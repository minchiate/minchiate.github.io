Colle carte delle Minchiate si fanno
due altri giuochi, diversi da quello,
detto comunemente Alle Minchiate,
descritto quivi sopra dal Minucci; ma
però simili fra di loro: e questi si chiamano
A' sei tocchi, e Al palio.

Si fa A' sei tocchi in due persone, ed anco in tre e in quattro, si
mescolano le carte: ed alzate, se ne danno sette per uno, le quali
ciascheduno tiene scoperte avanti a sé sulla tavola.  Di poi quello,
che ha fatto le carte, preso in mano il mazzo di quelle che sono
avanzate, ne trae una per volta dalla medesima parte, donde ha tratto
l'altre, che ha dato a' compagni, e scopertala, se quella tal carta
tocca, cioè è accanto, o di sopra o di sotto, a una di quelle, che sono
scoperte in tavola, chi ha questo tocco, la prende per sé, e la serba,
fintantoché non n'ha acquistate sei. Ed il primo che arriva a questo
numero, vince il giuoco.

Per esempio: nelle mie sette carte scoperte v'è il 25, se esce fuori il
24 o il 26 io dico tocco, e prendo quella carta.  E se per avventura
uno de' compagni averà il 23, o il 27, allora il 24 o il 26 non si da a
nessuno, e si pone nel mezzo della tavola, per esservi due che lo
toccano.  Chi fa Pappoleggio, vince il giuoco di posta, ancorché non
avesse aquistato alcuna carta.

Il Pappoleggio è, quando alcuno ha due carte fra le scoperte, che siano
distanti un punto l'una dall'altra, v.gr.\ il due e il quattro di
danari: se esce fuori il tre, si fa pappoleggio, e resta vinto il
giuoco.  E in questo modo si giuoca A' sei tocchi, come si dice, alla
piana, e senza pericolo di molta perdita.

Ma volendosi fare giouco più grosso, s'usano alcuni patti o scommesse,
che sono le seguenti: Primo tocco, Guasto, e Privilegio. Il Primo tocco
è, l'essere il primo ad acquistare una carta; il Guasto è, l'escir
fuori una carta, distante due punti da una delle scoperte; v.gr.\ uno
ha il 13 ed esce fuora l'11, o il 15.  E Privilegio è la carta distante
tre punti, che al 13 sarebbe il 10 e il 16.  Ed ogni volta, che si
vince una di queste tre scommesse, si segna una partita.  Si scommette
ancora al primo tocco in tavola (che è quando si tocca colla prima
carta che esce fuori) ed allora si segnano due partite; e si scommette
alle verzicole, che è, quando si fa verzicola colle carte scoperte, e
con quelle ancora, che s'acquistano.  Inoltre si vince il giuoco marcio
a coloro, che non hanno acquistate tre carte, che sono la metà di sei,
e si segna loro la posta doppia.

Ora perché questo giuoco (quando si fa spezialmente con tutti questi
patti) richiede molta attenzione, potendo passare a monte o esser prese
da altri molte carte, che si sarebbero potute acquistare per sé, di qui
è che si può dubitare, essere da giò derivato il detto, usato dal
nostro Poeta nel C.VI. St. 44. per dimostrar due che stiano
attentissimi a tavola a mangiare:

Sembrano a solo a sol due toccatori;\\ perciocché in verità certi tali
non muovono mai il guardo di sul loro piatto, ed insieme colla coda
dell'occhio guardano, se venga altra vivanda; siccome i detti
giuocatori a' sei tocchi guardano con tutta attenzione le proprie
carte, e danno nell'istesso tempo un'occhiata a quelle, che sono tratte
dal mazzo.

E in ordine a questa denominazione si può dire, che come coloro, che giuocano alle minchiate, si domandano Minchiatisti, così quelli che giuocano a' sei tocchi, si dicano con voce equivoca Toccatori.  Non riprovo per altro la spiegazione del Minucci, fatta nella sua nota alla citata Stanza 44, ma dico bensì che nondal ritrovarsi due toccatori all'osteria (che è cosa molto accidnetale e da non fare stato per un detto comune) ma dal loro ufizio proprio sia derivata l'origine.

Negli statuti della mercanzia libro 1 rubr 13 si legge: Amministrino almeno due di loro insieme il loro offizio, e se faranno alcun tocco, al quale non siano stati almeno due di loro presenti, e tanto propinqui l'uno all'altro, che abbino possuto sentire le parole l'uno dell'altro, ec. e di sotto Faccino di lor tre coppie, ec. Per tanto dall'andare così uniti ed attenti per fare il loro ufizio, ne sarà nato il proverbio.

In questa rubrica si dichiara, come anco dalle addotte parole si vede, che i Toccatori erano sei, ma ora sono ridotti a due soli, per esservi poco bisogno del loro ministero.

Ora passando a dire del giuoco Al palio, questo si fa nella medesima
maniera che A' sei tocchi, solo è differente in questo: che si debbono
acquistare dodici carte, ma si pignliano non solo quelle che toccano,
ma tutte quelle che sono più accoste, e che non siano distanti i
medesimi punti da quelle degli altri compagni. Quelle però che toccano,
contano come se fossero due, e l'altre che non toccano, contano per
una.  Il giuoco però si fince da colui che prima degli altri arriva a
segnare dieci, ovvero dodici lupini, o dichiamo segni; che ciò sta nel
concordato.  Chi è il primo ad acquistare dodici carte, segna un
lupino; chi fa il tocco in tavola, ne segna due, e ciò non seguendo,
chi primo tocca, ne segna uno, chi accusa pappoleggio, ne segna uno, e
venendo fuori il detto pappoleggio, ne segna tre, avendo nelle carte
scoperte, o venendo con quelle, che s'acquistano una verzicola, sne
segna uno, ed essendo la verzicola d'arie, ne segna due. si possono
ancora in questo giuoco fare scommesse di verzicole, tocchi e altro
conforme più piace a' giuocatori.


