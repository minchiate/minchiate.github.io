\documentclass[11pt,a5paper]{article}
\usepackage[utf8]{inputenc}
\usepackage[T1]{fontenc}
\usepackage[english]{babel}

\usepackage[left=13mm,top=11mm,right=13mm,bottom=14mm]{geometry}

% I like this font!
\usepackage{tgbonum}
\renewcommand{\rmdefault}{qbk}

% how to format and space chapter titles
\usepackage{titlesec}
\titleformat{\section}[display]
            {\huge\bfseries}
            {\vspace{-1.5em}}
            {0pt}
            {}
\titleformat{\subsection}[display]
            {\normalfont\fontsize{12}{14}\slshape}
            {\vspace{-1em}}
            {0pt}
            {\raggedleft}

\begin{document}
\section{XIII}
\subsection{Account of the Italian Game of Minchiate, \\ by Robert Smith, Esq. F.R.S.\ and F.A.S.\ \\in a letter to the rev.\ John Brand, Secretary.}

Read December 8, 1803.

Sir,\\
In the Eighth Volume of the Society's Archaeologia, are many ingenious observations on the origin of card, and their introduction into England; more specifically those of our late learned member Mr. Gough.
That gentleman has in a manner exhausted the subject; but as, in his account of the Italian game \textit{Minchiate}, he has given French names to the cards, I am induced to think that he was indebted for his information principally, if not wholly, to the ``Voyage d'un François en Italie'', and to the casual inspection of an imperfect set of cards of French fabrique, though inscribed with the name of an Italian maker.

I have the honor to present the Society a complete set of \textit{Minchiate} cards, such as have been long in use at Florence, and a small treatise in the Italian language, containing the rules of the game, and directions for playing it; both which I brought from the continent some years ago, and have had them by me ever since.

There is no game on the cards, of which I have any knowledge, that requires closer attention, a more ready talent for figures, or greater exercise of the memory, than this of \textit{Minchiate}.  It is held in high estimation among the fashionable circles in Tuscany, where almost every body exclaims, in the language of the treatise, ``è senza dubbio il più nobile di tutti i Giuochi ch siensi mai potuti inventare colle carte''.  I shall endeavour to sketch some of its principal outlines, referring to the treatise itself for a detail of its rules, and of the various combinations and changes of which it is susceptible.

A \textit{Minchiate} pack consists of ninety-seven cards, of which fifty-six are called \textit{Cartiglia}, forty \textit{Tarocchi}, and one \textit{Matto}.

The \textit{Cartiglia} is composed of the four suits, each containing fourteen cards.  The suits are, \textit{Spade} or swords, with us called Spades; \textit{Bastoni} or Clubs, \textit{Danari} Money, answering to our Diamonds; and \textit{Coppe}, cups or chalices, which may be considered as corresponding to Hearts.
All these names, though Italian, are evidently of Spanish origin.
In the two suits of Clubs and Spades, the ace, the deuce, the tray, etc.\ up to the ten inclusive, rise in value in the order mentioned; in Hearts and Diamonds the value increases, inversely, from ten downwards.
Next to the ten numerical cards of each suit follow the four picture cards, which are, \textit{Fante}, the knave, (in Diamonds and Hearts called \textit{Fantina}), \textit{Cavallo}, the horse, \textit{Regina}, the queen, and \textit{Re}, the king.
None of the picture cards have any intrinsic value, except the king, which counts five points independently of its value in combination with other cards.
Of the forty cards that compose the \textit{Tarocchi}, thirty-five are numbered in Roman capitals, and five are not numbered.  On the first five of the numbered cards are coloured representations of a juggler, an empress, an emperor, a pope, and a lover wooing his mistress; a combination perhaps not wholly accidental.
The rest of the numbered cards have on them figures of historical characters male and female, of beasts, and other animals real and fabulous, emblems of the four elements, the twelve signs of the zodiac, youth, old age, fortune, justice, death, the devil, and some ludicrous devices without any determinate meaning.
The remaining five cards of the \textit{Tarocchi} are called \textit{Arie}; but of these I shall make more particular mention by and bye.

It is to be observed, that the first five of the numbered cards are called \textit{Papi}, or Popes, as Pope-ace, pope-deuce, pope-tray, etc; pope-ace counts five points, the others count three each.
So likewise the numbered cards from vi to xii inclusive are called popes, as pope vi, pope vii, etc; but none of these count except the x, for which are reckoned five points.
Number xiii also counts five points, as does number xx, but none of the numbers from xiv to xix, or from xxi to xxvii count any thing.
From xxviii to xxxv each number counts five points, except number xxviiii, which counts nothing, unless when either the first or the middle card of a \textit{Verzicola}; it then counts five points also.

The five cards of the \textit{Tarocchi}, which, as I have observed, are denominated \textit{Arie}, though not numbered, rank as xxxvi, xxxvii, xxxviii, xxxviiii, xl.  They are called \textit{Stella}, \textit{Luna}, \textit{Sole}, \textit{Mondo}, and \textit{Trombe}; and are designated by rude representations of a Star, the Moon, the Sun, a winged female figure standing on a circle (the symbol of eternity) and holding in one hand a crown, in the other a sceptre, and of another winged figure or cherub, in a kind of glory, blowing a double trumpet.
Each of these counts ten points, independently of its value in a \textit{Verzicola}.

The remainder card of the pack, or ninety-seventh, is the \textit{Matto} or Fool, which counts five points, and has one property peculiar to itself, it can neither take nor be taken, unless the holder have no other card left, and then only with restrictions.
It may be tacked to every \textit{Verzicola}, the value of which it enhances five points.
The figure on this card is represented in his usual motley dress and long-cared cap, playing with boys.

Having had occasion more than once to refer to the \textit{Verzicola}, it may be proper to say something of its specific character and value.
A \textit{Verzicola} is a species of sequence peculiar to this game, and is of greater or less value according to the quality of each card separately, and to the whole in combination.
The regular \textit{Verzicola} is composed of more cards of the \textit{Tarocchi} in sequence, as i, ii, iii; ii, iii, iiii; iii, iiii, v; or, i, ii, iii, iiii; ii, iii, iiii, v; i, ii, iii, iiii, v; all of which are called \textit{Verzicole} of Popes.
The numbered cards xxviii, xxviiii, xxx, are called verzicola of \textit{Tarocchi}, or of thirty; xxxi, xxxii, xxxiii, ar \textit{Verzicola di Sopratrenti}, or above thirty; and xxxiii, xxxiiii, xxxv, a \textit{Verzicola di Rossi}, from the red colour of those cards.
There is also a \textit{Verzicola} of \textit{Arie} in conjunction with certain of the numbered cards, as \textit{Sole}, \textit{Mondo}, and \textit{Trombe}, in which are three \textit{Arie}; or the numbered card xxxv, \textit{Stella}, and \textit{Luna}, in which are but two \textit{Arie}; or xxxiiii, xxxv, and \textit{Stella}, in which there is one only.
Hence the first \textit{Verzicola} of \textit{Arie}, which consists of three \textit{Arie}, counts thirty points, the second twenty-five, and the last twenty.

Besides the regular \textit{Verzicola}, there are other called irrecular \textit{Verzicole}; and these also count according to the valu eof the cards composing them.
Three kings or four kings form an irregular \textit{Verzicola}; so likewise ace, \textit{matto} and \textit{trombe}; x, xx, xxx; xx, xxx, xl; and i, xiii, xxviii.

I fear that I have nearly exhausted the patience of the Society, and I shall therefore conclude.
The printed treatise, like other of the kind, contains rules for shuffling, cutting, dealing, playing, and throwing out the cards, the penalties for a breach of the rules, and a variety of other matters; all of which require the particular attention of those who would desire to obtain a knowledge of this most curious but difficult game.

{\centering
  
I am, with very great regard, \\
Sir,\\
Your most obedient\\
\hspace{12em}and most humble servant,

}

\noindent Basinghall street, Nov.\ 28, 1803. \hfill Robert Smith.


\end{document}
