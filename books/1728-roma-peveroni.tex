\documentclass[12pt,a6paper]{article}
\usepackage[utf8]{inputenc}
\usepackage[T1]{fontenc}
\usepackage[italian]{babel}
\usepackage{changepage}

\usepackage{tikz}
\usetikzlibrary{decorations.shapes,shapes.geometric}

% avoid orphans and widows, allow for (a lot of) letter spacing.
\usepackage[defaultlines=2,all]{nowidow}
\usepackage[tracking]{microtype}
\sloppy

% how to format and space chapter titles
\usepackage{titlesec}
\titleformat{\section}[display]
            {\huge\bfseries}
            {\vspace{-1em}}
            {0pt}
            {}
\titleformat{\subsection}[display]
            {\normalfont\fontsize{12}{14}\slshape}
            {\vspace{-3em}}
            {0pt}
            {\raggedleft}
% I like this font!
\usepackage{tgbonum}
\renewcommand{\rmdefault}{qbk}
\usepackage{lettrine}
\usepackage[left=14mm,top=12mm,right=11mm,bottom=14mm]{geometry}

\newcommand{\supersection}[1]{%
\clearpage
    {\scshape \centering \huge #1\\}
    \vspace{6pt}
    \hrule
    \vspace{12pt}
}

\newcommand*\sepline{%
  \kern 3pt \hrule \kern 2pt
}

\title{\kern -2em
{\huge REGOLE GENERALI}\\
\textls[120]{\Large Del Nobilissimo Gioco}\\
\textls[1000]{\normalsize DELLE}\\
\fontsize{34}{34}\selectfont{\textls[100]{MINCHIATE}}}
\author{%
  \vspace{-26pt}\\
\textls[122]{Con un modo breve, e facile}\\
\textls[44]{per ben imparare a giocarlo.}\\}
\date{%
\vfill
  \kern 3em\small In Roma, Per Raffaelle Peveroni. 1728.\\ \sepline
  \textit{Con licenza de’ Superiori.}}
\begin{document}

\pagenumbering{gobble}

\maketitle
 

 


 

 

 
Cortese Lettore

Il veder girare questo
picciolo Libricciolo sotto di un
titolo solo così specioso di
Regole generali del Gioco
delle Minchiate, son sicuro,
benigno Lettore, che ti recherà
qualche sorte di maraviglia:
prima perche hai veduto altre
volte stampate queste Regole
senza alcun frutto, e poi perche
hai trovato in ogni conversazione
ove prattichi, che ogn'uno
gioca questo gioco a capriccio,
e con regole diverse, e per
esser bravo Giocatore basta
asserire con franchezza la
pratica delle altre conversazioni
ancorché non vera; poco importando,
che questa prattica degeneri
in abuso; e siccome questo
é stato l'unico motivo, che hd
havuto in flamparle per levaye
da mezzp tante prattiche, e
tanti abusi: volendo dare una
regola generale, certa, e fissa
per tutte le conversarzioni, così
pero, che saprai compatire il
mio ardire, quale per renderle
accreditate bo pretefo ancora
di renderne la ragione, perché
devono tutte essere pratticate
uel modo da noi prefcritto; ma
fe poi queste ragioni da noi
addotte non foffero tali, che
poteffero capacitarti 2 baflanza
sappi che io mi contento
di moderarle in quella parte 
ove fossero defettose, purche ti
complacci direndermi una ragione
maggiore, la quale faccia
costare essere falsa la prima,
mentre io che non ho mai avuta
altra idea nel fare questa
picciola raccolta, che distruggere
ogni abuso, per publico beneficio,
e rendere questo Gioco senza
competenza di prattica eguale
in tutte le conversazioni, mi
saprò sottoscrivere di buona
voglia al tuo faggio, e miglior
parere ogni qual volta lo
conoscerò fondato su la ragione. Del
resto vivi felice .

La Gioventú de' tempi
nostri è così inclinata
al male per genio, e
per natura, che stima
debolezza di spirito signorile,
evilta d’animo nobile
l'intraprendere per suo divertimento
qualche onorato impiego, a segno
tale, che vuole più tosto
marcire nell'ozio per avere
campo di poterfi liberamente
dare in preda ad ogni forte di
vizio che fe gli faccia incontro,
che sottometterfi a qualche léggiera, e virtuofa fatica,credendo ftoltamente di non poterf
prendere altro divertimento
pil conveniente alla loro nobile
condizione di quello lisa fuggerire
un mal configliato cappriccio 3
Onde la vediamo giornalmente
pratticare baldanzosa
tutti li ridotti, e frequentare
tutti li poftriboli, ove il gioco
facendola da gran capitano tira
a fe tutta la Gioventu vagabon
da per condurla a militare sotte
il ftendardo della difperazione,
¢ con la fperanza d’un impofibile guadagno li conduce ad un
ftata miferabiliflimo fenza ri.
paro.

Tralafcio le frodi, che quivi
fempre fi meditano; Pinganni,
che fempre fi tentano con pregiudizio anche notabile del proptio carattere ; tralafcio le bestemic, che quivi continuamente fi fentono, le questionl, che fi
fomentano, livituperj, che fi
Pratticano, e dird folo, che qui
viregnano continue lecrapole:,.
nelle quali fanno folo macftofa

pompa: la prodigalita, Pintem
peranza,.e Pubriachezza madri
feconde ditutte le fceleratezze
più abominevoli,le quali ma
fcherate col nome digenerofita
grandiofa, dé spiritofa fortezzay
e di animofa gagliardia ftrafci~
nano a viva forza la Gioventt in

un precipitofelaberinto- di vizjj.

a fegno tale, che non sa più co-=
nofcere il bene, non sa più diftinguer ragione, ¢ fe qualche
wero, @fedele amico., amante
dell’onor fuo,procura con bella

forma.di farle concepire l’erros
re,. i cub abbagliato fen vive,
fi fa reo.d’infedelta-, ¢ colpevole di zelo indifcreto, quafi.che
voglia impedirle li fuorgiovanili paflatempi,non confideran
do. altro fpaffo per loro., che.
quello gli sa fuggerire un mal
configliato capriccio.

Onde io per vedere di diftogliere la nobile Gioventt ims 4
qualche parte da cotanti abominevoli vizj mi fono lufing ato’ di poter loro impedire una
firada cotanto precipitofa coll’
allettamento d’un folo vizio,
quale per altro abbia in fe fteffa,
¢ del dilettevole,e del virtuofo,,
accioche fodisfacendo in quetta,
forma alla loro mala inclina.
zione possa bel bello andarg
fcordando ogn’altro vizio peg—
giore.

Non vie dubbio, che tutte
le novita. rifvegliano nel?ani- ¢
mode’Giovaniun nonsochedi 
piacevole, che fortemente gli
ftimola per prefto condurli ad
una perfetta cognizione di quel1a novita ; cosi mi Infingo 5 che
per essere questo vizio non ha
1olto introdotto per tutte le
onverfazioni poffa con facilità
1overe l’animi della Gioventi
d applicarvi con tutto. il ferore, econ tutta lattenzione
er ben impoffessarfene ;, tanto
iu che fembra degno. di ginfta.
jprenfione colui, che pretende:
li pratticare tutte le nobilt con=
rerfazioni fenza fapere questo
jizio virtuofa..

E’ quetto il nobiliffimo., ediettevole gioco delleMinchiate,,
sotanto pratticato in tutte le
converfazioni, quale per altro,
yorta in fronte Vorribile nome:
li gioco, caggione d’ogn’altra
(celeratezza, ma racchiude pero. in fe tea un compleffo di
virti, quale lo rende amabile,
guftofo, ¢ piacevole, perches
oltre Vimparare l’aritmetica
s'impara ancora la politica,. la
prudenza »e economia ; onde
per non avere db viziofo, che jl
folo nome, pare che poffa meritar con ginuftizia tutta Pattenzioneé della Gioventi:, mentre
e maggiore il vantaggio,cheé
per apportare di quello.non sia
il danno, fe ne poffa temere,

Figuratevi in questo gioco di
vedere. due inimici a fronte,
quali con Parmi alla mano.,e
coir la rabbia ful cuore vanno
avventandofi vicendevolmente
replicati colpi di spada per cavarfidalle vene lun Valtro col
fangue la vita; cosi appunto il
Giocatore delle Minchiate ayi- .
do anch’egli della vittoria, va
eon bell’ arte machinando fecrete infidie al fuo caro inimiCo per farle perdere quella carta donore, che troppo —

“ i:

 
i
di pregindizio a fe fteffo fe no
la perdesse, e con bello, e polis
tico ftudio procura cavare ik
fangue dalla borfa del fuo inimico, fenza pero che egline
refti offefo; ma folo lafcia in lui
un bel desiderio delka vendetta,
perloche crefce in lui lo ftimoJodi andare piu guardingo nei
cimenti, e quando fe glifa in=
controuna dura necessita di re=
ftare perdente, deve almeno
con tutta Peconomia: possibile
procurare il fuo minor male,
perche fembra anche specie di
vittoria rl faper refiftere con
animo generofo alle avverfita
jnevitabili, mentre tutta la vaghezza di questo gioco, confifte
in fapere con prudenza approfittarfi del (uo maggior vantaggio, 0 pure Vevitar il fuo maggior male con daria i diElle
 

 

14

fenderfi dall’infidie fecrete d'un
inimico fcoperto, e con bella
forma fempre renderfi a lui fu
periore in ogni cimento .
Attefo dunque le così belle
qualita di questo gioco parm »
che il faperlo a perfezione possa
essere d’un gran vantaggio alla —
nobile Gienenti, qual’e inoltre
necefatata 4 ben faper contate »
fommare, e sottrare 5 pertanto
ad essetto,che possa ogn’une.con
tutta facilità impoffe flarfene dk
buona voglia intraprendo questa fatica 5 defcrivendo. prima»
tutte le regole generali, che
pofiono ben ammaeftrarvi 10
quefio gioco, © pol andaremo
bel bello difcorrendo del modo
di pratticarle. Tocchera pet?
avoia ben ftudiarle per potere
-poicon tutta facilità ridurle
buona prattica; essendo. che 14

; Teo15

Teorica da per fe fola non é ca*
pace a rendervene informatia
pieno: attefe le grandi appendici, che anno tutte le regole generali, e queste non puole impararle altri, che una diligente »
& efatta prattica, mentre ndscono da foli accidenti, che continuamente diverfi accadono.in
questo gioco 5 onde per ben imparare questo sioco fa di meftie~
reunire alla Teorica la prattica, la quale come ailai più bra~
va maeftra sapra meglio darvi
ad intendere quello, che non sa
fpiegarvi la Feorica.

Questo gioco trae la fua pri~
ma originedalla nobiliflimaCitti di Firenze inventato con bel-'
lo ftudio pert un piacevole ammaeftramento della nobile Gioyentw nell’ Aritmetica, ma le altre vistu,che Padornano Phanno

refo:

 

 
 

16 refo cosi univerfale, che ormai
fembra difetto di curto intendimento il non faperlo, mentre
fi fa reo d’incapacita chi ne trax
fcura il confeguimento,. folo vi
avverto pero,che per ben gioca=
xe a questo gioco non basta il
conofcere le-carte,ed il giocarlo
perche qnefto é ftato ginfto if
motivo,,che mi hamoflo.a defcrivere le regole generali di
questo gioco, essendo che in.»
ogni converfazione dove anda~
vo,. trovavo regole: diverfe,e¢
talvolta alcuno fi faceva lecito ne’ cafi, dubj: afferire con
franchezza per ferma-refoluzione cid,.che ne-meno.¢ da metterfi indubio, e fe poi glie ne
dimandavate laragione, non»
fapevarifpondere altro,foloche
così fi prattica in altre conver~
fazioni, fenza riflettere» che
ognt

ak ot mM RO Lt et ett a Of tie feed eh, ih it Oe CI IA ee OA kt OCMC COPS
rp
yeni converfazione fi regolas
on regole diverfe, & & capricio, ed il giocare fenza faper
‘endere la ragione del suo molo di giocare, oltre l’essere re~
ugnante atutie le nature de’
siochi, & anche contro Vinten~
ione dell’ Autore,. quale ha preefo nell’ inventare questo gioco:
li ftabilire regole certe, filfe, &
mmutabili, le quali doveffera
empre fervire per tutti, fenzas
afciare in arbitrio de’ giocatork
] variarle 4 loro capriccio. OnJe per togliere da mezzo questi
ibufi communemente praticati
n moltiflime converfazioni absiamo ftimato neceffario defcrirervi tutte le regole generali, —
\ccio. ch’ogn’uno pola reftarne
‘apace, eregolarfi in avvenire:
ie] modo, che diremo in ap-.
ouesto.. Intanto vi basti re
cne
 

 

 

 

 

 

 

 

19

che questo gioco fi fa con 9%.
carte, fra qualive ne fono 24
d’onore,0 vogliandire che contano, ¢l’altre tutte non col
tano niente.
Questo gioco tutto che abbia
un folo nome puol perd giocat#
in più modi tutti diverfi, men
tre fi puol giocare 4 ogn’uno 04
per fe folo, e fi puol giocare con
il compagno adue per due;
più fi puol giocare,in due,in tres
ed in quattro, e quando fi gioc®
in meno di quattro, allora fi 0°
ca ad ogn’uno per fe, fe bene®
puolanche giocare in quatttos
ad ogn’ano per fe, ma quando ®
g10ca in quattro il vero giocate
egiocar con il compagno:,.
Quando fi gioca in mene di
quattro fidanno fempre 25. CaF
te per ogn’uno, e quando fi 810°

ca in quattro fe ne danno fola men19
mente 21.con che pero l’ultima
deve fempre darfi fcoperta a

tutti,e fe questa fara carta d’ho-.

nore quello, a cui fara toccata
fegnera tanti puntia fuo favore
quanto importa la carta fcoperta, fe poi non e carta che»
conti, allora non fi fegna niente; dopo che tutti li Giocatori
averanno ayute tutte le fue carte, quelle che reftano fono fuori digioco, enon fi confiderano cofa alcuna in quel gioco, ne
é lecito ad alcuno de’Giocatori
il poterle vedere.

Quando fi'gioca ad ogn’uno

per fe quello che ha cattive carte puole non giocare in quella
mano, ¢ gettare le fue carte ab
monte, e pagare alcuni punti
a quello, che fa Vultima mano,
fecondo quello, che convengo~
‘ho, e fe bene pero lui non gioca

In
 

 

20
1D quella mano rifcuote la pena
dell’onori, che fi perdono da
chili perde giocando; H folito
quello, che ha la mano 5. fe non
vuole giocare paga 20. puntl, il
fecondo ne paga 39. il terZzo, ed
il quarto.ne pagano 4o.

Questo modo di giocare, fe
bene per fe fteffo é piu bello»
perche ¢ pill difficile giocandofi
‘totalmente all’ofcuro ; nulla di
meno fembra fmezzato, perche
in quelle poche carte, che reftano fuori del gioco vi possone
essere le migliori carte del gioco, e quelle, che poffono fare
tutto il gioco perd ¢ ftato inventato dopo il gioco della fola, quale porta feco per necefiita, che tutte le carte d’Onoré
siano in gioco, e questo non

ftato fatto ad altro fine, che pe

rendere il detto gioco pill tach
le,e piivago. — Foat

Fola vuol dire, che tutte le
carte d’onore siano in gioco a4
segno tale, che non puole reitare fuori del gioco una carta»
@onore,benche minima, e quelle, che reftanonell’ultime car~
te deve tutte pigliarfele per fe»
quello, che fa le carte; ma 58,
che voi mi rifponderete, come
e possibile, che fola vogtlia dire, che tutte le carte d’onores
siano in gioco, fe fecondo quello voi avete imparato poc’anzi
puole darfi ’accidente, che an-~
che giocando con Ja fola tutte
le carte diconto non siano in,
gioco; mentre fe fi gioca a ogn?
une perfe, ancorche fi giochi
con lafola qualched’un de’Gio.
catori puole gettare le fue car.
te al monte, fe bene avetle jp
mano dell’onori; onde in quel
cafo reftarebbero fuori del Zio-O Carte d’onore,. Tous

 
 

 

 

 

 

 

 

22

lo pero vi rifpondo,che ever
soquello, che vi ho imparato 4
principio; ma non per questo &
vero, che reftino fuori del gioco carte d’onore, quando uno
de’ Giocatori dice di non volet
gviocare, e gettale fuc carte al
monte, perche quel Giocatores
che getta le fue carte al monte »
dice di non voler giocare ¢
obligato a pagare quella pends —
che gl’é ftata aflegnata in prin
cipio, qual pena vain compensazione delle carte d’onore, che
puole avere in mano : mentre
none¢ credibile,che uno,il quale
abbia in mano tante carte d’onore, quanto puole importare. a
fua pena voglia fenza rifico Vo
runo pagare la pena certa, @
non giocare 5 onde ecco, che la
pena viene a compenfaré que

danno,che puole caggionarli ann,
quels
23
quelle poche carte d’onore posse al monte da chi non vuol
giocare,sicche non puole giuftamente dirfi in queito cafoy che
siano fuori del gioco cartes
donore, perche quelle vengono confiderate in gioco mediane
tela pena,perloche reftera fempre vera la propofizione, che
fola vuol dire, che tutte le carte d’onore devono fempre essere in gioco, ne puole niuno get;
tarle al monte, fe non in cafo
ne paghi il compenfo,quale non
fi deve permettere, fe non &
quelli, che hanno la facolta di
gettare tutte le fue carte al
monte, e non giocare in quella
mano; ma quando non fi possono gettare tutte le fue carte al

monte ; allora ne meno élecito.

il gettarne alcuna fola d’onore,
perche allora farebbe contro if
Z10»

 

 
 

24 :
gioco, ee potrebbe portare del
gran pregiudizio, come diremo
in appreffo.

Hora poi che havete impara-~
to che cofavoglia dir fola, dovete imparare 4 conofcere tuttele carte d’onore, e dopo che
averete ben apprefa la cognizione delle medefime allora doverete fare lo ftudio delle verzicole, quali vengono compofte
con fole carte d’onore,ne puole
mai essere verzicola, fe non yj
concorrono almene tre carte
d’onore; per ben conofcere tuttele carte d’onore, é necessa~
rio ancora conofcere quelle,che
non fono @onore per poterle
diftinguere, ¢ pero dovete sapere che tutte le carte fidividono
parte inTarochi,e parte in Cartiglia, la cartiglia po! fi divide

in quattro fpecic deter cioe
pa
 
spada, bastoni, coppe, e denaroy
ed ognuna di queste quattro-spe cle e compofta di quattordici
carte, quali tutte non contano
cofa alcuna, ad efclufione perd
del? ultima, che fi chiama communemente il Re, quale contas
cingque, ed.é carta d’onore, nel
eiocare poi dette carte, la magpiore prende fempre la minore ;
ad efclufione pero.di coppe, e
denari;-mentre la minore prenle lamaggiore, quando perd non
fino carte; figurate, essendo che
le figurate precedono femprey
alle numerate, ele numerate
folo.anno il privilegio di precedere alle maggiori dinumero.
Li Tarochi fono quaranta, e
col Matto fono quarantuno,fino
al trentacinque fono tutti numerati, ed i] maggiore precede
mprefe al minore; dopo il trens
oe B ta 

 

 

 

 

tacingue vi fono altre cinque
carte, le quali fi chiamano ari¢ »
e non fono numerate ; ma fi diftinguono peré dalla loro repréfentazione, mentre la più inferiore rapprefenta la Stella, 1a
feconda la Luna,la terza il Soles
laquartail Mondo, laquinta le

 

—a

Trombe. ; ¢ queste fono le mag*

giort carte del gioco. In oltre
vie un’altra carta,la quale é fen
zanumero » e non fa nessunas '

figura, ¢ fichiama il Matto »

i quale all’ofanza de’ Matti fa
quel che yuol¢ ; ma pero é infe>
riore, e tutte le altre carte tut
to che siano carte d’onore,men=

tre Ini non puole mai pigharé
neffuna carta .

Le carte che contano fond
dalf’uno fino al cinque inca
ve, il dieci, il tredict » ed il ven”
ti, ¢ dal yentiotto fino ie

3 ; yall

y
 

27
ranta efclufene, il yentinove
quale|non conta, fe non in cafo
che faccia verzicola, tré di ques
fte come dicemmo di fopra fanno verzicolas ma per essere vers
zicola’é necessario, che siano
confecutive.,. come uno, due,e
tre, 0 pure due,tre, e quattro, 6
pure 28, 29.30. © cosi di tutte
Paltre in appreffo fino al. quatanta ; s¢ poi foflero quattro, 6
cinque carte: confecutive tutte;
fanno verzicola,ma fe non fono
pero confecutive non poffonoe
fare verzicola oltre di tuttes
queste vetzicole, ve ne fono delle-altreycome unMatto;e Trom=
ba, uno, tredeci, e ventiotto,
dieci; venti trenta, ¢ quaranta,
ma per essere verzicola devono
fempre essere confecutive, di

“più tre Re'ancora fanno vérzjcola, il Matto poi entra in tuta
Ba te

 
28
te le verzicole, fe bene da per fe
folo non fa altra verzicola che
uno Matto,e Tromba. 9): ? °

Tutte queste carte che fanno
yerzicola contanto cinque©
ponti per ognuna,ed efclufione
delli quattro Papi, .che contano
folamente tré, ¢ le cinque arie
che contano dieci :per-ognuta >
Il vehtinove conta folamente
Cinque, quando fa verzicola, al=
trimenti non conta’ niente V1
fono alcuni che fanno ancora. —
la verzicola diMondo,Carne,¢ |
Diavolo, ed:in quel cafo quando —
i] Diavolo fa-vetzicola conta)
cinque sancor essos altriment!
non conta niente. 1 « ¢"!

Le verzicole: ficontano tf
yolte, cioe una prima di comity

ciave 4 giocare, maé pero is
a

 

‘che ‘bifogna anche mow a.
prima dicominciare 4 a i

 
/ 29
ne fi puolé più moftrare dopo
d‘haver giocata la prima carta,
né si deve fupporre moftrata, 6
per averla rubata,6 per averla
trovata nella fola; ma bifogna
necessariamente moftrarla prima di giocare chi vuole contarla,-alla fine pot-del gioco fi
conta di nuovo’ due | volte,
quando: pers. fi sia fatta .
Quando havrete ben ap
prefe tutte queste cofe come neeeffariiffime::. allora comincia=
rete avenire in qualche cogni+
zione diquefto gioco, quale fi fa
nella forma, che diremo adef=
~ fo;questo gioco deve giocarfi in
quattro perfone a due per due,
e primieramente fideve venire
al’elezione de’ Compagni, es
quando li Giocatori non fi accordano fra diloro, allora de.
ve rimetterfi Pelezione alla for.

Bg ae

 
 

 

 

 

eS

30
te, fatta che fi fara l’elezion®»

de iCompagni,.comincera D0 —

de’ Giocatori 4 mefcolare ben
bene tutte le carte affieme » ¢
poi lafcierd ad arbitrio dellas
parte Palzarle, quale dovra
riconofcere l’ultima carta alza@

ta, e fe fara. carta d’onore, 0 so>

praventi,allora dovra pigliarfe
la per fe,,¢ pigliera tutte quelle,

che ritrovera,¢ fe faranno car~ —

te d’onore, dovra fegnare a fud
favore tanti punti,quanti ne im
porteranno: le carte d’onore al
zate, perche finoa tanto, che

‘trovera sotto la carta alzata, 9

carta d’onore, 6 fopraventi, tur
te dovra prenderfele per se,e

se fi dasse l’accidente, che quer”
lo, che alza rubbasse per fe piu
di quattordici carte, 4 fegn? ta
le, che non vi reftaflero pil cal”
te d (afficienza pet tutti a

 
31
catori, come dovra regolarfi
Pultime? in quel cafo,non essenco dovere, che egli non abbias
le fue carte giufte, né essendo
dovere, che quello, che alza
tralafci di pigliare le carte d’
onore che trova nell’ alzata. In
quel cafo per altro impoffibiles
deve regolarfi il gioco in quella
conformita, che fi regolarebbe
fe uno face ffe le carte fenza mescolarvile carte della folla, es
Pav verfario rubasse tre, 6 quattro carte; mentre dovrebbes
finire di prendere il fuo compimento delle carte da quelles
carte, che fono rimaste fuora
del gioco,poco importando,che
foflero cartaccie ; mentre basta,
che non sia fuori del gioco carta d’onore per essere obligato if
gioco a tirare innanzi,e ne inco]pi la fna trafeuraggine fe non

4 ha

 

 
zy

‘trovera sotto la carta alzata 916

30
te, fatta che fi fara Pelezionés
de i Compagni,. comincera uno
de’ Giocatori 4 mefcolare ben
bene tutte le carte affieme se
poi lafciera ad arbitrio della;
parte Valzarle, quale dovry
riconofcere ultima carta alzata, e fe fara carta d’onore, 6 sos
ptaventi,allora dovra pigliarfe.
la per fe, ¢ pigliera tutte quelle,
che ritrovera,¢ fe faranno Cars
te d’onore, dovra fegnare 4 fuo
favore tanti punti,quanti ne importeranno: le carte d’onore ay,
zate, perche finoa tanto, che
0
carta d’onore, 0 fopraventi, tutte dovra prenderfele per fe, «
fe fi dasse l’accidente, che quello, chealzarubbafle per fe pia
di quattordici carte, a fegno tas
le, che non vireftaflero più cars
te a sufficienza per tutti li Gioca 

Oe

31
eatori, come dovra regolarfi
Pultime? in quel cafo,non essenco dovere, che egli non abbias
le fue carte giufte, né essendo
dovere, che quello, che alza
tralafci di pigliare le carte d’
onore chetrova nell’alzata. In
quel cafo per altro impoflibiles
deve regolarfiil gioco in quella
conformita, che fi regolarebbe
fe uno faceffe le carte fenza mescolarvile carte della folla, es
Vavverfario rubafle tre, 0 quattro carte; mentre dovrebbes
finire diprendere il flo compimento delle carte da quelles
carte, che fono rimaste fuora
del gioco,poco importando,che
follero cartaccie ; mentre basta,
che non sia fuori del gioco carta d’onore per essere obligato il
gioco a tirare innanzi,e ne incol, pi la fua trafeuraggine fe non
fe B 4. ha
 

 

 

 

 

32

ha tatto bene te carte, ¢ fe ha
lafciato fuori del gioco nel mescolar le carte della fola at
tecedente, cosi parimente ne
cafo, che chi alza ruballe pill di
quattordici carte, quello che fa
le carte farebbe obligato a
prenderfi il compimento delle
fae carte dallo fcarto di quelloy che ha rubato, ed incolparYy
fe ftelfe fe non ha faputo mefco=
lar bene le carte, ne occorre il |
dire, che lo scarto fono tutte
‘cartaccie, e che Inj non devé —
avere men carte dellaleri » pretendendo d’ obbligare quello >
che alza a non poter rubaré
più di 13. carte per poter an.
sciare le fue carte giufte aque’
lo, che fa le carte, il che fe 60%
fotfe quando uno tralafcia /#
fola fuori del gioco? » pravverer”

be in confeguenza x che ae 7
x ad

che alza tion potrebbe rubare >.
perche fi sa di certo, che le
carte. della fola antecedentes
fono tutte cartaccie, eficcome:
non é vera questa propofizione
perche quello,che alza,ha facol.
ta dirubare, ancorche sia rima.
sia fuori delle carte la fola antecedente,cosiancora-ha facolta&
di:pigliare alzando. quante car=
te, che trova da poter pighiare;.
e quello. che fa le-carte in quel
cafo deve prendere il compi-mento delle fue carte dallo scarto diqueHo:, che-ha alzato, ilquale dovra. prima essere me+
scolato tutto aflieme .

Finita poi che fara Palzata,
quello, che havra mefcolate lé
carte daraprimadieciearte pex
ognutio de” Giocatori, e poi co«
mincerd da capo ¥ @ ne dard
uideci; ma Pultima dovra dags

Bs la

 
 

34
la fcoperta, efe fara carta di

onore, quello a cui fara toccata

segnera tantiponti a suo fav 0re, quanti ponti importera lai

carta scoperta, e dopo, che ha
vra date atutti li Giacatori le)
undeci carte con prenderfeley’

ancora per fe’; allora offervera
ancora lui, fe dopo la tua unde

cima carta vi fara apprefio al=.

tra carta d’onore,o fopraventts
e quelle dovra prenderfele tutte per fe, fino che ve ne trove

ra confegnate @ fu0 favore tan:
ti pont, quanti importeranno
le carte d’onore, che ivi avré

trovate . :
Dopod che avra yeduto non
esservi più carte da potert
prendere, offervera allora fem
quelle poche carte,che le fara’

no rimaste vi fara carta d’one7
derfele
t
re, e quelle dovra prenae
tu

as

 
35
tutte per fe’, ad efclufione perd
delli fopraventi, quali non fi
poffono pill prendere fe non in
cafo, che pofflino’ fare verzico=
la, ¢ quando tutti li Giocatori
avranno fatto le carte una volta per uno,allora fi verra di nuovo alla divifione de’ Compagni,'
quale dovra. esser fatta nellas:
forma, che abbiamo detta di
fopra. |

Vogliono alcuni, che fe quello, che fa le carte trovasse all’
ultimo il 29., ¢ questi non fas.
ceffe attualmente verzicola,che
non fi possa prendere, ‘e debbas
reftare fuori del gioco; ma siccome questa carta puole fempre
fare verzicola,ancorche attuals
mente non Ja faccia cosi, con fa
speranza, che possa fempre far
gioco non deve mai reftare nel
monte, e percid in moltiflime

BG con
 

 

 

 
 

 

 

 

i :
26
converfazioni di gia e ftato le
vato questo abuso, perche veramente ¢ stato conofciuto. jngiu«
fto; Vorigine di quef’abuso-pro*
viene da usvaltro.abuso maggie.
re, il quale fi prattica ancora in
moltiflime converfaziont; ma
per essere ingiuftiflimo» ¢ col
tro ogni raggione » menita d’efsere abolito affatto, come dir
gia é ftato fatto in molti luoghly
perche. ogni volta » che refti Ievato questo abuso maggiore, al

Jora refta ancora levato ogni

altro abuso,.che da quello pro~
viene. we
Si ftilava prima, ¢ ftila ane
cora in alcune converfazionls
che quello:s. che fa le carte fe 1
yelle poche carte,che reftane
all’ultimo. non avesse trovata+
carta d’onore » doveffe nece#4"
siamente pagare afliem¢ 6 |

 

 
37
suo Compagno. un refto all’
avverfarijse siccome succede's
che alle volte: in quelle poches
carte non vi fi ritrova che il
folo 29. cosi volevano, che fe if
detto. 2g. non faceva attualmente verzicola,non fi fofle potuto prendere, ¢ fosse allora ftato neceffitato pagare ilrefto. ©
‘Ma perche if far pagare il refto a chi non trova ¢arta di
onore nel monte era una legge
troppo: barbara, e troppo ingiufta » cosie gia ftata communemente levata-;-e fi: spera, che
in breve fi levera da per tutto $
poko poiche chi-fa le carte non
sia foggetto a pagare questa pe-~
na ».allora non vi fara pin difiicolta per il 29. perche compira
ancora alla parte avverfa,che il
detto.29. siain gioco, perche in
guel.cafo.puole avere lei fola la
—— dpe
 

 

 

 
 

| | 38
 fperanza, che li possa far gioco
| per avere il 39,e cosi vorra,che
( fenz’altro refti in gioco .
Hl Che poi non sia obligato @
pagare ilrefia, quello che nel
far le carte non trovaalcuna»
cartad@’onore., in quelle poche
| carte, che le reftano in mano
| | fembra giuftiflimo, perche il
pagamento del refto in questo
cafo fi prefuppone in pene di
non haver trovata carta d’ono-)
re; ma non fie mai trovata lege
cosi barbara, che oblighi a pa-garela penad'un atto, che dir
pende puramenté dal accidente, anche in pregiudizio pro~
prio,la pena deve folamente pa~
garfi per quelli fatti, che possono essere maliziofi, e ridondana
in vantaggio di. chi fi commete
te, ancorche alle volte ne fie

gua al contrario ; ma que™?
chit

 

 

 

 

ee
oy ee

che ridondano folo in pregindizio dichili commette, ¢ hon»
poffono essere maliziofi,non devono condannarfi a pena alcuna; mentre.basta in quel cafo il
folo pregiudizio, che fi prova
fenzaaggravarlo con maggior
pena, perche fe fi doveffe condannare alla pena del refto quello, che non trova carta d@’onore
in quelle poche carte, che fono
rimaste al monte, con maggior
giuftizia dovrebbe essere condannato a pena maggiore colui,
che in 21. carta non havra talvolta carta d’onore,e pure non
fitrova legge,che poffa obligarlo a pagarne la pena, bastando+
le per fola pena la neceffita del
perdere.

Oltre di questi abusi ve n’é un’
altro forfi di non minor confiderazione, quale deve onnina
men
 

 

 
 

Q

mente levarfi, perche & contto
Jaraggione, ¢ la giuftizia del

| gioco, come diremo adeffo «
  ~Digidavete intefo di fopta,che:

| il giocare con fola,. porta seco
per neceffaria confeguenza, che:
tutte le carte d’onore, oche
pofiono far gioco,devono.cie=
re in gioco », perche quella ¢ la
yera definizione della fola; dt
piti.avete intefo,.che tanto ques
to, che alza, quanto quello: che:
fale carte, hala facolta. di:pigliare per fe tutte. le carte di
onote, con tuttbli fopraventi,
che trova, sia nell’alzata,sia nel~
lafcoperta; ora abbiamo avedere, che cofa‘fi abbia 4 fare di
quelle carte prefe in quel modo.
Dovete prima fapere che
juno de’Giocatort puole gioca”
re gon più, O meno cartes a ”

 
ar,
tregl’é ato pofto per pena il
non contare alla fine del gioco
li fuoi cnori,sia comunque fi voglia, non potende-contare che
Pultima fela fa » e le carte fer
pure ne vince, © laraggione fi
eperche il giocare-con pil, o
meno carte puole essere fatto
con frode,come faremo vederes
ed a questa pena refta ancora»
foggetto il proprio compagno 5
perche prima che cominci a gio~
care puole,e deve avvertitlo che

conti le fue carte, acciò abbia. a

‘ giocare'con carte giufte, ¢ per

cid refta ancora lui foggetto alla
pena in cafodi contravenziones
mentre chi ha più carte del fue
dovere prima di cominciare @
giocare deve fcartarle, ¢ gettarJe al monte, reftando in fuo arbitrio fcartare quelle carte che
pitile parera, e le piacera,pur
che
 

 

 

42
che non siano carte d’onore,do
one nelle reftare tutte ins

510CO cotro l’opinione mal fon
data d’alcuni, li quali pretende~

vano, e pretendono ancora che.
fi possano fcartare le carte d’o
nore, Con la raggione, che chi

scarta, cerca fempre di fare ik

fuo maggior vantaggio, si che

quando scarta una carta d’ono
rc, la scarta folamente, perches

crecde,che la carta fscartata pofJa essere di suo maggior vantag
gio fuori di gioco, perche fe re
sia in gioco teme di perderla

con suo pregiudizio.

Ma (e militasse questa raggione, dovrebbe altresi essere in
sua facolta il dire di non voler
giocar quando uno ha carte cattive in mano, perche in quel cafo cercarebbe ancora Iui il fro

maggior vantaggio,mentre HP
4:
lafciarebbe di giocare,folamente quando aveile timore di perderesma siccome non é giuftizia
che uno,che abbia cattive carte
poifa lafciar di giocare con tutto che tema di perdere aflai, per
che altrimenti non farebbe piti«
gioco, fe folo fi voleffe giocare
sii la certezza della vincita,
mentre il Giocatore deve essere foggetto alla perdita, e alla vincita, cosinon é ginftizia
che uno che rubbi una cartas
poffa scartare 4 fuo arbitrio lé
carte d’onore, perche col scartare carta d’onore viene 4 preZiudicare alla rettitudine del
Bioco. Deve bene il Giocatore
procurare il sao minor male,ma
pero fenza pregiudizio del Zioco, e fenza inganno ; 11 scartare
cartad onore,oltre diche é preBiudicale al gioco,'¢ ancora

frau
 
 

 

 

 

 

 

44 :
trandolento 9 perche chi fcarta
carta d onore fa contro la natus
ra del gioco, quale deve essere
fatto con tutte le carte d’ono¥e, come habbiamo detto di fopra, né deve essere in arbitriod’
un tolo il levare dal Bioco carta
che conti, perche quello che
scarta carta che conti,la fcanta
Jolo pertimore, che ha di per
derla, cosila fcarta 4 folo fine
che non faccia gioco contro di
fe medefimo, e questo.¢ trop.
po pregiudiciale alla parte contraria, quale fi trova di fare meno gioco diquello che farebbe
fe la detta carta d’onore fofie
rimatfta in gioco.

E poi fe confideriamo bene
il cafo, due folo fono quelli che
poffono naturalmente fcartares
cioé quello che alza, 6 quello

che fale carte,. fe ¢ quello ehe
a “
 

4
alza deve feartare’ per avere
rubata qualche ca#ti-d’onore,
0 fopraventi, onde per il vantageio’ che ha avuto di tubbare non puole caufare pregiudiZio al gioco; ma deve rigorofamente fervirfene in vantaggio
del medefimo gioco » Maflime
petche col rubare ha tolto i]
‘vantaggio 4 quello che fale car~
te, ed in questo cafo verrebbe
4 caggionare due pregiudizij,
UNO ioe a quello che fa le carte, e altro al gioco, il che non
e ginfto, née permiftibile .

Se poi ¢quello che s4 le carte,

che deve scartare é nemeno Inj
ha facolta difcartare carta che
conti,perche nella ftesta confor.
mita verrebbe a caufare previu.

dizio al gioco, ¢ non Puole, na
deve fervirfi del Vantageio qi
prendere per fe tutte le Carte

@one~

 
 

 

 

 

|
i
i
|

46
donore che trova nel monte
per poi sCartarne altre d’onore
a suo arbitrio. Stimarei forte
meno male il fcartare Piftesse
che trova nel monte, ma queste
ne meno fi puol permettere 5
perche farebbe un permettere
fervirfi della fola folamente 4
proprio beneficio, ¢ non a beneficio publico; e poi allora non
fi potrebbe pil dire gioco, ¢né
fola, perche Ia fola nessuno 1a
farebbe fe non in cafo, che yi
conofceffe il proprio vantaggio. Ecosi fi direbbe fola ad
arbitrio; ma ficome pare che la
giuftizia del gioco voglia, che
chi gode il vantaggio. di pren=
derfi per fe tutte le carte d’ono
. xe, che fono rimaste al monte »

debba ancora foggiacere al pericolo di perderle, altrimenti tl

Sloco non farebbe ne giufto » NC
: egua~
 

eguale,cosi per togliere da mez20 quest’abuso contro tutta 'e=
quita del gioco, é necessarid
confeffare, che quando fi gioca
con la fola non fi possino mai
scartare carte d’onore,

- E’vero,che a principio il gio~
co fi introdotto con la liberta
di fcartare quelle carte, che più
placevano, ancorche foflero d?
onore, ma ealtresivero, che
queito gioco a principio non fu
inventato con la fola » @ sicco~
me in quel cafo potevano reftare nel monte altre carte d’ono~
Te, Cosi non fi faceva ingiuria &
nessuno il scartare carta d’ono=
te. Ma al contrario quando fi
$10ca con la fola fi sa di certo,
che nel monte non vi poffong
essere rimatte carte d’onore, «
fe pure ven’é rimafa qualche.
duna, quella deve necessaria.

MER
48
mente essere di quello, che fa

- le carte, cosi non deve poi esse
re lecito.ad alcuno il voler mettere nel monte, ¢ fuori del gioco una carta d’onore, perche
quella carta d’onore, che non
gioca, deve fenza dubbio essere
pregiudiciale a qualcheduno
de’ Giocatori, il che non & deve permettere in conto alcuno,

Questo difcorfa ¢ bello, .e

 

buono, par difentirmirifpon. —

dere;ma quella cofa di non con.
tare quando fi gioca.con più, 6
meno carte fembra pena troppo
rigorofa, ¢ forfe anche ingiufta,
perche dal scartare, 6 non fcar=
tare, non ne nafce altro pregiudizio, che quello prova chi
gioca. con pil, 6 meno carte, e
Poi il non scartare proviene folamente da unpuro accidente
di dimenticanza, non essendo
cre
be
aa eee
credibile che uno voglia gioca«

recon pill, © meno carte .

Ma io vi rifpondo, che tutte
le pene nel gioco’ fono ftate inventate per oviare le frodi, che
fi poffono fare, ¢ per essere ob=
bligato alla pena non éneceffario, che sia fatta la frode, ma
basta, che vi possa essere, e quel

male, che per accidente e acca~.
duto basta che fi poffa argumen-:

tare, che possaeffere fatto con
frode; perche in quel cafo fi

prefuppone fempre la malizia,.

e siccome il giocare con più, 0

meno carte puole essere fatto:

con frode,cosi ancorche accada
per accidente deve esservi las
fua peva, quale deve essere affai maggiore di quella, che deve pagare quello che fa le carte
in cafo disbaglio, e questa pena

ferye per tenere attenti,ed ap-,

pli; 4

~
50

plicati i Giocatori,accioche per
laloro diftrazzione non fueceda ognimanounsbaglio, ede
-ancora dovere che sia aflai rigorofa, perche fe fi accorge d’avere carta di più,oltrediche hail
vantaggio di fcartare a fuo ar~
bitrio fa pagare di più lapenaa
quello, che da le carte, cosiin~—
cafo di non fcartare pernon ef
ferfi fervito del fuo privilegio,
deve essere obligato ad una pena di gran longa maggiore,quale é’quella di non contare,

Che poi il giocare con pill, 6
meno carte poffa essere fatto
con malizia evidentemente fi
conofce, perche fe tal’uno avetfe da farfi qualche carta d’onore,‘che li premeffe » ¢ dubitasse
dinon aver forma di poterlas
fare, potrebbe fcartare una, 0 —
due carte per farfiun faglio, © |

cQ
Z
 

Ou

I

cosi ¢fimerfi, dal pericolo di
perderla,6 pure col gettare con:
bello ftudio in terra una delle
fue carti prendere poi all’avverfarij qualche carta di gran confeguenza, e fe non vi fofle pena
verupa potrebbe con tutta faci-=
lita succedere spessiffimo. 

Col non scartare poi fi puole.
falyare qualche carta d’onore,la
quale dovrebbe necefsariamen-.
te perderfi, mentre puol tenerla per Ja fua ultima cartas e per
confeguenza fuori del gioco, e.
cosi farebbe ficuro di poterla fare, Oalmeno dinon potérlas
perdere, e siccome questo farebbe un giocare con vantaggio, econ malizia, cosinon,
deve permetterfi.

Inoltreé obligato alla pena
quello che da più’, o meno carte, ¢ poi tutto il vantaggio ¢

C3 di
52
s| di chi la riceve, perche in quels
| lacarta di più. puole aver avis
| | to una carta donore, e dip *
ada facolta di feartate a suo
, ‘modo, cosie dovere che sia ob"
| Tgato ad una pena maggiore
| chiégioca con una catta di pills:
odi meno, folo per nod efserfi
a fervito del fuoivantaggio=
Chi da più, meno carte ©
obligato alla pena » perche nel
dare le carte puol conofcer
qualche carta » © cosi darne ula
itl, O MEN, fecondo che |i tor
Li | na a conto; onde per oviate?
| | questa frode ee ftato nece stat!
e per eller
non nece!i le carte con
che vi pol

essere .
La pena di chi sbaglia le cat

te fono 20. punti, pet Ja primé@
| cate

 

 

j

ge

 
53
carta,e:dieci per ogni altra,fino
alla fomma d’un refto ; ma fe lo
sbaglio fosse in tutti, allora ft
deve pagare la pena per tutti,
e fi puole arrivare @ pagare un
refto per ogni sbaglio di carte:s
A questa pena e ancora foggetto
il compagno, perche ancora lui
deve fare attento, ed avertire
il (uo compagno, quando fi accorge dello sbaglio.

Chiha carte. di più datele per
sbaglio, ha facolta difcartare
a fuo arbitrio, prima che cominci agiocare, purche con lo
fcarto non fi faccia faglio, per=
che non deve fervirfi di quel
-valtaggio per ammiazzare. ul
RéalPaverfarij, fe poi ne ha di
meno, deve patimente prima
di cominciare a giocare domandare a quello, che fa le car
te tante carte » quante ne ha

C 3 di

 

 

 

 
yr ved

 

 

 

 

 

54
di meno, equefti dovra darlé
diquelle del monte ; pero prima di far la fola mefcolan 0
ben bene il monte, ¢ poi quello
a cui mancano domandera fer”
za vederle quelle carte » che
più le piacera 5 purche non G0”
mandi più del fuo dovere Puole di più darfi lo sbaglio
dicarte inuno, che avefie le
fue carte ginfte, ed obligare Mt
nel cafo quello, che fale carte
alla pena dello sbaglio» perche
fe quello, che fa le carte fi foor~
dasse di dare la carta scoperta
adalcuno,¢ quello, cid non
oftante, avelte le fue carte gil”
fe, none per questo, che quer
lo, che fale carte » non abbiv
sbagliato ; mentre deve oh
dere, che lo sbaglio sia succ®
duto nelle prime ‘carte, ee

gol non dare la carta con
ona
55

ha privato quello del vantaggio

di fegnare a fuo favore tanti
punti, quanti ne poteva impottare la carta scoperta, efe non
foife obligato 4 pena veruna,
verrebbe in quel cafoad efimerfi con inganno dalla pena dello
Sbagho ..

Se poi il Giocatore incominciasse a giocare fenza aver pris
macontate le fue carte, il che
é contro ogni buona regola di
gioco, efi trovasse 4 giocares
con più, Omeno carte; allora
non é pi intempo di rimediare |
al male; ma deve pagare las
pena di non contare, perche
quando uno ha incominciato a
giocare, non é pil padrone di
ritirarfi la carta giocata per
non essere quella più fua, mas
del gioco; Onde in quel cafo
non potendo rimediare all’erros

C4 Ie,

 

 

 

 

 
$6

Te, dovra foggiacere alla peoo

2 Siuftamente dovuta a chi
$10¢a Con più, oO meno carte .
© questa pena puole dirfi
troppo tigorofa rifpetto . al
ompagne, perche trattandofi
Mel fuo interesse, deve avertire
“/40 Compagno, prima ches
Slochi, che conti le fue carte,
© che scartl » fe ha.dafcartare;
e liccome Jyi gode del vantag.
810 della pena di quello., che fq
le carte, se le conta in tempo,
Cosi deve essere ancora lui foo.
Setto a questa pena, fe pér faa
trafchragsine il fuo Compagna
Sioca fenza aver contato. ley
fue carte.

Se fofle nel cafo.d’avere una
Carta di più; allora, ancorches
aveffe giocata la prima carta,
e siaccorge dellosbaglio, prima

quel

Che fosse copertala mano, in i
quel cafo puole tintediars’ af
male, con dire che la carta’.
ehe ha pofta. ful tavolino- deb
gioco é quella, che ha intefg
di fcartare,.e cosi-fe li. deve bo-.
nificare lofearto., e rifcuotere
la penada quello, che ha-fatte
le carte, fe la carta che ha di
più elie ftata data per sbaglio.

Ma fe per accidente nel rifpondere la prima volta dasse un
Re, 0 pure altra carta d’onore
come che quellencn fi poffono scartare ; allora non é più in
tempo a rimediare al male, con
tutto che fosse la prima giocata:
il fimile puol parimente succedere in qualfifia altra carta,,
che non siad’onore’; mentre fe,
avesse per accidente una carta
di più, la quale le fofle ftata da.
tada chi fa le carte, e fi trovaf=
fe una fola carta di fpada,- o.al.

C 5 tra
58
tra cartadidiverfa fpecie, efi
giocasse la prima volta fpada, a
dove quello avetle una fola cartase dopo di aver giocato siaccorgefie d’aver una carta di pitty
in quel cafo non potendodir che
scarta la carta giocata per non
poterfi far faglio,cosi non ha più
facolta di poter scartare; macs
deve continuare il gioco fing
alPultimo con la fua carta dj
pil, e poinoncontare.

Hora che abbiamo veduto jj
modo, che deve tener uno,'che
aveffe pil, Omenocarte,& dge
vere veder il modo, che dovera tenere chi fa le carte, in cafo
che fi accorgesse dello sbaglio
in tempo di poterlo rimediare.
Pertanto fi devefapere, ches
quello, che fa le carte deve»
flar molto attento 4 non sba=
gliare;ma fe mai fi accorgesse@?

aver
59
aver data qualche carta dipia,
o dimeno ad alcuno, deve pro-.
curar dirimediare al male, prima che quello fi volti le carte.
alla faccia,perche dopo lui-non
é piu padrone di quelle carte,né
giova il dire, che fi contino,perche l’obligo di contarle compete folamente.a quello, che fa Ie
carte fino a tanto che lecarte
fono coperte, e non fono vedute ;ma una volta, che le carte
fono vedute, 0 ha data la carta

fcoperta, allora non puole più 
in conto alcuno difporre di
quelle carte; ma deve lafciar
correre, e pagarne la pena, fe vi
ésbaglio, fe laparte fene accorge,enonfare, come fi lufingano molti, li quali quando
scoprono all’avverfarij qualche
carta d’onore, dicono fubito,
contate le carte, credendofi dj
C 6 po 

 

 

  

60
poterli levate quella carta di
onore, con lo sbaglio da loro
fatto, e non siavvedono, che
oltre if non potere aver quell’
onore y farebbero ci più obligatia pagar la pena detta di foptay
ogni qual volta pero la cartes
fofle fcoperta, ¢ veduta dali

-.Giocatori; mentre Pavvifano»
che conti le carte, dove al cone

trario fe non fi raccordasse ‘il
contare le carte, potrebbe darfi
Vaccidente, che fi dimenticafte
dicontarle, ¢ giocafle con una
carta di pil fe maiglic lavesse
data in sbaglio.

Refta ora a vederfi, che regola deve tenerfi in cafo che |
le carte foflero giufte di nus
mero, e non di qualita, e che Ht
Giocatori non fe ne accorse 10a
to che alla fine. del primo ge
co, ¢ fe bene in alcune ee
 

61
fazioni f prattica di non manda
re mai il gioco 4 monte’ col pretefto, chealgidco delle Minchiate mai devono rifarfi les
carte: Io pero fono di diverfo
fentimento, prima perche quest
affioma non & fondata fopra alcuna ragione’, e poi perche me
lo perfuade la ragione ; mentre
siccome il giocar con fola, por=
ta feco per necessaria confeguenza, che tutte le carte di
Onore siano: in gioco 5 comes
habbiamo detto di fopra, cosk
mancando qualche earta d’ono-~

re il gioco fi deve avere per hul-:

lo, e per non fatto’, perche tutte le carte d’onoré hot erano
in gioco, come e*doveresne
occorre, che fi dida, che se'qiiel=
lo, che fa le carte, fe ne accor?
fe prima di eomiliciare aeiocas
re puol far giconoftere da” ears
- ta 9.

 

 
  

 

 

|
ii!
¥
ih
il
iN

 

 

 

 

 
62
ta, che manca, e prenderfela
per fe, come rimasta nella fola,
ancorche fosse cafcata in terra,

© rimasta fuori del gioco per.

accidente, perche io le rifpondo, che il gioco deve essere fatto fenza frode, ¢ dove vi poffa
esser la frode, fe ne deve pagar
Ja pena, e siccome. in quettas
forma potrebbero commetterfi
moltiffimi inganni, quali non
devono permetterfi, cosi in.
quel cafo il gioco fi deve avere
fempre per non fatto; altrimenti fpeffo succederebbe,che quelIo, che fa le carte con deftrezza
di mano, econ malizia gettarebbe in terra, 0 lafcierebbes
fuori del gioco le migliori carte, fe non tutte, almeno qualche
dona, con la certezza, che fen2a pagarne pena, quelle toccarebbero a luinecessariamente ;
; on-=

ee ee ee ee ee ee ee ee ee. ee. ae. ae

ge ee. ot ce ie a ee Se ae
63
nde pet oviare questi inganni
“neceffario confessare, ches
nivolta, che nelle carte del
r1Ioco mMalica alcuna carta di
ynore, Overo ve ne fara alcuna
Ji più per non far nafcere un insonveniente maggiore; allora
ynel gioco sintendera fempre>»
yer hon fatto, fe poi le carte,
-he mancaffero, 6 che crefcefero, non fofsero d’onore; alora il gioco s’avera fempre per
yen fatto,e validiffimo.

Nafce un’ altrodubbio in maéria di fcarto, anche fra primi
Ziocatori di non poco rilievo,la
-efoluzione del quale io per me
timo affaichiara,ma la raggio1e pero é quella, che deve conyincerci; fupponiamo adunque,
che uno de’Giocatori abbia,al-~
-ando, rubate due carte, e quete Pabbia pofte da parte ful ta
yor
 

 

 

 

 

 

 

G4.
volino per doverne scartareo

due altre a suo tempo ; questo

fenza punto piu ritlettere 4 quelle due carte ha incominciato 4
giocare fenza fscartare; fi domanda adesso come debba an;
dare questo gioco, non parendo
giuftitia il condannare a4 pena
veruna quello,,.che ha rubato le
due carte, e non ha fscartato,
perche alla fine gioca con le fue
earte giufte,e fe poi vie qual]che pregiudizio in quel-cafo @
tutto di quello, che non ha scartato,, non potendofi negare, che
Jo fcartare non sia di gran yantageio.

Dovete prima avvertire-co-.

Me fi gioca, perche dal modo

del giocare ne nafce una refoluZione diverfa fe fi gioca all’antica, e fenza fola, alloranoné
fogectto a pena veruna, e deve
ti«=

tirarfi tananzi il gioco, come fe
quelle due carte non foffero in
gioco; fe pot fi gioca con la
fola caminando. fempre con.
Vifteffo principio, che tutte le
earte d’onore debbano essere in
gioco, dico che le carte, che ha
rubate» 0 fono d’onore, Onon
Jo fono, fe non fono d’onore
allora deve prenderfele in mano, e giocare con più carte,e
poi foggiacere alla pena di chi
gioca con più, Omeno carte,
fe non fono d’onore fenza essere obligato a penaalcuna puol
dire d' avere feartate quelles
- fteste, che ha rubato, né occorre dire, che il pregiudizio in
questo cafo é tutto di chi non
scazta, perche fi deve riflettere,
che il gioco nome fatto per li
ftorditi; onde puol essere, che
non abbia voluto scartare per:
timo  

66

| timore che aveva di perderes
i | quell’onore; econ fare lo ftorA dito efimerfi dal pericolo di
| perderlo, il che fe bene di rare
aa accade, nulladimeno puole ac
a cadere, eper questo giufto deve
esser foggetto alla pena del non
contare, non essendo necessario
il commettere la malizia per ef
fer obligato alla pena, ma bastay

che vi possa essere .
In oltre fupponiamo, ches

fi uello che fa le carte abbia preih te nella fola tre carte d’onore »
) ~~ enon ne fcarti folamente che
i due, e poi cominci a giocare +
| dopo d’aver giocato gl’avver™
l farij pretendono che debba 107
i care con una carta di piu pee
| avere scartato una carta meno
| del fao dovere ; ficerca aden?
Mh come debba regolarfi quelte, es

| fo, & aqual pena acne a

 

 

 

 

 

 
  
  

Gy
obligato if delinquente, non
essendo dubio, che questo cafo
‘merita qualche pena. . t

Primadidover venire allare~
foluzione della condanna, fi de~ .
ve riilettere quale sia ftata las
carta fcoperta di quello, che ha |
fatta la fola, perche quando |
Pavverfario fi ricorda precifa~ |
mente di quella carta, allora |
puole riconofcere fe quell’ifter i

“fa carta é rimastanel monte, & |
incafo'che non sia carta d’ono- — |
te, che sia rimasta nel morite, |
ogni qualvolta quello sioca con |
le fue carte ciufte’, allora noné
obligato 4 veruna pena, perche
puol dire, che per terza carta
ha fcartata la fua feoperta, & |
Ogni volta, che non fi poffa reconvenire di fatto, con avvers
. Urlo, che Ja fua carta scoperta
Ron e nel monte, allora se ]j de
ve

 

 

¥

 
 

Lt ve abonare lo scarto fenza per
iy na veruna, ma fe poi alcuno de’
i || Giocatori fapendo di certo la
I fua carta scoperta lo reconye+ nifle di fatto per non trovare la
ty detta carta nel monte, allora
|| dev’essere obligato alla pepe
| dello sbaglio per esserii prefa
i il ‘una carta meno, mentre ogtl
al qualvolta gioca con le fue cats
| te giuftc, enon ha lafciato fuctt
| del gioco carta d’onore non
| ‘puoleffere obligatoa pik.
i ~ Daquefte giufte .¢ veridiche
| refoluzioni ne deriva per neceli fita unaltra, non meno degia
| da faperfi da chi desidera ben
imparar agiocare a questo 10"
co per tutti glaccidentt > C

Mi potessero accadere 1n un ca y
| quafi incredibile, qual que!
|

 

 

di lafciar per accidente una a

iq ta d’onore nel monte, eral

 

 

 

 
69
del gioco, spettando questa necessariamenre a quello, che ha
fatto le carte, di modo tale che
dovera fempre esser confiderata
in gioco quella carta, come ses
fofle in mano di quello, che ha
fatte le carte,per non poter esser |
ladetta carta fuori del gioco; e |
fe quello, che ha fatte le carte |
con ladetta carta avera‘giocato
con una carta di più, fara necef- |
fariamente obligato avoggiacer |
alla pena di non contare,perche |
puol aver lafciato 4 bella pofta |
fuori del gioco quella carta;che: |
noi fupponiamo lafciata per ac
:

 

 

 

accidente, per timore che aveva |
di perderla pigliandofela. |

Giache siamo nella. materia \
dello feartu, e delPalzata, vi di.
ro alcune cofe circa Palzata nes

 

cessariiffime a faperfi, & & che’

. fe quello, che alza nellalzare
las |

)

 
40.

lafcia cadere una carta in tavoJa, quella ancorche non sia veduta dev’essere la fua alzata, ne
& lecito a quello, che ha alzato
il dire non la voglio, come fan...
no moltiflimi, perche per esser
feparata dall’altre fi puol cono~
scere cosi,O sia buona; sia cat.
tiva quella dev’essere necessariamente?llalzata,. ancorches
quella carta caduta ful tavolino
veramente non fosse Pultimas.
carta alzata. Taluni ancora
ufano dopo aver alzato di dare.
alcune carte di quellalzate al
fuo compagno, afpettando a pigliare per fe quella carta, che.
pill gli piace per fua alzata, ma
questoné meno fi puol fare,perche col tatto fi possono facilmente conofcere le carte; così’
per ovviare ofn’inganno ¢neceffario prima di venir all’alza
tay
k—

wt
ta, che fi spieghi fe vuol lafciare carte di sotto al fuo compa-~
gna, e che dica quante ce ney
vuol lafciare, accioche quello,
che fa le carte non abbia occafione di dolerfi, che le carte
poffono esser conofciute in quella conformita, che fi prattica
appunto quando fi vuol prendere per fua alzata, ola prima, 6
Pultima carta, le quali perche fi
poffono conofcere non fi poffono pigliar mai fe non fi dice prima in tempo, che fi mefcolano
affieme le carte, perche unas
volta ch’abbia fpiegato anima
fuo non puol pil recedere da
questa sua deliberatione,ma de
ve necessariamente efeguirla.
Se poi nell’alzare cadessero
di mano tré, O quattro carte
alanes allora per non met.
terfi in difputa quale fofle Ia

pris
72,
syima, foflero comunque fi sia
! q

le carte fcoperte, allora fi deve
di nuovo rifar le carte,e di nuovo alzarle, ma fe delle carte,
che cafcaffero ful tavolino una
folo fosse la fcoperta, e Valtre
tutte coperte,allora la fcoperta
fara la prima carta dell’alzata,
e poi succederanno Valtre carte
coperte.

Dopo che avrete bene imparate 4a memoria tutte queste regole, non per questo avretes
imparato 4 giocare, mentre fin?
ora non fi e difcorfo d’altro,
che delli accidenti,che possone
accadere nel fare le carte ; ora
@ dovere cominciar un poco a
difcorrere del gioco, e del modo di pratticarlo.

Dovete dunque fapere, che
il Giocatore ¢ fempre obligato
arifpondere di quella Pee >

che

eR

Ss mA A = WH.

 
 

73
she fi gioca, a fegno'tale, ‘che
e uno de’Giocatori giocafle una
arta di denaro,e l’altro rifponlesse tarocco, fe poi aveffe in
nafio carta didenaro farebbe
ybligato a pagare agl’avverfaii, fe questi fe ne accorgessero,
in refto per ogn’uno.

Ma siccome questi non fe ne
oflone accorgere, fe non quanloviene giocata quella carta,
er Ia quale deve pagarfi la pea, & allora per lo più il delinuente fuol negare il fatto, che
on: puol ‘provarfi in altras
orma, che col riconofcere lo
baglio, quale fe veramente fi

rova nelle carte giocate, non

ffendo credibile, che uno nel
‘ioco voglia accufare un altro
’un delitto falfo, quando queto fi ha da provare con la pura
erita del fatto, col veder tut:
D te
 

 

 

 

74
‘te le carte giocate, allora fi deye prefupporre, che lo sbaglio
VP abbia commeffo V accufatos
non bastando mai in alcun fatto
la fola negativa del delinguenté
a provare il contrario, altrt
menti non fi darebbe mai cafo:
che li delinquenti foflero obligati A penaveruna, © cosi po
trebbonfi commettere a mane
falva delle frodi, ¢ dell’ i
gan. |
Pofto dunque, che resti pro

yato ildelitto non vie dubi0:
che deve pagarfi la pena» Ja
quale dev'essere tutta a carict
del delinquente » al contrarit
ditutte le altre,cbe fi pagano if
compagnia, ela raggione I
perche nell’altri cafi i1 comps
gno puole, e deve avvertines?
in questo per non faper a f
carte possa aver in mano?
come

 

| |
7
Coy.) i
compagno non puolavvertirlo; |
onde lui non dev’esser foggetto !
» aduna pena per un fatto di cui |
lui @ affatto innocente, fe poi |
in vece di denaro dafle coppe, 6
qualche altra cartiglia, alloras
pare, che la convenienza voglia Payvertirlo, perche non
fe glipuole fupporre malizia, |
come probabilmente fi puole |
fupporre in uno, che fi faccia |
una carta d’onore, e percon- =|
feguenza non fembradegnodi |
pena, |
Il non rifpondere adequatamente diquella fpecie, che fi
gioca fi chiama rifiuto, essendofi fempre obligato a rifpondere di quella specie,che fi gio- |
ca, fino che fe ne ha in mano, fe
poi di quella fpecie,che fi gioca |
non fene ha più, allora é obli- |
gato 4 rifpondere Tarocco per ‘|
D2 fine, |

 

Tecate dis baineel cae ad cis eat Se Se ES

 

 
 

 

 

 

 

76 .
fino, che ne ha, e quando no
ne ha più puole dare che carta
vuole a fuo arbitrio, quando fi
gioca di qualche fpecie, che lui
non abbia, ma fene ha deve cid
non oftante rifpondere adequatamente per non essere pol obli

gato alla pena del rifiuto; pe!

efimerfi pero dal pericolo de
rifiuto, puole gettare in tavol.
le fue carte scoperte, lafciand'
in arbitrio de’Giocatori il pre”
derfi che carta vogliono; ™
in quel cafo le fue carte piu no!
giocano in quella data di carte
enon puole pill prendere
alcuna, essendo quelle carte tub

te perdute .

Dovra pero avvertire que!
lo, che vorra gettare in favo"
le fue carte di non gettat e

quando avra ancora un Re
Je mani, perche ancot AN

 
te DN Die Se ace

~

Sa

 

 

a
ha per perduto,e fe gli avverfariinon havessero da poter pren~
dere, che una fola volta,e di più

fi giocasse d’una specie diverfa - .

della quale quello che ha gettate in tavola le fue carte ne
avetle,cio non oftante deve dare il Re in quella mano, perche
non essendo obligato 4 rifpondere adequatamente poffono eli
avverfarij prenderfelo liberamente.

Se qualcheduno rifiuta per
prender qualche carta d’onoré
alla parte cotraria,oltre la pena
delli due refti, che deve pagare,
deve altresi alla fine del gioco
reftituire all’avverfarii la carta,

che li ha prefo, anzi deve refti- 
tuirgli tutte laltre carte ad ef.
clufione della fua, perche altrimenti farebbe un far vantaggio
al suo compagno con pregiudi
3 Z10

 

 

 

 
 

 

 

 

 

 

 

 

zio degl’altri, e ficome il gioco
non permette, che con frode f
pregiudichi a neffuno,cosi ¢ dovere direftituirla fempre;quefa
regola pero non milita a favore
del refintante, perche fe quello,
che rifiuta petde una carta di
onore non puole alla fine domandar la fua carta, tutto che
fii obligato alla pena, perche

-doveva non rifiutare, e poi puol

essere, che abbia impedito alla
parte il farfi una carta di maggior importanza, la quale abbia poi necessariamente da perderla, ¢ cosi fe ha perduto una
carta @onore ne incolpi fe fteffo,¢ per la fua trafcuragine ne
naghi la pena .,

ii hte, accioche pofta dir ft
tale,é neceffario, che quello»
che lo commette abbia rigioca
to, ne basta, che la bafe sia 60”
pet
ee
   
 
  
   
   
 
 
  
  
  
   
  
  
  
  

. 79

perta, come vogliono alchni,
perche ficome non fi puol tenere le mani ad alcuno, cosi puol
essere, che alcuno più follecito
degValtri copra prefto la bafe ;
fe poi la bafe fi cuopriffe da chi
commette il rifiuto,allora basta
anzi ancorche la bafe non foffle
coperta, ma avesse giocato immediatamente lacarta, per la
quale ha commeffo il rifiuto y
fe Ini non avvertifce il rifluto da
se prima di rigiocare quella carta tanto ¢ obligato alla pena del
rifiuto, fe perd Pavertifce da fe
prima di giocare Ja carta,allora
puol rimediare al male fenza
pena veruna, ma per rimediar-~
lo fenza penaé necessario prima di rimettere in tavola la»
catta, per laqual’e succeffo il
tifiuto, che firipigliin mano il
fuoTarocco,altrimenti fe lui la
D 4 scie
 
 

 

 

 

 

 

sciera in tavola il Tarocco,e 1a”

sciera parimente iu tavola la
carta del rifiuto, non fi dev
credere, ch’abbia pofta in tav®
la la carta del rifiuto con idé:
di riagiuftare il rifiuto, ma be™
sicon idea dirigiocare, ¢ ¢0%
allora ¢ obligato alla pena d¢
rifiuto,perche ogni volta che
carta é in tavola fcoperta i1Gi0

catore non é pil padrone
quella carta, ma quella fi dev!

intendere giocata, fe pero toe
ca giocare a vi, in quella guili
appunto, che fuceederebbe §
quel Giocatore, che ave fie la.
mano, & avesse una carta di pl!
fe nella prima carta gioca
una cartaccia di fpada fcoperté
enon avefe di fpada altro ©

quella fola carta, ¢ prima chi
tutti liGiocatori avessero rifp

fto s’accorgesse d’aver una a
a

 
 

ach, Svat i pink > eerie ln aa eh ee aa ite = ene aS ND WS | EN << st chil CAM ky

oy
ta di più, tutto che la bafe non
fofle coperta, non per questo
puol piu fcartare, e la ragione
fié, perche dopo che fi ¢ incominciato & giocare non ¢ più
luogo allo fcarto, ed intanto fi
ammette lo fearto nella prima
mano, perche quello che deve
scartare puol dire, che la carta
potta ful tavolino del gioco elie
FP ha pofta con intenzione di
scartarla, ma quando quellas
earta ¢ tale, che non puol esser
scartata, allora non fi ammette
piu fearto, perche quando la
earta ¢ in tavola fcoperta Ini
non e più padrone di ripigtiar
fela in mano, e dire non ho gio-~
“eato fe non incafo divoler ri
mediare ad un rifiuto,ma per rimediarlo deve prima ripigliarg
in mano la fua carta, che fa ful
tavolino, alfrimenti fe toccara

D 5 a S10

 

 

 
 

 

 

 

 

82

& giocare 4 lui,e fenza ripiglial
prima Valtra carta, ¢ lafciera 0
tavolala feconda carta, allor4
sintendera rigiocato, ¢ rifiuta:
to,ed oltre il pagare la pena de
rifiuto dovra raggiuftare 1a ba
fe obligando il rifiutante a 1%
pondere adequatamente » eth
pigliarfi in mano il suo Taroc
co: fe poi dopo il rifiuto folls
paflato pil d’una mano, alloré
fenza pil agiuftarfi la bafe, !
continuara il gioco fino alle
fine. i
fed é ben ginfta questa rage
ne, perche con il riftuto alcunt
yolte fi puol caufare il preg
dizio dipiù ditré refti, com
diremo appresso, coftando
refto di fali fessanta punt on7)
non ¢ maraviglia, che chi 1l
ta fii obligato alla pena © 7

folirefti per ogni volta. ae

 
wet OT a Ge ht Po eG OSE

83

fiuta, edi più fii obligato a rétituire la carta d’onore prefa
aglavverfarii conil rifiuto, ma
deve perd avvertirfi, che non
puol dirfi rifiutato piu di una
fol volta quando uno comincia
Arifiutare, ¢ continua rifiutare fine allultimo, tutto che
i rigiocafle di quella fpecie per
la qual’é successoil rifiuto quattro,  cinque volte, poiche non
fi puol mai dire atto confumata
fino a tanta, che non fi fcopra
e ficome il rifiutante non é obligato a pena veruna, fe primas
non fi scopre i rifiuto, cosi con
la fperanza, che non debba scoprirfi puot continuar a rifiutare
fino all’ ultimo; Se pot fi scopriffe prima che finitca il gioco, ¢ fi rimediafe nell atto ifteffo,
che fi fcopre, e quello cid non
oftante rifiutaife la feconda vol
D 6 ta

 

 

 

 
 

 

 

 

 

 

 

84
ta fopra l'ifteffa fpecie » alfora
deve di nuovo pagar la pena, se
viene scoperto, perche essendo
il fecondo rifiuto atto nuovo se
diverfo dal primo, dev’ancor4
Ini avere la fua pena come i
primo.

$9 bene, che a prima vitta vi
fembrara troppo rigore il dovet

pagare la pena, ¢ reftituire 14

carta d’onore, quafi che per um
folodelitto fi abbiano a pagaté
due pene; ma fe confiderareté
al pregindizio, che puol caufare
fon certo, che direte, che ¢

-fomma'giuftizia, perche la pe

na pofta al rifiuto é per punile
la malizia di chi lo commettes
ed ilreftituire Ja carta donore
fi deve fare, perche non fi pue
con frode rabar quella cart
aglavverfaril, e poi col rifiuta
-re alle volte fi possono aver
pl

 
 

5; —

più di tré refti, e fe il contrario
non fe ne accorge, quel che rifiuta non é foggetto 4 pena alcunas onde in quel cafo tornarebbe a conto il rifiutare, poiche alla fine non pagharebbe
che dui foli refti.

 Figuratevi d’avere in mano
da cinque, 0 fei carte folamente fra” quali abbiate li Re @
oro, la Tromba, quattro altri piccioli Tarochi, ¢ che toc
chi agiocare 4 voi. Voi gioca<
rete il R¢ d’oro, il fecondo che
fi trova avere il 30, in mano
fenza averlo mai potuto fares
‘per dubbio di non poterlo più
fare, Oper altro fine,non hayendodenari gioca il 30. ilterzo
che ha il Sole, col motivo, che
il 30. e una grancarta, per il
troppo gioco che puol fare yj
mette ilSole, Pultimo poiche

ha

 

 

 

 
 

 

 

 

ha il Mando,tutto che abbia denaro per non perdere iltrenta»
e prendere il Sole gioca il Moan
do, ¢ rifiuta,quale per altro non
farebbe più in ftato di poterfele
fare. Hora con questo rifiuto ha
jmpedito la verzicola di Trom
ba, Mondo, ¢ Sole,di dieci, ve
ti, trenta, ¢ quaranta, ¢ di tren
ta, trentuno, ¢ trentadue, ed 10
vece ha fatto verzicola di Mot
do, Sole, ¢ Luna,. dt yentiott
velitinove, ¢ trenta, edi tre
Re; ora ditemiun poco quante
importa questo rifiuto, ¢ fe fa
rete bene il conto,trovarete che
importa il fuo conto giufto.Ducento ottanta quattro punti che
yo! dire quattro reftis quaral
taquattro punti; ora non Vvl pat
egli giuftizia che chi rifiuta > 0”
tre la pena delli due refti,debhs
sacora essere. obligato or :
ae

hia

 
1 OW
tuire le carte d’onore prefe col
il rifiuto,bastandole folo Vavervi impedita. 1a- verzicola di
Tromba Mondo, e Sole, altrimenti bifognarebbe confeffare
che il rifiuto portafle più tosto
vantaggio, che pena al rifiutante,quando per altro la pena fta.
ta pofta per castigo, enon per
utile del delinquente; e siccome
in questo cafo andarebbe neceffariameate reftituito il Sole affieme con tutte fe altre carte ad
efclufione del Mondo, cosi ad
essetto che nan succedano sconcerti, é fi abbia ogni volta
confiderare il pregiudizio che
puol portare il rifiuto é neceffario ftabilire per regola certa,ed infallibile, che ogni volta
che fi rifiuta, e col rifiuto
prende qualche carta d’onore)
dalla parte contraria quella con

tut
 

 

 
 

 

 

 

 

 

tutte Paltre carte debbano fempre reftituirfi, edi più pagarfi
fa pena,mentre non fi deve ma!
permettere cofa che con ffode possa esser’in vantaggio di
chi ka commette. |

Avvertite perd che nel giecare il Matto mai fi puol rifiutare per parte di chilo gioca, al
corche aveffe in mano altre
carte di quella fpecie che fi glo
ca, perche quella ¢ una cart
che deve essere giocata da mat:
to, cioé quando fi vuole, c fi
tutte le figure che vuole chi J
gioca,con questa fola diftinzion®
che mai puol prendere, anco™
che vi foffera in tavola le catté
più inferiori del gioco, 4 fegn?
tale che fe tutti foflero cafoat!»
cioe havefsero potto le fue cart
in tavola, e chi ha il Matte
tenefse per ultima carta, ee j
tie fecatte intavola più hon»
giocano, in quel cafo non po
tendo il Matto far bafe, s'inten
derebbe che Pultima 1a facefse

uno di quelli che hanno gettate
le fue carte in Tavola .

Questa carta del Matto non S
vero che debba fempre giocarit:
per Tarocco, come commune=
mente s'ingannano molti, perche fe cosi fofse bifognarebbe
dirloTarocco,ed efsendo Farocco dovrebbe efsere nécefsariamente a tutta la Cartiglia fuperiore almeno; il che non e vero; onde fe non éyero che sia
fuperiore alla Cartiglia, non»
puol dirfi Tarocco, e non potendofi direTarocco,deve poterfi
giocare quando fi vuole .

Puole 4 queste gioco darfi P
accidente, che quello che ha il
Matto in mano mon abbia occa
fio
 

 
 

go
fione di pigliarmi, e fi trovi alla tine del gioco fenza carta da
poter dare in vece del Mattos
ed in quel eafo deve dare il Matto iftefso, e tutto che sia carta
donore, non per questo fi fegua
1a Mortespercheé carta che non
puol mai morire, fe poi fofse
obbligato. a dare in vece del
Matto una carta @onore per
non averne altre, allora si, che
fi fegna la Morte di quella cars
tad’onore. Accadé una volta»
ache quello che aveva il Matto
aveva ancora le Trombe gi0-.
candofiin tré ad ognuno per fe,
non fece altrabafe,; che quella
della Tromba, con la quale prefe i] Sole, ed il Mondo; domandava allora Pavverfario la carta
del Matto, e quello voleva darle Piftefo Matto » come meno
Pregiudiciale a fe ftelso » per lo.

che
1
che poftofi al dubbio in diteuta
i communemente da’Giocatori più periti,e piu efperti rifoluto, che fofse obligato a dare una
carta a fuo arbitrio,ad efclufion
Jel Matto, e della Tromba, con
‘1 folo motivo che queste dues
carte fono imperdibili; onde
non puole alcuno privarfi d’una
di efse a fuo arbitrio per fuo
maggior vantaggio, e pregiudizio dell’altrt .

Dovete pero avertire, che in
un folo cafo non ¢ lecito giocar
4 fuo arbitrio il Matto,perche fi
tratta di dover rifpondere carta
necessariamente obbligata, e»
questo succede quando uno de’
Giocatori giocafle per la prima
volta d’una Cartiglia, e quel
che viene appreffo non avendo
di quella fpecie che fi gioca met
tele Tarocco, allora quello che

ha
 

 

 

 

 

 

 

- teffe esser’fagliato il suo Rey

92
ha il Re,fe non ha ancora g10¢
to, necessitato a darlo, anco
che aveile altre carte di que
specie, 6 pure avesse il Matt
ma questa neceffita non obblig
fe non chela prima volta che
gio ca di quella fpecie, ¢ prit
che sia in tavola ilRe, sia ven!
to unaltro che habbia rifpo
Tarocco,fe poi o per inaverte!
za, © per genio quello che ha
Ré non lo ha voluto gioca
ma ha voluto più tosto gioc#!

un altra Cartiglia della mee
ma fpecie col dubbio che lipt

a

lora non é pitiobbligato #

Jo, fe non quando non have
più di quella specie, e fi trovall
fenza Matto; mentre fe fig!
cafse di quella fpecie per no
rifiutare, allora deva dare iB
per necessita . Più
a,
Parmi che tutte queste siano 1]
regole generali pil necessarie |
ad impararfi per ben appren- |)
a dere questo gioco, ¢ siccome 10 |
i 4 principio mi fono ideato di i
4 defcrivervi folamente le regole |
~ generali, Cosi crederei d avere |
M Abastanza fodisfatto al mio obli- |
" go, ee fe voi imparerete bene a |
0 memoria tutte queste regolen |
“ potretedire converita,chein- =|
i tendete il gioco, sebbene non
 saprete giocarlo a perfettione,
mentre il pretendere d’impa- |
rarvi questo gioco a perfettione 1
fenza una ben efatta prattica, |
“  farebbe temerita, essendo le re~ gole del giocare diverfe das
¢ quelle del gioco, nafcende que¢ = fte dalli continul, e varii acci1

 

denti, che accadono, e siccome
di queste ne @ fola maeftra la
prattica, cosi ne hd lafciato la
cus

 
 

|

Ha cura ad essa per ben impata
a vele; Le regole da me fino
| defcritte, come che apparte
Lei gonoal puro gioco, fono inv
a) riabili, ma quelle del giocaté
| poffono,anzi fi devono alle Vv!
Vt ei te variare fecondo le covtl
‘aia genze de’giochi, che accado?
Lil ii Dird bene peronon per mo!
| i vodi regola generale, maP
ai raggione del modo di giocat
Hh che &fempre meglio di farf
onori alla mano, fe sia possibil
| maflime lipiu gelofi, ma
|| yolte é più efpediente il git
HHI verfo del compagno, tutto ©
rn  sia quafi ficuro di perderne 4
Lh tl cuno,quando per altro con qu¢
Hi la perdita fifpera di fare maggior
vantaggio a se stesso, 4
it volte ancora potendofeli fa
i alla mano, ne meno¢ dove
i farfeli, perche con no? far
i puo

 

 

 

 

 
95
puo fperare di far un gioco mieliore » ma tutto questo dipende da quelli accidenti, de’ quali
ne puole essere fola maeftra la
prattica, & il pretendere di descriverli farebbe pazzia .

Le carte più gelofe da farfi
fono quelle, che possono fare
maggior gioco, per questo uno
30.35, S0le;e Papa tutte tre fono gelofiflime, perche queste
fono le chiayi di molte verzicole, cioé a dire, che possono fare
delle verzicole aflai, pero bifo~
gna cuftodirle più dell’altre, ¢
procurare fempre di farfele. |

Ad un faglio, & una feconda
fi puol paflare qualfifia carta»
gelofa,quando pero non vi siano
scarti, ¢ non vi sia forma di poterli fare con pil commodits

alla mano, ad una terza di raro
fi pafla, ma quando fosse fopra
Ta
 

 
96 /
Tarocco,con piii giuftizia fi puole arrifchiare, e fe poi la neceffita del gioco Vobligafle a pat fare,quaniunque non fosse fopra
Tarocco, fi deve fempre procurare, fe fi pud, di paflare una
carta dVonore grofla, accioche
per pigliarla ce ne yoglia un’altra, ma fi deve pero ayvertire »
che non sia carta troppo gelofas
perche fe é carta gelofa l’avverfario, fe puole, ci va ficuro ; se2
poi non e gelofa-per tema di
non perder Ini, forfe non ci andara, enon lacoprira, e questa
diverfita di regola V’ha da imparare dal modo digiocar deg!’
altri, e la difpofizione delles
carte, che fi hanno in mano »
mentre da quelle devefi conoscer il modo di contenerfi. |

Se poi il gioco andasse in fac~

cia al suo compagno, ¢ fofle,

ficuec

=e

; 97
ficuro, che lui deve rifpondere
Tarocco, allora fi puo girare
a lui qual fi sia carta d’onore,
anche delle più gelofe, fe perd
hon fi potesse dubitare,che giocando quella carta poteffe poi
impedir 4 Jui qualch’altra carta
di maggior confeguenza, per~
che in quel cafo verrefte a perder tutti due, come per ragione defempio fe voi girafte il
trenta, & il voftro avverfario
giocafle il Mondo, & il voftro
compagno aveffe in mano il Sole, ecco che voi perderefte il
trenta, & il voftro compagno
non puol fare il Sole, mettendo
a rifchio di perderlo,come probabilmente puol accadere, e»
cosi vi mette a rifchio di perder
tutti due, due carte gelofiffime;
pero nel girare bifogna ancora
andar guardingo, ¢ prima di gi
| rare
 

 

 

 

 

 

98
rare le carte più gelofe bifogna
procurare di girar carte gelole
si, ma meno gelofe,ad essetto di
scoprit il gioco » poiche fe aves
fe da farvi una verzicolas © di
pir aveffivo il trenta » & il Sole
prima deve girare qualch¢
carta della yerzicola,e tutta all

cora fe sia di bifogno» © quana¢
yedete, che non va sula carte
di mezzo della verzicol e fe
gno, che non ci puoles ed allo
ra fi puol girare il trenta, fe p?
la voftra verzicola fofle di Sole
Luna, ¢ Stella, allora 100 mm)
lita più questa regola, pete
vor avverfario cercara femp!

di rompere 4 voi la yerzicola
farla lui; cosi in quel cafo 1
di fat.

vra fempre procurats
Sole, ¢ la Stella» ¢ perderé p

tosto la Luna "Nel girarle carte » che fare
3 ¥ yvert
————
aS Od =— 1 wv geet, OF OS ee a ie

=

 

    

99
verzicola bifogna andar molto
guardingo; ma quando non fi
potefiero far alla mano, e che
bifognasse girarle fi deve procurare di girar quelle per le prime, che non poffono far agl’av~
verfarj, e quelle, che possono
far verzicola agl’avverfarj avete da procurare di farvela alla
mano, o'pure di farle con meno
rifchio, che sia poffibile .

Quando voi a principio averete verzicola dovrete procurare giocando di farvi prima le»
carte pili gelofe, e le carte pil
gelofe delle verzicole fono le
carte di mezzo, ad efclufione
dell’uno, ch’é gelofiflimo, perche fatte le carte di mezzo, alJora non fi puol più. perder la
verzicola fe fofle di quattro carte, perche col perderne una
fa altra alla mano; ma prima

E 2 pero

 

 
 

 

 

 

100
pero di girar la carta di verzic
la fideve procurar di toglier ¢
mano al fopra mano tutta [a
cattiglia, ad essetto, che gio
cando esso non siate obligato |
rifponder cartiglia, ma possiat
farvi le carte a voftr’arbitrio
perche altrimenti correrefte l
rifchio di perdere due carte
verzicola; onde prima e nece!
fario, anzinecessariiffimo d’im
parar a giocare Ia cartiglia; ©
per ben giocarla fideve proc
rare di giocar prima tutta que”
la cartiglia, che fi pud fuppor!®™
che quello, che fa fopra ma?
abbia in mano, e fe viaccotg™
fte, che di qualche fpeci¢ 90?
ne havesse più, tutto che vol #
avefte affai, dovete fempre pt
curar digiocarla per Pultim??
accioche voi giocando quel

gartiglia oblighiate il voftro 1”
e pra
l“

a eral SS

eoT rw

Pultima mano, la

10f
Pra mano a giocarvi in faccia,
e voi farvi quelle carte d’onore,
che più vi piaceranno.

Questa regola perd non é

¢mpre certa ye non deve pratticarfi ognivolta, ma deve,
pratticar folamente in cafo,che
Voi avette degl’onorj aflai da
farvi, enon pefte la forma dj
poterli fares fe poiavette pochi
onori, O pure avefte altra for.

soos eT Sc eau naa yh
he ar potcrirrare + anora dove, .

te procurare prima di glocare
tutta la cartiglia, della quale ne
fofle faglio jj voftro fopra mas
NO, accioche non Poteffe poj
farfi g?onorj alla Mano, ma dovesse farli tutti col suo rif.
chio,

Di più fe voi avefte Ja Trom.
ba vi fervira ancora Per potey
con pill facilitg artivare a far

quale conta

3 die.

 

 

 
A 2
dieci punti, pero bifogna diftin |

16

guer prima il gioco per faper .
ben giocare la cartiglia, poiche,

alle volte é più efpediente i]

iocare prima di quella cartiglia, della quale fe n’ha pitin
mano, « alle voltee piu efpedjente ilgiocar diquelle, che
fe ne hameno.

Tutto il maggior ftudio, che
fi deve fare in questo gioco é
@ impedir agravverfaij it pos
terfi far gVonori, fi che dovete
fempre ptocurare di non la~
sciarglieli fare alla mano, e>
questo lo farete col procurare
fempre di giocar quella cartiglia, che lui poffa aver in mano, e quando fofte ficuro di non
poter giocare cattiglia, allora
dovete procurar di non essere
piu obligato a giocare, fe perd
non vi ci obligafle qualche car—
, ta)

 

angl Saal © Pais; ell tte eS ee Bete Lee ee kp

 
 

103
ta gelofa, la quale vi obligasse
a prendere per farvela; poiche
tutta la bellezza di questo gioco
é il procurare di far la caccia
all’avverfario, e per farla alcune volte torna a conto i! perdere qualch’onore, per pren~
derne delli maggiori .

Ma prima di procurare di far
caccia all’altri,¢ più necessario il
difendere se stesso, ‘pero dovete
fempre procurare, che la giocafa vada in faccia al voftro compagno, accio possa farfi alla mano quelli onori, che fono piti gelofi, e che li possono portares
più pregiudizio,e maggior van=
raggio all’ Avverfarj, mentre in
juefta forma provarefte dues
rantagei, uno di tenere in fog«
rettione linimico, e l’altro,che
| voftro compagno fara quelehe

ruole .
Eq Quan
 
104
Quando avete Ja tromba,tut
tala vofira maggiore attézione |

deve essere ditenere in fogeet
tione l’inimico, e fe potete far—

lo, dovete far in modo, che Iui

non possa accorgerfene; perd 

non fta bene il fequeftrarfi 4

principio un Re nelle mani, ma

dovete veramente giocarlo, fe
perd non potefte credere di po~
terlo falvare col mezzo del vo~
firo compagno nel progresso
del gioco; fe poi il faglio vi fof=
fe fopra mano, allora potete poi
tenerlo con pill giuftizia, ma fe
é sottomano., allora e cofa molto pericolofa a riufcir bene,perche date campo alPinimico di

 

paffare sti quella cartiglia ogni.

volta che carta vuole, pero &
meglio in quel cafo il giocarle
fubito.
A’ principio quando havetes
da

4
es
10g

da prendere con Tarocco doves
te procurare fempre di prendere con li Farocchi più grofti,ma
dovete farlo in modo di non dare ammirazione all’Avverfario;
altrimenti vi ridurrefte alPultj* mo con tutti li Tarocchi grofii,
e farefte per confeguenza gb]jgato a prendere, e mandare las
giocata in faccia all’ Avverfario
il che e un’errore folenniffimo «
mentre alle volte non folo nop,
fi deve prendere con Tarocchi,.
ma ne meno con onori, essendo
più efpediente il perdere un?
onore, che prendere, perche in
quel cafo fivicne 4 guadagnar
più di quello nea fi puol perdere, ma di guetta regola Ia fola
attenzione ve ne puol esseres
maeftia .

Se con la Tromba avefte ancora della gran cartiglia,¢ nom

: ES avele
106
avefte forma di poterla fcarta~
re, allora dovete fare if conto,¢
vedere, fe vi puole riufcire dt
giocarla tutta con fperanza dt
arrivare all’ultimo con la tromba, e inquelcafo dovete pro~
curare, e far tutto ik possibile di
giocare prima tutta quella,della
quale ne havete maggior quan=
tita,ancorche fosse certo di fare:
il giocodellinimico, ¢ NOD faxe, come fanno alcuni, f quali
non la vogliono giocare fino all*
ultimo per tema di fare il gioco
dell’ Avverfario, e non rifletto~ 
na, che fe non la giocano.a prin
cipio, bifogna giocarla in fine 5
& alloraé peggio, perche oltre
di che ficorre il rifico di cafca—
xe con la Tromba 4 mezzo gio—
co, fi da campo all’inimico da
farfi quelle carte d’onore » che
forfi non fi farebbe fatto a prin~
Gi 5

|
 

10
cipio, e Cosi fi toglia l’occafione
di poterli prendere qualch’ono=
re, che necessariamente haverebbe perduto..

Con. la tromba ée fempre bene
tenere ancora un Tarocco gro&
fo, perche in cafo che fofte obli«
gato a giocare la Tromba per
motivo di prendere qualch’ono.
re, potete ancora avere la fpe=
ranza di fare Pultima mano, la
quale conta dieci ponti, ma avVertite 4nontenerne più d’uno.
per non essere obligato a prendere, e quando non aveftte al-.
tro, che un’onore di carte groffe, quello ancora dovete tenere,.
ma. dovete pero procurare, che
quel’onore non vi faccia verzicola, 6 non la faccia all’Avver-=
fario, perche in quel cafo é bene farfela prima.

Di piu fe avefte la Tromba,.¢

E 6 avet-.
Bi ‘708
avefte da fscartare, dovete regolare lo fcarto in modo, che vi
torni più acconcio, e dovetes
prima riflettere alla quantita, &
alla qualita del?onori, che havete nelle mani, & ilmodo di
farli, e poi regolare lo scarto it
modo di potervi fare li voftri
onori,e dare foggettione all’ ini-~
mico; ma prima di ogni cofas
havete da riflettere al danno vo| ftro, e fe conofcete, che vi poffa tornare in acconcio il (carta| - ge di quella cartiglia della qua| | il le ne havete piv, lo farete, altri| Hl menti scarterete di quella, delHa la quale havete meno, e queHal ftaregola ha piitosto per fuo
| fandamento la prudenza, ches
ogni altra cofa .
| kl Mondo poi vi da Vifteffa regola della Tromba tanto nello
fcartare, quanto nel modo def

gl0~

.
k————e
199
giocare con questa fola differen:
za, che il Mondo é inferiore alla Tromba, e per questo fi dice
per proverbio, che il Mondoe
fatto per perderfi tutto, che fi
debba procurare di perderlo
meno che sia poffibile; e fe bene
fi fuole per il più portare questo
fino alla fine del gioco, cid non
oftante questa regola non é femprevera; petche fi deve folagiente tenere fino atanto, che
uno ha fperanza di fare con il
medefimo un gran gioco cok
prendere qualche carta gelofas
all’inimico, 0 pure per difefa
del fuo compagno, il quale ha
| qualche carta da farfi, e non.
puol farla, che col girarla folamente, ma quando man¢ano
questi due motivi, allora non e
più raggionevole il tenere il
Mondo fino all’ultimo, e per
gue-=

 
 

10
questo vi e neceffario oltre l’attentione per ben faper giocare
una buona memoria, perche bifogna ftar attenti 4 tutte le carte d’onore, che fifono giocates
e ricordarfi di quelle, che fi hanho ancora da fare per vedere,
fe torni il conto iltenere a rifico il Mondo ; perche la fperanza
delPonore, la quale puole obliGare a tenere il Mondo, deve»
essere tale, che eguagii il rifico
della perdita, fe non intutto,,
almeno nella maggior parte, al.
trimenti non torna conto il tenerlo col timore di perderlo.
Ad essetto che il rifico paris
gli deve esser tale, che fi speri
di poter togliere all’inimico
qualche carta di gran confeSuenza, la quale quantunques
bon importi tanto, quanto pud
le portare il Mondo, importi

PS"
Bh tecnica
~~ 8 1 rt
perd affai, 6 pure per tema
-che il compagno fuo non abbia
da perdere qualch’altra carta,la
quale possa fare un gran gioco »
6 pure che abbia Ia probabilita
di potere co! Mondo fare per fe
qualche verzicola, col prendere
qualche cofa d’onore all’inimico; altrimenti fenza questi motivi nop torna mai conto 1 tenere-il Mondo finoal fine, fes
non in cafo che fosse certo: di
non poterlo perdere, che per
un qualche accidente impenfato; &alle volte é più efpedien-:
te di farfi il Mondo, ancorche fi.
abbino inmano altre carte di
onore da farfi, perche ¢ aflat
piu quello, che fi perde, perdendofi il Mondo di quello, che»
nole guadagnarfi, prendendo
ancora qualche onore all’ inimi-.
co; Jaonde perche il-rifico nom:
e egua 

 

 

 

 

 

 

 

112
éeguale, né meno deve eller
eguale il pericolo.

slSole, che é carta affai p!
gelofa del Mondo. deve essere
anche giocato con pili giudiai¢
prima perche a lui crefce upinimico, essendo foggetto.4
Mondo, & alla Tromba » ¢ P°
perche é carta, che puol fare
piu d’una verzicola,& ognl yer
zicola ove entra il Sole cont
il doppio delaltre, perche que
sia non puol contare meno
trenta ponti, e Paltre possom
contare folamente quindeciso"
de con questo motivo deve elle
re giocato con molta attene
me, & alle volte fara bene il 4
felo fubito, alle volte fara be |
il tenerlo fin’ all’ultumo 5 be
bifogna fempre faper diftungtt
re il gioco, e fecondo le con

genze regolarfi « ii
Ne) =?

9

S»

j=. >

S bt weer oe BN

~&

RIES PP

~ 3%

  
 
 

IT?

Se voi avefte 1] Sole, & avefte ancora pochi Farocchi con
una cartiglia malamente diftribuita,e dubitafte di potere alla
fine perdere il Sole’, allora alla
ptima congiontura, che vi ca~

piti dovete farvelo, ma prima:

peréd dovete riflettere, fe quello, che vi fta fopramano sia pia
capace a giocare cartiglia, las
quale voi abbiate ancora, per=
che fe fofte certo, che non potesse più: giocare cartiglia, allota potrefte tenere il Sole un poco piu, perche potrete sperare,
che la gioeata v’abbia a venire
injfaccia, e fare il Sole a voftro
arbitrio, e poi potrefte ancora.»
difendere il voftro compagno
in cafo, che vi girasse qualche
carta d’onore, ma fe poi non
avefte questa probabilita » allo~
ta dovete necessariamente farvi

il

 

 
Pare

 

 

 

 

 

 

 

 

114.

i] Sole tutto che il voftro com
pagno avesse una verzicola
farfi, perche dovete confidera
re, che puol essere maggiore i
danno,che vi caufarebbe la po
dita del Sole, del vantaggior’
vi potrebbe portare il fare de
la verzicola, ¢ pol veunaltl

gagione, Xe, che il yoftro in

mico vedendovi fare il Sole,
fapendo, che i] voftro compé
gno hala yerzicola da far fi»
ye necessariamente dubitate

Ja Lul

che vol abbiate almeno

“per difefa della verzicolas ¢ ©

si non fi azarda con tanta ac
tia venire su la verzicola»
non in cafo che abbiate und de
Ie carte superioti a tutte «

Se poi avefte con il Sole U
Cartiglia ben diftribul
una mediocrita

altri onor!, allora

 
= & No

\~ wo i OD

i
Cf

11g

re il Sole un poco pill, ancorche
il voftro compagno non avesse
verzicola, e dovete con prudenza deftregiarmi in modo che
vi capitila congiontura Gi pote:

- prendere qualche onore all’ini
mico, edevete riflettere, che
se vicapita la congiontura di
prendere qualche aria inferiore
ai Sole, li dovete tirare, fe pero non fofte ficuro di perderlo,
6 pure vi accorgefte, che il
rifico non é eguale ; ma quando
il rifico fosse eguale, Odiverfificafse folo in cinque punti tan
to ci dovete ritirare,perch’ aria
per aria fempre deve rificarfi, e
fe poi vi succede male non per
questo averete giocato male,
perche il male folo fara proceduto da un accidente, ed al contrario il giocar bene procedes
dattenzione.

s Als

 

 

 

 
 

 

116

Alle volté per provare ancora fe il voftro fopramano
habbia carta fuperiore al] Sole,
dovete prima girare al fuo compagno qualche carta gelofa, in
tempo però, che quello sia obbligato
a Biocare Taroco,e sebbene

avetti avutg,tempo di potervela
fare,dovete non farl4 per afpettare a girarla, perche feil sua
fopramano ha carta fuperiore,
allora deve tirarci necessariamente, e fe non vitira é fegua
che non ci puole, ma fe poi vi
tirasse con qualche altra carta
inferiore,la quale il voftro compagno non potesse coprire allora potendo dubitare che poteffe
ancora su il Sole, dovete subitamente farvelo, ¢e cosl potrete
ancora regolarvi rifpetto all’altrionori dopo il Sole. ,

Doverete pero avvertire che

sc
 

114
fe per voftra Magesior carta non

avefte che un folo trentadue 59
trentatre, © altra carta poco
gelofa, tutto che avefti un taglio
no per questo ve la dovrete fubjto fare ; ma dovete tenerla, ancorche potete forfi dubitare di
perderla, perche dovete feryirvene per potere ricuperares
qualche picciolo onore del vostro compagno, in cafo che non
haveffe potuto farfelo sq Ja cartiglia, ecosi fino atanto che
non conofcete, che il yoftro
compagno non ha in mano pia
onori piccioli da farfi dovete
fempre tenere la forma di poterli ricuperare, fe poi quel
voftro onore potesse fare verzicola agl’avverfarj, 6 pures
a voi potete procurar di far.
velo.

) Al modo di conofcere fe if

coms
ty i
118
compagno abbia più: in mano
onori piccioli fi é il riflettere fe
gli abbia potuti fare alla mano »
enon li ha fatti,o pur fe li abbia
potuti girare a voi, enon li ha

girati, questo fara fegno eviden—

te che non li ha, perche lPonort
piccioli che premono, non potendo più farli fenza l’ajuto del
compagno deve neceffariamente girarli fe puole .

Chi haveffe Mondo, e Sole in
mano non deve mai farfeli tutto
che fofle ficuro che sotto mano
vi fosse la Tromba, fe non in

‘cafo pero che li facessero verzi
cola, petche la perdita @una di
quelle carte importa fubito un
refto, altrimenti deve afpettare
che licapiti la congiontura di
tirare 4 qualche onore, e dovera tirarli con il Mondo; mentre
allora é ficuro di pigliar quello
ono  

It
onore all'inimico, 6 di fark ik
Sole, con la speranza di poter
prendere qualche altro onore
alPinimico, e di far Pultima.

. Alle volte voi non avretes
in mano alcuno onore » e pure
dovete tenere il gioco in tanta
ftima che havete da procurare
@ingannare Pavverfarij, e non
fare come tal’uno che fubita
fanno conofcere di non aver
niente,perche con scoprire il lo=
To gioco danno campo all’av=

verfarij di fare quel che voglio-.

no ; al contrario fe voi procurate di far credere al voftro inimimico d’ayer un gran gioco lo
lotenete in tal foggezione, che
procura di farfi tutti Ponori alJa mano, e voi con bella manie=
racol moftrare di fare un gioco di Tromba dovete procurare
qi farglieli fare ; ma in modo
pen

 
320

pero che non fi accorga che voi
lo -facciate per debolezza col
procurare fempre di allentare, ed in tanto prendere pct
mandarle il gioco alla mano col
far moftra di prendere per for—
za,perche in quella forma farete ancora il gioco del vouro
compagno con abbligare il votro inimico a giocarle in faccia, ¢cosi lui havera campo di
poterfi fare quelli onori che
yora .
Ma col procutar d’ingannare
Pinimico, non dovete pero iDgannare il compagno >» perche
potrebbe apportarvi up grave
pregiudizio, percio quando non
averete niente, edaverete procurato d@ingannare il compa
gno quando siamo vicipi alla»
fine del gioco dovete far conofcere al voftro compagno,, che
vol

 

TC

et

pr
qu
sa1
V

in,

Ct

re
1non havete gioco, ¢ lo fa
f¢ col prendere alcune volte,

arli vedere che potevate non’
-ndere, perche fe prendete

ando potevate non prendere

‘a fegno evidente,che voinon

lete più dare foggezzione all?

imico 5 e cosi dovra conosre che non avete carta da fa
la caccia.

Di più fe vi pareffe dal modo
giocare che il voftro compa10 poteffe avere la Tromba, 6
Mondo, allora voi dovete»
mpre girarlili Tarocchi più
‘offi ancorche foffero onori, ¢
‘ocurare dinon lafciarlo piiar mai, accio non fosse obbliato a giocare in faccia al fuo
imico; maficcome potrebbe
arfi il cafo che faceffe il gioco
i Tromba, ee veramente non
aveffe,allora pet ben coup ieee
ae

   

23

  
      
  

 
 

 

 

 

 

viet
“0 5 l
yy

 

 

 

We SI ee

s22
lodovete fempre oflervare I
carta che rifponde, perche £
principio comincia a darvi qué
che Papa, quando per altro Pp

-trebbe darvi carta maggiore,

il Papa non vi facesse yerzico
potete fortemente dubitaré, ¢
esso non Pabbia; ma fe po%
daffle qualche Tarocco picciol
che non. contasse porendovel
dare qualch’ altro maggiore,
lora é fegno certo, che nor Vk
ma folo fi ¢ pofto in caccia P

“qngannare [inimico;onde vol

lora dovrete fare il voftro 810
e non più quello del compagh
fe poi fofte quafi certo, ch’ave!
la Tromba il voftro compag”
e fuflimo all’ultimo in tres
quattro carte, e yoi non have
‘che un folo Tarocco » ed il ref
Cartielia, ¢ quel Tarocce fo!

fe non fol

‘ancora un onore » a
= 2 ae pu

eae
  
  

pee
più che onore gelofo,¢ toccafle
giocare &voi per fare il gioco
del voftro compagno,dovete pia
tosto giocar Ponore, e non mar
la cartiglia, fepoi fofte certo
che avelfe il Matto in mano, allora potete non giocar Tarocco,
e giocare la cartiglia perche
il Matto @ lui gli gioca come
vuole, e potete non arrifchiare
il voftro onore .
Se viaccorgefte ancora, che
il voftro compagno avesse in»
mano qualche picciolo onore »
il quale non avesse mai potuto
fari, allora dovete procurare di
andatle incontro con li Tarocchi più grofli, ancorche foflero
onori, ma fe dubitate d'un gran
gioco negl’ avverfarii, dovete
andarle incontro folo con Tacocchi grofii, e non @onore per
non obbligare Pinimico, ad imFa pes

 

 

 
 

 

 

 

 

 

 

 

 

 

 

 

 

124
pedire al voftro compagno ¢
qualche onore fuperiore il da
vi quel piccolo onore,e cosi vé
restivo a perdere il voftro 0?
re per averlo girato al comp
eno, & il voftro compagno Vé
rebbe a perder un’altro on0!
per non averfelo potuto fare
alla mano, e questo succede 4!
coratalvolta senza che pailia
alcun onore, perche quan Q
yoftro inimico pud creder » cl
voi non abbiate piu carta done
re gelofa,ma credendo che ‘al
bia il voftro compagno, ¢ fapen
do, che ilfuo compagnoh i
Tromba,allora,ancorche sj g10
chi cartiglia, ¢ 200 yj sia onor
alcuno, gioca il Mondo, no
peraltro fine, folo perche il vo
ftro compagno non pola fare
carta donore ; Onde in quelt

bifogna ftare affai avyertito P
. no

ee —«e,,
  

125
non essere. obligatia perderes
tutti due.

Abbiamo veduto di fopra come alle volte é necessario it
gannare Vinimico col moftrare
@aver un gran gioco, fe bene
non avefte niente, ma a folo
motivo ditener in fogettiones

| Pinimico; ora dobbiamo vedere,

come alle volte ¢ necessario di
far vedere all inimico dinon.,
avere ne Tromba, ne Mondo,
tutto che l’abbiate, ¢ questolo
potete fare,o con lo scarto quan.
do dovete scartare, ocol modo
di giocarey col giocare totalmente diverfo daquello, ches
giocarefte-fe volefte fare il gioca
della Tromba, ma doyete pero
avvertire, che questo gioco lo
dovete fare a principio, ma
quando siamo nello ftringere del
gioco dovete procurare di metF 3 ters

 

 

 
126
tervi a fegno dinon essere |
obligato 4 prendere, ma met
S allora il voftro avverfario
ogettione tale, che non po
piu farfi carta d’onore alla 2
no, maflimamente fe fofte ficu
che aveffe in mano delle cal
gelofe affai, e questo lo dove
fare, specialmente quando h
vete eran quantita de Tarocel
edubitate di dover essere P
in necessita di prender molt
|  fime volte, ¢ per confeguel
ie obligato 4 giocare speffo in 4
[| cia al voftro inimico.
In oltre fe avefte una cart
glia perfida, cioe moltiflimé
qualche fpecie,¢ fofte ficuro,¢!
afl C45 tocar
glaltri non la poteffero gioc
-allora dovete voi giocat femnt
quella, poco importa 0,
4i facia il gioco dell’inimice ae
mandarle fempre il gioco a

 

 

 

 

 

 

 

 

 

 

 

(gen
itl

[C*

fla

a".

cia, ancorche fofte ficure ; che
lui nom ne haveffe, perche: fe
non la.giocate voi fiete: ficuro ;
di nom arrivare all’ultima*con
la Tromba, e cosi vi togliete la |
speranza di poter far caccia, ¢
vi bifognaabbandonare ikvofiro :
Compagno in mezzo al gioco ; |
~ Tu fomma tutta la maggiog - |
difficolta di questo gioco con= |
fifte in faper ben giocare la cartglia, nel farfi Ponori a tem Pos
e girare verfo del compagno 3
Tpetto: al modo di giocare be.
ne 4a cartiglia parmi daveryl
detto tanto, che'batti, it refto lo
dovete Imparare da voi medefimi per mezzo d’una bei efatta
Pratticas circa del farvi Ponorl
a tempo confe(to il vero, che io
Per me non sddarvilume maglore. Refta ora folo 4 vedere jl

modo, che fi deve teneze nel gin
FE 4, Tada

 

 
 

 

 

 

aver memoria fifla dituttil’o

 

tare Yonori al compagna; Ed io
per me vi dico., che dovete pri
ma, riflettere a tutti Ponori,,che
avete In Mano, e poi girare ptir
ma quelli, che fonomeno gelofi,
e dopo li pil gelofi, e fe vedeté
che il voftro inimico.non ci tira
quello fara feeno evidente, che
non cipuole, fe perd. non folle,
ficuro, che il voftro compagn?
aveffe un onore fuperiore al fos
perche fe non sa dicerto, che it
compagno. ha/una carta fuper!o> |
re alla fua fopra Ponori gelott
deve andar necessariamente
puole, efe non civa efegnos
che non ha, carta d’onore groflar
onde allora vi potete mettere 1?
caccia al voftro sottomano co”
una carta d’onore, anche me" |
diocre,, ma fopra tutto cont
we re
fens

 

ti, che fono ftati giocati

—
perche fecon quella carta non
potete prendere carta che 4 voi
facesse. gran gioco, ma al contrario’ col perderla potefte far
fare un gran gioco aglavverfa~
tii, allora voi dovete farvi la
voftra carta d’onore, perche»
puol esser maggiore il rifchio
della perdita,che l'utiledel! guadagno,: e cosi quando il rifchio
noné eguale non fi deve arri
schiare, e per questo quando il °

Sioco: noné fcoperto aria per
aria deve arrifchiarfi . 3
Ma fe fapefte di certo, 6 pure

potefte dubitare dal modo dé

giocare, che il voftro.avverfa~

Tio. aveffe un aria fuperiore alla

voftra, allora dovete procurare

difarvi la voftra allamano, fe

bene alle volte ¢ dovere perdere

   

flettere, che carta potete più
prendere con quel voftro onorey .
 

 

 

 

 

 

 

130
re un aria per dover poi far magior gioco, perche con la pel
dita ?umaria si puol impedire
all’avverfarii il fare altro onoré
di maggior confeguenza, e pure
fe fofte in ftato di non poter più
Zare la voftra aria, puotete,co”
dranchezza coprire l’aria, 6 pure
qualfifia altro onore dell inimir
co,.con la certezza di farvi 14
voftr’aria, 0 qualch’altra carta
@onore alla mano ; mentre-c?
perdere un aria fiete ficuro, ch¢
al gioco vi debba venire alla m4"
No, e poi farvi quella carta che
pili vi piace. . ;

E talvolta ancora é necessa
vio paflare una carta d’onot? »

ancorche non ve ne sia un’altras
per folo. motivo di poterfene po
fare un’altra alla mano, percn®
in quel cafo fiete ficuro» di
farvi quello, che girate, 2
 

 

farvi quello, che vi é rimasto in
mano, dove al contrario farete
obligato a perderne due, fe non
ne girate prima uno.4 tempo debite, perche col reftringerfi del
gioco fi rettringe a voi la forma
di farvi le voftre carte d’onore;
laonde ¢ meglio perderne una
a tempo debito per poterfene
poi fare un altra, che essere ne
ceffitato a perderne due in ultimo per non averne voluta gitare prima una al fuo compagno.

Alle volte fi poffono, anzi fi
devono fcartare Tarocchi, di
quelli perd, che non contano
per poterfi diftribuire la-cartiglia, in modo dinon essere obliGato agiocare in faccia al fuo
avverfario, ma avvertite pero di
farlo con fomma cautela, ¢ non
mettere a rifchio li voftri onori,
meutre prima di penfare allas

cace
 

 

 

 

 

 

 

 

caccia delli onori altrui, bifc
gna penfare almodo di fare
fuoi, aflicurandoyi, che quand
avrete imparato 4 falvare liv’
Jiri onori, ed a procurare ¢
prender quelli deg!’inimici, f
rete un gran Giocatore, perch
tutta la bellezza di questo gioc
confifte nel faperfi difendere di
suo inimico, e procurare ne

ifteffo tempo di vincere il f

inimico. |
Tutte queste regole anno
loro apendici, né fono rego"
fempre certe, mentre fi varia”
ogni volta fecondo la difpe:
zione de’ giochi, onde vo! Hi
dovrete ftar fiffo 2 questa,
con la voftra prudenza, e ©
la voftra attenzione regolat!
fecondo vi portera il gioco: :
xo bene, che queste vi po a
dare up gran lume per pote! 4k
eee prens

4
 

 

f)

prendére con più facilità il gioco,
ma non per questo vi possono fare un gran Giocatore,per~
che per esser gran Giocatore v1
vuole la prattica, la quale vi deve imparare il modo di metter~
Te in efecuzione .

Onde parmi d’aver a bastanza
fodisfatto al mio impegno con
defcrivervi tutti quelli accidenti, che mi fono faputi venire in
memoria, e poi darvi il modo,
di provederyi; ma ficome fong
Infiniti g?accidenti, 'che poffa=
no giornalmente accadere in
questo gioco, cosi lafcio la cu

faalla prattica di ammaeftraryi

fedelmente nel refto, bastandomi per-ultima mia fodisfazione,
che in questo gioco procuriate

fempre di tener in foggettione

i voftro inimico, e che rifletUate fempre ogni volta alle cay.
te

— ets
 

 

134

te giocate, e che abbiate memo
tia di tutti Ponori, tanto dell
glocati, quanto di quelli da gi
carfi, che vi protefto, che g107
Carete prefto beniflimo. .
Tutti gPonori aMieme fann?
364. e con tutte le carte fan?
496. Ora tocca a voi il fare £0”
pta di qnefto gioco una matulé
rifleffione, & una longa pratt!
ca, avvertendovi, che questo®
il vero modo di giocare in quae
@ro con il compagno, per allt
3l puoi aicora giocare 10 oe
& in quattro,ad ogn’uno pet 7.
ma ficome non mi fon mal ide »
to d’imparar agiocare ad OS 0
uno per fe, perche in que “ 4
non fi poffono girar gl On ter’
compagno, main calo de he
necessitato a perder quale
onore fi deve fempre proce

« # 7l a
il minor danno; co! qraver
ae ere ee ae a a ee ee ee ee ee

rare
avetvi quivi brevemente decritte queste regole, acciò pofate poi imparar questo gioco
on piti facilità .

Se poi vi fosse qualcheduno,
che desiderasse @’'imparar a giocare ogn’uno per fe, deve faper
che chi sa ben giocare col come
agno sa ben giocare ad ogn’un,
ex fe, poiche la fine del gioca
sempre l'istessa, cioé di procurar di
farsi li fuoi onori, es
render quelli del compagno .

 Per altro a me basta per ora
avervi ammaeftsati nella forma
più necessaria, e più ufitata,,
Jual’é quella di giocare con ik 
compagno, e con la fola, e così
ier adempito nella forma più
possibile al mio impegno, mz

Se poi avessi mancato in qualche 
parte vi prego a compatir la 
mia temerita, e debolezza,
assicurandovi, che il difetto non é
stato della volontà, ma dell'incapacità,
pero sottomettendomi
sempre prima al ben regolato,
e prudente giudizio di voi
altri Signori Giocatori, dirò,
che per giocar bene basta giocare, e con riflessione, e con la
ragione; perche ogni qualvolta
uno sa render la ragione perché
ha giocato, più tosto; quella
carta, che un’altra dovete n¢eceffariamente
confessare, che
quello ha giocato bene, tutto
che l’evento sia succeduto males
laonde fe voi vi manterretes
sempre con quelli primi principj
di saper render la ragione del
vostro modo di giocare, credetemi,
che giocarete sempre bene.

IL FINE.

\end{document}
