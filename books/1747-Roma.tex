\documentclass[11pt,a6paper]{article}
\usepackage[utf8]{inputenc}
\usepackage[T1]{fontenc}
\usepackage[italian]{babel}
\usepackage{changepage}

\usepackage{mathabx}
\usepackage{array}
\usepackage{verse}

\newcounter{listombrecnt}
\newenvironment{listombre}{
  \begin{list}
    {\arabic{listombrecnt}.}
    {\usecounter{listombrecnt}
      \setlength{\labelwidth}{0.3pt}
      \setlength{\labelsep}{0.3em}
      \setlength{\itemsep}{-0.2em}
      \setlength{\leftmargin}{0em}
      \setlength{\itemindent}{0.6em} % equals \labelwidth+\labelsep
    }
}{\end{list}}

% needed for floating frames
\usepackage{float}
\usepackage{wrapfig}
\usepackage{framed}

\usepackage{titlesec}
\titleformat{\section}[display]
            {\huge\bfseries}
            {}
            {0pt}
            {}[]
\titleformat{\subsection}
            {\slshape\large}
            {}
            {20pt}
            {\flushright}[\kern -4pt]
\titleformat{\subsubsection}
            {\scshape}
            {}
            {0pt}
            {\centering}[\kern -4pt]



% avoid orphans and widows, allow for (a lot of) letter spacing.
\usepackage[defaultlines=2,all]{nowidow}
\usepackage[tracking]{microtype}
\sloppy
% how to format and space chapter titles
% I like this font!
\usepackage{tgbonum}
\renewcommand{\rmdefault}{qbk}
\usepackage{lettrine}
\usepackage[left=13mm,top=12mm,right=11mm,bottom=14mm]{geometry}

\newcommand{\textdoublecircle}{%
  \textcircled{\raisebox{1.5pt}{\textbigcircle}}%
  }

\newcommand{\textfisch}{\raisebox{-5.6pt}{\Huge\textbullet{}}}

\setlength{\leftmargini}{0pt}%
\newcommand{\literaryquote}[1]{%
\kern -6pt  \begin{verse}
    {\footnotesize \it #1}
  \end{verse}\kern -2pt%
}

\newcommand{\markerA}{{\small$\Downarrow$}}
\newcommand{\markerC}{{\small$\downarrow$}}
\newcommand{\markerB}{{\small$\downdownarrows$}}
\newcommand{\markerD}{{\small$\downuparrows$}}
\newcommand{\markerE}{{\small$\downtouparrow$}}

\newcommand{\supersection}[1]{%
\clearpage
    {\scshape \centering \huge #1\\}
    \vspace{6pt}
    \hrule
    \vspace{12pt}
}

\title{\kern -42pt\fontsize{30}{30}\selectfont{\textls[80]{GIUOCHI}} \\ \kern 1pt \fontsize{10}{10}\selectfont{\textls[600]{DELLE}}\\ \fontsize{22}{22}\selectfont{MINCHIATE, OMBRE,} \\ \fontsize{18}{18}\selectfont{\textls[480]{SCACCHI},}\\ \kern 2pt \fontsize{11}{11}\selectfont{Ed altri d'ingegno}}
\author{\vspace{-30pt}\\
\textsc{\fontsize{12}{12}\selectfont{\textls[520]{dedicati}}}\\
\textit{\fontsize{11}{11}\selectfont{Alla Ill$\widetilde{\textit{m}}$a, ed Ecc$\widetilde{\textit{m}}$a Signora la Signora}}\\
\textsc{\fontsize{18}{18}\selectfont{\textls*[500]{principessa}}}\\
\textsc{\fontsize{30}{30}\selectfont{\textls*[120]{donna giulia}}}\\
\textsc{\fontsize{28}{28}\selectfont{\textls*[33]{albani chigi}}}\\
\textit{\footnotesize D A}\\
\small D. FRANCESCO SAVERIO BRUNETTI\\
\small \textls[360]{DA CORINALDO}.}
\date{\vfill\small In Roma, per il Bernabò, e Lazzarini, 1747 \\ \textit{\textls[80]{Con licenza de' Superiori.}}}



\begin{document}

\pagenumbering{gobble}

\maketitle

\clearpage



\pagenumbering{roman}


\noindent{\fontsize{15}{15}\selectfont{\centering Ill$\widetilde{\textup{m}}$a, ed Ecc$\widetilde{\textup{m}}$a Signora.\\}}\vspace{80pt}



\lettrine{S}e la grandezza dell'animo
altrui argomentar si dovesse
dalla bellezza del corpo, dalla grazia,
dalle maniere leggiadre, Eccellentissima
Signora, Voi sareste la massima Eroina di Roma;
ma che lo siate molto più, da' più alti principj,
che non amettono errore, si scorge. Questi
sono le vostre opere
insigni, le quali non vi fan punto de
genere dalla Ill$\widetilde{\textup{m}}$a, ed Ecc$\widetilde{\textup{m}}$a Prosapia
donde veniste, e da quella generosissima
ove andaste, garreggiate per certo con
quelli Eroi, che il Mondo adorò sul Trono
supremo del Cristianesimo il vostro
Prozio Clemente XI, ed Alesandro VII.\
della familgia Chigi; alla propagazione di
cui due germogli donaste Sigismondo, e
Francesco, oh di quanta espettazione, mentre
si vedono promettere colla loro indole generosa
non inferiori progressi dei loro Maggiori.
Io non voglio quì annoverare l'efimie
gesta di tanti Eroi, quanti ambedue
le Famiglie produssero, perché non sarei
per mai venirne a capo: dico bene, che
Voi di tutti vi fate in ogni più chiara loro
virtù immitatrice. Costanza ne i giusti
impegni, potente protettrice de' miseri,
affabile cogl'umili, corretrice dell'ardire,
benigna, cortese, familiare con sommo
decoro coi costumati, pia, guardinga, illibata.
Queste sono le vostre doti ordinarie.
Siete poi nel conversare gioliva, sincera,
e leale di modo che chiunque una
volta vi tratta non cessa mai di celebrarvi
con meritate lodi. Ora essendo Voi di
tante, e sì preclare prerogative, ornata
che in Donna forte richiedonsi, e che sì
rara cosa è trovarne più d'una. Non vi
sia maraviglia se io ancora rapito da tanta
eccellenza di merito, e dal desiderio di
proporre ancora ad altrui un ottimo, e sì
raro esemplare, giacché Validiora, sunt
exempla, quam verba, \& plenius opere
docentur quam voce, S.\ Leone. Non
vi sia dissi maraviglia se io venga ora a
tributarvi in segno del mio più ossequioso
rispetto quest'Opera fatta, perché altri
apprendano quelle virtù delle quali Voi
abbondate. So, che altezza dell'Animo
Vostro userà meco la confueta degnazione
di accettare il mio piccolo dono, come
l'Amplissimo, e Sagacissimo Porporato vostro
Fratello Gianfrancesco Cardinale in
questo fausto giorno 10.\ Aprile 1747.\
meritamente ornato ed esaltato dal Sommo
Pontefice Benedetto XIV si degnò poc'anzi
di accettare da me un simile tributo
d'ossequio. Con che io spero di confermarmi
sempre più sotto l'alto patrocinio
dell'Eccellenza Vostra, e dell' E$\widetilde{\textup{m}}$o Principe
sudetto, alle quali fin da fanciulli ho esibito
una sincera, e cordiale servitù, in cui
con tutto l'ossequio, e rispetto mi confermo,
e mi dico

\vfill

D.V.E.

\vfill

Roma 10.\ Aprile 1747.
\vfill

{\flushright
 \textit{Umilis, Divotis, ed Oblig., serv.}\\
Saverio Brunetti Cappellano di N.S.

\vfill


}

\section{PREFAZIONE.}

\lettrine{L}{'U}omo altro non è, che una
mente, ed in corpo talmente
organizzato, e disposto,
che ogni sua parte, qualunque
ella sia, è in ordine al
moto. In esso, oltre il gran
Caos delli moti corporei spontanei, evvi il
moto dell'Anima, allorché pensa
incessantemente, o agli oggetti, che gli si paran
davanti da i sensi, co i quali si unisce in virtù di
quel potere, che cosi vuole: o al Sommo
Vero, la di cui connessione è tanto
indispensabile, e necessaria, che da esso separare
mai non potrassi, senza ritornare in quel
nulla, da cui fu tolta. Ecco l'Uomo da tanti
moti a lui naturali perpetuamente agitato;
che fa egli con quel molto, che di volontario
gli resta? Va sempre in cerca del suo
riposo, della sua pace, altro non desidera,
non fa mai altro, che cercare un'oggetto
proporzionato ad appagarlo. Vede per esperienza,
e per tutte le ragioni altro non esservi,
che la giustizia capace di farlo montare
alla intiera sua felicità, se a questa (non distratto)
si appiglia, eccolo tutto attento a
serj pensieri, di meditare il Vero, e di anelare
amando verso il Sommo Giusto, che per
tal fine creollo. Ma perché forz'è di natura
non qual fu, ma qual'è, non potere intensamente,
e senza mai cessare attuarsi così,
sa di mestieri, che pure qualche respiro si sia
talvolta alle cure più serie, acciocché l'animo
meno applicato in cose sì premurose,
quasi dopo breve riposo riassuma con più vigore
i negozj di maggiore importanza. \textit{Homines, quamvis in rebus maximis exerceantur, tamen si modo homines sunt, interdum.
animis relaxentur}. Cic.\ Tusc.\ q.\ 2.

Un'opportuno trattenimento, che con
piacere si costuma, è il giuoco, questo per
avventura è talvolta un'adattato mezzo
termine per trattenersi con gioliva applicazione
nella civile conversazione, ove lo spirito
impiegato in cosa di leggiero interesse, e di
allegro trastullo ristora l'illanguidito vigore.
\textit{Ludo \& joco uti illo quidem licet, sed sicut
 somno, \& quiete tum cum gravibus, seriisque
 rebus satisfecerimus}. Cic.

Tra questi il gioco delle Minchiate giudico
molto a proposito a rendere plausibile
conversazione, che per esser lungo, vario
e meschiato d'azardo, e d'ingegno può insieme
trattenere, dilettare, ed erudire colui,
che talvolta lo giuoca. Egli è lungo,
perché si tratta di 97 carte che debbono mischiarsi,
dare, giuocare, e poi contare in più
modi. Vario, perché ha 96.141.308.410.784.017.049
casi diversi v.\ la mia Arimetica
Sp.\ c.\ 5. D'azardo, perché le carte migliori
acaso capitano alle mani, e poi d'ingegno,
perché con poco il più esperto vincerà l'avversario;
egli è anche facile, benché a primo
aspetto paja difficile. Lucr.\ 1.2.
\literaryquote{
 Sed neque tam facilis res ulla est, quinca primii difficilis.}

Ora questo giuoco io intraprendo a spiegare
forsi con non minor diletto d'esso: scrivendo,
di quello, che altri abbiano a giucarlo,
e spero di nulla perdere col mio, piacere,
anzi di guadagnare, e far guadagnare
altro, che scienza di giuocare a chi leggerà
queste poche pagine, perché mi confido di
trattarlo. \textit{Lectorem delectando pariterque
 monendo}.

Con sano consiglio spero aver unito, alle
regole d'un giuoco i precetti più seri della
vita sociale di alcuni Uomini morigerati, e
dabbene, e ciò perché i Giovani avidi dei
piaceri giuocosi, volontieri leggendo questi
precetti a poco a poco senza avvedersene
coll'imparare il giuoco delle Minchiate,
apprendino ciò, che è più importante il vivere
civile, e savio. Tutti noi abbiamo in noi
medesimi i semi, delle virti, basta eccitarli
per vederli germogliare, e far frutti: \textit{
 Omnibus natura fundamenta dedit, semengue
virtutum: omnes ad omnia ista nati sumus:
Cum irritator accessit, tunc illa animi bona
velut sopita excitantur}. Per verità non
merita un giuoco l'applicazione, totale d'un
Uomo, o tanta, che per impararle bastasse ad
apprendere una qualche utile scienza. In esso
(cosa rarissima) lodevole è la mediocrità,
e'l saperne troppo, come dice il Castiglione
del giuoco de i Scacchi nel suo Cortiggiano,
non è molto comendabile, ond'io per togliere
questa nota da chi impara a giuocare, gli fo
in un tempo medesimo imparare a giuocare,
ed a convivere, acciocché l'eccellenza in uno
via più nobiliti la perizia nell altro.

Chi sa, se a sì bello scopo collimerà il mio
talento, che poi con retta ragione mi senta
dire con Anaxippo Ateniese riferito da Celio
Rodigino l.\ 22.\ c.\ 13.
\textit{Vah numquid fingi potest portentosius illo,\\
Qui Sophos in Triviis inflato mugit ab ore,\\
Rebus at in mediis stupet, ac velut iscius haeret?}
Io perciò quasi nulla v'ho posto del mio (se
si eccettui il difetto) tutto ciò, che dico,
altri lo dicono a nostra istruzione. T.~Calfurnio.
\textit{Ut mea rustictitas si non valet arte polita
 Ludendi, at certe valeat pietate probari.}

Se poi tutto ciò dica io sì bene, che altrui
piaccia, e con profitto, sappiano, che
\textit{De virtute loqui minimum: Virtutibus uti
Hoc Sampsonis opus.}

Facendo io dunque tutto quello che posso per
utile, e profitto d'altrui senza risparmio né di
applicazioni, né di spese; né d'incommodo.
Ed avendo io radunate nel corso di molti anni
in occasione di leggere varj Autori, alcuni
loro detti veramente memorabili, utili pare
a me alla retta economia del vivere d'ognuno,
gli ho qui disposti insegnando un giuoco, che
per essere tutti d'Autori Poeti, o Istorici, o
Oratori, non è inconveniente, che in un
giuoco da tali s'impari, come dice Ugone
l.\ 9.\ didascalicon. \textit{Ab omnibus libenter disce
quod tu nescis}.


Il Giuoco di grande interesse vien detestato
dalle Leggi, perché ivi si perdono le cose
più preziose della vita umana, il tempo, le
facolta, la salute, la riputazione, e la
tranquillità dell'animo, ivi si offende la carità
verso il prossimo, vincendo, e verso se stesso,
perdendo, e quello, che è peggio, i Giuocatori
non si persuadono giuocando di fare alcun
male. \textit{Ac, qui perniciosis est error, multi
 etiam, ne injuriosos quidem in proximos esse,
 eos, qui Alea ludunt: aut ullo se scelere
 contaminare, quamvit id magnam in pecuniam
 vel rem faciant, arbitrantur}. Pascalio
Giusto nella Prefazione. Perciò ognuno
fugga da quei giuochi, ove tutti i suddetti
mali s'incontrano, ed ove dice Giovenale.

\literaryquote{Ploratur lacrymis amissa pecunia veris.}

Incomincio dunque col massimo precetto,
che chi lo principia ad osservare, subito in
tutti gli altri è perfetto, e questo è la Carità
verso di se, verso de i prossimi, e sopra tutto
verso di Dio. Verino.

\literaryquote{Est Charitas perfettus amor perfellague virtus,
Qua fine perfelium nil reperire potes.}

\subsection{IMPRIMATUR}

Si videbitur Reverendissimo Patri Sacri Palatii Apostolici Magistro.

\nopagebreak
\hfill F.M.\ de Rubeis Archiep.\ Tarsi Vicesg.

\subsection{APPROVAZIONI}

\lettrine{A}{v}endo per commissione del Reverendissimo
Padre Maestro Ridolfi Maestro del Sagro Palazzo Apostolico
attentamente letto un Libro intitolato:
Giuochi delle Minchiate, Ombre, Scacchi etc.\ nè
avendovi ritrovate cose contrarie a'
Principi, ed a' buoni costumi, giudico se ne possa
permettere la stampa. Di Casa quello dì
2 Settembre 1746.

\nopagebreak
\hfill Giampietro Marchese Lucatelli.

\lettrine{P}{er} Ordine del Reverendissimo Padre
Maestro Fra Luigi Niccolò Ridolfi Maestro
del Sagro Palazzo Apostolico ho letto
un'Operetta del Signor D. Francesco
Saverio Brunetti, intitolata Giuochi delle
Minchiate, Ombre, Scacchi etc.\ e non vi ho
trovato cosa contro la Fede, e buoni costumi.
Ed in fede \&c. Da Campitelli questo dì 1 Ottobre 1746.

\nopagebreak{\flushright Alessandro Pompeo Berti\\
Consultore dell'Indice.\\}

\subsection{IMPRIMATUR.}
{\flushright Fr.\ Aloysius Nicolaus Ridolphi Ordinis Pradicatorum Sac.\ Palat.\ Apost.\ Mag.

}


\supersection{
Notizie
per il Giuocco
delle
Minchiate.}

\pagenumbering{arabic}

\section{CAPO PRIMO.}


\subsection{Mazzo delle Carte.}

\lettrine{Q}{uesto} Giuoco si fa con un
mazzo di 97.\ carte. Sono
queste le 4 seguenze volgari di Spade,
Bastoni, Danari, e Coppe;
ogni seguenza ha 14 Carte. Vi sono
poi 40 Tarocchi, e 'l Matto.

\subsubsection{nota allegorica}
{\footnotesize
Si può assimigliare questo gran mazzo di Carte
alla catastrofe delle vicende mondane; tutte
insieme è come il Genere Umano, che vive, alla rinfusa
su questa Terra; le 4 seguenze sono come le 4
Monarchie prima degli Assirj, o Caldei, che cominciò
in Nino suocero di Semiramide V'anno del
Mondo 1944 contemporaneo a Debora, e terminò in
Dario Medo figlio di Astiage contemporaneo a
Danielle l'anno del Mondo 3450. La seconda
Monarchia de' Persi incominciò da Ciro nipote di Sorella
al sudetto Dario, e terminò in Dario codomano
vinto da Alesandro il Grande, in cui cominciò, e
finì la terza Monarchia de' Greci, perché dopo esso
si spezzò in molti Regni, i quali a poco a poco
vennero sotto il potere de i Romani, che la quarta
Monarchia gloriosamente ottennero. I 40 Tarocchi,
che prendono tutte le altre seguenze possono alla
quinta universale Monarchia applicarsi, che sotto
suo manto raccoglie ogni Nazione. Evvi in fine il
Matto, che ad ogni qualunque carta risponde, non
prende mai, e non mai resta preso, se non quando
tutto va in rovina. Forsi per ricordarci, che fin
dalla sua origine l'Uman Genere impazzì, e
tuttavia i stolti dureranno fino alla fine del Mondo,
quando non nasceranno più uomini, \textit{Stultorum
plena sunt omnia.} Cic. ep. 9.

Tutto questo ho voluto indicare acciocché
coll'occasione delle Minchiate
venga a' Giuocatori il
desiderio di sapere le Istorie, e sopra
tutto la Cronologia, perché \textit{nescire quod
anteaquam natus sis acciderit,
id est semper esse puerum.} Cic. in
Oratore.
}
\subsection{Le quattro Seguenze.}

Il valore delle Carte nelle due Seguenze
di Spade, e di Bastoni il più piglia il
meno; ma Denari, ed a Coppe (prescindendo
dalle figure, che sono Re, Regina,
Cavallo, e Fantiglia) il meno piglia il
più, cioè
l'Asso piglia tutte le altre carte fino
al 10.~il 2.~similmente, come nel giuoco
dell'Ombre. Il solo Re conta cinque.
Tutte le Carte guadagnate agli aversarj contano,
uno per una.

\subsubsection{nota allegorica.}
{\footnotesize
Nella republica umana alle volte succede che
il minore supera il maggiore, ed un dispreggiabile
plebeo tal volta è salito alli più sublimi gradi dei
Magistrati. Due soli memorandi in vero tra tanti
ne scelgo, acciocché ci rendiamo cauti a non dispreggiare
mai alcuno, perché per abietto, che egli
sia può per vie impensate, o in tutto, o in parte
divenire arbitro dei nostri vantaggi. Claudio il più
stolido. uomo del Mondo, avendo inteso lo scempio
di Caligola \textit{rumore ce dis exterritus} (racconta
Svetonio) \textit{prorepfit ad folariun proximum iutergue
pratenta foribus vela se abdidit, Ibi iatemtem~
discurrens forte gregarius miles, animadversis pedibus,
e studio sciscitandi quisnam esset, agnovit
extractumque, et prae metu ad genua sibi
accidentem, Imperatorem consalutavit, perduxit ad
alios commilitones fluctuantes, nec quidquam adhuc
quam frementes, et lecticae impositum vicissim
succollantes in castra detulerunt, tristem, ac
trepidum, miserante turba obvia, quasi ad poenan, et
suplicium raperetur infans.}

Quando mai averebbe possuto pensare un'Aratore in tempo,
che refocillavasi con pane, e cipolla
sopra la sua gomèra, che sarebbe a lui venuto un
Cavallo, che lo avrebbe prescelto Re di Boemia,
il che per essere successo a Primislao l'anno 632
piacemi la giocondissima narrazione istorica come
Kranezio Wandal c.\ 17.\ trascrivere \textit{Boemis imperabat
Libussa, qua moriente patre Virgo coepit
administrare paternam ditionem, tam prudenter,
ut Rex diu nullus desideraretur in ea gente. Ita circumspexit
omnia, ita justitia insignis erat, ut nemo
sub illa quereretur se suo destitutum: adeo scelerum
Ultrix, ut impune nemo peccaret. Mansissent
Proceres sub foemina longius: sed mortalem
prospicientes, semen Regium ex illa desiderare
videbantur. Dum tamen nemo audet interpellare super ea
re foeminam, sumpsere audaciam non ulli ex majoribus
ut voluntatem Procerum, et totius Populi
confenfum, atque defiderium explicarent. Tila, torvo
primo vultu intuit referentes: quid est, inquit,
quod vobis desit in hoc rerum statu, ut Regem
desideretis? At illi. Nihil si te immortalem haberemus,
nunc, quoniam communi omnes forte sumus
defecturi, cum te carere jam debeamus, semen ex
Regium mitigaret Populi defidevium; \& ingentem nominis
tui affectum. Dissimulabat aliquandiu casta
Mulier Populi de se desiderium: tam illa delectabat
Thorus sine socio. Sed cum illi finem non
facerent, jussit ornari candidum gradarium Equum,
cui proficiscens insidere solebat. Illum sequi praeeuntem
Proceres jubet, et apud quem is consisteret,
Regem mox consalutarent: nam illam dignaretur
Thoro maritum, quem fata jumento imperantia
monstrarent. Sedulo implent, que jusserat
Regina, sternunt jumentum, et premisso illi fraeno
liberum sinunt praeires illi porro sequuntur,
observantes quo vadat. Abibat in Agros, viam non
parvam, nec per solita gradiebatur itinera sed
compendia querere visus ibat in directum per nemora
per dumeta, per Agros, donec ipsa jam meridiae
deprehenderet hominem labore fessum se ad refectionem
in Agro, quem araverat reponentem. Cum
hominem cerneret jumentum, ecce stetit immobile
nusquam divertens, aut circumspiciens. Aderant
dicti primores, qui observabant. Voto se potitos
interpretantur, et gaudent, mox sciscitantur de
nomine. Pribislao, inquit, mihi nomen est. Illi
vero surge protinus, et conscende stratum tibi
jumentum pertrabuntque ad Reginam, insinuantes
quid repererint. Illa virum complectitur, et pro
Regia dignitate, suis nudatum vestibus, ornat,
et thoro dignatur. Fuit optimus Princeps, leges
enim non paucas (si Fulgosio fides) tulit quibus
Bohemi in alta pace viverent, quibus quidem
meritis adeo affecti fuere Bohemi, ut in primario Urbis
Templo ligneas soleas, quibus arans utebatur,
diutissime servarint, easque novis Regibus, dum
diademate ornarentur praeferrent.}

Altri esempj senza fine di Uomini umili, ed
abietti, che divennero potenti Signori, e
Monarchi, potrei portare, che per brevità tralascio; onde
chiunque tu sia averti, che, quegli, che tu dispreggi,
non abbi un giorno da vederlo tuo superiore,
e patrone, o perché tale divenga, o perché
favorito del Prencipe. Vedi l'esempio delle piccole
bestie, che, o col veleno, o coll'industria atterrano i
gran bestioni.
\literaryquote{Parva necat morsu spatiosum Vipera Taurum\\
A Cane non magno saepe tenetur Aper.}
}
\subsection{Tarocchi.}

I Tarocchi sono notati co i numeri Romani
dall'I fino al XXXV; primi cinque si chiamano
Papi, gli ultimi cinque Arie, e sono
36 Stella, 37 Luna, 38 Sole, 39 Mondo,
40 Trombe. Il loro valore è dal 2.\ fino al 5., 3,
ed 1.\ 10.\ 13.\ 20.\ 28.\ Matto, e 30.\ fino a
35.\ vagliono 5, e le arie 10.

\subsubsection{nota allegorica.}
{\footnotesize
I Tarocchi in questo giuoco sono utilissimi,
come nella Republica sono utili quelli, che anno la
perizia, o delle Arti, o delle Scienze, che in esse
divenendo eccellenti giungono ad acquistare per se,
e per la Patria la fama immortale, e la vera gloria.
Ovid. de ponto.

\literaryquote{Artibus ingenuis quesita est gloria multis.}

Onde ciascuno dee procurare di esercitarsi in
quell'arte, o applicarsi in quella scienza, in cui
si sente
più idoneo, \textit{quam quisque norit artem in ea
 exerceat.} Aristot. appresso Cice. onde Ovidio
nella sua arte.

\literaryquote{Si vox est, canta: si brachia mollia, salta,\\
 Et quacumque potes dote placere, place.}

Questa è la maniera di divenire insigne, ragguardevole,
e ricco, a cui ancora le leggi portan rispetto
in virtù di quell'asioma \textit{excellens in arte non debet
mori}; e Orazio dice

\literaryquote{Dignum laude virum multa vetat mori.}

Dunque ogn'uno segua l'amonizione del Poeta I
de art, fatta a' suoi Concittadini.

\literaryquote{Disce bonas artes, moneo Romana juventus.}

}
\subsection{Matto.}

Questo non è nè Tarocco né Cartiglia,
entra in tutte le verzicole, e ne forma
una col massimo, e minimo Tarocco;
non si perde mai se non si perdono
tutte le carte, perché quando si giuoca si
ripiglia, ed in sua vece si dà una Cartiglia,

\subsubsection{nota allegorica.}
{\footnotesize
Giacché, come dice Cicerone in Or. \textit{Stultorum
 plena sunt omnia}; l'indole de' pazzi chi mai
raccapezza? Quelto è certo però,che chiunque finge esser
pazzo, e non è (se da sopranaturale impulso non è
guidato) va a finir male \textit{illegitime stultitia finis e
infortunium}. Stobeo; i matti nel Mondo non sono
qualche cosa determinata fan di tutto, si ficcan per
tutto; e scompigliano tutto, solamente cogli
estremi uniti (come coll'uno, e le Trombe) fanno a
caso tal volta qualche buon'effetto. La vera pazzia
non finisce mai se non si muore, e nel Mondo durerà
col Mondo medesimo. Nelle Minchiate è desiderabile
il Matto, ove il Savio lo giuoca a suo modo.
Soleva dire Solone \textit{Stultus numquan taceres
potest}; ora se v'è chi poco parla tale sarà ben savio,
ancorché altri segni s'industrj dare di pazzia; a tale
però intimo il parer di Stobeo. Sono però molto
più quelli che vogliono apparire savj, e non lo sono,
e diconsi impostori, che van vendendo \textit{magno
conatu nugas}; e sono per lo più con gran solennitè
cacadubj, de' quali Gellio \textit{homines diliros, qui
 verborum minutiis rerum frangunt pondera} i questi
sono gli uomini senza senno più nocivi nel Mondo.
Con tutto che la pazzia sia un male sopra ogn'altra
malatia perniciosa, pur sé trovato un'Erasmo,
che ha infarcito in un'Opuscolo molte lodi di essa,
sotto il titolo Moriae encomium.
}
\subsection{Verzicole.}

Le Verzicole sono gli onori del giuoco, e
si accusano da principio, e sono le
seguenti.

Tre Re, o tutti e quattro.

Uno, matto,
e Trombe.

1.13.28.\ — 10.20.30.\ — 20.30.40.\ o
tutte le decine.

Tre papini in ordine o più.

E tre, o più carte in ordine dal 28 fino
alle Trombe.

Il 29.\ non conta fuorché in Verzicola.

Il Matto si accusa, e conta in tutte le
Verzicole.

In fine del giuoco tutte le carte contano
354, e l'ultima levata 10 di più.

Un resto si fa da 60 punti, e questi resti
sono le vittorie del giuoco.

\subsubsection{nota allegorica.}
{\footnotesize
Universalmente gli Uomini cercano gli onori,
le ricchezze, ed i piaceri, le quali cose cercate con
moderazione, e possedute senza soverchio attacco
non rendono male il di loro seguace, né sempre dee
biasimarsi colui, che non ha di queste cose un totale
disprezzo, potendo esse all'occasioni avere ottimi
usi, e per la Republica, e per la Religione, e per
i prossimi. Terent.

\literaryquote{Atque haec perinde sunt, ut illius animus est, qui ea possidet.\\
Qui uti scit, ei bona sunt: illi, qui non utitur recte, mala.}

Nelle Minchiate si acquista onore d'ingegno, giuocando
bene; denaro, vincendo; e piacere, conversando;
or parliamo dell'onore, giacché

\literaryquote{Honore gaudet offici se quilibet.}

Non è mica l'istesso essere uomo d'onore, e
cercare gli onori; il primo vuol dire essere ornato
delle virtù orrevoli; il secondo vuol dire, il cercare
le Dignità, le Cariche, e le preeminenze; al primo
fanno corteggio gli ossequj, a i secondi, sono
strascinati gli uomini a fare i segni esterni. di rispetto,
onde queili, e non questi anno la vera gloria. Seneca

\literaryquote{Qui favoris gloriam veri petit.\\
Anime magis quam voce landart volet.}

Plutarco saviamente insegna quanta gloria a ciascuno
convenga desiderare; \textit{tantum gloriae}, dice,
\textit{desiderabis, quantum satis est ad conciliandum tibi ia
rebus gereudis auttoritatem, qua inde uascitur
quod vir probus babeavis.}
L'uso delle ricchezze è ottimo alli padroni di
esse, pessimo a chi esse comandano. \textit{Imperat aut
servit collecta pecunia cuique} Horat. e. 1.
Pessimo giudico il consiglio di colui, che disse

\literaryquote{ Pria i danar cercate, o Cittadini.\\
 E dopo questi le virtù cercate.}

Forsi questo fu detto, perché dice Giov.

\literaryquote{Unde habeas quaerit nemo, sed oportet habere.}

Fu interrogato un giorno Simonide cosa era meglio
essere dotto, o ricco, rispose \textit{ignoro, video Sapientes
divitum limina frequentare}; ora sentiamo un gran
Saggio \textit{divitiae saeculares si desunt non per mala
 opera querantur in Mundo, si autem adsunt, per bona
opera serventur in Caelo.} Ecco come si cercano, e si
conservano le ricchezze.
}

\subsection{Modo di giuocare.}

Si giuoca questo giuoco in quattro persone,
o in partita con l'entragnos (come
usa communemente) o in partita senza entragnos
(entragnos vuol dire, che si vedono, e
prendono dopo le rubate tutte le carte di
conto, che si trovano nelle residue) tutte
queste maniere debbono spiegarsi.

\subsubsection{nota allegorica.}
{\footnotesize
Diverse sono le maniere di vivere degli uomini,
altra è la vita d'un Causidico, altra quella
d'un Pecoraro, altra quella d'un Certosino. Le
cure d'un Generale d'armata sono molto diverse
dalle cure d'una donnicciuola, che fila, e poi
tutti, abbiamo varj i nostri desiderj
all'adempimento
de i quali s'indrizzano le nostre azioni.
\literaryquote{Vivere diverso Mortales more videntur, \\
Nam ratio cunctos non regit una viros.}

quindi è, che tanto bisbiglio vedesi nelle
popolazioni, tutti gli uomini in grandi e
diversissime faccende applicati. Verino.
\literaryquote{
 Hic opibus vacat innumeris, ille ambitioni: \\
Hic peragrat campos, navigat alter aquas \\
Sunt, qui bella petunt, sunt queis servire togatis\\
Contigit, af alios otia laeta juvant. \\
Unus amat Venerem, vino se ingurgitat alter.\\
Hic hilaris ridet, tristis at ille gemit,\\
Hic citharam movet, hic cantus emittit amoenos,\\
Viribus ille valet, litibus ille vacat. \\
Sunt qui mundanos tantum venantur honores\\
Sunt, qui in praeruptis montibus adificant \\
Ille est astridicus, tellurem sed metit alter, \\
Hic numeros curat, diligit ille notas.\\
Sic variis diversa placent sua cuique voluptas \\
Ad sibi res gratas quemque libido trabit.}

Onde chi si vuole cattivare l'animo di chi che sia,
veda di scoprire il suo desiderio; e quanto più gli si
mostrerà atto a farglielo conseguire, tanto più da
quello sarà frequentato, ed amato: questo è un
precetto di politica sommamente proficuo, che la
maniera di scoprire gli altrui desiderj, inclinazioni,
e pensieri sia il giuoco più abbasso diremo.
}
\subsection{Modo di giuocare in partita
con l'entragnos.}

Il giuocare la partita è giuocare in quattro,
due contra due, se i compagni non sono
determinati, si scelgono a sorte, alzando le
carte, e quei due, che alzano carte migliori,
stanno insieme, e si pongono al tavolino di
contro; e gli avversarj parimente, e poi
quelli, che alzarono più, comandano a chi
dee il primo fare le carte.

Questo meschia le carte, poi le pone in
tavola acciocché il suo contrario, che le sta
a sinistra le alzi. Questo scuopre ciò, che è
sotto, perché, se vi trova carta di conto, o
un sopraventi, lo prende, e se dopo questa
sianvi altre carte simili, si prendono tutte,
finché ve ne sono seguitamente, e questo si
chiama rubare, e segna tanti punti, quanti
ne vagliano le carte rubate.

Quindi chi fa le carte ne dà prima dieci
per uno, e poi undici, e scuopre l'ultima, che
se conta si segna il suo valore da chi l'ebbe.
Finito che ha di dare le carte, ne debbono restare
tredici, se non ne furono rubate, in queste
vede, se ruba, cioè osserva, se la vigesimaseconda
è carta di conto, o sopraventi,
che, se è, lo prende, e susseguentemente
quant'altre simili ve ne sono, seguitamente
poi scuopre tutta la fola, e prende quante
carte di conto vi trova. Fola sono le sudette
tredici carte, che restano in fine, le carte
trovate in fola non si segnano: fatto questo,
chi rubò scarta, ed ognuno dee restare con
ventuna carte in mano; poi giuoca quello,
che sta alla destra di chi fece le carte; a cartiglia
si risponde cartiglia dell'istesso palo, chi
non l'ha mette tarocco, dopo giuocata la prima
carta si mostrano i scarti. Chi rifiuta paga
uno resto per uno agli avversarj.

La prima volta, che si giuoca un palo,
al faglio si dà il Re, e non si può dare il Matto
in quest'unico caso; per salvare un Re bisogna
impiccarlo, cioè non giuocarlo la prima volta,
che si giuoca al suo palo.

Ogni volta che si prende una carta di conto
agli avversarj si segna il valore di detta carta
purché non sia stata accusata in verzicola,
perché allora non si conta da principio. Il
solo ventinove si conta in verzicola da principio,
ancorché sia stato preso dagl'avversarj,
ma poi non si conta nella medesima, perché
non v'é.

\subsubsection{nota allegorica.}
{\footnotesize
Lo scartare, e il giuocar bene si può adattare al
vivere con se stesso, e il fare bene i fatti, suoi.
Seneca ci dà un bel modo di vivere con noi medesimi
ep. 82. Sic certe vivendum est tanquam
an conspectu vivamus, sic cogitandum tanquam aliquis
in pettus intuitum trabere possit. E con gli altri:
Non erit tibi scurrilis, sed grata urbanitas.
Sales tui sine dente sint, joci sine vilitate, risus sine
cachinnatione, vox sine clamore, incessus sine tumultu,
quies tibi non desidia erit, \& cum ab aliis
luditur, tu sancti, honestique aliquid tractabis.
Giuocando poi non siate negligente, massimamente
quando vincete, perché è vero quello di Tito Livio
Negligentia nascitur ex re bene gesta.
}
\subsection{Modo di contare.}

Finito il giuoco, si pongono le carte a tre
per tre, cioè due che non contano, ed
una sopra di conto, e si fanno tanti di questi
monticelli, quante sono le carte di conto,
che si hanno. Quattordici di questi monti
fanno il numero delle carte, che si avevano,
cioè quarantadue, tutte le altre sono guadagnate,
e quante se ne guadagnano, tante subito
si segnano, se ne possono guadagnare
2.\ 6.\ 10.\ 14.\ 18.\ 22.\ 26.\ ecc.

Poi si conta tanto, quanto si ha di carte
accusate da principio, poi tutte le verzicole,
che si hanno, poi tutte le carte, l'ultima, ed
i segnati, da tutto questo computo si leva il
conto degli avversarj, e il residuo dà la vittoria
di tanti resti, quante volte il sessanta
entra in detto residuo, e uno di più, se avanza
qualche punto, che pur dicesi entragnos.

\subsubsection{nota allegorica.}
{\footnotesize
Sommamente accorto è quegli, che sa contare,
ed adoprare gli Uomini per quello, che vagliono.
Tutti hanno qualche abilità, e perizia, ma non tutti
vagliono in una cosa, in un'altra.

Hic satus ad pacem, hic castrensibus utilis armis.
Naturae sequitur semina quisque suae Prop. 1.\ 3.

I sommi ingegni tal volta non sono gli ottimi
in tutte le cose, singolarmente nel governo. Tucid.
lib. 3. Hebetiores ut plurimum melius Rempublicam
administrant, quam acutiores, e poi ne rende
ragione. Onde Euripide disse:
\textit{
Mens qua sapit nimium, non sine damno sapit.}
È necessario ancora, ché voi intendiate il vostro valore
per esercitarvi in quello Cic. ae quisque nocat
ingenium, et quam quisque novit artem in ea
se exerceat. Q.\ Tutc.
}
\subsection{Spiegazione de i termini\\ proprj di questo giuoco}

Non penso, che siavi alcun giuoco, in
cui si adoprino termini più stravaganti
di questo, de i quali è ben difficile a renderne
ragione, perciò noi faremo come i Geometri,
che non mai definifscono le cose, ma
i nomi, \textit{definitionibus Geometricis vocabula
artis explicantur}. Clau. in Eucl. ad lect.
\begin{description}
\setlength{\itemsep}{-3pt}
\item[Affogare un Re] ovvero {\bf impiccare un
Re} vuol dire non giuocare il Re la prima
volta, che si giuoca ad un palo. S'impiccano i
Re per non perderli.
\item[Ammazzare un Re] vuol dire prenderlo
all'avversario.
\item[Ammazzare un Papa] vuol dire prenderlo
all'avversario.
\item[Morire] vuol dire prendere qualunque
carta di conto. Le Trombe sono immortali,
e il Matto muore solamente nella rovina
universale di tutto il giuoco.
\item[Rubare] vuol dire, alzando, trovare
carte di conto, che si prendono. Similmente
si chiamano rubate le carte di chi fa la fola, se
dopo l'undecima trovi in fola altre carte di
conto.
\item[Smattare] vuol dire rispondere col Matto.
\item[Rifitta] vuol dire giuocar cartiglia, che la
abbia l'avversario posto a sinistra.
\item[Fare] vuol dire non aver più d'una seguenza,
e per ciò rispondere con tarocco.
\item[Far caccia] vuol dire lasciare il giuoco in
mano altrui per aspettare carta di conto
vantaggiosa.
\item[Far tenuta] vuol dire passare di sopra mano
carta maggiore a quella, che si vuol prendere
all'avversario, acciocché allora non se
la faccia.
\item[Far passata] è dar tarocco geloso sopra
 cartiglia con pericolo.
 \item[Ganzo] è quello, che assiste a chi
non sa troppo giuocare.
\item[Girare una carta] è quando si giuoca in
faccia al compagno mandalisi carta gelosa.
\item[Girare il gioco] è giuocare da principio
i tarocchi maggiori.
\item[Fumare] è giuocare un papino in segno al
compagno d'avere buon giuoco, si fuma
ancora con un sopraventi.
\item[Cascare] vuol dire non aver più tarocchi,
ed allora, se non vi sia in mano qualche Re
da dare al compagno; si pongano tutte le carte
in tavola, che di mano in mano le prendono
quei che fanno le levate.
\item[Entragnos] sono le carte di conto che sono
in fola, ed i punti, che avanzano, che
fanno un resto.
\item[Fola] sono le ultime tredici carte,
 che restano a monte.
 \end{description}

\subsubsection{nota allegorica.}
{\footnotesize
Il parlare, e lo scrivere polito, elegante, e con
termini proprj della buona lingua è cosa assai civile,
e comendabile; perciò deesi fare studio di adoprare
vocaboli buoni, e ricevuti dall'uso delle persone
culte, e letterate del paese. Non già per parlar
bene andar ricercando certi rancidumi de' primi Scrittori
se pure l'uso non gli richiami, perché
\literaryquote{
Multa renascentur, qua jam cecidere, cadentque.\\
quae nunc sunt in honore vocabula, si volet usus\\
quem penes arbitrium est, \& vis, \& norma loquendi.}
}
\subsection{Leggi del Giuoco.}

Chi non trova niente in fola, era legge
che dovesse pagare un resto, ma ora è
stata abolita, come crudele.

Chi sbaglia le carte, paga per la prima
carta sbagliata venti punti, per le altre dieci
per una agli avversarj. Chi ne ha avute di
meno, le prende in fola prima meschiata.

Chi non s'accorge d'aver ricevuto carte
di più, o di meno, in fine di giuoco non conta
nulla di quanto ha fatto, conta però le verzicole,
che accusò da principio, le carte, e
l'ultima.

Chi rifiuta, cioè non risponde al palo
che si giuoca, avendone, paga un resto per
uno agli avversarj, e si aggiusta la mano.

Chi non accusa la verzicola prima di
giuocare la prima carta, tal verzicola da
principio non conta più.

Chi casca, e mette le carte in tavola,
non n'è più padrone di ripigliarle.

Se nel mazzo manca qualche carta, non si
rifanno le carte, né si muta mazzo, la carta
per terra va in fola a monte.

Gli errori massicci di questo giuoco il Zipoli
li racchiude in una stanza all'ottavo cantare
del Malmantile stanza 61.

\literaryquote{\normalsize Apunto il Generale a far s'è posto\\
Alle Minchiate, ed è cosa ridicola\\
Il vederlo ingrugnato, e mal disposto:\\
Perché non ha accusata una verzicola,\\
 Le carte ha dato mal, non ha risposto,\\
E poi di non contare anco pericola,\\
Sendo scoperto aver di più una carta:\\
Perché di rado quando ruba scarta.}

\subsubsection{nota allegorica.}
{\footnotesize
Ottima cosa sarebbe non v'ha dubbio trovare
un'amico da conversare senza difetti, ma dove egli
stia di casa, non si è scoperto ancora: Ne hoc præceperim
tibi, ut neminem nisi sapientem attrabas.
Ubi enim istum invenies, quem tot sæculis quærimus?
pro optimo sit minime — Seneca d. c. 7.

Perciò i difetti tollerabili nell'amico si debbono
scusare, e questo è l'ottimo nelle umane cose, Orazio,

 Vitis sine nemo nascitur,\\
 Optimus ille, qui minimis urgetur,

Bisogna prendere gli Uomini come sono, purché
non siano insofribili, che se si muta la loro familiarità,
per l'ordinario si va di male in peggio. Hom,
Illiade.

\literaryquote{Nemo hominum in Terris vitio sine nascitur usquam.}

Poi sono vi alcuni difetti scusabili, come colui che
beve del vino, pare, che sia contro la sobrietà, pure
da Orazio, da Marziale, e da simili Poeti è preferito
a chi beve acqua.

\literaryquote{Aquam bibens, probum, \& utile paries nibil\\
 Fæcundi calices quem non fecere disertum?}

Questa regola di scusare in altrui i piccoli difetti
non abbastanza sarà mai ricordata a i cogniugati,
che, se tra loro fossero discreti, come conviensi,
non verificherebbero sì spesso quel di Giovenale
alla Sat.~6.

\literaryquote{Semper habet lites alternaque jurgia lectus,\\
In quo nupta jacet, minimum dormitur in illo}

Avvertano le Donne, che all'eccesso s'inquietano
co i loro mariti, per fino ad abbandonarli, poiché
dice Plutarco, che fanno come colui, che punto
dall'Api abbandona il miele. \textit{Sponsæ, qua flatim
offensa moribus maritorum, eos deserunt: perinde
faciunt, ac si quis ictus ab apibus mel relinquat}.
Poi s'intima agli Uomini, che, \textit{qui non communicant
lusus, \& jocos cum uxores faciunt, at alibi
quaerant insciis viris}
}





\subsection{Giuoco di Giro.}

Avendo scoperto in mano del compagno,
o per verzicole accusate, o per fola, e
per le scoperte, che ha molte delle carte
superiori, e voi vi trovate o tutte, o in gran
parte delle altre, allora si fa il giuoco di giro
per far cadere gli avversarj, e prenderli
quant'anno. Subito, che vi capita il giuoco
in mano, che dovete procurare, che sia quanto
prima, fumate o con un papino, o con un
sopraventi, oppure giuocate un sopratrenta
in segno delle Arie, che voi avete, e così
alternando, ora voi, ora il compagno a pigliare
sempre con le vostre superiori, farete, che
i contrarj finalmente, non potendo resistere a
tarocchi, vi cadino in mano con tutte le loro
carte di conto, o almeno con alcune di esse
le più gelose, che sono 1, 3, 20, 30, 28.

\subsubsection{nota allegorica.}
{\footnotesize
Qui mi cade in acconcio di parlare della Rota
della Fortuna, della qual Deità si ride Giovenale
alla Sat.\ 10.\ 3

\literaryquote{Nos facimus fortuna Deam Caeloque locamus.}

Ovidio meglio indovinò, che cosa è la Fortuna,
quando disse ep.\ 1.:

\literaryquote{Ludit in humanis divina potentia rebus.}

Le vicende umane girano tal volta gli Uomini
a tali felicità, alle quali neppure mai avrebbero o
aspirato, o pensato; e tal volta da sommi gradi precipitano
al fondo di tanta miseria, che divengono
obrobrio degli Uomini, e derisione del popolo a
chiunque, che o nell'una, o nell'altra sorte si trova,
avverto con Ausonio.

\literaryquote{Si Fortuna juvat caveto tolli. \\
Si Fortuna tonat caveto mergi.}

Certo è, che 1a fortuna sia ella favorevole, o
avversa, sempre è cattiva. \textit{Fortuna cum blanditur,
 tune maxime metuenda}. Salust. Ma quando sia peggiore,
non è deciso ancora, ed è problema assai disputabile.
Il Comune vorrebbe la buona ventura
nel Mondo; pochi, e nascosti si contentano d'ogni
cosa, né si lagnano nelle disgrazie. L'indole per altro
di ciascuno molto dipende dalle vicende. Terent.

\literaryquote{Omnibus nobis ut res dant sese, ita magni, atque
humiles sumus.}

Ma perché radi sentirete darsi il vanto di fortunati,
che tra poco non vediate precipitati, perché come
avverte Silvio Italico.

\literaryquote{Brevis est magni Fortuna favoris.}

Quindi è, che nelle traversìe, e nelle quanto si sian
grandi disgrazie, conviene all'Uomo saggio mantenere
l'animo tranquillo, dunque \textit{conare amentiam
fortunae, animo forti sustinere}. Menandro. Giacché
è verissimo quello, che maravigliosamente disse
Cicerone in 5.\ lib.\ Ep.\ \textit{\textsc{Praeter culpam, aut peccatum
homini accidere nihil potest, quod sit
horribile, aut pertimescendum}}. Tanto è: nel
giuoco della vita umana, che è mescolato di
fortuna, e d'ingegno, come lo è quello delle minchiate;
l'ingegno s'adopri per conservarsi giusto, e santo.
La Fortuna s'osservi per ridersi d'essa. L'animo
umano è troppo grande per soggettarsi ad essere
vellicato da queste cose esteriori, che sotto il
dominio della Fortuna vacillano. Concludo dunque:
\textit{Divitie si diliguntur, ibi serventur, ubi perire non
possunt; honor si diligitur, illuc habeatur, ubi nemo
indignus honoratur; salus si diligitur, ibi adipiscendo
desideretur, ubi adepta nihil timetur: vita si
diligitur, ibi acquiratur, ubi nulla morte finitur.}
}
\supersection{Avvertimenti per giuocar bene.}

\section{CAPO SECONDO.}

\subsection{In che cosa consista il giuocar bene.}

Quegli giuoca bene a Minchiate, che
con l'istesse carte procura, o di vincere
più, che sia possibile, o di perdere
meno che sia possibile: dissi procura di
vincere, e non risolutamente vince; perché
può darsi caso, che si giuochi con ottimo
consiglio, ma ne succeda male, ed in tal caso
non dee biasimarsi colui per l'infelice successo:
Siccome non merita lode, un felice successo
d'una temeraria impresa: \textit{Aliqui omnia imprudenter
agunt, et ipsis aliquando pro voto
succedunt: quidam ad omnia se consilio disponunt,
et sæpe in contrarium eis cedit.} Cassiod.\ 1.\ 3.\ p.\ 2.
Perciò il mestiere di consigliere
è molto pericoloso, perché, come asserisce
Demostene, \textit{duram esse conditionem corum,
qui in dandis consiliis, et in gerendis
rebus solent versari, cum eorum fides ex eventu
soleat extimari}. Quindi è, che talvolta si
giuoca una carta con ottima ragione, che poi
il caso conduce a mal porto, ed allora si sentono
le alte querele del compagno, le derisioni
degli Avversarj, le dispute de i spettatori;
allora meglio è tacere, e seguire l'aurea
sentenza di Cicerone, che dice, \textit{ferendas
esse linguas, et qualemcumque fortunam,
praefertim, quae absit a culpa}.

Siccome si dee vincere il giuoco con
quelle carte, che il puro caso ci fa capitare
alle mani, così nella vita umana, che altro
non è, che un giuoco, \textit{quomodo et fabula,
sit \& vita}. Sen. Le disposizioni del corpo, le
passioni dell'animo, gli Uomini, che ci circondano,
gli affari, che ci occupano, le contingenze,
che ci occorrono, sono come un
miscuglio di carte, che continuamente abbiamo
alle mani, che c'intrigano per lo traffico
della nostra eterna felicità, alla quale l'esercizio
delle sole virtù ci può condurre.

E perché il giuocar bene consiste in gettare
convenevolmente volta per volta ciascheduna
delle carte, che si anno in mano,
così ogni nostra azione dee essere concorde
colle regole del giusto, dell'onesto per il
fine sudetto. \textit{Vita humana ex multis actionibus
consistit: at felicitas actiones absolvit}.

Tutte le maniere di giuocar bene a Minchiate
riduco in quattro Capi. Primo scartare.
Secondo rispondere cartiglia a cartiglia, tarocco
a cartiglia, e tarocco a tarocco. Terzo, giuocare
cartiglia. Quarto, giuocar tarocco, per i
quali seguono dodici precetti utilissimi.

\subsection{PRECETTO PRIMO.}

Prima d'ogni cosa, ricevute le carte, bisogna
guardarsi di non dar segno alcuno
della qualità di esse, o col rallegrarsi o col
rattristarsi, o in qualunque altra maniera,
ma si dee stare sempre eguale, ilare, ed attento.

{\footnotesize
 Molto bene a proposito avverte Euripide:
 \textit{Palam enim fortem suam testando,
apud omnes imperiti hominis est, celare vero
sapientis. Hujusmodi enim gaudium hostium, et risum movent}, e Pindaro:
\literaryquote{Alteri aperire noli, quidnam taboris nos exerceat\\
 Hoc te admonitum volo.}}


\subsection{PRECETTO II.\\
 \footnotesize Scartare.}
Si faccino fagli più, che si può; primo,
quando non si ha buon giuoco, per potere
amazzare i Re; secondo, quando si hanno
carte gelose da salvarsi, per farsele su le
Cartiglie; terzo, quando si anno 10, o più
Tarocchi, perché in tal caso difficilmente si
cade. Si averta di scartare in quel palo, di
cui ne sono meno in fola, perché più sicure
siano le passate, se poi si anno pochi Tarocchi
per sostenersi, si scarti ove ne sono più in
fola.

\subsubsection{nota allegorica.}
{\footnotesize

Le Cartiglie sono, come il comune del volgo,
la natura di costoro, bene osservò Titto Livio dec.\ 2.\ l.\ 4.
\literaryquote{Vulgares, aut serviunt humiliter, aut superbe dominantur.}

I miseri, ed i ricchi faranno bene
a scansarne il comercio, e molto più la familiarità.
I potenti debbono essere loro affabili, e cortesi per
avere l'applauso, e l'opera universale. \textit{Caeteris
mortalibus in eo stare consilia, quod sibi conducere
putent. Principum diversam esse sortem, quibus
praecipuam rerum ad famam dirigendam}. Tac.\ 4.\ ann.
La pena poi di chi giuoca con gente plebea, e
misera si è di perdere, e pagare, e di vincere, e non
riscuotere, perché, come disse il Zipoli
\literaryquote{Di rapa sangue non si può cavare, \\
Ne far due cose perdere, e pagare.}
}

\subsection{PRECETTO III.}

Non si dee far faglio, quando si abbiano
le Trombe, e pochi Tarocchi, cioè
meno di nove, perché si caderebbe presto e
mancherebbe al compagno la difesa, ed a voi
il piacere di far caccia. Si averta di scartare
in quel palo, ove se ne hanno più in mano.

\subsubsection{nota allegorica.}
{\footnotesize
Tale è la parsimonia, che conserva le sostanze
per le gravi occorrenze. \textit{Non intelligunt homines,
 quam magnum vestigal sit parsimonia}. Cic. in paradox.
ed Ovidio de Arte.

\literaryquote{Non minor est Virtus, quam querere parta tueri.\\
Casus inest illic, hic erit Artis opus}

Quasi tutti cercano arricchire, ma pochi assai ne
sanno la maniera: una efficacissima si è il non ispendere
mai nulla in cosa superfiua. \textit{Divitiae grandes
 homini sint vivere parce}. Lucan.\ l.\ 9. Se mi si chieda
qual sia più lodevole quegli, che di povero divenne
ricco, o quegli, che seppe conservare il patrimonio, o l'eredità: rispondo con Claudiano l.\ 2.
\textit{Plus est servasse repertum, quam quesisse decus};
L'ottima maniera d'arricchire ce la insegna S. Agostino
nelle epistole: \textit{Si vis esse Mercator optimus,
foenerator egregius da quod non potes retinere, ut
recipias, quod non poteris amittere: da modicum, ut
recipias centuplum, da temporalem possesionem, ut
consequaris hereditatem eternam}.
}

\subsection{PRECETTO IV.}

Tenere a memoria il numero delle carte
di ciascun palo, che sono in fola, e
di mano in mano, che si giuoca fare il conto
quante ne restano, ed in mano di chi, per
regolarsi, si nel passare, si nel dare le rifitte,
e non far cadere il compagno, o non mandarlo
con qualche carta gelosa a morire, ma
farli fare su le Cartiglie il suo giuoco: e questo
è il più utile Precetto di questo giuoco;
chi sa questo eseguire è bravissimo Giuocatore;
chi non ha tal solerzia, e prontezza di
spirito, e di memoria, non sarà mai riputato
buon giuocatore di Minchiate.

{\footnotesize
E ben vero, che molti sanno questo precetto,
e volendo possono metterlo in prattica,
ma siccome ricerca un'attenzione assai
laboriosa, a chi giuoca con tanta fatica, stimo,
che si convenghino tutte le lodi, che Tolomeo
Filadelfo diede al Libro di Eraclide Sofista,
intitolato \textit{De Laudibus laboris} a cui Tolomeo
mutò questo titolo in quest'altro: \textit{De Laudibus Afini bestiarum laboriosissima}. Un piccolo
interesse, ed un giuoco, che si fa per
onesta ricreazione » non merita certo
un'attenzione, che affatichi l'animo, tolga il
tepore di conversare, e manifesti un'avidità
troppo grande di superare i compagni. Tengasi
a memoria quel bel detto di Terenzio
molto usato da Tommaso Moro; \textit{Illud nihil
nimis, nimis mihi placet}.
}

\subsection{PRECETTO V.\\
 \footnotesize Rispondere Cartiglia a Cartiglia.}

Prima passate il Re, o impiccatelo, se
dubitate del faglio per lo scarto dell'aversario;
secondo mettete su le seconde le
Regine, per fare la levata, e poi poter presto
giuocare nel faglio, o scarto del compagno;
se poi volete, che quello vi giuochi nel
vostro, mettete le carte inferiori. Quando il
Matto non può avere uso migliore tal volta
giova a salvare un Re affogato, cioè quando
è restato solo, e si giuoca a quel palo, al
quale il compagno non ha fatto ancora, si dà il Matto.

\subsubsection{nota allegorica.}
{\footnotesize
Può adattarsi il rispondere cartiglia a cartiglia
al trattare dell'Uomo con se medesimo, e prima
bisogna, secondo quel celebre detto Delfico, che
l'uomo conosca se stesso, \textit{nosce te ipsum}, cioè le
proprie passioni, inclinazioni, vizj, abiti, e simili.
Ogn'uomo, siccome è creato per la fruizione, e
possesso d'un'infinito bene, quanto sia vasto
l'oggetto, a cui aspira, e l'inesausta voglia del piacere,
né pur'esso comprende a bastanza, quindi è, che ha
una stima di se assai stravagante; ogni bene, che li
succede lo reputa dovuto al suo merito, ogni male
ingiusto, ecco l'origine d'ogni disordine, la superbia;
quest' Idra abbattuta nel cuore umano, subito
non solo farà d'ogn'altro vizio purgato, maeziandio
d'ogni più eccelsa virtù girà nobilmente fastoso;
a tali io dico con Prudenzio.
\literaryquote{Define grande loqui frangit Devs omne superbum.\\
Magna cadunt, inflata crepant, tumefatta praemuntur\\
Disce supercilium deponere, disce cavere \\
Ante pedem sovean, quisquis sublime minaris}

Per pungere al vivo questi palloni di vento legghino
pochi versi di Simone Nanquerio, ove parlando
dell'uomo dice
\literaryquote{Concipitur foeda primorum sorde parentum,\\
Nescitur, et nudus, plorat, et ortus humi\\
O miser interius secreta mente revolvas,\\
Carne quid humana vijius esse potest?\\
Cerne quid emittant nares, cunctique meatus\\
Corporis assidua sorde fluente tua.\\
Inspice quam fragilis, quam pauper, quamque misellus, \\
Vivas, tuta dies von tibi nulla datur.\\
Aspice vel prorfus quid sis, nisi putre cadaver, \\
Quod tamen minimis vermibus esca datur.
}
}
\subsection{PRECETTO VI.\\
 \footnotesize Rispondere Tarocco a Cartiglia.}

Quì ricorre il Precetto quarto, che chi
tiene a memoria le Cartiglie uscite,
e da uscire, chi vi fece, e chi nò,
passerà sicuramente le sue carte gelose; quì
aggiungo solamente, che su le prime si passa
qualunque cosa, e su le seconde ancora, su
le terze si può arrischiare un papino, o altra
carta di conto, ma di poca importanza; su le
terze sopra Tarocco si azarda più facilmente
un sopra trenta, s'averta però, che in fola
non ve ne siano più di tre. Chi osserva il
Precetto quarto, passerà sicuramente anche su le
quarte, e su le quinte.

\subsubsection{nota allegorica.}
{\footnotesize
Per applicare questo Precetto al nostro vivere,
trattandosi di Tarocco, che prende tutte le Cartiglie,
stimo, che convenga al superiore, qualor tratta
con gl'inferiori, ed eccone un'ottima regola,
che ne dà il morale Filofofo alla ep.97. \textit{
 Sic cum inferiore vives, quemadmodum superiorem
 tecum vivere velis.} Si noti quello, che dice
in appresso, \textit{
omnes tibi pares facias, inferiores superbiendo non
contemnas, superiores recte vivendo non timeas}.
Per altro il superiore non dee facilmente azardarsi
a scherzare coll'inferiore, massimamente in cose
d'ingegno; perché può accadere, che ne resti
confuso. E famoso quello, che fu risposto da un certo
Neka mal'uomo al Cardinal Filippo Repington
Inglese, gli disse il Cardinale.

\literaryquote{
 Et niger, © nequam cam sis cognomine Nekam.\\
 Nigrior esse potes, nequior esse nequis.}

Rispose Neka.

\literaryquote{Phy nota faetoris, lippus malus homnibus horis.
Phy malus, \& lippus, totus malus ergo Phylippus.}
}
\subsection{PRECETTO VII.\\
 \footnotesize Rispondere Tarocco a Tarocco.}

Se il Tarocco vi si giuoca in faccia, cioè
sotto mano, o sia dalla destra fatevi ciò
che volete, e potete; se sopra mano girate al
compagno; se dal compagno, quando non
avete voi giuoco da far caccia, coprite per
sostenere il vostro compagno, o per far tenuta
a qualche carta importante.

\subsubsection{nota allegorica.}
{\footnotesize
 Qui gioverà fare alcuna riflessione per trattare
convenevolmente, e cautamente cogli eguali. In
primo luogo si averta, che due sono i cardini su i
quali ogni azione umana comunemente ragirasi.
Amore, ed interesse; onde fra tanti, che ti
circondano, sappi, che rari sono quelli, che di vero
amore d'amicizia ti amano. \textit{Vulgus
 amicitias utilitate
probat}. Ovid., e l'Ariosto c. 4.\ ft. 2.
\literaryquote{
Che dopo lunga prova a gran fatica\\
Trovar si può chi ti sia amico vero,\\
Et a chi senza alcun sospetto dica\\
E discoperto mostri il tuo pensiero.}

Ma i loro ossequj i loro officj abili sempre sospetti,
che questi da te voglino un qualche loro vantaggio,
o di commodo, o d'interesse, o di onorificenza o
di diletto, e tanto più, quanto è migliore lo stato
tuo per alimentare questa loro fiducia; allora
t'avedresti quanti pochi siano gli amici tuoi, se
cangiassi
la felicità in miseria; onde molto bene cantò lo
stesso massimo Poeta al Canto 19.\ stanza prima.
\literaryquote{
Alcun non può saper da chi sia amate\\
Quando felice in fu la rota fede,\\
Però che ha i veri e i finti amici a lato,\\
Che mostran tutti una medesma fede, \\
Se poi si cangia in tristo il lieto stato, \\
Volta la turba adulatrice il piede.\\
E quel che di cuor ama riman forte, \\
Et ama il suo Signor dopo la morte.}

E' memorabile il detto di Archita Tarentino.
\textit{Quemadmodum difficile est piscem sine spinis
inveniti, ita, \& hominem, qui non dolosum quiddam
habeat}. Onde, se Chilone Lacedemone a tutti
diceva, \textit{cave te ipsum}, quanto più dico io, \textit{ceteros
homines}. E questo basti per renderci cauti nel trattare
cogli uomini. Per poi convenevolmente con essi
conversare aurea veramente è la regola data da
Seneca il morale nel 3.\ de Ira, si metta bene a
memoria, e molto più in prattica, che è cosa di somma
importanza. \textit{Dum inter homines sumus colamus
humanitatem, non timori cuipiam, non periculo
simus, detrimenta, injurias convicia, contumelias,
vellicationes, comtemnamus, ut magno animo brevia
feramus incommoda}. Riflettete poi, che l'amicizia
è un vero negozio, in cui i Mercadanti sono gli
amici; i commodi, i piaceri, ed i comuni sollievi
sono la mercanzia. L'amore, la sincerità, la fedeltà,
la confidenza sono le monete di questo banco amoroso;
fallisce il banco, quando uno paga sempre favori,
e beneficj, e non ne riscuote mai alcuno; onde
 Seneca in Octo.
\literaryquote{
{\hspace{14em} amor est}\\
Quem si fovere, aut alere desistat, cadit. \\
Brevique vires perdit extinctus suas.}

Uno dei massimi solazzi della vita nostra si è il
godere de i nostri beni con gli amici, e lo sfogare con
essi le nostre perturbazioni, o traversie, queste
senza il sollievo dell'amicizia ci opprimerebbero, e
quelli quasi nulla ci consolarebbero.
\literaryquote{
Quid tibi jucundum submotis esset amicis?\\
Cuncta tibi quamquam sint cumulata bona.}

}

\subsection{PRECETTO VIII.}

Suole bene spesso accadere, che, giuocandosi,
siate ripreso dal compagno di qualche
errore, o di qualche inavvertenza che
veramente abbiate commesso, o che egli se la
figura, o che di naturale è querulo, inquieto,
irrisore. In tal caso meglio è tacere, se
questo non basta, pregatelo a voler sospendere
le querele fino alla fine del giuoco, perché
allora pienamente giustificherete la vostra
condotta, non potendo dire le vostre ragioni,
che sono fondate nelle carte, che avete in
mano; se questo né pur basta, eccovi un
precetto il più utile, ed il più necessario, che io
abbia mai inteso, ed è d'un savio Cortiggiano
pieno di trionfi in simili battaglie, o per
dir meglio assalti di parole:
\literaryquote{Nobile vincendi genus est patientia, vincit\\
Qui patitur, si vis vincere disce pati.}

E incredibile quanto amore si concilia quegli,
che così opera, né mai si adira per si fievole
cagione, onde soleva dire S. Anselmo
\literaryquote{Ira odium generat, concordia nutrit amorem.}

Ussino ci dà un'ottima regola per conversare
con tutti con plauso, e decoro, ognuno se la
impari, e la prattichi nel conversare
\literaryquote{Sis patiens, verax cornis, sermone modestus,\\
Sint tua laudatis seria picta jocis.\\
Attamen extremum nemo laudabit utrumque\\
Serius aut nimium, sive jocosus eris}.


\subsection{PRECETTO IX.\\
 \footnotesize Giuocare Tarocco.}

Se tocca a voi a giuocare Tarocco, e avete
giuoco da far caccia, fumate al compagno,
acciocché vi sostenga, e poi lasciate il
giuoco in mano ad altrui, finché non capita
quella carta, che bramate, o per farvela, se
voi l'avete in mano. Ma sappiate, che di
tutte queste cose ne fa più scienza un mese di
prattica, che dieci anni di precetti: \textit{Usus
frequens omnium Magistrorum opera superat.}
Cic, 1.\ de Orat.

\subsubsection{nota allegorica.}
{\footnotesize
Conversando con altri procurar sempre si dee
imparar da loro qualche cosa, e questo è uno de'
maggiori emolumenti del Cittadino, che ha sopra
gli abitatori delle campagne, dicea Platone in Phae.
Agri, \& arbores docere me nihil possunt: sed homines,
qui in urbe versantur. Per imparare ottime
sono le regole date da Ugone in Didascalion. \textit{Mens
humilis, studium quaerendi, vita quieta, scrutinium
tacitum, paupertas terra remota. Haec referre
solent multis obscura legendo}. Se mi si chieda
per quanto tempo convenga imparare, rispondo
con Seneca ep. 76. \textit{Tandiu discendum est, quamdiu
 nescias}. Dico questo a proposito di acquistare
qualche carta degli avversarj, e di fare in modo, che il
compagno vi ajuti. }

\subsection{PRECETTO X.\\
 \footnotesize Cascare.}

Non deesi cascare, quando si hanno in
mano i Re affogati, finché non siano
dati, se si può, nelle levate fatte dal
compagno, o quando probabilmente tutti ancora
abbiano cartiglia. Fuori di questi due casi si
gettano le carte in tavola che le prendono di
mano in mano, chi fa le levate; consiglio però
a tenerle fino alla fine.

\subsubsection{nota allegorica.}
{\footnotesize
Siccome questi miei precetti sono diretti al savio
conversare con altri, il più arduo caso, che loro
accada, si è il conversare con gran Personaggi,
stimano essi il più fausto incontro trattare i gran Signori,
ed è veramente. Ma o quanto con essi facilnenté si
casca. Ecco dunque le avvertenze per evitare più
che si può le cadute, si osservi, se il Principe prevale
nell'ingegno, o nella potenza, o in ambedue.
Se è di grande ingegno, allora quel poco, che se ne
ha, s'impieghi in tener nascosti i suoi interni pensieri,
o difetti. Dante.

\literaryquote{
Abi quanto cauti gli Uomini esser denno.\\
 Presso a color, che veggon pur l'opera\\
Ma per entro al penfier miran col senno.}

Se poi prevalgono in potenza farai appò loro,
come il secchio del pozzo, che quando è pieno,
tira in alto, ma quando è voto, si precipita al fondo
del pozzo. Se tu loro sarai utile per l'adempimento
de i loro fini, salirai in grazia, in favori, onori, e
dove vuoi. Se cesserè in te questa dote, allora sì,
che tutto d'un colpo ti troverai sommerso fin nei
fondi dei più cupi pozzi della confusione: \textit{Brevia
momenta plerumgue extollunt homines, rursumque
deprimunt inprimis vero eos, qui in aulis versantur,
illi enim profecto similes sunt mensarum calculis,
qui pro eo, ac calculatori libet mode assem, modo,
talentum valent. Ita homines ad nutum Principis,
modo miseri sunt, modo beati}. Appresso Polibio
lib 5.

Quando i gran Signori sono potenti, ma di poca
elevatura, s'incorre con essi il gran pericolo delle
Corti, cioè della maldicenza e calunnia di coloro,
che gli stanno a fianchi, singolarmente, se costoro
mai fossero Uomini di perduta coscienza, vili,
e mal nati, de' quali dice Crispo Salustio \textit{ad Cæsarem.
 Ad reprebendenda aliena facta, \& dicta ardet
 omnibus animus: Vix fatis apertum os, eorum
 lingua prompta videtur}. onde loro ben s'adatta il
nome datoli da Apulejo: \textit{Insidiatores, \& Canes
clanculam mordentes, qui secretis criminationibus
infament ignarum, \& quod incautior decipiare palam
laudatum} e quel, che è peggio, ti fingono
amicitia, come avverte Tacito an. 14. \textit{Sunt, qui
specie amicitiae dolum tibi parent}, che, se sei buono,
e semplice senza dubbio veruno \textit{invenies supplantatores,
 \& funambulos simplicitatis tuae}, Tertull. Bisogna
dunque non mediocre avvedutezza per guardarsi,
difendersi, e premunirsi da costoro per non
cascare. Ecco adunque gli appoggi più efficaci.

Seneca ce ne dà uno bellissimo nel 3.\ de Ira.
\textit{Si imbecillior est, qui te lafit, parce illisf potentior,
 tace}. Un'altro pur'ottimo ne dà nel l. 2. \textit{
 Potentiorum injurias hilari vultu non patienter tantum ferendas
 esse}. Sofocle dice, che molto giova esser
sobrio, e non fidarsi di alcuno: \textit{Sobrius sis, \& memor
 esto nulli credere, seu confidere}. In fine voglio
raccontare, che Alessandro Magno aveva tra i suoi
un certo Calistene filofofo inettissimo nella scienza
del vivere sociale. Solea il Re spesso ripetere in sua
presenza: \textit{Sapientem ego odi, qui sibi opis nihil}

Ora resta a parlarsi, comte debbasi trattare con
quel Signore, che tale è veramente di senno, e di
potenza, con tali meno pericolosa è la cosa, leggasi
però il capo 13.\ dell'Ecclesiastico. Ma come dissi
intorno al savio, dotto, e giusto Principe, è difficile,
che l'Uomo d'onore non venga onoratdod, amato,
ed esaltato ancora alle occasioni, di questi però se ne
trovano rari, ed è forte non piccola in tali incontrarsi.

Infelicissimo certamente è colui, che con un
Prencipe ignorante, e bestiale è portato a vivere i
giorni suoi, perché con tali né ragione, né ben'operare
giova, anzi talvolta pregiudica Dante:
\literaryquote{
Che dove l'argomento della mente \\
S'aggiunge al mal volere, \& alla possa\\
Nessun riparo vi può far la gente.}

Dunque dovendo voi scegliere un'amico, e vi si dia
il poter scegliere, tra il potente, e l'eguale, questo
a quello preferite, benché vi paja del primo minore, perché
\literaryquote{
Dulcis inexpertis cultura potentis amici, \\
Expertus metuet.}

Ma se il vostro destino vi guidi a dovere altrui servire.
Dante Par. c.17.
\literaryquote{
Tu provarai siccome sa di sale\\
Lo pane altrui: \& come è duro calle\\
Lo scender, e 'l salir per l'altrui scale.}

Meglio di tutti è il consiglio di Ovidio a chi può eseguirlo.
Tristium.
\literaryquote{
Usibus edocto, si quicquam credis amico, \\
Vive tibi, \& longe nomina magna fuge.}
}

\clearpage
\subsection{PRECETTO XI.\\
 \footnotesize Valore de i Tarocchi.}

\begin{wrapfigure}{r}{3.8cm}
 \vspace{-12pt}
 {\scriptsize\begin{tabular}{@{}rr|r}
 \multicolumn{2}{@{}c}{Sole levano}&\multicolumn{1}{c}{Matto con}\\
 \multicolumn{1}{@{}c}{Matto}&55&\multicolumn{1}{c}{un'altra}\\
 30&55&105\\
 1&55&75\\
\multicolumn{1}{@{}c}{Trombe} & 50 & 95\\
\multicolumn{1}{@{}c}{Sole} & 50 & 105\\
\multicolumn{1}{@{}c}{Mondo} & 40 & 95 \\
20&40& 90\\
\multicolumn{1}{@{}c}{Luna} & 30 & 85\\
\multicolumn{1}{@{}c}{Stella} & 30 & 85\\
13 & 30 & 70\\
35 &10 & 70\\
34 & 10 & 70\\
33 & 10 & 70\\
32 & 10 & 70\\
31 & 10 & 70\\
29 & 10 & 65\\
28 & 35 & 80\\
 3 & 28 & 78\\
\multicolumn{1}{@{}c}{Re} & 15 & 70\\
10 & 15 & 70\\
2 & 14 & 69\\
4 & 14 & 67\\
5 & 9 & 64\\
Ultima & 20&
\end{tabular} }\end{wrapfigure}

Acciocché si sappia il valore de i
Tarocchi, quanto voglia dire averli, o non
averli ecco la seguente tabella, in cui il valore
della mancanza d'una sola carta di conto si
esprime. Tutte insieme le carte di conto contano
354, alle quali si debbono aggiungere le
verzicole da principio, le carte vinte,
dieci dell'ultima con i segnati delle scoperte,
e degli amazzati.


Questa Tavola si potrebbe seguitare sino a combinare due
carte, ma per due
motivi me ne astengo. Primo, perché
con essa si toglierebbe a questo giuoco,
uno degli ottimi fini, per cui è stato
istituito, qual'è
quello di addestrare
la gioventù ad impratichirsi de i numeri,
e prontamente sapere a memoria computarli. Secondo, perché ella
è sì lunga, e tediosa, che io non ho
ozio per simile faccenda: se però a
qualchuno venisse voglia di prendersi questa
briga, sappia che dovrà fare due Tavole con
le seguenti combinazioni de i soli Tarocchi di
conto senza i Re.

\begin{tabular}{@{}l@{ }l@{ }r}
Per le carte&\multicolumn{2}{@{}r}{a 2, a 2 combinazioni 231}\\
& a 3, a 3 & 1540\\
& a 4, a 4 & 7315\\
& a 5, a 5 & 26334\\
& a 6, a 6 & 74613\\
& a 7, a 7 & 170544\\
& a 8, a 8 & 319770 \\
& a 9, a 9 & 497420 \\
& a 10, a 10 & 646646\\
& a 11, a 11 & 705432
\end{tabular}

I Re fanno combinazioni a parte.

Chi in queste, o somiglianti cose
appartenenti alla scienza de' Giuochi è vago volersi
istruire, legga tutto il Capo V. della mia
Arimmetica Speciosa, ove abbondantemente
resterà appagato di questa dissicilissima scienza.
Qui non si vogliono proporre, che cose,
che solamente sollievino l'animo, e lo allettino
con profitto, non cose, che molto lo
applichino, e lo affatichino.

\subsubsection{nota allegorica.}
{\footnotesize
 Siccome in questa Tabella subito si vede il
 valore delle carte; e la loro efficacia in scemare il conto
dell'Aversario, così io trovo qui materia di svelare
una delle massime proprietà del Giuoco, che è
quella di manifestare un nomo tal qual'è nel suo intimo
del cuore, e nelle sue naturali, o acquistate perturbazioni
dell'animo. Poiché nel giuoco e squisitamente
si vellicano le più risentite passioni, ivi si
vedono alcuni, che investiti dalla cupidigia di
acquistare, usano ogn'industria, adoprano ogn'arte,
muovono ogni pietra per vincere; la fraude, la
menzogna, l'inganno sono i loro famigliari ripieghi; la
simulazione, e la dissimulazione sono i
mezzi, che usano i meno colpevoli: quegli per poco,
altri per nulla s'adirano, altri per dispetto
precipita il giuoco; v'è chi loda gli errori per farsi merito;
v'è chi per invidia biasima un'ottima, ed ingegnosa
attenzione. Sfavilla altrui negli occhi un non so che di
grand'incendio nascosto nel seno alla presenza di
qualche suo collusore, e ad altri un non so che di
odio ferale. Non pochi melenzi, ed inetti giuocano
per rendersi del tutto ridicoli. I loquaci, ed i
buffoni, quanto più pajono graditi tanto meno acquistano
di stima: gli ambiziosi, e vanagloriosi fanno il fracasso
più stomachevole, lo sboccato, e empio non
altrove meglio si scorge, in somma l'avaro, il fraudolento,
l'ingrato,, il molle, il lascivo, e se v'è altro
vizio nel Mondo, tutto nel giuoco si osserva, e
singolarmente spicca quella passione, che è la moderatrice
universale delle azioni di colui, che giuoca:
né per quanto cauto, ed avveduto egli sia può
nascondere giammai se stesso il giuocatore: perché,
siccome le vicende, le più strane, le sorprese, le più
inaspettate, le astrazioni, le più inconsiderate
succedono nel giuoco: cosi è difficilissimo, se non
impossibile in tanta strana tempesta tenere tranquillo
lo spirito internamente agitato dalle più vivaci
passioni.

Non però minor mostra fanno nel giuocate le
belle virtù, che ornano un'animo moderato, e
savio. Ivi il liberale, il giusto, il mansueto con
lepidezza, sofferenza, costanza, e moderazione, opera
sinceramente, vivace, ilare, giocondo, e pronto
riparatore, o precisore d'ogni sinistro avvenimento,
e tutto fa cosi bene, che tal resta dipinto nel suo
estrinseco qual veramente è; da ciò ben si vede, che
la via più sicura di conoscere chiunque, si è quella di
farlo giuocare; se tanto è vero, com'è indubitato,
qual'emolumento dal giuoco si cava? Lo dirò
brevemente, chi giuoca perde tempo, denari, e
riputazione (parlo ora de i giuochi sempre detestabili di
resto e di grande interesse, non già delle
Minchiate) ma perché non paja, che io esageri troppo
contro d'un contratto, e comercio il più frequente del
Mondo, eccovi ciò, che ne dice Giovanni Lottini
all'avviso 559. \textit{Il giuoco è cattivo in tutte le parti,
perciò, che mentre si giuoca si sta sempre con sete di
vincere, onde ogni piccola cosa turba il Giuocatore,
e cagione, che s'adiri contra qualunque si sia, non
avendo rispetto né a luogo, ne a persone, né a se
medesimo, senza che tira all'inganno, allo
spergiurare, ed alla rovina de' più cari amici, che egli abbia,
desiderando di vincere tutte le facolta loro. Finito il
giuoco, da poi se l'effetto è riuscito contrario al
desiderio, colui, che perde rimane dolente, e pieno di
disperazione, e per riscuotersi non è male, che ei non
pensa di fare, e potendo, che non facesse, ed avendo
vinto per lo più consuma i danari in cose vane, e
triste, di maniera che, ragionevolmente fu fatta un
legge in Egitto, la quale concedeva ad ogn'uno di
potere accusare il Giuocatore d'ogni sorte di vizj,
senza essere obbligato di stare alla medesima pena,
quando bene non si fussero verificate le accuse tanto
avevano per cosa ferma che di chi giuoca si possa
sospettare ogni male, per grande, che egli sia}

Sono vi poi alcuni, che si lagniano di perder
sempre; giuocando anche a Minchiate, a questi io
insegnerò un segreto di non perdere mai più, e lo
imparai nel Malmantile del Zipoli, e perciò non
voglio essere avaro di manifestarlo a tutti quelli, che
giuocando perdono, ed è sicurissimo, e provato,
eccolo :
\literaryquote{Capitano sai tu quel, che ai a fare, \\
 Se tu non vuoi più perder non giuocare.}

E Giacomo Beneventano di più ci comanda a fuggire
quel giuoco, ove si può perdere la robba sua.
\literaryquote{
Ludum quoque fuge, per quem tua perdere possis.}

Racconta Svetonio, ché Cesare Augusto dopo aver
perduto due battaglie navali giuocava frequentemente
alle carte. Un bell'ingegno ne espresse la ragione
cosi.
\literaryquote{Postquam bis classe victus naves perdidit\\
Aliquando ut vincat Judit assidue aleam.}

Se poi giuocando avete perduto, vi averto ad essere
puntualissimo a pagare per giustizia, per decoro,
e per vostra quiete; eccovi quelli, che Automedonte
stima fortunati.
\literaryquote{Rei anil debet fortunatissimus ille est, \\
oximus huic celebs, tertius orbus erit.}

La sventura è dopo la perdita restar debitore,
questa si toglie col pagare.
}


\subsection{PRECETTO XII.}

Alla fine del giuoco se vinci, o perdi restati
sempre eguale, né dar segni, o di
troppa displicenza dell'uno, di troppa allegtezza
dell'altro avvenimento; e sopra
tutto ricordati dj essere economo dei denari
vinti, perché, non sempre 1a forte ti sarà
favorevole; onde riperdendo tu, non ti abbi
ad incomodare del tuo, avendo guadagnato
l'altrui, Dalle congratulazioni poi dei circostanti
te ne sbrigherai in quella guisa che suggerisce Dante Purg. c. 5.
\literaryquote{
Quando si parte il giuoco della Zara,\\
Colui, che perde, si riman dolente\\
Repetendo le volte: e tristo impara.

Con l'altro se ne va tutta la gente:\\
Qual va dinnanzi e qual diretro 'l prende,\\
E qual da lato gli si reca a mente:

Ei non s'arresta: Or questo e quello intende\\
A cui porte la mano, più non fa pressa\\
Et così della calca si difende.\\}

Sopra tutto se hai perduto non ti affligere
perché non v'è male di peggior nacumento
quanto l'ansietà dell'animo; onde Ovidio
parla per esperienza 2.\ de ponto.

\literaryquote{Unda focusque nocent et causa valentior istis\\
Anxietas onimi, quae mibi semper adest.}

Se avete perduto pensate, che non è già
felice quello, che ha molti denari, ma quello
bensì, che non si rattrista, come disse
Euripide. \textit{Non felix appellandus est, qui pecunias
habet plurimas, sed qui non tristatur.}
E Cicerone nel 3.\ delle Questioni Tusculane,
dice \textit{Summa stultitia est frustre confici dolore,
 cum intelligas nihil posse profici}, E se il giuoco
è la cagione in voi di rattristarvi lasciatelo
perché \textit{non debemus causas doloris accerescere}. Senec. ep. 10.


\section{CAPO TERZO.}

\subsection{Giuoco in quattro ad ogn'uno per se.}

Ora in disuso giuocare a Minchiate a ogn'uno per se; ma pur volendolo giuocare,
eccone i principali precetti.

Primo. Si giuoca in quattro, si ruba si
danno venti carte, e poi la scoperta; si
prendono le carte buone della fola da chi fa le
carte, si scarta ecc. come sopra.

Secondo. Ogn'uno segna i punti delle
rubate, e degli amazati, contro tutti e tre i
collusori, tre segnano contro chi ha perduto
la carta di conto. Sicché ogn'uno tiene, o
può tenere accese trè partite, le quali
giuocando crescono, o calano, secondo le
vicende del giuoco.

Terzo. In fine ogn'uno fa i conti con
ciascuno de' suoi compagni, e può
succedere, che con altri vinca, e con altri perda, ed
allora si girano i resti.

Quarto. Per giuocar bene; dovendosi
perdere qualche carta, si procuri darla a chi
non fa verzicola, v.g.\ se vedete, che
qualch'uno ha fatto l'uno, el 28, il vostro 13, che
non potete farvi, procurate darlo, quando
fa la levata, ad uno degli altri due.

Quinto. Somma attenzione per farsi le
sue carte per far caccia, per conteggiare
esatto, e spedito, e per simili occorrenze, ove
come in ogn'altra azione umana \textit{Usus Magister est optimus}. Cic.\ pro reb.

\subsubsection{nota allegorica.}
{\footnotesize
Il conversare è all'uomo necessario, e spesso
inevitabile; farà però molto bene, se a tutto suo
potere sfuggirà la troppa familiarità di chi che sia;
resterà in vero privo d'alcuni piaceri emolumenti
della societa, ma anderà esente da molti incommodi,
rancori, e passioni d'animo, che incontrano
coloro, che troppo cogli uomini si familiarizzano.
\literaryquote{Si vitare velis acerba quedam,
Et tristes animi carere morsus,
Nulli te facias nimis sodalem,
Gaudebis minus, \& minus dolebis}

E' ben vero, che l'uomo, che poco prattica altrui,
o che vive solitario \textit{aut Angelus st, aut bestia},
dice Arist. 1.\ polit. però dee guardarsi di non
divenire come un bruto intrattabile, e rozzo, ma essere
buono, e della legge ottimo emulo. Un giorno
s'incontrò Crate Filosofo in un Giovanetto, che andava
solo, e cogitabondo, gli dimandò, che cosa
egli andava facendo così solo, e pensieroso.
Rispose il Giovane, se parlare con se medesimo, allora
Crate ripigliò \textit{Fili cave rogo, \& diligenter
 attende, ne cum homine malo loquaris.}
}

Ed ecco quanto del Giuoco delle Minchiate
per mio trastullo aveva io raccolto, il che visto da
varj miei amici, vollero, che pubblicassi, al che,
benché di mala voglia, pur condescesi, sperando di
fare colle annesse allegorie qualche vantaggio alla
Gioventù studiosa, ed insieme avida di giuocare.

Concludo col ricordarvi quello, che fin
da principio vi dissi, che se avrete la carità,
avrete ancora ogn'altra perfezione, ora,
vi aggiungo la grandezza della nostra gloria
dipendere dalla grandezza della carità.
\literaryquote{Cui major charitas, debetur gloria major:
 Quantus amor fuerit, premia tanta feret.}

E di più ogni industria, tutto il talento, e
qualsisia fatica si dee adoprare nel giuoco
terribile di questa vita, per vincere, e per
ottenere il premio eterno. Verino.
\literaryquote{Quam potes eterno pro munere ferre laborem?
Mercedi an tantae, par labor esse potest?}







\supersection{GIUOCO DELL'OMBRE}

\section{CAPO QUARTO}

\subsection{Avvertenze sul Giuoco dell'Ombre.}

\lettrine{D}opo il giuoco degli Scacchi senza
dubbio il più bel giuoco inventato da un'acutissimo
spirito Spagnolo, è a giudizio
comune il Giuoco dell'Ombre, è vero,
che egli è meschiato d'azardo, e d'ingegno,
ma questo v'ha tal parte che assolutamente
chiunque ha sopra i conpagni questa
prerogativa, a giuoco lungo resterà sempre
vittorioso. Perciò confido di fare altrui cosa
gratissima, se qui darò in esso qualche avertenza
particolare, perché talvolta molto diletta
un giuoco sì bello.

\subsection{Sucinta Spiegazione \\del Giuoco
dell'Ombre.}

Il Giuoco dell'Ombre si fa in tre con 40
carte Spade, Bastoni, Danari, e Coppe.

\subsection{Valore delle Carte.}

Spadiglia è l'asso di spade la prima carta
del giuoco, che sempre trionfa.

Maniglia è, o il 2 di spade, o il 2 di bastoni,
o il 7 di denari, o il 7 di coppe, secondo quale
de' sudetti pali trionfano, ed è la seconda carta del giuoco.

Basto è l'asso di bastoni, terza carta del giuoco,
che sempre trionfa; onde Spade, e Bastoni
si chiamano pali curti; perché anno un
trionfo di meno, che Coppe, e danari. Queste
tre carte si chiamano Stuccio, e Mattatori.

Punto sono: Asso di denari, e di coppe,
quando il loro palo trionfa, sono la quarta
carta di Stuccio, e prendono tutte le figure
quando non sono trionfi, sono la migliore
dopo le figure nel loro palo.

Nella Cartiglia di coppe, e di danari le
migliori sono le più piccole.

I Stucci sono tanti, quante carte si anno
consecutive dopo le tre prime sudette. E
questi stucci sono onori in mano dell'Ombre.

Far tenazza vuol dire avere spadiglia, e
basto, o due altre carte equivalenti, ed essere
ultimo in giuoco.


\subsection{Modo di giuocare.}

Si danno cento puglie tra fisce, e gettoni
a ciascuno de' tre collusori, che vagliono
ad arbitrio, si sceglie a sorte chi dee far carte.

Chi fa le carte mette in piatto 4 puglie,
meschia, fa alzare, e da 9 carte a ciascuno.
a tre a tre, e non altrimente.

Il primo a man destra di chi fa le carte
considera il suo giuoco, e se vuol'essere
Ombre, ciò è se si vuole impegnare a fare più
levate d'ogn'uno de' suoi Compagni, può dire
la giuoco, ed allora giuoca senza scartare, e
gli altri scartano, e prendono per ordine dal
monte quante carte lor piace dopo, che l'Ombre
ha dichiarato il palo che trionfa. Se
alcuno degli altri due non vuole, che l'Ombre
la giuochi, gli dice, mas di mas, se l'Ombra
dice dichiara il palo, ed allora è obbligato a
fare tutte le levate; se non dice, chi fece
dire mas di mas è Ombra esso, ed è obbligato
a fare nove levate.

Se poi vuol giuocare scartando dice entro,
e se nessuno gli fa dire di più, ciò è se
non v'è chi voglia giuocare senza scartare,
dichiara prima il palo, che trionfa, e poi
scarta, e gli altri prendono le carte, che avanzano
con ordine, e scartano anch'essi; e poi
ambedue si difendono procurando, o di fare
tre levate per uno, o uno quattro, e l'altro
una.

Se poi non vuole né pure entrare può fare
o Cascarone portando due carte, o Cascariglio
portandone una sola, o novena non
portandone alcuna; ed allora dopo dichiara
il trionfo, e se fa cinque base, o pure
quattro, ed i compagni uno tre, e l'altro due,
tira tutto il piatto con gli onori, che ha da i
suoi compagni, se poi fa tre, o quattro base la
ripone, cioè ne mette in piatto quante ve ne
sono, e paga gli onori; se poi uno aversario
da Codiglio, cioà fa esso più levate degli altri
l'Ombre paga a quello tutto il piatto.

Se non vuol'essere Ombre dice passo, e
pone una puglia in piatto, così si dica degli
altri due; la seconda volta o si ripassa, o si fa
novena, o cascariglio, o cascarone.

\subsection{Gli Onori sono.}

Una puglia per ogni Stuccio; due per la
giuocata, due pel Cascarone una per il
Cascariglio, e una per la novena, quattro per il
totos si pagano all'Ombre se vince.

Quando l'Ombre non ha speranza fondata
di vincere, dica mi do, perché, se qualch'uno
non lo prende, facilmente con l'ajuto
dell'altro potranno essere in due a riporla, e
talvolta in tre, quando nessuno lo prende, e
fanno tre levate per uno.

\subsection{Leggi del Giuoco.}
\begin{list}
{\arabic{listombrecnt}.}
{\usecounter{listombrecnt}
\setlength{\labelwidth}{1.5pt}
\setlength{\labelsep}{1.0em}
\setlength{\itemsep}{-0.3em}
\setlength{\leftmargin}{0em}
\setlength{\itemindent}{2.5em} % equals \labelwidth+\labelsep
}
\item È uso, che chi ha Spadiglia faccia giuoco
e non può passare, che con due
puglie, una per il passo, l'altra per la spadiglia.
Questa maniera di giuocare si chiama
spadiglia a forza, e basto a vista.
\item Chi ha il basto può passare la prima
volta, poi la seconda bisogna, che faccia
giuoco, o che passi con puglia: se giuoca, e
se gli viene la Spadiglia la dee subito mostrare,
alttimenti ripone il piatto, se perde
due volte, e se vince non lo tira.
\item Chi sbaglia le carte non ha pena veruna, ma si rifanno di nuovo.
\item Chi ha carte di più o di meno di nove
ripone la puglia, se non lo dice prima di far
giuoco.
\item Se dando le carte si scopre un Mattatore,
 cioè una carta di Stuccio, si rifanno le carte.
\item Se l'Ombre non nomina il trionfo s'intende Spade.
\item Se l'Ombre si è dimenticato di nominare
 il palo, e se non ha confuso con l'altre
sue carte lo scarto, può rifarlo di nuovo; ha
tempo di nominare il palo, finché non si è
voltato le carte, che prende, in faccia.
\item Se l'ultimo a scartare lascia qualche
carta la può vedere subito, e dopo giuocata
la prima carta tutti la possono vedere. Se
qualcuno la vede prima di essersi giuocata la
prima carta, ripone il piatto. Se restano più
d'una carta, non si vedono, se non è consenso comune.
\item Chi scarta se prende più, o meno carte,
 se se ne accorge prima di giuocare, si può
emendare, e prendere dal monte quella,
che gli manca, o rimettere quella, che ha
preso di più.
\item Chi fa vedere una o più delle sue
carte in pregiudizio dell'Ombre ripone la
puglia.
\item Chi rifiuta ripone il piatto tante volte,
 quante volte rifiuta. Si può emendare finché
 non si è giuocata un'altra carta, ancorché
 la base sia coperta.
\item Se uno prende una base non sua copre,
 e rigiuoca ripone il piatto, e restituisce
 da base al patrone.
\item I tre mattatori non sono obbligati da
 i trionfi inferiori; la Spadiglia obbliga la
 maniglia, e 'l basto prima giuocata, ma se ella
 giuoca seconda non obbliga.
\item I Stucci sono onori in mano dell'Ombre
 non in mano d'altrui,
\item Gli onori si possono dimandare, , finché
 non è incominciato un'altro giuoco.
\item Chi fa sei base dee farle tutte, altri
 menti paga quattro gettoni a testa a i compagni;
 basta giuocare la sesta carta per obbligarsi
 a far totos.
\item Qualunque parola, che si dice a questo
 giuoco è irrevocabile, chi dice passo, non
 può far giuoco, e se ha la Spadiglia ripone;
 chi dice entro non può fare altro giuoco, che
 entrare ecc.
\item I contr'ombre possono chiedere gano,
 cioè il Cavallo d'un Re, possono dire vada
 con la maggiore, possono dir mia, cioè
 non copra, e questo sempre, che lor piace.
\item Chi ha una carta di più o di meno,
 e se ne accorge dopo, che ha detto di far
 giuoco ripone la puglia, e poi sta in sua arbitrio
 di far giuoco, o non farlo. Se fa giuoco,
 ripone la puglia, due volte se la perde, e se
 vince non tira niente.
\end{list}
Supponendo, che si sappiano tutte le regole,
e tutte le leggi di questo piacevolissimo
giuoco. Prima dirò quali siano quelle combinazioni
di carte, con le quali, abbenché minime,
chiunque entrerà, potrà con ragione
sperare la vittoria, che se non li succede non
incorrerà la taccia, o di temerità, o di
troppo azardoso. Secondo, vi esporrò alcuni giuochi
sicuri per altro, ma per essere mal guidati
si perdono, e saranno casi tanto più belli,
quanto che parrà, giuocandoli bene, che si
giuochino male, e giuocandoli male parrà
d'averli giuocati bene.

\subsection{Giuochi minori, co i quali\\ si entra a palo curto.}

\begin{listombre}
\item Maniglia, Basto, Re, cinque.
\item Spadiglia, Basto, Re, cinque.
\item Spad. Man. Re, quattro.
\item Spad. Man. sei, sette.
\item Spad. Basto, Dama, sette.
\item Spad. Re, Dama, Fante, sette.
\item Man. Basto, Dama, sei, cinque.
\item Man. Re, Dama, Fante, fette.
\item Basto, Re, Dama, Fante.
\item Man. Basto, sette, sei, cinque.
\item Re, Dama, Fante, sette, sei, cinque.
\item Spad. Re, Dama, quattro, tre.
\end{listombre}

\subsection{A palo lungo si entra entra con}

\begin{listombre}
\item Spad. Man. punto, quattro.
\item Basto, Man. punto, due.
\item Spad. Basto, punto, tre.
\item Spad. Man. Re, Dama.
\item Spad. Man. Fante, tre, un Re.
\item Spad. Basto, Re, Dama, Fante.
\item Basto, punto, Re, Dama, Fante.
\item Man. punto, Re, Dama, Fante.
\item Man. Basto, Re, Dama, tre.
\item Man. Basto, Fante, due, tre, cinque.
\item Spadiglia, Basto, Re, due, un Re.
\end{listombre}

L'entrare con minori giuochi è rischio imprudente,
è meglio, avendosi la spadiglia, a
fare cascariglio, o cascarone.

\subsection{Si giuocan solo a palo curto.}
\begin{listombre}
\item Quattro Mattatori, un'altra, e un faglio.
\item Spad. man. Re, Fante, due Re, e un faglio.
\item Spad. man. basto, e due Re.
\item Stuccio, sei, tre, quattro, un Re e un faglio
\item Man. Basto, Dama, Fante, quattro, tre, e un Re.
\item Man. basto, sei, cinque, quattro, un Re, e un faglio.
\item Basto, Re, Dama, Fante, sette, sei, cinque, quattro, un Re.
\item Re, Dama, Fante, sette, fei, cinque, quattro, un Re.
\item Spad. Re, Dama, sette, sei, quattro, un Re, e un faglio.
\item Man. Re, Damas Fante, sette, quattro, Re, e due fagli.
\end{listombre}


\subsection{Si giuocano a palo lungo.}
\begin{listombre}
\item Stuccio, tre, quattro, cinque, un Re.
\item Basto, Re, fette, tre, un Re, due fagli.
\item Spad. man. punto, sei, tre, un Re, e due
Dame guardate.
\item Man, Basto, Dama, Fante, due, sei, un
Re, e due Dame guardate.
\item Basto, punto, Re, sei, tre, due, un Re, e
due Dame guardate.
\item Punto, Re, Dama, Fante, due, tre,
sei, un Re.
\item Man. bast. punto, tre, quattro, cinque,
un Re, e due fagli.
\item Spad.
punto, Re, Dama, due, sei, un
Re, e Fante guardato.
\end{listombre}

 Chi è primo, o ultimo a giuocare, può
prendersi qualche arbitrio di più.

\subsection{
Casi ne i quali si perde \\per giuocar bene
in apparenza, \\e la maniera di giuocarli.}

\subsection{CASO PRIMO.}

L'Ombre di mano ha maniglia, Re, Dama,
Fante, sei a spade trionfo. Re di coppe,
Dama, e tre di denari, Re di bastoni.

Il secondo ha sette, cinque, quattro. Dama; e
fante di bastoni; Re, e Fante di denari, sei di coppe.

Il terzo ha basto, e tre di trionfo, Fante,
quattro, due, tre di coppe, asso, cinque,
quattro di denari.

L'Ombre perderà questo giuoco, se incomincierà
a giuocare col Re di bastoni: l'ultimo
lo prende, e giuoca denari, il secondo
fa il suo Re, e poi il Fante di denari, poi
giuoco bastoni, fa mettere il basto, se l'ombre
lascia sperde codiglio, se prende, è riposta,
Dunque per vincere si giuochi così.

L'Ombre giuochi il Re di trionfo. Il secondo
prende colla spadiglia, il terzo dà il
tre.

Il secondo giuoca bastoni per la meglio,
il terzo mette il basto, e l'Ombra dail Re.

Il terzo giuoca denari, Ombre mette la
Dama; e il secondo prende col Re; poi fa il
Fante ancora, e poi tutto fa l'Ombre.

Se il terzo giuoca in vece di denari coppe,
Ombre prende col Re, e poi giuoca
tutti i suoi trionfi.

\subsection{CASO SECONDO.}

L'Ombre secondo ha maniglia, basto, Re,
Dama, cinque, tre di bastoni trionfo,
Dama guardata di spade, e Fante di coppe.

Il primo a giuocare ha sei di bastoni,
Re, e sei di coppe, Re, Fante, asso, e sei
di denari, sei, sette di spade.

Il terzo ha spad. Fante, sette, quattro di
trionfo asso, due, quattro di coppe, Re,
e Fante di spade.

L'Ombre perde il giuoco, perché amazza
col cinque il Re di denari, che giuoca il
primo, e il terzo glielo prende col sette, e
poi giuoca coppe, che resta del primo, e poi
giuoca spade, e il terzo ne fa due altre, e la
spadiglia fa la quarta, onde è riposta.

Per vincerlo basta, che lasci col Fante di
coppe il Re di denari, oppure l'assicuri con
un mattatore.

\subsection{CASO TERZO.}
L'Ombre terzo a giuocare ha maniglia,
basto, Re, Dama, tre, sei, quattro a
coppe trionfo, due Fanti di bastoni, e di denari.

Il primo ha punto di coppe, Re, Fante,
sette, tre di spade, tre, quattro, cinque, sei
di bastoni.

Il secondo spade, Fante, due e cinque
di coppe, Re di bastoni, Re, Dama, sei,
cinque di denari.

L'Ombre perde il giuoco così. Il primo
giuoca bastoni, il secondo pone il Re, e lo fa,
poi giuoca il Re di denari, l'Ombre mette il
Fante, e il primo da bastoni, il secondo poi
giuoca denari, l'Ombre mette il Re, el primo
il punto, se poi giuoca bastoni, il secondo
mette il Fante di trionfo, l'Ombre prende
con la Dama, e giuoca il basto, il primo da
bastoni, e il secondo da il cinque trionfo,
l'Ombre giuochi che che sia, ha riposto.

Si vincerà così. Se in vece. di mettere il
Re sul Re di danari, l'Ombre lo coprirà con
un trionfo basso.

\subsection{CASO QUARTO.}
L'Ombre primo a giuocare spadiglia, maniglia,
sei, cinque, quattro, tre di spade
di trionfo, ed una cartiglia per palo.

Il secondo ha il basto, e il sette di trionfo,
Re, e sei di bastoni; asso, e cinque di
coppe, Fante, tre, sei di denari.

Il terzo ha Re, Dama, Fante di trionfo,
Re, e sette di denari, Dama di bastoni, Re,
Fante, e due di coppe.

L'ombre perde il giuoco così. Giuoca la
spadiglia, il secondoda il ette, e il terzo il
Fante, e poi giuoca a bastoni, il terzo prende
col cavallo, poi giuoca il Re di coppe, e lo
fa; poi fa il Re di denari, poi giuoca coppe,
l'Ombre mette trionfo, e il secondo prende
col basto e il terzo pur fa la sua quarta con
Re, e Dama.

L'unico modo di vincere questo giuoco è
che Ombre giuochi subito spadiglia, e maniglia.
\subsection{CASO QUINTO.}

Il primo a giuocare ha maniglia, basto,
Re, cinque di denari trionfo, Re, Dama
di bastoni, Dama, Fante, tre di spade.

L'Ombre secondo spadi. punto, Dama,
due, tre, quattro, sei di trionfo, Re di
coppe, e Fante di Bastoni.

Il terzo Fante di trionfo, Re, sette, sei,
cinque di spade, sei di bastoni, Cavallo,
tre, cinque di coppe.

L'Ombre perde così. Il primo giuoca il
Re di bastoni, e lo fa, poi giuoca il Cavallo,
l'Ombre mette il due trionfo, e il terzo
prende col Fante, e giuoca coppe, il primo
mette cinque trionfo, e lo fa, poi giuoca,
come vuole, sempre farà due altre levate.

L'Ombre vincerà, se passerà la Dama sul
Cavallo. Insomma quando l'Ombra ha molti
trionfi, la meglio è giuocare le grosse.

\subsection{CASO SESTO.}

L'Ombre terzo a giuocare ha spad. basto,
Cavallo, sei, cinque di bastoni trionfo,
Re, tre, sette di danari, Re di coppe.

Il primo maniglia, Re, Fante, quattro di
trionfo, Fante; asso, sette di denari, due di
spade, e due di coppe.

Il secondo tre sette di trionfo, Fante,
uno, tre, sei di coppe, Re, tre, sei di spade.

L'ombre perde il giuoco così: il primo
giuoca il due di coppe, l'ombre prende col
Re, e giuoca il Re di denari, il secondo
prende col sette di bastoni, e giuoca tre di
trionfo. Eccola riposta.

L'Ombre vincerà, se in luogo di giuocare
il Re di denari, giuocherà il sei di trionfo,
il primo prenderà col Fante, e giuoca denari,
il secondo prende col sette, e l'Ombre da
sette di denari, il secondo giuoca o spade, o
coppe, e giuochi qualunque, sempre l'Ombre
vincerà, perché su le spade metterà il cinque
trionfo, e su la coppe metterà denari.

Tra i casi possibili, che possono succedere
in questo giuoco che sono 263,547.520 prescindendo
da i franchi, sonovene tra i dubbj
curiosissimi, e chi non ha memoria de i scarti,
de i fagli, e della previsione al caso possibile,
non s'azzardi contro chi in questo lo supera,
perché perderà.

I sudetti sei casi ben considerati molto
gioveranno a dar lume agli altri simili, che ve
ne sono moltissimi. Che è quanto di questo
nobilissimo Giuoco mi è parso bastevole indicare
essendovi di esso molti, e gravissimi
Scrittori, che diffusamente lo anno spiegato,
ed io d'alcun d'essi mi sono prevaluto per le
cose sudette.


\supersection{GIUOCO DEGLI SCACCHI.}
\section{CAPO QUINTO.}

\lettrine{I}L più bel giuoco, che sia stato trovato al
Mondo senza veruna controversia è il
Giuoco degli Scacchi; 1a di cui antichità è
si grande, che coetaneo lo vogliono a Palamede
sotto Troja, o a Diomede sotto Alessandro
il Grande, o agli antichi Arabi, o Persiani:
ma di ciò poco importa: certo è, che
ha dilettato sempre fuor di modo coloro, che
l'anno imparato, perché egli impegna, e picca
sollennemente i giuocatori, essendo tutto
pieno d'ingegno, e gli amici tra loro si donano
ogni sorta di cose rare, e preziose, ma
trattandosi d'ingegno poche volte, o non mai
si vedranno l'uno cedere all'altro.
\literaryquote{
 Aurum, et opes, et rara frequens donabis amicus,\\
 Qui velit ingenio cedere rarus erit.}

Avertì Marzial.\ l,\ 8. Perciò si vedono questi
tra loro costantemente combattere le ore, e
i giorni intieri con applicazione incredibile
per riscuotere la gloria, o il piacere d'aver
superato il compagno in quel feroce combattimento
d'ingegno; dico però, che non si
dee alcuno tanto lasciare trasportare dall'impegno
di questo giuoco, che poi l'abbia da
perdere troppo tempo, perché \textit{non ita generati
et natura sumus, ut ad ludum fatti esse
videamur}; disse Cic.\ 1.\ de officiis: dico però
che quando ci troviamo scarichi di negozj, di
studj, di applicazioni, e quando compiuto abbiamo
ogni obbligo del nostro stato siccome il
natural talento ci porta ad indagare cose
belle, nuove, e mirabili, come conobbe anche
Seneca l.\ de moribus. \textit{Cum sumus necessariis
negotiis curisque vacui, tunc avemus aliquid
videre, audire, dicere, cognitionemque
rerum aut admirabilium ad bene beateque vivendum
necessariam ducimus}. Cosi possiamo
dare qualche ora a questo Giuoco. Certamente
novità più ingegnosa, e più innocente
fra tutti i trastulli del Mondo del Giuoco degli
Scacchi vi farà difficile il ritrovarla, al quale
non manca né pure il mirabile d'una sottilissima
invenzione; ed in esso non si giuoca d'altro
interesse, che della gloria di vincere.

\subsection{Notizie per giuocare a Scacchi.}

E qualch'uno siavi, che non abbia veruna
contezza del Giuoco degli Scacchi,
premetto una succinta, ma sufficiente
spiegazione di esso giuoco.

\subsection{Tavoliere, e sito de i Pezzi.}

Il Giuoco de i Scacchi si fa in due in una
tavola quadrata, che dicesi Scacchiera distinta
in sessanta quattro caselle alternanti di
color bianco e nero con sedici pezzi bianchi,
e con sedici neri di sei specie, che stanno nelle
due prime file opposte; e sono

Due Rocchi, o Torri nelle case di cantone.

Due Cavalli nelle case seguenti.

Due Alfieri seguenti.

Re. Pezzo principale, in ordine a cui sono tutti gli altri.

Regina bianca in casa bianca, e Regina
nera in casa nera, si chiama ancora Dama, e
Donna.

Otto pedine stanno avanti alli sudetti
pezzi.

\subsection{Moto di essi pezzi.}

Il Re può andare, e prendere in qualunque
prossima casa d'intorno alla sua, che
attualmente occupa, una sol volta può saltare
nella fila de' suoi pezzi col Rocco, purché
non si sia mai mosso, e sia libero il passo da
Scacco, e da altro suo pezzo, nell'atto, che
ha scacco, né pure può far detto salto, e
facendolo, non può offendere l'inimico, né radoppiare
offesa.

La Regina va, e prende come il Re ma
per tutte le linee dritte, e traverse, ed oblique
da capo a fondo al Tavoliere dovunque
vuolsi, purché non vi sia pezzo, che impedisca la strada.

Gli Alfieri vanno per obliquo per le loro
linee, o bianche, o nere da capo a fondo.

Le Torri, o siano i Rocchi vanno per
dritto, e per traverso pure da capo a fondo
alla scacchiera e da un lato all'altro, se il passo
sia libero e possono prendere se vogliono, l'inimico
che s'incontra essere per queste strade.

Cavalli saltano, e prendono di bianco
in nero, e di nero in bianco tre case per tre
case nella fila contigua d'ogni intorno alla
loro, che attualmente occupano.

Le Pedine vanno avanti per dritto, e pigliano
per traverso. La prima loro mossa
può essere, se si vuole, di due case, e poi
s'avanzano casa per casa, né mai tornano in
dietro; arrivata all'ultima casa qualunque di
esse diventa o Regina o qualunque altro
pezzo si voglia.

Quando un pezzo piglia occupa la casa
del preso.

Le Case sono denominate dal pezzo, che
li sta in fila v. g. prima del Rocco, e quella
che ha, seconda è quella, che li sta avanti,
terza, e quarta le seguenti ecc.

Quando s'offende il Re con qualche pezzo
s'avvisa con dire Scacco è poi Scacco
matto, quando il Re non ha luogo da salvarsi, o
pezzo da coprirsi, o pezzo, che prenda il
pezzo offensore. Chi dà questo Scacco matto
porta la vittoria del giuoco.

Lo Scacchiere sia con casa del rocco bianca
alla destra delli giuocatori.

Pezzo toccato pezzo giuocato.

Casa toccata casa occupata.

Re non può saltare, o come si dice arroccare.
1. Se si è mosso. 2. In atto di Scacco.
3. Se passa per lo Scacco. 4. Se offende, o se
radoppia offesa.

Il Giuoco si fa patta in due maniere. Prima,
perché si resti con pezzi uguali, o con pezzi
ineguali; ma che non possino dare Scacco
matto, Come il Re; o col Cavallo, o
coll'Alfiere, o con due Cavalli. O con Rocco, e
Cavallo, contro un Rocco. O con Cavallo,
con'Alfiere contro un Rocco. O con Pedina,
e Dama contro Rocco.

Seconda, quando il Re è serrato senza
essere offeso, e non possa moversi senza essere
preso né possa movere altro pezzo.

Re, e Rocco: contro Re, e Regina impatta.

Se si dia un giuoco così. Il Nero abbia il
Re alla casa del Cavallo, pedina avanti, e
pedina alla terza di rocco, e rocco alla terza
d'Alfiere. El bianco abbia Re, e Regina dove
vuole. Il nero impatta se non moverà altro
che il Re, altrimenti perde senza riparo.

B. Re in casa di suo Cav. Rocco alla seconda di detto Cav.

N. Re alla quinta del suo roc. Dama alla
terza dell'Alf. di Re bianco. Se il Bianco
giuoca prima è patta.

B. Dà Scacco col roc.\ alla sua seconda.
N. Re alla terza di cav.

B. Rocco alla sua terza dà Scacco. N. Piglia.
B. Re serrato.

Pedina di rocco. E l'Alf. che non ferisca
la casa ultima di detto rocco, se il Re contr.
occupa dette case è patta.

Una ped.\ che non sia de' rocchi col suo
Re a lato. E'l Re contr. stia avanti, è patta,
se si opponga a casa di mezzo, a più d'una
non si opponga.

Ped.\ alla penult. di rocco, Re all'ultima,
e Cav.\ nella sua casa. Re contr. in casa
dell'Alf. è patta, purché non prenda mai il Cav.

Re alla casa del Cav. Rocco alla seconda
di cav.\ di Re nero.

N. Una ped.\ alla seconda di cav.\ di Re
bianco. Pad.\ a quarta di cav.\ di Re bianco.
Re alla terza di rocco di Re bianco. Rocco
alla casa di Re.

Il bianco impatta cosi. Dà Scacco col rocco
in sua seconda casa, e poi sempre si metta
in presa del rocco contrario.

Eccovi alcuni bellissimi, ed ingegniosissimi
colpi cavati per lo più da varj Autori, che
di questo giuoco anno fatto libri laboriosissimi.

Per intendere, e per gustare veramente
la bellezza di questi Tratti, bisogna disporre
i giuochi come si propongono, e poi provare
da se, se mai riuscisse fare quello, che si
prescrive, perché, o vi arrivate, ed avrete il
piacere del vostro ritrovato, o no, e vi sarà
gratissimo avere imparato una cosa da voi
disperata.


\section{CAPO SESTO.}

\subsection{Tratti curiosissimi.}

\subsubsection{TRATTO I.}

{\small
\noindent
\begin{tabular}{@{}p{3.84cm}p{3.84cm}}
{\bfseries\scshape Bianco}.& {\bfseries\scshape Nero}.\\
Re alla 4 del re contr.
Il rocco del re alla 4 dell'alf.\ del re contrario.
Il rocco della dama alla
4 della dama contraria.& Re solo alla sua casa seconda.\\
\end{tabular}}

Il bianco dee dar matto al nero in tre colpi a condizione di movere tutti i pezzi.

\vspace{6pt}{\small
\noindent
\begin{tabular}{@{}p{3.84cm}p{3.84cm}}
Il rocco del re alla casa dell'alf. di dama contr. dà scacco. & Il re alla sua casa.\\
Il re alla terza dell'alf.\ del re contrario & Il re alla casa del suo alfiere.\\
Il rocco della dama dà matto alla casa della dama contraria.&
\end{tabular}}

\subsubsection{TRATTO II.}
{\small
\noindent
\begin{tabular}{@{}p{3.84cm}p{3.84cm}}
{\bfseries\scshape Bianco}. & {\bfseries\scshape Nero}.\\
Re alla 2.\ casa del suo alfiere.&Re alla seconda del rocco del re contrario.\\
Cavallo alla casa del cavallo del re contr.& La pedina del roc.\ del re alla 3.\ del rocco del re contrario.\\
& La ped.\ del cav.\ del re alla 3.\ del cav.\ del suo re.
\end{tabular}}

Il bianco giuoca prima, e dà matto in 4.\ colpi.

\vspace{6pt}{\small
\noindent
\begin{tabular}{@{}p{3.84cm}p{3.84cm}}
Il cavallo del re alla terza dell'alfiere del re, contrario. & La ped.\ del cav.\ del re una casa alla quarta del cavallo del suo re.\\
Il caval. dà scacco alla 4.\ del cav.\ del suo re. & Il re alla casa del rocco del re contrario.\\
Il re alla casa del suo alfiere. & La ped.\ del rocco del re una casa alla seconda del rocco del re contr.\\
Il cav.\ dà scacco matto alla 2.\ dell'alf.\ del suo re.
\end{tabular}}

\subsubsection{TRATTO III.}

{\small
\noindent
\begin{tabular}{@{}p{3.84cm}p{3.84cm}}
{\bfseries\scshape Bianco}. & {\bfseries\scshape Nero}\\
Il rocco della dama alla quarta della sua dama.&Il re in sua quarta casa.\\
Il rocco del re alla quarta dell'alfiere del re.\\
Il cav.\ della dama alla 2.\ della dama contraria. \\
Il cavallo del re alla terza dell'alf.\ del re contr.
\end{tabular}}

Il bianco darà matto in quattro colpi.

\vspace{6pt}{\small
\noindent
\begin{tabular}{@{}p{3.84cm}p{3.84cm}}
Il cavallo della dama dà scaccho alla 2.\ dell'alf del re contrario.& Il re alla sua terza casa.\\
Il cav.\ del re alla 2.\ del rocco del re contrario.& Il re alla sua seconda casa.\\
Il cav.\ della dama alla 3.\ del rocco del re contr.& Il re dove vuole.\\
Il rocco del re dà scacco matto alla quarta del suo re.
\end{tabular}}

\subsubsection{TRATTO IV.}

{\small
\noindent
\begin{tabular}{@{}p{3.84cm}p{3.84cm}}
{\bfseries\scshape Bianco}. & {\bfseries\scshape Nero}.\\
Il re alla quarta del re contrario. & Il re alla casa del cavallo del re contrario.\\
La dama alla sua seconda casa & La pedina dell'alf.\ del re alla seconda dell'alf.\ del re contrario.\\
\end{tabular}}

Il bianco giuocherà, come segue, altrimenti farà
patta per forza.

\vspace{6pt}{\small
\noindent
\begin{tabular}{@{}p{3.84cm}p{3.84cm}}
Il re alla 4 del suo alf. &La ped.\ fa dam.\ novella.\\
Il re alla terza del suo cavallo, e darà matto nel colpo seguente.
\end{tabular}}

\subsubsection{TRATTO V.}

{\small
\noindent
\begin{tabular}{@{}p{3.84cm}p{3.84cm}}
{\bfseries\scshape Bianco}.& {\bfseries\scshape Nero}.\\
Il re alla casa dell'alfiere del re nero.& Re alla sua terza casa.\\
Il rocco di dama alla casa della sua dama.& La pedina del re alla 4.del suo re.\\
Il rocco del're alla casa dell'alfiere del suo re& La dama alla 4 del rocco della dama bianca.\\
Il cavallo del re alla terza dell'alf.\ del suo re.& Il rocco di dama alla seconda dell alfiere del suo re.\\
\end{tabular}

\noindent\begin{tabular}{@{}p{3.84cm}p{3.84cm}}
La pedina del re alla quarta del suo re.& Il cav.\ del re alla terza del rocco del suo re.\\
& Il rocco del re alla terza del cavallo del suo re.\\
\end{tabular}

\noindent\begin{tabular}{@{}p{3.84cm}p{3.84cm}}
& La ped.\ del rocco della dam.\ alla 3.\ del suo roc.\\
& La pedina del cavallo di dama alla sua terza.
\end{tabular}}

Il bianco dà scacco matto in tre colpi.

\vspace{6pt}{\small
\noindent
\begin{tabular}{@{}p{3.84cm}p{3.84cm}}
 Cav.\ del re dà scacco alla 4.\ del cav.\ del re negro. & Rocco prende cavallo.\\
 Rocco del re dà scacco alla terza dell'alf.\ del re negro & Re prende rocco.\\
 Roc.\ della dama dà scacco matto alla terza di dama nera.
\end{tabular}
}


\subsubsection{TRATTO VI.}

{\small
\noindent
\begin{tabular}{@{}p{3.84cm}p{3.84cm}}
{\bfseries\scshape Nero}. &{\bfseries\scshape Bianco}.\\
Re alla 2.\ di rocco suo. &Re nella casa del cavallo della sua dama.\\
\end{tabular}

\noindent\begin{tabular}{@{}p{3.84cm}p{3.84cm}}
Pedina del rocco alla 3.\ del roc.\ di dam.\ bianca.&Pedina alla seconda di re contrario. \\
Ped.\ del cav.\ alla 3.\ del cav.\ di dama bianca.&\\
\end{tabular}

\noindent\begin{tabular}{@{}p{3.84cm}p{3.84cm}}
Rocco in casa dell'alfiere di sua dama. &\\
Dama alla 3.\ del suo cav.
\end{tabular}}

Il nero dopo 4.\ mosse senza prendere la pedina dà
 scacco prima con una, e poi matto con l'altra.

\vspace{6pt}{\small
\noindent
\begin{tabular}{@{}p{3.84cm}p{3.84cm}}
Dama alla terza del cavallo di re dà scacco. &Re alla casa del rocco.\\
Dama alla casa di suo re. &Re alla casa del cavallo.\\
\end{tabular}

\noindent\begin{tabular}{@{}p{3.84cm}p{3.84cm}}
Rocco alla seconda dell'alfiere del re contrario. &Re alla casa del rocco.\\
Dama alla quarta di rocco di dama contraria. &Re torna alla casa del cavallo\\
& Se fa dama.
\end{tabular}}

Nero dà colle pedine i due scacchi consecutivi, e se
non fa dama, parimenti.

\subsubsection{TRATTO VII.}

{\small
\noindent\begin{tabular}{@{}p{3.84cm}p{3.84cm}}
{\bfseries\scshape Bianco}. & {\bfseries\scshape Nero}. \\
Re alla 2.\ di Re nero. & Re in casa del suo Rocco.\\
\end{tabular}

\noindent\begin{tabular}{@{}p{3.84cm}p{3.84cm}}
Cav.\ alla 3.\ di suo re.\\
Ped.\ alla 4.\ d'alf. di re.\ nero.\\
\end{tabular}

\noindent\begin{tabular}{@{}p{3.84cm}p{3.84cm}}
Cavallo alla 4.\ di cav.\ detto.\\
Alf.\ in casa di cav.\ di sua dama.\\
Ped.\ alla 3.\ di rocc.\ di re n.
\end{tabular}
}

Il bianco darà Scacco prima con una pedina, e poi
matto con l'altra in cinque Tratti.

\vspace{6pt}{\small
\noindent\begin{tabular}{@{}p{3.84cm}p{3.84cm}}
Cav.\ alla 4.\ di alf.\ di sua dama; poi alla 4.\ di re
nero; poi alla 2.\ d'alf.\
di re nero; poi Scacco con
una ped.\ e matto con
l'altra. & Il re vanellacasa di cavallo, e di rocco.\\
\end{tabular}
}

\subsubsection{TRATTO VIII.}

{\small
\noindent\begin{tabular}{@{}p{3.84cm}p{3.84cm}}
{\bfseries\scshape Bianco}. & {\bfseries\scshape Nero}. \\
Re alla casa di rocco suo. & Re in sua casa.\\
\end{tabular}

\noindent\begin{tabular}{@{}p{3.84cm}p{3.84cm}}
Pedine di cav., e di roc.\ alle loro case, ed altri
pezzi, quanti si vogliono,
purche sia libera la
fila di rocco di re bianco.
&
Alfiere di re alla quarta d'alfiere di dama.
Pedina di detto alla 5.
Dama alla terza di rocco di suo re.
Rocco alla terza dell'alfiere del suo re.\\
\end{tabular}
}

Il Nero dà scacco matto in due mosse.

\vspace{6pt}{\small
\noindent\begin{tabular}{@{}p{3.84cm}p{3.84cm}}
Prende la ped.\ del rocco con la dama. & Prende la dama col re.\\
Dà scacco matto col rocco alla terza sua.
\end{tabular}}



\subsubsection{TRATTO IX.}


{\small
\noindent\begin{tabular}{@{}p{3.84cm}p{3.84cm}}
{\bfseries\scshape Bianco}. & {\bfseries\scshape Nero}. \\
Re alla 2.\ del re nero; Ped.\ alla terza del rocco del re nero. &Re nella casa del suo rocco.\\
\end{tabular}

\noindent\begin{tabular}{@{}p{3.84cm}p{3.84cm}}
Pedina alla 5.\ dell'alfiere del suo re. &\\
Cavallo alla sua quinta.&\\
\end{tabular}

\noindent\begin{tabular}{@{}p{3.84cm}p{3.84cm}}
Cav.\ alla 3: del suo re.&\\
Alfiere del re alla quinta del rocco del suo re.&
\end{tabular}}


Il Bianco darà scacco con una pedina, e matto coll'altra in cinque colpi,

\vspace{6pt}{\small
\noindent\begin{tabular}{@{}p{3.84cm}p{3.84cm}}
Cav.\ della 3.\ del re alla 4.\ del Cav.\ del re.&Re in casa del suo cav. \\
Det.\ cav.\ alla 5 d'esso re.&Re tornain casa del roc. \\
Detto alla 2.\ dell'alfiere del re nero dà scacco.&Re in casa del cav.\\
Scacco con una pedina e poi matto coll'altra.&
\end{tabular}
}


\subsubsection{TRATTO X.}

{\small
\noindent\begin{tabular}{@{}p{3.84cm}p{3.84cm}}
{\bfseries\scshape Nero.} & {\bfseries\scshape Bianco.} \\
Re alla terza del re bianco con quatro pedine
intorno alle quattro case bianche, e due cavalli, uno alla terza di rocco di dama, l'altro alla quarta di cav.\ di dama. &

Re in sua casa, con suo
rocco in sua casa, e l'altro rocco in casa d'alfiere di sua dama. Alfiere
in casadirocco di dam. nera\\
\end{tabular}
}


Il Nero dà matto con pedina, che sta alla quatta di
re bianco in cinque colpi.

\vspace{6pt}{\small
\noindent\begin{tabular}{@{}p{3.84cm}p{3.84cm}}
Cav.\ che sta a 3.\ di rocco dà scacco alla 2.\ d'alf. & Piglia col rocco.\\
Avanza la pedina alla. seconda di dama. & La prende.\\
\end{tabular}

\noindent\begin{tabular}{@{}p{3.84cm}p{3.84cm}}
Scacco col cav alla terza di dama. & Rocco lo prende.\\
Prende con la pedina, con cui dà matto il colpo seguente.
\end{tabular}
}

\subsubsection{TRATTO XI.}

{\small
\noindent\begin{tabular}{@{}p{3.84cm}p{3.84cm}}
{\bfseries\scshape Bianco}. & {\bfseries\scshape Nero}.\\
Pedina alla 4.\ di rocco di dama contr.& Re in casa di rocco di sua dama.\\
Re alla 4.\ di cav.\ detto.& Pedina di cavallo in sua casa.\\
Dama alla 4.\ d'alf come sopra.&
\end{tabular}
}

Il bianco dà scacco matto di pedina in tre tratti.

\vspace{6pt}{\small
\noindent\begin{tabular}{@{}p{3.84cm}p{3.84cm}}
Dama in casa dell'alfiere del re contr. dà scacco.& Re alla 2. del rocco.\\
Dama alla casa dell'alfiere della dam.\ contr.& Pedina un passo.\\
Piglia con pedina, e dà scacco matto.&
\end{tabular}
}

\subsubsection{TRATTO XII.}

{\small
 \noindent\begin{tabular}{@{}p{3.84cm}p{3.84cm}}
{\bfseries\scshape Bianco}. & {\bfseries\scshape Nero}. \\
Re in casa del suo rocco,
e due pedine di rocco,
e di cav.\ alle loro case. & Re in sua casa; Dama alla seconda di suo rocco.\\
Rocco di dama in sua
casa. Altro rocco in casa
dell'alfiere. & Cav.\ alla quarta di cav.\ di re bianco.\\
\end{tabular}
}

Il Nero dà scacco matto in tre, ovvero in quattro mosse.



\vspace{6pt}{\small
 \noindent\begin{tabular}{@{}p{3.84cm}p{3.84cm}}
 &Cavallo dà scacco alla seconda d'alfiere.\\
Re alla casa di cavallo. &Cav.\ alla 3.\ di rocco scacco doppio di due pezzi in presa.\\
Se il re va in casa del suo alfiere. & Dà scacco matto colla, dam.\ alla 2. di d. alf.
 \end{tabular}
}

Se poi il Bianco andiede in casa del rocco. Il Nero
li dà scacco colla dama alla casa del cav. Il
Bianco piglia col'rocco, e il Nero dà scacco matto col
cav.\ alla seconda dell'alfiere.

\subsubsection{TRATTO XIII.}

{\small
 \noindent\begin{tabular}{@{}p{3.84cm}p{3.84cm}}
 {\bfseries\scshape Nero}. & {\bfseries\scshape Bianco}.\\
Rocco del re alla terza del suo cav. & Re alla casa del rocco suo.\\
Dama alla 4.\ del suo re. & Dama alla seconda del cav.\ della dama contr.\\
\end{tabular}

\noindent\begin{tabular}{@{}p{3.84cm}p{3.84cm}}
Alf.\ del re alla quarta
dell'alf.\ del re suo. \\
Rocco della dama alla
sua quarta. \\
Re alla quarta dell'alf.\ del re contr.\\
Pedina alla terza dell'alf.\ del re contr.\\
Ped.\ alla terza del rocco del re contr.\\
 \end{tabular}
}

Nero in sei mosse darà scacco con una pedina, e
poi matto con l'altra. Questo è di molto ingegno.

\vspace{6pt}{\small
 \noindent\begin{tabular}{@{}p{3.84cm}p{3.84cm}}
Dama alla casa del re, contr. dà scacco. & Il re va alla seconda del rocco\\
Rocco alla seconda del rocco contr. dà scacco. & Si copre colla dama. \\
Dà scacco coll'altro roc. & Lo prende colla dama.\\
Dà scacco colla dama alla terza del cav.\ della
dama contr. &
Il re vada dove vuole,
avrà prima scacco con
una pedina, e poi matto con l'altra.
 \end{tabular}
}

\subsubsection{TRATTO XIV.}

\hfill\textit{\footnotesize Bellissimo.}

{\small
 \noindent\begin{tabular}{@{}p{3.84cm}p{3.84cm}}
 {\bfseries\scshape Nero}. & {\bfseries\scshape Bianco}.\\
Re alla seconda del rocco della dama contr.& Re in sua casa\\
\end{tabular}

\noindent\begin{tabular}{@{}p{3.84cm}p{3.84cm}}
Dama alla seconda del cav.\ del re contr.& Pedina del cav.\ del re alla terza.\\
\end{tabular}

\noindent\begin{tabular}{@{}p{3.84cm}p{3.84cm}}
Cav.\ alla quarta del re contr.& Pedina del rocco del re alla quarta.\\
\end{tabular}

\noindent\begin{tabular}{@{}p{3.84cm}p{3.84cm}}
Cav.\ alla quarta della s. sua dama. &\\
Rocco del re alla sua 2. &\\
\end{tabular}

\noindent\begin{tabular}{@{}p{3.84cm}p{3.84cm}}
Quattro pedine quella del cav.\ della dam alla 3.\ del cav.\ della dama contraria.\\
\end{tabular}

\noindent\begin{tabular}{@{}p{3.84cm}p{3.84cm}}
Quella della dama alla 3.\ della dama,
contr. quella dell'alf.\ del re
alla 3.\ dell'alf.\ del re contrario e
quella.\ del rocco del re alla
terza del suo rocco.

 \end{tabular}
}

Il Nero in tredici mosse dà scacco colle quattro pedine consecutivamente, e matto con quella del rocco. Partito di grand'ingegno.

\vspace{6pt}{\small
 \noindent\begin{tabular}{@{}p{3.84cm}p{3.84cm}}
Dama dà scacco alla 2. dell'alf.\ del re contr. & Re alla casa della sua dama.\\
Dama dà scacco alla seconda del re contr. &Re va alla casa dell'alfiere della sua dama.\\
\end{tabular}

\noindent\begin{tabular}{@{}p{3.84cm}p{3.84cm}}
Ped.\ del rocco va avanti. &Va avanti con la pedina.\\
La prende con la sua. &Avanza la sua ped.\\
\end{tabular}

\noindent\begin{tabular}{@{}p{3.84cm}p{3.84cm}}
Avanza la sua. &Avanza la sua.\\
Rocco va alla seconda del suo alf. &Avanza la pedina.\\
\end{tabular}

\noindent\begin{tabular}{@{}p{3.84cm}p{3.84cm}}
Cav.\ della 4.\ della dama va alla quarta dell'alf.\ del re contrario. &Fa dama.\\
Dà scacco col rocco alla seconda dell'alf.\ della dama. &Sicopre colla dama alla terza del suo alfiere.\\
\end{tabular}

\noindent\begin{tabular}{@{}p{3.84cm}p{3.84cm}}
Dà scacco con la dama alla 2.\ dell'alf.\ tra la dama ed il Re. &La prende con la sua, e dà scacco.\\
Si copre, e dà scac.\ con la pedina, Poi prende la dama, e dà scac.\ coll'altra, poi dà scac. col1a 3.\ e poi matto coll'altra. \\
 \end{tabular}
}


\subsubsection{TRATTO XV.}
{\small
 \noindent\begin{tabular}{@{}p{3.84cm}p{3.84cm}}
{\bfseries\scshape Bianco}.&{\bfseries\scshape Nero}.\\
Re alla 2.\ della sua dam. &Re alla 2.\ del cav.\ della dama contr., con una\\
Rocco alla casa dell'alfiere della dama.& pedina ivi alla 2.\ del roc.\\
Pedina del rocco del re alla quarta del rocco.&Due pedine una del rocco del Re alla 4.\ e l'altra del cav.\ alla terza.\\
 \end{tabular}
}


Se il Bianco non giuoca come segue farà patta.

\vspace{6pt}{\small\noindent\begin{tabular}{@{}p{3.84cm}p{3.84cm}}
Rocco in casa del rocco avanti la ped.\ contr. & O lo prende o no: se lo prende. \\
Re alla 2.\ dell'alfiere. & Avanza la sua pedina.\\
\end{tabular}}

Il Bianco la prende, fa dama prima del nero, e vince il giuoco.

Se il nero non prende il rocco si ritira, e perde tutto.

\subsubsection{TRATTO XVI.}

{\small\noindent\begin{tabular}{@{}p{3.84cm}p{3.84cm}}
 {\bfseries\scshape Bianco}. & {\bfseries\scshape Nero}.\\
Re alla 3.\ del suo alfiere. & Rocco di re in sua casa. \\
Donna alla quarta del cavallo di suo re. & Rocco in seconda. Dama in quarta.\\
\end{tabular}

\noindent\begin{tabular}{@{}p{3.84cm}p{3.84cm}}
Alfiere di donna in sua casa. & Pedina in quinta.\\
Alfiere di re alla 2. di. cav.\ di donna contr. & Re in settima di detto rocco. \\
\end{tabular}}

Il Bianco dà matto al quinto colpo.

\vspace{6pt}{\small\noindent\begin{tabular}{@{}p{3.84cm}p{3.84cm}}
Alf.\ di dama dà scacco alla 4.\ d'alf.\ di re. & Re in casa del rocco del re contr.\\
Alf.\ di re alla 3.\ dell'alf. di donna nera. & Se la dama pigliala dama, e dà scacco.\\
Re alla 2.\ del suo alf.\ e dice scacco di scoperta. & Si copre con la dama. \\
\end{tabular}}

Il Bianco piglia con l'alfiere e dà scacco matto.

Se il Nero non piglia la dama, ma la giuoca alla sua
quarta, e dà scacco. Il Bianco va sempre col re
alla seconda del suo alfiere, e dopo farà matto, o,
con l'alfiere, o con la dama.

Se poi il Nero giuoca il rocco in casa dell'alfiere
della sua donna, o del cavallo de suo re. Il Bianco
re alla seconda dell'alfiere, e dice scacco di
scoperta. Il Nero piglia l'alfiere. Il Bianco dà
scacco matto con la dama alla quarta del rocco.

Se il Nero non piglia l'alfiere, ma si copre con la
dama. Il Bianco la piglia con l'alfiere, e dà matto.


\subsubsection{TRATTO XVII.}
\hfill\textit{\footnotesize Bellissimo.}
{\small
\noindent\begin{tabular}{@{}p{3.84cm}p{3.84cm}}
 {\bfseries\scshape Bianco}. & {\bfseries\scshape Nero}.\\

Re alla seconda di rocco
di dama contr. con tre
pedine alle 3.\ di cav.\ di
donna, di dama, e di
alfiere di re nero. &Re in casa di sua dama.\\
\end{tabular}

\noindent\begin{tabular}{@{}p{3.84cm}p{3.84cm}}
Rocco di re in sua casa. &Pedina di rocco di re alla terza.\\
Cavallo avanti il rocco.&\\
\end{tabular}

\noindent\begin{tabular}{@{}p{3.84cm}p{3.84cm}}
Dama alla quinta di detto rocco.&\\
Alfiere alla terza di rocco di dama.&\\
\end{tabular}
}



Il Bianco dà matto di tre pedine successivamente in
quindici tratti.

\vspace{6pt}{\small
\noindent\begin{tabular}{@{}p{3.84cm}p{3.84cm}}
Dama alla 2. d'alf.\ di re. &Re in casa dell'alfiere.\\
Cavallo alla terza d'alf. &Re in casa di sua dama.\\
\end{tabular}

\noindent\begin{tabular}{@{}p{3.84cm}p{3.84cm}}
Cavallo alla quarta di re nero. &Re come sopra.\\
 Rocco alla sua terza. &Re torna come sopra.\\
\end{tabular}

\noindent\begin{tabular}{@{}p{3.84cm}p{3.84cm}}
Dama dà scacco alla seconda di re nero.&Re come sopra.\\
Cav.\ alla 4.\ dell'alfiere del suo re.&Avanza la pedina.\\
\end{tabular}

\noindent\begin{tabular}{@{}p{3.84cm}p{3.84cm}}
Cav.\ alla 5.\ di suo re.&Avanza la pedina.\\
Rocco alla 3.\ del suo alf.&Avanza la pedina.\\
\end{tabular}

\noindent\begin{tabular}{@{}p{3.84cm}p{3.84cm}}
Rocco alla 3.\ del suo re.&La pedina.\\
Rocco alla 3.\ del suo alf.&Fa dama.\\
\end{tabular}

\noindent\begin{tabular}{@{}p{3.84cm}p{3.84cm}}
Rocco dà scacco. &Si copre con la dama.\\
Dama dà scacco alla seconda dell'alfiere. &La prende.\\
\end{tabular}

\noindent\begin{tabular}{@{}p{3.84cm}p{3.84cm}}
Dà scacco con ped.\ di cavallo.& Re in casadidama.\\
Prende la dama collapedina, e dà scacco.& Re In sua casa.\\
\end{tabular}}
Il Bianco dà scacco matto colla pedina dell'alfiere.

\subsubsection{TRATTO XVIII.}

{\small\noindent\begin{tabular}{@{}p{3.84cm}p{3.84cm}}
 {\bfseries\scshape Nero}. &{\bfseries\scshape Bianco}.\\
Re alla seconda di alfiere di sua dama. & Re alla seconda di rocco di regina nera.\\
Cavallo in casa di dama bianca. & Pedina di roc.\ alla sesta.\\
& Pedina di cavallo di dama alla terza.\\
\end{tabular}}

Il Nero dà scacco matto di cavallo in quattro mosse.

\vspace{6pt}{\small\noindent\begin{tabular}{@{}p{3.84cm}p{3.84cm}}
Cavallo alla 3.\ dell'alf. & Spinge 1a pedina.\\
Dà scacco col cavallo. & Re in casa del rocco.\\
Ritira il re in casa dell'alfiere. & Spinge la pedina del rocco.\\
\end{tabular}}

Il cavallo nero dà scacco matto alla seconda dell'alfiere.

Se il Bianco giuoca prima la ped.\ del cav.\ di donna.

\vspace{6pt}{\small\noindent\begin{tabular}{@{}p{3.84cm}p{3.84cm}}
Cav.\ alla 3.\ dell'alf. & Avanza detta pedina,
viene come sopra: ma
se va in cantone col re.\\
Cav.\ alla 4.\ sua. & Spinge la ped.\ del rocco.\\
\end{tabular}

\noindent\begin{tabular}{@{}p{3.84cm}p{3.84cm}}
Cav.\ alla 3.\ dell'alfiere contr. & Avanza la pedina.\\
Cav.\ alla 4.\ di sua dama. & Dà scacco colla sua ped.\\
\end{tabular}}

Il Nero la prende col cavallo, e dà scacco matto.

\subsubsection{TRATTO XIX.}

{\small\noindent\begin{tabular}{@{}p{3.84cm}p{3.84cm}}
{\bfseries\scshape Bianco}.& {\bfseries\scshape Nero}.\\

Re in casa dell'alf.\ della dama contraria. & Re in casa di rocco di dama,\\
Due rocchi alla loro 4. & con tre pedine avanti,\\
\end{tabular}

\noindent\begin{tabular}{@{}p{3.84cm}p{3.84cm}}
Dama alla 4.\ del suo alf. & E tre altre avanti al re bianco.\\
Due alf.\ alle 3.\ di cav.\ e di rocco di re.\\
\end{tabular}

\noindent\begin{tabular}{@{}p{3.84cm}p{3.84cm}}
Due ped.\ di da.\ e di cav.\ di dama alle seconde.\\
\end{tabular}}

Il Bianco dà matto di pedina in dieci mosse.

\vspace{6pt}{\small\noindent\begin{tabular}{@{}p{3.84cm}p{3.84cm}}
Rocco alla 4.\ del suo re. &Piglia con pedina.\\
Alf.\ alla 2.\ di cav.\ nero. &Promove la pedina.\\
\end{tabular}

\noindent\begin{tabular}{@{}p{3.84cm}p{3.84cm}}
L'Alf.\ alla terza di re.& Lo piglia.\\
Re alla 2.\ dell'alfiere. & Promove la pedina.\\
\end{tabular}

\noindent\begin{tabular}{@{}p{3.84cm}p{3.84cm}}
Dama alla quarta dell'alfiere di dama nera. & Promove la pedina.\\
Rocco alla sua terza. & Ivi promove la pedina.\\
\end{tabular}

\noindent\begin{tabular}{@{}p{3.84cm}p{3.84cm}}
La sua pedina quanto la va. & L'altra.\\
La sua. & L'altra.\\
Donna in casa dell'alf.\ del re neroda scacco. & Re avanti.
 \end{tabular}
}

Il Bianco dà scacco matto di pedina alla sesta.

\subsubsection{TRATTO XX.}
\hfill\textit{\footnotesize Ingegnoso.}

{\small\noindent\begin{tabular}{@{}p{3.84cm}p{3.84cm}}
 {\bfseries\scshape Bianco}. &{\bfseries\scshape Nero}. \\
Re alla 2. dell'alfiere di donna nera. &Re in casa di rocco di sua dama.\\
Rocco alla seconda di cavallo ivi. & Pedina alla quarta di re bianco.
\end{tabular}

\noindent\begin{tabular}{@{}p{3.84cm}p{3.84cm}}
Due pedine alle terze di detti dietro. \\
Rocco alla 2.\ del suo re. \\
\end{tabular}}

Il Bianco senza movere il re dà matto di pedina in
sei mosse.

\vspace{6pt}{\small\noindent\begin{tabular}{@{}p{3.84cm}p{3.84cm}}
Rocco alla 2.\ dell'alfiere del suo re. & Avanza la pedina. \\
Rocco alla 4.\ dell'alf. & Avanza la pedina. \\
\end{tabular}

\noindent\begin{tabular}{@{}p{3.84cm}p{3.84cm}}
Rocco alla 3.\ dell'alf. & Fa Dama.\\
Rocco dà scacco alla sua terza. & Si copre con la dama.\\
\end{tabular}

\noindent\begin{tabular}{@{}p{3.84cm}p{3.84cm}}
Rocco posto alla 2.\ di
cav.\ dà scacco alla 2.\ di
rocco nero. & Prende colla dama.\\
\end{tabular}}

Il Bianco dà scacco matto di pedina.

\subsubsection{TRATTO XXI.}

{\small\noindent\begin{tabular}{@{}p{3.84cm}p{3.84cm}}
{\bfseries\scshape Nero}. & {\bfseries\scshape Bianco}. \\
Rocco alla terza dell'alfiere della dama contr. & Re in casa del rocea del la sua dama.\\
Rocco alla terza del cavallo della dama contr.\\
\end{tabular}

\noindent\begin{tabular}{@{}p{3.84cm}p{3.84cm}}
Cavallo alla quarta della dama contraria.\\
Alfiere alla 4.\ del cavallo della dama contr.\\
Re in sua casa.
\end{tabular}}

Si dee dare dal Nero tre volte scacco, è l'ultima, matto.

Nero. Rocco dà scacco in casa dell'alfiere, e poi
nella seconda, poi matto coll'alfiere alla terza dell'alfiere contrario.

\subsubsection{TRATTO XXII.}

{\small
\noindent\begin{tabular}{@{}p{3.84cm}p{3.84cm}}
{\bfseries\scshape Nero.}& {\bfseries\scshape Bianco}.\\
Rocco alla quarta dell'alfiere del suo re. & Re alla sua quarta.\\
Rocco alla 4.\ della a Cav.\ alla terza dell'al- fiere del re contrario.\\
\end{tabular}

\noindent\begin{tabular}{@{}p{3.84cm}p{3.84cm}}
Cav.\ alla terza della dama contraria.\\
Re in sua casa.\\
\end{tabular}}

Senza dar scacco in tre mosse darlo matto nella quarta

\vspace{6pt}{\small
\noindent\begin{tabular}{@{}p{3.84cm}p{3.84cm}}
Cavallo del Re alla casa
del re contrario. &Re alla sua terza.\\

“Cavallo della dama alla
casa dell'alfiere della
dama contraria. &Re alla sua quarta.\\
\end{tabular}

\noindent\begin{tabular}{@{}p{3.84cm}p{3.84cm}}

Lo ftesso cav.\ alla z. del
cav. della dam.\ contr.&Re alla sua terza.\\
\end{tabular}}

Nero. Scacco col rocco della dama alla quarta del
suo re, ed è matto.

\subsubsection{TRATTO XXIII.}

{\small
\noindent\begin{tabular}{@{}p{3.84cm}p{3.84cm}}
{\bfseries\scshape Bianco}. {\bfseries\scshape Nero}.
Pedina d'alfiere di re in sua casa con avanti cav.\ due roc.\ e alla 6. il re.& Re alla casa del suo alfiere, tutti in linea.
\end{tabular}
}

Il Bianco dà scacco matto in sei mosse con la pedina.

\vspace{6pt}{\small
\noindent\begin{tabular}{@{}p{3.84cm}p{3.84cm}}
Rocco alla 4.\ dell'alf.\ va alla quarta del suo re. & Re alla casa del cavallo.\\
Scacco col sopradet.\ roc.\ nella casa del re contr. & Re nella seconda del suo rocco.\\
\end{tabular}

\noindent\begin{tabular}{@{}p{3.84cm}p{3.84cm}}
Scacco col cavallo alla quarta del cav.\ contr. &Re alla terza del suo rocco.\\
Il sudetto rocco alla terza del suo re. &Re alla quarta del suo rocco.\\
Il detto rocco dà scacco alla sua terza. &Il re va per forza a prendersi matto di pedina alla 4.\ del cav.\ contr.\\
\end{tabular}
}

\subsubsection{TRATTO XXIV.}

{\small
\noindent\begin{tabular}{@{}p{3.84cm}p{3.84cm}}
{\bfseries\scshape Bianco}.&{\bfseries\scshape Nero}.\\
Re alla casa del suo roc.&Re alla casa del rocco della dama. \\
\end{tabular}

\noindent\begin{tabular}{@{}p{3.84cm}p{3.84cm}}
Rocco del re alla casa dell'alfiere del suo re.&Ambi i rocchi nelle case reali. \\
Rocco della dama in sua casa.&Alfiere del re in sua casa.\\
\end{tabular}

\noindent\begin{tabular}{@{}p{3.84cm}p{3.84cm}}
Alfiere della dama alla seconda del roc.\ del re. Alf.\ del re alla 2.\ del re.&Alfiere della dama alla seconda del suo cavallo.\\
Cavallo alla terza dell'alfiere della dama.&Cavallo alla seconda di dam.\\
\end{tabular}

\noindent\begin{tabular}{@{}p{3.84cm}p{3.84cm}}
Pedine del cavallo, e dell'alfiere del re in loro casa. &Altro cavallo alla quarta del cavallo del re contrario. \\
Pedina del cavallo della dama alla quarta del cav.\ contrario.&Dama alla 4.\ del rocco del suo re.\\
\end{tabular}

\noindent\begin{tabular}{@{}p{3.84cm}p{3.84cm}}
Pedina del'alfiere della dama alla quarta dell'alf.\ della dam.\ contr.&Pedine dei due rocchi, e del cavallo, e dell'alfiere del re alle loro case.\\
Ped.\ del re alla 4.\ del re.&
\end{tabular}
}


Pare, che il Nero in questa disposizione di giuoco,
e per numero de i pezzi, e per qualità di essi sia
in istato di vincere, e pure il Nero avrà per
forza scacco matto al più in sei mosse.


\vspace{6pt}{\small
\noindent\begin{tabular}{@{}p{3.84cm}p{3.84cm}}
Il Rocco prende la ped. contr., e dà scacco. & Il re prende il rocco.\\
Ped.\ del cav.\ della dam.\ s'avanza alla terza del cav.\ nero, e dà scacco. & Re torna nella casa del rocco.\\
\end{tabular}

\noindent\begin{tabular}{@{}p{3.84cm}p{3.84cm}}
Dà scacco col rocco alla casa del roc.\ della dam. & Il re si copre coll'alfiere alla terza del rocco.\\
Lo prende col rocco e dà scacco. & Re alla seconda del cavallo.\\
Rocco alla 2.\ del rocco contrario, e dà scacco. & Re alla casa dell'alfiere.
\end{tabular}
}
Bianco dà matto di pedina, e se va alla terza dell'alfiere, avrà matto col rocco alla seconda di detto alfiere.

\subsubsection{TRATTO XXV.}
{\small
 \noindent\begin{tabular}{@{}p{3.84cm}p{3.84cm}}
 {\bfseries\scshape Nero}.& {\bfseries\scshape Bianco}.\\
Re alla casa dell'alfiere della dama contraria; Roc.\ di dama nella sua 3.; Ped.\ del cavallo della dama in sua casa. & Re nella casa del rocco della dama, e quattro pedine avanti, e quattroavanti alla casa del suo alfiere. \\
\end{tabular}

\noindent\begin{tabular}{@{}p{3.84cm}p{3.84cm}}
Dama alla 2.\ del suo alf.\\
Cavallo alla casa della sua dama.\\
\end{tabular}

\noindent\begin{tabular}{@{}p{3.84cm}p{3.84cm}}
Cav.\ alla casa del suo re.\\
Alfiere alla 2.\ del rocco del re \\
\end{tabular}

\noindent\begin{tabular}{@{}p{3.84cm}p{3.84cm}}
Alf.\ alla terza di detto.\\
Rocco del re, alla sua 4.
\end{tabular}
}

Ciò essendo disposto, il Nero darà scacco matto con
la pedina in tredici mosse.



\vspace{6pt}{\small
 \noindent\begin{tabular}{@{}p{3.84cm}p{3.84cm}}
Cav., che sta alla casa del re, si pone alla terza della dama in presa dell'ultima pedina. & Lo prende.\\
Promuove la dama nella casa lasciata dalla pedina. &Avanza la pedina.
\end{tabular}

\noindent\begin{tabular}{@{}p{3.84cm}p{3.84cm}}
Mette il rocco, che sta alla sua quarta in presa della pedina. &Lo prende. \\
Alf.\ alla 2.\ del suo cavallo, acciocché il Bianco non scopra. &Avanza la pedina, che prese il rocco.\\
\end{tabular}

\noindent\begin{tabular}{@{}p{3.84cm}p{3.84cm}}
Alf.\ alla terza della dama contraria. &Lo prende con la ped.\\
Re alla 2.\ dell'alfiere. &Ped, alla 4.\ del re.\\
\end{tabular}

\noindent\begin{tabular}{@{}p{3.84cm}p{3.84cm}}
Dama alla quarta dell'alfiere contrario. &Ped.\ alla 5.\ del re.\\
Ritira il rocco alla sua seconda. &Avanza la pedina prossima.\\
\end{tabular}

\noindent\begin{tabular}{@{}p{3.84cm}p{3.84cm}}
Avanzala sua. &Avanza l'altra.\\
Avanza la sua. &L'altra.\\
\end{tabular}

\noindent\begin{tabular}{@{}p{3.84cm}p{3.84cm}}
La sua. &L'altra. \\
Dama nella prima linea dà scacco. &Re alla seconda del rocco.\\
\end{tabular}
}

Il Nero dà scacco matto colla sua pedina.

\subsubsection{TRATTO XXVI.}

{\small
\noindent\begin{tabular}{@{}p{3.84cm}p{3.84cm}}
{\bfseries\scshape Nero}. & {\bfseries\scshape Bianco}. \\
Re alla sua terza. Pedina avanti alla 4. &Re alla casa dell'alfiere del re nero.\\
Dam.\ alla 4.\ di suo roc. &Pedina alla 4.\ di suo re.\\
\end{tabular}

\noindent\begin{tabular}{@{}p{3.84cm}p{3.84cm}}
Rocco di dam.\ alia sua 2. &Due rocchi uno in casa di dama,\\
Rocco di re alla terza di cavailo di re. &L'altro in casa d'alf.\ di re.\\
Cav alla 3.\ di rocco di re.& Cav di re alla 3.\ di rocco di re.\\
\end{tabular}
}

Il bianco dà matto in tre mosse Giuoca prima il Bianco.

\vspace{6pt}{\small
\noindent\begin{tabular}{@{}p{3.84cm}p{3.84cm}}
&Cav.\ Scacco alla sua 5.\\
Lo prende col rocco. &Rocco, che sta in casa d'alf. dà scacco. \\
Lo prende col re. &Dà matto coll'altro rocco alla 3.\ di dama nera.
\end{tabular}
}


\subsubsection{TRATTO XXVII.}

{\small
\noindent\begin{tabular}{@{}p{3.84cm}p{3.84cm}}
{\bfseries\scshape Bianco}. & {\bfseries\scshape Nero}. \\
Re alla terza della dama contr. con quattro&Re In casa della sua dam. \\
pedine bianche nelle 4 case bianche d'intorno. &Alf.\ in casa del rocco del re contr. \\
Cavallo alla 3.\ del rocco del re contr. &Roc della dam alla sua 2.\\
\end{tabular}

\noindent\begin{tabular}{@{}p{3.84cm}p{3.84cm}}
Cavallo alla 4.\ del caval lo del re dama contr. &Rocco del re in casa del suo alf.\\
&Ped.\ del cav.\ della dama in sua casa. \\
&Ped.\ del rocco deila dama alla terza del rocco.\\
\end{tabular}}

Il bianco dà scacco matto in cinque mosse con la pedina, che sta in presa dell'alfiere.

\vspace{6pt}{\small
\noindent\begin{tabular}{@{}p{3.84cm}p{3.84cm}}
Cav.\ alla 3.\ del rocco dà scacco alla 2.\ dell'alf.&Lo piglia col rocco.\\
Avanza la pedina posta alla 3.\ del re, e dà scac.&La piglia col rocco.\\
\end{tabular}

\noindent\begin{tabular}{@{}p{3.84cm}p{3.84cm}}
Cav.\ dà scacco alla terza del re.&Prende col rocco.\\
\end{tabular}}

Prende il rocco colla pedina con cui poi dà scacco matto.


\section{CAPO SETTIMO.}

\subsection{Maniere di giuocare.}

Ho scelto alcune poche maniere di giuocare,
pajono a me sufficienti per introdurre un bel
talento a prendere il buon gusto di questo ingegnosissimo
Giuoco, perché, come dice Plutarco in
Mor.\ ad un bravo ingegno: \textit{Satis est accensio, velut
incitamentum in materia, quo vis inveniendi, et
cupiditas veritatis incitetur.}

\subsubsection{Prima maniera.}
{\small
\noindent
\begin{tabular}{@{}p{3.84cm}p{3.84cm}}
{\bfseries\scshape Bianco}. &{\bfseries\scshape Nero}.\\
Ped.\ del re due case.&L'istesso. \\
Cav.\ del re alla 3.\ d'alf.&Cav.\ di dam.\ alla 3.\ d alf.\\
Alf.\ di re alla 4.\ d'alfiere di dama.&L'istesso.\\
\end{tabular}

\noindent\begin{tabular}{@{}p{3.84cm}p{3.84cm}}
Ped.\ d'alf. di dama una casa. &Cavallo di re alla terza d'alfiere.\\
Ped.\ di dama due case.&Ped.\ di re laprende. \\
Ped.\ prende la pedina.&Alfiere dà scacco.\\
\end{tabular}

\noindent\begin{tabular}{@{}p{3.84cm}p{3.84cm}}
Cavallo copre alla terza d'alfiere. &Cav.\ prende la pedina del re.\\
Re salta. &Cavallo di re prende il cavallo di dama.\\
\multicolumn{2}{@{}c}{\markerA}\\
Pedina prende il cavallo &Alfiere di re prende la ped.\ alla 3.\ d'alf.\ di da.\ contr.\\
\end{tabular}

\noindent\begin{tabular}{@{}p{3.84cm}p{3.84cm}}
Dama alla 3.\ di suo cav. &Alfiere prende il rocco.\\
Alf.\ prende la ped.\ d'alf.\ di re, e dà scacco. &Re in casa d'alfiere.\\
\end{tabular}

\noindent\begin{tabular}{@{}p{3.84cm}p{3.84cm}}
Alf.\ di dam.\ alla quarta di cav.\ di re contrario. &Cav.\ alla seconda di suo re.\\
\multicolumn{2}{@{}c}{\markerB}\\
Cavallo alla quarta di re contrario. &Alf.\ prende la ped.\ alla 4.\ di dama contraria.\\
\end{tabular}

\noindent\begin{tabular}{@{}p{3.84cm}p{3.84cm}}
Alf.\ di re alla 3.\ di cavallo di re contrario-.&Ped.\ di dama due case. \\
Dama dà scacco alla terza d'alf.\ di suo re.&Alfiere copre. \\
\end{tabular}

\noindent\begin{tabular}{@{}p{3.84cm}p{3.84cm}}
Alfiere prende l'alfiere. &Alfiere prende il caval.\\
Alf.\ alla 3.\ di re contr.\ e dà scacco di scoperta. &Alfiere copre alla sua terza. \\
\end{tabular}

\noindent\begin{tabular}{@{}p{3.84cm}p{3.84cm}}
Alfiere prende l'alfiere &Ped.\ prende l'alfiere. \\
Dam.\ prende terza dell'alfiere contrario, e dà scacco. &Re in sua casa.\\
\end{tabular}
}

Dama dà scacco matto alla seconda d'alfiere di re
contrario.

dopo l'{\markerB}

\vspace{6pt}{\small
\noindent
\begin{tabular}{@{}p{3.84cm}p{3.84cm}}
Cav.\ alla 4 di recontr. & Ped di dama due case.\\
Dama alla terza d'alfiere di suo re. & Alf.\ di dama alla 4 d'alf.\\
\end{tabular}

\noindent\begin{tabular}{@{}p{3.84cm}p{3.84cm}}
Alf.\ alla terza di re contrario & Ped.di dam- di re una casa\\
Alf.\ di da.\ dà scacco. alla 3.\ di rocco di re contr. & Re in sua casa \\
\end{tabular}
}

Alfiere di re dà scacco matto alla 2.\ d'alfiere di re
contrario.

Dopo la {\markerA}

{\small\noindent\begin{tabular}{@{}p{3.84cm}p{3.84cm}}
La ped.\ prende il cav. &Alfiere prende la ped.\\
Dama alla 3.\ di suo cav.& Alfiere prende il rocco.\\
\end{tabular}

\noindent\begin{tabular}{@{}p{3.84cm}p{3.84cm}}
Alf.\ prende la ped d'alf. e dà scacco.&Re in casa d'alfiere.\\
Alf.\ di dama alla 4.\ di cav.\ contr. & Cav.\ prende la pedina alla quarta di dama.\\
\end{tabular}

\noindent\begin{tabular}{@{}p{3.84cm}p{3.84cm}}
Dama dà scacco alla terza di rocco. & Re prende l'alfiere\\
Alfiere prende la dama. & Rocco prende l'alfiere.\\
\end{tabular}

\noindent\begin{tabular}{@{}p{3.84cm}p{3.84cm}}
Rocco prende l'alfiere. Cav.\ alla seconda d'alfiere di dama contraria. \\
Dama dà scacco alla 3.\ di cav. & Re in casa d'alfiere.\\
\end{tabular}

\noindent\begin{tabular}{@{}p{3.84cm}p{3.84cm}}
Dama prende il cavallo, e vincerà\\
\end{tabular}}

Ancora dopo \markerA

{\small\noindent\begin{tabular}{@{}p{3.84cm}p{3.84cm}}
Dam.\ alla 3.\ di suo cav.&Alf.\ prende la pedina alla quarta di dama.\\
Alf.\ prende la pedina, e dà scacco.&Re in casa d'alfiere.\\
\end{tabular}

\noindent\begin{tabular}{@{}p{3.84cm}p{3.84cm}}
Alf.\ di dam.\ alla 4.\ di cavallo di re contrario.&Alfiere copre alla sua terza. \\
\end{tabular}

\noindent\begin{tabular}{@{}p{3.84cm}p{3.84cm}}
Roc.\ di da. in casa di re.&Cav.\ alla seconda di re.\\
\multicolumn{2}{@{}c}{\markerC}\\
Alf.\ di re alla 4.\ di rocco contrario.&Cav.\ alla terza di cavallo di re.\\
\end{tabular}

\noindent\begin{tabular}{@{}p{3.84cm}p{3.84cm}}
Cav.\ alla quarta di re contrario.&Cav.\ prende il cavallo.\\
Rocco prende il caval.&Ped.\ di cav.\ una casa.\\
\end{tabular}

\noindent\begin{tabular}{@{}p{3.84cm}p{3.84cm}}
Alfiere di dama dà scacco alla terza di rocco.&Alfiere copre.\\
Rocco dà scacco alla 4.\ d'alfiere.&Pedina prende il rocco.\\
\end{tabular}
}


Dama dà scacco matto alla seconda d'alfiere di re
contrario.

Dopo \markerC{}



{\small
\noindent\begin{tabular}{@{}p{3.84cm}p{3.84cm}}
Alfiere di re alla quarta di rocco contrario.&Ped.\ di dama due case.\\
Rocco prende il cavallo contrario. &Re prende ilrocco.\\
\multicolumn{2}{@{}c}{\markerD}\\
Rocco dà scacco in casa di re.&Re in casa d'alfiere.\\
\end{tabular}

\noindent\begin{tabular}{@{}p{3.84cm}p{3.84cm}}
Dama dà scacco alla 4.\ di suo cavallo.&Re In casa di cavallo.\\
\end{tabular}
}


Rocco dà scacco in casa di re contrario, prenderà
la dama contraria, e guadagnerà.

dopo \markerD

{\small\noindent\begin{tabular}{@{}p{3.84cm}p{3.84cm}}
Rocco dà scacco in casa di re. & Re alla seconda di dama.\\
\end{tabular}}

Dama prende la pedina di dama, e dà scacco matto.

\subsubsection{Altra maniera di giuocare.}

{\small\noindent\begin{tabular}{@{}p{3.84cm}p{3.84cm}}
{\bfseries\scshape Bianco}.&{\bfseries\scshape Nero}.\\
Ped.\ del re due case.&L'istesso.\\
\end{tabular}

\noindent\begin{tabular}{@{}p{3.84cm}p{3.84cm}}
Gav.\ del re alla terza d'alfiere.&Cav.\ di dama alla terza d'alfiere.\\
Alfiere di re alla quarta d'alfiere di dama.&Littesso.\\
\end{tabular}

\noindent\begin{tabular}{@{}p{3.84cm}p{3.84cm}}
Ped.\ d'alfiere di dama una casa. &Daina alla seconda di re.\\
Re salta. &Ped.\ di dama una casa.\\
\end{tabular}

\noindent\begin{tabular}{@{}p{3.84cm}p{3.84cm}}
Ped.\ di dama due case. &Alf.\ di re alla 3.\ di cav-\\
Alf.\ di dama alla 4.\ di cavallo di re contrario.&Ped.\ d'alfiere una casa.\\
\end{tabular}

\noindent\begin{tabular}{@{}p{3.84cm}p{3.84cm}}
Detto alla quarta di rocco. &Ped.\ di cavallo di re due case.\\
La prende col cavallo. &Ped.\ prende il cavallo.\\
\end{tabular}

\noindent\begin{tabular}{@{}p{3.84cm}p{3.84cm}}
Dama dà scacco alla 4.\ di rocco contrario.&Re alla seconda di sua dama.\\
\multicolumn{2}{@{}c}{\markerA}\\
Alf.\ prende la pedina. &Dam alla seconda di cav.\\
\end{tabular}

\noindent\begin{tabular}{@{}p{3.84cm}p{3.84cm}}
Alf.\ del re dà scacco. alla terza di re contrario. &Re lo prende.\\
Dama dà scacco alla casa del re contrario.&Cav.\ lo copre.\\
\end{tabular}}

Pedina di dama dà scacco matto.

Dopo \markerA, se si difende così.

{\small\noindent\begin{tabular}{@{}p{3.84cm}p{3.84cm}}
Alf.\ prende la pedina. & Dama alla casa d'alf.\\
\multicolumn{2}{@{}c}{\markerB}\\
Alf.\ di re alla seconda d'alf.\ di re contrario. & Pedina di re prende la pedina.\\
\end{tabular}}

Dama dà scacco matto alla quarta di cavallo di suo re.

Dopo \markerB

{\small\noindent\begin{tabular}{@{}p{3.84cm}p{3.84cm}}
Alf.\ di re alla 2.\ d'alfiere di re contrario. &Cavallo di dama, alla seconda di re.\\
\multicolumn{2}{@{}c}{\markerC}\\
Ped.\ prende la pedina.&Ped.\ prende la pedina.\\
Rocco dà scacco. &Re alla terza d'alfiere di sua dama.\\
Alf.\ di re dà scacco alla casa di re contrario.&Re alla quarta d'alfiere.\\
Alf.\ di dama dà scacco alla terza di suo re.&Re alla quarta d'alfiere di dama contraria. \\
\end{tabular}}

Pedina di cavallo dà scacco matto.

Dopo \markerC

{\small\noindent\begin{tabular}{@{}p{3.84cm}p{3.84cm}}
Pedina prende la ped.&Pedina di rocco una casa.\\
Alf.\ di dama alla quarta di rocco di suo re.&Rocco alla sua seconda.\\
\multicolumn{2}{@{}c}{\markerD}\\
Ped.\ di dama dà scacco. & Re alla terza dell'alfiere di dama\\
\end{tabular}}

Alfiere di re dà scacco matto alla casa di re contr.

Dopo \markerD

{\small\noindent\begin{tabular}{@{}p{3.84cm}p{3.84cm}}
Ped. \&c. Re in casa di sua dama.\\
Ped di re una casa. &Ped.\ di dama una casa.\\
\end{tabular}

\noindent\begin{tabular}{@{}p{3.84cm}p{3.84cm}}
Dama alla terza dell'alfiere di suo re. & Ped.\ d'alfiere di dama una casa.\\
\end{tabular}

\noindent\begin{tabular}{@{}p{3.84cm}p{3.84cm}}
Pedina d'alf.\ di dama una casa. & Re alla seconda d'alfiere di sua dama.\\
\multicolumn{2}{@{}c}{\markerB}\\
Dam.\ alla 3.\ di suo roc. & Re alla casa di cavallo.\\
Alf.\ di re prende il cav. & Dama prende l'alfiere\\
\end{tabular}}

Alfiere prende il cavallo, e guadagnerà.


Dopo \markerB{} se si difende cosi.

{\small\noindent\begin{tabular}{@{}p{3.84cm}p{3.84cm}}
Dama alla terza di suo rocco. & Ped.\ d'alfiere una casa. \\
Pedina prende la pedina di dama. & Cavallo di dama prende la pedina.\\
Dama alla sua terza. & Rocco prende l'alfiere.\\
\end{tabular}

\noindent\begin{tabular}{@{}p{3.84cm}p{3.84cm}}
Dama prende il cavallo alla 4.\ di dama contr.& Rocco alla quarta d'alfiere di re contrario\\
Pedina una casa alla 2.\ di re contrario. & Cav.\ prende la ped.\ alla seconda del suo re.\\
\end{tabular}}

Dama dà scacco matto alla terza di dama contraria.

\subsubsection{Alcune maniere di giuocare il GAMBITTO.}

{\small\noindent\begin{tabular}{@{}p{3.84cm}p{3.84cm}}
 {\bfseries\scshape Bianco}. &{\bfseries\scshape Nero}.\\
 Ped.\ del re 2 case.& L'istesso.\\
 Ped.\ d'alf.\ di re 2 case. &La prende.\\
\end{tabular}

\noindent\begin{tabular}{@{}p{3.84cm}p{3.84cm}}
Cav.\ di re alla 3 d'alf. & Ped di cav.\ 2 case. \\
\multicolumn{2}{@{}c}{\markerA}\\
Alf.\ di re alla 4.\ d'alf. di Dam.& Ped di cav di re una casa. \\
 \end{tabular}

\noindent\begin{tabular}{@{}p{3.84cm}p{3.84cm}}
Cav.\ di re alla 4 di re contr. & Dam.\ dà scacco alla 4 di roc.\ di re contr. \\
Re alla casa del suo alf. & Cav.\ di re alla 3 di suo rocco.\\
 \end{tabular}

\noindent\begin{tabular}{@{}p{3.84cm}p{3.84cm}}
Ped.\ di dam.\ 2 case. &Ped.\ di dam.\ una casa. \\
Cav.\ di re alla 3 di sua dam. &Ped.\ di re una casa. \\
\multicolumn{2}{@{}c}{\markerB}\\
\end{tabular}

\noindent\begin{tabular}{@{}p{3.84cm}p{3.84cm}}
Ped.\ di cav.\ 1 casa.&Dam.\ dà scac. alla 3 di rocco.\\
Re alla 2.\ di suo alf. &Dam.\ dà scac. alla-2 di cavallo. \\
 \end{tabular}

\noindent\begin{tabular}{@{}p{3.84cm}p{3.84cm}}
Re alla sua terza. &Cav.\ di re in sua casa.\\
Cav.\ di re alla 4 dell'alfiere di suo re. &Alfiere di re alla terza del rocco suo.\\
 \end{tabular}

\noindent\begin{tabular}{@{}p{3.84cm}p{3.84cm}}
Alf.\ di re in sua casa. &Dam.\ pren. il rocco\\
Alf.\ di re dà scac. alla 4 di cav.\ di dam.\ contr. &Ped.\ d'alfiere copre.\\
 \end{tabular}

\noindent\begin{tabular}{@{}p{3.84cm}p{3.84cm}}
Aif.\ la prende, e dà scac. &Prende l'alf.\ con la ped.\\
Dam.\ prende la dama. &
\end{tabular}}

Dopo \markerB{} se il Nero si difende cosi.

{\small\noindent\begin{tabular}{@{}p{3.84cm}p{3.84cm}}
Ped.\ di cav.\ 1.casa. &Dama dà scac.\ alla 3.\ di rocco. \\
Re in sua casa.&Dam.\ alla 2.\ di cav contr.\\
Cav alla 2.\ d'alf.\ di suo re.&Cav. di dam.\ alla 3 d'alf.\\
\end{tabular}}

Alfiere di re in sua casa, e vincerà.

Dopo \markerB{} se il Nero si difende cosi.

{\small\noindent\begin{tabular}{@{}p{3.84cm}p{3.84cm}}
Alf.\ di re alla 4.\ d'alfiere di sua dam.&Ped.\ di cav di re una casa. \\
Cav.\ di re alla 4 di re. contr. &Dam.\ dà scac.\ al 4 di roc.\ contr.\\
 \end{tabular}

\noindent\begin{tabular}{@{}p{3.84cm}p{3.84cm}}
Re in casa d'alfiere. &Cav.\ di re alla 3 di suo rocco.\\
Ped.\ di dam.\ 2 case. &Ped.\ di dam.\ una casa.\\
 \end{tabular}

\noindent\begin{tabular}{@{}p{3.84cm}p{3.84cm}}
Cav.\ di re alla 3 di sua dama.&Pedina di re una casa.\\
Ped.\ di cav.\ di re una casa. &Dama dà scacco alla terza di rocco contrario.\\
 \end{tabular}

\noindent\begin{tabular}{@{}p{3.84cm}p{3.84cm}}
Re in sua casa. &Dama al 4 di rocco di suo re. :\\
 \end{tabular}

\noindent\begin{tabular}{@{}p{3.84cm}p{3.84cm}}
Cav.\ di re alla 4.\ d'alfiere di suo Re. &Dama dà scacco alla 4 di suo rocco. :\\
\multicolumn{2}{@{}c}{\markerC}\\
Alfiere di dama copre &Dama alla 3 di suo cav.\\
 \end{tabular}

\noindent\begin{tabular}{@{}p{3.84cm}p{3.84cm}}
Cav.\ di re alla 4.\ di dam.\ contr. &Dam.\ pr.\ la ped.\ di dama contr.\\
Alfiere di re alla3 di sua dama. & Dama alla 4 del suo alfiere.\\
\end{tabular}

\noindent\begin{tabular}{@{}p{3.84cm}p{3.84cm}}
Alfiere di dama alla terza di suo re. &Dama dà scacco alla 4 di suo rocco.\\
 Pedina di cavallo copre. &Dama alla 4 di rocco di dam.\ contr.\\
 \end{tabular}

\noindent\begin{tabular}{@{}p{3.84cm}p{3.84cm}}
Alf.\ di re dà scac. alla quarta di cavallo di dama contr. &Dama lo prende.\\
Cav di re pr.\ la ped.\ dell'alf.\ di dam.\ e dà scac. &Re in casa di sua dama.\\
\end{tabular}}

Il cavallo prende la dama.

Dopo * e se così.

{\small\noindent\begin{tabular}{@{}p{3.84cm}p{3.84cm}}
Alf.\ di dam.\ copre.&Dam.\ a 4 di roc di dam.\ bianca\\
\multicolumn{2}{@{}c}{\markerD}\\
Cav.\ di dam.\ alla. terza di rocco.&Pedina d'alf.\ di dam.\ 1 casa.\\
\end{tabular}

\noindent\begin{tabular}{@{}p{3.84cm}p{3.84cm}}
Cav.\ di re alla 4 casa di dam.\ contr. &Ped.\ di cav.\ di dam.\ 2 case.\\
Ped.\ di. cav- di dam.\ 1.\ casa. &Dam.\ pr.\ il cav.\ di dam.\ contr.\\
\end{tabular}

\noindent\begin{tabular}{@{}p{3.84cm}p{3.84cm}}
Alf.\ di dam.\ alla 4 di cav.\ suo. &Dama alla 2.\ di cav.\ di dam.\ contr.\\
Rocco di dama in 1.\ di cav.&Dam.\ prende la ped.\ del rocco.\\
\end{tabular}

\noindent\begin{tabular}{@{}p{3.84cm}p{3.84cm}}
Rocco in sua casa.&Dama in 2.\ di cav.\\
Alf.\ di dam.\ alla sua 3 &Dam.\ pr.\ il rocco.\\
\end{tabular}

\noindent\begin{tabular}{@{}p{3.84cm}p{3.84cm}}
Dama pr.\ la dama. &Ped.\ d'alf. pr.\ il cav.\\
Alfiere p.\ la ped. &Cavallo di dam.\ alla 2.\ di sua dama.\\
\end{tabular}}

Alfiere di re pr.\ il rocco di dama.

Dopo \markerD, e se cosi.

{\small\noindent\begin{tabular}{@{}p{3.84cm}p{3.84cm}}
Cav di dam alla 3 di roc. & Da alla sua 2.\\
Cav.\ di re alla q di dam.\ contr. & Alf.\ di re alla 2.\ del cav.\ di suo re.\\
Alf.\ di dam.\ pren. il cav. di re. &Alf.\ pren. il cav di dam.\\
Gav.\ di re dà scac. alla 3 dell'alf. di re nero. &Re in casa dell'alf-\\
\end{tabular}}

Cav.\ prende la dama.

\subsubsection{Altra maniera di giuocare il
GAMBITTO.}

{\small
\noindent
\begin{tabular}{@{}p{3.84cm}p{3.84cm}}
 {\bfseries\scshape Bianco}. & {\bfseries\scshape Nero}.\\
Ped.\ del re 2 case.&L'istesso.\\
Ped.\ d'alf. di re 2 case.&Ped.\ del re prende.\\
Cav.\ del re alla 3 d'alf.&Ped, di cav.\ di re 2 case.\\
Alf.\ di re alla 4 d'alf. di sua dam. &Alf.\ di re alla seconda di cav. \\
\end{tabular}

\noindent\begin{tabular}{@{}p{3.84cm}p{3.84cm}}
Ped.\ di dam.\ due case.&Ped.\ di dam.\ una casa.\\
Cav.\ di dam.\ alla 3 dell'alfiere.&Pedina d'alf.\ di dama una casa. \\
\end{tabular}

\noindent\begin{tabular}{@{}p{3.84cm}p{3.84cm}}
Ped di roc.\ di re 2 case.&Ped.\ di roc.\ di re 1 casa.\\
Ped.\ prende la ped.&Ped.\ prende la ped.\\
\end{tabular}

\noindent\begin{tabular}{@{}p{3.84cm}p{3.84cm}}
Il roc.\ prend il rocco.&Alf.\ dere prende il roc.\\
Cav del re alla 4 di re N.&Ped.\ di dam.\ pren il cav.\\
\end{tabular}

\noindent\begin{tabular}{@{}p{3.84cm}p{3.84cm}}
Dam.\ alla 4 di roc.\ di re nero.&Dama alla terza d'alf. di re suo.\\
\multicolumn{2}{@{}c}{\markerC} \\
Ped.\ di dam.\ prende la pedina. &Dama alla seconda di cav.\ di suo re.\ \\
\end{tabular}

\noindent\begin{tabular}{@{}p{3.84cm}p{3.84cm}}
Ped.\ alla 3 di re nero.&Cav di re alla 3 di suo alf.\\
Ped prende la ped d'alf. e dà scacco.&Re in casa dell'alfiere.\\
Alfiere di dama prende la ped.\ di re nero.&Cav.\ di re prende la dama.\\
\end{tabular}
}

Alfiere di dama alla terza di dama contraria, e dà
scacco matto.

dopo \markerC{} se 'l nero si difende così

\vspace{6pt}{\small
\noindent
\begin{tabular}{@{}p{3.84cm}p{3.84cm}}
Ped.\ prende la ped.\ d'alfiere, e dà scacco.&Re alla sua seconda.\\
Dam.\ alla 2.\ del suo re.&Alf.\ di dam.\ alla 3.\ del suo re.\\
\end{tabular}

\noindent\begin{tabular}{@{}p{3.84cm}p{3.84cm}}
Alfiere di re prende l'alfiere di dam.\ contr.&Re prende l'alfiere. \\
Dama dà scacco alla 4 del suo alfiere.&Re alla sua seconda.\\
\end{tabular}

\noindent\begin{tabular}{@{}p{3.84cm}p{3.84cm}}
Dama alla quarta di suo cavallo dà scacco.&Re prende la pedina.\\
Dam prende la pedina di cavallo dà scacco.&Cavallo di dama copre.\\
Dama prende il rocco.&\\
\end{tabular}
}


Dopo \markerC, se il nero si difende così.

\vspace{6pt}{\small
\noindent
\begin{tabular}{@{}p{3.84cm}p{3.84cm}}
Pedina prende la pedina d'alfiere, e dà scacco.& Re in casa di dama. \\
Dam pren la ped di cav.&Dama prende la dama.\\
\end{tabular}

\noindent\begin{tabular}{@{}p{3.84cm}p{3.84cm}}
Ped.\ fa dam.\ e dà scacco.&Re alla 2.\ di dama.\\
Dama prende l'alfiere del re.&Dama prende la pedina del cavallo del re.\\
\end{tabular}

\noindent\begin{tabular}{@{}p{3.84cm}p{3.84cm}}
Dama prende il cavallo del re.&Pedina una casa alla terza d'alf. di re bianco.\\
Dam dà scac.\ alla 2.\ d'alf. &Re alla 3 di sua dama.\\
\end{tabular}

\noindent\begin{tabular}{@{}p{3.84cm}p{3.84cm}}
Alf.\ di dam.\ dà scac.\ alla 4dell'alf. di suo re. &Re alla 4 d'alf. di sua dama. \\
\multicolumn{2}{@{}c}{\markerC}\\
Il cav dà scac. alla quarta di rocco. & Re alla quarta di dama bianca.\\
Pedina d'alf.\ dà scacco.& Re prende la ped.\ di re.
\end{tabular}
}

Cav.\ dà scacco matto.

Dopo \markerC{} o pure così

\vspace{6pt}{\small
\noindent
\begin{tabular}{@{}p{3.84cm}p{3.84cm}}
Cav.\ dà scac.\ alla quarta di rocco.& Re alla quarta di cavallo di dama.\\
Pedina d'alfiere dà scacco. & Re prende la pedina di re. \\
\end{tabular}
}
Cavallo dà scacco matto alla 4 d'alf. di dama nera.

Dopo CROSS se il nero e difende così.

\vspace{6pt}{\small
\noindent
\begin{tabular}{@{}p{3.84cm}p{3.84cm}}
Alf.\ dà scac.\ alla 2.\ di dam.& Re prende il cavallo. \\
Ped.\ di cav di dama dà scac. & Re alla terza di rocco di dama bianca.\\
Dama dà scacco alla seconda di re nero. & Re alla seconda di cavallo di dama bianca.\\
Dama dà scacco alla 4.\ di re nero. & Re prende la pedina d'alfiere.\\
\end{tabular}
}

Rocco dà scacco matto.


Dopo *', se si difende cosi.

\vspace{6pt}{\small
\noindent
\begin{tabular}{@{}p{3.84cm}p{3.84cm}}
Dama dà scacco alla 4 di re nero. & Re alla terza di rocco di.dama bianca.\\
Alf.\ di dam.\ dà scac. in sua casa. & Re alla quarta di cav.\ di dama contraria. \\
\end{tabular}
}
La pedina dell'alf. dà scacco matto.



\subsubsection{Altra maniera di giuocare.}

{\small
\noindent
\begin{tabular}{@{}p{3.84cm}p{3.84cm}}
{\bfseries\scshape Bianco}.& {\bfseries\scshape Nero}.\\
Ped.\ del re due case. & Il medesimo.\\
Ped.\ dell'alfiere del re due case. &Pedina prende la pedina contraria.\\
Cavallo del re alla terza casa del suo alfiere. &Pedina del cavallo del re due case. \\
\end{tabular}

\noindent
\begin{tabular}{@{}p{3.84cm}p{3.84cm}}
Pedina del rocco del re due case. &Ped.\ del cav.\ del re alla 4.\ del cav.\ del re contr.\\
Cav.\ alla 4.\ del re contrario. &Ped.\ del roc.\ del re alla sua 4\\
Alf.\ del re alla 4.\ dell'alfiere della dama.&Cav. del re alla 3.\ case del rocco. \\
\end{tabular}

\noindent
\begin{tabular}{@{}p{3.84cm}p{3.84cm}}
Pedina della dama alla quarta.&Alf.\ del re alla seconda j casa del suo re.\\
Alf.\ della dama prende la ped.\ contr.\ alla 4.\ casa dell'alf.\ del suo re. &Alf.\ del re prende la ped.\ contraria, alla quarta casa del rocco del re, contr.\ e gli dà scacco. \\
Pedina del cavallo del re copre. & Alfiere del re alla quarta casa del suo cavallo.\\
\end{tabular}

\noindent
\begin{tabular}{@{}p{3.84cm}p{3.84cm}}
Roc.\ del re piglia la ped contraria alla 4.\ casa del del rocco contrario.&Alfiere del re prendes l'alfiere contrario:\\
Ped.\ prende l'alf contr alla 4.\ casa dell'alfiere del suo re. &Pedina dela: dama una casa.\\
Cav del re prende la pedina alla quarta casa.&Alf.\ della dama piglia il cav.\ contr.\\
\end{tabular}
\noindent
\begin{tabular}{@{}p{3.84cm}p{3.84cm}}
Dama piglia l'alfiere. &Cavallo piglia la dama.\\
Rocco piglia ia il rocco, e dà scacco. &Re alla sua seconda~ \\
Rocco piglia la dama. &Re piglia: il rocco. \\
\end{tabular}
\noindent
\begin{tabular}{@{}p{3.84cm}p{3.84cm}}
Alf.\ piglia la pedina, alla 2.\ casa dell'alf.\ del re contrario &Cavallo della dama alla terza case del suoi alfiere. \\
Pedina dell'alfiere della dama una casa. &Re alla fa Seconda \\
Alfiere alla terza casa del cav. della sua dama.&Cavallo, del Re alla terza casa del re contrario\\
Re alla seconda del suo alfiere. &Cav.\ ‘del Re dà scacco ‘alla 4casa del cav.\ det, re contrario.\\
\end{tabular}
}
Re alla terza casa del suo alfiere, e guadagnerà





\subsubsection{Altra maniera di giuocare.}

\vspace{6pt}{\small
\noindent
\begin{tabular}{@{}p{3.84cm}p{3.84cm}}
{\bfseries\scshape Bianco}.& {\bfseries\scshape Nero}.\\
Pedina del re due case. & Il medesimo.\\
Pedina dell'alfiere del re due case. & Pedina del Re prende la ped.\ dell'alfiere del re contrario. \\
\end{tabular}
\noindent
\begin{tabular}{@{}p{3.84cm}p{3.84cm}}
Cav.\ del re alla terza casa del suo alfiere. & Pedina del rocco del re una casa.\\
Alf.\ del re alla 4.\ casa dell'alfiere della sua dama. & Pedina del cavallo del re due case. \\
Pedina del rocco del re due case. & Pedina dell'alfiere del re una casa.\\
\end{tabular}
\noindent
\begin{tabular}{@{}p{3.84cm}p{3.84cm}}
Cav.\ del re pren. la ped.\ del cav.\ del re contrar. & Dama alla seconda casa del suo re.\\
Dama dà scacco alla 4.\ casa del roc.\ del re contr. & Re alla casa della sua dama\\
Cav.\ dà scacco alla 2.\ casa dell'alf.\ del re contrar.& Re alle sua casa.\\
\end{tabular}
\noindent
\begin{tabular}{@{}p{3.84cm}p{3.84cm}}
Cav.\ del re pren. il roc.\ del re contr.\ e dà scacco della sua dama. & Re alla casa della sua dama\\
\end{tabular}
\noindent
\begin{tabular}{@{}p{3.84cm}p{3.84cm}}
Cav.\ dà scac.\ alla 2.\ casa dell'alf.\ del re contr. & Re alle sua casa.\\
Cav.\ prende la ped.\ del rocco del re contr. e dà scacco alla sua dama. & Re alla casa della sua dama
\end{tabular}
}
Cavallo prende il cavallo del re contrario, e guadagnerà.

\subsubsection{Altra maniera di giuocare.}

{\small
\noindent
\begin{tabular}{@{}p{3.84cm}p{3.84cm}}
 {\bfseries\scshape Bianco}. & {\bfseries\scshape Nero}.\\
 Pedina del re due case.& Il medesimo\\
Pedina dell'alfiere del re due case. & Pedina del re piglia la pedina dell'alfiere del re contrario.\\
Cav.\ del re alla terza casa del suo alfiere. &Pedina del rocco del re, una casa.\\
\end{tabular}

\noindent\begin{tabular}{@{}p{3.84cm}p{3.84cm}}
Alfiere del re alla quarta casa dell'alfiere della sua dama. & Pedina del cavallo del re, due case.\\
Pedina del rocco del re, due case. & Pedina del cavallo del re, una casa. \\
Cav.\ del re alla quarta casa del re contrario. & Rocco del re alle sue due case. \\
\end{tabular}

\noindent\begin{tabular}{@{}p{3.84cm}p{3.84cm}}
Ped.\ della dam.\ 2 case.& Ped.\ della dam.\ 1 casa. \\
Cav.\ del re alla terza casa della sua dama & Pedina 1 casa alla 3.\ casa dell'alf.\ del re contr.\\
Pedina del cavallo del re, una casa. & Dama alla seconda casa del suo re.\ \\
\end{tabular}

\noindent\begin{tabular}{@{}p{3.84cm}p{3.84cm}}
Cavallo del re alla quarta casa del suo alfiere. & Dama piglia la ped.\ del re contr. e dà scacco. \\
Re alla seconda casa del suo alfiere. & Dama alla terza casa del suo alfiere. \\
Dama alla sua terza casa. & Rocco alla seconda casa del cavallo del suo re. \\
\end{tabular}
}

L'alfiere del re alla quarta casa del cavallo del re
contrario, sopra la dama contraria, piglierà la
dama contraria, e vincerà, e tanto basti di questo giuoco aver detto.


\supersection{Gioconde riflessioni su varie cose utili
al comercio umano.}

\subsubsection{PER LA BILANCIA.}

Mi sembrano utili le seguenti notizie, perciò
non istimo inconveniente lo esporle. Siccome
da me l'anno scorso fu osservato nella Binomica,
che i termini della progressione Geometrica dupla
possono esprimere qualsisia numero, dalla qual notizia
tutto quel calcolo ne dedussi. Cosi ora osservo,
che nella progressione Geometrica tripla con i soli
suoi termini simpli, altri sommati, e altri
sottratti, parimenti si possono esprimere tutti i
numeri, che si vogliono sotto la somma di tutti li detti
termini: v.g.\ con i termini 1.\ 3.\ 9.\ 27.\ 81.\ si
potranno esprimere tutti i numeri da 1 fino a 121.\ somma
de i sudetti termini. Così, se vorrò esprimere 74.
farò 81-9+3-1=74 e se vorrò esprimere 63 farò
81-27+9=63, e così degli altri. Un uso preclaro di
questa osservazione si è, che in una bilancia senza
tenere tanti contrapesi, con pochissimi si possono
pesare quante libre si vogliono perché da tutta la somma
si leva il simplo, levando il peso, e si leva il doppio,
levando il triplo da una parte, e mettendo il simplo dall'altra; per
altro con 4 pesi si pesano da 1 fino a libre 40
\begin{tabular}{l@{ }c@{ }l@{ }r@{ }l}
con 5 pesi &—& da 1 fino a libre& 121\\
con 6 pesi &—& da 1 fino a libre& 364\\
con 7 pesi &—& da 1 fino a libre& 1093\\
con 8 pesi &—& da 1 fino a libre& 3280&ecc.
\end{tabular}

Quello, che si dice del peso in libre nelle bilancie
grandi, intendasi ancora di altre bilancette minori,
e de i seggiatori a grani.

Se
volete fapere il giusto peso d'una cosa, e non
avete altra bilancia, che una dolosa, pesatela prima
da una parte, e notate condiligenza il peso, che
mostra, poi pesatela dall'altra parte, e notate
parimente il peso, che mostra; ambedue questi pesi
saranno falsi. Il peso medio geometrico proporzionale
tra questi due sarà il vero peso di detta cosa;
cioè si moltiplichino insieme li due pesi falsi, dal
prodotto si estragga la radice, che sarà il giusto peso
di detta cosa.

\subsubsection{CONTRATTI}

Giova osservare in materia di contratti varje
cose, che a primo aspetto pajono impossibili, e
pure riescono, v g. Tre Mercanti di cavalli vanno in
fiera, il primo vi reca 10.\ cavalli, il secondo 30, il
terzo 50. Ognuno di loro vende i suoi al prezzo
corrente in fiera, la sera si trovano ogn'uno di loro
avere fatta l'istessa somma di moneta. Così. La mattina
il prezzo corrente fu 7 per cento, il primo fece
100.\ scudi, e gli avanzarono tre cavalli. Il secondo
fece 400.\ scudi e gli avanzarono due cavalli. Il terzo
fece 700.\ scudi, e gli avanzò un sol cavallo. Verso
il tardi si mutò il prezzo dei cavalli, e si venderono
300.\ scudi l'uno. Dunque tutti e tre i Mercanti
portarono a casa 1000.\ scudi all'istessa fiera con tanta
disparità di mercanzia.

Un Mercante da a due suoi figli mille scudi a
testa, accioché vadino in fiera a trafficarli. Il primo
compra vitelli a 10.\ per cento, poi va in altra fiera,
ed essendosi dimagrati, bisognò rivenderli a 20.\ per
cento. L'altro ebbe miglior forte, ed in altro paese
compro i Poledri a 20.\ per cento, e li rivendé a 10.
per cento: Si dimanda, che negozio fece il padre di
costoro? Pare, che non ne dovesse avere né utile,
né discapito; e pure vi guadagnò scudi 500.\ non ho
proposto a veruno questi casi, che non abbia mal
giudicato prima di considerarli, onde avverto con
Gellio: \textit{Sapiens sermones suos praecogitat,
 \& examinat prius in pectore, quam proferat in ore}.

\subsubsection{MISURE.}

Con due vasi, uno di cinque metrete, ed uno di
tre si può dividere per metà un vaso di balsamo
di 8 metrete. Così: empiasi quello di tre, e si voti in
quello di cinque, e di nuovo si empia quello di 3, e
si voti, finché è pieno quello di 5, questo si rimetta
in quello di 8 e vi si metta quell'una, che sta in
quella di tre; poi si riempia quello di 3, in quello
di 8.\ ne resteranno 4.

E se le misure fossero state, una di 5, ed una
di 11, e il vaso pieno di 16, e pur si dovesse divider
per metà. Si empia tre volte quella di 5, e si voti in
quella di 11, questa piena si voti in quella di 16, e vi
si mettino le 4, poi due volte si empia quella di 5,
e si metta in quella di 11., resteranno 3 in quella
di 5 che posti in quella di 11 prima votata in quella
di 16, riempiendosi poi quella di 5, resteranno 8 in
quella di 16:

Tre Cacciatori trovarono in un ripostino 33
vasi d'oro, 11 pieni di un liquore prezioso, 11
pieni per metà, ed 11 voti. Per dividerli egualmente
senza votarli. Si divide 11 in tre parti, ciascuna minore
della metà, che si può fare in quattro maniere,
perciò il quesito ha quattro soluzioni, prima 1.5.5.\
seconda 2.4.5.\ terza 3.4.4.\ quarta 3.3.5.\ questi
numeri indicano le piene, e le vote di ciascuno:
dunque nel primo caso il primo avra una piena, una
vota e nove mezze; Il secondo, e il terzo cinque
piene, cinque vote, e una mezza ecc., ed in altri
simiglianti casi si farà sempre l'istesso.

Un Pastore ha da condurre 90 pecore per 30
macelli a 30 per viaggio, e ad ogni macello ha da
lasciare una pecora. Si dimanda, se al fine può farne
restare alcune per sé. Si risponde, che per se ne
può fare restare 25. Prima a 30 a 30 le conduca a 10
macelli, e poi a 15 macelli, e poi agli altri 5 ne
resteranno 25 per se.

\subsubsection{PERIODO GIULIANO.}

Dati gli anni dell'Indizione, delli Cicli Solari,
e Lunare, trovare l'anno del Periodo Giuliano.
Si moltiplichi il Ciclo Solare corrente per 4845,
della Luna per 4200., dell'Indizione per 6916., la
somma di questi prodotti si divida per 7980, il
residuo farà l'anno del Periodo.

Il Periodo Dionisiano e il prodotto del Ciclo
Solare, e Lunare 532. Dati i numeri dunque de i
sudetti Cicli dell'anno corrente, si troverà l'anno
del Periodo, se si moltiplicheranno il numero
Lumare dato per 479, e il Solare per 57, e la somma si
dividerà per 532.

I detti numeri moltiplicatori sono i minimi
moltiplici di uno, diviso per l'altro che avanza 1.

E nel Periodo Giuliano i numeri moltiplicatori
sono i minimi moltiplici di due, che divisi per il
terzo avanza 1.

Per trovare il numero Aureo, l'Indizione, e il
numero Solare corrente, si divida per essi l'anno
corente, agli avanzi s'aggiunga 1, si averà l'Aureo
numero, s'aggiunga 3, s'averà l'Indizione, e se si
aggiunge 9, s'averà il numero Solare, o la lettera
Domenicale per tutto questo secolo.

Per trovare l'Epatta in questo secolo, si moltiplichi
l'aureo numero dell'anno corrente per 11, e
si divida poi per trenta il prodotto, all'avanzo si levi
11, e s'averà l'epatta corrente.

Un Cavaliere da una egual somma di dobloni a
tre povere Donne, con patto che ne distribuisca
ognuna egualmente per dote alle sue figliuole, e se
qualche cosa gli avanza, sia suo. La prima aveva
tre figliuole, li divide, e per se ne avanzarono due.
La seconda aveva cinque figliuole, e per se ne avanzò
uno; e la terza aveva sette figliuole, e per se ne
avanzo pur'uno. Si dimanda quanti dobloni regalò
il Cavaliere ad ognuna di esse?

Si moltiplichi il primo avanzo per 70, il secondo
per 21, il terzo per 15. Si sommino i prodotti, la
somma si divida per 105, e l'avanzo di questa divisione
sarà il numero de i dobloni, che il Cavaliere
diede per limosina a ciascuna delle sudette donne.
Il perché di questa operazione si cerchi nella mia
Arimmetica Speciosa c.\ 3.\ a.\ 2.

Se saravvi chi avrà piacere di biasimare, o tutto.
o parte di questo Libretto, o l'autore, sappia,
che l'uno ingegna, e l'altro prattica il precetto di
Catone.

\literaryquote{Cum recte vivas, ne cures verba malorum.\\
Arbitrii nostri non est quid quisque loquatur.}

Direi mille bagattelle di più, se qui non fosse

{\centering\LARGE\textbf{\textls[200]{IL FINE}}.\\}

\end{document}
