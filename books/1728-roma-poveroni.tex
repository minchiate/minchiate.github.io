 

 
 

REGOLE GENERALI

Del Nobiliffimo Gioco
Doo hioL eh:

MINCHIATE

Con un modo breve, e facile
per ben imparare a giocarlo.

 

 

 

In Roma, Per Raffaelle Peveroni. 1728,

 

 

 

 

|
| Coa licenga de’ Superiors « |
 

 

 

 

 

 

 
-Cortefe Lettore

Leveder girare que-
| to picciolo Libric-
Na ciolo fotto di un ti-
P™ solo cost _[peciofo di
Regole generali del Gioco
delle Minchiate ; fon ficuros
benigno Lettore 5 che ti reche-
ra qualche forte di maraviglia:
prima perche hai vedato altre
wolte flampate quefte Regole.

waza alcun frutto , ¢ poe per
che bai trovato rn ogni conver=

azione ove prattichi,che opin
uno gioca quesie gioco a capric-
cio, econregole diver fe, € per
effer bravo Giocutore bafta af-
ferire con franchezxa la prat=

ica delle altre conver/a zioni
A a ate

f

 

 

 

 

 
‘A. ‘
ancorche non vera; poco impor

tando,che quefta prattica degea
nevi in abufo; e ficcome queffo
é flatol'unico motivo, che hd
havuto in flamparle per levaye
da mezzp tante prattiche, e
tanti abufi: volendo dare una
regola generale, certa, e fifa
per tutte le converfarsoni, cost
pero , che faprai compatire 7]
mio ardire, quale per render]e
accreditate bo pretefo ancorg
di renderne la ragione, perche
devono tutte effere pratticate
uel modo da noi prefcritto; m3
fe poi quefle ragioni da noz
addotte non foffero tal’ , che
poteffero capacitarti 2 baflan-
za: fappi che io mi contento
di moderarte jn quella partes
E ove

Se
o
ove foffero defettofe, purche ta
complacct direndermt una ra-
giene maggiore, la quale faccia
coftare effere falfa la prima 5
mentre ioche #0n hd mai avuta
altra idea nel fare quefta pic-
ciola raccolta , che diftruggere
ogni abuso , per publico benefi-
cio,e rendere guefto Gioco fen=
za competenza di prattica egua~
le in tutte le conver [azioni, mé
fapro fottofcrivere dibuonavo-
glia al two faggio,e miglior pa~
rere ogni qual volta lo conofee-
v0 fondato si la ragione. Del
refio vivi felice .
  

49, al male per gevio, e
Nosema @ per natura, che ftima
debolezza di fpirito fignorile , -
evilta d’animo nobile Pintra-
prendere per {uo divertimento
qualche qnorato impiego , 4 fe-
gno tale, che vuole pit tofto
marcire nell’ ozio per averes
campo di poterfi liberamente
dare in preda ad ogni forte di
vizio che fe gli faccia incontro,
che fottometterfi a qualche lég-
giera, e virtuofa fatica,creden-
do ftoltamente di non poterf
prendere altro divertimento
pil conveniente alla loro nobi-,

A 4 le

iy

 

 

 

 
 

8
le condizione di quello lisa fug-
gerire un mal configliato cap-
priccio 3 Onde la vediamo gior-
nalmente pratticare baldanzofa
tutti li ridotti, e frequentare
tutti li poftriboli, ove il gioco
facendola da gran capitano tira
a fe tutta la Gioventu vagabon-

da per condurla a militare fotte

il ftendardo della difperazione ,
¢ con la fperanza d’un impofi-
bile guadagno li conduce ad un
ftata miferabiliflimo fenza ri.
paro.

Tralafcio le frodi, che quivi
fempre fi meditano; Pinganni,
che fempre fi tentano con pre-
giudizio anche notabile del pro-
ptio carattere ; tralafcio le be-
flemic, che quivi continuamen-
te fi fentono, le queftionl, che fi
fomentano, livituperj, che fi
Pratticano, e dird folo, che qui-

vi
: 2
viregnano continue lecrapole:,.
nelle quali fanno folo macftofa

pompa: la prodigalita, Pintem-

peranza,.e Pubriachezza madri
feconde ditutte le fceleratezze
pid abominevoli,le quali ma
fcherate col nome digenerofita
grandiofa, dé {piritofa fortezzay
e di animofa gagliardia ftrafci~
nano a viva forza la Gioventt in

un precipitofelaberinto- di vizjj.

a fegno tale, che non sa pit co-=
nofcere il bene, non sa pit di-
ftinguer ragione , ¢ fe qualche
wero, @fedele amico., amante
dell’onor fuo,procura con bella

forma.di farle concepire l’erros-

re ,. i cub abbagliato fen vive,
fi fa reo.d’infedelta-, ¢ colpevo-
le di zelo indifcreto , quafi.che
voglia impedirle li fuorgiova-
nili paflatempi,non confideran-

do. altro fpaffo per loro., che.

Ag quek
oe Fe ee EE

10
quello gli sa fuggerire un mal
configliato capriccio.

Onde io per vedere di difto-
gliere la nobile Gioventt ims 4
qualche parte da cotanti abo-
minevoli vizj mi fono lufing a-
to’ di poter loro impedire una
firada cotanto precipitofa coll’
allettamento d’un folo vizio ,
quale per altro abbia in fe fteffa,
¢ del dilettevole,e del virtuofo,,
accioche fodisfacendo in quetta,
forma alla loro mala inclina.
zione pofla bel bello andarg
fcordando ogn’altro vizio peg—
giore.

Non vie dubbio, che tutte
le novita. rifvegliano nel?ani- ¢
mode’Giovaniun nonsochedi _
piacevole, che fortemente gli
ftimola per prefto condurli ad
una perfetta cognizione di quel-
1a novita ; cosi mi Infingo 5 che

i per

ws

gcse ee ee i Ree:
 

It

er effere quefto vizio non ha
1olto introdotto per tutte le
onverfazioni poffa con facilita
1overe l’animi della Gioventi
d applicarvi con tutto. il fer-
ore, econ tutta lattenzione
er ben impoffeflarfene ;, tanto
iu che fembra degno. di ginfta.
jprenfione colui, che pretende:
li pratticare tutte le nobilt con=
rerfazioni fenza fapere quefto
jizio virtuofa..

E’ quetto il nobiliffimo., edi-
ettevole gioco delleMinchiate,,
sotanto pratticato in tutte le
converfazioni, quale per altro,
yorta in fronte Vorribile nome:
li gioco, caggione d’ogn’altra
(celeratezza , ma racchiude pe-
ro. in fe tea un compleffo di
virti, quale lo rende amabile ,
guftofo, ¢ piacevole, perches
oltre Vimparare l’aritmeticas

A 6 im-
Lg
smpara ancora la politica ,. la
prudenza »e economia ; onde
per non avere db viziofo, che jl
folo nome , pare che poffa me-
ritar con ginuftizia tutta Patten-
zioneé della Gioventi:, mentre
€ maggiore il vantaggio,cheé
per apportare di quello.non fia
il danno, fe ne poffa temere,

Figuratevi in quefto gioco di
vedere. due inimici a fronte,
quali con Parmi alla mano.,e
coir la rabbia ful cuore vanno
avventandofi vicendevolmente
replicati colpi di {pada per ca-
varfidalle vene lun Valtro col
fangue la vita; cosi appunto il
Giocatore delle Minchiate ayi- .
do anch’egli della vittoria, va
€on bell’ arte machinando fe-
crete infidie al fuo caro inimi-
Co per farle perdere quella car-
ta donore , che troppo —

“ i:

 
i
di pregindizio a fe fteffo fe no
la perdefle, e con bello, e polis
tico ftudio procura cavare ik
fangue dalla borfa del fuo ini-
mico, fenza pero che egline
refti offefo; ma folo lafcia in lui
un bel defiderio delka vendetta,
perloche crefce in lui lo ftimo-
Jodi andare piu guardingo nei
cimenti, e quando fe glifa in=
controuna dura neceflita di re=
ftare perdente, deve almeno
con tutta Peconomia: poflibile
procurare il fuo minor male,
perche fembra anche {pecie di
vittoria rl faper refiftere con
animo generofo alle avverfita
jnevitabili , mentre tutta la va-
ghezza di quefto gioco, confifte
in fapere con prudenza appro-
fittarfi del (uo maggior vantag-
gio, 0 pure Vevitar il fuo mag-
gior male con daria i di-
Elle
 

 

14

fenderfi dall’infidie fecrete d'un
inimico fcoperto, e con bella
forma fempre renderfi a lui fu-

periore in ogni cimento .
Attefo dunque le cost belle
qualita di quefto gioco parm »
che il faperlo a perfezione pofla
eflere d’un gran vantaggio alla —
nobile Gienenti, qual’e inoltre
necefatata 4 ben faper contate »
fommare , € fottrare 5 pertanto
ad effetto,che pofla ogn’une.con
tutta facilita impoffe flarfene dk
buona voglia intraprendo que-
fta fatica 5 defcrivendo. prima»
tutte le regole generali, che
pofiono ben ammaeftrarvi 10
quefio gioco, © pol andaremo
bel bello difcorrendo del modo
di pratticarle. Tocchera pet?
avoia ben ftudiarle per potere
-poicon tutta facilita ridurle
buona prattica; effendo. che 14

; Teo-
15

Teorica da per fe fola non é ca*
pace a rendervene informatia
pieno: attefe le grandi appendi-
ci, che anno tutte le regole ge-
nerali, e quefte non puole impa-
rarle altri, che una diligente »
& efatta prattica, mentre nd-
{cono da foli accidenti, che con-
tinuamente diverfi accadono.in
quefto gioco 5 onde per ben im-
parare quefto sioco fa di meftie~
reunire alla Teorica la pratti-
ca, la quale come ailai pitt bra~
va maeftra fapra meglio darvi
ad intendere quello, che non sa
fpiegarvi la Feorica.

Quefto gioco trae la fua pri~
ma originedalla nobiliflimaCit-
ti di Firenze inventato con bel-'
lo ftudio pert un piacevole am-
maeftramento della nobile Gio-
yentw nell’ Aritmetica, ma le al-
tre vistu,che Padornano Phanno

refo:

 

 
 

16 -
refo cosi univerfale , che ormai
fembra difetto di curto intendi-
mento il non faperlo, mentre
fi fa reo d’incapacita chi ne trax
fcura il confeguimento ,. folo vi
avverto pero,che per ben gioca=
xe a quefto gioco non bafta il
conofcere le-carte,ed il giocarlo
perche qnefto é ftato ginfto if
motivo ,,che mi hamoflo.a de-
fcrivere le regole generali di
quefto gioco, effendo che in.»
ogni converfazione dove anda~
vo,. trovavo regole: diverfe,e¢
talvolta alcuno fi faceva leci-
to ne’ cafi, dubj: afferire con
franchezza per ferma-refoluzio-
ne cid,.che ne-meno.¢ da met-
terfi indubio, e fe poi glie ne
dimandavate laragione , non»
fapevarifpondere altro,foloche
cost fi prattica in altre conver~
fazioni, fenza riflettere» che
ognt

ak ot mM RO Lt et ett a Of tie feed eh, ih it Oe CI IA ee OA kt OCMC COPS
rp
yeni converfazione fi regolas
on regole diverfe , & & capric-
io, ed il giocare fenza faper
‘endere la ragione del {uo mo-
lo di giocare, oltre l’effere re~
ugnante atutie le nature de’
siochi , & anche contro Vinten~
ione dell’ Autore,. quale ha pre-
efo nell’ inventare quefto gioco:
li ftabilire regole certe, filfe, &
mmutabili, le quali doveffera
empre fervire per tutti , fenzas
afciare in arbitrio de’ giocatork
] variarle 4 loro capriccio. On-
Je per togliere da mezzo quefti
ibufi communemente praticati
n moltiflime converfazioni ab-
siamo ftimato neceffario defcri-
rervi tutte le regole generali, —
\ccio. ch’ogn’uno pola reftarne
‘apace , eregolarfi in avvenire:
ie] modo, che diremo in ap-.
oueflo.. Intanto vi bafti re
cne
 

 

 

 

 

 

 

 

19

che quefto gioco fi fa con 9%.
carte, fra qualive ne fono 24
d’onore,0 vogliandire che con-
tano, ¢l’altre tutte non col
tano niente.
Quefto gioco tutto che abbia
un folo nome puol perd giocat#
in pid modi tutti diverfi, men
tre fi puol giocare 4 ogn’uno 04
per fe folo, e fi puol giocare con
il compagno adue per due;
pitt fi puol giocare,in due,in tres
ed in quattro, e quando fi gioc®
in meno di quattro, allora fi 0°
ca ad ogn’uno per fe, fe bene®
puolanche giocare in quatttos
ad ogn’ano per fe, ma quando ®
g10ca in quattro il vero giocate
€giocar con il compagno: ,.
Quando fi gioca in mene di
quattro fidanno fempre 25. CaF
te per ogn’uno, e quando fi 810°

ca in quattro fe ne danno fola-
 men-
19
mente 21.con che pero l’ultima
deve fempre darfi fcoperta a

tutti,e fe quefta fara carta d’ho-.

nore quello, a cui fara toccata
fegnera tanti puntia fuo favore
quanto importa la carta fco-
perta, fe poi non € carta che»
conti, allora non fi fegna nien-
te; dopo che tutti li Giocatori
averanno ayute tutte le fue car-
te , quelle che reftano fono fuo-
ri digioco, enon fi confidera-
no cofa alcuna in quel gioco, ne
é lecito ad alcuno de’Giocatori
il poterle vedere.

Quando fi'gioca ad ogn’uno

per fe quello che ha cattive car-
te puole non giocare in quella
mano, ¢ gettare le fue carte ab
monte, € pagare alcuni punti
a quello, che fa Vultima mano,
fecondo quello, che convengo~
‘ho, € fe bene pero lui non gioca

In
 

 

20
1D quella mano rifcuote la pena
dell’onori, che fi perdono da
chili perde giocando; H folito
quello, che ha la mano 5. fe non
vuole giocare paga 20. puntl, il
fecondo ne paga 39. il terZzo, ed
il quarto.ne pagano 4o.

Queflo modo di giocare, fe
bene per fe fteffo é piu bello»
perche ¢ pill difficile giocandofi
‘totalmente all’ofcuro ; nulla di
meno fembra fmezzato, perche
in quelle poche carte, che refta-
no fuori del gioco vi poflone
effere le migliori carte del gio-
co, e quelle, che poffono fare
tutto il gioco perd ¢ ftato in-
ventato dopo il gioco della fo-
la , quale porta feco per necefii-
ta, che tutte le carte d’Onoré
fiano in gioco, € quefto non

ftato fatto ad altro fine, che pe

rendere il detto gioco pill tach
le,e piivago. — Fo-
at

Fola vuol dire, che tutte le
carte d’onore fiano in gioco a4
{egno tale, che non puole reita-
re fuori del gioco una carta»
@onore,benche minima, e quel-
le, che reftanonell’ultime car~
te deve tutte pigliarfele per fe»
quello, che fa le carte; ma 58,
che voi mi rifponderete, come
e poflibile , che fola vogtlia di-
re, che tutte le carte d’onores
fiano in gioco, fe fecondo quel-
lo voi avete imparato poc’anzi
puole darfi ’accidente, che an-~
che giocando con Ja fola tutte
le carte diconto non fiano in,
gioco; mentre fe fi gioca a ogn?
une perfe, ancorche fi giochi
con lafola qualched’un de’Gio.
catori puole gettare le fue car.
te al monte, fe bene avetle jp
mano dell’onori; onde in quel
cafo reftarebbero fuori del Zio-
-O Carte d’onore,. Tous

 
 

 

 

 

 

 

 

22

lo pero vi rifpondo,che ever
soquello, che vi ho imparato 4
principio; ma non per quefto &
vero , che reftino fuori del gio-
co carte d’onore , quando uno
de’ Giocatori dice di non volet
gviocare , € gettale fuc carte al
monte, perche quel Giocatores
che getta le fue carte al monte »
dice di non voler giocare ¢
obligato a pagare quella pends —
che gl’é ftata aflegnata in prin
cipio, qual pena vain compen-
{azione delle carte d’onore, che
puole avere in mano : mentre
none¢ credibile,che uno,il quale
abbia in mano tante carte d’ono-
re, quanto puole importare. a
fua pena voglia fenza rifico Vo
runo pagare la pena certa, @
non giocare 5 onde ecco, che la
pena viene a compenfaré que

danno,che puole caggionarli ann,
quels
23
quelle poche carte d’onore po-
fle al monte da chi non vuol
giocare,ficche non puole giufta-
mente dirfi in queito cafoy che
fiano fuori del gioco cartes
donore, perche quelle vengo-
no confiderate in gioco mediane
tela pena,perloche reftera fem-
pre vera la propofizione, che
fola vuol dire, che tutte le car-
te d’onore devono fempre effe-
re in gioco, ne puole niuno get;
tarle al monte, fe non in cafo
ne paghi il compenfo,quale non
fi deve permettere , fe non &
quelli, che hanno la facolta di
gettare tutte le fue carte al
monte , € non giocare in quella
mano; ma quando non fi poflo-
no gettare tutte le fue carte al

monte ; allora ne meno élecito.

il gettarne alcuna fola d’onore,
perche allora farebbe contro if
Z10»

 

 
 

24 :
gioco, e€ potrebbe portare del
gran pregiudizio, come diremo
in appreffo.

Hora poi che havete impara-~
to che cofavoglia dir fola, do-
vete imparare 4 conofcere tut-
tele carte d’onore , e dopo che
averete ben apprefa la cogni-
zione delle medefime allora do-
verete fare lo ftudio delle ver-
zicole, quali vengono compofte
con fole carte d’onore,ne puole
mai eflere verzicola, fe non yj
concorrono almene tre carte
d’onore; per ben conofcere tut-
tele carte d’onore, é necefla~
rio ancora conofcere quelle,che
non fono @onore per poterle
diftinguere, ¢ pero dovete {ape-
re che tutte le carte fidividono
parte inTarochi,e parte in Car-
tiglia, la cartiglia po! fi divide

in quattro fpecic deter cioe
pa-

 
{pada, baftoni, coppe, e denaroy
ed ognuna di quefte quattro-{pe -
cl€ € compofta di quattordici
carte, quali tutte non contano
cofa alcuna, ad efclufione perd
del? ultima, che fi chiama com-
munemente il Re, quale contas
cingque, ed.é carta d’onore, nel
eiocare poi dette carte, la mag-
piore prende fempre la minore ;
ad efclufione pero.di coppe, e
denari;-mentre la minore pren-
le lamaggiore, quando perd non
fino carte; figurate, effendo che
le figurate precedono femprey
alle numerate, ele numerate
folo.anno il privilegio di prece-
dere alle maggiori dinumero.
Li Tarochi fono quaranta, e
col Matto fono quarantuno,fino
al trentacinque fono tutti nu-
merati, ed i] maggiore precede
mprefe al minore; dopo il trens
oe B ta-
 

 

 

 

 

tacingue vi fono altre cinque
carte, le quali fi chiamano ari¢ »
€ non fono numerate ; ma fi di-
ftinguono peré dalla loro repré-
fentazione , mentre la pid infe-
riore rapprefenta la Stella, 1a
feconda la Luna,la terza il Soles
laquartail Mondo , laquinta le

 

—a

Trombe. ; ¢ quefte fono le mag*

giort carte del gioco. In oltre
vie un’altra carta,la quale é fen-

zanumero » € non fa neflunas '

figura , ¢ fichiama il Matto »

i quale all’ofanza de’ Matti fa
quel che yuol¢ ; ma pero é infe>
riore , € tutte le altre carte tut-

to che fiano carte d’onore,men=

tre Ini non puole mai pigharé
neffuna carta .

Le carte che contano fond
dalf’uno fino al cinque inca
ve, il dieci, il tredict » ed il ven”
ti, ¢ dal yentiotto fino ie

3 ; yall

y
 

27
ranta efclufene, il yentinove
quale|non conta, fe non in cafo
che faccia verzicola, tré di ques
fte come dicemmo di fopra fan-
no verzicolas ma per effere vers
zicola’é neceflario, che fiano
confecutive.,. come uno, due ,e
tre, 0 pure due,tre, e quattro, 6
pure 28, 29.30. © cosi di tutte
Paltre in appreffo fino al. qua-
tanta ; {¢ poi foflero quattro, 6
cinque carte: confecutive tutte;
fanno verzicola,ma fe non fono
pero confecutive non poffonoe
fare verzicola oltre di tuttes
quefte vetzicole, ve ne fono del-
le-altreycome unMatto;e Trom=
ba, uno , tredeci, € ventiotto,
dieci; venti trenta , ¢ quaranta,
ma per effere verzicola devono
fempre effere confecutive, di

“pit tre Re'ancora fanno vérzj-
cola , il Matto poi entra in tuta
Ba te

 
28
te le verzicole, fe bene da per f€
folo non fa altra verzicola che
uno Matto,e Tromba. 9): ? °

Tutte quefte carte che fanno
yerzicola contanto cinque©
ponti per ognuna ,ed efclufione
delli quattro Papi, .che contano
folamente tré, ¢ le cinque ari€
che contano dieci :per-ognuta >
Il vehtinove conta folamente
Cinque, quando fa verzicola, al=
trimenti non conta’ niente V1
fono alcuni che fanno ancora. —
la verzicola diMondo,Carne,¢ |
Diavolo, ed:in quel cafo quando —
i] Diavolo fa-vetzicola conta)
cinque sancor eflos altriment!
non conta niente. 1 « ¢"!

Le verzicole: ficontano tf
yolte, cioe una prima di comity

ciave 4 giocare, maé pero is
a

 

‘che ‘bifogna anche mow a.
prima dicominciare 4 a i

 
/ 29
ne fi puolé pit moftrare dopo
d‘haver giocata la prima carta ,
né {i deve fupporre moftrata, 6
per averla rubata ,6 per averla
trovata nella fola; ma bifogna
neceflariamente moftrarla pri-
ma di giocare chi vuole con-
tarla ,-alla fine pot-del gioco fi
conta di nuovo’ due | volte,
quando: pers. fi fia fatta .
Quando havrete ben ap
prefe tutte quefte cofe come ne-
eeffariiffime::. allora comincia=
rete avenire in qualche cogni+
zione diquefto gioco, quale fi fa
nella forma, che diremo adef=
~ fo;quefto gioco deve giocarfi in
quattro perfone a due per due,
€ primieramente fideve venire
al’elezione de’ Compagni, es
quando li Giocatori non fi ac-
cordano fra diloro , allora de.
ve rimetterfi Pelezione alla for.

Bg ae

 
 

 

 

 

eS

30
te, fatta che fi fara l’elezion®»

de iCompagni,.comincera D0 —

de’ Giocatori 4 mefcolare ben
bene tutte le carte affieme » ¢
poi lafcierd ad arbitrio dellas
parte Palzarle , quale dovra
riconofcere l’ultima carta alza@

ta, e fe fara. carta d’onore, 0 {o>

praventi,allora dovra pigliarfe
la per fe,,¢ pigliera tutte quelle,

che ritrovera,¢ fe faranno car~ —

te d’onore , dovra fegnare a fud
favore tanti punti,quanti ne im-

porteranno: le carte d’onore al-

zate, perche finoa tanto, che

‘trovera fotto la carta alzata, 9

carta d’onore, 6 fopraventi, tur
te dovra prenderfele per {e,€

{e fi daffe l’accidente, che quer”
lo, che alza rubbaffe per fe piu
di quattordici carte, 4 fegn? ta
le, che non vi reftaflero pil cal”
te d (afficienza pet tutti a

 
31
catori , come dovra regolarfi
Pultime? in quel cafo,non effen-
co dovere, che egli non abbias
le fue carte giufte, né effendo
dovere, che quello, che alza
tralafci di pigliare le carte d’
onore che trova nell’ alzata. In
quel cafo per altro impoffibiles
deve regolarfi il gioco in quella
conformita , che fi regolarebbe
fe uno face ffe le carte fenza me-
{colarvile carte della folla, es
Pav verfario rubaffe tre, 6 quat-
tro carte; mentre dovrebbes
finire di prendere il fuo compi-
mento delle carte da quelles
carte, che fono rimafte fuora
del gioco,poco importando,che
foflero cartaccie ; mentre bafta,
che non fia fuori del gioco car-
ta d’onore per effere obligato if
gioco a tirare innanzi,e ne inco]-
pi la fna trafeuraggine fe non

4 ha

 

 
zy

‘trovera fotto la carta alzata 916

30
te, fatta che fi fara Pelezionés
de i Compagni ,. comincera uno
de’ Giocatori 4 mefcolare ben
bene tutte le carte affieme se
poi lafciera ad arbitrio della;
parte Valzarle , quale dovry
riconofcere ultima carta alza-
ta, e fe fara carta d’onore, 6 {os
ptaventi,allora dovra pigliarfe.
la per fe, ¢ pigliera tutte quelle,
che ritrovera ,¢ fe faranno Cars
te d’onore , dovra fegnare 4 fuo
favore tanti punti,quanti ne im-
porteranno: le carte d’onore ay,
zate, perche finoa tanto, che
0
carta d’onore, 0 fopraventi, tut-
te dovra prenderfele per fe, «
fe fi daffe l’accidente, che quel-
lo, chealzarubbafle per fe pia
di quattordici carte, a fegno tas
le, che non vireftaflero pid cars
te a {ufficienza per tutti li Gio-
ca-
 

Oe

31
eatori , come dovra regolarfi
Pultime? in quel cafo,non effen-
co dovere, che egli non abbias
le fue carte giufte, né effendo
dovere, che quello, che alza
tralafci di pigliare le carte d’
onore chetrova nell’alzata. In
quel cafo per altro impoflibiles
deve regolarfiil gioco in quella
conformita , che fi regolarebbe
fe uno faceffe le carte fenza me-
{colarvile carte della folla, es
Vavverfario rubafle tre, 0 quat-
tro carte; mentre dovrebbes
finire diprendere il flo compi-
mento delle carte da quelles
carte, che fono rimafte fuora
del gioco,poco importando,che
follero cartaccie ; mentre bafta,
che non fia fuori del gioco car-
ta d’onore per effere obligato il
gioco a tirare innanzi,e ne incol-
, pi la fua trafeuraggine fe non
fe B 4. ha
 

 

 

 

 

32

ha tatto bene te carte, ¢ fe ha
lafciato fuori del gioco nel me-
{colar le carte della fola at
tecedente, cosi parimente ne
cafo , che chi alza ruballe pill di
quattordici carte, quello che fa
le carte farebbe obligato a
prenderfi il compimento delle
fae carte dallo fcarto di quelloy -
che ha rubato, ed incolparYy
fe ftelfe fe non ha faputo mefco=
lar bene le carte, ne occorre il |
dire, che lo {carto fono tutte
‘cartaccie, € che Inj non devé —
avere men carte dellaleri » pre-_
tendendo d’ obbligare quello >
che alza a non poter rubaré
pit di 13. carte per poter an.
{ciare le fue carte giufte aque’
lo, che fa le carte, il che fe 60%
fotfe quando uno tralafcia /#
fola fuori del gioco? » pravverer”

be in confeguenza x che ae 7
x ad

che alza tion potrebbe rubare >.
perche fi sa di certo, che le
carte. della fola antecedentes
fono tutte cartaccie , eficcome:
non é vera quefta propofizione
perche quello,che alza,ha facol.
ta dirubare, ancorche fia rima.
fia fuori delle carte la fola an-
tecedente,cosiancora-ha facolta&
di:pigliare alzando. quante car=
te, che trova da poter pighiare;.
e quello. che fa le-carte in quel
cafo deve prendere il compi--
mento delle fue carte dallo {car-
to diqueHo:, che-ha alzato, il-
quale dovra. prima eflere me+
{colato tutto aflieme .

Finita poi che fara Palzata,
quello, che havra mefcolate lé
carte daraprimadieciearte pex
ognutio de” Giocatori, e poi co«
mincerd da capo ¥ @ ne dard
uideci; ma Pultima dovra dags

Bs la

 
 

34
la fcoperta, efe fara carta di

onore, quello a cui fara toccata

{egnera tantiponti a {uo fav 0-
re, quanti ponti importera lai

carta {coperta , e dopo, che ha-

vra date atutti li Giacatori le)
undeci carte con prenderfeley’

ancora per fe’; allora offervera
ancora lui, fe dopo la tua unde

cima carta vi fara apprefio al=.

tra carta d’onore,o fopraventts
e quelle dovra prenderfele tut-
te per fe, fino che ve ne trove

ra confegnate @ fu0 favore tan:
ti pont, quanti importeranno
le carte d’onore, che ivi avré

trovate . :
Dopod che avra yeduto non
effervi pit carte da potert
prendere, offervera allora fem
quelle poche carte,che le fara’

no rimafte vi fara carta d’one7
derfele
t-

re, e quelle dovra prenae
tu

as

 
35
tutte per fe’, ad efclufione perd
delli fopraventi, quali non fi
poffono pill prendere fe non in
cafo, che pofflino’ fare verzico=
la, ¢ quando tutti li Giocatori
avranno fatto le carte una vol-
ta per uno,allora fi verra di nuo-
vo alla divifione de’ Compagni,'
quale dovra. effer fatta nellas:
forma, che abbiamo detta di
fopra. |

Vogliono alcuni, che fe quel-
lo, che fa le carte trovaffe all’
ultimo il 29., ¢ quefti non fas.
ceffe attualmente verzicola,che
non fi pofla prendere , ‘e debbas
reftare fuori del gioco; ma fic-
come quefta carta puole fempre
fare verzicola,ancorche attuals
mente non Ja faccia cosi, con fa
{peranza, che pofla fempre far
gioco non deve mai reftare ne}
monte, € percid in moltiflime

BG con-

 

 

 

 
 

 

 

 

i :
26
converfazioni di gia e ftato le
vato quefto abufo , perche vera-
mente ¢ flato conofciuto. jngiu«
fto; Vorigine di quef’abufo-pro*
viene da usvaltro.abufo maggie.
re , il quale fi prattica ancora in
moltiflime converfaziont; ma
per eflere ingiuftiflimo» ¢ col
tro ogni raggione » menita d’ef-
{ere abolito affatto, come dir
gia é ftato fatto in molti luoghly
perche. ogni volta » che refti Ie-
vato quefto abufo maggiore, al

Jora refta ancora levato ogni

altro abufo,.che da quello pro~
viene. we
Si ftilava prima, ¢ ftila ane
cora in alcune converfazionls
che quello:s. che fa le carte fe 1
yelle poche carte ,che reftane
all’ultimo. non avefle trovata+
carta d’onore » doveffe nece#4"
fiamente pagare afliem¢ 6 |

 

 
37
{uo Compagno. un refto all’
avverfarijse ficcome fuccede's
che alle volte: in quelle poches
carte non vi fi ritrova che il
folo 29. cosi volevano, che fe if
detto. 2g. non faceva attual-
mente verzicola,non fi fofle po-
tuto prendere, ¢ foffe allora fta-
to neceffitato pagare ilrefto. ©
‘Ma perche if far pagare il re-
fto a chi non trova ¢arta di
onore nel monte era una legge
troppo: barbara , e troppo in-
giufta » cosie gia ftata commu-
nemente levata-;-e fi: {pera , che
in breve fi levera da per tutto $
poko poiche chi-fa le carte non
fia foggetto a pagare quefta pe-~
na ».allora non vi fara pin difii-
colta per il 29. perche compira
ancora alla parte avverfa,che il
detto.29. fiain gioco, perche in
guel.cafo.puole avere lei fola la
—— dpe-

 

 

 

 
 

| | 38
 fperanza, che li pofla far gioco
_| per avere il 39,€ cosi vorra,che
( fenz’altro refti in gioco .
Hl Che poi non fia obligato @
pagare ilrefia, quello che nel
far le carte non trovaalcuna»
cartad@’onore., in quelle poche
| carte, che le reftano in mano
| | fembra_ giuftiflimo , perche il
pagamento del refto in quefto
cafo fi prefuppone in pene di
non haver trovata carta d’ono-)
re; ma non fie mai trovata lege
cosi barbara, che oblighi a pa-
-garela penad'un atto , che dir
pende puramenté dal acciden-
te, anche in pregiudizio pro~
prio,la pena deve folamente pa~
garfi per quelli fatti, che poflo-
no effere maliziofi , € ridondana
in vantaggio di. chi fi commete
te, ancorche alle volte ne fie

gua al contrario ; ma que™?
chit

 

 

 

 

ee
oy ee

che ridondano folo in pregindi-
zio dichili commette, ¢ hon»
poffono effere maliziofi,non de-
vono condannarfi a pena alcu-
na; mentre.bafta in quel cafo il
folo pregiudizio , che fi prova
fenzaaggravarlo con maggior
pena, perche fe fi doveffe con-
dannare alla pena del refto quel-
lo, che non trova carta d@’onore
in quelle poche carte , che fono
rimafte al monte , con maggior
giuftizia dovrebbe effere con-
dannato a pena maggiore colui,
che in 21. carta non havra tal-
volta carta d’onore ,e pure non
fitrova legge,che poffa obligar-
lo a pagarne la pena , baftando+
le per fola pena la neceffita del
perdere.

Oltre di quefti abufi ve n’é un’
altro forfi di non minor confi-
derazione, quale deve onnina-

men-

 

 

 
 

Q

mente levarfi , perche & contto
Jaraggione, ¢ la giuftizia del

| gioco , come diremo adeffo «
_ _ ~Digidavete intefo di fopta,che:

| il giocare con fola ,. porta {eco
per neceffaria confeguenza, che:
tutte le carte d’onore, oche
pofiono far gioco ,devono.cie=
re in gioco », perche quella ¢ la
yera definizione della fola; dt
piti.avete intefo,.che tanto ques
to, che alza, quanto quello: che:
fale carte, hala facolta. di:pi-
gliare per fe tutte. le carte di
onote, con tuttbli fopraventi ,
che trova, fia nell’alzata,fia nel~
lafcoperta; ora abbiamo ave-
dere, che cofa‘fi abbia 4 fare di
quelle carte prefe in quel mo-
do.
Dovete prima fapere che
juno de’Giocatort puole gioca”
re gon pit, O meno cartes a ”

 
ar,
tregl’é ato pofto per pena il
non contare alla fine del gioco
li fuoi cnori,fia comunque fi vo-
glia, non potende-contare che
Pultima fela fa » e le carte fer
pure ne vince, © laraggione fi
eperche il giocare-con pil, o
meno carte puole eflere fatto
con frode,come faremo vederes
ed a quefta pena refta ancora»
foggetto il proprio compagno 5
perche prima che cominci a gio~
care puole,e deve avvertitlo che

conti le fue carte , accid abbia. a

‘ giocare'con carte giufte, ¢ per

cid refta ancora lui foggetto alla
pena in cafodi contravenziones
mentre chi ha pit carte del fue
dovere prima di cominciare @
giocare deve fcartarle, ¢ gettar-
Je al monte, reftando in fuo ar-
bitrio fcartare quelle carte che
pitile parera, e le piacera,pur-

che
 

 

 

42
che non fiano carte d’onore,do-

one nelle reftare tutte ins

510CO cotro l’opinione mal fon-

data d’alcuni, li quali pretende~

vano, € pretendono ancora che.
fi poflano fcartare le carte d’o-

nore, Con la raggione, che chi

{carta , cerca fempre di fare ik

fuo maggior vantaggio, si che

quando {carta una carta d’ono-

rc, la {carta folamente, perches

crecde,che la carta f{cartata pof-
Ja effere di {uo maggior vantag-

gio fuori di gioco, perche fe re-

fia in gioco teme di perderla

con {uo pregiudizio.

Ma (e militaffe quefta raggio-
ne, dovrebbe altresi effere in
{ua facolta il dire di non voler
giocar quando uno ha carte cat-
tive in mano, perche in quel ca-
fo cercarebbe ancora Iui il fro

maggior vantaggio,mentre HP
4:
lafciarebbe di giocare,folamen-
te quando aveile timore di per-
deresma ficcome non é giuftizia
che uno,che abbia cattive carte
poifa lafciar di giocare con tut-
to che tema di perdere aflai, per
che altrimenti non farebbe piti«
gioco , fe folo fi voleffe giocare
sii la certezza della vincita,
mentre il Giocatore deve effe-
re foggetto alla perdita, e al-
la vincita, cosinon é ginftizia
che uno che rubbi una cartas
poffa {cartare 4 fuo arbitrio lé
carte d’onore , perche col {car-
tare carta d’onore viene 4 pre-
Ziudicare alla rettitudine del
Bioco. Deve bene il Giocatore
procurare il {ao minor male,ma
pero fenza pregiudizio del Zio-
co, € fenza inganno ; 11 {cartare
cartad onore,oltre diche é pre-
Biudicale al gioco ,'¢ ancora

frau-

 
 

 

 

 

 

 

44 :
trandolento 9 perche chi fcarta
carta d onore fa contro la natus
ra del gioco, quale deve effere
fatto con tutte le carte d’ono-
¥e, come habbiamo detto di fo-
pra, né deve eflere in arbitriod’
un tolo il levare dal Bioco carta
che conti, perche quello che
{carta carta che conti ,la fcanta
Jolo pertimore, che ha di per
derla , cosila fcarta 4 folo fine
che non faccia gioco contro di
fe medefimo, e quefto.¢ trop.
po pregiudiciale alla parte con-
traria, quale fi trova di fare me-
no gioco diquello che farebbe
fe la detta carta d’onore fofie
rimatfta in gioco.

E poi fe confideriamo bene
il cafo , due folo fono quelli che
poffono naturalmente fcartares
cioé quello che alza, 6 quello

che fale carte ,. fe ¢ quello ehe
a “
 

4
alza deve feartare’ per avere
rubata qualche ca#ti-d’onore,
0 fopraventi , onde per il van-
tageio’ che ha avuto di tubba-
re non puole caufare pregiudi-
Zio al gioco; ma deve rigorofa-
mente fervirfene in vantaggio
del medefimo gioco » Maflime
petche col rubare ha tolto i]
‘vantaggio 4 quello che fale car~
te, ed in quefto cafo verrebbe
4 caggionare due pregiudizij ,
UNO ioe a quello che fa le car-
te, € altro al gioco, il che non
€ ginfto, née permiftibile .

S€ poi ¢quello che {4 le carte,

che deve {cartare é nemeno Inj
ha facolta difcartare carta che
conti,perche nella ftefla confor.
_mita verrebbe a caufare previu.

dizio al gioco, ¢ non Puole , na
deve fervirfi del Vantageio qi
prendere per fe tutte le Carte

@one~

 
 

 

 

 

|
i
i
|

46
donore che trova nel monte
per poi {Cartarne altre d’onore
a {uo arbitrio. Stimarei forte
meno male il fcartare Piftefle
che trova nel monte, ma quefte
ne meno fi puol permettere 5
perche farebbe un permettere
fervirfi della fola folamente 4
proprio beneficio, ¢ non a be-
neficio publico; e poi allora non
fi potrebbe pil dire gioco, ¢né
fola, perche Ia fola nefluno 1a
farebbe fe non in cafo, che yi
conofceffe il proprio vantag-
gio. Ecosi fi direbbe fola ad
arbitrio; ma ficome pare che la
giuftizia del gioco voglia , che
chi gode il vantaggio. di pren=
derfi per fe tutte le carte d’ono-

. xe, che fono rimafte al monte »

debba ancora foggiacere al pe-
ricolo di perderle , altrimenti tl

Sloco non farebbe ne giufto » NC
: egua~
 

eguale,cosi per togliere da mez-
20 queft’abufo contro tutta 'e=
quita del gioco, é neceflarid
confeffare , che quando fi gioca
con la fola non fi poflino mai
{cartare carte d’onore,

- E’vero,che a principio il gio~
co fi introdotto con la liberta
di fcartare quelle carte, che pitt
placevano, ancorche foflero d?
onore, ma ealtresivero, che
queito gioco a principio non fu
inventato con la fola » @ ficco~
me in quel cafo potevano refta-
re nel monte altre carte d’ono~
Te , Cosi non fi faceva ingiuria &
nefluno il {cartare carta d’ono=
te. Ma al contrario quando fi
$10ca con la fola fi sa di certo ,
che nel monte non vi poffong
eflere rimatte carte d’onore , «
fe pure ven’é rimafa qualche.
duna , quella deve neceflaria.

MER
48
mente effere di quello, che fa

- le carte , cosi non deve poi effe-

re lecito.ad alcuno il voler met-
tere nel monte, ¢ fuori del gio-
co una carta d’onore, perche
quella carta d’onore , che non
gioca, deve fenza dubbio eflere
pregiudiciale a qualcheduno
de’ Giocatori, il che non & de-
ve permettere in conto alcuno,

Quefto difcorfa ¢ bello, .e

 

buono, par difentirmirifpon. —

dere;ma quella cofa di non con.
tare quando fi gioca.con pid, 6
meno carte fembra pena troppo
rigorofa , ¢ forfe anche ingiufta,
perche dal {cartare , 6 non fcar=
tare, non ne nafce altro pregiu-
dizio , che quello prova chi
gioca. con pil, 6 meno carte, e
Poi il non {cartare proviene fo-
lamente da unpuro accidente
di dimenticanza , non effendo
cre-

be
aa eee
credibile che uno voglia gioca«

recon pill, © meno carte .

Ma io vi rifpondo, che tutte
le pene nel gioco’ fono ftate in-
ventate per oviare le frodi, che
fi poffono fare , ¢ per effere ob=
bligato alla pena non énecef-
fario, che fia fatta la frode, ma
bafta, che vi pofla effere, e quel

male, che per accidente € acca~.
duto bafta che fi poffa argumen-:

tare, che poflaeffere fatto con
frode; perche in quel cafo fi

prefuppone fempre la malizia ,.

€ ficcome il giocare con pit, 0

meno carte puole eflere fatto:

con frode,cosi ancorche accada
per accidente deve eflervi las
fua peva, quale deve effere af-
fai maggiore di quella , che de-
ve pagare quello che fa le carte
in cafo disbaglio, e quefta pena

ferye per tenere attenti,ed ap-,

pli; 4

~
50

plicati i Giocatori,accioche per
laloro diftrazzione non fuece-
da ognimanounsbaglio, ede
-ancora dovere che fia aflai rigo-
rofa, perche fe fi accorge d’ave-
re carta di pit,oltrediche hail
vantaggio di fcartare a fuo ar~
bitrio fa pagare di pit lapenaa
quello, che da le carte, cosiin~—
cafo di non fcartare pernon ef
ferfi fervito del fuo privilegio ,
deve eflere obligato ad una pe-
na di gran longa maggiore,qua-
le é’quella di non contare ,

Che poi il giocare con pill, 6
meno carte poffa eflere fatto
con malizia evidentemente fi
conofce, perche fe tal’uno avet-
fe da farfi qualche carta d’ono-
re ,‘che li premeffe » ¢ dubitaffe
dinon aver forma di poterlas
fare, potrebbe fcartare una , 0 —
due carte per farfiun faglio, © |

cQ-

Z
 

Ou

I

cosi ¢fimerfi, dal pericolo di
perderla,6 pure col gettare con:
bello ftudio in terra una delle
fue carti prendere poi all’avver-
farij qualche carta di gran con-
feguenza , € fe non vi fofle pena
verupa potrebbe con tutta faci-=
lita fuccedere {pefliffimo. _

Col non {cartare poi fi puole.
falyare qualche carta d’onore,la
quale dovrebbe necefsariamen-.
te perderfi, mentre puol tener-
la per Ja fua ultima cartas e per
confeguenza fuori del gioco, e.
cosi farebbe ficuro di poterla fa-
re, Oalmeno dinon potérlas
perdere, € ficcome quefto fa-
rebbe un giocare con vantag-
gio, econ malizia, cosinon,
deve permetterfi.

Inoltreé obligato alla pena
quello che da pit’, o meno car-
te, ¢ poi tutto il vantaggio ¢

C3 di
52
{| di chi la riceve, perche in quels
| lacarta di pid. puole aver avis
| | to una carta donore, e dip *
ada facolta di feartate a {uo
, ‘modo, cosie dovere che fia ob"
| Tgato ad una pena maggiore
| chiégioca con una catta di pills:
odi meno, folo per nod efserfi
a fervito del fuoivantaggio=
Chi da pit, meno carte ©
obligato alla pena » perche nel
dare le carte puol conofcer
qualche carta » © cosi darne ula
itl, O MEN, fecondo che |i tor
Li | na a conto; onde per oviate?
| | quefta frode e€ ftato nece flat!
e per eller
non nece!-
i le carte con
che vi pol

eflere .
La pena di chi sbaglia le cat

te fono 20. punti, pet Ja primé@
| cate

 

 

j

ge

 
53
carta,e:dieci per ogni altra,fino
alla fomma d’un refto ; ma fe lo
sbaglio foffe in tutti, allora ft
deve pagare la pena per tutti,
e fi puole arrivare @ pagare un
refto per ogni sbaglio di carte:s
A quefta pena e ancora foggetto
il compagno , perche ancora lui
deve fare attento, ed avertire
il (uo compagno, quando fi ac-
corge dello sbaglio.

Chiha carte. di pit datele per
sbaglio, ha facolta difcartare
a fuo arbitrio, prima che co-
minci agiocare, purche con lo
fcarto non fi faccia faglio , per=
che non deve fervirfi di quel
-valtaggio per ammiazzare. ul
RéalPaverfarij, fe poi ne ha di
meno, deve patimente prima
di cominciare a giocare do-
mandare a quello, che fa le car
te tante carte » quante ne ha

C 3 di

 

 

 

 
yr ved

 

 

 

 

 

54
di meno, equefti dovra darlé
diquelle del monte ; pero pri-
ma di far la fola mefcolan 0
ben bene il monte, ¢ poi quello
a cui mancano domandera fer”
za vederle quelle carte » che
pitt le piacera 5 purche non G0”
mandi pit del fuo dovere -
Puole di pid darfi lo sbaglio
dicarte inuno, che avefie le
fue carte ginfte, ed obligare Mt
nel cafo quello, che fale carte
alla pena dello sbaglio» perche
fe quello , che fa le carte fi foor~
daffe di dare la carta {coperta
adalcuno,¢ quello , cid non
oftante , avelte le fue carte gil”
fe , none per quefto , che quer
lo, che fale carte » non abbiv
sbagliato ; mentre deve oh
dere, che lo sbaglio fia fucc®
duto nelle prime ‘carte, ee

gol non dare la carta con
ona
55

ha privato quello del vantaggio

di fegnare a fuo favore tanti
punti, quanti ne poteva impot-
tare la carta {coperta, efe non
foife obligato 4 pena veruna ,
verrebbe in quel cafoad efimer-
fi con inganno dalla pena dello
Sbagho ..

Se poi il Giocatore incomin-
ciaffe a giocare fenza aver pris
macontate le fue carte, il che
é contro ogni buona regola di
gioco, efi trovaffe 4 giocares
con pitt, Omeno carte; allora
non é pi intempo di rimediare |
al male; ma deve pagare las
pena di non contare , perche
quando uno ha incominciato a
giocare, non é pil padrone di
ritirarfi la carta giocata per
non eflere quella pit fua, mas
del gioco; Onde in quel cafo
non potendo rimediare all’erros

C4 Ie,

 

 

 

 

 
$6

Te, dovra foggiacere alla pe-
oo

2 Siuftamente dovuta a chi
$10¢a Con pitt, oO meno carte .
© quefta pena puole dirfi
troppo tigorofa rifpetto . al
ompagne , perche trattandofi
Mel fuo interefle , deve avertire
“/40 Compagno, prima ches
Slochi, che conti le fue carte,
© che {cart} » fe ha.dafcartare;
€ liccome Jyi gode del vantag.
810 della pena di quello., che fq
le carte, {e le conta in tempo ,
Cosi deve effere ancora lui foo.
Setto a quefta pena, fe pér faa
trafchragsine il fuo Compagna
Sioca fenza aver contato. ley
fue carte.

Se fofle nel cafo.d’avere una
Carta di pit; allora, ancorches
aveffe giocata la prima carta,
€ fiaccorge dellosbaglio, prima

quel

Che foffe copertala mano, in i
quel cafo puole tintediars’ af
male, con dire che la carta’.
ehe ha pofta. ful tavolino- deb
gioco é quella, che ha intefg
di fcartare,.e cosi-fe li. deve bo-.
nificare lofearto., e rifcuotere
la penada quello, che ha-fatte
le carte, fe la carta che ha di
pid elie ftata data per sbaglio.

Ma fe per accidente nel ri-
fpondere la prima volta daffe un
Re, 0 pure altra carta d’onore
come che quellencn fi poffo-
no {cartare ; allora non é pid in
tempo a rimediare al male, con
tutto che foffe la prima giocata:
il fimile puol parimente fucce-
dere in qualfifia altra carta ,,
che non fiad’onore’; mentre fe,
avefle per accidente una carta
di pit , la quale le fofle ftata da.
tada chi fa le carte, e fi trovaf=
fe una fola carta di fpada,- o.al.

C 5 tra
58
tra cartadidiverfa fpecie, efi
giocaffe la prima volta fpada, a
dove quello avetle una fola car-
tase dopo di aver giocato fiac-
corgefie d’aver una carta di pitty
in quel cafo non potendodir che
{carta la carta giocata per non
poterfi far faglio,cosi non ha pit
facolta di poter {cartare; macs
deve continuare il gioco fing
alPultimo con la fua carta dj
pil, € poinoncontare.

Hora che abbiamo veduto jj
modo, che deve tener uno ,'che
aveffe pil, Omenocarte ,& dge
vere veder il modo, che dove-
ra tenere chi fa le carte, in cafo
che fi accorgefle dello sbaglio
in tempo di poterlo rimediare.
Pertanto fi devefapere, ches
quello, che fa le carte deve»
flar molto attento 4 non sba=
gliare;ma fe mai fi accorgefle@?

aver
59
aver data qualche carta dipia,
o dimeno ad alcuno , deve pro-.
curar dirimediare al male, pri-
ma che quello fi volti le carte.
alla faccia,perche dopo lui-non
é piu padrone di quelle carte,né
giova il dire, che fi contino,per-
che l’obligo di contarle compe-
te folamente.a quello , che fa Ie
carte fino a tanto che lecarte
fono coperte , e non fono vedu-
te ;ma una volta, che le carte
fono vedute , 0 ha data la carta

fcoperta, allora non puole pit -

in conto alcuno difporre di
quelle carte; ma deve lafciar
correre, e pagarne la pena, fe vi
ésbaglio, fe laparte fene ac-
corge,enonfare , come fi lu-
fingano molti, li quali quando
{coprono all’avverfarij qualche
carta d’onore, dicono fubito ,
contate le carte, credendofi dj
C 6 po-
 

 

 

  

60
poterli levate quella carta di
onore, con lo sbaglio da loro
fatto, € non fiavvedono, che
oltre if non potere aver quell’
onore y farebbero ci pit obliga-
tia pagar la pena detta di foptay
ogni qual volta pero la cartes
fofle fcoperta, ¢ veduta dali

-.Giocatori; mentre Pavvifano»
che conti le carte , dove al cone

trario fe non fi raccordaffe ‘il
contare le carte , potrebbe darfi
Vaccidente , che fi dimenticafte
dicontarle, ¢ giocafle con una
carta di pil fe maiglic lavefle
data in sbaglio.

Refta ora a vederfi, che re-
gola deve tenerfi in cafo che |
le carte foflero giufte di nus
mero, € non di qualita, € che Ht
Giocatori non fe ne accorse 10a
to che alla fine. del primo ge
co, ¢ fe bene in alcune ee
 

61
fazioni f prattica di non manda
re mai il gioco 4 monte’ col pre-
tefto, chealgidco delle Min-
chiate mai devono rifarfi les
carte: Io pero fono di diverfo
fentimento, prima perche queft
affioma non & fondata fopra al-
cuna ragione’, € poi perche me
lo perfuade la ragione ; mentre
ficcome il giocar con fola , por=
ta feco per neceflaria confe-
guenza, che tutte le carte di
Onore fiano: in gioco 5 comes
habbiamo detto di fopra, cosk
mancando qualche earta d’ono-~

re il gioco fi deve avere per hul-:

lo, e per non fatto’, perche tut-
te le carte d’onoré hot erano
in gioco, come e*doveresne
occorre, che fi dida, che {e'qiiel=
lo, che fa le carte, fe ne accor?
fe prima di €omiliciare aeiocas
re puol far giconoftere da” ears
- ta 9.

 

 
  

 

 

|
ii!
¥
ih
il
iN

 

 

 

 

 
62
ta, che manca, e prenderfela
per fe, come rimafta nella fola ,
ancorche foffe cafcata in terra,

© rimafta fuori del gioco per.

accidente , perche io le rifpon-
do, che il gioco deve effere fat-
to fenza frode, ¢ dove vi poffa
effer la frode, fe ne deve pagar
Ja pena, e ficcome. in quettas
forma potrebbero commetterfi
moltiffimi inganni, quali non
devono permetterfi , cosi in.
quel cafo il gioco fi deve avere
fempre per non fatto; altrimen-
ti fpeffo fuccederebbe,che quel-
Io, che fa le carte con deftrezza
di mano, econ malizia getta-
rebbe in terra, 0 lafcierebbes
fuori del gioco le migliori car-
te, fe non tutte, almeno qualche
dona , con la certezza , che fen-
2a pagarne pena, quelle tocca-
rebbero a luineceflariamente ;
; on-=

ee ee ee ee ee ee ee ee ee. ee. ae. ae

ge ee. ot ce ie a ee Se ae
63
nde pet oviare quefti inganni
“neceffario confeflare , ches
nivolta, che nelle carte del
r1Ioco mMalica alcuna carta di
ynore , Overo ve ne fara alcuna
Ji pid per non far nafcere un in-
sonveniente maggiore; allora
ynel gioco sintendera fempre>»
yer hon fatto, fe poi le carte,
-he mancaffero , 6 che crefcef-
ero, non fofsero d’onore; al-
ora il gioco s’avera fempre per
yen fatto ,e validiffimo.

Nafce un’ altrodubbio in ma-
éria di fcarto, anche fra primi
Ziocatori di non poco rilievo,la
-efoluzione del quale io per me
timo affaichiara,ma la raggio-
1e pero é quella, che deve con-
yincerci; fupponiamo adunque,
che uno de’Giocatori abbia,al-~
-ando, rubate due carte, e que-
te Pabbia pofte da parte ful ta-

yor
 

 

 

 

 

 

 

G4.
volino per doverne {cartareo

due altre a {uo tempo ; quefto

fenza punto piu ritlettere 4 quel-
le due carte ha incominciato 4
giocare fenza f{cartare; fi do-
manda adeflo come debba an;
dare quefto gioco, non parendo
giuftitia il condannare a4 pena
veruna quello,,.che ha rubato le
due carte, e non ha f{cartato,
perche alla fine gioca con le fue
€arte giufte,e fe poi vie qual]-
che pregiudizio in quel-cafo @
tutto di quello, che non ha {car-
tato,, non potendofi negare, che
Jo fcartare non fia di gran yan-
tageio.

Dovete prima avvertire-co-.

Me fi gioca, perche dal modo

del giocare ne nafce una refolu-
Zione diverfa fe fi gioca all’an-
_tica, e fenza fola, alloranoné
fogectto a pena veruna, e deve
ti-
«=

tirarfi tananzi il gioco, come fe
quelle due carte non foffero in
gioco; fe pot fi gioca con la
fola caminando. fempre con.
Vifteffo principio, che tutte le
earte d’onore debbano effere in
gioco, dico che le carte, che ha
rubate» 0 fono d’onore , Onon
Jo fono, fe non fono d’onore
allora deve prenderfele in ma-
no, € giocare con pit carte ,e
poi foggiacere alla pena di chi
gioca con pitt, Omeno carte,
fe non fono d’onore fenza efle-
re obligato a penaalcuna puol
dire d' avere feartate quelles
- ftefle, che ha rubato , né occor-
re dire, che il pregiudizio in
quefto cafo é tutto di chi non
{cazta, perche fi deve riflettere,
che il gioco nome fatto per li
ftorditi; onde puol eflere, che
non abbia voluto {cartare per:
timo-
  

66

| timore che aveva di perderes
i | quell’onore; econ fare lo ftor-
A dito efimerfi dal pericolo di
| perderlo, il che fe bene di rare
aa accade, nulladimeno puole ac-

a cadere, eper quefto giufto deve
efler foggetto alla pena del non
contare, non effendo neceflario
il commettere la malizia per ef
fer obligato alla pena, ma baftay

che vi pofla eflere ._
In oltre fupponiamo, ches

fi uello che fa le carte abbia pre-
ih te nella fola tre carte d’onore »
) ~~ enon ne fcarti folamente che
i due, € poi cominci a giocare +
| dopo d’aver giocato gl’avver™
} farij pretendono che debba 107
i care con una carta di piu pee
| avere {cartato una carta meno
| del fao dovere ; ficerca aden?
Mh come debba regolarfi quelte, es

| fo, & aqual pena acne a

 

 

 

 

 

 
  
  

Gy
obligato if delinquente, non
effendo dubio, che quefto cafo
‘merita qualche pena. . t

Primadidover venire allare~
foluzione della condanna, fi de~ .
ve riilettere quale fia ftata las
carta fcoperta di quello, che ha |
fatta la fola, perche quando |
Pavverfario fi ricorda precifa~ |
mente di quella carta, allora |
puole riconofcere fe quell’ifter i

“fa carta é rimaftanel monte, & |
incafo'che non fia carta d’ono- — |
te, che fia rimafta nel morite, |
ogni qualvolta quello sioca con |
le fue carte ciufte’, allora noné
obligato 4 veruna pena , perche
puol dire, che per terza carta
ha fcartata la fua feoperta, & |
Ogni volta , che non fi poffa re-
convenire di fatto , con avvers
. Urlo , che Ja fua carta {coperta
Ron e nel monte, allora {e ]j de-

ve

 

 

¥

 
 

Lt ve abonare lo {carto fenza per
iy na veruna, ma fe poi alcuno de’
i || Giocatori fapendo di certo la
I fua carta {coperta lo reconye-
+ nifle di fatto per non trovare la
ty detta carta nel monte, allora
|| dev’eflere obligato alla pepe
| dello sbaglio per efferii prefa
i il ‘una carta meno, mentre ogt}
al qualvolta gioca con le fue cats
| te giuftc , enon ha lafciato fuctt
| del gioco carta d’onore non
| ‘puoleffere obligatoa pik.
i ~ Daquefte giufte .¢ veridiche
| refoluzioni ne deriva per necel-
i fita unaltra, non meno degia
| da faperfi da chi defidera ben
imparar agiocare a quefto 10"
co per tutti glaccidentt > C

Mi poteflero accadere 1n un ca y
| quafi incredibile, qual que!
|

 

 

di lafciar per accidente una a

iq ta d’onore nel monte, eral

 

 

 

 
69
del gioco, {pettando quefta ne-
ceflariamenre a quello, che ha
fatto le carte, di modo tale che
dovera fempre efler confiderata
in gioco quella carta , come {es
fofle in mano di quello , che ha
fatte le carte,per non poter efler |
ladetta carta fuori del gioco; e |
fe quello, che ha fatte le carte |
con ladetta carta avera‘giocato
con una carta di pid, fara necef- |
fariamente obligato avoggiacer |
alla pena di non contare,perche |
puol aver lafciato 4 bella pofta |
fuori del gioco quella carta;che: |
noi fupponiamo lafciata per ac-

:

 

 

 

accidente, per timore che aveva |
di perderla pigliandofela. |

Giache fiamo nella. materia \
dello feartu, e delPalzata, vi di.
ro alcune cofe circa Palzata nes

 

ceflariiffime a faperfi, & & che’

. fe quello, che alza nellalzare
las |

)

 
40.

lafcia cadere una carta in tavo-
Ja, quella ancorche non fia ve-
duta dev’effere la fua alzata, ne
& lecito a quello , che ha alzato
il dire non la voglio, come fan...
no moltiflimi, perche per effer
feparata dall’altre fi puol cono~
{cere cosi,O fia buona; fia cat.
tiva quella dev’eflere necefla-
riamente?}lalzata,. ancorches
quella carta caduta ful tavolino
veramente non foffe Pultimas.
carta alzata. Taluni ancora
ufano dopo aver alzato di dare.
alcune carte di quellalzate al
fuo compagno, afpettando a pi-
gliare per fe quella carta, che.
pill gli piace per fua alzata , ma
queftoné meno fi puol fare,per-
che col tatto fi poflono facil-
mente conofcere le carte; cost’
per ovviare ofn’inganno ¢ne-
ceffario prima di venir all’alza-

tay
k—

wt
ta, che fi {pieghi fe vuol lafcia-
re carte di fotto al fuo compa-~
gna, e che dica quante ce ney
vuol lafciare , accioche quello,
che fa le carte non abbia occa-
fione di dolerfi, che le carte
poffono efler conofciute in quel-
la conformita , che fi prattica
appunto quando fi vuol prende-
re per fua alzata , ola prima, 6
Pultima carta, le quali perche fi
poffono conofcere non fi poffo-
no pigliar mai fe non fi dice pri-
ma in tempo, che fi mefcolano
affieme le carte, perche unas
volta ch’abbia fpiegato anima
fuo non puol pil recedere da
quefta {ua deliberatione,ma de-

ve neceflariamente efeguirla.
Se poi nell’alzare cadeflero
di mano tré, O quattro carte
alanes allora per non met.
terfi in difputa quale fofle Ia

pris
72,
syima, foflero comunque fi fia
! q

le carte fcoperte, allora fi deve
di nuovo rifar le carte,e di nuo-
vo alzarle, ma fe delle carte,
che cafcaffero ful tavolino una
folo foffe la fcoperta, e Valtre
tutte coperte,allora la fcoperta
fara la prima carta dell’alzata,
e poi fuccederanno Valtre carte
coperte.

Dopo che avrete bene impa-
rate 4a memoria tutte quefte re-
gole, non per quefto avretes
imparato 4 giocare, mentre fin?
ora non fi e difcorfo d’altro,
che delli accidenti,che poflone
accadere nel fare le carte ; ora
@ dovere cominciar un poco a
difcorrere del gioco, e del mo-
do di pratticarlo.

Dovete dunque fapere , che
il Giocatore ¢ fempre obligato
arifpondere di quella Pee >

che

eR

Ss mA A = WH.

 
 

73
she fi gioca, a fegno'tale , ‘che
e uno de’Giocatori giocafle una
arta di denaro,e l’altro rifpon-
lefle tarocco, fe poi aveffe in
nafio carta didenaro farebbe
ybligato a pagare agl’avverfa-
ii, fe quefti fe ne accorgeflero,
in refto per ogn’uno.

Ma ficcome quefti non fe ne
oflone accorgere, fe non quan-
loviene giocata quella carta,
er Ia quale deve pagarfi la pe-
a, & allora per lo pitt il delin-
uente fuol negare il fatto, che
on: puol ‘provarfi in altras
orma, che col riconofcere lo
baglio, quale fe veramente fi

rova nelle carte giocate, non

ffendo credibile, che uno nel
‘ioco voglia accufare un altro
’un delitto falfo , quando que-
to fi ha da provare con la pura
erita del fatto, col veder tut:
D te
 

 

 

 

74
‘te le carte giocate, allora fi de-
ye prefupporre , che lo sbaglio
VP abbia commeffo V accufatos
non baftando mai in alcun fatto
la fola negativa del delinguenté
a provare il contrario, altrt
menti non fi darebbe mai cafo:
che li delinquenti foflero obli-
gati A penaveruna, © cosi po
trebbonfi commettere a mane
falva delle frodi, ¢ dell’ i-

gan. |
Pofto dunque, che refli pro

yato ildelitto non vie dubi0:
che deve pagarfi la pena» Ja
quale dev'eflere tutta a carict
del delinquente » al contrarit
ditutte le altre,cbe fi pagano if
compagnia, ela raggione I
perche nell’altri cafi i1 comps
gno puole, e deve avvertines?
in quefto per non faper a f
carte pofla aver in mano?
come

 

| |
7
Coy.) i
compagno non puolavvertirlo; |
onde lui non dev’efler foggetto !
» aduna pena per un fatto di cui |
lui @ affatto innocente, fe poi |
in vece di denaro dafle coppe, 6
qualche altra cartiglia , alloras
pare, che la convenienza vo-
glia Payvertirlo, perche non
fe glipuole fupporre malizia, |
come probabilmente fi puole |
fupporre in uno, che fi faccia |
una carta d’onore, € percon- =|
feguenza non fembradegnodi |
pena, |
Il non rifpondere adequata-
mente diquella fpecie, che fi
gioca fi chiama rifiuto, eflen-
dofi fempre obligato a rifpon-
dere di quella {pecie,che fi gio- |
ca, fino che fe ne ha in mano, fe
poi di quella fpecie,che fi gioca |
non fene ha pid, allora é obli- |
gato 4 rifpondere Tarocco per ‘|
D2 fine, |

 

Tecate dis baineel cae ad cis eat Se Se ES

 

 
 

 

 

 

 

76 .
fino , che ne ha, e quando no
ne ha pitt puole dare che carta
vuole a fuo arbitrio, quando fi
gioca di qualche fpecie, che lui
non abbia, ma fene ha deve cid
non oftante rifpondere adequa-
tamente per non effere pol obli

gato alla pena del rifiuto; pe!

efimerfi pero dal pericolo de
rifiuto, puole gettare in tavol.
le fue carte {coperte, lafciand'
in arbitrio de’Giocatori il pre”
derfi che carta vogliono; ™
in quel cafo le fue carte piu no!
giocano in quella data di carte
enon puole pill prendere
alcuna, effendo quelle carte tub

te perdute .

Dovra pero avvertire que!
lo, che vorra gettare in favo"
le fue carte di non gettat €

quando avra ancora un Re
Je mani, perche ancot AN

 
te DN Die Se ace

~

Sa

 

 

a
ha per perduto,e fe gli avverfa-
riinon haveflero da poter pren~
dere, che una fola volta,e di pid

fi giocaffe d’una {pecie diverfa - .

della quale quello che ha get-
tate in tavola le fue carte ne
avetle,cio non oftante deve da-
re il Re in quella mano, perche
non effendo obligato 4 rifpon-
dere adequatamente poffono eli
avverfarij prenderfelo libera-
mente.

Se qualcheduno rifiuta per
prender qualche carta d’onoré
alla parte cotraria,oltre la pena
delli due refti, che deve pagare,
deve altresi alla fine del gioco
reftituire all’avverfarii la carta,

che li ha prefo, anzi deve refti- -

tuirgli tutte laltre carte ad ef.
clufione della fua , perche altri-
menti farebbe un far vantaggio
al {uo compagno con pregiudi-

3 Z10

 

 

 

 
 

 

 

 

 

 

 

 

zio degl’altri , e ficome il gioco
non permette , che con frode f
pregiudichi a neffuno,cosi ¢ do-
vere direftituirla fempre;quefa
regola pero non milita a favore
del refintante, perche fe quello,
che rifiuta petde una carta di
onore non puole alla fine do-
mandar la fua carta, tutto che
fii obligato alla pena, perche

-_doveva non rifiutare, e poi puol

effere, che abbia impedito alla
parte il farfi una carta di mag-
gior importanza, la quale ab-
bia poi neceflariamente da per-
derla, ¢ cosi fe ha perduto una
carta @onore ne incolpi fe ftef-
fo ,¢ per la fua trafcuragine ne
naghi la pena . ,

ii hte, accioche pofta dir ft
tale,é neceffario , che quello»
che lo commette abbia rigioca-

to , ne bafta, che la bafe {ia 60”
pet-

ee
   
 
  
   
   
 
 
  
  
  
   
  
  
  
  

. 79

perta, come vogliono alchni ,
perche ficome non fi puol tene-
re le mani ad alcuno, cosi puol
effere, che alcuno pid follecito
degValtri copra prefto la bafe ;
fe poi la bafe fi cuopriffe da chi
commette il rifiuto,allora bafta
anzi ancorche la bafe non foffle
coperta , ma avefle giocato im-
mediatamente lacarta, per la
quale ha commeffo il rifiuto y
fe Ini non avvertifce il rifluto da
{e prima di rigiocare quella car-
ta tanto ¢ obligato alla pena del
rifiuto , fe perd Pavertifce da fe
prima di giocare Ja carta,allora
puol rimediare al male fenza
pena veruna, ma per rimediar-~
lo fenza penaé neceflario pri-
ma di rimettere in tavola la»
catta, per laqual’e fucceffo i}
tifiuto, che firipigliin mano il
fuoTarocco,altrimenti fe lui la-

D 4 {cie-

 
 

 

 

 

 

 

{ciera in tavola il Tarocco,e 1a”

{ciera parimente iu tavola la
carta del rifiuto , non fi dev
credere, ch’abbia pofta in tav®
la la carta del rifiuto con idé:
di riagiuftare il rifiuto, ma be™
sicon idea dirigiocare, ¢ ¢0%
allora ¢ obligato alla pena d¢
rifiuto,perche ogni volta che
carta é in tavola fcoperta i1Gi0

catore non é pil padrone
quella carta, ma quella fi dev!

intendere giocata , fe pero toe
ca giocare a vi, in quella guili
appunto , che fuceederebbe §
quel Giocatore , che ave fie la.
mano, & avefle una carta di pl!
fe nella prima carta gioca
una cartaccia di fpada fcoperté
enon avefe di fpada altro ©

quella fola carta, ¢ prima chi
tutti liGiocatori aveflero rifp

fto s’accorgefle d’aver una a
a

 
 

ach, Svat i pink > eerie ln aa eh ee aa ite = ene aS ND WS | EN << st chil CAM ky

oy
ta di pid, tutto che la bafe non
fofle coperta, non per quefto
puol piu fcartare, e la ragione
fié , perche dopo che fi ¢ inco-
minciato & giocare non ¢ pitt
luogo allo fcarto, ed intanto fi
ammette lo fearto nella prima
mano, perche quello che deve
{cartare puol dire, che la carta
potta ful tavolino del gioco elie
FP ha pofta con intenzione di
{cartarla, ma quando quellas
earta ¢ tale , che non puol effer
{cartata, allora non fi ammette
piu fearto, perche quando la
€arta ¢ in tavola fcoperta Ini
non € pit padrone di ripigtiar-

fela in mano, e dire non ho gio-~
“eato fe non incafo divoler ri-

mediare ad un rifiuto,ma per ri-
mediarlo deve prima ripigliarg
in mano la fua carta, che fa ful
tavolino , alfrimenti fe toccara

D 5 a S10

 

 

 
 

 

 

 

 

82

& giocare 4 lui,e fenza ripiglial
prima Valtra carta, ¢ lafciera 0
tavolala feconda carta, allor4
sintendera rigiocato , ¢ rifiuta:
to,ed oltre il pagare la pena de
rifiuto dovra raggiuftare 1a ba
fe obligando il rifiutante a 1%
pondere adequatamente » eth
pigliarfi in mano il {uo Taroc
co: fe poi dopo il rifiuto folls
paflato pil d’una mano , alloré
fenza pil agiuftarfi la bafe, !
continuara il gioco fino alle
fine. i
fed é ben ginfta quefta rage
ne , perche con il riftuto alcunt
yolte fi puol caufare il preg
dizio dipit ditré refti, com
diremo appreflo, coftando
refto di fali feflanta punt on7)
non ¢ maraviglia, che chi 1}
ta fii obligato alla pena © 7

folirefti per ogni volta. ae

 
wet OT a Ge ht Po eG OSE

83

fiuta, edi pit fii obligato a ré-
tituire la carta d’onore prefa
aglavverfarii conil rifiuto, ma
deve perd avvertirfi, che non
puol dirfi rifiutato piu di una
fol volta quando uno comincia
Arifiutare, ¢ continua rifiu-
tare fine allultimo, tutto che
i rigiocafle di quella fpecie per
la qual’é fuccefloil rifiuto quat-
tro,  cinque volte, poiche non
fi puol mai dire atto confumata
fino a tanta, che non fi fcopra
e ficome il rifiutante non é obli-
gato a pena veruna, fe primas
non fi {copre i rifiuto, cosi con
la fperanza, che non debba {co-
prirfi puot continuar a rifiutare
fino all’ ultimo; Se pot fi {co-
priffe prima che finitca il gioco, -
¢ fi rimediafe nell atto ifteffo,
che fi fcopre, e quello cid non
oftante rifiutaife la feconda vol-

D 6 ta

 

 

 

 
 

 

 

 

 

 

 

84
ta fopra l'ifteffa fpecie » alfora
deve di nuovo pagar la pena, {e
viene {coperto, perche eflendo
il fecondo rifiuto atto nuovo s€
diverfo dal primo , dev’ancor4
Ini avere la fua pena come i
primo.

$9 bene , che a prima vitta vi
fembrara troppo rigore il dovet

pagare la pena, ¢ reftituire 14

carta d’onore, quafi che per um
folodelitto fi abbiano a pagaté
due pene; ma fe confiderareté
al pregindizio, che puol caufare
fon certo, che direte , che ¢

-fomma'giuftizia, perche la pe

na pofta al rifiuto é per punile
la malizia di chi lo commettes
ed ilreftituire Ja carta donore
fi deve fare , perche non fi pue
con frode rabar quella cart
aglavverfaril, € poi col rifiuta-

-re alle volte fi poflono aver
p}

 
 

5; —

pid di tré refti, e fe il contrario
non fe ne accorge , quel che ri-
fiuta non é foggetto 4 pena al-
cunas onde in quel cafo torna-
rebbe a conto il rifiutare, poi-
che alla fine non pagharebbe
che dui foli refti.

_ Figuratevi d’avere in mano
da cinque, 0 fei carte folamen-
te fra” quali abbiate li Re @
oro, la Tromba, quattro al-
tri piccioli Tarochi , ¢ che toc
chi agiocare 4 voi. Voi gioca<
rete il R¢ d’oro , il fecondo che
fi trova avere il 30, in mano
fenza averlo mai potuto fares
‘per dubbio di non poterlo pit
fare, Oper altro fine,non hayen-
dodenari gioca il 30. ilterzo
che ha il Sole, col motivo, che
il 30. € una grancarta, per il
troppo gioco che puol fare yj
mette ilSole, Pultimo poiche

ha

 

 

 

 
 

 

 

 

ha il Mando,tutto che abbia de-
naro per non perdere iltrenta»
e prendere il Sole gioca il Moan
do, ¢ rifiuta,quale per altro non
farebbe pit in ftato di poterfele
fare. Hora con quefto rifiuto ha
jmpedito la verzicola di Trom
ba, Mondo, ¢ Sole,di dieci, ve
ti, trenta , ¢ quaranta, ¢ di tren
ta, trentuno, ¢ trentadue, ed 10
vece ha fatto verzicola di Mot
do, Sole, ¢ Luna,. dt yentiott
velitinove, ¢ trenta, edi tre
Re; ora ditemiun poco quante
importa quefto rifiuto , ¢ fe fa
rete bene il conto,trovarete che
importa il fuo conto giufto.Du-
cento ottanta quattro punti che
yo! dire quattro reftis quaral
taquattro punti; ora non Vv} pat
egli giuftizia che chi rifiuta > 0”
tre la pena delli due refti,debhs
sacora effere. obligato or :
ae

hia

 
1 OW
tuire le carte d’onore prefe col
il rifiuto,baftandole folo Vaver-
vi impedita. 1a- verzicola di
Tromba Mondo, e Sole, altri-
menti bifognarebbe confeffare
che il rifiuto portafle pit tofto
vantaggio, che pena al rifiutan-
te,quando per altro la pena fta.
ta pofta per caftigo, enon per
utile del delinquente; e ficcome
in quefto cafo andarebbe necef-
fariameate reftituito il Sole af-
fieme con tutte fe altre carte ad
efclufione del Mondo, cosi ad
effetto che nan fuccedano {con-
certi, é fi abbia ogni volta
confiderare il pregiudizio che
puol portare il rifiuto é ne-
ceffario ftabilire per regola cer-
ta ,ed infallibile, che ogni volta
che fi rifiuta, e col rifiuto
prende qualche carta d’onore)
dalla parte contraria quella con

tut-

 

 

 
 

 

 

 

 

 

tutte Paltre carte debbano fem-
pre reftituirfi, edi pit pagarfi
fa pena,mentre non fi deve ma!
permettere cofa che con ffo-
de pofla efler’in vantaggio di
chi ka commette. |

Avvertite perd che nel gieca-
re il Matto mai fi puol rifiuta-
re per parte di chilo gioca, al
corche aveffe in mano altre
carte di quella fpecie che fi glo
ca, perche quella ¢ una cart
che deve eflere giocata da mat:
to, cioé quando fi vuole, c fi
tutte le figure che vuole chi J
gioca,con quefta fola diftinzion®
che mai puol prendere , anco™
che vi foffera in tavola le catté
pitt inferiori del gioco, 4 fegn?
tale che fe tutti foflero cafoat!»
cioe havefsero potto le fue cart
in tavola, e chi ha il Matte
tenefse per ultima carta, ee j
tie fecatte intavola pit hon»
giocano, in quel cafo non po-

tendo il Matto far bafe, s'inten-

derebbe che Pultima 1a facefse

uno di quelli che hanno gettate
le fue carte in Tavola .

Quefta carta del Matto non S
vero che debba fempre giocarit:
per Tarocco , come commune=
mente s'ingannano molti, per-
che fe cosi fofse bifognarebbe
dirloTarocco,ed efsendo Faroc-
co dovrebbe efsere nécefsaria-
mente a tutta la Cartiglia fupe-
riore almeno; il che non € ve-
ro; onde fe non éyero che fia
fuperiore alla Cartiglia, non»
puol dirfi Tarocco, € non pote-
ndofi direTarocco,deve poterfi
giocare quando fi vuole .

Puole 4 quefte gioco darfi P
accidente, che quello che ha il
Matto in mano mon abbia occa-

fio-

 

 
 

go
fione di pigliarmi , e fi trovi al-
la tine del gioco fenza carta da
poter dare in vece del Mattos
ed in quel eafo deve dare il Mat-
to iftefso, e tutto che fia carta
donore, non per quefto fi fegua
1a Mortespercheé carta che non
puol mai morire, fe poi fofse
obbligato. a dare in vece del
Matto una carta @onore per
non averne altre, allora si, che
fi fegna la Morte di quella cars
tad’onore. Accadé una volta»
ache quello che aveva il Matto
aveva ancora le Trombe gi0-.
candofiin tré ad ognuno per fe,
non fece altrabafe,; che quella
della Tromba, con la quale pre-
fe i] Sole, ed il Mondo; doman-
dava allora Pavverfario la carta
del Matto, e quello voleva dar-
le Piftefo Matto » come meno
Pregiudiciale a fe ftelso » per lo.

che
1
che poftofi al dubbio in diteuta
i communemente da’Giocato-
ri pid periti,e piu efperti rifolu-
to, che fofse obligato a dare una
carta a fuo arbitrio,ad efclufion
Jel Matto, e della Tromba, con
‘1 folo motivo che quefte dues
carte fono imperdibili; onde
non puole alcuno privarfi d’una
di efse a fuo arbitrio per fuo
maggior vantaggio, e pregiu-
dizio dell’altrt .

Dovete pero avertire, che in
un folo cafo non ¢ lecito giocar
4 fuo arbitrio il Matto,perche fi
tratta di dover rifpondere carta
neceflariamente obbligata, e»
quefto fuccede quando uno de’
Giocatori giocafle per la prima
volta d’una Cartiglia, e quel
che viene appreffo non avendo
di quella fpecie che fi gioca met
tele Tarocco, allora quello che

ha
 

 

 

 

 

 

 

- teffe effer’fagliato il {uo Rey

92
ha il Re,fe non ha ancora g10¢
to, neceflitato a darlo, anco
che aveile altre carte di que
{pecie, 6 pure avefle il Matt
ma quefta neceffita non obblig
fe non chela prima volta che
gio ca di quella fpecie , ¢ prit
che fia in tavola ilRe, fia ven!
to unaltro che habbia rifpo
Tarocco,fe poi o per inaverte!
za, © per genio quello che ha
Ré non lo ha voluto gioca
ma ha voluto pit tofto gioc#!

un altra Cartiglia della mee
ma fpecie col dubbio che lipt

a

lora non é pitiobbligato #

Jo, fe non quando non have
pit di quella {pecie, € fi trovall
fenza Matto; mentre fe fig!
cafse di quella fpecie per no
rifiutare , allora deva dare iB
per neceflita . Pit-

a,
Parmi che tutte quefte fiano 1]
regole generali pil neceflarie |
ad impararfi per ben appren- |)
a dere quefto gioco , ¢ ficcome 10 |
i 4 principio mi fono ideato di i
4 defcrivervi folamente le regole |
~ generali , Cosi crederei d avere |
M Abaftanza fodisfatto al mio obli- |
" go, e€ fe voi imparerete bene a |
0 memoria tutte quefte regolen |
“ potretedire converita,chein- =|
i tendete il gioco , febbene non.
> faprete giocarlo a perfettione ,
mentre il pretendere d’impa- |
rarvi quefto gioco a perfettione 1
fenza una ben efatta prattica , |
“  farebbe temerita, effendo le re-
~ gole del giocare diverfe das
¢ quelle del gioco, nafcende que-
¢ = fte dalli continul, e varii acci-
1

 

denti, che accadono, e ficcome
di quefte ne @ fola maeftra la
prattica , cosi ne hd lafciato la
cus

 
 

|

Ha cura ad effa per ben impata
a vele; Le regole da me fino
| defcritte, come che apparte
Lei gonoal puro gioco, fono inv
a) riabili , ma quelle del giocaté
| poffono,anzi fi devono alle Vv!
Vt ei te variare fecondo le covtl
‘aia genze de’giochi , che accado?
Lil ii Dird bene peronon per mo!
| i vodi regola generale, maP
ai raggione del modo di giocat
Hh che &fempre meglio di farf
onori alla mano, fe fia poflibil
| maflime lipiu gelofi, ma
|| yolte é pit efpediente il git
HHI verfo del compagno, tutto ©
rn _ fia quafi ficuro di perderne 4
Lh tl cuno,quando per altro con qu¢
Hi la perdita fifpera di fare mag

| gior vantaggio a fe fteflo, 4
it volte ancora potendofeli fa
i alla mano, ne meno¢ dove
i farfeli, perche con no? far
i puo

 

 

 

 

 
95
puo fperare di far un gioco mi-
eliore » ma tutto quefto dipen-
de da quelli accidenti , de’ quali
ne puole effere fola maeftra la
prattica, & il pretendere di de-
{criverli farebbe pazzia .

Le carte pid gelofe da farfi
fono quelle, che poflono fare
maggior gioco, per quefto uno
30.35, S0le;€ Papa tutte tre fo-
no gelofiflime, perche quefte
fono le chiayi di molte verzico-
le, cioé a dire, che poflono fare
delle verzicole aflai, pero bifo~
gna cuftodirle pit dell’altre , ¢
procurare fempre di farfele. |

Ad un faglio , & una feconda
fi puol paflare qualfifia carta»
gelofa,quando pero non vi fiano
{carti, ¢ non vi fia forma di po-
terli fare con pil commodits

alla mano, ad una terza di raro
fi pafla, ma quando foffe fopra
Ta-

 

 
96 /
Tarocco,con piii giuftizia fi puo-
le arrifchiare , e fe poi la necef-
fita del gioco Vobligafle a pat -
fare,quaniunque non foffe fopra
Tarocco, fi deve fempre pro-
curare , fe fi pud, di paflare una
carta dVonore grofla , accioche
per pigliarla ce ne yoglia un’al-
tra, ma fi deve pero ayvertire »
che non fia carta troppo gelofas
perche fe é carta gelofa l’avver-
fario, fe puole, ci va ficuro ; {e2
poi non € gelofa-per tema di
non perder Ini, forfe non ci an-
dara , enon lacoprira , e quefta
diverfita di regola V’ha da im-
parare dal modo digiocar deg!’
altri, e la difpofizione delles
carte , che fi hanno in mano »
mentre da quelle devefi cono-
{cer il modo di contenerfi. |

Se poi il gioco andaffe in fac~

cia al {uo compagno, ¢ fofle,

ficu-
ec

=e

; 97
ficuro , che lui deve rifpondere
Tarocco , allora fi puo girare
a lui qual fi fia carta d’onore ,
anche delle pit gelofe , fe perd
hon fi potefle dubitare,che gio-
cando quella carta poteffe poi
impedir 4 Jui qualch’altra carta
di maggior confeguenza, per~
che in quel cafo verrefte a per-
der tutti due, come per ragio-
ne defempio fe voi girafte il
trenta, & il voftro avverfario
giocafle il Mondo, & il voftro
compagno aveffe in mano il So-
le, ecco che voi perderefte il
trenta, & il voftro compagno
non puol fare il Sole, mettendo
a rifchio di perderlo,come pro-
babilmente puol accadere, e»
cosi vi mette a rifchio di perder
tutti due, due carte gelofiffime;
pero nel girare bifogna ancora
andar guardingo, ¢ prima di gi-

| rare
 

 

 

 

 

 

98
rare le carte pid gelofe bifogna
procurare di girar carte gelole
si, ma meno gelofe,ad effetto di
{coprit il gioco » poiche fe aves
fe da farvi una verzicolas © di
pir aveffivo il trenta » & il Sole
prima deve girare qualch¢
carta della yerzicola,e tutta all

cora fe fia di bifogno» © quana¢
yedete , che non va sula carte
di mezzo della verzicol e fe
gno, che non ci puoles ed allo
ra fi puol girare il trenta, fe p?
la voftra verzicola fofle di Sole
Luna, ¢ Stella, allora 100 mm)
lita pit quefta regola, pete
vor avverfario cercara femp!

di rompere 4 voi la yerzicola
farla lui; cosi in quel cafo 1
di fat.

vra fempre procurats
Sole , ¢ la Stella» ¢ perderé p

tofto la Luna -
"Nel girarle carte » che fare
3 ¥ yvert-

————
aS Od =— 1 wv geet, OF OS ee a ie

=

 

    

99
verzicola bifogna andar molto
guardingo; ma quando non fi
potefiero far alla mano, e che
bifognaffe girarle fi deve procu-
rare di girar quelle per le pri-
me, che non poffono far agl’av~
verfarj, e quelle, che poflono
far verzicola agl’avverfarj ave-
te da procurare di farvela alla
mano, o'pure di farle con meno
rifchio , che fia poffibile .

Quando voi a principio ave-
rete verzicola dovrete procura-
re giocando di farvi prima le»
carte pili gelofe, e le carte pil
gelofe delle verzicole fono le
carte di mezzo, ad efclufione
dell’uno , ch’é gelofiflimo , per-
che fatte le carte di mezzo , al-
Jora non fi puol pit. perder la
verzicola fe fofle di quattro car-
te, perche col perderne una
fa altra alla mano; ma prima

E 2 pero

 

 
 

 

 

 

100
pero di girar la carta di verzic
la fideve procurar di toglier ¢
mano al fopra mano tutta [a
cattiglia, ad effetto, che gio
cando effo non fiate obligato |
rifponder cartiglia, ma pofliat
farvi le carte a voftr’arbitrio
perche altrimenti correrefte }
rifchio di perdere due carte
verzicola; onde prima e nece!
fario, anzineceflariiffimo d’im
parar a giocare Ia cartiglia; ©
per ben giocarla fideve proc
rare di giocar prima tutta que”
la cartiglia, che fi pud fuppor!®™
che quello , che fa fopra ma?
abbia in mano, e fe viaccotg™
fte, che di qualche fpeci¢ 90?
ne havefle pit , tutto che vol #
avefte affai, dovete fempre pt
curar digiocarla per Pultim??
accioche voi giocando quel

gartiglia oblighiate il voftro 1”
e pra
}“

a eral SS

eoT_ rw

Pultima mano, la

10f
Pra mano a giocarvi in faccia,
€ voi farvi quelle carte d’onore,
che pid vi piaceranno.

Quefta regola perd non é

¢mpre certa ye non deve prat-
ticarfi ognivolta, ma deve,
pratticar folamente in cafo,che
Voi avette degl’onorj aflai da
farvi, enon pefte la forma dj
poterli fares fe poiavette pochi
onori, O pure avefte altra for.

soos eT Sc eau naa yh
he ar potcrirrare + anora dove, .

te procurare prima di glocare
tutta la cartiglia, della quale ne
fofle faglio jj voftro fopra mas
NO, accioche non Poteffe poj
farfi g?onorj alla Mano, ma do-
vefle farli tutti col {uo rif.
chio,

Di pit fe voi avefte Ja Trom.
ba vi fervira ancora Per potey
con pill facilitg artivare a far

quale conta

3 die.

 

 

 
A 2
dieci punti, pero bifogna diftin |

16

guer prima il gioco per faper .
ben giocare la cartiglia, poiche ,

alle volte é pit efpediente i]

iocare prima di quella carti-
glia, della quale fe n’ha pitin
mano, « alle voltee piu efpe-
djente ilgiocar diquelle, che
fe ne hameno.

Tutto il maggior ftudio , che
fi deve fare in quefto gioco é
@ impedir agravverfaij it pos
terfi far gVonori, fi che dovete
fempre ptocurare di non la~
{ciarglieli fare alla mano, e>
quefto lo farete col procurare
fempre di giocar quella carti-
glia, che lui poffa aver in ma-
no, e quando fofte ficuro di non
poter giocare cattiglia, allora
dovete procurar di non effere
piu obligato a giocare , fe perd
non vi ci obligafle qualche car—
, ta)

 

ang} Saal © Pais; ell tte eS ee Bete Lee ee kp

 
 

103
ta gelofa , la quale vi obligaffe
a prendere per farvela; poiche
tutta la bellezza di quefto gioco
é il procurare di far la caccia
all’avverfario, e per farla al-
cune volte torna a conto i! per-
dere qualch’onore , per pren~
derne delli maggiori .

Ma prima di procurare di far
caccia all’altri,¢ pitt neceflario il
difendere fe fteflo , ‘pero dovete
fempre procurare, che la gioca-
fa vada in faccia al voftro com-
pagno, accio pofla farfi alla ma-
no quelli onori, che fono piti ge-
lofi, e che li poflono portares
pitt pregiudizio,e maggior van=
raggio all’ Avverfarj , mentre in
juefta forma provarefte dues
rantagei, uno di tenere in fog«
rettione linimico, e l’altro,che
| voftro compagno fara quelehe

ruole .
Eq Quan-

 
104
Quando avete Ja tromba,tut-

tala vofira maggiore attézione |

deve eflere ditenere in fogeet-

tione l’inimico , e fe potete far—_

lo, dovete far in modo , che Iui

non pofla accorgerfene; perd _

non fta bene il fequeftrarfi 4

principio un Re nelle mani, ma_

dovete veramente giocarlo, fe
perd non potefte credere di po~
terlo falvare col mezzo del vo~
firo compagno nel progreflo
del gioco; fe poi il faglio vi fof=
fe fopra mano, allora potete poi
tenerlo con pill giuftizia, ma fe
é fottomano., allora e cofa mol-
to pericolofa a riufcir bene,per-
che date campo alPinimico di

 

paffare sti quella cartiglia ogni.

volta che carta vuole, pero &
meglio in quel cafo il giocarle
fubito.
A’ principio quando havetes
da

4
es
10g

da prendere con Tarocco doves
te procurare fempre di prende-
re con li Farocchi pit grofti,ma
dovete farlo in modo di non da-
re ammirazione all’Avverfario;
altrimenti vi ridurrefte alPultj-
* mo con tutti li Tarocchi grofii ,
e farefte per confeguenza gb]j-
gato a prendere , e mandare las
giocata in faccia all’ Avverfario
il che € un’errore folenniffimo «
mentre alle volte non folo nop,
fi deve prendere con Tarocchi ,.
ma ne meno con onori, effendo
pit efpediente il perdere un?
onore , che prendere , perche in
quel cafo fivicne 4 guadagnar
pitt di quello nea fi puol perde-
re, ma di guetta regola Ia fola
attenzione ve ne puol efferes
maeftia .

Se con la Tromba avefte an-
cora della gran cartiglia,¢ nom

: ES avele
106
avefte forma di poterla fcarta~
re , allora dovete fare if conto,¢
vedere, fe vi puole riufcire dt
giocarla tutta con fperanza dt
arrivare all’ultimo con la trom-
ba, € inquelcafo dovete pro~
curare, e far tutto ik poflibile di
giocare prima tutta quella,della
quale ne havete maggior quan=
tita,ancorche foffe certo di fare:
il giocodellinimico, ¢ NOD fa-
xe, come fanno alcuni, f quali
non la vogliono giocare fino all*
ultimo per tema di fare il gioco
dell’ Avverfario , e non rifletto~ _
na, che fe non la giocano.a prin
cipio , bifogna giocarla in fine 5
& alloraé peggio, perche oltre
di che ficorre il rifico di cafca—
xe con la Tromba 4 mezzo gio—
co, fi da campo all’inimico da
farfi quelle carte d’onore » che
forfi non fi farebbe fatto a prin~
Gi 5

|
 

10
cipio , € Cosi fi toglia l’occafione
di poterli prendere qualch’ono=
re, che neceflariamente have-
rebbe perduto..

Con. la tromba é€ fempre bene
tenere ancora un Tarocco gro&
fo , perche in cafo che fofte obli«
gato a giocare la Tromba per
motivo di prendere qualch’ono.
re, potete ancora avere la fpe=
ranza di fare Pultima mano, la
quale conta dieci ponti, ma av-
Vertite 4nontenerne pid d’uno.
per non effere obligato a pren-
dere , € quando non aveftte al-.
tro, che un’onore di carte grof-
fe, quello ancora dovete tenere,.
ma. dovete pero procurare , che
quel’onore non vi faccia verzi-
cola, 6 non la faccia all’Avver-=
fario, perche in quel cafo é be-
ne farfela prima.

Di piu fe avefte la Tromba ,.¢

E 6 avet-.
Bi ‘708
avefte da f{cartare, dovete rego-
lare lo fcarto in modo, che vi
torni pit acconcio, e dovetes
prima riflettere alla quantita, &
alla qualita del?onori , che ha-
vete nelle mani, & ilmodo di
farli, e poi regolare lo {carto it
modo di potervi fare li voftri
onori,e dare foggettione all’ ini-~
mico; ma prima di ogni cofas
havete da riflettere al danno vo-
| ftro , e fe conofcete , che vi pof-
fa tornare in acconcio il (carta-
| - ge di quella cartiglia della qua-
| | il le ne havete piv, lo farete, altri-
| Hl menti {carterete di quella, del-
Ha la quale havete meno , e que-
Hal ftaregola ha piitofto per fuo
| fandamento la prudenza, ches
ogni altra cofa .
| kl Mondo poi vi da Vifteffa re-
gola della Tromba tanto nello
fcartare , quanto nel modo def

gl0~

.
k————e
199
giocare con quefta fola differen:
za, che il Mondo é inferiore al-
la Tromba, e per quefto fi dice
per proverbio, che il Mondoe
fatto per perderfi tutto, che fi
debba procurare di perderlo
meno che fia poffibile; e fe bene
fi fuole per il pit portare quefto
fino alla fine del gioco, cid non
oftante quefta regola non é fem-
prevera; petche fi deve fola-
giente tenere fino atanto, che
uno ha fperanza di fare con il
medefimo un gran gioco cok
prendere qualche carta gelofas
all’inimico , 0 pure per difefa
del fuo compagno , il quale ha
| qualche carta da farfi, € non.
puol farla, che col girarla fo-
lamente , ma quando man¢ano
quefti due motivi , allora non e
pid raggionevole il tenere il
Mondo fino all’ultimo , e per
gue-=

 
 

10
quefto vi € neceffario oltre l’at-
tentione per ben faper giocare
una buona memoria , perche bi-
fogna ftar attenti 4 tutte le car-
te d’onore , che fifono giocates
€ ricordarfi di quelle, che fi han-
ho ancora da fare per vedere ,
fe torni il conto iltenere a rifi-
co il Mondo ; perche la fperanza
delPonore , la quale puole obli-
Gare a tenere il Mondo, deve»
eflere tale , che eguagii il rifico
della perdita, fe non intutto,,
almeno nella maggior parte, al.
trimenti non torna conto il te-
nerlo col timore di perderlo.
Ad effetto che il rifico paris
gli deve effer tale, che fi {peri
di poter togliere all’inimico
qualche carta di gran confe-
Suenza, la quale quantunques
bon importi tanto, quanto pud
le portare il Mondo, importi

PS"
Bh tecnica
~~ 8 1 rt
perd affai, 6 pure per tema
-che il compagno fuo non abbia
da perdere qualch’altra carta,la
quale pofla fare un gran gioco »
6 pure che abbia Ia probabilita
di potere co! Mondo fare per fe
qualche verzicola, col prendere
qualche cofa d’onore all’inimi-
co; altrimenti fenza quefti mo-
tivi nop torna mai conto 1 te-
nere-il Mondo finoal fine, fes
non in cafo che foffe certo: di
non poterlo perdere, che per
un qualche accidente impenfa-
to; &alle volte é pit efpedien-:
te di farfi il Mondo, ancorche fi.
abbino inmano altre carte di
onore da farfi, perche ¢ aflat
piu quello, che fi perde, perden-
dofi il Mondo di quello, che»
nole guadagnarfi, prendendo
ancora qualche onore all’ inimi-.
co; Jaonde perche il-rifico nom:
e egua-
 

 

 

 

 

 

 

 

112
éeguale, né meno deve eller
eguale il pericolo.

{lSole, che é carta affai p!
gelofa del Mondo. deve effere
anche giocato con pili giudiai¢
prima perche a lui crefce up-
inimico , eflendo foggetto.4
Mondo, & alla Tromba » ¢ P°
perche é carta, che puol fare
piu d’una verzicola,& ognl yer
zicola ove entra il Sole cont
il doppio delaltre, perche que
fia non puol contare meno
trenta ponti, e Paltre poflom
contare folamente quindeciso"
de con quefto motivo deve elle
re giocato con molta attene
me, & alle volte fara bene il 4
felo fubito , alle volte fara be |
il tenerlo fin’ all’ultumo 5 be
bifogna fempre faper diftungtt
re il gioco, e fecondo le con

genze regolarfi « ii
Ne) =?

9

S»

j=. >

S bt weer oe BN

~&

RIES PP

~ 3%

  
 
 

IT?

Se voi avefte 1] Sole, & ave-
fte ancora pochi Farocchi con
una cartiglia malamente diftri-
buita ,e dubitafte di potere alla
fine perdere il Sole’, allora alla
ptima congiontura, che vi ca~

piti dovete farvelo, ma prima:

peréd dovete riflettere , fe quel-
lo, che vi fta fopramano fia pia
capace a giocare cartiglia, las
quale voi abbiate ancora, per=
che fe fofte certo, che non po-
tefle pit: giocare cartiglia, allo-
ta potrefte tenere il Sole un po-
co piu, perche potrete {perare,
che la gioeata v’abbia a venire
injfaccia, e fare il Sole a voftro
arbitrio, € poi potrefte ancora.»
difendere il voftro compagno
in cafo, che vi giraffe qualche
carta d’onore, ma fe poi non
avefte quefta probabilita » allo~
ta dovete neceflariamente farvi

il

 

 
Pare

 

 

 

 

 

 

 

 

114.

i] Sole tutto che il voftro com
pagno avefle una verzicola
farfi, perche dovete confidera
re, che puol effere maggiore i
danno,che vi caufarebbe la po
dita del Sole, del vantaggior’
vi potrebbe portare il fare de
la verzicola, ¢ pol veunaltl

gagione, Xe, che il yoftro in

mico vedendovi fare il Sole,
fapendo, che i] voftro compé
gno hala yerzicola da far fi»
ye neceflariamente dubitate

Ja Lu}

che vol abbiate almeno

“per difefa della verzicolas ¢ ©

si non fi azarda con tanta ac
tia venire su la verzicola»
non in cafo che abbiate und de
Ie carte {uperioti a tutte «

Se poi avefte con il Sole U
Cartiglia ben diftribul
una mediocrita

altri onor! , allora

 
= & No

\~ wo i OD

i
Cf

11g

re il Sole un poco pill, ancorche
il voftro compagno non avefle
verzicola, e dovete con pru-
denza deftregiarmi in modo che
vi capitila congiontura Gi pote:

- prendere qualche onore all’ini-

mico, edevete riflettere, che
{€ vicapita la congiontura di
prendere qualche aria inferiore
ai Sole , li dovete tirare , fe pe-
ro non fofte ficuro di perderlo,
6 pure vi accorgefte , che il
rifico non é eguale ; ma quando
il rifico foffe eguale, Odiverfi-
ficafse folo in cinque punti tan
to ci dovete ritirare,perch’ aria
per aria fempre deve rificarfi, e
fe poi vi fuccede male non per
quefto averete giocato male,
perche il male folo fara proce-
duto da un accidente, ed al con-
trario il giocar bene procedes
dattenzione.

s Als

 

 

 

 
 

 

116

Alle volté per provare an-
cora fe il voftro fopramano
habbia carta fuperiore al] Sole ,
dovete prima girare al fuo com-
pagno qualche carta gelofa, in
tempo perd,che quello fia obbli-

“ gato a Biocare Taroco,e febbene

avetti avutg,tempo di potervela
fare,dovete non farl4 per afpet-
tare a girarla, perche feil {ua
fopramano ha carta fuperiore ,
allora deve tirarci neceflaria-
mente, e fe non vitira é fegua
che non ci puole, ma fe poi vi
tiraffe con qualche altra carta
inferiore,la quale il voftro com-
pagno non potefle coprire allo-
ra potendo dubitare che poteffe
ancora su il Sole , dovete {ubi-
tamente farvelo, ¢€ cos} potrete
ancora regolarvi rifpetto all’al-
trionori dopo il Sole. _ ,

Doverete pero avvertire che

{c
 

114
fe per voftra Magesior carta non

avefte che un folo trentadue 59
trentatre, © altra carta poco
gelofa, tutto che avefti un taglio
no per quefto ve la dovrete fubj-
to fare ; ma dovete tenerla , an-
corche potete forfi dubitare di
perderla, perche dovete feryir-
vene per potere ricuperares
qualche picciolo onore del vo-
{tro compagno, in cafo che non
haveffe potuto farfelo sq Ja car-
tiglia, ecosi fino atanto che
non conofcete, che il yoftro
compagno non ha in mano pia
onori piccioli da farfi dovete
fempre tenere la forma di po-
terli ricuperare , fe poi quel
voftro onore potefle fare ver-
zicola agl’avverfarj , 6 pures
a voi potete procurar di far.
velo.

) Al modo di conofcere fe if

coms
ty i
118
compagno abbia pit: in mano
onori piccioli fi é il riflettere fe
gli abbia potuti fare alla mano »
enon li ha fatti,o pur fe li abbia
potuti girare a voi, enon li ha

girati, quefto fara fegno eviden—

te che non li ha , perche lPonort
piccioli che premono , non po-
tendo pit farli fenza l’ajuto del
compagno deve neceffariamen-
te girarli fe puole .

Chi haveffe Mondo, e Sole in
mano non deve mai farfeli tutto
che fofle ficuro che fotto mano
vi foffe la Tromba, fe non in

‘cafo pero che li faceflero verzi-

cola, petche la perdita @una di
quelle carte importa fubito un
refto , altrimenti deve afpettare
che licapiti la congiontura di
tirare 4 qualche onore, e dove-
ra tirarli con il Mondo; mentre
allora é ficuro di pigliar quello
ono-
  

It
onore all'inimico, 6 di fark ik
Sole, con la {peranza di poter
prendere qualche altro onore
alPinimico , e di far Pultima.

. Alle volte voi non avretes
in mano alcuno onore » € pure
dovete tenere il gioco in tanta
ftima che havete da procurare
@ingannare Pavverfarij, e non
fare come tal’uno che fubita
fanno conofcere di non aver
niente,perche con {coprire il lo=
To gioco danno campo all’av=

verfarij di fare quel che voglio-.

no ; al contrario fe voi procura-
te di far credere al voftro inimi-
mico d’ayer un gran gioco lo
lotenete in tal foggezione , che
procura di farfi tutti Ponori al-
Ja mano, e voi con bella manie=
racol moftrare di fare un gio-
co di Tromba dovete procurare
qi farglieli fare ; ma in modo
pen

 
320

pero che non fi accorga che voi-

lo -facciate per debolezza col
procurare fempre di allenta-
re, ed in tanto prendere pct
mandarle il gioco alla mano col
far moftra di prendere per for—
za ,perche in quella forma fare-
te ancora il gioco del vouro
compagno con abbligare il vo-
tro inimico a giocarle in fac-
cia, ¢cosi lui havera campo di
poterfi fare quelli onori che
yora .
Ma col procutar d’ingannare
Pinimico , non dovete pero iD-
gannare il compagno >» perche
potrebbe apportarvi up grave
pregiudizio, percio quando non
averete niente, edaverete pro-
curato d@ingannare il compa
gno quando fiamo vicipi alla»
fine del gioco dovete far cono-
fcere al voftro compagno ,, che
vol

 

TC

et

pr
qu
{a1
V

in,

Ct

re
1non havete gioco, ¢ lo fa-

f¢ col prendere alcune volte,

arli vedere che potevate non’
-ndere , perche fe prendete

ando potevate non prendere

‘a fegno evidente,che voinon

lete pitt dare foggezzione all?

imico 5 € cosi dovra cono-
sre che non avete carta da fa-

la caccia.

Di pit fe vi pareffe dal modo
giocare che il voftro compa-
10 poteffe avere la Tromba, 6
Mondo, allora voi dovete»
mpre girarlili Tarocchi pit
‘offi ancorche foffero onori, ¢
‘ocurare dinon lafciarlo pi-
iar mai, accio non foffe obbli-
ato a giocare in faccia al fuo
imico; maficcome potrebbe
arfi il cafo che faceffe il gioco
i Tromba, e€ veramente non
aveffe,allora pet ben coup ieee
ae

   

23

  
      
  

 
 

 

 

 

 

viet
“0 5 }
yy

 

 

 

We SI ee

{22
lodovete fempre oflervare I
carta che rifponde , perche £
principio comincia a darvi qué
che Papa , quando per altro Pp

-trebbe darvi carta maggiore,

il Papa non vi facefle yerzico
potete fortemente dubitaré, ¢
effo non Pabbia; ma fe po%
daffle qualche Tarocco picciol
che non. contaffe porendovel
dare qualch’ altro maggiore,
lora é fegno certo, che nor Vk
ma folo fi ¢ pofto in caccia P

“qngannare [inimico;onde vol

lora dovrete fare il voftro 810
€ non pit quello del compagh
fe poi fofte quafi certo, ch’ave!
la Tromba il voftro compag”
e fuflimo all’ultimo in tres
quattro carte, e yoi non have
‘che un folo Tarocco » ed il ref
Cartielia, ¢ quel Tarocce fo!

fe non fol

‘ancora un onore » a
= 2 ae pu

eae
  
  

pee
pit che onore gelofo,¢ toccafle
giocare &voi per fare il gioco
del voftro compagno,dovete pia
tofto giocar Ponore, € non mar
la cartiglia, fepoi fofte certo
che avelfe il Matto in mano, al-
lora potete non giocar Tarocco,
e giocare la cartiglia perche
il Matto @ lui gli gioca come
vuole, e potete non arrifchiare
il voftro onore .
Se viaccorgefte ancora , che
il voftro compagno avefle in»
mano qualche picciolo onore »
il quale non avefle mai potuto
fari, allora dovete procurare di
andatle incontro con li Taroc-
chi pitt grofli, ancorche foflero
onori, ma fe dubitate d'un gran
gioco negl’ avverfarii , dovete
andarle incontro folo con Ta-
cocchi grofii, e non @onore per
non obbligare Pinimico, ad im-
Fa pes

 

 

 
 

 

 

 

 

 

 

 

 

 

 

 

 

124
pedire al voftro compagno ¢
qualche onore fuperiore il da
vi quel piccolo onore,e cosi vé
reflivo a perdere il voftro 0?
re per averlo girato al comp
eno, & il voftro compagno Vé
rebbe a perder un’altro on0!
per non averfelo potuto fare
alla mano, e quefto fuccede 4!
coratalvolta {enza che pailia
alcun onore , perche quan Q
yoftro inimico pud creder » cl
voi non abbiate piu carta done
re gelofa,ma credendo che ‘al
bia il voftro compagno, ¢ fapen
do, che ilfuo compagnoh i
Tromba,allora,ancorche {j g10
chi cartiglia, ¢ 200 yj fia onor
alcuno, gioca il Mondo, no
peraltro fine, folo perche il vo
ftro compagno non pola fare
carta donore ; Onde in quelt

bifogna ftare affai avyertito P
. no

ee —«€,,
  

125
non effere. obligatia perderes
tutti due.

Abbiamo veduto di fopra co-
me alle volte é neceflario it
gannare Vinimico col moftrare
@aver un gran gioco, fe bene
non avefte niente, ma a folo
motivo ditener in fogettiones

| Pinimico; ora dobbiamo vedere,

come alle volte ¢ neceflario di
far vedere all inimico dinon_.,
avere ne Tromba, ne Mondo,
tutto che l’abbiate, ¢ queftolo
potete fare,o con lo {carto quan.
do dovete {cartare , ocol modo
di giocarey col giocare total-
mente diverfo daquello, ches
giocarefte-fe volefte fare il gioca
della Tromba, ma doyete pero
avvertire , che quefto gioco lo
dovete fare a principio, ma
quando fiamo nello ftringere del
gioco dovete procurare di met-
F 3 ters

 

 

 
126
tervi a fegno dinon effere |
obligato 4 prendere , ma met
S allora il voftro avverfario
ogettione tale, che non po
piu farfi carta d’onore alla 2
no, maflimamente fe fofte ficu
che aveffe in mano delle cal
gelofe affai, e quefto lo dove
fare , {pecialmente quando h
vete eran quantita de Tarocel
edubitate di dover eflere P
in neceflita di prender molt
| __ fime volte, ¢ per confeguel
ie obligato 4 giocare {peffo in 4
[| cia al voftro inimico.
In oltre fe avefte una cart
glia perfida , cioe moltiflimé
qualche fpecie,¢ fofte ficuro,¢!
afl C45 tocar
glaltri non la poteffero gioc
-allora dovete voi giocat femnt
quella, poco importa 0,
4i facia il gioco dell’inimice ae
mandarle fempre il gioco a

 

 

 

 

 

 

 

 

 

 

 

(gen
itl

[C*

fla

a".

cia, ancorche fofte ficure ; che
lui nom ne haveffe , perche: fe
non la.giocate voi fiete: ficuro ;
di nom arrivare all’ultima*con
la Tromba , € cosi vi togliete la |
{peranza di poter far caccia, ¢
vi bifognaabbandonare ikvofiro :
Compagno in mezzo al gioco ; |
~ Tu fomma tutta la maggiog - |
difficolta di quefto gioco con= |
fifte in faper ben giocare la car-
tglia , nel farfi Ponori a tem Pos
€ girare verfo del compagno 3
Tpetto: al modo di giocare be.
ne 4a cartiglia parmi davery}
detto tanto, che'batti, it refto lo
dovete Imparare da voi mede-
fimi per mezzo d’una bei efatta
Pratticas circa del farvi Ponor}
a tempo confe(to il vero, che io
Per me non sddarvilume mag-
lore. Refta ora folo 4 vedere jl

modo, che fi deve teneze nel gin
FE 4, Tada

 

 
 

 

 

 

aver memoria fifla dituttil’o

 

tare Yonori al compagna; Ed io
per me vi dico., che dovete pri
ma, riflettere a tutti Ponori ,,che
avete In Mano , € poi girare ptir
ma quelli, che fonomeno gelofi,
e dopo li pil gelofi, e fe vedeté
che il voftro inimico.non ci tira
quello fara feeno evidente , che
non cipuole, fe perd. non folle ,
ficuro, che il voftro compagn?
aveffe un onore fuperiore al fos
perche fe non sa dicerto, che it
compagno. ha/una carta fuper!o> |
re alla fua fopra Ponori gelott
deve andar neceflariamente
puole, efe non civa efegnos
che non ha, carta d’onore groflar
onde allora vi potete mettere 1?
caccia al voftro fottomano co”
una carta d’onore, anche me" |
diocre ,, ma fopra tutto cont
we re
fens

 

ti, che fono ftati giocati

—
perche fecon quella carta non
potete prendere carta che 4 voi
facefle. gran gioco, ma al con-
trario’ col perderla potefte far
fare un gran gioco aglavverfa~
tii, allora voi dovete farvi la
voftra carta d’onore, perche»
puol effer maggiore il rifchio
della perdita,che l'utiledel! gua-
dagno ,: e cosi quando il rifchio
noné eguale non fi deve arri-

{chiare , e per quefto quando il °

Sioco: noné fcoperto aria per
aria deve arrifchiarfi . 3
Ma fe fapefte di certo , 6 pure

potefte dubitare dal modo dé

giocare , che il voftro.avverfa~

Tio. aveffe un aria fuperiore alla

voftra , allora dovete procurare

difarvi la voftra allamano, fe

bene alle volte ¢ dovere perde-
re

   

flettere, che carta potete pitt
prendere con quel voftro onorey .
 

 

 

 

 

 

 

130
re un aria per dover poi far mag-
ior gioco, perche con la pel
dita ?umaria {i puol impedire
all’avverfarii il fare altro onoré
di maggior confeguenza , € pure
fe fofte in ftato di non poter pit
Zare la voftra aria , puotete,co”
dranchezza coprire l’aria, 6 pure
qualfifia altro onore dell inimir
co ,.con la certezza di farvi 14
voftr’aria , 0 qualch’altra carta
@onore alla mano ; mentre-c?
perdere un aria fiete ficuro, ch¢
al gioco vi debba venire alla m4"
No, € poi farvi quella carta che
pili vi piace. . ;

E talvolta ancora é necefla-

vio paflare una carta d’onot? »

ancorche non ve ne fia un’altras
per folo. motivo di poterfene po
fare un’altra alla mano, percn®
in quel cafo fiete ficuro» di
farvi quello, che girate, 2
 

 

farvi quello, che vi é rimafto in
mano , dove al contrario farete
obligato a perderne due , fe non
ne girate prima uno.4 tempo de-
bite, perche col reftringerfi del
gioco fi rettringe a voi la forma
di farvi le voftre carte d’onore;
laonde ¢ meglio perderne una
a tempo debito per poterfene
poi fare un altra, che eflere ne
ceffitato a perderne due in ulti-
mo per non averne voluta gita-
re prima una al fuo compagno.

Alle volte fi poffono, anzi fi
devono fcartare Tarocchi, di
quelli perd, che non contano
per poterfi diftribuire la-carti-
glia , in modo dinon effere obli-
Gato agiocare in faccia al fuo
avverfario, ma avvertite pero di
farlo con fomma cautela, ¢ non
mettere a rifchio li voftri onori,
meutre prima di penfare allas

cace
 

 

 

 

 

 

 

 

caccia delli onori altrui, bifc
gna penfare almodo di fare
fuoi , aflicurandoyi, che quand
avrete imparato 4 falvare liv’
Jiri onori, ed a procurare ¢
prender quelli deg!’inimici , f
rete un gran Giocatore , perch
tutta la bellezza di quefto gioc
confifte nel faperfi difendere di
{uo inimico, e procurare ne

ifteffo tempo di vincere il f

inimico. |
Tutte quefte regole anno
loro apendici, né fono rego"
fempre certe , mentre fi varia”
ogni volta fecondo la difpe:
zione de’ giochi, onde vo! Hi
dovrete ftar fiffo 2 quefta,
con la voftra prudenza, € ©
la voftra attenzione regolat!
fecondo vi portera il gioco: :
xo bene, che quefte vi po a
dare up gran lume per pote! 4k
eee prens

4
 

 

f)

prendére con pid facilita il gio~
co.,ma non per quefto vi poflo-
no fare un gran Giocatore,per~
che per effer gran Giocatore v1
vuole la prattica, la quale vi de-
ve imparare il modo di metter~
Te in efecuzione .

Onde parmi d’aver a baftanza
fodisfatto al mio impegno con
defcrivervi tutti quelli acciden-
ti, che mi fono faputi venire in
memoria, e poi darvi il modo,
di provederyi; ma ficome fong
Infiniti g?accidenti , 'che poffa=
no giornalmente accadere in
quefto gioco, cosi lafcio la cu

faalla prattica di ammaeftraryi_

fedelmente nel refto, baftando-
mi per-ultima mia fodisfazione,
che in quefto gioco procuriate

fempre di tener in foggettione

i voftro inimico, e che riflet-
Uate fempre ogni volta alle cay.
te

_— ets
 

 

134

te giocate, e che abbiate memo-

tia di tutti Ponori, tanto dell
glocati, quanto di quelli da gi
carfi, che vi protefto , che g107
Carete prefto beniflimo. _.
Tutti gPonori aMieme fann?
364. € con tutte le carte fan?
496. Ora tocca a voi il fare £0”
pta di qnefto gioco una matulé
rifleffione , & una longa pratt!
ca, avvertendovi, che quefto®
il vero modo di giocare in quae
@ro con il compagno, per allt
3} puoi aicora giocare 10 oe
& in quattro,ad ogn’uno pet 7.
ma ficome non mi fon mal ide »
to d’imparar agiocare ad OS 0
uno per fe , perche in que “ 4
non fi poffono girar gl On ter’
compagno , main calo de he
neceflitato a perder quale
onore fi deve fempre proce

« # 7} a
il minor danno; co! qraver-

ae ere ee ae a a ee ee ee ee ee

rare

 

> = A

Ba
   

. 1135,
’avetvi quivi brevemente de-
critte quefte regole , accid pof-
ate poi imparar quefto gioco
on piti facilita .

Se poi vi foffe qualcheduno ,
he defideraffe @’'imparar a gio
are ogn’uno per fe, deve faper
he chi sa ben giocare col come
agno sa ben giocare ad ogn’un,
ex fe, poiche la fine del gioca
fempre lifteffa, cioé di pro=
urar di farfi li fuoi onori, es
render quelli del compagno .

_ Per altro 4 me bafta per ora
avervi ammaeftsati nella for- |
na pit neceflaria, e pitt ufitata,,
Jual’é quella di giocare con ik |
“ompagno, econ la fola, e cost |
ier adempito nella forma pitt
offibile al mio impegno, mz

€ poi avefli mancato in qualche |
arte vi prego a compatir la |
nia temerita, ¢debolezza , affi- |
Cll
af
126 <
curandovi, che il difetto noné

ftato della volonta , ma dell in- ©

capacita, pero fottomettendo-

mi fempre prima al ben regola~ »

to, € prudente giudizio di vol
altri Signori Giocatori, dird,
che per giocar bene bafta gio-
care ,€ con rifleflione , e conla
ragione ; perche ogni qualvolta
uno sa render la ragione pets
che ha giocato , pi tofto; quella
carta, che un’altra dovete n¢e-
ceffariamente confeffare, che
quello ha giocato bene, tutto
che l’evento fia fucceduto males
laonde fe voi vi manterretes
fempre con quelli primi principj
difaper render la ragione del
voftro modo di giocare,credete-
mi, che giocarete fempre bene.

IL FINE,
 
