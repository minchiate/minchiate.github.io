\documentclass[12pt,a5paper]{book}
\usepackage[utf8]{inputenc}
\usepackage[T1]{fontenc}
\usepackage[italian]{babel}
\usepackage{changepage}

\usepackage{etoolbox}
\apptocmd{\thebibliography}{\setlength{\itemsep}{-2pt}}{}{}

\usepackage{tikz}
\usetikzlibrary{decorations.shapes,shapes.geometric}

% avoid orphans and widows, allow for (a lot of) letter spacing.
\usepackage[defaultlines=2,all]{nowidow}
\usepackage[tracking]{microtype}
\sloppy

% how to format and space chapter titles
\usepackage{titlesec}
\titleformat{\chapter}[display]
            {\huge\bfseries\scshape}
            {\vspace{-1.5em}}
            {0pt}
            {}
\titleformat{\section}[display]
            {\large\bfseries}
            {\vspace{-1.5em}}
            {0pt}
            {}
\titleformat{\subsection}[display]
            {\normalfont\fontsize{12}{14}}
            {\vspace{-1em}}
            {0pt}
            {\centering}
% I like this font!
\usepackage{tgbonum}
\renewcommand{\rmdefault}{qbk}
\usepackage{lettrine}
\usepackage[left=13mm,top=11mm,right=13mm,bottom=14mm]{geometry}
        
\makeatletter
\renewcommand{\@makefntext}[1]{%
  \setlength{\parindent}{0pt}%
  \begin{list}{}{\setlength{\labelwidth}{12pt}%
    \setlength{\leftmargin}{\labelwidth}%
    \setlength{\labelsep}{2pt}%
    \setlength{\itemsep}{0pt}%
    \setlength{\parsep}{0pt}%
    \setlength{\topsep}{0pt}%
    \footnotesize}%
  \item[\@thefnmark\hfil]{#1}% @makefnmark
  \end{list}%
}
\makeatother

\renewcommand{\negthinspace}{\hspace{-3pt}}

\newcommand*\sepline{%
  \kern 3pt \hrule \kern 2pt
}

\title{%
  \kern -2em\fontshape{sc}\normalsize
\textls[180]{\Huge MALMANTILE}\\
\textls[360]{\normalsize RACQVISTATO.}\\\kern 8pt
{\LARGE POEMA}\\
\textls[240]{\Large DI PERLONE ZIPOLI}\\
{\small CON LE NOTE DI PVCCIO LAMONI.}
}

\author{%
{\normalsize DEDICATO}\\
\textls[220]{ALLA GLORIOSA MEMORIA}\\
\textls[-20]{\footnotesize Del Sereniss. e Rerverendiss. sig.\ Principe Card.}\\
\textls[320]{\textsc{\Huge LEOPOLDO}}\\
\textsc{\LARGE de' medici}\\
\textsc{e}\\
\textsc{risegnato alla protezione}\\
\textsc{del}\\
\textls[-20]{\footnotesize Sereniss. e Reverendiss. Sig Principe Card.}\\
\textls[40]{\textsc{\Huge FRANC. MARIA}}\\
\textsc{\LARGE nipote di s.a.r.}
}

\date{%
\vfill\scriptsize
\textls[120]{\small\scshape In Firenze}\\
\sepline{}
\textls[-30]{\scriptsize Nella Stamperia di S.A.S.\ alla Condotta.\ 1688.\ \textit{Con lic.\ de Super.}}\\
\textls[120]{E PRIVILEGIO}\\
\textls[60]{Ad istanza di Niccolò Taglini.}
}

\newcommand{\flagverse}[1]{\vspace{6pt}\hspace{-8pt}\makebox[26pt]{\fontshape{sc}\footnotesize \hfill{}#1\hspace{4pt}}}

\newenvironment{signature}{

\kern 1em

\hfill \begin{minipage}{4.5cm}\centering}
{\end{minipage}}

\newenvironment{poesia}{%
  \kern -2em
  \setlength{\parindent}{-1em}%
  \setlength{\parskip}{8pt}%
  \begin{adjustwidth}{5em}{}\-
    
  }
               {\end{adjustwidth}}
\renewenvironment{verse}{%
 \itshape\setlength{\parindent}{30pt}
  \setlength{\parskip}{0pt}
  \obeylines}
               {}
\newenvironment{ottave}{% 
  \setlength{\parindent}{18pt}
  \setlength{\parskip}{-1pt}
  \obeylines}
               {\-}

\newenvironment{argomento}{%
  \section*{\textsc{Argomento}}\hspace{18pt}\begin{minipage}{10cm}
  \setlength{\parindent}{0pt}
  \setlength{\parskip}{-2pt}
  \obeylines}
               {\end{minipage}\vspace{10pt}}

\newcommand{\stanzadash}{\rule[2pt]{54pt}{1pt}}
\newcommand{\markstanzablock}[1]{\item[\stanzadash] \textbf{#1} \stanzadash}
\newcommand{\makestanzalabel}[1]{\textit{\textbf{#1}}}
\renewenvironment{description}
                 {\begin{list}{}{%
                       \setlength{\labelsep}{4pt}
                       \setlength{\labelwidth}{8pt}
                       \setlength{\topsep}{0pt}
                       \setlength{\parsep}{0pt}
                       \setlength{\parskip}{0pt}
                       \setlength{\itemsep}{0pt}
                       \setlength{\leftmargin}{12pt}
                       \setlength{\itemindent}{0pt}
                       \let\makelabel=\makestanzalabel}\small}
                 {\end{list}}

\begin{document}
\pagenumbering{gobble}
\maketitle
\pagenumbering{roman}

\-

\vspace{6em}

{\centering\Large
{\footnotesize\textsc{al sereniss., e rev.\ sig.\ il sig.\ principe card.}}\\
\textls[44]{\huge FRANCESCO MARIA}\\
\textls[100]{\LARGE DE' MEDICI.}\\
\kern 2em}
Il
Sereniss. e Reverendiss. Principe Cardinale
Leopoldo de' Medici Zio di V.A.R.\ Principe
di quelle rare, ed ammirabili qualità,
che hanno fatto stupire tutto il Mondo,
fino da i più teneri anni dell'A.V.R.\
conobbe, che in lei dovea continuare quello
splendore, che hanno accresciuto alla
sua Sereniss. Casa le stimabili doti di
V.A.R; E per questo, siccome giudicò, che l'A.V.R.\ gli
dovesse succedere nelle virtù, e nella dignità, così volle, che 
ella fusse anche erede della sua singolar Libreria. In questa,
havea l'A.S.Rev.\ destinato, che dovesse ottenere il luogo la
presente Opera di Perlone Zipoli, a cui S, A. R, m'onorò
comandarmi, ch'io facessi alcune note, grazia compartitami
(siami lecito il dirlo) forse con qualche scapito del
prudentissimo giudizio di S.A.R.; Ed havendo io ubbidito nella
miglior forma, che havevo saputo, già si pensava alla stampa,
quando i Fati invidiosi tentarono di privarla di così pregiato
onore: e sarebbe loro riuscito, se la somma prudenza di
quel gloriosissimo Principe non havesse a i medesimi impedito
il corso, con prepararle il rimedio nel rifugio alla protezione
di V.A.R.

Se ne vien però il povero Malmantile a' piedi di V.A.R.
umilmente supplicando la sua benignità a volersi degnare di
riceverlo nella sua grazia, e, come erede obbligato; 
riverentemente convenendola al Tribunale della sua generosità,
perché gli faccia godere la giustizia, concedendogli il luogo
stabilitogli, acciò egli possa dirsi veramente rifatto dalle
rovine cagionategli da tante sue disgrazie, e da tanti suoi
sinistri avvenimenti: Ed io piglio l'ardire d'accompagnare
queste preci, che egli porge a V.A.R., come quello, che
conosco d'haverlo con la mia penna costituito in grado d'haver
maggiormente bisogno dell'autorevol patrocinio di V.A.Rev.\
alla quale intanto umilissimamente inchinato bacio
ossequiosissimamente la Sacra Porpora.

Di V.A.Rev.

\begin{signature}
Vmilissimo Servidore\\
Puccio Lamoni
\end{signature}
  
  
 

\clearpage
{\centering\Large
{\footnotesize\textit{Al Sereniss.\ Rev.\ Sig.\ il Sig.\ Principe Cardinale}}\\
\textls[44]{\LARGE LEOPOLDO DE' MEDICI}\\
       {\large PADRONE CLEMENTISSIMO.}\\
       {\normalsize PVCCIO LAMONI.}\\
       \kern 2em}

SERENISS. E REVERENDISS. SIG.

MENTRE stavo meditando d'ubbidire a i cenni stimatissimi
di V.A.Rev.\ col far le Note alla presente Leggenda di
Perlone Zipoli, mi cadde sotto l'occhio un sonetto del
Burchiello, nel quale havendo osservato, dove dice:
 Non sunt, non sunt pisces pro Lombardis,
mi saltò il ticchio d'esser' il Lupo nella favola, cioè che questo verso
m'avvertisse, che la faccenda da V.A.Rev.\ impostami non fusse 
carne da' miei denti, ond'io havevo già quasi pensato di far conto, che
passasse l'Imperadore: Ma considerando poi, che farebbe stato errore in
gramatica, e da pigliar con le molle, il far'orecchie di mercante a i
riveritissimi comandamenti di V.A.R.\ ho risoluto di non metterla più in
musica, o in sul liuto, ne mandarla d'oggi in domani, dando erba
trastulla, e menando il can per l'aia, ma (venendo a dirittura a i ferri)
non tener più questo cocomero in corpo, e così cavarne cappa, o mantello
più per eseguire gli ordini di chi può comandare a bacchetta, che
perché io resti persuaso d'haver forze sufficienti a portar sí grave soma;
E quantunque io sappia, che havrei fatto molto meglio a lasciar la lingua
al beccaio, perché così havrei sfuggito il farmi dar la quadra, o la
madre d'Orlando, e sonar dietro le padelle da coloro, che si pigliano
gl'impacci del Russo, e ficcando il naso per tutto, fanno poi le Scalee
di S. Ambrogio, come quelli, che havendo mangiato noci, apporrebbono
al sale, senza considerare che ognun può fare della sua pasta
gnocchi, e che [come disse colui, che s'impiccò] ognuno ha i suoi
capricci; tuttavia ho voluto (legando l'asino dov'è piaciuto al padrone)
dare a conoscere che V.A.R.\ non farà, come il Podestà di Sinigaglia;
Se poi ad alcune di questi tali rincresce, mettasi a sedere, e, se non gli
piace, la sputi o mi rincari il fitto; e se dirà, che in fare alla presente
Opera le Note comandatemi, io non habbia preso il panno pel verso,
ma più tosto fatti de' marroni, e pigliato de' granchi a secco, lo lascerò
ragliare; perché son sicuro, che non mi farà baciare il chiavistello, ne
Pigliare il puleggio dalla casa mia; ne mi può accusare di delitto da
farmi mettere in Domo Petri fra i due Apostoli, o da farmi meritare d' esser'
ammazzato con una lancia da pazzo; E se l'indiscretezza di questi tali
mi condannerà per gli errori, che troveranno nelle Note fatte da me, la
mia ignoranza m'assolverà. Non ne ho saputa più: ho soddisfatto al
debito d'ubbidire, e mi quieto col detto di Donatello: Piglia un legno,
e fann'un tu. Mi fara forse detto: Tu porti frasconi a Vallombrosa,
cavoli a Legnaia, ed acqua in mare, e vai contrappelo alla buona
strada a comparire avanti a un Principe così erudito con questi tuoi
scritti; ed io a lettere d'appigionasi, e di scatola, senza saltare in sulla
bica, o entrar nel gabbione, rispondo a costoro, i quali fanno tanto il
Cecco suda, che portano ben loro le mosche in Puglia, e i Coccodrilli
in Egitto, e dandomi il mio resto, hanno trovato il modo d'intisichire,
senza però dirmi cosa, che io non sappia; perché conosco-ancor io il
pane da sassi, la Treggea dalla gragnuola, e le cornacchie dalle cicale; e
sapendo quanto il mio cavallo può correre, sarei venuto di male
gambe, e quasi come la serpe all'incanto, a metter questo cembolo in
colombaia; se non mi fusse noto, che colui, che è avvezzo a mangiar
sempre starne, desidera talora carne di Storno, e non fussi certo, che
la somma prudenza di V. A. R, (conoscendo, che il pruno non produce
limoni, e che dalla botte non esce mai, se non di quello che v'è
dentro, che parimente è impossibile, che il Gufo faccia il verso del
Rusignuolo) non è per isdegnare di ricevere le baie di Perlone Zipoli con
l'abito da villa messo loro in dosso dalla mia zucca, poco atta a
rappresentar l'impresa degli Accademici Intronanti, perché le manca il
Meliora Latent.

Supplico però l'impareggiabile umanità di V.A.R. a voler restar
servita di far conoscere a questi tali, che io ho legato il Cavallo a
buona caviglia, con fare degne queste mie insipidezze d'un benigno suo
sguardo; non perché lo meritino per se stesse, ma perché bensì conviene
alla continuazione di quel generoso aggradimento, col quale si compiacque
ricevere in vita dell'Autore il medesimo Malmantile. Il quale
se con le mie ciarle haverà fortuna di comparire in pubblico, godendo
sí pregiato favore, si potrà dire, nato vestito, ed io cascherò in piè
come i gatti, e mi pioverà il cacio in su i maccheroni: E così con
haver'immitato il cane di Butrione, non havrò timore di coloro, che passano
per la maggiore; perché sapendo essi, che l'Aquile non fanno guerra co'
Ranocchi, sdegneranno abbassarsi tanto con la loro critica, mettendo le
mani in si vil pasta; e quegli Aristarchi, i quali non contano, e non
hanno voce in capitolo, per haver poco di quel che il bue ha troppo, e
che sono come monete stronzate, o come i cavalli di regno; non saranno
causa, che io alzi i mazzi; ne mi faranno venire la muffa, o il moscherino
col loro gracchiare; perché oltre all'essere scritto pe' boccali, che il
Cieco non può giudicare de' colori, si sa ancora, che raglio d'asino
non entrò mai in Cielo, che però conoscend'io, che essi son per fare,
Come colui, che tosa il porco, non gli stimo il cavolo a merenda, e gli
ho dove si da al bossolo da spezzie, e dove si soffiano le noci; Sicché si
possono andar' a riporre a lor posta, e fare un mazzo de' loro salci.  E se
bene dice il proverbio, che la carne di Lodola va a Piacenza a ognuno;
io non mi curo, che me ne sia data, anzi per non mangiarne, son
contento far sempre di nero, purché non mi dieno di bianco questi Correttori
delle stampe, che tiranneggiando le lettere, perché si stimano il
Secento, cercano i fichi in vetta, e 'l nodo in sul giunco. Ma se poi mi
vorranno pure strazziare, io gli assicuro, che e' non hanno a mangiare il
cavolo co' ciechi, quantunque io non sia tanto addietro con l'usanza,
che io voglia mai far credere a haver cattivi vicini, o sia di natura
d'ungermi gli stivali a mia posta. Mi mandino, pure: all'Vccellatoio
quanto a lor piace, e mi facciano anche dietro lima lima, non faranno
però causa, che io faccia come Chele Masi, perché me la farebbono di
figura, e mi scotterrebbe troppo; se bene mi persuado, che ancor'essi
non fussero per uscirne netti; e che fusse per succeder loro il mangiar 
noci col mallo, e far come i Pifferi di montagna, poiché, se essi si stimano
piccioni di Gorgona, ed io non son di Valdistrulla; perché sono uscito
di dentini ed ho rasciutto il bellico, e per questo so ancor'io quante
paia fanno tre buoi; onde a dirmi cattivo cattivo, la farà fra Baiante, e
Ferrante, perché io son d'una natura, che non posso ber grosso, e mi so
levar le mosche d'intorno al naso, ne mi morse mai cane, che io non
volessi del suo pelo, massimamente quando m'è saltato il capriccio di
voler la gatta, e badare a bottega, giuocando per la pentola; e s'io me
la son mai legate al dito, o l'ho presa co' denti, n'ho voluto vedere
quanto la canna; perché non mi suol morire la lingua in bocca, ed ho
tagliato lo scilinguagnolo, ne m'è piaciuto mai portar barbazzale, e so
lasciar la squola d'Arpocrate, quando è tempo, ed in particolare con
quei tali che, son più tondi dell'O di Giotto, e che stimando una stessa
cosa il chiacchierare, che il condennare, non sanno portare altre ragioni,
che quel maladetto \textit{non si può}.

Ma perché non paia ch'io saltando di palo in frasca voglia dar panzane
a V.A.R.\ e che questa mia lettera sia il vicolo di mona Sandra, conchiudo,
tornando a bomba, che stimerò d'haver toccato il Ciel col dito,
e tirato diciotto con tre dadi, se potrò conoscere, che l'A.V.R.\ resti 
servita di credere, che in questa parte io l'habbia: ubbidita giusta mia
possa, come riverentemente la supplico a degnarsi di far apparire con l'onore
di nuovi suoi comandamenti. Mentre facendo la festa di S. Gimignano
umilissimamente inchinato bacio ossequiosissimamente a V.A.R.\
la Sacra Porpora.

\clearpage
\noindent\textsc{\centering
\textls[180]{\large al cvrioso e discreto lettore}\\
{\large pvccio lamoni.}\\
\kern 0.5em}

La presente Opera di Perlone Zipoli si manda alle stampe, per soddisfare
alla curiosità di molti, che bramosi di pigliarsi il passatempo di leggerla
ne hanno fatta instanza. E perché in alcuni detti, e proverbi usati
in Firenze, de' quali si serve il nostro Autore, possa esser' intesa anche da
color, che lontani dalla nostra Toscana, non hanno la vera cognizione del valore,
e senso di essi, vi ho aggiunto alcune note, con le quali se non ho appieno
soddisfatto, mi basta, che havrò forse data occasione col mio cicalare, che
venga ad altri voglia di meglio discorrere. Tu intanto ricordati, che questa è
una novella; e così ti accomoderai a compatire, se alle volte mi fon fatto
lecito di dare qualche spiegazione favolosa. So, che havrai la bontà di sbandir la
censura, e ti tornerà commodo, perché facendo altrimenti havresti troppo da
fare, poche, o forse niuna essendo di quelle cose, che ho scritto, che non la
meritino con un nuovo foglio, e per questo non te ne prego: ti prego bene, se sei
Fiorentino, a legger' il Testo, e non le Note, perché queste non son fatte per te,
che, meglio di quel ch'io habbia scritto, intendi la forza de i detti, che ho
preteso dichiarare,

Dovrei notare gli Autori, a i quali son ricorso per tirare a fine la presente
fatica, ma perché gli bo nominati in tutti quei luoghi, dove è convenuto valermi
della loro autorità, tralascio di farlo; non voglio già tralasciare di confessar
l'obbligo, che queste mie Note, ed io habbiamo all'Eccell.\ e dottissimo Sig.\ 
Gio.\ Cosimo Villifranchi, ed agli Eruditiss.\ SS.\ Anton Casto, e Sig.\ Francesco
Maria Bellini, i quali m'hanno onorato di più erudite notizie; ed in ultima
attestar la fortuna che hanno havuto questi miei scritti di passar sotto l'occhio
dell'Ecc.\ Sig.\ Abate Anton Maria Salvini\footnote{Anton Maria Salvini, Firenze 1653 - Firenze 1729. Grecista, con Antonio Maria Biscioni, 1674-1756 figura sulla copertina delle edizioni 1731 e 1750 del Malmantile.} il quale non solamente s'è contentato
d'emendar molti miei errori, ma d'ingagliardire ancora le mie debolezze con non
poche sue bellissime erudizioni, a segno, che ha fatto nascere in me una speranza,
che sia per esser ricevuta volentieri questa mia Opera, e d'haver guadagnato
non poco appresso al Mondo letterato, per haver dato occasione a questo dottissimo
huomo d'esercitare la sua stimabilissima penna, i tratti della quale, come
non ho dubbio che nobilmente risplenderanno dentro all'oscurità della mia, così
son certo, che saranno da tutti benissimo ravvisati: Ne confesso però al
medesimo il mio debito, e ne porto al pubblico questa attestazione, perché si sappia
che quello, che sarà riconosciuto per non mio, non è latrocinio, ma regalo
fattomi da questo, e da altri huomini dotti per loro generosità, e per sollevar
Perlone dal discredito, che haveriano fatto meritare a questa sua Opera i miei scritti.\\
Lettore, vivi felice. 

\clearpage

{\centering\Large
\textls[244]{\LARGE PROEMIO.}\\
\kern 1em}

Lorenzo Lippi\footnote{Lorenzo Lippi, Firenze 1606 - Firenze 1665, pittore. ``Perlone Zipoli'', poeta, scrittore.} (che in Anagramma nella presente Opera si chiama Perlone
Zipoli ) è stato ne i tempi nostri Pittore non poco celebre, come testificano
molte, e molte sue fatiche. Ciò lo fece meritare d' esser chiamato dalla
Sereniss. Arciduchessa Claudia d'Austria\footnote{Claudia de' Medici, Firenze 1604 - Innsbruck 1648. Reggente del Tirolo dalla morte del secondo marito Leopoldo d'Asburgo nel 1632 alla maggiore età del figlio Ferdinando Carlo nel 1646.} per valersi dell'opera sua a Inspruk,
dove dette principio a questa da lui chiamata Leggenda delle due Regine di
Malmantile, e la dedicò alla medesima Sereniss.\ Arciduchessa Claudia. Haveva però
l'Autore concepita nell'animo suo quest'Opera qualche anno prima, e nel
tempo, che essendo in Villa de' SS, Parigi a S. Romolo nell'andar per quelle campagne
a diporto, vedde le muraglie di Malmantile; ed haveva discorso questo
suo pensiero col sig.\ Filippo Baldinucci\footnote{Filippo Baldinucci, Firenze 1624 - Firenze 1696. Storico dell'arte, politico e pittore, ``Baldino Filippucci''.}, dal quale poi nel tessimento del Poema
hebbe, come da persona erudita ( che tale lo dichiara la sua bell'Opera mandata
da esso alla luce intitolata Notizie de i Professori del disegno) non piccolo aiuto
in proposito della lingua, e d'altro, e particolarmente nei descrivere il Consiglio
de i Diavoli nel Canto sesto.

Tal composizione fece egli a solo fine di mettere in rima alcune novelle, le
quali dalle donnicciuole sono per divertimento raccontate a i bambini, e di sfogare
la sua bizzarra fantasia, inserendovi una gran quantità di nostri proverbi, ed
una mano di detti, e Fiorentinismi più usati ne i discorsi famigliari, sforzandosi di
parlare, se non al tutto Bocaccevole, almeno in quella maniera, che si costuma
oggi in Firenze dalle persone Civili, ed ha sfuggito per quanto ha potuto quelle
parole rancide, alle quali vanno incontro tal'uni, che per spacciarsi huomini
letterati, non sanno fare un discorso, se non vi mettono, guari, chente, e simili
parole, che per essere state usate dal Boccaccio\footnote{Giovanni Boccaccio, Certaldo 1313 - Certaldo 1375.}, essi credono, che dieno l'intero
condimento alli loro insipidi ragionamenti, e stimano, che quello sia il vero parlar
Fiorentino, che non è inteso, se non da i lor pari, e non s'accorgono, che
in tal guisa parlando, si rendono scherzo di chiunque gli sente, come bene attesta
questa verità il Lasca\footnote{Anton Francesco Grazzini detto il Lasca, Firenze 1505 - Firenze 1584} in quel suo Sonetto sopra l'Opere del Berni\footnote{Francesco Berni, Lamporecchio 1497 - Firenze 1535. ``che dice le cose sue semplicemente, e non affetta il favellar toscano''.}, dicendo:
\begin{verse}
\hspace{-1em}Non offende gli orecchi della gente
Con le lascivie del parlar Toscano,
Vaquanco, guari, mai sempre, e sovente
\end{verse}
Ed Antonio Abbati\footnote{Antonio Abati, Gubbio inizio secolo XVII - Senigallia 1667} dice
\begin{verse}
\hspace{-1em}Peggio non ho, che quel sentir parlare
Con tanti quinci,e quindi, e, ec.
\end{verse}
Anzi in questa parte l'unica intenzione del nostro Poeta è stata di far conoscere
la facilità, e pienezza del parlar nostro, e \textit{Cogliendo della lingua materna il più
bel fiore}, mostrare, che ancora ad uno, che non ha (come'appunto, era egli)
altra eloquenza, o poca più di quella, che gli dettò la natura, non è impossibile
il parlar bene. Questo, ed altri fini dell'Autore s'argumentano dalla seguente
Dedicatoria, che egli stesso scrisse alla Sereniss.\ Arciduchessa Claudia, la quale
lettera io pongo qui per confonder coloro, che pur vorrebbono fargli dire quel
che mai il nostro Poeta hebbe in pensiero.

\begin{adjustwidth}{1.5em}{}
  \itshape
Ati figliolo di Creso Re di Libia (se è vero, che io non ne so più la, e la vendo,
come io l'ho compra) vedendo il padre in pericolo, isso fatto cavò fuora
il limbello, e disse le sue sillabe, come un Tullio; Tutto il rovescio dovrebbe
fare il pesce pastinaca senza capo, e senza coda della mia Leggenda a mal tempo,
ch'io mando a V.A.S.\ perché vedendo ella quel dolce intingolo di quel
fantoccio di suo padre in procinto d'esser mandato all'Vccellatoio, e quasi ridotto
alla porta co' saffi, e che gli sien suonate dietro le padelle, anzi fra il
tocca, e non tocca di scior Pallino, potrebbe a sua posta far' un mizzo de' suoi
salci, e farsi ricucire la bocca per non haver più occasione di formar verbo.

Ma perché si compiace V.A.S.\ di volerne una secchiatina, benché questa mia
Leggenda non fusse degna di fiutare eziam i luoghi privati, verrà di gala col suo
ricadioso cicaleccio, che si strascica dietro una gerla di farfalloni, a farne una
stampita anche ne i Palazzi reali, perché ella è una prosontuosina da darle del
Voi; Ond'io conoscendo nella temerità di essa l'ubbidienza dovuta de iure a i
riveriti suoi cenni, gli è giuoco forza, voglia il mondo, o no, che ella si metta
giù a bottega a sfogare la fisima de' suoi fantastichi ghiribizzi, contentandomi
io, che ella, come nata da scherzo, mi faccia scherzo alle genti. Compatisca
dunque l'A.V.S.\ questa sconciatura partorita nel tempo, che io do
festa a i pennelli, mentr'ella non apprezzando un'ette gli applausi volgari, riceverà
per grazia sterminata, e per arcisbardellatissimo favore, se queste baie
riusciranno di qualche valezzo nel cospetto di V.A.S.\ alla quale profondamente
inchinandomi, con ogni debita rivereaza bacio la Veste.
\end{adjustwidth}

Da questa lettera adunque si viene in non piccola cognizione de i sentimenti
dell'Autore nel comporre la presente Opera; La quale fu da esso presso che
terminata in Inspruch, e dedicata come ho detto alla Sereniss.\ Arciduchessa
Claudia; Ma essendo S.A.S.\ in quei medesimi tempi passata all'altra vita,
convenne all'Autore tornare alla Patria, dove fu questa sua Novella veduta da diversi
amici suoi, fra i quali dal sig.\ Romolo Bertini Servidore del Sereniss Principe
Cardinale Leopoldo de' Medici\footnote{Leopoldo de' Medici, Firenze 1617 - Firenze 1675, cardinale dal 1668.}, e molto accetto per l'ottime sue qualità, virtù,
e dottrina, e da esso hebbe S.A.R.\ la prima notizia della presente Opera, e fino
da allora mostrò l'A.S.R.\ non piccola inclinazione, che si pubblicasse, e se
tralasciò di comandarne la stampa, fu, perché sentì dal medesimo Bertini, che
l'Autore pensava d'accrescerla.

Fu veduta ancora dal sig.\ Francesco Rovai\footnote{Francesco Rovai, 1605-1647. ``Franco Vicerosa''}, e dal sig.\ Antonio Malatesti\footnote{Antonio Malatesti, Firenze 1610 - Firenze 1672. ``Amostante Latoni''.};
ambi Poeti nel lor genere Eccellentitfimi, dal sig.\ Salvador Rosa\footnote{Salvator Rosa, Napoli 1615 - Roma 1673. ``Salvo Rosata''} non men celebre
nella Poesia, che nella pittura, ¢ dal quale il Lippi hebbe notizia Dello Cunto
de li Cunti\footnote{Pubblicato da Adriana Basile fra gli anni 1634-1636.} di Gianalesio Abbattutis\footnote{Giovan Battista Basile, Giugliano di Napoli 1566 - Giugliano 1632.}, di dove l'Autore cavò poi alcune novelle,
che si trovano in quest'Opera: La quale in somma fu veduta da molt'altri eruditi
ingegni; e fu il Lippi da essi consigliato, e poco meno, che forzato a metterla
alla stampa, con persuaderlo, che meritava la pubblicazione: ma ricusò egli
sempre di far tal passo, conoscendo molto bene, che colui, che stampa l'Opere
sue, s'espone ad un certissimo pericolo, per una incerta gloria, e massime nel
presente secolo, che vi è maggiore abbondanza di spropositati, e mordaci Satirici,
quali con invidioso livore lacerano le fatiche altrui, che di Censori discreti, i
quali con dotti avvertimenti n'emendino gli errori.

Dalle grandi instanze fattegli dagli amici suddetti, che egli stampasse questa
sua Novella, insospettito il Lippi, che il libro di detta sua composizione non gli
fusse levato, e contro a sua voglia stampato, andava molto circospetto, non lo
lasciando in luogo, dove fusse sottoposto a tal caso; Ma essendo una volta andato
in villa de' SS. Susini suoi cognati, e di quivi alla villa del sig.\ Don Antonio de'
Medici\footnote{forse Anton Francesco de' Medici, 1618-1659, frate dell'ordine dei Cappuccini}; dove havendo portato il detto libro per passare, leggendolo, la veglia,
la notte, mentre egli durmiva, il sig.\ Piovano Gualfreducci, ed il sig.\ Tommaso
Fioretti con l'assistenza del medesimo sig.\ D. Antonio sciolsero il detto libro, e
fra tutte due lo copiarono e la mattina lo rilegarono, e lo raccomodarono in
maniera, che egli non s'accorse del virtuoso furto. Questa copia capitò poi in
mano a Paolo Minucci\footnote{Paolo Minucci, Firenze 1606 - Radda 1695. ``Puccio Lamoni''}, il quale facendo al Lippi la solita instanza di metterlo alla
stampa, ed egli ricusando, gli disse il Minucci, che l'haurebbe egli fatto stampare;
¢ replicando il Lippi, che se ne contentava, se vi era modo, il Minucci
col mostrargli la detta copia scoperse il furto, e fece conoscere la possibilità, che
havea di farlo stampare, S'alterò non poco il Lippi veduto questo, ma come
huommo virtuoso, ed onorato volle, che la vendetta di tal disgusto fusse il costituire
il Minucci, ed ogni altro in grado di non si curar più di stampar quell'Opera;
questo fu con aggiugner'ad essa alcuni episodj, ed altro, in maniera, che in
breve tempo 1a ridufle da fette piccoli canti, che ell' era, alli dodici, che è la
presente; ¢ perché non gli avvenisse di questa, come gli era accaduto della prima
teneva l'originale di essa in modo riserrato, e ristretto, che non lasciava vederlo
ne meno all'aria, e poco altro poteva haversene, che sentirne recitar da lui
qualche Ortava alla spezzata, ed il Minucci più d'ogni altro haveva questo favore
da lui, perché col fargli sentire l'augumento, che dava a quest Opera, stimava
di fare scemare nel Minucci la volontà di stamparla, e conseguir l'intento,
che s'era prefisso, ma ne seguì tutto il contrario, perché havendo il Minucci
sparso fra gli amici, che il Lippi riduceva la sua Opera in stato ragguardevole,
pervenne questa notizia all'orecchie del Sereniss.\ sig.\ Principe Card.\ Carlo de' Medici\footnote{Carlo de' Medici, 1595-1666.}
Decano del Sa.\ Collegio, e S.A.R.\ curiosa di veder quest'Opera comandò
al Minucci, che operasse d'appagare tal sua curiosità. Il Minucci manifestati al
Lippi i sentimenti dell'A.S.R.\ esortò a non contraddire di ricever l'onore
che S.A.R\ gustava di fargli; ed egli conoscendo, che mal poteva negare d'ubbidire
a tanto Principe, per il quale (come fratello della Sereniss, Arciduchessa.
Claudia) riteneva congiunto al debito di suddito un genio non ordinario di servirlo,
e persuafo pure una volta; che il pubblicar detta Opera non gli poteva
apportar se non lode, condescese a lasciarne pigliar copia per S.A.R.\ la quale si
piacque di dar dimostrazione del suo benigno aggradimento con atti non piccoli
della sua solita generosità, e verso il Lippi, e verso il Minucci, che ne fece
la copia, perché così volle il Lippi, o per spaventar il Minucci con la gran macchina,
che appariva, e così levarlo dal pensiero di pigliarsi questa fatica, ed
addormentare intanto nel sig.\ Principe Card.\ la volontà d'haverlo (come disse il
medesimo Lippi) o pure, perché quella copia non capitasse in mano ad altri, che
del medesimo Minucci, del quale si fidava, e per sua bontà, e perché haveva
anche veduto, che di quella copia, che teneva detto Minucci della prima Opera,
non s'era mai saputo cosa alcuna, perché esso Minucci l'haveva sempre occulata,
e negata a ognuno d'haverla, Ma quel'ultima copia sendo in mano del
detto Sereniss.\ sig.\ Card.\ Decano, accrebbe nei SS.\ suoi Cortigiani la curiosità
d'haverla, e cosè per diverse vie ne trassero una copia. Da questa poi se ne sono
sparse infinite; ma perché l'Autore sopravvisse qualche poco di tempo, e sempre
accrebbe, o moderò qualcosa, ed oltre a questo, perché la poca avvertenza di
coloro, che hanno copiato, ha causato, che si trovino molte copie, e difettofe,
o guafte, il Minucci riputandosi in un certo modo cagione di questo disordine risolvette
per rimediarvi, di supplicare il Sereniss.\ Principe Leopoldo (allora non
Cardinale, al quale dall'Autore stesso fu quest'Opera dedicata, dopo la morte
della Sereniss.\ Arciduchessa Claudia) di permettergli il mandare la detta Opera
alla stampa, per rinnovare la memoria de] già defunto Lippi\footnote{Siamo quindi fra il 1665 ed il 1668.}, e S.A.\ glielo
concedette, con obbligo però, che gli facesse alcune Note, ed esplicazioni; E così
contento l'universale, che desiderava tal pubblicazione, e diede al Minucci il
gastigo d'esscre stato causa del suddetto disordine, ed al Lippi la soddisfazione\footnote{postuma}
dovutagli dal Minucci per la violenza fattagli, con obbligare il medesimo Minucci
a sottoporre ancor'egli i suoi scritti a quei danni, che dalle stampe ne
risultano; Sentenza veramente giusta, come appoggiata al fondamento della pena del
Taglione, ma troppo severa nell'arbitrio per la gran disparità, che è fra la vaga
Opera del Lippi, e l'insipide chiacchiere del Minucci, sopr'alle quali, e non
sopra gli scritti del Lippi si fermeranno, e poseranno tutti gli Aristarchi; con
tutto questo non ha il Minucci voluto intentare appello, anzi, sendosi accinto
subito a dare esecuzione alla sentenza, ha aggiunto all'Opera le Note comandate,
con le quali ha egli preteso d'operare, che fuori di Firenze, e della nostra
Toscana, e per tutta Italia possano esser meglio intese molte parole, detti, frasi,
e proverbj, che si trovano nell'Opera, forse non intesi del tutto altrove, che in
Firenze; e prega il Lettore a compatire, se non sia da esso soddisfarto appieno, e
ricordargli, che non è stata mente del Minucci il portare l'etimoiogia delle parole,
frasi, e proverbj, ma d'esplicargli in maniera, che possano esser'intesi anche
fuori di Firenze, ed habbia il medesimo Lettore la discretezza di riflettere, che
molti Fiorentinismi sono in uso, nati dal puro caso, senza un minimo
fondamento, o ragione, perché si dicano, e che;
\begin{verse}
Non omnium, quae a maioribus nostris scripta, aut dicta sunt, ratio reddi potest.\footnote{Adattato da Tommaso d'Aquino, Summa Theologiae, Q.\ 95, Art.\ 2. ``Sed non omnium quae a maioribus lege statuta sunt, ratio reddi potest, ut iurisperitus dicit.''}
\end{verse}


\clearpage
\noindent\textsc{\centering\large
\textls[240]{\Huge MALMANTILE}\\
{\small DISFATTO}\\\kern 6pt
\textls[360]{\LARGE ENIGMA}\\\kern 4pt
{\normalsize DEL SIG.\ ANTONIO MALATESTI.}\\
\kern 1em}


\begin{poesia}
Ov'è l'Etruria indomita, e infeconda,\\
Già fui per molti figli e ricco, e bello,\\
Or c'una fascia a pena mi circonda,\\
Povero, brutto, e vil non son più quello.

M'hanno gli amici più che 'l vento, e l'onde\\
Levate l'ossa, e toltomi il cappello,\\
E fino il nome par che corrisponda;\\
Vna mala tovaglia, o un mal mantello.

Così ridotto trovomi a mal porto,\\
Col corpo voto, e senz'un membro intero,\\
E pur con tuttociò non mi sconforto;

Anzi ora godo, e farmi eterno spero,\\
Mentre in Flora un' Augel per suo diporto,\\
Cantando in burla, mi rifà da vero.
\end{poesia}

 
\chapter{Primo Cantare}
\pagenumbering{arabic}

PRIMO Cantare. Ecco che il nostro Poeta mantiene l'intenzione
data di pubblicare una Leggenda,e non un Poema, mentre mette
sopra ogni Canto l'inscrizione, che si vede in diverse leggende
dove in vece di dire Canto 1., e Canto 2, ec. come usano nei
Poemi Italiani, egli dice Primo Cantare, e così seguita fino all'ultimo,
volendo per la sua modestia efler chiamato Compositore
di Leggende, non Autore di Poemi, ed in uno stesso tempo
con bell'arte difendersi dalle censure di chi lo tacciasse di non aver'osservate le
regole del comporre i Poemi, sapendosi, che a queste non sono sottoposti i
compositori di Leggende.

\begin{argomento}
Marte sdegnato perché il Mondo è in pace
Corre, e da letto fa levar la suora, 
E in finto aspetto, e con parlar mendace
Mandala a svegliar l'ire in Celidora,
Fa la mostra de' suoi Baldone andare
Indi all imbarco non frappon dimora,
E per via narra con che modo indegno
a occupate avea il suo Regno.
\end{argomento}

Gli Argomenti a tutti li Canti di quest'Opera sono di Amostante Latoni, cioè
Antonio Malatesti, fatti di comandamento del Sereniss.\ Principe Cardin.\
Leopoldo de' Medici.

\section{Stanza I}

\begin{ottave}
\flagverse{1}Canto lo stocco, e 'l batticul di maglia,
Onde Baldon sotto guerriero arnese,
Movendo a Malmantil' aspra battaglia
Fece prove da scrivern' al paese,
Per chiarir Bertinella, e la canaglia
Che fu seco al delitto in crimen lesa
Del far' a Celidora sua cugina,
Per cansarla del Regno, una pedina,
\end{ottave}

Mostra l'Autore in questa sua introduzione, che egli vuol descriver da Guerra
fatta da Baldone in aiuto, e difesa di Celidora, e vuol persuadere, che se ben
dice \textit{aspra battaglia} fu una guerra di nulla, e però seguita: \textit{fece prove da
scrivern'al paese}, del qual detto ci serviamo per derisione, quando altri ha fatta
una azione da lui stimata grande, e bella, che in effetto non è poi tale, anzi è
tutta il contrario, e si dice:  \textit{Hai fatto assai, scrivi al paese}.

\begin{description}
\item[BATTICVLO di maglia] Intende il Giaco, arme difensiva di dosso, cioè una
camiciuola composta di maglie di ferro, ed è la lorica ansulata, che usavano gli
antichi. E se bene \textit{batticulo di maglia} non è veramente buon Fiorentino, nondimeno
è spesso usato, ma per giuoco, ed è comunemente inteso per il Giaco, e si dice
così, perché coprendo quest'arme le parti di dietro, nel moto che fa colui, che
l'ha in dosso, batte in quella parte; come si dice Picchiapetto quel Gioiello, che le
donne usano portare al collo pendente sul petto.

\item[MALMANTILE] E' un Castello antico vicino a Firenze circa dieci miglia,
  oggi del tutto rovinato, e distrutto, ne vi si vede altro che lé muraglie Castellane.

\item[CHIARIRE] Questo verbo, che oltre a gli altri significati, vuol dire Far conoscere
  l'errore, o Render capace; nel presente luogo vuol dice Scaponire, o
  Sgarire: \textit{Il tale mi faceva l'huomo addosso, gli ho dato una buona quantità di pugna, e l'ho
chiarito}; cioè con questo l'ho reso capace, e fattogli conoscere la stima, che io fo
  di lui, e quella che egli deve far di me. Questo verbo è traslato dal verbo Chiarire,
  che è Purificare ogni liquore torbido, e contaminato da materie crasse.

\item[CANAGLIA] Gente vile, ed abietta, che tali saranno, come vedremo, i soldati
  di Bertinella, i quali il Poeta mette Huomini d'infima plebe, che Cicerone
  chiama Imi subsellij homines. Il Sig\ Francesco Maria Bellini in alcune sue bellissime
  reflessioni, che si è contentato fare sopr'alla prsente Opera, ponderando la
  parola Canaglia dice, che l'allungamento delle parole in \textit{aglia} sta Oggi in
  Toscana un certo avvilimento, e disprezzo del subietto, e s'usi solo in cose vili, e
  plebee, e però si dica de' Birri sbirraglia; della Plebe. Plebaglia, e gentaglia; de i
Fanciulli, e popolo infimo Spruzaglia, (metaforico da spruzolo, acqua minuta)
e che questo sia antichissimo Latino, sia di neutro plurale, del quale si servirono i
Latini per comprender l'appartenenze della cosa, della quale parlavano, v.g.\
delle cose appartenenti alle navi dicevono Navalia; alla Cacina Popinalia, e molt'altri,
è corrotto da noi con l'aggiunta della lettera G.

\item[IN crimen lesa] È delitto di lesa Maestà cacciare una Regina del
suo Regno.

\item[FAR' una pedina] Si dice Fare una pedina a uno allora che procurando questo
  tale di conseguire cosa di suo gusto, ed essendo vicino a ottenerla, un'altro, a
  cui haveva confidato tal negozio; gliela leva su. Viene dal giuoco di Scacchi, dicendosi
  propriamente: Dare scacco di pedina.

In oltre, chi è pratico del giuoco di Scacchi sa, che quando s'è perduta la
Regina, si procura di racquistarla con far' arrivare una pedina al posto dove
stava la Regina dell'avversario al principio del giuoco, e così intendere, che Celidora
priva del Regno conveniva, che sotto nome di Pedina tornasse a ricuperarlo,
se voleva esser detta Regina.

Si potrebbe anche dire, che il nostro Poeta seguitando il costume che habbiamo
di chiamar Dame le Signore grandi, e Pedine le donne d'infima plebe, habbia inteso,
che Bertinella, togliendo il Regno a Celidora, l'habbia cavata del nome di
Dama, per haverla ridotta in grado miserabile, le habbia fatto meritare il nome
di Pedina; ma l'esser' il nome, di Celidora nel terzo caso, e non nel secondo, o
nel quarto; fa languire questa riflessione.
\end{description}

\section{Stanza II}
\begin{ottave}
\flagverse{2}O Musa, che ti metti al sol di state
Sopr' un palo a cantar con si gran lena, 
Che d'ogn'intorno assordi le brigate,
E finalmenre scappi per a schiena;
S'anch'io sopr'alle picche dell'armate
Volto a Febo con te venga in iscena,
Acciò ch'io possa correr questa Lancia,
Dammi la voce, e grattami la pancia.
\end{ottave}

Quest'Ottava ha poco bisogno di spiegazione vedendosi chiaro, che il Poeta,
invoca per sua Musa la Cicala, e così dà a conoscere, che egli vuole scrivere affatto
mostrando, che per fare una composizione come egli ha in animo,
e per descrivere una guerra qual fu quella di Malmantile, gli basta haver
chiacchiere.

Si potrebbe anche dire, che il Poeta sapendo che non si trova, che le Muse habbiano
dato mai alcuno aiuto effettivo, ed evidente, come dette la Cicala a Eunomo
Locrense Suonatore nella disputa, che hebbe con Aristono, supplendo con
la voce al mancamento della corda strappata, come si legge in Strabone lib. 6.
voglia, come fece Eunomo, far più capitale della Cicala, che d'altre Muse:
E può anch'essere, che egli invochi la Cicala, perché stimi più nobili delle Muse le
Cicale per esser queste più riguardevoli, come nate avanti alle Muse (secondo la
favolosa credulità de' Gentili) d'Huomini, li quali per lo gran gusto, che hebbero
del cantare, furono in cicale convertiti, come si cava da Celio Rodigino lib.\
17.\ cap.\ 6.\ le cui parole sono queste: \textit{Fertur enim hosce homines fuisse ante Musas;
natis deinde Musis, cantumque monstrato, illorum nomnullos voluptare cantus usque adeo
delinitos fuisse, ut canentes cibum, potumque negligerent, imprudenterque perirent; ex
quibus deinde cicadarum genuss sit propagatum, ec,}

Dice il Doni nella sua Zucca, che tutti li Poeti hanno la loro Cicala, e che
questa serva loro per Fama publicando le loro Poesie, onde il nostro Poeta seguitando
l'opinione del Doni invoca la Cicala destinata al suo servizio, perché gli
faccia questo di pubblicare le sue Poesie.

\begin{description}
\item[PALO] Pertica, Bastone di legno, che si mette per sostegno alle viti, ed altri
  arbuscelli simili.
  
\item[LENA] Significa quello, che i Latini dicono \textit{respiratio}, cioè quieto, e
  tranquillo
  anelito, il che mentre è nell'Huomo, egli si mantiene senza difficultà, nelle
  forze: ma la troppa fatica di corpo, o di mente spesso fa affannare tal Lena,
  però che uno, che s'eserciti assai senza posarsi, appunto come fa la Cicala col
  suo cantare senza riposo, si dice Haver gran Lena.

  Dante Inf.\ C.\ 1.\ \begin{verse}E come quel che con lena affannata, ec.\end{verse}
  
  Al Canto 24.\ \begin{verse}La Lena m'era dal polmon si sì smunta, ec.\end{verse}

  Vedi sotto C.\ 4.\ stanza 6.

  Varchi stor.\ lib.\ 5. \begin{verse}Essendo egli di pochissimo spirito,
    e di gentilissima Lena\end{verse}
  
  Franco Sacc.\ Nov.\ 127.  \begin{verse}Alla fine perdendo questi ciechi
      la Lena per essersi molto bene mazzicati, ec.\end{verse}

  I Latini con la voce \textit{Vis}, e con la voce \textit{robur} esprimevano questa Lena.
  
\item[VENIRE in scena] Comparire in pubblico,  vedi sotto C.\ 4.\ stan.\ 6.

\item[CORRER questa lancia] Tirar' a fine quest'Opera.

\item[GRATTAMI la pancia] Col grattare il corpo alla Cicala, ti fa che ella canti,
  la Cicala a grattare il corpo a lui, acciò che'egli canti. Quand'altri
  sa qualcosa, ed è duro a manifestarla, si dice; \textit{Grattagli la pancia, che egli
    canterà},
  cioè interrogalo, ed esaminalo bene, che egli dirà tutto quello, che tu
vuoi; si che il senso di questo detto \textit{Grattare il corpo a uno}, è Incitarlo a discorrere.
Vedi sotto C.\ 2.\ stan.\ 8.

\end{description}

\section{STANZA III \& IV}

\begin{ottave}
\flagverse{3}Alcun forse dira ch' ia non fo cica y 
E ch' io farei'! meglio a farms Zito, 
Suo danno;innanes pur chi vual dir dica, 
Fo io per quelte qualche gran delitto 2 
S?* io dirt male, ilCiel /a benedica; 
A chi non piace, mi rincari il fitto: 
pion fo, se fela. Sanne questi feivcchi, 
Chognun pre far della fuapalta gnocchi.

\flagverse{4}Mi basta fal che Voftra Altexza accetta
Dionurarmi d' udir questa mia feoria
Seritra cosk come la peanagetta y
Per fuggir L0xs0,e non per cercar gloria;
Se non le gufea, quando l! aura letta
Tornerò bene il farne una baldoria:.
Che le daranno almen qualche diletto
Le Adonachine, quando vanno a letto, 
\end{ottave}

In queste due Ottave l'Autore piglia a difender se medesimo dalle male lingue,
¢ moftra, che poco gl' importa | elier lodato, o biafimato:in questa sua Opera 5¢
che, non eflendo obbligato a veruno,vuol soddisfare a se medesimo, ed.al suo ca-
priccio; e però dice; S'ia dirò male il Cicl /a benedica, che significa V adia il nego-
zio, come è vuole, che non m' importa, E seguita 4 chi nom piace ms rincart il
fitto,voiendo moftrare, che per non efiere obbiigato a render conto ad alcuno del-
le sue azioni; non teme d' esser riprefo, o di ricever danno; ¢ foggiugne: Ognun
puo far deda /ua pasta gnocchi, cio€ ogni huomo likero puo fare del suo,a suo.mo-
do. Conchiude in somma, che egli vuol dar gulto a se medesimo, ¢ lasciar dire
chi vuol dire, baftandogli, che S. A., cio¢ il Serenils, Principe Card, Leopoldo de'
Medici, a cui dedica ? Opera, si contenti di riceverla, ed' udirla, foruta comes
la penna gotta, cioé composta non ad altro fine, che dj spaflarsi; ne si cura d' ac-
quiftar gloria per ral composizione, anzi supplica S, A. ad abbeuci pegeve
V havera letta, che ricevera qualche gusto dal veder' andare a letto le Adon 4 8
per Monackine intende quello, che intendono i nostri Fanciullini, ¢ioé quelle pic:
cole scintille, che, nell' incenerirfi la carta, a poco a poco si [pengono, ¢ facen-
do un certo moto, pare che si dileguino, (embrando tante Monache, le quali col
loro Jume in mano scorrano per il dormentorio, andando a ietto, 5A

CICA, Niente. Anzi vuoi dire (s¢ si pyo ) Manco di niente, dicendofi in di-
minuzione Poco, niente, Cica, Viene dal ato Cicum, che vuol dir Quel velo,
che si trova nelle melagrane per divitione de' suo} granelli, che per efter così (ot~
ule, ¢ di piun valore, serviva ai Latum per dimolirare la poca stima, che faceva-
no d' una cola, dicendo:. Ve Cicum quidem agderim, ec. ¢ noi diciamo in quelto
proposito /appola, lifea, er, y z sftp ue
> ZITTO.. Quicto. Stare zitto vuol dire Non parlare, Viene dal cenno.. 2
che frsuol fare, quando fenga parlare si vuoi fare incendere a uno, o pil, che
quictino, come iacevano ancora i Latini, che per accennare.ad.aitri, che fiiquie
tatle protterivano le due confonantt o *

 

nd

 
  
 

 

;. Se BOE
GNUCCO.« Enna specic di Pane gramolato, mescolato con anici; ¢ questas

patta fra le nobili € la più vile: Li proverbio Ogun pus far dedia a
significa ognuno ha ai libero arburio, ea ciprime quello, che 1 Li
Vind quifque in re Jua moderator, arbiter, ec. ii aun S
* suo danno, Non mimporta, Non timo questa cola, E dicemmo; 4/¢ f

. t

   

at

 
' PRIMO CANTARE: 5

tal cosa mt nociva, /uo danno io la voglio non offante ec, Esprime Io la vo
Bene mi pud nuocere, ec. Vedi sorto C. 4. stan. 26. al ermine Jn ogni modo.
-. RINC-ARARE, Accrescere il prezzo. E questo detto Rincarare i fisto usato in
Sao significa: Non fo stima,ne temo le male lingue, perché non mi pos-
sono far danno.
 BITTOQ.. Pigione, Canone, cioè Quel danaro, che si paga annualmente per
una Casa, o Podere, o altri beni, che si posleggono d' altri con pagargit ua tan-
'to lvanno. Locarionis canones,
. BALDOR/IA, Fiamma accefa in materia fecca, e rara, come paglia, ¢ simili,
che prefto s'accende,¢ prefto finilce; detta forie Baldoria da Baldore, 0 Baldan-
-za, che vuol dire Allegrezza: quindi Liera significa poi Baldoria, come vedremo
foro C. 2. tan. 56. Diciamo anche Far baidoria, quando altri spende allegra-
mente, ¢ fida bel tempo confumando tutto il suo havere:; il qual detco vien tor-
se da un religiofo coftume, che cra fra gli Antichi, che delle vivande fagre
non @i lasciatjcro avanzi, ma quello che avanzava s' abbruciaffe; il qual rito
si cava dai Precetti di Moist in proposico deli' Agnello Pafquale. Quetta specie
di Sacrifizio fu usata anche da i Geatili Romani, ¢ la dicevuno: Proteruiam fa-
cere, che vuol dire Par' una fiamma,o baldoria; B pigliavano ancor'eifi proteruiam
facere nel fenfo detto sopra di confumare, ¢ mandar male il (uo, come si cava,
da Macrob. lib..6. Saturnal.2., dove si legge, che Catoue motceggiando ua talt
Albidio, che haveva confymato gurto il uo havere,¢ folo gli era rimatta una Ca-
fa; laquale gli abbrucid, disse: Proteruam fecit 5 proprerea quod ca, qua comfes
mon paruerit, quafi combufifer s ec,
racigh ites cia eet ee Vv, =
Oferta glie gid, loc lay Ma poi ch' ella la vusleseiolbo promeffo
u ee o poi morfe le mant, Nae mandarla più doges ae,
» Perch? il filo nen va ne ben, ne presso, Che chi promette,e poi non ta mantiene,
E versiv'é chtil Ciel ne feampi i canis Si fa,? anima sua non va mai bene,
+ Mofira:l: Autore; che layconyenienza per haver' ¢gli prometia a S. A. R. quest'
Opera, l'obbliga a mantenere la parola,quantunque égli conosca, che non sia co-
fa d' efler.veduta da S,A. R., ¢ per questo s'¢ morfo le mani,-cioè peatito
'd' haverla promefia, perché vede che la tefligurg dell' qpera noa ta
1a bene, ¢ vi fon versi che #/ Ciel we /eampi i cani, civg cos: strop-
' i, che tanto mae nop. vorrebbe ess. jpeno a un cane:
) feampare attivo,come ¢ in questo luogo,, significa Liberare, Ma con-
l ia che S.A.R.\ la Table » hon sta bene che egli ja mandi pia in
domapi,ma ¢ dovere offeruar la prometia; ai che fare s'accigne
-quelta conuchicnza, ma ancora per il umore deila pena me-
promette,¢ nom mantiene la quale ¢ che L' amma sua non va mai
a da i nostri Fanciulli; ¢ viene dali' aatico-, poich¢ Itula-
reci fecondo il Monofino Fior, Ical, lingue lib. 3.9.109.

  
 
 
 
 
   
 
 
 
 
   

   

|

 

     
 
  
 
 

    
  
 
 

<

    
  
    
  

  

rurfus eri

pi: Che fuona lo steflo che; Chi da, ¢ riteglie
ale Lo iteflo che

Chi promette 5 @ nor mantieng L) anima

STAN: »

  

2 Nos autem dicimus id, quod (olent pneri: —

   
  
   
a
:
*
Si
'
.
<
r.

   
   
    
  
 
 
   

6 MALMANTTLE.
STANZA' VI
etiache? si comead un che fempre ingolla Cos) la voftr Idea di già [aroha:
Del ben di Dio, ¢ trinca del migliore, Di quei libron, che van per la maggiore,
Ml vin di Broxzi, un pane,e una cipolla Fore potra, femcendofi fuocliata,
Talor per uno si berzo tocca il cuore; Ear di quest anche qualebevc cm

Ripiglia animo il Poeta; ¢ spera che S, A. R. sia per'contentarsi di le;
uctta sua Opera, se non per altro,almeno per distrarsi dagli Mud) più ferij;¢ con-

fidera, che si come colui, che ¢ folico far vita lautitfima', havea talvolea di
mangiare un pane, ¢ una Cipolla; € ber vino da niente, così chi-è solito legger li-
bri piu fenlati, talora avera non poco gusto a legger libri di baie, facezie ¥

INGOLLARE. Veo! dit Mangiat prefto, ed inghiottire senza mafticare.:
Stuf pitt il verbo Ingoiaresefendo il verb» ingodare usato nel Contado, se bene &
forse meno barbaro che #ygoiare, perché ¢ più proffimo alla sua latina origines,
che è la proposizione Zr, ¢ guia, ed in questa appunto 'inghiortita la leteera 'L. se-
condo la stretta' pronanzia comane Toscana, € mutato in I ferrato 50 confonante
si dice comunemente Ingoiare: Così dice il sig.\ Francefeo Maria Bellini.

DEL ben di Dio. Delle pi buone vivande; che i Latini dicevano Jovis neftar,
e noi diciamo (arte di gallina, che vedremo in 'quetto Cant, stanza'64.

TRINCARE. Bere aflai; Voce che viene dal Tedesco; € diciamo Trine, 0
Trincone, uno che beva sregolatamente; Vedi forto Cant. 7, Manza 1,
' DEL migliore, S) intende quel che vuol dire', 'ma il nfo più altrafo puro Fio-
rentino è¢, che gli Olli 'di Firenze' vendono fempre due specie di vino roffo 'y und
di poco prezzo, che lo dicono Vino di forco, o di baffla, percht viene da' luoghi
dj sotro a Firenze, dove fanno Vini deboli, ¢ leggieri; ¢ l'altro di maggior prez-
20, ché lo dicono vino di sopra, 0 de migiiore; ¢ di quetto intende il at

TOCC ARE il cuore, Dar soddisfazione intera ¢ 'altri mangia con gu:
sto, ¢ si conosce, che quella vivanda gli fa pro's diciamo: Le tal vivanda gli ha
toccato il cuore.

SATOLLO. Sazio, Ripieno') Dal tatino /arar. era ye ae

oe BROZZL. B un'di quei luoghi orto Firenze, dove nafee' i deto vino debote:
Vedi orto in questo Cant. stanza 47. meen 7

'PER scberzo, Yarendi non per taint, 0 fete; sma per Revo so tomagufo.
E' voce Tedesea, ¢ la pur fuona lo stelso

ANDAR per ta maggiore, Efler della prima ' fle: Traslato da i Magitteati
dell Arti della Città di Firenze, delle quali 5: ena: 'che sono

 

'Giudici, e Notai; Cambio; Mer 5 Lana 5 Seta; Speziati, i
se paflano a Cavalleria, Alere Minori, che art eenan *) Quota eee
non paflano, 0: ra non pafiavano aca 'quando 'in
ze si dice, // ale va per: 'delle:

    
   

maggiore ss Sete 'una

are Arti, ed' della cap sw claffe, Come s' intende ie laogo's
LIATO. Senz' appetito: senza puto di mungeyo

eae opie..

: FAR una corpacciata, Saziarli. Empiee benitiimo il corpo =
corpacciata, gu altri legge, ree ° fa altra cosa'
te'fa una volta.

 
' PRIMO. CAN TAR E ?
seS TAN ZA VIL STANZA. VIL
Già dalle guerre le Provincie flanche, Che d' baverle non n'¢ nevia.ne modo,
Non fol pits von venivana a batragha, Se dentr'ad un mar d'olio nun si tufa,
Ma fur banditi gh archize/ tarmibiiche, E repute il padron degno d' un nodo,
Edsexiamiliportar un fil di pagiia Che lolafeia indurive, e far la muffa.
Vedeanfi i bravi accniattar /e panche > Coun Marte, che wede Parmiaun chiodo
fol menar le man fu la tovaglia; Tete! appiccace malamente sbiffa,
» Kuando Marte dal Ciel fa capolino, Che metter non vi possa fu le zampe
pres iL topo dall-orcio, al marzolino 5 Eche laruggin v'babbia a far le stampe.
j 'a principio all' Opera,. descrivendo lo stato, in che crano le cose del
| tee Mondo 5 eidicey che turso era in pace, ne si usava pili arme di forta alcuna; ed i
bravi,ed huomini armigeri aceu/artavano le panche, cioè Stavano Ozioli, ¢ menava~

no le mani folo in su la tovagiia, che viene a dire Aucndevano folamente a mangiare.
E qui scherza con l'equivoco del menar le mant 5 che vuo] dir Combattere, vedi
sotto C. 10, stan. 2,, ¢ trattandofi de] mangiare vuol dir Mangiare aflai, ¢ prefto

j vedi sotto C. 6, stan. 46. Marie però s' adira, che non s' adoprino. più l armi
f L' Autore aflomiglia Marte quando s'afiaccia al Cielo, ad un topo, che s'affacci
4 alla bocca d' un' orcio picno di cacio, ¢ d'. Olio, che s' adira per veder tal cacio
eheroee: dal padrone, ¢ di non poterlo arsivare, se egli non entra in detto

“hae ee Ate berle. Spada 7 pugaale sed egai altra forta d' Armi, a distiozion

oe "Arnele noto fatto di pane per uso di federe, ¢ possono starui
più in una volta; detto dai ini fabfeltinms »¢ viene dalla voce Latina
Planca, che significa Aflamcati, ¢ tayolati piani.,

ACCHI eee se panche. Signitica ( siccome habbiam detto ) Starfenes
senza far cola », ¢ spenficrato.... Ter. in An. disse Ofeitanres di coloro, che
fianno in, maniera., quali dica. Stanno shavizliando, che noi.diciamo:
See ment 30 Fare aru megli hai,o Dondelarfela, ¢ simili, che
tutti ci servono per Peries s Perder' i tempo in-vano, ed & quello che i 'Latini
dissero; Afanum shes sub pailio; f

TOK AG Ll» 'panno lino che si distende, pralla ménfa dai Latini
“ac patemacot) thabbiame fo forse da Toralsa, che crano i pansi, che cirewm-

4

   
   
  
 
   
 
 
  

Ur si keris difeumbentinm,
EVAR se mani, lee fo affolutamente 5 ire Far ione;
' ak te, Quando sMto affol vuol dire Far. quifti
r ara Affretiarlial lavoro, che fara aggiunto; ¢ Gi ula dires
» duno che corra afiai, Mena le mania leggere d' uno
in somma d' ogni Operazione humana, ancorche non facca
qui vuol dire Mangiar pr ue cas forto C, 6. tan. 46.
Guardar di (oppiatto. Quand' altri procura di vedere, fen2a
lc 'Japan diciso aun muro, Oaltro,, ¢ cavar suo-
scuopra quel ch'.¢i vuol vedere,.¢ gucte fidice Far
Dee teas è lo steflo
le di terra y per ulo di conferuar' Olio, vino, ed a li-

uvi, ed uggerv il cacio.

 

ie oak

nae

  
   
  
%s

-
'
-

 

8 MALMANTILE

MARZOLINO. Specie di cacio tondo fatto a piramide, e'col manico nel
fondo dalla parte più grofla; chiamato Marzolino,perché si comincia a farlo nel.
mefe di Marzo, ed € il miglior cacio, che si faccia nei nostri pacfi. E nel/pre-
fence luogo, se ben dice Asarzoline, intende ogni forte di cacio.

DEGNO di nodo, Cioè merita la forca per |' errore che fa a-non-mangiare»
que! Marzolino, lasciandolo andar male.

TVTITE U armi appiccarea un obiodo, Dicendoli: tale ha appiccate l armi
all' arpione, ai chiodo,s' intende: Ll tale ha abbandonate ! armi, cioé Lasciato
@' eflere armigero. Cid viene dagli antichi gladiatori, i quali quando dal or-
Jo, col porger Joro una bacchetta-crano afloluti, ¢ liberati dal-far pu il gladia-
tore, folevano dedicar l'armi ad Ercole, appiccandole nel dilui Tempio, ¢o-
me ci moftra Orazio lib, 1. ep...

— —— Keianins armis.
Herculig ad postem fixis, latee abdirus agro,
Et lib. 3, ode 26.
Vixi puellis nuper idumens,
Et militavi, non fine gloria;
iVunc arma, defunttumqnue belle
Barbiton hic paries habebit,:

SBVFF ARE. Dar legnid' ira. Sbuffare & quel foffiare, che fuol fare per fo
pia uno, che sia in colloras Traslato forse da i cavalli: E si dice Shufare,
quando altri adirato si duole, ¢ in uno steffo tempo minaccia'con parole'.

Dante Inferno C. 18,: Ud.,

Quindi fenriamo gente che si nicchia
Nell altra bolgia, e che col mufo shuffic,
E [e medesima con le palme picchia,

Viene da Bufo specie di soffic 5 che vedremo sotto C. 3. stan. 57).

CHE larnggin v' habia a far le Stampe, La raggine, rodendo il ferto, vi fas
sopra certe impreffioni simili a quelle, le quali con acqua forte si fanno nel ra~
me per Stampare, ¢ pero le dice Stampe. E

STANZA IX. ae
Shircia di quis di ld per te Cittadi, / Si-voltaseda un' occhiata ne? contadi 5

We altreguerre,ograh Campion discerne, Che già nutrivan nimicizie ererne y

Che battagie di ginocoacarte,e adadi, Enon vede iVillan far pik quiftions

E Stomachi d Orlandi alle taverne, 'In fuor che'con la roba'del Padrone.

Marte, riguardando bene per le Città, vede folamente 'di giuoco, ¢s
gente valorola, ¢ brava nel coins + Voltatoli poi ne i Coneadi > che eran già
Pieni di nimicizic, e riffe; vede, che dai Villani non si fa alcra guerra, che

che fanno con la roba del Padrone. ha

S41R. Ld, Sbirciare vuol propriamente dire Socchindere gli occhi, acci
l'angolo della vitta, fatto più acuto, posla offeruare con pil facilità naa m
zia; Se bene si piglia ancora per Guardar per banda, a fine di non”

 
   

vato s 'come fanno (peflo gli amanti; movendo la pupilla alla volta dell' angolo

efternio dell' occhio, con quel muscolo, che per tal cagione da' Medici si chiausa~

amatorio; E quelto Sbirciare, 0 Bircio, e Sbircéo ha fortel'etimologia dal Laci-
*: oy eae ao

 

 

:
s
    
   
    
    
 
 
  
   

PRIMO CANTARE: 9

mus, che Vuol dir'? angolo dell' Occhio. Verg. Egl. 3. Tran/uer/a tuenti-
mis; 14 qual parola vuol Servio, che abbia origine da hirens,eflendo che
Questi animali infuriati per la libidine guardano obliquamente, ¢ torto le capre,
che amano.
E pero vero, che il nome Bircio, o Sbircio si dice non folamente di chi ha gli *
chi scompagnati, ma generalmente ancora di chi ha qualfiuoglia forta d' im-
petfezione agli occhi, etféndo noi in questo non differenti da i Latini, apprefio
i quali se ben /x/ens vuol propriamente dire Vno, che ha folo un' occhio, come
si vede in Giovenale Sar. X. che parlando di Annibale dice: Cam Getula ducem
'aret bellua lufeum; che il Petrar. disse: Sour! un grande elefante un Duce losco.
ic. dé orat. Hic lufeus familiaris mens Cains Sentius = Lusciofus vuol dire Quel-
To, che ha Ia vifta corta, come si pud dedurre da Varrone lib. 8. difviplin, Stra-
be Quello che ha gli occhi torti, da noi chiamato Guercio. Cie, 1, de Nan
Deor, Et quos infigni nota Strabones, aut Patos esse arbitramur;che Petns significas
Vno che abbia gli occhi leggiermente abbaffati, che noi lo diremmo Luschetto.
*Porfirione annot, ad Horat. lib. 1. Sermonum Sat. 3. 'Peti proprie dicuntur, quo-
rum bue, arqhe illuc oculi veloctter vertuntur,ec, Coclites Quelli, che fon nati ciechi
da un' occhio. Plau. in Cur, Vrocule falue; ex Coclitum profapia te esse arbitror ec.
Lucini; Quelli che hanno ambedue gli occhi piccoli Plin. lib. 10, cap. 37. 4
ijsdem qui alter lumine orbi nascerentur coclites vocant,& quibas parui utringne ocelli,
'bucini vocantur, ec. one Quelli di vifta così debole, che non veggono se non
“Quando splende il Sole. Plin. lib. 8. cap. 50. Ss caprinum iecur vescantur, reffitni
'Velpertinam aciem lis, quos Nyilopas vocant,ec. Non oftante, appreffo molti que-
se differenze si confondono, pigliando speffo l'uno per I'altro; così appreffo noi
si confondono i nomi Guercio, Bircio, Orbs, Lufto, e simili, ec, accomodandogli
ite a qualfivoglia imperfezione degli occhi, come vedremo sotto in questo
Cant.
vuol

   

-

 
    
 
     
  
    
  
   

. stan. 37. che Orbe, vuol dire Affatto cieco, cioè Ocxlis Orbarus, ¢ stan. 66.. *
I dir Lulco, 2
~ CHE abattagiia di ginoco, ¢ 4 carte, ¢ a dadi, Non vede nel Mondo altre risse
che di giuoco, nel quale egli non ha che fare. Perché torna non affatto fuor di
polito una riflessione sopra la voce latina diea,¢ la voce Talus: si contenti
al Beetore, Che io faccia un poca di digre(sione, Sono molti de' moderni Lati-
“ni, che si servono della parola zea per intendere la carta da giuocare; ma forse
ooo ica » se vogliamo credere a Polidoto Vergilio, al Meurfio, al
: ro, a Raffaello Volterrano, ed altri, che hanno trattato de i giuochi anti-
la chiamano tharta luforia; & Alea chiamano Ogni specie di giuoco
se forse quei tali non voleffero foftenere la loro opinione con dite,
voce dlea ¢ prefa in genere generalitimo;allora significhi ogni spe-
fortuna: ma prefa in genere speciale, significhi la carta da giuo- 
imetto alla prudenza del Saggio Lettore. So bene che fino if
fa detto lea, come'fi cava da Marzialc. = x
9 & non damnofa viderkr; '
tris abpulit ila mares ec.;, "
per Fortuna, fecondo Livio lib. 37. che parlando d
iu tofto — che pace co i Romani per le dure coa
x:

   
 

  

Soe

dials
dizioni, che gli offerivano, dices, Nihil ea moverunt regem, tutam fore belli aleams
ratum; quando perinde ac vitteians fibi lezes dicerentur,ec, E Colum, in Praefat. lib,
1, dice Afaris, @ negotiationis alea, Pare che errino ancora, coloro, che piglia-
no la yuce Ta/es per intendére il Dado, perché veramente il Dado si dice refera,
¢ rales veo). dire il Tallone, cio¢ Quel' offo, che è sopra il caicagno del piede-,
donde si dice vefte talare, la vefte lunga infino a i piedi; E questa voce Tals,
trattandofi di Mrumento per giuocare ¢ |' af/ragalo Greco., che ¢ quello che i, no-
stri ragazzi chiamano aéiofo; ma questo ¢ forse minore equivoco., poiché tal'osso
finalmente viene usato in cambio di dado, servendofi per numeri di quelle mac-
chie, o fegni, che naturalmente sono in dew' offo,, come pil largamente diremo
forto C, 8. stan. 69. Gioviano Pontano nel suo Dialogo di Caronte diltingue que-
fio aliotio dal dado,dicendo; 4eque ego nangquam talis juft, nec referis. Lo stesso fa
il Gellio lib. 1, Cap. 20. che dice Ta/us cabus non eff, cubus.m, eft fizura ex omni la-
tere quadrata, te/sera fex lateribus constat, Marziale pure nel lib.14.ep, 15. moftea
tal differenza, dicendo.: Non /um talorum numero par, teffera dam fit Adaior quam
talis alea fepe mthi ec, Tal ditterenza si deduce anche da Cicer. lib. 2. de Divinat.
Quid n, fors eft ? idem propemodum, quod micare, quod talos iacere, quod tefferas.

E tanto balti per risponderea quei che biaGimarono i'haver noi meffo per espli-
care le presenti duc voci Carte,e dadi il latino Charta Luforia,© Te/sera, che per altro
non importava al cafo noltro questa digreilione, ¢ torna pil a proposito il fape-
rc, che tali giuochi tanto di dadi, quanto di carte, dice Platone in Pedro, che
futiero inventati da un tal Theut Dio de gli Egizz). Demoni ance ipff nomen Thent,
hunc primum numerum,@ computationem numerorum, Gi jam » Ap iam y
talorum denique, alearumgque ludos audivi, e¢. Raffaello Volterrano, ¢ Celio Calcag.
de Ludo Talario, ¢ Tefleraria,, dicono, che questi giuochi fuflero trovatida Pa-
Jamede nel campo Greco sotto Troia, ¢ però gli domanda, Palamedis, alea; si co-

, me fa il Soutero; Ma Lfdoro lib. 8. Originum, concorda bensì, che haveflero ori-
gine nel detto Campo Greco, ma da un Soldato, che havea nome Alea, ¢ che
da lui il giuocorprefe ilnome d'alea, Herodoto lib,1. riportato da Polid.Verg. lib,
2. cap. 13. dice, che l'inventaflero 1 Lidi per le caute che si diranao forto C..6.
tian, '

6 STOMACH d' Orlando. Dicendofi: 71 tale ¢ buono stomaco, 0 vero. Euno
fomaco d' Orlando, ec. sintende,il tale ¢ coraggiofo, ¢ bravo; Qui pero valendofi
dell' equivoco di Buono stomaco, che vuol dir Gran mangiatore intends Gente bra-

4 va nei mangiare,:

DAR ur occhiara. \otendiamo: Guardar' alla sfuggita. 'its
 FAR quiftione, Far contefa, disputa, rifla; ma dicendofi affolutamente
“fenz' aggiunta: Par quitione, s' inseade: Combatter con le spade, ec.
SFANZA X

Ond'ei ch' in testa quell' umor 8 ¢ fittay Niun fiata percis J ren feta 3
Pet CheVhuom ficrocchi pur ginftafuapossa;  Perch'ella dorme,e appitzod in[u lagrof

10 MALMANTILE fl

      
 

 

    
  

 

Ay
Senza picchiar,ne altro, ci feonfitto.  Poiche la fera havea la bu me
Liscis a Bellona manda itz una foofsa;  Cenato fuora, ¢ prefo un na.
¢ Marte cifoiue d'unirfi con la forgila Bellona a fine di meitere scompigli nel
moa lo, ¢ andato a trovarla, la vedo ia letco a dormire beia ca ancora delia fera
padaca. (5) ean. PAO.

   

£

se:
if

   
    
PRIMO CANTARE. it

stent + Questa voce, che per altro significa:materia umida., ¢ mii 5
parlandofi d' animali significa Flemma, collera, malinconia, ec, viene [peffo da,
noi prefa per Fantafia, o pensicro come nel presente luogo, che dicendo: S'¢
Bevo quel? umore in refta,vuol dire ha stabilito, ha fermato il pensiero, ha risolu-
to. La pigliamo ancora per Desiderio. Bartolomeo Cerretani for. nell' anno
“T5002. dice; Sifenti che [ umore di Piero-de' Medici, di tornare sn Firenze non,
era spento, ec, ea Papa Ale[sandro, desiderando fare il Valentino sus figlinolo Signore
di Toscana, si volle anch'egli valere di questo umore de' Medici,ec, Diciamo Bell'umo-
re Voo che ha fantafie graziofe. Vedi fortorin questo ©, stan, 58, Si dice#ar'.il
bell' umore Vano, che vuol far da bravo, ¢ da ardito. “/tale-volle fare il bed umo-
re col falire sopra quell albero,e casco, ec. Donde habbiamo Ymoriffa, -che figailica
Vno di-cerucllo instabile, ed inquieto. 'Haver yrand' xmore vuol dit' cer superbo,
ed thaver'gran pretenfiont di se medesimo. Z

'CHE Phnom si crocchi, 'Che I. huomo si perquota. 1 verbo crocchiare del qua-
le ci serviamo alle volte per il verbo -cica/are;come si vedra in questo Cant. stan.
4., oC. 3. stan. 3., e che vuol' anche dire Quel fuono, che fa un vafo di terra
cortd feflo.y come-Pentola, o altro vafo simile; ci serve anche.nel 'significato di
dar buffe,.¢ questo intende nel prefeate luogo; propriamente? Quel cautare, che
fa la gallina chioccia, quando ha i pulciai..
GIST A [ua possa, Pee quanto egli pud; Prafe antica latina inxta meum posse,ec.
FIAT ARE, Significa parlare. Vedi forto C, 6. stan, 12.
- Ein (u la grossa. & in al buono del dormire. Dorme profondamente. Trasla-
“to dal' baco da fera, il quale quando dorme per la 3.-volta,, che è il suo dormire
pitegagliardo; si dice: & nella grofia.

WON fente u# xtto, Non'(ente verun rumore, cioé ne.put'.un di queivcenni.,
xi che dicemmo' fopi quetto»Gant, stan, 3. 1) Varchisfior, lib. 6, dice. Com,
<abdertir sche ne cenat yne ritti, we atei brutti si facefrero,;

      
    
   
 
   
       
  
      
    
 
  

 

 
    
  

   

“eae Tatendiamo Cenar in:conversizione fuor di casa propri
- “PIGLIAR — + Ambriacarsi. Ci sono più specie di briachi, fra' quali fon
“quelli, che si dicono corti monne, che fon-coloro, che per lo troppo vino be-

vutojdanno nelle buffonerie, ¢ faltano, € chiacchierano spropositatamente, fa~
cendo mille altre pazzie, ¢ pois' addormeatano; € si dicono ancora Corti nonne,
sti liar la roma | Equeo.€ nome a: il goae comprende tutte le species
chi, di che pa forto'C.2, staa. 69, la questo C.stan.77. dice. 5” im-
briacaron come tante monne dal che deduci., che si pad dire: Prefe'la nonna ye preje
 da monna, che in ambedue maniere halo stesso figaificato,
s bg, 2A sty cS TAN ZA XL
= \ Sta cheto thetose con due man dipiatto
 Barre ha spada sopr' ad una cassa,
105 Ghia qual s aperfe., edei viftevi drento
4 Robe wane/che, a tutte feve vento.
'ogni-romore 4" che faccia Marte. non ne. ed
rove quivi in'una caffa.,“Esprime il Poera:il genio ta-
bondo di Marte, ela natura del Soldato, che & lempre dedita al Swe:
(prime ancora la briachezza di. Bellona, dicendo, che lla dosiuniva riaasiet.
4. Ha «tee

 
     
    
 

 
   

    

  

 

 

   
 
   
   
   

bs,

12 MALMANTILE

nelle materaffe sopra un letto mal rifarto; il che moftra, che quando Bellona andd a
dormire era ip grado, che non fapeva distinguere le coperte dalle materaffe.

LESTO come un gatto, La voce lefto, che viene dal Latino /ubiefus, che vuol
dir Leggieri, frivolo, ¢ debole, appreflo di noi significa Pronto, agile, ¢ deftro;
E questa comparazione Lefo, come un garto; da noi ¢ usatillima per espsimere la
grande agilità d' uno. Vedi sotto C, 2, stan. 35.

SALOTTO, Intendiamo Piccola fala, cio¢ yn ricetto prima che s' entri nella
principal fala. i"

MATERASSA, Arnefe da letto, quello che si dice in Latino Greco Ana-
clinterinm a diltinzione di culcitra plnmea, che noi diciamo Coltrice; eflendo las
materafia un facco largo quanto ¢ il leo, ¢ ripisno di-lana., ed.impuntito nel
mezzo,

Chero cheto. Quietissimo. Nota che la replica d'yna steffa voce, appreflo di noi,
ha la forza del superlativo. y

DI piatro, Cioè per lo largo della spada.

MANESCO, Vno che sia, diciamo noi, delle mani, cio¢ pranto, ed-inclina-
to a perguotere, ed.no che sia inclinato a rubare.. Qui però vuol dire Robes
atte, ¢ comode a esser portate via. Roba manesca intendiamo Roba, che ci sia
prenta, ¢ comoda a valerfene. è 'i t

FECE vento 4 tutte. Porto via ogni cosa. Rubd ogni cosa. Che questo inten-
diamo quando diciamo; Far vento a una cosa. 4

STANZA XII,

Ma non fa si, che la forclla sbuchi, S'allunga, ¢ si rivolta, come i ciuchi:
Di modo ch'ei lachiama,e lifafretta; Ella ch'ancor del vin hala spranghetta,
Lafalletica,e dice:Quvia fuor bruchi: E, fatto un chiocciolin fu l'altro late
Lo Spedatingo vuol rifar ie letra, Le vien di nuovo t afine legato.

Con tutto che Marte faccia ogni diligenza perché Bellona si fuegli, folletican-
dola ye sridande » che € hora di levarsi, non trova modo di farla deftare; anzi,
¢flendofi elia alquanto folleyata per causa di que' romori, s' allunga, ¢ si rivolta,
poi si rannicchia, ¢ di nuovo si addormenta, perché il yino la tiene oppressa. 5
Ed ¢ bella is pirtinee duno, che dorma con gran gusto, ¢ volenticri; perché
guelto rale, fentendo strepito, si risveglia alquanto, ¢ facendo,, per lo pill, le»
operazioni, ¢ moti descritti nella presente ottava, seguita a dormire. >

SAPCARE. Intende svegliarhi, ¢ levarsi; Vicie da.guella buca, la quale si fq
nelle materafle col pefo della persona, A

FAR fretta 4 uno, 8 intende Stimolar' uno a far prefto, alta as

SOLLETIC ARE, Stuzzicare leggicrmente ugo.in alcuna di quelle parti del
corpo, le quali, toccate così, incitano a ridere, Vienedal. verbo Sedlicito., folli-
sTPE SSO) TL PO AP | Liaw shgeucs ay oce't | ae

FVOR bruchi., Dalla voce Brucp habbiamo.jl verbo Bracare, che yual dir Le-
var le foglic a gli alberi, ¢ per metafora ire Andar via, onde quando.di-
ciamo: 4 tale sbraco y inteadiamo, Andd via sed, il simile incendia

For brichi y cioè andate via... Luigi Pulci Bec. Ognun bruco,.che, Hera Ia tre-
senda, Onde qui s' intende Efe, dat Jette. Detto, usatissimo in, queste proposi-

10, 3 says leant *¢

 

 

 
' PRIMO CANTARE: By

© LO Spedalingo vuol rifar le letta. Questo detto significa, E' hora tarda,¢ cas
levarsi dal letto; ed ha origine da gu spedali, ne i quali si raccegtano i Pellegri-
'ni; dove, quando ¢ hora di levarsi,, ¢ che i poveri, ¢ i Pellegrini seguitano a.
stare nel letto, Jo Spedalingo, cioé il Guardiano, 6 Sopraccid dello Spedale fuole
per (vegliargli gridare: S' hanno a rifar le letra,,
CIVCO.. Afino giovanc, Oypoledro. Forse dal latino Cicnr, che par che vo:
. glia dire Beftia addomefticata, ed agevole,
HA la sprangherta; 0 fanghetta. Quel duolo di testa, ed inquietudine, che si
fente la mattina, quando, la sera avanti s'é wore bevuto, ¢ poco quella not-
te dormito, per lo qual duolo pare, che il capo sia sprangato,0 legato con /pran-
Gghetta, o frangherra. Che così si chiama ogni verga di ferro, 0 regolo di legno,
che unisca due materiali insieme; come si dice porta /prangata, una porta, in
- mezzo alle di cui imposte sia conficcato a traverso un regolo di legno, affinché
dette imposte non si poslano aprire, E frangherta pure si dice quel ferro, che ferra
insieme ' imposte de gli usci, il quale s' apre, ¢ ferra con la chiave facendolo
sCorrere in certi anelli, come il chiaviftello, dal quale è differente, perché il 7
chiayiftello non si pud, o almeno non è in uso aprir con la chiave.
E.ATTO un chiocciline. Cioè Rannicchiatafi, 0 raggruppatafi quafi in figura
di chioccioja, come sono quelle focattole, o stiacciate, che fanno le nostre don-
ne per i Bambini,le quali chiamano chiocciolini,perché gli fanno a figura di chioc-
siola; ¢ come vediamo, che nel dormire fa per lo pili il cane.
LEG AR t afino,, Addommentarsi, Detto, che vienc da i Villani yetturali, che
essendo es rac soprapprefi dal fonno, legano l'afino, e s' addormentano nel
Juogo 5 dove gli piglia il fonno,, E col dire: Mrale ha legato (enza l'aggiunta.
@ Afino, s' intende; Il tale sé addormentato. Francho Sacchetti nov. 171. dice:
Essendo Gulfo entrato nel letto,guando fu per legar l'afino,il compagno comincio col man-
taco 4 fofiare. Bocc. gior. 4. nov. 9. Diche la donna spaventata, per fuegliarlo co-
qnineio a prenderlo per lo nafo, ¢ tirarlo per la barba, ma turto Gra nulla, perché c7li j
raviglia legato L afino. ec.
STANZA XIIL
O corna disse il Re degli Smargiaffi, » Oche per lagran furiacgliinciampali,
, E intanta le coperte bavenda prefo Och' elle fulfon di fovere!
Le ne tira lontan cinquanta paffi, Bafha cl' ei barte ilceffo, eche gi

ss eee egli si trove distefo; In testa la beffemmia delle corna..

Incollérito Marte leva le coperte a Bellona, ¢ le butta in terra, dove cascd
ancor' egli,¢ capo 5 ¢ si fece un bernoccolo., o tumore nella telta, quali
tumoretti da |

ore. fon chiamati Corna per elicr nel Juogo, doves i
seers piglia la voce ptennia non nel si proprio G-:
cil y © leyare empiamente alla Divinità quello che se Ic convic-
significato di maladizione, 0 precazione > come & pres
stra Toscana, ed in altre parti d' Italia, ¢ ipectalmenes in
€ inteso comunem:

inte per Maledire. E qui dicendo:
fence Quell' imprecazione che ha-

        
    

 
 

         
 
  
 

 

 
   
   
 

     
      
   
 
 

 

  
    
 

14: MALMANTILE.

uno di quei bernoccoli, o tumoretti, che per-éffer nella testa scherzofamente si
chiamano Corna.

SMARG/ASSO. Huomo bravo. Armigero. Ma però l*usiamo:per derifio-
ne, ¢ per intendere Vn' huomo fuor dei limiti della ragione, e deila prudenza,
ed uno di quei petulanti,'¢ minacciofi, che pretendono di sp ognuno 'con
la lor pretesa bravura. sao nd + Arey

CINQUANT A pafi, Lontano assai, Detto iperbolico:usato speflo anche in
piccoliflime distanze.

INCIAMP.ARE, Dar co i piedi in qualcofa 'nel camminare: @ il' Latino

'endere..

Foran pefo, Pefo grande, pefo fuor dimifura, Petr. Canz.'17.
Atri ch' io fieffo.,'¢ ildefiar foverchio, 

E certo che le coperte eran di grandidimo pelo, perché Bellona si serviva per
coperte delle materatie, come's' ¢ detto sopra. 5

BASTA. Termine conclufivo usatissimo da'Noi,quafi diciamo': 'E a sufficien-
za, e¢ si dice anche i4 baffanza, dal verbo Baftare, che è il latino fufieie. 1 La~
tini dicevano Aat, Sat ef. Piau.'nel Penuo si servt della voce Bat, senza aggiunta
di Sat ef, edi Giofatori di eflo-dicono: Bar vox, qua htimur Chm quempiam ine
bemus tacere, 4 4

CEFFO, Vuol dir propriamente il mufo del cane, del porco, ofimili, mas
si dice anche del Vifo, o faccia dell' huomo, ma per'lo'pia 'in derifione, ¢ per
intendere una faccia bructa, e mal fatta. Vedi forto C. 4. 'ftanv'ro.

« STANZA XIV.

Ella fuegliata allora esc) del Nidio, Cosa cht a Marte diede gran faftidio,
E dicendo ch' in cio gli fa il dovere, “Ma perch' ei nonvnel darlo a divedere,
E ch' ei non ha ne garbo, ne mitidio, Si rixza, e froda il colpo che gli duole,
Non si puo dalle rifa ritenete, Poi dite chewuol diiedue parole.

Per |' infolenze 'di Marte.,:Bellona'finalmente 'si fueglia,\¢ la la'burla a Marte
perch egli è cascato-, ¢ Marte fingendo non sentire la percofla'si rizza,¢ dice a

cllona, che vuole alquanto discorrerle... oi

VSCIR del nidio, V scir del létto: quale chiama Nidio per la similitudine, che
ha nelle-materafle quel luogo-, dove s' ¢ dormito., col -Nidio, entro al'quale co-
vano gli uccelli.. 2 eee
GLI fra il dovere. “Gli & interuenuto quel ch' ei meritava'. Dovere  \ginffo;

 

ginftreia-, (ono finonimi. i.

NON ha garbo, Non ha accuratezza | Per i di questa parola Garbo
è da fapere che-erano in Firenze due luoghi pri » dove già fifabbricavano
panni Jani d' ogni'forta,-uno detto S. Martino da'una'Chiela's chequivi«è dedi-
“cata a-detto Santo:,'e l'altro'fi domandava il Garbo} 9 idi Mtradefi con-
servano'fino al presente. 'Nel detto il Garbo'fi icavano le pannine di

tutta perfezione; ¢quelle che si 'icavano in S. Maitino'erano 'fempre
“feriore.condizione nae venne iniufo il dire': 'La tal cosa.è del Garbo,

denotare la petfezione di quella tal cosa., *E dalle robe venne alle
comincid a dire:'Huomo di garbo., huomo, che 'ha garbo yee, ndo d'
V.a0 »che operi bene, econ accuratezza.,'Cosi dice il aoe ete

  

 

 

 
  
  
  
 
   
  
  
   
 
   

PRIMO CANT.ARE ay

alla parola Garbo. E noi diciamo ancora in questo Senfo: Non ha ne Garba, ne
S. H#artino,
 AUT LDI0, Gindizio; ordine; Parola corrotta da metodo.

VON si pu dalle rifa ritenere. Non pud far di non ridere.

DAR faftidio. Dac noia; dar disguito.

NON vuol dario a divedere »Non vuol farlo conoscere. L'aggiunta della par-
ticeila, di, al verbo vedere s' ula folo in questo cafo per esprimere, far capace,
© render bene informato. u ss

FRODARE, E noto il suo significato,, vengndo dal Latino fraudere,, che vuol
dire Ingannare; Ma noi lo pigliamo ancora per Occultare, 0 noa manife stare,
come ¢ prefo nel presente Iluogo; ed ¢ traslato da quel frodare:, che vuol dires
Nascondere quaiche roba alla porta della Città, o.alla Dogana per fraudare la
Gabella con il non pagarla, che si dice Far fredo Vedi foto C, 6. fan. 28.

STANZA Xv,
Di pura Dea risponde, chr io ascolto Quello non fol; ma quanto baveva tolte
» Hai tu finite ancora? Ovvia, di prefto: Di quella calfa, ci rendese mette in feffo,
Ma prima di queipanni faunrinuolto, E postofi a feder fu la predelia,
E gettaio in ful letto yh' io mi vefto. Con gravitd atpoi così favella.

Deicrive aflai bene il genio inquieto,.¢ furibondo di Bellona, meatre moftra
l'ardenza, con la quale cilia stimola Marte a dir quanto gli occorra, interrogan-
dolo se egli ha finito, quando fa che non ha ancora cominciato, ed in uno steffo
tempo gli comanda, che rimetta le coperte in ful letto: Vbbidisce Marte, ¢ s'
accomoda a federe per dar principio al dilcorso, che sentiremo,

EAR' un rinuolto. Blo stetlo che Affardellare, abballinare, 0 far balle,

 AUTTERE in fefto... Accomodare; aggiuftare.. E ik Latino aprare, ¢ das
AMtetter in feflo diciamo Rafercare, 0 mercer in assetto. Varchi Storia libro 8.
Hau 4 di, € notte favorato per mettese il Salone in assetto,.L' Autore delias
storia de' Piacevoli, ¢ Piattelli lib..2. dice Wo» pareva possibile difender. ta fila, al-
degare i lasci, ¢ dar fefto al tutto, ¢ pure ben tofto si vedde mettere ogni cosa in assetto.
PREDELL2A. Qui inteode Quella seggiola facta a cafletta, la quale si tien,
vicina al letto per l'occorrenze del corpo; che per alrro queita voce predelia ha
mol ificati., chiainandofi prede//a ancora-guell' arnefe lopra il quale si pula.
0 | ido partoriscono; Predella si dice quelio scaglioae di legno, fo-
pra il quale ffa il Sacerdote quando celebra Messa; ¢ quella seggiola dove ficde
il Sacerdote quando in Chiefa ascolta le Confe'fioni detta alcrimenti Confeiliona-
le. Pre pure ¢ deta sapete parts della briglia, che si tiene in mano, coinc si
dino esposizione a Dante nel Purg. C, 6,

da com' e[ea fiera ¢ farta fella

corretta dagls sproni,
HA 5 OC. è “ha
fon SY ae re, mentees 4 seeanrate vuol dire
¢ madia contra, i i Cicalare, grac-
ur Gal: A tale non Chee heres » Me cicalava, ma Boek.
¢ parlava con fondamenco, regolatamente, ¢ feriamen-

rm 1 STAN:

   
 

 
     
   
   
 
 
 
  
   
 
   

     

 
 

 
    
7"

16 MALMANTILE:
STANZA XVL
Sirocchia, male nuove; poi ch' in Terra Sai, che la Morte ne molefta, e ferra',
Veggiam ch'all'armi pin ne[suno attende, Che la/ua firegua anth'ella ne pretende,
Onde il nostro meffiero,ideft la guerra, E se non [e li d@ soddisfazione,
Che [tain ful taglionon fa putfaccende; La ci fara marcir n' una prigione.

Marte in questo suo discorso moftra alla forellyla neceflità, che ambedue han-
no che si faccia guerra, per il bifogno, che hanno di guadagnare almen tanto da
pagare il dazio alla morte, acciò che ella non gli faccia metter prigioni, ¢ qui-
vi morire, se non le pagano detto tributo.

a SIROCCHIA Sorelia. Parola Fiorentina; ma oggi poco in uso. 'Dante nel
Purg. C.-4, ¢ Canto 21.; 4
Che se Pigrizia fusse ua Sirocchia, ev.
L? anima sua ch' ¢ tua, e mia firocchia, ec.
5 ST A in ful tagtio, Due specic di Mercanti di drappi, 0 diciamo Setaiuoli sono
in Firenze.1 primi fabbricano drappi per mandargli fuor diStato, 0 per ven-
derglia merciai di Firenze a pézze intere; i fecondi fabbricano, ¢ vendono in
Firenze a braccia, 0 diciamio a minuto, ¢ quetti si chiamano Setainoli, che spanno
in ful eaglio, Marte dice alla Sorella, che la loro arte, che /ta in ful raglio non lavo-
ra più, ed il Poeta scherza con l'equivoco di Tagliar drappi,¢ tagliar huomini;
€ che di questa lor' Arte di taglio vuole la morte, che essi paghino il dazio,, dan-
do alla medesima tanti morti l'anno; onde ¢ la guerra non lavora, non possono '
pagar ee tributo. 3

SERRARE, 0 far ferra a uno, Affrettarc, stimolare, violentare uno. Vedi forto
Cy 9. Manga 13. è

ST REGV-A, Intendi quel dazio, che devono alla morte. La voce fregua y che
vuol dir Porzione dovuta, vien forse dal Latino frena, che significa mancia.
Varchi Stor. lib. 10, 47 alcune cose vanno quei tali rispettati, ma in molte pis devone

andare alla medesima firegua, e ragcuazlio degli altri, ev. r % 7
DAR soddisfaxione. Soddisfare, Adempire ogni forte di convenienza, 0 di
- debito che uno habbia con un' altro: Ma mente s' intende Pagar quel da-
4 naro, del quale uno è debitore.; ey

CL fara marcir n° una prigione. Ci fara star tanto in carcere, che noi vi morf-
remo di stento; V' infradiceremo.;
STANZA XVIL

'. Bifegna qui pigliar qualche partite,  - C' ha dato un toffo nelle feimunito,
ie si Se noi non vogliam' ir nella malora 5 Mentre di Malmantil fitrova fuora,
IS Ed un cen' ¢ ch' è buono arcifquifito, E paffandola fempre in piagniftei » —
Qual! è, che si rifuegli Celidora Pigra si sta, come non rocchi a leis
: Seguitando Marte il suo discorso, propone y che si ponga in animo a Celidora

gid cacciata da Malmantile, di risolverti alla vendetta 5 ¢ così far nascere las
guerra; per rimediare a' lor bifogni. o2)- aaeras oe
PIG LIAR partite, Risolversi a pigliar qualche modo di rimediaré» ¢

 

 

; ANDAR nella malora, \ntendi Andare in prigione per questo debito. E il

latino Zz malam Crucem abire. e
e4RoIS QVISITO. A buono, diciamo in augumento; buono, pil ee '
¥ age di -buo-.

   
1. wate 7, 1 ees Se
Sa yt he

PRIMO CANTARE, 17

- buonifiimo, ed in luogo di buonissimo diciame anche squifito, facendolo superla-
tivo di buono =e cosinon,dourebbe patire agumento; tuttavia si dice Squifito, più
squifito, (quiGcidimo,o arci(quifizosimitado forse i-Latini, che da optimus superlativo
di bonus, hanno, opsimifimus, Si trova anche nelli Scrittori antichi della lingua nb-
stra. Vaccrescimento al superiativo, il Bocc.nov,19.dice Così santissima donna, E, nov.
Go. Gosh ortimo pariatore, ec, Gio; Villani lib. 12, cap, 104, dice: Rima/e in pitt pef=
Simo haze, ed: al lib. 7, cap. 100, La guile era della maggiore diS. Gio:, ed era moito
Sortissima ¢ cap. 101, A pie delle Atontag ne deste Pirre molto altissime, & queto Au-
tore l'usdfempre, che gli venne occasione d' e(primer un gran superlativo; mas
da i moderni.non pare, che sia molto wlatoy e con ragione, perché con Paggiun-
tadi molto, così, piu, © fmili 5 i) superlativo che ha la natura del suo nome, riceve
moderazione, ¢ pili tofto scema, e torna indietro della sua eflenza;; ¢ così volen-
do dire, che-una Montagna sia altissima con Aggiungerui il molto, così,v assai, si
viene:a dire che la Montagna sia alquanto alta, ¢ non in tutto alta, o altissima >
rigevendo in questa maniera il superiativo limitazione,e non agumento. Saluftio
disse muito puleherrimam » quando riporta il discorso fatto da Catone Viicen(e as
Cefareim proposito della congiura di Catilina,
» ha particella arci, che vien dal Greco archos,
'che da i moderni pen esprimere (s¢ si, pud.) di la
nostro. Poeta l'ula anche nel Cant, 12. stan. 34
particella arci aggiunca.al superiativo ta |' ecco
-derare:,.e-non accreicere, ec, Ah ug
' » RISKEGLLARE,,, Non dal fonno, ma dalla Pigrizia, then
Rada data un tuffaneHofeimunito, Ha factauna azione da sciocea, ¢ da stoltas*
~  Metaforico dari vintori-y,i quali volendoy che la feta.,'0 altro, pigli il colore, I'in-
tingone nel' bagao-di quel'tal 'colore tante volte 5 quante par joro che serva. £
questo dicono Dare uniruffo 50 pie ruff. E dicendoti W/ tale ha daroun tafe nello cim
unite St intende che quel tale habbia fatca un' azione da scimunito, non però
i fiaideltutto scimunito.. Questo termine dar' un tdfopud forse anche ve~
; da coloro,, che aflogano, iquali prima di morire tornano alla faperficie del?
i Sra due,"0.tre volte 5 il chediciamo: Darei tuff; eche,'s! intenda ¢ profimo
Ee del tutto.scimunito.,, come è-vicino a esser del tutto. morto 'coluiy, che da i
a 'nell acqua. La, voce scimunito credo che sia composta di due dizioni, cioé
feemo, (che vuol-dir? uno che habbia manco giudizio di quel, che si conviene) €
'Hilton © venga a dire wnitamenre sceme, cig Iccmo ugualmente » Odel pari, oi
le partia'un modo,, che'conchiade affarto sciocco y:¢ infenlato..:
Sf trova fuor di Adalmantile., E priva di-Maimanti perché le.¢ stato tolto da
&B 'fence trova cflettivamente fuora. Diciamo: Jo fonsuora di tal pen=
Sero per intendere:'io non ho più questo pensiero. 2k
4, PAGNISTEL. Singulti, (olpiri mescolati'con pianti.. Voce-da donnicciuole,
Y cee 1s 23 én; 8
= COME non 'tocchi a lei, 'Ciok 'come l'intereffe in questo negozio non sia,
'Staspetti a lei, mavad un! altro-,: “

 

   
  

 
 
      
  
          
    
 
 

 
 

1

    
 
    
   
 

che significa Superiore, stula an-
» © piu fu del superlativo, ed it
ma appreflo dime anche questa
» che 1' altre dette sopra di mo-

 

 

\ donating irate.

 
   
 
     
     
  
  
  

     
      
 
 
   

    

 

  

 

 

   
 

 

ee

aoe)! ek

 

18 MALMANTILE

P STANZAKVIIE > ° 5
Afa come quella, pare a me, che aspetta, \Flor mentre ch'ella inarme'non fimetra
© Cheile piovano in bocca le tafugne,- 'Per racquiftar lo scettro,e sue capagne;
Senza pensar un' Lora allavendetta €4iolto male per noi andra il negozto,
La sua disgraxia maledice, @ piagne; \ Che muoiam di mattana,ecrepia d'oxio,
Marte pone in considerazione a-Bellona yche se non trovano il modo di far ti-
foluer Celidora ad armar gente per racquiftar' il (uo flaro' di Malmantile 5 11 snes
gozio andra'inal per loro, che non hanno faccende. 3D Bemnierse L
CHE le piovano in bocca te lafagne?, Vol del'betie', ¢ non vuol' durar fatica aus
domandarlo: come per efempio uno che ha'gran fame } fitaicia più tofo'finired
da 5 aioe » che chiedere il cibo ddvutogli,'ma aspetta che il 'cabo gli corra:in Boos
ca da se, Coftume di Cuccagna.' ' srobover
LAS AGNE, Specie di pasta tirata; ed affottigliata come un velo', '
VN Iota, Piccola lettera dell” Alfabeto Greco, ¢ si piglia per esprimer ib mente,
MORIR di mattana, Morit di malinconia'; quafi dica: E 'cost'grande laymax
linconia, che mi nalce dallozio¥, che mi fa'divenir marco; emorire, *Vieneda
mitto, mattas,¢ forse prima si diceva': Perite di'morte mattanay ec. che era una
occifione speciale, che si facevarda gli Aruspicj nell" immolar le Vittime;'le i
faentravano vive, ¢-così morivano & poco a' pdco'crudelmente;La onde i Lati=
ni aggiungono fempre a-quetto verbo 1a parola' morte '0 supplicio, come filtres
dein Cicerone, che dice Aforre. mattavit.& supplicia maktari, 8 oN SMI
CREP-ARE, Questo verbo Crepare, che significa Quando un legname si spac-
ca, 0 fende da p c (e: significa ancora Morire'a stento,ed in questo fenfo & pre~
fo nel presente luogo'; © forse ¢ prefo 'nel (enfo d' Alléntare,'che vuol dire Quan-
do a uno per la foyerchia fatica'cal gl' intestini, ¢ voglia'Lroni pars
Jando, che s' intenda; ¢ così grande la fatica, che duriamo 4, che ¢i fa*allentares
ST. XIX. SoRAN' Z°A OMKE 0 op
| | Fartene'dungue, e sn abito'di mage, +.

 

Chia? forfo cofherife ne Ha cheta \

 

> Perch? ella vede\esser legata corta', 1) Dopo il formar gran civcoli ye figures >
Che # el havefje un di gentese monera ~) Conchiadi, ¢ dille-che.twifer

age
Tu la vedrefti nfeir digatta morta; ' Che ee 3 4.4

Aa qui Baldon farò dal” A alla Reta

 

(So quel chi dico, quando dico torta')

Ritrova tu coffer, fra feco in thono, ©

Che quit'al refta anchtio farò dé buono,

tafie ella si a'progurare di ra
Jona y che lavadia.atrovare,'¢ la tincuayi ton ditle'y cheypretto 'riayera
flato, ¢ le metta addosso l'usbergoincantatos 9) Hi -osoi akg

CHE fa? Questo termine

i '(ignifica; la tal cosa pad cffere,o hon pad el vai |
dica: Chié colui, che fa di ficuro, che la cosa sia, 0 non sia cbsi?-? peng
forze baftantia farqucilo 5 che ella 3

E' legata corra,, Cio' non ha f

E-quel tuo core lie'di drago
Ambottire Surah ta bradure 5
Mairile in deffo, che vedrala poi)

Bar lo spavaido pin, che ru non vuoi;
Marte facendo tifleffiante'y che se Gelidora havelle chi la foccofretie; ed:aias

  

ifare Jo ttato,percio' ordin aie

Traslato dal cavallo,, afino, mulo, o simili, i quali quando fon fieri ye bizza si
Jegano dovungue si sia con lacavezza corta, affinché non deta va loro”

d' attorno.

 

VSUIR

 

ae

wore ee Te whet ee

a

oh ted

 
    

    
 
 

PRIMO GANTARE: 19 ie

© VSCIR. di-gattt mortay.y Farsi vivo dimoftrarsi, fiero. Far la gatta morte vuol
'aicSimulare oll LaliBn. Tran, Cant, 2. stan. 12. parlando dsl Cavallo Troiaso

»
ami'h ms (2 Beanie ixgoarinleolt eqsafin legney isiqi? Py se
. ib cE w'sttendono a far lagatta morta',

Bi eatiai differo lepus dormiens, E noi diciamo anche far la gatta di: Atafino.
Vedi foro C, 7. stan. 69.

FARA dal' A alla seta, Fara puntualmente quanto bifognas Para il tutto.
LA, la Z. sono il principio, e il fine del nostro Abbicei, onde. con questo ter-
mine-intendiamo Sardsfatro il tutro'y come appunte appreflo i Grech Alpha, C.
Omegza.;che ¢ lo fetlo. che a, Capire ad calcem.de' Latint. » >:

SSO quel ch' io, dicos quando dicoturta, So benissimo.come sta-questo negozio,
Elprime a intend' io, U Pulci nel suo Morgante fa dire a-quello scellerato i

—a

Opa» ca credo nella torts,¢ nel Tortello:
2 9 Soquel, ch' io dico, quand' to dica torte,
Ez voo! dre M pao io aque! ch' io voglio dire y¢ queHo ch' io intenda per
terta ys
DSTA feco be tweet "iia feco unita; Va @accordo feco.;Traslato dalla MuGcas
a FARO! di. buono. Negoziero. da vero. Paro quanto bifogaa.,. Quando uno
iuoca di danari si dice Far di buono, che vuol poi dire Operar con attenzione; il
e non si fa: epenieeneht si giuoca di buono, » non ponendofi attenzione quando
ginacarda burla,. (o. 0
a Magy 'Nonchanno a i Maghiabito particolare' mail Poeta se lo
digara inpquella guilar; ehe ha veduto in commedia, cioè veltelungay granbar-
ha}c.lavergalinmano +.B Adagae voce Perfiana, chefigaifica Sapiens 5 ¢ quello
che i Grecirdicono Filofofo. E di questa forte Filofofi furono quellisMagi, che
andarono ad adorare Gicsl bambino.) Ma perché Zoroafte fu anch' egli uno di
cali Pilofofi detti Magiy:esecondo Plin, lib. 30. cap. 1. fu inventore dell Arté>
dell' incantare,, però:talrarte¢è detta Magia, ¢coluro, che l'efercitano fon chia-
 mati: 'Magi..Tatio Gerulal. C, ro. stan, 29.
tee Som detto Ismeno,i Siriappellan Magoy.
3 Weis [oo ls Machewdedl arti incognite fon vago.
E) Sper ene ares i feptodacP lids Verg..l lib. a:cap, 33. & dist specie,cioè Ne«
oC A » ¢Hydro-
si = a 'dettitancora ablegromantes ec, Vedi forto Cant, 2.

   
 
 
    
 
   
    
     

 
   
 

b » parrebbe ihe Benificaile Scelleraggine, © Scia-
' acne si pigtia < 20k per Disgrazia..» Boccaccio Novella 36. La»
. della mia ena BN. 43. E-della [ua feiagura do-
; toe adel paredeer ne se ne servivano nello steffo modo, che>
 facciamo noi per intendere Big 'Plaut..in Cape, Atsior poritus hoftinns wa
hoc eft scelus ? Quaji'in orbuatem liberos produxerim, Ter. in Kun. Neque Que-
um esse ego hominem arbitrary cui magis bons Felicitates omnes aduerse fiat. P

“of feeleris ? Il medesimo significato ha la voce latina — che a not
voce Sei. iagurate,.
  

 

 

20,  IMALMANTILE

CORAZZONE, Corazza grande, Armatura di petto,¢ schiene; dal latino
Thorax, si dice anche Petto a botta, perché ¢ a figura d' una botta, o perché si
prefume, che regga a una botta d' archibufo.

'MBOTTITO. Ripieno, ¢ trapuntato non di cotone,'9 altro simile, ma d'in-
sulti ¢ di branure, che vuol' ingendere Incantato, come vedremo appreffo nell'
ottava 27,

'SP.AeALDO. Huomo avventato; Huomo inconfiderato, Dal latino /uperua~
fidus Soverchiamente ardito, ¢ quafi temerario, ¢ tutto impertinent.

See STANZA Xa4l, STANZA XXH.
Bellona cha il medesimo capriccio Ove doppo mosprate ogni accidente

Di far bracinole, va col farracchino. Di tucca la sue vita pel paffato:y
Con il bordoneje'un bel barbon posticcio, Seggiungehe per viad'un fuaparente
Sembrando un venerabil peliegrino; Jn breve tempo riaura ta fhato;
E fatto di parole un gran pastriccio Peri si metea in arme,ch' un presente:
Esser dicendo aftrologe, ¢ indovino, Le fad'um panceron, che ancorche usato
Che vien di quel discofto pik lantana Ripara i colpi ben per eccellenza y
La ventura le fa sopr' alla mano; E poi piglia da lei grata licenza

Bellona va a trovar Celidora, ¢ fingendofi Aftrologo, le dice molte cose-ocs
corsele per il paflato, per accreditarsi; poi le predice', che fra poco tempo cella
riaura il suo Stato, pero i metta in armi; ¢ ke dona la corazza incanrata.,"¢ si

arte.

CeAPRICCIO, E Pensiero, fantafia, volontà., come intende anche forto ©, 6,
stan. 101. E per altro capriccio significa quello, che i Latini dicono orrere, che
quando i peli s' arricciano;il che segue @ per lo freddo, 0 per qualche subito spa-
vento, © ne i cafi di febbre, come s' intende sotto C, 6. stan. 14.¢ C, 20. stan. 2,
ners 101 habbiamo il yerbo accapricciare, che vuol dire Havere spavento. Dan+
te Inf. C22. y aK

Lo viddi, ed anche il cor men' accapriccia,”

BRACIVOLE, Si dicono guelle ferte cobain di carne di porco, o d' altro
animale, che sono ¢osi tagliate per cuocerle sopr' alla bracie, ¢ però dette bray
cixole, Ma qui intende fette d' huomini, ¢ vuol dire che Bellona hayea la mede+
fiusa volonra di far guerra, che hayeva Marte. =

SARROCC HINO. Eun collarone di cuoio, il quale adattato al collo cuopre
tutte le spalle, ¢ buona parte delle braccia, ¢ petto a jadi Manteiio, ed &
u(ato da i Pellegrini, che vanno a piede.a i Iuoght fanti; B questi tali fo-
no da noi chiamati Pe/egrini corrottamente da Peregrint; la qual yoce ¢ latina y¢
ritiene appreflo di noi gli stetfi significati di singolare, ¢ graziofo, ed anco di fo>
refticro, Peregrinus in domo patris mei, Petrarca Canc, 12. t ete
3 Moffe una Peliegrina il mio cor vano., geen
Et intende, che una graziofa:, ¢ bella donna moffe il suo cuore, E la derta voce,
Sarrocchiny ceedo y che venga da San Rocco il quale portava forse questa parte
abito., quando ando peregrinando.il Mondo. LS
; SOKO. E nome particolare, ¢ proprio di quel baftone:, elie' portano

cllegrini. we owdey

PAST RICC/O. Maffa confula di diverse rcbe.. Qui vnoldire quantità di pa-
role mai' ordinaic,: £ ee Dal:

  

  

3

Bait
Sa
bt

 
 

—=-.1. ee) Ce nt TS meee, Pee Pee ee ees ns
eer 1
PRIMO CANTARE, 24 qe

© DAL discofto piis lonrano, Pitt lontano della jontananza 'tefla, come diremmo:
Vero più del vero, o della steffa verità..
| FAR la ventura, Stcolagare. Sono alcune donnicciuole originarie d' Egitto,
de quali in Tofeana vengono il più delle volte di Sicilia, ¢ si chiamano Zingane;
Queste, dando a creder d' efler perite di chiromanzia per buscar denari, vanno
considerando i lincamenti delle mani alle persone, ¢ palefano ( dicono esse ) le cos
se paffate, ¢ predicono le future: EB perché dilcorrono artifiziofamente con certi
lor generali empre di bene; esse chiamano,ed anche da tutti noi vien detta que-
fla operazione; Far /a ventura, ola buona ventura. r
PAKENTE. Intendiamo ogni forte d' afini, o confanguinei in qualfifia gra-
do; così ¢ inteso nel presente iuogo, che vuol dire Baldone cugino diGelidora.
Così l'inavefe Dante nel Parad, C,6,, ¢ il Petr, Sons 191. E se bene stretcamente
vuol dire il genitore, venendo dal latino Parens, ¢ usato da noi in tal fenfo aflai
dirado., ¢forse non mai fuor che nel numero del pill, come l'uso Dante Inf.
be Le: '
———— Homo gid fui
E ti parenti miei furon Lombardi,
Atantovani per Patria ambi dui,
Ed il Petr, Canz. 29.:
è © ¢ Madre benigna ye pia
Che cuopri t uno, € 1' altre mio parente,
* P-ANCERONE,Antende quella gran corazza detta sopra in questo C. flan 20.
ANCORCHE" Hae, Adoperato, Vecchio, Antico, * °°
PIGLIAR buona licenza, Pighiar commiato, Licenziarsi da uno per andarfene.
E-quell' epiteca di duona, ograra s' aggiugne pér esprimere, che quel tale parte»
con buona grazia dell' altro, ¢ con il di lui confenlo, ¢ hon forzato, o seacciato.
» STANZA XXIII STANZA' XXIV."
Già il termine d' ns anno era trafior[a Fece [palucce a Calcinaia, ¢ a Signa,

  
 

 

 
    
 
 

 
   

 
     
     
 

 

» Che Celidora haves perduto il Regno; Ma la pania al [uo folio non tenne,
+ Quando né pur le/piacqueilcafooccorso, Perché terren nun v'era da por vigna;
» Adavolle un tratto acor moftrarne fegno, Calo nel piano,e ad Arno se ne venne,
> Percio rithiefio ai connicin foccorso, Ove Baldon facta nella Sardiena
fatto ns haurian col pegno, Vele spiegare, ¢inalberar' antenne

   
 
 
 
 
  
 

Fermato havendo ts come buon fito
D! armati legni un numero infinito,
erdita dello Stato di Celidora, dice,
I ja patiato quando la medesima comincid ad haver pensiero di
ricuperarlo, ¢ per cid fare, richiele foccorso a diversi vicini, ma senza frutto; la
 onde si risolué di venirfene verso Firenze, ¢ trové in fu la rivad' Arno in uns
con una buona armata, ',

“voce tratto ha molti significati dicendofi rracri di
» che si daa i delinquenti' nel martirio delia corda.
ira 'mo Queili uicimi moti, che fanno i moribondi neil efaiar lo
irito. Trarto si dice in vece di eftratto, cavato, o dedotto, ec, Tratro val per
tanza,dicendoli tratto di tempos twatto di via, ¢ simili, Trateo. dicortefia per”
i Atco

 

 
  
   

   
  

   

    
 

 

 

 

2, MALMAN TILE 9
Ato di cortefia,7'ratto per manicra, Edin mate tuoge: ioe Peas &

il latino tandem aliquando

VN piacer fatto non haurian col pegno. Ss lees Vacs chemo
a yeruno:, eziam se li fufle,daco-il pegno ia'mano.

TENER il.suo in rispiarme, Venere il fao ate,econ riguada s > taal dicono
r isparmio 2 ri/parmiare,,

GIVSTO, Questo termine significa Perl? appunto.

ERA come leccar marmo,, Bravana ogni ist per. appanto;come vaniti
Tecear' il marmo.

»FECE spallucce. Si raccomandd., Questo detto seas dai poverelli', che per.
muovere a compaflione in domandando-l'elemofina, fanno tutte le fmorfic,» e»
gclti, che fanno, ¢ podiono, ¢ fra gli altriil pia comune i Fare /pallucce 5 Siok
StringerJe spalle alla, volea del collo..

LA pania non tenne, Non fece cosa di buono, cioè non hebbe ainto da, colora,
dy quali lo sperava; inteadendofi con questo dettato, che quel tale, che fu richie-
flo, von adempi il volere di chi lo richiefe; cite diciamo ancora: Vax.ha trovata
appicco. 1 Latiai pure ia questo proposico ditiero Evannerunt infidia, Rania inten-
diamo il visco, col quale si pigliano gli uccelliy, B diciamo dom tenere quando 5
© per il molle; o per altro la pania non appicca, ne li prendgne son) ae LA

AL suo felito, Secondo il sao coftame, Dice al fao (olito-per-dimoftrare » che
'in quei paefi era da sperar poco bene al solito, perch? mon v' è terreno da por viene,
che.vuol dire: Non ¢ da far fondamento so da [perare da loro favore alcuno, ¢
scherza con l'equivoco del parre wigne,, perché verameaze quei paefi non 'hanno
terreni buoni a poryite-viti..

C.ALO' nel piano. Scefe;ne) plano, perch', Calvinaia., ¢ Signaifono ncaa
cOllinette vicin¢ ad Arno...

OVE Baldon faces nella Sardigna. LD Autore y che vnol fears flare i in fa i;

 

 

* durle., ¢ (eruirfi dello scherzo degli equivoci, fa che Celidora crovi Baldone nella

Sardigna; ¢ pare che voglia dire l'lola di Sardigna, eddatende dium logo suo~
ridelle mura di Firenze in fa la riva d' Arno, così detto per il fetorey: che.quivi.
fem, re si fente acausa delle-beftic del più condo y che morte si fanno in quel luo~
B0 scorticare: ¢ tal nome viene dayi Latini; che chiamavano; Sardinia. queisiuo-
ghi, li quali per li mali odori sono fotopot. »all'infezione dell'.ariayy. come è I
Fan di Sardigna, la quale per Aayere da Settentrione monti altissimi, che ie im-
'pediscona i venti, ¢ se riraacaaee aria »,¢forropolta alla-peftilenza. Di qui
ancora [i nostri Medici ban 'hanno peers nome di Sardigna a quel:luogo, acila 'Spe-
es è Santa, Maria Nuova di dove si mettono gi infermi te ferenti-
laghe, o-altro se ip ear riva.d' Arno chiamataSardigna,, si — -
i s etait: 2) si rigaricano, iNavili, che da) Livorno vengono a, Fir
sper lo flume d? Arno, ¢ tali es » che quivi.fon fempre.in gran

che fieno. Jrarmata di Te detto. oun i
Sere ceae ae pool Renths Bretho: eG serve,

'yore /ito per pafo', ed in eff. Mat' oiore sche
ie cueceee yela eet
+ Quello medesimo: 04

 

  

atone Sos ee elo
fachi angheanel

ad

?

 

 
Pee Se Te PE eee) PE MTGE gage Ti, oF

PRIMO CANTARE: 23°.
Situm casprorum secondo Ces, de bello Gallico, ed intendono' 'aheorò puzzo fecon-' |
do Plin. lib. 21, Peffimam a Crocum, quod. feng redoler, © |. 43
STANZA X “STANZA XXVI.;
Coftui quando Bellona fu ii pane ' Roi che'pedoni egli hebbe, e gente'in fella te

A Celidora, come ge s*intese Tanta ch' al fin ff i chiame soddisfarro,
~ Da Marte ibaveon wuta una fardata, Render volendo il Regno alla Sorelia,

« (Che lo tenne balordo pitt d' un mefe, E farle far bandiera di ricatto iS
E vli meffe una voglia sbardeliata Destind muouer guerra a Bertinella,
Di far bactaglia, ¢ mille belle imprefe; Ch aleigiadatohavea la feaceo mateo;
Ona' egli entrato in fregola si fatta Cos} con quell' armata, ¢ quei difegni
Feée toccar tamburo a /pada tratta, | In Arno meffe i sopradderti legni.
Marte erd staco a trovar Baldone, conforme haveva detto alla Sorella ye l'ha-
veva fatto rifoiuere a metterfi in arme per aiutare Célidora, ¢ rimettecla nello
Stato; € percid con questa gente a tal fine s' cra imbarcato.

FARDAT-A, Percotla data con un pannaccio intinto in sporcizia; perché
farda yuo) dire fornacchio, che ¢ Vin grande sputo catarrofo. Vedi forto in que-
sto Cant. flanza 47. E's' intende ancora per Vna quantità di sporcizia bicumino-
fa yche tirata in qualche luogo s" appicchi,¢ s' interai in quel Juogo dove è butta-
ta, come farebbe una manata di fango, 0 altro simile buctato in un muro; Dal
che per metafora intende in quello: luogo per Vin ay  ohe s' appicchi, ¢ s*in-
terni, quella persuafione, che Marte haveva fatto a Baldone di far guerra.

* BALORDO., Quolta voce che vuol dir fonavvertito, Smemorato, che è il
latino mente ¢ peas 5 ci serve per intendere D' tino', che j per qualche accidentes
ito, enon sappia a qual partito appigliarsi, per rime-
eare al danno'che: da-quello accidénte gli rene 3 ¢ si dice anche Shalordito, Stor~
Vedi sotto C. 11, stan. 25.
cebubbann LLATO a coia che leceede i termini del naturale, ed in-un certo
modo: che si dice) Grande, pili grande j grandissimo, ¢
ja', ¢!poco'usata; Bi forse meglio Diforbicante; o Ya=
moderato s ss juonaad lo stetfo:. L#Autore del Capitalo i in ate de” peducei eer
7 (cde to cingueihore del giorno in mercato. °
<- Apafeer gli occhi di si belP oggertal, 2 96 ho BEM i
Von) \\9 Ewe cavowun pitcere shardeltato,
ifBBOO Voglia grande. Onde viol dire Entrata wnifreeit # fatta jatende
Essendogli venuta così gran ee  traslato dai pesci ¥ che fidice Andare ix '
quando's' adunano molti inficme per la gentrazione; ed è il latino Jibi
eae eae gatti,quando | (0 in'amore / Veli forto Cant.,
mh G unGevt 0, RAL
\ Vuol dir siiadice aacblehttn 'sintende 'Aruolare.s
dati 5: caffe Vedi Corto CG, 3 Ran. 56. ©
A spada: iposo%, Senza intrmulione 5 i Senate
' t ie
EAR bandera pane Ricactat Vendicart # Quetta voce Ricatto 3 hed
enidal verbo Ricatcar/ uO! propriamente dire Liberarsi di schiavitu-
cf dg een watery ved © il Latino par par

referee

      

   
     
 

    

 
  
 
 

     
 
 
 
 
 

 

 
    
      
   
     
   

  

ae

 

   
: MALMANTILE:

; 24
y roferre. Ul dettato Far bandiera' di ricatto timo che venga dal coftime dei Corsa=
ri, li quali, quando pigliano qualche legno, che stiminu d' flere in. gradoda ef
fer ricattato,, v' inalborano una bandiera bianca, con la quale,danao cepno alle
Terre vicine s¢ lo vogliono ricattace; il che se voglion fare, corrispondono.¢on
alzar bardiera dello stesso colore.; ¢ questo dicono Metter bandiera di ricarto.
DATO havea lo scacco matte, Le havea fatto questo dango.5.0 cagionata-que-
fla rovina. Li giuoco delli scacchi è antico, ¢ fu usato prima da i Greoj., che
oralo dicono Zarrici, ¢ poi seguitato da i Latini, che lo dissero Ludug latrun-;
culorum,, A quetto giuoco si da fine quando ¢ faito prigione i Re » ¢ si dice allo- f
ra feacco matto; onde qui vuol dice, che Celidora havea toccaro Scsccomarto, ha-
vendo perduto il suo Regn: Es' allarga quello detto a tutto quello, che adal-
tri fucceda di gran perdita, 0 di grave danno. ey
STANZA XXVIL,

Ou anco in breve Celidora.arriva Chel' usbérgo incantato della diva
Con armi.in doffo,ed altro da far fete, Liha fasto diventar | Ammarxafette,
Perché una volta al fin fatrafi viva Ed alle riffe incitalatalmente,
Ha rifaluto far le sue vendette; Ch' ella pixzica poi dell infolente. 4

Celidora arriva all' armata di Baldone nella Sardigaa, ¢.quivi comincia a. mo~
firare gli effetti della Corazza incantara. YO
è eh da far ferte. Intende la spada, ¢ vuol dire che ¢ra larga, edrabile a»
far fetce. i 5
FATT AS! viva, Rifentitafi, ¢ fattafi ardita., E lo stesso che P7cir di-garra
morta detto sopra in questo Cant. fan, 19.:,
YSBERGO, Ciok quella Gran,corazza di pelle.di drago: detta sopra,lacquale ib
Poeta qui dichiara, che ha intelo y ivcansata quando ha derto:sopra smbetrita a
infulti, e di bravure alla stan. 20.
3 AMMAZZA fete, Contano le donne una novella per trattenimento.de'Fan-
ciulli; ¢ per accomodarsi alla loro capacità 5 dicono::, Fuyna volta un beligiova»:
netto in Garfagnana detto Nanni, il quale per la sua:mendicità dormiva in una.
capanna da fieno; quivi eflendo egli un giorno per, ripolarsi, ¢ ripararfidal cals
do, si meffe a pigliar le mosche,'¢ ne -haveva ammazzate fette, quando com-
arue quivi una bella Fata,e gli disse; che se le donaya quelle sette mosche per ci-
ire una sua paflera, ? haurebbe fatto ricco, Gliele.concefle egli pili che volen-
tieri; ond' ella innamorata diquefta, sua cortele prontezza loprefe pera. mano, !
¢ lo'conduffe alla sua caverna, dove riveftitolo, ¢ datogli danari), ed apmi., gli
ofe in testa un' elmo, o bs cra lcritta@lettered! Ora, 2 lommanedn
'tte; ¢ lo mando al Campo de' Pifaniyi quali in quel:tempo. con lt-aiuto de Frans
zefi guerreggiavano co i Fiorentini'. Arrivato Nanni a “detto Campos:chieles
soldo i Pilani,edon i err te ena eal
: io folo in un giorno am es « soprannome: dmmazxa/fette, Bu per
Pa Se< questo;¢ eee anche ben orcas buon solde.¢: 'con: noniminore si-
2 ma accettato. Essendo poi fra pochi giorni in una scaramuccia 'mortal Capo
'delle truppe Franzefi, ¢ volendone essi fare un alteo,, crano fra didorain gran,
'diflerenza, perché eflendone propolti diversi, coloro, a' quali.'nom/piacevano. i 4
: fey edenee 2
ro,
.

 

 

   

    
 
 
 

exti propo gridavano Waniy Veni, onde i

 

 

 

i ae Fa a ii i

 
PRIMO CANTARR: as

0, che diceffero Nanni, Nanni, e che haveflero creato lui: cOminciarono as
gridar Nanni, Nanni; viva Nanni; ¢ così a voce di popolo Nanni detto ! Am.
mazzafette reftd eletto capo di dette trappe, ¢ divenne ricco, si come gli haveva,

romefio la Fata. E di questo intende ij Poeta, volendo moftrare, che Celido-
ra era divenuta brava, quanto questo Ammazzafette, il quale non fece mag-
gior bravura', che ammazzar quelle fette mosche, si come ne anche Celido-
f Ta non fece maggior bravura, che affettar quei Cavoli, che vedremo nell' ottava
29. seguente. r
ALEE riffe incitald talmente, cl? ella pizita dt infolente, Bellona le fa venir vo-
glia così grande di far risse, che ella vien poi a noia, ¢ si rende odiofa con i suoi
modi impertinenti. 11 verbo Pizicare vuol dire Cominciare a eflere, 0 Efleres
alquanto. // tale ¢ stato tanto tempo in Firenze, ch' atta di Fiorentino, Lo trovo
anche usato da i Bolognefi in questo fenfo, ¢ l'usd Francesco Negri nel suo Taffo
in lingua Bolognefé Cant. r, stan. dove El pizigava di fei ann' ch'i Tramuntan,
ec. per intendere, Bra già prefio a fei anni, ec.

INSOLENTE, Si dice colui che da faftidio, ¢ noia a ognuno, € che firende
ediofo a tutti con Ie sue azioni impertinenti.
STANZA XXVIII, STANZA XXIX.

Non così tofto al campo si conduce, Se guarda, ¢ dispertofa, ¢ impertinente,
'Come la fuora viol del Dio Soldato, Efempre vual che sia la fu di sopra;
La Marfifa di nuovo posta th lice, Talor' affronta per la via In gente
hell' esce afarto fuor del ferminaro; Cercando litt, quafi franchi  opra:
col brando che taglia,com*ei cuce We venga ( dice ) pur chi vuol niente,
ja far proprio morire un disperato, Pero che,chi mi da che far mi seiopra;
'nol trucidar' ognunoognun vuol mort, Giunta in quest in un capo pien di cavoli
 Eguas a quello, che (a guarda torto, WV afferto tanti, che Beati Pavoli.

Defective il Poeta una brava (proposicata, ¢ impertinente, per moftrare in Ce-
~ Jidora gli effetti del? incantata Corazza; ¢ con queste azioni, che le fa fave, dipi-
gne al vivo'wno di questi spacconi, ¢ ammazzatori, che noi diciamo che Cam-
“pano di fegati d' huomtni, ¢ fon poi il ritratto della poltroneria, ¢ sfogano la,
“ for bravura comé fa Celidora, in un campo di Cavoli. or
COME 1a fwora vuol del Dio soldato, Come vuol la forella di Marte, Bellona.,
per opra della quale Celidora ¢ capirata a quel campo
 MARFISe4. Donna guerriera nota, favoleggiata dall' Ariofto, e però la di-
 C0: di mnovo posta im /uce, ed intende una Marfifa moderna fatta brava da Bellona,
~'cioé Celidora.,; ov
=) VSCFR del feminato affarto, Perder' il (enno del tutto, Fmpazire. Quando al-
i per un grandissimo contento si railegra più del dovuto, diciamo: II tale #m-
¢ per V allegretiza; € così intende di Celidora, non che veramente sia im-
ita. s Latini hanno il verbo devirare, che vuol dire Impazire, ed è¢ metato-
ico 'dal bifo! i com dalla preposizione De, clie fuona extra, © li-
rare, che vuol dir Fare i tolchi nel campo con l'aratro; econ questo fol verbo

lelirare intendono extra liram incedere, dove noi diciamo Vicir del seminato, che
blo stesso che e-xrra liram incedere, 0 delirare, del qual verbo ci feryiamo ancor
'nei medesimo scalo,come si vede in Dante. Nt ha

: eet

 
 
 
 
 
   
   
     
  
  

ee

a

   
   
   
 
   
    
  
 
   
 
     
 
 

   

   
      
    
      

Pat.

  
26 MALMAN TILE

Ed egli a me; perché tanto delira
Hoegi l' ingeguo fuoda quel che fuole,;
E si dice anche dedivo uno,. che sia fuori del fenno, Dan. Par. C, 1.
Che madre fa sopr' al figlinol deliro,

Alcuni vogliono, che.questo vesborDesirare venga dal Greco, Lirin, che vuol
dir scioecheggiare. Diciamo nel medesimo significato Vjeire de/ feminario, E que-
fio forse deriva dal Latino Seminarixm, che secondo Colum, lib, 1. de arboribus
c. 1. 3. vuol dit quel luogo, nel quale si feminano le piante per trapiantarle,il che
quando segue,la pianta cavata.dal.detto Seminario refta come un pesce: fuor dell'
acqua, € piantata poi ripigiia il vigore,quando ha cominciato ad attaccarsi nella
nuova terra; € da quello, dicendofi huomo fuori del Seminario s: intende' Huomo
sbalordito. Si dice ancora fuoridel secolo, ¢ habbiamo rrafecolara, ed ii verbo
Strafecolare, Vedi forto Cant, 6, stan.36. pur tutto.a questo proposito.. Ma si que-
sto, come gli altri fuddetti termini, con cutco che poslano crederfi l'accennate»
derivaziont, io stimo che intanto s'ufino inquefto proposito, in quanto hanno il
principio della parola, che fomiglia quello della parola /enma; e-che fidica fuori
del Seminato, Seminario, 0 Secolo in vece di dire Fuori del fenno., E questa specie
di parlare, che è specie diyparlar furbe(to, è molto usarorin Firenze per scherzo,
¢ lo.dicono parlare /anadaitwo, il qual parlare riesce aflai graziofo,quando è ma-
neggiato da persone (piritofe,. perché taluolta con parole, chenon hanno che
fare con quelia materia, della quale si discorre, vien descritta per allufioni, 6
per metafore, 6 altrimenti quella tal cosa,della quale si parla. Per efempio; Ad
un Priore, il quale a tre mogli ». che haveva havuto, non-hebbe mai figiinoli, ed
havea nome Antonio,di Priapo annebbiato, Adan Proposto,.che haycas
nome Girolamo, ed era lungo, fecco, ¢ di colore olivaftro,dicevano; Pro/eintto gi-
rato. Di questo parlar' lanadattico si serve sotto C, 9. stan. 1, "

T AG LLA come ci cuce, Tanto ¢ buono a tagliare, to buono a cucire, che
vuol dir:non taglia. Detto usatitiimo per intender Ogni forte di coltello, o.ar-
me, O forbice, che per la ruggine, o altro non fieno atte a tagliare.

FAR morire ux disperato. Dicono che le ferite fatte con i ferri rugginofi, 6 in-
taccati, fieno pericolofe di cagionare spafimo, ¢ percid do si vede un coltel-
lo, o arme di tal forte, si fuol dire Farebbe morire. uno ai} roe » cioè.didolori ec~
ceflivi, 0 di spafimo, E tale era la spada, 0 brando di lidora x:

GVA a quello, Male, 0 gran disgrazia avverrebbe a colui 5.chelaguardaffe
torto. Bil Latino Ye id, )

GVARDA torto, Quand' uno non è molto nostro amico,diciamo;-Zi,tale nen,
smi vede con buon'occhio; 0-yero mi guards torte, Che iLatinipure diconoVen re-
is aspicere oculis, sgetier

ee ETTSSA » Huomo altero, ¢ chedisprezza,ognuno, ed' ogni piccala,
cosa.s' adira. i a, q

IMPERTINENTE. Vno che vuol più del fao-dovere, 0 del giufto; »\o;più di
quel che gli s*appartiene; scot a:

VVOL che ls [ua y stia,fempre di sopra, Waol fempre haver ragione.» che-si dice
anche Sopraftante, E questt tre modi cioé een » Sapertinente 5 eSopraftante
si posion dire Sinonimi, ¢ significanti Huomo: Pane. Gectaeampea OTRO ES ES °

i iper-

 

 

 

 
PRIMO CANTARE: 1

superbia, compagna indivifibile di tutti: gli Sgherri', € bravazoni a creden-
za

i

f AFFRONT ARE, Vuol propriamente dire Affaltare il némico, ma si piglia
's ancora per Andar' incontro, o Affacciarsi a uno per'parlargli, ¢ così ¢ prefo nel
presente luogo, per intendere che Celidora cercava spropositatamente I" occasio-
ne di far quiftione, ¢ tutto per descriverla simile a i detti bravi di parole.

CHI mi da che far mi feiopra, Dourebbe dire Mi sciopera', fecondo che'da al-
cuni troppo delicati',.¢ punto considerati ne fu'avvertito il Poeta, ma la figura
Sincopé-( ammessa fra i Latini) Verg. 5.nedice gubernacio in vece di gubernacule
da noi € accettata anche:nella profa, cd adopratacomanemente in molre voci,
particolarmente in quetta;dicendofi pili [peflo Opra', Adoprare, Scioprare, cheo
Opera, Adoperare, ¢ Scioperare, lo libera da-questa ceafura., EB questo termine Chi
mida che far mi sciopra proprio di certi Taglia cantoni', che yoglion con esso
moftrare, chechi'da loro vccafione di far quiltione gli sciopera, cio li leva dal
farne up' altea',, che han tra mano, ¢ Ji leva da un lavoro per impiegargli in un'

altro simile.
WN APPETT OP tami, che Beari Pavoli', / Ne“taglid in fette grandissimo nume-
ro. Quando. vogli beftare. amb: dardo, fog) dire; Gran,

 

danno che farebbe:cuftui imun' orto di cavoli,odiraduchi, E quel detto Zeati Pavoli,
ha origine da un Mvntanbanco, il quale vendeva il rimedio contro a' veleni con
dichiarazione-di- voler-donare ( come effettivamente donava ) la pietra di S.Pao-
lo-atutticoloro,, che havevano nome Paolo, onde infiaiti plebei per buscar quel-
la pictra dicevanordi haver nome Paolo; ficch® egli comincid ad esclamares.
© quanti Paol, orquanti Paoli. E perché quelli, che ottenevano quella pietra
i tenevano fortunaciper haver' havuto i) regalo', ne nacque il dettato. Sow pik
che'non furonoi Paoli Beati, che vuol dire, furom moltissimi; Che la voce Bears in
| eget @finonimo della voce felice', o forcuaato, Beso voi che fiete ricco, per

clice, o Fortunato voi, che fire ricco. <

STANZA XXX.

 

Così pienadi fumi, ed umor bravi Eva per infilzarne [ette otravi:
- Chetet! hanno cavata dt Calende, » Ma nel pensar di poi, che se gti ofendé
} Rivolze l'occhio at popal delle navi, Far no porrebbe lor, se non mal ginoco;
 La dove Brescia romoreggia, ¢spiende, Gli-vnol lasciar cainpare unt altro poco,

 Celidora facendo queitefue bizzarrie y vede la gente di Baldone, ed efiendofi
inferocità/in'quei cavoli', gli vien voglia di far' io fefio in quelle genti, ma si
ractien di fario pet non dar loro disgufto, ¢ per lasciargli campare un'altro poco:
PIEN Adi fumiche'tel' hanno cavata di Calende. Moftra il Poeta, che Cell-
dora sia meno, che briaca in questa (ua bravura-, i fumi della quale le hab-
Bias odlafeapoil ii\ceruélio, come fanno i fumi del vino a chi troppo beve, che
questo incendedi¢endo'l hanno:cavata di calende, ed @ quelio che i Latini dicono
extravcailem efeyedio credo:cheda' Jatino callem venga la corructela di ca-
lende;.¢ per»parlaellanadattico detto (opra in questo G. stan. 28. si voglia di'ca-
vara del calle per invendére ( come facevano i latini )'Cavata di Ceruelio.
. BRESLTA vomsrexgia, esplende. Si (ence romor d atmi, ¢ & vedono risplen-
der i¢ medctiin:. A Breicia fefabbricand buone, ebelic armi, ¢ però 1] Poca
x Di pigiian

ponte

   
   
  
 
     

 
shies

 

28 MALMANTILE,
aro
5 che:

pigliando Ia Cited per J'armi, che in quella si fabbricano, seguita-T' nfo-nostro; che:
€ di dire tale ha tutto Brescia addosso, per intendere Hs moit'armi addoss.

STANZA XXXL STANZAXKXA&IL.

Al fin, deposfo un' animo si fiero, 'S' abbocca appunto con Baidone fheffo's.
Un genio cangia a poco a poco l' ira, E fentendo ch  egliha tai gente fatte
E' come un' orfacchin.,c' apie d' un pero Per rimeiter in fefto, ed in posseffo
wd bocca aperta i pomi suoi rimira; Via Cugina sua ch' e-perie fratte,
Ferma impalata quevi'com' un cero Ben belo (quadra,edice:Egli¢ pur defsol
Fif[ado in loro il (guardo, fuiene,e/pira, Or fu ch' 10 casco in pic, come le gatees
LVe puo viver al fin se non domanda Ed esclama dipoi: quest! € un' aRione 5,
Ove ? armata vada, e chi comanda. Che veramente ¢ degna:di Baldone,

Celidora pero appiacevotitadi, si ferma a guardar con gusto grandiimo quei
Soldau, ¢ domanda di chi è P-Armata, ¢ chi la comanda; e s'abbatte a doman-
darlo a Baldone, il quale gli dice, che ha facto quella gente per aiutare una sua
cugina, ond' ella riconosciuco Baldone, si rallegra, ¢ dice: veramente quefia ¢
un' azione degna di Baldone.

Ce4NGLA l'ira in genio. Cioè dove prima haveva l animo @' infilarne fert'
ottavi, adeflo comincia ad haver genio con loro, ed a portargli aftetco.. Questa
voce genio se-ben non pare che Toscanamente significhi cosa alcuna, nondimeno
€ moito usata dicendofi Auomo di buon gemo, o di cattive genio per intendere Hue-
mo di buona, 0 cattiva indole, o inclinazione. Haver genio con uno E: jo stetio
che Haver fimpatia con uno. Appreflo i Latini pure se-ben genio non si-distingue-.
va dall' anima ragionevole, ¢ molti lo pigliaficro speflo per Lares; altri per ght
Dei Penati-, altri per il Dio-del piacere, altri per li quattro elementi, altri per lt
dodici fegni del Zodiaco, altri per lo Dio che faceva nalcere y ¢d altri per diver-
se altre cose; tuttavia efli pure se ne serviuano per intendere inclinazione, —;

' ¢imoftra Plauto in Truculento.1, 2. cam genijs fuis belligerare 4 ec. idem 4

defraudare geninm.: 1
COME un' orfacchino a più d' unipero, Si dice L' orfo fogna pere; Leva le peres
ecco l' orfo, Dal-che si cava, che questo animale sia molto ghiotto delle pere; il
be anche atiefta Vincenzo Martelli nel suo Capitolo in lode delle menzognes
jicendo: ' A “
Oggi i voi pitt ch' advaltri si connieney;
si eati Benché noi fiam tant' orf a quefte yidepite i. < t gin
E si dice che in rimirarle gioisca tutco cre ola tperene di conseguisse;
percio |' Autore afomiglia Celidora aun picciolo Orfo a piéd'un pero,perché in
yeder quella gente, la quale ella (pera che sia-per-lei, si rallegra, gode, ¢ brilla,
'come fa |' orlo stando a pié del pero,vagheggiando le pere. a
PERMA impalata quivi come un cero, Per esprimere la fupidità nella —
'trova Celidora nel vedere quei Soldati, 1' Autore dopo haver detto che, ae
bocca aperta come frat orfo a pic del pero, foggiunge che ella fava impalata, come
'cero, cioè ritta rita, ¢ nel polto, come favano quelle torrette, fate di
'carta, 0 di-panno, 0 di tavole, che la mattina di S. Gio; mettevano | ri
antichi attorno alla piaza del Tempio di S. Gio: Batifta, entro alle quali stava

- mn' huomo., che le-moycva, ¢ queste le domandayano corr lengegmnaprenie Bore

tt
oe

 

 

Be is
PRIMO CANTARE:

22

Dati 'nei suoi discorsi Storici lib. 6. ia fine. Hoggi in vece di tali torretee»portaao
in due, dello Spedale de) Bigallo; sopr' alle spalle proceffionalmente,uno sgabel-
i Jone, sopr' al quale ¢ fermato un gran cero fatto di legno, per sfuggire il peri-
' celo di romperlo fendo di cera, ¢ faranno 26. 0 vero trenta ceri, che mandas
detto Spedale per tributo al detto Tempio di S. Gio; Batifla. Si pud anche de-
durre questa similitudine da quei peveri Criftiani, i quali da i Turchi sono impa-
Tati, che verifimilmente stanno intivizzatiy ¢-come |'.Autore vuol.che s*intenda,

s che Aefle Gelidora:, me hf
} SVIENE,, ¢ spira, Svenire vuol dit Perdere ifentimenti, ¢ Spirare'vuol dire»
Efalar l'anima, ficché si poslono dir quai finonimi, ma in quelto luogo:il verbo
ii significa Volare, che vuol dir Guardare con desiderio di conseguire,come
uno che havendo grandissima fame., stia a vedere un che mangi, ed -habbia.

d'-avanti mole vivande; Vedi forto C. 14, stan. 34,

~A8BO°C ARST, Trovarsi, o abbatterfi in uno per parlargli. Zo non fon Lew'
snformaro di questo negoxio, ma m? abbocchero col tale, che m' informera.

E' per le fravce. EB' rovinato. -E' per la mala. Quello che i latini differo D:
eocattumeft. Pratta. 8' intendeBorroncello., 0 Macchia, che fuol render' aspro
un pacfe, ¢ vien dal Greco Brattin che fuona Far fiepe-.

BEN ben lo-fquadra, Lo guarda benitlimo., che la forza della-replica & di far
nascere il superlauvo, come accennammo sopra in questo C. stan. 11. Ed il ver-
| ho /quadrare, che vuol dir Mifurar con la squadra, significa Considerare, >

Guardare un' oggetto minytamente, ¢.con diligenza.

 CASC ARE ia pie come'i gatei, Orcener da un male,o da un-cattive acciden-
~ te,un-bene impeniato; che i latini dullero excidere extra mala,

j STANZA XXXII STANZA XXXIV.
} Maravigliato aliorail Sir a' Kgnano-, S? ell! ¢ ( dic' ei.) così noi fiam cuginiy
< E chi fer (disse) rusche fai il mio nome? E subito si fan centro accoglienze y
Mo ti conosco già ds lunga mano, Ed ella a hii ne vende mili' inchini-,
(Ellarispose) ¢ acto tu sappia il come, Egli altrectante a ei fa riverenze,
Celidora fon' 10 del Re Fioriano Così fanno talor due fantoccini
. Fratello d' Amadigi di Beipome, Al fuon di cornamusa per Firenze,
E con tutto, che gid fien' anni Domini Che luna incontroaltaltroandar fivede
Ch! io non ti viddi, fo come ti nomini, Mofo da un filche tié,chi fuonayal piede

Baldone,'¢ Celidora si riconoscono per cugini, ¢ G fanno molte accoglienze.,

CONOSCER ds-lunga mano. Conoicer di gran tempo. Langa mano: ad anni
tanto fuona quanto Lunga ferie d' anni, o gran quantità d'anai., che diciamo
anche 2' um gran pexxo ch' io ti conosco.,:

BALDONE, Celidora, eaAmadigi sono nomi a-cafo:, ma l'Lnfame Floriano &
anagrammatico, da Raffacllo Fantoni.

SON' anni Bomini. Son' anni infiniti. Sono tanti anni, quanti feno dalla na-
 scita di Nostro Signore-che diciamo Aano Domini.' iperbole usatisima in Fix
i enz

 
 
  
  
  
    

 

ce
 ACCOGLIENZeA1. Ricevimento con amorevolezza,,¢ cortefia., ¢ con unas
'eerta dimoftrazione d' affetto, che s' usa ver(o le persone grate. View dal Lati-
Colere.y che esprime Amar-con riverenza., ed honore~ Bis

 

AM

  

se

 
 

30 MALMANTILE

INC HINO. © lo stesso:che'riverenza facendofi con abbaffar la tefta', ¢ piega-
re le ginocchia, ed ¢ propriodelle Donne; | Riverenza si fa: con abbaflar la'te+
sta, ¢ piegandofi un (ol.ginocchio si manda s"altra gamba addietro a foggia di
genufictiione, ed & propria-degli huomini, come si vede nel presente luogo, che
dice,

Ella aluine rende mille inchini;

' li altrettante alei fa riverenze,

COSÌ fanno talor due fancoccini, Suol' andar per Firenze un -contadind, suo+
nando una cornamusa, ¢ porta alcune figurine di legno', che-hanno le congiun-
ture delle membra malticttate, ¢ cuntrappefate con piombo in modo, chef
muovono per ogni verso; queste infilza per lo petto in-una fortilitima corda da'
chitarra, 0 diciamo miaugia,la quale da una parte lega ad uno de" uoi ginocchi 5.
¢ dal' altra ad una tavoletta potta in terra a tal fine,e col muovere quella gamba,»
alla quale € legata la corda;ta, che quelle due figurine infilzutevi bailano ab tempo
del fuono della cornamusa. Lovefa dunque questa operazione, che: fanno*i* due: fis:
gurini, s' intende ancora come facetlero fra di loro questi due'pareati.

CORNAMVS.A, Zampogaa doppia, compolta d' un badopecpetuojedi uns>
soprano, che canta le note come gh altri Zufoli, ¢ fi'da it fiato ad ambeduercoa!
un facco di quoio., da colui che-luona', ripieno di vento: coh ofhare:imun picco-
Jo cannello animeliato; ed il suonatore premendo col braccio il detto-facco:da il"
fiato.a dette due Zampogne.

STANZA XXXV,

Poi che le fratellanze, ei complimenti Har mitre,ch ella fenfia aduepalmenti
Furon finiti, a lei fece Baldone Pigtiando un pan di fedici a'boccone,-
Quivi portar un po di feiacquadenti, St muove il campo,e [ore'alla uta infe
O volete chiamarla colezione, Ciascun paffa per ordine araffegna,
Dopo finite le cirimonie Baldone fa portar da'bere', € amuglars > © mentre

che Celidora mangia, si fa la moftra de' Soldati,..
EAR le frateilanze, EB tratto dal' uso chet nelle: nostre Compagnie, 6 Con-
fraternite di fecolari, neile quali a i tempi determinati si vanno*tucti-ad abbrac-
ciace Juno con l'altro;.¢ questa azione dicono Far Ie: frarelianze, E-da>quetto
dunque intendi dopo finiti gli abbracciamenti ¢ le cirimonie. Sis
SCLAC QV ADENT S, Quel che significhi lo dichiarail Poeta medesimo-dicen-
do; O volete chiamaria colazione,. Che vuol dire parcamente cibarlifuor del defi-
pare', e-della cena 5 ¢ viene-dal Latino collettio prandij vel cons. Ma titcome fon
diversi li pasti che si fanno in Firenze, così fon diversi li nomi che lorodanno. Lb

* primo mangiare che si fa fra l'alba, c il mezzo giorno si chiama' Asciolvere, ed

aile volte colanione. Quelio, che tifa amezzo giorno fixchiama de/fnare... Quel-
lo che si fa tra 'l mezzo giorno, ¢€ la sera fidice Aferends quali meridieredemdes >
Quello deiia ferarsi dice cena, ed-altora'che per il digiuao la (cra: si: mangia: poco
fr dice colazione; Blawoce:/ciacquademévual dire Quandolti:
qualche poco, per bere con gusto.

ee

SCVFELARE, Mangiar comingordigia, o'divorare. E voce Fioreaeiaa yma

hog 3i usata folo. per scherzo.5¢ vien foric daSemjina che ¢ unairaspaovlima da
Jezn9 desta così, perch? adoprdadoja leva malo wgag pee obs mmgpeo pense
eddut

Cablaaid anche? i ordiae,

 

S

*
ae

fics

 

 
PRIMO CANTARE,

 

ee

 

“sche vuol dir mezzo cieco, come vedemmo sopra in questo Cant, stanza 9,; Io fa

 

   

3t

A due palmenti, Da ambeduele ganasce: Traslato dal. Molino, che si dice»
Muavinare a due palmenti quando Due uote lavorano; che palmento vuol dire tut-
ta la macchina, che fa macinare, dicendofi Molino @' «n paimento, 0 di due pal-

ementi,quando Va molino ha una,.0.due macini «| E fimo che si dica Palmento,
quafi Palamento, perché le ruoce, che fanno.andar, la maciac fon compote di
tavole a foggia di;pale per prender |',acqua, che le fa girare.:

VN pan di fedici, ec, Con questa iperbole esprime l'ingordigia di Celidora;
perché peraltro un pane di fedici de' nostri quatteini malamente.si pud confum.~
re anche con fedici bocconi, intendendo Soccone quella quantità, che |' huomo
pud pigliar dentro alla bocca in-una volta,

P-ASScAR a raffegna, Quando i Soldati si portano avanti, al loro Gapitano,

Ȣ fanno scrivere il lor nome fidice Paffar a raffegna. Equi Baldone come fupre-
mo Capitau.o per fare honore alla cugina, Ha la raflegna, nominando, pero s0-
Jamente gli Vfiziali prificipali; il che pare che pil propriamente fidica Dare, 0
far la mofira, Vedi soro C, 2. stan. 36.

STANZA XXXVL

E per il primo vienfene in campagna E 1a sua [chiera aumerosa, e-magua,

+ PPappolone il Marchefe di Gubbiano, E perch' egti ¢ Soldato vererano
Goluiebe nel conjlitto della Adagna Ha nell' infegna una tagliente [pada,
Eftinfe il Gallo, ¢ sepeel ti Germano; Ch'ein pegno all ofteria di mewza strads
L' Aucore in qucita sua Opera mette una mano d'amici suoi sotto nomi ana-

j Btammatici, 1a waggior parte de' quali ¢ nominata in questa moftra, che Baldo-
ehofa dell efercito 5 descrivendone alcuni con qualche loro azione, &.con un' epi-
logo delia loro vita oltre all' Anagramma. i primo che viene inanoftra ¢ Pappo-
clone, ci0ePaoto- Pepi anagramma proprio, perch questo gentilhuomo: era gio-
-wanouo.grande.di persona, ¢ gratio, ¢ mangiava afiai; e fo il Poeta lo
sidice. me» che yuohdir gran mangiatore Vedi forto C. 6. stan. 70. » ¢ lo fa
eddarchefe di Gubbiano, che ¢ un Cattelio; ¢ /ngubbiare ( detto però piebeo.) figoi-
fica Bupier il veotre Dice nel conjlirto della Afagna, cioè Nel mangiare, se ben
par che voglia dire in una fanguinofa battaglia seguita in Alemagna. Eftinfe il
Galo, efeppeili il Germano; pac che dica. ammazo Francefi, ¢ Tedeschi, ma vuol
pdire ch' ci mangid-gaili,¢ germani;¢ gli, fa fare per infegna una-spada impe-
\gnata all ofte dijmezza strada, che ¢ un' ofteria fuor di Firenze-un miglio,e così
~mottra, che ogai fine di quelto tale era ij mangiare,
k STANZA XXXVIL
Bieco de Crepi Duca d' Orbatello Son L-armi loro, it-boffolo ye il randelio,
Menailfuoterzocthaitveder nel ratto, Non tiran paga, reegonfi d'accatro,
Cut perch ei da an acchio fa asportello, Sofiano, fon di calca, erborfainolr,
Soldati ha picfoe' hanno chinfo fatto, E nimici-mortal.de' muricciuoli,
Segue dopo Pappolone Bieco de Crepi, cio Piero de Becci-huomo di faccianon
bella, con occhi biechi,, ¢ lusco, ¢ pero ii Poeca con iequivoco d' orbo,

«Duca a' Orbatello, ¢ dice, che vedendo egli alquanto y ha prefo per Soldati gente,
“che è affatto cieca, avverando il detto.: Bears Advnocali in terra cecorum. Hanno
| queiti soldati Ul bofiolo, ¢ il baftone, non tirano.paga 5 ma vivono di: verte.

o fon

ie

   

      
  
 

2 MALMANTILE™

fen tutti spie, ladri, monelii, e nimici de' muricciuoli.

Vn terzo. Numero di soldati comandati da pil capitani, e dal Colonnello;che
i Latini dicevano /egionem, ed il Colonnello forse era Tribunus,

MEN ARE, Condurre, Ma qui fla proprio il verbo Menare fecondo il pro-
verbio che dice: Solo tciechi si menano, —~;

Ha il veder nel tateo, U ciechi non hanno altra vifta, che il tatto, el odorato

nelle cose corporee, ¢ materiali; ¢ 1' udito nell' incorporee.
. ST Aa spertello. Intende mezzo cieco. Metafora tolta da quelle botteghe; le
gualt quando non è fefta intera, ¢ comandata stanno mezze aperte, che si dices
Star' a (ported, perché aprono folo quella parte del legname, che si chiama se
tello; ¢ seguita la metafora dicendo: Su/dati ha prefo channo.chiufo affarto: cioè s0-
no affatto ciechi. Varchi stor. Hior. lib. 11. dice: Won si tennero le borteghe Aperte,
ne a sportello, ma chinfe affatto, 4 j:

ZOSSOLO, E' quel valoa foggia di calice, col quale si raccolgono i voti ne-
gli Squittini. Vedi sotto Cant. 6.stan, 109., ¢ per la similivudine intendiamo quel
valo di latta, di rame, d' ottone, 0 d' aitra materia, che ¢ usato da i ciechi per
ricevervi l'clemofine, ay

RANDELLO. Intende Quel baflone, che adoprano i ciechi per farfila stra-
da. Se ben randello s'mmtende un Pezzo ci baftone grosso quanto quello de'ciechi,
ma aflai pil corto, che s' adopra per firingere le legature delie baile, che però
tale operazione si dice edrrandeliare.

REGGONS! d' accatto, 1 verbo Reggerfi in questo laogo, ed in questi termini
vuol dir Cavar il guadagno per mantenerfi: M tale si regge cal far' il farto, Cive
vive col guadagno, che cava dal far' il farto, ec. 4

SOFFLARE, }n lingua furbesca vuol dir Far la spia, se bene & inteso comune-
mente. Ed il Poeta parlando di cicchi, i quali hanno per coftume di parlar fur-
beko, & serve di questa, ed altre lor parole, come E//er di calea', che vuol dir
Huomo da far qualfivoglia furfanteria, ¢ viene dalla voce Calcagno, che in lin-
gua furbesca vuol dir Moneillo, cioè /adro di calca nella quale entrano per rubar

'ic borfe, ¢ di qui si dicono Borfatolt, ¢ Faglia borfe. Vedi forto C. 6. stan. 64.

NIMICT de' muriceinoli, Chiamiamo muricciuoli quel pezzo di muro, che avan-
za sopr'a terra attorno alle cafe; d' altezza d'un braccio pil', o meno,e di
simile largheaza; fatto, o per wlo di federe, © per difefa de i fondameati. Di
guefti sono nimici i ciechi, perché speflo vi Pp jotono dentro co"! piedi, ingan-
nati dal sentir al vifo, ed alle mani l'aria libera, il che fa lor credere, che non
potia eflerui impedimento veruno anche in terra. nth

s N ZA xXXAVII.

      

 

    

La frradai pitt si fanno col baftone 5 Chi fuonatt ribecchin,chi il colafetone;
Altri la guida segue d'un suo cane, Cosh tntti fi'van buscando il pane. —
Chi canta a più d'un' uscio wt Oraxione, Han per infegna il djaval de'T arocchi,
E fa [corci di bocca, ¢ voci frrane; Che vuol tentar un fornopien dé.
Deferive il modo del marciare di questi ciechi, ¢ fa lor fare que!.

razioni, che fon (oliti fare andando a cercare elemofine, Dice che 'anno

la Prada col baftone; altri si fanno.guidare a un cane, ed altri vanno cantando Orarioni
a pic d' un' xscio; EB quetti fon ciechi stipendiaci daile pexlone pic, aseiocche ogni
F 5s giorno,

iz

 

 
 

S es — Se eee Ls =
F ws ae

SS

-Pighiar® & tiechi fuor ¢ all offeria

 

 

PRIMO CANTARE: 33

Le, 0 ogni fettimana vadano alle cafe delle medesime persone a cantare un'
razione avanti al loro ulcio, dove per efier sentiti fanno voci frane, cioè Gri-
dano forte, ¢ fanno bratri feorci di bocca; E questo avvien loro perché, per lo pill,
li ciechi oltre alla loro cecita, fogliono havere altri stroppi nella faccia. Molti
fuonano il ribechino, cio' il violino, altri il Co/a/cione: questo strumento (che da
i pi ¢ detto corrottamente Gavascione ) E' un corpo, come quello della tiorba,
con manico lungo, con due fole corde, il quale si fuona con un pezzo di fuolo
da scarpa, che volgarmente si dice taccone; & percio tale strumento è detto an-
che Tiorba a Taccone da Filippo Scrutendio da Scafato, il quale così incitola 11
suo graziofo Canzoniero Napoletano. Alcuni furbi per co/a/cione intendono las
forca, perché ancora a questo s'adoprano due corde, la grofla, ¢ la [ottile, co-
me alla forca. Questi ciechi fuonatori foglion fempre andar vendendo qualche>
Orazione, o Rapprefenitazione, o altre Leggende, e così tutti si vanno buscando
il pane, cioé guadagnano da vivere. E volendo il Poeta moftrare quanto la gen-
te di questo terzo sia affamata, le da per infegna wx diauolo, che tenta un forno
pieno di gnocchi; ¢ moftra che sia fempre intenta a procacciarsi ib vitto con ogni
forta d' invenzione, che il verbo rentare significa Procurare, © Provarsi di far
una tal cosa.,e fideduce, che questo diavolo senrafe, cioè si provafle a rubar da
a4 forno il pane, che vi era dentro. E per gnoceointende Ogni forte di pane;
bene gnocco & quella specie di pane, che dicemmo sopra in questo C. stan. 3.
SCORCL di bocta, ¢ voci frrane. Voci strane', ¢ bocche diverse dal naturale;
perch se bene la voce//corcio & termine di prospettiva, che moftra la figura esser
re(a capace della terza dimenfione del cor po; s'intende anche per positura di cor-
© parte d*etfo diversa dal naturale, aI:
T AROCC AF, Carte,con le quali si giuoca alle Minchiate.Vedi orto C.8. stan.
« 61. in.una delle quali carte al num. 14. ¢ efigiato un Diavolo; e guefto dice, che
senta il forno pien di gnocchi Il nostro Poeta haveva dato a questi Ciechi It imprefa
del Buio, come si vede in alcuni suoi sbozi, che diceva. '
Hanno un' imprefa, dove Bieco metre
Li buio che a fuegiiar vale Cinette,

STANZA XXXIX, STANZA XXX¥X.

» Dietro al Duca,c'ognunguarda atraver[o Signora', rispos' egli, benché ciecay >
Vanno cantando 0 aria di Scappino, Fu pero jonpre simil gente sgherra;

Ma non ginnfero al fin delterzoverso, Con quel batocchio zomba a mofeacieca

. Che venuto alla donna il moscherino, Senza riguardo,come dar' in terra;
Fattoa Bieco un rabbuffo a modo,e verso, Sort'ogns colpo intrepida s' arreca y
Gli disse: 5? ia v' alloggio dimmi Nino, Che non vede i perigl della guerra:
«Perch io non veddi mai in vita mia E' cieca ¢ ver,ma pur il pan pepato

< E! pitt forte yfe d' occhi egii è prsvato,
STANZA XXX41.. eoike

 

 

 Ovvia (diss ella) tocca innanzi il cocehio, Va dunque oforteye inuitto bercilocchio,

 £se cofforoa guerreggsar fon' atti Che i nimici da te [aran disfatti y
Tienteli puree' non mi par! a crocehias Perch' in veder la tua bella figura

 Mdentre gli tempo qui di far di fart, Cascan morti,fenraitro, at para |
Queiti cicchi and dictro a Bieco doiaria di Scappino,( che ¢ uaa
alse E can

 

oi

 

 
 

 

34 MAL MAN TELE €
canzonetta, la quale cantavano i ciechi in Piazza del G, Duca, quando l'Auto.
re principid la presente opera ) ma Celidora adirata di cid, dice a Bicco yche
non vuol tal gerite, ed egli rispole, che-se bene eran cicchi cran però-fieri, che
il non vedere 1 pericoli gli rendeva arditi,¢ forti, come appunto ¢ il pan pepato,
che è pil forte, quando non ha occhi; ond' ella gli dice, che se gli tenga,¢ vada
allegramente, che ella ha speranza dicavar fructo da lui folo senza loro, perché
stima, che il nimico sia per cascar,morto subito, che vedra il suo brutto vifo.

GV-ARDA a traverso. Vino che ha gli occhi scompagnati-, come haveva Bieco
diciamo Guardare 4 tranerfo. Vedi sopra in questo Cant, stan. 9. Tran/uerfa tuen-
tibus hirquis, Virg. Egl. 3.

VENVTO alla donna il moscherino, La donna, cioè Celidora, s'adird, Si dices
Venire il moscherino al nafo, perché si troyano alcune piccole mosche 4 le quali vo-
Jando, taluolta entrano nel nafo altrui, ¢ toccando quella parte così fenfitivas 5
danno grande alterazione, e mettono }' huomo in una subita impazzienzaye stiz-
za. Si dice ancora Venir la fenapa, 0 la Moftarda al nafo, perché nel mangiar la
moftarda (che ¢ un' intingolo facto di fenapa, ¢ mofto corto) quando ¢ ben.cari-
ca di fenapa,viene al nafo un certo pizicorey che forza a, lageimare.. Si dices
anche Venir la muffa, 0 altri puzi odiofi, ¢ sporchi, come fidice fatto GC. 4; flan.
23, E tutti significano Venir collera.

E-ATTO ua rabboffo,, Beavato. Fare un rabbuffo, 0 Rabbuffare vuol dire Ri-
prender uno con minacce, o Spaventarlo-con afj di parole.. Ii Landino nell'
esposizione a Dante C. 7. dell' Inferno alla parola Bufa.,.¢ Rabbuffare dice: Ada
proprio Buffa ¢ vento, onde diciamo Buffettare chi getta vento, per bocca,e Shiffare,quan-
do con snono di parole, 0 a dir meglio Con ventofe, edienfiace parole alcuno minaccia.
Di qui diciamso Rabbuffare,Conturbarese muover le cose dell' ordine loro, e scompigtiarley
e chiamiamo Rabbuffo,quando Con parole conturbiamo, e Scompigliamo la mented' uno y
Vedi forto C, 3. stan. 57, la voce Buff.  i '

A modo,e 4 verso. Con tutta perfezione. B il latino modis, & formis,

DIMMI Nino, Dimmi pazzo,e senza Ceraello, come fu Nino, il quale per lo
grande amore, che portava a Semiramide sua Meretrice 5.0 moglie » le concefic,
che per un giornovella fife affoluta Regina y ed ella in quel giorno:lo fece am-
mannaie » ¢ficonfermd Regina per fempre » come si legge in Plutarco ia Serm,
Amator, j ' oe

' PIGLIAR' i ciechi fuor all' ofteria, Quand' uno vince assai, fogliamo dirgli: Si
torrd + ciechi, © s' intende ail' oferia. E questo perché si fuppone, che quel
tale, che vince per l'abbondanza del aro. venutogli in. mano. fenzas
fatica, sia per spenderlo profulamente in pigliar& tutti.li suoi gustt fino cons
l'andare a cena all'ofteria, ¢chiamare alla faa menfa a fuonare alcuni ciechi 5 i

in fa !-hora del inno girando, perl' ofteric.a tale eficttogexques

“ eee
: hi sono i Ciechi, iiquall Oelidaracdive haver veduto pigliare all' ofterie, —

SGHERRO, Bravo... Ammazzatore; Tagliacantoni.. Vediforo, Canty 3.
Rp. ers auey 2h ee at
BATOCCHIO, Quel baftone; col quale si fanno la strada.iciechi si chiama
Batocchio dal et 7 oe fanno.iciechi, per aeneeneer!

dattere da gli altri cicchis E però vuol dire anche 11 Battaglio delle Campane.

 

| ZOGH-

 

 

 

 
 

 

vo LS ell ee

PRIMO CANTARE: 35

ZOMBA. Perquote, baftona. Vedi forto C. 6. stan. 104.,¢C. 11. stan. 28,
. MOSCA cieca. \\ giaoco detto Mof¢a cieca ¢ trattenimento da Fanciulli, che
deriva dall'antico, ¢ fidiceva Atu/ca aenea, ¢ si faceva nel modo, che usano
hope, che è in questa maniera. i

irano le forts fra più ragazzi a chi debba bendarsi gli occhi, ( che in questo
giuoco dicono Star sotto) ed a quello,a cui tocea,sono bendati gli occhi in modo,
che non possa vedere, ¢ poi con uno sciugatoio, o altro panno avvolto, che ciascu-
no tiene in mano, si danno da gli altri delle percosse a colui, che è sotto, ed egli
¢ost alla cieca va rivoltandofi, ¢.quello-che egli arriva con'la percofla deve ben-
darsi in vece del, percuziente, il quale si leva la benda,e va fra gli altri a percuo-
tere il nuovo bendato; Quello,al quale di-mano:in mano tocca a star forto, me-
na y senza riguardo, colpi spictati, si-perché commoffo da tanti colpi vorreb-
be: vendicarsi, si anche perché, cogliendo, il colpo sia in modo da non poter'
efler negato, procurando ognuno di non toccarne,¢ d' occultarla, se pud,
uando l' ha: toceata, per non haver' a stare in quel martirio,in-che € colui, che
fotta. E però dice Zomba a mosca cieca senza riguardo come dare in terra, Si di-

Ce maxzate da ciechi per inteadere Percofie spierate.

41 Pan pepato è pis forte se a! occhs egli e private, Si fuole in Firenze perla fetta
di,tucti i Santi fare un certo. pane che da noi fidice Pam pepato, il quale ¢ com-
io di fapa, aceto., farina, pepe, ed altri aromati-, ¢ mescolanui pezzetti di
ib di poponi., zucche', cedri,ed'aranci:conditi in gugchero, o miele, li qua-
di pezzetti,quando il pane si tagliasreftano nella tagliatura a similitudine d'occhi,
¢€ percio. da i nostri Fanciulli (on chiamati Occhi; E cavandofi dal pane tali
echt sche sono dolci;il pane refta cnet »cioè pil acido; ed il Poeta si serves
della parola Forre-in significato diGagliardo, dicendo che i ciechi fendo feng'
occhi fon pitt forti', ed intende gagliardi,scherzando con questo equivoco di forte,

T [8c innauzi il cocchio. Seguita il tuo viaggio, ¢ tanto's' intenderebbe a dir

teturainnanei senza-porui |' aggiuata Cocchio, ma il Poeta ve lo pone»
per seguitar l'uso Fiorentino.

ST AR a crocchio. Il verbo Crecchiare., ¢ la frafe fare a crocchio significano Ci-
ealare, o Ciarlare di cosa di poco frutto, o importanza per finire il giorno.
Onde questi tali si dicono Crecchions., Cicaloni 5 Perdigiorni, ¢simili. Vedi for-
to Cant. 3. stan. 5. Questo verbo Crecchiare serve anche per intendere Dar delle
buffe. Vedi sopra in questo Cant. stan. 19. Se

AZERCILOCC HIO. Epiteto compolto-dal Poeta', che vuol dir Bircio di'che

sopra in questo Cant, fan. 9.
STANZA XXXXII “ “STANZA XXXXUL

We Segue intanto Romolo Carmari ns * sen! infegua nera che v' ¢ drenta y
Cavalier di valore, ¢ di gran fama; » Cupido morto con & snor piagnoni
M14 sfortunato, perch coi danari - - Aarciar si-vede un groffo Reggimenta,
Cee ere nee 0 Chi egl bad innumerabili tritoni 5
'on le prllole berarj All cui arrive ugnun per lo spavento
L affetto evacns orc nike Si rincantuccia, od Smpieftealteny

Tal che fenz' un quattrino dmartellate ° Eda lontana infin
Alla guerra ne va per disperato, 9 iano s
a v \vo. we E

 

eee

 

  
 

36 MAEMANTILE

Segue Romolo Carmari, Quéfto fa un Fiorentino,del quale non flimd bere scio-
glicr l' anagrammma, edirne il nome. Questo Gentilhuomo havendo durato un
gran tempo a godere una sua Meretrice, ¢ (pefovi molto danaro, 0 gli fu tolta,
o ella non lo volle pik perché egli abbandond lo fspendere; come ¢ proprio di
simili donne; ¢ cid ¢sprime il Poeta ia quei due veri.

Con le pillole date a fusi erar},
L affeteo evacuo l' Arpia ch' egli ama,

I quali versi suonano: L' havergli fatta votar la borsa fece disperdere l'amo-
re, che ella fingeva di poreargli, Onde egli disperato, se ne va alla guerra;¢
moftra questo suo spento amore nell' infegna, che egli porta, in cui ¢ dipinto
Cupido morto, che ha d' attorno i suoi piagnoni. E perché questo Signore era
nel veltire positivo, ¢ senza boria alcuna, anzi pil tofto abbietto, il Poeta fa,
che egli conduca un reggimento di gente mal veftita, ¢ questi huomini chiama
Tritoni, pexcht Huomo trito, 0 Tritone tanto vale appretlo di noi quanto dire
Huomo mal veftito; E questa gente per esser così mal veltita ¢ stimata una schie-
ra di Monelii, ¢ di Ladri, e percid ¢ causa, che s' acerescano i serrami alle bot
teghe, ¢ che ognuno fugga per la paura, che ha di loro.;

DAMA, Vuol dic Donna nobile,venendo dal Greco Damar,fecondo alcuni;
¢ fuona Signora dal Francefe Dame, Madame, cioé Signora, mia Signora; ma
i piglia anche per |" amata, come ¢ prefo nel presente luogo y i

CON le pillole date a suoi erar), Con l'evacuatorio dato alla sua borfa,cioé con
avergli fatti finire i danari mand via dal fo corpo la bile amorofa, cio' lascid
d-amarlo. ?

L' Arpia « Intende Meretrice y ed esprime una donna rapacé, come sono le»
'Meretrici ( che Arpia in Greco fuona come Rapace ) © quali sono figu-
rate Arpie, che i Pocti fingono esser tre, Acllo, Ocipete, ¢ Celeno; ele
fanno figlie di Nettunno, ¢ della Terra; altri figlie di Thaumante, ed
Elettra, altri d'-altre Deita; basta che se ne servivano per e(primer l'avari~
zia. Vergil. 3. 4En.

Tristins haud illis monffrim, nec fevior ula
Peftis& ira Deum fiygijs fefe extulit-undis,
. Virginei volucrum vultus, feedissima ventris
Proluvies, unceque. manus, & pallida femper
Ora fame. ele
+E Dante nell' Inf. Cant, 13. seguitando Vergilio dice
Quivi le bruste Arpie lor nido fanno,
Che caccinr dalle Strofade i Troiani | i
“Con trifto annunxio di futuro danno-. ' i
Spalle hanno alate', colli,¢ vsfi humaniz.,
¢ Più-con artigli,¢ pennuto il gran ventres ww? '
'Fanno lamenti fu glivalberi, flraniy 0 patect, aed.
nome d'Arpia dette a una Meretricé ancheil Coppetta nel fio Capito-

Questo nome
lovin biafimo della Signora Ortenzia Greca dicendo
crudeli, infide, inique, ¢ ladte “
s venire a faffidio a mille Rome

Fes teva efease ofr madee; Sane

 

 
 

   

PRIMOPCANTAR 2.

AMMARTELLATO, Haver:martello, o:¢Ser' ammnarczilav var!
Quand”uno innamorato ha gelofia della cosa amata, ovvero ha qualens 12
con la medesima. 1] Firenzuola nel fa Capitolouin lode del legna fantoy chia
paazial' esser' ammartellaco.dicendo:

tor nnovamente vi dice che cava»
Di faftidioun', che crepi di-martello,
Guarda fe'quefia è un! opera brava.

E si parri voleffon provar quello
E conosceffon la lor matattia,
Tutti rirornerebbono in cernella';
C! altro non ¢ il martele una Paget.

PER disperata, La disperazione tuna foverchia inquietudine, cagionata das
” grave dilgufto, la quale-ci leva affatto'il dominio di noi medesimi.

PLAGNONI. Trova (peffo nelle storie Fiorentine questo nome Piagnoni, che
vuol dir Coloro che seguitavano la parte di F, Girolamo Savonarolajma qui vuo!
dir Quegli huomini, che si mettono ai mortori de i gran personaggi atcorno al
cadavero, tutti coperti di nero-, ¢ con lunghi-veli, ed in mano hanno uno sten-
dardo, o penaoncello di taffetta nero: E si dicono Piagnoni, dal piagnere che
dourebbon fare per la: morte di quel tale.

MARCTARE, Bib muoverG degli eferciti. Voce reftata a noi-dal Prancefe;
'eda molti si dice Marchiare,perché quefii ali, vedzadoia scritta con l'aspira-
'zione y la pronuaziano all' Italiana,non si curando di riflectere che il C-H fuona
fer, enon chi.

*REGGIMENT O), Quantita di Soldati comaniata da pitt Capitani, ¢ dal Co-
donncilo; ¢'forfe lo ttelo, che Terzo detto sopra in quetto C. Man. 37.

TRITON » Sono Dei, 0 Moftri Marini, iquali si dipingono ignudi, o al
pil coperti d' aliga, ¢ di qui gli huomini mal-veltiti si chiamano-da noi Tritoni.,

G huomini triti, che fuona Huomini vili, ed abbiecti. Vedi sotto in questo
'ant. stan. 86."
tee spinach + Nasconderfi:, o-metterfi per'i canti per non esser
veduto.

EAM PIESTi calzoni', Per la' pauta, (c li move iil corpo, e gli empie le brache.-
Quetto detto esprime, che Quei Tritoni facevano gran paura a-chi gli vedeva,non
'che veramente 'se gli empicilero i calzoni.;

 SAD DOP PIANO i ferrami alle borteghe Per afficurarsi da coftoro, che sono tima-
'ti tanti ladri, im gran tratto di pacfe rinforzano le ferrature alie botteghe. B qui
l'Autore dice tutto quello', 'che'egli pud, per moftrar coftoro affatco birbom,-¢

'vera canaglia-, 5
ee STANZA XXXXIV.-
etek Manmade © «\” 'Serive fonetri, canta ognor di Sillt.,
'nella guerrae fogeetto., © °E' bnon compagno, pi. i il vin pretto,
Che miciterebbe gl li Achilii; ~ 'Rabaro, - Seu nel eae
*:E quanti fon di loro in un caléetto: * Uiquattrodelle coppe c' bail monnino.
nella moftra Doriane da Grilli che t Lionardo Giraldi. Questo gentilhuomo
fu bellissimo humore 5 molto dedito alla poesia burlesca, huom difeorricors,
ed huomo di conversazione; ¢ perché egli haveva per coftume il dar de Monnini,
il Poeta gli fa fare per imprefa Vana carta da giuocare s nella quale in mezzo ay
un guattro di coppe è figurato.an. Monnino,

CHETTERE uno itz un calcetto. Confondere uno, Superar' uno nel fapere 5.0
nel valore, ¢ ridurlo tanto avvilito, che si vorrebbe nasconder dentro.a un cal-
cetto, viliflima, ¢ piccola parte dell' abito dell' huomo, come — che non,
cuopre se non il piede, Questo Doriano veramente non fu mai soldato, se ben
lV Autore dice, che egli & buon foggetto nella guerra; ma dice così di lui, perché ef.
fendo egli di sua conversazione,lo sentiva speflo.dilcorrer delle guerre con gran 4
fondamento moftrandofene aflai pratico,

VIN pretto. Vino puro, ¢ senza commiftione d'.acqua, 0 d' altro; ¢ fenten-
dofi in pili luoghi-del nostro Contado chiamarlo vino puretto, non fon lontano: da
credere, che la voce presto sia o figurata, 0 corrotta da puretto.

CASINO. lotendi quella Cala nella quale la-nobil gioventi Fiorentina s*adu-
fa per giuocare,

MONNLNO. Le carte de' Ganellini, 0 Minchiate:hanno in (¢ efigiate quate
tro cole diverle, che una parte hanno spade, unaparte baftoni, una parte das
nari., ed una parte coppe, ¢ tutte quattro queste noma di carve comingiano da
uno fino a 14. Nella carta del quattro di.coppe in mezzoée una bertuc-
cia a federeyla qual bertuccia da noié detta Adonming,.E.questa dice il Poetayche
è l infegna di tei i perché egli & (olito didare i Adonnini, che vuol dire»
Quand' uno parlando con un' altro,questo lo forza a) dir qualche parola, che rimi
con un' altra, che a quel tale dispiaccia; per efempio Doriano disse ad ua.Che-
rico: Won fu mai gelatina senza,-, «. B quisfiférmo fingendo non firricordare
della parola che finiva il verso; ed il Cherico, il quale ben fapeva. la fenrenza
glicla fuggeri dicendo: fenz' allore, ¢ Dorian foggiunle: Voi free il maggior bwe
che vada in core, E, quelto si dice dare i Monnini.

 

STANZA XXXXV, ' STANZA XXXXVI
Fra Ciro Serbatondi il Sir di Gello Di foglio per impre/a un bel Cartone
Che in, Pindo a Adana Clio foftiene ikbraccia,  Iafiame con lapafia egtibanno meffo 5
Egeno de Brodetti, e Sardonello, Dei lor Fantacci, i quali da Perlone
Vafari,.ch' ¢ padron di Butinaccio, |. Saglion copiare, 0 difegnar dal gefso 5
Conducon tanta gente ch' ¢.up flagella Ne 0 v'han dipinta d'inneneione
Da far che le pagnoste habbianospaccia, L loi inaaieqeas hanno e/prego

Di cus (perch il mefbar dilettaaognuno) Su le tre hore il venicel revaio
PENN me ane oRWreen jem neh nies
ano tre ati \ puno ¢ Fra Ciro Serbarondi

“the vuol dirt Crifefane.Berandl quale fa Sir di Galle, perchs ha fonte woa fone ]
villa così deta. Dice che fofieneril braccio,a Adana Clio, perché egli è huomo
SecaelaFlrss che ol dt defend Pat henate Sit at dosoaoaes

'ardone 'ari -vuol dire aloré, il, fa. x 3
perché ancor' egit ha una Villa così detta, Conducono. i molsa. 4

wale comandano vicendevolmente a un ne (per-uno; ¢ perché fix

ong stati tutti tee scolari dell' Autore,fa lor fare una bandiera de)
difegni,ch¢ hanno fatto in fquola sua; Ma perché gueiti

'Si pighano il comanda a un di per uno

   

 

    
> agoftos e per it

 

si alcuni di.
“Perr

gli huomini nel

 

 

> 7 >; + -

PRIMO CANTARE. 39

cheialla-pittura, però non fecéro altro acquitto in essa, che quanto baftava per
una certa infarinaturar, ¢ per faperne discorrere; egli volendo moftrare queito
lor poco profitto, fa che di lor propria invenzione ritraggano nella decta lor
bandiera una cosa invifibile, come appunto è il Vento. q

E 4n flagello. Questo termine significa Infinita, ed Abbondanza grandissima,
'ed esprime un numero indeterminato. Vien, forse dai Latino, che tal volta
significa Quantita immenfa. Martial, lib. 2. 30. Eromius laxas arca flagellat opes 5
parlando d' uno che havea gran quantica di danari,

CHE le pagnotre habbiano spaccia, Che s efitiy che si confumi molto pane. E pa-

<gvorta se bene non ¢ voce Fiorentina;¢ nondimeno spelio usata.

MEST ARE, Qui val Miniftrare, Comandare,

CARTONE, 1 pittorichiamano Cartone Quella carta grande fatta di più
fogli., sopr' alla quale fanno il modello di quaiche grand' opera, che devono di-
pignerencl muro a fresco, o a tempera 5.0 vero per teflere arazzi.

FANTOCCH, Figure mal fate. Pitter da:Fantocci s intende Pittore da pocds
appunto.come da quelta loro imprefa vuol l' Autore., che si argomenti che futle-
ro.questi Signori.

DAL gefs, Ciok dalle figure fatte di geflo. I pittori hanno per coftume di
chiamare dette figure di rilevo,( delle quali si servono per difegnare ) col folo
nome dige/> (enza dir figure, 0 itatue, come si vede nel presente luogo, che»
dice difegnar dal geffo,

LANTERNONE. Arnele noto, che serve a portarui dentroil lume, ¢ di-
feaderjo dalivento.).))
jy, BRVCIAT 410... Colui che vende marroni arroftiti alla fiamma, 0 nel for-
no, che noi chiamiamo Sruciate, donde Bruciataio,

n STAN ZA XXXXVIL

tho: >

Nanni Ruffa del Braccio, ed Alricardo Hanno acomune un lor vecchio fredarde
Conduce quer di Broxziy edi Quaracchi Da farne a corui tanti spauracchi,
e 'he bevon.guel lor vin gagliardo, E dentro per imprefa v' hanno posto
Le strade allagan tutte co i fornacchi, Gli [piragli deldi di Ferragofto.

Scguitano due altri Gentilhuomini Wanni Ruffa del Braccio, che vuol dire Ale/=

ro Brunaccini od.Alticarde che vuol dice Carlo Dati; a quali fa condurre le»

gentidi Brozzi., ¢ di Quaracchi.y,due loghi vicini a Firenze, ne i quali nasces

 vino deboliissimo 5 ¢ pero dice che questi soldati fon mal fani;¢ pieni di catarroy

perché bevono.quei vini deboli, ( che egli ironicamente parlando, chiama ga-

en cree danao prima alle gambe, che alla tefla.

E pecché tali infermi pare che si rihabbiano, ¢ piglino qualche vigore, quando

.trovano all' allegrie; percié fa loro portare una infegna nella quale sono espret-
b i, gozzoviglicy ed allegrie, che già si facevano il di di
'8 intende il-di primo d' Agofto, venendo questa voce da Feriaré
igenza di questo¢ da fapere, che anticamente folevanfi cele-
1.con grandi allegrie; ¢ cid si faceva forse, perché eflende
servore della state,erano neceffitati dal gran caldo a sta-
-ge allegramente, perché l'allegria ¢ il ptiino rimedio delia (quola Salernitana:
Hac tria mens bilarisyequics,moderara diera:Eisédo duague molto pericolofo in quet

. tempi

   
   
  

agofto,

-brar le ferie Ai

 

  
~~

 

GQ MALMANTILE

tempi d” infermarsi, e percid molti giorni infaufti allora si notavano dagli Egizj,
cfiendo vicino al Sirio, 0 Canicula da tutti detta peftifera y come ci moftra Sta-
2io lib, 1. Siluar, Zam nec calido latravit Sirius «fro, E' necefiario riposarsi,bere,e
maogiare, ¢ stare allegramente; al che consiglia nelle sue Odi Orazio più voice;
Ed habbiamo una cantilena assai praticata, che dice,

Quando fol oft in Leone,

Bonum vinum cum mellone,

Et agreftum cum pipione. ¥
E perché veramente il servore del So! Leone, o Sirio, ¢ allora nel maggior col-
mo, sono le flagioni molto calde; ¢ peggiori, che in tatto l'anno; onde appref-
fo a' Greci ancora si facevano molte allegric', ¢ facrifizzj.a segno, che appreffo
gli AttnieGi (ecGdo alcuni il mefe d'Agofto acquifto il nome a' Hecatombeon. Tai felte,
ed allegrie si facevano già a Firenze non folo per la detta ragione, ma ancora per
causa di alcyne vittarie ortenute da i Fiorentini in quei primi giorni d'Agotto,e se
ane conserua ancora il coftume, ma'non si fanno tance felte, quante già si faceva-
no, poiché folamente si fa correr al Palio alcuni Afini: Siche s' argumenta, che
il nostro Poeta intenda, che in questa infegna, o flendardo futile rapprefentato il
palio de gli afini, mentre dice /piragli del didi Ferragofto, che vuoldire un poca
di memoria delle gran felte, che già si facevano in quei giorni. © '

SORNACCHIO., Sputo groflo,e catarrolo, detto anche farda, Vedi sopra in
questo C, stan. 25. Monfignor della Casa nel suo Galatco dice; Di /ofiamenti di
nafo sporcamente, di tirar fornacchi, e /putamenti,

SP.AVRACCHIO, Così chiamiamo quei paunacci, che sopra ad un palo,per-
tica, 0 albero st mettono per li campi a hne di spaurire i colombi, ed aleri uecel-
li, Vedi sotto C. 5. stan. 49, poe

SPIRAGLIO, Vucl dit feffara in muro, © in tetto, © impdfte di usci, o di
fineftre, per la quale, trapela Itaria, 0 lo (plendore, che i Latihi dificro rima-,
In guclto luogo però è inteso metaforicamente per Piccola notiziaycome è atiai
in.ulo, ¢ forte non lontano da i Latini, che dillero Spiraculmm tantu! ius rei ad
me venit per intendere lo ho havuta di cid qualche notizia,

STANZA XAXXVIIL

Gustavo Palbi Cavalier di petto Van moltiagrucce,in seggiola,e nel letto,
Con Doge Paol Corbi hor ntincammina Perch non sono ancor neta farina;
Gt' Incurabili tutti, e 11 Laxzeretto; Fat per imprefasn un lenzuol chefuetola
Gente, che uscia di far la quarantina. Fn Pappino rampante a una pentola,

Seguono Gustavo Falbi, cioè Vgo Stufa Senatore Fiorentino, ¢ lo chiama Cava-
fier dé pesto, perché ha la Croce in ray efiendo Bali della Religione di S.Stefa-
no}; Bl'altro è Doge Paol Corbi, che vuol dire Canatier Lacopo del Borgo. A que-
sti due gentilhuomini fa condurre una q 1 s > ¢ di stroppiath,
per moftrare, che essi nel tempo; che |' Autore componeva la presenté Opera

non crano d' intera fanita per quaiche poca d' ipocondria, che gli miole(lava, ¢

fa però lor fare per imprefa un Servo dello spedale di S.Maria Nuova con les
mani aizate a una pentola, Ue
INCVR ABILI, Così si chiama in Firenze uno Spedale., ne} quale vannoa cu-

racfi1 Maitranzefati.
Laz.

 

+e

 
0

PRIMO'CANTARE: “gt
LAAZZERETTO. Luogo; o Spedale in cui fimettono gli huomiai,¢ robe»

- folpette di pefte per far lor fare la quarantina, ¢ renderle praticabili, che Far la

'quarantina vuol dire Star riferrato in uno di questi Juoghi quaranta, o pil, o me-
no giorni per/purgar il sospetto-d' infezione. E questo nome Lazzerctto viene»
da Lazzero rifuscitato da N, Sig, Giesi Crifto, quando era di già fetente il di lui
'corpo.:
GRYCCIA, Specie di baftone per gli stroppiati, sopra una teftata del quales
'efiendo confitto un legnetto fatto a guisa di mezza luna, si foftiene'il corpo met-
tendo detta mezza lina forto il braccio, ¢ l'altra teftata del baftoné in cerra; ¢
-perché questo' baftone è simile a una croce mi par di porer crederé, che 1a voce
'Gruccia sia corrotta dal Latino scipio cruciarus,

ANON fon netta farina, Non sono schietti, non sono affattd fani.

LENZVOL, che fuentola., Coftoyo in vece di bandiera, usano un lenzuolo,¢
cid per moftrare, che tutte le loro cose sono da spedali'; in esso leazuolo è dipin-
to-un' Aftante; 0 Servo dello [pedale di S. Maria Nuova, rampante a una pentole,
cioé con le mani alzate a una pentola, che @ in alto; a similitudine del Lione, il

aié quando si trova dipinco ritto con le bratiche dinanzi alzate a qualche cola,

'dice Rampant. -Pranco Sacchetti Nov. 133, £d hebbers ritrovato per cimicro wx

MeXXo orfo con le campe rilevate,¢ rampanti,
L

STANZA

Bel Mafatto Ammirato anch' egli pas

Lindo garzon d' ogni virtis dorato,
Che pus de' soldi bavendo nella caffa
Pifeiar a lerto, ¢ dire 3 lo fon fudato;
Ma per l'ipocondria, che lo rartaffa,
Ei si da acreder d' essere Ammalato;
Maé mangia,beve,e dorme il/uobifagno,
Chit fine 4 vespio,e poi si leva in fogno,

ST AN ZrASE.

Con lo scenario in mano, e il mondo fusra

Va innanzi 4 nobil suoi commilitoni,
Pancrazio, Pedrolino, ¢ Leonora

Lo seguon con un nugol d' [frien

C hanno una infegna non finita ancora,
Perché Anton Dei co tuttii voi garzoni,
Incambio di sbrigar quella faccenda',
E ito al Ponte a Greve a una merenda,

 

Patla Belmijatto Ammiraro, che € AZattias Bartolommei Marchele giovane di bell'
'aspetto, ricco, € letterato; il quale fu un tempo, che si persuadeva d? haver tucti
imali. E perché'questo Cavaliere si diletta di comporre commedie, ¢ volenticri
récita in esse lui medesimo', ed appunto nel tempo, che I'Autore accrebbe la pre-
fente Opera, havea detro Signore mefla insieme una conversazione di giovani no-
“bili, che recitavano all' improvvifo; però lo fa capo di nobili commedianti, ¢
gi da uno stendardo 'non ancor finito, perché Antonio Dei ricamatore ( ¢ questo

il vero suo nome, cognome, ¢ profeifione ) in cambio di finitgliclo, era anda-
to a un' allegria al Ponte a Greve, luogo poco lontano da Firenze. Cafo seguito
al detto Sig, Marchefe Bartolommei, che aspettando alcuni abiti per una com.
ti 1 garzoni della sua bottega fuori di Firenze. i

4LAVENDO de soldi nella ceffa. Eflendo ricco: Non gli mancando denari

PISCLAR' a letto @ dire: lofon fudato, E' proverbio assai vulgato, che igni-
fica. Pud fare a suo modo, che, o male, o bene che egli faccia, git € fempre
ascritto a bene; E s' intende d'Vno, che sia ricco, e fortuaato.

“media, che si dovea far la sera, il Dei in vece di finirgli fen' era andato con tut-

i LEVARSS iv fogno, Levarsi più prefto dell' ere solita di levarsi, quafi dica...:
ea ee E we

 

“aes

 

 

 
 

+

4

 
 

Qe MALMANTILE

S'é levato di notte,fognado esser'hora di levarsi,e qui 'Autore intende, chea questo.
Cavaliere il mezzo giorno, alla quale hora cominciava a deftarsi,serviva per aurora,
SCENARIO, Eun foglio, sopr' al quale fon descritti i reciranti, le scene della
commedia, la quale si dee recitare,ec. i luoghi,per i quali volta per volta devono
uscire in palco i recitanti, afinché quel tale, che affifte gli pefia fare uscire ag-
iuftatamente, ed a i tempi debiti. 'Lal foglio si domanda anche Atandafuora, se
il Aandafuora & alquanto differente dallo Scenario, perché questos appicca
al muro dictro alle scene affinché ciascuno recitante lo possa da se stesso vedere»,
ed il Afandafuora & tenuto in mano da colui, il quale inuigila,, che l' opera sia,
recitata ordinatamente; ma tuttavia, come ho detto, s' intende, ¢ si piglia spef-
fo l'uno, per l'altro. =

PANCRAZIO, Pedrolino,¢ Leonora. Nomi di recitanti nella faddetta con-
uerfazione. w=

NVGOLO a' Ifriani, Gran quantita di commedianti. Questa voce mugolo, che
nel presente luogo significa numero infinito, si usa più propriamente parlando di
volatili, perché questi volando gran numero insieme, come farebbono storni,
colombi,ec.occupano il fole,ed oscurano l'aria,appunto come fa il magolo.La voce
Usrioni & latina, tolea dall' antico Toscano, come dice Polid. Verg. lik.3-cap.14.
Ie cui parole fon queste. Et quia Hifter Fnfeo verbo Indus vocabatur, ideo nomen hi-
Srrionibus eff indizum, ec. Ma hoggi ce ne serviamo per nome speciale, chiamando
Iftrion: folamente i commedianti, che recitano per prezzo.

GARZONI, Intende javoranti; se ben Garzone vuol dir propriamente Giova-
ne scapolo, ¢ feaza moglie, come si vede nell' ottava antecedente /indo garzone;
'Tuttavia s' intende anche Servitore, o layorante, che stia a falario in botteghe
di qualfivoglia mefliero.

MERENDA., Specie di mangiare, che si fa tra mezzo giorno, ¢ fera. Vedi

sopra in questo C, stan. 35,
STANZA LI STANZA LIIL

Don Panfilo Pilori move il paffo
Che,tra che per usanza mai fra cheto,
or ch' ei fa moto fa si gran fracasso,
Ch? io ne diferadoil Diavol n' uncancto,
Aforda il mondo piis d'agn' altroilgraffa
Papirio Gola, c' appunto gli ¢ dreto,
4i qual vefti di lungo, ¢ fu guerriero,
Perocche poco gli fruttava il Clere

STA i ZA LIL

E n' ha fatto con esso de rammanzi,

C* un pp di campanile pee ae alloga y
E questa ¢ la cagion, che la trai lanzi
Da soldato n' andd in Oga Magoza;
We PY, al men tirato innanzs,
Posi la [pada, ¢ ripiglio la toga,

E per lo regloff rae fee
Tornar' 4 casa a quefie stiacciatine,

 

 

Al che tra molti commodi s' arroge;
Quel ber del vinych't troppo cofaghiorta,
Lua birre, qua Jalcraut, qua cervoge y
41 casa mia dicea,del vin s'imbotta,
Pero finianla; cedant arma toga:
Lonon laveglio,in quanto a me,più cotta;
Guerreggi pur chi vuol, s'amazs ognuno,
Ch'io per me non ho spixza con niffino.

STANZA LIV.

Così rinunzia lt arm a Gisve, ¢ stima
Defer il pik lieto huom che calchi terra,
Pensa frato mutar, cangiando clima, -
Ma trovara ? ltalia tutta in guerra,
E farzato ferrarsi, piit che prima 5

99 Eeco il gindizio human comespe/soerra
so tormar fra gente eer: ye gaie,
E fugge t' atqua jatto le grondate

BS qua satete §' STA

 
 

 

 

PRIMO CANTARE.

è STAN ZA LV.

Tra don Panfilo, ¢ lui uno fqnadrone
Dat Pontadera aspettano,e da V ico,
Che parte per la via vanno a Vignone y
E parte fanne un fonno a pié d'un fico,
Cofhoro empion di rena un lo» foffone,

. E quando sono a fronte all' inimico,
Gliela schizxan nel vifo, ed in quel mitre
Gli piglian gli alers la mifuraalventre,

STANZA LVI

L infegna di cofforoé un Atontambanco,
Cha di già dato alls suoi vafitl prexze,
E detto che fon buoni al mal del fianco,
E firolagato,e chiacchierato un pera 5
Ma trovandofi aifin fudato,¢ scaico
E non havendo ancor toceato xn bexzo,
Siscadolezza,ed entrain grade/mania,
Pot dice, che si parte per Germania,

Segue Don Panfilo Pilots, che & dpoliro Pandoifint gran chiacchierone, ¢ 'Papirio
Gola, che ¢ Paolo Parigi, il quale ne i suoi primi anni vefhi abito da Prete ( che
questo intende col dire Ye/ti ds dango ) ma poi lo posd, ¢ fen'.andd in Alemagna,
alla guerra vedendo, che quell' abito non gli era di frutto; Vifto poi, che anche
que! mefticro non gli fruttava,tornd alla patria, ¢ ripiglid ' abito. Ma trovato,
che ancora l'Italia era fottofopra per causa della guerra del Duca di Parma, fu
forzato dal deb)to di fuddito, ¢ dalla conuenienza della provvifione,a tornares
alla guerra in servizio del Sereniss, Gran Duca, ¢ a lasciar di nuovo l'abito da.
Prete, Finita deta guerra il medesimo Paolo Parigi si rimefie l'abito, ¢ fattofi
Sacerdote, mori pot Rettore delia Chiefa di S. Angelo a Vicchio: Questo Pao-
do Parigi fu figliuolo di Giulio, ¢ fratello d' Alfonfo ambedue Architetti celebri,
come fu ancor' egli, ed Andrea altro suo fratello, che fu Maeftro di campo, ¢
nominato dal nostro Poeta Paride Gurani sotto nel C. 3. stan, 10.

I fuddetti due conducono genti dai Pontadera,e da Vico, (Terre vicine a Pifa)
Ac quali genti dice il Poeta, che / a/pertano, perché venendo di lontano per la
Manchezza del viaggio s' erano fermate per la Mrada a riposarsi; E per moftrare,

ache questo Papirio cra. grand' ingegnere, fa che quelta gente habbia per arme un'

ordigno per faciiitare la distruzione del nimico, il quale ¢ un mantrice pieno
di rena 5 ¢ per alludere al genio vagabondo di Papirio, ed alle chiacchicres
di Don Panfilo, figura nella loro infegna un Montambanco, che sono genti
chiacchierone, ( ¢ però detti anche Ciar/atani ) ¢ che non hanno patria ferma,
fendo oggi in Firenze, ¢ domani altrove, fecondo che gli porta la speranza del

guadagno,

FReACeASSO. Strepito, romore; Vien dal latino Frangere, che vuol dit
Rompere, ¢ veramente il significato proprio di fracaff ¢ quel romore, che pro-
cede da frattura, o peeameno di materiali; s¢ bene si piglia per ogni forte di

9.

strepito. Dan. Inf.

 già venia fu per le torbide onde

Va fracasso a! un fuon pien di spavento.

E ncl Purg, Cant,
oc! Pulgghanes 14,

ecco l'alzra con si gran fracasso
Dove !'cspositore Landini dice, che Pracaffo vien dal verbo frangere
WE disgrado il Diavol »' wn canneto, Farebbe manco romore il Diavolo in uhs
stime di canne. Si figura il diavolo, per Jo più, un' huomo con le corna, con

! ali, ¢ co i piedi di gallo; onde si di

ice un Digvol n' um canneto, perché si fappo-

ne, che paflando il detto diavolo dentro a un postime di canne, pigit con le cor-
“a aN ae Ho fa»

 

 

 

 
 

 

*

 

44 MALMANTILE

na, conl'ali, ¢ con gli artigli le canne, le quali scappando dalle dette cor na;
ali, ed artigli a guisa di molla,perquotono nell' altre canne', che per efler yores
sinno frepito, ¢ rimbombo non piccolo. Quand' uno s'affatica per conseguir
gualcofa diciamo: 7 tale ha fatto it diavolo per baver La tal cosa, es' intende ha fat-
to il diavol n' un canneto, cioè gran romore, Il termine; Ve disgrado Vuol dire
Jo fimo manco: Io levo il luogo,, o grado: per efempio // tale compone versi La-
tini così bene, che io ne disgrado Vergilio, civé io stimo, che questo tale habbia tol-
to il luogo a Vergilio, ¢ faccia meglio di lui. Vedi forto Cant, 3. stan. 34. C. 6.
stan. 61.¢ C. 7, ttan. 25.

RAMMANZO, Far iin rammano, 0 rammanzina vuol dire, Riprendet? uno,
con minacce; ¢ fuona lo stesso, che far' un rabbuffo, o Rabbuffare detto sopra in,
guefto C. stan. 39.

NON gli aloza un po di campanile. Pigtia la parte per il tutto,¢ vuol dire Non
gii fa conseguire una Chiela.

LANZI, Così chiamiamo i Soldati a piedi guardie del Sereniss. Gran Duca;i
guali fon tutti Alabardicri Tedeschi: E pero dicendo: dade fra i Lanzi intende
Andé fra i Tedeschi, cioé in Alemagna; la yoce Lanzi ¢ Todesca lasciatacida
loro medesimi, che in falutarsi fogliono chiaimarsi Lanrzman', che fuona Paefano;
¢ Lanzchne% vuol dit soldato a piede, € per quelto git Scrittori Fiorentini si
servono della yoce Lanzichenecchi, per intenderé Soldati Alemanni a piede'. Et
it Varchi Rorie Piorentine lib, 2, dice così: Quanto pitt s* avvicinavano t Lanzi, che
così pee mazgior brevita gl chiameremo da qui avanti,e non Lanzichenecchi, ec,

OG-A magoga, Quand' uno va lontano dalla sua patria, dicono le nottre don-
ne, Gi è andato in Oga magoga, Ed intendono gli è andato a casa maladetta, nel
qual fenfo è prefo anche nella facra scrittura; ¢ 5, Gio; nell Apocaliffe' al 20,
dice Og magog, & congregabit cos in pralinm. Ed al cap. 7. dice In disperfionem i
tiem, ¢ si trova anche in altri' libri della Sac, Bibbia. Vedi Angel. Mons. Flo,
Ital. lingua alla parola oga magoga. Dicono ancora Gaga magoga. E forse inten-
dono dei Regno di Goaga in Attrica. Li Vocaboiifta Bolognele dice, che Og fu
pigante d' Aftarotte Rede Baraniti,della creazione del Mondo 2492. contro al
popolo d'ifrael ne i campi d' Edrai, ove fu deftrutto con tutto il suo efercito, es
cinguanta Città; ¢ che di qui venne ii significato Andare in disperfione,e in fumo,

0a casa del Diavolo, eflendo interpetrato Og magog, peril Diavolo. Sin qui
i) Vocabolifta. Gli antichi fecondo Plinio chiamavano Magog !a Città d' Edetia,

” (che Strabone dice, che è l'iftefia, che Hierapoli ) dove era il celebre ['empio

della Dea Atergatide detta la Dea Siria, ¢ dove gli Bbrei viflero in cattivita,on-
de da questo dicendofi Andare in Magog, per git Ebrei era lo stetlo che dires:
Aadar' in servith. Gio: Villani Stor. ior, lib. 5. Cap. 29. dice; Le genti, che si
chismano Tartari nscirono dalle eMontagne di Gog Magog chiannate in latino monti-di
&eleen, Conchiudo dunque, che oon dire andowm Ova Ativoza. Significa An-
do in pacG lontanissimi, ¢ di pericolo: ed & quali lo stcllo, che dice Andé 4 Buda,
che vedremo sotto Cant. 5. stan. 13. ave
TIR ATO snnanzi, Avanzato a gradi » a dignita, a utili, ec. wer
TOG. Vuol dir propriamente abito da Dottori, ma si piglia bene tpeffo per
Vabito da Prete, come ¢ prefa in quelto luogo. Mi oes * se oR

 

 

 
 

 

 

PRIMO CANTARE,. 45

~ TORNAR a casa a queste stiacciatine, Tornare a goder”i comodi della propria
casa, che si dice anche: Tornare al Pentolino, che i launi dissero: Redire ad prifti-
na Prafepia. Stiacciatina ¢ dimioutivo di Stiacciara,la quale è specie di pane, che
dopo lievito si stiaccia con le mani per farlo pitt fottile, affin che si quoca pili pre-
sto, ¢ faccia minor midoila.

S' arroge. ll verbo Arregere vuol dire aggiugnere. Al che »° arrege; al che»
s'aggiugne,¢ vuol dire; Cié anche di più. Il Lasca Nov. 5.

E cosh per non arrozer pezevo al male, si flava quiera, ec, Petr. Canz, 9.
Eduolmi, c' ogni giorno arrage al danno,.

COSA ghiotta. Cola desiderabile, cosa appetitofa; che ghiorto si dice Vno avi-
do di mangiar del buono; ¢ viene da /ndulgere gutturi.

SAL crave, Cavolo falato. Voce, ¢ vivanda Tedesca..

BIRRA, 0 Cernogia, Bevanda, che s'usa in Alemagna., ed in altri pacii,
dove ¢ poco Vino; ed ¢ composta di biade, acqua, ¢ fiori di luppoli; ed ¢ lo
fieflo Birra, che Cervogia, e quelta ultima è dal Latino.

sMBOTT ARE. Metter neila botte. Se bene qui si potrebbe intendere Bere,
costumandoli dire: Jo.mon imbotto acqua,yin vece didire: Lo non bevo acqua, si
come è inteso forte C, 7. tan. 4.:

NON (a voglio più cotta, Per la mia parte mi basta così,ne mi curo di meglio.
Sum presenti Catone contentus, dilic Auguito.
~STIZZA, Ira, collera; ¢ vale anche per Inimicizia.

FERRARST Antende Armarsi.. B decto scherzolo, perché Ferrare, senza dir
'pits s' intende mettere i terri allt unghie de' piedi de' cavalli, muli, cd altres
beitie. 

GENT gaie, Genti allegre, ricche, c abbondanti d' ogni comodo, e quiete;
che la voce Guio è forse fincopata da Gandio.

GRON DAME. Quel calcare, che fa |' acqua da i tetti, quando piove; ¢ si
dice Grondaia da Gronde, che sono quelle tegole pill Jarghe, le quali fon polte-
nicli”eAremita de' tetti, Ed il Proverbio Fuggir 2 acqua (otto le grondare vuol dire;
'Procurar di fuggire un pericolo, ¢ andarli incontro, che ¢ quello forse, che i La-
tini intesero col dire Lucidit im Scyllam cupiens vitare Charybdim, 3

ANDARE 4 Vignone > Andar nelle vigne altrui a' corre Puva; ¢ si dice così
'pet rendere il detto oscuro, moftrandofi d' intendere d' Avignone in Francia, 0
del Bagno di Vignone, che ¢ ucllo Stato di Siena.

SOPFIONE. Quel piccolo Mantaco, o Mantice,, del quale comunemente ci
serviamo per foffiar nei fuoco, usandolo a mano. |

SCHIZ ARE, Qui è verbo attivo, ¢ vuol dice: Gli gettano con violenza nel
vifo quella » che è dentro al foffione.. f
MONT AaB aNCO « Vno di coloro che vendono i rimedj nelle pubbliche

 ~~ piaze y detti Adontambanchi dal moncare sopra i banchi quando vogtiono vende-
© re'; € detti anche Ciar/arani dalle gean ciarle, che fogliono fare.
* TOCC-ATO wa bezz0, Preto, 0 bulcato un guattrino. Bezzo è moncta, ©»
Parola Veneziana, ma usiamo, s¢ non la moncta, almeno la voce bezo ancor not
- per intender Denariin generale.
Sl feandolezza. In quetto Inogo, éd'in questi termini significa Adicarsi, if mo-
— 4 car

 

CT See

 
E
.
;

 

46 MALMANTILE?

firar con se parole, € con gli atti la collera, che uno ha. Vedi foto C. 11. flan.
23. Verbo che viene dal Greco /eaudalizefthai che suona, a loro,come.a noi Offen=
dorfi, o adirarsi d' una cola.

ENT RAR in smania, Entrar in grandissima collera; che Smania è una fover-
chia inquietudine, cagionata da febbre, 0 da ecceilivo caldo, o da foyerchio

amore, la quale riduce |' huomo quafi infano, ¢ furiofo,

STANZA LVII
Hluomini bravi quanto sia la morte
Scandicci n' ha mandati,e Marignolle,
Gente, che si puo dir che habia del forte,
Poi ch'ellaammazzagli azlise lecipolles
Sue lance i pali fon, targhe le sporte,
sirchibigi le man, te palle xolle 5
Vaben di mira, ecolpocolpoimbreccia,

STANZA LVIIIL
Vien comandara da Strarildo Nori,
Ch'é Chimico, Poeta, e Cavaliere y
Ede quel, ch' in un quadro coi coloré
Fece quei fichi, che divenner pere.
E perché questo ¢ il Re de bell' bumori,
Per dimoftrar quanto gli piacciail bere; i
Ha per iprefaun Lanzoa due bracherte; q

 

Maffime quad attrui vuol dar lafreccia, CBilmolleinfegna trar dalle mexrette, a

Seguita la gente di Scandicci, e di Marignolle, Ville vicine a Firenze, doves
nascono Cipolle, Agi, ed altri fortum: simili in grande abbondanza. Questa
gente dice che ¢ brava quanto la morte, perché ella ammazza gli agli, e lecipolie, ¢
Ji puo dive che habbia del forte, E pare che intenda che ella fae in fortezza, es
bravura gli agli: E vuol poi dire, che ha molti fortumi, ed Ammazza, cio¢ Fa
mazzi delle cipolle, ¢ degli agli. E perché questi contadini habitando intorno
a Firenze praticano molto la Città, dove ¢ occasione di spendere più che nel
coatado, dice l'Autore, che fon genti che danzo la freccia, che vuol dir Chie-
der denari in prefto; ¢ par ch' ei voglia intendere che fon bravi tiratori di free-
cia, ed' archibufo. Son comandati da Straczildo Nori, cio Rinaldo Strozzi Ca-
valiete di S, Stefano; ed ¢ quello, che in fquola dell' Autore volendo dipignere
alcuni fichi non trové mai il modo di fare, che non pareffero pere. Questo fu
un geatilhuomo di grandidimo garbo, faceto, allegro, ¢ spiritofo, e buon be-
vitore; ¢ percio gli fa fare per imprefa un Lanzo, che vota una mezzetta di vi-
no,¢ gli fa comandare quelta gente, perché fu poi P...... in vicinanza dei
lor pacfi.

'SPORT A « Specie di paniere fatto di giunchi, ed ha due manichi; serve per
pereret dentro erbaggi, ed altro, che si provvede in piazza giornalmente per il

itto.

ZOLLA, Gleba, pezzo di terra follevata nel lavorare i campi, Vedi forto
in questo Canto stan, 82..

COLPO colpo. A ogni colpo. Intendi; fempre ch' ei tira; colpisce, che 1a for-
za della replica ¢ di far nascer il superlativo, provi t

IMBRECCIeA, Forse meglio imbercia; E Significa Pigliar di mira; donde»
imberciatore colui che fa pi ione di tirar d' archibufo; ¢ par che venga da»
sbirciare, ¢ bircio, che € guardar con occhi focchiufi, come dicemmo sopra in
questo C, stan. 9. ¢ come s' usa a tirar con l'archibufo. Ma puo anche essere che
venga da breccia che vuol dir Quelle rotture che vengon se Y ie>
dail' artiglicric, e si dica imbrecciare per colpire, si come intende nel prelente
Juogo pigliando colpire in (ealo di conseguir I" intento,.

 

sari

alas 5° < co pilmiila Rae aineae o
 

PRIMO CANTARE: 47

} 'DAR Ia freccia, Come habbiamo accennato, vuol dire Chieder denari in pre-
. flo; ¢ s'intende Vno che habbia poco modo, e minor voglia di rendergli. Gli
antichi Etiopi, ¢ gli abitatori di Maiorca, ec. non folevano dar mangiare alli
loro figliuoli, se questi con le frecce non facevano cascare dallo stile, 0 albero
il cibo, che vi era posto, ond' io stimo, che questo frecciar per vivere habbia da-
i to origine al presente detto. Vedi Alex. ab Alex. dier. gen. lib. 2. c. 25. [1 Mo-
nofino dice, che questo frecciare habbia origine dal Latino ferire che apprefio
loro haveya il medesimo significato, ¢ lo cava da Teren. in princ. Phormionis:
Porro autem Geta Ferietur alio munere ubi bera pepererit. Diciamo; idenari sono il
fecondo fangue; dar ferita cava il fangue, come il dar frecciate, cava il fangue;
¢ per questo dicendo dar freccia intendiama Dar freccia alla borfa, ¢ cayare que-
flo fecondo fangue, che è il danaro.
BELLY MORE, Huomo allegro, faceto, ec. vedi sopra in questo C. stan. 10.
Quando diciamo, Ii tale è Re della tal cosa; intendiamo Vale in superlativo
grado in quella tal cosa; onde Re de belli humeri vuol dire Grandissimo bell' hu-
more. Significato che viene da iGreci, i quali chiamavano Re colui, che nei
giuochi fanciuileschi vinceva, ¢ superava gli altri, ed Afino,o Mida era chia-
mato colui che perdeva; il che più ditfusamente vedremo nel 2. Canto.
: LANZO a due brachette, Lanzo dicemmo sopra, che vuol dir soldato Tedesco
a piede; ma qui vuol che s' intenda uno proprio di quelli della guardia del Sere-
} nissimo Gran Duca; dicendo a due bracherte, perché questi tali Lanzi vanno ve-
titi a Jiurea, con un paro di brache larghe, fatte a strisce, come fon quelle delli
*Svizeri del Papa in Roma, ¢ come quelle de' Trabanti dell' Imperatore.
INSEGNA trarre il molle dalle mezzette.. Infegna col suo bere,come si fa a vo-
tare i vafi pieni di vino, Che wezzerra ¢-un valo fatto di terra inuctriata, che
serve pee — il vino,ed ¢ capace della quarta parte d' un fiafto Fiorentino.

: ZA LIX STANZA LX.

. Morbido Gatti, Henrigo V incifedi Nell infegnahan ritratta uw huom canito,
A far venir innanzs ecco fon pronti Che troppo bavédoil crin(per offer vecchio)
I fanti, che ne da il Ponte a Rifredi, Fioccofo, ¢ lungo, un fanciullina aftuto a
Che mille sono annoverati,¢ conti. Dietrogligrida:Gli abbrucia il penecchio.
Han certs Santambarchi fino a piedi, Da queffa febiera qui st ¢ pravveduto i
Che chiaman' il Rimbel di la da monti, Gran ceffe piene a' huova,e di capecchio
E paton con la spada in fu le polpe Con fafee, perze, ¢ tafte accomodate
Vn che facia lo ferascicg alla volpe. Per farsi alle ferite le chiarate.

Pafla l'ultima truppa di Soldati, la quale ¢ composta d' huomini dal Ponte a,
Rifredi, che € un luogo vicino a Firenze. Coftoro fon comandati da ALorbido
Gatti, C1Oe Migiotto Bardi,e da Henrigo Vincifedi, che ¢ Vincenzio Federighi, due
gentilhuomini già [colari dell' Autore: E perché questi si pigliavano gusto di ra-
gionare speflo con un tal Dottor Cupers, glicio fa fare per imprefa.
}  A Questo Dottor Cupers negli ultimi anni della sua vita, che durd sopra ot-
tanta anni, entrd in frepefia d' esser bello, ¢ si perfyadeya che ogni donna s' in-:
namoraiie di lui, ¢ lo volefle per marito, ¢ pero andava lindo, e com la chioma
folta, ¢ lunga, ¢ ben coltivata; ma canutissima: onde i ragazzi quando pafiava.
per le strade gli gridavano dietro: Guarda il Pennecchio, gli abbrucia il Pennec-

thio s

    
Vas lle

48 MALMANTILE

chio, intendendo di detta sua chioma, ¢ lo facevano adirare,-¢ maggiormentes
impazire. E perché li contadini del Ponte'a Rifredi si danno a credere a' haver
maggior Civilta degli altri contadini per efice nati, ed allevati, si pud dire, nei
Borghi di Firenze, ed intorno alla Petraia, ¢ Castello, Ville spetlo habitare>
da Principi della Serenissima Casa, percid per Jo pili vengono 'alla Città col
ferrdiuolo, 0 fantambarco, che sono le Toghe de i Barbaflori, ¢ Dottori
del Contado; e per questo il Poeta dice Han certi Santambarchi fino a piedi, Che.
chiamano il Zimbel di ld da? monti, cio' incitano i ragazzia dar loro delle Zimbel-
late. E per efler questa l'ultima schicra fa, che ella conduca seco il bagaglio
de i medicamenti per |' Efercito.

SANT AMB ARCO, Specie d' abito, 0 sopravvefte, 0 diciamo mantello
usato da i nostri contadini per difenderfi dall'acqua,e dal freddo; ed € composto
di due larghe strilce di panno cucite in forma di croce con una buca in mezzo,
per la quale paflano il capo, e vengono coperti da una parte di detto panno les
schienc, ¢ il petto, ¢ dall' altra le braccia, ei fianchi, Si dourebbe dire Saita in
barco, © così dice Mattio Franzefi nel Capitolo del suo viaggio da Roma a,
Spoleto,

Gli offi, ¢ a profferir mai non fon parchs
Volean ch' io scavaleaffi a si mat re
E m offerivan fuoco, e Saltambarchi.

Ed è forse meglio detto Sairambarco; perché questo abito  composto in tal
forma; che tiene tutta la persona difefa dalfreddo, ¢ non l'impeditce il faltare
i foi, ¢ paflare i barchi. Ma si dice Santambarco perché così lo chiamano i con-
tadini che se ine servono, ed ¢ lor abito proprio.

CHIAMAR: una cosa di la da i monti, Questo termine significa Meritare una.
cola grandemente, come per clempio 4 tale € cosh injosente, ch' ei thinma le baffuma-
te di la dai monti, ff
. ZIMBELLO. In questo logo intende un facchetto pieno di crusca;
© di cenci, o di fegatura, legato a una cordicella lunga circa due braccia,
col quale i fattorini delle botteghe de fetaiuoli nel tempo del Carnevale, quando

aflano i contadini per quei Iuoghi 5 dove sono le botteghe de i fetaiuoli, uno di
loro perquote il contadino; ¢ mentre questo si volta per veder chi ha percosso,
gli altri ragazzi lo perquotono dall' altra banda: B questo per !o pili vien fatto a
certi 'contadini, che se ne vengono in Firenze intronizzati, ¢ in sul grave, come
appunto fanno quei del Ponte a Rifredi. E per altro la voce Zimbetio ha il signi-
ficato, che vedremo sotto C, 7. flan.76.0 e

FAR Io ferascico alla Volpe, EB' una tpecie di caccia, che si fa alla Volpe, piglian-
do un pezzo di carnaccia fetida, che legata a una corda si va fira(eieaniso pes
terra; per far venir la Volpe al fetore di cfla Carne; ed il Poeta aflomiglia il por-
tar della spada di questi Contadini a questa corda, dicendo che ttava pendentes
in [w le polpe ( cio' dietro alle gambe, che così chiamiamo cotelta parte ) appun-
to come sta a fune di colui, che fa lo strascico alla Volpe. sai”

PENNECCHIO, Quit prefo per chioma', 6 Zazzera, come habbiamo accen-
nato sopra, metaforico da quell' inuolto di lino, floppa; lana, © altra materia
simile, che adattano le donne sopr'alla rocéa per filarey il quale inyolto si dicc
Vennscchio. Ore.

i) 9 ihS

 
 

PRIMO CANTIARE: 42

¢ 5, cc api qui. La voce qui è superflua, baftando per farsi intendere il
dir folamente da guefta Regina senza aggiungere 1a particella gui: Manon per

il nostro Poeta ha fatto errore, havendo seguitato il nostro Fiorentinismo
wfatissimo; Dicendofi comunemente ( forse a maggior' emfali) Questo negurio qui,
questa cosa che è qui, ¢ simili; ¢ la particella qué esprime st megegioydel guale ragio-
niamo presentemente 5 Quefia cosa, la quale habbiamo fra le mani: Anzi stimo, che
Tyhabbia fatto ad arte, ¢ per moftrare questo nostro modo di dire, ( forse ripren-
fibile ) del quale non mi pare, che in tutta l'Opera si sia servito mai più; quan-
tungue non gli fieno mancate l'occasioni; E se bene nell' Oicava 65. seguente—,
pare, che l'usi nel medesimo modo, offeruifi, che quivi ¢ termine dimoftrative
neceflario, ¢ non riempitivo, operando che s' intenda di quella Cugina, che è li
presente, ¢ non d' altra, come si potrebbe intendere, se non vi metteffe la parti-
cella qui.

CEST A, Intendiamo un gran paniere, che fa mezza foma di beftiajed ¢ con-
tefto d' afficelle di castagno, o d' altro I¢gname a foggia di cafla, per uso di por-
tare da un paefe all' altro uova, vino in fiaschi, ed altre cole frangibili; ¢ per Jo

it fon fabbricati due attaccati |' uno all'alcro con quattro legni gagliardi aggiu
ati in maniera da adattarsi opra i bafti a traverso alla beftia, in modo che ten-
gono equilibrate, ¢ ferme dette due cefte anche senza legarle, Se ne fabbricano
ancora della ficla forma, ¢ materia sciolte, cioé senza i detti quattro legni, ©
s' adattano, ¢ fermano in fu i bafti con le funi, come si fa 1 Ceftoni, che»
ancor' essi panieroni di mezza foma fatti di vinciglie di castagno, o altro al-

bero inseflute de i quali si parla sotto C, 10. stan. 7.

'C-APECC IO.. La pettinatura, cioè quella stoppa più groffa, che si cava dal
lino sodo la prima volta;, che si pettina:detta capecchio, perché si cava dai due
capi del lino, cioé barbe, ¢ cime, le quali sono più ripiene d' immondezze, ¢ di
filo morto, e inutile.

FAR Ia chiaraca. M primo medicamento, che si faccia alle ferite & l'albume,
“o chiara d@ huovo, entro alia qual chiara s' iatigne il capecchio, ¢ si pone sopra
alle ferite; E questo si dice far la chiarara,

STANZA LXh

“Ei general ai tutta quella mandra Lascis gram rempo fale polpein Piddra,
Amospante Laton Poeta infigne Mentre si dava il are vigne 5
Canta improvvifo, come una calandra, Fortuna, che l'havea matto provato

Stampa gli enigmi,ferolaca, ¢ dipigne Ville, ch' ei diventaffe anche spolpate.

«> Generale di tutto questo efercito ¢ Amoftante Latoniy cioè wAnromo Adalatefti
Poeta celebre per ieee sue opere, ma specialmente per quella Sfinge, la quale,
come vedremo forto C. 8. stan. 26. ¢ una scela d' enigmi in (onewi y de' quali (es
ben la stampa ne fa goder pochi, se ne sperava numero maggiore, volendone>
gli pubblicare goo. scelti da una infinita, che ne ha compotti; ma la di hui mor-
'te seguita poco tempo fayci priva per 'ora di questa consolazione. Ne gli anni suoi
Piovenili canté ail improvvifo molto lodatamente, si dilewo d' Aftrologia, e nel
-difegno fu scoiare dell' Autore, ¢ suo amicissimo, come mottra, facendolo capo,
'e faperiore di cacti gli amici faoi, che aomina in questo clercito. EB perché queito”
wimoftanre cra di corpo aduito, ed havea le o fori, dice 5 che /a/cio se poipe

an

 

 
«

50 MALMANTILE? %
in Fiandra, ¢ ché la Fortuna che' t havea provato marto, volle:che egli diventaffes
anche /polparo, cioé senza polpe; ma aggiunto alla voce marto vuol dire marro af=
fatto; non che Amoftante fufle affatto privo di cervello; che la voce marro a
pretio di noi significa ancora Allegro, Faceto, ¢ simili, nel qual fenfo & ptefas
nel presente Iuogo; ¢ però vuol dire, che Amoftante era huomo facetissimo.
MANDRIA, Vuol dire Vna gran cuantica di bePies ma quirintende Granis
quantita d' huomini, Mandra è voce Greca', che fuona 'Spelonca + ¢lnogo, eri
tro al hs le pecore s' adunano all' ombra,ma la pigliavano anche per la gregs
ga medesima, ¢ da essa diflerd Archimandrita eee meeps grepgias
ante pure prefe Adandria per quantita di huomint, nel Purg. C. 3.
Si vidd' io muovere, ¢ venir la tefia
Di quella Mandria fortunata allorta,
Paudica in faccia, ¢ nell' andare onefta,
CANT A improvvifo. B coftume in Firenze al tempo de i granealdi la notes
cantare dell' ottave all' improvvifo, mentre ne i luoghi ey aperti-della Cictafi
va pigliando il fresco; e perché in tal' efercizio wave molto il Malatefti; il —
ta l' afomiglia alla Calandra uccello di bellidimo cantare.
ENIG AU. Indovinelli.Voce Latinogreca.V edi forto0.6, Nan.34.c C.8Man. 26.
LaSCIO' le polpe in Fiandra, Non è, che Amoftante futile mai stato in
Fiandra; ma, perché lo fa generale di quelto efergito, ¢ dovere,' che egli mo-
firi, che Amoftante ha vedute,, ¢ provate dltre guerre,¢ che egli si fa trovatoa
dar de' sacchi ne i quali ha Jasciate le polpe delle gambe;, il che Pea
ditarlo 5 2 peenghe si come ad un soldato gli stroppj, ¢ le cicatrici fon di gloria, cosh
ad Amoftante era di gloria  haver perduto le polpe delle gambo adie guerre di
Fiandra; ma il vero¢, che quand' uno hale gambe fortili, diciamondi lui: Reb x
ha lasciato te polpe in Fiandra': ed il Poeta con questo equivoco, che accreditas i
Amoftante, vuol dire, che egli haveva le gambe fortili; ¢ seguita con I mages E r
equivoco di e#ateo /polparo; che significa; come s'¢ detto » matto del tutto,
-vuol che s' intenda /exxa polpe affasto. E la voce polpa, che significa oe ez
quantita di carne, che sia fenz' offo, da noi si piglia/perle pole
quando ¢ detta aflolutamente.( Vedi T ottava's9. antecedente; E forto al oe &
Sian. 99..dice ofaccia senza polpe che s' intcnde tuttala carne di quel' corpo )¢
signitica pure Afatro spacciaro ¢
STANZA LXII,

 

Paffati tutti con baule, ¢ spada ch chlo biftraca ye come che ne vada
. Serranfi in barca, come ia fardelle; Gilt la vinaccia,e il fangue a catinelle 5
Gimil pan aaindamnrie « Eben eee eevanatens ae wud
O ferma un palo; guai alla sua pelle Aion gli da tanto tempoch' ei respiris
Dopo fatta la mottra se Renta la pide pulebarche con ogni suo arncle,
¢ Baldone affretta all"imbarco i sold; »255 a wees i
S4VLE, Vorendiamo ogni Gdeate i Gates waligia., o:-tambato, che: (
mente si posla adattare in fu yale ae d'un cayallo, mentre wife
' dal verbo bainlo, ¢ l'allarghiamo ad ogni fortardi cafla portatile-inda le fome, €¢

Qui intende quell inuolto., che oe i soldate sopr' alle reniiper for preere
bagagho,\detro alert! zaino ie oli 72, tb?

a SER-

 

 

 
 

PRIMO,CANTARE, “st

SERRANSL,, come te fardele.. Si serrano strettissimi appunto, come stanno le
Tardelle ne i cefloni,, quando da Livorno (on portate a Firenze, 0 nei bariglio-
ni, quando ¢ivengono falate. Comparazione aifai usata per intendere Aretti, ¢
ferrati insieme, che in voce marinaresca si dice stiuati,

 TENERE a bada., Tratteacre uno. Varchi ttor, lib, 4. Conoscevano y che erano
tutte cose finte ye folo per tenere a bada trovate, Viene dal Verbo Badare, che has
molt significati, Zadare al negozio per Artendere al negozio. Signitica Indugia-
re,0 perder il tempo, come ¢ inteso nel presente luogo, che dice tiene 4 badas,
ed intende, Chi gli è caula d' indugio,o gli fa perder tempo; il Petrarca Son.23.

Confolate lei dunque, che ancor bada,
| Cio' aspetta 1a venuta del Ponteficese perde tempo. Significa ancora covtizuare,o
| Leguitare a far una cola, Vedi forco.C.1, stan. 20. Sigaitica Oferware C.9. fan. 28.
| Significa Di/prezzare, non curare, per elemplo; /o non bade al tuo gridare, latende
jo non stima, onon curo il tuo gridare, Da quetto badare 50 bada habbiamo badaloncy
che.vuol-dire Vin' huomo perdigiorno, ¢ che non (a, ¢ non vuol far nulla.

GV AL alla sua pelle, Mai per iui. Vedi sopra in questo C, tan, 28,

BIST RATT -ARE, Tratcar male', Strapazzare, o Stranare,

VeA giit 1a vinaccia. E! neceflario tar pretto per sfuggire il danno, che si pati-
Ace ¢ che si teme più grave dali' indugio. Quando il mofo, cioè ib liquore ca-
vato.dall'uva, il quaic ¢ nei tino, ha boilito a battanza; perde il vigore, ¢ non
puo pil foftenere a galla, cioè nella sua superficie, la vinaccia ( che così Gi. chiar
mano iralpi y¢ bucce-deil' uve ) onde la Jascia calcare in fondo, ed incorporan-
dofi con i nuovo, si guata; B questo si dice uxdar git la vinaccia; che
poi paffato in proverbio figaitica Quel che habbiamo detto.

NE vail nea catineke, Ne " imnolto Ack mia Per intender, che Vn' in-
-dugio apporta grave dispcndio, ct serviamo di guefto detto; ¢ si dice anche: #
Soon Vedi Se c pain. 20.;

4ESTO. Qui vuol.dir Pronto, ed all' ordine,

WON gli-da tempo che respiri... Non gli lascia ripigliare il fiato.. Questo detto
e(prime ua grande athrewamento, 0 incalzamento.

h STANZA LAL,

Percid imbarcati tutti in-un momenta y

STANZA, LXIV.
Rifiede Malmantil four' un pazgerto 5

Poi che Baldon faces così gran ferra,  chiungue verso lui voltale cizlia
i Stspiegaron Vinfeznd, ¢ veie al ventas Dice, ch'i fondatori hebber concerto “
e Qu ~~ Di fabricar L ortava meraviglia,

te Navi si spiccar da rerra;
Ed egh: allora entro in ragi: 0
Di quel che lospingevaa far talguerra;
Ma per contarla pik diste/a,e pranay
Incomincid così dalla lontana.

L' ampro pacfe poi,ch' egli ha Seggetto
Non Fite, ginocare,a milie miglia;
Ve laria buona azrurre oltramarinay
E.non vi manca latte dt gallina,

Fatta la moitra, ed imbarcate in brevitiimo tempo le soldatesche, si partirono
le Navi dal Jido ¢ fecero vela spiegando le loro infegae.. Intaato Baldone da

 oprincipio a nareare la causa, che lo muove a far la guerra di Malmantile 5 ¢ co-

-amincia dal descrivere la fituazione, quaiita, ¢ dominio.
. FAR ferra. Afscettare. In,alzare. Vedi foro C. 9, stan. 13.

CONT ARLA dife/a,e plana, Intendi, Raccontarla punwualmente, ¢ cORe
G2 VON

~
: 35
* Bet

tutte le circoftanze,

 

 
 

gu MALMANTILE

NON si [a wvo gixocare a mille migia. Io giuoco, che non si trova chi sappia
© posla giudicare a mille miglia, quanto paele gli ¢ fuggetto; perché è così gran
paefe, che mille miglia non si considerano, eflendo paruita di numero, ¢ di ma~
teria in riguardo del tutto, che gli ¢ fuggetto. E questa voce /uggetto > che vuol
dir fortoposfe, s' intende Situato sotto, ¢ non fottoposto al dominio di Maimanti-
le, che per esser Posto nella fommita d'un poggetto, ha d'attorno molta pia-
nura, € colline fortoposte, cioé pili baile di lui; se ben par, che voglia dire, che
Malmantile ha dominio immenfo.

ARIA axzurra oltramarina. 1 pittori dicono buon' aria quella, la quale ¢ co-
lorita con l' azzurro oltramarino, perché questo non perde mai il colore, come
perde l'indaco, ¢ lo smaito; ma € pero anche vero, che quando I'aria si vede di
colore azzurro, come è il buono oltramariao, @ fegno, che è purgata da ogni
imperfezione di nebbia, o d' altri maligni vapori, ¢ per conseguenza ¢ aria buo-
na; il Poeta però dice, che a Malmantile ¢ aria aczurra oltramarina per intende-
re, che a Malmantile ¢ aria, che dura fempre azzurra, come fa quella colorita
con l'oli ino, cioé fempre buonil ~ BP oft i € quel colore »che si
cava dalla pietra detta Lapisiazzuli.

NON vi manca latte di gallina, Vi sono tutte le cose squifite,t abondante dogni
bene. Detto antico, si come si cava da Strabone lib, 14., dove discorrendo del-
lc campagne di Samo dice, che crano così fertili, che si diceva comunemente,
che produceffero fino il latte di gallina, cioè quelle cose, che ¢ imposlibile, ch'
altrove si trovino, come è il Jatte di gallina. Samus, dice egli, feracissima, unde
laudantes non dubitaut illud ci proverbium accommodare » quod fer i i
lac, ec,

 

 
 
 
  

STANZA LXV, STANZA L

Ut Re di questo 'Regno giunto a morte Gobba, e xoppaé coftei, € mancina,
La mia Cugina qui, che fu sua Donna Ha ilgox20,e da due sfregi il vifoguafte,
(Non havendo figlinoliza altri in Corte Storfe in Firenze ognor lacavallina
P. opingui pik)lafeio donna,e Madonna: Ae i lupanari con grate pompaye fafto s
Ata come volie la sua prifta forte, E perch ofequij havea fera,e matting,
Vn certo diavol a' una Mona Cionna E il ritol di Signora a tutto pasto, )
Figlinola d'un guidone ignudo, ¢ scalzo Patra arrogante, al fine alzs il pensiero
Ne venne presso a farie dar (0 shaizo. eA voler quefis onuri da dovero.

Narra Baldone, che ii Re di Malmantile inflitui Celidora erede del Regno, €
che quetto le fu usurparo da Bertinella, la quale descrive per una donna tuttas
contraffatta,, ¢ la moftra una vera fgualdrina: ed imita Dante nel Purg. C.19.
che dice.;
Hi venne in fogno wna femmina batba,

Con gli occhi guerci,e sopra i pie distorta,
Con le man monche, e di colore scialba, i

Qui & da considerare, che i tanti diferti da Baldone attribuiti a Bertinella y
realmense im lei non fullero, perehé, ed egli non se ne farebbe innamorato,come
si dice souwo nel Cant, 9., ed ella non haurebbehavuto tanti altri amanti; Ma.
Baidone non I havendo mai yeduta, e voleado concitar contro di lei odio di
quet folaau, che lo seguivane, per ustigargli ad andar pil volensicri alia ricupe-

razion:

Bae oe ee

 

  

 

 
ee: ~s on =. 'Sa ot

PRIMO CANTARE: 53 '

sazione di Malmantile, la rapprefenta loro una donna così nefanda -
SV-A donna, Sua moglie, Se bene i Poeti dicendo La mia donna, o La sua.
donna, intendono l'amata. "
LASCIO' donna, ¢ madonna, Termine notariesco, e curiale, che significa Pa-
drona affoluta. Sincopato di Domina. è is
VN certo Diavolo, Si dice così quando vogliamo esprimere uno, che è cagione
di qualche nostra disgeazia: per elempio: / negoxio andava bene, ma un certo dia-
volo d' un Senfale con le sue chiacchiere lo rovine quali dica » di diavolo, che guafto que-
+3 fo negoxio, fu un Senfale.
MONA Cionna, Bun devto di-disprezzo, che significa Donna da poco in
ogni operazione: ed il fenfo della voce Mona, Vedrai sotto C. s. stan. 18.
GVIDONE. Intendiamo huome vilissimo, abietto, senza roba, ¢ senza crean-
t za, 0 riputazione. 'i %
| DAR (0 sbaizo, Mandar via; Scacciare,
} ORO. In queito luogo vuol dir Vno, che vede poco, che noi chiamiamo
lusco, se bene il suo vero fenfo ¢ dicieco affatto. Vedi sopra in questo C. flan.
g. alla voce sbirciare.
MAN. INO. Vno che per affuefazione ha maggior forza, ed attitudine nel-
la mano finiftra, che nella deftra; E perché quelto tale si pud dire djfettalo;
— huomo mancino., vuol dire Huomo non buono; ed in questo feafo ¢ pre-
fo nel presente luogo. E però voce che ha del furbesco. Se ne servi il Lalli nella
sua En. trav. nel C.2. stan. 4a,dicendo,
. Perch' io gan fui mai orbo, ne mancina,
Edal C, 4. stan. 67.
E rinscito in famma un buom mancine,
Vina delle pitt vili creature
: 0" habbia fio mondo; ¢ paxxo da catena;
HA il goxxo, EB' parola nota, venendo dal latino gattar: Ma qui vuol dire
wn gonfio, o ferafa, che vien nella gola, che i medici, che scrivono di fiuil 4
s male pongono al trattato il titolo de Boccis.

SFREGIO, Cicatrice di taglio nel vifo. Ed una donna sfregiata ¢ numerata,
fra le infami, e per la deformita del volto, e per la causa, per 1a quale si suppo-
ne, che le sia stato fatto. Vedi forto C, 2. stan. 3. dove si moftra eller tali stregi J
vituperosi anche negli huomini, ed al'C, 6. stan. 54.

SCORRER /a cavallina. PighiarG wutti li (voi guifii liberamente, ¢ senza riguar-
do alcuno. Havere scorfa la cavailina ne i lupanari, vuoi dit, che era merctrice
vecchia, ed avanzata ai bordelli, ¢ lupanari. Gli antichi Egizj, quando vo-
levano esprimere la sfacciataggine meretricia, figuravano una cavalla senza fre-
no; il furore della quale nelle cose Veneree esprime Vergilio 3, Georg. dicendo.

, Seilicer ante omnes furor eit infignis equarum.:
| » TL titol ds Signora a tutto pasto, Cioé continovamente era chiamata Signora. &
F  Termine usatissimo per intender voglia cosa, che si faccia molto,¢ contino- '
\yatamente'. Il Mauro nel Capitoio in lode della Torniella dice,:
2 ». Eragionidi voi a tutta 2
Daddovero. Per debito: Per giuitiziay Per merito. Intendi che volle proccu-
ror

 

 
34

STANZA LxVIL
Così la mira ad alto havendo meffa s

A suoi Fruftamationi ua da vicorfa,
Bramar dive una grarsa ye che in essa
Non si tratta di feorporo di borfa;
Ma, perché aspira a farsi Principeffa
Desidera da loro esser fovgarfa
Col loro aiuto,volendo, € consigtia.,
Provar,va Malmanil puo dar di piglo,

'MALMANTILE®

rap d havere stato, 0 Ggnoria per meritare il titolo di Signora y-¢0, ed,
che quel.da dovero non ¢ la yoce vero conl'aggiunta della sillaba do; ma ¢ il no-
me dovere mefio in uso di dirlo così corrottamcnte in cali simili.a queste » © per
eiprimere una ¢ofa di doverey0 doverola, ¢dovuta,¢ gina.

 

SLANZAY LEVIN,
Pronto ciascunose vuoltra miile frocchi
Esporre si ventre, <ameun Paladino: y
Che per servire a Dame,.raii allocchs
Cercan l occasion col fuscetlino;
Ma eordi pas » otratti ds baiocchi,
Lerche no hanounbeccod'nn quactring;
Ecredon, promettendo Roma,e.Toma,
Di spacciar l'oro dela bionda chioma:,

 

 

Bertinella havendo facta la fuddetta risoluzione y richie(e li suci amantiy:che
la voletiero aiucace a farsi Pringipeda con mpadronirfi di;Maimantile, ¢di.taoi
Deudi 8 ¢fibiscono a (eruirla, perché fentono di nou haver a (penderes il che &
cercato da tutti coloro, i quali con simil donne pretendono di. patiar per belli5
che ¢ una delle tre (pecie di persone, che voglion quette femmine dntorao,\cioé
1/ belio per sua propria (odisfazione. Ll bravo per farsi riipettare. Bd il ricco
minchione, 0 corrivo, per cavar danari da lui, per campare se medelime 5.edé
primi due, I) Perfiani dice, vd

Ul bravo, ed il corrivo, ed il valente «

Nella mia Mea fallifee 9 ctoae

Questo antico detratoy

Per c' al bravo, ed al bel nom apparifee 5

Ma fol vorredbe il suo minchione allAtone A)»

PORRE ad alto la mira, Alpivare'a cose granti. Aira si dice. quel fegno,» che
è nella canna dell' archibufo y o nelle baleftre, nel quale's' affitla |" occhio per ag-
giuftare il colpo al berzaglio, E di qui Porre La mira a unacofas' intende Volgere
al pensiero, 0 aspirare a una cosa, i hema Hm

£RVST AMATTONL, Si dicono Quelli, che giornalmente vanno in una
casa, 0 bottega, enon vi spendono mai ud soldo y o vi portano. utile aleuno,
E Gi dicono Fraftamatroni, perché non fon d' altro giovamento y che frustares,
cioè spazzare, ¢ ripulire con le i marroni; iquali fon quelle Jaftre fatte di
terra cotta, con le quali si laftricano i pavimenu 'delle stanze ».da1Jatini detti
Lareres:” s eg

SCORPORO di borfa, Spendere.. Scorporare vuol dit Eftrarre dauna mafia,
da un corpo, o quantita > O.una porzione di essa.. Ain hs cogent

DAR di piglio.. ln quelto luogo vuol dic Pigliare', impadronirGi; ed.alle volte
vuol dir Principiare come fotco C.6, fans. weno S

ESPORRE il ventre a mille flocchi, Vanti a' innamorati d' andare-foli contro
aun' efercito:intero, come i Poeti fa iano, che faceflero i Paladini, che
sono quei dodici Conti di Palazzo, ordinati da Carlo Magno per combatrere
coatro ai nimici della S, Fede Cattolica, che furono detti Comites Palatini, cioè
'Compagni nel Palazzo, che sono forse gli odierni Pari di Brancia': the noi poi

'cor-

 

 

 
Te

PRIMO'CANTIARE: 35

corrottamentechiamiamo Paladini, ¢ con questa voce intendiattio. Haomé bravo.

ALLOCCO. Specie d' uccello con il capo cornuto, come |' affivolo, ma ¢

| più grande', ¢ di colore lionato, con occhi grandi, ¢ lucenti, E* animal goffo,

¢ (e bene vive di rapina, tuttavia è tanto poltrone, che per cibarsi aspetta di pi-

gliare gli uccelli »quando gli vanno scherzando atrorno, tratti dalla di lui gof-

| faggine; ¢ quando se:liavvicinano,, non con rapacita, ma'con flemma, ¢ gra-

} vita non ordinaria gli prende col roftro, o con gli artigli; EB da questa goftaggi-

ne nel far all*amore, ed aspettare gli uccelli, per-Allocco intendiamoV no,. che

se ne ia perdendo il giorno in vagheggiar Damerfenza proficto. 5 edȢ Lo stesso

che Fruffamattoni, Colombi di geffo, ¢ simili.

Con nome -di/occo in molte parti d' Italia è chiamata ancora la Civetta,

| e credoy perché ¢ di figura, se ben pili piccola; simile a quella dell' Allocco, e>
vive con Ie medesime arti,;

-CERE-AR cof fusceline, Cercar minutamente ye con diligenza; \st tale cerca le
buffe col fuscelline yyuol dite; Il tale fa tutto quel che egli puo, per esser percoflo,
© per toccarne + Questo detto vien da quei ragazzi dell'.infima plebe, i:quali do-
po.che &venuta in Firenze una gran pioggia', che habbia fatta- corer IL acquas
per la Citta', vanno.cercando per le strade vicine alle gran fogne, che portano in
Arno, se trovano fra le commenticure delle laftre delle strade spilli, chiodi 5 ed
s altre cose simili portate, ¢ lasciate quivi dall' acque correnti; ¢ per far cid si
servono d' uno: stecco, o fulcelletto di opa',o d? altro', col quale: vanno rifru-
gando bfelli di dette commettiture 5 perché cos gran diligenze fon troppe al
poco utile,n°¢.natoiil fuddetto proverbic, checha l'acceanato fenfo', ed ¢ lo
itefio:che chiamar'una cosa di 1a da i monti, detto sopra in questo C, stan, 19.

- BALOC CO... E parola, ¢ moncta.romana', la qual parolaé talvolta usata da
noi per intender Danari y come qui, che dicendo Won si park di baiocchi intende
Won si parli di danari, cioè di Spendere.
<0 NOM hanno un becco d'4n quattrino, Non hanno pure un denaro, ¢ quella pa-
rola Beceo si metce a maggiore elpreifione, quafi dica Non hanno ne pure un fol
quattriuabecco; cioè cattiyo, ¢:non il cao' a spenderfi; Se non voleitino dire,
che-veniffe queito detto dail' antica moncta Romana di rame; pella:quale era im-

da uoabanda il yolto di Giano.con le corna, ¢ dal' altra un toftro di na-
we eche al dire; Wa-becco d' un quattrino sia lo steffo 5 che dire, ne-anche la
ete di pio' la faccia di Giano, che è cornuta.

. BROMETTE Roma,e Toma, Promette cose grandissime,\e che da personas
alcuna non si posiono mantenere,o offervare; i Latini ditiero Adaria,> Afomtes polli=
eer, La voce toma non fo che habbia nel nostro idioma significato alcuno, ¢ sti-
mo; cheyfiaulaca in ydetto per darle la rima con la parola Roma; Se for.
denon ho spagnuolo somar, che vuol dir torre:,.0 pigliare, ed intender-
si Ti prometto Roma, ( che ¢ a dir tutto il mondo ) e ru toma, ciok piglia quel che

refhava, di follecitarla promettendole Koma ye toma y
det mondo,;

   

piace. Lafea Nov,
 igome se gli fulfe il primo

a <8, 'Però-non

BAYS t ¢ 2 cS

STAN-

 

 

 

 
 

so MALMANTI LEB)

2 STANZA LXIX,

Eva tra molt snoi piie fidi amanti
Vn ciarla, che peré detto 8ilCornacchias
Ed ¢ di quei pittor, ch' i viandants
Con lo stioppo dipingono alla macchia;
E perché nella linguaha ilfuoincotanti,
Molto si vitayal]ai prefumesegracchia;
E finalmente colorisce, ¢ tratta

STANZA' LXX.
Scrive un viglietto poi fecreramenté
Ad un compagro suo capobandito y
Dicendo, che weduta la presente's
Li suo bagaglio subito. ammannito 5
Di norte tempo meni la sua gente
A Rimaggio alla Svoita del Romito;
Ata vada allaspeRrata, ¢ pei tragetriy

 

 

« Questo negorso, come cosa fatta, E senza pensar' altro svi L asperti,
65 = STANZA LxXxX se a
wind la carta, e queic' bebbe L inte/a,

Come quel ch' inuitatoera alfuogiuoco
Andonne, ¢ guido feco'a quel? imprefa Anch eglino con groffia ye folta schiera.
Cent' huomin con le lor bocche di fuoco, D' una gente da bosco ye dariviera.
. Fra questi suoi pil: fedeli amanti era un tale detto il Cornacchia. Coltui eras
uno con tal soprannome; perché havea la voce d' un fuono simile al gracchiare>
della cornacchia, ed cra un folennissimo briccone, ¢ ladro, € spia.. Questo daa
Bertinella il negozio per fatto, e s' ammannisce a far ja ite di Maimantile;
con scrivere ad un capo di ladri da strada suo corrispondente, che si conduca a.
Rimaggio con l¢ sue genti.con armi, ¢ panni, e 1 aspetti alla Suolta del Romi-
to, che @ una contrada in vicinanza di Malmantile. Bscgui )amico, giunfes
'con cento huomini ben' armati nel luogo ordinatogli: fra poco vi arrivd ancora
al Cornacchia con Bertinella, con grande schiera di bravi furbi, che questo in-
tende gente da bosco, e da riviera; che i Latin dissero homines omninm horarem
CLARLONE, Vno, che chiacchiera aflai, L'Autore intende, che chiacchie~
rava afiai alla giuftizia, cioé faceva la (pia, e percid detto Cornacchia, che ¢ uc
cello di cattivo augurio; perché il suo ciarlare era didanno al profimo. Ed in
vero cofivi, mentre vifie, fu fempre chiamato i} Cormacehia 5 o per questa.cauia,
© per quella che habbiamo accennato sopra, >
DIPINGERE alla macchia, Dipinger uo Ritratto fenz' haver d'avanti-l ori
ginaie yma col folo-haverlo veduro, £ 1' Autore pero intende 5 che egii era la-
dro di flrada., ¢ pigliando la voce macchta nei suo vero fenfo di selua denfa s dice,
che alla macchia ritraeva i viandanti cen lo frioppo, ed intende Afialtava la gente alia
strada con l'archibufo per rubarla, Questa però è finzione,perché ii Cornacchia,
se hebbe la malizia,non hebbe gid tanto cuore di far' il ladro di strada 5. ¢ l'Auto-
relo finge tale per moftrare 5 che egli era un furbo da far qualfivoglia sciagura~
taggine. 4;;:
tA nella iingua il suo in contanti.. Vuol dire eloquénte, pronto dilinguay.:
VANTeARST, Prometterfi molto di se: medesimo, Elakar-le propric-opere,
éil Latino Zatare. wij uy" 5584),, a
. GRACCHIARE, Cicalare con poco fondamento, Vedi forto C. 4. flan 2g.
C. 7. stan. 9, ¢ C. 8. stan. 65. Ma eoftui'é chiamato Cornacchia, il Poeta si
serve del verbo gracchiare per esprimer il cicalar di eflo. +
COLORIRE, Metafora assai usata, ¢ viol dire discorrer d'una cosa con ag-
 Biwfatezeay con termini proprj, ¢ con colori rettorici per perluadere, ¢ fare»
apparir

Kuiviil Cornacchin,e quedta bnonaspefa
Li Bertinella giunfero fra me '

J) ee a ee

   

 
PRIMO'CANTARE: 57

lapparir vera quella' tal cosa, della quale si discorre. ?

VIGLIETTO., obigiietto. Vuol dir lettera; Ma strettamente significa quella.
lettera;, chefirmanda in luoghi vicini, come da una casa all' altra, dentro alla
amedefima Città, o Terra, Voce che forse viene dal Francefe Poulet, che vuol dir
ettera,amorola, o1da Bidet, Vedi sotto C. 6. stan. 54.

BAG AG-LAIO.. Quelle fome, che si conducono appreffo gli eferciti per utile, ¢
«comodo dell'armata,.0 dictro qualfivoglia viaggiante per (eruizio della propria
periona.; si dicono Sagaglio, forse dal Francefe Bagage; 0 dal verbo Bainlares,
che val Portare, come habbiamo offeruato sopra in questo C. stan. 62. alla voce
Baule, ed è quel che i latini dicevano smpedimenta.

} AMMANNIRE, Meter' all'ordine, Allefire, approntare; quai dica 4d
manus babere. Dante Purg.'C, 23.
Di-quelich' il Ciel veloce loro ammanwa,
edalC.r9, Lavirtit yc! a ragion discorso ammanna,

| ALLA (pexzata., A pochi insieme per volta, non in (quadre o'teuppe Forma-
te. Sidice anche. dda sfilata, Vedi sotto C. 6.stan, 85. ¢d ¢ il diminutins dei latini,

PEi tragetts, (Per le balze, per luoghi., ¢ ttrade.non praticate; ¢ il puro La-
tino Traiectus.. ?

HAVER Vintefa Rit d' do.Haver l'inftruzione dicome si debba cé.
wT AR uno al fuoginoco, Chiamar' uno a fare una cosa, che sia di suo ge-
nio,.¢.gusto.d Latinidissero Adu/as hortari ur canant, ec,

. °BOCC HE: ds fusco.,. Intendiamo Ogni-arme da fuoco,:atta a portarsi addosso,
«come Mot iy archibuli,, piltole, ¢ simili,
BVONASpe/a.-Huomo aituto,¢ scaltrito,esuana lo stesso, che Trifto, e Vol-

 

ype vecchia.. '
STANZA LXXIL STANZA LXXIIT.
Dopo ch' insieme tutti fur coflore to foc' aun ignorante, aun idiora
. Si fece de' pi degni una femblea, L' esser il primo a favellar non toca;
Del. come difeorrendo fra di loro 41a perdonate aqueftarucca vora,
Sorprender' il Capello fidovea-, Signori, s'io-vi-rompol'hvovo in bocca;
Ond'it Cornacobia in mexzo al coviffora Scricchiolafempre la pitt trifha ruota,
| « iRixzaro'in più con gran prefopopea, Così la lingua mia più-rorza, ¢ feiocca
} Ed'una toccatina di cappello y V infaftidisce,¢-ver ma v afficura
| » dn.eal modo cave fuora il limbello.. Che Adalmantilee nostro a dirittura.,
i coftoro insieme, quei più degni si riftrinfero a contiglio, per fermar
al y he si doveva:tener per forprender Malmantile, ed il Cornacchia, fac-

te suecirimonie, comincia a moftrare il.modo certo di pigliare detto Malmantile.
PRESOPOPE A. 'Questa voce, che vien dal Greco Profopopea compostasdi
due dizioniPrefopon, che fuona perforam (ed a noi Personaggio ) ¢ jpoceo., che
fuona facio, feibene ¢ una figura con la quale fingefi un perlonaggio, come fareb-
be introdurrewna cola inanimata., che-parlicon una animata., & contra, tut-
-tavia noice:ne serviamo per intender'una'certa superbia., -arroganza,, 'fasto., o
refunzione di se medesimo., dimottrata con gli atti; diche vedi sorto C.6. stan.
5. Ed in'tal fenfo., fecondo il. Monofino era pigliata ancora da.i Greci... Si dice
da noi anche fufliego-, derivando la voce dallo Spagnuolo.
: VNAtoccarina ds cappello.Atto che e(prime detta Profopopea. =H Cate

   
58, MALMANTILE

CAPO! fuora il limbello. Comincid a parlare. Limbelli; Si dicono quei pezzi
di pelle di beftia, che dalle dette pelli tagliano i Conciatori', donde poi démbel-
/ucci i ritapli delle pelli più: fortili, come di cartapecora, che servono'per far col-
la da Pittori. E perché tali /imbel##, quando fon freschi; ed umidi sono simili-alle
lingue, percid per /imbelfo'intendiamo lingua; ¢ però detto scherzofo, come si
vede, che !'usd il nostro Autore anche sopra'in quella sua lettera alla Sereniss,
Arciduchessa, riportata da me nel Procmio. Cave fuora il limbello, ¢ disse le sue»
Siliabe, come un Tullio, ec,

IGNORANTE, & fdiota, Sono Sinonimi, ne vi si fa alcuna differenza', se»
bene strettamente /gvorante vuol dire uno, che non sa nulla, ¢ /diora par che si
conuenga a coloro, ch¢ non hanno cognizione di lettere. -.

ZV CCa, S intende il capo del' huomo per la similitudine |, ¢ Zxcca vera vuol
pe dire testa senza cerucilo, 'che si dice vera di fale, 0 non baver fale in zucca.

questo perché ¢ solito nelle cucine tenere il fale in'una Zucca fecca-appefa'al
muro del Cammino. Vedi sotto Can. 4. stan. 15. 1 Latini pure dicevano fale per
giudizio, ¢ trovafi in Catullo,: bake

Walla in tam magno corpore mica falis
Vedi sotto C. 8. stan. 26., ¢ Marziale C. 7.
Nullaque mica falis, nec amari fellis in ilfis

ROMPER t hvovo in bocca. Torre la parola di bocca a uno, cid ¢ Dire 'che
doveva, o voleva dire un' altro. Terenzio'disse Bulus ereprus ¢ fancibus eff.

SCRICCHIOLARE. Stridere, strepitare « 8" intende quel romore, che fas
nel muoversi un legno fortemente stretto, o aggravato da altro legno, o mate-
riale duro; come appunto segue nelle ruote da carro. Ed il proverbio: Sempre
Sericchiola la peegio rugts del carro, Significa 7/ pitt fetocco della conversaxione, vnol
Jempre parlare, Detto antico, ¢ vien dal Latino, che dice femper deterior vebicu-
4 rota per ferepit, ec. R

 

A DIRITTVRA, Civ' affolutamente, ficuramente, e senza difficulta aleuna,
s

STANZA LXXIV.
Credete a me: Ciascun si lia nascofto
Fn quefke macchie,in quofti boschi intorne
Ed io da vor fra tanto mi difeofto,
We questa notte faro più ritorno.
Rinedremci cola doman ful posto,
Perché vicino aj tramontar del giorna
Vi faro cenna, bor voi ponete mente,
E poi venite via allegramente
STANZA LXXV.
Parte il Cornacchia, ¢ corre prefto prefia
Da corti suci ami epithet 5
Dal qualt le lor beftie piglia ix prefto
2 covtch pla foes bona oh,
E di [opptatto, come fante lefta
Cave di tasca certi cartoccini
Preni a' alloppio, e dentro al vin li pone
Quelle impepando, senza discrerione.

   

TANZA LXXKVL
Cosh carreggia, ¢ gitmto a Malmantile
All' aprir della porta la mattina
Scarica in piazza il vino, ed ut barile
A regalar ng manda alla Regina,
Poi vende il resto a prezzo tanto wile y
Cognit necopra,e in he che n'haincatina
Per rivenderlo altrui, il fiasco attacca,
Si cala al buon mercatoya quella macche
STANZA LXXVIL
Due, 0 tre fiaschi davane a quattrinoy
Ed a' poveri davalo a Lonne 5
Tal che tutti suffandofia quel vino
S' imbriacaron come tante monne 5
E subito dal grande al piccaling
delle dine

Tanto de gli huamin, "
Cascaro in fannolenga si gagliarda
Che defti non Guacilleane bebarde:

 

1

 
 

 

PRIMO CANTARE, 59

= Cornacchia instruisce i compagni di quello devon fare, ¢ si parte, ¢ va da,
certi contadini suoi amici, da' quali piglia le lor beftie in prefto, ¢ Ie carica di
vino alloppiato, quale porta in Malmantile, ¢ Jo vende così a buon mercato, che
Ognuno ne comprd,e bevvero tanco, che tucci s' imbriacarono,¢ si meffero a dormire
~ PRESTO prefto. Prettitfimo: per la replica d' una stessa parola, che ha forza di
superlativo, come habbiamo detco altrove,.

Di foppiatto. Di nascofta, Vien dal verbo impiattare, che vuol dir Nascon-
dere unacofa corporea, coine s' ¢ detto altrove,

FANTE lefo.. Huom (agace, aftato, ¢ che fail conto suo.

CARTOCCINO. Diminutivo di Cartoccio, che ¢ una piegatura di foglio,fatca

a Piramide usata da gli speziali per metterui détro zucchero, pepe,ed altro simile,
ALLOP PIO. Specie di fonnifero composto di fago di papavero, coagulato,
fecco ye poluerizzato, ¢ d' altri ingredienti; ¢ si chiama oppie.
<CARREGGIARE, Venendo da carro dourebbe intenderfi folamente per Cam-
minar col carro, o traghectar robe col carros»ma ci (erue per lo pill per inten-
der ogni forte d'andare, 0 camminare, a picde, 0 a cavalle, conduceado o non

BARILE, Vato di legno per ulo di portarui olio, vino, ed ogai altro liquo-
re simile, ed è la mifura comune del vino, capace di 20. sia(chi, ¢ quello da olio
di 16, fiaschi. Tali vafi fon composti, ed aggi(tati ia maniera da adattarne duc
per volea addosso a una beltia da (oma. '
~ ATT ACC Ail fiasco, Coloro, i quali in Firenze vendono il vino a fiaschi alla

“propria casa y-attaccano per (egno di cid sopr' alla porta un fiasco, acciò che il
ib luogo', dove si veade il vino: ¢ pero quando si dice H/ tale ha oggi
attaccato il fiasco, s' intende, il Tale oegi ha cominciato a vendere il vino a fiaschi.

Sscala.abuon mercato, Si lascia persuadere dal prezzo vile a comprarac. B'
traslato da gli uccelli, che si calano alla vifta della preda.

MACC.A, Abbondanza grande. Vien. forse dal Latino Adadus, che Sinten-
de abbondanza grande, quafieAsagis an'tus. Plau, milit, 4.22. Adete amare. B
si trova Pxer matte nirewe; giovanetto virtuoGiimo. Dice il Vocabolifta Bolo-
gnefe, che macco vuol dir' abbondanza y che induce disprezo, e così ¢ vero acl

~parlar nostro, che si dice smaccare per ittender Vituperare, o fereditare.

A lfonne', Pee niente. Senza spefa, EB detto plebeo, ed è usato per lo pik tra
i battilani, i quali hanno per tradizione, che [fonne falle già un' huomo de'loro,
il quale mangiava tanto volenticri a spele d' altri, che essendo morto, ¢ feppelli-
to già di qualche mefe, scappafic dell' avello al discorso, che da alcuni si faceva
di-voler dar mangiare a tutti i Battilani per tre giorni, senza che spendeflero,
Coftui havea due fratelli ' uno detto Salicone, ¢ l'altro lo Scrocchina y € però
feroccare mangiare a Salicone, a Scrocco, ¢ a donne significano tutti Mangiar fea-
za spendere, che Tx io ditle edsymb poslo dalla proposizione A, che
fuoaa Senza,¢ symbolum, che vale ae » 0 scorto, ¢ significa (enza denari; & si
come ne i Latini quelto Arymbo/um,tu usato da i paratliti,c guatteri, così il aoitco
donne, ¢ usato dalla plebaglia, fra la quale è nato.

Pud anch' essere, che questo detto donne venga da un Iiogo poco fuori di Si-
renze detto //onne, dove anticamente andavano a definare aicune volte pate
2 Ha

 
60 MALMANTILE

,
molti battilani, senza spendere, non perché veramente non spendefleros:ma fier-
ché il denaro, che si (pendeva in quel definare, era di mance este perile Pa

5. Giovanni', ¢ Carnevale, che mefioin una lor corbona, si ferbava\, ¢ distri-
baiva per — definari; ¢ pud eflere, che questi battilani.deffero tal nome»

Lfonne a que

luogo dove andavano a far questi lor definari., chiamati da loro.de~

Jfinari a Monne; ma sia come si voglia, basta che appreflo noi il termine Sonne è

inteso per Senza spefa.

TVFF.ANDOS/, Tuffarsi a una cosa,significa Pigliare,o fare assai una tal cova.

S'imbriacaron come tante monne.V edi quel ches} è detto sopra in

STANZA LXXVIIL
Quando il Cornacchia vedde ilsuo aifegne
Già riuscito, ando sopr' alle mura,

efto C-stan.i0.
TANZA LxxIx.
E perc' ognun dormiva, come un Talos
La donna fece farne.una funata,

Ed ai compagni fece il detto fegno, E condurfegli apiedi a baciar baffo.y
Che bene havendo al tutto poste cura y E renderleil tribato ognun prorata y
Saliro al poggio fenz' alcun ritegno 04 Celidora:pa: reftarain Naffos

Senza Sospetto haver, senza paura
Dietro alc ornacchia lor guidonese scorta

Cive da' suci vaffalli rinnegata:y
Già che tutti voltarohavean mantelloy

 

Dentro al Castelloentraron per la miners Comando che baciaffelil chiavifeelle.
STANZA LKXX:
Ell ubbidi, cemendo, ancor di.peegio., Coss finito il solito corteggio Shi
E ben che fuffe un pexxo in la di nottey Con due strabellijeun par di fearpe rotte
MI pigliarfene subito il pulezgio ——  Triffa,¢ firascina poi perila boccolica

Vn xucchera le parue di tre cotte. Vin 10°20 mendicava all accattolicn s)
1:Compagni di Bertinella yeduto il fegno dato dal Cornacchia, andatono a»
Malmantile, ed entrati dentro, ¢ trovati tutti a dormire gli legarono,.¢-gli con-
duficro a render ubbidienza a Bertinella,la quale comandd a Celidora, che nicif>
se del Castello, ed ella mutta mai' all' ordine se n' ando, benché futile afflai di not-
te, ¢ si condufe a.mendicare il vitto
GVIDONE, ¢ feorta, Guidone s' intende Colui che guida; ¢ Scortaé quello che
moftra laftrada; ma la voce Guidove & forse per scherzo prefa dall? Autore:nel
fenfo, chefopra stan. 65. ¢ sotto.al Cant, 8. stan. 72.:
FAR una funata. Legat con una fute più persone: Quando molti infiemes
commettono un delitto., si suol dire: Se vengono ibirri 5 vogiion far la bella funata,
Non perché crediamo, che vogliano effettivamente Iegargli tucti a una funes ma
intendiamo, Vegliono farne molti prigiont, ¢ così intend) nel presente luogo ~
BACAR baffe. Cioé inchinarfia baciar i piedi in fegno di vaflaliaggio «
RIMANERE in Nall. Dai più si dice rimanere im Ajso,e cid segue per corru-
zione nella pronunzia., che tanto fuona rimanere in afso che rimsancre in Wafso
come si dourebbe dire,¢ significa abbandonato., senza-aiuto, ¢ senza, consiglio;
Ed è derivato dalla fayola d' Arianna abbandonata da'Lesco nell' J/oladi Natio;
E si dice anche rimanere in [u le fecche di Barberia, il che corrobora che si debbas
dire i WNafo, ¢-non in allo che non ha verua fenfo, o allegoria. Vedi fosto C.
10. stan, 2. i"
VOLT-AR mantello. Rinnegare. Ribellarsi; andar da.un partito all altro. Il
Lalli En, trav. C. 2, stan. 39.. 3 3 sun
r

 
or Clee 7 wae

PRIMO CANTARE: or
+ i Flor che mi lice divoltar mantelo, '

. BACTARE il chiavifeello. Andarfene senza speranza di tornare.. Vfiamo que-
flo detto per esprimere che non si vuole, che quel tale, che è stato per li suof
mali portamenti scacciato d' una tal casa, viva con la speranza di ritornarui, ¢
pero si potrebbe dir con Vergilio Supremum vale.dixit,:

CHIAVISTELLO. Scrratuta da porte, © fineltre, che confifte in un ferro
lungo, il quale fa la sua operazione, paflando per diversi anelli pur di ferro
adatcati nel legname; ed ¢ il Latino veitis.

PIGLLAR i paleggio:,. Andar via. Pigliar il cammino, E' frafe marinaresca, ma
pero usata comunemente in quefii termini d' andar via prefto. Dante Par. C. 23.

Non ¢ puleggio da prccola barca
Quel che fendendava t ardita prora
We da nocchier, ¢ a se medefmo parca, «

Da questa voce Puleggio viene /pulezzare, che vedeemo sotto C, 7. stan. 18, che
pure significa Andar yia.. Forse si potrebbe dir anche prueggiare verbo pur mari-
; nare(co, che significa Andar via bel bello.

i Vincenzio Tanara nella sua Economia del Cittadino in villa Lib..6. tratando
y dell' erba 'Px/eggio dice »che sparfa in luogo dove fieno pulct ha virtù di scac-
; ciarle; onde puo essere che da questo effetto.dell' erba Pu/eggio venga il presente>
i dettata,, Da pulegeia forte anche vengono Pulegge., che sono quelle piccole girel-
' le.,.che si congegnano,ne.ilegni per facilitare i veicoli, come farebbe dentro a i
regal da piede alie (cene, 0 propane da commedie.per'renderle pili facili a+

Atralcicarsi dentro.a,i\canali in occasione di mutazione delle medesime scene.
AN suechera le parne di tre cotte. Le parve d' haverla a buon mercato: le par-
ve d' haver fortuna grandilfima., perché s'aspettava malto peggio.. Lo Zucche-
ro di tre cotte fatte bene si stima che sia aj miglior grado di perfezione, della.
ae sono trei gradi. fecondo il detto omne trinum eff perfetum. Bd i Franzzfi
lenominano il superlativo col tre, clot buono, for buono, ¢ tre buono, per

buona, molto buono,.¢ buonissimo,:

STRAMBELLE, Vehi yecchic,¢ stracciate.. Vedi forto C, 3., stan. 65.»
¥N tox20,Detco così affolutamente fenz' altra aggiuntavuol dire un pezzo di pa-
Ne. E frufiumpanis, che usd Dante nel Parad, C, 6.AMdendicanda/navicaafrufaafrufeo,
TRIST A, ¢ Strascina « Huomo.trifto vuol dire Huomo mal yeltito,¢ Stra/cino
fuona quafi lo fieflo, perché, Stra/cini.chiamiamoaicun: huomini, i quali vanno
comprando carne fuori delia Città, el' introducono in.Firenze occulcamente pr
deans ¢ perché coftoro fon fempre uati, fudici, ¢ stracciati, per-

cid dic Strascine intendiamo. mal' all' ordine di veltico, ec, ¥
 BOCCOLK a5 ¢.accattolica. Sono due parole dette per icherzo, ¢ per 1a fimi-
litudine che hanno con Boccay¢ con,Accattare, ¢ per pariare lanadattico;, non
sono però fuori dell' uso della gente pitt Civile y la quale speflo si serve di parole
ating a quel proposito., che le pare: che facciano. giuoco stroppiandole, ¢ inter-
pretandole a lor modo,come le presenti Boecolica,e accattolica che l'una vuol dir Boc-
a, ¢ l'altra Accattare,e così intendefi che Celidora accattava per mangiare. Tal'
uso d' allufione (cherzofa ra pur'anche appreflo.ai Latini trovandofi 42 ilie nan-
 Gham recedis, che par ch¢ voglia dire w non ti parti mai dalla Cua di Troia, ¢
LOX s'in-

es

 

ae
e MALMANTYILE®! ©

s' intende poi; tu hon abbandoni mai I" Ilo intestino', cioè (empré mangi.
MENDIC ARE, Vuol dire durat fatica'a conseguire. tale mendica le parole,
'cioe Dura fatica a parlare; ma il suo significato pili intelo ¢ Chiedere elemofina,
Dante Parad. C. 6,
Indi partiffi povéro, e'vetufto,
E 8 il mondo fapeffe il cor ch' eghi hebbe 5
Mendicando [ua vita a frafto a frusto, ec.
STANZA LXXXE

Ih tanto Bertinella del Reame
'Garbatamente fecefi padi ona 5
E de' villaggi, ¢ d' ogni [uo beftiame
Prefe tl posseffo in petto, ed in persona
. STANZA LXKXAIL
Toff che ci hebbe fitto il capo, volle
C' ognun ferraffe ul rraffico,e il negorie,
'Donando a ciascheduno entrate,e rolley
. Aecio se ta palfaffe da'buon forio,
Ed allegro, 4 più pari, ed in panciolle

Poi per letizia cavalieri, ¢ dame
Regalo di confetti,'¢ di pattona;
E segueogn' anno di mandarne attorno,
Per la dolee memoria di quel giorno,
STANZA LXXAlill,
Cos} mai fempre in fefte, ed in tonuaito
Tirano innangyi questi [pensierati;
Neinsobelbes per ao axe lun dite',
Ben ch' eh credesson a' esser impiceari;
Won treme della Corte, chi ¢ fallitoy

 

 

Senzabriga vivele m pace, ¢ in oxio y Che tutti i giorns a lor fon feriati;
Ognun vi s' arvecd ds buona gana, Non v'e ginffizia,neilbargel vafuora,
Che la poca fatica a tutti? fana, Se non aftigur chiunche lavora,

&
Sbandita Celidora dal regno, Bertinella prefe ! attwal posseffo di tutto lo fla
to 5 ¢ per acquiftarsi la b x de? fudditi comincid dal regalare le dame;'€
cavalieri, con regali degni della vilidima condizione di se medesima, ¢d appro-
priati alle qualita de' Cavalieri 5 e Dame di Malmantile; poi con fefte, ed alle-
rie per contentare il popolose con levare i Mini(tri della giuftizia tanto odiofi al-
da plebaglia, e con fare altri ordini che si leggono nelle presenti ottave.
AN petto, ed in persona, Attualineate, € corporalmente. -Arimo & z
PATTONA, ees > 0 pane fatto di farina di castagne, con altro nomes
detto polenda y dal Latino Po/enta, che era vivanda fatta di farina d' orzo cons
altre polveri odorifere fecondo Varrone. E' vivanda vilissima appreffo di noi; €
'da questa sua vilra habbiamo un detto di disprezzo y che; Afangeapatrona;
Mangiapolenda a un huomo vile 5'¢ buono, a poco. Qual detto usd Plauto chia-
mando questi tali P/tsphagj; ma il disprezzo non nasceva dalla vilta della polenta,
(che era finalmente il cibo comune anche per le persone di garbo,¢generalmen-
te mangiando questa forte vivanda i Romani viflero lingo tempo, Vedi Plin.
lib. 18. cap. 8. ) nasceva bene dail' intenderfi con tal detto un huomo buon'a,
poc' altro, che a mangiare 5 ¢ come noi diciamo Sparapani; Voramadi¢,  timili.
V bebbe fitto il capo, Sen' era'impadronita+ N' haveva prefo l' actual posletio;
perché eflendo il capo la pil nobile-;¢ eae parte della persona ', noi dicia-
Mo Ficcare il capo in kn logo per intendere Entrare in un luogoy ¢ pigliarne il
poslefio personalmente. nh
TRAFFICO, ¢ negoxio. Sinonimi, se bene trafic par, che si riftringa all' ar-
ti manuali; onde con dire 77 » ¢ negoxio invende non layorare y me mercan-
teggiare, 0 negosiare, è ” Bok

 

 
 

 

PRIM O/C AIN T/ARE:? 'e

ZOLLA. E il Latino gleba y che) vuol dire Pezzo, o maffa di terta smoffa,
come s' è accennato sopra in questo C, stan. 57.,ma qui pigliando la parte per il
tutto, intende terrent fructiferi: 11 e4/eha'delle zoie', comuncmente s' intende»
Ha de' terreni. '. x

SOZl0.. Dal latino Socius. Compagno.. iver da buon foro vuol dir Viver
da buon compagno, alla reale, ed alla (chietta. E questa voce Sozio non fa che
sia usara se non in questo cafo, ¢ con l'aggiunta di bueno, O male: dicendofi 7
tale ¢ buon foxia, 0 non è mal foxio, per intendere. EB' galant: huomo.

A più pari, ed in panciolle.. 3' ula quelto detto per esprimere Vn» huomo pol-
trone, che non voglia far' altco, che godere i suoi camodi, ¢ la voce pancio/les
€ composta di due parole, ciaé pancia, ed ol/e, ¢ (uona pancia di pentola, la quale
col posar pari » ¢ con quella sua gran pancia ¢ il vero ritratto della: comodita, ¢
poltroneria. 11 Bronz. nel Cap, in lode della Galea dice.

Guari, ma in capo al gingco, come valle
 ACiela, ne fu tracto il poverino y
E fui privato di fare in panciolle,

BRIGA, Noia,fattidio,fatica. Qui ¢ oe per faccenda,o pensiero d'operare.

D1 buona gana. Molto volenticri. E detto spagnuolo,e la voce gana é'ulata da
noi per intender Voglia,o gutto grande. Mi tale mangia di gana; Lavora di gana;ec,

' SCIOPERATO. Vino che non ha y¢ non vuole haver faccende. Vedi sopra,
F stan. 29. Scioperati s' intcndono quei Cittadini y che senza arte, o impiega vivono
on le loro entrate.
» CORTE} Antendi la Corte della giuftizia da i Latini derra Curia a differenzas
dt Anta;e wuol dire Miniftri della ginftizia,;
FALLITO. Vino che negoziando ha fatto così gran debito, che nan ha pos-
fibilita di pagarlo. E il latino decu'tus,qui fallit creditores,ip/umaue fefellere negacia
TVTTA i giorni fon feriati, Sempre è felta per loro; Fertaro s'iintende-quel gior-
no, nel Ze ancor che lavorativo non Gi tien da i Magiftrati ragione, e non &
possono fare esecuzioni civili contra a i debitari, e questo intende dicendo Now

Nile aed

SS,

teme della corte, chi è falco, perché è feriato, ¢ non pud esser menato prigione,.
STANZA Laxxlv,;
Ua 2 ia non erro il tempo € gid vicino, Così panna [ard di Cafentina,
Che n' ha a venir ia piena de distyrhi, Ne se lamenti alcuno, 0 si sconturbi;
Mentre daman per far un buon borting Che chs nuoce al copagnoin fatti,oin detti
<Andremo a dar' fo a questi furbi., Deve faper che y Chila fa l'asperti,

Baldone, havendo fatto il detto raccanto della cacciata di Celidora, dice (pe-
rare, che sia vicino iltempo, nel quale faranno gaftigati coloro, che hanno sor-
» prefo Malmantile, perché il giorno futuro vuol! andare a dar ioro addosso.
' HA da venir la piena de' diffurbs. Ha da venir grandissima quantita di disgufti a
t flurbare i loro commadi, E Piena diciamo quando Arno, 0 altro Fiume cresces
per le pioggie. '<
\ SARA' panno di Cafentino. Cafentino & una Regione in Toscana, doye si fab- ee
-hrica una specie di Pa » che bagnati scemano di Iungheaza, ¢ larghezza per-;
ché rientrano. E da questo detto /ard panno diCafentino, intendiamo Rientrera,
cide tu hai fatto a me questo, ed io tard a te il simile, coe Mi vendicherd.
2 cHl

   
 

64 MALMYNTILE *
CHI la fa? aspetti, Chi favun torto al compagno, alperti pure'd' eine contrac.
cambiato. Il Petr. disse;
Chi si prende diletto di far froile,
Won si dee lamentar 8 altri? inganna,
E questi due verli possou servire per dichiatazione delli quattro uti della,

 

pre(ente otrava.
STANZA: LXXXV.
Qui racque il Duca; € subito rarcacca,
Col dire alla cuginain voce bafsa, ca)
Che. sperch'egli hit laboccaafeintra,e frac
Hi fuggiunger a tei qualcofa lafja
Non ho che din( gli rispond'ella)un bacca,
Ottre che la Sarebbe carne grafsa,
Di pits tofto, in che mo noi fiam parenti,

STANZA LXXX¥I1.

Ed jo che non nebo gran cognizione 5
E fempre me ne sono fhata avderra. \
(Che tutta la mia gente ando al caffone,
Come tu sai cb io ero fancinlletta:)
T' udira volentieri.» Allor Baldone
Soggiunfe: Or or ti ferno,e ararafretta,
Perché non gli morialatingua in bocca y

ta dailo Cunto degli'Cunti di Gianalefio Abbattutis

Ch' io nor: paia a coftor de gl Innocenti; Ricomincio quef? altra filaftrocca.

Baldone termina il discorso, ¢ volto.a'Celidora le dicey che ella foggiunga.,
se ha di più; edefia dicendo, 'che non ha che foggiugnere lo prega a narrare, in
che modo fieno parenti: E Baidone's' accinge a contentarla.. Equi termina il
nostro Poeta il suo ee.

NON ho che dire un hacca, 1! H vogliono, che non sia lettera\, ma femplices
aspirazione, ¢ pero.dicendofi Avon ho che dire un haccas 10 ficfio cche'dire: Vox
ho che dir nulla,

SAREBBE carne grafsa.. Stuccherei il popoloy; Mi fendeiel odiofa, I) Lasca
Nov. q.dice 2 £ poi io non vorrei anche tanto inf apbidirto, che exlim'. bave/ie wdire,
che io fufi carne grafsa. La carne grafia fuole ai. pil che la mangiano cagionare
naule a; il che diciamo stuccare.

CH io mene 4 coftor de gl' Innocenti., Che coflaro'non 'pensino, the io fa.
batarda,-o senza parenti. In Firenze lo spedale-de gl' Innocenti si 'chiama quel-
Jo, nel quale si mettono.ad allevarei bambini\y per Jo pi, nati di-congiunzioni
illecite, i quali corrottamente chiamiamo Woceurini.. Vedi Totto Cant,vooftan. 7.

ME ne sono feata a detta (Non ho cercato di fapernepib 4a; macho creduto ge
che m''è stato détto, o raccontato..

LA mia gente ando al-caffone. -Mio padre, mia\madre;'¢ tuttingli aaleti miei pa-
renti morirono;che per mia gente in quefio luogo',) i in at termini's'' inten-
de Mici parenti, ¢ 'non altri.

et tanta fretta. Subito, Preftissimo.,

NON gli moria la lingua in bocca, Era loquace; cloquente. Havea facies ss
a 'Elo stessoiche Havere il /uoin minioenaiana Sige 'come s-accennd fo-

ra bE
2 VLA RODE Serie di parole; e per lo più Hinténiestumciistartocrtalls
ordinato, e proprio del racconto,'che talora fanno le ae a' Fangiulliyis
le lor'novelle, 'come appunito & quella che narra Baldone, che ae
haverla sentita 'forse ueseneneenie sue donne,' A a cra fat

 

 

\section*{FINE DEL PRIMO CANTARE}
\end{document}

 

om

 
 

i

q

 

 

ARGOMENTO. i
De i due gran figli del Signor d' Vgnano
Prodigiofo il natal narra Baldone;

Gomes acquifta moglie Flortano,:
E vien dak' Orco poi fatto Pprigione.

Seth at

Come Amaaigi livera il germano;

E il moftra [paventofo a terra pone y
E dice al fin, che' un di questi dui
Fu padre 4Celidora, el altro a lui',

2
Sn py ngt

STANZA I.

E Ra in Venano il Duca Peridne, We per altro era tutta bacchettone,

Bu Che fempr' all' Altarin fidecommiffo Che per un suo pensiero ecerno,e fifa
Faveva notte, edi tanta orurione, D'haver prole, perché deliafuaschiatta
E tanze carita, ch era un fubbiffo + Won v' era, morto lui, ne can, ne gatta,

Ui. Duca Baidone da:principio-alla narrativa del parentado., che pafia fra lui,
¢ Celidora, come havea promedfo neil' antecedente Cantare ye dice; Che fa già
in Vgnano il Duca Perione, il quale faceva mole opere pic per disporre il Ciclo
@,-<oncedergl) prole, La favola del nascimento di queti figtiuoli trovafi-nello
Cunto degli Cunti-diGianalefio Abbattutis Giorn, 1, Cunto 9. ll nostro Poetas
pero non la cavo di quivi; ma la narro, come l'haveva fenuica contare alle sue
donne, quando era tanciullo; ¢ guelto è certo, perché queita era nel suo primo
Trem fatto molto prima., che ii Bafile Autore dello Cunto-de li Cunti la stam-
pafie, si Tht
ALT ARINO, Così chiamiamo un' inginocchiatoio a foggia d' altare, iliqua-
Ieper Jo più G tiene allato.al letto.p inginocchiarfije fare orazione.

. STAR fidecommifso in un luogo, & detto iperbolico, che significa Star molzissimo.
in un Juogo; che qui vuol dire Stava fempre 0 non si Jevava mai dail' Altarino;
che s'intende faceva orazioni infinite. ' rs

TANTE carita ch era-un fubsfvo,, Carita » ed clemofine infinite. Per denotare
Una quantita indicibile usiamo dire: Son rantivche e un fubsfso, un fracasso, un fla-
Selo, ¢ Gmili. Quelta.voce Subbifso vien pre dal Greco aby/sos,, che significa vo~

ragine

 

 

 
 

*

ee

66 MALMANTILE

ragine, o fmifurata profondita'd' acquie s'come fuond ancora nel foftfo idioma;
donde /ubifsare Andar nel profondo, quafi dica /ub aby/so.;

BACCHETTONT, Così chiamiamo.noi certi colli torti, ¢ grafhafanti, che
flimano peccato il portare un fiore in mano, ¢ credono poi di far' un' atto me.
ritorio a dare a usura; con aitro nome chiamati Ipocriti, cio¢ Pfeudobeati; huo-
mini'da bene pér interefle, e*per gabbare il cofipagno; ¢sono infomma cploro,
de' quali Giovenale disse: Qui Curios fintulant, & Bacchawalia dinunt, E diciamo
Bacchertone, quali Va chetone, perché quetta Canaglia, che fudia di fimuiare la.
bontà, per arrivare a suoi fini,è simile all' acque profonde, che vanho chete.,
delle quali patlandé Q. Curzio dice: Altissima queque jlumina minime labuntur sono,
E come queste acque fon fempre di pericolo, così li bacchertoni nella loro taciwur-
nita occulcano il malo animo, che hanno contro al proffimo. 11 coftume di co-
storo tocea Orazio Jib. 1. Ep. 17. dicendo che fon devoti di Laverna Dea de»
ladri.:

Labra movens; metuens andirs; Pulchra Laverna,
. Da mihi fallere; da influm, fanttumque videri.;

Di quelta voce Baccherroné si serve anche il Tationi nella sua Secchia. Mimico
natural de' Bacchettoni, Ed un dottissimo de' nostri tempi, il quale fa un
discorso poetico sopra a coftoro, lo termina con dire Furfanre, © bacthetton /uuna
il medesimo, Vedi sorto C. 6. stan. 97. dove si dice eller lo Melo Bacchertoni, che
Jpocriti, i quali S. Matteo chiamd similes fepulchris dealbatis; i) Berni nel'Orlando
disse. O agghiacciati dentro, è di fuor caldi; In fepoleri dipinti gente morta,.

Giovenale aggiunge al detto di sopra.

Fronti nulla fides; quis enim rion vicus abundat
Triftibus obscoms ? castigas turpia, cum fis
Amer Socraticos notissima fofsa Cinnedos.

Di quefii tali parla in diversi tuoghi la Sacra Scrittura deteftando tal vizio, eo-
ine abominevole, ma per brevita tralascio di riportarlo, contentandomi di chiu-
dere col detto dell' Evangelilta Atendite a falfis propheris, qui veniunt in veftimentis
ovinm, intrinfecus vero funt lpi rapaces = ¢ rimetter il Lettore a quello, che (crive
S, Matteo Euangelifta al Cap. 6. 15.23. -

Tale era appunto questo Perione, che faceva le dette Opere pic, non perché
veramente futic buono, ma perché con efle pretendeva d! eftorcer dal Cielo las
grazia d' haver figliuoli. 4

SCALATT A, Stirpe y Profapia, famiglia.

NON v' era, we can ne carta, Non vi rimaneva pur' uno. Plauto disse: Wve
weisca quidem domi eff, Del qual detto si servi quel servo dell' Imperator Domi*
ziano che domandato; se Domiziano cra folo in camera, rispole: We mufeas
quidem eff, Percht Domiziano stava la dentro ammazzando le mosche. Ter.
dit: Ve Sannione-quidem relitto.

 oaeteegel at 'STANZA-I. susie
Così divs gran tempo, ma dacerto y. E quanv ti far yn in dispr
» Vedendo cli ei ton era efandito Senza voler pik dar del pr 5
Essendo omai con gli anniin la. un pereo, Gertarofi al! avaro, ed al furfante
A mangiar comincid del pan pentito 5 'Cambio la diadema in wren:
3 OR-

 

 
> at eee hh

 

ey. ee a ree

 


SECONDO CANTARE, 7

Continud gran tempo Perione a far-le narrate opere pic, ma yeduto ch'ci non
¢ra efaudito, ¢ ch' ci non haveva figlivoli, ¢ trovandofi già vecchio, percht ve-
ramente egli era un di quei Bacchettoni furbi, che habbiamo detto sopra, ¢ che
faceva bene folamente per interefie, si penti d' haver fatto tante elemofinc, ed
altro bene, ¢ muto coftume.

DA 7exx9. Da ultimo. Forse meglio /exo, venendo dal Latino /ecius opposto
dj ocius. Vedi foto C. 4. stan. 72.
| ESSENDO un peo in id con gli anni, Essendo grave d' eta. Havendo molti
| anni. Vedi sotto C, 12, stan. 36.
be MANGLAR del pan pentuo, Cioè si duole, si pente d' haver fatto del bene; ed
.

.

& quel fatti penitere di Cicerone,

POST O in disprexzo quanto far folea. Cie lasciando fare di fare clemofine, ¢
Qrazioni, ed altre opere pie come folea fare.

SENZA voler dar del profferizo. Senza yoler dare pili niente; ¢ ne meno quel-
lo, che havea. prometio, o proferta,

GETT ATOS/ all' avaro... Divenuto.ayaro per clezione, o diremmo A postas.

FVRF ANTE, Vuol dir furbo scellerato, ¢ ladro, ¢ simili venendo dal latino
barbaro foris factens, operante fuori del dovere, ma si piglia anche per Spilorcio,
ed avaro, come ¢ pre(o nel presente luogo.
 CAUBIO' la diadema in un turbayte. Di Santo divenne Turco, che Diadema
apprefio di noi vuol dire quell' ornamento, 9 corana di splendori, che si vedes
dipinto attorno alla testa de' Sa nti. Dice che cambio la diadema, che meritavas

\ come Santo, in un turbance, cioé cappello da Turco, non che veramente si met-

tefse il Turbante;.maintende, che d' huomo da bene-diventd qutto il contrario.

hg STANZA
Di poi tutto diverso ye ma! disposto La moglie un miglio si tenea discofto,
 da modo degli Dei faceafi beffe, E dow' ci dava a' poveri a bizreffe y
| Che segli udia trattarne,bauria pik toa Quando picchiavan poi dalla finefira y
 Valuco ful moftaccio.uno sherlefe; Facea lor dar il pan con la balefra,

gees Perione tutto diverso da quel che era, come sé detto,,comincid
anche a non stimar pil gli Dei, anzi g i Arapazava in modo, che haurebbe vor
Juco più tofto un, sfregio (ul vifo, che sentirgli nominare; sbandi la moglic, ed in
x t limofine a i poveri gli baftonava. ay

2 RECERSO 9 -differente da quel ch' cra prima. Se benquefta yoce diver/o
significa ancora strayagante.. Vedi sotto C, 8. stan, 17. ed in questo fen(o 1a pigla
Franco Sacchetti Nou. 29, £ quc/ta natura pare a me, che fufse delle firane, ¢ dix
verse che trovar si pore/sero.,. B Noy. 78; Ed era wn! buome maliziofo y reo, ¢ di di-

aii iy ey sama | tar iogs 5. se td ¢
FACE AST befe. Si burlava. Non faceva stima. E il latino flocei facere.

SBERLEFFE...Taglio,. 0 sfregio', che i Latini dissero frigma; Rigido signara
 Htigmate frome. E. perché gli iin ful yifo sono cosa ignominiofa, come s\¢

ieee fears ch stan, fs cid si deduce fhe Perione hanria pie toll foppor-
4414 ogni grande ingiyriayed ignominia, che sentir nominare gli Dei. joppet-
tA nel Cap,in lode della sig.Qusnaia, iglia la voce sherlefe asta: di buriare a 3
Nao » con oltraggi s.¢ puaiure, che Beast - molsi si dice Fare uno scappenco a
org z=

. aller rae

  
 

68 MALMANTILE,

'Allor l amico in mezzo a i'dolor mici
Ati fece uno sberleffe di velluro
; E mi fece arroffir dal capo a piei:. othe!
E più sotto nel medesimo capitolo fo stesso mofira, che habbiamo ancoilverbo
sberleffare dicendo
E col rider di grazia andate piano,
Che non è per infermi ucil conforto,
E chi viol sberleffar, sberleff in vano, \

Lorigine da questa voce sherlefe vien forse da Berlina it quelto modo:

Si faole alle volte, dopo*haver tenuto in Berlina i ladroncelli, fegnargli ing
qualche parte del corpo con un ferro infuocato, acid: che fiend dalla Giuitizia,
riconosciuti, se altra volca per commeffi delitti li tornaffero 'nelle mani'. E di
questi fegni vedcemo foto C. 6. stan. 54. Cid si costumava ancora appreflo. gli
aatichi Romani ne i servi fuggitiviy, € gli fegnavano-nella fronte comic si cava'da
Aulonio Epig. 15. che parlando di un servo.nominato: Pergamo dice',

: Lam fegnis scriptor, quam lentus', Pergdme, curfor
Fugifti, & primo captus es in fhadio; t
Ergo notas scripto tolerafti Pergame vultu,
Et quas neglexit dextera, frons patitur.

Et aggiungefi alla voce verdina quella finale efe, da quella lettera maiuscola F,
che è il segno, o marchio, col quale si marchiano i detti delinquenti. Che co
sia berlina. Vedi (otto in questo C. flan ig.

MOST -ACCIO. Faccia, Volto, eo, we

TENEA la moglie discoffo-un miglio, Tenea la moglic lontana da-se,intendi non
volea pik commerzio con la moglie.. Lat; fecubabat. ap ty

DARE 4 Bizxeffe, Dares o'donare largamente, QuéNe voce; che' & ¢ompo-
sta dal latino bis, & efe., cioé due volte, f, vuol dir pienamente-, largamente,
abondantemente, ¢ simili; Quando il fommo Magiltrato Romano intendeva.
fare ad'aa supplicante la grazia senza limitazione, ma pienamente*faceva il re-
scricto fotto'al memoriale, che diceva Fiat Fide y che poi perbrevitd tro.
no di dimoftcare questa pieacza di grazia con fegnare 1 memoriali con Tole' du»
esse s onde quello che conseguiva tal graziavdiceva: Io hovhavuta la grazia a bis
efe, cioè due volte ff che s' intende grazia intera, e/pitna, a} coftrario di quél-
la limicata, che'era con'una sola esse aggiontavi la limitazione, 0° condiz
con la quale il Magiftrato havea conc la graziay E'da quelto bis efe's' &
corrottamente introdotto il dir Bizzette, che ha il signiticatu,“che 'habbidm

 
 

lea be gout TTh By of pret Ms gait CPavee
DARE il pan con la balefray, Vol dice Gvapazare «. Fare in maaitra', che il
'denefizio fa di i a chi lo-riceve. Deriva forse dallufo-, obtera ire

ze avanti che: andar a caccia tonl'archibufo', di tenere-al suo

midi a posta i quali'con quaiche falvaticina manteneffero le | Pers ede e

questo efercizio Eifendo d' utile, ma assai laboriofo, pud hai ¢ a

quelto Proverbio dare il pan'con la balefira, cioè/accompagnato da'fatica, ¢ difa-
gio

 
wan Peeters.

SECONDO'CANTARE: 69
| io grandissimo. Ma nel presente!luogo intende che effettivamente faceffe ticare
eftratea i poveri.

Si dice ancora in questo proposito.. Porger il pane con la spada,e cid forse de-

riva da quello, che fece Dionifio Tiranno.a un tal Democle Filofofo, il quale

( perché adulando cccedeva in lodare le grandezze di quello stato di Dionifio )

egli féce (edere ad una menfa ripiena delle pia e(quifite vivande, che per un ban-

cherto realeinuentar fipotetfero; ¢ fece attaccare per il manico ad una ferolas
pendente con la punta sopr' alla sua testa, una spada sfoderata, la quale vedura

dal Fitofofo,gli cagiond così grande spavento, che egli aon pott se non con mol-

if ta paura, ¢ con poco-gusto pigliare di quei cibi, Di coftui parla Orazio Od,

pre lib. 3.:;
Ky 2 7 Distrittus enfis cui super impia
Ceruice pendet, non sicule dapes
} ws (08 \Duleem elaborabunt Saporem
| - Sidice-ancora, aqaelto proposito, dare if par col baffone che ha-origine das

'quel che fece il Piovano Arlotto; il quale lp Baltigar I" indiferetezza d' alcani
caccidtori; che gti havevano lasciato 1n casa ua branco di-cani, quando a quetti
dava il pane 5 ' accompagnava con una mano di'baftonate, onde-i poveri cani
s' crano affuefatti quando vedevano il pane a fuggire; per lo che divennero co-
tanto fimagti 5 che pena firreggevano in Piedi. Ritornati i cacciatori per li loro
Caini'y vedutigiivcost sfacti si dolevano del Piovaho; ma egli prefo in mano il fo-
'oli yticd a ate alcuni pezzi di pe edi sie dico-
me era solito patiare il negozioy, in vece @ actoftarsi al pane fugeivano 3 Onde il
Pidvane si feusd c6-i cacciavoti dicendo: 'Come volete sche ingraffino-,\(e quan-
pre ner faggono come vedere ? £ da' questa facezia venne
e Oak it pe ed/ ha/tone, che tignifica moftrat di yoler far del bene a uno,
-fargli del male \»Seneca'ci fa'veder guelto modo di dire anche appreffo.a i La-
fini, raccontando il detto di Fabio per soprannome Verracofo-, che il piaceres
Fate daper(ona 'zotica', € con indmiera [aivatica chiamava Panes lapidafum 9 che
€ Appropriatd at ndftroldetta-Dare i/ paneye ta fafsara,;
 BALEST RA' Sttumento's o arine da caccia, col quale si (eagiiano palle di
fecea jhnelia guild che si fa delle frecce; © serve cer alton 'uccelletti,,
“d'un arcod' 'acclaio accomodaco in cima a un'atta 5 '© degno torto,,
e (Ono adactati altri ordinghi di 'ferro' pet 'facilitare'operazione.,
ietie- dal? antica baliifta-arme Suerricra, che dicevano ballifta 'forse dal Greco
'ballein » che significa (cagliare “= y
ee HG 9 eal Iv.
La plebe's i grandis mi) Vedutolo cos? 'mutar'regiffro:
Chit DisvaiMuwbosacyraen E diventar pserintiy S
ene aun Pe lor finifro vt au talmente dt 'tninvo catrivo:,
 Edin lor pro r rato Chel biurebbon voluro ingoiar vive;
> Per giueita murazione del Duca di 'biiono in cattivo, li faoi fudditi, che pei-
 “ma'l' amavano., cominciarono:a portargli'odio, ¢ bramareli ogai niale,
81 farebbe sparato in lor pro, 'Haurebbe fatto loro ogni eee immaginabile..
| -Hiatirebbe meffa,'c (pela la propria vita.a-benefizio lero 3 ¢!> voce pro-è-ua fu-
flantivo

  
    

 

 

 

 
 

70 MALMANTILE 05%

fantivo che significa giovamento, utile, ec. dal:latino prodef

MUTAR regifiro, Mutar maniera di fare., Registro diciamo quell'.ordine di

ferri, il quale & negli Organi strumenti musicali, con ciafeuno de' quali ferri al-
zandolo, o abbaflandolo si da, o leva il fiato a quelle canne, le quali si vuol,
che fuonino o nd, ad esserto di far mutar voce all' organo, il che si dice smuear
regifire, che pafiato poi in proverbio significa Mutar maniera, o- modo di
in qualfivoguia cosa. Vedi forto C8. stan, 52. alla voce protocollo Regi/tro in
altro significato.

INGOLARE. Trangugiare. Mandar gil in corpo una cosa senza anche ma-
flicarla, che si dice anche sngullare. Vedi forto C. 1. stan. 6.

s

TANZA V.
duvenne, che gid inteso un Negromante E per ridurlo all' opre buone, efante
C'un'huom com'era quei si ginfloye magne, Non per [peranza di verun guadagno
Faceva novita si firavagante, Fintofi un baro, adargliandol' afsalte,
Vn' atto volle far da buon conipagno; Fon po di ben chiedendo per fant' alto.

Stando le cole ne i fuddetti termini, Va tal mago, inteso che un -huomo
bene come era Perione s' cra cangiato in così cattivo, volle fare un' acto da hus
mo da bene, cercando di rimettere Periane nelia*buona firada, ¢ però fintoft
un' accactone, andd.a chiedergli l'elemofina per amor di Dio.

WEGROMANTE. Flo fleffo che Mago: Se bene Negromante venendo da
negromanzia s' intende colui, che per mortuos vaticinarur, che è una delle sei spe>
cie di Magi detti sopra C, 1, stanza 20.5 tuttavia da noi si piglia per nome geng-
rico, ¢ per intendere ogni specie di ae ye di magia. i:

BARO. Biante. Accattone fallo.» ien forse dal Greco Barijs, Bareos., che
fuona molefus, importuno, sfrontato, come appunto sono questi tali; ¢ se beng
questa parola ha del furbesco pure s' ula comunemente, ¢' usd il Varchi St. Fior,
lib. 11, Ed in fegno, che lo rifiurava,¢ non gli oreduea pil, havendolo per baro af
giumatore,arfe i suoi libri, i

PER Sant' alto. Cioè per Dio. E,parlar furbesco y il quale forse &noto fuori
della nostra Tuscana, come inventato da Vagabondi, Monelli ¢ tianti per non
esser intesi, se non da i lor pati, ¢ poi fattofi familiare a mole' altri, a.fegn
che ne-è facto, stampato il vocabolario.. Si dice anche parlare im gexgo,ed in lingwe

furfantina, come Cl moftra il Vaichi Sr, Fior, lib. 15. Apparifeono piu lertere scritte
non ins cifra, main gerga a uso ds lingua furfantina moairo mse « I nostro Poeta. |
ferye di tal parlare nella persona di quelto Biante perché, come ho detto; si
huomini fon soliti pariar in Pa ao i; eoaipouis ponte'

Rispose Perione -. Pratel mia, « See bai bifognos che posso fare iedns,
se ru te lo credeffi tat? inganm 5 Che fon Frafaxia sche rifaccia':
Tu anoi ch' io doni per ? amar di Dio, B che penst sche qus ci sia la cava?

Ne [ai ch' s0 pigliorei per San Giovanni, Non ¢ piu sompo che Lerta filava

'Aila richiefta del Mago Perione non si muove a far limolina., anaid = che»

pigtersDhe anch' egli qualcofa, ¢ che ¢ palato quel tempo che egli dava via
1b tua., ¥ e

PIGLIEREL per San Giovauni. §. Gio; Batifta ¢1l Santo provenore seen

atta

 

oh

;

 

 
 

 

 

 

SECONDO CANTARE, vhs

Città di Firenze,e percid il giorno delia sua fefta ¢ grandememe folennizzato 5 ed
in quel giorno fon ficuri nella Città finovi banditi capitali, ficché gli Sbirri non,
Hidnipislanatins + Daquefto è nato' equivoco Proverbio; Pigiterebbe il dé
di San,Giovannir, 0. per San Giovanni, che: vuol dice Piglierebbe anche quel di,
nel quale ne meno i birri pigliano,; ¢s' intende pigdieredbe, cioè accetcerebbe tutto

che gli futle-dato ih-ogai occaiione y ed in ogni tempo. B lo scherzo & nel
verbo pigliare che vuol dir Far cattura, o Catturare,e vuol dire anche Accettare,
© ricevere, come s'intende in questo proverbio; che esprime; Lo piglicrei, ed
accetterei fempre, ¢ non darei mai.

CHE fon Fraffazio, Raccontano una favola d' una donna non troppo hone-
sta, la quale havendo commerzio con un tai' huomo detto Fraffazio, fu con esso
una volta trovata dal marico; ed essendo ella altrettauto fagace, quanto il ma-
rito femplice:, ¢ di cervello grosso, gli dicde facilmente a credere, che colui cra
un' huomo da bene 5 che andava rifacendo i danni a chiunque occorreva qualche
disgrazia, e che l'haveva chiamato in casa affinché le ricompraffe una faa con-
ca, la quales' era rotta,¢ che appunto gli narrava questo suo danao; foggiun~
gendo; B come, Marito mio! Non conoscete dunque Fraffazio? Il buoa ma-
rito se la bevve, e così 1a donna scampé la furia, E da questa favola, quando &
dice: fer Fraffazid, vuol dit: » Esser colui che [pende il suo per folevar t aitrui mi-
Seriesye che rifa. i danni come dice il nostro poeta.

CHE pensi, che qua ci sia la cava, Pensi che io habbia la cava de' danari, cioè

+ Torna bene a questo detto quel che si trova in Saluftio; Cen/es me vi-
em ararij praftare. Non ¢ pero che cava voglia dire la Zecca, ma si piglia per
questa nel presente detto ( da noi usatiimo tn questo proposito ) perché si (up;
ne 5 ed è verifimile che la Zecca, come luogo dove si batte la moneta, ne fais
abondante, come sono abondanti le cave di quelle cose, che da efle eftraggonfi.

NON ¢ pik ib cempo che berta flava. Non € più il tempo, che le cose andavano
come si bramava. 1 tempi fon mutati. Pipino Re di Francia per mezzo di suoi
Amba(ciadori sposd Berta dal Gran pi figiwuoia di Filippo Re d' Vagheria, las
quaic havendo saputo\, che questo suo Sposo era brutto, © nano, malvolenticri
s? accomodava a dare H conienfo; ma pure, vinta dalla riverenza dovuta ai pa-
dre, condescefe, Arrivata in Prancia, la(ciandofi governare dal giovenil senti-
mento;richiefe Elifetta di Maganza ua fegretaria ( la quale 4'Vagheria,dove era
naca del Conte Guglielmo di Maganza ribello di Prancia,(e ne vemiva-con Berta a
Pacigi ) che voleile, fingendofi la sua persona, in sua vece sposarsi con Pipino
il quale,¢ pera fomighianza, che era fra lor due, '¢ per non haver Pipino mai 4
veduta Berta, non' haurebbe aflolutamente riconosciuta, Bliferta da ere
si moftro renitente; ma persuala poi da Grifone,¢ Spinardo di Maganza sui
parenti, condescele a i voleri di 3, B.così arrivatia Parigi, Elifecta si spo-

80 con Pipino in vece di Berta. La quai Berta in tanto di consiglio di detti due

Maganzefi s' era ritirata in ludgo vicino a Parigi, con pensiero fermato cons

-decti Maganzefi di quindi occultamente partir, ¢ tornarfene alla patria com

Aaiuto de' medesimi; ma guelti la cradirono, perché in vece di servirla alla vol-

ta della patria sua, ' inniarono ad-un bosco, con urdine a quelli, che la con-

'ducevano, che l'uccidetlero: Mu coitoro mou a picea, in veced' sagen Ihe
bi spo-

 
 

 

p MALMANTILE

spogliarono, ¢ legatala ad un' albero la lasciarono in preda alla Fortuna, ¢ tor-
narono a i Maganzefi, dicendo che l'haveano uccifa 1 Maganzefi per occulta~
re si atroce delitto fecera morire tutti quei ficarj, havendo prima anche-d' arri~
vare a Parigi fatte ritorpare in Vagheria tutte le dame, ed\altre: personenons
complici, ne confapeyoli di si geande scelleraggine 6} cs 2th aawmiss
Berta intanto, che se ne stava così legata:dolendoGi 5 ¢:lamentandofi fu sentita
da un tal Lamberto Cacciatore dei Re Pipino; Coftui seguitando la voce ficcon-
duffle dove stava Berta legata all' albero., ¢ (cioltala., alla propria.casa la con=
dufle, ¢ la consegno alla moglic yeftendola d' abiti vili, ¢ conformijalla 'posibili-
ta di lui, ed alla poyera condiziane, della: quale Berta disse deflere 2» Quivi
stette Berta circa cingue anni 5 nci qual tempo guadagad molti: denari di) filare}
ed altri lavori,, che insieme con:le figliuole di Lamberto facevacs: Avvenne una
giorno, che eflendo Pipino a caccia si condutic folo alla Cala di Lamberto, ove
veduta Berta s' inuaght di Jei 5.¢ con essa si:congiunfe sopra.ad-un suo carro 5 nel
qual congiungimento fu gencrato Carlo, coskdetto dal amedefimo Carlo. dn ta»
le occasione Berta scoperfe a Pipino ii tradimento. de i Maganzeft narrandoli
tutto il seguito; perloché Pipino fece abbruciare Elifecta, cd una mano di Mas
ganzefi, ¢ rimefie nel trono Berta, 3h bh fob eomearr« plidhoss
Da questa favolosa foria nacque ilproverbio; Wom è pik. ikvempo che Berta fie
Java, Cioé non ¢ pil il tempo che Berta flava neile feive filando., ¢ ricamandoy,
che-significa; Le cose fon mutate, ' i Pas X\

Di guefto dettorfi (ervi Berta moglie d' Arrigo 1V,Imperatore, come si vede
nello Scardeonio Monameata Patavina lib. 3. Cialie 1g. de Berta ex Montagna~
no, le di cui parole fon queite. Ademonatur ia iifdem Pacavinis eAnnalibns celebris
fama Berteex Vico Montagnani, qus quidem fait ruflicano genere, fed moribus certe
perquam nobilis CO animo perguam generosa, ach
Hee enim tempore Henrici IV, Imperatoris, cum eius uxor, Berta & ipfa muncupa-

ta, Pacavij moraretur, vel cinfdem force nominis similitudine, vel propria generositas
te animi allecta, obrulit ei dono filum tenuiffimum 5 quod-eleganter [amer neverat mar
nin 5, in Vrbem venale detulerat Quod munus Regina bilari vulew accepit 3:
cum, cognoviffet nomen, OF animum mulicris, cam indignam cenfuit, xt vitam inopem
Samineo colo amplius fuspineret fuam,, Dato icague filo procuratori suo, inber ad Pagnm
Monragnani frarim proficisci, ubi mufier habicabat,& pro referends gratia tor terra
ingeraei ex publica adferihi, quantum [pacij filum dono datum extenfum. comprehen-
dere,@ cwrcumdare posset, Quod.cum catere mulieres vidifsent, ilico Berta exemple
attulerunt »& ip[e filum, quod Regina dono darent. At ipfa renuens id ab alijs acci
pere percanté re/pondir, Pertranfye tempus, dum Berta filabat. 3 sree
Gliantichi digevano Aon ef amplins atas Cyclopum,ed in. moluc.a)tremaniersficame
Ancor noi diciamo: E finita dacuccagna,o la vignuoladVen ¢ piri cempo ai Bartolommery
ec. Cont quali, ed alpri detti intendiamo Non:si godono pil quelic felicitache già:

   

si godevano. STANZA. VIL 2
» Signor ( foggiunfe il Adago) mi [a male Hor bapa; Chi del miofac.
Di veder, ¢' un si gran limofiniere, ( Difs egli) fa la xappa nel '
E4 huom tanto benigno, ¢ liberale Pero va in pace tu ca' tuoi bifeget 5
Caduto sia nel mal del miferere.. Pevche per me tx mangerai de' her's

  

"igi

 

 
SECONDO CANTARE: 3

ll fegromante vedendofi cacciar via con tal risposta; replicd, che gli dispia-
eeya,;ch' ei fuile diventato avaro. E Perione li foggiuafe, ch' ci non sperafle da
Jui faffidio alcuno..

CADVTO net wial del miferere. Divenuto mifero, cioè avaro, tenace, che se
bene il mal del Miferere è una infermita mortale; Noi ci serviamo della voces
AMiferere nella forma che habbiamo detto sopra:C. 1. stan. 80. della voce boceo/i-
ea; per intender mifere, che nel presente luogo vuol dire avaro; ¢ così è inteso
comunemente, se bene la voce 44/ere propriamente vuol dire infélice 3

FAR capitate. Par' aflegnamento; o sperare nell' aiuto d' alcuno. Vedi forto
C. 7. stan. 82. Questa voce capitals € dedotta da capirario capitationis, che era una
tafla, otributo, che determinavati ix capita Leaner per aflegnamento; ¢ pro-

tk \priamente capitale del Principe, come ¢€ forse la Decima, che pagano hogei i
nofter contadini, chepure fidice decima in fu latefta.
| RANIERE B.un valo intesluto, ¢ composto:di fili.di vetrice, o-d'\altra spe-
| cie d' albero,.0 di fottilissime strisce di legno in-figure:,¢ forme varie, in tuttes
i le\quali che tieno., ha fempresil manico; che senza manico si chiama corbello,o
' panicra 5 € servono Perporcar frutte, 0 altra.che sia; detto paniere, 0 panicra
torfe es » perehé gli antichi tenevano il pane in tal forte di cefta in mezzo
alic menfe, ¢ percid dat Lacini detto Patarium.
\ Fed Kila ruppanelipanere. Questo proverbio dice;
Chifal alsrni meftiere
ring ta es eeude. en
JE ¢osidichiara iluo significato., quale ¢: Che colui, il quale si mette a fare
tuna.cosa,, che non fa) fare, non fara nulla di buono; ed in fultanza vuol dire; Af-
anne wane. Ovid. libs )2.
» Kique liquor rari fub pondere cribri
© Ede forse:meglio dir Juppa, che 2xppa yenendo dal verbo /uppurare, che vuol
dire attrarre ? umido; 0 da Suppen r Tedeseo.- Vedi sotto C. 4. stan. 25. Ma lufo
isiobligaa dir zup
ok tn pace, aan pfamo. dire y quando:mandiamo via i poveri, che accatta>”
oj |) asd.in un certo, modo Pjauto in milit. dicendo Pax, abi,
MANGERAT de fogni. Mangerai cose immaginarie, clog non mangerai.
ee ele Capitolo della paverta dice.:
of can ls vn Chevsfacciara ralor non si vergogni
| lo permetta ye faccia male,
Latin i on » che non cake viver di fegni,
if  A Latin Latini pure bayevan Gmil modo di-dire, come si vede in Givyenale Sat. 6.
ia Qualiacumaue voles ude fomnia vendunt.

E.coloro, che map veaeie accents dana cosa,fogliono fognarla; perché

Saag non ¢ il fogno, che
Vn'

ha corvette
. 9. introduce un, Pastore, che raccontando le sue felicica

Poss ideo quecumyue Tolent in notte videri
Ln [omnis, vim magnam ovium — capellas.

 

'La onde, Teocrito,
(Gosi ragiona:
 

4 MALMANTIDE

Er anco noto Nonio, che appreffo gli antichi Roniani, il verbo ve/cer signifi-
caya vedere: Prius quam infans esserytni oculi facinus vefeuntir ciok videnr;come noi
Pure diciamo; Afangiar un con gli occhi, quando altri guarda uno:con grande ac-
tenzione; ¢diciamo anche:. Dar paffo acti octhi. Dan. Par. Ci 27,08

Efa natura, ed arte le pasture
Da pigliar occhi: 2 ¥

Si che dicendo mangerai de fagni, si pud anche intendere, Ti faxieruijeifeddisfa.
rai con dar pasto a gli occhi; 0 della vifea; che ¢ ho tictioche non mangerai. Vedi
sotto C6. fan, 55. che er la vifta.

 

 

 

STANZA. V4ill, STANZA IX)

Comer replico quei.) se.e' si.cicalay E non barteva la mia fine alerove 0%
Che tu darefti via fin ta gouneliay C'adhaver prima ch' io ferraffi li ocehi
Vedendomi [pedato, ¢ per.la mala dn ricompenfa un d:, piacendo a Giove',
Petra haver' jt eranchio.alla fearfella? Della miadonnaquattr'sfeimarmocchi,
foi che tu grarti ij corpe alla cicala Ma finalmene dopo mille prove 3
(Life it Duca) ia levi qucfta cannella Didar' il lustro a marmi coi ginocchi',
ser quel ch'io ti diro, percht se gid Tenendo giochi in molte,e il colloavite,
Donai-, non era tuttn carita. E Le nocca cot perro fempre in lite;

STAN ZA) Kina ere ' ars

Lot bebbi bianca a femmine,ed a mafebi, Perché. po poi(difs' io)2li¢ me'chtiocaschi
Ond' io sbraciar volendo a bel diletto, Dalle fineftre prima, che dal tetto;
Mi risoluei levar quel vin da fiaschi, - Bil cavarmii di mano aiteffo un pelo,

vE aomdar pile quant'un pinrabd'agherco, Sarebbe un voler dare un pug hioin Cielo,

1, Mago mofira di-non ipoter eredere, che havendo Perione nome'di liberalif-
firio, non s' habbia a muover' a compatlione di Iwi 5 ¢ Perione vinto dal” impor-
tunira di coftur, gli dice, che fu già liberale'per disporre'il Cielo 'a concedergli
fig\iuoli; ma perché eg? non era stato efaudito, lascid di far pili limofine, ed
hora era impotlibile cavargli di mano'un picciolo.,

Sf cicula, Ciok si dice; Si dilcorre. Il verbo cicalare-usato in'quéfi termini
«sprime discorso di-cosa incerta, che frdice anco bucinare y oburicare', E si dice:
la tal cosa non fu poi vera;ma fu una cicalatayciok sene parlo; ma non'é poi stata

vera. “ terigacih;
'DAREST 1 via fin'la ¢onnella « DaréRi via fino al'proprio veRivo; datetti via
tutto il'tuo havere.. E se bene gonnellas' intende una specie'd' abito da donna, in
qucito proverbio diventa nome generico per ogni forte d' abito.
SPEDATO,, Cie co' piedi laceri dal viaggio. ' vere
PER la wala. Cioè per la mala via'; e's' intende thal condotto di fawita,e mal'
all' ordine di veftito, ¢ feaza danari '. tiowk wie? 3 mS as
HAVER il granchio alla fearfella'. ChiatniathorGrianebio, © ee ndspetie di
malattiadi (pafimo, la quale quando viene alle mani impedifee'il maneggiare le
dita; E da questa quando diciatho W rule bait eranchio alla forfella intendiai
non pud adoperare lé tani incorno alla borfa, che'vuol dire; & pigro acavar de.
nari della borfa, cioè, adire: ¢ tenace,, o avaro., ed uno, de' quali parlando
Marziale dice.: aie

 

ys

 

 
 

 

SECONDO CANTARE. #8
tpi. a Bitigat'y& podagra Diodorus, Flave', taborant;
ne nh ' Sed mil Patrono porrigit; bac Chiragra eff.: j
tx. | Enoi pure diciamo di questi tali; Haver la gorta alle mani, Haver i pedignoni

alle mani; Haver le mani acgranchiate; farebbe a pagar co' monchi,
SCARSELLA, Intendiamo ogni forte di tasca, 0 borfa di danari, come si
vede forte C, 3. stan. 5., se bene scarfella ¢ propriamente una borfetta di quoio
if Con ferrature di ferro fatta alla foggia delle Carniere da'cacciatori; fa qual forte
ei di borfa usava già in Firenze portarsi da tutti legata a cintola..
GRATT AR' il corpo alla cicala, Incitar' uno a discorrere. Vedi sopra Cant.
primo fan. 2.1 Latini pure difflero in questo propolito Crcadam ala comprehendere.
LEV AR 1a cannella, Defiftere di fare una tal cosa. Traslato 'dalla botte, alla

hi fileva la cannella,quando è finito il vino, che ¢ra in essa, K cannella inten-
t) amo quel legnetto tondo forato per lungo, che si adatta al fondo della bottes
bi, per cavarne il vino,la quale da i Latini con voce Greca si dice epiffominm, Si dice

anche in questo proposito.
i LEVAK A! vino da fiaschi, come vedremo apprefio.
; PRIMA ch io ferraffi eli ochi. Prima ché io mor:ffi.
MARMOCC HE, Ragazzi. Queita voce marmocchio in significato di fanciullo,
viene da marmo, alla pulitezza,e liscio del quale s' 9flomiglia il liscio, ¢ pulitezza
del volto de i fanciulli, ¢ delle fanciullette. Or. Od.-19. lib. 1.
h Vrit me Glycere nitor
Splendentis "Paria marmore purius.
DAR il lustro @ warnii'eo' i ginocchi. Cioè Naya tanto tempo, € così (peffo in,
ginocchioni, che il lungo fregare con le ginocchia faceva divenir lucentt i mar-
, ini', sopra i quali s'inginocchiava.
1) FENENDO gli occhi in molle. Cio' lagrimando, ¢ cos) tenendo gli occhi in
molle nelic lagrime. 2
COLLO a'vite. Collo torto, come fanno i Bacchettoni. Si dice a vite per si-
militudine, essendo /a ve uno strumento; il quale serve per ferrar' un materiale
con l'altro, che peg essere attorcigliato come /a vite pianta, che produce l'uva,
sn <r? nome', ¢ si dice anche marti ¢ chiocciola: quello dal torcere, col
' > fa 'operazior ione; ¢ questa per la similitudine, cHe ha la sua figura con
 il'gufeio della chiocciola a.
BLE nocea cal petto fempre in lire, Ciok dandofi delle pugna net petto; il
che moftra che le nvcca fiego in lite col petto, mentre non ceflano di pei quoterlo.
E nocca'intendiamo nodelii delle dita. Vedi foro C, 3. stan. 8., ¢C. 9. Itan. 54.
In somma i} Poeta con queste quattro maniere di dire, ioe Dar' il laitro a' marini
0", ginocchi § 'Tenere chi in male, Haver il colle a vite; ¢ le nocea fempre in lite»
se eee Pie meer” orande; ¢ descrive assai bene un' Hipo-
a; ao € Al» "
ALO F bebbi Slee pens ha da conseguire per via d'eftrazioae “
di-polizze ( come si fa al lotto ) sono scritte Wolaniedele polizze premiate,¢ I'al-
tre fon bianche; ¢ chi ha una polizza bianca,non conseguilee premio alcuno. E =
di qui viene:il detto 40 / ho bavuta bianca, che è fatto comune, ¢ per intender di;.
futte quelle cose, che si tenta di conseguire, e non si conseguiscono.
ene c K2 SBRA-

      

 

tt
Mar

 
96 /MAEMANTILE

S8RACIARE, Vuol propriamente dire, allargare, ¢ follevare la brace a fine,
che meglio s' accenda, € renda più calore; ma per metafora intendiamo spender
prodigamente, ¢ largamente, come.s' intende nel presente luogo, ¢ sotto Cant, 3.

an. 2. is

A bel diletto., A posta; 0 per gusto, ma senza buon fine, ¢ utile 5 ¢ fidice an-
che a bello fiudio., a bella posta, a bella prova., che wuiti si posion Pigliare in guefto
fenfo., Se bene alcune volte significano quel, che i latini dissero dedita opera es»
maffime quando non v'é l' aggiunta di be//a, che in questo.calo ¢ detto ironica~
mente, ed ha forza d' csprimere Liafimevole y come per elempio Veramente tu bat
atta wna bella cosa, cide w hai fatto una cosa biafimevole,.¢ che fla male « 4irg,
Egreciam vero landem, & spolia ampla reportas.

NON darei quanto un puntal d' agherto, Liaghetto una cordicella fatta di.feta,
© d'altro, che serve per afhbbiar le velti, ¢ adattarle alla persona, alla qual. cor-
dicella è solito fare una punta di sottil lamina d ottone, o d'.altro metallo,,¢
gucfte punte si dicono panrali, ¢ di queste punte fen' hanno due,.0.tre per un»
quattrino;¢ da questa vilta serve,ilpresente detto per esprimere; Non dares wiente,
ne meno una cola, che non val nulla. Che 4 latini difiero fra' altre molto, /7-
tiofam nucem non dederim. E noi pure diciamo we fico fecce, um dnpino-, ¢ Gmili.
Vedi sotto C, 3. stan. 8.

LEV AR il vin da faschi, 1 fenlo metaforico è lo stesso, che levar la cannella
detto poco sopra stan. 8,

- PO poi, Alla fine. All ultimo de gli ultimi.. Opera anco.in questo detto la,
forza della replica, che induce superlativo, Vedi sorto in questo C, stan. 73,

GL' è me ch' io caschi dalle fineftre prima che da} tetra, Nel male & il meglio,' ¢-
leggere il meno. Intende; egli ¢ meglio, che io lasci flare di dare il mio. che se-
guitare, ¢ darlo via tutto, cioé mi conteati di questo danno,, ¢ non lo faccia
maggiore col seguitare a profondere il mio. E que! me per meglio ¢ la figuras
Apocope da noi speflo usata; ¢ I'uso Dante pili volte, ma notabilmente nel C.
32, dell' Inferno, che |' usd nel principio del periodo..., aceatijaat

Me fofte fhare qui pecore, 0 zebe. sda ene t nos
Ma di questa figura Apocope, ¢ come | usiamo, vedi forto in questo C., stan, 36.

CAVARMI di mano un pelo, Conleguir da. me cosa alcuna, ancor, che di nian
valore. -; slob ned

SAREBBE un voler dare un pugno in Cielo, Sarebbe un yoler tentar,unacofas
impotbile, Facilius Calum digito actingeres, s>

  
  

 

       
 
  

eng ANZA Xt. RIAN Zh AM, Se

Che pagherefti ( disse f editro da te nov alpettar ch' io. chiedds;\

dm “here
Se cio fulle ( rispe/e Perion « Rerehisamanciique abiesta altri mivedas
Mocor eis nod yh fabtia tins dfegoa, \ deboine, vale jd li von
Eraluoglia appiccara habbiaaltarpiones rf fra se ru brami a day
to ti vorrei donar mez" il f pil regno dopo regoverniy m!
Sogginn[e quei: Non vo pur'uns sane 2a ling y
4a folamente La tia 4 Cua cuor th portin a. i

13 Siglo th s D389. 3G

 

* STAN

 

 

 

 
 

ae per = en

SECON DO CANTARE, 77
S T.A.N)Z.A XH.
Ed ordina di poi, che se ne quoca.» « Prefa che thaglie fateoit becca allvca,
> Ga terza parte in circa arrofto,e leffa, Che subito ch'in corpo se eacfa,
» (Chim tutti modi è buona)edan'un poca Senza che tu pits altro le apparecehi,
nds nquel modo 4 mangiar alla Ducheffa; Dotrela pregna infin [opr'a sii th
1 mago s' efibisce a dare a Perione il modo, che la sua moglieimpregni;

Perione glidice che se cid segue li vuol donar t mezzo il suo regno; ed il ak
ricusando il tutto y.da.a Perione laricetta dellAfino marino per impregnar la.

maoglic..
CHE: pagherefti ? Quando veggiamo uno, che fommamente brama di fapere,o
a otenete una cosa; per moftrare, che € in nostra potefta l'adempire il suo defi-
ae »fogliamo dire; Che pagherafti? Che /pendere/ti ? Quanto darefti? © simili,
ares fied » o.diceffi la tal cosa ?
| EGONVE »\ Maliardo, Mago, Negromante, ec, Viene dal.Jatino, feeon,
| doh oer Murcto nelle sue varie lezioni lib, 12, c.19. emendando ua jug-
di, Plauto nelle B « Longurs eft Stris ion... Strigas
( dice egli ) vocabant mulieres y quas etiam noitn velare arbicrabantnn » codemgue.sode
rer: bamines.maleficos, 5 quorum vocabuiorum vulgus in Italiansitur, Vedi (oto
3: 69.
, L0.non ve pite.difegne. Yo non ho pila (peranza d' ottenere questa cola. N'ho
affatto levato l'animo, ° il pensicro e
= ARRIGUCARE la: ong te arpione.. Haver lasciata la voglia y 0 i} desiderio
Eee lo. che e-dppicear' al chido vio sopra C, 1. fan. 8, Eque-
se procede, da i voti-, che anticamente facevanoii Gentili,

 

   

jn Tempio; i.quali non & potevano levare di dove eran polti,

DS coupe ticels in uso comune, © profano.
0 RINE, E una specie di chiodo uncinato per ulo di regger l'imposle delle
parte netinetire,, girando, quelle sopra di edi. Da i Latiai decti Cardizes.

Se 40 pear Eas Non, yoglio danari. Crazia & delle più vili monete

rESOtO pas eflendo, ? ortava parte del giulio.

sn: 2 ¢ maflime dalla gente vile per cA ceaense Nona

sects ta cola.

 
 
 
 
  
  
   
   

a servitore agli huomini xirmofi, edi gar- i
> I tale» & un' hanna egpiesode il.detto di Diogene

diamo huomo dotto, virwolo, ¢ di tutta perfezione.

serine de ale PAutore si serve anche acliottava, 7, ante-

sedente.), seaoecs is aserstaone dun discorso, ¢ paflaggio ad
'oro sonia d ee 3 Ha baftanza quanto. habbiamodetta,per
'gonchiudere iLcome., ndo; 0 pe 90 non farelaral cof... —~;
« REDA. Ciok lyecettion ide ° danende fighuoli..: Seer: reday
iL tale ha hanuto.nn figh E egies Vinee iorentina »: sprepufata x€
folamente per i contadi; d intendono anche i Peet Lic beitic.

MOSCA, Bind hat be venditori di pesce, che vivevuno al tempo,

che l'Autore compole quef?

) GLI è farto il becco aro bs Deneeid &conchiulo, che i Latini diflera: JZ
' ales. s) Lailt nella sia Bh. Hee. 2. than. 64. ae “xe

    
78 MAEMANTILE:
We vanno tuiti: il warcie hora-figineca,
Non v2 rimedio;. E fatto il becco all' oca "

Dice Francesco Cieco da Ferrara nel suo Poema intitolato il Mambriano(Ope-
ra nota per efler l' origine, ed- antefatto dell' Orlando innamorato, Poema del
Boiardo, ed in conseguenza dell' Orlando fariofo di Lodovico Ariofto) al'Can-
to fecondo 5 che ibe & SORE
» Pugia nel Regno di Cipri un Re chiamato Li¢anoro il quale' havea una fola
y» figuola nominata 'Alcenia, la quale amande egli-al pari di se stesso 5 vole fat
»» pere, se buona, o ria fortuna ella fule per havere 3 fatti però chiamare*alcuni
»» Aftrologi fece fare la nativita alla. medcfima (ua figliuola; © tutti concordaro-
x» HO, che ella farebbe prima flaca madre, che moglic; 'Onde il Re per evitare
»» il prefagito vitupero, fece fabbricare un giardino contiguo al feo palazo rea>
»» le, ¢ dentro al detto giardino edificd una fortifima', ed altissima Torte cons
x» Molte stanze, ¢ con tucte le' comodita, ma senza fineftra aleuna, che riuscif-
» se fuori della Torre: Dentro a questa mefié la figlia'con alcune Matfone,e
xy Damigelle, aficurandofi dell*ingrefio della medesima » non folamente col te:
»» Herne cgli proprio le chiavi della porta, ma con haver deputate accuratissime,
»» ¢raddoppiate guardie di soldati intorno, ed alla: porta della*torre, ed alles
2» mura del giardino; ne altri cotrava nella torre, che una fola'donna, della
» quale il Re si fidava, © le dava la chiave ogni volta, che a lei occorreva an-

»9 dare alla Torre con provvifioni di vitto, od' altro. bias * O33eik
>»  In questo tempo mori un tal Co, Gio: di Famagufta huomo ricchiffimo';- ed
29 alquanto parente dg] Re, © Jalcid érede delle fue'immente faculta Caffandro
23 unico suo figliuolo; Questo giovane fece fabbricar wn palazo fonwofissimo, ia

x» cui teneva corte bandita con tanta splendidezza, che fino al medesimo Re ven-

»» ne voglia d' andarui, ¢ lo mefie ad effetto / Andatovi dunque fu dal giovaned
» inuitato a cena, ed il Re accetto l'inuito, credendo fargli conofer, che non
»3 era in grado di banchettare decentemente un Re all improvvifo. Ma tutto il
»» Contrario avyveane, perch il Re fu così ben servito', ¢ di vivande,¢ di mufi-
sy che, ed' ogni altra cosa coaveniente ad un banchetto regio, che gli parve++
by che Cattinio havefle maggior poslanza,' che non haveva égli; onde “eS
») cid ad havergli inuidia, ed a pensare come' porefle 'morti carlo'; Haven
pera veduto sopra ad una maravigliofa fonte, che era nel giardino, un- motto
ay Che diceva Omnia per. pecuniam fatta funt. Si volto a Caflandro,'¢ difie+

3» Motto € troppo prefontuolo, efiendoci molte eofe, che non si posion fare col
» danaro, AJ che rispose Caflandro: Sire, lo ho posto quivi quel motto | per-
3x ché mi fon fempre creduto 5 che il denaro apra la frada anche al? les
31 ¢ fino a hora mi € riv(cito, come appunto m: (Gn figurato,Horfu (replied il! RE)
yy Già che tida il cuore-di porer fare gni'cofa col'dcnard, io' ti' do 'tempo

»» anno a procurare per le trade, che vorrai y'di godere 1a inia'fighuola,ehe io
sy tengo pella torre guardata,come tu (aise (@ dentro a quetto me quae
eleva

  

x t0, fara toa moglie; quando no, la tua testa paghera la pena. EB

» il Re, perché eflendo entrato in sospetto della potenea di Catlandro 4
y sotto qualche pretesto levarfelo d' avanti-. 7 A
y»» I povero Caflandro ritnatto sbalordico da tal-proposta,meditaya

 

<4

 

 
 

 
 

SECON DOC AINT AIR E, 99

y» bando dalla patria, “quando Euripide sua Balia.; fapata lacagion® de! fao di-
» sgusto gli disse, che si confolafse., perché ella haveva un sao nipots dotazo di
'97 Così grande ingegno y che afsoltitamente gli haurebbe aperta la strada all' in.
9) \grefso' nella Torrey =.) ee
x» Questo nipote della: Balia,Ewripide fabbricdunOca di legname 5 grande,
y tanto, che potelse agiatamenté asconderfelesin corpo \un':huomo y chev' en-
»» trava,"eutciva perdi forte Pali, e\per via discerti ordinghi facevafare a tal'
35 Oca tutte loperazioni;e'moti-, come se fulse fata'viva, edera del-tuttoper-
99 fettary senon che lemancava il becco. Cafsandra fece sparger voce', che era
syeandato! in lontani-paefi»; ed: intanto 'havendo: fatta portare: occultamente 1a
+9) detta Ocatin-un luogo-remoto:;¢entro nella medesimay ed/Buripide sia Batia io
“¥; abito: moresco la guidava | sfingendo di venirdal Cairo dove'era verimente >
y nata, ed allevata detta Euripide ) e"parlando \in-guelladingua ben? intela a1
Ҥ) 'Gafsandro,'toccava con:una 'bacchetta |l'Oca 5 cd'eva ih concerto', che Cal-
~gy fandro iper via di certe Zampogne facefse cantar Ivoca. L' aftutasBalia, ac-
y» cennate a a l'operazioni dell' Oca, andava dicendo, che'a-volerla vedose
$y operar colegalanti,:¢ maravigliofe, bifognava spendere; ¢'perd il popolo,
»» mefsa insieme buona' fomma'di monete,la diede alla Balia plaquale fece tare
1p alrOca diver belle'operazioni. i
+4) Arrive la fama di quel? Oca al? orecchic del Re, \¢ della' Reginay ond»
«39 'fattala'venire a se, dopo haveria veduta operare', regalata Euripide,la man-
3 darono! ad: Alcenia toro: figliuola per farle-pigliar qualche spalso, € diverti-
«Jy mento hei ginochi dell' Ova; 1a quate condotca nella Porre;il negozio ando ia
“59 maniera,-che-per via deteattati della Batia, Cafsandro'nelto farein camera
«59)'d" Alcona afedtoinquell'Oca:; si godé Alcenin-, efi diedero.ja fede di Spot.
'fs! Batro-quests,Cafiandro accomods'all' Oca il beceo; econ la Bulia'ascofto nell'
y» Oca fen' usci delia torre, ¢ presentacafi laBalia'con l'Oca dv avanti al Re, ed
§5)° alla Regina' per domiandar' licenea; i Re'difwe: Quest' Oca-hail becco, evpri-
» ma non l'havea? £ la Balia rispose: “Non se le'era meso » perché non eras
x» ancor fatto: e Voftra Machatenga aimemoria-quel che ora ha detto,
«3° Fra pochi giorni(pird il termine', dentro-abquale\Caffandro doveva baver
“Sy” goduta Alcenia,, onde il'Re se lo fece condurre-avanti., ¢ Caffandro disse; Si-
“> te'V. Mvfacciarvenize Euripide mia Batia, I) Re lo-compiacquey'ecomparla
© Sys “Buripide'com l'Oca, fu'dalRe subito riconosciura'yed ella girdifse:°V. Me Gi
“yj 'Tivordi che @ facto if becoo all' ea; ¢ fatta quivi condurre Oca fece en-
a» trarui dentro Cafsandro, ¢ lo fece fare le solite operazioni, acciò che il Re
vj, 'Conolcefse »che quella era la steSa Oca,-che 'in quella stefsamaniera era di-
“3, morata pili giormi con Alcenia nella' Torre: onde 1! Re conosciuta l aftuzia di
Sy, 'Cafsan,¢ fay recifameateril fatto, ¢ che Alcenia era gravida, ed
“49 havea data la fede di (posaia Cafsandroyiconfermé i) matrimonio per ofseruar
la parola,contentandofi di cedere alla disposizione delcfato'; Eda questa trave-
~* Nita trasformazione diGiove in)Gigno & nato il'proverbio: E furto il becco
-alt' Oca; che significa ( come habbiamo detto ) il negozio è fatto,-0 perfezic-
snato. Questa,-o simile novella leggefi in quelle di Giovanni detto il-Peeoroas.

 

STAN-

 

 
2 ou

 

80 2M AIM ANTIAUE BD 3 2
STANZA XIV.: " STANZA sKVeb-
O ques ( disse i Duca) è veramente. Benche fuffe coftuicom! una pinay).
Da pighar con le malle; Ch unfamaras << Tanto targo, y errs af
Possa col cuore ingravidar la gente; Per non balzar.un trarto alla 2
Vedi non ti fan finta:, io non la paro «; 1 pefextori wennera in pacfes \)
For sis il prowar stom ha acoftarmente,| Così pefeando Lingo ta marina;
E quando mi coftaffe ancoben cara Leche benedett' aofino si-prefe m
Vo farlay per veder, se cid riesce; E il cnor nun bel bacina snargentato,
Pero si mandi al mar per queste pesce. 4 fuon di pive al 'Duca fu portate,

I) Duca fentendo che il cuor d',un' Afino- marino ¢raatto\asingravidarila:mo-
glie, si ride-del mago; ma tuttavia era cos ernie deere shaver figliuoli,
che volle p ze do|, che i p gediedi si-
nalmente lo prefero, e portarono il cuore alsDuca gi 139) nupvaile ho sara

E DA pighar con le male. Buna grofia minchioneria 5 ésna0,(peapotito. gran-
dissimo. elle intendiamo quello fleumentordi ferro eg pypnitor anamrsan a
boni ardenti:, ec.

VEDI, Questo termine ha del giuratorioy quali dica: oe fede mia: onescin ne
to credo, Credi a me che tu fai male, ec, Vedi foto: CyB, Nano 63.

NON Ia paro. Non la credo. Tratto dalla Riffa 'so-Maila ginoco dicdadiael
quale side tino tien la polta dice'; Parodi y ¢ nomilastenendo dice Won la pare.
LARGO come-una pina. Si dice largo com' una. Ss verde,la quale strettiflima,

¢ ben ferrata; Comparazione ironica, perché ) fargo vuol dir liberale, ed
huomo frerta "viol dire avaro »etenace; Si che fendo agi pina, verde, strettidiima,
comparandofi un huomo a quelta; sintende trettissimo;.cioé tenacissimo, ava-
rissimo, che i Latini dissero Laro facrificar; che fuona;,Glié divoto della folaga,
la quale perché @ di natura vorace y servivaa iJatini per siprimeresieileome
avido del denaro, ¢-lo dicevano Larus hians,,

JGNORANTE. Voo che non fa.) Vedi sopra C. 1. han. 73., Ma wale ancora
per ingrato y Zotico y villano, €'poco amorevole ed in quetto luogaé oo vinta fen-
fo, nekqualeé fempre » oper lo: pill, prefo-nel contado..)

PER non balvare. Cio' per non andare... Si costumaidire belesre pee. andare '
o cadere injcole di disguito, come balsar infermo in un letta,, baizare in una prigion
He y€ee Non si direbbe balzare ainn banchetros ¢ simili; -Per non balzarein wna pri
kiove j quanti noi fiamo, fara necefjario che altri di noi balzino in ae yedaieri si
Latuino in Chiefa, Difie ! Autore, che seride la vitadi quei tre si ladei Fio~
rentini.

BERLIN.A, Euna (pecie di Moment 40 gaftigo pcheifi daca i Jadroncelii
re Joro eS 1h a ae alate
an iu pape i frequentati ye quivifila ae ' in
folome tals = Quel firamento si ebiama anvora Gogna.. itMeah fo110.C. 3.

 

 

 

sear eae tisuuist 1

VENNERO inpacfe. Ciok coat earn 5 fidafeiaron, trovare..«) ean ri-
trovamento di cole ascofe 3; Ed © lo stetlo.che venire in foena deco nt.
1, stan, 2,

QUESTO benedetto Afine, L? epiteto benederto in tali occaloal a 'dir' nag

   

 
 

Fae

SECONDO CANTARE, Sa

to bramatu. To cerco del tale, del quale ha grandiffio bifogno }¢ questo bene-
detto huomo non si trova ' a & chimneys t

BACINO, 0 bacile» EB' un piatto d' nto 5/0' d'altro metallo'grande'pit:
della solita mifura de i piatti da cavola.,¢ serve propriamente per ricever l' acqua,
che fda alle manv alle tavole de'grandi, se ben s'adopta anche in molte altre occa-
fioni, ¢ per altri effetti. 4 i mE

PIVA, Dicemmo, che cosa-sia sopra C. 1, stan, 34. alla voce cornamusa, T
contadini fogliono per il maggio andare attorno cantando, ¢ fuonando la Cor-
namusa, ad effetto di ragunar denari per far'con essi regalo.a qualche !uogo pio,
¢ ricevono l'elemofine, che vengono lor fatte in un bacino, ed in un' altro'por-
tano quel tal regalo, che voglion fare; 0 vero l'appendono ad.un ramo d'alloro,
aaltro albero:, ¢.dicono.questa:lor gita., andare acantar maggio, 'Tal coftume»
tocca il nostro. Autore:con questo modo? di. portdre if cuore deil' Afino mavino al
Duca.

STANZA XVI. STANZA XVIL
Ed.egli prefoil prelibaro Cuore', Allor vedefti partorire il letto

Lodiede a! Cuatvoyalqual métre boicofse, Vn tenero: ye verzofo lertuccin';

Si fece una trippaccia la maggiore Di qual) armadio fece uzo Ripetto,
OCT sdilde' wachmai veduta fofse 3 La seggiola di ldun seggioline 3

Le robe, ¢ mafferizic a quell? odore Latavola figiio un bet buffetro.,
Anely elle divencaron tutte grofe:y xi. >. 9 Hacalfawn vago, @ piccol cafretrino,
« Ein pocasompo.4 un' otta tutte quate. Bildefirenncanteretto mando fuore,)
» Becer A! sccorda,il pargeletto.infante',. C ana bocchina havea tutta fapore.

STANZA XVILL

4, Cuoco anchegli poi non fu minchione,: Ch! in far vivande faporite, e buone 5

¢ bxce witofi n'un fiance, Fu fubso fqnifite ye molto franco,
i vedde prima uscirne uno fidione; Ein quel ch' il padre stettefopr' aparto.,
; oe Guatterinoingrembiulbianco, -  CavindinCorte, s lhi,e al terzoye al quarto
ol dette-il Cuore, al Cuoco, il. quale nel cucinarlo ingravidd, si come an-
cora tutti gli arnefi, ¢ mafferizi¢, che ne sentirono.!,odore »,¢d ana medesima-
hora Partarirondaisios sig 4 oom of 3 A

 Quivorreiry, leetore si ricordaffe che il Poeta, nel comporre ued? Ope-t
ra ha, Pe pA pce cei queldeinonslieseied alte iat n't:

contate ai Fanciulli (come habbiamo decto ) ¢ che pera (ta dentroa' termini di»

quelle favole y le qualt com: per lo pili invenrate 5 ¢composte da quelle meded-

me doanice  superare la capacita di queitey ne: di-quelli, ¢ si
contentaffe di non, inasione nel (eosin dan tuna, cola tanto favolo.
fa, ¢ faori del, natur. ¢ ih far parcorire le:mallerizi¢; ed! oftervare j che

Oe. » doreo, nel (uo Cuntorde:li- Guati

ancora, Gio, E pur fashuomo

ha pais queffa, ed altre novelle fimuli; a folo oggetto di cratteneee.li. picci-
illi 2 COME €Bhi dict.) 9! fon on asl sane) + Goons, oveMe KAO
 LRELIBATO » Viol dire una cosa gutofa 5.0 singolare,, ma significa ancora
eo eperermenta narrata, o deta avanti, come ¢ nel presente luogo, che Signi

 

fica il suddetto, 0 accennato cuore; ed habbiamo anche il verbo preibare Dan.
 Purg. Cant, 10, ' a
“WEL L toy
-

 

 
82 MALMANTILE

Hor ti rimanclettor soprail tua banco
Dietro pensando a cio, che si preliba.

4 di de nati, Non nacque mai veruno, che vedeffe un ventre maggior di quel-
lo., che haveva il cuoco «E un-termine:, che aroplifica la voce mai; V.giNeflu-
no di quelli » che sono stati al mondo, mai vedde, cc. Py? bominum memoriam.

A un' ota, A uno steffo tempo; a una medesima hora.. Vsandofi da noispefio
la voce ora in vece d' hora:: adefta in vece a' allora, Che orta éegli? ia vece di che
hora ¢ egli? ne

FECER a! accordo il pargolettoinfante, S'accordarono'a partorire a un' hora.
medesima.

LETTVCECINO., Intende piccolo lettuccio, Ma lettucciointendiamo una gran
ca(sa, la quale per di dietro ha wna spallicra,e dalle teftate i bracciuoli, sopr' alla
quale ¢ solito tenerfi uno Mraptnto, ¢ serve per riposo, ¢ per dormirvi sopras
dopo definare.

ARMADIO ec, Atnefe di legno per riporvi ogni forte di roba,il quale per lo
più si tiene affido, o aveosto al muro, ¢ si apre come le porte, ed ha dentro 'di-
versi palchetti,, o catlette; ¢ per fiperro qui intende piccolo armadio.

SVFFETTO, Intende piccola tavola.

DEST RO. Quello che diciamo anco luogo Comune, ed è quello, dove si va
a scaricare il ventre..

CANTERETTO.. Piccolo Cantero, e questo &un vafo di terra, 0 di rames
o @ aitra materia, il quale si mette dentro: alle predele 'per recipiente all' uso
fuddetto, chiamato così per efler per lo pil) di figura: simile a quel bicchicre che
i Latini chiamavano Cantharas.

VINA bocchina havea tutta fapore. Il Poeta scherza, fapendofi bene, che simil
forte d' —- fuol' efler fempre fetida, ¢ però dice che eraeurto fapore, cioè fape-
va di qualcofa.

AUINCHIONE, Vuol dir femplice, corrivo: Ma qui vuol dire uno, che non,
fa meno di quello, che fanno gli altri v. g. Se tx pigli della tal cosa, non weglio esser
Minchione, ne vag tia pigliar' anch' io, iva ae

SC HIDIONE, 0 fridione, BE -questo ultimo & pil comune; 'Vuol dire quello
firumento da cucina, nel quale s infiiza la' Carne, o Vecelli, per care arrofto,

GV ATT ERINO, Ditinutivo di'Guatcero, chet colui, che serve d' aiuto al
cuoce. Qui intende piccolo cuoco.

GREMBIV LE, Bun panno, col quale si cinge la persona forto lo omaco
per difendere il veftiro da' g)i untumi; decto così 94a 'regicoreminm; ed in altri
luoghi d' Italia Senale quia fimum tegic;-¢ moltt Zomule da Bimie,

MOLTO franco, La voce franco, che vudl dir sees eer
— un'huomo ardito, coraggiofo, pratico 5.0 to 5) ne
nel fente luogo. au30 & (tiem Oug ROL odie: Ne ola
SOPRA parte - Quel tempo, che le donne flanno nel letto:dopo'haver parto-

i cagionati loro-dal'parto pei ef ca
: uae

 

ti. AYIb50

   
 

rito, per riaversi da gli feoncerti
parto. i

 

Ou sty
STAN-

.- lt i

 

 

|
 

SECONDO CANTARE.

STANZA XIX. STANZA XX.

La Ducheffa ch' il cuore havea inghiorrito,  Crebbera insieme, ed all' adolefeenza
Cotto ch' ci fu com ogni circoftanza, Peruenuti mangiaro il pane affatto;

. Anch' ella con gran guffo del marito Nel far fanta,nel far la riverenza,
Stampo due Bamboccioni d'importanta; Hebbero il corpa 4 meraviglia adatto:
Grazie,e bellexe baveano in infinito, Tra lor yon fu mai Inte,0 differenza,
E coss grande, e tanta fomiglianra, Ma di accordo voleanfi un ben matto;
T ant eran fatti uguali,ed a capelle, L! Infante Floriane uno hebbe nome,
Che non si diffinguea questo da quello. E quell' altraeAmadigi di Belpome.
La Dachessa pure partori due bellissimi figliuoli, tanto simili di fatteze, che

non si distinguevano |' uno dail' altro, Questicrebbero, ¢ furono allevati con,

buona creanza, ¢ fra di loro cordialmente s'amarono. Vno di efi hebbe nome
P:-Jnfante Floriano, che vuol dire Raffaello Fantoni, ¢ l'altro Amadigi di Bel-
pome; E queito è nome a cafo.

ST AMPO' due bamboccioni a importanza. Pastori due bellissimi figliuoli, ¢ che
havevano tutte Ie condizioni, ¢ parti desiderabili; B nota che il termine d'impor-
tanza usatifiimo da noi ia simili occasioni, vale in questo cafo quanto il termine
di garbo, ¢ per esprimere una tal quale perfezione del fubietto. Li Lalli Za. Tr,
C. 1, stan. 54. dice.

E produrra, se ben non senza duale,
Due garbars bambucee a xn party fal.

ef capelo. Perl appunto. E il latino.ad wxguem. Termine usato da coloro,

che si regolano col filo nello fquadrare, come sono i muratori, ec. E vuol dire

non vicorre la grofieza d' un capello dall' uno all' altro; ma si ula in ogni con-

giuntura di paragonare, 0 milurare una cosa conl'altra, non folo in quantita,

coine Ho riscontrate i denari, ¢tornano a capello; ma auche nella qualita come nel

cafo nostro, che s' intende: erano uguali di mole di corpo, ¢ simili di fatteze.

MANGLAR il pane affatto. Mangiar bene, ¢ senza far rofumi, 0 tozi; mas
significa huomo di buon pasto. Vedi foro C. 8. fan. 56.

FAR sand, E10 fefio, che far la riverenza; ma éun termine, che ¢ proprio
dei bambini,, ido comingiano a imparare.a andare, che quel lor muoversi
timidamente ¢ detto dalle balie far /auta, 0 pure è, quando fanno la riverenza

-baciando altrui da. mano; ed¢ così detto eae fanita, cioé fare faluce; falu-
tare« 'Diciamo infegnare al Bue far fanta per intendere: Zufegnar le feienze, oi ter

saini ciyili aun' bueme xatico, villano, ¢ di difficile apprenfiu

83

 

 

SJ wolenano ne ben marto, S 8! 20 fuile - Equel
termine A4attus, del quale habbiamo dexto sopra C. 1. stan. 76.
STANZA Xx STANZA XXIL
drrivati che furono ambiduai Di modo che fdegnato, come ho detto,
A conoscer bomai il pan da' faff, Ch'il Duca per La sua spilorceria
 Efaper quante paia fan sre busi, Og hor vie piie tenevalo a freccherts,
Se ben dal padre havean de gli spaffi, Vn di fri ed andar via,
Vedendofi gid grandi impwcatoi Ma tdcquelo per fare il gioco netto,
Ed a soldi tenusi baffi baffi, Fuor ch'al frarelioyal qual n'unaofferia
Oftico gli pareva, e molto strano, Difse(veduto havendo 4 un fiafeo ilfide)

Bd ia particolare a Fioriano.

Volerjene ramingo andar pel mondo,

 

 
 

84 MALMANTILE>

Cresciuti quefii due Giovani, ed arrivati a condscer il:ben.dal nialé, vedendofi
così grandi pareva lor malagevole il hon haver denari 5 perché il padre perlaifua
spuorceria non gliene davaydi che'pid d'Amadigi sentiva disgufto Piorianosonde
si rifoiuette dsandatovia 5 ¢ perché: ' adempimento di tal faa risoluzione non gli
folle unpedito, non ne parld.ad alcuno, fuori che al fratello Amadigi,
CONOSCER il pan da faffi;\efaper quante paia fan tre buoi, Significano Jo stel
fo, cive conoscere il ben dal male. Hor, disse, Novit quid dient era tupinis Si
dice ancora in quetto proposito Sapere 4 quanti di ¢ San Biagw, E quetto'denoha
origine da.un coftume antico, il quale era in Pirenze, che i ragazi fattori délle 
bosteghe d' arte di feta:, che (on fituate nel Mercato Nuovo vicino alla' Chiefa di
5. Biagio, havevano dicenza, paflato il di della fella di elo Santo ( che fax
sebbe alli 2, di Febbraio, ¢ se ne fa alli 3. per causa della Purificazione, il che
ha daso occasione di ulare questo dettato ) di fare alle falace, € pigliarsi ogai
forte di paflatempo in alcune hore del giorno, ed abbaadonare la bottega per in<
fico a. tucto il giorno di Carnovale; e per questa causa era quel giorno tanto defi-
derato da i ragazi, che fapevano benitfimo il di, che si tolennizzava la deta
f€sia; onde colui, che non fapeva tal giorno, era fra i ragazai ripucato ua bag=
&<0, ¢ che non havendo notizia delle cose del mondo ( giudicata da Joroquelta
una delle più importanti ) non fufle persona abile, ¢ di tanto°giudizio da faper
fare i fact suoi. E questo proverbio s' è fatto poi comune a tutti gli huomini per
intendere un' huomo (ceruellato, melenfo,e buono a poco. Il Lasca Nov, 4. dices
La Stheggia yed t-Pilneca\, che. Sapevano a due once, quanto colut pefava, ed a quanti
dit San Biagio.
SE ben dal padre havean-de gli spafi, Se bene il padre dava loro de gli avverti-
menti., ¢ paflatempi. Nota che per scherzare il nostro Poeta, subito che ha det-
to duoi (eguita dal padre, ¢ questo fa per coccare quel coftume burlesco; il'qualeé
iu Firenze (ma pero fra gente:bafla ) che quando uno nomina bae, beccos 0 ca-
firone,Valtro dira di tuo padre, edicendo vacca,dira di twa madre, e simili, Vedi
forto C, 12. stan. 49. annot.al termine wmorirecon la grillanda
GRANDI impiccatai. Proibiscono le leggi Y-impiccare chi non paffa 18/anni;
¢ di quinel diciamo.grandi impiceatoi, cioè abili a elicr*impiccat, per antender
guelin,y che pafiano la decca eta dir8. anni. $ ”
st SOLDIL tenmti baffi bai, Tenuti con pochi denari, Traslato dail' acque,delle
guali quando ne fon poche nei laghi, pozzi, o fiumi, si dice bafe. Vedi forto in
quetlo C, stan. 61. € parlando d' uno che habbia pochi denari G dice: iL acgue
Jon bafe si come intese colui con quel suo motto ZL' acque Jon baffle, et! ache banno
gran fete, cioè Alle gran veglie i danari fon ne « 2:
SOLDO. Vale per intender danari,riccheza..E soldo moneta immaginaria
(hoggiin Firenze eftettiva di bronzo)che vale tre de nostri quattrini;Spetio usiamo
quelto termine per una certa.generalita: Il tale.ha de' soldi,de' quatcrini,dell'oro,
Rer intendere € ricco» nonche habbia quantita di soldi, di quattrini,.0 dro ef-
fectivamente yma molti-ne vale il uo stato; Equi intende-Monete «-—
. OSTICO., Spiacevole, Malagevole, lnfopportabile. E il Latino J he
vale per cosa da nimico.
STLANO. Qui ha lo stesso significato a' ofico. Vedi (orto C, 3, stan. ae

. oz ' Te tro

 
 

 

 
 

SECONDO' CANITARE. 85)

altro vuol dire stravaginte da eatranens. E molti dicono' rate ajuno che habbia
cattiva cera ye perinfermita sia mal condorto. + } Hal
-SPILORCERTAS Sordidezza, Avarizia. lo credo che questa parola venga da
Pilorci, che i pellicciai chiamatio ao ritagli di pelle,' che non eflerido-buoni as
metter' in opera ygli-riducono'in spazzatura, la quale poi veadono per governa-
rei terrenijse li dica /pilorcio quali huomo vile,ed abietto-quato sono questi pilorei.
“ TENER' uno aftecchetto, Bare flar'a fegno, '0 far patire uno di quello; che»
églicha bifognio; come'non'lo taftiar mangiare 'quanto ei vorrebbe; 0 haver de'
danari quanti bramerebbe ¢ Quand' uno per la(carfezza di danari vive mifera+
mente si fuol dite: Atele ( difende?, si febermisce, ec) ond' io ton fon lontano dat
credere, che questo termine sia corrotto, ¢.che*fildovelie dire'a focbherro da stac-
cheggiare, che è l'iftetlo che schermirG, ¢ pud significare Eifere (carfo, 0 haver.
bifogao di denari.
' VEDVTO il foudo a un fiaféo, Dopo haver bevuto un fiasco di'vino;'e così ha-
ver veduto il fondo'di dencro/del fiaico; ed in fuftanza qui-vuol dire; Dopo ha
ver bevuato molto'bene; ovaflai. a

ANDAR ramingo pel mondo, Andarfene errante. Ramingo vien da ramo,¢
si dice Ramingo de gli vecelliidi Rapina, come elprime dl Crescenzio nel Cap, 3.
della *bonca degli Sparuicri lib. 18. con le seguenti parole: Si chiama nidiace, v
wero che di nidio uscito di ramo in ramo va seguitandola madre,e pero fichiama Ramingo,

Ed alli sparuicri & danno tre nomi, cioè Widiace, che & quello, che ¢ cavato di
nidio., ¢dallevato., amingo quello che u(cito di Nidio non fa gran volate; e>
Grifagno quello', che già patiato I" anoo ha mutato alla Campagna. Ma quelte
aoateat 'noitro:, baftandoci, che a' similitudine*ditali uccelli,, dicefi
Andar ramingo coli; che hora va in un luogo', bora's* incammina ip un' altro,
senza fapere politivamente, dove egli vogha andare, '
&. STANZA XXII

Anadigi 4 distoris tutto un giorno Tn vnoiir difse se verote vain un forno:
Sr arrabbio, s aggird com'un Paleo; E dopo un grande, è lungo piagnifteo;
Ma perché quanto peu eli ava interno HHorsn-vanne(difs' egli)io men' accordo,
Egii ord piu'oftinatod uno Ebreo, Ma lasciami dite qualche ricordo.

» Amadigi (entita questa risoluzione del fratcllo, molto s'affaticd per distornelo;
ma veduto'che per la di lui oftinazione s' affaticava in vano, concorse con lui,
con questo però che gli la(ciafle qualche ricordo'di se,
 P-ALEO Cosivchiamiamo una specie dt erba', che nasceintorno alle lagune.
Ma diciamo anche Palcouno strumento di legno, che (erue per traftullo, ¢ giuo-
co de' ragazzi, il quale ¢ di figura piramidale al' ingiti; e nella teftata, che viene
+di sopra ha'ua manichecto condo:, il quale.avvoltato con uno spago, 0 cordiccl-
las' infila in uo' atlicella,bucata,e tirandofi quello (pago fifaolta, ed i Paleo feap-
»pa dal buco dell' aificella', © va per terra girando,portato dail'ampul(o di quelio
 'spago. Tale dtrumento da i-Latinié detto Tu*ho forse dala figura piramidales.
+ WVerg. 7. Aneid. Cex quondam torto volitans fub verbere turbo, T ibull, Nam

ue aSOry
“at per plana citus fola verbere turbo, Dance nel Paradifo C, 18. a
Ed al nome del alto Uaccabeo
Vidi moversi un' altro roteanda, =
E letizsa era ferza del paleo « i EG

   
-
86 MALMANTILE

E dice,così, perché a tale strumento si fa continovare il girare perquotendolo
con una sferza, dopo che egli ha hayuto il primo moto, ed.impulfo dal fuddetto
spago. Ed il proverbio aggirarsi come ux paleo vuol dire affaticarsi aflai, ¢ conchiu-
der poco; che i Latini pure difleru Trochi in morem circumagi, perché dicon Tro-'
chus tanto il paleo, che la trottola, portandolo dal Greco Treches, che vuol dir
ruota, 0 altro strumento che giri. Vedi forto C, 6, stan,22. E forse ancheJa yo-
ce latina T-«rbe significa tanto il paleo. che la trottola, perché Turbo vuol dire
ogni cola che habbia figura Piramidale, a rovescio, cioé il largo di sopra,.¢ da
piede acuta, come appunto ¢ il Paleo, ¢ la Trottola; se bene non (ono Jo fefio
come ci teftifica una certa cantilena aflai praticata fra i ragazi, che dice,

E il Criffiano non ¢ gindeo.,
E la trottola, non ¢ paleo,
E paleo non ¢ trottola, ec, q

PIV? oftinato d' uno Ebreo, Oltinatitfimo, che non si trova nazione più oftinata
nella sua legge, che quella. de gli Ebrei, che pero ha meritato)il titolo 5 che le da
la fanta Chicfa di pertidi, Cino da Piltoia, O vei, che fere wer me sigindes: ciok

erfidi,
. VA in nn forno, Va dove tu vuoi. E specie d' imprecazione, che fuol far' uno
vinto dall'impazienza, E si suo] dire anche in questo proposita: Vain malorayva
al diavolo, va in galea,¢ (mili, Abi in malam erucemse Plaut, Epid, Ato ts se.2-

ditle; Atala iff usmodi mihi amicos furna merfos, quam fa
XIV. Ss

STANZA
Allor per fadisfarjo Flariano,
Accio che più tener non Labbiain ponte,
Con un baften fatato, c' hayea in mano
Tocco la Terra,e fece uscirne un fate »
E disse: Quindi poi ben che lontano
Vedrai sto vivo,o s'ia sono a Caroute;
Perché quef'acquagzn' or di pina inpito
In che grado so faro diratti appunto,
STANZA XXV.
Sal corso di que? acgua porra cura,
Tutto il carfo vedrai di vita mia;
AMentr' ella ¢ chiara, criftallinase pura,

foro,
TAN ZA XXVI.
Cio dette in capo il berrettin fiferra,
Aerte man,chiude gl occhi,e frrige i déti
E da si forte una imbroccata in terray
Ch' il ferro entrovvi fino ai fornimiti.
La quel che i grills,e + bachi di forterra
Sgombrano tutti i loro i.
Pullula fuori un cefto di mertella,
E di nuovo Florian così favella
STANZA XXVIL
Fratel mio caro, questa Piauta ancora
Com' io la paffi ti dara epee ao
Cioè mentr'ell'e verde,anch' io allara

Di pur ch'io vwva. in feftaged allegria; Son vivefresco,e verde com' un' aglio;
Ed all incantro, se torbida, e feura 5 E quand' ella appaffisce, efi coloray

Ch ella mi vacome dices la Cia; Anch'io lagui/ivod ho qualche travaglio,

Ma k gusapcdadlenste Sern il corso, In somma sell' fecea, leva i moceoli.,

Di ch'io sia itoa veder baliar LOrfo. Per farmidire ilcantoinfscarpezoccoli.

Ficri wo per contentare il fratelio, toccd la terra con un baftone incantato,

; che haveva in mano, ¢ ne fece.na(cere una fonte, ¢ disse che dalla mutazione di
guell' acque haverebbe egli conosciuto lo stato, nel egli f tovafle. Dip

mefle mano alla spada, ¢ con ¢ffa buco la terra,.¢ scappo tuori. anor-

tella; E moftrd ad Amadigi, come egli si davea contencec in conalcere ancora.
; da questa mortella, in che grado egli si trovatic. i

 

<= ¢ pts act a |

 
SECONDO CANTARE: |

87

TENERE in ponte. Tener un sospefo, o irrefoluto. I Latini pure differo: 2

pdetinere; ¢ però ttimo, che questo nostro detto venga dall' uso antico de'
omani, che nell' clezione de i Magiftrati chiamavano Pontes quelle piccole ta~

yole, sopr' alle quali eran posate le paniere dei voti; di che fa menzione Cic. 1.

Rhet. Pontes diffurbar, Ciftas deijcit; ¢ canto Ravano incerti, ¢ sospefi coloro, che

devano; quanto le cefte de i voti stavano sopra i detti Ponti; E' pero di-
cendo: Ego /um super pontes, vaol dire il mio Voto è ancora nelle Cefte, o coper-

to, ¢ per conseguenza io sono folpefo, ed incerto di que} che habbia a efler di

me, Eci serve poi questo detto Tener' uno in ponte per esprimere; trattener' uno

con le-speranze, 0 con altro fecondo il fubictto. '

SONO a Caronte, Son morto. Son fra l'anime', le'quali paffano la Barca di
Caronte, che fecondo la faifa credulita de'Gentili era il Navaleftro,il quale con-
duceva | anime de i morti con la Barca alla Città di Dite. Vedi forto C. 6. stan.
19: & feqq.

COME dicea la Cia, Miva male, ¢ peggio. Che questo voleva inferire una
tal Ciay © Seia Fruttaiola con un detto sporco da lei molto usato.

SON itoa veder ballar tOrfo. Anche questo 'detto significa fon morto.

IN cape itberrettin si ferra yec, Con guefti due versi esprime uno, ches' accin-
ga a fare un' operazione'; nella quale sia neceflario ular molta forza', perché-ia
efi: moftra quelle azioni, che per lo più fon (olite farsi in simili congiunture.

METTE mano, Quando ditiamo aflolutamente meteer mano; intendiamo met-
ter mand allrarmi. Diffringere enfem.

iene a via; mae it '

'qui comm pare i proposito il norarewuna ja generale portata dal
varehi 'nel fo Hercolano; civt che la lease tain hel frei qual-
ia dizione ne} nostro parlare ha la forza di privazione, come ai

Latin la particela m ha forza di negativa, come doftus, indottus, ec. Ed appref-

fo di noi eaitare y fealuare ec, Ha però questa regola anch' essa le sue eccezioni,

come sbilordite vitol dirbatordo, € non vuol dire senza balordaggine; T urbare,stur-
bare, diffeobare, che faonano'lo fieflo con l'aggiunta, che senza. Taluoltas
anitor' s* aggiunge alla 'deta yS), la particelia a, e particolarmente quando la,

eee ens vocale, come amare, difamare'; intereffato, difintere/-
Salto 5 0 ORES.

* CESTO', Intendiamo pianta di virgulto, o & erba, come Cefto di lattuga, di
mortella-,-ee.' Se bene de ¢ virgulti si dice anche Pianta, come si vede nella pre-
fente ottava 27.Fratel mio caro queffa Pianta ancora, Viene dal latino Ce/pes, ¢ noi
pure diciamo' Cespugtio. lo fimo, che pianta sia nome generico, poiché serve,
per tutti li vegetabill, dicendofi Pianta di prezemolo, pianta di grano, e pianta
di oe, €¢. € noni direbbe di tutti cefto', ne cespuglio.

'RDE come un' Aglio, Vn bel verde si paragona ad un' Aglio, perch? questo

ha le sue frondi di bellitiimo color verde,€ che f) mantengono ver=.

di, ¢ fegno di sua pérfezione » E però dic Ui tale ¢ verde come un' aglio, s'ia+ 4

tende; ¢ di fanita perfecta' cruda Deo, viridifque fenettus, Horat, Dumques

virent genna, Questa bmi si piglia da tutte le piante, la fanita delle quali

8' argumenta dail' efler ben verdi, che dimoftra non havere esse patito, ne eflere
12

 
 

 

MALMANTILE?

in grado di feccarfi'. Edialle volte s' intende uno di-mala'fanita quando fi'dices
verde come. un' aglio, mas' intende non la frescheza, che denota il verde delitaglioy «
mail colore, che efiendo verde neila faccia dell' huomo denota pocayfanitas 10
LEV.A i moceoli per farmi dire il canto in fearpe,e xoceoli,Compra la cerayper far-y
mi il funerale:: che moccoleyuol:dire ogni piccola candeladi-céra ye quit prefol
per ogni forte dicandele di cera'. B quel farmé dire il canto fearpe Zoceoli & detto.,

gioco(o usato fra 1 nostriContadini; 1] cual.detto non è forse senza fondamento
neaffatro: improprio, che posia haver origine dalla diligenza, che si pone nel fae,
che i morti quando son portatialla sepoliura habbiano y se sono huemini un parr
di scarpe nuove,e se fon donne un par di pianelie,o zaccoli puovi; eRveco/e\e-unay
scarpa col fondo di legno, che serve pen difendere i piedi dall'acqua, che¢perterra.
Ss 0 s hovel

TANZA: XXVIII: TAIN ZiA>-X SAX. >
Poi che queste parole hebbe finite, è Eth prima giorno fece tama via y\ 9.09 0h
Dal suo caro eAmadigi si licenza y Chi suoi Lacché spedati, e conci male
A qual rimafe tutto, sbigortito, Sirimafero 5 Punoall' offeria,
Pero che gli dolea la sua partenza, c Ent altro fearmanato allo spedale;
Quand' in feha Florian di gid falito 9», Ona' ¢i più non havende compagnias'.
Senza gran doble, o lester di credenza Se bene accanto havea spadase pugnale;
Andonne abenefizio di natura i 'Per non baver paurain andarfoloy
Con dug ferni cercande faa ventura,, » Cantava ch' ci pareva unrofignnelo,
exgh onssiber wid AD CAL MMe obyteD oir a \ '
Così muove canzoni ogn' bor cantando Onde ai timori al fin dato di bando.
Con una voce tremolante in quilio, on Tirava innanzi il voiontarwefilio 5,
E quaiche trillettin.di quando in quado E ginntova Campi, li fermar si volle
Alle fielie n' andava ye in vifibilio: A bere, ¢ far la Rolfe per bi malle, -»

Floriano si parte dal fratello Amadigi, il,quale ne rimafe aiflitto. Lalcio per,
la flrada i Lacché stracchi,ed egli folo si condufle a Campi, dove si fermé a bere.
SSIGOTTITO, Afflino; perduro d'animo. I Latini dissero «daimo deietius,
Quand' uno sia allegramente diciamo: Il tale fa in.gote, 5.0 fha in barba di micio.,
Vedi in questo C, stan. 48. Si che uno che non stia allegramente si dice vom /ta im,
gore, non sia in barba di micio;.E però non farebbe gran fatto, che questa voces
shigettito venifie dallo Spagnuolo bjeorses 5 che vuol dir bafette y,¢ che per-ia lette-,
ra, $, che aggiunta al principio d' uha parola ha forza di privazione (come,
habbiamo detto poco sopra ) significatie senza bigorres, che vuol dir (enza balette,
cio¢ non in barba,, non allegrameate: 0.forse sbigottico,quafi sbattuto...
(od BENEFIZ10 di natura. A cafo; dove la Fortuna lo.guidava... »
LACCHE', Servitori, che corrono.a pié:; ¢ per lo) pil fonowwagazzi
yanetti. Vedi forto.C. 11. stan. 9. 51k HObHat
SPEDATT., In questo cafo non vuol dir Senza,

" eftanchi dal viaggio. Be it by AV Svan: sets ah
Scakddanart tna pecie d'infermitd, che viene. c:
'caldau per violente fatica, 0 viaggio

   
  

 

 
 

che dope esserfi foverchiamente Hi orate
freddano 0 col bere.0,conJo (tare al vento, 0 in luogii frc(cht;¢ si dice +) és
gliar una fearmana, 0 fearmanare.«.E forse specie di quel maics che i medici chia~
mano Pleusitide,edé comunemente chiamato wal di petto.Qui sotcndh, ABBEY

 

 
 

SECONDO CANTARE: 89

dal viaggio, in maniera che l'anelito se li rendea difficile, ¢ però 'non poteva-
no camminar pil.

CANT AVA che pareva un Rofignuolo. 1) Rofignuolo, Vecelletto noto, da i
Latini detto phifomela, ha il più bello, ¢ gagliardo cantare di qualfivoglia Vccel-
letto, e per questo quand' uno canta bene, lo paragoniamo al Rufignuolo.

VOCE tremolante. Voce, che tremava per cagione della paura; Si come'i tril-
Ki eran fatti per timore, e si potevano dire pil tofto tremoli, o interrompimen-
ti di canto cagionati dalla paura', che veramente 77id che sono un riperquoti-
mento di voce musicale nel medesimo tuono. Horazio disse: Cantu tremulo.

LN quilio,.. Secondo che mi disse il Signor Nigetti, fra i mufici del nostro feco-
Jo il Maeftro; la voce quilio significa un cantare in voce non sua, come se uno
havefle voce di bafio, e cantafle di soprano; Si che s' intende, che Floriano can-
tava per la paura in voce falfa, enon sua naturale, che i Latini fecondo Cic. lib.
3.de Orat. la dicevano Vocula fal/a.E Titinio appreflo Fefto ditle Succrotilia vocula,

ANDAR alle feelle col canto, Cantar in tuono alto. Se ben qui par che voglia
dire, fen'.andaya in gloria, cioé cantava con gran soddisfazione, ¢ gusto; poi
che foggiugne én vifibiie che appreflo di molti de' nostri vuol dire Andarfene in,
eftafi, ¢ perderei sentimenti per il gran guflo, Matteo Franzefi nel Cap, del suo
viaggio da Roma a Spoleti dice.

Vedea pafsar con toruo supercilio
Qualche Sarrapo tronfio, ed appoggiato
il tappeto, n° andava in vifibilio.

Vergilio Egl..5. ditle: Voces ad Sydera iattare,

Ed ottavo Ma. Effundere voces ad athera,

TLR AVA innanzi il volontario efilio, Continovava il viaggio, che egli medefi-
mo s' era eletto,cfiliandofi dalla propria casa.

BAR la zolfa: Detto scherzolo, che signifi a Cantare, far musica, ed ¢ com-
posto di tre note musicali, la, fol, fa. Ll Signor Salvador Rofa in una sua bella
Satira parlando della musica dice,

Quanta gira la terra a tondo a tondo,
Lago alcuno non v' ¢ che di schiamarzi
.) «Edi xolfe non sia pieno, ¢ fecondo,

PERS mole. Ib molle ¢ chiave musicale, o fegnatura di femituono; Mas
qui dicendo far /a xolfa per b molle, si serve della voce mulle per incendere: am-
mollare la bocca, cioé bere, E così scherzando sopra alla musica, ed havendo
detto, che Floriano cantava; foggiugne,, che voicva seguitare a cantare anche
nell' ofteria, ma per b molle, ed intende Vuol bere.;

STANZA XXXL STANZA XXKIL
A Campi, hora spiantato alla radice Com' io diffi, Florian nella Cittade
Dominava in ques i Storditano, Entré per rinfrescarsi,e tocear bomb,
Se ben Turpine ferive', ed altri dice, — Mail gra fraftuono,cb in quelle cotrade
Ch' ei regnaffe in we luogo piss lontanos —— —- D'arini,di beftie,e d'haomini rumbomba;
Hebbe una figlia detta Doralice, Al sentir fu pei canti delie rade
C'bavea un'occhioc'uccides il Criftiano, Tute a cavalio rifuonar la tromba 5

Ma quel che pin tirava la brigasa «Ed il voler faperne la cagione,
El cffer fola ye ricca sfondolata, M = Lo fecero mutar a' opinione.

  

 
p

as

he

i) MALMANTILE

1) Poeta finge Città Regia il Castello di Campi, luogo vicino'a Firenze y che
hoggi ha poca forma di Castello, per efler distrutto, ¢ dice che già vi regnava
Stordilano, che hebbe una bellidima Figliuola nominata Doralice, 1a quale per
etier fola, ¢ ricchissima, era da moiti bramata in moglie. E perché questa non
sia creduta la stessa, che quella che l' Ariofto fa Figliuola di Stordilano Re di
Granata dice: Se ben Turpino ferive, ed altri ( cioé  Ariofto ) dive, ch' ei regna/-
Se in un Inogo pitt lontano, coe in Granata. \

Floriano dunqgue, il quale era entrato in Campi folamente per pigliare un po-
co di riposo, e rinfrescarsi, e andarfene, fentendo tanui strepiti d' armi, € ro-
mori di tamburi, si risolue di trattenerfi alquanto per intenderne la.cagione.

HiVEA un octbio c' ucvidea il Criftiano, Havea così begli occhi, che facevano
innamorare ognuno. Questo detto vien forte dalla comune opinione di quel fer-
pente da i latini detto Regulus, ¢ da i Greci, ¢ da noi chiamato Bafilisco, 11 qua-
le col folo fguardo avvelena, ed ammazza coloro, che egli mira. E moiti Poeti
nostrali per ledare l'occhio di bella donna hanno detto: Occhio di Bafilisco, in-
tendendo, che han forza di metcer nel cuore il veleno d'amore. Apul. morficans
tibus oculis,

TIRAVA la brizgata, Lufingava, incitava, allettava il popolo a desiderarla -

RICCEA sfondolata, Ricca senza fondo: Ricchiflima. Diciamo Ricco in fon-
do, senza fondo, sfondato, 0 sfondolato, per denotare una ricchezza. senza nume-
ro,omifura.

RINFRESC ARS, Ciok reficiarsi col riposo, e col cibo. I Latini pure dice-
vano tal volta-rinfre/car/i per ristorarsi,trovandofi refrigeratus in vece di refociliarus,

TOCC AR bomba, Arrivare in un luogo e dimorarvi poco. Questo detto &
tolto da un giuoco fanciulle(co detto birri e ladri, il quale fanno in questa manic-
ra. S'uniscono molti Fanciulli, etirate le forti a chi di loro debba efier birro,
chi ladro, quelli che ono eletti birri si mettono in mezzo della stanza, o piazza
dove s' ha da fare il giuoco, ¢ ciascuno de i ladri piglia il suo posto, il quale &
già stato consegnato per immune; ¢ questo luogo da essi è chiamato bomba, che i
latini dicevano mera in questo medesimo giuoco usato ancora da i loro ragazzi, ¢
da quelli de i Greci, se beac in qualcofa differentemente. Questi ladri vanno
scorrendo da ua luogo all' altro, e i birri procurano di pigliargli, ed i ladri,
quando si veggono stracchi, corrono a trovare un di quei Juoghi immuni detto
bomba, dove ttando, sono franchi, ed i birri non poslono pigliargli, e si guada-

gna, 0 si perde il premio stabilito,fecondo che fon convenuti d' efier prefi,onon

prefi in tante gite; ed il ladro prefo ( continovandofi il givoco ) diventa birro,
ed Ml birro, che ha prefo diventa ladro. E perché nel toccar bomba si trattengo-
no 3 pero diciamo toccar bomba per ¢sprimere arrivare in un luogo, ¢ par-
tirfene prefto. E questa voce bomba vien dal Greco bombeo, che vuol dire Strepita-
ore, 0 far fuono, ( donde rimbombare ) è da quel romore, che fanno i ragazzi con
la voce, ¢ con le mani per far conolcere che toceano i} luogo immune, questo
Juogo è¢ chiamato bomba. Diciamo tornare a bomba che significa ternare al primo
discorso-, Vedi forto C. 8, stan. 15. ws i

FRASTVONO. Fracasso, Strepito,romore confufo, quafi dica fuor di tuono.

CANTO. Cioè l' angolo che fanno le cafe.a capo a una strada che eae

Q — van"al-

 

 

 
SECONDO CANTARE: gt

ian' altra; detto così fecondo alcuni, dal Greco Canthos, che vuol dire Angolo
dell occhio, o dal canto, che nello sboccar delle strade in su le cantonate folcva

farsi dagli antichi, come si cava da V

. Egl. 3.

Won tu in trevijs indotte folebas
Stridenti miferum stipula disperdere carmen ?

Ma è detto dai Greco camptin, che vuol dire Piegare.

TVTT 14 cavalo, Così chiamano i Soldati quelia suonata di tromba, che fa in-
tendere a i medesimi il montar'.a cavallo, la quale par che esprima; Tati a ca-
valle. Coftume tolto da i Latini, che per significare il fuono della tromba dice-
vano fecondo Servio, ed Ennio Taratantara.

At tuba terribili fonitu taratamara dixit.

STANZA XXXIIL

Era gta feavaicato ad una Oftefa,
Per far, si com' ei fece, un conticino,
Ne altro bebe che pane,e capra leffa,
Che fitra anche gii fu per mannerino
Bevve al pore una nuova manomelja,
Perch' il vinaiohavea finito il vino;
Fece conto, ¢ pag ben volentieri
Poi chiefe il fin ditanti Strombettieri,

s

TANZA

Ma c' occorre ch' in cio pitt mi distenda,
Mentre 1a cosa è tanto dinulgata ?
'Pero lasciami andar,ch' ioho faccenda
Havendo sopra un' altra tavolata

STANZA XXXIV.

Ella rispose: E come; E non lo fai?

Se per Campi non è altro discorso,

Che havendoil Re una figliaye' hoggi mai

eAbbraccerebbe un' hué prima Cun or fo;

E percht reda ell' ¢ bell', ¢ d' assai,

Di pretendenti bavendoungrancocorso y

Bandire ha fatto, acid nefun si lagni,

Chin gioftra chi la vuol fela guadagni,

XXXV.

Dicé Florian che ai suoi negoxzi atteda,

Scufandofi d' haverla feieperara

E rimeffa la brigha al suo giannetto,

Come un pardo faltovvi /u di netto,

Floriano eflendo scavaleato a un' ofteria, dopo che hebbe mangiato, e pa-
gato intese dalla padrona dell' ofteria, che quei romori di trombe si facevano
perché il Re voleva maritare la Figliuola a quel Cavaliere, che meglio si portafle
1p gioftra; onde Floriano monté subito a cayallo per andare a veder quetta fefta,

F ARE un conticino, Così usiamo dire per farsi intendere copertamente Andar a
mangiare all' ofteria.

FITTO gli fu, Gli fa fatto credere. Gli fu dato ad intendere che ¢' fufles $
Mannerino., Il verbo ficeare usato in questi termini serve per esprimere, che:
quellatalcofa fudata per maggior prezzo di quel che ella valeva,o per di miglior eS

+ ualita, che ella non era. Vien da ficcar carote, che vedremo forto questo Cant,
s jan. 70, € Cant. 6, stan. 68. Lat. imponere alicui.
i MAN. a ie d' agnelli castrati, che nel!a nostra Toscana è ottima
nel Territorio., econtado di Piftoia, ed ¢-carne (quifica al contrario della ca-
pra, chee ja pepgione; che si mangi, ed.in particolare cotta a leffo.

MANOMESS A, Quando all' Ofte arriva portatogli dalla montagna il vino
primo cavato dalla,botte si dice: 2 offe ha bausto la manomefa, Ed i Fiorentini, ':

othe fon di buon gusto,o pil tofto ghiotti nel bere, lo pighano pitt volenticri,

quando è vino di » non tanto per la curiosita di gustare quel nuovo

vino, quanto perché non piacendo loro le fondate,hanno caro di bere del primo,

che esce della botte, onde pare che il = voglia intendere, che Speene fes
14 2 ene

“a4

 
 

or MALMANTILE

bene bevve acqua hebbe nondimeno gusto,, perché era nuova manomeffa,, maia
essecto gli da la burla dicendofi che bevve una manomefa nuova cioè infolita, nons
efiendo solito, ne coftume, che si manometta il pozzo, se non per le:beltie.
VIN AIO, Ciok colui che nell' ofterie da il viao. Per maggior intelligenza di
questo è neceflario fapere', che nell'Ofteric di Firenze stanno due maeltri, ¢ ten-
gono garzoni differenziati; Vno di questi macftri ¢ il padrone principale ed in
Jui dice l'Ofteria, ¢ questo si chiama il Vinaio; altro ¢ macltro anch' egli, ma
folamente della Cucina, della quale paga un tanto il mesc di pigione al Vinaio,
dal quale pud etier mandato via. Ho voluto dir questo, perché s0 che a i Fore-
flieri è di non poca confufione questa distinzione, perché si fanno'far il conto da
uno, ¢ peafanao d' haver finito; gli sopraggiugne poi il fecondo Olte, che fa lo~
ro il conto della Cucina, e cresce la somma del primo conto fatto dal Vinaio.
FECE conto. Domandd quanto dovea pagare.,. Trattandofi d' ofterie Far con
to stintende Haver finito di mangiare.
ST ROMBETT /ER/, Incende il romore, che fa il fuono delle trombe.
ABBRACCEREL BE un huom prima c' un' orfo, Così diciamo d' una Fanciulla,
che sia in eta da.maritarsi, ¢ che sia bella, grande » ¢ ben formata, intendendo
che sia in eta da bramar ? huomo., ¢ da distinguerlo.da un' orfo 5 0 da 'non fug-
girlo, come farebbe all' orfo. Virg, am maruraviro, plenis © nubilisvannis,
*" D' -dSS-AL, Valente,contrario di Dappoco: pare che fuonio stetio chein la-
tino preffans.. f
REDA, Vedi sopra in questo Canto stan, 12. Quié prefo nel suo Poe si-
gnificato d' herede,o fuccetiore nelle faculta; ¢ vuol dire che essendo ella Figliuo-
la unica del Re, dovea hereditare tutto quello:che ¢gli posledeva., vise
TeMVOLAT £, Così chiamano li nostri Ofti tutti coloro, ehe 'vanno a 'man-
giare alle tavole delle loro ofterie, canto se fuffe un folo per tavola, quanto
se fuflero più, pur che feggano a mangiare a tavola..:.
SCIOPERAT A. Levata dal lavoro, o dall'opera.. Vedi sopra C, 1. st. 29,
GIANNETTO., 'lntende cavallo, Sendoi giannetti specie di cavalli», che ven-
gono di Spagna del paefe d' Afturia., e perciO dai Lavin detti4spurcones ) -
'P-ARDO, I Gatto pardo¢ animal noto, come ¢ anche nota la di tui feroces
agilita, e deftrezza; e pero-appreflo di noié in uso questa 'cemparazione quando
vogliamo intender l'agilita di vita d*alcuno3 Vedi sopra C. 1, stan..11, Le /oale
corre lefto come un gatto.. vita ol

STANZA XXXVL STANZA XXXVIL
Tocca di/proni,e vanne,e giunge in pi Floriano in comemplar facciass) bells
'Dovietlibiatatese che riafer lagiofira, > Dave quel evade, bale de

Che per vedere il ws AY Rib y
B apeen | Cova factanie asia.”
Sedevail Re presentela Ragaca,
Che quanto adorna,e bella si dimoftra,
Tanto è confufabavido ahaver coforte,
Won afuo mo ma qual verra ta forte.,

 
Le ee eee

SECONDO CANTARE.

i è STANZA XXXVIIL,
Po far'\(dicea) che bella creatura | Capperipud ben dir d baver ventura
nell' Offeffa da vero havea ragione, Quelloia cui tocca cos) buon boccone;
Perch' ella ¢ bella fuor d'ognimifura Ma's ellas' ha da vincer con la lancia,
Per me non faprei darle eccerione. Hoggie quadoci arrischio ach'o la picia
Floriano giunto in Jenne veduta Doralice così bella se ne inuaghilce, ¢ risol-
ue pero di tentare la fortuna, ¢ cimentare la sua persona per avventurare i) con-
seguirla per moglie.
LL Popol vis ammazza, V'é tanto popolo per veder quella gioftra, che s' ani-
mazzano l'un J'altro per la strettezza. Hiperbole usatitlim' in queilo proposito
per esprimere la gran calca, 0 quantita di popolo. i
F.ANNO la moftra,, Quando i Cavalieri, 0 soldati, o altre genti 5 che devono
fare qualche operazione guerricra (ancor che finta ) avantidi.cominciare a ops-
rare compari(cono in ordinanza questo si dice far /a.mofrra,
LA Ragazza. Intende Doralice figliuola del Re.
2A. SVO mo. Secondo il suo gusto. Quel me vuol dir modo, usandofi da noi,
come da ii Latini, ¢ da iGreci la figura Apocope, che leva l'nltime sillabe alle»
role, ¢ da noi alle seguentiparticolarmente; Afodo., meglio fede y vagliv 5 vedi,
Saendaate » piede,ec. Che diciamo + mo, me, se, vo'. ve,fra, fan, più. Howo-
Juto-notar queste, perché speflo nel nostro parlare ci vagliamo di quelta figura,'¢
fitrovera ancora speflo usata nella presente Opera, come habbiamo.accennato
ancora sopra C. 1. stan. 10,
TIRA frecciate.come la rovella, Tira dardi,e frecce in quantita. Di questo
termine come la rovella, come la rabbia,.come il canchero, ci serviamo per.c(prime-
re quantita grande 5.0 vero operazione wiolenta infuperlativo grado; come per
efempio Mtale.corre fortissimo, il tale perquote gagliar ate diremmo.// tale.corre
la rovella, rabbia 0 canchero, 0 perquore come., ec, E si.deduce la. comparazio-
Rian violenza., con la:quale opera il male della rabbia, o del.canchero.. Las
evoce fovela:, O.rovello s \credoiinuentata dalle donnicciuole per.non profferire la
arola 'rabbia,.come'fi dicecappira in vece.dicanchero, EB se bene hanno del fur-
ten » fon tuttavia, molto:wlate:, ¢.]' usd il Malateftiin,alcune. sue ottave,
: Da poi ch'.io,a servito per rimbelloy
i) \Befonovandato.trenta mefi aioni
| Gridando per larabbia., ¢ pel rovello x 4
8 Come fail Gatto quand' hai pediononi ec, 'eis
Ed habbiamo il verbo. Mare, e}'.addietti Jato. 'In formma'in.que~
» flo luogo dicendo Tira frecciate come, /a,rovedaiintende,.che |Doralice.con le sues
 gcan bellezze faceva ianamorare ognuno, che la vedeva,
LE Grazie, | Poeti-fingono, che le grazie-fieno tre figlie di Giove nominates
Ve Aglaia, Eufrofine,, ¢ Thalia. Ag/aos\in Greco val per splendido, Eufrofine, ila-

93

ne

   

 

 

rita, allegrezza, ¢ Thalia, verdeggiante. Si che dicendo 7 feorge in quel.volto les wn
 grave vien'.a dire: Si sce:in lei splendidezza., alleg: » ¢ fre(chezza, cioè
gioventi fana. <

RACGOLTO in uno. Vnito in un folo'luogo, Termine latino, usato.alle.vol-
ste anche da noi in questo proposito, 3

    
 

94 MALMANTILE

LE trombe. Nella pitt stimata carta de' Ganellini, o Minchiate è effigiatala3
Fama con due trombe alla bocca, e da questa tal carta si chiama le Trombe; &
per efler questa la superiore a tutte l'altre carte quando si dice:\ La tal cofart les
rrombe s' intende, che questa tal cosa sia la meglio, che si trovi nel suo genere
Ed è detto afiai usato per esprimere l'eccellenza d' una cosa, ed ha la forza del
superlativo. ?

NON plus ultra, E, noto il motto delle colonne d' Hercole, che vuol dire;
Won si vadia più avant:; E noi ce ne serviamo nelle congiuncure simili alla pre-
fente, che s' intende; non si pud andar pil la, cio¢ non si pud avanzare,o fupe-
rare tal bellezza, o vero non si pud far più bella. Esprime anche questo termine
un superlativo,

PVO! fare, E' termine d' ammirazione,o flupore quafi diciamo: Pud mai fare
il Cielo, o la natura una cosa tanto bella, ¢ perfetta come quetta ? L:

CAPPER/? Ancor guefto ¢ termine d' ammirazione; ¢ si dice ancora cappita,
canchita, canchigna for(e per.non dir canchero: Voci inventate dalle donne;come
habbiamo accennato poco sopra alla voce reve/la, Confuona col latino Pape, che
noi diciamo 4 ! ¢ col latino babe, che noi diciamo, 0 habbo | E la parola capperi,
che tanto in Greco, che in Latino vuol dire il capper frutto noto, serviva anche
a' medesimi per termine d' ammirazione, o giuratorio, come si vede in Laerzio
nella vita di Zenone. Sed, @ per capparim iurabat, ficut Socrates per canem,ec. LO
stesso riferisce Alex, ab Alex. dier. gen. lib, 5. cap. 10. I Lalli nella faa En, wrau.
C, 1, stan. 85. 2

Capper disse Enea, come si tofto
Fatt' ha si gran Città questa Signora \:
ef CHI toca così buon boccone. Chi haura così buona forte. Chi haura per mo-
glie così bella, e ricca Giovane.
Cl arrischio anch' io la pancia, Ci avventuro anch' io la vita.
TANZA XXXUX

O per tute' hoggi beccomi sue moglie Cid detto falta incampo,e un' aftatoglie,
Nobile, ricca,¢ bella; 0 veramente Intruppandofi ld dov' ei gid fente,
Vi lafeto Loffa; s* ella cogkie, coglie C' appunto il ReYolleciea, ¢ commette,
Se x0 a patires O Cefare, 0 niente. Che pe' è prims si tirin le bruschette.

 Risoluto Fioriano di provarsi in questa gioftra si fa innanzi', ¢ piglia una lan.
cia. Qui bifogna fupporre, che Fioriano, ¢ gli altri Cavalieri futicro armati di
doffo, come € necetiario, che fieno i Cavalieri, che gioftrano a corpo a corpo.

BECCOMI fu moglic, Questo-verbo beccare ha signiticato di rubare, guada-
“— © acquiftare, Gio, della Casa nel Capitolo in lode del martello d' amore
ice::: dase,

So che fapete del ladro fottile, 4 oa
C' 4 Giove se la harba gid ai stoppay ie
: Quando glibecco fut esca, eilfucile. ae Si

E però afato per Jo pi scherzando in occasione di maritaggi, come appunto

nel presente luogo, EB si dice M tale piclio moglie, ¢ becca fu una buona dore., Elo

aan nace dal verbo beccare', che € novo quel che signitichi trattandoli d' am-
mogliati. a: t
+ SE

 

SMES

 
 

 

SECONDO CANTARE. 95

©) S* ELLA coglie, coglie.S' io m' appongo,fara bene. S' io vincerd l'haurd indo.
winata, ¢ sard felice, Se no 4 patire, Se non m' ayponee » fara disgrazia, haurd
pazienza. In somma con que i due detti yuo! moftrare, che Floriano ha I'animo
accomodato a tutto quel che sia per fuccedere, o male, o bene che sia.

O Cefare,oniente, Aut Cafar, aut Nihil, O morire y 0 esser qualcofa di gar-
ho. Quetta fentenza latina si profferi(ce da noi corrottamente, O Ceferi,o Nic-
colo, ed esprime Aut Rex, ant afinus de i Greci, cioé uno de due eftremi.

STL tirin le buscherte. Si tirino le forti. Credo che si chiamino bruscherte,¢ non
buschette, 0 forse in ambedue i modi; che ¢ un giuoco da Fanciulli, e si fa con,
pigliare tante fila di paglia, o altra materia simile, quanti sono coloto, che han-
no. a concorrere al premio proposto, ¢ quel filo, che tira il premio, si fa o pilt
lungo, o più corto de gli altri; detti fili s' accomodano fra due afi, 0 in mano
in modo, che non Gi veda se non una delle due teftate di essi, per le quali teflace

ciascuno de' Ragazzi cava fuori il suo, ¢ quello che tira il più lungo, o il pil
corto, sc do che è defti » consegui(ce il premio proposto; Questo giuoco
serve ancora ai Ragazzi per fare le divifioni ne i loro giuochi Panciulle(chi,come
ofarebbe ne i Birri,e Ladri detto sopra in questo C. stan. 32. aila voce Bomba, che
allora pigliano tant fili, quanti sono i Ragazzi, la meta Junghi, ¢ la meta cor-
ti, e:cavandoli da loro a uno per volta detti fili; quelli, che hanno i lunghi,van-
no da una banda, ¢ quelli de' corti dall' altra; ¢ così serve a loro, come serve nel
presente lnogo, per un modo di tirar le forti. E da questi bruscoli, o fili di pa-
ia mi do a credere, che si dica brascherte; e che bu/cherte sia quel giuoco, che si
con certi pezzetti di mazza rifefla, ¢ che fitirano, come 1 dadi, con altro
nome dette /e buffe. Vedi forto C, 11. stan, 42.
STANZA XXXXI, STANZA XXXKIL
Come volontarofo Floriano, Piglian del campo,e al cenno del trombetta
Senza cbieder licenza, 0 cosa alcuna, Sivannoincontro con la lancia in refta; £
Si fece innanzi, ¢ postavi la mano Ii Marchefe a Florian t' havea diretta;
j Di trarne (a pin langa hebbe fortuna, Per chiapparto nel merxo della testa;
Poo dopo il Adarchefe. di Soffiano Ma quei,ch't furbo, aun teposacivetta,
a... Simile a quella anch' egli ne traffe una E aggiufta lui,disendo: Afjaggiaquefia,
Ond' essi, come priaifn destinato,

 

male abbattuto. AZarche/e di Soffano, E nome a cafo, ¢ fa Marchefato una con-,:
trada,o villa vicina a Firenze detta Soffiano. 'f,

CHIAPP ARE, Val per colpire.

FVRBO, Se ben ja voce furbo deriva dal latino Fur, che vuol dir Ladro, tut-
tavia ce ne serviamo per esprimere un' huomo scellerato, ¢ che habbia ogni for-
 tadivizio, come s' è.detto sopra in-questo C. stan. 2. Ed ancora per denotare un'

huomo aftuto, ¢ che sappia il conto suo, come segue nel presente luogo »
ee FACIVETT A. Abbaiia la tella.. Viene dal giuoco di eivetta, che da i gid-
vanotti si fa in questa maniera. S' accordano tre; ¢d uno di loro, al quale @ toc-
cato in forte, si pone in mezzo a gli altri duc, i quali s' ingegnauo di a i}
' erEsh

Perché gli diede si spierata borta '
. Furono i primi a correr lo freccato, Ch' egli ando exit come una pera cotta,
2 ¢ questi due furono i primi a correre la lancia, n¢! qual' inconcro il Marchefe ri-
ida
york

Floriano pre(e una di dette Bruschette,ed una ne prefe il Marchele di Soffiano;
j
}

by.

 

  
 

 

96 MALMANTILE

berrettino di testa con le percoffe della mano; € —— egli tocca terra con le
mani, aon puo esser percoflo; e però hora alzandofi, hora abbaflandofi; tiras
guando all' uno, e quand' all' altro di gran moftaccioni; dura il giuoco:
che da uno delli due gli sia fatta cascare con un colpo Ia berretta dalla testa, che
allora perde il premio proposto, ¢ lo vince colui, che gliel' ha fatta cafeare, il
wale ( seguitandofi il giuoco ) va nel mezzo in luogo del primo. Tal-giuoco si

rs a tempo di fuono, ¢ piglia il nome dalla Civetta yccello, che per bulcare if
vitto scherza con gli uccelletti alzando, ed abbatiando la testa, come appunto fa
colui, che fla nel mezzo.” E da questo poi far civerra s'intende Abbaflareil capo.
Da Scops, che € un'uccello notturno de! genere delle Civette. Era appreflo i Greci
una forta di giuoco, 0 paflatempo detto Scopias, nel quale veniva contratfatto a
tempo di balio il muoversi in giro, ¢ l'alzare, e¢ l'abbaflare della testa di quell'
uccello; onde ne fu formato il verbo Scoprein irridere, che appreflo i Greci vale',
quel che appreflo noi Toscani, Vecellare. V. Giulio Polluce |. 4. cap. 14.

AGGWST A iwi. Aggiultar uno, s' intende Bargli il fao dovere, ¢ trattare uno
come eglimerita, Lat.coxcinare, Vuol dire ancora conciar male uno,come s'intende
nel presente luogo, ¢ sotto C. 11.stan. 50. E per altro vuol dire Saldareso pagare
un debito. Lat. pariare.

BOTT A. Colpo, o percofla. E questa voce bortayper altro vuol dire una spe-
cie di Rospo. Lat. rabera.

ANDO gis com' una pera corta, Calcd gid facilmente,ed a piombo,come fanno
Ie pere cotte dal Sole, che cascano facilmente dall' albero; o forse come le
cotte.al fuoco, che fon facilissime a andar gilt in corpo quando si mangiano
Plauto disse: T'am crebri ad terram decidunt ut. pyra'; da che si deduce che s'intea-
da delle pere, le quali cascano dail' albero,

STANZA XXXKXIL J

tn quanto a Sposa, homai questoe ascoito; Che mette lui per morto, anzi fepolto,
S?ei rocco terra, ancor la voglia spati: Ma il giovane, che da di quei faluti
Così Florian dicea; ne Sette molto Gli moftra in avviarlo per le

» CH il fecondo ne viene alpren battnti, LL error di chi fai conti fenzal Ofte
Comparue il fecondo Cavaliere il quale si dava a credere d' haver già morto

Floriano; ma questo col burtarlo.a terra, gli fece conoscere quanto s' era ingan-
nato.
£ ASCOLTO. EB licenziato. 1 ragazi, che vanno alle fquole, quando sono
stati sentiti leggere dal Maeltro si dicono a/eolrs, ¢ s' intendono licenziati: ¢ così
iefto: Cavaliere efiendo paflato per le mani del Maeftro, che ¢ Floriano, si pud
ire a/colto,¢ licenziato dalla Sposa.. e
TOCCAR terra, ¢ [putar la veglia, Dicono le donne, che quando fon pregne,

» venendo loro voglia di qualche cosa, se in quello ante G toccano con le proprie

mani in alcuna parte di » quivi nasca alla creatura un fegno simile a quel-
la tal cosa desiderata; ¢ 1 fegni poi chiamano vogiic; ¢ che per sfuggire che
la creatura non nasca con tali fegni 5 0 voglies il rimedio sia, che la Donna pre-—
goa, quando le viene tal desiderio, tocchi subito terra con la mano, e4puti di-
cendo:. 4 terra vadia. B però il Poeta, seguitando questa opinione, dice, che se>
il Marchele ha toccato terra per liberarsi dalla vogiia della Dama, ¢ neceflario
anco-

  

 
 

SECONDO CIANTARE. '97
ancora 'chelegli sputi, a voler che il rimedio sia fatto' compitamente, |'Tal detto
| sputar la voglia, & afiai vulgato per intender' unosche habbia gran desiderio d'una

-bilcofa; jche-sia-a Iii imposiibile.a-conseguire » Vedi Plin, lib. 28.c. a

\) (A SPROW baceati,, Acvatta carriera; Velocemente! Fran. Sacc, Novella 'mibi
3%. E così falito a cavatlen', ando'd sprom battnti al Palace de! Signori.

) LO. meite per marto, anzi fepolto. Intende; che questd secondo \Cavaliero non
folo credeva di havere auccidere Floriano; ma git pareva'gia d' haverlo uccifo..
Esprime la gran prefunzione, che havea di s¢ stesso muri Cavaliero., ¢ la poca
stima, che faceva di Fioriano.

D1 quei faluti, Invende di quelle percoffe,

PAR il conto senza Ofte. Stabilire per fattayna cosa, alla quale deve inter-
uenire, ¢ concorrere anche la volontà d' un' altro. Doveé T intereffe del com-
paguo, si pud metter ig ficura la propria yolonta, ma non: quella del compagno.

STAN ZA XXXKUL

—

Comparfo il terzo, in te/a della lixza All' altto manta il fettimo indirizza;
S? affronta feco, ¢ paffalo fuor fupria Liortavose il none appre/so inveffe,e fora;

. Soggiunge ilguarteed coli tel infirza. E cotna tutti con suo vant, e fama
Shudella il quinto, efreddailfeftoacora Cave di teftail ruzxo-della Dama.

In questa otta va It Autorenactala vittoria', che hebbe Floriano di fette'Ca-
valient ye descrive la lor, perdita'in fette modi di diré diversi; il primo'lo pafsa fuor
fuora,il fecondo, <a(si dourebbe dire infilzazma non folo perché gli f rmel-
fa oot tea per. ¢aulaidella.rima,quanto anche petché per i più si dice infizza,

. enon #85¢\facto lecito dirlo anch' egli) ib terz0 do pafsa fuor fuori, i) quar-
toua fredda's il quinto Jindirizzaall' altro mondo; il seto Pinnefte, edvil. settimo
dofora, EB 2 quell fette  i dire havendo quafi eutti lo -stetio significato d' am-

TT

 

 

' dana l'artifizio de) Poeta-in moftrate la fecon-
. ascenien. Jingua jenn
4. Che si dice anche Nizza. Vuol dir linea; ma da noi s'intende quel

  
   
  
  
   
  

tayolato,"o/muro, rafente al quale corrono i Cavalieri le lance al Siracino.
CAVO! ditelea ilrnx20 della Dama, Fete ulcir'di testa il desiderio della dama.
Eaormenes » che dal.verbo razcare vuol dir Baie, wfata in questi. termini si-
Prucio, — sidefiderio, ec, si che dicendofi. Wi fale ha queffa ruzzo in

esmeld ale ha ono +e » gucfto liumore 5c, il Laica nov, mihi
8. dices... cosh fates gafhigatura, ee rhe re fener £08:
M07 ey 8 wren ta

wn T: A NZA XXXXIV.
AM Re firal 3 con Plaine Ond' ogni-altro ne fu mandato fano;
ny, Seefe di iconsaPiglnola | o>) (Bde nelle ze infino a gola

 

  

a se fae ieinteaiaagee 4 moi | ho. > Bem pastinc, servite, ¢ ringrasiato
. Gome nel Bando havea date parola;. ~ Rimsfe quivi a goder il Papaty.
he Re fesse da Foca iano alla ee as er mo-
2 nr -Bioriano rimaie quivi a godere que
ta fisop 'eh Siitins 10

TOCC AR la mano. Elo steffo in were 'calo che che diciamo impalmare,
AL depeisoee tat toccamento, chefi iad Sins nama gry
N

3;

 

rc

 
ee
98 MALMANTILE?,

si;che il primo atto che si faccia per lo stabilimento del contratto del matrimo-
nio, Vedi forto C, 12. stan. 50. '

CH ANDAT O fana, Cioé licenziato, ed esclufo. Il verbo: valeo, (che signifi-
ca Star fano, ¢ usato da i latini anche per licenziarsi: parentibus vale\dixit, ed il
simile facciamo noi, come si vede nel presente luogo., che diciamo Mandar fani
in vece di licenziargli. Anzi 11 medesimo verbo vaieo è tal volta usato danoi per
intendere Addio, cioé licenziarsi. 11 Vai in una faa fiortola ( se ben points)
lo moftra dicendo.

Hore liete,

Tam vatlete. t
dam valete amati ferculi; y q
Etnnale, gyoesse,
O fedale, 7
Che maneggi i miei liberculi; i

U1 nostro Poeta sotto\C. 6, stan. 18.
Refti la donna, ed er le disse vale

 

oar:

WELLE dolcexze infinoa gola, Immerlo nei piaceri,e ne igufti, foto C. 4.

fan. 42. dice esser ne guai agola.

GODERE il Papato. Goder le felicita concedutegli dal Cielo e

STANZA XXKKXV.
Tre di fuonaro a felta le campane',
Ed altrertanti si band? il lavoro,
Eil Suocero., che meglioera del pane,
Viu' huom discreto ed ua coppa d'oro,
Faceva con cli Sposi a fealdamane,
Talhord'a Mona luna,e Guancial d'oro,
E fece a' Paggi recitare a mente
Rofana,e la Regina d' Oriente,
STANZA XXXXVI,

LD andar il giorno in piazza ai Buratriniy
Ed agli Zanni furon'le tor gite;\
Ogni fera facevanfi fepini
Di oe » ¢ di bakar veg lic bandite;
E chi non eraingambe, nein quattrini
Da trinciarle,e da fare ite, ¢ venite,
Dicea novele, o stavale a ascoltare,
Faceva al Maxxolino, 0 alle Comare,

In queste quattro ottave il Poeta narra le fele, ed allegric, che si fecero ing
Campi per lo sposalizio di Doralice con Floriano; le quali fefte:fa'che non tra~
scendano eo nds pucrile per continovare a scrivere una novella per i Fanciulli.

meglio che il pane. Era un' huomo buonitfiaio, un' huomo che si

accordava a ogni cola, appunto come è il pane, che s'accorda, ed unilce con,
tutte le vivande, almeno appreffo a i Fiorentini. In questo proposito i Greci

. dissero, Columba mitior. f

VAN-A coppa a ora, Vn0 »l quale non sia da apporre alcunrdiferto, omni exce-

ERA un

 

STANZA XXXXVIL
Altri pik la vedevanfi confondere
A quel ginoco chiamato gli Spropositi,
Che quei ch' esce di tema nel rispondere
Connien ch' il Subio depositi 5
Aa altri piace pik Capanniscondere,
Hani? altri vary) humor yvarij propositi y
Perch? ognuno aun mo none composto
Pero chi la vuol leffa, ¢ chi arrofto,
STANZA XXXXVIIL.°
Chi fale Merenducce in ful bavaglo;
Chi con amico fa a Stacciabburatta
Chival! Altalena, ¢ chi a Beccalaglio;
Va quello a Predellucce,un s' acculatta;
Per tutti in somma fempre vi fu taglia
Di fhar lieto cos} in barba di gatta,
LE tra Floriano, il Re, ela Figlinola
Mai fu che dir n' unt anno una parola,

 

pe

 
 

 

 

 $ adunano pil Fanciulli, ed uno,

SECONDO CANTARE: 99

eptiont maior, Credorche si-dica coppa d' are, periintendere oro coppellato; 0 di

ella, cioé raffinato, che Coppelia si dice quello strumento, col quale si ri-
duce l'oro alla sua yera purita, ¢ perfezione;¢ Coppa vuol dir bicchiere, o altro
valo simile, donde poi Sortocoppa quella tazza, sopr' alla quale si portano i bic-
chieri, dando da bere ye Coppiere quel che porta da bere al Signore.

SC ALDAMANE. Quattro, 0 pili s! accordano, ¢ sates Sac ordinata-
mente le. mani sopra del.com »€\poi vanno cavando per ordine quel~
la mano, che sehen ¢ meets sopra all' altre sane © con quelio
modo; €.confricazionepretendono scaldarfele; ¢ però tale operazione ¢ detta,
Scaldamane; ed ¢.giuoco/Fanciulle(co, che ha la sua pena' per chierra cavando la
mano, quando non tocca a lui. at

MONA luna... S' accordano molti Fanciulli,e tirano le forti.a chi di loro hab-
bia a domandar consiglio a Mona luna, ¢.quello.a cui tocca vien fegregato dalla
conversazione, ¢ ferrato in una flanza,.accioO.che non possa intendere chi sia,
quello di Joro, che, refti elerco in Mona luna, deila qual Mona Juna si fa l'ele~
zione fra gli altri, che reftano dopo che coluié ferrato. Bletta:che ¢ Mona lu-
na; si mettono, tutti.a federe.in fila, ¢ chiamano colui, che è ferraco, \acciò. che
venga a domandar il.consiglio a.Mona,luna.,, Questo tale se ne viene,\¢.doman-
da il consiglio a uno di quet ragazzi, quale egli crede, che sia slato eletto in Mo-
na luna, ¢ se s' abbatte a trovarlo,ha vinto;se nd; quel tale, a cui ha domanda-
to il consiglio gli risponde; lo non fon Mona luna, ma fla più gil, o pili fu, se-
condo che.veramente.¢. polto quel tale, che @ Mona luna; ed il domandante per-
de il premio proposto, ed ¢ di nuovo riferrato nella stanza per tanto, che dai
Fancuulli sia creata un' altra Mona luna, alla, quale egli torna a domandar confi.
glio, ¢ così seguita fin a che una volta s'apponga, ed allora vince; e quello.che
= Mona luna perde i) premio,, ¢ vien riferrato nella stanza,diventando colui, che
deve domandare, ¢ quello che s' appose,s' intruppa fra gli altri ragazai.. 1 do-
mandante richiede fino,a quattro volte il consiglio, ¢ pud perder quattro prevj,
¢ poi fimescola fra gli altri ragazzi, efente però da dover pil efler domandan-
te, se non nel cafo,, che fatto Mona luna, egli perdeffe, ¢ sempre fitorna as
ercare nuova Mona luna, ¢ si deputa nuovo domandante, quando il primo s'ap-
ponga » 0 habbia domandato, quattro volte il consiglio, 1a qual fuazione, come

detto » non pud esser forzato a fare, se non quattro volte: edi premj si adu-
nano, ¢ feditribuiscone poi fra di Joro riparticamente, ¢ dal rendergli poi a di
chi sono, cavano un' altro paflatempo, come diremo, Da questo ginoco viene
il proverbio Più fu fa Afona luna, che significa Nella tal cosa è mifterio pil im-
portante di quel che altri si pensa. 7

Nota che. taato questo giuoco, quanto ogni.altro, che troveremo nella pre-

fente Operas' altera,.7 ¢ diversifica secondo li gusti, ¢ conucazioni pue-
— nili;¢.noa mi Tipe ne haveffi nella tua puerizia Eatti » 0 veduii fare

alcuat, o tutti diversamente da quello, che io gli descrivo *
GV ANCLAL @ oro, Quetto eure € giuoco Panciullesco, quale € fatto così:
mette.a federe sopra.a una seggiola, ed un'
altro se li pone inginocchioni avanti, ¢ pola il suo capo in grembo a quel che>
ficde, il quale gli chinde gli occhi con le mel » acciò che non possa vedere chi
. 2. ha

Oe ees 6
be

 

 

 

100 MALMANTILE 52 if

sia colui, che lo percoffe in unaimano »che egti si tiene dietro \sopr' alle ren; doz
vendolo egli indovinare;.¢'calaiche gli fertagli o¢chi:, dopo \che questotale ¢
flato percosso glidice 2 Chi sha percofa? edegli risponde: Ficefeccho y ef altro
replica: Aderamelo qua per un' orecchio, Ed allora quello fi'tizza',¢ va @ pigtiar
colui, che egiterede it perdutiore y¢ (e $” appone, ha vinto y ¢ ponetil percutfo-
re in lnogo suo;¢ li fa.dare il premio in mano a quello chic.fiede's @ se hom s' ap-
pone perde ih premio.; quale. coniegna, ai derto fedente, ¢ ritértia al udgo di 'pri-
ma per continuare; fin tanto.che s*appone,ed alla quarta vol 'si fa huova'clez-
zione,come sopra a>Monayiona; 'Guetto mi par di potcrcredere j\che sia quel
gioco; che i Greci.chigmavano Coldubi/mo siferivo dal Baleng. de lad.veteap.37:
gual giuoco da quel Propheriza: quis te percufit 7 detto per disprezzo da Giada a
Giesti Crifto sig.\ noft'o, si pudvarguinentare, che fafle anco appreflo a-i Latini.
ROSANA,¢ la Regina @Oriente. Sono duc Leggende,o Rapprefentaziont n0-
Usfime, per esser cantate giornalmente da ogni donniccinola ngeBh ONO?
BVRATT IN1, litehde quei Figurini di: legho, che for fatti: muover da une, che
a cal effetto s' ascondé in un castelletto di legna coperto di' pannd; € gli fa operas
re.mettendofegli sopra alle punte delle dita,e cd un certo uo fifehio git fa parlare,
ZANNI, Per Zanni, che's'intehde servo feioceo Lombardo, qui intendé ogni
forta di Bagateellieri, che fanno ilibuffone per le piazze ys) 8.
FEST INI di ginocojec, Quando's' adanano'in wna casa ae Dame', e Cavalieri
per giuocare insieme, 0 per ballare nella 'prima parte della norte, dice fare un
Feftine, 0 Veglia. E te bene veglia strettamente prefa, pare che significhi più trat-
teaimento di-ballo, ch¢ di giuoco, tuttavia la pigliamo per intendere ogni forta
di teattenimento, 0 di Giuaco',' 0! di Ballo yo di qualfivoglia altra cosa', nellas
wale si spendano Ie prime hore della notte, dicendofi: Aoi facemmo la deglia &
dudiare » 4 ballare, a cantare, ec, Ma voleado pigliare queste due voci nel suo pro-
prio significato; Feitino, S' intende adunanza di persone nobili, sia per ballare,
© per giuocare in quelle hore deila notee; ¢ Feglia ¥ intende d” ogni forta di per-
fone ordinarie; E si come s' avvilirebbe ve: fo fui alla veglia nel Palazzo
del Principe così pare, che si burlerebbe dic*ndO: Fué al se/Pino im cafd'uh Battilano,
Quando si dice Feffino'pubblico, 0 Vee liw bardica s intende Feffino, 0 Peglia & por-
ta aperta,'dove pud' andare ognuno ¢ Vedi sotto: G@ 9 stan. 51. ¢ Cant. ro,

stan, 28. ' i &
NON era in gémbe; ne it quaterin' | Now si sentiva'gagliardo da ballare,.¢ non
flani23:

haveva monete da poter giuocare. © '

DA trinciarle, Intende da far capriole, cide fattare'. Vedi forto C, 7, flan.

DA fare ite, ¢ venite, Cio givocare. Quando fi'giudea, € rdendo si paga
la posta volta per rota 50 rlguot quia la vine dca oO fare ire',
nite 5 ¢s' intende pagare il subito pérdata la posta'; ¢ riceverio'nello'
niodo vincendo; ed ¢ il contrario del detto Parenti ae-gli bai; che figaifica
care in fu la fede, 0 a-credenza, RF 2 OF Die gTo AND Vib AE er

| MAZZOLINO'. Ancor 'queito'® trateehitienco da Panchull', ¢ fifa in tal gui-

fa. Pia i adwnano inlieme;e si piglino'i nome & tn fiore per ciascuno,
¢ di questi fori un di loro, che @ il' Giatdiniere compone un mazzo', € poi dice:
Questo mazzo non fa bene per causa délla Viola; € colui, che ha'prefo a

 

 

 

    

 

  

ice

ies

  

 

 
 

 
  
   
 
 
  
  
 
 
  

\

SECONDO CANTARE: ror

delta: Viola deve risponder fabito: Dalla Viola non viene,ma si ben dal Giglio, o
altro fiore, che' a' lui verra nella mente'; ¢ se'non risponde subito', o vero se no-
mina un'fiore') che non sia in quel mazzo, perde un premio, i) quale si da at
Giardifiiere; 'E così vannio seguitando fino a che il Giardinere habbia in mano
tanti. premj da potere alla fine del giuoco distribuirne almeno uno per ciascuno di
quei ragazzi, che sono nel giuoco; ed il Giardiaiere ¢ fottopotto anch' egii alla
perdita del premio', perché f€ ua fiore dara !a colpa a Ini, ¢ che egli non rispon-
da'fubito, ¢ nomini un Fiore, 'che non sia nel mazzo'; perde come gli altri, ¢ it
fio premio va dato in mano a colui, che I" ha fatto errare; ma core in deposi-
to', perché 'alla' fine' del Giuoco va poi con gii altri distribuito 'dal Giardiniero
il quale'non Jo pud però dare a se medesimo; E questi premj ff domandano pegvi,
edi questi intende il Poeta dove dice: Convien ch' if pegno fubiro deposit!.

Finito il Giuoco i} Giardinicre distribuisce ripartitamente ¢ pegui pigliandone
anebra per se.Tali pegni poi sono da coloro, che gli hanno dal Giardiniere havuti,
reftituiti a i proprj padroni, i quali, se li rivogliono', devon fare una cosa fecon-
do il gusto'di colui), al quale ¢ toccato in forte il detto pegno; E questo dicono
far ta penitenea, Ya quale se egli non fa, il pegno refla in mano a colui', al quale
€ roceato'; ¢ però questi pegni devono esser di qualche valore, accié che i padro-
ni habbian caro di riavergii. Alle volte fanno questo giuoco iGiovanetti di mag-
giore eta ) € riducorio questi pegni a moneta, quale depositano ogni volta, che

'in'mano a un'déposirario, ¢ se ne fernono per far merende, ec, fal giuo-
eo peo diffimile a quello, che facevano i Greci dctto Bafilinda riferito da Giu-
lio 'Polltice tab:'9. '¢. 7..¢ dove noi dicianio Giardini¢re essi dicevano Re, comes
facevano anche i Latini, ¢ cid si deduce da Hor. Ep. pr. ub, pr.

3 ——— At pweri ludentes, Rex evis', ainnt,
Si rette facies', bic murns abenens essa, ce.
'Rofeia, die fodes 5 melior lex, an puerorum

: WNama ? que Regnum rette facientibus ofert.

Se bene potrebbe dirfi, che Orazio tion imtenida di questo giuoco particolarmen-
te,perché in tutti li givochi Fanciulleschi tanto i Greci, che i Latini chiamavano
Re colui', che vinceva, ed afino quello che perdeva; ma perché nel giuoco pre-
fente cra farto Giardiniere (0 diciamiolo Re Wghetto » che in altri giuochi era ri-
mafto superiore a tutci, però sion m' ajlontano da interpretare Orazio, ed appli-
'care 3 suo 1ndgo al presente proposito, nel quale, se it Re errava diventava
YP afino, ¢ Re si faceva colui, che havea fatto errare, 0 tenendofi il conto di
'chi di loro haveva meno errato, quello alla fine era il Re,¢ quello che più volte
haveva errato era l'Afino', 0 Re Mida. Vedi il Meurfio de Ludis vererum, Gli
Spartani similmente per Legge di Licutgo, econo che riferifte Plutarco'nellas
vita del medesimo', ai Ragazzi di pil di fer' anni, ei come Principe
il più favio tra loro, che sopranrendetie a' loro giuochi, ¢ Fanciulleschi efercizzj.
~ “ALLE comare. 6 ginoco è trattenimenco di Fanciullette,e 10 fanno così:

Mettono una di loro in un letto con un barrboccio fatto di celici, ¢ flngendo;che

“guefta habbia parrorito, le fanno ricever le vifite da-altré Fanciullette con far

quelle cirimonie, ed accompagnature, che f coflumano in occasione di vere par-

tucienti. Nk,: 3
t i tal

 
102 - “MALMANTILE

Tal giuoco era usatoancora dalle Fanciullette Greche fecondo Giulio Pol.lib.9,¢i7;
ma in-vece d' una Parturiente fingevano una Sposa; ¢ lo dicevano Phitramelia.
Qual giuoco fanno pure ancora le nofire Fanciulline, ¢ lo chiamano far' alle Zie
Non ha questo giuoco delle Comare, 0 Zie altro fine, che di paflare il giorno in
quelle loro tirimonie, ¢ ricevimenti, ne i quali alle volte si confuma quello, che
Ie Fanciullette hanno havuto per merendare.

GLI spropositi, E Jo stesso in fultanza, che quello del mazzolino, se non che
dove in quello si finge un Giardiniere; in questo 1 Ragazzi s'adattano a qualfivo-
glia altra cosa, con pigliarsi quei nomi, che attengono a quella tal cola;
¢lempio: Faranno il giuoco sopra il pane;il Macfiro fara il Fornaio, e gquefto fara
quello che nel Mazzolino fa il Giardiniere; uno fara la farina, uno l'acqua, uno
il forno, ed altre cose attenenti alla conftructura, ¢ perfezione del pane; IL
Fornaio dira: Questo pane non è buono per caula della Farina; quello che has
il nome della Farina, deve risponder subito: Dalla farina non viene, ma dail'
acqua, o da altra cosa che gli venga in mente, atteneote al pane, ¢ che sia frais
loro Ragazzi; ¢ se non risponde prefto,, o non da la colpa a qualche cosa, il no=
me della quale non sia in quella adunanza,o non sia attenente al pane, perde 5 ¢
deposica il pegno; ¢ fifa nel refto per appunto come nel giuoco del Mazzolino:
E questo giuoco universale è forse quello, che habbiamo detto sopra, che face-
vano i Greci detto Bafiinda, E da noi si chiama il ginoco de gi S; » perché
dovendo quei Ragazzi risponder prefto, attribuiscono al pane cose (propositanf-
fime,¢ che non hanno che far punto col pane, 0 sua bontà, oltre a non esser
il nome di quella tal cosa in veruno di quei Ragazzi. £ quello vuol dire VJew di
tema,

Habbiamo un' altro modo di far questo giuoco, ed è così: Mettonfi più per-
fone a federe in giro, e ciascuno dice al compagno in uno orecchio una parola,o
due al pil, ¢ finito il giro, ciascuno ordinatamente dice force quella parola, che
gli ¢ fata detta dal vicino, ¢ volendone comporre il periodo si sentono gli Spro-
politi, che rifultano-da quelle parole; ¢ fida la pena a colui, che ne @ stata las
cagione.

re 4 niscondere. Vino si mette col capo in grembo a un' altro, che gli tura
gli occhi, ed un' altro, o più si nascondono, € nascofti danno cenno, € colui sche
haveva gli occhi ferrati si rizza, e va cercando di coloro, che sono nalcofti, e»
trovandone uno basta per liberarsi da tornare in grembo a colui, dove mettes
quello, che ha trovato, ¢ questo perde il premio propotto, ¢ il trovatore va as
nasconderfi; ma se non trova il nascofto in tante gite, o in tanto tempo, quan-
to sono convenuti,perde il premio, ¢ ritorna a flar con gli occhi chiuli come pri-
ma; ¢ seguita così fino a quattro volte, perdendo quattro premj, come sé detto
sopra a Aduna luna, ed i premj poi Gi distribuiscono come si fa al giuoco del A¢az~
zalina, E quello far con gli occhi ferrati si dice far forto, che i Greci in un Gmil
ginoco dicevano catamyerm, Lat. connivere. E coiui che è stato fowto quattro vol-
te,¢ non ha mai trovato il nalcofto, ¢ per consegucnza perduti i quattro premj,
occupa il Iuogo di colui, che teneva fowo, € quelto s-iniruppa con gii altri Ra-
gazzi,fra i quali si tira la forte a chi dee Mar foro, 0 nasconderfi; E così segui-

 

tano canto y che si riducano tutti liberi; perché quello che ha pagati li quatcro~

prem)

 

3
5
&

 

 

 
 

 

SECONDO CANTARE. 103

© premj nel modo fuddetto, ed ha occupato il luogo di tenere gli altri sotto, come
ne vien cavato nella maniera accennata, refta fuori del giuoco, del quale folo
attende la fine per conseguire anch' egli la sua parte de 1 premj da distribuirfi, Era
“ancor questo giuoco appreflo a 1 Greci, ¢lo chiamavano -Apodidra/cinda secon-
do Giulio Polluce lib. 9. c. 7. 5 ma diversificava alquanto; Ed in questo giuoco
pure il vincente era detto il Re, ed il maggior perdente ? Afino. Vedi il Buleng.
de lud. Grace, cap. 22. éd il Meurfo in verbo e4podidrascinda. Simile a quetio
era ancora il giuoco detto da' Greci Myinda,

OGNVNO 4 un mo non è composto. In questo proverbio fentenziofo habbiamo
f ancor noi come i Latini pi modi di dire, come: Le nature fon diverse. Tanti
)— huomini tante berrette, 0 tanti cervelli, Tutte non possono esser aun modo, Chi la
te 4 leffe ye chia rofto,¢ molti altri; ene i Latini si trova. Quot homines tor fen.
tentie,, Suns cuique mos, Trabit sua quemque ia « Won omnes cadem mirantar,

amantque, ed altri infiniti, ¢ tutti con Jo steffo significato.
PAR le merenducce. \ nostri Stovigliai in alcune Fiere, che si fanno in Firenze
il giorno della festivita di San Simone, ed in quello di $, Martino conducono

  

 

; gran quantita di stoviglic piccolitfime,come piatti, tegami, pentole, ed ogni al-,

tra specie di arnefi,¢ vafellami da cucina, che da essi si fabbricano di terra. Di
; este si provveggono li nostri Fanciulli per quanto vien loro permefio dalla loro
~ borfa, ¢ da queste vien poi loro l'occasione di far le AZerenducce, percht haven.
-sdo altre'mafierizie adeguate, come tavole, sgabelli', bicchieri, faluiette,¢ fimi-
'li); imbandiscono una menfa, accordandofi più Fanciulletti, ¢ Fanciuiline a por-
tare quello', che ¢ dato loro per merenda, ed accomodando tutto in piccole par-
ticelle, le distribuiscono in quei piattellini,figarando di fare un Banchetto, e met-
*tono a federe a quella tavolina li loro Bambocci; E queste fonda loro chiamate
© Merenanece,delie quali parla ij Poeta, ¢ le quali erano usate ancora dalle Panciul-
line antiche in occasione del fuddetto appellato Phitramelie » come si ca~
va dal Meurfio, dal Soutero, ¢ dal Buleagero.
BAV-AGLIO, Saiuietta » 0 Tovagliolino da Bambini, che si lega al collo con
due cordelline, o nastri, detto così dalla bava, che sopra vi casca dalla bocca de
~bambini; i Latini pure fecondo 1l'Onomastico lo dicono pe'torale falivarium,e con
A mee Bavazlé come lor proprj arnefi apparecchiano Ie loro piccole tavole quan-
~ do fanno le A4erenducce, ¢ Gi mangiano quelle particelle distribuite in quei piattel -
“Aini3-come s*è detto sopra. EB di queste Aderenducee parla il Poeta.
ST ACCIABBVRATT A, Due seggono incontro |' uno all' altro, ¢ si pigliano
per le mani, ¢ tirandofi innanzi, ¢ indietro; come si fa dello staccio abburattan-
do la farina, vanno cantando una lor frottola, che dice.
} Staccia abburacta
» Martin della gatta:
La gatta ando pel vino, ec,
E questo è trastullo usato dalle Balie per acquietare i Bambini di quella eta, che
 appena si reggono in piedi 8 |
\\ ALT ALENA, Paflacempo da Fanciulli; Legano due funi al palco,o vero a
* due al beri, ¢ le fanno calare a doppio fino pretio a terraun braccio, ¢ sopra di
» eff funi accomodano un'afle, sopr' alia quale si pone uno, 0 pits a federe, ve
; are

 
   

 

ea | ee
 

104 MALMANTILE

dare il moto a detta affe vanno cantando alcune canzoni con, tin! aria aggiustata
al tempo dell' ondeggiamento di quell' affe, ¢ questa ¢l' £ora de', Greciy,dai La-
tini detta Oscillatio, ed alore, yolte. Petaurnm penjile, ¢noila diciamo Alealena dal
Latino Todewen, che vuol dir quella Macchina di legno.,,con,la quale si cava.Jtac-
qua de i pozzi ( come si vede in Plin, lib, 19, c, 4. Vel Tollenonum haufkn rigandos)
da noi detta Mazacavallo. Vedi forto C. 6. staan. 86.,E questo perché faceyano
l'Altalena, come la fanno talvolta anche Ji nostri, Fanciuili con iacrocicchiare
una trave sopr' all' altra, ¢ ponendofi ugo o pil ragazzi per teftata della trave,
che ¢ di sopra,la fanno alzare, ¢ abbailare a foggia di Atayzacavallo. Diguestas
parla il Bulenger, de Ind. vet, c, 11. Questa ditaiera, in aicuni luoghi di, Tofeana
€ detta bictancole, 5,
BECCALAGLIO. E' un givoco simile alla mosca cieca detta sopra. Cox, flan,
40. ne vi è altra differenza, che dove in quello si da,con yn panno.avwolto, 9 altra
cosa simile,in gquelto si da con la mano piacevolmeate una fola, volta da.colui, che
bendé gli occhi a qucl, che sta forto, ed il bendato in wece di dare, 5 affanna di
pigliare un di coloro, che in quella flanza sono del giuoco, ¢ colui che reftera
prefo,deve bendarsi in Iuogo del bendato, ¢ perde,il pegno,.0 premio.,ed.il pri-
mo bendato refta libero, ¢ s' intruppa fra quelli, che hanno.a eden prefi sie fifa
come sopra nel giuoco di Guancia] d'oro, Sidice Becealaglia,. questo tale
bendato vien condotro in mezzo della stanza, 0, piazza, doye s! hada fare il giuo-
co; ¢ colui che lo bendo, ¢ che quivi l'ha condotto gli dice; Che fei tu-venuto\a.n

   

 

fare in piagza? Ed egli risponde; 4 beccar /'aelio, E quello,dandogli leggiermen-
'te See fur' una spalla foggingne:, O beccati codefto, Dopo. la qual fungio-
ne il bendato s' affatica,di pigliar uno per metterlo in suo,luogo. 1 Greci appel-

lavano questo pernen Claeiede da pencola che in Greco, si dice Chysray edo fage-
vano nella stessa manicra; ma in vece di, bendare gli,occhi,mettevano.a colui, 0
fingevaG, ch' egli tenefle colla finiftra una pentola in capo, ¢ girandogli intorno
Jo [olleticavano, o percotevano; onde, se egli rivoltandofi, prendeva chi gli
tirava; il prefo rimaneva in cambio suo a.eflere quel della pentola, 4 Latini lo
dicevond tidus ollarins s 4 &, aillapre>- Su
Simile.a questo era un'altro giuoco usato dalle Ragazze Greche, detto, Cheliche-
Jona, vel quale, messa,a sedere quella, a cui dayano nome di Chelona x chevuol
dire Teftuggine; Ie dicevano: Chelichelona quid facis ix medio ?. quella risponde-
va: Lanam sexo,@ filum miltfium con quel che segue riferito dal-Buleng. de>
Jud. vet. cap. 41. F ae Tere
Nel giuoco poi della Chyrrinda, ovvero, ludus ollarins dicevano: Quis ollam ?
¢ chi teneva la pentola rispondeva: Ego Adidas, cs! affannava non di pigliare un
di coloro, ma di toccarlo co i piedi, ¢ quel tale casi, tocco perdeya, ¢ si metteva
la pentola in capo; E perché ( comes' ¢ detto sopra ) i Greci havevano per co-
flume di chiamare Re il vincitore, ¢d afino il perditore., pecd quelto tale, che

havea la pentola in capo si. a Adida, clot Re efino, Vedi Giulio Polluy —

ce lib. 9. c. 7. ed il Buleng. de Lud. Vet, c, 17.

ANDAR a predellucce, Duc si pigliaag peri pol a' ambedue le: st tno!

con l'altro in croce, ¢ formano come una seggiola, ¢ un' altro vi fiede: 3¢

quelto si dice andar' 4 predellucce. Das feis=is-ninne un guinea ae aekans*>,
ae ae

 

  
  

+

&

4
 

 

SECONDO CANTARE: 105

ed era il portare uno in fu le spalle, ¢ reggerlo, tenendo Ie di lui ginocchia nelle
paime delle mani voltate dietro alla persona, ¢ detto Zz Coryla, cic nella,
siotola, © cavo della mano. Ma questo credo che sia un' altro giuoco, che noi
diciamo 4 cavalluccio, che vedremo sotto C. 3. stan. 30. tanto più che i Greci se-
condo Io stesso Polluce chiamano questo giuoco detto 4x Cory/a, per altro nomes
Hippada dal verbo Hippazin, cavalcare. E questo se bene € giuoco, tuttayia &
specie di pena per quei, che portano per haver perduto ad altri de' fuddetti giuochi.

ACCVLATT ARE, BE' pafiaiempo da Ragazzi, ma è specic di pena, ¢ di tor-
mento dovuto a colui che € acculattato. Quattro ragazzi pigliano uno per les
braccia, ¢ per i piedi, ¢ formandone un quadrato, Jo follevano, ¢ gli fanno bat-
tere 11 culo in terra tante volte, quanto merita il suo delitto, o perdita, che ha.
fatto in altri giuochi, come sopra. E questo si dice acculattare, che in altro si-
guificato vedemmo sopra C. 1. stan. 7. Gli Spagnuoli chiamano l'Acculattares

“mantear,perché mettono colui che si ha da acculateare in una coperta,o mantello,¢
tenendola da quattro capi, lo sbaizano in alto, € lo fanno ricadere in essa, ¢ noi
lo diciamo dar la coperta,

V1 fu caglio per tutti, Vi fa da dar soddisfazione a tutti. Ognuno hebbe in che
impiegarsi. Traslato da' Sarti, che dicono:1n ete roba ci è raglio per un' Abi-
to,0 per due,ec, per intendere,ci ¢ tanta rcba, che si pud fare un' Abito, o due, ec.

ST AR in barbadi Gatta, 0 di Micio, come si disse sopra in questo C. stan. 28.
annotazione alla voce sbigortito, Pare che oo detto possa venire dall'anticas
superitizione degli Egizz}, i quali credendofi, che il Gatto fufle consegrato alla

fide:, che era ia loro Deita maggiore, non folo nutrivano con granditiima
cura, ¢ splendidezza questo animale, ma fecondo Pierio Valeriano reputavano
degno di morte colui, che ne ammazzafic, o faceffe loro oltraggio. E riferi(ce
Alcx,ab Alex, dier, Gen; lib. 3, cap.7. ¢ lib. 6. ¢. 14. che quando moriva un Gat.
to,i medesimi Egiz2j per contraficgno di dolore firadevanole ciglia,e poi metren-
do addosso al morto gatto fale, ed aromati, e coprendolo con un panno bianco lo
feppellivano, facendoli taluolta fepolcri notabil;tanta era la flima che ne facevano.

XXXXIX. STANZA L.

Mai fu tra lor fin qui nulta di guafto, Bench" if Suocero altura, ¢ la Conforte
Se non che Florian volto ale cacce, Maledicefrer questo [uo motivo

. -Hatvendane più volte tocco un taffo, Dicendogli che la fuor delle porte

 E fentendofi dar fempre cartacce, Va' Orco v' è st perfido, ¢ cattive y

Dispose al fin di nan'veler pi pasto, Che perfegutta ? huomo infino a marte,
We curando lor preghi, ne minacce E che lt ingoierebbe vivo vivo;
Fece innitar da i faliti Bidelti Con gentt, ed armi usct ful axrora
Per Paltra di i Piacevoli,¢ i Piattelli, Gridado: Andiane,adiane,eccolafuora.

Non hebbero ( come s' ¢ detto ) questi Sposi mai occasione d” addirarsi, se non
che Floriano inclinato alla caccia si risoluette andarvi a dispetto della Moglie,

¢del Suocero:.

NON fu nulla di guafto « Non furono tra loro mai rotture; cioènon s' adira-
Tono mai; ¢, come si dice 5 non s' ingroflarono i fangui.
 HAVENLONE toveato un vasta, Havendo di cid domandato alla sfuggita, 0

 discorfone con brevita. Tratto da i tafti del Cimbalo, o'vero Organo strumenti

ae

musicali. oO DAR

*

3%,
106 'MALMANTILE

DeAR cartacce, Non rispondere fecondo il gusto di chi richiede; Traslato dal
giuoco di minchiate, nel quale si dicono cartacce quelle che non contano.s:¢ fo-
no di niun valore. Vedi forto C, 8, stan. 61.

DAR pasto, Trattenere uno con scufe, o chiacchicre. E il latino verba dare; /pe
laitare. E si dice così,perché il polmone degli animali(che da noi fidice pafo)stracca
colui, che lo mangia, ma non Jo fazia. Si dice anche dar paffo, quando uno, che
fa giuocar bene a un tal giuoco,finge di faper poco, ¢ si lascia vincer da princi-
pio, a fine d' indurre il femplice a far grofle potte per vincergli aflai.

SIDELLO, Donzello, o Servitore d' Vniversita, 0 d' Accademia, come fa-
rebbe quel Donzello, che serve allo Studio di Pifa, o ad altri simili. E questo
nome di Bidello fecondo }' Autore deile Notizie Ecciefiaftiche ¢ corrotto da Pe-
dullus, perché questo Viiziale, ( dice egli ) che nell' Accademie, ¢ negli Studj

pubbiici haveva cura d' efeguire le commiffioni appartenenti allo studio, folevas
portare in mano un baftone chiamato Pedo; Quantungue aleri ( foggiunge il me-
see) tirino la sua etimologia dalla parola Saffonica Bydell, che vuol dire il
anditore,

Ma io.credo che il nome Bidello sia tolto da Berul/a, che & quell' albero, del
quale si facevano le verghe per i fasci, che anticamente portavano 4 -Littori
d' avaati.a i Magifirati del popolo Romano, ¢ che da questo portare i fasci di
verghe di Betulla, sia poi venuto il nome di Bidello a tali serventi di Vaiversita,
i quali faono figura di Littori, € nello studio di Pifa portano ancora una grofia
mazza d' argento ( significante gli antichi fasci ) quando vanno in funzioni pub-
bliche avanti al Collegio de i Dottori.. Alex, ab Alcx, dier. Gen, lib. 12.17. in
fine, dice così.

Quodque fascibus, quos praferebant Lictores, betullas virgas maximt commodas dis-
sere, itague ex illorum virgis tum proper candorem tum propter tennitarem publices
Lasces, qui magifiratibus prairent, efecere. E Plinio lib.6. c. 18. Gander frigidis for-
bus,& magis Berulla; Gallica bac arbor, mirabilis candore arque tenuitace, terribilis
Maziftratuum virgis. Lo steffo attefta Polid. Verg. lib. 4. c. 3. 1

oa » ¢ Piattelli, Sono in Firenze due conaerfazioni di cacciatori ~

wali andando alle cacce gareggiano fra loro a chi faccia. maggior predayequeila,
ane rimane superiore, nanos fuole entrare nella Città teionfante cebaoth 5
carri, ed altro; ¢ l'una si dice la Compagnia de' Piacevolé,¢ Valtra de' Piatrelli;
¢ ciascuna ha la sua fanza catro alla quale s' adunano. gli Vfiziali, ¢ Servcati,
¢ Altci; ¢ questi fon quelli de' quali dice il Poeta, ¢ chiama i Joro serventi
idelli.

VN' Orco. Questa ¢ una beltia immaginaria inventata dalle Balie per far paura
ai bambini, figuraadola uno animale specie di-Pata, nimico dei bambini catti-
vi, ¢d il Poeta, che noms” allontana mai dal genio'pucrile, moftra che ibfuoce-
ro Stordilano voleva indurre nel. Floriana ihtimore per farlo aftenere dais
andare a caccia, con dirgli che fuori della porta v' era l''Orco, ers,
huomini: Questo nome però viene dall' antica faperftizione de i yi quali
chiamavano Orce l'Inferno Virg. a. lib. 6, Priemifque in faucibus orci, Bd inten-
devano per Orco anche Plutone, quali wrgus, five Kragus ab urgende egli
sforza, ¢ Spinge rutti alla morte; ¢ percid dalle madri, ¢ nutrici per' Peas
ial

> Skee? pp seeth seh ce etre,.. x

 

 
SECONDO CANTARE: 107

alli lor bambiai si dice che ' Orco porta via: il che pure vien da i Gentili, che
igliando Orco per la morte, lo chiamavano Ineforabile,-¢ rapace. Orazio
Bae 18. lib. 2, Nulla certior tamen
Rapacis Orci fine destinata,
~ GRID ANDO andianne andianne., ec, Così vanno gridando i cacciatori faddetti
la mattina avanti giorno per fuegliare i compagni. Lo stesso, che Alo Alo; ovs
vero lon dal Pranaele eAilons.
STANZA LL

 

 

Senza veder ne anche.un' animale
Frugo, bufso y gird pi di tre miglia;
Pur vedde un tratto correr un Cignale
» Ferace, grande, e groffo a meraviglia,
STANZA LIL
Che a posta prefabavea quella fembianza,
E glk pafso fuggendo atlor a! avanti
Per traviarlo folo con speranza
D: haver a far di lus piit boccon fanti;
Cosh guidollo fino alla [ua spanra
» Dov'ei pense di porgli addossoi guanti;
~ Poi nan gli parue tempo, perché i cani
» blauersan piit tafto lui mandate abrani,
STANZA if,
Pero.volends andave in ful ficura
Won a perdira pitt che manifefia,
Perché a reder roglieva un' offo duro
~ M€entre non to chiappaffe testa testa;
Glisparid' vcchio,e fece un tempo [euro
Per incanto levar, vento, ¢ tempefta,
£. uolash gros comparire,
Cc Dearehte ph ans may che mi dire,

usioriang scorfe moita campagna, ¢ ce

Ond'ei, cht il di dovea capitar male
Si moffe a seguitarloa tutta briglia,
Won essendo infor mato ch'in quel Porce
Si trasformava quel ghiotton dell Orco,
STANZA LIV.

A cacciator, che quivi erain farfetto,
E dal (udore omai tutto una broda,
Havendo un veftituccio di dobretto,
Ed nn cappel di brucioli alla moda,
“Per non pigliare al ventoun mal di petto,
O altro', perché il Prete non ne goda,
Won trovado attra cafain quelfainatico,
Che quellagrotta, infaccavi da pratico.

STANZA. LV.

Atal gragnizola, a venti così fieri
C* ogni cosa mandavano in rovina,
Tal freddo fu che tutti quei quartieri
Sen! andananoin diaccio,¢ in gelatina,
Ed ci ch' era veftito di lercieri,
E mai meglio facea la furfantina,
Won più cercava capriole, 0 damma,
414 dafar,s' ei poreva,nn po di famma,

  

rcd buon pezzo, ¢ non trove mai nulla,

se non che pur vedde un grotio Cignale, a] quale si mefle dietro co i suoi cani,
* non sapendo, che ¢ra l'Orco trasformatofi in quel cignale per pigliar Fio.

riano dalla vitta lc spari, ¢

pioggia, ctempefta, 1a-quale obbligo Si

t via de' uoi incanti fece venire una gran
loriano a ricovrarsi in una grotta, che cra

vi fra quelle macchie, nella quale entrato, si meile a cercare se trovava modo
deme un po tiesto 4
£.

AVGO.. Cioè cerco:minutamente
do con le pertiche per tutto... ~

'frugando per le fiepi con i cani, ¢ buffan-

DOVEA capitar mate. Doveva haver disgeazic. Doveva rovinare, E il Lat.

-— Perdsyperire,
. cd TVTT Abrigiia, A quto corso

.. GHIOTTONE. Epiteto solito da

«

 

f  senza punto fermarsi, come fa il cavallo
quando se gii lascia ee « Laxatis babenis
( ) 'aun' huomo maligno, e di genio cattivo,
€luona quafi lo itetio, che Briccone, furbo, viziolo, scellerato.
+ O2

py

 
108 MALMANTILE

'PIV boccon fanti., Più buon bocconi. La voce fanti in cafi simili significa per-
fezione in generale. Vedi forto C, 3, stan. 8.

PORRE iguanti a deffo. Piglia guanti per mani, ¢ vuol dire Pigliarlo.; Hab-
biamo il verbo agguantare, cioè pigliare », Guanto dal Germ, Hend:, mano.

ANDARE in ful ficuro, Andar enza paura. Metterfi a fare un negozio cons
ficurezza di non efler' impedito, e che riesca fecondo l'intento.

TORRE a rodere ut? offo duro, Pigliare a fare una cosa difficile.

CHIAbP ARE, Qui val per ritrovare,e sopra in questo C. stan. 41.per perquo-
tere; ed il suo propeie significato ¢ Pigliare; dal Lat, capere.

TEST A testa, Cioè a folo a folo. Remoris arbitris, Diciamo anche a quat-
tr' occhi.:

GRAGNVOLA, Grandine, che è gocciola d' acqua congelata nell' aria per
forza di freddo, ¢ di vento, e si fa di vapore freddo, ¢ umido stropicciato nelle
parti interiori.del nugolo, La pioggia nalce da vapori freddi, ¢ umidi adunati
ne i nugoli, La xeve ¢ impreffione generata di freddo, ¢ d' umido; ¢ questo fred-
do @ minore di qucllo,col quale dalla pioggia vien generata la grandine, ed ha in
se qualche parte di caldo. La rugiada ¢ gencrata di freddo, ¢ di umido non rap-
prefo, e questa congelandofi nell' aria diveata la brinata « Ho voiuto,benché fuor
di proposito, notare l'origine de i sopraddetti accidenti dell' aria, perché da.
questa s'intendano i loro nomi; in qualche parte d'Italia per ayvencura differenti.

HAVREBBE infranta nun fo che mi dire, Haurebbe schiacciata, o diciamo an-
che ammaccata qualfivoglia cosa per dura che fufle; Non fo immaginarmi, ne
dire cosa tanto dura, che ella non l'havefie infranta. Questo termine 2on fo che
mi dire usato nella forma, che si vede nel cafo presente, significa quel che s' ¢ det-
to; ma per altro.' usiamo anche per denotare di non havere, o faper trovar
modo di rimediare a qualche accidente » per clempio: Lo non fa che mi dire, se it
tale vuol far male i fatti suoi, ©

IN farfetto, Veltito leggiermente. Farfetto hoggi intendiamo ogni forta d'a-
bito leggieri, ¢ difinuolto, che sopr' alla camicia si porta sotto gli altri abiti, co-
me farebbe camiciuola, 0 giubbone, ec. “

TVTTO una broda di fudore. Tutto molie dal fudore; Sudatissimo per la fati-
'ca del viaggio violento. i

DOBRETT O.Intendiamo una specie di tela di Francia fatta dilino,e bambagia
(che è il cotone filato ), Sidice anche Dob/erro da duplex,perché nel tefferio,e fatto
di doppia orditura, ¢ riempitura. Così. dobb/a © dobbra dissero gli antichi.

BRYCIOLI. Quelle fortili strisce, che il Legnaiolo cava da qualfivoglia legno
lavorandolo con Ia pialla, si dicono bracioli, forse dalla similitudine de' brucioli y
bachi;e da questi si diconocappelié di brucioli quelli, che son composti;ed intesiuti di
stcisce d' un' erba particolare, nello stcflo modo, che si fa con la paglia y alla
similicudine, ¢ larghessa della quale sono ridotte le dette strisce.

e4LLA moda. Cioè alla foggia che ula; la quale cra nel tempo, che l'Autore
compote 1a presente Opera 5 che i cappelli havevano piccola falda. Si che non
tanto per efier di brucioli., quanto per esser piccolo, cra poco atro a difendere»
dai acqua.. Si dice alla mods quali all' usanza, che ¢ modo,cioè adeflo,Pr, alla moda,

ei AL di petto, Così.chiamiamo volgarmente quell' infermita., che 1 Medici
Ficena Pletritide. PER.

 

 
 

 

LR Sree Te ga ere ne MRS OA MMeare | ice nared

SECONDO CANTARE: 10g

» PERCHÉ il Prete non ne goda, Cioè per non'morire, € così far che il Pretes
non goda il guadagno della cera del funerale.

QVEI quartieri. lotendi per quelle campagne, per quei contorni. Che per al-
tro noi Fiorentini per guartsere intendiamo una delle quattro'parti, nelle quali ¢
divisa la nostra Città. E guartiere in lingua militare significa Habitazione ¢ dar
quartiere al nimico significa faluargli la vita, ¢ farlo prigione.

1NSACCAVI da pratico. V' entra dentro come se egli,per eflerui entrato altre
volte, fapefle la strada, e vi fufle pratico.. Se bene huomo pratico usato nella ma-
ty che ¢ qui, vuol dire huomo favio, ¢ da faper pigliar compenfo in ogni oc-
ea

GELATINA, Vivanda nota fatta per lo pili col brodo di carne di porco cot-
ta in aceto', © poi congelato; Ma qui per Ge/atina intende che l'acqua s' andava
congelando sopra il terreno, ¢ fa Gelarina finonimo di Diaccio,come fa D, inf. 32.

PAR la Furfantina, Si ova una specie di Bianti, i quali per muover le per-
fone pie'a far loro elemofina, dopo haver bevuca buona quantita di gencrofo vi-
no,ne i tempi più freddi si diflendono mezzi ignudi nelle strade-più frequentate, €
tremando fingono di morirfi dal fieddo, e questo lor tremare si dice far /a Pur-
fant ina, cio' fare it giuoco-che fanno questi furfanti,ch' ¢ poi paflato in dettato,
che significa,, ¢ comunemente s' intende Tremare.
~ MA meglio, Benitimo., Già mai si trove chi facefle meglio. Quel ma vuol
dir mai; la figura apocope.

'DAA A1 A.E' \o stesso, che Daino specie di capron faluatico.Lat, dama D. Inf. 4.

Sh si farebbe un'cane infra due dame, ec.
STANZA LVI.

Trove fucile,ed esca,¢ legni var}, Così con tutti commodi ae... pari,
Ondun buon fuoco in uncantone accefe, Dopo una lieta, ilcrogiolo si prefe 5:
E in fu due faffi postt per alari, Essendofi a far quivi Siu >:
Sopr'un' altro fedendo i più difhefe. Mentre pioveva, come quei da Prato i

Bloriano: havendo trovato' ia 'quella grotta comodita d' accendere il Fuoco,
P-accefe:, ¢ vis*accomodd a scaldarsi, alpectando che intanto ceflafie la pioggia.
FVCILE. Intendiamo quello strumento d' acciaio, del quale ci serviamo per
battere nella pietra focaia ad esserto di cavarae il fuoco; detto Fucileda fuoco,
quafi fecaio, 0 facile. Che per difiefi anche Focile.
£SC.A, Quel fango, 0 sia cuoio corto conciato'col falnitro, che facilmente»

 

 

iglia fuoco., ¢ serve per tener sopra alla pictra quando in essa si batte per trarne
i oa 3 dai Latini detta fomes. La qual rete,fot ben per translato fighifica inci-
tamento., © flimolo, che noi pure diciamo fomite, nondimeno era intesa per
ogni cosa facile a pigliare quel fuoco, che Vergilio-appella seers eth-
Sirufa in venis filicis < Si come noi, ancora diciamo:E/caogni forte di cibo d' ani-
mali, pure dab latino £/ca.,. che vuol dir'cibo,'ed incendiamo ancora questa ma-
teria, che ¢ atta a pigliare subito il fuoco, quafi sia il cibo del fuoco; anai a que-

sia non diamo altro nome, che a' ¢/ca, e dicendofi £/ea aflolutamente, ¢ (enzas 3

Aggiunta, s' intende folamente-questo cuoio cotto, © fungo conciati con falnitro. = ih

“ALAR!, Sono due Ferri 5 o Safi, che si tengono nei focolare', perché man-

i 'tengano folpele le legne, acid che pil facilmente ardano. £' voce ease i
ue

 
“110  *MALMANTILE

Latino /ares, la qual voce spefle volte era prefa per fuoce ».come si pud dedutre
da Ovid. 1. faft. 18. 'i % ve

Omnis haber gemings hincsarque bine ianua frontes

E quibus hec Populum /pettar 5 @ ila Larem.

Eda Colum. lib, 11, cap, 1..de Villico.,.Con/uescat rufticus. circa larem Damini,
focumque familiarem femper epularé. 1| Sipontina dice così:) Lares Di erant apud
Gentiles, & colebantur domi, focu/que illis [acer erat, unde vulgus focum focolare ap-
pellat quafi laris focum. Molti in vece di dire 4lare dicon arali, o sia corrotta~
mente, 0 pure, perché gli piglino da era, intendendo strumenti da mettere in
fu l'altare per foftenere le Jegne per il fuoco de i facrifizzj, ¢ così fanno che fias
ben detto tanto arali, che alari.; 'oe )

AC, pari. Agiatamente si dice anche 4 pie pari.. Vedi(opra;Cant. pr, flan.
82. Lasca Novella 4. lib. 2, Servsti delle buone vivande:y che voi fapere bene acconces
e fragionatg se ne frettero a pic pari, Si dice anche agambe larghe. Vedi.forto.C. 9.
stan. 32. Ed in miolti altri movi, che tucti maftrano la spenfierata agiateaza duno,

DOP' una liera. Dopo una famma.  Diciamo Aer una fiamma chiara, senza
fumo, ¢ che prefto paflia detta diera da (etitsa, come anche baldoria,da baldore(cio®
baldanza ) voce antica.. Gli Spagnuoli similmente dicono alegro, un fugco dal
legria. Vedi sopra C. 1, stan. 4.0 fore si dice seta selngeae Gieramente, che ap>
preGo ai nostri Contadini vuol dire prefamense, cioè cosa, che pafla preftamente.

PIGLIARE il Cregioio. Stagionarsi, Quando fon format i bi chieri, ed altri
vali di vetro, gli mettono così caldi in un fornelletto, che a tal fine ¢ sopr' alla
Fornace, dai ord chiamato Camera, dove ¢ un.caldo. moderato, ¢ quivi gli
lasciano flagionare, ¢ freddare a poco.a poce, conducendoli con un ferro alla
bocca del detto Fornello per da batio,dove non si fente più.caldo, il che da edi G
dice dar asempra, temperare 5 0 dar il Crogiolo., 0 Cragiolare. E. di qui parlando
dell' hyomo intendiamo piglare i/ Cregiolo, quando dopo una fiamma egli conti-
nova a fare attorno al fuoco, fino che sia tutto incenerito. E da quelto verbo
Crogiolare piglia, 0 ha l'origine, il Gregivele sche ¢ quel valetto. di terra cot-
ta, il aale serve per mextervi de ' @ liguefare, 0.fondere i metaili nella Fors
nacc,detto corrottamente Corergimalo, is gabe BAN

FAR come queida Pax Proyerbio vulgatissimo, che. significa La(ciar piovere;
1 Fopoli della Città di Prato., che ¢duddita, ¢ vicina a. dicei miglia a Fircnzes '
nel- tempo,\che i Fiorentini fisreggevano.a Repubblica, domandarono Jicenza di
poter fare una Fiera jl di 8, di Settembre, ( ja qual Fiera Gi continova fino al pre-
sente in detto giorno ) ¢ per tal' efietto. mandaropo Ambalciadori alli SS. Priort
di libeeta, da 1 quali fa Joro.conceduca la domandata Jicenza ».con Neincliag:
pagaticro una certa s di denaro. Accordato ib aegozio gli Amb; a
partirono; Ma ¢flendo nell ulcir del Palazzo, fowvenoe loro, che sein talgior-
no fufie piovuco, non haurebbono potuto far la Fiera, ¢ nondimeno farebbe loro
conyenuco pagare il danaro accordato; onde per aflicurar quello punto tornaros
no indictro, cd entrati di nuoyoida i SS. Priori, uno di ch ambafeiadon feng
altre parole disse: Signori, sse ¢'pioveile? Alicheuno.de'Signori iybita eispole. =
Lasciate piovere, E di qui nacque quelto proyerbio Far come quel da.Prato', che
signitica Lalciar pioveres 1 ' ie 6 Asal Eas

3

 

 
Le a et eT et ee ee

 

=

SECOND'IO CANTARE. =

STANZA LVIL

LZ' Orco fratantocon mike atri,e feorci,

etffacciatofi all nuscio, ch' era aperto,
Prego Florian con quelgrugninda Porci
Tutro quanto di fango ricoperto,

Che ( perch'ella veniva gin con gli orci)
Ricever o voleffe un po ul coperto,
'Ritrovande/i fuora fealzo, ¢ ignudo
A sigran pioggia,e a tempo tosh trudo

TIL
STANZA LVII1,
Hebbel giovane allora un eran contento

Dhaver di nuovo quel beftion veduto,

E favendogli addossa affegnamento,
Luafi in wn pugno gid Phaveffe hanuto,
Rispose: Volentieri; entrate drenjo,
Venite, che voi frare il ben vennto,
Che dopo ilfugeir voi Uamite, eit Zielo
Fate a me; ch' ero fol, fernizioaCiels.

 

Mentre Fioriano flava a fealdarsi; 1' Orco s' affatcio alla bocca della grotta
fenz' haver mutata la figura'di Cignale, ¢ pregd Florian, che !o ldsciatle entra-
re; Eiglirisponde, che entriallegramente, ¢ che ne riceve servizio, perché
essendo folo,ha cara un poca di Compagnia.

. Non si maravigli il lettore, che un Cinaate parli; ¢ si ricordi, che ¢ una No-
vella per i Fanciullini, e che queste cose seguivano.
i Al tempo, che volavano + pennati,
: Tutie'le cose fapevan parlare;
Secondo, che dice quel che de(crive la guerra di Carnovale con Madonna Que-
fefima « Apill. As.) i2. Parietes locuturos,boues,o id genus pecora dittura prefagint.

GRVGNO. S' intende ia faccia del Porco 5 da grannitus, che ¢ lo stridere del
Porco. Grugnino ¢ detto per vezzi, ma qui ¢ ironico, ¢ per derifione Guardate
bela faccettina, 0 bel grugnino', 0 bel. grugno, quando yogliamo jitendere una
brutta faccia 5 EB si dice baver i/.gruero,dell* huomo quando ¢ incollera, donde ix-
gragnare por entrar in coliera'. Vedi forto C. 8. stan. 61. ¢ /erugoni si dicono le
pugha dace net vil. '

ELLA vith gi ton gli orci. Cio' piove Ne di¢a: Ogai goccib-
lad di tanta acqua; quanta rie cade a dar la volta a un' Orcio, che ne sia piend.
Sidice anche Zila viene a bigonce, a carinelle, ec, tutte iperboli per denotare, che
piova gagliardamente. Vedi forto C. 10. stan, 20.

FALENDOGLI addosso usiegnamento, Difegnando quello, che yoleva far di
aa i# in fito potere Ȣ dominio, come esprime il Potta medesimo di-

: Quafiin ne gid l'hawe/sé haunto, ' vt
BAR i naan fun servizio, 6 favote accettissime, '6 gratidissimo,
STANZA LIX.) STA EX.;
Poi disse > Hor vin venité Alta ficnra..
Rispose ? Oreo: Jo non Verrd ne arco,

Credi tu pur ch? io sia così merlotto! Guarda la gamba | perch' ia he panra
Se non glicanfi ci verrd domani, | ~ Diquellalrifeiagh iwrlveggo ut fiico,
5° altro,dice il garzon,non ¢°è di rotto Allor Florian la cintura',
Die pieche te gli vo" legar lontani', Ed impiarth la [pada ott' un banco,
© Eprefo allora il [uo guinziiglio in mano' Diffe'l' Oreo: ( dedutald riporre ¥

© Lagi in un canto T ehero y¢Giordane, “Jo ti ringracierei; ma non accor.

STAN-
i ens

112
STANZA LXL
E lasciata la forma di quel verro,

Prefal' antica,e moftruofa facia,
Con due catene salto la di ferro,
E lo lego pel colle, e per le braccia,
Dicendo: C acciatar tu bai pres' erro,
Perché credendo di far predaincaccia,
All fin non bai fart' altroch unavescia,
Ment' il tutto ¢ seguito alla rovescia.

'MALMANTILE

 

STANZA LXIL

Rimafto ci fei tu, come tu vedi
Senza bifogno haver di teftimoni,
E perché con leurieri, ¢.cami se spiedi
Far me volevi in peri, ed in bocconi;
Coss perch' ella vadia pe' fusi piedé
Faraffi ate, ne leva piit ne pani,
Accio che, procurando IL altrui danno,
Ler te ritrovi il male, ed s1:malanno,

STANZA. LXIIL

Ed io c* hebbi mai fempre un tale scopo
D' accarezzar ognun, benché nimico,
Come la Gatta,quando ha prefo il topo,
Che, se ben' ¢ tra lor quell' odio antico,

Scherzandocon esso alquanto,e poco dopo
Te lo (granocchia come un beccafico y
Così perché piit a fila tu mi metta
Veglio far' io, ¢ poi darti la firetta.

L' Orco alla cortefe ofierta risponde, che ha paura de' cani, ¢ della spada; e»
Floriano lega quelli in un canto, € ripon questa fetto un banco; Allora l'Orco
si scuopre, ed entrato nella caverna prefe Floriano, ed incatenollo.

S/ch? E un termine, de} quale ci serviamo per dimoftrare che habbiamo,co-
nosciuto l'inganno, 0 cattivo trattamento., che alcuno ci habbia fatto, o hab-
bia in animo di farci, quafi dica: Cos} eb vorrefti.ch' iofaceffi? 0 vero Così mi
tratti eh ?

FATE motto, Proferito col primo,o, stretto.,. Vuol dire ascoltate, sentite..
Fate motto a me; ed usato nella forma che è nel presente luogo,ha forza d'escla-
mazione, e vale per un certo modo di domandar consiglio, quando ci detta una
cosa, che sia imposfibile a farsi, o a crederfi, quafi chiamiamo altra gente, che ci
consigli se questa tal cosa sia da farsi, o da.crederfi; ¢ che fenta lo sproposito che
cié stato deito. Dird per efempio; Cofui dice che ha trent' anni ye Sono pin di cin-
quanta ch' ¢i nacque; Fate motto! Cio' udite sproposito; O vero giudicate, se»
cid pud essere.: 3 è:

SLA così merlotto. Cink sia così femplice, così minchione., così privo di fenno.

Ci verré domani, Detto ironico, che significa Non ci verro mai. Questo De-
mani ¢ if Domani eterno di quell' Oite, che hayeya (eritto sopr' alla sua bottega
Doman si daa credenza,¢ heggi no, Ghe l'hoggi era fempre;,¢ il Domani havea
fempre a venire.Berni 4 rivederci alie Calenae Greche,prcio da Such. in Aug. c. 87.

DYE picche.. Detto indetecminaro, se ben pare determinato,, ¢ significa molto
lontani, ¢ non per appuaro la lunghezza di due picche ma forse aGiai pili, ¢ for-
se aflai meno. ) 4

GVINZ AGLIO, Si quella corda, o striscia di gor » con che si tengono. i le-
vrieri a lafla;e da molti è prefo per ogni force dj legame, derivandolo.dal verbo
latino wincio, come vincafire, ynciglia, ec. ma strettamente guinzagiio »\ 0. vinza-
gis 3 intende folo la corda, 0 quoio,col qe si tiene al Jevriero alla lalla; se»

ene da qualcuno ¢ inteso ancora per quel Jegame, col quale $\accoppiano in-
fieme i bracchi, o altri cani da caccia, Lat. copula. ':

GVARDA la gamba! 1 Cielo me ne liberi, Ll Cielo mi guardi, che io sia per,
far questo. In Firenze nella Corte della Mercanzia, che ¢il Lee dovefi

anno

¢:

 

 
 

pet
ti

iF

 

SECONDO CANTARE. 313

fanno ¥ esecuzioni Civili,sono alcuni Donzelli, i quali si chiamano Toccatori.
'Questi dopo che in una causa si fon fatti tuces gliatti, ¢ si vuol venire all' esecu-
zione personale, vanno ad avvisare il debitore, che se-egli non paghera in te:mi-
ne:di ventiquattro hore (ara condotto in carcere; ¢ fenzatale atto, che si dice»
Toccare,o fare il tocco, non si si pud con Cittadini Fiorentini:venire a detta cfe-
cuzione per(onale. Tali Togcatori anticamente pet esser conosciuti portavano
una calza d'un colore,ed nad' un' altro, onde nel paflare che facevano fra le
Borteghe,¢ peri i noghi pil gepeeass ixagazzi gridavano: Guarda la gamba;
affin che chi era in grado d'esser toccato si porefic fuggire, ¢ guardarsi, non po-
tendo i Toccatori far tale azione ne i luoghi iumuni; ¢ si dice Toccare perch
non serve, che coftoro avvisino con la voce il detto debitore, ma devono for-
malmente toccarlo con la mano, E da guefto è venuto il modo di dire.
Guarda la gamba; che significa mi guarderd, 0 fuggira di far tal cosa. [1 Lalli
neil' En. trav. lib, pr. stan. 67. si serve di questo detco ne] medesimo proposito.
Venere allor rispose; Honor Celefie
Guarda 1a garuba | usurpare io non vegtio,

IMPIATT ARE, Naicondere, ¢ si dice di materiali; € non pare che
fuonerebbe bene il dire Impiattare la verita, 4a virtù, ec. Vedi sopra C, 1. stan,
75 +41 Poeta fene servetotto C. 19. stan. 5. parlando dell'Aurora; ma la conside-
ra. come donna 5 ¢: corporea, come si considera il Sole, la Luna »tle Stelle,
delle-quali si dice Lmpiatcarfe 9 0 rimpiattarf: dictro a i nugoli, o dietro le monta-
ce ' f oo lei-non fering: che s appiatta,¢ fugge.

dir la Tayola, sopra alla quale si posano le vivande per man-

giare: ae bene:Banco ha: molti altri significati.

0. dO thringraxierei, vanon occorre; Cirimonia che si usa con.chi ci habbia fatto
sun favore a rovescin,.o vero ce l'habbia fatto quando noy occorreva,o quando
hawevamo'gia fattoda per'noi quel-che speravams da lui; 0 che difua cortefia ci

faccia un Tavares del quale non havevamo bifogno; ed ¢ lo fieflo che dire 4 "ho

smegli orecchi, ee ¢ fimnili 5

40F2 Porco maichio senza castrare. Dal Latino verres.

TV has Pree ero, elaine stete E — hogei poco usato fy che il

t orstm'c
BARE wpa vefeia “og conchiudere Non 'adempire il Yao intento', come.
aoe diquella i Sten fm r Sees mettono nella canna minor
gu richieda, ¢ fearicando poi non ono, ¢ fantio
uno feoppio. ychea pena 6 fente, ¢ tale sopeenen rd Si dite
ancora ve/ia una:specie di Sr B ve/cie dicono le donne un racconto de fatti
' ¢ velciai donna > che ridice tutto quello che fate

cor 2
“* 5:0 Devine pi ov gibi ion “dpgtungere;'¢ non levare. Cio' farai trattato
'cepa eae oe cae E
e 2iath ie Mian ia woo

Tinie, edi walense pais, e io ch' il male.
(os hggigamels diag cot, 'econ ogni cosa; ed i Posta mete

y

   

 
114 MALMANTILE;

mo lo dichiara,dicendo: come um beccafico, i quali uccelletti da i pil si mangiano
senza buttar via  ofla.. E /eranocchiare se ben s' ula alle volte ne i cafi come il
presente, non lo trovo usato se non per esprimere il romore:, che fa coi denti in
romper quell' offa colui che le mangia, il qual romore è simile a quello.che fa il
ranocchio quando canta.

HEBBI un certo feopo, Hebbi un certo fine, un certo genio, un certo riguar-
do» La voce scopo vien dal Greco scopos, che tanto appreflo a Greci quanto ai
Latini, ed apprefio a noi vuol dir Berzaglio, ¢ ggr metafora significa quel fine,
al quale tende, ed ¢ diretta la nostra mente nelle nostre operazioni, per lo pil
in bene; che non flimerei si potefle dire senza riprenfione. Scopo di rubare. Si
dice anche baver mira, il qual termine ¢ per avventura pil generico, dicendofi
haver mira di far bene,¢d baver mira di far male.

METTERE a filo, Bar venir gran voglia, Traslato dal coltello, ed altri ferri
tagiienti, i quali quando sono ben' arruotati ( che fidice meffi in filo, 0 affilati)
tagliano meglio.

DAR (a Sretta, Vuol dire opprimere uno. Ma qui aia nel suo vero si-
gnificato di stringere, ed intende stringere co i denti, ci i

mangiare.
STANZA LXIV.
Così spogkollo tutto ignudo nato, Lo racchiufe, ¢ lo tenne foggiornato 5
E veduto ch' egli era una fegrenna, Perch' ei facefe un po miglior corenna,
Adeit asciutto,¢ ben condixionato, Però che a guisa pos di mettiloro
Snello, lefto, ¢ leggier com' una penna, Voleva dar di Zanna al suo lavoro,

L' Orco spoglid Floriano per mangiarfelo, ¢ vedutolo così magro risolvé di
'non toctarlo, ma lasciarlo flare tanto che ingraflafle ye poi mangiarfelo. 4

JGNVDO rato, Cioè ignudo, come quando ei nacque. Diciamo così per in-
tender uno, che non habbia in doffo ne pure una minima parte di veftimento, ed
ha la fiefla forza che dire Zenudoignude,, che per la ragione della replica', vuol
dire Ignudidimo, '0 Affatto igaudo.

SEGRENNA. Quella voce, usata per lo pilt dalle donnicciuole, vale per

'esprimere una persona magra,sparuta,¢ di non buon colore, che i Latini, tol- 

to dai Greco, dicono Afonogrammus; ed il Poeta medesimo la dichiara dicendo;
Tdeft asciutto, che bxomo asciutto intendiamo huomo magro; ond' io mi credo che
JSegrenna veaga da fegaligno che vuol dire Animale magro edi tempéramento non
atto a ingratlare. Diciamo ancora mummia, che sono quei Cadaveri fecchi nel
mare.d' Etiopia, 0 ne i fepolcri dell' Egitto: come vedremo forto C. 6. stan. 52,
per intendere Huomo foverchiamente magro. Diciamo Segrenna a una donnas
'magra, dispettola, maligna, incontentabile, ¢ che non approva, ne loda: mai
l'operazione @' alerui inv

fs delkg Sid RO NGS THINY'D a: ' 2
BEN condizsonato, Questo termine, se ben pare riempitura del verso.y-0( ¢o-
me diciamo )borra, non € così ma ¢ pure che quando si vuole intendér un ma-
gre » habbiamo questo dettato vulgatifiimo scintto ye ben condizionato; xolto for-
se da quello che Ernie denne a >
eben condizionata, per-avvisare il Corrispondente della diligenza de) Latore, o
'Condottticro.. vba Sate

SNELLO leitosleggier come wna penna Queste tre yori nel presente luego Sono f-
ri nonimi

 

ie

 
SECONDO CANTARE: 4s

nonime significando, ed esprimendo tutte 1a poca carne che haveva addosso Plo-
riano, ¢ che era al maggior fegno magro. & la voce /ve//aha forse origine dal
Tedesco Skye!, che vuol dir Veloce.

ZO tenne foggiornaco. Lo trattava bene di mangiare. Gli faceva buone pele.
Che /oggiornare uno vuol dire Spender il tempovin ben cultodire, governare, eo
riftorare uno con quello che occorra', ¢ s' ula questo termine per lo pil, trattan-
dofi di beftiami, ¢ percid appropriatamente detto in quelto Juogo, perché, se
ben Floriano era huomo, era gpndimeno trattato dali' Orco come beitia da in-
grafface.

F ACESSE miglior coteyna. \ngraflafic. Per intendere uno assai graffo diciamo:
Egli ha buona cotenna; trasiato da i porci, la pelle de i quali si dice propriamenic
cotenna, che dell huomo si dice corenna folamente la pelle del capo 5:0 per di-
sprezzo, ¢ per intendere un' huomo Zotico, che si dice huomo:digrofsa corenna,
o Cotennone, 0 Coticone,

AAGVIS A di mettiloro, Volea dar di zanna al suo lavoro, Coloro che indorano i
legnami si chiamano Azeri 1 ore, ed in una parola sola Azettilori,, Questi per
brunire, o dar il Iuftro a i loro lavori si servono de identi pil lunghi,0 diciamo
maettre di cane, di lupo,o d' altro animale simile, (i quali denti chiamiamo <az-
ne, 0 fanne come vedremo sotto C, 7. stan. 54. ):¢ tal lavorare dicono xannare,o
dar di zanna, Ma qui dar di cannas' intende il naturale adoperar de i denti, che &
mangiare; ¢ (cherzando con l'equivoco dice che l' Orca voleva dar di anna al
suo lavoro, cioè mangiarsi Floriano, che era il suo lavoroy che egli havea fatto pi:
Bliandolo, ed ingraflandolo.

STANZA. LXV.

STANZA LXV}1.
— Amadigi c andava per diporto

Due volte il giorno almeno a rivedere

E piangendo diceva; O.T ato mio,
Se tu muori, che ver [ata par troppo 5

La fonte, ¢ la mortella., che nell orto S' hava dire anche dimes tele dich iog

Lascio Florian per tante [we preghiere; Ttibus, come difse BP.... Pioppo,
Trovato il cefto spelacchiato,e¢ smorta y Cosh, senza.dir pure al Padre addio,
. El acque baffe purzolenti,e nere Adonta four' un cavalo, ¢ di galoppo

Qui(dice)Fratel mio noi fiam ful curra

» Diandar a far un balloincapo azzurra,

Vici & Vanano molto ben' armato,
E feca.un cane alany havea fatata.

In questo tempo Amadigi s' accorse dalla fonte, ¢ dalla mortella, che Floria-

cane incantato., and6.a gercar di lui,

he

ny ipa,
a no era in pericolo, € percié montato a cavallo. bene armato, ¢ con un groflo
os

si partend. Spelacchis

SPELACCHIATQ, Pelato in

BP qua,¢ in la, cioz parte delle faglic cascate, ¢
s' intende un' huomo ' che stia male a fanita, ed a roba oe

sia mal veftito per la sua poyerta. 4
(oe SMORTO, S' intende che nonha il suo natural colore buono.
'ah  SLA ful curro, Siamo in procinto; fiamo all' ordine; fiamo vicini, Cxrro
yo = fon pezzi di quali G metton sotto alle pietre,o ad altre cose gravi per
rm facilitargli il moto quando si frascicano, dai Latin detti Palange.
rey EAR un ballo ix ¢

' ' azzurro. Vuol dire Esser' impiceato; perché campo az.

i Rurros' intende il campo, che fa l'aria, il quale ¢ azzurro, ¢ colui, che ¢ im-

0 f j 'Piscato movendo le gambe, pare she palit in aria, Per maggiore inteliigen2a la
dais 2

voce
1

 
|

 

116 MALMANTILE

voce campo pieeeluiemencients 5 wuol dire quel luogo, che avanza in. uns
quadro fuori delle figure, ed-altra che. vi fia'dipinto » come si dice una infegna»:
entrovi un lione in campo azzurro. Ed i medesimi Pittori ne cavano' il verbo
campire, ché vuol dire Dare il colore, de) quale:ha da efiere il campo.

7 ATO. Vuol di Fratello, B' parola usata dalle Balie per infegnar parlare a i
Bambini, come Habbo.in vece di Padre, Mamma, Bombo, ¢ simili, che per ef
fer parole labiali tornano più facili a proferirfi. Furono usate anche dai Latini
come si vedein Marz.lib, £. 95.

Aammas, atque tatas habet Aphra, fea ipa tatarnm
Diti, & mammarnm maxima mamma poteft.

Vedi forto C. 3. stan. rz., ¢ C. 4. stan. 5.

TK lodich'io. Vale per Te logiuro; Ti afficuro., Vedi Oraz. lib: 2, Ode 17.
dove parlando con Mecenate infermo, dice:

Ab te mex si partem anima rapit
AMatiirior vis, quid moror alters?

Con quel.che segue simile al presente lamento, che fa Amadigi per if Fratello,
che: Orazio fayper Mecenate:.

1T [BV S-come disse P..«. Pioppo, Significa sha dire anche dime: gli mor-
to. Questo P..... Pioppo.era 'uno, che havea poca amicizia con Prisciano., e
non oftaote fempre slatinava', ¢ fra l'altre quando voleva dire i) tale ¢ morto di-
ceva fibas, © intendeva Egli ¢ito. EB da questo fuorderto diciamo Come disse
P., 2. %Pioppo, B-s*intende il tale & morto,. ost th

Dik' addio, Intendiamo quel faluto,, che si fa nel pigliar congedo,o licenziarsi
da uno., ed ¢ lo steffo, cite i Latino Yale, usato da noi ancora come dicemmo
sopra, e vedrémo forto-C, 6ttan. 18. ¥ '

GALOPPO:, Corso divcavalio,ida i Latinhdetto'earfus gradarixs, che & in
mezzo tra il trottare, ¢ il correre. Forse meglio gualoppe fecondo Dante Inf.

Cant, 22, Bs
di rintoppo
A gli altri disse a lui, se tu ts cale e
Jo non tiverro dietro di guatoppo,
CANE eAllano, Cane groffo per caccia da Cignali,e simili animali feroci, ed &
maggiore, pil fiero, ¢ pir gagliardo del Mattino. re
STANZA LKV

 

It STANZA oe
E cavalcando con la guida, ¢ foorta L'apparir a! Amadigi agti abit
Dil sue fadeie,eul hicescars lame j Raddoler? agro dei lor mefti- vif
~ Chinnanzi gli facea per la più corta. Che per la fomigtianza atucti quanti
La fhrada per lo monte, e per lo piano; iapahio dapenaion @ Campi Elifi,
A Campi giunfe:, dove in su la porta 0 2 mance, © paraguantt
rapa Leggenidi, Stan's ' davon moltva darne al Re gli avvisy 4

- Che perchi fucreduta dwognuno, —§ ——-—' Alrrs alia figlia, ed ambi a quest
KraleCeveyovthiie Cohpoabionss oPRercia Cae è
Amadigi ee » dove dal bruno, che vedde addosso' a gli abitatori

conobbe, che era mortoiitlor Principe; subito che coftoro veddero Amadigi,

credettero ch' i fulle Florianos e'peccid molti corsero a darne avyviloal Re, ¢

a Doralice. oie ERA

 

 
 

 

SECON DOCANTARE. fry

ERA laCorte; ¢ ratroC ampica bruno. Cioé i Cortigiani, e gli abitanti di Cam-

i crano velliti di nero in: fegno di meflizia, per la morte del Re Floriana. Pecr.
4a; E.vedrai nella morte de eAMariti
Tutte vefiite a brun le donne Perfe

© Da aleuni'fi dice wefire #turto y 0 a feorrucciv.\ Ma credo che essi habbiano ac-
¢actate queste voci da i moderni Romani. t

AGRO dei lor mefti vifi. Viforagro vuol dir Malinconico; @ si dice agro perch
ung, che habbia hauato qualche disgutto; fuob moftrarlo nelia faccia con incre:
spav la fronte, ¢ fare:altri gefti appunto come fa uno, che mangi cole aspre- 5
acide,oagre. E però dice Raddoler ? agro dei lor mefti vift, che significadi me-
lancolici, gli fece ricornare-allegri:. Ad

CREDIT O «i Campi Elifi, Creduto nellvaltro mondo:;¢reduto morto., che
eens Eiifi dalla superftiziofa Geatilita erano creduti-ii Paradifo.. Vedi foro

. 6. tam. 32. '

PARAGYANTO, Mancia,o regalo. Paraguanto, dono, iregale, mancia ap-
pretio dinoi si possono dir finonimi; E se bene molti vogliono-che: manvia ye pa-
raguanto si dica quello, che dal Superiore si da all”inferiore; © donoieregalo G
dica quello, che dal' inferiore si da al superiore (che-in-questo cafo now si dircb-
be mancia ) 0 dali'uguale, all' uguale, nondimeno nel buon parlar familiare si pi-
glia uno per l'altro, nes offerua tanta strettezza, ed il nostro Poeta pure si
vede nel presente luego, che non oficrua questa distinzione come poco, 0 punto

c Orin ta STANZA LHIX
Doralice brittande a tai-sovelle Enon fear pil nella pelle
<A rinfronzirfi ardoffene allo specchio, Salts fuor dipatarre innanxi al vecchio,

Sb mefse il grembinl bianco le pianclie Ed invontro correnda sil: sia cognate;

UI veRxo ab collo,e i ciondoli all'erecchio, Ecco Florian ( dicea ) rifucirato,

Dordlice sentieaquefta nuova si raffazzond, ¢ subito-corse incontro al suo co-
gnato: Amadigi, credendolo Floriano suo marito'.

BRILL ANDO. Giubbilando.; Brille si dice uno che sia allegro per haver beuuto
molto vino. Vedi foro C. 6. stan. 35. ed è il primo grado di briaco dicendofi in
agugumento Brivo scorto, briaco, spolpato, Molti voghiono,, che questa voce brilla~
re venga da'biril/è tpecie di gioia', e che brillare significhi (cintillando-tremolare,
appunto come fa il biril/e, ¢ come fanno coloro, che sono fonmmamente allegri,
©che habbiano foverchiamente beuuto.:

RINF RONZIRSI, Ratfazzonarsi, abbellirfi, aggiufarsi la persona tolto dal
Latino refrondefeere, che vuol dir quando gli alberi si veftono disnuove frondi,
le quali nell' antico Fior, si dicevano fronze. Terenz. in Heaur.

: —— Et noffi mores mulierum; s

¥

> % Dammolinntur,@ comunrur, annus off,
+ Cioé si rinfronzi(cono f dice l'espositore Landino js" accomodano, ed accon-
iano la oom ee accu ae og Hh

~ CIONDOLI al? io, ini. le gioi portano
“denti all' orecchic, Latino Zaawres Bis aor cama pendent per cere

ciondoli.

 

 

 
118 MALMANTIEE

VEZZO. Quell'ornamento di gioie, che le Donne portano alicollo

PIANELLE, Specie di scarpa, che cuopre folamente la parte dinanai del pie+
ce,da i Latini dette fandalia, E,con dette gioic adornandola,moftra il Poeta qua-
le posla eflere una Regina di Campi', che non eccede il Inflo d' una pulita con-
tadina de i Contorni di Firenze.

NON pus frar nella pele. Non ped aspettare, perché l'allegrezza le ha:cagio-
nata una inquictudine tale, quale /ogliono havere tutti coloro, che dovendo con-
seguir qualcofa di lor guflo, ogni kbra d' indugio. stimano mille. A questo
si pud applicare quell' 4 Sermento torus eff de i Latini, che pare che espri-
ma quella inquictudine, che fuol cagionare l''ira; Lasca Novella 5. Si che per la
paffione, e per la rabbia non poteva [rar nelle cuoia,

COGN-ATO. | Latini per cognazione intendevano ogni forta di parentela.Ma
noi per cognaro intendiamo un Fratello di noflra moglie youn marito d! una fo-
rella di nostra mogli¢, 0 un marito di notira Sorella, ¢ nello stesso modo respet=
tive il Fratello del marito, si dice cognato, come intende nel presente luogo s

INNANZI al vecchio, Cioè prima che ulcifie di casa il Re suo padre, inten-
dendofi comunemente Padre quando in questi termini si dice il sage

 

taluolta il Padre sia giovane,
STANZA LXX.

Noi vi facevam morto; o gindicate
Selacarotac era fata fital
Pur noi ci rallegriam, che voi tornate
A confolar la voftra gent' afflitta,
Domandar non vcorre come state 5
Perthé v' havete buona soprascritta,
E fiate graffo, ¢ tondo com xn parco
Per le carezxe fattevi dail' Orca,

STANZA LXXL

AM immagine così perch' io non vera:
Tu fat com' ell' ando, che fufti in cafo,
So ben, che mi dirai, che non fu vero
Ma la bugia ti corre fu pel nafo,
Hor basta, Tx ritorni fano, ¢ intero,
(C' a pexzi tu dovevi esser rimafo )
Per 1a Dio grazia,¢ sua particolare,
Perché tel' ha voiuta risparmiare,

STANZA

Mio padre te lo disse fuor de denti,
Ed io pur te lo aiffi a buona cera
Lon una volta, ma diciotto, o-venti
Che l' Orce ti faria quatche billera;

io,ancor che

STANZA LXXII,
Dunque s ei fa cosh gli è neceffaria,
Gh'ei non sia la quelfurboch'unlotiene,
Anzi tutto iLrevescie, ed il contrario
Mentr' egli tratta i foreftier si bene g
(Ed io, che gid havea ful calendario,
Gli voglioinquato.ametuttoil miobene,
Perch'ei non t ingoio; Se ben da wn lato
Ti flava bene, havendola cereato,
STANZA LXXIIL
Cast nel mezzo a tutta la pancaccia y
Ch'é quivi corsa,e forma un giro tonde,
La sua caponeria gli batta in facia,
E quel ch'ei ne cavo po poi ingquelfonde
Già che (dicea.) con l'andar' a caccia
Ai disperto.di tutto quanto il mondo
Cavafi,fenra far alcun guadagna
Die occhi ate,per trarne una alcopagno,
LXXIy.
Ma tu volefti fare a gli feredenti,
Perché te ye firuggei come la cera y
E quafi un rischio tal fuffe una lappola
Voleffi andaxxi, ¢ defti nella trappola.

» In queste cingue ottave moftra,, che Doralice ingannata dalla fomiglianza,

che haveva Amadigi con Floriano,gli fa un dicorso di congratulazione mefeola-
ta con rimproveri, col quale il Poeta esprime aflai bene il coftume delle nostre»
Eemmine in simili caf; tacendo che.dal principio del discorso, che ¢ 1a congra-

tula-

 

 
SECONDO CANTARE., 119

tulazione, lo tratti del Voi, ¢ quando viene a' rimproveri lo tratti del Tu.
SE La carotac' era spata fitta. Ficcar carote vuol dire quand' uno inveatando
qualche novella, 0 trovato,lo racconta poi per non suo,, acciò che pil agevol-
mente gli sia creduto; fiche Doralice vuol dire; guardates' ella c' era stata data
a crédere. Vedi forto Can. 6. stan, 67. ¢ 68. Mattio Franzefi nel Capitolo sopr'
alla Corte dice:
: 'Chiama piantar carote il popolacciv
Quel che diciamo: Adofirar nero per bianco
Per distrigarsi da quaiunque impaccio
E per tutto il medesimo Capitolo discorrendo sopra questo detto, moftra che
thabbiamo anche iliverbo Carerare 5 ¢ Carotiere, quello che ficca carote. LU Lalli
En. Tr. lib. 2. stan. 2.
Egli che ben conobbe al primo tratto
Ch! era in un campo da piantar carote
Si dice Piantar carote, perché questa pianta fa grofla radice, ecresce assai nei
terreni dolci, ¢ teneri, ed uno facile a credere si dice Homo dolce ye tenero.:
VOL havete buona sopraferitra, La faccia fuol' efler dimoftratrice delle paffiont
interne, ¢ però dicendofi haver buona sopra/critta's' intende haxer biota fanitd,co-
me dichiara il Poeta medesimo dicendo; Von occorre domandarni come voi feate, per
whi ficondsce dalla buona soprascritra, cioè la fembianza, la buona cera, ¢d aria.
del. volto ci dice, che vai state bene. E cosila voce sopra/critra, che vuol dires
Inscrizione, che si fa alle lettere, ci serve per intender quanto sopra s'é dot-

to.
LA bugis vi corre fu pel nafo. Tu daicolore. Tu timuti-dicoloré in vifo, per-
ché tu hai -detto una falfita, Twi oculi declarant, Lo Scoliafte di Teocrito spic-
gando-quei versi dell' Iditio.12. che in Latino furono così tradotei: Verim ego te
« laudans yformofe; baud mentiar umquam, Nec tenni gravis innascetur puftulanari;
sdice così,.Vuol dire; che tiekdodarti, io non mentird,. non mi nascera sopra.,
al nafo la bugia; poiché alcuni fogliono chiamare certe bollicine bianche, che»

vengono fu pel nafo', bugie: c\colui che leaveva, era natato, come bugiardo.
'Fin qui lo Scoliafte.

RISPARMIARE 4 0'ri[pinrmare, Vale:per petdonare.. Quis intende 1' Orco
~ehe non ha voluto far male alcuno.. '
HAVER uno ful.calendario, Havere a noia, o'vero odiar' uno.
QUANT O w me gli vo turto il miobene.. Pee quanto s' aspetta ame gli porto
tutto quell' affetto, che si pud portare; |' amo di tutto cuore.
TI fava bene. E' \o stetio che Ti flava il dovere.. Tornava bene,-che 1' Orco
» t havefle ingoiato, perché t' haverebbe fatto quello che tu meritavi.
PANC.ACCIA, Così si chiama da noi quel luogo dove si ragunanoi novelli-
si per darfirle nuove PunsVaitroyed ha questo nome di Pancaecia, perché nel tem-
di Rate questi rali si radunavano già per sentire il fre(cowicino alla Chielas
» Cattedrale, fedendo sopra aun muricciualo coperto di tavoloni, 0 panconi', e>
«da questi prefeil nome di-Puncaccia. Eda questa pancaccia, Pancaccieriye Pancac-
vai intendiamo quei perdigiorni., che stanno oziofamente ragionando de i -fatti
daltci, ed in questo fenfo ¢ prefo nel presente Iuogo, che dicendo quei della pan-

ALLIS y

 
120 MALMANTILE

caccia, intende una quantita di questi Crocchioni.. Vedi sotto C.6: stan. 69. Can-
ti Carna(cialeschi,.@hi.vxol udir bugie, 0 movellacce Venga ascolar coffuro; sare
Sf fian wntia it dd fs te pancacce,

GOLA butrarin faccia La sua caponeria., Gli rimprovera la sua oftinazione »

VEL ch' ¢ ne cavd po poi in quel fonda, Quel ch' ei guadagnd, pert ona
fine delle fini, 0 in ultimo degli ultimi. Tanto servirebbe dir po: fenz' aggiu-
gnerui ix quel fonda, ma così & il nostro coftume in simili cafi per dar maggior
emfafi, quafi dica una fine pitt la delle fini, Vedi forgo C. 8. fan. 51.

CAV-AR due occhi a te per trarne uno al compagno,, Detto vulgatitiimo, che ci
serve per e(primere Far 4 se molto male, per farne  packissime al nimica,

FVOR de' denti,, Apertamente; chiaramente¢ il Lat. Eoqui, ed &il contrario
di parlar fra denti, o a mezza bocca, che significa non si laiciaré intendere, for-
se ¢ il Atuffitare de i Latini.

et BVONA cera. Con allegea faccia; cioè non sopraffatto da collera,o altra
paflione, ma con apimo ripolato; diciamo anche ful fodo, ful ferie tolto par
Serio, admonere. li Lalli Eo. Te. C. 4. stan. 103.

Prega y Seonginra, ¢ dille a buona cera,

AILLERA, Burla-nociva,o snon cattiva del tuto, aimeno che non piace;
voce corrotta da Wil/era voce antica che vuol dic Villania,, «2%

TE WE frnegei come ta cera. liverbo fruggerfi, che vuol dine Ligutart fer-
ve a noi per farsi dered' uno che ard ae » Ue Lali
En, Tr. C,.4, fan, 109.-disse. 5 i d¢ Motsiss

Che se ne firugge come le candele.

LAPPOLA, Cofada non timarf. L'erba da nostricontadini'chiamata Lap.
pola fa un feme picno d'acute spine, ma fragili; EB però dicendofi: nom do stime tna
Lappola,s' intende non lo fimo punto,¢ s' ula per.lo più trattandofi divbravura,
¢valore, alludendo a quell' armatura di spine y chehala La » le qualifes
pon fon saraee 89 aeety Sananne. se belesemme te ag

agilissime..

DEST I nella Trappola. V' incappatti, Vi rimaneRti gee Te dnquewtn inci
si. Trappoja intendiamo ogni forte d' antifizio, che si trova per pigiiare ani.
mali tanto di terra, quanto d' arias ed? acqua, donde 7'rappalare valyper Ingan-
nare. Ma 7 rappo/a strettamente prefa s' intende un' aria per per eerste
ed una esicdia rete da pescare ha ae di 7

 

Su egsd Tr. da.quaterini y Pe Invenzioni:per e fare i
ANZA TXKV a STANZA L XVL

ai eee 97 Ma perch' ei non credea veder mai l'hera
E “as il fordo ad: ogni suo quefice »  DY baver il.suo Fratello a faluamento y
da fiben' attingea.da questecofe >... Daun ganghere a tuctiy.e rorua fuora
price 4 Florian porea eer seguita ys -Dietra al sua can veloce'come it vento;

immaginandofi es apposey  Neera un trar di mano andacaancera

vicemne Atoglic, ci [uo Mariea, ed caccia al! Orcoch' ei vi dertedrento
Bch' egli essendo tutto lui maniate.....  «Come il Fratel vedendo un bebsignale,
Fulje pel (uo Fratel dacgnuncambiato, si se wale quante lui dolce di fale.

ab b>

STAN-

“RMS i ie aa ae

q

 
yh
ity

 

—

SECONDO CANTARE: — 120
~~ ASTANZA LXXVIL STANZA LXXVIII.
Che fegnitollo anch' ei per quelle strade EiquandotOrco poi venue anc!ia lui
Dond' ei conducel huomo allafuatana, et dar parole com quei tempi firani,
Ove rmentre diluvia,edal Ciel cade > 'Bd \allufeio fatea Pin da Montus
OB broda ye cect, Criftianello intana., Affin chet' arme  ¢4 caniegli allonrani
OcEd-egli tanto pai lo persuade Eidiffe: Suipiccin pighacald,
oCw ef lege ivcani:, ¢ pif durlindana, E chiapparata [pada con dueimani
Stavende havuto innanzi la lezione, Si lancto fara, equivi a più non peffe
Si fvetcefempre mai fodo al macchione. Gli comiacio aimenarile man pel defo.
; a S THAW ZA LXXIX.
E mentre chor di punta, ed bor di tagtio \\ ~ Tal-che tatto forato come nn vaglio
Di gran finefire fa, di Linghe Witee.; Ut power: Orco\al fin cade,  bafifee,
Pike preftc che non va firale aberzactio Edheraquelte raps, e quelle: macchie
~\Mbcan savicuta anch' eg lize ribadisce; Rimafe a far banchetro alle Cornacchie,

Amadigi/argumentd dal discorso di Doralice', che-ella fulle Mogli¢di Floria-
nO.,,.¢ compre/o quanto poteva efler' avyenvto al medesimo; ¢ però senza dar al-
tra' risposta dette addietro, ed uscico di Campi; fu'dal Cane guidato alla tana
dell' Orco |, il.quaic.fa da Juiicon ajuto' del suo cane', ammazzato.

44.4f, Questo avvcrbio che significa In alcun tempo serve anche per negativa,
come & nel prefenicJuogo 5 ¢ come l'usd pity volte il Boccaccio ed'in specie Nov,

. Mai frate il Dinvel tivcirecay ccs B Nov. 54. Che mai ad animo-riposato si fareb~
gee witrevare yo Nov, 77. Adai di cid che hora mi parti dubitai, Matteo Villani
lib, 8. cap. 39. / Perugini mas si vollero dichiarare, ed in molti altri luoghi de) Boc-

caccio, del Paflavanci 5 ¢ d' aleri Scrittori del buon secolo si trova usato per ne-
gativa. Ho voluto dir cid in questo luogo per toccare'la difefa dell' Autore dalla
critica datagli d' haver usato questa voce 44si per negativa senza l”aggiunta della
particella we, © om, ¢ senza correlazione alla negativa anteposta nel medesimo
periodo:, ¢ che tanto vale il dire /o now farò mai questo, quanto il dire. Jo nai fas
xo.questo, E.mi rimetto all' uso yed al TORTO,, E D/R/TTO del P. Bartoli,per
la difefa:di 88 a ee ai
i sFECE it fordo,..Finfe di-non sentire. s
WAT TINGEA da quefie cose. MW verbo attingere 0 attignere, che & il Latin'
me eens un Juogo 5 0.4 un fine; Azeram attingere: da noi € preloy
ed usato come il verbo aurio, che vuol dir Cavar V'aequa da i pozzi, che noi di-
ciamo attignere y ed in significato di Comprendere, vedere', udire, ocults © auribus
haarire « EB nel signi di cen € prefo nel presente luogo.
8' APPOSE. Verborncutro che-val per indovinare: Ed attivo vuol dire Dar
la i Bund « Jom apposi di chi baveva fatto il male, e però lappost a lai. \o
a ist oe havea fatto il male, € pero ne diedi las
adei.> onc aa! o5 Yb chsh ee Y
 TVYTTO lui maniato, Come lui per appunto: Similissimo a ui: Facto a cay
pelle, che vedemmo (opra in questo C, stan. 19. Lacy Nov. 7. dice: M1 qual fan-
taccio veftsto de' panns del Pedacoga,tutto maniato parea Ini, lo credo che sia parola

Sorrotta da mizvaro cioé diligentemente dipinto, 0 forse corrottamente derivato
dai Latino barbaro Emanaeus, tanto simile a lui, che pare emanatus ab illo,
3 2 NON

 

ae ak ee
se MALMANTILE

NON credea di veder mai ? hora. Amadigi havea così gran desiderio'di vedere
il svo Fratello libero, che dubitava non fade per arrivar mai quell' hora, ed ogni'
momento, gli pareva un' anno, 2
' un ganghero, Da volta addietro. Ganghero diciamo uno st per
uso d' affibbiar le velti, fatto di filo di ferro, 0 d' altro metallo, il quale € fatto
in forma d' uncino, eda quella rivolta, che egli fa, dare il ganghero intendiamo
tornar indietro. 'Retrorfum vela dare. Dare il Banghero » diciamo quando la lepre
fuggendo avanti al cane, torna indietro, ¢ lascia correr il cane, che portato
dalla velocita non si pud ritenere, ¢ yoltarsi subito come fa efla, che in tanto pi-
glia campo in manicra ch'.ella (Campa, dal che diciamo Far lepre veechia per in-
tender tornat indietro. Vedi sorto C. 10. stan. 23. '
NON fu si doice di fale. Non fusi credulo: Si minchione: Si sciocco. Viaas
vivanda poco falata si dice dulce di fale, cioè sciocca. Donde esser senza fale, 0
non haver fale in zucea vuol dire Huomo sciocco, senza giudizio, senza ceruel-
lo. Safechiamiamo IL arguzie, e detti ingegnofi. Vedi otto C, 8. stan. 26. Di-
ciamo if tate ¢ dolce,¢ senza l'aggiunta di fale intendiamo è corrivo, creduloy
minchione, ¢ senza giudizio; ¢ per coprire più questo detto, usano molti dire»
Lupinaio (che vuol dir colui che vendendo per Firenze Lupini va gridando dolcé
dolci ) per intendere Cofui ¢ dolce, Qui dunque vuol dire, che Amadigi non fu
corrivo quantojera stato il Fratello a credere all' Orco. Boce.Gior. 4. ns 2, A444.
donna Zucca al vento, la quale era anzi che nd a. dolce di fale, Latca Nov. 2,
Experch? egli era nato in Domenica, non fendo la gabelia det fale aperta§ senne fempres
molto bene del dolce.
TANA, Caverna, grotta, buca. Donde intanare, entrar nella tana,
BRODA,ececi. lntendi acqua ye gragouola. Fu un ragazzo ghiotto*delle
civaic, per il quale fo padre ( per mortificare geen sua gola ) ordind, che nel-
la sua feodella non si metteffe altro, che il puro brodo de'ceci, o d'altre civaie
respettivamente, onde il peer ragazzo vedendo gli altri con le fscodelle piene»
di legumi si disperava,, Ed eflendofene andato un giorno in camera mentre pio-
veva se ne fava alla finettra gridando acqua, e gragnuola, € quetto per la ia,
che haveva, che si stagionatlero i legumi per gli altri, enon per lui. Senti il
padre questo uo gridare, che gli disse: perché preghi il Cielo a mandar Ia grandi-
ne, cola tanto nociva? L'aftuto ragazzo per scampar la furia subito rispose:
Padre mio io non ho mai desiderato, 0 pregato male per nefluno, ¢ se io prega+
vo che insieme con l'acqua veniffe anche della grandine, ho voluto intendere, che
il Ciclo vi metteffe una volta in testa di farmi dare con tanta broda una voltas
anche de'\ceci, che dixquefti intendevo quando dicevo gragnuola. Ii Padre rife
dell' aftuzia 5. dette ordine, che per 1' avvenire fuffe trattato., come gli altri.
E da questo,intendiamo ai » ¢ gragnuola, quando diciamo broda, e cect.
CRISTIANELLO. E' detto d' avvilimento, ¢ significa Huomo dappoco,0 di
'a fortuna 5 © di piccola figura; che i Latini dicono bomuncio, ¢ nob talvolta
in questo fenlo diciamo Homicciuolo. ae
DIRIND-ANA, Intende la spada,e piglia questa denominazione dalla famo-
sa spada d' Orlando Pajadino, la quale da i Poeti hebbe il nome di Durlindana,
0 Duyindana y '
W #iA-

 

 

 
  

 

 

SECONDO CANTARE, 123
4 HAVENDO havuto innanzi la lexione, Essendo stato prima informato; ayvila-
t0,, instruito.: Cioé havendo comprefo dal di(corso di Daralice-, che questo cra
quell' Orco, che ingannava. 3

STAR fodo al Macchione. Intendiamo non condescendcre alle richieRe, o'non
Gi lasciar lufingare dall' esortazioni di alcuno. Questo detto viene da quegli-uc-
celletti, che stanno per le ma¢chie, dove si tendono le ragne, i quali, per eflere
Mati altre volte molefati, hanno imparato, che quello (cacciargii col battere la
macchia era di lor poco danno flando fermi, però ncn si muovono a ogni romo-
re,¢ questi si dicono Par fodi al Afactbione, Di tali uccelli si dice anche accivereati,
Veditotto C, 9, stan. 22,

FACEA Pin da Montui, Cioè facea capolino, che vuol dir quel che accen-
sd sopra C, 1, stan. 7. Questo detto viene da una canzonetta, 0 villanelia,
che dice. A

Pin da Montui, Fa capolino
Dreto è Menyhino, E Mon con lus, ec,
Plauto disse Ex infidijs clanculum ancupari.

SP-piccino. B* modo di incitare il cane contro a uno, El irritare, oimmittere
de i Latini, che noi diciamo anche ammetcere. Vedi (otto C, 11. stan. 29. si di-
ee anche ai/sare verbo originato da quel fyono, che fa la voce dicendofi: /u /a; 0
dalla parola Ra voce antica » che vuol dire Ira, dalla quale habbiamo il verbo
aizzare 50 adizzare, 0 aiffare, Dan, Inf. C.27,

Dicendo, fa ten va: pitt non? aizxo,

A PI non posa. Con ogni maggior potere; Quafi dica con animo di seguita-
rea far quella tal cosa fino ache non sara stanco, € non possa pil.

MEN AR le man pel deffo, Adoperar le mani nella persona d' uno, cioè-Per-
quoterlo. La voce dof dal Latino dor/um, da aoi s' intende per tuto il torfo
dell' huomo, ches' eccettuino da molti il capo, le braccia, ¢ le gambe,
Lasca lib. 1. Nov..7. Non contento di ricercargli col baftene le braccia y ¢ le gambes',
volle ancora con esso ritrovargli tutto il doffo,

GRAN fineltre, ¢ lunghe frrifee. Gran ferite di punta edi taglio Punitim, C-
tefim, disse Vegezio. Dice frri/ce per la similitudine che ha una hunga ferita di
taglio con la stri(cia, ¢ lo fa per esprimere che eran ben lunghe » come dice fize-

requelle di punta ppctche s'intenda, che eran larghe. 3
CAVVENT ARS!, Spingerti, gettarsi,o andar velocemente » 0 con impeto
alla volta d* uno, che i Latini dicono irruere.

R16 ADIRE, RibattereQuando si mette un chiodo dentro a una tavola,¢ che la
punta di cflo chiodo pafla dall' altra parte,la detta punta si piega, ¢ si riconficcas
perce il chiodo facia V effetto d' una Jegawura; ¢ per far Qacfio, uno baie in

u la punta del chiodo 5 eV' altro tiene a riscontro in ful capo'delchiodo un fer~
£0 5 € quelto si dice ribadire; e però Amadigi da una parte, ¢ il ca~
ne mordendo dallt altra l'Autore per esprimer questo atto si serve del verbo yba-
dive usato da molti ed in questi termini, ed anche per licare,

FORATO come un vaglio. Havevano fatto nella persona dell' Orco pili buchi,
€ tagli che non ha un vaglio, firu le si fepara il grano dall' immon-

: in vaglio, col
dizi¢, decto dal Latino Yannns + etal Cavell dal Latin Crsiram ye rie
2 lum

 

 
¥G

  

124 MALMANTILE

Jum, voce usata dall' Agricoltore Palladio:, Questa-comparazione' era: usitaan.
che da i Latini trovandofi in Plauto Carnificum cribrumspariando diua fetvo, che
era Nato mal concio dalle baftonate.

BASISCE. Muore. Questo verbovha forfel' origine dalla Greca voce Bafis =
che vuol dire incefxs, ¢ che intendiamoiil rale se n' andé, peril tale mori, che di-
ciamo basi; vedi.l'Orcava 82, seguente,¢ da questo verbo deriva la voce basto, che
vuol dir huomo senza sentimento.,.¢ quafi morto. Meller Gio; della CaQnel

Capitolo del Martello:d' Amore 'dice.

Perché ti guardi torto'la Signora'; t
Parti haver le budelia in un catino,
E doventi bafiro alloraallora.

Vedi sotto C. 6. fan. 97.
STANZA LXXX

Amadigi dipoi fece pulito,

Perché trovato havendoil suo Fratello

Con una barba lunge da Romito,

E pitt lordo, epiit unto d'un panello,

Lavatolo, e rimeffugli il veftito,

Ch' era ancor quivituttoin unfardello,

Lo ricondufse a Campi, ove la Adoglie

Di lui già pregna,appunto waren lee le doglie.

Ata prefto come luiporrai dir mio.
Hor senti pur: Bafite Periine
eAnco eAmadizi subito tuo Ziv

Vine ator donaye n' hebbe ut bel garzone,

NZA

STANZA LXXXI
Corse la Levatrice, ed in essetro
\Framille boime,fe' soldi,e doglien hora,
Partorigli una bella pi/ciallerto
Che fufti tu, poi'derta Celidora,
E maritaca al Re, come s' ¢ detto',
Di Maimantil del qual tufei Signora;
Ne fei, e ne farai, io lo raffibbio,
'Sebennen puoi per bor dircomeitmibbio,
LXXX rs:
Che Baldo fu chiamato je quel fon' io,
Che poi crefeiuto detto fon Baldone '
Hor eccotisdal primoval terzo grade
Narrato tutto il nostro parentado.

Amadigi trovato: il Fratello Floriano:lo: rivefth; ¢ lo-riconduffe a Canglt dove
Doralice partori Celidora; ¢ d' Amadigi nacque Baldone.. E. con. terminare il
racconto,termina il Poeta il fecondo Cantare's

FECE pulizo, Feceil negozio aggiuftatamente, come: andavSoihadls

BARBA da romita, Barba lunga,¢ incalta:, che-tale perio pu eer

barba:de i Romiti.,

LORDO. Sudicio Ihifo. Dal tno Lari sche wiol dir Livido 2 quali' 'per

lorum cuffum, © livid E

Be P
mo, ma ancora ad ogni più 7o stcumento,sopra il q
chiamiamo un:

PANELLO,. Così

nati da iwenti,a! squetti refiftano,
flor. lib. eee

parturiente da i Latini'd
HOLME. Voce' che esprime:
diceyano bei mibi 'ye noi forte

'habbiamardal-Greco hoi moi
SG [aldi 5 ¢ dogtion' bara pot. aS 2 burlare chi'taiuolew: oe

E... Raccogliteice;. sams ooh, ries! la. Creatura dalla

 

'alPhuo-
quale sia schifezza.

viluppo'di cenci intinti-nell' olio'; fego i °
altra materia oleacea, ¢: bituminofa. gail.
narie in occasione webises iene

tron Tome ergy
eminent?, ¢ domi-
hin i val lo stesso. Vatchi

-difétto'd? clio'.

 

'animo. ype di a che i
E quell'

  

 

 

 
SECONDO CANTARE. 125

firathihiirica, © farlezzj senza cagione yo per dolori leggieri, che noivdiciam>,

Fareil monelio', enone riempituca intentata dal/Poeta, ma è pur cos! in uso, di-
cendofi a questo tale: O pover' huomo! dime | fei soldi, ¢ dogliene bora; ¢ si no~
shimano Toth di monete.per haver occatione di dire dagtiene, che & il verbo
dare, ed in questa occasione si dice, perché ha similitudine con la voce doglia.

PISCIALLEFTO } Wnabathbina', Quando una donne partorisce una Fem-
mina, nitda diguelle donne che sono attorno alla. parturlenterlé yuo) dar las
nuova, che ella sia femmina, ma perché pure al fine ella lo deve fapere, per non
profferire la parola femmina dicono: Vaaipiscialletro;-Vnacome'me ye finili, E
da questo' noi habbiamo far” ua bambina, che vuol dir Fare un' errore.

LO rafibbio. Lo replico.

NON puoi dir come it nibbio, Cioè non puoi dir Mio. Il Nibbio uccello rapace
non fa altro canto, ne si fente da lui altra voce, che ua certo filchio gio strido 5
sche par che suoni mio mio, ¢ da questo per avventura i Latini lo diconAdiluus,o1i

gauolieA4ilano, ¢ i Francefi Azilan; E noi da quelta sua voce volendo espri-
mere, che una cosa fia' veramente mia,dichiamo: Posso dire come il nibbia,cioè Mio;
l'autoré Jo dichiara nel primo verso dell' ottava seguente dicendo > eta prefte
come lui potrai dir mio,

BASITO, Vedil ottava 79, antecedente.

Z/O. Fratello del padre, o della madre, o-matito'd' una forella del padre, 0
della madre: 'Quié fratello del. padre'.

VN betgarzone. Cioeun figiiuol maschio, Equi il Poeta seguita.a moftrare if
coftume delle nostre donne accennato nell' ottava antecedente, che quando il par-
to édi ma(chio joghunia di loro vorrebbe esser la prima a: darne lal nfova, es
dannosalia:creatura fempre qualche epiteto,, come ma bel garzone.s ninbel giovane,
un.garbate fantoccione, um bamboccione a' importanza. Vedi sopra in questo C. stan.
491 ma quandoé fematina-, tutte: le afiiftenti ammutoliscono, 0 quando pur' al
t ine lovdicano®, dannovalla creatura epiteti d' avvilimento, come 2i/ciallerto, Pi

sciatcbera', zm ¢uaintuccia,, © simili, come habbiamo detto poco.sopra.
« 4B nostroparemeado pla poltca Genealogia: In che modo noi fiamo parenti «

FINE DEL SECONDOGANTARE.

sod opita 4

~ -RS

aa.

i

  
 

INARA TA
Ce Paes aetealaet
TERZO CANTARE,

al 70.
'22 ARGOMENTO. §
5 Vengon d Arno 2 [econda i legni Sardi, ?
Sharcan le genti, e vanno a Malmantile,
& she
ote Nascon grandi scompigli in quella piazza,
e335 E ognun si fugge in veder Adartinazza,
cuve SPL Cte ws cure
PRE IAT

Ma per vari accidenti i più gagliards
STANZA TL STANZA-IL

  

   

Non fan quel tanto, che di guerra è file.
Arma i suoi Bertinelia, alza flendardi,
E moftra in debol corpo alma virile, R

Nhe sia avoerxo a frarfeneafedere E pur chi vive y sta fempre to

| V Senzafar nullacon le mani in mano, va ber a Lincs '
E lantamente puo mangiare, ¢ bere, Perché al Mondo non è nulla di netto 5
E in fofha,e giuoco viver litte, ¢ fano, E non si puo mangiar boccone in pace 5
Se gli fon rotte nova nel paniere, Hor ne vedremo in Malmantil Veffetto,
Considerate se gli pare firano 5 Che immerso nei piacer vivendo abrace,
Ed to lo credo; ¢' a un' affronto tale Non pensa che patir ne dee la pena,
e4 certo ognun l'intenderebbe male. E che fra poco s' ha mutare feena,

Ii Poeta volendo trattare dell' aflalto dato a Malmantile, ¢ del disturbo, che
@ per apportare l'elercito di Baldone a quelli spenfierati, che sono nelia Terra,
introduce il presente Cantare con nna reflefione 5 che sia un gran disturbo a co-

loro, i quali standofene coi loro

commodi, ¢ senza un minimo pensicro, si v

gano (opraggingnere chi gli privi di.
rebbono di gran disguito, pelt an

ucfti loro agi; mentre simili accidenti fa-
a coloro, che\non steficro con tutti i lor

commodi; perché niuno, 0 bene,
tutti fiamo fottoposti alle dilgrazic

omale, che gli stia, vuol mai ricordarGi, che
,¢che nel mondo non si aa felicica perfecra.

ST ARSENE con le mani in mano, A cintola, 0 in feno. Si dice d' uno, che
sia tutto dato in preda all' ozio, ed alla poltroneria, ¢ che non vuol Jayorare.
Viv accidiofo, nighittofo, 0 scioperato, 1 Greci, ¢ Latini ditlero: Jn choenices
fearre: de bomine ociofo, & desidiofo. '3

GVAST AR ? wove nel paniere, Guaftare i difegni altrui, Traslato. guaftar

a are y nuova

in we

 

 
TERZO\CANTARE.- 127

Ituova nel nidio.; dove fon dalla chioccia covate. Vedi Efopo Favola dell'Aqui-
la; edello Scarafaggio. E il conatum frangere de i Latini.

SE gli pare frrano. Se gli par.duro', ¢ difficile a foffrire. Vedi sopra Cant, 2.
stan,'21. 5 ed il proprio ¢diffrano. Stravagante, o foreftiero 5 o non
del nostro parentado; valendocene in tutti questi, ed altri significati, come segue

| ne i Latini della voce extraneus,

; AFFRONT O. Significa Aggreffione, affalto,, abbaccamento.. Vedi sopras
Cant,1. stan.29. ma si piglia: ancora per Soprufe, come ¢ prefo nel prefeate»
luogo. '

BERE unasciroppo, che dispiaccia, Sopportar per forza una cosa, che sia di
disgutto 5 che in Latino: si disse: Calicem bibere; perché Calix era una specie di

) bicchiere, col quale gli antichi beveyano caldo, come appunto si bevono gli sci-

s roppi; ¢ lofacevano ancor' essi per: medicamento; ¢ per conseguenza era tal be-
+yanda', come.a noi, per lo più, di poco gusto,
° WEL eAiindo non è nulla ds netto. 11 Mondo non ha felicita perfetta. Vricuigue
dedit vitium natura creato,

VIVER a brace. Viver'acafo, senza regola, o considerazione. Ha forse-

efto. detto origine dalla mifura, che si fa della brace, che per esser cosa vile, ¢

i poco prezzo si mifura inconfideratamente senza guardarea darne un poca pill,

©um\poca meno. Da questo poi habbiamo /braciare veduto sopra Cant. 2, stan.

10, che significa Confumare il suo inconfideratamente.

MVT ARE foena, Mutar faccia, 0 stato, mutar maniera di vivere, Traslato

 

 

dalle prosp 3 dove si le die » quali prolp sono da noi
vu Igarmente chiamate Scene.
' STANZA IIL, STANZAUIV.
Era in quei tempi la, quando i Geloni Quand in rerra  armata con la foorta
Tornano a chinder l'offerie de' cani, Del gran Baldone a Adalmantilsinuist,:
'E talun, che si [paccia i mitiioni Ond' un famiglio nel ferrar la porta
~ | Manda al prefta il tabi pe panni lani; Senti rumoreggiar tanta genia.
Ed era appunto l oraych' i Crocchioni Vn vecchiocraqueft hnom di viffa corta,
Si calano ull affedio de' caldani; Che L erre ogni bor perdeva all offeria,
» Ed escon con le canne,e co! i randelli . Tal che trail bere, el esser ben a! eta
* Tragaxzi a pigliare 5 pipiftrelli. Won ci vedeva pik da terza in la.:
Descrive la flagione, che correva; quando la soldatesca sbarcd in terra, ¢ s'av q

 

vid verso Malmantile foto la condotta di Baldone; e dice che era ful finire dell'
Autunno, poiché cominciava a diacciare, ed i ricchi finti mandavano a impe-
gnare i veltiti da state per rifquoter quelli da inverno; coftume assai usato da co-
loro, che sfoggiano in veftire quantunque fieno poverissimi, ¢ questi intendi
vicchi finti, che si pacciano i milliont »che si fuol dire; Adexzettin non rifquate Pan-
talone, cs' intende, che gli abiti da state non vagliono tanto, che impegnandoli
poslano rifquotere quei da inucrno', come appunto è }abito povero di Mezzetti-
no servo sciocco in commedia,¢ I'abito ricco di Pantalone vecchio in Commedia.
Narra parimente l'hora appunto che era, quando coftoro s'accoftarono'a Mal-
mantile, ¢ dice, che fu fa l' annortare, che € quell' ora, fa la quale i Crocchioni
si mettono nelle botteghe intorno a un caldano per paflar la veglia. In tale fla-

gione

  
 

we
o

se BE

 

TERZO CANTARE: “ag
Fiiova nel nidio, dove fori dalla chioctia covate. Vedi Bfopo Favola dell'Aqui-
: dello faggio. E i} conatum frangere de i Latini.

 

- SE gli pare firano. Se gli par duro, ¢ difficile a sofftire. Vedi sopra Cant. 2.
stan. 1 eli propdo i 'ato ¢ difrano, Stravagante, 0 foreficra,Onon

ado; valendocene in cutti questi, ed altri significati, come segue
ne i Latini della voce extraneus.

AFFRONTO, Significa Aggreffione, aflalto, abboccamento, Vedi sopra.

Cant. 1. stan. 29. ma si piglia ancora per Sopru/o, come è prefo nel presente>
Iuogo.
NRERE uno » che dispiaccia. Sopportar per forza una cola, che sia di
disgutto, che in Latino si disse: Calicem bibere; perché Calix era una specic di
bicchiere, col quale gli antichi bevevano caldo, come appunto si bevono gli sci-
roppi; ¢ lo facevano ancor' essi per medicamento; ¢ per conseguenza era tal be-
vanda, come a noi, per lo pil, di poco gulto,

WEL eMondo non è nulla ds netto, 11 Mondo non ha felicita perfetta. Yricuize
dedit vitium natura creato, e

VIVER a brace. Viver'acalo, fenzaregola, 0 considerazione. Ha forse-
we detto a dalla mifura, che si fa della brace, che per eter cosa vile, ¢

i poco prezzo si mifura inconfideratamente senza guardare a darne un poca pilt,
© un poca meno. Da questo poi habbiamo /lraciare yeduto sopra Cant, 2. stan.
10, che significa Confumare il suo inconfideratamente.:

MVT ARE feena. Mutar facia, o stato, mutar maniera di vivere. Traslato

dalle prospettive, dove si recitano le commedie, quali prospettive sono da noi
vu Igarmente chiamate Scene.

TANZA IIL STANZA IV.

Era in quei tempi la quando i Gelont Quand in terra armata con la feorta

Tornano a chinder l'offerie de' cani, Del gran Baldone a Malmantilsinuit,
Etalun, che si (paccia i millions Ond' un famiglio nel ferrar la porta
Mazda al prefta il rabi pe panni lani; Senti rumoreggiar tanta genia,
Bd era appunto l orach' i Crocchioni Vn vecchiocra quest huors di vifta corta,
Si calano all affedio de caldani; Che l'erre ogni bor perdeva all offeria,
Ed tscon con le canne,e co' i randelli Tal che trail bere, el' esser ben d' eta
Tragazzia Piglares pipisprells A Von ci vedeva pile da terza in la,

De(crive 1a flagione, che correva; quando la soldatesca sbarcd in terrae s'av-
vid verso Malmantile sotto la condotta di Baldone; € dice.che era ful finire dell'
Autunno, poiché cominciava a diacciare, ed i ricchi finti mandavano a impe-
gnare i veltiti da (tare per rifquoter quelli da inverno; coftume aflai usato da co-
loro, che sfoggiano in veflire quantungue fieno poverissimi, ¢ questi intendi
ricchi finti, che [i [pacciano i millioni, che si fucl dire: Adexzettin non rifquote Pan-
talone, ¢ s' intende, che gli abiti da fate non vagliono tanto, che impegnandoli
poslano rifquotere quei da inverno, come appunto è |' abito povero di Mezzetti-
no servo fselocco in commedia,e l'abito ricco di Pantalone vecchio in Commedia.
Narra parimente l'hora appunto che cra, quando coftoro s*accofarono a Mal-
mantile, ¢ dice, che fu fu l'annottare, che ¢ quell' ora, fa 1a quale i Crocchiogi
si m:ttono nelle botteghe miorno'2 un caldano per paffar la veglia, Ia raie ta.

prone

 

 
 

ye

TERZO CANTARBE; 129
gab pane: ete. 7 ' = 39 re notte a Terza, che è
uafi il principio « = ie si ire, che i fuffe fempre al buio,
© non vedefle punto entro il giorno. E*detto aflai vulgato per intender uno

 debole di yifla, come intende nel presente luogo. Vedi spra C. 1. stan. 9. E for-

se vuol intendere Vno di coloro, che perdono la vifta alla levata del fole, ¢ las
eee fole ya sotto.
“ SSPTANZA' V. STANZA VI.
Per questo mette mano alla scarfella 1 quali sopra il nafo a Petronciano
Ow ha più ciarpe assai a un rigattiere, Con la [ua flemma pose a cavalcioni;
Perché vi tiene infin la faverella, Tal che meglio [coperfe di lontano
Che la mattina mette ful brachiere; Esser di gente armata pitt [quadroni.

Come [uol far chi ginoca a cruscherella,

Spaurito di cio, cala pian piano,

Due Ando alla cerca intere inrere, Per non dar nella scala i pedignont;
E poi ne traffe in mezzo a due fagorti E giunto a bajo lagrima, ¢ lingrra y
Vi par a occhiali affumicati,e rotti. Gridando quanto mai n'a nella strowza,
STANZA VIL
Dicendo forte, percht ognun  intenda: Perch quaggik nel piano è la tregenda,

Al} armi all armi, uonifi a marcello
Ss lasci il ginoco,il ball, ¢ la merenda,
E ferrinfi le porte a chiaviftello,

Che ne viene alla volta del Caffelio;
E se non ci ferriamo,o facciam tefla,
Mentre balliamo vuol fuonare a fefta:

Il detto famiglio s se col metterfi gli occhiali, che era gente armata, es
questo si mefle a gridare; allt armi,

SCARSELLA, Tasca, Vedi sopra C. 2. stan. 8,

CIARPE, Intendi robe vili, firacci, bazzecole, che i Latini differo Scruta.;
ed in altro fenfo Ciarpa vedi sotto C. 5. stan. 33.

RIGATTIERE, Rivenditore d' ogni forta mafferizie, ed arnefi da i Latini
detto Propola dal Greco; ed a noi viene da rigaglic, che intendiamo robe diverse
di poco prezzo, ed avanzumi pfati. L' Autore affomiglia la ta(ca di coftui a una
bottega di Rigattjere, perché queste per lo più fon ripiene di diversi arnefi, fra i
quali ¢ caluoita difficile ritroyarvi una cosa, quand' altri la voglia,

PAVERELLA, Fave macinate, ed impaftate con acqua. E di questa si fanno
torte cotte nel forno, che si dicong ancora AZacco forse dal Gree, AMatto.Lat, pinfo,
'Tale Favereda dicono, che sia lenitivo a i dolori d' ajlentatura, ed habbia yirti
d aflodar quelle parti; ¢ però dice, che coftui /a mette in ful brachiere, che & quel-
la fasciatura, che s* applica afl' eftremica del ventre per foftenere gl' inceftini,

CRYSCHERELLA. E? giuoco da Fanciulli. Fanno in fur' una tayola yn mon-
ticello di Crusca, ¢ yi mettono dentro quelle crazie, 0 quattrini » che vogliono
giuocare, ¢ mescolando poi bene, si fanno da uno del giuoco, a cid depucato,
tanti monticelli di detta crysca, quanti sono i giuocatori, i quali ( lasciando da

quello, che ha fatto i monti, perché deve efler ' ultimo a pigliare il monti-
cello') tirano le forti a chi debba esser il primo a pigliare uno di detti monti, ¢
ciascuno nel monte, che gli è toccato va cercando de i denari, che la fortuaa,
v habbia fatti reflare. Stimo, che questo giuoco faffe nfato ancora da i Fanciul-
li Latini, perché si trova Ludere furfure, Ed a questa ricerca, che fanno i ra-
gazzi del denaro aflomiglia quello, che  il famiglio per trovare gli oc-
chiali PA.

 
 

 
  

   

  
 

 

    
 
 

 
 
  
  
 
   

 

 
  
 

 

le m:
4"

 
 

 
     
  
      
    

  

      
   
    
    
    
   
 
   

130 MALMANTILE; >
FAGOTT/, Inuolti, o fardelli piccoli» I ncorg
PET RONCIANO, ¢, pi ae ee

forse specie di Mandragora; ¢

simile alla Zucchetta, ¢ fla appicca }

ghianda, alla quale s' affo a figuea;

si appella Atarignano, A quelto Perroncians s' allomiglia ¢

ti un nafo di firaordinaria grofiezza, ¢ di colore roffo livido.,

s' intenda', che havefle quelto famiglio. wend
e4C AVALCIONT, Vuol dire una gamba da una p:

come si sta in ful cavallo, e come,stanno gli occhiali &

da una parte, ¢' altro dall' altra.. ee a
PLAN piano. Cioè adagio adagio, bel bello: Adaaie. »VOCe p

aggiunta al verbo fare, ed al verbo andare significa quel, che hel prefen te Ju

cioè Adagio, e con diligenza,, che i Latini dicono placide incedere; ed aggiu
verbo parlare signitica parlar con voce bafla, Submifva voce.
PEDIGNONT, Specie d' infermita, che viene,ne i piedi, ¢ nell

troppo freddo dai Latinidetti Perwiones, A pecan svelte
S/GNOZZ ARE,0 fingozzare, o finghiozzare. E' un moto del

fuer (0, 0 mediattino, cagionato da foverchia -yotezza,o ripienezza;

litudine significa anche sospirare vehementemente con pianto, come

prelente luogo, | Latini ancora s ne servivano nel primo significato, ¢

condo; Singultus-, & fingultire, © fingultibus ingemere.: etee
GRIDA quanto mai n' ha neda Hrozzas Grida quanto pnd pilseg

refifter la gola. Che froxza vuol dire La canna della gola, altrin

Gorgoxzule. 1 Latini pure di¢evano in gusture exclamare, £ da questa

vicne strozzare, che yyol dire Strangolare sh

Dinte Inf. C. 7. Quef? inno si gorgoglia nella frozza, |

gE, C28, Con Ia lingua tazliaca nella ferozza
SVONISI a martello. Si fuonino le campane a rintocchi, che si

corr' homo.:
TREGENDA, Moltitudine, ¢ quantita di gente, Dalle perfo

crede, che vadano fuori la notte anime dannate, ed altri (piriti per

gente, ¢ quelte chiamano la Tregenda. Tal' opinione se bene ei Dp

plici, ¢ idiote, nondimeno pare che venga seguitata da S. Agoftino

lib. 4. de Civit, Dei dice. Lamia dicuntur anima hominum depravata,

te meritis macufofa, qua a corpore feparate terriculamenta funt mortalibs

fente lu go è intesa per moltitudine di gente.: )

SVONARE.. li verbo fuonare si piglia taluolta in vece del verbo |

e¢ però ne nate l'equivoco del fuonare mentre colora ballano y che vuol,; K
tergli, se ben pare, che voglia dire fuonare alloro ballo:, Ed in cid'

Latini, che hanno il verbo py//are, che vuol dir perquotere y & 4
fuonare ogni forta di frumento musicale, ¢ le campano; ed il
pelaror..

 
 

   

 

  
TERZO CANT ARE: 131

f ava <o STANZA! IX.

Tn qm A Tra quespi paittl acer a fore afsai,

© EB che ne cup ogh 'baloce “Ole d Marthefi, Principi, e Siznori;

Cad omparita 2S Fyamin' di Conto,e eP offi botregai,

a 1s eee mooyct "Wana Siralbe, ¢ Battilori 5

, Quivi le una progenie ardita Lanaiiiol, Ovefici', ¢ Merciai

' a ye erstiey phe ee ti Norai, Legisti, Medici, e Deter 5

“OES ne'viene all'erta lemme lemme “In somma quivi fon gente, ¢ brigate
Col Batthil Toffie tutto Biliemme. D' ogni forva; chiedere, ¢ domandare,
be bd fiuddetto vecchio andava ptidandé, ¢ che non oftante quelto, colo-

fo, erano in 1 if€seguicavano: a darsi bel cempo, l'armata arrivd

Panera: ¥

 

a
z
=
3

   
    

soe

=

Bese

thufa; 1} Poeea'natra la' qualica di quefti soldati.

'A Viol dir adnate, o'cantata, Bocce. Nov. 97. Cox una sua viola
fronr ia Pampire  Vatchi itor. lib. 10, ALdarespa ando in persona fopi a il baftio-
ne di S, Miniato con tutti li [uoi fuanatori, e dopo piit lunghe Rrombertate, ¢ flampite,
ec. Ma gui iatende romore:, € cicalamento odiofo, che ¢ il fenfo, nel quale oggi
per lo pire tela da'noi gueft# parola, ed ha Jo steffo significato che bordeiio,
d Paar Le Binill red'p ¢iictaforicamente, il' che vedremo altrove.
ip ALOCCARST:"Prettullar®, Perder”il tempo', € trartenerl in cole di poco
momento, 0 traftalli da ragazzi, dé i quali è proprio il verbo baleccarft, 0 balocco;
¢ forse & fincopato daf verbo Baddlaceare, ¢ Badahicta; Vedi foto C. 6, Nan. 32,
Spr ewncOEMEee.» Dictatho Wich Bicacea”™” Vareli MOF. 1B: 15. 214 fureno pore
tare' le'tbiavi di non fo! che Biticea'; Vuol dir forcezza piccola', ¢ di poca considera-
zione posta in luogo eminente, come appunto ¢ Malmantile, il quale con glie-
fla fola patold Biccieocea', il Poeta benitliio descrive; percht per Bitcicocea vol.
garmente intendiamo un Cafolaré', 0 castelluccio poito in luogo eminente, ma
da:farne poca flima'. Lasca Nov. 3. Salita che hebbe ton non poca diffiultd quell al-
pehre Montagna, credeka entrare in un bel caspello, ma riguardando all' inturno, ved-
de che era ina Biecicoece piit per vefugio di capre, che per ricetto di soldati,

'ST confide nelle fante'nocca. Ha ja sua fidanza nelle pugha. EV' epiteto unre &
meffo per esprimere if odo'de} parlare de i Battilani: Se bene ¢ usato dalla gen-
te anche'piil civile per interider perfezione come vedemmo sopra C. 2. stan, 52.
E quié benissimo posto';' perché /anttxs vuol dir determinato, © fabilito, fendo
fincopato da fancitas, ¢ le pugna sono s' armi stabilite, ¢ proprie de' Battilani,
Che' per mocea'; che foo 1 nodelli delle dita, s intende tutta la mano feriata, che

in questo pili, che in altra maniera si scorgono le nocca',

* della medesima natura, ed ha lo fiefio significato di pian
piario dette in questo C: stan. 6.; ma ¢ termine reftato née i Battilani, o feo
we @/usate da altri (ara detto lieme lieme, che viene dal Latino ieviter, 0 leve,e
' 'leggierttiente; 0 dal'Téscano Lieve, che vuol dit Leggieri.

°BATTL, e Teffi. Bauilani, che fon coloro, che cdnciano la lana , ¢ Tefi
'quelli che lateffono, 8) 8 es cerns ' $
0 TETTOBiliemme. Chiamiamo Biliemme quell' ultime contrade della Città
di Firenze, dove abita questa forta di gente, la quale veramente, benché
'nata, ed allevata in Pirenze,è affarto onary da gli altri Fiorentini - i'co-

: 2 lumi,

2.
=

as

  

  

SARA KLEEN.
dh
e
2
5

 

Digitiesenipuar

 
a

 

132 MALMANTILE

flumi, ¢ nel parlare; farebbe leggi a suo modo; mangia d' ogni forta spo
come gatti, cani, pesce, ¢ carne fetida; beve ogni (orta di vino |
mente, come afferma il nostro Poeta forto in questo C. stan. 60. dicendc
che a bere ¢ peggio delle spugne. In somma è un Popolo da se, che noi chia
gli Vasi, il Batti, 0 Biliemme, \a qual voce serve ancora per esprimere la p
plebe, come è nel presente luogo. ak
GVITT/, Guidoni, ace » fudici, sporchi, ¢ fordidi. E' parola che;
Napoletano, se bene il Varchi stor. lib, 10. € ne serve anch' egli per esprimeres
un' hvomo d' animo vile, dicendo: Eli era tanto d' animo guitto, e tanto mefebings
che usava dire: Chi non va a bortega ¢ ladro. (rape
HVOMINT di como. Huomini di stima; huomini riguardevoli.. Trans
forse dal giuoco delle Minchiate, nel “ giuoco si stimano, ed
Jamente le carte, che contano, le quali fon quelle, che vedremo sotto C. 8.f
61, Si dice 2 tale conta per intendere; il tale ¢ huomo adoperato, 0 ¢ buonoas —
walcofa. ah
a BATTILORI, Mercanti d' oro filato. Banchieri. Mercanti di cambio, che
si dicono Negozianti. Serainoli Mercanti di drappi, ¢ di eta, Lanaiuols. |
canti di pannine,¢ Lana. Orefici. Mercanti d' oro, e d' argeato fodo. Alem
ciai, Coloro, che vendono naftri, feta, telerie, ed altre merci simili,
questi fuddetti in generale si chiamano Mercanti, o mercatanti.
BRIGATE. Quantita di gente, Vedi sopra C, 1. stan, 2, 3
D ogni forta, chiedete, ¢ domandate. Cioé domandate, ed eleggete '
forta di gente volete, che la troverete fra coftoro; perché vi è d' pmete

      
   

     
     

mur te

 

 

   

mete

persone. Hs
STANZA X. STANZA XL

Sul Colle compartisce questa gente 1 nome di coftui, dice Turpin
Amoftante con tutti gli Vfriali; Fu Paride Garani, ¢ il legno prefty
Tra' quali un graffo v' ¢ conualescente, Perch' ei voleva darne un rivelling —
©' haveva prefo il di, tre serviziali; A un [uo nimico traditor France,
E appunto al corpo far' allor si fente Che per condurlo a seguitar Galina
L operazione, dar dolor beftiali 5 Lo tira pe' capelli al fuopacley — *
Tal che gridando fenz: alcun conforto E per fuggirne ai paffi lagabellay
In terra si butte come per morto « Lo bolla, marchia, ¢ tutta lo fuggella,

Ii Generale Amoftaate distribuisce ful colle di Malmantile i Soldati, fra iqua-
li era Paride Garani, che havendo prefo un gran vacuatorio sentiva duloci acer
biissimi, ¢ però G rammaricava. Il nostro Poeta per accredirare questa ope
ra, come fece il Pulci nel suo Morgante, ¢ ' Ariofto nel Furiofo, le da anche
egli il fondamento della storia; allegando |' autorita di Turpino., come fece an
che sopra C, 2, stan. 31. ¢ da quello che scrive Turpino, cava che coftui havea
nomie Paride Gatani, il quale havea pre(o il legno per dare una quantita di le
gnate a un (uo nimico Francefe, cheyper condurlo a fegnitar Calvino,s lo yoleva
tiraré pe i capelli in Francia, ¢ per risparmiarne la an  hayeva già mat
chiato, ¢ bollato, ¢ figillato. E scherzando l'Autore con. questi. equivori, wl
dite che Paride prefe il Legno fanto per medicarsi del mal Franzefe. |,

PRESE id legno, Cio bevye il decorto di Legno Santo pet medicare il Mal

t r: 'ran-

 

 
 

 

TERZO CANTARE? 133
Franzefe; se ben par che voglia dire, prefe un pezzo di legno per baftonare quel
Sa:

DARE un rivellino, Dare una quantita di legnate. Rivellino ¢ una specie di
fortificazione, che si fuol fare d' avanti alle porte delle Città, 0 fra le cortines
delle Fortezze, così detto forse perché revellitur a linea, 0 perché revellat hoftium
vim; ¢ da questa rivolta nelle cortine, 0 dal quafi fivolarh cal al nimico hab-
biamo il presente translato, che ci serve per esprimere, Rivoltarsi a uno cons
gran quantita di baft » bravate, riprenfioni, ec, E dicendofi aflol
¢ (enz' aggiunta: Gui fece un rivelline, s' intende Gli fece uma folenne bravata,o buo-
na pafsata,o gran rabbuffo; E dare un rivellino,s' intende dar quantita di percofle.

RIDVELO a seguitar Calvino. Par che voglia dire ridurlo a seguitare la fetta
di Calvino Eretico, ¢ vuol dire, che per farlo divenir calvo, questo suo mal
Francefe lo tira per i capelli, ¢ glieli fa cascare.

£0 balla, marchia, ¢ tutto lo fuggela, Fa bullette, marchia, ¢ fuggella. E vuol

dire che ee suo mal Francefe gli havea cagionato bolle, crofte, ¢ lividi; che

il verbo fuggellare vuol dire Far de i lividi nel vifo a uno con le percofie, i qua=

li noi chiamiamo Pesche: 1 Latini in questo fenfo differo; /uzgiliare. Vedi ford

C 6, stan. 54. metaforico da /uggellare che vuol dire imprimere in cera, oftia, ¢
simili nelle lettere, ec. ¢ si dice anche /igi//are Dant, Purg. C. 7.

La sua impronta quand' ella figila.
E fuggellare Dante Purg. C. 10, Come figura in cera si fuggella, E Canto 3}.

Ed io si come cera da fuggello.
STANZA XIL STANZA XIIL

Dife Amoftante, viffo il cafo forano, Gloria cerca Lion, piit che moneta,
© Noferi di casa Scaccianoce: Pero ch' ei bada al giuoco,efa progreffo;
Per Ser Lion Magin da Ravignano, Per l' acqua in Pindo andocome Poera,
Ch' sk venga a medicar, corri veloce; Ondt agl infermi da le pappe a lefso.
do dico lus, perché ce n' ¢ una mano, Gis è quel che attende a predicar dieta
Ch infilza le ricette a occhio,e croce y E farebbe a mangiar con L' intere/so;
O fa sopr' al! inferme una bottega, Ma perché gid tu n'hai pits d'uno indizio,
E pos il pitt delie volte lo ripiega. Va via, perché l'indugio piglia vivio,

Amoftante veduto lo firavagante accidente, ordind a Noferi Scaccianoce (che
vuol dir Francesco Cionacci ) che andafle per Ser Lion Magin da Ravignano
(che vuol dire Giovann' Andrea Moniglia ) ¢ facefle venire lui medesimo, che &
un valent' huomo, € non come quaicuno, che non fa dove s' habbia la testa, ed
in vece di medicare un' infermo 1! pil delle volte  ammazza con le sue spropos.
tate ricette, ed ¢ di quelli, de i quali si pud dire.

His, & si tenebras pep ant, off facta poreftas y

Extenuandi agros, bomine/que impuné necandi,
. [che non si pud dire di Lione, che procura pil d' acquiftar gloria che oro.
Egli ¢ Poeta, ¢ pero non ¢ maraviglia, se andando egli per J' acqua al fonte di
Parnalo dia poi molte pappe con  acqua'a gli ammalati. L' Autore dice così,
perché in una sua leggieri infermica non voile questo medico, che e¢gli pigliaties
amedicamento alcuno, ma lo volle curare con ia fola dicta, facendoli mangiar
fera, ¢ mattina pappe; ¢ però dice; sstende a preduwar dicta, E farebbe a man-

già

 

 

 
ee

 

 

*

134 MALMAN TILE T
iar con U interefo; perché veramente itil quel tempo Lione
sano e robufto, mangiava aflai. Questo Lione non cra sta
tore nel primo componimento de! te sia Oper: 28 suo
havendo folamente eat quel 1 ream ad aie lf
dremo pocoapprefio,ia dopo la fudderta infermita,per ven
dell nalce whet to a Tieta ce lo volle ae 'Hor tornandd
no. 1] Generale dopo haver dato a Noferi molti contrafiegni'
scefle questo medico, manda'a'cercarne. © © ssi

CE n' è una mano, Ce ne fon molti. Termine'che vien dal Latino ©
En, /unenum manus emicat ardens. aes -

INFILZ A le ricette a occhio, ecroce, Si'dice anche'a Occhio V0
ricette senza regola, considerazione, o fondamento. Opera senza f
prova, E' termine meccanico, - « eee ee ee
FAR una bortega sopra uno infermo, Far allungare il male per cavarne'®@
guadagno, E questo termine s' usa in qualfiyoglia negozio, del quale uno pra
ri di prolungar la spedizione per buscar pity denaro. nse ae
RIPIEG ARE uno. \ncendiamo Far morir uno, Vedi sotto C, 10
BADAR al ginoco, Attender con applicazione a quella profeifioné
fa, 0 a quel negozio-, che ha fra mano, e si dice anche Badare-a dott
sopra C, 1, stan, 62. quelto verbo badare in altri significati, z

PAPPA, Cioé panc bollito nell' acqua; o'in altro liquore. E' dig
le inventate dalle Balie per facilitare il parlare a i bambini, come B
ma,¢ simili. I Latini difflero, pappare, ¢ i Greci pure dicevano ?. eb
in altro fenfo, volendo-esprimere il Padre, i) Habbo, Vedi sopra C, 2, flan!
E sotto C, 4. stan, $e 12.:: bad ovale

ATTENDE 2 predicar dicta, Sempre dice che si mangi poco; che q
tende per far diera. Se bene appreffo a' Medici diera vuol dire regola
versal¢. Dieta si dice congreflo di gran personaggi per trattare n
mi, come si dice Dieta il Congreflo de i Priacipi Elettori all' Ele.

eratore.:

F AREBBE a mangiar con l'interefso, Mangerebbe fempre di giorno, ¢di
te, come fanao i-cambi, 0 usure', che mangiano di, 'notte, mentre che: a
po fa crescer la somma degl'intereti. L' usura in Ebreo dicefi mor/>', ie 'one

L LN DVGIO piglia vizio. L' indugiare, 0 trattenerff ¢ pericolofo'di cagio
quaiche danno, o far perder la congiuntura di conseguir l'intento. Adoré

   

   

   

   
  
  
 
    
 
   
 

   
  
  
  

damnum, 7%
STANZA XIV. 0 ¢ Tai
Noferi vanne, ¢ fente dir ch? egli era Bedi foglirdiste/a una gran eras
"Con un compagno, entratoin nn fattoio, Ha bell", ¢ ritto quivi il [uo se '
Ow egli ha per lanterna, efsende feras\ ~ Siche prejfo lo trovi, eid
Li orinal fitto sopra a-un [chirxatoio, ©. Dell uned Sndio gh facta ]
Noferi trova il Medico nel Fartoio da olio, che quivi era il suo studig? 3

 
 

fa l'ambaftiata.:
FATTO/O, Quella lanza, dove & la macine per infragnere l'olive 5

ma
firettoio, ed altri ordinghi per cavar l'olio dalle medesime olive « en it

tino Oles fattorinm. Ree

 

   
  
 

 
     
     
  
  
   
 
  
 
 
    

 
 

gi?

ae
SS

 

TERZO CANTARE,. 135
elfednehtetse 0.0 altea gaapaveris » ocl quale s!orina, da i Lati-.

ma csc ane » donde i Sanefi chiamano scafarda '
piers effetto usanole donne.

Beis canna di stagno, o d' altro metallo, con la

sf agi' infermi. Vedi forro C10. fan. 4. >
ames ae Sparfa una quantita di fogli. ee era per la si-
quella diltefa di fogli con le ere, 0 mercati, che alcune

y tan ei eae nelle quail per le piazze si veggono moltiii-
ee v dilegnt » leggende, ed. altri arneft confufamente..,
ra;

  
 
    

 abbiamo forse questa voce fiera dal Latino forwm, che era intclo
a dove si facevano le fire 0 mercati, 0 pure dal Latino ferie,.

l ¢ritto si Hacon facilita aggiuftaco il (uo scrittoio; che la voce bello,
terthini ale

in 0 non vuol dire, che Ormai,o di già,,¢ serve per cmfali,
¢ per denotare la franchezza in terminare una opcrazione: Si dice riccare unde
bortega., rizzare wo negorio per dar principio a un negozio..

VNTO si: Oe chiama fiudio quella stanza., nella quale uno faa Gutters?:

epee Medico haveva depurata per suo (tudio la stanza del fattoio, lo

 

 
  

IN? en au stanze sono, 0 verifimilmente.devono essere uate.
' NZA X STANZA XVI,
writ chiamato Brae Era quest' huomo.un certo Medicaftro y
y (ponde ide haver' allora altro che fare, C? al dottorato. Luo se piover fienos
nen RR 6 upa.sua commedia ing distende Eperch' ei. vi pati (pefe., e difaftro
* datizalaca. U,Confole di.Adare y E frato fempre grofjo-con Galena;
“Eche/e? opra Jua cold s* attende Egiuntola: Vofar(aife)ua' impisftro >
Pee feast qari sue scolare Onde s' it mat venifJe ds velen
> nd (persmentata ed in Sua vece Prefto vedrema; in tanto egli se [pogli,
ia mandato lis; ¢ casi fece. E fiami dato aduenties efogli.

'Scntendo Lione d' efler chiamato a medicare » tisponde, che per allora nons
;wenire, mache. mandcra un (uo (colare valent' haomo, Coftui cra un gran
— O.giuata doye eral infermo., comincid subito con. gli (propositi.
COM OLE di mare, Questa tu una Commedia intitolata La Serie nobile, nel-
Jaqualc ¢ introdotto per |' Broe.un Confoic di Mare in Pifa,onde molti la chia-
mano il Con/ole di mare, ancor che il titolo stampato in fronte,di cla sia,La Ser-
na nobiles ¢ tt composta dal medesimo Lione,, ¢ recitata. in) mulica: con grandi
Apparati d' ordine del Serenifsimo Principe Cardinal, Gio; Carlo nel. suo belliti~
mo Teatro fabbricato allora.di nuovo, Ed il nostro Poeta nella presente ottava
vuol moftrare la poca applicazione, che Lione haveva in quei tempi alla medi-
cina,, come giovane, s¢ bea per altro dotto; ¢ che poi voltacofia tale studio ha
saputo acquiflarsi la fama, che ha acquiftato, € meritare una delle prime Catee-
dre deilo itudio di Bila, ¢ di feryire attualmente'al Serenifsimo Gran Duca per

CUEDICASTRO. Medico di poca scienza,.0.s come diremo:) faluatico,

SE piover fiera nel suo dottoraro,, Quando si fente uno » che vaole spacciach per
huomo dotto, ¢ dal parlare si fa conoscer per. uno igaorante, si fuol dire quaa-
do ci parla Tirate già del fieno intendendoG: Per darg.a queito bus che pacia.. Si

whe

 

 

 
 

 

~ Galeno, ¢ non fapeva quel che egli dicelfe, fiche in fuftanza vuol dir un

   
   

136 MALMANTILE

che dicendo che nel addottorarsi coftui, piovve fiena,intende che coftui si
to per un folennissimo bue; e però venne gran quantita di fieno fenz'
sto, che diciamo; La roba ci piove per intendere vien roba in abbon

chiederla.;.
E' STATO fempre groffe con Galeno; Esser groffo con uno vuol dire

collera, o efler adirato con uno; Si che dicendo, che coftui ¢ Pato fempr
can Galene, perché l'haveva difaftrato,¢ fatto penare, s" intendeera:
feco; ¢ però non lo guardava mai, ¢ conseguentemente non havea p

  
      
       
   
     

dissimo ignorante nella Medicina. 2

VELENO. Questa parola ha due significati: uno proprio che & toffico,
tro improprio, che ¢ fetore. IL primo ¢ quello, che s' intende nel presente
go, il fecondo si vedra nell' Ottava seguente. ae
STANZA XVIL eae
Confermata pero la sua credenza '
Rivolto at eeapagl oe a dire bed |

  
 
  
  

  

Mentre ¢ spogliato, per ta peftilenza,
Ch' egli efala, si vede ognun fuggire,

Pernenne una zaffata a Sua Eccellenza, uefto ¢ veleno,e ben di quel profonde,
Che fu per farlo quafi che fuenire; Ses voi ch' egli avvelena it

Mentre che Paride si spogliava ognuno per lo gran fetore comincid a i
onde il sig.\ Medico, che fente ancor' egli l'orrendo fetore, si conferméd nel cre
dere, che fufle veleno, percht avvelenava. ie

PESTILENZA, Intendi fetore grandiftimo. E si serve della parola pe
zy per la parola veleno prefa in significato di pyzzo, o fetore, ¢ per altro Peft:
lenza vuol dire mal contagiofo,: ? -

Z AFF-AT A. Parte del vapore di quel puzzo, portato dal moto dell! arias:
E fidice anche 7afaea d! ogni liquore per intendere /prxzzagiia d' ogni liquore «
Franco. Sace, num, 136, L'orina gli ando ful Cappuccia,e nel vifo,ed alcune rafate io
bocca, coe
4S, Ecc. Questo titolo benché non sia così conueniente a' Medici,nondimeno
 usato dalla nostra plebe in vece dell' Eccellentissimo, el' Autore lo daa
medico per derifione. f

PROFONDO., Per traslato significa Grandemente, fmoderato, 0 perfettifi-
mo, come usavano anche i Latini. F

AVVELENA. Rende puzzolente. Ecco la voce veleno, ed err
fa nel fecondo fenfo detto di sopra di paxzo, oferore; El" equivoco, che
cid ne nasce, serve a questo Medico per farsi stimar dotto moftrando conolcere
che questo ¢ veramente veleno,perché egli avvelena, che vuol dire far uutire,ed!

lo piglia in significato d'attofficare,c Veleno in significato di toflico, Vedi forto in

questo C, stan. 54. la voce lezzo. t
“STANZA XVUL 5
Rispose il general, commoffo 4 Sdegno + A cio foggiunfe il Medico: Buon fegney
Come veleno ? 0 corpo di mia vita | Segno che la natura inuigorita
Edoveeil Mefroaahauedtre ingegno? A' morbi repugnante, adeffo queste
Lovedrebbeil miabuesch'egl bal'nscica, ef nostri nafi manda si moleflo.

Ui Generale s' adira, ¢ dice; Che non hayete odorato da sentir questo puzZ0»

 

   
 

TERZO CANTARE, 37

—- conolcere, che egli ha l'ulcita! Alche.replica il Medico: Que-
foe fegno, perché la natura havendo prefo vigore, come quella, che re-

pugna ai morbi, espelle ora questo morbo, ¢ lo manda ai nostri nafi. Per in-
- Render fito, fons direa questo Medico, è:neceflario fapere, che
“lay significati, il primo ¢.iofermita, ¢-dicendo repygnante aii

morbi intende all! infermita; ed il fecondo è fetore 0. puzz0;-¢ dicendo manda a'
nostri nafi queffo morbo intende Manda questo fetore, Ea il buon medico, che sti-
mas che natura morbo repugnans. voglia dire repugni al puzzo, cava la conseguen-
za, che il sentir questo puzzo sia buon fegno, perché la. natura scacciando il puz-
20, dal corpo dell' infermo, lo manda a i nafi de' circoftanti, ¢ così va scemando
il morbo al iente..

 £0. vedr mio bye. Lo vedrebbe uno, che non haveffe punto di giudizio.

YSCIT A, Stemperamento di Corpo, Soccorrenza; da' Latiai con voce Greca
detta Diarrhoea.

SVON fegna. L' Autore.moftra in questa Ottava il modo, col quale foglion,
parla i Medici ignoranti per accreditarsi apprefio agl' idioti, dando ragioni
(propositate, ¢ inducendo ott improprj; pur che lufinghino il pazziente con
una eerta apparenza-di sperar bene, come fanno gli Zingani, ¢ i Montamban-

chi.
wot iaDaceamy § eeoS DANZA XIX.

Vedendo poi y chtil flufso raccappella Chiamagli aspati, cbinfermieri appella,
(Come quelle ctha in zucca poco fale ) Ui Cerafico chiede, ¢ lo Speriale,
Comincia a gridar:Guardia,lapadella; E veuuto Linchioftra, al fin si mette

BE ( quafi fufse quivi.nno spedale ) A feriver una rifma di ricette.
L leatiflimo Medico vedendo, che il corpo faceva nuova operazione,co-

mincida chiamarla Guardia, che portaffe la padella, pensando che quelle pa-
role havefiero virtii di fermare il flutfo, havendole sentite dire negli Spedali in,
occasioni simili,¢ però credendo esser ne/lo Spedale chiamava gli Aftanti, ec. ¢
poi si meffe a feriver una gran ricetta.

RACCAPPELLA, Opera di nuovo. Reitera, Replica. Raccappellare si di-
ce quando coloro, che stringonol' olive per cavarne I lio, o le vinacce per ca-
varng il vino, dopo haver dato qualche siretta, allentano lo strettoio, ¢ nelle>
gabbie mettono nuove olive, o nuova vinaccia sopr' all' altra, che v' era prima.
Alcuni dicono rincoppedlare, tracndolo dalle coppelle de' purgatori d' oro, nelies
quali rimetcono pits volte lo steffo metallo per rathaarlo,il che dicono rincoppeliare.

HAVER poco fale in xucca, Haver poco cervello, poco giudizio. Boce.n.2,
Be 4: Per porre la fun belezza innanzi ad ogn' altra, si come quella che haveva poco fa
4¢ in xucea,, Vedi sopra C, 1, stan. 73. ¢ forto C. 4. tan. 15.

GV-ARDLIA, la padella, Questo ¢ un detto, che s' ula, quando si fente, che
altri facciaromore per di sorto per causa dell' u/cita del vento, e si dice così, per-

gl infermi, che sono negli (pedali, quand' hanno bifogno di vorare il.ven-
tre, chiamano colui > che è di guardia, che porti la padel/a., che € un valo di ra~
me, ¢¢, il quale ¢ adattato in maniera-da poterfi mettere,in cao di bifogno,nel
0 sotto all' infermo, acciò che posla fare il facto, suo.; tenza muoversi dal
etto. 3 Bibi 2% oo ui wha

7 ee

s “sTan-

 

 

 
 
   
        

138 MALMANTILE

ST ANT?, 0 Afanti, Son coloro, che affifiono al servizio deg!'
2 me vedemmo sopra C. 1, stan. 48. Lat. «d/antes. i628

INFERMIERE, Chiamano negii spedali Znfermiere colui. 5
che gl? infermi fieno mefhi a ietto, quando fon condotti allo spe p
nota per fargli vifitare dal Medico, ¢ gli regiftra al libro degli entrati, ¢
usciti, ed al libro de' morti. (andi 04

CERVSICO. Quello che medica le ferite, piaghe, ed altri ma!
richieggonc opera manuale, ¢ cava langue, ec, detto ancora con voce!
usata da' Latini Chirurgo.

LISMA, ori/ma, Diciamo un fagotto, o balletta di carta, che
a 500, fogli DalGr.arithmos, Qui però è detto iperbolico, ¢ per mofir
quello Medico (crivetie afiai,non che veramente confumaffe una Liima di

TANZA Xx STANZA XX
Dove diceva ( dopo millioni Peré prefto boliir farere a fodo
Ds feropoli, ai drammeye libbre tante) Voit agnello,o caprette in um pi

  
    
   
    
   
     
     
      
 
 

 
 
   

   
 

Che già, che questo mal par che cagiont QV un' altro vafo nelle ste/so
Stemperamento forte, umor piccante y Vn lupo per infin che sia disfe
Per temperarlo; Recipe in bocconi Poi fare un servizial col prii
Colla, gomma, mel,chiara,e diagrante, E col fecondo un' altro ne sia fai

Quindici libbre in una volta fola Farò quefea ricerca operazsone

 
 
 
  

  

  
 
 
 
   

 

Di fangue se gti tragga dalla gola; Senx' alcun dubia, ed eccola rag
STANZA XXL
Accio che tiri per canal diverso Questi animali efsendo per nat
L'umor che tende al cétro, ut One grave Limici, come i tadré
Che se duraffe troppo.a far ral verso Ritrovandofi quivi per

Dir potreble dinfermo: Addio fave. 1 lupo correra dietro all'

Pot tengafi due di capo riverso Lagnelio, che del lupo baurd

Legato per i piedi a unatrave 5 Ritirandofi andra per ibd

Se questo non facefse giovamento, Così va in fu la roba y &.

'Compote gli faremout argomento, E i due contrarj fan, cht. zodd,

To queste sue ricette moftra |' Eccellentissimo Medico la sua g ine COs
proporre farmachi, ¢ rimedj spropositati, come € quello de i due brodidi lupo, —
ed' agnello, ¢ quello del tenere il pazzicate appiccato al palco per i a

. ca

    

   
  
     
       
   

    

'capo all' ingid, b Ph Sayy sky
eu/LLIONE, E! un numero determinato di dieci centinaia di migl
è prefo per indeterminato; come fuccede speflo, che per esprimer, u
quantita di cose., si dice E' un millione delle tali cose, ancor che fieno mol
no, ed aile volte molte pil. Così i Latini in questo fenfo fexcenra;
4 Greci myria., cio' diecimila. ou), isbn aN
STEMPERAMENT O forte. Stemperare vuol:dic Ammollire;10
nel ventre di coflui cra follevamento d' umori, ¢ stemperamentodi
ti, clot acide, ¢ diumori piccanti.. Gli epireti di forte, ¢) piccante fon'
conyenienti al yino,dicendofi vino forte quello, che comincia a diventare ace
ved in molti lugg hi d' Italia si dice Vin forte,il vino:gagliardo, o grande
Bectl: mee

 

   
  
  

    
 

 

  
 

piccante quello che in beverlo fa friazare le Jabra, ¢ la lingua. Questo Bé

  

  
 

TERZO CANTARE: 539
ti?  - lentissimo Medico /però intende quel forte per acido, ¢ per grande, ¢ gagliardo;
v4 E piccante dal ieee % che vual EE Duguere % Offendere che si dice anche
ph dar nel nafo » Vedi forto C; 7, stan, 59. I Eccellentiflino cava l'argumento, che
ee i umori fieno'piccanti, perché danno nel nafo col loro fetore; Ora per ra(-
Oe] fodare, ¢ coagulare'tal flemperamento vuole il prelibato Medico, che si dia al d
pazziente a bere gran quantita di col/a, miele, gomnia, chiara d wow, ¢ diagran-
tit te, le cole nella ens antita, che egli pone ses incorporaffero, in

ase

Ge grandifiima® quantita: d'.a ¢ fsrésbous atte a coagulate, ¢ feccare un Iago;

¢ (e vi havefle aggiunto gefio, ¢ matton pefto-haurebbe dato una ricerta da ilop-
ait pare quante'rorture si poslano mai troyare ne i vivai.
rat \ DIAGRANTE, Specie di gomma, 0 colla, che serve per incollare i drappi
aa ne i'rovesci de i-ricami', 0 per altee cose simili.
1. SE li-tragga 15. libre di fangue'per la gola, BE cavandofi 15. libbre di fangues
dalla vena della gola del pazziente; ¢ legandolo per i piedi al palco.col capo

Po all" ingid (che questo vuol dir caporiverso ) preteude il Medico y che ia roba sia,
ya per mutar viaggio, se vorra condurfi al suo centro, che non & più nel Inogo,
fen dove era prima, ma stante la positura del corpo è diventato suo centro il capo,
a CONTINOVASSE afar tal verso, Continovafic a fare nella medesima forma,

'se o maniera, Vedi sotto C. 7. stan..1.

AD DIO fave, Significa Noi fiamo (pacciati; Noi fiam finiti; Siam morti, Fa
et un Villano ne! contado d' Imola d' ingegao pi) tofto groffo che no, il quale ha-
i veva un bellissimo campo di fave, ¢ nel mezzo di esso era un gran ciriegio carico
o di ciriege. A tal Ciriegio haveva il villano fatta una fortidima prunata, perché
is pops soe gli fuflero colte; ¢ vantandofi di questa sua diligenza, fu sentito
. da un Cieco suo amico, il quale glidiffe: Con tutti li tuoi pruni io vi falird, e:
sf se non lo faccio, voglio perdere dodici lire, ch' io mi ritrovo + ed il villano re-
de
eh

 

plicd: Setu non pigli la scala, o vero non porti il forcone, 0 altro per levare

1 pruni io voglio giuocare questo campo di faye, e che tu non vi (ali. il Cieco

si contentd; ¢ così.conuennero. L' aftuto Cieco si coperfe tutta la vita con buone

ie pellidi bue, ¢ così armato paffando per mezzo de i pruni senza sentir puntura,
y alcuna, fali sopra il ciriegio. Li villano, veduto quetto, tardi accortofi della sua
g* ee » piangendo il suo danno gridava: Addio fave, cioé io ho perduto le

uoh fave, Vedi il Cornazzano Novella 10. dove troverai questa fayola non travetti.
i ta, e meglio espreffa.

TRAVE. Legno groflo,¢ lungo, che s' adatta a reggere i palchi. z
inet ARGOMENTO., E'lo stesso, che Serviziale, 0 Crificro detto sopra in questo

lt C. stan. 10. ¢ 12. E.quitorna bene, perché vuol medicarjo per via d' argumenti
si ——Jogici yma di canfeguenze sproposirate.
ao BOLLIRKE a fodo, Ciok bollire molto tempo, ¢ gagliardamente,

BRODO.. Decorto di carne. Acqua ingraflata con carne.. Se ben la parola
PA brodo è comune a ogni forta di decotto, o mineftra, aucorché non di carne.
ie 1 DVE contrar} fan che it terzo goda, Inter duos stigantes tertius gaudet, Con que-
oe flo argumento, ¢ con queta fentenza, ¢ con altre ragioni da fquartati, pretende
cA F Eccellentifiimo d' haver trovato il:modo di fermare i Hluilo.;

S2 STAN-
 

 

140 MALMANTILE

    
 

STANZA XXIV; » STANZA XKV,

Cio detto rivolteffi al mormorio In quel che questo t

Di quell' ambrette, ov' a meftar si pose; We dice ogni or delt'

E, perch' elle fapevan di stantw, TofelloGrani, ilquale tu

Teneva al nafo un mazzolin di rofe. Mofo a pierd, con una fun

Soggiunfe poi: Coftui vuol dirci addio, Tagliace havea aun

Che quefie flemme putride, ¢ viscofe Sopr' alte quali a foggia di

Moftran, che ben' affetto agli artolani Fu Paride da certi Conradini

     
  
  
   

Ei vnol' ire a ingraffare i Petronciani, Portato a' suci poder quivt vicini.

L' Eccellentissimo Dottore, dopo haver fatte le fuddette belle ordinazioni
mette'a Muzzicare quella materia,¢ da quel puzzo fa pronoftico, che il
te sia per morire; ¢ ' argumento, che egli fa-di cal morte non ¢€ didimile.
ricette. In canto Tofeilo Gianni accomodé una barella, sopr' alia qual
fu posto, € portato da certi contadini ad una villetca de' Signori —
Malmantile in Inogo detto Santo Romolo; nella qual Villa trov:
concepi nella mente il far la presente Opera, come dicemmo sopra ne)

wIMBRETT- A. Così chiamiamo guanti, ed altre pelli conciate con
d' ambra.. Ma qui intende, ironicamente parlando, quella materia fetida,

SAPEVA di stantio, Haveva cattivo odore. Quando una materia per
ghezza del tempo ha cominciato a perdere la sua perfezione,si dice /antia; che
se sia carne, 0 pesce, non da troppo buono edore; e quelte si dice payee
tio, La qual voce viene da stanziare lungo tempo, ed ¢ il Latino

 

sotto in questo C, stan. 54. sith,
VVOL dirci addio, Sc ne vuol' andare. Ci yuo) lasciare, cioé prire.
FLEMMe4. Vmor freddo, ¢ umido che i Medici chiamano in

munemente si dice hemma dal Greco, reid

VVOL! andare 4 ingrafare i Petronciani, Vuol andare a ingraffare gli orti col
suo corpo, facendoli forterrare; ¢ piglia Perronciani (che vedemmo it
tio.C, fan, 6, quello che fieno ) per tutto lorto. E nota che per care la
castroneria di questo Medico, ' Autore gli fa dedurre il. pronoftico della morte
di Paride dal credere, che il suo corpo sia già corrotto, ¢ ridortofi tutto in quel
la terza putrida fuftanza, ed in conseguenza.atto, ed il'calo.a it i
ni; E vuol dire, che Paride morra: Digendofi vulgarmente per intender que
sto U tale ando a ingraffare i cavoli, cio' il tale mori. ¥ oiioen at

CAPO a affiuele, A-uno ignorante si dice Capo di Bue, Capo di Casteones
Capo d' ativolo,¢ simili, ZL' afixolo è un' uccello in tutto simile alla' Civetta, se
non che ha sopra il.capo, alcune penne ritte, che fembrano cornas ©* atobigt

TOSELLO Gianni, Agoftino Nelli Gentil' huomo Fiorentino buon:
¢ veramente huomo da bene, Che intendiamo baor figlinale: » SANA GE

COLTELLA. Specie di scimitarra, Arme ches! usaiportare 4 va

    

a caccia.. 3 0
BARELLA, Aruefe fatto di tavole » che ha quattro manichi 5 serve pet por
tar faili, ¢ aleri pefi in due persone; qui intende una barelia da porearesane
d huomiai inferm), 0 morti,, che è Gimile alle bares o-catalettico i quali si 10
glion postare detti corpi, ¢ da Bara ¢ chiamata baredla.. Vedi oan”
dane 54. SEAN

 

 

li ee a
 

TERZO CANTARE. 14t
to STANZA XXVI.
Fu del Garani ascritto fucceffare Dicon ch' ei nacque al tempo delle more,
 Puccio Lamoni anch'ei grad' ingegnere, Per ch'eglié di pel brunoye membra neve;
Bravissime Guerrier faggio Datrore, Hor qua di Cartagena eletto Duce

Cortigiano, ante, ¢-Tanerniere, i fior de' Adammaganuccoli conduce.
Al Garani fu dato\ per fucceffore Puccio Lamoni, il quale &, Paolo. Minucci.
Il Poeta dice che coftui era ingeenere, ¢ Adercante;.ma tali attributi gli (no fin-
ti, perché io, poslo giurare., che egli non fa ne dell' una, ne dell' altra profettio-
ne. Loichiama guerriero, ¢ questo perché detto Puccio fece una campagna.
neil' efercito Pollacco in Pruifia,seguitando quella Real Corte, alla quale era
fiato inuiato dal Sereniflimo Principe Mattias di Toscana alla Maefta del Re Gio:
Cafimiro. B perché detto Puccio godé per melti anni, ¢ fino che S, A, visse,
l'honore di servire. all' A, S. in qualita di Segretario, però dice che era Corti-
giano. Dice che ¢ Dostere perché veramente egli ¢ addottorato in Legge, sc be-
ne per l'applicazione alla corte.5.non clercito tale profeifione. Lo chiama Ta;
verniere., perché speflo lo vedeva entrare nell' Olteric, ¢ trattare con Ofti, il che
seguiva perché egli vendeva loro del vino raccolto nei fui beni, € gli conucni-
va lasciarsi rivedere speflo per risquoterne il prezzo. Dice che si vocifera, che
gli nascefse al tempo delle more, Perch' egli è di pel bruno, e membra nere, eficndo
li così in effetto: E facendolo Duca di Cartagena dice, che egli conduce if
ore de' Mammagnuccoli,cioè i migliori, ¢ più valorofi Mammagnuccoli, Questi
M gnuceoli ¢rano una fazione di galant' huomiai, i quali f
profeifione di fapere il conto loro in ogni cosa, ¢ particolarmente nel giuocare,
© pendere bene ildor-danaro,.¢ d' cflere il fiore della reale, ed onorata
apialanite + Havevano ua loro capo, che si chiamava Abate, dal quale erano
galtigati 5 quando facevano qualche crrore o nel giuocare,, 0 nello spendere, ma
però tutto ¢ra in galanteria. Le loro adénanze si facevano in cala l'Abate, do-
ve si giuocava a giuochi più di (paflo., che di vizio, ¢ si facevano altre allegrie,
dicene, merende:s ed altrispaflatempi. Coftoro crano tutte persone serie, es
quiste se:della pidriguardevole Civilta, ¢ percid era la lor conversazione molto
bramata,, onde cra-pumerolissima; Se bene non era ammefio a quella veruno, che
non haveile provata prima la sua dabbenaggine, ¢ non fuffe stato riconosciuto
dal Abate, ¢ da altri (uoi Consiglicri -meritevole d' eflere ammeflo. Fra coftoro
era detto Puccio, ¢ perché egli era forse de' pili affezionati, i1 Poeta.lo fa loro
Condotticro., ¢ per la stima che faceva di lui nel giuoco delle Minchiate, era fo-
lito chiamarlo il Re:delle carte; percid lo fa Duca di Cartagena, ed ¢ ancora ap-
propriato, perché detto Puccio per esser di faccia bruna, ha qualche fembianza,
ed ariadi Spagnuolo; oltre che nel tempo, che l'Autore Jo aggiunfe a questa sua
Opera, il detto Puccio, era flato destinato dalla Maefta del Re Gio; Cafimiro
per (uo Segretario dell' Amba(ciata di Spagna.
STANZA XXVIL

 

L' Armatahaveatragli altri unCappellano Faceva da Pittor, da Tiziano;
Dottormailfnofaper fu buccia a Maquat'ei fece main'adava agrnccia y
'Pero ch' egli Pudio col fafeo in mano, Hebbeuna Chicfa, e quiviabiscaaperta
Ed era più bafon.a' una Bertuccia » Si ginoce fnoi soldi dell' hee ©

Pee

 
 

 

142 MALMANTILE: &
STANZA XXVITL

 
   
 
    
  

Franconia si domanda Ingannavini, Lelie havea in casa it
E fu a come il pie valente, Gid fatta una lerione,e falla a
Perch' eghs fapea leggere i Latini, Subito accetta y € fiede in alto
A far quattro parole a quella gente', Senta metterui fu ne fal',

Fra gli altri Cappellani, che erano nell' Armata, era un Dottore y ma dig
scienza; perché il suo studiare era stato il darsi bel tempo. Fu feolare d
re nella pittura, ma impard poco, ¢ se bene fipprefumeva diaper'
fece mai cosa, che non fufle stroppiata. Fu Rettore della Chiefa di Petriolo;
Villaggio vicino a Firenze circa due miglia,¢ perché egli era huomo allegto,
di conversazione, dice che egli ff gidscd fino i soldi dell? Sa} ed intende che co
fomava tutte le sue entrate in allegrie. I suo nome cra Franconio th
cioè Giovannantonio Francini, A questo dunque » come al più dotto fu fatta'
stanza, che facefle un poco di discorso a quei Soldati, ed'egli che “haveva' ums
tempo fa recitata una lezione nell' Accademia del Coltellini 5 ¢ 1 a ace

¥ j Wil

a memoria, si content® di fare quanto gli era stato imposto, ¢ senza
tempo in mezzo montd in pulpito.

BYCCTA buccia, Leggicrmente. Cicé sapeya poco; non haveva gran fonda
mento; che si dice anche s# pelle in pelle. Vedi foro C, 8, flan.58. edi
differo superficie renus. we ri

PUL buffone ad una bertuecia. Huomo arguto yallegro, © facetoyBaffune die
ciamo colui, che tiene il popolo allegramente con facezie, e moti, &
Scurra, Vedi forto C, 11. stan. 42. E Sertuccia diciamo la scimmia,

TIZIANO. Pittore celeberrimo. Econ dire facea da Tiziano; intende pet
antonomafia, che egli si prefumeva d' esser il pi valente Pittore del Mondo.

QVANT ' ¢i facea,n' andava a gruccia, Tutto quel che egli faceva rap
piato, cioé mal fatto, mal dipinto, Vedi sotto C. 11. stan. 41. mete

BISC-A. Luogo pubblico, dove & permefio giuocare a ognuno; Egiwecsre#
bisca aperta, vuol dire Giuocar fempre, ¢ senza riguardo alcuno. re

JL Coltellini, Quetto & il Signor Agofino Coltellini Avvocato Fiorentino huo-
mo dotto, ed amatore de i Letterati, il quale in molte opere com, da tui si
chiama col nome anagrammatico Oftilio Contalgeni. In casa di eflo @raguilas
l'Accademia degli Apatifti da esso fondata, nella quale si fannoidifeorfi Acad
mici,ed altri efercizzj virtuofi: Mirabile per haver saputo far durare per lo (pa
zio di cinquanta, € più anni la detta Accademia, (empre in florido, cola inl
lita a' nostri secoli in quelta Città. Lntcrueniva spefio in detta Accademia quell
Francini, ed alle volte vi faceva qualche lezione; nelle qualimofro i suol
ed eruditi talenti 5 ¢ f¢ bene l'Autore dice che il (uo fapere fu buecia it, OM
to lo-chiama huomo (caza forndamento, non è però, che egli fulle tale 5
gli huomini de' nostri tempi non era dei fecondi in dottrina non meno '
che profana; ed era veramente Dottore di legge. aim

SENZA metterui fu ne fal,ne olio, Prefto, iubito, senza replicare, «o' mettet
dificulta, Nulla interposira mora. Fu un tale, che tornato la feraa cafa'y difeal

suo servitore: Fammt una infalata, ¢ fa prefto, ch' io forv aspertato, ¢ noms
yoglio mangiare altro che quella; fa preito. dico. Ll servitore eee

 

 
 

i
ai

 

TERZO CANTARE. 143
senza condire la portd in tavola al padrone'; il quale cid vifto lo fgridd; Ma il
servitore rispose; Signore per servirui prefto, non vi ho meffo fu ne fale, ne olio,
E da questa goffaggine del servitore viene il prefeate detto, che significa Fare una

 

 

cosa subito 5 ¢ senza considerazione.
+ neue toh Seca Se TAAN ZA.XEIX,
Sale in Bigoncia com due torce a vento, Che ben si (corse in Ini quel fondamito,
eicio lo wegga ognun pro tribunali, Che diede alla [ua casa Giorgio Seali,
Ove, moftrar volendo il [uo talento, E piacque si, che tutti di concordia
Fece un discorso, ¢ fece cose tali, Si meffero a gridar: mifericordia,

Il Poeta continuando, a voler moftrare, che Franconio fufle di poco valore,
¢ che però il discorso da Jui fatto futle scimunito., ¢ senza alcun fondamento, lo
burla, ¢ dice che piacque tanto, che il popolo, si meffe a gridar mifericurdia; del
qual termine ci serviamo per moftrare, che qualche cosa ci sia venura a faftidio,
come per efempio. Ei duré tanto a discorrer, che mifericordia, Disse tante feiocche-
rie, che mifericordia, Ob mifericordia,, quanto volete voi durare? Quali dica, hab-
biate mifericordia, ¢ compaffione di noi, ¢ non ci tediare pili,

BIGONCLA. Eun valo di legno, del quale si servono 1 Contadini in tempo
di vendemmia per pigiarui dentro ' uva, prima di metterla nel tino, ¢ ce ne fer-
viamo anche in alere occorrenze, come di portar' acque, ¢ simili,

» Hi Bini nel Capitolo del Pilo dice;
Viua dir, che se ben' ella il più mi deffe,
ind Ed opraffi,( non ch' altro) una bigoncia,
srobe; Ognun direbbe, che ben fatto haveffe.
eo Epperche fo. vafo demo Bigoncia ¢ molto simile a una cattedra tonda,però
da moiti tai Cactedra: si chiama bigoncia, come anche tutte l'alere cattedre. Il
Davanzati ne} suo Cornelio. Tacito postilie al 2. libro num. 18. dice: Arringa-
vans i nostri antichi al popoloin piacza, in ringhtera, e nei Consigls in bigoncia, che
era un pergamo in terra a fogcia di bigomia.

TORCE a vento. Torce grofie che si fanno di funi di cotone filato attorte per
servirfene a far lume la notte per le Arade; ¢ si dicono 4 vento, perché refiftono
alivento;¢ a-distinzione di quelle, che si fanno a Venezia, che per esser gentili
si spengono a ogni poco di vento. E Torcia, che da i Latini ¢ detta fusalia, fu-
natinm, viene'a noidal Francefe Terche

CHE diede alla [ua casa Giorgw Scali., Giorgio Scali fu in Firenze an riputatif-
fimo Cittadino Popolano, it quale nelle diflenzioni, che seguirono a suo tempo
fra i nobili, e Popoiani di Firenze, si fece capo di questa parte, con promefia, e
speranza d! esser follevato a cose maggiori, cio¢ all' affoluto dominio di Firenze,
ebenché per altro accortiffino, ¢ prudentiflimo, lasciatofi portare dal dolce de-
fiderio di domiaare, si fido nelle vane promefie della instabil plebe, con la qua-

lep id haver forze baftanti per conseguire 1' intento, s' accinfe all' ope-
ra; ma nel pil bello il popolo,.o spaventato., 0 pentito.l'abbandond, ond' egli
venuto in potere-del Governo fu decapitato: Eda Ini ¢ detto il Proverbio: Far
come Giorgio Seali, che vuol dir Pigliare a far' una cosa senza fondamento, che i
Latini con similitudine della Scrittura., dissero Scipione arundineo inniti, Di que-
flo calo di Giorgio Scali parlano tutti gli Storici, che scriveno le cose di Firen-
2

 

var”
144 MADMANTOELSST 9
ze digquei tempi, ed il Nerli fra gli aleriaggiunge, che allora'¢omincid
proverbio.; fancy Sutahogwies

STANZA XXX, i
Li tema fu di quefia sua lexione, Così, dicea, la vofirase mia,

Quand' Enea già fuor del suo pollaio SV Quis viva, e fanaye della 2
Paceta andar in'fregola Didone, » » Cacciara fu dal empia Mm

   
 

     

  

Com! una gatta bigia di Gennaro;~ » Tredira ancl ella fuor ¢

E che se i Greci ascoft in quel ronzone Pere sun rantoardire', etal 1

In Troia fuoce diedero al pagliaio 5 Parui, @ adefo gaftigar si

Ein man a Enea posero il lembuccio, ¥ bavete il modo senza cht a

Ond' ¢i fuggi col padre a cavalinccio; to ha finito, 1) Ciel vi benedica
Il tema del discorso,, che fece Franconioy fu quando Enea eflend:Soggiand
Troia fece innamorar Didone, 'ed aflomigliando:Celidora cacciata di Malman-
tile ad Enea scappato da Troia, esorta quei soldati a gafligar Pardire di Bente
nella, € rimettere Celidora nel suo ttato, già:che hanno il modow
POLLAIO, Si dice da noi quella:stanza, nella quale anno, edo
li: E chiamiamo pollaio quelle felue, o macchie, dove la sera vanno gli uece
a dormire; Ma qui intende per translato la nofira Casa, Patria yorluogo, dove
fiamo soliti abitare. > dog Gat
ANDARE in fregela, Dicemmo quel che significhi sopra-C. 1. stan. 25, Mas
che Didone fulle innamorata d' Enea, come favoleggia Vergilio, &
ché oltre che Didone fu così casta, che vedendofi violentata da Iarba f
ritania a rimaritarsi seco, volle pi tofto da se-stessa ucciderfi, che il
suo morto marito Sicheo con nuovi sponfali'; EB' anche vero, chene
seguire il detto innamoramento, perché Enea fu 360. anni prima di

 

verita si cava da diversi Autori, e si (corge in Darete Frigio', ¢ Ditti a
che (crifflero la vera Storia dell' eccidio di Troia. Che il nostro ic
ti gusita bugia-di Vergilio, dicendo nell' Inf, C, 5. “aie a
Li altr' ¢ colei, che s'ancife amorofa, y me
E roppe fede al cener di Sicheo, se gantiel 2

Non è meraviglia, perch Dantes' era eletto per suo Maeftro,  guida Vergilid.
Che Enea fude tanto tempo avanti a Didone, si deduce anche songs
Didone fuggendo l'infidie di Pigmalione suo fratello, che per desiderio i
le haveva ammazzato il marito Sicheo, come pure accenna' Dantes Parg, C.20.
Noi ripetiam Pigmatione allotta, onda
Cui traditore, ¢ ladro ye patricida toe illid
Fece la voglia sua dell oro ghiorta:, ». Seip 5
Portandofene il teforo in Affrica, chiefe a quegli abicatori tantovdi tert d
to poteva circondare una pelle di toro, cl' ottenne; Bd altutamente
detta pelle in firisce così fottili, che abbraccid con efle tanto terreno,
fico Cartagine, il che fu dopo 70. anni della edificazione di Roma » 7
edificata cirta 300, anni dopo la morte a' Enea, Sant' Agoftinodifein di It
done, che quando Vergilio non fufle stato dannato per altro, i i
no per quelta falfita cotanto pregiudiciale alla ripucazione di Didone sl gl?
difende ancora Aufonio col seguente Epigramma tradotto:dal Grecow 0 ©

  

Ad

 

 

 
 

TERZO CAINT/AA RE:

145

  
 
  
     

2 S.\Ad Didus Tmaginem CXI.
Dida 5 nis guiase con/pici '
Seances (ase modis ipleagen aise sy

9») Talis cram, fed non Adaro quam mibi finsit erat men:,

|. Vita nec inceftis Leta cupidinibus. ©

Namque nec dneas vidit me Troius unquam,

dig Neo Lybiam aduenit Clafibus Miacts;

L fens 5 atghe arma procacis Larbe

94 > vmorte pudicitiam

nificco 5 caStos quod pertulie enfes
5 wut lasa crudus amore dolor 2 *

 

nA ot Vita virnm, posiris meenibus oppetiy..
Jnnida cur in me stimulafti musa eMaronem,
Fingeret ut nostra damna pudiciria ?
Vos magis Hiftoricis lettores credice de me,
SL Quan qui fart Denn eonoubitufqne canunt;
bates Vates, temerant qui caPmine vernm,
: Humani[que Deos affiemilanc vibijs,

GATT A bigia Er quella, che noi chiamiamio'Soriana, che è un mifto di co-
lor bigio, € lionato ferpato di neroy-qual colore foriano ff dice folamente di Gat-
ti, onde io argumento, che'i primi' gatti di questo colore veniffero a noi di So-
ria, come vennero alcuni anni addietro quelli del colore del topo portati da Pie-
tro della Vaile dalla Perfiaye petd da molti chiamati Perfianini,. Vedi sorto C. 9.
stan. 19. 9) oe; a:

RONZONE; Conia jz;-¢ruda vuol dir Cavallo stallone', o per la monta, da
i Latinidetto eguus admiffarins':¢ per ronzone,-ronzine, 0 rozza jntendiamo
eavallo cattivo, Ronzone'con la, z, dolce vuol dire una specie di Moscone 50
tafano. Qui} Autore intende quel cavallo di legno fabbricato da j Greci per in-
gannare i Troiani come dice Vergilio. In. alcum Tefti si trova scritto caffone in»
vecedi rontone,. ma nel mio, che ¢ di mano dell'Autore, è seritto ronzone,
PAGLIAIO, BE” proprio guel cumulo, o maffa di paglia, che si fa dat Conta
dini dopo haver battuto il grano, per lo più avanti alie cafe; ma dicendofi dar
fuoco al pagliaio, s' intende Dat fuoco alla Cala.
PORRE il'lémbo'; 0 él lembuccio in mano, Significa Mandar via uno; E questo,
perché quand' altri vuol mandar via' uno di qualche luogo senza parlare, gli fas
il ferraiuolo addosso, e gli mette un lembo di eflo ( che /embo vuol dires
'na parte dell'eftcemita del ferraiuolo, 0 d' aitro abito, d yefte simile ) nelles

« Sic eecidiffe imvat; Viet ine vninere fama;

mani; ¢ da questo 'colui's aeoees d? esser licenziato, efiendo notissimo, che»
uclto detto Pigtiare', o-dare il lembo significa Eller licenziato; Tratto dai mac-
ri delle bore #'i quali, volendo licenziare un garzone, gli dicono: piglia il
lembo; piglia il'cencio, ec. ¢ intendono Vattene,
ef CAFALIVECIO.. Cioè in fw le spalle. B ndidiciamo portare a cavalluccio
da un giuoco, che fanno i nostri ragazzi in'quefta forma. Vino mette il capo fri
le gambe all' altro per di dictro, ¢ Sue così da terra, lo porta fra le spal-

le,

 

=. ay
    

146 1 MALMANTLELBE
'ce, cil collo, e per questo fii dice,a cavalinecios, Tir
cevano lo diceyano è coryla. fac Y
sopr' alle palme delle mani del portatore rivoltate dietro.

non accavalciava le gambe al collo, come fanno.i nostri, «
teneva al collo del portatore; € lo dicevano ix cotyla io
pt

 
 
     
   
 

mano di colui, che portava, come si cava dal Buleng..
da Cel, Rodig, le, rig lib. 27..cap.27. E questo.era.
una pena data a quei fanciuili, che baveano perfo
giochi, che habbiamo accennati sopra nel 2. Cantare »..B.
modi, con li quali portavano 5 così crano diversi i,nomi,.che dava
giuoco; perché si trova chiamato Cabefinda,ed Hippas, si come si ved
Polluce lib. 9,. 7. Che guefto giuoco fufle,usato anche dai L
re da Vergilio En. lib. 2. il quale-dice » che Enea. portd il Vi
padre in fu le spalle in tal muaniers Lawn seta se N

Ergo age chare pater ceruici imponere

Ipfe [ibibo bumeris, nec anidodhearaen A ikea
allegro,.¢.con buona

 
   

 

 
     
   
  
 
 
  
  
 
 
  
 

 

DELLA buona vozlia, Intendiamo fano.y,

Lalli En, Trau, lib. 1, stan. s1..diffle. >

 Stanne, diletta.mia, dibuana.

Parafrafando Vergilio, dove dice: Parce merx, E noidiremmo: 4

EVOR di questa foglia, Cioè fuori di Malmantile, Pigliala s

parte di sotto della porta, per tutto Malmantile; 0. intende foglia

reale.. aria. dgis: B4ine 5 Oa:
STANZA XXX Mica civ STANZA X

Poiche da esso inanimite furo caviond ince
Leschiere, si portaron a itor posti,

E gid fdraiaco ognun laffo,e maturo

Ingremboalfonnogliocchihavevapeffiy Mojtrando wwoler farne /
Quanda un trattoletrombe,ed il taburo Segui c-un' Vfirial [uo

  
   
 
 
    

Reppe i riposf, eifonni appena imposti; 'Che pik d'agn'altre men

Safi, presto cos) aoe fracasso, Tocco la redone 4 suoi intern

Ch'il fiatonitrombettier feappo da baffe, De' tamburini ye tro bert
Dopo che Franconio hebbe dato animo a i soldati ognuno andé,

¢ già tutti stracchi s'erano addormentati, quando in un subito fu dato:

be, ¢ ne i tamburi, che fecero fuegliare tutta la soldatefea; ma.

prefto celsd, perché i trombettieri, ¢ tamburini.lasciarono star di fonar

paura, che hebbero del Generale, il we entrato in collera di così gran

giurd di voler gaftigar colui, che era flato il capo di al follevamento, € 10:

dd ad effetto, facendo dare la corda a uno Vfiziale suo favorito, ch

farebbe mai aspettato, ¢ gli fece mettere i tamburini, ¢i i

' SDRALATO, Diltelo con comodita. Voce da,

confolazione, che'fente uno, che sia stanco a distenderfi

eratamente. Vedi forto C. 6. stan.26. E non crederei

di Cerbero, paraftalando Vergilio; dove dice

  

  

Loo his 1D. Oe a co si bs > te os ee Cs

 
 
  
}
i
i
a
i
A
@
ai
y
-

 

TERZO CANTARE: 147
6  teqce immania tered refoluit
ue a Fufies bumi, toroque ingens extenditur antro. xq

'A Vivato, In un subito. E quelto termine @ an rrarto fighified anche tutti
due,0 pili alla volta, ¢ si pud intender, che le trombe, ed i tamburi, cioè uno,

eee fiato da baffo at trombettieri, Cafeare il fiato vuol dire Haver paura,
o timore; onde con questo dire intende, che i trombetticri hebbero paura' det
een. di fonare; non perch veramente perdeficio, o
ulciffe loro il fiato dalle parti da baffo.

YNCOLLORITO... Adirato-. Entrato in collora.

OCCHIO true, Prafe latina; usata da'noi, ¢ significay ¢ moftra lira che»
uno habbia;'¢dicendofi: 11 tale mi guarda 'con mal' occhio, 0 con occhi torti,
s'intende il tale @ adirato meco - Hec autem toruitas a taurorum ferocia dicitur,

MIN ACCIO! col dito', Coloro che vogliono gaftigare qualche delitto, o ven-
dicarsi d' alcuna ingiuria, fogliono brandire il dito indice verso quel tale, che
vogliono gaftigare, ¢ tal brandimento si dice minacciare dal Latino Afinari, o wi-
nitari, sur:

CHE più d' ogni altro meno se l'aspetta. Per esser quelto soldato amico, e molto
in grazia al Generale; non havrebbe mai ¢reduto, che egli l'haueffe a gaftigare,
TOCCO! locorda, In' Birenze danno la corda legando il paziente per le mani
legate insieme dietro alle reni; ¢ per quelle ran @ un groffo canapo, che
paflaper-una carrucola, tirano il p in fu, lasciandolo leggier: feor-
ter in git, ¢ poi ritirandolo in fu'tante volte, a quante ¢ condennato, ¢ quetto di-
ciamo; dare rratti di corda. Qual tormento da“1 nostri antichi era detto dar /a,
volla, 0 collare, ©'noi diciamo: dare la corda. Soggiunge poi: Co' /uoi intermed) di
tamburini, ¢ trombettieri a' piedi; cio con tutto quello che ci andava; il che era,
che i tamburini, ¢ i trombettieri, i quali erano stati complici a tal delitto, stel-
Bp ys! Hr lui affiflenti a vedere efeguire la piuftizia, come si cofluma,
quando molti (ono complici d' un delitto, per lo quale vien gaftigato feveramen.
- te il capoy 'ipale, € gli altri complici ricevono minor galtigo,, ed adiftono a
\\wedere iligaftigo del 'loro principale, Io però non sono Jontano dal credere, che
il Poeta per foftenere 'a faa Opera fempre in si le burle., habbia -yoluto in-
5 che i camburini; ¢ trombettieri fuflero essectivamente legati a i piedi di
coini, che era tirato sa, ¢ voglia moftrare con questo il coftume, che si tiene ia
Firenze di legare.a' piedi di tali pazienti qualche cosa, che significhi il delitto da
Jui commefio, acid che il popolo comprenda la cagione di quel martirio, come
per efempia:;a un fornaio, che habbia facto il pane cattivo, 0 di minor pelo del
dovuto, faranno legare a' piedi un filo di panc, ¢ così gli daranno 1a corda:e mi
la(cio indurre a cia 5 che il Poeta habbia voluro intender questo, dal vedere,
che: egli: nel? Orrava seguente dice; alla corda onole che sia attaccato così: i qual
detto pare che esprima, che il paziente debba toccare la fune co'i trombetti, ¢
tamburini legatigli a i piedi,:

 

 

T2 STAN-

 
 

 
    
 
 
   
   
   
  
    
   
  
    
   

  

148: methathethadhalet ities

STANZA Pepe ake:
Alla corda così vol ches' attacchi 5 vstene
Perché a arbitrio ye senza consigtiarfiy aed di
Facea venir all! armtiallor che
Bifogno havean più di riposarsi,
Ed eran mexxs morti; e come bracchi 5
Givano anfando inordinati  esparsi,)
E con un fuor di lingue,e orrendavifta -

Sofiavan,ch'ioho spp un Alchimifia.

ll Generale fece dar la corda a quell' Vifiziale non | 20 flo, perch
fol' arbitrio di far dar' all' armi (enga il suo confenfo ma \ancora
u(cito fuori del concertato, il quale era di offeruare prima dim;
se le flelle prefagivano buona,o trifta forte, B qui il lettore fir
sta in fale burle, ¢ sappia, che l'Autore non flimayva che I aft
a tanta precognizione, ma si bene, che idabeaue faa Sepnarig sites

ifti, injilisa 6

D ARBIT RIO xe propria eerrefia Saonahe lo stesso; ed  ambesie si
Di suo capriccio, 0 volontà...) “th
cANS ARE, E} quell' impeto 5 o.romore, che fa il relpiro 24
il - ( che noi pure dal Latino diciamoanelare ) e vient a

aCCO, Cane per uso di caccia, il quale quando è stracco re

ae, ¢ tiene la Jingua fuori; B se bene fanno così tutte le sp
nostro solito far questa comparazione /olamente ai bracchi
meate sono pil fortopostia fracear hij i percio che Rimolati a
di trovar preda, fanno maggiore; ¢ piu violento viaggio che
fio Sat, 1. Nec lingue quantum fitiat canis Appula tantum.
ORRENDA vifta. Vista spaventevole;, che tale € il veder vat
bocca aperta, ¢ con la lingua fuori, perch Pee: lo pil reftano in
gl' impiccatiy. 9 a

SOFELAV-AN ch' io-ho spoppare we Alchimifta. Alchimiti.fon'
fiano nel fuoco per trovarl'oro, ¢ senza nominare Alchimiftay
sale fofia s' intende,, ¢ Alchimifta, Se.bene s' intende, ¢ Bada
cennammo sopra c 1, Nan. 37. anzidicendoli Ji cal. fag, Me bimiffed
tale fa la spia, ¢ tutto è fondato ful verbo fofliare » che signitica Far

10 ho foppato Significa io repre meno, 0.io non 'timo punto il
fanno gli Alchimifti in paragone di-quello, che foffavano questi
stesso tignificato, che il termine ne ferede dss pra Ga, 1068 ns
mo fotro C, 6. stan. 61, ate

TAMBVSS-ARE., Rerquotere y aie dle bl Biparola gei pr
macellari, che dicond Tiamibaffare qu: se
perché la pelle si spiochi bene salislivaats © ae anc
dremo forto C. 11, stan. 26, E tutto ha Origine dal.ta

 
  
 

che fa esso,s' aflomiglia al romore, che fanno i macellari 5 '
 

ase

SAAR RERERAEEORE ERGEE CE Set

-

TERZO CANTARE- 149

STANZA XXXVI.
Hiomai la Bama, che riporta a volo
| Diagn' ii nnove, elegazzette,
 Spargeper Ma ilche ar mato fiuolo
View per tagliare a tutti te calzetre 5
Gig molti impauriti, ein preda al duolo
Non piie co i naftri legan le foarperte,
Macon buone,, ¢ faldissime minuge,
Perché stien forti ad wn cumores tage.
STANZA XXXVIL
In tal confufione, in quel vilume y
| All udir queilamenti, ¢ quegli affanni
4 molti cht eran gid dentr' alle piume
Lo shucar fuori parue allor mill anni:
Chi per veftirfi riaccende il lume,
Pera ch' al buio non ritrova i panni 5
«Chi nudo foappa fuori,¢ non fa stima,
Che dietro gli sia facto lima lima.
Sparfo per

ntile 1" avvilo dell'

STANZA XXXVIIL

 Rerché s'egli ha camicia, 0 brache,o ve/P4,

Non bada che gli facciano il baccano;,
Ben si del sriffe avuifo afjlitto refha
Onde pitt a' un poi ginoca di lonrano 5
Chi torna indierro a fasciarl 1a testa y
E chi si tinge con il raferano,
Chi dice, ¢ una dogisa fegii ¢ pref »
Per non haver aire a far dife/a.
STANZA XXXIX,
Altri, che fugge anch' ci simil burrasca,
Finge l' wnferamo, ¢ vanne allo spedate,
E benchi [ano ei sia come una lasca
Col medico s' intende, € col /pexiale,
Perché alt'uno, edall'altrocpielatasca,
Accio gli faccia fede ch' egli ha male;
Ed essi quefho, e quel servvon malato,
E chi piis da, do fan dt gid /paccsato.

-Maimai arrivo di detta Soldatesca, gli abitatori
s' acciafero pil al fuggire, che al difenderfi. Narra il Poeta diver-

-diquel lnogo,

Sen tale spavento, ¢ le varie scule, ed invenzioni, che rrovano coloro
sper nion hayer adandare alla difefa della muraglia,.

“ GAZZETTE., Novelle, Avid, Carte d'-avvili. Egazcetra diciamo anche

-lacrazia,Veneziana.

TAGLIAR ie calzetre. Tagliar le gambe. Es' intende, dare delle ferite in.
pore del-corpo, se ben le.ca/zere non veftono se non le gambe: Come
jamo anche rompere ia tetla, ed intendiamo Ferire il nimico in quelle parti
idel-corpojeherci. verra farto.. E diciamo fiaccar le braccia a uno con le baftonate, se

bene:in ogni altra parce

giidaremo che nelle braccia...

NaST RO. Et una specie di tela, 0 benda che non eccede la larghezza d' un
feltordi-braccio., eferve'per legare, o falciare sda i Latini però detto, Vitra, ed

in alcuai uogbi d' dralia derco fersuccia.

MINVGE,. Corde da strumenti musicalicome Tiorbe, Liuti, ec. fatte di bu-
<dellaidi beftie;'¢ pero Dante Jaf. c. 28. per intender budella disse,
wich ili \Tralegambe pendevan le minugia, i
-\) Diceche non 6 fonodegare de scarpe coi naftrt,.ma con le minuge, perché fo-
no pil fode, ¢ da refifter pill; Ed ¢ coftume nfatissimo il dire: // tale sera dega-
16 le fearpe bene', o.conile minuge y per intendece Correva forte, 0 volava: fuggen-

«do i pericoli, cheicidiintende con

quella fentenza;, Rumores fuge »

- CONEYSIONE,,¢ vilume., Sono in quetto uogo quaG finonimt havendo Jo
stesso significato di Viluppo, imbroglio, ec.

DENT RO alle piume

Ji quando -vogliono

FAR lima lim Define, dicggiate
AR lima lima, Beffare, dileggiare. |
dar la-burla AUN 5

un modo proprio da Fasciulli, i.qua-
si fregano al dixo indice fapra Iindice

« dell' alta mano.a guifa'dicoloro cheJimano.s ¢iNoltanioG verlocolui, che ve-

va

 
   
 
 
 

150 MALMANTILE a

glion burlare dicono. Zima, lima. Vedi forto C. 9. stan. 66. annot; ©
WON bada. Non cura; Non offerva, Non gl' importa'; Il verb
vuol dire oflervare, ha più significati, come Attendere,
ligenza, curare, stimare, ec. Bada a tuoi negozzi. Badaa
viene. In somma hala forza del Latino Cwrare, Vacare: si dice: 7
bade, per intender Trattenerlo. Star a bada d'uno: per inteadere
do I opera, i favori ec. d' uno. ae FF
BRACHE, Calzoni. Brache da noi propriamente si dicono qu
ghi, che usano i Soldati a piede Tedeschi guardie del Serenissimo Gri
1 Paggi nobili. B si dicono taluoita Brache quei calzoni che si
chiamati ancora Mutande; Vedi foro C.6, flan.z0, ©
FAR il baccano, Qui vuol dir beffare, dileggiare con fischiate'yo st
mili; ed il suo significato proprio è Fare strepito, far romore ¢ viene:
nalia, ve
GIVOC A di tontano, Cioè non s' accofta: ed è lo stesso'che Parfene
che vedremo nell' ottava seguente cook ta
BVRRASC A. S intende propriamente il travaglio del mare; ma 10:
per ogni forta di sturbamento, 0 pericolo. Forse meglio borrafea:

  

 
 
     
 
     
 

     
    
 
 

 
  
 
 
  
    
  
  
   
 

SPEZIALE. Colui che manipola, e vende medicament; ¢ però
detto Pharmacopota; ed altrimenti -dromatarius da aromata ye noi lo
giale da spezicric, come si trova anche in Latino aeenaae
TASCA, Scarlella, che ¢ un facchetto appiccato ai 3
uso di tenerui dentro quello, che occorra alla giornata, e particolarm
ri; ¢il Latino mar/upinm, Ed empier le tasche a uno, vuol tie Dargli
naro. BN
LO fanno spacciato, Cioè dicono, che egli è in grado di morte,
ta, che i Medici regolando le atteftazioni delle infermita con le fomm
nari, che crano lor date, facevano fede esser in grado di morte q
ne dava; ¢ quel che ne dava ri atteftavano, che era leggi¢rment
STAN xX. 7
Si che con queste finte, ¢ con quef arte D' uno fheffo voler La maggior parte
Coffor,c! usan la taxxa,e non latarga, Trovan la'via di sparfene'all vd n
Servir volendoa Bacco,enona Adarte, Ed il reftante non si afbute 7
Che non fa fangueyma vnol che fisparga, Compari/ce,perch? ei nom puo far alt
Quetti abitanti di Malmantile con tali feufe, ed invenzioni cercano di for
si dail' andare alla guerra, ¢-folo vi'va chi non ha danari, ne da!
berarfene. ' 5 ae (eae
TARGA, Brocchiero,Scudo, Rotella'. Intende 5 che fon più avvezai a
re che a guerreggiare, ed-hanno piu genio con Bacco 'Re del vi ~
hanno con Marte Xe delle guerre; percné quello fa nascere nel carpo il
equetto lo fa disperdere. ' 1b coreakant
ST ARSENE aa larga, Significa non s impacciare d' una cola, ed è.
che gixecar di lontano, che vedermmo nell' Ottava antecedente. © 5
eASTVTO,¢ fealtro, Sinonimi di fagace, ed accorto, Huomo, che fa il con-
to suo. Ma per maggior intelligenza di quelte parole tute 5 ¢ /eakro, ee

 
 
  
   
 
  

 
     
 
  

 
 
   
   
ds)
eile

“

er ihe

 

'TERZO CANTARE} 352

td dccorto & da fapere che, se bene ce ne serviamo per finonimi, tuttavia ci ¢
qualche differenza; particolarmente fra,/agace, ed a/Puto; perché l'arti, che dal-
la fagacita s' adoprano, non meritano biafimo, per non esser se non avvedimen-
tivortili, ma schietti, reali, ¢ senza fraude, o ingaani: El afuzia oltre alles
fuddette lodevoli arti si serve anche delle menzogne, fraudi,¢ falfira, ¢ d' altre
cose poy scarslepgs nobile. E però Scaitro, ed accorto pare che meglio s' adat-
tino per finonimi a fagace, che ad a/furo, al quale più proprio finonimo farcbbe
Maliziofo, 0 trifto, 0 furbo; quando però la yoce furbo ¢ prefa in fealo.d' huo-
mo, che fail conto suo; Ma, come ho detto, ncl comun parlar civile nots
usiamo così efacta com »¢ puntualita; ma pigliamo l'uno per l'altro.

eee men Xl, ree' XXXXIL

lentr? in piarza si fa nobii ar fa 5 Vilta  arretra, honor ds poil innita

Auch in Palagzo Meda tans A cimentar la sua ee in guerra,
Con unatreccia avvolta,e Paltra/par/4, L cfortal' una a conseruar la vita,
Corre alla Mdaimantilica rovina; L' altroa difender quanto puolaT err.
Benche ne i paffi ape vada pik fearfa, Pur fatto conto di morir veftica
Perchi all' uscioda via mais' avvicina; Voltoffi a bere, ¢ divenuta sgherra

+» Da ferte volte in fu gid s' ¢ condotta ( Pero che Bacco ogni timer dilegua )
Fino alla fogha; ma quel faffo scotta. Dice:O de'mici:cht mi vualben mi segus.

« Mente che la men codarda gente si raguna in piazza, anche la Regina Berti-
nella. al romore, nuova Semiramide con i capelli non ancora finiti d' aggiufta-
xe, correa difender Mabmantile; ma non con tanto ardire, perché questa nostra
Semiramide non s' arrischid così subito a paffare la porta della Casa; ma si fermd
in quella, sospefa,¢ travagliata da due gran paffioni Poltroneria, ed Honore,
che quella J' esorta a starfene; ¢ gquefto Y obbliga ad andare, Al fine lasciataG
persuadere dall' Honore prefe animo, ed esorto i suoi a seguirla.
go MT RECCLA. 1 capelli delle donne si chiamano #recce, perché per lo pil foglio-
no le donne far due parti de i lor capelli, ¢ ciascuna di quelle fuddividere in tres
altre parti, ed inteslerle in terz0, il che si dice treccia; B Bertinella stava così
Antrecciandole, quando senti il romore, per lo che lasciato il lavoro corse cons
una. parte intrecciata,¢ l'alcra nd, comic dicono, che faceffe Semiramide, quan-
do senti il pericolo, che fouraftava a Babillonia.

MA 1a foglia scotta, Quando uno o per debiti, o per delitti fla ritiratoin ca-
fa, 0 in Chiela, diciamo: Nor esce, perché la faglia feotra; ciok se egli ulcitie di
cala., 0 di Chiefa, farebbe fatto prigione: ed a Hertinclla scorra quella foglia,
perché se ulciffe di quella, pericolerebbe di toccarac.

VILT A', Qui vale per poltroneria, o codardia.

MORIR vefiite., §' mtende di coloro, che sono ammazzati, i quali muoiono
con le vefti in doflo, e però dicendo che fa conto.di morir veftita, s' intende che
ella ha risoluto d' andar a farsi ammazzare.

SGHERRA, Braya, Animofa; fatta così dal vino, che leva di tefla ogni timo-
re. Bacco dai a fu detto ibe, rch liber l'huomo da i pensieri noiofi,
€ pero dice ogni pensier dilegua, ed il Chiabreca disse.

ia ssi oe 9 ¢ dianfi al vento
4 vorbidi pensicré,

 
 
    

192 MALMANTICE! 27

Seneca de Tranquillit. disse; 2vonnunguam ad ebrieracémv
gat nos fed ut deprimat curas = Elevat enim curas,@ ab imo:
norbis quibufdam, ita trifitie medetur, Di tr
Generale del? Imperadore Ferdinando 2., il quale mai si
glio di — » ne si mefse ad imprefa alcuna importante,
molto bevuto. E Bertinclla imita questo gran guerriero }

STANZA XXXXIIL ' STANZA XX
Dietro a suoi paffi metteft in' cammino Piaccianteo uo se
“Maria Ciliegia illuftre damigella; t
Tutto lieto la segue il Ballerino, ci a
Che canta il titutrendo falalella. Ed ¢ la diffrurion

 
  
    
 
 
    
   
 

Va Meo col paggio, Zoppica Atafino, Gid miftro le doppie con la

Corre il Maffelli,e il Capitan Santella, Finiro poi che fu quella bona

Molti,e molt' altri amici la seguiro, 0 omtagit

E pie Mercanti channo havuto il giro, Ed hora in Corte ferne a

Alle voci, ed ordini di Bertinella obbedirono diversi suoi seguaéi'
Matti. ses Vie

MARIA Ciliegia, Fu una Donna creduta pazza, la quale andava'
ricevendo elemofina senza domandarla, Coftei con una flemma,
ordinaria difeorrendo fempre da per f€, diceva belle, e fenfate

de da molti non cra flimata pazza, ma uguale a Diogene, che abitava
te; ¢ per tale azione farebbe stato riputato matto, se non have
belle Eiietac Ȣ dogmi, come appunto fece questa madonna Mai
la quale, o parte di essi sono staui raccolti da un buon fetterato 5
volta gli dara alle Rampe: Come Diogene,anch' essa non si curava
dormiva nelle frad¢ sotto qualche portico o loggia, ¢ percid porta'
pre un granatino per spazzare quel luogo, dove si metteva a do
spazzola per spazzolarsi la vefte, 1a quale benché poverifiima, era
molto pulita, ¢ se bene piena di toppe, atiai bella per efserui le n
mefie forte anche senza bifogno, con vago, ed aggiuftato ordine.
ta sua sporta haveva ancora qualche biancheria, ¢ molte volte un
danetto pieno di fuoco, nel quale, pa/seggiando per le rade, ai
le sue vivande; forto la gonnella haveva pil facchetti, entro ©
pentola, ¢ piatti per suo uso, ¢ quello che le avanzava a' suoi man
va forelle, ¢ nipoti i quali si trattavano comodamente, ed -habitas
buona caforta, che era di detta madonna M aria, dove ella alle volte
mutarsi; ma non volle mai fermaruifi, ne dormirui ancar che prega
ta anche da' detti suoi parenti-a volere flar con loro, Bu(cava mo!
li quali comprava quello, che parcamente le bifognava, ed ogni
va per ? amor di Dio tutto quello, che le avanzava, ¢ per Jo pi
nache, dove alle volte port® anche fino a dieci fendi, Domandata'
qualche parere, non rispondeva; ma seguitando il suo solito chiaceh
ma, che quel tale i partifse-da lei reflava appagato con qualche fencenaa,
to, che ella diceva a proposito del quefito. Per efempio: Vna mattina,
clia (otto le logge d' avanti al Tempio della Santissima Annonziata, un

  
    
    

   

    
 
 

  
   

   
  
  
  
 
   
        
   
    

Bmnwceocacow =~
 

 

TERZO CANTARE.: 153
wie netto le domandd,se ella credeva, che la sua moglie bella, da madonna Maria
we, molto ben conosciuta, fufle honefta; ma gliclo disse con la più sporca maniera,
: che dir si le. Madonna Maria senza alzar la testa, o dar fegno d' attenzio-

ne al del giovane, seguitando il suo discorso, che faceva del poco rilpet-

to, si portava alle Chiele; dopo molte chiacchiere disse; Vedete voi questo
oe sboccato, il poco rispetto, ch' ei porta alla Chicfa? La sua moglic &

la, ¢ la prefe, che ella era onefta; ma che puo ella havere imparato da lui, se
non il modo di diventare altrimenti? ed hora io ho, che ella sia diventata; perché
ogni gelofo ¢ becco. E seguitd il suo cicaleccio, entrando in diversi altri gine-
prai, come era solita; ¢ così chiacchierando tutto il giorno dalla matting alla
fera, buscava molti denari. Coftci mori, ¢ si trové nella sua sporta una borfet-
ta, nella quale era una ricevuta di cinquanta sCudi, dati a certe Monache con,
obbligo di far dire una meffa il mefe all' alrare della Santis, Nunziata per l'ani-
ma sua 3 dil che fl-cava-argumcato.. che.clla non fufle pazza. |!) |

FAL. LA. Così ¢ chiamato un contadino trifto,il quale non hayendo vo-
glia di lavorare, s'é dato a chiedere clemofina; ¢ per far venire le donnicciuole
alle fineftre, ¢ cavar loro di mano robe, ¢ danari, va per le strade cantando al-
cune sue ottave amorofe, ¢ ad ogni due versi fa intercalare con la voce dicendo
Falarera tututrendo, con che si persuade d' imitar il suono del Chitarriao; ed all'
ultimo dell' Ottave,al medesimo suono della voce, si mette a ballare, ¢ per que-
fio i) Poeta lo chiama il Ballerino; ¢ poi va attorno chiedendo la limofina.

4420. Era uno scemo di ceruello pro vnSAe dal Palazzo; e perché egli
fon si reggeva benc in piedi, pero andava fempre appoggiato a un ragazzo; e+
Patent Palos ct peers.

SRELEEER ELSSEREPES Shs

AL ASINO. Exa uno stroppiato nelle gambe, ¢ nelle braccia; il quale era an-
ia' i q

ch' egli provvifionato dal Palazzo per quelia sua figura cotanto contraffatta das
ale gli te og s
% MASSELLI, Era un mato 20 creduto tale, provvifionato pure dal Palazzo.

Coffui haveva in mente tutte le fefte del anno, ¢ quali Ofizzj, ¢ commemora-
zioni dovean farsi da i Preti giorno per giorno. Sapeya in oltre,quali erano quei
Rettori, ¢ Curati di Chiefe, tanto in Firenze, che nel Contado, i quali nelleo
fefte trattavano bene, © male ai loro definari; ¢ da essi si Jalciaya in tali giorni
rivedere; ¢ mangiava, ¢ beveva tanto, che è imposiibile a crederlo anche da chi
I ha pil volte veduto. Era soprannaturale nel digerire, ¢ s' è veduto fmaltires

SN

ae

we eerie vt: ¥.
gran quantita di roba, si puo dire impotlibile, come farcbbe un gran piatto di
= catia teacsiA bollita in brodo di bue, e condita a guisa di maccheroni; fee vol-
è te biffo, ¢ tela d' olanda nella fiefla forma, ¢ questo in breve tempo, ¢ fenzas
i”  difficulta, 0 dolori, Li Poeta dice; Corre il eiaffelis, perché veramente coftui,
“ benché decrepito, era di gamba velocissima. Haveva ui Serenifs. Gran Duca da-
i to per servitore al Maflelli un giovanotto gagliardo, perché Jo seguitafle per tut-
i to dove egli andava, ¢ offeruatic tutte le suc azioni, senza mai contradirgli, 0
ee impedirlo, ed ogni fera riportafle quanto il Matielli haveva fatto in quel giorno,
4 Quando il Mafielli riceveva alcun diigulto da coftui, non s' alterava seco, mas»
wy si metteva la via fra gambe, ¢ senza mai fermarsi, 0 voltarsi ae meno a dietro »
i

non la guardava a camminare di buonuilimo palio 25., 0 trenta miglia con gran-
digimo

aS!

 

 
    

154 MALMANTILE | 4

dissimo travaglio, ¢ rabbia del servitore, che non poteva, ne do
conveniva, che lo seguitaffe; onde andava molto cauto in strapazzarlo
ful principio del suo servire faceva fino a baflonarlo ) non tanto per
gaftigo da S, A, S. minacciatogli, quanto per il timore, che il Mafie
detta non viaggiaffe.
CAPIT AN Santella, Questo fu un soldato della Banda di Piftoia,il g
la volta al cervello ( 0 così finfe ) perché gli fu rubata la moglie da chi ne}
più di lui. Coftui venne in Firenze, ¢ vi dimord qualche tempo,
se pazzie; ma perché fu conosciuto, che forto questa sua finta pazzia si
deva una gran triftizia, fu mandato forzatamente in Candia al servizio
Veneziani, donde non è pil tornato. s
© MERCANTI, ¢ hanno havato il giro, Cioé gente impazzata. Si serve
parola Giro per intendere il girare del cervello, che vuol dire lmpazzare
per il Giro de' Mercanti, che si dice, quando un Banchiere tiene in mai
naro di tutta la Piazza; il che in Firenze tocca a fare una volta per uno
banchieri, o negozianti più groffi per tanti mefi; il che ¢ fatto per co
mercanti; ¢ dicefi: vere il Banco giro.
P/ACCIANT EO, Fu un Fiorentino di cos) vili natali, che non si fa
la cafata, ne il vero nome suo, eflendo fempre stato inteso col folo fop
di Piaccianteo. Coftui dalli parenti suoi fu lasciato aflai comodo, ma
lo, che era dedito alla crapula, confumd in breve tempo tutto lo flato
a pena haveva dato principio Serer le milerie della poverta, ¢ gli
la Fortuna di nuovo lo follevé facendoli redare da un suo congiunto
considerabile di doppie; ¢ però il Poeta dice: Gid mi/uro le doppie con Ia fh
queste ancora il buon Piaccianteo diede prefto fine, pensando d' haver
rare il fentenziofo proverbio, che dice; e4 uno scialacquatore non ma
denari, Mas' ingannd, perché ridotto in eftrema poverta, ¢ non
meftiero alcuno, si riduffe a portare quella barella, con la quale fip
ammorbati al Lazzeretto nel tempo, che fu la Pefte in Firenze, €
tal contagio campé di cotefta sua fatica; finita poi la pelle viveva
buscava con far servizj alle meretrici; e però il Poeta lo fa servitore
la, ¢ suo Aio, e direttore. 'Piaccianteo voce che ha dell' antico Péacent
MANGTAR le cacchiatelle col cucchiaio. \perbole usatissima per i
gran mangiatore, Cacchiate/la, E' una specie di pane finissimo fatto
ed alla grandezga d' una pera bugiarda; onde con questa iperbole, int
che pigli in bocca in una volta cante di quelte cacchiatelle, quante pigli
delle fragole, 0 pifelli, o altra cosa simile, ¢ così viene a ¢flere ipafbae
perché il cucchiaio comune è capace a fatica d' una fola cacchiatella
ca dell' huomo difficilmente riceve una fola cacchiatella per volta:
di, che mangiava le cacchiatelle in grandissima quantita, e senza m
me non si numerano le fragole, ec, che si pigliano col cucchiaio.
ELA distrugione della Vernaccia, B' gran bevitore. Vernaccia & una
no bianco, mal' Autore per Vernaccia intende ogni forta di vino.
AUSVRO' Ie doppie con lo spaio, Haveva gran denari. Ipetbole fata |
der un gran ricco; ¢ ci viene dal Latina Agdio pecuniams metitur

  
     
  

 
 
     
   
 
   

e

   
    
   
      
    
      
   
  
   
    
  
 
    
    
   
  
 

  

 

 
 

 

 

Bh

EEN

Tae

eae

as

TERZO CANTARE. 155
4 SON. V.ACCIA. Significa placidezza di mare; ma noi la pigliamo anche per
ogat forta di bene (lare, ¢ di buona fortuna, come ¢ intela a presente oe '
“ BARELLA, Specie di veicolo Gmile alla bara, 0 ferétro, col quale si porta-

 

ert:

a fotterrare; ma quelta che serviva per pertare gli ammorbati cra

no
Coperta sopra con cerchiate, ¢ tela incerata a foggia di calsa tonda di sopra, co-

me i tamburi da via,

Comanda la padrona ch' egli scenda,
E sia gik fuori com gli ovecchi attenti
Fra (nth fheesfe ch'ei non intends
| Ache fine fon Id cutante genti;
Ma queglial qual no piace tal faccéda,
» Sela trimpelia, af pillein complimenti,
E, perché at fichi il corpo ferbar vnoley
Ekin crf in queste, 0 simili parole,
STANZA XXKXV1,
Alea Regina, perché a Obbedire
Pitt d ogni altro a' tuoi cenni mi do vito,
Cold n' andro, ma (come si fuel dire )
Come ta ferpe, quando va all incanto;
Won cbt 10 fusga il pericol di morire,
 Perch? so fo buon per una volta tanto;
Ma perché,s*io mi parto, non ti refea
Vet buom, che sappia,dov'eglibalarefta,
STANZA XXXXVIL
Non ti fdegnar, s'io dico ul mio pensiero,
at jibil non è ch' io tacciaofinga,
E,(e w andaffe it collo, fempr' il vere
Son per dirti,e chi Pha per mal,ficinga,
Ti fernird di cor vero, ¢ fincero
Senz? intereffe d' un puntal di stringa y
E non. come in tua Corte ono aleuni
Adulator, che fanno Adeo Raguni,

 

STANZA XXXXVIIL
Jo dunque che non voglio esser de' loro,
ea tengo l'adular peffimo vizio,
Soggiungo,e dico, per ridurla a oro,
Che mal distribuita ¢ questo usixioy
E che non pus palar con tuo decora;
Peicht moftranda non haver gindizio,
Vn tuo dio ne mandi a far la spia
Quali d' huomin tu havelfi careftia.
STANZA IL,
Manda manda a spiar qualche Arfafatto,
O un di quei, che piscian nel Curtile 5
uefio farò il meftier, come va fatto
Senza sospetto dar nel Campo oftile:
Oftile dico, mentre cofta in fatto
Che cinto ha d'armi tutto Adaimantile,
Tal gente si puo dire a noi contraria,
Berche nevien quafsh per pigliar' aria,
TAN f

E perch' ¢i non vorrebbe uscir del cova
Soggiunge dopo queste altre ragioni;
Ma quella, che conosce sl pel nell'vove,
S' accorge ben, che fon tutte invenzioni;
Pero [enza pin dirglielo di nuovo
Lo mands fers afuria di [pintoni,
Eynitr'ei pur volea vabrogliar laSpagna
Gli fal uscioferrar fu le calcagna.

Bertinella vuol mandar Piaccianteo nel Campo di Baldone a [piare; ma egli,
che non vorrebbe andare, adduce mille scufe; quali non gli sono ammelse, ed &
cacciato fuori di Malmantile a furia di spinte,

 TRIMPELLARE, Intendiamo quel suonare adagio, ¢ tentoni 1a chitarra,
liuto, © altro strumento simile, che fanno coloro, che imparano a fuonare:

da questo per tri

0 si fen.

er trimpellare, 0 BIA S
Za profitto,rempeliare che diciamo anche metteria fuk into, o metterla in musica,

€ fuona quafi lo steffo che,

SE (a paffa in complimenti, Che significa Perder il tempo in yane cirimonie; ¢

senza toceare la fuftanza dei negozio.

VVOL [erbare il corpo ai fichi.. Vuol veder di viver, quanto ci pud, ¢ nony

metterfi a rischio d' efscre ammazzato,

 

Vi VR.

 

a
q
%

-
. = ee

—

  
  
   
 
      
    
  
  
   
   
    
  
  
   
  
   
     

156 MALMANTILE

OBBEDIRE a tuoi cenni mi do vanto, Profelso defer' il più o
re che tu habbia, e di fapere intenderti anche aicenni. —
COME Ia ferpe quando va all' incanto, Cicé mal volenticri,
Folens nolenti animo, Omero, Ii Lalli En. Tr. C. 2. stan. 32. dice
Come la biscia all' odiofo incanto:
FO buon rr una volta tanto, Poslo morire una fol volta, Quando |
danaro, ches' ha in tavola, allora che uno ha perduta quella porai
veva, cava di tasca nuovo danaro, o vero dice: fo Laon, cioé pi
scudo, o per due, fecondo che gli pare; ¢ s' intende, che non vuol
la (omma, per la quale ha fatto buono, cioé promefio, Per efempio
no per uno scudo, ' avvertario inuita di due, io tengo la posta, ma
vincere, ne perder pili che uno feudo, perché non fo Fiat di pil.
SE w' andaffe il colle, Se bene io fapeth, che ci fuffe pena la vita, WV
curim in manibus tenens aliquis ceruici esset incurfurus mes, conticerem. —
CHIP ha per mal, si cinga, Non m' importa, che altri l'habbia
si cinga pur la spada, ch' io fon pronto a rispondergli. Nel primo
dell' Autore dice si cinga, ¢ vuol dire si levi pur da lato la spada,
modo io non voglio far quiftion feco. L' Autore, che fapeva, che
i modi si dice, stimo fore meglio detto /i cinga, perché nel fecondo
di sua mano, dice f cinga.
SENZ! intereffe d' un puntal di springa, Non voglio da te cosa alt
che minima: Suona lo stesso che um puntal d' agherto, che vedemmo
stan, 10. ¢ che il Lat. We ligulam quidem.
FANNO Meo Raguni, Cio' ragunano danari. La forza fa nella ¥
che se ben pare, che sia il cognome di Meo, è il verbo ragunare, ¢l
meitere insieme,¢ 44eo € prelo in vece di meus, mea, meum, © vuol '
raguni marfupio, cioè raguni alla mia tasca,
ETENGO I adular pefimo vizio, Non è dubbio, che l'adulazione'}
trando, ¢ percid Dante mette gli adulatori nell' Inferno gaftigati cont
vera pena, che si legge al C, 18, dell' Inf. Cicerone nel suo lib. de
de gli adulatori così: lis denique temporthus cavendum eff, ne afs
ciamius anres, neve adulari mos finamus, in quo falli facile eff; tales enim
wt inre laudemur, ex quo innumeratilia nascuntur peccata, cum homines
nibus turpiter irridentur, © i maximis versantur erroribus. Di:
mandato qual beftia mordeffe più ferocemente rispose; Nelle faluatiche
tore, nelle domeftiche l' aduiatore, perché con le sue falfe jodi ti
rovine; Bd aggiungeva; che'le parole composte non per aprire il ¥
compiacere 5 sono un caprefto melato. Si potrcbbono addurre
gravitfimi Autori, ma si lascia di farlo, perché non torna affatto al pi
si rimette il lettore a Plutarco nel suo libro de digno/cendo amico ab
PER ridurla a oro, Per ridurla alla perfezione del difeorfo; Per
conchinfione. Vedi forto C. 8. stan, ys
COME se tu havefi carefia a huomini. Come se ti mantaflero hu
'to. Ancora apprefio di noi quando fidice: W/ tale oa
buono a quaicofa, seguitando il detto di Diogene Alomintem quare
x2, Corsorramini » & virr ofoe, Omero, Viri gfe.

    

  
    
   
   
  

  

 

 
 

 

 

 

TERZO CANTARE: 157

@RPASATTO. Huomo vile, mal fatto, scimunito, ¢ da poco; che i Lati-
ni dicono Vappa, Cerdo, ¢ simili, come si vede in Plauto da noi in quelto pro-
> citato C. 6. stan. 98. E questo nome d' Arfafatto viene da Arfaxad
della scrittura fagra, che nel barbaro secolo non essendo dal volgo inicio, fu
reso per und Habbaleo,o Habbano. —-

Dl quei che pisciano nel Cortile. Pisciar nel Cortile vuol dire Far la spia, ¢ que-
flo, perch coloro, che fanno la spia, cfiendo veduti entrare, ¢ uscire del Pa-
Tazzo della Giuflizia, hawno qualche roflore, e però eflendo veduti da alcuno
lor conoscente, si fermano nel cortile di detto palazzo a pisciare per scufa. Si
pud anche dire, che il verbo pi/ciare sia prefo in significato di buttar fuori, ed in-
tendere che pifeino, cioé buttino fuora quello che sanno nel Cortile della Giutti-
zia, ove è la Cancelleria del Bargello, nella quale le spie portano le denunzie.
Si pud anche far refleffione, che detto Cortile fla fempre pieno di Sbirri, i qua-
Ji fon' anche per lo più spic, e vi sono due pisciatoi spefissimo adoprati da loro,
ed intendere, che venga da questo il detto Pisciar nel Cortile. Ma fiacome efler
Gi voglia, l'effetto ¢, che pisciar nel Corriles' intende comunemente, Far la spia.

CAMPO oftile. Campo nimico, Dice che ¢ campo oftile, perché ofta; ¢ fa.
na(cere il bifticcio dalla parola offile, ¢ dalla parola cosa, la quale nel parlare>
pare che dica che v/a, che vuol dire s' oppone, ¢ fa oftacolo, facendola di duc
dizioni, cioè che, ed offa, quando è Puna sola, cioé coffe dal verbo cofare 5 cho
vuol dire Esser manifefto. Modo usato da Franc, Harbarino ac' Mottetti,

NON vengon quafsh per pigliar' aria. Vengon per altro fine, che per andare a
spatio, o pigliare aria. Detto usatissimo per intendere uno, che vada forto al-
ti pera in qualche luogo, ¢ sia poi per negozio importante, ¢ per cavar uti-
le da quella gita; che i latini dissero: Won fine ratione lupus ad urbem. E noi pu-
Fediciamo: OQusfta cosa non ¢ fatta fine quare. Vedi sorto C. g. stan. 11,

CONOSCE ii pel nell vove. E' fagace, ¢ aftuto, ¢ fa considerare ogni minuzia:
forse è quello, che i Latinidissero: Ventura per dioptram pro/picit,

A furia di spintoni, Con quantita grande, ¢ spelsa di spiate, che tale è la for-
za della parola faria in questi termini forse dal Greco Phura, che vuol dir' abbon-
danza, © moltitudine, Vedi orto C. 9, stan. 49.

LMBROGLIAR la Spagna, Quand'uno s' aftatica con chiacchiere fuor di pro-
posico per divertire uno dal priacipiato discorso, per non gli dire quel che egli
vorr fapere, 0 non fare quel che cgli ¢ imposto diciamo; Egdé tmbreglia la,
Spagna,

PERRAR 2 uscio in Ju te calcagna, Vuol dir Scrrar'uno fuori della porta. E? il
contrario di dare dell impo/ta ful moftaccia, che vedremo foto C. 10, stan. 27., che
vuol dir proibire l'ingretio a uno che venga per entrare; € quello vuol dire Ob-

biigar uno a uscire.
STANZA LL
Sperance refta alla Regina intorne La pala nella defira tien del forne
Spianator di pan tondo riformato; Wella finiftra un bel teglion marmato
Gridan elle remo,e Livorno, dn cambio di rotella, chagli guarda
Ed ha un Co. che pare up vicinate 5 Da j colpe il magaxrin della mofrarda,

STAN-

 

 

 
 

 

  
   
 
 
 
 
 
 
 
 
 
 
 
 
 
 
 
 

158 MALMANTILE ~
STANZA LIL '4 STAN
De i Rovinati anch' ei pafso la barca,;
Perché la gola,il ginoco,e il ben veftire
Gii baveano il pane, la farina,e Parca
Jn fumo fatto andar come elifire 5
Tal che,cantando poi,come il Petrarcay
a2 CAmore io fallo, € vecgio il mio fallire,
Ail ginoco del barone, ¢ alla baffetta
Giocava,apparecchiando alla Crocetta,
STANZA LIII,
Fu dalle dame amato in generale,
(4 dico dalle prime della perza)

Poi Bertinella ffavane si male, Gi dal usizio,e
Ch' ella fece per ini del ben bellezza, Crn la solita faa p
Perché [pefa la rola, e concra male, Perch sin quéfto cafo\aleun
Fatta più bolfa a! una pera TUERZA y Siscnopre, facil sia, farie pri

Potea dt notte, quanto a mezzo giorno, dccid ful lerto poi di B. «ch
eAndar ficura per la fava al forno. Se gli facia ferrare il m
Partito Piacciantco refta appreflo Bertinelia Sperante; questo era B
fai comodo; ma tra il suo mandar male >» ctra l'effergli stata fara
tega, si ridufle anch' egli malissimo, ¢ nondimeno non usciva tai di
retrici, dalle quali yeramente cavaya il yitto, perché essendo bell” hi
efle amato, ¢ se ne servivano per bravo, e per ogni occorrenza loro:
sto il Poeta lo fa consiglicro, ¢ Bargello di Bertinella,

SPERANTE. Così veramente haveva nome coftui, ¢ faceya il
Fornaio, ¢ però dice Spianator di pan tondo: E lo dice riformato, p
bito a quei rempi il fare i tondo (che così si chiama ij pil n
faccia in Firenze per il pubblico ) in rgnardo dell' appalto, che fu |
sta forta pane; ¢ però gli conuenhe ferrare la bottega, Ci è però.anel
zo dell' equivoco, perch /pianarore di pane vuol dire Colui che fa il
significa ancora uno, che mangi molto pane. Vedi forto C. 6, sta
si pud intendere gran mangiatore di pan tondo, ma riformato;
pud pil mangiar tanto, per non havere il modo da comprarlo
mine militare, ¢ s' intende quel soldato, che è privato della
vea; che si chiama poi Vfziale riformato, ' 2 14g

GRID AN le spalle sue remo, e Livorno, Ha spalle cosh grandi,
'rate a Livorno per mettere a un remo di galera. Questogridare, ec, &
dire, che ha lo stesso significato, che Chiamar di id da'monti, Visto sopra (

Van C..,. che pare un vicinats. Haun C..:. grande quanto uaa
Tperbole usatissima per denotare un federe eflremamente grande j-¢
intendiamo una contrada, - AS: ula

TEGLIA marmata, Coperchio fatto di marmo minutamente pefto,, ¢ ter
col quale, sendo infuocato, si cuoprono le teglie, © tey er rololare le
de: edé forse il Latino clibanus; che per altro yuo) d a
cotto, se crediamo a Pietro Viloa Vita di Carlo V.

 
    
 

  

 
 
 

 
 

 
  
   

 
 
    
    
 
  
      
        
    
   

 

   
  

   
    

   

 

  
 

TERZO CANTARE: 159
IL midgaezino della moffarda. Cioé il ventre. AsoParda & uno intingolo fatto
y ofto cotto, ¢ fenapa 2 ¢¢. ma qui ¢ prefa ( come da molti ) per quella roba.,
a che fla nel ventre per qualche similitudine che ha quell' escremento col colores
della ¢ magazzimo diciamo una stanza destinata a riporui,¢ conseruar.
vi Spagna. almazén.

— PASSO' ta barca de' rovinati. EF nel numero de' poveri.
is ARCA, Voce latina, che vuol dir Caffa in generale, ma noi intendiamo spe-
at

  

 

cialmente quella gran madia, entro alla quale  Fornai tengono il pane cotto, 0

FATTO andar' in fumo d' elifire. Fatto andar male fenz' alcun frutto appun-
to come fa l'elixire, che lasciato in un vafo aperto fuapora, ¢ si disperde.
nem AL Barone, ¢ alla Bafferra, Sono due giuochi noti, i primo di dadi, e l'altro
16% — di carte; ma qui scherzando vuol dire, che cra divenuto Barone, cio' mal velti-
i to, guidone, € ridotto al baffo, che vuol dire Impoverito; traslato dalla botte,
van che si dice efer' al bao quando il vino che v' ¢ dentro è alla fine, ¢ che la botte ¢

| quafi vota.
. APP ARECCHIA alla crocetta, Vuol dir non haver da mangiare. Far degli
(ee — sbavigli significa non haver da mangiare. Vedi sotto C. 4. stan. ultima. Ed ef-

 

wil' sendo i¢ di molti nello sbavigliare farsi 1a croce col dito pollice incontro }
uk! alle fauci, pero far le crocette intendiamo stare a bocca aperta, e vota, che in fu-
ii@ — stanza vuol dire non haver da mangiare, Qui il Poeta rende il detto più oscuro,

1? = € pill coperto dicendo apparecchia alla crocetta, che & un Conuento di Monache,
iB nel qual uogo par che voglia dire, che coftui defini, ¢ ceni: che questo significa
il verbo apparecchiare, quando ¢ meffo affolutamente, ¢ senza aggiunta.
ie PRIME dela pexxa. E' Jo feifo che di prima Claffe, o paffar per la maggiore
detto sopra C, 1. stan. 6. p
ST AVANE male, Tribolava per |! amore, che gli portava, Era grandemen-
4 te innamorata di lui, Latino deperibar.
a FECE del ben bellezza. Cioé spefe, ¢ confumd, quanto ella havea, Havendo
4 confumato tutto il suo bene, le rimate folo la bellezza, o vero fece bellezza, ed
nf? ene ogni suo havere. E' quel Procerusam facere, che vedemmo sopra C, 1,
iat D,
2 BOLS A. Mal fana per troppa umidita, ¢ ripienezza. E perché questi tali
1i@ elf foglion esser per lo più ripieni di carne liquida, ¢ di colore fra il verde, ¢ il
giallo, gli paragoniamo a una pera troppo matura, 0 fracida, che questo vuol
ol dire pera mezza. Virg. mitia poma; ciok maturi,

we POT EVA andar ficura, ec. Quelto si dice d' una donna vecchia, € brutta, in-
jc tendendo, che ella ¢ ficura di non esser rapita.
a LEZZO, Puzzo., Fetore, Propriamente /ezzo ¢ un' odore che dispiace, il

pi quale non nasce da corpo corrotto, come ¢ quel puzzo,, che nasce da una carne
troppo frolla, 0 altra cosa marcia, o fracida, che si dice stantia; ma ¢ odores
yt Raturale, © procede da fudore  o da altra evaporazione, che getra un corpo,
nt beaché non sia corrotio, onde quello che si fente dal becco, ¢ dalla capra vivi, si
ya!  dice lezzo, e quella che si fente ca i medesimi quando fon morti, € corrotil si di
| ce puzzo 0 fetore, 0 fito di stantio. Vedi sopra in gucfto C, fan. 24. Queite
a.

 
——

|
4

 

 
 

16° MALMANTILE,,

Jezzo,così dda olezxo,¢proprio quello, chei L.dicono Virus,Not
veleno, morbo, ferore,, he 1 ¢ simili pigliando l'uno per l'al

che l'altro ¢ vocabolo di mezzo, perché tutti si poslono
re, come si cava da Caio Lurisconfiulto: Qui igitur ( dice egli )
ber adijcere utrum bonum, an malum. E Statio lib, 2, Syluarum:Atque
Virus, odoriferis eArabum; quod crescit in aruis, Noi ancora diciamo
purzo di muschio; [a dé mnschia ch' egli avvelena. Gls ammorba d'ambra
to ch' egli attoffica, ec. sue

PASCIONA. Intende Comoditi,c abbondanza d' ogni cosa neceffari
to, se ben pa/ciona vuol propriamente dire Il pascolo delle beflie.

N'IMP.AZZA affatto. E! di tal maniera innamorata di lui, che ha
ilcervello. L, efflittim, perdite amat,

NON (a vede a mezzo. Non gode la vifta di lui alla meta di quello,
rebbe; termine, col quale s' clprime l affetto grandissimo, che uno
un' altro, Won veder più avanti; ne pik qua, ne pin la; usd il Boce,

SALAMISTRA, Maeltra di fala. Ma iol intendiamo una donna
dottorefla, affannona, ¢ simili, ma per derifione, diciamo ALadonna Sa
Qui intende direttrice del governo; ¢ la chiama Sa/amiffra pur per di

V-A in capo as liftra, Cioè toltone Bertinella, ¢ Martinazza eglié il
il primo huomo che sia in Malmantile,

£ DI nidio, E' trifto, E' aftuto fino dalla culla., e4 incunabulis

Noi pigliamo questo detto da gli uccelli cavati dal nidio, ed allevati
V uccellatura fon fempre migliori, che i preficel,

NAVICELLO, Vuol dir huomo lefto, ¢ che fa tutte le furberie', che
sa navigare a tuttii venti. Ha lo stesso significato che esser di nidio.

JL letto di balocchino, 5' intende le forche. Da un tale detto Balo
fu impiccato in Firenze al Canto alle rondini per ladro di beftie, delle
Senfale, ¢ si chiamd anche il Parola. Vedi sotto C, 6, stan. 67.

" SERRARE il nottolino. Vuol dire strozzare: intendendofi per Nottolit
parte della canna della gola, che vulgarmente chiamiamo gorgorzule, €Qe
la similirudine, che ha nell' andare in gii,e in fu,quando ' ing hiotti ce

   
  
  
  
  
   
  
   
   
 
    
   
    
    
   
 
   
   
 
   
  
 

     

in git, ¢ in fn delle nowole da ferrar porte, ec. 2
STANZA LVL ee
Fa in tanto nel Castel tocear la cafsa, Ch' in fretta alla raffegna se
” Einalberar U infegna del Carreccio, Con le schiere pero fatte a bal
E comandante elegge della malfa Che ad una ad una
U nobil Cavalier Mafo di Caccio, Sotto /ua guida,e [orto sua t

Bertinella fa toccar tamburo, ¢ inalberar l'iniegna generale, e d
nerale della sua gente Malo di Coccio, il quale subico si metee a far la
ed accomoda tutti i soldati sotto i suoi Capitani, e Comandanti.

CARROCCIO. Questo era anticamente un gran Carro di figura qu:

ra il quale s' inalberava appiccata a una grande antenna ! infegna
della Signoria di Firenze, ¢ si metteva fuori in occasione di trionfi, 0 |
Fiorenuini uscivano in campagna alla guerra con cfercito formato, ed è
flefio Carro, ¢ della stetla Sgura,e grandezza quello, sopra il quale si,
ii Palio di S, Gio; Bauita, MA

 

 

 

OS eee a a a

Ss peer
ty

E PELLESE

WREARE BEALEE

TERZOQCANTARE, 161

. &€ASO di Coccio, Tommalo di Coccio fu un Pescivendolo huomo ficro, ¢ di
gran seguito di snoi uguali, a i quali egli in tutte I occasioni di fefte, cacce, ed
altre cose simili comandaya come a' suoi servitori, ed era benissimo ubbidito da
chi.per genio, ed affetto, ¢ da chi per timore, ¢ però il Poeta lo fa Generale de'
soldati di Bertinella, che fon tutti di condizione simile a !ui, come vedremo.
Lo dice mubil Cavaliere, perché in Firenze egli era conolciuto,e nominato pitt che
qualfivoglia gran Cavaliero.

4A HABBOCCIO. In confulo,.a cafo, e senza considerazione.
STANZA LVI

Si primoé il Purba nobile Bradiere, 1 \fateude il Vecchina il gran Barbiere,
Che non ginoca alla buona,e meno a gofi, Che vnol chrogni hor fitrinchi,e si sbafof,
A noccioli bensì fifa valere E dove 4 menfa metter puo la mano,

,

Perch ci da benet buff,¢ meglio i Sofi. Si fa la fefta di San Gimignano,

Al Poeta mette in questa raficgna una mano di piebei noti per qualche loro
azione © buona, © cattiva, ¢ gli nomina con i loro soprannomi. Ii primo è il
Furba firadiere, cioé uno di coloro, che alle porte della Città cercano i patseg-
gicri se hanno reba da gabella, i quali pizzicano di spia; ma questo Furbo era
anche in effetto spia. li fecondo ¢ il Vecchina Barbiere.

ALLA buona, ed a gofi. Sono due giuochi di carte afsai noti: ma con dir così
intende, che coftui non era ne buono, cioè femplice, ne goffo, cioè corrivo.

A NOCCIOLL hen sit, Già che il Poeta porge la congiuntura di narrare, qual
sia appeelso a inoftri Ragazzi il giuoco de' noccioli, ed in quante maniere si
faccia,, il Leteore si contentera, che io spieghi con un poco di digreffione i mo-

i 5.€¢ i si traftullano i nostri Ragazzi a questo giuoco de' noccioli, ¢ non
si idegnera di volgere gli occhi a leggere il discorso di quei trattenimenti, a'qua-
Ji,non idegnd.di volger l'animo, ed impiegar l'opera un Cefare Augufto, fecon-
do che riferisce Suctonio Trang. riportato, ¢ considerato da Alex. ab Alex, dicr.
Gen. lib. 3, cap. 24. ¢ ricordandofi che tutta quest Opera è fatta per i Fanciul-
Hale » che.per quelle persone, che già reliquerunt nuces, haura la bontà di con-

»fenon per nece(saria, almeno per non affatto fuori di proposito tal di-
greflione « Dicodunque-che il ginoco, che fanno i nostri Ragazzi co' noccioli
di ( Cofiumato anche da 1 ragazzi Greci, e¢ Latini, che lo dicevano ladus
acellatarum, fecondo i| Buleng, de Lud. vererum, & Alex. ab Alex. dier. gen, lib. 3.
cap. 21, ade di cui parole poco apprefso riporteremo ) ¢ usato in molte maniere;
ma specialmente giuocano, 4 Cavalea, alle Cafelle, alla Serpe, a Ripiglino, a Shree
Seid, 4 Cavare,.aShricchi quanti, aTruccino, ed alle Buche. Di tali giuochi,e
ai ciascuno di edi narreremo ii modo, che tengono a efercitargli, ¢ diremo qua-
li Geno simili, 0,gli Mei, che erano usati da gli antichi, |

A cavaica. S' accordano due o più, ¢ tirano sopra un piano i noccioli a un,
per uno, ¢ tanti ne seguitano a tirare, ane stieno a far falire sopr' agli altri
trati un nocciolo che sopra vi refti, ¢ si regga senza toccare altro che noccioii;
€ cojui che ha tirato il nocciolo rimafto sopra, vince, ¢ leva via tutti i noccioli
tirati. Lo dicono a Cavalea da quel cavalcare, che fa il nocciolo sopr' a gli altri,

ALLE Cafelle, 0 Capannelle.. Mettono sopra ad un piano tre noccioli in trian-
golo, ¢ sopra dieii.un' alteo nocciolo, ¢ rs maiia dicono Ca/ells., 0 Capan.

nella;

 

 

 
| 2 9

* Giulio Polluce lib. 9.c. 7. moftra che faceflero questo giuoco ancora

 al quale & toccato in forte,deve,girando in rnota con quello'

    
  
  
  
   
   
    
  
   
   
    
  
   
   
   
   
 
   
   

162 MALMANTILE™

nella;¢ fatto di éffe il numero tra loro conuenuto, ed
concordata, tirano in dette Cafelle un' altro nocciolo 5 ¢ colui cl
vince cutte quelle cafelle, che fa cascare col colpo. Questo fu wi
gli antichi, ¢ dicevano Ludere Caffello nucum fecondo il Buleng. C.
selle vengono descritte da Ovidio in Nuce in quei veri M
amplins, alea tora off 5 Cum fibi fuppositis additur una tribus,
ALLA ferpe. Fanno una di dette cafelle, la quale figura po d
da quella fanno partire un filare di noccioli, che figura il refto del corp
ferpe, e poi vi trrano dentro con un' altro nocciolo, € chi fa col tiro
uno, © pili noccioli del tutto fuori del detto filare, vince tutti lino
sono dalla rotwura in git verso la coda di decta ferpe, ¢ durano così, fino ache
sia rovinata da un di loro queila cafella, che figura il capo della ferpe.
pure era usato da i Greci, ¢ Latini, ¢ forse facevano co' noccioli altre fig
come si cava dal Buleng. Cap. 8,, dove si vede, che in vece della fei
co i noccidli un triangolo equilatere, o [ come dice egli } il delta &
ARIPIGLINO, Pigliano quella quantita di noccioli, che conuel
randogli all' aria gli ripigliano con la parte della mano opposta alla
in tal' atto sopr' alla mano non refta alcun nocciolo,colui perde la gita, '
colut, che segue; ¢ così si va seguitando fino che refti sopra detto luo
mano qualche nocciolo, ¢ quelto al quale ¢ rimafto il nocciolo,dee di qui
Jo all' aria, ¢ ripigliarlo con la palma, e¢ non lo ripigliando perde la git
reflafle pi d' uno sopra alla mano, pud colui farne scalare quanti:
che ne refti uno; che se non reftafle, perde la gita. Ripigliato il
conda volta, deve coftui tirarlo all' aria, ed in quel mentre pigliare
de i noccioli cascati, ¢ con essi in mano ripigliar per aria quello che
seguendo, posa i noccioli prefi, e perde la gita; ¢ se ne ha pigliati
senza fare errori, reftano suoi, ¢ si (eguita il giuoco fino a che fiend

dissero Pentalitha, perché ulafiero di farlo con un numero det
faflolini, 0 aliofi.
SBRESCIA, E lo stesso, che ripiglino, se non che nella

vonfi ripigliare quei noccioli, che cascarono in terra la feconda volta

uno, o due per volta, ma tutti a wn tratto s il che si dice fare sb

dovene pur' uno, 0 cascandogliene, perde la gita, e così fiva seguitando,sia
uno pulitamente gli raccolga tutti. Sd

CAVARE, Infilano un nocciolo con una fetola di crine di 'eavall
ual fetola ridotta in forma di campaneila', o anelletto legano uno'!
fegnato un circolo in terra, vi mettono i noccioli, che fon d' accord

  
  
 

filato,a tal girare,buttar con esso nocciolo fuori del circolo uno y © pil
di quelli y che fon dentro al circoloy ¢ vince quelli, che cava je
the gira, tocca terra, perde la gita; ma guadagna i noccioli eavati, eda
ciolo da girare a un' altro. E così si va seguitando fino a ¢he fien
noccioli, Similmente nel giuoco detto da' Greci Eis amillan delerit
¢ehio, dentro 'l quale però si doveva buttarel' aliosso.in mani

  

 

 

ian ite deen cna:

2 ante

ASL Be 2 62g & 2 eo pop
i
i
a
y
it
=
:
i'
re
:
'
of
i

a

as

nll

ad

vi

&

“4h!

va
a
yi

v
è
a

%
9

wiceeimidi —

TERZO CANTARE. ¥63
se,¢ non ulcisse di detto cerchio. Appreffo di noi anche negli Alioffi si fa aca.
vare, Canti alcialeschi; Perch' al cavare un' elioffe bruito, ec,

 SBRICCHI quanti, Occultano dentro al pugno, o dentro ad-ambe le mani
ita ioli, che vogliono, poi domandando ad altri, che indo-
vinino ni e'noccicli occultati, ed indovinandolo vince tutto, se no; de-
ve dare quel numero di moccioli, che ha detto di pil, 0 di meno; E questo si fa
una uno, dovendo il primo, che domando.far' anch' egli domandare,
¢ cosifi va continuando i giuoco.. Quelto sbricchi quanti & lo stesso, che pari, o
casto, nel si domanda, se il numero è pari, 0 caffo, ¢ chis' appone vince
tutti li noceioli occultati; se no, perde altretcanta somma. I Latini differo: /u-
dere par impar. LGreci artiazcin, Di questo giuoco parla Giulio Polluce sopra
citato, ed il Meurfio de /adts vererum, i quali moftrano, che si faceva, comes
pure oggi si facon i danari, econ altra materia, come mandorie, ¢ simili, at-
ta a:poterfi accomodare dentro alle mani, Ovidio in Nuce. Ef etiam par fit nu-
merus qui dicat.,.animpar Vt divinatas axferat augur opes.

A TKYCCINO. Vno tira un nocciolo in terra, ¢l' altro tira un nocciolo a,
quello, che @ in terra, ¢ cogliendolo vince, se no; quello, che tird in terra il
primo, raccoglie il suo nocciolo, ¢ Jo tira a quello, che tird ! avversario, ¢ così
continovano., ¢ chi coglie vince il nocciolo che coglie, 0 quello che fieno conue-
nuti. Ex simile al ginoco detto da'Greci Streprinda.

ALLE buche. Fanno diverse buche in terra in giro, formandone come unas
rofa, nelle quali tirano i noccioli, ¢ colui vince, che entra in una di dette buche,
quella somma, che ¢ prezzata quella buca,nella quale entrd il suo nocciolo: per
efempio le buche sono fette, la prima che ¢ volta verso donde si tira, che è la pil
facile a entrarui non fa vincere,non efiendo ta(sata in cosa alcuna, ¢ da i aoftri
fagazzié detta la buca del Niffo ( forse da wibil ) E dell' alere una vince tre, una
quattroyec. EB percid ho detto, che vince chi v' entra quanto è preazata la buca,
€ poi va.con gli altri ad aiutar condurre il nocciolo nella buca a colui, che al pri-
mo tiro non v' entro, € spingendolo di dove ¢ alla volta delle buche col dito in-
dice ( che dicono limare ). Ovidio ut pronas di

igizo bifue femelue ee > 0 col buf-
fare, 0 col foffiare nel nocciolo,¢ ¢ la differenza da buffare a fo

fiare vedremo
poranepeeliy ).nel che adoprano ogni arte per difficultare all' avversario il con-
jurre il nogciolo dentro alle dette buche; E così facendo a una volta per uno a
limare, buffare, 0 foffiare, colui vince, che ha fortuna di condurre il nocciolo
dentro a una di dette buche, ancor che il nacciolo sia degli avver(arj. Similes
al fare alle buche & quel d' Qvidiq. Vas quoque fape canum spatio diffance locatur, In
quod. milla levinnx cadat una manx, Banno quelto giuoco ancora con una palla, ¢
giuocano danari, come vedremo forto C, 8. itan, 69. alia voce Alinfo. Edé fimi-
le quello che i Greci, fecondo Giulio Poll, lib. 9. c. 7. chiamana pherinda: © se~
condo il Meurfio de Lud, Grae, alla voce Apherinda, & alla voce milla, ed il
Buleng. cap. 14. € go, Se bene tanto nell' dpberinda quanto in quello, che si chia-
mava Eis amillan; tiravano ia un circolo, ¢ non nelle buche. Alla buca bens}
tiravano in quelltaitro detto Tropa, che corrispondeva a questo nottro. Conchiu-
do we » che la maggior parte di detti giuochi crano usati anche da gli an-
tichi; & se ben pare, che si servitiero delic ae > 10 non soa lontano dal crede-

2 te,

 

 
we

   
  
 
 

ey MALMANTILE ©

re, che la parola Nwces voglia dire ogni forta di nocciolo,
lib. 15. cap. 21., dove mette in dubbio, se ie noci in:
ancora arrivate in Italia; ed oltre a questo trovone i gla
ed ardirei pero affermare, che ancor' essi adoperafsero noccioli di p
(come fanno anche i ragazzi de' nostri teaspi ) alle volte noci, ¢
cioli di pesca, seguitando Alex. ab Alex. lib, 3. c. 21.5 che dice
Gos viros super nucibus ocellatis einfmodi, qus essent, ancipitem ditt cogicationt
Se, variaque in opinione versari y © alias nuces avellanas, alios amygdalas pa
neque fatis ratam fententiam ferre super Tranquilli verbis, quibus Ang ds
animi canfa cum pueris facie liberali ocellatis nucibus lnfiffe dict.
mus, & probabilins putamns id ef: Einfmodi nuces ocellatas nucleos 5
pomis fitos inspicimus dicamus esse, quibus perfape Iudere nostrares pueros d
dittafque ocellatas propter ocellos, & foramina, quibus muniuntur undiqne
ansyedalas, aut avellana, ficut error haber 5 fed de perficorum offibus, quibus |
debatur 4°& nunc frequens puerorum Indus eff, intelligi conuenire credumus
© non umbigua fenrentie fore. Dalle quali parole s' intende, che
cora si ginocava a questo giuoco de' Noccioli, Ovidio de Nuce,ct
verita, € moftra che havefsero molti de' fuddetti giuochi, 0 poco d
Marziale attefta, che crano gli steti genj ne i fanciulli de' suoi tempi, ct
d' oggidi, ¢ che il portare in tasca noccioli causava a quelli delle maz;
segue ne i noltri, dicendo; y 4
Alea parna nuces,& non damnofa videtur;
Sape ramen pueris abjpulit illa nares
Ec altrove. Zam triffis nuctbus puer relictis
Ed Horatio, ?offquam te talos, Aule, nucefque
Ferre finu laxo vidi y ec,
Sono dunque, ¢ furono fempre puerili tutti li fuddetti giuochi; e
biamo un detto di disprezzo; Va 4 giuoca a'noccioli, che significa Tuo
gior giudizio di quel che habbia ua fanciullo: Qual detto cra usato
pure, come si cava da Perfio Sat. s.
Et nucibus facimus quscumque relittis i
E dicevano reliquit nuces d'uno, che dalla puerizia paflava a mani
rie; Dal che potrebbe argumentarsi, che 11 Poeta dicendo, che
ca bene a i noccioli, intendefle, che egli fufle huomo di poco giudizio, e cher q
nucibus imcumbat; Ma si conosce, che non intende.questo, perché prima
Non ginsca alla buona ne 4 goff, significando che non era ne buono ne
ora col dire, che egli ginoca bene a' noccioli, perchéda bene i buffi, ¢
vuol — ben la spia, che baffare, ¢ fofiare vuol dir Bar la spia
C. 1, stan. 37. 5 «
BVEPI fh. Buffo ¢ un fofiare non continuato, ma fatto:a un tratto - |
si farebbe a sputare, 0 a profferire la parola buff, donde buferd., 0 bufea un grat
nodo dj vento, che paffa prefto, Sofiodun foffiare con la bocca:tanto quanto
ud durare senza ripigliare il fiato, ¢ cid dico per moftrare la differenza'
Ee buffo, ¢ foffa; che per altro sd che fof & generico, ¢ comprende og
sompimento d' aria fatto col fiato di che che sia, dicendofi /ofiare y

 
 
   
  
 
   
 

   
   
     
 

   
 
  
 
   
   
   
   
   

    
   
     
     
   
   
 

a

Sreiteetes

Sat

3 3%

Se

cARE

aee

- TERZO CANTARE;: 165

vento, Che manda fuori il mantice, /offare fidicono i Venti, ec. Vedi sopras
C. 1. stan. 39, la voce rabbuffo,

1L Vecchina, Era tin barbiere così chiamato, il quale ogni fera andava ricer.
candoiper Postérie le conversazioni, che erano a cena, ¢ trovandone di suoi ami-
ci, con varie'chiacchiere poco a poco fenz' essere inuitato si metteva a federe,
© mangiava', € beveva quanto più poteva, ed al far de' conti fen' andava senza
feeds era comportato, è faceva il buffone; Procurava, che

conversazioni di cene si faceffero in a sia, dove apparecchiava, e prov-
vedeva assai pulicamente, ¢ bene, ¢ con spela aggiuftata faceva star bene,e avan-
zava tanca roba per' se da viver pili giorni, ¢ però dice Viol che ogn' hor si trinchi

che dal Tedesco rinchen vuol dir bere ) ¢ /7 sbafoff, cioè si mangt assai, donde:

ve un che mangia aflai: Quelte voci ha/ofia, e ha/ofione sono in ulb appref-

fo alla plebe pili bata, edi pili civili ! adoprano per (cherzo, per intendere uno

foverchiamente graflo, ¢ che mangi molre mineftre, le quali si dicono ha/offe dal
Latino vas oft, cioé Valo pieno di mineftra.

St fala fefta di San Gimignano. San Gimignano è una grofsa Terra del Domi-
nio Fiorentino nel Vescovado Volterrano; ¢ la principale, ¢ più folenne fella,
che si faccia in questa Terca & di Santa Fine, la qual Santa fu di quel luogo: E
dicendofi far la feftn-di S, Gimignano' intende si fa fine; € qui ale esprimeres,
che questo Barbitre dava fine a ogni cosa, che veniva in fu la menfa.

' 3 TA

NZA LVIIL
Dalle freddeacqie il Mulaifanti approda Co i pescatoré al Mula hora # accoda
A. id snilitar fra fronde,e frasche, DimeoT receon de ghioxzi,e delle lasche;
Ata nobil bardarura tina in broda Pericol pallerino ancl? ei ne mette
Divcedri edi ciriege d' amarasche, Dugento suoi armati di raccherte

4L mula dalle fredde acque, Pu uno che nel tempo di state vendeva l'acque diac-
kanes a Pare che questo Mula sia un gran sig.\ di lontani paeGi
evicino al Mar gelato, di dove approdi alla spiaggia del mare; ma approda,cioè
s' accoffa alvreftante dell' armata di Bertinella. Dice fra frondi, ¢ fra/che,perché
questitali veaditori d' acque diacciate fogliono per all ornare le loro

di verzure, fiori, ¢ frasche.

8' ACCOD A. Seguita, o vien dietro immediatamente. Quafi ad caudam ire,
Noi wliamo questo verbo per 1e beftic da foma, che seguitando in viaggio Yuna.
I altra viene alla prima legata la feconda, alla feconda !a terza, ¢c, cons
la cavezza alla groppa dell' antecedente, ¢ così chi seguita va con la testa vici-
na alla coda di efla, ¢ quelto si dice accodare, benitiimo usato qui dal Poeta,
per il Mula, fendo che a i muli pil, che ad ogni altra beftia segue questo acco-
dare.

DOMMEO. EF' una parola fola, ¢ dourcbbe dire Dommeone, che così cras
chiamato un venditore di pesce, ¢ falumi, il quale era amato da rutti i ghiotti
di Firenze, perché vendeva fempre il miglior pesce,, che veaiffe in mercato, ed i
giorni di geaflo haveva fempre qualche ee » © ghiortornia singolare. £
pero lo chiama treccone, che vuol dire Rivendugliolo, cioè rivendicore di cole»
commettibili di poco prezzo [ che si dice anche barnllo } forse dal Latino ¢rice,
bagattelle, cose di poca stima, ¢ di vil pregio; Marziale, Sunt aping, triceque y

of

 

 
  
      
    
   

166 MALMANTIEB ¢

& si quid vilius iis, Dice di ghioxzi, e di lafebe ( duc specie di '
per intendere, che yendefie sokamente questi, ma per moftrare
pesce in generale. phi thad t
PERICOLO. Quefio fu un tale Alcflandso Violani detto.
nato per il suo gran valore nell' abbaco, come diremo foro.C, 4
perché egli cra anche bravissimo giuocatore di Palla a corda,e 2
po a fitto una di quelle Rtanze dove si giuoca a tal giuoco, if 1
armate di racchette, 0 daccheste, che sono meftolescon le quali si giuo
a corda, ¢ sono composte d! un cerchio di legno col manico, ¢d il ya
no d' una rete fatta di grofia minugia: per /accherea intendiamo anche.
di dietro del porco, ¢ del castrato; Non (0 già se la /acchetea da ginocare,
nome da guefta, 0 questa da quella, so ben che si chiamano cogil' une 5 ¢
er la similitudine, che è fra di loro della figura. Questa da gi r

 

tini detta reticu/um da quella rete, della quale € composta, come si ¢a.
Ovidio: Reticwlogue pile leves fundantur aperto. Vedi foro C. et
at

   
  
  

viamo per mapdare a casa le robe commeftibili, che si comprano in
vecchio, ¢ ci servono ancera per Quochi. Cofloro fon per lp più
e Cantoni Suizzeri, ¢ dimorando in Firenze foglion far camerata co i
che vendono i tartufi, ¢ per questo dice che egli conduce Norcia, ¢ la Vallatas ®
perché egli era hvomo pulitissimo,gli fa per (oprayyefta un grembiule candido »
come veramente cgil fempre portava. Z SIWOET
GIANNETT A, onde Giannettina; specie d' arme in asta, nella guerra wat
da gii alfieri, Ginera in Spago. ¢ una piccola lancia; corsesca. jo amet
PENNACCHIO, S' intende una quantita di penne di Struzzolo; ma coftul
I havea di Cappone come trofeo di Googe. 5 BOE
Z#N4. Specie di panicre senza manico composto di strisce di Jegno gentile
eda tale Zana coftoro fon detti Zanaro/s. Di questi tali il Poeta fa Capitano!
licche, perché in vero egli era riverito da efi, coe. quelli che nel loro. ?
} havevano veduto efercitare Cariche riguardevoli, ¢ fapevano,, che era d
reputati delja sua patria, dalla quale era in quei see «dhe
SGARVGLIA. Fu un Battilano assai celebre, ¢ fra i (uoi pari Capopolo, ©
da coflui quando in commedia ¢ stato introdetro il Battilano }' hanno:

rola Pijlotta. ol
; STANZA LIX, STANZA L

Melicche quoco all' ordine s° apprefia, L' unto Sgaruglia con frittelle a tof
Per giannettina bain mano unoftidione, Alla [quadra de Qaochi-hora fogging
Ed un pasticcio per vifiera in testa Lucha de' Battilani assai
Con pennacchio di penne di cappone Genre che a bere ¢ peggio
Vn candido grembinl per sopravvcfta A cut battiers (diceva, )
Gii adornailc..,¢l'nno,el'altroarnione, Ch' affeddeddieci 1a dove si git
Vina zana è il [uo scudo, e nel? armata Noi non habbiamoa se i
Conduce tutta Norcia, ¢ la vallata, 4Ma-s ha a far fempre la ¥
Segue Melicche Zanaiuolo di Mercato vecchio, uno di coloro, de' quali (¢

 

Sgaruglia. Questi condnce la schiera de' Battilani, che dice famo/«, ¢! ee
wi

gf

 

do con l'equivoco, vuol dire Attamata, da Fame, ¢ non $4 Sensis
 =

 

 
 

BeSakeswFe gs

=a

TERZO CANTARE. 164

| PRITTELLE. Cosichiamiamo una vivanda facta di pasta quafi liquida feieta
nell' olio da i Latini detta 4rro/aganus; ¢ si come essi mescolavano con detta pa-
sia latte, ed altro, così noi pure vi mettiamo delle mele affettate, uva feccas,
latte, rifo, erbe, ed altro fecondo i gusti. 1 nostri contadini nel tempo, che fan-
no ¥ olio costumano di far molre di tali frittelle, indotti a cid da havere olio ia
-abbondanza, ¢ ne danno anche a i vicini, e parenti; sono però soliti coloro, che
'vanno a veder lavorare, chiedere le frictelle, ed i lavoranti con poca grazia, ¢
“meno diferezione spruzzano Polio addosso a quel tale dicendo: Eccoti le frittelle.
-E da quetto forse per frirte/le intendiamo macchie, che vuol dire Ogni fegao, o
“tintura, che sia nella superticie d' un corpo diversa dal proprio colore di quel tal

corpo, come ieee > quando l'olio casca sopra ad un punno. Ed il Poeta dicen-
“do, che coftui molte frittelle, intende, che egli era alfai unto, come fempre
*sono i Bactilani per il continuo maneggiare olio, ¢ lane unte,

A IOSA, In quantita grande. Diciamo nel medesimo signifitato a cafifo,in.
chiocca,a biftia, a fufone, voce usata da Giovanni Villani, a similitudine della.
Franzele 4 foifon, cioè con effutione, senza risparmio, 4 furore, 4 precipizio, a»

“bi > 4 Wome, ¢ simili, Che se bene fon modi baifi, nondimeno sono tuluolta
usati anche fra la gente civile. E questo a 4o/a credo sia parola corrotta, e che
doveffe dire a chiofa, che significa quelle cappelle, che hanno le bullette, ¢J
'ogni piccola piaftra di piombo, di rame, o d? ottone ridotta tonda, ¢ simil »
'alle nostre monete, delle yuali chiofe i nostri ragazzi si (eruono per giuocare alla
«trotrola ta vece di monete, ¢ però chio/a s' intende per moneta di niua valore;
Hi Perfiani disse:

ail © * Ma vin tafea non ho pure wia chiofa

A mantenermi, in tanto qua pars eff ?

Siche dicendofi: Delia tal mercanzia ue n' era a Fof4, 0 a chio/a s intender y
the di quella mercanzia ve n'era così grande abbondanza, ¢ per questo era a così
vil prezzo, che se n' haveva fino per una chiofa, Ii Berni nel suo Capitolo ia le-
de de' Ghiozzi dilic - 3 i

Segue da'que/to un' altra disciplina,
Che havend' ingerno, ¢ del ceruello 4 iofa,
A ' -  Bifogna che v' habbiate gran dottrina,
H Domenithi in lode della Zuppa.
an E'quincs vien, ch' ella si fuol gradire
Da chiha ceruello, ed intellerto a iofa,:
'vote vhio/# per similitudine significa ancora le Crofte delle bolle, E vuol

anche dire E(posizione, o comento, forse dal latino greto Glofa.. Dante num,2,
Purg. C, 11,:

E ferbolo # chiofar con altro refto,
E nelInhC.2y.disse Paranno s) the tu porrai chiofarlo,
Hi Varchi nel Capitolo dell' uova sode dice: 3
Es io fuffi Dottor, consiglieret 5
'Che sopra questo si dovelfe fare
“| Leagi, e statuti, e pos oli chioferet.
© PEGGIO delle spugne, Succia id vino più che aon farébbe uaa spugaa; cide

deve

 

 
168 MALMANTILE®S o@

beve affaissimo, come veramente fanno i Battilani, i quali chi f
pra in questo C, stan, 8. ' subi
BATTER la Calcofa. Frafe Furbesca, che vuol dir batter la firad:
€ questo parlar furbesco è praticato assai da ie forta di gente.
AFFEDDEDDIECT, Giuro proprio de' Battilani profferit
una fola parola con due ff, ¢ quattro d. i Bargilani
¢ sono molte persone a lavorare, hanno ogni dieci huomini un.
chiamano il Capo dieci, che ¢ da loro ubbidito, ¢ Mimato, ¢ per
se del Dicci, intendendo di coftui, flimano di fare yn gidramento. fo)
Credo mondimeno che dicano a se de Dieci per non dire a se di Dio,
dicono per Dianora, Corpo di Dianora per la medesima ragione.
SCARD ASS AR /a lana, Cioè pettinare la lana con ini
no Cardi, perché hanno i denti torti, ¢ simili a i
foglie, il fulo, ed il fiore deli erba detta cardo, del qual fiore
si servono per pettinare, ed unire il peio.de i pauni, ¢ pero lo
ed ¢ i) latino car minare. Vedi sotto C. 7. stan. 37.: 3
FAR la lunediana, Apprefio a i batiilani significa non lavorare;.¢ que
ché nel tempo, che l'arte della Jana Javorava, coftoro guadagnayano. lai, ed
erano pagati dalli loro maeftri il lunedi, dove gli altri oo i
fabato, ¢ però questo giorno del lunedi,cfiendo per loro giorne d'.
la riscoffione, era da essi Jolennizzato, ¢ non voleyano lavorare, (ma fl
fefta) a confumare in bere, ed in mangiare quel denaro, che hayevano'
e guefta loro folennita chiamavano Lunediana, cd alle volte Lunigian 15 ed
da essi tal fefta così offeruaza, che tra loro era la seguente cantilena,

 
  
   
 
 
    
   
 
 
        
  
 
 
   

  

  
 
 

 
   
 
 

Chi non fa la lunediana, on entngaite

E' un gran figlio di puttana, z ae
Ed oltre a questa ce n' è un' altra che dice me

UVenerd: de Beccai, 4 if

Mt fabato de gli Ebrei, r

La Domenica de' Criftiani, TE

E il lunedt de i Battilani, '

Si che dicendo /unediana s' intende fella, come si yede nel presente
che Sgaruglia dicendo »° ha afar Sempre la Lunediana, ¢c, intende hada 7
pre fefla. Questo nome di Lunediana refta ancor' hogsi » ma come che i

'orza flare alle volte le See

Jani sono pochi, ed i lavori meno, conuien loro per
timane intere senza lavorare, ¢ Così non € mefla troppo in plo detta fo
anzi hanno di grazia, lavorare anche il lunedi,

, ' Stren ZA oe

   

Conchino di Melone ecco s' affaccia, Che tutti allegris¢ rubicondi in
Che ? Offersa tenendo de gli alloré Cantando nua canzone A'
Col fineye aldo d'un buon pro vi faccta Di gran coltellise di ragliers arpa
Ha dato un frego a tutti s debitori, Si fon per amor [uo fatti soldati,
ue Conchino di Melone, il quale si.conduce dictro una mano de* (oi dh

ae che si fon fatti soldati per 1a cortefia, che ha fatto loro di sc
ti il debito, che havevano seco, fu eoflui già quoco d' Ofteric, ¢ per eli }

 

 

ee. Se ee

ae
 

“ee Tt. 4
TERIZO CANTARE: 169

BR to gtaffo, edi flatura piccolo fu chiamato Conchino} gli venne voglia Ui diventar

.  macftro, onde prefe sopra di se un' Ofteria detta ii allori, dove subito hebbe»

am = molti i, ma tutti a credenza, per Jo che prefto falli; e non trovando mo-

 

do di rifquotere un soldo gli venne rabbia, ed abbrucid i libri per-non haver di
pitt paflione di vedere scritti i suoi denari, ¢ non gli potere spendere. E
intende dicendo ¢ol fine, ¢/aldo d! un buow provi facia ha dato frego a tutti è

» S*e4 SP ACCIA, Si fa innanzi.. L' Autore si serve di questo verbo afacciarf,
per denotare, che coftui havea la faccia larga; (cherzo assai praticato con uno,
cre habbia gran ceffo dicendolegli afacciarevi, facciami favore, facciami buon vifo,
efimili.

TAGLIERE » Intendiamo un' arnefe da cucina,fatto di legno,tondo a foggia
di piatto per uso d' affettare sopra di efio carne, ¢ per triturarla con quei gran,
coltelli', e farne polpette, 0 altri batcuti. I Tedeschi usano in molti luoghi 1 piat-
ti da tavola fatti di legno, ¢ gli chiamano T-alier con voce venuta d'Italia, come
si pud eredere; gid che i nostri antichi i piatcelli, © tondini dal tagliarui fu le»
vivande, domandavano taglieri, onde il proverbio-. Due ghiorti a un ragliere,cioè
4 uno fheffa piatto'. Trovali questa voce nella antica lingua Gallefe, o Francesca;
¢ dicevano railfeor; come leggefi in un' antichissimo libro in quella lingua,dal Lat.
volgarizzato, appellato de] Conquijfo della terra Santa di Gerufalemme, i) quale fié
ritrovato essere di Guglielmo Arcive(covo di Tiro; ¢ si conserua nella preziofiti-
ma libreriadi Manoferitti del Serenifs. Gran Duca, appreffo alla Chiela, ¢ Col-
legiata di S, Lorenzo. 1) paffo tutto: volrato in Tofeano dice così; La dentrofin

irea') fu'trovato un vafello di pietra verde, e chiara aflai di troppo gran.
belta, fatto così, come un tagliere. Li Genovedi pensarono, che cid fuffe uno
fmeraldo, Percid lo prenderono a lor parte, de) guadagno della Città per trop-
po gran somma d'avere. Portaronncio in lor Città, ¢ ? appefero nella Maftra
Chica, ove egli'¢ ancora. L' huomo vi mette la cenere, che si prende il primo
giorno di Quarefima., ¢ si moftra altres! come ricchissima cosa, Perché ¢' dicono
veracemente, ch'egli ¢di smeraldo. Nel margine vi ¢ questa postilla in nostras
lingua, ido\, ¢ dove ¢' Genavefi guadagnano el catino di fmeraldo, che ten.
gono ancor'

gio Criflo Giesi alla gran cena.
STANZA LXII.

Scarnecchia che di guerraé un ver copidioy
L! Eroe degli arcibravt, ¢ dico poco,
A cui dourebbe dar piatto, e stipendio
Chiungue governa in qualfivoglia loco,
Percht quando seguiffe qualche incendio
Ei fa il rimedio per guarir dal fuoco,
Mena gente avanzata a mitre,e gogne,
Da vender Siabeschiacchiere,e menzogne,

i nel monte di S. Giorgio, ¢ credefi, che sia i piatte, dove man-

STANZA LXIII.

Rofaccio con aleissime parole

Movendo il pie yacconta,c 4 pigione 5

Fa per quel mefe dar la casa al Sole,

E nel zodiaco alloga lo Scorpione;

Cusi shallando simil ciance, ¢ fole

Si tira dictro-un nugol di persone,

Fa per imprefa in mezzo all internallo

Di due sue corna un giobe di crispalio.

Seguita Searnecchia. Questo fu un Montambanco 0 Ciarlatano, il quale ven-
deva unguento per medicare scottacure, ¢ montava in palco sempre in abito da
Coviello col nome di Capitano Scarnecchia,. faceva una mano di braverie a

' fine

 
 

     
   
 
    
 
    

170 MALMANTILE( ¢

fine di ragunate il popolo, ¢ però It Autore lo dice
de li arcibravi, B perché ¢ Ciarlatano, lo, faycapo di Monelli; © ON
alla berlina » ¢ che & buona.a vender bugie, come perlo pi sono
chi. Dice che doverebbe esser provvifionaro,; pecehe hs iene
dal fuoco le cafe, che abbrucialsero, e s(cherza, burl: Q
deva detto Scarnecchia buono.a-guarire le scostature in godin
dolo buono a rimediare a gl' incendj.

MIT RA, 0 Mitera », Diciamo)quel:foglio, chera foggiadi coront si
capo a coloro, che per delitti fon Tota © mandati in ute 'afing
C. 6, stan, soeC, 12. stan. 19

GOGNA., E' lo stesso che Berlina detto sopra C, 2. stan. ay. I aaa
no Wumelle, se ben-questa era pili toNo una specie di ceppi da serrare ip
de forse meglio con Piauto,-e.con Lucilio la chiameremo colfare. 9
FLABE, ¢ menzogne* Sinonimi, che significano Bugie. Fiaba Pe fab
menorna dal verbo mention, 1 ate
Dopo li faddetti vien Rofaecio y il quale conduce feco una. gran mano. i
ne tirate dalle sue chiacchiere. Coftui fa ung de i più superbi ciarloni
mai stato nella Ciariataneria, e spacciavafi per Attrologo. Noma
banco, ma stava a cavallo allato,a una tavola elevata, sopr' alla qual
una faragine di cartapecore di privilegi havati s diceva egl ) pee il
da i maggiori Potentati della [eon » qualche seheretro di gatro;0
sfera d' ottone, tre corni neri lunghi, ail uno de' quali era appefe unip
calamita, all' altro una palla di lumpidissimo Criftailo dil Monte, ed
corno, che cgli diceva eliere d'Vnicorno.. Vendeva una fuacmeftura:dailut chia~
mata con vocabolo Greco Wepensbes, che diceva-efler buona a rutte Pi
conforme al medicamento d' Elena chiamato con queste medesimo
penthes ( cio' di contrartoal dolore ) da\ Poeta nel4.dell' Viiflea, ed-a chi lacom-
prava donava un' anelletto d' oflo, che (pacciava' per ottimo aldol fla;
per esser fatto di dente di Cavallo marino, Diceva havere, impa;
già da un gran Mattematico, ed Alirolego suo Zio nominate Gio
cio., che predifig s vantava egli }.la rovina della palla della Cupol
di Firenze molto tempo avancd, che cella seguifie. In somma:con le
fandonie ragunava fempre, che 1 montava a cavallosinfinite persone;@
buone fomme di danari; 11 Poeta lo fa condotticre di questa ge
le chiacchiere, ¢ gli fa fare per imprefa quei wre suoi corni persiaier
di criftallo. pocikrutagt

eALT/SSIME parole... Chiama parole altifoime quelled Rofaccio; pete
fempre dilcorseva di plancti, di stelle,  d? alevevcofe:celefti.comeme
tore con dire, che egit ha affttaralacafa al Solose meffole Se
Senza ironia Dante nf..4, chiamo Virgilio; A aleiffiens Poera -By
Così vidi adunar la bela scola Di-guet Signor det atei/jima canto, O
rissime canto chiamala on gale In Otbimo;e ornarissime
cia turte le dottring,e maffimela Teologia,imperochéi primi P
- SBALLARE, Vuol Propriamente dire cava
esprimere uno che racconti moire 5 ¢ molte cofecpiut

 
   
 
  
  
    
    
  
   

 
     
  
   
 

¢} Domo

  
 

     
 

 

oe ee
 

AS

SAN

ut

SERRE

Bak

TER ZO'CANTARE: i 71
verita,ed dil medesimo, che//chianrare, che vedremo sotto C. 10. 'stan. 66. Questa
voce sballare in al ficato vedremo forto'C.'11. stan. 4.

~ CLANCE ye fole, nimi}; eP ultimo @ Sincope di favole; ed intendiamo-

chiacehiere lontane dal vero. Petrarca Sogni d'infermi, ¢ fole di Romanzs, li
Mauro jin biafimo'dell' Onore disse: a

DS Har-abdieh® ia y che le fon butte-fole,

SS uttiargumenti da ingannar gli feiocchi

HO OL 9 a Le cafe che confiftone in parole,
Ti Perfiani'in'una sua canzone dice oon,

> i Se con ragliare o fole
ity stew 'Ve pagar di-bravara

“Ottavio Pertari nelle sue Origini-dedudele parole Ciance, © Cianciare da Can-
tiones j Cantionate's It Boce. Now: 61. quando disse tla landa di donria Dfatelda', e
corali altri ciancioni volle dire senza dubbio canzoni, le quali ( perehé erano molto
in pregio le Provenziali, o:le fatte fa VY ariedi'Provenza, come si vede da alcu-
neinttolazioni'di Lande antiche*) chiama come' per iftrazio, € contraffacendo
in questo, ficome in molti altri luoghi,la pronunaia delle lingue Mraniere; cian-
ciont'; Scherzando anche nel medesimo tempo sull' altro significato, cioé di ciancta,

VN nugolo di persone. Quetta voce nugolo per Quantita grande,è afiai usatas
dainoi y el*usdiil nostro Poeta sopra-C, 1. fan. 50. Così Giuvenale Sat. 13. imi-
tando inci Omero'; chiamd la molcitudine delle combattenti griy, nubem fo.

 

 

nora ON1 s2
: an iz §$ TAN ZA LXIV.:
Sopr' un lettoriechissime fiorito E pur, vin arme ei non fu gran perito,
“Rartar: Pippa fifa del Caffigtione, Guerrier comodo almen nel padiglione,
Ove coperte fea tutto vefito, - Queffo impera dal morbido piumaccio
| Ch'in tal mado (0 foalda al suo padrone; et quelli del meftier di Michelaccio,

Seguita Pippo det Ca/tiglioni portato in un ricco letto, di dove comanda a i fol.
dati 5 the on ata get ees vo di lavorare. Coftui era il più graziofo,'e
faceto umore, che'fia mai stato in Firenze, ¢ i chiamd Pippo del Castiglioni, per-
ché servi lungo tempo a i SS. di Casa Castiglioni con fedelta indicibile, ¢ pero da'
medesimi $$, aniato a/fegno', che non oftante le burle, che in diversi tempi, ed
occasioni: faceva.a efi SS, non potettero mai mandarlo via, perch, s¢ lo licenzia-
vano', egli trovava fempre vaghe invenzioni per non fen' andare,' come fra le»
molte fu questa..: Il sig.\ Cavalier Vieri da Castiglione, al quale per ordinario fer-

-viva,lo-licenzid con queste parole: Sgombrami di Casa. Pippo andato in Piaz-

za chiamd prot Carrettai, e condottigli con le loro carrette d' avanti alla,
porta delltabicazione di essi SS. in sy Y ora, che i) sig.\ Cavalier Vieri foleva tor-
nare a definare, ordino loro, che, se il medesimo sig.\ Cavaliere gli domandaffe

quello, che facevano quivisgli rispondeflero, che ve gli haveva mandati Pippo; si
come

segui ed il Sig, Cay. disse: che hada far'Pippo delle carrette? Ed-egli a
queste parole scappato di dietro a una di esse carrette', rispose: Sgombrare, co-
me VS. Llluftrils. m' ha'comandato; Onde il Sig, Cav. ridendo della faceta in-

a amen => del- sao comandamento Jo richiamo in cala, ¢ pagati i carrettai gli

' = Ys IN

 
. be flato assai di notte. Pippo si scordo di mettere il caldanetto nel let

ip MALMANTILE

LN un letto riechissimo fiorite, HW medesimo Sig, Cay, una fera con
che facefle, che il letto fuffe caldo, quando egii tornava a dormire 5

 
 
   
 
  
   
    
     
   
  
   

tornato il Padrone, ¢ volendo andare a dormire, Pippo si trovo:ii
perché stante |' ora tardissima non vi era modo di trovar fuoco; ricorle:
solite afluzie, ¢ questa fu, che egli per la parte di dietro del letto v' ent
tro così veftito com' egli era, ed il padrone, credendo che sli andafle mo
lo scaldaletto, si poglid da per se per non lo scioperare', e spogliato a
volta del letto, ¢ difie: Cava il fuoco, ed alzata-la cortina y
vedde Pippo, che follevata alquanto la testa disse; Signore il letto non
caido a baftanza. Il sig.\ Cavaliere vedutolo così, ¢ conoscendo l'umore:
beftia fenz' alterarsi Jo fece uscire, ¢ toltafela in pace entrd nel letto così com
era. E per alludere a questa, facezia il Poeta fa venir Pippo portato in un
chiissimo letto.. ors
PiVatACCIO., Guanciale lingo quanto la larghezza del letto; della grok
za d' un facco ordinario da grano, ed ¢ ripieno di piumeye però + Pinns
cio. Qui per piumaccio intende tutto il letto. 1 ah
QELLI del mepiero di Michelaccio, Gente, che non ha voglia di
che il meftiero di Michelaccio dicono, che era mangiare, bere,e,
Qui pure bifogna, che il Lettore si contenti ch' io faccia un poco di.
ne per narrare alcune delle facezie del detto Pippo, meritando: la
cita di questo huomo, che si spenda qualche di tempo in sentire:
guzie, il quale è viffuto fino:a pochivmefi addietro d' cta di 8s. anni se
la medesima bizzarria, fauo che, dove prima frequentava molt
trovar le conversazioni, che gli pagavano Jo scotto, ( perché
quattrino,dando egli tutto quello » che guadagnava alli suoi vecchi
dre, alli quali continovo d' ubbidire come ua fanciullo fino al” eta sua di sopras
75. anui, che essi paflando cento anni, morirono ) dopo la morte del Pad
quento pil le Chiefe pregando S. D, M. per la falute del. Serenifs.: G. 'Daca » dal
quale gode fino, che wile, onorata proyifione per il buon servizio
nissima Cala. hbase
Essendo una volta il medesimo sig.\ Cav. Vieri al Poggio a Caianot
'Serenifs. G. Duca }.a scruire il Serenifs, Sig, Principe Card, Gioy a
Pippo a Firenze la vigilia del Santifs, Natale ordinandogli, che si facefle-dare dil
farto un suo veftito nuovo.,'¢ lo portaffe al Poggio., <T ordine, cheigli diedef
con.quelte parole: Va a Firenze, ¢ fasti dare dal farto il anio veftito ye portale: We.
bidi Pippo > ¢ la fera-medesima tornd col -detto veftiro del padroneun f
entrato in Chicfa do ve era tutta la Corte per-udir la Mefla-(mancandovi
sig.\ Cav. Vieri, che se ne flava in camera aspettandoiil veftitu per
veduto da wttii Cort igiani, ¢ da-tuui li SereniGs. Principi che quivi.
il sig.\ Principe Card. Gio, Carlo gli disse: sig.\ Filippo che:colaé questa? Val
fiate molto nobile ? Ed egli tispole: Screnil..queste fon graziesche mi
Padrone. & S. A. Rev. immaginandofi di come stava:il fatto si,
Pippo sil quale fatte, pity, (paiicggiate per la Chiefa. fen.andd alle stami
Padrone;, che vedutolo con quell' abito in doffo lo sgridd dicendo.; Briccone!

 
 
 
 
 

 

See Se ——— ee

 

 

eR gS ce a
Se Eatetetarkies

ae.

SShERs

54
ee

RERLRASEE | Et

&
=

SERA!

TERZO CANTARE: 473

Siam fratelli? Rispole Pippo: Perché sig.\ ? Replicd il sig.\ Cav. Che furfanteria.,
¢ la tua metterfi il mio veftito? Mi maraviglio di V. S, Lluftrils. ( foggiunfe Pippo)
non me I'ha ella donato ? Come donato ! (disse il Sig.Cav. ) Ti par' egli abito da

co eae,¢ mi fla benifiimo; E V.S. Iiluftri, medesima a
ha detto., che io me lo

cia dare dal sarto, ¢ Jo porti, ed ecco ch' io 1" ubbidi-
sco, già tutta la Corte ha saputo questa generosita di V. S, llluftrifs., ¢ si sono
rallegrati meco del:regalo, che V. S, Uluftrifs. mi ha fatco in questa folennita.
Il Sig, Cay. conofeendo, che non era svo decoro il metterfi quel veftito, che era
flato yeduto in doflo al suo servitore, stimd bene il quietarsi, ¢ fargliene un re-
galo, per-non — far' altro; Ecosi Pippo si godé quell' abito, che per la sua
ricchezzacra ite a. un Principe.
Era grande amico di Pippo il Poon Fantacci oggi vivente Rettore delia Chiela
di Varlungo fuori di Firenze circa un miglio,il qual Prete è flato fempre huomo
aflai faceto, ¢ piacevole; ¢ fra eflo, ¢ Pippo fon seguite diverse graziofe buries
¢ fra I altre il Fantacci difegno una volta di fare star Pippo senza.cena, ¢ necel-
fitarlo a.dormire all' aria; € per questo l' inuito ad andare alla sua Chiefa a Cena
ella fera appunto, che il Prete havea fermato d'eficre acena nella Villa de' SS,
Pont quivi vicina; ¢ ad effetto, che gli riu(cisse il difegnoshaveva ordinato alla.
serva che andafle a dormire a casa una sua parente, ¢ detto al Contadino, che

on alla Chiefa, che, se.fufle accaduta cosa alcuna attenente alla curayman-
7 'a oe

Prete di Rovezzano,Chiefa vicinissima a quella di Varlungo. Pippo chie-
fla,ed ottenuta licenza dal suo padrone,la (era al ferrare delle porte della Città,se
anvandd.a Varlungo, ¢ trovata ferrata la porta della Casa del Prete,.dopo haver
molto picchiato,conosciuto » che:non era veruno in casa, disperato s'accofd alla
cafa'diquel Contadino., che haveva l' erdine di mandare la gente a Rovezzano.,
eda eflo intese, che il Prete.era andato.a-cena fuor di cura, ¢ gli ordini che ha-
ea lasciato.s, Pippo accortofi molto bene, che il Prete ' haveva burlato, volles
renderglilapariglia, « percid fare trovata una scala a pivoli, con essa-montd sa-
pra il tetto della chiefa ye.quivi portata buona quantita dijpaglia, ed altro ciar-
= combultibile,¢ raro., gli dette fueco, ed andatoalle funi delle campane

meffe a fuonare a rintacchi. 11 Prete Fantacci., che era.poco lontano fentendo
fuonare a martello, st affaccid\a una fineltra -per sentire, che cosa fufle quella.,
evedutoil fudco sopr' alla sua Chiefa, tutto spaventato lascio la cena, ¢ l'alle-
gria, e-corsealla volta della sua cafas ncljla quale (ubito.entrd per-vedere doves
era il fuoco., ¢ rimediarui-can.}'aiuto d' una parte de' SS. Commenfali, ¢ con,
uina-quantita di contadini., che.già erano quivi.concorsi con zappe, ¢ pali per ro-
winare),¢ tagliare dove bifognafle. Pippo intanto scefo.dal tetto se.' andd
-ad.arno,¢-si fermo a:cena da.un tal Boni mugnaio suo.grande amico,, baltan-
doglid-havere furbata I allegria,nella.quale era.il-Prete, il quale girato ¢ oy
to.,\¢sopra..per tutta la casa.,.c non-havendo trovato ne meno-fegno di fuaco.,
fece viltase il tetto della\Chiefa,:¢ trovo:la.paglia, che era.finita d'ardere, es
vitta la-feala appoggiata alla.muraglia., s'.accorse che era Mata una contraburla.,
di-Pippo, tanto ne sche silcontadino, detto di fopia-disse haverlo.yeduto poco
prima, ¢percid (opportandofela in pazzienza,tornda ceaare, dove non man-
carono le minchionature 5x¢ barzellette y che fucono da quei SS..della conuee(a-
zione dette.al Prete. Cow-
174 MALM ANTIDE 7

Commeffe una volta Pippo non fo che mancamento, per'
volle mortificarlo col mandarlo in carcere, onde gli fece dare
un biglietto, acciò lo portafse al Segretario del Magiftrato
glietto diceva, che fulse ritenuro il Latore in fegrete fino a nuovo
prefe il viglietto, e indovinatofi del contenuto, ¢ parendogli duro
in prigione in tempo di Carnevale, ¢ sapendo, che il'non portare il ¥i
delitto da galera, andava mulinando come potefse faluare la'capra jt
quando la fortuna,nell' andar' egli come la ferpe all" incantosgli fece
nanzai un Tedesco giovanetto (eruitore di liurea del medesimo sig.\ C
Padrone, alla volta del qual Tedeseo andato Pippo » quali brava:
Padrone ¢ in collera, che tu fei flato tanto a venire', perché vole
taffi questa lettera al Sig: Segretario de gli Omo, ¢ perché én
mandava me; febene, ho da fare afsai fu in Palazzo; pigliala ye
do. Il buon Tedesco non pensando alla malizia porto la lettera\, in
degli ordini- della quale i} Tedesco latore fu ritenuto in carcere', ¢
che S. A,S. era reftata ubbidica.. Pippoil dopo definare'del medefi
vefti da donna, ¢ senza maschera con le sue propric bafette ye barba se
seggiava il corso delle maschere,havendo d' attorno un popolo infinito *
téa vedere questo tumulto i} Sereni(s. G, Duca, che pa(sava:in carroza
la firada, onde (pedi uno flafiere per intendere che cofafulse. Lo
no, dicendo che cra Pippo del Cattiglioni in'matchera da: donna 5
che già fapeva del viglictto,replico: non pud efsere, ondesil Caporal
fieri andd da per (¢, etornd replicando efser veramente Pippo nel
veva detto lo flaffiere'; in tanto S, A. S.s' accofto,¢ Pippo che gli
tro,ed hayeva ofseruato, che S.A.S,hayeva mandato due volte a veder chieglies
fattole una grandissima riverenza dilse: Seréni/s, io fon io,io fon'io,percht
wm' ha fatto il servizio di portar la lettera lui; Finalmente conosco bora
chi fifa ben volere po [perar fempre questi, e maggiori servizz).
rife dell' afluzia, ¢ ordind che fufse scarcerato 11 Tedesco.
ig. Cav. Bernardo fratello del sig.\ Cav. Vieri haveva prefa I
efta dama volendo elser servita da Pippo per bracciere,

uomo ¢' eta, ¢ veltiva'di nero, ¢ non con la liurea',come gli altri'!
quella Casa, prego il suo sig.\ Conforte, che lo chiedelse al; 'atello, perché fer-
vifsea lei, 11 sig.\ Cavaliere Vieri gli compiacque, se bene cop poco 3
perché era avvezzo a anes fuori di quelle i bizzarrie lo se: rar
e con meno gusto di Pippo, che non avvezzo a servir dame gli a
versi ad aveensate in sua vecchiaia 'e mal! volenticri lalchea il suo padroney it
diferecezza del quale non sperava trovare in chi che sia; onde prego la 4
che'lo yolefse lasciare al servizio, che era solito; ma la'Signoranom
mutatfi di proposito; per lo che Pippo sigettd alle invenzioni il
con riputazione, ¢ con operare, che la Signora Jo licénzialse,
mettelse mancamento, Chiamd dunque a sse alouni ragazziy ¢ dil
alcuni pochi soldi, impose loro, che quando lo vedevanovcon)!
dafsero tutti a gridare Pippo, Pippo, Ecco Pippo,'¢ glitacelsero
I ragazzi invitati al loro giuoco, ¢ che haurebbono-daco quaicola'

   
 
   
   
  
   
    
     
  
 
      
   

  

  

   
    
  
   
      
 
   

  

  
  
 

—. —we oe Row eee

 

 

2e 2 ow an =o SS fF ea eB H.-S ese oe

x
 

 

5

TER:Z:O0 CANITARE.
nit reloceafione di far quel chiafso, appena Jo veddero ulcir di cafaydanto il braccio
alla.Padrona,s che cominciarono a strepitare, ¢ tagunarono quivi quanta gente |
i ¢ra in quei contoral'.¢ Pippo favio, feaza mutarsi in facia seguitaya a dare il
sai) bracciovalia Signora,,.1a quale vergognandofi, che il suo servicore fae lo scicr
volun 20'del Popdlo., che egit fulse trattaco come un pubblico buifone ys) aftretto, di

giugnere in Chicla, pen(indo,, che quivi almeno dovelse fermarsi il baccano 5),
niayfe.celsd il pelacajion fini il inicia perché quei ragazai standoli cuttiat.,
» non geidavano per rispetto della Chiela, ma erano cagione y che wiro i
pepo guardalse verso quella parte; per lo che la Signora per liberarsi ordind a»
ippo, che andafse a casa,¢ mandaGe un' altro servitore, ¢ tornata poi 4 ca/a
le parue mill'anni render Pippo a chi glicl' havea cunceduco; E così egti ricorna
al primo ferwizio, ficuro, che alla Signora non farebbe mai più venuta yogiia ai,

eum 208 (ernireda lui. yeaah
jade, Havewadl sig.\ Cav. Vieri una bellacagna da Fermo, la quale diede in cura.a
f dicendogli: Tien conto di quelta cagna 5 ed avvert a non la fimasrire y

om perché se la finarrifei non ti aspeteare altra licenza. Prefe Pippo la cura della ca-
ga $24» €col trattarla bene avvezed a fare mille giuochi, ¢ se la refe così alic~

| 2onata', che era imposfibile, che egli la smarrife. Avvenne, che Pippo fu in-
vitat® a una fella', che & dovea fare in un iuogo poco lontano da Firenze, dove
era per tratcenceli almeno tre giorni, onde chicle al padrone ticengia per a quel

  
 
 

 
 

te

   

oe tenipo';-ma non ltottenne, Pippo senza moftcar di cid disguftoy la mattina avan-
| tivalla-wigilia di dewa fefta:comparue in-cala (enza la cagna, ed il sig.\ Cav. do-
J mandi dov' ell'era.. Pippo dilse quafi piangendo:. sig.\ io non1o.s03, quando io
i fubvicino a: cafe mix ierfera ella cominciò a fuggire, ¢ per hholto., che 40 le core

mf relsi diettd chiamandola, non fa postibile farla cornare y ne arrivarla.. Replicd il
| sig.\ Cavaliere; Tw fai i parti; pero va a fare i faci moi, ¢ non haver' ardire di
od mettere il piede ineala postea (enza la cagna. Pippo fingendo un dirottissimo
4 pianto fen' ulci di casa, ¢ ando alla fefta, alla quale era stato inuitato, e pafsati
®  alcuni giorni in grandidima allegria se ne torad a Firenze 4 e andato fuori della
porta alla Croce da uno Ortolano suo amico, al quale haveva lasciata la cagna,
; se la prele, ¢ I" infango tutta, ¢ le infanguino l'ugaasaccid parelge (pedata, ¢ Ie~
i gatala con una corda: lx condu/se al padrone, il quale veduto Pippo con la cagna
giidibe: Dovel hai trovata? In Cafentino,Liuttrits: Sig,, ¢ nomci voleva altri
| cheme:per trovare il-luogo dov' ell' era fitta. I sig.\ Gav. credette quanto dife
Pippo, il quale con tale invenzione gode la soddisfazione, che bramava, E tan.
torbatti ptrun faggio delle facezic di Pippo, il di cni.intero nome, © cognome

oe
SPAN ZA LXV. STANZA LXVI.
Cenro fuggerti egli ha della fuselaffe

inch egtine Pigmei disterti., e brurti
Fanti che nacquer nelle magne baffe,
Mita se ben fon piccini, vi fon tutti,
Mangian /pindci,arrufian le mataffe.,
Ea ha più viz2j egnunydi fei. dargusti,
Cosa è quota che va per il sue dritto,

Che non è in corpo storto animo drittas

tf girea Bariffone adele rocca
Gran gigante da Cigoli di quelli,
'Che vanno a corre i ceci con la brocca
| Ebattoncon le perriche i baccelli:
Ler sue bellexze amore hasipreincocca
Per ferir Dame i dardijed ¢ guadrelli,
. Fa il Cavaliere nelle cavalcare,
2 va [pel furiero alle nerbare.

 

|
|

 
 
   
    
   
 

196 MALMANTELBhAT

Segue Batiftone Nano con una gran quantita di compag al
bene fon così piccoli, fon tutti viziofissimi, ¢ non Segoe 3
ché in un corpo mal fatto, di rado si trova anima ben outa

BATISTONE, Questo fu un Nano levato da guardare le pecore  ¢
a servire il Serenissimo Principe Mattias di Toscana, dove in! 6
in ful posto di bello; ¢ facendo lo spafimato di tutte le Dame, ©
ce: Per sue bellexze Amore ha fempre in cocca Per ferir Damei
arrivO a fegno questa sua inclinazione aile dame, che per porere liberas
ticare con efle, si contentd che il suo Serenissimo Padrone lo facefle
come segui, ma però in burla, ¢ stette nelle mani di Maeftro Agnolo”
' Castratore circa un mefe, sempre credendo d' essere flato castrato: EB
H egli, non oftante che fufle di statura piccolidima impard afai bene a
| € maneggiare ogni cavallo aggiuftacamente, supplendo con la mano'

che gli mancavano le gambe, era (olito ancor egli andare nelle ca
i Cavalieri, ¢ pero dice: Fa i/ Cavatiero nelle cavaicate. Ma percht quel
( di Caramogi ¢ aflai fottoposta alle mazzate del padrone, ed egli ne hai
sua parte, però il Poeta dice; Va speffo Furiero alle mazzate. Questo Ni
! la morte del Sereni(simo Principe Mattias servi al Serenissimo Gran Duca if
lita pure di Nano,ma efercitava anche la cucina fegreta diS. A, S., nel
stiero s' era fatto peritifiimo, per lo che oltre alla buona provvifione¢
buscava gran mance; ma la Fortuna |' abbandond ia ful buono,
si egli innamorato d' una bellissima giovane sua pari disnatali 5 la

glic, ed in pochi giorni mori. Lo chiama Gigame da Cigoli Ȣ
quelli che colgono i ceci con fa brocca, come si fa de i fichi, ¢ che base:
(a pertica, come si fa delle noci, non potendo arrivargli altrimenti.
Gigante da Cigoli,in una collinetta vicina a S.Minjato al Tedesco,si
le donnicciuole, una Iperbolica cantilena antica, la quale dice, ) joe

Ed! una punta d' ago rogebiols

Ne facea pugnale, ¢ spada, itis it
E di quello che gli avanzava de
Ne faceva uno spuntoncin,

ae

 
  
  
 

 

   
   
    
 
      

 
 

 
  

E is questa ilena con altre iperboli retrograde simili ae
re la picciolezza di questo Gigante da Cigoli; e di qui ¢ in uso comune il dires
Gigante da Cigoli a un Nano, che i Latini differo Pumitio, ¢ noi dici ne
Ledina, similitudine tratta dal giuoco della dama; Sericciole da un'
coliflimo di questo nome, Pimmeo dalla voce Greca Pygmaios 5 che' t
dell' altezza d'un pene « 1Greci dicevano Manus, Pujilius quantus Molo y ed ae
tre volte gutta; ed un Pedante lo chiamo Titsviditinm Scarabei umbra. a
Strada nelle sue Prolufioni, parlando d' un Nano dice: Fangino be seem
capite se torum tegit, Ed altrove, pure nello stesso proposito dice; iis #
cin 5 Somninm hominis, falsllum anima. xa

BROCC.A, Voce, che viene dal Greco Brochos fecondo il Monofino, ¢ &
condo altri dal Greco Prochoos; il che ¢ più verifimile, eflendo quetto valo d
acqua, ¢ quello vafo da vino; ¢ vnol dire un vafo di terra per uso di |
acqua, € pero detto Aydria, ¢ noi lo chiamiamo brocea;, Chiamali broc

 

 

   

a <n
 

AA

SERL

We

zt

SR

Le ' om |

q

TERZO CANTARE,; 177
ancora uno flrumento fatto di canna rifeffa in più parti; se quali allargate,¢ ria-
'teflute con falci, formano comé una piramide'a rovescio, ¢ di tale strumento
fermato in cima a una pertica, ci serviamo per corre i fichi,quando non si potlo-
no arrivar con le mani; € di questa brocca dice nel presente Iuogo

~ FVRIERO, Si dice colui, che va innanzi a preparare gli alloggi nel viaggia-
re che fa un' Efercito, o altra gente in buon numero. Lat, metacor mwanfionum,
Tn Latino barbaro dicefi fodrarins da fodrum voce che vien dal Germanico, la.
in buon Latino si direbbe'alinentum, pabulum', annona; Onde Foraggio, e
Foraggiare', Provifione di guerra,e provvedere l'efercito. Tuto cid si offervd dal
Ferrari nelle Origini alle voci Foraggio, ¢ Foriere, Ma erra quando piglia Frie.
re dello /pedale, che si trova in Gio: Villani lib. 8. c. 95. per accorciato da Foric-
re, sia Provifor bespirij poiché quivi, si come appreffo al Bocc. Nov, 92. si-
i. Srate dal Pranzefe pees cone si domandano anche oggi i Cavalieri di

alta. Qui si serve della voce Furiero per intender fur:a che fuona quantita,
come dicemmo sopra in questo Cant. stan. 50. ¢ vuol intendere, che queito Nano
speflo toccava qualche furia, cioè quantita di nerbate. Vedi forto C, 9. fan. 49.

PIMMEL » Beano popoli nani, che habitavano nell' ultime parti dell' Indic,
i quali. crescevano fino all' altezza al più d' un braccio, ¢ le loro mogli di cinque
anni partorivano, ed otto erano vecchie. Di eu fa menzione Plinio lib, 4.
cap. 11. ove dice i barbari chiamarli Cathizi, ¢ lib. 7. cap. 2. Cofforo per efscr
così piccoli erano infeftati, € rapiti dalle Gru, onde per difenderfi andayano
armati di'frecce; ¢ cavalcando sopra alle capre in granditlime schicre,a guattare
iloro nidi, ¢ romper loro ! uova. Di questi parla Giuvenale fat. 13. dicendo.

Aad fubicas Thracum volucres, nubemque fonoram
Pygmans parnis currit bellator in armis,

Mox impar hofti raprnfque per aera curuis
Vugiibusa fava fertur grue: Si videas hoc
Gentibus in nostris, rifu quatiere, fed illic,
Quamquam eadem alfidue /pettemur, pralia ridet
Nemo, xbirora cobors pede non eff altior uno

NELLE magne baffe. Intende che sono di flacura baffa, se ben par che dicas
fieno nati nella bafla Alemagna. Lat. Germania inferior,

SE bene e' fon picein? vi fon tutti, Benché piccoli hanno malizia quanto un,
grande. Tydeus corpore, animo vero Hercules; da Omero, il quale descrive Tideo
il padre'di Diomede piccolo si di statura, ma gagliardo.

MAKGVT TE, Che Nano fuffe coftui, ¢ quanto fagace, ¢ scellerato, vedilo
nel Pulci nel suo Poema intitolato il Morgante ? Questo nome di Azargusre forse
fu finto' dal Puici a similitudine di 4¢ergite, Personaggio famofo per la sua sccm-
piataggine, il quale fa il faggetto d* un intero Poema burlesco di Omero; e cid
pore avere imparato il Puici da} suo dowo amico meffer Agnolo da Montepul-
ciano. ©
NON è in corpo forts anima dritta, Non & in corpo mal fatto, animo ben,
compolto, giufto, ¢ che tirial buono; che tanto significa la voce dritto in que-
sto luogo. Sidice anche: Vn fegnato da Dio, non 4 mai buono: (alludendo per
avventura a Caino, Gen, c. 4. vers. 15.: quali che quel tale sia in un certo mo-

do

 

 

 
 

178 MALMANTILE ~ %

do contrafflegnato, affine, che ognuno, che Jo vede si guardi ) qu:
praticata comunemente, ¢ si vede da i seguenti versi maccheronici

Nulla fides gabbis,& parum credite ruse, s

Si guercius bonus eft, inter miracula feribe, i

Vn' altro Poeta in questo proposito disse: Chiude un' anima bigia
Che huomo bigio intendiamo huomo cattivo, di poca coscienza, ¢
gione, Marziale. Crine ruber, niger ore, brevis pede, inmine lafus Rt
préftas Zoile, si bonus es. Quel Terfite, che quanto sconcio di vilo, ¢ sc
to nel corpo, altrettanto era bructo nell' animo, ¢di coftumi
fopportabili; vien descritto da Omero al 2. dell' Iliade ( fecondo la tra
Pictro la Badefla Meilinele, stampata in Padova l'anno 1564.)

Lusco a' un' occhio, ¢ a' um pic oppo, e firetto

Wegli omeri, che gobbi ha sfin' al colle;

Aguxro il capo, e'l capel cre/po ye raroy

Sucido ye ner, lentiginofo, € marcio,

ZA LXVII.

   

Piena di fudiciume,¢ di frambelli eUacfiro de? Bianti, e de' Monell,
Gran gente mena qua Palamidone, E vefte la corazza da baftone y
Chril giorno vanne a Carpi,ed a Borfelli, Perch egli quant' egnialtra (uo alte
E la notte al Bargel porta il Lancione, E' tutto il di figura di riliewe,

   

Palamidone conduce feco una quantita di birboni, stracciati, ¢
cra Jui. Questo fu un guidone mezzo matto, ma tutto trifto, ed al
gno birbone, il quale faceva servizio a' carcerati,¢ perché continovamente br
tolava, dicendo di pazze (cioccheric, haveva fempre dietto una gran quantita di
ragazzi che lo facevano stizzire. La notte per guadagnar qualcofa portaya'
tro al Capitano, o Caporale de'Birri un' arme in asta solita portarsi —
glia del bargello, quando la notte va facendo la guardia, la quale arme noi
detta /ancione, Ma che egli rubafle non posso crederlo, perché afflolutamente nom
havea tanto giudizio, ¢ stimo che il Poeta dica questo nel presente luogo, ¢ a
trove per descriverlo per uno di quei furfaati, de' quali si pud credere ogai ribal-
deria. Palamidone ¢ accrescitivo di Pa/amudes, Eroe noto nella guerra Hy
fecondo la pronunzia Greca pil moderna dicefi Palamide,e non Palamedes onde
è fatto il (oprannome di Palamidove; che significa un lungo ¢ fortile, come wie
palo, una persona grande di stacura. =o

eANDARE 4 Carpi, ed a Borfelli, Carpi ¢ un Principato in Italia notitfimo j ¢
Borfeili è un luogo ful Fiorentino, ¢ scherzando con questi due nomi Carpi itt
tendiamo carpire, cio¢ rubare., ed a 4or/edé, cioè alle borfe per rubare. Ati
stofane Poeta Greco nella Commedia inticolata i Cavalieri, citato dal
nel Flos Ztalica lingwe, ( ove egli tocca la maniera di parlare Fiorentina; eng

ebbe per San Giovanni, usata anche dal nostro Poeta; ) dice così: manus in Attt-
lis hades « Vuol dire: fempre chiede, ed ¢ apparecchiato a pigliare; scherzando fal
nome di certi pork chiamati 4ro/i, per  allufGione che ha questa voce alla pa
rola atein che significa chiedere  ate

PORT ARE il Lancione al Bargello. Quelto meftiero (olito farli da birro novi2idr
lo faceva alle yolte Palamidone., comes' è deo. 5 he ee

:; BiANTI

   
 

 

      
   
  
 
 
 
  

 

a

reo.
RARER LARP LEE

weERAS

RNLSESE

mA

Be

 

TERZO CANTARE: 179

BIANTT, Si trova una specie di Bricconi,e Vagabondi che v anno buscando
danari con invenzioni, come si vede da un libretto intitolato Sferza de' Bianti, ec,
E si dicono anche Monelli; (¢ ben veramente per monelli intendiamo quei pove-
ri, che si stroppiati, malati, impiagati, o morti dal freddo per muover
Ie persone a far loro elemofine, donde rab po far it monello quel ragazzo,
che havendo toccate leggiermente delle dal Maeftro, o da altri, metre as
fogquadro il vicinato con le strida per moftrare d' essere stato dalle bufle strop-

iato., ed in vero non ha mal nefiuno, che si dice anche far marina: vedi (opra
. 1. stan. 37. alla voce fofiano, ¢ forto C. 4. stan. 8. Di questi intende il Perfia-
ni nei seguenti versi.
Signor non fo se voi fapere il bando

Di chinder tutti dentro a! Mendicanti

Mascalzon, vagabondi, ¢ maleftanci,

Che vanno per le frrade mendicando,
fo che sono in arnefe tanta male

Mi ritrovo in grandissimo viluppo;

Temo esser prefo in vece d un Gaiuppo,

E finir la mia vita allo Spedale.

VEST E la coracxa da baffone. E' armato a baftonate, vefte un' armatura da,
difenderlo dalle baftonate; $' intende che ¢ fortoposto a toccare spello delle ha-
stonate.

. RILEV-ARE. Intendiamo buscare, conseguire, ottenere. Petr. Canz. 22.
M fempre sospirar nulla rilieva,

Onde (¢ bene figura ai rilievo yuo! dire Ratua di marmo, o di altro materiale,
noi incendiamo rilevare, cioè bafeare ¢ qui intende buscar mazzate. I) verbo ri-
devare piglia questo significato da rilievo, che sono gli avanzi delle menfe de' gran-
di, quali avanzi si buscano per Jo più da coloro che servono a tavola, donde di-
ciamo Viver di rilievi che vuol dir Campare d' avanzi. Vedi (otto C., 5. stan. 47.
Franco Sacch. Nov. 154. Quando la croitata fu mangiata tutta, senza far rilievo ne
meno de' topi, Rilevare yuo) dir Quello esprimere che fanno delle parole i ragaz-

zi,g imparano a compitare.

STANZA LXVIII.

Comparisce fra tanto umcarro in piaga Con che la formidabil Martinazza
Da Farfarel tirato, earbariccia A lor, ch' ¢ ch' è, le coftole fRropiccia,
Vobidiente al cenno della\eayz4 E quei Demon} in forma di Camoxza
Soda, nocchinta,ruvida,e mafficcia. Vin tirando a battuta la carrozza.

In tanto, che si fa la moftra de' soldati di Malmantile comparisce in piazzas
un carro tirato da due Demonj in forma di capra faluatica, che questo vuol dir
camoxza, la quale per lo pil si trova ne i monti del Tirolo. Plin, lib. 12.cap.37
la chiama Rapicapra. | nostri antichi dissero Stambecco, il Lat. ibex..

 PARFARELLO, ¢ Barbariceia, Nomi di due Demonj dal nostro Poeta cava-
- da Dante, del significato de' quali nomi vedi gli Spositori sopra il medesimo

ante.

, NOCCAIVT A, Piena di nocchi, che sono quei piccioli rilevati come bolle,
iquali si veggono per Jo più ne i. di pruno, di forbo, ec, che gli rendono
2

rnvidi

x.
ita MALMANTILE (5

ruvidi, eli chiamamo ancora wedi, come fanao i Latini.,..
MASSICCE, Intendiamo tutte quelle cose,, che dal.
te di materia stabile, ¢ folida, ¢ non vote, 0 vane,0 in
deboli, siqech ¢ rralnepeiegig
CHé ch'é, Ad ora ad ora,.di quando in, quando 4 | argh
ST ROPICCIARE, Fregar qualcola con)p altro,ed i Lat
Forse € corrotto da froppicciare, che pare si dove ie an ee
cio, con che per lo pil si fropicciano gli arnefi per liberargli )
Sropicciar le coftole a uno vuol dire Baffonare uno... 07 RA
TIRANO (a carrozza a battuta, Nona battuta di musica »maab
mazza, con la quale Martinazza la baftona, » 'eerie
STANZA LXIX, STANZA LX
Coftei ¢ queila firega maliarda, Ove la notte al nace eran concorse
Che manda i cavallacci a Tentennino,. < Tutte le Streghe anch'effe ful ca
Ed egliun punto a comparir non tarda Z Liavosi col Bau y le Biliorfe
Quand ella fa lo feaccio,o ul pentolino, A ballare ye.cantare,e far te
Come quand'ella si unge,e s'inzavarda Ma quando presso al di
Tate ignuda nel canto del cammino, Fa di moffi
Ler andar col Barbuto foto il mento Come a coftei, chor vienfene:
Con la granata accefa a Benevento, E in [yu quel carro nelCaftell
STANZA LXKL
E la cagion si ¢, ch' ella ne vada Perché vi,
Adeffoacafa tutta in caccia,e in furia,  C alla [ua patria
L' haver veduto dentro alla guaftada Percio, se nulla fuffe di
Vn fegno, che le ha data cattiv' uria; We viene anch'effa a dar il
Martinazza è una di quelle fireghe, le quali. coftringono il Dia
lo staccio, e il pentolino, ¢ con ungerfi per farsi portare a B
greflo de' Diavoli (otto il noce: Questa Martinazza adeflo, si fa
famente daquei Demonj a Malmantile, perché ha veduto-nella cara
da fanguigna, che le prefagilce la caduta di Malmantile, onde vi fis
ancor' efla per dare il suo aiuto. (Questo nome di Martinazza ¢ nome 3
quella strega, ¢ fregherie fon tutte dal Poeta dette pep accennare l'opinioneé
alcune donnicciuole, le quali portate dall' illufioni diaboliche y si danno a ¢rede
re d' havere effettivo commerzio col Diavolo. > tit oo
STREGA. Vedi sopra C, 2. stan, 11. Viene da frix uccello
detto a fridendo, secondo Ovid, fatt..6. aati
Eft illis frigibus nomen, fed naminis bnius, —
Canfa,quod horrenda stridere noite folents
E questo uccello ( che forse era l'Arpia., ma Plinig,dice 5 che non
fofle ) credevano gli antichi più superftiziofi., che rapifié i bambini-
Et ab huius avis nocumento feriges Latini appellabant mulieres puellos'
raflu, E diqui ancor noi le chiamiamo streghe, che le:
da far malic, fattucchierie, ed incantefimi, ¢ però chiamate.ancora J
MANDARE un cavallucio, Magdaceuna citaai sioé chiamal
gindizio criminale con polizza. E quette polizze de Giudizz) Criminal

    

  
 

 

 

   
  
   
 
 
    
     
   
 
  
        
 
 
 
  
       
 
 
   
  
  
    
  
 
   
     
 
 
 
   
 

R SRSCER Esa ase EE

—
=

wERERRE

TERZO CANTARE. 181

renze si dicono ¢avallucci a differenza di quelle de' gindizzj Civili, che si chiana-
no Citazioni; ¢ questo nelle polizze criminali ¢ stampata l imprefa, o
contraffegno del Magiftrato criminale, che ¢ un' Huomo a cavallo armato; qual
con ¢ chiamato comunemente Cavalluccio.

-TENTENNLNO.Nome dato dalle nostre donne 2] Demonio per non !o chia-
mare Diavolo; quali rexrarore; col qual nome ¢ nominato preflo San Matteo
Cap, Vers. 3.00 t i:

HA lo paccio ye il pentolino, Favoleggiano, che quelle donne Maliarde, ¢ Stre-
ghe, che habbiamo detto, aes fare diversi incantefimi per ritrovare cofes
perdute, © per ottenere altri loro intenti, ¢ fra questi incantefimi fare lo Paccio,
0 it Pentolino, o la caraffa; Si che dicendo Fa /o staccia, e il pentolino intende fa in-
cantefimi. Quei che indoyinano per via di flaccio sono detti dat Greci Co/cino-
mantels.

COME quand' ella s unge, es inzavarda. Inzavardare,¢ uno impiaftrare con
materia morbida, e viscofa, atta a distendere come il lardo, I) Poeta seguita,
la vana., ¢ superftiziofa opinione, che queste tali donne vadano ogni tanti
giorni al congreffo de' Diavoli sotto il Noce di Benevento: Ove da notre a/ noces
eran concorse; al qual luogo dicono efler portate dal Diavolo in forma di caprone,
che questo intende if Barbuto forto al mento, ¢ cavate dalle loro cafe per la gola.
del cainmino (¢ però dice nel canto del cammino ) dal medesimo diavolo forzato a
far tal funzione da quegli uatumi, che dice esserfi meffa addoflo la medefinas
donna; a quale poi a detto congreflo fa rempone, cioé si da buon tempo; si piglia
tutti quei piaceri, che le vengono in fantafia quella notte; Ma ful far del giorno
Ie conuien partire, ¢ il Diavolo in:un baleno la riporta al suo paefe. Tale opi-
nione hanno simili scimunite; ed o sia per effetto di matrice, 0 pure per opras
del Diavolo, che per illufione faccia-loro apparir per vere tutte quelle (ciocche-
tie, che esse si fingono nella testa, l'effetto ¢, che esse si credono d' efler' anda-
te veramente a Benevento, ed cffere state riportate dal Demonio al loro paele,
quando effettivamente non si sono moffe del letto..

. GRANAT A, Bran mazzetto di scope, od' altra cosa simile, che s' adopras

spazzare,¢ ripulire le stanze. E con queste granace accefe in mano dicono,
che tali streghe vadano cavalcando sopra un Caproneal detto Noce di Benevento.

BAV, ¢ Biliorfe, Questi nomi bau, biliorfe, orco., befana, versiera, ¢ altri
simili, (ono tutti inventati dalle Balie per spaventare i bambini, ¢ rendergli ub-
bidienti,persuadendo loro, che.questi fieno spiriti infernali, ¢ però il Poeta nu-
mera fra i Diavoli il Bau, ¢le Biliorfe,, per accomodarsi alla capacita de' Fan-
ciulli, per li quali profefla-d' haver composta la presente opera. Vedi sopra Cs
2, flan.'50. 1 Greci il cemibalo per chetare 1 bambini dicono Carabax

FAR tempone., Darfibeltempo; Stare allegramente, pigliandofi tutti quei:
poliskhoee pud, ¢ si pigliarsi., che diciamo anche /euazxare; trivnfare; far
juonacera; Genioindulgere, litare Genio, dissero i Latini. La Compagnia della
Letina infegnando., in-qual luogo si deva pigliare laccafa per risparmiare, dice:
Vorriano le nostre.cafe esser in una quafi dall altre feparata comrada, lontana'da vie, ©

piagze pubbliche y dove all? occasiani si fefteogi,¢ si facciatrebbi, e tempone.

BATTER il taccone.. Elo stesso, che barter la caleofa, dewto sopra  questar

2 lan

 
YY Bate

* trovafi nei Latini /olwm vertere. &

    
   
  
 
  
  
 
    
   
   
    
   
 
     
     
   
  
   
 
 

182

C, stan. 60,,cioé ¢amminar via; andarfene. Si dice an
dice il fuolo della scarpa, cioé quella parte, che posa in
- th
VENIR di punta. Venir con velocita, a dirittura; che diciamo:
vela. Vedi sotto C. 6, stan. 10, Credo sia originato dalle barche,
venir di punta quando vengono a dirittura senza volteggiare.
IN caccia, ein furia, Cioè in fretta, frettolofamente, e con furia,
no coloro, che fon cacciati; che però diciamo; Corre, che par ch'egit| ha
dietro, Incedit quafi in fugam versus. A
GV AST ADA. Specie di vafo di vetro per uso di conseruarui
stesso, che caraffa dai Latini detta Phiala, L' Autore disse sopra nell'
antecedente, che Martinazza era (olica fare lo Staccio, ¢ il Penroline, €
la Guaffada; queste maliarde, ¢ fireghe empiono di superftiziofi li
raffa, 0 guaftada, e facendovi mirar dentro da un fanciullo innoc
dire di vederui dentro quel che hanno desiderio di fapere, ¢ tutto
le persone femplici, ¢ cavar loro denari di mano. Questo indovinare
acqua, fu anticamente prefio i Perfiani, ¢ da'Greci si chiama Aydro)
questo habbiamo un detto Géi ha +i diavoio nell' ampolla per intendere
dovina ogni cola. Z
CATT IV' uria, Cattivo augurio. Questa voce Vria corrotta da
ta per lo più dalle donnicciuole, deta senza aggiunta di cattiva, o bt
tende cosa, che non piaccia. La tal cosa mi dé uria: © s' intende mi
mi da impedimento, mi da noia; da che si pud credere che sia
xegia, che pure vuol dir noia, faftidio, impedimento, ec. 0 forse ia
che fuona Jo stesso, che xgeia, o forse in vece 4' ombra, che è il
do vale per impedimento, /a tal cosa mi dd ombra, per la tal cosa
Siche Vria, xegia, ubbia, ed ombra faonano tutte lo stesso; Vria, e
usate per lo più dalle donne, ¢ ' altre fon pi: comuni. Si potrebbe
fecondo il Monofino, che la voce ria veniffe dal gteco Vria, che
prospero, ¢ che si come habbiamo per coftume di dire buona, 0 ¢è
quantungue /orre significhi affolutamente bene,c felicita;così habbiamo.
di dire buona, o cattiva Vria,quantunque Vria significhi (empre feli
Greco Vria, Nello stesso modo, benché prefso i Francefi bear signi
licita; voce a loro derivata similmente da) Latino augyrinm; dicono
malheur, quali buona, ¢ cattiva wria, cioè buona, e mala ventura; €
doci servir bene di questa parola Via, come vocabolo di mezzo, dour¢!
giungerci buona, 0 cattiva,¢ non dirla afsolutamente, ¢ senza detta
come habbiamo accennato, che molti se ne servono; ma l'uso ci Lil
aftrufe stiracchiature. '
SE nulla sue, Per tutto quel che potefse fuccedere, Se accadefse q
grazia. 1 Latini in un simil modo per isfuggire il cattivo augurio, ¢
nare cosa infanita, come ¢ la morte, dicevano: Si quid patiar. Si gi
manitus acciderit, Se Dio facelse altro di me, con tutto cid, ¢c.
NE viene anch' essa 4 dare il suo difegno. Con queste parole moftra
quanta gelofia haveva Martinazza di non perdere J' autorita, che «

  

     

   
    
    
    

 
 
 

ale

=
=

Se

 

wo
off
3
o
è
34
po!
4
eo

Y

 

 

TERZO CANTARE; 183
Malmantile, ed il sospetto di non efser levata dal grado di Salamiftra, che go-

deva, come accennammo sopra in quelto C, stan, it,

STANZA LXXIL STANZA LXXIIL

Fuggh tutta la gente [paventata Figuriamci vedere un [acco pieno
ell' apparir dell' orrido spertacolo, Di zucche,o di popon sopr' aun giuméto,
La praca fu in unt attimo spayzara, Che rottafi la corda, in un baleno
Pur un non vi rimafe per miracolo, Ruzziolan tutti fuor ful pavimento
Così corrende ognuno all imparzata E nell' urtarsi batton ful terreno:

Si se Cun l'altro alla carriera offacolo; Chi si perquota,e chi s'infranga drento

Chi da un'urton,quell'altro da un tracollo, Chiff sbucci in un fafso,e chi sintrida,

Chi batte il capo ye re colle, Ed un altro in due parti si dsvida.
STANZA LXXIV.

Così fa quella raxza di coniglio, A tal che in veder quello scompiglio,
Che nel fuggir 1a vifta di quel cocchio obo ben prefo( dice) qui lo ferocchia,
Chi se rompe ta bocca yo fende un ciglio Mentre a coftor così comparir voli:
E chi si torce un piede,echiunginocchio; Sapeva pur chi erano i mici polli,

1) Poeta descrive aflai vagamente il timore, ¢ lo spavento, che eatro addosso a

ei di Malmantile per la vifta del Carro di Martinazza, la quale vedendo colo-
ro così spaventati, si pente d' eer quivi arrivata in quella guila.

IN xn atrimo.\n un momento.Corrotto da atomo.Si dice anche nn baleno,come
nell' ottava 73. seguente. Jn un batter d! occhio. V. sotto C, 10. stan, 42. dal Lat.
Etu oculi:En atomo difiero i Greci. Dante Inf, C. 22. Subito,e (peffo 4 guisa di baleno,

NON ve ne rimafe sy miracolo, Fuggiron tutti, che non ve ne reftd pur'
uno. Tanto esprimeva se havefie detto: Von ve ne refi pur' uno, Ma col dires
miracolo da maggior' emfafi, ¢ seguita I'afo; ¢ vuol dire farebbe ato creduto mi-
racolo se un folo vi fufle reftato.

CALL impazzata., Acalo; Come fanno i pazzi, cio senza considerar gquel-
Jo che facevano, o dove essi andavano. #' il latino perperam,

VRTONE. Percofia che si da con tutta la vita in un' altra persona, 0 in uns
muro, oaltrove, cd lo fleflo, che Spinta, ne vi (0 fare altra differenza (es
non che Vrtare vuol dir percuotere.a cafo, ed € il Latino ofendere; © Spingere
vuol dir Mandar uno innanzi, o-indietrocon violenza, ed ¢ il latino émpellere.;
Ma nondimeno-#rtone, ¢/pmta si pighiano |)' ano per l'altro 5 se bene non si di-
rebbe Dare una spinta in un muro,'0 altra cosa immobile, che fatta mobile co-
mie farebbe un muro sciolto per farlo rovinare, si direbbe Dare una spinta. A
= quafi recifo da piede per atterrarlo 5 si direbbe Dar la spinta per farlo
cadere, ec,

TRACOLLO. Accennamento di cadere. Extra collum pedis ire; 0 pure detto
così quafi Tracrello. Vocabolario delia Crusca. Tracollato addiettivo da rracol-
dare 4 che vale lasciar' andar gid il capo per fonno, o simile accidente.

GIVMENT®O. Si dice propriamente l'afino  benché s'intenda anche ogni be-
fiiaccia da foma.. Così prefio i Latini: Quello che in $, Gio, cap. 12, chiama-
to pullus afine yin S, Matteo cap, 21, & detto pulls filins fubiugalis, Puledro, figline-
do della giumenta,:

RVZZOLARE. Gisare per terra; che diciamo anche Rotolare.

I3-
 

184 “MALMANOTILE?

INFRANGERSI, Sflagellarsi, ammaccarsi » disfarsi,
76. C.11, stan. 12. i I ti
RAZZ A di Coniglio, Gente timida, ¢ codarda. Si dice poltrone come
giio, perché questo animale, che è specie di lepre; come quella,
PIGLIAR (0 ferecchio., lngannarsi, Far' errore. Lo sono st
credendo di (tar bene, ma ho pre(o lo (crocchio; cioé mi sono it
sono stato male. Il proprio significato della parola, ferecchio &
trovar danari,piglia a credenza una mercanzia per ventici

   

 
 

 
 

questo,quando noi facciamo una cola, che non ci torna poi bene, ne
utile, ¢ gusto, ma pil tofto ci ¢ di danno, si dice pigliar fo ferocchio,
S-APEVO chi erano i miei polli
Cognosco oves meas.

   

STANZA LXXV. STANZA LXXV
Scefe dal carro poi per impedire Percio si ferma strambasciata ye.

Cosigran fuga,¢ rovinofa fola;
Ma quei vit pis si frudiano a fuggire y Dalla Carretta subito di
E moftraognun se rotte hain pie le (uola, Eqgli si lancia addossa ac
Chi finalmente, come si sual dire Così correndo tutra si
Chi corre corre, ma chi fugee vola, Perché quel Diaval vanne
Ond' ella y ben che adopri ogni putere, Pur ( dicendo: arrila 5
Vede che fara tordo 4 rimanere. Lo fruga si,ch' al fin la
Martinazza scefe dal carro per fermar quella gente, che fuggiva,@!
correr lor dietro, ma allora si, che coloro fuggivano, onde ella
a uno di quei caproni al fine gli arrivo. EB qui termina il terzo Cantat
FOLA, Quantita di popolo, che furiofamente corre a qualche luo
to da i Cavalieri, che gioftrano, che dopo, che si (ono soddisfacti li
a uno per volta a giofirare, in ultimo corrono al Saracino ( così
mezza figura, 0 bufto 3 di Moro, o Saracino, fatta di legno, efi
corrono dico al Saracino tutti in truppa, uno però dopo l'altro,
far la fola, In Latino potrebbe dirfi: exerceri ad palum. Vegezio
lib. 1. cap. 14. Tiyro, qué cum clava exercetur ad palum, baftilia quogue
gravioris y quam vera futura sunt iacula, adversus illum palum tamquam:
minem sattare compellitur, E si dice fola, 0 folata d' uccelli, di popolo
tender di cose che velocemcate si muovono.in quantita, ¢ prefto finile
ta di vento, Studiare a folate. Lavorar a folate,ec, Forse meglio fola y
fica quel che i Latini dicono Adagna hominum vis, vel turba, aut fummafi
bomsnum, Si come noi dal calcare le strade, che fa il popolo.e daliu ee
¢ stretti, diciamo Vna molticudine numerosa di gente, una gran calc
Franzeéi nella lor lingua la dicono fowle, cioé fol/a dal verbo fouler,
calcare, Da folla abbiamo fatto Afollarsi, ¢ Folto, denlo, calcato
tarsi, far furia,far preffa: lo stelso quali che -Afollarsi tutto deriva
tura dal Latino follss, nel quale sta l'aria ferrata in modo, che pi
capire. i ' vi

 
 

  
 

Sie
Vitae

 

 
     
  
   
    
   

non ne vale venti, ¢ poi la vende quindici, ¢ questo si dice pigliar lo fero
Plauto disse: Emere caca, vendere oculata die. Vedi forto C. 6, stan. 60. |

  
  
   
  
  
   
    
  
  
    
   
  
      
  
   
 
  
 
  
   

- Sapevo di che qualita eran coftoro, @ il

Kitorna indietro, ed un de' si i

 

a la

TE Ee a SE
 

SARRERLL ELLA TERRES DEN

~

 

TERZO GANTARE. 185

      

 STVDI. SZ, I) verbo fudiarsi er affaticarfia far prefto, o spedire
tuna cosa, che diciamo anche menar le ioe. Per efempio: pie 5 nee il
tempo è breve, ¢.non finirete, se non fate prefto. Qui intende s' affaticavano a
fuggire infeare: al che s' adatterebbe i yerbo szeumbo, labore, ed anche+

Studeo, ¢ quelto dal Greco spexda, afrertarsi...Nel,Salmo: Domine ad adivandium,
me feftina. conn Tddio, Tease ad! aiutarmi. eae Sic feftinanti femper lucn-
pletion obfear y a colni che si fiudia @ arricchire il pi riceo da impaccio.

MOST RAR le fuola deile fearpe. Corscr velocemente; perché così s' alzano
aGai i piedi, ¢ si moftrano le (uola delle (carpe. I Greci pure dicevano in questo
propolito Canum pedis offendere, Si dice tan Battere iltaccone, che vedemmo
sopra in questo C, stan. 79.

CAL corre corre y machi fuece vole. Detto fentenziolo, che significa, che mol-
to pili forte corre quello, che ¢ perfeguitato, che non corre colui, che lu perfe-
guita, perché la. paura gli mete l'alia' piedi,¢ per questo dice Chi fugge vole..
Vergilio dilses Pedibus simor addidit alas,e Dante Inf.C. 22.

E poco walle yches' ali al sospetto, Non potcro avanzar.

Intendendo,.che il gran timore, che hebbe del Demonio quel dannato,|o fece

efser pitt veloce, chet! ali di quel Demonio, che gli correva dictro. Della pa-

,, rola agit Ipiegantidima della yelocita appreflo Vergilio,vedi Seneca Epift, 108.

PARE tordo a rimanere. Cioè rimarra a dietro, ¢ non arrivera quella cana-
glia'. sfl.giuoco de' tordi ha qualche similitudine con ' Amilla de' Greci, guia de
certo iafhu inter dudentes certemen eff, come dice il Buleng. de Ludis Veterum cap.
a4 clagara si dice in Greco.amide. Nell' Amilla si tirava una palla dentro as
wa fegno, o-circolo,.¢ colui perdeva, la di cui palla usciva, o non entrava nel
ciccolo., Nel cordo non si fa ne segno, ne circolo, ma si tira una piccola palla,
( da noi a distinzione dell' altre palle detta grille, come vedremo forto C. 6. stan.
pe colui., chelatiradice: 4 pe/sare, cioè a pafsare con la palla il detto
gsillo, © a rimanere, cioé rear con la detta palla di qua dal detto grillo; così
sirando ciafeuno,s' ingegna di pafsare, o rimanere il pill vicino a detto grillo, che
egli pud; perch chi meno lo palsa, o meno addietro gli rimane vince la posla,
¢d.a quelli, che-non pafsano, o non rimangono, quando devon rimanere, o paf-
fare, vince il,doppio,¢ questi perdenti si chiamano Tordi, € sono di tre forte,
perché tre sono i cafi del tiro; cioé Tordo a pafsare è quello, che pafsa di la dal
grillo quando deve rimanere. Tordo.a rimanere quello che rimane di qua dal
grillo,quando deve paGare. E Tordo femplicemenie si dice quelio,la di cui palia
refta in dirittura del.grillo per banda,e questo da alcuni fifa che non vinca,ne per-
da, daaleuni, che perda folo la meta degli altri tordi, se ¢ pili lontano dal gril-
Jo di oo che vince 5 efe è più vicino non perde; da alcuni gli € perme/so riti-
rare fino a tre volte, quando però fempre refti in dette tre volte nella medesima
dirittura del grillo; e quando non paffi, o non rimanga perde una fola posta: ¢
fsempres' intenda pafsata, o rimasta la palla quando fra ¢fsa, ¢ il grillo pola,
interporfi un filo in fquadro,se però non 1o tocchi non per banda, ma per quella
parte,dove hada rimanere, o reftare; ¢ tutto si fa secondo le conuenzjoni, e+
atti. Questo giuoco per lo pil ¢ usato da' ragazzi, o dagl' infimi botiegai di

Firenze; i quali nei giorni felte, uscendo dalla Città per andar' a pigliar'
3 Aa aria
%
ar, tie 5:

 

 
 

 

    

- Aty I
(a 1 Al
' in z 'hie sa
186 MALMANTIER()
aria nel camminare giuocano a quelto giuoco, \
'no a chi perde, ¢ quando n' hanno fegaatitanti, ¢
bere, e da mangiare, si férmano alla prima Ofteria'; €
quantita di danaro, che ha pérduto. Hor tornando a pro
Unazza farò tordo arimanere, ed intende, che rimarra (ro >
quella ciurina. ' pt ge aag
STRAMBASCIAT A, Affannata; Oppréa dal' ambascia 5:
dificulta di re(pirare cagionata dalla violence fatica nel correre,
prabbondanza d' alito. Dante Inf. C.24. & però leva si; vinci I
qui per avventura e4mba/ciadore, che piglia a fare amba/cia, cio'
dare a quel Personaggio, o Città, a cui eglié inuiatoys ©
S/lancia. Si getta; cioé con un falto monto preftamente a.
rone. o eal
: S/rinfacca, AGomiglia Martinazza (che cavalcata in fal suo Capi
a quando s' empi¢ un facco di roba leggieri,la quale si mandi gil co
stiuarla, ed empier bene il facco, questo s' alza, es' abba(sa
faceva Martinazza a cavallo ia ful Caprone, il quale faceva a lei
andando baielloni, cio' a falti,come ¢ il proprio correr delle capre
ce balzeloni viene da balzellare, che lo diciamo il faltellar delle le
di Maggio, ¢ Giugno, che elle (ono in amore, e la caccia che in talt
si dice andare al ha/zedo. Del cavalcare la beftia nera, ¢ cornuta V.
ARR 1d, Cammina li, Va la. Termine stimolatorio usato ju
ec, dai vetturali. B' ben vero, che vedendofi uno a Cavallo, i;
ciamente, si fuol dire per derider colui 4rri /@ quafi diciamo yaa cavalea un' af.
no, ¢ portato da questo uso 1' Autore fa dire a Marcinazza Arré jd. |
Jo fa venire dal Greco Errbe, cio, va via, a
CARNE cattiva, Animale vituperoso. Diciamo carne cattiva', 0 cat
'di carne ancora a quegli huomini, che sono di genio sciagurato,€ + Oe
de si dice quafi in proverbio, ¢ per ironia di chi sia magro, opi di perl
ma sia maligno,e aftuto,e come si dice ne' faoi panni @ vi sia tutto 5” 0
Stornello, poca'carne,e cattiva, Equi si pud anche dire, che I Autore la'

carne cattiva, perché era capra, che fra le carni, che si mangiano', ae
. VMs oyet

 
   
 

 
   
  
  
    
    
    

  
   
   
    
   
   
  
  
 
  

ya.
CIVRALA.Dal Lat. turmaSi dice propriamente degli Schiavi
Jera: Ma Gi Piglia ancora per quantita di gentaglia,e qui intende di
glia, che fuggiva. Vedi (otto C. 5. stan. 16.5 € C. 11, stan. 16;

ae

FINE DEL TERZO CANTARE,
' ve

ee
atk

 

 
 

 
   
 

cee
OCANTARE,

ARGOMENTO.

Tguerrier di Baidon fon mal disposti
Perché la fame in campo gli travaglia;
Lifendefi,¢ Perlon lasciamo i posti,
Won vedendo arrivar la vetrovagiia.
Pfiche non tiene i fusi pensiers ascofti
et Calagrillo Cavalier di vaglia,
Che promette aiutar la damigella,
E poscia ascolta una gentil novella.

 

At

 

STANZA I.
Maia vincit amor: dice un Teftoy
Exun'altro diffese dette piit nel fegno:
Fames Amorem superat. E questo
E' certoye approwacgniic'ha un po d'igegno
Perch? quantunque mor sia si molefto,
Che tutti i Martorelli del sue Regno
Dicano ogn' ora; Ahi laffo,io moro,to pero,
Enon si trova mai, che cio sia vero,
STANZA IL
Non ha che far niente con la fame y
Che fa da vere, pur ch' ella ci arrivi;
Posson gli amanti star senza le dame
1 mefi, ¢ gli anni se mantenerfi vivi;
Ma se due di del consueto frame
1 poveracci mai rimangon privi.,
Ei basta, che de fatto andar gli vedi
© porre il capo dove il Nonnoba i piedi.

SAAS SSS

STANZA IIL
Tal che si vien da questi effetti in chiaro 5
Che d' Amore, la fame ¢ piit potente,
Ond'écognun di lui pitt questa ha cara,
E quand' alle sue hore ei non la fente
Lamentafiye gli pare oftico, e amaro;
Percto riceve torto dalla gente,
Mentre ciascun la cerca, ¢ la defia y
Es ella viene, vue} mandarla via.
STANZA lV.
eAnzi la scaccia, come un' animale
Sul buon del definare, e della cena,
Per questo ella talor, che i'ha per male 5
Psu non glitorna;ovver per macgior pena
In corpo gli entra in modo,e nei canale
Che non Lempiercbbe Arno con la piena,
Come vedremo,c' a Perlone ha fatto,
C” a questo conto grida come xn matto,

Il nostro Roeta riflettendo., che nel presente Cantare gli conuien de(crivere las
fame, che era.neb campo di Baldone, per non esserui ancora comparfa la muni+
zione di bocca, s' introduce col provare, che la fame è superiore ad Amore, quan.
tunque la maggior parte degli dupatinl, leapitando Vergilio Eg), 10. dove canto:

è; a2

Omnia
———————————7~E

 

 

   
      
 
 

188 MALMANTILE
Omnia vincit 'amor; @ 'nos cedithnys ante
éica che Amore sia più 56 fup
ver provata questa sua in si maravi;
più potente, e più stimabiley ¢ desiderabile ych
fere scacciata neJla maniera, che ognun procura di
habbia ragione di vendicarsi di tal disprezao, 6 con I" and
de\ mangiate,' col venir troppo 5 Quand6 nof si ha chen
moftrare ch' è seguito a Perlone. oat
MATTORICLS agen nea es ma
4H! laffo. Inverpolizione, che deniota dolore. dica fon
da} dolore, dal travaglio, ec. B il Lat. bem, bei mihi, Francele Helas,
NON ha che far niente. Non ¢ ¢ luogo da far comparazione. Non
isperto alla fame.;
TRAME Si dice il fieno, paglia, © altro simile che si dap
fic: Maqui lo piglia per cibo degli huomini, come ¢ scherzofo
ciamo /rameggiare,quando uno va trattenendofi col mangiare alqui
do che venga in tavola la vivanda per definare, o per la cena, che
concellare, Vedi sotto C, 7. stan. 10,
'POVER ACC/IO, Epiteto che esprime la compaffione, che s*ha
di colui, il quale finomina. Vale per infelice, disgraziato, ec.
PORRE il capo dove il Nonno hai piedi. Farsi fotcertare. Motire.
tura si dite; Appowiad parres fuor. kr eae
RICEVE torto, Non (e le fa il giufto: Non se le fa il dovere, 7%
rio di diritto. E significano questo Giufto; ¢ torto Ingiafto,“cotne
ra C. 3. stan. 66. None sn corpo florto anime dritto. enh
ANIMALE. E' nome generico, che significa ogni' specie di
coftume pigliarlo in specie, e per azmale intender folamente le!
gue poi che dicendofi animale a un huomo's”intende un hnuomo
giudizio,in somma un huomo beftia. Bocc.n.79, dice: Conofeendo
efer un' animale, Vedi forto in questo C. stan. 5 1.°Cic, Wonne vides, J
ZL canale, cine il canal del-cibo, che & la, Zola' + il comiderto adesbarconiy OY
così vien descritto in lifgua*furbesca dalla plebe Fiorentina. ea
NON ! empierebbe Arno con la piena, Non V-empicrebbe.Atnojqiia
pioggie vien groflo. Iperbole usata per intender"nno,vchenon si
gordo tanto del cibo, 'quanto dei denari, che ilatini differoD
dun huomo, quem eos non nutriet, illum nec-Bgypras.“Empiti
per dispetto a uno, che non si trova mai fazio; modo'baffo.- oe
STANZA V. STAAwNoZ A OVd,
Defta ? Anrora omai dal letto feappa,
E cava fuor'le peyue di bucato,
Poi barre il fuoce,e quocerfaila pappa
Per il giorno bambin c? allora ¢ navo;
E Feboch'é il Compar gid con la cappa,
Econ wr bel veftito di broccato,
C a nolo egli ha pigliato dal' Ebyeo,
Tuste splendente vienfene al Corteo,

      

r

   

 

  
 
 

  

  
 
 
   
 
 

Z

 
 
     
   
    
   
   
      

  

 
 

 

fest!
5
“le
¢
ys
yi
i
wo
5
a

 

wee

QVARTO CANTARE. 189.

»Inoftro Poeta ( come habbiamo detto altrove ) hebbe notizia da Saluadore
Refa d'un libro Napoletano intitolato LO CVNTO DE Li CVNTI, ed in.
comporre l'aggiunta alla presente opera se ne val (e,cavandone qualche peafiero,
© concetto', come vedremo; ¢ questo è quello della presente de(crizione delia lc-
vata del Sole. Dice dunque che /uegliata ' turora, esce del letto,e cava fuora le
perze bianche di tucato; il che allude alla chiarezza che apHOrA l'Alba. Di poi
accende il fuoco!, e fa quocer 1a pappa per darla al Giorno bambino che allora ¢ nato.
E per questo fuoco intende quell' albore che si vede all' apparir dell Aurora, il
va crescendo, ¢ piglia un colore gialliccio per lo vicino apparir del Sole;
e però dice che Febo viene con  abito di broccato d' oro tutto [plendente al Corteo del
warno bambino. E così intende che alla levata del Sole i Soldati di Baldone non.
ino ancora hayuta la provvifione per vivere, onde sono in collora, epartico-
larmentemolti diloro, che sono afluefatti a far fempre col ventre pieno.
PELZE di bucatePezee bianche pulite perché sono di bacato,cioè non adoprate
dopo che furono imbucatate; ed intende quei panni lini, che servono per falcia-
se, ed inuoltare i bambini.
BATTE il fuoco, Accende il fuoco, Così diciamo, quando per accendere. il
fuoco si batte nella pietra focaia, se ben non si batte il fuoco, ma la pictra, Ver-
gilio nel 6, dell' En, dice.

 

quarit pars femina flamme

Abftrufa in venis filicis ——————

PAPPA, Pane boilito in acqua; è la vivanda solita darfia i bambini quan-
do s' allattano, ¢ cominciano taliittaro » ¢ si dice pappa perché eflendo la let-
tera, P..puramente labiale, ¢ facile a profferirfi come sono le lettere B, M. ¢
pero ne ibambini-si-trova maggiore attitudine a profscrir quelte, che l'altre
confonanti, si che pitr facilmente profferiscono habbo, mamma, pappa, bombo s
che padre, madre, mineftra, bere, onde le balie si (ervano di queste parole per
facilitare. la loquela.a i baibini, Tal coftume cra forse anche negli antichi te
mani, come si cava da Varrone, (nel libro ane ee » Ovvero dell' alle-
vare'® figliuoli ) che per Papas intende guello, che intendiamo noi Toscani
Pappa yoda Pein »che oe Satira 3. dite cs sia

Et similis Regum pueris pres minutum.,

I Grecivpute per «i loro bambini 4i seraivano come noi, ¢ come i Latini, di
voci di due sillabe.con raddoppiarae la prima sillaba,, per maggiore agevolezza
del rilevare layparola... Di.queste parole bambinesche ne troveremo molte nella
»presente Opera., usate dal Poeta per scherzo., o per accomodarsi alla qualita di
colui che fara parlare,¢ non perché fieno in afo altrimenti. Vedi focto in questo
Cant, stan,12.dove dice d' un bambino.che impara.a parlare.

BROCC.ATO. Buna specie di drappo fatto.a fiori, es' intends Deappo tel-
futo'con.oro..

- A NOLO eli ha pigliato dal' Ebreo, Dice che il Sole ha pigliato a noloil suo
splendente.abito., per significare che lo.rende la fera,, come lo reftitui cone caio-
ro', che:pigliano gli. abiti.a nolo per.un giorno; ed intendere che il Sole alcon-
dendofi la fera alla nostravifta, la(cia guell' abito risplendente, che.s' cra mello
a mattina,, Y
3 COR.

 

 

 
 

 

 

 

196 MALMANTILE |

ao
CORT EO, Corteggio | Codazzo di donne,ec, che gn
quando va a marito, o un bambino portato a
VONANESI genti, | soldati del Duca d' V;
pellar  efercito dal nome del Generale, come Vaimarefi
COMP ARIRE in (cena, Venire in pubblico. Vedi sopra C.
LA materia che da il portante a' denti, La materia, che fam
cioé 1a roba da mangiare; si dice anche Da far ballare il mento.
uefto C, stan. 23. 2 portance si dicg una specie d' andare di cavalli. Il Lali
fr. C, 3. fan. 58. dice. 1 aE
Per dare il lor pertante ai denti asciutti,
LENA. Vedi sopra C. 1. fan. 2.
EA maiticavan male, L' intendevano male, la fop n
E solito quando si pensa a qualche cosa fifamente, ¢ con applicazio
care, onde Perfio delle composizioni ben pensate disse: Remorfum |
Suem: E tal mafticare cos: pensando si dice auche raminare,o dig
mafticare che fanno gli animali del pié feflo percid detti ruminantia
Vedi forto C. 6. stan. 5. Qui fa bell' effetto ' equivoco del verbo
che pare che voglia dire / iutendevano male, ¢ vuol poi dire che n
Ic, perché non mangiavano, non havendo che mangiare. i
STANZA VII. STANZA V
E tra coftoro un certo girellaia, E, perch' ei non bavea tutti

  
   
 
   
    
   
 
 
    
    
  
      
     

   

    
 
 
  
 

Che per U' asciutto va fui fufechini, Fu il primaad esclamare, r¢
Male in arnefe ye indoffo porta un faio Forte gridando:Obime
Che fu fin del Romito de Pulcini, Pel mal che vienein

Cit chi viel dir ch'ei dorman'ungranaio Onde Eravano,e Dow
Per c'bail maxzocchio pien di farfallini
E' matto in somma,pur potrebbe ancora
Wan di guarirne,percht il mal da in fuora,
STANZA

Mentre di gagnotar gid mai non vefta E per vedere il fin dé
Colni ch' è senza numero ne rulli y Se ne van discorrende g
Anxi rinforza col gridare a testa, Del bifegnoch' essi han cb'il
Lasciano il fuoco ye ¢ vani lor traftulli, Perché fentono omai fons
Fra li fuddetti soldati affamati |' Autore pone se medesimo descri
erfona, ¢ genio; ¢ dice che egli fu il primo a gridare per la fame,
ravano,¢ Don Andrea Fendefi ancor essi affamati s' accoftarono a
tir la cagione di quelle strida, 3
Nota che il Poeta divide il periodo nelle due ortave,ottava,e nona,di ¢
to da qualcheduno criticato d' errore, ma pero senza ragione, non a
regola poetica, ia a pale vieti il poterio fare, come habbiamo detto
, G/RELLAIO. Huomo firavagante. Huomo che gira,s' intende

hs pre »¢ che fa scioccaggini, ¢ pazzie.

ANDAR ? ascivtto, Signi efier ro, ¢ con poca

Vedi sopra Ca: stan. 68. % aad celia
VA infu fefeeliim. Ha gambe così fortui, che rafiembrano

  
   
 
 
 
 
 

     
      
  

  
    
  

   
  

 

 
 

 

eee

QVARTO CANTARE, 19%

Mine wfatifiimo da noi in questo proposito; che diciamo, Camminare fu fulcelii.
 ALAL? in arnefe. Mal veltito: Mal' ail' ordine di sanita, d' abito, ec. Lalli
tr, lib, 1. stan. 34.
eben Pcs navi ao che gli avanzaro
Qui si conduffe afai mate in arnefe.

ekildiise:Dotee ta inde dello sputo dice.
iAutee  Eccomi qui per raccontarne centey

Ben ch' io non sia d' accordo col ceruello,

E malagiato in arnefe ms fento.
Il Perfiani sCrivendo al Serenissimo Principe D, Lorenzo dice.

do, che sono in arnefe tanto male,

    

site Mi ritravo in grandifsimo viluppo,

ics Teme efer pref in vece d' un galuppo y

via E finir la mia vita allo Spedale.

4 Franco Sacchetti Nov, 122. // Saccardo era guarito, e stava bene in arnefe. Bocce.

walt) 2+0. 8. Partitofi aljai povero,¢ mal' in arnc/eda colui, col quale lungamente crds

ato.
it DEL Romito de' Pulcini. Questo fu uno che abitava poco lontano da Mal-
- mantile, ¢ teneva vita eremitica, veftendo di lendinella a foggia di Francesca- |
yy nefealzo; Da coftu prefe il nome di Romito quel luogo vicino a Malmantile
ae che dicemmo sopra C. 1. stan. 70. E perché egli oltre al procacciarsi il vitto con
. chiedere gwinae aiutava ancora col autrire nella sua abitazione buon nume-
ai ro di Polli per vender.' uova, fu nominato il Romito de Lulcini, Quando I Aa-
he © torecompole la presente Opera, detto Romito era morto di gran tempo prima,
f © pero dice che il /aio che eg\i haveva addosso fu fino del detto Romito, volendo
inferire che era gran tempo, che qucli' abito cra fatto, ed in conseguenza oltre
, | all'effer vile per eficre-ttaco d' un povero Romito, era ancora lacero, ¢ confa-
oA mato dal tempo.
yil S AIO, Gonnelletto, o cafacca, o simile parte d' abito da huomo; dal Latina
Sagums. WVarchi flor. fior. lib 9, E forte il Lucco chi porta un faio, chi nna gabba-
gl nella, 0 altra.vefticciola di panno chiamata cafacca.
wis DICONO ch' ei dornsa inun granaio, L' Autore medesimo lo dichiara, segui-
a tando + perch? ha il mazzocchio pien di farfallini, se uno dorme, o si trattiene iny
ifet ua granaio, si fuol' empiere di quei farfallini che stanno fra il grano; e quando
id diciamo: I}taleha de' farfalliai,o delle farfalle,intendiamo E' mezzo matto; ¢
wif — dicetuello volante, 0 instabile. E per mazzocchio intendiamo il capo, perch?
wil mazzocchio era una parte del Cappuccio, che già portavano i Fiorentini, se-
,  "ondo chediceil Varchi nelle sue storie Kiorentine lib. 9, ll Cappuccio s dice egli)-
wt ha tre parti, cioè il maxzocchio,il quale ¢ un cerchio di borra, che gira, ¢ fascia intor-
wt! = no intorno alla refia, ¢ di sopra, foppannato di nero di ravescio, copre tutto ilcapo, Si
got dice a er mazzucco, © Così havea detto |' Autore, ma havendo il
yl - medesimoa dipingere uno dell' antico.Magiflrato di Firenze, mi domandd come
i era veramente l'abito Civile antico, ed 10 gli feci vedere questo Juogo del Var-
a chi, onde egli poi mutd, ¢ disse mazzocchio per quanto vedo dal suo feconds
.  Originale, che ¢ appretio di me~ 4
we moi IL,

 

 
}

  
 
  
   
  
  
 
  
  
    
    
  
   
 
  
 

192 MA LMA NTILED

IL mate da in fuora, Quando il male da in fuora, cioé 'man
te l'interna malignita, (uol' essere indizio di falute; cofui eflendo infer
pazzia, il dare in fuora di tale infermita ¢ il far pazzie; ¢ il Poet
potrebbe guarirne, perché il mal da in fuora, c1oe spera ch' et
olte pazzie, che ¢ lo sfogo del suo male, ed il fuodare in

  

ha tutti i suoi mefi. BY spropotiraco. Non ha l'iatera pe:
uello. Non è stato tutti a nove i meli nel ventre di sua madre a p
ceruello. In fomima vuol dire Non ha giudizio; ¢scemo. tf
£.4R ma ina. Diciamo far marina coloro, che fingendofi stroppiati, e
piagati gridano, ¢ si rammaricano per farsi creder tali; che tanto vale inigi
propolito Marinare,0 Jar Adarina, quanto rammaricarsi, o dolerfi di cola,
dispiaccia, ma per lo più s' intende di coloro, che fingono; come per
lo (colare battuto dal macitro,si dice far marina, quando fingendo che il
gli faccia gran male, piange, ¢ firide a più non posso; che di dice anche
monello, Vedi sopra C, 3. itaa. 67. ak
VADO a Scefi, Quando diciamo; I tale ¢.andato.a Scefi, intendiamo
to, se ben pare che diciamo è andazo alla Citia di Scefi, o Affi, p
ho scendere ci servc pec intendere morire, Virg. fucilis descenfus.
PEL mal, che viene in bocca alla gallina, M male che viene inbocca t
na da noi € detto pipint dai Lat, peruira, E perché fra da gence baila in}
dire apperito si dive appipio, pero cavano questo detto':. / tale ha stimal
in bocca alla gallina, c10t la pipita, © intendeno appipite, cioè fame. E

tende il Poeta nel presente luago con quelto derto piebeo. peo! t
ERAVANO. Cioè Averano Seminctti. Den Andrea Fendefi. Besdinando
Mendes. oat

PASCINA, Fascetto dilegne;Ed abbraciare insieme una fascinayy
ascaldarsi,, € spender ciascuno ja sua porzione nelle legne; E vuol dit 0+
pertamente andare all' ofteria, Oraz. Ligna /uper foco /arge reponens. 6
STRVZZOLO. Vecello noto, il quale mangia così voracemente, che it
ghiotcisce fino il ferro, Dicendofi veutre di fruzzolo stintende Ventre i
Vlin. degli struzzoli. Concoguendi fine deleitu devoratu miranatura, agat
AUNVZZ OLLI, Quci minuti fragmenti, che cascano dal pane, quando
spezza. E quest' atio di cercare i minuzzoli nelle tasche,e(prime uno che:habbia
grandissima fame. odes
GAGNOLAKE. Voce corrotea da cagnolare, che & il guaire, chefanno!
cagnolini quando hanno bifogno della poppa. Se per avventura non lo
vatfimo dal verbo Latino gannire, che signitica Rammaricarf con. parole no
affatto intese mescolate con sospiri., ¢ fingulti, che è quelio, che nel prefeate?
uogo vuol dir gagnolare. =
E SENZA namero ne i rulli, E' matto. Nel giuoco de rulli fipighi v
© pil, o sr eis SORT > a. de ses hail suo
che uno, il quale jama 11 Matto; E però dicendogi: 4 zale ¢ ih fengammme?
frairulli, i uathnide @ il rocchetto, che ¢ senza numero, cio? il mateo «Quel
rocchetti si chiamano radi, perché rizzati in terra in socal oa
a

nel mezzo, vi si tira dentro con un Zoccolo di leguo grave tondo di

 

 
 

QVARTO CANTARE: 193

midale, il quale si chiama rullo, € il giuoco si domanda «'Rutli, ed alle volte
Samer chi pil ne fa erlest i kee tiro vince. Si costuma anche tirare

dil no. wu
SRINPOREA | Ciotctelee lo Grider,

GRIDARE a

 

'tefl, Gridar quanto pitt

0 il guaire. L. ingeminat. Si raddoppia,

si pad. Si dice anche gridare 4 corr'huo-

mo,0 quant' uno n ha nella frrotea; nélle canna; o' nella gla. Vedi sopra C. 3,
Ranbgjeiq ish si. pasalty

TRASTVLLI. Trattenimenti*. E' voce da Fanciulli, ¢ qui vuol esprimere.,
che futiero veramente traftulli da bambini, perché aggiunge l'epiteto vani, come
era veramente il cercare de i minuzzoli nelle tasche.:

PER vedere il fine di lla feta. Per vedere in che haveva a terminare 7» Oa,
che fine' fuffe (aeoqelbnomice + Quando un discorso, o un fuono » 0 un Can-
tare, © altro romore cémincia a venirci a faftidio diciamo: Quando finird questa
Sefta; questa musica 5 questo chiaffo'; questo bordello; questo baccano; queffo mirscaiore
fmili, Vedi forto'C. 9: stan. §1.€€, ro. stan. 53.

GRVLLO., Int

'eadiamo'uho melancolico,sbattuto da cattivi effetti,e non affat-

to fano, che si dice anche Acquacthiato; E tal voce € prefa forse dalla Grue uc-
cello (Spyruila)che quando sta fermo posa un fol piede, ¢ tiene Pale baffe in ma-
niera', che pare un pollo ammalato; che pero tal pollo, ed ogni altro uccello
Cost'ammalato fi'dice Zruilo, 0 che porta i frafeoni, Vedi fotta C10, stan. 20.

SENTONO fuonar la lunga, Quando il Prete per

inuitare j popoli alla Meffa,

fuona la campana, e' dura Hs tempo, in contado dicono /uanar la lunga. B

da-questo durate lungo tempo

STANZA X.

Così domandan chi sia quei ch' esclama,
E metre grida jd urli st bespiali |
Glié dette; Quefioeun tale, che fichiama
Perlone dipintor de' miei ffivali,
Vahuom @al mondos'acquifta gran fama
Nel far de' ceffantts pe' boccali,

E con gt induspri,¢ dotti suoi pennelli
Suo nome eerno fa negli sgabelli,

icendofi: il tale fente fuonar ja lunga, s' intende
me per esser lungo tempo, che non ha mangiato. E
pertamente diciamo: Eeli ha quella de! Carmine, s' intende
Chiefa del Carmine di Firenze,avanti si dica la prima meffa
na-per un grande spazio di tempo, ¢ questo fuonamento si dice da tu
del Carmine.

Per significar più co.
la lunga, perché nella
,fuonano una campa-
ttl fa lange

STANZA XI,

Si trova in bale frato, ani meschino,
Ma ben che il furbo ne mancect pochi,
Ginocherebbe in fw pettini da lino -
Che ur'ora non puo viver ch'ei né iginochi,
Ma £ti vincelfe un di pur'un quatiring
dn vero si potrebbon fare è fuachi 7;
Perch giocando fempre Siorno,e notte >
Farebbe a perder con le tasche rotte,

STANZA XII,

Ginocoffi un suo fratel gid la sua parte;
Suo padre fu deleinoco anch'egli amico,
Pero natura qui n incaca l'arte

Havendo itato un genio antico,

Coftoro 'intesero, che'colui, il quale cos} gridava cra Perione

Coftni teneva in man prima le carte >
Che legato gli fuffe anco il belico:
Epriache mamma, habbo,pappa,e Poppe
Chiamp [pade,bafton, danari,e coppe,
> cioé Periones

Zipoli, che vuol dive Lorenzo Lippi Autore della presente Opera; ¢ fa che ven.

§a deicritto per uno sfortunato, ed oftinato giocatore.
i Bb

MET.

 
 

 

 

 
    

194 MALMANTILE,
METTE frida, ed urli beffiali, Stride, ed urla gagliards
perché lo fridere € proprio del porco ferito, ed wrlare &
cane, ¢ lupo; s¢ ben ce ne ferniamo anche per l'huomo i
DIPINTORE de' miei stivali. Pittore dappoco. EB'
ro, che sanno poco in qualfivoglia scienza, 0 arte. V
E frvale diciamo un huomo gotfo, ¢ di poco giudi:
scarpa, che cuopre tutta la gamba, es' ula per ca)
poco si dice Pittor da sgabelli, da boccali, da colombaie, ec. come si
fente ottava, che dice: Fa de' ceffaurti ne + boccali, Econ gl indufirss
eterna il suo nome negli sgabelli. Ma perché quelta sua modeftia, ed h
sia di pregiudizio al merito di così gran valent' huomo, repli
tore riputatiflimo, come le belle opere sue chiaramente teltifi
firera il sig.\ Filippo Baldinucci, se mandera alle stampe la faa,
Pictori, Opera degna d' ammirazione si per le belle notizie, che si
fa, esi ancora per saperfi, che questo erudito huomo |' ha ritrovate a
fieme in brevissimo tempo rubato alli tanti riguardevoli affari, che p
benefizio Jo tengono continovamente occupato,;
CEFF AVYTT/, Voce composta delle note Musicali Ce fa, wt, ¢ 00)
ficato veruno, se non che moftrandofi di dire la chiave del Cé fol fa at
Ceffo, che si piglia per vifo, 0 faccia, se bene appreflo di noi cefo vi
di cane, 0 grifo di porco, E quantunque venga forse dal Greco ©
dir Capo, onde anche i Latini, chiamano Cephalea un certo dolor di
in Franz. chef sia capo; nondimeno noi non ce ne serviamo se non peril
per intendere una facia brutta, e fatta male; © perdl Autore 5 ¥'
tenda, che Perlone dipigne male, chiama cefi quelle facce, che egli dipl
per altro parlando pittorescamente chiamerebbe Tefte. sped B,
bocc-dZE, E' una milura fatta di terra cotta invetriata capace deli
dun fiasco a, 3 ne ogni forta di vafo sia pil par
rande; che sia però di questa materia, e figura. E perch? quetti boccali da
ai, che gli fabbricano in Montelupo (on dione ted eae se senza un mal
mo dilegno, però a uno, che dipinga male si dice Pitror da Boceali 5 0» Pittoest
eMontelupo. oc et le
BASSO ffato,anzi meschino. Povero mendico; Poverissimo
FVRBO. Propriamente ladro dal latino. fur, ed & parol ingen
tavia si piglia per 4/tuto, (agace, caltrito,¢ che sa il conto suo: Qui vuol
fo, perché ha il vizio del giuoco, Fur a furuo, i, migro diétus, Papiat.
 AVE maneggi pochi, Intendi: maneggi pochi danari. Non gli vengs,
gran quantita di danari. 2
GIOCHEREBBE sx i pettini-da lino. Intendiamo uno, che giod
ogni iore scomodo, come farebbe, s' egli stesse a federe in fui
no, che fon composti d' acutissime punte di ferro., pote Ne
'SI potrebbon fare i fuochi. Si potrebbono fare i fuochi in fegno d'
come d' una cosa infolita, Detto usatissimo, quando si qualco
gusto, che fiamo stati buon pezzo aspettandola; Che si dice anche Sa
doppia, Vedi forto C, 6, Ran. 107. #

  
    
     
 
    
   
      
 
 

  
     
    
     
   
   

  
      
    
 

  

 
 
 
 

  

 
a

QVARTO CANTARE; 195
well PARERBE 4 perder con le tasche rotte. Perderebbe sempre: Farebbe a gara 24
ua chi pili con'te tasche rotte, quantunque queste perdano tutti li'danari, che

ha ineffe fimettono, 

 INCACARE, Disprezzare: La natura non sa grado, ¢ non ha obbligo «/'
pCéh arte, non eflendo flato opera dell' arte, che egli giuochi, ma effetto della natura,
éiy che" ha prodotto con questo vizio di giuocare. Dan. Pur. C. ro. disse;
eye Na la natura gli haverebbe a feorno,
wht VN genio, Vedi sopra C, 1, stan. 31.

“e PRIMA che gli fulfe legato il belico, Subito ch' egli ulci del ventre della madre:

j,i Bellico', Diciamo quella parte del corpo, d' onde è prefo i) nostro primo alimen-

coyitg £0 nel ventre della madre; 1a qual eae nel venire al mondo è¢ legata dalle nutri-

jumt@ ci. B cid serva per dichiarazione del presente detto.

fu Goa SABBO, Mamma, Pappo, e Poppe. Sono delle prime parole, che si profferi-

ceil = scono dai bambini, come s'é detto sopra in questo C. stan.s. Ma questo Perlone
prima /pade, bafton, denari, ¢ coppe, che sono li quattro fegni differenti

|
— Srasaoe arte da ginocare, che si appellano femi, come vedremo sotto C. 8.
. stan. 6, E qui 4 sn fa dire per moftrare, che prima d' ogni altra cosa questo Per-
gat Jone ona il giuoco, ¢ che venne fuora con cotefto a eee Ms giuocare.
a ZA XIII, A XIV.
jest Ma Toe voi sappiate il personaggio, E' swo amico, ed ¢ pur feco adeffo
a ofa Saluo Rofata un huom della sua tacca,

i cib'racconta,è il Franco Vicerosa,
pi i Cavaliero, del gual non è il | più, Saggio; Pero che anch ei sabbeverain Permefo,
Scrittor fubblime in ver/oyquate in profa; E Pittor paffa chiunque tele imbiacca;
mt Dipinge, ne pus farsi da vantaggio Tratra d ogni ers at ex profeffo,
, e in qualfivoglia cosa: E in paleo fa si ben Coviel Patacca,
cg Vince mel Canto i mufici più rari, Che fempre ch'ei si muove,och'ei favela
E nel portare ecchiali non ha pari. Fa proprio seangherarti le mascella,
' STANZA ¥ Vv.

ap 2
4 Hor percht Pranco, ed egli ogni maniera La dove minchionando un po la fiera
: 'Proceuran fempre ai piacere altrui, Mt Franco disse lor; Queffoé coli
iA Di Pertone dan conto ye, don' egli era, Ch in xucca non ha punto,anziragionaft
Di conserua n' andar con gli altri dui, Diappiccargli alta teffa un'appigionafi.
Acciò che si sappia chi ¢ colui, che da tal notizia di Perlone, dice; i
haveva nome Franco Vicerosa, cioé Francesco Rovai Cavaliere dotto, Poeta,.
7 Mutfico, Pittore, ¢ veramente dotato di quelle buone qualita, ¢ virtù, che dice
' jl Poeta,e che flanno benissimo in suo pari, come teftificano alcune poche sue.
 Poesie stapate dopo Ia di Ini morte, che non sono anche le migliori, che egli facefie
| Dice che nel portare occhiali non ha pari, perché haveva nafo aquilino aflai grande.
Con eflo & Saiwo Rofata, cioè Saluador Rofa huomo anch' egli dotto, e Pittores
eccellente, il cui valore ¢ notissimo, moftrandolo a baftanza le di lui stimatissime
Opere; ¢ aoe valeffe nella Poesia si conoscerebbe da alcune Satire da lui fat-
te, le quali ra vedere una volta alla tampa. Questo era amicissimo dell'Au-
tore, € fu causa, che egli tirafle avanti la presente Opera, persuadendoli, che»
era ce godere l'aggradimento universale, ¢ gli dette anche notizia de lo Cunto
degli Cunti pubblicato in quei tempi.  Saluator Rofa recitava da Napo-
B 2 Jetuno

!
che egli
PAS Br FRE, FM S

i

Quel! aggiunta di fera ¢ solita metteruifi, ma non so gt a

 

196 MALMANTILE |

Ietano in commedia mirabilmente, ¢ firfaceva.chiamare
sto Franco Vicerosa, e Saluo Rofata infegnarono dunq
defi chi, e dove cra Perlone. a7,
AVOMO della sua tacca, Huomo simile.a lui. Vniformi di ge
ca detta anche raglia ¢ un pezzo di legnetto feflo in due parti
quale serve per libro di conti a coloro, che non sanno leggere, ini
Vniscono dette due parti di legnetto, € nella parte più spianata f
tacche, 0 fegni col coltello, 1 - fegni denotano il numero delle ¢
credenza, 0 dei danari, che fidevono,, o de i lavori fatti, pezzo:
eflo legno rimane appreflo al creditore, ¢ l'altro appreffo al debitore:
si voglion dar nuoyi danari, o fegnare nuovi lavori, s' uniscone detti
vi si fanno i fegni che occorrono; E volendo aggiuftare i conti si
gni, ¢ si vede la quantita del debito, 0 credito: ne vi pud nalcere i
ché se in una delle dette parti di legnetto fara fatto un fegno di più 5 'a
far nell' altra, perché non riscontrera, se il debitore, ¢ creditore non!
dono scambievolmente detti pezzetti. Era in uso questa maniera di
anco appreflo ai Latini, che tal Jegnetto, che noi appelliamo Tagia yo
la dicevano tefera;: Swam uterque teferam habet; ratio conftat. Ha}
un' altra taglia, che chiamavano Tefera ho/pitalis, la quale servi
re gli amici, e corrispondenti di diversi pacfi, ferbando ciascuno
goetto; il quale si lasciava anche a gli Eredi; E quando andava
dell' altro portava la parte del legnetto;e unendolo & dava:a conole
te; ¢ pero detti legnetti crano cuftoditi diligentemente. Questo
Plauto in Pen, Ezo fum ipfus, quem tu quaris. P, hem quid ego audio?
gnatum eje. PB. Sé ita ef, Telferam me conferre bospitalem Si vis ceca a
tli, Donde havevano poi, T¢/seram frangere ho/pitalem, che significa
hospity. Dal che si cava, che homo eix/dem te/sere, sia lo stesso, che!
medesima taglia, che significa delli stetfi genj, ¢ corrispondente. Diguihi>
biamo il verbo attaccare, che vuol dire Vnire due materiali insieme 'ire
ho atagliare, che vuol dire Esser uniti di genio. Ricord, Mal. Sror-Fionapiy
dice: Lucca, Pifoia, e Volterra feciono taglia co' Fiorentini,.¢$' i
garono, © fecerolega; E si trova ne gli antichi noftti Storici &
lega. Om
PASSA chiungue tele imbiacca. Supera ogni Pittore. I co ee
FA sgangherar le mafeella, Fa ridere scegolatamente, che &,quel Rife quae
che dicemmo sopra C, 3. stan. 66. alla voce Pimmei. E veramente (
ne gli anni suoi più giovenili, che dimord in Firenze recitaya.(
detto ) questa parte di Napoletano così bene, che si pud. dire, che eglil
Maeftro in far questo Personaggio, wukseuee;
eANDAR di conserva, Andare insieme.. Detto Marinaresco,
significato.. wise oe
BU MINCEUON.ANDO (4 fora. E' il latino derideo, E tanto yale ilvebom®
chionare, che CO...... Che non si dice per eflere sporco, ed-usato ah Wt

         
    
   
  
      
     
 
 
 
   
 
 
    
 

  

Se

   

to fuona il folo verbo mixchionare, s¢ non che

pes

 

  
    

 
 

QVARTO CANTARE: 197
era, efler,detto da coloro, che non avendo voglia di comprare paffeggiano per
ie fete J del prezzo di questa, 0 di quella cosa, ¢ non offerendo tea

 te, 0 pochissimo; ¢ flanno a vedere, ¢ offeruare chi compra. E venuto poi a»

ee tines affolutamente, ¢ si dice ancora Adinchionare la Matter..
edi fonto.C, 7, stan. 15. EB pur qui ancora senza l'aggiunta di A¢arrea suonas

i | £W.xxcea non ha punto; cio' punto di fale, €s* intende: Non ha ceruello in te-
nie) © fla, Vedi sopra C, 1, Man. 53. 1] Mauro in lode della Cacia dice:
vVaetss Ed io, che sono un buom materiale,

  

| pe Cencande cit ben mafirerci cl io false

foots Da dovero una Zucca senza fale.
Catullo di Quinzia disse: 2

to.s mica falis,

= 0, $93 Wudla in tam magno off ci
odpiil ATT ACC ARGLI alla testa un' appigionafi. Efiendo ia sua testa vota;per mo~
firare, che ella si pud afficare si discorre a' appiccargli -appigionafi, che così chia-
mo quella cartelia,in cui fla scritco a lettere grandi APPIGIONASI, ¢ s' ap-

miamo que!
icca sopr' alle porte delle cafe difabitate, affin che si conosca, che quella è cala

mani
aia ha affittarsi, 0 appigionarsi, appunto come dice, che era la testa di Perlone, che
feaingt per esser yota di ceruello, era in grado da poterfi affitcare, 0 appigionare. Iny
lawl alcuni i d' Italia conferuano |' uso antico, scrivendo in L. Ef locanda,
ania  STANZA XVI. STANZA XVIL
cl Spiscqued (ito male ad ambi tanto tanto, Se forse dice; tu fei frato offefo,
y, E mentre @ piange, che si getta via, Che fai tu della spada il mio piloro?
ch pt 4 pietofa Eravan pianfe al uo pianto e-4 che tenere al fiance questo pefa
as! Verbigrazia per fargii compagnia; Per startene a mangiante come un boto?
ai Poi tutto liero postefecti accanto S' al corpo alcun dolor t° have/se poi
Ler cavalo di quella frencfia, Gli è qua chi vende I olia dello Scoro;
oo Di quelle firida, ¢ pianto si dirotto, Set' bai bifogno a! oro io ti fo fede,
Che quaifivegta Banca te lo crede,

,
gt. Che fa per nulla il bretolon mal corto.
ca  A coftoro dispiacque molto il male di Perlone, ed Eravano dopo haver com~-
' planta questa (ua 'disgrazia, si mefie a confolarlo, ¢ ad efaminarlo strettamentes
# per fapere la cagione di si gran suo pianto.
Ji BLETOLONE mat corto. Huomo sciocco infipido, fuenevole, appunto come è
la bietola: Marzial. 13. Vt /apiunt fatus fabrorum prandia beta, voce Sie-
tola, che viene dal Latino bera, che vuol dire una specie d' erbaggio, tanto nel
 nostro idioma, quanto nel Greco, ¢ nel Latino serve ancora per esprimere un'
# —huomo seiocco,ed infipido-, Laerzio nelle vita di Diogene Cinico dice così; Cir.
| cumfbantibus se adolescentibus eff dicentibus: Caveamus ne mordeat nos: Bono inguit
| essete anima filioli 5 carnis enim betis non vescitur. Plin, lib. 20. cap. 22. moftra, che
i mariti volendo dire villania alle mogli dicevano loro b/irea, raccogliendolo dal-
le commedie di Menandro; ¢ si legge in quelle di Piauto., intendendo una cofas
fsciocea, ¢ che non è buona a nulla; E come noi da bieto/a caviamo il verbo svie-
tolare, che vuol dire Scioccamente piangere.( Vedi sotto C, 7. stan. 93.) ¢ imbie-
solire, che vuol dire Commoversi, o esseminarsi. ( Vedi sotto C. 9. stan. 57. ) così
gli antichi hayevano berizare, che ha lo stcflo, o poco difference signiticato.
A Bie-

 
 

 

   
  
   
     
 
 

198 MALMANTILE \— i
Bierolone dunque fuona lo steffo, che Scimunito; ma con l'aggiu
vuol dire Scimunitissimo, perché 1a bietola cotta poco 5
della cruda
en; oe colui, = governa la Nave: dagli an
to Pedorto forse dal L. pedes prefo per remi, come apprefio Planto
© per funi da mie eolar efecto « Ma questa voce Piloto ci
mere un' huomo da poco, ae » irreffoluto, ¢ flemmatico;ed in
è prefo nel presente luogo, Vien forse in tal cafo dal Lar. pi n
mo, che per havere i piedi troppo piatti, ¢ contraffatti cammina male.
to C. 6, flan.
4 CHE portare, A che fine portare? Che occorre che tu porti?
facis? Ad quid venifti ? nel Greco dice eph' boo, cio per l'appunto
ST ARSENE 4 man ginite come un boro, Boti chiamiamo quei F.
tue, che si mettono attorno all' Immagini miracolofe per 8
ricevute,e però si dourebbe dir Yori, ma per iscambiamento di lettera si

Berni in biafimo d' un' huomo brutto.
Fuege da' ceraioli
Acciò che non lo vendan per un boto; ce
Che anticamente detti Fantocci si facevano di cera, ¢ per lo pits
giunte in atto d*orare, ¢ per questo dice /rarfenea man giunte come
s' intende d' uno, che non sappia, 0 non voglia operare, em
lavorare; ¢ vnol' inferire, Che fai tu delle mani, edella spada, che
doperi a vendicarti, se v ¢ stata fatta ingiuria ? Mons, della Cala
boto per modo di dirlo fempre. i
LO Scoro, \ntende di quel Ciarlatano, che vendeva Lattovarj 5
a veleni detto lo Scoto.
TE ls crede, Scherza con l'equivoco, dicendo ogni banca te lo
banca ti crede, che tu +habbia bifogno dell' oro, ¢ pare che voglia
banca ti fidera, o preftera l oro.
STANZA XVIIL
Dopo Eravano poi neffun fu muto,
CP ogaun gli volle fare il uo discorso
Offerendo di dargli ancora Aiuto,
Mentre dicefse quanto gli era occorso;
STANZA XIx.

Won v' è rimedio amici alla mia forte; =
” I. tutto è vano y gid che 1a fentenza MU soldato ciat nel ciabattino
E' stabilita in Ctel.della mia morte y Peri:che miscomuien.,

Che vnol ch'io muoiase muvia in mia presiza,

Già l alma ffivalata in fu le porte
Omai dimoftra a? esser di partenza.
Gid con il corpo tutti i sentimenti

Le cirimanie fanne 5 €4 complimen

 

   

Ed ¢ che foto fon come wn
Eldinnanzi a Minos,cagh
Rapprefentar mi devo co,

 

   
      
     
         
        
    
  
      
 

“di noi si piglia in diversi

QVARTO CANTARE. 199
STANZA XXL STANZA XKXIL
Wa ecco omai L bora fatale ¢ ginnta, Hormai di vita for nscito, e pure
"Chto lasci il mio terreftre cordovano; Lon trove al mio penar quicte,o céfarto,
Gid gid la morte corre che par' unta O Cielo Mondo, 0 Giove, 0 creature
'fo dime con la gran falce in mano; Dite, s' udiffe mai così gran tor to ?
inge ella il ferro nel bel fendi punta, Se Morte ¢ fin di tutte le feiagure,
3 iomancar mifento.a mano amano: Come alupar mi fento ancor che morte?
Pero lo spirto, ¢ il corpo in un fardello Ecome, dove ognuno esce di guai,
Tivo fuor della vita, e vo all' avello, Mi 8 agugxa il mulino pile che mai?

 Anche gli altri dopo Eravano gli offerfero il loro aiuto, ed egli fingendofi paz-
zo\comincia a dire una mano di (cioccherie,e moftrando di creder d' efler morto,
si maraviglia, che mors, qua omnia foluit non gli habbia levato l'appetito di ci-

HAVERE feorfo col ceruello, Esser' impazzato. Haver dato la volta al ceruel:
Jo. Metafora tolta dail' orivolo a ruote, che si dice gualto, quando le ractes
scorrendo escono del lor moto regolato.
APFFISS AR gli occhi in uno. Guardare senza punto movere gli occhi; atto da
pazzo di quella specie, che domandano Maniaci.
ALLA mia forte, Di quel che m' ha da fuccedere. Questa voce forte appreflo
Woidcari » come seguiva anche appreffo a i Latini, da i
quali fidiceva fors ogni avvenimento di Fortuna. Cic.lib.2.de Divinatione. Quid
enim [ors eft? idem propemodum, quod micare, quod talos iacere, quod tefseras, ed in
fenfo & prefo nel presente luogo. Si dice tirar le forti,per intender quel

super veftem meam miferunt fortes dell' Buangelifta.

La pigliavano per carica, o incumbenza, fecondo Livio: Si id gravaretur fa-
sere, quod non sue fortis id negotium e/set.

La pigliavano per stirpe, fecondo Ovid. 6. faft.

a; Si genus aspicitur, Saturnum prima parentem
. Feci, Saturni fors ego prima fui

La dicevano anche il capitale, ¢ quello che noi pure diciamo forte principale;
Plaut. Mott, Quatuor quadraginta illi debentur mina, Et fors,& fenus DA,tantum eff.

Altre volte pigliavano /ors, pro ixdicio fecondo Verg. 6. Aneid,

Nee vero be fine forte data, fine indice fedes,

Perché ( fecondo Servio ) non s' udivano le caule wif per fortem ordinate, nam.,
quo tempore cause agebantur conueniebant omnes, 7 ex forte dierum ordinem accipie~
bant, quo post trigefimum diem causas fuas exequerentur,

Dicevano forte gli Oracoli, 0 risposte, 0 le polizze sopra alle quali si (criveva-
no le risposte. Val. lib. 1. Cuins rei exploranda gratia legati ad Delphicnm oraculit,
vetulerunt: scent fortibus, ut aquam eins lacus emifsam per agros diffunderent, Virg.
in questo fenlo disse: Lycie fortes. Apprefio noi aucora ( come ho accennato )
Jorte si piglia per fortuna, 0 destino, per condizione, flato, 0 essenza. B dicia-
Mo toccare in forte, che significa ottenere la benefiziata, quando s' eftraggono
Ie polizze, che ¢ quel mittere fortes; ¢ se bene in significato di fortuna vogliono
alcuni, che si debba dire forte, ed in significato di qualita, 0 condizione fore,,
hoggi ( almeno nel parlar familiare, ¢ Civile ) non trovo, ches' usi tal distinzio-

ne,

ib aa
200 MALMANTILE —

ne, ma fento usare alcune volte ' una per l'altra indifferentemente} ) ~
ClLABATTINO. Vno che raccomoda scarpe rotte; iabatta
re Scarpa vecchia, ¢ scarpa all' Appostolica, che (ono quelle, che:
Cappuccini. In molti luoghi de' contorni Fiorentini chiamano C
ra quelli, che fanno-di nuovo; che noi chiamiamo Calzolai, in Up
similusente zapaceros; € questo nome di Ciabatraviene
cioé (carpa ferrata con chiodi; quali (on quelle che usano i contadi
ciatori. i
TIRAR le quoia, Havendg detto, che di soldate doveva diventar
la ragione perché; ed ¢ questa, che gli conuien tirar le quoia, come fanne
battini, ¢ 1 Calzolai, che tirano i quoi per condurglia quella mifura, ¢ i
no, delle quali quoia dice, che si dee servire per rimcalzare it pino., cio' fark
scarpe al pino. Nota che Jo scherzo dell' equivoco, nasce da rirar le quoia,
vuol dir Morire, ¢ rincalzar con efse il pina, che vuol dire Parsi forerrare
del pino, ¢ così alzandogli la terra attorno rincalzarlo., che questo vu
ealzare un' albero. Ofierua ancora, che facendolo parlar da pazzo'
coloro credano, che egli habbia concepito nel cerucllo questo spro 1
a fare le scarpe a i pint; perché quando un Calzolaio dice; Io calzoil ti
tende Io gli fo le (carpe. Plut. in Dem, E ca/zandoft dices, Il Gr, erepidas si
SOTTO, fon come un cammino. Sono schifo, ed ho le carni fudice,
cammino, dove si fa il fuoco, Comparazione usatifiima particolarn

donne. W s
MINOS, ¢ gli altri Gindici, 1 Giudici dell' Inferno fecondo le
tichi Poeti, ¢ della Gentilita sono tre, cioé Minos figliuolo di Gi 3
a, che fu Re di Candia, Eaco, che fu figliuolo di Giove, ed' }
¢ d'un Ifola già detra Enopia, la quale egli poi dalla madre chiamo Bgi
Radamanto, che fu figliuolo di Giove, ¢ d' Europa, che fu Re di Li
sti Re,perché furono feveri amatori della giuftizia,dicono i detti Poeti
tone gli cleggefle per Giudici dell' Inferno, affinché efaminaffero } anime
aflegnaffero loro le pene, che meritavano, ¢ da quello, che di loro ferive
2a. 6. si pud comprender il lor precilo, ¢ particolar ofizio 5 che di
Quafitor Adinos urnam movet, ille filentum
Concilinmque vocat, vitas, & crimina discit,
E di Radamento dice;
Gnofius hae Rhadamanthus habet durissima Regna,
Castigatque, anditque dolos, fubigitque fateri, §
D' Eaco parla Ovidio così;

 
  
 
  

 
 
   

 
    
 

   

  
    
   
  
   
   
 
 
   
   
 

    
   
 
   

ytd

 
    

 

Tualque ae

Eacus in penas ingeniofus erit, td
E conchiude i} Poeta, che uno di questi Giudici efamini, 1" altro giudichidl
zo mandi ad esecuzione. Se ben Dante nel 5. dell' Inferno.dice:
Stavvi Minofse orribilmente, e ringhia, ss
Efamina le colpe ned entrata, aa a
Giadica, e manda fecondo ch' avvinghia. oi

CORDOY-ANO. Specie di quoio da fare scarpe, la concia del uae a for

 

 
 

QVARTO CANTARE. zor

inventata in Cordova,e percid tali quoi chiamanfi riamente cordovani,e fon
poll di Castroni podta "re: qui ioteade, pelle humana', ¢ dicendo
tafei ly cordovane intende io muOia, come intendon quelli, che dicono
Terrefire falma;T errena /poglia,e simili,Cunto de li Cun, Peffoe concio per cordonano.
CORRE che xnta., Corre velocemente; comparazione dalle carrucole, ©
puleges » 0 altre simili, le quali quando sono unte con olio, fapone; 'o altro,

nate ' a os:
PALCE. Strumento,col quale si fega il fieno; ¢ col quale speffo si vede dipin-
ta lamorte com efla in mano. anne A>
GVAl, Travagli, fuenture, sciagure, afflizioni. Vedi sopra'C. 1, stan. 28.
(ALLER ARE, iiceor gran fame', perché dicono, che il lupo fempre habbia
gran fame; quindi il volgo chiama Male della Lupa quello di coloro, che fem-
pre mangerebbono, perché da joro vien preftiftimo fmaltito il cibo con pochiffi-
mo nutrimento, ed ¢ quelia infermita, che i Medici chiamano Fame caninaa.
Vedi forto C. 5. stan, 61. E da questo male chiamato della Lupa diciamé allnpare
uno 'che habbia gran fame -
\AAGVZZARL it mulino, Par-venire s0crescere V appetito: perché aguzzare
da-macine del mulino yuo! dire Metterla in tagiio in maniera, che si renda più
dngorda.. Vedi fatto C. 7. tan. 31.

STANZA XXIII.
Va'a dir che qua si trovi pane, o vino
O altro da infegnar ballare al mento;
1 Se nonifi fain cenaidi Salusno,
» Qaaato a iar non te afeenamito,
oO iepecdineen 70 i ¥
yuihavete a ireal monumento,
Vor l intendete, che nel catalerto
Com voi portate il pane,ed il fiaschetto,
oa STANZA XXIV,
¢ Let pet old dal cimitero,
S' it Ciel danari's e fanita vi dia
Empiete il buxzo aun morto foreftieroy
O infegnateli almena un ofteria;
Se ben voi fare qui fempre dinero,
Perché di carne havete carcftia:
E' tale appetite che mi scanna
Chiun Diavol corto acor mi parra mina,
STANZA XXV.
Se ben non c ¢ da far cantare un cieco
| Diguefta [pada alOfte foun presente,
C'ad ogni mo, da pai cht ella sia meco,
Mai bate colpo, o volle far niente;
Per una xuppa dollaancor di greco,
Mache gracchioia? Qui neffun mi fente.
Che fo? s* i morti fon di pietd privi

STANZA XXVI.

Qui racquese per fuggir la via si prefe
Facendo fempre il Nanni,ed il corrive,
Perch'eglie un di quei marti alla Sanefe,
C' han fempre mescolato del cattivo;
Per haver campo a feorrer il pacfe
We fere pol di quelle con l'ulivo
Miffrando egn'bor pitt dar nelle girelle,
E tutto fece per faluar la pelle.

STANZA XXVIL

Perch uno, ch' it soldato a far #  meffo,
Mentré dal canipo fiigde, e si rravia,
Sendo trovato, vien senza proceffo
Caldo catdo mantdato in piccardia;
Però £ ei parte non wiiol far lo Refs,
Ma che lo feufi, ¢ falui la parria,
Onde minchion minchion factdo itmatto,

Se ne [cantona', che non par suo fatto,
STANZA XXVITL ©

4M Fendefiascappare anch'ei fu leffo

Con gli altri tre correndo a rompicollo,

Volendo rificar prima un caprefto,

E maorir con ta fromaco farollo, A

Che refar quiviiamenarsi Pa...

Ed allungare a quella fozgia il cello;

Li danno certa è fempre da fuggire,

Meglofara ch jotorniaftartraivivi, Co S'egli avvien peggio puinonc'e che dire,
 

202 MALMANTILE

Petlone seguitando a dire i per esser tenuto matto si 3 e per fale
var la vita continovo a fare ie iioctiee ie, sapendo, che un to aaa
pa dal campo, ¢ si parte senza licenza ¢ reo di morte, ed il Fendefi 5 ¢ gli altri
scamparono anch' essi. i Seek

Vda dir che qua si trovi, EB! yanita il credere, 0 dire che qua si trovi;s' ingan-

na chi crede che qua si trovi. “vy

INSEGNAR baliare al mento. Mangiare. E' lo fleffo che Dar il portante a'
deati detto sopra in questo C, stan. 6. q

FAR la cena di Saluino, Andare a letto senza cena; che la cena di Saluino era
Pisciare, ¢ andare a letto. *

O SER H/ac,o Abramo, 0 Iecodino; Lptende tutti gli Ebrei, ¢ seguitandd ¥ opi-
nione del volgo, il quale crede, che quando gii Ebrei feppelliscono i loro morti
metiano lore appreiio del pane, ¢ del vino dice; Voi / intendete che morendo por
rate con voi il pane,e il vino, poiché nel mondo di qua non si trova ne da mangia-
re, ne da bere. Wig

CAT ALETTO, Quella barella, entro alla quale si pottano i morti al sepol-
cro, che i Latini dicevano fererrum. Voce compolta di Lertoye Cara preposia. Gr.

ORBE, old, alo, E simili; sono voei, ¢ termini usati per farsi (entire da chie
alquanto lontano; come fa il Latiao eas, Orbé, ¢ fatto da Ora bene; Or beat
Latino age vero; Alo dal Fr, ailons; andianne.

CIMITERO, Piazza, nella quale si fanne i sepoleri per li morti, Voce che
viene dal Greco she, che suona dormire y rip » Onde jon, lo
stesso, che Dormentorio. Quindi i Cretenfi chiamavano Cimeterid una calas
pubblica, la quale ferniva per alloggiare i pellegrini. Vedi forto C, 7. stan. 27.

S°LL Ciel danari, ¢ fanita vi dia, Dice questo sproposito per acerescere in
ro la credenza, che egli sia matto, sapendo bene che i morti non hanno bifogno
di fanita, ne si curano di denari. gt 04

3VZZO. \itendi il ventre dell' huomo, da bufto che s' intende tutta quellas
parte del corpo humano, che è dal collo al pettignone, senza le braccia. -

FAR di nero, Mangiar di magro. I venerdi, abati, Quarefima, ed aitre vi-
gilie si chiamano giorni neri, quaG giorni di lutto destinati alla penitenza, il
Poeta scherzando con l'equivoco del acro, col quale è solito farsi !apparato a'
morti, par che voglia dire non mangiate mai carne, perché foggiunge di carne
bavere carefiia, ¢ par che intenda non havete carne da mangiare, ¢ vuol dires
bon havere carne in sa l' ofa, perché i morti in breve tempo reftano puri (chele-
tri senza carne. a

e4PPETITO che mi scanna, Fame così grande, che mi fa morire, che mi fa
perder ja canna della gola; che scannare uno, vuol dit Tagliarli la canna della
gola. Cunto de li Cunti Giorn, 1.\Se la necefjita non la foannava, 6

MI parr manna, Mi parra buonissima; come paruc, ¢ sua gli Ebrei la Man-

na, che mando loro Dio nel deferto, che ricevendola elclamayano Adanu cio. -

'Che è questo ? onde forti il nome.

WON ho da far cantare un cieco, Non ho ne meno un quattrino da darlo a un —

cieco,perché canti un' Orazione.:
LA ogni me, Per: ogni modo. E' termine assai usato in Birenze in diversi fen,
per:

 
      
    
 

=

EF ar eae aoe

CaS pc FF ES.

LL ER FEE

 
 

 

 

QVARTO CANTARE: 203

ereeeaeens come nel presente luogo, Haglio dar via la spad.,
4d ogni modo non baste mai colpo, cine perché io non la fimo per non hayer
i rato, O significa ne ceffita di fare, 0 non fare una cola per efempio,
a » che ad ogni modo s' hada morire. Significa contentarsi di
' sche uno ha confeguito: fo. ho guadagnato poco, ma ad ogni modo io mi
contento. Significa Oftinazione. So hate anaes po muocere 5 ma La voglio
fare ad ogni modo. Vedi sopra Can, tr. stan, 3. il termine; /vo danno, che pat che
habbia correlazione al termine, A ogni modo, V.gr. Se '0 ho perduta la tal

' anno; ad ogni modo io non mene servivo, E quel mo per modo &
Ta figura da noi molto usata come vedremo altrove.

tAl batsé colpo. Diciamo + il tale non batte mai colpe per intendere, il tales
ora mai, ¢ qui intende, che la spada di Perlone nelle sue mani non lavo-

  
   

 
    

 4¥PP A. Pane intinto nel vino, 0 in altro liquore, Forse meglio Suppa, Fran-
0 Sacc, Nov. 86. E fatea la fuppa con le spexie,subite porta in tavola il ventre, ¢ la,
Lippe. Stimo che venga dal Tedesco /uppen, che vuol dir Brodo di carne,o ¢'al-
tro, che si quoca leflo. In questo fen/o una forta di mineftra chiamiamo xuppa,
Lombarda., Vedi sopra C.2. stan. 7. Ma l'uso ha introdotto i) dir corrottamen-
tezuppa,¢da molti ineuppa; come 2ulfa,€¥ézz0, ¢ zinfonia in vece di solfa,
C2z0 5 | ia, ¢ simili.
 GRACCHIARE. Dilcorrer senza proposito, o profitto. Da Graccio Latino
us. Utale mi chiefe dieci scudi in prefto, ma 10 lo la(ciai gracchiare. Ve-
10 C, 7. stan. 59. ¢ C. 8. stan. 65. '
"il nanni, ed il il corrive, Finger& corrivo, goffo, femplice, baséo.
MATT alla Sanefe. Si dice Sane/i Matti, ma in effetto fon pil fagaci degli al-
tri ene Matti alla Sanefeye' han fempre mescolato del cattivo; cioè dell' aflu-
fagace, ed ingegnofo. '
fece di quelle con I ulivo, Fece delle (cioccherie grandissime. In alcune fo-
a » fuole la generosa pieta del Serenifs, G. Duca liberare dalle carceri aleu-
icon pagare il loro debito, o parte di eflo;¢ guchti tali vanno procef-
tea render grazie a Dio al Tempio della Santifs. Annonziata, o di S.
10: | 5 ¢ quelli che hanno pagato tutto il debito, ¢ (ono affatto liberi por-
tano in mano un ramo d' olivo a distinzione di quelli, che per non haver pagato

 
 
    

   
   

tutto il debito, ma parte di esso devono tornare in carcere, i quali nou hanno
Volivo in mano, ma fon legati. Da quetto ramo d' ulivo, che ia tal congiuntu-

rm to inter, credo che sia nato il dettato; La tal cosa ¢ com I'uli-
%, che significa cosa grande nello stesso modo, che i Latini differo pa/maris, ed
f un' azione ardita, che diciamo anche marchiana; da pigliar con le mole,
€,, Come s' intende qui, che vuol dire, che questo fece cole grandi, ed ardite.
DAR nelle girelle, Impazzire. Vedi sopra C. 3. an, 43., ¢ fortoC.9.Man.10,
MANDATO in Piccardia caldo catdo. Impiccato subito prefo senza far pro-
Gelso: Calde aldo subito, ¢ prima che la cosa si raffreddi. Piccardias,

“in ipl ardore criminis, Provincia della Francia, serve, scherzando con la.

Amulitudine della parola, per intendere émpiccare. I Latini pure havevano
MN termine coperto per fare ome impiccare 5 che era diteram longam facere,
oy cz come

 

 
— 2° Rey: Cena are

  
    

204 MALMANTILES / >

come si vede in Plauto; il che ha data occasione-a molti letterati didilcorrere per

chiarire qual fulle questa lettera  ¢ Celio Rod. leét, Ant:

chinde, che fufle il T maiuttolo, che dGmite alla forcay che facevano

Noi ancora diciamo': Andare * Lamia chore un Porto in Toscana' 'te ae

ligno, ciod a fune'y ¢legno; Dar de' calei wl vento: “Ballar incampo:

C2, fan, 65. Ballar 'nel Paretaio web Nemi (00 Cy an, yor B tute:

no Bfer-impiccato. saath t

@UINC HIONE; Da imitichiet detto sopra in guefto a stan. 15.

SE ne feantona, che non par fio fatto. Se neva via e ee ean

questo per andarfene, E fete quell' Agere se di Te ih

CORRER a rompicolo. Cortet velocemente; € 3 precip iio senza wonbiderates —

la strada buona, o cattiva,
ARRISCHLARE un caprefto, Avventurare a essere impiceato, Corre it

sto il rischio a' andare in fu le forche, che aa di morir di fame.;

 
 
     
   

Ss oom see oe

ss

   
 
 

   

  
 
 
 
 
  

MENARSIU A...... Perder il tempo senza far nulla. Se vuoi intender
ne questo.detto, leagi difeor(o d' Anibal Caro in tin di Ser' ee
STANZA XXIX. TANZA 1!
Lasciam coftoro, ¢ vadan pure avanti Danae ¢ i guerriero, € via'
oe il vitto Li per quel contornos Cavalcando ne va con fefa, &

Che se fame glicaccia ye "fon poi fanté Ognor tenendo tt chirarrino in
Da batrerfi ben ben Seco in un forno: Perché il viaggio non ¢1 a
Perched! un era guerrier conit ch' io cati E' bravo sh, ma'poi buon
Ate2z0 impaniata, perch'eglihad'iterno E farebbe servizioinfino
Vasa donna firaniera in veffe. brane, Venga chi vuol, a tute
Ches' afjligge, ¢ si duol della forenna Se bene fuffeit Carat FB svecchit
STANZA XXXL ' a
Poiché bella ¢ colei che si dispera Percio convifo ar, sana cera
Sempre piangendo fens.' alcun ritegno y Par unEbreoe® beth ee
E vanne,come to diffi, in cioppa mera E as quanto l'affligge,e [a trave
Ler dimoftrar di sua mespiziail fegno, lagrille il Campion quivh
11 Poeta lafeia il discorso di quegli afamati,'¢ si mette 4 narrare Ja' fav
vettita di Pfiche; la quale chiede aiuto 2'Calagrillo, che ¢ Carlo Galli
di Cavalli' 5 gli racconta'i suoi travagli. -
SON fanti, $? intende fon huomini c' hanno cuore, € 'pirito da 'fare quella
tal cosa; ¢ da pigliare ogni risolazione.
Dat batterfi ben ben feco in'wn forno. Da combattet con la'fame' <>
un. forno pien di pane, © mangiandofelo » vincerla', ¢ facia
MEZZO impaniato., Ubrogtiato; Intrigato: Trastato da | alin

   

 

   
      
 
  
  
 
 

SPLF2F RESTS Een eee EE

   
 

 

  
 

  

    
    

vendo toccata la pania, volano'st, ma con Uifficulta per tim
. Joro la*pania y.che hanno sopra alle penne.' +e
BVON papricciane. Huomo dolee, groffotano,"huomo alla bubint Pisfriccia:
wo specie di Pastinaca. 1 dettorantico ¢ Buon pasticcione, cioè eee:
Placidus tanquam aqua filens.
BATT I feravecchio, Molti vogliono, che si dica il Bratti ferra'
le fa aa haomo facultofo, ma di cattiva fama: Coftui lascid poi tutto

  

2 meric 3 sana
 

SSSRew

aS S58 oe

es tS SS

a
3

ee &

QVARTOVCANTARE. 20s.

C di fecolariintitolata in.S, Giofeppe, perché delle renui-
te fine'; come segue fino al. di d' hoggi;:ma.a me pare,
io hia dire il Bares; ¢ il Batti, ciod i Bactilani s quando noo poilo-
no pii rare "altfa arteyy si mettono.a fare. il rivendirore di
cenci,¢ ferri vecchi » e-dall' andar gridando per ia Gitta Chi ha ferri vec)', han-
ito il nome ! divFerraveccine ~ E perohé queste sono vilissime persone aed
lle quali si ha poco riguardo; + quando vogliamo esprimere, che uno sia di maa-

  
    
   
 

fueta,, ed umil natura, ¢ indifferente.con tutti, fogliamo qualificarlo con questo
termine | Salute' yofarebbe fernixio, anche al Batti ferravecchio, Che se dicette il
calzerebbe tanto bene; pera finalmente il Braeei, fu persona di qual-
che siguardo', © Civilta... Zmbratra soprannome trovali ne) Bocc,
RESACELE E' nota la favola di Ai pedecleriita maravigliofamente da Apulcio,
il Poeta incalftra in questa sua Opera, ¢ l'immaschera assai aggiultaca-

(Pio aren. Vito aspro sche denota dolore, o altra paffione travagliofa Lar,

TMITT Aces, era, Haver brutta, 0 cattiva:cera vuol dire Faccia, che dal suo
indi¢hi poca fanita, 0 grave disgufto., che travagliando I animo,
il corpo 5 E bratra cera yuo) dit ancora Fifonomia cattiva.

un' Ebreo c' habia perduto il peguo. Quand' uno per qualche disgufto mo.

i¢a ci serviamo di quefio detto, perché o sia vero, o sia no-

fra opinione'y rarisiimi sono gli Ebrei, che habbiano faccia allegra; ma un' E-

breo che habbia perduto i} pegno aggiunge melanconia.a malenconia, ¢ pero
moftra facia. deformatidima.

SPANZA XXXII STANZA XXXIIL

Fn vg incta ) devi fapere E quando poi io 'bo bell, € trovato,
wr bel marito, ma perch' io Martinarra, chre fempre lo Scompiglia,

Die eis era contro al suo volere Fa si. che pur di nuovo m'é è foappato,
er fet anni n° ho pagato il fio; Ed in mia vece.ali' amor suo s " appigtia,
aallor per farmela vedere Tal ch'io rimange cacciator seraziato;
Stigtate meco fer' ando con Dio Scnoprolalepres.e un' altropoi lapiglia
a lingo; ebea volerle ritrovare Ti dico questo; perchi hanrei voluto
trier volea da ene e Che tu mi deffi a raccatrarle aiuto.
STANZA XXXII.
a ie pomaree ginra sch il marito Edd ella lo ringraziaye del, Lfeguito
' pero non si feomenti. Di sante sue fatiche,¢ parimenti
“Ef ndpteed « quel che ha smarrito, ( Fata più lieta per le sue promeffe )
¢ [ei bifognando,e dieci,e vents. Così da capo.a raccontar si meffe.

Psiche espone a Calagrillo il suo bifogno,e lo richiede d'aiuto; Ei glielo pro-
mictte 5 ed ella fatta alleere per tal promefia, incomincio.a discortere » Narrando
tutte le-fatiche 7 ¢ difagi meee -da lei in ricercare del Marito.

WV-HO pagato il fio..N' ho pagato la poms 3 ¢ibLat. penas.dare, Fio & voces
Fiorentina antica, che ant dir fexdo. Gio, Villani lib. 5. cap. 1.Scomumicd Fede-

Afoluette tutti li suoi Baroni da fio, e (aramento, ec, ma da noi hoggi nor.»
ee eneiene 3 nel qualc anche iusd Dante Purg, C, 10, es

 
206

MALMANTILE |
Di tal superbia qui si paga il fio. >»

Ved os Bey aA

Cl ebleva la carta da Navicare, Bra impomibit ritrovac sgun bag. fone haves,

la carta da navicare, ola buffola.

ZL! HO bell'e srovate, L ho già trovato. Vedi sopra C. 3. fan. 14 a foraa di

guefto addiettivo bedo in questi termini.

4M' HA feartato, M' ha rifiutato. Traslato dal giuoco delle carte, 'che ae

do una carta, che habbiamo in mano non fa per noi,la buttiamo s
delle carte; il che si dice scartare, vedi foto es i
e4 RACCATT-ARLO, Cioè ritcovarlo, riaverlo, ricuperarlo. Il,
significato di raccattare ¢ Ragunare, mettere insieme »:
NON si szomenti, Non si perda d' animo, non si sbigortilca. Petr. ete vila
E fol della memoria mi sgomento.
Dante nel Purg. C. 14. in tignificato attivo. '
Cacciator di quei lupi in fu la riva 2 e ih

     
  

—
. fan, 6. alla voce

Vedi sotto C, 1

 

 

i 65

Del fiero finme, ¢ tutti gh sgomenta, * gh
SMARRIRE. E' un certo perdere con speranza di ritrovare. Dan. Inf, C, rE
Che la diritta via era smarrita vide

QVATTRO, fei,¢ dieci., e venti. Scherza facendo,, che Caria proses

più di quel ch' è richiefto., come fanno tutti i bravazaoni,¢ in tanto
a una bella donna non mancano mariti.

STANZA XXXV.
Cupido è la mia cara compagnia,
Riccogarzon,feben la carne ha ignuda,
efnzi non è, t' ho detto una bugia,
Perch'ei no' mi vol piie covta,ne cruda,
Ma senti pure, ¢ nota in cortefia:
Quando la madre [ua ch'era la Druda
Del Fiero Atarte,ideff la Dead amore
Gravida fu di queffo traditore;
STANZA XXXVL
Perch una srippa havea, che conueniva,
Che dale cigne bomai le fulle retta,
Cagion ch'in Cipro mai ai cofausc.vs,
Se non con i braccierijed in Seggetta,
Pur fempre co. gran gente, € comitiva y
Com' a Regina; com' elle, 8 aspetta,
i J paesi ha dietro,e gli fraffier dinanzs,
E dagt inlati due filar “2 Lanzi.

Se non ch ¢ miei m è pase
Mio padre ch i i ne lo scanna y
Con un mio Zio ch' andava zientey

i Cathy

TANZA

 

 

STANZA XKXXVIL )

Essendo cos: fuori una mattina se
Per suot negorzi se pi
Vito per cafo nna Vacca Trenina >

E tocca a pena, in terra la
Ond' ella un alta rammanting >
Perch'unalinguaellhache ragliase fede:
Va, che tw facia, quando ne sia otta
Vn si Leos ice) in forma a wna betta
ZA XXXVIIL
E ae oh? in-vece d'un bel figiia
Di suo gusto, edi tutti i Te ARAB
Vn rospo fece come un pan di miglies
Cc ee fatto flomacare i cant;
Che poi cresciuto, feceft consiglo
Di dargli un po di moglie ma i merzant
Non Pela pee
ae ee ne voleffe, ofentir nulla
Sperando tutti tre d' ungere il dente,
E dire: O corpo mio fatti cs >
E riparare ad ie lor co

Me gli oferiro;e fecefi impiafiro.

E un mio fratello anch'e
eect Psiche a Calagrillo la dolorofa storia, ¢ facendofi dalla nalcita si

Cupido dice, che nacque in forma di rospo per la maladizionc d' una veschia»€ i
chy poi cre(ciuto fu a lei dato per marito.

   
  
   
   
    
   
    
 
  
 
   
  

Sis cS ee ee 6 ee

egg?

(FEZ EEFE SHEET

SEH 6

£.
 

 

QVARTO CANTARE: 207
een eas Hida Nealeffo, nea rofto. Non mi vuol pi ia,

 
  
 
 

yl Tr. lib. 2. fan. 42. dice:
) Nom gli volle annafar cradi, ne cotté
'DI « Innamorata, tanto in bene quanto ia male; perché si dice amante,
4 » damo, non fempre in significato difonefto.
Da C.12z, Dentro vi nacque L amorofo Drude
RIE j: Delia fede Criftiana il S, Atleta, Parla diS. Domenico.
Se bene nel presente Inogo s' intende Meretrice, concubina.
| CIGNE. Sono firiscie di quoi, od' altra materia adattate a foNenere, ¢ te-
"ere insieme qualfivoglia cosa, dette cigne, da cignere.

 BRACCIERL, Coloro, sopr' alle braccia de' quali con una mano s'appoggia- -

no le Dame andando a piedi per la Città.
 DAGL inlati, Dalle bande, da i lati. Idiotifmo usato assai in /ati per lati.
LANZI, Così chiamiamo ii soldati Tedeschi della guardia pedeftre del Seren.
G, Duca. Vedi sopra C. 1, stan. 52.
VACC-A Trensina', Così chiamiamo certte donnicciuole poco honelte, sfac-
- Giate, ed ardite, che non portano rispetto a veruno; € credo che si dica così pez
militudine, che hanno con le Vacche di Trento, le quali per esser' avvezze a
fempre per le campagne del Tirolo, (ono faluatiche,  fecoci.
— Re IANZINA. E*\o stesso, che rammanzo detto sopra C.1.st. 52.,¢ che
mbbuffo nel med. C.t 39. Da alcuno ¢ definita così: Riprentione fatta con parole
; Minacceyoli, ¢ ingiu:iole.. Forse dalle dicerie de' romanzi.
oy yep ieee ae Ȣfende, Ha una cattiva lingua, che dice ogni forta

|

   
   

 

; 'male senza rispetco, o riguardo alcuno, che lacera  alerui ripurazione.

4 HAVREBBE fatto spomacare i cani, Così sporco, ¢ nefando, che haurebbew

7 provocato il vomito fino a i cani per la sua schifezza. ia questo feafo i Latini
“pure fiferuivano del verbo fomachari.

s  - | DARGLT un po di moglie. La voce poco ¢ usata da noi in diverse maniere; 0

w  declinabile, che significa quantita, come dureg/i un poco di carne; O indeclinabile
Der % io; come andare un poco a Roma; Dategli un po di moglie, ¢ serve per

- eufafial discorso, ¢ non per quantita, potendofi dire andare 4 Roma: Dategli mo-

flit, che tanto esprime schza la voce poco, 1a quale però nel presente luogo non
tipithezza, 0 (come diciamo ) borra; ma € così detto per moftrarne l'uso,

. che apprefio di noi ¢ frequentissimo, ma nel cafo come il presente è tanto usato,

6

*

%

'

w —'tillima da noi in questa, ed in altre voci enunciate sopra C. 1, ttan, 36.

oh | MEZZANT, Senfali. Coloro che sono mediatori a conchiudere ogni forta,
y afare,

AL bifegno ne lo Jeanna. E? poverissimo;muore di neceffita; la voce feannare,

a Sula da noi per esprimere an foverchio desiderio di qualfivogtia cola, se bene il
a Myo più proprio è della fame, come s' è veduto sopra in quelto C, stan. 24,

' PEZLIENT INTE, Povero, che chicde limofina. Deriva dal Latino perere onde,

h, povere pexriente vuol dir pauper petens cleemofinam; ed € lo stesso che povero in can.

 %4, quafi ignudo come una canna; altri vogliono, che quello ixcanna sia una fo-

iy! Waparola, ¢ voglia dire imcannatore: Che quando un' huomo si metre a incanna-

ee; re

 

 

'che non pare si possa dire altrimenti. Quel po per poco € la figura apocope usa~ -

 
  
   
   

208

re, &fegno, che ¢ miferabile
4] Varchi Stor. Fior. lib, 124:
grnrzolato, ¢ diventarone poveri
Jogi, per guardar fempre it Cielo ¢
VNGER il dente; Mangiar roba, che unga il dente come carne
fempre pane, come fon neceffitati fare i mendichi; ¢ vuol dire Far
mangiar un po meglio. Wee OE ied dos sana
DIRE al corpo: fatti capanna « Haver tanto dam
pregare il Cielo, ia diventare il suo corpo capace quanto.
riporre il soe ir C.
tanta roba. Viiam questo termine quando veggiamo. uno avvezzo.a
ramente, e che si trovi poi a ua banchetto Jautifimd.) 6 obasbe
SI fece limpiafiro, Cioé x accorde, si conchiufe ii negozia.. s
STANZA XXXX. 9 STANZA XXXXE A
Fu volentier la scritta frabilita, Saggiunfers di tui mill aleve bowxey
To dice fol du lor, che fan pensiero Ata quandoda me poi lo
Di non havere a dimenar le dita y:
Ma ben di diventar lupo ceruicra 5
E,, perché e' fon bugiardi per la vita. y
Dimoltrano a me pai il bianco pel nero
Dicendomi, che m' hanno fatta sposa Ogni volta con mio maggior dolare.
D! un giovanetto sch! ¢ st-belta cosa. Sentivo darmi ana frvccataalcueres
Psiche continova il racconto, ¢ dice, che finalmente-fu-conchinfo db parenta-
do fra lei, ¢ il Rospo figliuolo di Venere. Z 1 be
ST ABILIT Ala ferieea, Fermato, ¢ conchiufo il contratto del Matrimonio,
che appreffo di noi fi'dice La scritta del parentado. oma
NON haverea dimenar le dita.Cio' haveraviver stza iavorarejatan dau
DWENT-AR lupo Ceraiero, Divorare, mangiar yoracemente ycome fail Lupo |
ceruiero. Plin, J. 8.c.22, de Lapis dice così: Sunt in eo genere qui Cornary v0~
cantur, qualens è Gallia Pompei, Atagni arena Spectarum diximus 5 buic ae
fame mandenti si respexit, oblivionem cibi furrepere ainnt digrefumque quarere alind »
£ da tale agonia di-mangiare s' aflomiglia un huomo, che mangi sonaednes,
ad un lupo ceruicro.” 23 ices
BOZZE. Intendi bugie, fandonic, trovati non veri; finziont » )
Quando non vogliamo credere qualche novita, che ci sia raccontatadiciamo:
fot ho per boxza. Traslato da iPitori, che dicono bozze, € abbensare Gace), 1
prime pennellate, che danno in una tela, ¢ gli Sculzori quei primi colpi, che>”
danno in un marmo, O altro; i quali additano un non so che del vero
faranno col finirle.. Vedi sotto C. 7. stan, 5. 3 i
'MI casco le braccia, M' abbandonai; mi perdei d' animo; mi sgome
4 STANZA XXXX1 HGS
Now lo voleva; pur miv' arrecai Quando pik
Veduto baxendo i tc

sip este
Ma perché non è il Diaual fempre mai
Cotanto brutto com! eglié dipinto,

    
 

      
  
      
 
 
    
  
     
    
    

 

    
  
  
   
     
    

   

 

  
    
  
  
 
  

  

 
 

  
     
 
   
    
   

 
 

     

    

  

   

 
 

 
   
  

QVARTO CANTARE: ty

'A XXXXIIL STANZA XXXKIV.

un bel g ' E perché quivi ancora haurd paura

Chrionon vada a stucbargli il (uo riposo,
'\uHlaurd sopr' ad un monte fepoltura,

Che mai si vedde ul pik precipiteo,
Ed alto poi così fuor di mifura,
Che non v'andrebbe il Bartoli ingegnofo;
Oltre che innanxi ch'io vi possa gingnere
4 Ci-vuol del buono,e ci [ard da ngnere.
forma d'un bel giovane,ia(ciata la fozza figura del
jalei fa comadamento, che di.cid in maniera alcuna non parli,perché altri-
ficendo; fara-cagione, che égli la jasci, ¢ f¢ ne vada in luogo da non po-

trovaro.

i. Condescefi; acconfentj, mi v' accomodai; vedi in questo Can,
prefo per accomodarsi col corpo;¢ qui ¢ prefo per accomodarsi con,

  
 
   
  
   

    

 
 

 
   
  
  
 
  
  
   
 
 
    
 
 
 
  
  
 
   

a ee

—

partite vinto. Veduto che la cosa haveva a andare in quella guifas;
dia diversi significati: perché vuol dire Servtinio, che noi corrot-
n0/quitrino, Vedi foro Can, 6. stan. 109., ¢ di qui Viffe il partite
dire Vifto, che il negozio era stabilito così, perché quando il parti-
il negozio §' intende stabilito. Adetter il cernello a partite, significas
metter in dubbio uno se deva fare, o non fare una tal cosa. Donna di partite vuol

ice.'Si piglia'in vece 4° accordo', patro, baratto, 0 condizione, Io vendo
4 col tal partito, ec, Significa ri/oluzione, 0 determinazione. Io ho prefo

=

Rs. Ps -s =

wm i ine. Significa termine, pericolo, 1] tale ficonduffe a mal parti-

4 MO, Clot a catrive termine, 0 4 pericolo di vita, 0 powertd. Ci (erue per espri-

ch, Mer maniera, modo: lo non vi verrd a partito alcuno. Significa rimedio, e/pe-

i? dente. Prefero per partito di fegargli la gamba,cc.

A “AL Diaval non ¢ brurto con egli ¢ dipinto. 11 Maleinon è poi fempre tanto,quan-

ed taro.

f ila gola., Immerfo nelle disgrazie. Vedi sopra C. 2. stan. 44. il suo
; AQVATT R cechi, A folo a folo, Remoris arbitris,

ist 'Sila »de'farté mia', Non voglia faper più nulla dime. Tratto dall'an-

g@ 'ico, Come.si vede in Pilaco, che col lavarfi'le mani pretese di non haver, che fa-

gh tenella Sentenza data contro al nostro sig.\ Giesi Crifto. Ii Lalli Eneid. Trau,

C4. stan. pa.

oy wid E mi lavo le man de fatti tok

s AL Bartel ingegnafo di Bartoli, che ha stampato un trattato dell'architettura,
perch diceingernofo cioè ingegnicre, che appreflo'di-noi vuol dire Architecto;
 €non Bartojo legifta(come si trova in altunr tefti,,dove dice Bartolo, enon il

«® Bartoli »»perché tratandofi di falire un luogo erto pud giovarpil i fapere d' un'

oe itetto:, 'che quello d' un Legitta. ' SE Stetson

tl Ol val del buono. Ci fara molto da faticare, 0 da spendere, o da camminare,

, — @simili, servendoci questo termine per — tutto quello ci posla ates necet-
Ss fario

 
 

eee

aio MALMANTILE )

fario in uno affare, fecondo la fubietta materia, come per efempio: A feriver
la presente Opera ci vuol del buono, € s' intende ci vuol molto tempo, moltas —
fatica, molti fogli, ec. ed è lo stesso che ci farò da mgnere. Ll che viene dal me-
dicare i feriti,e però-per Jo più s' usa in cose di poco gutto,e faftidiofe, per efem-

pio: Ll tale amaiazzo uno, vuol haver da ugaere,cioé vuol hayer

molti trava-

gli, (pele, difficulta, ec. ad aggiuftare il negozio. 11 Mureto lib. 9. cap.13.Var,

cur,
STANZA XXXXV.

Pos ch' una firada trovero nel piano y
Che veder non si puo gid mai la peggiay
Poi giunto apie del more alpeftre, eftrane
Con due uncini arrampicar mi se
Menado alt'erta bor l'una,por U'altra mandy
Come colui, che nnota di spalfecgio y
Ed anche andar con fléma,e co gindizio
'S' io non me ne vogl' ire in precipizio,

let, dille; Non parna & panca, fed multa & magna ad hoc efscienduns req

STANZA XXXXVL
Scoscefo è il monte in somma, e dirupate y
Edl viaggio lunghissimo, ¢ diferto, —
Così disse Cupido (mascherato,
Dopo civt ch' ei mi si fu feoperto;
Ond? io promeffi di non dir mai fiato y
E che prima la morte bauria foferts,
Che tra(gredir d'un pittoin fatti in detti
Afuoigufti,alucicenni fu

Cupido accenna a Psiche parte delle fatiche, e travagli, che ella havra nell
andare a ricercarlo; ¢ Psiche gli promette di non dir mai nulla a nefluno,

VNCINI, Strumenti di ferro adunchi, ed aguzzi,, servono per a
gualcofa, ¢ si fanno anche di legno per uso di corre frutti, ¢ per altre occorren-

ze ruftiche.

eats

RAMPIC-ARE. E proprio dei-gatti 5 ¢ d! altri animali simili, che falgono

fu per gli alberi, appiceandofi co' rampi, cioé con l'ugna delle
sotto in questo C, stan. 68. E ci serviamo del verbo rampicare per esprimere uns —

. Vedi

che falga in qualche luogo difficile, ancor chedo faccia senza rampicare.) Vedi

forto C. 9, stan. 25.

tere
NVOT A di (paffezgio, Nuotare di spaffeggio diciamo quand? uno essendo tut-
to nell' acqua dalla testa in fuori, cava fuora di essa un braccio per volta ordi-
natamente, battendolo sopra all' acqua per romperla, ¢ (pingerfi avanti.
NON dir fiato, ¢ non fiatare. B lo steffo che non parlare. Vedi foro C.6. st
42. Si dice anche non alitare. Non far verbo, Berni Orland. ae
E ferra piit fiatar mi.Pava chiasto, Vedi sopra C, 1. fan. 10.7
GVSTI, cenni, precerti, In questo Inogho hanno tutti tre lo steffo significato
di comandamento. Considerandofi gu/o per il meno stimato, cenno nel
dJuogo, ¢ precetto pet lo pi stimato, denotando dominio.

STANZA XXXXVIL

We tal cosa a persona haurei feopertas
Perché rusravia ta gente soiocca
Riden del rospo, ¢ davami ia berta;
Ed io, che quand'ella mi Venne in coca,
Won fo tener un cocamero all'erta,
MMs lasciai finalmente uscir di bocca,
he quel non era un rospoyma in esserta
Yn Srariofo y ¢-vago giovanesso »

STANZA XXXKVIIL

E che y se lo vedeffon poi la notte

“Quandin camerd messin
Exgetra via la feorxa delle bette
Chun fole proprio par Sputate »
Le male lingwe forse Mparian chiettes
Che si doteiars Se
Pero che now si pus tiraran peo
Chil comento ay veglian fre ree *

   
 

sn

S@eenvwris =

anes

ear

ES SsHAewvaesak 7a e FES AF ES

wa

 

eae

 
 

ee

= 3

,

a

“

QVARTO CANTARE: zit

- Vinta Psiche dalla collera, che le venne per esser burlata dal' altre donnes,
Scoperle il fegreto 5 E nota che PAutore moftra il coftume delle nostre femmine,
e quelle di tutto il mondo, le quali obligate a narrar qualche loro mancamento;
si fanno dalla lontana, e “ rfaadere d' haverlo cx i xo
forzate da' maggiori mancamenti d' altri.
pe cme berta, Mi davano la burla, mi beffavano, mi minchionava-
no, Berta si dice 3 col l¢,impernato sopra i pali, si fanno le paliz-
7 Patieead barcode fares i pe via di corde se ras ensets » che (as in
detto ceppo. E il Latino irridere, Raccontano le nostre donne, che quel fagace
villano nominato Campriano, del quale diremo sotto C, 11. stan. 48. cffendo ve-
nuto in mano della giuftizia per le suc cattive opere fu condennato a efler mefio
inun facco, ¢ buttato in mare; In cfecuzione di che fu meffo dentro al facco, ¢
oC to a i famigli, che lo buttaflero in mare. Nell' andar coftoro ad efegui~
re ini imposti furono per strada affaliti da alcuni mafnadieri,i quali si cre-
derono, che in quel facco fufie roba di valore; onde i famigli per scampar la vi-
ta lalci ivi 11 facco. con Campriano, si fuggirono, Campriano piangendo
fidoleva della sua disgrazia, il che sentito da uno di quei ma(nadicri gli doman-
'db perché piangeva, ed a qual fine era flato meflo in quel facco, Il fagace Cam-
'priano gli rispose; Io piango di quel, che altri gioirebbe, ed &, che questi SS.
voglion rmi per ie Berta unica figliola del Re nostro, ed io non la voglio,
conofeendomi inabile a tanto grado, per ¢fler' un povero villano. E perché efi
dicono, che se ella non si marita a me, l'oracolo ha detto, che questo Regno an-
“dra fortofopra, m' hanno meffo in questo (acco per condurmi a farmela pigliar
'forza;.¢ + oe ela causa del mio pianto. il mafnadiero credendo alle paro-
dicoftui, si concertd con i Compagni d' andar' eflo a pigliare questa buona.
fortuna, ¢ ripartirla con essi: onde fattofi mettere dentro al facco da Cam-
> che non reftava di pregarlo a volergli tar del bene quando fufle poi Re,
'feceallontanare i compagni, ¢ ferratolo entro al sacco, stette aspettando, che
titornaflero coloro, i quall non flettero molto a comparire con nuova gente, es
veduto quivi il facco abbandonato, lo riprefero, ed essendo vicini alla riva del
mate, velo precipitarono, ¢ così sposarono a Berta i) balordo mafnadiero. E
di qui venne dar la berta, 0 /a figlinola del Re, che vuol dir burlare, minchionare.,
come habbiamo accennato. Si dice anche dar /a madre d' Orlando, percht das
alcuni ficrede, che la madre d'Orlando Paladino havefle nome Serta,
— QVAND' ella mi viene in cocca, Quando mi viene in proposito di dire. B si di-
ce anche ella mi viene in cocca per intendere quand' io entro in collora, come s' in-
tende nel presente luogo. E cocca diciamo quella tacca la quale ¢ nella freccia.
per adattarla in fu la corda dell' arco da i Latini detta Crena, donde poi diciamo
eee eae 3 © feflura, che è nella parte ae alla punta dell' ago da
cucire, dal Gr, e-cocche; effremitd acuta, Dan, Inf, C, 12.
a cere, Chiron prefe lo firale, ¢ con la cocce
oh; Fece la barba indietro alle mascello
TUN Etiiee na caccinnre all erie a Dioogee far di meno di non la dire. Si
Eft comparazione al cocomero, perché eflendo questo di, figura sferica., e»
lilcio, facilmente ruotolando pud scorrer gil re un' erta 5 © monte, ¢ facilmen-
War 2 te

Oh

 

 

 

 
 

 

 
    
  
  
  
    
  
 
  
   
   

212 MALMANTULE) 5.

te pud esser anche tenuto fermo; onde molto -ben. si-dice Non fa tenet wn ¢
mero ali' erta d' uno che sia facile a palelare:un fegreto y che co ug
potria tacerlo,: «. claticwn stitrranity
PRETTO sputato. Similifiimo a lui: per appunto come jui,e senza ale
ne alcuna come è il vino oe 2 eae —- een 3
juclla aggiunta di /pacato si toglic da coloro, che pigliano:le-m cl
eon ae éninale » i quali in qualche Sesion per andi
punto fogliono tirare il filo 5 e sputandovi sopra lasciano cascar.
Parte, che gli è sotto,e da quello conoscono se il lavoro ¢ per appunio,
CHIOTTE. Chere. Voce Fiorentina, ma poco usata fuor di schet
ne, come poco sopra s'é vifto,! usd il Berni nell' Orlando E senza pits
frava chiotto tO Homa} onus
ST danno piato de' fatti d' altri. Gli danno pensiero; Gj sono a cuore
altri, Si mecterebbero a litigare per i fatti d' altri; Che Piato yuo)' d
Vedi foto C, 7. stan, 27. tb orng
VON si pud sirar un peto ec. Non si pud far una cosa bench minima,
popolo non vi voglia far sopra i suoi discorsi, Anh sole t
STANZA IL... STANZA Lhioy

 
 

Le ciglia inarca, ¢ tien la bocca feretta
Chiunque da me tal maraviglia ascolta;
Ma quel ch' importa afordono fu detta,
Che Vener, ch'ogni cosa havea ricolta,
Per veder s' elle verayo barzelletta,
Poiché a dormire ognun fel' era colta,
Entra in camera,e vien pia, 'Piano al letto,
E trova il tutto appunte come ha detto.

  
 

  
  
 
  
  
   
   
 
 

STANZA L
E nel vedere in terra quella Sposigla LVon tivo dir com io reftaffi allora Tya,
Che per celarsi al mondoil giorno adopra, Che mi fovvenne subito di a lw
Di levargliela via le venne voglia, 4 primo di mi si fueld, ¢ ancora —
Accs con off pik non si ricwopra: Mi fece l espertissimo comando, 2
Così la prendeye poi fuor della faglia Chtin alcun rempo io non ta deffi fora,
Fa un gran fuoco, e ve la getta Sopra 5 Ed io fon' ita scioceaya farne un bands, è
We mai di ti si volle partir Venere = poi mi pare rano,e mi arcoy dy,
Infin che non la vedde fatea cenere, Segli ¢ in valigiased ha copratosiporct. hy
STANZA LIL Uae yeas \
Sospe/a per un pexxo io me ne fretti, Guarda fu pel camin, civo in fai vetti, a
Chi io aspetrave pur ch' ei ritornalfe; etpro cli armarjye a Scoftar le cafe
A cercarne per A/a poi mi derti We trovanidelo mai, al fin'msi a i
Per le fhanze di soprase per le bafe; Per non fermarmi fin ch'ia no letrove, tay

Il fegreto palefato da siche, venne all' orecchie di Venere » 1a quale quando
Cupido dormiva gli abbrucié la vefte da rospo; il che veduto Cupido la
se ne faggi,¢ Psiche si mede a cercar di lui m

ha,

aca desta a fordo, Fy detta a chi ne fece capitale 4 8 chi importava (a mt
pe Oo « i i

  
   

213

   

|, Haveva sentito', ¢ inteso ogni cosa ?

Cosa non vera,ma detta per scherzo. E si dice Barzellet-
'lando, ¢ scherzando.

fo termine, che vuol dire Adagio adagio, significa ancora

c }) Senza far punto firepito, 0 romore.

'LE. Piccolo piumaccio, sopra il quale si posa la guancia, quando

; “rn mame guancia, come in diversi luoghi, ¢ detto origdie-

Rea Ie Oiler

 Riveftirfi'da rospo. Ecco la voce generica animale, che noi

le, come accennammo sopra in questo C. stan. 4.

dire. E? lo stesso termine, che pensare voi, vilto sopra in questo C.

one voglio dirlo, perché da per voi vel' immaginerete 5

    
    
  

   
      
 
    

fuora, Non la manifeftaffi, ed io n' ho fatto un bando; ed io Y ho
tutto. Won modo tubam, fed etiam praconem adhibui,

Scontorcerfi ¢ proprio delle ferpi ferite; ¢ parlandofi d' huomini
n certo atto, che denota dolore per qualche disgufto, o travaglio in-

     
 

valigia, E' in collora, in ira; Nel bugnolone, nel gabbione, e simili,
ni ne habbiamo in questo significato.
AR it porco. Significa andarfene; ed ¢ come l'interpetrazione di /ui~
3 quafi voglia dire fuinam, cioè fuillam emere, 0 che pili tofto sia detto /ui-
si feappar via dalla viena, e fugcirfene, come quei che sOn colti a coglic-
we uva nell” alerui vigna. Diciamo batrere sf raccone, batterfela, cor-
je se ben fon voci, che hanno del furbe(co, sono però comunementes
pre intese in questo fenfo. Vedi sotto C. 11. stan. 11.
ANZA LiV. STANZA LVI.
¢ via v0 fola fola y Ripongo la noccinola, ¢ la castagna
ancora una giornata y E rimetto le gambe in ful lavoro
tedsperrami figliuolay Per una lunga, ¢ ferile campagna
dietro veggomi una Fata, Difabirata pitt che lo Smannoro;
mi diede una noccinola, Dupo cinque ani giuntaa una-motagna,
jee lio, difs' io, d' una faffatra M1 si se inmanzi un grade, eorribil toro,
un' altra faa compagna Che ha le corna, ei più tutti d'acciaioy

  
  
     
   
 

mano anch'ela una castagna, Etira che correbbe nel danaio.
STANZA LVIIL.
ei mangiato i faffi E come Cavalier ch' al faracino
'accomodai per darui fu di morfo, Corre per carnovale, 0 altra feta, »
ch'io non La spiacciaffi, Verso di me we viene 4 capo chino

gran bifagno non mi fuffe occorso Con la (ua lancia biforcata in tefba,
ata di cid con eli occhi baffi Lo gid con le budella in un catino
tai del lor discorso, Addio dicevo al Atondo,addiochirefta.

   

Seufe,e refe ad ambe eAddio Cupido dove tu ti sia,
le lascio, dolla a gambe, A rinederct ormai in pellicceria,
STAN-

 

Digitizers: By
 

aig MALMANTILE' ©

STANZA LVIII,. tigeesbiadecs WT ORS
O Mamma mia, che pena, e che (pavento Pur come volle il Ciele io ini rammite
Hebbe allor questa mezxa donniccinala? Del dono delle Fate, ¢ la nocciuola =
Tremavo giufto come giunco al vento, Prefa per cafo prefto fur un fafo
Che quivi mi trovavo inerme, ¢ fola; La scaglio,ella si rope, en'e/ce un maffo,

Medflafi in viaggio Psiche s' imbatté in due Fate, dall' una delle quali wee
una nocciuola, ¢ dal' altra una castagna, ¢ le differo, che non le stiacciaffe, f
non aun gran bifogno. Dopo cingue anni di cammino per un deferto arrivd a»
pi¢ d' una montagna, dove le venne incontro un toro con le corna d* acciaio;
dal quale spaventata Psiche stiaccid la n occiuola,¢ ne nacque un maflo,

FATA, Fate sono donne indovine dette fecondo alcuni dal Greco Phatis che

fuona Donna indovina, ¢ quelle forse che i Latini co' Greci chiamano /ibilies
ma dalle nostre Balie nel contare le novelle a i fanciulli fon prefe;
buon genio, ¢ che fanno servizio al proffimo con le.loro azioni, € & contrarie
all' Orco, al Bau, ¢ alle Befane, che foo nimici de' bambini, a i quali questes
fempre fanno servizio, ed il Poeta, col regalo, che fa lor fare a Psiche, moftra
guefta verita. Da gli antichi furono anche chiamate Ninfe,¢ Dee, el'
nel suo Puriofo cid afferma, dicendo: sia
Queste c' hor Fate, da gli antichi furo Agee
Chiamate Ninfe,e Dee con pik bel nome, weil
Di queste Fate discorre ' Autore foto, nel Canto fettimo, ed è credibile, che
questa voce Fate venga dal Latino Fara fatorum, che Dan, Inf, ¢, 9, disse le fata,
Che giova nelle fata dar di cozzo? A
QVESTO ¢ meglio a una falfata, Quando si riceve da uno qualche regalo di po-
co valore, fidice per (cherzo: Queffo ¢ meglio d' una faffata, 0 vero a' un calcio di
mosca: volendofi inferire, che da quello, al nocivo, 0 al nulla vi è poca i
za. Plau. in Tr. disse Afelins eff quam deterrimum, E
ALLOTT A haurei mangiati i faffi. Allora havevo così gran fame, che haurei
mangiata qualfivoglia cosa,ancor che dura quanto un faflo, lo crederei, che il ve-
stitore di questa favola havefle seguitato i compositori de' Palmerini, degli Ama-
dis, ed altri Cavalieri erranti, che mai in tanti viaggi, che fanno lor fare, par'
una volta si trova, che in campagaa mangiaflero; ma il sentir, che Psiche
scorre di mangiare, ¢ che fu levata dond' ¢ll' era, perché non vi moriffe di fame,
mi fa credere diversamente, cioé che in questo suo iungo viaggio le Pare le em-
pieffero il corpo, che clia non fen' avvedetic, %
SCHIACCLARE, Corrottamente diciamo anche fiacciare, vuol dir Rompere,
6 infragnere, ed ¢ proprio di quelle cose, che hanno gu(cio, come noci, man-
dorle, uova, ¢ simili. i
DOLLA agambe. Comincio a camminare; è lo stesso che rimetto le gambe in [
lavoro, che è nell ottava 56. seguente. I La)l. Eo. Tr. C, 2, stan. 33.
mand' so la diedi a gambe,¢ dentro ann foffo
Lasca Nov. 6. Temendo, che colui non gli uscife dietro, s* usch di casa s pleted, è
la dette agambe,e per la fretta si feordo di ferrar l'nscio, 1 Lat. pure dilfero conijcert
Se in pedes.
ZO Smannoro, Così è detta una gran pianura posta poco lontana per —

Es iae e-em ee

 

 
  
    

PF See Se

  

SEs eEee eat peeiiak

&
=

SPREE FFG

 
 

 

 

QVARTO CANTARE: ary
alla Città di Firenze, 1a quale dura più miglia per ogni verso,senza mai trovarsi

una casa, se bene è tutta coltivata. Si dourebbe dire Ormannoro dalla famiglia
-antica degli Ormanni, la quale era già padrona di tutte quelle pianure, che si di-
Ormannorum

A che correbbe in un denaio, Tira così aggiuftatamente, che egli correbbes
Piccolo berzaglio, come è un denaro, che è la quarta parte del quattrino
3 con altro nome detto picciolo, ed un giulio ne vale 160.
iCZNO. Così chiamiamo quella statua, o fantoccio di legno, che figura
0. armato, al quale ( come a berzaglio ) corrono i Cavalieri le lance;
dice anche Buratto, che ¢ ua' altra forta di berzdglio( il quale si mettev
vece del Saracino ) ed ¢ una mezza figura fecondo alcuni', che nella fiailtra.
i¢ lo feudo, nella deftra la spada, o baftone; 1a quale se non è colpita nel pet-
si rivolta, ¢ percuote colui, che falll.
Ld biforcata, Intende le corna del Toro.
CON le budelin in un catino. Mi credeva già morta; Mi credeva già essere sia-
'ta sbudellata dal Toro. Luigi Groto Cieco d' Adria, in una sua lettera al Petr.
dice: Quei cani con il loro bau bau ci fecero parere d' havere le budella in ua,
“eatino, E Catino Intendiamo un valo di terra, o d' altra materia per servizio di
ucina, ¢ per uso di lavar piatti, ec.
} A RIWEDERCL in pellicceria, A rivederci fra i morti. Questo è il comiato,
ky

es

 
   
  
   

GREERFE Ge

a che noi finghiamo, che G diano le volpi ' una con I altra, perché (apendo, che
bm devon efler' ammazzate, ¢ le lor peli vendute, dicono alli lor figli, quando da
fle si feparano:. A rivederci in pellicceria, che così si chiama in Reena quella
jp firada,, nella quale (ono le botteghe di coloro, che comprano, ¢ vendono pelli
oi  Gianimali per foderare abiti, ec. ed in mano di coftoro, o tardi,o per tempo
g¢ -fanno che devon capitare.
I O MAMMA mia, O mia madre. Esclamazione di spavento, e di timores,
wata propriamente da' fanciullini, quafi dica: O mia madre foccorretemi in
t icalo.
~ SONNICCIVOLA, Vuo0l dir Donna di spirito minore di quel che converreb-
€ al suo naturale, da i Latini detta Aduliercu/a, Siche mezza donnicciuola vuol
dir Donna quafi da aulla, ¢ senza spirito.
WNCO. Specie di virgulto, che nasce in lwoghi padulofi, del quale si servo-
20 i Villani per legare i cralci teneri delle viti, ec.
444550. 8 intende un faflo grande, Questi nostri scarpellini chiamano il

 
    
  

maffo La cava delle pictre.

 STANZA LIX. STANZA LX.

Tal pietra per di fuoraé calamita Sfavilla il maffo al batter dell' acciaro,

| E ripiena di fucco artifiziato y £ da fuoco al rigiro ch' ¢ nascofo,

Hor mai arriva il Toro, ed alla vita Ea egli a ragzs cb allor ne feapparo

f Con un lancio mi vien tutto infuriato y Vincolpo fatto haver vede a suo cofto,
Ata dietro al maffe ero fuggita Perch non vi fu feampo, ne riparo,
. re riman quivi scaciato,. Chrei fra le fiamme non si mucia arrofscs
| CW in fo dando ha ferrata testa Ed iofeanfatoilfuoco,e ogni altro arate,

da quclte celamica afifo rea, Lieta mi parto, 0 tire innanzs 1 conto,
Ni: se sss; rf

 
 

 

216 MALMANTILEY |

Il detto faffo cra per di fuori calamita, ¢ dentro era fuoco lavorate, onde il
Toro perquotendovi con le corna ch' erano d'acciaio vi rimafero eo
da quella percofla nacque il fuoco, il quale 's'appiced allt ordigno 5 'abbrucid.
il Toro. Psiche libera da questo incontro seguito il fu viaggio.) ) ae
CALAMIT A.B: \a pictea fimpatica del ferro yo forse madre dai L

detta Adagnes. Vedi foro C, 8, fan. 45. € 66.

oy logeny!

FYOCO artifiziata, Vuol dire ogni forva di composizione fatta con
(che diciamo Da archibufo ) tanto per guerra, quanto per fefley.
RIMANE scaciaro, Riman burlato, E' lo stello, che rimaner con un Z

nafo, che vedremo forto C. 6, fan. 5.

mafio.

2o0h Hd

RIGIKO. Intend l'ordigno di fuoco lavorato.y che ¢ composto dentro al

v} of sala,

RAZZ, Raggi di fuoco 0 del Sole, 0 d*altro (cintillante..Ma dicendovaf
folutamente razz!, intendiamo quei fuochi artifiziati, che-si fanno in Occalione
di fefte con poiuere d' archibulo conttipata,e benisfimo Jegata entro alla —

dotta come pezzi di canna,

TIO innanzi il conto, Seguito il mio viaggio, Vedi sotto C. 6.stan. 16, Fane

to serviva tivo innanzs, © senza metterui if conto fuonava i) medesimo;

nato da quei, che tengono libri di debitori, ¢ creditori ci obliga a dir così,

» STANZA LXIL

Piit la ritrovo un grand' uccel erifone,
E tops assai, che giran.come PARR y
Perch' egli entrato in lor conver/axione
Gli becca,grafiase ne fa mille [Praxxj y
Di lor mi venne gran compalfione,
E vo per ovviar,ch'ei,non gli ammarzi,
Ma quei mi séte al moto,einpic firizza,
E per cavarsi, vien con meta flizza,

STANZA LXIL

Questo animate ha il buffo di cavallo

Di bue la coda,e in fucte spalleha laley

M capo, e it colo ginffo come sl gallo, «

Li pie di nibbio.vero ye maturale 5).
Gii artigl di fortissimo metalle
Grandi groffise.adunchi in modo tale
Che non vedefti quando leges, 0 ferivi,
Mai de tuoi di pin bei imterrogacini,
STANZ on
Son? att poie'a far pis acuta
' Seecmaige.aedieeie ighe y
Tal che,s' al vifo fuffinaiwenuto
Con essi, mi lasciava assai pik righe
D' un sibro di macfiro di linto,,
Ed una flamperia di falfarighe
Con farmia life come le gratelle.
. Da quocerui le trigtie,e le fardelle,.

 

STANZA LXIVE ©
Hor os tornare, In quel chia hetimere
Ch' il

mio grifo sia scherze

La cafagna ch! io in + ecio fuore

La rompo, en' esce subite un Ltone,*

Che mi scomo non poco tl barvionore»

Perch egli in mia difefa a lui soppone,

E moftrogli bor con Pigna,ed bor ee dati

Jn che mo si gaftigan gh infolenti.”

STANZA Lky,

L' uccelle anch' eel, che non ha pana

Géi rende molto ben tre per

Ada quel che haver del suo '4

Al contraccambio subito
E ben ch' ei owltepauiiae
+ Liafferraye firinge tanto el
Di poi garbatamente gli iesca
Gli flinchi fu's nodelli, e me gli rech:
STANZA LXVL st
AMetto uno firide ye mi ricive imdrete
Loch he paura aller ch'e: mmm ings »
Ata quegii ie

 
 

a Che mai vedelje st

Cio conoscenao sutta egal
Gti lascia.in terrase va perfacti fuck >
Ed so gti prendo aliora, efsemdo certs
Diaverne ahaver bifogno in figrad'erta

 
  
 
   
 
   
  
  
   
     
  
   
  
   
 
   
  
   
   
 
    
  
  
   
 

ee a

a

Q

3S

——— eS Re

 

 

   

217
STANZA LXVIIL
piedi, Evconuenne talor farsi da piedi
1 \Bawrendo git di erandi flramazzeoni,
mi Perché non ve dove fermar il pafsa +

| morte brancoloni y Cagion che /pefio mi trovai da ha/so,
trato il pericolo del Toro s' imbatte in'un' uccello Grifone, che ha-
@ acciaio, onde roppe la castagna, en' usci un Lione, che la difefe

|» ¢ tagliandogli gli artigit, li porté.a lei, Ia quale gli prefe, ¢
ttaccandofi ail' erto monte, comincid a falirui.
'che girano come pazzi, Sorci, che vanno in'qua ¢ in la correndo senza
ve determinatamente, appunto come fanno i pazzi.
RS/ la fizza. Sfogar ja collora, la rabbia, I ira.
0... Vecelio di rapina noto. Qui descrive il Grifone, ¢ lo fa mezzo ca-
mezzo uccello, ¢ con la coda di bue, e se bene da i pi ¢ descritto mez-
€ mezzo uccello, ¢ nimico mortale de'.cavalli, come si deduce da Verg.
rantur iam Gryphes Equis,tattavia non fa errore a comporlo di che be-
iuto » perché questo moftruofo animale in ogni maniera che sia è
fayolofo., fecondo Plinio lib, 10. ¢. aan Pegafos ( dice egli ) equino capite
Gryphes Aurita aduncitate rofiri fabulofos reor, illos in Scythia, hos in

BRROG-AT IVO, E} un contraflegno d' ortografia, i} quale si pone in fine
che conchiudono sn » Orichiedere, ¢ weet ¢ detto Punto
E:perché tal contrafiegno ¢ di figura simile a un' uncino, pero a
q igliamo gli artigli degli uccelli, come fa qui il Poeta, affomigliando-
pli a quelli del. grifone,
LI baaeftro di linto, Intendi libro da musica,, che fon pieni di righe,afi-
se di icriverui sopra le note musicali.
PALSARIGHE.. Carte rigate, ¢ lineate dinero, le quali si mettono sotto al
al quale si (crive,affine di far i versi diritti,'ed uguali camminando
0, che dalla faifariga per trasparenza si vede sopra il foglio, ove

PS

LIST]:. Qui vale per firiscette di ferro, 2 le ees composte le gratelle
firumenti dacucina, che servon per metterui sopra 1! pesce, 0 aliro.a quocere»
arrofto Econ mute queste similitudini intende, che se ucceliot ha vette meta
fli artighi addosso a Psiche, l'havercbbe malamente graffiata, ¢ segnata.
G. Vuol dir Faccia di porco, o simili; ¢ s' intende alle volte: la faccias
3 gan ame per naneeae » 0 per disprezzo; € qui il Poeta se ne serve per far
bifticcio rh e¢ Grifone,
x “4 VORE ~Panra 3 tumore. Da quella frequenza di battere, che fa il
a dalla ee del — » quando si ha qualche spavento: 1 Latini pure di,
anitah  velcordis percnffio.
INSOLENTE. Acrogante, faftidiolo, petulante,. Vno che tratta., ¢ proce.
~ per copia « Gli rende pib del suo dovere., perché.a 'render
che ¢la coppia,si ms la meta più del dovere: B con quetto
€. modo

jiized le
s
Se sna

218 MALMANTILE.¥

modo di dire s' intende, che uno Gi difenda da un' altro con pare
sempre con vantaggio, che diciamo anche render pane per

NON si cura haver niente di sue, Intendi Non vuol' esser da lui fope

e4f PERRARE, Abbrancare, pigliare stretto; 1 apprehenfam
NODELLI, Intendi la congiuntura delle gambe co' piedi, ©
eANDAR carponi. Camminar co' piedi, ¢ con le mani per terra,
fieflo, che Andar brancolone, che si vede nel verso seguente; se non che qu
vuol dir Salire adoperando le mani, ¢ i piedi; ¢ carponi ¢ camminare alla p
con le mani, ¢ co' piedi, Dante Inf. C, 26. descrivendo una simil falita dice
E proseguende 1a folinga via *
Tra le febegge, ¢ trai rocchi dello feoglia
i più senza la man non si [pedia, ae
STRAMAZZONI. \ntendi Calcate; che per altro #ramazxone intendono gli

schermitori una specie di taglio.
STANZA LXVIIL

Tusei quei topivia ne vengon ratti,

E furon per mangiarmi dalla fefta,,

Pero che dale granfie io gli hofottracti

Di quella beftia a lor tanto molefta;

Così ve rampicando come i gatté

Suil' aspro monte dietro alla lor pefta,

Sopportando fatiche, frenti, e.guai,

E fame, ¢ fete quanto si puo mai,
STANZA LXIxX,

Pur finalmente in capo 4 due altri anni
Giungemmo al luogo tanto defiato;
Ma non finiron qui mica gli afanni,
Perché di muro il tutto ¢ circondato;

EB qui s'aggingne ancor malea malanni,
Chto trove Cuscioyma'l trove diacciato;
Penssa 8 allor mi venne larapina,
Es' 10 dscevo della Violina,

 

see
STANZA LXX.

Hora tu sentirai ch' il dare aiutca

A tutti quanti fempre si connine y

Percht gid mai quel tempo s'? perdate,
Che dicepirws in foretell,
Non dicofet althuom, ma ico a un britd,
Che forse immrondo, ¢ inutile fitiene y
E che tx non lo fRimi anche una chioft,
Pero che ognuno è buono agqualehecifa,
STANZA LXXL
Setu giovi al compagno, allor tu fai
( Quafi gli prefti roba ) un capi
Anyi talor per poco, cheghaai
Ti rende psa fei volte che non vale.
Ma non Fase io pretender mai,
Perch? ell è cosa, che starebbe male;
Quefod un censo il quale achiloprende
Rrchieder non si pwd s' ci non to vende,
facendole

 
 

1 topi, che Psiche liberd dagli artigli del Grifone la seguitarono
gran fella, ¢ con quella campagaia in capo a due altri anni arrivé Psiche al luo
go dove era Cupido, che era un recinto di mura, dentro al quale non si poteva
palsare se non per una fola porta, ¢ questa era ferrata, oh Reed
VENGONO ratti. Vengono velocemente:dal Latino rapidus, D, Infer, ©. 21.
Perch' io mi moffi, ed a lui venni ratto Ee

4 whe
Ed habbiamo rartezza,per preftezza, 0 velocita. Varch, Stor. lib. 4. Za ele j
Lae

70 il sig.\ Sciarra Colonna fenl ¢on gran rattexza da Roma,;

FAR fefta anno, Ral

egrarsi conuno. Ricevere, © trovar uno con atti di

amorevolezza:, ¢ cortefia; Che nelle beltie si conosce tal rallegramento da i
fj, come nel cane dal dimenar della coda, ne i gatti dal fregarsi u
ed altri animali dal moto degli orecchi, come forse si conosceva in quei topi.

Lat. adulari fanno yenire alcuni da ad,& wra, che in Greco significa coda quafi fit

   
 
 
  
   
 
    
 
 
 
  
  
  
  
  
 
  
 
 
 
 
 
 
 
 
 
 
 
  
  
 
 
  
   
  
 
 
 
 
    
   
     
  
       
 
   
 
  
   
   
   

VARTO CANTARE: 219

ndi falire appiccandofi con gli artigli del Grifone, co.
i. Viene a. che s' intende ugne di gatto, lione, tigre, €
anche snerpicare ¢ ico Mtrumento ruftico da romper le terre.
Franzefi sopra alle maschere dice:

| Nom-vi crediate, che qualunque faglie
Haveffe da [ua posta tanto ardire,
Chr inerpica/se sopra alle muraglie

si dice inmarpicare,e annarpicare. Vedi sotto Can. 9.stan,

w
'alla lor pefta, Seguitando le lor pedate,
a icella riempitiva in i er emfafi

ir a 'B'
appunto come i Latini dicono ne quidem » se bene & diferente dal
¢ non s' usera per affermativa, io voglio mica, come essi dicono ero
che se bene ¢ per emfafi ha però qualche parte del negativo, quafi
fo now voglio ne pur' una mica, che vuol dir minuzolo di pane, o granel-
Petr. Son..91. We mica trove il mio ardente defio.
Dolori di cuore, che fanno quafi venire in angoscia, Petrar.

 

 

Buy Leh >

  

ris

Se la mia vita dal aspro tormento

Si puo tanto febermire, ¢ dagli afanni,

GER male a malanni. Al male accrescer male, ¢ peggic.

(0 diacciato. Cio' porta ferrata. Vedi sopra C. 3. stan, 3.

la rapina', Mi venne rabbia, collora, o stizza.. Rapiva vuol dire ru-

quindi uccello di rapina; ma dalle nostre donne è prefa in

» per sfuggir di dire rabbia creduta parola peccaminofa, ¢ dico.

i rv arrabbiare, ed arrabbiato

bicevo del male fra me medesimo, perché le cose non

mio modo. Questo so che significa Dir della violina, non so già da.
gine questo dettato, che ¢ lo stesso che Dir l'oraxione della ber-

ee

i una Chiofa. Non lo stimi punto. Vedi sopra C. 3. stan. 60, alla

capitale. Metter insieme una somma considerabile di denaro per ha-
a ogni suo bifogno: Si dice anche far un' afsegnamento,
$0. La namra del censo, ¢ che colui, il quale presta danari a censo,non
richieder la somma principale, che egli da, ma folo i frutti d' efla; pud ben
steel la medesima somma principale a ogni suo piacimento,
la diede @ forzato a riceverla, come dice il Poeta affomigliando co-
ilpiacere a un' altro, a uno che dia a censo, e dice, che colui che
non dee, ne pud pretender la ricompenfa, ma la pud bene sperare,
creditore: ow ben dice Seneca de Beneficijs \ib.3.c.14, Vide etiam
crit, nulla repetitio, B lib, 4.cap. 39. Alia conditio

beneficio.

SEE

 
  
  
      
    

 
 

su Se

Ee 2 ' STAN-

citized bylibgie
*
 yorture di cose di sua qualita, ec.

 

  
 
 
 
 
 
   
 
  
 

220 “MALMANTILE)

STANZA LXXIL

Guarda s' el? è così; Lo per la mia
Picea di prender di quei topi curay
Da lor vinta respai ds cortela,
E wt hebbi la pariglia con?) usura y
Peri ch' in quefia xxx ricadia,
Ch' io ho d'haver trovata claufura.,
Eglino tutti ful cancel faliro,

E si fermara, ove ¢ la toppa, ingiro,

STANZA LXXIIL i

E gli denti appiccando.a quel legname, Cupido etmor, che tanti ha
Come s' in bocca haveffero un trapano y Berzaglio qui si giace della
Prefeo prefto vi fecero un forame Ei chera fuoco,il nafo
Da porre il fiasco,e vender iltrebbiano, i 0
Tal ch' in terra cascando ogni ferrame
Spalanco l' xscio di mia propria mano,
E paffo dentro,¢ refto pur confufa,
Perch' acor qusvie un'altra porta chinfa, Non farò U urna, che glié qui daca
1 Topi fuddetti rimunerarono Psiche, perché rodeudo fino a fette porte, che
erano in quel Serraglio,fecero cascare i ferrami, ¢ Psiche entrata dentro,trovd il
sepolcro @ Amore, ¢ dail' Inscrizione, che in esso eras comprefe quello'
reftava da fare. t $

HEBBE la parigla. Hebbi il contraccambio.. B' il Latino Par
Pariglia intendiamo due cole uguali nel giuoco di Carte,,0dadi,, come due!
due alli, due figure, ec, ¢ di tal voce non ci serviamo se non nel giuoco 5 0 nel
cafo del presente luogo di reader contraccambio si.in bene, come in male }
forto C, 6, stan, 69. Io l'ho per voce Spagauola, ed il Varchi nella. 8.
} usd ia ua certo modo come straniera dicendo.: Dopo eferfi vendicari
renduto i contraccambio, 0 y come si fuol dire, la pariglia. y

CON ? xfura, Col frutto. Cioé mi contraccambiarono, facendo
vizio a. me, che non havevo io fatto a loro;:

ZEZZA, Vitima. E' voce antica hoggi poco usata fuor che nel,
Vedi sopra C. 2, stan. 2, Si trova anche/egee y/ezzaiay 0xerraid, >>

IC.ADIA, Noia, travaglio y avversita, moleftia, © simili che vengono\
aun' altro dilgutto; da ricadéa, che ¢ quando uno infermo già quaGi fanato, vie~
ne a riammaiarsi, o per lo mal governo, 0 per altro. Nella storia di Semifonte
Trattato terz0. Con li loro misfatti, dando alli Fiorentini non ic
Sac, Noy. 98. Che ricadiaé que/pa di questi porci?.. t rs 19:0).
CANCELLO. lntende il legaame, che chiudeama.porta: ma pro}
cancella diciamo una chiufura di porta fatta di stecconi, orftrisce di legno
ferro feparate |' una dal' altra a guia di gabbia.; ora.gve

TOPPA, Intendiamo quella piaftra di ferro, sopr'alla quale fon
ingegni della ferratura, detta affolutamente, o senza aggiunta, perché per alt
Toppa G dice ogai pezzo di panne, legno, quoio, ferro, ec. che s' adatti:

 

 
    
      
 
 
 
   

    
      
   
   
    
   
       
  
    
 
  

          
 
    
  

   
  
 

ia

 
  

 

    
  
  
    
   
    
   
  
  
   
   

Be

EST SPSS SSeS Fo BF

Fee

 

 

QVARTO CANTARE? a2r
fruménto specie di fucchiello, col quale si forano mate.

tre 'metalli, ec. eel odeeaa

@°, Coloro che vendono il vino a
loro » come dicemmo sopra C. 1. stan. 76, ed oltre a,

più nella porta, o nel muro una fineftrella,per la quale dan-
vendono'; a'questa'fineftrella atfomiglia il foro fatto da i
0 iano pigliando questa specie di vino per tut-
'intende esser questo tale sfondato simile a quello, che si fas

vendere if
dere il vino'.

Ateneo

ca
MOR 2a.20

che Cupido ¢'freddo, cioè morto

schi, appiccano un fialto

'£. Aprire largamente, quanto si pud.
bere un novo. Fu cosa facilissima,come ¢ il bere un' uovo: i Gre-
'in questo proposito Oxo patio quis ovum forberet je trovali questa,

0 agrafio. ipingere-a grattio, sgraffio, o graffito,è un' imprimer
anwterisacee nell' ifesthacdties fee(ca de' muri con deteo ferro,
ja graffio, forse dal' antico graphium, che era lo filo di ferro, col

(ONARE; 0 sbolzonare'. Sacttare, frecciare, da bolzone [pecié di frec-
Mattio Franzefi sopra alla boria dice:
Di qui Amore accorto balefriere
\* “Bolzona quaiche giovane galante
lato', Ha il nafo freddo. Pighando la parte per il tutto,vuol dire,

A? Animiale'noto; ma qui si dice una, che chiacchierando assai,non

se fa tener
ree

we

Feta cosa alcuna; ¢ degli huomini diciamo Cicaloni, Appretfo
'cicala non sona male, poich? alle cicale sono da essi raflomigliati in più

aa peril continovo cantare, che fanno, ¢ questi, e quelle. EB
quelto Poeta graziofamente chiamo Musa la cicala sopra C. 1, stan, 2,

STANZA LXXVL
Non tivo dire adeffo sin quel cafo
ro gli occhi due fontane,
“E feci'come thi 2 rotto il nafo,
* Che versail fing ne,e corre al lavamane;
“malin oorer 4 quel vafo
Durante a lagrimar fei fertimane,
¥. il pite voglia di piagnere,
inte ceils miebbi iniricobe
Pret aes LXXViL
Quand io ch' egli era poco meno
“Mn fach Muh ape 4 buon porto,
Valli innanzi ch'e fulfe affatro pieno,
Ech il marito mio fuffe riforto.
Lavarmi it vifo ye raffettar mi il feno,
Accio st lorda non m' baveffe scorto;
Percio miparto ye corro,se in quel monte
Per avventura fuffe qualche fonce.

 

STANZA LXXVIII,
In quel clt io m' allontano com' io dico y
AMiartinaxxa, che era in Stregheria,
l'fio di la portata dal nimico,
Che non porette tar per altra via;
E perché fempre fu [uo modo antico
Di far pertutto aalcun qualche agheria;
Lefie il pitaffio, (quadro Purnaye tenne,
Che li fufse da farne una folenne,
STANZA LXXIx,
Se qua, dice fra se, Cupido dorme,
Vito rifuegliarlo per veder un tratto
“ Sregli ty come F dice, ¢ se conforme
eA quel che dai Pittori vien ritrarto *
Se ben chi lo fa belo, ¢ chi deforme,
Basta mi chiarird com' egli ¢ farto;
Per questo ad emprer mettefi quel vafo,
ef cui poco mancava ad efser raso,

Zed bya
 
 
   
   
   
   
 
   
     
 
   
   
  
 
   
 
 
 
 
 
    
  
 
   

222 MALMANTILE 5)

2 STANZA LXXX..%
Con l'animo di pianger vis arreca, Al fin si pone a
4a ponza pontia, lagrima non getta Si che per forza a pianger è
Si prova a far cipiglio, e bocca bieca, Onde la pila in Ai

Ne men qucfiaé pero buona ricetta; Refto colma, ¢ Cupido se
In ordine al Cartello havendo Psiche con le sue lagrime quafi piena l'urn
ando a Javarsi il vifo, e raccomodarsi la testa; Intanto Martinazza arriyo:
polcro, ¢ con le lagrime sue fini d' empier l' urna, ¢ Cupido usci dal Sepoley
WON ti vo dire, Questo termine (crue per esprimere. Date puoi ben fa
Sta cola meglio di quelo che io fapeffi dirti; 0 vero so che tu hai da per te tant
gindicar come io rimaneffi, senza che io te lo dica, Suona lo stesso che pensa
dica tu,tu puoi fapere, ec, Vedi sopra in questo C, stan. 41. stan. 52,5 ¢ fan,
Simile è quello: Non domandar, se Durlindana taglia.  +p ae
LAVAMANE, ¥ uno strumento di legno, 0 d' altro, che con tre piedi
ma come una piramide in triangolo equilatere, ¢ sopra esso si pola la catinell
altro vafo per Javarsi le mani. ri;
ERA poco meno che ail' rie, Era quafi pieno. L' acqua arrivava
mita del yafo: che questo vuol dire or/o, che viene dal latino.ora y si
¥ eftremita di qualfivoglia cosa.
LORDO., Schifo, intrif©. Dal latino Luridus. eee
VA in fregheria. Dicemmmo sopra C.2. stan. 11, donde derivi tal nome di Stre-
ga, c¢dal C. 3. stan. 69, dicemmo esser fama, che tali Streghe vadano la notte:
cavallo in ful caprone a Benevento al congrelso de' diavoli. E questo: (
cendo Andare in Stregheria portata dal zimico, che vuol dire il Demonio, in for-
ma di Caprone. Che queste donnicciuolucce credure Streghe vadano in ful Cas
prone a Benevento ¢ opinione vulgata,¢ molti di ceruello debole I hanno per
indubitata, e le medesime Streghe se }o credono, perché il Diavolo con illuf
fa loro apparir per vera questa falfita; Mala graziofa fagacita d' um Superiore
ne fece chiarire tutti i dubbj in questa forma. ee
Fu condotta alle carceri una di queste tali inquifita di maliarda, ed il Gindi
dopo molte efame havendo troyato, che veramente coftei era una donna, che si
credeva far malic, stregar bambini, ed altre scioccherie, ma in effetto non v'era
cosa di conciufione, o di proposito, risoluette di gaftigarla per la mala i
ne, ed in tanto soddisfare alia propria curiosita. Fattala però venire a sé 'inter~
rogd se andava anccr' ella a Benevento, rispose che si, onde egli le die: Tovi
voglio perdonare se voi andrete questa notte a Benevento, ¢ domattina mi race
conterete quanto vi fara fucceflo. bifogna che mi diate la liberta,replicd la don-
na, acciò io possa nella mia stanza fare i miei scongiuci, ¢ le mie unzioni; il
Giudice gliela concedette con questo che voleya dargli da cena insieme con
compagno: il che accettd la donna, baftandole esser fuori di quel luogo., dove il
Diavolo non poteva capitare. Andata dunque a casa cend con il detto compa:
goo, che cra un giovanotto ortolano,¢ con un' altro giovane, che la donna
© che egli conducefle, ¢ beyuto abbond: come era il sao coflu
me in tali fere di viaggio, la(ciati i commenfali a tavola fen' entrd nella solitas
camera, € quivi spogliatafi, senza ferrar la porta,ne le fineftre della medesimas
camera »

 
    
 

223

# Yordine del Diavolo ) s'unfe con più forte di bitumi puzzolen-
liacere in ful letto, subito s' addormentd; I due compagni, così
camera, ¢ legarono la donna per ecia, ¢ gambe alle
del letto, ¢ benissimo 1a strinfero con funi, ¢ si meffero a chia-
voc!, ma come fuffe morta non faceva moto, ne dava fegno
onde i detti cominciarono a martirizzarla bruciandole hora
hora una coftia, ¢ finalmente così l'impiagarono in diverse parti del
arfero fino alla cotenna la meta della chioma; Cominciando a veni-

} donna con sospiri, ¢ lamenti diede fegno di suegliarsi, onde i det-

Jegami, ed uno di loro ando per una feggetta, ¢/' altro la rivefti
¢ dal fonno, e molto più da 1 martorj; giunta la feggetta,in efla
tarono al Giudice, il quale l'interrogé s¢ cra stata a Benevento, ed ella.
che'si, ma che haveva patito gran travagli, ed era stata baftonata cons

ferro infuocaté, ¢ strascinata, ¢ legata per le braccia, ¢ per le gambe,
riportata dal suo Caprone, che nel la(ciarla le haveva abbruciate con la
fa mezze le trecce, ¢ questo perché ella haveva ubbidito al Giudice, ¢ che
itiva morire dal gran doloredelle piaghe. Il Giudice ordind, che subito ful-
ita, come segui; ed intanto disse alla donna: Io v' ho fatto scottare, «
gaftigo del tuo errore, ¢ perché tu conofea, che non altrimenti a.
yma in casa ua hai ricevuto questi travagli, ¢ ti risolua a lasciar que-
Re falfe credenze; che se lo farai, io ti perdonerd. Da questo bel modo di gafii-
cay Pt arguro Giudice quella verita, che apprefio Jui era certissima.

NON, far per altra via, Non potette cflere in altra manicra, perché
non havrebbe mai potuto falire su quel monte; se non ve l'havefles
iavolo.

R/A, Violenza, dispiacere,soprufo. Viene dal Latino greco Anga-
4 cuaétio. Varchi Stor. Fior. lib. 2. E perché i Fiorentini nuovi tributi,
ritrovare havevano.

  
   
     
   

   
    
   
   
   
  
 
 
  
 
   
  
     
 
  
   
   

tuna folenne. Fare un' angheria delle maggiori, che si possano fare.
me & da noi spefio usata in vece di grandissimo, ed è tolra da i riti
t Chiefa, che si dicono fefte folenni, le maggiori fefte, che seguono nell'An-
0. Così bieros, cioè fagro, pretio i Greci, ¢ facer prefio i Latini vale taluolta
Brandissimo, e4nchora facra 5 Adorbus facer, & lo stesso, che Anchora maior,
us maior, B, Virgilio quando disse; uri facra fames, per avventura inteles
ma..

VIEN ritratto, Vien dipinto. Se il dipinto 2 come il vero. Dice: chi to fa.
9 ¢ chi deforme, per intendere, che i pittori da pochi soldi lo dipingono
“eAD fer rafo, Ad esser pieno affatro. Viene dal mifarare il grano con lo

Maio, che per dare, ¢ ricevere il dovere s' empie lo staio, ¢ quando è pieno si

- sttifeia sopra con un baftone, ¢ si fa ca(care quel grano,, che è sopr' alla boccas

ha questo si dice radere, ¢ tal baftone si dice rafera, € lo staio così pie-
Oli dice rafe 5 cioè picao per appunto fino all' orlo della bocca. a
F
le

 

   
 
   
 
 

224
VIs' arreca, Vis' accomo
42, s' arrecé con l'ani
PONZ A po
fizto,quafi riducen
mandano fuora il parto
tare, come si vede dal rca, che dice; oie
le riconobbi a ula huom che ponta
L' Espositore dice ideft che ipinga. Vedi 'Alunno fabr, num,
Ed il termine ponza ponza serve per esprimere uno, che aff ora
da poco; che si dice anche tresca tresca. Ticche ticche, denneinne y,
forto C. 5, stan. 51, Za vanum laborare. Se bene qui Deaheaiers
nazza moltissimo ponzaffe. joke pibale
C/PIGLIO, E! uno increspamento della fronte fatto in git la
occhi, ed è una guardatura d' uno adirato, 0 d' uno eftremamente fup
piglio del ciglio, Gli antichi, come,Daate differo Pigtia,la guardatura.
BOCCA bieca, Bocca storta. La voce biero Latino obliguus, ¢ usata assai
Legnaioli per intendere l'incgualita d' un legno, ¢ digono sbiecare quan
reggiano, ¢ fanno uguale.; tahoe
PILA, E' proprio quel fodo, sopra il quale posano gli archi de i
si piglia oat fs quel vafo grande di pictra, nel quale si mette ac
heverare Ie beftie, o per altro uso simile; in (omma per pila intendiamo.
fo di pietra che tenga, 0 riceva acqua. owasell
STANZA LXXXI, STANZA L
Quand' ella verso lui volte le ciglia, Fermoffi a Adaimantile,
E vedde quella [ua ea, figura Lo vole, ¢ gid le moze
Disposta  ¢ graziofa a meraviglia, Come fai tn ( dirai ) custo dfeguita?
Cie pik are KS 'far n' una pittura, Lo so, che ub ho a le.:
Gli s avventa di subito, e lo pigha, Aeeee mi donar quel
E,senza ricercar della cattura, Chiin due Aquile essendo trasformatt,
Dat suoi frafieri renebrofi, ¢ but Perché lafsi facea degli shavighiy WS
Portar se ne fa via con esso lui, Mt han traportata qua ne hy it Be
Mactinazza porta via Cupido, ed in Malmantile lo piglia per marito; ©
havevano raccontato a Psiche le Fate., le quali trasformate in due A ic Vhave.
yano portata via da quel monte co' loro artigli. E qui finisce il quarto Cantare,
CATTVRA, Si dice quella somma di danaro, che si da a i birri quand' haa:
no pigliato uno; ¢ si dice anche catrura quella polizza, ¢ ordine che si da alli sbir
ri perché piglino uno. Di qui il Poeta cava lo (cherzo dicendo, che Marti
za piglio Cupido feng' haver |' ordine della cattura, ¢ to port via,¢ nona
t0, che le futle dato il denaro della cattura, che havea fattadilui,
PACEA degli sbavigli. Si dovrebbe dire shadigi.Dan. Inf. C. 45.
: Anxi co pie fermati shadighava
ie Pur come fonno,ofebbrel' afjaliffe i
Ma hogei si dice sbavieli, ¢ sbavigiiare; che un' aprimento
do il fiato, ¢ poi mandandolo fuora, i) che per lo pil ¢ cagionay (
penGeri, da triftizia o malinconia, o da altro rincrefscymento,

 
 
 
 
      
    
    
  

pee

  

 
   
 
 
 
  
    

 
 

 

   

   
        
    
     
  
    
       
   
 
   

'
   
 

   

TO CANTARE. ary

¢ frigidi generati nello stomaco da ozio, ¢ da pigri-

bocca per la via del cibo, ¢ spargonfi per le ma(cella, ¢

gli fuora, alita con aperta bocca, il che da i Latini
ili, Significa non haver roba da mangiare, ne»

> ¢d habbiamo una rima, che dice; I

aviclia non pud mentire *

- Ocgliba fete, 0 egli ha fame, o & vuol dormire,

overa Psiche fando in quel Inogo, dove non eva da' mangiare, ne da
Wa Occasione di sbavigliare non potendo cavarsi la fame,ne la fete,

GLI, Dal Latino articnli. Zampe degli uccelli, o altri animali ditati,

le mani delle Pate, le quali convertite in Aquile,havevano artigli in

mani. Se¢ bene diciamo taluolta artigli le mani dell' huomo. Bocc,Canz.

Go? Amor ys io polfe uscir de' tuoi artigh, i

A pena creder possa,

Che alcnn altro uncin mai pile mi pigl,

FINE DEL QVARTO CANTARE;

INTO CANTARE,

 
 

 
  
   
    
 
 
    

 
   

  
    

         
 
  

    

 

Se
. VERSIE 3
tan' Conme Coane seiee NS
ie ARGOMENTO, ¢
/ Viol con gl' incanti dar la eAaga aita ?
F Jn Malmantile al popolo afsediato,
'an Ma dagli [pirti è così mal fernita,.
Che trai zimici ¢ il suo faper beffato; -e.
me Vien Calagrilio,e a duellar U inuita y Me
h EP inuito è da lei toffo accettato,
'il H Fendefi, ¢ altri due com' è usanza,
ie Sparsr di Piaccianteo fan (a pieranza, 2
go CR CIID CR BI CUM DBE
SR Sarath 7s
¢ art
ot STANZA L STANZA IL
ip Gli eftremi non fur mai degni di lode:
Ci vuol la via di merxz0,¢ chi ha ceruelle
Se vere, 0 falfe novitadiesli ode

A crederle al compagno va bel bella:
Le crede, s'ele fom fondate, e fade i
Ma s' elle fhar non possone a martelfo
Won le gabelia mica di leggieri,
Come fa il Duca a certi mefsaggieri,
Ff E Vo-

  
 
    
 

226

Volendo il Poeta nel presente Cantare: wince iadia vy
mandati da Martinazza per far diloggiar Baldone'y ¢10
le, per lo quale apparvero a)Baldone diversamente da quello
che fu causa, che egli non preftd fede alle loro cmgpcta
Che l'esser' huomo teftardo,.e capone non @ bene;
bene I esser così credulo, che si dia fede a onan ra che si fente d
degno di lode colui, che fa pigliare la via del mezzo, dando credito a' pele
se, le quali-egli conosce haver fondamento di-veriea y come fece Bald
meflaggieri di Martinazza De ES Ie:
CAPONE, Tellardo. Huomo ofitio nella faa opinione. In 'Baca
potrebberfi chiamare questi cali Capirones; da noi alerimenti Capardi,
TONDO. Huomo groffolano; femplice, facile, credulo, ec,
4 ai panni lani, che si dice sends, quando sono groffi, contrario i fini,
diciamo huomo fine, che & il contrario 4' huome rondo, Laica Novella 2
t0 fu hnomo di si i ereffs pafba, € così rondo di pelo, che in quattr? anni ai squola
tette mai imparare t' eAbbicci. Vedi forto C, 6, stan. 80.
MINC HIONE, Semplice. Vedi sopra C, 4. stan. 15.
SE le beve tutte. Crede tutto quello, ch' ei fente dir.
BAB BV-ASS!, Igooranti, huomini di ceruello groflo'. Vedi frto'!
CREDEREBBON cl? un afin volaffe.. Per esprimer' uno, » che cred
le cose imposfibili a crederfi, ci serviamo di questo detto. In Empoli |
lenne dell' anno, fanno una antica fefta, o rapprefentazione di ae
no: Quindi è, che nel Capitolo in lode delp “Ate, che va colle rime:
dice:

  
   
  

 
 

 
 
   
 
        
   
   
   
   
  
    

Ben moftran gli Empolefi aver cervello,
uanto conkienfi ad ogn' huomo da bene;
Che ? Afin diventar fanno un uccello,
NON pué tare a martello. Non cortilponde al vero, Tratto dat Cimento dell?
argento, che quando non fa, cioè non refifte al Martello, ae i e eee
1 Latini pure direbbero in questo proposito mon ef? aurum ignt prob. 5
NON le gabella. Non le pafla per vere, Non le crede: dal Pelee Ped
Gabeila delle porte, 0 de' pati; onde il verbo Gabellare, per ammecterey € a]
vare una cosa per buona, ¢ per vera. <dica particella Heer
enfafi della negativa,come già,e mai, ec, Lo non vai mai, che si dica im sis
che si dica, Io non vud micayche fidica, Vedi sopra C. 4, stan. 69. '

 

 

 

STANZA IIL STANZA v. i

Ma, perché chi m' ascolta intenda bene; Ella ahaa allor, ih deb pers

Tornar a Martinazza. mi bifogna, Chit pigliar Panes

La qual dianzilasciat, levi fovviene, Che per ta pet ce i

Ch' in ful Caprinfernal, Pigra earegna, Ai farfibravo ye bee

Quel popolaccio ha aggiuntoye lo ritiene Se ben fra tanto: moana

Dal gear via com tantafuavergogna, S* ellacon Gambaforrn,¢ B

Perché quando per lei larafiguray mode

Railenta il corso, ¢ piscia la paura,
 

   
      

QVINTO CANTARE: 227
AL pibieroin: “STANZA VI.
l 0.4 Cid dette balxain casa, € cold dentro
0 Per aenerfi dispogliafi in capelti,
) Ecacciatafi addofso quant' unguento
Haveva ne! [uci feridi alberetli,
Vin gran circolo fa nel pavimento,

   
   
    
    
  
 
 
  
 
 

E con un uafo in man,scriteiye Cartelli,
Borbortando parole tutravia,
Che ne men si direbbono in Turchia,
STANZA VIL
pari in mezzo al feonoy O colaggiie dal forreraneo Regno

wdo all ordine ogni cof, Cornuti moftri, e gente spaventofa
ad effetto il [uo difegno Pilizginofi eee Dees
voce firepito/a: Badace a me; le mie parole udite.
0 a Marcinazza, 1a quale sopra nel C. 3, fan. 76, la(cid, che mon-
cioni in ful Caprone, haveva arrivato quel popolo, che fuggiva per
riconosciutala, la prega a dar' aiuto a Malmantile, ¢ far, che essi
iano a-combatter,se fijpud. Bila dice, che Rima neceffario il combat-
che intanto vuol vedere, se gli riesce cacciar via il nimico per altres
5 ¢ vaflene in casa a fare i suoi incantefimi a questo effetto.
(FERN ALE.Duedizioni.come ridottein una,significante Caprone d'In-
i quel Diavolo in forma divCapra sopr' al quale era cavalcata,
Za j5e il quale si favoleggia, che vadano.le Streghe a Benevento,
sopra C, 3. stan. 69.
Vuol dire Cadavero d' huomo, o di beftia. Cavalcanti flor,
p. 2. dice;. Se volere veder quanto la lor per fidia si disse/e contro al fan-
macgiori, cercate i Connenti de'Frati, e troveretegli pieni di corpora, e di
vi antichi. Da questo dire del Cavalcanti m' indugo a credere, che
na ifichi cadavero d' huomo ammiazzato con ferite, ¢ straziato,
cl ere di tal voce per intendere una beftia piena di mascalcie, ¢
¢ flinio con Pier Vettori nelle Varic lezioni, che venga da Charonia,
ano già le voragini del fuoco,, che in diverse parti del mondo fj tro-
dicevano Charonia da Caronte, perché la superftiziofa Gentilica sti-
yche tali vorapini fuflero bocche d' Inferno, ¢ che per quelle s' andafle da
E perché hanno fempre puzzo orrendo, che procede da acque (ulfu-*
flo cominciarono a chiamare Charonia tutte quelle cose, che grande-
vano; E noi seguitando gli antichi diciamo C aragna a tutte le coies,
wutono, come fanno le beftiaccic guidalescofe, ¢ le morte. Dicigmo Caro-
un' huomo, che habbia cattivi sentimenti, perché un' azione mal fat-
dire 4 putes; onon ha buono odore., 3
Atenieli chiamayano Charonia quella porta del Pretorio, 0 Palagio del Po-
r uscivano cojoro, che erano condorti al supplizio, fecondo
iulio Polluce nell' Onomattico, ¢ Alex, ab Alex, lib, 4.c. 16. ¢ Cel.
lect. antiq. ¢, 8, ¢ lib, 47. c. 9. Tolta la derivazione di tal vace pures
ate, che conduce J' anime al hppltsinpatenee in barca, ¢ si dice mane
2 dar®

 
   
  
 
 
    
   
 
     
  
  
   
  
 
 
 
    
    
 
   
  
  

   

 
 

 
 
   
    
   
    
      
 

228

dar' uno a Cavonte per intender wala uno alia morte. vile
PISCIA la paxra, Ripiglia animo.. Non ha più para.
s no azauffaci fogliono pisciare; ¢ comunemente dalla plebe si dice
li paura; € da questo diciamo pisciar ta paura 5 quand' uno -spaventato
rito,perde quel timore. * Achy
L! ABF-ANNO in fulla pena, Eca aggiunto alla pena, che hebbe per la
) atfanno cagionaro dal correre, Vedi la voce Affanno sopra C.g. stan, 69.
VEKMENA. Va foil, ¢ giovane ramo d' una pianta si dice Vermena
Lavino Himen, Que) patio di Vegezio; de re militart lib, 1. cap. 11. Quemadns
dut ad feuta viminea, vel ad palos antiqui exercebant tyrones: L' antico vo
tore traduce csi. Come a feudi farti di vermene, 0 paliy tt provavand 4C.
GLI giunta addosso la piena, Sono accadute loro tutte le maggiori:
piena è prefa nel fenfo detto sopra C, 1. stan. 84.
eo FAR in mo che non s' habia a metter la spada al fanco, Far in modd ch
negozio s' aggiufti, scnz' havere ad adoprare le armi, che si dice Agginftarla
la spada nel fodero, i
Sé si puo far ds manco, Se la necefita non forzi a fare in questa maniera.
GAMB ASTORT A, ¢ Baconero. Nomi di Diavoli inventati qui dal' | Poetas }
nelio feflo modo, che inventati furono i nomi di Barbariccia, € Parfarelte
simili.
BALZA in casa, Va velocemente in casa. Zalcare propriamente si ¢
faltare, che fa la palla, o pallone perquotendo in terra, Vedi sopra CG;
SPOGLLASLin capelli, Si spoglia ignuda, e scioglie le trecce dei
vuol intender il Poeta, se bene si scrue del detto /pegliar/i im capes che si
adoperare ogni suo fapere, ¢ tutta l'applicazione per fare una tal cosa; per in-
tendere ancora che Martinazza s'era tutta applicata a fat, che Balser per
via d' incanto diloggi da Malmantile,
€ACCIANDOS! addosso,, Metiendofi addosso, E se bene il verbo ite uo!'
diy intromettere con violenza, noi lo pigliamo in fenfo di mettere 5 come i vede
nell' Ottava antecedente cacciar 1a (pada per metter la spadas
ALBERELLI, Vali diterra, Odi vetro, entro a' quali si ccobsladalatan,
guenti, ¢ cole simili; ¢ fon forse quei vafi, che i Latini chiamano alweolt
giiano il nome da questi. 2b Pai
BORBOTT ARE, EB' un certo parlar fra i denti poco inteso da chi l'Seana
che diciame anche brontolare, E' il Latino fubmnrmurare. Borboryttein
Greci è il romoréggiare, © mor morare che fanno le budella: Verbi psiiies al rian
fleflo naturale. !
e4 Più pari, Cio' a piedi giunti insieme. Questa voce pari y che per
vuol dire xgwatied di numero, ed il suo contrario ¢ dispari ( che diciamo se
i Latini dicono par, © impar, servc ancora per denotare ugwalita di
corpo » come gui; che s' intende, che un — non era ne pill innanai, er A
indietro dell' altro Si dice efer pari quando uno $'é vendicato con penn
ha pagato tutto quello che doveva, E ancora + esser pari ¢ gat &
quando non si pende per neflun verso. Strada pi ari per. re faut In
wa |' adoptiamo in tutte quelic cole, dove entri aa Peet

   

    

  

 
 
   
  
   
    
 
   
    

 
 

   
      
  
    
    
      

   
   
 

    
  
   
  

ah
 STANZA VIIL.
p vi scongiuro, € vi coman:
Pokies, € virtis ai questi incanti,

 QVINTO CANTARE:

229

YOST; Affumicati. Tinti da fumo, come sono i cammini, che fon
filiggine, che ¢:composta di fumo, ¢ d' umido. Lat. faliginof'.
ATE ame. Attendcte a me; Offeruate le mic parole, ¢ state attenti a

STANZA Ix
Per gl' imbrogli vi chiame,e I invenzioni,
Che ritrova il Legifta, ed il Notaio,
Quando per pelar meglio i buon pippiont
Gii aggira, che ne anco un' arcolaio;
Florsu, pexri di Sacchi di carboni,

   

|porcheria de' guardanfanti Per ques ladri del Sarto,e del Adngnaio,
le donne De per cofume, Che ti voglion rubar a tuo disperto; x
di pulci, ¢ fudiciume. Vycste fuor 7 venite al mio sae.

con diversi (congiuri chiama gli Spiriti infernali, per (eruirfenes
a far diloggiar Baldone da Malmantile: & l'Autore moftra il disprezzo, che egli
fa degl' incantefimi, facendo che Martinazza ¢oftringa i demonj con le cose ri-
— ditoleé, che egli mette in queste due Orcaves «

, SCONGIVRARE. Queito verbo t da noi usato per inteddere Eforcizzare,ciod

b “il Diavolo per via di giuramenti di formule facre dette per quetto
 Elorcifmi, cioé (congiuri; ¢ comunemicnte ¢ prefo in quetto fenfo, ed anche pill
atgamente si tira, come qui, alla manicra d' inuocare gli (piritizufata da'Maghi,

se bene il suo proprio significato è¢ domandare, o chiedere con grande ardenza,

! edéin to del verbo pregart dicendofi. “i prego, vi supplico, vi feongin.

ORCHERLA. Si dice non folamente un' atto sporco, ed illecito, ma ancora
una matetia schifa, sporca, ¢ brutta, o otal fatta..Come per efempio: 1/ tale
Seve wis crarione, che rinfed una bella porcheria, La voftra mercanzia non bebbe efite,
perché a una porcheria: I Libri di quel Mercante furono abbruciaci, perché
rat Pieni di parrire falfe, a' altre porcherie. Varchi nelle Rorie Fiorentine dice:
Era appunto sparfa in Firenze  usanza a? andar in 2axrera,  mantello, che era una
Ia por « Questa voce Porcheria significante disprezzo potrebbe venire dal
Latino porcaria, che vuol dir I utero delle Vacche, 0 délle Troie, dopo cht han-
NO partorito, o per dirla colle parole di Plinio lib.11. Cap. 37. Yuluam partu edito,
€tali vulue, particolarmente quaado non avevano condotto il parto, ma si era~
no feonciate, dagli antichi Romani erano manger per una cosa conser
la Porcaria non la mangiavano tanto voientieri,forse per efler cosa più schifa.Era
chiamata porcaria in un certo modo per disprezzo, € così ha portato ay
nol il ignificato, che ritiene di disprezz0, ed abbominazione. Ma la pil fem-
plice origine € da porco animale immondo; € così deta porcheria, cosa da porci,
Some furfanteria, cosa da furfanci,¢ simili.
- GVARDANP ANTE. E? uno strumento compolito di cerchi di filo di fetro in
tondo; il quale portano le donne Spagnuole, © circonda loro fa cintura foto le
Velli, le quali fa gonfiare: E lo dicono guardanfame, perch egli difende dalles
offe l'infante, cio' la creatura, che hanno le doane prtgne dentro ali'urero,
Perché questa foggia di veftire, che havevano cominciata ad usare Ie donne di

. Firenze,

/ 70. Latino obfecro, obre/for.

  
  
 
 
 
 
   
   
   
    
  
Yo!

  
  
 
        

230
Firenze, conosciuta rae
dava a poco a poco difufa

ne il Bando, cioé l'efilio, ¢ proibizione «

PIPPIONT. Piccioni. S' intende gente femplice,
sono i pippioni, colambarum pulli, colombi giovani. £
Cavar danari di mano al corrivo. i
ARCOLAIO, Strumento, sopr' al quale s' a
@' altra materia per incannarle, 0 aggomitolarle co
ed ¢ un moto perpetuo, ¢ però dice ageira che ne anc!
gira bene, ed aflai: ed aggirare in qucito luogo vuol dire ingannare; ¢
ratore, ingannatore, Coa Bind » si prende per huomo aggiratore; ¢
fare per girare, cioé non si rinuenire col ceruellogL. delirar: \ gg
Ingannare; Latino circumuenire.: Ns

STANZA X.;;
Tutto lt Inferno a cos; gran parele

Vien fibilando, ¢ intorno le faltella

Come dall' alba al tramontar del fole y

Fa quel, ch' è morfo dalla Tarantella, 1 Com

DIRT ANZA), SL K

Ed a far ch'ei si pigli quella spracca Ma perché tu mi voglia far

Senza cagion,gli par ch'ell'abbia iltorto, 'Di darmi Baconero,
Perché dalla Pro 'onda sua baracca Perch'jo mi va dell'
A Malmantil non ¢ la via dell orto. 4n cosa che mi preme, ec
Corpo | ( dic'ellayed al C elon l'attacca) Plutone allor quei due fa?
A venir infin qui tu [arai morto} E la frrada si piglia della p
Ma senti il mio Pluton, non t adirare, Seguito da i sugi sudditi,
Che venir non t' ho fatto fine quare, Posson fondar la C ompagnia.
Agli scongiuri di Martinazza le comparisce avanti Plutone con moltil
ed ella gli chiede Baconero, ¢ Gambaftorta, Eile la(cia quivi
nj,¢con gi altri se ne torna all' Inferno..

SIBILANDO, Soffiando,filchiando. E' yoce Jatina, che ritiene il suo.
10. Verg, En. 11, edrrettis horret fquamis, © fibilat ore. Iptendiamo
mente il filchiare de i ferpenti.

SALTELLA. Fa spelfi, ¢ piccoli falti; & il faltar delle Rane, Vedi
6. stan. 37,

MORO dalla Tarantella, Per la Calavria, ¢ Puglia dicono si trovi un
lo ragno detto Tarantola, o Tarantella, il quale nato ex purrs scappa
fure della terra in tempo di state. Geeta mordendo un'huomo,gli mepte ad
una infermita specie di rabbia, che Jo forza a ballare contiaovament
vata, al tramontar del fole, ne prova quicte, se non quando fente
chitarra 0 con altro strumento simile,un' aria detta percio la Tanane
faono guefto rale attarantato si affatica a ballare tanto', che flracco
morto; € stato in guefto fuenimento qualche hora,si rizza, ¢ ccfla,
flando fano per qualche giorno: E perché in quel paefe si trovano moll

   
 
  

      
    
  
   
 
 
 
 
 
       
  
   
    
      
  
 
   

 
 

    
  
   
   
        
  
   

-QVINTO CANTARE: 231

“Dicono, che tale infermita duri quanto dura la vita di
attarantato, la quale dicono, che non paffi tre anni;
posta pagati da quei Comuni, i quali vanao cercando
gli per universal benefizio, e ne hanno un tanto
un Rettore a cid deputato. Dicono in oltre, che
ficato provi la detta infermita ogni anno per un méfe poco pill,o
torno a' quei giorni,ne' quali fu morficato, che fara intorno al Sol
: trovino di quelli che la provino ogni mefe per qualche giorno,
> 0 Tarantella dalla Città di Taranto, nel cui territorio forse
te si trova. Ti Lalli nell' En. Tr. lib. 1, stan, 22. dice,
| Enea quantunque bravo anch' ei tremante
, Morfo dalla T arantola parea,
Gl'introna la testa con le strida: lo sbalordisce; lo fa aflordare

'vecchie. E' invecchiato, s' intende uno che si tratti da vecchio;

Valo di rame, col quale si caval' acqua dai pozzi. Vedi forto

an. 3. Ha if decto far come le fecchie fenz' altra aggiunta, significa andar in
af come fanno le fecchie infunate nella Carrucola.

Intende abitazione, Che baracca vuol propriamente dire quel

(0 i soldati in campagna per loro abitazione, nel quale 2a

€ capannello di frasche, 0 d' altro,col quale si difendono dal

icque. Viene dal verbo barrare, che vuol dir Circondare, 0 accer-

ice anche frabacca, © corrottamente, © pure ¢0 guid trabibus conftrue

 
    
      
 
   
   

   

  

via dell' orto, Quelto dettato significa; la via & Junghissima, e difa-
r ordinario dall' orto alla casa non è pil lungo viaggio, che ca~
ri delia porta, la quale di casa esce nell' orto, essendo per lo
a Città gli orti appiceati alle cafe.
PO! ed al Celon ? attacca, Vuol dir Corpo del Cielo. Si dice Corpo del
a del diavolo, ec, Ma quando uno pafia pib ia beftemmiando les
: Bil ateacca al Celone per intendere 5 egli entra nel Ciclo, cio'
jumi Celefti; EB per render più oscuro questo detto,ci serviamo del-
ne, che wuol dir quel panno, che si mette sopr' alla tavola da meafa
di distenderui sopra 1a tovaglia..
. Detto ironico per moftrar la poca stima, che si fa della Fatica,
durata uno a nostro pro, ¢d il poco grado, che gli fen' habbia,mafii«
tale ne fa grande oftentazionc. 
: + Voci latine usate nel suo significato; ¢ dicefi: Won fine qua-
3, ¢ significa non senza qualche fine, o cagione, Franco Sacch,
de Gli venne vogla d' andar a strovare il Re Adovardo, ¢ won fine quare, perché

tree molto lodar|o.

IN fondar la Compagnia de' brutti. Sono'tutti bruttissimi. Habbiamo in.
tun' Accademia, o Compagnia detta de' Brutti, 1a quale si raguna ogni
giorno di Befana ( che così si dice il giorno dell' Epifania ) ed in un lau-

tulimo,

 
 
  
 
 

  
   

 
 
  

 
 

  
     
  
 
 
 
 
 
 
 
 
 
 
 
   
   
  
   
  
    
  
  
  
    
  
  
    
    
    

age MALMANTILB, 5
tissimo, e stravagante fimposso si crea il |
iftim 8: maf Confininanmmnt

la il Fondatore,¢ si fa sempre il pil brutto.
Poeta.;
STANZA XIII

Lascian Plutone,e corroz dalla Druda
J dug spirti, aspettando il suo decreto,
Ed ella allor che fa da Ceccofuda
Per far si che Baldon dia volta a dreto,
Ed anche se si puo ch' ei vada a Luda,
Gli prega, che le dien qualche fegreto
Da far Jenz' altre guerre, ovver conte/e,
Che quelle genti sfrattino il pacfe.

STANZA X vo

Pers se non finghiam ch' egli le feriva
Ch'il suo rivale (adeffo ch'egli ha inteso
Chrei #¢ partite) con la gente arriva,
Per volergliela (u leyar di pefa,

E che se propria è ver, che per lei viva Hor dunque tu che [ei saputa
( Com' ei /peffo giurd) d' amore accefo, Che non (a cedi manco a Ci
E feglit cara lo dimoffri, e prenda, Scrivi la carta, che tu fai ch
Ed armi,e braviye corra,e la difenda, Sian tutti un monte d'a

I Diayoli rrovano l'inyenzione di far diloggiar Baldone da Mal
guefta € fargli intendere, che la Geva sua dama ¢ in pericolo d' efler
cono a Martinazza, che sCriva la lettera. i

DRVDA, Innamorata, amante, ec, se bene non fempre si piglia
to difoneftofo; Qui intende dama di Plytone, che era Martinazga, che » some
firega, haveva Jui per innamorato. AY

FA da Cecco fuda, S' affanna,s' affatica. Scherza con questo nome:
perché quand' uno s' affatica, ¢ s' affanna (enza proposito, moftrando d
cose diclamo: “tale /uda. Di questa natura era quel Cortigiano descritt
Berni nelle Rime. Ser Cecco non puo far senza la Corte, Ne la Corte puo iar
Ser Cecco,: a oa

VADA4« Buda, Vada via per non tornar pi. Proverbio nato dalla guerra
che già fece il Turco contro Lodovico Re d' V ria quando acquifto Bud
ca l'anno 1626., che vi morirono quaii cutti li Criftiant, che yi andaronos ¢
il medesimo Re; E però da quel tempo in qua dicendofi: // rale ¢ andato 4 Bu
s' intende è andato via per non ritornar pil, 0 vero € morto, ed ha il mede
fenfo,e per la medesima cagione; // rale ¢ andato a scio. F andato a Patraffojschet-
zo fulla Città d' Acaia famofa per tl martirio di S. Andrea, come s¢ si dicefleua
Latino: ivir Patras; ¢ fulla frale uiata dalla scrittura, sopra quei che muolon0 »
¢ si feppelliscono, quafi dica; E' andato ad patres /uos è
SFRATTINO, 0 sbrattino il paefe. Ripuliscano il paefe, cioé (ene vadanO.
DAR' a due tavole'a un tratto, Far due negozzi in uno ste(io tempo. Tratt0
dal giuoco di sbaraglino, nel quale con un fol tro, si dia duc, ¢ tre tavole, 0
girelle. Si dice ie: far' wn Viaggio, ¢ due fernizj. Vedi sorto C, 6, stan. ae.

  

   
  

     
  
   

 

  
 

 
 
   
  
    
   
  

QVINTO CANTARE, 233
a in due faffe. Attendere a due partiti. Vitumeligere, G alte-
eaiasicinn o. eae ahbe dirfi Mont' Vghi dalla fa-

illaggio vicino a Firenze.

antichissima Fiorentina. Ricordano Malespini nella Stor. Fio-
l no ebbe nome V 60 5 questi anche sue nobilissimo
no sei ql dicefona gli Vghi, ¢ per innanzi il poggio,
'si chiama Montughi, s'é chiamato per loro... Lo stesso conferma Gio,
RE
7 balan. Allora allora; Subito subito... Vulla interposita morula.
INE, Specie di Serpe, detto così, perché forse vada yeloce comes
facta, ¢ credo tuber dei Latini,
y 4A. Propriamente vuol dire Remiganti di galera: Ma qui @ prefa per
, come si trova anche prefa in pib Storie Fiorentine antiche, ¢ sopras
76.5 ¢ fora C. 11. stan. 76. dal Latino tarma, se bene propriamente si
di soldati a cavailo.
OL ammazzar beftie, ¢ persone. Vuol difertare il paefe. Quando vogliamo
ler uno, che vanti di voler far gran brayure, ¢ non lo giudichiamo atto a
me Veruna, diciamo Vol ammazzar be/tie,¢ persone. Ed in tal senfo di derifio-
'DEE preio nel presente luogo. IL Berni nelle rime congiunfe queste due voci cu-
allor che disse: Con wn mondo di bestie, ¢ di perfane,
saputa. Sci dotca; sei sCientifica. Donna fapura, facciura, faccente vuol
di Vna donna, che in tutte le cose vuol far da macftra. Colla steffa figura di
se faccente, diceli e4uuertito, edccorto,edunifato, ¢ dagli antichi Senrito
ee che avverta, ¢ che s' accorga delle cose, ¢ che stia full' avvilo, ¢
t, ll participio paffivo in forza di attivo.
| SLAMO una mana a afini, e di buoi, Siamo tanti ignoranti. Per lo più a que-
fle dug ie ed al Castrone affomigliamo coloro, che non hanno scienza alcu-
ha. S¢ bene  Autore fapeva, che il Demonio possiede tutte le scienze ( che cosh
a uo Greco nome Daemon, cioé fapiente; ¢ noi d' uno che sappia eccel-
a che cosa dichiamo; Egli ¢ un Demonio; nondimeno ha voluto,
efi due Diavoli si dichiarino ignoranti, acciò che si creda. più facilmence
Vertore, che fecero di scambiare le palle, come vedremo.
STANZA XVII, STANZA XVIIL
Non ti dé contre atispond' ella, a questo, E per dar al negoxio pitt colore,
Ed che voi vi conoschiate: In forma voglio wr' io d' una gomare
| Her si, dice i Demonia, ferivi prefta Della sua Geva detta Monafiore
parole in tal genere agginftate Confidente del Duca in ogni affare;
' a; ma vedi, io mi prote(to, Gambafforta verra da fernitore,
6 portai mai lettere,o imbasciate, Che moftri di venirmi a accompagnare,
Scriniforgiunge queische quato al porta E gid per questo ho fatte far di cera
ito qui con Gambaftorta. Due pallennach'e bianca,e Laltra nera,

 
    
   
  
 
 
 
  
   
   
  
  
  

——

SS ———

= TaAsS ASD

   
 

s
#
si
¥
8
i]
¥
4

Eccoms

 

Gg STAN.

 
 

 

  

" i
234 MALMANTILE >
STANZA XIX. shoo eg rah
Quand' un tien quefia neva in una brica, La nera a lui dare c
Di subito a un' huom prende figura; 1

E s'ei vi chinde quellaltra ch' bianca,
dn femmina si muta, e trasfigura. v
Si che riguarda ben 8 altro ci mance, La Strega qui gli dice,
E distendi mai più quefha feriteura; Perch ella ferive,e guaspoleha
Chil mio compagno,ed io qua per viaggio Ma lo [cancella, ¢ mettelo in |
Ci marerem Leffigte, ¢ il personaggio. ° Così fie" la carta, ela figilla

STANZA XkKI. van
Le fa la soprascritra, ¢ poi finisce
A più a' un ghirigere ta propria mano,
E con eff quel diavolo /pedisce

   
  

  
  
 
 
       
  
 

    

 
 
 
 

no la medesima lettera per portarla un di loro trasformato in Mona
1 alco ia un Servo per via di due palle, e se ne vanno così da Baldo:
havere feambiate le palle, chi dovea apparire la Fiore, appare il
rono scoperti.:
NON portai mai lettere, 0 imbasciate, La maggiore offela, che si
certe donnicciuole,¢ il dir loro; porta lettere; porta imbasciate; fa servig:
polli ( detto credo io dal Franzefe Puuler, che significa letterino @' amy
portatrice di lettere amorofe ) perché vuol dire Rufhiana: B però
Martinazza, che non vuole quest' offefa addoflo si dichiara, che non (
portar lettere, 0 ambasciate, cioé da far la ruffiana. vb
ECCOMI lefto, Eccomi pronto: Eccomi all' ordine. Zefo in ' f
vaol dir difinuolto, e senza imbarazzi. oth SBE!
DAR colore.al negorio, Bar' apparic = vero quel che è incerto; Da:
similitudine. Quetto fanno appreffo i Rettorici quei, che da loro foro
Colori, Givvenale dice: dic, Quinttiliane, colorem, + yo
COM ARE. Quella che tiene la creawura'al Battefimo. E qui il Poeta
il coftume, che in simili amori per Jo più la Balia, la Comare sono ne »
portano le parole, yagi
4ONA. E parola fincopata da Madonna, ed è il titolo che si da comunt-
mente alle donne d' infima plebe dicendofi in diminuzione Signora » Madonna»
Monna, come Signore, Meffere, Sere. Ma perché Afonna oltce al significato dl
Bertuccia, ha ancora altro significato osceno (almeno in lingua Veneziana) no!
per sfuggir l'equivoco; hoggi coftumiamo dire Agora enon Monona:
ALAT più, Hormai, Cioé finifeila una volta: E' termine dimoftrative @
certa impazienza, ¢ si dice: Ob mai piis: ed: il latino eandem aliquande:¢
confa con I imperativo s dicendoGi: Ob mai pis: finitela, OE RB S|
POSTILLA, Nel nostro idioma ha diversi significati; perch? 6 vaol }
( figuratamente fecondo Dante ) immagine d' un' oggetto, che ritorni alla
weduta da un yetro, o dall' acqua chiara, Dan, Par. C. 30,

  
      
       
 
      
 

           
  
    
   
  
   

  
 
        
     

">
 

=

<= SSaeu SE BS

a

QVINTO CANTARE:

 

 
 

Ls 235
Quali per-verri trasparent?, e terff,
Over per acque nitide, ¢ tranquil,
Wem Wardens « Won si profonde, ch' i fondi fien perfi';
6 Lernan de nostri vifi le postiile
Pgs » Debili st, che perla in bianca fronte
at  Won vien men tofto alle nostre pupille.

O viiol dire annotazioni, 0 glofa, che i Latini dicono expo/tio. O si piglia per

aggiunta, ed.¢ composta di due dizioni po?, & ila. Quali dica,

Postilla verba, cio dopo quelle parole, scrivi, o aggiungi questo, ¢ questo, E

annotazioni, glofe, o aggiunte hoggi per /ofilla intendiamo anche»

del libro, cioé quel bianco che si la(cia di sotto, ¢ di sopra, ¢ dalles

foglio scrivendo, o flampando: Si che scrivere in possilla vuol dire scri-

Margine;¢ s' intende ogni aggiunta, che si faccia al tefto scritto,o

in qualfivoglia Iuogo della carta © sia di sotto, 0 di sopra, 0 dalle ban-

idei versi ordinati, ¢ regolati; ed in questo modo, © luogo, dice che
"Marti

—. E'un tratteggio di penna usato per Io pitt nelle sopra(crittes

» come moftra il Poeta nel presente luogo, che faccia Martinazza..

Ghirigoro da' nostri antichi era detto in volgare il nome Latino di Gregorio;
Ghirigoro trovafi fempre coftantemente scritto nel Malespini, ¢ nel

Villani; come era 1a lingua di quel tempo. Ma qui Ghirigoro apparilce per av-

Ventura dal girare, ¢ rigirare della penna così detto.. E le parole /n propria mano

8 usago nelle soprafericte di quelle lettcre, le quali si mandano a yno, che sia ne!
luogo, o Città, 0 vero poco lontano da colui che scrive.

 
    
 
 

el.

STANZA XXII
Che Baconera ii quale ¢ un' avventato,
Neb dar la palla alt' altro di nascofto,
“Senza grardarla prima havea [cabiato,
a: fattoun grad' arrofto,
i quand' a Baldone egli ¢ arrivata
Dice cafe dal ver troppo discofto,
i afferma d'effer dina,e stbra
Huamo alla barba all abito,e alle mebra,
STANZA XXIIL

ECambaflorca anch'ei balordo, e frolto,

Mencr'apparir ficrede un' hue dabbene,
Alla Favela, alla prefenza, ¢ al volta
Per Vna fa fernizrj ognun 1a tiene,

» Afoglio intantoil Duca havea lor tolto,
Eveduto lo scritta, ¢ quel contiene;
Refha certo di quanto eraindovine,
Ch i furbi vorrian farlo Calandrino,

 

STANZA XXIV.

E poiché gli hanno detto, che la Geva 5
A lui gli manda con quel foglioa posta
Ma prima che da loro lo riceva
Hann! ordine d' haverae 1a risposta;
E foggiunto y che mentr' ella scrived,
Getrava gocciolon di questa posta,
Per il trabufograndech'cllahahavuto,
Come potra sentir dal contenuto,

STANZA XXV,

Egli ¢ (dic? egli ) um gran parabolano y
Chi dice ch' ell' ha scritto la presente,
Quand'ella no piglio mai penna in mata,
E fo di certo ch' ella n' ¢ innocente
Che poi tu sia la Geva, ch' in Venane
A me fu molto nota, ¢ confidente,

E tu sia buom, 4 dirla in coscienza,
AA me non pare y¢ nego confeguenza,

Gg STAN:
 
  
   

236 MALMANTILE —
STANZA XKVL»~

1 non compagni a una risposta tale - Ed alle

Guardanft in vifo,inquelfendofiaccorti,

Chregli hanno equivocato,e fatto male;

Refhan quivi allibbiti, ¢ merzi morti, 'Di Baldone, e di tutta

Giunti quei Diavoli da Baldone, credendofi rapprefensare uno
V' altro il Servo, non essendo accorti d' havere scambiate le palle,se
ambasciata: Ma Baldone, comprefo, che questa era una furberia, non
cid, quanto dall' essergli noto, che la Geva non fapeva ferivere 5 se gli cl
nanzi con una gran quantita di filchiate. ocx prcee
tVENT ATO, Vno che opera senza considerazione,e furiofamente.
mo fd 3 € precipitofo; dal ivo Latino ad. in si
to d' avvenirfi, cioé imbatterfi in una cola con velocita »e con furia,
DI nascofte, B \o feo, che Di foppiatto detto sopra C, 1. stan. 75.
PIGLLAR un eros « Pigliare errore;Intender una cosa per un' altra, §
pigtiare un granchio a fecco quando uno nel picchiar qualche materialesh !
si batte il marteilo sopr' alle dita, o si (erra le dita fra due materiali
ecrore intendiamo poi far un'errore, quando diciamo pigtiare un gr Beral +
Che Virgilio ha prefo Vn granciporro, "phase:

FAR un arrofto. Far un' errore. E' lo steffo che pigliar un granchio, vad
per avventura dal verbo arrofarsi, che vuol dir affaticarsi spropositatamente
furiofamente; ¢ le cose fatte in furia non si fanno mai bene, a

BALORODO, ¢ folto, Sinonimi che significano Huomo senza giudi a vO~
ce stolto è pura latina, e balordo ¢ lo stesso che in Lat, bardus. aeons )

VNAF A fernixxj. Come's'é detto sopra s' intende una Rufiana, ©

VOG LION farlo Calandrino, Calandrino, fecondo che dice il
sue Novelle, fu un' huomo tanto credulo, che gli fu dato ad intendere fino, che
egli era pregno, ¢ però da coftui diciamo 7% mi vnoi far Calandrino per intend —
re Tu mi vuoi far credere quel che io fo, che non è vero, Si dice anche far
pedino, da uno de' nostri tempi della natura di Calandrino

HANNO ordine @ haverne 1a risposta, 1 Poeta per maggiormente a
castronaggine di coftoro, fa che chieggano la risposta prima di p a

ropotta. ¢

CETTAVA 'coccioloni di queffa posta, Lagrimava gagliardamente. Il termine —
Di questa posta ignitica grofiezza; erano pere ds questa posta, cioè pere grodidi-
me,¢ si ja: » che colui, il quale dice cos,, accompagni il parlare col geft
delic mani dimoftrante la groffezza di quella tal cosa', Si dice anche samo fatte;
tanto groffe, come vedremo sotto C. 10, stan. 17. 18. ¢ 36.

TRAMBVSTO, Travaglio, rimescolamento, follevamento
fa di disgrazie. 3: aig

PARABOLANO, Bugiardo; chiacchierone; [propositato: Da Parabola, cide
similitudiae,o Racconto; ne'Capitoli di Carlo il Caluo si legge. Par: itty
fimul, & considerauernnt. Parlarono insieme, Du Frefne alla V. Parabola,

SO ch' ella n' 2 innocente. Intende; io fo ch' ella non fa scrivere. Per esprit
re uno che non habbia ne pure una miaima notizia d' una tal cola

  
 
   
 

   

  
 
  

 
 
 
 
 
     
 
  
   
    
  
  

   
  
  
     

   
  

   
 

d'animo per eae

        
   
  
 
   

QVINTO 'CANTARE:

alcxno nella tal cosa 0 ¢ innocente della tal cosa:

go il tutto: perché negando la confé apn viene a
4 e tutto I arguimento, € così tutto il dif

un subito timore 50 wergognie,e¢ percid
'solore fmorto y¢ gialliccio, come, feccandofi, diventano le potatu-
che si chiamano éibbie,dalla qual voce viene alubbitoye altibbire, Ve~
rio della Cra/ca alla Voce “Alibbire + IL Varchi Stor, Fior. lib, 10.
ova il quale incontinente ( quafi le fuffe venuta meno la terra

237

 
 
   
     
 
      
    
 
    
  

: ie,
ATA. Romore di yoci, filchi, urli, battimenti di mani, ¢d'altro,
uno per dargli la burla, Far le fischiate.a uno, quel che i Lat.

STANZA XXIX.
Che la padrona il tutto le comparte
Come s'in Malmantil fien due Regine,
 Psiche egnor di attorno, Anzi il bando si manda da sua parte,
ggni quattro palfi fa un Jamento. Perch' ella foffia il nafo alle galline 5
to tutto quanto il giorno » Così poi c bebbe dato libro,¢ carte,
cento volte,¢ cento Entra nell' un vie un, che non ha fine y
Malmantile, ¢ Similmente Coftui, che qusvi s' ¢ posso a bottega
tinaza,e fev'é ds presente. A legger sopra il libro della Strega,
TANZA XXVIIL STANZA XXX,
xn, ch' al fin la mette per la via Quest altro, che non cerca da coftui
sche quest orrida Befana, Di questi cingue soldi, havendo fretta,
un.toz%0 haveva,careftia Poichegli ha ee quel che fa per luiy
sitet erba porcedana y Spronailcavalle tutto aun tepo,eshietta,
di.gran soldi.in [ua balia, La donna che trovare il suo,colui
una cefacome una Degana, Digiorno in giorno per tal mezro aspetta,
Corteeingrado,e giunta a fegno, Per no loperder adocchio,ech'ei le machi,
i totum continens del Regno, Segue la frarna,egli nasipresi ifianchi,
| Poeta a parlar di Calagrillo; il quale camminando Psiche s' imbatte»
he le da avvilo di dove sia 'Martinazza.
MARCHIARE, Si dice marciare, che vuol dir camminare. Voce Francele,
ma già fatta Italiana. Vedi sopra C. 1, stan. 43. EB più accofto alla pronunzia,
-Oltramontana, dicefi anche Aerciare, forse da Marcia, contrada, pace, cam-
'Mino dane/marce disse il Villani la Danimarca; cioé Danefe Contrada,
'ANA. Intendiamo Donna brutta, malfatta. Vedi forto C. 8. stan, 30.
C.9, fan. 1.
T0ZZO. S intende pezzo di pane. Haver careftia d' un toxxo, Vuol dire ef-
mendico, pezzente.
ST AVA come la porceliana. Cioè terra terra, come l' erba porcellana, che fer-
2 per terra, € non alza mai virgulti; detta porcellana dal Latino Portulaca.
questo detto significa Vno che sia in povero stato, ¢ non habbia modo di falle-
-Varsi, che i ooo pure dicevano humi sacere.
sea balia. In (ao potere,e dominio. Balia ¢ voce fatta venire dal Mroh-

  
 

   
 
    

 

       
 
      
     
    
  

*
BS
5
€

 
 

  
   

   

238 MALMANTILE™)
ni dalla Greca Buleia, che fuona lo steffo che Bule; cic

Senato. A noi fuona Potefta,giurifdizione,autorita,e quel che i Latini
poreflas imperium, Dan. Purg, C. 1. bt

 
    
    
     
   

Che pure
Petr.C.36. Afentre ch' il corpe è vivo y i
Hai tu il freno in balia de' pensier tuoi
HA una casa come una Dogana, Cioè picna di robe, come sono le Dog e
ne di mercanzic. set
IL Bando va da parte sua, Ciok, ella comanda'. 7A oP
SOFFLA it nafo alle galline. Ella fa tutte le faccende. E questi tre
Totum continens de/ Regno;Rando va da parte fuaje fofiail nafoallegallineh
lo steffo significato;ma di quetto ci serviamo per lo pili per derifione,per in
uno che habbia ambizione d'efler creduto gran miniftro,ed habbiaim
neggi d'un governo,e non sia vero; che per ischerzo direbbefi anche
En.Tr.l.4.st.15.SopratrnrtoaGinnon, che del far ragza E'detta Varcifanfana,
DAR libro ye carte. Dare fata notizia d'alcuno, 'Viene da'coloro,i
havendo debito co' Magiftrati, fon mandati in efazione a i Miniftri forenfi
wali Miniftri i Magiftrati mandano il contrafegno del libro, nel quale €
il debito di quel tale, il nome, ¢ cafato di eflo, !'origine, ¢ somma delde
ed a quante carte ¢ la ua partita: E questo si dice dar libro, e carte,
in proverbio, significa Dar notizia chiara, ed efatta d' alcuno; 0
habbia fatta un' azione per altro occulta. 4
ENTRA nell' un vie uno, Faun discorso da non ulcirne mai, cor
be se uno voleffe seguitare Vn vie uno fa uno, due vit due fa quattra y ec, che sande-
rebbe nell' infinito. Dice il Varchi nel suo Ercolano, che in questo fenforfidice
Cantar la canzone del? uccelline', Con tal dettato s* elprime un chiacchierone'y
che cicalando, faccia 'molte digreffioni spropositate per allungare il suo cicala~
mento con racconti assai sConuenevoli, che si dice; Entrare in un ginepraio
re di palo in frasca, elk
S* £ meffo a bortega, $'& prefo per arte, per suo mefticro, o negozio. Quan
do uno fa qualche operazione con tutta applicazione, ed attenzione, ¢ con dime
strazione di voler durare assai, diciamo; Coftui sé meffo a bottega, oh
LEGGER ful libro d aicuno. Narrar le azioni, qualita, ¢ stato d'aleuno,
NON cerca questi cinque soldi, Non cerca, non gl' importa, non proccuras
fapere questa cola. Quand' altri fa un discorso, ¢ fa una digreflione senza tornat”
piu al primo proposito, se lidice: Voi pagherete la pena de' cingue soldi. Vedi fot-
to C, 8. stan. 15. E pero dicendo: Non cerco que/ti cingue soldi, § intende;non mi
curo di oes gnefta pena de' cingue soldi, con obligarti a seguitare il prin>
cipiato discorto. ai
SBIETT-A. Scappa via prefto, Vedi sotto C. 7. stan. $7, 4
IL suo colui. Ui fao amante, cioé Cupido. Se
PER non Ws perder a' occhio, Perché non le esca di vifta, Per non Jo fmarrite. —
SEGVIT A la flarna, Quand' uno seguita un' altro per haver da lui qualche
favore, diciamo: Zifeguita la starna. E Gi dice la farna, © non altro uccell0s

   
    
 
 

    
   
 
  
   
      
   
   
  
 
     
 
   
  
   

 
 

ie

SEPRTA TES ESS

SS “eS wt a

SAS

a

3. flan.s, Franzele.

 
 

paletettrorcs:

ne ilo
4 STANZA XXXL
— were i

ipeepet dot ons 3e ar fide,

'inttorna fon piie delle pecchie;
eh soldayed a Suis faite udati,
Che havido del guerrier noticievecchie,
 Gliva incontro, Vaccoglie ye riverisce,
t a luicon DY arms 8 offerisce.
TANZA XXXil.
i, fazginnfe, ch? io si preghé
l'donna rimaner fernito,
a questo ferro lei stimpieghi
| Per conto qua Paces — j
t tanto Cavalier nulla si nieghi,

~Risponde acio Baldon tutto complito,
* Tu fei padrone; fa cio che tu vuoi y
Nom ci van cirimonie fra di noi,
Sita he
Over cl? io me La metta in [ul liuto,
“Ot veglia tener Poche in pastura,
"Come quel che ci vada ritenuto

Per mancanza di cuore, v\per paura,

    
 

ss QVAN TO CANTARE:
Q seguitarle,ofleruandole dove si posano,e straccanioie

239

STANZA XXXIIL
Ti servsro di feriverti alla banca,
E in tanto per adeffo io ti ceafegne
M1 gonfalon di questaciarpa bianca,
Che tra te schiere? il nostro cétraffegns:
Tal-che libero il paffo, ¢ feala franca
Haurai per dar' effetto al tno difegno;
Che non fo qual si sia, ne lo domando;
Pero va pur ch'io refto al tuo comande,
STANZA XXXIV,

Ei lo'ringrazia, E ito più da preja,
Ove sia chinfo di Psiche il bel Sole,
Ad essa dice:In quanto al tuo intereffo,
Fin qui non t'bo servito,e me ne duole,
Che tu non pensi, bavendati prome(fos
Ch' io facciafango delle mie parole,

E ch'il mioindagio, eilnorifoluer nulla
Sia fatoun voler darti erba traspulla,
XXXV.

Perché si come haurai date vedite '
Won ho fin qui trovata congiuntura
Di chi m' indirizzaffe qua al Castello,
Per porerne cavar cappa, 0 mantelle.

~ Jo-con Psiche arriva al Campo, ¢ chiede soldo: Baldone l'accetta, ¢
nza'd' ancare a servir Psiche, con la quale avviandofi verso Malmanti.

da,
£ Calagrillo'fi scufa di non' haver pri

'ima servita.

alla banca, Atrolare uno per soldato: Banca diciamo quel luago
ee soldati, e dove fon loro pagatii denari ae fipend}:

7 (LONE. Vuol propriamente dire vefillo; ma
@infegna, Vediril Voilio de vitijs fermonis lib, 1. ove di qui

& piglia per ogni forta
la voce.

CLARPA, E' una legaccia di drappo, che dai soldan fi'cinge come la cintu-
'ra della spada.. E per altro ciarpa vuol dire quel che accennammo sopra Cans.

Escharpe.

SCALA franca. Franchigia; Liberta d' andare,'o fare. Paffo libero.

I ston fango delle sue parole. Dilprezzare la parola data ¢ non offeruar Ie pro-

DAR erba'eraftulla, AMdetterla ful linto; miandar voche in paffara hanno tutti tre
verbie dare,

Jo Reffo signific

ato, che € trattener' uno con Soa. Lat.
$1

NZA XX

Risponde Priche a questa'diceria:

| Lo non entro Signore in questi meriti y
Non ho parlato mai, ne che ti sia 2
So o spediro, o ver che tm ti peritis

 

Quel ke tu (fii, tutte tua cortefia,y
Per tal? accettose'l Ciel te lo rimeriti,
'Co darti invita honor,fama,ericchezza,
Sanita dopo morte 5 ed allegrenza.
STAN-
Se

240 MALMANTILE.
STANZA XXXVIL STANZA XX¥XVHE
Sta quieta, le dic egli, e ti conforta, Van eee @! occhiacci orlati di favore 'a
Chia vegtia adeffa dar fuoco al ve/paio, 03 addolfa & unt eratte ght [qnaderna

Così col Corno, il quale Al colle porta', Che par quand' il Faina alle se bare
lanternay.

Chiama la guardia, 0 vero it portinaio, In facia mi spalanca la i
Non ¢ si prefeo il gatto in fu la porta y E mediante.un certo pirxz \ ee
Quand es fente la voce del beccaio; Chrei fente al colo, pixzicotti alterna,
Quanto veloce a questo fuon la Ronda Ond? alle dita egli ha farsi i digali
Sopr' alle mura accoftafi alla sponda D' imorno a innumerabili mortali, —

Psiche rende grazie a Calagrillo della carita, che le promerte, ¢ facendo le lor
cirimonie, s' accoftano al Castella, dove Calagrilio, (onande il Corna, chiama
la sentinella, la quale subito s'affaccia alle (ponde delle mura, oy

DICERIA, Vuol dire Ragionamento, Discorso, Orazione: ma
voce è usata per lo più per intendere Ragionamento flugchevole, € odi perla
lunghezza, io

NON entro in questi meriti. Non parlo di queste cole. Ma questo detto has
wna certa forza d' e(primere: io.non ardisco d'entrar tanto in 1a col discorso;mae
niera, che viene dal folerfi dire; il merito della lite, 0 della causa, cio¢ limpor-
tanza del fatto. re

SANIT A, ed allegrezza dopo morte, E detto giocofo, perché un corpo mor
to non pud haver fanita, ne allegrezza,ne altre paflioni. Ma si potrebbe anche
dire, che questa donna, parlando iperbolico, voglia dire che eglt viva fano, ed
allegro fempre eziam dopo morte, il che ¢ imposhibile, come ¢ imposibile viver
mill' anni, € pure si dice: vi prego mille anni di vita, Sanied-è un* augurio sche
corrisponde at Greco hygiainein, cioè far fano, che metteva innanai alle sue ¢pi-
flole Pittagora devotissimo della sanita; degrezza corrisponde a quel fal
in principio espri i Greci co} nelle lor | sperché dove i La
tini pongono Salutem dicit, essi scrivevano Chairein, cioè come tradufle Orazio
in una sua Epiftola Gaxdere, yolendo dire, Ii tale,al tale desidera allegrexea fic
come in quell' altro modo usato da Pittagora: il tale al tale,desidera fanita.

DAR es al a 3 Violentare a ulcir fuora uno, che sia dentro; come se+
gue, quando si da fuoco'a un velpaio, che le velpe fon forzate dal tyaco a seap-
par fuori. Vedi Omero lib, 16, dell' Iliade, }

LA voce del Beccaio,. Vanno per Firenze alcuni Beccai, 0 Macellari yendendo
carne per dare a' gatti, ¢ fanno certe lor voci così ben conoscimee da i ot
gatti, soliti havere la carne, che appena coftoro hanno aperta la bocca, che i
gacti sono in fulla porta, A questi gatti afomiglia la guardia di Malmantile, che
a pena sentito il fuono del corno s' aftaccia alla muraglia, Delle voci, ¢ de'verli x
che fanno j venditori, che vanno attorno per inuitare il compratore, Senecacp. 14
56. lam libarij varias exclamationes,@& botularinm, & eruftularium »@ omnes pope M
narum infeitores, mercem sua quadam,& infignita modulatione vendentes $ 4),

t
”

   

SE

  
  

= 2.
seg

 

£ ia eS ezZenre eee

RONDA. Sidice quel Soldato di guardia, che rigita, e pafleggi Ia)
raglia della fortezza, vifitando la Sentinella, detca wa isi andsee incall ee
come i Franzefi dicono, aller en rond, stat

SPONDA, Parapetio delia muraglia; Quel pezzo di muro, che avanza alle

 
 
   
 

Ss SSREREE BS TELA SSA TAs SAT. SS ae

 QVINTO CANTARE: 2gt

    
   
  
   
   
  

z del terrapieno, ¢ si dice /ponda quel muretto, 9 spalice-
ilterreno, a i pozzi, a' fiumt, ec.

f di favure', Circondati di cispa per la similitudine, che ha con la cispa
co; E/avore ¢ uno intingolo fatto di noci, ¢ pane pefto, ¢ liquefatto
+ sec eSnpemeef quell' umor craffo, che si conden(a intorno alle pal-
i sli occhi.

a Eicaaepccinns gli occhi addosso. Subito fifla sopra di lui gli occhi
« Equelto verbo /quadernare s'ula per divolgare, maniichare, ec.

pats Lv 33+
plires,: Cio che per  nniverso si fgnaderna

WN, Celebre Luogotenente di Birri così chiamato per soprannome ¢
ILANCARE. Aprir quanto si pud una porta, un' armario, ¢ simili: le-
aca, cioé il palo, che tienc in alcune porte fermato tutta, o unay
slia porta; aprire affatto. Vedi sotto C. 6. stan. 43.
ZILOTTO, #' uno stringimento, che si fa in qualche parte del corpo,
sliando la pelle col dito indice, ¢ stringendola'co) dito pollice; ¢ così facevas

intorno al collo, a/ternando i pizeicorti, cioè facendoli hor con |' una, hor

/mano per pigliare i pidocchi, che sono queghi iznumerabili mortali, che

ue loro gli hanno fatts i ditali, cioè ricoperte le dita; Che ditale inten-
diamo parte del guanto, che cuopre il dito.
— STANZA XXXIX. STANZA XXXX,

  

Non tanto s* abburatta per la rogna, Bu bis, bu by comincia, ch' il buon ciorno
(Epe brnfeol, che vanno alla goletta, Vorrebbe dar al Cavalier, ch' ei tiene

n dir non pud quel che bifogna Ii Corrier, mediante il fuon del Corno y

line feiligua ache abacchetta, Del popol d'Israel chor va, hor viene;

Qual ib quartuccio le bruciate fogna, Van le parole a balzi, ¢ per ifforno
Nefenza quattro scofe altrui le getta, Prima cal fegno voglian colpir bene;
Talfi dibarte,¢ a vite fa la gola Pur pinfe tanto, che gli venne detto;

volta ch! ei manda fuor parela, Buon di Corrier che nuovac'é diGhetito;

(ctive il Poeta la guardia, la quale havendo creduto che Calagrillo fuffes
un' ', lo faluta come tale.

S'2BBVRATT A. Si dimena: Si dibatte. Abburattare propriamente vuol
dire Separare la farina dalla cru(ca con lo stacciv.
BRYSCOLI che vanno alla goletta, Intende i pidocchi, che vanno alla pola.;
Goletta intendiamo Veftremita dell' abito da Huomo intorno a!la gola. Ed il Poe-
ta copre a detto con l'equivoco di Goerta, fortezza in Barberia, e con las
Voce bru/coli, che sono minutissime particelle di legno, o paglia, o simili, ed egli
TART AGLLARE, Intoppare nel profferir le parole; pronunziar con difficul-
i: ¢ /eilingware vuol dir Balbettace..:
4 BACCHEITA, Ci dare a bacch vuol dire C dare affolut
Mente ¢ dispoticamente in ogni congiuntura, come Re, o Capitano, che porti
scettro, mazza, 0 Raftone a wf 3 ¢ di qui defi, che coftui eile "
¢ (cilinguava ogni lettera.

LFARTYCCIO, Milura Fiorentina capace della arg aacaner partes

D flaio, ¢ per lo più ¢ un vafo di legno, BRP.

 

 
 

5s eee

 

   
  
 
 
    
   
   
   
   
     
     
  
  
  
  
 
    

242 MALMANTILE) |

BRVCIATE, Marroni cotti arrofto in padella 5-0 in forn, 0 forto la
FOGNARE, Fogna vuol dire quel vacuo fatto ad arte |:
paiia l'acqua, ¢ si conduce feolando al fiume dal Lat, fovea:
mifwra vuol dir wetter la roba nella mifura in maniera, che
ma dentro vi fieno molti vacui, come facilmente fegae nel ia,
quale non si posiono flivare i marroni, i quali per cher di figura rotonda non,
ricmpiono lo spazio, ma fanno nacuraimeate, che rimangano fra I'und, ¢!'al-
tro molti vacui nella mifura; la quale poi, volendoli votare, io fquo-
tere; perché s' aftrontano nell' ulcire, € foqquadrano alla bocca. del quartuccio
in maniera, che non potriano (cappar fuort, se non Gi fquoteffe il vafo,ed uscen.
do, fanno un romore simile a uno che tartagli, le di cui parole pare, che non,
potiano uscir di bocca, se egit non si (quote, dibatte, o Sores equ aae
Jo che egii mette fra uaa parola, ¢ l'altra lo figura il vacuo che fla fra un
rone,¢l altro. E quctto intende col dire qual il quartuccio le bruciate fogna', ciok
fogaa le parole con interuallo di tempo, ¢ non di luogo.. Daa
EAR la gola a vite. Storcer la gola. Vedi lopra C, 2, stan. 9. atlas
PER frorno, Si dice quel ritornare indietro, che fa la palla che ha
nella parte opposta dove ¢ fata rata o sia muro, © sia altro, ed termine,
prio del ginoco delle pallottole, ¢ s' intende quand! yno tira per accoftarfial se
gno per via di detto florno, ¢ non direttamente: E così indirettamente
di bocca a coftui le parole. Ln somma vuol dire, che egli impuncava nel parla-
re, tartagliava, ¢ parlava a falti. valighelitya
GHETTO. Così chiamiamo il Serraglio, nel quale stanno in Firenze, ed in
altre Città gli Ebrei: E perché questi hanno nome di tener di mano afregheric,
pero dice che il Corriere di quel luogo è solito Ipetio andare a Malmantile a to-
var la flregha Martinazza.. Ghetto ¢ voce Caldea, che significa libello di
dia; onde noi diciamo Gherro per intender luogo di gente fegregata, ~~.
dal commercio degli altri huomini. Gli Ebrei quando vogliono dire loro
mogli, che le gaftigheranao col repudiarle dicono.; Ti manderé al Gher. a
STANZA XXXXL STANZA XXXKIL
Rispose t altro, tal parola udita + 41a che vo il tempo qui buttande vid
D' elfer corriere gid negar. non possey In disputar con matti., econ buffoni
Perch' io Pho corsa afar questa falitay 4 trattar teco credomi che sia ~~
Ata quato al Ghetto ia ndlavoglioaddosso; Come a' Birri contar le sue ragioni;

  
  
    
   

     
  
     
      

Non ho che far con-gente Ifraelita.; We diffi mal, perch hai fifonomia
Benti fara il mio brando ilcappel roffay D' un di color, che cinffan pe' caltonk y
Ecol darti [ul vifo un soprammano El' cffer tu coftt, par chvella quadriy

D Ebreo fara mutarts in Siciliano Ch' i Birri fempre van dove fon ladri.
STANZA XXXXIIL Joa
Dell! alma fala quei:fon soddisfatti'y
Aa, v0i col corpa la portate wa,
Hor baftasfe bs voi tant' odio corres
disporre.

  

Bench? voi fiste come cani,egatti,
Ch'effi non han com, voi gran fimpatia y
Perché peggia de' dtavol fete fatti,
Vande nel pighar pis tiraunis. 5 Mezlio,a-i.lor danni ti:posras
  

i eh he ke le SE had

}

 

eae tn i

Sessa BPres &
>

243
STANZA XXXXV.
La frefto-devi oprar 5 &° a tei sia farto;

et cui(perch ei confente in tal baratro
ag porrebbe far le fufa torte;
@i si cerca efstr mandato Hn tratto
Salt' afin con due rocche dalla Corte,
Si che, [tu nol fai, ts rapprefento,
C” un difordine qui ne puo far cento,

XXXVI.
t Mentre pero Cupido non rimetta:
non impiccate questa Troia, Ma se la rende non vi do pik nota,
vK Pigharmi questa derta Va ditone,enarra a lei quanto tho detto,
i Birro,e in fulle forche il Boia, Chri gui rattendo,e la risposta aspetto,

Sadira Calagrillo, che colui I'habbia prefo in cambio del Corriere degli Ebrei,
¢lo minaccia di rompergli ia velta, ¢ sfrepiarlo; ¢ dopo havergli detto molt im.

Oper) » gli ordina, che da sua parte avvisi Martinazza, che renda Cupido; al-
titi fara render per forza.

LHO corsa.. Ho fatca questa cosa senza considerazione. Quand? altri fa qual-
che risoluzione, che non riesce poi buona, di¢iamo: E# l'ha corsa dall' armeg~
gist, e'dalcorrere la gioflra. Similmente diciamo; Fare una carriera. Qui fa
giuoco la voce corsa, che ¢ cosa da Corrieri. si
NON la veglio aitdoffo. Non la voglio fopportare, Si dice anche non /a veglio in
sul giubbone.

Gente Israelita, Intende Ebrei: Popolo d'Israel.

IL cappello rosso. Gli Ebrei in Firenze portano per contrassegno il Cappello
rosso. Il Poeta dice, farò ben' io diventare Ebreo te col farti il cappello rosso col
sangue. E poi d'Ebreo ti farò diventar Siciliano tagliandoti il viso, ed intende
quel Siciliano Montambanco, che per accreditare il suo Olio da Ferite si faceva
agg persona, e con esso se le medicava.

AUMANO. Quel colpo, che si da con spada, o baftone, comincian-
do da alto, ¢ calando a batlo., Vedi forto C, ro. stan. 52,:

SVEPONE » Vino che'fa profeffione di trattener la brigata'con facezie.
DIR Me sue ragioni a Biri, Raccomandarfia chi non pwd', € non vuol far fer-
vizio'; anzitha caro'il tuo male. Vuol' anche dire discorref con und, che nons
a 'ta dica;o vero buttar le parole al vento, Plauto disse nel Pfeudolo;
spud novercam queri, '. t
CHEFAN pecalzont, Ciov i Bitri; i quali pigliano pe' calzoni. [1 verbo
cisfare hha del furbe(co; €*vuol dir Pigliar con prefa Mabile, ¢ buona, come &
feleste ate, pigliando uno per il ctuffo, cioè pe''capelli. Petrarca', Le san
'4vef? jo avoolte entro a' capegli, fn
“ESSER come vani,egatts, Eller poco d' accordo, 0 poco uniti, anzi fempres'
fhimi¢i,come naturalmente fonoi cani, ¢ i gatti. 2
NON ha gran fimpatia. La voce fimpathia Greca fatta Toscana significa incli:
hazione scambievole, o similitudine di - » di voleri, ¢ d' aftetti,
. amity 5: 5:

eu AE.

Mentr'a tofitinonrendailfnoCoforte.

  
 
   
   
   
   
   
 
     
  

 
 
   
 

 
 

  
  
 
   
  
   
     

ve
244 MALMANTILE | —

MAESTRO Baftiano, Intende il Boia, che allora cos
cra stato Mieftro Biagino. Vedi forto C, 6. tan. 56.
LETTO a tre colonne, Cioé le forche,le quali veramente fon
una flanga sopra a traverso, ed in molti luoghi sono, ey:
LAVORAR di mano, Vuol dir rubare. scherza dicendo, «
( cio€ il Boia ) perché essi ricevano qualche riposo da tanto lavo!
gil mette in ful letto.a tre colonne pee in (ulle forche ) ed in fuftan t
gl' impicca, perché fon ladri, E Calagrillo, seguitando l'equivoco del ri
dice aila guardia, che f¢ ella ha punto di pieta,¢ discrezione, dovrebbe
sto riposo in ful letto di tre colonne a Martinazza per il suo tanto
impiccarla, perché¢ ladra, I Latini pure per dir copertameate
manu finifira uti secondo Catullo in Afinium. ' sey
Marrucine Afini, manu finiftra
Non belle uteris in ioco y atque vino;
Tollis lintea negligentiorum.
E per dire eopertamente Impiccar'uno,dicevano; diteram longam.
bsamo notato aitrove.
NON cede un grano: Non cede punto. Che grano si pud dire ana
inconfiderabile del pelo, poiché 24. grani fanno un danaro, 24.
V oncia,e 12. once fanno la libbra.
NON uccella a pispole. Non si cura di confeguir cose di poco mon
è fra gli uccelii la pispola. 1 Latini dissero vom capeat mujcas.
FAR le fufa sorte, Far le corna. Vuol dir quand' una donna si Of
altri huomini, che col suo marito. I Burchiello Poeta capricciofo, VAs
sotto nome d' Accademico Fiorentino incerto, nella Raccolta delle Rime Piace-
voli del Berni, Casa, ec, i.
Non ti fidar di femmina, ch' è usa si '
e4 far le fufa torte al suo.marito,
Il Berni nel suo primo capitolo dell' orto dice:
E finalmente non fara mai fufa
Donna alcuna per lui torte al marito; Ky
Si dice fu/a torte per intender copertamente Corna.;
MANDATO con due rocche in full' afino. EB coftume in Firenze, al
dclitio del pigliar più d' una moglic,aggiugnere una dimoftrazione
che ¢ il far' andar' per la Città il delinquente legato sopra ad un' afino',
mitra di foglio in capo y ed a-cintola due, 0 pill rocche inconocchiate, che
ficano le due, o pili mogli. ses eeh
, QUEST A troia, Quelta porca. Epitcto vituperofifiimo nelle donnes
ynoldire Laida meretrice:: nell' huomo non € tanto ingiuriofo i). dirgli
Ml x0 pighar questa detta, Vud pigliarmi V'aflunto di far questa
derra vuol dire prometter per un' aitro, o star mallevadore, cioè di far'
cosa, fenon la fara quello, che  priacipalmente obbligato. Comprar.wna 4
yuo] dir comprar un' avviamento, un credito, ec, Derta & dai plura
Devitt»

 
      
    
      
    
  
   
   
   
  
   
 
      
    
    

fucere

 

\

   
  
 

  
 
 
 
  
 
 
 
    
  
   
   
 
   
 

245
STANZA XXXXVIIL
Lascia la sentinella, e caracolla
Gin pel castello, dando queste nuova
E benche il Adaggioringo della bulla
Gi habbi eff,r ech'ei si mova
: 'Di fargii porre a' piedi la cipolia,
r non possa dele pacche; Cercando della morte in bella prova
to havendo si Ciel turbato Vuol avvisar di cio Mona Cofofiola,
¢i par un porcellin grattato, Ch' è per bafire a questa battifoffola,
» che ¢un vero poltrone, fentendo le bravate ai Calagrillo, zitto
/¢ tremando va a dare questa nuova a Martinazza.
RDA l'armi dalle tacche.. Non vuol cavar fuori la spada, per non la,
:. Intendi che coftui era un codardo, perché per dir copertamente pol-
un soldato, se gli dice: Rispiarma foderi.
i Facche. Parole latine corrotte', ¢ ridotte in una, usaro assai dalla
- plebe nte per intendere Andare in faluo,ed è il Latino 4d asylum confugere,
Rl LEV-AR delle pacche.. Bulcare, 0 toccar delle ferite, che questo intendiamo
hey ma ¢detto plebeo. Li Vocabolifia Bolognefe dice che pane significa per-
agliarda.. La forza di questo verbo rilevare vedemmo sopra C. 3. stan. 67.

hi stor, Fiorent. lib.6, dice 4/ figlixole del quale nominato Lorenzo,rilevo unas
3

   
   
    
    
      
  

IDO veduto il Ciel turbato, Havendo conosciuto, che coftui era in col-
lice' anche /4 marina turba,
iB che pare un porcellin grattato. Similitudine assai usata per intender uno,
| nrispenda alle grida d' un' altro o per paura, 0 per riverenza, o per lao
coscienga macchiata »© per altro; ¢ si fa la comparazione al porco, perché il
Porto che firide,grattandolo si quieta, ed i porcai gli rendono maneggiabili col
Peg
OLLA, Il verbo caracolfare vuol propriamente dire Volteggiare col
na non ofante qui torna aflai bene per esprimere, che coftui per las
laffle girando per il caflello, non gli parendo trovare luogo ficuro. B
anche in uso caracol/are per camminare a piede,volteggiando d' una strada in
altta,¢ diciamo far un caracel(o per intendere una girata. Viene dalla voce
[ Spagnuola caracol » che ynol dire chiocciola, '
ene deta bolla, Termine della lingua furbesca 5 che in Firenze vuol
il Fiscale 3 ma s' intende per il Superiore in quegli affari di che si tratta, Va-
> il Maggiore delia Città, chiamata in quella lingua Bolla dal Greco Polis es
arbaricamente, Polla,
| FARGLI mettere a' piedi (a cipolla,, Fargli troncar la tela, ¢ mettergliela a
i: come si costuma in Firenze quando, il cadavero del giuftiziato dee Mares
elposto per qualche ora al pubblico; che gli mettono la testa.a i piedi.

PER bafire. Bi per tranfire 5 per (ucnirGi 5 per morirfi. Vedi sopra Cant, 2,
NA Cofofiola, Nome usato pet intender una donna faccendiera, aftanno-
Ofudatora. Scbbene Ce/osso/a [ fecondo ul Varchi nei suo dircojano all. voce

4: Batti.

    
    
   
   
  

|

 
         
  
 

Rartifoffiola 1810 Reflo che batrifofiola, ¢ significand affanno >)
mento grande, ma breve, che cagioni battimento di cuore, 0
il che si dice fofhare: Franco Sacc, Nov. 44. Ad' bai data eos),
io non [ard mai più litro', @ forse me ne morro. Non credo che fit
guello che dictamo fepraffalro at exore; lo stesso che batticudre y affanno
to per paura,o dolore improvvifo dagli Spagauoli detto,/obre/

| 246 MALMANTILE™
i
Corn, Tacito lib. 5. dice: Exterrite unt acri magis qudm dinenrno rimore, Bail n0-

i stro Davanzati parafrafando queste parole dice bebbero batrifoffia, woth

| STANZA IL. STANZ Ay Be

I Ella insieme le schiere ha già ridotte Atentre del farto poi le da context' ake

! Di genti, che non vaglionoun piftacchio, Com quell'ambasciaye ling ua di frullone
Cive di quelle, a cus fece la notte Fa ( perché nulia mai fora Se

} Col [uo carve si grande spauracchio, Chi lo fente morir di paffione;
Ed hor quivi parare, e dar le botre Ma quellasc'a fenrirlo> forse avvertAy
Infegna lor, che non ne fan biracchio Li intende un porcost i/ is è
Ma quand innanRi a lei coftni, ififerma Equi fnifoom le legion di wera y |
Cos} tremante,la cave di scherma, Perch'ella ni'da-pin:ne inCielneinverra,

Martinazza stava appunto instruendo quej soldati, che s evan -faggiti per pa
ra de' suoi Caproni, quando arrivd un la sentinella con Panbotetara a Gale
grillo, che la tarbd cutta', ond' ella la(cid flare il darlezione, si
NON vagliono un piffacchio. Non fon buoni a nulla. Si dice un piftacchio yun
lupino, una lisca; una forba, una lappola', un pelo yun baiocco, wa baa,
un picciolo, un zero, nn' ctte, un fico, cica.y un iota, una chiarabaldana, ui
puntal di stringha',o'd' aghetto, una fucciola, un soldo:,.an quaterino, un-cor-
no; tutti per e(primer'la poca flima;che si facia d' uno, 0 d' alcuna cola, Eft
dice anche: non lo stimo 1) cavolo a merenda, Latino cicwm, titiviliitium ~~
SPAVRACC HIV, Significa quel che accennammo sopra C. pr. stan, 40, Bab
li si dice fare /pauracchio a uno per intendere spaventar uno, o mettergli pauras
con fatti, 0 con parole.,
NOW ne fan biracchio, Nomnefanno nulla. Si dice anche straccio, brand, 0
brandello, e simili, id
! CAV-ARE un di (cherma, Vuol dire far perder il filo del discorso a uno, ed &
| Jo stesso che cavar di'tema.. Ma qui: vuol dir' anche far lasciare far di-
re, ¢ torna bene, perché Martinazza.la(cio la scherma', ed usci di tema, ¢'
proposito per l'ira, che'le cagiono  ambatciata fattale in'nome di Calagrillo.
eAMBASCIA, Attanno, o difficile re(pirazione d' alito, Fran. Sace, N.139
T ofto colui di-chi erano frati., (en' ando con l' ambascia della morte a ripigliarl. oh
LINGY-A di Frulione, Cioè che parla a salti, 0 a intoppi'y comeeil rumore}
che fa il frullone, che ¢ quell' ordingo, col quale; per via d' una»ruota:
fepara la farina dalla crusca dsmaxy
NON raccaperza nulla, Non intende nulla. Vedi sotto C, 6, flan rot.”
LP INTENDE per discrezione, Quando per' altro ci & noto un-negozio, ¢' che
taluno ce Jo racconti confufamente, o lo scriva con cattivi, ¢ non intelligibili'ea-
ratteri, sentito,o letto da noi, fogliamo dire; 2 habbiamo inze/o per diferexionts
cio¢ habbiamo havuto la discrezione di non gli far ripecere il discorso cacy

=

a

 

streeceée hag feet

we ei

   

2A# FEZ

 
   
     
   
 
     
     
   
   
    
 
    
   

se”

&

= Sas * BS

id

A= 2% BE

SF

 -

* mente dalla vergogna, la quale però si dice anche erubescenza.

— PE He.

  
 

247
quel fat-

 
 

er qualche: informazione, che havevamo di
orfo, © scritto:. i
,we im terra; E? fuori di se, Non fa quel che ella si faccia.
piel 5 we terra; dissero anche 1 Greci in questo proposito; ¢ l'ula Lu-
Plendamante, ovogliam dire F aifo indovino,
ANZA LL STANZA LIL
wedefi cambiare Rabbiofa, it capo versa il crel tentenna,
Quafi col piede il pavimento sfonda,
tor figratrale chsappe,jvor la corenna,
Hor dice al meffaggiero che risponda,
Hor larichiama métr'egli in Chiarcna,
Grida,¢ minaccia,e par che ficunfond,
Hille difegni tro al pensier racchinde
Lenne inne ye nulla mai conchiude,
LULL.
Che lavandole il collo lordo,e intrifa
Laghi formano in fen di poxxi neri;
el fin tornara in se,con la gonnella
Yi come fonagli da (parnicri, S' ascinga,e al meffaggier così favella,
Narra gli accidenti, ed i moti diversi cagionat: in Martinazza dall' ambascia-
tadi | ed in fine Martinazza s' accinge a dar la risposta. L' Autores

 
 

 
 

  

deferive.. per-una folenne sgualdrina poiché dice, che & così grande il
'udigitme che-clia ha-addotio., che le Jagrime che le cascano dagli occhi fanno
)arerle nel collo tanti laghi di pozzi neri, cioé di cedi, i quali ella s' alciuga

  
 
  
 
 
 

 
 
  

BIANCA come il mia collare, Diventa bianca comie un panno curato, E que-
te mutazioni di colore fon proprie d' uo che habbia l'animo alterato si in ma-
€, comein bene, perché la palidezza, ¢ sbiancamento deaota follevamento d'a-
dimo-non essendo altro-, che un mancamento di (angue, il quale per la paura se
ne fugge al cuore, ¢ lascia le vene del voito; ed il roflo denota ira percht questa

tibollimento di fangue intorno al cuore y che scorre per tucce le venc,
Ma apparisce pik nella faccia, perché quivi (ono molte vene intercucance, 0 vo-
gliamo dire im pelle,.che faciimente lo scuoprono; ¢€ lo: stessoeffetto viene pari-

   
  
 
 
 
   
   
  

DOPO ch cgit ha toccata una spogliaxza. Dopo che egli è stato fiuftato in ful
©...,dal maeftro. Spogliazza quali expoliatio, spogliagione si dice quando il
Macftro fa cavare i calzoni a uno (colare, ¢ mettendolo sopr' alle palle d' un'
altro, gli dd con la sferza in fulc..... E quando gli da nella (tela forma, ma,
senza i mandar gill i calzoni si dice dare una mula, o un cavallo. A questo
&,.,. fruftato affomiglia |'Autore il vifo di Martinazza quando le diventa rotio,
Vna simile spogliazza, quafi come a ragazzo infolente, © minacciata la nel se-
dell' Liiade a quel brutto moftaccio di Terfite, a cui Omero [ fecondo la,
one Latina ad verbum del Gifanio } fa dire da Vlisse: Ne posthac Viyfi
apne bumeris adfit, Gc, Si non ezo te comprehenfum, & charts veflibus exutum Paltio-
que, © tunica, que pudenda contegunt., Flenvem veloces ad naves dimifero, Cadens ¢
Concioue duris verberibus. TEN-

     
 
     
     

  

 
 

 
 

f

eae

EE ————— ==

—ooeee

  

 
 

248 MALMANTILE! 79

TENTENNA il capo innerfo il Cielo, Dimena la testa verlo il
§ fa da molti quando accade Joro cosa di-poco gusto, quafi voglia
il Cielo perché cagiona loro quella tal di(grazia: i Latiabdissero; caput 9
SEONDA il pavimento col piede. Per la collora batte i piedi in terra
mente, che fa quafi rovinare il Palco. Properzio. Et «repitwm dubi
ede. tf
ST gratta le chiappe, e la cotenna, Si gratta le natiche, il capo chee
to solito farsi per lo pili dalle d6ne quando fuccede loro qualche disgrazia,Pei
vas' intende il capo, perché la pelle del capo-dell' huomo si dice cotenna;
vuol dire 1a pelle del porco, ed impropriamente si dice la pelle d' ini
vedi sopra C. 2. stan. 64. ed in cid noi ci conformiamo co' Latini, che cuit
ta pelle del capo dell' huomo, ¢ dicono anche cutem detrabere per scorticare qual-
fivoglia pelle, il proprio vocabolo della quale è pellis. amd Ae
QLVAND?' egit ¢ in chrarenna, Quand' egii ¢ molto lontano.. Zp oras “
¢ da quetto noi diciamo: Quand' ezli ¢ in orinci. Viato dal Davanzati nel.
J ENNE inne, Di questo termine ci serviamo per esprimere uno che L
di operare,e non conchiuda. Viene da quello stento che fanno i ragazzi r
imparano a compitare; quafi dica compita compita, e mai rileva., ed halo fle
fo significato, ¢ forza che ponza ponza detto sopra C, 4. stan. 80, sea
SON*GLI da sparuieri, Intende lagrime grofie come sono i fonagli, che s'ap-
piccano a i picdi degli sparuieri;comparazione ipcrbolica,ma aflai 'inten
der grofle lagrime.Ain.1 1.41 lacrymans.guttisghumettat gradibus acon
nels chiamiamo quelle gallozzole, che fa l'acqua quando pioye, cadendo sopra»
i rigagnogli; o altrimenti neilo scorrere., '
2022/1 NER/. Bottini. Quei luoghi forterranei, entro a' quali Anraes
forta d' jmmondizia; ma propriamente pozzo wero è bottino, 0 fogna
del ceflo, a differenza di quella degli acquai, i
STANZA LIV, STANZA LV. >
Torna,¢ rispondi a queffo Scaizagatto y Pero s in questo mentre umor non varia y
Che si crede ingoiar con le parole: Domani al far del di facciami mottos

  
    
 
 
 

   
  
 
   
   
 

    

      
   

  
   
  
 
  
    
     
   

   

  
   

       
 
 
 
 
  
  

Ch'io no fo quel ch'ei dica,e s'eglié matto E s'io gli faro dar le gambe all! aria
Won ci posso far' altro ye mene duole, Quella sua landraba da pagar lofeette,
Poi circa alla domanda, ch' egti ha fatto; Mia se la forte fofse a me contraria
Che gli daro Cuptdo,e cid ch' ¢ vnole, Viol 'a me tocchia adar col capo roti,
Se con la spada in mano,o ver co laa #renda Cupido allor, ch io le prometto
Prima di guadagnario,il cor gli basta, Lasciarglielo fegnato, e benedetto, —
STANZA LVL
Cid detto partese quei chieralbnomo e/perto Ed in vifo vedendolo scoperta 5
( Essendo lato Cavailaro,e Atefo ) Ruch' ha bifegno dice d'un buon lett y
el Cavaliere ad unguem fa il referto Perch'egli è duro,e non punto pupilles
Di quel che Martinazzagh ha comeff[o, Lo conosco bensì, gli¢ Calagrillas —

Martinazza manda a dire a Calagrillo, che gli dara Cupido, s'¢i lo gl
gnera con! armi; ma se ella vince, vuol Psiche: la ronda porta l'ambaleiata,®
riconosce Calagrillo

SCALZ 4GATTO, Huomo vile, Guidone.
 

 
  
 
  
 
 
  
 
  
   
  
  
 
 
  
 
   
  
 
   
   

TO CANTARE. $40
ral eeenesircl neha con Ie chiacchiere'. E si dice:

teu varia Se fra tanto non si muta d' opinione.
ina, donna di bordello, ed intende Psiche; Landra è epi-
pinfami, ¢ Jaide meretrici, quali datrina, che la fogna,

ro. Hada pagare la pena. Pagar lo scotto vuol dire pagar
sé mangiato, pagar la sua porzione, la sua quota; Tercnzio
+ Ma qui intende il Latino penas mere, Dan, Purg. C. 30.
» L) alto fato di Dio farebbe rotto
Se Lece si paffaffe, e tal vivanda
Fuffe gustara fenx' alcuno scotta
pentimento, che lagrime panda,
è « Andar con la peggio; cioé ch' io perdeffi il duello;
ATO,e Back. Liberamente,¢ senz'eccezione alcuna.Fran.Sacc. Nou.
e ogni hora pur Segnato, ¢ benedetto, Esprime ua dar via qualcofa, o
uno volenticri,¢ con anime di non rivolerio; Vn licenziare af-
Peecceit vale,ingust Iola,

. BE' un famiglio, che porta le citazioni criminali mandate da
s Trent Cavallaro, pecché flante il largo dominio, ¢ giurisdi-
il suo tribunale, ¢ necetiario che vada a cavallo; 4 Ade/s0 € quello
eitazioni i pure de i Miniftri forenfi, ¢ fa i gravamenti, ec. es
lo, perché non gli occorrono lunghe gite, come al Cavallaro; a
panda Cxrfore; nome simile al Viator,col quale era dilegnato dagli an-
iil donzello,o fante pubblico.
hem» Per appunto. Frafe latina usata assai da noi.
“CARA cater. Riferilce. Frafe curiale, che vuol dire quando il Cavallaro, o

i data la citazione, riferisce in atti d' haverla data, che dicono an-
rapporto. Ev Autore si serve ee frale ( per altro non usata in,
i) perché ha detto, che questa Guardia era stato Cavallaro,e Meffo,
tbifogno d'un buon lefso, E' carne dura, e pero ha bifogno di bollires
7 Sore paaglonds EB detto vulgato per esprimere un' huomo, che sa il conto suo,
ye difficile a superarsi, che diciamo: O/so date per efempio; Il
“cies @ rodere un' offo duro.
+ Non ha bifogno di Tutori, fiona lo stesso che ha bifagno a un
brian è pupille si rifttinge a faper tare i fatti suoi, ed ha bi/orno
oe e(prime faper fare i fatti suoi, ed efler bravo,e valcate in ogni

    
  
  
   
  
   
   
  

'=
:

STANZA LVIL
wi tie dame Calagrille vefti, Che seguitaron come voi inten defi
giorno rivedremg li poi. Periun, che (en' ando pe' fatti suoi,
ekeongeon apprefti Che trovereme ti, se venir valere

Per ginger it 7 Fendefi gti altri duoi, Pin presto afsai di quel che vi credete,

oH STAN.

ee <

Digitized

 

  
 

 

 
 
 
 
   
    
    
 
 
  
  
 
 
        
   
   
 

”

250 MALMANTILE.
STANZA LVIIL
Che zio cio se ne vanno git pel piano
Shattuti com? io diffi dala fame;
114 non fon iti ancoraun trar di mano A 'per soddisfar s.
Che fenton raxzolar fra certo firame;  ——-Hla fatto in quattro di Pil

  

    

Percio con b armi subito alla mano El: con la sua spada se

Corron dicendo: Qui c'e del beftiame, Delt' honor della quale

Si che quando crediamo ditirar minze, Che havendola fancinila

Ji cor pu forse caverem di grinze, Non gli par ben ch' ienuda

STANZA LIX. STANZA LX

Curtofi quei che sue di vedere Ata perché un huom pix vsl mas,

Dentr' a una fialla inabitata entraro y Si pente esser'entrato in talc

E vedder,ch'e: a un'huom postoagiacere Pero che a fearui folo egti ha,

Sopr' alia paglia a guisa di fomaro; Che non lo porti via la Trentan

Accanto havea da mangiare,¢ bere y E perché tutto il giorno quant'es

E gli occhi distiliava in pranto amaro, Egli ha il mal della lupa, che

E trai disgufti,e il vin ch' era quifite  Non va mai fuor #4 cinta

        
     
 
 
 

Pareva in vifo un gambero arrospico, DL? ascioluar col suo fiasco nell
STANZA LXIL

Ovungne elie, d'untumi fa un bagordo, Aggira il beccafico,¢ pela il
Ch' ognor la gola gli fa lappe lappe; E a poveri cappon ruba le
Strega le botti di lor fangue ingordo, E prega il Cielche fac

 
    
    
  

E le fuftanze usurpa delle pappe; Quant le melagrane,
L' Autore torna a parlare di Perlone, ¢ degli altri, che Jaicio sopra si
28., i quali per la fame s' andavano ailontanando dal Campo,e¢ narra 5
floro trovarono in una Capanna quel Piaccianteo, che fu da Bertinellan
fuori a spiare, come vedemmo sopra C, 3. stan. 45. il quale haveva feco
giare,e da bere. Nella presente Ottava 62.de(crive aflai vagamente la)
nia di Piaccianteo, 3 ere
G/0' cio. Adagio adagio. B' la figura aphere/is. nf ist
RAZZOLARE, Fregare, raspare, fragare; ec. Qui vuol dir quel romor
che fa la paglia, 0 cosa simile, quando ¢ maneggiata in mafia. ae
STKAME, Paglia, fino, 0 aitra materia simile per cibo delle beftie +
sopra C, 4. stan. 2. a
TIRAR minxe. Vuol dite stentare. Ma s* intende moriré: Si dice milzi
il Poeta si serve della licenza, ¢ seguita intanto i pil che dicono; minga
milzas
C AVARE il corpo di grinze. Mangiare aflai, che in questa maniera gonb
il ventre s si levano le grina¢ al corpo. Plauto disse ventrem diffendere
Georg: disténdunt netare cellas, cioè empiono « Paes
PAREV-A un gambero arroftito, Bra rosso in vifo come sono i
ae aflai usata per esprimere un roffo in vilo, per il
evuto,
ALA fatto Sillidé mia, Ha finito, ha confumato, o mandato male tutto |
havere. E' detto fanadatsico Filide per fine, Ma per avventura hala fa 0

   
   
  
 
   
   
      
  
 
 
  

    

bands 2
Lia

© eae anges
 

 
    
    

 -QVINTO CANTARE, 251

figliuola di Licuego Re de i Traci, 1a quale s! innamoré di Demo-

di Tefeo, ¢ di Fedra, quando nel tornare dalla guerra di Pera
) stato spinto da i-venti contrarj nel Regno di Tracia, fu da Sillide rice.
on fegni di grande amorevolezza; ma egli senza riguardo a i benefizzi das
efla ricevuti, fen' andd; per lo che Sillide disperata s' impicco. Da questa dispe-
'rata morte di Sillide, quando diciamo far Fidiide, intendiamo finir la vita, ¢ fini-
re!

   
  
   
   

. - ¥
 IMPLATT ATO, Nacoto', Vedi sopra C. 2-fan. 60.
DELL! honor detla quale ha gelofia. Ha gelofia dell' honor della sua spada, per:
he havendola tenuta fempre fanciulla, cioè vergine ( che s' intende now mai
perata ) stima poco honeflo il lasciarla vedere ignuda, come è veramente po-
@una vergine Jasciarsi vedere ignuda. E con tali scherzi vuol dire, che
codardo, ¢ vile,¢ di poco animo, ed uno di coloro che wmbram fam

'ANC ANNA, Vna beltia ch' ingoia o tracanna trenta per volta;
/è una di queile Jarue immaginarie inventate dalle Balie per far paura a i bam-
'a come bau, befana, ¢ simili dette al trove.

 dL male della Lupa, £ inteso da noi per una infermita, che fa stare il pazien-
te in continua fame, ed i Medici-la chiamano fame canina. F
¢ | CHElofeanna. BE' un termine che significa grandezza di paflione, ed ha forza
1 davanzare jl superlativo, perché dicendofi, Ha ana fame, una fete, un desiderio,

tc. che le feanna, s' intende fame, sete » 0 desiderio grandissimo, ¢ più, Vedi fo-

ei = praC.4. stan.2z4.
ASCIOLVERE. Solucre il digiuno; sdigiunarsi, fare colazione. Vedi sopra
“ stan. 35. ma qui è prefo per mangiamento in generale, cioé per la materia

A

Hi Tad! Intende roba da mangiare, che fiaunta, come polli, carnes,

yg
a ec,
' 2 ME -

. ' 'BAGORDO. Bagordare, o far bagordo vuol dir Gioftrare, giuocar d'armi,

+ far conviti, ed ogni altra forta d' adunanza feftiva, ancorch¢ non d'armi. E
o potrebbe dirfi (cherzando bagordo, quafi vagus ordo, confufione ordinata; onde
a di gente in confulo » la quale interuiene a tali bagordi, Pigliamo
— ~pol do per commiftione di varie cose, come nel presente luogo, che intende
 mescolanza d' untumi. Vedi sotto C. 6. stan. 2. Del refto Bagordo viene da Zi-
tite vuol dire eda. E Bigordare trovafi prefio gli antichi; per corger la

 

» Fazio degli Vberti nel Dittamondo al Canto 32,
; Giovani bigordare alli chintani,
w hig E gran tornti,e una,e altre Gioftra
Bete 'i Farsi veder con giuochi nuovi,e (rani,
"Poi si disse Bagordo,¢ Bagordare; ¢ si trafiero quelte voci a Ggnificare ogni forta
# 'di stravizio, ¢ di ricreazione. Che Bigordo voglia dice eda, ciel efempio di
Giovanni Villani lib. - rubric. 132. £ recoffi patio di drappu ad oro fupra capo
i Helfer Amerigo di Nerbona portato sopra bigerdi per pix Cavalieri, Eclgo-
è z da San Gimignano Rimatore antico citato dai Conte Vbaldini nelle Annota-
' Meficr Brance(so da Barberi an + Brompere, ¢ ficcar bigards ye lance,
ee LA

  

 
 

 
  

 
      
 
 

23s MALMANTILE) | 5

LA gola gli fa lappe lappe', Sigaitica detiderar ardentem
nate dal cao chet il palato con la linge £0 VES
za havere nulla in bocea, che ¢ fegno di » qual fuono pare.
lappe; donde poi il verbo al/ampare, che vuol dire haver gran fam
in Greco, che è lo stesso, che Lambo in Latino, & fatto dal medesim

ST REG-A (e botti, Stregare vuol dir fucciare il fangue, perch dicono,
Streghe ficciano il fangue a i bambini; ¢ però dicendo frega de borts ini
cia 1 fangue delle bottt 5 che ¢ il vino y del quale ¢ éxgordo, cioè aviditlimo

VSVRP A le (ustanze dele pappe « Divora la carne, che ¢1a foftanza del
del quale si fanno le pappe. 2 nlohetoendt

cAGGIRA il beccafica, e:pela ibtords, Aggirare, € pelare,metafo!
lando, significa ingannar' uno, ¢ cavargli da dosso danari,; come habbia
nato sopra in quetto C. stan. g. Li Poeta scherzando pigha decti due verb
vero fenio, ed intende girar nello spiede i beccafichi, ¢ pelare i tordi p
cergli s ¢ mangiarfegli. '

LiVA ie cappe ai capponi, Ciok divora la pelle de' capponi.

E PREG A il Ciel che faccta, che gli agnelli, ec, Dove git agnelli hanno se
te due granelli, (cio€ tefticoli ) vorrebbe, che ne haveilero' quanti n' hs i
melagrane. E così descrive un folenne ghiotto; e crapulone. Similmente un Cet
to Filofleno folenne mangiatore 5 siccome 'riferitce Ariftotile lib, 3. delle Morali
indirizzate a Nicomaco, cap, 10. desiderava d' avere il collo più lunge d' unas

'i ate

  

 
 
    
  
  
   
   
  
  
     
 
    
   
  
    
  
 
 
 
  
 

grue supponendo, che così fuffe per eflere il gusto maggiore, tae:
STANZA LXIiL comanaal L ake
Vedenda quini comparir repente E quei foggiunge: Adi rallegro,e Gedo ~
L' infolite cae shigortisce il ghiotta, Che ee | facciate bene,e vi fon febiave;
F dal timor ch' egli ha di tanta gente Ma s' il patire ¢ fattoa yoo y
Trema da capo a pic, si piscia sotto: Penitente di voi non ¢ pin bravo,
Con tutto cio digruma allegramente, Tal ch'io per mevi mando a

E spefjo speffa bacia il suo barlorto 5 Non nel fettime Ciel, ma
E accio frremara non gli sia la vita Donde ai midani,ea meche[owoileapi,
Non dice mendegnateo aber gliinuitas Pisciar potrete a voftra polkain capo,
STANZA LXIV. STANZA LXVL
e/a i Cavalier famofi a quel plebeo s Ata perch al certo Voftra Reverenza
Che nou proffer: lor della rovelia, Ch' è frenuata, come-un Carnovale;
Furon per infeguare il Galatea Hanra fatta fin' hor tant' affinentay
Con batrergli gik in terraiimama/cella, Che bafti a soddisfar a ogni gran tally
Chi fei? ( difs' un di loro)e Piaccianteo, flor puo lasciar a noi t
Chie xa pover huorispode,ein quella Cella dccio baciam ta terra r
Molt? anni in aftinenza ha confumati Per piit mondi accofharsi aquest avans
Per penitenza de' suoi gran peccate, Delle retiquie sch' ell' ha gi eek
Piaccianteo vedendo comparir coloro armati, hebb'un lef
non per questo abbandono ii mangiare, anzi si ——— peril
lomandato

   
   
 
 
 
 

  
 

    
 

 haveva, che coloro non gli stremafiero la provvifione.

era,rispote eller uno, che faceva penitenza de'fuoi peccati ia quella cella
caitinenze: Dalla qual risposta accortifi, cheegli era un birbone 5 |

*
 
 
   

  

NTO CANTARE: 253
Bli-dice, che lasci un po fare il medesimo digiuno,

4/Si perde d'animo. Vedi sopra Gastan8.Dan,
Così mi fece sbigortir lo ALefro,

A'S i gli vidi s} turbar La fronte;
9 Golofo; Avido di mangiar del buono. Lat. ¢éuto;
'nol dire haver gran paura. Vedi sopra in questo C. stan. 3.
, Intendi mangiare; se bene digrumare ¢ il matticare, che fan»
più feflo, che si dice anche ruminare dal Latino, che perdchiama
e dette — come habbiamo accennato sopra C. 4. stan. 6.5 ¢ ve-
9 forta.C, 6. fan. 5.
ACIA il barlotto, Bove. Barlottoéun vafo di legno di figura simile al barile,
i: hé fara di tenuta o pils,o meno fino a dieci fiaschi, chetenédo
chi si chiama mezzo barile.Qui pero n6 intende strettamente capt specie
t@, ma un vafo da vino portatile addoflo,comunque si sia o di vetro,o di
una Zucca\, anzi fimo che intenda pil tofto di terra, perché più git
camo la terra del boccale.
- STREMARE, Vale (cemare, fminuire, quafi ridurre allo stremo.
 DEGNATE, Eun modo di dire usato da coloro che mangiano all' ofteria,
intorno alla loro tavola alcun lwro conoscente, ¢ dicono: deguate,
wi di bere, E perché ¢ termine usatiflimo dalla plebe, il Poeta fa.,
¢ si maraviglino, che Piaccianteo non |' usi,e fa prendere argumento,
ped afi per paura, che non sia accettato Vinuito., ¢ scematagli la.

.
CAVALIERS famoft. Cavalieri illuftri, ¢ di fama. Ma qui famofo non deriva
sma allude a fame, ¢ vuol dir Cavalieri aftamati.

We + Vuel dire -huomo di Plebe; ma ce ne (eruiamo anche per intende.

te 'infame,senza honore, ¢ senza creanza. Qui se ne serve per contrap.
lieri famofi, ¢ vuo) dire, che si come quelli erano famofi, cioé af.

bul cra infame, cioè senza fame, perché havea ben mangiato,

, Von, 2} della rovelia, Non offeri nulla; usandofi speffo il verbo proferire,

In vece-del verbo oferire; ¢ la parola della revella & posta a maggior' emfafi per

tiprimere non offeri nulla, ne meno una cosa nociva.,

~, ANSEGNARE il Gatateo, Infegnare |e creanze, i buoni termini, Galateo. è in-

titolata un' Operetta di Monfigaor Gio. della Casa., la quale infegna le buones

 
 
  
 

   
   
 
  
   

  
  

  
 
   

 
  
 

RESleG =

  
  
   

&

 
   
   
   
   

    
    
 
     
   
 

Creanze,

. eae ERGLI. 'gilt una mascella, Dargli un tagliovnel vifo, e fargli cadere una
analcia,

40 vi fon febiavo, Vi son servitore. E' un detto usato, quando alcuno faccia,
la azione, che meriti lode, per efempio Il tale fece una beliima Orazione;

fo gli fon schiavo, I Caporali nella vita di Mecenate:dice 5

H E si legge ch' erugufto un di gli disse:

Gari Capitan Mecenate io vi fon [chiava,.

 NELL' ottave Ciclo. L? Autore tenendo l'opinione, che i Cieli ficno otto dice,

Ha: che

  

        

aS SELES SEEPS ES

BAG

  

Digi

 
 

    
      
   
 
 
      
     
  
   
     

254 MALMANTILE™ >

che coftui merita d' andare nell' ottavo, cioè nel fup p
penitenza, che merita il fourano posto nel Ciclo.
MONDAN!. Intende peceatori. Coloro che fono'd
dani. i
ST ENV-ATO come un Carnovale., Magro, come un Carnov:
ironica, che vuol dire Graffissimo, come si figura il Carnevale, —
BACLAMO la terra dei boccale, Baciar la terra è un' atto, che si
fone divote per umilta s Ma coftui foftenendo l'equivoco del far
haver detro, che gli piace il modo del digiunare, che fa Piaccianteo, d
vuol ancor' egli far' un' atto d' uiilta con baciar la terra, ma
boccale, cioè bere. Bocca/e ¢ un vafo di terra capace della meta d' un
si piglia per tutti li vafi di terra a quella foggia, ancorché maggiori, ¢
ta di un fiasco anche più, t
PER accoftarsi pitt mondi, Per accoftarsi
nitenza,¢ d' umilta con baciar la terra.
RELIQV1E. Avanzi, fragmenti;¢ scherzando fempre con la bontà
fezione del penitente, par che pigli re/igure nel senfo speciale, che I
noi, cioè offa, ed altri fragmenti di Santi,ed ci vuol poi dire gli avanzid
lui mangiamento. Latino mense relique. Ed in quest ottava |' equivoco:
ficnuto da coftui in moftrare a Piaccianteo di credere, che egli fuffe u
te, che flefle quivi per fare aftinenza, come haveva detto; ¢€ per i
tentarsi, che essi ancora 's' accomodino con lui a far la penitenza nell
nicra, che faceva egli.
STANZA LXVIL STANZA LX
Qual madre, che ripara il suo figlinolo Così fam carua di pik rigaglie —
Ch ¢ sopragginnta da mordaci cani, Oltr' ad un'Oca groffa ar j
Ei cuopre tutto con il. Serraiuols, Ma vedendo pits 1a fra quelle re
Ed eglino gt danno in fale mani; Dun perro d'arme luccicar,
E col laza del Piccaro Spagnuolo, E del giaco feappare alcune ma lie f
Che dalla menfa-vnel tutts lontani, Da quella sua cafacca untae
etecio pot a tal cose non arrivi, Infospettiron, com' un' altra volta
Con due caici lo fan levar di quiri, Patra sentir chi volencier m° ascolta,
Piaccianteo vedendo, che coftoro s' accoftavano per torgli la roba, cerca di
faluarla,copréndola col ferraiolo, ma essi con una mano di calci V' allontanaro-
no,¢ d'accordo si mefiero a mangiare: Ma intanto,ofieruato, che egli era at-
mato, prefero sospetto, ¢ fecero quello, che sentiremo fozto nel C. 8. stan. 60,
RIPARARE, Rimediare. Val per difendere. Ed in questa comparazione> —
'imita Dante Infer, C. 23. che dice:
Come la madre, ch' al romore ¢ defta,
E vedo prefo a (e le fiamme accefe,
Che prende il figlio, ¢ fuge, ¢ non s'arrefta,
Havendo pri di lui, che di se cura;
Tanto che folo una camicia vefta, -
FERRAIVOLO, Mantello. Vn panno ridotto tondo, ¢ adattato a coprires
tutta la persona sopra agli altri abiet, metcendolo in fy ic spalle.

  

pil puri, havendo fatto Pau

  

   
 

 
         
    
  
     
 
 
   
    
   
    
 
 
   

p2Ep SRE EE EE

 

    
  
 
 

O-CANTARE;: 255

nuolo, Gli zingari, quando s' abbattono nel corrivo;
fa, che gli habbiano vedata, trovano diverse in-
di farlo ballace, o cantar con loro, o fargli mettere in capo
go, che gli'occupi la vifta, o con fargli metter il capo in ua' arma-
Eee oe, = ae ed invenzioni per nw ed haver
i rubargli ¢ hanno difegnato, mentr' egli aftratto da tali ope-
a badaa ry gli Ducts attorno; clea (pets veggiamo eat
commedia, che il servo aftuto, per truffare il servo stolto si vale di simili
tic. E questo si dice il ¢axo def Piccaro Spagnuolo, cioè invenzione dello Spa-
0 » Donde poi /azo, dazecgiare significa qualunque azione, che fuc-
oi Comici per ¢sprimere il ior pensicro. E /azo, che in Spagnuolo significa
'prende da noi per quel che i Latini diccbbero capeio, (ophi/ma,commentur,
versuria, fallacia, artes, doli, Ed in questo significato va profferito con,
» €non cruda, ed aspra, perché con la cruda significa fapore aspro,
ate, come que) della prugna, della forba mal macura, ¢ simili, che i
il dicono acide; Dante Inf, C. 15,
ie Ed è ragion, che la trai layzs forbi
ge 7 Si disconuien fruttare il dole fico
Z » perché ¢ frutca di fapore, /axz0, cio' acide dicefi da gli Spagnuoli
quafi dai Lat. diminutivo acidu/a,
'AR carita, Fra i Bacchettoni s' intende mangiare insieme. E tra gli antichi
( iconuiti, che si facevano a' Poveri; di limofine, si domandavano dea-
pat, clot Caritadi, EB Pietanza, voce confervatafi tra' Prati, ¢ tra le Monache,
Piatto, o mangiare offerto dalla picta, ¢ carita de' benefattori; non,
indo altro Pieranza, che Piecd, 1] Beato Fra lacopone: Vorria trovar
skune, Che avefe pictanya De lo mio cor afflitto.
ARCT raggiunta, Grathiiima. Vccello soprammodo grafio si dice raggiunto.

 APCCICARE, Rilpiendere; Rilucere. Viene da Lucciola.
: CASACCA. Parte d' abiro da huomo, che copre la persona da mezza la pan.
vin fy al collo. Così Ca/x/a in Lauino; se bene altra forta di vette, diver-
fa \Cafacca, fu detta così, perché copre tutta la persona a guisa, che fa la
tala j se crediamo a Ifidoro nel jib, 19, delli Origini, al cap. 24,

FINE DELQVINTOCANTARE,

 

 

   

Dy

 

S=sSTO

 

 
 
  
   
   
  
  
    
   
   
  

 

 
lll

——

SS OSE See

——

 

  
 
 
  
  
 
 
 
 
 
 
 
  
 
    
 
 
 
 
 
 
 
 
    
  

(ESSE Se ARS th WE
© PA SA a aes oe dP
SESTO CANTAR
Ese WES CSS
ARGOMEN TO, ' 8

5 Nel tenebrofa centro della Terra,
Ove regna Plutone entra la Strega,

oF E vnol che [eco per finir la guerra
Di Malmantile entri f Inferno in leva.
" Fanno concilio i moffridi fotterra,
Ove ciascun buone ragioni allega;
2 Certa al fin le promette ? affiffenza,

Rend' ella grazie, efa di li partenza.

Be secxrige asin
Bia PS A

aoe
STANZA I STANZA IL
Miler chi mat oprando si confida: Di chi creas Letcor tu.qui cht ia tratti®'
Far' alla peggivse ch'elia ben gli vada, Tratto di Adartinazza inigua Seregay
Perché chi pyglia il vizio per jua guida, Cha pin peccati, che non ¢ de' fattiy
Vs contrappelo alla diritta firada. E pel Demonio ogni ben far rinnegay.
E benche qualche repo ei (guarzrijerida Di darsi a lus gid feco ha fattod patti, a
Col vétoin poppain quel che pingli aggrada, ecio ne' suci bagordi la ay 'A,
E' vienposl'ora, ch'ei n' ha arender coto, | Ma frate pur; perché rards,e per-tempe iy
E far del tutto 5 dondola, ch' io feonto, Lo sconterd;.da ultima' ¢ buon tempos hy
STANZA LU. Pee ky
Non si penst dhaverne a uscir netra; E quand' ei possa, non se lo prometta, ty
S'inrighi pur col Diavol, ch'io le dico, Perch'ei, che fempre fu nofire wimite y
Se forse haver da iui gran cose a/petta, We puo di ben verun vederci ricchi,
Che nulla dar le puoch'egli ¢ mendico, Vana fune daralle, che L impicchi.

Ji Poeta havendo pensiero di narrar la gita, che fece Martinazza ai Regno di
Plutone per muoverio ad aiutarlo a diloggiar Baldone da Malmantile, ed a g*
stigare Gambattorta, e Baconero, fa ' introduzione al presente Cantare cons
una riflefione morale ponderando, che quei, che opera male, non pud sperare
d' haver mai bene, ¢ principiando comel'Ariofto C, 6.

Atifer chi mal! oprande si confida
Conchiude, che Martinazza y la quale nou fa se aon sciagurataggini 5 es' édata
al Diavolo, non pud sperar d' haver @ hayer bene, perché il Diavolo e es
 
   
      
      

SESTO CANTARE:

non pt irepral bon paces Hig ak
«10 weet senza riguacdo alcuno.,
va per il verso.buong. Va al contrario di quello, che
da diritta,via.. Sen, epift, 122. Omnia vitia contra naturam
'dinem deferunt; boc off Ingcuria propositum gaudere peruer fis:
dere a reito, fed quam longissime abire; deinde ctiam ¢ contrario flare.
andare a.rigrofo dai.Lating retror/um, Dan, Purg. C,.10, in simil

PPE wii fou
“gy Ofgek Criftian miferi g¢ laff y

) 4 ) (Che della vifta délla mente infermi
pAb Fidanzahavete nes ritrofi paffi.

@' andar contrappelo ¢ tolta da i pezzi di panno, 0 di pelle pe-
in cucirle insieme.s' oficrua, che il pelo vada. tutto per un.verlo, ac-
> iano.. A taftar un panno, o pelle pelofa per il verso, che vail
pilfacile, ¢. non si trova refillenza alcuna, come.a andar contro as
By Jatin

j
» Goda allegramente. |
:
|
;
|
|

 

  
  
 
    

 

 

|
cou vento in poppa. Secondo che ¢j desidera: Come fuccede quando si ha il
Vento in poppa della pave:-¢ significa ¢ m¢gonzj vanno bene, | Greci pure differo

vento navigare.

i OLA ch' io sconto, Vuol dire scontera il buon tempo, che ella fié data.,
; alcrewtanti disgutti, E' detto usato dalla Plebe, nella quale ¢ nato; ef
endo lato detto.da un maceliaro, a cui ¢ra flata rubata in pil volte gran quan-
ita di Catne 5 ed eflendo facto ritrovato il ladro, fu impi¢cato., ed, il maceliaro
'appelo alle forche disse: Dondola, ch' io feonto; intendendo'a vederti
 dondolare Sconto il debito, che hai meco per la carne rubatami.. Dondolare, &
lo Re ciondolare, come appunto fa |' impiccato; ¢ tal Verbo dondolare
i il nome da quel don don,, che fa il fuono delle Campane. E da quello me-
» che faceya quel tanto rinomato vafo dell' Oracolo di Giove, che
rain. Città dell' Epiro, Mima, e con molta ragione, derivarsi il nome
' 41, Dodona Abramo Berkelio Olandefe nelle Offeruazioni al Frammento dell' O-

Peta originale di Stefano de Vrbibus. Dondolare, 0 dondolarfela vuol dire Star-

fenea federe senza far nulla, di dove Dondolne vuol dire un perdigiorno. Quin-

diua moderng Poeta insendendo di questi tali disse:
Voi dal notturno al mattutin crepuscolo
ah hay Vi dondolate,e¢ fate atu me gli hai,
8st. Soaisconer We conchindete 0 proponete mai,
Se non rovine al popolo minkscolo.;

© HA più peccari, che non ¢.de' fatti « Ha pili peceati-ella fola, che non sono
quelli, che sono flati fatti, 0 commeffi da tutto il mondo insieme infino a ora,)

SAGORDI, Fefteggiamenti. Vedi sopra C. 5, stan. 62,

TARODI, oper tempo. Diciamo anche Tardi, @ accio ( cioè avaccio, parola,
antica, rimafa in contado, che vale tofto ) 0 vero} tardi, o avale; che dissero
& ancora gli antichi agwale; cioè ora, in questo punto; vuol dire;questa seguira una
s olta:opreflo,otardi. Lat. /erins, ocyus. 3
i. Mare fh Kk DA '

us SA EE

SaaS TS

'

 
 

 

 
    
       
   
 

ase MALMANTILE ”
DA ultimo è buon tempo, Da ultimo verra il ferena Pe
deito ironico, perché significa, che da uttimio per Martinazza'
tivo, cice fara gaftigara del suo mal fare, ©
INT RIGARS?, Vuol dire impacciarG, o intereffarsi: ¢ vuol
giiare, 0 mescolar una cosa con un' altra in-manicra di
go per imbroglio. Bihes ae BR
VIA fane daralle, che? impicchi, Quand' altri ci ha im: niti, pe
gli, che non merita rimuncrazione, si fol dire; Gli vud dare un par
Vn par di funi, 0 una fune, che impicchi. if = ie
STANZA IV. STANZA

    

 
    
 
 
 

 

Horsit tiriamo innanzi, ch' io ku finito Ella ch'in tanto havuto havea,
Perch' a questi discorsi le persone Che quei due spirti feiocci ed
Von mi dicefer: Questo feimunito Havean dinanrs a lis fatto U

 
  
 
 
    
   
 
 
   

Virol farct qualche predica ofermone, Si che dat esso furono Scoperti,
edirenti dungue. Gid v'havete udito Se la digruma, che ne va il fuo'
L incanto, ch' elia fece a petixione Mentre gli accordi se
Di quei det luego, c' hebbere concerto Rinsciti alla fin tutte,
“Scacciarne il Diuca;ma fuani lefferto, Con un palmo di nafo ne vil
Ii Poeta lasciando da parte la moralita,viene al racconto, ¢ torna alla
tia de] Lettore ' incanto fatto da Martinazza per cacciare il Duca
hebbe effetto, per lo che ella è in collera, 4 le pare di perdere<
ma, nella quale era tenuta dai popoli, ¢ soldati di Malmantile.
SCIMVNITO, Sciocco', scempiato. Vedi sopra C. 1. stan. 17:
SVANL: Posserto, Non riu(ci? effetto: il negozio ando in fumo, 1 Lat.
'dissero Exanuit, & evanescere. 198
SE la digruma, Seco stessa 1a pensa, e maficandola non la pud inj
'cioé rion la pud fofierire. E si dice digrumare, ¢ ruminare, ¢ dagli an
'to rugumare, onde forse & fatto digrumare; (che ¢ il rodere che | Ie
ipi¢ tefio, come vedemmo sopra C.g- stan. 6. ¢ C. 5. stan. 63.) perché
fucceda cosa di poco suo gulio, fuole per lo pil stando pensofo ma
scrare appunto come fanno dette beftic quando digrumano., al che
ebbe riguardo Omero in quel verso tradotto da Cicerone.
Ipfe funm cor edens, hominum veftivia vitans, a
quafi che chi maninconico rumina,'e biascia mafticandola male; 'moftri di
carsi il cuore. %; a
RIVSCIT I tutti panzane., Son riusciti tatte vanita, tutte chiacchiere
anzave, bubbole, chiacchiere, ec, vuol dir promettcre » ¢ NON mantenere, ©
dice inzampognare, infinocchiare, ed' il Lat, Verba dare. 4
RIMANE con tin palrio di nafo, Riman burlata', beffata, I-Lalli En, trl
stan. 11. dice. tet T stkaigit

 
  
  
    
 

   
   
  

Ed io fon per reftar in quefie caps
Con fei palmi lunghiffini di nafo'e
 

 
 
  
     
   
   

SES TO C'AIN TAR E:

WZARQVEE cad oon il STANZA VU.

se Basta, chella fel ¢ legata al-dizo,

 Etha prefa co' denti,, ¢ fer! affenns;
Tal ¢ andarfene in Dite ha feabilito,
Perch ne viol veder quanta la canna,
Ed oprar,, che Baldon-refti chiarito
Crambisce in Malmiatilfedereaferana;
Hor mentre a quefia volta 8 indivi,
Potra far un viaggio a due servizj.

non si perde d' animo, ¢ vuole in ogai maniera scac¢iar l'elercito

a Malmantile. Risolue pera d' andare all' inferno ia persona a tro-

-» per ottener da lut il gaftigo di quei due diavoli, che feccro i'errore,

jo modo di far diloggiar Baldone da Malmantile,

shiz eee fiperde d' animo; Non si gomenta..., Vedi sopra C.

8.6 C. 5. stan. 63.

b lifterrefe. Viebbe finito di cono(cergli, Hebbe viflo quanto essi va-

Si dige Ta m' bai dato il mivrefio: Tu m' hai preno: Son fazio, fon feufo di

r intendere Now mi varro mai pili dell' opera tua.

hanno fatta di figura, Le hanno fatto una ingiyria grandissima, unas

ima buria. Tratio dal giuoco di primicra, quando uno havendo buon,

ed efiendo per vincer la posla, un' altro con figura fa una primiera,e gli

  

   
   
  
   
    

ANNO un caprefto. Reftino impiccati. Chiamano caprefto quella cor-

jie, che il Boia lega aj cojlo a coloro, che egli impicca, la quale di-

morto il paziente si rompa; ¢ però dice rompano un caprelto; detto

tidimo per intendere farsi impiccare.

DERRE in lowatura, Ridurre in minutissimi pezzi. Limatura si dicong quei
i che cascano dal ferro, 0 altro metallo, quand' altri lo lima.

i morfe mai cane, ch' 10 non voleffi dei fyo pelo, Nefluno mi fece mai in-

40 non mi voleffi vendicare. Nefluno mi morfe, che io non lo rimor-
si, D che il pelo del cane sia medicamento alle morficature fatte dal me-

'  defimo cane. Vedi sotto C. 9. stan. 58. Eda questo rimedio ha origine il prefen-
te dettato; che i latini dissero Nemo impune abyt, qui me aufus fit ledere,

it SEL' è legara aj dito. Ne ha prefa memoria per vendicarsi. Sogliono molti per

 haver memoria di qualche negozio, che deyano fare,legarsi un filo intorno a} di-

a = i che ha dato origine al presente dettato. Ii Lalli En. 'Lr. Can. 2, flans25,

Tigran Sel' attaccd, come fuol dirfi, al dito.
Nel Deuteronomio alfefto, Eruntque verba hac, qua ego precipio tibi hodie in corde
tu: © narrabis ea filijs.tuis., & meditaberis (edens in domo tua, & ambulans in itino=
1, dormiens arque confurgens: Cr ligabis quafi signum in manu tua, B (ono al cap.it,
Ponite hac verba mea in cordibys 5  animis veftris, & fuspendite ea pre fieno in man
 tiibus, Bra Giordano Predi antico Domenicano; nel Vocabolario della.
 Ctusca alia Voce Filareria. Le filaterie si erano una carta, ove erano scritti i co~
Mandamenti della Legge, ¢ portavanla — al braccio apertamente. B quivi

S$4NE2 2 va

 

Dia

 
= Bi Asi - SS ee oe

= Se eS

ee

 

254 MALMANTILE?

va spiegando, cred' io, il paffo di San Matteo cap. 23.) Di
Jua, B voce Greca; da phylattein > puardare, di 1
di quoio, o di cartapecora, che gli Ebrei si legano albraccio'
mente a memoria-i padi delia Scrittura, che.quivi sono nota
domandano:Tephilim, wa haba sof eo eseytt -

L A profa covdenti » S' & adirata grandemente,°¢ sé meffa in a
dicarsi. Vuol impiegare ogni suo-stadio per vendicaré icalzolai
venire il quoio a quel fegao che loro bilogna; tirarlo co'.denti 5 di
presente termine, che esprime uno; che si sia prefo 'a cuore di' un
e'che vogiia impiegare ogni suo talento:per conchiuderloy

SE »' afanna, Sel & prefaa cuore: N' ha premura * Sene da pena 5
fiero., |

 
     
 

ne,|'uno,¢ l'altro nome significado ricchezze delle quali,perché si cavano di
ra, facevano Cuftode,e Padrone quel loro Dio forterranco; ma qui si piglia
per la Città, ¢ per il Regno di Dite,: aa
Ne vnol veder quanto la canna. Cio' quanto tira, 0 & lungs la canna da mifu
rare; ¢s' intende vederla per la minuta, ¢ quanto si pud, ¢ fare ogni sforzo pet
arrivare al suo intento, °
REST I chiarito. Refti (garito: Scaponitox Vedi sopra C. 1/stan. 12”
SEDERE a feranna, Vuol dire comandare'; esser padrorie s “Scrannay
me diciamo noi ) ci/cranna,è una specie di seggiola da i Latini detta se
Dante Purg. C, 19. dice: ' a cS
Hor chi fei tu che vuoi federe a feranna a
Per vindicar da lungi venti miglia alee
Con la veduta corta d'una spanna ? Ue aan
Buratto nell'Apologia contro al Castelvetro dice + Aon habbiare tanto cermelle, he
baffi, se ben volete (edere a feranna per giudicare gli altri, uy
FAR un viaggio a due servizzj, Che dichiamo anche + Fare un viaggio 5 e dies
servizx). Con un medesimo viaggio far due negozzj, che è impetrar da Plurone
il gaftigo di quei due diavoli, e lo sfratto di/Baldone, Ne i Latini si trova ims
questo fenlo Duos parieres de eadem fidelia dealbare,E si dice anche Dare a diet
vole 4 un tratto, Vedi sopra C, 3. tanet4. > oe

On

 

., STANZA VILL STANZA 'IX.
Git da Mammone andar vuolein persona, Percid s* accontia, e-vie tutta ph y
Che pile non ¢ dover, ch'ella pretenda 5 Col drappoin capo,e vol véraglioin mant
Che sua bravicornissima corona © cercar chi? informi della cite;
Salga a suo conto a veni poco, e scenda, Ne meglio'fa, che Giulio Padovanes
Chieder grariese dar brighe no céfuona, Chet ha fu 'per le punta delle ditt)
E chi ha bifogno,si suol dirys' arrenda, E pik ds Dante, e pi del Manrovans
Per questo a lei tocca apigliar la firada, Perch eglitio vi furon di a
'Per calla fin conuien, che chi vuol vada, E questo ogni tre di vi '
, ol vt sain wan?
at oie

t: 3h wath ~ 1,

IN Dite. Dite, fecondo il favolosorcreder de i Gentili'é lo steffo see

 
 
   

  
   
     
  
      
     
  
  
  
   
  
 
  
    
 
    
   
 
 
 
 
 
   
    
 

 

SESTO CANTARE: "255
WaAE PCS. S. - VOW ZA)

  
     
     
    

E poi per abbondare in cautela,
che in Di prefume) Volendola fernire infino al fiume,
che gente, ¢che loquelay Le porge un fardellin piccolo, € poce
ple dd conto,¢ lume; Di robe, che laggth le faran giuoco,

a'rifolue' ae in persona a aan x oe che
y che questo Re per lei a ogni scomodi; ¢ però sapendo, che
Padovano è pi aibithaes d oll alevo' della strada dell' Inferno, se ne va
r da lui informazione, ¢ della gita, e dei coftumi di quei paefi; ed egli
ice, ¢ per servirla meglio la vuol accompagnare fino.al fiume Acheron-
intanto le da un fardellino di robe, che laggib verranno.a bifogno'.
VICORNISSIMA corona, Epiteto, ¢ titolo composto dall' Autore a Pla.
gh they Lalli' Bn: Tr. lib. 1: tan. 16. parlando d' Eolo Re de' Venti dice;
ie sh » Dunque poi che Giunone alla prefenza
ro) pis) ies Di sua Real ventofita fu giunta.
'  - MAMMONE, Da Mammon; parola wfata nell' Evangelio. Alcuni esposi-
ue | Sacra Scrittura vogliono, che Mammona sia voce Caldea, ¢ significhi
$5 ed altri che sia voce Siriaca, ¢ significhi quello, che in Greco significa,
» che € divitie, si che concordano, e tanto è a dir Mammone, che»
 Demonio, ovvero Plutone, che qui s'intende per il Re dell' Inferno. Vie-
a ne dalla radice Ebrea Taman, che propriamente significa a/condere, riporre, es
Per Così dire-intanare; onde si fece AZatmon, ¢ alla Siriaca Adatmona, cio' ricchez~
Xt nafeofie, © vogliam dire teforo. Mammona poi venne a dirfi per più agevolez~
za zia
2. + Dare scommodi, dar moleltie. La voce briga significa opera-
2ioni (coirimode, faticofe, ¢ noiofe. eit
yf CHP ha bifogno # arvenda. Chi ha bifogno non sia superbo, ma si pieghia rac.
Ȣpregare; Che il verbo arrenderfi val per cedere piegarsi, 0 con-

 

' Cc.
ie CHI wolf vada. Chi vuol ottenere una cosa vada a chiederla da per s¢, ed il
et Ndice Chi non viol manat, e chi viol vada da se, Che diciamo anche Non i
ie «ibe mee, Che le speffo, 0 vero, Chi va lecca, E chi fha si fecca, i

ACCONCIARS!. Rinfronzirfi, raffazzonarsi. Vedi sopra'C, 2.stan, 69,
DRAPPO, Dicendoli drappo affolutamente s' intende drappo da-donna s che
una strifeia di taffetta,o d' ermifino Jarga fino a due braccia,e lunga fino aquar-

tro, la quale dalle donne Fiorentine di condizione ordinaria € portata in capo,

Oalle spaile quando vanno fuori di Casa. In Venezia drappo significa ogni forta

diveftimento, si come prefio i Toscani antichi scrittori. Vedi foto C.7.Man.az,

VENT AG LIO, Strumento noto usato dalle donne la state per farsi vento,

ZL INFORMI della gita, Le infegni la strada, che conduce all' iaferno,

GIVLIO Padovano, Io veramente non ho saputo ritrovare chi sia questo Giu-
 lio Padovano, se forse non ha inteso di Giulio Hygino scrittore d' Aitronomia,

Ma coftui fu liberto, o vogliam dire schiavo atirancato d' Augufto; condoreo da

lui ra d' Aleflandria, fecondo che alcuni vogliono; i quali percid lo stima-

no Al rino; o pure di nazione Spagauoio, fecondo la teltimonianza di Sue~
¢ Maio nel Libro de illuftribus Grammaricrs. L' HA

 

 

Sea S KE =

 
 

 

   
 
 

256 MALMANTILE ©

L' HA fu per le punte delle dita, La fa benitfimo; Latino im m
do Manuzio nella dedicatoria di Giuvenaie disse: Quando eas tench
digicas ungue(que twos, Cicerone nella Orazione cont i
uid cum accufationis tua membra dividere ceperit 5 @
canfe conftisnere ? Quid, cum unumquodgue tranfigere, expedire,

DANTE, ¢ il Mantovano, Dante Poeta Fiorentino; ¢ Vergilio
te finge y-che fuffe sua guida all'Inferno, ¢ pero dice: Egéine vs furon

| OGN tre di, Questo modo di dire, se bene € determinate, significa tp
fo, ©.a ogni poco indeterminatameate, eden ah

ANDAR via divela, Andar via velocemente, ¢ a dirittura, come |
quando va a vela. » one

PER abbondare in cautela, Cioè per servirla bene. Diciamo abbondi
quando uno fa pitt di quel che Ga richiefto, o pil di quel che sia n
elempio. lo dard diect feudi a uno, perché mi compri una mercanzia, la«
fo che non vale così gran somma; ma per aflicurarmi del cafo, che valeffe:
ir cautelato'y

   

 
 
     
     
  
   

pil, li do due altri (cudi per abbondare in caurela, cio per anda
ful ficuro, che non gli manchi denaro,se ella valefe pi. Qui pegd
Abbondare, ed eccedere in corcefia nel servirla. rune
LE faranno ginoco. Le torneranno a proposito., Le verranno a bifogno, Le
faranno d' utile. sod
STANZA XI. STANZA XI

Così la Maga se ne va coneffo, Questat la via, che mena
* Che f introduce in una bella via Perch'ellac allegrayo
Tutta frorita, st che al primo ingreffo Perché 4 martello poi non we
Par proprio un Paradifo, un' allegria; LA feorre ognor gente di mal afarts —
Ma nipin prefol buom il pic v'ha mefo Le ferpi sono ogni opera ribalda y
Cb' ella diventa un' altra mercanzia Chrella ci fale quali a lungo ani
Per i gran morfi,¢ le punture acerbe, Di quanto ha fatto, scavallata, ¢fenfe
Che fanno i ferpi ascofi fra quel! erbe, Ci fa sentir al cuor qualche rumorfa,
STANZA XIL STANZA XIV. 5

   
     

 
   

    
   
    

    
     
   

Entravi Martinazza,e fente un tratta tMa se ravvisia un tratto del sue,
Dueyo tre morfi a più dove calpefta, Bada a tirar innanzé alla balorday —
Percio befemmia, che non par [uofatta, Perch'il vizio rifiglia,e mecte il tally
E dice:O Giulio mio, che cosa e quefia? Vie stpre pix aaggravar/iinfulacerds,

Ed ei ridendo allora come un matto; Ul male invecchia al fine,e vi fa ilealle 9

Non è nulla ( rispose ) vien pur lefta; Siche venga un Serpente parese '

Che pensi tu ch'io sia privilegiato so Chrei x6 fente ne meno anctun ribregeny oa
bY 19 mi fento mardere,e non fiato. Cos} peggio che mai la da pel mezts

pa kar "STANZA XV, er

  
   

lla neve si f4 lo steffo giusco y ' Al fine ei ff rifealda come un fucco
Ebert ment Jul primo dacciafi le dita, Si che non lofarwits mai finita

Poiquelgragelo par che maanchi un poce, We gli darebbe punto di spavento'

E fempre pine nell! Agitar (a vita; Quandei thavelfe acoraa dorami

Martinazza se ne va con Giulio, il quale la conduce per una strada,
primo mgreiio pare una belia cosa, ma prefto si conoice, ch'ell'é al

SSLPEE sR Eee Sp Ge eae gis
 

    
  
    
    
  

SESTO CANTARE, zy7

i-€ i i ascofi infra quell' erbe; Giulio moftra a Martinazza,

 strada, che guida all' Inferno è facile, e gustofa, ¢ se bene ¢ ripicna.,
i, non fon sentiti ne conosciuti da quelli, che la camminano, perché
afluefatti; appunto come fanno coloro, che mettono le mani nella ne-
i ipio la toccano fredda, ¢ col seguitare a maneggiarla, par loro
PARE un Paradifo., Pare wna-cosa tanto allegra, ¢ vaga-, che più non si pud
 fare. Telemaco figliuol d' Viifie nel quarto del' Viiflea, arriyvato in Sparta; nel
 considerare attentamente la ricchezza, ¢ |' ampicaaa del Regio Palazzo di Me-

i — in quella ¢sclamazione: Ta/ dentro ¢ del gran Giove ilgran Pa-
tO,

4 ENT A wn altra mercanzia. Diventa un' altra cosa. \Véiamo dir mercanzia
i ogni forta di cosa ancor ehe incorporea, come 40 frudiare sé una cer-

py eC,
par (ue fatto + Non par che faccia quella tal cosa. Vedi sopra Can, 4.
aj stan. 16.
-) CASA Calida, Intende V' Inferno. Il Lalli En. Tr. parafrafando facilis de-

 ferfus Auarni ec. dice:
= Eves mio bello
: A casa Calda si va prefto prefto;
' | seen 'CHa ritornar infu, quale è il bordello.
] NONE nulla, Quette due negative fecondo la buona regola doverebbono afier-

| Mare, ma è nostro idivtifino tanto inveterato, che I” uso-ci libera dall' errore, se
"'@eneseruiamo in questo modo per negativa. Apprefio i Greci due negative, 0
. = affermano, ma negano maggiormente, ed ¢ maniera, siccome appref-
40 noi; così appreffo loro usatissima.
¢ WNONfaa marcello, Non regge alla prova. Noné-com' ella pare. Metafora
yf 'tolta'dal Cimento dell' oro. Vedi sopra C. 5, stan. 2.
'4 LINGO andare. Col tempo... In procetio di tempo'; Se continoverai lun-

ee. '

fo SCAVALLATO, Ciok datafi ogni forta di bel tempo. Si dice anche scorrer
i,  latavallna » Virg. 3. Georg. Scilicer ante omnes furor eff infignis equarum, Bt men-
io 'tem Venus ipfa dedit. E poi: dilas ducit amor trans Gargar astranfeque fonantem,&c,
»  VedifopraC. x. an..66.:

|) REALCHE vimorfo. Senton rimorder la coscienza'per gli ¢rrori-commefi.,
@ — ALLAbalorda, Senza-considerazione.

METT £ il tao. Talliice, fa nuove mefle, Vuol dire:sun vizio ne genera,
g  Mholti, Tallo @ parola veriuta a noi dalia lingua Greca, che significa germoglio,
et 'usata ancora dagli agricoltori-Latini..

»  VIENE «aggravarsi in fu lacorda. Vien più che mai a crefeere il male; perché
ido uno tocca il martirio della corda, ¢s' aggrava in fu la medesima corda,
~ fa crescere il dolore; 'Ed altrimenti 4g¢ravar/? in fu ta corda vuol dire quando uno
#  sfaminato'in fu la-corda dice-cose, che fanno crescere I"indizio, che egli hab.
y EMS somuneo-un dette. Ang

§ | “PAilecatlo. Vis' afiucka. Er ab afuctis non fit paffio, dice, che-non fen-
si “se pcmeno.un. c. a RL

 

 

 
Foe,

a ee a

 

 
  
  
   

258 MALMANTILE,

A/BREZZO, Che vuol dire'capriccio di febbre; cioè quel
che si fente prima, che entri la febbre « Latino rigor avalc
lib,2.cap.21, dice: Antipatro di Sidonia in quel giorno, che egl
gh arrivava qualche ribrezzo di febbre, ¢ tanto continua, s ce
mortale accidente, Ma Dante nell' Inf. C. si
Qual è colic! ha si prefa
Della quartana, ¢' ha già fugna smorte
1 E trema tutto pur guardando il raze.
BalC.za.dice: Pascia vedd' io mille vifi cagnagri;
Patti per fredds, onde mi vien riprenroy

E verrd fempre de i gelati guazzi,
Ma noilo pigliamo anche ( come ¢ pre(o nel,presente Iuogo ) per ogni leggi
follevamento d' animo, o-spavento, 0 per un jemplicitiimo dolore, Bdal
te per fattidio, o travaglio per efempio // rale commelfe quel mancamento; ne,
haver de! ribregzs, Vedi foto C, 11, stlan.2., 9 spkway ¥
La dd pel mezzo, Fa tutto quello, che gli vien yolonta senza riguardo aleuno.
E' dedotto da quelli; che in tempo di pioggia camminando per la Città yanno
per il mezzo della firada, ¢ non si guardano dal' ammollarsi per J' acqua cadu

 
 
   
 
     
     

     
 
  
     

   

Se SEES ST ee

=

 
 
   
 
 

  

  
  
   
      

ta, che scorre pel mezzo, ¢ per quella che vien dal Cielo, <i ied
STANZA XVI, ' STANZA XVIL sp
Hor tu m' hai inteso:rafferena il volto, Refta, dic? ella, omai ch' io ti ringrazia We
Che tu vedrai tirando innanzi il conto Dellinfernxionsch'appiio: li
( Perché di qui a poco non c' è molto ) Promiffio bons viri off obligatiay
Che delle ferpi non farai pin conto, 1 Die egli; Tho promeffoge intends a
Ma dimmi, c' ba' tu fatto del rinwolto 2 Ancor seguirti questo po iay) ha
Lho qui, dic'ella,fempre lefto,e pronto: E quivicon un tibi = "
Sta ben,foggiunge Ginlio,adungue corriy Ail) in qua ripigliands.il mio.cammind

 

Perché qui non ¢ rempo da por porri, Ti lascio, come io diffial
. Giulio ¢forta.Martinazza a non haver paura, ed a camminare; ed
grazia dell' instruzione datale, ¢ lo prega a partire, ed egii ricul di farlo, pet
¢hé l¢ ha promefio di accompagnaria infino al fiume-Achcronte.. 3... 4 ki
D1 qui a poco non ¢? ¢ molto, Questo termine giocofo ¢ usato per esprimere r+
ochissime tempo. ' 4: vowed Sag
TIR ANDO innanzs il conto, Seguitando ll suo viaggio, EB' termine mereaatilt,
che vuol dir portare un conto avanti da un libro a un' altro, oda unacattaae
un' altra nel medesimo libro, Donde poi tirar innanzé il conto vuol die Cammina
re avanti. Vedi sopra C. 4, stan. do, ot, nls To i
NON è tempo #a por porri, Noné tempo da perdere, Non & da indugiares.
Quando si pongono i porri, sono così fottili, che Hehe
i a Wnt
| big

 
  

   
 
  
 
 
 
  
  
    

*Zezesaerea

[Ras

porgti; eda questo habbiamo il presente proverbio, che si dice anche +
tempo da dar fleno aoche. i w naobs
PROMISSLO boni viri eff obligatio, Sentenza latina, che vuol dire un':
da bene ¢ obligato a mantener la parola,.¢d offeruare quel che ha p: a2
CON un tibi me commendo, Detto latino, che fuona con un. mi & a
te; ciog con falutarti.. Quando diciamo: Addio, C1 s' intende; vi.raccomandd «
' ms Sadun

     

 
 

f ut  -gRSTO CANTARE; 19
alut lo; Catullo: Commtenide tibi me.

  
     
 
    

colonnino, Tial

AANZA XVIIL

ti '4 il capo, @ rocca,

¢ ferpi elt a qualche paura;
2 x tase fatto def enor roccay
Vi otal teense ntl

is

£.

 

 & Bed

disse: Vale.

til famofo fume d' Acherorte,
ve s'imbarca ognun, che quivi arriva,
| Saffaccia ach'effayma il nocchiere ar ite,
7 che trate ognuxo hebbe dariva,
dietra,grida a lei con toruafrote,
Che quad non pafsa mai anima viva;
~Ond' ella, meffi fuor certi baiocchi,

2° Lascia? ad Colosnine yuo! dir lasciar uno
laren i mo
vanti alle forche, €'vi leg

Ha colonnetta di legno traforata—,
inialfactori 1 gli strozzano.
STANZA XxX,
Ed égli, che da e/a bebbe il fapone,
E che si trove li come tl ranacchia y
Prefo dalla wiedefima al boccone,
Menty" ella faite in barca, chinfe locchia;
La Strega sia quell anime si pone,
Luaic nlebrach: fon fino al ginocchio y
Duvendo a' Sopraffindagi di Dite
Presentar de' lor libri le partite.
STANZA XXL
Piangendo, come quando yno ha partite
Le cipolle fortissime malige:
Pafsan quel fiume,e poi quel di Cocito;
Vitimamente la palnde frige,
Che a Dite inonda tutto il circuito,
E in se racchide furbi,e anime bige
Ove Caronte al fin [endo arrivata

Gli getta un po di poluere negli occhi, Sharco tutti; ed ognun fu licenziato}

'Martinazza seguita il suo viaggio, e non fa più stima delle morficature de i
sespi yed artivati al fiume d' Acheronte', Ginlio si licenzia dalla donna, la qua-
Te ) per entrar nella barca; ma Caronte lo rie dicendo, che non poteva
eotrarui, ond' ella gli diede un poco di mancia, ed ei finfe di non la vedere en-
Wat in barca, dove ella si mescold con gli altri, ¢ fu condotta all' altra riva,¢
guivi con essi sbarcata.
oe, « Sidice tocca il cocebio, ¢ significa: cammina innanzi. Vedi sopras

41.

ZAMPETT A. Muove le gambe: Cammina. Zampettare si dice propriamen-
te de'bambini quando cominciano a imparare a andare.

NON si fente aprir bocca, Non si fente parlare. Sono infiniti i modi, che hab-
biamo per esprimer il filenzio d' uno, come far Zitto; non fiatare; non far verbo
mmansolire; star chiopto, lasciar la lingua al beccaio, baver viffo il lupo; diventare Ar»
porrate ec.

GLI difie: Vale. Gli disse Addio -

ACHERONT £. | fivmi dell' Lnferno da i Gentili si dicevano quattro, ¢ che
fialteflero dalle Jagrime de' mortali, per lo Mato de' quali figura Dante la flatua,
the vedde in foguo Nabucdonofor, che havea la telita d' oro, le braccia, es
Vie d'argenio, il sorpo fino alle cosce di rame, le gambe di ferro, ed il defiro

di terra cotta; da questa dice che scaturiscono le dette lagrime, le quali for-
Mano li detti quattro finmi Infernali, ¢ così la deferive nell' Int. C, 14,
Dencro dal monte fra dritto un gran veglio,
Che tien volte le Spalle in U Te;

A% BBt BW BEERS. SS Sak

=

SAFE

E Ra-

 

 
260 MALMANTILE

E Roma guarda si come suo [pegtio,
La sua tespa ¢ di firt oro formata y
E puro argento fon le braccia, ¢ il, A
Bhd dl nec tcan eee — ah x Tan
Da indi in ginfo¢ tutto ferro eletta, 7 tae
Salno che'! deffro piede è terra cotta y ref 9
E fla in fu quel pinych'in fu U altro, eretto a
Il primo dunque di detti fiumi ¢ Acheronte, che ia un certo modo
privazione d' allegrezza;da Acheronte nasce Stige, che significa cosa dispi
odiofa, quale ¢ il Dolore; perché guefto ne viene dopo la privazione dell'
grcaza, Ll rerzo è Flegetunte, che signfica pensiero ardente travagliolo.
questi tre fiumi si genera il quarto, che ¢ Cocito stagno,o fiume del
pianto. Questa favolosa opinione de' Gentili tocca Dante nell' Inf, C, 14.!
tando i sopraddetti versi. om

  
        
    
 

  

Ciascuna parte, fuor che l'oro ¢ rotta ity
D' una fefsura, che lagrime goccia 5 ait
Le quali accolte foran questa grotta, one

Lor corso in questa valle si diroccia ae
Fanno Acheronte, Stige, e Flegetonta; ihe
Poi fen va gik per quefea firetta doccia y;

Infin la dove piit non si difmonta, ~ alt
Fanno Cocito, e qual sia quello fhagno
Tu'l vederai, però qui non si conta,

 

 

ih

at '
CARONTE. Notissimo barcarolo dell' Inferno. Vedi sopra C. a, flan.2g) |,
HEBBE tratto ognun da rina, Hebbe Jevate d' in fu ja riva tutte ! anime, im |
barcandole, hu My
TORVA fronte, B latino usato da noi; E vuol dire Vifo burbero,asp toy
arcigno. Rs,
ANEMA via, Intendi huomo, che non fiamorto. Virg, 6, Em. Corports | (ty
vina nefas Stygia vettare carina. Sa bene il nostro Poeta, che anime sono immors | i
tali, ma seguita il coftume d' intendere huomo viveate, quando diciamo animas jij
viva [ Genefi cap,2, Ee fattus eS homo im animam vinentem ] ed imita Dante !
C, 3. che dice; }
E tn che fei coft} anima vina, |
Partiti da codefti, che fon morti, |e
Ii Lalli Ea, Tr. C, 3. stan. 16. ea
E non v' è mai entrata anima vina y iy
GLI gettd wn po di poluere negli occhi. Gii decte un po di mancia, I Latini pute jp
difero: Puluerem oculis offundere, Es' intende dar mance per corrompere il git %ir
sto, quafi diciamo: Abbagliare gli occhi del gindice con l'aro 5 accivcche mom ty
instizia. 194
£ pire il fapone. Exa flato fubornato, e corrotto con la mancia; Gli er: y

flare infaponate le carrucole (che vuol dire Tirar' uno al nostro volere, erent
derlo facile a quel che noi bramiamo, ¢ fare che non strida contro di noi ) coma
dargit la mancia; come con! iafaponare nna carrucola, o na ruora si facilites

| y
:

 
 

   
 
 

  
  
    
    
    
 

261
cOlO 5 }y che non frida, Ed & 10 steffo che getrar fa poluere negli ecchi
poce Dicefi anche: Vener le mam. Bocc, Nov.6. 2 bucno husmo por
i fece ngner le mani,
come il ranocchio. Obbligato a tacere, per havere havuta la
li faddetti due modi di dire, cio& Havere il fapone, e>
Qui non vorrei che ij Lettore credefle, che il Poeta
'egali potcficro corrompere i demonj, se bea la fentenza
lice munera(crede mibi)placant homine/que deo/que,ma (apefie haver'
-moftrare che l'oro arriva a corromper quelli, che ne meno si
¢ meno dovriano lasciarsi arrivar dail' oro, ¢ finalmente ha vo-
la posianza, che hanno i regali di far confeguire cid che si vuole.
per pecuniam fattafune, Si racconta di Filippo Macedone, che haven-
9 riconoicere una fortezza', ed cflendogli riferito, che era imposfibile il
la, domandafie agli sploratori,se vi era modo di farui andare un' afino ca-
volendo inferire, che dove non potevano I armi, farebbe arrivato
5 uri Sacra fames,quid non mortalia pettora cogis ? BE Orazio. aurum per
ire fatellites, Et perrnmpere amat faxa potentius Iitu fulmines,
YSEL oechio. Finfe di non vedere. £7 il latino connivere, Vedi forto C.

  
 
 

    
  

      
     
     
      
       
 
   

 

 
 

Rieke fino al ginocchio, 11 proverbio Ca/char Le brache & i) medesimo,
le braccia; che vuol dir perderfi d' animo, Omero: <dnimus in pedes de.

il cnore; cascd l'animo a' piedi, Onde dicendo, che coftoro have-
che fino al ginocchio, intende che eran loro cascate affatto, cioé erano
> — @ animo, perché doveano render conto delle loro azioni. Vedi
SINDIACT. Così chiamiamo noi quel Magiftrato, che ha |' autori-
cr icontia tutti i Magiftrati, Ofiziali, ¢ Miniitri del dominio Fioren-

  

 
 

 
  

maligia. Specie di cipolla da mangiare, che è fortissima, ¢ fa ve-
ime a tagliarla,¢ maneggiaria; Bocc. gior. 8.n.2, £ talora un mare
cipolle malige, o di Scalogni, 1 Lalii Eo, Tr. C. 3.
Così dicea, ¢ tutto ii volto molle
Haves di pianto, come [e [chiacciato
> ¥ Vi fuffe sopra il fugo di cipolle.
COCITO, Vedi sopra alla stan. 19, alla parola sAcheronte, ¢ quivi troverai
ancora quel che sia la Palude Stige, della quale vedi anche sotto in questo Cant,

3 Ibige; Genti scellerate, ¢ da non se ne fidare. Per comporre il color
310 i Pittori mescolano tutti i colori, ¢ lo chiamano il coler dell' afino; e però
lic | huomo bigio s' intende uno, che ha tutti i vizzj. Va moderno Poeta.,
€¢ notammo spra C. 3. stan. 66. dific parlando d' uno di questi tali, che era

   
   
     
        
    

 

  
  

Y; Chinde un' anima bigia un corpo nero,
di questa parola bigio in quelto significato stimo, che nasca da questo.
in Firenze pe foci paiisti tre ai, una de' Fautori di Er, Giblin
= Sree ke Savona-

 

 
 

 

 

 
 
 
   
 
  

264 MALMANTILE =~

Savonarola, la quale era detta de' Piagnons, I altea de? conttarj a detto Fr. Gite: |»
Jamo chiamata gli e4rrabbiati, 0 Compagnacci; ¢ sta'di loro ¢1 teo'nimi- = |
ci, e discordi, faluo che univano nel? esser contrarj alla terza fazione;, ¢
de' fautori de' Medici, la quale era detta de' Padefehi, i quali non conue
ne con l'una, ne con l'altra fazione. Di questi che inclinavano alla |
Patieschi taluolta alcuno per suoi fini particolari s' univa 0 con Puna, 0 con!
tra delle prime due, ma era ricevuto con sospetto, che non fale)
ro deliberazioni, ¢ pero dicevano: Won ¢ da fidarsi di lore', Son Bigi. E
ucftowforfe ha havuto origine quetta voce bigio in significato' 4' huomo da
se ne fidare, Vedi la Relazione di Firenze del Foscari, ¢ il Nardi nell
Florentine fib, 2.:
STANZA XXIL
Cli entrar dovendo in Dite,e faltn,e gira,
Che par quando mi barbera latrottola,
endar non vi vorrebbe, ¢ si ritira
Grattandofi belande la collottola;
Pur finalmente forza ve la tira,
Come fa il pefo al grillo,una pallottola;
Così ne van quell' anime nefande
Chi dal piccin tirata,e chi dal grande,
STANZA XXIIL

   

    
 
 
 
 
 
     
   
 
 
 
 
 
 
 
 
 

Perch gli è offa,e pelle,
Ch'ei par proprio il ritratto dello
STANZA XKV,

Per la gran calca nel paffar le porze Si che quand! ei si fence il tow
Connenne a ognino andarne cola piena, Lerche la fame quivi ne c
Ma la Strega non hebbe tanta forte, Liingoxxa, che ne manco non gh tocca
Che trenla il can, che quivi Pain catena; Ne di qua, ne di la gih per 3
E perché per tre bocche abbaia forte, Ma fubico gli venne il,,
Ella dice: Ti dia la Maddalena; Ond'ei sallunga in terra afar lai

Chil papavero,e il loglioch

Faria dormir un' orfo, non
ZA X&KVI '
Sdraiata dorme; e riff com wh tf,

E in tatotrovail pane,e in pexsiltaglia,
E in tre gole ch' egti apresgliene feaglia,
STAN

  
 
    

'Hor mentre fa il fonnifero il [uo corso,
La donna che piit la facea la [corta Lerno da bore fa versa la porta y
(Peroccht havea timor di qualche morfo) E poi (benchrella fuffe alquanto [rect
Vedendo che la beffia, come morta Da unacor[aye in Dite anchrellait
L' anime rimafte attorno alla Città di Dite moftrano co' gefti,
lentieri vadano dentro alla Città; ma i loro peccati @ forza ve le
anime nell entrar della porta fecero così gran calca, che la Strega nof
flar con esse, € tanto pili, che ell' hcbbe 'paura di Cerbero'; 'onde'pel
fene gli gett del pane fatto col fonnifero; per lo che il cane' si ad
ella entro nella porta. E quiil nostro Poeta imita Verg.nel 6,
dare a Cerbero dalla Sibilla ana fliacciata'col fonnifero, e nelle pr
23. 24.25. parafrafa,si pud dire,i seguenti versi del medesimo
Cerberus hse ingens larratn rege i
Personat, aduerfo recubans insmanis in dntro
Cui vates borreve videns iam colle colnbris 22

          
    
 
 

       
 

 
 
 

SESTO'CANTARE:

2 9 © medicatam frugibus ofam
Aes, Obijcit; ille fame rabida tria guttura panders
9 Corripit obietam, atque immania terga refoluit
5 Bafus humi, torque ingens extenditur antro,
Il verbo barberare è usato da' nofiri fanciulli per intender quan-
gira a falti, enon va unita per cagione dell' efler mal contrappela-

  
  

 
   
   
    

LA, Strumento;del quale si servono i ragazzi per giuocare, ed è un
foggia di piramide, che fini(ce in una punta di ferro, Vedi sopras
si fa girare avvoltandola con uno spago,¢ poi feagliando 3 ter-

 
 
  

3.

14, tirando con velocita a se la mano, alla quale ¢ legato detto spago.
GRATTANDOS! ta collottola, Grattandofi il capo nella parte di dietro dai
tini detta cerwix. E questo ¢ un' atto solito farsi per lo piti dalle donne, ¢ da'
ndo hanno qualche di(grazia, o gran disgufto. Vedi sopra Cant. 3.

 
     
  
 

  
 
  
 

VDO. Vale piangendo: perché sc bene il belare ¢ proprio delle peco-
li, ¢ viene dalla voce, che fanno tali beftie, che fuona be be, ce nes
lamo anche per esprimere il pianto dell' huomo, ma per derifione; donde si
belone, pecorune a uno che pianga assai. Vn moderno Poeta disse;
Hor ch' è per te finita la pasciona,
Gf Che fai che tu non beli, 0 pecorona?

GRILLO. E' un verme piccolo volatile noto: Ma trattandofi di pallottoles
Grilles intende que!la piccola palla, che si tira per fegno nel giocare alle pallot-
tole,0 alle pialtrelle,omurelle, Vedi sotto in questo C, stan.34, e C.9.staa.17.
PALLOTTOLA, \ntende quelle palle di legno, che servono per giuocare., j
nelle quali sono tre contrappefi di piombo, per via de' quali si fanno fare alies ?

F operazioni, e voltamenti, che si vuole, ' uno di questi si chiama. i
¢atetia, V altro il grande; ed il terzo il piccino, ed il Poeta, aflomigliando quell'

        
  
   
 

=

ASSES EES:

 
  
 
 
 
  

  
    
 
 
  
  

 

ge  *himea queffe pallottole, dice, che ancor' efle fon forzate a entrar nell' inferno \
dal,¢ chi dal grande, cioé chi da i peccati piccoli, ¢ chi da i grandi.

ye! Quantita grande di popolo; folla.

5 ANDAR con Ia piena, Andar co' più; andar in truppa con tutte quelle anime,

m lie piema per similitudine significa inondazione, o furia di popolo. Virg. Georg,
i lane falitantum rotis vomit adibus undam. Andar con la piena significa ancora se-
,g  Biitat Popinione comune; andar co' pill.
ee AL Cane che quivi fa in Catena, Cerbero cane con tre tefte,due delle quali stan-
4  nolempre fuegliate. Hercole lo lego, ed i) nostro Poeta imitando Vergilio co-
' me s'€ derto, lo fa addormentare col/pane alloppiato.
4  TTdiala Maddalena, Posla tu cfler impiccato. Dicevafi porta di Caronte da-
m gli Atenicfi quella porta del Palagio del Podefla, dond”uscivano coloro, che an-
;  davano alle forche, come accennammo /opra C. 5. tan. 3. ¢ noi diciamo Ti dia
ta Maddalena, da quella Campana, che ¢ nella torre del Batgello, la quale suo-
ma, <n va alle'forche, e si chiama la A¢addalena, perché con tal no-
mee » per efler 1a Cappella di quel Palazzo forto i ticolo di'S, Maria

GLIE.

.

 
oT

 

 
   
      
   
 

264 ' - MALMANTILE:
GLIENE feaglia, Gliene tira da lontano; Glien' av
ra nen segli volle accoftare.,
HAVEKIA mangiato Salerno, Haurebbe mangiato i fat, Ve
disse; fume rabida. E si trova Satylnm voraret, che batylum chi

piecra, che si divord Saturno, sDEDy
SER faccents, Si dice: Ser faccenti, o Barbaffori quafi Valvafori,
dajc,a coloro, che tutte le cose fanno, ¢ dicono magiftralmente, ¢
degli altri; E' però detto (cherzofo, ¢ per burlareuno. Qui intende:
natori dell' Inferno. E' parola derivata dail' anti.o verbo /accio, per,

   
    
    
  
       
     
 
    
    
    
   
    
   
  
 
     
 

 

  

  

Lopio'g > vA
PER il mal gaverno, Per il poco mangiare, che gli danno... Nell
Governare le gailine; cioé dar loro da mangiare. Similmente i La
soidati pigliavano un poco di rinfresco,dicevano; corpora curare,
Governare gli uliyi disse Pier Vettori,cioè concimargli;quafi questo sia
Si servtto che tien ! anima co' denti, Si macilente, ¢ magro y.che p
lerebhe ? anima, se non la riteneflc con lo stringer i denti. Giobbe pe
se medesimo emaciato, ¢ confunto. Pelli meae, confumptis carnibus y
meu, ce 3
EGLI è ofa, ¢ pelle. Non ha carne addosso: E' magrissimo, Plauto d
questo proposito Offa, atque pellis. E Dant. Purg. C.23. dice;
Wegli occhi era ciascuna oscura,@ CAVA s
Pallida nella faccia, e tanto scema y
Che dal! offa ta pelie s* informava,
SPENTO, S intende al maggior fegno magro.
LA fame lo scanna, Muore di fame. Vedi sopra C, 4. stan. 24.
CANNA., Intendi la canna della gola, la quale si dice canna per.
ne, che ha il gargarozzo con la canna, Dan, Inf,C, 28, f
Reftato a riguardar per meraviglia she, *
Con gli altri, mnanzé agli aleri apr} la canna s cnt
Onde Scannare, sgozzare: tracannare., ingollare, pha
GLI wiene il fonno in cocca, Cioè nell' eftremita delle palpebre 5 che vengonos
chiuderfi, Gli vien voglia grandissima di dormire. one
S* ALLVNG A in terra, Si distende in terra, Lmmania terga refeluit Fufus hima
totoque ingens extenditur antro; dice Verg. com' habbiamo accennato sopra.
A FAR (a nanna, A doimite. Termine infegnato dalle Balie it
imparano a parlare, per efier più facile a dir nanna, che dormire, Lala N.
Non lajcio mai certi detti che haveva imparato da bambino,chiamando pappo il
vino bombo, i quattrini dindi, e quando voleva andare a dormire, diceva
ta nanna. | Launi fimimente !'addormentarsi de'bambini alla Ninna
tilena delle Balie, da lor detta Ladus, ¢ da' Greci Wynnini; dicevano La
MENT RE il fonnifero fa il uo corso, Ul fonnifero fa la sua Operazione +
PAP AVERO,¢ Loglio, li papavero è quell' erba, il feme, ed eftratto
quale compone |' oppio, 0 fonnifero; ed ii loglio € un' erba, che nasce si
ni, il feme della quale mangiandolo, dicono, che faccia sbalordire, ¢
no. £ da questi mali effetti del Loglio habbiamo un proverbio, che

  

Lad

   

 

       
 

 

 
  

oo CANTARE?.
significa Io non fon balordo.
sopra C,3.stan. 32. /draiarsi è il verbo recumbere; E Ver-
u pacule recubans fub tegmine fagi; stimo che intenda sdra~
te ne stai all! ombra d' uno spaziofo faggio. E nota,
, che vuol dir largo, 0 spaziofo, ¢€ staco cavato il verbo
I eas » ¢ paffare il tempo senza pensieri, il che chia-

tifmo afiai usato. *
. Ronfare: Quel romore, che si fa da molti nel re(pirare dormen-:

  

 

   

ae;
da bette. Vuol dire accoftarsi. Perché le doghe, ¢ l'altre parti
'botte fon lavorate in modo, che si compaginano,, ed unilcono,

STANZA XXVIIL.
S' ell è come voi dite a questo modo:
(Si le risponde ) andate pur madonna,
Perch' altrimenti c* entrerebbe il frudo,
E voi fharefti in gogna alla colonna,
Horsit correte pria che freddi il brodo

 

   
  
 
   
   
 
     
 

dice) ola, che roba ¢ quella?
i ( dic' ella) nel forame,

   

non ho qui roba da gabella, Che la Regina poi farebbe donna
— Se non un po a' alors' a Proferpina Da farci per la frizza, ¢ pel rovello
Porto, perch' ella fa la gelatina. Buttar' a' più la forma del cappella,

za havea forto alcune rame d'alloro; ¢ da i gabellieri le fu doman-
+ ma essa con dire, che era per servizio di Proferpina, si libera,
a del gabellicre. 11 Poeta imita Vergilio, il quale fa che Enea d'or-

Sibilla porti a Proferpina il ramo di quell' albero con le foglie d' oro,
Come si vede al lib. 6, dell' Enetde.

—

 
  
  
   
       
 
  

 

Latet arbore opaca
$ Aureus, & folijs,& lento vimine ramus
iis Iunini Tnferna dittus facer,
HELL. Quella cosa mala; cio' la (pia.
> (A della fame. Ha grandiffiima fame, perché non guadagna denari
omprar roba per mangiare. Quando i meftieri non lavorano si dice: i /egna-
, alzolai, ec, arrabbian della fame, cioé non hanno da lavorare.
erai il forame. Per befiar' uno, che dandofi a creder d' haver fatto
gno a [pele, ¢ dispetto nostro, e non tha facto, diciamo: Tx ti
. Qui vuol dire: tu credevi di haver guadagnato il quarto, che
sca alle spie, ma non ¢ flato vero.
»1W.A, Fu figiivola di Giove, e di Cerere, 1a quale fingono gli an-
» che efiendo un giorno a corre i fiori,futfe rapita da Plutone Re del?
» 5 fatta sua moglie: Ma Cerere non potendo comportare, che la figliuo-
imanefle apprefio ai rattore; supplicd Giove, che volefic Jevarla dall'lnterno,
eae le, pur che ella non haveffe prefo cibo alcuno; Ma havendo
pina mangiato alcuni granelli di Melagrai:a non potette ulcire; Cerere di
0 supplicò, ¢ stimolò tanto Giove, che ottenne, che Prosperina steile fei
'dell' anno nell' Inferno con Pintone, ¢ fei mefi con la madre in Cielo. EB
. con

 

     

   
   

  

 

 
 

 

366 MALMANTILE ©

     

così Praferpina ref sei mefi in Cielo, dove ¢ chiamata Luna', @ fel

  
  

ferno, dove ¢ chiamata Proferpina, ed in Terra è chiamata Diana,
£

triplicata efienza Verg. disse:

Tergeminamque Hecaten, tria Virginit ora Diane,
E perché la Luna fei mefi dell' anno cresce, ¢ fei mei feema, perd'i
till finfono, che ella steffe sci mefi in Cielo, € fei mefi nell' Inferno,

  

 

+ teeth die

goers

 

no splenda in Terra, ed ¢ detta Diana. A questa finzione allude Dany

Ada mon cinquanta volte sia raccefa
La facia dela Donna che qus regge ¢
GELATINA, Brodo fatto con la carne di porco, € rapprelo; ¢ si fa
' i)

     

brodo di pesce. Vedi sopra C. 2. flan.

2. stan. 15.

STIZZA, Ira, Vedi sopra C. 2, stan. 78, al termine fw piccine, Era
velo, collora, ¢ simili, i potiono dir finonimi di stizza, quando è prefa in
fenfo; che per altro diciamo fizza Vna specie di lebbra, che vien€

ad altre beftie.

S:AREBBE derma. Questo termine significa Haurebbe animo: Si farebbe
to, ardirebbe, non la guarderebbe, ed ha lo steffo significato, che Son | i

detto sopra C, 4, stan. 29.

BVTT AR 4: pie la forma del cappello, Cio' buttar 1a testa a i piedi j) Teoncare

il capo, che è la forma del Cappello.
STANZA XXIK

La Maga senza dir piit da vantaggio,
Metr'egli a/petta un po di miacia,e intuona;
Ripiglia prontamente il suo viaggio
E incontra Nepo gid da Galatrona,
C” havendo dato Id di se buon faggio,

Jn ongi ¢ favoritoye per la buona,

Perché Breuffe in oltre a' premi, ¢ lode

L: ha di pin fatto Diavolo a due code,
s tA NZA XXX,

Hor che gli arriva ail' improvifo addosso
4 venir della maga ch'e il [uo cuore,
Lui Mago pur tagliatole a fus doffo
Le [pedisce per suo trattenitore,
Alentr'il petardo col cannon pi groffo
Sentefi fargli frrepitofo onore,
Cavalier Nepo, com' io diffi dianzi,
Col riverirla se gis affaccia innanzi,

C'ENTREREBBE il frodo, Ci farebbe la pena d' haver frodata;
nifeftata la roba, per non pagare il dazio, 0 gabella. |
LN gogna. Alla berlina, che ¢ quel gaftigo vituperoso, che dicemhmo:

   

veh a

     
    
 
  
   
   
  
   
 
 
    
   
 
   
   
   

ira

its

STANZA XXXL
E perché 4 Benevento eff com luk,
Com! ¢i di lei,bavuto havea
Won prima si riveggon ch' D
Rifanno il parentado,e 0 a
Tra i diavoli poi vin nei '
E percht Martinazza v' ¢ novitidy
E non intende il gracidar che @ fam,
L interprete fa egli, ¢ il torcimanits
STANZA XXXIL
Per via informa, ele da molti
Diufanza,¢ lyoghi,e intanto di bua es
Lo guida ai fortumati Campi Elifis
Dove si mangta,e beve hi we
E tra quei rofolaceé, € fior
Si peat tong far diquatcroeaaty

Chi un baloccose chi ut z
Che li non ¢ un negorio per | if

 
 
  

     
 
 
     
 
 
     
      
  
 
 
      
 
    
 
  
   

CANTARE

PV paar ar XXXIV.
 Quivi fifa al palione,¢ alla pilerta,
| Parte ne ginoca al Suffise alle Murelle,
Conte carie a Primiera un'alerafroca
A confartini ginoca,¢ le ciambelle,
etri fanno a Ci è alla lotta;
indovinells.,¢ chi novelle; (gio
7 lie fiorienn'altrounramo aun fag-
ths cagliata, ¢cou esse canta maggio.
A XMXXV,
Altri pigha, 0 dispenfa del tabacco
Altre piglia le mosche, un' altro grilli
E cates quanti in quei traftulli immer se
Sirengonail tenor y(t vanno aiversi,
Za iL Ȣ-s?incontrd ia Nepo da Galatrona molto
o da Plutone', il quale per fare onore a Martinazza da Jui tantoramata.,
i ptrartenitore 5 apendo cheerano amici-, Così dunque.ac-
a Nepo sche de faceva J' interprese, perché ella non intendeva il
voli, se ne pafsd ne i Regai-bui; edi) primo Juogo, che ved-
; ono. Campi Gli;,¢ quati il Poeca descrive ripieai di quei trattenimen-
; ¢ fanciuileschi., che (on-soliti facli dai bottegai pili vili per le feftivica
1 fubucbani,come sono le Ville degli Strozzi, Pucci, ¢ Gerini, doves
f pola per godere allegramente, ¢ seaz' un peofiero al mondo quel-
fa, che concede la campagna-, ¢-sospendere alquaato i pensicri aviofi del 1

|, | hvorare

wy

  

 
 
   
 
   

  

  

IL AMANCLA. Vedi sopra C, 2, an, 68.
bh ANTVON ARE, Vuol dive dac priacipio.al canto; Ma qui significa chiedere
y/ —$ON inetti, © cennila.maacia;.¢ ci serve per intendere domandare con ceani, o
o i quaifivogiia cola: per efempio: Ll talc insuona, vorrebbe andar' a cena,
a ta bostega, ec.
o 'aiatrova', #uvuno nel contado di Galatrona luogo nel Valdarno di
gt »9. conpolueci fimpatiche, o.con altro medicava tutte le ferite, ¢ 1
yl} buomini,, come di beflic,senza vedere il paziente ma folo'ia fu le» \
y@ = pezzebagnate nel faaguc di ello, o sopra ua panuo, che havetfe toccato lo Rrop~
IL PiO} per le-betic in qualfivoglialor malore, pigliava la Joro cavezza', o bri-
ge ia » ¢ sopra quelli diceva alcune parole, ¢ le medicava; ¢ per que-
'i sua lica superflizione da molti fu stimato stregone, come lo stima il Poe~
i seers. ioe eonofeiuto con Martinazza a Benevento, ¢ che era mago
afuo-doflo.
è DAR bran fore di se.Pacfi conolcere con le sue azioni-per huomo di garbo,.
wt = © prudenre, o-virtuolo.
eo 'ER (a-buona,. S? intende,t per la-buona: firada;\¢ vuol dire.. B' in\ buono
yf Mato; Gitira innanzi bene..

|, BREPSSE.,. Intende Plucone; ed & Jo steffo',.che la: Bilisrfa y colla' qual voce»
inno paura le Balic a' bambini, furfe dal Lat. &rebus, originato così; Erehufe, y
le. Me uv Lies

—S

 

 
 

 

   
 
  
      
    
   
    
   
  
   
    
   
   
       
   

# I er
oR

oMALMAN ore *

 

268

   

addosso una' lucertola 'con'due code sia fortunatissimo in:
larmente nel gitloco, ¢'percid vuol dire, che questo' 0 atiffi
grandemente privilegiato da Plutones percht haveva le due code }! =)
GLl wrriva addofo, Cio' sopraggiunge inaspettatamente a Plutone''t
Martinazza tanto amata da lui.) > nImAS a8: /
T AGLIAT OLE #'fuo dof, Fatto per appunto come lei, che havi mede
nj, ed inclinazioni, che ha lei. Traslato da' gli abiti 5 che si dicono sagtiati 4
doffe quando tornano bene in doflo. e 2S arand on
TRATTENITORE, Si dice quel Cortigiano, che vien di a ferniret
Ambatciatore, 0 altro foreftiero, che sia ricevuto, ¢ spefato'dalla Cortew >
PET ARDO, Specie d' artiglieria nota, che serve per'buteare a terra 5 1¢ por
te delle Città. In Latino fa detea da Famiano Strada con' voce Greta
Pyloclaptrum, Quafi Spezzaporta. op i» 2001y(4 ob oats
RIE ANNO il parentado, et amicizia, Quando due amici flati 'lango!
Jontani l'uno dal' altro senza vederfi, si ritrovano insieme, ¢ fanno le:
diciamo; Rifare il parentado, ¢ l'amicixia, NOVICE Tt
VB novizia, Non v'é pratica, perché non v' è mai fata in qu
hospes, € noi lo traslatiamo ad uno, cheé nuovo, ¢ non praticato in
affare. Lat. nonus, rudis, i ages oe
GRACIDARE, E' proptio delle ranocchie,'ma'qui intende il
voli, che forse se lo figura come quello delle ranocchie'.' Dan. Inf.
E come a gracidar si sia la rana,
INTERPRETE, ¢ Turcimanno. Si poslono dit finonimi,se non che Znterprete¢
propriamente quello, che esplica i fenfi delle parole, ¢ Turcimanno & ¢
parla in vece di coluj, che non intende il hnguaggio,riportando le parole; che»
fente dire nella lingua dell' uno, ¢ dell' altro respettivamente, Da alcuni dicefi
Dragomanno dalla voce Greca Dragomenos, che significa Znterprete usata da' Greti
Orientali de* tempi baffi; da Tbargum, che in Levante significa interprerazione,
Thirghem in Caldeo vale ¢/porre jesplcare, e da questa radice ¢ detta a
Thargum la Parafrafi Caldea della Scrittura. Ma hoggi Turcimanne da i pill q
tende ruffiano da quel portare le parole. BH
DI buon trotro. Paaiainaids di buon paffo. Trotto diciamo una specied' an-
dare del Cavallo, che è fra il paffo ordinario, ed il correre, ed è il latino ie:
cajare. =.
eal Elifi., Bil creduto Paradifo de i Gentili. Vedi sopra C,2. stan. 68:
e4 BERTOLOTTO. Senza pensare al pagamento, che si dice anche 4 Vf
a Youne; a Scrocco; a Salicone, Vedi sopra C, 1ftan.77, ¢ forto C, 7, stan. 5.
ROSOLACCH, e foralife. Specie di vilissimi fiori Aloette A
PAR di quattro, ed' orto, Seiben par-che voglia dire giuocare inuitando di
quattro, ¢ d' otto; tutra via s' intende fMarfene senza far nulla, che si dice::
'ar ateco mec, dondolarfel4, farea tn me gli hai, ondg un nostro Poeta moderno —

 
  
 
 
   

    
    

  

  

   

       
    
 

  

H ae '
SESTO CANTARE:

ative. errno al mattarin crepuscolo
Weddin — me gli bai,
nathan 'oponete, a concludete mai,ec,
oe vine Trattenimento «Da Badalueco, che vol dire pro-
: ia.» © leggicre combattimento. Latino velitario, ¢ figurata-
» © trattenimento piacevole. Ma la parola balocco, © balocarsi &
bambini; e nel contado è prefo per indugiare.
' grandifiimo, quafi dica spaziofo tanto quanto un' occhio ¢
pO +
UETT£,. Diminutivo di mucchio, che vuol dir quantita di cole riftret-
» quafi monticelletti, Latino cumuli  acerni; © Così mucchietti di gente
s d otto, 0 dieci tare riftrette insieme. Dan. lof, C, 27.
| B di Pranceschi fangainofo mucchio
i  » Sorte le branche verdi si ritrova..
pure il mondo in carbonata.. Diventi carbone, ¢ abbruci pure il Mondo,
i, ¢ vadia fottofopra il Mondo.
un faftidio di niente.. Non vuol sentir noia, 0 pigliarsi pensiere
che si vuole, o dibene, o di male,
. Ballare fenz' ordine, 0 regola. Vien forse da Ballunchiare
»» chefe bene è parola non usata,pur |'usd il Boccaccio Nou. 72. pe:
ballo di contadini.:
Strozzini.Gii Strozzini,come habbiamo d., una villa de'SS.Strozzi po.
a da Firenze,così detta.Si come.il Prato del Pucci, ¢ del Gerini sono due
aburbane.de' SS. Marchefi Pucci, ¢ Gerini; a' quali luoghi, fuole l'eftates
plebe Fiorentina'a spaflarsi, con far merende, balli, ed altro, che le tor-
o,come dice il Poeta nelle presenti Ottave.
pallone  ¢ alla pillotta, 1) pallone ¢ una groffa palla da giuocare fata di
€tipiena di vento, alla quale si di con il braccio armato d' un bracciale
nO: ela pillotta ¢ una palla piccola pure ripiena di vento, ¢ se le da con
a di legno. Quefii giuochi di palla, sono antichi, perché fecondo Pli-
% 59. furono troyati da un certo Pytho. Herodoto lib,.1, riportato da
slid. Verg. lib. 2. cap. 13. dice, che l'inventafiero i Lidi. Alea verd teffe-
» farumgue ludos, & pila, cateraque luforia recreandi animt gratia inventa,
a» preter quam talaria, Lydi populi Afi omnium primi, cxcogitavere &c. Ac-
-» qui Lydos ciufmodi aleatorias artes non tam voluptatis, quam.compendij,gra-
» Ua excogitafic idem Herodotus tradit, nam cum gravitate annone patria tem.
» pore Atydis Manis Regis filij premeretur, fic famem confolari foiebant, alte-
$9 £0 quidem die cibum fumentes, altero ludis operam dantes; atque hoc modo
, -% inediam folantes vixere annis duodeuiginti, E da' popoli Lyds alcuni voglio-
D0, siccome è Ifidoro nelle Origini, che venga la parola Ludus, 0 Ludius, che
lo stesso, che Iftrione. E ognun fa, che i Lidi dal' Afia pafiarono in Italia, ¢
popolarono l'Etruria, ovvero Toscana; E da loro i Latini le cirimonie facce,.
dudi, che si domandavano scenici particolarmente appreero; EB Hifer in lin.
ica, onde ¢ detto Jfrioni significava in Latjno Ledio siccome dice Tito
3 Poi questo nome /udus significante a cae spettacolo attenente, o far.
a. m2 to

 
     
   
    
   
    
   
       
   
     
     
   
     
   

 

 

 
 

 
   
  
     
    
   
    
    

aye MADMANTILBES ¢

to per canfa di religione, si stele a significare: in generale 0
ib 1, ¢ Suida dicono, che Anagallide Gramatica diCorfl i
mento della faltazione a palla,-cioè del gi alla palla at
a Naafica figliuola d' Alcinoo Re di' Corsu'y wolendo fare questa
il vanto d' una tale invenzione a/una faa paefana, & veramente Naufica:
» Del reflo Di

trodotta fola tra ' Eroine da Omeru a givocare alla palla

attribuisce quest' invenzione a' Sicionj, ¢ Hippafo altro Autore citato da,
a'Lacedemoni,come ache tutti gli altri corporali efercizzj-E che-futie mol
to dagli Spartani, o Lacedemoni lo mottra Properzio in quel vero
veloct fallit per brachia iatiu, delY Elegia che cominicia,) Atuita tna\, t
vamur inra paleftre, Dal che si viene in chiar, che il giuoco della;palla sia ante
chiflimo 5 ¢ si pud credere col Soutero de Jud, Veterum libs 3) Ci 14. e! id,
Verg. lib. 2. cap. 13. che questa'variazione d' origini proceda dall'havere havuto
gli antichi diverse ipecie di-paila, i come habbiamo noi', ¢ che gli accennati ia-
ventori habbiano ciascuno 'inventata la sua species perché:se noi habbiamoiil pale
» lone; i Latini havevano, ipfe follis, pila, & ipfis genus; conftarque V

3» to inflata. Habbiamo la pillotca', & eff ib follicuius, pilay 6 ipfa parva, &
»» similicer conttat aluca vento inflata. Simile a questae la palia i

in-vece d' efler ripiena di vento,°¢ ripiena di borra'; Ja' qual palla hoggi per lo
pil € usata da i contadini, © questa havevano anche gli antichi ¢ la diceyano 2-

fa paganica, F 3 Syaapedt
Marz. lib. 14. Hacyqna difficilis turget paganica plama y—. bd
Folle minus laxa oft, © minus artta pila; ee °

Habbiamo la palla simile alla bonciana', ma aflai minore ¥ che chiamiamo pala
defina, che pure ? havevano anche \fecondo alewni i Latin, © la dicevano Pila
fixentina, perché forse nel pacle Fiérentino si lavoratiero le miglioris Habbiamo
la palla facta di cenci impentita, che i Latini pure havevano, e- la chiamavand
co' Greci Phannida, 0 vero Harpafium, pesche te ne servivano per far il gvoto
y» da noi detto il Calcio fecondo 1] Sipontino, che dice, Harpattumy pile genus
x» elt; grofhor, quam pila paganica, tenuior, quam follis; E panno fere fit,
aliquando ex pelley lana, «omentove impletur, Non repercutitur, fed cum,
» multi fint Judentes in duas partes'divifi, ita ut utrique &-regione fibi inwicems
oy Oppositi fine, ad fuos quilque transmittere pilam constur y quam aduerlari) co
y» Hantur arripere; Alarpajtum diem a Giseco Aarpayin, quod eftirapere, quia
3» proietam pilam mulu fimul conantur ariipere, fed ob cam causam inuicem
> profternuntur, ve
Marz. lib. 7. ep. 31. Won harpasta vagus pulnerulenta rapis, A aie
Habbiamo la paiia a'corda, che terue per giuocare con'la-raechertasnelle Manze
fabricate per tale effetto; ed'etli havevano pram trigomalemscost decta 'non perché
futie di figura triangolarejma perché era triangolast la stanzajéove conefa
cavano, ¢ per dare a questa pallayfi servivano del rericixd y che & JotieNOy chet
racchetta 5 0 laccherta,come accennammo sopra C.3. tan, 58. -Di questa lacche
ta parla Ovid. lib. 3. 1.4 M
Reticutoque pile leves fundantur aperto, - nt
Nec, mfi quam tollas, ulla movenda pila off.

ss

         

  

 

BB aeFF FEES SEH TF ROPSCFARRP xf RE MASTS SOW HEELS Sees.
SESTO CANTARE.

tebe wit DARE Dawe
tepidum dextra, levaque trigonem.
tichi wfafie la palla-ripiena di borra od' altro pelo, si cava
; tino riportato qui sopra, ¢ dal nome di efa, perché
» che sia decta Pia dal pelo,col quale è ripiena;.se bene altri vo-
wenga dal Greco Pefeo:, ideft equo, perché € di figura sferica, che &
ogni parte, 0 pure ( il che € pil: verilimile) dal pee rh cioé
ibrata, ¢ sbalzaca, ¢ percid anche in Greco, si come in Toscana è det-
Dionifodoro antico gramatico, dove nel tefto deil' Viiflea co-
leggevafi Spheran, col qual nome chiamano i Greci da paila; si di-
i (criveile Patian. come per chiofa, e interpetrazione della voce d'Ome-
questo vien riferito da Euftazio, che sopra quel Poeta il gran comento
, Che i Greci ancora haveflero motte specie di palle,si pud dedurre non folo
cfere stati inucatati i giuochi di palia nel tempo, che fiorivano i Greci, es
| flo di oro la Spheromachia, |' Amilla, ed altre specie di giua-
ileriti da Giulio Polluce, ¢ dal #ulengero; ma da quello., che scrive
ino lib, 20, C. 14. dove dice, che fra i Greci giuocavano alla pallas
huomini, che le donne; ¢ cid cava da Homero. Si trova in oltre, che
'Siracufano giuocava alla palla, ed alla pillotta per ricuperar le forze.
ex ab Alex. dier. gen. lib. 3... 21. £ si pud credere, che si come noi habbiamo
'diverse palle, e\diverty modi di giuocar con esse, così non mancaffero a Joro an-
“coral invenzioni per soddisfart.
| AL foffi,£! wn givoco solito farGi per lo più da ragazz' in questa maniera. S'uni-
-eono dues più ragazzi,¢ pigliano una pietra, € polatala per ritto in terra vi
c ra quel danaro, che fon conucnuti di givocare, ed allontanatifi in,
liftanza, che sono-d'accordo, tirano una jaftra per uno ordinatamenie
q i¢tra ritta;sopr'alla quale sono i denari,¢ che si chiama il Suffije se que.
A 'Wien colpico, ¢ fatto cadere,i danari, che cascano,sono di colui, la lattra.
del quale ha fatto cascare il fut, se però sono più vicini alla sua Jaftra, che al
- fal moneta, cheé pili vicina al fufli,te gli rimette sopra., ¢ quello a,
cui) ira, ¢ seguitano,come sopra, tanto \che la moneta mefia sopra al ful
“tei finita'di ievare ne) modo, che sé detto. Da questo giuoco: habbiamo un,
Proverbio che dice Efer w/usfi, il che significa efier queliberzagiio, dove ognuno
“tira,cio€ sopra il quale devon cadere tutte le burle,-c tore le minchionature..
 Quetto giuoco & forse lo stesso, che da' Greci era detto Epbedri/mo, feconge Giu-
Tio Polluce, il Buieng. c. 48., ed il Meurs, de lud. Graecor,, te bene non ginoca-
~Vano denari, ma colui, che non butcava in terra ai futh,portava a cavalluccio
» quello, che to bucrava,il quale gli turava gli occhi colle mani, finche (enza errare
bb portale alla Jafira, 0 pietra, che si chiamava diores, cloe Adeta 0 Confine,es
-facevarquello, che comandava il vincitore, il quale in questi loro giuoch era,
hiamato'Re, ed il perditore era detto Mida, 0. vero Afino, come habbiamo vi-
“flo aitrove.
 | MPRELLE. E? giuoco simile alle pallottole, se non che in vece di palle ado-
oo eee laftrucce, ed un piccolo faflo per grillo, ¢ tal ginoco si dice anche pia-
eae wee «®

 
 

27 MALMANTILE? 4

PRIMIERA, Giuoco noto, che fifa con le carte. = =)
FROTT A, Flotta,, o fiotta. Vuol dire quantita di gente unite i

    

muove; dal Latino fluttus, Virg.Georg. dane Salucantum toris vomit edibus andi,

Varchi Stor.lib. 15. Z vedendo sopra a un monticello non molto
frotta di. contadins. DEIN
CIAMBELLE, ¢ confortini, Sono specie di pafie fatte col zucchero.
uova,e queste fon poe a vender da alcuni pi pel contado, dove si
¢ raddotti, che in Città; ¢ questi portan feco anche le carte per giocare,:
quali hanno diverse invenzioni di ginochi, come la mora, il tocco, ec. E
venditori quando giuocano, danno in vece di danari quei confortin' 5 ¢ cis
se perdono; ¢ se vincono,ricevono danari. L, circali, ernfiula. /
CIVETT-A, Quel giuoco fanciulle(cho, che dicemmo sopra C. 2, stan. 41.
INDOVINELL1, Latino griphi,enigmata;Quello che in latino dal greco i
enigma, noi circoscrivendolo diremmo detto oscure, ¢ diffeile a i
E la voce enigma s'¢ fatta Toscana, ¢ |' usiamo come l'usd il Malatefti nellay
sua Sfinge, Vedi eras 8. stan. 26, a vege!
CANT A Haggio, Nel principio di Maggio fogliono le Ragazze plebes
di Firenze, o del Contado sbeceane scomeni foe Pr) eee ¢
di joro-in mano un ramo d' albero adornato di fiori andar cantando ere
diverse canzonette per l'allegria del nuovo maggio, ¢ per buscar mance da
ro, che si pigliano i) paflacempo di farle cantare al fuono d' uno strumento.
cembolo, che è un' afficella ridotta in cerchio, ¢ fondata di cartapecora da una
parte fola a guisa di tamburo. Questo coftume di rallegrarsi il Maggio viene dal'
antico, ¢ si trova, che appreffo i Romani Xalendis 5 Nowis, @ Zdsbus maij Lari
Deo facra ficbant afello panibus coronato, Quindi forse ancora Maggio si chiamavdl
mefe de gli Afini, che per altro fu detto men/is hilaritatis, Che nel mefedi Mag:
gio si faceffero allegric forse pil di quello, che comportafie ' onefta, ¢ lavert-
condia, ne fanno fede gl'Imperatori Arcadio, ¢ Onorio nella loro Costituzion
inferita. da Giustiniano nel Codice lib. 11. 45. de maiuma, la quale era una alle
gria, che si faceva per il Maggio fecondo che spiega Suida. Da questo mele quel
ramo d albero, che i contadini piantano la notte di Calen di Maggio avantiall
uicio. delle loro innamorate, si chiama Adaio; Questo coftume d? appiceare:
maio alla casa della Dama ¢ riferito come proprio anche della Francia da Mat
ziale d' Aluergna ne' suoi Arrefti d' Amore, all! Arrefto quinto, il quale scritt
re fiori nel 1400. Qual Inogo Benedetto Curzio comentando dice; Prima d
Maij menfis invenes pluribus lndis, ac iocis fefe exercere confueverunt, arborem fapean
mero deportantes, ac in loco publico, aut etiam aute alicnius egreci virt januam 5
frequensiits amica fores pl vel promifinis adamantil
Signijs, atque emblemucibus.

isarote
BRANCO., Quantita di popolo indeterminata; ma si dice pili di beftie; com?

branchi di polli di pecore,di buoi, di aGni, ec, Vedi in questo C.  Ortawa
seguente. 4 sacl
HA mofo} Ofte a facca. Cioè mangiato, ¢ bevuto quanto I Ofte vi haveva-»
nel modo » ¢ con quella furia, che segue nel dare il sacco a una Cittas ae
sopra:

(EN-

MEZZL brilli, AMez2i briachi, Brillo vuol dir briaco allegro. Vedi &
2, stan. 69. At

 
  
   
  
  
  

  

Bee coc eee e2@Heekse ewer es) =!

 

eseF SF AS Soa

 

 
   

273
a bacco, Vna villantella che si canta per incitare
Aidlors eg MeL gave

»\\) Faceiam brindis a bacco, © vi:
questa,va il bicchiere attorno, ed ognuno' beve,intuonando prima.
però dice mentre ice 3 cioé mentre il bicchiere va a tor-
/perché tal-coftume è usatissimo in simili allegrie, però il Poeta, che s' in-
moftrar, che quivi si sta in felte, ¢ in giuoco, dice che facevano brindis
jot cantavano'bevendo. I Latini dicevano Propinare, cioè prebibere dal
tim, che fuona lo stesso che il far brindis, ed usavano anchreffi questo
bere in giro', che dicevano ia orbem bibere, & circumferebant feyphums
¢d essi pure cantavano in tale occasione di bere; come scrive Dione, che
e Roi aC l'quando al banch che fece
bevve a un bicchiere, che li fu porto da una bella femmina.
'brindsfi. Se ben pare che venga dal Tedesco pringen, percht volendo
) a*nazione bere, 'ed inuitare il compagno,fuol dire: Zk Mellan-
'y che vuol dire' /o ve lo presento; ¢ questo già facevano, perché quel vino,
havevano'a bere reftaffe benedetto dal Compagao, il quale foleva rispondere
nges, che vuol dire Dio lo benedica. Tuttavia il Lalli nella sua Moscheide
“61. graziofamente gli da origine dalla Città di Brindis, dove chi va ad
f è da ogni veflazione curiale tanto Criminale, che Civile, onde a.
 faevbrindifi par che sinviti uno ad andar ad abitar quella Cited, cioé a lasciar a
¥ 'parole del Lalli fon-queste: i
|. \Brindifi bella s io m appongo al vero,
Date fon meffi i brindifi in usanza,
Quafil buom dica; Lascia ogni pensiero;
Beviamo allegri, ¢ rinfreschiam la panga 5
E se pui il creditor duro ye fevero
Ci fa da' birri apparecobiar la franzAy
Brindifi habbiamo, Brindifi diletta,
Che quanto pik si bee, vie pss n' alletta;
paglie,o /pilli. E' un giuoco da' fanciulli, che si fa così: Pigliane
due corte fila di paglia, ¢ posandole sopra un piano liscio vanno
'4spingendole con le dita tanto, che uno di detti spilli, 0 fli cavalchi l'altro,e quel-
Jo, che refta di sopra vince, giuoco così detto dal Ter?, cio' tog/i, regi. In La-
tind tudere aciculis, E perché questo ginoco-¢ di niuna, 0 poca conchiufione, hab-
- biamoil proverbio s Fare 4 tes? con gti spillerti.; che significa affaticarG, ¢ per-
'Mere il tempo fenz' utile, © profitto; ed esprime ancora Far una cofacon fordido

    
    
  
    

   
   

 
 
 
 
 
  
    
 
  
    

       
    

S' aiutano l'un I altro, e's' accordano.
Ww XXXVI.

me ZA
| “Za donna refia litrafecolata,. Per tutta la Cutta vien falutata,
©) Fedendo quanto bene ognun si spalfa; E infinle stanghe,e ogni forcon s'abbaffa,
i he Nepo I ha di gid infor mata, Ed ella bor qua,bur 1a voltando inchini,
ragiona di lorma guarda,e pala: Pare nna bandernola da cammini.

STAN:

ST vengone il tenor, si vanno a' versi.
* STAN

  

 
 
   
 
  

 
 

   
   
 
   
   
 
 
   
 
 
  
 
   
 
 
 
 

a MALMANTILE::
STANZAXXXVAL. STANZA KEI.
è

Pera che tutti quanti queit Demoni y Percio pafjano in casa ©

Per vederla, n'uscian di quelle grote, Firatoconta Stree il Reda
'Ronzando con un brance di moscioni y. Le da la ben vennta,e vento
Che saggirin d'actorno.a una-botte 5 dt Le ipieaie sete: nar ds
Saleellam per de firade,e fui balconi y Elia per confeguir ogni (uo jutento:
Comal piover a' agofto fan le borte, hf "

Sees

     

   
 

E fan, vedendo [ue fembianze belie,, bails
Voei altese fiochese fuon di man con elle, -grazia anchiet di dar i
STANZA XXXVIIL, STANZA KXKK Og,
Così fra quel diabolico rombarzoa Sta pur, dic' ey cont anime (ato y g
La fhrega se ne va con lo firegone » C' a servirts mo mo vuo dar di piglia,,
Sic alla fine arrivanoa Palayro Jo.gid,come tu fai, baveo impranate;,
La-dove s' abboccaron con Plucone, Ma il tutco.d andato poi in iscompigha

   
 

Ma. perché tra di laro entré-nel mare Horfu: fra poco adsunerdil fensta y. ¢
Scwwccamente il eALandragora buffone y E sopra quetto si farò confizka y.0
Chiin quel cailoquio fesigran fraftuono, eAicio Baldon batta ig ritirates,
Che finalmente ognuno uscs dt tnono-, E tu reftt consenta,, econfolata,. >

Martinazza refta maravigliata, che coftoro stieno cos: allegramente 5 ¢ pak
fando pel mezzo a una infinita di Demonj, che-cutti la riveralcono., giunle coms

Nepo a Palazzo, dove se le fece incoptro Plutone, che la condutie dentro\,,e>

quivi havendole essa detto il suo bifogno, Plutone se peomess di confolarla..

REST A trafecolara, Refta.maravigliata: Strabilisce. Vedi sopraC, 1, st. 28.
ST ANG A. Pezz0 di travicelio,.cioé ua legno groffo.pii d' un baftone.
FOKCONE.. E'un' afta'di legno sopra, alla quale ¢ adattato un tridente di

ferro, ¢ serve per uso delle Malle.

INC HINO. Vedi sopra C..1, stan. 34:

BANDERVOL A da Cammini, Bandecuola vuol dir piccola bandiera, o pen

noncello, che è quel pezzetto.di drappo, che già portavano:j Cavalleggieri appie-

cato vicino alla punta dellatancia a guisa di bandiera; ed a guisa di questa ims

Firenze se.ne vedono fatte,.di lama di ferro potte in piu. emincati luoghi delles

cafe, come (ono le pergamene, dond' esce il fumo dei cammini, equelte servo

mo-per far conoscere i venti col lor girare, € voltard in ful. ferro: y.nel quale sono
jnfilate, ¢ bilicate; ed a quette allomiglia Martinazza, af

RUNZANDO. Ronzare si dice propriamente delle mosche; ¢ però.dice Ce

me fanno i moscioni, che sono-quei piccoli- vermi alati, che nascono-dal vino.
COME fanno le botte-al piover a' Agosto.. S'¢ veduto dalla sperienaa, che la piog-

già di state,cascando nella poluere scaldata dal Sole inuigorisce le rane,.o bore
nate di poco,,se bene molti. hanno creduto, che le faccia nascere quell" acqua,coo

guel Sole; il che è falfo, perché prefe (ubito scappate dalla poluere si fon trovate

 
   
 
 
  
 
   
  
 
 
   
 
   
 
   
 
  
    

     
     
 
   
 
   
 

 

Exé SE vo ESE ZZ mERZ SHE aes TLS ES

 
 

 
     

col ventricolo pieno-d' erba,. Ma sia come si voglia,basta che atal”acqua

gono-faliar, ma:d'.un salto debole,.¢ fico, spon come il Poeta: sesptl-
“mere yche faltaflero quei Diavoli. Va Posta faccto Piorcntinodel¢rivend a
* sual. cavadi Mancii-ia. unsuo Sonetta-dice:; i

  

Seae
 

H SESTO CANTARE. 275

    

ro Si fivergognan che paffan di norte,
seat oe oe ae feet
Pe " sides \ Trottando, ¢ faltellando come bette,
OCT alte, ¢ foche, e fuon di man con elle Così canto Dante Taf. C. 3.
Ke @”. Intendi frida, e'colui, che continova 'a gridare,afhoca per l'affati-
"cam 10 dell" aspera artetia, si che il fecondo nasce dal primo. E /uon di man,
im elle'; cide con quelle voci accompagnano il romore; che fanno co: batter le»

| “ROMBAZZO, Rombazzo vien dal verbo rombare, che vuol dir ronzare,o frul-
Tare, che @ quel romore, che fa perl' aria una cosa lanciata con yiolenza, ¢ si
'Piglia per ogni forta di Arepito', o fracasso. I Varchi Stor.lib. 10. in questo me-
¢ signiticato dice bombagze voce formata dal fnono,nella stetia manicra, che
; )formd bembus: Torma eAimatloness implernnt cornwa bombas, perché dice
lunge prombertare, ¢ spampite fatve con invredibile Bombazzo, quafi in tal
Mn paffero ¢ nimici. Ma |'Autore oon Storia di Semifonte dice al trattato 4,
! atone la T erra;allotra fenritofi per quelli della Città il rombazxu, E V'ulo
puechecl obbiighi adire romibazo v

| le nel mazzo, S'accompagnd con loro, Che diciamo; incrn/carsi, fic»
è ien'dal giudco del mazzolino detco sopra C. 2. stan. 46.

' a apora', Coftui era un buffone, © pili cofto un matto di Corte, che
)  chiacehierava (empre', ¢ senza propotito, o conchiufione.

j KeVIO. Voce latina fata di rado in Firenze,e vuol dire Ragionamen-

  

LSS" CO

 

s

to, che fanno insieme due, 0 più persone. Corrisponde alla Greca Dialogos, che

ifica indo la: parola Znrerlocutio, dilcorlo che si tiene fra due,o pili persone;
dai Pranzefi detto Emtrerien quati Trarrenimento,

VSCH di ruono, Perde¢ il flo del ragionamento,si dice anche:a/cir di tema, fmar-

rite argumento, il proposito. Vedi sopra C. 2. stan. 47. E' pref la similicudine

3 feherzando fal doppio significato della parola scordar/f |a quale tan-

tOVi dice' d* un' huomo, che non si ricordi piti di quel che ha proposto di dire;

'uno stromento, che non sia in corde, ¢ non sia temperato al giutta

O@ vnc, che non canti ginflo, ¢ fuor del legitimo tuono, il che si dice

TIRATISs de banda. CondottiG in un' altra parte della stanza, feparatifi, o
allontanatifi da quel congreflo.
CHE vento? bat [pinta in quelle bande, Qual cagione l'ha moffa a andar in quel
0". S.

 TRABALLARE. E' quell ondeggiamento, che fa uno, quando non pud fo-

: in piedi, Mattio Franzefi in lode della Posta dice,
cd Chi domanda per nome la cavala,

Chr eels ha sentito dir, ch' è favorita
Wy} Poi partendo chi trotra,e chi trabala,
Qui vuol dire, che Malmantile cra in pericolo di cadere, cio' esser prefo da Bal-
done, Diciamo in questo fenfo anche baienare, barcollare, \n certe rime mano-
scritte nella' Libreria di S, Lorenzo, si dice d' un cotto, che barcollava: Es'¢
Yalena, @ non batena a (ecco. Qui si nea ful doppio significato di balenare.
? i a Mo

5 FS = BERS SRE SAS 

   
*

 

 

=

 

     
   
 

276 MALMANTILE ©
40! md, Adeffo adeffo. BE' il latino, omb:
Firenze. L' usd più volte Dante nel suo Poema,si co
re altre parole Lombarde; B il Bocce. Nov. 32. 4@ vi
Jaca della donna, ch'era Veneziana. preys
DRO' di piglio. Dard di mano, cioé comincerd.
ficaya quafi quel, che 1 Latini dissero Expilare; i Franzefi p
dier nel (angue, ¢ nell' aver di piglio, E.'l suo cont ) Fazio.
Poema che fece in terza Rima, ove ¢ introdotto Solino a dettare a.
di ecografia, ¢ del mondo; che percid lo intitold Dita mundi,
ando; dice così al canto 142. ( Parla del Saladino ) i
a Coftui per (ua francherza, ¢ gran consiglio y
Tolfe la terra fanta a' Criftiani
Vincendo quegli, e dando lor di pe lion SNe
FLAVEO imprunato. Havevo ordinato il rimedio, Vien da quell' imp
che dicemmo sopra C, 3. fan, 21, Addio fave. i
HOR sik, Termine esortativo,e conclufivo,e diciamo nello steffo fenfo. 01
quafi Or via, Latino Eia age, Vedi forto C, 12, stan. 47. Diciamo + hor fu
diciamo bac ipfa hora furge,& hoc factas, aga
BATT Ala ritirata, Sene vada da Malmantile, Batter la ritirata
0] tamburo si fa quella fonata, per la quale i soldati intendono doversi
¢ lasciar ! imprefa. Gio, Villani cid disse; fonare (a ritirara; quali
il Franzele, retraite..
STANZA XXXXL STANZA XX
Jo ti ringraxio st, ma non mi place Dico di Gambasporta il tuo v
Percio ( gli rispond' ella ) di manieray
Ch'ionon voglia pivlinr laspada'lgiaco,
Ch in bagnela fon piit di quel ch'io era;
Così con quei due [pirti bavend» ilbaco
Sagginnge (per c' alor vnol far Japera)
to Vho con quei briccon furfanti indegni, Ches'egl adaffe un polafruftain ve
C' hanno frurbato tutti i miei difegni,  Imparerebbon per nn? aitra volts
STANZA XXXXIII,

  
 

 
 
  
     
     
        
     
   
 
 
    
  
 
   
 

   

   
 

E di quel pallerin di Baconero 4
Che ib oo | ginoco con “nt i
Scambiando it color bianco per
Error, che nol farebbe anc' un cava
Ma e'vit che gli firapaxzanoil.

See

 
 

7

  
  
     
  
 
 
  
        
    

as

    
     

a oi

a

Risponde il Re; Facciam quanto ti piace, Non penso di reftar gid contumace y —
Ma ti verranno a chieder perdonanzA, S'io non ti servo,perch'iofo a fidanta. '
Si che ru puoi con offi far la pace Dunque ti Lascio,e fone al to piacere;
Pero racquiera,e vane alla tua fiaza, Fatti servir da questo Cavaliere.,

Martinazza ringrazia Piutone, ¢ dolendofi del danno cagionatoli da G:
florta, e Baconero lo prega a gaftigargli: Plutone I esorta a placarsi,¢
che andranno a chiederle perdono dell' crrore; ¢ fatte con essa sue cirimonie
rimanda alle stanze. a '

WON vogtia pigiar la pada, ¢ ilgiaco, Non voglia armare contro di loro pet
yendicarmi. 4 =i eee
SONO in bugnola, Sono in collera, Bugnola si chiama un' arnele fatto dicot —
doni di paglia, entro al quale si conferua grano, biade, ec, da i Latini dettas
cumera, Bfidice cfler' in byguola, nel bygnolone, in valigia, nel gabbione ee

eS.

  

=

     
 

asget
 
 

   
 

SESTO CANTARE: 277

ee in cOllera. E tutte queste manicre vogliono esprimere il gonfiare,

un fa per l'infiammazione della bile commofla. Orazio Bile wmer iecar; dove
vaveva detto: mexm iecur vrere bilis, Ovidio ne' Balti, Intumuit Luna, cioè

onfidzentro in valigia. Gli Spagnuoli similmente dicono, embotijar/e.

| HAVENDO il baco, Havendo ira. Traslato da i cani, i quali quando hanno

: & n certo baco nella lingua per di forto, par che fieno (empre adirati, ed il simile,

 
 
    

 
 

segue ne i Montoni,quand' hanno il baco, o tarlo dentro alle corna,
'AR la pera. Anticamente s'abbruciavano i corpi morti sopr' ad un montes

 

; = » qual monte quando era accefo, chiamavano ?yra. Lall. En, Tr, lib. 5.
teil
ae: Già l'alta pira di Didone ardea,
pate £ vibrava lontan fiamme,e faville.
'Edda quefio credo, che venga il nostro far /a pera, € che s'intenda anche am-
na 'uno,quafi si dica: 40 vagtio far /a pira al tale, S' intende anche far /a [pia

  
   
   
 

'AR fallo, Far' errore. E' termine del giuoco di palla: ¢ però il Poeta se ne
ue'; perché l'errore fu fatto con le palle. Properzio lib.3. we pila veloces fal-
lit per brachia iattus.

NOL farebie ance un cavallo, Error groffissimo, ¢ che non lo farebbe anche
una beftia; e si dice wn cavailo, perché questo animale pare, che habbia discorso, e
giudizio pi che ogni altro animale. I Greci di sippos, che vuol dire cavallo, se
ne per una particella, che aggiunta a' nomi, importa grandezza. Hip-
pomarathram id @ il finocchio faluatico, ¢ Aippomyrmeces, certe formiche,
che paffano di grandezza l'ordinarie, ¢ comuni. Onde errore,o sproposito da.
cavallié mn' error grande. O pure si dice così, perché sia degno di cavallo, cioé
i  di gaftigo, qual si suol dare nelle scuole a' fanciulli.

 STRAPAZZ ANO il meftiero, Cioé nell' operare, non considerano quel che

 

f — eANDASSE Ia fruspa in volta. Se la frufla andaffe attorno. Se fuffero di quan-

~ doing baftonati, fruftati.

\ NON penso di reftar contumace, Termine di cirimonia, che significa:non penso
di commenter mancamento. La voce contumace ¢ Latina; però il Lettore si pud
soddisfare circa i suoi significati,

FO 4 fidanza, Confido, che per tua cortefia non l'haurai per male, ¢ mi scu-

A ferai; termine usato fra gli amici intrinfechi,¢ fidice anche; Fo 4 ficurta,

 SONO al tno piacere. Termine usato da' superiori con gl inferiori, in yece di

fF aetgeps Coenen, torent N

ia 'avaliero, Intende Nepo,

J STANZA XXXKXIV.

( ses mena allora alle sue spanve, Ove gli orfi facendo alcune danze

f Cha paraméti havean di quoi darn ats Dan la vivanda, e da lavar le mani;
; Ricamati di signoli, e di stianze, Volati al cibo al fin,come gli affori,

| Efepeans di via de Pelacani Sembrano 4 foe fo due toccateri
= Nn 2 STAN.

 

 

 
 

278
STANZA XXXXV.
Fioritaé la tovaglia, ¢ le faluetce
Di verdi pugnitopi, ¢ di froppioni,
Saldate con la pece,e in piega frrette,
Infra le chiappe frate de' Demanj.
Nepo fra tanto a mavinar si mette 5
E cheto cheto fa di gran boccont;
Offeruando Caton, ch' intese il gioco,
uando disse: in connito parla poco.
STANZA XXXXVL
Fa Martinazza un bel menar di mani;
Ma più cheilvitregli occhi al finfi pasce,
E quel pro falie, che fab erba a'cani,
Che il pan le buca, e sloga le ganasce,
Perché refee vi fon come trapans,
Ne manco se ne pro levar con Pasce.
Crudvéilcarnaggio,e si tirante,e duro,
Che non viene apuntareipiedi al muro,

 

Prexioft liputvl seve ne foaé i
Portati ciascheduno in sua guafiada
Essendovi aqua fortes inchifro buss
Di quel proprio, c'adopera to Spada,
Ella che quivi frar voleva in tnano, ~
E non cambiar, partendofi,la firada,
Perché i gran vini al cerebro le danno
Ben ben Vannacqua con agreftojeranno,

STANZA IL

E fatte due tirate da Tedesco
La taxxa butta via subito sm terra;
Lero ch'ell'é di morto un teschio fresco,
Che fuona, e tre di fa n'ando fotterra,

Nepo, che mai alzi vif da defeo,
E intorno.aibuon boccotiratohaaterra;
Anch'egli al fine dato a tutto il

La bocca follevo dal fiero pasto. '

Nepo conduce Martinazza alle sue stanze,dove era imbazidita la menfa, ¢ fu-
bito si mettono a mangiare. L' Autore descrive la qualita de i j dell'
imbandimento, de i trattamenti, € de i cibi, tutto appropriate a uno appartas

miento, ¢ banchetto da Diavoli.

QVOF humani, Pelli a' huomini. Se ben quoio vuol dir pelle di beftia conciat'
si piglia ancora per pelle d' huomo, come s' è veduto sopra C, 4. stan, 20, € coms

lo prefe il Ruspoli dicendo:

Vn certo ch' in full offa ha [ecco il quoio;
SIGNOLI, Specie d' apostema nella cute;da i Medici detti Puruneull.,
STIANZE. Quelle crofte, che fa nella pelle 1a rogna; o altre bolle; dai
Latini dette erafe. Varchi Stor. Fior.lib.1.4, G4 trewarono rofo dello fomaco quant
un giutio con una fianga nera sopr' a quel rofo.
SAPEAN di via de' Pecalani, Puzzavano di beftia morta di più giorni, Las
via de' Pelacani si dice in Firenze quella; dove fon le conce delle pelli, nella»
quale ¢ fempre wn puzzo orrendo cagionatoye dalle conce 5 € dalla corruzione di

quelle carni

VOLAT I al cibé come gli aftori, Entrati a tavola veloceiente. Avventacifi al
cibo, come fa l'afore, il quale, benché habbia il cibo a fuc dominio,vi s\avven-

ta,¢ lo divora con rapacica grandissima
SEMBRANO 4 folo a fol due Toccatori,

 

Dicemnmo sopra C. 2. stan. 66. quel che

* fieno i Toccatori, Questi sono folamente due, e volendo andare a cena all'ofte-
ria fon foreati andar da lor due foli, che le conversazioni de' galaathuomini oon

li

 

ee Fa ESOS ee Oe Ren eee se. Se

 

Rey

 

 
  

 

2
7
e
e

=

>

=

SE Tyr

,
'
j
7
s
f

it

 

RET TE in piega, Le faluctte, ¢ tovaglie si piegano in diverse maniere, ¢
fifa loro pigliare la figura, che si vuole, col tenerle così picgate Mrette in un tor-
0, ofirettoio fatto a posta per tal' effetto, in vece del quale strettoio, guelte
0 state firette fra le natiche de i Demonj; ¢ cid dice. per esprimere, che fons

Deseo
 ANTESE il giuoco... Sapeva,come era conueniente fare, quando disse: Pauca in
loguere.
 FAun bel menar di mani, Si tadia; 8 affatica a mangiare, Vedi sopra C. 1.

LE f il pro, che fal erba a cani, Non le fa pro. Quando i cani mangiano ler.

REST E. Quei fili fottilissimi, che flanno appiccati alla spiga del grano, dell'
orzo, ¢ della fegale; dal Lat, aris.
 TRAPANO., Specie di succhiello, o foratoio atto.a bucar pietre 5 ferro, ed
i altra materia per dura che sia, ¢ s' adopra facendolo girare com una corda,
loi ? habbiamo dai'Greco Trypanon, Vedi sopra C. 4. stan. 73.
NON se ne puo levar cont' ace, E' così duro, che ci vuol l'asce alevarne uns

iT NON viene,a puntar i piedi al muro, Non se ne pud strappare a fare ogni mag-

\ BAR lo spiano a casa a altri, Mangiare a casa d' altri (enza spendere. Vedi

\©, 3. stan. 51. Questo detto viene dallo spiano del grano, che vien dato
dal Magiltearo dell” Abbondanza a i Fornai per fmaltire il vecchio, che si ritrova
hei magazzini pubblici, ¢ da questo rifinimento /pianare, 0 far lo spiano a casa d'
altri intendiamo rifinire, 0 confumare quello, che colui ha di commeftibiic in,

ECASEO barca, e pan Bartolommeo, Precetto della squola de' ghiotti, che vuol
dire Mangiar 1a midolla del cacio, ¢ la corteccia del pane.

 FREMERE. EB' voce latina, che conferua appretio noi lo fieflo significato:
'Verg, 1. Bn, Cuntti fimul ore fremebant, E altrove descrivendo il Furore; fremir
borridus ore cryento,

BRANO, Pezzo dicarne (forse dal Latino membrana ) 0 altro strappato
- violenza, ¢ si dice sbranare; ¢ sbranalo, Vedi lopra C, 2. fan, 52. mandato
abrani.

 CIBREO, Guazzetto fatto di colli, ¢ ventrigli di poli, 2zinueal. Pud essere
Originata questa parola dalla Latina Gigeria. Feito Gramatico: Gigeriaex muitis

js ape.
MAGNANO, Quali machinarius fabbricatore di ferti minuti, ¢ di piccoli ia-
i: Begui

 

 

 
 

    
 
 

280.
gegni,come chiavi, toppe; a distinzione di Fabbro, che

ine Zappe, vanghe, ec, ¢ del Manescalco, che fabbrica ferri

ci¢ i Magnani fon fempre tinti di nero, il Poeta dice che il cil

lero interiori,per esprimere, che era nero. ate sh SOR G
VENT RIGL/O, Ventricolo degli uccelli; in altri luoghi oe theta “i

    
 

 

STRIGOLI. Diciamo quella membrana, 0 rete gratia; che sta
budella degli animali. i

AC QV A alle mule', BE! un detto di gente baffla, che significa date

GVAST ADA, Vafewto di vetro corpacciuto, ¢ col collo lungo, ¢ fire
serve per lo pia tenerui l'acqua per annacquare il vino, quando si beve.
antichi dissero Jngaifara. I) Canini la fa venire dalSiriaco Ga/far, che' v:
ficflo. Potrebbe anche comodamente dedurfi dal Greco Gaffer, che v:
corpo; ¢ così Guaftada esser detca dalla figura corpacciuta, nello: stesso
punto che Graffa voce Siciliana usata dal Boccaccio neile Novelle in ]
te viene, si come molte della Sicilia, dalla Greca Ga/ira, un. poco tt
Iettere; la quale significa as vafo che habbia pancia,

LO Spada, Valerio Spada celeberrimo Macftro di (crivere, huomo si
¢ che non refta addietro a veruno nella galanteria del tratteggiare con
mano, ¢ frappeggiare, ¢ far paefi con la penna; come d' intagliare in
bulino, ¢ acqua forte, fu amicissimo dell' Autore, ¢ suo scolare nel
vive ancora; € ben che d' eta sopra settant' anni,indefeflamente lavora p
nare il suo nome.
VOLENDO star in tuone, Volendo star' in cervello, ¢ noms' imb
CAMBIAR Ia strada, Quando vogliamo dir copertamente a uno, Tu fei bria-
co;diciamo. 7' bai fmarrita la Strada, ¢ pero intende;non si vuole imbriacarey
KANNO. Acqua paffata per cenere, detta anche 4/eia, dal Lat, dixivinm, I
dottitfimo Ferrari nelle Origini della lingua Italiana, dice così; Ranno; lixsvia
Vude vox ortum trabat, omnibus veftigijs indagara battenus fefellit, Chi fa, che non —
si origini dalla voce Greca Xhanis, che significa, Milla, gocciola, perché il sann0
stilla a gocciola a gocciola da que! vafo, che percid diceli Colatos ? (ee

ia del vinos4

    
  
 
   
  
   
 
     
    

SEP FRE SSS pec he es ee:

a:

F,

4 ef ert Fz Ez.6 =F

    
 
 
 
   
   

FATTE due tirate da Tedesco, Fatte due gran bevute. Manda gii
Latini dicono: pocula obducere;i Franz, avaller,

SVONA. Di questo verbo fonare ci serviamo per intender copertamentes

mere.

l'MAL alzé vifa dal desco., Stette empre attento alla roba, che era in tavolas+
Termine usato per intendere uno, che a tavola mangi con avidita, e non pigli div —
vertimento di forta alcuna; E de/co se ben vuol propriamente dire la tavola,dove
si fla a mangiare, onde il dettato: Chi non mangia al desco Ha mangiato di,
oggi ¢ poco inteso per altro, che per quel iegno, sopr' al quale i macellari tagliae
uo la carne, ¢ per quel banco, a) quale nelic Confraternite, 0 compagnie de'
colari ficde il Giovernatore,
TIR ATO ha aterra ai buon bocconi, Ha mangiato assai de' buon bocconi 5
Jo fico, che menar le mani detto sopra. aM

La bocca follevs dat fiero pasto. Verlo di Dante Inf, C, 33. Lascid star dimane
giar quell' orride vivande,. ioe

- STAN-

    
 
 

  
   

   

 
       

  
 

SESTO CANTARE 281

 
     

   
 

on SUBTAN ZA. STANZA LI.
Lasciatii voti, e i piatti scemi Spargon le rame in varia architettara
in anno al giardino pieno di emente Scheretri bianchi, ¢ rosse anotomie,

    

Di berlines di mitere,e di remi, Gi aborst,i moftriei gobbi in fu le mura
Edi firumenti da castrar da gente; Forman spailiere in lnogo di lumic;

       
 

 
   
       
    
      

    

ifede in me2x0 sl paretaio del Nemi Dugna, di denti, e simil' ffatura z

, D'un pergolato,il quale a ogni corrente Lnfeliciate fon tutte le vie;
i con quattro braccia di cavexra, Nun bel fepoteroanicchia il fore butta
. Penoloni, che sono nna bellexra. Del continuo morchia, ¢ colla firutta.

ee STANZA LIL
sono abbroffolite, ¢ feure Sui dadi i torfi nebili feulture

ie del mar venute della rena, (Perch'in rovina il tutto iitempo mena)

'intorno intorno in varie positure Rispaurati sono, ¢ rifarcité

  
  

d | Sep rem leggiadra feena, Da vere, ¢ fresche tefte divanditi,

lito che ro di mangiare, Nepo condufie Martinazza nel giardino. Qui
icipia a descrivere un giardinu da Diavoli moftrandolo ripieno di tutti quei
Malanni, ¢ disgrazic, che-alla giornata accadono a i mortali.
 LASCLAT Li bicchier vori, e+ piacti feemi. Havendo bevuto,e mangiato quan.

piaciuto.
> GLARDINO. Luogo dove si piantano fiori, ed altre delizie similida i Latini
detto Florarinm 5 fen pomarinm. Vicne questa voce dal Tedesco Garten, ¢ questo
dal Latino bortus, fecondo il Ferrari, ii quale biafima il Perionio, che la fa ve.
nire dal Greco ardevein, innafiare, seguitaco in cid dal Monofini. Ma tanto que-
glinella sua lingua Francefe,quanto questi nella nostra Toscana,sono troppo ap-
Paflionati acl far venire le voci dal Greco yilche non & fempre vero, ch' elle»

     

ee ee

'vengano.

 SERLINA. Gogna. Vedi sopra C. 2. flan 15.,¢ C, 3. stan. 62,
HITERA, B) quel berrettone, 0 cartoccio di foglio; che dalla Giuftizia si

fa meteere in testa a coloro » che sono fruftati in full afino. Vedi sotto Can, 12.

Bt

KP.

» 4h Pareraio del Nemi. Intendiamo le forche, perch queste fon fituate in uns
campo, che era, ¢ forse è ancora della famiglia de' Nemi, ¢ Jo diciamo Pareta-
40 per coprire il detto.. Li Pareraio € un boschetto fatto per uccellare a fringuelli,
ed altri uccelletti simili nominato Pareraio dalle retiyche s' adoprano a tal caccia,
Ie small fichiamano parere. Vedi sopra C. 4. stan,27. al termine mandatoin Pic.
tardia,

~ PERGOLATO., Le viti che foftenute in aria da pali, e pertiche, formano co-
Me Una coperta, o tetto si dicono pergole » O pergolati, come dicono anche i La-

Se

 CORRENTE. E10 fieffo che travicello » cio® un legno lungo,groffo più d'un
> € 8' adatea a formare, ¢ foftenere i palchi, ¢ vetti delie cafe,

| 46 a¥LZZA. § intende quella fune 5 con la quale si legano per il capo le be-
-..

| fit ye però € detta cavezza-quafi capo, ¢ il Poeta la chiama così, perché è lega.
, #2 per il-collo, ecapo degi' impiccati a quei correnti, ¢ gli chiama Penzoli, per-
-Sh€ gli figura grappoli d'vua pendenti a questa pergola,;
, BRA: Shek;

 
 

 
  
   
   
    
   
     
  

282 MALMANTILE- 2
SP-ARGON le rame. Gli alberi che sono in questo.giarditio distendon
rami in diverse mani¢re; ma in-vece d' alberi (ono schelétri.
tomie. Scheletro, 0 scheretro diciamo tutta |' offatura dtumco
ogoi altro animaie,ripulita dalle carni, ¢ rimeflainfiéme con
105; e4notomia chiamiamo il corpo d'un' huomo, ed” altro'ani I
moftra tutti li nerui'y wulcoliy ¢ vene, chefoho foro Jaipelley: soe
SPALLIERE, Quelle piante, ed alberhy che si fannovdutendere fu perte,
ra con i rami, come limoni, ¢ fufini, ec, si dicono spallieré se qui:pig
mie per ogni specie di pomi d' agrumi, dice, che in vece di tali pomi
questi alberi a spalliera gli aborti., i miofiti €s gobli.< ai Se
INSELICIAT E. Seliciavo dal Latino filices diciamo up lafiricd fatto

ma firettamente,intendiamo quei laitrichi fatti-di plete: piccotidinies 5 o- tan
giion fare ne i viali de i giardini a foggia di Mofaico con pietreyperd maggiori di ji
quelle del mo/aico,e minort atial di quelle degit acciortolatiy e sono diary colo- |

ri in manicra che (ene formano figure,¢c. Come col Mofaico., Binovece di gi
fic pictruzze,dice che (on fatte d' ugaa, drdendi, ¢ d' aitreoflature minute.)
a MORC HLA, Intendiamo la fondata deil'oli0 dai Latuno-amarca, © gets dat
fr, aan. 4 2
CABBROSTOLITE, Abbronzate. e<bbrofolire propriamente viohdire qu
abbruciamento che si fa agli uccelli pelati, agcio fiabbrucino quei pr 'the
non si fon potuti levare con le mani; ma qui vuol diretince dal fuoco ¢on un
leggieri abbronzamento ¥ che diciamo: abbractacchiate -3 egiggdted
MV MMIE. Sono cadaveri d' huomini che-hanno la-carne appiceata Ws
full' ofia seccatavi sopra da balfami, bitumi,ied aromati, come fon cOlpl,
che si trovano forto le Piramndi di Bgicco, 1 quali (ono di persone péincipaliy che
gli Egizzj havevano per coftume di riewpucre di baifami, ed aromaci, fate
gli con strette strisce di tela', odi drappo com murabilitfima maeftria, ¢ pt
hi insieme con qualche Idoletto fatto di metaijo dentro a una cafla, che sate
se

VANS

 

'

uy

Ni

iy

i

:

faccia d' huomo; così gli riponevano sotto quelie piramidi, dove non &
cevano; ma si feccava la carne, ¢ si riduceva tanto queila, che |' offo come

trito; per Jo che si ono-conferuati quet corp: fino a1 tempi nottei, ed f

ne trovano, Polid. Verg. de Rer, imuen, lib. 3. 6, 10. riferisce-con te seguentipe |

3» tole il modo di questo fotrerrare i cadaveri degh Egizzj:Agypuj Hatin mor hg

>» tuo homine ferro incurto cerebrum per-nares educebant, jocam iiusmedt

we is expl; deinde lapide A chiopico circa-ilia i 'a

>» bant,atque illac omnem alucum protrahebaat » & ubi repurgaverant, rorfam lig

»» Odoribus contafis:refarciebaat., ide iterum Contucbine. Vbi hae fecitlent,fa- toy

>» liebant nitro aduito feptuaginta dics, nam diutius (alire non ticebav; quib

y»» exactis Cadaver findone inuoluebant gummi iilinentes; Ho deinde.

&

fy

A

w

I

 

 

   

9» Pitiqui ligneam hominis efigiem faciebant, in qua inferebane y

>» lumque ita reponebant; Eeid, ut arbitror,ica facticabant ju eo acto” >

»» ta cadavera diucurnius incorrupta servarent. ae
Altri cadaveri fecchi ci vengono pure dagii Egizzj,i quali corpi |

teriori, e-tutto, fecco, ¢ come impictrito; ¢ ono iciza farciature; equell a

corpi d' huomiai, che dal vento sono staui fotterrati vivinelia rena 5 € quivi com

   
 
   
   
   
    
   

 

tidal' tar della rena, Di

 

chi, ma p:

SESTO CANTARE:
fertiatifi forse per causa de' venti meridionali, ¢ però il nostro Poeta dice: Fenn-
queste Mummie si servono i Medici per diversi farma-
a particolarmente per la Triaca, La'voce Mummiaé Araba; ¢ il Voffio

tira da Atam, che'in Arabesco vuol dire, cera (de vitijs Sermonis lib, 2. cap.
la cera ¢ '! miele faculta conferuatrice; ¢ della cera si servivano gli

per mantenere i cadaveri fecondo Brodoto, Jib, 1. Ma la pece mescolata
bitume, era forse quella materia, per quel che apparisce,con la quale
gli Egizzj condivano tali corpi, 1a quale in Latino greco dicono Pi

 

289

magi s
?

DI, Intende quelle bafi, sopr' alle quali fon posate le statue.

R51, Intende torfi d' huomini, che pittorescamente parlando vuol dire il

fenza'tefla', e'braccia, ¢ cosce Latino truncus; ¢ questi dice, che sono
ilareiti; cio€ raccomodati, rappezzati, riftaurati con haverui mefie in veces

- delle lor tefte già confumate dal tempo, alere tefte nuove, ¢ fresche di banditi;

uol ¢ meta » che alle volte si veggono al Palazzo della Giuftizia, eo.
sopr'alle forche elposte alla vifta del popolo, eflendo fate tagliate di poco tem.

po ai maifar
/ STANZA Lil

Inter '8 quadri di cipolle
Snip i fior Youyatee nariche;
Soma teiccioni, i signoli,e te bolle,

Le posleme, ta rigna, e le volatiche.

Vil mal Pricefe entrante alle midolle,
CW feminaro dalle male pratithe,

'Teticberi, le rabbie 5 ¢ gli altri mali,
- | Che vi mandano gli Offi,e è Vettnrali,

  

malfartori bandit, « per frelshe

STANZA LIV.

Pescheinsu gli occhi fonui arzurre,egialle,

I marchi, che fiorir debbon le spalle

Ai tagliaborfe, ¢ ladri ancor scolari;

Le piaghe a maffe, 4 pererecci a bulle,

wend ventofe,¢ gonghe in pik filars, ©
 ¢ il fior di rofolia, € pik rofoni

D' ortefica, vainolo,e pedignoni.

Cu re for per chi gti porta pari,

o ita a descrivere i) giardino dell' Inferno', ed in queste duc ottave narras
i - guelche contengono gli spartimenti. è
<QVADRI di cipolfe. Intende quelli spartimenti, che si fanno in terra ne i giar-
Me\gquali si pongono le cipolle de' fiori. Latino areole, puluini,
» PRA foglie, e-natiche. Dice così per moftrare, che questi mali vengono nella
carne ef mente, ¢ pigliando natiche per tutta la pelle dell' huomo, dice che
fra quelle foplie nascono questi mali in fu le natiche, intendendo la pelle, ¢ per-
ché anche la maggior parte de' medesimi mali per Jo pili viene in fu le natiche,
'come laogo pib carnofo.;
CHE vi mandano gli Offi 5 ¢ i Vetturali, Questa forta di gente ha per coftumes
itnprecar fempre male, come venga la rabbia, il canchero, la pefte, © firnili,
» PESCHE in fu gli occhi, Qucei-lividi, che vengono attorno agli occhi, quando
sono flati percofti da pugna, o da altri, e sono di colore azzurriccio, € intorno
: 7 $ Dar le pesche: i Latini dicono /uggillare alquem, vedi sopra C, 3,
'1. y che noi purt diciaino anche figilli tali lividi,¢ diciamo anche: figillare un'

uno, *
 GLI sfregi fior per chi gli porta pari, Gli sfregi fon fiori, che anno bene in ful
v ) di coloro che portan as tani, cioé fanno bene il raffiano, che portar i pollé
woo) dir fareil rufhano dalla voce pouler Francefe che vuol dir 5 vigtierro amorofo,
diciamo porta poulers. Qo CUAR.

SSS HSS ETE SaaS ers

 

 

 
  

290 MALMANTILE

MARCH, Tntende quei fegni, che dalla giuftizia si fanno
droncelli, gai per esser giovanettinon sono capaci della pena
Higmata, Vedi opra C. 2. st, 3. alla voce sberlefe. 6

PLAGHE a maffe, peterecci a baile, Piaghe, ¢ peterecci ins rand
ma, Nell' ulo diciamo anche Patereccio; ¢ Panareccio dal Greco, usato ai
da' Latini Paronychia, postema che si forma alla radice dell' ugna, che i
chiamano Redivias., 0 Reduvias yp eas

GONGHE. Intendiamo gavine infermita che viene nel collo, ¢ quei t t
che fon taluolta /pine ventofe, perché diciamo haver le gonghe os ee e

venga apparentemente nella pelle della gola sotto le ganalce, Latino tonfilla,glan
dule faucium, t VE
Ma perché non paia che io voglia fare un trattato di chirurgia, i
splicazione di questi mali; tanto piu che io stimo, che faranno inteli per
Italia, nella quale fon chiamati nell' iftefla, pore differente mamiera,
intelligenza dell' opera serve fapere, che in questo giardino sono tutte
ta, che vengono agli huomini efteriormente, le quali il Poeta vuol moftrares,
che si generano nell' Inferno, come sentina di tutti i mali. 10.04
} LV. STANZALVL,.

Alla ragnaia al fin si fon condatti
Di fils da toccar la margheritey »
Ove de' tordi cala, ede' merlottt
Alla ritrofa quantita infinitay

    
   

ai Lae

   
  

    

  

aes

Si maraviglia, si fupisce, epanta
Martinazza in veder si vaght fiori,
Erimirando hor questa,bor quella pianta
Non fol pasce la vifta in quei colori,

Ata confortar si fente tutta quanta
Alla fragranza di sh grats odori y
E ai non corne non pxo far di meno

Che fon poi da Biagin pelati,¢ corti
Sgoxzando de' più frolli ana partita y
Altrane/quartaye quellach'e puafrefea

Be ssh ee eh

Vu bel maxxetto, che le adorni il feno,

Nello Stidione infilza alla Turchesca,
NZA LVI.

= BEG ES.

Veduto il tutto, Nepo la conduce

Chi per la pizzicata, che produce
A! bagno, on ogni schinvoy ¢ galeotto )

| dl uazo,fa tragedie sn ful.capportas 
Opra qualeofa: Vn fa le calze,un cuce, Vn mangia,un fofia nella verrinala y
etltri vende acquavite,altré il biscotto, Vin trema in sentir dir:fuor camiciuels,
Martinazza refta maravigliata, ¢ si fupisce, ¢ rimirando tutte quelle piante,
paice la villa, ¢ soddisfa all' odorato con quella fuave fragranza, ne pud non fa-
re un mazzo di quei fiori galanti per adornarfene il seno. Vifto il giardino, Ne
po la conduce alla ragnaia, dipoi al Bagno, dove stanno i galeorti » descritto ¢-
me è appunto guello di Livorno cirea operazioni, che fanno i galeotti, |
SP ANT ARS!, Dallo Spagnuolo e/pantar/e. Vuol dire efttemamente mart
vigliarsi, ¢ si dice in augumento maravicliarsi 5 frabilirft, (pantarsi, che & il verbo ui
§paventarsi fincopato. Habbiamo l'addicttivo/panro, che significa eftremament —
marayigliolo. Ma forsc ¢ da Spandere, quali voglia dire largo a ie
de, ampio, ¢ in confeguenza maravigliofo. E di Spamse addicttivo, del, Ay
Spandere cen' ¢1' efempio in Meffer Cino, Quando ha per gli occhi (ua poem &;
fad 7 wap Sah
4 VN bel marzetto, che le adorni il feno, Bello ornamento del send d' una few.
* pa havervi crofte, rogna, ¢ simili galanictic, delle quali potcya efler fy
~

UGA

 

Serer,

 

 
 
 

SESTO CANTARE)

Poeta scherza per esprimere la laidezza di Martinazza.
BE' una felua, o macchia folta posta per Jo più lungo i rivi, per
si pe eee sospefa a due Mili, e questa rete si chiama ragna
'a imitazione di quei veli, che fanno i ragai per pigliare le mosche,
ragne. Pietro Angelo da Barga nel suo Poema della caccia de-
celli: Hos caffes, has ipfa plagas y bac retia quondam Ante alias omnes telamts
ere dotta Innenit dixitque (uo de nomine Arachne, E da questa rete ragna si
nna ea » Ove Gi tende per pigliar tordi, beccafichi, ec.
'teccar la margherita, Cio quelle Aanghe,sopr' alle quali si da il mar-
lla Corda, che quetto vuol dir toccar /a margherita.
DI, merlotti, Vuol dir merli giovani, ma dicendofi merlotto, o Tordo
stintende Huomo femplice, corrivo, che cala; che si lascia pigliare.

3

 
 

st. 59.
g Gabola fatta a foggia d' una trappola da topi, con la quale per
certo Ordigno si pigliano vivi gli uccelli, detta così per efier 1a parte, da
eferrare rivolta in dictro. Vedi sopra in questo C. st, 1. alla voce con-
Qui per ritrofa intende Carcere.
; » Maeftro Biagio, o biagino vuol dire il Boia, che così havea no-
me, quando I Autore compose le presenti Qrtave; ed a quelto fucceffle Maeftro
Baltianodetto sopra C. s. tt. 44.;
0 + Poco gli manca a essere stantio; s' intende animale morto di pik
giorni, Vedi sopra C.3. stan. 24. la voce stantio,
INELLARE alla Turchesca, Cioeimpalare,
BAGNO. Così chiamiamo quel ferraglio, entro al quale si tengono gli schia:
vi, €coloro', che per delirti fon condennati alla galera, detti pero Galeotti, i
quali dimorando quivi,fanno i meftieri enunciati dal Poeta, che si serve della vo-
 ¢ bagne per l'equivoco,il quale fa credere, che in questo giardino sia ancora il
g0 da bagnarsi per moftrarlo ripieno d'ogni delizia;come il Paretaio, ¢ la ra-
-  E.questo ferraglio di galcotti credo, che si dica bago, perché in esso quei
iquenti purgano i loro misfacti, come con |' acqua del bagno si purgano le»
'delle membra. Gagno si disse ancora un luogo simile, Li Pulci nel Mor-
pante: Dife Aforganre allora: ia fon nel gagno De' diavoli,
» PIZZICAT A, Specie di confezione minutissima, ma per 1a similitudine della
figura di essa confezione, ¢ per il fenfo del verbo pizzicareintendiamo ( comes
gui §' intende ) pidocchi.; pe
FA ie in ful cappotto, Ammazza pidocchi in ful cappotto, che ¢ quella,
fo tche portano gli (chiavi, o galcotti, remiganu, ed ogni altro mari-
; detto, siccome Cappa, 4 capiendo, perché piglia, ¢ cuopre tutta la vita,
- SOSPIAR nella verriola. Cio bere, perché bevendo si fofha, o: respira col na-
s0 nelia vetriola  cioé nel vetro. Detto che ha del parlar furbesco.. Vetrivéa er-
s herba parietaria detta daalcuni. 11 Monofini lib. 9. Indicare.yo-
Aen muito vino se ingurgitafle, dicimus. Egis ha toccato ben la verrigla,
Vesrivola eff herba infeftoribus notissima, de qua Petrus Cre/centius lib, 6. ¢, ult, pocula
itrea vulgo fiune, q
. do f Auzzino vuol baftonare un galeotto per qualche
Oo 2z suo

 
 
 

 
 
 
   
  

od
see ce

  
 
  
  
  
  
   
   
   
    
   
   
   
 
   

 
 

  

292

suo mancamento suol dire fuor camicinola, intendendo, che

ha da efler baftonato; ¢ però dice: Chi trema in (emir dir

trema per il timore delle baftonate. i
CAMICIVOLA. Bun piccolo farletto di panno lino, b2

che fecondo la Ragione si forto gli altri abiti(opra alla Camicia

derGi dal freddo, come 10 detto sopra alla voce Farfetto: gli

chiamano gix/ecca, vod awieirseles

STANZA LVIIL TAN:

Vanno pik innanzi a'gridi,ed a'romori
Che fanno i rei legati alla catena
Ove a ciascun fecondo i snoi errors
Datoe il gafligo, e la dovura pena,
esi primi che for due Proceuratori
Cavar si vede tl fangue d' ogni vena,
E questo lor avvien, perché ambidui
Furon mignatte delle borfe altrui, Con

STANZA LX, Oy

Quei, dice Nepo,t il Re degli xfurai, 1 gran se gli marcy dentro a'

Che pel guadagno scortico il pidocchio, Che nol vendea se non vaiea

Vn fernizio ad alcun non fece mai, Così fece det ged hor
Se non col pegno, e dandoli lo (crocchio GP intarla il doffo,e da'fiohfe fh
Paflano avanti a vedere i delinquenti legati alla catena, ¢ gait er lo

falli. I primi sono due Causidici, ed il fecondo è un' Viuraio y ti

fecondo il merito. » Seah

PROCCVRATORS, Agitatori di liti. Cautidici tanto Civilijche criminali

MIGNATTE. Sanguifughe. Quei vermi acquaticijde i quali si servono 1 Ce»

rufici per cavar fangue; e perché si dice, che i danari sono il fecondo fangut

però esser mignatta delle borfe alerni vuol dir Succhiare, roe ¢avar il de i

altrui borfe, come fa la mignatta fucchiando, ¢ cavando il fangue dalle vent»,

diciamo mignarta, 0 mignella a uno, che € firetto del suo, ¢ volenti sig:
quello d' altri: A questi tali pud quadyare cid, che disse Orazio. Lon milfura ch
tem nif plena cruoris birudo, a:

V-AGLIARSI, Intendi dimenarsi come fa uno, che habbia rogaa, © altro pe

la vita, che si dimena, ¢ scontorce per grattarsi il prudore; o pizzicore conl'a

bito, che ha in doffo, ¢ fa con la vita un moto simile a.quello, che fa und, che

vagli il grano. sae
TONCHI, Forse dal Latino sondere pre(o per mictere Ȣ divorare, Sond vet

mi piccoli, 0 infetti, che si generano nelle fave, pilelli, ed in altri i

 votano i granelli rodendoli; da i Latini detti Curcudiones. Virg, 1 z

pulatque ingentem farris acervum Curculio, es visage

TIGNVOLE. Bachi simili ma si generano ne i pani 5 ¢ fogi impaftari; dai
latini detti Tee, Di queste ne nascono ancora dal grano, ¢ si chiamano prt

noli. peas ah
™ MOSCIONT. Quei moscherini 5 che na(cono dal vino, che dicemmo sopra in
questo C, stan. 37. oe

Son

   

  

      
   
  
   
 
   
    
    
 

   

  

 

 

Se Fern Ee eae SE Sr FEKETE

 

    
 
      
 
 
 
 
 
  
  
 
     
 

SESTO CANTARE: 293
So eae eet si generano nel legno 5 ¢ lo rodono; da i Jatiai

er) eet.
| RARE « Intende quei farfallini, che si generano nel grano. Pyrau/fa,cou
4 areca (ono app: = farfalle pil grandi, le quali papain oe al
lume, e vis'abbruciano. Di queste disse il Petrarca. Semplicetta farfalla al lume

 

 “COCCIOLE. Piccoli tumoretti, o enfiature cagionate da' morfi danimal
come zanzare, bruchi,¢ simili.:

'S8RANF, Rotture; Scorticature. Vedi sopra in questo C. stan. 47.
PER riftoro. Per ricompenfa. Dan, Par. C. 5.,

Gabino Dunque che render puoffi per riftore?

Equife ben pare, che il nostro Poeta voglia dire »per riftoramento, o alleggeri-

ia de i-teavagli,¢ pene, nondimeno è tutto il contrario, perché @ parlare -

.
P| ' ¢ vuol dire; oltre a gli altri travagli ha di pil, che lo flageliano,¢ pelta-
foe dese pieno di feudi d' oro'. Questa voce. rifore vien dal verbo ri~

 

ie derivante dal verbo reffawrare, ed ha quafi, lo: stesso. significato,
non che questo vuol dire Acconciare, o raffettar cafe, ed altri materiali; ¢

i dir Ricompentare, o rifar danni,
ia Lo,4 Sy Nae ppi aunacordicella; i dendofi per
Py mbello quel facchetto-pieno di segatura,0 di cenci, che adoprano i ragazzi
'Perquotere i contadini,come dicemmo sopra C, 1. tt. 59. Zimbello detto, cred'
10, quafi cennelio, civé piccol fegno, argumentandolo dallo Spagnuolo, che il
chia five '

© Ub Re degli nfurai. Wmaggiore usuraio del mondo. Detto che viene da i Gre-
| Gi yiquali chiamavano Re,quello che avanzava,superava, e vinceva gli altri ne i

en: 9

i) 29h giuochi fanciulieschi; ed Afino quel che perdevay come habbiamo detto altro-
Vebs iy):

Yi, SCORT-1CO? il pidocchio. Significa esser avido del denaro, ¢ far' ogni maggior

a per guadagnare; si dice scorticar if piducchio, per vender la pelle, € con

¢ Planto 6 pod dire, Vel unguinm prafegmana colligere.: 6

, DAR be ferecchio, Preftar danari a usura »ed in vece di dar. denari effettivi,dar
aoe vaglia dieci oe venti. Vedi sopra C, 3. st, 74. ¢d è la più efecrandas

'2, che si trovi, ¢ forse la più praticata.
; MARCIRE, Intendiamo infradiciare, corromperfi, Dal Latino marceres;
|

SE non valeva un' occhio, Se non si vendeva caro,¢ a prezzo rigorofidimo: Non
vit cola pil cara dell occhio. Onde Catullo. Ni te pins oculis meis amarem
INT ARLARE, Esser mangiato da i tarli, o tignuole, che i Latini dicevano:
Cariem sentire,

E PESTO dai fui soldi, Inscanto dalle percofle di oa facchetto pieno delle
ae monete. Vuol moftrar in somma il nostro Poeta, che
= Per qua quss peceat, per bec torquetur.

STAN.

 

 
 
 
   
  
    
    

294 MALMANTILE?@

STANZA LXL oe STANZAS
Va! altro ad un balcon balla, e coruetta y Dice la maga questoe
Ch un diavol con tasferzaacentocorde uand ella i
Chrun grad'occhio dibue ctascihainverta, | Cofkui ha fates quale
Prima gli da certe picchiate forde y Par non fo nulla,e no
Con una spinta a baffo poi to getta Domandaa
dn cert' acque bitumofe 5 elorde, ° Tal penaa chifi debbagda
Chee' n' esce poi sch' ione disgradogli orci, Ed et che per servirla è 9)
O peggiv d'un Norcin mula wines ' Prontamente cosh le da risposa,
STAN ZA LKIL + SD
Quei fu Zerbino, ed! amorofe dardo Ma dell' occhiate sue ben più '
Moftrando,il cuor ferito,€ manomeffe, Hor fentene il riverbero', ¢:
Credeva il mio fantocciocon un sguarde E com'.ci gid pensd far alle das
Di (briciolar tuttoil femmineo feffo; Dalla fineftra è tratto in

Quel che segue ¢ uno che peced d' ambizione di bello'y¢ lindo 5 ¢ credeva'
la sua bellezza di far' innamorar tutte le dame, ed hora'riceve la a
suo peccato, p 94> being
CORVETT A, Salta. Cornettare & un certo faltellar de*eavalli 5dal Laci eure
uari, Spagnuolo corwar; piegare, innarcare, torcere» EB quelto'
appropriato in questo juogo per esprimer i) moto, che faceva coftui, il,
evitare le sferzate,era neceflario che falcella(se a tempoy ed in quella'
to, che fa il cavallo, quando coruetta.: >> SRDS
VN grand? cchio di bue ciascuna ha in vette, Pone in vetta, cioè nella cima di
queste corde, tocchio del'bue, ¢ non d' altro animale; perché bovis ocnle oculorum
pulchritudo, & nitor fiemscatur, e trovatene l'efempio in Omero, dal quale
Giunone è chiamata boopis, cioè bovinos oculos habens 0 vero Dea dagli occhi grandi,
¢ percid maeflofa. E coftui doveva efler gaftigato con la bellezza degli occhi;
perché con la pretesa bellezza de' suoi occhi, haveva egli 10 AO
PICCHIATE forde. Picchiate, ¢ percofle gagliarde, Percofle » che facciano
molto male, e non paia che lo facciano; servendoci in questo cafo la voce fords
per la voce occw/to,.come si dice ricco fordo, per ricco non palefe, 0 non' cond:
sciuto. Ie
LE disgrado, Quel che vaglia questo termine vedi sopra C. 3. stan, 37. al ter
mine ho froppato. AMY
ORC/0. Che cosa fieno orcj. Vedi sopra'C. 1. st. 7. Qui intende orci da olia,
che fon fempre schifi. a
NORCINO mula de' porci, Coloro che in Firenze ammazzano i porci, € così
morti gli portano sopr' alle spalle alle botteghe de' Macellari,(ono per Jo più del
paefe di Norcia, ¢ pero gli chiama mule Norcine, cioè portacors da Wercia ¢ O-
storo fon fempre tutti unti di gratio di porco, lorditimi, ¢ (chifi di fangue.
QVEST Ac ariofa. Questa € cosa grande, ardua, e-che arreca:stupore; otra
ordinaria, ¢ stravagante,¢ che non si pud credere, me
NON ono far gindizio. Cioé giudizio temerario,e falfo; Maniera da Ipocriti,
¢ faifi bacchettoni scrupolofi, chelp
ZERSINI, Così chiamiamo quei giovani, che persuadendofi d' esser belli, faa
Z no

 

=

Ce et ee ee ee ij

aL
= aie oi

Sef asl 2 &

 

 
 
    

- STO CANTARE 295

 vanno lindi credendofi di far innamorare ognuno con la lor

quel dohenae PAriofto nel Furiofo deferive per il pili belio, ¢

¢ di quel tempo. E si dice anche Mirtillo;nome cavato'dal Gua-
fins Vedi forto C, 10, stan, 30,

10 il. cuor ferito, ¢ manomeffo aeasmerslo gacdds Facendo da inna-

   
     
 

2. Nibbiaceo 5 Vecellaccio, ec. tutti servono per intendere un
cimunito.

aR re in minutifiimi pezzi, o-ridurre in. bricioli,ed in-

morir di 'patimo, ¢ disfarsi per amor di lui-tutte le dame.
0. +) Sinonimi che figaificano:li riperquotimenti, che fan-
i del Sole, 0 il fuoco nella parte oppostaa quella,dove direttamente
i Chimuci dicono; Fuoco di riverbero., 0 di riflefo.. Qui inten-
'coftui con quelle fruftate piene d' occhi, ha il gaftigo dell' occhiate amo-

egli nel mondo dava alle donne.
Reta is ake dame. Cioé si come egli pensd che le dame cascaffe-
la sua bellezza, ( il che appreflo di noi vuol dir farle morice
re), così egli ¢ buttato da qusi balconi entro al litame, per maggior
0 efti. tali sono schizzinofi ne poslono. vederfi addosso un,
che guafti la Joro attillatura,¢lindura,,

] STANZA LXV.

; ANZA LXIV.
an ch' ¢ legato, e che gli ¢ posto Qui Nepo scuopre la di lui magagna,
berrettin baffo a tagliere, Moftrando ch' ei fu nobile, ¢ ben nato,
-colpe colpo da discofto E fempr'hebbe il Pedante alle caleagna;
fra. gliene facadere. Cc ontuttocio voll' esser mal creato;
Mifero sia quivi immoto, e tofto Perché se e fulfe fhatoil Re di Spagna,
do gli occhi ai colpi dell'arciere, Mi cappello a nellun mai s' è cavato;
muave Punto, ochinaorizza, Pero s ei fu villano, hora il maeftro
€.42 cultello che & infizra, GU! infegna le creanze col baleftro.
STANZA LXVI.
4 par comune usanza, Se ¢' faltan la granata,addio Creanza,
4.risponde al Galatrons; ae ch? e° fien nati nella Falterona,

 
   
    

    
   
   
    

  
  

  
   
   

   
 

 

      
  
   

     
  
     
     
  

   
  

noi Fanciusi un pocon offer uanz A, Ata per la loro afinita Superba,
eyes: il. ahi baffona. Son poi fuggiti pitt che la mal erba;

   

“« Dattro. che segue è uno, che nel Mondo non volle mai imparare.i buoni coftu-
Bi. non si yolle mai cavar il cappello di testa per riveric nefluno,per grande
j 'ch ¢gli fulle, onde gli avviene il gafligo, che si dice nelle presenti ottave; E
Masoasaa dice a Nepo, che hoggi di questa forta mal creati ¢ pieno il Mondo.

pies (TINO a tagiiere. Berretta bafla ¢ piatta,nella quale non si vede la

del capo, come sono /e coppole Napoletave,
eat Qgni volta ch' ci tira. Vedi sopra C. 2. stan. 57.
« Sta duro; Sta faldo; Sta fermo; Non si muove.

(RCIERE. Colui che tira con la balefira « -4rciere in molti luoghi del nostro
do s' intende il Caprone, 0 Becco. Lat, aries.
(AG AGN.A. Mancamento, difétto. E parlandofi d' huomini s' inane tan-
'animo, che di corpo, Dante Jaf. C. 33. dice. OGe-

    
     
 
   

  
   
      
 

  

  

 

age 'MALMANTILE”
O Genovefi huomini diversi 08 OND
“D egni coftume, ¢ pien d! ogni magagnd
Lalli En. Trau, ston erga i j up reeves J
¥ i bE pias vas

Ogni trattate conte ogni magagna |

Magagna in Lat, barb, & detta Mahaminm, Swan) Franz. Aabain,

¢ vuol dire propriamente mutilazione di membra,¢ si stende a significare « i

no,¢ detrimento. Vedi Du Frefne nel Gloffario alla Parola Aabamiam,
BEN nato, Nato di nobili, ed honefti parenti,

       
   

HEBBE fempre it: Pedante alle calcagna, Hebe fempre il Maahre a te 4

gl' infegnava i buoni coftumi, ¢ termini, cet Sar
MAL creat, Senza creanza, Vao-che non fai buoni termini 0 coftumi,
VILLANO. Contadino', Stintende uno scortele, ¢ mal creato. Planco ra
merum, intende un' huomo ruftico, senza civilta, fenga galanteria, un pre
villano, Catullo, Péeniruris, & inficeviarum, [1 contrario di vidlane®,,
SE faltan la granata, Se essi cscono di (orto la cura del padre;'¢ del macho,
Si dice faltar la granata; quand? uno e(ce de' pupilli che ini diero's ema
re ex Spbebis, Dicono che quando uno ¢ arruolato per birro,debba far
mee a fare il noviziato,¢ fnito questo tempo gli faceian fare una cirion
faltare topr'a una granata, che gli mettono d'avanti in terra,e che fatta questa azione
refti libero dal noviziato, ed.in un certo modo esca de' pupilli; ¢ da que r
monia (che se non è vera, ¢ aflai vulgata ) credo 10, che habbia origine il pre
fente detto, F 'hae?
PAIONO nati nella Falterona, Paiono nati in luoghi incolti,e difabigati,come
sono le montagne della Falterona in Cafentino, dove poche creanze im-
pararsi, non essendo in quei luoghi con chi praticare, se non con pecore,€ por
ci, Ci ferniamo però di questo termine per esprimere un' huomo incivile, ¢ foz"
zo, eche tratti da villano; come ¢ quercubus, aut faxis natus, ae
SON fuggiti più che a malerba. Nefluno gli vuol praticare. Sono sfuggiti 42
tutti. Malerba intendiamo l'ortica erba nora, la quale & sfuggita da tuttl » pet

ché pugne.
STANZA LXVIIL STANZA LXVIIL

Ma chi è quel, 0 hai denti di cignale Ora per queste sue finzioni eterne,
E lingua così lunga 5 ¢ moffruofa ? Chi egli bebbe fempre nella mercathrs,
Si vede, che fon fuor del naturale Lucciole dands a creder per lanternt y
A me paion radici, 0 simil cosa. Sharbata gli han la lingnaye denratirss
Wepo rispose; Quello ¢ un Senfale Main bocca havids pos di gran cavertty
Che si i act il Parola,ma (a glofa Perché non datur vacuum jn naturhy
Huom di fandonie, dice, ¢ di bugie, Glibanno a mifferio in quelle fhanze we
Perché in esse fondo le fenferie. Composto denti, ¢ lingua di carate «

Segue un Sen(ale, il quale ¢ gaftigato delle bugie, che anna cavato
la lingua, ¢ identi, ed in quella vece meflovi delle carore. Ji Poeta si serve dell!
affioma Peripatetico! Won datur vacuum in natura, col quale ingende che fulle ae
cefiario riempier quei voti,cagionati dal' eftrazione della lingua, € deati » Ae
(cherza, sapendo bene auch' egli, che quei medesimi voti erano gid ripienl #
aria. >) ” ae;

    
 
  

S82 rennrer® BRfs2-nPhee

~

See

 

 
   

'fon mediatori afar vender una mercanzia:
vin Birenzeun'Senfale di beftic, huomo sccl-
oy che per le sue farberie:fu impiccato a: forche erette a posla per
0 ee eealns 3 ed è lo stesso, che quegli che fu,
chino detto | 3. fhe 55.
OIE ye bagi s aiclonaoe dal vero, ¢ sono si pud dir finonimi, se
Ml dir chiacchi vana, ¢ bagia propri vuol dire attefta-

fenferia. S' intende,quando uno di questi Senfali fa vender qualcofa,e

    
 

per lanterne, Dar a creder una cola per un'altra, 1 Lalli

     
 
 
  
   
  
   
     
   
 
   
          

| Lucciole qui rimiro per lanterne,
£4, Bi quel vermicello alato, che di notte riluce da i latini detto Ci-
Nottiluca; da'Tedeschi animaletto di'S. Giovanni, ¢ da' Greci Lampyris dal
e fai egiare nelle tenebre, come egli fa;\¢ /anterna & quello arne-
; 'quale-si porta il lume la-notte ferrato da talco, offo., 0 vetro per di-

jo dal vento; ed ¢ voce.pura latina,

4°) Specie di radica, Latino /iser, Mail proverbio Pisntar, 0 fecar
ca dare a creder bugie. Latino imponere alicui, Onde Impostura, es
febene si dice in pil grave significato. Vedi sopra C. 2, st. 70. Dices
yperché vi fon meffe tali carote,è non folamente per riempiere i
perdar il gaftigo a'coftui delle tante carote, che esso haveva piantate,
era in'vita', facendogli haver sempre dentro alla bocca effettive, ¢ natu-

a. 'ANZA LXIX, STANZA LXX,
See volta ha la facia, Vedi colui',¢ al colle ha un' orinale;
“Bute diavol legnainolo in ful groppone Cieco, rattratto lacero', ¢ piagato ?
— Gli ascia itlegname,fega,ed ipialliaccia, Ei fu Governator a uno spedale,
(Ste servir per [uo pancone; Ow ei non volle mai pur un malato,
a fu; c'alla pancaccia Ora per pena ogni dolore', ¢ male,
aglian le legne addofsv alle persone, Che gl infermi y' haurebbono portato
tener sa lingua in brigiia ( Mentr' alia barba lor pappo st bene )
ender. lapariglia, Sopr' al uo corpo tutto Guanto trene,
'il gaftigo dato-a* Mormoratori, ed a quelli, che, eflendo Aati Sopranten-
| Spedali,non hanno havuto carita; ma solo hanno attefo a crapulare per
Manodion 3 che dovevan fomminiftrare a' poveri, ed infermi.
edi

    
   
  
 
 
  

VE; Codrione. Le parti di dietro del'huomo fra le reni, € le nati-
fowo C, 10; st. 50. LU Perfiani disse,
“\ Céascun teme, e si caca nelle brache
Tn vederus appiccato ful groppane
t ® “Lo frocco da scannar le pastinache, -
si cava che ¢ usato, ma per lo più in scherzo,- Viene fecondo i} Ferrari dal

 
   

 

Orrhopyginm, che significa lo steffo. Hs
ARE, Tagliar con l'asce, che è uno strumento da legnaiuoli noto,chia-
Pp mandolo ©

. =.
a ooo

   

 

 
 

Tar ae

298 MALMANTILE ©

mandolo così anche i Latini, che lo dicono ¢4/cia. Ifidoro neilé Origint lib, 19)
6.19. Ascia ab affulis dita quas-a ligno eximit yenius diminutionm nomen eff asciole
( forse accetta ) Eff autem manubrio brevijex aduerfa parte referens vel, i
Jexm 5 vel canatum, vel bicorne rafirum, Vitruvio difie Asciare Lib. VI. ¢. 2. Suma.
tur Ascia, © quemadmodum materia ( Qui invende il ego; che gli Spagouoli dal
Latino chiamano, madera ) dolatur, fic calx lacn macerata ascierur, Am,
[MPlALLaccLa, Qui la rima forse ha necefitato  Autore a servith di
guefto verbo impiatiacciare in vece del verbo piallare, che vuol nee
-gnami con:la pialla come intende qui 5 ed il verbo impiallacchare vuol dire tito
prire un legname con piallacci ( fefftles lamina, famine pratenues \e disse Plinio}
fond fottilifene afircelle di noce, con le quali si cuopre altro legname più vilei
far cafle, tavole, ed altro, nella forma che si fa con ' ebano, granatiglia, ed
tri legnami nobili. Plinio discorrendo di legnami, de' quali gli antichi si (erui-
vano per impiallacciare lib. 17. 43. Que i laminas fecantur, quorumque Z
weftiatur alia materies, pracipua funt cedrus, terebinthus, etc, E poco 2
prima origo lnxuria 5 arborem alia imtegi, © viliores ligno pretiofiures cortice fier; B
PO, Lvcugitate funt, & ligni brattea, nec fatis, Capere tingi animalinm cornua.
dentes fecari, liguumgue ebore dsspingui, mox operiri =
LALLA, Chiamano i Legnaiuoli quello strumento di legno', che ha un ferro
incaflato, col quale aflottigliano, appt » pulil 9 ed addiri: ile
gnami, da i Latini,fecondo molti,detto Dolabra, ma forse con qualche
'Vn' antico Grammatico pat che 1a confonda coll' alcia.. Dolare fabri' Co
ascia ledere, Si legge in Colum, lib. 3. Qua falce amputari non possunt, dette doles
bra abradito,il che pare che voglia dire pili tofto accetta, 0 pennsto, ovanga;
che pialla: E corrobora questa opinione il medesimo Colum. lib. 4. ¢-24, serven
dofene in diminutivo; Semper circa crus dolabella dimovenila 'ef revra, clot Inter.
no al cansbo della vite ¢ da levare (a terra con una dccettina, 1) Calepino tiene sche
da pialia si dica runcina, e porta ' autorita di Plinio lib. 16. cap. 42. edd éncitares
runcinarum raptus, ove pare, che descriva appunto l'operazione della yf
per infino l arricciolinamento de' trucioli: Tutto il tefto dice così: Br ad quack
que libeat intoftina opera aptissima ( parla de}l' abeto ) five Graco, five Campanty,
ficulo fabricae artis genere spettabilis, ramentorum crinibus pampinate ae
be se voluens ad incitates runcinarum raptus, Ma io ardisco conteaddirgli ots
Y autorita d' Hermolao che dice: Runcine [unt maiores ferra, quibus fabri materis-
rij fecant arborum moles fubiettis canterijs, Si che non la pialla, ma fa fega grande,
'che adoperano i Marangoni per ricidere i legaami, adattandoli sopra quel C4
Valletti, che noi chiamiamo canreo ( dal Latino cantherins, cio' cabalus: -e pil
volgarmente pietiche, i quali sono one di due correnti inchiavardati ial
a guila di cefoie (che propriamente si dicono pietiche ) ¢ d'un' aleco pezzo'di cor
rente, che si merte a traverso alle pictiche (¢ queito si dice Canteo ) € a
così un triangolo vi adattano per via di piuoli il legao da fegarsi, Runcare è tet
'mine d' agricoltura, 'che vuol dir propriamente tor via, onde se ne formd per
yeotura la parola antica Latina averruncare, cio' avertere; ¢ se ne Iddio
wAverruncus detto Così, perché ab'eo precari folent, ut pericula avertat; si come dice
Watrronc., E in proposico d' agricultura (ene fabbricarono le parole aie.

 

   

fF EP Re ewe

a
wt

 

 
 

SESTO CANTARE: 299

Ronconé yle quali significano strumenti da nettare.i campijda rimondare fructi, ¢
i cinne raat Plinio lib. 18, ¢./21. Siligioem gfe striven. sfemen, shepeleas

accato  farrita 11 iB « Runcatio, cum feres in articulo off, evulfis inu-
. joo eoremng apes radicem indicat, Segetemque discernit a cespite. EB Catone cap.
even cremarique; ie che pili tofto Runcina parrebbe, che avel-
 fead eflere la roncola, 0 cosa simile, che la fega, o 1a pialla. Ma forse non.
tanto. il Calepino, quanto anche il Vocabolario delia Crusca dal levar via, ¢
a faellere,¢ ripulire ( che questo significa, comes' ¢ vitto il verbo Runcare ) hanno
dato il nome di rancina alla pialla),perché clla pulisce, appiana, ¢ leva il fover-
da! Jegnami.. Tuttavia anche per questa ragione 1a dires do/abra, perché si-
questa ancora pulilce, ¢ rade, come dice Colum, nel juogo (opra cita-
sia. come efter si voglia,poco fa sal rth nostram, baftandoci intendere, che

[ene guello strumento da legnaiuoli; che habbiamo accenaato,
 PANCONE. Chiamano i degnainoli quella loro panca grotia, sopra la quale
ilegnami per lavorargli, deta pancone, perché ¢€ fatta d' un paycone
aoe un' afie grofla-circa un. quarto di braccio » che sono affe da rifen-

 

“ALLA pancaccia Così si chiama quel luogo dove in Firenze si tiene il croc.
dilcorre de' fatti d'vaitri, edelle nuove. Vedi sopra C. 2. st. 73, E per-
ildir male del profimo si dice Tagliar le legne addosso.a uno. Latino famam.,
icnius lacerare prosciadere, pero.a coftoro vien dato il gaftigo adeguato, cons
Halab loro addosso il legname essertivamente.
TENER (a lingua in brigiia, Parlar consideratamente, ¢ con riguardo, ¢ si di-
ce anche: Tener la lingua a freuo.
ghtNDeR la pariglia. Render il contraccambio. Parigiia vuol dire una cosa,
pud dividerfi in due parti yguali; come nel numero due si pud far' uno,¢ uno.
£ 'to render pariglia vuol dir render ugual contraccambio.. Vedi sopra C, 4. st,
ma pwr pari referre de' Lat. Dan. nel Parad. C. 26. dice:
x Perch' io lo veggio nef verace /pegtio,
yh « ~ Che fa di se pareglie L altre cofes
E nulla fece tui di se pareglio.
Hoggi però in questo fenfo,¢ maniera, che si serve Dante di questa voce pare-
44 non mi pare, che si usi, se non da' Franzefi 3 che dicono pareil,

ALLA barba loro. A spele loro, Questo termine esprime Pigliare, 0 confuma-
re una cosa d' altri contro al gusto » € volontà del padrone di efla; 0 a dispetto,e
208 del medesimo.

AP PO. Cioé mangid, Donde Pappolone uno che mangia aflai che vedem

FER SSLA SES SCES CLESSSE RES

 

Hi

laedinee 1 6.

4 eee NZ A LEXL

5 'Chia, coltui, ¢' habbiamo a. dirimpetto Che non ne pag mai un maladetta 5

è (Dice la donna) a cui guegli animali Tenne gran posto, se spele beftiali;
 Sharban con le tanaglieilcuor del petto? Ma pai per soddssfare ei non hauria

fw risponde:.Questo ¢ un di ques tali, Voluto men trovargli per la via,

a a

; Pp z STAN-

 

 

 

|
|
 

300 MALMANTILE ©
STANZA LKXILe 0 ooo) oo SDANZA\LERMD
Colni, ¢ hail vifo peffo, eit otto... Riferva il. muragche c' ni
'Da quei due [pire in femimils spoglie ».. Donne, che se a eH
Hus vile fu, ma bifeaiuoloxe ghiogtoy' >  D* arir giviellare, ¢ luce

Che si volle cavar tutte le voglies\ Dar ile... al mario in i
Ocni fera tornava a casa cote, she Hor le superbe pierre; t
E dava col baiton cena alla moglie; Alla lor liberta fanno il

Hor finti quella Peffa quei demoni, © Pero che tanto erandi, ¢ tant!

Sopra di lui fan trionfar baton...» ».. | Chan fatto per lor carcere'
Termina la moftra delle:pene date'a idelinquenti con tre forte
il primo è dato a coloro, che non vollero mai pagare i loro debiti. Iba,
dato a i crapuloni strapazzatoridella mogtiex ll terza & quello dato alle do
ambiziofe,e vane. ag janguu vik > > sitll
TANAGLIE, Strumento di ferro fatto a foggia divcefoia, e serve V
chiodi da i legni,, ec. Da i latini detto forcipes, ig Mie ta hea's
NON ne pag un matadetro. Non volle mai pagareun debito, Non pagd mii
un quattrino di debito. L' epiteto ma/aderto ha la forza d'un becco d'un
no decto sopra C, ¢. f..68. ) vast tye
TENNE gran posto, Si trattd alla grande), e'fece spele bepiali', cioè grant
inconfiderate. Lat, immanes.: 23 elit Held eg?
NON hauerebbe volute crovargli per la via, Quand'anche egli' havefle trovato per
la strada il denaro, del quale era debitore;non haurebbe ad ogni modo' pagato il
suo debito. Questo termine ci serve per esprimere', che nefluna cosa hat
potuto muoverlo dal suo proposito, ¢ fargli venir'voglia di pagate. ~~
PESTO, Infeanto, ed ammaccato dalle-battonate, che gli danno quel De-
moni finti la sua moglie. E quetto vuol dire trionfar 'bastoni..) |» t
AYO M vile, Qui vuol dire huomo di bala condizione. *
RISC AWOLO. Huomo che pratica le bische «| Bische diciama quei raddotti
pubblici, dove si giuoca a carte, ¢ a dadi; nome forfe'venuto dal verbo bi/ear-
sare, che vuol dir Mandar male sproposicatamente il fao havere: ¢ corriff
al Latino prodigere. L' usd Dante nell' Inferno C, 10,
Biscarca ye fonde te sue faculeade, ag A
GHIOTTO. Huomo,a cuipiace mangiar del buono'. Vedi sopra C. es
DAV-A col bafton cena alla moglie » In vece di portar da-cena alla moglie; la be
stonava, Coftume assai usato dalla gente' d' infima plebe, imbriacarsi all'ofteri¢,
¢ non pensar' a mandare da cena a casa alla moglie, ¢ così-briachi'tornare @/¢a-
fa, ¢ perché la povera moglie si duole d' eer digidna', baftonarlas
DAR del c,,. in fal laferone, Quand? un mercante fallisce; diciamo + vitalelt
dato ile...ful laftrone. Beanetto Latini nel/Patafio-disse Dar det Ad wy
Questo proverbio è nato'da un'proverbio antico, che era 'in Firenze; chee *t
i quali fallivano, © rifiutavano-l' eredita del padre', “andavano 'nel mezzo:
Mereatunuovo ( luogo dove si ragunano i mercanti pers iare)e\gaiei era y
¢d € ancora una gran laftra di marmo tonda, che si chiama i} carrotcia( pere evi
& poita per fegno,dove si fermava il carroccio, sopra il quale s' inalberava Pinte
gua generale de' Piorentini, quando andavano alla guerra ) ¢ sopra —
a:

Shy

  

  
   
  

Bese e see we OE pm ERE ESSE SS ee EE

OPS SSEREG EES a oe

 

 
  

jor

. awifta del popolo, che nell' hora, che si doveva fare tal
10; & questo atto afficurava la loro persona dalle mo-
'di debito s:ne potevano i creditori moleltare se non la roba, las
deva ceduta tutta a favore de i Creditori, non essendo per questo
(0 il debitore a pagaie ultra vires, eflendo questo come un cedo bonis del
'dus. Così questra laftra alle persone de' falliti, che a quellarifug-
era come una Ara, © vogiiam dire altare, 0 luogo facro, 0 afilo, o 1
già, che dail' efier prefi gli afficurava, ¢ quelto, perché cflendo dedicata
pubblico di foftenere 1i folenne:carro, ¢ la tanto famofa infegaa della
endeva per questo riguardo franchi, ed immuai coloro, che col se-
 prendevanne folennemente, ¢ con cirimonia il posscfso. Di qui dar
laftrone yuo) dir-fallire. Edi qui pure,quand'uno casca, ¢ batte il c...
lle Jaftre diciamo < // tale ha rifiutato il padre, Fallire ancora dichiamo
i pole: Ef rale U' ha infilate; che corrisponde al Latino decoxit,
TTON/, Sono il latino dateres detto sopra C,1. st. 67. £ fare, 0 dare il
» Vuol dir fare a uno qualche danno grave; ¢ qui vuol dire; sono il lor '

,¢ pena.

TANZA LXXIV. STANZA LXXV.

in orecchiche mi par che e fuont Dice la Maga; Vo venir anch' io,

po tabellaccio del Senato, Perch'il veder pin altro non m' importa,

| Sichee' mi fa meftier chrior abbandoni, Ed in questa Cittd così a bacio

\ LiPeri cht io non-voglio esser' appuntaro; A diria mi par @ esser mexza morta;

i v ch reStavano i lion, Vaglia trattar col Ke d'un fatto mio,

Ma non posso venir chro fon chiamato, Ed andarmene poi per la più corta

Ed ecco appunto Diavoli co' lucchi; Ed ei le dice in burla; Se ru parti,
Peri lascia ch'io corra,e m'imbacucchi, Vaviain un'ora,etorna poi intrequarti,

. li fuddetti gaftighi dati a i delinquenti, Nepo fentenda la Campanas

del Senato si licenzia dalla Strega, ma dovendo efser' anch' ella nel Senato per

Bat Re, dice volerlo seguir sin quivi, di dove spedita se ne yuo] andare per

  
 
   
 
  

 
 

 
 
 
    
  
  
  
   

''*

 
   
  
   
   

   

 
 
 
 
    

STAR i orecchie. Ascoltare con attenzione. e4uribus arrettis aufeulrare,

© TABELL-AC C/O. Così è chiamata da molti la Campana del palazzo de} Po-

deft ( hoggi del Bargello, 1a quale & detta la Maddalena,come yedemmo sopras

ein guefto C, stan. 23. ) forse dal latino T-abeltiones, che vuol dir Notai, i quali di-

Moravano, ¢ tenevano i lor banchi dentro, ed attorno al detto palazzo,ragunan-

~d0vifi al fuono di detta campana, la quale hoggi ¢ detta anche /a furba, perché

lori d'alcune feftc, non fuona, se non per esecuzioni criminali di tefte, ¢ forche,

 €/a nowe per moftrar l'hora, che non si pud pill portare armi; 0 pure è così

-detta dal fuono oscuro,¢ malinconico, o che almanco rapprefenta cosa mefta,
'come il fuono delle tabelle ne' giorni Santi.

4 NON celia essere appuntato, Coloro che'fon de! Consiglio del Dugento, ¢ d'al-

“tri Magiftrati di Firenze se non vanno al detto Consigiio,quando si raguna a suo-

: erence condanaati in certa somma di danaro; ¢ questo diciamo

 
   
    
    
   
     
  

2YCCO, Br la sopravvelta,o mantello Curiale di Firenze, ed cra anticamente
nay

V abito 4

 
:

   

    
      

302 MALMANTILE: >

l'abito civile ordinario; e perch questo haveva già un
metteva in doflo detto lucco, si doveva dire imbacuccarsi. Va
141, Subito fu prefo; ¢ imbacuccata col cappuecio, fu condotto alle carci
C. 11, st.22. a Pou
e4 B.AC/0, Campagna, dove batte poco il Sole, che diciamo Al rezi
uggia. Vedi sopraC, 3. st. 71. alla voce Vria, ¢ foto C, 9. st. 44, ¢C, 10,
1 contadini in vece di dire: Iuogo o piaggia volta a mezzo giorno, dicono:
Jatio, ed in vece di dire: volta a tramontana, 0 a fettentrione dicono: ab
© a paggino che ¢ i) contrario di folatio, Credo venga dal Latino
si come natio da natixus. Da molti si dice meriggio quel luogo,dove
no i raggi del Sole per interposizione di che che sia,¢ ( pare a prima'
troppo lodevolmente, perché meriggio da meridies vuol dit mezzo giorno,.
do appunto i raggi del Sole sono più: quocentije però andare al meriggio p
be che volefle dir più tofto andare a scaldarsi a' raggi del Sole di mezzo
che andar all' ombra per difenderfi da i raggi del Sole. Per corroborazion
questo idiotifmo, si uova in Autore approvato per buon Scrittor Toscano
vollero fare il viaggio di notte per lo gran freado, ma si bene in full' ora met
allora cheil Sole con i suoi raggs haveffe addolcito i rigori hiemati.Maqueftitalifid
dono conl'uso, ¢ potrebbe dirfi anche colla ragione, perché meriggio nel si
to di luogo ombrofo, e difefo dal Sole,è lo fiefio, che juogo da paffare IL ore
del mezzo di, la quale cosa i Latini dicevano meridiari, Catullo. dube adte
aeridiathm, Ora dal meriggiare, cioé fare all' ombra nell' ore calde è
meriggio, ¢, da meriggio, rezzo.. Va in un' ora, ¢ torna pai in tre g.
€ uno (cherzo usato aliai fra gente bafsa,ed intende Va hora in uno,ciok
¢ torna poi divi(o in tre quarti; fij impiccato; se ben pare che voglia dire: Va»
in un guarto d' ora, ritorna in tre quarti. Cirimonia da Diavoli,
STANZA LKXVI, STANZA LXXVIL —
Tun vnoi gli rispos' ella,fempre il chiafo; Ed ella per oferta così magna hah
Wel Con/figtio così ne va con esso Ringraziamenti fattigli abarellay.
Ove ciascun l'honora, e dalle il paffo, Dice,c'hor mai sbrartar vuol la capagnly
Sbirciandola un po meglio,e piu da preffa, E tornar a dar nuove a Bertinella, —

Ella baciando tl manto a Satanaffo Pluton le dd ticenza,el parse
Fino alla porta ye [i se ne (gabellay

Lopregad' ofsernar quanto ha prome/so,

Ei.ghe! conferma, e perché stia ficura, Ond' ellain Dite aun Vetturin saccopty
Per la Palude Stige glielo giura. Che la rimeni a casa per la posta *
La Maga così scherzando, ¢ burlando con Nepo se ne va con esso in Confi-

glio, dove ognuao l'honora.. Fa riverenza a Piutone, ¢ lo prega a ma

quanto le ha promeffo; Eigliclo giura foleanem:a:2, ¢1 accompagaatala fino
aiia porta del Consiglio la liceazia, ed ella va a cercar d' ua Veccuriag, che la

riconduca per la polta a Casa,, 8
Tu vuci si chiafso, Tu yuoi la burla. Tu scherzi. Chiaio nel proprio ¢ 3

strezta, vicolo Lat, vicus quali erano le strade di Ro oa aatica, edel pri

cerchio in Firenze. Gio, Vill, 10, 29. S apprefe fuoco ix Firense in Borgo S.

Appostolo nel Clafso tra' Bonciani ye gli Acciainoli,B pecché in quette straducole abl-

tavano taluoita donne di mal aftare, Chuatio ¢ detto forse da Vicus Vicario, Ha-

sas

     
 
   
   
    
 
         

 

       
     
 
 
   
   
   
  
    
   
  

   

 

a2EH@ Ze O- Se LED

  

 
 

FS Sue

SESS SEAS a”

==

 

ESET ER EC SEE 8 Se

a

 

an SESTO CANTARE jes

ga; in buon Latino Vicinia ) venne a significare Posribolo, ¢ perché in tali difo-
nefti Inoghi si fa gran baccano,e ff scherza, ¢ si burla senza rispetto; percid
' iglia per burla,,per ischerzo. Se bene ¢ molto verifimile, che in questo

hielo
ultimo significato di strepico, ¢ di baccano, quale fanno quelli, che licenzioia-
-menze tratcano,¢ burlano, venga dal Latino de' tempi baifi; che il fuono di

» © degli organi, ¢ degli altri strumenti domandavano C/aficum,

tutte le campane
che i buoni Litini dicevano della rromba, a cui fon fuccedute le campane. Lt

lo dice Glas,

| SBIRCLAN'DOLA, Guardandola bene. Vedi sopra C, 1. stan. 9.

9.
 GLIELO ginra per la Pande Stige. Giuramento folenne, ed inuiolabile degli
Dei fecondo la falsa credenza de i Gentili, come si cava da Omero in pil luoghi
del Lliade 5 ¢da Verg Zn. lib. 6.:

: Stygiamque paludem,

 

ea
>t: Dif cuins inrare timent, & fallere numen.
« la ragione, per la quale questo sia giuramento folenne,fecondo Servio,¢ questa

4» Styx'meerorem significat, Dij autem lati (uat femper; ergo qui meerorem non
y» featiunt rant per triftitiam, que res eft sue macure contearia; ideo Lufiu-
per execrationem habent. L' altra ragione ¢, perché havendo Vitto.

iuola di Scige aiutati gli Dei nella guerra contro ai Gigaati Ticani, Gio-

ve per rimuneracla, volle che coloro, che giuravano per Suge di ici madreo,
fullero privi del nettare delli Dei, e non offeruavano il giuramento. & queste
“sole furono finte, ¢ credute di Stige, perché (econdo Teofrafto queito Suge era
un fonte in Arcadia, le cui acque, ¢ pesci erano velenofi per la di ini eltremas
frigidita; ¢ di questa acqua dice Plin, lib. 30. cap. 16. che Aatipatro voleife da-
re ad Alctiandro Magno, quando volle avvelenarlo per consiglio d' Arifotile.
yy Vogulas tantim mularum repertas, neque ullam aliam materiam, que non,
a» percoderewur a veneno Stygis aque, cum id dandum Alexandro Magno Anti-
a» pater mitteret, memoria dignum eft, magna Ariftotelis infamia excogiratum.
A barelia. ln quantita geande, Si dice a balle a maffe, a facca, ec. sono
pero modi bafli, ¢ più tofto scherzofi, ¢ s' usano parlando tanto di cose corporee,

quanto incorporce,

\ SBRATTAR la campagna, Andarlene: Sbrattare propriamente significa net-
tare 50 ipulire, contrario d' /mbrattare; si che sbrattare él paefe vuol dire ripu-
dice il 9 © per confeguenza andarfene da quel luogo.

SENE (gabella. La la(cia; Sisbriga; si libera, ¢ filicenzia da lei. Dedotto
dalla Gabella, che si paga, perché, come è pagato il dazio, o gabellad' una.
Mereanzia, si dice sgabellata, ¢ così si spedisce, e manda via.

BITE «. Qui la Città di Plutone, detta così da divitie, le  ci vengono tut-
tedi sotto terra. I Latini chiamarono Due, quel che con Greco yocabolo dice-
vano altrimenti P/zsone, che vuol dire il medesimo, € significa il ricco Lddio,
Addio detie ricchezze, come s' è veduto sopra.

WET TVRLNO, Coini che prefta cavallia nolo, 0 a vettura.

STAN.

 

'
4
|
 

  
   

304 MALMANTILE: —
3 STANZA LXXVIILA
I Re fatta con tei la dipartenza | Saliro alla,
Al falon del Consigho fene torna,
Onde ciascuno alla eos
Alza il Civite sé abbaffa gii le corna,
Plutone licenziata la Maga fene torna in
sua refidenza si prepara a discorrere.
FATTE le dipartenze. Licenziatifi (cambievolmente.
ALZ A il Civile, Alza le natiche. Civile € una prospetti
fentancte abitazione di Città; contraria a quella, che-si dice of
pagna. I Latini simil ue entrate principaliin;/
di quelli che venivano dalla piazza,o dal mercato; l'altra di coloro., che si
geva che venifiero di lontant pacfi, o di fuori dalla Città; La prima ente:
diceva a foro, 1' altra 4 peregre, siccome riferisce Vitruvio. Noi per quelto —
miamo Foro la parte in Paccia della scena, Lin eRe
RAGNI, Quci veli che fanno i ragni.. Narrano le favole degli antichi
li, che in Lidia fa una femmina detta Arachne nata in contado di bafla
quale fu così valorofa nel ricamare, ed in ogni forta d' artifizio di tela ¢!
che non folo superava tutte |' altre femmine, ma hebbe ardire di co;
la Dea Pallade; onde Pallade superata, ¢ vinta da lei, per dispetto le;
Javoro,¢ la converti in Aragno verme, che è quell' infetto che si
veli per pigliar le mosche da noi chiamato, ragno, 0 raguarelo. Ovid,
tam, Dante nel Purg. C. 12. rocca questa favola, -
O folle-Aragne, si vedeva io te
Gid mexz? aragna triste in [u gli straceé 2
Dell' opera, che mal per te si se. bie è
DR APPELLONI, Così chiamiamo quei pezzi di drappo i quali ee
no pendenti al cielo de i baldacchini, ¢ delle refidenze de i Prinejpi;,
rano le Chiefe, ec. Varchi St. Fio. lib. 14. Ed al vano dela Cupola era tirate in fu
Le funi wr belissimo ortangolo di drappelloni. Matt, Villani lib. 9,.cap. 43 deferiven~
do le nobili efequie fatte nella fepoltura dei Cavaliere Mefier Biordo degli 4
uni. E sopra nha xn drappo a oro con drappelloni pendenti coll' arme del peice}.
comune,e di parte Guelfa',¢ degli Vbertiné. Tali drappelloni coll' arme si
appiccati in gran numero nella infigne Chicfa Collegiata di $, Lorenzo un
giorno dell' anno, per memoria di antichi benefattori, ee
SPVT-A un ciabattine. Quando uno per soprabbondanza di catarto ha difficulta
in spurgarsi, fogliamo dire: gli ha wn ciabattino giis per la gola ye doy
Sputa un ciabattino, — a ee oe ¥ - cme se nel Lal
oni, Coll' o¢chiaia lnvida toffire, e spurar far, te 2 r
Spee Fan STANZA LXXIX. noaniaie mM
Spiegar volendo poi quanto gli occorre, Onde nui fiam quaggit in. fondo di
Comincia il [uo proemio intal maniera; Gente, a cui si fa notte avanti fers
Voi che di sopra al Sole in queste forre Voich' in malizia,in ogni frode,e
CadcSts meco all' arta oscara,¢ nera, Siateé Adacftri di color che a)

  
    

  
  
 
 

 
      
  
  
  
   
  
 
 
   
 
         
   
  
  
   
   
  
  
 
  

  

ug

Ce eR eek ai kB

   
 

 

 

+

SESTO CANTARE, 305
STANZA LXXXIL
Cominci il primo: Dite, Mdalebranche,
r ¢ } Quel che e'vi par che qui v'adafe fatto,
'bazzicar taverne, e chialfi Levato i Tocco, ¢ follevace t anche

5 agnun di voi st bravo,edotto,
ea vo.
ib pincrife

1 Alor quel Diavol n' un medefmotratto
a

Vn capitombol fa sopr' alle panche,

 

  

I a un famiglio a' Otto; Efalta ae nel mero com! un gatto s
aunque, benche pare Cittadini Ma perch'il Lucco s'appicco a nn chiodo,
el vieupero ingeg ns peregrini ' Si ric e, ¢ parla a questa modo:
TANZA EX xr, STANZA LXXXIII,
tusti'in correfia O Re, cus splende in mano il gran forcone
Da Martinazxa nofira confidente, Sil Cappello (periale ha quel fegreto y
 Poithe Baldone ancor cerca ogm via Col qual si fa feornare un pedignone,

Dh entrar in Malmansiscon tantagente, fo ? ho da far tornar un' buomo a dreto.

ar-ch?eglisbandi, e trucchi via So gidche qualche debito ha Baldone
| Adbope Mctitentensy z che ¢ lo vuol pagare in (ul tappeto, '
pe ere [opra questo il suo parere Percid manda Pedino (a in campagna,
. 'the ©! ci fuffe da tencre', Ch' ei ginocherd di posta di Calcagna,
: io de i Diavoli fr: composo dail' Autore; dopo che egli ottenne
meat » nell' efercitare i} quale conobbe l'autorita, che si usurpano i Can-
: +s anon hoe be » metce per Cancelliere di questo Consiglio un Ciappellet-
celliers che fun notaio (cellerato, fecondo che riferi(ce il Boccaccio nelle sue Novel-
leye bcontraddica a tutto quello', che vien proposto. I nomi di quetti
colt pis fon cavati da Dante nel suo Inferno; ¢ sappia il Lettore, che li spro-
he dicono,fon poco lontani da quelli, che PAutore sétiva dire nel areating
rid iperfonaggi che finge in questi Diavoli sono simili alli faoi Colleghi,
ed egli medesimo in leggermi questo Canto mi diceva; il tal Diavolo è simile al
tal mio Coliega ye il tale, al tale; ¢ mi parvero appropriati benissimo; non sti-
mo già bene nominargli. Ma tornando a proposito dico, che Plutone volendo

 
  

 

fens re de' suoi Senatori, fatta una breve orazione nella quale inferiice
un ver | Petrarca Gente, 4 cui st fa notte avanti fera, ed uno da Dante Siese i
Mathri di color che fanno, ordina a Malebranche il dire, quel che egli farebbes
per mandar via Kaldone da Malmantile,ed egli,fatte prime sue diaboliche cirimo-
nie dice che il suo pensiero farebbe di farlo citare alla Mercanzia da qualche
suo creditore, salea:

FORRA, Valle lunga, ¢ stretta posta fra poggi alti, onde poco dominata dal

3© però ben detto forra il pacfe infernale dove non batte mai fole.

 GENTE 4 cui si fa notte avanti fera, Con quelto yerfo del Petrarca, ? Autore

 intende che coftoro fon fempre di notte, cioc al buio.

, BABLV ASSO, Huomo senza giudizio, icimunito. L' origine sua & scura;
forse da Valwaffor parola feudale, dalla quale ¢ fatto anche Barbafforo, lo Netio
che » o-dortoraccio; faccente; ¢ che si da scioccamente ad intendere di
fapere 0 pure da Bwaccio peggiorativo di bue. Vedi sopraC. 5. stan. 1. Ul Bini

in lode del Malfrancefe dice.
Qq Eri-

 
 
 
   
    
   
 

Mi =p.

 

 
 

TPE

 

305 MALMANTILE | 3
Erispondendo a certi Habbuaffi, scea 2c bia
Che voglion dir, che questa malattia i eS ntn se
Tatto il corpo ci florpi, eck fracali. Ah eat RRR

 

Ed il Molza in lode de' fichi: My
Hor fa tut argumento, habbuaffo. ono AE
TONDO pii che ? O ds Giotto. Huomo tondo vuol dire huomo groffo d inge-
gno, ed ignorante, come s' ¢ accennato sopra C. 5. stan. 1, si che pik rondo dell''O
: Grotto vuol dire ignoranuthmo, ¢ pil, perché lO, che fece Giotto Pitore fu
tonditfimo, fecondo che riferisce Giorgio Vatari nella vita di eflo Giotto,
BALZICARE, Praticare; Converlare: Bocc. Giorn. 9. Nov. 5. £ vatrene nel
la cosa dela paglia, ch' ¢ sh mighor nego che ci sia, perciocché non vi baxzica mai per
“ote. 1

 

ona.

CHTASS, Bordelli, lupanari, luoghi, ¢contrade, nellequali habitano les
meretrici, come era in Firenze il Chiaffo de' Buoi, ¢ il luogo, dove ora € il Ghet-
to, detto anticamente Chiafo & perché in tali moghi ula di fare fracatio, e rumo-
re difonefto; di qui forse ¢ che chia/so, ¢ bordedio si prende ancora per tumult die
fordinato, infolente, ¢ lascivo..: swash

'PIV' cattiuo di tre afi, Affo si dice il numero-uno-de i dadi, che ¢ i)
numero, ¢ per confeguenza nel pil è il peggiore che vi sia tirando tre dadi,)
quetto il presente termine significa cattiviflimo:, che vale per aftutiffiime, ed ¢ lo
ficflo che Pil trsffo a' un famuglio a' Orto, che pur vuol dire fagacissimo,eche fail
conto suo, Famigio a' Oro. B' uno de' Birri del Magiftrato degli Orto di Bali
di Firenze, che ¢ il Magiftrato Criminale; ¢ perché si sappone che cofloro fap-
piano tutte le furberi¢e, però si dice: Il tale ¢ pis triffo a' un famiglio d' Onto, per
esprimere; ¢ huomo fagacissimo. 1 Greci dissero Cantharo. afturior, che qui
Cantharo fu un' ofte d' Atene aftutissimo. 4/*m in antico Latino voleva dire,
foto, fens accompagnatura; onde chi cantava  senza strumento che L/accompa-
gnaile, si diceva coftui: canere affa voce, Di qui pud esser venuta la voce Afoes
Kespare in affo, ciok esser la(ciato folo, se bene altri gli aflegnano altra origine: 0
pure da «fino che così chiamavano ne' dadi /' #nita i Greci, dicendola Ones. I
nostro Proverbio: O a/s0, o/ei i Greci dicevano, o diciotto, orre. O sre fei, ore
afi. 'Giulio Polluce lib, 9, al cap, di giuochi fanciulleschi, ¢ de' trattenimenti de

1i antichi. AS
PAZZO Cittadino, Questo epiteto si fuol dare 4 colore che fanno sutte le tor elt
4 casa, ¢ senza considerarione; ed € lo stesso che dire ux cernellaccio,

SBAND-RE, Disfare le bande, cioé licenziare i Soldati.

TKYCCHI via, Se ne vada. E' modo baflo, cavato forse dalla parola Ze
ruck Tedesca profferita da i Lanzi, quando con Ie loro alabarde fanno allonta-
nare il popolo; O forse dal giuoco del Trucco, che si dice truccare, 0 trncciart
la palla, quando cogliendola con un' altra palla si manda via dal luogo', doves

era; dal frequentativo Latino tra/fare usato da Catullo. ' ai

TOCCO. Con il primo o largo; Specie di berrettone, che anticamente ulava
in Firenze in yece di cappello. Varch. Stor, lib, 11. Cow le calze foppannate: a Ie
jerra bianca,¢ le berrette, 0 vero tocchi di colore roffo.:

SOLLIFATE ? anche. Alzati i fianchi, cioè rizzatofi da (edere, —

jicias

  
 

ek=s & re:

BEERS Ee Seek Sec s-

Se

Le fF Pe coor Fae

=
=

Sar erree

 

 
 

  
 
  
     
     

ore cubito,

Dan. Inf. canto 34,
8

 

7 la quale

zioni civil

STANZA LXXXIV.

Pluton diede con tutti una rifata,

\fiantar fino il brachiere,

RB difeegi: Va via beftia mcancara
Com' entra célafsedia il dare,e havere?
Segualalero che vien dela pancata,

Rizzato Barbariccia da federe
Sichina,ementre abbafsa gti la chioma

Alea le pe,e moftra sf bel di Roma,

STANZA LXXXV.

Poi # intirizzaye dice in rauco fuono:

- Se non si leva dalle fquadre tl capo,
Quale ¢ Baldone,e non si da nel buono,
Mai si verrd di tal negorio a capo,

| Dove y se manca lui quanti vi sono,
Reftari come molche senza capo,

- A poco # poco, a truppe, e alla sfilata
Partendoyn breve disfaran 0 areata,

——— a

 

Qqz

SESTO CANTARE; 307

parte del corpo, che è fra il fianco ye la coscia, da Ancon greco

ire gomito;¢ si piglia per ogni (ora di piegatura, come Jo moftra il

Città d? Ancona così detta dal gomito, che faquivi la spiaggia; Pli-

pio libs 3. caps 13. La iifders colonia Ancona apposica promontorio Cumero in ipfo
q se if

«© Quando noi fursmo la dove 1a coscia
Gait sis Ss volge appunto ful grofso dell' anche.
Edi qui sciancato'é un zoppo, che habbia mancamento in tal luogo. Vedi
foto C. 11. stan. 40. B il Latino Coxendices.
ty PITOMBOLO B' quando uno, posando il capo in terra, volta sopr' as
quello tutta la vita, Vedi sotto C, 7. st. 20.
| ORB, cui plende in mano sl gran forcone. Fingono che Nettunno Re del mares
atello di Plutone usi in vece di scettro una forca con, tre punte, ¢ però dettas
in realta è una fiocina da pescatori, Latino fu/cina, e Plutone
tun Bidente, cioé forca con due pente; Equefto & il gran forcone.
er Speziale.B uno Speziale in Fireaze, che fa per infegna un cappelio.
aiubeaowe - Enfiagione che viene ne i piedi',.¢ nelle mani per causa del
r Latino Pernio. Vedi 2 C, zt. 6.
- LOamal pagare in ful t: « La vuol pagar per via di Corte, con tutte le fo-
temipebe, non vuol oa Saede non feglt mandano j birti a gravarlo, o cattu-
en dice che Baldone gimecherd di calcagna, cioè fuggira per la paura
teller prefo per debito, quando vedra Pedino, che così G chiamava uno già bir-
ro della Mercanzia » che éil Magiftrato, per yia de] quale si mandano l'efecu-

¥.
re + Subito. Latino ¢ veffigio. Traslato dal giuoco di Tan » che si dice
dar di posa — si da alla palla prima, che tocchi terra. Vedi
e Ly s

forto C. 7,st.92.
TANZA LXXXVL
Circa il pigliarlo,# ionoal' ho, eglit fallo:
Facciam conto ch'in braco alla pastura
Vin toro sia coftui 0 xn cavalo;
Tiriamgl addofso qualche accappiatura
Legata innanzia un bel maxzacavalla
Collocato in castel prefso alle mura,
Ond' ei si levi un tratto all'aria, e pai
Si tiri dentro,e dove piace a noi,
STANZA LXXXVIL
Buono, rispose il Re,non mi dispiace;
Ma il Cancellier di subito riprefe:
Sia detto,o Senator,con voffra pace,
Tant! oltre il poter nostro non 8 esse/e,
Li tutto (aria nulo, ef foggiace
Ad efser condennato nelle pefe,
Ed io farei flimato anc' un Marforio;
econfentir a Kn! atto perentoria,
STAN-

 
  
   
 
 
 
 
   
 
 
 

we
308

ae NZA ieee
Perché fempre de ire i

y slrapacs 4 ete 5m rAgion®y
Pei Sella è in morayvienfi a ua! inibitay beg upalerele
E non giovando, alla comminazione
Ch' in pena caschi delle forche a vita y
E se la parte innova lefione y

Aller puo condennarsi, havende ofate
Di far causa pendente un' attentato >
Plutone, ridendo con gli altri della coveaaaeens

fecondo, che viene nea pancata,nominata Barbariccia, cheidica i)
e questo propone che si tiri un laccio a Baldone} € per vid d'un
s'alzi, ¢ G porti dove pil piacera; ma cié:non ea C
de Piutone ordina al ter2o nominato Calcabrinal 3 dica il suo parere
fiui si rizza,, ¢ fa riverenza al Re per far il discorso, meer

 
   
   
   
      

ti Ottave.
' SCHLANT ARE, Denes » spezzare detto da. Splenare, BB
lo, che fidisse sopra C, 3, st. 5. A

BESTIA incancara, Così diciamo per e(primere n*huomo feo 9
traslato da quelle beflie, che alle volte conducono. con Joro.i Monts
quali essi fanno far molti giuochi, ¢ dicono che tali: beftie-fieno:
operino per vie diaboliche. Si dice be/tia smcantataa und di poca confi
ed avvedimento, come il Lalli En, Trau. C. 20. 56.
Così gridammo, e con.la propria appa
Ci deffimo in ful pie beftie incantare
COAL entra con l'afsedio, Significa come s' accorda, 0 che i che re
I afledio,
IL bel di Roma, Così diciamo per intender apertamente c... 5
Roma intende il Colofseo, da noi corrottamente detto Culifeo,
SINTIRIZZA, Si vizza,si distende in fu la — 'EB' un' atto,¢l:
ta una certa superbia, e prefunzione di se stesso y ed & quella prefopopea' py ches:
dicemmo sopra C, 1, st. 72. 5 a eae
NON si verra a capo dé tal negorio, ec. Non si conchiudera», 0: terminéra if nt
gozio. ne woagiher
REST ATI come mosche fenxa capo, Cioè senza oe direzione '0g
Senza fapere che cosa havere a fare., 0 risoluere: i infect fo
capo, $' aggirano inutilmente, strascicando il sane di bey
dove.
ALLA sfilata, Senza ordine; confulamente, ¢ senza andare in ila,
nanza: Sbandati ' so29
S' 10 non V ha, ezli éfalle. Yo fon ficuro di pigliarlo. Seionom lo-p
per errore, E' specie di giuramento vantatorio, come: appease
forto C..8. fan. 72. & mio danne che vedremo C, 10, stan. 49.
ACC APPLATVRA, Vna fune accomodata,, € —- cay
do, che — y ibqual nodo firdice-cappio scorfoio.,

   
 
 
   
   
   
  
   
  
   
  
   

   

page e st FER w Gees = FL TFAF EE

 

=>

 
 

  
   
    
   
 

- SESTO'CANTARE: 309

MALZAC AVALLO', B un corrente, o pertica grofla congegnata per tra-
——-yerfo, come: acavallo 'un legno ritto; la quale's' alza-da-una parte
'con tirare a la parte | » E questo ordingo ¢ usato assai ne i piani di

Firenze per cavar I" dai i. [ Latini lo differo rolenonem a toliendo,
dete Smiles quella awbiid, della quale si servivano i nostri antichi as
Acagliar pictre:chiamata Azangano.. Livio dice: ariere Tollenonibus Inbramenta
i iy aur fa » sp vobuffos incuriebant, sta hina milirare
fien descritta da Vegezio così; Tollenc dicitur, quoties una trabs in terram praatee
a ry cui in fummo verrice alia transuers4 trabs longior, dimenfa medietate, 6on~
neititur, €o —— 5 at si unum capue-deprefjeris, alina erigatur, L' antico Vol-
“garizzamento lralenoé detto, quando una trave alta si ficca in terra, alia quale nel
J una altra trave pik lunga per lo traverso, enel meyxo mifurata, si com-
«mete in tal modo che se! wno capo si china, l altro in aito si leva. Da questa voce
-alvaleno ( Lat. toileno)si dice ? e4italena giuoco, che i ragazzi fanno con due travi
* incrociate, ¢ bilicate l'una sopr' all' altra a foggia di Mazzacavallo. Vedi sopra
Ga, stan. 48, Mattio Franzefi contro alle sberrettate dice.
6 Biggetnslo- Ma chi trovalfe il modo a bilicalle,
ee
os

 

Ma» 2 Sarebbe un [chifanoia, e faria bene
vail, Van contrappefo d! un maxzacavallo,
SIA detto con voftra pace. Perdonatemi; s'io v' offendo in dirlo, Non vi adi-
wvioffendete, io lo dico. Frafe de' Latini Pace rua hoc dicam, Nell'
igen di Quinto Catulo, Pace mibi liceat, Caleftes, dicere veffra. Adortalis
sue pulerior se Deo, Che Annibal Caro nel primo Sonctto delle sue Rime vol-
CO olfimi j ¢ *icontra a lei mi parue oscuro, Santi Nums del Ciel, con vofira paces
—— LO vieme', che dianzi era si bello.
(2 —--—« BSSER condennari nelle spefe'. Cioè buttar via sa fatica, e il denaro, oleum, &
«Opera perdere. Ma propriamente ¢ffer condannato nelle /pefe vuol dire, quando
=“ UNO!Per aver litigaco wna cosa ingiulta, ¢ dal giudice condannato a rifar cute
le spefe all avveriario; ¢ però quelto Cancellicre dice, che noa vuole acconfenti-
8a tale'atto-per'efiere ingiufto 5 ¢ da efer condannato nelle spefe.
Ss imato'un Adarforio, Sarei stimato un' huomo senza sentimento,o giu-
dizio; come @ la'starva di Marforio in Roma,
' ATTO fruffratorio  Awo vano, fatto senza proposito, E questo termine,
come tutti gli altri-delle (eguenti stanze 88. ¢ 89. fon termini Curiali jche veaca-
do dal latino', ed eflendo praticati in cucti li Tribunali d' Italia non-dubito, che

 

,
,  farannointefi da'ognuno; però ne tralalcio la spiegazione.
, ~ STANZA LXXXX. STANZA LXXXX1L
E poi ha fatte riverenze in chiocca Aa in vece di quel cappio da beltresca,
C0 fuivi più Lindi-a pianta di pattona, Ch'é il:toffice de ladri, si prouuegga
Si foffia it nafo ye [parzafi la bocca y Pua bilancia, 0 rete per La pesca;
- Epostain equilibrio la persona Con una lnnga fune, che la regea 5
© Come quel che si pensa dar' in.brocca E perch' sl fatto meglio ci riesca
Tutto sfromato dice: Alta Corona, Si ringa tutta, accio che non si veega 5
Circa Pordingo, pur si merra in opra; Einverra quanto ell' apre, ivi fispanda,
© Perch*ioconcorrose affermo quatofopra, Fino-ch' +l porco vengane alla ghianda,
Bate tit: STAN-

 

 

 
OO EE Ls

 

 
 
 
 
 
 
 
  

310

STANZALXXXXL...,
Perché 8 ¢ muovyon |' armi,di ragione
(Se dal capo l efercita ¢ condotto )
Annan a tutti marcerd Baldone,
E quand' ¢i giunga,ed ha la rete sotto,
Fate che lefie allor fien pitt persone
A farla tirar fu con l'avannotto,
Operando in maniera,ch' egli infacchi
tn lnogo, ove si vede il fole a feacchi, Lodando il fa
S.T.AN, ZA \ LXSXXIVi 5 copies
'Qui, dice il Re, si da fempreinbudelia, Gli ha fempre pik ritorte cl
'Siche mi cascan le braccia, ef ovaiay Mace' non locredes'einonvaals
Mentre cofini a ogni cosa appeila,; i
E co' /uoi punti mena il can per l' aia;
li terzo Diavolo, che ¢ Calcabrina, dopo haver fatta rive
mano di smorfie, come fanno certi Oratori affettati, dice, che app
cavallo, ma che in vece del cappio scorfoio piglierebbe una rete da
il Cancelliere s' oppone; onde Plurone sgridando il medesimo Canc
al quarto Diavoilo, che ¢ Cappelluccio, che dica il suo parere. at
IN chiocca, In quantita grande, in abbondanza, un diluvio di rive!
PATTONA. Specie di pane fatto di farina di castagne, che per ¢
più di figura lunga, s' aflomiglia a un piede mal fatto di un' huomo,
da, Prolusione Plautina prima dice: Qui enim pedibus fant planis ploti:
che piede di parcona si pud dir plotus dalla voce Latina Plautus, che fig
fo; ¢ questa dal Greco Plarys lato, largo; donde noi a tali huomini,
i piedi malfatti-diciamo Pileri. Vedi sopra C. 4. st. 17. li Franzefe dice P:
Spagnuolo Pata il suolo del più di bue, gatto, oca, ¢ simili; dail Gr, Parei
vuol dire battere col pié; calpeftare; calcare; EB Patdn similmente in
2 il contadino, che porta le scarpe grandi, ¢ grofle, ¢ rozzameate fa
trebbe anche esser detta Partona, in un certo modo quafi Pafona, cic
pata grea s perché¢ quella a similitudine d' un pines groffolano,e
'Pattume dilie Ser Brunetto nel Pataffio quello, che oggi dichiamo 2.
spaccatura ye mescnglio di cose fracide; ¢ ClO pure cred' io, dal Greco
peltare. Ed sl pattume vien rammuricando, Il che ha qualche simili 0
Patrons, cola fordida, ¢ vile, ¢ di brutto colore, s Greci ( per dire anche q
lo sterco, perché si scarica i) ventre lungi dalla strada comunale, che dal?
firada batcuta si dice Pates; dificro dpoparema, il che pud aver dato origine al
arole Pattume,¢ Pastona,, Gli dice findi, ma per ironia, che in, vece d'
picde ben fatto, & attillato, vuol dir piede (concio,¢ mal fatto. Lindo
ja venuta a noi modernamente di Spagna; ¢ & come /enda in quella lingua Vi
da) Latino /emita, ¢ linde da) Latino mite; così indo credo che sia d i
mito, cio' limitato, aggiuftato, ben afletto, composto. Da Lindo diciamo
che Allindarsi,e Allindirfi Sp. alindarfe, '3 eng ela
$1 /offia it nafo, e spaxafi la bocca. Eipurga il nao, ¢ spura, ¢ con Ia lin
netta identi, che sono.quei lezz), che fanno moiti Ocarori, come porre in

 

     
 
   
   
   
  
   
    
 
   
     

  

    
    
    
 
    
  
   
  
   
  
  
   
   
     
    
  
     
    
     
  

   

SESTO CANTARE: gu

brie ta persona; cio' dopo haver dimenato in qua, ¢ in 1a il corpo,fermarsi in po-
fitura intir }, come ha detto nell' Octava antecedente, che sono tutte smor-

fie, che denotano nell' Oratore una sciocca superbia, ¢ prefunzione di se stesso;

ed il Poeta lo tocca col verso che segue, dicendo: Come quello che se pensa dare in,
b che vuol dire, @ima di haver trovata l'inucnzione buona, ¢ d' haver im-

; cioé dato nel fegno.

O sfrontaro. Arditamente, sfacciatamente. I) Franzefe similmente ¢f-

ERT ESCA, 0 Bertresca, o belrresca; E' una specie di cateratta, ches' alza,
abbaffa, ¢ serve per riparo di guerra in fu le torri, ein fu le mura fra uns
, ¢l'altro; © così si dice ogni luogo, sopr' al quale si falga con pericolo
ecipizio. Di qui viene il verbo berre/care, 0 bertrescare usato da molti per
ndere Armeggiare, 0 affaticarsi intorno a un lavoro, ¢ non trovar la via as
hes i per berte/es intende la forea; per similitudine delle berte/che, le quali
i di legname, che si ponevano in alto. Gio, Villani lib. 9. 114. Pers
@ il porto era tutto impalizzato, ¢ incatenato e@ di sopra di eroffo legname imber-
+ Queste bertesche, 0 torri di legname alzate fu le mura dovcano (eruire
cose a gettar pictre, onde forse € la parola pertrechor, che significa,
pre i Spagnuoli munizioni, ¢ ripari da guerra, cioé le nostre berre/che, det~
ta forse così da echar las pedras.
- BILANCTA. Specie di rete da pescare, detta così per esser a foggia di bilan-
sia; firumento, col quale si pefa la roba.
a ella apre. Cioé quanv' ella allarga per ogni verso.
"FINO a ch' il porce vengane alla ghianda, Fino a che venga a dare nella trappo-
1a; ficali al zinkello. Esintende fino a che Baldone andando alla volta di Mal-
antile dia nella rete fuddetca.
\ SIENO Iefle. Sc bene leflo vuol dir Agile. Vedi sopra C. 1, st, 11. Tuttavias
far leffo vuol dire star pronto, all! ordine, 0 preparato.
~ AFANNOTTO. Pesce piccolissimo. Voce corrotta da Vguannotto, 0 Vw
annolto 5 che significa pesce nato quell' anno: perché g#ann0,0 wnguanno vuol
ir quel anno, se bene usato folo nel contado,¢ |l'Autore se ne servc in bocca,
@un contadino forto C. 10. st. 35:1 Latini dicevano Hornus, ed hornotinus unas
colad'tn? anno. Il Poeta da nome d' avannorto a Baldone, percht dovea esser
prefo con la bilancia, che € la rete, con la quale si pigliano gli avannotti.
IN lnogo, ove si vezga il Sole a scacchi, Cie in prigione; perché le fineftre fer-

a

—

we

=
=

a

ye 'tate della prigione, battendovi i raggi del Sole, fanno a figura dello scacchiere,
g! nel luogo dove termina il loro sbattimento, o ombra dei ferri. Da queste fine-
oo r te, Ograre di ferro delle prigioni, si formo 1] verbo e4egratighare usaco
if dal Boce. Nov. 85. Tw m' hai aggratighato il cuore colla rua ribeba', Clo€ imprigionas

       

to col suono della tua rideca, come oggi diremmo: ¢ da Brunetto nel Patafiio.
TVTT At fava. Tutta è una stessa cola. Sol eff Apollo, ipfe pollo Sol, Di-
Geil Cornazzano Nov. 11. che fu una Signora, la quale yolendo riprender co+
Potomee il mario, perché la(ciando lei andava dalle Meretrici, gli fece uns
utidimo definare, ogni vivanda cra condita, ¢ ripiena di fave con diversi stra-
'Vaganti ma delicati fapori. 1) marito le domandava; Che cosa ¢ quelta? ed el.
we la

ee!

 

io.
 

 
    
   

gin MALMAN TILE?
la rispondeva; Fava, E quest' altra? Fava. In somma
gaor marito sceglieve quanto volete, perché sattae fava;: egli.
guta, e faceta riprenfione della lie, mut vita, conoscer 3
na ail' altra non pud esser' altra differenza, che quella che nasce da un for
sfrenato appetito. E di qui poi venne il dettato 7 ase ¢ fava che significa &|
anne daxke Meee a eee so of OO OM
/L Cipolla, Autore noto, che ha scritto.in Criminale.
a Plutone, che se bene quivi, e/cln/a ogni ragione Civile s* attende
Tuttavia gli Autori criminali non approvano quell' operazione
rimette dicendo; Se tu lo comandi,io non ho che replicare, € conc
anche tu Jo voletli far' impiccare, ¢ squartare; che questo iatende / i
lo squarto. Tole 4
7 ad in budella, Non si conchiude cosa di buono, Questo proverbio.
copertamente: Far come il cane de/ peducciaio, ¢ s' intende dare.in budella. | s
e(prime discorrer' aflai, ¢ conchiuder poco, ed ¢ Jo stesso che dar.in cenci
MI cascano le braccia ye? ovaia, Mi perdo.d' animo affatto.. Si dice.
cuore, le braccia, le brache, il fegato, il fiato, eda moltis' ovaia peri
pertamente è tefficoli, ¢ tutti hanno lo stesso signiticato, dl perderfi d' animo.
qui accoppiandone due, cioè /e braccia, ¢ /' ovaia, esprime perderfi affatod! a
nimo. Latino ovaria, che si (ono sCoperte ultimamente nelle donne, dagli
erano creduti, ¢ detti 1 loro telticoli. % si
AOGNI cosa appella. Non 'é cosa che Nia a suo modo,, da: difficulta a ogni
cosa,a ogni cosa ha che dire; ¢ non se ne fla, ¢ non fen' acquieta,
appeharsi termine legale. Toa
CO! suoi punti mena il can per aia, Co' suoi punti legali, e con le difficalta 5
che oppone, manda in lungo le cole senza venire a conciufione aleuna. e4its
vien dal latino area, ¢ vuol dir quel pezzo di terra spianata, ed accomodata per
batterui, ¢ mandarui sopra il grano, ¢ biade, ds
ALA piit ritorte, che faftella. Ha più ripieghi, ¢ compenfi, che non a
cidenti, che faccedono, Ovvero egli trova subito riparo a ogni accula «
si dicono-quei legami fatti di vinciglie d'alberi, coni quali si legano i falci di
legne, € di fieno, o d' altro, detti ritorte, perché quella vinciglia si attorce pet
renderla maneggiabile, ¢ fleffibile a fine d' adattarla a legare. Dan. Inf, ©. 19:
Che spexzate bavertan ritorte,¢ frrambe, et
El non lo crede, Questo termine significa Tu non ti vuoi emendare; ¢ si dices
Won crede al Santo, /e non fa miracoli; cioè non crede d' haver a efler gaftigatoyin
che ei non prova il gaftigo. Qui dice se ei non va a degnaia, cio se egli non & Ie
gnato, ¢ baftonato: Legnaia ¢ un borghetto vicino a Firenze, ed il nome di
gnaia ci scrue per esprimere legnate, o baftonate. Vedi forto C, 11, st, 116 gr
tar /a tigna., Dove fimettono diversi modi di dire per intendere Baftonar wn
CAPPVCCIO, Il Varchi Stor, Fiorentina lib. 9. dice; 11 Cappucgio ha te
>» parti: il Mazzocchio, che ¢ un cerchio di borra coperto di pango, che
»» facia d' attorno alla tefla, € di sopra, foppannato dentro di rovescio
x» tito i) capo. La 1 opsia € quella, che pendendo in fu Ie spalle, difende
»> guancia finiftra. Li Becchetto ¢ una strilcia doppia del medesimo panes

     
  
      
  
 
  

   

  
 
  

oa
a
rm

BSB eee Fa.

Pe ee ed

=

 

 
  
 
 

BRE eBas FS:

 

 
   

 
    

SESTO CANTARE Re s
ra si, in fir la spalla, ¢ bene spafio s' avvolge al collo, e» %
eller pil deftri, ¢ più, intorno alla tela, ec. EB
che già portavano le persone civili 5.¢ del quale parla il
st. 7. alla voce Adaxxocchio., ¥
STANZA LAXAXV.
che direi,0 Sire, Perch eil ha, detto.con si texfo dire,
te ch? io dica mi vien detto, Ghiioffoper dir che mais' uds tal detto;
np non ofa, ch' io non ho che dire, Pero dico.ch' a dir non mi dd il cuore 5
ir quate qui quel? altro ha derto; Elascio dire a un' altro dicitore,
ecio,, che è il quarto diavolo, fatee sue cirimonic, fa un dilcorso fen-
¢, come si yede nella presente Octava tutta di scherzo sopra il yer-
le non richiede spicgazione, ma folo rifleifione al graziofo, ed in.

STANZA XCVIIL
Valeati, dice il Re, spropositato;
S? alcuna cosa qui non bas proposta
Come vuoi tu buaccio che'l Senato
Yada in Cancelleria per a risposta?
Par fento,rispond' ¢i ych' in Mdagiftrato
Così dir s* hi ed io l'ho detto apposta;
Mas ioviscadolerxo,e alcun m'incolpa
hiandellino. Dica Baciapile, Drerrore in questo,iomeneredo in colpa,
ANZA XCVIL, STANZA XCIX,
Non occorre brunir co i labbriifaffi,
Dice Plutone, ofsaccia senza polpe,
E fare il torcicollo, e ovunque pafi
Semmar discipline,e dir tue colpe,
Ch to foyche chi per lepre tt compraffi,
Haurebbe almen tre quarti dell.
ua in mexco,bacia terra,ein fine Pera va a fieds,¢ segua il Tiritera;
7 Auago piovon discipline.. E queis' affeteaye parla intal maniera +
rende Cappelluccio, ed in tanto il quinto Diavolo, che ¢ Libicoc-
re sbocear' Arao in Malmaatile, qual consiglio ¢ riprovato co.
ile; Oade Plutone ordina al fefta Diavoio, che ¢ Baciapile,il propor.
-¢questi dice, che vadano in Cancelleria per la ri/potta, che € lo stesso che
Vi
VE

 
  
     
 
 
   
 
 
 
 
 
 
 
   
 
 
 
    
  
 
 
     

 
 
  

 
    
 
    
   
  
   
 
  

Bao SO pero Plutone lo fgrida, ed ordina al Tirirera che ¢ il settino

10 dica, ed eglis' accinge a parlare.

INE. Quel che significhi diceamo sopra C, 3, st.27. E il Latino fearra,

shine. Vn poco poco. E qui, clicndo deteo ironico signitica; ¢ un,

(pazio da Arno a Maimantile..

'ASEO, Balordo, melcato, stupido, bafofo, A questa voce allude la Pran-

Smarrite, confnfo, quafi sbafito. B far il bafeo vuol dir finger di nou in-

3 erfi huomo senza giudizio, dal verbo ha/ire vilto sopra C, 2, faa.
Reflo che far /4 carta di mafino, 0 la gatta morta, vio sopra C. 1, st. 19.

? Ipocrifia, E' ua SH ipocrito. La voce Ipocrito yi dal

© reco

>
?
+

 
 

 

THe

314 MALMANTILE §

Greco Hypocrinephai, che faona contraffare; ¢1' Ipocrifia si difinifee Vina calli.
da, ed afluta palliazione del vizio occulta; perché Ipocrito si chiama colui, che
eflendo uno scellerato, nondimeno nell' abito; negli atti,e:

d' eficr buono, es' affatica di parere quel che egli noné,¢ ep rmer iamente J rin
ta significa commediante, iitiens ~S. 'Aposbad nel Sermone da 'enerdi dopo lan
s» Domenica della Quinquagefima. Hypocrita Greco fermone fiailator ie

>» pretatur, qui, dum intus malus fit, bonum se palam oftendit.
»» faifum, crifn vero mdicium (onat. Nomen autem hypocrite translacum eft a
3» specie eorum, qui eae tecta facie inceduat, distinguentes vuleum ccerulco,
x) hivcogue colore, & cozteris pigmentis, habentes fimulacra oris lintea gypfata,
yy» & vario colore distinéta, nonnumquaim colja, & manus creta' Z
yy utad personx colorem peruenirent, & populum, dum in ludis agerent, falle.
3» reat, modo in specie viri, modo in forma feminz, & reliquis preeftigijs. I
sy Berni nell Orlando contra gl' Ipocriti, Won han'da fare lémaschere a ——
i, Questi (ciagurati sono di tre forte. La prima è di coloro, che fingono |
cospetto degli huomini d' esser pieni di religione 5 ed internameare sono ateifli,
La feconda è di coloro, che fanno del bene non moffi dalla virtù, o dall' amore
del bene y ma per efter creduti buoni. La terza è di coloro, che dimoftrano di
non esser buoni, perché altri credano, che eglino fien buoni da vero, enon,
ipocriti, In questo Diavolo si scorgono tutte tre queste specie d' Ipocriti 5 che
appreffo di noi sono lo stesso, che Bacchettoni; detto sopra C, 2, stan, 1. Dante
neil' Inf. C, 23. parlando di loro dice: ih
Laggils trovammo una gente dipinta;
Che gira attoruo assai con lenti palpi,
Piangendo, e nel fembiante franca 5 e vinta; Lai
E gui dite; i/o /morte, cioè faccia pallida, ¢ scolorita; e'dice*che pioveno
/cigline per intender uno di tali Bacchettoni falfi 0 diciamo Ipotrito. B foto
neil' ottava 99. seguente dice, Seminar discipline, che ha lo stesso fenfo. Bs' ula
assai il servirli di questi due termini per esprimere: B? paflato per questa stradas
un bacchettone. Veramente questi tali infami non ia(siane di valerGi di tute le
forte d' apparenze, ed io ne conofeo uno della prima specie d' Ipocriti, che tro-
vandofi in una pubblica adunanza, ia cavarG ii fazzoletto di talea lascid cadere
una disciplina a vifta d' ogauno; ed essendogli detto, che avvertifi', che gli era
cascato non fo che dalla tasca, egli raceogliendola 'diffe: Non @ mia roba;'
fon così buono s che io adopri tali arnefi. Di/ciplina chiamiamo quella sferza- 5
che le persone veramente buone adoprano a batterfi per far penitenza; così
dall'ammunire, ovvero gaftigare-il corpo 5 per rénderlo servo ubbidieate al fu0
Signore 5 e ben disciplinato; cioé instrutto del suo dovere, che & la fummilfione
alia ragioné. L! uso frequente della disciplina comincid in Tolcana y'¢ si diffule
per tutta Italia,e si ereflero Compagnie de' Disciplinanti,o Batcati 1 aring' 1460,
Sigonius de Regno tralia. i bx ats i
SPROPOSIT ATO, Vino che non fa, ne dice éofa a ptoposite. =.
BV ACCIO. Ignorantaccio. Che si dice anche edfiraccio y Ci 4
dual, bue di panno, Vedi sopra C, 3. stan. 49. la voce arfafarco 1 Lz:
hayevano diverse voci, che esprimevano queito stessojcome si vede in rid

  

 

 

  
 

BEG ec kf eke? oe tS eS eee

arr =

u
tty

 

Bee

 
 
 

SESTO CANTARE:; 2s

. Sc, 1, dove dice.Qui ubique unt, quifuere, quique futu-
ed Gesdepucblici-) fod, fang:; hard', Bdeani Buccones, Solus
'ante eo flultitia; & moribus indoétis, & Terent. in Heaut. 5.
haram rerum conuenit que sunt diéa in flultum:, caudex,

plumbeus.
4 plea. B' quello, che i Latini dicono wltro, confultd, ovvero dedita
ioe non per errore, Oo inconfideratamente.
angolezzo. Il verbo scandolezzo portato dal Greco al Latino, e dal Lati-
 noanoi, ha significato d' inciampare, ¢ d' adirarsi come vedemmo sopra C. 1.
- stan. 56. ¢ se gli da anche il significato di quelle parole Si oculus tuus fandalizat te
te, come è nel prefeate Iuogo » che prefo in significato attivo vuol dire: Se io vi
dé occasione di far errore: se io vi sono cagione d' inciampo; ff ribi offenfioni /um;
2 | afero y per efempio, fo credeva, che il tale fulfe huome da bene, mail fen-
pai, che ecli da a usura, mba feandolezrato, cio fatto mutare il concetto, che

 
  
   
    
   
 
  
    
 

   
 

 BRYNIR e@ labbri i fai. Brunire, parlandofi di materiali fodi come ferro,
ilo, oro 5 ec, vuol dire Dar il lustro, ¢ però intende qui dar il lustro ai faffi co
labbri, baciandoli spefio, atto, che si fa da i Criftiani devoti per segno d'u-

y sopra C. 2. fan. 9. disse dar il lustro a' marmi co i ginocchi.
ACCLA Jenza polpe. Carne cattiva ame quando si compra Ia carne, che
sia con molto offo, si dice: Vi ¢ poco del buono; ¢ da questo dicendofi a un
-huomo o/sa senza carne s' intende trifto, ribaldo, o (cellerato.

CHI ti comprafse per lepre, haurebbe almeno tre quarti di volpe, Chi ti credeffes
femplice, troverebbe poi in te tre quarti almeno di maliziofo, 0 furbo. In La-

tino fidirebbe: Pro fimplci columba, afluta vulpes. In tutta questa Ottava narra
«| -Moltedi quelle azioni be fanno gl' Ipocriti, e Bacchettoni falfi.

jp. aS STANZA

i La che sono un! infano, eignaro ogni hora, Finché lo spirto sporti al foro fora,

Bb erche faper fupir non voglio, 0 vaglio, Dond' ti fa i peti,e puted oglio,e d'aclio,
oy 'al Duca,percht a' muri ei mora Accio ' accia fu? aspo doppo atdoppi
it Tofoin tefha si dia Jie meglioun maglio, La Parca,eil porco con la fhoppa Poppi:
@  Wiiritera, che ¢ il fettimo Diavolo propone che si dia in ful capo a Baldone,
i €s'ammazzi. 1 Poeta lo fa parlare in bifticcio a imitazione del Pulci nel suo
f Morgante lib. 23. che dice. ° 3

La casa cosa parea bretta,¢ brutta
ah è V inca dal vento /a natta, e la notte y

eF

goo Stilla di flelle, ¢? a tecto era tutta,

BB. Mapnifrnte E fuina, ¢ fuena di botto una borte.
BH Pere havea pure,¢ qualche frasta frutta,
io Del pane a pena ne deste a tai dotte

 

Poseia che pesci, ¢ lasche prefe all' esca,

 

Y Lt Metta allorra alla frasca fu frefea.
MAGLIO, Dal Lat. malleus. Martello grande di legno per uso di battere i

 

.

ie

  

- Setchi alle botti, o per ammazzare i buoi, o per altri layor: di legname, nei
squali richicggano

percufioni gagliarde, ¢ gravi.
Rr z

SPOR-

 

   
   
  
 
 

316

SPORTARE. Avanzare in fuora, come avanzano le gronde de i tetti fuori
dclle muraglie delle cafe; donde Sporti quelle aggiunte che son fatte alle case
fuori del muro maeftro, e rette da' beccatelli, sorgozzoni, o colonne, (in Latino
Afeniana, che il Filandro sopra Vitruvio definisce prorette proiet pergula
dicate a Menio, Oc.) € qui vuol dire = feapps, 0 esca fuori lo io EBT

PETO. Quel romore che fa il vento stappando all'huomo dalle parti di basso.
Lat. peditus. 

ASPO. E' un baftoncello con due traverse in croce contrapposte, e distanti
alquanto l'wna dail' altra, topra vi qualei raguna il filo/per ridurlo in'
ane dal' ane pare, nape pelyeapeuaaaeal Gnindolo onde Agenina

PARCHE, Le tee donne appellate Clra, wAcropo ye Licheft, e dere
quia nemsni parcunt, five quod parce,@ pene avare vicnm eribuant. La Gi
ilimava, che quelte futicro Figliuole dell' Erebo., ¢ deta Notte., se
iVatura Deor, ¢ fecondo altri, che faflero Fighie di Demogorgone; €
figuratiero le tre cole necefiarie all' hnomo, cioé il nalcere, ii vivere
re; dicendo che una di loro detta Cioto fila, cheé il nafeere, ta se ) detta
trope annaspa, che è il vivere, la terza detta Lachefi taghia il t ce il mo
re. Le chiamarono anche Nona, Decima, ¢ Morte. 1. a Ee i

STANZA Cl. STANZA CHL —
Ben tu puxxs ai pazro ch' ¢ um pexro y Lonon fo se Baton fornia, 0} iy
Disse Pluton, beftiaccia, per bifticcio, Perché #¢i vuol,
Perch' io per me non fo, ne raccaperro Famate i conti 5e conta,
Quel che tu voglia dir neltuocapriceio, Wel zerolho frat wnaye:
Ata non fon Re, s'io non'te ne divezRo,
E perché tu non tent grattaticcio,
Mentre flima non fai delie bravate,
Quest altra volta le (aran pectiate. Trémande andranne come
STANZA Cll. STANZA CIV.
via seguite: Sui lo Scamonea Ola, dove fiam nos ( dice P
Si rizza, in vrfo tutto infauguinato, Eche Yi bite chtio
Perch' ei,ch'eun faftidiofo,appio havea Daro benvio fat a 4
Fatco a graffi con un, che "ha en allato, Si calle fiele iv cnderd it bade

Pero con la bifunta sua geornea, Guarda quel'thetu di barone
La qual traluce come Ciel frellato, E va piit'tespo ye col 'aki
Sich'ella un' Argo par fatto alla macchia, Sta ne i vermini-, e parla con gindiciy
Sinettayal Res' inchina,ecostgracchia; Che per min se ti privo dell' nfixis.
Plutone dopo haver riprefo il Tiritera, comanda, che 'dica Scamonea ottav
Diavolo;'il quale da anch'-egli un consiglio spropositato,¢ con parole eel
onde Piutone lo fgrida,minacciandolo di levarglivia degnira Senatoria, ©
non s' avvezza a parlare con termini oneiti,'¢ rilpettofi. Poe «
BISTICCIO, E: \a figura'che i Greci dicono Parecheff yed & quando si
due parole che hanno lo stesso', 0 poco differente fuono-, diverso
come si vede neil' antecedente ottava 100. ene i due primi veri
101. Detto Bificcio quali Dificcio dal Latino greco Disti¢hnm; rela
che Biforto & fatto dal Lar, diPortus; Biffemto, dal Lat,

     
  
   
   
 
  
 
 

   

 
  
 

    

£0

Oat FS OSes ten renzEesa 

  
    
      
 

    
   
   

   

   

Be RE fee as

 

 

 
 

  

 SESTIO CANTARE;? 317
»Ci0e maltrattare,efimili,. Imperciocché i primi bifticci, de' quali ci
gli Efempi'yconfiftevano in Distici, o eel dire coppie di versi

lia stessa vocesla quale significava duc cole diverse,fecondo che o piuilar-

b stretta, o intera, o'dimezzata fiyprofferiva. Fra Guittone d' Arezzo,

' Poeti'Antichi di. Mons, Allacci, tutta una Canzone va tefiendo

i diparole ed' quella che si trovaa carte 385..nelia Licenza,

qual Canzone dice cosis

   

    
 
   

 
  
 
  

   
 

ny yedo,
Sen' ake mido,

aera Edi, che prefofo,

iihies tex i912 cio vuol di tornar fo,
la in'primo Juogo vale ad banc ipfam bor am,siccome adeffo vale ad boc ipfurs
| fecondo:luogo %d ¢/savuol dire ad ¢/sa mia donnaya les, 1 fait eda,
coll secondoymeido,L, me dedo, || primo fo vuol dir/ono verbo. I fecon-
-Ne'fonoefempi in Bindo Bonichi, ed in Francesco da Bzrberino.
raccapegzo.. Non fo'ridurre'a capo: Non rinucrgo: Non rinucngo:

vo: Non intendo. \

C70. Qui vuol dire opinione, o pensiero. Vedi sopra C. 1, st, 21.
'fon Re, Laicio d'etier Re, E' termine giuratorio che esprime Tanto
“€vero che iovho fatta, © farò la tal cosa., quanto è vero cheio sono quale io fo-
#0 Non (60 Padre ui Telemaco, cioè non sono. Viifie feio non ti feufto; Dit

f 4 Terfite prefio a'Omero.

ui te ne dwexzo, \S' io non ti fo lasciar questo vizio,-0 questo tuo modo
-diteattate. E> il contrario d' avvezzare. Vengono da Vizio sens avvitiare pec
eallietare a'un vizio difuxiare per liberare da un vizio. E questi due verbi
attivi, che neutri hanoo fempre lo iteilo significato. Diciamo per efempio
'Phateeces I del tabasco', cickie/serfi afiwefarro a pigtiarne.
tem gratcaritcio.,. Twaon fai Rina de i piccoligaftighi; Tu non temi
; enon cri le riprenfioni.. Nelle Raccolce de' Greci trovafiun certo
ico', che voltato.in 'Latino fuona così:
sls (6a Tncus maxima nom timer Strepitus.
Egrattaticcio intendiamo grattatura, che leggicrmente offende la cute.
PECCIATE, Petcofie nella peccia, Caici nei ventre. Termine baffo, e più
toffo feherzofo. Peccia loiflefio, che pancia, se\bene della parte, che è dallo ito.

 
 

 
  
 
  
  
  
 
 
    
  
  

 

      
   
   
   
   
  
  
  

Maco al none Peccia pare più verso lo stomaco, Pancia più verso il petti-
“Btione, Quelta'é dal Latino pancices; inictini; quella forse dallo Spagnuolo pe-
Latino pettus, onde Rimpecciare.

| BISUNT A giornea.. Velte atiai wnta.. 'BE per giornea's' intende la sopravveRe
 Mdei soldati', che i Latini dicono Ch/amydem ye Lispigiia per vefte d' aurorita,
 donde habbiamo un proverbio che dice.
AP PIBBIARS! la giornea,, Ohe Gignifica prefumerfi molto di se medesimo. 11
'Bn,. 102, parlando di Didone dice:

Come

 

 
 
  

 
  

  
   
    
  
    
 
   
 
 
   
  

pugs. -IMALMANTILED®

Came Diana allor che xscirne acacia ys
Lungo ? Exrota, 0 pure in Cinto (nales damipcabs
Fratutte U' altre la giornea s allaccia

E fuol parer fra le sue Ninfe un fole
Il Forti, parlando della Prammatica delle donne al caps mibi 242.
parole da i libri pubblici di questa Città,dice: Wen porevane portare
© mantello 0 altro veftito sparato, ne maniche sparateso tagliate per il lunga
cia. Donde si deduce, che questa cra yna sopravvelte, oizimarra aperta t
dinanzi, usata anche dagli huomini,di conto nelle cafe..Ma da noi hoggi-
glia per toga, o vefte curiale, che chiamiamo Jucco 5 ¢ nel. presente 1uogo |

  

dir questo, *

RALVCE, Traspare; E s' intende, che era piena di buchi, perel
giunge pare un' Argo fatto alla macchia, cio s' aflomiglia a un' Argo malfa
Argo fu quei pastore, che havea cento occhi., ¢ fu lasciato.da Gi
dia d' lo figliuola d' Inaco convertita da Giove in vacca; ed a sen
miglia i buchi, che crano nella vefte di Scamonea.« Plautoy se i
chiamé casa illuftre quella, per la quale per eflere il tetto rotto 5 si vedeva il Cite
lo. Quel che voglia dire aipingere ala macchia, vedilo sopra C. 1. st. 69, doves
vedrai anche il significato di gracchiare. ew

PRAT IC A, Intendiamo Confulta, o Congreffo di Confultori dallo Spagnud-
lo Platica ragionamenio, discorso, donde Praticare um megezio vuol dir,
© maneggiare un negozio. Varchi St. Fior. lib. 14. Ragunafi la eraser
ro, che per esser la Città ferma, non faceva bifogno fare altra spefa. wa
volo credo, che intenda furbar la noffra pratica, cioè dar diflurbo a 2.3
nostra amica, perché baver xna pratica si dice quand' uno ha,o fitiene qualche» | a,
donna, 0 innamorata: ¢ corrobora questa opinione il fapere, che Baldone noms Uy
flurbava il Consiglio de' Diavoli, ne Ji loro congreffi, o pratiche, ma

=

Martinazza con aflediar Malmantile, “ae AG
L! HO nei zero. L' ho nel forame: Non lo stimo. Zero è la figura tonda: 'ay
Abbaco detta forse da Giro, la quale forma le decine, ¢ per similitudine s* inten Pie
de il torame, ¢ ci serviamo di questa parola per coprire il detto sporcost' hols | hi
c..., usatissimo fra la gente bafia in questo significato di disprezzo; equitormas | %
bene, perché dice con rasta la sua aritmetica, cioè abbaco, io/'bo nel zero, che® | ity
figura di aritmetica. 4

BACCHIO, Baltone, 0 pertica dal Latino baculus. E felleticare qui intendes
perquotere; ¢ parla ironico, perché le baftonate sono contrarie del folletico, 

NON fara in gramatica, Non fara difhcile, ¢ che ci voglia grande stadio.
Gramatica preflo gli antichi volea dire /ingua Latina, come quella', per Ji
la quale ci bifognava lo studio della gramauca. E percid la Greca anticajoyvero I
Ellinica, ¢ litterale,.che si conferna folamence nelie (crivure; a differenzadelld |\,¢
volgare, ¢ moderna, la quale oggi si parla, corrotta da quell' antica, ¢ fichiama =
Komeca, cioè Greca de' sempi baffi, ne' quali i Greci non più tennero il lor antico
nome di Aellines, ma per gl Imperatori Romani, che in Oriente avevan trasfe-
rito ! imperio Romei comincjaronfi a nominare; quelia Greca antica, dico, tr-
vafi chiamata gramatica greca; perché gli Odierni Greci per apprenderla my

fogno

2

 

 
  
   
  
    
 
   
  
   
    

 
 

SESTO CANTARE: 319
fognd di gramatica, si come noi per imparare la Latina. Nel principio dell' an-
tico ss enero ee delle vite di Plutarco si legge. Qui comincia la
di Plutarco, la quale fne traslatata di gramatica greca in voleare greco in Rodi,
B perché la Grammatica'é cosa spinofa, ¢ difficile; per questo il dichiarare,¢
re l'intelligenza di qualche fatto, o questione oscura, ¢ imbrogliata di-
sgramaticare
RACHE piene, Per la'paura si movera loro il ventre, ¢ s' empicranno le
» Vedi sopra C. 1. st. 43.
sT1CO., Vno che difiicilmente ha il benefizio del corpo.
ME paralitica, Cioè tutta tremante come sono | paralitici.
VE sia noi? Dove credi tu d' ellere ? Termine che significa Porta rispetto
jerfone sed al nega dovetu fei, Alcflandro fentendofi recitare da uno, che
distefa la Storia de' (uoi fatti, una narrazione lontana dal vero; disse allo
|5, Evdove eramo noi allora? quali dicefle: Che non ti ricordi; che io v' era
5? Altre volte significa: Che non hai gindizio? per elempio T dai cexto
tale, che non ha haver 50,, dove fiam noi? cioè dove fiam noi col cerucllo?
E si? Termine usato per indurre timore, ed ha del giuratorio; E che si,
? quali dica: Giuro che si; ch' io ti zombero, se tu nox parli meglio. Si
per fare flar a fegno i fanciulli, E che si, che io vengo cofid, e vi sferzo,
Sidice anche, Vale 0 ginochiamo, 0 stiamo a vedere, che io visferze ? Vin Poeta
moderno se-ne servi per giochiamo, dicendo:,
'ahscas E che si, padron mio, ch'io m' indovine
SD ei Del voffro andar girando la cagione ?
SCORRETT ACCIO. Huomo scorretto diciamo colui, che senza rispetto al-
Gund dice parole'sporche' ed oscene, ed indecenti in ogni laogo.
ZOMBARE. Perquotere. Bil Latino Verberare, Dal faono. Così Typro de?
 Greci sche vuol dire verbero, ¢ verbo fatto dal fuono; onde ne nacque Typanon,
=. we yi Tamburo; dal quale abbiam fatto noi Tamburare, ¢ T ambuffare;
i ympanum, Zombare. Apprefio i Greci bombes ¢ il rombo, 0 romore delle
Pappreflo i Latini bvméxs è il fuono che fa il corno. Appreffo di noi Bom-
bérda ¢\detta dal gran rimbombo neilo (pararsi;; € così tutte queste lingue si (ono
accordate 5 contraffacendo il fuono medesimo, che da cose concave ulcendo, ¢
rigitando 5 e:ampliandofi perwene all' orecchio.
&/MBOMBO. Rifaonamento, l'Eco, cio' quel fuono che refta alquanto
ug romore 5 ¢ maffime ne i luoghi cavernofi. Dante Inf..C, 16,
Gid era il loco, ove s' udia il rimbombo
by i Dell' acqua che cadea nel? altro giro
ac hi ' Simil a quel che  arnie fanno rombo,
A VA col calzar del piombe, C: ina adagio; ¢ fd nelle tue op
4 diy Governati con prudenza, Lat. A¢arura ler, Dante Par, C. 13,
yg 2 3 E questo ti sia fempre pivmbo a* piedi
4 = Per farti muover lento come buom laffo y
o
i

   
  

 

 

RADDA Ed al si, ed al mo, che tu non vedi

 

 
 

     
 
 
 
 
 
 
    
 
 

320 MALIMAN TIE BS @
(STANZA CV.: Z
S* alza Scorpione alloraye wiere da ef
ate it Corno orvibile, proposhoy
Che gli eferciti dive in fuga ha, melo.
Conforme ferive, ¢,accerta, 2 Arioffo.
Si rallegra Pluton,e dice; Adefso
Naw ci fara: dal Cancelliere opposhes
Perché ci calza bene, € certo questa
Cosa del corno a mevarper Ia testa.
STA
Vuoi forse darci qualche eceezsone,?
Stiamo in decretis; diy peta veffito;
Va ben, risponde il Sere, ch ex proponey
Cosa, che non deprava ordines9 Ti0% ognun.
Fatta che hebbe Plutone la.bravata a Scamonea,si riazo Scorpione
volo, e propole, che si pigliaiie. i| Cormo.d: Afiolfo,, il che piacque a.
per quetto si volioal Cancelliece domandandoli,se ci havewa difhicul
provo; Onde Plutone ordino, che si.faceffe il partito.. F
SOGGHIGNARE.. Mottrare, 0 far fegno di ridere quali dafu
ere per fegno di. dif

  
   

bene in:faa forza & il latino fubridere » ed ¢ un certo,
zo, 0 di poca stima, che altri faccia di, qualcofa 5. ¢ si, chiamayrifo
cio¢ non puro, non vero; ma, fate.;

JO non fon qui per candeliiere. 1o.non fon qui folamente per far
devo dire ancor'io i mio. parere, quando occorra.

DOTTOR de' mei firvali. Termine di disprezzo, e vuol dire
Vedifopra C, 4, tt 10. Aewe

PET O-veftite « Che cosa sia peto, vedemmo nell' ottava roo, d
quando.il vento esce dalle parti da baflo accompagnato con qualcofa altro,
ce peto veltito. Eda guefto il Lettore pud comprendere quel che significhi,,
SONATE un doppio, Quand' altri dopo molte cole mal fatte ne fauna bent
dal medesimo solita farsi di rado., 0 vero dopo, che uno habbia terminata
faccenda con grande stenta, ed in molto tempo, diciamo.: Sonate wm ¢i0
tutte le campane per l'allegrezza di quelta cola infolita » © della rerminagions
di questa faccenda, che si pensava non -haveffe a efler terminata may) |

F-AR il partite, Fas.loferutinio, che noi volgarmente diciamo far lo /gxitine}

 
 

© fquittinare.
STANZA CVIIIL, STANZA CIX.

Vanno le fave attorno, edi lupini-, Vauno i danyelti ognun dalla sua banehy
E fentefi fiuonato,¢ fuor di chiave Ma perché ne ricevan: Be
Alle panche gridar: Tavolaccsné 5 * Che pis neffuna ardy a it.Re comand
Raccogliete pel numero,¢ le fave Se mon vuolyche a pier popolo si sferti
Pigliate in man; che questi cittadini, Di nuovo attorno s boffoli si manda
Che in simil Luogo far dourianfulgrave Da vincerfi il partite pe' due rere
Rendano( il capo havendo pien ds baie ) E cercate alla fin tutte le panchty 
Male i partivi, e mangian le civaie. Fu vinto non oftante cana:

te

.

RPS R RE SRA HE GRE ERE REESE

 
 
 
  
  
 
 
 
  
 
 
 
  
  
 
   
 
  
 
  

Bess ~~

 

 
  

| j donzelli vanno raccogliendo i vor!
ti in contrario fu vinto, che si

lone da Malmantile. E qui ters
~ Vedil Ariofo nei suo Orlando furiofo., che lo finge uns

i fyono fugava la gente.
fave arr edi Inpini. E' coftuine in Firefze, come era anche
di fare i partiti, o (quittini con fave~, e Jupini; ¢ pero havendo il Poe-
uto, che nel ConGglio grande di Firenze chiamato il Consiglio dei Dugen-
yhel quale inte ono centinaia, ¢ cehtinaia di persone ( come in questo
Consiglio de' Diavoli ¢ neceflario, che interueniflero sopra 300, Demonj, mentre
to voti non impedivano il yincere il partito) i Tavolaccint, Donzelli van-
F endo le fave, ed i lupini a coloro, che devon rendere i) partico, fas
'il medesimo coftume nel presente consiglio de' Diavoli, dove dice che si fen-
idare spuonato, ¢ fuor di chiave, cioé in voce, che non intuona, ¢ non accorda,
) procede y perché efiendo pili d'uno, ed in diverse parti delia stanza a,
impotfibile che s' accordino nel tuono, come anche perché dette yoci
ite'fra tanta gente, che bisbiglia., il che le rend ottule, ed offulcate.
YOLACCINO, Servo,0.Donzello di Magiftrato; così detto fecondo al-
r abellio detto fopta in quetto C. st. 74., ma io credo, che i Tavolaccini,
che sono un 'numero determinato, ¢ differenti dagli altri Donzelli, fieno quelli
che al rempo della Repubbiicha stavano fempre in palazzo, ¢ servivano alla ta-
] vola de' $5. ciascuno il fu', ¢ due n' haveva il Gonfalonicre, ¢ si dicevano Ta-
# — volaccini dal servire alle Tavole; ¢ che habbiano conferuato il nome, si come si
conferua ancora J" usizio, eflendo coftoro obbligati a andare a servire alle tavole
eo in palagzo del Serenifs. G, Duca in occasione di Forefticri, 0 di Spolalizzj, ec.
ma per altro aprono ogni mattina, ¢ ferrano ogni (era le Porte della Città. °
» RACCOG LIE le fave per il numero. A fine di faper con facilica, quanti fieao
coloro, che rendono i} voto, il Tavolaccino pigia in mano'da ciascuno una fa-
va, © poi si contano., ed indicano il numero de ivotanu, equelto si dice
c i numero, E pigliano le fave in mano, enon nel boflolo, per aflicu-
-rarsi che non vi sia chi ne metta pi d'una, ed alteri il numero,
STAR ful grave. Tener il decoro, la gravita. Star favio,
HA il capo pien di baie. Sempre vuole icherzare.
RENDER jl parrito, E' quel dare, 0 mecter 1a fava, 0 lupino nel boflolo, che
si dice: dare il voto.
4 PIEN popolo, In prefenza, ed a vilta di tutto il popolo.
“BOSSOLO. Quel vafo, nel quale si metiono i voti dagli Ateniefi detto Camus,

 
  
 
 
  
   
 
  
  
 
 
  
  
 
 
 
  
 
 

   

 

4 Vedi sopra C, 1. st. 37.

ams

es

if FINE DEL SESTO CANTARE.

J i

si

h ot

; pw

i Ss \ SET-

 

 

 
 
  
 

  
  
   
 
  
 
 
  
 
 
 
  
 
       
        
   
   
    

SUSE le a
oe

a

®

= =

ARGOMENTO.
Paride dop' haver molto bevuto
Entra,' andar' al campo, in frenefia, ae
E come il fonno havea pel ber perduto,;
3 Perde nel gir di notte anche la via: te
Cade in un fofso, onde a donargli aiuto ath:
. Corron le Fate,¢ gli nfan cortefia; eee
Vien condotto sn un' antro, e per diporte e
La froria gli è narrata di Magorto.:

Sepepapeapeaye pases

joinaatuaiaa
?

Beis -gFf essete fer

STANZA L STANZAII
V Ino tempera te disse Catone, Perché se quel s marrage ne einuecebidy Me
Perché si dee berue a modo,eaverso, Ed ¢ burlato il tempo di [ua vithy =| ii

E non come cola qualche trincone, Almen Sent ilfapor di quel cb'el

Che, giorno, e notte fempre fa un verso; E tien la faccia roffa ye colorita

Ond! et si quoce,e percié ei va aGirone, Buvlar anche si fa chi va alia.

La fauola divien dell' universo, E infacea senza gusto a

E vede poi morendo in tempo breve Che'jo tien sépre bol/oyein man del

Chie ver sche chi più beve mance beve. Aquall we a Pe morir di tified.
lL

        
    
   
 

Bre aa Ee

STANZA I STANZAIV,
S? il troppo vino fa, che 'hnom foggiace Pero sia chi si mae egli, è um dappoce
etal' error di tanto pregiudizio; Chi imbotta al sayy come gli
Chi non ne beve,e quedo,a cui no piace, 5S! avvegzi a ber del vinoa peeve,
Aquefto cite dunque ha un gra gindizio; Chiei fa che Vacqua fa marcire si pally

 
 
 

 

  

Anzi che nd, sia detto con Sua pace, Aa com' io dico si vnol berne pipe;
Per c ogni eftremo finalmente ¢ virio, Basta ogni. volta cingue, 6 se

E se di biasmo ¢ degno Punose taltro Perch' egli è poi nocivo il ersncar tame,
Questo ha il vataggio al mio parer stz'altro, Com' udirere adeffo in queste C
Volendo il Poeta narrare in questo Canto l'accidente occorso a Paride
ni, per haver troppo bevuto, s' introduce col riflettere, che siccome ¢ male
molto vino, così che sia anche male il bere folamente acqua; ¢ 2 che
dovendofi eleggere uno dei due mali, sia meglio eleggere quello del ber vi,
ma pero regolatamente, et MO-

    
  
    
 

    
     
  

SETTIMO CANTARE: 323
AMODO,¢ a verso. Regolatamente, E' il latino vulgato: modis, o formis
dmipblein co: i
 TRINCONE.. Vno che beva afflai. Da Trinchen Tedesco bere, tirar git.
i sopra C. 1, st. 6. Si dice anche pecchiare nella prefeate Ottava teraa, quafi

  

'facciare il vino come fanno le pecchie, ( cioé l' api che fanno il miele, così der-
; ee me le quali fueciano il dolce da i fiori, ed i vini bianchi geac-
'tolit edaldetto verbo pecchiare si dice pecchione a uno, che beve assai; ¢ pecchione
ichiama ua' ape faluatica,e maggiore dell'altre, che fuccia il miele prodotto dail'

api da' Latini chiamato fucus.. Virg, gnauum fucos pecus a rabepibas arcent,
dice ciencare nella presente Qttava quarta. Vedi il Landino esposizione a Di.
nf, C, 9, alla parola cionca nel verso Che fol per pena ha la /peranza cionca, do-
dice, che cionco è parola Lombarda, ¢ significa moxxo, ma cioncarc in Fiorentino
mnifica difordinatamente bere; Si che quetti tre verbi trincare, pecchiare, ¢ cioncare
fanno lo flefio significato, ¢ se bene hanno del foreftiero, wuttavia sono usati in

  

os

na

ot.
"SEMPRE fa un'verso. Sempre fa la medesima cola. Diciamo Ver/o il canto
pce “4 Verso del rufignuolo, Verso del fringucilo. E da tal verso vienes
trato.

'WA AsGirone. Huomo, che gira; intendiamo pazzo. E però servendoci della
voce Girone, che ¢ un Villaggio vicino a Firenze, copertamente intendiamo
uno che fa delle pazzie, come s' intende nel presente luogo.

DIVIEN 1a favola dell universo, E' burlato da wtti. 4 ore eff omni populo, It
Lalli Ha, Tr. C. 4. 2. 78.

Son fatca ime la favola del mondo

I) Pett, Ata ben veggio or, si come al popol tutto Favela fui gran tempo. Tibullo lib,
1, Ne turpis fabula iam. Nella scrittura. Et fattus /um illis in parabolam,

CHI pitt beve manco beve, Cioé, chi troppo beve s' ammaia,e muore,e così vive
poco, € per confeguenza beve manco, cive dura a bere manco tempo di colui,
che beve poco. Marz, lib. 6. Jmmodicis brevis eff atas, © rara fenettus, che das
noi poi si dice in proverbio Poco cs vive chi troppo sparecchia. A similitudine di que-
flo si dice: Chi piie fudia, manco findia,

OGNTeftremo è vizio. Ogni eftremo ¢ male. Ogni troppo è troppo. Questa
fentenza wfiamo dirla // troppo, e il poco Guasta id gineco, Al che pare, che facciano
molto a proposito i seguenti versi di Orazio.

Eft modus sn rebus, funt certi denique fines,
nos ultra, citraque nequit confiftere rectum,
E Terenzio mettendo in Latino una fentenza d' un favio della Grecia disse; We

ee a es ee ee

quid nimis,

SENZ' altro, Affolutamente; fenz' alcun dubbio. Latino fant, procul dubio

VA ala fecchia, Beve acqua. Secchia diciamo quel valo, col quale si cavas
— da i pozzi dal Latino firwla. Vedi sopra C. 5. st. 10,
: ACCA, Per Gmilitudine diciamo facco al ventre deil' huomo;quindi dnfacs
tare vuol dir Mandar git nel ventre. Pulci Morg. C. 19. st. 137.
sot E mangia, ¢ beve, e infacca per due verri
Peril contratio/acar in iipagubelo è trarre, —— fuori,

8 2

i ee

scl

 

 
 

 

324 MALMANTILE: >

S¢lelT A. Che non ha fapore alcuno. Deh kagions ia, SR
BOLSO. Vedi — C. 3. st. 53. Graffo non naturale, con di direlpi-
ro. Cavallo bolfo i Franzefi dicono pou/if dal pullare, cioè | a f
la Jena affannata. Lucano lib. 4. Pettora rauca gerunt, qua creber anbelitus urge
Ex defetta gravis longé trabit ila pulfus. r aivye
LN man del Fifico, Col medico fempre attorno; cioé fempreinfermo, ©
CAL imborta al porzo. Chi beve fempre acqua. E' lo stello che dafaceare
to sopra. 0 aged
ANIM ALE. Intende animale irrazionale. Se bene la voce animale & generi-
ca,¢ comprende sotto di se anche |' huomo, noi ce ne serviamo per speciale, in~
tendendo folamente le beftie, fiche dicendofi a un' huomo T# fei wn! animale, in.
tendiamo Tx fei una beftia; Vat srragionevole, ae
Ss? AVEZZI, 8 afuefaccia., Vedi sopra C, 6, st, 101. Nth > Aa
FA marcirei pali, Vuol dire:il vino si guafla annacquandolo, quafi dica;
infradiciare i pali, che reggono le viti, che producono il vino; o se fara
infradiciare il vino, che nasce dalle viti., che sono pitt deboli de i pali, mentre
fon da essi foftenute, Dichiamo anche per biafimare l'ulo dell' acqua: 2? acqua
rovina è ponti: quafi s' abbia a intendere + O pepfate, se non royinera gli foma-
chi deglt huomini, che sono più deboli ! > KE
BOCC ALE. E una milura capace della meta d' un fiasco Fiorentino, Dice
cinque 0 fei boccali per scherzo,sapendo bene, che ogni maggiore bevitore non

     

  

 

bevera mai Gi gran quantita in una volta, van
STANZA V. STANZA. Vide
Omai ferra gli ordinghi, ¢ le ciabatte E Paride, ¢' anch'egii si ritrova
Chiunque lavora,e vive in sul travaglio, A corpo voto in quelle e hie}!

E difilato a cena se la batte

A cafayo dove piit gh viene il taglio.
Chi dal compagno aufo il dente sbatte,
Tanti ne vaatavernach'é unbarbaglio,
Parte alla bufea,e infin,pur che firoda,
Per tutto ¢ buona fhanza, on'altri goda,

D Amor chiarito figliod' una Lous,
Che sualiziar gli ha fatto le bufecchie y
Dice al villan:Va a coprarmi delves
Ecco fei gink y sonne ben parecelne 5
Piglia del pane ye sopra tutto areca
Buon vino fai\ non. qualche cerboueca,

STANZA VII, Abe

Eset' avanza poi qualche quattrino y

Spendilo in cacio, non mi portar reffor
Meller fine, rispose tl Conradino,
Jo torva, 8°10. ne trove, ancor corefto.

E partendo gli ride? occhioliney
Sperando haver a far un po dagrefe;
Aa, facendo i snoi conts per la via,
S' accorgeche e' non v' ¢ da far calia.

 

Deferive afai vagamente il venir della notte; fu la quale ora Paride affalito
dalla fame comanda a Mco suo contadino, che vada a comprar roba da-ma
giare,¢ da bere, ¢ per tale effetto gli da fei giuli, con ordine che gli spendas

i 5G

tuttl. e by. soi 92
ORDINGHT, Intende ogni forta d' arneGi, ingegni, machines firemsentiod'
Javorare. Diciamo anche Ordigni; anzi gli anuchi non difero algrimentis

CIABATTE,, Vuol dir propriamente scarpe vecchie, ¢ quelle scarpe all! Ap *

stolica, che usano i frati fealzi, ma s' intende anche i frammento di ma-
tcriali di coloro che lavorano', ¢ per ogni sorta di mafieriziuole veechies'¢ 6
fumate, che i Latiat diconoscrara., Z WWE

Ay

ZeRm =

— t

Bee essezeane28=

zs


Fae BF ee SSR est avEeESeF®.

 

 
 

  
     
  
  
    

SETTIMO CANTARE: 325
 VIVE in faultravaglio, Latino manribus viitum quaritat, Campa delle /uabpaccia.

'Travagliare in lingua Francefe vuol dir lavorarc, ed in Firenze pure è usato in.
6 denbo diegndns: cosa ben travagiiata in vece di ben ieleaans edi qui si di-
in vece di viver col lavoro, o con le sue fatiche, cioé di quel che si
alavorare. Petr. C, 3.
es ynque animale alberga in Terra,
Se non se alquanti.c' hanno in odio il Sole,
: Tempo da travagliare è,quantoe il giorno:
Ma poi che 'l Ciel accende le (ue Hele,
Ree rteit. 9 * Qual tornaa casa equal s' annida in felua,
ve % portent Per aver posa almen infin all' alia.
es ben per altro travagiiare yuo! dire efler' anguftiato da infermita, 0 da altro.

  DIFILATO. A diriuura: Latino re%a. Con preftezza, ¢ senza fermarsi.
ze serve anche sotto in questo C. st, 63. Varchi Stor. Fior, lib. 9.

non prima giunto a Firenze che andandofene difilato, fenra pur cavarsi git

i SE la bate..Se ne va via. E' termine assai usato fra la gente bafla per espri-
ib yia, o partirfi in fretta, ed ha del furbesco batsere /a caleofa, ciok
: utter la trade, andar via, camminare, donde /frada battuta vuol dire strada.,
a 2 camminata, 0 strada di paflo. Latino via trita, Lucrezio Avia Pie-
i' Tidum peragro loca y nullins ante Trita fol, 1\ Petcarcha disse: Ogni fegnato calle,
—- Prove contrario alla tranquilla vita.
DOVE gli viene il tagio. Dove gli torna più comodo. Vedi sopra C. 2. st. 48.
' ', Senza. spendere. E' detto plebea... Si scrivono da i Magiftrati di Si.
 renee: di,commiflioni ai Miniftri forenfi, le quali da coloro, che le chieg-
g0n0, ele presentang; si pagano.a i Magiftrati, che le fanno, ed a i Miniftri,
I¢ le gicevono; ¢ quando non sono chielte, ma fon fatte, ¢ mandate per pro.
'Prio interefle di quel Magi(trato, che le fa,non vi ¢ spefa alcuna, ¢ pero afiaché
tall lettere, le quali non si pagano,si potiano distinguer da quelle, che si pagano,
fetivono nella sopra(critta ¢x ofitio, ma l'abbreviano scrivendo ex Vifo, ed i ta-
volaccini, o donzelli, che le confegnano non leggono se non ex fo, ¢ distin-
Buono queste due specie di lettere, dando a quelle, che si pagano il nome di let-
tere col diritto, cio' con la dovuta.spefa, ¢d all' altre il nome.dell' Hf, cioè fen-
za spela + Edi qui ¢ nato guefto detto a Vo, che vuol dir senza spela, ¢ serve in
gai uccafione.

ckseee=

*~

A SBATTE if dente. Cioè mangia. 3

J E un barbaglio. Son tanti, che fanno abbagliare; Non se ne pué raccorre il

f conto senza sbagliare; o abbarbagliarsi., cioè errare; dal Parpaglione, che disse-
70 gli antichi alla Provenzale;cioé dal Latino papilio;farfalla;di cui ¢ noto lerrare

® — intorno al jume.

è CILLA busca, Cercando sua ventura.. Bufeare.. Vuol dir Acquiftare, otte-

iy ere, puadagnare.. E dalla Spagnuola ha/car yenuta a noi questa voce infiemes

) SON molte altre negli nltimi cempi.

i Si reda, Simangi. Sc bene rodere si dice de' topi,de' tarli, ¢ simili. Per tutto

y Pbuena anza on' alsri gode, Voi bonum, ibe patria. Dove si tla bene, quello è

' 42)" buoa

 

 
—E

326 ' MALMANTILE *©
buon pacfe; E per ogni patfe, ¢ buona Stanza, Disse come in proverbio il Pe.

trarca. ' Sp heist tity
CAT APECCHIE, (ntendiamo luoghi orridi, inculti', € dil. Mattio
Franzefi in lode delle gotte: Alor per uscir di queffe carapecchie, N
do che pecchra & fatto da apes, apecala, 0 apicu/a così verifimilmente carapere
puo dedurfi da apex apichlus, che vuol dire piccola fommitdy ¢ cara preposizione —
Greca, la quale dice un certo ordine', o € aggiunta pet maggior forza, come si
vede nelle parole, Carafalco, Cataictto y Caruno', che differo gli antichi per =
Scheduno,¢ simili. > te
CALARITO. Aggiuftato Vedi sopra C. 1. stan. 1.Vuol dir che Amore l'ha
= accomodato, perché s' era pieno di mal di chiafio, come si disse one
in. 11. Oe ie
LOVA. Lorda; Poltrona. E' parola d'ingiuria a tina donna. E*yoce fira-
niera; ¢ yuo) dir Lupa; che similmeate gli Spagnuoli dicono soba; ¢ si
maeretrice. Gio. Vill, lib. 1. cap. 25. parlando di Romulo'; ¢ Remo allevati da
una Lupa dice: Questa Laurenza era bella, ¢ di suo corpo guadagnava come
ce,€ pero dai vicini era chiamata Lupa; onde si dice furono nutricati da lupa 3 il che
cavo egli da Livio lib. 1. /une qui Laurentiam valgato corpore lupam voratam inter
Stores putent: inde locum fabula, & miraculo datum, +:
SVALIG/ ARE. Cavar della valigia. Qui intende; gli ha fatto'confamarei
denari, perché ha/ecche se bene si dicono i ventricini del porco Boce. git
Noy. 10. Dove le femmine vanno in xoccoli [u pe i monti riveftendos pores delle lor bu
Jfecchie medesime noi le pigliamo per tasche, 0 borfe, nelle quali i tengono ida
nari. E /wasgiare propriamencte intendiamo, quando i Jadri digtrada rubano @>
uno tutto quello, che egli ha addosso; ¢ lo pigliamo per finonimo di /accheggiare;
'PARECC HIE, Numero indeterminato che esprime, Molti; dal Lat.
que, secondo alcuni: Volgarizzamento di Palladio manoferitto; Nel mele di
Marzo al cap. de ficu. Si metta sotto alle barbe parecchie pietre, *
CEKSONEC.A, Vino fradicio. L' Accademico Fiorentino incerto 5 cos! n0-
minato in una Raccolta di Rime piacevoli, che dicemmo altroye eflere il Bare
chielio, descrivendo un cattivo vino dice, Te et

 

 

Staccio non pafferebbe ne framigna
Tiant' è morchiato,¢ con la feccia miffo;
Sciroppo mi par ber, ma non di vigna; Ly
Chi ne beve non ghigna, ej
Ch' egh: è ciprigno, e cerboneca fina; ' b
Chindendo gli occhi, mi par medicina,

Brunetto Latini nel suo Pataffio disse Cerbonea.
Wel ver quest' ¢ pur nuova Cerbones Hem
Forse si dourebbe dir cerconeca, derivando questa voce da cercone che vuol dit
Fea ease at si dice cercone dal circolare, che fa il vino quando da la volta s §
fig ¥ ac SAM

 

 

 
  

" Gulbinges verso;
Hciibiekh Rife, & argutis, quiddam promifit ocellis,

agreffo. Avanzare; ma intende d' avanzo illecito, come farebbe, quan-

acomprare roba, dice havere (pelo più di quello, che ha (pelo,

quell' avanzo. Vien da i cantadini, che per rubare al padrone piglia~

f ey va non matura, ( che si chiama agreffo ) e ne fanno fugo, ¢ lo veadono.

j

s

SETTIMO

CANTARE: 327

QUESSER fine, Vuol dir Meffer si, Ma dice Meffer fine, perché fa parlare a

jun contadino: noffri fic rire toquuntur.

t occhioline. Vuol dir si rallegra. Ul rider dell' occhio forse accennd

 

    

le cravaglic de la vita,

 

\ lo termine ha lo steffo significato anche in Napoli, come si cava da lo Cun-
to deli Cunti di Gianalefio Abbattutis gior. 1. Cunto 8. dove dice; A¢oPrannole
kefrifolesco' li quale maritattero turte L autre fizlic, reftannole pure agrefta pe' gliottere

IN v' ¢ da far calia. Non y' è da far avanzi. Calta si dicono queirimafu-
a hares argeato, che nel lavorarlo cadono, ¢ si dicono calia quali calo
a ?

rz 0 dell' argento, che ridotto poi in proverbio esprime ogni forta di pic-
it Solo avanzo..
o STANZA VIIL STANZAIX.

   

Perch'egl ¢ tardi,, ed ha voglia di cena,
Poi c ogni cosa ha bell' ¢ preparato,

st v4;il pane,e ilcacto,es! vin rocac-
ep fatto un guazcabuglio nella [porta, Si frrugge, ¢ si confuma per la pena 5
un » Le quattro lire slazzera ye si lpaccia. Che ti non torna il meffo,ne il mandate;
Lialerol asperta agloria,e insis la porta Ma quand' ei vedde con la sporta piena

. eee seh av ognor s! affaccta, Giunger al fine il [uo gatto frugaso:
of EB per anticipare, il fuoco accende, O ringraziato, dice, sia Minoffe,
ae — Lavai bicchicri,e fa L altre faccende, Ch' una volta le furon buane mofe.
ie | bast S.T.AN.Z.AoX,

Chiappa le, robe, ¢ mentre ch' ¢i balecca Sbhocconcellandointanto,il fiasco shocca,
at In quocer t vova,e tl cacioch' ¢ fupendo; Econ due man alzatolo bevendo,
ih Sente venirfi 2 acquolina in bocca y Dice al villan, che nominate è Meso:

E far lagola come un faliscendo,

Hlorsit ti fo briccone, addio, io beo.

-M.Comtadino mandato da Paride a provveder la roba, ando all' Oite per fbri-
arli, compro il tutto. Paride in canto stava aspettandolo con grande anfieti;
¢ fubico. giunto, egli mefie a quocer l'uova,¢ il cacio, ¢ in ranto viato dal' im-
-pazienza, ¢ dalla fame comincio a mangiar del pane, ed a bere.
» PER la pi corta, Vuol dir per la rada pik corta; ma qui intendi per sbri-
garsi pitt prefto. -
.» PROCACCIA. Provvede. Vuol propriamente dire cercar di trovare una co-
fa, ¢ trovarla; Lat. per/equi & affequi, e(primendofi con questo folo verbo pro-
eacciare la diligenza, che s' ula in cercare, ¢ andare a caccia d' una cosa, ¢ las
fortuna, che s' ha di trovare quel che si cerca; onde poi molti dicono: buon pro-
| €aecino uno che s' ingegna per ogni maniera di guadagnare.
|. GY AZZ ABVGLIO. Mescolanza, mescuglio, 11 Casa acl suo Capitolo del
Martello di amore dice;

Nox

 

 
 
  
   
   
 
 

habe Ve > pay
  
 
  
    
   
    
  

328 MALMANTILE! ©
Non eva donna rica yo poverina yo)
Si facea d! ogni cosa un guarrabuglio —
Ogni fhanza eva camera, ¢ cucina, |
Mattio Franzefi nel (uo viaggio di Venezia dice:
Ear a una tavolata allegra cera, «
. Edi var} discorsi un guazxabuglio >
Il Lasca Nov, 10, Versarono aceto, vino', olio, fale, e farina, @ fecero un euat
lio il macgior del mondo, Dal che si cava y che questa yoce esprime mescolanza.
di cose maceriali, ed anche di non materiali; Voce composta di Guazzare', ch
 dibattere cosa liquida, ¢ di Boélire: quali da una Ricetta che dica »Guarzs,e
Bolli; fattone Guarzabuglio.: s
LIRA, E' una moneta Fiorentina, che vale un giulio ¢ mezzo, detto anche
Cofime, perché il nostro G, Duca Cofimo I. inventd, ¢ fa il primo', che' bat
in Firenze questa moneta.: 33 Sepia
SLAZZERA, Cava, conta, mette fuora', fa venir fyora'a forza', E*
furbesca, se bene assai usata. gree #
S1spaccia, Sisbriga: Si spedisce.;
L' ASPETT Aa gloria* L' a(petta con gran desiderio, con pazienza efrema,
Si dice anche a/pettare a bocea aperta.. Larus bians. wine
HA bell' ¢ preparato, Ha di già mef' all' ordine. Vedi sopra C. 3... 14 =
NON torna ne il Meffo, ne it Afandato, Non torna lui, e non manda alcunoa
dir quel che sia di Jui. Diciamo anche 4 ho mandate il.corno, dal coruo, che man-
dd Noe fuori dell' arca, il quale 'non tornd mai. pee
GATTO frugato, Così fon chiamati per ischerzo da i ragazzi i contadini.
Carus in Latino è cauto, aftuto; € con queste nome chiamafi anche il Gatto anl-
male.notas il\quale.quando.¢ fato frugato con pertiche, o con baftoni, non fas
altro, che volgerfi spaurito, € che guatare; onde vogliono alcuni, che abbia il
nome. Così i) contadino, quando scende alla Città, Dante Purg. 26,) ©
Non altriments flupido si turba o }
Lo montanaro, ¢ rimirando ammuta, ae
Quando Hx 9 ¢ faluatico s inurba, ANH
VINA volta furon buone mie, Vana volta ci tornd, Questo detto wfatissimo in
ucfto significato, vien da coloro, che flando a veder correre al palio per Jo grat
desiderio, che hanno di vedere arrivare i cavalli, speflo gridano; Eccagl se bea
veramente non sono; ma pure al fine vengono, ed allora dicono, Queffe Jie iy.
buone moffe. 4 che pafiato in proverbio; significa a terminazione di ¢
¢vento, 0 negozio. ite Ty
St balocca. Si trattiene. Si dice anche: Par' a bada, 0 badaluccare; Bi YOO Wl

usata per ibambini. Vedi sopra ©. 6. st. 32. he i, &

} = 2 ak

  

SPP eee SF

  

rescezrr=

    

STVPENDO. Buonissimo. Vedi sopra C. 6. &. 35. Cosa maravigliola, hy
perfetta, che induce stupore, ae
VENIR l'acquolsna in bocca. Si fente confumar dal? appetito, © per te

 

soprabbonda la faliva in bocca, la qual faliva ¢ caula che /e gola gh fac bi
Saliscendo, perché il gorgozzule gli va ingid, ¢ insu per inghiottir quell' umid by
E faliscendo & una strilcia di ferro, che s' adatta a lerrar le porte, apeaaia liy

 

 
 
  
 
 

  

  

'di pane,¢ mangia.

01, Onofrio; ed altri infiniti.
Tl fobriccone. Ti

'quidice Briccone per brindift,
 STANZA XI.

Così per celia cominciando a bere,
d un forfo,e dagliens il fecondo;
Fesigche dal vedere, e non vedere y

Ei ditde'al vino totalmente fondo;
tayola di poi meffo 4 federe,
| Lafeiato it. ixfee voto sopra il tondo,
Viltoffi a dieci pan da Adeo provvsffi y
Eis th momento fece repuliffs.
STANZA XII.
4 i pan dotto,e uinginlio di formaggio
» Non glittoccaron I'ngola,e s'inghiotte
Due par diferque d'uova,e da vataggio,
| Potdice; Meo spilla quella botte,
Chet'hai per Popre,e dami il vino afaggio,
To vib feafera anch' io far le mie lotte,
| Ben th'io sia bere, fis ripieno,e /uentri,
Percht mi par,ch' una latcata ¢ entri,
STANZA XIII.
URufticoche dar del suo non usa:
- Non faper, dice, dove sia il fucchiella,
~ Che per casa non v* & fhoppa ne fufa,
E che quel non ¢ vin,ma acquerello.
Civuol, risponde Paride,altra fenfa,
By itty » di canna fa un cannello,
-E, in fa ha borte posso a capo chino,
~ Con eff, pel cocchiume fuccia il vino.

 
 
 
  
 
     
  
 

   

  
      
 

SASS Ek

  
   
 
   
    
 
  

eee SE Ae

  

SETTIMO CANTARE:

con'alzarla, ed abbaffarla. In questo significato'diciamo ancora:
I » vedi sopra C, 5. stan. 62.

VCELLANDO. Diciamo sbocconceilare,
compagni a menfa, o che sia portata

SBOCC A il fiasco. Stura il fiasco, ¢ fquotendolo butta fuora il vino, che & nel-
per arlo dall' immondizie, o fiore, che vi pos' essete.
hot Bartolomeo » Ela figura Apherefis (peffo usata da noi ne i nomi
me Cecco per Francesco fatto da Cesco ( che trovafi nel Decamerones
cioé Francesca ) Menico per Domenico; cos! Lippo, Stagio, Coppo,
, Noferi, accorciarono i nostri antichi da Filippo, Anaftagio, Iacopo,

329

and' uno, mentre aspett2,
roba in tavola, piglia de'

 
 
 
  

 

 
 
   
 

Ti fo brindifi. Questo ¢ quel modo di parlare, che dicono Za.
» come accennammo sopra C, 1. st. 28, al termine:u(cir del feminato; ¢

STANZA XIV.

E perch? ¢ buono,e non di quello,il quale
E nato in fu la schiena de' ranocchi,
A Meo, che piit tofto a Carnovale,
Che per Vopre lo ferba,e/ce degli occhi,
E bada a dire; Ovvia, vi farò male,
Ma quegliche non vuol ch'ei Pinfinocchi,
Edé la parte sua furbo, e cattivo,

Gli risponde: Ob tu fei caritativo.
STANZA XV.

   

= *

Lasciami hie la bocca ascintta,
Che diavol penst tn poisch'ia ne bea?
Jo poppo poppo, ma il cannel non butta,
Risponde etieo: Po far la nostra Dea,
Che sei buttaffe, la berefti tutta,
O' diferezione s'e' cen' ¢ minuzzolo;
Paride beve,e poi gl da lo spruzolo,

STA I

Non vi fo dir se Meo sllor tarocca:

Ma l'altro, che del vin fu stpre chiotte 5
Di nnovo appicca al suo canel la bocca,
E lascia brontolare, ¢ tira sotto,

Ma tanto esclama,prega,dagli,e tocca,
Chrei lascia al fin diber già mexxo cotro,
Dicendo ch'ei non vuol ch' il vin lo quoca,
eta che chi lo trove non era un' oca,

if Patide in barla in burla bevendo, votd il fiasco, ¢ poi si mangid dieci pani,
Prova, il cacio provveduto da Meco, il quale egli prego, che gli defle a laggio

 
    

 

 

botte, ¢ Meo adduce diverse scule per non glielo dare; ondes
; re

Paride

 

 
  
 
 
 

    

330 MALMANTILE

Paride fatto un bocciuolo di canna si meffe a fucciare il vino per
= s —S cui duole il <a seatats suo 4!
ere; ma egli seguita, ¢ per farlo più arrabbiare gli sbruffa
torna a bere. Al fine già fazio, laid star di Croniede
buona cosa, e che l'Inventore fu un gran valent' huomo; ma
ber pil, per non' imbriacare. 2 ety X iy
PER celia, Voce usatitiima in Firenze, per denotare buria, feberze.
una giovane Commediante, la quale era di genio scherzolo, ¢ burlesco, ¢ face

   
 
  

 
   
    
    
 

 

 
 
    

la parte della serva; ¢ si domandava Celia.
| Ui Perfiani. Ji tno canto ¢ pik dolce d' una auelia; ea
Ma feufami, se reco io fo la Celia.

DAGLIENE un forse @c, Cioé bevi un poco, ¢ poi un' altro p
la quantita di vino, o d' altro liquore, che si pud bere senza ripigliar
Latino /orbere. ' Mee
FAs) che dal vedere, ¢ non vedere. La cosa andd in maniera, che it
mento; in un batter d' ecchio. /n stfu oculi.
DIEDE fondo al vine, Cioè vord il fiasco. Fini il vino. Dar fondo a un
fa vuol dir confumare affatto, Termine marinavesco; ¢ si dice dar fondo
la nave si ferma in porto, finito il viaggio.;
TONDO, Così chiamiamo quel piatto spianato di stagno, o d' altra}
pra il quale in tavola si posano i bicchieri.
FECE repulifti, Fini; ripuli, confumo ogni cosa, ne volle veder la
inine baffo, e usato dalla plebe.;
NON gli toccaron ? ugola. Non gli scemarono } appetito. Quando a tn gran-
de affamato si da poco cibo, diciamo: Won gli ha toccato ? ugola, ¢ ancora + Now
elt ha roccato un dente,e proverbialmente: E ffata una fava in bocca all orfo.
non palatum rigat. Vgola si dice quella particella carnofa, che pende fra le faucl
per uso di formar conucnientemente la voce. Latino wa, columella, =
SERQVA, Numero di dodici, ma si dice d' vova, di pere ye simili, che}
altro si dice dozzina. cay
SPILLA la bore. Buca la botte. Spillare si dice da spillo, che & quel
to, col quale si bucano le botti, ¢ questo forse dal Latino /picu/am, 0 pure
Spinula, Crescenzio lib. 4. c. 41. chiama /pina fecaria, € '| suo antico —
zatore, (pina fecciaia, la cannella posta nel fondo de' yafi da vino, per
uscire la feccia. $
OPERE, Coloro che aiutano lavorare a i contadini, ricevendo il p
Ic loro fatiche giorno per giorno si dicono opere, 6 opre. In Latino fimi
opera si dicono | lavoranti. + Om
VVO far le mie lorte, Voglio far le mie forze. Voglio pigliarmi tutte lo
disfazioni possibili, Diciamo; i/ sale vnol troppe lorte, troppe invenie, troppi
troppe cirimonie: quand' uno in far' un'-operazione la vuol far con ogni
ancor che superfluo, ¢ non neceflario. a
SVENT RI. Scoppi per lo troppo mangiare, e bere. ”
VINA lartata centri, Ci stia bene una lattata. Diciamo: fare uma lat
do dopo che s' è mangiato,, ¢ bevuto beac, si fa venir in tavola nu

   
 
 
 
 
 
   
   

 
 
   
    
  
    
  
  
    
    
 

Pow eaenw Re BB OSS. fo. ee ee

 

      
 
 

 

  

 

    

tetas Hess:
Stes *

= Fs
 

SETTIMO CANTARE:? 331

nuovi bicchieri puliti. Che per altro /atrara & una bevanda fatta con zucchero,
-orz0, ¢ femi di popone, che benissimo pefti, ¢ liquefatti con acqua gli fannd

Pe

yt paflare per stamigoa, la quale si da per Jo più a' febbricitanti per rinfrescare: ed
. t PY pat gr i pad. hese! abbisto pot il nome di /attata ot fuddetto nuovo
bere f » come che vogliano intendere, che questo fecondo bere non fias
 $propositato, ne per gola, ma per rinfre(care l'ardore del vino bevuto, come fa
alla febbre la datara, la quale diciamo più comuncmente orzata.
- S¥CCAIELLO, Diminutivo di /ucchio, che vale lo Neflo. Strumento d' ac:
ciaio per uso di bucar legnami: ¢ il Latino Terebra.
NON ha froppa, ve fufa, 1 villano per non dar bere, trova (cufa di non poter
metter fa cannelia alla botce, perché non ha stoppa da avvoltare in fulla cannel-
la per adattarla al buco della botte, ne meno pud bucarla, perché non ha fulas
da turare il buco dello spillo, delli quali fui ( che per altro servono alle donne
Se sopra il filo, quando filano a rocca ) ci serviamo per turare simili
gi bucht yperché per esser ben tondi, ¢ di figura piramidale,ferran bene ogni buco,

A di pi r scula, che quello non ¢ vino, ma acquerello, che è la lavatura

   

ai
io

 

 

gi delle vinacce, ¢ serve per bevanda de i contadini, da molti detto vinello, ¢ das
2 altri mezzingo, ¢ da i Latini Lorea, 0 Lora. Ma Paride, che molto ben conosce,

che: oat sono tutte invenzioni, gli dice: C+ vxol altra feufa, ed intende; Non
~ - Mailerrd per questo di far quel che io ho in animo, cio¢ di bere.

COCCHIV AE. Quel turacciolo di legno, col quale si tura la buca di sopras
sh della botte; ¢ si chiama così anche la stessa buca. | Latini lo dicono do/ij opercu-
him.”

ie SVCCIARE. Attrarre a se l' umido, 0 fugo. Dal Latino /were.
a NATO in fu le schiene de' ranecchj, Nato ne i pantani,dove stanno i ranocchi,
j@  chenoné vin buono.
i  ESCE degli occhs. Non pud vederlo confumare + Jo da mal volentieri, Gli

»duole il veder confumar quel vino, quanto gli dorrebbe il perdere il lume degli
@ Occhi. Detto assai usato in simile proposito.
2 NON vnol che ? infinocchi, Non vuol che con le chiacchiere lo ritenga dal bere
gi — 'Tnfinvccbiare & lo stesso, che dar panzane, bubbole, 0 chiacchiere ed è il Latino Ver-
a) ha dare, 1 Lalli En. Tr. C. 4. st. 107. dice.

: - Per ch' il parlar di lei non l'infinocchi,

b OHTY fei caritativo Tu hai la gran pieta dime. E' detto scherzofo, usato in
simili congiunture, ¢ si dice + 7% hai caritd pelofa, 0 la caritd di mona Candida, che
Pe biascicava i confetti agli ammalati per levar loro la fatica,
is NON fo se tu minchioni la mattea. Non s0 se ww burli, Vedi sopra C. 4. st. 15.

 Che pensi tu mai ch' io ne bea? Quanto pensi tu, ch' io al fine ne beva. Altro~
i ve habbi: detto di questa particella mai, che altre volte afferma, altre volte
6 hega, ed altre volte significa tempo, come qui, che vuol dire, quanto pensi tu,
4 Piblesnihiximene boda; In Latino dircbbelt, Waid demim cenfer?

40 poppo Poppe « Cioé io attendo a fucciare, ma io tiro fu poco vino, perché il

Cannello ne da 3
of PVO! far la nena Dea, Esclamazione, 0 giuramento di contadini; quafi yo-
a Jendo significare /a Dea Pales. Virg. 3. ark) Te quoque magna Pales Oc, <
4 12 Te 2 SE

 
va Se

—

——

 

 

332 MALMANTILE a2, y
SE @ cen'? minwzob, Se cen' & punto. Se ei cen' è pur un poto, Ser:
to Latini nel Pataffio. lovee for pata |) me i
GLA da lo spruRzole. Gli sputa il vino nel vilo a minute sille. Sprags j
ciamo quando comincia a piovere minutamente, onde Spragzaglia offerud il Vet-
tori dirfi da' contadini una piccola quantita di ope per similicudines
7 -AROCC A. Entra in collora; arrabbia.. Voce usata in Firenze, e in
Lombardia. Francesco Negri nel suo Taflo in lingua Bolognefe, portan:
quello il verso d' un' argumento, che dice 4/ Re si turba alla novela rea, pari
4 Re al fente,¢ ¢ minza a taruccar,
SRONTOLARE. E un rammaricarsi, o dolerfi di qualche soprufo, o finifiro
vvenimento con parole non affatto esprefle, ma confule, ¢ male articolate, ¢
fra i denti, che si dice anche bofenchiare; ( Nella Valdinievole meer
to il calabrone ) Viene per avventura dal Greco Bronean, che vuol dir P
Virg. in quel verso, ove nomina i Ciclopi affaccendati a lavorare il ferro, ¢:
mini nella fucina di Vulcano. Bronte/que, Sterope/que © nudus membra y
I primo nome lo cava dal tuono, il fecondo dal folgore, il terzo dall' ancudine,
¢ dal fuoco, ~
TIRA foto, Attende, continova, (eguita a fare quella tal cola. Si9
DAGLI, ¢ tocca, Questo termine significa, fa, ¢ rifa la tal cosa, ovvero pre:
£2, ¢ riprega; ¢ si dice Dagii, picchia, € rocca. Ovvero Dagli, toca, valle
martella, i
MEZZO cotto. Quafi briaco. Vedi sopra C. 6. st, 35. s
CHE lo trovo non era un'oca. Chi lo trovd non era huomo senza ceruello, ma
un valent' huomo. Ceruel d' oca, 0 capo d' oca vuol dir huomo di poco giudi-

zio.
STANZA XVII, STANZA XIX,

Wit rn

 
  
  

Che scufa non gli pare baver, che vaglia,
Che non gli fea a viltade attribuito;
Così ribeve un colpettino, ¢ in cambio
D' andar. 4 lett s'arma,e piglia Pabio,

STANZA XVIIL

Senza lume, ne luce wia spulerza,

E corre al buio, che ne anche il vento
Non ha para mica della brezza,
Perch' egli ha in corpo chi lavora dréto;
Per la mora si ben si (candolexa,

Che dando il ¢,,interraacgni mométo,
Quanto pik casca,e nella méma pesca,
Tanto pin fentech'elt'? molle, Puta '

Poiche dai cibo,e da quel vin che fmaglia, Dopo cht ei fu cascato ye vicascatly
Si fente tutto quanto ingazzullito y Ler non sentir quel mollee frescoacert,
Risolue ritornar alla bactaglia, Che'lvino,e quato diatiavea i
Donde innocentemente s' ¢ partite, Opra di dentro si, ma non di fuera,

Giiito al mulin dal mexzin giis sbracciate
Si (ciaguatta i calzoni in quella gore
Per dopo nella casa di quel loco
Farfegli tutti rasciugar al foto,
STANZA XX,
Mentre si china dando ilc,., a leva;
Ei fece us capitombolo nelacqua -
Ona' avvien ch'una voltacilacquabews
Sopra delvinyche maiper
Quanto di buon si ¢; cheisei vplens
Lavar i pauni,jlcorpo anche rifeiacgisy
E divien lacqua si feremexegiallay
Gh' i pesci vengon sutti quanti a gall

—

Be GBB PRR E S3SsS> Pe

=

ee en EC!. Be SE we 2 ares

 

 
 

SETTIMO CANTARE 333
BiwbineeDAN ZA XXL Hoi

 
  
   
   
    

tutte a lui fon note, Lotanto si conduce fra le ruote,
ne per muotar bene sl Romano;. Che fan girando macinare il grano
ei corpo, confie fa le gore, Ben fen' avvede, e gid mette a entrata
Anna/pa cof piede,e con la mano, Di macinarsi,e fare una fRiacciata,

wide fentendofi inuigorito risoluette di ritornare al campo; ¢ così (enz' altro
si mefie in viaggio, ma fendofi infangato, volle lavare i calzoni in una go-
 4, €vicalcd dentro, ¢ se bene egli fapeva nuotare, ¢ s' affaticava per ulcir dell”
'Acqua, tuttavia conobbe, che portava pericolo d' entrar forto le ruote del muli-
no, ¢ reftarui infranto, se non gli accadeva quello, che sentiremo appreflo.
» VINO che fmaglia, Vino potente, ¢ generolo. Si dice /magliare, perché il vi-
 no nel mescerfi nel bicchiere lascia nella superficie una stummia, che fa certe cose
q come maglic » le quali il vino gencrofo rode, ¢ confuma subito; e questo disfac
que ie si dice /magliare, ¢ quando non le disfa, ¢ fegno, che ha poco spi-
ya ito. Edi quiicicchi hanno un detto: Laloccom! io, 0 vommene ? ed intendono
¢0si didomandar al compagno alluminato, il quale ha mesciuto nel bicchiere, se
quella fummia se ne va, o si trattiene, ed in confeguenza s'il vino ¢ buono, o
tattivo.,, Lasca Nov, 4. fecero uno scotto regio con quel vino, che /magliava,
AINGAZZVLLITO. Forse meglio ingazzurlito. Vuol dir rinuigorito, rin-
» 0 rallegrato di quella allegrezza, che mette addosso il buon vino.
i dice entrar in xurlo, 0 in zurro 5 corrottamente da rvzzo, ¢ questo dal Latino

 

 

&

ruere,

ANNOCENTEMENTE # ¢ partito, Dice innocentemente, perché in vero Pa-
tide non haveva errato a partirfi dal campo, poiché n' era stato cavato da colo-
TO, chelo portavano via infermo, come s'é detto sopra C. 3. st. 25.

YN colpettino. Vn' altra volta, Vn' altro poco. 1 Franzefi similmente dicono
per efempio; boire encore un cowp. Bere un' altra volta. Provarsi a bere un' altro
poco. Ad è traslato dal provarsi in gioftra.

RIGLIAR ? ambio, Andarfene. Voce corrotta da ambulo latino, che vuol dir
andare, o pur vien da amb io specie d' andatura di cavallo, con altro nome detto
Portame., pecché per esprimere andavfene diciamo Pigtiare il portante,

SENZA lume, ne luce, Afiatto albuio. Senza lume terreno, ¢ senza splen-
dor celeite..

SPFLEZZ.A. Va via furiofamentc. Parmi che possa venire da (pulare il gra-
no, che i) vento furiofamente porta via la pula, cioé i gusci del grano; 0 das

pigliare il puleggio detto sopra C. 1. st. 80. *
 “OTA, Terra inguppata nell' acqua, ¢ ridotta quafi liquida. Così appreffo
i Franzefi moire & il Latino dus, madidus, ¢ quel che noi diremmo mole,
pA MEMMA, 0 melma. Quella terra, che nel fondo:de' fiumi, foffi, laghi, ¢,
ae » tidotta liquida, che la diciamo anche bellerta per me/metta Latino Limus
  Verifimilmente dal Greco Adigma, che vuol dire miffura.
: a. Meflo in corpo. Detto plebeo. Vedi sopra la voce Gubbia-
mC, 1, st 36.
DA'mexxo in gilt sbracciato. Così dice per (cherzo, sapendo bene che sbracciato
Significa,quand' upo tirando la manica in fu fino al gomito,la(cia ignuda quella, i
' parte

ea

SSBB SE”

 

 
 

,

 

 
   
 
     
    

334

arte del braccio, ¢ non quand' uno si cava i calzoni', co!

aride, il che si dice sbracato; ma-l' Autore si serve della voce
tendere spogliato; enon è-vero che habbia a dire sbracato
corretto, non folo perché l'originale di mano dell' Autore, che
ed in un suo primo sbozzo dice sbracciato, ma anche perché fed
mezzo ix giit ' intenderebbe che ei si fufle tirato fu i calzoni fino a
enon che se gli fufle affatto cavati, come era neceflario, che,
voleva lavargli. ¢
SCLAGV-ATT ARE, Dimenare un panno, o altro simile nell' acqua.
GORA. Vuol dire un canale d' acqua, che corre, ¢ propriamente s*
quella fofia, per la quale si conduce |*acqua a i mulini per macinare
ali fofie, o gore si fanno a quei mulini, che sono in fa' rivi,o p
quali ¢ feacfita d' acqua, non essendo neceflaric a i fiumi reali, nei:
servi abbondanza d' acqua, basta un foftegno, o steccaia ( che no
scaia ) che volti l'acqua al mulino, ¢ serva per Cola, chet ae
alla quale si raguna tutta I acqua, che porta la gora. Gli antichi finiv
voci in Ora non folumeute quelle, che aveano simili udi se col Lat, co
le quattro tempora, come ancor oggi diciamo; mia anche le Bergora,
le Campora; E simili. Onde il Sannazzaro nelie Ecloghe della sua A
se licenza di dire Pratora per Prats: Gc. Si pote dunque dare benifiimo:
quef' acque così ragunate essi chiamaflero Lacora dal Lat. Jacus, € poi
a staccare la voce, ¢ dirfi La gora. Da i Jatini si trova esser tali, o si
d' acqua chiamati Euripi, e dVili, ma credo che fuflero iperboliche a
come si pud dedurre da Cic, 2. de legibus, dove dice; Dattus ag
Wilos, Enriposque vocant quis non irriferit ? E veramente ¢ cosa da rid
Euripus è nno stretto di Mare,ove è il fluflo, ¢ reflufio; Ed il Nilo ¢ de'
ri fiumi del Mondo; E quette fon fofle femplici, e laghetti, che gli
mani fecero correre infino di vino in occasione di fefte; e da cid piglio
to, che gli adulatori per piacere a' Signori, le chiamaffero Wii, ed

DANDO ile... aleva, Cioeaizando ile....ed abbafiando il ca:

FECE un capitombolo, Rivolto il corpo ful capo fottofopra; fece un
capo, rivoltandofi foctofopra. Vedi C. 6. stan. 84. '

e4 GALLA, Nella superficie dell' acqua. Dai verbo galleggiare.,
origine da galle, che sono quelle leggierissime palle, che nascono dalle
donde /eggieri, com' una galla,

JL Romano, Fu uno Stufaiolo, che infegnava nuotare alla gioventh Fiorentit
MOLTO annajpa. Annaspare vuol dir mettere il filato sopr' all' alpo”
durre i) filo in matafle, edipanare, Lat, géomerare,afhne d' adattarlo
re, dal Greco ana/pan, che vale retrabere, revellere, E da questo qu
perde molto tempo a far qualche operazione,¢ non conchiude cofadi bi

ciamo annaspare. Qui vuol dire, che egli muoveva i piedi,.¢ le mani,

ve le maui colui che annaspa;¢ si puo anche intendere che armeggiava |

naspava molto, ¢ conchiudeva poco. eal
GLA mette aentrata di far una stiacciata, Già tien per certo d' havere@

infranto dalle ruote del mulino, I caflieri, ed ogn' altro che tenga libri

%

 
    
   
  
 
 
 
   

   
 
   
 
 
     
    
    

Tg eee ee ae oe ee

 
      
 
  
   

   
 
 

es ES ee
    
 
     
   
  

  

SETTIMO CANTARE: 335
: trata,¢ uscita,mette a entrata, quando ha ricevuto il denaro; € da questo noi

mo Tien per certo, o ha già per ricevuta quella tal cosa.
Stearn ¢ a ccsatabuperr
» che il meschin gra si prefume Ognun si tenga pure il [uo parere;
andar a far la cena alle feanae 's O quelle, 0 altre, a me non fa farina,
¢ una porta,e in chiaro (ume Baftini per adeffo di fapere,
ise 2 iar conocchie, Che queste non fon beftie da dozrina;
Che le Naiadi Ninfe di quel fume, E, 8' ella non m' ¢ feata data a bere,
—— Coronate di giunchi,¢ di Elle fon Fate c' han virtù divina,
—— Corrono ad aintarlo infin c a viva E che sia il vero, fede ve ne faccia
La dove il di riluce) in falnoarrina, Li Garani feampato dalla fliaccia.
STANZA XXIII. STANZA XXV.
— Bvede all ombra di falcigne frasche 11 quale così molle, ¢ sbraculato
io Fra le più brave mufiche acquavle MU cadavero par di Mona becca,
&s == Parte di loro al fuon di bergama/che, Ch' essendo stato allor difatterrata,
we » ¢fefhe ragliar le capriole, Hlabbia fatto alla morte una cilecca;

ei Chitien che queste Ninfe fien le lasche
[Chile firene, ed altri le cagznole;
4  Lonon fo chi di lor dia pin nel buono,
i Ble lascio nel grado, ch' elle sono,

is Ah % $

jd Male Fate, che (pecie fon di pesce,

ys Edbimoilcorpo aftar nell'acqua avvexzo
st Piiche 2 a bagnate a lor rincresce,

—, cos: fradicio merz0;

Si [quote,¢ trema si,ch' io ho fopparo
Per San Giovanni il carro della Zecca,
Ementr' ei ff debatee, e il capo feroiia,
UI pavimento,¢ i circofhanti ammolla,

TANZA XXxVI.

Percio lo spoglian; ma perch? riesce,

Quéido un vuol far pits prespo,fhar un pexro,
Pertrattenerlo(mentr'hor quefia,bor quelia
L) asciuga) una conto queita novella,

Paride stava con timor d' affogare, fu foccorso da alcune Ninfe', les

meffero a spogliarlo, ed intanto una di loro contd 1a novella, che vedremo

of

;:

3 lo cavarono dell' acqua, ¢ lo conduffero alle loro stanze, dove dette Ninfe
af

MESCHINO, lafelice; Povero. E voce, che denota commiferazione.
ge ANDAR a far la cena de' ranocchi, Cioè affogare, annegare, ¢ così diven-

tar cibo:de' ranocchi.

yi CONOCCH/z, Pennecchi in falla rocca, che sono quei rinuolti di lino, o
è Jana, 0 altra materia simile, che le donne per filarla accomodano in fulla'rocca
to da esse usato per filare; Voce corrotta da cannocchie, fecondo il Fer-

rati, perché-le rocche per Jo più (ono di canna; Il Voflio la fa venire dal Lar.

, Golus; quaGi Rorpiata da colwcula.
i

~DRAPPL, Ciok quei drappi da donna, che dicemmo sopra C. 6. stan. 9.
5 ~~ CAMPEGGIAR conocchie, Sappotto che le muca di quelle stanze fuffono bian-

che,ogni cosa di

alfivoglia colore vi si discerne ben sopra, e però ( servendoti

; del verbo pittoresco campeggiare ) intende; si distinguevano sopr'a quel bianco i

,  drappi,

¢ fuentolavano, ¢ le rocche appiccate alle muraglie,

GIFNCO, Pianta, 0 virgulto noto, che nasce vicino all' acque, ed in luoghi
| Umidi, ¢ padulofi, ¢ non fa foglie, ne tronchi;ma falti,come paglia,li(ci, (en-
E 2a nodi, se non uno ia vette, dove nalce il feme. E per quelto habbiamo ua,

 

pro-

 
     
    
       
  

336 | MALMANTILE * ©
proverbio, che dice: Cercar if nodo-in ful ginnco; Lat, sodum in scpe ghree
significa cercar le difficalta, dove elle non sono. V2 POT en

PANNOCCHIE, Spighe, che si producono dalle canne, dalla ei

panico, &c, dal Latino Panicu/a, voce usata 'da Plinio y ove tratta
Carerum gracilsas nodis dispintta leui faftigio renmatur in Caxmina,
la coma,; HeeO SNS

SALCIGNE frafehe, Frondi di falcio albero noto, che nasce; € vien pil
rofo in luoghi padulofi, Lat. frondes /aligna. a

eAL fuon di bergama(che. Chiamiamo Bergama(ca' un*ballo compone' rato
di falti, e capriole, e-però dice guinre, ¢ fespe rrgliar le-capriale; vest

CHZZVOLE, Sono certi animaletti neri, che vivodo sae:
ti pancia, ¢ coda, ¢ col teinpd diventano ranocchie, € met 1¢ gambe,
cascado loro la coda,mutano colore di nero in verde macchiatoje
mo la meftola da muratori; Lac, trudla, ¢ che It Abate Baldo da Vebi
zionario sopra Vitruvio dice al suo paefe chiamarsi Cacchiara,

LE lascio nel grado ch elle sono, Sieno chi elle si voglionojio non do'
nome, che un' altro; perché cid zon fa fartmarcio' non m' importa; eT
proposito mio. E qui l' Autore moftra d' haver notizia delle diverse
Gentili circa alle Ninfe, le quali tutti concordano esser Figliuole
¢ conchiudono che le piii fuflero Deita aquatiche; le quali Deita noi”

retiamo, che ficno diversi effetti, che produce Pumidita, E che parte
infe fieno de i prati, parte de' boschi, parte de i monti, ¢ con divers
Nereidi, Napee, Oreadi 5 ec,: '

NON fon beftre da dozzina, Non fon beftie ordinarie, ¢ da farne poca Mima:
Diciamo cosa da doxzina, 0 doxxinale, quella, che ¢ lontana dalla perfe
che @ lavorata con poca diligenza. s

S' ELLA non m' ¢ frata data a bere, S ella non m' è stata data a credere] *

FATE. Vedi sopra C, 4. stan. 54. '

STIACCTA, Si dice quella trappola, che si tende con le laftre a i topi,ed ag
uccelli, così detta, perché nel cadere addosso all' animale, Jo stiaccia.
SBRACVLATO. Senza brache, e senza calzoni. 2
C-ADAVERO di Mona Checca, Si (uole in Firenze nel giorno della commen
razione di tutti i morti,ne i fotterranei della Bafilica di S. Lorenzo, che fond il
fepoltuario, esporre uno scheletro di morto con veli in tea', ed altri:
menti, ¢ questo da i ragazzi è detto Atona Checca; cioè Afadonna Fr.
questo nome poi comuncmente s' usa per esprimere uno sbattuto, ed -afflitto dale
la fame, dal freddo, ¢ da altro stento. Ariftofane portato in Latino dice: AF
bil a Charephonte differ. ww
FARE nna cilecca, 0 feilecca, Far una burla; cioè finger di voler 1
fa, ¢ poi non la fare; Sicché vuol dire: habbia finto d' efler morto', &
sia stato vero. Habbia gabbato la morte. Diciamo anche pare uit m b
rato. Ll Bini nel fecondo Capitolo dell' orto dice: 2 og ae
' Ho una vaca, ma ell! ha una pecca 1 EY
D! un certo suo turacciol benedetto,
C' ogni volta mi fa qualche cilecca,

 

  

REF FS ee 8a Ea ee

2.

 

Er

eee ae oe ee a oe

 
 
  
   

ee
1) TP Sr TIMOsCuNTUReE 337
20.

So fips. Qui bao fleffo significato, ché mb difetadé detto sopra. 1. 0.
34. ¢C. 6, stan. 61. er altro havere ffoppato uno, vuol dire H2-
negli orecchi, ec. per efempio. Tu mi hai fatto il servizio tanto tardi, che
ho havuto più bifogno, ¢ però io ¢* be fropparo.
lella Zecca, 1) giorno di S. Giovanbatifta ¢ la maggior folennita, che
ri in Firenze per efser del Santo Avvocatu y € Protettore della Città, ed
oP ome agifteati di Firenze, e tutte le Terre, ¢ Castella fubordina-
; inio fanno la citimonia dell' offerta al Tempio dedicate al detto Santo,
fra gli altri il Magiftrato delia Zecca offerisce un gran Carro trionfale in figu-
piramidale alto circa 20. braccia, ¢ nella fommita di esso Carro è un' huomo
tutto coperto di peli, legato.con func a un palo di ferro alto circa un brac-
cio ¢ mezzo, che formando in cima un mezzo circolo gli fascia lo flomaco,dove
f detto huomo,acciò non cachi, il quale rapprefenta San Giovanni nel
to. E perché tal:Carro nell eflere stra(cicato brandisce, ¢ squote, però
che ¢ nella cima del Carros' agica grandemente ancor' egli; Ed il Poeca
huomo intende dicendo, che Paride si squote più del Carro della Zecca,
Colui, che è sopra detto Carro*
CB yO ineresce. Vuol dir venire a noia, 0 a faftidio, ed 8 il Latino
Bocce, gior.5. Nov. 6. lo fari s} 5 che lavedrai tanto, che ella ti increfeerd.
gnilica haver dispiacere,c' una cosa sia fatta,, o non fatta. Bocce. Nov. detta.
Ma di cidyche facto, glincrebbe, Significa compaflionare uno, come nel pre-
ee eis questo C, stan. 50. Significa'ancora haver dispiacere inten-
dendofi eflerinelle Fate maggiore la compaflione, che havevano di Paride per ve-
derlo così mal condotto, che non era il disgufto d' esser bagnate; E sono questi
due significati tanto profimi; che speffo col folo verbo rincrescere s* e(prime.»
Punoe Valtro, come fegae qui, ¢ nel Petr. Son. 44.
dP ote Onde il lasciare el' 4/pettar m' increfee,
Rs a intendere mi pefa, mi dispiace il lasciare, ¢ mi viene a noia l'al-
pettare, Li Perfiani nelia lettera al sig.\ Principe D.Lor. disse:
Ml mio bifegno ho gra detto a parecehi
(art 's) EB ciascun se ne duole, ¢ gli rincresce
4Omexo. Coml', ¢, firetta, ¢ con una fola,z, che fa alpro ( per-
ché:con Pelarga., ¢ con due zete, che fanno dolce, fecondo l'opinione del dot-
4. Carlo Dati, vuol dire meta ) significa bagnato assai; ¢ la voce fradi.
¢io che wuol dire corrotto, qui signitica inzuppato d' acqua. La voce mezo vuol
dire una:cosa tenera per efler troppo matura, come farebbe una mela o pera, ec.
vedi sopra C, 3; stan. $3.0 una cola intenerita per haver inzuppato molto umi-
do una spugna intinta nell' acqua, e questo ¢ il fenfo de! presentes
hiogo:.. Adezo & dal Lat, mitis per maturo; ed € il contrario di acerbo, che così
chiamiamo la:frutca'n6 per anco matura.V olgarizzamento antico di Palladio,nel
mefe di Gennaio; 'tit, 15. Serbanfi le forbe:, se si colgano dure; ec. ¢ ivi comin-
)  Glanfigimmerzare. Ul Lat, dice: wbi mitescere caperint, —

       
 

 
 
 
 
  
   
 
 

      
 

    
  
 
 
    
 
  

 

2
.
"

= BBtesei ES

3
a
7

=

xe x vy STAN.

os

 

 
'
i

 

338 MALMANTILE ~~

STANZA XXVIL
Fro un tratto una dama,eun Cavaliero
Moglie,e Marito in buono,e ricco (Pato,
Che fatti vecchi, contr' ognipenfiero 5
Dopo a' haver qualche anno 'higare
La grinza pelle con il cimitero
Conuenne loro al fin perdere sl piato'y
E fenz' appello haver a far proposito
Di dar per ficw tal' offa in deposito,
STANZA XXVIIL
Lusciaron due Figlinoli è pik compliti
Che'l mondahaveffe mai sule /ue sceney
Perch' essi havevan tutti srequifiti
Donati aungalat'hnomo,e un bom dabbene;
Aggiunto che di solds eran.gremui,
(Che questo infomac quel che vale,e tiene)
Stavan d' accordo, in pace,ed inamore,
Er eran pane ye cacio, anima,e core,

gidi non usa pil.
PIATO ye piatire.

 
 

 

Perciocche il nostro
gis habent vigorem,

v

potevano piatire per La lor

La Fata principié a contare la novella (1a quale ¢,tolta'da:lo Cunto deli
ti gior. 4. Cunto 9, ¢ gior. 5. Cunto 9, ) ¢ dice: Furon già unadama, ¢ ut
valicro marito, ¢ moglie, i quali venendo a morte lasciarono due Figlit my
costumati, ¢ ricchi, i quali s' amavano grandemente l'un l'altro. Qui
fa wna digreffione, ¢ considera, che questo modo di trattacfi fra i Si

Lite, 0 litigare d' avanti a' tribunali, detto dal Latybar+
baro placitum per lite,¢ placitare;la qual voce ritengono bella ¢ intera i Veneaia-
ni, Placitum è il decreto, fentenza del giudice, 0 Magifttato, e quel che i Fran-
z¢si dicono 4rreffo secondo il Budeo da arefeein, che in Greco vuol dire placere.
Ne' Senatuscon(ulti, ovvero Decreti, ¢ Sentenze. del Senato di Roma ulayand
guefta formula: Sexatui placere &c. come si ricava da Cicerone Filippica 3. € 5+
Nell' Ordinanze Regie in Francia si legge fempre in fine: Car tel eff nostre plaifin
iacere è tale. EB nella legge si dice; che Principam le

enne poi da' Latini bafii a tirarsi questa parola'a se
il proceffo della lite medesima., si come anche éudiciwm significa Ja' fentenxa jt
4a lite medesima, che fa nalcere la fentenza. Piatire lo Spagauolo dice;
Franzefe plaider; wutti dail' iftefla fonte Latina, Il Doni.nel suo Cancelli
Sempre ne piati la rovina va innanzi, ¢ chi piatisce ha quant' ei vysle il
Ed il Varchi Sc, Fior, lib, 14. Erano affegnate le canfe delle pouere
ertd.. E poco appreflo, dice; Perché
care quel piato al terxo posefore. Ed in quett ultimi versi della presente:
27, dice metaforicamente, che.a coftoro già fatti vecchi dopo haver fatta

  
   
 
 
 
 
  
 
 
 
 
 
 
 
 
 
  
   

apt eh gS
Ce fare i es
ca è neces

E fr as de cei osre

ll contrariocoftor di chi io favelle
1 quai di cortefia furon due spe

E trattavan ciascun da buon Prac
5S' haurebbon

    
    
  
    
   
 
 
        
  
 

   

il
pene chee
bifegnsos. mi

 

  

SS = 61 PERERA EES eeeeeeee

rar lungo tempo la loro carne a i (epolcri., conuenne morire, ¢ fart

   

Il proverbio piatire i cimiseri vuol dire Esser d' eta cadente, che Luciano portal
in Latino dice: lterum pedem fepulero, 0 vero in cymba Charontis habere; Cs

noi pure diciamo; Havere il più fu la bara 0 vero il più nella foffa, —

    
 

 

GA
 

 
  
  
  
  

SETTIMO'CAN'TPARE. 339
GALANT" nemo yed huomo dabbene. Si posion dir finonimi; ma Mrettamente
galant' buomo vuol dire huomo di garbo, ¢ come dicono i Franzeli, ones" uomo;
oltre acid amorevole, ed alla mano, ed huomo dabbene vuol dire huomo di co-
»huomo d' anima, ¢ che fa opere buone. Spagn, hombre de bien. L' uno
I altro comprendono i Greci colla fola parola Caoscagathos. Calos ignitica.,
adig eh buono, da bene. “y
GR. 4. Ripieni. Bil latino Spifus.. Denfus... E qui vuol dire havevano

 

 

pieno di frutti, un luogo pieno di mosche, o simili; perché tal voce 4 do-
usare in quelle occasioni, nelle quali cade la similitudine del proprio di
Hy « Greto-vuol dire terreno'ghiaiofo, ¢ pieno di faili, come sogliono ri-
S tive de i nostri tinmi, scolata che ¢ l'acqua piovana, quali rive però
og — chi )Greto 5 come greto d'arno, greto di mugnone, ec, Ora Grero addict.

fies no di danari; se bene ¢ detto improprio,perché gremito s' intende un'

dice i) Vocabolario della Crusca, /o diciamo in significato di speo; forse daila
titudine [pela de' fuffi de' greti;e diciamo anche in queffo significats Gremito. Quan.
ame,inclinerei a credere, che Gremity dal dirli propriamente degli alberi,quan-
'ono 'picni di fiors 5.0 carichi di frutta, venitic da Greminm perciocche if
quella parte, che fuole empicrfi di tali cose. Gli antichi Volgarizza-
che i Latins dissero diteus eth traduilero greto; laonde potrebbe ad alcuno
 questa ia fatta da quella. Seneca epilt, 15; Liles repersi en littore cal-
oie ao aouisiem amet delettant c Panciulli & sthcesanc in cose di
piccol pregio, si come.tono pictre, che |' huomo truova nel viaggio, € uel gresv
del mare, ¢ ne' fiumi. Palladio nc] Gennaio tit. 14. favellando della lattuga.
Candida fieri putancur, si fiuminis arena, vel litoris frequenter [pargatur in medias,
rm E posiono diventare bianche se entra loro, ¢ intra le loro foglie spefle volte si
spatga rena del fiume, 0 del greto, Qnde a dire gremito di soldi s' invenderebbe,
= hi sopra il veftito,o sopr' alia persona sparfo gran numero di soldi,
i somMeenemito di mo/che 8 intende haver molce mosche addosso, € non nella tafea,
i" Oinicafla.; Tuttavia, sc bene. improprio, è alle volte usato, come qui,
}, ESSER pane, ¢ cacio 5 anima,e cuore, Andar' uniti, ¢ d' accordo in ogni ope-
A razione, Bene conmeniunt, © in una fede morantur,.
wht iy oterra al Sole, Se hanno mafierizie, 0 poderi; per esprimere,
6 uno che. i peep roba diciamo: // tale ha quattro cenct, © se ha beni stabili
in terreni: Egli ha della terra al Sole,
» SLAMO di si perfida cottoia, Siamo così iniqui, ¢ di mal animo, Quei legu.
Mi, che per moito che si tengano al fuoco non si quocono, ne inteaeri(cono
Mai, fidicono di cattiva cotteia  € però con dire huomo di catriva cottoia, §? in-
tende di genio maligno,¢ difficile a persuaderfi al bene. Gr, ateramon,
~ ESSER al dumcine, Vuol dire eller in cftremo di vita; € vicue dail' uso, che &
nello Spedale di S, Maria Nuova di mettere uo piccolo lume a ua Crocititio al
letto di coloro, che sono agonizzanti. Si dice an.ora; cfier alla candela,
) NON gli fovverrebbon d' un tnpine, Non gli darebbono un minimo aiuto. Sov.
yenire neutro vuol dir covalent. Non mi (ovviene, quando fu questlo. Non mi
eieanam fugqueito. Lat, mentem /ubire in mentem venire, fuccurrere, Fr,
(e fovvenir 5 2

Vv2 " eHOz.

oe

Seba thasa

STEED

  

et

=—s=

o

 

 
 

 
 
  

  
    
   
    
 
   
      
 
 
 
 
      
 
 
 
 
  

340 MALMANTILE) |

1
HOZZORECC HI. Huomo scelleratoyed infame: EB questo,perch? s
fattori, che per la tenera eta sono elenti dalla ordinari: i
fiizia contraflegnati, come dicemmo sopra C.2. stan. 3. ¢ C. 6. stan, $4. ¢ fra
gli altri contrafiegni uno è il mozzar loro una parte degli arecchi,. \. mye
. LOKT AR acqua per orecchi, Fare a uno watt i servizzi i

HAVBEBBON volute indovinare. Questo termine e(prime la
che uno ha in servir l'altro, ¢ compiacerli in tutto Veen.
STANZA XAXL sT ZA XXXL
Essendo un giorno infiemo a um conuito y E tutti quei che feggon quiviamenfa
uad'appanto agurzato hinoil mulino, Lferui 5 i circoftanti,ed f
4 mangian con buonissima appetite 5 i
4Von focome il maggtore dette Nardino
Nell' affettar il pan taglioffi um dito,
51 ch' egli infanguind st tovagtioline
E parwegli si bello a quel mo intrifo,

  

Ch ei si pose 4 guardarlo fifo fife. Lfangue:
STANZA XxXIl, STANZA XXXIV,
E refta a feder li tutsa infenfato, Che gli par di veder ymentre in

CL! ei par di legno anch'er come la fedia, i for ver
Luo is ( tanto nel wifo è dilavato ) qualche Dea di Ciele
Con la tovaglia i simili in commedia Composta colafsis di rofe, ¢, Z
E mirando quel panno infanguinato E si gli piace: y.¢ tanta ght
Hor mai tant! allegria mutain tragedia, Che finalmente mentre ch'
AMeatre nel pin bel fuon delle scodelle Vna moglie d' un tal componimene
Si vede — riposar le mafeelle, 'Non fad de i fuci di mai pil contente,
Edendo gli faddetti giovani a un conuito, Nardino, che era.il maggiore,afiet-
tando il pane,si taglio un dito, ed infanguind il tovagliolino, e nel mirae quel
bel roflo in ful bianco, s' innamoré in maniera, che si propose di non haver mai
a reflar confolato, s' ci non piglava una moglic compotta di quel colore del 10+
vagliolino infanguinato, 5 nhy
CONMITO. Definare, 0 cena splendida, Dal latino Consivixm, 0 c
da Conuitare nel fenfo che gli Spagnuoli pigliano il loro. Combidar, per fuaicare,
¢ nel quale il prefe il Boccaccio, che difie, Commie « mangiare:,. E, Conuirati alt
ravole, ii B
AGVZ ATO il muline.. All' ordine con la fame per mangiare.. Cosh eratta lt
similitudine dal mulino; dicefi Adacinarea due palmenté, c10e mulini; di ¢
preftezza, © voracita maftica da amendue i lath aun. tranoy
itanza 22. > ¢ ORE ong B
APPETITO. Vuol dir. appetenza, ¢ desiderio im generaley»ma
detto aflolutamente,¢ sen2? aggiunta,vuol dir Fame © voglia 4 0) *
giare. Vedi sopra C. 4. tt. 8, # mal che viene in bocca allagalina, > 2 sb Oi
TOV-AG LIOLINO. Quafi piccola tovaglia. Quel pezzo di panne line
tiene avanti,quando si mangia etiendo a menfa, Boccaccio disse
Jo dichiamo anche falwietta dalla voce Spagnuola. Servillera, perché serve mol
al minificro, ¢ al scruizio della tavola, hae Be = HERR,

og

 
  
 
  
 
  
 
   
 
   
  
  
  
 
  
  
 
 
  
      

#&eFRESELLCFE - Pee stsiz=

Tr

gg #E 2 se FZ.

=

2 ge (Ha
  
  
        

 

 

SETTIMO CANTARE: 34

INT RISO » La poluere } © altra materia simile stemperata con liquore, come
e:farii Wa fidice:imtrifo, ¢ intridere Ma significa ancora imbrat-

| tao, [porcato, ec. come significa in questo |
PISO fifo. 'Senza batter' oechio 5 wha greta attenzione: dntentis, inconni-

eculis. 1 Greci dicono in una parola e4/cardamytti, che @ lo stesso che:

 

 

s o irca'y

8 28S Cash vede/sio fifo,
BSB NG Come Amor dolcemente gli governa
Seles Sol un giorno da prefso,

2m 8 Senza volger giamai rota superna,'

Bp%b-on pers | We pensaffi a' alerui, ne di me Steff,

 Purcdiwg sci El batter eli occhi miei non fuffe [peffo.

- DILAV. idito. Smorto. Si dice dilavato ogni colore, che nons

' “ATO; Impallidi

-ariva alla perfezione della sua essenza: come roffo dé/avato si dice un color roto,
'che sia pitt sbiancato,e pity'chiaro del vero roflo. Latino difurus.

PPO far con la tovagiia i simili in commedia, Intende ch' egli ¢ bianco appunto
come? latovaglias Latino' non oxnm: fic ouo simile, I due simili ® wn fuggetto di

', come quello de Menechmi di Plauto, a molti vi hanno (cherzato,per-
 fecondo d' intrecci.
A, Specie di tela lina fatta a un' opera, che si chiama renfa, detta così

dalla Citta'di Rein Francia. Così 'Perpignano forta di panno dalla Città della.
Navarra di questo nome. e4razzi dalla Città d' Arras in Fiandra':¢ Dxagio al

) Boccaccio si diceva un panno, che veniva di Dovay Città di-Fiandra.,
che Gio; Villani fecondo |' uso de' suoi tempi, chiama Doagio. Latino Duacum.
Baldacchino j deappo di Levante; da Habbillonia, che i Levantini chiamano 24-
g4ady inoftrivrantichi Ha/dacco, Gio: Villani |. 7. E'me/so fuori delta Città, sopra
(a sua persona umricco palio di Baldacchini di seta ed' oro,

LENZA,olenfa, Lat, tinea, filum piscatorinm, detta così quafi dal Latino
linea, Quelia'cordicelia fatta di crini di cavallo, o di feta cruda, con 1a quale
filegaitiamo da petcare:» Franco Sacchetti Nov. 163. Egii haves prefo l'allumi-
white ale lente acscandole con 200, Fiorini a' ord, Lalca Nov. 166. Fau'nn pescatore

i puceote i e/cando con lami, e con lenze,

* Wid oer. Acetta. Pezo di tela ia larghezza det suo essere, € lungheza

4d libitum; come un telo di lenzuolo, 6 di paramento sdrucito in tutta la lun-
ta#di esso lenzuolo, 0 paramento, Diciamo ref da pane quella rovaglietra,

O.triscia'di panno lino, con la quale si cuopre i} pane in fu l'afle', Qui intende

iktovaglinolo. Te/econ I", ¢ slargo usato da alcuni in Poesia, vuol dire il dar-

do. Lat. telam..

“'GLfvaapelo, Glivaa genio. Se gli confa: ¢ fecondo il suo gusto; él) op-

posto:d” contrappelo detto sopra C. 6, itan. r. }

"2 a ete mine XKXV,. 2
da figura nel pensiero, E come chet la vegga daddovero
ieee paket, eas Divoto se le inchina y € le favelld,
Co' suci capelli-d' oro, ¢ t* occhio nero, E le promette, # egli haurd moneta,

“Che più ne men la matturing frella. Di pagarie la fiera al? Lmproneca,
: STAN.

 

ig”

 
 

 

 
 
 
    

34%
E vuol mandarle il cuore in un pasticcio y
Perch' sila se ne serva a colarione;
E gli s' interna si coral capriccio
E tanto (ene va ip contemplarione s:

Nardino s' immagina, ¢ si compone nel pensigro una'
parendogli d' haverla veramente avanti a gliocchi., le parla, ¢ se
Ie dona il cuore; ed in questa guisa s' inaamora ardentemente d' una b
maginaria. sie eye Han HOR ee IPE

PRES C.A. Trattandofi d' huomo s' intende Vino dipoca eta; ed h
donna freschi s' intende fani, gagliardj 5. di buona cera,quantunque
grave. Virg. cruda deo, viridi/que fenetius. Frefeo Secondo il Ferrari

  
 
  
      
     
    
 
  

  

ney

       

Origine dal Latino vire/cens. La marturina Hella. Virg. Qualisie
fer undis,; to

PAG ARLE Ia fiera all' Improneta, Pagarle un regalo all
giorno di S, Luca 18, d' Oucbre all' Impruncta', la quale'é una Chiefa
tana da Firenze, celebre,¢ frequentata per un' immagine miracolofa
stima vergine, che ¢ quivi; la quale in tempo di calamita, ¢ di
portata folennemente a Firenze; ¢ nella venuta di questa Immagine f
una Lauda in una Raccolta antica di Laude spiriwali,

E SE gli imerna si cotal capriccio, Gli si ficca nel ceruello, o-gli
mente questo capriccio, fantafia, opinione. Vedi sopra C, 15M. a4
S' INMAMORA come un miccio., S' innamora come un' afino
mente, perché l'afino ¢ oftinatifiimo,¢capone. 5 itt

STANZA XXXVII STANZA X
Cos} a credenza infacca nel frngnuolo,
Ma da un catoegliharagion davidere,
Che s'egli -veryc'eAmor vuol esser folo,
Rivale non ¢ qui con chi contendere,
Ma Brunettoilfratel, che n'hagra duolo,
Poich'ilsuo male alcun non pus 'coprédere.
Tien per la prima un' ottima ricetta 5
'Di rimandarlo a casa in una feggetta,
STANZA da
Ei che vagheggia fott' alle lenzuola Replica quello, ¢ feccafi la.gola:
i gentil volto,e le dorate chiome, Lo fruga, tira, e chiamalo per nome
Ne anche gli risponde una parola 5 Ed ¢i pianta una vignaye nulla
Won che gli voglia dir ne chene come, Pur tanto Caltrofa,ch' ei si rifente,
Così Nardino sianamora ardentemente senza faper di chi. Brunetto fu)
tello lo fece portare a casa, dove lo meflero in sul letto, ¢ vennero,
Speziali a vifitarlo, ma non conoscevano ne meno essi il di lui male; onde
netto si mefie a pregarlo, che gli diceffe quel che egli havea; e Nardino!
la sua contemplazione non rispondeva; pure alla fine vinto da tanti
fratello parlo nella maniera, che vedremo nell' Octave seguenti.
eA CREDENZ A, Vuol dire, quando si compra qualche mercanzia,

  
 
  
   
 
 
 
 

 

  
  
 
  
 
   
 
   
 
 

  
 

          
 
   
 
 

See aS SELB RTE SE &

ASE

=~

 

SETTIMO'CANTARE. 343
si sborfa il danaro allora, as garlo in altro tempo. Ma qui vuol
dire feniza proposito, 0 senza fon *

mento. fH Varchi nel Cap. dell' vova fede.
©) Chiba fquadrato ben la quintefenza,
“. 9) Dite ch' ella non ha color neffuno-,
© 5 Bebe quel giallo v ¢ posto a credenza.
pTr@ng) Rir67." > ° '
Contro di noi bravavano a credenza.
Questa maniera è corrispondente al graris de' Latini. Perfecuti funt me gratis, La
version Greca dice dorean; in dono, cioè di lor cortefia, senza che io il meritaffi.
INS ACCA nel frugnuolo: S' innamora; Se bene entrar nel frugnolo vuol dire
anche entrar' in collera. Frugnuolo è  Janterna; con la quale si va di notte
a cacciaagli vecelli,ed'a pescare; ed ¢ parola corrotta da fornuolo, perché tal
trnaefiendo simile alla bocca d'un forno, così ¢ chiamata.
EGLI ha ragion da vendere Gli avanza delia ragione. Ha grandissima ragione.
\ SEGGETT-.A; Seggiola portatile con due stanghe. Vedi sopra C. 1. st. 48.
 GOMITO, La congiuntura del braccio dalla parte di fuori, dove si piega a.
mezzo il braccio,, dal Latino cubirs.

VAGHEGGIA, Fa all'amore, amoreggia, con desiderio d' avere la cosa amata,
Yagguarda., come difie il Buti cittadino, ¢ Lettore Pifano nella sua lettura sopras
Dante, Vedi sotto C. ro. st. 44. Dan, Purg. C. 16.
aT A Esce di mano a [ui, che la vaghegvia,

\ S.* Prima che sia a guisa di fanciula.
Enel Parad) 10. | Eli comincia a vagheggiar nell' arte Di quel Macfire.
Fazio degli Vberti nel Dictamundi; canto 143.
ale Efe @' udirlo proprio ti vaghegsi.
(cioé feivagho; ardentemente desideri ) E canto 144.
we Bios va pur, che quanto priego', e chieggio
Al fommo bene, ¢ fol, che tofto sia

2 Vaeay Wel paefe, ch' i bramo, ¢ ch* i vagheggio.
cio' desidero, ne fon vago; col quale io fo all' amore; ea cui mi pare un' ora,
mille anni di ritornare » Vagheggiare il Ferrari deduce dal Latino viftare, frequen-
ter ae, citaa ptoposico i versi di Lucrezio lib, 1. che descrivono Marte, che

Venere. uae
—— in gremium qui fepe tuum se
'yO Reycit aterno devinitus vulnere 'amoris,
Arque ita suspiciens tereti cervice reposta,
Pascit amore avides inbians in te Dea vifus,
O:pure view da Vago, avido; perché chi ¢ avido di godere la cosa amata, va at-
torno percercarla, ¢ fi'rigira come farfalla intorno al lume della bellezza di
quella, Dante in un suo Sonetto.
. To fon se vago dela bella luce
Degli occhi traditor, che m' anno occifo,
Che la dov' io fon morto, ¢ fon derifo,
La gran vagherza pur mi riconduce.
NE che, ne come. Intendi, che non folo non gli volle dire ne il male, ne la,
Caul@ di efig, ma ne meno yolle parlare, SEC-

    
  

  

 

'a

 
344

MA LMDANMTILE ©

  

SECC-ASI la gola., Se glia icequanie fauci per. isemnan eb Li strc d
PIANT A una vigna.,. Non bada, 0-non attende.a, ice «Che noi

diciamo anche far orecchte di mercante » che &:'

titi, che lif

propongono, attento folo al, suo vantaggio., irc 57 'Ear conte che

L Imperatore 0 far conto, che uno cants.. Per il conteario schi parla
non bada, o non vuol badare, dicefi Predicare al defer

C. 10, st,

bere.
Studio iaktabat inani,

46. Jn Latino pyre.trovanGi molt: detti in quetto
Vento loqui., Surdo canere, Frufira 3 velin wannm cantare y cum pisce,
Aliam rem agere » Oc, Virg. Ech 2, tbi bec sanouiteelis

@ gente,
10 5 Predicare:
significato, come:
Lire

z ve è

 

SZrifene. Ciok si sloeglia da quella applicazicne 0 filamin unis sili

 
    

 

STANZA X STANZA XXEXIL os
Dicendo; Fratel mio, se nae mi vuoi Kedi jSoggsnnfest' altro, och' io m' adirey'
Quel benyche tu dicei volermi a faced, 0. O\par. Rpiuernred etme
Non mi dar noia,va pe' fatti.tuor y Hai tus quiftione? hai tu qualche rigire
Lerche ii mio mal non è male da biacca, Tx me 0s 4 dire in tutte le manicre,
Al quale ad ogni mo trovar non puoi  Lardin rispose, dopo on ne

Vn rimedia, che vaglia una patacca, Tu fei importuno oi pbanmal

Perch'eglie ¢ firavagante, ed alla moday,
Che non se ne rinuien Caposne coda s
STANZA XXxXX1L.

Brunetto udito il cafoje quanto e' sia

Ma da chiio devo, iro eccomi prow; 4
Così guivi-di tutto fa unragcontes.)

STANZA XXXXUL ]k

E conoscenda, c' a ridurlo in fefto

Ji suo cordogtio,anch' eidolente refea y Ci vuol'alero che il medico,oilbarbiert,
Se ben per fargli cnor moftra allegria Vifi spenda la visa,e vadailrefios. |
AMa(com'io dico)dentro ¢ chi la pefia Vuol rimediarui in tutte le ae

Perch' in veder si gran malinconia,
Ed un umor si fifo nella testa,

E sav Ff risolue pr
D andar girando il i meal kh
Jn quanto a lui gli par che la fucchielli, Di trovargli una mogtic di suo gfe,
Per terminare il giuoco a! pagrerelli, Com' ¢i gliel' ha dipinta ginfto gino,
Fratel mio, se veramente.tu,mi porti quell' affetto, che ww-dici, lafeiami Mary |
¢ non mi dir pill altro, perché ad ogni modo tw pm rimediare al mio mal
che & grandithmo, Brunctto di nuovo Jo prega, onde Nardino, vii
importunita gli racconta tutto il cafo, e Brunetto, se bene dentro ha
travaglio facea buon vifo, ¢ datogli animo si risolué d' andar girando il Mondo
per veder di trovare una donna Jecondo il gusto di Nardino,, ¢ cavarlo di quella '
frenefia.
Vna cfortazione,, ¢ richiefta simile a quella, che fa Brunetto a Nardino, fail
Ma scherone allo Gnocco meek faper fa. di hui affiizione come si vede ne i
versi dello Stef
Atto pr. Sc. pr. » ae riporto qui, perché il Letore veda, che a un' Let
rerato ( come era Jo Stetonio ) non si dildice alle volre la(ciare gli fludj pil fer
per le bizzarric fanciulle(che., ¢-spero, che non fara Ailgaee queita poca di digrel-
ne.

 
 

 

 
 

SETTIMO'CANTARE, 34y
6G NOCCHVS, ET MACCHERO:?
Ga ne Mundo traviare venivi,
pay ” Cur non tum morui, cum primim lucis in auras
- » Sborfavit genitrix ? Cur me disgratia emper
49 Perfeguitat manigolda fenem? Cur ladra placerum
gy, Abftulis', & cunctis caricas me feeva malannis ?;
7 Quando finalmentum dabitur mifura teavai ?
x9 Quando refinabis streghidima filia streghae?
» Dum me-pensabam biancain reposare vechiezam,
x Mille diabolicis straziorque, creporque ruinis.
» Vh me mefehinum | Poterit quis ferre focorfum ?
'Ma. 55 Appuntum Gnoccum video. Quid brontolas? ola !
gy Fronte malinconica, quid tecum, Gnocche, favellas ?
 Sigy-Deh poverome, pares viridas magnafle lucertas;
> 5, Tam demagratus, tam difuenutus apares.
gy Testa dolet forfan ? Sciatica ? Fiftula ? peius ?
»» Ag potius placidam flurbant penseria mentem ?
y» Dig mihi quzfo tuam (cannat quid, Gnocche, coradam 2?
3» Vade viam, Macherone, tuam. Pradele, fogare

a ~~ 4, Mevolo', nec quidquam poteris fuccurrere Gaocco.
i Ma,, Ohimé! cur sprezas fradelli verba pregantis ?
o 3» Quis scit ? parlando paffabit forte dolorus,

| gy, Praefertim-caro dum palefatur amico,

Ga 4, Deh nolis, quxfo, nolis mihi rumpere teftam:
ea 5, Deh lafia me star; fum pienus; vade bonhoram,
p » Nec des impaccium, quoniam mihi crescis afannum.
wt | May »y Deh possar mundus ! Tortum mihi facis adeffum.

io ' »» Cur mihi, Gnocche, tuum non vis sfogare'Jamentum ?
ie x» Sum pro te chi 16: prafum dic, quafo, travaium'.
eh Gn.,, Pur ibi: Vade tuum, cancar ! tu vade viagginm.

 

". » Me miferum ! ad mundum veni trascinare cordam;
oe x Mancum nonne malum fuerat non nascere, vel i
i 9 Nalcere debebam, plus prattum nascere fungus,

ib 9» Quam malé stentando scontentus vivere femper 5
più » Omnibus & giornis centum morire fiatis ?

Ma. 5. Maide ! Cordoglio (ciappas, & (pernis aitam?
ry » Vadis & ad guilam matt, Lanzique briachi ?
ia a» Infuper, & fdegnas, si quis tua vulnera-curat?

Gn, !
> a »» O bellum tempus, Machero, poca(que facendas !
oi > Otmnes'confilium femper dare novimus'aletis

i yy Sed fibi medesimis nolunt procurare parerum.

a - ay Bene dicit.vaigi proverbium: Ducere danzam,

i » 'nuces OMnes, qui fedent, bactere norunt,
ys Cum funt ad terram. Me lafits dico, malhoram.
Ma,  » Ah Zucarine meus, meus, ah Gaocchine, galantus,
a Xx

»» Quid

 
346

Gn.

Ma.

Ga,
Ma,

Ma.

MALMANTILE 4)

9 Quid facies hofti, fidedegnaris amico?) ) — wanes,
yo Cur mihi nafoonsdisy pbc vulnera
»» Non ego partibo, nifi contes ante' marezam.

x» Su, fradelle, euam crep: oaconien ieee raconta.
x» Non parlas ? Deh butta fora, meschine, venenum;
» Dic mihi, que-carpunt faftidia triftia mentem, ) 5)
5» Que lacerant cuce 5 que te fulpiria rumpunt?.
>» Nonue recordaris tirictos nos esse parentes ? HPiy
>» Eft tua mamma mee carnalis,Gaocche, forella,
x» Atque ego nawura si,non caenalis, amore
3» Sum tibi fradeiius plus quam carnalis: aitam,

3» Quam potero tibi, Gnocche, dabo,; fac denique provam eu

»» Nam abi porto beaum, nec me fradelle licenties. ai
x» Namque amo te plus quam me stessum » ane
»» Dicito cunéa mihi, nec ce meschine' fa
»» Confilium forfan potero tibi dare galantum. &

2» Quid turbulentus guardas ? fu butta deh foras;

»» Eia, valent' homus; non finghiottire bilognat;

"5 Valneris ascofti nunquam medicina trovatur; '5 7)
x» Atsborfando foras fanatur fepe dolorus; 2
»» Fiftula, qua tumuit, totos corrumperet artus, itt
» Ni lancetta viam barbieri Jefta taiaret, #
3, Sufum, Gnocche valens » cordolia dire comenza. 4b

»» O fortuna mihi nimium traversa tapino,
sy Que mihi per forzam non firappas ventre magonem;
x» Eft-ne poshbilum, quod non sborfare fiatum,

» Vnam nec potero gambam distendere voltam ?
x» Sum desperatus: volo me impiccare da verum;
x» Cerne, mei, Machero, cavezam porto fomari.
» Impiccare ? mai. Non impiccare te,non.non;
a» Mattelcis; coftat troppum impiccare: nieatum

» Tu facies. Guardes gambam ! impiccare ?,Diavol !

» Et te, meque fimul piccares, Gnocche « Gas 'orlanauting
” Maidé » quis tantum milzam tibi rodit afannus?

x» Dic, faporite meus, que te fuentura chiapavit ?

») Sime impiccabo, cunctos scappabo, travaios.
>» Pur iJluc; iftam mattezam manda malhoram.

», Sola meum stentum poterit sbandire,caveza\. Mt bed
>» Ab nimium certé te steflum, Gnocche', fafinas: ) )
5 Mancum donna timet, mancum se donna sgomentat: a
»y Ne facias cofam talem pazelcis adeflum,;

»» Incidis ia brafam cupiens evitare padellam, te
>» Qui fugiens damnum, foccorsum a Morte rechiedisy «

»> Qua nullum maius damoum reperitur inorbe, >)
» Dicas, quid peius furca maginare poteftur?, 5 «4

  

   

 

  
    

FF pe pe

=

 

 
   

 

SETTIMO CANTARE;

> Nonne vides furcas ipfos odiare (afinos y

» Millantas furcas meritant qui mille fiatis ?

x» Forse putas bellam cofam piccare feltefium ?

» Nullos audifti, nullos nec, Gnocche, latrones

93 Esse volenterum piccatos., Canchere | robbam

x Perdere, poderos,, filios, atque moieram

»» Possumus; at contum non muttit perdere vitam |

xy Parlemusd' altro, bona notte; porge cavezam,

» Fac fennum matti, caveas non talopram.

» Si fennum matti facerem, mattiffimus essem; '
y» Sum deliberatus cannam truncare una volta;

2» Nec parles; quoniam mandas tua verba Patraflum;
x» Et liquidas tentas accoglicre retibus auras;

 

347

 5» Dextra orecchia bibit, fed versat lava parolas;

y» Surdo verba canis; oleum fimul opera perdis.

»» Qui pro te robbam propriam, vitamque gitarem,

»» Pocum flimo malum pro te gittare parolas.

»» Indarnum gracchias, indarnum dico, va viam.

» Litera vis tandem ficri longissima ? Ga, Certum.

2 Et godis cortum laqueo difrumpere collum ?

» Audis, Ma. Et tandem cornacchis essere pastum ?

» Sentis. Ma. Bavofam buccam torquere ? Gn, Cofiaum;
2» Et tralunatos oculos moftrare ? Gn, Davanzum.

» Lucentem faciem, lucentia bracchia, fula

» Viscera, contradam totam peftare fetore,

2 Et vitiare diem vitiato viscere letum ?

»» Sinum; si dico, finum; volo rumpere cannam.

» Heu ipfis fugiende lupis, buttande fofatis,

» Terribilis stratiande modis, privande facrato.
» Denique penserus nullus te, Gnocche, tuorum

yy Tangit ? Cui laffas pupillos, paze, chiatinos ?

x» Cui robbam ? cui confortem ? miferofque parentes ?

» Teque finalmentum ? Cafe qui (cribitur heres ?

»» Vis proprias carnes tecum mandare Patrafium ?

»» Vis proprios natos panem cattare per uscios,

2 Disperios pueros pitocorum more per urbes 2

x» Et post de fora veniet qua fama da verum ?

»» Gloria que Caf laflatur? Respice tandem:
2» Teque, ruofque fimul, mifere miferere fameia 5
» Et miferere tui, qui proijciere fofato,

x Indignum facro corpus recoprire tereno..

2 Forfan ad Stygias ibis ? (eu forfan Acheum

»» Ibis ad Infernum ? pensa, pover' home, to feétos;
2 Pensala, dico, benum: facile eft calare dcoiium,
»» Sed montare super cancar; stentare bifognat;

Xx 2 >> Sed

 
sj

 

348

Ga,
Ma.
Ga,
Ga,

Ma.
Ma.

Ma.

Gn.

 

Sum contentus; abi, grarum fed fiascum, —
' 4 Nam stio cesta 6 rampas beulaon 6 a 4

 
 

MALMANTILE ©).

j» Sed nec stentando brutto feapulabis ab Oreo. »

»» Horfus tornemus ca(as; fu, Gnocche: cavezam

>» Cafe mitte tue. Pensas piccare? bel opram; «

»» Essere non vellem, Veneto pro boia teforo,

» At tw, te stessum si piccas, boia farabis.

» Ah tibi, ne quel, tibi fis ne boia medemo, ~~

2 Et qui pro centum mundis non essere velles;

sv Bflere pro nihilo nolis. pec porge,
ico

pocum, ff l'3
Forfitan ipfa dies (aldabie, Gnocche, fericam, — è
Dura remolicfeune paicis,& tempore forbay 99
2 Nespula dura die mivescunt, nespula dura
»» Guarda mo-, si Gnocchi poterit mitescere noia
Tu bene cicalas; dottorus, & esse videris::
Sed cicala purem; gicttas nam carmina faxis.. \ «
Al facias i » Gnocche, pl >
»» Extremumque mihi praftes, care Gnocche; favorem.
2» Quem nam ? dil, Ma, ura; facies, quod certe domando
»» Dummodo fare queam, fabo, sta fupra parolam, —
Et potes, & legrus facies. Gn. Dic ergo,quid optas
Eft mihi botazus vinetti, Gnocche, rubentis,
xs Quod difamoratis posset rubare coradam,
»» Illius humore taze cum plena plaaura cht,
Saltitat, & brillat, brillando lumina frezaty
Et rubor in vitro liquefatti more rubini,
»» Ac dicto citius spumat; hune inde dileguat
» Puri sbottigliata meri vis fernida, qualis
»» Cum foffiat Boreas, nubes sfrattare per auras
»» Cernitur, & Calum laté purgare ferenum.
ay Sat scio,, si nafum preettabis ad ante bicherum',
»» Optabis fieri totum te, Gaocche, nafonem; $ ”
»» Piccantum retinet pulerum, garbumque galantum, ©
2» Quod refucitaret mortos. De hoc, questo,pochertum «
»» Gustes, ante tuum claudas quam tofte fiaum',: sie
xy Atque mei hoc portes extremi pignus amoris, f
x Vis rechem chi 16? Gn, Reches, fed frettola paffym. - P
» Nigotta proderic, cum fim piccandus adi ys ta
»» Auamen hanc lafles, dum torno, Gnocche, cavezam, «
» Ne te gire viam tua tantum spafima cogant,
»» Et fine gustando vinum, morire, galantum',

  
  
  

     

   
 
 

      
   

 

 
 
 

 
   
 
 
    
   
      
  
 
   

   

VOLER bene a facea. Portar granditfimo affetto.. BE' f
Va pe' fatti tuoi. Cioè vattene,e bada a te, Res tuas ribi babero;

 

riti anticamente alle mogli, quando fecondo le Romane
Vedi sopra C, 5. st. 57. 7 —

 

  

ah
8S Geta
 

 
   
 
 
    
  
   
   
 

SETTIMO:\CANTARE: 349

| NON 2 mal da bideca', Non è male ordinario,e che firifani con pocdrimedio,
perché la Biacca; che ¢un biahco cavato dal piombo, ¢d è adoprato da i Pitio-
ri, serve anche per fare un' unguento buono.a poco altro, che ad alleggerire il
ng Lp erat però dicendofi.: Won ¢ mat da biacca, s' inten-
de, Manatee Sri aia:

 CHE waglia una patacca, Che,vaglia nulla. Che patacca è moneta, che. in,
Firenze non vale, Paracomé una moneta di ramevufata in Portogallo, che vale
guattrini, Cos} noi d*una cosa da noi tenuta in poco pregio, diciamo. Wo

ale un soldo, Nonne darei un soldo See

ALLA moda. Vuol dite all usanza, come vedemmo sopra C. 2. st. 54. mrs
in questo luogo vuol dire stravagante, 0 nuovo y ¢ non più sentito, 0 vilto, ¢ del
tutto infolito; Diciamo: ceruello alia moda per significare ceruello fravagance, ©
fantaftico; dal mutar che si fa tutto giorno dela moda nel veltire.
— WON si rinniene ne capo, ne coda. Non frritrova, ne il-principio, ne la fine di
wm 5 yeofa. Non'fi fa, non's' intende, o non si ritrova come la cola si tlia, Vee
—- baput, nec pedes, disse Cic, EB' traslato dalle mataffe del filo, ¢ si dice anche Now
ritrov, che @ il principio della matafla.
ih) «| AL tu quiftione? Antendiamo havere inimicizic.

| HAL eu qualche rigiro? Har ta qualche innamorata ? Che la voce rigiro usata.,
] 'come nel presente luogo,vuol dir Pratica di donne per vizio; che per altro,rigiro

'significa Ripiego, dicendofi: 1] tale fa molte faccende, percht egli ha molui ri-
giré, cive neigh ed oe di vendere la sua roba. Alle volte Gi piglia per

« Vedi sopra C.4. st. 60,
. DENT RO è chi |e pefhe' Quand' uno si sforza di moftrarsi nel vil allegro, ed

'ha travagli da star malinconico,diciamo; Ei fa beon vifo, ma dentro è chi la pefa,

hes dentro sta in altra guisa. Kifus in ore y fletus in corde, Virg. Spem vulti fimu-
lat, premit altum corde dolorem.

5 Heaiore fie in teha, Pensiero, o fantafia oftinata. Vedi sopra C. 1. ff. 10,
PAR ch ei la fnechieli. Egli fla fra i) st, ¢ il no di fare una tal cosa, che direm-

mo Irrefoluto. Dante Inf. 8.

Padi age Che'lsi,e'l no nel capo mi tenzona,

Traslato dal ginoco delle carte, che firdice fucchielare quando si tira fu la carta,

'adagio adagio 3 if che pure è traslato dal bucar col fucchiello, che ¢ una aziones

simile al tirar fu la carta. Qui vuol dire, Pare che questa sua fiflazione lo voglia.

adagio adagio fare impazzire,¢ ridurlo a i Pazzerelli, che ¢ lo spedale,dove si
mettono i pazzi.
RIDVRLO in feffo. Ridurlo alla giufta mifura; Raggiuftarlo, rimetterlo in,

buon' essere: fargli ritornare i} giudizio.. Vedi sopra C. 1. st. 15.

SU spenda la Vita, € vada il reffo. Si (penda ta vita, ¢ la toba. Tratto dal giuo-
0, le ff faole feommettere, ¢ dire. Yada il refto; fo del reffo, Bquic det.
to per figura; perché quando è andata la vita, che ¢ la pil cara cola, che noi

ha iamo, par che non ci refti quafi altro da buttar via.
g 6 "O cinffo, Perappunto. Ela replica ha la solita forza di superlativo.

—— Catullo.. Af igis magis increbrefeunt. Nell'Ebraico Azeod, che vuole dice af/ai, rol

ot, raddoppiato vuol dire afaiffine,moltijimo.,

   

 

|

 

STAN-

   

 

 
 

359
STANZA XLIVs.0%
Percio d abiti, ¢ soldi si provvede, |
E da buone [peranze alfno Nardino.

Esce di casa,e mettefi in cammino,

Shirciandofempreinqua,e inlafevede >

Donna di vifo bianco, echermifino;
E se ei ne incontra mai di quella tinta,
Vuol poi chiarirfi,vellae verayo finta,

STANZA

Di modo cl ei non vuol reftarui colto
Ma fharui lefto, ¢ rivederla bene
E per questo una [pugna feco ha tolto 5
E fempre in molle accanto se la tiene,

 Suegetto y che gis occorra farne prove,
Brunetto date buone speranze.al suo fratello,montd a cavallo; ed
0 un' huomo a piedi, fen' ando cercando d* una donna bianca', ¢ rofladi
naturalmente, ¢ sapendo che tutte le donne hoggi si lisciano, haveva
spugna bagnata,per far con quella la prova,se il colore era finto,04
per molto, che egli cercafle,non trove mai donna, nella quale occorrefie far tal
Prova, perché si conosceva senza farla, che twtte eran tinte y e-lisciate. Quello
colore finto, che chiamiamo li(cio, o belletto,si dice anche frco-, che |
buona a tignere i pani; da i Latini detea faces 5 ¢ l'intendevano
si per quetto lilcio, o belletto. Plaut. Moft. 4.118. Vetule edentula,
corporis fuco occultant. EB di qui i Latini per fuco intendono una forta d
che ricopre con artifizio un mancamento in una mercanzia, ec, onde;
cere,

pk
S8IRCLANDO, Guardando attentamente. Vedi sopraC, 1.f9,
CHER4MISINO. Rofo di Chermisi, 0 Cremesi, E' il roflo porporino, chef
fa col fangue di certi vermi chiamati con voce Spagnuola Cocciniglia dal Latino
coccineus color, colore di grana, colore vermuelio; ed & il più nobile, ed a 0
te, che si trovi, ne mai perde il suo colore:e da questo nel presente luoge inte
de rosso naturale a perfezione, ¢ che non perde, come farebbe il finto +
o Karmes in Arabico vuol dire grana.. Latino coccum,(econdo lo Scaligero eferti
tazione 325.

D1 quetia tinta. Di quel colore. E' termine pittoresco, cotumandosi
dire: La tale ha una carn i

di carne.
VVOL chiarirfi, Vuole accertarsi.

MALIMW/AN TILE © 9%

  

4 STANZAKLV. | '
— \

> Che non si mini 0 si taftri le quai;
Epreso un bud cavalloyeunbuomoapiedey 00 B
Cc

ragione nella quale feno beile tinte, per intendere Belli

     
   
  
 
    
  
   
 
   
    
     

'

Pas.

“Wella pare il ritratto

Quattro dita vi lascia fu di loiay
Oe
Chrella par proprio un' Angiolin di
XLV" a pots
Con che paffan le sopraiil volte,
Vedra: s'il color 0 se vii

Aa gira girasin fasti ei non yitroyh,

i=

Ss
sa

 
 
 

fucum

a eee

aachil
ea

NON si minij. Non si tinga, Minio & specie di color roffa cavato ¢
8

\> adit
ae 1 si:
no; € miniare € una (pecie di dipignere con finidimi colori sopra cole e
me cartapecora, ec,

S/ lustri le quoia. Si \isci la pelle.

MOST ACCIO infrigno. Vilo grinzolo,

refroigne.

© cresposo,o rinfrignato of
ANCROIA, L' Ancroia è finta una donna brava in un Poema

   

 
 

 

SETTIMO.CANTARE - 351

~ Regina Ancroia; ¢ perché questo Poema è degli antichi, che si trovino nella.

 lingua nostra, mi do a credere, che quando si dice l'Ancroia,' intenda una vec-
chia. di Berni, de(crivendo la sua serva in un Sonetto dice.
' Lobo per cameriera mia l' eAncroia,
. Madre di Ferrak, Zsa di Morgante,
y w  )  ehicavola maggior dell' Amoftante,
phe Baliadel Turco, ¢ fuocera del boia
 Ma pnd esser ancora, che queita voce Ancroia sia un' addiettivo, che venga da
i@, che vuol dire Zotico, ¢ duro dai Lat. corium quali inquoito, fatto duro, co-
il quoio. )
thee Col pugno gli percoffe I epa croia,

 

ae Da questa voce croio habbiamo il verbo tncroiare, che vuol dire aggrinzare,.¢d

= =
ee

ce
a

 

SRSLERESE

 

renee per intender pelle grinza, ¢ fecca, ¢ indurita, come € quel -

vecchie, alle quali però si dice per (cherzo Aduna incroia, che nel parlare
'Pultima lettera di 44ona confonde, ¢ mangia la prima d' ixcroia, viene a
ancroia, che vuol dir vecchia grinzola. Jncroiato si dice un qaoio, che per
flato preflo al fuoco, sia divenuto duro, ¢ grinzofo, ed il simile una carca-
abbruciacchiata. Si dice incroiaro anche un panno divenuto sodo per gli
mi', ¢ lordure; ma di questo'é più proprio incorezzato, dal Lat. corryia. Il
bolifta Bolognefe dice, che Ancroia signitica vecchia, che va crollando il ca-
po, ¢che viene dal Greco Craein che vuol dir croilare. Ma venga donde si vo-
glia, basta che apprefio di noi vaol dir Donna vecchia, ¢ brutta y ed im quelto
Aenlo ¢ prefa nel presente luogo.
 LOLA, Sudiciume. Terra stemperata con acqua, ¢ ridotta liquida, che cons
altro nome chiamiamo mota. Qui vuol dir quelle materie, che si mettono in ful
vilo le donne y le quali s' imbellettano, Voce fatta per avventura dal L, i/luvies..
. IMPIAST RA, S' unge con materic bituminofe,¢ viscofe come è l unguento.
STVCC.A, Stucco è quella composizione di geffo, ¢ colla,¢ d' altre materie
4tenaci, che serve per riturar feflurejo magagne ne i legnami. E facco è una spe-
cic di geflo, o terra, 0 altra composizione, con che si tanno le figure di rilievo,
i per fucco intende quelle materie, che le dunne si mettono sopra il vifo per
la faccia, ¢ turarsi le margini del vaiolo,,o altre cicatrici; che il
verbo fuccare yuo) dire intafare, cio¢ riempiere i buchi, ¢ ragguagliare una.
I¢; donde gli Orefici dicono fluccare, quando con una certa loro lima
detta lima stucca, spianano i lavori d' argento. Sewccare vuol dire ancora quan-
-do wn cibo ci apporta naufea, 0 i discorsi d' alcuno ci vengono a faftidio.
. ViN* Angioline di Lucca.. A Lucca fabbricano certi figurini di cera, di geffo,
Od' altra materia,a' quali dopo formati danno il-colore di carne con ua rotio lu-
Sirante; per questo d' una donna lisciaca diciamo; Pare un' Angiolino di Lucca.
Gosi iGreci, che le belle persone afiomigliano alic statue ben fatce, le chiamano
Agalmata,e Properzio,ditie che il colurito del vifo della sua donna era giutto co-
»me quello, che si scorgeva nelle pitture del famofo Pittore Apelle. Quals Apelicis
<¢f color in tabulis. in un' Bpigramma Greco una faccia impellettata, ¢ lilciata,
con clegante bifliccio vien detta Profopeion, con Profopon, cioè maschera'y © non
faccia « Vedi Cel, Rod. Leét, antig. lib. 29. C. 7.

AON

 
 

    
  

55%
NON vnol resparni colto., Non ins rimanare q
ST ARV1 lefto. Stare'accorto, O.avvertito, 9) © 96
G/RA gira, Cammina in diversi luoghi; cammina
IN fatti, E' lo stesso, che in somma, o in pen? L,
STANZA XLVIL..
Dopo che tanto a ricercare ¢ itoy
Che i calli alc,. ha fattoinfulafellay
Giunfe una fera al luego d'un Romito,
C' areftar L innita nella fan Cella y quel delle ¢
A lui parne toccar il Ciel col dito Di che speffa ciascun
(Per non baver a iar fuori ala stella) Stettero a croschio ii
4M pafjar dentro, ed egli,e il feraitore,
Ringragiando idbuon buom di val favore,
STANZA XLVIILL
Veftia di bigio il Vecchio Mdacslente,
Facendo penitenca per Adacone Dice chi sia echo di cafael
E perch' ei fu nell accattar frequente, Aon per fa conta,mad! un si
Per nome si chiamo fra Pigolone. Del quale infino alt' Tad
Coftui y( com' io diceva ) allegramente Perché gli pare uscito die
dn Cella raccetto le lr perfoue, Non fifas ei fifia pile carne,
Spoglia il cavalio,e gli trito la paglia; Cosh piangendo in far di cid'm
Sul desco por distele la tovaglia, Per la mmuta conragl la.
Capito Brunetto una fera alla Cella d' un Romito, dove efleada
tato, stando a tavola raccontd al Romito ibcafo del Fratello, dicendo
fuora per far servizio al medesimo suo Fratello,
TOCC AR il Ciel col dito, Confeguir I imposfibile. Ap
ST AR alla fella. Dormire all' aria; a ciclo scoperto; alla fella diana 5 Lat
ub dio, NNR
MACILENTE. Mal fano; Cioè magro per lo stento, ¢ giallo i
ione.
: EV frequente nell' accattare, Duc tefti di mano dell' Autore dicono uno
te, edé) ultimo; ¢l'altro servente, equefto ¢ la prima bozza, ¢ se i
¢ Iraltro pud stare, io pighierei ? ultimo, perché in fultanza vuol dire che
€ra attento,¢ diligente nell' accattare, ¢ fempre chiedeva, che da,
importunita, s acquiftd il nome di fra Pigolone che così chiamiamo
fempre chieggono, ¢ che moftrando una certa ingordigia di roba,si
pre dello stato loro. Pigolare €il verso de' puicini, che beccano. Lat.
Spagn. piar dal fare pio pio, che così ¢ il lor verso. ovat
DESCO, Tavola, sopra la quale si pongono le vivande, quando si
dal Lat. discus, che ¢ pierra rotenda, 0 laftra da scagliarsi, Vedi sopra!
TVTTO accatrato. Ogni cosa havuta per limofina.
FIORITO quanto un Adaggio. Fiorititimo;percht ii mefe di 'Maggio' la s
ne de i fiori; O pure perché queili, che vanuo a cantar maggio,porta
ad aealbire tutto picno di-diverti tiori, il qual camo @ albero ch
Bio » ° maio. Diciamo: vizo foecisay quando o per ctier ab ton

 
   

  
  
 
 

 
  
 
   
         
 
  
    
 
   
  
   
  
 
    
    
   
    
 
  

 
  
  

 
  

SETTIMO°CANTARE, 333
per altro mancamentoj il vino dofi nel bicchieresha 'nella superficie minu-
tissimi frammenti d' una cerca specie di muffa'bidrica; che è il panno, che si fas
dal vino, equ 'chiamano fort; si che quis" intende, che il vino era vicino al
fondo dell ', 0 havea altro mancamento, che' produce la detta muffa; se be-
I 'chevoglia dire Vino ifquifito; perché /ro itoeattribuco di perfezione in tut-
4 syeccetto che nelvino, che |' efler fiorito è fegno d' imperfezione.
' centuna bette. Questo numero centuna, benché sia determinato,
dee | t per indeterminato; ¢ vuol dire Cavato da infinite botti di coloro,
}haveyan dato per limofina, E questo pure è imperfezione del vino, che
perde lo (pirito., ¢ la bontà in tanti travafamenti, ¢ mescolamenti,,
 STETTERO a crocchio. Stettero chiacchicrando.. Vedi sopra C, 1, st. gt., €
Bietemin così detto dallo strepito, che si fa ridendo, ¢ chiacchicrando
ielle conversazioni di trattenimento, percid dette Crocchi, Dal romore similmen-
 teedal faono che rendono, sono dette da' Prancefi Cloches le Campane. Così
i — 'saccordano nel rapprefentare con |' arte i femplici fuoni inartico-
jt lati che (ono un' inalterabil linguaggio della natura. '
ed batte dove il dente duoie, Si dilcorre empre volentieri di quelle cose,
j@ dove hala paffione', o sia di gusto, o di disgutto.
' a il campanello. Parlaya sempre lui, Questo detto viene da i Magiftra-
/
Cd

  
 

       
  

Bet
*

tidi ¢, ne i quali uno dei Colleghi si chiama il Proposto, e questo fempre
Sa aj litiganti, ¢ chiama, ¢ licenzia dall' udienze, ed i compagni
' a cheti; ¢ questo Proposto tiene allato alla sua seggiola un campa-
nellowE da quefio, quand' uno in una conversazione fempre parla lui, diciamo:
yp Bi tleneil caimpanetio.,
APINCRESCE fino all anima' Gli ho grandissima compaffione'; Vedi sopras
in questo ©; st, 26. Mi'dilpiace, mi pefa. Dante Inf. 6,
se RDS “Mi pefa'st, ch' a lagrimar m' innita,
DU Greco dice Achthomai, mi dolgo; ¢ 10 Spagnuolo similmente pe/ame. Onde quel
che'in Toscano si dice' dare il mi dispiace, esso dice, dar ef pefame: La stessa forza
ha MMe Y AP inere/ee, quali mibi merave/cir, secondo il Ferrari; mi grava, e»
BS Pope nage Amore ¢ pefoscomincid Dante una Canzone. £' m' incre/ce di
” Z PMO. Sus '
WON fap ei ff fra carne'; 0 pefee. Non fa quel ch' ei si sia. Noné in cervelio,
Non ha' vO conofeimento. Awevo pefee dicevano gli antichi un' biomo /Pra-

Eee,

ae

 

 

 

 

wm math
ip! neces ANZA LL STANZA LIL
ee Sta Pigolone attenro a collo rorto Egli ha un giardino posso in un bel piano,
i Ad 'ascaltarlo s€ poi ch' egli ha finito; Ch' ¢ ognor frorito,e verde tutto quato;
“a: F igiiuol,risponde a Ini 5 datti conforto, Giardiniero non v' t, ne Ortolano,
oy + Bfappi, che th fei nato veftito, Che a entrarui nefjin pus darsi vanto,
» Che qui? Pbbnom falnatico Magorto, Da per se lo lavora di sua mano,
a Ch'e un beftione, un diavol traveftito, E da se (0 fondo per via a! incanto,
o & “Che se te lo vedeffi vb eglie pir brutto! Con una casa bella di frupore,
ye Balke a suo tempo conterotti it rxtto, Che vi potrebbe ar  Imperadare.
5 4 vy STAN:

 

  
  
  
    

 
354 MALMANTILE

STANZA LIIL
Ma io ti uno dar' adeffo un? abbozzata
Lui prefto presso della sua figura.
Ei nacque a' un Folletto, ed' nna Fata
A Fiefol n' una buca delle mura,
Ed ¢ si brutto, poiche (a brigata
Solo al suo nome crepa di paura;
O questo ¢ il cafo a por fra i nocentini
ed far manciar la pappa a quesbabini,
STANZA LIV,
Oltre ch' ei pute come una carogna
Ede pin nero della mezza norte,
Ha il ceffo d'Orfoye it collo diC arogna,
Ed una pancia » come una gran borte
Va in sui balefirs, ed ha bocea di fogna
Da dar ripiego a un tin di mele cotte
Zanne ha di porco,e nafo di csvetta,
Che piscia in bocca,e del continuo getta,

STANZA

ea lasciando per bor  altre da parte,
Cocomeri vi fon di certa raza,
Che chi ne puo haver uno,e poi la parte,
Vi trova una bellissima ragazza,

Pigolone incefo il bifogno di Brunetto, gli da animo con dirgli, che
huomo faluatico ha quivi un' orto,dove fon cocomeri, che tagliandoli n'esce suo-
ra una bella fanciuila, la quale chiede da bere, ma sce feglida., ella
Deicrive ancora in queste quattro Orta ve la qualita di
SE/ naro veftito, Hai havuto buona fortuna, 0 qu wd
questo termine per esprimere,quand' uno desiderando qualcofa difficile a y
s abbatte accidentalmente a trovarla per appunto,,come ci la desideraya, eda
propolito del (uo bifogno. Dicono te Levatrici., che taluolta nascono ban
con una certa spoglia sopr' alia pelle, la quale spoglia non si leva loro subitos
ti, ma si jascia, ¢ ca(ca poi da per se in proceflo di giorni.; eral creatur
si dice vara vefiira ed € prefo per augurio di felicita di quella tal.

ha dato origine al presente dettato.

VN diavol traveftito. Vin diavolo immascherato da huomo; intende un'

brutto, quanto il Diavolo.
BELLA di flup
yvede;
VO!

I Pittori dicono Abbozzare

Jerto ¢ Gianni Schicchi, dice che i P

STANZALVL. —
Dell' ofa poi ne fa si ce

ere. Bellissima mirabilis vifu. Tanto bella, che fa flupire
ma per venire la yoce /tupore dal latino,pud ognuno intendere il suo}
IG LIO darti un' abborzata, Ciok ti vo;

Y > quelle prime pennellate, che danno in.una tela;
trove, dove voglion fare una pittura. Vedi sopra C. i uy

FOLLETTO. Vno di quelli spiriti infernali, che dicono che stieno per Hari
1) Ferrari nell' Origini alla Voce Fuile,citando Dante Inf, 30, efi disse, quelf

   
 

z

    
    
       

E della pelle ne fa maccheroni, =

Diente in Jomma v't, che vada ma
Sreche Brunetto figlinol mio, tn femti,
Ch' egli è un cattivo,ed orrido ammale,
Hora torniamo a suci scompartimemi,
Ove fon frutte buone quanto il
Vaghe piante, bes fiori, ed altre vole
Com' to ti potrei dir maraviglofe,
LVI, AYE
Che per esser aftuta la sua parte,
Dirattiche tu gli tpia una fun rare,
A un di quei fonti li si chiari,e freddi,
Ma se la ferni, a Lucca ti i

efto Mi i
ota. cadena

nis

glio de(crivere alquanto, 0
st, 41.

?olletti sono la/civ) genij ac Lemures rifu aS
domos implentes,. = E.

 
 

 

  
 
 
   
 
   

SETTIMO CANTARE. 335
FAT A. Vedi sopra C, 4. st. 45. i
eA FIESOL 0 una bucadelie mura, A Ficlole si veggono ancora alcune reliquie

delle mura di antica Città, ed in ci frammenti di muraglie fra l'altre si
 yede una gran buca di »od'altra cosa simile, la quale dalle donnicciuole è
- ereduta, ed è dataac ai fanciuili per abitazione delle Fate, ¢. pero vol-
ms @idetta /a buca delle Face. E questa è quella buca, nella quale dices
-» che Magorto era nato d' un Follerto, ed' una Fara. Angelo Poliziano
~ al titolo Lamia dice: Vicinus quoque adbuc Fefulano ruscnlo meo lucens Fon-
ticuins eff, fecreta in umbra delite/cens, ubi fedem esse nunc quoque Lamiarnm narrant

      

a imuliercule « Questa credo sia quella caverna, che-oggi si chiama /a fonte fotterra
¢,  luogo orrido  ¢ (paventevole, ma fempre pieno di limpidissima, ¢ freschiffims
lt NS TEV ' 2

ry “SNocewr 17 « Cioé quei ragazzi, che s' allevano nello Spedale degl'Innocen-
so erensa +5: ton

ca ~ CAF AR mangiar la pappa a quei bambini, Così diciamo d' un' huomo, o donna
nil a ¢ brutti, quaGi che fieno come il Bau, la Befana, ¢ simili larue in-
ga  uentate dalle Balic per render i bambini ubbidienti, ¢ fare che per il timore man-
- Cai wa. Vedi sopra C. g. st. 3. E.quetto putire da i Latini cra espreffo
0 co} paragone, perché dicevano vixum cadaver. 11 Monofini.

ut nero della mezza notte. Negritimo, piii nero del buio,

 

 VAin fui balefri » Ha le gambe fottili, e torte come sono i baleftri, compa-
'fazione vulgata), sendoci una cantilena di Balie, che dice.
° Ben ne venga Mignamau,

Saif oe j Cha le gambe a baleftrucci.
O81. 3 ¢ Sbilenco, dicefi chi ha le gambe torte; ¢ ancora Aver le bilie;
tratta la similitudine da certi legni torti, o randelli, co” quali i vetiurali legano
itetto, ¢ arrandeliano le fome; da loro dette bilie.
 BOCCA di fogna. Alla bocca delle fogne maeftre, o principali, che ricevo-
'no acqua delle strade,quando piove, ¢ la conducono nel fiume d' Arno, è figu-
rato up ma(cherone di pietra, il quale ingoia |' acqua ed oga' altra sporci-
zia., € di queste intende il Poeta; ¢ da questo diciamo: Bocca di fogna a uno, che
mangia, ed ingoia ogni forta di cibo, se bene sporco, senza distinzione, o ri-
yaleuno. Latino bedwo, gurges. Queste fogne in altri luoghi d' Italia sono
lette Chiaviche dal Latino Cloaca.

DA dar ripiego., Cioè dove entrerebbono tante mele cotte, quante n' entrereb-
be in un sina, che quel gran valo di Iegno, entro al quale si mette 1' uva pigia-
$a bollire per farne vino. rare
», ANNE, Denti: Propriamentes' intende di quei denti Junghi, che hanno i
Signali, i lupi,i.cani, ec. che noi li chiamiamo anche denti Adac/tri; 0 Adaeftre.
Vedi £ aes, st. 64. Borle & meglio dir /anne, ed ¢ pill conforme all' origine,
Onde /ubfannare buriarsi d' uno ridendo, in maniera, che tutti identi, come di-
$e il Boce. si poteflero trarre; moftrando le sanne. Daa. Inf, C, 6,

Quando ci feorfe Cerbero il gran vermo,
Le bocche aperfe,¢ —— le fanne.
yz

 

SER GF BRLLSES weeks

Se

it
4

eC,

   
 

  
        
 
    
  
    
 

356 MALMANTILE?D*)2

¢C.22. E Ciriatto, a cui di bocca nscia > Vo
D' ogni parte una Janna come 4 porcoy eres Pepe)

Gli fa sentir-comel' nie sdrucia. » 8
NASO, che pifeiain bocce, Ciok nafo aquilino' che ha la a
la bocca, ¢ pare che vi colidencro. vb~orn saokay
BERLING ACCIO « i Giovedi geaflo, chee I ultimo giovedt

detto Serlingacciv da Berlingare, che vuol dire bere, ¢ mangiare',

mente, come si fa in quel giorno: ¢ così Magorto, quando pi a
faceva conto, che quel giorno fufle il Berlingaccio, toleani:

menti, pappalecchi, e Gorxeviglie,daligodere, Latino ga

    

   

ennizzandolo con
c wifare, conic si
antico Glofiario, onde lo Spagaualo gozar., godere pel nostro gavayzare
ti finonimi, che voglion dir ghiotcornic Bocc, g. 8.n. 2. Sé cues
pis volte insieme fecero corzovighe sec. << i8927 toil?:
MIGLLACCIO, Sangue di porco, o d' altro animale mefeolato'con
farina', ¢ poi frittoynella padella a uso di frictata'da aleuni* Latiot
chus; se bene quetta era una composizione-dicacio, ¢ falamevdal
che vuol dir cacio, etarichas, che vuol dir falame. 1
STVZZIC ADENTI, Nettadenti: Sottilissimi,ed acuti stecchi
d' offo, o d' altra maceria per uso di nettare i denti + Latino
BYONI quanto il fale. Saporitissimi.. Vaaevivanda con molto fale'
rita, che vuol dire il contrario di sciocca, oinGipida ye: senza Yale 7 ¢
faporito è meglio al gulto, che Pinfipido,'¢ pero per faporito i i
¢ dicendofi; buoni quanto il fale, s' incendefaporiciimi, cioè gustofissimi

fapore. es
TCOCOMERO. Specie di mellone acquofo di fapore'dolce, che nella
stagione calda per rinfrescarsi. In moiti luoghi d' Italia chiama
così la chiama il Mattiolo, e dice che era incognita a iLatini, se bene G crovas
cuckmis, ma intendono il cetriuolo, che pure in alcuni luoghi si chiama eeeome™
Anguria, dice il Perrari, ¢ detta quafi cucumus anguinens 5 © così questo nome
che era proprio del cecriuolo,per mancanza di vocabolo fu tratto a
fructo, che noi Toscani chiamiamo cocomero. i
e4 LVCC Ati riveddi, Questo detto significa Non 1a vedrai pitt,
Buoni da Lucca nel suo teforo de“Proverbi dice, che havendo un
Lucchefe veduto ua Gentilhuomo Pifano a Lucca,usd feco cortefia
definare a casa sua, dove condotto, fu trattato con ogni forta <r
titofi il Pifano, e ritornaro alla patria,avvenne che fra poco tempo'
ando a Pila, dove paruegli conucnevole vilitare il Pifano fuddetto: Ts
pero alla casa di eflo, dopo haver molte volte buflatosal fine sa ffaccit
© gli disse che nom lo conosceva; onde il Lucchete disse: 2: e4 Lucta ci-vede
Pifa ti conobbi, © con quetto fiticenzid. Così forive un Lavcchefe:, ma tt
voltano il proverbio dicendo: 4 Pifativedat pea Lacca si i
grato,¢ scortefe quello da Luccaje:non quello da Pita') Seibene il
era ne Lucchefe ne Pifano nella sua En. Te. C. 3. st. 4. dice:;
E dicon spefo alernis Ti veddi a Lucca,

 
    
  
 
 
  
   
 
 
     
 
  
 
  
 
 
  
   
 
 
 
     
 

 

  
 
 
      
    
 
  

—nw ese E-epeEE PBB TL ete

   
 

 

 
  
      

SETTIMO CANTARE! 337

STANZA LIX.
Efe en ean
Dirad, che tu buon Cavalier non sia,
entre conforme all' oblige non vft
Servitie con le Dame, e cortefia.
ea lascia dire,e tien gli orecchichinfi,
Non ti piccar di cio, sia pure al quia,
Gracchi a sua postayty non le dar bere,
Accio non fugga;e poi ti sia il dovere.

Con questa, che fara farta a pennello, Vientene dunque meco,e sia in ceruelo.
| Come te cerchi, lenerai dal cuore Cammina piano, ¢ fa poco romare,
—  Ogni dogti i affanno al tuo fratello, Chefee' ci fentea forte, scuopreil cane,
16 --Edioten' entro già mallevadore. No occor' altro;noi habbiam fatto ilpane,

ite seguita:a narrar la favola del Cocomero, ed instcuito Brunetto di co-
a' ome contenere, perché la fanciulla non gli scappi, s' avvia con esso alla
volta del giardino di Magorto. -
s far conto @ haveria vifea. Ti puoi dare a credere d' hayerla veduta qua.
4: i a.vedere, perché non la rivedrai pill. 2
Al uno stivale. Refterai beffato, Retterai uno scimunito. Vedi sopra
yg  (Geqfero, LGreci dissero Bagas con/fieyti, da un tale detto Baga, 0 pure Bagoas
a} 'nome da Eunuco; che fu un' huomo infipidissimo; Donde poi noi diciamo Bag-
sg) 60» © Baggiano, a ua' huomo scimunito se non forse da Ha/eo,, ¢ da Habbano; 0
da Bageiano forta di fave maggjore dell' aitre.
of va di forche 5 edi moine, Vina quantita grandissima di finte carezze,€
'Nezzi; i Latini dissero blandicie, Ed in quelto proposito tanto è dire far le forche,
5 « dexai, quanto mome, significando cutte tre una forta di Jufinghe fatte con
wit Bti,ocon parole, ¢ sono quafi lo stesso che adulazione; perché ancor le»
dt “nine sec, son atti, gefti, ¢ discorGi, i quali contengono, se noa falfe lodi, co-
»meccontienc ? adulazione, almeno falfe dimoftrazioni d' affetto affine di com-
eo. jiacere ye di acquiftar ia grazia di colui, a cui si parla,¢ queste fon proprie di
ie di femmine, ¢ |' adulazione ¢ conuentente ad ogat forta di persone,
ma è fempre indizio d' animo vile, ed esseminato. Ll Landino nell' esposiziones
a Dante Inf, C. 18. dice, che gli adulatori in lingua Fiorentina fidicono moinieri;

   

f=

¥ »Ma questa voce non si dicendo in oggi, ac avendo autorita di Scrittore nell' an.
si tico, mi fa credere, che il Landino la derivaile a capriccio daila voce Fiorenti-
i na Meive non trovando parola corrispondente alla Latina ddulatores, I) Cala

nel Galateo volendo mettere in volgare il Latino ada/ari, lo esprelie colla paro-

SSL.

» la Piaggiare, L Bini in lode del mal Francefe dice:
uhangl Jo non roppi gid mai; ne carfiiancia 5
Machi mi va con si fatse moine,
Vorrei porergli sfondolar 1a pancia.
 La Stor. di Semifonte Trattato 4. Quand! altri ha ofefo un fupremo, non ¢ da si~
< darsi di lui', ne delle sue aftute moine,¢ Lufinghe.,
-  NON+i piccare. Non v' offendere none' adirace; Non cntrare in gara; Non
ats ti

 

©!

=

 

 
as

ae
mI cL

358 MALMANTILE 4) |

ti flimare ingiuriato. Vedi sopra C, 3. stan, 20. Tanto il Franzefe quan:
to lo Spagnuolo Picar voglion dire Pugnere; forse da Picca;

  
 
  
   

    
     
  

   

colina

wale Omero appella nyttein, cioè pungere. Vino piccame & que
iecda Ȣ che punga, lee they cP amma
bio; Tienle caro. ll Perfiani Tesan taeda bea
Va menati l agrefto 5:
Ceruellaccio peftato per Lambiceo ?

Che 'l tuo mordente ha trove poco appicco.

Di questo iv non mi picco “ct

Che s* io non ho la nobilta a bigonce, yet

Mi basta ds non esser a' undici once, (cioè baftardo) —

PICC ARS!, Vuol dir anche persuaderfi, o darfia creder d' etfer eccellentes
in una cosa, come piccarsi di bravo, di bello, di dotto, ec, ¢ vale quanto efier am
biziofo, o haver ambizione. sx pee

SSE £8 ® Ss ome

 

ST-A al quia, Sta sodo: Non badare a quel che ella dice; enon tilafeiares | tf
fuolgere, o persuadere a darle da bere. Dante. State contenti, wmanagenits, |v,
al quia, ' hive ow

GRACCHI a sua posta. Gridi, cicali, eflami pure quanv'ella vuole; lasciala | yy
dire, la(ciala cantare. Quand' uno vuol quaicofa da un' altro 5 ed ' ty
mandarglieia,¢ colui non glicla vuol dare,suol replicare a i detti di: Rs
chia, gracchia; quafi dica: Tanto mi muove il tuo dire 5 quanto il geacchiar) | ti,
d' una cornacchia. Vedi forto C, 8, stan. 64. far Fup

TJ (tia il dovere, Ti fucceda quel che w meriti. aa

SAKA fatta a pennelo, Cioè fara similissima, ed appunto come cs

T” entro Mallevadore, Te ne afficuro. Ti fo ficurta, che leverai u
Fratello questa frenefia. Adadevadore ¢ il Latino Fdeinffor, quafi afidarore, afi
curatore; detto Maiievadore secondo il Menagio, dal /evare in alto la.
fegno d' afficurazione, Lo Spagnuolo lo chiama Fiador, la qual voce in
co Vo)garizzamento Toscano manoscritto delle Vite di Plutarco tra
lingua Aragonefe, refid senza interpretazione insieme con alcune altre y il)
guiva in gucfte tali traduzioni, o per vezzo del traduttore, 0 per i
gine, o perché non ne fapefle pi la. Caro mon volle il diposito, ma fiette t
tutti, wah:

NOL habbiam fatto il pane, Noi habbiam dato nel laccio. Noi i
vuro la disgrazia senza rimedio. Diciamo ancora; Voi habbiam fritto, Vou
fouo C. 8. fan, 54. sega

STANZA LXI. STANZA LXIL ©
Zitti dunque 5 nefjun parts, 0 risponda + A casa lo strascina ete lo fica |

     
 
 

 

eAndiamo che e's' ha a ir poco lontano,
Così va innanzi, ¢ I altro lo feconda,
Oikjernitor lo segue anch' ei Piano piano,
Ma quel Demonio, che va, fempr' inroda,
Gii fente, e gli vnol vincer della mano,
Perché gli asperta,e il vecchioc'alla fiepe
Vien primoghiappa/,come dir: pepe.

7s facta,e conlacorda ve

E fatto questo a un canapol'

Che vien dal palco oueea a vertas
E per pigtiar il refto della critthy
Ejce poi fuora, ma ncl fate'
Che quand ei prefe q '
Ad aspettarlo havute

 
 
 
  

SS ee. ew SE FF ee eR Keke eee

 
 

   
 
  
 
 
  

 

SETTIMO CANTARE: 359

Soot a Selb SRRSTANZA LXE
| Edoggimai si trovano in franchigia, Sfogarsi intende,¢ a quella vefte bigia
ene Vuole un po meglio feardalfar le lane,
Kabel: june, en'érantoin valigia Percio /u verso il bofeo col pennato
Che ne manco daria la pace 4 un cane; A tagliar un Quercinal va difilato.

Pigolone esortando i compagni a far romore,s'avvia con essi verso il giar-
dino, ma appena giun(ero alla fiepe, che Magorto gli senti, ¢ prefe il Vecchio,
che era op vicino alla detta fiepe, ¢ condottolo a casa lo ferro in ua facco,¢

palco, tornd per pigtiare il reito, ma non gli trovando, fen' andd

| alco 5
al hosco per fare un buon battone, col quale haveva ia animo di baftonare Pi.

 2ITTL, Cheti, Vedi sopra C. 1. stan. 10.

LO feconda, Gli va dietco: Lo seguita, Petr. Canz. 8.

pies Ed un gran vecchio il fecondava appre/so.

 EB spelfo in ronda. Gira per orto facendo la guardia. Ronda dal Lat, retun-

dus; dal quale è fatto il Franzefe Rond ritondo.

 GLI enel vincer della mano, Vuole efler pis diligente, ¢ pitt lefto di loro; gi
wool prevenire. E traslato da quei givochi di dadi, ec, ne 1 quali il punto ugua-
Ie noné pace, ma vince quello, che ¢ il primo a tirare; per efempio, io fond il
primo a tirare, ¢ scuopro fei; tira il secondo, ¢ parimente scuopre fei, ¢ se be-
neil punto ¢ uguale, vinco io, che sono stato il primo a tirare; ¢ questo si dice
Vincer della mano, perch colui » che ¢ il primo a tirare,si dice baver la mano.
tanto basta ai noftco proposito, f€ bene moiti altri giuochi di carte danno questo
Privilegio alla mano.

s SIEPE, Chiudenda, 0 riparo fatto di pruni, ed' altri sterpi agli orti, eda
icampi, E' yoce latina. Franco Sacc. Nov. 83. E giungende dove era la vigna,
qucftaera molto affoffara, ¢ con una buona fiepe.

CHLARPA fu, come di pepe. Piglia subito,¢ senza contrafto, o fatica alcu-
na. Credo, che questo detcato sia corrotto,¢ che si debba dire: Come dir: pepe,
che è facilidimo a profferirfi, come tutto labiale,¢ di sillaba raddoppiata; ¢ che
da questa facilita si cavi il fgaiticaco di facilita in dire 50 fare una tal cosa, per-
ché'a dire; 'Come di pepe non ci fo trovar figaificato, 0 fale alcuno. Chiappare
dal pecaaere. Da Arripere fece il Bocce. Arrapare, Nella Lettera del medefi-
mo scrittay a Meffer Francesco Priore di Santo Appostolo, E fimalmente can
più largo parlare ferivi, che io non doveva così subito il partire, anzi la fuga dal tuo
Mecenate arrapare, Volle esprimere il Lat. fugam arripere con dare a quel verbo
wna terminazione Toscana. Così #rappare abbiamo fermato da extra, € rapere.

STRASCIN ARE. Stra(cicare un materiale per terra senza follevarlo,o por-
Jo sopra veicoli. Lat, Trabere.

FICC-ARE, Vuol dir mettere una cosa in un recipiente con violenza dal La-
tino figere,

 CRICC-A, § intende conversazione, 0 compagnia di più persone: metaforico
da quei giuochi di carte, ne i quali tre figure uguail insieme si chiamano cricea,
come tre Re, tre Dame, 0 tre Fanti.

» | AVRIANO banuto del bue. Haurebbono havuto poco giudizio, poco avve-
be

.

 

 
———

360

SI trovano in franchigia. Si ttovano in ficuro, in Inogo, dove n

refi; che franchigia intendefi un luogo immune per pri

tincipi, Lat, asy/m y che pure alcuai Toscani dico alte
ine eT

mofi di yoci nuove,dallo Spagnuolo dicono amparo,
RIMANE un bel minchone. Riman buriato, riman beffato. weno
stan. 15. si dice ancora reffare uno fivaie sopra in quettoC, spo
E in valigia, Erin collera. Si dice anche im bigencia yin' nel
nel gabbione, ec, come habbiamo notato sopra C. 6. tans 41. &
un' arnele di quoio, entroal quale si mettono cose necefiarie per la
fona, quando si viaggia, e's' adatta in fulla groppa dei cavalo, e quelli
vanno a piedi la portano in su le reni, ma questa propriamenfe si dice
NON darebbe la pace a un cane. Non darebbe la pace a Veruino; cioè
stizza, 0 collera, che egli ha, che se gli venisse avanti un' amico,
be come nimico, perché la.rabbia gli ha fatto perdere il conofeimento, Si dice
xn cane, © non un' altro animale, perché |' alo nostroé di dire + Wow
do guardi in vifo; Non ha cane che cli vogiia bene; nom ha cane che lo foccorra 6p ai
t#, € questo perché il cane ¢ timbolo della fedelra', ne-si trova animale pill
liare, ed amico dell' huomo, che il cane; e pero dovendofi pigliare un'
vicino all' humanita, ¢ profiimo al ragionevole; nel prelente luogo 5
i sopraddetti proverbi, pigliamo il cane. ta
SFOG ARS/ intende. Si vuol cavar la rabbia. Vuole sfogar ¥ ira;
all' ira, come si fa del fuoco, del fummo, che gli si da apertura,
VVOLE un po meglio scardaffar la lana A quetia vefte bigia, Scardaflar'
vuol dir battere, e pettinar la lana; con denti di fil di ferro a i an.
che cara: ( dalla similitudine del cardo erba spinofa ) raffinare la lana, accioeeht
si posia fiare. Vedi sopra C, 3. stan. 60. ¢ per metafora significa baflonare ind;
¢ però qui dicendo, vole scardaffare, ec. intende Vuol battonare ue
torna bene l'equivoco, perché par che voglia dire rilavorare,¢ di;
re la lana, con la quale ¢ fatta la vefte di Pigoione. Li Puici nel Morgantes: ”
Adattera it bartaglio ancor dal Cielo ee
In qualche modo a feardaffargli il pelo, a
PENNATO, Coitcllone adunco, il quale serve per potar le viti 5 app
forte così da quella crefta, © penna tagjicnte, che ha nella parte di
nio Marcello alla Voce Bipennis dice così: Bipennis manifefium ef id we
utraque parte fir acutum, Nam nonnulli gubernaculorum partes tenuores ad D
mulitudinem pinnas vocant eleganter, Pennato ancora è epiteto, che ¢ stato'
Latino a' yolatili.. Onde tcherzando sull — » ditie 1) Boece, Gi ]
18. / vidi volare i pennati y cosa incredibile a chi non gli aveffe veduti, EB n0i'
a raccontare gualche novella, per renderla più credibue, factiamo
fegnito nell' antico afiai, quando gli huomini eram più femplici', &
che volavano i pennati, Palladio de Re ruftica tit. 43. discorrendo de'
deContadini vi nomina è pennati,e gli chiama falces a rergo acutas, atque laiitl,
DIFILATO. E jo fietlo che Andar di vela,di filo, addirinura,
C. 6. stan, 10, Vedi sopra in questo C, stan. 5.

ob”

 

  
 
 

 

MALMANTILE!D 04%

eaEFRS rere &F Peerae

oe. pers er kro 2e028 825 5

 
  

Bot
STAN ZA LXV.

Ed ei le corde alfacco aun tratto feialte,
£ fatto quel meschino uscirne fuore,
Che lo ringraria, ¢ bacta mille volte,

el cht del vecchio. E fa un falto poi per quell' amore,

chiufo in quel sacco iltrova pohe >. > Vi merce il can cin e guarda le ricolte y

oe. 4 mal por! Dandogti aint, ed ezli se il servitore,

Poi con i piatri ye pie vafi di terra

Due fiaschi di vin rojo, ¢ lariferra,

LxVL
Quando Magorto in gik viene a ricifa
Con una fhanga in man cotanto fara,
fesse crspands delle rifa Perchée gli par mill' anni con quel tronco
wove con quegli altri firimpiatta; Difar vedere altrui ch' ¢i non è monco,

0, che stava naicofto a offeruare, veduco partirii Magorto, corse alla

9 ¢ trovato il vecchio nel sacco jo cavo., ¢ vi mefie dentro il cane con

di terra, ¢ duc falchi di vino, ¢ rattaccatolo come stava prima si na.

oo vedde venir Magorto con una grande stanga in mano.
felice, ' paroia di commilerazione,come meschino,¢ simili.

YANDOS! 4 mal porto, Trovandoli a cattivi termini.

arrucolada poxzo, Carrucola ¢ una catiecta di legno, e tal volta di fer-

alla quale ¢ impernata una gircila scanalaca, ¢ (ope'a tal girella s'a-

, 0 catena per tirar fu pefi con facilita, e quelta carrucola si tiene co-

ente appiccata al pozzo per tirar fu acqua, ed il moto, che fa cal girelia

ta cagiona per lo piu strepito, al quale il Poeta atiomiglia i sospiri,
; 'igolone.
ae SFA fais, per quell amore. E' un detto faceto, col quale s' esprime la gran-
a, ¢ contento d'alcuno: E tal detto viene da quzi Ciechi, che per
i Popolo fanno nelle piazze giocolare i cani, ¢ fra gli altri giuochi gli

s ¢ al baftone con dire: fa un falto per amor d' un pane, ed il cane tutto

» © per il contrario dicendogli; /alta per uaa mano di baftonate, il ca-

ein atto di mordere, ¢ non saita; ed il termine per qucil' amore figati-

lazione, O in riguardo; come Lo fo la tal cola per amor tuo, s! in-

bh tende Io la fo in riguardo, 0 a contemplazione tua per |' amore ch' 10 ti porwo,

 SERATT-A, Vedi sopra C. 5. stan. 13.

flere f delle rifa. Rider gagliardamente. Rider come fece Margutte, che

baenpp:s fecondo che favoleggia il Pulci nel suo Morgante; Ll' verbo

a altro vuol dire allentarsi gi' inceltini, vale anche quaato /eoppiare,

parities pur fidice: Scoppiare,¢ morire dalle rifa, Bd & quel re quati che

“habbiamo decto sopra C, 3. ttan. 65. Li Pulci nella Beca dice:

Petty wit ' Ta fet nel letto, e crepi dake rifa.

st enone Sitorna a nascondere. Vedi sopra C. 20, stan. 60. ¢ forto C.9.

bis he fa cht ek s* appiartd miffer gli denti.

ia era i emi a Trattaco —— dice: Quejte cose ho cavate da un six

bro

   
 
 
 
 
 
 
 
 
 
 
 
 
  
   
 
  
 
  
 
 
    
 
    
    
  
    
  
  
   

oraeeaal

 

  

 
 

   
 
   
   
  
 
   
 
    
 
   
   
 
  
 
   
   
 
 
   
  
 
  
      
    

362 MALMANTILE —

bro del Comune, che fu impiattato da uno de' Buonhuomini,¢
4 ricifa, Senz' intermidione; senza fermarsi, a p
difilato detto poco sopra Octava 63. antecedente. I) Pulein
ES io mi metto a cantar a ricifa, a
COT ANTO fata, Grofia in questa guisa. Vedi sopra C. 5.
stan. 36. Tam
Par veder, ch' ei non è monco, Far conoscere ch' egli ha le mani; 0
non ha mancamento alle braccia. Jonco vuol dir uno che ha manco
tutte due le mani. Lat. A¢ancas,
STANZA LXVIIL.
errriva in casa, ¢fbracciafi,.€ si mette Ed ei, ch' ¢ 1 fulle furie non vi
( Serrato V uscio ) con il sue randello Che infin.ch' ei non fisfoga
Sopr' aquel [acco afar le sue venderte, Sta intato il vecchio all'uscio,
Suonanao Zuant'ei pao fodo a martello, Ad origliare per udir qualedfa s
Ll Romito che stava-ale velette, E fente dire: O lecca
Perché? nscio hadi fuora il chiaviftello Carne feantia, barba pi
Andi ( benché tremando,e con spavento Ribaldo, Santinfizza, €
Che havea di lus) e ve lo ferro drento, C'aquel d' altri pon cingue, el
STANZA LXIx, 1

   
 

  

  

  

Guardate qui la gatra di Mafino, Ma quel' hai toltoa me,
Che riprendeva il virio, ed il peccato, Won dubitar ti coftera fa
+Se il monello-ha le man fatte a uncino Che tante volte al poxzova'
Per gire a [erafignar pel vicinato? Ch' ella vi lafeia il manicog

   

  
 

  
    

Magorto, arrivato a casa, si meffe a baftonar quel facco, credendo che vi
fufle dentro Pigolone.; Ma quelto-efiendo uscito di-casa mefle il ci
di fuori alla porta, ¢ fermatofi alquanto quivi, senti che Magorto mn
facco gli diceva una mano d' improperj. win ti
SBKACCLARS!, Vuol dire Denudarfi'il braccio da mezzo in git te |
mano come accennammo sopra in questo C. stan. 19, B sbracciarsi; ee
mente parlando vuol dire Impiegare ogni sua forza, diligenza, ed mol
in un' affare. Lat, mamibus, pedibu/que eniti. want 1
SVONANDO a martelio. Cioè baftonando. Suonar' a martello si <<, m
do la campana fuona a rintocchi., come fa il martello full ancudine, ii che i | %
quando si vuol ragunare il popolo per li bifogni della Città. Il verbo fumaretil | &
Latino puifo, vale appretio di noi, come apprefio i Latini per fuonare, ¢ pet |
perquotere. Vedi sopra C. 3. stan. 7. aie
ST AVA alle velette., Stava offeruando. Veletta, 0 vedetta diciamo
to, che sta in fulle mura d' una Città, o Fortezza a far la guardia d
munemente/entinells., edil lwogo dove sta detto soldato si dice velerra
Sumo che sia trasiato da i Marinari, che tengono la detta guardia
albero delia nave, ¢ dicono metter l'huomo aila vela, 0 veletta forse
piccola vela, che sia in quel luogo. Tarcagnotta Stor. lib, 5. p. 3. 7
Partitofi pero il Priore Stroxzi da Marfilia con 2 3. Galere, ed una g
welette in mare lo venne ad sacontrare. Dal che ficava che si chi:
-gune barche, le quali camminino avanti a una armata con huo
 

 

SETTIMO CANTARE. 363

le, opure da vedere vederta'e poi corrottamente veletta. Si come da /pecio anti-
¢ Latino significante lo veggio » si fece /pecula luogo eminente che signo-

reggi molto paele. Ma sia come si sia basta il sapere, che stare alle velette vuoi
dire Stare a offeruare.
| Bin fale furie, E'colmo d' ira.
ORIGLIARE, Star in orecchi, Star a sentire, ¢ vedere con attenzione, edi
iy! cofto.Pranzele oreillier. Spagn, otear forse dal Gr, Ora,orecchie, che i Fiaa-
fini spiega:/piare, eguardare da (uogo aito, come fanno le sentinelle.
 PEVERADA, Brodo di carne, o d' altro, E /ecca peverada vuol dir Brodaio,
se Beiniignifica porco, perché il porco mangia volentieri ogni forta di broda...

 War, St, Fior. lib. 14. dice: Gli diede una mineffrina bolita, cotta in peverada di

- pollo. Detta Pewerada dal Penere, cioé dal pepe, che per dar (apore si metteva.s

fa le mineftre, come fu da altri dottamente offeruato.

CARNE stantia, Carnaccia vecchia, ¢ frolla. Vedi sopra C. 3. stan. 24. ¢ $4.
ye  SAKBA piattolofa, Termine ingiuriofo per un vecchio;¢ vuol dire barba schi-
i 2, epiena di pidocchi, ¢ d' altre lordure,

SANTINFIZZ A. \pocrito; de i quali a baftanza s' è detto altrove; EB per
yy  satinfieza's' intendono certi Torcicolli, che stanno tutto il giorno d' avanti a
 una immagine d' un Santo, perché si creda che essi facciano orazione.
yi, GABBADEL. Rinncgato. Vno che gabba, cioé inganna le Deita, adoran-
fio, Oggi una, e domani un' altra, rinnegando ja prima. Se bene Deus non ir-
yal 'Tetur. Si dice ancora Gabbafanti. §
ay, PON cingue, ¢ teva fei, Vuol dire Tu (ci ladro; perché ponendo cinque dita
we della mano, fai il numero di fei con aggiugnere alle cinque dita la roba, che
sath porti via. Plauto disse: Trism literarum Homo, cioè tees Habbiamo diversi
modi di dire copertamente Ladro, come Sgrafignare. Havere le mani a oncim,
: che si vedono nella presente Otttava 69. Bespemmar con le mani, Andar aCarpi,
¢ 64 Borfelli., Par il Lanzo ( che in lingua lanadattica vuol dire Ladro ) gixocar, 0
# lavorar di mano,¢ Gimili.
i 'è LAgatia di Mafiro, Quetta fingeva d' esser morta,¢ nen era,e però vuol
è " dire huomo finto. Huomo che fa il femplice,e non è. Lat, Lepus dormiens, Te-
nere gli occhi aperti, baver L occhio, ed aprir l'occhio vuol dire andar cauto nell'
Operare: ¢ perché tanto Ia lepre, che il gatto tengono gli occhi aperti anche dor-
mendo, servono a i Latini, ed a noi per esprimer un' huomo vigilante, cd ay-
yeduto, e che moftri di non efiere. Vedi sopra C, 1. stan. 19.

MONELLO. Così chiamiamo quei guidoni, che per Firenze battono mari-

 Ma, comes'é detto sopra C. 4. stan. 8. Siccome Guidone di nome proprio fié
fatto appellativo, così forse anche Monello, in principio diminutivo dt Adone,
accorciato dal nome proprio di Simone è venuto a significare una tal razza di

 
  

persone.
'. ASSASSINO. Vuol dir ladro di strada, ma quié detto in vece di furbo,o
' Ȣ pud anche intenderfi ladro di strada,
NON dubisar ti coferd falato. Sta ficuro, che ti ha da coftare aflai, 0 che ne

-pagherai un gran fio.
— TANTO va la fecchia al poxzo, ec, Tante volte si torna a fare un male, che
Seri tey ' Zz

2 una
i
i ~
"

ha ee a

   

 
364 MALMANTILE ©
una volta vi si riman colto. Vna volta' fa per molte; e diciamo ancora; Tate ) wij

volte va la gatta al lardo, che unavolta vi lascia la zampa\, ec

violantium malus eff, ed orecchie della fecchia diciamo quelle due | tL

rate, nelle quali ¢ infilato il manico di efia (ecchia. sear Ne at
S 4

TANZA LXXx. STANZA LXXL ai
Poi fente, ch'egli dopo una gran bibbia Ben ch ci creda finua '
D ingiurie dd nel facco una percuffa, Tira di nuovo, eda vicino
Che rurte le frovigiie /perra,e tribbia, Ed il suo cane acchiappa i
Ech eidiceva; Horsugiihorottol ofa; Che fa-urliche van nell
E che di nuovo un' aitro ne rafibbia., » Ona' egli fiupefarto afjai ne
E che ( facendo il vin la terra rofja ) Dicendo: Qui è quand iomi
Soggiunge:O quanto/ague banelievene! Se nce' il fangue egli ha di gi
Quella ghicttene, a me, beeva bene. Come a gridar puo egli
Seguitando Magorto a dire ingiurie, da una baftonata in ful facco, €
i piatti, ¢ fa verlare il vino, ¢ credendolo il fangue di Pigolone refta
to, che ne posia haver tanto; € replicando un' aitra batlonata, ae
po il cane; 11 quale comincid a urlare, ed ei credendo, che fuifero strida di
lone, strabilisce ¢ non retta capace, che egli pola haver piu forza di 7
ara

  

ae

PoetEstEe

frida, mentre ha versato tutto ul fangue.
DOP PO una gran bibbia, Dopo una lunga diceria, 0 filaftrocca;

Dopo haver dette tante ingiurie, che farebbono un gran libro, da Biblia Greco
Latino, che vuol dir br:; E se bene la voce Bibbia oggi comunemente ¢ istela
per il libro deila Sagra Scrittura, cuctavia noi la pigiiamo ancora ne i cascome
il presente nel detto fenfodi libro, o di lettera, © di discorso lungo, come pate
che la pigliaficro gli antichi fecondo Herodoto lib, 1. dove dice > Alarpagum
clufife, leporis ventri biblion.ad Cyrum, Se bene qui ¢ Viguerro 5 letrera, Dal po
ma d' Omero intitolato l'Lliade, il quale è d' una prodigiofa quantita di vert,
come quelli, che a(cendono al numero di quindicimila (etcecento oreantatre; ut
gran moltitudine di cose, 0 di parole, dissero i Latini 4ias, o Hiades, Propeaid
41.2, clegia 1.

oe rRs oe ST

Tune vero longas condimus Iliadas,
Seu quicquad fecit, fine eff quodcumqne locura
Ataxima de nibilo nascitur bifforia, ne

RAFFIBEIA, Replica. 'Irasiaco dal congingner con fibbia bottoni,
il che si dice -4fibbiare, Vedi sopra C, 2. tt. 81.

STOVIGL/A, Intendiamo ogai forta di piatti, e vafellami di terra per nfo di
cucina. Ll Ferrari, Seovigue, Fittiia, vafenla, & frivola. Vandena, to
comperi. 1o fimo che sia parola florpiata dalla Latina. Veenfilia, Crefe, 12:12.
E molti altei arnefi,e fovg 4 di bilogno. Pallad. volgarizzato lib, 1. tity 6 Faber
da far terramenti, ¢ dilegname, e di /ovigli da vino, da Javorare, eda
Questo ultimo non è nel Launo, ed è aggiunto nella traduzione per impiegate
voce Mowtgli, 7 3

TX/BSLARE, Propriamente vuol dire Batter il grano in fulltaia dab Latino
Tribula tribule, 0 tribulum tribuli, che vuol dire una specie di carro, ¢ol già
fquoceva il grano in fit!" ata, come si cava da Colum. 'lib, 2, cap. ie

@uifa eo PEEP ~ se Ee oe eT

 

 
 
   
   
 
  

   

SETTIMO CANTARE. 365

'unt adijcere Tribulum, & trabam posis, ¢ Varr. lib. 1.C, 25. E'/picisin area

cM tivwencis iunitis', © tribula. B questo dal Greco eribein peftare, trita-

. Latino terere, o da thlibein schiacciare, dal qual verbo viene il Latino trib.

travaglio  dettoanche da' Santi Padri prefura, '

£. Questo termine significa A mio giudizio; Secondo me. Secondo il
4/0 intendimento; ¢ per ie si dice replicatamente 4 mé a me, Quan-

» cl0é per quanto io giudico i Franzefi Quant' a moi, 1 Greci similmente

» cioè fecondo me, fecondo il mio giudizio.

DE haver finita la fefia. Crede haver terminato il negozio, cioè d' haver'

Pigolone, Similitudine trata dalla folenaita, colla quale fon facti

i, che si giuftiziano.

CHIAPP A. Coglie: perch? se bene «cchiappare vuol dir pigliare uno con
¢ violenza, ci serve er esprimere colpir bene. Latino Certo iétw a/-

Spagnuolo, acertar, Vedi C. 2. st. 41.

EF ATTO. Rimafto stupido per la meraviglia grande. Latino ob/upe-

STANZA LXXII. STANZA LXXIIL
in questo mentre:col suo fante Perch'ei del certain quanto a contentarla
 Haven di gid scorrendo pel giardino Non ci ha ne meno un minimopenfiero,

   

   
      
   
    
  

 
  
   
     

 
 
       
   

it pritrovato, e quelle piante E pero quante volte ella ne parla,
we coles, che chede sl suo Nardino, Mura discorso, ela riduce al zero;
vot! tha trata fuor belle galante, Ma perch'ellae mozzinaye con laciarla
tif be mon si vedde mai il più bel fennino, Le Afonache trarria del Monaftero,
Econ un sno bocchin da sciorre aghetts Vedeyche s*ella bada troppo a dire
i  Chiede da ber ma non gid fel' asperti, Si lascerebbe forse connertire,
" av ' STANZA LXAlV.
wil) Peri per non cadere in queffo errore E ch'ei ne venga ch' ei l'aspetta fuore,
st | Lapigtianun tratto,efe la portain frrada, eAccio con essi anch' egli se ne vada,
yl Ed ai vecchio fa dir pel servitore, Che i non vuol lasciarlo nelle pefte y
* Che'pit tempo non è di frar' a bada, 44a condurlo al pacfe alle lor felke.

“Mentre che Magorto si fludia a baffonare, il favio Brunetto col servitore eras
andatoinell' orto, ed havea trovato il Cocomero, ¢ tagliatolo n' era ulcita las
fanciulla 'che egii cercava, la quale si mefle a pregario, che egli l'empictic las
tazza, maei non volle contentarla, anzi la prefe, ¢ la porto in firada, e man-
dO i teruidore a chiamar Pigolone per condurlo seco alle nozze di Nardino.
a ANTE, Si dice i) servitore; dail' intero infanre,si come in Latino Puer signi-
" fica ferno, da noi detto anche garzone, se ben Fante però comunemente vuol dire

" - soldaro'a piede, perché ne' tempi dell' Imperio baflo, che la milizia comincid a ri- |
of a tarsi pil per ja cavalleria, che per la soldatesca a piede; il pedone G venne as |

'ttimare come miniftro., ¢ servitore del Cavaliere; ¢ percid fu detto fanre,
| SENNINO. E' una parola, che si dice per vezzi a una femmina bella, favia,
~¢ pulita, ¢ che operi cen giudizio con fenno,¢ con puntualita. Latino scita pue/~
la,feitula. z
~~ BOCCA dat feiorre agherti., Così diciamo di quelle femmine, le quali per parer
“belle tengono la bocca ferrata, ¢ ridotta forzatamente pi Mretta del. suo nau.
ss rale;

    
  
 
  
  
  
 
 
 
  
 
   
   
   
 
  

366 MALMANTILE ©

sale; ne muovono i labbri di come se gli sono accomodati allo specc
par proprio, che habbiano la bocca accomodata a feiorre un ng
Aghetto è quello, che vedemmo sopra C.2.f. 10,
'NON se? aspetti. Non lo speri. Cioè non asperti, che le dia bere « |
gnuolo ¢/perar ¢ lo stesso, che a/pettare. bes pe a My
LA riduce al vero, La riduce al nulla; Zero quella figura d'abbaco, che
se stessa non rileva numero alcuno, ed accompagnata, forma le decine, ¢ eile
per esprimer # nuda, '
eHOZZINA, Huomo aftuto, triflo, ¢ che fa il conto fino, mas'inte
genio maligno. Latino Vulpis reliquie. Questa voce vien forse da orecehi m
che così fon fegnati quei furbi, che meriterebbono le forche, ma perla
eta non ne fon capaci, sopra C. 6. st. 54., ed in questo C, st. 30. Ȣ credo
perché diciamo Azoyzorecchi in vece di moxzina nello stesso significato,
TRARRIA le eAtonache dal eAionaftero, Confeguirebbe l'impotlibile con fas
sua induftria, periuafiva » ed cloquenza. Diogene disse: Oratio non ex ani

Sfn2i* 8

proficiscens, fed ad gratiam composita meleus eff laquens, quod [cilicer blandé x
ens hominem ingulet. té
NON è tempo di star' a bade, Non & tempo di trattenerGi. Non v' étempods | &
erdere.; R
LASCTAR' uno nelle pefte. Abbandonar' uno nel pericolo, Vino fa' R
folenza, o mala creanza, ¢ per non efler percoflo fugge viaye la(cia i Ra
€ questo si dice /a/ciar nelle pefte, cioé nelle pedate, o nella strada, che &
mancamenti ha fabbricato ai pericolo,colui che è fuggito;si pronunzia cont ay
ma c stretta a differenza di pefte infermita, che si pronunzia con l'é lagaies | 4
pero questa rima ha un a di falfita, ma tollerabile, ed¢ ammefla. &
STANZA L&XV. STANZA LXXVL s
Così di la poi ructi fer partita; Brunetto si ridea di Pigolone mm 1A
Ma piis dogni altro allegra la faciullay Perch' ei parea nel vifo un fico vittty | %i
Perché non prima fu dell' orto uscita E menaua a due cambe di ee pth
Crognt incanto,agnt vogliain lei s'anulla, Com' egli haveffe hauuto i Birri drt; aay
Anzi ai lor preghi in ful caval, Salita, E la donna diceva: Grambracont, hey
Che la duri; ed il vecchio manfuttt, | 4

Senza pitt ragionar di ber, ne nulla,
Va sipreinnazs ag altri wn trar di mano Che si vedeua fatto il lor xsmbello:
Fiera, ¢ bizzarra come un Capitano. Dagli pur (rispondea) ch'eglie fafitl. | "sp
Vicita che fu sa fanciulia dell' orto cefsd incantefimo, e la voglia del bere4y
anzicon la maggior' allegria del mondo monté a cavallo scherzando, € moe | &
teggiando il vecchio, il quale era ancor pailido per lo spavento havuto, i)
"RIZZARKO. Wuol dir lracondo, Suzzofo, o cola simile, fecondo chelule | &
rono gliantichi, Ma si piglia anche per spirito(o, ¢ vivace, come è !
presente luogo. In Spagnuolo Zixarro significa uno che vada bello, ¢ fupecbo nel Ge
veitire. B similmente roba bizarra, che 1 Pranzcfi direbbero bigearree, ie) 4
roba, cio' vette bellissima, varia, ¢ pomposa, donde poi da noi si prende Bare |
ro per capricciofo, firano, stravagante. eae Bk
FICO vieto. Fico annebbiato, o afato. Vn fico, il quale al colore, ¢ tene
rezza par maturo, non è, ma dalla nebbia è ridotto giallo, come se fulle ma:

 

 
 

SETTIMO CANTARE? 367
furo: comparazione, che esprime assai bene la faccia gialla, e grinza di Pigolo-
ne. El'epiteto Viero ¢ proprio delia carne falata, lardo, burro, ¢ olio, quando

. per eflere stantij, ¢ corrotti mutano il colore, l'odore, ed il fapore.
, MENAR di spadone 4 due gambe. Fuggire; Correre. Spadone a due mani si
quella pada pil grande delle spade comuni ordinarie, la quale s' adopra
-ambe ie mani, ¢ per derifione di coloro, che, vantandofi di bravi, all' occa-
poi fuggono, col folo dire; meno di Spadone, 0 gioco di spadone, s' intende a

ye gambe, che vuol dir Fuggi. Vedi (otto C. 10, st. 3.
COM egli havefe havuto s Birri arety, Detto usato per esprimere, che uno
corra velocemente

GIAMBRACONE, che a duriDubito, che voi non fiate per durare a cammina-
re. Giambracone fu un mato, che fempre andava gridando: Che /a duri, e»
| però quando noi veggiamo, che uno faccia un' Operazione con grande attenzio.
'Re,€ che noi dubitiamo, che egii non sia per durare fogliamo dire Giambracone,

an) © (enza dire, che /a ders intendiamo; piaccta al Cielo, che egli continovi, € così & Co-

 

inteso.
BATT O il loro Zimbello. Divenuto lo scherzo. Zimbello,oltre al significato,
i @ 0 (opra C, 1. st, 59.,vuol dire aacora quell' ucceilo, che si lega per
un piede allato al hoschetto de' paretai, 0 altri luoghi, dove si tende per pigliare
ig uecelli, che tirandofi quelia cordicella, che ha legata al piece si fa fuolazzares
Per incitare gli altri uccelli a calarsi. Latino amis illex, € dallo strapazzo, che
ry tale uccello riceve diciamo Zimbellouno quando ¢ burlato, beffato, ¢ strapazato
ad da tutti; nel qual fenfo ¢ prefo nel presente luogo; ¢ forto C. 9. st. 66.
7 — DAGLI ch' egii¢ faffedo. Dag, ch' ci lo merita. Olleruifi che 1 verbo Dare
nei cafi come i presente,vale per continovare, seguitare, durare, ec. ¢ con dire
rin folamente dagés icnz' altra aggiunta s' intende /eguita; ma s'aggiunge ch' egli ¢ faf-
Selle per una certa vaghezza, ¢ per un genio,¢ naturale inciinazione, che han-
N01 Fiorentini'd: paciar per proverbio, metafore, comparazioni, o similitudini;
i

 

- € forse ¢ aggiunto per confondere,ed oscurare il detto,perché dare al fafedo vuol
se dir perquoterio, ¢ nov vuoi dic seguitare. Habbiamo due specie di tordi, cioé
: botraces ye fafjedi 5 1 primi fon meno aftuti, ¢ piit facili a la(ciarGi pighare, i fecon-
| di sono più aftuti, ¢ ad ogni poco di romore scappano, pero quando la notte col
# "s frugauolo si scuoprono, si dice dagli con la ramata, che qucfto ¢ fa/sello, che alpet-
«' ta poco. In fuftanza nel presente luogo vuol dire continuate, 0 Seguitare, a burlar~

i mi, beffarmi, ¢ firapazarmi, ch' io lo merico, Da questa aftutezza del faffello si di-
a si S¢ fafsello a un' huomo, che sa il conto suo, ed efercita il suo fapere a vantaggio,
my" pretendendo fempre pil del giuflo, e del dovere, avido di guadagnare, € tenace
* el suo più del conueniente.
w STANZA LXXVIL

us Così feberzando, com io dico,in brigiia Percio dopo baver fatte molte miglia,
i Ne vanno Lenya mai sentirfi franchi y E che tor parue un tratto d'e/serfrachi,
) Efempr' ognun pin calda se la pigtia, Tutts affannati per st lunga via y
wo. Percheilcimor glispinge,e /pronai fachi; D' accordo si fermaro a un' Ofteria,

il
ty
# STAN.

 

 
 

MALMANT DLE) S G0 —

  
  
 
   
 
  
   

368
STANZA Laie au a
Dove il padron che intende fare. pasto >» Ben. ”, '
Trovagran vroba per yi garbato 5 'Guamioiinfa ha a0,
Chreitien che afar no habia rroppornaspo, E che quelia »
Mae? non fache enon hanno definate; Che's,

Brunetto con la sua compagnia seguita allegramente 'il suo vi
do per il timore, che hanno di Magorto y ma sti ia 7
un' Ofteria, dove mangiaron più di quello, che il padrone non: z

SC HERZARE in briglia, Quetto detto, che significa uno, che flando|
faculta, ¢ d'ogni commodo, non oftante G duole dello stato fuoy éd
anche per intender'uno, che stia allegramente,¢ scherzando fenzae
che egli è in grandiflimo pericolo;¢ così s'intende nel presente luoga,.

scherzano senza pen(are al pericolo,nel quale sono » che Mas arrivi
dosso, da-chanhp, peas

OGNVNO se la pigtia pits calda,~Ogauno se ne piglia maggior
sto pigtiarfela caida i Franzefi esprimono col verbo chalsir, ¢ noi cal
dal Lat. calere; Boccaccio nel Poema in ottava 'rima intitolate il

 

de' fatti di Tefeo 1, 2.
Oude li se nuova vifion vedere; s OSes i
Perché di ritornar li fu in calere. 2 Mey
E appreflo. Vici d' Atene, ne li fu in calere, ae
D' Ipolita amor dolce, e pudico. me
Spiegd la forza di questo verbo il Petrarca quando disse } ee

We dentro fento, ne di fuor gran caldo;
Che fa come una spiegazione de' due versi immediate precedenti:.
Ne del volgo mi cal', nedi fortuna; oy gaialah
Ne di me molto me 'di cosa vile, ome
GLI parne d' esser franchi, Parue loro d' esser in ficuro, ed esser liber da Mo

orto. hoped
OO ARE 4 pasto. Si dice quando I Ofte senza prezzare cosa per cola di quell

che mette ia tavola vuole ua tanto per persona, ¢ mette in tavola quello yee

are a lui.
f NON habbiano a far troppo guafto. Non habbiano.a mangiar molto, Le.

  

aPEFER

 

feo incognito dice.
Jo ero fario, ¢ non fei troppo guafto,
Il Berni in lode delle pesche +
Dioscoride, Plinio ye Tecfrafto
Lon hanno scritto delle pesche bene
Lerché non ne facevan troppo guafto,
Cioé non ne mangiavano molte, perché 'non gli piacevano.
V? & rimafto, L' ha fgarrata. E' rimafto ingannato,, come chi
trappola. ssf
LON vi refta fate. Non vi refta nulla. Vedi sopra in questo ©, stan. 7}
Mattio Franzefi contr' alle sberrettate dice;; cals
“Hed
Ae

FREER

we

 

 

ss,
 
 
 

  

SETTIMO CAN TARE 369

A cavarfela, ¢ metter pitt di cento
> *Folte per hora, +l che non serve a fiato,
va dietro alla cafserta, Cioè non si gaadagna, ma pil tofto si perde,
TANZA LXXIX. - “STANZA LXXX,
sntante | frracco S' 16 percoffi quel vecchio marivolo
wil randello a quel partito, Com' ha io fatto, disse, un canicidio?
'ciolte,ed apertohavedo omai quel/acco Sa ch' io lo prefi ye la ferras qui folo y
sencinar la carne del Romito, Che gnun porea vedermi,o dar faftidio,
Ed in quel cambio viftovi il suo bracco Won fo s' 0 sono il Graffo Legnaiuolo
coceh 5 vetri macolo,¢ bafito, A queste metamorfofi d' Ovidio,
fa miaravigliato in una forma Che sono in ver meranigliofe, e frane

     
   
  
  
 
  
     
  

  
       

Ct ei non fa s'ei sia defto,os'ei si dorma, Poi cnn Romito: mi dinenta un cane,
; STANZA LXXXL
e'| povera Melampo Lo ho una rabbia addofso ch'io avvam:
Che | nétte gua tencé oui Jaci; Con quel veechiaccio barba d'Olo sae

  

Chi più farò la guardia al mio bel capo Ch al certo fatto m' ha così bel ginoco;
defio, che t' hai cliinfe le lanterne? Che dubbio! metcerei le man nel fuoco,
eo Magorto'dal baftonar quel facco lo spiccd dal palco, ed apertolo vi
il suorcane; ¢€ reftando maravigliato, fuppone che sia stato Pigolo-
li habbia fatca questa burla.
ere In quella guisa; in quella forma, in quella maniera,
kntendi frammenti di piatti, pentole, ed altri vafi di terra,
Pe.Badro, giuntatore. E' voce Napoletana,'ma già facta Fioren-
tina, A 4 7
CHE gine pores darmi faftidio, Che niuno poteva impedirmi, La voce gaxno
per nino, hogei @ usata folo da 1 nostri contadini,
NON fos' io sono 11 Grafo Legnainolo, Non sos' io mi sia diventato ur' altro, il
'Graflo Legnaiuolo fa vn Fiorentino, il quale fu tanto femplice, che gli fu dato
@ credere, che non era pili lui ma diventato un' altro  ¢ per questo tale fu mefio
! ¢ alloppiato, ¢ fatto dormire quando si rifenti, s* accordé a paga-
Te le tpee je le cancellature per il precefo delitto, del quale fu affoluto, benché
havefle Confeffato'd' haverlo commefio come nuovo perlonaggio, ¢ pagd il dena-
10 un fratello di quello, che il Graflo si credeva d' eflere, ¢ duro in questa cre-
  denza qualche temipo; ¢ fin che li suoi veri parenti lo fecero riconoscerfi, ¢ ritor.
share: che egii cra. La Novella pare a me, è stampata dietro alle cento No.
vellea dell' edizione de' Giunti. Da coftui digiamo i Graff Legnaiuolo per
intendere un' huomo femplicissimo., ¢ facile a creder ogni-cola, bench' ei sappia
non efler vera, ed esser' imposiibile, che ella sia. Si dice ancora Calandrino, ¢
Cap,,comie aecennammo sopra Cy 5, st23. t,
VE Romito mi dineneaun cane, Se bene intende, cheil Romito era diventato un
'caN';/perché nel! facco trove il cane, vi haveva meflo i) Romito, si potrebbes
'anche 'che intendefie parergli gran metamorfofi, che un Romito y cioé ua'
i bene jdiventi un cane, cioé nno (cellerato. i a
- MAL chitfe-ie lanterne'; Hai chiufi gli occhi;-ed.intende fei morto,, Chiamanfi
Anche gli Occhi sccicanré'in lingua furbesca; € Così li chiamo in un verio del suo
fio Brunctto Lauini Macftro di Dante. Aaa 10

an

+225 28

 
  
  
  
  
    
   
 
 
  
    
  
  
 
 

wi tie Ea =

   
  
 

 
   
   
    
    
  
  
 

370 MALMANTILE ©&|

10 ho una rabbia addosse ch' io avvampo, Latino Jn fermento totus si
collora, un' ira grandissima. vvampare significa abbruciare leg)
cfempio; Vn panno bianco accoftato a una fiamma s' infuocola,e piglia
si dice arfo, o abbronzato, o avvampato. » # Sie

BARBA d' Oloferne. Barbaccia. E' nota la Storia facra di Iuditta,
la testa ad Oloferne. Nel pepeeiones detta storia, li Pittori per far
Oloferne per un' huomo crudele, dipingono la di lui testa tagliata brate:
barba lunga, folta, ¢ rabbuffata; ¢ da questo il dire a uno barba a' Olofer
giuriofo, perché fuona anche lo stesso, che refta d' impiceato, “

 

&8& set ise

wr

   
  

   
   
   
 

METTEREL 1a mano nel fuoco. Mi par d' esser così certo di gaefta cola, cheio | »,
la giurerei con metter la mano nel fuoco. Vno de' giudizzi, che chiamavano' wi
vini, appreflu i Safloni era la prova, che faceva il reo, via del suo > 1
nendo in mano ferro infocato. E le folennita, colle quali si veniva a qu -
va, sono descritte puntualmente dietro all' Iftoria clica di Polidoro Vi j "

TANZA LXXAIL STANZA LEXXIV be
Oimé le mie stoviglie, ¢ il vin di Chianti Ma perch' ei vede quivi le a Tis

Chrio tolfi in dar la caccia aun vetturale Volte al giardino,e poi versolavity | Gy

eA cagion di quel trifto Graffiafanti CheBrunetto,equegiialtri: tu

in um tempo ¢ versato, ¢ ito male, Quando v'entraro,e quando andarovia R

Giuroal Ciel ch'io non una ch'ei fene vati, Infospettito, lascia andar il Frate, Pe

E,s'¢i non vola, puo far capitale Ed entra nel giardino,e a y i

Chr io voglia ritrovarlo,e s'es c' incappa Scorge quel suo cocomero dit an

Che mi venga (a rabbia s'ei mi feappa, Ch'e frato il fargli un fregio sopr w

STANZA LXXXIIL STANZA LXXXV, | 'y
Lo troverò bensì, perch' io vue ire Poiché levata gli han quella fighualay | x

Qua intorno per veder s'io lorintraccio; Chiin essa(cons' io ho dette) si trouwvs, |»

Cos} corre alla porta per uscire y Per la stizza non puo formar )

Ma cei nd puofarlo,perché e' v's il chianaccio Si soraffia, barre s denti, efa z

Lo fquote, e shatte per volerlo ape eS E spalancando poi tanto di gola dey

Edhor v'attacca 'uno,hor altro braccio; Verla,befemmia il Ciel, z ne

Noiato al fine vanne,e corre ad alto, Dicendo; QO Macomettoe ¢

E dai balconi in frrada fa un falto, Che si facciano al mondo i")

STANZA LXXXVL mime | i
fa quanto a te chi ti pisciaffe addofso Sapro ben' io a coftor far shy. «thy \ y

So ben che th non ne farefti cao; Credilo pur, percht, se si da il cafe h

Ma io che da miei di mai bevvi grofso, (Che si dara feny'altro) chia, eT is

E le mosche levar mi so dal nafo 4o me gli vue di posta ingoiar vit, t

Seguita Magorto a dolerfi della sua di(grazia; poi fata risoluzione d'é ty
cercar del Romito, falta dalla fineftra in strada, dove vedute alcune. kj
fo il giardino, infospettito la(cid il pensiero d' andar cercando di Pi bry
ne va alla volta del giardino  ¢ quivi accortofi del ratto della fanci u
di yoler trovare coloro, che gli hanno fatto questo torto, ¢ di, volergli tu &
goiar vivi. Nota che il nostro Poeta in qn ottava 84. ¢ stato cri dy
ché s'¢ servito della voce ia in tutte tre le rime, ma tal fottigliezza fb iy
tofto chiamare ignoranaa, perché se bene & fempre la stessa yoce, “—— 7 j

 

 
SETTIMO CANTARE: 371

fempre diverso significato, perché la prima significa strada; la feconda significa,
altrove, 0 moto da un luogo a un' altro, ¢ la terza significa modo, guila, ma-
~niera, ec, E di simili rime troverai altrove in questa Opera, ¢ fempre le vedrai

    
  
    

 

 lodevoli per l'artifizio, più tofto, che biafimevoli per ta poca avvertenza.

- AOL. Esclamazione, che esprime disgulto, 0 dolore. Latino Hei mibi;
- CHIANTT, E' una regione in 'Toleana dove nasce vino buonissimo. E Vettu-

 intendiamo colui, che sopra alle beftic conduce vino, ed altre robe da un,
ogo all'altro; a differenza di Vetvurino, che prefta, ed accompagna caval-
a lettighe, ec, a i Viaggianti. Vedi sopra C. 6, tt. 37.
| DAR fa caccia, Correr dietro a uno. E propriamente si dice Dar /a caccias,
 quando i birri corron dietro a uno per pigliarlo.
git = GRAFFIASANT!, Bacchettone, lpocrito, E' lo steffo, che Santinfizza det-
igi to sopra in questo C, st. 68.
più PVO? far capitale, Pud efler certo. Qufta voce Capitale significa lo stato, o
Ill faftanze d' uno: tale ha 10, m, fendi di capirale. Significa aflegnamento. Chi
yt del mio fn capitale detto sopra C. 2. st. 7. Significa forte principale. Latino Sors,

i detta
i

   

,

yah

   

qs dat Greci cephalaion, cioè caput; dagli Spagnuoli candal, che corrisponde»
niall nostro Capitale, ¢ Candalofo dicono colui, che ha gran capitale, cioé grandi
we = fallanze. U/ rale ha havuto la fentenza contro, ed ¢ Stato condennato nelle [pefe, ed 4

are cento fendi di frutti, e mille di capitale. Significa quello vedremo foto C. 8,

 

u@ 1.65. Qui significa pud credere, pud esser ficuro.
jah «SET c' inciappa. S' ci mi da nelle mani. Se, c' incoglie. S' egli casca ne' mici
ei) Meguati

i) UI venga la rabbia, Giuramento imprecativo contro se steffo. Giuro di voler

yj far latalcofa,, ¢ se non la fo, mi fottopongo a ogni maggior tormento.

ait 8" 10 to rintraccio, Traccia significa orma, 0 veltigio; onde tracciare vuol dir

«at 'seguitare le pedate, ¢ per confeguenza qui intende: Se io lo ritrovo; Traccia si

iti dice quella strada, che fa il cane per la paflata della lepre, o d' altro animale

im fiutando; viene questo verbo rintracciare, che vuol dir Ritrovare, ¢ rraccia-

jus) ecetcare, Latino vesticare. x

ynt © CHLAFACCIO. Elo steffo, che chiaviftello detto sopra C. r. st. 69. che i Sa

se nefi dicono pestio dal Latino pefu/us. 11 Conte Vgolino preflo Dante Inf. 33. Ed
io fent) chiavar ? nscio di sotto all' orribile torre; cioè mettere il chiavaccio.

rit «= A QVELL Avia. A quella foggia. Inquellaguifa.

ull PARGLI uno sfregio in ful vifo. Fargli una ingiuria ignominiofa, si come sono

iii Bl sftegi. Vedi sopra ©. 2. st. 3.¢C. 6. tt. 54.

PAlabava. Intendi ha gran rabbia. Latino fromachatur, Che bava quell'
que. UmMOre viscofo, che da per se fleflo ca(ca dalla bocca come schiuma, come si vede
eS se i cani arrabbiati, donde ¢ prefa la presente metafora. Si dice ancora: Afi
: fs venir (a bava'; di chi mi fa entrare in collora, ¢ di chi noia forte,

La ML Ciel minaccia', e brava. Sgrida, ¢ minaccia il Ciclo. Vedi sopra C. 5. st
ff Gx, che dice Rabbiofa,il capo verso sl Ciel tentenna, che & quel minacciare il Cielo.
2 'Di questo verbo bravare, che vien dal Provenzale il Varchi ne fa un lungo discor-
ut fonel suo Hercolano, ¢ lo giudica molto esprimente il latino obrargare. Catullo.
ty:
6

è Aaa z Gel:

 

 

anes

Ate

 
| 372 MALMANTILE >

t+
] Gellins audierat, patruum obiurgare. falerb § 5 SOV si Dayne
| St quis delitias oe. 9 aut factret, et)
| TANTO di gla; Gola assai larga. Vedi fortoC, 16. st, 18, '
| ce tanto usata in questi termini, è tote Fig i
NON ne farefti caso,quand' uno ti pisciafe addofso, Non ti
j non t' importerebbe quand' uno ti pilciafle addosso; ed intend: Sei
ne, ¢ codardo, che fopporterefti qualfivoglia grandissima ingiuria
ne. Vn'antico Poeta per voler esprimere uno scellerato,¢ ingiuri
| ria di suo padres dice:patrios mincerit in cineres, B Pittagora in uno de'
boli per dinotare il rispetto, che si dee portare alla Divinica,
non si pisci in faecia al Sole.: itty
NON bevvi grofo, Non fopportai mai ingiuria alcuna. Ber
Non la guarda così per la minuta, ma m5 8 ogni ingiuria senza
ne, fingendo non fen' avvedere. Tratto dal bere le medicine, le quali
faporano, ma si mandano git a occhi chiufi, » 0: Soa sos Say
| MI fo levar le moscbe a' intorno al nafo ~ Mi s0,vendicare dellvingiuri
cilitd, Omero nell' liiade La preftezza, colla quale un Dio fa tornare indietroi
colpi avvelenati contro a un' Eroe compara al cacciare d' una mofea, che fa las
Madre dal corpo del suo figliualo. © J oenetige
FAR ilc,..rosso auno, Galtigar' uno.. Tratto da i Pedanti, i quali )
i ragazzi perquotendola in ful ¢.. 5 ¢ gliclo fanno roffo con'le)
sopra C, 4. st. §1. i: 1 a 0b if Settee SIR
oe. Subito, Viene dal giuoco di palla, che si dice Dar di nr.

  

 
   
  

 

 

si da di primo tempo, cio¢ avanti, che la pallatocchi terra «Lavinond seftigio«
INGOLARE, E' lo stesso, che ingollare detto sopra Cas st: 63, ¢ vuol die mat:
dar la roba gil: nello stomaco. ' ol Owe
STANZA LXXXVIL, STANZA LXXXIX:
Ma dove col ceruel fon' io trascorfor Quel detia Cella del Romite eiil
Pits buesdi me non ¢ forto le frelle, Ove trovaede if ibrvepuenia
Perchinnanzi ch'io habbia pref Porfo Intana dentro, e non wi scongeniing)
Vio (come si fuol dir) vender ta pelle 5 Fruga,erifrugainquaye iia aea
Fatei ci voglion qua, perch' il discorsa Sgomina cia che v'édafommoywinty
Fuor che ai Senfali non frutto covelle, M14 tutto in vanoyondeghial,
E mal per chi ha tépo,e tempo aspettas Sen esce con le man pieae ns yente s
Che mitre piscia il can,ia lepre sbierta, Ma dieci volte più dimal talento,
STANZA LXXXVIIL STAN ZAULX Xia ly
E peri prima, che 4 vila a gamba « Entra neh bafeayeogni,
Vaa fuga mi fuanin divconcerto E in fammea ne cored pax
A casa Pigolon vuogt ix-di gamba s | LB wedde' 5 fanz ain '

Che vi [ard coi compliciidel certo, vorrde pigiato esser bidval far decent.
ares fice ribinfe

Così conchinfo, correch? diff gamba y 4 «| Onde nelifine alf

N

Ne np bp cw @ eee FZ cee Fees SF ee hese KS

E come un bracco taper-quel deferto Che pur mol vendicar segtandiane

Tutti quanti quei benghi a uno 4 uno Così v' ar rivera. po' pai in
Cercando.s'ei vi scmopee,o fentecaloxne 1 Seveifuffe( di i ne

K aia MS

 
    
   
    
 
 
 
   
   
  
 
 
 
  
  
  
 
 
 
  
   
 

  

SETDEMO CANTARE: 373
° STANZA LXXXXIL;
Poiche Brunetro, ¢ le sue camerate

Pagaron L offe,( il quale assai contefe,
Perché le gole lar. difabnare
Gli eran parute meeety Spefe)-

». Partiron,, ¢ poi dopo-altre fermate y,
Sie SUnapdeledee

  

i di quanto have 10 E giunto a casa,ringrazianda il Cielo
Ve più 5 ne manco ne segui  effetto. Entra in fala,¢ di posea fa un belo.
yi STANZA LXXXXIUL
Trovan Nardino acor di male oppreffo,
E sbietolar Jo veggono ancor Lui,
L' Alhante, che porgevali horxata
faper. 9 me men per cui, >, Purine faceva lafua quattrinata
Magorto lai lamenti, ¢ si,mette a ceecar di coloro., che gli havevano ru-
t la Figliuola Ȣ nen gli trovando nella Cella del Romito, ne in alcun altro
ricorfe-a gl'.incanti; co i quali coftrinfe tutti della casa di Brunerto a pian-
3 onde Brunetto con i compagai arrivato a casa subito comincid, ed
icompagni a piangere.;
fon' ia feorfo coi ceruello' Che armegg' io? Che giro.io? Che frenetich' io?
SD pWVOMe fetta le frelle it più buedi me... Ao sono il maggiore ignorante che fa nel
Mondo Vedi sopra C. 6. stan. 98. Sarr /a Luna; i Petrarca..,Arda.s 0 mora,o
dangnifer sun pin gentile Stato del mo non ¢ forte la Luna, '
thsi Bi: da.pelle dell' orfo. prima di pigliarlo. Fax aflegnamento fopna una cola,
che ancora non s' ¢ confeguita, ed ¢ anche molto dubbio/o.Ji confeguirla..Eflen-
ido anati cre Giovani per ammazzare \un' orfo'si| quale faceva molto danno,

prima che atrivafiero.al luogo dove foleva trovarsi l'orfo, si fermarono a un'
Bieria ed havendo aflai ben mangiato, ditlero all' Ofte., che lo paghercbbono
son denaci: del danatiy a, che haure bbano dato loro le, Comunia per V'orfo,
\uhe walevano ammageare 5 ¢d.amapifi verlo dove stavala ficra, subito, che las
veddero fidiedero a fuggire, ¢ uno di loro fali sopra ad un' albero.s 1' altro scap-
- PP Viayyediiltcrzo fu lypraggisnto dail' Oxlo ei quale bavendolelg,caceiato fot-
~todtinfran bene beng sdi'por gli.accoflo digrife ail' orecchio,, ed intanto quel
meschino se neftava come monto, senza muoyerf punto; ¢ perché J' orfo naw.
cralmente (secondo. dicona alcuni.) quando: ee ede, che.! apimale da Jui. afaltato
sia morto, non gli da più fattidio, credendo che coftui fatie morto,fen! andd,7¢
'€Gluiideyd si 5 ed ay vind vero la Città cutto mal coacio.. Quella y.che,era fa
 litodimaulimalbero:iccle-s ed.accompagnatoli coneflo, gli domandé quel che,gli
 havefleidetto.? orfo nell' orecchia y sd egii rispole = Mijha derto, che io non mi
“fidi pili difimilicompagai come fei G28 Che 19. now veoda.a, pelle.dell or(o. se
jeteemeete ho prefo,.: B.da questa.novelia-habbiamg il prefeuce proverbio, che
idictlanche: Vender t' uscelio in fu la frascas L Geesi ditiero: Anrequam pisces

  

    
 
    
 
    
    

 
  

BSELEAL:

 
   
   
 

 
 

  
     

smuriam misces.. J.
- MAI frutd covelle. Non fa d utile alcuno, Covelle & voce romagnuola.e vuol
dire. Qualcofa, E' poco usata nok Fiogcatino fuor sie da qualche.consadino. Il
TANS:: valore

ERLETELLE

=

=

 
    
  
     

374 MALMANTILE ©

valore di questa voce è assai copiofamente espreffo dal Copetta inun f
sopra il non covelle, Nel Decameron trovati Cavelle per lo steffo
Lat. quod velles.

 
     
     
 
   
   
 
 
 
 
     
  
  
  
    

epee ae!
E' mal per chi ha tempo, € tempo a/petta, che mentre, ec, Mal fa colui, che ha
do l'occasione pronta perde il tempo, ¢ non la piglia,perché mentre si
J occasione fugge: E' noto il verso: Fronre capillaca post i
verbo sbiettare l'habbiamo anche sopra C. 5, stan. 30, Adentre il can piscia,
se ne va. 1 Latini ditfero Semper nocuit diferre pararis; fecondo Lucano, di¢
forse Dante nell' Inf. C. 28. disse:: 7
Questi feacciato il dubitar fommerfe 
In Cefare affermando, che il fornite
Sempre col danno L) attender foferfe.
PRIMA che a viola a gamba, ec. Incende prima che d' accordo se ne
Viola a gamba ¢ il baflo di viola, Fuga è specie di fonata a capriccio,
vuol dir Suonata concertata con diversi Rrumenti, ec. Econ questi
tende quel che s' ¢ accennato. }
INT ANA, Entra dentro. Si serve di questo verbo anche forto
25. se bene ¢ improprio; perché vuol dire Entrare in una tana, 0 buca
rebbe intanare una volpe, un taflo, un granchio, ec, cutcavia ¢ pur
to come nel presente luogo. ei
AMO. Niuno. Dal Lat nemo. Voce oggi usata folo dai contadiai ell
nostro Poeta se ne serve anche forto C, ro. stin. 37. in bocca d' un
SGOMINA, Si dice anche (gombinare,( contrario di combinare,

piare, unire ) e vuol dir mettre in confulione 5 o fortofopra tutto | che si

maneggia. Lat. perturbare,.
DA fommo aime, Frafe latina, che significa Da capo a piedi: Dalla fomal-
ta della casa, fino a i fondamenti di essa: Petrarca Trionfo della Fama, ene
Onde da imo Perduffe al fommo t edificio fanto, + ae
LE man piene di vento. Cio' fenz' haver trovato, 0 conchiufo nulla. Nellie
Scrittura. Et nibil invenerunt in manibus (nis; che diciamo ancora Con le troment
facco Ter, disse Infetta re. ee |
DI mal talento, 1 collera,e con volontà di far del male y ¢ di vendicatl:
Varchi Stor. lib. 4. Erano verso i nobis di malissimo talento, ne altro per manvweit:
gli alpettavano, che quel che avvenne. BY frale usaca dai Boccaccio. =)
NE cerco per mars e monti, Questo detto iperboiico è uiacutimo per esprimett
Ne cercd da per tutto; Viene dal Latino. ee
SENZA metteria in forse, Senza dubicar più. Senza metterla in dubbio. Dd
mettere in forse fece Dante il verbo inforfare, seguivato in cid dal Petrarca +
IL pigiato esser (ui al far de* conti. A consideraria bene} offefo', e-beffaroem
folamente lui. Quattro giuocano infieme', tre vincono, ed un di loro folamet”
te perde; questo tale si dice è/ pigiato, cio® quello, che ha gli altri addosso, et
cui G spreme il denaro. Bs' intende in ogui cafo, che ja disgrazia tocchia ue
folo della conversazione, ¢ tutti gli altri habbiano (oddiienioasl “, outile dal
danno di lui.: Ae:
POPOL in quel fondo, Vedi sopra C, 2, tian,,3.

 

 
 
 
  

   
 

y '"SETTIMO CANTARE.: 375

 BeANNO avanga. Vanno fecondo il desiderio. Ex animi eins fententia ille cr
wnt. Noil habbtamo da i Contadini, che quando si rende loro facile il lavo-
'la terra con la dicono; 4 lavoro va 4 vanga, cioè bene,¢ come si de-
« Bvangad strumento ruftico fatto a foggia di pala,ma di ferro pil
» © pill acuta, del quale i contadini si servono per rivoltolar la terran.

edi sopra C. 6, fan. 69, al verbo impiallacciare., Columella lib, 3, 1a chiama do-
ira,¢ perché questo nome vuol dire pil tofto la piaila, forse Columella inten-
ee flrumento usato a suoi tempi, che faceva sopra alla terra l'effetto che
t pialla sopra il legno, ( come ¢ hoggi la marra scopaiola, della quale si fer-
Uono i contadini per ripulire,¢ radere i boschi di scope per disporgli alla semen-
 ta della fegale ) perché,se volefie dire la vanga,haurebbe detto acuta dolabra fodi-
%,¢ non abradito: E la vanga si trova bipalinm, in Varrone: /d priss bipalio

 

  
  

  

  

STVMMIA di furfanti, Scelleratissimi, ex omni vitiorum colluvione concreti.

i 'Stammia,scbiuma, 0 spuma, & quello escremento, che nel bollire una pentola.
* piena di carne, ¢ di acqua manda alla superficie, il quale si butta via, perch ¢

th Imm ia; onde fummia di furfanti, 11 peggio, che sia nella furfanteria. i

hay! ( difabicata. Lat. gurges, Così diciamo di colore, che fempre mangia-

yi —— si veggono fazzi.

“an paruti cari per le (pee. Exa parfo all'Ofte, che coftoro haveflero man-

ot 9 troppo. D' uno che sia buono a poco, ¢ mangi aflai, ¢ che vada a servire
od | }; Beli ¢ caro per le spefe; ¢ intendefisfe gli da pil del dovere,¢ di que! che
wr en sua abilita a dargli solamente mangiare, senza dargli danari per prov-
a ¢. li Lalli nelia sua En, Ir. C. 2. stan. 130.

Sit Non vagiio un pel; fon caro per le spefe.:
Dit! DI posta fa un belo Subito comincia a piangere a tn Vedi sotto C.9, st.21.
ig) © SSHETOLARE., Cio piangere. Vedi sopra C. 4. stan, 16.

AST ANTE, Intende colui, che aifisse al servizio di Nardino infermo, 4fan-
oiil 34 si dicono quei Serventi, che affiftono a servire gl' infermi negli Spedali,¢ guefti
it Aoglion efler chiamati dalle persone comode ad aififtere alli loro infermi, ¢ pe-

10 qui lo chiama col nome 4' 4/fante, (upponendolo uno di questi tali.
in ORZ ATA, Bevanda rivfrescativa fatta di feme di popone, orzo, ¢ zucche-
yo 79 deniflimo peiti ¢ liquefatti con acqua,¢ paffati per (tamigna, si da per Jo pib
ra febbricitanti;detta anche /arrara come habbiamo veduto sopra in questo C. st. 12.
NE faceva la Sua quattrinata, Cie faceva la sua parte del pianto,

Ae STANZA XCIV. STANZA XCV.

oe /ardin vede colei bell' ye verzofa Mettere pur così le mani innanzs

ro Com' appunto ? haueva nel pensicro 5 ( Rispond' ella) Signer per non cadere,

i E dices Benuenuta la mia [pola, AMentre,temendo ch' io non mici feanzi,

1) Ko 9 tt piacere a se da Cavaliero. Specorare si bench? è un piacere:

we 4a voi piangere ? ditemi una cosa Ch io mi vi levi, ditems, dinanzi,
Bs Koi ci venite a malincorpo, ¢ ¢' vero? Che voi non mi porete pix vedere 5)
RD. Non vogliate risponder che ¢' non sia, Senza darmilaburla,ch' io m' acquicta,

" Perché

#

 

bé vei mi direfti una bugia.

E Senza replicar do voita a dreto.

STAN-

 
      
 
      
      
 

+
  
 
 
 
 
 
  
   
   
 
 
 
  
   
  
   
   
   
   
   
   
  
  
  
   
  
  
 

376

STANZA XCVIV

Ne fofopra la man non volterciy
j Che Pandarese lo far mi fontitt nia,
| Eben c'al mondo t6\sia come gli Bbreiy
| Che non han terra fer mayo patriaalcuna
i Andrd pensardo incanto a farki ries
Per veder di trovar migtior fortuna',
ct fond » come diceva Afona Berta: vr

t Chi non mi vuol fegn'e che non mi merta, - Pero non vogliar
? STANZA XCVHL: %

   

Ella foggiunge, ed Egli ribadi/ee j
| Ella non cede, ed ei risponde a thiond.; © ” ogmoraincafa 5 fuora y
Pur gliacquiera Brunetto,e al fin glsunifee,  ( Perch fempre si fnmena,
Sicché 2 un 0 altro chiedefi perdond} « ©) Hantioa tener agli ovcbi |
Nardino vede la Fanciulla; ¢ la trova per appunto'comie fel" era'
|; ma vilto che ella pi angeva le dice, che dubita, che ella' sia venuta mi
ed ella gli risponde, che dubita, che più tofto egh non la riceva vo!
pra questo seguitavano a contraftare, ma Brunetto al finé gli raj
tutto questo ognuno seguitava a piangere. i ¢
j VOI ci venite a matincorpo, Voici venite malvolentieri,'¢ con
| soddisfazione;cétra' flomaco,cOtra voglia,fatrone'taa fola parol: come
METT ETE te mani innanzi, Queito certnine ci ferae per esprimete
accufa un' altro di quaiche mancamento, del quale merita dj esset 4
per efempio: I ragazai dello Spedale degi*imhdceati, i quali si
fieno tutti baftardi, in occasione di contraftare'con alcri ragazzi,
giuria che dicano a quelli ¢, 7 /ei bastardo, pecche non sia detro a”
fio fidice: Aderrer le mani sananzé + © vi si aggiugne anche; per
prevertere,occupare. -
NON mi ci fanzi. Non mi fermi in que(ta Casa per fempre.'
SPECORATE. Piangere. Diciamo be/are per piangere per 1a
che ha cobbelar degti agnelli, € delle pecore certo planco Jango', che!
rei bambini, come accenuammo sopra C. 6, ttn, 22. e'da queito
Specorare in vece di belare, ¢ s intende piangere.; nist
St ben ch' è un piacere. Tanto bene, che € un gusto a sentirui, ¢ vedertis
NON ne volterei la mano fortofopra, In quetta cola io fouo imdife;
poco ny importa il faria,o non farla, Vicne da i Latini che discvano
Ne manum quidem verterem, + tye
ESSER come gis Ebrei, Ciok non haver luogo che sia\suo propri
ra il Poeta medesimo dicendo > Wun ho terra jermayche intende terta'y
abitazione fermata, ¢ stabilita per-lei, che per altro Lerra* termat
paefe, che non Ifota di mare', Lar, continens', + syageael
VOI vi levate in barca, Vou cntrate in colicra, Vedi\sopra Ce
dice anche abarcare 5 ¢ 2 iracondo, o vero facile al? iray chek
» derocholor& dette da NOt Aacmo as poca levacurascioe che'ci'vudl poco |
rein collcra.

~ ida non per
Co

 

   

a
3
 
 

SETTIM/O CANTARE; 377
Qui vuol dire fofferenza, 0 pazzienza, che per altro' Flemmas
accennammo C. 3. flan ages % ho
$4, Iraconda, Vedi sopra C, 1, Aan, 29, Alewni critici hanno
dra quelta refs; giudicandola rimafaifa in'tigaardo-dell' 5s, dolce di
'studardi ai/perte/a.,¢ dell', 0, Jargo diquelle, ¢ fretto di queste,
voglio quictare, ¢ difendere il holtro Poeta col Riscelli, 0 cons
'non mi fon'voluto pigliar la briga di vedergli come non neceflaria,
bem loro un' efempio d' Autore claifico il quale dice.
Hag. Lb 9.30 La verginellae simile alia rofa ~
9 5 Nei bel giardin /u ta nativa spina,
Mentre fola,¢ ficura si riposa
Ne gregge, ne\pastore se le aunicina
, Li aura fuave, el' alba rugiadofa, ec,
Eg con questo efempio.( il quale sia per regola, 0 per licenza) di fal-
ire il nostro Poeta, ¢ quietargii, aucor per J' altre, che hanno offeruatese sopra
stan, 13. Rofa  prola, ¢ cola; ¢ foro in questo C, stan. 103. Sposa, cola,

BADIRE, Ribattere, conficcare
Vedi sopra'C, 2. tan. 79.
DE 4 two, Rilponde aggiuttatamente, ed a proposito di quel, che si
verbum audit, tale dicit. Si dice anche Rispondere per le Rime. Las
iilitudine € tracta dalla Musica; la feconda dalla Poesia; B allude al co-
he de' Poeti che indirizzando /' uno all' altro Sonetti,e¢ proponendofi que-
levano,e le scioglievano in altra eguale composizione teffuta delle
t eliine rime, il qual coftume venuto dall' antico, si mantiene anche in oggi.
ful - Sl fmacica, ¢ B cola, Si manda escremeati dal nafo,¢ lagrime dagli occhi
oi M 'piaato, che/moccicare vuol dire mandar fuori mocei, che € quello
 eleremento del ceruelio, che esce dal nafo detto da i Latini mens.
PEZZVOL A. Pazzoletto, 0 Moccichino; ed & quel pezzo di panno lino, che

  

 
  
 
     
     
   
  
   

dall' altra parte un chiodo, Vale per re-

 
     
 
   

 

i si lo di se per uso di nettarsi i nafo, —
sl) py, SEANZA XCIX, STANZA ©,
fs wa in un continuo pianto, E veduto ch' ell! ¢ tra buona gente,
; om - ees: 3 i &
| Phangona i fer wise piangon gls animali, Moglie a! un ricco, € nobil Baccalare,
i Onde th guazzo per terraétale,e tanto, Eche Sok le puo mancar niente,
iB Chee Portan tutti quanti gli frinali. Per ch' elinein unacafacome un mare,
, iamoa Magorto, che fra tanto Won vi fo dir ¢ ei gongola, e ne ente
a Per faper quel che sia di questi tai, Contento grande, € guffo singolare,
il Ez dowe 1a jua figlia si rnrovi, De-modo ch'ei si pente,affligge,e duole
a = Hes fato ab confuse incanti nuovi. ao — ha fatto ye rifarcir lo vuole,
RS ae abled. 5) o STANZ yt
Pi un fae cogno, E poi che dentro pitt non ne puo porre 5
& Sumnonipencace, a nae foot; Biphedi pte "Luo aspetto ¢ molto brutto
lage ¢ 0,enecomincia a corre, Si lata, vipilisce ye raffaxcona
a ~ Darando fin che hebbe pieno tuto; E rimbeuilce tutta ta persona,
0 SS AAMERabs he ig

Bbb STAN-

 

 
  
  
  
    
  

 
  
  
 
   
   
    
   
 
  
   
   
   
   
 
   
    
   
  
  
 
 

378 MALMANTILE; =.

 
 

. STANZA CIL
E prefa addosso poi quella sua cafsa,. Che al suo ver
CL? ¢ tanto grave,ch' ei vi crepa forto y Mirando in rif
St metic 1a wa, @ prefto fene passa.;  Eversad pomiin
Ow ¢ la figlia s ¢ il flebile raddotso Poi st

Mentre che coftoro piangono, Magorto per via-de'
& la Figliuola, snaolcendonche ella è bene aipentt si. %
folue di regalare gli sposi d' una quantita grande di pomi d' co + A008;
to, € Così fece, ed ap arrivo oo in casa degli sposi tutti ceflarono di pia
GV.ALZO. Luogo picno,d'acqua, dove fiiposla guazzare, cioè pall
picde scnza navilio, che noi dal Jatino diciamo, vadoy o gaado; onde ilk
Vaaa così detto perché quel luogo dicevai Yada Volaterrana's ¢ guadare x
¢ paflare: Ma i piglia ancora per ogni grande ammollamento, che si fa
neile cafe, 0 altrave in sul suoloy come € prefo nel presente ye: aque
calo viene da guazza, la quale cade dal Cielo, altrimenti detta 4 at
prvina come gelata disse Dante dal Lat, gelu; ¢ non da guazzare il flume; Sefore
ie non voleflimo pigliarlo per parlare iperbolico, come ¢  adoperare
per paflar-tal molle, che è in quella lanza. 0) oop oe
8ACC-ALARE. Huomo di stima. Vno dei principali del paefe
anche Barbafforo, Baccalare da Baccalaureus si dice colui, che nelle
acquiftaco un grado proffimo al Dottorato, 0 Maeftrato detto altriments I
ziato; il che ula nelle Fraterie, ¢ corrottamente lo dicono Saccelliere
grado si ritrovava anche nell' ordine della Cavalleria.:
E una casa come un mare. Cioè sempre picna di zoba ed abbonda
bene, si come il mare, che ¢ immenfo, detto percid. da Omero atrygi ih
non ha fin y ne fondo, Si dice anche Vna casa come una Dogana, a
GONGOLA. Greco cancharei » Giubbila:-Si rallegra 2-5i
certa allegrezza interna. B' voce usata afiai dalla piebe.: Re
KIS-AKCIRE, Riftorare; Rifare il danno,0 ricompen(argli qd havergli tenn.
ti tanto in pianto. E per altro questo verbo ri/arcire vuol dir raffer
vifto sopra,C, 6. stan. 52,
cOGAO, E' una mifura,immaginaria di vino, che contiene dieci barill
quale corrottamente si dice Cento; Deriva.dal Lat. congims. Onde Bigonet §
da un Lat, bicongins; a Piftoia percid dette pili profimamente all' origine Bite
Gio, Villani lib. 8, rubr, 116. Valle lo fraio del grano in Firenze solds 8. Leagan del
mofto in certe parti meno di (oldi 40, Ma qui è prefo, come & coflume », per une
certa forte di cafla, o più toflo cefta fatta, ¢ contefla di firilce dal
corbelli, ma è di foggia Junga,.ed ha il coperchio, come hanno le, c
S/rafazzona. Si ripulilce; Si rinfronzilce. V.cdi ops. Cy 2 flange
si rifa; si rimette in fazione, in abito; fala Pre tecr yi la bella
nicra. Gli autichi dal Provenaale dissero agenzare, cio' &, '
ce Gente usata dagli antichi Tofeani ancora per Gentile, Br
voi, donna gente M' ha prefo.amor, non è.già maraviglia. L
il fenno, ¢ li gents coraggi.. I Beato lacopong disse che la 2
geaz@, clot non rilciacqua, come spiegd alcuno, ma rafaxzuna, ri
'

 
 
 

SETTIMO CANTARE: 37
forto pet lo fverthio pefo j ed'il verbo crepare, che»
, come vedemmo et: stan. 18.qui¢ nel suo vero
ire, perché quella gran fatica pud cagionare l'allentamento.
0€ si cava di Se.; fila errcor tact propriamente il piles
eflendo il nostro cappello pili tofto il pera/us.
Fras ) STANZA'CIV.
2) Eperché qualfinoglia-donniccinola,
2 Lpauen: Porta la dote, ed il corredo appreffo,
digni cosa,.——*. Acciacch' in quella casa la Figliuola
Sdegho*toritmenté ha spento; ~~ ~Polfa moftrar a baker qualche regre/so,
<1 SS Ne che gli abbin a aner quelcalcioingola
“C' un piccolo ne anche v' habia meffo,
'La vuol dotar conforme al grado loro
pss 'Con quel gran-monte di bei pomi a! ora,
SM Evopny shaves Ps ANN FAs +P;.
llor brillando con Brunette © Edegli poi al fin com ogni afferto
gr axiejefanaratanccoglitza; °°\° Riker? rutti, è volle far partenza,
inato un grande ye bel banchetto” > Ledandofi del furto del Romita,
“ar le noxze in sua prefenra, 'Che si grand' allezrezza ha partorito;
orto si fa. conolcere per il padre della, Sposa, ed aificurando Pigolonc, ¢
i perdonato, ed' haver guito', che segua quel parentando, colti-
vl lla caffa piena di pemi d'oro. Si fanno però di nuovo gli spon.
il banc! ¢¢ Magorto se ne corna al suo paéfe, dando molte lodi a,
per efler'egli stato autore di così gran 'conteato. E qui con la fine de]-
la'raccontata dalle Pate a Paride termina il fettimo cantare,
  AMAN vote, Senza nulla in mano: cioé si mariti(enza dare dote aleuna;
we} «= CORREDO, Quegli arnefi, abiti, ed altre robe j'che fi'danno alle Femmine,
 Oltre alla dote, quando si maritano, che i Giureconfulti digono Parapherna dal
yep Greco Para, che vuol dire oltre, ¢ pherna, che vuol dir dote,
git HAVER regrefio Termine legale, che vuol dire haver azione di domandare,
 Sontro\a tino, per rifarsi del pagato ad un' altro; Vedi (otto C, 8. st. 42, E co-
jj MUunemente significa un certo ardire, ed autoritd sopra ad una persona, o sopra
i i suoi beni ed effetti: 1 rale gli ha prefo regre/so addefvo, per intendere ha prefo
yg) atdire sopra di Jui.

a gli abbin a haver quel calcio ia gola, Non habbiano a poter rinfacciar-

  
  
    
   
   

   
     

 
  
 
 
  
 
 
 
 
    
 
  
 
   
   
   
  
  
 

  

us om; “ oe non v' habbia portato nulla: Noa habbiano a ha.
'caufa di conculcarla, '

| ELANDO. Giubbilando, Vedi sopra C. 2. st. 69.

4 CACCOGLIENZE. Vedi sopra C. 1. st. 34.

le mgxe. Cioé di nuovo si fecero gli sponfali, e folennemente ff
di sposi.

   
  

' 'Gohan

FINE DEL SETTIMO CANTARE:
Bbb 2 OTTA-

 

  
 

tk
 

 

   

    

    

  

&3 ARGOMENTO.
oY
D' un! avventura grande è po

OTTAVO.CAN
a; J 3
SERS]
Dalle sue Fate Paride veftita. ous. a ae
Vede lagalleria di quel? albergo;\.\ a
Seco eer eenin

    
      
    
    
   
   
 
   
  
 

   

Ond' ei pigliavicenza, e voltaiLcergo.,

 

  

      

 

a

5 Vien Piaccianteo condotto al Generale y.... »

ey Che non glivolle far ne ben, nemale,

BAP 25 CEPR CUE MFI CED
N75 h
ENA AAACN |
STANZA STANZA THM. Ta
Orrei, che mi diceffe un di coffora, La notte, disse y¢nn wafo dit
Che giofran tusta notte per le vie: y Che versa affronti, rifichi, € iy.
Che gusto v' ¢,perch' a ridurla a ora Pero, he nel:fiuo sempo shucan fuora kg
Non v'é guadagno, ¢ fon tutte parries Tutes i ribald ydadrt, e rompicali;

Poiche ( lasciando, che enon ¢ decoro ) Onde sia ben'riporfi di buon' boa, !
L? aria cagiona cento malattie, E dene efempio Lhuom pig is al iy
Mille disgrazie possono accadere, Che  unds loro al piis vale. at T
Mille malanni, Diauoli, e Versiere, E pria ch' il. fol sr amoncé se ripome thy
STANZA IL. STANZA LWe be
Sapete, che e' s'inciampa, ¢.che e' si cafen, Edegli, che at un. mondo assai pitt vale t
Si puo in cambio a' un' altro esser'offefo, Sta fuori tutta notse,o dineth, Pa
O dar mun, fet' bai monete intasca, £ gira al bmo come un' ani 4
C alleggerir ti voglia di quel pefo 5 hie
Maca: qual ma si pudcorrer burrasca, t
Però vi ginro, chiio non ho'mai inteso ny
La fin di questi tali, € tengo a mente de
Quel c'wn tratto mi disse un buom valéte, to modo, che non ve da le
STANZA V. » DSR hebeR Cy
Perché le fon tutte cose provate, Come al Garani quand! a gal K
E vere, che non v'? spina ne offo, e-4ndato era la norte n
E non si tronan poi fempre le Fates Che, mentie oi by
Che vengano 4 leuarti il mal da doffo, Da esse ebbe un fanor di &

 
 

 

   
    
  
  
 

2

OTTAVO CANTARE 381

Poeta ifeguitare a narrare quanto avvenne a Paride s' introduce col
che nocumento sia ' andar fuori di notte, ¢ che però sia cosa da,
, dente il'non considerare quanti pericoli si possono correre; Ed
nigliando la notte al Vafo di Pandora conchiude, che si dourebbe imparar
i polli, che vanno a dormir subito, chee' s'é riposto il fole, ¢ così sfuggires
te le disgrazie, perché non si trova fempre chi liberi dal male, come avvenne
a Paride, che dalle Fate fu liberato dal pericolo di morte.
 GIOSTRARE. O armeggiare. Mctaforicamente s' intende andar girando, o
pafleggiando senza faper dove, 0 senza fine determinato, che si dice anche anda-
» Oagironi.
ARIDVRLA « oro. Per ridurla alla. conchiufione. Vedi sopra C, 3.st. 43.
e malanni Diavoli,e Versiere. E' un modo di dire assai usato in simili
aioe per esprimere possono avvenire tutte le forte di disgrazic.
VEKSIERA, orl infernale, che dalle nottre donnicciuole ¢ intesa per una
fla moglie del Diavolo. Forse viene dal Latino Ver/usia, che vuol dir
'malizia; ¢ si dice Versiera un ragazzo maliziofo, faftidiofo, ¢ infolente, ma
Spill veri » che venga dal Latino aduer/arins,col quale nome ¢ difegnato il
7 nella scrittura., ddnerfarins nofter diabolus, Petearca.
5 1; Si che anendo le reti indarno tefe,
U1 mio duro avversario se ne scorni,
Da aduerfarius nelio stesso modo, che 1 Francefi fecero aduerfaire, così i nostri an-
: iz i's Auuerfiere ? anuerfiere, ¢ poi finaimente /a Versiera. Ll Beato lacopone da

 
     
 
    
 
 
 

 
 

i

 

   

if

Hl canto 62.

ih “A Lo nemico ingannatore

oe o8aiiin::; * Anerfier de la Signore.

xs) Eecant2r. | Fata gli anerfere venire,

8 ' Chel degian accompaguare,

an Nell? uso dicefi Far la Versiera, fare il Diauolo, e peggio,

ib  INCTAMP.ARE, B il latino ofendere. Vedi sopra C. 1. st. 13.

i OT ASC.4. Quella facchetta, che si porta comunemeate appiccara agli abiti per
nfo diyportar roba necellaria alla giornata, come denari, ¢ simili da' Latiai detta

it Pera, o Zona. 8

/ @ALLEGGERIRE di quel pefo. Cio portar via i denari, ¢ cos} alleggerirlo del

', pelo, edelia noia, che per quello gli veniva.::

a MANC A in che mo, Cioè sono infiniu i modi. Il termine mance in questo ca-

oe a0 sufato ironicawente, perché s' intende: Vou mancano s modi.

a  CORRER burrasca, E termine Marinare(co, che significa Correr pericolo, ed
in questo Ggnificato ¢ prefo comunemente, se bene berrasca vuol propriamentes

dire follevamento di mare per il cattivo temporale di venti, ec.

VASO di Pandora, E) nota la fayoladi Pandora, 1a quale fauna Femmina.,

( che Giove fece fabbricare da Vulcano, ¢ darle.in dono di ciascuno degli Dei je

 

' parti, affine di farne innamorar Prometco,.¢d indurlo ad aprire un va-
7 fo pieno di tutti i mali, che Giove haveva dato alla medesima, che lo donaffea
jf Promotco » che vuol dire Prevvidente; che antivede, per vendicarf dell' ingiu-
, tia da eifo fattogli quando rubo u fuoco celefte, ma nonl'havendo Prometeo
oe voluto

 

rie =
*

   
  
   
   
    
   
   
     
     
  
 
    

   

382 MALMANTILE ©

voluto accettare, lo prefe Epimeteo suo fratello, 4
fatto, il quale |' aperfe., ¢ vennero fuori tuttii mali, che
questo ¢ il vafo, che il Poeta intende nel presente luogo, ¢'
ni nel fecondo capitolo della pefte dicendo:
To leffi gid & un vafo di Pandora,
Che n' era drento il canchero, ¢ la febbre,
E mille morti, che n' usciron fuora

 
 

Orazio lib, 1, Ode 3.
Post ignem atheria domo
Subduftum, macies,& nova febrium
Terris incubuie cobors,
La favola, e raccontata da Esiodo. è
RISICO., Riftio, 0 rifico dal verbo arrificarf?, arrischiarsi,o d
vuol dire Esporfi al cimento, 0 avventurarsi a qualche icolo. In'
Risco significa, rups pricipizio, luogo pericolofo. Cie, se bene mi f
quam in diffictle, & scopulofo loco verser, rificofo. “he a;
TRACOLLI, Da tracollare: altrimenti barcollare, che & fm
il Latino metare, o ritubare; ¢ qui vuol dir Disgrazia,, o pericolo, by
ROMPICOLLI. Huomini; che consigliano, o inducono altri a far 2 |

Latino in omnem audaciam proieéti. A a Cu
T#STONE. Moneta Fiorentina, che vale tre giuli,o paoli, 9 | 4
VAL pik a! un mondo., Questa iperbole significa non vi ¢ prezzo, Ta

Star discoffo un mondo, disse il Bronzino nelle rime burlelche; cio Me

Spario. u

CERCAR di Frignuccio, Cercar le disgrazie. Andar incontro a' pores
Frignuccio dalle nostre donnicciuole è prefo per il Diavolo, e diciamo ane ny
cercar il male come i eMedicr, I Latini in questo proposito dissero; Camarinam me \ 4s
xere da una pianta,.4a quale ha le foglie così fetenti, che movendole, @tocta | Mi

 
  
  
 

dole lasciano un puzzo terribile: o forse da una palude detta Camarina! do
cina al castello detto Camarina in Sicilia, la qual palude, perché cagi \
detto Castello la pefte, i pacfani domandarono ad Apollo, se era bene Q
re detta palude, ¢ l'Oracolo rispole: Camarinam non esse mouendam 5 I
fatto poco conto di detta risposta,vollero feccarla, ¢ n' hebbero il gaft q

i nimici paflando per quella palude già fecca, entrarono nel Castello, €
“TW bella proua. A pola; ¢ I addicttivo beds s*ula in questi ca pes eal
IN bella proua. a; ¢ I addiettivo s'ula in i cal '
r seaen un Gpuinivonmal dica in prouffima. Vedi sopra se
nell' uso: L' ho bell' ¢ fatea questa, o quella cosa; cioé l'ho fatta fa
terminata, fornita. x
CHI cerca trona, Detto fentenziolo, che significa, che colui, che
al male, merita che gli fucceda:
NON 0' ¢ spina, ne offo.. E' negozio spianato. E cosa liscia, Non'
bitare, non ci ¢ da incontrare difficulta alcuna i
AGAMBE alzate, Cioè col capo all' ingit'. Si dice anche ve a
are, Vs0 quelta frafe Agambe alzare. Ser Brunetto Latini maeftro di D

  
   

tase n= ~sxse.

 

 
 
 

  
 

-OTTAVO CANTARE, 3 83

ovvero Capitoli pieni di gerghi, ¢ di vocaboli Fiorentini; ¢ vole spie-
'atto di chi iemwomnla in terra per iscaticare il ventre. Zvidi a eae als

ecanteee: con riverenza, cacava ) che questo vuol dire torrire in

ested col, ut

“EGGLAV A con la morte. Faceva conto di morire. Temeva di morires

nel mulino.
STANZA VI. STANZA IX.
efto vuol pur ch' io di Ini difeorra, Circa questo,pensiero elle non hanno,
Onde di nuono a i fatti suoi ritorno. Ne di fare altre spefe, come accade
Le Ninfe, eb? il vedean barter la borra Ad ogni galant' huomo.a capo a anno
Tutte gli fon co' panni caldi attorno, D acconci,taffe, laftrichi di trade:
B gid tra loro par che si difeorra UL vito,e il freddo non puo far lor dino,
Di fargli dare una scaldata in forno, Perch il tetto, che feorre,e mai non cade

 
  

 

 

  
 
 
 
 
  
    
   
 

   

 
  

ta perché questo in danno suo rifulta LZ? Inverno fui pilaftri di coralto
+ volleil/uo parere anch'ei inCfulta, Si ferma,e forma un palco di criftallo.
:S8TANZA VIL. STANZA X.
ino di non farn? altro; ond' esse Di Stare il Sole giu ne' suci quartieri
riveftire a [pefe loro; Non puo col frugnolone haver l'ingre/io,
7 icia nuova una gli meffe, Tal ch' elle flanno bene, e volentieri,
« C'ba dal colloe da man trina ¢ lavoro, E gedono un pacifico pufse/so,

 

Pr altra il ginbbone yn'altra le bracheffe Paride intanto infra tazze,e bicchieri,
"un ricco,¢ nobil quoio a' oro, E di piu forte vini, ¢ fructe apprefso,

  
 
      
  
  
 

« Fit altra gli ravvia la capeliera, Con è/se ritrovandofi in cantina, —
J&€ ette il benduccto,e la montiera. Valle provarne almeno una trentina,
oe STANZA VILL STANZA XI,
Alpalfe poi lo menan per la mano Ve per questo alterato egli ne reftay

4 ta lor bella abitaxione, Ovengach'egli¢ avverzoin dlemagna,

Ma poi pits buona,benché sia in patano, Oc' a faluar quel vin faccia la testa,

ea pagar non hanno la pigione, Ed in quel cambio dia nelle calcagna;

¢,un negoriv odio/o, e frrano Ragio, che quadra bene,e quedaye questa,
: quell' infolente del padrone Perch' ei non urta mai chi l'accépagna,

oad na a casa,e co si poca graria Ma sipre in tuono,e dritto com un fufo

Chiedeilfemeftrech'ei nov'é una crazia, Con efse per le scale torna fufo,

-olox gh pic's STANZA XIL

nv! Ow egli entrato in una bella fala, Di li poi falgon sopr' 4 un' altra sala
6 Ch ella sia l'Accademia si figura, Di bafton congegnati infra due mura,

' he vi fon aratolo, ¢ la pala Donde, arpicando come fan le gatre

ti d Strumenti da Siudiar ? agricoltara, Vanno a pafsar per certe cateratte.,

, DiParide dunque vuol seguitare a discorrer il Rocta, ¢ dice, che conoscendo
i ke Ninfe,, che eglifentiva un gran freddo 5 volevano metterlo a rasciugare, es

 Hilcaldarsi in un forno.,ma.egli non volle, onde efle gli fecero un veftito nuovo

 
 
 
      
 
      
 
   
 
  
 
 
 

   

' (pefe nella maniera,.che viene esprefio, in guetta Stanza fettima; Di poi
Jomenarono a vedere la loro abitazione, ed in cantina dove bewve afiai;e.nony
-danno per le ragioni; che adduce il Poeta; ¢ di cantina falirono alles

   
 

0

   

2 er

 

  
    

 

 
  
   
  
  
      
        
  
    
 
    
  
   
    
    
   
   
  
 
     
      
     
    

22,

384 MALMANTILE?

BATTER [a borra, Iorendiamo Tremare, ¢ battere i dei
do: E si dice così per la similitudine, che ha tal b
che G fa della borra:1a quale ¢ speci¢ di lana triturata col coltello ¢ |
empiere i bafti delle bettie da foma, ec. ¢ per liberar devta borra dalla)
si mette sopra a un' afle forata con piccoli:, € speffi fori, © filbatte
di corde adattate a questo effetto; ¢ questo battere fa uno firep: ct
che similiiudine col batter de i denti, che faccia uno tremante per ¢:
do, ec, Si dice anche batter la Diana; tremare tutto, flando allt aria, a€
scoperto; Latino /ib dio... Vedi foto C. 9. tt..6. os 8 Coe Raa

BRACHESSE, Brache, caizoni, Voce Veneziana taluolea wfata anc

  

On ' '
QVOF d' oro, Pelli di beftic conciate, e dorate', servono per
ze in vece di drappi. ' at

GLI ranua la capellicra, Gli pettina la zazzera yO chioma, 3

benaa. Striscia di panno lino bianca, che s' appicca pendente alla
cintola de i bambini, perché si posiano con essa nettare ij malo, -

MONTIERA, Specie di berretta usata dai bambini. Dallo §;
tera, berrettino,

PANT ANO. Palude', che diciamo anche padule, luogo pieno:d?
ma, che renda il terreno inzuppato, riducendolo come fango, da i
detto Palus, paludis,,

PIGIONE, Cioé quel denaro, che si paga per fitto d'una cosa; E
con termini proprj tro si dice quel danaro, che'fi paga per poderi, et
pigione si dice quel denaro, che si paga per Cafe, 0 botteghe, dicendo
borteghe,o cafamenti: Ed appigionare cafe,e botteghe. Di quelte si dice
ma dei terreni mai si direbbe appigionare. Pigione dal Latino
forse da fexdum, fio, e questo dal Latino fides, ¥

STRANO., Stravagante. Qui intende noiofo, odiofo, faftidiolo..
frrano dal Latino extraneus ritiene anche appretio di noi il significato di
ro, 0 lontano dal parentado nostro. Vi/o ffrano, vuol dir vilo arcigno
© crucciofo; vifo rane vuol diranche faccia macilente, ¢ pallida,”

SEMEST RE, Numero di fei mefi; ma intendi il denaro, che si
pigione di fei mefi. Le

T ASSE, ¢ laftrichi di firade. Spe(e, che occorrono farsi alla giornata
ro-, che posleggono cafe in Firenze; che /africhi, intende quella spela
partice fra i padroni delle cafe per raflettamento, ¢ Jaftricamento
della Città.

TETTO, che fempre feorre,¢ mai non cade., Abitano forto acqua 51
il loro retto', che fempre feorre, e mai non cade," 1 abt
PILAST Ri di Coralie, Pilattri si dicono quelle colonne fatte di
altri safi, per foftener volte, Latino pile. B percht'il corallo nafo
finge, che questo tetto f regga loprai pilaftri di coralloje vaol di
werno s' agghiaccia |' acqua, ¢ fiferma. 1 boy poner
NON pro col frugnolone baver ? ingreffo. Non pud il Sole trama
netrare i suoi raggi fotco l'acqua, Fragnojone da Frugnuolo detto

  
 
 

  

 

2.2. -Seeeepeee seers eee =

 =

BRipnmaes..

 
 
 
  
 
  
 

OTTAVO CANTARE. 385

ae 'RATO. Commoffo, o perturbato da qualfifia accidente. Ed alterato
dal vino vuol dir Briaco. Onde gli Alterati Accademici già famofi in Firenze
Bee ates

 

 

o per Impre(a un Tino; in cui Gi pigiava l'vua, ¢ ogni Accademico ula-
per imprefa particolare cose attenenti a vino; si come quella della Crusca,
le fuccedé, ula per imprefa tutte cose attenenti a grano.
ACCIA a faluar la tea. Non offenda cot suoi fumi la testa, perché ¢ vino
- Detto (cherzofo tratto da quelli, che giuocando di scherma non fanno
gioco, ma pattuiscono di faluare la testa, cioé non si colpire nella te(ta.
GION che quadra benese quellaye questa. Tanto pud efler per questa ragione,
per quella, che egli non Sa rimatto alterato dal tanto bere.

NON urta chi 0 accompagna, ma è fempre in tnono. Non barcolla come fanno i
riachi, ¢ non da spinte a chi è seco, ma sta in ceruello, ¢ va dritto.

| ARATOLO.. Si dice anche aratro dal Latino. EB erato si trova nell' antico
“Volgarizzamento di Palladio:; donde ¢ fatco il diminutivo drarolo. Strumento

quale i villani'rompono la terra, facendolo tirar da i buoi.

'PIC ANDO, Bi il verbo arrampicare fiacopato, ¢ vuol dire il falire, che

oi gatti sopr' a ua' albero, 0 simili, ¢ viene da rampicone, che & un ferro
inde Die, che usano i marinari per pigliare, ¢ fermar le navi. Latino
ro  harpagonis; da che noi pure lo diciamo anche arpazone, ¢ arpagonare,
CAT ARATTE. Et voce latuna, che vien dalla Greca catarrbattes, con las
intendiamo ancora quelle buche fatte ne i palchi,per le quali si pafia di for-
entrarein luoghi superiori con scala a pioli, come farebbe falire per di
faltetto + E per lo piij tali cateratte s' usano per entrar nelle colambaic;
a forta era la cateratta, che dice in questo lyogo.

 
  
  

  
  

  

 
 
   
   
 

TANZA XIII, STANZA XV.
4 qui la Mula vnol ch' io mi dischiari Horsi per ch' io non caschi nella pena
Circa il deferiner queste loro Stanze, De cingue solds; ecco ritorna a bomba
Che stio vi pongo addobbi un poordinari, A Brache d'or, che nel (alire arrena
Non per dir bugie, ne ffranaganze; Per quella seala, che va fu per tromba,
© Péréhiile Ninfe han folo i necefsari, Perché se bene ei fail Adagia da Siena
wv ie moderne usAnre, Gli è difadatto, e pefa chregls spiomba,

i
Per infeonare'a noi c habbian le borie E con le Ninfe a correr non puo porfi,

 
 
 

4 adri',¢ letti d'ore,e tante lorie, Maffime liche v' ¢ un falir da Orfi,
atl STANZA XIV. STANZA XVI
i; Ch ognun vuol far il Principealdid'eggi, Elle di gid, com' to diceua ade/so

   
 

rl | Seben chi la vole/se rivedere y Vicite fon di sopra a stanze nuone,
Melti firvegvon far grandexze,e sfoeei, eApettando, che facia anch'ei Liste/so,
cae,

 
 
   
 
  

(pecchto poi col rigattiere: C' appunto com il cambero si muone;
a sa lnfsobgrande, egiaregnainsns poggi, Onde connien poi loro andar per efso,
 Efon nelle capanne le portiere + Ed aiutarlo fin, che piacque a Gioue,

4 | Bera icannelt infin qualfiuoglia unto Che quafi manganato,e per strettuio

Had fuck fhiperti, ¢ seggiole di punto. Pafsafse ad alto il Caualier di quoio,
Pr I Autore di voler dire la yerita, prega il Lettore a non pigliare
wzione, se in descrivere le maficrizie delle Ninfe metterad addobbi, ed ar-
Refi un poco ordinarj, perché in eftctto ead così; ¢ da questo pigiia occasione di
tye Lee biafi-

 
 

SS

 

 
 

  
  
    

   

    
    
    
 
  

 
 
   
 
 
 
 
  
  
 
 
  

 
 

  
 
 
  
 
   
   
   
  
  
    
    
   
    
  
   
    
 
    
   
   
 
  
  

386 MALMANTILE

biaiimare il Info, che è oggi in Firenze. Di poi tornando
che le Ninfe falirono alle stanze di sopra, doye con gran fa
de, il quale chiama il Cavalier di quoio, perché era v
demo. shee
ADDOBS!, Mafferizie, ed arnefi per uso 5 ed ornamento
verbo addebbare, che vuol dire Adornare. Du Frefne nel Gloffari
dig Latinitatis. addobbare, armis inffruere, militare cingulum alic
confetka ex adoptare, quod qui aliquem armis instruit, ac militem ne
modo adopter in filixm, Si che Addobbare fecondo questo autore vi
folennita del vettire i Cavalieri,
ZORIA, Aibagia. Vanagioria. » oe
SFUGGI. Vlanze fontuote canto di veftire, quanto d' addobbamenti
fatti con (picndideza, € pil del consueto; Donde si dice fare sfoggio, 0
quando i trutti fanyo quantita grandissima di frutte, 0 quando chi
piu del folico; ed in somma s' intende d' ogai operazione, che esca del
© cel naturale; come si dice frutta sfoggrata quella, che eccede img
beilezza, ¢ supera l'altre fructe della sua specie. EB la forza della
venendo da seggia, cioé ulanza, el solito, antepostavi I', s, vuol dir.
foggia, cice tuor del solito, e del consueto. Gio, Villani quel che noi
foggi, chiamna difordinati ornamenti lib. 9. ¢, 245.5 ¢ lib. 10. Cap.t
mo autore lib, 12, cap. 4. £ mon ¢ da Lasciare as fare memoria d' una,
mintazion d' abito, che ci recaro di nuovo i France(chi. EB poco foro,
natura fiamo dijposti moi vani cittadint alle mutazioni de' nuomi abiti
trafare. Sfoggio dunque vale fuori di foggia., cloé dellafuzione, 0 VO
y.cniera' di fare ordinaria, ¢ usitata; che il Villani comes'é villo
sformata mutazione a' abito; ¢ disordinati, ¢ sconuenenoli, ¢ difonefi, ef
mienti ye nuoni, ¢ iffrani abiti,:
CHI (a volefse rinedere. Cio' chi la volefle bene efaminare, 0 rice!
maniera quefii cali poslano fare Gimili sfoggi + 'i
SONO a specchio. Hanno debito. Traslato da coloro, che hanno d
decime, che si pagano al Principe y i quali &i dice esser'.a specchio, p
notati a un libro, che si chiama lo specchio.; Qui dicendo.; sono
ghattiere, da:due colpi, uno che coltoro, che fanno tante borie non
gate, ¢J''altroy che questi loro sfoggi sono di robe usate, € vedute
poiché l'ha prefé'dal rigattiere, che vuol dire Vao, che vende mafleri
ed abiti usati. Vedi sopra C. 3. st 5.
POKTIERA, Paramento di drappo, o d'altro, che serve per
porte delle Ranze nelle cafe Civili. Da alcuni detta in Latino velum ada
TRA icanneili. Vuoldire fra la gente pili vile; perché fra i cannelli.
mo fra i tefitori di lana, che fon gente d' infima plebe, ed.¢ lo stesso
qualfineglia unto; perché questi tal: maneggiando fempre lane unte
fempre unti; ¢ qui aggiungendo al detto fra i canned, il si
intende, che fino i Batulani, che fra gli unti sono i più vili »fanno le!
SEGGIOLE di punto, Cioé seggiole ricamate, o trapuntate di
mo: Panto Vaghere,0 punto Franzce, 3

 

   

BABeRew eft gFen2n2 se 8s =

=

a=
Se =z

SURFER
 

 

 

*' OTTAVO CANTIARE: 387
CASCAR nella pena de' cingue soldi. Quand' altri nel discorso fa una digrettio-
ne, €non torna mai a) primo proposito, gli diciamo: Voi cascherere nella pert.
de' cingxe soldi, 1 Varchi nel suo Hercolano pariando di questa pena dice: E chi
cominciate alcun ragionamento,e pot eutrato in un' altro, non si ricordaa prit di

nave 4 bomba 2 fornire il primo, pagava gid, fecondo teftimonio dal Burchiello, ni.

te

'a

4
wail
a

 

offo, #1 g 'o non valeua per aunentura in quel rempo più di quei cingue soldi, che
a ced Nae quali Lacie vegghiamo, che ST rarchs si serve del detto
Tornare « Bomba per tornare a segno, 0 al proposito del primo discorsa, come fa
il nostro Autore nel presente luogo. L' Ariofto Satira prima dice;

= Ma perché i cingue soldi da pagarte,

t Tx che leggi, non ho, ritornar vuglio

7 La mia favola, donde ella si parte.

 eARREN A, Intoppa; Si ferma; Non seguita il viaggio. Traslato dalle na-

Viquando si fermano, perché tuccano il lette dell' acqua, che si dice arrenare, 0

incagliare, De 1 gual) verbi ci serviamo per e(primere non tanto il fermarsi in un

'Wiaggio, quanto il fermarsi in un dilcorso, o nel proseguimento di qualfivoglia.s
'aaione, negozio. Latino hnerere.
 PAil mangia da Siena, Fa il bravo. Fa il valorofo. Il Mangia da Siena è
\ di metailo atiai grande, la quale è posta sopra la Torre dell' orivolo
del Comune di quella Città, la qual figura dicono, che sia il fimulacro d@' uno an-
tico huomo bravo detto ii Mangia; Ma io fon d' opinione, che ella sia il fimu-

lacro di qualche antico Podefta di Siena, e che habbia acquiftato il nome di 423-

54 da qualche inferizione, che havefle appreflo, la qual diceffe Il eA/agna di Sio-
WAS COR i) ALagnifico di Siena, che s' incendeva già il Podefta: Ma sia com ef-

fer fivoglia, a noi basta fapere, che questo detto serve per in tender con derifio-
ne un bravo, o valente; quafi voglia mangiare le persone, e ingoiarle,

DISADATTO, Contrario d' Atto, deftro, agile, ec, Vno che duri gran,
fatica a maneggiarsi,o muoversi per la gravezza, o per altro accidente, Sciar-
feancora ¢ contrario di arto, ¢ significa uno, che fa male, o negligentemente
quel 'ch' e' fa; poco pulito nelle sue faccende, ¢ nella persona,

CON Ie Noha correr non puo porfi, Non pud gareggiare con le Ninfe a chi pil
corre. Interide, che le Ninte al ficuro lo superercbbeno nel corso;

VP Bun falir da Orfi. V' & cattivo, o difhcil falire. L' Orfo è un' animale, che
f ben, ir goffo, e difadatto, nondimena ¢ afiai deftro, ¢ facilmente fale anche
in ionghi inaccefibili; donde noi habbiamo: Efer come L' Orfa, cioè Sofuse deftro,

Ui Berni nel Cap, al Fracaftoro dice:
Shy. Conniene ivi lasciar t" xfato corsa,
«ta £ falir fs per una certa [cala,
i Dove hauria rotto il colle ogni deftr' Orfo,
'Ostiero nell Iliade al nono chiama una rupe 50 balza Aigitips, cio8 dalle capre abe
“Vandonata; © queito medesimo nome di Leeips danno gli antichi a una Città dell'
Afola di Cefaionia, € ua' aitra deil' Epico. Noi diciamo di Jaoghi simili erti ri-
'Pidi, © feosceli: Won vi falirebbero le capre, le ee Virgilio nell' Egloghe dide:
repe. Quella montagna altidima nell' India; fu'la quale fu il primo Ale(-
fandro Magno a falire, fu detta da' Greci eornos, cioè senza uccelii, quafi mon.
oksal Cec 2 tagna

 

 

ca i
 

   
     
   
 
  
 
 
   
  
 
 
 
 
 
 
   
   
 
     
     
    
  
  
      
    
      
 
 
     
 
    
 
  

 

388 MALMANTILE

tagna da non poterfi ne anche da chi aveffe l'ale formontare
S7 muove come il gambero. Cio' va all' indictro, Wepam
MANG.ANATO, lnfranto; Mangano ( dal Greco mage:
na, con la quale si distendono 5 ¢ si-da il juftco.a i panni, ¢
fare a forza di rulli sotto un gravissimo pefo, ¢ tal panno 5 0
si dice poi manganato. E Mangano come s'accennd sopra Cw.
na militare della quale i nostri antichi si (eruivano per (cagliar'p
fiediate,'¢ con essa scagli anche | ini, che dicevano poi
ganati, cio' sflagellati, ¢ pelti dalla percofla; ¢ così si potrebbe inten
ride; ma perché foggiunge paffaro per frretrow, che è un' altra machina, ¢

 

   

 

 
  

 

 

ue per stringer ulive, ec., © per mettere in piega a panni, si vede, che
quel mangano da panni.
Ss ZA XVIL
WN un Dormentorio grande, ma diverso,
Ove ciascuna in proprio ha la fuaceltay
Che sia com' io dir per questo verso,
( Se non erra Turpin, che ne favella )
Vana fanga a mez aria tuna travorfo, ¥
Dow' alla tien se calze,e la gonnella, w
i penzol delle forbe, ¢ del trebbiano, ag
E quel che più le par di mano in mano; fa
TANZA XVIIL. ae
Pitt git da banda un tavolin si vede, er
Che sa i tre/polifa la mnna nanna, Sxpenfa, che vi ina
E sia spatiiera al muro, ove si vede Ada trova im ozso tutti gli 1
Via fiuvia di giunchi,e fottil'canna, E i piatei ripulisi come sp fai
Evvi una madia zoppa da un piede, Teglie, e padelle, inutile a
E il filatoio con la sua ciscranna, Star'appiccare al muro per gli si
Won v'é letti, se non un per micliaio, Ed anche fon per sparut pilt a et
'Che tutte quante dormono al pagliaio, Perché il gattoa dormir vedein, | dy
: STANZA XX14, sth ng
Ond? egli offefo molto se me tiene, ») (Gliaccanan ch'ei vedra fel ta
Ch' una mantita per la golatocca; Ed ei ghignando allor pits noms (i
Ma quelle, che s' avveggon molto bene, E con esse ne va di compng §
Chregli bal'arme diSienarpreffain bocca Per ultimo a veder la Gi 7 hk
De(crive nelle presenti Octave il dormentorio delle INinfe,e ledorowmafieria® | Yu
Arrivano ee cucina, dove Paride a (aol e de pr ta
arata 'cosa alcuna 'per mangiare; Ma ie Ninfe Jo quictano'con dirgli, ch e
ey ada eiare} 0d de lo'condi a veder la Galleria, Pe
'DiVERSO, Differente., 0 'diffimile aghi-altri Dormentorj, perché: «
Celle non 'fon fatte di muraglia, ma fon tutte in-una grande flanza-y §
vile con stanghe app al:palco ciondolot foa r
quali poneng jo ciascuna'le sue 'robe, e panni le 'fa servire per muro
.così vengono 'fortnate'le Celle:. 'Si pud anche dire, yche la voce
dae figaificati il primo,'che vuol dire diferente ( ¢ gquefto f a
 

 

   
  
   
 

OTTAVO CANTARE 389

'meffo per contrapposto, come la-tal cosa ¢ diversa dalla tale ) il (condo quando
po 'ate che vuol dire strano, o stravagante, il Poeta lo piglia ias
Recornto significato. — lo piglid Dante Inf. C. 7,
 Entrammo g pen via diver[a, Oc,
Cavaleamti nelle sue storie lib, 12 parlando di Cammillo quando ifefe il
lio dice:, Non guardo all' ingiufto cacciamento, ma con grandiffi-
i yy mo efercito corse alla dife(a della patria, ¢ liberolla da così diversa fortuna.
7 es, Ricordano Malesp. Stor, Fior. cap. 80. dice: E cid fu per l'inuidia della Si-
> ae i » che non ¢ra.al loro volere, ¢ fu diversa, ed aspra guerra. Vedi. fo-
int 2, stan. 3.
bis Beene del trebbiano. Che cosa intendiamo per penzolo vedemmo sopra C.
6. stan. 50. ¢ Trebbiano € specie d' uva bianca, ma.qui ¢ prefo in generale eee
(IL ogni forta.d'uva,\che's' appicca nelle stanze per ferbare all' Inverno.
 DI mano in mano. Di tempo in tempo. Lat. Deinceps, che s' intende fuccelfiue
neat:  Cic, 7, Ep. Fam, disse De manu in manum. Dan. Par. 6, dice:
jas E [otto t' ombradelle Sacre penne
we \Governo il mondo li di mano.in mano.
yin
ye
si
yh
ue
ae
lf

  

. Bd  detto figuratamente-dal far paflaggio una cosa dalla mano d' uno nella.
'mano dell' alero.. Dal ginoco. detto Lampade dromia, nel quale colu aveva il
-vanto che va una fiaccola accefa correndo., ¢ così bella, ¢ accefa la confe-
gnava.a chi aveva.a correre dopo di lui; disse Lucr. lib, 2. Augescunt alva LEnbes 9
lia minuuntur, Inque breni [patio musantur fecla animantum, Et quafi curfores vite
Po eamema 9 Gi0e fuccede t'.uno vomo all' altro, l.nne vinente all' altro di mano

0 roc, Dal Lat. tripus, odes, E un pezzo di legno, 0 ceppo., in cui
a fon fite tre mazze, fope' alle quali posando., serve;per fohence tavole, ¢ deschi,
oe da i Latinidetto Trapecophorus.s'quali menfam ferens.

' | PAlaninnananna. Non Ma forte in terra, ma dimena o:per |' inegua lita de-
ri N 'te tre mazzc, 0 del fuolo,,.o per altro mancamento; ¢ diciamo far /a ninna manna
Od eda,quel dimenare;che si fa-della,culla.de ibambini, 'quando dallebalie si procu-

ache dormano, che si dice ZVmnare, spetche per lo più fogliono accompagnare

4 -talmoto.con una lor cantilena, che dice Ninna nanna il mivbambino. Vedi sopra
iS 'Cae Renaes. Questo dimenare fidice anche:ewlare pur dalla Quila de' bambini..
, SPALLIER A, Quella'parte della seggiola, alla quale:s' appoggiano le (palle
mA Aedendos |B per /paliiere intendiamo quelle nuragli¢ 5 alle-qualt sono.appoggiate

spianted'agrami, ec. come's'¢ detto sopra\C. 6, stan.'51. Questo artitizio-di
Farele:mura:coile piante-dicefi-da alcuni in Lat.-opus topiarium.. Equi nee
'quel-'muro parato di stuoie tatte di giunchi, o-canne paluftri, che fourasta.alla.
oat »sopr' alla: quale dice;che fedevano le Ninfe, ¢serve,per spallicra alla.me~

 

J

3 STVOLA, B il Latino Storeache conlerua appreffo noi il suo:significato.,

il HUADIA, Dal Latino maétra,i| qual pure ¢ Geeco;.ed una cafla sadatrata
it sopra-quattro:piedi, dentro alla quaic si lavora la patta per far-il pane; La dices
3 Zoppa.da'un'piede perché le: mancava., 0 crarottouno diguefli piedi.. Zoppa si-
'il siete den tee cain tavon della vecchierella Bayside la;presso rida

 
 

390 “ MALMANTILE

lib. 8. delle Trasformaziuni; ma ella la fece stare pari con me!
to; menfam fuccintta, tremenfque Ponit anus; menfa fed erat pes ters
sia parem fecit, i, 2/1 ihe
FILATO/O. Strumento col quale per via d' una gran ruota si fila Jan:
napa, ec, e si fanno le'funi. 1) OL HRS
CISCRANNA. Specie di seggiola come accennammo sopra C,
DORMONO «i pagiiaio. Cio' dormono in fu la paglia.
HVOMO alia buona, Huomo schietto, fincero,e senza malizia; Huo

za cirimonie, ¢ nimico del luffo, e delle boric fine fuco, © fallacijs, | Ve

sornm, ed Hxomo posirixo intendiamo uno, che non fa sfoggi nel veltire, ¢
ogni cosa si tratca senza lufflo. SOS
SENTITOSI allegare i dents. Vuol dite sentitofi (timolare dalla golae dal
desiderio di mangiare; se bene allegare i dents vuol dire quando i deat pert
matfticata qualcola acida, 0 agra. coine 'il limone:, ec. s*iavormentileono, ¢ i
fente una certa difsculta nel mafticare.Ma usandofi come nel pretente iuogo,vu0
dir venir yoglia di mangiare.:
TEGLIA. Specie di tegame fatto di rame stagaato per di dentro, serve pe
quocerui torte, ¢ migliacci, ec. (| Monofini lo fa venire dal Greco Telia y a
gual voce tra l'altre cose signitica 1' a/se da pane, e"| turacciolo,o coperchio del fum
maiuolo, 0 vogliam dire di quel canale, che gli antichi, in vece di cammino,ave
vano per servizio di cucina, buono folo a ricevere,¢ porcar via it A
dicendolo molt Tegehia, ¢ gli antichi in particolare, mi muovo a
venga pil cofto dal verbo Latino Tegere. Queste teglie hanno nell'
ta una campanelia di ferro per comodita d' appiccarla, ¢ le padelle hanno un
anclio in cima al magico per il medesimno essetco'; € questi fond gli orecchi de'qua-
li parla il Poeta dicendo: Stanno appiccate al muro per els orecchi., Ovidio lid. &
Metam, erat aluens illic Taginexs, dura clauo fuspenfus z anfa, hia
TORN/ZE, Parlando di gatei s' intende quel ronfare che fanno; perché ¢ &
mile a quel romore, che fa il tornio quando gira. + Aba
TOCC A una mentita per la cola, Dar una mentita per la gola a uno ¢ quando
se gli dice, che egli afferma il failo, ed ¢ grandissima ingiuria, e che muove al
¢ pero il Poeta scherzando dice, che\Paride si adira per I offela, che ri
quella mentita per la gola, cio¢ di quel tupposto che vi fufle roba per la golasy
che fu falfo, WS EE
Li arme di Siena imprefsa in bocca, L' arme di Siena è una Lupa, ed il mal dé
la lupa ¢ inteso comunemente per una infermita, che fa stare il pazziente in col
tinova fame; onde quando vogliamo intendere; il tale ha gran fame diciamd:
Egil ha il mat della iupa, ¢ pis copertamente Egli ha ? arme adi Senay es'
la lupa, cioé la fame. Vedi sopra C. 3. stan. 22. Kh
VEDKA', # il corpo teene, Cioé mangiera, ¢ bera. Detto assai afato”
gente di vil condizione. » 1. 3a
GHIGNANDO., Ridendo leggiermente. Lat. fubridere,
GALLERIA, Cosvin voce straniera chiamiamo aicune Manze piene
nate di-galanterie, ¢di-cose singolari, ¢ maravigtioe 3 quali ttaazes
fon dee Pmacorheca dal Greco Pmax, che suona tabula pita, © theca

oe

 

 

  

  

 
   

    
  
   

oe. >Eeerec-se.= ez ER PTE

 
  
   

  

orre al
STANZA XXIL

fb Principi ritvatti ye di Patrizzi,
 Originali farti gid in Fiarenga

4) Da quel, che gis vendea sotto gl' usizri,

» Ed euns dello feeffo una Sibilla

 Eduna bella Cittadina in villa.

Bt STANZA XXIIL

me

SE aa

tapelPa fole, e sgabelli
intorna inalzan sopra al piano
 Statue eccellents di quet Prafiteli
BH Già shalt danno il moto in. Settignano,
Ce * Buonarrnoti, e i Donated

Caer

— Aquel baffa ritseva di lor mano

z

 

ORTAVO CANTARE. jor
erste: Sper altro Galletia voce militare ¢ specie di fortificazione.
xX s

TANZA XkXIV.

Si che que? opre, che non hanno pari,
Quantoi (uddetti quadri,c' han del vago
Non si polfon pagar mai con danari,
Perché fon giore 5 che non hanno pago;
Vao feaffale v' ¢ di libri vari,

Ch' eran La libreria di Simon Mago,

C' abbellita di feorie 5 ¢ di romanzi

Fu pot venduta lor dal Pocauanzi,
STANZA XXV.

Exni un tomo fra gli altri scrittoa penne,
C' ame par bello, ¢ piace fine fine,
One si legge in carta di cotenna
Tradotte le librettine in feftine,

E che Gateno, ¢ il medico eduicenna
In musica mettean le medicine;

ww &

Leo, s' sl corpo fempre a chi le piglis

GC as paari scalzi pur si vede ancora
' Gorgheggia,¢ canta,noné meraxiglist,

| Sw t arco della porta per di fuora,
+3 9A da principio'a descrivere la Galleria deile Pate, ¢ narra la bellezza
4 aicune pitture, ¢ statue non diffimili dal refto delle maflerizie, per efler' opra
ade ad pilicimuniti: Artefici 5 (e bene scherzando gli efaita sopra i più eccellenti
Macitri, Oitre alle picture ve anche wo foaffale pieno di libri dei medelima yaio-
IE ye. » che sono te pitture, ¢ scolture.
FRONTESPIZZ/. Vedi sorto C. 9. stan. 15. i
MAIOLIC 4. Specie di piatti, ed altri valeilami di terra, la quale meglio,
Che im aites iuogh: si Javora oggi in Faenza;¢ questa terra è detta maiolica dall
Mola di Adsiorica, 0 Adaiorca dove già si fabbricava; €1' lfola che diciamo oggi
Maiorca già si diceva Maiolica,, come si vede in Gio; Villani lib. 4. cap. 30. WVe-
54 anni ds Cristo 1117. Gui Pifani fectono nna grande armata di Galees e Navi, ed
ndaronofopr' ali' Hola di Adavolica., B che iaquelta lola si fabbricafiero tali va-
lami si deduce, non folo da] nome, che ritengono di Maiolica, ma anche dal
Vederfi nelle fabbriche antiche di Pifa 4 etparticolarmente nelle facciate delies
Chiele murati di tai piatti come per trofeo, ¢ memorie delle vittorie havute da
i Pilani contro ai Maiorchin: 5
VNA belia Cutadina in vila, Era già in Firenze un Pittore da pochi soldi, il
quale faceva ritratci di Principi, di donne Fiorentine in abito da Villa, ¢ da Cit-
ta, de Sibuie yle Mules ec, -¢ tucto.così malfatto, che non ¢ran comprate tali
picture se non da genti di contado,¢ per vilidimo prezzo. Dette pitture si ven-
devano forto le Logge, che (ono d' avantia quelle flanze, dove si radunano i Ma-
giltraui di Firenze,c questo luogo si dice forto gh Vfizei y ¢ per una bella Cittadina in
Villa, € una Sibilia incende di queste belle pitture.

D1 quei Prafitelli, Di quelli Scultori valorofi, ¢ celcbri, come fu Prafiteles;
/parla però ironicamence, ¢ per derifione. Praffirelle detto poeticamente come
Annibaile, Eetorre y ¢ simili per ia rima,in vece di Praffitele, Annibale, Ben '

“hoo 0

=SERS BQ SSESHER ESS

AS

2s
Se

=

a= EE

   
    

392 MALMANTILE™
Così i Latini raddoppiarono la Lat. in Relligio, x Relient a ¢

la legge del verso. 4
CHE a i faffi i. daensit mined Settignano. Dar il moto ai fafti,
si vuol dire Formar figure di pictra: Virg. vines ducene de marmore
Settignano Borgo vicino a Firenze abitano quafi cutti fearpellini
fabbricano poco altro che stipitt, (Caglioni Primi if
che di cafe, ec, taluolta Javorano anche delle figure » ma per lo
le fuddette pitture; ¢ pero il Posta (cherzando dice, dannoi moto
che voglia dire animano i fafi,, fabbricando statue, che peiolowive
de, che danno il moto ai (atli, cioé gli muovono, ed ¢ s I
quali sono ia.quei monti di Settignano, lyogo detto così quali
dere, 0 posieilione della casa Seprimia, antica Romana, siccome
della Perronia, ¢ altri molu iuoghi dello Stato 5 che risepgane anon
padroni, nobili Cittadini dell antica Roma,

QLVEL bafso ritievo di tor mano, Se, Perché fir' c d
erano queste statue » porta I efempio d' una figura y che &: nell irchi
ste della Chiefa di 3. Paolo de i Carmelitani Sealzi, che è-una

afio rilievo, la quale rapprefenta, o almeno dourebbe rap,
Jo, maé lavorata così maravigiiolamente male,.ches'é rela
sua storpiataggine; ed €-compagna delle (tupende pitture del Pamo
Zannino da Campugnano. Jntendendo dunque al omen Boera
tre figure, che le sono attorno fatte della medesima maniera vuol
che si vedevano in quella Gaileria eran maligimo fatte,

NON hanno pago. Non hanno prezzo: E? parlareironico, e-wuol
hanno prezzo, clot non s' apprezzano 5 non si fimano, non vaglion A

SC.AFFALE. Armadio aperto fatto.a palchetti per uso di tener libri. |
nome di Sehapha, e di Scapbos fidicono in Greco molti arnefi,e stramenti, 1
tutti'o concavi, o- (cavati per uso di tener roba 5 dal verbo scapresm
re cauare,scanare, Onde scaffaie, arnefe y che ha varie capacita 5 €
ne' quali si ordinano,¢ si pongono i libri Lat. platens armarium

SHMON Mago, Fu lt Autore, ¢ capo de'. Simoniaci; essendo-
che tentafie di comprar da 5, Piero i beni.Sacri, ¢ Spirituali, come si
acti degli Apostoli. & che cosa sia Mago, Vedi sopra C, 1. stan. 20,5

POC AVANZI, Fu un Libraio Fiorentino ¢osi detto, ii quale nel'
 Autore compose la presente Opera era ridotto in poverta, € vendeva'
che leggende.

CART A di cotenna. Intende Cartapecora.

LIBRETTINE, Quel libretto, che infegna conolcere le figure dell"
¢ le prime regole del medesimo. Il Burchicilo..Vedilo andar,¢h e* par
tine, cioè ¢ tanto magro, fecco ye ee C' pare una signrad
tini un macilente, efienuato » ¢ deforme nelio fieflo modo.
grammo, ioe delineato solamente y¢ fattovi il (olo,¢ puro din

o colorito.
MEDICINA. Quando si dice femplicemente medicins da noi?
 bevanda folutiva, che si beve'con la preparazione, oe ¢
ta prima con alcuni sciloppi, cc.

 
    
  
 
 
 
  
   
  
 
   

 
 

  
  
 
  
    
     
  
   
    
   
        
   
  
  
 
   
 

 
  

OTTAVO CANTARE., 393

| GORGHEGGIARE, E} termine mufico da i Lat. detto Vibrifare, ed & un tril-
lo di voce fatto con la gola, al quale in un certo modo € simile quel romore, che
fa nel.corpo il vento, 0 altra follevazione d' umori cagionata dalla medicina,
ed il Poeta ii » di questo romore, che fa il corpo dice, che il pazziente>
pud far di meno-di non cantar così, poicht Galena, cd Avicenna haveyano
in musica tali medicine

  
  

b STANZA XXVI,
ave n't in rima, che la Sfinge ¢ detto Perch' ci, chefa, chee Sale bebbe concetto,
| Seelta d Enigmi, che non hanno nguali, Acciò che i versi suct fieno immorcali,
Perch! agnune ¢ distinto in um fonetto, £ i vermi dell'obblio non dien or noia
we Che if Poeta ha ripien tutto di fali; Porgli fra fale,e inchioftro in falamoa.

Bra questi libri delle Fate si crova anche la Sfinge, che è una scelta d' Indoyi-
i distinsi ciascuno in un fonetto, opera del sig.\ Antonio Malatefti; la qua-
Ieil nostra Poeta ( facendo di efla quella stima che merita ) non haverebbe metia
ion le, se il medesimo Malatelti non ! havefle forzato a farlo,com-
lo egli medesimo la presente Ortava nog alterata punto dal nostro Poeta.
fale Opera conticae ( come habbiamo detro ) Indovinelli, il Malateiti
il nome di Sfioge, che fu un Maitro appreflo a Tebe, Figliuolo ( fecondo
no )del Gigante Titone, ¢ di Echidna, che significa Vipera; ¢ Fratel carnale,
il, della spaventola Gorgone, del Can Cerbero, del Serpente
di pi tefle chiamato Idra, ¢ di pi altri moftri ¢ animalacci, il qual moftro di-
4 -tmorava in-un monte contiguo a Tebe sopr' ad uno scoglio vicino alla strada, ed
| a chiunque paflava proponeva wx dubbie[ che i Greci dicono evigma, i Latini
nt 'uphas pure dal Greco; ¢ noi indoninello come sé detto sopra C. 6, stan. 34.]e
; Leqlioa ace Jo scioglieva, il moftro improvvifamente lo pigliava,e}' uccide-
'i va. Agcadde, che Edipo figlio di Laio Re di Tebe fu quivi mandato, ed il Mo-
, fico gli propote: Qual' era quell'Animale, che da principio andaya con quattro
“ piedi, poi con due, ed in ultimo con tre = Edipo rispole, questo esser ' huomo,
"i, che da bambino va carponi con le mani, ¢ co1 piedi, € così con quattro piedi,
se poi rittoin fa due piedi, ed in vecchiaia con tre, perché va col baitone; E con
tal folygione vinfe il moftro, che percio si mori. Pe
 RIPLENO di foi. Ripicno di belli, ed arguti pensieri. I Latini ancora chia-
vif mavang falil' arguzie, trovandofi in Orazio.. Nofri proaui Plautinos landanere
fale, Giulto Liptio Aatig. lect. dicit se amare elegantes Plauti fates, Lucano: Non
ie Solici ifere fates. Tor. in Eun, Qui baber falem, qui in te edt, intende scienza, fa-
“if 'pere. Ma qui.' Autore scherzando con l'equivoco del fale dice: Che il Mala-
teftisil, (a che cosa ¢ il fale,¢ che cifecti partoriica [ perché egli era guar-
dane azzini del Sale di Firenze] 'ia meffo de i fali.ne i suoi fonecti, per
#1 fr loro falamoia con } inchioftro, athaghé i suoi versi si conferuino, ¢ G
mw?  difendano da i tarli della dimenticanza, sapendo, che il fale conferua, © difen-
we ic ins; ¢ le composizioni si conferuano da i vermi dell' obbiio con,
g@  Icriverle, € queito si fa con ".inchioftro, ¢ pero lo chiama falamoia, I Latini
cone la 1a Murra, del che noi componghiamo la voce /alamoia, quali
falis mursa, 1' iachiottro da Monfignor Ciampoli fu chiamato dal con(eruare s¢
Orie € i noint degli huomini Bai/amo della fama,
t Ddd STAN.

   

  
 
 

“

a

 

  
        
     
      
          
      
   
 
  
 
    
  
 
     
 
   
  
      
    
 

394 MALMANTILE ©

STANZA XXVIL
Altri Poemi poi vi sono ancora, E uncerto Mal
£d hanno caparrato alla Condotta Ecco subito bell! ¢
Grillo ilGiambarda, Ipolito,e Dianora Le Deecol Babi, chel ha
1 ferre Dormienti, e Donna Ifotta; i Z
Narra che moir' altri Poemi sono in detto scaffale, ¢ mette t
frottole composte da' Cicchi per le donnicciuole, ¢ per i fanciulli.
genie dice, che fara ancora la presente fuaopera, '
sNC AP ARR ATO, Data la caparra cioè dato danari innanzi per fert
mercanzia per conto proprio. ( Voce formata, dice il Perrari, da cape a
Qui vuol dire che hanno chicfto lu MALMANTILE, Gili antichi d
rare da Arra, caparra. "
ALLA Condotta, Così & chiamata a Firenze una strada, nella quale
botteghe i Librai, e alcuni Stampatori, ed ¢ così appellata, perché
stima strada haono i magazzini coloro, che tengono 1 muli per lao
mercanzie a Roma, a Bologna ed altrove..
MESSE in rotta le Dee col Bambi. 11 Bambi era uno, che vendeva
maggio, ec. che noi chiamiamo Pixzicagnoli. Dice che le Ninfe sono p
car lite con detto Bambi, perché eflo impedira, che elle non habbiano il B
di MALMANTILE,, volendolo egli per farne alle accinghe tance ¢:
per inuoltar falumi. Ed in fuftanza vuol dire, che la prefenve sua Opera'

2 2 eee.

r=

na per vendere a pefo per carta al pizzicagnolo; che così diciamo
che un libro non habbia in se di buono altro che la carta.. E qui se
dice questo per sua umilta, ¢ modeflia [ non essendo 1a sua Opera da
pelo per carca j tuttavia, non sapendo che la mia penna dovea farle meritare
tal fine, fece buon pronoftico, € non dubito, che havera dato nel fegno I
Lalli nella sua Franceide C. 4, stan. 21. Si servi di
E le cartacce lor servono al fine
Per avvolger U acciughe ¢ le Tonine,

STANZA XXVIII.
Bovvi anch' un libro ds fegreti, il quale
Gioua a chi legge, ¢ infegna di bei tratti
Ed infra' altre a far che le cicale
Cantsn senza che'l corpo se le gratti,
Ea far ch' i tordi magri con? occhiale
Guardandogli divengan tanto fatti,
Deferive pos moltissimi rimedi
Per chi parisce de i calli de' piedi,

STANZA XXX.

Perchi la donna come altera, e vana
Sopr' agli sfoges ognor pen[a,e vaneggia,
ae cht el?” abi un ceffo di befana
Pomposa,e riceavuol che ogni la veggia;

 

   
     
  
     
 
   
  
    
  
   
    

questa medesima frale. ee

STANZA XXIXi
S? io vi narraffi tutto il
Coftui, direfti, ha it cera j

Pur vuo! contarnen' wna folamentt R
Chie vera, ne crediare eb io sarfilh "i
Racconta a! una tal parturientt:
Ch' una carrenea seen faeae 5 &
E ch' una voglia fu, che bawen bavwsy b
Ed io lo crederé senza dispura.
Percio colei bebbe 1a voplit;
Della grandexza dell' I
eanceeioeadeg robe ik i
Le girelle vorrian, ebe'tfa 1
E è
   
  

rts

a

SB SSE CRELESE EGE

SEES SRSA Seth

i

Ma hafti circa i libri quanto ho detto,

 

OTTAVO CANTARE. 393
“STANZA XXXL

ed qualch' error novoglio far fuggerte,
 Perch'ioche negli Pudi non m'imbrog lio, Che pur eroppin' ho fatti for' al fogiio,
eee altri non ho letto E pot perché fom tanti,¢ tanti i tomi,

Lorfe i fatti lor faper non voglio y Che ne anche fo dir d'unterzo: nom:,
eos il racconto de i libri, che sono nello (caffale,¢ narraado un favoloso

=,  iperbolico parto, fa una leggieri fatira contro al luo delle donne.

 10 sfarfaili. lo aggiunga al vero: Io m' avvantaggi acl racconto. Dalla far.
falla, che gira ¢ s' avvolge or qua, or la, ¢ detto sfarfallare.

 ¥NA vglia fu, Che cola sia voglia in questo proposito. Vedi sopraC. 2. st. 42.
— ALTIERA,e vana. Altiero, si pud dir finonimo di superbo, pigliandofi
speffo ' uno per I altro; se bene a/tiero si dice colui, che per grandezza d' animo
non riguarda,¢ non applica a cose vili, anzi dimoftra vers di quelle una cerca
schifezza generola, ¢ senza vizio,¢/uperbo G dice colui, che per vizio, ¢ per

apriccio spropositato disprezza tutti, ¢ tutte te cole indifferentemente, ¢ senza
Thasoee alcuna. Qui, dicendo a/tera intende piena di prefunzione di se stel-
fa, che ¢ lo stetlo che /uperbo; e Vana dedita alle vanita, o vanagloriofa, boria.
fa, li Petrarca distingue queste due voci, dicendo nella Caaz, 22,

costs + Ch' in vifta vada altiera, ¢ difdegnofa,
Non superba, ¢ ritrofa.

 BEF ANA, Significa Donna malfatta: perché befana diciamo un fantoccio fat-
todicenci, che si fuole da alcuni mettere alle tinestre il giorno deil' Epifania, il

jale da Epifania ¢ detto. corrottamente il giorno di Befana. Vedi forta C. 9,

Sis

I,

TREGG/A. lntende carrozza. Se ben tregeia è un veicolo ruftico senza ruo-
te per ulo di portar paglia, ¢ legne, ec. facendolo tirar strasciconi da i buoi.
Servio sopra quel verso di Virg. 1. Georg. Tribulaque, traheaque, © iniquo ponde-
re rafirs dice così. Traha genus vebiculs dittum a trahendo; nam non haber rotas,
edé la nostra Treggia. F:

4L fangue tira, L? inclinazione, 0 genio le spinge, le forza, Intende che le»

irelle, che le donne hanno in testa, havendo fimpatia coal' altre girelle, fanno
Seiderare alle donne quelle della carrozza.

NON m! imbroglio negli fudi. Cioè; non attendo agli spudi; nan ho che fare con,
loro; nom mi intrometto di fiudiare; nan me ne impaccio,

PUR troppi n' ho fasti ful foglio. Per modettia intende; Pur trappi sono gli er-

rori che ho fatti nel comporre la presente Storia.

STANZA XxXU,
Però seguiam con Paride le Dee
A veder cose belle, e Strauaganti;
E prima tronerem di gran miscee,
 Corpi di Mummie,ed ofa di Giganti;
 Her in corpo a pesce due galee,
Tmpietrive com turti i naujganti y
eps 9 li quali esse han per tradizione
Ci

fur fatti del gingerol di Nerone.
Larti del gingguol di Ner Hid

STANZA XXXIIL

Chinfe nel vafo poi vedrem le cotte
C' bebbe quel Vecchio Chioccia di Sileng,
El asta che fu, dicon, di Nembrotte
Con che voile infilzar 2 Arcobateno;
Benché si creda più di Don Chisciotte,
E veramente non puo far di meno,
Perché in vetta nel mexzo delia lama
V' è scritto Dulcineach' erafuadama,

2 STAN.

 
396 MALMANTILE

STANZA XXXIV.
Pende dal palco un fecco gran Serpente,
Che uh al Cocodrilo s' afomigtia,
E dicon che 1a coda folamenre
Per laliighexza arrina a cingne miglia;
A1a quel che pitt curioso di niente
E' certo, è una grandissima conchiglia, /
Ouxe fra minuta alga, ¢ poca rena Chi vi dipana fa quant'
Sta congelaro un' uouo di Balena, C” al fin @ ogni gomitol si
La(ciato il raceonto de' libri, torna l'Autore a narrar le cose mai
singolari, che sono in questa Galleria, E perché in tali Gallerie i proc
le fa di riporui cose flravaganti, ed ancicagliec ragguardevoli, ¢ molte da
ne fingono per accreditare il luogo, € pero 11 nostro Poera mette anche
mano di cose iperboliche, come sono due galec impictrite in corpo 4 |
€ favolose, come un vafo pieno di gotte, ec, Vedi Liaciano neil' fitoria
ove delcrive terre, ed huomini in corpo'a una'balena; B Efiodo, ove
il vafo di Pandora, ove erano tutti i malori, ¢ tutti i malaoni,
AUSCEE. Intendiamo bazzecole, mafieriziuole, ed arneli vecchi di
prezzo, che habbiano del curioso; metcuglio di bagattelle, di curiosita ¥
AV MME, Vedi (opra C, 6. fan. 52. i
GWVGGIOLO di Nerone, Habbiamo un 'nostro detto, che è: Meron
ginggiole, che serve per esprimere; 4 fortuna mi s' artranerfa; Ml Diaual
disce l'efecuzione del mio pensiero, E viene non da Nerone iimperadore,
contadino chiamato Neri, il quale stava sopra un giuggiolo, offeruando
che entravano in casa sua pee rubare, ¢ toftoro accortifi a' efler;
moftrare che gli volevano fare una burla, ¢. non rubare: gli ditiero; 4b WV
\w fei in ful giuggiolo, intendendo: Noi t' havevamo ben veduro. E del lepname
di questo giuggiolo dice, che eran fatte le due alee impietrite incorpo.al pele.

VECCHIO chioccia, Vecchio malandato. ' uno, che sia alquanto infermo de
ciamo chiocciare; dalla chioccia, gallina vecchia,e spelata, che cova i puleitl,
come il malato cova il letto; ¢ !Autore chiama Suleao vecebio chioccia
Icno Pedante, ed Aio di Bacco si faceva portare topra aun' afino, C
mezzo infermo; ed i Gentili dicevano, che egii si trattava in quelta forma, pet-
ché eflendo egli il maeftro di Bacco, il quale € numerato fra gli Dei el
amici delle comodita, ¢ del piacere, era gitilto, che fudde un' huomo di tuttil
suoi comodi. ar tag

VOLLE infilzar  Arcobaleno, Volle infilzar \Areo-Celefte; che il chia
mavano Iride, ¢ la dicevano insieme co' Greci. Atmbafeiatrice degii x

wee

4B. 5.
: Frinde Colo mifit Saturnia Tino,
Ed il nostro Poeta dicesche Nembrotce vole injfilgar & Arcobitteno
fu quello; che Pe eer si pensd di voler guerreppiar col Cielo ed.
to fabbricd la famofa Torre'di Babel, cioé della confufione. ar
DIN Chifeiorre, Che in nostra lingua voreébbe dire: Di
mile, Fu ua Giteadino-delaMuntia, il quate havendo letti molti

     
   
  
  
  
 
    
  
  
 
 
    
 

 

  
    
  
      
 
 
   
    
   

 
  
     

pA seop ok FLFR e wn lere2t sere res e-2s

ae

 
 

    
   
 
    
    
   
    

OTTAVO CANTARE 397
valleria, cio? Amadis di Gaula, Palmerino d' Oliva, ec. s' imbriacd, eddinuaghi
dei meftiero di Cavaliere Brrace di tal maniera, che si meffe ad immitare le azioni
di detti Cavalieri » facendofi armare con quelle cirimonie, che eran soliti fare
“quei; anch' egli a cercare l'avventure, come graziofamente rac-
conta 26 Michel Ceruates'nel suo D6 Chi(ciotte,il quale fu molto bene tradotco
nostro volgare da Lorenzo Franciofini da Castel Fiorentino, assai benemerito
' a Spagnuola; (1 aggiunta, 0 fecondo libro del qual racconto' voglio-
 no, che sia flato composto da Carlo V. Imperatore ) E perché i Cavalieri Erranti
Ron erano stimati veri Cavalieri, se non havevano l'innamorata, però quelto
@ Don Chisciotte si finfe ancor egli la sua, che fu Dulcinea del Tobofo; E da questas
ae il nostro Poeta prova (cherzofamente, che questa Atta fulle più tafo di
| Don Chilciotte, perché nella lama', che era.in cima alla detta asta v' era (eritvo
'Daulcinea, ed intende, che questo ferro era doice, cio¢ di cattiva tempera.
FN gran Serpente, Quetta iperbole del Serpente ¢ posta qui ad immitazionc, o

iat per dir megho, in derifione di coloro, che scrivono le Storie d' Etiopia, che>
wi di -eflerui tali Serpenti, che ingoiano un Ceruio, 0 un Bue intero per volta

 €sono di lunghezza di piii di trenta piedi; E che M. Attilio Regulo nella prima
til > oda ai Cartaginefi ne uccidefle uno in Affrica preflo al fiume Bagra-
it che era lungo 120, piedi.

MANTICE, 0 mantaco. Vedi sopra C. 1. stan. 55.
si) = SARCOL AIO, Steumento fatto di canne rifefie, 0 stecche dilegno, sopra il
wai 'wales' adatta la matafla per comodita di dipanarla, 0 incangarla come s' ¢ det~
wi WifoprarC. 5, stan. 9. E dipanare € raccorre il filo,formandone una palla per co-
shi imetterlo in opera, ¢ tal palla si dice gomitolo dal Latino glomerare, e+
i Soma che il gomirolo, che a Roma ancora si dice glomero.
4) STANZA XXXVI. STANZA XXXVUL
se Van Sfera bellissima si vede, S'.in Grecia fatra fu la criftullina,
nis © (Ch sopr' aun ben tornito piediffallo, E quelta di vesciche vien da Troia,
ae 'Che per ginfiexza tutze l' akre eccede, Che a Fiefol fu portara a Catilina
ail on farte di legno, 0 di metalios ue norte ch' ei ee verso Piftoia,
es pure, ¢ fotrerrifi Archimede Ch' ei non giunfe ne anc! alla mattina,

i Con lla sua, ch'ei fece ai Criftalla, Chet arate wi tio le quoia,
am % Che bifogna guardarla,e /parsi addietra Sicché due Capitan sue camerate
; ie “ “Per'timor di non romper qualche verro, La prefero ye la diedero alle Fate,
2 STANZA XXXVIL STANZA XXXIX,
Che quefia 5 che con ogni diligenza Mentre s ammira così-bel lauoro

Di purgate vesciche fu commilfa,

E vi si fanno fu cento argumenti,

iv Se perdisgraia, 0 per inavvertenza Paride guarda 5¢ vede una di loro
ae Perquote ocade,ell' ¢ Sempre la fiefa; Canarsi un' occhiolaparrncta,esdenti,
E sel criftalio ha in se larrasparenza, E dargli aun' altra,perch'inturto ilcoro

 LA vescica al Diafano s' apprefia, Delle. Naiadi ch' ini fom presenti,
if --Edé\un corpoyche giammai non varia, O fuora (che pur anche fon parecchi )
eo E-quel si cangia ognor fecondo t' aria, Ha fol quei détrynn'ecchio,e due cernecebi

Se.

 

STAN-
 
 
 
 
 
 
 
 
  
  
  
 
  
   
 
     
   
 
     
     
  
   

398 MALMANTILE. ©

STANZA XXXX.
Pero ch' elle fon cieche 5 e vecchie tutte y
E loro i denti fon di bocca nsctti,
Ma ni per questo ell' appariscon brute,
Ch' ell! hanno volti bel, ¢ coloriti,
E se mangiar non possoncarne, efrutte
Elle s' aintan con de' pambolliti
Perché quei denti,come gli occhi,eiriccs
Non hanno pin virtit, che fom posticci.
STANZA XXXXIL
Così per offernar le lor vicende
lucha ch' io dico se gli caua adcffo, Cedendo ogni ragione y¢ ogni
Già ritornata dalle sue faccende, Perch' inqueff'oraa è
Perch' il portagli pin non le ¢ permeffa, La fronte escape,erife
Descrive una Sfera fatta di ve(ciche di Porco, ¢ moftra, che sia
re di quella di Crifallo, che fece Archimede Siracufano, perché ¢ pili f
più ficura. Mentre che Paride stava mirando, ¢ dilcorrendo sopra ilb
della Sfera di vesciche, una delle Ninfe si cavo la Parrucca sun' Occhio, 1
¢ dette il tutto a un' altra, perché così ¢ l' ordine fra loro, Qui pate, che:
alle Lamie, Donne, o Larne per dir m lio, che con carezze allettatrici
stimate da' superftiziofi Gentili mangiarsi i bambini; le quali fea cutre
no un' occhio folo, ¢ quello usavano a viceada hor quetta her quella,se
deicrive Angelo Poliziano lib. 3. tit, Lamia, che dice: Lamia h
excmptiles, hoc eft quos fibi eximunt, detrahuntque cum libuit, curl
y» cum libuit refumunt, atque affgunt; alice vero ctiam dentibus utuntur eque
y» exemptilibus, quos noéte non aliter reponunt, quam togam, ficut ha CO
>> mam (uam illam dependuiam, & cincinnos, &c. Sed lamia hac quoties do
egreditur oculos fuos fibi afhgit, vagatur per fora ee plateas, &c, domum ye
»» ro cum revenic, in ipfo fatim limine demit illos fbi oculos, abijcitque in le
»» culos; ita femper domi cca, foris oculata,,
PIEDIST ALLO. Bi quceila pictra, che ¢ foto al dado, sopra il quale pola
colonna: ¢ qui ¢ prefo per tutta la bafe, che regge questa sua Sfeca, comet prt:
fo comunemente. aia
VADA, efotterrifi eArchimede. E'o(curata la gloria d' Archimede; Quan?
uno fa un'operazione meglio d' un' altro diciamo al superato; T# ti pwoi ire ari:
porre,oafotterrare. Intendendo; Tu hai perduto cutto il credico, o la flimas
che ¢ quella senza la quale uno è tra gli huomini come morto; i che
che non si dee più far taata Mima della Sfera d' Archimede fatta di cri
ché questa facta di veiciche l'ha tuperata. 2
DATvoia, Non dalla Città di Troia, come pare, che ogi dice
Troia femmina del porco, delle cui vesciche era formata guetta sfera
V1 tira e quia. Vimori. Vedi sopra C, 4. tt. 20, Qui cocea la con
nione, che Catilina famofo capo di congiura descritco da Salu(tio mori
floia «
Vi fanno cento argumenti. Cio' discorrono afiai sopra questa Sfera. —

xy

  
    
   
 
 

 

 

 
 

 

OTTAVO CANTARE. 399

Ml PARRVCO-A, Voce straniera fattz nostrale, ¢ vuol dire Zazzera, dichioma
®@ finta, che diciamo: Zazzera posticcia dal Francele. Perroxque, chioma. Potreb-
» be forse dirfi in Latino capidamentum.
- CERNECCHI, Capelli pendenti alla testa; qui intende quella parrucca,o ca-
peli postice:; se ben cernecebi Gi dicono quei foli capelli, che pendono dalle tem-
* pie agli orecchi con altro nome dette faceagore, che i Latini, fecondo i) Polizia-
no nel luogo sopra citato dicevano cincinnos, E noi diciamo cincinns quei ciondo-
 lidi pelo, che fogliono haver i capretti, ed i Becchi foro la gola, i quali hanno
st qualche similitudine con quetti capelli, che noi chiamiamo cernecchi.
PAN bollite, Pappa fatta di pane, bollito in acqua.
 MASC-ALCIA, Magagna; Difetto; mancamento. E' lo steffo, che guida-
" Ielco, ma questo si dice folo nelle bettie, e mascalcia, che farebbe veramente so-
Todelle beftic, ' usGiamo anche per gli huomuni, ¢ taluolta per i materiali, Vie
'un' antico libro Toscano intitolato Libro d+ e#a/calcra, che @ dell' arte del ma-
nescalco, de re veterinaria,
DA quella via 01a quelia via, Subito. Senza metter tempo in mezzo. Latino

=

SEE

ill extemplo, ¢ veffigio. Se bene si potrebbe intendere ancora per In quella manicra;

ind in quella guisa, come ¢inteso sopraC. 7.0.84. 0

na oN ogni regreffo. Cede ogni azione; ogni autorita. Vedi sopra C. 7.st.104.

git. RIFERRAR (a bocca, Incende rumettere i denti. Bocca sferrata si dice a uno,

em =the habbia meno i denti dinanzi dal ferrare le beltic, ¢ rimeteer loro i chiodi a'

il pied, si sono sferrate.

mw STANZA XXXXIIL STANZA XXXXIV.

0 Pitna di cibi intanco nna credenza Credilo a me ch' eglié del eloriofo

pal Wit pari pari aperta spalancata, Pero qua dentroyvia,distendi il braccio,

8 OB fatta da vicin la rinerenza Che trouerai del buono, ¢ del gustofo,

ye Parole pronunzio di guespa daca: Se tu voleffi ben del Castagnaccio,

pi ° Caualier, se tu voi far penitenza, Paride fece un po del vergognofo,

on 'Ein parte a noi piacere,e cosa grata Hla nel veder le bombole nei ghiaccio,
440 munizion da caricar la canna, Mando prefto dabanda la vergogna,

gl E poi da bere un vin ch'é una manna. E fece come i Ciechi da Bologna,

oF STANZA XXXXV.

Levategli poi'via la calamita Sicch? in quanto ad bauer taglioo ferita
ost Di quel buon vino,e maffime del bianco Jn altra parte era ficuro,¢ franco,
ipo Gli fararon le Dee tutta /a vita Poi dangli un brando con la sua cintura,
gil  WDalla baferra in fror del laro manco, E del trattarlo  intavolatura,
ye Mentre stavano guardando le fuddette galanterie,comparue und credenza aper-
i ta piena di roba da mangiare, ¢ da bere, ed inuiid Paride a soddisfarsi; egli dopo

haver fatto alquanto lo Riiakian'> mangid, ¢ bevve; Terminato il mangiare se

46  'Ninfe lo fatarono,rendendogli impenctrabile tutta la persona, eccetto che la ha-
'fetta mancina + Qui il Poeta immuta |' Autore, che favoleggia Orlando impene-
o@  trabile in tutta la persona, eccetto che nelle piante de' piedi.
@ | CREDENZA. Così chiamiamo un' armadio, entro al quale si ripongono, ¢
tonferuano gli arnefi, ed avanzi della menfa; il quale armario si dice ancora,
4 tredenziera  perché quei bicchicri vaii, ¢ baciji d' argento, ec. che si mettono al-
Ic

 
400 MALMAN TILES ©
le tavole de' Grandi per servizio; 0. per apparato della
diti tutti insieme, si dicono credenza, ¢ i si rij
vriano riporre in detto armadio, che però lo chiaa
tino eroacns re

SPALANC AT A. Affatto aperta. Vedi sopra'C. gf. 38, Pal
cato diciamoa la chindenda\, o riparo fatto con è pali, a un >
vuol dir Senza palanca, ¢ per conleguenza totalmente aperto, ©
tegno, 0 inipedimento. ei è

PAROLE di quefia data, Parole (mili a\queste, 0 di
ta, la quale si attende moltifiimo nel gioco delle'carte', per ef
chiate; Onde si dice: Ha farrauna buona,o una cattiva data,

SE tu vuoi far penitenza, Se tw vuoi mangiare. Termine usato per:
inuitar' uno a definare, o cenar connoi, guafi ditiamo:venite a digi;
ché la nofira menfa ¢ povera, ¢ (eatlardi cibi. Sidicc.amcoralfar carisa 5,00

s'é vifto sopra C. 5. st. 68, otis verb 9!
HO munirione per caricar la canna. Ho roba da mangiare\, eda
care la canna della gola., ¢ non quella dellarchibufo <2 5, Gate
VN vin, ch' una manna, Vino ifquifitissimo, che tale si legge fulle!
che mando Dio nel deferto al Popolo cletta., Vedi ferro Cip.st58..Ma
stranicra, ma fatta nofirale, che significa una brina:condenfata: tenera'y'
detta cos} dali' Ebraico, Azanbm; ioe Quid of bee: come: si dice nell?
16. Poiché maravigliati gli Ebrei di questo nuovo, ¢ faporolo cibo 5
uno all' altro; Che ¢ cid, che no? mangiamo ? Dalquefta dolcezza viene '
nostro detto. 1 Latini dicevano in questo propolito dowis Velar..
EGLI¢ det gloriofo. I Battilani chiamano vino gloriofo il vino pga 4
rofo, e buonissimo, e dicono groliefe in vece di gloriofe; cio' valor lo
vaalle stelle. In certe Profe Toscane antiche, delle quali alcune si ritroyanom
nuicritte nella Libreria di S, Lorenzo date fuora dal Doni, vi ¢ una leteera amt:
rofa, nella quale ¢ accennato Amore con dire: Quel gloriofe; titolo dato

da' nofiri Battilani al vino; ¢ veramente Amore non imbriaca meno di

si faccia il vino il pith gloriofo, a
VIA. Quetto termine serve per follecitare, © incitare uno. Latino Eia ae,
CAST AGNI AC C/0.Pane fatto di farina di Castagne: qui vuol

per opera d' incanti quella credenza dava tutto quello, che uno fapeva

itis g

tern gab

   
    

rare.: a
FECE il vergenofo. Finfe dinon si ardire a mangiare. Moftrava vergogaitt
d' accettar l'inuito, che gli faceva quella credenaa. ig
BOMBOLE. Vali di vetro, i quali servono per mettere il vino in
ghiaccio, 0 neve, detti così ( fecondo alcuni ) dai fuono, che fanno-nel |
fuori il vino, che par che fuoni bombof. Al Rotenano' vuole, che i Latini
da tal fuono le diceflero amphore bilbina; ma pud anch' eflere, che agi ie
così da bombo voce.puerile 5 che vuol dir bevanda, decta così dal fueno, —
COME i Ciechi da Bolugna, Si da loro un toldo, perché cominci

c bifogna poi dargliene duc, perché si chetino. Ci iciue per espri:

 
   
  
 

 

 
 
    
      
    

i
mt

SBER ote

=
=

Ask RESS Pa SeETLE

 

T~s-e

>

 
    
   
     
   

   
 
   
 

   
    
   
 

fs

rad
ike

i
wb 1

-

4

  

4
4
of

4

   

a 4
 Pigliual
| Ainqud franortese thcome,e ilquddo,e il doue,

 
 

Escciam

Ciechi da Bi

 
 

re.
~ STANZA XXXXVI'
'Eperche if tempo ormai era rrascorso y
jarlo dowean di quini altroue
Prima in sua lode fatto un bel discorso,

( differo ) quanto t° ¢ occorso

Py ae tutto per appunto,

Ot è gus ra nofira giunto,

oc SPAN ZA KLE X Vil.

Akcibsuvada incontro a un'aunentura

< d prod' un pover huomo questa notte;
Quelle è un tal cognominato il Tura,
CH in Parion confiaua le pilotte.

0 Eta tebellenze un mofiro di natura,

Sicebe tutse /e donne n' eran corte 5

Elafeiando i roccberti, ed i cannelli

2 Per-lui chee cht ¢ facevano a capelli,

“STANZA XXXXVIIL

Non ch? eine deffe loro accafione,

— Come qualcbe narcify inesbertato,
C'una caffia, che e' vegga a un verone

?0/ «far lo spafimato;

' Anis wn diqueie'al Addo ta 4pigione,

«A bioscio nel veftire, ¢ feiammannato,
©' addosso i panni ognor tutti mincfira

0) Tirati gli parean dalla finefira,

“OTTAVO CANTARE,

maghana a Adarte,al Soleje aGiove,

gor

 ptegare a far'una tal.cosa moftrando non voler farla, ¢ bifogaa
ehe refti di farla, Orazio.:
» Omnibus hoc vitinum eff cantoribus, inter amicos
0) Wt munquam inducant animum cantare rogati,
drinffi numquam defiftant,
Sid > da Ferrara, 0 da Milano, 1 Latini in questo proposito
'ditlero Arabicus Tibicen. Qui intende, che Paride si fece pregare a mangiare,'¢
se ¢ poi non si trovava 11 modo, che egli reftafie.

LAMIT A,B! \a pictra Adagnes, la quale ha proprieta d' attrarre il ferro,
punto ha il vino di tirare a se Paride, ed è fra etio, ed' il yino la stetla,
»cheé fra jacalamita, ¢ il ferro. Vedi sopra C, 5. st. 59. B sotto ing
— guelto C, tt. 66,
| | Di trattario ? intavolatura', L? instruzione di come si debba adoprar quella spa-
t+ Intavolatura ¢ scritwura, che per yia di note, ¢ di numeci regola la mano del

STANZA IL,
Ed esseeran capone; ma chiarite;
el fin lasciando quel fuocnor di fmalto,
Fecer come la Volpe a quelia vite
C* hauea si bell' Una ye tanto ad alto,
Che dopo mille prone, anzé infinite
Arrinar non potendoni col falto,
Glié mé,dilfe,chio cerchi altra pastura,
Che quespa da ogni mo non è matura,
STANZA L,
Cosh non Ia faldo era Martinarza,
La qual non vitrouado anch'elja attacco,
Poicht gran tempo andata ne fu parra
Hanédo if rerzo,e il quarto,e ognuno firac
Codurreun giorno fecelo alla mazza, (co
E per via d' un che le teneua il facco
Aunezro a tofar pecore, ed agnelli,
Mentr' ci dormina gli taglios capelli,
NZA LI.

Quei capelii c un rempo hanea chiamati

Del suo falcto mortal funi, ¢ risorte,
Le bioae chiome wh Dio,queicriniaurati
Che ricoprivan rante piazze morte,
Onde feoperti furo s trincerati

Onc il nimico si facea si forte;

Perché (per quanto un Autore accenna)
Lo rimondavon fino alla cotenna,

ale fate dopo haver lodato Paride per bravo, per bello, ¢ per valoro(o gli dif-
fero sche IL havevan fatto capitar quivi, perché egli andatie a liberar il Tora,
quale loda ironicamente, ¢ dice, che tutte le donne crano innamorate di iui; ma

 

Ece

accor-

 
   
     
     
   
 
     
   

qo2 MALMANTILE (9

accortefi, chenon corri leva a niffuna, lo lasciarono, ¢:

egli non volle mai corrisponderle, haveva fattagli la malia
ottave seguenti.

1 DE
AVVENTVRA. 1 Romanzatori Spagauoli in quei loro Amadis di Gaul
Palmerini d' Oliva chiamavano ayventure ( awenturas) quegli i
ne i quali s'imbattevano i Cavalieri Ecranti, ¢ però il nostro:
creato il Cavalier di Quoio,vuol, che ancor' egli sia fimato i

che vada a provare l'avventura di liberare il Tura dail' incantefimo. 1
similmente dissero aduentures, Bi nostri Toscani ancora, fentendofi:
termine cavalleresco, chiamarono gli accidenti, che accadevano

davan loro materia di fare prodezze dawenture. L' Alamanni nel
principio.

 
  

3 teal

Narrero di Girone l'alte auwenture. a ad

E da cid il Boce, Tels. lib. 5. disse: 5 aabiitly
eetterfi in aunentura:. ee

Ma non li parne via ben ben ficura. poe

Pero non se ne mife in auuentira, sung

4L Tura, Coftui cra un pover' -huomo, che gonfiava le pillotte in] }
che in Firenze è la strada, dove si giuoca alla pillotea detta così da
perché in efla anticamente haveano le botteghe coloro, che lavoravano:
mi, 0 pure ( il che forse ¢ pitt verifimile ) quafi Ripar Regio Ripe Roine
tale strada sbocca ful Pafleggio di Lung' Arno; [n Roma ancora vi ¢ la
da di Parione detta similmente cosh detta quali Rione a Ripa, Regio Riper
pure è così chiamata, quafi Parte di Rione; Pars regionis, come mi vien riferito
leggerfi in alcune Carte, o Contratti. E perché veramente coftui era brut
di faccia, ed haveva la zazzera avviluppata', elorda, lo chiama moffro
va in bellegza, ed intende Deforme, se ben par, che voglia dire, di belleam
sopranaturali. aise
PILLOTT A, Specie di palla da giuocare, Vedi sopra C. 6. st. 34.
WV' ERAN corte. Erano abbruciate dal fuoco d' amore per lui Virg, Mir
infelix Dido: dice briache del suo amore, cs' intende innamoratissime di Iai. Lt
ebrie amore. Piauto nel eAilit glorifo,o Soldato,al quale da nome di
cio' di Abbartitore di Torri,e di Città; 0, came noi diremmo Taghacantoiy©
Spacca Montagne; fa dirgli da e4rtorrage, cioè in nostra lingue Sparapane Pata
to, suo adulatore; che tutte le donne sono di lui fieramente innamorate » Le
tibi ego dicam; quod omnes mortales (ciunt; Pyrgapolinicem te unum in tere rn
Virtute 5 & forma, & fattis muittiffimus? Amant te omnes mulieres,neque herce i
ria, Qui fis tam puicher. Ed egli sprezzatore altero di tali amori iange !
Jamente la sua disgrazia, beccandofi fu queste lodi; dell esser hwoWd, |
da fare innamorare di Jui tutto il Mondo, WVimis eff miferia
pimis. '

LASCIANDO i roccherti 5 ed i'cannelli. Lasciando star di lavorare.
prefe tanto forte l'amore; ¢ tanto le teneva fisse nell' amorofo penk
non potevano pili atcendere a' loro usati lavori. Quando Didone si
'rata d' Enca., non tirava innanzi gli edifizj.,.¢ le fabbriche della faa

  
  

 

 

   
  

 

preaereseei ns

 
  

OTTAVO CANTARE:

Virgilio ebbe a dire: pexdemt opera interrupta, minaque Adiroxam ingentes) come
che era occupata da pitt possente pensiero. Co! presente detto di lasciare
droccherti, 4 canneilé, s' intende questo, perché le donne a' infima plebs (che ta-

1 epeweneenye > che erano l'innamorate di coftui ) per lo più non hanno
lavoro

» che ? incannace, ¢ teffere, a' quali lavori s adoprano i Rocchetti (che

fon legnetti tondi forati per lungo, ¢ servono per ragunarui sopra la feta, ed

',

oo LU BRE

AaeTEEsS

TS A A Se ee, ae

altro filo: ed i Cauneii, che sono pezzuoli di canna tagliata fra un nodo, e
Paltro, dai Latini però detti ixternodia, ¢ servono per lo medesimo effetto d' a-
dunarui sopra la feta, ec. per adattaria a telsere; I che si dice incannare.
hone éch'é, Ad oraad ora; Di momento in momento. Vedi sopra Can. 3.

 PACEVANO a i cape, Si perquotevano, S'azzuffavano. Quando due donne
combattono tra di loro diciamo fare «4 capelli; perché il lor perquoterfi, ¢ per lo
 più) pigliarsi 1 una I altra per i capelli.
 CVFFIA. Berretta a foggia di facchetto, entro alla quale le donne & ferrano
icapelli in testa, ¢ quando noi diciamo nel modo, che ¢ detto ne! presente luogo
tna ¢xfia, un ciapperone, ¢ simili arnefi usati dalle donne, intendiamo una Don-
na, Così dal portare lancia, o barbuta; i toldati medesimi si chiamavano Lance,
¢ Barbute, come ficava da Matteo Villani, 11. 81. ¢ Erodoto volendo dire, che
pera si ritrovavano avere in. piedi ottomila soldati, che portavano rotel.
90 brocchiere; disse ottaci/chilian aspida, cioé scudi militari,o rotelle ottomila.
VEKONE. Latino moenianum, podium, pergula, e in Greco fecondo alcuni Pe-
ribolas da peribaliern abbracciare, circondare, che i Francefi dicono enxironner.
Propriamente vuol dire andito, o terrazzo scoperto': Qui credo, che habbia a dir
Baleone,¢ non Verone. Verone & detto quafi girone, cioè giro, dall' andarui sopra
¢rigirare, eAndito, che ¢ lo stesso par facto da e4ndare. Latino ambulatio,
EGLiéa Pigione al mondo. Così diciamo d' un' huomo spensierato, sciatto,sen-
2a considerazione, ¢ che vive a cafo, che si dice anche Auomo a bioscio, sciaman:
nato ( cioè male ammannato,male all' ordine ) e che i panu# gli paiono tirati addofsa
finefira,B cd questi quattro modi di dire |'Autore delcrive l'attilatezza del Tu-
ra;del.refto, parlando fecondo moralita, ognuno dourebbe flare in questo mon-
do, come a pigione; perché la nostra propria casa è nel Cielo. E nel Salmo 118,
4ncola ego fum in terra, i) Greco dice Parcecos, ¢ alcuni Salteri dicevano, come ri-
ferisce S. Agoftino sopra i Salmi, inquilinus, cioè pigionale. F
CAPONE, Oftinato Latino. Pertinax. Pertiwax.
FAR come (a voipe alla Vite. La Volpe dopo haver molto faltato, e dopo esserfi
i per arrivare un grappolo d' vua, ¢ non I havendo potuto arci-
vare disse: La voglio la(ciare flare, perché ad ogni modo ella non ¢ matura..
Pud aver data occasione a questa novelletca quella d' Efopo, della Volpe, € del
Pruno; in cui la Volpe, che voleva falire una fiepe, mi fuppongo, per mangiar
Tuva, della quale è ghiottidiima, pensando di troyare il Pruno buon' amico, «.-
sto ingannata del suo pensicro; poiché attaccandovili refto intaccata, e ! appog-
gio le fu ferita, e volendola poi disputar con lui, ebbe il-torto: E questo detio
¢i serve per esprimere uno, che habbia usata ogni poslibil diligenza per confegui-
Te una tal cosa,¢ non ' havendo potuta ottenere, 9 habbia abbandonata i' im-
Ece 2 prefa

ae
F
7

 
 

404 MALMANTILE ©

prefa come imposiibile, o sia quella tal cosa fata data a un' al
vanti di non |' haver voluta, perché:non era buona, 0 non!
diciamo: farsi honore d una cosa n
COSÌ' non fa falde eartinazza y Cos\ non fini, 0 termind ?
nazza la quale non troxando attaeco, cioé non trovando luogo di f
suo amore ver(o il Tura, del quale ando paz ca,cio' sterte innamoratiti
CONDFRRE uno alla mayza, Tradir' uno: Condarre uno con inga
finghe in mano de' suoi nimici, o della giuftizia, o in qualche altro
come si fuol dire; al macedo, Latino /n infidias ducere. re
TENER il facco, Tener di mano. Aiutare a cometter un delitto. Ha
un proverbio fentenziofo, che dice: Tanto ne va a chi ruba, quanto a
Jucco, che esprime Agentes, & confentientes pari pena puniuntur. B diciamo anche
Tenerfi il facco l'un t altro; che esprime il detto di Teren. Tradere operas mutnas
FEY NI, ¢ ritorte del suo fascio mortale. Metafora amorofa + Si ne
ritorte tengono unite pill legne in un falcio, 0 faftello, così i capelli del Tura,
quafi funi, € ritorte tengono unita col corpo |'anima, cioé tengono in
Amanti del medesimo Tura, E riorre dicemmo, che cosa fieno sopra C.
PLAZZE morte, Si dicono i luoght vacanti de i soldati; per efempio
no € pagato per cento soldati, ¢ non ne ha f€ non novanta; quei dieci
cento, che mancano si dicono piazze morte. Ma qui intende quelle pi
la(ciano le margini, o cicatrici de i mali, che vengono nel capo; sopr”
li non natcono capelli. “|
1 TRINCIERAT 1, V luoghi, dove erano le trinciere. Intende,
gliargli i capelli si sono scoperti quei Inoghi, i quali con- elle: margini
una campagna piena di trincicre. done tl nimico si facena forte s clot di 0
devano i pidocchi. we tage
TRINCIERA, oT rincea, EB' un' alzamento di terreno: condotto a foggia d
baltione, nel ricinto del quale dimorano i soldati per difenderti dallartiglierigg®
de i nimici. Franzefe rrenchée, cioè tagtiata, yene
LO rimondaron fino alla cotenna, Gili tagliarono i capelli fino rafente la pelle.
Rimondare vuol dir Tagliare a un' albero i rami: B curenna's invende folo lap
le del porco, ma quando si tratta del capo s' intende anche quelia dell huom
Vedi sopra C. 5. st, 52. * 4a?
STANZA LIL STANZA LALO
E cos) Aartinaz2a hebbe il suo fine E questo Lupo raggirar si vede 4 2
Volendo vendicarsi per tal via y Intorno a un montnofo cafameme—
Pero sche buona parte di quel crine,
Ch' aleun non few avvedde, leppo vidy
E fabbriconne al Tura le rove
Con una potentissima malia,
Che revifrata in Dite al pratocollo
fa un Lupo rapace trasfor mola,

  
 
 

 

   

4

NS ne ears ESR ene

D' una Genteycheymentre mice ilpitl

 
 
  
 
   
    
 
  

r

yor bS

 

Zexv7ek
 

OTTAVO CANTARE 403,
o» STANZAEIV. 5 STANZA LV;
d vanne,e perché tu non facia Eli la prende con il libro insieme,
— Qualche marronyma vegas arar drittoy Dicendo, che varraffi dell! avviso y
- Acco tal: ero si disfaccis

f Pi disfaccia y E.ched' incano ye dianoli non teme,
- Percht feattado un pel tu baurefti. ifritto, Perch? eglirehuom, che fa moftrar ilvifo,

 

Yi he questo libro qui faccia per faccia Si parte,e per c'al capo andar glipreme,
hl ordine, ei modo si ritrova [critto, da due parti vorrebbe efer divifo;
ipa Portalo teco, e.accio che rm difeerna, Pur vnol servirle, perché si figura
Perch! egli ¢ buio to questa laterna, Che non ci vada gran manifattura,
i Metono: STANZA LVL
wit poi nel suo ceruello, Ricerca nel [uo maftro feartabello
ia Che sa quel luego a bambera s'inuia Di quei pacfi la Geografia y;
| Potrebbe andar a Roma per eMugello, Aa quel(per quato noi potré coprendere)
oPerchvei non si rinnien dow' ei si sia 5 Non si vorria da lui lasciar' intendere,

Hi
id ~~ Martinazza:hebbe il (uo intento, perché prefa buona parte de i capelli del Tu-
sf 4 con essi gli fece una malia, che Jo trasformd in lupo, ¢ lo confind in ua mon.
ati tevicino:a Maimaatile. Finito questo racconto le Fate licenziaron Paride,¢ gli
r diedero un libro, dove era scritto il modo da tenerfi per disfar quell' incanto, ed
we wna lanterna per farsi lume; ¢ Paride si parti con risoluzione di sbrigar questas
we faccenda prima d'andare al Campo.
 LEPPO' via, Portd via di nalcofto. Il verbo /eppare ci serve per esprimere
svélocita 'hell' andar via 0 nel levar via quaicofa.
4 MALIA. IncanteGimo, fattucchieria, stregoneria. i
1” © PROTOCOLLO. Libro pubblico tenuto da i Notai per scrivervi sopra i con-
oe trattiyeteftamenti 5 ¢ così ¢ inteso da noi; se ben protocode vuol dire libro da re.
| giftrarvifopra, che che sia. 11 Berni fonetto in biafimo d' una mula dice:
4 > — E troppo sia diginna
ie ts srry? Cht il prorecollo memoria non fanne.
Perché veramente Prorocoo.è ua libretto,sopra il quale 4 fegnano, ¢ regiftrano
r brevementé le cole, per diflefderne poi scrittura pil largamente, ed autentica.
i  Msnce: deteo così quali 5 primo libra incallaro, ¢ legato. Liber ex glutine compattus y
o in guematta referuntur, Ma il noftco Poeta jo piglia nel fenfo, che oggi usiamo
di libroida Notai, ¢ inteade che Martinazza haveva fatto contratto cal Dia-
mM volo di questa malia; il qual contratto era già mefio al libro del Notaio del Dia-
it volo yeper questo detta malia era autenticata, ¢ non si poteva alterare, perché
' era paflaca per mano di Notaio, ¢ regiftraca al suo protocollo.
#  - CASAMENTO montnofo, Intende il Castello di Montelupo, che ogg? quali
# — distrutco siperd pili colto Cafotare, che Ca/iello, e lo dice montuofo, perché &
sopra un monte come lo moftra il nome medesimo, E nota, che ancor qui il no-
® fico Poeta vaimitandoi Romazatori Spagnuoli, che fanno parlare oscuramente,
| e come gli Oracoli quei loro:Alchifi, Zirtee, Wrgatide, cc, incantatori.
« MENT-RE move il più sopra alia terra v' ¢ rinuolea drento: Le reliquie di questo
/ — Castelio sono abirate da persone, che fabbricano valellami di terra 5 come pen-
tole, boceali, ec, quali si fabbricano per via d' una ruota, la quale va moffa cot

piedi, ¢ fa efsctto del tornio, ¢ perché in muover desta ruota, € fabbricare il
it valo,

  

aiies

 

 
   
  
 
  

oh MALMANTILE
vafo, la terra schizza addosso a chi lavora però dice Ademtre shane il più sopra alla

terra v' è rinuolta drento,
FAR' un marrone; Far' un error grandissimo; wx crrorome,
e4¢KAK aritto, Operar giuftamente, Non fare errori. Tolto dal Bifo
ciamo ancora, rigar diritto. 4a) aA
SCATT ANDO un pelo. Se tu uscifi punto dell instruzione 5 che tu'
tare, 0 scoccare, si dice della freccia quando seappa dalla cocca, t
di gui € tolta la metafora, o forse dall'orivolo'a ruote. See
TV hauerefti fritto. \\ Proverbio dice: Come difela Tinca ai Ti
altra aggiunta s' intende: oi habbiam feito, Qui intende tu hanrelti
tu haurelti rovinato questo negozio, EB' lo steilo che: Noi habbiam
ne detto sopra C. 7. st. 60. ow
HVOM che fa moftrar il vifa. Huomo ardito, e che non fee cen
ABAMBERA, Acaflo. Latino Jrconfultd. Vien forse da ¢ è
vuol dir ragazzuolo,(enza giudizio, B il ragazzo in alcunt luoghi chiamato Bar
berottolo. Dicefianche. 4 fanfera. 2 6a
ANDAR 4 Roma per Adugello, Far' una strada al tutto contraria, come &
rebbe andar da Firenze a Roma, ¢ pigliar la flrada per il Mugello, che € dirt
tamente contraria. oa
NON si rinuiene, Ciok non riconosce in che parte ci si sia, ¢ non fa quel chid
si debba fare. aed
MAST RO scartabello. Tntende quel libro, che gli haveano dato naa

  
  
  

  
 
 

è il suo maeftro, e direttore; Quetta voce scartabello, ¢ corrotta da C.
anticamente era intesa per un libro di stima; come moftra il Dortissimo 5:
ditissimo sig.\ Francesco Redi nelle annotazioni al suo bellissimo Ditiramboac
18. Gli Spagnuoli chiamano Careape/ una scrittura continuata nel foglio fens
voltarlo, comes' usa negli editti; dal' efere cred' io, non ripiegata, come ifr
gli, ma stefa, come una pelle; 0 perché Gi distendeflero tali forte di seriteure a0.
3

in carte ordinarie, ma in pelli, ovvero in cartapecore.

 

” we an ee ae Le Lae eeasS we

STANZALVIL STANZA LVIIL ~~
Fu Paride persona letterata, Ma benche 1a lettura sia fantafticay
Che gid udiato havea più d'un faleero, A un che, si pu dir non fa niente 5
AAa pei, non ne volendo pik fonata, E 0? altro di virtit non ha se
Alla squola fiudi di Prete Pero, Che pelle pelle 1 Alfabeto a mentt,
Pera s' ei non ne intende boccicara Tanto la biascia, strologaye rimafice
EB? da scufarlo; e poi, per dire il vero, C' 4 compito leggendo finalmente —
Lettere, ed armi van di rado unite 41 funto apprende,e fra Lalere sue ciate
Per ¢' han di precedenza eterna ite. Ripone il libroye sprona i le foarpe.
. STANZA LIX, otra
Cos} commina, ¢ a quel Castello arriva, Aa perche' non ¢ tempo cb'
Paffa dentro, lo gira,¢ si fiupysce, Quanto col Tura a Paride
Che quixi non si vede anima vina Con buona gratia vofra Fare pasft »

Perc'aquell ora in casa ognun poltrisce, Per difinir di Piaccanseo la catfte

 
 

» Jee cold

OTTAVO CANTARE; 407
STANZA LX.

 

Che da.quei trifti, com! io diffi ananti Di poi gli feeffi fel cacciaro innanzi
| ( Fatto mentre pappana affegnamento Ginfto come un Villano in fu il giuméto,
 Diinfaccarsi per lor -quei boccon fanti ) Pungolandolo, come un' animale
 Tocco de è pie nell' Crsenal del vento; Fin, che lo spinfer dove ¢ il Generale,

. Deferive le qualita di Paride, ¢ dice, che egli cra letterato, perch? havea let-

@ to pidd' un faltero,, che ¢ quel libricciuolo, contenente alcuni Salmi; che si da a

  

Aeggere a' ragazzi quand' hanno imparato a cono(cer le lettere dell' Abbicc:; E
Tamuiodke, samedi che vm fapeva troppo leggere; ¢ dice, che non ¢
da far meravigiia di questo, perché l'armi, ¢ le lettere mai furon d' accordo,¢
»però egli, che era armigero, cra scufabile, se non era letterato; con tutto cid
'compitando leffe in quel libro, ed intese quel ch' ei doveva fare; ed arrivato al
-Cafamento montuofo trovd che ogauno dormiva. E quil' Autore lascia il parlac
i lui, ¢ torna a parlar di Piaccianteo, che la(cid sopra nel fine del Canto 5. e>
dice, che a-furia di calci ¢ pungolate fu da coloro condotto doy' era il Generale,
© NON ne volendo fapere pin fuonata. Non volendo pii: sentirne discorrere di tare
tuna tal cosa, ¢ qui intende non volendo più studiare.
LA fquola ds Pretepero, \nsegnava dimenticare.
© NON intende boccicata, Non ne intende punto. Non conosce a pena le lette-
“res perché boccicata stimo che venga da abbiccs, ae dica non fa lt Abbicci, che
pt oma, che con i Greci ancor noi diciamo diphabero, el' usa il nostro Poeta,
fente Oxtava 58. Procopio nelia Storia segreta narrando l'ignoranza di
- Giuttino imperadore, che poi si adottd Giuftiniano; dice che egli era Analfubeto,
»tioé sche non fapeva I abbicci; ac scrivere il suo nome.
» PELLE pete. Superficialmente. &' lo stetlo che baccia buccia detto sopra C.

se flans27, b 4
© BIASCIARE. Mafticare senza denti; cioé con la lingua,¢ col palato. Qui
'lntende quello studiare, che fanno i fanciulli, quando imparano a leggere, che

“prima di rilevare, o profferic ja parola, che leggono, la compitano [utte voces,
'con la bocca il medesimo gefto, che fa uno, che biascia; ¢ lo steilo vuoi

dire quel rimaftica, ec, ¢ firoluga intend: cerca d indovinare quel che dica queila

scrittura.

. + Leggere a compito, ¢ quello accoppiar le lettere, ¢ sillabe, che

fannoi fanciuili, quando commciano a unparare a leggere, il che si dice compi-

tare. cio contare a una a una Ie lettere, per poi fommarle, per così dire, in

“ana parola; il che si dice, rileware,

~ CLARPE, Bazzecole, Vedi sopra C. 3. stan. 5.

SPRONAR te fearpe. Detto afato per burlar' uno, che viaggi a picdi

ANIMA vina. Ancor sopra C. 6. stan. 19, si serve di questo detvo aflai usate
“anol, se ben si fa che l'anima fempre vive;e qui vuol dire, che tutti dormivano.

POLT RIRE, Dormire. Vien da Poltro, che vuol dir letto; circa che vedi
forto C, 9. itan. 39.

FA-CIAM panfa, Riposiamoci: 0 fermiamoci. Frafe Latina venuta dal
Greco, usata anco da noi, i quali da Paufa abbiamo fatto Poa, eda Pan/are
'usato pure da' Latini de' tempi bali, 'y/are~

; ae.

 

 
 

  
 
 
   
 
 
  
 
  

408 MALMAONTILE TO
ARSENAL del vento. Ripostiglio deViyento § cioè il
dire una stanza, entro-alla quale &i fabbricanoi Dante laf. <
unle neil Arzana de' Keneviani, » 6 6) es
Mu hoggi si dice:ax/enale, e credo che sia parola corrott. 3
@rx manalis, ia quale origine viene approvata-dal Ferrari.)
PYNGOLARE; Stimolare. Pangolo ¢-quel-baftone con tuna' punta a
@ acciaio in'cima j\del quale si erupnoi congadihi per re
camminino; Lat. fimulus 7B quetto fidicepangalare',: caiy iss%
STANZA LXA vogcl now pS STPPARNg a
etppunto rl Generale a faris' eposto. * 'Cofteroal fine fegli

a

      
      
 
  
 
 
   
     
    
    
   
       
     
  

 

 

Alle minchiace,¢d ¢ cof ridicola. la) Rersdingli del prigion
Jl vederlo ingrugnato, ¢ mal disposo. Ma e¢ possom predicar
Perchigli ¢ fata morta una verzicila, ©. Perch'eglich'e' mele
Le carte ha dato mal, non ha risposto, ©. Oo “Eiperde una gram

E poi-dt non-conrare anco\pericola E gliene duole} ¢ now
Sendoscopertobaner di\pin nna carta yi.. Lornonddrerta,e:

Perch dtrado, quando rubay foatee, 00° Pletofamente fa qucfte si
-Cofloro, che conducevano Piaccianted parrivarono al Ge eye
va ginocando alle Minchiate, ma perchéegli-+haveva fatto-unaln a
perdeva, ¢ però rain colicra, in-vece d afeoitare quel che essi dice
se a dolerfi della Fortuna -; come sentiremo apprefio, D i nonedyy
MINC HI ATE,£ un ginoco assai notodetto anche 7%
Germini » Ma perché ¢-poco usato' fuori della nostra Toscana,o d
mente da quel che uGiamoinoi:, pér intelligenza delle presents Otraves
ceflario faperfi, che il giuco delle Minchiate fifa nella maniera che
E' composto questo ginoco di novantafette carte, delle quali 56. d
racce, € 40. si dicond'Tarocchi', ed una, che si dice i/ matto: 1é carte
in quactro specie, che si dicono femi, che in quattordici sono efigiati De
da Galeonto Marzio diconGi eflere pani antichi contadineschi ) 10 1.4: Coppe
14. Spade, ed in 14. Baftoni, ed in cia(Cuna carta di que(ti fea cominciad
(che si dice afo ) fino a dieci','¢:nell' undecima ¢ figuraro un Bante,
Cavallo, nelia 13. una Regina, € nella 14. un Re, ¢ pucte quelte carte
fuor che i Re si dicon cartacce, Le 40. fidicono Germmis 0 Tarocabiy ©
yoce Tarocchi vuole il Monofino che venga dal Greco: Etaroohi; quah
egli con ' Alciato, demotunrur fodales s1li'y gus cibi causarad dufum-com
quella voce non fo »che sia; fo bene, che Afereroi, ¢ Hetaroiwuol dire
da questa voce diminuita all' usanza: latina si pud-etiere fatto 4
Compagnont, Germini forse da Gemini fegno celette, che ¢-fra Ta
è il maggiore. In queste carte di Tarocchi (onw efigiati diverti
Segni celetti, ¢ ciascuna ha il suo numero da uno fino a 35,,¢0"
no a 40. non hanno numero, ma. si distingue dalla figuea am
maggioranza, che è in questo ordine Strela, Luna, Sole, Afonds eT)
éla idre, ¢ farebbe il numero 4o. L' allegoria ¢,iche siccome le)
vinte di juce dalla Luna,¢ la Luna dal Sole, così il Mondo ¢ maggi
ela Fama figurata colle Trombe, vale piu che ii Moado;

ale
a

  
  

   
    
  
  
   

  
  
  

REF FSSRSETFESTLRR SLE EESS EE

 

 

SRE

“PRSPEREVF eRe SBBEZE
 
 

OTTAVO CANTARE; 409

'I huomo n' è ulcito, vive in esso per fama, quando ha fatte azioni glo-
. li Petrarca similmente ne Trionfi te come un giuoco, perché Amore € fu-
daila Castita, la Castita dalla Morte, la Morte dalla aw > ¢ la Famas
| Diviniva, la quale eternamente regna. Non ¢ numerata ne anche la carta
ma vi è imprefia la figura d' ua AZarro, ¢ questa si confa con ogni carta,e»
ogai auacco, ed ¢ superata, da ogni carta, ma non muor mal, cioè non
i umat nel monte dell' avversario, il quale riceve ia cambio dei detto Maco
altra cartaccia da quello, che detce il aa ¢, ( alla fine del giusco queltv,
'dette il Matto, non ha mai prefo carte all' avversario, conuiene che gii dia
(0, On havendo altra carta da dare im sua vece, ¢ questo è il cafo, nel
si perde ii matto; Di tali Tarocchi aleri si chiamano mobili perché conjano
chi gli ha tn mano vince quei punti, che etli vagliono ) altri ignobili, per=
snon'contand. Nobili ono Ve, due, tre, quattro,e cingue, che la Cartas

%
del Fao conta cingue,¢ l'altre quattro contano tre per ciascuaa. Ui numero 10.
:

   
     
  
  
    
    
  
  
  

43.20. € 28. tino ai 35. inclufive contano cinque per ciascuaa, ¢ l'ultime cingue
| Guutanio dieci per ciascuna,e si chiamano sd, [i Afatro conta cinque, ¢d ogni
Re conta cingute, ¢ ono aacor' eifi fra ie carte nobili, s1 numero 29 non contas
se niin quando ¢ in verzicola, che allora conta cingue, ed una voita meno delle»
'compagae felpetcivamente; Delic dette carte nobyli si formano le Verzicole, che
Ordini, ¢ (egucnze almeno di fre carte uguuli, come tre Re, o quattro Re;
: di tre carte andanti, come xe, due, © tre, quattro, ¢ cingue, 0 compote, co-
i bet 'hie wo y 13.¢28. Vo, matto, ¢ quaranta, che sono le Trombe, Dieci 20. € 30.
tq OVEFO 20 30.¢ go. E queite verzicole vanno moftrate prima, che si cominci il
ginoco,¢ meife ia tavola, il che si dice acenfare la Verzicola, Con tutte le verzi-
i 'evi si confa il matto, ¢ conta doppiamente, o triplicatamente come fanno I'al-
a “tre, che sono in verzicola,la quale efilte senza matto, € non fa mai yerzicola se
i 'non nell' une, matto, ¢ trombe, Di queite carte di verzicola si conta il numero
“che vagliono, tre volte, quando pero l'avversario non ve la guafti ammazando-
a “Vene wna Carta, 0 pir, con carte superiori, che in quelto cafo quelie, che refta-
f 'DO; coMtano due voite, se però non reftano in (eguenza di tre, per efempio: Io
a. “'moltrO 4 principio del giuoco 32. 33. 34.¢ 35. se mi mi muore il 33.0 il 34. cho
os rompone 1a seguenza di cre,la verzicola ¢ guaflata, ¢ quelle, che vi reftano con-
tano foiametice due volte per una, ma se mi muore il 32. 0 il 35. vi refta la se.
Suchza di tre,€ per confeguenza ¢ verzicola, ¢ contano il lor valore tre volte»
ciafeheduna. Mf Atato, come s'é detto, non fa seguenza, ma couta fempre
è 1i (uO valore due voice, 0 tre fecondo, che conta la verzicola 9 guasta, o falua-
a ta} © quando s' ha pill d' una verzicola, con tutte va i Adarto, ma una fol volta
i) conta tre ed il refto conta due; € questo s' intende delle verzicole accufate,/es
mottrace, prima, che si conunci 1! giuoco, perché quelle fatte gon Ie carte am-
" mazzate agh avversarj, come farebbe; se hayendo io il 32. ed il 33. ammazzaii
ai' avversario ti) 31, 0 ii 34. ho fatca la yerzicola, ¢ queita conta duc volte,
Quando ¢ ammazzata alcuna delle carte nobili, ciascuno avversario fegaa a co-
'lav, a cui € thata morca canti fegni, 0 punt:, quanti ae yaleva quella tal carca;
Ccvetto però di quelie, che sono state moftracte in verzicola, delle quali, sendo
Auiazeate non si gaa cola alcuna (. a da quello, che per privil:gio non
3 f giuo-

 

 
     
    
    
    
 
  
   
   
  
    
   
   
 
   
  
   
    
   
    
  

4r0 MALMAN TILES

» giuoca ) perché tali fegni vengono dagli avyerfarj guadagaati
del valore di efla verzicola, che douria contar tre volte,
ed il 29, morendo la verzicola, dove esso eatrava, conta folo cing
carte poi, le quali si dicono carte ignobili, ¢ cartacce non contand
mazzano tal volta le aabili, che coa;ano comei tarocchi dal aunero 6,
amwmazzaa tutti i piccin, cioè l'1.2. 3.4. ¢ 5. dal 14. in fu am nazzano
il tredici, ¢ dal 21. in fa ammazzano anche il 20, ed ogai tarocea
Re ) ma servvno per rigirare i giuoco; il qual giuoco appreilo di noi non
non in quattro persone al più, ed allora si danao 21, Carta per ci
do si giuoca in due, o ia tre, (¢ ne danno 25. E giocandofi in quai
primo che seguita dopo quello, che ha mescolace le carte in fala mano
dice bauer (4 mano] ha la faculta di non giuocare, ¢ paga fegai trenta
che nel giuoco pigita ' uitima carta, ¢ questo che piglia l'ultima |
dice far f' ultima ) guadagna a ciascuno ai gueili, che hanno giuocato
Colui, che non giuoca guadagna ancor' egli de i morti, cioè fegna,
lore della carta a colut,al quale ¢ amimazzata deta carta. Se
giuoca, il fecondo ha la faculta di non giuocare pagando go. fegni,se
ca il 3. ha derta faculea pagando go. fegal, se il 3. giuoca paifa la faci
che paga 60, fegni come sopra. Ma se il giuoco ¢ folamente tn tre
ci questa faculta di non giodare.

Me/colate che sono le carte, quello de i giocatori, che è a mano
guello, che ha mescolato,n' alza una parte,e se v'é volta nel fondo di guel
del mazzo, che gli refta in mano una delle carte aobili, o ua tarocco dal
27. inclufive, 12 piglia, ¢ seguita a pigiiarle fino a che aoa vi trova und
ignobile: Quello, che ha me(colate le carte dopo haverne date a ciale

PSPS PPTs

se stesso dodici la prima girata, ¢ tredici la feconda, ¢ scoperta a %
carta la feuopre anche a se medesimo, ¢ poi guarda quella, che segue,¢! a
se fara carta nobile, o tarocco dal 21,al 27. ¢ seguica a pigliarne come +

gu-sto si dice rubare, ¢ queste carte, che si rubaao,¢ si scuoprono, sendo a
guadagnano a colui,a chi si sCoprono,o che le ruba, tanti fegni,
gliono; ¢ coloro, che le rubano € neceffario, che scartino; ciae si levino:
altrettante carce a loro elezione, quaate ne hanao rubate per ridurre le
al numero adeguato a quello de i compagai; ¢ chi non scarta, o per
dente di carte mal contate, si trova da uitimo con pil carte, o con.
avversarj per pena del suo errore non conta i puati, che vagliono le fu
ma se ne va a monte; Colui, che da le carte, se ne da più, o meno d
flabilito, paga 20. puati a ciascuno degli avversarj, e chi se ne trova
pill, ¢ deve scartare quelle, che ha di più; ma non pud far vacanza
deve rimanere di quel feme,, che egli scarta; Se ne ha meno, la deve
monte a sua elezione, ma senza vederla per di dentro, cioè chieder la qu
o la fefta, ec. di quelle, che sono nel monte, € quello, che mescoid le
si dice far /e carte ) fattele alzare gli da quella, che ha chiefto,
Cominciafi il giuoco dal moftrar le verzicole, che uno ha in mano
mo dopo quello, che ha mescolate le carte in fu la mano deftra,
quna carta, ( il che si dice dare ) quegli altri, che seguono devon dare

 
4

 

OTTAVO CANTARE. 4it

mo feme', se ne hanno; € non ne havendo devono dar tarocco, e quello si dice
nes » E dando del medesimo feme si dice ri/pondere. Chi non risponde,
ed | reece, feme, che è stato meffo in tavola, paga un feflanta punti
ciascuno, ¢ lia carta nobile, che haveffe ammazzato; per efempio il

Toes ai di danari, ed il econdo benché habbia denari in mano, da.un

 

acco sopra il Re, ¢ l ammazza; scoperto di haver in mano denari, rende
acolui dichiera, ¢ paga agli avversarj feflanta punti per ciacuno, come
MY gE deo. Ogni tarccco piglia tutti i emi, ¢ fra lor taroccht il maggior numero

 

 

-pigiia il minore, ed i matto non piglia mai', ¢ non è prefo, se non nel Calo dec
| todi sopra. Così si seguita dando le carte yeu il primo a dare ¢ quello che piglia
; Tecarte date; ed ognuno si fludia di pigliare all' avversario le carte, che conta-
—-HO€ quando s'¢ finito di dare tutte le carte, che s' hanno in mano ciascuno con-
b “tale carte, che ha prefe', ed havendone di pil delle sue 25. fegna a chi l'ha me-
ge MO taati punti, quante sono le carte che ha di più 5 dipoi conta i suoi onort, cod
l'il valore delle carte nobili, e verzicole, che si trova in esse sue carte, ¢ fegoas
all' avversario tanti punti, quanti con li suoi onori conta pitt di efl> 5 ed ogni
F et si mecte da bai.da un fegao, il quale si chiama wn fefanca, € que'ti
Sfane valutavano secondo il concordato. E tanto mi pare che bafti per facili-
tare |' intelligenza delle presenti ottave a chi nonfulle pratico del giwoco delles
Minchiate, che usiamo noi Toscani, che è assai differente da quello 5 che con le
'medesime carte usano quelli dala Liguria, che lo dicono Gsmellini; perch Adin~
¢hiate in quei paefi è parola oscena. Da questo'giuoco vengono molte manicres
dire; come E/sere il matro fra tarecchi; entrare in tutte le verzicole; Efsere le»
Trombe 5 carracce; Contare; nom contare; ¢ simili.
\ INGRVGNATO. In collera. Chi s' adira, 0 entra in collera fuol moftrarlo
con la Mutazione di volto, torcendo la bocca, 0 increspando la fronte, com,
atti simili, che si dice anche far mufo, ¢ far grugno, 0 ingrugnare. Vedi sopra C,
2, Man. 57, Lasca Nov. 10 Ata Beco non la porendo [goxzare fene fRaua ingrugnato
 anki che no, Viceli anche portare tener broncto; imbroncrare,. Nonio Marcelio an~
tico Gramatico. Bronci /unt producto ore, & dentibus prominentibus.
DS AMMAZZ AT A una Verzicola, Ammarzare,rubare,foartare, dar mal les
artenon contare, verzicola, non rispondere y se/santi, ec. lepgi queiche habbiamo
~detto qui sopra alla voce Afinchiare. read
» HVOMO roto. Huomo collerico. Lat. praceps in ira, che si dice ancora in

yb + mn ena huomo precipitolo.

ES

BE

RUEBRERLER SSeS.

,¢ © NON ci puo fear foto, Non la pud foffrire. Lat. /ustinere, pati.
gi | LOR nem da retra', Non bada, 0 non attende a que! che eff dicono. Non da
. Lat. mon faciem accomodat aurem. Dar vetta in altro fenlo dissero gli
“f antichi nelle cose di guerra, per quello che i Latini dissero, imperum /ustinere,
yy » GAGNOLARE. Rammaricarsi. Vedi sopra C. 4. stan. 9.
n.% STANZA LXIIL
| Che t* ho io fatto mai fortuna ria Lucho non si farebbe anch' in Turchia,
/ Chee? bas con me si grand' inmicizia; Lie proprio un'impierade un'inginftizia;
, Mentre tu mi fai perder tuttauia Vedi, non lo negar che tu? kai meco;
5 Che @ nem mi tocca pureadir:Galizia? E poi fen'. aunedrebbe Nanni cieco,
2 Es £2 STAN.

 

ee Low

 
© SSR SS aes eee

a

 

 
  
 
 
 
 
 
 
  
 
 
 
 
   
   
   
   
 
   
   
  
   
  
        

412 MALMANTILE

STANZA LXIV,
eMa, se volubil fei quanto /degnofa
Facciam la pace, manda via lo fdegno;
E [e tu fei de' miferi pietofa, a
Danne,col farmi vincer, qualche fegno, Si 3} 5 ma basta po
2» Fu il vincer fempre mai lodeual cosa y O Baccellacero
a» Vincafi per fortuna 50 per ingerno y

ee FFE CEES SP rsa ererersEas

Percio de' danni miei refhando faria, Capitale\ Sarthe
La Fortuna mi sia non la Disgraria, Se tu nd voi pil

STANZA
E cosh finiran tanti [chiamazri
Dichiamar la Fortuna,ei ginochi inginffi, — Ov'io ritrowo ognor exttit
Che mentre vi ti ficchie vit ammarzi Per forza al ginoco mi ric e
T4 [pendi, e paghi il Boiacherifrupi, Appunto, come il ferre-a cal
1i Generale si duole della Fortuna perché gli ¢ contraria, ¢ lo fa
pre: la prega a volerfi mutare, ed essergli una volta favorevole: © 0
sto C. 15. stan. 1. dice Px i vincere, ec, Ma poi accorgendoli, che il suo
è inutile, riprende se medesimo, del vizio, che ha di giocare, ma conok
l'ammonizioni non (ono abili a farlo defiftere dal giuocare..
WON mi tocca a dir; Galizia, Non ho punto il conte mio. I
de della Galea disse:
E se non ne facean tanto romore
Non saria lor toccato a dir: Galizia 3
Tanta gente » andaua per amore. ae
Ed il Perfiani dolendofi, che un suo fratello cra più lefto, epi aflute
disse;
E prima: 1 mio fratello è una ciuftizia y
Che mi riuede moito bene il pelo,
1 credeu' eljer furbo, e giuro al Ciele
Che feco non mi tocca a dir; Galizia, ket
Da quiefto che dice il Perfiani pud,chi legge,comprendere il vero
sto detto. 3
NON si farebb! ancl' in Turchia, Non si farebbe in luogo veruno 5
foaa del mondo, (¢ ben fulle il maggior nostro nimico, come ¢id Treo
sopra C. 5. stan. 6. i, Suagtod Cane
SEN' avvedrebbe Nanni cieco, Lo conoscerebbe uno, che non havelle
Lo vedrebbe un Cieco,come era Nanni. li Proverbio dice: éome:
cieco, ¢ (enz' altra aggiunta s' intende, vedere 5 perché questo Nanni
va fempre; vedere, Si dice anche femplicemente Vamnicreca., © 8!
defimo. Si dice anche: Le vedrebbe Cimabues vibe ncn ciecos 05\che
>

eed

vt

occhi di panno, detto h » venendo da
tura in Firenze, non perché eghi fuffe cicco » ma & voieva denota
fufle nato al mondo cieco y vive affatto al buio del difegno. 1
THM.

444A che gracchia io? Ma che sto io a ciarlare in vano. @

 
 

OTTAVO CANTARE, 43

re della Cornacchia yo del graccio, quafi Lat. graceutare, Ma ci serve per clpri-
un cicalare senza mento 5 senza frutto, oal vento, Vedi sopraC. 1.
staa, 69. C. 4. stan. 25..¢C, 7. Rao. 59. Ser Brunetto Latini nel Patathio; in quel

-weelo: Adi aific 5 #10 non fo.y ch' aurem cornacclie ? volle dire in gergo; alludendo
eal faono della cornacchia; Che auremo per il ores di domani, Lat.cras,

a

 DISDETT 4. Dilgrazia. Maia fortuna. B' ii contrario di Desta, che vuol
 dir buona forcuaa nei Eee, Oinaltro. Sp. defdicha L, malum fatum,mala fors.
 FINCER la posta. Guadagnare quello, che va in giuoco. Vedi forto in questo

wp ©. stan. 7..¢ vuol dire vincere una volta foia.
— PORRE 4 Caualiere, Rimaner superiore, Caxaliere si chiama quella Torretta,
4 nelic Fortezzeavanza sopra a tutte le muraglic della medcfima fortceza;¢ di

Essere yo fiare a Canalere, vuol dire Efscr superiore, © avanzare il compa-

- gno, Varcit Stor, lib, 9. Zara questa parte delle mura di qua d' Arno non banendo

wile Me monti, ne colli sopraccap!, non puo dal di sopra, (come si dice)a canalicre essere offefa.

rd  BACCELLACC I/O, Scimunito 5 Sciocco; Infeafato. Auguilo Imperadore»
all” diceva bacelus. €

'isp -Lorfe fogna pere, Ognuno Gi figura di goder quel ch' ei vorrebbe, ogauno fo-

: ch'er bramna. Virg. ed. 8. 4% qui amant ipfi fioi somma fingunt. Vedi [0-
 pra C, 2. fla, 7. B per qual caula ti dica / oro, ¢ non altri aaunali, Vedi C. 1.
ca 31. Teocrito ditie; Omnis canis panem fomniat, ec.
| ) @APIT ALE, Quetto termine oltr'a i signincati, che dicemmo sopra C. 7.
Mian, 82, protterito nel modo, che ¢ nel presente uogo, ha la forza del Latino
Fiinam © yuoi dire piaccia a Dio, che non sia per essere,¢ che non segua, in
contrario.

y SCHLAMAZZO. Romore, Strepito. Traslato dalle galline, il gridar delle
quali Gi dice ichiumazzare, Ll Vocaboiitta Bologaele dice, che 1 verbo schia~
-Mbazzare significa Kiciamare io darao, dal Verbo Greco Sciamocheo, che vale

} mare cum umbra, Ma ¢ yvanita; perché schiamazzo vien dal Latino exc/smatio,
V1 ficchi evi  ammazzs, ha queito cafo son quaGi Sinonimi, € figaificano
— immergerti, o applicacti cutto a una cola, A "4
bp PAGE boia che +i frajti, Spend per haver danno. Teognide disse: Sibé

Oe sph vineula cudie.

bp LABRICCINO del Paonazei, Lntende carte da giocare, perché già un tale de'

te Paonazzi fabbricava dewte carte.

APPYNTO come ti ferro 4 calamita, Per fimpatia, come fa la calamita al fer-

i ro; Beeausito detta da Franzeli simaat, cive Pica amante.

df SA ANGA LXVIL STANZA LXVILL

is E Sard. ver, ch' so habbia a star feggetto Datemi dungue un marzo in sula tefia
. | a una cosa, che mi da tormento? Vedere; eccoms qui ch io non mi muaitay

   
 

St

a

Come tormento ? oibo \ s' 10 ci ho dilette.

» St ua intanco per lui vine scontenta.

O per fido giuocaccto | ¢ maledetto
Cin ¢ ha trouate,e me, chetifrequente y

Ne voi farete cosa men che bonefta,
Se dal.giocar, morendo, io mi rimond y
Soc! ogni di farebbe questa fefta,

C! altro dilerto, che giocar mon proKo y

i Tu non cs bai colpa tu, 4 me il gaftiga Ed a ginocare omai fon tanto avenge
a — 2 poicht cou te m' intrig? he'd pentirms non giana £4 Se

 
  
   
  
 

414 MALMANTILE ©
STANZA LXIX.
LD usare ogni fapere, ogni mi
Non vale a far mi cotro al gioco,
Imperocch' io t ha fitto si nell' ofa
C? amos mio mal qual afferato inferno,
E forse giochero dentr' alla fofia,
Che forse? diciam pur:tengo per fermo;
E se trouar le carte ini non pufio y
Fari, ( pur chee si eiecbi} all aliofiso. + I quarti anro,oo'far
Seguita il Generale a lamentarsi, ¢ combattendo in lui la voglia
con la ragione, ¢ con la conucuieuza', prega gli amici 5 che Pai
ché vede, che non c'é altro modo; che egli si rimanga di'giocare}
d' efier certo d' havere a giocare anche dopo morte, ¢ che alla fepoltura
dare con le carte da giocare nel feretro nelia manicra,, che esprime
va 70. *
b7z0" « Questa voce ha diversi significati, perché ce ne serviamo
come nel presente luogo: per dimoftrazione'di naufea, come oii 5
e quefia ? (orto C, 10, tt. 23. per riprenfione, © difapprovazione: Oibe.
cosa,ed esprime il latino Kab, & espace, E gue) che i Greci distero e4ib
ciamo anche: aibo, eibo, ¢ tbo, 4 Oe
SCONTENTO, Scontclato, disguftaro » La \ettera', sy aggiunta
pio di nomi, verbi, ec, ha nel pariar nofire la forza, che apprefio
particelia i» privativa di Circa di che vedi il Varchi nell' Hercola
de alla particella ex.
MAZZO, Quei martellone di legno, che adoprano i. Macellati
la tela a' buoi, donde mazznola queiia, che a Roma adoprano per
i malfattori. Si dice anche magéio, nia quetto € propriamenteq
prano i bottai a cerchiar le botti, Dal Latino malleus. 18
FARE schermo contro al gioco, Difenderfi, 0 riposarti dat non gioeare.
dal verbo schermire, che vuoi dire Eiercitarsi per imparare a difenderhi
il qual viene dal Germano be/chirmen, siccome vuole il Voto. Dan,
O Grscopo dicea da Sant' Andrea, La ie
Che t' è viovato ds me fare schermo? Y
i Petr, Son, 17. Ch' i non fon forte ad aspettar la luce
Di questa donna, ¢ non fo fare sebermo
Di luoghi tenebrofi,e a' hore tarde ? i rue
L' HO fitto nell ofa. Ho wn desiderio di giocare internatissimo;
giovane innamorato difie, Georg. lib. 3. Quid ivvenis magnum cui
ignem Durus amor? Bil Petrarca.: Dee ai
eee

 
 
 
   
   
   
    
  
  
      
   

    
     
   
    
    
  
   
   

E ricercami le midolle yet ofa, © ~ A
AMO il mio mal qual ajerato infermo. Come brama: il febbricitante di bees
che gli ¢ nocivo, così bramo io di giocare, che mi ¢ dannofo, she?
e4ALIOSSO. Come habbiamo detto sopra C, 1. st. 9. tutti li gi i
da i Latini si dicono alea: da che io deduco, che quelta voce Aliso
Latino alea, & of, ¢ significhi, come in efictto significa ofo da gu

   
   

 
 
    
  
 
 
  

   
 
 
  
 
   
  

 

pe

Ste

:

s

it
r|

 

 

OTTAVOCANTARE. is

i l'aftragalo.de i.Greci., Dicefi ancora Catrioffo; quai. gasdy
uct otio bared et gambe didietro di tutti Pen o
on ¢ nell' agnell

 

> Pagnello, bue, ec, che negli animali d' ugna sode, come il
ec, © ditate come il Lione, ec. non si trova, eccetto, che nell' Alicor-
o Pol. Virg, lib. 2, cap. 13.. ¢ Dianel Soutero de Aleatoribus lib, primo
+» Buleng. de lud, Veter. c, 58. ed ¢ un' offetto di figura quadrilunga das
concavo,¢ dall' altra conueflo: Nel mezzo del concavo apparisce un
co,, ed il conuetio., che è la parte opposta al concavo, forma: in cia-
luefiancate duc piccoli buchi; nelle teftate del fianco al concavo,¢
flo | due superficie quafi piana, se non-che in una si vede un fegno come
»¢ hell' altra un fegno come un 8., € queste duc parti quando |' Alioffo si
in tavola sono le più difficili a rimanere scoperte, perché ono di più dificil
del concavo, ¢ del conveffo, ¢ l'altre due fiancate non reftano mai (co,

    

  

f wee perché niuna per la sua rotondita pud posare «<I nostri ragazai dell'infima
 plebe,nel giuocare con quett' offo s' adaitano a quei fegni, servcndofene per nu-

 Miero.con fare il concavo il numero 4nd, il conuctfo farina, cioè naila, per effler
qusito 11 pity facile arimanerescoperto, la parte dove ¢ il fegno 8. vince otto
tiene la figura di quci numero;>¢ da' Greci quello numero, di otto negli
chiamato Srefichora,s¢ la parte dove è il. fegno.S, vinca dodici,) perché
haf 'a quafi di libra, che si divide in 12. parti; 0 secondo, che conuengono,
ado, o variando questo giuoco, fecondo i patti:(B l'usano dettira-
dalla Pafqua di Refurrezione ( nel qual tempo s' ammazzano gli agaclli,
mpe de' quali si trovano.questi offi } fino a che vengono le pelche, ed al-
lato.' Auuofo.,.¢ giuocano ai nocciolisne i modi detti sopra C, 3. st. 37.
qual giuoco durauo.a giuocare, fino a che stiacciati i noccioli vendono l'ani-
me di ef aghi spcaziali., che fara per tutto Octobre in circa  ¢ da quelto tempo
fino a Quaiefima givocano alla ruila, 0 alle buche com la palla di legno nel
Che si difie opra C, 3... 57.; ¢ per tuttaJa Quarefima giocano alla trot-
« E così distribuscono 1 loro trattenimenti per tutto |' anno., Ma tornando
all' Aliofse; appretio agli antichi Romani era usato dagli huomini più fenfati, ed
in diverse maniere; ¢ fra I altre il concavo.era chiamato Cane, 0 canicula forse
da. fiella lucida, che si yede nella bocca del Gane Celefte;, stella cattiva, ¢
malefica; € colui, che tirando faceva apparire detto lato,, posava in tavola due
denari, o guello.che ¢rauo conuenuti fra loro i giocatorl, ed era cattivo, onde
Pecfio dif. Damnofa Canicula quantum Kaderet y la parte oppolta,a, detta era
ohiamata Venus stella benigna, ¢ benelica; ¢ significava il num, fer Latino Serio,
da noi detto Size, nel giuoco dello Sbaraglino, quafi Seino da' Greci chiamato
 Hexites,¢ chi tirando scopriva quelta Venere guadagnava [ei, ¢ tutto ello,
che haveyano polazo in tavola coloro,, che havevano scoperto Cane, 0 Canico-
la, Giulio Poiluce lub. 9. dice 5 che da ipill, il Sei era chiamato C0,.¢ il Cane,
| Ovverol'Aflo; Chio: ¢ che in questo lor talo non havevano, ne il duc, ne il cia-
gue. Con queiio offo giocavano tanto i Greci, quanto 1 Laci in altre maniere,
~ © fino con fei, ¢ oti offi per volta, ma a me balla haver accennata la fuddetta»
“per teflimonio, che anticamente ancora era in uso questo-giuoco; ¢ tralalcio di
harrare J' altre manicre che fon molie, perché non fa a proposito noltro ee

 
 
 
 

SE

 
 
  
 
 
 
 
  
  
     
  
  
  
 
     
  
   

o0ti('(“' Cz AT

se il Lettore ne faffe curioso legga Polid. Verg. lib. 2.
Aleatoribus lib, pr, cap. 29, Buieng, de lud, Vet. Gap ye
rum gen. lib, 3. Cap, 21. Ho decto, che questo Aljotio ogg
zi, ed il nostro Autore ci addita quetla verita, faccndo r
prrché si giochi, all' altofso, Se trovar le carte ivi mon posso; ¢ intend: V
fempre, ¢ f€ non troverd carte,giuocherd all' a/io/so, quantungue
fagazzi, pur ch' io soddisfaccia al viziofo genio, che ho di giocare.

VAN co libri, ec, A' Dottori, quando portati alla fepoltura j

mettere nel feretro,o bara i librised a i Cavalieri la (pada al fiance f

dice, che fara fatto a lui, che per far conolcere, che meiitre ville era |

re, gli faranno una ghirlanda di quei fiort, che sono itmpreifi nelle

vefte fara ricamata di picche, e di cuori, ¢ forto la testa git c

mattoni; ed in questa maniera hawia anch' egii actorno tucei quattro'

sono impretfi nelle carte da giocare a primicra. + a ee
Far sn quarto @ germini, Giocare in quaccro alle minchiace y Vedi fop

aS etek

  

   
 
  
 
     
  
     
    
   
  
  
      
 
      
   
  
  
  

   
 

quetto C. st. 61. 5 00? th Se Ta
STANZA LXXI. STANZA LXX th
Volea seguir, ma tutti della lanza Amoltance ch ¢ buow ai bei
Gii dieron fu la voce con il dire, E por da bene,acor chia a
Che il perdere ¢ comune,e fhar' nfanza, Dt questo suo viocar, don't si he
E perde una miferia ds tre lire, i p ow
Pero si qusets pure,e habbsa speranta Dicendo &° a impiccarie hon, y
C' an giorno la difdetta ha da finire, L! bauer femp.icemienteunpo dm o
'Pero che i tempi variabili sono, Ma quand snch ezti havefse ivan Ga
E dopo il triffo n' ha a venire il buono, Del far la [pia non se we fa pr by
STANZA LXXIl, STANZA Lax th
Intanto gli moftraron il Prigione Ed al prigion preterito imperferto” oy
Che fort' il manto deit lpoerifia Rinolto con le carve im man l'itty \ >
In carwa, dicendo, in divozione Gid fattofele porre a dirimpette &,
Faceva lo scultore, ideft la spia; i giocar a' nna crazia la tay
Berd, perch' im essetro egli è un euidone Ouver si metra fuor in [i a |
L  impicchi s* ei vuoi far opera pi: Vn teftoncino, ¢ sia guerra finitay | fy
Serragli pur, dicean, la gola, ¢ poi y Così lo prega, lo sconginra,e inpatt |
S' ci dice pik nulla, apponlo a noi, Bada pur fempre « mescolar leo
Voleva il Generale contiouare il sao lamento, ma 1 circoftanti lo &
tare confolandolo,¢ moftranuogli, ch' ei si faceva (corgere a far tanto ke ¥
per una perdita disi pochi foidi: Intanto gh prelentarono Piaccianted
it, che lo facefle impiccare, perché eglt era Spia; Ma il Generale buonhio® )

jo fece liberare, dicendo, che un poco d' indizic non era baftance a
care, ed oltre a questo del fara spia non se ne fa ne meno procetllo 5
che se s' haveflero a fare impiccare tutte le spic ci farebbe facceada, |
medesimo Generale inuita Piaccianteo a giocar seco di poco,¢ (olo per'
Nei che il Poeta esprime il vizio internaco di giuocare, che era
ché nello stesso tempo, che determina di non voler mai più gi i
terfi a giocare fino con ua vil prigione, con /' anticta y che muitea Ma

 
   
        
  
  

   

OTTAVO CANTARE. 417
\der fempre a mescolar le carte; come fanno coloro, che punti dal gino-
per haver perduto, vorrebbono pur trovare con chi giocare per ricattarsi.
'LI dieron fu la voce, Lo fecero chetare. Latino.: Vocem alicui comprimere,.
CDE una miferia di tre lire, Perde poco. La voce miferia, che per altro
ifica infelicita, 0 avarizia, usata in questi termini serve per avvilire; ¢ pero
ime qui una somma di niuna considerazione.
i SOTTO a manto d@' Ipocrifia, Sotto feula, (otto pretesto, sotto coperta di far

  
   
    
 

teh

: - BACEVA ta feultore, Cie faceva I alcoltatore, ¢ non lo statuario, ed inten.
de, Stava alla feolta, cio fava ascoltando 1 ditcorfi d' altri per ridirgii; ¢ cons
| termine equivoco viene a dir copertamente Far /a /pia, come dichiara il

medesimo
-G71DONE. Furfante, Huomo d' infima plebe (enza riputazione, Vedi sopra

 
  

Gr, 63. '
AP. a noi, Mlins crimen affinge nobis, See' fa più la spia, gaftiga noi.
4 'Tiathcuriamo » OP entriamo mallevadori, che e' non fara pil la (pia, Elo
[ll fleflo', che mo danno, che vedremo [otto C, 11, st. 49. cioè mio sia it danno, se non
'jail Segue tort, Come iv dico,

 HVO.MO di buona paffa. Huomo di buona natura, Latino Oleo tranquiltior.
i Plauets in Poenuio', dra hunc canem faciam pibi-oleo tranquilliorem, farò stare zitto,

   

| 60m' olio,
yet = — DOP ci fr enasta, Dove egli pecca. Con che egli varia la sua buona natura.
4 ~ DEL far la spia non se ne fa proce(so, Gaftigar uno senza far proceffo vuol dir

iio fommariamente. Latino indica canfsa, 0 più tolto, de plano, cioè
ein nea ita di giudizio, senza federe a banco di ragione, © come si dice an-
4 che volgarmente pro tribunals; ma qui par che voglia due', che le spie noa folo

we non si gaitigano, ma ne anche se ne fa proceffo. %

yal. PRIGION preterito imperfetto. La voce preterito, che fuona paffato, qui vuok

wie dir, che il prigione era dictro al Generale; ¢ 1a voce smperfetro denota Vimperfe-
zione, ¢ vighiaccheria di Piaccianteo.

uli. “
 FN teffoncino. Teflone ¢ una moneta, che vale tre paoli, ¢ da molti in occa-
il fione di ginoco si dice Vm re/toncino, per intendere giochiamo folo un teftone,¢ fis

ai S774 fimta, coe non si giuochi più.;;

BADA a mescolare ve carte, Con questa azione di badare ( cioè continovare ) a
mecolar le carve inuitando colui a giocare esprime, come habbiamo detto, las
ye BFAD vVoglia, che il Generale ha di giocare,
; “STANZA LidY, STANZA LXXVIL
i Queeli che compracerto nun gis cofka, Duraro a battagliar forse tre hore,
£ vede bauerl'hauuta a buon mercato; Poi la levaron quafi che del pari;

4 Li inusto tiene, ¢ regge a ogns poita, Se non ch' il General fu vincitore

'i Ben ch'ei non habbia un bagattino atiato, Di certa po di somma di danari,

= E dice, al più faremo una batospa E perché gli domanda, ¢ fa. Sealpore,:

4 Kuddei mi vincaye vogiia esser pagato, Quei, che gli spefe in cene,e in definari,

“p Li rapa fangue non si pnd cavare Lon bauer ( dice) manco affegnamento
Ne far due cose y perdere, ¢ pagare « Tal ct Amoftance refta al fallimento,

a tele: gre en

 

 
    
   
      
     
        

ge MALMANTILE =

Piaccianteo actetta l'innito, e mefiiGi.a giocare il
@ alquanti denari; ma perché Piancianteo non ne haveva
grit Così fa la Fortuna, quando perleguita un giuocatore
o|amente quando oon vi è modo d ciler pagato
L! HA baunta a buon mercate. Ha (campato un gran perict
non ha havuta quella pena, o gafligo., che egli conolceva c
TIENE L inuto, Accetta |' inuito,¢ s' accorda a giocare,..
REGGE a ogni posta, Posta ( trattaadofi di giuoco ) vuol dir
danaro ? che 1 giuocatori concordano, che corra volta per, eal
si dice inuirare, e reggere 4 ogni posta, 8' intende tenere tutti gl' inuil
BAGATTINO. La quarta parte del quattrino Fiorentino, con altro none
deito Picciolo, Latino We obolum quidem, Voce, € moncta Veneziana,
FAKE una batofta, Combattere, ¢ gueftionare con parole, ec. Latino,
cari, ed habbiamo aucora ii verbo barofare, per combattere prope
ria di Semifonte trattato quarto, Non havendo tanta.geate, che
Terra batoftare, E pili (otto. Hor dt qud, bor di la si baoftafe., j
NON si pus cavar di rapa fangue. Non i pud cavare una cosa di, wee
&. Latino. gum è pumice postalare. Plauto. Nam tu aquam ¢ pumice nun pr

stulas, qui ipfus fitiat. iva
LA levaron quafi del pari, Cis' intende /a scrittura; Non vi corle qa iene,

cio' si vinfe, ¢ si perdé poco. mitidiy
FA fealpore; Fa romore; Contende alzando la voce. 5 Oe

NON hauer manco afegnamento. Non haver danari, ne modo da trovames.
Ela voce manco in questi termini ha la forza del Latino, nec etiam, ome
quidem, che noi pure diciamo, ne pare, ne meno, ne ance. lo credo,

Ce corrotta da ne anco.

REST A al fallimento, Refla con quel credito da non tlguoter mai pt

fallito s' intcnde colui, che non ha denasi, ne aflegnamenti,

FINE DELL' OITTAVO CANTARE,

  

ae

 

 
  
  

 

  
 

: ARGOMENTO
" Ginnti i rinfrescbi, ¢ inusgorito il Campo

 

ie

Qe

a
Corre all' affalto, ¢ segue aspra baruffa;

~ Malmantil quaft ¢ prefo, ond' al ua feampo %,
f Chiama all' accordo, e termina la rufa, [ae
i Chi tratta pik di guerra hor trova inceampo, >

a = Perché nell' allegrezze ognun si tnffa,
5 Faffi in Corte il conuito, ¢ poi, dal vino
. Riscaldati quei Principi, il feltine. Ol
«ERPS EARS Pb Pape Pape ce ere? 7
: Nie
as 48
STANZAI. STANZAILIL
ye Aguerra, ch' in Latino ¢ detta bello Si che e' mi par ben tondo,ed un corriva,
Parbrutta ame in volgar per fei Befane, Chi pus fear bene in casa allezro,e fano,
Non cr altro se e comincia quel bordello E lascia il proprio per ? appeliativo
Di quell' artigtierie, che fom mal fane, Cercando miglior pan, che que! di grano,
| Eche enon v'é da metter' in caffello; Cen' un' altra ancorch'io non arrixe 5
E Slenti poi per altro com' un cane Ch' ¢ quell'afsalir un con Parmi in mano,
Sere' un quattrino,e pien di vitupero Che non fol non m' ha fatto viliania,
( Ditelo vei, se questo ¢ un bel meftiero, Ma, che mai viddi in vifo in vita mia,
STANZAIL  STANZA IV,
E pur la gente corre,¢ vi s' accampa Florsit cerchi chi vuot bartagliae rifse,
- Ognit per farfiua'huomo,e acquiftar gradi, E si chiarifeaye prow: un po le chiare,
~ Quafi degli buomms cola sia la frampa Che s' io. credeffi farmi un altro Viifse
Mentr! il canarne l'ofsa avvien aradi, L'armi,percioné m'hano ainzapognare,
LA gli buomin si disfanno,e chi ne feapa Ognuno ha il [uo capriccio, come difse
#14 tirato diciatto con tre dadi, Quel Lanzoghe volea farft impiccare,
E pria ch' ei ginnga a efser Caporale 'Pero mi quiero, ma perc' bora brama
CHangierd certo, piu d'un fraiodi fale, Atoftrarus il vero;attenti,e cominciamo,

,Per introduzione de! presente Cantare, nel quale il Poeta vuol deferiver | af.
Ito dato. a Afalmantie, si serve della dimoftrazione, che la guerra (ia una brut.

ta cosa, e che pero habbiano poco giudizio coloro, che vt vauno; perché se be»
nei Latini la chi 0 Bello ( il che fecondo alcunt facevano per aatifrali, cig'

Gee 2 pec

 
   
 
  
  
 
     
   
   
    
     
    
  
    
  
   
 
   
 
  
 
     
 
 
     
 

420 MALMANTILE

per una figura di parlare contraria a » che s'intende, c
bosco, che € senza luce; Parce le, che memine proctnt
guerra, che non ha in se cosa aleuna di bello, egli nondimeno
tissima, e ripiena di pericoli, come farebbe a dire i colpi 3
abbondante di patimenti, ¢ stenti come farebbe il non haver, che
non haver mai denari; onde un Poeta per ispicgar la-bructezga di
Lelia horrida bella, Oltre a questo @ contro alle ragioni della
gnar I armi a danno di chi mai ci fece ingiuria alcuna, disse un G
lum a beluis dicirur, perché & cosa darbeftie, Si maraviglia pero
vada volentieri ingannata dalla speranza, che in quella si face
¢ non s' accorgono, che più tofto vi Gi disfanno, ¢ quand' anche g
ci vuol degli anni prima, che uno confeguisca i minori gradi della o
la guerra Vx fol ne premia, € un million ne ammiazra. Conchivde p
vo di giudizio colui, che potendo stare a casa sua con ogni commodo,
trigarsi con la guerra, ¢ che quanto a se, quand' anche fufle certo d
ventare il maggior' huomo del mondo, non si lascera mai lufingare da
ranze: Ma perché egli fa, che ognuno pud far di se a suo modo, sosp
scorrer pil de i mali, che nascono dalla guerra, ¢ s' accinge a;
con deferivere l'affalto dato a Malmantile dall' efercito di Baldone.
IN volgare, Cioè a parlare chiaro, fuor di gramatica. '
BRVTT A per jei befane, Befane come dicemmo si C..8, st. 30. vu
Panioccio fatto di cenci,e di qui per Befana intendiamo non folamente
na brutta, e mal fatta; Ma le Balie si servono della voce befana per i
una di quelle Larue, che nuocono a i bambini, come il Baw, er.;'¢ gli p
no, che ci sia la Befana cattiva,¢ la buona, ¢ che venga nellecale perk
del cammino del focolare, ¢ però la notte avanti al giorno dell' Bpifania,
Gio, Villani lib. 7, ¢ I nostro popolo anc' oggi chiama Befania, onde;
mente vien questo nome di Befana, come s'& detto sopra, fanno che i
appicchino le calze a i cammini, perché le dette Befane gliel' empiano di
buona, 0 cattiva, fecondo, che essi sono stati 6 buoni, o cattivi ze tali
buone, o eattive si figurano fempre brutte; onde bratro per fei befane vuol dit
eftremamente brutto. J Filofofi icolaftici per esprimer più 1a, che i
dicono M2 «fo, dando alle qualita gradi fino in otto, e volgarmente per elprimt
lo Reflo si dice Sei, come di fei corre, ec, se bene ¢ un termine, che ha
furbelco, Cscala per fei putte, ¢ simili. Ll Ferrari cavando la definizione
na dal Politi Aucor Sanefe la descrive così: Larwale fimulacrum, ——
nia puerss terriculamentum Suspenditur; unde nomen invenit, B foggiunfe w
mulieres deformes Befane dicuntur larua illa turpiores. Dice finalmente, che i Frar
cel dicono T#phanie dal Greco Tbeophama, cioè Apparizione.d' iddio.
nocte danno ad intendere le superftiziot(e,¢ ignoranti femmine a' femplici
li, che seguano molte cose fuor dell' ordine della natura., miracoloic,
per esser la vigilia della fefta de' Magi, né sanno, che con questo nt ¢
Perfiani, ond' ebbe origine, eran chiamati i Savi, ¢ intendenti
Natura, delle Stelle, e de! Ciclo. ia 3 NM

    

  

at

EPS &

eee FF FaTRRSEs

oe: esx EB >is

 

WEL bordello, La voce bordello, che propriamente vuol ie i igo

 
 

 
    
  

NONO CANTARE. qat

blico dove abitano:le meretrici., ¢ prefa da noi in più fenfi, come per frepito 5 0

per una cosa flucchevole:,¢ noiofa, come è prefa nel presente Juogo, ¢ altri la

iglian inteoder Difficulta, o fatica »comela prefe il Lalli nella sua Ea.tr.

le paroled: Verg: Hoc opus bic labor,:

enn Ene aio bello 5

8 et Cafacalda si va prefto prefto;

}) gameboeene| 2-1 | Ma vitornar in fu, quefioe il bordello,

aa "0 è da mettere in Castelo. Specie di pariar Janadattico, del quale par-

2 a. sopra C, 1. st, 29. alla voce /eminato, es' intende Non v' è da mettere in
~,

   

» che significa poi; non v' ¢ reba da mertere sm corpo, cioè non y'é da man-

'. In furbesco; Non v'é da smorfire; Non-v' ¢ da empiere il fuflo, che così

, dicefiil corpo nello stesso modo, che in Greco volgare si dice Cormi da literale
a Corner, che vuol dire Fuffo,o Ceppo, Latino /ipes, candex.

| ~ STENT A come uncane, Patilci, ed hai careftia delle cose neceflarie.al vivere.

: eo della Caccia lib. 5. Ergo age duro dffuescant vittu catuli, Si dice frentar

bracco, quando uno per la sua poverta ha male il modo di provvederfi il

we

ie mee

-PIENO ai vitupero. Pieno di pidocchi, rogna, ed altre tattere, ¢ porcheries
4 i indivitbil della soldzvefea yi chet dice anche: Pieno “4s Bobbio, dal

— Latinovepprobrinm, ebbrobrio ) ¢ Peno di faftidio; del refto Vitupere significa infamia

bye vergegna, Bocc, Nov. 63.
in ET.. Abi vitupero del guasto mondo
ptt T] medesimo Boccaccio nella Teleide lib. 1.

BOs Abi vitupero delia gente Achiva,

ee Omero,¢ Epimenide citato da', Paolo diticro in questo fenfo mala probra, cio'
id vituperosi.
o Per farff haomo. Per diventar un' huomo valorofo: Che e/sere un huomo, 6
sit an'huome, serve apprelso di noi per intender quello, che intendeva Diogene,
of ES diceva 1 Aominem quero, diccfi Efler un' huomo Givven. f wis efe aliquis,
ie scrittara Confortamini, & essere robuffi, Omero, Viri effore, & forte cor fumite,
af VCHivescampa. Scampare vuol dire fuggire, scappare, © liberarsi da un peri-
PI colo + € qui intende chi eicé vivo, o avanza alla guerra, Scampare 5 quali u/cire
J dal.campo; dalla battaglia. !
| © MA twraterdiciotto con tre dadi. Ha havuto la maggior fortuna, che si posla,
haere y/perché il cum, 18, ¢)il maggiore, che si potia fare con tre adi. 1.Gre~
J cl pure ond eae dicevano: Ter /ex iattare, come si ricava da Giulio
| Pollucesnell? Onomattico. Sy aah its
Bi CAPORALE. Capo di fquadra, che fra gli Vfiziali ¢ il minor grado, che si
j dia nella milizia, Caporati differ gli antichi per Principale,Latino Capitalis. Gio,
Villani 1. 28. parlando di Roma dice:

ee Fu caporale regno di se medesima
— Biib. 12.89. eA tutte le caporali Citia a' tralia.
La voce è formata dail' antico plurale 'Capora', come Campora, Borgora', es

simili. °°:
- MANGERA pri: duno fraio di fale, Significa-confumesa molto tempo, perch
x molto

 

 
—

  
    
  
   
 
    
   
   
 
    
   
   

4uz MALMANTILE™
molto tempo ci vuole aun' huomo folo a confumare 'uno f

chi, quando volevano significare-un 'tempo lungo; dicevano com
che sled da mangiare a d' un -moggio di sale, Cicerone de Ami
que illud eff, quod vulgo dicitur wultos modior 'falis fimul edendos efse
nus expletiim fit, Questamaniera proverbiale pure in. pro
usata da Piurarco nel libro della multiplicita degli amici » Si pud
che inghiottira pil d' un boccone amaro 5: ¢ di poco suo:guflo.
con troppo fale si dice amara; ¢ pero mangiando molto faleman
amaro. ' ' +.
TONDO, e-corrivo.. Si poslon dir-finonimi; ¢ il primo signific
fo, ed inGpido, ed il fecondo 5 che:si dice\anche: Corribo., huomo leg
cile a creder' ogni cola. Latino credu/xs:.. 1\Napolerani dicono ¢
minchionare, burlare.,.¢ dar. pasto'a uno; sopra-C..6..f, 80, disse.«
tondi pik dell' O di Giotto, chesuona loyftefla, Tonsa fimiimente: pre
}i vale balordo, dappoco., femplice, goffoxCunto degli cunti?
Bue.:
LASCIAR il proprio per  appellative. Maniera di dire tratta dalla
in cui si danno nomi di due forte, alcuni chiamati propri, aleri appell
dire; Lasciar il certo per It incerto, Bar come il Cand' Biopo ci
che haveva in bocca,per pigliar quella,della quale yedeva,lo shattimento 4
qua, che gli pareva magguore, ¢ Jo stefto significato ha; Cercarmmighor
grano, Eliodo Poeta Greco: Folle ¢ colui, che lascia andar le cose facili
¢ con certa speme segue le pin difficili, ¢ lomrane..\4\ pene
10 non arrino, Cioè lo noa comprendo + lo non arrivo col mio giudizio a it
tendere. In lingua furbesca.. foo» ammasco s non redo  cive non piglio, nonae
zanno, non comprendo. Lat. non affeguor. iru

ESPs SGP SSE

ee ee rset

 

VILLANIA, [ngiuria,Soprufo y mal termine s LG

S? io credeffi farmi un nuouo Viiffe ec, 8' io credefGi di diventare il maggior hut
mo del mondo. Diciamo Va nuoxe Orlando. L Greci Alter Hercules, 'gh

SI chiarisca col pronar le chiare, S' accerti di que(ta cosa con provare le feri
perché chiara intendiamo quell' albume deli' uova, i quale s' adopra a medicit
le ferite, vedi sopra C, 1. stan. 60. ed il Poeta servendofi del verbo ehrarive che
vuol dire (caponire, o (gannare,€ della voce chiare fa nascere lo (cherad.
 INZAMPOGNARE., Ingannar con infinghe. Lat, Verba dare:ed ¢ 10 hielo
che iatinocchiare detto sopra C, 7, stan. 14. Dalla natura del fuono, e della Me
fica.; incancatrice delle meoti degli nomin.. Fra tutti gli trumenti però. que d
fiato, levano pili di (efto, ¢ pare, che percuotano l'anima più gagit ¢
onde furono, ad esclufione degli altri,usati nelle battaglie, nelle quali facevad
meitieri tor via da cuori |' appreafione del pericolo., ¢ infonderni la a
speranza. Noi habbiamo un Proverbio. Far come i Pifferi di montagna (|
nator di piffero strumento di fiato contadine(co.) che andarono per piferare
rono pifferats. Volcano minchionare gli altri coi, darne, ¢. furonc.
col toccaine. Fare uno cornamu/a apprefio i\ Puici, ¢'] Burchiclioe
inzampognare verbo facto da stampogna strumento di fiato rufticale, così a
Symphonia, della qual yoce servcndofi Daniello al cap. 3. neil' Litoria

 
   
  
 

cn x aie

 
 

 ciulli 5

 

   

= a

ae

Ss

S=etis &. BEx es a i EO

, NONO CANTARE.

43

¢ narrando che efi non attefero-punto il cenno., che per comando Regio
si dava, @ adorare la Statua, col fuono di tromba, di cevera, di finfonia:, ¢ di

ees ae fuoni; sg si pud dire [ fiami lecito qui dt servicmi. di questa baga ma-
inzampognare,come git altri. Tromper in Vranz,

=» ¢ pur dal Latino carmina,

we

be Dis LEAN ZAWV..
2 aurora ye come diligente
azza le stelle in Cielo, ¢fapulito,
eae ffi alla fineftra d! Oriente
Evora l orinal del suo Marito,
. Ma perché il Carretton ricco, e lucente

. Acciocch'ei non la vegga/cociase/ciarta,
eee: amegerneved ei iirimpiatta.
: STANZA:
Quande il vutto easone ' (rinfresco '
St che,chi hauea col mafticar dinieto,
pe oma iecamente il corpo al desco y
E come si /uot dir ) riebbe il pero;
ae Hi General, che tutta notte al fresco
nda con? Afirolabio innanzi,e indreto,
Batrendo la Diana in ful lunario
Hanea fatto di Stelle un calendario,

~ Edi nostro Autore dice =

 

. Gid muone il Sole,ed ella U' ha sentiza,.

as = forse a corno, 0 tromba de' ciurmatori: E Charmer Ancantayes >

UGNFNO ha il suo capriccio. Virg. Quifque fuas patimur manes, Ogauno ha je
: fantalic, Vo Lanzo, essendo riprelo, perché faceva cose da efler impiccato,
ve Che folerce tire » lasciatte far a ie 5 percht ho ancor ie mie pelle capricce, BE chi
ha Lanzw, Vedi (opra C, 1. stan. 52. ¢C. 4. stan, 36.

STANZA VII.
seaienaat era anch' egli riuedere
Tutto quanto aggrez.rato al pappalecco,
Done per hauer meglio it suo doxere
Fece in principioun bel murare afecca:
Quando fu pieno,al fin chiefe da bere,
E poi ch' egti hebbe in molle polto ilbecco:
Fighnoli, 3 4iffe, omai venutat l'hora,
Ch' e' si tratta d' hanerla acauar fuora.
STANZA VIIiL.

S' a mensa ognun di voi tanto s' affolta,
Atangia per quattro,e bewerpoi per fecte 5
Che par proprio che sia giuntoa ricolta,
Anrich'egls bablia afar le fuevendette,
Tat ch' io pensai vedern' anc' una volta
La tonaglia ingoiare,¢ le faluette,
Ed bebbi un tratto anche di me paura,
Per una spalla dauola ficura,

“I nostro Poeca de(crivendo la levata del Sole imita Daate nel Purg, C.2.dove,
descrivendo anch' egli il parcir dell” Aurora dice:

65 bid Sicche le bianche,¢ le vermiglie.guance
La doue io era, dela bella Aurora,
Per troppa etade dieninan rance.

Accio ch' ei non la vegea feoncia ye [ciatta,
Manda git impannata,¢ si rimpiatta.

Bd intendono Vaoy et Alero, che quei colore, 1: quale appariva nell' Oriz-
lente per caula deil' Aurora, era quai (parito; ed in iu queit' hora comparue la
munizione da bocca, edi soldati i rinfrescarono. Dopo di che 1) Generale det-
'We principio a far 1' orazione per inanimire i Soldau j quaic Orazione militace si
Soutiene nelle presenti stanze fettima, ¢ ottava, ¢ nelle quaccro segueati.

ere de fielle in Cielo, ¢ fa pulico,. L' Aurora coi ivy ipicndore, offuscas

quelio

 
 
  
 
 
   
  
     
   
 
  
   
   
  
  
  
   
    

  

4z4 MALMANTILE ~

quello delle Stelle, € così le leva dai Cielo ¢ lo fgombra;
VOT AP orinas del uo Mdarito, Cioè del vecchio Titone favol
la Aurora, Virg, Tithont crovenm tinquens Aurora cubile. D
cubina di Titon antico gid s* imbiancaua al balzo a' Oriente Puor delle
dolce amico, Qui pero descrive Aurora nei suo primo app
la parola # imbiancana. Li noitro Poeta poi-, per 'Vorsmale ¢ J
tende quella rugiada, la quale caica sopr' alla terra cicca'l apparic
la qual' hora l'Alba, 0 Aurora si perde 5 pero dice Adanda gin 0 impan
rimpiatta, cioè ferra le tinciire, es' asconde J “
SCONCLA, ¢ feiatta. Si potion dir Sinonimi.. Se bene /oon cia
mente dire una Donna, che non si sia ancora accomodata icapelli
quale accomodaiento di capelit dice Accunciatu ase feiatea vaold
scompotta, e che habbta gu abiti male adactati,y¢ agguitats
sconcio ¢ pil generica, che nome la voce /sarto 5 core:
tine. Znconcinnus, inbonestus, wdecens, incompositus',
1M? ANNAT A, Così chiawiiamo queiteiat di legao sportellatt
tono alle fineilre per chiuderle con carta, tela 50 vetsr, che vi si
fenderG dai treddo,o dal Soe 5 & mandar git  émpannaca vuoldir se
tclio di gueito telaio, ¢ chiuder la fineitra; perché per lo pile deceit:
aggiu(tati in manicra, che per aprire, ¢ Chiudére s'\ alzano, ed. abbul
diciamo tar fu, ¢ manaar gin. 6
SJ rimpiatta, S? a(conde. Vedi sopra C, 7. stam. 66.
HAVE A col mafwar dimcto, A chi cra vietato i mangiare t
havevano ) traslato da 1 Magittract di Firenze y Re' quaii ti dice baxer
non poter confeguirgli, ¢ aver proibizione per quaiche tempo di et
jut, che v' habbra parenu, oche gi habbia efereman di corto, Oo) per  @
givni ttabilite dalle Jeggi. Dan. Purg, C. 14. one
Lav' ¢ meffier ds conforto Diniero, asthe
Negli Statuti Fiorentini diceti barbaramecate Dewerum ou itl
LIET AMENT E, Vuol dire-Ailegramente da lito; se bene i noltti Contile
pi dicono /eramenre in vece di prettamente; ¢ forse qui i Autore Jo Cee
fio tenlo; perché si pud credere, che 1 soldatis' accoftafiero & mangiare:
gramente, ¢ preftamente. Li Lat edacer donde ¢ venutu il Poscano Allegri s*
1 Frangele Alaigre ( che pil mostra la iua origine ) vale pronto, H
E /efo per avventura puo eiler fatto da servs ae
AP POGGLARE il corpo al desco. Si dice anche di chi rifeuote danari 0 prove
fione da banco, 0 Juogo pubbico. Cie accoltarti alla menia per mangiare.
RIEBBE wf pero, Svritociiia + Ripreie forza sok pero quello tia, vedi
6. ttan, 107. Del riavere i) peto vedi wna curiosa noveilettain Giovannt:
te,derto Gioviano Poatano,ne! Diaiogo iniacolato earenio p
cipio. Del maic che:fa al vento caccaiuly, © del beue, che neit:
cice 5 se ne legge un'epig) Greco di Nv » melita 3 1
dire Fiorita Kaccolta de' medetunt bpigramun 51 quaic cradon ave
fuona così. Peditus occadst muitos incinjus in aluo; Lipiojts batoo,
Seriat y@ occidie rurfum si peditus; ergo Regibus auguftis quis

 

  

Fz ER THELIST.

Be

eee 2 baw ere

  
  
  
  
   
   

 
 

cue NONOCANTARE ~- — 45

BATTENDO 14 Diana in ful lunario, Tremando dal freddo per essere thao
all' aria a considerar le stelle. Batrer la Diana, Vuol dir battere il tamburo all'
pparir del giorno, quando si vede la Stella mattutina, ovvero Stella Diana, cioé

del di. Ma per mecafora intendiamo battere i denti per il freddo, che di- ae

mo anche barter la bora, Vedi sopra C. 8. fan. 6, >, a

 TVTTO aggrexzato, Intirizzato per il freddo; Affiderato; Agghiacciato;;

sghiadato; morto di freddo. sggrinzato truovafi nell' antico per secco, es
liato di carne, quali fogliono reftare i morti ( appellati percid da Greci /i-

res, ci0é privi d' umidore, fecondo che vuole Pjutarco nel libro intitolaro 'J
inal sia de' due pitt profitrenole; ! acqua, 0 pure i fuoco,¢ quali si veggono cffere

 

is mie structe, (munte, ¢ fecche. da Aggrinzaro forte ¢ nato Aggrizzato, ps
| PAPPALECCO, Antende al mangiamento in generale: che per altro Pappa-

 decco se - leccornia, ghiottornia ( Franzcle; friandife ) come habbiamo veduto

1C.7. stan. 55.
i hes Os niece il suo donere, ec, Moftra che il Generale, essendo affamato,
yi aifolratle anch' egli a mangiare, acciocché gli toccaffe la sua parte; intenden-
j ' do che mangio aflai prima di bere Tee murare a fecco, vuol dir murare senza
eaicina 0 alcro bicume, ma con i foii safi, ¢ trateandofi di mangtare vuol cir
jot Mangiare senza bere. Neil' antico facevano la parte a mangiare, ¢ a cia~
feheduno toccava ja sua; il Juffo poi levd questa usanza; dice Plucarco nelle Que-
 stioni Conviviali lib, 2. g. 10.
; MESSE il becco in molie, Vuol dit bere, pigliandofi la voce becco, che vuol dir
re il rofteo degli uccellr, per la bocea deli huomo, Queito detto merrer il becco in
molle Gguinca auche parlare, aprir 1a bocca. Gli Spagauoli la faccia dell' humo

dicot roffro da quella degli uccelli.
i 'S' afolta'. S? atfacica con furia, ¢ con vehemenza.
im STA Gitmo 4 ricotra, Cioè ch' e' G sia nell' abbondanza maggiore, come si fup-
pone che e' si sia nel tempo, che si faono le raccolte: Se forse nua voletim» dire,
che coftoro mangiando facevano uno sparecchiare simile a quello, che tanao co-
loro che fegano 1 grano, ec.

PAR cbt egli habbia a far le sue vendette. Quand' altri mangia,¢ beve aflai,o
fa quaififia operazione fen' iatermiiione, riposo, o rispiarmo, ci serviano di
queito'detto, affomigliando quel tale a uno, che per vendicarsi portato dail' ira
Opert veementemente.

PER una spalla davola ficura.M'era entrato così gran timore, che non mangiaflero
anche me, che d'accordo havrei daca una delle mie spalle per confecuarim: 1 ceito,

STANZA IX. STANZA xX.
Redeamus ad rem; Se ( come ho detto') Che quafi fui per dar nelle girelle,
Qua fufte al ber infer mie al magiar fani, Perché dopo ch' i punti della Luna

  

 

Eco+ coltelli sn man, (Pandoui a petto y Hebbs deferitti, ¢ che extse'le Helle
| Runfeiste si brani (parapant, fic Haneuo rafsegnate ad una ad una
bli battaglia vederui ancora aspetto Trouo [marrite bauer le Gallinelle:
| Con la spada così menar le mant y Ma dopo è, ch' io mi dauo alla fortund,
Ona ib aimico vino, ed abbartuto Che fra le elle fiffe, efral' erranti,
NNe sia, come franotre ho preveduto « Won vedenone anche i Mercatanti,
VR Hhh <2 Ska

CRRREBALERE EMASE.

 

=

 
 

 

 

      
 
  
   
   
 
    
 
  
 
     
    
 
    
   
     
       
 
  
 
 
  
   
  
  
     
   
  

26 MALMANTILE

STANZA XI.

M€a diffi poi da me, che poco importa
Se quel branco di Polli non si troua
Ani che questo a noi risparrio apporta y
Peroche magian molto,e non fann' nova;

E [e ne anche alcuna Stella ho scorta
De! Mercatanti, gui creder mi gioua,
Che e'fieno in fierayo vero al lor viaggio,
Per laViaLatrea a mercatar formaggio, Essi cerchin la roba, ¢ mo
Seguita il Generale la sua orazione militare, con la quale dopo hai
suoi Soldati di bravi nella maniera, che si vede, termina suo
che si vada ad affaitare il nimico, perché spera y che fieno per h;
tuna per le ragioni, che dice, con le quali da un poco di bur! ara
FVSTE al bere infermi, al mangiar fani, Bevelte, ¢ mang te aflai,
gi' Intermi per lo pil vorrcbbono fempre bere, ed i fani mangiano
caflai.:
ST-ANDOV1 a petio co' coltelli in mano. Par che voglia dire,
fronte per far alle coltellate » ed intende, che flayano a menfa uno
altro co' coltelli in mano per tagliar pane, ¢c,, ec.
SPAR AP ANT, Così diciamo per derifione a un bravazzone, ¢ qui ton
ne, perché questi soldati mangiavano gran quantita di pane, 4 '
PIÙ per dar nelle gireile. Fui per dare la volta al cerucllo. Vedi sopraC.t.
GALLINELLE, Quelle sette Stelle, che si veggono fra il Tauro, ef
dette Pleradi; in Lat. Vergilie, Il comento d' Arato Latino. Pleiades 4 plartits
te Graci vocant, Latini eo guod Vere exoriantur Vergilias dunt. Aicum dil
Pleiades fieno nominati, quafi Plefiades cio che si Ranno accoflo,per.
ci le chiamaton anche B try, cioé Grappol d' uva,¢ noi Galinelley p
piccole,¢ in un mucchio. Lt Vberti nel Dittamondo.
Poi disse: guarda nella frome a quelle y
Le qua' da' fani 'Pliadi [on dette,
E che i volear le chiaman Gallinelle, 4
AU! dauo alla fortuna, Mi tribolayo: Mi disperavo: Si dice an
alle freghe, al diauolo, alla versiera, alle bertucce, a' cani ¢ simili,
fortuna: tratto per avventura, da' Marinari, quando disperati, ab
in braccio alla borra(ca; 1a quale da' nofiri Toscani fortuna di mare 5¢
folutamente vien detta. Il Petrarca s' era dato in un certo, modo alla
quando,descrivendo il suo stato infelice diceva. a wi
Fra si contrari venti in frale barca.
Ui trouo in alto mar senza gouerno,.
E poi. Ch' ia mede/mo non fo quel ch' io
MERE AT ANT1. Le tre stelle del cingold @
Tauro, così dette perché sono infeme, ¢ paion compagae,
ragione. Adercatante dicevano gli antichi quel che noi. oggi p
-reante. L' arte de' Mercatanti nella nostra Città ancora al,
servato l'antico nome.,: % '

SREERGERE

2RERSES

ee ae

  
 
 
   
  
      
 
   
  

 
  

NONO CANTARE. 27

 BRANCO 4i polli.Latende le Gallinelle dette di sopra.ll Ferrarialla voce Branca
dice in fondo: Branco eréam pro grege.Vin branco-di pecore.Vaa mano di pecore.
Mon n pro mulritudine, ec, Manus autem eff branca, ut alibi anumaduerfurm,
REDER mi vious che fien per la Via Latcea, ec. Scherzando con queiti aoint di
clot Gallinelle, ¢ Mercatanti discorce di esse, come se quelle fudero gaiiine,
che fon difatili,perché mangiano, ¢ non fanno uova,¢ che quetti Mer-
i non eran nel Cicio, percné erano andati a provvederd di formaggio
Via Lactea y 1a quale egii fuppoae di latte, ¢ che pero vi sia il formaggio a
Mercato; ¢ conchiade, che ancor questi sono difutili, perché fond intenti
ente a' guadagni, ¢ aon si curano di gloria di guerre; ¢ pero che ¢ bene, che
. questi non Gi trovino ia Cielo, perché torna a ior favore, ¢ pero si poilas
8 “ entrar' in guerra con buono augurio. Ridicole confeguenze altrologiche, con le
'quali mottra la poca stima, che egli fa dell' Aftrologia come di cosa frivola,e vana,
— Fra laren, 8 quel circolo bianco, che divide da una parte all' altra l'Oriz-
"-zonté, edi nose i vede 1m Ciclo la meta, il quale dicono tia formato di miaucil
fime fielle; Da molti è cniamato /a va Romana, Dan. nel Parad, C. 14. la chia-
| m0 Galafia, dalla voce Greca, colla quale queito yalibul cercnio del Ciclo si caia-
Ma Galaxsas, cive laccco,
| Come distinta da minori in maggi
ee Lum biancheggia tras pols del mondo,
a Galafiass, che fa dubbiar ben Sages,
SON boti; Son huoauni di geflo, ¢ di Aucco; che s'intende huomini buo-
ot at ia yilolidi; Lat. frpites, caudices. Vedi sopra C. 4. tan. 17. ¢ sotto C.
ws Tt, fap, 41. Similicudine tratta da quell' immagini, che appicca nelic Chicle chi
ge 8 botato. In ispagayoio Sore ¢ (puatato, che ha il cagho morto, Lat, hebes,
age tt Oftde boro de ingenso vale huomo d' ingegao poco vivace; ouylo.
se | DANNO te ferste con (a penna, Cioè terilcono sella borla, quando scrivono
Te partite in debico a uao. EB verameute le partite in debita sono ferite, perché
GidiceL denars sono it fecondo fangue, i) quale con tali ferite si cava d' addosso al
Proilimo, Così i dice volgarmcnte Tarare ana frecesa, calui, che chiede a uo' al-
tro in danari,vedi topra ( 2.¢ insdguinarti chi comincia a toccar guattrini,
sh) Dl dar foro, Deve dare, coe divicae lor dzbitore, ¢ per l'equivoco inten-
de deve Perquocergli; ¢ da cio cava la coalegueuza, che noa fiea buont per las
Suerra, poiche se cia piantaav una partica ( snteadi dispongono una parte, una
# quaama di Soidati Jogauno gh dee dare (taccadi perquocere tali Soldati ) es
j gueilt che da tutti ac coccane, boo fon buoui per la guerra, Psancare wna par-
Ma Cinferire, o descrivere nel Giorudle, 0 uubro di uegozio uaa parte, 0 arcico-
lo, capo di (crittura, che da dcbuo, € credito a chi s' alpetia; 1 che si dices
anche decendere una parsita y decendere uno debore ye creanwe, toric dal Latino
recerfere, deiccivere y regiltrare.
STANZA XIIL

     
   

   
 
   
 
  
  
    
  

 

| Nun prima fabili l'andare in GMErTA y Com un bratcod uccelliil quale in terra
Che vede/ts pie prefto ch' 10 nol dico Sts calato a beccar grano, 0 paniva;
Vitleuaiena, «ur trattoyun ferra ferray Va che si muons basta, che quct folo
Ed ir correnas contr' Alil inimne. £4 fuoice pyuare a tats nw volo,
è: Hhh z STAN:

 

zat,

 

 
 

 
 
 
 
 
 
 
 
 
 
 
 
 
  
  
   
 
 
  
     
     
    
   
 
   
   
   
    
  
 
 
  
 
 

428 MALMANTILE™
STANZA XIV.
J coraggioft al primo, che si moffe,
Gli altri (gid fendo meglio [ui piccenali )
Non poterono star più alle moffe,
Ma corsero ancor lor come Terzuoli 5
Giunti di Malmantile in fu le fofe,
Drizrate al muro afsai feale a pinuoli
i falirui renewano una baia
Com' andar pe' piccions in colombaia.
STANZA
Gh fiipits, le foglie, e gli architraui
A quclto efecto efsendo gid (murati
Per via di curri, dargani, ¢ ditrani
Gli hanevan su le mura firascinati, Faceano un venga addofsoat
Stabuito d' entrare in guerra,¢ dar U affalto a Malmantile i più ¢
rono i primi a muoverdi, ¢ gli altri meno coraggiofi (eguicarono. &
Dante, che nei Purg. C, 2, dice:
Come quando cogliendo o biada, 0 loglio
J colombi adunati alla paffura
Quieti senza moftrar U usato orgoglio 5
Se cosa appar ond' essi habbian paura
Subitamente lasciano fRar ? esca
Perché affaliti fon da maggior cura,
Arrivati dungue alle mura di Malmantile, credendofi di trovar fac
s' ingannarono, perché quei di sopra gagliardamente si difendevano
altro. Qui ¢ da considerare, che se bene Capitelliye Srontespizzs fon me
shitettura, il Poeta (cherzando con I equivoco.di capi, ¢ fronti, ¢ serve
verbo Pampare nel fenfo, che lo pigliano i Legnaiuoli, ec, che dicen
C. 1, tt. 8., vuol die, che tali merii pictre, ed altro devano sopra 1 2
alic fronti dei soldati, ¢ gli stampavane, cive gli faceyano di quei-
chiamano stampe, ed in fuftanza vuol dire, che rompevano tefle,¢
fuono, che rendono i corpi battuti fecero i Greci il lor verbo typrein,
re; da queito verbo ne venne Typus voce pur Greca accettata da'|
una forma imprefia, o cavata fuori col battere: Se ne fece ancora 7}
tamburo, che Omero pil conforme all' origine disse Tympanon seguito
Catullo nel Poema Gailiambico. Noi abbiamo voci da riferire a quelte' \
come farebbe Stampa, Stampita, Stampare, Stampanare, Ma in pro
fiampe fatte ful moftaccic d' un' antico Giucatore di pugna, evvi un
gramma del Greco Lucilio, che in nostra lingua voltato dice Così;
2 un vagho, Appollofane, il tua capo,
O qual fu mai pin traforato arnefe,
Son tane di formiche 90r dritte, or torte,
E par, che con bizzarre, e varie nore
Vn Lirico eccellente il Lidio v' abbia
Inravolate sopra, ol Frigio canto.

esceftie?

jie te te en ee i a.
 
   

  

NONO CANTARE,
6 Or franco vibra il minacevol pugno
 Ecombarci pur liero in duro arringo 5
 Che se colpo novelio a te discende,:
Quel ch' ai riscoffo, aurai, ma non gid nuond
et Capir nel capo tuo potra ferita,
PIP prefo chrio nol dico,, Preftitiino confumaron manco tempo a far tal cosa,di
silo, che io confumo a dirlo. Latino dicto citsus.
“N lena leua, un ferra ferra, Quando vogliamo intendere, che una gran quan:
: di popolo adunata in qualche luogo si sia partita in un subito, ¢ velocemente
ia one di questo.detto 5B signiticano quafi lo stesso, se aon che l'ultimo ef-

    
   
      

» quando uno è da altri incaizato a correr, ec, vedi (opra C. 1, st. 63. e»

ke
- f hail

pero nel p luogo si potrebbe anche dere, che i primi volon-
 tarj, ed 1 fecondi forzati dalla riputazione. 11 Varchi Stor, lib. 2. dice: Pa /ubir
| Wegridato: armi armi, lena lena, ferra serra, ec, Dal che si cava, che questo detto
tog significhi Leva la roba di sopr' alle;moftce delle botteghe, ¢ ferrale come (eguiva
at | Firenze nelle follevaziont di popolo, ¢ che ii medesimo detto sia poi facto co-

Mune a oga: forta di tumulto, ¢ per ¢sprimer un moto turiofo di quaatita di po-

4
Ll

| AR correnda. Andar correndo. Il verbo ire venendo dal Latino, vale appreffo
di aot quaato il verbo anaare, ma ci serviamo folo deil' Intinito ire, del partici-
Pi9 ito, © folo, o accompagnato col verbo efere, e dell' Lmperfetto ina, ixano,
che si dice poi, giva, ¢ giwano, Nella vita di Cola di Rienzo (critta in lingua Ro-

Mana antica trovali jio, ¢ seffero, ¢ simili, che i Toscani cangiando |' [ coafonan-
foi *ealpra nella doice lettera G dicono gio, cioé andd, € gifero, cioè andaflero.
wi fimiimente prende alcuai tempi, come farebbe i presenti di tutti i modi,
'i dai verbo Vado, io vo; ancorche Dante viatle forefticramente, edadi per Vada;
gg © 0i0 cofretto dalia rima.

» ST ANDO mestio in fui piccinoli, Essendo pi gagliardi nelle gambe; ¢ questo
gi AVVeniva, perché havevano mangiato. £ piccinols, che & il gambo delle fruttes.
g Latino pedicutus, ¢ pref comunemente in questo cafo per le gambe dell' huomo,
ia NON porersero Rar falc alie moffe. Non potettero contenerfi, che non corref-
a fero. Toho da j Wavalli Bacbari, i quali corrono a i palj, che eflendo tenuti per

Jo freno dai loro Stallon: al luogo donde a) fuono della tromba deeono partirfi,
7 che si dice le moffe ( Latino carceres ) molte volte scappano, prima che sia dato i)
' detto fegno,e questo si dice non far ferme aie moffe, che poi paflato in proverbio
! non haver pazzicnza, © lofferenza, ma per il gran desiderio d' arriva-
i Tea Uo luogo, partirfi prima del dovere; ed esprime quella inquietudine, che uno
, hanelitaspercar, che /egua una tal cosa da iui anfiofamente bramata. Del Ca-

vallo generoso Virg. Georg. 3.
Stare loco nescit, micat auribus,G tremit artus,
Colettumgue premens volvit sub naribus ignem,

CORSERO come terzxoli, Corsero con la stessa velocita,con ia quale vola alla.
preda il terzuoio (pecie di falcone. Perché così sia detto rende ta ragione il Tua-
No de re accipitraria lib. 1, edtrque ad co. cum tres foetu enitatur eodem Predones gene~
rofa parens mas kitinus imo despectus letto incet appeliatur y & inde Tertius,

SCA-

 
 
    
   
   
    

4jo MALMANTYPLE &
SCALE 4 pivoli. Scale fabricate di due corredti «
glioni sono pivoli ficcati fra 'uho ¥'¢ I altrore C
fine in distanza uguale a riscontro, ovvero'i detti f
© stecche, © regoli di legno conficcati in deeti correntt Mampati
riscontro. B pinole, ( Latino clanicx/a, civxt cavicchio; ovvero
de ogni pezzo di-baftone adattato a porerd mettere in un buco,
TENEV ANO una baia, Stimavand cof: facile;*Stima
burla, ec. Latino mage, Ii Ferrari dice poter venire questa voce da
iflar' a bada, in ozio, Latino wataré, © O01 i
COLOA18 ALE, Quelle flanze fabricate per lo pity nelle form
per uso de i colombi, € nelle quali'wascono i piceionit) «>
FEC ERO parergli altro fuono, Fecero lor conotcere, che |
ment.. voy
ewERLI, Qvei picco}i murelli'; in distanza uguale'y ned quali per!
mioano te muraghie delle Città, ¢servond per: parapetti'. ad soldati,
per difefa delia muraglia; così dette quali. murnlesdice il Berrari; fume
primes parus murs,Dichiamo @una-cola;che ancora abbia delle dific
rarsi,¢che non Gi fiano per anco spuntate: £ ci è de merio, cio' non è elpy
to il cutto, Ci rea ancora qualche parte da abbattere 2 Vedi foro © 12)
ISSO fatro. Subito. Due voci Latine corrotte, ¢ ridotte Toscane,
loro lo feflo signincato.
DISEATTO (e reftuggini, Infrante le Teftuggini animali Terreftri,
che hanno la coccia, © guscio durissimo da alcuni'detti Tartaruche
he, da altri bexzache ( dal bezzicare,.ch' elle fanno raspando in terra
atinl Tefudmes, E § potriaanche dire, che ? Autore intendetle di qu
razions da guerra, che usavano gli antichi dette Te/udines; nelle gi;
no foo alle mura, reggendofi fulie spalie gli uni gli altri, ¢ aiutandofia m
tarui sopra, coperti turu di feudi, € terran iteme per ripararsi da' colpi, che
si (cagliavano per di sopra; E quetta operazione s' addimandava refixggine spe
ché flavano col capo, € 'colla vita dentro agli icudi, come flanno le
(in Lp. torragas in Beanz. ortaes ) dentro aile loro scodelle 3 le quali )
dette da' quei dello stato di Muang, come racconta il Ferrari bi/se fo
bijce (codeliaie, perché anno 1i capo di bilcia, ¢ stanno rinchiufe cone i
delia; Onde potrebbenfi dire, dom:porte, come un' antico Poeta chiamé le chien
de. Autione famoso ceteratore ¢ fatto parlare da Pacuvio così, delcrive
tetluggine con que' versi portati da Cicerone de divin, ub. 2. Q@madrapes 1am
da, agreftis, bumilis, aspera, Capite breui, cernice anguina, adjpettu trad
ruche,¢ BR2uhe, sovo voci usate dai Caro ne' Mauiaccint; ¢ i} Veneziaiol
chiama Gv/ane dal Gr. Chelonei, da noi si dicono anche butte seodellaic,
BAST le NO Seré, Celebre, ¢ nouttime scrittore d' archucuura.
EbDIF/Z/0, Preto largamente s' inteuce Ogni forta di faborica, €
ma prefo ttrettamente vuol dir faia, ec, Cafe, ed altre niuraghe, |
ades, @ facio; ed in queito andiamo uniti co' Latini, che per earfien
no ogni forta di scrittura. Gio, Villani t, 128. Pauose/f ad ascdin, 00, ©
difici, ¢ per cane per forza ebbe, Li lib, del conquiito, Per joraa a

 

 
       
  
  
     
   
   
  
  
 
  
  
 
 
  
 
   
 

 
 
   
  

gE Es PSs SEES. =

 

'> ie PS

= Fo

 

 
 
 

  

NONO CANTARE.

1. Capiteli, ¢ frontejpizi,, Columnarnm capitula, © fronts bespitii, >
(ATT H Srglie.s ¢. aui, Stipi (ono le pietre de i tianchi y¢ foglie quel-
a parneey quelle dilopra, che tutte insieme formano una por-
a» Suipice dal Latino #:pes.. Architrave; quafi trave principa-

: « Quei ruotoli di legno, che servono per facilitare lo strascico de i pei;
atini li ditiero Palange, Vedi sopra C. 2, st. 65. Dichiamo: mertere une fal exr-
Spiguerlo a poco.a poco, ¢ condurlo doicemente a fare alcuna cola, La
Voce viene probabilmente dal Latino baiudare; questo aggiuttar' un corpo
}a un' altro in maniera, che quello Jo porti con ficurezza. E la feconda
| Latino xmbdicus, cioè punto ne) mezzo, Bilicare quali ponere in umbitica,
ARGANO. Strumento, che servc pct tirar fu pefi in alto, che da huomini è
" moflo in giro per via di leve. Alcuni Latini lo dicono Sucu/e, i Greci oniffi, cioè
 Afineli:, ¢ quelto & V argano,fecondo il Filandro, cum axe iacente, quello pui cum
axe ereite, dice che in Latino ¢ Ergeta, cioé macchina da lavoro; donde, 0 da
voce(lecondo i} Baido sopra Vitruvio)è fatta la noltra Argano,
MSADATT 1. Scommodi; Non atti a efier portati, o Arascicati.
MC ATI, Meili in bilico-,-0. equilibrio., Latino Jibratis.. Diciamo.bilico
ofitura d' un corpo sopra ad un' altro in maniera, che posando quafi in un
non penda, o aggravi pil da un lato, che dall' alo. L nostri Scarpellini
 dicono baggiclare per biluare. i
it. BOTT O porto, Si dice. Ch' è cb' € 5 colpo colpe sec. ¢ 8' intende Spefiime volte

      
   
   
      
      
    
    

 
  
 
 
 

PAR* un venga, Tirar roba da alto a batio sopra auno, che sia foo.

 

 

 
 

“a ay STANZALXVI. STANZA XVIIL
a Le Donne anch? esse corron co' figtinoli y Chi, perché gik non piglin l imbeccata
f i 2 dy che troxan, gettan dalle muray Cuopre i capi con tegoli y ¢ mattoni,
o con la conca, 0 vafo da vinolt Chi verssa git bollente la rannata,

a 9» Pighia a qualcun del capo la. mifura; Che pela i vifie porta via i bordoni,
a8 Profuma il piscio i panni, ei ferraixoli Nei? olto un'altra intigne la granata,
yet Ne guardan vc v'é penail far bruttura, E fal asperges sopra i morioni,
ps Chi tira gi: unjastrone alic cerned y Altre buttan le caffe,accio i soldaté
ie Che se ewe orili serva per murella, Partir si debban, poiché fon cafjati,

ie ooNarraiil Poera la difefa, che facevano queidi Malmantile, ¢ descrive diverle
we" Opérazioni militari adeguate alla composizione burie(ca di cutta. opera.
CONCA, Valo grande fatto di terra cotta, entro al quale si fanno i bucati
Ke ASO da viyoli, Sono vatetti di terra cotta simili alle conche, ma piccoli, en.
| 80a! quali Gpongono vivoli, cd altre pianterelle d' erbe, 0 fiori. Dice che.con
v — gucfi pigliano la mifura a.ijcapi y perché hanno il vacuo capace della tetta d? unt
Td huomo; al quale quando i Cappellat voglion pigliare la mifura della testa, metto-
u# ~—-'NO in capo un tappelio; € ceftaco di Malmanzile per pigliar tal milura, in vece
sso un cappelio., mettevano-un valoda vivoli: ¢ cosìscherzando intende y che ti-
@ — ravano (ule tefte a i soldati di Baldone i deni vali.;
@ \SEvi dipenail far brurture', Se\vi ¢ pena il fare sporcizie; Dice che tirano fino
Dorina, ¢ non guardano,-se. cid sia proibito,: ¢ con questo dire, accenna i} co-
ef flume, che ¢ in Firenze a” affiggere alle muraglic dove non si vuole, che fien fat.
r te

 
    
 

432 MALMANTILE

te sporcizie, certe tavolette di pietra, nelle quali & scritto il
flrato degli Otto, che proibisce, ¢ mette la pena a'chi fa
niuno si posia pretendere ignoranza; Ed intende anche di
¢ grave pena, che è in Firenze a buttare dalle fineftre nel
torno a' quali dispone anche la ragion comune, come si vede
De his, qui deiecerine, vel effuderint, ' '
SE v' ¢grilli, Sopra nel C, 6, st. 22. dicemmo, che grille si cl
cola palla, che si tira per fegno, giucando alle palloctole; ed all
firelle, qual giuoco dicemmo come ti facia sopra ia detto C.6.t,
rché tirandofi, or qua, or la alla ventura, 0 alla volontà
a il falto del grillo, che dopo un breve falteilare si ferma, e-poi
-dicefi ancora Lecco, quali i/ex eMurelle chiamanfi anco
nelle sue Rime. orate
Ch' io do fempre nel lecco alle murelle OP R
dal Toscano antico e#ora, che ¢ lo stesso, che il Latino Moles }ép
si dice di pictre. A'awer la refta piena di griili s' intende uno, che ha capric
vaganti; ¢d il Poeta scherzando'con questo equivoco di' grille dice
quelle laftre a' grilli, che sono neile tette di'coloro, come se piocatietd
strelle, 0 murelle. Dal pazzo similmente,¢ curioso faito del grillo fon detti
icapricci, ¢ fantafie firavaganu, che faltano in capo, ¢ per così dire
PIG LIAR' un' imbeccata, Infreddare: B diciamo ancora: Pighare df mitt
caffrone, perché il beccd, ed il castrone hanno una tal raucedine, che
pre, che cofiano, appwato come fanno gl infreddati.
Té£GOL/, Pezzi di terra cotta adattati a coprire i tetti delle cafe.

ap

   
 

 

     
    
  
  
 
   
   
    
   
  
 
  
  
  
    

 

HlAe.
: RANNAT 4, Liscia forte; che è quell acqua bollita'con cenere; ¢
dalla conca, quando si fanno i bucati. Lacino /ixininm,
BORLON/, Inteudiamo quelle penne, che non de} cutto spun
scorgono dentro alla pelie degli uccelli, ¢ per similitudine intendiamo il)
spunta nella faccia degli huomini « way
FAI alperges con la granata, Diciam far ? asperges quando con spugha
tra cosa si (pruzza acqua, © altro liquore,.a minute stilie; 1a qual cola il
chiama e4/pergere, qui dice, che spruzzavan' Olio con le pranate;
aiciato un mazzo di scope, © d' altro simile adattato per (pazzare,)

stanze.

SOLDAT! caffati. S' intendono quelli, che sono stati pri
la milizia, perché cafare vuol dire cancedare: Ed il Poetas
guivoco di <afaté, cioè percotli dalle cafie', dice, che se fon
nou dal Campo, perché non fon più nel numero de” float,

SLANZA XIX,

 

Vi? altro con un gatto vwol la berta, Ed il primo ch' et trova
Legato il cala,ond' es fra quei.d'Vgnano - Che dou'ticbiappar
Sguawnialugna, econ la bocca aperta

Griaa ina/prio in sue parlar Soriano s

  
   
       
   

  

oF

ee ee ee ee eT oe eS Ol ee

=~ aw

 

 
 

NONO CANTARE: 433

arnt) re Bie XX, e
Miagola, ¢ foffia it gatto, es' arronciglia y
Ed Gite endian heerees
janes quel che oa " trattopigla
Beli è miracol poi se pite gli feappa;
thie oat peter tee cos riglia y
jie Lo tira fu con quaiche bella cappa,
a «Ci qustcheciarpayo qualche pinacchiera,
ye  Ecosi gli riesce di far fiera,
ame cool (STANZA XXL
due Quand una volta lasciale calare
ib oi iaers al buffo di Grazian Molletto,
Che fu;di posta per ispiritare,
«Quel pelliccion vedendo intorno al petto,
we Le beftia intanto falta, ¢ dal coliare
'hoe "

=

 

Tutto prima gli firaccia un bel gigisetto,
fet  Dipos si lanciaye al capo se gli ferra,
ebst  Si che il cappelio gli mando per terra,

STANZA XXIL
Non.sa Grarian, che Diauol si sia quello:
Pur tanto fac' al fine ei se ne sbriga,
Ea aiza il vif per farne un maceilo,
¢Ha vedendo il rigiro, e ch'ei s' intriga
Con dame, vuol canarsi di cappello;
Ma perch' il micio gis ha tolto La briga,
La Dama accsuetrata, anzi civetta
Lo burla, che gli è corsa la berretta,
STANZA XXIILL
Ed ei, che da colei punger si fente y
Onde al nafo lo fironzolo gli fale,
Perde il rispetto,¢ quiui si rifente
Con dirgli, Atona merda, ¢ ogni male,
Vain questo al aria ungraromar digéete,
Che 4 terra feende a mafse dalle scale
Fiaccate,erotte ach'elfe dagii /prazrolt
'Di pierre, c' ancor grattana § cocuzzoli,

oa Continova il Poeta a narrare gli accidenti, che (eguono nell' aflalto di Mal-
ie mantile, e dopo haver detcritto una Donna, la quale con un gatto legato a uns
" i miazzacavallo andava levando rcba da dosso a quetlo, ¢ a quello, come segue a
ol Graziano Molletto ( che ¢ il sig.\ Conte Lorenzo Magalotti ceicbre per aobilta,
HF 'e dottrina ) dice che le scale degli AGalitori furon rotte dagli Allediati: ¢ con i
r faffi, ¢ con altro, che tiraco di sopra alle mura, dava ancora addosso a i soldati.
at IL (a berta, Vuol la burla ( vedi sopra C, 4. st. 47..) onde shertare, lo stef-
4 fo y che beffare. [i Davanzati ped dite Swerrare nella (ua traduzione di Tacito.
mY Corte poesie senza antore, che fuertavano le sue crudeid. Se bene in questo luogo si
; poirebbe intender per berta quello strumento, che serve per ficcare i pali ne i
ea pfiumi nel far le fleccaie, che ¢ un gran ceppo di legno ferrato, il quale infilato in
“ln pernio, 0 ago di ferro confitto sopr' alia testa d' un palo, s'alza per via di fu-

ni, ¢ si lascia ca(care sopr' alla testa del detto palo ( già fitto in terra) per fario

sf andar pita drento. E perché in questa medesima guila faceva Colci coi gatto,in-

yo teade, che defie così /a berra, eruendofi del mazzacavaljo, che appretio gli an-
ti"  tichi era usato per arnefe militare, come s' ¢ toccato sopra C.6. st. 86. In propo-
i)" fito di Berta per Bxrla, il Ferrari dice così: ognuno poi la creda, come gli pare
4 f verifimile, Dopo aver detto, che que' delio flato di Milano chiamano Berta
8 ta Gazzera, ¢ cid dal balbettare,ch' ella fa; foggiugne; (aoniam autem fanne,
gil! At que irrifionis [pecies eff aliena verba imitando reperere,inde Berta pro Inda,ae derifione
gi accipitur, © fare una berta illudere, & decipere. O pure finalmente ¢ forte più
credibile, che venga questa manicra di dire dalla novella raccontata sopra nelle
Annotazioni alla St, 47. del quarto Cantare,
d& —. SGVAINA I agna, Cava fuori' ugna, che tiene alcofte dentro alla pelle, la
we bed gli serve per guaina, ed il Poeta scherza, dicendo /guaina U' ugna. lock ques
gnano

Wo + Appropriando benissimo wns, a Vgnano. 
yd 4NASPRITO. Incollorito, meflo in ira, in stizza, in rabbla. Latino exa/~
lii 1N

peratis,

i

 

 
 
  
   

54

IN parlar Soriano, Ciok oer gatti in ling
si dice quello, che ha la pelle di color lionato ferpato d
ché si dia in altri animali, o in panai, non si dice foriana; se
perché i gatti di tal colore fien venuti di Soria, come ai
di Perfia quelli di color di topo portati da Pietro della Vaile,
chiamati Perfiani, o per Perfianini. 'one

DISERT A, Cioè ttroppia; concia male. Guasta.

VVOL ievarne il brano, Brano dal Latino barbaro mn.
il pezzo, Vedi sopra C. 6. st. 47.

MIAGVLARE, 0 ignaulare. Bi ii

  
   

  

   
 
    
 
 
   
  
 
   
  
  
    
   
  
   
 
  
      
 
  
 
 
  

  

I gridar de i gatti; + il'fofiare dic
quello strepito, che fanno aprendo la gola, quando fond in rabb

S' ARRONCIGLIA., si torce in sse stesso, come fa la ferpe quan
viene da ronca, roncola, ronciglia; specie d' arme; 0 pitt” 2
agricoltori, ed ¢ fata come una spada, ma è torta in cima a guisa d
serve per eflirpare i pruni: o pure da Ronciglio, usato' da' Dante per graf
fauto a uso d' uncino. i "

E MIRACOL 8 egli feappa, E cola soprannaturale, © imposiibile,
degli artigli, £1 Petrarca. soe eee
E cio, ch' in me non era

Mi pareua un miracolo in altrui
cioé una cosa, che non potefie stare. '

LO tiene in brigla, Cioè 10 maneggia bene, facendolo operat

CLARPA, Dal Franzefe e/charpe, banda, bandiera.. Quel draj
tano i soldati cinto:de' soldati era proprio il cintolo, onde cinguoie fol
dalla milizia, Vedi sopra C. 5. st. 33. 5) ie 7

FAR fiera, Buscar, 0 acquiftar roba = per efempio ends pirando per
torni, ¢ chi gli dette pane, cht voua, chi una cosa, ¢ chi un' altra tanto,
Satta un poco di fiera, se ne tornd, mn.

D1 posta, Subito: Di primo tempo. Vedi sopra C. 7. st. 92. BY
giuoco di palla, che si dice dar ai posta quando si da alla palla, prima
terra, ed ¢ il Latino ilico, ¢ vefigio, Gli antichi dillero: Di colpo y
fo, che di Borto. 7

FV per spiritare. Hebbe un grandissimo spavento, o paura.

GIGLIETTO. Specie di trina con punte; così detta, perch ha
col giglio.

Avr. Cioé gnell' ordigno, col quale la donna alza, ed ab
Vedi sopra C. 4. st. 69. Se bene & pud ree la voce rigire nel
mo sopra C. 7, st. 41,, ed intender, che Graziano, alzando il ca
giro, cioé la donna, ¢ dedurre questa opinione da quel, che foggiung
Vedendo, che s' intriga con Dame,,

ACCWETT AT A, Afiuta; Sagace. Tolto dagli uccelletti,
civertats, quando havendo altre volte veduta la civetta sono dit
non si la(ciano lufingare a volarle attorno, come fanno quelli
mai pil veduta. ae

eANZL cinetta. Pili toto troppo ardita, ¢ sfacciata. Si dice'

, eel ee8 TR

e— lL eBeuwe ete

— 7 ate

= ~ - — =

 
 

x A juomo da poco, però con tale equivoco

 

NONO CANTARE: 435

vane troppo ardita nel trattar con gli huomini, quafi faccia con essi, come la
corerasss gi uccelletti, che cerca con gli (uoi gefti di tirargli a se. Vedi for-
to in questo C, st. 60, E Plin, lib. 10. cap. 17.;
CHE gli ¢ corsa la berresta; Che il gatto.ha fatto preda, ¢ gli ha portaro yia il
ppello.. Ma perché, La/ciarficorrer, pee via la berretta, vuol dice Elicres
mentando G h diveherch iandnban tones
raziano womo da poco dal veder, che si lascia rubare, € portar
via il cappello, gli an burla; di che egli s' adira, perché si fente fete
if r¢ dali' etiere burlato da que(ta donna,
-, GLI fale lo frronzufo ai nafo. Derro sporco, che significa entrare in collera, ma
= poco usato, dicendofi pil tolto fair la muffa, 0 la fenapa, 0 la moftarda, 0 it
herimo, ec. Vedi sopra C. x, st. 39. Bil Lalli En, Trau, C, 2. st, 65,
Waapn 6 Airs Corebo un tale firazio,e tanto,
i Con la moffarda al nafo,e nol comporta,
AGli Ebrgi.colla fiefla voce significano, ¢'/ na/o, ¢ ira, perciocché par, che qui-
¥iclla particolarmente rifegga, siccome disse Teocrito + acris bits ad nafum fedet,
Onde noi dichiamo Arric¢iare il nafo per ifdegaarG; simile in parte quel che dice-
wano gli anuichi Leware il miffo. La voce Ebrei fie Aph, in Siriaco Apha; ondes
. itorcec: ¢ venuta la nostra 4fa, colla quale a ete una cola fomi-
giiaptitima alle vampe dell' ira; cio¢ un vapore, ¢ yn caldo fallidiofo, ¢ affan-

    

 

HO!»
t 'sop SLrifenre. S'adira: Entra in collera, perché ¢ burlato,
pjat 'A merda, Detio ingiuriofo usato fra le donne di vil condizione, ¢ del-
Ta voce mona vedi sopra C, 5. tt, 18.1 Lagini similmente (asum, conum, frerquili-

me,

. FLACCATE. Spezate, Fiaccare & verbo proprio per esprimer, quando un le-
£00, © altro. materiale si rompe in mezzo per fouerchio pelo, Latino fari/cere.,
 springs. Donde poi bxeme fiacco vuol dir huomo affaticato, ¢ stracco; se bene &

ver) imile sche venga dal Latino faces, faccidus, dichiamo, fiaccare |e braccia

A uno, clive infragnerglicle, ¢ romperglicle colle baftonate.

SPKVZZOLARE. Vedi sopra C, 7. st. 15. E qui è derto ironico, ed intende

f Bingge pict 7 '
V2ZZOLO. Latino vertex, cacumen. La parte di sopra del capo diffefi an-
she. Zwecolo 5 siccome da Cocuzza de' Napoletani ( Latino cacarbita ) ¢ si dice an-
Gora. comiznole, se bene questo ¢ proprio delle fommita de' tetti, ¢ de' camumini;
dal Latino cudmen quali culminnlum «
ares ST NZA XXIV, STANZA XXV.
Chi con, chi per banda,.¢ chi fupino Quantungue il.campo annaffi tal rugiada,
i se ne viene, ¢ fa certe cascate, Come le zucche, annarpican le [cale,
Che manco ie farehbe un' Arlecchino y Onde più a' xno in gik versala firada
, Quand in commedia fa le sue fealare; Fa pur di nnono un bel [alto mortaic;
Si che y stinnanzi fecero il fantino, M44, piché ammonti ne traboechije cada,
Le brache in fasti glieran pui cascate, Sardonello [2a forte, ¢ in alto fale,
> B infranes, ¢ pefti andando gis nel foffo E trai mimici al fine a lor mal grado

Mette [u il piede,e agli altri ye Uguado
2 PAN.

, Hanatolere a quchlo nuove scate adddfe, y
geet ii

 
 

436 MALMANTILE™
STANZA XXVI.
Chi vidde in un pollaio, ove fisrona | *
Vn numero di polli senza fine
Tra lor cascar qualche pokafira nuona,'
Che roft addoss' elt ha gullie galline
Ciascun per far di lei l'ultima'prona 5
Eye aa folelapariea athee, 1
Che la difende, ¢ da beccar (e porta Ma Eravan, che
Stroppiata rimarrebbe, ¢ forse morta, Aiuto a un cempo,ed
Rotte le scale coloro, che erano sopra di esse cascarono nel fofla
r0 corpi furon polate nuove scale, in fa le quali intrepidamente
neilo falto ful muro y ¢ feel nella Terra, dove fu da mojti di quei
falito: Ma Eravano, che lo vedde in pericolo d' efler ammazzato
¢gli dentro a dargli aiuto.

BOCCONIL. Dittefo in terra, 0 altrove con la pancia, e faccia ve
no, Lat. pronus contrario di Sapino, fulle reni; Lat. fupinus ye Per,
la doppia posicura che refta, diversa dall' una, ¢ dal' altra, la diciamo 4 x
Per franco,¢ Per latv, Lat. in latus. Bocconi  detto colla stessa forma, che!
nocchioni, Brancoloni, Saltelloni, ¢ simile 5 che si -dicono anche Boccone
vhione, ec, anzi questa ultima maniera è l' usata dagli Autori antichi Ti

eARLECCHINO. Va fecondo Zanni, cioé un servo.femplice in
Così nominato, il quale faceva assai bene le scalate, che fon quei giuoc
Ai fuol fare detto Zanni in commedia con una scala a pivoli, sopra alla
affaticandofi di voler falire, casca in diverse manicre. f

FECERXO il fantine, Pecero il bravo, l' ardito, il coraggiofo, Si
gura. Egli ¢ fantino cioè persona, da fare queffo,e altro, Fantino di
faate. Lat, infans, cioè Ragazzino usato dagli antichi in generale, @
oggi a ua significato particolare. Chiamando noi fantini quei R i, ¢
pr' a cavalli spogliati corrono al palio, Si dice anche fare if Baiardina, da
lardo celebre Cavatlo di Rinaldo Paladino, così detto dal suo mantello y
yea efiere Baio accefa.:

GLI eran cascate le brache. G\i era entrata la paura addosso
animo. Vedi sopra C, 6. stan. 20, Lat. aninsum desponderant,

ANNAF SI tal rugiada, Annaffiare vuol dire Ammollare 5 0 af
giada vuol dire quel che accennammo sopra C. 2. stan. 55. alla voce gr
Ma qui da nome di ragiada a quelle pietre ec, che buttavan gid gli

-dnnafiare detto da Adacqware, che si dice anche /anacquare,e Annacquare y
Ui duc ultimi verbi diconfi propriamente del remperare coll acqua il vino;
equare propriamente ¢ dare [ acqua alle piante. + Ia
INARPICARE, Aggrapparsi, forse dal Gr. herpein chet in
Pere, reptare, Salire in alco, appiccandofi con le mani, € co' piedi y
no i gatti. Si dice anche rampicare sopra C. 4, lan. 68. ed-«
vedremo nella seguente ottava 28,:
SALTO mortale. Chiamano i Giocolatori falto mortale,quando
tecra Con le Mani', o con alcro faltano, voltandy la persona fo}

 
       
   
 
  
  
 
 
   
 

> eran

  
 
 
 
 
 
 
  
  
     
   
  
   
 
 
 
    
        
  
   
 
     

WEITAF.

wee Se peg RRFESZLTE=

 

  
 

NONO CANTARE, 437

verifimilmente facevano coloro, che ca(cavano, o erono gittati da alto 'a batfo.
) TRABOCCARE, Intende precipitare, 0 cascare da alto a baflo, romperfi
la bocca; andar colla bocca per terra. E se bene il proprio significato di trabuc-
“care è quando mettendofi in un vafo maggior quantica di liquore, 0 d' altro, di
PS yche posla capire', casca dalla bocca del vafo quel, che vi ¢ di pili; onde per
figura si dice'un Trabocco di fangue, ec, tuttavia si piglia ancora in fenfo di calca-
te. Traboceo ne i vizzi, ec).
hie = ROMPE il guado. Apre \a strada, 0 il paffo. Ovid. de arte amandi,comandando
'ex che si rompa il guado per via di viglietto, dice: Cera vadum tenter, Guado vuol
s dir quel luogo ne i fiumi,per dove si pud paflare senza navilio, che si dice guada-
ve; Eda quelto guadare, o rompere il guado s' intende aprirfi il paflo in qual.
“Voglia occasione, o congiuntura.. Parrebbe che fletie meglio vado dal Latino
mis » siccome si dice ancora yolgarmente il porto di Yada, dal Lat. Wada VYo-
 taterrana; perch così Gi fuggi V equivoco di guado (pecie di tintara, mas
ivell quelli stitichi, i quali si vergognano, che la nostra lingua sia aiutata dalla sua+
frit madre Latina,non ci concorrerebbono, ¢ darebbono una turbativa a chil' usaiic.
hist = MANDAK 4 Purafso. Par morire; E perché significa il medesimo che man-
aoe » 0 4 Scio credo che derivi da i foccorfi maadati in diverse occafoni,
| “tempi ai detti tre Juoghi, da i quali non essendo tornato veruno di quelli, che
al —andarono, quando si vedeva mancare uno in paefe, si cominciafle a dire. Eel
stl ¢ andato a Buda, a Scio,0 4 Patrafso; per intendere egli € andato in luogo, don-
de non tornera mai più, duc, unde negat redire quemquam; ¢ s' intende egli è
i = Morto. Vedi sopra C. 5. itan. 13.
j TIRAR l'ainxolo. Vuol dir morire, dalle cunvulfioni della persona, che pa-
§&  tilcono quei, che si muoiono, Aixslo è (pecie di rete da pigliare uccelli. E la for-
2a, che fa ' uccellatore nel tirare l' aiuoio, o simil forta di rete, ¢ deferitta das
id Petro de Angelis da Barga in que' versi +
0! Tum vero innitens pedibus confurgit,& omnes
Intendens neruos magno trabit impete funem.
4 ZO feorge debito. 1.0 vede in pericolo di morte. '
STANZA XXVIII. STANZA XXIX,
1 ' Chinmgue è 'n Castelle allor pien di paura - Auitiene a lor ne pri, ne meno un' iota
% Corré per far © auanti et prit non vada, Com' ai fancinlli, quando per la via,
Fan la tura at rigagnol con la mota,

 
    
  

 

«RB memrtil vuol rispinger dalle mura,
' “\ Ch altri pik la 2 arrampica non bada; El! acqua ne comincia a portar via,
| | itr db ouniare anco di gua proccura Che,mentr' affodan quixi ov'ellaé vota,
| Main fete Ini ghit ged farts la frrada, Essa distende altrone la corsia,
, E se riparan la, prt qua fracafsa,

| E a cogs intorno tanto il popol cresce,
C" ogni riparo innalido riesce. Tal ch' ella rompe,e a lor dispetto pa/sa,

«\ [Soldati di Baldoné superate tutte le difficuita, finalmente entrarono in Mal-
© mantile, éd il Poeta paragonando questa cacrata ad un' acqua corrente, che rom-
 pe, € paffa ogni oftacolo, che le 4 pari avanti, esprime I" inutil difela, che fan-
“no i Terrazzani.
ARRAMPLARE. E' jo ficflo che inarpicare detto poco sopra, ed è il Latino
Perreptare.
VN

™"

 

a
 
     
   
 
 
      
     
    
  
    

438  MiALLIMIAN TLE Bot

VN ita, Vn niente, detto sopra C. 1» stan. 18.)

RIG AG NOLO. Diminutivé di ome 5 Piccolo riva,
è proprio per intendere da parte più: bafla, che ¢ nel 0
di Firenze per dove feotre l'acqua's che piove y efic 7
incende nel presente luogo 4 € ¢' aacénide comuacmente s che ua
rigo, 0 rio diremmo rixolo 0 ra/celloy dewro così da Riuiceday la
presso alcuno antico. Se bene Dante nell' Inf. C, 1g.dices Bd
Sente rigagno, ec. ed intende quel fiatnieell@, 0 rivos il, 0
nali. Li Varchi Stor. Fior, libro 13. Commiciarono ad nscar fuara
e che i rigagnoli correuano, ele ve erat piene di motayedifarge rt
Nov. 16. 4 rigagnolo delia qual via corre, chepare un fiumicclian |

MOT A, 'Lerra ben inzuppata acl? acqua. Ai Percariz, Lupums
 immora, Per intelligenza della \iuddetta comparazione ¢ ince
i ragazai dell' 1afima piebe di ae 'fogliono per loro pa
dopo ja pioggia (corre l'acqua per detti rigagaoh pigliate del
ond ae come un Danian opposte ai corso dell' aon
paflaggio al fume, ¢ questa chiamano la twra.; ma fiocome d'
quel iuogo fempre va crescendo,)così 0. per 10 pelo, rompe
bondanza traboccando la superay ¢ pada via noa oltaace dri
v' appiichine, come dice il Poeta. Qunero nell' Aliads ib, a 5s,

De! Troiani fereci allagranturbay...
dt folgorante eApollo andanainnanze
Tenendo in mano il preziufo fondo:

Ei degls Achini il muro aterra Sefe;
Ne coffogli fatica, appunto.come
Lungo il mare il fanciulfacoll arena y
Che poich¢ fabbricato ha per. suo gsoco
Va gentil fanciullecsco alto lanoro;
Colle mani, ¢ co' pie scherzando il guasta,

A lor dijpetto,, Contro alor voglia. Lat. ijs innitis, Il Boce, disse
Per di(peto. A Dante prima, ¢ poi al Petrarca ia uecedlica della rl
il servirli della parola De/picto accordandofi in cid, siccome ima
col dialetto. Provenzale, o Francelco. Virg, ecl. 2. Despectus tibi Jum ne
queris, Tu m' hai in dispetto,ne ti cale il fapere,chi io mi sia, Confiache
la strada, che è.per il mezzo della galera; onde que) groilo Canaone.
diceli Cannone di corsia, S' intende ancora per la correate dell' acqua..

       
  
      
   
      

FeSlFaer- aes

eet

it Se OR Peas aw

   
 
 

STANZA Xxx, opqas ae a
Gid tutti fon di sopr' alla muraglia, Celidora a due man 4
Che 1a circonda un lunge terrapieno; Che ne-anche un vi
Già si fiorifee in si crudel barcagha Tanti fil d'erba gol

Di fanguinacci la gran madre il feno: Lane' buomini così
 

4
NONO CANTARE:  439.
o- - STAN ZAXXEL oo. STANZA XXX.
ee, jth Amiffame —. — Adafa di Coccio a questo,e quel comand,
Da toccatori fan col brandispocco, Ed all'un dane,e aun'altronepromerte,
d h Lacompagnia del Furbainnanci mada,

“Pere che della morte almen Ceffane,

'Se non prigion si fa chi è da lor tocce, Che refti ai fianchia Batiston commette
AIP incontro ritrovafi Sperante; Com Pippoyil quale (Pa dal' altrabanda,
WA) + Che fa menando (a sua pata, il fiocco, Ma egli imretreguardia poi si mette,
Wh E se gid le fuftanze ha difipace, E mentr'ognun favanza agloriasmtente
a Ei fiede a gambe larghe,¢ si fa vento.

 
 

Hor mand'a male gli buomini a palate,
+ Essendo già wtci i Soldati di Baldone faliti sopr' alla muraglia, ¢ padati oclla
PS di dentro si mettono alla difefa, Sinarra la bravura di Celidora, di
y edi Amoftante, s' accenna 1l valor di-Sperante, !a diligenza di Mafo
S eraccc pane wragtoe ut Coot cies A
La gran madre si se i fanguinacci 11 feno » Ci terra s'asperge di fanguc:
#88 Ounero nell” Lliade (petisind « =:
pm 8 di fangue la terra intrifa corre.
® La Gran madre per la Terra intese if Petrarca nel Trionfo della Morte.
elf SEG ORY O ciechs 5 if tanto affaticar che giova?
jeu 08 Tutti tornate alla gran madre antica 5
pone E'L nome voftro appena si ritrova.
  TOCCATOR?, Vedi sopra C. 2. tan. 60.¢ C. 6. stan. 44. 3
 “ BRANDISTOCCO.. Specie a' armein asta; simile alla picca, ma l'asta pik
corta, ed i ferro più' largo y ¢'pily lungo, che non ¢ quel della picca; ¢ credo
venga dal Tedesco froch, che vuol dir battone, € brando che da' Pocti Eroici mo-
derat si prende per Iipada, ¢ significhi Spada in ful baffone.. Stocco ¢ dal Greco
Felechot Lat. Pipes, candex, da cui è facta anche la voce feecco,  perciocché pri-
ma per batterfi si adoprarono le-mazze, ¢ poi si venne a ferri;( Orazio Serm.
1.1. Sat, 3. Vaguibus © pugnis dem fuftibus, atque ita porro Pugnabant armis » que
'pelt fabricaverat nfus i nomi potleduti già dall'arme di legno, furono ereditati
'dalle arme di ferro, che a quelle fuccederono. Onde Stocco, che in Germanico è
baitone, a nOi significa /pada corta, ¢ floccata ia ferita, che si da con quella. Brand
* jn Saflonico ¢ riz one, 0 fuoco; onde Brandispoccbi poterouo eflere cio che Virgi-
“tio lib. 7.¢ 11. chiaina /fipires, © /udes pranffas, ovvero obuftas cioè baftoni, 0

mazze appuntate col fuoco. 3
' CESSANTE. Si dice quel debitore, che essendo stato toccato da i toccatori

“pud esser fatto prigione dopo le 24. hore da che è lato toccato, ( del quale ato
me rt e. (a 60. ¢ C. 6. stan. 44.) ed il Poeta scherzando coll'

'Paclammo sopra ©.
egnivoco toccare, cide efler percoffo; dice che quello, che da coftoro è tocco di-
viene almeno Cefante della moree, se non prigione, ed intende che quello, che da
coftoro è ferito o muore; 0 refta vicino al morire, com” è proto ad andar in

Prigione colui che ¢ tocco « ' <

FAR il focco, Fioccare vuol dir quando nevica gagliardamente, € da questo
diciamo fare il fioceo per esprimere un' abbondanza di che che Ga, per elempio si
fa ii fioece delli uccelli, 0 de' pesci, 0 de' denari, ec. si direbbe a uno, che pigliaf.

se molti uccellt, molei pesci, o molti danari, ¢¢. & così nel preteate luogo inten-
de

%

 

 

a

 
 
   
 
   
  
       
    

440 MALMANTILBE

de che Sperante ammazzafle molti huomini con
il vello della lana Lat. foccus., Si trae anche come's' ¢detto
ve, che Marziale appella tacitarum vellera aquarum, La
in abbondanza, si dice Fioceare; ¢ stendefi anche r
aver dewo di Mcnelao: Poco dicea, ma bene, viene a dire d'
Atandaua fuor diluvi di parole 5 '
Come allor che di verno ilnembo fiocca y.
E fu pe' monti nena a! ogn' intarnos dohlgioee
MANDAR male a palate. Vuol dire mandar male il fay
gamente ed inconfideratamente. E qui ii Rocta ia Spe
vendo havuto per coftume di mandar male ii tuo a 0
l'antica ulanza di mandar male a palate ancora gli huomini 5 ¢d
con quella sua pala, concia male moltihuomini, '
A chine dd, ¢achi ne promette. Diciamo cosid' uno infolente
che tutto il giorno facia risse, perquotendo quand' uno .< quand"!
con questo dettato il Poeta deferive la,natura. di Malo di Coccio, il
s' ¢ detto sopra al suo Inogo ) era huomo di conversazione,¢ nelie tel
ordi, ne 1 quali si trovava, foieva vOler (empre sopraftare gli aluri
¢ ca Cf farsi ubbidire con le grida, ¢ tainolta con ie butie,
# gambe larghe. S' elprime con questo termine la commeaita, ¢ (pe
ginc,con la quale uno ficce a pigharh riposo;(¢ si dimotira un pimuo ¢
sare, ed amico dell' ozio, ¢ delja pigrizia ) che fidice: Stare iw Rane
C, 3. stan. 72, € C. 3. stan. 1, 60m s¢ mani in mano; Con ie mani in cintola, —
STANZA -XXXiLL STANZA AARIV,
Amoftante alt incontro un nuoko eAarte.  Vedendo i Terrazzanigbe stannoin fa
Senbra fra tutti anants alia testata 5 Che il nimico ad S[padeye gioca
Lo segue Pao C orbi da una parte s Ler non far Mole 1H fab MALCON KG i
E aa quest aitra Egeno alta franceta, Ritsranfi, ¢ non sengon pik
Vengonfiin tanto a mescolar le carte Ma speron ben ( moftcanaoas
E vien /pade,ebaston per ogni armata, Denari,e coppe)indurghs a far p
Ectidam puche,e 4 gsmocar none leflo a) si
Vs perde ia figkra, ¢ fa acl r¢sto. Speaiscon, che pario in
eile preicnu due otrave il Poeta dopo haver lodato per vaiorolo
seguicato dai Corbi, ¢ da Egeno, icherza in sull' equivaco del ginoco 5 &
sucne rai as/corso dai proverbio « Vengonsi a mescolar le carte,( che
€ \¢ ne Locca, O se ne 1iceve, Come vedremo sotto C. 10, Ble
auibedue 1 campi vanno ( cioè s' adoprano ) /pade, ¢ ha/tom, ¢ che chi
che ( ive urta nelle picche ) perde /a figura (che € una di quelle carte, nell
Ji sono efhgiaui gues fantocci, che ne 1 ginochi di daia tono te carte,
cive perde la propria perlona,e fa del refto ( cioé muore ). £ Terr
in fors, C1U¢ hanno i lor punto in fiori, ( ed incende tanao ip
Bria ) vedende che 41 nimico ad /pade ( cioè adopra ic ipade). Per non,
+ maitom: ( cloe per non fare un monte di mori in iu 4 mattoni, ¢ ¥
fui terreno.) ff r#tir ano da chore ( cide Jasciano J' ardire,) me tengon
Vuoi dike HU VoOguon più giuocare y ed intends non vogiion pil

  

 
 
 
  
   
   
    
    
 
    
 

  
  
    
 
 
       
   
   
 
  
  
   
  
  

 

gs gp emer Ee ae PP ee EsP soo eee eee FEE

 
 

« NONO CANTARE;

ano di ridurgli a far partite, cioè accordarli, moftrandogli

 

44t
i ddwari', e coppe, cine

 ofterendo loro dell oro: E pee questo mandano al Campo un' Ambasciadore,

che parld nella maniera che se
 STANZA XxXxy.

 Spida Signori ? armi ognun sospenda,

Ache far questa guerra aspra,e mortalel
Fermi per grazia; più non si contenda,
Per c! alsrimenti vi farete male.
Fate che la cagion aimen s' intenda,

| Ca cherichedi a questo mo.non vale

F

ni

it

ee

| Bchi pretende venga con le buone,
Che dara glifard soddisfarione,

'ntiremo nelle seguenti ottave.

“STANZA XXXVI.

Con queiyche dona per amor non s' nf4
4n tal modo ta forza,e la rapina,
Chiedere,imperciocch? giammai ricufa
Ui ginfto, ed st douer la mia Regina,
No entraron mai moschein bocca chinfa,
E con chi tace qua non s indonina ?
Poss' egli accomodarla con danayi?
Dungue parlace, ¢ vengafi ai ripari,

 L Ambaiciadore de 1 Terrazani espone la sua amba(ciata, ¢ chiedendo tregua,
-elolpenfione d' armi conchiyde che la Regina di Malmanule ¢ pronta a dar loro
fodistazione, pero domandino, che faranno efauditi.
|. SPiDA, Questa è una parola usata da j ragazzi ne i loro giuochi fanciul-
ye non hay ( ch'io sappia ) significato nefluno universalmente, ma nel
modo, che se ne servono i ragazz: signitica sospenfione di giuoco, o permuffione
@ eleacarsi per alquanto da efio senza pregiudizio, appunto come si fa con la fo-
spenfione d' armi in occasione di distide o particolari, 0 generali, ond' io crede-
rei che G potefle dire, che questa voce /pida futle corrotta da ssida, 0 disfida, I
- Fagazai si servono di queita voce così, per efempio. Wel ginoco de' birri, ¢ ladri
detto sopra C, 2. flaa. 32, quand' uno occa bomba 0 per qualche sua faccendas
on attenente al giuoco, vuol partire,per afficurarsi dal' efler catturato dice;
Spida, E con queita parola s'intende per lui fatta sospenfione di giuoco: E quan-
do il ragazzo, che & Ggnore del giuoco dice Spida s' intende sospenfione generale.
Ed il Poeta » che si ricorda che egli scrive una Novella per i fanciulli s' accomo-
daa i termini da loro praticati,ed intesi, facendo servirfi a questo Ambasciadore
della voce Spida per farsi intendere che vorrebbe sospenfion d? armi.

Cae hericheli + Chetamente; occultamente, senza parlare. Varchi St. Fior.
lib, 15. Per Ze cafe si facenano delle ragunate a chetichelli.

WON vale. Questo pure ¢ termine fanciullesco, se ben taluolta usato anche
dagli huomini d' eta, ¢ significa Non è dovere, Non conuiene, Non sta bene,
ec. Prefo per avvenitura dal giuoco, in cui chi scommette dice per efempio; Va-
dedi tanto ? E quegli che non accetta dice: Non vale, cioé non fo buona questa
erate 90 pure quando si fa contra le leggi del giuoco, si dice similmentes

NON entraron mai mosche in bocea chiufa, Chi non chiede, non confeguisce;
chi non parla non. inteso. Lo Stefonio nella sua Gnoccheide ato primo sce-
a prima dice,

hee pak Vulneris alcofti nunquam medicina paratur,

£ viene a fonar lo stetlo che con chi tace, qua non ' indoxina, Plauto nel Pseu-
dolo Att. 1, se. x. ove introduce lo [chiavo, che così parla al suo giovane Padro-
Ae Innamorato,

Kkk Si

 
 
  
     

PrViot ovoy

  

E poi conchiiile:
ina fuggire i litigi.

dice Così

“STANZA AdXxVIL
A quel sl General,c' ha un pod' ingegno
Kusene il colpo,e in dietro si discofta
Che si fer mina i suoi, aipoi fa fegno,
' Pala parola, ¢ manda gente 4 posta,
Ne bado molto a fargli har a fegno,
Chela materia si trove disposta;
Crascun a! amie le parci ferte faldo,
pC? ognun cerca fuggire il ranno caido.
STANZA XxxXvVill,
ch della pelle ha punto panto cara y'
ch che von vorrebbe esser nccifo
'empre de feiarre di fuggir procexra s
~ BYe mai c entra, ha caro esser ditsfo,
' Ben ch} ei. mostré non, baner pasra
S? in quel Cimento lo guardate in vif
Lisciato lo vedrete d' un bellerto
* Composto di giuncate, ¢ di brodetto,

 
   

* Ordiaa i) Generale, che si fermi il combatteré, ¢ trova i'Sol
dieatidivai » perché a ogauno piace il vivere; ¢ sia wno'coraggiols
mai essere, al cimento poi non haura careftia di timiore. Fermato gue'
battére, Chi era ferito s' ando a far medicare ¢ ah

PASS AR parola, & termine militare, che significa far fapete |
Capitano' per tucco l'efercico con dirlo a uno, che'lo dica a ua?
vada seguitands fiaché lo sappia ogauno senza che si faccia
Qi. Gli aatichi Capicani fa

fiziali fubordinati wa
si conteneva l'ordine di cio, ch

'di yoo! ior cl icvar qiaao da ij i

aes ie Hic Sen aes

eee UM AN TY Le Mi

v hominum parsi vem
6 °)\Nees te rogandi, © eileen ye x
Nunc quoniam id fieri non poteft
| Me fubiget, ut ve rogitem'; Fjord
Eloquere ut quod ego ne/cio, id tet '
“PVOS S? egis actomviarla con danari. Ci è egli modod
trovVar rant denaro, che aggiuitj questa ca

*''Dungue parlare, Quest' ultimo verlo pat tolto: di oda g
1, ove Teti patia al'iuo Figiinolo addolarato;

Parla; sis Wb habs sit digi by beds, ue

Tener la'cofd Wath tua'beentt ascofa'y mins aA

eAiciocché tu ae thee as
TA

  

oe

sa Pees

Dewo uli

 
     
  
  
 

1h te ada,

0G ease
Bao i as
ade

     
    
     

SE FLSFFSRPSR oS Staetes

A
Sien: Projo brau,
Se mai vengono a
Crediare che elo fan
Perec' a rutei viene il bi
Ech ela palferebban
Se lo potefser far con tor
snithewsate i a quella opiniiie
Di veder Cuanro viner fa
STANZA XXWK
E questi che badauane ax
in Malmanty, 8 accorfe
Che que; none meflier
Pero si contetaron dell”
Gai tagle alcuno impi
Hitri rimette braccia,e ¢
Altri da capo a
Echi fifa uae ed

SEF

pest

     
  
      
    
 
  
 
 
 
 
 
  
     

as
. =.

 

ih
nye
aaa

- o = & wo Z.

 
 

f es haat 1803 ae Udsarntare Teffera. Amminiand'

 
 

eee -

NONOPQANTA RE

} #88
'Siliodtalico,, eee etn ee ff
pases temmalumeaee €.con ording,.0 de Da by eat re
ea - oa Leib se sittin

ST ROV Oar niaseria,disposta. Veove. prontezza d! ubbidire » perch? cialcur

- inclinava a lasciare il combattere. Sante eT ae
\ \ EVGGARE it ranto valde  Buggire i pericoli,o le fatiche, ~
| HA care eferdinifoHa caro che-qualcuno entri di mezzo,.¢ impedi(ea i
tocombatreresiche queito vuoldire diwidere una quiftione. Lac, pugaam dir e.
elLilcio Lateadiamo tutte quelle mefture,, con le quali aicune»
sper parce-bellefi lisciang. ta faccia 5 che diciaino imbelietrarfe: decto [:con-
do aleuaisda wRerlerra. cio'. melmay fango. In Franaefe il (cio dicefi Farò, onde
ciog unbratwace 5 ¢44re ana farda, ¢ wna fardaca, il che figuratameate>
- Sluergognare uno.con mato pungente in pubblico, che alccimenti dice si; dur /a
 Ceretata 5 E. dare una cenciata fudices, ccacta dal coftume de' Ragazzi Fiorentini,
che il'di di yuezza Quarefima, quando ( per usare un loro idvotismo ) si (ega las
eal cioé viene ad.ctlere partita per mezzo quella Stagione di penitenga;
Peete ior abufo,ednfolenza batcono el vifo alla gente grotlolana, 0 fenipiice
pd al COntado.cenci intinti.nell'tnchiottro, 0 in altro fudiciuine.. Branco Saccheci
disse Dane ca Fare, ¢ dare nna zapare, per offeadere coa marto. Vedi sopra, a,

ae Pilla 45.0: base wid Ca Te ya
jit «OM Ne ATA, Latte rapprefo,, ¢ (errato in fogli¢ di farfara con giynchi,, e»
Gdecta ginncata, la.quale mescolaca con broderro, che ¢ mincitra, fata d'
Wlovauidette liquide con brodo, o acqua,¢ agrelto, o fugo.di limone,,. farebbe
ua color¢ fra ij, giallo 5,¢ il bianco, appunto come diventa ia faccia di coloro, che
& i da subito timore, sink, 1
ASN AD/ERL, Huomini fanguinarij: Da Mafnada, che vuol dire truppa.
l'di Soldatic: what, militum manus Ma per lo piii intendiamo compaguia di ajiaii-
at poeaid Aieada.
TIRARLA

fuori, Cio.cavar fuori la spada per combattere. Virg. vagina.

VOISN Gx

  

aetkbEiks =

TREES

RESEES

er aay;.
= SATEICVORE. Ecceffiva paura, ¢ spavento. Dicefi folo dal frequente bat-
'eres che si fence dalla parte del cuore in uno, che habbia timore. Se bene af but-
ter del cuore ¢ indizio ancora d' altre pastioni, che futte anno quivi lor, seggio;
“eae gran defio, congiunto colla speranza di vicino conféguimeato del defi.
rato bene, la quale pero dai timore, non è mai io tuto disgiunca,:
sualptelten'arctben 4 Jeegiert, Paciimente lascerebbono (tire dt far quella quittio.
'Re. ln un frammcnto di Storia, Fioreaciaa manoscritca, che dame oa tisfa di -
i ncarui il principio G legge: 5, Gli difero ua monte di villagia, ¢
ond 'ingiurie, ma.il Cattellano, che era di uci Soldat, che avg iano canioin
dt ight doula Cavalleria, se la palso di leggiers,¢ la Ciaaiogii gracchiare,
sgnattendeva.a star. deatro;.ed a i suoi Suldaci, che Jo pregavag» a ulcire, ¢ dare
vs, addosso.al nimico,, rispondeva; Lo noa vogity ultirs, percaé nog voglio cae
Se CEDER guurs [a vivere un poltrone. Con quelto termine descriviamo 490, che
yuo brighe, ac faciche, o.pensiert, a¢ meno f yuule esporce.2 rifthi, o.pe~
SS Ree: ye oe

 

MSS ERE ES

ae
ar eet

  
 
  
  
 
 
 
    
   
  
 
 
    
  
  
 
   
     
     
     

444 MALMANTILE

ricoli di forta alcuna.. Il Ferrario seguitando il Salmafio nel |
le che la voce poltrone venga da Police trunco, dicendo che:
andare alla guerra si trova che si troncaffero a posta da lor
dito grosso; B dovea essere usata tanto questa furfanteria, ¢
tali il soprannome, e furono appellati Azurci fecondo che
Cellino lib, 15. il che.volea dire poltreni; poiché Murcia pret
mava la Dea dell' oziofita, e della poltroneria, Origine et
non la credo vera, stimando che 1a voce polerone venga pill: sto da
poledro, ( come alcuni spiegano quel be/fie poltre di Dante Purg. )
Poltrone a.uno, che non vuole, 0 non pud durar fatica, appu 0
dro, il quale non è ancora atto alla fatica. Ovvero da poltro, che
fecondo 1 Landino sopra quel patio di Dante Inf. 24. che dice

Hor mai conuien che tu cosh ti spoirre,

Disse it maeftro; che feggendo in pinma

1n fama non si vien, ne sotto coltre.

Donde poltroni gli huomini pigri  ¢ dormiglioti, dice il

zione di questo patfo.

PREG Sk FS oye = oe

— meftiero da abborracciare, E' cosa da farsi consideratan t
cafo,
LMPLAST R ARSI con le chiare, Medicarsi con le chiare d' uovo le ae
di sopra in quetto C, stan. 4 A a Re
PARSI aar de' punti in ful cefs, Ricucired tagli, che ha nel vifo,: quale cae 9 pe
ma cefo, perché guatto da i tagli, non merita nome di faccia. Cefe o Fran a
se € parola nobile, che significa Capo, come alcuai vogliono, dal Gr. gi grps mH
nol ¢ parola di dispregio, ¢ significa vifaccio brutto. ae 'a
STANZA XXxxl. STANZA XXKX ui
Baldane in questo per la più ficura * Et essi andaron con la lor patente tp
Due gran Dottori atrattamentiinuia, Di poter dire ye fare, € alto ¢
Lun Fitfolan Branducci che proccura Lor camerata fa tra? a
D' haver se non po in Pifa,oin Paxia, Che gli seguia curioso per. =
<ilmeno in refettorio una lettura 5 Baldino Filippucei lor yy
ZL! altro è Meinforcon da Scarperia, Huom, che pis tofto canta py
ChefeVbuom vine per mangiar vi ginro, Crescer volea come gli altri appa e
Ch' ei vuol campar mill anni del ficuro, 3 44a si pent),quand'a e
. STANZA XXXXIL. STANZA XXXXIV, 9 &
Calfandro Cala Cheleri fra tanto Son alti gli altri due fuor di mifar «
Del Duca allora il primo Segretaria Ond! ei nel me? 0 camm
“ 7° loro un discorso di quel tanto Refha aduggiato sv hed)
evan dire al lo aunerfario Ne men pro crescer pits
Cacciatof, Giosieiae: ar
Escorso turto if [uo vocabolario

Scriffe in manierayefeceun tale Spoglio y
Che mefe un mar diCruscain mexico feglio,

 
 

 

NONO CANTARE: 445

PRES os | HROMOVE TH ANZ AX X "BV. lov
ella pure alor quiui's'inchina, Purche il nome conferui di Regina,
Dando a ciafennoi fut debiti riroli., Luando per t annenire altras' intitoli,
Econ essi ferme IL altra mattina Che questons le nieghin, chiede al mato.
| Mdiscorrere, ¢ far patti, e capitoli, Wel resto por da loro il foglio bianco,

manda suoi Amba(ciadori a Bertinelia, i quali con efla fermarono di
flabilire i capitoli della pace per la matuna seguente, promettendo la medesima
| Bertinella d' acconfentire a tutto,pur che le retti il titolo di Regina.
DE gran Dertors, Dice due grandi, perché veramente erono ambedue di. sta~
a ce alta, ed un folo di essi era veramente Dottore, cioé Ficlolano Branducci,
ai che ¢ Frdncesco Baldovini giovane dotto, ¢ spiritofo; ma perché nel tempo, che
i fu composta la pretente Opera era afiai difapplicato, pero lo motteggia, dicendo,
che egii proccura d' havere una lettura in un refettorio, se egli non la pud otte-
 Berein Pifa; 0 in Pavia. Ma non voglio già io lasciar nelle menti di chilegge-
 fala presente Opera I imprefiione', che questo Baidovini fulie lettore da' Retet-
fod t0rj, € pero dico, che le (ue beile, ed erudite composizioni lo fecero conolcere»
infin in Parigi, dove eflendu fate fenuite in diverse Accademie dall' Em. Sig.
ym Card, Chigi tino di la lo fece chiamare a Roma, ¢ lo diede per Segr. all' Em. Sig.
» Cardinal Nini, la qual carica eghi efercito pi anni molto Jodevoimente; mas
kit Beceilitato dalla poca buona fanita, che godeva in quel clima, se ne tornd allas
| patria, dove efiendo stato prowvilto d' una Pieve, quivi se ne vive godendo mag-
b,@ Blor quiere, ¢ miglior faluce, che non godeva a Roma. i
él MELN forcon da Scarperia, Pierfrance(co Mainardi grandissimo di statara, ma
G8 ware dottore. Questo per esser,si pud dire,un colotio, ed in ful fiore della gio~
veotl thangiava ati,¢ però il Poeta dice, che se 1 mangiare fa campare, ¢gli
(Ill Per viver molto tempo. L'iperbole di mile anni (e bene & di numero determi-
'ge ato; si piglia per indeterminaco, ¢ signitica lunghissimo tempo.
I * CASS ANDRO Cheieri, Cive il sig.\ Alessandro de' Cerchi Cavaliere, e Sena-
we tore Fiorentino Segretairo della Sereni(s. Granduchefla, e però ii Poeta lo fa pri-
mo Segretario del Duca. E perché veramente egli € un Gentilhuomo di gutto
"i isquifito, ¢ d' una cloquenza aggiuftaciflima, dice, che con la direzione del Boc-
sil caccio (le cuj opere regolano la lingua Fiorentina per efier' egli il nostro Cicero-
Ne ) ¢feorrends il suo Vocabulario ( cive il Vocabolario della Crusca ) meffe um mare
di crufea in mezzo fostio, ¢ (cherzando |' Autore con l'equivoco di Crusca buccia.s
uv del grad, ee CRVSCA Accademia Fiorentina, intende, che questo'Caflandro se-
id 'ce un diflefo compotto di parole approvate dalla medesima Accademia della,
», 'Crufea, nella quale si fa proteifione di pariare, € scriver pulitamente la veras
“| lingua Fiorentina.
7 PER far un diffefo di quello, che doveano dire, Cioé per metter loro in scritto
I Iattruzione di come doveano'contenerai in trattar 'accordo,si come si faa tutti
gli Ambasciadori,e plenipotenziari, che G mandano da' Principi, Repubbliche ec,
 FAR to spoglio a! wn libro, Mercantilmente's' intende copiare le partitede' i de-
 bitori; ¢ per altro s'intende quando si cavano da un libro quei concetti, tentenze,
'parole, delle quali ci voguamo servire in far qualche composizione.
POTER dire,¢ fare, ¢ alto, ¢ bao, Potcr negoziare, ¢ conciudere a lor gu-
Os

d)
i

 
e,
flo', € vélonta, the ih
dicono: Peni; j
patentee Bis

libero.

LALDINO Filippucci, Filippo Baldintcci d
e quelto intende il Poeta dicendo Huomo', che canta'ben
¢reicera più, perché egli ¢ duggiato da quei due huomini lunghi
e Meio, de' quali egli lo dice'parevte, non perché vera

eg ee ee

e accomodarsi alla rima. Queito¢
jamo detto sopra nel Proemro. ~
* LVYOGO

5 STANZA XXXXVIL
Eperché ore già finian del giorno
Siconfuled, che fulfe fatrafera s
, Percio tutti alle spanze fer ritorno
Com! un fatco digatti, fuor di Schiera,
I Cittadini Pavan @ ogn! intorno

* Welle radesfu i cantize alla fronciera, Che non si
Bicivcgh' ognun fecondo il suo porere Gis teiehnzs Gene Bl
© 5 foreftieri in ala dia quartiere, Sti Mab spefa dicey men Wid

a ST AN-Z/A*REXRWVNA ome DAVIN
©"Del Principe a' Vgnan pot si domanaa, Poeperre ners

“\  perché la labarda anch' egls appoogs
* 'Staffer attorno a rivercar si manaa;

  

  

un facco, a quait
LA quarticre »
fied a ME i

ee

 

uae

 

BG

anggiaco, Vuol dir luogd, dove nonatt
Pinterposizione di muraglic } 0 d” altro, EY Gail doghile pian 00
tate, € con poco vigore, ¢ i dicona auggiare; da Yggia » ombra,;
TENNE un mexro miglio di pace. 'Per mbitrar', che queni t
haveano le gambe lunghe, si servc di queste"iperbole'd? un imezzo mi
DA loro il fogito bianco, Apptova tutto quello'; che essi conchi
Joro Jil foglio, bianco firmato di tua mano,acctocche vi ferivano lee
capitoli delia pace, come pili piacera loro, 'Che ¢ Jo stesso, chedit
in voi in tuto, ¢ pertutto, In quelto fenfo dific il Petrarca ». my

"Chi Lhabbia racceteato, e chil' alloggi; x
Etiendofi già fatta (era ciascuno sbandd, €d i Terrazami tts
sex dar' alloggio a | soldati di Baidone. Bertinelia iawn Pala x

¢d il Generale, 1 quali accctcarono Pinuito. Si'cered deiDuca per co

'ch' eli in Palazzo, dove-bnalmente egli venne dopo quaiche di

© che non voleva parurfi dalla iocanda, nella quale s' era accomodato..

COME un facco ds Gatts, Cr0e lenz' Ordine, o'regola 5 ma con!

~ tende, che ifoldau sbandarono, chi io qua y chivin Jay come

Gi dja! andare.
rova aliogyio, Dar

   

  
   
  

aan ta
a i

 
 
   
 
  
  
 
   
  
 
  
  
  
 
 
 
 
   
 
   
   
 
   
  
 

os MBA

aa

: wa
STANZA XXEK
Grants a palarro Bi
In Amospame eC
E-wuol che (gli odj mai:
Stien feco 5 ma ciafe
» Puer' finalmence ne i preg

Se, es 8. SERS PES EL EETE RBPRERPS SR

S” era décniarovoue
Priaichiei n'wferfs
Nand per:

 
  
  
  
  
   

  
   
    
  

dort ~ 1
quarticre significara
ae Swan grote hk ey

  

      

 
 
   

em, Sista 30a sobre ipaaies
dA 12. epill. 33. quidem,
r Fer ime —m fed egoa egh, ur eee - Croe noo
wesmercnen gliteci croppe cirimonie. E apprefio. Pall pot C. Ca~
» Hlorum ego vix attigi penulam; ramen remanferunt Dichia=;
e ferraiyo'o jinuitare uo. aitaseawate » © pregario a voler 'rima-
co noi. £ ta/ciarsi tirare pel ferrainole, ¢ non accettarc |' inuito » € ari
Koa > '
CH! vs difagio, Quand' altri ¢ inuitato a un conuito aed
teatro. datalcuno.y.per licenziarsi da chi Jo tratticne ta full' ora del ¢o.

s te la-causa speria quale ei i parte, fuol fernirfi di qu:flo ates
al eons (a, non dia aifaeio: cioè se 10 fon caula, che egli (peade, aun ¢ dovere 5
'difagio-col tarmi aspettare.
“ ee ~Andar a mangiar a casa d' altri senza (pendere...:
operat ferraiuolo, o ¢appa.s perché 1n vece di quello ia porcano ful-
i:Alabardieri + i quali in occasione 4' avere aire a tavola  s¢ ne, spa-
ae appoggiuala-aila parece 5 ¢ perdo.con quest) decto intendiamo. Posare ra
ior (ad! aters5c.quivi mangiare, se bene Pe/are tl ferrainolo.s' ay
“4 'aucora'un giovane, che non ha provifione, ma serve in uo banco,, 0 'in who ff.
2ibegravissy baitandogl d' edereimpiegato, ¢ d” abuuart per poter goder€ col
oe
MWAMBRA locanda. Incendiamo reli Alberghi, 0 vero Offerie, che danno, das
 dOrmice a vforetticri.
SERA nce wiare. a era nome eed Havea eletto quel lyogo per' Abto
Fipotor, exis Wiens t
VOLLE mille Porei. Vole iacpdofiaith di citimonie y¢ lufinghe: ed. ¢ io. neiio hc
'chevwererderto: itopra'< Com fran Bche Janene, così dewto dal Latido vente c1oe
di corpo, ¢ gi fl
“WCODAZZO\, Incende seguito di gente “dictto.« Warchi Stor, Fior, lib. I2.Faé
al Primt Cittadini eli fecero codazrodietro, accompagnandolo, ¢raccompagnandolo gaila
we ius Cufanl Palarrxo; comes' ei fufféril padrone di Firenze,

      

hat

Ltd fate

oh “WHSPANZA thy STANZA L...,
A cena (perché il giorne in questo loco dn cambio di guarir dell' appetica ~~

a: * Lblebbertvairra faccenda le brigate, Facenano un collo come nna. Giz eff

; 8 arta cucinave intorno al foco ) Se vien frictate, og un Sana accinits,
wt Senses furia ds friteate', Che per aria chi puofe.la fearaffa;
od ¢ nem ipresba si 5 ma duran poco, Si riduffero in brene a tal partita.,
3 \Che-uppena farte ellveran gid ingoiate, C' ogms volta faceanoa rufa raffac,
a Presse gente a rauolaera molta', tn ultimo seguendo Bertinella. >»
gi sR, We" miangiawan dueye tre per wolta, L! andanano @ cauar.dela padella.
gf oWDelerivetarcena fatta'da' Bertinella a i Foreftieriy la. aleconfiflettga, in,
pt fritcate » mangiate con fa fiiria, che egli dice: paflo Reale, e cirimonie conue-
if se a una Regina di Malmantile.

iin fueria di fritrare, Beitvate in quantita; 3 Waa gran quantica di Fricta.
sopra C, 3. st. 50. EXIT:

eet

  

    
   
    
  

   
 
 
   
 

.

448 aan F EDR
PRITT APA SEE viv eda! factard WOVa bE
felid'pddella' asfoge ia aveortah,ielde mene a
125 appreffo 'atirort baslerebe dine 5 petcheirgioy
sce Sen eal: as tra ng “
GIRAFF-A, 'Avimale quadeupede § ikqualess se bene
fidema,€ s citaaaiea Dencaaneg eine toy -havil €onouid
a'quello del' Cammello'ylégambe'dinanai abo i quelled
coda j ed è del colore meuctia®,- che q
i Latini lo dicono' Camelopardalis y cio' bela Yeheticne! I
'Pantera, Pannoil-coo comenine ewafnd inwndealiangas Lio eel
interpretare 5 che non' fifazialleroy" perchemméareare | dial
cibo con gran'deiiderio';Latino-¥ehiare } 0) chesaliuagatiero ene
betas

 
  
  
 
  
    
 
 

a

 
 
  

    
 
 
     
  
 
  
   
     
 

pet vedere donde, ¢ quandowenivanolle Feiecace ena
refize'a tempo tuo fa menzione'ibPolizignd-nelie® pellance 5 » Gitiog

 

Scaligero' simil dit quetto:
ail Efercitazionie 209. nutn 3s OVedice "hei Perfiani Girmafa P. f
E Abts il BOM Gina Parse Omit » Ha" eH) bho mee
o STAPA accinite, Sravarateetito'y Teo} oprepataco sidal Laci
-didiatho stavalattento', <u'all' ordiné cones, tnleMtaro.chigmaw,

ho tifato i ahtivo's particolarmenté dation Villanty*s fempre in

spele fei ptovvedere'danatir: "Ora /peritintratctared! Origine softy
nendofi il danaro a fructo, la Corte' prititipale 9 siccome da"Greciy dalla
detta Capo scost-da nO1'fi ehiamo Capitalc; ¢ Fondo! ancora, dai tei idére.y 6
la petunia data a intereties a:pensa' di fondo 5 e»pedere!, orpotieth
ta'; Che'pérd:' nftra y come geactata dai danaro \y-the! ayprincipi
Greti Chiamarono Torr stioeParre, 1 Latiniyemes siqua nig
fu Ud Varrdne', ¢ da Norio Marctlio Oticrpabome apiraies p
posito'; ff disse Ia forte' pquafipecinia capitales principal ndan
che'da quelta pechaiirpolta 12/a%phincipio s hevenivd poirdngu

da' Holtri anticht Crvaren, voce che finulmentestrovatiun Gio «
la) éhé i Franzefi didero chewanee,'cioe rendita envratayda Chef, capo.Ora
cinire, che anche dillero, Cixamgare, ¢ lo fleflo, che Provvedere ti
<cidé & chiefact, aflegnar fondi's*¢ ludghi da rischotere; foraire ye:

 
 
  

See. > ea ae PERFEPE RSET ERE RRR ES

    

rnito, nogeiLefto » sircensp
. oP OPP Dee oes ai
y uP Via Con firia, come si-fardellescara

atrornd Peiitrelehh Voce alle vdice usacar; enn Jaycredo

i rofto 'fied' per bi 'iaS* a 1a asad
Pi Tn ape. Si dice' ido sono pili gente d' act
Gialcuno # affatina con preftezza's € (eti2"Ordine 5 O-regola dip
'egli pud'dic Shae Sad repair med, toa? inciutlese

i ¢ da notare 'Poeta | ' i
Pin pane sopraveiene fiipro
fritearemifttvie? dalle macenier Unica feu

 
         
   
  

ve

 
 
   
      
    
   
  
   
  

|

!
!

 

 

Stanchi di mangiar, non sazz}

Finito

BPA ficsass-

 

STANZAL
'at anna
Tal musica fini po poi in quel fondo;
Ma perché dopo cena sl vin lauora
Facean parzie le 'ior del mondo y

| Fra' akre Bertinella, e Celidora
inganancieree per burla un bale tando,
 Eapooa

4 0. entrouni altra brigata
Tal che si fece poi veglia formata.

sien STANZA LIL

'Fano poi com' è  usanka
Moite candele intorno alla muraglia,

 Lesplendor delle quali in quella franca
E sale, e tanto, chelagente abbaglia,

+ he diffinte si vedeva in danza

bt meglio capriole intreccia, ¢ taglia
Wannaccio in tanto [opr' alla spinetta 2
S' era mefioa xappar la Spagnoletta.

NONO CANTARE.

Z rel taano gestive insane discadess lnnetira nazioneda
orit quali dicono, che i Fiorentini fanno je frittate d'un' uova !'una per rilparmiare;
 & però dices che durano poco, ¢ per questo ce ne vogliono molte pi + si che per
sta ragione non è vero, che si facciano fortili per risparmiare, eflendo certo,
he tanto. 3¢ tanto unto si con/uma a far' una frittata d'un' uovo [olo,quan-
wm to a farne una-di sci; onde si viene a confumare cinque volte pill, perché unas
- fristata di sei uova faziera tre persone, ¢ fet frittate d' un' uovo |' una.non sazic-
un' huomo folo. Si che non di fordidi, ma di ghiotti in questo partico-
potion esser tatiati i Fiorentini, che fanno ie frittate di poche uova |' una,
inché fieno più cotte, ¢ più gustofe. Di questa verita si puo chiarire, chi non
erede, con fare a quattro persone due frittate di fei uova l'una, ¢ vedra, che
eranno fatica a finirle » come le finiranno ben prefto quattr' altri, a'quait fa
dieno dicci anche di due uova l'una, purché ben cotte, ¢ quetti si ridurrando
a rufa raffa, ed a rubarle anche dalla padella, come facevano coioro di
tile, Raffa raffa & lo stesso, che il Latino rape, rape, dal Latino rapere,
 fifece rabare, ¢ si poté ancora formare, rappare, come il Boccaccio in una sua
'manolcritta da fugam arripere, formd Arrapare, © dillero la fuga.
r « Leppare, voce della lingua furbesca puo venire di qui, o pil toflo da
vare, significando portar via con preftezza, La figura è ja medesima, comes
Tose dice Prometter Roma, ¢ toma, per avvcatura dallo Spag. tomar; quali;
E piglia, ch' 10 la fo già ua, ¢ tela dd. Tre agiole,¢ barugule. L. naga, varie,

mgé. Daa rufa è facto gure; scompigliare.

 

449
ei detrat-

STANZA LIID

Vn gobbo [no compagno wn tal delfino
C' alle borfe. più rofto, che nel mare
Tempesta induce; prefe un violino,
Che fonando parea pien di zanzare,
Intanto un ben dipinto mefolina
Si porge in mano a quei ch'ha dainitare,
Et Ygnanefe, al quale il balle tocca
Sciorina a Kertinella in fulle nocca,

STANZA LIV.

2' grave il colpo,¢ gingne in modo tale,
Che quanto piglia tanta pelle sbuccia:
La Danna, bench fentafi far male
Senx' alterarsi in burla se la fugcia,
No vol parer ma infel'ha poi per male,
E dice l' orazion della bertuccia
Sorride, ma nel fin par che riesca
tn un rider più tofto alla Tedesca.

» che ebbero di cenare i Conuitati cominciarono a ballare così in burla,

Ma crescendo il popolo riusci poi veglia formata. Così per lo più segue fra lay

dalla quale nel tempo di Carnevale, dopo le cene solite farsi

x i, si da ne i fuoni, ¢ cominciano a ballare fra di loro pa-

Ren, ¢ fenvefi da chi patia per le Se ¢ da i viciui vi concorre altro Boge
it: 1 e

 
 
 
   
     
   
       
   
   
    
  
   
      

ayo MALMANTILE

e si fa vera veglia di ballo, come segui fra questi connitati
quali essendo toccato a fare da mai 'del batto alla meffola
egli inuité Bertinella, perquotendola co! meffolino
che le sbuccid le nocca, di ché la donni's'adird,se bea non ta
ballo alla mefola si costuma in queste veglie per introdu
Jo, che è eletto Maeftro rocca con que! meftolino le mania
vita al ballo, e poi tocca le mani ad alcrertanti huomini, ¢q
vitate vanno a ballare, e nel ballare il Maeltro da il me!
ella va con esso a toccare tanti huomini, ¢ tante donne, € così
tri usano questo ballo con fare, che il Maeftro tocchi ante:
lato che hanno alquanto fra di loro, vanno senza meftola a
mini come ¢ solito, ¢ si seguita senza adoprar pitt la'mefola',” Q
si dice batlo alla meffola, si ta anche colla pezzuola, 0 Oy
lando si getta a quello, che si vuole inuitare, ¢ così di mano in
chiamato Ballo alla pexruola, 6
ST ANCH di mangiare, non faxxj. Stanchi dal? affaticarsi a maflicar pi
ma non già fatolli, perché havevano mangiato poca roba. Ll Petrarca nel T
fo d' Amore, nel principio:; ne
Sranco già di mirar, non fagio ancora,
Giuvenale Sat. 4. ragionando di Meffalina moglie di Claudio
Et laffata viris, nondum fatiata receffit.
TAL mifura fini po poi in quel fondo, Alla fine delle fini tal' opet
nd: Pur una volta fini. Latino ad extremum, tandem, aliquando,
C, 4. st. 9. in questo C, st. 1, alla voce Bordello, € forto C. 10,
ne po pot, ec, Vedi sopra C. 2, st. 73.

sR SERS TSE RPESEES

=
=

Ha SPR a=

  

   
  
 
     
 

a W

iL vin laxora, 1\ vino opera,fa la sua operazione con dar” alla teflaye '
briacare. Del suo lavoro, € della sua operazione si pud dire quel che difie} ka
delle pecchie. Ferner opus. i ty

B ALLO tondo, Specie di ballo, che si fa, pigliando più persone per! »
¢ formando così di tutti loro un circolo, ch' è forse Latino Choreas m
nostri Toscani detto Carolare. ee Ye

VEGLIA formata. Veglia vera, ¢ folenne con tutte 'le formalita, i 4
Vedi sopra C. 2, st, 46. dove teoverai Jutrecciare, ¢ tagliar capriole, & ie 4
st A

23. q
Nunn acco. Questo fu un tale nominato Giovanni, € si diceva
cio per la sua (ciattezza, ¢ spensicrataggine [ poicht fo nome &
del vero nome Giovanni; sopra il qual nome è da vedi tole
della' Casa ]; Questo infegnava fonare la chitarra 4/ed if
pochissimo come quello, che non haveva cognizione cna della
rd dice epee 4a spagnoletta ( specie di danza ) aflomighando il
cato delle dita in fu lo Arumento, a uno, che zappi: ¢ Spinerra
balo, o Bonaccordo,,
VN gobbo. Intende il gobbo Trafedi, il quale faceva p

violino, ma fonava assai male, ¢ per questo iI Poeta dice: ch
@i xanxare, aflomigliando il fonar di lui al ronzare delle

   
    
  

   
    

   
 
 
  

 

d NONO CANTARE,
'It! mipiccoli alati, co acutidimo pungiglione, Questo Gobbo servl alla Sere-
oleemmt aioe. quaita di Nano, ¢ per le sue facete manicre piacque
" salia Serentis, Arciduchelia Anna d' Auftria, chg o condufle con se, quando an-
do dove entro tanto in grazia al Serenils, Arciduca Ferdinando Car-
Jodi lei marico, che  arricchi non folo con li suoi gro fipendj, € molto piit
con I regaii', ma ancora con 4 denari, che questo generoso Principe si lasciava.
da efio nel Bins delle mane » nel quale il Trafedi era aftutidimo, ¢ face-
 'Ya grofle,potte, perché fapeva, che perdendo S, A, S.non voleva eller pagata,
lige se vinceva cra pagato puuwwalmente. E per questo il Poeta dice, che ip un di
Wh quei Delfini, che'predicono rempesta alse borfe, come vogliono; che il pelce Delfino
ica la tempelta nel Mare, ¢ perché quelto pesce pare, che sia gobbo, però
i ) per coltyine chiamar Lojfini, + gobbi, Mori poi questo Trafedi, ¢ 1a-
jit scid mece.ie sue faculta a una donna di camera della Serenifs. Arciducheffa, della
Co qual donna haveva tatco scmpre¢ da innamorato, con patto, che si maritafle con
un Fiorentino suo amuco, che era in Insprug, come segui.
1 MESTOLINO, Cucchiaio di jegno per uso di cucina: Diminutivo di 4zefo-
#4, la quale in Lombardia chiamano 44¢/cola, dal mescolare,
Ada inuitare. k4a da chiamare ai bailo, '
—— SCWWRINA, Chog batte gagliardamente, Il proprio di sciorinare & quando si
get ort > abit: di paano fuori delle caffe ne i tempi di State, ¢ si disten-
 dono per targlt pigliac aria, batcendogii con (curisci,( che dichiamo camari dal
pot Greev camaces) donde feamarare si dice quelto battere, per cavargli la poluere,
st © Per liberacgis dalle cigauole - E da queito scamatare, o perquotere j panni, ec.
igel Pighamo il verbo sciorinaré per perquotere, E sciorinarf? intendiamo uno, che per
 A gran caldo Gi leyi gli abiti daddotia; Dal Latino ara detta poi ora coll' o lar-
f £9, quale Gi fence, quando.ia plebe de' ragazzi con sua antica canzone grida al-
sath Ie matchere u carnovale efiora Ter, in Adelph, Accipiunds, © muffitanda in iyria
adalescentium eff. L' huomo se ladeve fucciare. Quivi Donato, Adafitare enim,
pe
4

4

 
   
   
  
     
   
 
 
 
 
 
 

Proprit'ef? difimulandi canfatacere. E Sopra. eHufficanda; Patienda, consideranda
cum filentio, Gc, ¢ dal sao diminutivo non usato orina, cioè auretra, ne riufei il
verbo Sciorinarsi, che ¢ lo stelio, che se dicetle,con Latino barbaro, ¢ ridico-
fo exawrinare. Netia Valdiaicvole dicono; scfobacare quando exopacare, cavares
i day'. opaco,;
1N buria se la fuccia, La comporta come fatta in ischerzo; dal fucciare-, che
"| fifa, quando G feate grave dolore; tirando a se il fiato,
| NeMivuel parere, mat' ha poi per male. Non vorrebbe, ch' e' si conosceffe;
mane ha veramente havuato diigulto. Virg. premit alcum corde dolorem,
DICE Porazione della bertuccia, Dice de] male borbottando, o brottolando
sotto voce, e così facendo con la bocca quei getti, che fa la bertnceia, 0 scimmia,
“quando@in rabbia, che pare, che elJa borbouti, ¢ discorra dentxo a i denti; che
-diciamo comunemente, che ella dica orazioni. —;
| RISO alla Tedesca, Rifus fardonicus. Kifo finto,¢ che par più tofto pianto.
In lingua Tedesca ridere si dice Jache; ond' io credo, che il noflro Autore, che
“haveva qualche cognizione di quella lingua per essere stato alquanto tempo ia Ia-
sprug » habbia detto ri/o alla Tedesca ° aon perché Bertinella ridetie, come fanno
12 i Te-

 

    
 
  

 

 
 
   
   
   
   
  

st¢
pero pla
argla
Fase bine ) Che fiand' similt 1
meazione.
STANZA' ae ai
Al Det veramebie pare Bratig? 2G Ya beffii ?
vse "babii bya A onde or Bhreded 00 ee arvlnesy z

Perch gii par a' baverle dato prano, Ci morde in quaiche part
ernei d baverla tocca a malo envoy “" Ech) se
Ma quando fanguinar vedde la mano,” “ts ee
Io mi difdico, disse, ¢ me ne penta y?\ "Faia
Finalmente to ho tl diauol nelle braccia, uel mespolino
E [ono,¢ faro fempre una beftiacci@ ha vette!

STANZA LVR?
Wer carargliene pena, è 'Biri
nonfacome,al paren h
Dror

  

Sl ¢ WSRoERRER

    
 
  
     
    
     
   
     
  
   
   

 
 

ae
Rin arap in Canberit ih fablerro 2 « © 1009 Syaadermapuoraee:
2” aaanai più TPinig ane Se raceme ie Casaliadonma,

: STANZA LVIL: STANZA Lk
He Principe'a quel oriad ) Wigule? emairep 'LLG ridsa\ Dortma ator come'
3 'a foggiradrd ¥2 Wictrtdto here', 192% «Bldipolaize mance,

= call tro 'du hhh VeAbite dhe JOU 2 « 28 IRE feowe/l ah aldorne gal
Co amore in tui vuol far le sue vendette, —- OUR GNMEGC cated fareRiifwe

Ui quel vive fhiattin combean picébio', "0 « Ds iene yx

erkriwet:

     

ess

“CG abiriih aiiplerofkdW maiekirE\2) 6 OG RLU enareeinura pei

IL mefolina 5 © quei, che glienc dette Di non moftrar in ranto 8
«NB per BP laa bdr qa Ol 19 LY pPERRE OS Gece vel wsuforat medics
SO Po igeres ia terrain Cente rnild pores |. 129 UD anguenre che teyfan

   

“0: Bj doe 'G mara vighias che ta Donna*faccit st gran/laniento pparéndyy
Osporer haverle® ad maida (anpucuccortifi,, che ib male
2 Se G7 QUEP EY pli HOU eredcvaYy-ripreh de se Reffo |) 2 si metre ivo!
WY CON Medica Me HE biedtlediAratite si feuop! namiorato 0
OCP ARS, enande torino.' Rifentirts'¢ dolerf tanta) 1c! ol amie le
>. Ona ABD Read, “ANfatiday AppehalNon giipard' haverla quai
Sreata ye da Stevicate; ¢ Srénravee dal Liarino fettentarey come owii ¢
z ati Cie, Si adPAtcic, ALE wir msiferdque fuftento..10 MeHor} cioè paul
yea thiala pera mi Condued's @ ati'reggo'.° Non folametite dicta
Sich ea oial caeueanae ” | @ mala faricay
C'ferto'y Batinovint' ech ytenid} catkanten\ B fitcometi dies 464
Bebe » cio' grandissima. Ho auuta una buona malartiey
1
Ou s

ee ae on ane:

imal¥orza'; pochissimo,.»vsn wed wy
  

 
 

'

ere as
icp. divenfamnenne 4 ne

races ne aa Ow at

 
  
 
  

sor we! iy 51
nym ORG Shots ovo

vale CMOS en no Mle ghinibizzef.
a RS nme S'S on ioeaeemaneenenie gee

 

 

wh ib sen oh
+ CANom artes adenine
a “od medesimo in lode dellt,Vimor.malancolico,..,. A drow Seqeeay Bw '9 sConUL AE

611 » Bvan fuggendo ogns altra compagnia, ) 2 ASWA bE
aaa SOL op Ae Cd ghizibizein 98 concerti y ¢ 4 CARTE 5 96 sui grktis WY
yo viens ©, Lheecompagman pur fempre.vada.. a8 iA sce s\c08 DAN
j Story Bior-liba.t5e.dige < acca, LAA AER, Seonpes ghicibiczands
Plow dy Dihoeae'D sua | ee
@bArvcca il-ticchio,.. Giiwien, questa volonts ».pen eG @ gape 0 afoul dal
« Branzcle,%ix.,. mosca caninay;,Sumili, ma Aatnabate Penie, bvalilla.s.¢ Al
adaliailillo.»che ¢,una-molca pungentitima » che infelta 4 i da noi ia
i coenaadad pacerba faransy quo tora aermorndeyert peta NB
mda S 4 APS AS
ae Relacanantciocanstiye! dolore. che; prova uo pazeiente,, quap-
See una fericafirmettelale, accto, o,altra.cola, simile.4.che, moktihica, e>
Corrodede. partielle de'iquali carpi acri, ¢ mordags fembragg..al, ate a
Buila difrecoje feriicanms ©PRAZRHO. —.sisshwo9 90. 9\ml)o 5
RAR ua tira 4 an Suniendettar uo mal Ceri: c0 a che a iactia a
UNOswiaw A oie ar re we-sfhow vow wh srsivsusily stls,, in
h\STAACCLA comme 4 pleebiawsE grand a.collera Bg i
' sofehiacciare Ggnificabatiene identi per 1a collera, —- per a ae; ed ha
piquetto: Gignificato fenz! aggivagerus come vom picrhio ma,tal Gmilitnding s ay
questo. uccelioiha propriesa: naturale dij batter, sceau cere
rofted.in fu sramiideg|t aibert per; fueginsdefarmu ¢ sliggual ce
concbellittima giz »che;¢ queiasiMope haysr, molgo., eet 2
'¢ ville uscir le formiche si diflende some morte sopra, quel amo,» €, Ca'
ladingua g, che éJunga 5 ¢.carnola.,¢ quella.distends opera il, medesimo a an 2c
ose formighe, vi vanao sopra.per.palcerti, ¢ quando.al Picchio, pare di haveruenes
——— abaftanea), ura ate taolinguayeddngoia, aDa.quett '0 uccelio deco in
» Gea Oryscalaptes » 0198: Pictinatere di quencery € InnLaty pics li.¢.formaso,probabil-
ovamente il, verbo Picchiare,cioè. batierese. chi batted demtlperila stizea,paresche face
lorfiedle romore,ca.tdenuyche fa ak prcchio cal becco », Plasto spel, pro-
Seeremersaniice srond Sai S108 OH,. ai )
= MANDA git Trinigante,eAacomisto,. Beftepymia >maledice tua tal Be,

WAKA

 

=

adil SeEEELEe dpe

 

   
    
  
  
 
  
  
 
   
    
  
   
       

a

454 MALAEAN TILE Oe
¢ suoi falfi Profeti ¥ pn eee
colle maladizioni, coprecesteats ¢ beftemmie oe
GV AIRE, RawmaricarS, eoeee aie: i's
gagnolare. Vedi sopra-C, 4. stan. vventura da wagire ee
guaina; perché i cani quando ne ae tocche,fanno um mug
gito de' bambitii'. 'Si pud anche dire, che venga da #1 ase i
rammaricarff dell' huomo. 1 wales Now, 2 bn R
comincia # ffridere, ¢ Puaire., a he wl
METTE 4 foqquadro, Solleva, ¢ mette axofgr tutti i vi
re, Soqquadro & voce usata dat muratori'y eee »¢ simili, ¢ v1
squadro, che ¢ quando per accidente d*
mancamento un pelo tirato, 0 strafeiaatonon pud fare” ib suo corlo,
rd cagiona, che git steomenti del veicolo, o treno facciano si ito at
per lo sforzo, ed affaticamento yche riceyono, eda Yale
drare, ¢ mettere a fogquadro iv vecedi Rordirecobromorey) &
/MBLETOLIRE, Commuoyerti } Intensrire « Vedi sopra C.
tini pure in vece di /anguere, dicevano'volzarmente ne! sane
eficr cénero., e moscio, pigliando la similitudine das real ¢
signitica erbageio 20 ortaggio; Auguito Imperadore formé una 5
rola, e dilie Serizare pigiiando ia similudine dalle bietule, ~per vi
languids'; non iftar bene. Vedi Suetonio'nella Vita d Augulto » Ove:
voci,¢ maniere particolari, che quetto Principe ulavaynel. par
Celio Rodigino lib. 15. c10. Now similmente, diciamo! fauna
si, illanguidirfi per il aah d'amore, B Bretolone pre a hu
mii fatta;
BESTIA scimunita, Spend spropositato senza jmenlitnaiasiiya -
zio affatto. Lisca Nov. 2, dts perché. ellaera ponera, a queste se
torre senza dote, ec, Scimunito; sciacco, Scimunito'é lo steffo che wren
Lat. incaftigarns, Gr. acolafes, che not riceve'lammoniziani;) €
fictti, monitoribus aff ah E perché questi, 0 simili a loro fogiiono eflere
ale il giovane deloritto da Orazio y Sublimis cupidusque,o amara reli
nix; E qual'é quei, che difvaol cio, che volle: come disse Dante nfs
ro nell' Fliade al terzo libto; Delle giowani genti rigogliofe Sempre per:
tere menti; cioè per dirla volgarmente hanno il ceruello sopra Jab:
@ che Scimwunito', che di sua natura yale Non ammonito, non riprefo 5
stigato, o che non vuol eflere amimonito, ne riprefo, ne galtigato; ¢
rio, € mentecatti fanno; venga' a 'signiticare /eiocco, e haomo dt
to, L' efempio del Bocce. nel Filocolo lib, 4. dove: parlando come
Il tno diletto ¢ dimorar ne! vani occhi delle foimunte femmine, pwd elle
voglia dire ancora licenziofe, immodette, intemperanti, ¢ non
ze folamente,
RAGNATELO. Ragno, infetto noto, Dicono che perm
dej cane si pigiia del (uo pelo, ¢ fiipone sopr' alla parte offela,
HO sopra C.6 stan. 6.¢ che il ragno, ¢ 'o scorpione aumpa
foper a la piaga che hahao faita coi loro morfo,suaino il pazziene

   

Page

eFEEEES

 REEES

 
   
      
     
  
    
    
    
  
  
    
   

ea

   

Se Ss See Se
 
 

'*
NONO CANTARE: 455
necredendo chest pezzi delmeftolino, habbiano la stessa virtù; lega sopralia se-
rita, che ha fatta col meftolino a-Bertinella, idetti pezzi Maforle Baldone:, co-
me Soldato bravo » haveva notizia della jancia,con la quale Achille feci Telefo,
ee nea sehen havea detto J' Oracolo, i, Qua
. iabit medebirur, Donde Dante afer. C, 31, disse:, '
lo) loi Cosnod! toche foiena la lancia y

he 14 5 0D! Achille y¢ del fu padre esser cagione

tHe Prima di trifta,e poi di buona mancia,
| -\Biérede; che il meftolino habbia la medesima virtù della detta lancia.

>

Buk

qt ALAN del Cielo » Quali che Adanna def Cielo, ¢s' intende orto rimedio per
at fanar male,»come fu ottimo rimedia per liberar.daila fame 11 popojo eleno
wiytt inane che. Dio git mando nel deferto.. diFirenzuola in lode del iegno fante
io, 3 > <oSy
se) sbiaib shoizwe2 S& uno'non mangia, s' un non si riposa,
lags i Osha il fegato guafto, ole budella,
Rab > Bgli è a man del Crelo a ogni cosa.

** Nota!che in:questo detto la parola:4¢an:non vuol dir mano, non, essendo pa-
Ola figurata'per apocope, ma nell*intera sua efieaza Adem, che così si trovan
scritta nelSacro Tefto quella, che Dio mando al (uo.Popolo (che noi poi chia~
jamO manna )¢tal man si dice nelia Sapienza al capo 16. che havetle ogni buon
x vien chiamata quivi Paze approncato, e appreftato dal Cielo fenya fatica
© pero iniqucito detto credo che fr debba intender e#Zanna y ¢ non mano per si-
se uba cosa ottima in ogni gencre; e-che cid sia vero y quando sopravvie-
he a*yao® qualcofa di suo gulto,suoi dire: #' wa manna,e non mano: ¢ se uno
ricercaté | se per un su6 conuito una tal vivanda gli piacera prisponde farò Atan-
adScome si Vede 'fopra G, 8. stan. 43. Se bene potrebbe anche dirfi», che collas
feta parola Gi aljndetle a due signincati, ¢ a quello che ora di sopra si è detto,
WMtan; cioè manna, ¢ dian, cioè mano, E ALano de! cielo potrebbe parer det-
ta'Colla medcfima forma', con cui diciamo di qualche rimedio., o medicamenio
cfitace Kyi ¢ (Paro la man di Dio, il che coceisponde a cid»; che dice Piutarco
fOnumM Conuiuialiam lib. 4. quacit.1.)cheun certo Filone medico,aicuni me-
'Witdinenti Reali, così decti perché erano da Re ) enon da Poveri, 0 per efiere
i*! fepreti di Ré jo per 1a loro eccellenza; ¢ che dal (occor(o potente, che se ne ri-
; ceveva y erano-chiainati /exipbarmaca, appclld com-particolare.appellaziones
mani degl Idaij.
jd) WPREGLAT-A, ¢ neva, Intrifa, sporcata, tinta, Da i venti, che portanan via le
i mmelecine Bal gran vento, che per le parti da baflo gli usciva dal corpo accom-
ip 'pagnato da qualche altra cosa; la-quale ricoprendo le'mele che sono quella par-
ce pitt eafnola delle:cosce, che forma il federe } ” alconde alia vita © costin un,
w? Cert modo fe' porta via; Si che il Poeta Meoppiando quel verlo 5 che dice. «Dai
md 'venti, che Portanan via le vele, intende, che la Camicia di Baidoneera tinta dallo
z)
6
è

RELILE wtuateale

'sterco ':
SQVADERNA fuori, Cava fuori de i calzoni,¢ la distende. Morg. Le chiap-
/quaderno con rinerenza, Dante Par. 33. Cio che per 0 uninerfo si /quaderna.

! tele, cid che ¢ sciolto, ¢ s(parfo per l'universo, prendendo la fimulicudine da'

J libri sciolti, ¢ squadernati. DR

  
 
   
 
   
     
 
 
  
 
  
   
 
   
   
      
 

'436 MALMANTILE |

DIRGLI manco che meffere, ec. Dirgli iurie
tia differo i Lat, ed il Lalli Bitar kon by eHloee Lich
è Teitt m' ha detco peggio che meffere. 6)
Molti dicono + Ase/sere él' afino: ond' io stimo:che dic
che meficre s' intenda, l'ingiurid più che se gli havefle
Comico Fiorentino nella Moglie Ato 4. sc, 10. in-derifione del t
dice: Si; Adefsere ¢  safine, che va nel mezzo. Quali dica:
quando paffa per le strade gli fa largo, eva nel mezzo,
BEL vedere, 1 bel di Roma ¥' intende it Colofico ycheinoi
Ciamo Culilco; eda questo per belmadere y Obel di Kama iptei
che Bertinella pericolava di moftrare alzando le gambe.
Bellofguardo, son nomi di juoghi, ¢ ville nobilidime nel Fic
vato, ¢ donde si scorge molto, ¢ bel paefe.
eHEDICO da fucciole, Medico spropositato,e dipoca:scienza,
mo i marroni cotti col guscio nell' acqua, ¢ preadono tal nome dal /ucciare 5§
fanno i ragazzi per trarne senza aprir wutto 1 gu(cio, la pasta, che vi & dent
E perché questo cibo ¢ vilissimo; pero foros iamo da si i
nulla. I Latini dissero bomo manct cioé di niua pregio,
fico; per Naucum intendendo il Gufer, o buccia di quaifivoglia cof
la, che si bucta via, ¢ aon buona a aulla,
LE fa veder le luccicle, Le fa pianger per il dolore, Quando uno}
tale, che gli muova. le lagrime, pare al pazziente di veder per ari
14 di minutissime stelle, simili alle lucciole, il che ¢ cagionato dall'

lagrime, ¢ che pafiando sopra alle pupille offende, ed altera la virtù v vas
Oe STANZA LXL STANZA LXUL,

Non dimoftra la taccia così mefta S' impiccherebbe, ma dall' altro:
Quel ragagxo scolar,quel cauczzmola y Ei va pos retinente,e¢

Allor che motti giorni e (ato fefta
E che finita poi quella vignuala,
Ji mataderto tempo ecco s' apprefta,
Ch' e's ha di nuouo atornar alla/quola,
We si gualta belando si la bocca
uuand il matftro col bajion to chiecca, Gli vada in (u le forche
STANZA LXiL STANZA Lxl
Qrante cambiate in vifo,¢ mal contento, Poiche '1 cundotto delle pat
Adefo pare il pouero Baldone, S' ha da ferrar(dic' egli,
\ © ha nna stizna,ch'ei si rode drento, Perché si ia leva alle sue.
Per non bauer ceruel, ne discrizione,
Che benc' altrni la morte dia [pauento,
Se e' non fuffe che e'c' ¢ condenvariong
Achis ammarza pena della vita,
Con una fune baurebbela finita,

Con quella mane' alei dif

 
 
  
  

un mutmameess ofa. sealers a2 ESEG~0FE

oe Se
 

  

pas sro LIDUE

 

8 ois! me nyo 34 fan ne'iipauni,
Ps em wre - dif oe vs onngoy eRereheymensre th' I ami, ella v anuifea
'chap yh Ob bu) Chomsany, u lite
a ye fooppia dali th Gi Sent habbia on acgquauite.
intovaci) Poetara Sabeshens cepaeha cheba baleen
ps9 sorim re eran moped Da quefio a srcorgentof Bering
di.lei 3-4 h

 

a iste ine goticiod si sbaseni tmodh 1b i od LE": sy sais: Nas. aoe}
ianbopkematendonstetniens faney Olaltea forta di)logame, con
eaeioees ed<alere:beftiefimili + Evcaves.t; filidice ancora,

fa merce) collooa: malfastori pquandog)' i =
Oy: 6. Saatgo! aah Eda emadl noiidiciamora un rapazzo-malig

    

 lioiywenn\ LiVai facendorparlare tn Pedantésdice 2a (09 11309 isos to att?
doped d jwoda, Hee, Seana & osjuy 1k1gGs esm2) 901813 Iq issagayi Onnst
ab omtetharion Seieababtanitioher ants omiiiiv s odio olaup dioieg ¥
of Mey aba so, O folerro-vrifar ciferory PO Over owsd Orstid imaed bellow
OR iiteade 4a sesete st O, eyed |i ccaaiate marae) 19g ¢ Colt
' A quella id nines ehocien en quel-pdatemps,
itp aeaast iid? eavpolonannavOaiee:
ivcen se sar bebe vignnola', l'ela dnrage pes inten:
credo che sia pata He Tulnadaeeibe deseo opera

mad C.9, nf eke an la) Petite Tn: Eictirhnadutemne 'a a' Buohs

cL aa

   
 
   
   

uh fa già'an val Sse da 'Panzano, ib quale havendo din' fola pic
ue co) Poe facevz a ex faye barilittivino; ede ae,
on rei hi achiorbailt, ed haVeva 2*OB Hi Torte frattesyichie tro -
7 AO' al aon eglnop it si meow rubsiidoltuva,

mee ore Ha" mvae ¢ fempre dieeva";' 'Chie'ratdoplievaer bani bofa
nefla; rem POscoPLe®, thie per fuor bifog ni" tai vende iadetenvigng yes
fre do pia Fitoperta della derta'vight', HoH potevarabare 'come
ee: or Salis fo S*aFri(chiava a imbdttare ance witio's pet fo” che
dom abdate alii fo? amici 'da CHe*procedeva' phe eel" err 'vino 5
edali Prisporidéva, che era fina la vignuota ae a teiccoe dice: il
pad eifer che” venga it i decacos ee ae a) “wip reheas
ov mang an =q oon

rig Pgh Syeepoda balie Roca Pane ©.°6. fan.
nares @¢ fo steffd', titi duc verb} stei'dal Tao's IP BalerNov, 7,
ane A ractomandana ' a phi poteres; e coloré antendtnand a vbinctirts 'chi di

pinseebed ls sha olehnon aol,
te wna fine baurebbela fin ita: Haurebbe fiaito ee 'fub" CRN aziO' ton-im-

TANTO, 0 quanto, Termine, che significa piccola quantita, ed 2 lo feffo.
che par un poco; alquanto, Petrarca. E tu, se taxto, 0 quanto.d' Amor senti,

4 un sopractieni', Parca waa folpenfione, un preety di foptatrenere;
' 'Profiiagato il termine. Mm coNn-

bie —

 

 
  
  

 
 
  
 
 

   

453 MALMANTILE >

CONDOTTO delle pappardelle\, Cioè la cannaidella gola,
del cibo detto da' Greci Ocfophages, ¢ da noi scherzofamente él condotto de! ho
che risponde alla parola Greea significante il porta cibo, o i} Port i
piglia pappardelle, che sono lafagne corte nel brodo di carne pet ogni cibo,
ti chiamano pappardelle la ricotta stempcrata con acqua rola), eu ova 5 a
¢€ poi fritta a toggia di frittelle.

TLR AR le quoia, Signitica morire, come dicemmo LC. 4. 20,
scherza, moitrando, che pet la legge del Taglione si gattigar le gu
( civé la pelle ) dei Duca per haver egli commeffo un delitto nel
nella, rompendogli quella della mano, € seguita lo (cherzo dicendo, «
morire in /« tre degus  [ che vuol dire in sule forche ] perché con un
col meftolino J fece la decta ferita nella mano di Bertinella; ¢ di pil
Ballerino a vento (che vuol dire ballerino da qulla ) per moftrare:
egli commefio ' errore bailando, farebbe gaftigaco con esser fatto mori
do, come pare che muoia colui che ¢ impiceato, Vedi sopra\C, 2, st
re un ballo in campo aczurro; che ¢ lo stesso, che Tirar de' calci a Ronaio,
vento Borea, 0 Tramontano, Quel che sopra dice: in /u tre legni per i

forche; è simile a quel di Plauto, che volemdo intender Far, cioè ladro 5 difles
trinm literarum homo, vel
FACENDO il Nanni. Facendo il goffo. Fingendo di non badare, oofferm
re,Vedi sopra C, 4. stan, 26. Moftrando di non s'accorger di quel che faceva Bal-
done, facendo le vifte di non vedere. *
SCOPPLA dalle rifa, Ride secgolatamente. Vedi C. 3, stan. 66, alla yo
Pimmei, ¢ C, 7. stan. 66. 0S

 

 
  
     
    
 
   
   
   
  
   
  
 
 
    
   
    
 
 
 
   
     

&
Been SSB SRS Ewe ow o=

PER l'ablegrezza non puo fear nei panni. Si rallegra geandemente.
capir nella pelle. Per il gran gusto Gi rallegra tanto, che non trova qui
di sopra C, 2. @aa, 69, Piatone acl Carmide, poco dopo al principio, volend
esprimere una gran paiione di piacere, ¢ di gioia fa dire a Socrate, &
pitt in me flefo. i 0 ae

cANDARE in fumo ad acquauite, Risolucre in nvila. Suanire. Lat, 4
re. Sidice anche in tu:no d' elisire, Od' eferuite, sopra C, 3. han. 52.

STANZA LXVL STANZA LXVIL-
Atentre Baldon qual fempluerto uccello, 4a ridan pure, ¢ faccian ci
Coast d! tntorno alla cinetta armeegia Per ch' ci vuol far orecehieds Merci,

 
  
 

Lo burtino te genti, Amor ta,
C” ad ogni mo farò fido,¢
Come talor 3! abbrucia ico
“i garto al fuoco, ¢ frau

ed tuti guint serve per zimbello,
Senzache mai vi badi, o fen' annecgia
Ogun lo burla, e dice; Pelle vello;
Crafexn dice la sua,ciascun motteggia,, «

 

Beato chi pu bella te la feianta Baldon già fenve tl fuocose
E pot leuanfi crosci dell ottanta, Aa com un pan di ”
; STANZA LXVIH. 6

ne ot See wa

E cos} wa,per ca principio Amore, Ma nel getrarla allor
Par bella cosa, efembra ginsto ginfto Perché riftringeye rides
Vira pera cotogna, il cui colore y Ecosi Amor, al primae ani
Odor » faper aslesia,¢ piace al gxfboy C! allerta y ¢ piace 54

     
 
 
 

lal

aa,t
ie
ib,

NONO CANTARE. 459

STANZA LXIX,

Ed agli cht impaniato, € 4 qualche fegno ta lasciamla per hor cbt io'fo diferno,

 Credeil suo amor da lei esser gradiro, Che quefho canto refti qui finito,
 Altero vanne, ¢ fhima a' esser degno, Perch? dife un Dottor da Paleftrina

——— Diinuidia pitt che d'effer moftro a dito, Breuis oratio penetra in cantina,

. era così fit di la, che faceva mille me-
lenfaggini 5 per le quali era da ogauno burlato, ed egli Fingeva di non se n' ac-

c » © continovava a fare (cioccherie oftiaato in quell' Amore, come tal

volta @ un gatto oftinato a stare intorno al fuoco, ancorché si feata abbruciare.

4 Poeta adsaugiia Amore alle pere cotogne, le quali dilettano con l'odore, col

colore, ¢daano gusto nel mangiarle, ma si dura poi fatica a digerirle, € diven-

do che Baldone si reputava pitt degno d' esser inuidiato, che compatito, termina

il nono Cantare.
| CWETT A, Vedi sopra in questo C. stan. 22.

SERVE per zimbello. Servc per scherzo di tutti. O pure per allettatore degli
altriamanti a venire ad amar la sia Dama. Ii Malatefti parlando in persona d
un villano mandato d' oggi in domani, ¢ burlato dalla sua Dama, disse;
' Da poi ch io ho fernito per zimbello,

E fon andato trenta mefi aiont

Gridando per la rabbia', e pel ronello
vibr od Come fa il gatte quando ha i pedignoni
ud id « Alla mia Betta ho pur dato? anello, ec, 7
DICE: vello vello, Termine, che fenifica Derifione', quafi dica; guarda, >
guarda lo feiocco, il pazzo,o simili, ed ¢ lo stesso che Esser moffrato a dito per de-
rifione, che vedremo apprefio nell' ottava 69. ¢ che far lima lima dittro a uno vi-
sto sopra C. 3. stan. 37.

MOTT EGGIARE. Burlare, o beffare copertamente uno con detti acuti, e+
mordaci. 1 Greci di C diare uno; noi p biarlo 5 egiarlo, Da
motto, parla; che si piglia anche dagli antichi per fentenza, 0 concetto, o det-
to intero; B Azorsetto, cine breve detto, e fentenziofo, come fon quelli intitolati
Motterti ne' documenti d' amore di mefler Francesco da Barberino. Asutire, loqui
disse Fefto, foggiugnendo 1' autorita d' Ennio nel Drama intitolato Telefo. 2a.
am missive piebero piaculum eft, EB ttimato un delitto a ud plebeo il far motto,cioè
aprir bocea, ¢ parlare: onde Azertegesare non è altro, che parlare con qualche.
bel dettoy @acuto. Dal Greco Azythos viene il Latino murire, €'| noltro Adorze,
Ui Casa)però nel Galateo col definire i Motti /pectal pronrexza, e leggiadria, ed
oftano movimento a' animo; pare che in un certo modo lo faccia venire 5 O pures
scherzaquafi, che venga'da A4oto, movimento.

BEAT O chi pik belo te ta frranta, 'BE' lodaco colui, che la dice più bella in bef-
famento di Baldone; ci serviamo dell' cpiteto bearo per felice, avventurato,
fortunato.,'efimili (come se ne serve il Poeta anche sopra C. 1. st. 29. come nel
presente:luogo-, cheesprime, Fanno a gara a chi più bene lo burla: Latino Cer-

 

 

 

sare conuitijs ) Petr. i
NED Beato venir men che 'n lor (Rhos:
Me piit caro il morir, che viner fenxa;
= Mmm 2 Le
& ca

q

 
    
 
 
 
 
 
 
  
 
 
    
  
   
  
   
   
   

460 MALMANT ELE:

LEV AN crosci dell! ottanta., Si ride fmoderatamente. La vt
quel bollore gagiiardo, che fa la pentola, Fee cra 9 Op
€ si dice croscrare dal fuono +. ik gal verbo
Dan, Inf. C. 24,

O giuftizia di Dio quant ats
Che carai colpi per venderta crefeia oi

Tl termine dedil otrawta significa squifitexza., 0 ps
ne logico a to: © forse daile, ralce specie dipannine; le quali
tanta paiole sono a buonidimo grado di perfezione 50 finezza..

Ck ALECC!, 0 cicalices. Dilcorfi faytida pil persone insieme
priamente dire Discorfi dell! azioni, ed snteredi altrui con.
di bene: éd intended per Jo più, Cigalamenti fatti dadonn
digiorni, novellieri; per questo quando si fente Ree nuova
dice € un cicaleccio, 0 una cicalata. >

FARK orceciue dt mercante. Finger di nom ascoltare 2 © nan.
che altri ti discorra. E propriamente s' intende far oregchie di mercante coll,
che efiendo richiefto di qualcofa, 0 riprefo d' 4leun vizio non
richiefte, 0 non si emenda agli avvertimenti, 0 riprenfioni... Si dice piantare me
vyna lopra C. 7. It. 39. Far conto, chee paffi l. Leperadore. ne to. si

COSTERECCT, intendi le Coftole: Li coftato..

EVN certo imbroglio, E' un certo negozio imbrogliato, is difficile, cele
mo anche ana cosa così fatta, intendendo una cola. che pon ha eo del banat
del giufto, dell' onefto 0 del fattibile. ons,

WEL gettarla, Dicono, che la pera cotogna viloinga il venton-a coed stil
mangia, ¢ lo rifecchi rendendolo stiticho, ¢ però dive;Vel.gerranla da dolore se
piu lotto dice; Nel fine ti vogtio, nello smaitirla si man. ia fuori
mu dica le ti riesce così di gusto come pel principio s:cio¢iquando lama

41d impaniato, E' rimatto prefo alla pania, come rimane-il pettiroflo
do ja Civetta, intende s' è innamorato 4moris yorte dmplicitus y aK or
parazione, che ha fatta sopra dicendo,
etientre Baldon qual, femplicerta angela 2

". Così d intorna alla Civetta armecoia..
Quando uno ha male grave, da non ne potere ( non iisimene errs
dichiamo; £g/i ha impaniato, eq o¢ eam

ALTERO vanne, Vedi sopra C. 8. st. 30, Qui-vuol dire gout,
mando, che questo amore Jo renda degno d' eGere inuidiato per haver
bene, come stima l'amore.di Bertinella, che d' eles ¢ompatito del
d' cllerfi innamorato di coftei. B così si da.a,credere digodere ogni
sapendo, che come disse Erodoto nel libro intjtolaca, Talia 5-2 meglio
diato, che compatito; 1a quale fentenza colle efi parole appunta, a
fa l'usd Erodoto, dichiamo noi comunemence tutto giorno; E.chee ji: ue
ce Pindaro nella Raccolta morale dello Stobea eHMiglian Minuidiak F,
le quali fentenze dalla nostra plebe ridotte in una Cantilena Fiores
Così ¢ sa sincoomate

Meglio ¢ inuidia fop| tare h

Che di se compajfion dare,

 
 
 
     
 
 
 
   
 
    
 
 

 

    
    
  
 

    

  
 

NONO CANTARE. 451

 DOTTOR! di Paleftrina, Se ioffapeti, che Catone haveffe detto. Brevis ora

Caios crederei 5 'che volefle dir di lui, perché fu originario di Tusculo,

di Prafeai »eche havette pigliato Palefrina, cioè l'antico Prenelte per Fra-

7 € S'i0" fapeti » che un montambanco, il quale si faceva chiamare il Dotto-
redi Paleftrina, e faceva da Attrologo fufle solito dire tal fentenza, stimerci, che
ee questo, Ma intenda di chi egli vuole, basta che con questa fencenza
dai opps ha voluto significare, che i difeort brevi piacciono inating ai

2 icantinieri, ( perché ne' suoi Originali trovo una volta im excints,

'ra volta i in cantina ) ed in fultanza intende, che ancora gi' idioti amano,e>

ei eee idiscorfi brevi.
fo
ime i

   

  

nt FINE DEL NONO CANTARE.

DECIMO CANTARE,
Peeabasdlasibabastiasdbarls 8

ARGOMENTO,
Per far la Adaga col Rival quiftione
Va y ma in vederlo pot le spalie volta,
E, con lui dietro,
Ove ¢ la gente per balare accolta,
Del Lupo in traccia Paride si pone,
Ui trova,e'l prende con induftria molta, we

ugge nel falone,

i E uccifo quel, da fine alf avventura,

STANZA I.
wanti ci fan, che veftono armatura
0° Dartor di feberme, e ingoiator di fquole
4 ditminedaces » che fanno altrui paura,
© Premar la Terrase [paventare tl Sole;
' © BE ratcontande ognor qualche branura
f

os

Sempre ogn'un cone parole;
St fda sl cafo di venire all' ergo,
Labial om! olia, poi voltano «| tergo.

STANZA

tpien mofirain zucca bauer del Sale,
hb ee jon [fanio fempre fugge ta guiftione,
Anxi veder facendo quanto ei vale
odMebpicare al bifogna di spadone,

| Ed wu tal guifaé liberate il Tura,

| pene Reps ep pe geste eer aS

we

STANZA II,

Mae fon da compatir fee fanno errore,
benché non fembri mancamento questo,
Se chi 4 menar le man nonglidailcuore
In quel cambio a menare è piedi è leo,
Ob mi direte: Vanne del tuo bonore

Si, ma un po di vergogna pala prefto,
Helio è dir: Vn Poltron gui si fugvi,
ee: qui fermofi un bravo,e si mori.

L,

E che ( chi a neffun vorria far male )
Sa ritirarsi dalt' occaftune y

E Senza pagar tafteso chi lo medichi
La campo, che ai ni re se Fee

«dh theme

 

i eee}
 
  
  
 
 
  
  
 
  

462 MALMANTILE.

STANZA TV.
Ma voi, che di question fate bottega
Credendo immortalarui; e che vi giova
Far la spada ogni di com! una fega y imparate
E porni a rischise far ogni gran prova, eg
Il nostro Poeta volendo deferivere nel presente Cantare la di
lagrillo a Martinazza, per la paura, ¢ poltroneria della
segui, s' introduce con dire, che quei Bravazzoni,ed Amm
pre discorrono di far riffle, ¢ quiftioni, quando si vien poi ai
ratamente, ¢ loda il lor pensiero, contiderando, che 1
la vita, che far fermo, ed esser' ammazzato per il vano pretesto di rij
eche non pud esser biafimato colui, che non havendo cuore a menar |
mena in quel. cambio i piedi, ¢ fa intanto un' azione degna di lode, fug;
male. Conchiude al fine, che tali bravi, che cercano d*immortalarai
ro bravure, ¢ smargiafferie s' ingannino, perché dopo la lor morte:
ur minima menzione di loro: Git esorta pero ad imparare da i
DOT TORI di scherme,e Ingoiatori di (quole.. Cioé che fanno da mae!
ma, ¢ che si prefumono di faper tenere in mano la spada meglio di chi
da nelle squole di scherma. Ma qui scherzando.con I'equivoco di (quola'
che cofioro fon bravi mangiatori, poiché ingetano /e /axole, che fo
ne fatto di farina mescolata con anici, ed € chiamato fquola,
figura d' uno strumento,col quale si tefe,detto corrottamente /guola
dixs, come vuole il Ferrari; ed è quella cafletta fatta a foggia di na
ro chiamata anche navicella)entro alla quale s' adatta il cannello pieno dil
paflarlo a riempier l'ordito: Si dovrebbe dire (paola, ma l'uso ha
ja notizia di tal voce. Dan. Inf. C. 20,
Vedi le triffe, che lasciaron U ago
La (puola,e il fufo, e fecerfi indovine,
E nel Purgatorio Can. 31.
E, tirandofi me dietro, fen giva.
Sour! esso ? acgua liene come pola. ?
FANTONSACC!/, Huomaccioni; Huomini di statura grande; ma dicendol
Fantonacei §' intende in un certo modo grardi,e poleroni,o difutili. B dict:
Galeonaces, @Uanizoldacei, ec, Omero nell' Liiade lib, 3. introduce Extore,
del male a Paride suo fratello. £ tra gli altri mali, che gli dice, unoedi
marlo, Eidos ariffe, cioè un bel fantone, d'ottime fattezze; 0 come meer
significando la bellezza del corpo,disgiunta daila virtù dell' animo;un
wn Dongelione, 0 come dice qui il noitro Poeta; un Fantonaccio, cio? che!
moftra, ma ¢ poco buono a auila, *
AMMAZLAR con le parole, Legiones difflare spiritu,come disse Pl
dato millantatore. Pretender di farsi stimare, ¢ temere col dilcorrer
ritie, quiftioni, ammazzamenti, ¢ con efercitar fempre con chi fil
arrogante superiorita. Di-quetti parla Famiano Strada Jib, 2, Pro
Gloriofi ifti duces. Det homsnumque contemprores, & gut se atijs faci
Calo minitabundi gre 'p ATLis, Guam profil d 08

   
   
   

      
    
   
  
   
   
 
  
   
  
 
 
   
   
  

DR Pweg er ge ep roeae: =. wa

Sener

  
 
   
  
  

DJECIMO CANTARE: 493.
tini chiamano milites gloriofos, questi vantatori poltroni, de i quali intende il Poe.
ta nel presente luogo,e se ne dichiara col dire: Se view mas il ca/o di venire all'ergo,
~ ifica, se vien mai il cafo d' haver ad adoprar |' armi, non parlano pili, ¢
fuggono, che € quell' abijcere Clypexm de i Latini.

VN poco di vergogna paffa prefto. Quel poco di roflore, che si ha per una cosa
mal fatta fuanisce, efi disperde: Seatenza usata, ¢ praticata da coloro,
che fanno poca stima della riputazione.

(i MEG LIO ¢ dire: Vin Poltron qui si fuggi, ec. Buona fentenza,¢ vera, ¢ prati+
jig cata da coloro, ehe bramano pe tofto vivere con poca riputazione, che glorio-

gi famente morite; il che bene esprime il detto Latino Vir fugiens denuo pugnabit.

m Der, che s'era srmato, ed havea fatto (Crivere nel (uo scudo a caratteri
iamt d' oro BON FORT VN& vantandofi di voler-far gran bravure, (¢ egli entra
è,g Va in guerra; quando si venne al combattere, buttd via lo (cudo, ¢ si fuggi, ed
misit a. coloro, che lo taflavano poi di codardo disse: Vir qui fugie, ruxfus redinregra-
nme bit pralinm, indicans ueilins Patria fugere, quam pralio mori, mortuus enim non pi.
sen grat (che noi diciamo: / morti non fan pin guerra; ) at qui falurem quefiuit in fuga,
poet pote/? sm multis pralijs patria u/ui efe. Tuttavia anche appreffo gli Antichi era vitu-
dda Peroso questo tuggire; ¢ si trova, che 1 Lacedemoni bandirono Archiloco sola-
digi mente, perché havea scritto, che cra meglio abijcere clypeum, quam interire,
jue a del fale im <ueca, Kaver giudizio. Vedi sopra C. 4. st. 15. ¢ C. 8. st,
wi, © CHOCAR di spadone, Par che voglia dire, che questo tale si difenda con gio-
jgad care di spadone a due mani, ma incende, che gioca di spadone a due gambe.,
yal Slot fugge: motteggiamento usatissimo verlo coloro, che fuggono per paura il
ie dite Ginora ben di /padone, ¢ \enza dite a due gambe s' intende fuggi. Vedi sopras

| C, 7.0.76. Giuocar di spadone si ula ancora di dire in proposito d' una casa, che

sia igauda, ¢ (pogliata di maflerizie; in questa manicra. Vi si puo ginocare di [pa-
done, ciaé Non vi ¢ cola alcuna, che possa arreftare, 0 impedire questo efercizio,
che ha bifogno di iuogo largo, ¢ difimbarazzato.
T#aSTE, Vedi sopra C. 1. st 60. Talte fila, che G mettono nelle ferite, dette
così dal taflare, che fanno la lunghezza, ¢ larghezza di quelle. Latini panicidi
ai Vulnerary, lineamenta, i
g DAR campo, che si predichi di ivi, Dac' occasione, che si discorra di lui cons
wm) lode. £1 ver! predicare usato in questi termini figaifica Far' cn:omj, 0 lodare,
| Quand' uno fa quaiche azione bella, ¢ di cia G pavoneggia, (ogliamo dire in de-
Be 2 Chese ne predich,
PAR botreca di quiftioni, Viuer di risse. Haver care le riffe per guadagnares.
E tanto questo detto quanto far da spada come una fega, cioè intaccaria nel far qui-
fione, come è intaccata, o denotata una fega ) sono detti deriforj a tali Bravaz-
zoni,¢ Tagliacantoni.
LA morte vi si piega, Voi morite, ¢ dopo la voftra morte non si discorre più
de! voftri gran fatti, ¢ si perde la unemoria delle voitre azioni » ¢ vanne del pari

la bravura, ¢ la codardia » Quell' importuno, che per la via facra s'avvid dictro

a Orazio, enon lo voleva lasciare; domandatorda lui, se avava netiuno de'fuoi,

che' aspettafiero a caia; pee maggior suo dolore gli rilpose: Omues compo/ui,(a-

no accomodati, la morte gli ha ripicgaui tucci, Sa) th ee SUN

 
 

 
  
 
 
 

44

Colei c ha fatto buio
Paga di fogni i debiti a ciafiuno, .  (Benche si
Quella, che dianzi tolfe al di la vitay Per fuggir
Cagion, che tutto il mondo porta ae Comincea a
Descrive con vaga maniera in quest' Ottava V apparir
con equivoci; uae far buio vuol dit Confumar tutto il suo Sed '
tendédo della notte)vuol dire ha oscurato: ¢ se ha confamato | !
¢ fallita » ¢ non prod pagare i suoi debitife non con i
ricca se non di fogni;e pagar di fogns vuol dir pagar di moneta
non pagare, Vedi sopra C. 2. st. 7. fugge dungue la notte per
giona non folamente, perché è fallita, ma ancora i ella te
sia fatta la spia, che ella poco diana. uccife il giorno perché la
oscurita uccide il giorno ) per la qual morte tutto i) mondopi ee
dir, che per tutto il mondo la notte ¢ buio, enter: bruno,e €0
te di gualche nostro conginen i se bene ella non dourebbe temere di tal!
zione 5 perché Si chinde gli ocehi a, che fgets on off.
re, finger di non fapere; ¢ il eos connivere., Vedi sopra C, 6.8 vit
vuol dire che si chiudano effettivameate gli occhi, perché og ne
fuggir I iba c' ha le calze gialle, per fuggix V Alba, A ¢ spia del gi
che ha le calze gialle, perché il primo albore del giorno è i colore frail
€ giallo, ¢ così s' accomoda all' equivoco delle calze gialle shee
ze il contraflegno delle spie, 0 de i toccatori come accenn sopra C
fan. 60. 03 99g
COMINCTA a ragionar dt far le balle + Comincia a ragionare, 0 r
partenza, che questo intendjamo quando diciamo: 4 rale fa te baile

fa colligar:
TANZA VI, STANZA VII
E denna » che di quci balletti Jf aftidita poi da ranto fran) +
Sarebbe in corte tutto il condimento, Suvi mulinelli, forge cL

  
 
   

  
   

  
      
 
   
    
    
 
  
 

 
 

  
   
 

Ler ch' in un tempo fol con i calcerti £ data nna Seofferra come i fae
Ballandosuona al par d' ogni firumento, La laciachiede, britdospi 7
Lupo cena per degni suci risperti Perché il mmico all? alba de' Ta
Prefe dag altri un canto in pagamento, Vuol trucidare in singolar
E sopra un pagliericcio angufto y ¢ fod Ed a fargli servixio, pil
Fino ad horas' ¢ cotta nel | suo brodo. Vuol ee
STANZA VIL. ANZ

Pero che wel pensar che la mattina i vi intrepid
Entrar in campo dee alla tenzone, Espaccia il Baiardino, eit
Fa ginfto, come quella Nocentina, Chi la fringeffe,
C's giorno andar douendo a proceffione, Pagherebbe quaicofa ay

Occhio non chinde, ¢ tuttania mulina, Ma tutto questo

 
 
    

(ZRF, BORED ER RESEP RL Bae. ELLER ew eee

Tanto che ud capoell' bacome unceftone; La faccia tofta 7
Così la Strega in cella solitaria Sperando
eAtrende afar mille caspelli in aria, Chie! non fen,

  

101 Sig

 
 
  
 

 

DECIMOCANTARE. 465

-'Martinazza, che farebbe stata la perfezione di quella veglia, se ne ritiro in
camera, ¢ poslafi in ful letto stava pensando alla battaglia, che doveva fare con
jagrillo, ed alla fine, se ben veramente non farebbe voluta andare a combat-
ere, finge coraggio per non esser cae codarda, ed in sul far del giorno chie-
le sue armi, (perando pure, che habbia a fucceder qualcofa; che impedilca, ¢
® sia causa che non segua il detto duello.
SAREBBE fata ii condimento, Cioè Carebbe stata la perfezione di quei bali,
# di quell' allegria. Così quando sopraggiugne qualche persona gradita in una con-
" jone, si dice per ilcherzo, Venir ella, come il cacto fu maccheroni, come lo
- xuechero in fulle fragote, 0 fulle vinande; valendo con queste batie similitudini si-
gaificare cid che più nobilmente fidirebbe. Essere ella il condimento della con-
tm ucriazione, ¢ non vi mancare altro per renderla gustofa, faporita, ¢ perfera.
hued SVON-A al par d' ogni firumento, Ghediio vogliamo dir copertamente, che una
wet cosa pute diciamo: La talcofa fuona, Vedi sopra C. 6 stan: 49, ed il Poeta cava
da cid lo scherzo dell' equivoco, moftrando di dire che Martinazza fuoni d' ogni
mit strumento, ed intende che le putano assai i piedi, poiche dice, che ella /uona co'
'mj ¢alcetti, che sono scarpini di panno lino, che si portano in piedi in su ja carne fot-
shay to te calze; ¢ si dicono cascerti ancora quelle scarpe di quoio forcile, senza fuolo,
gum ma con la fola piantella, che usano i ballerini, e che ulavano già l¢ nostre donne
ga di portare sopr' alla calza quando portavano le pantofole.
ott  PIGLIAR un canto in Pagamento. Significa Andarfene. I debitori, che volen-
rag ticri (cantonano i suoi creditori,si dicono dare un canto in pagamento,cioè fug-
gigi gite il creditore per non pagarlo, ¢ per non avere occasione di trattare con Jui:
|. PAGLIERICCIO. E quel gran sacco pieno di paglia, che usiamo tenere in fu
gig Fletti forto le materaffe, detto anche /accone.
wt ~~ 8° 6 cotta nel suo brodo, Non ha havuto veruno d' attorno. Quando alcuno f2
: qualche risoluzione, che non è approvata, 0 non piace agli altri, e non ¢ da ve-
yi tuno in quella seguitato diciamo; E /* quocerd nel /no brodo, cice senza che altri
vi a & nulla del suo; 0 vero Farò come gli (pinaci, ¢ s' intende che G quo-
cono ir brodo
già FA come quella Nocentina, Nello Spedale deg!*Innocenti di Firenze (che & quel
“4 nel guale s' allevano i nati per lo più di copula iliecita, si come accennam-
i Te sopra C, 1. stan, 85. ) stanno riferrate molte Fanciulle, che noi chiamiamo
a Mocentine le quali non escon fuori se non una voita ! anno, che è la mattina,
, della vigilia'di San Gio: Batifta, che vanno per la Città procethionalmente; es
Pe ciascuna di Joro ha gran desiderio di far tal gita, non vi € aubbio, che
f speranza d' haver a godere si bramata foduistazione, fa, che pare a' ciascuna
 mill' anni, che venga il giorno y¢ che per tal pensicro poco derma la notte avan.
 £1, rivoltando per la mente wweti li modi di comparire atullata, ¢ bene all' ordi-
ne; il che è caula, che la mattina ella ha poi un capo c me un ceffone, cice grof-
© €pieno di confuficni per haver poco dormito, ed affaticaia la mente in quei
Pensieri; € guefte fon quelle, alle quali il Poeta aflomigiia Martinazza.
MVLINARE + Pensare; Difegnare, andar vagando con la immaginazione;
che diciamo anche: Ghiribizzare. Vedi sopra C. 9 stan. 56. Viene dal Latino
molior » che vuol dir wacchinare,O ne dal volgare Aduino, quali girare coi pen-
aa ficro

  

   
    
   
   
 

te
 

 

 

      

466 MALMAN TILLEY

ficro come un mulino. Virg, disse spedissimo +| Corde:
che fanno le persone innamorate peulando fidamen
giamente ne diede la descrizione in Didone,
Multa viri virtus animo, multufque
Gentis honos  barent infixs pe'tore vultus
Verbaque, nec placidam membris dat
Tutta la notte va mulinando « E lo stesso, chevaculer. Ho
Quid brexi 'fortes iaculamur auo
multa ?
E' detto ballo scagliarsi col pensiero ora in una cosa ora, inua)
Mattio Franzefi acl Capitolo delle Nuove,
Lasciamo aftroiegare a chi indovina
Per wie di conetiure, ¢ di difeorfi,
E col vernel fantaitica, ¢ mulinay.

HLA il capo come un cefeone. Gli si confonde ik cerucilo, Pai p
do diciamo fa ii capo grosso, 9 se gli ingrofia il capo, intendiamo
de il giudizio: EB Cefone & un gran paniere fatto di vinciglic dt
te, ed & capace di mezza (oma, ¢ perché ha la figura a:
queta comparazione. vil

CAST ELLO in aria, Pensieri senza fondamento, ed affegnamenti
nt, ¢ che non poslono riuscire. Laili Ha, Tr. C. 2. st, 2470 ADA

Fra me facea mille Caffelli im aria ode
Ariftofane intitola una sua Commedia, in cui. Gi burla di
Nuuole; ¢ lo fa falire, ¢ pafleggiare in aria.y per moftrate, pr !
vana, ¢ senza fondamento la (ua filofofia. Noi quando vogliamo dire!
badare a' discorfi terij, ¢ avere il capo aitrove,¢ a bagatelle; Dichiamo i
fare a' nuuoli, (e non vuol dire pitt toflo in lingua Lanadattica: Pen/area milla.

MVLINELLO. E uno firumenco di ferro, che serve per follevar peli
derivandojo dal verbo malinare detto (opra significa inucnzioni,
ne, difegni, ec,

DATA una scofetta come i cani, S intende, che Martinazza I
veilita, ¢ levandofi dal paglicriccio, fece come fanno 1 Cani, quando,
no, che per lo più si fquotono. A

ALBA de' Tafani, Si dice quell' ora del giorno sche il Bolee
re vigore, nelia qual' ora i Tafani sono più vivaci, Tafano. Lati
un verme volatile simile alla vespa nel colore, ¢ nella figura; ma
aflai maggiore, ed ha ancor' egii un' acuto pungiglione 5, ficche
de' Tafani s' intende leyarsi di la da mezzo giorno wi) |) say Pr
PAR vegnia uno, Far cortefie, 0 carcazeasuno, an iL
no affetcate, fidicoao /exzi, quali iddicia'y © intedtus » come k i
sca Novella 10. Sérallegro con Nencio [poso della Ragaread y #6 &
bene, ¢ le faceffe verxi. Col dire.  farls servizia, e pin chee
orecclu fieno i maggiar pexzé,intende, che Martinazza gli fara g
tarlo in pezzi così minuti, che un' orecchio intero sia: 1 mag)
trovi del suo corpo »detto acim per suena ua

 
 
 
 

   
  
  
   
  
  
    
    
   
 
 
    
  
 
    
      
   
 
      
  

 
 

a
om

DECIMO'CANTARE 467

'SP ACCTA il Baiardino, ¢ il Rodomonre, Si fa Mimar bravo, come favoleggia,
' Ariofto, che fulle il Cavallo di Rinaldo Paladino appellato Baiardo, € quel
¢ Saracino detto Rodomonte. Pud anche essere, che far il Baiardino, signifi-

chi far il bravo da un tal Pietro Terragiio soprannomniato Baiardo, che fr uns
soldato di-valore, ¢ d'inufitate forze, il quale mori forro Milano militando al

-servizio del Re Francesco di Prancia, come narra il Varchi Stor. Fior. lib, 2,
| CHI la fringeffe fra uscio, ¢' mare, Chi l efaminatie bene; chi glielo do-

mandafle da folo a folo,

segua, 0 non vada la posta, o l'inuito
tutte le cose, che intenzionate, non s' ¢!

 PAGHEREBBE quatcofa a farne monte, Spenderebbe qualcofa a non far questo
“duello. in ructi i giuochi si dice far monte, quando si reita d' accordo, che nons
roposto; ¢ questo ¢ fatto poi comune a.

i(cono: per efempio / tal matrimonio,

he era gid conchinfo', ando'poi 4 monte, cioè non si stabili. lo voleva andare a Ro.

with
joie

ma, ma poi ne feci monte, cio non andai.

IN se tien duro, Lo tien fegreto in se. Non si confida con veruno,

FA factia tofta,, La faccia fol' efier dimoftratice delle interne paffioni; ¢ pe-
ydiciamo; / rale fa faccia toa, intendiamo il tale si sforza di non sco-

iia prit co mutamenti del voito 1 suoi fegreti, eflendone richieflo, ¢-di non confet-

waco

a

git
:
3
eo
è
:

i

:

“STANZA X.
Spada,e lancia fra taro un Servo apprefia

i Col perto.a borta in man Laltro galoppa,

'a altro 0 elmo da coprir la testa
Da distder unalcro,e bracciaye groppa,
Di che coperta in ricca sopranuefta
Par un pulcin rinuolto nella floppa,
Ed allestica in ful cantar del gallo
eitro quivi non refta, che il Canalo

fare itdelino »eflendone claminato. Latino frontem perfricit,

STANZA XL

Percio fa comandare a i Barbereschi,
Che lo menin n' un campo di gramigna
Accioech'ei pasca un poco,e si rinfreschi
Perché per altro il poverm digriena.
La marca bebbe del Reeno,es enidalescbi
Gis hanno rifatta quella di Sardigna,
Maglie,e reti ha negli occhi,ode per cena
Vanne a pescar nel lago di Bolfena

B servi di Martinazza le portano l'armi, delle quali armatafi, ordina, che le
sia condotto:i} Cavallo, quale il Poeta de(crive per una folennissima Carogna.
“GALOPP A, Cioè Corre', Verbo usato in questo significato,ma però impro-

prio, perché galoppare, o gualappare & specie di correr di Cavallo; la qual voce
concorrono gli eruditi a farla venire dal Greco calpareia,

GROPPA, Si dice la parte di dietro del cavallo, o simile animale, ma qui in-

tende la schiena di Martinazza.

PARE un pajein rinnolto nelia floppa, Quando si vede uno, che non fa portare

l'abito in dotfo, ¢ che pare impaftoiato nel camminare per causa deg!i abbiglia-
menti, che had' attorno, l'aflomigliamo a un palcixe, o poliaftrello rinuolto
nella floppa; e non fiamo is cid diffimili dai Latini, che in questo proposito
didero. Herer ranguam mus in pice.

SVL cantar del gato, All' apparir del giorno, che a talora fogliano per Io pik

cantare i Galli Vedi forto C. rr. st. 5. Orazio.

etd galli cantum con fultor ubi oftia pulfar,
BARBERESC HI, lntende gli Stalioni; (¢ bene Sarbere/chi chiamiamo coloro,
N ai

on 2 i quali

 

Bikes e, 4
  
 
 
  

468; MALMANTILE —

t quali cvflodi(cono, ¢ gevernano i Cavalli Barbari, ¢
Poeta gli chianya così per derifione del Cavallo di Martina
Firenze 1 Cavaili, che corrono a i palj della Città, ton
frica, che noi chiamiamo Barberia, s
CRAMIGNA, Erba nota buona per pascolo degli Afini pil
li, ma a quelio di Martinazza non par poco haver di questa,
zerin digrigna, clue s¢ nou havefle di questa, non havrebbe.
ci serviamo del verbo digrignare per incendere flentar per la fa
nare, ¢ acrocare i denti per non hauer altro, in che ado
canl, ec. che si dice digrigware, quando per la rabbia
Tat Cas.
x Non vedi tu, che digrignano i denti
Econ le cigha ne minacctan anoli?
Ed egliame: Non vno, che cu paventi y
Lascsagli digrignar pure 4 lor fenno,,
MARCA, Contraflegno. Es' intende quel fegao, che hannoi

li, o di razza in una coscia, o nel collo, perché da essi si posla
razza sono. Virg. 3. Georg. Continuoque notas 5 nomina gentis inurunt,
che quelto Deftriero di Martinazza havea già la Marca del Regno di
sono oggi i migliori) ma che i guidade/chi gue n' haveano mutata in
digna, € non intende dell' Liola di Sardigna, ma di quel luogo fuori
Firenze, dove si scorticano le beftie morte detto la Sardigna, came
pra C, 1. st. 2g., ed intende, che questo Cavallo per li guidaleichi, ed
fetti, che haveva, era buono a mandare in Sardigna allo Scorticatoio
te/co diciamo ogni scorticatura fatta alle Beftie dalle selie, balti, o altro. Mau

Franzcli descrivendo un cavallo fintile a questo disse; wig
Dinanzi ¢i non ¢ 21d troppo gagliardo; “iy
Ma in sa ja scbiena ha quaiche curdalescho,
E le spronate mostran, ch' ¢ infingardo, ™

MAGLIE,¢ veri, Così chiamiamo alcuni mancamenti, che vengono si
occhi aile beftie; ed i} Poeta servendofi dell' equivoco dice, che con 'quelle ra
pud andar a pescare nel Lage dé Bolfena; ed intende, che il cavallo-era bof
dicemmo sopra C, 3. st. 53. » che cola sia. E così sotto quetti equivoci iroa

mente loda il-Cavailo di Martinazza. sagt
STANZA XIl. STANZA XIIL
Hor mentre pajce 1 mifero animale, E ti faluta,e tt si raccomanda, —
Eche si fala cerca aclla fella, E per cha inteso, che rm fai duclly
Giunge un Diavol più ner aet caviale Vn rotelion di fughero ti manda,
Con un marteile in mano,e una rorella, Spada non gid,ma ben gnejto:
Ed un liquor botiente ix un pitale Con una potentissima benanday —
'Ed inchinato a lei cos favella: Ch' 10 ti prefemo emr'a
I Re dell' Infernal Diavoleria Bell! ¢ caiduceta come la.

Con queste trescherelle a te m' innia, edilo [pedal si ad ta medicinal
;: aie

    
 
 
 
 
 
  
  
    
 
   
 
 
 
  
  
 

 

 
 

DECIMO CANTARE. 469

. STANZA XIV. STANZA XVI.

Hor fenrj; che qui batte sl fondamento Ma se per non haver buon corridore'
Quand' ih nimico ti verra a ferire Quivi a canfares tu non fulfe leffA,
Va pure innanzi,e non haver spavento, O per altra disgrazia; 0 per errore
el ferro questa targa a offerire, Ei r'appoggiaffi qualche calpo in tej 5

 E tuffo ch' ei la paffa per di drento, Vorlio, che tu per ficurtd maggiore

» Sia prefto col martello a ribadire, Hor per allor4 ti tracanni quests,

Ma lasciagnene subito alla spada Quale ¢ una bevanda sh squifita,
Peich'egli a se tirando, tu non cada, Che chi Lha in corpo no pua uscir di vita.
Ni STANZA. XV. STANZA xVIL
Face' egli poi con essa quanto vuole Così le fa rngoiar tanty dt micca
| Che pix di punta non puo farts offe/a, D! una colla renace di tal forte,
» Di taglio manco, essendo c' una male Che dove per fortuna ella si scca
Si fata a maneggiar pur troppo pela; wl mondo non è prefa la pik forte;
Portila dunque per ombrello al Sole, Luefta ( die' egli) Uanima t appicca
» Pere alia refta non gis muona scefa Ben ben col corpo, ¢ s'aitre non ¢ morte
 Edigli( gid che queila non ¢ il cafe) ©? una fepararion di que/ts Aussi,
Che # egli ti vuol dar, ti dia di nafo, Oxgi timor non hai de' fasti [uci.

» che Martinazza aspetta i} suo Cavallo riceve un regalo da Plutone. 5
confiftente in armi jd in. una bevanda per difenderfi dalle ferite,¢ dalla morte,
Nota che in questo bel regalo il Poeta immita coloro, che hanno scritto le pro-

 dezze d? Amadis di Gaula, ed altri Romanzatori, i quait, quando il loro Erces
dee esporfi a quaiche battaglia pericolofa, fanno fempre, che qualche Mago
“amico di efso Eroe io mandi a regaiare d' armi incantate, © altri difenfivi, ed
inttruziom, '

St fata cerca delia fella. Si ta cercando della fella, Dice così per moftrar, che
ae cra tanto iniolito ad adoprar la (ella, che non si lapeva piu dov'
ella fufle, >

PIÙ ALE. Alberello, 0 vafo di terra, come dichiara il medesimo Autore nell'
Ottava seguente dicendo; ch io ti presento entr' a questo aiberelio, Se ben Pitale &
Piopriamente quel va, che si mette centro alle predelle con altro nome detto
tantero..L' uno, ¢ il altro nome dai Greco, quello da Pitharion, piccol valo di
terra, doioiwm; quetio da Cantharos voce usaca anche da' Latini. © significa ua
vao lungo, ¢ stretto in fondo. E con manichi 5 quale ¢ queilo, che si vede cal-
volta figurato in mano a Bacco.

« TRESCHERELLE, Lato trice, Bagatelle; Coferelle di poco prezzo, Ve-
di sotto in queito C, f, 28.

SVGHERO. Pianta aota simile alla Quercia, ¢ fa le ghiande ferotine, ¢ las
faa leggierisfima (corza serve per far lavort da refiitere all' acqua, come farebbe
caiietce per metterui bomboie di vetro piene di vino, o d' altro per diacciare.

 BELL ¢ calduccia,, Temperatamente calda; ¢ come si da la medicina, che intea-
diamo bevanda folutiva. Vedi sopra C, 8. tt. 25.

CHUVOVERE fiefa, Fer yenire l infreddatura. Scefa diciamo una distillazione,
© catarro, che daila testa cafea nell' altre membra per causa del freddo.

Tl dia di nafo. Detto iporco usatisfimo nelia Picbaglia ia fegno di disprezzo,e

sin-

 
————————————< =

i

se

472 MALAANTILES @
s intende di nafoine,... che per ricoprire si dice 6
serve = esprimere la poca stima, che si fa della ——

NON fuffi lefiaa canfarci. Noa fai prefta a-fuggirli,.
Effugere, delinearesy nes lisdab Greco compre ara
detto così quali CG x F

TRACANNI, they bevay logolli i

TANT Adi mica, Vina gran quantita di ininefied = "
tore del Capitolo in lode de' Peducci, parlando:della min Secccea i
E gli ho tutes per cari, non che buoni

Von oftante, che sia chi dica espreffa,

Che tanta micca e cosa da bricconiy
Ser Brunetto Latini servendofi di questa voce nel suo\librovco
tutto di gerghi,¢ vocaboli,¢ proverbi Hinsanwsats 7 intitolaco
che sia antica Cittadina-di Firenze, 1s

Non ti darei una mica di beata; ¢
Se bene qui par, che voglia dire wn bricivlo, dal tele
tanta si pronunzia col gelto, che accennammo sopra OC. soft.
Luefia pea, e vedremo (orto nell Ortava 18.¢ 36. seguenti

FICC-A. Ficcare vuol dir Mexeré » 0)Cacciar per forza'.

NON è prefa (a pix forte, Diciamo fan prefax, quando la collay cal
© simili s' appiccano gagliardamente in quet noghi »ne\bquali-sono

L'ANIMA & appieca, Si ricordi il Lettore, che quella 6
fu le burle, ¢ particolarmente dove G trata diyincanti,ne iquali, q
trava luogo di fare apparir quaiche azione spropositata,non lafera
segue in guefla bevanda, la quale dice, che appicca ' Anima al
che egli creda, o voglia periuadere, che cid posla per incanto farsi
firare la goflaggine di Martinazza, ¢ di coloro j che hanno tanta a
caatelimi, ¢ ne i Demouj,

STANZA XVIIL
Quando la Maga vede un tal presente,~
C' ha in se tanta virtù, tanto valores
Da. morte 4 vita riauer si fente y

Si ringalluxza, ¢ fa tanto di cuore 5 'Cusiabe 'hontai i se
E dove fares ita un po.arilente Percio fatracal ronzin ha fell
Nel far con Calagrillo il bellumore, Vi monta sopra, € poi te xomb

 

    
     
 
   

Hor ¢ ha la barca afficuraca in porto

Pere! adeffo ch' eg ha ratte
Per fette volte almanco lo vnol morte, rddy

Camminerebbe pik in
STANZA &X

Perch? ei bada a spudiar declinazioni Pur.grazia del mated
Pin non si pua farlo levare « panca; Tentenna tanto y
Le polizze non Pwo, parca i i frasconi, Chiesvien. done n'
E con lo spalle s*¢ givcaio un' anca; M14 « carinetie il fang

Martinazza inansmita dai regalo mandatole da Piutone, etlendo-
Sole, monta a cavallo, ¢ taaro io fruga con gli sproni,, e col im.
zoppicando pur alia fiac  conduile ai luogo dove hayea ote \

si
at

  
  
 
    
 
 
 
   
 
 
   
  
  
  
   
  
   
   
   

reese erProczsleezeTEt.2f:r2=...

 
 

DECIMO CANTARE. 471
ST fente viauer da morteavies.Cioè le pafla quel timore, c' havea dvefieres
- ammazzata da Calagrillo | x,.
mi. SI ringaliarza. Si caliegea. Lat. Gefire, Si dice ringalluzzarsi, quafi mo-
~ firarli ficro.,.¢4 animofo come fanno 1 Galletci,quando si preparaao per -
“ ter fra oro, @ dopo che hanne combattuco, e vinto. Lucilio 4ib, 8, satyr. dice:

eee Galli nacens cum victor se Gallus bonefte
ee ite © Sufulie in digitos, primore/que erigit ungues.
@ [Lalli En, Te. C.5. dan. a6. ditle 3. Jn guetta nacas anor si ringalluzza. Stor:
di Seumifonte TLratt, 321 Semifoutefi, credendo d'hawer ogni dsfficuied fopita, rinesl-
— bnzxaronfi, @ fidandofi di (ua valentia y ec, B pi Lowe dice: Veds quanto noi fama
om 4 iti, e 8 mimici ringalluRrati, eC.
gibi FAltanto-de cuore, Pigiia animo, le cresce V' ardire. E il termine Tanto nel fix
infos gail » che diceauno nell' Occava 17, antecedeate ed altrove,.¢ si fuppones
i sho deteo.aicrove:), che colui, che per la faczia la dimottrazioac con las
| Mano accennando la grotiezza, ¢ pants di quella cal cosa, Quei che i.La-
ect ual daimus, vooltci quali fempre dicono coraggia, ¢ cxore.
ia | SAKEBSE a a rilente., Sarcbbe andata adagio. Circospetta ) O rattenuta a4
¥ risoluerfi,, )L? haurebbe pensata, 0 contidcrata. Significa infomma operar coa
tuwore. Leace per lento, siccome Violente per Violento dicefi da alcuat; come
Queito filo, queita corda ¢ fenre,.cloe non tela, non urata. Da Lente si fece Ri-
ing (fn sche noo ti ula se von in quelta Manicra: eFadare a rilente, ¢ significa lo
cai stetio, che Lente cioc ientameate., Nello tteflo modo chel'antica voce Diricapo
ail usatardal?-anuco volgatizzatore di Virgilio;c lo iteilo che Dacapo,
PAR ih bed: umore. de dea huomo dell' umore, vuoi dire huomo faceto, es
SFAZ1010,5-come vedenyno fapra C. 1, stan. 1o..¢.58. s*intende anche wao., che
si Vogiia: sOpcattarc 1 Compagna-di parole, ¢ di fatti, ¢c, comes' inteade nel. pre-
feute Moga, f
aia AOR eC ha la barca aficarats in porto, Ciok le par d' haver afficurata la vita col
regalo mandatoie da Piuwone
ih nae Vheche racing a | bucats wi fu i terrazzi', Cio' il Sole, che asciuga i panui
“moi deabucati, Dereazzo! »( quali Terrazzo) diciamo quella parce fupériore >
jul dele cafe la quale per loipiu edasciata da ana banda aperta', ¢ feoza muro, in
Se vece dei quaie lita, solteacre al tetta- da colonac, ¢ fom fabbricati in quetta forma
ww per comodita d' havere idole epercid das Latinidetti Solarimm, eda i Greci
Wil hewocaminus', Cio fornace del Sole,
iM CAM AMNGREBBE più in tre di che in uno, None dubbio., che qualfivoglia
m! — Animaie.camuninerebbe pith ia tee giorni, che.in uao, ma uGamo quetto modo
wih di dure per moitrar la fiacchezza d'uao Aaiuale y quafi diciamo: Quel viaggio
che egit na da farein ua giorno, 10 farcbb¢e pwd voleauieri in tre giorni, che in
yun foiow ue ¥
ul BADA a fhudiare-declinagioni, Attende,:o-continovaad accennare di -cadere
w” ~— perladebolezza. Deciinare's' intende uno, che*etfendo in buono stato: 5.0 dt fa~
ie hita,o di roba, cominci amancare neil' uuo, O.nell' altra; equi (cherza cont
4) equivoco delle declinaziom de 1 noim 5'ed-insende, che-il cavalo per la deboica-
#) 2a cra feinpre per. cascare..
0 Wow
  
 
 
 
 
 
 
   

47% MALMANTILE ©

NON si pui far lenare 4 panca, Non si pud farlo riavei
flar ritto: quand' uno è stato lungo tempo afflitto da i difaftri
to per terra, 0 vero terra terra) € che a poco a poco si va
Comincia a rizzarsi a panca; BE' traslato da 1 Bambini, quando:
dar ritti appoggiandofi alle panche; onde habbiamo un detto per
uno sia pil aftuto d' un' altro, che dice: Quando it Diauolo del tale nat
del? altro andaua alle panche, Franc, Sac: Nov. 158. dice: ach 60)
nostra mercanzia, che non ce ne rizzerems piit a per questo anno, ©
NON pwd le polizze. Non ha tanta forza ch' ei posla portare una po
Latini pure dissero: We folium quidem fuspinet. t

PORT Ai frasconi, ec. Diciamo portare i frasconi uno, che sia alg
mo, traslato dagli uccelli, ne i quali ¢ contraflegno d' infermita Y
abbaflate, che paiono beftie cariche di faftella di frasconi. Vedi. y
g. alla voce grado. Qui vuol dir che il Cavallo era infermo, ¢ malandato pet lt
vecchiaia. | Lb
CON 10 [palo s* ¢ giuscato un anca, Scherza con l'equivoco del giuoco di
nel 8 quand' uno piglia tante carte, che col lor contare
31. si dice /pallace, 0 ha baxuto lo /pallo, ¢ perde, si che intende che il

 
 
     
 

&

  
  
   

  
   
     
 

   
    
  

Martinazza è spallato. von lil
GRAZIA del martello, e degli sproni, Con ' aiuto del martello, che le mand)
Plucone, ¢ degli sproni, cio perquotendolo col martello, epi 1
gli sproni: Diciamo anche mercé del martello, ec, er
S* arranca, Diciamo arrancarsi, quand” uno per qualche difetto non pot
muover le gambe s' affatica per camminare, e/forse ¢ il verbo p
pato. Vi chi lo fa venire da Anca, che è l'offo tra "I fianco, ela coleiay
questa dalla Greca Ancon,colla quale si significa il gomito, ¢ si stende ad
gature, che fomigliano quella del gomito, Onde Sciancaro, quafi ex:
pun ha intere, enon senza mancamento |' anche. B Arrancarsi quafi tirarh, 2
firaicinarsi distro l'anche. 15h) aga
NE ha da ire il fangue a catinelle, ¢ ha bigonce, Ha da verlari moltissimo far
ue. Vedi sopra C. 2. stan. 57. (perbole usaca quando due Poitroni
ducllo, Vedi sopra C. 1, stan. 62. in altro signiticato. BC, 3. Ran, 29, che ol
sia bigoncia, Quando l'indugio piglia vizio, e-che fa di bifogno 1a preftezza jl
altro proposito dichiamno. ee ne va il fangue a catinelle, aah
STANZA XXI. STANZA XXIL
Quand! ti Nimico, ch' ius faa difagio Se tu fapeffi, come tu non faiy
A tal pigrizia,grida ad alta voce, C' armi fon queste pie
Vieni Afinaccia, moniti Santagw Farefte forse il brauo mance,
Cb' so fon qui pronto acaricarti anoce, O parlerefti almen-a' altro ling
Ella risponde: A noce? Biagio; Ma gid che tn venifti a tno
Fate un popian Barbier ohe'lranoquoce; i
S' altro vifo non haivallo a procura,

    
 
    
  

 

    
   
 
   
 
 
  
 
   
 
 

SeweR.> es > sre

cr

a repo RB re =

atten

o

alee

 

 
   

Lerche codefto non mi fa para. rrotté
Arrivata Martinazza al luogo dove s' haveva a fare il duello.wi tr
¢o Calagrillo, il quale vedendola venire così adagio ia fgrida y ela

 

 
 

SSk ELS

8 EE Sei ESei a oak

=
&

DECIMO CANTARE. 4B
ella gli risponde;che non ha tanta furia, dicendogli ch' ei non' farebbe tante»

bravure,(¢ egli fapefie di che armi ell' ¢ armata),¢ che ella veniva per ammaz-

zarlo.

“STA 4 difagio. Patisce aspettando: Sente incommodo in aspettarla,
 eASINACC/A. Parola ingiuriofa, ¢ benissimo iesire in questo cafo as

Martinazza, perché veniva pigramente, come fal' Afino.

. SANT AGIO, Si dice veramente Ser egio; che fu un Medico così nominato,

perché taceva tucte le sue faccende con ogni maggior suo agio,¢ commodiia fino

a tirighare, ¢ ripulire la faa mula, senza muoversi dal letto; ed ¢ paflato poi in

verbio, ¢ yuo! dir Huomo di turti i suoi comodi, e tardo nell' operare, che

ju una parola diciamo - Agiato. O forse-dalla voce Toscana, che vuol dire Len-
fecha y Comodird,

A caricarts a noce, Quando il noce è carico di noce, si scarica con le baflona-

te, ¢ pero dice, che wuol caricarla alla foggia, che si carica il noce, pec (cari-

Carla poi-con le perco!

se

» @LAGIO Biagso. Modo di dire usatissimo, ¢ particolarmente de i Fanciulli,
€ credo che fidica per caula della rima, ¢ del bifticcio, perché per altro il nome
Biagio ¢ (uper fluo all”'e(preffione, valendo tanto il dir folamente adagio, quanto
adagio Biagio, S¢ bene ci ¢ una favola notissima d' un certo Contadino nominato
Biagio, i quale perché non gli fullero rubati i suoi fichi, se ne stava cutea la not-
te a far loro la guardia; onde alcuni Gioyanotti per levarlo da tal guardia, es
poter a lor gusto corre 1 fichi, fintifi Demonj una notte s' accoftarono al capan-
nettoid) Biagio mentr' era dentro, ¢ discorrendo fra Joro di portar via la gente,
ciascuno narrava le sue bravure; ed uno di coltoro disse ad alta voce; Se voglia-
mo fare un' opera buona catriamo nella Capanna, ¢ portiamo via Biagio; Bia-
B10 cid -udito,scappd dai capannetto tutto pieno di paura gridando Adagio adagio.
o di qui puo forse havere origine il presente dettato Adagio Biagio, 0 adagio disse

ago,

FAT £ pian Barbiere che'l ranno quoce. Di questo dettato ci serviamo, quando
Ron voghamo acconscutre che si faccia qualcola in nostro danno.

» COT ESTO vifa non mi fs paura, Quando vogliamo moftrare di non temeres
diciamo: Ha tu altro vifo?e qui Martinazza dice: Va 4 cerca d! wn' altro vifo
perché corefho non mi fa paura.

SEVER AGGIO. Invende quella colla fehe Ie ha fatta bere il Diavolo, 1] Fran-
zele dice bexarage corcispondentemente alla nostra voce.

A tao ma' guar, Cioè a tuoi mali guai; Mal per te, che ci venifti, Ci sci ve-
puto per wrovare il tuo danno, Cusi 44a' paffi diceli alcuna volta per cattivi pal-
si; ome 'Piano a ma' paffi,

MANDA 1 faggio. Quando si da una piccola porzione di quella mercanzia,
che si vaol yendere:, acciocché il compratore posla riconoscere 1a qualica di etla
mercanzia si dice; dare, 0 mandare il faggio. B Martinazza dice a Calagrillo,
che intanto mandi il faggio delia sua carne ai vermini, perché fra poco vuol
mandargii nell' avello tutto il corpo.

NON volti portar bafto. Non son solita fopportare ingiurie.

Ooo STAN:
414 MALMANTILRS 2
STANZA XXIIL t ]
Horsh, dic' egli, all armiv apparecchia,
E vedrem se farai tante corenne,
© questo fuono allor mona Pennecchia

y

Dice fra se: No,no:Non taro Ammenne, \-  E ch* io t° infegni far

Sard meglio qui far da lepre vecchia, Così tn ch' items

E fenva star a dir pur al ©... vienne, Milafis a}

Fa proua ( gid dilcefa dal deftriero ) Ma fa pur quitof

Se le gambe (c\dicon meglio il vero. Bt ual, se eu fu z.
STANZAXXV20 1)

  
 

S? al cimento, dic' ella, del duello C
A furta corsi, bor fuggolo qual pefte y Però che dop' al muro f
Pero va ben, che chi non ha ceruello Grid egli quanto vnol,
Habbia gambe, e così mena le [este, Che per le grida it Lupose
Mortinazza, vedendo, che Calagrillo non cede alle sue bravate,

che fara meglio per Jei non indugiar pil a fuggirfene, pero (non si:

cavallo) fmonto, e fuggi così a piede verlo il Castello!
rimproverandole il mancamento, ma essa stimando più il peri

la perdita della riputazione fen' entra in Malinantile, ¢ lo lascia
SE farat tante corenne, Se farai tante bravure. Detto di derifione a wu

vantatore. wR
MONA Pennecchia, Detto derifivo alle donne. Da Pennecchio', ig

priamente si è quella quantita di lino, 0 lana, o cosa simile, che si

rocca per filarla, detta così quafi pensiculum.. Dal Lat. pensum.
NON tanto ammenne. Non fara così. Ogai parola non vuol risposta Per

io non voglio poi anche fidarmi in tutto di Platone'. Amen & parola Bl

vale In verita, Per verita. er
FAR da lepre vecchia, Cioé tornare in dietro, La lepre vecchia per'

gnar terreno, quando ¢ seguitaca dal levriero da in dictro, (il cS atto

La un ganchero, Vedi sopra C. 2, stan. 76. ) ed il cane furiofo se

scappa innanzi, ¢ perde |' occasione di pigliarla. L' aftuta maniera

delia Lepre è descritta mirabilmente da Eliano nella Storia degli animali'
cap. 14.. are

SENZA dire alc,,., vienne, Andarfene subito, © senza 'merter tempo it

mezzo. II Pulci nel Morgante, £ non è tempo da dire ale.... vienne.
SE le gambe le dicon meglio il vero. Se cilia fara più prefto a fuggire

a cavallo, Quando le gambe, braccia, 0 altre membra fanno bene 1a

razione diciamo: Le gambe, ec, mi dicono sl vero, cioè non mii fallifeone

mancano foto., Wee

Cl haueffi detro ulmen Salamelech, Almeno ci' havefii'ta detto.

Turchesca usata da noi per (cherzo; ¢ significa, Pace, 0 Salurea voi. ~

FARM le feilecche. Betfarmi.. Vedi sopra C, 7. stan. 25, 11 Vor

goefe dice, che Cilecca wien dal Greco Cileo, che wwol dir mulceo far

feilecca far tl contrario di carezne, civ far burle. Ma pud essere, che |

licta Gi fece Lezei forca di delicatczzc cosh Scileccke il contrario, che A

aliettare 5 ¢ poi burlare,

E intana di riterno

 
    
 
   
  
   
   
 
   
   
      
   
  
 
        
 
  
     

   

 

 
 
 
 
 
 
 
 
 
      
    
 

= pe eet S*e2 ce oc We ieee se*8. BS screes.es.ess
  

SSeS

 

 

DECIMO CANTIARE. 475
WMI lasci a prima giunta in fulle fecche. Subito,m') abbandoni + Milasci:fenz'

- alcoltarmi..B' lo stetio.che lasciar in Naffo,vifto (opra C, 1, an. 79, Si dice an-

che /asciare in Seco; lasciar fulle fecche di Barberia. Lat. Syrtos.

AO teco il sarlo. Ho.rabbia teco., perché ilxoder. deila rabbia s' affomiglia al
roder del tarlo nel legname:,Per il contrario fidice: auer baco.con wna persona y
cio¢ averci paiione. Petrarca: Afentre che il cuar daeli amorofi vermi fu confumato

TI vegito se tu fuffi in gremboa Carlo, Tiarriverd per.tuto, Diciamo; J.
grembo 4 Carlo, cioè Carlo Magno Imperadore, per moftrare che si vuole arri-
vare uno, ¢ vendicarsi in ogni maniera, quand' egli anche si fuggisse fotco la pro-
tezione del più porente, valorofo Principe del. mondo, come fu Carlo Magno;
econ i Latini diciamoanche,in grembo a Gione.. at

| CORRER a furia,, Eo fefios che far una cosa senza considerazione.. Vedi
sopra C. 5. stan, 41. E qui (cherzaintendendo se corse nel venice corre anche
nel tornare in dietro. '
 CHL nan hs ceruello habbia gambe. Significa chi non ha havuto giudizio,o me-
moria di pigliare, o fare tutto quello, che egli doveva in uo viaggio, habbia gam-
be, cloe lo faccia in due, 0 più viaggi, ma qui il Poeta scherza,.¢ motteggian-
do Martinazza si serve del proverbio, per intender, che se ella non hebbe cer-
ucllo ad accettare, ¢ venire al cimento del, duello, habbia hora le gambe per

ire
MENA le fefte, Pa speti,¢ lunghi padi, Le (ele, cioé il compaffo, s' affo-
miglia alle gambe dell' huomo; ¢ pero mexar /e feffe s! intende adoprar prelto les
gambe, cioc camminar velocemente, correre.
ANT ANA. Intendi se n' entea nel Castello di Malmantile. /ntanare da Tana;
cava foterranea.
DIET RO ai muro faluns efte, Chi ha un parapetto di muraglia non ¢ dubbio,
che € ficuro dalle stoccate.. B/se dal Lat, &è, formato all' usanza nostra, de'
li niuna parola intera finisce in confonante. Ii Burchiello nelia fine del primo
Sonetto. on funt non funt pisces pro Lombardi. il primo fant va seritto, ¢ letto
funte come qui Efe, acciocché il vero torni. E in quel verso, per dire anche
spe 28' aliude a un vero Racconto, che si trova (critta nelle Craniche de' Pre-
icatori, alla vita di Giovanni da Vercelli Generale.
DALLE grida scampa il Lupo, Detto ulaciflino per moftrar la poca Rima,
che si fa di coloro, che gridano,

STANZA XXVL STANZA XXVIII.
Poich* egli vede in somma che coftei, CHartinarza, che teme del suo male,
Alsrimenti non torna, fa i suot conti, Vedendo che 'l nimico se le accaita,
Che fara ben ch' ei vada a trouar Lei, Tre (caglionc'ba la porta,a un tépo fale,
Come faceua Macometto a i montis E gli da nel moftaccio dell! imposta «
E perch' ell ha due gambe, ed egli fei Ds poi dandola a gambe per ie soale y
( Mentre pero di fella ei non i/monts ) Senta dar tempo altempo,apigliar fofta
L arvriuerd:ne primaildeftrier punge, Infacca nel falon, la done ¢ il ballo,
GC? all entrar dé Palazzocs te lagsunge. Ed ei la segue foefo da cauallo.
O00 2 STAN-
.
 

   
   
 
   
   
   
   
   
    
 
    
  
       
 
    
    
      
 
  
   

476 MALMAN TIVE) 0

STANZA XXVIIL:
Appunto era seguito in ful feftino, \
(Come interuienein ee
Che due di quei che fannoda xerbine
S' cron per Donne disfidari «morte
L' un foreftiero, e /mentico pel vino he
L' a mi lafera,anch'eicenddoincorte } -
Ha Spada accato il Cortigian,ch'é l'altro,
Ma piit per ornamento, che per altro, Alle spalle

ca STANZA XKX.
In quel ch'ei morde i guati,efaquei sees © Ohe im

Che van de plano all arte del Adirrilto, ©

Ech'egtihafempriall'ufeioguoctes:

Dietro alla Serega giunge Calagrillo y Più des pie e

Calagrillo seguitando Martinazza entra con Lei nel is oO
che già fatto giorno ) continovavano a ballare,'¢ mette paura a
larmente a un-zerbiaello, che ¢flendofi sfidato con ua suo tivale
fufle quelio, ¢ pero si fuggi codardamente. 3
COME faceva Macometto ai monti, cioè se NON VengZORd Pr aendi si
noi da loro, che così ¢ fama, che dicefle Macomeito, per mofti
miracolo, comando a i monti, che scenueilero gilda iui, “e veer
venivano 'dicefse; Horst: andremo noi da loro.

HA fei gambe, Cioè due sua, e- quattro del Cavallo. 0

GL1 da l'imposta nel moftaccio, Gii ferra la porta in faecia Che T
mo quel legname, che'chiude le porte,'¢ fincitre da: Launo poiter) B
Serrar la porta in faccia:, per intendere operare - fare in modo 5 the =
vicino alla porta non entri,'¢ ferrar (4 porta tn fa le calcagna, 'incendere
uno fuori di casa., come vedemmo sopra C, 3. st. 50. 'Nenehe serial
T imposta nel vifo., o ne i piedi. ae
DANDOLA a gambe, Cominciando a correre. Vedi sopra 0.4
SOST-A. Riposo.. Vien dai verbo /ofare, chee: il. Laune/ ey
re,o fifere,

FESTINO, Trattenimento di giuoco;o di ballo. Vedi ropa Ca
celi Fefino., quafi felta piccola,, come quella, che'fi fa\felle'ca!
delle grandi, che si fanno-nel pubbiico..

TRESC.A, Così-anticamente dicevafiuna speeie di allo dal qual
hoggi Tre/cone specie di'bailo, come vedremo sotto C; 11g. U
Purg. c. '10, la piglia per specie diballo, dicendot
Trescando alzatol' umile §. rosacea
E nel presente logo ¢ prefa per adunanza wi. gence', che'
che la piglia il medesimo nell' daf.C.14.

-  Senza'viposo mai eralatrefea
Da trefea; trefeare, ches' intende operaré; ¢ Tre)
telle, che vuol dir cose di poco prezzo, o stima. Vedi a
£ANNO da Zerbino, Fanno dei bello, ¢ del galante,

ao Pe oes SP THE SS Es

-

a aie aa Ae

 

 
 

¥

DECIMO CANTARE 477

\ TVTT AY architettura's ec. Vuol dite, che quel tale usava nel veltire ogni ar~
te, € s' aggiuftava con ogni maggior lindura, diligenza,edifegno.
GONFIO, Alticro, ¢ superbo per la sua bellezza, come fa 11 Pavone,. che al

i detto delle persone più femplici,' gonfia perché si stima bello; donde poi pavonez-

giarsi, che vuol dir considerarsi, ¢ vagheggiarsi per bello; E guefto verbo <(pri-
'Me quel che vuol dir i/ Poeta nel presente luogo.

CREDE turar le Dame in Veffunio, Crede far perder 'tutte le Dame per il suo
amore. Crede, che la sua bellezza sia per far' ardere del suo amore 3 ¢ Vefusio &
il monte del Regno di Napoli, dove sono le voragini di fuoco.. rf
 HA paura del dilnnio, Cioè del diluuto delle percofle, le-quali spengono amor
nel cuore, e ' accendono nelle spalle ma differentidimo.

» VAN deplano all' arte del Mirtilio. Son-douute,-¢ si richiedono all' arte dell' ia-
namorato, da que! Mirtillo introdotto per innamorato dal Guarino avila fuss
Tragicommedia incitolata Pafforyfide.

HA gli occhi a' mobi, Bada, oflerua, sta vigilante. E diciamo «' mochi, ¢ non
allfaltre biade di maggior valore, perché eflendo i Mochi cibo proprio de i Co-
lombi, sono da' essi prt, che l'alcre danneggiati quando sono di poco feminati,
€ peroé neceflario haver J'.occhio, ¢ badare con piit attenzione a i mochi, che

Ll alte biade.

[pochi, Detto ironico, 'che significa moltissimi.
ha pik cnor dun grille, EB' codardo, non ha animo, Sotto C.11.z9.dice,
Han facte di Leoni, ¢ cnor di scriccioli, Appretlo i Greci per il contrario trovafi
Thymaleon, cioè Cuar dt leone, per vomo valoro(o, forte, cortaggiofo.

FA più capicale de' piedi, che del ferro, Si confida pil ne i piedi, che nella sp2-
da; cioè Mimd più ficuca dife(a quella del fuggire, che quella dell' armi: ¢ circas
queita voce capit aie. Vedi sopra C. 7. It. 82. ¢ C. 8. st. 6.

STANZA XX&K1. STANZA XXXIL
Toffo tornando l' amicizia in parte, Prima, che tra coftoro altro ci nasca,

Si viene allarmi, che ciascuna armata
Cid tien del altra un fegno fatto adarte
Per darle atradimento la-pierrara:
'Di qui si viene a mescolar le-carre,
Tal ch' in vederlatante scompigiara,
Rittrandofi a dir badan le Dame:
Baha basta; non pik; dentro le lame.

£ che la rabbia affatto entri frat cani,
E ms conuien fattar di palo in frasca,
E ripighar la Storia. del Garani,

Chre dietro a far che'l Turacirinafia,
eAccio,tornato pot come i Criffiani,
Ad onca della Strega-ogni mattina
Ritorni a vifitar ta' Kegolina

Di questo follevamento ciascuna-del'e-parti prefe sosperto di tradimento,.e per=
cid si venne all' armi dentro al medesimo falone.. 'Qui l' Autore.lascia coftoro, ¢
torna a Paride Garani, il quale egli latciodopra C. 8. st. 59..

TORNO' ? amicizia inparce. L amicizia si divile; cive ritornd inimicizi

“mMeera*prima. Parre t quella; che i Latini'dicevano parter, 'cioè fetta, fazione;
'onde Parziale, cioè affezionato,difenditore. Quel che sia parte per womo di spa-
da ch' egli era, ¢ non di lettere, Jo defini assai bene Farinata degli Vberti ti vec-
'chio, 'pretio.a Gio. Villani |. 12. Volere, ¢ disuolere; € per oltraggi, ¢ grazie ri-
Ceuute,

DAR ta pietrata, Dar colpo mortale; 0 conclufivo, dare a tradimento la pic-

trata
  
   
   
    
   

478 -MALMANTILE &

trata è ver in quel verso di Plauto; leera manu fere lapidem; panem oftemit
altera, Che risponde anche per appunto al nostro proverbio ane y¢ (a
Sajata.; o% SW
ST viene a mescolar le carte, Si me(cold la zuffa. Vedi sopra C. 9. st, 35.
SCOMPIGLLAT A. Confula. Qui intendi, rottalapace.
LA rabbia, e fra i cani, Così diciamo quando yogliamo esprimere »
s' azzuffano indiftintamente: Ii Latino Xabies inter canes, Ee
SALT AR di palo im frasca, Paflar da un discorso ad un' altro assai
dal primo. Far digreffione. 11 Monofini dice, che con questa nostra
s' accorda quella de' Latini usata da Tertulliano, De calcaria in carbonariam..
Ma guefta s' accorda più con quell' altra,. Dalla padelia nella brace. I |
Tertulliano nel lib, de Carne Chrifti dice così. dgitwr de calcaria, quod:
carbonariam; a Adarcione ad Apellen, i ose
LA regolina, Così chiamano i Ragazzi dell' ipfima Plebe Pornia ube
ga, la quale sta aperca in tempo di Quarefima, ed ivi si vendono frittelle,
Ii, baccala fritto, ed altre forte d' untumi simili, praticata, ¢ frequentata da' ra
gazzi, ed altre genti vilissime, come era il Tura, che speflo v' andava.
STANZA XXXIIL STANZA XXXV.

 

Paride giunto in mexxo ai cafolari,

Ove meffer Morfeo aun tempo folo
Fa dir di sia molts in Pian Giullaré
Strepitando fugeir lo fece a volo,
Sicognun deffo vanne a'fusi affari,

Ed ci, che Star non viol quivi a piuolo

eAnzi dare al negurio [pedizione,

Domanda di quel luogo infor marione,
STANZA XXXIV.

Vn gran Villano, un bnom acta matura

De' Quarantotti li di quel Contado,

Che perché ei non ha troppa [effitura,

Ed ¢ profontuofo al quinto grado

Junanzi se gi fece a dirittura,

E concerts (uoi inchin da Fraccurrado,

Benevenga disse, Voftra signoria,

E Le buone Calende sl Ciel vs dia,

Jn quanto al Lupo egli ¢ un! animales
Aa che aninial dich? io bue,
Via fiftol ds quei veri, un faci
C' ha fatto per sngenito gran dant,
E gid con i forconi, ¢ con le pales
J popoli affilliti rurto mguanno
Quin' oltre gli enno feati tutti riete
Per levar questo marbo da nn

STANZA XXXVL

Ma gli ¢ un fetanalfo foatenato,

Che non teme legami, ne ea.
S? ¢ carpito pits voilti 5 ed ammagliatty
Ed ha ricifo funi tantogrofe,
Le bastonare non gli fanno fates
Chie' navha abriga rocehechethaledt
D' ammayyario co' ferri non c' ¢ viy
Cb' egit ¢ come frncar n' uaa matity

 

STANZA XXXVIL

  

La entro a quella felua ei si rappiarra, Che tutti gl' animaliycht ei raccats

* Perch' elia égrande,dirupata ye fitta y Cudfando gli trascina lvirittay
wacciocche nimu.un tratto lo cumbatta,, E chi guatar poteffe; io.fopenfierd
Quand egli ha dato a'Socci la sconfitra, Chie' v' habbia fatto a' ofa uns "

Paride entraso ne i Calolari di Montelupo trovo, che tutti dormiyand,|
con firgpitare fece (uegliargli, ed havendo caro di sbrigarsi, proccurd
intormazione da qualcuno delle qualita ed abitazione del Lupo, ¢s' ak W
un Villano Sateapo del paefe, che gliene diede puntual ragguaglio. Ecol dif
fo., che.fa fare a questo Villano, moftra il modo di parlare del cont
KODZCp

      
  
 
 

 

 

tel on oe Ge Cee:

i el i
 

 

 

ee.

tak

 

Zo
7
a
a
q
i
5
j
:

%

DECIMO CANTARE:? 479

CASOL ARI, Intendiamo più cafe insieme in campagna scoperte, € spalcate;
qui intende di Montelupo, il quale se bene ¢ Castello, ha pil figura di Cafolares
per esser le Cafe cutte quafi rovinate, ¢ distrutte.

MOREFEO, Favoloso Miniftro del Sonno, il quale i Gentili tenevano, che a»
i comandamenti del Sonno suo padrone si trasformafie nella facia, nel paflare, €
ne i coftumi in qualfivoglia vivente, ¢ però fu scritto: Hominum fittor Morpheus',
beftiarum imirator. Ed altri, Atorpheus, © varijs fingit nova vultibus ora, detto
Morfeo da Morphe, che in Latino vuol dire forma, faccia; onde noi Smorfie,
per brutto arto, o gefto fvenevole, che si facial particolarmente col vio. E
roan in furbe(co; mangiare. Qui dal nostro Poeta Morfeo, ¢ prefo per'lo Nef

fo fonno. \

FA dir di si a molti in Pian Ginllari, Fa dormir molti; perché colui, che dor-
me senza posar la tefla, l' inchina, ¢ fa con efla il medesimo atto, che fa colui',
il quale con efia accenna di dir di si. In Piaw Gintlaré intende nel letto, che anti-
camente'fi costumava il dire. / vo in' Pian Ginllari per intendere, io voa letio,
© mi pongo gil a dormire: Ma questo detto come oggi poco usaco è ancora poco
inteso. Per altro Pian Gindari & chiamato un Borghetto di Cafe nel concorno de'
Vilage di Firenze non troppo distante dalla Città, che anticamente era de'Giul-
Jari cafata Fiorentina. Giullari, e Giulleria, dal Latino iaculares, vuol dir butio-
ne, ¢ buffoneria, 0 allegria. Vedi il Varchi nei suo Hercolano; ed il medesimo
nelle Stor. Fior. lib. 15. Won gridavan con quella fefta, ¢ ginlleria ch' eran soliti.

STKEPIT ANDO fuegir lo fece 4 volo. Facendo romore, fece fuggir Morfeo,
cioé fueglid i popoli.

NON vuol far a pivolo, Non vuole star' a difagio aspettando; diciamo: Tener
uno a pivolo, quando lo facciamo aspettar pili del dovere, o pil di quel che egli
vorrebbe, quafi che egli flia legato alla nostra volontà contro a sua voglia, come
si fanno star legate le bettie a i pinoli, che (ono pezzi di baftone, che fitti per le»
mura servono a i Contadini per legarui le beftie.

DE' Uuarantotte del contado, De i più riputati, ¢ Aimati del paefe; perché il
Quarantotto in Firenze è la dignita Senatoria, la quale ¢ il maggior grado, che
godano i Cittadini Fiorentini.

NON ha feffitura. E' huomo ardito, e libero nel parlare', non ha vergogna, o
-riguardo 0 timore, che lo ritenga; ¢ s' intende anche Vn' huomo, che operi, c
viva inconfideratamente, Sefirwra chiamano le Donne guella filza di puoti radi,
che fon solite fare da piedi, 0 nel mezzo delle loro vefti per farle divenir pib cor-
te, © per aliungarlo con sdrucire detti punti fecondo, che torna loro in acconcio
dal Latino /ectura, come vuole il Ferrari, Le Romane moderne 1a dicono ritrep-
pio, quafi piccol ritiramento deila velte, ed ¢ lo stesso, che imbaftitura, che ve-
dremo sotto C, 12. st. 33.

PRESONTVOSO, Pili che ardito, e poco men, che impertinente: Vno che
prefume afiai di se medesimo, e s' arroga piii di quel ch' ei merita. Vn' arrogan-
te. Daa. Purg. C. 11, dice.

Bd ¢ qui perch fu prefontnofo

DA Fraccurrado, Da Fantoccino; da burattino; che intendiamo quei bam-

bocci, che dicemmo sopra ©. 2. st. 46, 11 Bini nel Capitolo del Bicchicre <
Kuch

 
 

 

    
 
  
   

MALMANTILE

Questi perché fon grandi, ancor fon belli
Sends poca betta senza grandeRra y
\ wei paion Fraccurradi y¢ Spivitellig
Tra' canti Carna(cialeschi vi ¢ un canto intitolato. Canta, ni

Fraccurradi, e Bagattelle, ove sono descritti, i giuachi, che
© giucatori di mano con tali legnetti, ¢ burattini, detti-Frac

LE buone Calende il Ciel vi dia. Virconceda il Cielo, tutti i.
dia ij buon' anno.

SVE di panno, Sciocchissimo ch' io fone, Io ho manco giu
dicenci. Vedi sopra C, 6. st. 98. Lyf

VN fistolo. Le nostre Donnicciuole intendono Demonio, Diavolo. Vi
male maladetto, Bocce, gior. 7, Nou, 6. dufino a tanto, che il fiftolo us
Juo marito. Così detto dal filchiare de' ferpenti, a' quali egli ¢ affo

F AC/MALE, Huomo maligno, ¢ da fare cout it
lefactor, Cavalcanti Storia lib, 9. cap, 11. Cerri huomini befti
i quali mai alcun bene fecero, ¢ now hanrebbono saputo farne y huomini faci
futili, n't

PER ingenito. Per naturale instinto, che questo vuol' intender quel Ci

eASSILLIT I, \oucleniti, adirati. L' Affillo è un vermicello volati
alla zanzara, ma pill grande, ed ha un forte, ¢ lungo pungiglione
quando il Bue ¢ punto, entra in grandiilima smania,¢ tem eda qu
tadini quando vogliono intendere, che uno è in collera dicono; Eel:
o¢ afiduo, Sula in Firenze ancora questo termine, ma per ischerzo, y
con ammogliati con i quali farebbe termine ingiuriofo, quando non fulle ula
in burla, perché ¢ un dirgli Bxe, ve hfe

¥GVANNO. Quest' anno, Vedi sopra C. 6. st. 92. alla voce auannotte,

SMINOLT RE glienno feati tutti rieto, Qui intorno gli sono stati cust dietro ct
cando di pigliarlo, Enno, ¢ la terza persona del numero plurale dell' indi
del verbo efere, hoggi poco usato in guefla forma fuor, che da i contadini;¢!
uso Dante Parad. C, 13, me

Non per faper lo numero, che enno oot

PER levar quefio morbo da tappeto Per levar queita pefte, ¢ questa tribolazion
dal mondo; J sappete serviva già in Firenze per firato ai Supremi
quindi /euare uno da tappero figuibca levario, 0 privario di quella dignita
quale ¢ posto, che por pafiato in proverbio yuoi dire privare, © levar uno
qualfivoglia luogo, come qui che s' intende levar dal mondo,

SET AN ASSO, Satana; Demonio, dai Latino Saranas,come
nuovo teflamento. Appelliamo Saranafo uno, che sia fiero, ¢
scrua di tal jua forza per far del male: ¢ usato però dalle donne contro,
ciulli fieri, ¢ vivaci, 1 quali chiamano anche WVabifi. In Ebraico:
onde il nostro Dante. i? acai

48a

 

 
     
 
 
  
    
   
  

 

 

Pape Satan pape fatan aleppe. ) aries

Evuol dire Aduer/arins, Aduer/arins nofter dsabolus, ate
CakPITO. Cioé pigiiato con violenza, dal Latino carpere.
i Contadini. 'sili

  

zeseseEer: |

ae pee ee ee

— a
 

ees

=

fet

he

ete

RERt ES

DECIMOCANTARE: 481

2. Vedi sopra in questo C. st, 18. il termine santo di cuore,
NN git fanno fata, Non gli fanno male, 0 danao

'TANTO

 NON? ha 4 briga tocche, che Ube feoe. Subito, che ¢gli ? ha toccate gli pal-
fa il-doiore, non stima 'e percoise. Quando i Cant hanno toccato delle baftona-
te si squotano, ¢ reftano di guarite, che ¢ indizio, che non fentono, O non cura~
no più il doiore, ¢ di qui viene questo significato di squotere |e bufic, ¢ ne hab-
biamo il dettato Tw fai come i Cams, es' intende cu (quoti le bufie, che significas
Non le cur:, non le senti, non ne fai thma, ec. Vedi forto C. 11, tt 44,

MACT A. Con Vi longa. Monte di fatii dal Latino Adaceria,

Sl rimpiatra, Sinaconde, Vedi sopra C. 9, fh. 5,

 dVia40.. iano « Latino nemo. Won sopra C, 7. st. 89.
. £0 combatra. Gli dia noia;! impedisca,,

QVAN DL egis ha dato a' Soccs la feonfitta, Quand' egli ha meffo fortofopra, o in
contufione le mandrie', cioè fatti fuggire i bettiami afialtandogli: Che Socciv.s'in-
teade quel beltiame, il quale si da a ua Contadino per far' a mezzo del guadagno,
quafi dica a Sccio, cloe a compagnia. L'azione, che nasce dal contratto di So-
Gita, si domanda da' Legifti Azione Pro focio; Ma noi per Seccio intendiamo
una focieta, 0 compagnia particolare, ovvero una Accomandita di beltiame, che
si.da altrui., perché lo cuftodi(ca, e governi » a mezzo guadagno, ¢ perdita. So-
Zi /poj pure dal Latino Sectas intendiamo quel, che i Latini dissero /edatis iures
Sodalitijs iunétus, 0 Buon forse dichiamo a colui, che non guasta mai, ¢ che acco-,
da le conversazioni,

CA' ei raccatta, Ch' ei raduna, Ch' ci trova, ¢ piglia,

CIVEF ANDO. Cioé¢ pigiiando con voracita; rubando.

LU ritza, Cioè in quel luogo li, Termine ruttico, Dal Latino #i rea, Qui-
via diritto; in quella dirittura, 0,, come 1 Franceli dicono, en cer endroit,

10. fo pensiere ch' e v? habia fatto a' ofa un cimitero, lo credo ch' ci v' habbia ra-
gunato una gran quantita-d' offa. Che Cimitero diciamo 1] luogo, dove si forter-
Tano imorty. Vedi sopra C. 4. st.2g.¢ C. 7. tt, 27,

STANZA 'AxXVill
Sta Paride afenttrio molto attenta y

Ada pai vedendo quant' ei si prolunga

. Frafe dice; Coftui cs ha dato drento
Come quel che vuol far mela ben lunga,
Gli ¢ me troncargli qui il ragionamento

 checio prima, che il ds mi sopraggiunga
40 polfa lasciar l'opera compira,
Peri gis. dice: O via falia finita.

STANZA XXXIX,

Poi ch' egls ha intelo dow' ei possa bartere

e4.un diprefja 4 rinuergare il Tura,
Lell'efer foleo il bofeose a' altre tartére,
Che gli narri coftui, faper non cura:
La laterna apre,e il libro,od'alcarattere
Poa, vedendo., dar' una lettura,
Così leggendo senti darsi norma

Di quanto debba fare, in questa forma,

STANZA XXxX,

Vicino al boschereccio Scannatoia
Mentr' il froco di stipa vi riluca,
Palton groffe, Bracctale,¢ Schizzatoia
Co' Gucators a palleggiar conduca;

Ai rumbombar del suo diletto quoia
Toffe vedra, che 'l Gocciolone sbuca
Keuei ricchi arnefi vago di mirare,
Che gid in Firenze lo facean gunfiare >

, Sta Paride attento al discorig dei Villano;ma conoscendo ch' egli era entraco
ip on discorio da non finir mai, lo fece chetare, ¢ prefo il libro, da cflo compic-
BEDS tate;

se quel ch' ci doveva fare.

co.
 

 
    
 

482 MALMANTILE

COSTVI ci ha dato drento. Coftui & entrato in un discorso da non'
fine; ¢ me la vuol far (unga, Ciok vuol far' una lunga diceria,
OVVTA, E' \o steBo, che ors. Latino' Eia age. Termine, che
spedizione. ms " i ast
DOP' ci pud battere, Cioè da qual parte egli habbia andare per ir:.
Tura. Tey et
APN diprefso, Alquanto vicino a dove egli sia. Si dice 0 a ited “tid
vel circa, Dal dirfi per efempio: Furono tanti, quanti io v' ho detto vel cireay
Cit, 0 in quel torno, haa alle se
RINVEKG ARE. Rinuenire; Ri; Ri iare; Raccapezzare. © ist
ALTKE tattere, Altre zacchere, minuzie, © circoftanze di poca considera- a
vione. Se ben Tattere per (cherzo s' intende una specie di malore, che viene in 4/4
torno al fefio per crescenza di carne. i ?
CARATTERE, La forma, 0 figura delle letcere dell' Abbiccl. Voce latinas
tolta dal Greco Character, ¢d i) Monofino vuol che itia lio dir carartolo, ma si
non fo per qual cagione, se non fufle per allontanarsi dal Latino, che per altro te
non ho letto tai, ne sentito dir carartofo, se non a qualche Villano del tutto ru- ne
fico. eee
SCANNATO10, S) intende il luogo dove s' ammazzano i buoi, edaltrebe>
flie, ma qui intende quella felua, entro alla quale si nascondeva il Tura,¢las Ki
chiama scannatoio, perché quivi il Lupo scannava le beftic,
BRACCIALE, Manica di legno dentata, della quale s' arma il braccio pet laf
giocare al pallon groflo. Vedi sopra C. 6. st. 34. any Ta
SCHIZZ AT O/O ( gui intende il piccolo). Strumento d* ottone, o d' altro Ne
metallo fatto a foggia di canna da crilteri, ma aflai minore, e serve per metter '
vento in qualunque luogo con violenza, come si faa gonfiar palloni,0 pillotte, wt
o per (chizzar liquori; ¢ 'i maggiore per far serviziali. Latino e/yfer detto così,,

quafi frumento inondante, ¢ lavativo. Vedi sopra C, 3. st. 14. Che

PALLEGGIARE, Dare alla palla, 0 Pallone, mandindoio, ¢ rimandandolo Che
per tra(tullarsi, ¢ per avviare 1) giuoco; ma non giocare regolatamente, Onde» Ry
quando uno tira ia luogo un neguzio, coll' avviare chi glielo raccomanda, 2 un? Cad
altro, ¢ che quello lo rimanda, al primo, ¢ tutti due si accordono a burlare il Tes,

pover' huomo; fidice + Tra loro se ¢a palieggiane; che in Latino forse i direbbe Giaj
Coludunt, ast SOE?
GOCCIOLONE. Si dice a uno, che ta guardando una cosa con grande atten- Tu

zione, ¢ con desiderio a' ortencria, ¢ propriamente si dice di quelli innamoratt » Beri
che stanno i giorni interi appit d' una casa a guardar la dama, che € alla finellsa, Ng
¢ si coniumano, € si struggono a poco a poco, ¢ per così dire a filia a flilla, © vu
però dice Gocetolone al Tura, ¢ vuol' elprimere, che egli cra innamorato di que L
guarnefi. Lucrezio lib, 4. Pariando degl'innamorati. ee fog
Wamque voluptatem prafagit multa cupido, Ri
Hac Venus eft vobis, bine autem eff nomen amiorit; - Ca
Hine ila primum Veneris dulcedinis in cor B g

Stilauit gutta, © fucceffit frigida cura, ?
CHE gid lo facean gorfiare, La yoce gontiare vyol dire Andar superbo, comes oy

4
ra
53,

 
 

DECIMO CANTARE.
dicemmo sopra in questo.C. st, 29. sed il Poeta (cherzando con l'equivoco di gon-

fiar

3 ma in effetto vuol

483

Ic pillotte, ¢ palloni, che era il mefticro del Tura, come acccnnammo sopra

Gf st, 47. pare, che voglia dire, che quegli arnefi eran caula, che il Tura (eo

andava sup poi dire, che quegli aracfi eran caula ch'ei

J Sontava le milous » ¢i palloni, ¢ che egli gonfiava la pancia, buscando per mez-
zo

imi arnefi da comprar roba per empictia.

— STANZA XXXXL
Paride in soofe fatice ubbidisce 5
Accender fa le feope, ¢ intorno al si
 Gid questoje quel st spaglia ed alleftisce
 M faa braccialeye si comincia il gimoco;
Al suan del qual! Amico comparifee,
M4 ritenuto, perch' e vede il fuoce,
 Elemento, che vien dali' animale
Fugritaper instinto naturale.
STANZA XXX AIL
NGarani che fava alle-velette,
 Fedendo che'd Compar viene alla cefta,
Che te feope si (pengano commerte y
 Edin ua tempo a i Giscator da fefta:
WD un batrer docchio il ginoco si difmeste
La fipa si sparpagha ye si calpefta;
Tal che ficuro t' animal ridotto,
Va Paride pian piano, e fa fagotto.
5

STANZA XXXXIII.
Cid ch'é in ginoco in nn fascio egh ravvia,
E tra gambe la ferada poi si caccia
A tutto firascicanao per la via
Con una fune a otto, 0 dieci braccia.
Spinto dal genio a quella ghiortornia
Da lunge il Tura scguita la traccia,
Come fa il Gatto dietro alle vivande,
E il Porco a' beveroni,ed alle ghiande.
STANZA XXXXIV.
Vaghecgialo, s'aliunga, xappa,e mugola,
Talor 8 appre/sa,econlexampe iltoces,
Hor moftra shavigliando aperta l'ugola
Hor per leccarlo appoggiavi la bocca,
Tutto lo fina, lo roniftia,e frugola;
Così mentre il suo cnor givia trabocca
Ej, che non rocea per letizia terra,
Entra nel Borgo, e in gabbia si riferra:

TANZA XXxxv.

Perché Paride fa ferrar le porte,
E poi comanda a un branco di Famigli,
Che quiui farti bauea venir di Corte,
Che di loy mano l' Animal si pieli;

Ma i Birri, che buscar temean la morte
Non voglion accercar simil consigli,
E fan conto ( se ben' ei fa lor cuore )
Che @ paffi cutrania 2 Imperadore,

 

Paride in ordine a quel che trovd scritto nel libro datogli dalle Fate, fece acc
cender il fuoco d' avanti a) bosco, ed attorno vi meffe gente a giocare a} pallo-
Ne: a quel romore il Tura ulci dal bosco, ed allora Paride fece un falcio de'brac-
Ciali, pallone, ed altri arnei, ¢ legatolo a una fune lo fece strascicare per las
scada, la qual conduce al Castello di Monte Lupo, dentro al quale i conduffe il
Tura, seguitando quegli arnefi, e Paride fece ferrar le porte, ed ordind ad alcuni
Bi tri, che quivi haveva per questo fatti venire, che lo pigliaffero, ma essi impau-
titi aon yollero accoftarsi.;

C4ALLEST/RE. Metter' all' ordine: Approntare

L) AMICO comparisce. Cioè il Tura esce dal bosco,¢ vien fuora spinto dal gu-
sto di vedere il pallone.

RITENVTO., Renitente; cicé non alla libera, ma con qualche timore per
¢aufa det fuoco:, del quale il Lupo n1cura'mente ha timore.

ST AVA alle velette, Stava offeru'ndo. Vedi sopra C. 7. st. 67. It Burchiello
nella Novella del Medico Bolognef=. ¢ dello Scolar femplice dice: Andando ¢ri-
dando cerci tutta ia casa, ¢ tronarlo non gli fu ordine, onde tratte dalla disperarione si

Ppp 2 parti,

 
484 MALMANTELE:
parth, e lo Scolare, che flaua alle velette Vitorhato in'cafay ec,
. IL Compar viene alia cefta, Cioè Animale vien fuor
zimbello de i braceiali 5'¢ palloni 5 ec, iy HgOWs
DA sia ai Ginocatori, Ha veftardi giocare; Licenzia iG)
agli Scolari vuol dir Licenziar la Squola, ¢ di qui dicendofi dar,
cenziare ogni forta di lavoro, Daag
IN un barter ad? occhio, Inun momento. 1 Latini pure ditono-Jr%è
SPARPAGLIATE, Spandere contufamente, ¢ ae ee
come si fa della paglia, quanido'si batte, ¢ si spoglia il'grano'. 1 Pulei dite:
Sopr' alle spallela treccinsperpagtia., 0
FA fagotto, Fa un fa(ciode rbracciali, paltomt ec. Par fagotto; ¢ 10!
quafi, che far le baile per bacterfela, per andarfene. Latino v4/a colligere. ~
Sl caccia la via fra gambe, Comincia a camminare. Latino + viem,
SEGVIT A Ia traceia, Seguita, 0 va dictro allapefta, oalla sed étol-
to dai Bracchi, i quali si dice/egwitar /a traccsa, quando mel cercar della %.
ec. fiutando seguitano quella firada, ¢ quel tratto', per dove ella ha tirato;
per dove ¢ paflata: di qui habbiamo il verbo inzracetare-detto sopra C, 7. st,
BEVERON!, Così chiamano i -nostri Comtadini quella bevanda grofia fatta di
crusca, € d'acqua, ec, la-quale danno a i Porci. Vega eh aie
LO vagheggia, 'Lo guarda aftewuofamente.. Sivaledi-questo verbo vaghectiog
per esprimer il gulto, col quale 1) Tura guardava quegli arnefi;:eflendo tal ¥
proprio degl' innamorat', Vedi sopra C. 7. st79. (4 aay
MVGOLARE, Buna voce indiltinta, ¢ che aon finitamuore fra i denti.
ROVISTIARE; Ravoltolare, netter soflopray Forte aeglio-romifia dal verbo
rovistare, che vuol dir Muovere da un ldogo all*aliro. Ji Pulci., Morgante vas
rouiftando ognt cosa.. Hh Wx
PER letizia non tocca terra, Sopra C. 9..st.63, Per V allegrezza nom pud*star
n¢ i panni, 'che € lo stetio; ¢ figaisca haver'aliegrezza', o gufo grandissimo; Si
dice ancora; ma in modo batio, Lacamicianon gli tocca il federes Ml Boccaccio
Novella 32. iam
FAMIGLI, Qui sintende Famighi di Giuftizia,cio' Birri;la famiglia debPode-
fla,dal Boccaccio detti fergents, quali ferxientes, siccome'da noisfamigli,cic' fa
FA conto y che pafii 'dmperadere,, Finge di-non intendere, o-di nom lentire qu
che si dica\, Detto forse questo dal tempo', quando'era I'dmper: rec
vanni Paleologo-in Firenzeal Conciiio »che per cferfi già tata familiate la ua
vifta, ¢ forse, mancandogli i danari., non comparendo:così pompola, ne Cos
bella compagnia; ¢ appagata anche dalla prima volta in fu, lacuriosita; quan-
do paflava per le strade, non doveva far muovere la 'gente come prima, ¢ come
ando egli arrivd; Onde si venne a dire, quasdo uno non si cura di quaiche co-
f: Facciam conto, che paffi lo Laperadore, t a7%8 one
: ST, AN 2, AoREXKEV End OVS
Poiché gran perso ha i porri bapredicate, Senza pin (har a burtear via il fiato,

 

  

 

E che fan conto tuttania cb'eb cantiv, Totti di mano abc. iiguanti,
Pero che.da i Ribaldi gli vien dato: Bifogna, dice, con quefia canaglia
Li udienz.a, che da il Papa i furfanti, Far come il Podeftdidi

'AN:

 

   
   
 

 
   

DECIMO CANTARE: 485

ZAXXXXVIL ©» STANZA XXXXVIIL.
ds caps Si refta il Lupo, e'l Tura buomo diviene;
Ma non pero, che libero ne sia,

    
   
  
     

 

ad una delle [ue legacce

    
 

a addosso al' Animale C' ambi fone appiccati per le rene
eee a uso di bifacce: Formandoun Leong ha ¢ la Bugia,
r di tal.concia dé cauiale Dice Turpino,e par ch' et dica bene,

 

Ch' essendo questa si crudel malia,
ina di iupo,ed una d'buomo fembra, Lon erano.a disfaria mai baftani
di sua [pecie oguunna ha le/ue mzbra, » Gli odor birreschi femplici dei guanti,
4)! STANZA IL
opri tal mafferizia E Paride, che gra ' chbe notizia
molto pin fatto le mani, Da quel suo libro,si da quint ai cani,
 Percheglincants in man delaGinffizia Perché pin oltre il libro non ispiega,
i fichi- alla nebbia vengon van, 'Ona' et fa conto al fin di tor ia lega,
ide veduto che i Birri non ubbidivano, ed havendo per avvertimento dal
bro datogli dalle Fate, che gl' incanti rimangon vani iv mano della Giuftizia,
sdiede a credere che haveffero tal virtù ancora i guanti dei birri, ¢ per questo
f eae al Caporale, ¢ gli mefle addosso alla beftia, la quale si converti
Induce corpi appiccati insieme, che uno d' huomo,¢ I altrodi lupo. A tal me-
tamorfofi refta Paride stupefatto, ¢ non sapendo che cosa farsi, perché il libro
bon inlegna da vantaggio » risolué di chiamar due fegatori per(eparar It Animal
bruto*dal razionale. In questo moftro il nostro Poeta imita Dante nell' Inf. C.
 25. nella commiftione di que! Serpe con ' anime di quei cingue Cittadini Fioren-
i € la delcrizion di tal moftro comincia al verso: Se tu fei hor Lertore acreder
dente,
 PREDICARE #3 porri. Predicare al deferto,, Affaticarli in-vano a esortares
uno.a far bene, che i Latini differo vento logui; Surdocanere.
 PANNO conto ch' ei cantiB lo stetlosche dar Pandienza che da il Papa ai furfanti
che ia fuitéza vuol dire n6 fare stima delle parole d'un0,0n6 badare a quel ch'es dice.
CAPOXKALE. 'Capo di squadra di birri, Grado che si di anche sia i Soldati.
Vedi sopra'C.'9. stan. 2.
 BAR come il Podefid di Sinigaglia, Cioè comandare, ¢ farda se.. I) Duca di
Calauria Sigifmondo havea aflediato Sinigaglia,nella qual Terra era per Gover-
shatore foltituto da Gio: de Castro, Petruccio Piccolomini; Coftui tentd di ab-
' la Terra, dicendo efler. meglio uccello di-campagna, che di gabbia,
¢d a lutaderiva il Podefta, ma i Cictadini featendo questo differo di volergli get-
“tare dalle fiaeftre se più parlavano d' abbandonare la Città, ¢ vennero tanco in
odio ved in disprezzo de i Cittadini, che. quando comandavano noa erono ubbi-
Giti, edi qui venne il Proverbio: Far.come it Poreftd di Sinigaglia, cioè Coman-
dare, e'far da se, Cavalc. Scor.:
 Deeaa + S'intende quei Jegami, con i quali ff legano le calze, cingendo
ambe.
MSACCE, Così chiamiamo due facchetti appiccati "uno contro all' altto a
'due cigne, i quali si mettono a traver(o ai cavallo, ec. sopra il quale si cavalca.,
'€ servono per porcar robe, come si fa con una valigia, (ono appellate —

  
   
  
 
 

me

 
 
 
     
  
   
    
 
   
  
 
    
   
   
  
  
     

vit

 

 

 
486 MALMANTILE

bis facche, due volte facche, 0 facche a doppio. Lat, AMdantica Bocce,
nov. 10.5, Haveva Frace Cipolla comandato che bea guardafle, che let
»» (ona ada toccaile le cose sue, ¢ (pectalmeace le sue bilacce nelle q
»» cose rare. B pil otto nella medeGa novella. La prima cof che venac
» prefa fu la bifaccia, agila quale era la peana. weet
CONZI 4A. Quando Gi dice coacia di guaati s' iateade profumameato,

si dice guanti di coacia di Rona, di Venezia » di Spagaa 7 ec. ¢ 8 intend
mati alla foggia di Roma, ec. Qui dice concia di Cauiale y cioè feteatt »
fragore, 0 feagraaza ¢ Detto ironico., Were ta
LA Sugia. La Bugia Gi figura uaa Femmina con due facce differenti, comes
@' orfo 0d' huo ny, o di lupo, ¢ d' huomo, come è aci prefeace luogO,
DICE Tarpino, Scherza cone fa sopra C. 2, stan, 31, autorizzando en

te (ua Novella com i detti di Turpino, come fa ? Aciofto. lve
MALIa, Iacantefimo. Suregoneria. Vedi sopra C, 8, stan. 52. Donde 44-
liarda una strega. iy
T AL maferizia, Iatende i guanti del birro. cust cee
Sd aicani, S'adica, Quando uno per la stizza grida, ¢ fa alere dimoftra+
zioni d' impazzienza, 0 di rabbia diciamo; Si daa' can, Vedi (opra C. the 10,
STANZA L STANZA LiL...

 
 

 
   
  

 

Per cid fatti venir due Marangoni y E morta re la dd per cofacertay =
Con tutto quell' ordingo, che s' adopra M4 quel Demonio insieme firappicch,
eA fegare i legnami, edi panconi, E qual porco ferito agolaaperta

ef dinider il Moffro metre in opra;
Mitre la fegaim mero ai dusigropponi
Scorre cosiva il mondo fortofopra
Mediante il rumor de i due parrienti,
Che un fa d' urli el altro dilamenti,
STANZA LL
Pur senza ch' inraccato elit habbia un offo
La [ez infino ail' uitsmo aileefe
Lasciando il Tura libero, ma roffo,
Dietro ds fangue com' un Genone/e;
La Be(tia gli volea tornare addosso,
Ma Paride, che (ubito ? intese
Prefa la (pada la cagio pel mexrd
Pensando di madarla un trattoalrezro,

  

aa
Per dinorarlo forte se gli ficeay - -
Ed eslic! alt' incontro stan. all! eta y
la [a Latefta un sopramman gli appicedy
Ch in due parti diuifela di netto
Com' una tefticcinola di capretto.
STANZA Luh
M4 ritornato a penna,¢4calamaio
Pur quello Heffo a Paride si volta y
Che per veder il fin di quel mofeaio
See' fulfe mai possibile una volta y
Mena le man chee pare un Berrettait,
Ed a chius' occhi pur fuonas r4ccilta
E dagli, e picchia,rifuona se mA 7
4a forbice, t ¢ fempre bella.

Paride fatti venir due Segatori d' affe, fece fegare il Moftro in fu It artasatu-
ra deli' huomo con la beltia, e così gli fepard; Ma la Beftia tentava di
carsi onde Paride caglid la Beftia pel mezzo, ma eifa prefto firappiccd 5 B qui
il nostro Autore immita l'Ariofto nella favola d' Orillo; levata da Vergilioacil
Eneide, che finge un tal' Erillo Re di Paleftrina che haveva tre anime, onde cra
neceflario tre volte ammazzarlo per finirlo a.

tHARANGONT, Si dicono i Garzoni de i Legnaiuoli che lavorano peropra,
quando in una bottega, ¢ quando in un' altra a tanto il giorno, ¢ non jn
una boticga a falario di tanto il mefe; ma qui l'Autore intende Segatori di le-

 

goami,

 
a ese

—————

eee

me

=

- +e

 

 

 

DECIMO CANTARE. 487

| guaind 3 €gli ordinghi, che ? adopra, sono la fega a due mani, lima per metteres
} Gags denti, ¢ il cavalletto per adattarui sopra quel materiale, che i dec (e-
ox. cavalletco si chiama pietiche. Vedi sopra C. 6. stan. 6g. alla voce im-

 PANCONT. Sono afi groffe circa un quinto di braccio, le quali si rifendouo
per farne o affi più fortili, che si dicono panconcelli, o per farne correnti.

GROPPONE. S' intende la parte di poms di tutti gii animali, o bipedi,o

yadrupedi, ¢ lo diciamo ancora codione, ed ¢ propriamente quella parte che re=

fra le natiche, ¢ le reni. Vedi (opra C. 6, fan. 69.

VA fottofopra il mondo, Lo strepito confonde l'universo. I Latini pure dicono
Mundi fumma readit ima, © ima fumma;¢ vuol dire, che jo ttrepo cra gran.
'dithmo per le firida del Tura, ¢ per gli urli del Lupo.

 ROSSO come un Genoxeje, &* in Firenze una Compagnia, 0 Confraternita di
Secoiari detta de' Genovefi, perché ¢ formata di gente di quella Naziwne s Co-
storo hanno per coitume d' andar proceflionalmente la fera dei Giovedi Santo as
vifitare le Chiefe, si battono le reni ignude con mazzi di corde entrovi alcune
ficiie di metailo acute come quelle degli sproni, ¢ queste forando la pelle ne trag-
gono il fangue, il quale bagna loro le reni, ele tigne di roflo; E di questi in-
tende il nostro Poeta nel presente luogo.

. tH ANDARE uno al rexzo. Mandare uno nell' altro mondo, df fresco, ciok
il corpo suo forto terra. Ammazzar' uno. Rezo, vuol dire un luogo dove non
arrivano i raggi del Sole per interposizione di che che sia, ¢ fidice anche, me-
riggiv, bacio, ombra,¢ uggia. Vedi sopra C. 6. stan. 75 ¢C. g. stan. 44.

ST.AV-A aif erta, Stava ocnlato; flava avvertito. Erta si dice la talita d'un

BRIO; ¢ are all' erta ¢ termine di caccia, percht la Lepre ha per propria di
for fempre alla volta della fommita de' monti, per non efler così facilmente
arrivata, ¢ pigliando i suoi riposi, scoprir paefe, ¢ minchionarc icani; ¢ pera

in caccia State al? erta s? intende Habbiate |' occhio, ofieruate; il che ¢

poi pafiato in dettato comune a ogni cola.

PN sopramman gii appicca, Gli da un soprammano, che è quel colpo, che si da
¢ spada, baftone, ec. cominciando da alto, ¢ calando a balio. Vedi sopras

5- stan. 41.

D1 netto. S' intende lo taglid pulitamente in un fol colpo,

TESTICCIWOLA, Le telte degli Agnelli, ¢ de i Capretti da noi si chiamano
Teftucinote, ¢ per friggerie si tagliano nel mezzo per lo lyago in duc parti ugua~
li; eda questo taglio afiomiglia quello, che fa Paride alia tetta det Lupo.

4 penna,¢ 4 calamaio, Per ! appunto. Vedi sopra C. 2, stan. 19.

VEDER il fin di quel mofeaio, Veder il tine di quetia cola noioia. Vedi sopras
C, 4. stan. 9.¢C. 9. stan. 51.

MEN A le man ch' ¢i par ux Berrettaio, Menar le mani dicemmo sopra C, 1, st,
7. quel che significhi, ¢ qui intende che. mcnava le mani con ceierua,come fauna
1 Berrettai, ¢ Cappellai, che nel felcrare i cappelli, o berrette menano le mani
Prelto in riguardo dell' acqua bollente, con ia quale si fa tal lavoro,

 SVONA a raccolra, Continova a perquoter a jungo, che così fuona la campa-

Ra; quando fuona a raccolta di popolo per le prediche, ¢c, ed 1 verbo fonare si

° gailca
+

ee

 

   
   
  

- — “gk ae
2 488
a
488 MALMANTILE |
gnifica anche perquotere, ¢d ¢ della medesima natura, che il
habbiamo detto altrove. 6 ay eee
DAGLI, picchia, rifuona,e marvella. Questo di dire
re uno, che adopri ogni fea induftria, per fare una cola perf
do pitt vole le diligeaze. Vedi (opra C. 7. stan, 16. Similitudine,
tratta da' fabbri, quando Javorano il ferro sopra l'incudias; Qui:
d' Orazio incudi reddere versus, mettergli alP incudine, forto
critica. Cio¢ efaminargli, rivedergii di nuovo.con somma, rigorofa
diligenza. La nottra maniera; Barrere il ferro quando è caida, ebbe
meate da questa prontezza, ¢ macitria talieme, che si adopra per lavorat
nalmente |' dcudir degli Spagauoli, che vale aixeare, voce ormai si
è fatta dal Latino ddcudere, ciod battere insieme il medesimo ferro.
dichiamo per efempio. La prego a volere accudive « quefke megorio; © si
FORSICE.. Questo termine significa oftinazione,, per elempio. fo 2,
che tu non faccia la tal cosa; e tu forbice, cioè Tu oftinato I'hai voluta |
modo. Dicono che venga da uaa Donna offinata, ¢ capona,, 1a quale
chieito al Marito un par di Forbice, e non havendogiicie il marito mai:
te,ella ad ogni cosa, che i} marito le domandava rispondeva: Forbice;
impazzientato da queita sciocca oftinazione,le proibi il dirlo
piu Jo diceva 5 per Jo che il marito la baflond, ma. non per: ella se
maneva, ficche egii un giorno sopraffauto dalla collera la gewo in ump
cila fino che potette parlare fempre dite; Forbice, ed in ultimo goa p
valerfi della voce, si valfe delle mani cavandolg fuori del' aequa con le
giori alzate ed allargate in figura di forbice,per mottrare che moriva |
oitinazione, ¢ caponeria. Questa novella ¢ vulgatitiima fra le nostre
io ho trovata tra una raccolra di efempi facta da ua Buontempr
mano del medesimo tengo fra i miei nianoscritci. 2 eS
Lit fempre quella bella; L' ¢ sempre quella medesima. Questo yien da un
co, 1] quale andava accartando,¢ cantava una cerca orazione al fuono di un
tarrino, fermandofi alle porte de' suoi benefators i giorni destinati;
venuco a faftidio, do fempre la defima cola, inci:

  
   
   
 
    
   
 
  
    
    
  
   
    
    
    
   
   

 

  

 
 

 
 

 

  
    

quelli, che gli facevano l'elemofina a dirgli, che se non cantava q 'ae
orazione non gli haurebbero dato pil nulia, ed egii rispandéva; Pa
se', cht domani ve ne vaglio cantare una bella, Ma pecche il Povererto ai 4
se.non quella, tornaya l'altra mattina, e cantava la steila, laonde i f ”
fattori.accortifi, che il Meschino non ne. fapeva altre compathonaadolo, git te
cevono. L' è fempre quella bella, ed intendewano |' ¢ tempre quella 1 ig
che ¢ poi venuro in detrato, ¢ significa noi fiam fempre-alie medelim a
quanto racconto ancora fra gli scriti del medcfimo Bugnrempi top z
pucato ali' origine del presente dettato. ren “a i
S. TAN ZiAisbl Moni tap 'y¢:
Tal ch' ei si scofta none, e dieci paffi, Pervia gli anuenta m
E piglia fato, perch! es pronar vuvle, i
Selavirtude a forte gli giouaffi, i;

C* hanno! erbe, le pictre, ¢ le parole;

 
  
     
 
 
  

489
gout STANZA  i
recaffe a scorno, Resta in parata', molto gira il cnaré
alle gioftre,e alle quitaney | 'Pimceis pikes anc ielibbienicfie,
we b gli vada incorna, > Merce ch ei fache'l Diauvloe bugiardo,
EB latrartigo' faffi, come un cane; “E quanto en sia furtile,¢ filigroffo;
i, ver ch' e' fufse ! apparir del giorno, oPercia si merte un pezro a bellofguardo,
; L! Ombre,il Bau, ele Befane oCredendoognor che gli faltafse addofso,
Sparyce affatco, e più non si rinede, Aa poi ch' ei vedde omas d' ¢/ser ficuro
Ma Paride per questo non gli crede | = Ando all Ofte, ¢ cauollo di pan duro,
Vedendo Paride, che quel Moftro si rappiccava fempre » ¢ che ci non trovava
'modo di liberarfene per ferite, che glisdette, gli venne'in pensicro, che se era la
Werita 5 che in herbis, verbir, & lapidibus stesse la virtù, poteiic eflere che alcune
di quette cose havetie virtù di fare sparire, ¢ svaniresl Moltro; ¢ pero prefo il
 [xa dove, il quale era pieno di parole, ¢ dliverle erbe, € de i faili ogni cola tird
addotio a quel Moftro, ¢ l'indovind, perch subito egli spari, ed il Tura rima-
se libero”, 'Con tutto questo,Paride non si fidando, stette buon pezzo a offeruare;
ma veduro, che il Lupo non compariva pil si parti, ¢ ando all' olteria a man-

Patt. i
Ors ' fiato, Cioé si riposa,
. MLAEST RO Grillo Contadino, 8 nota la favola'di Grillo Contadino, il quale
per fardispetta @un sue fratello Medico sche non gli volle dar parte d' ua tefo-
F0, che infizme! havevano trovato, si fece Medico anch' egii,¢ con i sui forcuna-
a fiti's' acquifto la grazia del suo Re, non folo per havergli: rifanata las
cavandoie una tilca di pesce della gola con ungerle ilc,..., ma ancora
per haver saputo indovinare i fegreti dél medesimo Re, ¢ chi erano coloro, che
 aluirubato hayevano, in somma fece diverse scioccheric, le quali tutte per gli
} spares fidondarono in stima del suo valore, ¢ l'accreditarono per un valoro(o
Medico, ¢ grandissimo Indovino, come si legge nella di Jui favolosa vita, 0 di-
Ciamo spiritola Satira.
WINT ANA} Bruna campanella, che si tien sospefa in aria (oftenuta da una
molla dentro a un canacilo, alla quale per infilarla corrono 4 Cavaiiert con las
Jancia', come fanno anche'al Saracino, che dicemmo sopra C. 4, than, 57. € si di.
Ce ancora Chintana, Varchi Stor, Fior, lib, 15. Fecera metrer delia rena a! avanti
al palazze, ed appiccare /a chintana, Dai noltri Ragazzi ¢ detta corrottamentes
Timana 9 ed ¢ iatelo quel lor patiatempo, che fanno, infilando una zucca fresca
in una corda, ¢ pottala in aria attraverso a una Arada corrono con alle 1a mano
@ dare in detta zucca, unmitando i Cavalieri, i quali corrono alla quintana, 0
al Saracino, Dice che Paride era avvezzo alle gaintane, ¢ alle gioffre [che nel
Prelence inogo fon fitioninu; s¢ ben gioftra's' intende quando i Cayaiieri corrono
a corpo a om 70 al Saracino, ¢ quintana significa quello, che diciaino qui fo-
Pra) perché Paride haveva pil aout militato im Spagna, dove haveva cfercitaco
1 jor! gradi della mulizia, ¢ tornato alla Patria tu dal Serenityaio Gran Duca
fatco Governatore deija forcezza veochia di Livorno, ed hunorato del titoio di
Macttro di Capo, I nome tuo era Andrea Parigi, fu fratello d Aifon(o, ¢ di
Paoio detto sopra Papirio Gola, & Figliuolo di Giulio, ¢ fu come custi questi va-
sa = Qqq jen-

     
 
   
 
  
  

 
   

       
 

   
 
 
  
 
   

 

-

 
   
   
  
  
 
 
 
  
  
   
 
 

Se

:

©

i

=i

    
    
 
  

aa

 

 

 
—

' Ye
*

490 MALMANTILE™
lentissimo Tngegnere, € periti archi Qui
Ferrari cusi. Ludus equeltris,cum diretta in encun fimulachrnn:
gehtat, bala incurritur, Alcunt han detto come Vguecione Pifano.s
zionario, che Già così detta dalla quinta parte della piazza yin
tri, come Balfamone sopra Fozio da un certo Quinto inventore 2ed
la vera origine mottra il Pertari eflere da Comrus.cioè ee i
punta di ferro; ¢ si raccoglie dabtitolo nel Godice:, de i Y
radore chiama quelto giuoco con voce Greca Kynranos., In ordi
Chintano, ¢ non Chincana pare, che lo chiamaile, se sha a
ma, Fazio degli Vberti nel Dittamondo.:
Gionani bigordare alli Chintani y
E gran tornei, ed. una, ed altrag
Far si vedea con ginochi nuoui se ferant. -\

  

   

jofira '
>

CALAPPOLERIE. Cosa di poca stima: oda farne poco conto i “Apine; '

triceque,¢ buttubata, V. Feito, ¢ ivi sopra lo Scaligere.,
BAV,e Befane, S' intendono quelle Larue inveatate dalle Balie per far paura
ai Bambini, come habbiamo decto sopra C. 2. stan. 50. et
REST A sn parata, Si ferma in guardia, cioé con 1a spada pronta, ed in posi-
tura comoda a ferire, E' termine da schermitori. yori
MERCE', Con la prima, €;, firetta, ela seconda longa, vuol dir mercede
che profferito al contrario vuol dir mercanzia: Nel modo che:è detta nel pre-
fente luogo, ed in molt' altre occasioni mere vuol dire per causa di cid: qual di
ca io riconosco tal mercede, tal benefizio da questa cosa, o da i,
ec, ficome Paride riconosce questa mercede, 0 benefizio di non si fidare del Dia.
volo dal fapere, che quello ¢ bugiardo, ed ingannatore. Questoiderto ¢ lo'ftelio,
che Grazia del marcello, ¢ degli foroni, che vedemmo sopra in que(to C, fran, 20,
1L Diauoloé futtile, ¢ fiia grofo. 11 Diavolo & fagace, ed inganna l'huomo,
facendo il goffo, ed il balordo. * inet
REST Aa bellu (guards, Reled guardando attentamente. Bello fguarde® unas
villa poco lontana da Firenze: ¢ per 1a similitudine che ha questo nome bella/enar-
do con il verbo guardare si piglia in detto significato. pn amaetir
 CAPOLLA di pan duro, Mangid adai. Gii mangid tutto il pane, che haveva
in casa, gliclo rifint. Detto usatifiimo per esprimere Aéangiare assai ee,
spy paler

ays Nae

FINE DEL DECIMO CANTARE. eae

 
  
   
  
 

 
 
   

 
  
 
      
 
  
 
 
 

 
    
    
     
  
    
 
   

 
  
 

|

 

eee
eh ARGOMENTO, '
St

Cangia le dance in rifsa un? accidente, aS
iSe

Fuggonfi Bertinella, ¢ Martinacza,

-VNDECIMO CANTARE,

Vien fuor Biancone, ¢ fa morir gran gente;

5 Ma gli Orbi a tui fan poi sentir la mazza, 6%
es Da Celidora, ¢ da Baldon possente 33
ee Mezza defirntta ¢ quella trifta razza; th

Taghanfi a pezei in quelle squadre, ¢ in queste '“*
E così in ata? fanfi le feffe. e a ge

2 —
RAPALA AAAS AS

om STANZAL STANZA Ik
Chi mi.dard la voce, ele parole ui ci vorria chi scortica L' agnello,
'antia dir la guerra indiavolata; Es al mondoé persona pil inumana,
Ond' oggimas dara le barbe al Sole © descriver la frrage,ed il flagello
Bertinella con tutta la sua armata; Che seguir si vedrd di carne humana;
C'alCiel Gagliarde alzando,e Capriole, Ch' io gid oni fento, mentre ne favello,
\Farò.verso Volterra la Calata, A tremito venir della quartana.,
. Efe d' amor canto con cetra in mano, E n' ho si gran terror, ch'io vi confefi0,
<Derd col ferro il ve/pro Sicilsano ? Che mai piu de'miei di farò quel de/so,

Tinoftro Poeta volendo.nel presente Cantare narrar la battaglia seguita ia Mal-
mantile, ¢ le crudcita grandi, che (uccefiero nel Palazzo della Regina, dice, che
a fac tale descrizione vorrebbe efier un' huomo fanguinario, quanto è colui, che
feortica git agnelli; che non si spavencerebbe, come fa egli acl rammentarsi i]
grande firazio, che fu fatco di carne humana in tal batcagiia. Qui immuta Dan-
te-nel principio del C..8. dell' Inf. che dice;

i Chi porrsa mas pur con parole scialte
Dicer del fangue, ¢ delle piaghe a pieno
Ch! 10 hora vidi, per narrar pitt volte ?
4 mi lingua per certo uerria meno, L
— avventura seguita Vergilio nei 6. deli' Kncid., che dice, imitando pures

°

Qqq 2 Non
 

 
 
   
 
  
     
   
      
     
 
  

492 MALMANTILE
Non mibi'y si Sassen or ag
Omnia penaru ee omina polfem.—

E così rende l'uditore attento,  curioso, col promettere di vol
venimenti così maravigliofi, che non ¢ per trovar parole adegu
ne esprimere. A! > bet: F e

'DARA Ie barbe al fole, Morira,. E' traslato dalle piante, le qu
cioé si feccano, quando si fuelgono, ¢ si voltano loro le barbe al So

GAGLIARDA, e Calata, Sono-duc specie di danza, ob
scherza con la voce ealata., che vuol dir caduta, oftela, d
ver fatte qui Gagliarde, e capriole fara la calata,, cioé calera verso
comunemente s* intende andar forterra, cioè morire. Jay +e

DIRA il Vespro Siciliano, Dopo haver cantato versi amosgh ante fj
Siciliano, che s' intende; vedra, € provera stragi. B' nora la follev ne de
ciliani (orto Gianni di Procida contro a i Francefi nel cempo, che questi ti g
giavano la Sicilia nella qual follevazione fu il egno, che un determina gi
al fuona del Velpro ciascuno si moveffe contro a i Prancefi, come se
fuccefle granditfima Mrage di essi Franceli; E da questo & nato il '
Vespro Siciliano; che vuol dir fare steagi,ammazzare. Vedi Gio. Villanil
61.¢ Giachecto Male(pini nella Continuazione della Storia di Ricor
cap. 209, >

Hil festive l'agetib + Sona tial yarenaisinmeeltaie
i quali nel tempo, che sono gli agnelli, vanno per Firenze gridando. Ch
scorticar l'dgnello; per bulcar denari in ammazzare, ¢ scorticare Metti ani
il nostro:Poeta da quello (canaare,\¢:scorticar un' intinica di'effranil,
puta huomini crudeli, ¢ senza pieta, ¢ questa per'accomodarsi abgenioy"e cap.
cita de i fanciulli, che stimano quell' atto una granditlima inumanita,
nando quelle beftiuole innoceati. ny sieht

FLAGELLO. Qui è prefo in significato di eopine, farts ee CA

   

   
    
   
   
 
      
     
 
 

  

  

di. Vedi sopra C.1. tt. 45. invaltro signiticato. In Gio. Villani trovafi nel fen ayy
usato qui dal Poeta; F/agello, ¢ Fragelio; come costuma di dire anche aug
piebe Fiorentina, ¢ come dissero i Greci, ¢ si legge ne} tefto Greco dell®; pac
fey

uy

dy

 
 

hio, Phragellion per quello, che i Latini dicono Fracetium Omcto
sgrazia,sferza, 0 2agello ds Givve vinci Node tibro 12) verlo 397%
831. Attila Re degli. Vani tu soprannominato per quelo, Frage! t
TREMITO dela quartana, Quci brividi, che G-fentono' dal pazavente nell'en-
trare della febbre quartana, i quali sono aflai maggiori diquegli 5 che foglions fie
venire, quand' uno ha qualche spavento}-eperd 'con dives VA tvemiep dela pias edy
sana, intende, che lo (pavento cra grandissima,€ fuori dell' ordinario: E «ali tend
brividi, 0 tremiti vengon' allt huomo:, perthéla'patira fringe il cuore; per lo

  
 

che il fangue corre tuctovin aiuto di eflo;¢percio--membri efteriori, ¢ ie parti te
superficiali, ed cftreme rimangon\fredde; edi steddo facendo riftrit i pori, be
cagiona quel che i Latini dicono rigor » che farizeare i capelli, © pels "€ Cagio~ ni
na il cremito, il quale si domanda capriccio, ¢ rsbrexzo, Vedi C. 6 Gig

MAL più, de' miei di fare quel defo, Spaurisco tanto, che esco

 
 

 

 
    
   

cro prima.

ets DAN ZALHI8.'

be il galio apportator del giorno
La notte nera pits d! un Calabrone,

Bil sua buio,e quant'abre eli'ba dintorne
Diognise qualungue grado,e condizione,
| Acid ficuri omai faccian ritorno
\ Gli nccei, cantando il lor falfo bordone,

AIncitr'al Sol,ch'in quespa parte,e in quella
| Fa pel lor gorxo nascer le granella
lead ety

 
    

Perché-crafeun » che quini si ritrova,

 

VNDECIMOCANTARE. 493

“fino a che viverd » non farò mai più allegro, come era mio solito, perché quelto
- spavento m' ha fatto mutar compicifione, ¢ temperamento: Non saro piii, quel

STANZA IV.

Quand' infra Dame, ¢ Cavalieri erranti,
C' al trescone in Palazzo eran intentiy
Comprefeun dietro all'alero i duellanti,
Armati tutti due, come fergenti,

-Si shallo il ballo, andar da cantoicanti,
Ele chitarre, ei mufics Srumenti
Ai proprj fuonatori, ¢ balierini
Divenner rante cnfie,¢ berrestini,

STANZA V.

Si fa pero bifbigtio, ¢ si rinnuous

 Kedendo entrar quell' armi coid dentro, L? odio fra te farion gid quafi sperto,
 Subirovdiffe: Qui garta cicacca: Che tirando ai rispere: gu la bufa,

: =, £ trama di qualche tradimento, Ruppe la tregua, e rappicce la xufa,

4 iver la Jevata del Soley ¢ dice, che in fu quell' hora entrarono nella stan-
22, ove si faceva il ballo, Martinazza, ¢ Calagrillo, che la seguitava con l'armi
F inmano:; per lo che si lascid flar il baliare, ¢ si venne all' armi., rompendo las

tregua, perché ciascuna delle parti sospetto d' esser tradita, ¢ che questo fufle uno
— militare, come i ditie sopra C, 10, flan.31. dove laicid questi duel-
EL gaily apportaror del giarno sbandina 1a notte. 1) gallo ¢ solito cantare in full'ap-
pariridel giorno, ¢. però dice ch' eglié apporrasore del giorno, e che da 11 ban-
do alia notte col suo cantare. Somniaque excuffit nuncia lucis aus, disse ua Poeta;

Excubitorque diem cantu predixerat ales, canto un' altco, & erifta /pettabilis alta,

Auroram gallus vecat applandentibus als, Disse il Poliziano nel suo Villano.

CALABRONE. E! uva specie d' infetto, o verme alato di figura simile allas
mofta »maatlai pil grande, ¢ di colore ncgriflimo, ed ha un jungo, forte, e»
acutissime pungigiione. Con questo nome chiamiamo,ancora il tafano detto fo-
Corot. 8. 1 Greci Prouerbilti ditiero fearabao mgrior, Pitt nero dello scara-
B10, che ¢ un' altra specie:di mosconaccio. i
4N comro.al Sole, Giivucceili vanno incontro al Sole cantando in ringraziamen-

to delbenefizio, ch' ci fa joro, maturando le biade per loro alimento.

* GOZZO. E! il primo ventre degli uccelli, cloe quella vescica, che hanno ap.

Ppit-del colio, dove si ferma il cibo, che beccano, edi guivia poco a poco si di-

Mtribuilce al ventricolo; ¢ da noi si piglia ancora per la gola dell' huomo, perché

vien da gutrur. re

° CAVALIERI erranti, Così fon chiamati quei Cayalieriavyenturieri, che fon

descritti ne i Romanzi Spagnvoli da loro detti Cauaheros andanes; wa qui inten-

de, che erravano perché stavano ballando aliora, che bilognava combateere.

“| TRESCONE, Specie di ballo, cos detto da Tre/ca balio anuco. Vedi iopra

G. 10. st, 28, Dante Purg. 10.

 

: ee S=TePiie etre

= ee

a

i

 

 
 

  
 
 
 
  

494 MALMAINTILE® (0
Li precedewa al benedetto Vafo

Trescando alzato, 0 umile S. ane cond
cioé faltando, ballando. M As +
SBALLO'. \\ verbo shallare vuol dire disfare le balle; ma qui

re il balio, In buon Toscano non si direbbe shallare il dar fine al
pis la forza della lettera 5s, aggiunta al principio di verbo, 0
ignificato contrario si come la particella, i», appreffo i latini, |
tare, spiantare; grariofo, fgrariaso, ec, ma il Poeta se ne s
scherzo di ballare, e sballare, e (eguita il bitticcio # dar da canto s canti
figuratamente sbaf/are, per eccedere la verita ne' racconti; ¢ © ¢
numeri di cose con vantaggio, ¢ con caricatura. " *
DIVENT AR caffe, ¢ berrettini, ec, Cuffia, come s'e detto sopra C, 8, fh. 48:
una berretta fatta di velo, o di tela.a foggia di facchetio usata dalle |
ferrar dentro i capelli in capo; dice, che gli Prumenti vennero caffe ye
perché le chitarre, ed aitri strumenti simill corpacciuti, eflendo bateuti in
capi di coloro, ¢ per la loro fottigliezza sfondandofi, fecero I effetto
be in ful capo la cuftia, o berrettino, cioé lo ricoperfero, e ferrarono in
E' detto usatitfimo. Ti faro wm berretrino delia chitarra, per intendere i
chitarra in fu la cefta. Vina timil frafe venne in capo a Omero nell' Iliade, quan-
do disse, Lapidea indui tunica, per voler dire, Essere tapidato', quafi il ricoprires
uno di faffate, sia uo fargli un veftito di pietre, che gli stia bene alla vita.
GATT A cicova, Ci€è mifterio foro. Ci ¢ inganno, eum Tras tiled
i Latini. tet ain
TRAMA, Si dice quella feta, ec., che serve per riempiere le a
renza dell' altra, che serve per ordire, che si dice orfoio; che per la più n
si dicono ordito, ¢ ripieno. Dante Parad. C. 17. t soi aged Rl hn,

  
   

  
  
  
      

    

  

    
   
    
    
 
  
         

Poiché racendo si moftro [pedita Tat

L! anima fanta di metter la trama Che

Jn quella tela, ch' io le porsi ordita, (SRE LS Sir

'Ma trama Gi piglia per concerto, ene habbiamo il verbo tramare, cheiwuol dir bag
negoziare copertamente, ¢ forco mano, dilegnare,, concertare, Mraletrami ge ha,

fio affare,ec, Bdicendo: Queffaé trama ds quaiche tradimento, intendes/Queho Oe)

  
  

 

@ tradimento concertato. Latino /ute/a doi. Varchi Stor, Fior, lib Cm
d' una conuenzione facta senza saputa d' un terzo dice + Orazio se ne? ada
rugia, senza che il Sig, Gentile fuspicasse non che sapesse cosa alcuna di questa' i
trama di gocciola per intédere specie d apopicsia,quafi una coperta apoplethiaye da 4
questo si potrebbe intendere per rrama, uaa (pecic; ¢ dire questa è specie di qual i
Che tradimento. Storia di Scmifonte Trattat, 3. dice. 4 popolo fa fallewe 5 ¢ grida tha

na, [uspwcando, che trama ui falfe, contro di lus, speotepecnh aot ¥

  

BIS BIGLIARE, Dilcorrer in fegretor, che si dice anche Far Pith pifft; ij
Pispigiiare, che usd Dante Parg. C. 5. Skit ise Bap w
Che si fa cio, che quini si pispielia, “ ¥ they

E si dice pi/pigio., ¢ pispigiio, forta di cicalamento; e viene da quel fafurrio, che: hi

featiamo da coloro, che parlano in fegecto.. toggi pia comunemente fidiceb® =
Soighiare, bifvigtio, ¢ bifrigtio, 5 te te
Th Ry

 

ae ae
 
 

  
 

na, O rispetto
 STANZAVL
metre man da buon Soldato y
imico ritorna a Bertinella,
f quale in quel punto casco il fiato,
UM fegato, la milza, ¢ le budella,
Vedendo, quando men' hauria pensato,
 Vicire i pefei fuor deta padelia,
 Mtentre 1a fa venir Adarte vighacco

     
   
  

Col suo Baldone alle peggio del (acco.

STANZA Vil,

 

 

 

} VNDECIMO CANTARE.
| | TIRANDO git La buffa.a i rispessi. Non havendo pili rispetto, 0 riguardo al-
cuno. Sxffa intendiamo una berretta, la quale ¢ fatta a

f » € mandata gil cuopre anche tutta la faccia, ¢ i collo: Eda questo
la faccia, mandar gis /a buffa., vuol dire oprare senza riguardo, ¢ scaza

 
   
   
 

495

foggia di morione, che

STANZA VIIL.

Mentre 8 alcun t' offerua, ella pon mente
Per canfarsi enon esser appostaca 5
Ecco in un tratto vedefi presente
Martinazza la sua confederata,

Che poco dianzi anch' ells fimiimeute

Di man di Calagrillo ¢ feapolata,

E feco vanne in luoghi occulti, ¢ fenré

A fare wncanti, es faliti (congiuri,
STANZA Ix,

eit a © un certo vento non le gusta, Nes quali aiuto ella chiede a Plutone,
Che fa le (pade,e ognor per l'aria sischia, Ed ¢i comparfo quixi in uno ispante

wiil —- E.grd vedendo che (a morte aggiufta Dice, c' ha fatto a lor riquifizione
yee] Chipievnol far det brano,e pin starrischia, Gid [pedire un tacche per un gigante
at Bel bello fuigna, ¢ vanne alla rifrufta Qual' è quel famofissime Brancone,
it | Dun luego da faluarsi da tal mischia, Che col bartaglio,ch' era di Morgante,
pt} —- Adtischiayche non gli par di porer credere, Verrd quini tra poco in lor foccorso
- Ee Percio sospira, ¢ non si puo discredere, ef dar picchiate,e' hanno a pelar  orfa,
& votes: f STANZA X. xi

Ed eccolo ( foggiunfe) ovvé battaglio\ E 8 anuedra,c' al fin piscio nel vaglio,

© desi fo dir ych'il primo,ch' egit accoppa,
Tatra l'armata a irfene in sharaglio
Che la barba penso farci di froppa;

E che al pigliar un Reeno non è loppa;
Cot scaciata abbaffera la crefta
dn veder, che de' suci non campa testa,

Si rappicca la battaglia, ¢ Bertinella eflendofi perduta d' anima, per vederes
i ritornato suo nimico., quand' ella pensava d' haverlo tutto dalla sua, es
-temendo di non efler ammazzata in quella Foote » meditava di faluarsi in qual-
4 che ficuco, ed appunto-s' imbatcé in Martinazza scampata da Calagrillo,
J € con essa en' ando in iuogo appartato a fare incanteGimi, per coftringer Plutone
F -ad aiutarle; ed Egli comparfo quivi dice, che si fara venire il Gigante Biancone,
il uals in questo dire arrivO quivi, ¢ Piutone rincuora le donne con raccontare
la bravura di flo, dalla quale da loro per distrutta l'armata di Baldone.
LE casca il fate. Si perde d' animo. E foggiungeado: 4 fegato, la milza ye,
, te budelia, intende Si perda d' animo affatto
— QFeANDO men fet è pensaro. Quando meno dubitava. Non expettato valvus
ab hoffe culit..

VSCIRE i pesci fuor della padella. Perder quel ches' era acquiftato, ¢ sopra di
che s' era fatto aflegnamento certo, ¢ ficuro.

VENIR alla peegio del facco, Venire al maggior fegno di discordia, e di rottu-
ta, Nelle guerre il peggior grado, che sia, €, quando le Città,0l'Armate fon
meffe a facco; ¢ però dicendofi /e peggio de! facco in peggior grado, ¢ condizio-
ne, che è haver il facco. VL

,

 
|

 

 

   
 
 
 
   
   
   
      
     
    
 
    
 
     
   
     
    
   
  

ee
496 — MALMANTILE © 7
VIGLIACCO,, Vile, codarda., EB voce spagnuola, vells
significa furbo,¢ furfante, poltrone. i
SEL bella, Con bella maniera, ¢ senza dar © del
antichi differ; bedlamente,manonéinufo..
SVIGNA. Se ne va con preftezza, o fugge. Forfeda questo
viene ¢ omprare if porco, che vuol dite anch' egli Andarfene
fuinam, 010% fuillans emere. Ed ¢ usate quetto verbo fuignare
besco. Vedi sopra C. 4. stan. 51, Si potrebbe anche dire, come pei
erudito, che questo verbo fuignare ligniticaado scappar dalla Vigna, s°:
scappare di foro la Vigna, strumeato o macchina milicare, che serviva
tichi per andare (otto ie muraglic a combatier le Piazze, con le quali”
difeadevano gli atiecianti da i (aii, ed altre cose, che erano: buttace lor
dagli affediati, le quali necetiitavano quelil, che vi erano.coperti a
forto alle medesime vigne; extra vineam exire, che (uona fuignare.
VANNE ala rifrufia, Vuol dice cerca mioutamente, ¢ con diligenza
NUN si pus discreaere, Non pud non credere. Non pud creder, che
a cffer così, ¢ non habbia a eficre altrimenti. Non pud capacitarli
SCAPULAT A, Fuggita; Scuppata. 3' intende scampato il pericolo
LACCHE', Ragazzi, cae corrogo appiedi per servizio de' loro
di sopra C. 2, fan. 29. 2 ae
BLANCONE. B' quel coloffo di marmo bianco., fattura dell' Ammannato, il
quale ¢ posto in Firenze nella Piazza dei Gran Duca, dentro a una valea gran-
de, la quale riceve l'acqua da diverse fontane, che scacuriscono da detto: fo
¢ suoi annetii; ¢ se bene rappreicnta Nettugno, ¢ chiamaco da cutta M Biancones — ui,
ai ee ray Vaca; 1 hi
MORGANT E, 11 Pulci in un suo Poema intitolato il Morgante narra'; che

  

  

   
    

 

  

Ms
quetto era un Gigante, 1 quale nog adoprava per coubattere alt': he ua Ya
gran battaglio da campana, joe alo tf

PICCALATE ¢' hanno a pelar  orfo. Picchiate gagliarde, perché il) pelo dell' Oh
orfo efiendo difficile a suellere, ¢ pelare » non si fa caicare con' ky
se leggieri, Pelare, wattandofi di muraglie, 0 pietre vuol dire-space; ol
si, 0 (crepolare, onde potrebbe dirli hanno a peiare  orfo, cioè tare fore yi
rompere l' orfo, che Gi dice quel pictronc, che adoprano gii fiyfaiuol FN
lire i piano delle stufe, onde nabbiamo poi menar 1' orfo.a Atoaan Pre
re ripulir Modana, ¢ Ggnitica mecterii a far una cosa umpolsibue uk

PENSO' farci la barba di Stoppa, S'intcnde; E poi dargh tuoco.
Penso ingaonarci,.¢ por farci ogni maggior danuo, ie
PISCIO' nel vagiio, Blo stetio che far 1a zuppa nel paniere desto sopra C.
flan.7. E.che cola sia vaglio, Vedi sopra C, 2. stan, 79. Luciano in ab
co volendo spiegare, che il far bene a' crifti ¢ come un tar la 2upp.
perché 1 benetizzi riceuti (cappano jaro prettissimo dalla memora; 4
buomo cattivo,e sconoicente a una bog forata, che uo quello, che va i met. tes,
te, si ver(a. Plauto nei Pfeudolo, o vogiiam dire Bugiardello; 2Vae piuris refert, ="
quam si imbrem in cribrum geras, Corcisponde questa maniera alia noltes (char
nel vagtio, Luciano nei Live dic; come da in cofano forato, ©

ey

  
 

 
 

VNDECIMO CANTARE:

 
  

497

)zuppa nel panicre. Playto pure nel Pleudolo, la pertu/um ingerimus. dicta do-

opera ludimus. La favola delle Danaidi ha fatto luog
nifica non ¢ cosa facile. Loppa; che si dice

19 al prouerbio.

 
 

NON: ne Detto bafio, che
anche lolla,
anche

ce » gli c levata.

a STANZA XL
Qui tacqueil Diawol,perch' ¢ fatto rece,
“ él aria al capo git è maligna,
anuerzo 4 flar fempre nei fuco,
Vatea alle donne il dietroacafa,e/uigna,
EB lafesaus il Gigante nel sua law,
Che douendo a Baldon grattar latigna,
 Sull ulcio det falon gid perwenuto,
edge Hf batragiinje questo fu il faluto,
STANZA Ali,
Sei braccia era ti bascagtio aito, e ds paffo,
| Bm injragnena aimen arciotto,o vent,

  

4; Ma dando fu nei patcormando a baffo
cha  Van trang intatiata, ¢ tre correnti,

; E fece tai frafiuano,e cal fracasso
14 Che shalord? « un tratto i combattenti,
oh OE per pawra, a chi non fu percoffo

we |, Nomrimafe sr quel punto/anguc addosso,

il gulcio, che si leva di sopr' al grano quando si bacte, che si chia-
inche pyle. Lat, apinde secondo Nonio Marcello gramatico. 5

Y SCACIAT A ~ Rimanere scaciato; vuol dir Rimaner buriato, ches' intendes
; nd' ugo credendofi confeguice una cosa, ¢ facendolela sua, 0 non Ia confe-

 
   
 
    
  
   
 

 ABBASSERA la crefta. Gli feemera!* umore, o I alterigia, I Galli d' In-
dia, quand' entrano in frenefia, gonfiano,, ¢ cresce loro la crefta, € patleggiano
on una certa intronizzatura, che par (uperbia; ed usciti di quella frenefia, sce~
ma, ed abbaifa loro ja creita,¢ di qui vicne il presente dettaco, che significas
readerGi umiie, contrario di Rizzar (4 crea,;

STANZA XIII,

Ed infra gli altri Piaccianteo, il quale
S' era schermito bene infizo aliora,
Vedendo un fantoccion si badiale,
Dopo il terror di tanre [pade fuora,
Di quel derto farebbe capuaie,

a9 C' un bel fuggir faina la vita ancora,
444 perché in quae in la v'é mal riscotro,
Vede hauer vifo di fentenra coniro..

STANZA XIV,

Poiché non fa tronar modo, ne via
Per nellun verse da [campar laguerra,
Ech' ovis ¢ forza, che chi v'é vi stia,
Pond morto, gettafi gilt in terra,

E ritrouando la botrigleria

Apre t armadia, ¢ dentro vi si ferra,
Con pensicro di fiarni fempre occulto,
Fin che si quiet così gran tumulto,

! Plutone si paite dalle Donne,e la(cia quivi il Gigante Biancone, il quale andd
, alla lanza,dove si faceva ia zuffa, ed arrivato in fu la porta alzd il battaglio,
: per comigciar con esso a perquotere, ma al primo colpo dette in una traye, la.
quale per efler fradicia, si fraca(so insieme con pill correati. Tal colpo spauri

» tutti coloro, che eran quivi, ¢ particolarmente Piacciantco, il quale fino allora

8 era ben difefo, ma per Jo spavento, che hebbe dei Gigante,

getto in terra,

s fingendofi morto, ed a poco a poco si condufie all' armadio della bortiglicria '

b nel quale entrato vi si erro.

 

si #ATIOr«0, Divenuto fioco. Vno, che per catarro, 0 per altro impedi-
} mento aell' aspera arteria ha perduta la chiarezza della voce, li dice rancus,don-
y de rancedine, ¢ reco. Dan, Int. C. 14.:

A, Erendele a colui ch' era già reco, £.
(| Li aria glié maligna, 1? aria gli nuoce, gli cagiona danno,

1, dietra a casa,¢fuignua. Volta le reni, ¢ se ny « Bil verbo /ujznare,detto
rr GR.

Poco sopra neil' ovtava fectuma, 

AT.

 

 
  

   
  

  

MALMANTILE® ©
S' initende perquotere. 'Così I intende
. fo direi anche, maio temo, che'ella

ny Won s apparecchi a grattarmi la tigne.

Si dice anche cacciar 1a mo/ca da deffo, in questo C. stan. 20, !
dajfar la lana, sopra C, 7, stan,63. Adandare a Legnaia,sopra C.
ter Ta poluere, sorto C, 12. stan, 1, E tutti hanno lo steffo signi

INFRAGNERE. Ammaccare, 0 pigiare una cosa tani
forma., come farebbe Peftare un fico maturo, ec, ¢ il Lat. ¢/
Vedi sopra C, 4. stan. 76, ¢ forto in questo C, stan. 17.) *

INT ARLAT A, Rofa dai tarli, che sono quei vermi, li
dentro al legname.,, ¢ di ele si nutricono; da i Latim detti rer
'C.6. stan. 59.

FRASTVONO, Fracasso. Sinonimi, che significano Romore, stre

NON gli rimafe fangute in deffo. Acbbero così grande spavento, che

mate spirito, Dicono, che a uno, che habbia ha'vuro un granditimo sp
© paura, se in quel punto gli fule tagliata una vena, non gliu
per le ragioni accennate (opra in guefto C, stan. 2, d

S' eva Jchermito bene. Cio',s' ra difefo.. Havea scampato il toccatne

BAD/LALE, Grande, Si dice anche machofo, imperialc, € simili,
'scherzo; ¢ significa grande più del naturale. Kose ee
VN bel fuggir falna la vita ancora, Alla (entenza che dice Vn'bel morir tutta le
vita honora, rispondono coloro, che flin:ano più il vivere 5, oe

 
  
   
 
  
      
   
   
   
    
   
    
  

Sa ney

bony

         
  
    
  

Vo bei fuggir fainn In vita ancora, 7 ag
V" è mal viscontro, V' è male il modo. Non W'@ buona congiuntura, ~ io
VPEDE baker vifo di fentenza contro, Conosce di non'haver ragione, cioè, che il Mt

'ncgozio non è per seguire, com' ei vorrebbe. A tthe? Wy
CAl ve vi sia, Chi ha havuta la disgrazia, se la ianga: E si dice: Obi v'é ui

vi sia, ¢ chs nén 0° è non v" entri, qui però intende; chi in quella stanza viftia, tt

perché aon se ne pud ustire. Reet eee
BOTTIGIERIA, Armadio,o flanza, ove si tengono V afi da Vino ' a)

¢ servizio della menfa. Voce, che vicn dal Francele Borteille, che e Cl

'fiasco, 0 altro vafo simile da vino. 4 4

STANZA XV. 7

'Col battaglio di nucuo agile, ¢ prefto © già ch' egli non puo tt

Tira il Gigante 5 ¢ da nella lumiera', etrmeggiar col bat: 'et lento y f k,
Ls qual cadendo fece del suo refto, 'Pero che il toga non ha gran diffanza, fr
Perché i [pense, ¢ rope cid che v' era; “Cagion ch' ei trowa fempre' mento; a
Hor, 8 eglie in beftia, dicanelo questo, 'Lafeialo andar bawendo pin fidana x
Mentre ch' ei da ne' lumi intal manera, Nelle sue manych' in simile Srumenb hei
E dice che'! Demonio lo jtafila, E piglia guells ciurma abbietra,e sbricia a
Poiché eli fa faltir due colpi in fila. 'eA4 menate 5 com' anici im camicia yj
' STANZA XVIL. oe ee fg
'Così tutto arrabbiato, come un-cane Talche'l me/chin non mangera più par ad
Piglia un pel coho,¢ feactialo nel muro, Percid gli amics [uci, a
Di forta, che disfatto ei ne rimane “We voglion, che il ribaldo,
Som' wie ficaccia piattalo matures Gli andaron alta-vien tush quant

Stet a. ac? ae
 
     
    
  
   

VNDECIMOCANTARE. 492
STANZA XVIIL STANZA XIX.

"sion cofforo un brance di galletti, E come la mia Serva, quana' in fretta
Quando la fhate, a tempo di ricolta, Dee fare ilpesce a uovo,e che si caccia,
Antorne a qualche bica units, ¢ spretti Trama due nova einfigmele picchiet:s,
nun di loro a berricar s' afolta, Sicche in untempoturte due le(chiaccias

Pere il Gigante fa certi feambierti, Bs che dall' tra ¢ [pinto alia yenderca
Che re ne [uifa quattro,o se per volta; Softien quei due,es' apreneliebraccia;
Infaffidico al fin da quel baccano, Poryciacche,pacte insieme quello,e quefie;

Si china,ed aggavignane un per mano, Stcche e diwentan prit che pollo pefto.

. Biancone con un coipo fracafia la lumiera, ¢ spegne tutti i lumi. Nota che,
i se bene era di giorno, la lumiera era tuctavia accela, il che speflo aveiene in ta-
lioccafioni di veglie y che i segiivorl distratti dal gusto del ball,fanno mezzo
— fenz' avvederfi, che sia pafiata la notte, Ll Gigante in collera Jascia il
ttaglio, ¢ comincia a pigliar quelia gente, ¢ bacteria per le mura, onde tut-
tian tratto gli corsero addosso, ma egl si difendeva, facendo di loro ua gran
-maccilo.
LVMIER A, EB vno strumento, col quale si softengono in aria pitt lumi acce-
si, che i Latini dicono Lychauchus pensius, luceraiere in aria.
FECE del sue refto. Far dei retto s' intende fipire ja roba, la vita, ec. qui dun-
que vuol dire si speafero atfatto 1 lumi. <
B in beftia, B in collera., Dar ne i lumi, vuol dire entrar grandemente'ty col-
Tera, dar nelle (candescenze; ed è Jo steilo che dar nelle furie, ed il Poeta (cherza
 con questa metafora di dar ac' lumi, ed intende dare etfettivamente col batcta-
- glio ne i lumi della lumiera.

ail; 4L Dianol to feafiia, 11 Diavolo lo perfeguita; Gli ¢ contrario.

IN fila, Vo doppo l'altro, fenz' intramezzo.
ot CARMEGGIARE, Questo metaforicamente significa Aggirarsi, o affaticarsi in
ibs vano; ¢ signitica anche ingaonarsi, per efempio: Tu armeggi, se tu (peri d' ot-

tenere, ec, ma qui ¢ prefo anche nel suo proprio signiticato di mineggiar lara;

gh Cnell' altro d' aggirarGi. —

wo) CWRMA. Genraccia vile. Vedi sopra C, 3, fan. 76 ¢ C. g. stan. 16,

ABBIETT A, ¢ sbricia, Sinonimi, che figaincano vilitfima, minurifsima gente,

A manate, Da i pili si dice menare. Quanti a' entrano in uaa mano; ¢ per la
grandezza della mano del Gigante fuppone il Poeta, che fica moltiimi per vol-
ta, perché dice: came anici sn camicia, che sono anici coperti di 2ucchero, de i
quali con una mano se ae pigliauo le centinaia. 

FICO piattole, E' una specie di fico detta così.

NON voglion ch' ci se ne vanti. Lo voglion gattigare, perch' ci non s' habbia a.
gloriare d' hayer ammazzato quel loro amico.

». BlC-AQuafi da il Lat. Barbaro apica dal buono -dpex. Così chiamano i Conta-
dini quel monte di grano in paglia a mazzi, da loro così accomodato, affinché
si flagioni, pec poterlo cavar dalla spiga; deta da 1 Latini rrieict congeries. Das
questa voce bica habbiamo il verbo sdbicare per accamulare. Dante laf, C. 9.

Come le rane innanzi alla mmica,
Biscia per l'acqua si dileguan tucte
Per ¢ alla terra ciascuna s abbua, Rrr2z~ BEZ-

SEE CERCA ES

 

 

 
 

 
  

500

MALMAN TYLER: 1 v

BEZZIC ARE, MW beccare'de i pollaftrelli si dice bezs
FA certi feambietti, Cioè contraccambia le percofie,

  

ra

Scambietto * termine di ballo, che significa mutanea'
INF AST IDITO da quel baccano, Klicndogii v«

  
 
 

 
 

si
sopra C, 4. stan. 9.

 

 
 
 

 

Allor Bieco non ha pite fofferenza,
E giura, che di questot: Bacchillone
Von andra al Prete per la penitenza,
Perch'ei vnol, chee' la faccia col baffone;
Ei fui, che di ral arme ban da teenza
Gite ne daran a una fanta ragione's
Così guida i fuvi ciechiyow' ¢ il coloffe,
Accto gli caccin le mosche da defo.
STANZA XXL.
Eglino tutti quini fermi a tiro
Presso.a Biancone aun fiscbioco' baitoni,
Senza tramexzo alcun, senza respiro
We diedero un carpiccio di queé buoni,
Ed egli con un piede alzato in giro
Fa lor sentir, s' egli ha fodii talloni,
E mentre questo paffa ye quel rientra,
'Con quel pedino te li chiappa,e /uentra,

 

'Bieco veduto questo fa vehire-i suoi Ciechi,i quali tutti in giro ini

la importunita. La voce baccano, che significa combat esett
piglia nel fenfo, che si piglia musica, felta » bordello, '

Quand! ecco rt veccbio Paolino

 

 

Ve anti

AGG AVIGNA, Piglia,¢s' intende cinger con la 'mano 'tu

glia, in manicra, che si possa tenere stretto con factiita,
PESCE a' weno, Vova fritte »0 frittata, che dicemmo sopra C. 9
s' intende propriamente la frittata, che dopo eer cotta 5 0
ruotolo, pure nella padella; rifritca, ¢ ridorta in figura “di p
ta pesce d'uono, La Compagnia deila Lefina dice: La 'consner
antichi, i quali conrenti a' un pesce d' uouo di due woun al pile

ClACCHE. Questa parola non ha verun significato', ma folo
no, che fanno l'uova, ed alere cose similt, quando si rompono, edil
ne serve pr esprimer quel bateere, che fa il Gigante di'quei due hi
tr' all' altro, ed immita Dante, che nell' laf. C,32,dice:
LVon hauea pur dail' orlo fatto Crich
E seguita i Latini, che pure 'hanno a finta voce Tax,
come si vede in Plauto in Perla; dove per intender buie dice > Tax
meo. E noi pure diciamo'tach, ¢ pach; anzi le percotie da molti in F
cono pacche, come dice anche il noltro Poeca sopraC, 5. st. 47. Da
ta la parola Fiorentina dcciaceare,, che ¢ lo ttetio, che' Pefeare
dicefi 'Pepe acciaccary; modeftamente infranto,e Acciaceo sopi
do uno per così dire calpefta,¢ maktratta un'alero., ”
5 j

3¢ per

  

la hile elbétae,

a;
STANZA XXN

Aquat fa pits cagon,cblT efti,e'
E ( perchegti e bizzarre) bam
Condotti com' ei fuole,un par a

OveSalito a Petigion di

Vavol matel,ch'egis' ha de ee

T aftando,owe il Gigs

 
  
 
 
 
 
 
 
 
   
 
 
 

£ darel ccc ieP bocca

 
 
 
 
 
 
  

SEafi Si =F eo &

      
      
   
       
     
     

 

oy
Wes,

ee
a

aie

 
  
      

VNDECIMO CANTARE. yor

affaltano co baftoni, ¢ Paolino falito sopr' a i suoi trampoli metie i) suo
iuolo sopr' alla faccia-di eflo Biancone, il quale però s' adira, ¢ beltemmia

i suot falfi Dei. Pah
| BACCHILLONE, 0 Bacchiglone, E nome d'un fiume, che paffa dalla Cita
| Vicenza, in Latino detto Azedoacus minor (econds Fra Leandro Alberti; ed ¢
ida Dante Inferno 15.-ove discorre d' uno, a cui fu permutato il Velco-
irenze in quello di Vicenza, che dal servo de' servi Fu trafmmutato d' Arno
one. Da quelto fatto di Mefier' Andrea Mozzi, che così si domanda-
Vescovo, o pure dal verso di Dante nacque in Firenze il proyerbio; del
fanno teftimonianza il Varchi nell' Ercolano, ¢ il Borghini. Sacare d''e4r-
in Baechilione, aitudendo al saito dal Vescovado di Firenze a quello di Vicen-
y che significa faltar d'un proposivo in un' altro s Saitar ai palo i frafea: Ma
-quelta voce Bacchillona aggiunta a huomo significa huomo infipido, ¢ buono 4»
. oe » ancorché di persona grande; ¢ fuona lo tteflo, che Gaicone, Palamidonc,
i: » ¢ simili,.¢ credo, che sia il medesimo dire a un! huomo Lacchillone,
scheCaftrone, e che venga da Bacchio, che in alcuni juoghi di Toscana vuol dire
we — agnello,e cos: Bacchi/one voglia dire agnelio grade,cioè Caffrone. O pure viene dal
© | Lat. bacuius,quati Perticone, Scuriscione, O vero & deo quali Baleceone; che si
»¢€ non fa niente dibuono, ne di ferio.

WON andra al Prete per la penitenza. Quelto modo di dire usiamo per fare in-
'tendere, che ci vogliamo vendicare del oprufo, 0 torto fattoci, o che yogliamo
galligare uno di qualche mancamento commeffo; quafi diciamo: lo medesimo
i dard la pena di questo suo fallo, (enza che egit vada per efla al Confefore sed

il Poeta l' e(prime dicendo: Perché vnol, ch' ei la facia col baffane.,
| AIANNO ficenza-di porter tale arme, Cioè hanno permiflione di portare il ha-
it scherza, peso ivciechi portano il bafione per necefita, per farsi lan

  

  
   
 
   
  
    

  
   

 

 
 

    

 
 
   
     

QW VINA fanca ragione, Gli daranno le'baftonate,.come vanno date, ¢ quella

pi |  WoCe Sama, se ben pare riempitura per emfali, nondimeno detta in questi termi-
sf ablignifica perfezione, quafi dica divera, ¢ di tutta ragione, ¢ d' intera giufti-
a Zia, che la voce Sanétus fiacopata da Suncitus vuol dire Nabilito, determinato.,

» Nov. 10. £ battnrala adungque d' una fanta regione, cioè.con una folenne ma-
niera; dateglicie delie'buone. Vedi l'Orava 25. seguente..

GLI caccino le''mofebe da defo, Lo battonino. Vedisopra in questo C. st, 11,

SENZA tramexzo, ¢ senza respiro, Senz' intermiffione di tempo, ¢ senza pi-
igliare riposo.

NE dettero un-carpiccio di quei buon, Ne detterouna buona,'¢ gran quantita.
Carpiccio viene dal verbo carpire,-¢ pero vuol dire. manata., 0 manciata, ¢ cence
Aeruiamo per intender quantita., ma per Jo pili di bufie, comel'intese ilFiren-
-2uola neil' Afin d' oro + £ poscia, che per nua volta gle x' hebbe dati un-carpiccia de

i

TALLONI + Quella parte del piede, che ¢ tra la noce., ¢ il calcagno,:ma qui
'piglia la parte per cueto il piede. Vien dai Latino Tans. C. 8, st..69.
 PEDINO, Deito ironico., ed.intende gran picde, pedone,

SPER

 

 
 

 

    
 

goz MALMANTILE

SVENTRA. Rompe, spezza, 0 sfonda il ventre,
attivo, che fventrare neutro ha il figaitca
PAOLINO Cieco. Questo fu un Cieco compo!
zonette, le quali si fentono ancora cantar per Firenze da al
azzi, ¢ per questo il nostro Poeta dice: Fs pil canzoni, ch
oeti celeberrimi del nostro secolo. Tali sue canzoni anda'
le piazze, dove per adunare il popolo faceva fare diversi
cani, ed egli medesimo, benché affatto cieco, ¢ decrepito, 2
trampoli di legno a i piedi, Questitrampoli erano duc pertiche y in
ciascuna,delle quali era fitto un pivolo, ¢ sopr'a questi dae pivoli falis
sopr' ad essi i piedi, ¢ foftenendo la persona col rimanente di de
con adattarfele forto le braccia,camminava con granditima franchezza
poli da' Latini si domandano Graiie, 'ccondo Nonio Marcelle; ¢ quei,
minano su' trampoli, Gratlatores. Feito dice; Grattarores i
ni, qui, ut in faltatione tmitarentur agipanas, adiettis perticis furculas h
que in bis superftances aa similitudinem crurum eins generis gradiebantir
prer difscuteacem confiffendi, Plauto Vinceretis curfuceruas,© gallatorem.
D1 cento scampolt, Tutto rappezzato; che scampole Jiciamo quel pezzo d
no, 0 drappo, ec, che al mercante avanza d'uua tela quafi pezzo,così
pato, cioè avanzato a far' un' abito 1nccro; ¢ qui intende toppes 0 pezei
anno. ere a
. (MB ACVCC ARE. S! intendé coprire il capo,¢ ilwifo.. Vedi
si. 73. Varchi Stor. Fior, ub. 1.4 Subso fu prefo,¢ smbacnceato col eapp
dotto alle carceri,
Sl feandolexza, S' adira. Vedi sopra C.
di scandolezzare ¢ quel, che dicemmo sopra C.
BREZZA, Vento freddo; Vedi sopra C. 7. st. 18. ue
PAbP AICO. E' un pezzo di drappo incre(pato da una parte, e ridotto quai Ht
in forma di facco, quale portano in capo le donne per difenderfi freddo, ed 'afl
oggi lo chiamano anche cufia, Mattio Franzefi in lode delle Malehere dice ¢ » all

  

  
   
 
 
 
     
    

 
 

 
 

  
 
    
  
     
  
    
   
  
  
     
   
   

£Lvvi un fegreto, che a noi dir si puore, vet ~ Yay
Che la mascheraé me' a! un pappafico, si
E pero si vente in van. cufola, ¢ [quote ty
Ed il medesimo in lode delia Potta uso il verbo impappaficarft di aay
Chi ale tempse si fascia gli vechiati 'ake
Chi sopr' a i berrettin impappafica, ine
PORCO, Aggiunto a huomo vuol dire Schifo. ps a
0740". Intend, Che schitezza ¢ questa? Vedi sopraC, 8.67 yy
ALLEZZA, Vedi sopra C.-3. st, 64. & wota, che il verbo allezeare tantoat =”
tivo, quanto neutro ha Jo stetio significato,; 3 sur (oy
SA di refe azzurro, Per tigncre in azzurro adoprano i Tintori ere
fetore orrendo, o sia galla, 0 sia guado', 0 uno, 1' altro infiemes 1M

rimane per qualche rempo in fu la roba tinta, ¢ particolarmence in ful 1in0
pero dice quel cenciaccio fa ai refe azzurro, ed intende. Ha gran fetores'
verbo appeftare ha lo fictio significato, ¢ natura, che ha il verbo 4
di al detto C, 3, st. 54.

bee

    
   
    
 
    
  

STANZA XXIV.
levare intanto hawea Perlone
| La srane dal Gigante roninata;
Abe ancor quini ciondolone,

he la lumiera gid tenea legata,
“Ed 4 foggia d! eAriere, 0 eMontone

7 nla addietro, e dannole l' andata

- Verso quel torvion, che si distefe,

   
     

> STANZA XxV.
Hor' quando ( perch' egls sbalordito,
~ Etutto intenebrato in terra giace )
 LCieehi pik che mai fanno pulito,
 Edegli se le piglia in fanta pace,
OB fra le maxe innolto.a quel partite
Vn facco diventato par di brace,
© Eben quel panno al vifo gli è dovuto,
—— Dovendofi si-cappuecto aun batturo,

 
  

   
       
  
   
 

 

lo steffo significato.

= =.
—

orribili Giganti.

 
 
 
 
 
 
   
 

| Col si pile voite in bocca del Franzefe, Perché quivinon è troppo-buon' aria.

 

VNDECIMO CANTARE 503

TI vuo' dar l'incenso con le peta. In vece di farti honore, ed incensarti, voglio
sprezzarti, offerendoti cose puzzolenti, come suol'esser il peto, del quale Vedi sopra C.\ 6.\ st.\ 100, Orazio. Vin tu Curtis Iudaeis oppedere?

STANZA XXVI.

Mentre gli rompon Poa, € poi gli fanne

Così t incannucciata co' randelli
E talor, non wedendo ove si danno,
Si tamburan fra lor come vitellt',

 

Gli altri soldati a gambe se la danno,

Ed ognun dice: alla larga seabells;
Euege la parte amica,e la contrariay

 

STANZA XXVIII,

Ma reftin pure a rinfrescarle gli orbi,

Con quell? snfalatina di mazzocchi,
Ed et riposi all? ombra di quei forbi,
Che gli grattan la rognaco' lor nocchi
Mentre quivi per far dispetto aicorbi,
Sotto quel cencio tien-coperti gli ovcht 5
Che sugnun parte,ed io mi partoacoray
'Pen tornare a Baldone,¢ Celidora:

   

~ Con inucazione, ¢ macchina di Perlone, il Gigante ¢ atterrato, ed i Ciechi
'gli vanno tutti addoflo col baftone, ed in questo grado Jo Jascia il Poeta, ¢ torna
'a dilcorrer di Baldone, ¢ di Celidora.

CIONDOLONE. Vna cola, che fla pendente da alto a baffo [enz' esser ferma
'in verun' altro luogo, che dove è appiccata, come farebbe il battaglio.nella cam-
. » fidice far ciondolone, 0 ciondojoni dal verbo ciondolare, come dal verbo pen-
Gee si dice pendotoni, 0 penzoloni; da dondolare., dondoloni, che tutti hanno quali

ARIET E, 0 montone, Macchine,'0 strumenti bellici antichi, de' quali si servi-
| -vanoiper rovinare le muraglie; Sono notidimi., parlandone tutti gli Storici La-
“tin; ma particolarmente Giulio Cefare ne' suoi comentarj.
quel sorrione. Così & chiamato dal nostro Poeta il Gigante, perché
avanza sopra gli altri huomini., come avanzano i torrioni sopra lemuraglic; ed
anche perché servendofi dell' Ariete, 0 Montone, lo deve adoperare, non in un'
huomo, ma inuna torre, come è solito adoprarsi simili arnefi. Da questa gi-
'gantesca flatura, per la quale.essi sono affomugliati alle torri; fece Dante il ver-
'ho Torreggiare afiai galantemente. Inf. 31. Vorreggiavan dé mezzra ta persona Gli

* COL si del Franzefe in becca.'Gridando®: bud, but, che voce dimoftrativa di
p| dolore, ed in lingua-Franzefe vuol dire si. 4

¥ - SBALORDITO. Siordito, fuori del sentimento'per le percoffe ricevute..
¥) © INTENEZRATO, Si pwd dir finonimo di sbalordito: ¢ qui vale per intormen-
i * 'tito daile percotie. Vn fatio, muraglia, 0 altro simile materiale folido, ¢ dura,

 

 

“Ai dice intenebrato, quando, per le peccolic., che se gli danno per romperlo., ¢ 4i-

Bio,

 
    
     
   
      
      
    
    
   

504 MALMANTILBE. | ¥

dotto in termine, che dal fuono si conosce, che si comincia.a
F ANNO puiite, Vuol dire Ripulire 5 ma detto in questi te
da vero, o perfettamente; E' Jo stesso, che Fardi buono detto fo
SE le pigia in fdnca pace. Se le piglia con tutta, ed intera quiete. Ci
baftonare, e non si rivolta, ne's'adira. E la voce Santa ha la forza,
detto sopra in questo C, st. 20. ' olay
KINVOLTO fra le mazze. Coloro, che portano la brace a vende
ze, la mettono ne i facchi; ¢ per ammagiiarii, ¢ legargli (opra tie,
tatamente gli rinuojtano in alcunc imazze; ed il Poeta schereando dice
gante ¢ simile a uno di questi facchi picni di brace, perché egli € rinu
mazze,¢ intende di quelle mazze, con le quali i ciechi Jo baftonano.
BATTVTO. Chiamiamo Barrasi coloro delice Contraternite fecolari
proceflionalmente vanuo con velti line in dowlo, le quali chiamiamo faccht
figueino vefti di penitenza ) cappe, 0 velti da bactu,, cioè, che f bane,
si disciplina, ed il capo, ¢ faccia coperta con ua cappuccio appiccato a dettas
vefle. Ed il Poeta scherzando con l'adicttivo barrute, cio baftonate, ¢ col sa
flantivo barruto, cioé humo di Confraternita, dice, che ai Biancone flaya
il Cappuccio, perché cra barrato; ¢ per cappuccio piglia quel ferraiuolo, che
lino Cieco havea meflo in capo al Gigante. '
INC ANNVCCIAT A co' randelli. A coloro, che G sompono braccia,gambe,
© cosce, ec, Nel raflettare tal rottura, «fhuche ' off Rando fermo al luogo,ac-

  

  
 

 
 
 

comodato si rappicchi, fanno una fa(ciatura con pezzi d' afficeile, o ttecche, la is
gual fa(ciatura chiamano / incannucerata, ¢ pero dice, che, hayendo rouse | offa Us

al Gigante, gli fanno hora l'incannucciata co' randelli, cioè con quei afloat, Dk
0' quali lo perquotono. 37h Une)

s1 tamburano come vitelli. Si baftonano ben bene. Quando i Macellari hanne Ni
ammazzato un Vitello, o Bue, ec. lo gontiano, ed acciocché il vento pall Da
da per tutto faccia spiccare la pelle daija carne, baftonauo la beftia con alcune>

f
mazze, ¢ questo si dice tamburare » 0 tambu/sare, che vedemmo sopra C. 2. AL34. ie
ed a questo ramburare aflomiiglia le baftonace, che si danno fra loro i Ciechi 5 whe

wuol dire molte, fode, e spete « Sidice samburare, perché date in quelle pelli di si
Bue, ec. gonfie, fanno il fuono simile a quello del tamburo strumento 3 i

E per altro ramburare uno vuol dire quereiario; ¢ questo perché anti

Firenze 4 tenevano in alcuni Juoghi pubbiici de' Magiltrati certe;

hi da chiunque si voleva, crano meile le denunzie legrete, € guefte calle C
vano tambari, ¢ da essi tamburare, era il medesimo, che acculare, 0 quetelare.
Vedi gli Staunti di Firenze al libro intitolato. Ordinamenta snspicia contra Magnares
(citau aicune -voite da Gio, Villani ) al capitolo, ove Gi tratta del mettere nel tam~
buro. ieee
ALLA larga /gabelli'. Allontaaiamoci. Quando dopo la cena si fa balla, 0al-
tro paflatempo simile nella medesima stanza, nelia quale s' ¢ cenato sche 1 com
menfali si rizzano, ¢ per dar Juogo si fanno levar via le tavole, le seggiole, ¢ Blt
sgabelli; ed ogn' altro, che potetic dare impedimento, si suol dire: alla

belli, ¢ s' intende; si levi di mezzo ogui impedimento; il che ¢ in ¢
che significa; facciafi ala, o si taccia largo 5 ma per lo pil s' 1

 
 
 
 

HOEZBE SE 2E x PRZ

  
 
 

VNDECIMO CANTARE. 505

er ae: ' [
troppo buon aria, Li gon' y'é buono tare; Intendi: v'é pericolo di
- MAZZOCC H!, Così chiamiamo i Talli del radicchio, ne i quali nasce il se-
de iquali si fanno infalate, che sono rinfrescative, ed il Poeta, (cherzan-

'con I equivoco dimazzocchio., che vuol dire: baftone, dice che con questi

1 i taano al Gigante l'infalata per rinftescarlo, ed intende; /e ha/tonare,
SURAT. 1 baftoni de'Ciechi per to più sono di forbo,o a' altro legnaine simile
chiuto, fodo.,.¢ grave,¢ dicendo 1) Poeta; Si riposf alPombra di guei sorbi, che
i grartan ia rogna co' lor nocehi, inteade + si riposi forto quelle baftonate de i

 

      

Ai “@ t
© PER far disperco a i corbi tiem coperti gli ccchi, Per fare Mizza a i corui per la.
» che hanno di non poter beccare, ¢ cavare glivocchi al Gigante, poiché gli
, © difefi col mantello di Paolino cicco,
PANZAXXVIIL 5 STANZA XXIX,
Che la-nel mezzo a's suoi nimics comba Su via figtixoli; orto buon piceini,

     
  
  

“Di moda, ch? essi foeman per bollire, Faccian di quepti furbiyun tracto,ciccioli,
Che dove i colpi ella indirizza,e pidha, Nim remete di questi spadaccins,
Te sli manda in un subito a dormire, C' alcimento non vaglion poi tre pictiali;

    
 

“Che ne meno col fhan della (ua tromba

¢ Es in vista.vi paton Paiadini
NCamprian eli fardbbe rifentire

Han facce di Lionije cuor di fericciell;

 
   
  
 
 
   
   
  

|B quamo brava, similmente accorta', Efel-gridare,e ilbravar lor v' afforda,
ait Acombatrere i suci così conforta, Ut can chabbaiayraro avyien che morda,
ya ov Deferive laibravaca, ¢ prudenza di Celidora, ¢ riferisce #* orazidne da essa fat-

z Pe inanimire i foidati, la quale € veramente appropriata al personaggio, che
ae::

~ ZOMBA, Perquote ». Vedi sopra C, 6. st. 104.
| SCEALAN per bollire, Vuol dire sminuiscono, e quell' aggiunta per bollire, si
'poneiper un coftume introdotto da un quoco goffo, ¢ ghiotto, il quale havendo
mMeflo'a quocere Ieile alcune merle, se ne mangid pil della meta, ¢ portate il re-
A 'in' #gli domando il padrone, che cosa havea fatto dell' altre merle? ed
“il'quoco gli rispose; Sig, sono scemate per bollive, E da questa goffa aftuzia quando
diciamhoe Laral cofaé (cemara per bodire, intendiamo, che una tal cosa ¢ (cemata
aflai, senza poterfene ritrovare il conto, o faperfi la causa del mancamento.
~~ PIOMBA.. Precipita'; lascia calare, 0 calcare il colpo.
"LA tromba di Campriano, Questo Campriano fu ua contadino aftuto, come sé
'accennatosfopra C. 4. st. 47.,¢ Come si vede dalla sua fayolofa foria ttampata,
€Ol titolo Storia di C ampriano, 11 quale per'far denati trovd diverse inucnatoni di
gabbare le persone femplici; ¢ fra l'altre quelia d* una pentola, che bolliva fen-
(2a fuloco yperché da efio levata, mentre gagliardamente bolliva, € portat®in,
~mezzo a) una stanza, la fece vedere al corrivo, a cui voleva venderla; coflui ve-
dmala veramente bollire, fenz' haver fuoco avanti, subito se ne inuaghi, ed ac.
» Sordofi di compraria per il prezzo, che conyewncro. Giunto poi guetlo tale a,
casa con la pentola, e volendo senza fuoco farla bollire, ¢ non gli riuicendo, si
~ quereld con Campriano, dicendogli, che I ne ingannato; Campriano chiamé
ss la

   
   
   
  

SU OS eae st ey

 
 
   
   

 

 

    
   
  
    

 
 

    

506 MALMANTILE

la moglie, ¢ la sgridd, dicendo, che non potev efler 5

cambtata, La donna fingendo un gran timore, con gran la;

per haverla inavvertentemente rotta, glien' havyeva data un' a!

paura, che havea del marito. Di che Campriano moftrand

to, cavo fuori un colrello, ¢ con esso feri la moglie nel petto

ascofa sotto i panni una gran vescica piena di fangue, il quale fg

che uscitie dalla terita factale da Campriano; per la quale fingendo |

fer morta, calcd in terva. LU gonzo si doleva, che Campriano per ¢ C

gicra havetle commedo un delitto così grave; Ma Campriano con facia.

£'i die + Sc ben 1a donna € morta, 10 1apro rifulcitarla, quando vorro

basta, ch' io fuoni quetta trombetta; ¢ stimolato dal fempiice a far

piacque » ¢ fonata ja tromba, ja donna ff rizzo, moftrando di rifalc

il (emplice con grand' instanza chiefe la tromba a Campriano, il quale d

te preghicre a gran prezzo gliela vendé: Coftui andato a cala prefe o

gridar con la moglic, ed in fine le diede una pugnalata, con la quale

€ poi si mefle a (onar la tromba, ava quella infelice elendo veramente morta,n0a

rifalcitd altrimenti, B per questa caula, ¢ per altre sue fei. aggini fu Cam:
riano condaanato alla morte, che dicemmo sopra C, 4. st. 27, E di questa trom-

Es parla i) Poeta nelprefente luogo. sot then is ERE
SOTTO buon piccini. Esortazione, che si fa a' cani, quando s' incitano,o am- —

mettono contro qualche fiera, come vedemmo sopra C. 2. £.87,; ed il

si softiene fempre in fu le burle, fa che quetta Capitanefla esorti, edit

suoi soldati con questi termini da cani.: - 1

cicciold. Frammenti di graffo di porco, che avanzano nel tegame,o altro

va(o, quando i fa lo strutto, 0 lardo, da alcuni detti ancora dardings, ficche> — jy

  
   
      
   
       
  
 

     
   
  
  
   

  

 
    
  
  
   

 
 

vuol dire facciamo di coftoro minutissimi pezzi. Ciccio/o diminutivo, che vieoe> (
da Ciccia; la quale nel linguaggio delle Balie 5 ¢ de” fancimili vale'apprefodinot
Carne; siccome appreffo i fanciuili Greci Tria. ei Whi
SPADACCINI, Così si dicono per derilione coloro, che portano laspadas =p,
folo per pompa. juin seemnigadgtit Une
PALADINI, Ciok Conti Palatini, Quegli huomini bravi, evalorolidifran- —j,
cia cantati dal Boiardo, dail' Ariofto, ¢ da altri; ¢ da questi dicen ty

Mena (e mani come un Paladino, intendiamo buome valorofo; poiche t O tay,

   

do. Cosisapprefio gli Antichi,Ercole, e Achille si veniva a chia a
rofo,¢ dicevano: editer Hercules,¢ di Lucio Sicinio Denotato agg
mano braviflimo, riferisce Gellio lib. 2. cap. 11.5 che per la gt: eras Oy
appellato Achilles Romanus, Di guefti Conti Paladini, '0 del Palagzo intele il ei

 
 
 

Petrarca nel Trionfo della Fama Cap. 2, ro) Ne
Cingean coftus + uci dodici robufti, + ' d } yy

FACCE di Lioni, ecuor di scriccioli, Moftrano d' efler bravi', ed animofit te

codardi. Lo scricciolo eflendo il più piccolo uccello, che ti trovi, ha per conie- ».

guenza il cuore piccolissimo, ed huomo di piccol cuore s'i huomo timido,

e codardo. Vedi sopra C. 10. st. 30, Latino pari, © angu(ti anim Mi-

eropsychvs«

4

 
 
 
  
   
  

 
 
 
 

 

I ee

 

 

VNDECIMO CANTARE.

597

   

eet. di rado morde:,. Chi fa molte parole, fuol far pochi faci. SE

ape

Suol far poche parole.
STANZA XXX.

bb wel ch? Ella da ritto,e da rove/cio,
— Condicende, va fenando a doppio,

Da ful vifaal Cornacchiann marove/cio

Cun mighio si senti lonan sale 3
et =
anc' egli eA cantocneall pio,
| Mail fapor non gusto già de' buon oo
Come chi prefe il suo de' cartoccini,
» STANZA XXXI,
Sperance per di id gran colpi tira
Con quell' “infornapan della sia pala,

We barte in terra,fempre ch' ¢ la gira,

5 shafiti per la fala,
Tal che ciascuno indietro si ritira,
»O per franco schifandolo fa aia,

Bhi l' asperta, come bavete inteso,

' ek elon Ai) ie i pf

   

Perch Alsicardo, c' al pafsol? attende y
4 goxz0 gli trafora col pugnale
Ete lo manda a far le sue faccende;

ANZA
r ome il fuggir questa volta non gli vale,

proverbio con dire, Caneche murde y non abbaia s' esprimera la

Curzio:, Aleffima queque siumina minimo labuntur sono; ed. anche
Polidoro. Vergilio: Cave ribs
lontano il detto di Catone

se flefic fentenze,habbiamo in uso. anche\nel pariar nostro dicendofi:

a @ acque chete, Guardati dail? acque chere; Chi far di farsi vuole;

acane mnto., & ab aqua filenics A
1 Demiffos.animos, tacitos vitare me-

    
 

STANZA XXXIL

eAmoftante, che vede tal fiagello

D? se! arme non usata più in battaglia,
erica la spadaye quando vede il bello,
Tira unfenditeein mezoglicla taglia;
Riman brusto Sperante, e per rowcllo
Li refto, che gli auanza all aria scaglia;
Vola il trovone, eil Dianol fack' eicaschi
Sula bottiglierra tra vetri, ¢ fiaschi,
STANZA X&XLIL

Dalle diacciate bombole,e guaktade

i vino sprigionato bianco ye rosso
Fugge per b afse, ¢ dann felso cade
Git dow' è Piaccianteo,e dagli addofso:
Ei che nel capo ha fempre stoccht,e spade,
04 quel fresco di (ubita riscofso,
Pensando sia qualche [pada, ocoltello,
Si lancia fuora,evia farpa fraselle.,
XXXIV.

Così dal gozz0 venne ogni (uo male,.
Per tui fadi, per Ini la vita spende;

E vanne al Diavolche di nuouo piccalo,
A ustolare a menfa appie di Tantaio,

-Celidora esortando i suoi a combattere non lascia di menare le mani; Si nac-

tano diversi avvenimenti, ¢ la morte del Cornacchia, e di Piaccianteo.
SVONA a doppio, Intendi perquote inceflantemente. Suonare a doppio inten-
do tucte le campane, o la maggior parte dicfle, che sono in un
campanile » fyonano insieme.

Vedi sopra C, 6, st. 107. Sonare per percuotere,,

il Boccaccio Novella 67. E alzato il baflone i comincié a fonare. Latino

are.

 MANROVESCIO, E} quel colpo, che si da col braccio all' indietro,cio¢ con la
|p conuefia della mang, ¢ da quella parte con baflone, 0 altro, che s' habbia

in mano,
ako feepi se h seid Meneoee un miglio, Il romore si senti molto da lontano, Ioke:

roposito.

een rere sopra C, 3. st. 21.
SICLIANDO un fempiterno aloppio « qu 20 alloppiarsi, 0 pigliar L oppia;

© cor-

 
 

  
 
   

. | ee
so8 MALMANTILE 10%
© corrottamente'? alloppie vuol dire addormentarsi da Opis
Sicch€ qui intende, che prefe un fonnoeterno, cioè mori.)
Oui dura quies oculos, & ferrens unger Somnus; in arernam cli
Dice; che per —— ¥ oppi rein perché It haveva dato.
tempo, per moftrare, che quis peccat,per hac torquetar,,
di Pincha y che per caula dab guacnisoeetipibans F c
zo muore, " sees

INFORNAP ANE, Cioè la pala da infornare il pane,
per arme, oh obi
SBASIT!, Morti, VedifopraC.2. f. 79.0 a a
FA ala, Fa largo; fa piazza'. 'Latino Viam prebere 3 win decedere y fun
HA finito il pefo. sa fivito di fare quel, che gli era flare ordinaro; ha
compito; € s' intende ha fino ta vita: Metaforico di quetta porgione di
che si da alli bactilani dali loro Capodieci di tance libbre@i lana, che
vorare, la qual porzione chiamano wn pefo, ¢ dicono bauer finito il pefo
peafum, quando hanno finito di lavorar quel tanto', che era stato loro daro.
QUANDO vedde il bello, Quando vedde il deftro; il tempo a proposito.. >
REST A brute. Kiman bettato, eflendogli avvenuto quello y.che egh non s'al-
pettava 5 nel qual cafo il vifo refta macchiato di triftezza'y € i.
confufione.. We
SOMBOLA. Vedi sopra C. 8. st. 44.,.
FESSO. Fetura apertura di legname, 0 d' alera'materia, ©
vafi di terra cotta, Latino Rima, ' ' 3
WEL capo frucchi, e spade. Dubita, che tutto quello, che egli fente, fieno ar-
mi per l'immaginazione depravata della paura; per la quale #%¢ rifeofo
tremore, che viene per qualche accidente inaspettato; 'che. ci cugioni

   

  
 
 
  
 
  
   
  
  
 
    
   
 
 
 
  
  
 
 
  
    
  
 
 
     
   
 
   
   

per lo spavento, ches' abbia di qualche cosa improwvifa. Vedi fo he
C. st.2. se RitEe '
SARPA. Se neva. E verbo marinarclco. Latino foluir, anchoram vellit. Bog.
l' aggiuata della voce fratello è posta per emfafi, e quali per un giuro "g hl
LO manda a far le fne faccende. Lo spediice. Quis' intende 'ammazzay — te
PIANT ALO a ustolare. Latino ardere, inbiare. Lo mettevaliato a Tantalo 2 i
desiderar ancor' egli il cibo. Ed usuiare è latino; 'quafi dica + re dal ri
desiderio d*haver quella tal cosa, che egli vede. 'Ovidio negli Ai ¢
indomitis ignem exercentibus curis Fertilis, accenfis menfibus arder %
proposito ci feraiamo anche del verbo spirare. Vedi sopra C. 1, A. 31, diciamo ch
anche Vrolare; particolarmente de'cani, che fanno col mofo atte vie
vande, € per così dire le mangiano coal occhi, € col desiderio. ee, &
TANT ALO. E' nota la favola di Tantaio hglivolo di Gioves-e di Plotenin'2, 7
il quale per far prova del valore degli Dei 'gli convitd, € diede loroim tavola cot. i
to, ¢ spezzaco un suo figlinolo detto Pelope; Ma gli Deis' aftenaero op
cibo, eccecto Cerere, che mangié le (chiene, le quali le furono'poi Fit 1
Dei, che lo fecero rifalcitare, ¢ confinarono all' Inferna T: r be
cendolo patire di concinova fame, ¢ fete, per thaggior suo te
'metcere sopra il flame Ereditaao, che moltra acque doiciifims,a!;.
 
 

VNDECIMO CANTARE:; 509
1 felabbra, ma non tanto, che ne posla bere, ¢ sopra alla tefla ha un'
albero-carico di frutte bellissime le quali s' allontanano quand' egli s* allunga per
'pigiiarle' 41 nostro Poeta, che-ha de(critto Piaccianteo per un' huomo golofo

'y che morendo,egli fara confinato all Inferno, € per questo suo peccato di
ola fara mefio allato a Tantalo a #/flare anch' egli, come fa Tantalo,vedendo

ha da faziarsi, e che non posla haverla. Bologninus. i
Tantalus bic etram fitiens potare vetatur,
Ha 'a quod Pelopis Dijs epulanda dedit,
quali Omero nell' 11. dell' Viiflea descrive la pena di Tantalo, tradot-
Latini fuonane così:
Stat mifer in medio; medijs exardet in undis

Tantalus,& fruftra circumfert pallidus ora,
Proximus illudit mento circumfluus humor
Et prope Yorantes contingunt corpora grtra y
Et crines,@ barba madent a/pergine crebra;
Dumque undam captar fitienti Tantalus ore

  

 
 

  
 

   
    
   

 
     
 
 
        
   

STANZA XXXV.
Era ua camer ata un tal Guglieimo,
Cha la labarda,e ifuci calzonia strisce
Virbigonicinolohaincapo in vece d'elmo,
E'tutto il reffo armaro a flocchefisce.
» “Alemnnno è coftui Perneiter [celmo,
» Econ quel dir che brava,ed atterrisce,
Sbruffi ferenti (earicando; e rutti
Ln un tempo spaventa,e ammorba tutti,

STANZA XXKVL

Humoremque cavis, fentat tomprendere palmis.
Hen /upito, ben longe fugitura recurfitat unda,

STANZA XXXVIL
Perché voltando il ferro della cappa

Verso Alticardo a vendicar [ amica,
Quei ghetascafaye glittra forto,e'l chinppa
Con la spada meixo del bekico,
Ond'sl vim pretto in maggior copiascappa,
Che no mesce in tre dil Inferno,e tl Fico,
Ala non va mal, perch'e: caduto allotta,
Hentre boecheegia tutto lo rimborta,

STANZA XXXVIIL

 

     
    
 
 

 
   

 
       
 

 

 

1 Coftud a quel ghiortone a tutte Uhore Gira Sperante pegeio a' un mulino

Fu buon compagno a ber la maluagia, Perch'arme alcuna in manpiiend.gl refta,
rer non cadere adeffo in qualch'errore, Par trova un tratte un pie d'un tavolino,
yi E far' un torto alla cavaleria, £ Ciro incontra,e gls vuol far la fefta,
a] Pur'anco gli vuoi far,mentre chrei muore Aa quei prefo di quivi un sharagline,
, Con farsi dar due crocchie, compagnia, Voa casa con esso a ini fain refia,
E non duri molta farica in questo, Perché paffando ? offo oltr* alta pelle,
: ~ Chet trove chi [pedilo'e bene,e prefto. Nel capo gli raddoppra le cirelle.,
. Seguitando il Poeta a narrare gli accident occorfi in questa zaffa, dice, che

 

   

 

Alticardo ammazz0 Guglielmo Lanzo, che volle seguicare in morte Piaccianteo,
come l'haveva seguitato fempre all oiterie; B Ciro Serbatondi ammazza Spe-
rante, con battergii un tavoliere da giocare a-sbaraglino in fu la testa,
GVGLIELMO Tedesco. Fu quetto Ledelco Soldato della Guardia pedeftre del
Serenitfimo Gran Duca, la quale ¢ composta d' Alabardicri veltiti a livrea con,
brache larghe fatte a strisce paonazze, ¢ role,¢ si chiamano Linzi. Vedi fo-
pra C. 4. stan. 4. E perché questi non portano ferraiuolo, o cappa, diciamo per
ascherzo ferraiuolo, o cappa quella labarda, che portano in spalla., come vedre-
me

 
      
 
  

 

  
 

 
 

 

  
   
 

gre MALMA NETLLTW

mo appreffo stan. 27. ¢ s' ¢ accennato sopra €. 9. fan. 48..€  r
date, 0 percofie colla jabarda. Coftui era molto amico di-

aiuto a mandar male la roba, ¢ però il Poeta dice, ch' ei lo vuol

in morte 2) La OORT

BIGONCIVOLO.. Diminutivo di bigoncia, detto sopra C, 10, stan. 7
coftui con un bigonciuolo, arnefe, che per lo pi s' adopra al vino,
che in tutte le (ue operaziont egli haveva l'animo al viao, ¢ con
( che vuol dir pesce baftone, vivanda aflai usata dai Tedescht ) per m
alla voglia del vino haveva unita ancora quelia del mangiare. Si. ars
ancora, che il Poeta voglia moftrare, che coftui era fudicio,.¢ c
in effetto egli era, ¢ come per lo pili (ono questi Lanzis a caula forse di
pelce, che veramente ha fempre malo odore.

BEKNEIDEK Scelm. Voci Todesche le quali in nostra ipa fuonane
cone, scellerato,

ATT ERRISCE, Spaventa. La pronunzia Todesca ha un certo accento,
fa credere, che colui, che parla bravi fempre, € per quelta rozzezza di 'al
gua dicono che ella sia propria, ed il cafo a comandare eferciti, come la Fran-
ccle a aoe con dame, la Spagnuola al comando politico, ie cuaanaraerey
guefte cose Pr

SBRVEFL, BE? quel mandar fuori per bocca il vento jonato in carded
prabbondanza di ae E ratti si ie dire lo stesso, aso che per rasse inten
diamo il puro vento,¢ sbruffo si dice quando il vento vicn fuor del corpo «
no firepito, che non viene il rutto, ma accompagnato con un poco sae;
¢fiendo lo cheafare un mandar fuori di bocca con violenza vino, 0 altro i
AMMORBA, Fa putire. Vedi sopra in questo Cant. stan, 23. quie pe

significato attivo, cio¢ appefta; mette la pefte in tutti. '
GHIOTTONE,, Gran go.olo; Gran ghiorto. lntende di Pisesbame a

MALV AGIA, Specie di vino afiai noto; ed a noi viene di Vi qui ie
pigliando la specie per il genere, intende che gli fu fempre compo be a tain
forta di vino.

  

 
       
          
 

 

  
 
 
  

CROCCHIE; Percofle, Da ereechiare che in significato attiyo vuol dire Ps
motere

2 SPEDILLO bene ye prefo. In poco tempo gli diede buona sp t 7"
ammazzo prefto, ed affatto. Questo detto bene, ¢ presso era il mol 3 7
cademia Fiorentina detta de' Rifritti, ed il Poeta se ne (erue, p pil
fu già di detta Accademia, ed immita un' altro Poeta, che nelj' umprovvila, © \s
byona morte d' uno pure di detta Accademia difie; 'aban bar
E per moftrar, come Rifritto ville y pide eh e to

Mor:, come Kifriteo E PRESTO, E BENE, Ca

EEE ¢ il Fico, Sono due Ofterie di 'Eirenze così nomina' die oo i;
Infega

Bosc HEGGIARE. Quel moto, che fanno con aprire, ¢ (errage la bosaia
mandar fuora gli ultimi spiriti coloro, che muoiono 4

LO rimborra, Rimette nella bors 9 lO¢ in corpo 5 'ribeve -
che gli cra ulcito di corpo.

 

   
 

 

 

 

 

.

VNDECIMO CANTARE) Sat

CLI vel far la feta, Cide lo vuole finire, lo vuole ammazzare '
GLI fauna casa in testa. Nel giuoco di sareg eee una casa,vaol dire rad-
iar le girelle, 0 tavole sopr' a uno de' 24. fegni, che sono nel tavoliere, cd
i. scherza con questo addoppiar Ie gireile con dire che batrendogti il ta-
 yoliere in cefta gli raddoppia le girelie, che quiui haveva, ¢ così gli fa una cafe,
- intesta, che haver girelle in testa s' intende tuomo col cerucllo che gira. Vedi

C. 9. stan. 10.
STANZA XXXX.

-) STANZA XXXIX.
Ritvaffe gid Perlone un certo Marte, Tofelloych in fere.ra ad bxom non cede
Riesce adefio qus tutto garbato,

© Cthaucua il nafo da fiurar poponi,

| E perch ei nol pago mai del ritratto y Perch'ei rifana un zoppo da un piede,
Pere fa feco adeffo agli fgrugnoni; Cregnor fu quella parte andd seiancato,

| Ediegtien' un si forte. ch' in quell' atto Mentre di taglio un sopramanglidiede

Gli si fhianto la firinga de' calzoni, dn quel, che fano havea dali' alrra late,
| Che qual tenda calando alle calcagna Che pareggiolio, ond' ei fu poi di quci
 Scopri scena di bosco, ¢ di campagna, Che dicon: qui¢ mioye qua vorres,

% STANZA XXXXIL
Grazian di fangue in terra ha fatt'un bagno Che vie da un trcbettier di Carla Atagne
Onde glié ya 4 chi va gin che nnoti; Quando le molfe dar fece ai tremors;
 Afetta un Salta,e xn Birrocolcopagns Toglie ad unl'asta,tl qual fail Paladine
1 E frroppia uneal, che fale erucce aiboti, Se ben con efsa fu [parzacammino,

“Seguita a narrare varj accidenti occorfi in quella zutla, ¢ le racconca le bravu-
re di Tofello Gianni, ¢ di Grazian Molletto.
SU ffianto la firinga de' caizoni, Si roppe la stringa, cioé quel legame, che ferra
calzoni in fulia pancia.
TENDLe4. Incende nel presente luogo quella tela, che si mette d' avanti a i
chi »sopra i quali si rapprefentano Commedie, affinché cuopra le scene per
Doprine nel dar principio alia Commedia; Lat. /iparinm, e però dice, che i snoi
calzoni. eflendogli cascati,, scoperfono scena di bosco, ec, cioé quel, che da loro
'eraycoperto. Cafo veramente seguito a Perione, che,per voler ae pagato d'ua
Fitratto., che egli havea'fano a uno, gli conuenne fare alle pugna, ed ia quel
re gli cascarono i calzoni.
SCIANC ATO. Vno, che va zoppo per haver difecto nell' anche, offo princi-
pale delle cosce. Vedi fupra C. 6, stan. 82.
. CHE dicon; quie mio, ¢ qua worrei, Così diciamo di quelli zoppi, che vanno
a gambe larghe per difecco, che habbiano nell' anche, o in ambedue le ginocchia,
€ non posano i piedi in dritto, fecondo J' uso comune, ma pare, che vogliaao
can un piede andare in un iuogo ye con' altro in un' altro, ¢ che accennino qui
# mio, €qua vorrei, Di questi tali diciamo ancora Andare a feiacquabarili, perch
fanno lo fieflo moto con ia persona, che fa uno, che (ciacqui un barile «
APPETT-A, Taglia da una parte all' altra, come si fa al pane, del quale pro-
-Priamente si dice affectare,o far fette.
VN Sata, Si chiamano Salti quei famigli, ¢ donzelli dell' Arte dell' honefta
“(che in Firenze € il Magiftrato, al quale fon fottoposte le Meretrici ) i quali fan-
'NO ogni forta d' cfecuzione tanto Civile, quanto Criminaie contro le Meretricé,
t VN

me

i ee
 

 

le figure di carta petta:, le
di boto, ¢ d' haver ricevuto ae
cono Bari. Vedi sopra C, 4. tts

 
 

 
 

fente 'uogo il nostro Poeta,

5 tz MALMANTILE | (¥
VN tal che fale erucce a boti « Intende' uno seultore dappoed 5 che.
qnaii @ mettono alle Immagini facre.
razia; ¢ queste figure:co
c. £7. Gruccia è dal Lac, barb:
€ baflone fatto a croce; onde in alcuni Juoghi della '
Far le grucce a una figura, s intende fra.i pittori.
fan, 27, Intendi dunque, che coftui era (cultore stroppiatore dit
fabbricava se non faacecci di carta pella, formati confornie di gi
no di quella bellezza, che pud yedere?chi andra nelle! Chiele
miracolofi; ¢ queste figure faceva così male, che le strop
da fapere che /eultor da bori suona fra gli scultori lo stesso, che fra i
Pittor da fgabelli, dewo sopra ©, 4. stan, 10, Questo tale ancorché fulie:
¢ nato d' intima plebe,  ttimava un Buonarruot; efi piccavaidi nobile
dice, che yen da wn tromberta di Carlo Adagno, quandevle moffe dar fac
Cioè ha origine da un trombettiere,dei
re.i bandi, che dar de mofe a' tremori, vuol dir comandarfo\

ticamente, se bene in deco scherzolo,¢ per derifione, come se ne serve nel pre.
> aa

arias
java affarto ¢ Ino

quale:Carlo Magno i serviva per manda

Ae

SPALZZACAMMENO.. Vanno per Firenze aleuni © Marchigiani o Lom-
bardi con una pertica in spalia gridando: Spazcacammina y acciocthé pia
che efi ripuliscono le cappe., 0 gole de i cammini-dalle filiggine » Vino:

tait era cului y il quale con queli' alta, clod con ta pertica tr

ladino.
STANZA XXXXIL
Tutto tinte ne va Puccio Lamoni
Stoccheggianda nel merzo della Vuffa,
£ in Pippa un tratte da del Castitioni,
Che majcherato ancor tira di buffa;
Ea ci che nel sentir quei farfallont,
Venir pitt tofta fentefi la musa,
Paffandalo pel petto banda banda
Ai far rider le piattole lo manda,
STANZA XXAXKTL
Nanniruffa ha piu la pien di ferite,
Pericolo, che fu [copa meftieri,
Fu pailaio, Senfale, etitor di lite;
Srette Bargelioy ed abbaco di xeri
Prefel appalte alfin dell! acquavire:
Ala pris fuaniro i fuvi peafieri,
Lon pite il wana fhillando, ma il cernello
Per metterui poi il moffo,el'acquerelio,

Continoya a narrar quel, che segue neheombattimento y

mazzamenti,

TVTTO tinte, Vuol dire adirato, ma il Poeta si serve'di
ché detco Puccio è di faccia bruna, come s'€ detto sopra C.

6 OP
STANZA XXXREV) ©
Con Duriano ii Purba eccoalle wank ©
Di ferro da fradceri i Safe,
Ev altro una paletta tq
E con eff a tui cerca,e sbracia
Ma percht quei le fqnete, tome canis
Gi fraricatt fuaf hs chibufo, (4

Chreghi ha a' Monnini,evane

Fatto d ognun polpette
S' 4 tanto mal non se:
Col dar ful grifo-a tui Salue Rofata y
Chef. oui 2
Vuol ch' e facia pere

Cb? essendo prefa
Lo spinge fuor

   
  

  
      
  

     
     
 
 
  
    
 
  
    
 
      
     
   
  
  
 
     
 
 
 
 
 
 
 
   

«=. FREER

=a.

 
 

wii =

a a ae

Pre Sst = = -

Sate Sb tes eee

 

VNDECIMO CANTARE 513

 TIRA4di bufa, Fa i buffone. Le buffe, come accennammo sopra C. 2. staa.
2. alla voce bu/chette, sono pezzetti di mazza rifefla, e formano quaai un dado,
se non che hanno te parti piane, ed una conuefia, ¢ si tirano come idadi, fa-
eendo Con esse quei giuochi, che si refta d' accordo con fei, orto, 0 pil di cali

 buffe; ¢ per me stimo, che s' usino, come s' usavano dagli aatichi gli aliogi: ma

: è i ¢ giuoco da fanciulli,percio habbiamo il detto sirar ds buffayche vuol
ire Far cose da fanciulli, ec. da persone di poco giudizio, che poi da questo in
una parola si dice buffone, e far il buffone; che i Latini dicendolo scarra lo delcri-
vono per uno, che rifum ab audientibus caprar, non habita ratione verecundie, aut di-
gnitatis, © così per uno, che non habbia l'intero giudizio da distinguere i tempi,
ee wetl ne le persone, come ¢ per lo più il giudizio d' un fanciullo. UI P.
', Vincenzo Maria Carmelitano Scalzo nel suo viaggio all' [adic Oricntali lib. 4,
¢.26. descrivendo un' uccello detto Buffo [ che è forse.quello che i Launi Bubo,
€ noi chiamiamo Gufo } dice così,, I nottri antichi lo chiamaron Buffo, onde:
y» forse hebbe origine il nome di buffone, poiché è incredibile, quanto quetto
a uscello sia inclinato agli scherzi, ed alle burle, con Ie quali bene (peffo atcer-
y rice di notte, ed inganna la gente.
|. BARFALLONI, Denti spropositati, ¢ sciocchi.

SENT ES! venir la muffa. Si fente venir V' ira; Entra in collera.

LO manda a far rider te piattole, Lo manda a far il buffone nell' altro mondo,
dice /e piartole, perché questi fon vermi, che stanao negli aucili, ed hanno oc-
¢afione di rallegrarsi per 11 nuovo cibo che a lor viene dall' andar egli nell' avello,

PERICOLO, che fa Scopameftieri, Si dice Scopameftieri colui, il quale seguita
poco tempo a far un' arte, ma Jasciandola stare ne vada a fare un' altra, perché
la prima non gli piaccia,come appunto fece questo Aleilandro Violant detto
Pericolo, nominato sopra C. 3. stan. 58. il quale veramente fece tutti i mefticri
enunciati nella presente Ottava 43. ed in ultimo si diede.a trovare invenzioni di
Mettere appalti; comincid dal Tabacco, e poi l'Acquavite, i quali senza suo
utile, 0 pochiffisno conchiufe per altri. Dice, che abbaco di zeri,perché veramen-
te ci fu un grandissimo abbachifta,e per questo havendo saputo trovar degli erro-
ri. contro a' miniftri grandi, fu da essi perfeguitato si, che fu mandato in gaiera;
Ma havendo le notizie date da lui fatto al fine (coprir la verita, furono i delin-
Foe gaftigati, ed egli cavato di galera. Dice abbaco; ma percht questo verbo

gnifica ancora flar dietro a fare una cosa,¢ non trovare la via a terminarla,
per non haver tanto giudizio, o scienza che a cid bafti, il Poeta piglia tal detto
in questo luogo neil' uno, ¢ nell' altro fenfo, cioé, che egli fulle veramente gran-
de abbachifta, ¢ che egli abbacaffe, cioé armeggiafle col ceruello fenz' utile es

conchinfione, ¢ però v' aggiunge di zeri, perché, sia pur grande un' abba-
¢hifta quanto si vuole, che mai non rilevera somma alcuna, se non si servira d'
altra hgura che del zero, Cos} in effetto fu coftus che con tutto il suo grand' ab-
baco non pes mai far conto, che gli tornafle bene, ¢ con tutte le sue arti, ed
invenzioni si pud dire che abbacaffe, perché in ultimo si mori quafi di fame,

PIGLIAR ? appatto. Quand' uno col pagare ai Principe una somma convenuta
Piglia ' assunto di provvedere uno Stato d' una mercanzia, ¢ fa proibire che -al-
tri la posla vendere, o fabbricare senza sua licenzia, diciamo pigiare appaito, che
Sil Las, Adonopolinm. Tre MET-
 

 
   
    

514 MALMANTILE si

MET TERVI il mofto, eI acquerello « Confumarui tanto le bu
tive fuftanze.. Oleam, & operam perdere,

FVSO da StradieriChi fiend gli Stradieri dicemmo sopra C. 3.
sto lor fufo ¢ un ferro fortile lungo, ed acuta, col quale forano i
altro a fine di vedere, se vi sia occulrata roba 5 che paghi gabella.

PALETT A da Caldani, B' una meltoletta di ferro con manico: g0 » che
serve per iftuzzicare i fuoco 'nel caldano,0 focone,il quale, che cola sia, Vi
C. 3. stanza 3.; ee

SBRACTARE.. Vuol dire iftuzzicar la brace, perché s' acceada, 0 P'accelas tie
spandere alquanto, ¢ qui diceado: gf sbracta il mufo, intende, 10 perquote con 1a

   
  
    
       

paletta nel vifo, ¢ gli¢ lo sCortica. 1. 20.) aga Deke
4 LE squote come fanno icani, Non ttima, Non cura le buffe.. Vedi sopra C, 10, Suan
anza 36. 5 'sun Obey Mec
eARe HIBESO ch' egli ha a' Monnini, Doriano fa morire il Fucba con 'uno: Sino
quei suoi Monnini detti sopra C, 1. @. 44. i quali Monnini ij Poeta insieme cons Nan
ogai aitro flimava tanco sciocchi,e odiofi, che credeva fuflono abil a far morire Celido
uno di naufea,; al fen

SQV ARCINA, Spada corta,e larga,altrimenti detta colrellao mexza/pada. Te
POLPETT A... Vivanda nota fatta di carne benissimo bactuta con coltello sed
impaftata con uova, cacio, pan grattaco, fale, spezierie,ecs > an Oh Difer
CERVELLAT A, & specie di falficcia fatta di carne, © di ceruelli di poreo La
triturati, ed imbudellati come la falficcia. E dicendo far poiperte 5 ¢ ceruellaths Tere
4' huomini intende far macello, ¢ strage d' huomini. OLS AES
CONT ADINA., Specie di danza usata nel Carnovale 5 la quale confifte tutta hag

 

   

in forze in questa manicra,, Octo-, 0 dieci huomini si fermano ritti col im Cel
fieme in giro con le braccia alla coliottola l'uno all' altro; opr' alle. di ha
quciti faigono quattro, o fei», sopra i fei altri tre,¢ soprai tre wao,¢ fatea que- atts
ita regolata mafia vanno girando a tempo di fuono,, ed in ultimo quello, che € to
cima (opra a tutti,fa ua capitombolo sopr' alle spalle di quei tre alla volta delter= e
reno, dove ¢ ripigliato da due, che sono quivi a tale effetto;:nello feflo modo ty
fanno poi i tre,¢ poi i (ei, ¢ dopo questi gli otto, 0 i dieci fanno iltcapitombolo ile
in terra; ¢ questa dicon far /a tombeiata. EB percht Mato di Coccio.in: for- te
ta di bal'o era Maeftro, € però dice, che Salvo Rofara sapendo, bea la re
Contadina, lo fa fare la tombolata gil perla scala. aan Ui
STANZA XAXXXVL STANZA XX¥EKVIL- ti
Palamidone in tanto con la mano, Quafi di viver Bariftone uso, > '
In tasca a Belmaforto andana in volta, Egeno affronta con un prmerwoloy hes
Per tirarne la borfa in suw pran piano, E perché quei |" uccedia come nn gifoy i
Per carita che non gli fuffe tla; Salea ch' ei pare'un gailestovmapanele ip
Mail buon pensier ch' egii bayrie/ce vano E raito fa cht iL manbeaeenpe y ta
Perch' egli col pugnal se gli rinolta's Manda My
E fa per carizade anch' e che muoiar, E por to pi: the
'extecio fa vita non gli tolga il boa, 'Per dario per un 1

 
 

 

EB paffagli un veftir

ee

STANZA XXXXVIIL

  

Exquei gti duol che'l rinnono quell anno,
- Bfee' si muor vnol che gli paghi il danno,

ae VNDECIMOCANTARE, 515

STANZA XXXXIx,

Romolo infilza “to mezzo al bufto > L' armi Papirio ad un Prandron guadagna,
'tes iyoenunise un canto erafugviasco, Che. fae apiacuhine lo Swillerra;
Efe ne muor con molto suo difeuito, Ma # a parole gli è Spaccomontagna,
«Perch? egli haveva a esser aun fiasco; AUP ergo poi riesce Spada fanta,
Tira inun tempo fifo aun bell imbufto, Perchheifactee it al Cel dar lecalcagna,
demmafeo, 'Won una voir dice, ma cinguanta:

Sta[uch'in terra i pari miei non danno
Ed ei risponde: S'io sto (uy mio danno,
L

 

STANZA
riga il Mula, ePoste degli allori, E nelle parti git posseriori
» Son mandati per fempre a far un fonno, Panfiloagginfta Meoyche vendeil tonno,
 Miccioge'l Baggina da Strazildo Nori Tal che s* allor putina,bor chi accofta

Sono inuiati done ando il lor Nonno, Sente che raddoppiata egli ha la potta,

\ Narra' morte d' alcuni difenfori di Mal mantile, ¢ le bravure de' Soldati di

a, Se'brami tanto d' intendere i nomi anagrammatici, quanto di fapere
chifieno gli altri. Vedi sopra al C. 1. ed ai C. 3,
STVEO. Sazio. Annoiato. 2

 PENT ERVOLO.. Piccolo file di ferro aeuto, del quale infra gli altri si servo-
no i farti per far buchi agli abiti.

DB aecelta > Lo baria; lo schernisce. Dice come un gufo, cioè come fanno gli
ucceilétth al: gato, che è uno uccello notturno, ¢ simile alla Civetta, ma aflai più
grande } chey Latini dicono babenem, donde bubbofone si dice a uno spropositato
chiacehierone; ¢ bubbole i racconti spropositati, ¢ non' veri ( forse da Bubbola uc-
cello, Lat. «pupa. ) In questo uccello detto gufo, o barbagianni, favoleggiano
git atichy Poeti, che fufle mutaco da Proferpina quell' Ascalafo, che fece la spia
a.Proferpina d' haver ella mangiato ja melagrana, il che fu causa, che ella non

¢ ulcir daii' Inferno. Ovid, 5. Met. Questo uccello € forse lo stesso, che quel

Pgeedel quale habbiamo detto sopra in questo C. stan. 42.

~ GALLETTO marzxolo, | galli, che nascono del mefe di Marzo, quando poi
fifega il grano fon pil grandi,e fs gagliardi di quelli, che nascona d' Aprile,
eper queitofaicano piii alto alle spighe del grano, onde col dire: Salea come un
galletto marxvalo, s' intende falta gagliardamente.

 LL mal tarenfo, Vuol\ dire huomicciuolo di cattivo animo, che i Latini purer
dicono boma fungini generis.

4VEFETTO. lntendiamo una specie di tavolino; ma quis' intende un colpo,
che si da'col dito di mezzo accomodato a guila di molla a! dito pollice,o ( come
diciamo ) dito geoffo, ¢ poi lasciato (appar con violenza al Juogo, dove si vuol
colpire «| Moiti pero per bufferto, o buffertune, intendono.colpo di tutta la mano;
¢ appreffo gli muoli Boferada, 0 Boferon vuol dire moftaccione, guanciata.,

Macon quetto huomicciuolo, che non era da pugna, 0 simili, si pud credere,
che intenda veramente pufferro dato con un fol dito.

BAR querciuole, Ciok con le gambe alzate all' aria, € s' intende st ammazza,

-Lnoftri ragazzi dicono far querciuolo, quando no pola le mani, ea testa in,

terra, ¢ manda le gambe all' aria; quaft moftrando qd essere una.pianta, la sc
od Tee "2 a

 
 

——

  
 
     
    
  
   
     
   
 

 
   

516 MALMANTILE | ©

 

 
   
 
 
 
  
 
 
 
      
 
   
 

 

  
  

ha,della quale sia il capo, il corpo sia il futto,e i rami le zampe. ho
seguente dice dar /e ca/cagna al Cielo sche vuol dir caduto in terra b Bul
così si moftrano le calcagna al Cielo, ¢ fi'dice anche mandare a gambe | no
FVGG/ASCO. Riurato, fuggitivo. Vao, che per paura de' birri sg
vedere, se non ne i luoghi immuni. we ky
HAVEVA a offer a un fiasco, Croe 8 haveva a trovare a bere i 5
Quando alcuni voglion bere insieme un fiasco di vino, € pagarne i
ii valore per mettere insieme la cricca, dicono Chi vaol essere 4 un fiasco? Mi,
tende chi vuol accordarsi a bere, € pagar cia(cuno la sua parte? BY termiae! Bad
fo, ed usato fra l'infima plebes ate a0
BELL imbuffe. Bella preteaza, Va di coloro, che Manno in fa la ky
quaii non hanno di buono che la prefenza, da 1 Launi soprannominati 4
per metatora, perché /folones si dicono quci bet rami, che noa ab
donde noi diciamo folly a uno che non € buoao se non a far comparla,o v
za,come si dice qui #7 bell' smbuffo, che diciamo ancora wa bel coram Vobis. A
Tulipano, diciamo a uno, che abbia buono aspetco; ¢ poche altre quali Ti,
similitudine del fore così detto, venutoci di Turchia, che va imitando 1a! hare
¢ la vaghezza delia Tulipa, 0 del turbante Turche(co,ondehailnome, =u
DOMMASCO, Deito così dalla Città di Damatco in Levante. Specie di v
drappo fottile di feta fatto a fior1, 0 ( come diciamo ) a opera. os baa
RINNOVO! quedl'anno, Se ' era fatto di nuovo quell' aano, Pare che sia foli-
to quando altri si fa un veltito nuovo per li primi giorai, che -adopra havers = nd
git qualche riguardo di più, come faceva coftui, che per esser ii (uo vettito nuo- T
vo, l'apprezzava pili della propria vita, poiché rinfaccia, e proreiladeldanno
del veftito, ¢ di quello della vita non ne dilcorre, oem oie ¢
FlaNDROWE, Huomo di Fianiira, Ma perché huomo di Fiandea diciamo j
Fiammingo, la voce Fiandrone ci fertic per esprimere Vino spaccone, éhe si vanti P
di bravo,raccoatando le prodezzc tacte da im fuori di qua, ed uno di quelli, che b
i Latin dicono milires gloriofos, ed in questo fenfo Jo piglia il Poeta nel presente i
luogo, se ben (cherza con l'equivoco; Ed egli stesso lo dichiara dicendoy Che» I
fan Taghiacantons,e lo Smillanta; all' ergo poi riesce Spada fanta, ciok fa da bravo 5 ha
ma dovendo venire a i fatti, ¢ alia conclufione, riesce una (pada, che aon fa mal ¢
veruno, ¢ pero Santa; ed in fultanza un poitrone. Dicefi nell' uso, "i
buona pada; cioè € huomo, che fa bene adoprare la spada. Nel Pianto che't Pe
Carlo Magno neila morte di Rolando da' nostri Poeti detto Orlando, appreflo vy
Tarpino Arcivescovo di Rems, ¢ compagno in guerra del medesimo Carlo: 6 die fing
ce. O brachium dextrum corporis mei, barba optima, decus Gallarums, inf hg
Carlo chiama Oriando Spada deila giuftizia alludendo alla formidabile spada da ie
Turpino detta durenda, da' duri colpt ch' egli dava con etia da' poeti Darindana, Hs
oh wrath rf 5 © fmill. dich ua nottro pi bio in. di,
che dice La fradera del' kiba, che vuol div vantatore di gran cole 50;
re; Equefto perché la stadera dell' Elba; che serve per pefare barche piene ey
ferro, acile sue tacche comincia a contar da/ mille, ¢ seguita s -a migiiaia
Tagliacantoni, cioè, che tira gill pezzi di muraglia corrifj | Pyrgo ii
wices di Riawto, Che vorrcbbe dire in noltra Lingua Atrerrasor ty

 
 

ee eS

4
Ai. si

a
VNDECIMO CANTARE. 537

 

Lo Smillanea, ciok Smillantatore si esprime dal Greco Thrafon, cioè Audace.,

BHES Ske

ee

Sesh © £F

SPTVSILRS Pee Thr ees CUR ET

 

Baldanzofo; ¢ dal Latino Adiles gloriofus. E la parolaé fatta da Adidanea, (cher~
'zofamente usato dal Boce. in vece di mille; dandogli la definenza di quaranta,
cinguanta, ¢ simili; quafi uno non sia contento di dire la femplice parola di mil-
le, ma la voglia go > ¢ far parere 1a cosa pil di quel ch' ell' ¢ in esserto.
'S' io Ho fu, mio danno, Non mi rizzo al certo. Questo termine mio danno ula-
to in questa forma, ¢ specie di giuramento, ed ha la forza del termine appon/o 4
noi, decto sopra C. 8. stan, 72. € 3° io non' ho,egli ¢ fallo,detto sopra C. 6. (tan, 86,
MiCC4O, Così era nominato un garzone della pallaa Corda, che ¢ uno di
coloro i quali stanno nel mezzo della stanza, mentre si gioca, a raccorve la pals
la, ¢ rammentare il giuoco.
BAGGILANA, o Baggina, Eva un Battilano, che in occasione di felte serviva
ai Bawtilant per tamburino.
DOVE anao il lor Monno, Ciok nell' altro Mondo. Vedi sopra C. 4, tan. 2.
 NELLE parti posseriori. Cioé nel c....0 come baflamente si dice, nel prete-
rito, dove dice che ¢ prima putiva, hora pute il doppio, che questo vuoi dire
ha raddoppiato la posta.
e4GGIVST A. B' prefo ne) fenso medesimo, che è prefo sopra C, 2. stan. 41.
CHEO che vende il Tonno, Fu un venditore di peice falato,¢ tali huomini
hanno (empre addosso cattivo odore.

STANZA LI. STANZA LIL
In abito Scarnecchia da Coviello, Gustavo Faibi con un soprammane,

Tinta de brace l una,el altra guancia, Di nerty il capo fmoccola a Santella
EB per sua [pada sfodera un fuscelio, Scaramuccia si muar fotte Erauano,
C" al pome a' una bella melarancia, C' aimazza anche Gaba da Berzighella,
Rinolto con quest' armi a Sardonello, E fuentra quel birbon dell Ortolano,
Perma, gis dice, guardati la pancia, Che fa il minchion per non pagar gabella,
Ed enrisponde: ueftoé pensier mio, Ma colto poi vi reffa ad ogni modo,
z rant un colpo, ete lo manda a Scio.  Mentr' adeffo gli va la vita iv frodo,
Descrive } abito, ed armi di Scarnecchia, che refto morto da Sardonello;

Eravano ammazza Scaramuccia, Gaban da Berzighella,¢ l'Ortolano.
COVIELLO. Cioè lacoviello maschera, che finge un bravo sciocco Napole-
tano, 'a quale s' aggrotte(ca con fargli i bafi alla Spagauola col nero 41 b ace,es~
PerO dice Tinto di brace? una, el' akira guancia,¢ con armaria d' una spada faca
d' una mazza, che ha in vece di pome una mela, o melarancia, o altra frutras
simile per rendere il personaggio piu cidicolo,e così veftiva quelto Montambanco,
facendofi.chiamare Scarnecchia... Vedi sopra C, 3. tt. 62, Così Cola, ¢ Zanni,
personaggi ridicoli di Commedia sono nomi proprj de' loro paefi, donde si fingo~
.no»s accorciati dagl'interi nomi Niccola, ¢ Giovanni; onde va in terra lorigine
di Zanni, che alcuni ingegnofamente hanno tirato dal Latino Sannio, mis.
LO manda-a Scio, Lo manda all' altra vita, ed ¢ lo stelio, ¢ si dice per ia me-
defima ragione, che mandar a Pasraffo,0 a Buda, derto sopra C, 5. st. 134
- SMOCCOLA il capo. Taglia il capo... Smoccolare si dice tagliare i} Lucignolo
di una candela, o altro lume per levar quegli escrementi, che fa la fiaccola, che
hiamali f i. » che queiti Spagi sear'

 
 

   
   
  
   
   
     
  
 

8 MALMANTILE

 

  
   
 
   
  
 

desfavilar quafi exfavillare; il Vives disse exfungare formando 1a all
Virg. 1. Georg, Scintillare oleum, & putres concrescere fungos, ol
SCARAMVCCIA, Vo' aitea maschera, come Scarnecehia - tit
Ourava 51., ma questo era Iftrione, e non Montambanco. i owe Roi
GABAN da Berrighella, Quetto pure era Iftrione, ce rapprefentava wo |
dt un Romagauolo ttoito. ' = Oe
LORTOLANO, Coftui fa un yeechio aftuto, che: per ein
dovutali per aicuni delitti commeili, s' era finto ae 1 Del
chion per non pagar gabella, Menandro, Rufticum essete fimulas, tam Par
vi refta colto, cioè viene (caperta questa sua malizia da Bravano, che Per
vita in frodo, a colui, che non volea pagar la gabella,e¢ vuol wae Sin
in vece di frede folamente l'usiamo di dire dalla fraude, che si comm el
pagare la gabella. Ta
STANZA LIL is
e4rmato a priuileo} omai Rofaccio Che piove al
Marte sguaina, e Venere influente, Ond''ci in quel pumoandada Nan
Ma ae Sardonello ful moftaccio Vede le elle, e linac t altrasfera ua
Gli fece con la spada un' ascendente, Nel vifo ectifia,e dice: Ty
Rofaccio ricoperto di privilegj cava fuora Marte; ¢ Venere; che Pe
tivi influffi, ma Sardonello fece piombare sopr' a di luiun pefimo % Tee
tagliandogli con un soprammano parte del vilo, ¢ del collo, ed un braccio Rani
il qual dolore egli vede le stelle, ed eclifiando |' una, € laltra sfera del Coats
ferrando gli occhi dice: Buona fera, cioè perme, fatto buio, «B Mi
sto Rolaccio si piccava d' Aftrclogo, come s' ¢ detto sopra C, 31M. 63.5 11 Poeta tg
con la presente Otrava descrive la di lui morte con equivoci di termini affrolo- pred
ici. f Lapa
: ARMATO 4 priniteg|. Questo Rofaccio, come ancora gli altri Montamban- ro
chi per accreditare i rimedj, che da essi fon dispensati, moftrano una infiuna di iw)
privilegj concefli loro da diversi Principi; ¢ pero ii Poeta lo fa axmato di privi- the
legi. Uontanlald ken

SGVAINA, Virgilio vagina eripit, Sfodera Marte, © Venere; che predicono
rovine; B dice /gnaina, che vuol dir cavar la spada dal fodero, o guaiaa, perché
s*intende, che aon haveva alcr' armi offenfive, che Venere, ¢ Marte wnfluili
cattivi 'a duaaiead

ASCENDENTE, Termine aftrologico, col quale qui intende colpo di taglio, Un

che viene da alto a baflo, piovendo, cioè calando in ful capo, ec,

OCCIDENTE, Intendiamo l'occalo del Sole', maqui intende ocealo y cio' &
morte di Rofaccio, ily oni aalaan ey baatee tm
VEDE le free. Quand' uno fence gran dolore; si dice + Eeli ha veduto le fielle y hi
perché le lagrime, che vengono in (ugli occhi per il dolore, G

la rescazione della luce, che yi batte, una cosa simile a una quantita di mi -

nute stelle in Ciclo, che pil volgarmente diciamo veder 'nce i

mo sopra C. 9, st. 60, 5 ma qui si serve di questo, perché.gii

re di farlo morire aftrologicamente, i
ECLISS.A. Chiude, cuopre; ficome alla Luna 'reftano i

»hajean > x

#3

    
 

VNDECIMO CANTARE. sip
 dail interposizione della Terra 1 raggi del Sole, quando seguono I ecliffi.
DICE buona fera, Cioè si fa buio per lui, ven donate 10. st. 5. Qui intende
 & finito il giorno del mio vivere. Virgilio in-aternum clauduntur lumina noflem, ©
i asadostndelcginnanalo » che, havendo:manco un' occhio, ¢ Ji fa ca-
vato l'altro, disse: Buona worte per tutto lo tempo, '

i STANZA LIV. STANZA LV. i
Mein per fiancofentefi percoffo Gid per la franca il fangue era a tal segue
Datlo stidion del cuciniere Melicche, C” andar vi si potea co' mauicelli —

 Parafiraccio porco grande, ¢ groffo Istrion Vespi tutto furia,e sdegne
Perch' il ghiowso si fa di buone micche; Rinualto ha quivi tl povero Adaffelli,
| Sirivolta eAeino,¢ da al coleffo E col coltel da Pedrolin di legno
Nelda gola ch' egli ha pien di pasticche, Su pel capo eli squotola icapellig,
«Tal che morendo dolcemente il guitto: acleciopratcane poi la lifoa, el Lota...
» Addio cucina dice, ch'iobo frito, Pius bella faccian la conocehia a Cloto.,
ils STANZA LVL
NGatsi,, e Paol Corbi inveleniti A tal ch'i pacfani sbigottiti,
| Quali villan ch' i tronchsyed i rampolli E dal difagio feonquaffati, e froki
 Taglin di marzo ai fratti ed alle viti (Oktre che a' pachi il numero è ridotto)
| Potanda i bafts braccia, gambe,e colli; Cominctaron le gambe a tremar forto,
. Termina con te presenti Ortave i} racconto del combattimenco (egaito in Mal-
mantile,, ¢ dice la morte:di Melicché, ¢del.Mailelli., ¢ qui tinisce ' Vadecimo

re

 MELICC HE, Vedi sopra C, 3. st. 59, lo chiama Parafiraccio,perché era huo-
»¢del continuo havrebbe mangiato: EB questa voce Parafito, che ap»
prefio'dinoi ha dell' ingiuriofo,non era così apprefio gli antichi,come si pud de-
durre da molti Autori tra'guali Luciano; ma particolarmente da Piutarco, dove
fitrova': Parafitos nontancumappellabant strici adalatores illos, qui apud Dinitums
tmenfas wutriuntnr, fedietianm tos,qus ob rem egrecit geftam,publico /umptu in Prytaneo
atebautur Oc, Ondedelie Stinche di Firenze, nel capitolo in lode del Debito, il

Bernt; è
Voi fore quel famofo Priranco y è
Ab bower yas Doe renews in grafsoin fisoi baront
I popaly che discefe' due F efeo,

Exit Atheneo Parafiti olim appelabuntur foci, 7 fideles Pontificum, eAMagiftratiiz,
Ibmedefimo Plutarco.. ¥

PASTICCHE., Specie di confezione fatta col zucchero\muschiato y:ec; ¢ però
dice more 'doicemente, perché ha gii per la gola 11 zucchero, Pa/feca voce Spa-
boas » siccome anche 'Pa(figiia, che vale Jo stedo; ¢ sono tutte due diminutivi di
pasta.

GVITTO, Huomo vile, abbietto, fudicio, sporco., e sciatto. Vedi sopras
C, 3.\f..9.è:voce Napoletana, ma usata oggi anche da noi, 'Nella raccolta de'
 Poeti antichi-deil' Aliacci, Pra Guittone-scrivendo un Sonetto, siccome da esso si
raccoglie.a Meflere Oneito da Bologna 'Poeta, ¢ amico suo; scherza sul nome di
turer € die, *

— SAS QF Cisasn sees

=

Pita

A.. SSeS

 

 

 
 

  
  
  
   
   
 
 
  

g20 MALMANTILE | © ¥

Voktre nome, Mefsere,¢ caro,e onratoj
Lo meo afsai ontofo,e vil pensando, =
Ma al voftro non vorrei auercangiato, =
10 ho fritto, Scherza col verbo friggere, che vuol dir Quocere carne,0
padella con lardo, 0 olio; ed il detto ho fritto, che significa il
in malora. Latino Attum eff de me; perij. Vedi sopra C. 8. st. 54, tor
nel presente luogo, perché par che dica; Addio cucina, ti lafio non
pili bilogno di te, perché io ho già fritto, ed intende ho finito di vivere.
IST RION Vespi. Pietro Sufini. Questo fu cognato dell' Autore, ¢ git
grandissimo (pirito, copiofissimo d' invenzioni, come si vede in una
commedie da lui composte, ¢ da altre sue Opere poetiche, B pecige p
fentava in commedia ottimamente tutte le parti, ma in specie quella del se
zanni, ( cioè servo sciocco Lombardo ) che ula armare con un coltello di Tegao
simile a quello,col quale si batte, ¢ si scotola il lino per purgarlo dalla lisca,
percid chiamafi Scotola; però il Poeta lo fa azzuttare col Mafielli, ¢ sc
con quel coltello la zazzera. Dice coltello da Pedrotino, perché con tal
ceva chiamare in commedia detto Sufini nella parte di servo feiocco. Questo mo-
ri giovane poco dopo l'Autore; ¢ con esso si pud dire, che in Birenze morifles 4
la moderna arte comica, 0 almeno la franchezaa, ¢ leggiadria nel maneggiarlag =
SQVOTOLARE. Vuol dire battere il lino. Ma qui intende squotere i capelli
per facilitare a Cloto, una delle tre Parche, il farne 1a conocchia, aleism
INVELENIT/, Ancrudeliti, inviperiti, inaspriti, incancheriti, arrabbiati
fon finonimi per intender' uno, che sopraffatto dalla collera operi of
te, ¢ con ira, in maniera, che non sappia quafi distinguer ch'eififaccias, =
Similitudine prefa dal ferpente in collera; di cui Virgilio lib, 2, En, tcolentems
tras,& coerula colla tumentem. wm abean
POT-ANO. Latino amputant,demetunt, obtrancant, tutte similitudini trates
dal' agricoltura. Potare si dice de' traici delle viti, € de' rami degli alberi; ma il
Poeta si ferne di questo verbo per corrisponder' alia similicudine, haveado dewto
quafi viltan ch! e' tronchi, eds rampolli taglin ds Marzo, ec,. sd
SCONGVASSATI, Stanchi, € rovinati walla fatica del combattere.
FROLL], Qui vale per stanchi, ed indebolits 5 t¢ ben per altro Frode vuol di-
re stantio. Vedi sopra C. 3. st. 55. alla voce Leazo, iahesh
TREMAR le gambe foto. Vuvi dir haver paura. Virg. Eo. ry.
ae folvuntur frigore membra, Se ben si puo anche intendere, che le §
mente tremafiero per la debolezza, ¢ thancneaza.

FINE DELL' VNDECIMO CANTARE. +

  
   
   
    
   
    
  

 

 
   

Ze

 
 
 
 

Berbnanhéansr ~ekk dé

 
  

BER TEER BEES

 

 

 

 

HifwMAMALAM

Fea dattacbute

A R GOMENTO,
e A, Montelupo. da Paride 1). nome,
Poi gapigar la Maga,e Biancon vede,
Rimef[a sn. Trano è. Colidera 3 3 e1come.
~ Aarito, al general dd ln fuafede.
a Baldon, che la fortuna ha per.le chieme
Con Calagrille aVgnan rivolgeil picde y
E al suo bel. Regno con Amor va, Psiche
A corre il frutto, delle sue fariche.

 

 

 

ae

speyrepeage

sae PP Regen

STANZA HL
Che sono fratt com! io diffi sopra,

were STANZA I.
Swanco gid di vangar tutta mattina

   
    

“Abconcadino al fin a va-a rifelnere,
- Te forniar Vopresed-in chiamar la T ina

* Cokmerize guarto,eil petal dell'afoioluere;

( Nella Maga affidatifi) asperranda
Da' Diavoit im lor pro veder quale'eprs;

ea chi-vive a speranta muor acids;

eee tn Caffelle ancor non firifina Perch in Dite fon tutti fottofopra,
Phu quei-marei di squoterfi la poluere; 'Per non faper dove, come, ne quando
Onde: Badldon quei popoli-di/per de Laftiaffe il Cornocapolfo,c! ale (chiere
Tal che a' joldati Malmantileé al verde, Esser tromba dovea nelle carricre.
STANZA Il. TANZAILV

E vase Sta, perché porevan dianzi,

vedean col peggivandar ficuro,

~\ Cederil campo, @ non tirare innanzi

ra Star avoler coxzar col mura:

< E così va, che questi fon gli avanei,

Che fafempre colsie'ha is capo duro,

» Che dentro.a feifi reputa un' Oracolo,

Ne crede al Santo,se non fa miracolo,

Di modo, che Plurone omai scornato,

Poiche quel corno pitenon si ricrova,
Pel Proconfolo dice baver pefearo +
Pero connien pensare a invenzion nuova;
Ha innanss ch' ¢i-risolua col Senato,
Eche'l:foccorso 4 Atalmantil si muova,
Ch'egli habbia.aeffer proprio pot savvisa
Di Meffinail foccorso, v quel di-Pifs,

wey introduce i Poeta in questo Duodecimo Cantare con 1a rifleilione, che i (ol-
7

vv dati

|

 
 

    
     

22 IRKTVUA MAL MAN TILE > 1009

dati-di Bertinella non, haurebbono ricevyto,così gran danno\ |
sono accordati,, ¢ non fufione faut in tanta 'tingaiones la.;
in loro per la speranza, che havevano negl'incanti di,Mar
havevano havuto effetto alcuno, 1 Diavoli non feppe:
dove fufie ii Corno d' Aftvlfo., aon si ricordando, che. an
quando Affolfo ando per il fenno.d' Orlando, comedice.|'/
| KANG AKE...Lavorar la. cerca conia vanga... Bipalio

FERALAR l'opre,.Cioè far defiltere dal lavorare eer an
raion Depera. fra. i\contadini.s' intendedlJayoro;, che fa.un'
no, ¢ s' intende.ancora lo Relio huomo.s.che ya.alavorare a
io.ho; chamato due, opere, per iacender due huomint; In quetto lavoro ci
dicci opere, per intender dicci giorai di iavoro, ec,

p44 Ting. La Caterina, intende ladonna del Contadino

MEZZO, quarto... Così chiamano i Contadini ua gran valo —
foggia. da boccale 5.del.quale si leswo, fespartag da bere ai Javor.
po, ¢ gli danno questo cee perché'e forse di ce een
staio. x ae
PER SL afeigluere 1 cont ania ebaainesior il definare asciolvere, Seren csnidal
foluere il digiung, dali. sdigunarsi, ¢d.il definare, lo chismano wien
terzo mangiare Aicono./a cen

eA non si rifina., 'Nanay refla,,
esprima una.op ¢ feng'
Ciog perquoterii,, baftonarsi.. Vedi rae C7. tt. 63. t by

ESSER? al verde., Eji¢r' ajla.fine. 'Tratto dalle candele. di, se ce seieprion dig
fon unte di verde nel piede.. Viano nel Magiftrato del Sale di.
le tafe dell'Oftcric.,,¢ darle.al pili cfferente., ¢ agl.tempo, che aMtneenapi thd
colissima candela di cera tinta da picde di color verde ognuno puo.otferine, es ida
conlumata quelia noo pud. pil veyung offerire sopr' a quell' ofscria, ma shintende Ps
reflata.a, colui), che ha cfiertoyi maggior prezzo, ovver0 non arrivando.lotier. Cm
ta.aldovere, ' Ofteria ai suoyo si dubasta un' altro giorno con nuova candelerta', deat
EB digui habbiamo il dewato hs ha che dir,dicada candela ¢ al.verde, che significa ted

      
    
  

 

on, fifa fine. Ma — che p00 iar

   

BINED a ET SERRE SSESE

 

 

 
 
 

sbrighiamoci y che il cmpo fugge.. E quelto eficr' al verde ¢ pafiato in thes
per tutte le cole, come.cficr' al verde ot danari, vuol dire esser' alia ka
pari,...Va mpderan Poeta teleth scritto neil' Ofteria di Radicofaa Pre
trata » qénm ° Re he age

| Cohanr, p Spebater ridotto alverde.: ee ua ie ' hy

 

. » Gineca, uper ricattarsi, ¢ Sempre perdes

COzzAR col muro», Tentac l'impotlibile,. Contrattar con chiha pb: forea di re
poi, Clavam, ¢. manu Herculis extorquere. Diceli anche: saree a co4hi co! mre othe
ciuoli. Nell' Ecolcfiafico cap, 3. Ditiori re ne focins fuerss 3 Qu Ep
cabys ad ollam ? Quandoenim se colliferint, confringerur. La Feces Fest
tole nel. fume galleggianti 2 una di rame, l'altra dicerra fa a. ¢ oka
quale viene, Anaae ad. Efopo,¢ troyafi refa in versi Latini gala,
CAPL dur ag te oftinati. Dure pom ebb —o Ne
'ST tacas lends » Amico della sua opinionc,¢. che Li, Gh
uw

rs

 
 

~~ oN eeenrvc

S88

SSBB ER CEES £5:

 

DVODECIMDJSIVLTIMO'CANTARE, 8
reat fate', ¢ dit meglio w ogni altro. Huomo di quefd naritat dice de'
= Setpe thea Nim di lapete;e d-etere wngtan” buotao - Baxi.
; edi fe'micltefimo ',¢ pereid ine diviene contumace 3! &

PUR Hess ONT I OM 9G; ty
VENOM orede al Sarito';\se*hon fa miracoli, Non crede'; che una cosa pli poma'ii-
teruenire', fe'rion 1a vede fegitire.Generario prava quarit signum videre. B per lo
più s' usa in' occasione' dammonire, 0 rinfacciare j'come ¢ nel/pretente hiozo} ll
tale è lato pir volte: avvertito dition contindvare @ fat'guella tale operazione-,
perche'gliene' potrebbe seguir male'; ma' egli oftinato wor erede at Bantoy se nor se
miracoh, cioè non da retca agli aveertimenti; ma! vudl-seguitare?,finvhe 1a die
ee fucceda » 4' Proverbifti Greci mettond un Proverbio,-che dice: Primes
a rem. PURI Toe h) LL Mes bas! Pe TNE Dey 99@ 1951b

CHI vive con (peranzia mior cacando. Detto. sporco » ¢d usato per lo pil fa,

genterviles;\c vuol dire -'chiMfi palce di speranza-,'muore di faine'y"ed-in fulliinza

a €*vanitaril'fondarsi nelle speranze. “ai /pe'neratar 5 wi reer

mats ig 2. 0g

  

SON tutti fortofopra, Sono in grandissima confufiune. '
sI DOVEA fer tromba alle carriere.' Dovea' fare feappat tut? peome facev't il
Corno 4' Aftolfo: e'come fa' scappare dalle motfe i cavaili'barbati'y che edrreno
al palio quella tromba, che fuona il banditore, pet dare if feghd della [otpperied
SCORN ATO' »» Vuol dif beffato; ma qui 410 (cherzu di /eorWard\, che Vadeiie
senza corna, come era rimafo Piutone fenza'¢orno, cine senza it Corti dA Nbk
fo. Var animale, che abbia perdute; © tronche le cortia, vient ad avere per
del décoro; onde scornato diciamo per beffato. Acheloo 'fiume; 'efleatlogli d2*Er-
colelévato un corno', rimafe feornato; ¢ fuergonato. Onde Ovidio 9; Met Muh
tas Achelons agrefies 5 Et! Laverne cornu; meuijs capur abdidlit sndis,  Hivtc tanith abla
ti dommie idibure decoris, Gc} 229 10109 Lb 2h 4 HW) 199 > piri
> SPBSCAR per il Proconfalé. Ho Neff, che durar fatica per impoverite; sean,
CG operam perdere. \\'Proconfolo'é ia Firenze il Magiftrato', che soprdutenie 21
dottori,¢ Notai, ed ha la'saa refideriza otto Ie logge,dove sono git altri Viizzi,
acll'ultima abitazione versoril fiume d' Arno; il qual fiume pet quello spazio',
che ¢ fra l'un ponte, e/' altro} ', 6 almeho efa già fortopotto alla 'giurifdizione
del medesimo Magiftrato del Proconfolo'; come pesca'ad elo rifetuata ne' vr ti
poteva pescare senza licenza del detto Magiftracs 3! non vi era-già ditra pena aIfi
contraffacienti, se non la perdita delle reti, ¢ del pesce, che hanno prefo 5 feads
acchiappati in ful fatto; E Pett utes! aie s

STAN ZAV E>) > STN ZA VEU.
ae Paride ritorno, OO Ada quegti 5 ¢° obligate si non Witende,
onGhte nellvoffe alla quarta sboccatura;\ Wor vnol phr quanto un capo di spillerto;
« Eperché dal pacfeegls ha in quelgiornd E subito ogni cosa indietro vende,
0 Foleo ogni nota', liberando ib Tura; Ringrariande cinscsin det buon' afetto,
8| La\gente quini corre @ intornd E'dwe', che da lor nulla precende;

ed rallegrarfidelia fuabravkrh >) 0! 2) EB Te aiteddisfarle bhnho concerto,
Ne lo ringraxiasewrallegrarsi intenta, °°! Perital niémoria gh fara pitt griro
Chi gti da chin lt dona z'chi gli-avvina, ©: Che it tuogo'Aonteliipa sia chizmaro;

ang, ~:

Vvv 2 STAN-

 
 

 

 

  
  
 
 
  
   
  
 
 
  

524

Si si, ch' eli è dover da tutte quanté
Gli fu risposte, ed in un tempo stefo
Li editto pel Caspello fu pe i canti
Per notizia de' Popols fu meffory>
Che dinuleato pos di te avanti yo. «\
Fu offeruato si, che finoadefo-..- \
Lucho nome confernan quelle mura,
E'l manterranno,fin che'l mondo.duna, ¢

STANZA AX ) f \“ A

E che fuor del Caspella il,popoh proves: «3000 (- ) SERB;

Che ognor ne feappa quaiche sfucinata,

Per to piit gemse yeh? a peta 31 i loxofexes fer o

Cotantaé rifinita,¢ maltrattata, — Qui pinto innansi stwile sentiva) >

Tornaril Poeca a discorrer di Paride, il quale hayendo ridosto il Tura nely

fino flaco, haveva liberato quei popoli, i quali per riconofeimento del

ordinarono, che que! luogo si chiamafie daallora avanti Montelupo

torna al campo, ¢ trova ogni cosa murata..
LA quarta sboccarura, Cie ha sbaccato, 'cioe.: manomeflo

vuol dire: ha bevuto tre fiaschi di vino, ec cominciato ibquarto2\Ipérbole, che

significa: ha bevuto molto vino, sborcare propriamrnte Qgettare via

vino, che ¢ nel collo del fialco., per purgarlo affaordalll'obia.yec, LiAQpesas!
CHI gli da, chi gh dona, ¢ chi git avventa iB' detto giecofo nfato per burlares

uno, che figlorij d' eflere sspeflo:regalato; es) intende; chido ee 1

avventa, cice fafiate, ec. € lo scherzo dell'equivoco't:nelwerbomare, ¢ '
NON enol, quant' un puntale d' agherto, Racufarurto.. Vedisfopra Capt, 10;
RINGR AZIO' del buono affetto. Termine di cirimonia julaci si

ringrazia uno'del regalo, ¢ nello fefid tempo si Ficnfa di rice -dicia- f

mo;non voglio,© non stimo il regalo, servendo, per obligarmiy Pinclinazio- '

ne, che io veggio in voi di farmelo; ¢ questa teftimonianza  chehio dal:voltro
affetto verso di me. ist
eH/ONTE Lupo. Finge, che Montelupo Castelio wicino a
anch' eg) quafi distrutto bavefle nome da quota azionedi:
biamo per tradizione vuigata, che eglilfufleanticamente
flare il Castelio di Capraia luogo allora forte' fituato rincontro
cendo coloro, che.' edificarono: Perdifiragger questa Capra 'Non sci guole altro, che
un Lupo, ¢ percid lo nominarono Castello Lupo, che. per esser Mopraiun monte si
detto Monte Lupo. Coca bg ED
GLI venne il grille. Gli venne voglia: E' 1o-stesso, che tocedsill

sopra G. 9. st. 56. con Sp bei
ST &VGGIMENTO. Vn continuo ardente pensiero 50% I

iftruggimento vuol guarire, cioè vuol' adempire questo.fne-desiderio

all armata. Ii Burchiello, fe'ben mi ricorda; Se/piri:d.amo rd

 

 

  
  
 
  
  
  
    
 
 
 
  
 
   
   

 

  
  
 
   
   

    
   
  
 

SPARITO cid che v' era, Non v' cra pil personasalcuna, ip
Baldone era diloggiato, ed entrato'in Malmanule. 495

   
 

 

“DVODECIMO;ETVLTIMOCANTARE. 525
SEVCINAT A, mola neg Vana gran quantita, Fuciaayyicn dal

che wuol-dic
ani

  

(O facina @ i

» 0 luogo dove Gi ri

no mercangic; es
be capire una fucina prela. pec
operasoni

le ic
' et re Bocce, Nov. 2. ee ane eena di diaboliche

igion dire; 4

si erika, 'vuol anche dire il Barccaten de' fabbri 0 delle fonderie,.ec,
| RIELNIT. « Malconcia,@aaca, finica, sopunatai ¢s.intcnde di fanita,e roba,

or STANZA a
pala 5 ¢ ne ri/contra un branco,
Preeti lemgean,
'bi dietro fr ascicar fivedeun fianco
; | gh gi

   

 

STANZA XL

Chi ha scatole, chi sacchi,¢ chi sieehiee

Di givie di mifoee 5 dibiancheriay \
Va" altro ha una ranaca di scrittwe 4,
Ch' agli ha @un Pinto della dtercariby

agli senza.adar albaco, £ piange,ch).¢i le vede mal ficure y
& nee Sete egli ha riscoffo; Pero che *l vento gliene porta via;
 Ciascuno hail/uofardel) diguelletre/ibe, Vat altro dopo bauer mille imbarazzi,
a a og si ha potuco beans ee Port' addofso nna gerla di ragaxzi,
STANZA XIilL
reimbacuccato Arete feretto Le dine agliocchihantutteilfazzoletto,

a ria > 'eJpelse, Ipe/so si szattiene 9

E sgombrane 2 Py rocche,e pergamene

 

tra ys' elle, le stanno. 

Chi'lf il ye chi >

Chi porta nngatto,e La caninainbraccio,

Sono

 

lex
te yede una gran quantita di gente, che fugge da Maimantile, per (cam-
parila vita, e porta feco,le.cose più grate; nel che il Poeta s' accomoda a' gen) di
quelle tali perionc, che fuggono, ed a quello che,per Jo piu,luol seguire ia simili
Seapets;

at ENC INGO. Se ben significa quantita di polli, o di pecore, o simili, tuttavia

ne serviamo per elprimere ancora quantita d' huomini, Lat. bomnum manus.
Vedi aC, 6, tan. 35;

T ASCIC A dittro on Fucwene Va zoppo., per efler Mroppiato da.un fianco.
HA »ifeoffo Senza aspettare al abate, Glioperarj ordinariamente ri(quotono le
ro mercedi, ¢ prezzidelliloro lavori il giorno del fabato 5. ed il Poeta scherza

col. verbo rifquotere, che vuol dire ricever denari ¢ ce.ne serviamo ancora per
intendere Ricever butle.

GVIDALESCO. Malcalcia; Scorticatura. Vedi sopra C, 10, fan. 11.

TRESCHE. Qui intende bagattelle, bazzecole, arneli di poco,prezzo; Lar,
trica, Vedi sopra C, 10. stan, 12,

SCATOLA, Lat, cap(ule. Sono caflette con fondo,e coperchio, fatte con,
fottilissime afiicelle in varie figure, fecondo che richiede la roba, che dentro ay
efie si ripone.

SLANCHER/E, S' intende ogni forta di panno.lino., come tovaglic, lenzuo-
la, camice, ec.

PLATO, Lite civile » dal Lat, placitum, VedifopraC. 7, stan.27.

MERE ANZI A. Altrimenti Afercatanzia. Così chiamiamo, in Firenze quel
Foro, 0 Magittrato, al quale si ricorre, per far l'efecuzioni civili,¢ ai we fon

fone-

 
 

ee

 

    
  

526
fortoposti tutti li Mercanti, ec. il quale ha particolari fat
'MB ARAZZI, Spagnuolo, Embarazes » Roba', th
6 feommodo; ed' aBBHaED il verbo imbarakzare y cht
nefi(, te tina Qanza ¥ ec * » ASSAM vaya
GERLA Da gero Latino','che vuol dire
Ma Voce il nostro Chimentelli nel' Azsr nie?
di baftoni a guisa di gabbia da uceelli
larga je fondato hella' parte pity tretca 5 det
per portare'il pane eotto da un Iuogo all*aitro'y adatrandofelo”
alle reni; € et eeitind nim
firo Autore nella-eecéra alia Seréniilima Atciducheda'Clauiia,
nelProemio'j dove Wie' Che i Prascica diecro'lina gerbi- tdi farfaallond COR
gran quantita dipropositi) Pad bene atiche efere Che il' Poeta intend'
mente ger/a, ¢ che voglia dire, ché haveflero due }o tre bambini'in u
talé gerle §'per:portari'pilr comodamente's coiné veggiamo'tutco”ll B
parire povere donne della Garfagnatiay ¢ d* altrove, che portino due, 0
gaaai addotio imgerie y 6 altri trabicvolifimill /-)) >) 9
tM BACVEC ATO} Copertd5¢ Ito 'bene, ¢'s* ihtende pi
pert ibcaporn Vedi sopra C)1't, Man22: se bendal Cy 6, fan. "64.
ne serve per intendere Metterfi |' abito addosso, tuttavia ¢ da norare,'
intende il lucco, che ¢ l' abito Curiale "if quale'aiicledménte haveva il
per coprir la testa, ¢ però mietrerfivtal"abito si diceva Pmbackecarff; Si
inbavagliare. Giovanbatifta Bufia? asBehedetto 'V archi lettéra nona, | ¢
da AMona'coiet, “ed imbavagliatala la conddffero alle Palle se 3
OLE risconera » Cive riconta la moncta 7 per vedere} i '
traruino y vuol dire imbatcerfi in-who Pma risconttare libri;' ferirrtite 5
danariy contipecyvuolidit Rivederé je rorha®l Ay
Z HO29! ib. 920074 2» MH299)  tegal cuba
» conteaffegnd Wi piabtd', 6'di dolore'
il fazzokeio agli ocehi', Veli topra G..9! stan.48! ahaa
SCOMBRANO « Portan via Seombrare [ quati dal Latind excumiiliré, ton?
trario d? Ingombrare, che ¢'come se fotle dal Lit! ficidmindare] deco
t6, ci (eruc per intendere. portar le? mafferivie"die ania casa a tn” altras
mo in-vece'del verbo diloegiare', sieggiare 'Biaiken archi
BeASPL rocthe je pergamene' 7 re fruthenti atcenenti
habbiamodenowoped nel ©. ye Rad.'>\\E -pertamend intertddid tes aS
Carta con'la quale'fermano ia 'condechia' ia (u'l# roced "per fadifitarell Blare
la'dicond paielibéad #pérshe per tovpit: ol efier facta di carta pecora,€ he ti

  
   
        
  

     
     
     
   
    

  

et anche, carta perdaminay i 9
199 9.99 roel vp SAB AON Z AOR 249 Pr cxortert
Entra: Paride al fimdentro alla portarys 9 * Ma quel che mardni¢ha p tt dppor
OOue\g lipar a! entrasdentroun matelles \ -<Si% st' veder tn' pi m Capen
Chr ad ogni palje troua gente morta. Di scope, e di fascine f
1Oiper lo-mer 5 che (Ps per far fardeliay > ani? i iy:
Oud i IMGselis ¢ edusiiwi wh sil? - otary ate wage "A Ve Loi 4

   

aw
 
   

——— a

BOE Gg CUaio 'Set etree eee

 

 

DVODECIMO;EDVETIMOCANTARE: = 527

oo ye STAN ZAXWE oo cen! Singeatte:
arriuato in pragza, Egliftaben, pere una simil raza,
Perchi(domanda) ésigran fuoceaccefor.. C? ha fatto se @ ogni lana un pefo,
 Egh érisposto:egheper Martinarra, E' si vorrebbe ( Dia me lo perdoni )
bid v'e dentro,escrine:latopre/o;. Gaftigar a milura di carboni. 7
'aride entra oe! Castello, ¢ vede molta gente morta, o malameate ferita., e+
jartinazza mefia nel fuoco per gaftigo deilc sue stregonerie.
 MACELLO, Beccheria. Luogo dove s' ammazzano le beftic per vitto dell?
mo: E per macedo intendiamo Strage, 0 difipamento di che che, sia. Qui
Iptende, che a-Paride par d' entrare in una bottega di ua macellaro in riguardo
=| molto fangue, che vede (parfo per il Castello. Così quel che dice Dante, che
V go Ciapecta tofle figliuolo d' un beccaio di Parigi., Sccfano Pafquier va interpe-
trando, che abbia voluco dire di un bravo soldato, quale era suo Padre, che per
la @trage che faceva, era riputato, come un maceliaro. '
CHE fea per far fardelio, Lat, vafa collig't, Che & vicino a morte;, fla-per an-
darlene da questo mondo. Vedi sopra C, 4, stan, 21.
CAPANNELLY di feope. Piccola capagna, mucchio, monte di (cope., ec, il
eee quando era per l'cffetto, che era fatto, quelto, cra dat Lacini:detto eon

Inc reca Pyra dal Greco Pyr, vuol dir fucco,e noi pure lo diciamo Pira, Dang
126.: ii
i Chi è in quel fuoco, che vien.si dinife,
eed z Ds sopra, che par furger dalla pira,
iets: Ove Exeocle cul fratel fu mifo. »
SCRIVE; lato prefo, Antendi; ha cieito per sc quel luogo - /edem occupauit;ma
Per maggior chiarczza di questo detro, ¢ da fapere, che in Firenze G fanno ogni
@nco tra gli altri quattro mercati, uno per Quartiere, che il primo nel Quar-
Gere, ¢ in fu ja piazza di S, Maria Novella il primo giorno di Quarefima, ach
quale Gi vendono Icgumi, feccumi, ¢ frutte. Li fecondo nel giorno di SS, Simone
nel Quartiere,, ¢ in fu la piazza di S. Croce, Li terzo la.vigila dituitii Santenel
Quartiere ¢ in fu la piazza di S, Giovanni, acl quale si vendevano oche.; mas
questo € andaco in defuciudine; perché ¢ perduta l''ulanza di regalar |' oca lay
mattina di cucti i Santi. LI quarto nel giorno di San Martino nel Quartiere,.e»
in fu la piazza di S. Spirito. 1n questo, come nel fecondo Gi veadono abiti, pans
Hine, ed ogni forta d' arncfi, ¢ maflerizic;.¢ come-che acile dette fire concorros
Ho molti mercanti di panni, ed altri artefici d' ogai forta.,. così alle. volremanca
doro il luogo, dove polarsi, per farui.ia quel giorno la lox boxtega; onde. piglia-
Ho il luogo qualche giorno avant, ¢ fegnano jo spazio dei juogo,, che piguano
con getio 50 altra unta, ¢ vi (crivono in leere cubicali LATO PRESO,, ¢ que-
flo servc per impedire, che altri entrino in quel juogo.: Edi. gui diccadofi; I
tale ha (critto /ato prefo in quelia cala, ec, intendiamo: quella cala, ec. ¢ per iui,
wne gli pud efier tolta. Così. dice, che Martinazza scriveva dace pre/o in quel mon=
te di scope, per iagendere y.chc havea tatto in modo, che.qucl fuoco.non le po-
teva esser tolto.... 4g Neds e402 an ic
|fatto a' ogni Jana un pefo, Ha commefio ogni forta di de'i\to-senza riguar-
do alcuno. “Si dice anche far d' ogms erba fa/cio, Che in (uitaaza s' jntende un' nue.
mo

  

 

|

 
 

 

 

 

 

 
  
     

38 “1 ALMA NIP 1

mo scellerato, di coscienza larga fhe Hon' tetne
giuttizia; che'in Latino' pure si ditebbe, ex guoliber,
mea quella; Aivdum fie-pratum, quod non persranfeie lit

b10 me loperdent, Detto da Ipocriti, perch ¢ in' un' certo
cenza a Dio di fare un peccato impune.1 Latini havevano'una i
che parte simili + Si Dijs«placet', " eee
. GASTIG ARE a mifura di-carboni. Dar maggior gattigo di
il detingtente. 11 carbone ¢ fra le pitt vili/ mercanzie; chef
mifura, € per questo nom ff guards così: per la minuta in darne
bra, ¢ pero habbiamo questo detrato, che significa: dar' pile |
nel Morgante. ef mifura di crufea, ¢ dt carboni, + 0 RE Oa

STANZA XV.; STANZA HVE)!

Ia quespo., ¢ ognum parla della Strega, i i a
Si fente dire; A voi; largo, Signori,
E un bnomaccion più lungo a' unalega,

Dal Palazzo si vede conaur- fuori, Per esser vogavanti di galere

 
      
  

     
  

  

Poi sopra il Carro, ove Birrenoil leva, Chetal fa d Amoktante
E cinto ( come già gl Lmperadort ) Eperch'egli@un ?
Dialorowmvece, a' uncarton le chioma, Sentengtaro I hanea' nfarey
Va trionfante al Remo, non a Roma, Che Atalmantil non ha legniyne Mare
STAN ZA XWPLY

Percid, mentre che tutto ignudo nato, Lat confulte it decreto ha renocate 5 ~
Senonch' egli ha due frasche per brachetra, Sicche di luimndn' ordine 8 >
Sh) bel trofeo si muone, ed ¢ tirato Ed ¢\ Stato spedito un Cancel res
Da quattro canallaccs dacarretta, €on pik famigli « farlo-ratzenere o>

. I. Gigante Biancone legato ignudo sopra un carro ¢ condorto fuori di: Palazzo
per efler menazo in Galera; ma quella esecuzione refta sospela, perehé Malman+
tile non haveva', ne Mare, ne galere-, Haba 3 sun-

LARGO Signori', Date luogo; Fate ala. I Latini far far largo dicevano Sum
monere, Orazio. Neque confularis Summoner liter. Vedi sopra C, 11. fam

PIV" lungo a wna lega, Iperbole usatifiima per esprimere Lunghitiimoy Di
atiche pis /ungo a una picea, 6 LO alae

BIKRENO, Intende birro, e'fi dice'cos) per 1a. similitadine
con Kirreno, che fu amante d' Olimpia,fecondo |" Arioito', dal! snes.
capertamente birro diciamo: lo /poso a' Olimpia, th ial ene

CINTA di cartone (a chioma, A coloro,, che per delitti-fon las
frufta, afino, o berlina, fogiiono per maggior vilipendio meceereinteta un bet
réttone di fogiio', che per-efier a foggia ai mitra-epiicopale lovehiamano milena,
quali' sono 'quelle, colle quali farono:dipinti nelle itira del PalagiodehPorelta
oggi derro del Bargello', 1 seguaci del caceato Duca @ Areael, le
per l'antichita appéna si veggono'. VeditopraG, 6. Man, 56, €eque
per cartone, che pet altro vuol dire quella carta grotia, che (erie
incartar pauat, cc, r

HAVO MO abandiera, Haomo a cafo, inconfiderato » volubil
riofo nelle sue operazioni. +k Saga, Url al

 
 
   

 

 

SS ae
DVODECIMO,ETVLTIMOCANTARE, 525
IGNVDO nate. Affaito'igdudo % Vedi sopra ©, 2. fan.-64. IL Coloffo ad*noi
; e"mto ignudo; faluo.che ha due frasche per braghertas cio' duc
fogliedi vite-fatte di ferro 5.0 d' altro metailo dorato, che gli cuoprono. le parti

  
   
 

& ¢ SESLOU Re Ub =e a
« CAVALLAGCCE da carretta, Coloro., che in Firenze tengono carrette a vet
ra? per-portar mercanzie yed arnefi da un luogo.a un' altro hanno fempre caval-
lacci vecchi, rifiniti, ¢-ai poco valore, ¢ pero dicendofi cavalio da carretta ys"
intende cavailaccio di tal forta. Qui il Poeta finge, che il Gigance Biancone fal
smelo sopra.avun carro tirato da quattro di questi cavallacci » perché 1l Colosso
detto Biancone sta sopra ad un carro, che si. figura tirato da quattro, Cavalli
anarini,. > ' 2 a
1A vinocate il Procefso.. Intendi ha: mutata la fentenza, o decreto della galera
havendo considerato, che non se li poteva dare esecuzione, perché Malmantile
non ha gaiere,ne dominio di mare.; '

» » STANZA XVIIL STANZA XIX,
~Hragaxzi infrattanto, che fon triffi, E perch! ei.nonha in doffo alcuna vefay
© Aveder cio che fuffe, essendo corsi, Lo fegnan colpo colpo in modo,tate

Epaich' eglié un prigion,/i fona avvisti, Ch' mmnanzi ch' e finiscan quella feta,
let Bich eglieben legara, ¢ non puo sciarsi, Ne lo fuifaron, ¢ conciaron male;
ly) Wnitamente in un balen provuifi E al miteron, che atorre haueainsefa,
Di bucce, di meluzze, rape, etorfi, ( Bench giammaispuntate auefel' aig,)
* Cominciarono a far achi pri tira, Conquei suot merli, che non ban lepeane,
Ed anche non tiranan fuor di mira. Pigtiar volo alt aria al fin conuenne,
Narra gli strapazzi, ed infulti, che yengon fatti al Biancone, ¢ con questo
smoftra il coflume de i ragazzi Fiorentini, i quali quando ua malfattore ¢ condot-
-to per la Città in full' afiao, 0 metio alia berlina, lo trattano nella forma, che
dice del Biancone, tirandogii torli, cio¢ gambi di cavoli 5 bucce di poponi, ¢ si-
“mili immendizie. £ nota che havendo egli.detto, che Biancone haveva Jamice-
»ra, perché il Coloffo detto Biancone ada ha yeramente la mitera » fa che i sa-
~gazzi la levino co i faifi di capo-al Gigante Biancone». i
-4N-nn baleno. Subito; In.un batter d' occhio, detto sopra C, 11, stan. 42. Di-
ciamo anche: in men che noo,balena; eflendo il baleno, o il Jampo 4. siccome yil
vento,¢'l fulmine cosa velocifiima, Onde noi d' uno yche corra ¢ sparisca.yia
fuggendo, diciamo = £' pare il vento, Ha fatto comenu baleno. Corre y come units
Yacsta, Pare che"! vento se loporti, Virg. En. |. 5- J,
Primus abit, longeque ante omnia corpora Nifus ont
Emicat,& ventis 5 & fulminis ocyor alis,
Dove quell' Emicat vaic: Scappa fuora,¢ innanzi agli aleri, come um lampo, Si
Swede correr la piazza in un baleno,
«»LVON tiran fuer di mira. Colpivano nel Inogo, dove fegnavano.. Vedi sopra
~C. 1. flan..37. dove troverai co/po colpo, che significa ogai coipo, che ¢' tirana.
Che diciamo anche Zorto bette, Mira ¢ lo stelio che Scopus, voce Greca usata.da'
« Latini,; facta da Scopein, mirare,;
le PkIa Ache finife ques foffa. Primachee' finisse quell? operazione; Si dice
anche + quel/a musica; quel baccano; aes Ȣsimili, Vedi sopra C. ae fh 53+
c xx.; +, tbe

rad

=eSh 2

See CUR SSS

=e

~  a ae

MBAS, %
eng

 

 

 
ss

 

 

     

530 MALMANTILE

MITERONE a torre. Quel foglio, che per derifione si métte i
fattori detto mitera, come habbiamo accennato poco |
doil capo al delinquente, apparisce a i circostanti una roronda t
la parte di sopra di detto foglio molte volte l'intagliano a guisa d
farsi sopr' alle muraglie delle Città; ¢ così havevano fatto a quelle
e'perd il Poeta scherza con la voce merlo, che è un' uccello note
glia dicendo, che se bene i merli, che haveva in capo Biancone n
mar mefle le penve, ¢ non havevano mai spuatate / ali; tuttavia
vouare,ed intende, che quel Afirerone fu fatto volare dalle bucciate,
che gii tirarono quei ragazai, con le quali glielo levarono di testa, s
STANZA AX. STANZA XXIL oy
Paolin Cieco, il qual non ha fuvi pari Ed ci lo donaa Bieco,e a Pasian
Nel fare in piazza ginocolar' i cani, Col carro,e tutcel' altre ap,
E vendea l' operetic, ed ¢ lunari,
E proprio ha genioa spar coi Ciarlatani,
Pens[ato ch' ex farebbe eran denari,
Se quel bestion veniffe alle sue mani,
Pere' baurebbe,a moffrarsi,quel Gigante
Pix caica, che non hebbe l Eiefante.
STANZA XX1
Così prefa fra se risoluzione,; Subito qui Paolino feende,
| Vain Corte a Bieco,¢ lo conduce fuora; Per trouar qualche st buon.
Gili dice il suo pensiero, € lo dispone Havendolo ferrato fra due ee
etchieder il Gigante a Celidora; Accio non sia veduti da persona, Vey

  
 
 

 
  
  
  
    
      
    
 
 
 
 
 
 
  
 
  
  
      
   
 
   
 
 
   
     

E Bieco andato a ritronar Baldone Bieco a tenerlo con due altri atendey
Tanto l'infipilla, ¢ allora allora E se lo vede muouer 510 ha;
Ei corre alla cugina, e gliene chiede; Ma egliha fortuna, perch écni grande,
Ed ella volentier elielo concede, Che non gli arrina mancod

fande,
Paolino Cieco ottiene da Celidora in dono il Gigante insieme co! carro, ful quale
era, ¢ ful quale lo condufle a Firenze, ¢ si fermo ia fu la Piazza della Signoria,
havendo chiufo dewto Gigante fra due tende; affinché non fatie venduto, ¢ men-
tre.così flando, Paolino cerca d' una flanza, per metteruelo, ¢ farlo poi vedere
a coloro, che havefiero pagato un tanto per uno, come si faceva dell' Biefaate,
fuccetle quel, che sentiremo appretio, * ie
“ ELEF ANTE, ¥u condotto in Firenze più anni ono un' Elefant
il popolo per la curiosica correva in gran numero a vederio forto ie logge
Signoria ( hoggi detta de' Lanzi, perché quivi € il quartiere de' Trabanti, 0 fan-
ti della guardia del Serenils, Gran Duca da noi chiamati Lanai') dove fava rin-
chiufo in un tavolato, ¢ si pagavano alcune crazie per entrarvia vederlos ¢
fio animale fingulare ne i noltri paefi, mori in Firenze per lo gra freddy elas
sua pelle ripiena, ¢ lo scheletro nettato, ¢ meffo inficme si confervano nella Gal-
leria del Serenifs. Gran Duca. ucoensini aie
INZIPILLO'. Inttigo, stimold, pregd instantemente, ¢ forse voce corrottas —
Sill:

    
   
   
 
      

da hbillare, Latino foilare, infufurrare, trovandolt nella' flor

traccaco feume: Di-ninwa miferedenca era fhato antore 5 ¢ nulla male:
date, ta

 
   
 

DVODECIMO,ED VLTIMOCANTARE, = 31

TRAINO. Diciamo guella quantita di roba, che possono strascinare duc buoi,
che i contadini dicono trainare, ed il veicolo chiamano traino, 0 treggia, La-
tino traba, 0 trahea, a trahendo, Virg. Georg. 1, Tribulaque, trabeaque, © ini-
que pondere rafiri. Si dice anche sraine una mafura di travi, che contiene quattro
Breccia quadre. Qui intende quel carro, sopra il quale era il Biancone con tutti
phate arnefi, ¢ pigia la voce sraino nel significato della voce rreno usata per

rfi intendere carro, ¢ bagaglio dell' artiglierie; !a qual voce s'accorda 'colla.s
Franzefe Train. Noi percio ja diciamo ora Treno,rapprefentando quella prooun-
zia; ora 77a:mo coll' accento fulia prima, non facendo conto della pronunzias
Oltramontana, ma della (crittura. Qui il Poeta dice Traine coll' accento fulla.
penultima; per accomodarsi alla neccitsta della rima. Franco Sacchetti nelle Ri-
me fimiimente pose quelta voce nelia fine d' un verso,

Per tirar colti piedi un gran traino,

LA Piazza dela Synoria, La Piazza, che hoggi si dice Piazza del Gran Du-
a,¢ si diceva de' Signori, 0 delia Signoria, perché è d' avanti al Palazzo de'
Priori, ¢ Gonfalonicri di Firenze, che si dicevano la Signoria, nella qual Piazza
@ la fuddewta loggia, detta de' Lanzi

CHE non gli arrsva manco alle matande, Cioé non gli arriva ai bellico, perché
mutande chiamiamo propriamente certe piccole brache, le quali si potiany,quan-
do si va a bagnarsi in Arno, per coprire le parti vergognofe, le quali mutagde»
per ordinario cuoprono dai bellico fino al principio della colcia.

STANZA XAlV, STANZA XXV.
Piange Siancone, e chiede altrui mercede, Quei tre yc ognor came cuciti a i fianchi,

E mentre il Fato,e la Fortuna accufa,

Euor delle tende si guardo gira,e vede
« Perfeoy'ha in man la testa di Medufa,

E immoto refta li da capo a piede,

Ne piit si duol yma tien la bacca chinfa,

Perché col Carro, e tutta la sua muta

De cavallaccs in marmo si tramuta,

Gi favan quivi,accioch'ei nofeappaffe,
Privi di fenfo allora,e freddi,e bianchi
eAnch' eglino si fanna immobil faffo.
Ata perchs'l protungarmi non vi stachi,
Giie me',c' a Malmantile io mene palji,
Ove git amici Paride ritrova,

E fente,¢' ogni cosa si rinnova,

 

Ii Gigante Biancone era così grande, che avanzava il capo sopr' alle tende;
nel girare, che egii fece la telta verlo la loggia de' Lanzi, vedde i tcichio di Me-
dufa tenuta in mano da Perfeo; per la qual vilta rimafe immobile, ¢ diveanes
faflo tanto lui, quanto il carro, i cavalli, ¢ coloro, che gli crano d' attorno; B
così il Poeta da la sua fine, ¢ si sbriga dal Gigante; di poi ritcorna a dilcorrer di

 quel che si faceva a Malmantile.

PERS EO, ¢' ha in man la testa di Medufa, Questa è una statua di bronzo, las
ae € fiuata sotto un' arco di detta loggia de' Lanzi; opera di Benucnuto Cel-

i; ¢ rapprefenta Perfeo con la testa di Medula in mano, verso ia quale flacua,
guarda il Coloffo detto Biancone, percht ¢ di marmo bianco. E nota la fayola
di Perfeo figliuolo di Giove, edi Danac, 11 quale uccile Medufa figiiuola di For-
co strupata da Nettunao nel Tempio di Pallade, la quale percio sdegnata conver-
tii capelli di Medwla in ferpi, ¢ fece che la sua facia faceili diventare di faffo
coloro, che la guardaflero: Ma il detto Perfeo havuti da Mercurio gli stivali, ¢
la scimitarra, mentre Medufa dormiva s le tagho ia celta, 1a quale pot ee

xX 2 mefle

 
 

 

 

   
  
 

sg IFATHAI OM ES +9 vi h
AS3HL ) pire Ofisip MALM he thy; 4 ie on si
miéfie nel proprio 'feudo, Di questa favola si servé il! Poeta 4 } 7
gante;dicendo, che per haver' eghi mirato questa tettadé-]
marmo, € così da graziofamente una favoloia origine a questo f
rapprefenta Nettenno Dio del Mare', ed! è»posto nella: Piazza' del G
sopr'ad un carro tirato da-quattro cavalli marini nel mézzoa una
quale riceve I acqua', 'che featurifee davaleuni niechi, © conchiglic
in mano da alcune statue di Tritoni-alte quanto le gamberdel d
or dette flawe flanno attorno:"E queste il Poera finge', che fieno

mipagni, che dice fargli cucits a i fianchi, ¢ che non gli arrinano a le
dé'; E così viene a conformarsi col gruppo, che si vede di queste ttarue
fo tutto di marmo,

CVCIT 1 ai fanchi, Stretti attorno, come se fuffero euciti, Detto uk
per'esprimere uno, che mai si levi-d' attorno a un' altro;€ qui corna bene,|
Ché quelle flatue sono così strette attorno aj Coloffo, che paiono cavate:
fo marmo, del quale ¢ cavato il Colosso.

GLle me', Glié meglioy. Vedi sopra C. 2. st, 10. <a? S

PSTANZA XXVEy STANZA XXVHE
Poicht Baldone eAalmantile ha prefo, Cos} cercando le grandexe i |

E tutte quelle povere brigate Soe @altrihor feo Ve, ?
Saluopera chi non si fuffe arrefo ) Onde tornata Celidora, il Lage
mii se ne fon ite a gambe akace 5 De i popoli padrona, ¢ dello Stato
Sitché'da queite havendo al fin coprefo Temendo ancor de'

 
  
    
        
     
 

.

    
     

 

    
   

 

 

       
 
 
  

'Pot Bertinella, ch ella l ha infilate; Nuovi Miniffri fa, nuove ve i
Perammaxzarsi sfodera un pugrale, Se ben de i primi poco ha da temere
Ada quei,ch'é buono,non le vuol far male, Che tutes ban ripiegate le bandiert i
STANZA XXVIL. STANZA BALK
Clienon fo come gli esce fra le dita, E per eftinguer la memoria i
B/fulta in Strada, che le gabe ha deftre, Di Bertinella in ogni gente ye-loco }
Ov" ella a ripigliarlo'é pos /pedita Si levan le sue armi, il suo ritratto,

Tagliato in croce si condanna al fusce's
aE perch'elt habbia a raccorciar, la gita, Vn bando va di poi, & averum patto
Le fa pigliar la via dalle finefire; Neffan ne parti pite punto ye poco
NEMa wa sh 5 ma poco poi le importa Sotto pena di fear in fu la fume
MT rovaricht amarza,se viginnge morta, Quattro mefi al palarzo del Com
Celidora tornata padrona di Malmantile fa buttar Bertinelia
ordina nuovi Magiftrati, ¢ comanda, che non si parli pitt di Berti

villime'pene. jo faites

Dix'chi dopo di lei fa le mineftre;

    
        
 
    

  

ELLAL ha infilate. \ofilar le pentole, vuol dire Esser rovinato
ver finito-, 0 perduto la roba, ¢ la vita, ec, clie di tutto s*in cok
mente. “tale ? ha inflate. Latino decoxit. +20 sett

LE gambe ha dere, Non, che quel pugnale haveffe gan
dire } cheetlendo grave, gli fu facile andar' a baffo in strada';
perie 'fineftre anche Bertinella da chi fa le minefre, ciok dachi
avichi comand; chee Celidora ritornata padrona di Malmanule.
gacge ae peccato, Ha la pena det suo fallire, ¢ che ha m
whet; Fs

   
   

  

 
 

  

1 v¢ t E LAN 7
-.  DVODECIMO;EDVLTIMOCANTARE. | 533
' flaver voluto per strade indirette farsi Regina', usirpando queld' altri iio! is > /
be i. icsanlig voghamosintendere uno, che piocenea oe taper fare Ogni-cosa
meglio degii altri diciamo; M.raleéit Lagi, Che il Lagi fu anticamente un Sen-
icato wv Firenze, che faceva tutti i negozzj della piazza': Si dices
rO per scherzo, ¢ per una certa ironia, ¢ derifione. ho “ogee
* HANNO ripiegato le bandiere, Cioéhanno finito; Son morte, Il Petfiani,
parlando di se medesimo in quetto proposito disse + ty
core edi primo tramontano a quest® ascintte ae)
si Be Ditems pure sl requie,¢ il Miferere,:
Perch' so fo vela, e piego le bandiere; '
 E buona notte; a rinederci tutti,
LE fue' arm, Intendi'}*infegne della sua cafata, 0 stirpe. ues
7 ~ STAR in [u la fune quattro mefi. Now & posibile' star in fa 1a coda quattro
y hore, non che quattro mefi., ond' io penso, che con questa iperbole voglia iaten-
: sia condennato alla morte, alludendo agi' impiccati, che in un certo modo

 

quando pendono daile forche a vifta del

popolo; st poslono dire are in fulla corde,

be in fulla fune.
¢ STANZA XXX. “STANZA* XXXIIL
jee | Yr Orarore intanto de' pitt brani Spiegafi se desea 4 ttn tavolotto
" ACelidura Aaimamue inuia, Vol abito mavi di mexzalana,
a 'Che det Caffelo ad essa da le chiavi, Che infu fianchi appiccato ha per diforto
Evende omaggio con la diceria; Pn lindo rid aief alla Romana;

 

Ed ella in detti macffofi, ¢ gravi
Pronta risp a tant' Ambasceria;
Inds le chiavi piglia ye nn' altro mazzo
= Wi quelle delle flanze det palazzo.
ae STANZA AXAIL,

E perch? gti è un perro, ch' eli' ha voglia
Di riveder, come ad arnefi e pieno;
Del Mamoye d'altri addobbsfi di/poglia,
E comincia a girarlo dal terreno;
4Guardarobi aspetta, ead ogm fogiia,

Poi viene un verde nuouo camiciotto
Con bianche imbaftiture alla balana;
E poi due trincterate camicinole,
Che fanno piatza d' arme alle tignuole,
STANZA XXXIV,

Vua Rimarra pur difaianera, ~~
Per dove fifa a' faffi arcifquifita,
Perché gli aliorti, ¢ it banero a spalliera
Pavan la teftaye in giu meza la vita,
Portandola alle

'i; te,o0anna fitra,
\C* ad aprer gli usci patono it baleno; Torre,e comprar si pio roba infinica,
è E subito poi lefto-uno safiere Cb elt" hadue manicon s) badiali,
mn Quand' elta palfa, le alza le portiere. Che è ine quattordici arfenali.
f STANZA XXAIL, STANZA -XXXV,
Ed ella se ne va ficura,e franca, Vina cappa tane bella, e pula
b Sapendo ogms traforo a munadito, Di cotone; se ben vefta indecifo,
ie Perché troppo.non è, ch'ella ne manca, S' ell'¢ di drappo, o pur ringiovanita,
EP abito, fin quando havea mario, Perché non se ie vede pelo in vifo,
'0 Scefe; )£i70 y fali ne mat fu hance s Evvi @ abiti pur copiainfinica,
2 t Sin che non hebbe di veder finite; Mia chi unto, chi roto, ¢ chi ricifo;
2 All' ulssvia si fece in guardaroba Che il tempo guasta tutto; ¢ per marura
% eAprir gli armadi,e cavar fuor 1a roba, Cosa bella quagzit pala, ¢ non diira,
4 Malmantue manda un suo Ambalciadore, 0 Depataco a renaer' wbienes:
a Ce.
f.

 

 

E
 

  
       

44.534
a Celidora; ¢d ella attualmente, ¢ corporalmen
tutte le flanze del Palazzo, ed in Guardaroba fa la
veramente adeguati a una Regina'di Malmantile.. 3
RENDE a la diceria, Cioé fece una Orazione d'
mone, 0 Discorso, col quale refe ubbidienza. is. 4
HA voglia di rinedere. Ii Poeta (prime benissimo il genio unit
fire donne, quale è di rivedere tutte le caffe 5 armadi, ec. subito, che
© maritaggio entrano in una casa a loro nuova, ho isch ete
TERRENO., S' intendono qui, fecondo l'uso., le prime fanze d' una cal
che sono al piano della frada, Del reo Terrenoé la tetra stessa così,0 così ¢
dizionata. Latino terrenum; folum, ager.» - send 5 -
PALONO il baieno, Ciok tannopretto, Dante Pars 25. Subito
di baleno. Inf, 22. i2 men, che non bafena, vatiot ”
OGNI traforo. Antendi ogni porta, ognicriuscita,/ogni minima:
4A MENA dito, Sa benitimo. Latino caller, Le sono notissime st
L ABITO' fin quando banea marito, Celidora, comes' ¢ detto sopra C,
Fu moglie del Re di Malmantile, ¢ da lui haveva ereditato i Regno, i)
MAVE, Color wrchino chiaro. Azzurro sbiancato, i
GV ARDINF ANTE. Vedi sopra C. 5. tt. 8. *: geomet
MEZZ ALANA, Tela fatta di lino,è lana, che inuna fola parola si dices
ancora acce//ana, quali accia, ¢ (ana; roba aflai da i nostri Contadini.) |
C.AMICIOTTO.. Così chiamano le Contadine,quella'velteda donna, cheles
Fiorentine chiamano fortana, Et
CON bianche imbaftiture alla baixana, Costumano le nostre Contadine di fares
nelle Joro vefti yicino a terra una cintura con punti di refe bianco in ful nero jun-
ghi, acciocché si veggano da lontano, ¢ queiti punti foftengono una piegaturas
fatta nel giro di detta velte per accortarla, ¢ serve a loro per ornamento,0 guat-
nizione, ¢ si danno ad intendere di far creder nuova la medesima — causa
di quella punteggiatura, ¢ che aliora sia uscita deile mani del Sarto; il ee
quando vuole imbaftire,.0 dar priueipio a cucire yo' abito per mettere int 9
eda fegno i pezzi, che vuol cucire, ¢ solito fare tal punteggiatura larga y das
queito imbaffire si dice imbaftitura altrimenti feffitura, © ritreppio, Latino /ubfutnr4.
E questo verbo smba/tire servc per intendere ogni cola principiata,e non perfezio-

 
    
      
        
  

   

  
 
    
   
 
   
 
  
    
   

    
 
 
 

 
   

nata; come éo ho imba(Pito L' oraxione, che debbo recitare 5 ed in poche ere ”:
che diciamo abboyzare. we

BALZ ANA. Iniendono il giro da piedi della vefte; altrove Pideos 'Latino
limbus « LF

TRINCIER AT E.camicinole.. Vuol dit camiciuole confumate dalle tignuoles »
per la similitudine, che ¢ tra una campagna pieaa di trinciere, ed.un panno ple
no d' intignature, che percio apparisce bucato, € trinciato, Vedi sopraC. 8. st
51. E.che cosa sia camiciuola. Vedi sopra C, 6, st: 57, at otwe att

BANNO piagza a arme alle tignuole, Vedi opra Co. 51. -questo medesimo
concetto sopra il capo del Tura; B che sia tignuola al C, 6. st. 54. € Cs 10. (h 12+

ZIMARRA, Abito, che già ulavano portare le Donne Fiorenti all?
altro abito detto /orrana; il quaic da i Latini ¢ detto amiculam, il qual'

' YY

 

 

  
   
 

tie
a

“= SSeeresiut

 

 

 

DVODECIMO,EDVLTIMOCANTARE. ~ 535

'veramente assai decorofo, e modeflo, ¢ non come quello, che usano hoggi, del
quale si pud dire:con Quinto Curzio lib. 5. Feminarum conniusa inenntinm in prin-
cipio modeftus eft habitus, dewde fumma quaque amicula exuunt, panlatimque pudoré
profanant, ad ultimum ima corporum velamenta proyjciunt, Ma tornando a proposi-
to: Quelta specie d'abito detto Zimarra haveva intorno al collo un collare gean-
de (che chiamayano bavero ) fatto di tela incollata,e cartone,e ripieno di stecche
d' offo di balena; ed in fu le spalle, dove ha principio il braccio un giretto actor-
no al braccio farto della stessa roba, che il bavero ) qual giretto il nostro Autore
appella aliotti, perch così si chiama, ed alle volte si dice piffagne ) dal quaies
pendeva una manica larga.,¢ grande quanto una buona sporta, la qual manicas
non s' imbracciava, ma serviva così pendente per ornamento, ¢ per una certas

“grave accompagnatura; ed oltre a questo dava commodita di riporvi fazzoletto,

Oaltro, che occorretie. Di queste maniche, tali se ne fon vedute a' mici giorni,
che farebbono fiate capaci di cinquanta libbre di grano l'una, ¢ pili; © però il
Poeta dice, che sono 11 cafo per andare alle nozze, ed ai mercati, perché vi si
pud mettere molta roba dentro: E gli-aliorri, ¢ banero difenderebbono da un col-
Po in riguardo della roba, di cui fon compolli; E dice /a rea; perché questi ha-
veri, nascondevano dentro di loro tutto 11 capo di chi gli portava; ¢ tali aliorti
si sono veduti, i quali coprivano pili di inezzo il braccio.

DOVE si fa ai fafi, Dove si tirano le fafiate; il che segue in Firenze in Mer-
cato nuovo, dove 1 garzonetti delic butteghe de' Setaioli quindici, o venti giorni
avanti alla Solennica di S, Gio, Batilla fra il mezzodi, ¢ il vespro fanno fra- di
loro alle fafiate, ¢ necetiitano tutti li bottegai di quelle contrade intorno al Mer-
cato nuovo a star ferrace per quell' ore; ¢ questo fanno per folennizzare la decta
fefta quel tempo innanai; ¢ per questa ragione tutte le botteghe, che sono in quel-
la firada, dove tirano i fatfi, hanao la riuscita in aleca strada per di dietro, di
dove entrano i macitri, ¢ lavoranti, senza aprire Jo (portello principale, ¢ quivi
attendendo a i lor lavori, laiciauo che i loro ragazzr Gi piglino per quell'ore tale
spaffo 5 anzi ci (ono taiuoica de i maeitri, che comandano a1 loro ragazzi, che
vadano a pigliarii, spaveatati da un profetico detto: Guai a Firenze, quando in.
Mercato non si fara ai faffi; V sano di fare a' fai anche in Roma i ragazzi Tra-
fleverini. E fare a' faji, Hgucacaiente s' iotende » Mandar male, rovinarsi, get-
tar via il suo, Latino di/apidare, fare alla peggio, ¢ operare senza giudizio; si
faceva a' faffi ancora in Firenze per accafione d' allegreeze pubbliche, ¢ una fine-
fica di rame traforata fu posta al Palazzo de' Medici,oggi de' Marchefi Riccardi
Per vedere questo spettacolo, come ¢ sato da altri scritto, ed offeruato.

ARCISLVISITO, Ui cafissimo, buoniitimo,, attissimo, ¢ pil, se più si pud E

dire. B' un termine, ches' usa per farsi intendere; più fu, che il superiativo, di-
cendofi buono, pil buono, buoniflimo, ed arcibuonitime. Ma dicendofi buo-
NO, Migliore, in vece di pil buono, ¢ ifquifito in vece di buonissimo, che fa.
V effetto del superlativo di buono, non pare che sia ben detto pili ifquifito, e»
isquifitissimo,facendosi cosi'un superlativo di superlativo; tuttavia per J'ulo inteo-
dotto non farebbe riprefo chi Jo facetle; ed io crederei, che fufle meno biatime-
vole dire, arcifquifite, che isquifititfimo, perch non trovo troppo in uso il dire
pil isquifito 5 onde non pud s' uso antrodurre isquifitisimo.s che toguirebbe al pitt
squi.

ae

 
    
     

536, uuie nohace dae aaa
isquifito..;L Latini dicono bonus, melior 9 che: Q d F
byono,, migliore 5 ¢ i/quifite; ed io conde cic i che Bctedfinn piste ass s
miffiaus:, che faonerebbe più isquifito, isquifitissimo, se it I

trova eptimiffimus.. Appretio det nostri Autori Toscani si trova, 1
molto, aijai, ¢ simili a i superlativi, come notammo Coat 17. >} ia r
buona grazia di efi, lo flimo.errore, perché molto, piu 5" » Hiufilis
faculta di scemare, e non cre(cere il superlativo,. aa

er efempio if tale ¢ Luonissimo, vuol dite il tale è perferramente
iamo molto, certo', che (cemiamo la perfezione di buono 5
molto buono, ma von perfettamente buono, eficado maolte una'
s2.5 € non indeterminata, come ¢ i} superlativo: EB — » che
1iguifito, ¢ isquifieissimo, © arcifquifite, hanno prefa la vace s
tivo da per se, ¢ non come per superlative di buano; il che vi 7

tofna poi all' addigteivo aigiiore, che non riseve alerazione 5 nomdicendof » nondicendof i;
migliore, Be migliorissimo, le hen fidicewmalte migiione 5 e:alfai mia
marlod' eflenza;. come ia bbiia thes detto s»perché solo 5.0 affat miglit
men buono, che non fa migéiore aflolutamente detto:, se non comparando:
all: altra quale sia.di loro meglio, st Amr
i ZANE, Colore fra il paonazo,¢ i} lionato. 2p
OTONE, Vuoldire bambagia non filata, Manoi per cotone:
forta dipanao col pelo annodato; come.è la saia rovelcia 50 il rovele [
Hon si dicono corone se non. hanno il pelo aanodato, che allora si dicano di coteney
© actoronati, Dice, che num ¢ certo se sia rowescia 5 0 drappa 5) pceaim
la feta. 2 ellendogli caduto il pelo, per efler logoro je perché:è senza pelo dice

che € riagiouanito; Sicch¢ in fuftanza vyol dire che: era usato, i lal.
R(CISO, Qui vale per intendere confumato nelle piegature d'un di !

  

   
 
 

    
    
   
 
     
  
 

  

epanno 5, per efiere stato così piegato lungo tempo; che per altro ri
“un Jegho, o altro materiale tagliato ne] mézzo yed ¢ il contrario' div rife se

   
 
 
 

nel oy pela per illungo, Vedi sopra ©. 11, tan, 36, ricife,
ANZA XXXVI. STANZA X&K.
Basta es eve qualcofa un po cattind, Due altre 'armadj poi i fur

Che Celidora ha quint abiti, ¢ panni, Che ? anoe tutto
Che al certo (tuttanolta ch' ella vina )
Puofrancamence andar in lq co gli anni y
Ma perché al [uo char magnonos'arriua E un' altro di pin tr
'Di certe roppe, scampoli, e foppanni + Bealze ye fearpe ye)
Top Wimpaccio vollese a quella gente, Chea vederfi p er
"Ch edt'ha a' intorno,farne un belpreséte, Ve poi'la nidoigi
. STANZA xxxViil st
mui se si parte ed Apre uno Riperto A
2 intagli,e a' arabe/chi ornato,e ricco,
“E trois due cafferse di belletto
Cort! altre di pezrette, ¢ @ orichitco
va il Poeta a narrare glia arnefi,e
hon si parte dallo feh
' faye | Ae a ee

  
  
 
 
 
 
 
 

 
 

n

    
  
  

ME

   

  
 

Si elcr ibs.

a

SERRE ES © =

SSE ST Pesrsr st i ta.

 

DVODECIMO,EDVLTIMOCANTARE: 537

contro alle donne, moftra; che se usano il belletto, ed il liscio, hanno anche
bifogno della medicina da rogna, ¢ del rottorio.

VN po cattiua, Quel po vuol dir poco per la figtira Apocope; ed un poco cat-
tiva 5 trattandofi di abiti, ¢ d' altri materiali, s' intende per lo pit', confumati,
2

vecchi.
TVTT AVOLT A, ch' ella viva, Pub francamente andar in da con eli anm, Pav

che voglia dire, che se Celidora vivera, ha tanti abiti, che le bafteranno molti.

anni senza farsene di nuovo; Ma dall' essere gli abiti della detta qualita, si com-
prende, che scherzando vuol dire, che se Celidora vive, invecchiera, percht
andar in Id con gli anni vuol dire invecchiare, come s' accennd sopra C, 2, stan. 2.

(siginines Ritagli, pezzi di panno, 0 drappo. Scampoli, vedi sopra C. 11;

in. 22.

SOPP:ANNI, Fodere, cioè tele vecchie, che hanno servito per fodere d'abiti.
Scherzando burla la generosita di Celidora, la quale con queste galanti ciarpe,
che fon fondacci d' una bottega di rigatticre, o ferravecchio, regala i (uoi pil:
cari per non apparir meno generosa di Bertincila, che regalo 1a patcona, come
vedemmo sopra:C. 1. stan. 81. a:

D* oronetro: Par che dica d' oro pulito, ¢ puro, ma intende wetto d' oro, cioè
puro; fenz' oro, Equivoco usatifiimo in'quelto propotite,

LA miaferizia per la casa. Incendiamo 11 Cariello', 0 turacciolo del ceffo; es

flo 5 perché un tale detto Galeno, che andava per Firenze vendendo tali cariel-
li, gridava shi vnol la mafferizia per la casa, in vece di dire, chi vuol Carielli; od
¢ra bene inteso.da cutti,

RABESCHI, 0 Arabe/chi, Specie di pittura fatta a fogliami, fiori, masche-
roni., © altro, tutto aggrottelcato, cioc sproporzionato dal naturale, detto co-
si, perché forse tal maniera sia venuta d' Arabia, fecondo che si pud dedurres
da. Cel. Rodig. Jib. 29. ¢. 5. dove trattando delle Lamie, ¢ delic Sircae, dice;
LaAmmiam vero opera parerga ex Arabia maftichen vocant,

SELLETTO. Liscio. Mestura, con la quale si lisciano, ed imbellettano les
donne « Vedi sopra C. 9. stan, 38.

PEZZETTE. Sano pezzi di tela bambagina tinti col cremisi, e zucchero, ed
altre sono di carta fabbricate in Spagna, e se ne servono le femmine per colorirsi
di rosso la faccia.

ORICHICCO. Gomma di Ciriegio, di Pesco, o di Sufino, ec. della quale si
servono le femmine per lustrarsi la faccia, e per appiccarsi veli in fu la teita.

“PER Jambicco. Adagio adagio scaturendo da piccioli fori fatci nel coperchio
del fiaschetto., come s'ufa dei' acque odorifere. Lambicco ¢ il nao della campa~
na,¢ d' ogni cappelio per uso di stillare, donde /ambiceare, ¢ pafsar per lambicco,
# incende stillare; B /ambiccare, 0 lambiccarsi it ceruello, ¢ lo stello che mulmare,
detto sopra C, 10, stan.7.

ALLERA, Pianta nota, le di cui foglie eruono per cauteri; ¢ ¢osi i ceci bian.
chi, li quali per tal effetto erano ia quello (tipo.. Da queite cose vili comprenda
il Lettore, che il Poera si maaticne fempre in fu gli (cherai, deferivendo una Re-
gina, ¢ Palazzo ricchi di quegli addobbi, che fon conuenienti a una beac stant
cOntadina, ¢ decenti alla grandezza d'una Regina di Maimantile,,

% Yyy. STAN,
 

 
  
 
 
 
 
 
 
 
 
 
  

Sh. MALMAN TILE 1980
STANZA XXXIX,.., ith NZ 3
dun caffon diferro vada REREO y c i i co#lor |
L Quiuitvoua il morto, nia dd vero, s -
Che i diamantiye le givie di gram pregro
Lon v'bano che far nidla,e sono un zero;
Lerche si tratta, che vi. Safe un wero
bi perle, che se ben pendeana in nero
Examsi groffe, che st [parfe vace,;
Ch' ell' eran poca manco d' una. noce; Sun i quartrini 5 i precioli ye i bateati,
STANZA XXXX STANZA -XXXXIL
D? anells ya! orecchini Vé1h marame} 'Poi ne venixan gli occhidiciueste;
“Tanti gioie!ls pot, ch' ¢ un fracasso; Ma il proseguir piu olere fa interrortes,
Perc' alla donwa:

Di medaghe dorate 50, vavindi-rame' a
dir, che" Duca levolea far

Vn. moggio ne mifurano,, @ di palo;
Ala quella ¢ sparr ates, ed nn litame Ond? ella il tatto nelcafjon rimette

    
 
  
 

  

    
 
        
      
 
 
 

Risperto alle monere 5 che più baffo E riferrato feende giwdi (orto,
Le piit belle comparfero del mondo; Oue Baldon ? aspercarn iftinali,
Ch! in faseri poses creffi flanvo al fonda: -» one partir di quini fha'im ful? ali >

: STANZA: EMBKI MO vinnd cow 2

Per ¢ agginftare omas tutte le cole 5 In punto, @ questo fine aller
Che pin desiderar non si potea in: ier 'bined
Egli, ch' eva per far come le/pose La puliva.per metterie la fellay

  

LA ritornata s idef? alla Dacea, Licenrioffs costidullaforellay © >
Celidora trova il caffone de'.danari,, ¢ coi tal-occatione i Poera'
monete Fiorentine eficttive, ed immaginarie.. kn tanto che Celidora va vedendo
guefte ricchezze; vien da lei Baldone-suo cugino per liceoziativ) 9
TROFA it morta, Cioé trova il buond. Diciamo rrewar it morta, 0 fare nits
morto, quand' uno trova ripod qualche gran vallente, © fa in gua-
dagno.. A. P
LON 0 ha che far nulla, Par che voglia dire non si stimano, vispette al? altres
Givie, che sono in.quet /uege; ma in eisai vuol dire; che quedo non ¢ luoge per toro
cioè non ve ue fond, i b tone Ses
Sf trata. Si discorre; Termine aflai usato per esprimere una che
s' habbia di qualche cola'; quafi-dica > Si difeorre comunemente, che'
cosh..
AL marame. Voa quantita grandidioma. aferame propriamente.wuol dite ogni
rifiuto di mercanzia, come quella, che dak mare ¢-geteata'a' iva' bi i”
tum, Ma quando diciamo marame nel modo; che! & detto: nel eel
intendiamo abbondanza così grandé.d' una cosa y che generi naulea, €
disprezzabile la medesima cola. Fra i nottci Contadini Gedice
tendefi ? avanzo 5 ¢ rifiuto delle frutte rimatte lord, dopo. la celta', o° vel
delle migliori » noa fo s¢ essi Rroppiano'la nostraparola y o-feonoi Cori
la loro 5 dico bene che mi pare più fighificante; Amaramejehe J
Fiorentino quello 5 che questo 5 che per così dire', ha del Nape
Vedi il Vocabolario della Cru(ca alla voce Cerna',

      
     

y

 
   
  

   
 

DVODECIMO,EDVLTIMOCANTARE 439

UN fracasso. È lo stesso che un flagello, un barbaglio detto sopra C. 7. stan. 5.

UN moggio. Il nostro moggio è di staia 24, lo staio è di libbre 50.\ di grano, e
la nostra libbra è once dodici, Ma qui è detto iperbolico, è significa quantità
grandissima.

RISPETTO a questo, A paragone di questo; cioè a paragone delle monete,
che son più basso.

I pesci grossi stanno al fondo, Detto, che significa: Il meglio sta nel fondo.

PIASTRA, È lo Scudo, o Ducato d'argento Fiorentino, che vale lire sette
ed è moneta effettiva. Il Fiorino è moneta immaginaria, e valeva quando più,
e quando meno, essendoci anche il fiorino d'oro, che forse è quello che habbiamo
ancora hoggi d'oro effettivo, e lo chiamiamo zecchino gigliato, ma il fiorino
ne immaginario, ne effettivo appresso di noi non è più in uso, Scudo d'oro
è moneta immaginaria usata da i Mercanti per facilita di scrittura, valutandolo
lire sette, e mezzo, se ben molti per scudo d'oro intendono la mezza doppia.
La Lira moneta d'argento effettiva, e si chiama Cosimo, e vale dodici crazie.
Il Giulio, che si chiama anche Pavolo è moneta d' argento, e vale otto crazie,
Il Carlino pur d'argento effettivo ne vale sci; ed il Testone val due lire; questa
moneta già in Firenze si chiamò Riccio, dall'impronta della testa del Duca
Alessandro de' Medici, che era ricciuta. La mezza piastra e d' argento effettiva,
e vale lire tre, e mezzo. La crazia è moneta d' argento basso, ed è l'ottava
parte del giulio. Il quattrino è moneta di bronzo effettiva, ed è la quinta parte
della crazia. Il soldo moneta immaginaria che vale tre quattrini; ed il battuto
ne vale due: hoggi l'habbiamo ambedue di bronzo effettive. Il quattrino si divide
in quattro denari di bronzo effettivi, ma hoggi non se ne vedono, se non in
occasione di tributi Ecclesiastici, che sono presentati, e son poi resi, perché gli
possano haver un'altr'anno.

OCCHI di Civetta, intende le monete d'oro, come il doblone, che vale lire
quaranta. La doppia, che vale lire venti. La mezza doppia, che vale lire dieci, Il
quarto di doppia, che vale lire cinque. L' ottavo di doppia, che vale lire due, e
mezzo, che tutte sono d'oro effettive. Habbiamo ancora il zecchino, il quale
chiamiamo gigliato, che vale lire dodici, ed è il più purgato, oro che si conij, e
si può dire il nostro unghero. Si trovano ancora de' dobloni di quattro, e cinque,
e di sei doppie l'uno, di conio Fiorentino.

SPAKTIMENT!, Divifioni, feparamenti. Chiamiamo spartimenti quelle,
divifioni di'tereeno, che Gi fanno ne 1 giardini per piantarui le cipolle da tiori.

ali (partimenti:, se bene sono di diverle figure, si dicono anche quairi. Vedi
pe C,6,-ttan. 63..E per similitudine aiciamo spartimenti te divifioni » che si
trovano ineafiecte, 0 scatole, come crano queiti delle monere,

VENNERO pris hafere. Intendi Avvisi, 0 imbalciace 5 che Staferta appreffo
di noi,¢:1o stesso, che Corriere. Sp. efafera.

\ BAR matte'. Elo stesso che abbaccarli con uno ¢ parlargli, Vedi sopra C,
2, stan. 59.,in altro significato, — > ne

STA sm full aii. EP all' ordine per partirfi. SST

. FAR come le spose. Significa ritornare; lo dichiara il Poeta medesimo,dicendo:
Tdeft 1a ritornata; E quetto perché già coflumavafi, ¢ forse ancora in alcuni Iu

   
 

es:

aS

, ee ee ee

=e 2 SS Sw

PR 6S we

d-
Koyy 1s ghi

 

 
=

 

 
 

s4o
hi@eoRitma, che le si dopo'effere state dicti', 0 pre:
foie rotniao alla casa paceraa's Fer sephe qui git

Teniarns dell Achinea. Taupe lo fallone, 'che* cated
che Achinea, 0 Chinea, intendiamo il cavailo buon rer
éuina specie di cavaili particolare «Sp, bacanea. Franz, bacquenen'y
STANZA XXXXIV, ts “STANZA. Xx
O mai è tempo, cara Celidora,
Ch! inverso li miei [udditi m' apprefi >
Che 'l trattene*mi di vanvargio faora y

  
    
   
 
 
   
    
 
 
 
 
 
 
 
 
 
 
  
 
 

Pregsndicar potrebbe a' miei intereffi Dite, non ci oi fulle corda,

Pero qui refea tu co! tuoi,sn buon bee, Bifog a Lmeteae epee a
E farti anwe, e rispercar da essi y ( Rispsfe il General) 3 ella 8

Ed in ordine a questo i conviene ee ome t

Fare anche un' altra cosa per tuo bene,
STANZA XXXKXV.
Perché, s' io parte ei »cugina mia,
Non fo 2 se tn ci havraituttiitnsigufti,
Che qui non è neffun., che per te sia,
Mentre forsee poi nuowi difeusti,
Ma voglia il Ciel, ch' io dica la bugias

Ed ogni modo vo', che tut' aggiuiti, 'tipo prefto sles of
Per ficurtd con an compatie » Ugquale Vuolotu? parla. a Her =e |

S accafi teco,¢ qucfto, ¢ il Generale, D: mat pitt si, ¢ daccela in fa

STANZA XXKXVL STANZA AKAM;

LT byei hati difender si da vanto, Ed ella nel sentir, cons eit affrin

Che tn vedi,egli ¢branoquarun Marte, A dar pronta: rispotta atal do

E se finor per noi ha fatto tanto, D' un modefto roffor tutta,

Pifa quel che eifara,s'egli entra aparte,

Orsit  daglt la mano; cana [it ilgnanto;

E voi non ve ne fiate pitt in disparte,

Casa Latoni, 0 Amoftante nostro

Fareui innanzi, dite il fatto voftro

STANZA L

Degli dunque la mano in mia prejenzas 3 Ma per non recar tedio

E voi, 0 General, datela a les, Ideft a chi ascolta i versi mitiy

Ch io 'voglio prima della mia partenca

Veder folennizzar questi Flimenci. La[cidgliyadiame;

Baldone da per sposa Celidora al Generale Amoitaate Latoni “ai

dopo haver narrato il discorso fatto da Baldone a paliow per indurlaa

tarsi d' haver questo marito, ed i soliti lezzi donneschi farti da 2

dir di si; paffa a di(correr d' un' altra sposa, che ¢ Psiche, cone ee i

Lagere ouave.

hai neJunsche per te sia « Non hai nefiyno, she aid

  
     
 
 
 
 
   
  
 

 
 
  

a

    
'

DVODECIMO,EDVLTIMOCANTARE. 548

OVVTA. Termine che significa spedizione, © incalzamenio a far prefto. BE' il
Latino Hia,age. Vedi sopra C. 6, fan, go. alla voce, horse, aiegh 3
PASS ATE gud. Venite qua. Lat. ade/dum. B: modo di dire, che significas
comandar con imperio,.¢ con (everita, ed ha del bravatorio. R59,
SE vi piace la pannina, Se vi piace la mereanaia y cio¢ Celidora.
NON ¢i tenete piit in fulla corda. Non ci fate più Aentage,o desiderar la rispo-
fla. Nom cé renere piis coll' animp dubbio, ¢ sospese,:
SON bell' ¢ accordato. lo sono, affatto d' accordo; fon contentissimo. Vedi fo-
pra C. 3. faa, 14, Questo termine bee, '
TERREL d bauerne di beato, Lo riputerei mia gran felicita, Stimerei d' haver
gran forte, WV' avrei di carti, Mi terrei d' etfer beato, ee
EGL1¢ dower sentir  altca campana. E' cosa giulta sentir I altra parte,
TRANA, Questa voce non havrebbe alcun significaco, se bene ¢ assai usata 5
ma perché pace, che immiti il fuono della tromba, quando si da la moffa a i ca~
vali, che corrono al palio; ci serve per esprimer mxovité  /pedi/citi, sbrigati a.
far la tal cosa, Q pure ¢ detto Trana, cioè tra' pur/d tira avanti; dal verbo Tra-
nare, che vale trarre con fatica qualche cosa, ¢ strascinarla.

, ALAL pitt. Quetto termine usato nel modo, che è nella presente Ortava, ci è
familiarissimo, ¢d ha quati lo stesso significato che evvia detto poco sopra, e s'ula
Pua per F altro in occatione di stimolar quaicheduno a spedirfi; ed esprime unas
certa impazzienza di colui, che stimola. E' il Lat. ea tandem. Finiscila,. dille
ana volta,

DAG ELA in fanore. Rispondi fecondo il nostro desiderio, Quando si vince
una lite, si dice haner 1a fentenza in fanore..

CUOKIE con (a ghirlanda. Significa morir vergine. A coloro che muoiono in
¢oncetto di vergini, quando si portano al sepolcro, costumafi di porre in telta
una ghirlanda di fori in fegno della loro castita. Qui il Poeta scherza, come &

, solito farsi, quando si discorre d' una donna impudica, che Gdice Elba giurate
di morir con (a ghirlanda, ¢d & detto ironicamence, ¢ per intendere, e//a vual por=
i tare il vanto ye La corona delle donne impudiche, Ma non per queito il Poeta (che
molto ben si ricorda, che Celidora, per essere flaca moglie del Re di Malmanti-
le, non è pi da ghirlanda, intende, che Celidora fofle impudica, ma dice gosh
per ischerzo, ¢ per segu tare il coftume della plebe, la quale, quand' uao nomi-
t na forella, madre, 0 moglic, fuol dire; purtana di me, ¢ fimui. Se si parla d?
amumogilati fuol dire becco del diavolo, ee. Tal cohtume moitrd il Poeta ancor fo-
praC. 2. stan. 21. dove dicendo: 4 faper quante paia fan tre buoi, foggiugne sfybi-
to Se ben dat padre, ec. ¢ vuole intender padre bue, fecondo lo scherzo fuddetto:
' Non è pero queito stimato offefa, percht avvien fempre detto per ischerzo; ma
4 ricice bene odiolo, ¢ riaferescevole l'eder.u/aco speflo, ed in ogni congiuntura,
y come è usato fra i pil vili, che lo fanno per parer fagaci, ¢ concettof.
¢ Sl riftringe nelle [paile. Cioè 8' accorda, ed accop/ente a quel, che altri dice,
ib
v

aS Sen

© propone. EB' un' arto solito farsi da quelli, che & rimettono, o aderiscono alla
yoloata d' uno, per non poter fare alttuncnti, 0 conuinti dalle ragioni, o indo
ti dalla necedlica, quafi dicano: Parienza; Bifogna frarct. Bocc. Giorn, 2, nov, 8,
' Ada pure nelle [palte rifiretto casi quella dagiar a daee setae mole Mas /ihoorre AnGa,
yar 2. se

 

 
 

   
 
    
   
 
  
   
  

Sate
Eefubetiesaivolen nos si faceia essert “7
volta della testa 5 non dimend dictarho >. re;

0 garbate: O così sta'bene Lat, edge, perphtore belle Te
sue ii contento |, che's' ha», he una cofa  fucceda fecondo chefi desid
APREST Oye male, te cone dafser Meglio'¢ farimale'y¢ pre!
si mai col pensiero “dis volér far benew Chi fa o|,,emale pfiaalineare'
cha facenuy adagio 5 ¢ bene, mainon conchiude, o-termina'quel'cheha
moidi fare, non si puddin che facciay ¢ yeramente nonfa'y e pend nell'c
dei fare ¢ meglio far male, che non fare. '
DATE (a mano Dar ia mado (Latinoviuagere\ dexreras yO la.
nia, che fr faccia negii spontaiizai') ¢ dice impalmare O:far Limpalmamentos,
STANZA Lie oS BAN ZAMLDL oub
Sogwitoical sxb Lvoe gud Phiche bakes,
(olLanSenegee |y. che sn: last frggiafh patra
mand eiskincdrfecon La ging iddaes, y\

 

 
   

  

ee © al dueilo non volle la gatta; Per eat sa i
quefamnalivnara Medesy) >) sion Lagwale
ieplaeaens ere ot mqe) asBe Pe ep

neater (grades ust & Biche trt/ud honor ae s
ones', aldan pian, Ces ie perdadayy 00% ~ nel e es aero ae we }
ia

    
   
 
   
 
   
    

STANZA LIL; oralga on nenaaet
Bit won potends bauer Cupide sposoy 08 90 Pereincomintance/m: ets a4 F
hori: Amardai martha tontana >: \\Bacendo com? il'can delParcolano,, ©)
Ou Revael,s elapher (can iia Ucanegdlah iG 9% (O'all! infatara now P
Che pur veduto sia da corpo humane: E non pao ines eae
Martinazza haveddo prdiilto, che dovea esser fatta imoriré, eiche per Gupi
do non dovea esser piirfucsspolo,, inttidiofa, che.questo'be ne havetioa epodie dd
alteiy: !-haveva incancatoun-udga igacto per impedire: yoche'altrinon havefe
\ EFOGIFA ratta., Boggiva velotemente, Ratto viene dal ae eee
verbio Fiorentino;.Cbrva-pianosa ratto, corri(pondente ai Latiaoy
GING ADEA, Intend laspada, come s'intende: conunenene wb al
deta dail'impugnarficé tutte cinque le dicasete bene itbaftone pure simpugna coeur hur
te cinque de-dita, non si di¢e-cinquadea y pecché questo fipud im}
digch jal che\non Gi — fare delianspads ordinariayy 0fe pur ff
© con difficulta VS RSH As
wuollagstea, a vuolattendete pNomwuol'badarey
Rissmnneiir quel tal:negozio. Hl Berni nell\Orlando y=
« Chey come si fuol dir, voglit la gutta, ~~
OVA Aeden B+ uora lacrudela, che eh Medea si
Oza Re de' Colchi:,»versaril fratello Absyreo- opr )
fo, Glauca sua rivale y¢' co yet th suo ne per 4

  

"ihe 'Vee Rte Sceey cr

jeicodeind Mamie) mateo; A Gatto ata

goD

 
Se peeve fats = &

-

ie
6

|

 

DVODECTMO,ED VLTIMOGANTARE '5432

 

 

 

ne fuan'o\pibes inet faitem a <, be aL a oe
do)0-da quaiche donni at iftra ye wih won; sfiss alist ctloe

TIRA per dado. “Conia aplageresrnoraands pil Bettilenels
la milizia, soldati insieme habbiano commefio qualche delitto 'ca-

pitale, farmorice tn di loro'y,¢ falvar.la:vita a tutti gli alert, facendo loro tiz
rariila-forte ne s€ però 5.1 orcas dettirdadi, ¢ da-credere, che ace
compagnine tal funzione con, fo! i xe con pianti;.¢ fimo però sche il Poetas
digcndo ztiraper-dade y intenda, toipieay © plange pill di cuore che mai; /eguirae
Piangeress pisces gagardamene yes sie pare, she non heaas here aim > 6 sia -
da principio,

“hssan wage. Esser desiderosa d' una tahoe. « Saiwere vago, che vuol dir be
lo, adarne.yec. Sig igiia\ ancora in questo fendi bramefo, ec. Tiraleé — divbes cir
vuol dire: Zi tale genio', ha gusto di betle burle, ¢ feberzi.

HA gid fanoil-pianto. Liha già pianto per perduto. Termine assai usato rims:
Gimili congiunture.. Pianto & quellamento, che si fa-sopra il morto,decgo:così.dal
batterti, per.dolore il petto.» Latino planitus, rodalia = voce Lavina:hanne fat-
ta Gmiimente i Pranzefi la loro Péainte, seh ats eh

eA LZ AR capanne, eo, Cioè quei monti di scope 9 ec. chevsaveno fatei per &b-
biuciar Martinazza come fredetto sopra in quetto G.t.-3» equeftefonove 2o/e
as Fusco, ie quali dices che sshanno a fare per-hongrdi Jet; \cheper altrovyquan-

do diciamo:: s! banno a fare.cose di ey on st afarcofe, bole; wine.

frofe, ¢ fuori del consueto,

FAR come il cane dell' ortolano y Cioè non voleres ° 'non. potere havert uaa cO+
fa 5 ed-impedire y che altri J*habbia, come fa ilcané dell*orcolano'} che nin,
fuangia-! crbaggio y ¢.non vuole che altet lo. Piglt Canis in Prafepi + Provetbio
nlato da bucianoy

eT AN Ze: bItbbs

  

 

 

“iquid 6 of STANZA LV M
tio, \e Bsiche bebbtrogeuife Cos 'byes: affanni ¥¢ le fatiche 0!) Ob
< WDE extte quello eb' è fegiito'ie Corre; » Soffente per rant' anni, ¢ lafri ee

~Gbda il teiogoappinte now si farprecifoy

Risrovatofi, Amore; ed egli, e Priche

AtRena si fainsaprer'tattele porte; » Rappattumato fu dai cavalieri 30
se Amanro crofeiar fenve/i wrgran rife, >: Onde foordats deli"ingiurie amicbe 5 a
atest \obie peg gio; poi fwonsr; ma forte Eriuniti più che volentieri:
& Aeafivmare.di-pefe sr aboccanti 5 + vad regp sposi fero i bactabaffi yoo 9
i obSemea sunosceriehi rece, Contantics' bq Reftando # parte diver foe je (pai
STANZA LV. STANZA shVTR cor 6
“Gir per peniescate ognnn preftoaddirizza y (i Gluntis cialdéni pots e fare i bile,
Che dal timor gli # arricciane è peli. Ml Duca diede al fin 2 ultimo Addie' -
Ma C alagrillo aitiero, ¢-pien di fiicxa “E Jubiro conagni [uo vaffallo
ib oGem talus frrifeia fa colps cradels 5 =: dnnerfa Venano Spiele it-pendio
wi Wa per le stance fende,taglia.e infizza, E Catagrilio ix groppa al-fud <aualld
jlut 444 mon cliappa, se vende' logences cil' © \Preforoon Pficle it Raretrare Dia,)

~ien Rar tde inns ¢ i¢ok fucertibroinranty y.
E il Diavol cacciaye manda vialincato,

o8e\Gupido per opra dij Pakide Aixicrova je per inekzo di-quei Cavalieri'

“3uVO

'¢Aashrei pars); eintefolildor 20
Gb ricomduffe 'ali' Amoroso. as
 

 
   
  

344
con Psiche, si fanno le fefte delio spolalizio di
Jo di loa Bache con iain beioenta ~ dy,
lo accompagaa Psiche se Regno d',
€ROSCLAR an ria. Rider gagliatdamenre- Vedi
GRAV, traboce: Gravi:pil del giufto pelo,
on delle nee ee on se ne feeuc per
¢ seguita 5 chi recé contanti ( che termine proprio
= intender, chi dava se heeds '
e4ADDIKIZZ 4Ciok va via. Fugge per la pitt hota temas
STKISCLA; Intendila spada, come intese (apra:C. 2: st, 60. ° ae
CHIAPPA, Coglig s ritrova, perquote s¢aipilce. Vedi sopra C7.
RAGNATELI, Ragni,piccolt vermis o inferti nati... Vedi sopra.
Le flanze piens di ragnateli tignifica vote dogui.altea fa. Siauimente |
yolendo dire il borficchio voto, dite; Plexys facculius eff arancaram,
RAP PATTV MATL. \ocendiamo rappacificati. Da molti si dice
ge di pace donde: O vincere, o patrare, clo' pareggiare; far pace» ae
gredo venga quetto verbo rappatramare, il quale ¢ atlai usato, mala) 4
da pochi fuori della plebe.::

CIALDONI, Specie di pasta confetta, contorta sottile come l'ostie, ed attorta,
e ridotta come un grosso cannello di canna

STANZA LVIILED VETIMA.,
Finito è il nostro scherzo: hor facciam festa,
Perché la Storia mia non va più avanti,
Sicché da fare adesso altro non resta,
Se non ch'io riverisca gli ascoltanti.
Ond'io perciò cavandomi di testa
Mi v'inchino, e ringrazio tutti quanti;
Stretta la foglia sia, larga la via:
Dite la vostra, ch'i' ho detto la mia.

SCHERZO, Qui vale per trattenimento, Latino lusus, Sogliono i nostri
Contadini, quando fanno le loro veglie di ballo, dopo che hanno un pezzo ballato;
introdurre qualche intermedio, rappresentazione, o giocolamento di forze, o
altro, q questo chiamano lo scherzo, che per lo più finisce in burlar qualche
semplice, e dar'occasione di ridere, e questo tale è poi anche detto lo scherzo, e così
l'intendiamo comunemente, ed il nostro Poeta molto bea l'esprime servendosene
nella sua lettera alla Sereniss. Arciduchessa Claudia d'Austria, riportata sopra
nel Proemio, dicendo: Contentandomi io, che la mia Leggenda, come nata da scherzo,
mi faccia scherzo alle genti.

FATE festa, Cioè siate licenziati, Vedi sopra C. 10, st. 42.


Nota, amorevole Lettore, che il Poeta per terminare la presente sua Oera,
ringraziando con questa ultima Ottava gli uditori, si serve della chiusa inventata,
ed usata dalle donnicciuole, quand' hanno raccontata una novella; cioè
Stretta la foglia sia, larga la via;
Dite la Vostra, ch'io ho detto la mia, 

E conchiude, che ha contata una Novella, come diede intenzione sul principio
di quest'Opera. Ed io pure me ne servo per incitare altri a dir qualcosa
meglio di quello, che habbia fatt'io, non so s'io mi dica nel dichiarare, o pure
confondere, ed intrigare quello che nella presente Opera ho stimato poco
intelligibile fuori della Città di Firenze, e prego il discreto Lettore a compatir
me, che per ubbidire ho pigliato a far' un volo superiore alle mie forze, ed a
contentarsi di biasimar me solo, e non quei, che mi comando, perché habbia
fatto errore nell'elezione. E fo punto.

FINE DEL XILEDVLTIMOCANTARE.
   
 
 
    
   
    
 
   
   
    
   
    
 
 
   

è; MEE LY Big

I Molto Rev, Sig Gio; Domenico
cia di riconoleeté con ogni di
. Opera fouo il Titolo di Malmant
Zipolt, vi sia cov alecuna, che ¢
~ Cattolica, eda' buoni 'Coftumi 4
» Maggio 1686.; =e aoe '

Niccolo Castellam Vic. Cen. Fiorent, dam Ry ia

si
Mluftrifs. e Rev. Sig, g
Ho attentamente letto Oe cor Operetta al
le Racqusftato di Perlone Zipoli, insieme con le fae note
spiegazioni, ¢ per non avervi trovato cosa, ne
alla Santa Fede Cattolica,ed a' buoni costumi,
mano mi soscrivo. Firenze 20, Settem. 1686,
Gro. Domenico del Bruno en Sac, Ti

Attesa la sopraddetta relazione si stampi > osservati gli ordini
soliti, Data z0.Settemb, 1686. > Z;
Niccold Casteliani Vic. Ge

I Molto Rev. Padre Lettore Dolci Minor Otfrvante Conf i
tore del Sant'Vfizio di Firenze legga attentamente la
fente Opera di, Perlone Zipoli,. intitolata Malmantile
quiftato, ¢ ritrovandovi cosa repugnante alla Sat
Cattolica, ¢ buoni costumi, riferisca, Dal 9, Vfizio.
renze 17, Ottobre 1686.::
Fr, Francefio Agoftino Gambaroua Min,
Del S, Vyizso.

Reverendifs. Padre,:
Ho rivista, ¢ ben considerata l'Opera intitolata A
le di Perlone Zipoli, ¢ per non esslervi cosa repu
-aggiunte, stimo possa
D'Ogni Santi li 24. Febbr. 1686, Pe
Fr, Bragio Dolei Ain, Offer. Conf, del S. j

Attenta prefata relazione.
Imprimatur;

Fr. Ces. Pallarvicinus Ordimis Min, Convent, Vie, Ge
S. Off. Florentia.

Ruberto Pandolfini Senat. ¢ Aud. di S, A. S.

 
 

Stel 3% oy rm i

 

BT 2Mh ood ww

LocalWords:  havrei Sinigaglia habbia habbiamo crazia crazie donnicciuole
% LocalWords:  Celidora Bertinella
