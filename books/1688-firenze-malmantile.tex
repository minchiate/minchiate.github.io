\documentclass[12pt,a5paper]{book}
\usepackage[utf8]{inputenc}
\usepackage[T1]{fontenc}
\usepackage[italian]{babel}
\usepackage{changepage}

\usepackage{calc}% http://ctan.org/pkg/calc

\usepackage{etoolbox}
\apptocmd{\thebibliography}{\setlength{\itemsep}{-2pt}}{}{}

\usepackage{tikz}
\usetikzlibrary{decorations.shapes,shapes.geometric}

% avoid orphans and widows, allow for (a lot of) letter spacing.
\usepackage[defaultlines=2,all]{nowidow}
\usepackage[tracking]{microtype}
\sloppy

% how to format and space chapter titles
\usepackage{titlesec}
\titleformat{\chapter}[display]
            {\huge\bfseries\scshape}
            {\vspace{-1.5em}}
            {0pt}
            {}
\titleformat{\section}[display]
            {\large\bfseries}
            {\vspace{-1.5em}}
            {0pt}
            {}
\titleformat{\subsection}[display]
            {\normalfont\fontsize{12}{14}}
            {\vspace{-1em}}
            {0pt}
            {\centering}
% I like this font!
\usepackage{tgbonum}
\renewcommand{\rmdefault}{qbk}
\usepackage{lettrine}
\usepackage[left=13mm,top=11mm,right=13mm,bottom=13mm]{geometry}

\makeatletter
\renewcommand{\@makefntext}[1]{%
  \setlength{\parindent}{0pt}%
  \begin{list}{}{\setlength{\labelwidth}{12pt}%
    \setlength{\leftmargin}{\labelwidth}%
    \setlength{\labelsep}{2pt}%
    \setlength{\itemsep}{0pt}%
    \setlength{\parsep}{0pt}%
    \setlength{\topsep}{0pt}%
    \footnotesize}%
  \item[\@thefnmark\hfil]{#1}% @makefnmark
  \end{list}%
}
\makeatother

\renewcommand{\negthinspace}{\hspace{-3pt}}

\newcommand*\sepline{%
  \kern 3pt \hrule \kern 2pt
}

\title{%
  \kern -2em\fontshape{sc}\normalsize
\textls[180]{\Huge MALMANTILE}\\
\textls[360]{\normalsize RACQVISTATO.}\\\kern 8pt
{\LARGE POEMA}\\
\textls[240]{\Large DI PERLONE ZIPOLI}\\
{\small CON LE NOTE DI PVCCIO LAMONI.}
}

\author{%
{\normalsize DEDICATO}\\
\textls[220]{ALLA GLORIOSA MEMORIA}\\
\textls[-20]{\footnotesize Del Sereniss. e Reverendiss. sig.\ Principe Card.}\\
\textls[320]{\textsc{\Huge LEOPOLDO}}\\
\textsc{\LARGE de' medici}\\
\textsc{e}\\
\textsc{risegnato alla protezione}\\
\textsc{del}\\
\textls[-20]{\footnotesize Sereniss. e Reverendiss. Sig Principe Card.}\\
\textls[40]{\textsc{\Huge FRANC. MARIA}}\\
\textsc{\LARGE nipote di s.a.r.}
}

\date{%
\vfill\scriptsize
\textls[120]{\small\scshape In Firenze}\\
\sepline{}
\textls[-30]{\scriptsize Nella Stamperia di S.A.S.\ alla Condotta.\ 1688.\ \textit{Con lic.\ de Super.}}\\
\textls[120]{E PRIVILEGIO}\\
\textls[60]{Ad istanza di Niccolò Taglini.}
}

\newcommand{\flagverse}[1]{\hspace{-8pt}\makebox[26pt]{\fontshape{sc}\footnotesize \hfill{}#1\hspace{4pt}}}

\newenvironment{signature}{

\kern 1em

\hfill \begin{minipage}{4.5cm}\centering}
{\end{minipage}}

\newenvironment{poesia}{%
  \kern -2em
  \setlength{\parindent}{-1em}%
  \setlength{\parskip}{8pt}%
  \begin{adjustwidth}{5em}{}\-

  }
               {\end{adjustwidth}}

\newcommand{\backspace}{\-\kern -1.5em}
\newcommand{\verseprefix}[1]{\hspace{-5.5em}\makebox[5.25em]{\rm #1\hfill}\hspace{0.25em}}

\renewenvironment{verse}{%
 \itshape\setlength{\parindent}{5.5em}
  \setlength{\parskip}{0pt}
  \obeylines}
               {}
\newenvironment{ottave}{%
  \noindent\hspace{36pt}\begin{minipage}{10cm}\vspace{4pt}
  \setlength{\parindent}{-18pt}
  \setlength{\parskip}{6pt}
}
               {\end{minipage}\\}

\newenvironment{argomento}{%
  \section*{\textsc{Argomento}}\hspace{36pt}\begin{minipage}{10cm}
  \setlength{\parindent}{0pt}
  \setlength{\parskip}{-2pt}
  \obeylines}
               {\end{minipage}\vspace{10pt}}

\newcommand{\ellipsis}[1]{\makebox[#1]{\rule{#1-2pt}{0.5pt}}}

\newcommand{\stanzadash}{\rule[2pt]{54pt}{1pt}}
\newcommand{\markstanzablock}[1]{\item[\stanzadash] \textbf{#1} \stanzadash}
\newcommand{\makestanzalabel}[1]{\textit{\textbf{#1}}}
\renewenvironment{description}
                 {\begin{list}{}{%
                       \setlength{\labelsep}{1em}
                       \setlength{\labelwidth}{0pt}
                       \setlength{\topsep}{0pt}
                       \setlength{\parsep}{4pt}
                       \setlength{\parskip}{0pt}
                       \setlength{\itemsep}{-2pt}
                       \setlength{\leftmargin}{1em}
                       \setlength{\itemindent}{0pt}
                       \let\makelabel=\makestanzalabel}}
                 {\end{list}\setlength{\leftmargin}{0pt}}

\begin{document}
\raggedbottom

\pagenumbering{gobble}
\maketitle
\pagenumbering{roman}

\-

\vspace{6em}

{\centering\Large
{\footnotesize\textsc{al sereniss., e rev.\ sig.\ il sig.\ principe card.}}\\
\textls[44]{\huge FRANCESCO MARIA}\\
\textls[100]{\LARGE DE' MEDICI.}\\
\kern 2em}
Il
Sereniss. e Reverendiss. Principe Cardinale
Leopoldo de' Medici Zio di V.A.R.\ Principe
di quelle rare, ed ammirabili qualità,
che hanno fatto stupire tutto il Mondo,
fino da i più teneri anni dell'A.V.R.\
conobbe, che in lei dovea continuare quello
splendore, che hanno accresciuto alla
sua Sereniss. Casa le stimabili doti di
V.A.R; E per questo, siccome giudicò, che l'A.V.R.\ gli
dovesse succedere nelle virtù, e nella dignità, così volle, che
ella fusse anche erede della sua singolar Libreria. In questa,
havea l'A.S.Rev.\ destinato, che dovesse ottenere il luogo la
presente Opera di Perlone Zipoli, a cui S, A. R, m'onorò
comandarmi, ch'io facessi alcune note, grazia compartitami
(siami lecito il dirlo) forse con qualche scapito del
prudentissimo giudizio di S.A.R.; Ed havendo io ubbidito nella
miglior forma, che havevo saputo, già si pensava alla stampa,
quando i Fati invidiosi tentarono di privarla di così pregiato
onore: e sarebbe loro riuscito, se la somma prudenza di
quel gloriosissimo Principe non havesse a i medesimi impedito
il corso, con prepararle il rimedio nel rifugio alla protezione
di V.A.R.

Se ne vien però il povero Malmantile a' piedi di V.A.R.
umilmente supplicando la sua benignità a volersi degnare di
riceverlo nella sua grazia, e, come erede obbligato;
riverentemente convenendola al Tribunale della sua generosità,
perché gli faccia godere la giustizia, concedendogli il luogo
stabilitogli, acciò egli possa dirsi veramente rifatto dalle
rovine cagionategli da tante sue disgrazie, e da tanti suoi
sinistri avvenimenti: Ed io piglio l'ardire d'accompagnare
queste preci, che egli porge a V.A.R., come quello, che
conosco d'haverlo con la mia penna costituito in grado d'haver
maggiormente bisogno dell'autorevol patrocinio di V.A.Rev.\
alla quale intanto umilissimamente inchinato bacio
ossequiosissimamente la Sacra Porpora.

Di V.A.Rev.

\begin{signature}
Vmilissimo Servidore\\
Puccio Lamoni
\end{signature}

\clearpage
{\centering\Large
{\footnotesize\textit{Al Sereniss.\ Rev.\ Sig.\ il Sig.\ Principe Cardinale}}\\
\textls[44]{\LARGE LEOPOLDO DE' MEDICI}\\
       {\large PADRONE CLEMENTISSIMO.}\\
       {\normalsize PVCCIO LAMONI.}\\
       \kern 2em}

SERENISS. E REVERENDISS. SIG.

MENTRE stavo meditando d'ubbidire a i cenni stimatissimi
di V.A.Rev.\ col far le Note alla presente Leggenda di
Perlone Zipoli, mi cadde sotto l'occhio un sonetto del
Burchiello, nel quale havendo osservato, dove dice:
 Non sunt, non sunt pisces pro Lombardis,
mi saltò il ticchio d'esser' il Lupo nella favola, cioè che questo verso
m'avvertisse, che la faccenda da V.A.Rev.\ impostami non fusse
carne da' miei denti, ond'io havevo già quasi pensato di far conto, che
passasse l'Imperadore: Ma considerando poi, che farebbe stato errore in
gramatica, e da pigliar con le molle, il far'orecchie di mercante a i
riveritissimi comandamenti di V.A.R.\ ho risoluto di non metterla più in
musica, o in sul liuto, ne mandarla d'oggi in domani, dando erba
trastulla, e menando il can per l'aia, ma (venendo a dirittura a i ferri)
non tener più questo cocomero in corpo, e così cavarne cappa, o mantello
più per eseguire gli ordini di chi può comandare a bacchetta, che
perché io resti persuaso d'haver forze sufficienti a portar sí grave soma;
E quantunque io sappia, che havrei fatto molto meglio a lasciar la lingua
al beccaio, perché così havrei sfuggito il farmi dar la quadra, o la
madre d'Orlando, e sonar dietro le padelle da coloro, che si pigliano
gl'impacci del Russo, e ficcando il naso per tutto, fanno poi le Scalee
di S. Ambrogio, come quelli, che havendo mangiato noci, apporrebbono
al sale, senza considerare che ognun può fare della sua pasta
gnocchi, e che [come disse colui, che s'impiccò] ognuno ha i suoi
capricci; tuttavia ho voluto (legando l'asino dov'è piaciuto al padrone)
dare a conoscere che V.A.R.\ non farà, come il Podestà di Sinigaglia;
Se poi ad alcune di questi tali rincresce, mettasi a sedere, e, se non gli
piace, la sputi o mi rincari il fitto; e se dirà, che in fare alla presente
Opera le Note comandatemi, io non habbia preso il panno pel verso,
ma più tosto fatti de' marroni, e pigliato de' granchi a secco, lo lascerò
ragliare; perché son sicuro, che non mi farà baciare il chiavistello, ne
Pigliare il puleggio dalla casa mia; ne mi può accusare di delitto da
farmi mettere in Domo Petri fra i due Apostoli, o da farmi meritare d' esser'
ammazzato con una lancia da pazzo; E se l'indiscretezza di questi tali
mi condannerà per gli errori, che troveranno nelle Note fatte da me, la
mia ignoranza m'assolverà. Non ne ho saputa più: ho soddisfatto al
debito d'ubbidire, e mi quieto col detto di Donatello: Piglia un legno,
e fann'un tu. Mi fara forse detto: Tu porti frasconi a Vallombrosa,
cavoli a Legnaia, ed acqua in mare, e vai contrappelo alla buona
strada a comparire avanti a un Principe così erudito con questi tuoi
scritti; ed io a lettere d'appigionasi, e di scatola, senza saltare in sulla
bica, o entrar nel gabbione, rispondo a costoro, i quali fanno tanto il
Cecco suda, che portano ben loro le mosche in Puglia, e i Coccodrilli
in Egitto, e dandomi il mio resto, hanno trovato il modo d'intisichire,
senza però dirmi cosa, che io non sappia; perché conosco-ancor io il
pane da sassi, la Treggea dalla gragnuola, e le cornacchie dalle cicale; e
sapendo quanto il mio cavallo può correre, sarei venuto di male
gambe, e quasi come la serpe all'incanto, a metter questo cembolo in
colombaia; se non mi fusse noto, che colui, che è avvezzo a mangiar
sempre starne, desidera talora carne di Storno, e non fussi certo, che
la somma prudenza di V. A. R, (conoscendo, che il pruno non produce
limoni, e che dalla botte non esce mai, se non di quello che v'è
dentro, che parimente è impossibile, che il Gufo faccia il verso del
Rusignuolo) non è per isdegnare di ricevere le baie di Perlone Zipoli con
l'abito da villa messo loro in dosso dalla mia zucca, poco atta a
rappresentar l'impresa degli Accademici Intronanti, perché le manca il
Meliora Latent.

Supplico però l'impareggiabile umanità di V.A.R. a voler restar
servita di far conoscere a questi tali, che io ho legato il Cavallo a
buona caviglia, con fare degne queste mie insipidezze d'un benigno suo
sguardo; non perché lo meritino per se stesse, ma perché bensì conviene
alla continuazione di quel generoso aggradimento, col quale si compiacque
ricevere in vita dell'Autore il medesimo Malmantile. Il quale
se con le mie ciarle haverà fortuna di comparire in pubblico, godendo
sí pregiato favore, si potrà dire, nato vestito, ed io cascherò in piè
come i gatti, e mi pioverà il cacio in su i maccheroni: E così con
haver'immitato il cane di Butrione, non havrò timore di coloro, che passano
per la maggiore; perché sapendo essi, che l'Aquile non fanno guerra co'
Ranocchi, sdegneranno abbassarsi tanto con la loro critica, mettendo le
mani in si vil pasta; e quegli Aristarchi, i quali non contano, e non
hanno voce in capitolo, per haver poco di quel che il bue ha troppo, e
che sono come monete stronzate, o come i cavalli di regno; non saranno
causa, che io alzi i mazzi; ne mi faranno venire la muffa, o il moscherino
col loro gracchiare; perché oltre all'essere scritto pe' boccali, che il
Cieco non può giudicare de' colori, si sa ancora, che raglio d'asino
non entrò mai in Cielo, che però conoscend'io, che essi son per fare,
Come colui, che tosa il porco, non gli stimo il cavolo a merenda, e gli
ho dove si da al bossolo da spezzie, e dove si soffiano le noci; Sicché si
possono andar' a riporre a lor posta, e fare un mazzo de' loro salci.  E se
bene dice il proverbio, che la carne di Lodola va a Piacenza a ognuno;
io non mi curo, che me ne sia data, anzi per non mangiarne, son
contento far sempre di nero, purché non mi dieno di bianco questi Correttori
delle stampe, che tiranneggiando le lettere, perché si stimano il
Secento, cercano i fichi in vetta, e 'l nodo in sul giunco. Ma se poi mi
vorranno pure strazziare, io gli assicuro, che e' non hanno a mangiare il
cavolo co' ciechi, quantunque io non sia tanto addietro con l'usanza,
che io voglia mai far credere a haver cattivi vicini, o sia di natura
d'ungermi gli stivali a mia posta. Mi mandino, pure: all'Vccellatoio
quanto a lor piace, e mi facciano anche dietro lima lima, non faranno
però causa, che io faccia come Chele Masi, perché me la farebbono di
figura, e mi scotterrebbe troppo; se bene mi persuado, che ancor'essi
non fussero per uscirne netti; e che fusse per succeder loro il mangiar
noci col mallo, e far come i Pifferi di montagna, poiché, se essi si stimano
piccioni di Gorgona, ed io non son di Valdistrulla; perché sono uscito
di dentini ed ho rasciutto il bellico, e per questo so ancor'io quante
paia fanno tre buoi; onde a dirmi cattivo cattivo, la farà fra Baiante, e
Ferrante, perché io son d'una natura, che non posso ber grosso, e mi so
levar le mosche d'intorno al naso, ne mi morse mai cane, che io non
volessi del suo pelo, massimamente quando m'è saltato il capriccio di
voler la gatta, e badare a bottega, giuocando per la pentola; e s'io me
la son mai legate al dito, o l'ho presa co' denti, n'ho voluto vedere
quanto la canna; perché non mi suol morire la lingua in bocca, ed ho
tagliato lo scilinguagnolo, ne m'è piaciuto mai portar barbazzale, e so
lasciar la squola d'Arpocrate, quando è tempo, ed in particolare con
quei tali che, son più tondi dell'O di Giotto, e che stimando una stessa
cosa il chiacchierare, che il condennare, non sanno portare altre ragioni,
che quel maladetto \textit{non si può}.

Ma perché non paia ch'io saltando di palo in frasca voglia dar panzane
a V.A.R.\ e che questa mia lettera sia il vicolo di mona Sandra, conchiudo,
tornando a bomba, che stimerò d'haver toccato il Ciel col dito,
e tirato diciotto con tre dadi, se potrò conoscere, che l'A.V.R.\ resti
servita di credere, che in questa parte io l'habbia: ubbidita giusta mia
possa, come riverentemente la supplico a degnarsi di far apparire con l'onore
di nuovi suoi comandamenti. Mentre facendo la festa di S. Gimignano
umilissimamente inchinato bacio ossequiosissimamente a V.A.R.\
la Sacra Porpora.

\clearpage
\noindent\textsc{\centering
\textls[180]{\large al cvrioso e discreto lettore}\\
{\large pvccio lamoni.}\\
\kern 0.5em}

La presente Opera di Perlone Zipoli si manda alle stampe, per soddisfare
alla curiosità di molti, che bramosi di pigliarsi il passatempo di leggerla
ne hanno fatta instanza. E perché in alcuni detti, e proverbi usati
in Firenze, de' quali si serve il nostro Autore, possa esser' intesa anche da
color, che lontani dalla nostra Toscana, non hanno la vera cognizione del valore,
e senso di essi, vi ho aggiunto alcune note, con le quali se non ho appieno
soddisfatto, mi basta, che havrò forse data occasione col mio cicalare, che
venga ad altri voglia di meglio discorrere. Tu intanto ricordati, che questa è
una novella; e così ti accomoderai a compatire, se alle volte mi son fatto
lecito di dare qualche spiegazione favolosa. So, che havrai la bontà di sbandir la
censura, e ti tornerà commodo, perché facendo altrimenti havresti troppo da
fare, poche, o forse niuna essendo di quelle cose, che ho scritto, che non la
meritino con un nuovo foglio, e per questo non te ne prego: ti prego bene, se sei
Fiorentino, a legger' il Testo, e non le Note, perché queste non son fatte per te,
che, meglio di quel ch'io habbia scritto, intendi la forza de i detti, che ho
preteso dichiarare,

Dovrei notare gli Autori, a i quali son ricorso per tirare a fine la presente
fatica, ma perché gli bo nominati in tutti quei luoghi, dove è convenuto valermi
della loro autorità, tralascio di farlo; non voglio già tralasciare di confessar
l'obbligo, che queste mie Note, ed io habbiamo all'Eccell.\ e dottissimo Sig.\
Gio.\ Cosimo Villifranchi, ed agli Eruditiss.\ SS.\ Anton Casto, e Sig.\ Francesco
Maria Bellini, i quali m'hanno onorato di più erudite notizie; ed in ultima
attestar la fortuna che hanno havuto questi miei scritti di passar sotto l'occhio
dell'Ecc.\ Sig.\ Abate Anton Maria Salvini\footnote{Anton Maria Salvini, Firenze 1653 - Firenze 1729. Grecista, con Antonio Maria Biscioni, 1674-1756 figura sulla copertina delle edizioni 1731 e 1750 del Malmantile.} il quale non solamente s'è contentato
d'emendar molti miei errori, ma d'ingagliardire ancora le mie debolezze con non
poche sue bellissime erudizioni, a segno, che ha fatto nascere in me una speranza,
che sia per esser ricevuta volentieri questa mia Opera, e d'haver guadagnato
non poco appresso al Mondo letterato, per haver dato occasione a questo dottissimo
huomo d'esercitare la sua stimabilissima penna, i tratti della quale, come
non ho dubbio che nobilmente risplenderanno dentro all'oscurità della mia, così
son certo, che saranno da tutti benissimo ravvisati: Ne confesso però al
medesimo il mio debito, e ne porto al pubblico questa attestazione, perché si sappia
che quello, che sarà riconosciuto per non mio, non è latrocinio, ma regalo
fattomi da questo, e da altri huomini dotti per loro generosità, e per sollevar
Perlone dal discredito, che haveriano fatto meritare a questa sua Opera i miei scritti.\\
Lettore, vivi felice.

\clearpage

{\centering\Large
\textls[244]{\LARGE PROEMIO.}\\
\kern 1em}

Lorenzo Lippi\footnote{Lorenzo Lippi, Firenze 1606 - Firenze 1665, pittore. ``Perlone Zipoli'', poeta, scrittore.} (che in Anagramma nella presente Opera si chiama Perlone
Zipoli ) è stato ne i tempi nostri Pittore non poco celebre, come testificano
molte, e molte sue fatiche. Ciò lo fece meritare d' esser chiamato dalla
Sereniss. Arciduchessa Claudia d'Austria\footnote{Claudia de' Medici, Firenze 1604 - Innsbruck 1648. Reggente del Tirolo dalla morte del secondo marito Leopoldo d'Asburgo nel 1632 alla maggiore età del figlio Ferdinando Carlo nel 1646.} per valersi dell'opera sua a Inspruk,
dove dette principio a questa da lui chiamata Leggenda delle due Regine di
Malmantile, e la dedicò alla medesima Sereniss.\ Arciduchessa Claudia. Haveva però
l'Autore concepita nell'animo suo quest'Opera qualche anno prima, e nel
tempo, che essendo in Villa de' SS, Parigi a S. Romolo nell'andar per quelle campagne
a diporto, vedde le muraglie di Malmantile; ed haveva discorso questo
suo pensiero col sig.\ Filippo Baldinucci\footnote{Filippo Baldinucci, Firenze 1624 - Firenze 1696. Storico dell'arte, politico e pittore, ``Baldino Filippucci''.}, dal quale poi nel tessimento del Poema
hebbe, come da persona erudita ( che tale lo dichiara la sua bell'Opera mandata
da esso alla luce intitolata Notizie de i Professori del disegno) non piccolo aiuto
in proposito della lingua, e d'altro, e particolarmente nei descrivere il Consiglio
de i Diavoli nel Canto sesto.

Tal composizione fece egli a solo fine di mettere in rima alcune novelle, le
quali dalle donnicciuole sono per divertimento raccontate a i bambini, e di sfogare
la sua bizzarra fantasia, inserendovi una gran quantità di nostri proverbi, ed
una mano di detti, e Fiorentinismi più usati ne i discorsi famigliari, sforzandosi di
parlare, se non al tutto Bocaccevole, almeno in quella maniera, che si costuma
oggi in Firenze dalle persone Civili, ed ha sfuggito per quanto ha potuto quelle
parole rancide, alle quali vanno incontro tal'uni, che per spacciarsi huomini
letterati, non sanno fare un discorso, se non vi mettono, guari, chente, e simili
parole, che per essere state usate dal Boccaccio\footnote{Giovanni Boccaccio, Certaldo 1313 - Certaldo 1375.}, essi credono, che dieno l'intero
condimento alli loro insipidi ragionamenti, e stimano, che quello sia il vero parlar
Fiorentino, che non è inteso, se non da i lor pari, e non s'accorgono, che
in tal guisa parlando, si rendono scherzo di chiunque gli sente, come bene attesta
questa verità il Lasca\footnote{Anton Francesco Grazzini detto il Lasca, Firenze 1505 - Firenze 1584} in quel suo Sonetto sopra l'Opere del Berni\footnote{Francesco Berni, Lamporecchio 1497 - Firenze 1535. ``che dice le cose sue semplicemente, e non affetta il favellar toscano''.}, dicendo:
\begin{verse}
\backspace Non offende gli orecchi della gente
Con le lascivie del parlar Toscano,
Vaquanco, guari, mai sempre, e sovente
\end{verse}
Ed Antonio Abbati\footnote{Antonio Abati, Gubbio inizio secolo XVII - Senigallia 1667} dice
\begin{verse}
\backspace Peggio non ho, che quel sentir parlare
Con tanti quinci,e quindi, e, ec.
\end{verse}
Anzi in questa parte l'unica intenzione del nostro Poeta è stata di far conoscere
la facilità, e pienezza del parlar nostro, e \textit{Cogliendo della lingua materna il più
bel fiore}, mostrare, che ancora ad uno, che non ha (come'appunto, era egli)
altra eloquenza, o poca più di quella, che gli dettò la natura, non è impossibile
il parlar bene. Questo, ed altri fini dell'Autore s'argumentano dalla seguente
Dedicatoria, che egli stesso scrisse alla Sereniss.\ Arciduchessa Claudia, la quale
lettera io pongo qui per confonder coloro, che pur vorrebbono fargli dire quel
che mai il nostro Poeta hebbe in pensiero.

\begin{adjustwidth}{1.5em}{}
  \itshape
Ati figliolo di Creso Re di Libia (se è vero, che io non ne so più la, e la vendo,
come io l'ho compra) vedendo il padre in pericolo, isso fatto cavò fuora
il limbello, e disse le sue sillabe, come un Tullio; Tutto il rovescio dovrebbe
fare il pesce pastinaca senza capo, e senza coda della mia Leggenda a mal tempo,
ch'io mando a V.A.S.\ perché vedendo ella quel dolce intingolo di quel
fantoccio di suo padre in procinto d'esser mandato all'Vccellatoio, e quasi ridotto
alla porta co' saffi, e che gli sien suonate dietro le padelle, anzi fra il
tocca, e non tocca di scior Pallino, potrebbe a sua posta far' un mizzo de' suoi
salci, e farsi ricucire la bocca per non haver più occasione di formar verbo.

Ma perché si compiace V.A.S.\ di volerne una secchiatina, benché questa mia
Leggenda non fusse degna di fiutare eziam i luoghi privati, verrà di gala col suo
ricadioso cicaleccio, che si strascica dietro una gerla di farfalloni, a farne una
stampita anche ne i Palazzi reali, perché ella è una prosontuosina da darle del
Voi; Ond'io conoscendo nella temerità di essa l'ubbidienza dovuta de iure a i
riveriti suoi cenni, gli è giuoco forza, voglia il mondo, o no, che ella si metta
giù a bottega a sfogare la fisima de' suoi fantastichi ghiribizzi, contentandomi
io, che ella, come nata da scherzo, mi faccia scherzo alle genti. Compatisca
dunque l'A.V.S.\ questa sconciatura partorita nel tempo, che io do
festa a i pennelli, mentr'ella non apprezzando un'ette gli applausi volgari, riceverà
per grazia sterminata, e per arcisbardellatissimo favore, se queste baie
riusciranno di qualche valezzo nel cospetto di V.A.S.\ alla quale profondamente
inchinandomi, con ogni debita rivereaza bacio la Veste.
\end{adjustwidth}

Da questa lettera adunque si viene in non piccola cognizione de i sentimenti
dell'Autore nel comporre la presente Opera; La quale fu da esso presso che
terminata in Inspruch, e dedicata come ho detto alla Sereniss.\ Arciduchessa
Claudia; Ma essendo S.A.S.\ in quei medesimi tempi passata all'altra vita,
convenne all'Autore tornare alla Patria, dove fu questa sua Novella veduta da diversi
amici suoi, fra i quali dal sig.\ Romolo Bertini Servidore del Sereniss Principe
Cardinale Leopoldo de' Medici\footnote{Leopoldo de' Medici, Firenze 1617 - Firenze 1675, cardinale dal 1668.}, e molto accetto per l'ottime sue qualità, virtù,
e dottrina, e da esso hebbe S.A.R.\ la prima notizia della presente Opera, e fino
da allora mostrò l'A.S.R.\ non piccola inclinazione, che si pubblicasse, e se
tralasciò di comandarne la stampa, fu, perché sentì dal medesimo Bertini, che
l'Autore pensava d'accrescerla.

Fu veduta ancora dal sig.\ Francesco Rovai\footnote{Francesco Rovai, 1605-1647. ``Franco Vicerosa''}, e dal sig.\ Antonio Malatesti\footnote{Antonio Malatesti, Firenze 1610 - Firenze 1672. ``Amostante Latoni''.};
ambi Poeti nel lor genere Eccellentitfimi, dal sig.\ Salvador Rosa\footnote{Salvator Rosa, Napoli 1615 - Roma 1673. ``Salvo Rosata''} non men celebre
nella Poesia, che nella pittura, e dal quale il Lippi hebbe notizia Dello Cunto
de li Cunti\footnote{Pubblicato da Adriana Basile fra gli anni 1634-1636.} di Gianalesio Abbattutis\footnote{Giovan Battista Basile, Giugliano di Napoli 1566 - Giugliano 1632.}, di dove l'Autore cavò poi alcune novelle,
che si trovano in quest'Opera: La quale in somma fu veduta da molt'altri eruditi
ingegni; e fu il Lippi da essi consigliato, e poco meno, che forzato a metterla
alla stampa, con persuaderlo, che meritava la pubblicazione: ma ricusò egli
sempre di far tal passo, conoscendo molto bene, che colui, che stampa l'Opere
sue, s'espone ad un certissimo pericolo, per una incerta gloria, e massime nel
presente secolo, che vi è maggiore abbondanza di spropositati, e mordaci Satirici,
quali con invidioso livore lacerano le fatiche altrui, che di Censori discreti, i
quali con dotti avvertimenti n'emendino gli errori.

Dalle grandi instanze fattegli dagli amici suddetti, che egli stampasse questa
sua Novella, insospettito il Lippi, che il libro di detta sua composizione non gli
fusse levato, e contro a sua voglia stampato, andava molto circospetto, non lo
lasciando in luogo, dove fusse sottoposto a tal caso; Ma essendo una volta andato
in villa de' SS. Susini suoi cognati, e di quivi alla villa del sig.\ Don Antonio de'
Medici\footnote{forse Anton Francesco de' Medici, 1618-1659, frate dell'ordine dei Cappuccini}; dove havendo portato il detto libro per passare, leggendolo, la veglia,
la notte, mentre egli durmiva, il sig.\ Piovano Gualfreducci, ed il sig.\ Tommaso
Fioretti con l'assistenza del medesimo sig.\ D. Antonio sciolsero il detto libro, e
fra tutte due lo copiarono e la mattina lo rilegarono, e lo raccomodarono in
maniera, che egli non s'accorse del virtuoso furto. Questa copia capitò poi in
mano a Paolo Minucci\footnote{Paolo Minucci, Firenze 1606 - Radda 1695. ``Puccio Lamoni''}, il quale facendo al Lippi la solita instanza di metterlo alla
stampa, ed egli ricusando, gli disse il Minucci, che l'havrebbe egli fatto stampare;
e replicando il Lippi, che se ne contentava, se vi era modo, il Minucci
col mostrargli la detta copia scoperse il furto, e fece conoscere la possibilità, che
havea di farlo stampare, S'alterò non poco il Lippi veduto questo, ma come
huommo virtuoso, ed onorato volle, che la vendetta di tal disgusto fusse il costituire
il Minucci, ed ogni altro in grado di non si curar più di stampar quell'Opera;
questo fu con aggiugner'ad essa alcuni episodj, ed altro, in maniera, che in
breve tempo la ridufle da fette piccoli canti, che ell' era, alli dodici, che è la
presente; e perché non gli avvenisse di questa, come gli era accaduto della prima
teneva l'originale di essa in modo riserrato, e ristretto, che non lasciava vederlo
ne meno all'aria, e poco altro poteva haversene, che sentirne recitar da lui
qualche Ortava alla spezzata, ed il Minucci più d'ogni altro haveva questo favore
da lui, perché col fargli sentire l'augumento, che dava a quest Opera, stimava
di fare scemare nel Minucci la volontà di stamparla, e conseguir l'intento,
che s'era prefisso, ma ne seguì tutto il contrario, perché havendo il Minucci
sparso fra gli amici, che il Lippi riduceva la sua Opera in stato ragguardevole,
pervenne questa notizia all'orecchie del Sereniss.\ sig.\ Principe Card.\ Carlo de' Medici\footnote{Carlo de' Medici, 1595-1666.}
Decano del Sa.\ Collegio, e S.A.R.\ curiosa di veder quest'Opera comandò
al Minucci, che operasse d'appagare tal sua curiosità. Il Minucci manifestati al
Lippi i sentimenti dell'A.S.R.\ esortò a non contraddire di ricever l'onore
che S.A.R\ gustava di fargli; ed egli conoscendo, che mal poteva negare d'ubbidire
a tanto Principe, per il quale (come fratello della Sereniss, Arciduchessa.
Claudia) riteneva congiunto al debito di suddito un genio non ordinario di servirlo,
e persuafo pure una volta; che il pubblicar detta Opera non gli poteva
apportar se non lode, condescese a lasciarne pigliar copia per S.A.R.\ la quale si
piacque di dar dimostrazione del suo benigno aggradimento con atti non piccoli
della sua solita generosità, e verso il Lippi, e verso il Minucci, che ne fece
la copia, perché così volle il Lippi, o per spaventar il Minucci con la gran macchina,
che appariva, e così levarlo dal pensiero di pigliarsi questa fatica, ed
addormentare intanto nel sig.\ Principe Card.\ la volontà d'haverlo (come disse il
medesimo Lippi) o pure, perché quella copia non capitasse in mano ad altri, che
del medesimo Minucci, del quale si fidava, e per sua bontà, e perché haveva
anche veduto, che di quella copia, che teneva detto Minucci della prima Opera,
non s'era mai saputo cosa alcuna, perché esso Minucci l'haveva sempre occulata,
e negata a ognuno d'haverla, Ma quel'ultima copia sendo in mano del
detto Sereniss.\ sig.\ Card.\ Decano, accrebbe nei SS.\ suoi Cortigiani la curiosità
d'haverla, e cosè per diverse vie ne trassero una copia. Da questa poi se ne sono
sparse infinite; ma perché l'Autore sopravvisse qualche poco di tempo, e sempre
accrebbe, o moderò qualcosa, ed oltre a questo, perché la poca avvertenza di
coloro, che hanno copiato, ha causato, che si trovino molte copie, e difettose,
o guafte, il Minucci riputandosi in un certo modo cagione di questo disordine risolvette
per rimediarvi, di supplicare il Sereniss.\ Principe Leopoldo (allora non
Cardinale, al quale dall'Autore stesso fu quest'Opera dedicata, dopo la morte
della Sereniss.\ Arciduchessa Claudia) di permettergli il mandare la detta Opera
alla stampa, per rinnovare la memoria de] già defunto Lippi\footnote{Siamo quindi fra il 1665 ed il 1668.}, e S.A.\ glielo
concedette, con obbligo però, che gli facesse alcune Note, ed esplicazioni; E così
contento l'universale, che desiderava tal pubblicazione, e diede al Minucci il
gastigo d'esscre stato causa del suddetto disordine, ed al Lippi la soddisfazione\footnote{postuma}
dovutagli dal Minucci per la violenza fattagli, con obbligare il medesimo Minucci
a sottoporre ancor'egli i suoi scritti a quei danni, che dalle stampe ne
risultano; Sentenza veramente giusta, come appoggiata al fondamento della pena del
Taglione, ma troppo severa nell'arbitrio per la gran disparità, che è fra la vaga
Opera del Lippi, e l'insipide chiacchiere del Minucci, sopr'alle quali, e non
sopra gli scritti del Lippi si fermeranno, e poseranno tutti gli Aristarchi; con
tutto questo non ha il Minucci voluto intentare appello, anzi, sendosi accinto
subito a dare esecuzione alla sentenza, ha aggiunto all'Opera le Note comandate,
con le quali ha egli preteso d'operare, che fuori di Firenze, e della nostra
Toscana, e per tutta Italia possano esser meglio intese molte parole, detti, frasi,
e proverbj, che si trovano nell'Opera, forse non intesi del tutto altrove, che in
Firenze; e prega il Lettore a compatire, se non sia da esso soddisfarto appieno, e
ricordargli, che non è stata mente del Minucci il portare l'etimoiogia delle parole,
frasi, e proverbj, ma d'esplicargli in maniera, che possano esser'intesi anche
fuori di Firenze, ed habbia il medesimo Lettore la discretezza di riflettere, che
molti Fiorentinismi sono in uso, nati dal puro caso, senza un minimo
fondamento, o ragione, perché si dicano, e che;
\begin{verse}
Non omnium, quae a maioribus nostris scripta, aut dicta sunt, ratio reddi potest.\footnote{Adattato da Tommaso d'Aquino, Summa Theologiae, Q.\ 95, Art.\ 2. ``Sed non omnium quae a maioribus lege statuta sunt, ratio reddi potest, ut iurisperitus dicit.''}
\end{verse}

\clearpage
\noindent\textsc{\centering\large
\textls[240]{\Huge MALMANTILE}\\
{\small DISFATTO}\\\kern 6pt
\textls[360]{\LARGE ENIGMA}\\\kern 4pt
{\normalsize DEL SIG.\ ANTONIO MALATESTI.}\\
\kern 1em}

\begin{poesia}
Ov'è l'Etruria indomita, e infeconda,\\
Già fui per molti figli e ricco, e bello,\\
Or c'una fascia a pena mi circonda,\\
Povero, brutto, e vil non son più quello.

M'hanno gli amici più che 'l vento, e l'onde\\
Levate l'ossa, e toltomi il cappello,\\
E fino il nome par che corrisponda;\\
Una mala tovaglia, o un mal mantello.

Così ridotto trovomi a mal porto,\\
Col corpo voto, e senz'un membro intero,\\
E pur con tuttociò non mi sconforto;

Anzi ora godo, e farmi eterno spero,\\
Mentre in Flora un' Augel per suo diporto,\\
Cantando in burla, mi rifà da vero.
\end{poesia}

\chapter{Primo Cantare}
\pagenumbering{arabic}

PRIMO Cantare. Ecco che il nostro Poeta mantiene l'intenzione
data di pubblicare una Leggenda,e non un Poema, mentre mette
sopra ogni Canto l'inscrizione, che si vede in diverse leggende
dove in vece di dire Canto 1., e Canto 2, ec. come usano nei
Poemi Italiani, egli dice Primo Cantare, e così seguita fino all'ultimo,
volendo per la sua modestia esser chiamato Compositore
di Leggende, non Autore di Poemi, ed in uno stesso tempo
con bell'arte difendersi dalle censure di chi lo tacciasse di non aver'osservate le
regole del comporre i Poemi, sapendosi, che a queste non sono sottoposti i
compositori di Leggende.

\begin{argomento}
Marte sdegnato perché il Mondo è in pace
Corre, e da letto fa levar la suora,
E in finto aspetto, e con parlar mendace
Mandala a svegliar l'ire in Celidora,
Fa la mostra de' suoi Baldone andare
Indi all imbarco non frappon dimora,
E per via narra con che modo indegno
a occupate avea il suo Regno.
\end{argomento}

Gli Argomenti a tutti li Canti di quest'Opera sono di Amostante Latoni, cioè
Antonio Malatesti, fatti di comandamento del Sereniss.\ Principe Cardin.\
Leopoldo de' Medici.

\section{Stanza I}

\begin{ottave}
\flagverse{1}Canto lo stocco, e 'l batticul di maglia,\\
Onde Baldon sotto guerriero arnese,\\
Movendo a Malmantil' aspra battaglia\\
Fece prove da scrivern' al paese,\\
Per chiarir Bertinella, e la canaglia\\
Che fu seco al delitto in crimen lesa\\
Del far' a Celidora sua cugina,\\
Per cansarla del Regno, una pedina,
\end{ottave}

Mostra l'Autore in questa sua introduzione, che egli vuol descriver da Guerra
fatta da Baldone in aiuto, e difesa di Celidora, e vuol persuadere, che se ben
dice \textit{aspra battaglia} fu una guerra di nulla, e però seguita: \textit{fece prove da
scrivern'al paese}, del qual detto ci serviamo per derisione, quando altri ha fatta
una azione da lui stimata grande, e bella, che in effetto non è poi tale, anzi è
tutta il contrario, e si dice:  \textit{Hai fatto assai, scrivi al paese}.

\begin{description}
\item[BATTICVLO di maglia] Intende il Giaco, arme difensiva di dosso, cioè una
camiciuola composta di maglie di ferro, ed è la lorica ansulata, che usavano gli
antichi. E se bene \textit{batticulo di maglia} non è veramente buon Fiorentino, nondimeno
è spesso usato, ma per giuoco, ed è comunemente inteso per il Giaco, e si dice
così, perché coprendo quest'arme le parti di dietro, nel moto che fa colui, che
l'ha in dosso, batte in quella parte; come si dice Picchiapetto quel Gioiello, che le
donne usano portare al collo pendente sul petto.

\item[MALMANTILE] E' un Castello antico vicino a Firenze circa dieci miglia,
  oggi del tutto rovinato, e distrutto, ne vi si vede altro che lé muraglie Castellane.

\item[CHIARIRE] Questo verbo, che oltre a gli altri significati, vuol dire Far conoscere
  l'errore, o Render capace; nel presente luogo vuol dice Scaponire, o
  Sgarire: \textit{Il tale mi faceva l'huomo addosso, gli ho dato una buona quantità di pugna, e l'ho
chiarito}; cioè con questo l'ho reso capace, e fattogli conoscere la stima, che io fo
  di lui, e quella che egli deve far di me. Questo verbo è traslato dal verbo Chiarire,
  che è Purificare ogni liquore torbido, e contaminato da materie crasse.

\item[CANAGLIA] Gente vile, ed abietta, che tali saranno, come vedremo, i soldati
  di Bertinella, i quali il Poeta mette Huomini d'infima plebe, che Cicerone
  chiama Imi subsellij homines. Il Sig\ Francesco Maria Bellini in alcune sue bellissime
  reflessioni, che si è contentato fare sopr'alla prsente Opera, ponderando la
  parola Canaglia dice, che l'allungamento delle parole in \textit{aglia} sta Oggi in
  Toscana un certo avvilimento, e disprezzo del subietto, e s'usi solo in cose vili, e
  plebee, e però si dica de' Birri sbirraglia; della Plebe. Plebaglia, e gentaglia; de i
Fanciulli, e popolo infimo Spruzaglia, (metaforico da spruzolo, acqua minuta)
e che questo sia antichissimo Latino, sia di neutro plurale, del quale si servirono i
Latini per comprender l'appartenenze della cosa, della quale parlavano, v.g.\
delle cose appartenenti alle navi dicevono Navalia; alla Cacina Popinalia, e molt'altri,
è corrotto da noi con l'aggiunta della lettera G.

\item[IN crimen lesa] È delitto di lesa Maestà cacciare una Regina del
suo Regno.

\item[FAR' una pedina] Si dice Fare una pedina a uno allora che procurando questo
  tale di conseguire cosa di suo gusto, ed essendo vicino a ottenerla, un'altro, a
  cui haveva confidato tal negozio; gliela leva su. Viene dal giuoco di Scacchi, dicendosi
  propriamente: Dare scacco di pedina.

In oltre, chi è pratico del giuoco di Scacchi sa, che quando s'è perduta la
Regina, si procura di racquistarla con far' arrivare una pedina al posto dove
stava la Regina dell'avversario al principio del giuoco, e così intendere, che Celidora
priva del Regno conveniva, che sotto nome di Pedina tornasse a ricuperarlo,
se voleva esser detta Regina.

Si potrebbe anche dire, che il nostro Poeta seguitando il costume che habbiamo
di chiamar Dame le Signore grandi, e Pedine le donne d'infima plebe, habbia inteso,
che Bertinella, togliendo il Regno a Celidora, l'habbia cavata del nome di
Dama, per haverla ridotta in grado miserabile, le habbia fatto meritare il nome
di Pedina; ma l'esser' il nome, di Celidora nel terzo caso, e non nel secondo, o
nel quarto; fa languire questa riflessione.
\end{description}

\section{Stanza II}
\begin{ottave}
\flagverse{2}O Musa, che ti metti al sol di state\\
Sopr' un palo a cantar con si gran lena, \\
Che d'ogn'intorno assordi le brigate,\\
E finalmenre scappi per a schiena;\\
S'anch'io sopr'alle picche dell'armate\\
Volto a Febo con te venga in iscena,\\
Acciò ch'io possa correr questa Lancia,\\
Dammi la voce, e grattami la pancia.
\end{ottave}

Quest'Ottava ha poco bisogno di spiegazione vedendosi chiaro, che il Poeta,
invoca per sua Musa la Cicala, e così dà a conoscere, che egli vuole scrivere affatto
mostrando, che per fare una composizione come egli ha in animo,
e per descrivere una guerra qual fu quella di Malmantile, gli basta haver
chiacchiere.

Si potrebbe anche dire, che il Poeta sapendo che non si trova, che le Muse habbiano
dato mai alcuno aiuto effettivo, ed evidente, come dette la Cicala a Eunomo
Locrense Suonatore nella disputa, che hebbe con Aristono, supplendo con
la voce al mancamento della corda strappata, come si legge in Strabone lib. 6.
voglia, come fece Eunomo, far più capitale della Cicala, che d'altre Muse:
E può anch'essere, che egli invochi la Cicala, perché stimi più nobili delle Muse le
Cicale per esser queste più riguardevoli, come nate avanti alle Muse (secondo la
favolosa credulità de' Gentili) d'Huomini, li quali per lo gran gusto, che hebbero
del cantare, furono in cicale convertiti, come si cava da Celio Rodigino lib.\
17.\ cap.\ 6.\ le cui parole sono queste: \textit{Fertur enim hosce homines fuisse ante Musas;
natis deinde Musis, cantumque monstrato, illorum nomnullos voluptare cantus usque adeo
delinitos fuisse, ut canentes cibum, potumque negligerent, imprudenterque perirent; ex
quibus deinde cicadarum genuss sit propagatum, ec,}

Dice il Doni nella sua Zucca, che tutti li Poeti hanno la loro Cicala, e che
questa serva loro per Fama publicando le loro Poesie, onde il nostro Poeta seguitando
l'opinione del Doni invoca la Cicala destinata al suo servizio, perché gli
faccia questo di pubblicare le sue Poesie.

\begin{description}
\item[PALO] Pertica, Bastone di legno, che si mette per sostegno alle viti, ed altri
  arbuscelli simili.

\item[LENA] Significa quello, che i Latini dicono \textit{respiratio}, cioè quieto, e
  tranquillo
  anelito, il che mentre è nell'Huomo, egli si mantiene senza difficultà, nelle
  forze: ma la troppa fatica di corpo, o di mente spesso fa affannare tal Lena,
  però che uno, che s'eserciti assai senza posarsi, appunto come fa la Cicala col
  suo cantare senza riposo, si dice Haver gran Lena.

  Dante Inf.\ C.\ 1.\ \begin{verse}E come quel che con lena affannata, ec.\end{verse}

  Al Canto 24.\ \begin{verse}La Lena m'era dal polmon si sì smunta, ec.\end{verse}

  Vedi sotto C.\ 4.\ stanza 6.

  Varchi\footnote{Benedetto Varchi (Firenze, 19 marzo 1503 – Firenze, 18 dicembre 1565), umanista, scrittore e storico.}  stor.\ lib.\ 5. \begin{verse}Essendo egli di pochissimo spirito,
    e di gentilissima Lena\end{verse}

  Franco Sacc.\ Nov.\ 127.  \begin{verse}Alla fine perdendo questi ciechi
      la Lena per essersi molto bene mazzicati, ec.\end{verse}

  I Latini con la voce \textit{Vis}, e con la voce \textit{robur} esprimevano questa Lena.

\item[VENIRE in scena] Comparire in pubblico,  vedi sotto C.\ 4.\ stan.\ 6.

\item[CORRER questa lancia] Tirar' a fine quest'Opera.

\item[GRATTAMI la pancia] Col grattare il corpo alla Cicala, ti fa che ella canti,
  la Cicala a grattare il corpo a lui, acciò che'egli canti. Quand'altri
  sa qualcosa, ed è duro a manifestarla, si dice; \textit{Grattagli la pancia, che egli
    canterà},
  cioè interrogalo, ed esaminalo bene, che egli dirà tutto quello, che tu
vuoi; sì che il senso di questo detto \textit{Grattare il corpo a uno}, è Incitarlo a discorrere.
Vedi sotto C.\ 2.\ stan.\ 8.

\end{description}

\section{Stanza III \& IV}

\begin{ottave}
\flagverse{3}Alcun forse dirà ch'io non so cica, \\
E ch'io farei 'l meglio a starmi zitto, \\
Suo danno; innanezi pur, chi vuol dir dica, \\
Fo io per questo qualche gran delitto? \\
S'io dirò male, il Ciel la benedica; \\
A chi non piace, mi rincari il fitto: \\
Non so, se se la sanno questi sciocchi,\\
Ch'ognun può far della sua pasta gnocchi.

\flagverse{4}Mi basta sol che Vostra Altezza accetta\\
D'onorarmi d'udir questa mia storia\\
Scritta così come la penna getta,\\
Per fuggir l'ozio, e non per cercar gloria;\\
Se non le gusta, quando l'avrà letta\\
Tornerà bene il farne una baldoria:\\
Che le daranno almen qualche diletto\\
Le Monachine, quando vanno a letto.
\end{ottave}

In queste due Ottave l'Autore piglia a difender se medesimo dalle male lingue,
e mostra, che poco gl'importa l'esser lodato, o biasimato in questa sua Opera, e
che, non essendo obbligato a veruno, vuol soddisfare a se medesimo, ed al suo capriccio;
e però dice: \textit{S'io dirò male il Ciel la benedica}, che significa Vadia il negozio,
come e' vuole, che non m'importa. E seguita \textit{A chi non piace mi rincari il
  fitto}, volendo mostrare, che per non essere obbligato a render conto ad alcuno delle
sue azioni, non teme d'esser ripreso, o di ricever danno; e soggiugne: \textit{Ognun
  può far della sua pasta gnocchi}, cioè ogni huomo libero puo fare del suo, a suo modo.
Conchiude in somma, che egli vuol dar gusto a se medesimo, e lasciar dire
chi vuol dire, bastandogli, che S.A., cioè il Sereniss.\ Principe Card.\ Leopoldo de'
Medici, a cui dedica l'Opera, si contenti di riceverla, e d'udirla, \textit{scritta come
  la penna getta}, cioè composta non ad altro fine, che di spassarsi; ne si cura
d'acquistar gloria per tal composizione, anzi supplica S.A.\ ad abbruciarla quando
l'haverà letta, che riceverà qualche gusto dal veder' \textit{andare a letto le Monachine}. E
per Monachine intende quello, che intendono i nostri Fanciullini, cioè quelle piccole
scintille, che, nell'incenerirsi la carta, a poco a poco si spengono, e facendo
un certo moto, pare che si dileguino, sembrando tante Monache, le quali col
loro lume in mano scorrano per il dormentorio, andando a letto.

\begin{description}
\item[CICA] Niente. Anzi vuoi dire (se si può) Manco di niente, dicendosi in
  diminuzione \textit{Poco, niente, Cica}. Viene dal latino \textit{Cicum}, che vuol dir Quel velo,
  che si trova nelle melagrane per divisione de' suoi granelli, che per esser così sottile,
  e di niun valore, serviva ai Latuini per dimostrare la poca stima, che facevano
  d'una cola, dicendo: \textit{Ne Cicum quidem dederim}, ec. e noi diciamo in questo
  proposito \textit{lappola, lisca, ec.}
\item[ZITTO] Quieto. \textit{Stare zitto} vuol dire Non parlare, Viene dal cenno. \textit{Zi},
che si suol fare, quando senza parlare si vuol fare intendere a uno, o più, che
quietino, come facevano ancora i Latini, che per accennare ad altri, che si
quietasse profferivano le due consonanti S.T.

\item[GNOCCO] È una specie di Pane gramolato, mescolato con anici; e questa
pasta fra le nobili è la più vile: Il proverbio \textit{Ognun può far della sua pasta gnocchi}
significa ognuno ha il libero arbitrio, ed esprime quello, che i Latini dissero:
\textit{Unusquisque in re sua moderator, \& arbiter, ec}.
\item[SUO danno] Non m'importa, Non stimo questa cosa. E diremmo; \textit{io so che la
tal cosa m'è nociva, suo danno io la voglio non ostante ec}, Esprime Io la voglio, se
bene mi può nuocere, ec. Vedi sotto C.\ 4.\ stan.\ 26.\ al termine \textit{In ogni modo}.
\item[RINCARARE] Accrescere il prezzo. E questo detto Rincarare il fitto usato in
  questi termini significa: Non fo stima, ne temo le male lingue, perché non mi
  possono far danno.
\item[FITTO] Pigione, Canone, cioè Quel danaro, che si paga annualmente per
una Casa, o Podere, o altri beni, che si posleggono d' altri con pagargit un tan-
'to lvanno. Locarionis canones,
\item[BALDORIA] Fiamma accesa in materia secca, e rara, come paglia, e simili,
  che presto s'accende, e presto finisce; detta forse \textit{Baldoria} da Baldore, O
  Baldanza, che vuol dire Allegrezza: quindi \textit{Lieta} significa poi Baldoria, come vedremo
  sotto C.\ 2.\ stan.\ 56. Diciamo anche \textit{Far baldoria}, quando altri spende
  allegramente, e si da bel tempo consumando tutto il suo havere; il qual detto vien
  forse da un religioso costume, che era fra gli Antichi, che delle vivande sagre
non si lasciassero avanzi, ma quello che avanzava s'abbruciasse; il qual rito
si cava dai Precetti di Moisè in proposito del'Agnello Pasquale. Questa specie
di Sacrifizio fu usata anche da i Gentili Romani, e la dicevano: \textit{Proterviam
  facere}, che vuol dire Far'una fiamma, o baldoria; E pigliavano ancor'essi \textit{proterviam
facere} nel senso detto sopra di consumare, e mandar male il suo, come si cava
da Macrob. lib.\ 6.\ Saturnal.\ 2., dove si legge, che Catone motteggiando un tal
Albidio, che haveva consumato tutto il suo havere, e solo gli era rimasta una Casa,
la quale gli abbruciò, disse: \textit{Proterviam fecit, propterea quod ea, quae comesse
non potuerit, quasi combussisset.}
\end{description}

\section{Stanza V}

\begin{ottave}
\flagverse{5}Offerta gliel'haveo già, lo confesso,\\
Ma sommen'anche poi morse le mani, \\
Perch'il filo non va ne ben, ne presso, \\
E versi v'è ch'il Ciel ne scampi i cani:\\
Ma poi ch'ella la vuole, e io l'ho promesso\\
Non vo mandarla più d'oggi in domani,\\
Che chi promette, e poi non ta mantiene,\\
Si sa, l'anima sua non va mai bene.
\end{ottave}

Mostra l'Autore, che la convenienza per haver'egli promessa a S.A.R.' quest'Opera,
l'obbliga a mantenere la parola, quantunque egli conosca, che non sia cosa
d'esser veduta da S.A.R., e per questo s'è morso le mani, cioè pentito
grandemente d'haverla promessa, perché vede che la tesstura dell'opera non sta
ne bene, ne presso a bene, e vi son versi \textit{che il Ciel ne scampi i cani}, cioè così
stroppiati, che tanto male non ne vorrebbe vedere, ne meno a un cane.
Ed il verbo \textit{scampare} attivo, come è in questo luogo, significa Liberare. Ma
conchiude poi, che già che S.A.R.\ la vuole, non sta bene che egli la mandi più in
lunga da hoggi in domani, ma è dovere osservar la promessa; al che fare s'accigne
adesso, non solo per questa convenienza, ma ancora per il timore della pena meritata
da colui \textit{che promette, e non mantiene} la quale è che \textit{L'anima sua non va
  mai bene}. Sentenza usatissima da i nostri Fanciulli; e viene dall'antico, poiché
l'usavano ancora i fanciulli greci secondo il Monosino Fior. Ital. linguae lib. 3.9.109.
dove cava dal Greco le seguenti parole: \textit{Nos autem dicimus id, quod solent pueri:
quae recte data sunt non licere rursus eripi}: Che suona lo stesso che: \textit{Chi da, e ritoglie
il Diavol lo ricoglie}, che vale lo stesso che: \textit{Chi promette, e non  mantiene L'anima
sua non va mai bene}.

\section{Stanza VI}

\begin{ottave}
\flagverse{6}Ma che? si come ad un che sempre ingolla\\
Del ben di Dio, e trinca del migliore, \\
Il vin di Brozzi, un pane, e una cipolla\\
Talor per uno scherzo tocca il cuore; \\
Così la vostr'Idea di già satolla\\
Di quei libron, che van per la maggiore,\\
Fore potrà, sentendosi svogliata,\\
Far di quest'anche qualche corpacciata.
\end{ottave}

Ripiglia animo il Poeta; e spera che S.A.R.\ sia per contentarsi di leggere
questa sua Opera, se non per altro, almeno per distrarsi dagli studj più serij, e
considera, che si come colui, che e solito far vita lautissima, havea talvolta gusto di
mangiare un pane, e una cipolla; e ber vino da niente, così chi è solito legger
libri più sensati, talora averà non poco gusto a legger libri di baie, e facezie.

\begin{description}
\item[INGOLLARE] Vuol dir Mangiar presto, ed inghiottire senza masticare.
  S'usa più il verbo Ingoiare, essendo il verbo \textit{ingollare} usato nel Contado, se bene è
  forse meno barbaro che \textit{ingoiare}, perché è più prossimo alla sua latina origine,
  che è la proposizione \textit{In}, e \textit{gula}, ed in questa appunto inghiottita la lettera 'L'
  secondo la stretta pronunzia comune Toscana, e mutato in I serrato, o consonante
  si dice comunemente Ingoiare: Così dice il sig.\ Francesco Maria Bellini.

\item[DEL ben di Dio] Delle più buone vivande; che i Latini dicevano \textit{Jovis nectar},
e noi diciamo \textit{latte di gallina}, che vedremo in questo Cant. stanza 64.

\item[TRINCARE] Bere assai; Voce che viene dal Tedesco; e diciamo \textit{Trinca}, o
  \textit{Trincone}, uno che beva sregolatamente; Vedi sotto Cant. 7, stanza 1.

\item[DEL migliore] S'intende quel che vuol dire, ma il senso più astruso puro
  Fiorentino è, che gli Osti di Firenze vendono sempre due specie di vino rosso, uno
di poco prezzo, che lo dicono Vino di sotto, o di bassa, perché viene da' luoghi
di sotto a Firenze, dove fanno Vini deboli, e leggieri; e l'altro di maggior prezzo,
che lo dicono vino di sopra, o de migliore; e di questo intende il Poeta.

\item[TOCCARE il cuore] Dar soddisfazione intera: Quando altri mangia con gusto,
  e si conosce, che quella vivanda gli fa pro, diciamo: \textit{Le tal vivanda gli ha
  toccato il cuore}.

\item[SATOLLO] Sazio, Ripieno. Dal latino \textit{satur}. Qui vale per Stracco di leggere.

\item[BROZZI] È un di quei luoghi sotto Firenze, dove nasce il detto vino debole.
Vedi sotto in questo Cant.\ stanza 47.

\item[PER scherzo] Intendi non per fame, o sete; ma per stravizio, o tornagusto.
E' voce Tedesca, e là pur suona lo stesso

\item[ANDAR per la maggiore] Esser della prima ' fle: Traslato da i Magitteati
dell Arti della Città di Firenze, delle quali 5: ena: 'che sono
Giudici, e Notai; Cambio; Mer 5 Lana 5 Seta; Speziati, i
se paflano a Cavalleria, Alere Minori, che art eenan *) Quota eee
non paflano, 0: ra non pafiavano aca 'quando 'in
ze si dice, // ale va per: 'delle:

maggiore ss Sete 'una

are Arti, ed' della cap sw classe, Come s' intende ie laogo's
\item[SVOGLIATO] Senz' appetito: senza puto di mungeyo

eae opie..

\item[FAR una corpacciata] Saziarsi. Empier benissimo il corpo =
corpacciata, gu altri legge, ree ° fa altra cosa'
te'fa una volta.
\end{description}

\section{Stanza VII \& VIII}

\begin{ottave}
\flagverse{7}Già dalle guerre le Provincie stanche,  \\
Non sol più non venivano a battaglia,   \\
Ma fur banditi gli archi, e l'armi bianche, \\
Ed etiam il portar un fil di paglia \\
Vedeansi i bravi acculattar le panche \\
E sol menar le man fu la tovaglia;      \\
Quando Marte dal Ciel fa capolino,    \\
Come il topo dall'orcio, al marzolino

\flagverse{8}Che d'haverlo non v'è ne via ne modo,  \\
Se dentr'ad un mar d'olio non si tuffa, \\
E reputa il padron degno d'un nodo,   \\
Che lo lascia indurire, e far la muffa. \\
Così Marte, che vede l'armi a un chiodo \\
Tutt'appiccate malamente sbuffa,      \\
Che metter non vi possa su le zampe    \\
E che la ruggin v'habbia a far le stampe.
\end{ottave}

Il Poeta dà principio all'Opera, descrivendo lo stato, in che erano le cose del
Mondo, e dice, che tutto era in pace, ne si usava più arme di sorta alcuna; ed i
bravi, ed huomini armigeri acculattavano le panche, cioè Stavano oziosi, e menavano
le mani solo in su la tovaglia, che viene a dire Attendevano solamente a mangiare.
E qui scherza con l'equivoco del menar le mani, che vuol dir Combattere, vedi
sotto C. 10, stan. 2,, e trattandosi del mangiare vuol dir Mangiare assai, e presto,
vedi sotto C. 6, stan. 46. Marte però s'adira, che non s'adoprino più l'armi.
L'Autore assomiglia Marte quando s'affaccia al Cielo, ad un topo, che s'affacci
alla bocca d'un'orcio pieno di cacio, e d'olio, che s'adira per veder tal cacio
abbandonato dal padrone, e di non poterlo arrivare, se egli non entra in detto
olio.

\begin{description}
\item[ARMI bianche] Spada, e pugnale, ed eggi altra sorta d'Armi, a distinzion
  dell'Armi da fuoco.
\item[PANCA] Arnese noto fatto di legname per uso di sedere, e possono starvi
più in una volta; detto da i Latini \textit{subsellium}, e viene dalla voce Latina
\textit{Planca}, che significa Assamenti, e tavolati piani.

\item[ACCULATTARE le panche] Significa (siccome habbiam detto) Starsene
senza far cosa alcuna, e spensierato. Ter.\ in An.\ disse \textit{Oscitantes} di coloro, che
stanno in questa maniera, quasi dica. \textit{Stanno sbavigliando}, che noi diciamo:
\textit{Starsene con le mani in mano}, o \textit{Fare a tu me gli hai}, o \textit{Dondelarsela}, e simili, che
tutti ci servono per Per esprimere \textit{Perder' il tempo in vano}, ed è quello che i Latini
dissero; \textit{Manum habere sub pallio}.

\item[TOVAGLIA] Quel panno lino che si distende, sopr'alla mensa da i Latini
  detto Mantile, e noi l'habbiamo forse da Toralia, che erano i panni, che
  \textit{circumponebantur in toris discumbentium}, ec.
\item[MENAR le mani] Quando è posto assolutamente, vuol dire Far quistione,
E con aggiunta, vuol dire Affrettarsi al lavoro, che sara aggiunto; e si usa dire
Mena le mani a correre, d'uno che corra assai, Mena le mani a leggere d'uno
che legga presto, ed in somma d'ogni Operazione humana, ancorche non fatta
con le mani, e qui vuol dire Mangiar prsto, ed il simile sotto C. 6. stan. 46.
\item[FAR capolino] Guardar di soppiatto. Quand'altri procura di vedere, senza
esser veduto, suole asconder la persona dietro a un muro, o altro, e cavar fuori
tanta testa, che l'occhio scuopra quel ch'ei vuol vedere, e questo si dice \textit{Far
capolino}. Sotto C. 2. stan. 78. dice \textit{Fa pan da Montui}, che è lo stesso.
\item[ORCIO] Vaso grande di terra, per uso di conferuar' olio, vino, ed altri
  liquori, si come per conservarvi, ed ugnervi il cacio.

\item[MARZOLINO] Specie di cacio tondo fatto a piramide, e'col manico nel
fondo dalla parte più grossa; chiamato Marzolino,perché si comincia a farlo nel.
mese di Marzo, ed € il miglior cacio, che si faccia nei nostri pacfi. E nel
presente luogo, se ben dice \textit{Marzolino}, intende ogni sorte di cacio.

\item[DEGNO di nodo] Cioè merita la forca per l'errore che fa a non mangiare
quel Marzolino, lasciandolo andar male.

\item[TUTTE l'armi appiccate a un chiodo] Dicendosi: tale ha appiccate l'armi
all'arpione, al chiodo, s'intende: Il tale ha abbandonate l'armi, cioè Lasciato
d'essere armigero. Ciò viene dagli antichi gladiatori, i quali quando dal popolo,
col porger loro una bacchetta erano assoluti, e liberati dal far più il gladiatore,
solevano dedicar l'armi ad Ercole, appiccandole nel di lui Tempio, come
ci mostra Orazio lib. 1. ep. 1.
\begin{verse}
\makebox[12em]{\dotfill} Veianius armis.
Herculis ad postem fixis, latet abditus agro.
\end{verse}
Et lib. 3, ode 26.
\begin{verse}
Vixi puellis nuper iduneus,
Et militavi, non sine gloria;
Nunc arma, defunttumqnue belle
Barbiton hic paries habebit.
\end{verse}

\item[SBVFFARE] Dar segni d'ira. Sbuffare è quel soffiare, che suol fare per lo
più uno, che sia in collera, Traslato forse da i cavalli: E si dice Sbuffare,
quando altri adirato si duole, e in uno stesso tempo minaccia con parole.

Dante Inferno C. 18,: Ud.,
\begin{verse}
\backspace Quindi sentiamo gente che si nicchia
Nell'altra bolgia, e che col muso sbuffi,
E se medesima con le palme picchia,
\end{verse}

Viene da Buffo specie di soffio, che vedremo sotto C. 3. stan. 57.

\item[CHE la ruggin v'habbia a far le Stampe] La ruggine, rodendo il ferro, vi fa
  sopra certe impressioni simili a quelle, le quali con acqua forte si fanno nel rame
  per Stampare, e pero le dice Stampe.
\end{description}

\section{Stanza IX}
\begin{ottave}
\flagverse{9}Sbircia di qua di là per le Cittadi, \\
Ne altre guerre, o gran Campion discerne, \\
Che battagie di giuoco a carte, e a dadi, \\
E Stomachi d'Orlandi alle taverne, \\
Si volta, e dà un'occhiata ne' contadi \\
Che già nutrivan nimicizie ererne     \\
E non vede i Villan far più quistione    \\
In fuor che con la roba del Padrone.
\end{ottave}

Marte, riguardando bene per le Città, vede solamente guerre di giuoco, e
gente valorosa, e brava nel mangiare. Voltatosi poi ne i Contadi, che eran già
pieni di nimicizie, e risse, vede, che dai Villani non si fa altra guerra, che
che fanno con la roba del Padrone.

\begin{description}
\item[SBIRCIA] Sbirciare vuol propriamente dire Socchindere gli occhi, acciò che
  l'angolo della vista, fatto più acuto, possa osservare con più facilità una
  minuzia, Se bene si piglia ancora per Guardar per banda, a fine di non essere
  osservato, come fanno spesso gli amanti; movendo la pupilla alla volta dell'angolo
  esterno dell'occhio, con quel muscolo, che per tal cagione da' Medici si chiama
  amatorio; E questo \textit{Sbirciare}, o \textit{Bircio}, e \textit{Sbircio} ha forse l'etimologia dal Latino
  \textit{hirquus}, che Vuol dir l'angolo dell'Occhio. Verg. Egl. 3. \textit{Transversa tuentibus
  hirquis}; la qual parola vuol Servio, che abbia origine da \textit{hircus}, essendo che
  questi animali infuriati per la libidine guardano obliquamente, e torto le capre,
  che amano.

  È pero vero, che il nome Bircio, o Sbircio si dice non solamente di chi ha gli
  occhi scompagnati, ma generalmente ancora di chi ha qualsivoglia sorta d'imperfezione
  agli occhi, essendo noi in questo non differenti da i Latini, appresso
  i quali se ben \textit{luscus} vuol propriamente dire Uno, che ha solo un'occhio, come
  si vede in Giovenale Sat.\ X.\ che parlando di Annibale dice: \textit{Cum Getula ducem
  gestaret bellua lufcum}; che il Petrar.\ disse: \textit{Sour' un grande elefante un Duce losco}.
  E Cic. de orat. \textit{Hic luscus familiaris mens Catus Sentius} :

  \textit{Lusciosus} vuol dire
  Quello, che ha la vista corta, come si può dedurre da Varrone lib. 8. disciplin.\

  \textit{Strabo} Quello che ha gli occhi torti, da noi chiamato Guercio. Cic.\ 1.\ de
  Nat.\ Deor.\ \textit{Et quos insigni nota Strabones, aut Paetos esse arbitramur};che Paetus significa
Uno che abbia gli occhi leggiermente abbassati, che noi lo diremmo Luschetto.
Porfirione annot.\ ad Horat.\ lib.\ 1.\ Sermonum Sat.\ 3. \textit{Paeti proprie dicuntur, quorum
  huc, atque illuc oculi velociter vertuntur}, ec,

Coclites Quelli, che son nati ciechi
da un'occhio. Plau.\ in Cur.\ \textit{Unocule salve; ex Coclitum prosapia te esse arbitror ec}.

Lucini; Quelli che hanno ambedue gli occhi piccoli Plin. lib. 10, cap. 37. \textit{Ab
ijsdem qui alter lumine orbi nascerentur coclites vocant, \& quibus parvi utrisque ocelli,
lucini vocantur}, ec.

Nyctilopes Quelli di vista così debole, che non veggono se non
quando splende il Sole. Plin. lib. 8. cap. 50. \textit{Si caprinum iecur vescantur, restitui
  vespertinam aciem his, quos Nyctilopas vocant}, ec.

Non ostante, appresso molti queste
differenze si confondono, pigliando spesso l'uno per l'altro; così appresso noi
si confondono i nomi Guercio, Bircio, Orbo, Lusco, e simili, ec, accomodandogli
spesso a qualsivoglia imperfezione degli occhi, come vedremo sotto in questo
Cant. stan.\ 37\ che Orbo, vuol dire Affatto cieco, cioè Oculis Orbatus, e stan. 66.
vuol dire Lusco.

\item[CHE a battaglia di giuoco, e a carte, e a dadi] Non vede nel Mondo altre risse
che di giuoco, nel quale egli non ha che fare. Perché torna non affatto fuor di
proposito una riflessione sopra la voce latina \textit{Alea}, e la voce \textit{Talus}: si contenti
il Lettore, che io faccia un poca di digressione. Sono molti de' moderni Latini,
che si servono della parola \textit{Alea} per intendere la carta da giuocare; ma forse
pigliano equivoco, se vogliamo credere a Polidoro Vergilio, al Meursio, al
Soutero, a Raffaello Volterrano, ed altri, che hanno trattato de i giuochi antichi,
i quali la chiamano \textit{charta lusoria}; \& \textit{Alea} chiamano Ogni specie di giuoco
di Fortuna, se forse quei tali non volessero sostenere la loro opinione con dire,
che quando la voce alea è presa in genere generalissimo; allora significhi ogni specie
di giuoco di fortuna: ma presa in genere speciale, significhi la carta da giuocarel
nel che mi rimetto alla prudenza del Saggio Lettore. So bene che fino il
giuoco de' noccioli era detto Alea, come si cava da Marziale.
\begin{verse}
  Alea parva nuces, \& non damnosa videtur,
Saepe tamen pueris abstulit illa nates, ec.\end{verse}
Altra volta la presero per Fortuna, secondo Livio lib.\ 37.\ che parlando d'Antioco
il quale volle più tosto guerra, che pace co i Romani per le dure condizioni,
che gli offerivano, dices, \textit{Nihil ea moverunt regem, tutam fore belli aleam
ratum; quando perinde ac victo iam sibi leges dicerentur}, ec, E Colum.\footnote{Lucius Iunius Moderatus Columella; Lucio Giunio Moderato Columella (Cadice, 4 – Taranto, 70) scrittore. }\ in Praefat.\ lib.\
1.\ dice \textit{Maris, \& negotiationis alea}. Pare che errino ancora, coloro, che
pigliano la voce \textit{Talus} per intendere il Dado, perché veramente il Dado si dice tessera,
e \textit{talus} vuol dire il Tallone, cioè Quel'osso, che è sopra il calcagno del piede,
donde si dice veste talare, la veste lunga infino a i piedi; E questa voce \textit{Talus},
trattandosi di strumento per giuocare e l'astragalo Greco, che è quello che i nostri
ragazzi chiamano aliosso; ma questo è forse minore equivoco, poiché tal'osso
finalmente viene usato in cambio di dado, servendosi per numeri di quelle macchie,
o segni, che naturalmente sono in dett'osso, come più largamente diremo
sotto C.\ 8.\ stan.\ 69. Gioviano Pontano nel suo Dialogo di Caronte distingue questo
aliosso dal dado, dicendo; \textit{Atque ego numquam talis lusi, nec tesseris}. Lo stesso fa
il Gellio lib.\ 1.\ Cap.\ 20.\ che dice \textit{Talus cubus non est, cubus .n, est figura ex omni latere quadrata, tessera sex lateribus constat}. Marziale pure nel lib.14.ep, 15. mostra
tal differenza, dicendo: \textit{Non sum talorum numero par, tessera dum sit Maior quam
talis alea saepe mihi} ec. Tal differenza si deduce anche da Cicer.\ lib.\ 2.\ de Divinat.\
\textit{Quid .n. fors est? idem propemodum, quod micare, quod talos iacere, quod tesseras}.

E tanto basti per rispondere a quei che biasimarono l'haver noi messo per esplicare
le presenti due voci Carte, e dadi il latino Charta Luforia, \& Tessera, che per altro
non importava al caso nostro questa digressione, e torna più a proposito il sapere,
che tali giuochi tanto di dadi, quanto di carte, dice Platone in Pedro, che
fussero inventati da un tal Theut Dio de gli Egizzj. \textit{Daemoni autem ipsi nomen Theut,
hunc primum numerum, \& computationem numerorum, Geometriam, Astronomiam,
talorum denique, alearumque ludos audivi}, ec. Raffaello Volterrano, e Celio Calcag.
de Ludo Talario, e Tesserario, dicono, che questi giuochi fussero trovati da Palamede
nel campo Greco sotto Troia, e però gli domanda, \textit{Palamedis alea}; si come
fa il Soutero; Ma Isidoro lib. 8. Originum, concorda bensì, che havessero origine
nel detto Campo Greco, ma da un Soldato, che havea nome Alea, e che
da lui il giuoco prese il nome d'alea, Herodoto lib. 1. riportato da Polid. Verg. lib.
2. cap. 13. dice, che l'inventassero i Lidi per le cause che si diranao sotto C. 6.
stan. 34.

\item[STOMACHI d'Orlando] Dicendosi: \textit{Il tale è buono stomaco}, o vero. \textit{È uno
stomaco d'Orlando}, ec. s'intende, il tale è coraggioso, e bravo; Qui pero valendosi
dell'equivoco di \textit{Buono stomaco}, che vuol dir \textit{Gran mangiatore}, intende Gente
brava nei mangiare,:

\item[DAR un'occhiata] Intendiamo: Guardar' alla sfuggita.
\item[FAR quistione] Far contesa, disputa, rissa; ma dicendosi assolutamente
  senz' aggiunta: Far quistione, s'intende: Combatter con le spade, ec.
\end{description}

\section{Stanza X}
\begin{ottave}
\flagverse{10}Ond'ei ch'in testa quell' umor s'è fitto,\\
Che l'huom si scrocchi pur giusta sua possa;\\
Senza picchiar, ne altro, giu sconfitto.\\
L'uscio a Bellona manda in una scossa;\\
Niun fiata perciò, non sent'un zitto,\\
Perch'ella dorme, e appunto è in su la grossa,\\
Poiché la sera havea la buona donna\\
Cenato fuora, e preso un po di nonna.
\end{ottave}

Marte risolve d'unirsi con la sorella Bellona a fine di mettere scompigli nel
mondo, e andato a trovarla, la vede in letto a dormire briaca ancora della sera
passata.

\begin{description}
\item[UMORE]
Questa voce, che per altro significa materia umida, e liquida, e
parlandosi d'animali significa Flemma, collera, malinconia, ec, viene spesso da
noi presa per Fantasia, o pensiero come nel presente luogo, che dicendo: S'è
fisso quel'umore in testa, vuol dire ha stabilito, ha fermato il pensiero, ha risoluto.
La pigliamo ancora per Desiderio. Bartolomeo Cerretani stor. nell'anno
1502. dice: \textit{Si senti che l'umore di Piero de' Medici, di tornare in Firenze non
era spento, ec, Ma Papa Alessandro, desiderando fare il Valentino suo figliuolo Signore
di Toscana, si volle anch'egli valere di questo umore de' Medici}, ec, Diciamo Bell'umore
Uno che ha fantasie graziose. Vedi sotto in questo C., stan. 58. Si dice Far' il
bell'umore Uano, che vuol far da bravo, e da ardito. \textit{Il tale volle fare il bell'umore
col salire sopra quell'albero, e cascò}, ec. Donde habbiamo Umorista, che significa
Uno di cervello instabile, ed inquieto. \textit{Haver grand'umore} vuol dir' esser superbo,
ed haver gran pretensioni di se medesimo.

\item[CHE l'huom si crocchi] Che l'huomo si perquota. Il verbo crocchiare del quale
  ci serviamo alle volte per il verbo cicalare; come si vedrà in questo Cant.\ stan.\
  4., o C.\ 3.\ stan.\ 3., e che vuol' anche dire Quel suono, che fa un vaso di terra
  cotta fesso, come Pentola, o altro vaso simile; ci serve anche nel significato di
  dar busse, e questo intende nel presente luogo: propriamente Quel cantare, che
  fa la gallina chioccia, quando ha i pulcini.
\item[GIUSTA sua possa] Per quanto egli può; Frase antica latina \textit{iuxta meum posse}, ec.
\item[FIATARE] Significa parlare. Vedi sotto C. 6. stan. 12.
\item[È in su la grossa] È in sul buono del dormire. Dorme profondamente. Traslato
  dal baco da seta, il quale quando dorme per la 3.\ volta, che è il suo dormire
  più gagliardo; si dice: \textit{È nella grossa}.
\item[NON sente un zitto] Non sente verun rumore, cioè ne pur' un di quei cenni,
  \textit{zi} che dicemmo sopra questo Cant.\ stan\ 3. Il Varchi stor.\ lib.\ 6.\ dice: \textit{Con avvertir che ne cenni, ne zitti, ne atti brutti si facessero}.
\item[CENAR fuora] Intendiamo Cenar in conversazione\footnote{``conversazione'' sembra indicare il moderno ``circolo'', o equivalente dell'inglese ``club''.} fuor di casa propria.
\item[PIGLIAR la monna] Imbriacarsi. Ci sono più specie di briachi, fra' quali son
  quelli, che si dicono cotti monne, che son coloro, che per lo troppo vino bevuto,
  danno nelle buffonerie, e saltano, e chiacchierano spropositatamente, facendo
  mille altre pazzie, e poi s'addormentano; e si dicono ancora \textit{Cotti nonne},
  o \textit{pigliar la monna}. E questo è il nome generico, il quale comprende tutte le
  specie di briachi, di che parleremo sotto C. 2. stan.\ 69. In questo C.\ stan.\ 77.\
  dice. \textit{S'imbriacaron come tante monne} dal che deduci, che si può dire:
  \textit{Prese la nonna}, e \textit{prese la monna}, che in ambedue maniere ha
  lo stesso significato,

\end{description}

\section{Stanza XI}
\begin{ottave}
  \flagverse{11}Le scale corre lesto com'un gatto,\\
  poi dal salotto in camera trapassa,\\
  E vede sopr'a un letto mal rifatto\\
  ch'ell'è rinvolta in una materassa;\\
  Sta cheto cheto, e con due man dipiatto\\
  Batte la spada sopr'ad una cassa,\\
  La qual s'aperse, ed ivi vistevi drento\\
  Robe manesche, a tutte fece vento.
  \end{ottave}

Bellona non ostante ogni romore,  che faccia Marte, non si sveglia, ed egi
ruba alcune cose, le quali trovò ivi in una cassa. Esprime il Poeta il genio furibondo
di Marte, e la natura del Soldato, che è sempre dedita al rubare.
Esprime ancora la briachezza di Bellona, dicendo, che ella dormiva \textit{rinvolta
nelle materasse sopra un letto mal rifatto}; il che mostra, che quando Bellona andò a
dormire era in grado, che non sapeva distinguere le coperte dalle materasse.

\begin{description}
\item[LESTO come un gatto] La voce lesto, che viene dal Latino \textit{sublestus}, che vuol
dir Leggieri, frivolo, e debole, appresso di noi significa Pronto, agile, e destro;
E questa comparazione \textit{Lesto, come un gatto}; da noi è usatissima per esprimere la
grande agilità d'uno. Vedi sotto C.\ 2.\ stan.\ 35.

\item[SALOTTO] Intendiamo Piccola sala, cioè un ricetto prima che s'entri nella
principal sala.

\item[MATERASSA] Arnese da letto, quello che si dice in Latino Greco Anaclinterium
  a distinzione di \textit{culcitra plumea}, che noi diciamo \textit{Coltrice}; essendo la
  materassa un sacco largo quanto è il letto, e ripieno di lana, ed impuntito nel
  mezzo.

\item[Chero cheto] Quietissimo. Nota che la replica d'una stessa voce, appresso di noi,
  ha la forza del superlativo.

\item[DI piatto] Cioè per lo largo della spada.

\item[MANESCO] Uno che sia, diciamo noi, delle mani, cioè pranto, ed inclinato
  a perguotere, ed no che sia inclinato a rubare. Qui però vuol dire Robe
  atte, e comode a esser portate via. Roba manesca intendiamo Roba, che ci sia
  prenta, e comoda a valersene.

\item[FECE vento a tutte] Portò via ogni cosa. Rubò ogni cosa. Che questo intendiamo
  quando diciamo; Far vento a una cosa.
\end{description}

\section{Stanza XII}

\begin{ottave}
  \flagverse{12}Ma non fa sì, che la sorella sbuchi,\\
Di modo ch'ei la chiama, e li fa fretta;\\
La solletica, e dice: Ovvia fuor bruchi:\\
Lo Spedalingo vuol rifar le letta,\\
S'allunga, e si rivolta, come i ciuchi:\\
Ella ch'ancor del vin ha la spranghetta,\\
E, fatto un chiocciolin su l'altro lato,\\
Le vien di nuovo l'asino legato.
\end{ottave}

Con tutto che Marte faccia ogni diligenza perché Bellona si svegli, solleticandola,
e gridando, che è hora di levarsi, non trova modo di farla destare; anzi,
essendosi ella alquanto sollevata per causa di que' romori, s'allunga, e si rivolta,
poi si rannicchia, e di nuovo si addormenta, perché il vino la tiene oppressa.
Ed è bella espressione d'uno, che dorma con gran gusto, e volentieri; perché
questo tale, sentendo strepito, si risveglia alquanto, e facendo, per lo più, le
operazioni, e moti descritti nella presente ottava, seguita a dormire.
\begin{description}
\item[SBUCARE] Intende svegliarsi, e levarsi; Uscir da quella buca, la quale si fa
  nelle materasse col peso della persona.
\item[FAR fretta a uno] S'intende Stimolar' uno a far presto.
\item[SOLLETICARE] Stuzzicare leggiermente uno in alcuna di quelle parti del
  corpo, le quali, toccate così, incitano a ridere, Viene dal verbo \textit{Sollicito},
  \textit{sollicitas}, quanto val per Tentare.
\item[FUOR bruchi] Dalla voce Bruco habbiamo il verbo \textit{Brucare}, che vuol dir Levar
  le foglie a gli alberi, e per metafora vuol dire \textit{Andar via}, onde quando diciamo:
  \textit{Il tale sbrucò}, intendiamo, Andò via, ed, il simile intendiamo nel dire
  \textit{Fuor bruchi}, cioè andate via. Luigi Pulci Bec.\ \textit{Ognun brucò, che,
    l'era la tregenda}, Onde qui s'intende \textit{Escì, dal letto}. Detto, usatissimo in
  questo proposito.
\item[LO Spedalingo vuol rifar le letta] Questo detto significa, È hora tarda, e da
  levarsi dal letto; ed ha origine da gli spedali, ne i quali si raccettano i Pellegrini;
  dove, quando è hora di levarsi, e che i poveri, e i Pellegrini seguitano a stare
  nel letto, lo Spedalingo, cioè il Guardiano, o Sopracciò dello Spedale suole
  per svegliargli gridare: \textit{S'hanno a rifar le letta}.
\item[CIUCO] Asino giovane, ò poledro. Forse dal latino \textit{Cicur}, che par che
  voglia dire Bestia addomefticata, ed agevole.
\item[HA la spranghetta] o \textit{stanghetta}. Quel duolo di testa, ed inquietudine, che
  si sente la mattina, quando, la sera avanti s'è troppo bevuto, e poco quella notte
  dormito, per lo qual duolo pare, che il capo sia sprangato, o legato con spranghetta,
  o stanghetta. Che così si chiama ogni verga di ferro, o regolo di legno,
  che unisca due materiali insieme; come si dice porta sprangata, una porta, in
  mezzo alle di cui imposte sia conficcato a traverso un regolo di legno, affinché
  dette imposte non si possano aprire, E stanghetta pure si dice quel ferro, che serra
  insieme l'imposte de gli usci, il quale s'apre, e serra con la chiave facendolo
  scorrere in certi anelli, come il chiavistello, dal quale è differente, perché il
  chiavistello non si può, o almeno non è in uso aprir con la chiave.
\item[FATTO un chiocciolino] Cioè Rannicchiatasi, o raggruppatasi quasi in figura
  di chiocciola, come sono quelle focattole, o stiacciate, che fanno le nostre donne
  per i Bambini, le quali chiamano chiocciolini, perché gli fanno a figura di chiocciola;
  e come vediamo, che nel dormire fa per lo più il cane.
\item[LEGAR l'asino] Addormentarsi, Detto, che viene da i Villani vetturali, che
  essendo per strada soprappresi dal sonno, legano l'asino, e s'addormentano nel
  luogo, dove gli piglia il sonno. E col dire: \textit{Il tale ha legato} senza l'aggiunta
  \textit{d'asino}, s'intende; \textit{Il tale s'è addormentato}. Francho Sacchetti\footnote{Franco Sacchetti (Ragusa di Dalmazia, 1332 – San Miniato, 1400), letterato. Visse principalmente nella Firenze del XIV secolo. È oggi ricordato soprattutto per la sua raccolta Trecentonovelle. } nov.\ 171.\ dice:
  \textit{Essendo Gulfo entrato nel letto, quando fu per legar l'asino, il compagno cominciò col
    mantaco a soffiare}. Bocc.\ gior.\ 4.\ nov.\ 9.\ \textit{Di che la donna spaventata,
    per svegliarlo cominciò a prenderlo per lo naso, e tirarlo per la barba, ma tutto
    era nulla, perché egli haveva a buona caviglia legato l'asino}. ec.
\end{description}
\section{Stanza XIII}
\begin{ottave}
  \flagverse{13}O corna disse il Re degli Smargiaffi,\\
E intanto le coperte havendo preso\\
Le ne tira lontan cinquanta passi,\\
Ma in terra anch' egli si trovò disteso;\\
O che per la gran furia egli inciampassi,\\
O ch' elle fusson di soverchio peso,\\
Basta ch' ei batte il ceffo, e che gli torna\\
In testa la bestemmia delle corna.
\end{ottave}

Incollerito Marte leva le coperte a Bellona, e le butta in terra, dove cascò
ancor' egli, e batté il capo, e si fece un bernoccolo, o tumore nella testa, quali
tumoretti da molti per scherzo son chiamati Corna per esser nel luogo, dove nascono
le corna a gli animali.

\begin{description}
\item[DICE bestemmia delle corna] e' piglia la voce bestemmia non nel suo proprio
significato di attribuire, o levare empiamente alla Divinità quello che se le conviene,
ma nel significato di maladizione, o imprecazione, come è preso
tal volta nella nostra Toscana, ed in altre parti d' Italia, e specialmente in
Napoli, dove \textit{iastemiare} è inteso comunemente  per Maledire. E qui dicendo:
\textit{Torna in testa a lui la bestemmia delle corna} intende: Quell'imprecazione che
haveva fatta, venne addosso a lui, e viene a dire Si fece un corno nella testa, cioè
uno di quei bernoccoli, o tumoretti, che per esser nella testa scherzosamente si
chiamano Corna.
\item[SMARGIASSO] Huomo bravo. Armigero. Ma però l'usiamo per derisione,
  e per intendere Un'huomo fuor dei limiti della ragione, e della prudenza,
  ed uno di quei petulanti, e minacciosi, che pretendono di spaventar ognuno con
  la lor pretesa bravura.
\item[CINQUANTA passi] Lontano assai, Detto iperbolico usato spesso anche in
  piccolissime distanze.
\item[INCIAMPARE] Dar co i piedi in qualcosa nel camminare: è il Latino \textit{offendere}.

\item[SOVERCHIO peso] Peso grande, peso fuor di misura, Petr.\ Canz.\ 17.
\begin{verse}
Altri ch'io stesso, e il desiar soverchio,
\end{verse}
E certo che le coperte eran di grandissimo peso, perché Bellona si serviva per
coperte delle materasse, come s'è detto sopra.
\item[BASTA] Termine conclusivo usatissimo da Noi, quasi diciamo: \textit{È a sufficienza},
  e si dice anche \textit{A bastanza}, dal verbo \textit{Bastare}, che è il latino
  \textit{sufficit}. I Latini dicevano \textit{Bat, Sat est}. Plau.\ nel Penuo si servì
  della voce \textit{Bat}, senza aggiunta di \textit{Sat est}, ed i Giosatori di esso
  dicono: \textit{Bat vox, qua utimur cum quempiam iubemus tacere}.
\item[CEFFO] Vuol dir propriamente il muso del cane, del porco, o simili, ma
  si dice anche del Viso, o faccia dell'huomo, ma per lo più in derisione, e per
  intendere una faccia brutta, e mal fatta. Vedi sotto C.\ 4.\ stan.\ 10.
\end{description}

\section{Stanza XIV.}

\begin{ottave}
  \flagverse{14}Ella svegliata allora escì del Nidio,\\
E dicendo ch'in ciò gli sta il dovere,\\
E ch'ei non ha ne garbo, ne mitidio,\\
Non si può dalle risa ritenete,\\
Cosa ch' a Marte diede gran fastidio,\\
Ma perch'ei non vuol darlo a divedere,\\
Si rizza, e froda il colpo che gli duole,\\
Poi dice che vuol dirle due parole.
\end{ottave}

Per l'insolenze di Marte, Bellona finalmente si sveglia, e dà la burla a Marte
perché egli è cascato, e Marte fingendo non sentire la percossa si rizza, e dice a
Bellona, che vuole alquanto discorrerle.
\begin{description}
\item[USCIR del nidio] Uscir del letto: quale chiama Nidio per la similitudine, che
  ha nelle materasse quel luogo, dove s'è dormito, col Nidio, entro al quale covano
  gli uccelli..
\item[GLI fra il dovere] Gli è intervenuto quel ch' ei meritava. \textit{Dovere},
  \textit{giusto}, e \textit{giustizia}, sono sinonimi.
\item[NON ha garbo] Non ha accuratezza. Per intelligenza di questa parola \textit{Garbo}
è da sapere che erano in Firenze due luoghi principali,  dove già si fabbricavano
panni lani d'ogni sorta, uno detto S.\ Martino da una Chiesa, che quivi è dedicata
a detto Santo, e l'altro si domandava il \textit{Garbo}, quali nomi di strade si
conservano fino al presente. Nel detto il Garbo si fabbricavano le pannine di
tutta perfezione; e quelle che si fabbricavano in S.\ Martino erano sempre
d'inferiore condizione, onde venne in uso il dire: La tal cosa è del Garbo,
volendo denotare la perfezione di quella tal cosa. E dalle robe venne alle persone, e si
cominciò a dire: Huomo di garbo, huomo, che ha garbo, ec. intendendo d'uno
che operi bene, e con accuratezza. Cosi dice il Monosino Flor.\ It.\ linguae
alla parola Garbo. E noi diciamo ancora in questo Senso: \textit{Non ha ne Garbo, ne
S.\ Martino},
\item[MITIDIO] Giudizio; ordine; Parola corrotta da metodo.
\item[NON si può dalle risa ritenere] Non può far di non ridere.
\item[DAR fastidio] Dar noia; dar disgusto.
\item[NON vuol darlo a divedere] Non vuol farlo conoscere. L'aggiunta della particella,
  di, al verbo vedere s'usa solo in questo caso per esprimere, far capace,
  o render bene informato.
\item[FRODARE] È noto il suo significato, venendo dal Latino \textit{fraudare}, che vuol
  dire Ingannare; Ma noi lo pigliamo ancora per Occultare, o non manifestare,
  come è preso nel presente luogo; ed è traslato da quel \textit{frodare}:, che vuol dire
  Nascondere qualche roba alla porta della Città, o alla Dogana per fraudare la
  Gabella con il non pagarla, che si dice \textit{Far frodo} Vedi sotto C.\ 6.\ stan.\ 28.
\end{description}

\section{Stanza XV}
\begin{ottave}
  \flagverse{15}Dì pur: la Dea risponde, ch'io ascolto;\\
Hai tu finito ancora? Ovvia, dì presto:\\
Ma prima di quei panni fa un rinvolto,\\
E gettalo in sul letto ch' io mi vesto.\\
Quello non sol; ma quanto haveva tolto\\
Di quella cassa, ei rende, e mette in sesto,\\
E postosi a seder su la predella,\\
Con gravità dipoi così favella.
\end{ottave}

Descrive assai bene il genio inquieto, e furibondo di Bellona, mentre mostra
l'ardenza, con la quale ella stimola Marte a dir quanto gli occorra, interrogandolo
se egli ha finito, quando sa che non ha ancora cominciato, ed in uno stesso
tempo gli comanda, che rimetta le coperte in sul letto: Ubbidisce Marte, e
s'accomoda a sedere per dar principio al discorso, che sentiremo.

\begin{description}
\item[FAR un rinvolto] È lo stesso che Affardellare, abballinare, o far balle,

\item[METTERE in sesto] Accomodare; aggiustare. E in Latino \textit{aptare}, e da
\textit{Metter in sesto} diciamo \textit{Rassettare}, o \textit{metter in assetto}. Varchi Storia libro 8.
\textit{Havendovi dì, e notte lavorato per mettere il Salone in assetto}. L'Autore della
storia de' Piacevoli, e Piattelli lib.\ 2.\ dice \textit{Non pareva possibile distender la
  fila,allogare i lasci, e dar sesto al tutto, e pure ben tosto si vedde mettere ogni
  cosa in assetto}.

\item[PREDELLA] Qui intende Quella seggiola fatta a cassetta, la quale si tiene
vicina al letto per l'occorrenze del corpo; che per altro questa voce \textit{predella} ha
molti significati, chiamandosi predella ancora quell'arnese sopra il quale si posano
le donne quando partoriscono; Predella si dice quello scaglione di legno, sopra
il quale sta il Sacerdote quando celebra Messa; e quella seggiola dove siede
il Sacerdote quando in Chiesa ascolta le Confessioni detta altrimenti Confessionale.
Predella pure è detta quella parte della briglia, che si tiene in mano, come si
cava dal Landino esposizione a Dante nel Purg.\ C.\ 6.
\begin{verse}
  Guarda com'essa fiera è fatta fella,
  Per non esser corretta dagli sproni,
  Poi che ponesti man alla predella.
\end{verse}
\item[FAVELLARE] S'intende Ragionare, discorrere; Strettamente vuol dire
  Parlar con ordine, e massime quando è contrapposto agli verbi Cicalare, gracchiare,
  chiacchierare, e simili. \textit{Il tale non chiacchierava ne cicalava, ma favellava
    e discorreva}. Cioè parlava con fondamento, regolatamente, e seriamente.
\end{description}

\section{Stanza XVI}
\begin{ottave}
  \flagverse{16}Sirocchia, male nuove; poi ch' in Terra\\
Veggiam ch'all'armi più nessuno attende,\\
Onde il nostro mestiero, idest la guerra,\\
Che sta in sul taglio, non fa più faccende;\\
Sai, che la Morte ne molesta, e serra,\\
Che la sua stregua anch'ella ne pretende,\\
E se non se li dà soddisfazione,\\
La ci farà marcir n' una prigione.
\end{ottave}

Marte in questo suo discorso mostra alla sorella la necessità, che ambedue hanno
che si faccia guerra, per il bisogno, che hanno di guadagnare almen tanto da
pagare il dazio alla morte, acciò che ella non gli faccia metter prigioni, e quivi
morire, se non le pagano detto tributo.

\begin{description}
\item[SIROCCHIA] Sorella. Parola Fiorentina; ma oggi poco in uso. Dante nel
  Purg. C.-4, e Canto 21.; 4
  \begin{verse}
    Che se Pigrizia fusse sua Sirocchia, ec.
    L'anima sua ch'è tua, e mia sirocchia, ec.
  \end{verse}
\item[STA in sul taglio] Due specie di Mercanti di drappi, o diciamo Setaiuoli sono
  in Firenze. I primi fabbricano drappi per mandargli fuor di Stato, o per vendergli
  a merciai di Firenze a pezze intere; i secondi fabbricano, e vendono in
  Firenze a braccia, o diciamo a minuto, e questi si chiamano \textit{Setaiuoli, che stanno
    in sul taglio}, Marte dice alla Sorella, che la loro arte, che sta in sul taglio non
  lavora più, ed il Poeta scherza con l'equivoco di Tagliar drappi, e tagliar huomini;
  e che di questa lor'Arte di taglio vuole la morte, che essi paghino il dazio, dando
  alla medesima tanti morti l'anno; onde se la guerra non lavora, non possono
  pagar questo tributo.
\item[SERRARE] O far serra a uno, Affrettare, stimolare, violentare uno. Vedi sotto
C. 9. stanza 13.
\item[STREGUA] Intendi quel dazio, che devono alla morte. La voce stregua, che
vuol dir Porzione dovuta, vien forse dal Latino strena, che significa mancia.
Varchi Stor. lib. 10, \textit{In alcune cose vanno quei tali rispettati, ma in molte più devono
andare alla medesima stregua, e ragguaglio degli altri}, ec.
\item[DAR soddisfazione]. Soddisfare, Adempire ogni sorte di convenienza, o di
  debito che uno habbia con un'altro: Ma strettamente s'intende Pagar quel danaro,
  del quale uno è debitore.
\item[CI fara marcir n'una prigione] Ci fara star tanto in carcere, che noi vi moriremo
  di stento; V'infradiceremo.
\end{description}

\section{Stanza XVII}
\begin{ottave}
  \flagverse{17}Bisogna qui pigliar qualche partito,\\
Se noi non vogliam' ir nella malora\\
Ed un ce n'è ch' è buono arcisquisito,\\
Qual'è, che si risvegli Celidora\\
C'ha dato un tuffo nelle scimunito,\\
Mentre di Malmantil si trova fuora,\\
E passandola sempre in piagnistei,\\
Pigra si sta, come non tocchi a lei.
\end{ottave}

Seguitando Marte il suo discorso, propone che si ponga in animo a Celidora
già cacciata da Malmantile, di risolversi alla vendetta, e così far nascere la
guerra; per rimediare a' lor bisogni.
\begin{description}
\item[PIGLIAR partito], Risolversi a pigliar qualche modo di rimediare.
\item[ANDAR nella malora] Intendi Andare in prigione per questo debito. E il
  latino \textit{In malam Crucem abire}.
\item[ARCISQUISITO] A buono, diciamo in augumento; buono, più buono, buonissimo,
  ed in luogo di buonissimo diciamo anche squisito, facendolo superlativo
  di buono e cosi non, dovrebbe patire agumento; tuttavia si dice Squisito, più
  squisito, squisitissimo, o arcisquisito, imitando forse i Latini, che da \textit{optimus}
  superlativo di \textit{bonus}, hanno, \textit{optimissimus}, Si trova anche nelli Scrittori antichi della lingua nostra.
  L'accrescimento al superlativo, il Bocc.\ nov.\ 19.\ dice \textit{Così santissima donna}, E nov.\ 60. \textit{Così ottimo parlatore}, ec, Gio.\ Villani\footnote{Giovanni Villani (Firenze, 1280 – Firenze, 1348) mercante, storico e cronista. Scrisse la Nuova Cronica, un resoconto storico della città di Firenze e delle vicende a lui coeve. } lib. 12, cap, 104, dice: \textit{Rimase in più pessimo stato}, ed al lib. 7, cap. 100, \textit{La quale era della maggiore di S.\ Gio.\ ed era molto
    fortissima} e cap. 101. \textit{A pié delle Montagne dette Pirre molto altissime}, e questo
  Autore l'usò sempre, che gli venne occasione d'esprimer un gran superlativo; ma
  da i moderni non pare, che sia molto usato, e con ragione, perché con l'aggiunta
  di molto, così, più, o simili, il superlativo che ha la natura del suo nome, riceve
  moderazione, e più tosto scema, e torna indietro della sua essenza;; e così volendo
  dire, che una Montagna sia altissima con Aggiungervi il \textit{molto}, \textit{così}, o \textit{assai}, si
  viene a dire che la Montagna sia alquanto alta, e non in tutto alta, o altissima
  ricevendo in questa maniera il superlativo limitazione, e non agumento. Salustio
  disse \textit{multo pulcherrimam} quando riporta il discorso fatto da Catone Uticense
  a Cesare in proposito della congiura di Catilina.

  La particella arci, che vien dal Greco archos,    che significa Superiore, s'usa
  anche da i moderni pen esprimere (se si, può) di là o più su del superlativo, ed il
  nostro Poeta l'usa anche nel Cant, 12. stan. 34    ma appresso di me anche questa
  particella arci aggiunta al superiativo fa l'effetto    che l'altre dette sopra di
  moderare, e non accrescere, ec.
\item[RISVEGLIARE] Non dal sonno, ma dalla Pigrizia.
\item[HA dato un tuffo nello scimunito] Ha fatta una azione da sciocca, e da stolta,
  Metaforico da i vintori, i quali volendo, che la seta, o altro, pigli il colore,
  l'intingono nel bagno di quel tal colore tante volte, quante par loro che serva.
  E questo dicono \textit{Dare un tuffo}, o \textit{più tuffi}. E dicendoti \textit{Il tale ha datoun tuffo nello scimunito}
  S'intende che quel tale habbia fatta un'azione da scimunito, non però
  che egli sia del tutto scimunito. Questo termine \textit{dar' un tuffo} può forse anche
  venire da coloro, che affogano, i quali prima di morire tornano alla superficie
  dell'acqua due, o tre volte, il che diciamo: \textit{Dare i tuffi}; e che, s'intenda è prossimo
  essere del tutto scimunito, come è vicino a esser del tutto morto colui, che da i
  tuffi nell'acqua. La voce \textit{scimunito} credo che sia composta di due dizioni, cioè
  \textit{scemo}, (che vuol dir' uno che habbia manco giudizio di quel che si conviene) e
  unito, e venga a dire \textit{unitamente scemo}, cioè scemo ugualmente, o del pari, o in
  tutte le parti a un modo, che conchiude affatto sciocco, e insensato.
\item[Si trova fuor di Malmantile] È priva di Malmantile perché le è stato tolto da
  Bertinella, o se ne trova effettivamente fuora. Diciamo: \textit{Io son fuora di tal
    pensiero} per intendere: io non ho più questo pensiero.
\item[PAGNISTEI] Singulti, solpiri mescolati con pianti. Voce da donnicciuole,
Vedi sotto C. 2 stan. 23.
\item[COME non tocchi a lei] Cioè come l'interesse in questo negozio non sia, o
  S'aspetti a lei, ma ad un'altro.

\end{description}
\section{Stanza XVIII}
\begin{ottave}
  \flagverse{18}Ma come quella, pare a me, che aspetta,\\
Che le piovano in bocca le lasagne,\\
Senza pensar un' Iota alla vendetta\\
La sua disgrazia maledice, e piagne; \\
Hor mentre ch'ella in arme non si metta\\
Per racquistar lo scettro, e sue campagne;\\
Molto male per noi andra il negozio,\\
Che muoiam di mattana,e crepiam d'ozio.
\end{ottave}

Marte pone in considerazione a Bellona, che se non trovano il modo di far risolver
Celidora ad armar gente per racquistar il suo stato di Malmantile, il negozio
andra mal per loro, che non hanno faccende.

\begin{description}
\item[CHE le piovano in bocca te lasagne] Vuol del bene, e non vuol durar fatica a
domandarlo: come per esempio uno che ha gran fame, si lascia più tosto finire
da quella, che chiedere il cibo dovutogli, ma aspetta che il cibo gli corra in
bocca da se. Costume di Cuccagna.
\item[LASAGNE] Specie di pasta tirata, ed assottigliata come un velo.
\item[UN Iota] Piccola lettera dell'Alfabeto Greco, e si piglia per esprimer il \textit{niente}.
\item[MORIR di mattana] Morir di malinconia; quasi dica: È così grande la malinconia,
  che mi nasce dall'ozio, che mi fa divenir matto, e morire. Viene da
  \textit{macto}, \textit{mactas}, e forse prima si diceva: Perire di morte mattana, ec. che era una
  occasione speciale, che si faceva da gli Aruspicj nell'immolar le Vittime, le quali
  sventravano vive, e così morivano a poco a poco crudelmente; La onde i Latini
  aggiungono sempre a questo verbo la parola morte o supplicio, come si vede
  in Cicerone, che dice \textit{Morte mactavit}, \& \textit{supplicio mactari}.
\item[CREPARE] Questo verbo Crepare, che significa Quando un legname si spacca,
  o fende da per se: significa ancora Morire a stento, ed in questo senso è preso
  nel presente luogo, o forse e preso nel senso d'Allentare, che vuol dire Quando
  a uno per la soverchia fatica cascano gl'intestini, e voglia Ironicamente
  parlando, che s'intenda; è così grande la fatica, che duriamo, che ci fa allentare.
\end{description}
\section{Stanza XIX \& XX}
\begin{ottave}
  \flagverse{19}Chi sa? forse costei se ne sta cheta\\
Perch' ella vede esser legata corta,\\
Che s'ell'havesse un dì gente, e moneta\\
Tu la vedresti uscir di gatta morta;\\
Ma qui Baldon farà dall'A alla zeta\\
(So quel chi dico, quando dico torta)\\
Ritrova tu costei, sta seco in tuono,\\
Che quant'al resto anch'io farò di buono.

\flagverse{20}Vattene dunque, e in abito di mago,\\
Dopo il formar gran circoli, e figure\\
Conchiadi, e dille che tu sei presago,\\
Che presto finiran le sue sciagure,\\
E quel tuo corazzon pelle di drago\\
Imbottito d'insulti e di bravure\\
Mettile in dosso, che vedrala poi\\
Far lo spavaido più, che tu non vuoi.
\end{ottave}

Marte facendo riflessione che se Celidora havesse chi la soccorresse, ed
aiutasse, ella si muoverebbe a procurare di racquistare lo stato, perciò ordina a
Bellona, che la vadia a trovare, e la rincuori con dirle, che presto riavera il suo
stato, e le metta addosso l'usbergo incantato.

\begin{description}
\item[CHI sa?] Questo termine significa che la tal cosa può essere, o non può essere,
  quasi dica: Chi è colui, che sa di sicuro, che la cosa sia, o non sia così?

\item[È legata corta] Cioè non ha forze bastanti a far quello, che ella  vorrebbe.
  Traslato dal cavallo, asino, mulo, o simili, i quali quando son fieri, e bizzarri si
  legano dovungue si sia con la cavezza corta, affinché non offendano chi va loro
  d'attorno.

\item[VSCIR di gatta morta] Farsi vivo, dimostrarsi fiero. \textit{Far la gatta morta} vuol
dir Simulare. Il Lalli En. Trav, Cant. 2. stan. 12. parlando dsl Cavallo Troiano
dice:
\begin{verse}
  e stanno i Greci ascosti in questo legno,
  e v'attendono a far la gatta morta.
\end{verse}
I Latini dissero \textit{lepus dormiens}, E noi diciamo anche \textit{far la gatta di
  Masino}. Vedi sotto C. 7. stan. 69.

\item[FARÀ dall'A alla zeta] Farà puntualmente quanto bisogna. Farà il tutto.
  L'A, e la Z. sono il principio, e il fine del nostro Abbicci, onde con questo
  termine intendiamo \textit{Sarà fatto il tutto}, come appunto appresso i Greci Alpha, \&
Omega; che è lo stesso che \textit{Capite ad calcem} de' Latini.
\item[SO quel ch'io, dico, quando dico torta] So benissimo come sta questo negozio,
  Esprime \textit{m'intend'io}, Il Pulci nel suo Morgante fa dire a quello scellerato di
  Margutte.
  \begin{verse}
    Io credo nella torta, e nel Tortello:
    Sò quel, ch'io dico, quand'io dico torta,
  \end{verse}
E vuol dire M'intend'io, quel ch'io voglio dire, e quello ch'io intenda per
torta.
\item[STA seco in tuono] Sta seco unita; Va d'accordo seco. Traslato dalla Musica.
\item[FARÒ di buono] Negozierò da vero. Farò quanto bisogna. Quando uno
giuoca di danari si dice \textit{Far di buono}, che vuol poi dire Operar con attenzione; il
chee non si fa quando non si giuoca di buono, non ponendosi attenzione quando
si giuoca da burla.
\item[ABITO da Mago] Non hanno i Maghi abito particolare, ma il Poeta se lo
  figura in quella guisa, che ha veduto in commedia, cioè veste lunga, gran barba,
  e la verga in mano. E \textit{Mago} è voce Persiana, che significa \textit{Sapiens}, e
  quello che i Greci dicono Filosofo. E di questa sorte Filofofi furono quelli Magi, che
  andarono ad adorare Giesù bambino. Ma perché Zoroaste fu anch'egli uno di
  tali Filosofi detti Magi, e secondo Plin. lib. 30. cap. 1. fu inventore dell'Arte
  dell'incantare, però tal arte è detta Magia, e coloro, che l'esercitano son chiamati
  Magi. Tasso Gerusal. C.\ 10.\ stan.\ 29.
  \begin{verse}
    Son detto Ismeno, i Siri appellan Mago,
    Ma che dell'arti incognite son vago.
  \end{verse}
  E perché quest'arte, secondo Polid. Verg. lib. 1. cap. 33. è di sei specie, cioè
  Negromanzia, Geomanzia, Chiromanzia, Piromanzia, Aeromanzia, Hydromanzia,
  però questi Magi son detti ancora Negromanti, ec, Vedi sotto Cant, 2. stan. 5.
\item[SCIAGURA] Questa voce  parrebbe che significasse Scelleraggine, o
  Sciagurataggine si piglia da noi per Disgrazia.  Boccaccio Novella 36. \textit{La
    storia del mio ardire, e della mia sciagura vi racconti} E N. 43. \textit{E della sua sciagura
  dolendosi}. I Latini pure dicevano \textit{Scelus}, e se ne servivano nello stesso modo, che
 facciamo noi per intendere Disgrazia. Plaut. in Capt. \textit{Maior potitus hostium est,
   quod hoc est scelus? Quasi in orbitatem liberos produxerim}. Ter. in Eun. \textit{Neque quemquam
 esse ego hominem arbitror, cui magis bonae Felicitates omnes adversae sint. P. Quid
hoc est sceleris?} Il medesimo significato ha la voce latina — che a noi ha la
voce Sciagurato.

\item[CORAZZONE] Corazza grande, Armatura di petto, e schiene; dal latino
\textit{Thorax}, si dice anche Petto a botta, perché è a figura d'una botta, o perché si
presume, che regga a una botta d'archibuso.

\item[IMBOTTITO] Ripieno, e trapuntato non di cotone, o altro simile, \textit{ma d'insulti
  e di bravure}, che vuol'intendere Incantato, come vedremo appresso nell'ottava 27.

\item[SPAVALDO] Huomo avventato; Huomo inconsiderato, Dal latino \textit{supervalidus}
  Soverchiamente ardito, e quasi temerario, e tutto impertinente.
\end{description}

\section{Stanza XXI \& XXII}
\begin{ottave}
  \flagverse{21}Bellona c'ha il medesimo capriccio\\
Di far braciuole, va col sarrocchino.\\
Con il bordone, e un bel barbon posticcio,\\
Sembrando un venerabil pellegrino;\\
E fatto di parole un gran pastriccio\\
Esser dicendo astrologo, e indovino,\\
Che vien di quel discosto più lontano\\
La ventura le fa sopr'alla mano;

\flagverse{22}Ove doppo mostrato ogni accidente\\
Di tutta la sue vita pel passato,\\
Seggiunge, che per via d'un suo parente\\
In breve tempo riavrà lo stato;\\
Però si metta in arme, ch'un presente\\
Le fa d'um panceron, che ancorché usato\\
Ripara i colpi ben per eccellenza,\\
E poi piglia da lei grata licenza
\end{ottave}

Bellona va a trovar Celidora, e fingendosi Astrologo, le dice molte cose occorsele
per il passato, per accreditarsi; poi le predice, che fra poco tempo ella
riavrà il suo Stato, però si metta in armi; e le dona la corazza incantata, e si
parte.
\begin{description}
  \item[CAPRICCIO] E Pensiero, fantasia, volontà., come intende anche sotto C. 6,
stan. 101. E per altro \textit{capriccio} significa quello, che i Latini dicono \textit{orrore}, che è
quando i peli s'arricciano; il che segue o per lo freddo, o per qualche subito spavento,
o ne i casi di febbre, come s'intende sotto C. 6. stan. 14. e C, 20. stan. 2.
Donde poi habbiamo il verbo \textit{accapricciare}, che vuol dire Havere spavento. Dante
Inf. C22.
\begin{verse}
  Lo viddi, ed anche il cor men' accapriccia
\end{verse}

\item[BRACIUOLE] Si dicono quelle fette, o strisce di carne di porco, o d'altro
  animale, che sono così tagliate per cuocerle sopr'alla bracie, e però dette
  \textit{braciuole}, Ma qui intende fette d'huomini, e vuol dire che Bellona havea la
  medesima volontà di far guerra, che haveva Marte.
\item[SARROCCHINO] È un collarone di cuoio, il quale adattato al collo cuopre
tutte le spalle, e buona parte delle braccia, e petto a foggia di Manteiio, ed è
usato da i Pellegrini, che vanno a piede a visitare i luoghi santi; E questi tali
sono da noi chiamati Pellegrini corrottamente da Peregrini; la qual voce è latina, e
ritiene appresso di noi gli stessi significati di singolare, e grazioso, ed anco di
forestiero, \textit{Peregrinus in domo patris mei}, Petrarca Can. 12.
\begin{verse}
  Mosse una Peliegrina il mio cor vano
\end{verse}
Et intende, che una graziosa, e bella donna mosse il suo cuore. E la detta voce
Sarrocchino credo, che venga da San Rocco il quale portava forse questa parte
d'abito, quando andò peregrinando il Mondo.
\item[BORDONO] È nome particolare, e proprio di quel bastone, che portano i
Pellegrini.
\item[PASTRICCIO] Massa confusa di diverse robe. Qui vuol dire quantità di
  parole mal' ordinate.
\item[DAL discosto più lontano] Più lontano della lontananza stessa, come diremmo:
Vero più del vero, o della stessa verità.
\item[FAR la ventura] Strolagare. Sono alcune donnicciuole originarie d'Egitto,
le quali in Toscana vengono il più delle volte di Sicilia, e si chiamano Zingane.
Queste, dando a creder d'esser perite di chiromanzia per buscar denari, vanno
considerando i lineamenti delle mani alle persone, e palesano (dicono esse) le cose
passate, e predicono le future: E perché discorrono artifiziosamente con certi
lor generali sempre di bene; esse chiamano, ed anche da tutti noi vien detta questa
operazione; \textit{Far la ventura}, o \textit{la buona ventura}.
\item[PARENTE] Intendiamo ogni sorte d'affini, o consanguinei in qualsisia grado;
  così è inteso nel presente luogo, che vuol dire Baldone cugino di Celidora.
Così l'intese Dante nel Parad. C.6., e il Petr. Son. 191. E se bene strettamente
vuol dire il genitore, venendo dal latino \textit{Parens}, e usato da noi in tal senso
assai di rado, e forse non mai fuor che nel numero del più, come l'uso Dante Inf.
Cant. 1.
\begin{verse}
  \makebox[8em]{\dotfill} Homo già fui
E li parenti miei furon Lombardi,
Mantovani per Patria ambi dui,
\end{verse}
Ed il Petr. Canz. 29.:
\begin{verse}
Madre benigna, e pia,
Che cuopri l'uno, e l'altro mio parente,
\end{verse}
\item[PANCERONE] Intende quella gran corazza detta sopra in questo C. stan 20.
\item[ANCORCHÉ usato] Adoperato, Vecchio, Antico.
\item[PIGLIAR buona licenza] Pigliar commiato, Licenziarsi da uno per andarsene.
  E quell'epiteto di \textit{buona}, o \textit{grata} s'aggiugne per esprimere, che quel
  tale parte con buona grazia dell'altro, e con il di lui consenso, e non forzato,
  o scacciato.
\end{description}

\section{Stanza XXIII \& XXIV}
\begin{ottave}
  \flagverse{23}Già il termine d'un anno era trascorso,\\
  Che Celidora havea perduto il Regno;\\
  Quando non pur le spiacque il caso occorso,\\
  Ma volle un tratto ancor mostrarne segno,\\
  Perciò richiesto ai convicin soccorso,\\
  Che un piacer fatto non havrian col pegno,\\
  e tenevano il lor tanto in rispiarmo,\\
  ch'egli era giusto, come leccar marmo.

  \flagverse{24}Fece spallucce a Calcinaia, e a Signa,\\
  Ma la pania al suo solito non tenne,\\
  Perché terren non v'era da por vigna;\\
  Calò nel piano, e ad Arno se ne venne,\\
  Ove Baldon facea nella Sardigna\\
  Vele spiegare, e inalberar' antenne,\\
  Fermato havendo lì come buon sito\\
  D'armati legni un numero infinito.
\end{ottave}

L'Autore toccando la finta storia della perdita dello Stato di Celidora, dice,
che era già passato un'anno, quando la medesima cominciò ad haver pensiero di
ricuperarlo, e per ciò fare, richiese soccorso a diversi vicini, ma senza frutto; la
onde si risolvé di venirsene verso Firenze, e trovò in su la riva d'Arno in un
luogo detto Sardigna Baldone con una buona armata.
\begin{description}
\item[UN tratto] Una volta, La voce tratto ha molti significati dicendosi \textit{tratti di
  fune}, Quello scarrucolamento, che si da a i delinquenti nel martirio della corda.
  \textit{Tirar i tratti}, diciamo Quelli ultimi moti, che fanno i moribondi nell'esalar lo
  spirito. \textit{Tratto} si dice in vece di estratto, cavato, o dedotto, ec, \textit{Tratto} val per
  distanza, dicendoli tratto di tempo, tratto di via, e simili, \textit{Tratto} di cortesia per
  Atto di cortesia, \textit{Tratto} per maniera, Ed in questo luogo significa Finalmente, ed è
  il latino \textit{tandem aliquando}.

\item[VN piacer fatto non havrian col pegno] S'intende Uno, che non fa mai servizio
  a veruno, eziam se li fusse dato il pegno in mano.

\item[TENER il suo in rispiarmo] Tenere il suo a fe, e con riguado, molti dicono
  \textit{risparmio}, e \textit{risparmiare}.

\item[GIVSTO] Questo termine significa Per l'appunto.

\item[ERA come leccar marmo] Era vana ogni diligenza per appunto, come è vanità
  leccar' il marmo.

\item[FECE spallucce] Si raccomandò. Questo detto seas dai poverelli, che per
  muovere a compassione in domandando l'elemosina, fanno tutte le smorfie, e
  gesti, che fanno, e possono, e fra gli altri il più comune il \textit{Fare spallucce}, cioè
  Stringer le spalle alla, volta del collo.

\item[LA pania non tenne] Non fece cosa di buono, cioè non hebbe aiuto da coloro
  da' quali lo sperava; intendendosi con questo dettato, che quel tale, che fu richiesto,
  non adempì il volere di chi lo richiese; che diciamo ancora: \textit{Non ha trovato
    appicco}. I Latini pure in questo proposito dissero \textit{Evanuerunt insidia}. \textit{Pania}
  intendiamo il visco, col quale si pigliano gli uccelli. E diciamo \textit{Non tenere} quando,
  o per il molle, o per altro la pania non appicca, ne li prende.

\item[AL suo solito] Secondo il suo costume, Dice al suo solito per dimostrare, che
  in quei paesi era da sperar poco bene al solito, \textit{perché non v'è terreno da por vigne},
  che vuol dire: Non è da far fondamento, o da sperare da loro favore alcuno, e
  scherza con l'equivoco del \textit{porre vigne}, perché veramente quei paesi non hanno
  terreni buoni a porvi le viti.

\item[CALO' nel piano] Scese nel piano, perché Calcinaia, e Signa sono piccole
  collinette vicino ad Arno.

\item[OVE Baldon facea nella Sardigna] L'Autore, che vuol sempre stare in su le
  burle, e servirsi dello scherzo degli equivoci, fa che Celidora trovi Baldone nella
  Sardigna; e pare che voglia dire l'isola di Sardigna, ed intende di un luogo fuori
  delle mura di Firenze in fa la riva d'Arno, così detto per il fetore, che quivi
  sempre si sente a causa delle bestie del piè tondo, che morte si fanno in quel luogo
  scorticare: e tal nome viene dai Latini; che chiamavano; Sardinia. quei luoghi,
  li quali per li mali odori sono sottoposti all'infezione dell'aria, come è l'isola
  di Sardigna, la quale per havere da Settentrione monti altissimi, che le
  impediscono i venti, è sempre di cattiva aria, e sottoposta alla pestilenza. Di qui
  ancora li nostri Medici hanno dato il nome di Sardigna a quel luogo, nello Spedale
  di Santa Maria Nuova di dove si mettono gli infermi più fetenti
  per piaghe, o altro simile. In detta riva d'Arno chiamata \textit{Sardigna}, si fermano,
  e scaricano, e si ricaricano, i Navili, che da Livorno vengono a Firenze su
  per lo fiume d'Arno, e tali legni, che quivi son sempre in gran numero, finge
  che sieno l'armata di Baldone. Su questa riva, come s'è detto sono gli scorticamenti
  delle bestiacce morte, e però dice, \textit{che vi era buon sito}, e si serve di questa
  voce \textit{sito} per \textit{posto} ed in effetto vuol dire Puzzo, o Mal'odore, che scaturisce da
  quelle Carogne, e la parola \textit{sito}, che vuol dire l'uno e l'altro, fa nascere un bello
  scherzo.  Quello medesimo scherzo può farsi anche nel Latino, perché dicono
  \textit{Situm casprorum} secondo Ces. de bello Gallico, ed intendono ancora puzzo secondo
  Plin. lib. 21, \textit{Pessimum esse Crocum, quod situm redolet}.
\end{description}

\section{Stanza XV \& XXVI}
\begin{ottave}
  \flagverse{25}Costui quando Bellona fu inviata\\
A Celidora, come già s'intese,\\
Da Marte haveva havuta una fardata,\\
Che lo tenne balordo più d'un mese,\\
E gli messe una voglia sbardellata\\
Di far battaglia, e mille belle imprese;\\
Ond'egli entrato in fregola sì fatta\\
Fece toccar tamburo a spada tratta.

\flagverse{26}Poi che'pedoni egli hebbe, e gente in sella\\
Tanta ch'al fin si chiama soddisfatto,\\
Render volendo il Regno alla Sorella,\\
E farle far bandiera di ricatto,\\
Destinò muover guerra a Bertinella,\\
Ch'a lei già dato havea la scacco matto;\\
Cosè con quell'armata, e quei disegni\\
In Arno messe i sopradderti legni.
\end{ottave}

Marte era stato a trovar Baldone, conforme haveva detto alla Sorella, e l'haveva
fatto rifolvere a mettersi in arme per aiutare Celidora, e rimetterla nello
Stato; e perciò con questa gente a tal fine s'era imbarcato.

\begin{description}
\item[FARDATA] Percossa data con un pannaccio intinto in sporcizia; perché
  farda vuol dire sornacchio, che è Un grande sputo catarroso. Vedi sotto in questo
  Cant. stanza 47. E s'intende ancora per Una quantità di sporcizia bituminosa,
  che tirata in qualche luogo s'appicchi, e s'interni in quel luogo dove è buttata,
  come farebbe una manata di fango, o altro simile buttato in un muro; Dal
  che per metafora intende in questo luogo per Un colpo, che s'appicchi, e s'interni,
  quella persuasione, che Marte haveva fatto a Baldone di far guerra.
\item[BALORDO] Questa voce che vuol dir Inavvertito, Smemorato, che è il
  latino \textit{mente captus}, ci serve per intendere D'uno, che per qualche accidente
  occorsogli, resti sopraffatto, e non sappia a qual partito appigliarsi, per rimediare
  al danno che da quello accidente gli resulta, e si dice anche \textit{Sbalordito},
  \textit{Stordito}. Vedi sotto C.~11, stan.~25.
\item[SBARDELLATO] Una cosa che eccede i termini del naturale, ed in un certo
  modo avanza il superlativo, perché si dice: Grande, più grande, grandissimo, e
  Sbardellato; è però parola bassa, e poco usata; È forse meglio Disorbicante, o Immoderato,
  che suonano lo stesso. L'Autore del Capitolo in lode de' peducci dice.
  \begin{verse}
    Io sto cinque hore del giorno in mercato
    A pascer gli occhi di sì bell'oggerto,
    E ne cavo un piacere sbardellato,
  \end{verse}
\item[FREGOLA] Voglia grande. Onde vuol dire \textit{Entrata in fregola sì fatta} intende
  Essendogli venuta così gran voglia. È traslato dai pesci, che si dice \textit{Andare in
    fregolo}, quando s'adunano molti insieme per la generazione; ed è il latino \textit{libido},
  o \textit{cupido}, E diciamo \textit{In Fregola} I gatti, quando sono in amore. Vedi sotto Cant.
  3. stan. 30.
\item[TOCCAR tamburo] Vuol dir Suonare il tamburo, ma s'intende Arruolare
  Soldati, il che si dice anche \textit{Batter la cassa} Vedi sotto C. 3 stan. 56.
\item[A spada tratta] Incessantemente, senza riposo, Senza intermissione,  senza levar mano.
\item[FAR bandiera di ricatto] Ricattarsi, Vendicarsi. Questa voce Ricatto, che
  vien dal verbo Ricatcarsi, il quale vuol propriamente dire Liberarsi di schiavitudine,
  da noi è presa per Vendicarsi, e Far venddetta, ed è il Latino \textit{par pari
    referre}. Il dettato \textit{Far bandiera di ricatto} stimo che venga dal costume dei Corsari,
  li quali, quando pigliano qualche legno, che stimino d'essere in grado da esser
  ricattato, v'inalborano una bandiera bianca, con la quale, danno cenno alle
  Terre vicine se lo vogliono ricattare; il che se voglion fare, corrispondono con
  alzar bardiera dello stesso colore; e questo dicono Metter bandiera di ricatto.
\item[DATO havea lo scacco matto] Le havea fatto questo danno, o cagionata questa
  rovina. Il giuoco delli scacchi è antico, e fu usato prima da i Greci, che
  ora lo dicono \textit{Zatrici}, e poi seguitato da i Latini, che lo dissero \textit{Ludus
    latrunculorum}. A questo giuoco si da fine quando e fatto prigione il Re, e si dice allora
  scacco matto; onde qui vuol dice, che Celidora havea toccato Scaccomatto, havendo
  perduto il suo Regno: E s'allarga quello detto a tutto quello, che ad altri
  succeda di gran perdita, o di grave danno.
\section{Stanza XXVII}

\begin{ottave}
  \flagverse{27}Ov'anco in breve Celidora arriva\\
Con armi in dosso, ed altro da far fette,\\
Perché una volta al fin fattasi viva\\
Ha risoluto far le sue vendette;\\
Che l'usbergo incantato della diva\\
L'ha fatto diventar l'Ammazzasette,\\
Ed alle risse incitala talmente,\\
Ch'ella pizzica poi dell'insolente.
\end{ottave}

Celidora arriva all'armata di Baldone nella Sardigna, e quivi comincia a mostrare
gli effetti della Corazza incantata.
\begin{description}
\item[ARME da far fette] Intende la spada, e vuol dire che era larga, ed abile a
far fette.
\item[FATTASI viva] Rifentitasi, e fattasi ardita., E lo stesso che P7cir di-garra
morta detto sopra in questo Cant. stan. 19.:,
\item[USBERGO] Cioè quella Gran corazza di pelle di drago: detta sopra, la quale il
  Poeta qui dichiara, che ha inteso, \textit{incantata} quando ha detto sopra \textit{imbottita
  d'insulti, e di bravure} alla stan. 20.
\item[AMMAZZA fette] Contano le donne una novella per trattenimento de'Fanciulli;
  e per accomodarsi alla loro capacità, dicono::, Fu una volta un bel giovanetto
  in Garfagnana detto Nanni, il quale per la sua mendicità dormiva in una
  capanna da fieno; quivi essendo egli un giorno per riposarsi, e ripararsi dal caldo,
  si messe a pigliar le mosche, e ne haveva ammazzate sette, quando comparve
  quivi una bella Fata, e gli disse; che se le donava quelle sette mosche per cibare
  una sua passera, l'havrebbe fatto ricco. Gliele concesse egli più che volentieri;
  ond'ella innamorata di questa sua cortese prontezza lo prese per la mano,
  e lo condusse alla sua caverna, dove rivestitolo, e datogli danari, ed armi, gli
  pose in testa un'elmo, o berretta in cui era scritto a lettere d'oro: Ammazzasette;
  e lo mando al Campo de' Pisani, i quali in quel tempo. con l'aiuto de Franzesi
  guerreggiavano co i Fiorentini. Arrivato Nanni a detto Campo, chiese
  soldo a i Pisani, e domandatogli del nome rispose: Io mio chiamo Nanni, e per haver
  io solo in un giorno ammazzato sette, ho per soprannome: \textit{Ammazzasette}. Fu per
  questo, e per esser' anche ben formato, con buon soldo, e con non minore stima
  accettato. Essendo poi fra pochi giorni in una scaramuccia morta il Capo
  delle truppe Franzesi, e volendone essi fare un altro, erano fra di loro in gran
  differenza, perché essendone proposti diversi, coloro, a' quali non piacevano. i
  Soggetti proposti, gridavano Nani, Nani, onde i Soldati Italiani, che credettero,
  che dicessero Nanni, Nanni, e che havessero creato lui: cominciarono a
  gridar Nanni, Nanni; viva Nanni; e così a voce di popolo Nanni detto l'Ammazzasette
  restò eletto capo di dette truppe, e divenne ricco, si come gli haveva,
  promesso la Fata. E di questo intende il Poeta, volendo mostrare, che Celidora
  era divenuta brava, quanto questo Ammazzafette, il quale non fece maggior
  bravura, che ammazzar quelle sette mosche, si come ne anche Celidora
  non fece maggior bravura, che affettar quei Cavoli, che vedremo nell'ottava
  29. seguente.
\item[ALLE risse incitala talmente, ch'ella pizica d'insolente] Bellona le fa venir voglia
  così grande di far risse, che ella vien poi a noia, e si rende odiosa con i suoi
  modi impertinenti. Il verbo \textit{Pizicare} vuol dire Cominciare a essere, o Esseres
  alquanto. \textit{Il tale è stato tanto tempo in Firenze, ch'ei pizica di Fiorentino}, Lo trovo
  anche usato da i Bolognesi in questo senso, e l'usò Francesco Negri\footnote{Giovanni Francesco Negri, Bologna 1593 - Bologna 1659, pittore} nel suo Tasso
  in lingua Bolognese Cant. 1, stan. \makebox[1em]{} dove \begin{verse}El pizigava di sei ann' ch'i Tramuntan\end{verse},
ec. per intendere, Era già presso a sei anni, ec.

\item[INSOLENTE] Si dice colui che dà fastidio, e noia a ognuno, e che si rende
odioso a tutti con le sue azioni impertinenti.
\end{description}
\section{Stanza XXVIII \& XXIX.}

\begin{ottave}
  \flagverse{28}Non così tosto al campo si conduce,\\
Come la suora vuol del Dio Soldato,\\
La Marfisa di nuovo posta il luce,\\
Ch'ell'esce affatto fuor del serminato;\\
E col brando che taglia, com'ei cuce,\\
Da far proprio morire un disperato,\\
Vuol trucidar' ognuno, ognun vuol morto,\\
E guai a quello, che la guarda torto,

\flagverse{29}Se guarda, è dispettosa, e impertinente,\\
 E sempre vuol che sia la sua di sopra;\\
 Talor' affronta per la via la gente\\
 Cercando liti, quasi franchi  l'opra:\\
 Ne venga (dice) pur chi vuol niente,\\
 Però che, chi mi da che far mi sciopra;\\
 Giunta in quest' in un campo pien di cavoli\\
 N' affetto tanti, che Beati Pavoli.
\end{ottave}

Descrive il Poeta una brava spropositata, e impertinente, per mostrare in Celidora
gli effetti dell'incantata Corazza; e con queste azioni, che le fa fare, dipigne
al vivo uno di questi spacconi, e ammazzatori, che noi diciamo che Campano
di fegati d'huomini, e son poi il ritratto della poltroneria, e sfogano la
lor bravura come fa Celidora, in un campo di Cavoli.
\begin{description}
\item[COME la suora vuol del Dio soldato] Come vuol la sorella di Marte, Bellona,
per opra della quale Celidora e capitata a quel campo.
\item[MARFISA] Donna guerriera nota, favoleggiata dall'Ariosto, e però la dice:
  \textit{di nuovo posta in luce}, ed intende una Marfisa moderna fatta brava da Bellona,
  cioè Celidora.
\item[USCIR del seminato affatto] Perder' il senno del tutto, Impazire. Quando altri
  per un grandissimo contento si railegra più del dovuto, diciamo: \textit{Il tale impazisce
    per l'allegrezza}; e così intende di Celidora, non che veramente sia impazita.
  I Latini hanno il verbo \textit{delirare}, che vuol dire Impazire, ed è metaforico dal
  bifolco, sendo composto dalla preposizione \textit{De}, che suona \textit{extra}, \& \textit{lirare},
  che vuol dir Fare i solchi nel campo con l'aratro; e con questo sol verbo
  \textit{delirare} intendono \textit{extra liram incedere}, dove noi diciamo Uicir del seminato, che
  è lo stesso che \textit{extra liram incedere}, o \textit{delirare}, del qual verbo ci ferviamo ancor
  noi nel medesimo senso, come si vede in Dante. Inf.\ C.\ 11.
  \begin{verse}
    Ed egli a me; perché tanto delira
    Hoggi l'ingeguo suo da quel che suole.
  \end{verse}
  E si dice anche deliro uno, che sia fuori del senno, Dan. Par. C. 1.
  \begin{verse}
    Che madre fa sopr' al figliuol deliro,
  \end{verse}

Alcuni vogliono, che.questo verbo \textit{Delirare} venga dal Greco, \textit{Lirin}, che vuol
dir scioccheggiare. Diciamo nel medesimo significato Uscire del seminario, E questo
forse deriva dal Latino \textit{Seminarium}, che secondo Colum, lib, 1. de arboribus
c. 1. 3. vuol dir quel luogo, nel quale si seminano le piante per trapiantarle, il che
quando segue, la pianta cavata dal detto \textit{Seminario} resta come un pesce fuor dell'acqua,
e piantata poi ripiglia il vigore, quando ha cominciato ad attaccarsi nella
nuova terra; e da quello, dicendosi huomo fuori del Seminario, s'intende Huomo
sbalordito. Si dice ancora \textit{fuori del secolo}, e habbiamo \textit{strasecolato}, ed il verbo
\textit{Strasecolare}, Vedi sotto Cant, 6, stan. 36. pur tutto a questo proposito. Ma si questo,
come gli altri suddetti termini, con \textit{tutto che possano credersi l'accennate
derivazioni, io stimo che intanto s'usino in questo proposito, in quanto hanno il
principio della parola, che somiglia quello della parola senno}; e che si dica fuori
del \textit{Seminato}, \textit{Seminario}, o \textit{Secolo} in vece di dire Fuori del \textit{senno}. E questa specie
di parlare, che è specie di parlar furbetto, è molto usato in Firenze per scherzo,
e lo dicono parlare Ianadattico, il qual parlare riesce assai grazioso, quando è maneggiato
da persone spiritose, perché talvolta con parole, che non hanno che
fare con quella materia, della quale si discorre, vien descritta per allusioni, ò
per metafore, ò altrimenti quella tal cosa, della quale si parla. Per esempio: Ad
un Priore, il quale a tre mogli, che haveva havuto, non hebbe mai figliuoli, ed
havea nome Antonio, dicevano \textit{Priapo annebbiato}. Ad un Proposto. che havea
nome Girolamo, ed era lungo, secco, e di colore olivastro, dicevano; \textit{Prosciutto girato}.
Di questo parlar' Ianadattico si serve sotto C, 9. stan. 1.
\item[TAGLIA come ei cuce] Tanto è buono a tagliare, quanto buono a cucire, che
  vuol dir: non taglia. Detto usatissimo per intender Ogni sorte di coltello, o arme,
  o forbice, che per la ruggine, o altro non sieno atte a tagliare.
\item[FAR morire un disperato] Dicono che le ferite fatte con i ferri rugginosi, ò
  intaccati, sieno pericolose di cagionare spasimo, e perciò quando si vede un coltello,
  o arme di tal sorte, si suol dire \textit{Farebbe morire uno disperato}, cioè di dolori eccessivi,
  o di spasimo, E tale era la spada, o brando di Celidora.
\item[GUAI a quello] Male, o gran disgrazia avverrebbe a colui, che la guardasse
torto. E il Latino \textit{Vae illi}.
\item[GUARDA torto] Quand' uno non è molto nostro amico, diciamo: \textit{Il tale non
  mi vede con buon'occhio}; O vero \textit{mi guarda torto}, Che i Latini pure dicono \textit{Non
  rectis aspicere oculis}.
\item[DISPETTOSO] Huomo altero, e che disprezza, ognuno, e d'ogni piccola,
cosa s' adira.
\item[IMPERTINENTE] Uno che vuol più del suo dovere, o del giusto, o più di
quel che gli s'appartiene.
\item[VUOL che la sua, stia sempre di sopra] Vuol sempre haver ragione, che si dice
anche Soprastante. E questi tre modi cioè \textit{Dispettoso}, \textit{Impertinente}, \textit{Soprastante}
si posson dire Sinonimi, e significanti Huomo d'una certa imperiosa arroganza, o
superbia, compagna indivisibile di tutti gli Sgherri, e bravanzoni a credenza.
\item[AFFRONTARE] Vuol propriamente dire Assaltare il nemico, ma si piglia
  ancora per Andar' incontro, o affacciarsi a uno per parlargli, e così è preso nel
  presente luogo, per intendere che Celidora cercava spropositatamente l'occasione
  di far quistione, e tutto per descriverla simile a i detti bravi di parole.
\item[CHI mi da che far mi sciopra] Dovrebbe dire Mi sciopera, secondo che da
  alcuni troppo delicati, e punto considerati ne fu avvertito il Poeta, ma la figura
  Sincope (ammessa fra i Latini) Verg. 5. AEn. dice \textit{gubernaclo} in vece di \textit{gubernaculo}
  da noi è accettata anche nella prosa, ed adoprata comunemente in molte voci,
  particolarmente in questa, dicendosi più pesso \textit{Opra}, \textit{Adoprare}, \textit{Scioprare}, che
  \textit{Opera}, \textit{Adoperare}, e \textit{Scioperare}, lo libera da questa censura. E questo termine \textit{Chi
  mi da che far mi sciopra} è proprio di certi Taglia cantoni, che voglion con esso
  mostrare che chi dà loro occasione di far questione gli \textit{sciopera}, cioè li leva dal
  farne un'altra, che han in mano, e li leva da un lavoro per impiegargli in un'altro simile.
\item[N'AFFETTÒ tanti, che Beati Pavoli] Ne tagliò in fette grandissimo numero.
  Quando vogliamo beffare un bravazzone codardo, sogliamo dire: \textit{Gran
  danno che farebbe costui in un'orto di cavoli, o di raduchi}, E quel detto \textit{Beati Pavoli},
  ha origine da un Montanbanco, il quale vendeva il rimedio contro a' veleni con
  dichiarazione di voler donare (come effettivamente donava) la pietra di S.Paolo
  a tutti coloro, che havevano nome Paolo, onde infiniti plebei per buscar quella
  pietra dicevano di haver nome Paolo; sicché egli cominciò ad esclamare
  O quanti Paoli, o quanti Paoli. E perché quelli, che ottenevano quella pietra
  si tenevano fortunati per haver havuto il regalo, ne nacque il dettato. \textit{Son più
    che non furono i Paoli Beati}, che vuol dire, furon moltissimi; Che la voce \textit{Beati} in
  questo caso è sinonimo della voce \textit{felice}, o fortunato, \textit{Beato voi che siete ricco}, per
  Felice, o Fortunato voi, che siete ricco.
\end{description}
\section{Stanza XXX}
\begin{ottave}
  \flagverse{30}Così piena di fumi, ed umor bravi\\
 Che te l'hanno cavata di Calende,\\
 Rivolge l'occhio al popol delle navi,\\
 Là dove Brescia romoreggia, e splende,\\
 E va per infilzarne sette ottavi:\\
 Ma nel pensar di poi, che se gli offende\\
 Far non porrebbe lor, se non mal giuoco;\\
 Gli vuol lasciar campare un'altro poco,
\end{ottave}

 Celidora facendo queste sue bizzarrie, vede la gente di Baldone, ed essendosi
inferocita in quei cavoli, gli vien voglia di far io stesso in quelle genti, ma si
rattien di farlo per non dar loro disgusto, e per lasciargli campare un'altro poco.
\begin{description}
\item[PIENA di fumi, che te l' hanno cavata di Calende] Mostra il Poeta, che
  Celidora sia poco meno, che briaca in questa sua bravura, i fumi della quale le
  habbiano offuscato il cervello, come fanno i fumi del vino a chi troppo beve, che
  questo intende dicendo l'hanno \textit{cavata di calende}, ed è quelio che i Latini dicono
  \textit{extra callem esse}, ed io credo che da questo Latino \textit{callem} venga la corruttela di calende; e per parlare Ianadattico detto sopra in questo C. stan. 28. si voglia dir \textit{cavata
  del calle} per intendere (come facevano i latini) Cavata di Cervello.
\item[BRESCIA romoreggia, e splende] Si sente romor d'armi, e si vedono risplender
  le medesime. A Brescia si fabbricano buone, e belle armi, e però il Poeta
  pigliando La Città per L'armi, che in quella si fabbricano, seguita l'uso nostro, che
  è di dire \textit{Il tale ha tutto Brescia addosso}, per intendere \textit{Ha molt'armi addosso}.
\end{description}

\section{Stanza XXXI \& XXXII}
\begin{ottave}
\flagverse{31}Al fin, deposto un'animo sì fiero,\\
In genio cangia a poco a poco l'ira,\\
E' come un'orsacchin, c'a pié d'un pero\\
A bocca aperta i pomi suoi rimira;\\
Ferma impalata quivi com' un cero\\
Fissando in loro il sguardo, sviene, e spira,\\
Ne può viver al fin se non domanda\\
Ove l'armata vada, e chi comanda.

\flagverse{32}S'abbocca appunto con Baldone steffo,\\
E sentendo ch'  egli ha tal gente fatte\\
Per rimeiter in sesto, ed in possesso\\
Una Cugina sua ch'è per le fratte,\\
Ben ben lo squadra, e dice: Egli è pur desso!\\
Or su ch'io casco in piè, come le gatte,\\
Ed esclama di poi: quest'è un'azione,\\
Che veramente è degna di Baldone.
\end{ottave}

Celidora pero appiacevolitasi, si ferma a guardar con gusto grandissimo quei
Soldati, e domanda di chi è l'Armata, e chi la comanda; e s'abbatte a domandarlo
a Baldone, il quale gli dice, che ha fatto quella gente per aiutare una sua
cugina, ond'ella riconosciuto Baldone, si rallegra, e dice: veramente questa è
un'azione degna di Baldone.
\begin{description}
\item[CANGIA l'ira in genio] Cioè dove prima haveva l'animo d'infilarne sett'ottavi,
  adesso comincia ad haver genio con loro, ed a portargli affetto. Questa
  voce genio se ben non pare che Toscanamente significhi cosa alcuna, nondimeno
  è molto usata dicendosi \textit{Huomo di buon genio}, o \textit{di cattivo genio} per intendere Huomo
  di buona, o cattiva indole, o inclinazione. \textit{Haver genio con uno} È lo stesso
  che Haver simpatia con uno. Appresso i Latini pure se ben genio non si distingue,
  va dall'anima ragionevole, e molti lo pigliassero spesso per Lares; altri per gli
  Dei Penati, altri per il Dio del piacere, altri per li quattro elementi, altri per li
  dodici segni del Zodiaco, altri per lo Dio che faceva nascere,  ed altri per diverse
  altre cose; tuttavia essi pure se ne servivano per intendere inclinazione, come
  ci mostra Plauto in Truculento 1, 2. \begin{verse}cum genijs suis belligerare, ec. idem quod defraudare genium.\end{verse}
\item[COME un'orsacchino a piè d'un pero] Si dice L' orso sogna pere; Leva le peres
ecco l' orso, Dal-che si cava, che questo animale sia molto ghiotto delle pere; il
be anche atiefta Vincenzo Martelli nel suo Capitolo in lode delle menzognes
jicendo: \begin{verse}
  Oggi a voi più ch' ad altri si conviene,
  Benché noi siam tant' orsi a queste pere, ec.
\end{verse}
E si dice che in rimirarle gioisca tutto per la sola speranza di conseguirle; e
perciò l'Autore assomiglia Celidora a un picciolo Orso a pie d'un pero, perché in
veder quella gente, la quale ella spera che sia per lei, si rallegra, gode, e brilla,
come fa l'orso stando a pié del pero, vagheggiando le pere.
\item[FERMA impalata quivi come un cero] Per esprimere la stpidità nella quale si
trova Celidora nel vedere quei Soldati, l'Autore dopo haver detto che \textit{stava a
bocca aperta come fra l'orso a pié del pero}, soggiunge \textit{che ella stava impalata, come un
cero}, cioè ritta ritta, e fermata nel posto, come stavano quelle torrette, fatte di
carta, o di panno, o di tavole, che la mattina di S. Gio.\ mettevano li nostri
antichi attorno alla piazza del Tempio di S. Gio. Batista, entro alle quali stava
un'huomo, che le moveva, e queste le domandavano \textit{ceri} secondo che dice Goro
Dati\footnote{Gregorio Dati, 1362-1435, mercante fiorentino.} nei suoi discorsi Storici lib. 6. in fine. Hoggi in vece di tali torrette portano
in due, dello Spedale del Bigallo; sopr' alle spalle processionalmente, uno sgabellone,
sopr' al quale è fermato un gran cero fatto di legno, per sfuggire il pericolo
di romperlo sendo di cera, e faranno 26. o vero trenta ceri, che manda
detto Spedale per tributo al detto Tempio di S. Gio. Batista. Si può anche dedurre
questa similitudine da quei poveri Cristiani, i quali da i Turchi sono impalati,
che verisimilmente stanno intirizzati, e come l'Autore vuol che s'intenda,
che stesse Celidora.
\item[SVIENE, e spira] Svenire vuol dir Perdere i sentimenti, e Spirare vuol dire
Esalar l'anima, sicché si possono dir quasi sinonimi, ma in questo luogo il verbo
\textit{spirare} significa \textit{Ustolare}, che vuol dir Guardare con desiderio di conseguire,
come fa uno che havendo grandissima fame, stia a vedere un che mangi, ed habbia
d'avanti molte vivande; Vedi sotto C. 14, stan. 34.
\item[ABBOCCARSI] Trovarsi, o abbattersi in uno per parlargli. \textit{Io non son ben'
informato di questo negozio, ma m'abboccherò col tale, che m'informerà}.
\item[E' per le fratte] È rovinato. È per la mala. Quello che i latini dissero \textit{De
eo actum est}. \textit{Fratta}. S'intende Borroncello, o Macchia, che suol render' aspro
un paese, e vien dal Greco Frattin che suona Far siepe.

\item[BEN ben lo squadra] Lo guarda benissimo, che la forza della replica è di far
  nascere il superlativo, come accennammo sopra in questo C.\ stan.~11. Ed il verbo
  squadrare, che vuol dir Misurar con la squadra, significa Considerare, e
  Guardare un' oggetto minutamente, e con diligenza.
\item[CASCARE in pie come i gatti] Ottener da un male, o da un cattivo accidente,
  un bene impensato; che i latini dissero \textit{excidere extra mala},
\end{description}

\section{Stanza XXXIII \& XXXIV}
\begin{ottave}
  \flagverse{33}Maravigliato allora il Sir d'Ugnano,\\
  E chi sei (disse) tu, che sai il mio nome?\\
  Io ti conosco già di lunga mano,\\
  (Ella rispose) e acciò tu sappia il come,\\
  Celidora son'io del Re Fioriano\\
  Fratello d'Amadigi di Belpome,\\
  E con tutto, che già sien' anni Domini\\
  Ch'io non ti viddi, so come ti nomini.

  \flagverse{34}S'ell'è (dic' ei) così noi siam cugini,\\
  E subito si fan cento accoglienze,\\
  Ed ella a lui ne vende mill' inchini,\\
  Egli altrettante a lei fa riverenze,\\
  Così fanno talor due fantoccini\\
  Al suon di cornamusa per Firenze,\\
  Che luna incontro all'altro andar si vede\\
  Mosso da un fil, che tien, chi suona, al piede.
\end{ottave}

Baldone, e Celidora si riconoscono per cugini, e si fanno molte accoglienze.
\begin{description}
\item[CONOSCER di lunga mano] Conoscer di gran tempo. \textit{Lunga mano d'anni}
tanto suona quanto Lunga serie d'anni, o gran quantità d'anni, che diciamo
anche \textit{È un gran pezzo ch'io ti conosco}.
\item[BALDONE, Celidora, e Amadigi] sono nomi a caso:, ma l'\textit{Infante Floriano} è
anagrammatico, da \textit{Raffaello Fantoni}.
\item[SON' anni Domini] Son' anni infiniti. Sono tanti anni, quanti sono dalla nascita
  di Nostro Signore che diciamo Anno Domini. iperbole usatisima in Firenze.
\item[ACCOGLIENZA] Ricevimento con amorevolezza, e cortesia, e con una
  certa dimostrazione d'affetto, che s'usa verso le persone grate. Vien dal Latino
  \textit{Colere}, che esprime Amar con riverenza, ed honore.
\item[INCHINO] È lo stesso che \textit{riverenza} facendosi con abbassar la testa, e piegare
  le ginocchia, ed è proprio delle Donne; \textit{Riverenza} si fa con abbassar la testa,
  e piegandosi un sol ginocchio si manda l'altra gamba addietro a foggia di
  genuflessione, ed è propria degli huomini, come si vede nel presente luogo, che
dice,\begin{verse}
Ella a lui ne rende mille inchini;
Egli altrettante a lei fa riverenze,
\end{verse}
\item[COSÌ fanno talor due fantoccini] Suol' andar per Firenze un contadind, suonando
  una cornamusa, e porta alcune figurine di legno, che hanno le congiunture
  delle membra mastiettate, e contrappesate con piombo in modo, che si
  muovono per ogni verso; queste infilza per lo petto in una sottilissima corda da
  chitarra, o diciamo minugia, la quale da una parte lega ad uno de' suoi ginocchi,
  e dall'altra ad una tavoletta posta in terra a tal fine, e col muovere quella gamba,
  alla quale è legata la corda; fa, che quelle due figurine infilzatevi ballano al tempo
  del suono della cornamusa. Intesa dunque questa operazione, che fanno i due figurini,
  s'intende ancora come facessero fra di loro questi due parenti.
\item[CORNAMUSA] Zampogna doppia, composta d'un basso perpetuo, e di un
  soprano, che canta le note come gli altri Zufoli, e si da il fiato ad ambedue con
  un sacco di quoio, da colui che suona, ripieno di vento: col soffiare in un piccolo
  cannello animellato; ed il suonatore premendo col braccio il detto sacco da il
  fiato a dette due Zampogne.
\end{description}

\section{Stanza XXXV}
\begin{ottave}
  \flagverse{35}Poi che le fratellanze, e i complimenti\\
Furon finiti, a lei fece Baldone\\
Quivi portar un po di sciacquadenti,\\
O volete chiamarla colezione,\\
Hor mentre, ch' ella scuffia a due palmenti\\
Pigliando un pan di sedici a boccone,\\
Si muove il campo, e sott'alla sua insegna\\
Ciascun passa per ordine a rassegna.
\end{ottave}

Dopo finite le cirimonie Baldone fa portar da bere, e da mangiare, e mentre
che Celidora mangia, si fa la mostra de' Soldati.
\begin{description}
\item[FAR le fratellanze] È tratto dall'uso che nelle nostre Compagnie, ò Confraternite
  di secolari, nelle quali a i tempi determinati si vanno tutti ad abbracciare
  l'uno con l'altro; e questa azione dicono \textit{Far le fratellanze}, E da questo
dunque intendi dopo finiti gli abbracciamenti e le cirimonie.
\item[SCIACQVADENTI] Quel che significhi lo dichiara il Poeta medesimo dicendo;
  \textit{O volete chiamarla colazione}. Che vuol dire parcamente cibarsi fuor del desinare,
  e della cena, e viene dal Latino \textit{collectio prandij vel coenae}. Ma siccome son
  diversi li pasti che si fanno in Firenze, così son diversi li nomi che loro danno. Il
  primo mangiare che si fa fra l'alba, e il mezzo giorno si chiama \textit{Asciolvere}, ed
  alle volte colazione. Quello, che si fa a mezzo giorno fi chiama \textit{desinare}. Quello
  che si fa tra 'l mezzo giorno, e la sera si dice \textit{Merenda} quali \textit{meridie edenda}.
  Quello della sera si dice cena, ed allora che per il digiuno la sera si mangia poco
  si dice colazione; E la voce \textit{sciacquadenti} vuol veramente dire Quando si mangia
  qualche poco, per bere con gusto.
\item[SCUFFIARE] Mangiar con ingordigia, o divorare. È voce Fiorentina, ma
hoggi usata solo per scherzo, e vien forse da \textit{Scuffina} che è una raspa, o lima da
legno detta così, perché adoprandola leva molto legno per volta, e per questo è
chiamata anche \textit{ingordina}.
\item[A due palmenti] Da ambedue ganasce: Traslato dal Molino, che si dice
  \textit{Macinare a due palmenti} quando Due ruote lavorano; che \textit{palmento} vuol dire
  tutta la macchina, che fa macinare, dicendosi Molino \textit{d'un palmento}, o di \textit{due
    palmenti}, quando Un molino ha una, o due macini. E stimo che si dica \textit{Palmento},
  quasi Palamento, perché le ruote, che fanno andar le macine son composte di
  tavole a foggia di pale per prender l'acqua, che le fa girare.
\item[UN pan di sedici, ec] Con questa iperbole esprime l'ingordigia di Celidora;
  perché per altro un pane di sedici de' nostri quattrini malamente si può consumare
  anche con sedici bocconi, intendendo \textit{Boccone} quella quantità, che l'huomo
  può pigliar dentro alla bocca in una volta.
\item[PASSAR a rassegna] Quando i Soldati si portano avanti, al loro Capitano,
  e fanno scrivere il lor nome si dice \textit{Passar a rassegna}. E qui Baldone come supremo
  Capitano per fare honore alla cugina, Fa la rassegna, nominando, però solamente
  gli Ufiziali prinicipali; il che pare che più propriamente si dica \textit{Dare}, o
  \textit{far la mostra}, Vedi sotto C, 2. stan. 36.
\end{description}

\section{Stanza XXXVI}

\begin{ottave}
  \flagverse{36}E per il primo viensene in campagna\\
Pappolone il Marchese di Gubbiano,\\
Colui, che nel conflitto della Magna\\
Estinse il Gallo, e seppellì il Germano;\\
È la sua schiera numerosa, e magna,\\
E perch'egli è Soldato veterano,\\
Ha nell'insegna una tagliente spada,\\
Ch'è in pegno all'osteria di mezza strada
\end{ottave}

L'Autore in questa sua Opera mette una mano d'amici suoi sotto nomi anagrammatici,
la maggior parte de' quali è nominata in questa mostra, che Baldone
fa dell'esercito, descrivendone alcuni con qualche loro azione, ò con un'epilogo
della loro vita oltre all'Anagramma. Il primo che viene in mostra e Pappolone,
cioè \textit{Paolo Pepi} anagramma proprio, perché questo gentilhuomo era giovanotto
grande di persona, e grasso, e mangiava assai; e per questo il Poeta lo
dice \textit{Pappolone}, che vuol dir gran mangiatore. Vedi sotto C. 6. stan. 70.,  e lo fa
\textit{Marchese di Gubbiano}, che è un Castello; e Ingubbiare (detto però plebeo) significa
Empier il ventre. Dice \textit{nel conflitto della Magna}, cioè Nel mangiare, se ben
par che voglia dire in una sanguinosa battaglia seguita in Alemagna. Estinse il
Gallo, e seppellì il Germano; par che dica ammazò Francesi, e Tedeschi, ma vuol
dire ch'ei mangiò galli, e germani; e gli fa fare per insegna una spada impegnata
all'oste di mezza strada, che è un'osteria fuor di Firenze un miglio, e così
mostra, che ogni fine di questo tale era il mangiare.
\section{Stanza XXXVII}
\begin{ottave}
  \flagverse{37}Bieco de Crepi Duca d'Orbatello\\
Mena il suo terzo c'ha il veder nel tatto,\\
Cioè perch'ei da un occhio sta a sportello,\\
Soldati ha preso c'hanno chiuso afatto,\\
Son l'armi loro, il bossolo,e il randello,\\
Non tiran paga, reggonsi d'accatto,\\
Soffiano, son di calca, e borsaiuoli,\\
E nimici mortal de' muricciuoli.
\end{ottave}

Segue dopo Pappolone \textit{Bieco de Crepi}, cioò Piero de Becci huomo di faccia non
troppo bella, con occhi biechi, e lusco, e però il Poeta con l'equivoco \textit{d'orbo},
che vuol dir mezzo cieco, come vedemmo sopra in questo Cant. stanza 9., lo fa
\textit{Duca d'Orbatello}, e dice, che vedendo egli alquanto, ha preso per Soldati gente,
che è affatto cieca, avverando il detto. \textit{Beati Monoculi in terra caecorum}. Hanno
questi soldati il bossolo, e il bastone, non tirano paga, ma vivono di limosine,
son tutti spie, ladri, monelli, e nimici de' muricciuoli.
\item[UN terzo] Numero di soldati comandati da pil capitani, e dal Colonnello;che
i Latini dicevano /egionem, ed il Colonnello forse era Tribunus,
\item[MENARE] Condurre, Ma qui sta proprio il verbo Menare secondo il pro-
verbio che dice: Solo tciechi si menano, —~;

\item[HA il veder nel tatto] U ciechi non hanno altra vista, che il tatto, el odorato
nelle cose corporee, e materiali; e 1' udito nell' incorporee.

\item[STA a spertello] Intende mezzo cieco. Metafora tolta da quelle botteghe; le
gualt quando non è fefta intera, e comandata stanno mezze aperte, che si dices
Star' a sported, perché aprono folo quella parte del legname, che si chiama se
tello; e seguita la metafora dicendo: Su/dati ha preso channo.chiufo affatto: cioè s0-
no affatto ciechi. Varchi Stor. Fior. lib.~11. dice: Won si tennero le botteghe Aperte,
ne a sportello, ma chinfe affatto, 4 j:

\item[BOSSOLO] E' quel valoa foggia di calice, col quale si raccolgono i voti ne-
gli Squittini. Vedi sotto Cant.\ 6. stan.\ 109., e per la similivudine intendiamo quel
valo di latta, di rame, d' ottone, o d' aitra materia, che e usato da i ciechi per
ricevervi l'clemofine, ay

\item[RANDELLO] Intende Quel bastone, che adoprano i ciechi per farfila strada.
  Se ben randello s'mmtende un Pezzo ci bastone grosso quanto quello de' ciechi,
  ma assai pil corto, che s' adopra per stringere le legature delle baile, che però
  tale operazione si dice \textit{Arrandellare}.

\item[REGGONSI d'accatto] 1 verbo Reggersi in questo laogo, ed in questi termini
vuol dir Cavar il guadagno per mantenersi: M tale si regge col far' il farto, Cive
vive col guadagno, che cava dal far' il farto, ec. 4

\item[SOFFIARE] In lingua furbesca vuol dir Far la spia, se bene è inteso comunemente.
  Ed il Poeta parlando di ciechi, i quali hanno per costume di parlar furbesco,
  e serve di questa, ed altre lor parole, come \textit{Esser di calca}, che vuol dir
  Huomo da far qualsivoglia furfanteria, e viene dalla voce \textit{Calcagno}, che in
  lingua furbesca vuol dir \textit{Monello}, cioè ladro di calca nella quale entrano per rubar
  le borfe, e di qui si dicono Borsaioli, e Taglia borse. Vedi sotto C. 6. stan. 64.

\item[NIMICI de' muricciuoli] Chiamiamo muricciuoli quel pezzo di muro, che avanza
  sopr'a terra attorno alle case; d'altezza d'un braccio più, o meno, e di
  simile larghezza; fatto, o per uso di sedere, o per difesa de i fondamenti. Di
  questi sono nimici i ciechi, perché spesso vi Pp jotono dentro co' i piedi, ingannati
  dal sentir al viso, ed alle mani l'aria libera, il che fa lor credere, che non
  possa esservi impedimento veruno anche in terra.
\end{description}
\section{Stanza XXXVIII}
\begin{ottave}
  \flagverse{38}La strada i più si fanno col bastone,\\
Altri la guida segue d'un suo cane,\\
Chi canta a più d'un'uscio un'Orazione,\\
E fa scorci di bocca, e voci strane;\\
Chi suona il ribecchin, chi il colascione;\\
Così tutti si van buscando il pane.\\
Han per insegna il diavol de' Tarocchi,\\
Che vuol tentar un forno pien di gnocchi.
\end{ottave}

Descrive il modo del marciare di questi ciechi, e fa lor fare quei gesti, ed operazioni,
che son soliti fare andando a cercare elemosine, Dice che \textit{I più si fanno
la strada col bastone; altri si fanno guidare a un cane, ed altri vanno cantando Orazioni
a pié d'un'uscio}; E questi son ciechi stipendiati dalle persone pie, acciocché ogni
giorno, o ogni settimana vadano alle case delle medesime persone a cantare un'orazione
avanti al loro uscio, dove per esser sentiti fanno voci strane, cioè Gridano
forte, e fanno \textit{brutti scorci di bocca}; E questo avvien loro perché, per lo più,
li ciechi oltre alla loro cecità, sogliono havere altri stroppi nella faccia. Molti
suonano il ribechino, cioè il violino, altri il \textit{Colascione}: questo strumento (che da
i più è detto corrottamente \textit{Ganascione}) E' un corpo, come quello della tiorba,
con manico lungo, con due sole corde, il quale si suona con un pezzo di suolo
da scarpa, che volgarmente si dice taccone; E perciò tale strumento è detto anche
Tiorba a Taccone da Filippo Scrutendio da Scafato\footnote{Felippo Sgruttendio de Scafato. Ignoto, forse anagramma di persona reale, in vita nel 1646.}, il quale così intitola il
suo grazioso Canzoniero Napoletano. Alcuni furbi per \textit{colascione} intendono la
forca, perché ancora a questo s'adoprano due corde, la grossa, e la sottile, come
alla forca. Questi ciechi suonatori soglion sempre andar vendendo qualche
Orazione, o Rappresentazione, o altre Leggende, e così tutti si vanno buscando
il pane, cioè guadagnano da vivere. E volendo il Poeta mostrare quanto la gente
di questo terzo sia affamata, le da per insegna un diavolo, che tenta un forno
pieno di gnocchi; e mostra che sia sempre intenta a procacciarsi il vitto con ogni
sorta d'invenzione, che il verbo tentare significa Procurare, o Provarsi di far
una tal cosa, e si deduce, che questo diavolo \textit{tentasse}, cioè si provasse a rubar da
quel forno il pane, che vi era dentro. E per \textit{gnocco} intende Ogni sorte di pane;
Se bene \textit{gnocco} è quella specie di pane, che dicemmo sopra in questo C. stan. 3.
\begin{description}
\item[SCORCI di bocca, e voci strane] Voci strane, e bocche diverse dal naturale;
perché se bene la voce \textit{scorcio} è termine di prospettiva, che mostra la figura esser
resa capace della terza dimensione del corpo; s'intende anche per positura di corpo,
o parte d'esso diversa dal naturale.
\item[TAROCCHI] Carte, con le quali si giuoca alle Minchiate. Vedi sotto C.8. stan.
  61. in una delle quali carte al num. 14. è effigiato un Diavolo; e questo dice, che
  \textit{tenta il forno pien di gnocchi}. Il nostro Poeta haveva dato a questi Ciechi l'impresa
  del Buio, come si vede in alcuni suoi sbozi, che diceva.
  \begin{verse}
    Hanno un' impresa, dove Bieco mette
    Il buio che a svegliar va le Civette.
  \end{verse}
\end{description}

\section{Stanza XXXIX, XXXX, XXXXI}

\begin{ottave}
  \flagverse{39}Dietro al Duca, c'ognun guarda a traverso\\
Vanno cantando l'aria di Scappino,\\
Ma non giunsero al fin del terzo verso,\\
Che venuto alla donna il moscherino,\\
Fatto a Bieco un rabbuffo a modo, e verso,\\
Gli disse: S'io v'alloggio dimmi Nino,\\
Perch'io non veddi mai in vita mia\\
Pigliar i ciechi fuor c'all'osteria
\end{ottave}

\begin{ottave}
  \flagverse{40}Signora, rispos'egli, benché cieca,\\
Fu però sempre simil gente sgherra;\\
Con quel batocchio zomba a moscacieca\\
Senza riguardo, come dar' in terra;\\
Sort'ogni colpo intrepida s'arreca,\\
Che non vede i perigli della guerra:\\
E' cieca è ver, ma pur il pan pepato\\
E' più forte, se d'occhi egli è privato,

  \flagverse{41}Ovvia (diss'ella) tocca innanzi il cocchio,\\
E se costoro a guerreggiar son'atti\\
Tienteli pure, e non mi star' a crocchio,\\
Mentre gli è tempo qui di far di fatti.\\
Va dunque o forte, e invitto bercilocchio,\\
Che i nimici da te saran disfatti,\\
Perch' in veder la tua bella figura\\
Cascan morti, senz'altro, di paura.
\end{ottave}

Questi ciechi andavano dietro a Bieco cantando l'aria di Scappino, (che e una
canzonetta, la quale cantavano i ciechi in Piazza del G. Duca, quando l'Autore
principiò la presente opera) ma Celidora adirata di ciò, dice a Bieco, che
non vuol tal gente, ed egli rispose, che se bene eran ciechi eran però fieri, che
il non vedere i pericoli gli rendeva arditi, e forti, come appunto è il pan pepato,
che è più forte, quando non ha occhi; ond'ella gli dice, che se gli tenga, e vada
allegramente, che ella ha speranza di cavar frutto da lui solo senza loro, perché
stima, che il nimico sia per cascar, morto subito, che vedrà il suo brutto viso.
\begin{description}
\item[GVARDA a traverso] Uno che ha gli occhi scompagnati, come haveva Bieco
  diciamo Guardare a traverso. Vedi sopra in questo Cant. stan. 9. \textit{Transversa tuentibus
  hirquis}, Virg. Egl. 3.
\item[VENUTO alla donna il moscherino] La donna, cioè Celidora, s'adirò. Si dice
  \textit{Venire il moscherino al naso}, perché si trovano alcune piccole mosche, le quali
  volando, talvolta entrano nel naso altrui, e toccando quella parte così sensitiva,
  danno grande alterazione, e mettono l'huomo in una subita impazzienza, e stizza.
  Si dice ancora \textit{Venir la senapa}, o \textit{la Mostarda al naso}, perché nel mangiar la
  mostarda (che e un'intingolo fatto di senapa, e mosto cotto) quando è ben carica di senapa,
  viene al naso un certo pizicore, che forza a, lagrimare. Si dice
  anche \textit{Venir la muffa}, o altri puzi odiosi, e sporchi, come si dice sotto C. 4. stan.
  23. E tutti significano Venir collera.
\item[FATTO un rabbuffo] Bravato. Fare un rabbuffo, o Rabbuffare vuol dire Riprender
  uno con minacce, o Spaventarlo con asprezza di parole. Il Landino nell'esposizione
  a Dante C. 7. dell'Inferno alla parola Buffa, e Rabbuffare dice: \textit{Ma
  proprio Buffa è vento, onde diciamo Buffettare chi getta vento, per bocca,e Sbuffare, quando
  con suono di parole, o a dir meglio Con ventose, ed enfiate parole alcuno minaccia.
  Di qui diciamo Rabbuffare, Conturbare e muover le cose dell'ordine loro, e scompigliarle
  e chiamiamo Rabbuffo, quando Con parole conturbiamo, e Scompigliamo la mente d' uno}.
  Vedi sotto C. 3. stan. 57, la voce \textit{Buffi}.
\item[A modo, e a verso] Con tutta perfezione. B il latino \textit{modis, \& formis}.
\item[DIMMI Nino] Dimmi pazzo, e senza Cervello, come fu Nino, il quale per lo
  grande amore, che portava a Semiramide sua Meretrice o moglie, le concesse
  che per un giorno ella fusse assoluta Regina, ed ella in quel giorno lo fece ammazzare,
  e si confermò Regina per sempre, come si legge in Plutarco in Serm.
  Amator.
\item[PIGLIAR i ciechi fuor c'all'osteria] Quand' uno vince assai, sogliamo dirgli: \textit{Si
  torrà i ciechi}, e s'intende \textit{all'osteria}. E questo perché si suppone, che quel
  tale, che vince per l'abbondanza del denaro venutogli in mano fenza
  fatica, sia per spenderlo profusamente in pigliarsi tutti li suoi gusti fino con
  l'andare a cena all'osteria, e chiamare alla sua mensa a suonare alcuni ciechi, i
  quali in su l'hora del mangiare vanno girando, per l'osterie a tale effetto, e questi
  sono i Ciechi, li quali Celidora dice haver veduto pigliare all'osterie.
\item[SGHERRO] Bravo. Ammazzatore; Tagliacantoni. Vedi sotto, Cant. 3. stan. 42.
\item[BATOCCHIO] Quel bastone, col quale si fanno la strada i ciechi si chiama
\textit{Batocchio} dal batterlo in terra, che fanno i ciechi, per farsi riconoscere per quel
battere da gli altri ciechi. E però vuol dire anche il Battaglio delle Campane.
\item[ZOMBA] Perquote, bastona. Vedi sotto C.~6. stan.~104., e C.~11. stan.~28.
\item[MOSCA cieca] Il giuoco detto Mosca cieca è trattenimento da Fanciulli, che
  deriva dall'antico, e si diceva \textit{Musca aenea}, e si faceva nel modo, che usano
  hoggi, che è in questa maniera.

  Tirano le sorti fra più ragazzi a chi debba bendarsi gli occhi, (che in questo
  giuoco dicono Star sotto) ed a quello, a cui tocca, sono bendati gli occhi in modo,
  che non possa vedere, e poi con uno sciugatoio, o altro panno avvolto, che ciascuno
  tiene in mano, si danno da gli altri delle percosse a colui, che è sotto, ed egli
  così alla cieca va rivoltandosi, e quello che egli arriva con la percossa deve bendarsi
  in vece del, percuziente, il quale si leva la benda, e va fra gli altri a percuotere
  il nuovo bendato; Quello, al quale di mano in mano tocca a star sotto, mena
  senza riguardo, colpi spietati, sì perché commosso da tanti colpi vorrebbe
  vendicarsi, sì anche perché, cogliendo, il colpo sia in modo da non poter'
  esser negato, procurando ognuno di non toccarne, e d'occultarla, se può,
  quando l'ha toccata, per non haver' a stare in quel martirio, in che è colui, che
  sta sotto. E però dice \textit{Zomba a mosca cieca senza riguardo come dare in terra}.
  Si dice \textit{mazzate da ciechi} per intendere Percosse spietate.

\item[IL Pan pepato è più forte se d'occhi egli è privato] Si suole in Firenze per la sesta
  di tutti i Santi fare un certo pane che da noi si dice \textit{Pan pepato}, il quale è
  composto di sapa, aceto, farina, pepe, ed altri aromati, e mescolanvi pezzetti di
  bucce di poponi, zucche, cedri, e d'aranci conditi in zucchero, o miele, li quali
  pezzetti, quando il pane si taglia, restano nella tagliatura a similitudine d'occhi,
  e perciò da i nostri Fanciulli son chiamati Occhi; E cavandosi dal pane tali
  occhi, che sono dolci, il pane resta \textit{più forte}, cioè più acido; ed il Poeta si serve
  della parola Forte in significato di Gagliardo, dicendo che i ciechi sendo senz'occhi
  son più forti, ed intende gagliardi, scherzando con questo equivoco di forte.
\item[TIRA innanzi il cocchio] Seguita il tuo viaggio, e tanto s'intenderebbe a dir
solamente \textit{tira innanzi} senza porvi l'aggiuata \textit{Cocchio}, ma il Poeta ve lo pone
per seguitar l'uso Fiorentino.
\item[STAR a crocchio] Il verbo \textit{Crocchiare}, e la frase \textit{stare a crocchio}
  significano Cicalare, o Ciarlare di cosa di poco frutto, o importanza per finire il giorno.
  Onde questi tali si dicono \textit{Crocchioni}, \textit{Cicaloni}, \textit{Perdigiorni}, e simili.
  Vedi sotto Cant. 3. stan. 5. Questo verbo \textit{Crocchiare} serve anche per intendere Dar
  delle buffe. Vedi sopra in questo Cant. stan. 10.
\item[BERCILOCCHIO] Epiteto composto dal Poeta, che vuol dir Bircio di che
sopra in questo Cant. stan. 9.
\end{description}
\section{Stanza XXXXII \& XXXXIII}
\begin{ottave}
  \flagverse{42}Ne Segue intanto Romolo Carmari
Cavalier di valore, e di gran fama;
Ma sfortunato, perché coi danari
Giuocando egli ha perduta anco la dama.
Con le pillole date a suoi erarj
L'affetto evacuò l'Arpia ch'egli ama.
Tal che senz'un quattrino ammartellato
Alla guerra ne va per disperato.

  \flagverse{43}Dop'un'insegna nera che v'è drento,
Cupido morto con i suoi piagnoni
Marciar si vede un grosso Reggimento,
Ch'egli ha d'innumerabili tritoni,
Al cui arrivo ugnun per lo spavento
Si rincantuccia, ed empiesi i calzoni,
E da lontano infin dugento leghe
S'addoppiano i ferrami alle botteghe.
\end{ottave}

Segue \textit{Romolo Carmari}, Questo fu un Fiorentino, del quale non stimo bene scioglier
l'anagrammma, e dirne il nome. Questo Gentilhuomo havendo durato un
gran tempo a godere una sua Meretrice, e spesovi molto danaro, o gli fu tolta,
o ella non lo volle più perché egli abbandonò lo spendere; come è proprio di
simili donne; e ciò esprime il Poeta in quei due versi.
\begin{verse}
Con le pillole date a suoi erarj
L'affetto evacuò l'Arpia ch'egli ama.
\end{verse}
I quali versi suonano: L'havergli fatta votar la borsa fece disperdere l'amore,
che ella fingeva di portargli, Onde egli disperato, se ne va alla guerra; e
mostra questo suo spento amore nell'insegna, che egli porta, in cui è dipinto
Cupido morto, che ha d'attorno i suoi piagnoni. E perché questo Signore era
nel vestire positivo, e senza boria alcuna, anzi più tosto abbietto, il Poeta fa,
che egli conduca un reggimento di gente mal vestita, e questi huomini chiama
\textit{Tritoni}, perché Huomo trito, o Tritone tanto vale appresso di noi quanto dire
Huomo mal vestito; E questa gente per esser così mal vestita e stimata una schiera
di Monelli, e di Ladri, e perciò è causa, che s'accrescano i serrami alle botteghe,
e che ognuno fugga per la paura, che ha di loro.
\begin{description}
\item[DAMA] Vuol dir Donna nobile, venendo dal Greco \textit{Damar}, secondo alcuni;
e suona Signora dal Francese Dame, Madame, cioè Signora, mia Signora; ma
si piglia anche per l'amata, come è preso nel presente luogo.
\item[CON le pillole date a suoi erarj] Con l'evacuatorio dato alla sua borsa, cioè con
avergli fatti finire i danari mandò via dal suo corpo la bile amorosa, cio' lasciò
d'amarlo.
\item[L'Arpia] Intende Meretrice, ed esprime una donna rapace, come sono le
  Meretrici (che Arpia in Greco suona come Rapace) e quali sono figurate
  Arpie, che i Poeti fingono esser tre, Aello, Ocipete, e Celeno; e le
  fanno figlie di Nettunno, e della Terra; altri figlie di Thaumante, ed
  Elettra, altri d'altre Deità; basta che se ne servivano per esprimer l'avarizia.
  Vergil. 3. AEn.
  \begin{verse}
Tristius haud illis monstrum, nec sævior ulla
Pestis et ira deum Stygiis sese extulit undis,
Virginei volucrum vultus, foedissima ventris
Proluvies, uncaeque manus, et pallida semper
Ora fame.
  \end{verse}
  E Dante nell'Inf. Cant, 13. seguitando Vergilio dice
  \begin{verse}
    \backspace Quivi le brutte Arpie lor nido fanno,
    Che cacciar dalle Strofade i Troiani
    Con tristo annunzio di futuro danno.
    \backspace Spalle hanno alate, colli, e visi humani;
    Piè con artigli, e pennuto il gran ventre;
    Fanno lamenti su gli alberi, strani.
  \end{verse}
  Questo nome d'Arpia dette a una Meretrice anche il Coppetta nel suo Capitolo
  in biasimo della Signora Ortenzia Greca dicendo
  \begin{verse}
  Arpie crudeli, infide, inique, e ladre
  da venire a fastidio a mille Rome
  Voi, la vostra fantesca, e vostra madre.
  \end{verse}
\item[AMMARTELLATO] Haver martello, o esser' ammartellato vuol dire
Quand'uno innamorato ha gelosia della cosa amata, ovvero ha qualche sdegno
con la medesima. Il Firenzuola nel suo Capitolo in lode del legno santo, chiama
pazzia l'esser'ammartellato dicendo:
\begin{verse}
\backspace Hor nuovamente vi dico che cava
Di fastidio un, che crepi di martello,
Guarda se questa è un'opera brava.
\backspace E s'i pazzi volesson provar quello,
E conoscesson la lor malattia,
Tutti ritornerebbono in cervello;
C'altro non è il martel c'una Pazzia.
\end{verse}
\item[PER disperato] La disperazione è una soverchia inquietudine, cagionata da
  grave disgusto, la quale ci leva affatto il dominio di noi medesimi.
\item[PIAGNONI] Trova spesso nelle storie Fiorentine questo nome Piagnoni, che
vuol dir Coloro che seguitavano la parte di F. Girolamo Savonarola; ma qui vuol
dir Quegli huomini, che si mettono a i mortori de i gran personaggi attorno al
cadavero, tutti coperti di nero, e con lunghi veli, ed in mano hanno uno stendardo,
o pennoncello di taffettà nero: E si dicono Piagnoni, dal piagnere che
dovrebbon fare per la morte di quel tale.
\item[MARCIARE] È il muoversi degli eserciti. Voce restata a noi dal Francese;
  e da molti si dice Marchiare, perché questi tali, vedendola scritta con l'aspirazione,
  la pronunziano all'Italiana, non si curando di riflettere che il C-H suona
  sci, e non chi.
\item[REGGIMENTO] Quantità di Soldati comandata da più Capitani, e dal Colonnello;
  e forse lo stesso, che Terzo detto sopra in questo C. stan. 37.
\item[TRITONI] Sono Dei, o Mostri Marini, i quali si dipingono ignudi, o al
più coperti d'aliga, e di qui gli huomini mal vestiti si chiamano da noi Tritoni,
quasi huomini triti, che suona Huomini vili, ed abbietti. Vedi sotto in questo
Cant. stan. 86.
\item[INCANTUCCIARSI] Nascondersi, o mettersi per i canti per non esser
veduto.
\item[EMPIESTI i calzoni] Per la paura, se li move il corpo, e gli empie le brache.
Questo detto esprime, che Quei Tritoni facevano gran paura a chi gli vedeva, non
che veramente se gli empiessero i calzoni.
\item[S'ADDOPPIANO i serrami alle botteghe] Per afficurarsi da costoro, che sono stimati
  tanti ladri, in gran tratto di paese rinforzano le serrature alle botteghe. E qui
  l'Autore dice tutto quello, che egli può, per mostrar costoro affatto birboni, e
  vera canaglia.
\end{description}

\section{Stanza XXXXIV}
\begin{ottave}
  \flagverse{44}Hor comparisce Dorian da Grilli,\\
che nella guerra e così buon soggetto,\\
Che metterebbe gli Ettori, e gli Achilli,\\
E quanti son di loro in un calcetto:\\
Scrive sonetti, canta ognor di Filli,\\
E' buon compagno, piacegli il vin pretto,\\
Rubato, per insegna, ha nel Casino\\
Il quattro delle coppe c'ha il monnino.
\end{ottave}

Segue nella mostra Doriano da Grilli che è Lionardo Giraldi. Questo gentilhuomo
fu bellissimo humore, molto dedito alla poesia burlesca, buon discorritore,
ed huomo di conversazione; e perché egli haveva per costume il dar de Monnini,
il Poeta gli fa fare per impresa Una carta da giuocare, nella quale in mezzo a
un quattro di coppe è figurato un Monnino\footnote{La bertuccia, nel mazzo delle Minchiate.}.

\begin{description}
\item[METTERE uno in un calcetto] Confondere uno, Superar' uno nel sapere, o
  nel valore, e ridurlo tanto avvilito, che si vorrebbe nasconder dentro a un calcetto,
  vilissima, e piccola parte dell'abito dell'huomo, come quella che non
  cuopre se non il piede, Questo Doriano veramente non fu mai soldato, se ben
  l'Autore dice, che egli è \textit{buon soggetto nella guerra}; ma dice così di lui, perché
  essendo egli di sua conversazione, lo sentiva spesso discorrer delle guerre con gran
  fondamento mostrandosene assai pratico.
\item[VIN pretto] Vino puro, e senza commistione d'acqua, o d'altro; e sentendosi
  in più luoghi del nostro Contado chiamarlo \textit{vino puretto}, non son lontano da
  credere, che la voce \textit{pretto} sia o figurata, o corrotta da \textit{puretto}.
\item[CASINO] Intendi quella Casa nella quale la nobil gioventi Fiorentina s'aduna
  per giuocare,
\item[MONNINO] Le carte de' Ganellini, o Minchiate hanno in se effigiate quattro
  cose diverse, che una parte hanno spade, una parte bastoni, una parte danari,
  ed una parte coppe, e tutte quattro queste specie di carte comingiano da
  uno fino a 14. Nella carta del quattro di coppe in mezzo è figurata una bertuccia
  a sedere, la qual bertuccia da noi è detta \textit{Monnino}. E questa dice il Poeta, che
  è l'insegna di Doriano; perché egli solito di dare i \textit{Monnini}, che vuol dire,
  Quand'uno parlando con un'altro, questo lo forza a dir qualche parola, che rimi
  con un'altra, che a quel tale dispiaccia; per esempio Doriano disse ad un Cherico:
  \textit{Non fu mai gelatina senza \makebox[1.5em]{\dotfill}} E qui si fermò fingendo non si ricordare
  della parola che finiva il verso; ed il Cherico, il quale ben sapeva la sentenza
  gliela suggerì dicendo: \textit{senz'alloro}, e Dorian soggiunse: \textit{Voi siete il maggior bue
  che vada in coro}. E questo si dice dare i \textit{Monnini}.
\end{description}

\section{Stanza XXXXV \& XXXXVI}
\begin{ottave}
  \flagverse{45}Fra Ciro Serbatondi il Sir di Gello\\
Che in Pindo a Mona Clio sostiene il braccio,\\
Egeno de Brodetti, e Sardonello,\\
Vasari, ch'è padron di Butinaccio,\\
Conducon tanta gente ch'è un flagello\\
Da far che le pagnotte habbiano spaccio,\\
Di cui (perch'il mestar diletta a ognuno)\\
Si pigliano il comando a un dì per uno.

\flagverse{46}Di foglio per impresa un bel Cartone\\
Insieme con la pasta egli hanno messo,\\
Dei lor Fantocci, i quali da Perlone\\
Soglion copiare, o disegnar dal gesso,\\
Nel mezzo v'han dipinto d'invenzione\\
L'impresa lor, nella quale hanno espresso\\
Su le tre hore il venticel rovaio\\
C'ha spento il lanternone a un bruciataio.
\end{ottave}

Seguitano tre gentilhuomini scolari dell'Autore; uno è Fra \textit{Ciro Serbatondi},
che vuol dire \textit{Cristofano Berardi}, quale fa Sir di Gello perché ha forse una sua
villa così detta. Dice che \textit{sostiene il braccio, a Mona Clio}, perché egli è huomo
letterato. L'altro è \textit{Egeno de Brodetti}, che vuol dir \textit{Benedetto Gori}. Il terzo è
\textit{Sardonello Vasari}, che vuol dire \textit{Alessandro Valori}, il quale fa Sig. di Botinaccio,
perché ancor'egli ha una Villa così detta. Conducono questi molta gente, la
quale comandano vicendevolmente a un giorno per uno, e perché si conosca che
sono stati tutti tre scolari dell'Autore, fa lor fare una bandiera de i fogli di quei
disegni, che hanno fatto in squola sua; Ma perché questi attesero più alle lettere,
che alla pittura, però non fecero altro acquisto in essa, che quanto bastava per
una certa infarinatura, e per saperne discorrere; egli volendo mostrare questo
lor poco profitto, fa che di lor propria invenzione ritraggano nella detta lor
bandiera una cosa invisibile, come appunto è il Vento.

\begin{description}
  \item[È un flagello] Questo termine significa Infinità, ed Abbondanza grandissima,
ed esprime un numero indeterminato. Vien, forse dai Latino, che tal volta
significa Quantità immensa. Martial. lib. 2. 30. \textit{Et cuius laxas arca flagellat opes},
parlando d'uno che havea gran quantità di danari,
\item[CHE le pagnotte habbiano spaccio] Che s'esiti, che si consumi molto pane. E pagnotta
  se bene non è voce Fiorentina, è nondimeno spesso usata.
\item[MESTARE] Qui val Ministrare, Comandare.
\item[CARTONE] I pittori chiamano Cartone Quella carta grande fatta di più
  fogli, sopr'alla quale fanno il modello di qualche grand'opera, che devono dipignere
  nel muro a fresco, o a tempera, o vero per tessere arazzi.
\item[FANTOCCI] Figure mal fatte. \textit{Pittor da Fantocci} s'intende Pittore da poco,
  appunto come da questa loro impresa vuol l'Autore, che si argomenti che fussero
  questi Signori.
\item[DAL gesso] Cioè dalle figure fatte di gesso. I pittori hanno per costume di
  chiamare dette figure di rilevo, (delle quali si servono per disegnare) col solo
  nome di gesso, senza dir figure, o statue, come si vede nel presente luogo, che
  dice disegnar dal gesso.
\item[LANTERNONE] Arnese noto, che serve a portarvi dentro il lume, e difenderlo dal vento.
\item[BRUCIATAIO] Colui che vende marroni arrostiti alla fiamma, o nel forno,
  che noi chiamiamo Bruciate, donde Bruciataio,
\end{description}

\section{Stanza XXXXVII}

\begin{ottave}
  \flagverse{47}Nanni Russa del Braccio, ed Alticardo\\
Conduce quei di Brozzi, e di Quaracchi\\
Che, perché bevon quel lor vin gagliardo,\\
Le strade allagan tutte co i fornacchi,\\
Hanno a comune un lor vecchio stendardo\\
Da farne a corvi tanti spauracchi,\\
E dentro per impresa v'hanno posto\\
Gli spiragli del di di Ferragosto.
\end{ottave}

Seguitano due altri Gentilhuomini Nanni \textit{Russa del Braccio}, che vuol dire
\textit{Alessandro Brunaccini} ed \textit{Alticardo} che vuol dice \textit{Carlo Dati}; a quali
fa condurre le genti di Brozzi, e di Quaracchi, due luoghi vicini a Firenze, ne i quali nasce
vino debolissimo, e però dice che questi soldati son mal sani; e pieni di catarro,
perché bevono quei vini deboli,  (che egli ironicamente parlando, chiama gagliardi)
che per la loro debolezza danno prima alle gambe, che alla testa.
E perché tali infermi pare che si rihabbiano, e piglino qualche vigore, quando si
trovano all'allegrie; perché fa loro portare una insegna nella quale sono espressi
alcuni di quei bagordi, gozzoviglie, ed allegrie, che già si facevano \textit{il dì di
  Ferragosto}, che s'intende il dì primo d'Agosto, venendo questa voce da Feriare
agosto, e per intelligenza di questo è da sapere, che anticamente solevansi cele
brar le ferie Augustali con grandi allegrie; e ciò si faceva forse, perché essendo
gli huomini nel maggior fervore della state, erano necessitati dal gran caldo a stare
allegramente, perché l'allegria e il primo rimedio della squola Salernitana:
\textit{Haec tria: mens hilaris, requies, moderata diaeta}. Essendo dunque molto pericoloso in quei
tempi d'infermarsi, e perciò molti giorni infausti allora si notavano dagli Egizj,
essendo vicino al Sirio, o Canicula da tutti detta pestifera, come ci mostra Stazio
lib, 1. Silvar, \textit{Illum nec calido latravit Sirius astro}, E' necessario riposarsi, bere, e
mangiare, e stare allegramente; al che consiglia nelle sue Odi Orazio più volte;
Ed habbiamo una cantilena assai praticata, che dice.
\begin{verse}
Quando sol est in Leone,
Bonum vinum cum mellone,
Et agrestum cum pipione.
\end{verse}
E perché veramente il fervore del Sol Leone, o Sirio, e allora nel maggior colmo,
sono le stagioni molto calde; e peggiori, che in tutto l'anno; onde appresso
a' Greci ancora si facevano molte allegrie, e sacrifizzj a segno, che appresso
gli Attniesi secondo alcuni il mese d'Agosto acquistò il nome d'\textit{Hecatombaeon}. Tal feste,
ed allegrie si facevano già a Firenze non solo per la detta ragione, ma ancora per
causa di alcune vittorie ottenute da i Fiorentini in quei primi giorni d'Agosto, e se
ne conserva ancora il costume, ma non si fanno tante feste, quante già si facevano,
poiché solamente si fa correr al Palio alcuni Asini: Sì che s'argumenta, che
il nostro Poeta intenda, che in questa insegna, o stendardo fusse rappresentato il
palio de gli asini, mentre dice spiragli del dì di Ferragosto, che vuol dire un poca
di memoria delle gran feste, che già si facevano in quei giorni.
\begin{description}
\item[SORNACCHIO] Sputo grosso, e catarroso, detto anche farda, Vedi sopra in
questo C. stan. 25. Monsignor della Casa nel suo Galateo dice; \textit{Di soffiamenti di
naso sporcamente, di tirar sornacchi, e sputamenti}.
\item[SPAVRACCHIO] Così chiamiamo quei pannacci, che sopra ad un palo, pertica,
  o albero si mettono per li campi a fine di spaurire i colombi, ed altri uccelli,
  Vedi sotto C. 5. stan. 49.
\item[SPIRAGLIO] Vuol dir fessura in muro, o in tetto, o imposte di usci, o di
finestre, per la quale, trapela l'aria, o lo splendore, che i Latimi dissero \textit{rima}.
In questo luogo però è inteso metaforicamente per Piccola notizia, come è assai
in uso, e forse non lontano da i Latini, che dissero \textit{Spiraculum tantum ius rei ad
me venit} per intendere io ho havuta di ciò qualche notizia,
\end{description}

\section{Stanza XXXVIII}
\begin{ottave}
  \flagverse{48}Gustavo Falbi Cavalier di petto\\
Con Doge Paol Corbi hor n'incammina\\
Gl'Incurabili tutti, e il Lazzeretto;\\
Gente, che uscia di far la quarantina.\\
Van molti a grucce, in seggiola, e nel letto,\\
Perché non sono ancor netta farina;\\
Fan per impresa in un lenzuol che sventola\\
Un Pappino rampante a una pentola.
\end{ottave}

Seguono \textit{Gustavo Falbi}, cioè \textit{Ugo Stufa} Senatore Fiorentino, e lo chiama
\textit{Cavalier di petto}, perché ha la Croce in petto essendo Bali della Religione di
S.Stefano; E l'altro è \textit{Doge Paol Corbi}, che vuol dire \textit{Cavalier Iacopo del Borgo}.
A questi due gentilhuomini fa condurre una quantità di convalescenti, e di stroppiati,
per mostrare, che essi nel tempo; che l'Autore componeva la presente Opera
non erano d'intera sanità per qualche poca d'ipocondria, che gli molestava, e
fa però lor fare per impresa un Servo dello spedale di S.Maria Nuova con le
mani alzate a una pentola.
\begin{description}
\item[INCVRABILI] Così si chiama in Firenze uno Spedale, nel quale vanno a curarsi
  i Maifranzesati.
\item[LAZZERETTO] Luogo, o Spedale in cui si mettono gli huomini, e robe
  sospette di peste per far lor fare la quarantina, e renderle praticabili, che \textit{Far la
    quarantina} vuol dire Star riserrato in uno di questi luoghi quaranta, o più, o meno
  giorni per spurgar il sospetto d'infezione. E questo nome Lazzeretto viene
  da Lazzero risuscitato da N. Sig. Giesù Cristo, quando era di già fetente il di lui
  corpo.
\item[GRUCCIA] Specie di bastone per gli stroppiati, sopra una teftata del quale
  essendo confitto un legnetto fatto a guisa di mezza luna, si sostiene il corpo mettendo
  detta mezza luna sotto il braccio, e l'altra testata del bastone in terra; e
  perché questo bastone è simile a una croce mi par di poter credere, che la voce
  Gruccia sia corrotta dal Latino \textit{scipio cruciatus},
\item[NON son netta farina] Non sono schietti, non sono affatto sani.
\item[LENZUOL, che sventola] Costoro in vece di bandiera, usano un lenzuolo, e
  ciò per mostrare, che tutte le loro cose sono da spedali; in esso lenzuolo è dipinto
  un'Astante, o Servo dello spedale di S. Maria Nuova, rampante a una pentola,
  cioè con le mani alzate a una pentola, che è in alto; a similitudine del Lione, il
  quale quando si trova dipinto ritto con le branche dinanzi alzate a qualche cosa,
  si dice Rampante. Franco Sacchetti Nov. 133, \textit{Ed hebbero ritrovato per cimiero un
  mezzo orso con le zampe rilevate, e rampanti}.
\end{description}

\section{Stanza IL \& L}

\begin{ottave}\flagverse{49}Bel Masotto Ammirato anch' egli passa\\
Lindo garzon d' ogni virtù dotato,\\
Che può, de' soldi havendo nella cassa\\
Pisciar a letto, e dire : io son sudato;\\
Ma per l'ipocondria, che lo tartassa,\\
Ei si dà a creder d'essere Ammalato;\\
Ma è mangia, beve, e dorme il suo bisogno,\\
Ch'è fino a vespro, e poi si leva in sogno,

\flagverse{50}Con lo scenario in mano, e il mondo fuora\\
Va innanzi a nobil suoi commilitoni,\\
Pancrazio, Pedrolino, e Leonora\\
Lo seguon con un nugol d'Istrioni,\\
C'hanno una insegna non finita ancora,\\
Perché Anton Dei con tutti i suoi garzoni,\\
Incambio di sbrigar quella faccenda,\\
È ito al Ponte a Greve a una merenda.
\end{ottave}

Passa Belmasotto Ammirato, che è Mattias Bartolommei Marchese giovane di bell''aspetto,
ricco, e letterato; il quale fu un tempo, che si persuadeva d'haver tutti
i mali. E perché questo Cavaliere si diletta di comporre commedie, e volentieri
recita in esse lui medesimo, ed appunto nel tempo, che l'Autore accrebbe la presente
Opera, havea detto Signore messa insieme una conversazione di giovani nobili,
che recitavano all'improvviso; però lo fa capo di nobili commedianti, e
gli da uno stendardo non ancor finito, perché \textit{Antonio Dei} ricamatore (e questo
è il vero suo nome, cognome, e professione) in cambio di finirglielo, era andato
a un'allegria al Ponte a Greve, luogo poco lontano da Firenze. Caso seguito
al detto Sig.\ Marchese Bartolommei, che aspettando alcuni abiti per una commedia
, che si dovea far la sera, il Dei in vece di finirgli sen'era andato con tutti
i garzoni della sua bottega fuori di Firenze.

\begin{description}
\item[HAVENDO de soldi nella cassa] Essendo ricco: Non gli mancando denari
\item[PISCIAR a letto, e dire: lo son sudato] E' proverbio assai vulgato, che significa.
Può fare a suo modo, che, o male, o bene che egli faccia, gli è sempre
ascritto a bene; E s'intende d'uno, che sia ricco, e fortunato.
\item[LEVARSI in sogno] Levarsi più presto dell' ere solita di levarsi, quasi dica
S'é levato di notte, sognado esser'hora di levarsi,e qui Autore intende, che a questo
Cavaliere il mezzo giorno, alla quale hora cominciava a destarsi, serviva per aurora,
\item[SCENARIO] È un foglio, sopr'al quale son descritti i recitanti, le scene della
commedia, la quale si dee recitare, ec. i luoghi, per i quali volta per volta devono
uscire in palco i recitanti, afinché quel tale, che assiste gli possa fare uscire
aggiustatamente, ed a i tempi debiti. Tal foglio si domanda anche \textit{Mandafuora}, se
bene il \textit{Mandafuora} è alquanto differente dallo \textit{Scenario}, perché questo s'appicca
al muro dietro alle scene affinché ciascuno recitante lo possa da se stesso vedere,
ed il Mandafuora è tenuto in mano da colui, il quale invigila, che l'opera sia,
recitata ordinatamente; ma tuttavia, come ho detto, s'intende, e si piglia spesso
l'uno, per l'altro.
\item[PANCRAZIO, Pedrolino, e Leonora] Nomi di recitanti nella suddetta conversazione.
\item[NUGOLO a' Istrioni] Gran quantità di commedianti. Questa voce \textit{nugolo}, che
nel presente luogo significa numero infinito, si usa più propriamente parlando di
volatili, perché questi volando gran numero insieme, come farebbono storni,
colombi, ec.\ occupano il sole, ed oscurano l'aria, appunto come fa il \textit{nugolo}. La voce
Istrioni è latina, tolta dall'antico Toscano, come dice Polid. Verg. lik.3-cap.14.
le cui parole son queste. \textit{Et quia Hister Fusco verbo ludus vocabatur, ideo nomen histrionibus
est inditum}, ec. Ma hoggi ce ne serviamo per nome speciale, chiamando
Istrioni solamente i commedianti, che recitano per prezzo.
\item[GARZONI] Intende lavoranti; se ben Garzone vuol dir propriamente Giovane
  scapolo, e senza moglie, come si vede nell'ottava antecedente lindo garzone;
Tuttavia s'intende anche Servitore, o lavorante, che stia a salario in botteghe
di qualsivoglia mestiero.
\item[MERENDA] Specie di mangiare, che si fa tra mezzo giorno, e sera. Vedi
  sopra in questo C, stan. 35,
\end{description}
\section{Stanza LI ... LVI}

\begin{ottave}
  \flagverse{51}Don Panfilo Pilori move il passo\\
Che, tra che per usanza mai sta cheto,\\
Hor ch'ei fa moto fa si gran fracasso,\\
Ch'io ne disgrado il Diavol n'un canneto,\\
Assorda il mondo più d'agn'altro il grasso\\
Papirio Gola, c'appunto gli è dreto,\\
Il qual vestì di lungo, e fu guerriero,\\
Perocché poco gli fruttava il Clero
\end{ottave}

\begin{ottave}
  \flagverse{52}E n'ha fatto con esso de rammanzi,\\
C'un po' di campanile non gli alloga,\\
E questa è la cagion, che là tra i lanzi\\
Da soldato n'andò in Oga Magoza;\\
Ne quivi essendo men tirato innanzi,\\
Posò la spada, e ripigliò la toga,\\
E per lo meglio si risolse al fine\\
Tornar' a casa a queste stiacciatine.
\end{ottave}

\begin{ottave}
  \flagverse{53}Al che tra molti commodi s'arroge;\\
Quel ber del vin; ch'è troppo cosa ghiotta,\\
Qua birre, qua salcraut, qua cervoge,\\
A casa mia dicea, del vin s'imbotta,\\
Però finianla; cedant arma togae:\\
Io non la voglio, in quanto a me, più cotta;\\
Guerreggi pur chi vuol, s'ammazzi ognuno,\\
Ch'io per me non ho stizza con nissuno.
\end{ottave}

\begin{ottave}
  \flagverse{54}Così rinunzia l'armi a Giove, e stima\\
D'esser il più lieto huom che calchi terra,\\
Pensa stato mutar, cangiando clima, \\
Ma trovata l'Italia tutta in guerra,\\
E forzato ferrarsi, più che prima;\\
Ecco il giudizio human come spesso erra\\
Crede tornar fra gente quiete, e gaie,\\
E fugge l'acqua sotto le grondaie.
\end{ottave}

\begin{ottave}
  \flagverse{55}Tra don Panfilo, e lui uno squadrone\\
Dal Pontadera aspettano, e da Vico,\\
Che parte per la via vanno a Vignone,\\
E parte fanne un sonno a piè d'un fico,\\
Costoro empion di rena un lor soffione,\\
E quando sono a fronte all'inimico,\\
Gliela schizzan nel viso, ed in quel mentre\\
Gli piglian gli altri la misura al ventre.
\end{ottave}

\begin{ottave}
  \flagverse{56}L'insegna di costoro è un Montambanco,\\
C'ha di già dato alli suoi vasi il prezzo,\\
E detto che son buoni al mal del fianco,\\
E strolagato, e chiacchierato un pezzo,\\
Ma trovandosi alfin sudato, e stanco,\\
E non havendo ancor toccato un bezzo,\\
Si scandolezza, ed entra in grande smania,\\
Poi dice, che si parte per Germania.
\end{ottave}

Segue Don Panfilo Piloti, che è Ipolito Pandolfini gran chiacchierone, e Papirio
Gola, che e Paolo Parigi, il quale ne i suoi primi anni vestì abito da Prete (che
questo intende col dire \textit{vestì di lungo}) ma poi lo posò, e sen'andò in Alemagna,
alla guerra vedendo, che quell'abito non gli era di frutto; Visto poi, che anche
quel mestiero non gli fruttava, tornò alla patria, e ripigliò l'abito. Ma trovato,
che ancora l'Italia era sottosopra per causa della guerra del Duca di Parma, fu
forzato dal debito di suddito, e dalla convenienza della provvisione, a tornare
alla guerra in servizio del Sereniss.\ Gran Duca, e a lasciar di nuovo l'abito da
Prete. Finita detta guerra il medesimo Paolo Parigi si rimette l'abito, e fattosi
Sacerdote, morì poi Rettore della Chiesa di S.\ Angelo a Vicchio. Questo Paolo
Parigi fu figliuolo di Giulio, e fratello d'Alfonso ambedue Architetti celebri,
come fu ancor'egli, ed Andrea altro suo fratello, che fu Maestro di campo, e
nominato dal nostro Poeta Paride Gurani sotto nel C. 3. stan.\ 10.

I suddetti due conducono genti dal Pontadera, e da Vico, (Terre vicine a Pisa)
le quali genti dice il Poeta, che \textit{l'aspettano}, perché venendo di lontano per la
stanchezza del viaggio s'erano fermate per la strada a riposarsi; E per mostrare,
che questo \textit{Papirio} era grand'ingegnere, fa che questa gente habbia per arme
un'ordigno per facilitare la distruzione del nimico, il quale e un mantrice pieno
di rena, e per alludere al genio vagabondo di Papirio, ed alle chiacchiere
di Don Panfilo, figura nella loro insegna un Montambanco, che sono genti
chiacchierone, (e però detti anche \textit{Ciarlatani}) e che non hanno patria ferma,
sendo oggi in Firenze, e domani altrove, secondo che gli porta la speranza del
guadagno.

\begin{description}
\item[FRACASSO] Strepito, romore; Vien dal latino Frangere, che vuol dir
  Rompere, e veramente il significato proprio di fracasso e quel romore, che procede
  da frattura, o spezzamento di materiali; se bene si piglia per ogni sorte di
  strepito. Dan. Inf. C. 9.
  \begin{verse}
    già venia fu per le torbide onde
    Un fracasso d'un suon pien di spavento.
  \end{verse}
  E nel Purg. Cant, 14,
  \begin{verse}
    ecco l'alzra con si gran fracasso
  \end{verse}
  Dove l'espositore Landini dice, che Fracaffo vien dal verbo frangere.
\item[NE disgrado il Diavol n'un canneto] Farebbe manco romore il Diavolo in un
  postime di canne. Si figura il diavolo, per lo più, un'huomo con le corna, con
  l'ali, e co i piedi di gallo; onde si dice un \textit{Diavol n'un canneto}, perché si suppone,
  che passando il detto diavolo dentro a un postime di canne, pigli con le corna,
  con l'ali, e con gli artigli le canne, le quali scappando dalle dette corna;
  ali, ed artigli a guisa di molla, perquotono nell'altre canne, che per esser vote
  fanno strepito, e rimbombo non piccolo. Quand'uno s'affatica per conseguir
  qualcosa diciamo: \textit{Il tale ha fatto il diavolo per haver la tal cosa}, e s'intende \textit{ha
  fatto il diavol n'un canneto}, cioè gran romore, Il termine; \textit{Ne disgrado} Vuol dire
  lo stimo manco: lo levo il luogo, o grado: per esempio \textit{Il tale compone versi Latini
  così bene, che io ne disgrado Vergilio}, cioè io stimo, che questo tale habbia tolto
  il luogo a Vergilio, e faccia meglio di lui. Vedi sotto Cant, 3. stan. 34. C. 6.
stan. 61.¢ C. 7, stan. 25.

\item[RAMMANZO] Far un rammanzo, o rammanzina vuol dire, Riprender' uno,
  con minacce; e suona lo stesso, che far' un rabbuffo, o Rabbuffare detto sopra in
  questo C. stan. 39.
\item[NON gli alloga un po' di campanile] Piglia la parte per il tutto, e vuol dire Non
gli fa conseguire una Chiesa.
\item[LANZI] Così chiamiamo i Soldati a piedi guardie del Sereniss. Gran Duca, i
quali son tutti Alabardieri Tedeschi: E pero dicendo: \textit{Andò fra i Lanzi} intende
Andò fra i Tedeschi, cioè in Alemagna; la voce Lanzi e Todesca lasciataci da
loro medesimi, che in salutarsi sogliono chiamarsi \textit{Lantzman}, che suona Paesano;
e \textit{Lanzchnect} vuol dir soldato a piede, e per questo gli Scrittori Fiorentini si
servono della voce \textit{Lanzichenecchi}, per intendere Soldati Alemanni a piede. Ed
il Varchi storie Fiorentine lib. 2, dice così: \textit{Quanto più s'avvicinavano i Lanzi, che
così per maggior brevità gli chiameremo da qui avanti, e non Lanzichenecchi, ec}.
\item[OGA magoga] Quand' uno va lontano dalla sua patria, dicono le nottre donne,
  \textit{Gli è andato in Oga magoga}, Ed intendono gli è andato a casa maladetta, nel
  qual senso è preso anche nella sacra scrittura; e S. Gio; nell'Apocalisse al 20,
  dice \textit{Og magog, \& congregabit eos in praelium}. Ed al cap. 7. dice \textit{In
    dispersionem gentium}, e si trova anche in altri libri della Sac. Bibbia. Vedi Angel. Mons.
  Fio. Ital. linguae alla parola oga magoga. Dicono ancora \textit{Gaga magoga}. E forse
  intendono dei Regno di Goaga in Affrica. Il Vocabolista Bolognese dice, che Og fu
  gigante d'Astarotte Rede Baraniti, della creazione del Mondo 2492, contro al
  popolo d'Israel ne i campi d'Edrai, ove fu destrutto con tutto il suo esercito, e
  cinquanta Città; e che di qui venne il significato Andare in dispersione, e in fumo.
  o a casa del Diavolo, essendo interpetrato Og magog, per il Diavolo. Sin qui
  il Vocabolista. Gli antichi secondo Plinio chiamavano Magog la Città d'Edessa,
  (che Strabone dice, che è l'istessa, che Hierapoli) dove era il celebre Tempio
  della Dea Atergatide detta la Dea Siria, e dove gli Ebrei vissero in cattivita, onde
  da questo dicendosi Andare in Magog, per gli Ebrei era lo stesso che dire:
  Andar' in servitù. Gio: Villani Stor. Fior. lib. 5. Cap. 29. dice: \textit{Le genti, che si
  chiamano Tartari uscirono dalle Montagne di Gog Magog chiamate in latino monti di
  Belgen}. Conchiudo dunque, che non dire \textit{andò in Oga Magoga}. Significa Andò
  in paesi lontanissimi, e di pericolo: ed è quasi lo stesso, che dice \textit{Andò a Buda},
  che vedremo sotto Cant. 5. stan. 13.
\item[TIRATO innanzi] Avanzato a gradi, a dignità, a utili, ec.
\item[TOGA] Vuol dir propriamente abito da Dottori, ma si piglia bene spesso per
l'abito da Prete, come è presa in questo luogo.
\item[TORNAR a casa a queste stiacciatine] Tornare a goder'i comodi della propria
  casa, che si dice anche: Tornare al Pentolino, che i latini dissero: \textit{Redire ad
    pristina Praesepia}. Stiacciatina è diminutivo di Stiacciata, la quale è specie di pane, che
  dopo lievito si stiaccia con le mani per farlo più sottile, affin che si quoca più presto,
  e faccia minor midolla.
\item[S'arroge] ll verbo Arrogere vuol dire aggiugnere. Al che \textit{s'arroge}; al che
  s'aggiugne, e vuol dire; Ci è anche di più. Il Lasca Nov.~5.
\begin{verse}
  E così per non arroger peggio al male, si stava quieta, ec,
\end{verse}
Petr. Canz, 9.
\begin{verse}Eduolmi, c' ogni giorno arrage al danno.
\end{verse}

\item[COSA ghiotta] Cola desiderabile, cosa appetitosa; che \textit{ghiotto} si dice Uno avido
  di mangiar del buono; e viene da \textit{indulgere gutturi}.

\item[SAL craut] Cavolo salato. Voce, e vivanda Tedesca.

\item[BIRRA] o \textit{Cervogia}, Bevanda, che s'usa in Alemagna, ed in altri paesi,
dove è poco Vino; ed è composta di biade, acqua, e fiori di luppoli; ed è lo
stesso \textit{Birra}, che \textit{Cervogia}, e questa ultima è dal Latino.

\item[IMBOTTARE] Metter nella botte. Se bene qui si potrebbe intendere Bere,
costumandosi dire: \textit{Io non imbotto acqua}, in vece di dire: Io non bevo acqua, si
come è inteso sotto C, 7. stan. 4.

\item[NON la voglio più cotta] Per la mia parte mi basta così,ne mi curo di meglio.
Sum presenti Catone contentus, dilic Auguito.
\item[STIZZA] Ira, collera; e vale anche per Inimicizia.

\item[FERRARSI] Intende Armarsi. È detto scherzoso, perché Ferrare, senza dir
più s'intende mettere i ferri all'unghie de' piedi de' cavalli, muli, ed altre
bestie.

\item[GENTI gaie] Genti allegre, ricche, e abbondanti d'ogni comodo, e quiete;
che la voce Gaio è forse sincopata da Gandio.

\item[GRONDAIE] Quel cascare, che fa l'acqua da i tetti, quando piove; e si
dice Grondaia da Gronde, che sono quelle tegole più larghe, le quali son poste
nell'estremità de' tetti. Ed il Proverbio \textit{Fuggir l'acqua sotto le grondaie} vuol dire;
Procurar di fuggire un pericolo, e andarli incontro, che è quello forse, che i Latini
intesero col dire \textit{Incidit in Scyllam cupiens vitare Charybdim}.

\item[ANDARE a Vignone] Andar nelle vigne altrui a corre l'uva; e si dice così
per rendere il detto oscuro, mostrandosi d'intendere d'Avignone in Francia, o
del Bagno di Vignone, che è nello Stato di Siena.

\item[SOFFIONE] Quel piccolo Mantaco, o Mantice, del quale comunemente ci
  serviamo per soffiar nei fuoco, usandolo a mano.

\item[SCHIZARE] Qui è verbo attivo, e vuol dice: Gli gettano con violenza nel
  viso quella che è dentro al soffione.

\item[MONTANBANCO] Uno di coloro che vendono i rimedj nelle pubbliche piazze,
  detti \textit{Montambanchi} dal montare sopra i banchi quando vogliono vendere;
  e detti anche \textit{Ciarlatani} dalle gran ciarle, che sogliono fare.

\item[TOCCATO un bezzo] Preso, o buscato un quattrino. \textit{Bezzo} è moneta, e
  Parola Veneziana, ma usiamo, se non la moneta, almeno la voce \textit{bezo} ancor noi
  per intender Denari in generale.
\item[SI scandolezza] In questo luogo, ed in questi termini significa Adirarsi, e
  mostrar con le parole, e con gli atti la collera, che uno ha. Vedi sotto C.~11. stan.~23.
  Verbo che viene dal Greco \textit{scandalizesthai} che suona, a loro, come a noi
  Offendersi, o adirarsi d'una cosa.

\item[ENTRAR in smania] Entrar in grandissima collera; che Smania è una soverchia
  inquietudine, cagionata da febbre, o da eccessivo caldo, o da soverchio
  amore, la quale riduce l'huomo quasi insano, e furioso.
\end{description}

\section{Stanza LVII \& LVIII}
\begin{ottave}
  \flagverse{57}Huomini bravi quanto sia la morte\\
Scandicci n'ha mandati, e Marignolle,\\
Gente, che si può dir che habbia del forte,\\
Poi ch'ella ammazza gli agli e le cipolle,\\
Sue lance i pali son, targhe le sporte,\\
Airchiusi le man, le palle zolle,\\
Va ben di mira, e colpo colpo imbreccia,\\
Maffime quand'altrui vuol dar la freccia,
\end{ottave}

\begin{ottave}
\flagverse{58}Vien comandata da Strazildo Nori,\\
Ch'è Chimico, Poeta, e Cavaliere,\\
Ed è quel, ch' in un quadro co i colori\\
Fece quei fichi, che divenner pere.\\
E perché questo è il Re de bell'humori,\\
Per dimostrar quanto gli piaccia il bere; \\
Ha per impresa un Lanzo a due brachette;\\
Ch'il molle insegna trar dalle mezzette.
\end{ottave}

Seguita la gente di Scandicci, e di Marignolle, Ville vicine a Firenze, dove
nascono Cipolle, Agli, ed altri fortumi simili in grande abbondanza. Questa
gente dice che è \textit{brava quanto la morte, perché ella ammazza gli agli, e le cipolle, e
si può dire che habbia del forte}, E pare che intenda che ella superi in fortezza, e
bravura gli agli: E vuol poi dire, che ha molti fortumi, ed Ammazza, cioè Fa
mazzi delle cipolle, e degli agli. E perché questi contadini habitando intorno
a Firenze praticano molto la Città, dove è occasione di spendere più che nel
contado, dice l'Autore, che son genti che \textit{danno la freccia}, che vuol dir Chieder
denari in presto; e par ch' ei voglia intendere che son bravi tiratori di freccia,
e d' archibuso. Son comandati da \textit{Strazzildo Nori}, cioè Rinaldo Strozzi Cavaliere
di S. Stefano; ed è quello, che in squola dell'Autore volendo dipignere
alcuni fichi non trovò mai il modo di fare, che non paressero pere. Questo fu
un geatilhuomo di grandissimo garbo, faceto, allegro, e spiritoso, e buon bevitore;
e perciò gli fa fare per impresa un Lanzo, che vota una mezzetta di vino,
e gli fa comandare questa gente, perché fu poi P\ellipsis{2em} in vicinanza dei
lor paesi.

\begin{description}
\item[SPORTA] Specie di paniere fatto di giunchi, ed ha due manichi; serve per
portarvi dentro erbaggi, ed altro, che si provvede in piazza giornalmente per il
Vitto.
\item[ZOLLA] Gleba, pezzo di terra sollevata nel lavorare i campi, Vedi sotto
in questo Canto stan.\ 82.
\item[COLPO colpo] A ogni colpo. Intendi: sempre ch' ei tira; colpisce, che la forza
  della replica e di far nascer il superlativo.
\item[IMBRECCIA] Forse meglio \textit{imbercia}; E Significa Pigliar di mira; donde
  \textit{imberciatore} colui che fa professione di tirar d'archibuso; e par che venga da
  sbirciare, e bircio, che è guardar con occhi socchiusi, come dicemmo sopra in
  questo C, stan. 9. e come s'usa a tirar con l'archibuso. Ma puo anche essere che
  venga da breccia che vuol dir Quelle rotture che vengon fatte nelle muraglie
  dall'artiglierie, e si dica imbrecciare per colpire, si come intende nel presente
  luogo pigliando colpire in senso di conseguir l'intento.
\item[DAR la freccia] Come habbiamo accennato, vuol dire Chieder denari in presto;
  e s'intende Uno che habbia poco modo, e minor voglia di rendergli. Gli
  antichi Etiopi, e gli abitatori di Maiorca, ec. non solevano dar mangiare alli
  loro figliuoli, se questi con le frecce non facevano cascare dallo stile, o albero
  il cibo, che vi era posto, ond'io stimo, che questo frecciar per vivere habbia dato
  origine al presente detto. Vedi Alex. ab Alex.\footnote{Alessandro d'Alessandri, ``Alexander ab Alexandro'', Napoli 1461 - Roma 1523. Umanista e giurista.} dier. gen.\footnote{Genialium Dierum, Parigi, 1532.} lib. 2. c. 25. Il Monosino
  dice, che questo \textit{frecciare} habbia origine dal Latino \textit{ferire} che appresso
  loro haveva il medesimo significato, e lo cava da Teren. in princ. Phormionis:
  \textit{Porro autem Geta Ferietur alio munere ubi hera pepererit}. Diciamo; i denari sono il
  secondo sangue; dar ferita cava il sangue, come il dar frecciate, cava il sangue;
  e per questo dicendo \textit{dar freccia} intendiamo Dar freccia alla borsa, e cavare questo
  secondo sangue, che è il danaro.
\item[BELLUMORE], Huomo allegro, faceto, ec. vedi sopra in questo C. stan. 10.
Quando diciamo, Il tale è Re della tal cosa; intendiamo Vale in superlativo
grado in quella tal cosa; onde \textit{Re de belli humori} vuol dire Grandissimo bell'humore.
Significato che viene da i Greci, i quali chiamavano Re colui, che nei
giuochi fanciulleschi vinceva, e superava gli altri, ed Asino, o Mida era chiamato
colui che perdeva; il che più diffusamente vedremo nel 2. Canto.
\item[LANZO a due brachette] Lanzo dicemmo sopra, che vuol dir soldato Tedesco
  a piede; ma qui vuol che s'intenda uno proprio di quelli della guardia del Serenissimo
  Gran Duca; dicendo a due brachette, perché questi tali Lanzi vanno vetiti
  a livrea, con un paro di brache larghe, fatte a strisce, come son quelle delli
  Svizeri del Papa in Roma, e come quelle de' Trabanti dell'Imperatore.
\item[INSEGNA trarre il molle dalle mezzette] Insegna col suo bere, come si fa a votare
  i vasi pieni di vino, Che \textit{mezzetta} è un vaso fatto di terra invetriata, che
  serve per misurare il vino, ed è capace della quarta parte d'un fiasco Fiorentino.
\end{description}

\section{Stanza LIX \& LX}
\begin{ottave}
  \flagverse{59}Morbido Gatti, Henrigo Vincifedi\\
A far venir innanzi ecco son pronti\\
I fanti, che ne dà il Ponte a Rifredi,\\
Che mille sono annoverati, e conti.\\
Han certi Santambarchi fino a piedi,\\
Che chiaman' il zimbel di là da monti,\\
E paion con la spada in su le polpe\\
Un che facia lo strascico alla volpe.
\end{ottave}

\begin{ottave}
\flagverse{60}Nell'insegna han ritratto u' huom canuto,\\
Che troppo havendo il crin (per osser vecchio)\\
Fioccoso, e lungo, un fanciullino astuto\\
Dietro gli grida: Gli abbrucia il pennecchio.\\
Da questa schiera qui s'è provveduto\\
Gran ceste piene d' huova, e di capecchio\\
Con fasce, pezze, e taste accomodate\\
Per farsi alle ferite le chiarate.
\end{ottave}

Passa l'ultima truppa di Soldati, la quale è composta d'huomini dal Ponte a
Rifredi, che è un luogo vicino a Firenze. Costoro son comandati da \textit{Morbido
  Gatti}, cioè \textit{Migiotto Bardi}, e da \textit{Henrigo Vincifedi}, che è \textit{Vincenzio
  Sederighi}, due gentilhuomini già scolari dell'Autore: E perché questi si pigliavano gusto di
ragionare spesso con un tal Dottor Cupers, glielo fa fare per impresa.

A Questo Dottor Cupers negli ultimi anni della sua vita, che durò sopra ottanta
anni, entrò in frenesia d'esser bello, e si persuadeva che ogni donna s'innamorasse
di lui, e lo volesse per marito, e però andava lindo, e con la chioma
folta, e lunga, e ben coltivata; ma canutissima: onde i ragazzi quando passava
per le strade gli gridavano dietro: Guarda il Pennecchio, gli abbrucia il Pennecchio,
intendendo di detta sua chioma, e lo facevano adirare, e maggiormente
impazire. E perché li contadini del Ponte a Rifredi si danno a credere d' haver
maggior Civiltà degli altri contadini per esser nati, ed allevati, si può dire, nei
Borghi di Firenze, ed intorno alla Petraia, e Castello, Ville spesso habitate
da Principi della Serenissima Casa, perciò per lo più vengono alla Città col
ferraniuolo, o santambarco, che sono le Toghe de i Barbassori, e Dottori
del Contado; e per questo il Poeta dice \textit{Han certi Santambarchi fino a piedi, Che
  chiamano il Zimbel di là da' monti}, cioè incitano i ragazzi a dar loro delle Zimbellate.
E per esser questa l'ultima schiera fa, che ella conduca seco il bagaglio
de i medicamenti per l'Esercito.
\begin{description}
\item[SANTAMBARCO] Specie d'abito, o sopravveste, o diciamo mantello
usato da i nostri contadini per difendersi dall'acqua, e dal freddo; ed è composto
di due larghe strisce di panno cucite in forma di croce con una buca in mezzo,
per la quale passano il capo, e vengono coperti da una parte di detto panno le
schiene, e il petto, e dall'altra le braccia, e i fianchi, Si dovrebbe dire \textit{Salta in
barco}, e così dice Mattio Franzefi nel Capitolo del suo viaggio da Roma a
Spoleto.
\begin{verse}
\backspace Gli osti, c'a profferir mai non son parchi
Volean ch'io scavalcassi a sì mal tempo,
E m'offerivan fuoco, e Saltambarchi.
\end{verse}

Ed è forse meglio detto \textit{Saltambarco}; perché questo abito è composto in tal
forma; che tiene tutta la persona difesa dal freddo, e non l'impedisce il saltare
i fossi, e passare i barchi. Ma si dice \textit{Santambarco} perché così lo chiamano i contadini
che se ne servono, ed è lor abito proprio.
\item[CHIAMAR una cosa di là da i monti] Questo termine significa Meritare una
  cosa grandemente, come per esempio \textit{Il tale è così insolente, ch'ei chiama le bastonate
  di là da i monti}.
\item[ZIMBELLO] In questo luogo intende un sacchetto pieno di crusca;
o di cenci, o di segatura, legato a una cordicella lunga circa due braccia,
col quale i fattorini delle botteghe de setaiuoli nel tempo del Carnevale, quando
passano i contadini per quei luoghi, dove sono le botteghe de i setaiuoli, uno di
loro perquote il contadino; e mentre questo si volta per veder chi ha percosso,
gli altri ragazzi lo perquotono dall'altra banda: E questo per lo più vien fatto a
certi contadini, che se ne vengono in Firenze intronizzati, e in sul grave, come
appunto fanno quei del Ponte a Rifredi. E per altro la voce Zimbello ha il significato,
che vedremo sotto C. 7. stan. 76.
\item[FAR Io strascico alla Volpe] E' una specie di caccia, che si fa alla Volpe, pigliando
  un pezzo di carnaccia fetida, che legata a una corda si va strascicando per
  terra; per far venir la Volpe al fetore di essa Carne; ed il Poeta assomiglia il portar
  della spada di questi Contadini a questa corda, dicendo che stava pendente
  \textit{in su le polpe} (cioè dietro alle gambe, che così chiamiamo cotesta parte) appunto
  come sta la fune di colui, che fa lo strascico alla Volpe.
\item[PENNECCHIO] Qui è preso per chioma, ò Zazzera, come habbiamo accennato
  sopra, metaforico da quell'involto di lino, stoppa, lana, o altra materia
  simile, che adattano le donne sopr'alla rocca per filare, il quale involto si dice
  Pennecchio.
\item[QUESTA schiera qui] La voce \textit{qui} è superflua, bastando per farsi intendere il
dir solamente \textit{da questa Regina} senza aggiungere la particella \textit{qui}: Ma non per
questo il nostro Poeta ha fatto errore, havendo seguitato il nostro Fiorentinismo
usatissimo. Dicendosi comunemente (forse a maggior' emfasi) \textit{Questo negozio qui},
\textit{questa cosa che è qui}, e simili; e la particella \textit{qui} esprime \textit{il negozio, del quale ragioniamo presentemente}, \textit{Questa cosa, la quale habbiamo fra le mani}: Anzi stimo, che
l'habbia fatto ad arte, e per mostrare questo nostro modo di dire, (forse riprensibile)
del quale non mi pare, che in tutta l'Opera si sia servito mai più; quantunque
non gli sieno mancate l'occasioni; E se bene nell'Ottava 65. seguente,
pare, che l'usi nel medesimo modo, osservisi, che quivi è termine dimostrativo
necessario, e non riempitivo, operando che s'intenda di quella Cugina, che è lì
presente, e non d'altra, come si potrebbe intendere, se non vi mettesse la particella
\textit{qui}.
\item[CESTA] Intendiamo un gran paniere, che fa mezza soma di bestia, ed è contesto
  d'assicelle di castagno, o d'altro legname a foggia di cassa, per uso di portare
  da un paese all'altro uova, vino in fiaschi, ed altre cose frangibili; e per lo
  più son fabbricati due attaccati l'uno all'altro con quattro legni gagliardi aggiustati
  in maniera da adattarsi sopra i basti a traverso alla bestia, in modo che tengono
  equilibrate, e ferme dette due ceste anche senza legarle. Se ne fabbricano
  ancora della stessa forma, e materia sciolte, cioè senza i detti quattro legni, e
  queste s'adattano, e fermano in su i basti con le funi, come si fa i Cestoni, che
  sono ancor'essi panieroni di mezza soma fatti di vinciglie di castagno, o altro albero
  intessute, de i quali si parla sotto C.\ 10.\ stan.\ 7.
\item[CAPECCHIO] La pettinatura, cioè quella stoppa più grossa, che si cava dal
  lino sodo la prima volta, che si pettina detta capecchio, perché si cava dai due
  capi del lino, cioè barbe, e cime, le quali sono più ripiene d'immondezze, e di
  filo morto, e inutile.
\item[FAR la chiarata] Il primo medicamento, che si faccia alle ferite è l'albume,
o chiara d'huovo, entro alla qual chiara s'intigne il capecchio, e si pone sopra
alle ferite; E questo si dice \textit{far la chiarata},
\end{description}

\section{Stanza LXI}

\begin{ottave}
\flagverse{61}E' general di tutta quella mandra\\
Amostante Laton Poeta insigne\\
Canta improvviso, come una calandra,\\
Stampa gli enigmi, strolaga, e dipigne.\\
Lasciò gran tempo fa le polpe in Fiandra,\\
Mentre si dava il sacco a certe vigne, \\
Fortuna, che l'havea matto provato\\
Volle, ch' ei diventasse anche spolpato.
\end{ottave}

Generale di tutto questo esercito e Amostante Latoniy, cioè \textit{Antonio Malatesti}
Poeta celebre per molte sue opere, ma specialmente per quella Sfinge, la quale,
come vedremo sotto C. 8. stan. 26. è una scelta d'enigmi in sonetti, de' quali se
ben la stampa ne fa goder pochi, se ne sperava numero maggiore, volendone
egli pubblicare 400. scelti da una infinità, che ne ha composti; ma la di lui morte
seguita poco tempo fa, ci priva per ora di questa consolazione. Ne gli anni suoi
giovenili cantò all'improvviso molto lodatamente, si dilettò d'Astrologia, e nel
disegno fu scolare dell'Autore, e suo amicissimo, come mostra, facendolo capo,
e saperiore di tutti gli amici suoi, che nomina in questo esercito. E perché questo
Amostante era di corpo adulto, ed havea le gambe sottili, dice, che \textit{lasciò le polpe
in Fiandra}, e che \textit{la Fortuna che l'havea provato matto}, volle che egli diventasse
anche \textit{spolpato}, cioè senza polpe; ma aggiunto alla voce \textit{matto} vuol dire
\textit{matto affatto}; non che Amostante fusse affatto privo di cervello; che la voce
\textit{matto} appresso di noi significa ancora Allegro, Faceto, e simili, nel qual senso è presa
nel presente luogo; e però vuol dire, che Amostante era huomo facetissimo.
\begin{description}
\item[MANDRIA] Vuol dire Una gran quantità di bestie; ma qui intende Grani
  quantità d'huomini. Mandra è voce Greca, che suona Spelonca, e luogo, entro
  al quale le pecore s'adunano all'ombra, ma la pigliavano anche per la greggia
  medesima, e da essa dissero Archimandrita il governatore della greggia.
  Dante pure prese \textit{Mandria} per quantità di huomini, nel Purg. C. 3.
  \begin{verse}
    Sì vidd' io muovere, e venir la testa
    Di quella Mandria fortunata allotta,
    Pudica in faccia, e nell'andare onesta,
  \end{verse}
\item[CANTA improvviso] È costume in Firenze al tempo de i gran caldi la notte
  cantare dell'ottave all'improvviso, mentre ne i luoghi più aperti della Città si
  va pigliando il fresco; e perché in tal'esercizio valeva molto il Malatesti; il Poeta
  l'assomiglia alla Calandra uccello di bellissimo cantare.
\item[ENIGMI] Indovinelli. Voce Latinogreca. Vedi sotto C.6, stan.34.c C.8 stan. 26.
\item[LASCIO' le polpe in Fiandra] Non è, che Amostante fusse mai stato in
  Fiandra; ma, perché lo fa generale di questo esercito, è dovere, che egli mostri,
  che Amostante ha vedute, e provate altre guerre, e che egli si sia trovato a
  dar de' sacchi, ne i quali ha lasciate le polpe delle gambe, il che serve per accreditarlo,
  poiché si come ad un soldato gli stroppj, e le cicatrici son di gloria, così
  ad Amostante era di gloria  haver perduto le polpe delle gambe nelle guerre di
  Fiandra; ma il vero è, che quand'uno hale gambe sottili, diciamondi lui: \textit{Egli
  ha lasciato le polpe in Fiandra}: ed il Poeta con questo equivoco, che accredita
  Amostante, vuol dire, che egli haveva le gambe sottili; e seguita con l'altro
  equivoco di \textit{matto spolpato}, che significa, come s'è detto,  matto del tutto, e
  vuol che s'intenda \textit{senza polpe affatto}. E la voce polpa, che significa ogni pezzo, o
  quantità di carne, che sia senz'osso, da noi si piglia per le polpe delle gambe,
  quando è detta assolutamente. (Vedi l'ottava 59. antecedente; E sotto al C.6.
  stan. 99. dice \textit{ossccia senza polpe}, che s'intende tutta la carne di quel'corpo) e
  significa pure \textit{Matto spacciato}.
\end{description}
\section{Stanza LXII}

\begin{ottave}
\flagverse{62}Passati tutti con baule, e spada\\
Serransi in barca, come le sardelle;\\
Gli affretta il Duca, e chi lo tiene a bada,\\
O ferma un passo; guai alla sua pelle,\\
Ch'ei lo bistratta, e come che ne vada\\
Giù la vinaccia, e il sangue a catinelle,\\
E ben che lesto ciaschedun rimiri,\\
Non gli dà tanto tempo ch'ei respiri.
\end{ottave}

Dopo fatta la mostra se n'entra la soldatesca nelle barche con ogni suo arnese,
e Baldone affretta all'imbarco i soldati.
\begin{description}
\item[BAVLE] Intendiamo ogni sorte di cassetta, valigia, o tamburo, che facilmente
  si possa adattare in su la groppa d'un cavallo, mentre si viaggia. Viene
  dal verbo \textit{baiulo}, e l'allarghiamo ad ogni sorta di cassa portatile in su le some, ec.
  Qui intende quell'involto, che portano i soldati sopr'alle reni per lor proprio
  bagaglio, detto altrimenti zaino.
\item[SERRANSI, come le sardelle] Si serrano strettissimi appunto, come stanno le
  sardelle ne i cestoni, quando da Livorno son portate a Firenze, o nei bariglioni,
  quando ci vengono salate. Comparazione assai usata per intendere stetti, e
  serrati insieme, che in voce marinaresca si dice stivati.
\item[TENERE a bada] Trattenere uno. Varchi stor, lib, 4. \textit{Conoscevano, che erano
tutte cose finte, e solo per tenere a bada trovate}, Viene dal Verbo \textit{Badare}, che ha
  molti significati. \textit{Badare} al negozio per \textit{Attendere al negozio}. Significa
  Indugiare, o perder il tempo, come è inteso nel presente luogo, che dice \textit{tiene a bada},
  ed intende, Chi gli è causa d'indugio, o gli fa perder tempo; il Petrarca Son.23.
  \begin{verse}
    Consolate lei dunque, che ancor bada.
  \end{verse}
 Cioè aspetta la venuta del Pontefice, e perde tempo. Significa ancora \textit{continuare}, o
 \textit{seguitare} a far una cosa, Vedi sotto C.1, stan. 20. Significa \textit{Osservare} C.9.
 stan. 28.  Significa \textit{Disprezzare}, \textit{non curare}, per esempio; \textit{Io non bado al tuo gridare}. Intende
\textit{io non stimo, o non curo il tuo gridare}, Da questo \textit{badare}, o \textit{bada} habbiamo \textit{badalone}
che vuol dire Un' huomo perdigiorno, e che non sa, e non vuol far nulla.
\item[GVAI alla sua pelle] Mal per lui. Vedi sopra in questo C.\ stan.\ 28,
\item[BISTRATTARE] Trattar male, Strapazzare, o Stranare.
\item[VA giù la vinaccia] È necessario far presto per sfuggire il danno, che si patisce
  e che si teme più grave dall'indugio. Quando il mosto, cioè il liquore cavato
  dall'uva, il quale è nel tino, ha bollito a bastanza; perde il vigore, e non
  può più sostenere a galla, cioè nella sua superficie, la vinaccia (che così si chiamano
  i raspi, e bucce dell'uve) onde la lascia cascare in fondo, ed incorporandosi
  con essa di nuovo, si guasta; E questo si dice \textit{andar giù la vinaccia}; che
  poi passato in proverbio significa Quel che habbiamo detto.
\item[NE va il sangue a catinelle] Ne va molto del mia. Per intender, che Un'indugio
  apporta grave dispendio, ci serviamo di questo detto; e si dice anche: \textit{a bigonce}.
  Vedi sotto C.\ 10.\ stan.\ 20.
\item[LESTO] Qui vuol dir Pronto, ed all'ordine.
\item[NON gli da tempo che respiri] Non gli lascia ripigliare il fiato. Questo detto
  esprime un grande affrettamento, o incalzamento.

\end{description}

\section{Stanza LXIII \& LXIV.}
\begin{ottave}
\flagverse{63}Perciò imbarcati tutti in un momento,\\
Poi che Baldon facea così gran serra,\\
Si spiegaron l'insegne, e vele al vento,\\
Quando le Navi si spiccar da terra;\\
Ed egli allora entrò in ragionamento\\
Di quel che lo spingeva a far tal guerra;\\
Ma per contarla più distesa, e piana,\\
Incominciò così dalla lontana.
\end{ottave}

\begin{ottave}
\flagverse{64}Risiede Malmantil sour' un poggetto,\\
E chiungue verso lui volta le ciglia\\
Dice, ch'i fondatori hebber concetto\\
Di fabricar l'ottava meraviglia,\\
L'ampio paese poi, ch'egli ha soggetto\\
Non si sa, vuo giuocare, a mille miglia;\\
V'è l'aria buona azzurre oltramarina,\\
E non vi manca latte di gallina.
\end{ottave}

Fatta la mostra, ed imbarcate in brevissimo tempo le soldatesche, si partirono
le Navi dal lido e fecero vela spiegando le loro insegne. Intanto Baldone dà
principio a narrare la causa, che lo muove a far la guerra di Malmantile, e comincia
dal descrivere la situazione, qualità, e dominio.

\begin{description}
  \item[FAR serra] Affrettare. In alzare. Vedi sotto C. 9, stan. 13.
\item[CONTARLA difesa, e plana] Intendi, Raccontarla puntualmente, e con
tutte le circostanze,
\item[NON si sa uno giuocare a mille miglia] Io giuoco, che non si trova chi sappia,
o possa giudicare a mille miglia, quanto paese gli è suggetto; perché è così gran
paese, che mille miglia non si considerano, essendo parvità di numero, e di materia
in riguardo del tutto, che gli è suggetto. E questa voce \textit{suggetto}, che vuol
dir \textit{sottoposto}, s'intende Situato sotto, e non sottoposto al dominio di Malmantile,
che per esser Posto nella sommità d'un poggetto, ha d'attorno molta pianura,
e colline sottoposte, cioè più basse di lui; se ben par, che voglia dire, che
Malmantile ha dominio immenso.

\item[ARIA azzurra oltramarina] I pittori dicono buon'aria quella, la quale e colorita
  con l'azzurro oltramarino, perché questo non perde mai il colore, come
  perde l'indaco, e lo smalto; ma è però anche vero, che quando l'aria si vede di
  colore azzurro, come è il buono oltramarino, è segno, che è purgata da ogni
  imperfezione di nebbia, o d'altri maligni vapori, e per conseguenza e aria buona;
  il Poeta però dice, che a Malmantile è aria azzurra oltramarina per intendere,
  che a Malmantile è aria, che dura sempre azzurra, come fa quella colorita
  con l'oltramarino, cioè sempre buonissima. E \textit{L'oltramarino} è quel colore, che si
  cava dalla pietra detta Lapislazzuli.
\item[NON vi manca latte di gallina] Vi sono tutte le cose squisite, è abondante d'ogni
  bene. Detto antico, si come si cava da Strabone lib, 14., dove discorrendo delle
  campagne di Samo dice, che erano così fertili, che si diceva comunemente,
  che producessero fino il latte di gallina, cioè quelle cose, che e impossibile, ch'altrove
  si trovino, come è il latte di gallina. \textit{Samus}, dice egli, \textit{feracissima, unde
laudantes non dubitant illud ei proverbium accommodare, quod ferat etiam Gallinae
lac}, ec.
\end{description}

\section{Stanza LXV \& LXVI}
\begin{ottave}
  \flagverse{65}Il Re di questo Regno giunto a morte\\
La mia Cugina qui, che fu sua Donna\\
(Non havendo figliuoli, o altri in Corte\\
Propinqui più) lasciò donna, e Madonna:\\
Ma come volle la sua trista sorte,\\
Un certo diavol d'una Mona Cionna\\
Figliuola d'un guidone ignudo, e scalzo\\
Ne venne presso a farie dar lo sbalzo.
\end{ottave}

\begin{ottave}
  \flagverse{66}Gobba, e zoppa è costei, e mancina,\\
Ha il gozzo, e da due sfregi il vifo guasto,\\
Scorse in Firenze ognor la cavallina\\
Ne i lupanari con gran pompa, e fasto,\\
E perché ossequij havea sera, e mattina,\\
E il titol di Signora a tutto pasto,\\
Fatta arrogante, al fine alzò il pensiero\\
A voler questi onori da dovero.
\end{ottave}

Narra Baldone, che il Re di Malmantile instituì Celidora erede del Regno, e
che questo le fu usurpato da Bertinella, la quale descrive per una donna tutta
contraffatta, e la mostra una vera sgualdrina: ed imita Dante nel Purg. C.19.
che dice.;
\begin{verse}
  Mi venne in sogno una femmina balba,
  Con gli occhi guerci, e sopra i piè distorta,
  Con le man monche, e di colore scialba.
\end{verse}

Qui è da considerare, che i tanti difetti da Baldone attribuiti a Bertinella,
realmente in lei non fussero, perché, ed egli non se ne farebbe innamorato, come
si dice sotto nel Cant. 9., ed ella non havrebbe havuto tanti altri amanti; Ma
Baldone non l'havendo mai veduta, e volendo concitar contro di lei odio di
quei soldati, che lo seguivano, per istigargli ad andar più volentieri alla ricuperazione
di Malmantile, la rappresenta loro una donna così nefanda.

\begin{description}
\item[SVA donna] Sua moglie, Se bene i Poeti dicendo La mia donna, o La sua
  donna, intendono l'amata.
\item[LASCIO' donna, e madonna] Termine notariesco, e curiale, che significa Padrona
  assoluta. Sincopato di Domina.
\item[VN certo Diavolo] Si dice così quando vogliamo esprimere uno, che è cagione
  di qualche nostra disgrazia: per esempio: \textit{Il negozio andava bene, ma un certo diavolo
    d'un Sensale con le sue chiacchiere lo rovinò} quasi dica \textit{Il diavolo, che guastò
    questo negozio, fu un Sensale}.
\item[MONA Cionna] È un detto di disprezzo, che significa Donna da poco in
ogni operazione: ed il senso della voce Mona, Vedrai sotto C. 5. stan. 18.
\item[GUIDONE] Intendiamo huomo vilissimo, abietto, senza roba, e senza creanza,
  o riputazione.
\item[DAR lo sbalzo] Mandar via; Scacciare.
\item[ORBO]. In questo luogo vuol dir Uno, che vede poco, che noi chiamiamo
  lusco, se bene il suo vero senso è di cieco affatto. Vedi sopra in questo C. stan.
  9. alla voce sbirciare.
\item[MANCINO] Uno che per assuefazione ha maggior forza, ed attitudine nella
  mano sinistra, che nella destra; E perché questo tale si può dire difettoso;
  perciò huomo mancino, vuol dire Huomo non buono; ed in questo senso è preso
  nel presente luogo. E però voce che ha del furbesco. Se ne servì il Lalli nella
  sua En. trav. nel C.2. stan. 40, dicendo,
  \begin{verse}
    Perch' io non fui mai orbo, ne mancino.
  \end{verse}
Ed al C, 4. stan. 67.
  \begin{verse}
    E riuscito in somma un buom mancino,
    Una delle più vili creature
    C' habbia sto mondo; e pazzo da catena;
  \end{verse}
\item[HA il gozzo] È parola nota, venendo dal latino guttur: Ma qui vuol dire
un gonfio, o scrofa, che vien nella gola, che i medici, che scrivono di simil
male pongono al trattato il titolo de \textit{Boccijs}.

\item[SFREGIO] Cicatrice di taglio nel viso. Ed una donna sfregiata è numerata
  fra le infami, e per la deformità del volto, e per la causa, per la quale si suppone,
  che le sia stato fatto. Vedi sotto C, 2. stan. 3. dove si mostra esser tali sfregi
  vituperosi anche negli huomini, ed al C, 6. stan. 54.

\item[SCORRER la cavallina] Pighiarsi tutti li suoi gusti liberamente, e senza riguardo
  alcuno. \textit{Havere scorsa la cavallina ne i lupanari}, vuoi dir, che era meretrice
  vecchia, ed avanzata ai bordelli, e lupanari. Gli antichi Egizj, quando volevano
  esprimere la sfacciataggine meretricia, figuravano una cavalla senza freno;
  il furore della quale nelle cose Veneree esprime Vergilio 3, Georg. dicendo.
  \begin{verse}
    Scilicet ante omnes furor est insignis equarum.
  \end{verse}
\item[IL titol di Signora a tutto pasto] Cioè continovamente era chiamata Signora.
  Termine usatissimo per intender voglia cosa, che si faccia molto, e continovatamente.
  Il Mauro\footnote{attribuito come ``Mauro'', autore di versi inclusi le opere burlesche di Francesco Berni et al.} nel Capitolo in lode della Torniella dice.
  \begin{verse}
    E ragionò di voi a tutto pasto
  \end{verse}
\item[DA dovero] Per debito, Per giustizia, Per merito. Intendi che volle proccurar
  d'havere stato, o signoria per meritare il titolo di signora, ec. ed osserva che quel
  \textit{da dovere} non è la voce \textit{vero} con l'aggiunta della sillaba do, ma è il nome
  \textit{dovere} messo in uso di dirlo così correttamente in casi simili a questo, e per
  esprimere una cosa di dovere o doverosa, e dovuta, e giusta.
\end{description}

\section{Stanza LXVII \& LXVIII}
\begin{ottave}
\flagverse{67}Così la mira ad alto havendo messa\\
A suoi Frustamattoni un dì ricorsa,\\
Bramar dice una grazia, e che in essa\\
Non si tratta di scorporo di borsa;\\
Ma, perché aspira a farsi Principessa,\\
Desidera da loro esser soccorsa\\
Col loro aiuto, volendo, e consiglio,\\
Provar, s'a Malmantil può dar di piglio,

\flagverse{68}Pronto è ciascuno, e vuol tra mille stocchi\\
Esporre il ventre, e come un Paladino,\\
Che per servire a Dame, tali allocchi\\
Cercan l'occasion col fuscellino;\\
Ma non si parli, o tratti di baiocchi,\\
Perché non hanno un becco d'un quattrino;\\
E credon, promettendo Roma, e Toma,\\
Di spacciar l'oro della bionda chioma.
\end{ottave}

Bertinella havendo fatta la suddetta risoluzione, richiese li suoi amanti, che
la volessero aiutare a farsi Principessa con impadronirsi di Malmantile, ed i suoi
Drudi s'esibiscono a servirla, perché sentono di non haver a spendere, il che è
cercato da tutti coloro, i quali con simil donne pretendono di passar per belli,
che è una delle tre specie di persone, che voglion queste femmine d'intorno, cioè
Il bello per sua propria sodisfazione. Il bravo per farsi rispettare. Ed il ricco
minchione, o corrivo, per cavar danari da lui, per campare se medelime, ed i
primi due, Il Persiani dice,
\begin{verse}
Il bravo, ed il corrivo, ed il valente.
Nella mia Mea fallisce
Questo antico dettato
Per c' al bravo, ed al bel non apparisce,
Ma sol vorrebbe il suo minchione allato.
\end{verse}

\begin{description}
\item[PORRE ad alto la mira] Aspirare a cose grandi. Mira si dice quel segno, che
  è nella canna dell'archibuso, o nelle balestre, nel quale s'affissa l'occhio per aggiustare
  il colpo al berzaglio. E di qui \textit{Porre la mira a una cosa} s'intende \textit{Volgere
    il pensiero}, o \textit{aspirare a una cosa}.
\item[FRVSTAMATTONI] Si dicono Quelli, che giornalmente vanno in una
  casa, o bottega, e non vi spendono mai un soldo, o vi portano utile alcuno,
  E si dicono Frustamattoni, perché non son d'altro giovamento, che frustare,
  cioè spazzare, e ripulire con le scarpe i mattoni; i quali son quelle lastre fatte di
  terra cotta, con le quali si lastricano i pavimenti delle stanze, da i Latini detti
  \textit{Lateres}.
\item[SCORPORO di borsa] Spendere. Scorporare vuol dit Estrarre da una massa, o
  da un corpo, o quantità di roba, o una porzione di essa.
\item[DAR di piglio] In questo luogo vuol dir Pigliare, impadronirsi; ed alle volte
  vuol dir Principiare come sotto C.6, stan 60.
\item[ESPORRE il ventre a mille stocchi] Vanti d'innamorati d'andare  soli contro
  a un'esercito intero, come i Poeti favoleggiano, che facessero i Paladini, che
  sono quei dodici Conti di Palazzo, ordinati da Carlo Magno per combattere
  contro a i nimici della S, Fede Cattolica, che furono detti \textit{Comites Palatini}, cioè
  Compagni nel Palazzo, che sono forse gli odierni Pari di Francia: the noi poi
  corrottamente chiamiamo Paladini, e con questa voce intendiattio. Haomé bravo.
\item[ALLOCCO] Specie d' uccello con il capo cornuto, come l'assiuolo, ma è
  più grande, e di colore lionato, con occhi grandi, e lucenti, È animal goffo,
  e se bene vive di rapina, tuttavia è tanto poltrone, che per cibarsi aspetta di pigliare
  gli uccelli, quando gli vanno scherzando attorno, tratti dalla di lui goffaggine;
  e quando se li avvicinano, non con rapacità, ma con flemma, e gravità
  non ordinaria gli prende col rostro, o con gli artigli; E da questa goffaggine
  nel far all'amore, ed aspettare gli uccelli, per Allocco intendiamo Uno, che
  se ne stia perdendo il giorno in vagheggiar Dame senza profitto, ed è lo stesso
  che \textit{Frustamattoni}, \textit{Colombi di gesso}, e simili.
  Con questo nome di \textit{Allocco} in molte parti d'Italia è chiamata ancora la Civetta,
  e credo, perché è di figura, se ben più piccola; simile a quella dell'Allocco, e
  vive con le medesime arti.
\item[CERCAR col fuscelino] Cercar minutamente, e con diligenza; \textit{Il tale cerca le
busse col fuscellino} vuol dire; Il tale fa tutto quel che egli può, per esser percosso,
  o per toccarne. Questo detto vien da quei ragazzi dell'infima plebe, i quali dopo
  che è venuta in Firenze una gran pioggia, che habbia fatta correr l'acqua
  per la Città, vanno cercando per le strade vicine alle gran fogne, che portano in
  Arno, se trovano fra le commettiture delle lastre delle strade spilli, chiodi, ed
  altre cose simili portate, e lasciate quivi dall'acque correnti; e per far ciò si
  servono d'uno stecco, o fuscelletto di scopa, o d'altro, col quale vanno rifrugando
  i fessi di dette commettiture, e perché così gran diligenze son troppe al
  poco utile, n'è nato il suddetto proverbio, che ha l'acceanato senso, ed è lo
  stesso che chiamar' una cosa di la da i monti, detto sopra in questo C, stan.\ 19.
\item[BAIOCCO]. E parola, e moneta romana, la qual parola è talvolta usata da
  noi per intender Danari, come qui, che dicendo \textit{Non si parli di baiocchi} intende
  \textit{Non si parli di danari}, cioè di Spendere.
\item[NON hanno un becco d'un quattrino] Non hanno pure un denaro, e quella parola
  Becco si mette a maggiore espressione, quasi dica Non hanno ne pure un sol
  \textit{quattrino becco}; cioè cattiyo, e non il caso a spendersi; Se non volessimo dire,
  che venisse questo detto dall'antica moneta Romana di rame; nella quale era impresso
  da una banda il volto di Giano con le corna, e dall'altra un rostro di nave,
  e che il dire; Un becco d'un quattrino sia lo stesso, che dire, ne anche la
  parte d'un quattrino, cioè  la faccia di Giano, che è cornuta.
\item[PROMETTE Roma e Toma] Promette cose grandissime, e che da persona
  alcuna non si possono mantenere, o osservare; i Latini dissero \textit{Maria, Montes polliceri},
  La voce toma non so che habbia nel nostro idioma significato alcuno, e stimo;
  che sia usata in questo detto per darle la rima con la parola Roma; Se forse
  non fusse il verbo spagnuolos tomar, che vuol dir torre, o pigliare, ed intendersi
  \textit{Ti prometto Roma}, (che è a dir tutto il mondo) \textit{e tu toma}, cioè piglia quel che
  ti piace. Lasca Nov. 8. \textit{Però non restava, di sollecitarla promettendole Roma, e toma,
    come se egli fusse il primo Principe del mondo}.
\end{description}

\section{Stanza LXIX, LXX \& LXXI}

\begin{ottave}
  \flagverse{69}Era tra molti suoi più fidi amanti\\
Un ciarlon, che però detto è il Cornacchia,\\
Ed è di quei pittor, ch' i viandanti\\
Con lo stioppo dipingono alla macchia;\\
E perché nella lingua ha il suo in contanti,\\
Molto si vanta, assai presume, e gracchia;\\
E finalmente colorisce, e tratta\\
Questo negozio, come cosa fatta.
\end{ottave}

\begin{ottave}
  \flagverse{70}Scrive un viglietto poi segretamente\\
Ad un compagno suo capobandito,\\
Dicendo, che veduta la presente,\\
Il suo bagaglio subito ammannito,\\
Di notte tempo meni la sua gente\\
A Rimaggio alla Svolta del Romito;\\
Ma vada alla spezzata, e pe i tragetti,\\
E senza pensar' altro ivi l'aspetti.
\end{ottave}

\begin{ottave}
  \flagverse{71}Andò la carta, e quei c'hebbe l'intesa,\\
Come quel ch' invitato era al suo giuoco\\
Andonne, e guidò seco a quell'impresa \\
Cent'huomin con le lor bocche di fuoco,\\
Quivi il Cornacchia, e quella buona spesa\\
Di Bertinella giunsero fra poco,\\
Anch'eglino con grossa, e folta schiera\\
D'una gente da bosco, e da riviera.
\end{ottave}

Fra questi suoi più fedeli amanti era un tale detto il Cornacchia. Costui era
uno con tal soprannome; perché havea la voce d'un suono simile al gracchiare
della cornacchia, ed era un solennissimo briccone, e ladro, e spia. Questo da a
Bertinella il negozio per fatto, e s'ammannisce a far la sorpresa di Maimantile;
con scrivere ad un capo di ladri da strada suo corrispondente, che si conduca a
Rimaggio con le sue genti con armi, e panni, e l'aspetti alla Svolta del Romito,
che è una contrada in vicinanza di Malmantile. Eseguì l'amico, giunse
con cento huomini ben' armati nel luogo ordinatogli: fra poco vi arrivò ancora
il Cornacchia con Bertinella, con grande schiera di bravi furbi, che questo intende
\textit{gente da bosco, e da riviera}; che i Latini dissero \textit{homines omnium horarum}.

\begin{description}
\item[CIARLONE] Uno, che chiacchiera assai, L'Autore intende, che chiacchierava
  assai alla giustizia, cioè faceva la spia, e perciò detto Cornacchia, che è uccello
  di cattivo augurio; perché il suo ciarlare era di danno al prossimo. Ed in
  vero costui, mentre visse, fu sempre chiamato il Cornacchia, o per questa causa,
  o per quella che habbiamo accennato sopra.
\item[DIPINGERE alla macchia] Dipinger un Ritratto senz'haver d'avanti l'originale,
  ma col solo haverlo veduto. E l'Autore però intende, che egli era ladro di strada,
  e pigliando la voce macchia nei suo vero senso di selva densa, dice,
  che alla macchia ritraeva i viandanti con lo stioppo, ed intende Assaltava la gente alla
  strada con l'archibuso per rubarla, Questa però è finzione, perché il Cornacchia,
  se hebbe la malizia, non hebbe già tanto cuore di far' il ladro di strada, e l'Autore
  lo finge tale per mostrare, che egli era un furbo da far qualsivoglia sciagurataggine.
\item[HA nella lingua il suo in contanti] Vuol dire eloquente, pronto di lingua.
\item[VANTARSI] Promettersi molto di se medesimo, Esaltar le proprie opere,
è il Latino \textit{Iactare}.
\item[GRACCHIARE] Cicalare con poco fondamento, Vedi sotto C. 4. stan 29.
  C. 7. stan. 9, e C. 8. stan. 65. Ma perché costui è chiamato Cornacchia, il Poeta si
  serve del verbo gracchiare per esprimer il cicalar di esso.
\item[COLORIRE] Metafora assai usata, e vuol dire discorrer d'una cosa con aggiustatezza,
  con termini proprj, e con colori rettorici per persuadere, e fare
  apparir vera quella tal cosa, della quale si discorre.

\item[VIGLIETTO] o \textit{biglietto}. Vuol dir lettera; Ma strettamente significa quella
lettera, che si manda in luoghi vicini, come da una casa all'altra, dentro alla
medesima Città, o Terra. Voce che forse viene dal Francese \textit{Poulet}, che vuol dir
lettera, amorosa, o da \textit{Billet}, Vedi sotto C. 6. stan. 54.

\item[BAGAGLAIO] Quelle some, che si conducono appresso gli eserciti per utile, e
  comodo dell'armata, o dietro qualsivoglia viaggiante per servizio della propria
  persona; si dicono \textit{Bagaglio}, forse dal Francese \textit{Bagage}; o dal verbo Bainlare,
  che val Portare, come habbiamo osservato sopra in questo C. stan. 62. alla voce
  Baule, ed è quel che i latini dicevano \textit{impedimenta}.

\item[AMMANNIRE] Metter'all'ordine, Allestire, approntare; quasi dica \textit{ad
  manus habere}. Dante Purg. C. 23.  \begin{verse}
    Di quel ch'il Ciel veloce loro ammannna,
\end{verse} ed al C. 29.\begin{verse}La virtù, c' a ragion discorso ammanna.\end{verse}

\item[ALLA spezzata] A pochi insieme per volta, non in squadre o truppe formate.
  Si dice anche \textit{Alla sfilata}, Vedi sotto C. 6. stan. 85. ed è il \textit{diminutim}
  dei latini.

\item[PE i tragetti] Per le balze, per luoghi, e strade non praticate; e il puro Latino
  \textit{Traiectus}.

\item[HAVER l'intesa] Rimaner d'accordo. Haver l'instruzione di come si debba contenere.
\item[INVITAR uno al suo giuoco] Chiamar' uno a fare una cosa, che sia di suo genio,
  e gusto. I Latini dissero \textit{Musas hortari ut canant}, ec.

\item[BOCCHE di fuoco] Intendiamo Ogni arme da fuoco, atta a portarsi addosso,
  come Moschetti, archibusi, pistole, e simili.
\item[BVONA spesa] Huomo astuto, e scaltrito, e suona lo stesso, che Tristo,
  e Volpe vecchia.
\end{description}

\section{Stanza LXXIL \& LXXIIT.}
\begin{ottave}
  \flagverse{69}Dopo ch' insieme tutti fur costoro\\
Si fece de' più degni una semblea,\\
Del come discorrendo fra di loro\\
Sorprender' il Castello si dovea,\\
Ond'il Cornacchia in mezzo al concistoro\\
Rizzato in pié con gran prosopopea,\\
Ed una toccatina di cappello,\\
In tal modo cavò fuora il limbello.
\end{ottave}

\begin{ottave}
  \flagverse{69}Io so c'a un'ignorante, a un'idiota\\
L'esser il primo a favellar non tocaa;\\
Ma perdonate a questa zucca vota,\\
Signori, s'io vi rompo l'huovo in bocca;\\
Scricchiola sempre la più trista ruota,\\
Così la lingua mia più rozza, e sciocca\\
V'infastidisce, è ver ma v'assicura,\\
Che Malmantile è nostro a dirittura.
\end{ottave}

Ragunati costoro insieme, quei più degni si ristrinsero a consiglio, per fermar
il modo, che si doveva tener per sorprender Malmantile, ed il Cornacchia, fatte
sue cirimonie, comincia a mostrare il modo certo di pigliare detto Malmantile.
\begin{description}
  \item[PRESOPOPEA] Questa voce, che vien dal Greco Prosopopea compostasdi
due dizioni \textit{Prosopon}, che suona \textit{personam} (ed a noi Personaggio) e poeeo, che
suona \textit{facto}, se bene è una figura con la quale fingesi un perlonaggio, come
farebbe introdurre una cosa inanimata, che parli con una animata, \& è contra, tuttavia
noi ce ne serviamo per intender una certa superbia, arroganza, fasto, o
presunzione di se medesimo, dimostrata con gli atti; di che vedi sorto C.6. stan. 85.
Ed in tal senso, secondo il Monosino era pigliata ancora da i Greci. Si dice
da noi anche sussiego, derivando la voce dallo Spagnuolo.
\item[VNA toccarina di cappello] Atto che esprime detta Prosopopea.
\item[CAVÒ fuora il limbello] Cominciò a parlare. Limbelli; Si dicono quei pezzi
  di pelle di bestia, che dalle dette pelli tagliano i Conciatori, donde poi
  \textit{limbellucci} i ritagli delle pelli più sottili, come di cartapecora, che servono per far
  colla da Pittori. E perché tali \textit{limbelli}, quando son freschi; ed umidi sono simili alle
  lingue, perciò per \textit{limbello} intendiamo lingua; e però detto scherzoso, come si
  vede, che l'usò il nostro Autore anche sopra in quella sua lettera alla Sereniss.
  Arciduchessa, riportata da me nel Proemio. \textit{Cavò fuora il limbello, e disse le sue
    Sillabe, come un Tullio}, ec.
\item[IGNORANTE, \& idiota] Sono Sinonimi, ne vi si fa alcuna differenza, se
  bene strettamente \textit{Ignorante} vuol dire uno, che non sa nulla, e \textit{Idiota} par che si
  convenga a coloro, che non hanno cognizione di lettere.
\item[ZVCCA] S'intende il capo dell'huomo per la similitudine, e Zucca vera vuol
  però dire testa senza cervello, che si dice \textit{vota di sale}, o non haver sale in zucca.
  E questo perché è solito nelle cucine tenere il sale in una Zucca secca appesa al
  muro del Cammino. Vedi sotto Can. 4. stan. 15. I Latini pure dicevano \textit{sale} per
  giudizio, e trovasi in Catullo.
  \begin{verse}
    Nulla in tam magno corpore mica salis
  \end{verse}
Vedi sotto C. 8. stan. 26., e Marziale C. 7.
  \begin{verse}
    Nullaque mica salis, nec amari fellis in illis
  \end{verse}
\item[ROMPER l'huovo in bocca] Torre la parola di bocca a uno, ciò è Dire che
  doveva, o voleva dire un'altro. Terenzio disse \textit{Bolus ereptus e faucibus est}.
\item[SCRICCHIOLARE] Stridere, strepitare. S'intende quel romore, che fa
  nel muoversi un legno fortemente stretto, o aggravato da altro legno, o materiale
  duro; come appunto segue nelle ruote da carro. Ed il proverbio: \textit{Sempre
    Scricchiola la peggio ruota del carro}, Significa \textit{Il più sciocco della conversazione, vuol
    sempre parlare}, Detto antico, e vien dal Latino, che dice \textit{semper deterior
    vehiculi rota perstrepit}, ec.
\item[A DIRITTVRA] Cioè assolutamente, sicuramente, e senza difficultà aleuna,

\end{description}

\section{Stanza LXXIV.}
\begin{ottave}
  \flagverse{74}Credete a me: Ciascun si stia nascosto\\
In queste macchie, in questi boschi intorno\\
Ed io da voi fra tanto mi discosto,\\
Ne questa notte farò più ritorno.\\
Rivedremci colà doman sul posto,\\
Perché vicino al tramontar del giorno\\
Vi farò cenno, hor voi ponete mente,\\
E poi venite via allegramente.
\end{ottave}

\begin{ottave}
  \flagverse{75}Parte il Cornacchia, e corre presto presto\\
Da certi suoi amici contadini,\\
Da' quali le lor bestie piglia in presto\\
E carica più some di buon vini,\\
E di soppiatto, come fante lesto\\
Cavò di tasca certi cartoccini\\
Pieni d'alloppio, e dentro al vin li pone\\
Quello impepando, senza discrezione.
\end{ottave}

\begin{ottave}
\flagverse{76}Così carreggia, e giunto a Malmantile\\
All'aprir della porta la mattina\\
Scarica in piazza il vino, ed un barile\\
A regalar ne manda alla Regina.\\
Poi vende il resto a prezzo tanto vile,\\
C'ognun ne compra, e in fin che n'ha in cantina\\
Per rivenderlo altrui, il fiasco attacca,\\
Si cala al buon mercato, a quella macca
\end{ottave}

\begin{ottave}
\flagverse{77}Due, o tre fiaschi davane a quattrino,\\
Ed a' poveri davalo a Isonne,\\
Tal che tutti tuffandosi a quel vino\\
S'imbriacaron come tante monne,\\
E subito dal grande al piccolino\\
Tanto de gli huomin, quanto delle donne\\
Cascaro in sonnolenza sì gagliarda,\\
Che desti non gli havrebbe una bombarda.
\end{ottave}

Cornacchia instruisce i compagni di quello devon fare, e si parte, e va da,
certi contadini suoi amici, da' quali piglia le lor bestie in presto, e lo carica di
vino alloppiato, quale porta in Malmantile, e lo vende così a buon mercato, che
Ognuno ne comprò, e bevvero tanto, che tutti s'imbriacarono, e si messero a dormire

\begin{description}
  \item[PRESTO presto] Prestissimo: per la replica d'una stessa parola, che ha forza di
superlativo, come habbiamo detto altrove.
\item[DI soppiatto] Di nascosto. Vien dal verbo impiattare, che vuol dir Nascondere
  una cosa corporea, come s'è detto altrove.
\item[FANTE lesto] Huom sagace, astuto, e che sa il conto suo.
\item[CARTOCCINO] Diminutivo di Cartoccio, che è una piegatura di foglio, fatta
a Piramide usata da gli speziali per mettervi dentro zucchero, pepe, ed altro simile.
\item[ALLOPPIO] Specie di sonnifero composto di sugo di papavero, coagulato,
secco, e polverizzato, e d'altri ingredienti; e si chiama \textit{oppio}.
\item[CARREGGIARE] Venendo da carro dovrebbe intendersi solamente per Camminar
  col carro, o traghettar robe col carro, ma ci serve per lo più per intender
  ogni sorte d'andare, o camminare, a piede, o a cavallo, conducendo o non
  conducendo roba.
\item[BARILE] Vaso di legno per uso di portarvi olio, vino, ed ogni altro liquore
  simile, ed è la misura comune del vino, capace di 20. fiaschi, e quello da olio
  di 16 fiaschi. Tali vasi son composti, ed aggiustati in maniera da adattarne due
  per volta addosso a una bestia da soma.
\item[ATTACCA il fiasco] Coloro, i quali in Firenze vendono il vino a fiaschi alla
  propria casa, attaccano per segno di ciò sopr'alla porta un fiasco, acciò che il
  popolo vegga il luogo, dove si vende il vino: e pero quando si dice \textit{Il tale ha oggi
  attaccato il fiasco}, s' intende, \textit{il Tale oggi ha cominciato a vendere il vino a fiaschi}.

\item[SI cala a buon mercato] Si lascia persuadere dal prezzo vile a comprare. È
  traslato da gli uccelli, che si calano alla vista della preda.
\item[MACCA] Abbondanza grande. Vien forse dal Latino Mactus, che s'intende
  abbondanza grande, quasi \textit{Magis auctus}. Plau, milit, 4.22. \textit{Macte amare}. E
  si trova \textit{Puer macte virtute}; giovanetto virtuosissimo. Dice il Vocabolista
  Bolognese, che macco vuol dir' abbondanza, che induce disprezo, e così è vero nel
  parlar nostro, che si dice \textit{smaccare} per intender Vituperare, o screditare.
\item[A Isonne] Per niente. Senza spesa, È detto plebeo, ed è usato per lo più tra
  i battilani, i quali hanno per tradizione, che Isonne fusse già un'huomo de' loro,
  il quale mangiava tanto volentieri a spese d'altri, che essendo morto, e seppellito
  già di qualche mese, scappasse dell'avello al discorso, che da alcuni si faceva
  di voler dar mangiare a tutti i Battilani per tre giorni, senza che spendessero,
  Costui havea due fratelli l'uno detto Salicone, e l'altro lo Scrocchina, e però
  \textit{scroccare} mangiare a \textit{Salicone}, a \textit{Scrocco}, e a \textit{Isonne} significano tutti Mangiar senza
  spendere, che Terenzio disse \textit{Asymbolum} composto dalla proposizione A, che
  suona Senza, e \textit{symbolum}, che vale quota, o scotto, e significa senza denari; E si
  come ne i Latini questo \textit{Asymbolum}, fu usato da i parassiti, e guatteri, così il nostro
  \textit{Isonne}, è usato dalla plebaglia, fra la quale è nato.

  Può anch' essere, che questo detto \textit{Isonne} venga da un Iiogo poco fuori di Firenze
  detto \textit{Isonne}, dove anticamente andavano a desinare aicune volte l'anno
  molti battilani, senza spendere, non perché veramente non spendessero, ma perché
  il denaro, che si spendeva in quel desinare, era di mance fatte per le Pasque,
  S. Giovanni, e Carnevale, che messo in una lor corbona, si serbava, e distribuiva
  per questi desinari; e può essere, che questi battilani dessero tal nome
  \textit{Isonne} a quel luogo dove andavano a far questi lor desinari, chiamati da loro
  \textit{desinari a Isonne}; ma sia come si voglia, basta che appresso noi il termine \textit{Isonne} è
inteso per Senza spesa.
\item[TVFFANDOSI] Tuffarsi a una cosa, significa Pigliare, o fare assai una tal cosa.
\item[S'imbriacaron come tante monne] Vedi quel che s'è detto sopra in questo C. stan. 10.
\end{description}

\section{Stanza LXXVIIL}
\begin{ottave}
  \flagverse{78}Quando il Cornacchia vedde il suo disegno\\
Già riuscito, andò sopr'alle mura,\\
Ed ai compagni fece il detto segno,\\
Che bene havendo al tutto posto cura,\\
Saliro al poggio senz'alcun ritegno,\\
Senza sospetto haver, senza paura\\
Dietro al Cornacchia lor guidone, e scorta\\
Dentro al Castello entraron per la porta
\end{ottave}

\begin{ottave}
\flagverse{79}E perc' ognun dormiva, come un Tasso,\\
La donna fece farne una funata,\\
E condursegli a piedi a baciar basso,\\
E renderle il tributo ognun pro rata,\\
A Celidora poi restata in Nasso,\\
Cioè da' suoi vassalli rinnegata,\\
Già che tutti voltato havean mantello,\\
Comandò che baciasse il chiavistello.
\end{ottave}

\begin{ottave}
\flagverse{80}Ell'ubbidì, temendo, ancor di peggio,\\
E ben che fusse un pezzo in la di notte,\\
Il pigliarsene subito il puleggio\\
Un zucchero le parve di tre cotte.\\
Così finito il solito corteggio\\
Con due strambelli, e un par di scarpe rotte\\
Triffa, e strascina poi per la boccolica\\
Un tozzo mendicava all'accattolica
\end{ottave}

I Compagni di Bertinella veduto il segno dato dal Cornacchia, andatono a
Malmantile, ed entrati dentro, e trovati tutti a dormire gli legarono, e gli condussero
a render ubbidienza a Bertinella, la quale comandò a Celidora, che uscisse
del Castello, ed ellam tutta mal' all'ordine se n'andò, benché fusse assai di notte,
e si condusse a mendicare il vitto.
\begin{description}
  \item[GVIDONE, e scorta] Guidone s'intende Colui che guida; e Scorta è quello che
mostra la strada; ma la voce \textit{Guidone} è forse per scherzo presa dall'Autore nel
senso, che sopra stan. 65. e sotto al Cant, 8. stan.~72.
\item[FAR una funata] Legar con una fune più persone: Quando molti insieme
  commettono un delitto, si suol dire: \textit{Se vengono i birri, voglion far la bella funata}.
  Non perché crediamo, che vogliano effettivamente legargli tutti a una fune, ma
  intendiamo, \textit{Vogliono farne molti prigioni}, e così intendi nel presente luogo.
\item[BACIAR basso] Cioè inchinarsi a baciar i piedi in segno di vassallaggio.
\item[RIMANERE in Nasso] Dai più si dice \textit{rimanere in Asso}, e ciò segue per
  corruzione nella pronunzia, che tanto suona \textit{rimanere in asso} che \textit{rimanere in Nasso}
  come si dovrebbe dire, e significa abbandonato, senza aiuto, e senza consiglio;
  Ed è derivato dalla favola d'Arianna abbandonata da Teseo nell'Isola di Nasso;
  E si dice anche rimanere in su le secche di Barberia, il che corrobora che si debba
  dire \textit{in Nasso}, e non in asso che non ha verun senso, o allegoria. Vedi sotto
  C.\ 10.\ stan.~2.
\item[VOLTAR mantello] Rinnegare. Ribellarsi; andar da un partito all'altro. Il
  Lalli En. trav. C. 2, stan.~39.
  \begin{verse}
    Hor che mi lice di voltar mantello
  \end{verse}

\item[BACIARE il chiavistello] Andarsene senza speranza di tornare. Usiamo questo
  detto per esprimere che non si vuole, che quel tale, che è stato per li suoi
  mali portamenti scacciato d'una tal casa, viva con la speranza di ritornarvi, e
  pero si potrebbe dir con Vergilio \textit{Supremum vale dixit}.
\item[CHIAVISTELLO] Serratura da porte, o finestre, che confiste in un ferro
  lungo, il quale fa la sua operazione, passando per diversi anelli pur di ferro
  adattati nel legname; ed è il Latino \textit{vectis}.
\item[PIGLIAR il puleggio] Andar via. Pigliar il cammino, E' frase marinaresca, ma
  però usata comunemente in questi termini d'andar via presto. Dante Par. C. 23.
\begin{verse}
  \backspace Non è puleggio da piccola barca
  Quel che fendendo va l'ardita prora
  Ne da nocchier, c' a se medesmo parca.
\end{verse}

Da questa voce Puleggio viene \textit{spulezzare}, che vedremo sotto C. 7. stan. 18, che
pure significa Andar via. Forse si potrebbe dir anche \textit{prueggiare} verbo pur
marinaresco, che significa Andar via bel bello.

Vincenzio Tanara nella sua Economia del Cittadino in villa Lib. 6. trattando
dell'erba \textit{Puleggio} dice, che sparsa in luogo dove sieno pulci ha virtù di
scacciarle; onde può essere che da questo effetto dell' erba \textit{Puleggio} venga il presente
dettato. Da \textit{puleggio} forse anche vengono \textit{Pulegge}, che sono quelle piccole
girelle, che si congegnano, ne i legni per facilitare i veicoli, come farebbe dentro a i
regoli da piede alle scene, o prospettive da commedie per renderle più facili a
strascicarsi dentro a i canali in occasione di mutazione delle medesime scene.
\item[UN suechera le parne di tre cotte] Le parve d' haverla a buon mercato: le parve
  d'haver fortuna grandissima, perché s'aspettava malto peggio. Lo Zucchero
  di tre cotte fatte bene si stima che sia il miglior grado di perfezione, della
  quale sono tre i gradi. secondo il detto \textit{omne trinum est perfectum}. Ed i Franzesi
  denominano il superlativo col tre, cioè buono, for buono, e tre buono\footnote{bien, \textit{fort} bien, \textit{très} bien}, per
  buono, molto buono, buonissimo,:

\item[STRAMBELLE] Vesti vecchie, e stracciate. Vedi sotto C, 3., stan.~65.
\item[UN tozzo] Detto così assolutamente senz' altra aggiunta vuol dire un pezzo di pane.
  E \textit{frustum panis}, che usò Dante nel Parad. C. 6. \textit{Mendicando sua vita a frusto a frusto}.
\item[TRISTA, e strascina] Huomo tristo vuol dire Huomo mal vestito, e Strascino
  suona quasi lo stesso, perché Strascini chiamiamo alcuni huomini, i quali vanno
  comprando carne fuori della Città, e l'introducono in Firenze occultamente per
  rubarne la gabella, e perché costoro son sempre unti, sudici, e stracciati, perciò
  dicendosi \textit{Strascino} intendiamo mal' all'ordine di vestito, ec.
\item[BOCCOLICA, e accattolica] Sono due parole dette per scherzo, e per la similitudine
  che hanno con Bocca, e con Accattare, e per parlare Ianadattico, non
  sono però fuori dell'uso della gente più Civile, la quale spesso si serve di parole
  latine a quel proposito, che le pare che facciano giuoco stroppiandole, e interpretandole
  a lor modo, come le presenti \textit{Boccolica}, e \textit{accattolica} che l'una vuol dir Bocca,
  e l'altra Accattare, e così intendesi che Celidora accattava per mangiare. Tal'uso
  d'allusione scherzosa era pur'anche appresso ai Latini trovandosi \textit{Ab Ilio nunquam
  recedis}, che par che voglia dire tu non ti parti mai dalla Città di Troia, e
  s'intende poi; tu non abbandoni mai l'Ilo intestino, cioè sempre mangi.
\item[MENDICARE] Vuol dire durar fatica a conseguire. \textit{Il tale mendica le parole},
  cioe Dura fatica a parlare; ma il suo significato più inteso è Chiedere elemosina,
  Dante Parad. C. 6.
  \begin{verse}
    \backspace Indi partissi povero, e vetusto,
    E s'il mondo sapesse il cor ch' egli hebbe,
    Mendicando sua vita a frusto a frusto, ec.
  \end{verse}
\end{description}

\section{Stanza LXXXXI, LXXXII \& LXXXIII}
\begin{ottave}
\flagverse{81}In tanto Bertinella del Reame\\
Garbatamente fecesi padrona,\\
E de' villaggi, e d'ogni suo bestiame\\
Prese il possesso in petto, ed in persona\\
Poi per letizia cavalieri, e dame\\
Regalò di confetti, e di pattona;\\
E segue ogn'anno di mandarne attorno,\\
Per la dolce memoria di quel giorno.
\end{ottave}

\begin{ottave}
\flagverse{82}Tosto che ci hebbe fitto il capo, volle\\
C'ognun serrasse il traffico, e il negozio,\\
Donando a ciascheduno entrate, e zolle,\\
Acciò se la passasse da buon sozio,\\
Ed allegro, a piè pari, ed in panciolle\\
Senza briga vivesse in pace, e in ozio,\\
Ognun vi s' arvecò di buona gana,\\
Che la poca fatica a tutti è sana,
\end{ottave}

\begin{ottave}
\flagverse{83}Così mai sempre in feste, ed in convito\\
Tirano innanzi questi spensierati;\\
Ne moverebbon per far nulla, un dito,\\
Ben ch' ei credesson d' esser' impiccati;\\
Non teme della Corte, chi e fallito,\\
Che tutti i giorni a lor son feriati;\\
Non v'e giustizia, ne il bargel va fuora,\\
Se non per gastigar chiunche lavora.
\end{ottave}

Sbandita Celidora dal regno, Bertinella prese l'attual possesso di tutto lo stato,
e per acquistarsi la benevolenza de' sudditi cominciò dal regalare le dame, e
cavalieri, con regali degni della vilissima condizione di se medesima, ed appropriati
alle qualità de' Cavalieri, e Dame di Malmantile; poi con feste, ed allegrie
per contentare il popolo, e con levare i Ministri della giustizia tanto odiosi alla
plebaglia, e con fare altri ordini che si leggono nelle presenti ottave.

\begin{description}
\item[IN petto, ed in persona] Attualmente, e corporalmente. \textit{Animo \& corpore}.
\item[PATTONA] Torta, o pane fatto di farina di castagne, con altro nome
  detto \textit{polenda}, dal Latino \textit{Polenta}, che era vivanda fatta di farina d'orzo con
  altre polveri odorifere secondo Varrone. È vivanda vilissima appresso di noi; e
  da questa sua viltà habbiamo un detto di disprezzo, che è; \textit{Mangiapattona},
  \textit{Mangiapolenda} a un huomo vile, e buono a poco. Qual detto usò Plauto chiamando
  questi tali \textit{Pultiphagj}; ma il disprezzo non nasceva dalla viltà della \textit{polenta},
  (che era finalmente il cibo comune anche per le persone di garbo, e generalmente
  mangiando questa sorte vivanda i Romani vissero lungo tempo, Vedi Plin.
  lib. 18. cap. 8.) nasceva bene dall'intendersi con tal detto un huomo buon'a
  poc'altro, che a mangiare, e come noi diciamo \textit{Sparapani}, \textit{Votamadie},e simili

\item[V'hebbe fitto il capo] Se n'era'impadronita: N'haveva preso l'attual possesso;
  perché essendo il capo la più nobile, e principal parte della persona, noi diciamo
  \textit{Ficcare il capo in un luogo} per intendere Entrare in un luogo, e pigliarne il
  possesso personalmente.
\item[TRAFFICO] e negozio. Sinonimi, se bene \textit{traffico} par, che si ristringa all'arti
  manuali; onde con dire \textit{Traffico}, e \textit{negozio} intende non lavorare, ne
  mercanteggiare, o negoziare.

\item[ZOLLA] È il Latino gleba, che vuol dire Pezzo, o massa di terra smossa,
  come s'è accennato sopra in questo C. stan. 57., ma qui pigliando la parte per il
  tutto, intende terreni fruttiferi: \textit{Il tale ha delle zolle}, comunemente s'intende
  Ha de' terreni.

\item[SOZIO] Dal latino \textit{Socius}. Compagno \textit{Viver da buon sozio} vuol dir Viver
  da buon compagno, alla reale, ed alla schietta. E questa voce Sozio non so che
  sia usata se non in questo caso, e con l'aggiunta di \textit{buono}, o \textit{malo}: dicendosi
  Il tale è buon sozioxe, o \textit{non è mal sozio}, per intendere E' galant'huomo.

\item[A piè pari, ed in panciolle] S'usa questo detto per esprimere Un huomo poltrone,
  che non voglia far'altro, che godere i suoi comodi, e la voce \textit{panciolle}
  è composta di due parole, cioè \textit{pancia}, ed \textit{olle}, e suona pancia di pentola, la quale
  col posar pari, e con quella sua gran pancia è il vero ritratto della: comodità, e
  poltroneria. Il Bronz. nel Cap. in lode della Galea dice.
  \begin{verse}
    \backspace Guarì, ma in capo al giuoco, come volle
    Il Ciela, ne fu tratto il poverino,
    E fu privato di stare in panciolle.
  \end{verse}

\item[BRIGA] Noia, fastidio, fatica. Qui è preso per faccenda, o pensiero d'operare.

\item[DI buona gana] Molto volentieri. È detto spagnuolo, e la voce gana è usata da
  noi per intender Voglia, o gusto grande. \textit{Il tale mangia di gana}; \textit{Lavora di gana}, ec,

\item[SCIOPERATO] Uno che non ha, e non vuole haver faccende. Vedi sopra,
  stan. 29. Scioperati s'intendono quei Cittadini, che senza arte, o impiego vivono
  con le loro entrate.
\item[CORTE] Intendi la Corte della giustizia da i Latini \textit{detta Curia} a differenza
  di \textit{Aula}; e vuol dire Ministri della giustizia.
\item[FALLITO] Uno che negoziando ha fatto così gran debito, che non ha
  possibilità di pagarlo. E il latino \textit{decoctus, qui fallit creditores, ipsumque fefellere negacia}.
\item[TUTTI i giorni son feriati] Sempre è festa per loro; Feriato s'intende quel giorno,
  nel quale ancor che lavorativo non si tien da i Magiftrati ragione, e non si
  possono fare esecuzioni civili contra a i debitari, e questo intende dicendo \textit{Non
  teme della corte, chi è fallito}, perché è feriato, e non può esser menato prigione.
\end{description}

\section{Stanza LXXXIV}

\begin{ottave}
\flagverse{84}Ma s'io non erro il tempo è già vicino,\\
Che n'ha a venir la piena de' disturbi, \\
Mentre doman per far un buon bottino\\
Andremo a dar'addosso a questi furbi.\\
Così panno sarà di Casentino,\\
Ne se lamenti alcuno, o si sconturbi;\\
Che che nuoce al compagno in fatti, o in detti\\
Deve saper che; Chi la fa l'aspetti,
\end{ottave}

Baldone, havendo fatto il detto raccanto della cacciata di Celidora, dice
sperare, che sia vicino il tempo, nel quale faranno gastigati coloro, che hanno sorpreso
Malmantile, perché il giorno futuro vuol andare a dar loro addosso.
\begin{description}
\item[HA da venir la piena de' disturbi] Ha da venir grandissima quantità di disgusti a
sturbare i loro commodi. E \textit{Piena} diciamo quando Arno, o altro Fiume cresce
per le pioggie.
\item[SARA' panno di Casentino] Casentino è una Regione in Toscana, dove si fabbrica
  una specie di panni, che bagnati scemano di lunghezza, e larghezza perché
  rientrano. E da questo detto \textit{sarà panno di Casentino}, intendiamo Rientrerà,
  cioè tu hai fatto a me questo, ed io farò a te il simile, cioè Mi vendicherò.
\item[CHI la fa, aspetti] Chi fa un torto al compagno, aspetti pure d'esser contraccambiato.
  Il Petr. disse;
  \begin{verse}
    Chi si prende diletto di far frode,
    Non si dee lamentar s'altri l'inganna,
  \end{verse}
E questi due versi posson servire per dichiarazione delli quattro ultimi della
presente ottava.
\end{description}
\section{Stanza: LXXXV.}
\begin{ottave}
\flagverse{85}Qui racque il Duca; e subito rattacca,\\
Col dire alla cugina in voce bassa\\
Che, perch'egli ha la bocca asciutta, e stracca\\
Il soggiunger a lei qualcosa lascia\\
Non ho che dir (gli rispond'ella) un hacca,\\
Oltre che la sarebbe carne grassa,\\
Dì più tosto, in che mo noi siam parenti,\\
Ch'io non paia a costor de gl'Innocenti;
\end{ottave}

\begin{ottave}
 \flagverse{86}Ed io che non ne ho gran cognizione,\\
E sempre me ne sono stata a detta \\
(Che tutta la mia gente andò al cassone,\\
Come tu sai ch'io ero fanciulletta:)\\
T'udirò volentieri. Allor Baldone\\
Soggiunse: Or or ti servo, e a tanta fretta,\\
Perché non gli moria la lingua in bocca,\\
Ricominciò quest'altra filastrocca.
\end{ottave}

Baldone termina il discorso, e volto a Celidora le dice, che ella soggiunga,
se ha di più; ed essa dicendo, che non ha che soggiugnere lo prega a narrare, in
che modo sieno parenti: E Baldone s'accinge a contentarla. E qui termina il
nostro Poeta il suo primo Cantare.

\item[NON ho che dire un hacca] L' H vogliono, che non sia lettera, ma semplice
  aspirazione, e però dicendosi \textit{Non ho che dire un hacca}, è lo stesso che dire: \textit{Non
  ho che dir nulla}.

\item[SAREBBE carne grassa] Stuccherei il popolo; Mi renderei odiosa. Il Lasca
  Nov. 4. dice: \textit{E poi io non vorrei anche tanto infastidirlo, che egli m'havesse a dire,
    che io fussi carne grassa}. La carne grafia suole a i più che la mangiano cagionare
  nausea; il che diciamo stuccare.

\item[CH' io non paia costor de gl'Innocenti] Che costoro non pensino, che io sia
  bastarda, o senza parenti. In Firenze lo spedale de gl'Innocenti si chiama quello,
  nel quale si mettono ad allevare i bambini, per lo più, nati di congiunzioni
  illecite, i quali corrottamente chiamiamo \textit{Nocentini}. Vedi sotto Cant.\ 10.\ stan.~7.

\item[ME ne sono stata a detta] Non ho cercato di saperne più là; ma ho creduto quel
  che m'è stato detto, o raccontato.

\item[LA mia gente andò al cassone] Mio padre, mia madre, e tutti gli altri miei parenti
  morirono; che per mia gente in questo luogo, ed in questi termini s'intende
  Miei parenti, e non altri.

\item[A tanta fretta] Subito, Prestissimo.
\item[NON gli moria la lingua in bocca] Era loquace, eloquente. Havea facilità a
  parlare. È lo stesso che \textit{Havere il suo in contanti nella lingua} come s'accennò
  sopra stan. 69.
\item[FILASTROCCA] Serie di parole, e per lo più s'intende d'un discorso male
  ordinato, e proprio del racconto, che talora fanno le balie a' Fanciulli in quelle
  lor novelle, come appunto è questa che narra Baldone, che l'Autore oltre all'haverla
  sentita forse raccontare alle sue donne, quando era fanciullino,
  ha tratta dallo Cunto degli Cunti di Gianalesio Abbattutis.
\end{description}
\section*{FINE DEL PRIMO CANTARE.}

\chapter{Secondo Cantare}

\begin{argomento}
De i due gran figli del Signor d'Ugnano
Prodigioso il natal narra Baldone;
Come s'acquista moglie Floriano,
E vien dall'Orco poi fatto prigione.
Come Amadigi libera il germano;
E il mostro spaventoso a terra pone,
E dice al fin, che l'un di questi dui
Fu padre a Celidora, e l'altro a lui.
\end{argomento}

\section{Stanza I.}
\begin{ottave}
 \flagverse{1}Era in Ugnano il Duca Perione, \\
Che sempr'all'Altarin fidecommisso \\
Faveva notte, e di tanta orazione, \\
E tante carità, ch'era un subbisso.\\
Ne per altro era tutto bacchettone,\\
Che per un suo pensiero eterno, e fisso\\
D'haver prole, perché della sua schiatta\\
Non v'era, morto lui, ne can, ne gatta.
\end{ottave}

Il Duca Baldone dà principio alla narrativa del parentado, che passa fra lui,
e Celidora, come havea promesso  nell'antecedente Cantare, e dice; Che fu già
in Ugnano il Duca Perione, il quale faceva molte opere pie per disporre il Cielo
a concedergli prole. La favola del nascimento di questi figliuoli trovasi nello
Cunto degli Cunti di Gianalesio Abbattutis Giorn. 1. Cunto 9. ll nostro Poeta
pero non la cavò di quivi; ma la narrò, come l'haveva sentita contare alle sue
donne, quando era fanciullo; e questo è certo, perché questa era nel suo primo
Poema fatto molto prima, che il Basile Autore dello Cunto de li Cunti la stampasse,

\begin{description}
\item[ALTARINO] Così chiamiamo un' inginocchiatoio a foggia d' altare, il quale
  per lo più si tiene allato al letto per inginocchiarsi, e fare orazione.
\item[STAR fidecommisso in un luogo] è detto iperbolico, che significa Star moltissimo
in un luogo; che qui vuol dire Stava sempre, o non si levava mai dall'Altarino;
che s'intende faceva orazioni infinite.

\item[TANTE carità ch era un subisso] Carità, ed elemosine infinite. Per denotare
  una quantità indicibile usiamo dire: \textit{Son tanti, che è un subisso}, \textit{un fracasso}, \textit{un flagello},
  e simili. Questa voce \textit{subbisso} vien forse dal Greco \textit{abyssos}, che significa voragine,
o smisurata profondità d'acque, come suona ancora nel nostro idioma,
donde \textit{subissare} Andar nel profondo, quasi dica \textit{sub abysso}.

\item[BACCHETTONI] Così chiamiamo noi certi colli torti, e graffiasanti, che
  stimano peccato il portare un fiore in mano, e credono poi di far'un'atto meritorio
  a dare a usura; con altro nome chiamati Ipocriti, cioè Pseudobeati; huomini
  da bene per interesse, e per gabbare il compagno; e sono insomma coloro,
  de' quali Giovenale disse: \textit{Qui Curios simulant, \& Bacchanalia vivunt}. E diciamo
  \textit{Bacchettone}, quali \textit{Va chetone}, perché questa Canaglia, che studia di simulare la
  bontà, per arrivare a suoi fini, è simile all'acque profonde, che vanno chete,
  delle quali parlandé Q. Curzio dice: \textit{Altissima quaeque flumina minimo labuntur sono}.
  E come queste acque son sempre di pericolo, così li \textit{bacchettoni} nella loro taciturnità
  occultano il malo animo, che hanno contro al prossimo. Il costume di costoro
  tocca Orazio lib. 1. Ep. 17. dicendo che son devoti di Laverna Dea de
  ladri.
  \begin{verse}
    Labra movens, metuens audiri; Pulchra Laverna,
    Da mihi fallere; da iustum, sanctumque videri.
  \end{verse}

Di questa voce \textit{Bacchettoni} si serve anche il Tassoni nella sua Secchia. \textit{Nimico
natural de' Bacchettoni}. Ed un dottissimo de' nostri tempi, il quale fa un
discorso poetico sopra a costoro, lo termina con dire \textit{Furfante, e bacchetton suona
il medesimo}, Vedi sotto C. 6. stan. 97. dove si dice esser lo stesso \textit{Bacchettoni}, che
\textit{Ipocriti}, i quali S. Matteo chiamò \textit{similes sepulchris dealbatis}; il Berni nel'Orlando
disse. \textit{O agghiacciati dentro, e di fuor caldi}, \textit{In sepolcri dipinti gente morta}.

Giovenale aggiunge al detto di sopra.
\begin{verse}
  Fronti nulla fides; quis enim non vicus abundat
  Tristibus obscoenis? castigas turpia, cum sis
  Inter Socraticos notissima fossa Cinaedos.
  \end{verse}

Di questi tali parla in diversi luoghi la Sacra Scrittura detestando tal vizio, come
abominevole, ma per brevità tralascio di riportarlo, contentandomi di chiudere
col detto dell'Evangelilta \textit{Atendite a falsis prophetis, qui veniunt in vestimentis
ovium, intrinsecus vero sunt lupi rapaces} e rimetter il Lettore a quello, che scrive
S. Matteo Evangelista al Cap. 6. 15.23.

Tale era appunto questo Perione, che faceva le dette Opere pie, non perché
veramente fusse buono, ma perché con esse pretendeva d'estorcer dal Cielo la
grazia d'haver figliuoli.

\item[SCHIATTA] Stirpe, Prosapia, famiglia.

\item[NON v'era, ne can ne gatta] Non vi rimaneva pur'uno. Plauto disse: \textit{Ne
  musca quidem domi est}, Del qual detto si servì quel servo dell'Imperator Domiziano
  che domandato, se Domiziano era solo in camera, rispose: \textit{Ne musca
  quidem est}, Perché Domiziano stava là dentro ammazzando le mosche. Ter.
  disse: \textit{Ne Sannione quidem relicto}.
\end{description}

\section{Stanza II.}
\begin{ottave}
 \flagverse{2}Così durò gran tempo, ma da zezzo,\\
Vedendo ch' ei non era esaudito\\
Essendo omai con gli anni in là un pezzo, \\
A mangiar cominciò del pan pentito;\\
E quant'ei far solea posto in disprezzo\\
Senza voler più dar del profferito,\\
Gettatosi all'avaro, ed al furfante\\
Cambiò la diadema in un turbante.
\end{ottave}

Continuò gran tempo Perione a far le narrate opere pie, ma veduto ch'ei non
era esaudito, e ch'ei non haveva figliuoli, e trovandosi già vecchio, perché veramente
egli era un di quei Bacchettoni furbi, che habbiamo detto sopra, e che
faceva bene solamente per interesse, si pentì d'haver fatto tante elemosine, ed
altro bene, e mutò costume.
\begin{description}
  \item[DA zezzo] Da ultimo. Forse meglio \textit{sezo}, venendo dal Latino \textit{secius} opposto
di \textit{ocius}. Vedi sotto C. 4. stan. 72.
\item[ESSENDO un pezzo in là con gli anni] Essendo grave d'età. Havendo molti
  anni. Vedi sotto C. 12, stan. 36.
\item[MANGIAR del pan pentito] Cioè si duole, si pente d' haver fatto del bene; ed
è quel \textit{facti poenitere} di Cicerone,
\item[POSTO in disprezzo quanto far solea] Cioè lasciando stare di fare elemosine, e
orazioni, ed altre opere pie come solea fare.
\item[SENZA voler dar del profferito] Senza voler dare più niente; e ne meno quello,
  che havea promesso, o proferto.
\item[GETTATOSI all'avaro] Divenuto avaro per elezione, o diremmo A posta.
\item[FVRFANTE] Vuol dir furbo scellerato, e ladro, e simili venendo dal latino
  barbaro \textit{foris faciens}, operante fuori del dovere, ma si piglia anche per Spilorcio,
  ed avaro, come è preso nel presente luogo.
\item[CAMBIO' la diadema in un turbante] Di Santo divenne Turco, che Diadema
appresso di noi vuol dire quell'ornamento, ò corona di splendori, che si vede
dipinto attorno alla testa de' Santi. Dice che cambio la diadema, che meritava
come Santo, in un turbante, cioè cappello da Turco, non che veramente si mettesse
il Turbante, ma intende, che d'huomo da bene diventò tutto il contrario.
\end{description}
\section{Stanza III}
\begin{ottave}
\flagverse{3}Di poi tutto diverso, e mal disposto\\
In modo degli Dei faceasi beffe,\\
Che s'egli udia trattarne, havria più tosto\\
Voluto sul mostaccio uno sberleffo;\\
La moglie un miglio si tenea discosto,\\
E dov'ei dava a' poveri a bizzeffe,\\
Quando picchiavan poi dalla finestra,\\
Facea lor dar il pan con la balestra.
\end{ottave}

Divenuto Perione tutto diverso da quel che era, come s'è detto, cominciò
anche a non stimar più gli Dei, anzi gli strapazava in modo, che havrebbe voluto
più tosto un sfregio sul viso, che sentirgli nominare; sbandì la moglie, ed in
vece di dar limosine a i poveri gli bastonava.
\begin{description}
\item[DIVERSO] Cioè differente da quel ch'era prima. Se ben questa voce diverso
  significa ancora stravagante. Vedi sotto C. 8. stan. 17. ed in questo senso la piglia
  Franco Sacchetti Nov. 29, E questa natura pare a me, che fusse delle strane, e diverse
  che trovar si potessero. E Nov. 78. \textit{Ed era un'huomo malizioso, reo, e di
    diversa natura}.
\item[FACEASI beffe] Si burlava. Non faceva stima. E il latino \textit{flocci facere}.

\item[SBERLEFFE] Taglio, o sfregio, che i Latini dissero stigma; \textit{Rigido signata
 stigmate fronte}. E perché gli sfregi in sul vifo sono cosa ignominiosa, come s'è
  detto sopra C. 1. stan. 66. da ciò si deduce che Perione havria più tosto sopportata
  ogni grande ingiuria, ed ignominia, che sentir nominare gli Dei. Il Coppetta
  nel Cap. in lode della sig. Ortenzia piglia la voce \textit{sberleffe} in significato di burlare
  uno, con oltraggi, e punture, che hoggi da molti si dice Fare uno scappeneo.
\begin{verse}
Allor l'amico in mezzo a i dolor miei
Mi fece uno sberleffe di velluto,
E mi fece arrossir dal capo a piei.
\end{verse}
E più sotto nel medesimo capitolo lo stesso mostra, che habbiamo anco il verbo
sberleffare dicendo.
\begin{verse}
E col rider di grazia andate piano,
Che non è per infermi util conforto,
E chi vuol sberleffar, sberleffi in vano.
\end{verse}

L'origine da questa voce \textit{sberleffe} vien forse da \textit{Berlina} in questo modo:

Si suole alle volte, dopo haver tenuto in Berlina i ladroncelli, segnargli in
qualche parte del corpo con un ferro infuocato, acciò che fieno dalla Giuitizia
riconosciuti, se altra volta per commessi delitti li tornassero nelle mani. E di
questi segni vedremo sotto C. 6. stan. 54. Ciò si costumava ancora appresso gli
antichi Romani ne i servi fuggitivi, e gli segnavano nella fronte come si cava da
Aulonio Epig. 15. che parlando di un servo nominato Pergamo dice.
\begin{verse}
  \backspace Iam segnis scriptor, quam lentus, Pergame, cursor
  Fugisti, \& primo captus es in stadio;
  \backspace Ergo notas scripto tolerasti Pergame vultu,
  Et quas neglexit dextera, frons patitur.
\end{verse}

Et aggiungesi alla voce \textit{berlina} quella finale \textit{effe}, da quella lettera maiuscola F,
che è il segno, o marchio, col quale si marchiano i detti delinquenti. Che cosa
sia berlina. Vedi sotto in questo C. stan 15.

\item[MOSTACCIO] Faccia, Volto, ec.

\item[TENEA la moglie discosto un miglio] Tenea la moglie lontana da se, intendi non
  volea più commerzio con la moglie. Lat; \textit{secubabat}.

\item[DARE a Bizzeffe] Dare, o donare largamente. Questa voce, che è composta
  dal latino \textit{bis, \& effe}, cioè due volte, f, vuol dir pienamente, largamente,
  abondantemente, e simili; Quando il sommo Magistrato Romano intendeva
  fare ad un supplicante la grazia senza limitazione, ma pienamente faceva il rescritto
  sotto al memoriale, che diceva \textit{Fiat Fiat}, che poi per brevità costumarono
  di dimostrare questa pieneza di grazia con segnare i memoriali con sole due
  effe, onde quello che conseguiva tal grazia diceva: Io ho havuta la grazia a \textit{bis
    effe}, cioè due volte ff che s'intende grazia intera, e piena, al costrario di quella
  limitata, che era con una sola effe aggiontavi la limitazione, o condizione
  con la quale il Magistrato havea conceduta la grazia. E' da questo \textit{bis effe} s'è poi
  corrottamente introdotto il dir Bizzeffe, che ha il signiticato, che habbiamo
  detto. Nella storia di Semifonte scritta sopra 300 anni sono, si legge al trattato
  terzo. \textit{La Terra di Semifonte era piena di torri merlate, e piombatoie, e di Torricelle
    a bizzeffe}.
\item[DARE il pan con la balestra] Vuol dice strapazare. Fare in maniera, che il
  benefizio sia di disgusto a chi lo riceve. Deriva forse dall'uso, che era in Firenze
  avanti che usasse andar a caccia con l'archibuso, di tenere al suo servizio huomini
  a posta i quali con qualche fsalvaticina mantenessero le mense de i grandi, e
  questo esercizio essendo d'utile, ma assai laborioso, può haver data origine a
  questo Proverbio \textit{dare il pan con la balestra}, cioè accompagnato da fatica, e disagio
  grandissimo. Ma nel presente luogo intende che effettivamente facesse tirare
  balestrate a i poveri.

Si dice ancora in questo proposito. \textit{Porger il pane con la spada}, e ciò forse deriva
da quello, che fece Dionisio Tiranno a un tal Democle Filosofo, il quale
(perché adulando eccedeva in lodare le grandezze di quello stato di Dionisio)
egli fece sedere ad una mensa ripiena delle più esquisite vivande, che per un banchetto
reale inventar si potessero; e fece attaccare per il manico ad una setola
pendente con la punta sopr'alla sua testa, una spada sfoderata, la quale veduta
dal Filosofo, gli cagionò così grande spavento, che egli non potè se non con molta
paura, e con poco gusto pigliare di quei cibi. Di costui parla Orazio Od.\
pr. lib.~3.
\begin{verse}
 Districtus ensis cui super impia
 Cervice pendet, non siculae dapes
 Dulcem elaborabunt saporem.
 \end{verse}

Si dice ancora, a questo proposito, \textit{dare il par col bastone} che ha origine da
quel che fece il Piovano Arlotto; il quale per gastigar l'indiscretezza d'alcuni
cacciatori, che gli havevano lasciato in casa un branco di cani, quando a questi
dava il pane, l'accompagnava con una mano di bastonate, onde i poveri cani
s'erano assuefatti quando vedevano il pane a fuggire; per lo che divennero cotanto
magri, che a pena si reggevano in piedi. Ritornati i cacciatori per li loro
cani, vedutigli così sfatti si dolevano del Piovano; ma egli preso in mano il solito
bastone, tirò loro in terra alcuni pezzi di pane, ed i cani ricordevoli di come
era solito passare il negozio, in vece d'accostarsi al pane fuggivano, onde il
Pidovano si scusò co i cacciatori dicendo: Come volete che ingrassino, se quando
io do loro il pane, fuggono come vedete? E da questa facezia venne questo
proverbio \textit{dar il pan col bastone}, che significa mostrar di voler far del bene a uno,
e fargli del male. Seneca ci fa veder questo modo di dire anche appresso a i Latini,
raccontando il detto di Fabio per soprannome Verrucoso, che il piacere
fatto da persona zotica, e con maniera salvatica chiamava \textit{Panem lapidosum}, che
è appropriato al nostro detto \textit{Dare il pane, e la sassata}.
\item[BALESTRA] Strumento, o arme da caccia, col quale si scagliano palle di
terra secca, nella guisa che si fa delle frecce; e serve per ammazzare uccelletti.
È composta d'un'arco d'acciaio accomodato in cima a un'asta, o legno torto,
dentro al quale sono adattati altri ordinghi di ferro per facilitare l'operazione.
Viene dall'antica ballista arme guerriera, che dicevano ballista forse dal Greco
\textit{ballein}, che significa scagliare.
\end{description}

\section{Stanza IV.}
\begin{ottave}
\flagverse{4}La plebe, i grandi, ed ogni lor ministro\\
Ch'il Duca così buono havean provato,\\
Mentre fu scudo ad ogni lor sinistro\\
Ed in lor pro sarebbesi sparato,\\
Vedutolo così mutar registro,\\
E diventar un turco rinnegato,\\
Eran talmente d'animo cattivo,\\
Che l'havrebbon voluto ingoiar vivo.
\end{ottave}

Per questa mutazione del Duca di buono in cattivo, li suoi sudditi, che prima
l'amavano, cominciarono a portargli odio, e bramargli ogni male.
\begin{description}
  \item[SI sarebbe sparato in lor pro] Havrebbe fatto loro ogni favore immaginabile.
    Havrebbe messa, e spesa la propria vita a benefizio loro, e la  voce \textit{pro} è un
    sustantivo che significa giovamento, utile, ec. dal latino \textit{prodest}.
\item[MUTAR registro] Mutar maniera di fare. \textit{Registro} diciamo quell'ordine di
  ferri, il quale è negli organi strumenti musicali, con ciascuno de' quali ferri alzandolo,
  o abbassandolo si dà, o leva il fiato a quelle canne, le quali si vuol,
  che suonino o no, ad effetto di far mutar voce all'organo, il che si dice \textit{mutar
    registro}, che passato poi in proverbio significa Mutar maniera, o modo di fare
  in qualsivoglia cosa. Vedi sotto C.\ 8.\ stan.\ 52.\ alla voce protocollo \textit{Registro} in
  altro significato.
\item[INGOIARE] Trangugiare. Mandar giù in corpo una cosa senza anche
  masticarla, che si dice anche ingollare. Vedi sotto C.\ 1.\ stan.~6.
\end{description}

\section{Stanza V.}
\begin{ottave}
\flagverse{5}Avvenne, che già inteso un Negromante \\
C'un'huom com'era quei sì giusto, e magno,\\
Faceva novita sì stravagante, \\
Un'atto volle far da buon compagno;\\
E per ridurlo all'opre buone, e sante\\
Non per speranza di verun guadagno\\
Fintosi un baro, a dargli ando l'assalto,\\
Un po di ben chiedendo per sant'alto.
\end{ottave}

Stando le cose ne i suddetti termini, un tal mago, inteso che un huomo da
bene come era Perione s'era cangiato in così cattivo, volle fare un'atto da huomo
da bene, cercando di rimettere Perione nella buona strada, e però fintosi
un'accattone, andò a chiedergli l'elemosina per amor di Dio.

\begin{description}
  \item[NEGROMANTE] È lo stesso che Mago: Se bene Negromante venendo da
    negromanzia s'intende colui, che \textit{per mortuos vaticinatur}, che è una delle sei specie
    di Magi detti sopra C. 1, stanza 20., tuttavia da noi si piglia per nome generico,
    e per intendere ogni specie di mago, e di magia.
\item[BARO] Biante. Accattone falso. Vien forse dal Greco \textit{Barijs Bareos}, che
  suona molestus, importuno, sfrontato, come appunto sono questi tali; e se bene
  questa parola ha del furbesco pure s'usa comunemente, e l'usò il Varchi St. Fior.
  lib.~11, \textit{Ed in segno, che lo rifiutava, e non gli creduea più, havendolo per baro, e
  giuntatore, arse i suoi libri}.
\item[PER Sant'alto] Cioè per Dio. È parlar furbesco, il quale forse è noto fuori
della nostra Toscana, come inventato da Vagabondi, Monelli e Pianti per non
esser intesi, se non da i lor pari, e poi fattosi familiare a molt' altri, a segno
che ne è fatto, stampato il vocabolario. Si dice anche parlare \textit{in gergo, ed in lingua
furfantina}, come ci mostra il Varchi St. Fior. lib. 15. \textit{Appariscono più lettere scritte
non in cifra, ma in gergo a uso di lingua furfantina molto strano}. Il nostro Poeta si
serve di tal parlare nella persona di questo Biante perché, come ho detto; simili
huomini son soliti parlar in questa forma.
\end{description}

\section{Stanza VI.}
\begin{ottave}
\flagverse{6}Rispose Perione : Fratel mio\\
se tu te lo credessi tu t'inganni,\\
Tu vuoi ch' io doni per l'amor di Dio, \\
Ne sai ch'io piglierei per San Giovanni,\\
Se t'hai bisogno, che posso far'io?,\\
Che son Fraffazio, che rifaccia i danni\\
E che pensi, che qua ci sia la cava?\\
Non e più tempo che Berta filava.
\end{ottave}

Alla richiesta del Mago Perione non si muove a far limolina, anzi dice che
piglierebbe anch' egli qualcosa, e che è passato quel tempo che egli dava via
il suo.

\begin{description}
\item[PIGLIEREI per San Giovauni] S. Gio. Batista è il Santo protettore della nostra
  Città di Firenze, e perciò il giorno della sua festa e grandemente solennizzato, ed
  in quel giorno son sicuri nella Città fino i banditi capitali, sicché gli Sbirri non
  posson pigliar nessuno. Da questo è nato l'equivoco Proverbio; \textit{Pigiterebbe il dì
    di San Giovanni}, o \textit{per San Giovanni}, che vuol dice Piglierebbe anche quel dì,
  nel quale ne meno i birri pigliano, e s'intende piglierebbe, cioè accetterebbe tutto
  quel che gli fusse dato in ogni occasione, ed in ogni tempo. E lo scherzo è nel
  verbo pigliare che vuol dir Far cattura, o Catturare, e vuol dire anche Accettare,
  o ricevere, come s'intende in questo proverbio; che esprime; Lo piglierei, ed
  accetterei sempre, e non darei mai.

\item[CHE son Fraffazio], Raccontano una favola d' una donna non troppo honesta,
  la quale havendo commerzio con un tal' huomo detto Fraffazio, fu con esso
  una volta trovata dal marito; ed essendo ella altrettanto sagace, quanto il marito
  semplice, e di cervello grosso, gli diede facilmente a credere, che colui era
  un' huomo da bene, che andava rifacendo i danni a chiunque occorreva qualche
  disgrazia, e che l'haveva chiamato in casa affinché le ricomprasse una sua conca,
  la quale s'era rotta, e che appunto gli narrava questo suo danno; foggiungendo;
  E come, Marito mio! Non conoscete dunque Fraffazio? Il buoa marito
  se la bevve, e così la donna scampò la furia, E da questa favola, quando si
  dice: \textit{esser Fraffazio}, vuol dir: \textit{Esser colui che spende il suo per sollevar
    l'altrui miserie}, e che \textit{rifà i danni} come dice il nostro poeta.

\item[CHE pensi, che qua ci sia la cava] Pensi che io habbia la cava de' danari, cioè
  la Zecca.  Torna bene a questo detto quel che si trova in Salustio; \textit{Censes me
  vicem aerarij praestare}. Non è pero che cava voglia dire la Zecca, ma si piglia per
  questa nel presente detto (da noi usatissimo in questo proposito) perché si suppone,
  ed è verisimile che la Zecca, come luogo dove si batte la moneta, ne sia
  abondante, come sono abondanti le cave di quelle cose, che da esse estraggonsi.

\item[NON e più ib cempo che berta stava] Non è più il tempo, che le cose andavano
  come si bramava. I tempi son mutati. Pipino Re di Francia per mezzo di suoi
  Ambasciadori sposò Berta dal Gran pié figliuola di Filippo Re d'Ungheria, la
  quale havendo saputo, che questo suo Sposo era brutto, e nano, malvolentieri
  s'accomodava a dare il consenso; ma pure, vinta dalla riverenza dovuta ai padre,
  condescese, Arrivata in Francia, lasciandosi governare dal giovenil sentimento;
  richiefe Elisetta di Maganza sua segretaria (la quale d'Ungheria, dove era
  nata del Conte Guglielmo di Maganza ribello di Francia, se ne veniva con Berta a
  Parigi) che volesse, fingendosi la sua persona, in sua vece sposarsi con Pipino
  il quale, e pera somiglianza, che era fra lor due, e per non haver Pipino mai
  veduta Berta, non l'havrebbe assolutamente riconosciuta, Elisetta da principio
  si mostro renitente; ma persuasa poi da Grifone, e Spinardo di Maganza suoi
  parenti, condescese a i voleri di Berta. E così arrivati a Parigi, Elisetta si sposò
  con Pipino in vece di Berta. La qual Berta in tanto di consiglio di detti due
  Maganzesi s'era ritirata in ludgo vicino a Parigi, con pensiero fermato con
  detti Maganzesi di quindi occultamente partir, e tornarsene alla patria con
  l'aiuto de' medesimi; ma questi la tradirono, perché in vece di servirla alla volta
  della patria sua, l'inviarono ad un bosco, con ordine a quelli, che la conducevano,
  che l'uccidessero: Mu costoro mossi a pietà, in vece d'ucciderla, la
  spogliarono, e legatala ad un'albero la lasciarono in preda alla Fortuna, e tornarono
  a i Maganzesi, dicendo che l'haveano uccisa.  I Maganzesi per occultare
  sì atroce delitto fecero morire tutti quei ficarj, havendo prima anche d'arrivare
  a Parigi fatte ritornare in Ungheria tutte le dame, ed altre persone non
  complici, ne consapevoli di sì grande scelleraggine.

  Berta intanto, che se ne stava così legata dolendosi, e lamentandosi fu sentita
  da un tal Lamberto Cacciatore del Re Pipino; Costui seguitando la voce si condusse
  dove stava Berta legata all'albero, e scioltala, alla propria casa la condusse,
  e la consegnò alla moglie vestendola d'abiti vili, e conformi alla posibilità
  di lui, ed alla povera condizione, della quale Berta disse d'essere. Quivi
  stette Berta circa cinque anni, nel qual tempo guadagnò molti denari di filare,
  ed altri lavori, che insieme con le figliuole di Lamberto faceva. Avvenne un
  giorno, che essendo Pipino a caccia si condusse solo alla Casa di Lamberto, ove
  veduta Berta s'invaghì di lei, e con essa si congiunse sopra ad un suo carro, nel
  qual congiungimento fu generato Carlo, così detto dal medesimo Carlo. In tale
  occasione Berta scoperse a Pipino il tradimento de i Maganzesi narrandoli
  tutto il seguito; perloché Pipino fece abbruciare Elisetta, ed una mano di Maganzesi,
  e rimesse nel trono Berta.
  Da questa favolosa storia nacque il proverbio; \textit{Non è più il tempo che Berta
    filava}, Cioè non è più il tempo che Berta stava nelle selve filandode., e ricamando,
  che significa; \textit{Le cose son mutate}.

  Di questo detto si servì Berta moglie d'Arrigo IV, Imperatore, come si vede
  nello Scardeonio Monumenta Patavina lib. 3. Classe 14. de Berta ex Montagnano,
  le di cui parole son queste. \textit{Memoratur in iisdem Patavinis Annalibus celebris
fama Bertae ex Vico Montagnani, quae quidem fuit ruslicano genere, sed moribus certe
perquam nobilis \& animo perquam generosa},

\textit{Haec enim tempore Henrici IV Imperatoris, cum eius uxor, Berta \& ipfa nuncupata,
  Patavij moraretur, vel eiusdem forte nominis similitudine, vel propria generositate
  animi allecta, obtulit ei dono filum tenuissimum, quod eleganter suamet neverat manu,
  \& in Vrbem venale detulerat. Quod munus Regina hilari vultu accepit; \&
  cum cognovisset nomen, \& animum mulieris, eam indignam censuit, ut vitam inopem
  famineo colo amplius sustineret suam. Dato itaque filo procuratori suo, iubet ad Pagum
  Montagnani statim proficisci, ubi mulier habitabat, \& pro referenda gratia tot terra
  iugera ei ex publico adscribi, quantum spacij filum dono datum extensum comprehendere,
  \& circumdare posset, Quod cum caeterae mulieres vidissent, illico Bertae exemplo
  attulerunt, \& ipsae filum, quod Regina dono darent. At ipsa renuens id ab alijs accipere
  percante respondit}, Pertransiit tempus, dum Berta filabat.

Gli antichi dicevano \textit{Non est amplius aetas Cyclopum}, ed in molte altre maniere, si come
Ancor noi diciamo: \textit{È finita la cuccagna}, o la vignuolaxe. \textit{Non e più tempo di Bartolommeo},
ec. Con i quali, ed altri detti intendiamo Non si godono più quelle felicità che già
si godevano.
\end{description}

\section{Stanza VII.}
\begin{ottave}
\flagverse{7}Signor (soggiunse il Mago) mi sa male \\
Di veder, c' un sì gran limosiniere,\\
Ed huom tanto benigno, e liberale \\
Caduto sia nel mal del miserere.\\
Hor basta; Chi del mio fa capitale\\
(Diss'egli) fa la zuppa nel paniere.\\
Pero va in pace tu co' tuoi bisogni,\\
Perché per me tu mangerai de' sogni.
\end{ottave}

Il negromante vedendosi cacciar via con tal risposta; replicò, che gli dispiaceva,
ch' ei fusse diventato avaro. E Perione li soggiunse, ch'ei non sperasse da
lui fastidio alcuno.
\begin{description}
\item[CADUTO nel mal del miserere] Divenuto misero, cioè avaro, tenace, che se
bene il mal del Miserere è una infermità mortale; Noi ci serviamo della voce
Miserere nella forma che habbiamo detto sopra C. 1. stan. 80. della voce \textit{boccolica},
e per intender \textit{misero}, che nel presente luogo vuol dire avaro; e così è inteso
comunemente, se bene la voce \textit{Misero} propriamente vuol dire infelice.
\item[FAR capitale] Far' assegnamento; o sperare nell'aiuto d'alcuno. Vedi sotto
C. 7. stan. 82. Questa voce capitale è dedotta da \textit{capitatio capitationis}, che era una
tassa, o tributo, che determinavasi \textit{in capita populorum} per assegnamento; e propriamente
capitale del Principe, come è forse la Decima, che pagano hoggi i
nostri contadini, che pure si dice decima in su la testa.
\item[PANIERE] È un vaso intessuto, e composto di fili di vetrice, o d'altra specie
  d'albero,  o di sottilissime strisce di legno in figure, e forme varie, in tutte
  le quali che sieno, ha sempre il manico; che senza manico si chiama corbello, o
  paniera, e servono per portar frutte, o altro che sia; detto paniere, o paniera
  forse dal pane, perché gli antichi tenevano il pane in tal sorte di cesta in mezzo
  alle mense, e perciò da i Latini detto \textit{Panarium}.
\item[FAR la zuppa nel paniere] Questo proverbio dice:
  \begin{verse}
    Chi fa l'altrui mestiere
    Fa la zuppa nel paniere.
  \end{verse}

E così dichiara il suo significato, quale è: Che colui, il quale si mette a fare
una cosa, che non fa fare, non farà nulla di buono; ed in sustanza vuol dire; Affaticasi
in vano. Ovid. lib. 12.
\begin{verse}
  Vique liquor rari sub pondere cribri
  \end{verse}

 Ed è forse meglio dir suppa, che zuppa venendo dal verbo suppurare, che vuol
dire attrarre l'umido; o da Suppen Tedesco. Vedi sotto C. 4. stan. 25. Ma l'uso
ci obliga a dir zuppa.
\item[VA in pace] Così usiamo dire, quando mandiamo via i poveri, che accattano.
  E l'usò in un certo modo Plauto in milit. dicendo \textit{Pax, abi},
\item[MANGERAI de sogni] Mangerai cose immaginarie, cioè non mangerai.
  Mattio Franzesi\footnote{Mattio (Matteo) Franzesi, San Gimignano 15.. - 1555, poeta burlesco.} nel Capitolo della povertà dice.
  \begin{verse}
    Che sfacciata talor non si vergogni,
    E che spesso permetta, e faccia male,
    Si scusa, che non può viver di sogni.
  \end{verse}
  I Latini pure havevan simil modo di dire, come si vede in Giuvenale Sat. 6.
  \begin{verse}
    Qualiacumque voles Iudaei somnia vendunt.
  \end{verse}
E coloro, che hanno una vogllia ardentissima d'una cosa, sogliono sognarla; perché
altro non è il sogno, che
\begin{verse}
  Un'immagen del dì guasta, e corrotta
  \end{verse}
La onde Teocrito Eglog. 9. introduce un Pastore, che raccontando le sue felicità così ragiona:
\begin{verse}
Possideo quaecumque solent in nocte videri
In somnis, vim magnam ovium multasque capellas.
\end{verse}

Et anco notò Nonio, che appresso gli antichi Romani, il verbo vescor significava
vedere: \textit{Prius quam infans esses, tui oculi facinus vescuntur} cioè \textit{vident}; come noi
pure diciamo; \textit{Mangiar un con gli occhi}, quando altri guarda uno con grande attenzione;
e diciamo anche: \textit{Dar pasto agli occhi}. Dan. Par. Ci 27,08
\begin{verse}
\backspace E sa natura, ed arte le pasture
Da pigliar occhi \makebox[5em]{\dotfill}
\end{verse}

Sì che dicendo mangerai de sogni, si può anche intendere, \textit{Ti sazierai, o soddisfarai
con dar pasto a gli occhi}; o \textit{della vista}; che è lo stesso che non mangerai. Vedi
sotto C. 6. stan. 55. che dice \textit{pascer la vista}.
\end{description}
\section{Stanza VIII — X.}
\begin{ottave}
\flagverse{8}Come (replicò quei) se è si cicala,\\
Che tu daresti via fin la gonnella,\\
Vedendomi spedato, e per la mala\\
Potrai haver' il granchio alla scarsella?\\
Poi che tu gratti il corpo alla cicala\\
(Disse il Duca) io levsi questa cannella\\
Per quel ch'io ti dirò, perché se già\\
Donai, non era tutta carità.
\end{ottave}

\begin{ottave}
\flagverse{9}E non batteva la mia fine altrove,\\
C'ad haver prima ch'io serrassi gli occhi\\
In ricompensa un dì, piacendo a Giove,\\
Della mia donna quattr'o sei marmocchi,\\
Ma finalmene dopo mille prove\\
Di dar' il lustro a marmi coi ginocchi,\\
Tenendo gli occhi in molle, e il collo a vite,\\
E le nocca col petto sempre in lite;
\end{ottave}

\begin{ottave}
\flagverse{10}Io l'hebbi bianca a femmine, ed a maschi,\\
Ond'io sbraciar volendo a bel diletto,\\
Mi risolvei levar quel vin da fiaschi,\\
E non dar più quant'un puntal d'aghetto,\\
Perché po poi (diss'io) gli è me' ch'io caschi\\
Dalle finestre prima, che dal tetto;\\
E il cavarmi di mano adesso un pelo,\\
Sarebbe un voler dare un pugno in Cielo,
\end{ottave}

Il Mago mostra di non poter credere, che havendo Perione nome di liberalissimo,
non s'habbia a muover' a compassione di lui, e Perione vinto dall'importunità
di costui, gli dice, che fu già liberale per disporre il Cielo a concedergli
figliuoli; ma perché egli non era stato esaudito, lasciò di far più limosine, ed
hora era impossibile cavargli di mano un picciolo.

\begin{description}
\item[SÌ cicala] Cioè si dice; Si discorre. Il verbo cicalare usato in questi termini
esprime discorso di cosa incerta, che si dice anco \textit{bucinare}, o \textit{buzicare}, E si dice:
la tal cosa non fu poi vera; ma fu una cicalata, cioè se ne parlò; ma non è poi stata
vera.
\item[DARESTI via fin la gonnella]  Daresti via fino al proprio vestito; daresti via
tutto il tuo havere. E se bene \textit{gonnella} s'intende una specie d'abito da donna, in
questo proverbio diventa nome generico per ogni sorte d'abito.
\item[SPEDATO] Cioè co' piedi laceri dal viaggio.
\item[PER la mala] Cioè per la mala via, e s'intende mal condotto di sanità, e mal'all'ordine
  di vestito, e senza danari.
\item[HAVER il granchio alla scarsella] Chiamiamo \textit{Granchio}, o \textit{grancia} una specie
  di malattia di spasimo, la quale quando viene alle mani impedisce il maneggiare le
  dita; E da questa quando diciamo \textit{Il tale ha il granchio alla scarsella} intendiamo
  non può adoperare le mani intorno alla borsa, che vuol dire; è pigro a cavar denari
  della borsa, cioè, a dire: è tenace, o avaro, ed uno, de' quali parlando
  Marziale dice.
  \begin{verse}
    \backspace Litigat, \& podagra Diodorus, Flave, laborant;
    Sed nil Patrono porrigit; haec Chiragra est.
  \end{verse}

  E noi pure diciamo di questi tali; \textit{Haver la gotta alle mani}, \textit{Haver i pedignoni
    alle mani}; \textit{Haver le mani aggranchiate}; \textit{farebbe a pagar co' monchi},
\item[SCARSELLA] Intendiamo ogni sorte di tasca, o borsa di danari, come si
vede sotte C. 3. stan. 5., se bene scarsella è propriamente una borsetta di quoio
Con serrature di ferro fatta alla foggia delle Carniere da cacciatori; la qual sorte
di di borsa usava già in Firenze portarsi da tutti legata a cintola.
\item[GRATTAR il corpo alla cicala] Incitar' uno a discorrere. Vedi sopra Cant.
primo stan. 2. I Latini pure dissero in questo proposito \textit{Cicadam ala comprehendere}.
\item[LEVAR la cannella] Desistere di fare una tal cosa. Traslato dalla botte, alla
  quale si leva la cannella, quando è finito il vino, che era in essa. E cannella intendiamo
  quel legnetto tondo forato per lungo, che si adatta al fondo della botte
  per cavarne il vino, la quale da i Latini con voce Greca si dice \textit{epistomium}. Si dice
  anche in questo proposito.
\item[LEVAR il vino da fiaschi] come vedremo appresso.
\item[PRIMA ch'io serrassi gli occhi] Prima ché io morissi.
\item[MARMOCCHI] Ragazzi. Queita voce marmocchio in significato di fanciullo,
viene da marmo, alla pulitezza, e liscio del quale s'assomiglia il liscio, e pulitezza
del volto de i fanciulli, e delle fanciullette. Or. Od. 19. lib. 1.
\begin{verse}
 Urit me Glycerae nitor
 Splendentis Pario marmore purius.
\end{verse}
\item[DAR il lustro a' marmi co' i ginocchi] Cioè stava tanto tempo, e così speffo in
  ginocchioni, che il lungo fregare con le ginocchia faceva divenir lucenti i marmi,
  sopra i quali s'inginocchiava.
\item[TENENDO gli occhi in molle] Cioè lagrimando, e così tenendo gli occhi in
  molle nelle lagrime.
\item[COLLO a vite] Collo torto, come fanno i Bacchettoni. Si dice a vite per similitudine,
  essendo \textit{la vite} uno strumento; il quale serve per serrar un materiale
  con l'altro, che per essere attorcigliato come \textit{la vite} pianta, che produce l'uva,
  da essa piglia il nome, e si dice anche \textit{torchio}, e \textit{chiocciola}: quello dal torcere, col
  quale  fa la sua operazione; e questa per la similitudine, che ha la sua figura con
 il guscio della chiocciola.
\item[E LE nocca col petto sempre in lite] Cioè dandosi delle pugna nel petto; il
  che mostra che le \textit{nocca} sieno in lite col petto, mentre non cessano di perquoterlo.
  E nocca intendiamo nodelli delle dita. Vedi sotto C. 3. stan. 8., e C. 9. stan. 54.
  In somma il Poeta con queste quattro maniere di dire, cioè \textit{Dar' il lustro a' marmi
  co' ginocchi}; \textit{Tenere gli occhi in molle}, \textit{Haver il colle a vite}; e \textit{le nocca sempre in lite
  col petto}, Intende, che \textit{costui stava sempre orando}; e descrive assai bene
un' Hipocrito, o devoto in apparenza, e falso.
\item[IO l'hebbi bianca] Quando un premio s'ha da conseguire per via d'estrazione
  di polizze (come si fa al lotto) sono scritte solamente le polizze premiate, e l'altre
  son bianche; e chi ha una polizza bianca, non conseguisce premio alcuno. E
  di qui viene il detto \textit{Io l'ho havuta bianca}, che è fatto comune, e per intender di
  tutte quelle cose, che si tenta di conseguire, e non si conseguiscono.
\item[SBRACIARE] Vuol propriamente dire, allargare, e sollevare la brace a fine,
che meglio s'accenda, e renda più calore; ma per metafora intendiamo spender
prodigamente, e largamente, come s'intende nel presente luogo, e sotto Cant, 3. stan. 2.
\item[A bel diletto] A posta; o per gusto, ma senza buon fine, e utile, e si dice anche
  a \textit{bello studio}, a \textit{bella posta}, a \textit{bella prova}, che tutti si posson pigliare in questo
  senso. Se bene alcune volte significano quel che i latini dissero \textit{dedita opera} e
  massime quando non v'è l'aggiunta di \textit{bella}, che in questo calo e detto ironicamente,
  ed ha forza d'esprimere \textit{biasimevole}, come per esempio \textit{Veramente tu hai
  fatta una bella cosa}, cioè tu hai fatto una cosa biasimevole, e che sta male. Virg.
  \textit{Egregiam vero laudem, \& spolia ampla reportas}.

\item[NON darei quanto un puntal d'aghetto] L'aghetto è una cordicella fatta di seta,
  o d'altro, che serve per affibbiar le vesti, e adattarle alla persona, alla qual cordicella
  è solito fare una punta di sottil lamina d'ottone, o d'altro metallo, e
  queste punte si dicono \textit{puntali}, e di queste punte se n' hanno due, o tre per un
  quattrino; e da questa viltà serve il presente detto per esprimere; \textit{Non darei niente},
  ne meno una cosa, che non val nulla. Che i latini dissero fra l'altre molte,
  \textit{Vitiosam nucem non dederim}. E noi pure diciamo un fico secco, un lupino, e simili.
  Vedi sotto C, 3. stan. 8.
\item[LEVAR il vin da fiaschi] Il senso metaforico è lo stesso, che levar la cannella
  detto poco sopra stan. 8.
\item[PO poi] Alla fine. All'ultimo de gli ultimi. Opera anco in questo detto la
  forza della replica, che induce superlativo, Vedi sotto in questo C. stan. 73.
\item[GL'è me ch'io caschi dalle finestre prima che dal tetto] Nel male è il meglio, l'eleggere
  il meno. Intende; egli è meglio, che io lasci stare di dare il mio che seguitare,
  e darlo via tutto, cioè mi contenti di questo danno, e non lo faccia
  maggiore col seguitare a profondere il mio. E quel me per meglio è la figura
  Apocope da noi spesso usata; e l'uso Dante più volte, ma notabilmente nel C.
  32, dell'Inferno, che l'usò nel principio del periodo.
  \begin{verse}
    Me foste state qui pecore, o zebe.
  \end{verse}
  Ma di questa figura Apocope, e come l'usiamo, vedi sotto in questo C. stan. 36.
\item[CAVARMI di mano un pelo] Conseguir da me cosa alcuna, ancor che di niun valore.
\item[SAREBBE un voler dare un pugno in Cielo] Sarebbe un voler tentar, una cosa
  impossibile, \textit{Facilius Caelum digito attingeres}.
\end{description}

\section{Stanza XI — XIII}
\begin{ottave}
\flagverse{11}Che pagheresti (disse lo stregone)\\
Se la tua moglie havesse il ventre pregno?\\
Se cio fusse (rispese Perione)\\
Ancor ch'io non ne faccia alcun disegno,\\
E tal voglia appiccata habbia all'arpione\\
Io ti vorrei donar mezz il mio regno\\
Sogginnse quei: Non vo pur'una crazia,\\
Ma solamente la tua buona grazia.
\end{ottave}

\begin{ottave}
\flagverse{12}Altro da te non aspettar ch'io chieda,\\
Ne c'alcuno interesse mi predomini,\\
Perché, quantunque abietto altri mi veda,\\
Io ho in c\ellipsis{18pt} la roba, e schiavo son de gli uomini\\
Hor basta se tu brami d'aver reda,\\
ch'il regno dopo te governi, e domini,\\
Commetti al Mosca, al Biondo, e a Romolino,\\
C'un cuor ti portin d'asino marino.
\end{ottave}

\begin{ottave}
\flagverse{13}Ed ordina di poi, che se ne quoca\\
La terza parte in circa arrosto, o lessa,\\
(Ch'in tutti modi è buona) e dann'un poca \\
In quel modo a mangiar alla Duchessa; \\
Presa che l'ha, gli è fatto il becca all'oca,\\
Che subito ch'in corpo se l'è messa,\\
Senza che tu più altro le apparecchi,\\
Dottela pregna infin sopr'algli orecchi.
\end{ottave}

Il mago s'esibisce a dare a Perione il modo, che la sua moglie impregni;
Perione gli dice che se ciò segue li vuol donar mezzo il suo regno; ed il mago
ricusando il tutto, da a Perione la ricetta dell'Asino marino per impregnar la
moglie.

\begin{description}
\item[CHE pagheresti?] Quando veggiamo uno, che sommamente brama di sapere, o
  d'ottenere una cosa, per mostrare, che è in nostra potestà l'adempire il suo desiderio
  sogliamo dire: \textit{Che pagheresti? Che spenderesti? Quanto daresti?} o simili,
  \textit{se io ti dessi, o dicessi la tal cosa?}
\item[STREGONE] Maliardo, Mago, Negromante, ec, Viene dal latino, secondo
  che osservò il Mureto\footnote{Marc-Antoine Muret (Muret, 12 aprile 1526 – Roma, 4 giugno 1585) filologo e umanista. Dovette fuggire la Francia per eresia e sodomia. Mureto è il nome con cui è conosciuto in Italia, dove poté stabilirsi protetto dalla Chiesa.  } nelle sue varie lezioni lib. 12, c.19. emendando un luogo
  di Plauto nelle Bacchidi. \textit{Longum est Strigonem maleficum exornarier. Strigas}
  (dice egli) \textit{vocabant mulieres, quas etiam noctu volare arbitrabantur, eodemque modo
    strigones homines maleficos, quorum vocabulorum vulgus in Italia utitur}, Vedi sotto C.
3. stan. 69.
\item[IO non ne fo più disegno.] Io non ho più la speranza d'ottenere questa cosa. N'ho
affatto levato l'animo, o il pensiero.
\item[APPICCARE la voglia all'arpione] Haver lasciata la voglia, o il desiderio
  d'una tal cosa. È lo stesso che \textit{Appiccar al chiodo} visto sopra C. 1. stan. 8. E questo
  modo di dire forse procede da i voti, che anticamente facevanoi i Gentili,
sospendendogli nel Tempio, i quali non si potevano levare, di dove eran posti,
ne convertirgli in uso comune, o profano.
\item[ARPIONE] È una specie di chiodo uncinato per uso di regger l'imposte delle
parte, e finestre, girando, quelle sopra di essi. Da i Latini detti \textit{Cardines}.
\item[NON vo pur' una crazia] Non voglio danari. \textit{Crazia} è delle più vili monete d'argento
che habbiamo,  essendo, l'ottava parte del giulio.
\item[HO in c\ellipsis{18pt}] Detto usatissimo, e massime dalla gente vile per esprimere: non stimo,
  non apprezzo questa tal cosa.
\item[SCHIAVO SON de gli huomini] Son servitore a gli huomini virtuosi, e di garbo.
  Quando noi diciamo Il tale è un' huomo (Seguitando il detto di Diogene
  \textit{hominem quaero}) intendiamo huomo dotto, virtuoso, e di tutta perfezione.
\item[HOR basta] Questo termine (del quale l'Autore si serve anche nell'ottava, 7. antecedente)
  è usatissimo per denotare la terminazione d'un discorso, e passaggio ad
un'altro conclusivo del primo, quasi dica: \textit{E a bastanza quanto habbiamo detto per
conchiudere il come, o il quando, o il se si deva fare, o non fare la tal cosa}.
\item[REDA] Cioè successione, heredi, e s'intende figliuoli. \textit{Il tale ha havuto reda},
\textit{il tale ha havuto un figliuolo}. E buona parola Fiorentina, ma hoggi poco usata, e
solamente per i contadi; dove per \textit{reda} intendono anche i figliouli delle bestie.

\item[MOSCA, Biondo, e Ramolino] Tre venditori di pesce, che vivevano al tempo.
che l'Autore compose quest'Opera.
\item[GLI è fatto il becco all'Oca] Il negozio è conchiuso, che i Latini dissero: \textit{Iacta
  est alea}. Il Lalli nella sua En. Tr. C. 3 stan. 64. disse:
\begin{verse}
Ne vanno tuiti: il marcio hora si giuoca,
Non v'è rimedio. È fatto il becco all'oca.
\end{verse}

Dice Francesco Cieco da Ferrara nel suo Poema intitolato il Mambriano (Opera
nota per esser l'origine, ed antefatto dell'Orlando innamorato, Poema del
Boiardo, ed in conseguenza dell'Orlando furioso di Lodovico Ariosto) al Canto
secondo, che
\begin{adjustwidth}{0.5em}{}
  Fu già nel Regno di Cipri un Re chiamato Licanoro il quale havea una sola
figuola nominata Alcenia, la quale amando egli al pari di se stesso, volle
sapere, se buona, o ria fortuna ella fusse per havere; fatti però chiamare alcuni
Astrologi fece fare la natività alla medefima sua figliuola, e tutti concordarono,
che ella farebbe prima stata madre, che moglie; Onde il Re per evitare
il presagito vitupero, fece fabbricare un giardino contiguo al suo palazo reale,
e dentro al detto giardino edificò una fortissima, ed altissima Torre con
molte stanze, e con tutte le comodità, ma senza finestra alcuna, che riuscisse
fuori della Torre: Dentro a questa messe la figlia con alcune Matrone, e
Damigelle, assicurandosi dell'ingresso della medesima non solamente col tenerne
egli proprio le chiavi della porta, ma con haver deputate accuratissime,
e raddoppiate guardie di soldati intorno, ed alla porta della torre, ed alle
mura del giardino; ne altri entrava nella torre, che una sola donna, della
quale il Re si fidava, e le dava la chiave ogni volta, che a lei occorreva andare
alla Torre con provvisioni di vitto, o d'altro.

In questo tempo morì un tal Co. Gio: di Famagusta huomo ricchissimo, ed
alquanto parente del Re, e lasciò erede delle sue immense facultà Cassandro
unico suo figliuolo; Questo giovane fece fabbricar un palazo sontuosissimo, in
cui teneva corte bandita con tanta splendidezza, che fino al medesimo Re venne
voglia d'andarvi, e lo messe ad effetto. Andatovi dunque fu dal giovane
invitato a cena, ed il Re accetto l'invito, credendo fargli conoscer, che non
era in grado di banchettare decentemente un Re all'improvviso. Ma tutto il
contrario avvenne, perché il Re fu così ben servito, e di vivande, e di musiche,
e d'ogni altra cosa conveniente ad un banchetto regio, che gli parve
che Cassandro havesse maggior possanza, che non haveva egli; onde cominciò
ad havergli invidia, ed a pensare come potesse mortificarlo; Havendo
però veduto sopra ad una maravigliosa fonte, che era nel giardino, un motto
Che diceva \textit{Omnia per pecuniam facta sunt}. Si voltò a Cassandro, e disse: Quel
motto è troppo presuntuoso, essendoci molte cose, che non si posson fare col
danaro. Al che rispose Cassandro: Sire, io ho posto quivi quel motto, perché
mi son sempre creduto, che il denaro apra la strada anche all'impossibile,
e fino a hora mi è riuscito, come appunto mi son figurato, Horsù (replicò il Re)
Già che ti da il cuore di poter fare ogni cosa col denaro, io ti do tempo un'anno
a procurare per le strade, che vorrai, di godere la mia figluola, che io
tengo nella torre guardata, come tu sai, e de dentro a questo tempo ti verrà fatto,
sarà tua moglie; quando no, la tua testa pagherà la pena. E questo fece
il Re, perché essendo entrato in sospetto della potenza di Cassandro, voleva
sotto qualche pretesto levarselo d'avanti.

Il povero Cassandro rimasto sbalordito da tal proposta, meditava di pigliarsi
bando dalla patria, quando Euripide sua Balia, saputa la cagione del suo disgusto
gli disse, che si consolasse, perché ella haveva un un suo nipote dotato di
così grande ingegno, che assolutamente gli havrebbe aperta la strada all'ingresso
nella Torre.

Questo nipote della Balia Euripide fabbricò un'Oca di legname, grande
tanto, che potesse agiatamente ascondersele in corpo un'huomo, che v'entrava,
e usciva per di sotto l'ali, e per via di certi ordinghi faceva fare a tal'Oca
tutte l'operazioni, e moti, come se fusse stata viva, ed era del tutto perfetta
se non che le mancava il becco. Cassandra fece sparger voce, che era
andato in lontani paesi; ed intanto havendo fatta portare occultamente la
detta Oca in un luogo remoto, entrò nella medesima, ed Euripide sua Balia in
abito moresco la guidava, fingendo di venir dal Cairo (dove'era veramente
nata, ed allevata detta Euripide) e parlando in quella lingua ben' intesa da
Cassandro, toccava con una bacchetta l'Oca, ed era il concerto, che Cassandro
per via di certe Zampogne facesse cantar l'oca. L'astuta Balia, accennate
a pena l'operazioni dell'Oca, andava dicendo, che a volerla vedere
operar cose galanti, e maravigliose, bisognava spendere; e però il popolo,
messa insieme buona somma di monete, la diede alla Balia, la quale fece fare
all'Oca diverse belle operazioni.

Arrivò la fama di quest'Oca all'orecchio del Re, e della Regina, onde
fattala venire a se, dopo haverla veduta operare, regalata Euripide, la mandarono
ad Alcenia loro figliuola per farle pigliar qualche spasso, e divertimento
ne i giuochi dell'Oca; la quale condotta nella Torre, il negozio andò in
maniera, che per via de trattati della Balia, Cassandro nello stare in camera
d'Alcona ascoso in quell'Oca, si godé Alcenia, e si diedero la fede di sposi.
Fatto questo, Cassandro accomodò all'Oca il becco; e con la Balia ascosto nell'Oca
se n'usci della torre, e presentatasi la Balia con l'Oca d'avanti al Re, ed
alla Regina per domandar licenza; i Re disse: Quest'Oca ha il becco, e prima
non l'havea? E la Balia rispose: Non se le era messo, perché non era
ancor fatto: e Vostra Maestà tenga a memoria quel che ora ha detto.

Fra pochi giorni spirò il termine, dentro al quale Cassandro doveva haver
goduta Alcenia, onde il Re se lo fece condurre avanti, e Cassandro disse; Sire
V.M. faccia venire Euripide mia Balia. Il Re lo compiacque, e comparsa
Euripide con l'Oca, fu dal Re subito riconosciuta, ed ella gli disse: V.M. si
ricordi \textit{che è fatto il becco all'Oca}; e fatta quivi condurre l'Oca fece entrarvi
dentro Cassandro, e lo fece fare le solite operazioni, acciò che il Re
conoscesse che quella era la stessa Oca, che in quella stessa maniera era dimorata
più giormi con Alcenia nella Torre: onde il Re conosciuta l'astuzia di
Cassandro, e saputo più precisamente il fatto, e che Alcenia era gravida, ed
havea data la fede di sposa a Cassandro, confermò il matrimonio per osservar
la parola, contentandosi di cedere alla disposizione del fato;
\end{adjustwidth}

E da questa travestita
trasformazione di Giove in Cigno è nato il proverbio: \textit{È fatto il becco
  all'Oca}; che significa (come habbiamo detto) il negozio è fatto, o perfezionato.
Questa, o simile novella leggesi in quelle di Giovanni detto il Pecorone.

\end{description}

\section{Stanza XIV \& XV.}
\begin{ottave}
\flagverse{14}O questa (disse il Duca) è veramente\\
Da pigliar con le molle; Ch'un samaro\\
Possa col cuore ingravidar la gente;\\
Vedi non ti son finto, io non la paro.\\
Hor su il provar non ha a costar niente,\\
E quando mi costasse anco ben caro,\\
Vo farlo, per veder, se ciò riesce;\\
Però si mandi al mar per queste pesce.
\end{ottave}

\begin{ottave}
\flagverse{15}Benche fusse costui com' una pina\\
Tanto largo, ignorante, e discortese;\\
Per non balzar un tratto alla berlina,\\
I pescatori vennero in paese:\\
Così pescando lungo la marina,\\
Questo benedett' asino si prese,\\
E il cuor n'un bel bacino inargentato\\
A suon di pive al 'Duca fu portato.
\end{ottave}

Il Duca sentendo che il cuor d'un' Asino marino era atto a ingravidar la
moglie, si ride del mago; ma tuttavia era così grande il desiderio d'haver figliuoli,
che volle provare, e comandò che i pescatori vedessero di trovarlo, ed essi finalmente
lo presero, e portarono il cuore al Duca.

\begin{description}
\item[È DA pigliar con le molle] È una grossa minchioneria, è uno sproposito grandissimo.
  \textit{Molle} intendiamo quello strumento di ferro, che serve per pigliar carboni
  ardenti, ec.
\item[VEDI] Questo termine ha del giuratorio, quasi dica: \textit{in fede mia}, ec, \textit{io non
lo credo}, \textit{Credi a me che tu fai male}, ec, Vedi sotto: C. 8. stan. 63.
\item[NON la paro] Non la credo. Tratto dalla Riffa, o Massa giuoco di dadi, nel
quale quando uno tien la posta dice; \textit{Paroli},  e non la tenendo dice \textit{Non la paro}.
\item[LARGO come una pina] Si dice \textit{largo com' una pina verde}, la quale strettissima,
e ben serrata; Comparazione ironica, perché huomo \textit{largo} vuol dir liberale, ed
huomo \textit{stretto} vuol dire avaro, e tenace; Sì che sendo la pina verde strettissima,
comparandosi un huomo a questa; s'intende trettissimo, cioè tenacissimo, avarissimo,
che i Latini dissero \textit{Laro sacrificat}; che suona, Gli è divoto della folaga,
la quale perché è di natura vorace, serviva a i latini per esprimere un huomo
avido del denaro, e lo dicevano \textit{Larus hians}.
\item[IGNORANTE] Uno che non sa. Vedi sopra C. 1. stan. 73. Ma vale ancora
  per \textit{ingrato, zotico, villano, e poco amorevole}, ed in questo luogo è preso in tal senso
  nel quale è sempre, o per lo più preso nel contado.
\item[PER non balzare] Cioè per non andare. Si costuma dire balzare per andare,
  o cadere in cose di disgusto, come \textit{balzar infermo in un letto}, \textit{balzare in una
    prigione}, ec. Non si direbbe \textit{balzare a un banchetto} e simili. \textit{Per non balzare in una prigion, quanti noi siamo, sarà necessario che altri di noi balzino in campagna, ed altri si
salvino in Chiesa}, Disse l'Autore, che scrisse la vita di quei tre famosi ladri Fiorentini.
\item[BERLINA] È una specie di tormento, o gaftigo, che fi dà a i ladroncelli
mettendo loro al collo un' anello di ferro incatenato a una colonna, o a un muro
in luoghi pubblici, e più frequentati della città, e quivi si lasciano esposti
all'insolenza della plebe. Quel strumento si chiama ancora Gogna. Vedi sotto C. 3.
stan. 62. e C. 6 stan. 50.
\item[VENNERO in paese] Cioè comparvero, si lasciaron trovare. Esprime un ritrovamento
  di cose ascoste; Ed è lo stesso che \textit{venire in scena} detto sopra nel Cant. 1. stan. 2.

\item[QUESTO benedetto Asino] L'epiteto \textit{benedetto} in tali occasioni vuol dire tanto
  bramato. Io cerco del tale, del quale ha grandissimo bisogno, e questo benedetto
  huomo non si trova.
\item[BACINO] o bacile. È  un piatto d'argento, o d'altro metallo grande più
della solita misura de i piatti da tavola, e serve propriamente per ricever l'acqua,
che si dà alle mani alle tavole de' grandi, se ben s'adopra anche in molte altre occasioni,
e per altri effetti.
\item[PIVA] Dicemmo, che cosa sia sopra C. 1, stan.\ 34. alla voce \textit{cornamusa}. I
  contadini sogliono per il maggio andare attorno cantando, e suonando la Cornamusa,
  ad effetto di ragunar denari per far con essi regalo a qualche luogo pio,
  e ricevono l'elemosine, che vengono lor fatte in un bacino, ed in un'altro portano
  quel tal regalo, che voglion fare, o vero l'appendono ad un ramo d'alloro,
  o altro albero, e dicono questa lor gita, \textit{andare a cantar maggio}. Tal costume
  tocca il nostro Autore con questo modo di portare \textit{il cuore dell'Asino marino} al
  Duca.
\end{description}

\section{Stanza XVI — XVIII}

\begin{ottave}
  \flagverse{16}Ed egli preso il prelibato Cuore,\\
Lo diede al Cuoco, al qual mentre lo cosse,\\
Si fece una trippaccia la maggiore,\\
C'a i dì de' nati mai veduta fosse,\\
Le robe, e masserizie a quell'odore\\
Anch'elle diventaron tutte grosse,\\
E in poco tempo a un'otta tutte quante\\
Fecer d'accordo il pargoletto infante.
\end{ottave}

\begin{ottave}
\flagverse{17}Allor vedesti partorire il letto\\
Un tenero, e vezzoso lettuccino,\\
Di qua l'armadio fece uno stipetto,\\
La seggiola di là un seggiolino,\\
La tavola figliò un bel buffetto,\\
La cassa un vago, e piccol cassettino,\\
E il destro canteretto mandò fuore,\\
C'una bocchina havea tutta sapore.
\end{ottave}

\begin{ottave}
\flagverse{18}Il Cuoco anch'egli poi non fu minchione,\\
Perché bucar sentitosi n'un fianco,\\
Si vedde prima uscirne uno stidione;\\
Dipoi un Guatterino in grembiul bianco,\\
Ch'in far vivande saporite, e buone,\\
Fu subito squisito, e molto franco,\\
E in quel ch'il padre stette sopr'a parto,\\
Cucinò in Corte, a lui, e al terzo, e al quarto.
\end{ottave}

Il Duca dette il Cuore al Cuoco, il quale nel cucinarlo ingravidò, sì come ancora
tutti gli arnesi, e masserizie, che ne sentirono l'odore, ed a una medesima
hora partorirono.

Qui vorrei, che il lettore si ricordasse che il Poeta, nel comporre quest'Opera
ha havuto per fine il mettere in verso quelle novelle, che dalle Donne son raccontate
ai Fanciulli (come habbiamo detto) e che però sta dentro a' termini di
quelle favole, le quali come per lo più inventate, e composte da quelle medesime
donnicciuole, non possono superare la capacità di queste, ne di quelli, e si
contentasse di non prender ammirazione nel sentir da lui una cosa tanto favolosa,
e fuori del naturale, come è il far partorire le masserizie, e d'osservare, che
ancora Gio. Batista Basile, che pur fu homo dotto, nel suo Cunto de li Cunti
ha descritto questa, ed altre novelle simili, a solo oggetto di trattenere li piccirilli,
come egli dice.

\begin{description}
\item[PRELIBATO] Vuol dire una cosa gustosa, o singolare, ma significa ancora
  leggiermente narrata, o detta avanti, come è nel presente luogo, che significa
  il suddetto, o accennato cuore; ed habbiamo anche il verbo \textit{prelibare} Dan.
 Purg. Cant, 10.
\begin{verse}
Hor ti rimanclettor soprail tua banco
Dietro pensando a cio, che si preliba.
\end{verse}
\item[A dì de nati] Non nacque mai veruno, che vedesse un ventre maggior di quello,
  che haveva il cuoco. E un termine, che amplifica la voce \textit{mai}; V.g. Nessuno
  di quelli, che sono stati al mondo, mai vedde, ec. \textit{Post bominum memoriam}.

\item[A un'otta] A uno stesso tempo; a una medesima hora. Usandosi da noi spesso
la voce otta in vece d'hora: \textit{allotta} in vece \textit{d'allora}, \textit{Che otta è egli?} ia vece di che
hora e egli?
\item[FECER d'accordo il pargoletto infante] S'accordarono a partorire a un'hora
medesima.
\item[LETTUCCCINO] Intende piccolo lettuccio, Ma lettuccio intendiamo una gran
cassa, la quale per di dietro ha una spalliera, e dalle testate i bracciuoli, sopr'alla
quale è solito tenersi uno strapunto, e serve per riposo, e per dormirvi sopra
dopo desinare.
\item[ARMADIO ec] Arnese di legno per riporvi ogni sorte di roba, il quale per lo
  più si tiene affisso, o accosto al muro, e si apre come le porte, ed ha dentro diversi
  palchetti, o cassette; e per stipetto qui intende piccolo armadio.
\item[BUFFETTO] Intende piccola tavola.
\item[DESTRO] Quello che diciamo anco luogo Comune, ed è quello, dove si va
a scaricare il ventre.
\item[CANTERETTO] Piccolo Cantero, e questo è un vaso di terra, o di rame
o d'altra materia, il quale si mette dentro alle predelle per recipiente all'uso
suddetto, chiamato così per esser per lo più di figura: simile a quel bicchiere che
i Latini chiamavano Cantharas.
\item[UNA bocchina havea tutta sapore] Il Poeta scherza, sapendosi bene, che simil
  sorte d'arnesi suol' esser sempre fetida, e però dice \textit{che era tutto sapore}, cioè sapeva
  di qualcosa.

\item[MINCHIONE] Vuol dir semplice, corrivo: Ma qui vuol dire uno, che non
fa meno di quello, che fanno gli altri v.g. \textit{Se tu pigli della tal cosa, non voglio esser
Minchione, ne voglio pigliar' anch' io}.
\item[SCHIDIONE] o stidione, E questo ultimo è più comune, Vuol dire quello
strumento da cucina, nel quale s'infilza la Carne, o Uccelli, per quocerli arrosto,
\item[GVATTERINO] Diminutivo di Guattero, che è colui, che serve d'aiuto al
cuoco. Qui intende piccolo cuoco.
\item[GREMBIVLE] È un panno, col quale si cinge la persona sotto lo stomaco
per difendere il vestito da' gli untumi; detto così \textit{quia regit gremium}, ed in altri
luoghi d'Italia \textit{Senale} quia \textit{sinum regit}, e molti \textit{Zinale} da \textit{Zinne}.

\item[MOLTO franco] La voce franco, che vuol dir libero, ci serve ancora per
  esprimere un'huomo ardito, coraggioso, pratico, o disinvolto, come intende
nel presente luogo.
\item[SOPRA parto] Quel tempo, che le donne stanno nel letto dopo haver parto
  rito, per riaversi da gli sconcerti cagionati loro dal parto, diciamo: Star sopr'a parto.
\end{description}

\section{Stanza XIX. \& XX.}

\begin{ottave}
\flagverse{19}La Duchessa ch' il cuore havea inghiottito,\\
Cotto ch'ei fu con ogni circostanza,\\
Anch'ella con gran gusto del marito\\
Stampò due Bamboccioni d'importanza;\\
Grazie, e belleze haveano in infinito,\\
E così grande, e tanta somiglianza,\\
Tant' eran fatti uguali, ed a capello,\\
Che non si distinguea questo da quello.
\end{ottave}

\begin{ottave}
\flagverse{20}Crebbero insieme, ed all'adolescenza\\
Pervenuti mangiaro il pane affatto;\\
Nel far santà, nel far la riverenza,\\
Hebbero il corpo a meraviglia adatto:\\
Tra lor non fu mai lite, o differenza,\\
Ma d'accordo voleansi un ben matto;\\
L'Infante Floriano uno hebbe nome,\\
E quell' altro Amadigi di Belpome.
\end{ottave}

La Duchessa pure partorì due bellissimi figliuoli, tanto simili di fatteze, che
non si distinguevano l'uno dall'altro. Questi crebbero, e furono allevati con
buona creanza, e fra di loro cordialmente s'amarono. Uno di essi hebbe nome
l'Infante Floriano, che vuol dire Raffaello Fantoni, e l'altro Amadigi di Belpome;
E questo è nome a caso.
\begin{description}
\item[STAMPO' due bamboccioni d'importanza] Partorì due bellissimi figliuoli, e che
  havevano tutte le condizioni, e parti desiderabili; E nota che il termine \textit{d'importanza}
  usatissimo da noi in simili occasioni, vale in questo caso quanto il termine
  di garbo, e per esprimere una tal quale perfezione del subietto. Il Lalli En. Tr.
  C. 1. stan. 54. dice.
\begin{verse}
E produrrà, se ben non senza duolo,
Due garbati bambocci a un parto solo.
\end{verse}
\item[A capello] Per l'appunto. E il latino \textit{ad unguem}. Termine usato da coloro,
  che si regolano col filo nello squadrare, come sono i muratori, ec. E vuol dire
  non vi corre la grossieza d'un capello dall'uno all'altro; ma si usa in ogni congiuntura
  di paragonare, o misurare una cosa con l'altra, non solo in quantità,
  come \textit{Ho riscontrato i denari, e tornano a capello}; ma anche nella qualità come nel
  caso nostro, che s'intende: erano uguali di mole di corpo, e simili di fatteze.
\item[MANGIAR il pane affatto] Mangiar bene, e senza far rosumi, o tozi; ma
  significa huomo di buon pasto. Vedi sotto C. 8. stan. 56.
\item[FAR santà] È lo stesso, che far la riverenza; ma è un termine, che è proprio
  dei bambini, quando cominciano a imparare a andare, che quel lor muoversi
  timidamente e detto dalle balie \textit{far santà}, o pure è, quando fanno la riverenza
  baciando altrui la mano; ed è così detto fare sanità, cioè fare salute; salutare.
  \textit{Diciamo insegnare al Bue far fantà} per intendere: \textit{Insegnar le scienze, o i
    termini civili a un'huomo zotico, villano, e di difficile apprensione}.
\item[SI volevano un ben matto] S'amavano grandemente, o svisceratamente. È quel
  termine \textit{Mactus}, del quale habbiamo detto sopra C. 1. stan. 76.
\end{description}
\section{Stanza XXI \& XXII}
\begin{ottave}
\flagverse{21}Arrivati che furono ambiduoi\\
A conoscer homai il pan da' sassi,\\
E saper quante paia fan tre buoi,\\
Se ben dal padre havean de gli spassi,\\
Vedendosi già grandi impiccatoi,\\
Ed a soldi tenuti bassi bassi, \\
Ostico gli pareva, e molto strano, \\
Ed in particolare a Floriano.
\end{ottave}

\begin{ottave}
\flagverse{22}Di modo che sdegnato, come ho detto,\\
Ch'il Duca per la sua spilorceria\\
Ogn'hor vie più tenevalo a stecchetto,\\
Un dì si risolvette d'andar via,\\
Ma tacquelo per fare il gioco netto,\\
Fuor ch'al fratello, al qual n'una osteria\\
Disse (veduto havendo a un fiasco il fondo)\\
Volersene ramingo andar pel mondo.
\end{ottave}

Cresciuti questi due Giovani, ed arrivati a conoscer il ben dal male, vedendosi
così grandi pareva lor malagevole il non haver denari, perché il padre per la sua
spilorceria non gliene dava, di che più d'Amadigi sentiva disgusto Floriano, onde
si risolvette d'andato via, e perché l'adempimento di tal sua risoluzione non gli
fusse impedito, non ne parlò ad alcuno, fuori che al fratello Amadigi.

\begin{description}
\item[CONOSCER il pan da sassi] e \textit{saper quante paia fan tre buoi}, Significano lo
  stesso, cioè conoscere il ben dal male. Hor. disse, \textit{Novit quid distent aera lupinis}. Si
  dice ancora in questo proposito \textit{Sapere a quanti dì è San Biagio}, E questo detto ha
  origine da un costume antico, il quale era in Firenze, che i ragazi\footnote{sic. Dalla Wikipedia: anche se l'ortografia italiana distingue in posizione intervocalica, per motivi storici, una -z- scempia e una -zz- doppia, a tale differenza grafica non corrisponde nessuna differenza di pronuncia: la zeta intervocalica, che sia scritta scempia o doppia, che sia sorda o sonora, si pronuncia sempre e comunque intensa, cioè come se fosse scritta doppia. } fattori delle
  botteghe d'arte di seta, che son situate nel Mercato Nuovo vicino alla Chiesa di
  S. Biagio, havevano licenza, passato il dì della festa di esso Santo (che sarebbe
  alli 2. di Febbraio, e se ne fa alli 3. per causa della Purificazione, il che
  ha dato occasione di usare questo dettato) di fare alle sassate, e pigliarsi ogni
  sorte di passatempo in alcune hore del giorno, ed abbaadonare la bottega per infino
  a tutto il giorno di Carnovale; e per questa causa era quel giorno tanto desiderato
  da i ragazi, che sapevano benissimo il dì, che si solennizzava la detta
  festa; onde colui, che non sapeva tal giorno, era fra i ragazzi riputato un baggeo,
  e che non havendo notizia delle cose del mondo (giudicata da loro questa
  una delle più importanti) non fusse persona abile, e di tanto giudizio da saper
  fare i fatti suoi. E questo proverbio s'è fatto poi comune a tutti gli huomini per
  intendere un'huomo scervellato, melenso, e buono a poco. Il Lasca Nov. 4. dice:
  \textit{Lo Scheggia, ed il Pilucca, che sapevano a due once, quanto colui pesava, ed a quanti
    dì è San Biagio}.
\item[SE ben dal padre havean de gli spassi] Se bene il padre dava loro de gli avvertimenti,
  e passatempi. Nota che per scherzare il nostro Poeta, subito che ha detto \textit{buoi}
  seguita \textit{dal padre}, e questo fa per toccare quel costume burlesco, il quale è
  in Firenze (ma pero fra gente bassa) che quando uno nomina \textit{bue}, \textit{becco}, o
  \textit{castrone}, l'altro dirà \textit{di tuo padre}, e dicendo \textit{vacca}, dirà di tua madre, e simili, Vedi
  sotto C. 12. stan. 49. annot. al termine \textit{morire con la grillanda}.
\item[GRANDI impiccatoi] Proibiscono le leggi l'impiccare chi non passa 18 anni;
e di qui noi diciamo \textit{grandi impiccatoi}, cioè abili a esser'impiccati, per intender
quelli, che passano la detta età di 18. anni.
\item[A SOLDI tenuti bassi bassi] Tenuti con pochi denari. Traslato dall'acque, delle
quali quando ne son poche nei laghi, pozzi, o fiumi, si dice basse. Vedi sotto in
questo C. stan. 61., e parlando d'uno che habbia pochi denari si dice: \textit{L'acgue
son basse} sì come intese colui con quel suo motto \textit{L'acque son basse, e l'oche hanno
gran sete}, cioè \textit{Alle gran voglie i danari son pochi}.
\item[SOLDO] Vale per intender danari, riccheza. E soldo è moneta immaginaria
(hoggi in Firenze effettiva di bronzo) che vale tre de nostri quattrini; Spesso usiamo
questo termine per una certa generalità: Il tale ha de' soldi, de' quattrini, dell'oro,
per intendere è ricco, non che habbia quantità di soldi, di quattrini, o d'oro effettivamente,
ma molti ne vale il suo stato; E qui intende Monete.
\item[OSTICO] Spiacevole, Malagevole, Insopportabile. È il Latino \textit{Hosticus}, che
vale per cosa da nimico.
\item[STRANO] Qui ha lo stesso significato d'ostico. Vedi sotto C.\ 3.\ stan.~1. E per
altro vuol dire stravagante da \textit{extraneus}. E molti dicono \textit{strano} a uno che habbia
cattiva cera, e per infermità sia mal condotto.
\item[SPILORCERIA] Sordidezza, Avarizia. Io credo che questa parola venga da
Pilorci, che i pellicciai chiamano quei ritagli di pelle, che non essendo buoni a
metter' in opera, gli riducono in spazzatura, la quale poi vendono per governare
i terreni, e si dica spilorcio quasi huomo vile, ed abietto quanto sono questi pilorci.
\item[TENER' uno a stecchetto] Fare star'a segno, o far patire uno di quello, che
egli ha bisogno; come non lo lasciar mangiare quanto ei vorrebbe; o haver de'
danari quanti bramerebbe. Quand'uno per la scarsezza di danari vive miseramente
si suol dire: \textit{Il tele si difende, si schermisce}, ec. ond'io non son lontano da;
credere, che questo termine sia corrotto, e che si dovesse dire a \textit{stocchetto} da
stoccheggiare, che è l'istesso che schermirsi, e può significare essere scarso, o haver
bisogno di denari.
\item[VEDUTO il fondo a un fiasco] Dopo haver bevuto un fiasco di vino; e così haver
  veduto il fondo di dentro del fiasco; ed in sustanza qui vuol dire; Dopo haver
  bevuto molto bene, o assai.
\item[ANDAR ramingo pel mondo] Andarsene errante. Ramingo vien da ramo, e
si dice \textit{Ramingo} de gli uccelli di Rapina, come esprime il Crescenzio nel Cap, 3.
della bontà degli Sparvieri lib. 18. con le seguenti parole: Si chiama \textit{nidiace, o
vero che di nidio uscito di ramo in ramo va seguitando la madre, e però si chiama Ramingo}.

Ed alli sparvieri si danno tre nomi, cioè \textit{Nidiace}, che è quello, che è cavato di
nidio, ed allevato. \textit{Ramingo} quello che uscito di Nidio non fa gran volate; e
\textit{Grifagno} quello, che già passato l'anno ha mutato alla Campagna. Ma questo
non fa a proposito nostro, bastandoci, che a similitudine di tali uccelli, dicesi
Andar ramingo colui; che hora va in un luogo, hora s'incammina in un'altro,
senza sapere positivamente, dove egli voglia andare.
\end{description}

\section{Stanza XXIII.}
\begin{ottave}
\flagverse{23}Anadigi a distorlo tutto un giorno \\
S'arrabbiò, s'aggirò com'un Paleo; \\
Ma perché quanto più gli stava intorno \\
Egli era più ostinato d'uno Ebreo, \\
Tu vuoi ir disse: e vero? o va in un forno:\\
E dopo un grande, e lungo piagnisteo;\\
Hor su vanne (diss'egli) io men'accordo,\\
Ma lasciami di te qualche ricordo.
\end{ottave}

Amadigi sentita questa risoluzione del fratello, molto s'affaticò per distornelo;
ma veduto che per la di lui ostinazione s'affaticava in vano, concorse con lui,
con questo però che gli lasciasse qualche ricordo di se,
\begin{description}
  \item[PALEO] Così chiamiamo una specie d'erba, che nasce intorno alle lagune.
    Ma diciamo anche Paleo uno strumento di legno, che serve per trastullo, e giuoco
    de' ragazzi, il quale è di figura piramidale all'ingiù; e nella testata, che viene
    di sopra ha un manichetto tondo, il quale avvoltato con uno spago, o cordicella
    s'infila in un'asticella, bucata, e tirandosi quello spago si svolta, ed il \textit{Paleo}
    scappa dal buco dell'asticella, e va per terra girando, portato dall'impulso di quello
    spago. Tale strumento da i Latini è detto \textit{Turbo} forse dalla figura piramidale.
    Verg. 7. Aneid. \textit{Ceu quondam torto volitans sub verbere turbo}, Tibull. \textit{Namque agor,
    ut per plana citus sola verbere turbo}, Dante nel Paradiso C. 18.
    \begin{verse}
      Ed al nome del alto Maccabeo
      Vidi moversi un'altro roteando
      E letizia era ferza del paleo.
    \end{verse}

E dice, così, perché a tale strumento si fa continovare il girare perquotendolo
con una sferza, dopo che egli ha havuto il primo moto, ed impulso dal suddetto
spago. Ed il proverbio \textit{aggirarsi come un paleo} vuol dire affaticarsi assai, e conchiuder
poco; che i Latini pure dissero \textit{Trochi in morem circumagi}, perché dicon \textit{Trochus}
tanto il paleo, che la trottola, portandolo dal Greco \textit{Trechos}, che vuol dir
ruota, o altro strumento che giri. Vedi sotto C. 6, stan. 22. E forse anche la voce latina
\textit{Turbo} significa tanto il paleo, che la trottola, perché \textit{Turbo} vuol dire
ogni cosa che habbia figura Piramidale, a rovescio, cioè il largo di sopra, e da
piede acuta, come appunto è il Paleo, e la Trottola; se bene non sono lo stesso
come ci testifica una certa cantilena assai praticata fra i ragazi, che dice,
\begin{verse}
E il Cristiano non è giudeo,
E la trottola, non è paleo,
E paleo non è trottola, ec.
\end{verse}
\item[PIÙ ostinato d'uno Ebreo] Ostinatissimo, che non si trova nazione più ostinata
nella sua legge, che quella de gli Ebrei, che però ha meritato il titolo, che le da
la santa Chiesa di perfidi. Cino da Pistoia, \textit{O voi, che sete ver me si giudei}: cioè
perfidi.
\item[VA in un forno] Va dove tu vuoi. E specie d'imprecazione, che suol far' uno
vinto dall'impazienza. E si suol dire anche in questo proposito: \textit{Va in malora}, \textit{va
al diavolo}, \textit{va in galea}, e simili, Abi\textit{ in malam crucem}, e Plaut. Epid. Atto 1. sc.2.
disse: \textit{Malim istius modi mihi amicos furno mersos, quam foro}.
\end{description}
\section{Stanza XXIV — XXVII}
\begin{ottave}
\flagverse{24}Allor per soddisfarlo Floriano,\\
Acciò che più tener non l'abbia in ponte,\\
Con un baston fatato, c'havea in mano\\
Toccò la Terra, e fece uscirne un fonte\\
E disse: Quindi poi ben che lontano\\
Vedrai s'io vivo, o s'io sono a Caronte;\\
Perché quest'acqua ogn'or di punto in punto\\
In che grado so sarò diratti appunto.
\end{ottave}

\begin{ottave}
\flagverse{25}S'al corso di quest'acqua porrà cura,\\
Tutto il corso vedrai di vita mia;\\
Mentr'ella è chiara, cristallina, e pura,\\
Di pur ch'io viva in festa, ed allegria; \\
Ed all'incontro, se torbida, e scura\\
Ch'ella mi va come dicea la Cia;\\
Ma quand'ella del tutto ferma il corso,\\
Di ch'io sia ito a veder ballar l'Orso.
\end{ottave}

\begin{ottave}
\flagverse{26}Ciò detto in capo il berrettin si serra,\\
Mette man, chiude gli occhi, e stringe i denti\\
E dà si forte una imbroccata in terra,\\
Ch'il ferro entrovvi fino ai fornimenti.\\
In quel che i grilli, e i bachi di sotterra\\
Sgombrano tutti i loro alloggiamenti\\
Pullula fuori un cesto di mortella,\\
E di nuovo Florian così favella
\end{ottave}

\begin{ottave}
\flagverse{27}Fratel mio caro, questa Pianta ancora\\
Com' io la passi ti darà ragguaglio,\\
Cioè mentr'ell'è verde, anch'io allora\\
Son vivo, fresco, e verde com'un'aglio;\\
E quand'ella appassisce, e si scolora,\\
Anch'io languisco, od ho qualche travaglio,\\
In somma s'ell'è secca, leva i moccoli,\\
Per farmi dire il canto in scarpe zoccoli.
\end{ottave}

Floriano per contentare il fratello, toccò la terra con un bastone incantato,
che haveva in mano, e ne fece nascere una fonte, e disse che dalla mutazione di
quell'acque haverebbe egli conosciuto lo stato, nel quale egli si trovasse. Dipoi
messe mano alla spada, e con essa bucò la terra, e scappò fuori un cesto di mortella;
E mostrò ad Amadigi, come egli si davea contenere in conoscere ancora
da questa mortella, in che grado egli si trovasse.
\begin{description}
\item[TENERE in ponte] Tener un sospeso, o irresoluto. I Latini pure dissero: \textit{In
  pontes detinere}; e però stimo, che questo nostro detto venga dall'uso antico de'
  Romani, che nell'elezione de i Magistrati chiamavano \textit{Pontes} quelle piccole tavole,
  sopr'alle quali eran posate le paniere dei voti; di che fa menzione Cic. 1.
  Rhet. \textit{Pontes disturbat, Cistas deijcit}; e tanto stavano incerti, e sospesi coloro, che
  pretendevano, quanto le ceste de i voti stavano sopra i detti Ponti; E pero dicendo:
  \textit{Ego sum super pontes}, vuol dire il mio Voto è ancora nelle Ceste, o coperto,
  e per conseguenza io sono sospeso, ed incerto di quel che habbia a esser di
  me. E ci serve poi questo detto \textit{Tener' uno in ponte} per esprimere; trattener' uno
  con le speranze, o con altro secondo il subietto.
\item[SONO a Caronte] Son morto. Son fra l'anime, le quali passano la Barca di
  Caronte, che secondo la falsa credulità de' Gentili era il Navalestro, il quale conduceva
  l'anime de i morti con la Barca alla Città di Dite. Vedi sotto C. 6. stan. 19. \& seqq.
\item[COME dicea la Cia] Mi va male, e peggio. Che questo voleva inferire una
  tal Cia, o Scia Fruttaiola con un detto sporco da lei molto usato.
\item[SON ito a veder ballar l'Orso] Anche questo detto significa son morto.
\item[IN capo il berrettin si serra, ec] Con questi due versi esprime uno, che s'accinga
  a fare un'operazione, nella quale sia necessario usar molta forza, perché in
  essi mostra quelle azioni, che per lo più son solite farsi in simili congiunture.
\item[METTE mano] Quando diciamo assolutamente metter mano; intendiamo metter
  mano all'armi. \textit{Distringere ensem}.
\item[SGOMBRANO] Vanno via; Si partono.

E qui non mi pare fuor di proposito il notare una generale portata dal
Varchi nel suo Hercolano, cioè che la lettera 'S' aggiunta nel principio di qualsivoglia
dizione nel nostro parlare ha la forza di privazione, come appresso a i
Latini la particela \textit{in} ha forza di negativa, come \textit{doctus}, \textit{indoctus}, ec. Ed
appresso di noi \textit{calzare}, \textit{scalzare}, ec, Ha però questa regola anch'essa le sue eccezioni,
come \textit{sbalordito} vuol dir \textit{balordo}, e non vuol dire \textit{senza balordaggine}; \textit{Turbare}, \textit{sturbare},
\textit{disturbare}, che suonano lo stesso con l'aggiunta, che senza. Talvolta
ancora s'aggiunge alla detta 'S' la particella \textit{di}, e particolarmente quando la
parola comincia per lettera vocale, come \textit{amare}, \textit{disamare}; \textit{interessato},
\textit{disinteresato}, ec.
\item[CESTO] Intendiamo pianta di virgulto, o d'erba, come Cesto di lattuga, di
  mortella, ec. Se bene de i virgulti si dice anche \textit{Pianta}, come si vede nella presente
  ottava 27. \textit{Fratel mio caro questa Pianta ancora}. Viene dal latino \textit{Cespes}, e noi
  pure diciamo Cespuglio. Io stimo, che pianta sia nome generico, poiché serve,
  per tutti li vegetabili, dicendosi Pianta di prezemolo, pianta di grano, e pianta
  di querce, ec. E non si direbbe di tutti cesto, ne cespuglio.
\item[VERDE come un'Aglio] Un bel verde si paragona ad un'Aglio, perché questo
  ha le sue frondi di bellissimo color verde, e che si mantengono verdi,
  è segno di sua perfezione. E però dicendosi \textit{Il tale è verde come un'aglio}, s'intende:
  è di sanità perfetta Virc. \textit{cruda Deo, viridisque senectus}. Horat. \textit{Dumque
    virent genua}, Questa similitudine si piglia da tutte le piante, la sanità delle quali
  s'argumenta dall'esser ben verdi, che dimostra non havere esse patito, ne essere
  in grado di seccarsi. Ed alle volte s'intende uno di mala sanità quando si dice
  \textit{verde come un'aglio}, ma s'intende non la frescheza, che denota il verde dell'aglio,
  ma il colore, che essendo verde nella faccia dell'huomo denota poca sanità.
\item[LEVA i moccoli per farmi dire il canto in scarpe, e zoccoli] Compra la cera per farmi
  il funerale: che moccolo vuol dire ogni piccola candela di cera, e qui è preso
  per ogni sorte di candele di cera. E quel \textit{farmi dire il canto scarpe zoccoli} è detto
  giocoso usato fra i nostri Contadini; il qual detto non è forse senza fondamento
  ne affatto improprio, che possa haver origine dalla diligenza, che si pone nel fare,
  che i morti quando son portati alla sepoltura habbiano, se sono huomini un paio
  di scarpe nuove, e se son donne un par di pianelle, o zoccoli nuovi; e \textit{zoccolo} è una
  scarpa col fondo di legno, che serve per difendere i piedi dall'acqua, che è per terra.
\end{description}

\section{Stanza XXVIII — XXX.}
\begin{ottave}
\flagverse{28}Poi che queste parole hebbe finite,\\
Dal suo caro Amadigi si licenza,\\
Il qual rimase tutto sbigottito, \\
Però che gli dolea la sua partenza, \\
Quand' in sella Florian di già salito\\
Senza gran doble, o letter di credenza \\
Andonne a benefizio di natura\\
Con due servi cercando sua ventura.\\
\end{ottave}

\begin{ottave}
\flagverse{29}E il primo giorno fece tanta via\\
Ch'i suoi lacché spedati, e conci male\\
Si rimasero, l'uno all'osteria,\\
E l'altro scarmanato allo spedale;\\
Ond'ei più non havendo compagnia,\\
Se bene accanto havea spada, e pugnale,\\
Per non haver paura in andar solo,\\
Cantava ch'ei pareva un rosignuolo.
\end{ottave}

\begin{ottave}
\flagverse{30}Così nuove canzoni ogn' hor cantando \\
Con una voce tremolante in quilio,\\
E qualche trillettin di quando in quando\\
Alle stelle n'andava, e in visibilio:\\
Onde ai timori al fin dato di bando\\
Tirava innanzi il volontario esilio;\\
E giunto a Campi, lì fermar si volle\\
A bere, e far la zolfa per bi molle.
\end{ottave}

Floriano si parte dal fratello Amadigi, il quale ne rimase afflitto. Lasciò per
la strada i Lacché stracchi, ed egli solo si condusse a Campi, dove si fermò a bere.
\begin{description}
  \item[SBIGOTTITO] Afflitto; perduto d'animo. I Latini dissero \textit{Animo deiectus}.
Quand' uno sta allegramente diciamo: Il tale sta in gote, o sta in barba di micio.
Vedi in questo C. stan. 48. Sì che uno che non stia allegramente si dice \textit{non sta in
gote}, \textit{non sta in barba di micio}; E però non farebbe gran fatto, che questa voce
sbigottito venisse dallo Spagnuolo bigottes, che vuol dir basette, e che per la lettera 'S'
che aggiunta al principio d'una parola ha forza di privazione (come,
habbiamo detto poco sopra) significasse senza \textit{bigottes}, che vuol dir senza basette,
cioè non in barba, non allegramente: o forse sbigottito, quasi sbattuto.
\item[A BENEFIZIO di natura] A caso; dove la Fortuna lo guidava.
\item[LACCHÉ] Servitori, che corrono a pié; e per lo più sono ragazzi o givanetti.
  Vedi sotto. C.~11. stan. 9.
\item[SPEDATI] In questo caso non vuol dir Senza piedi, ma con i piedi affaticati, e
  stanchi dal viaggio.
\item[SCARMANATO] Scarmana è una specie d'infermità, che viene a coloro,
  che dopo essersi soverchiamente riscaldati per violente fatica, o viaggio si raffreddano
  o col bere, o con lo stare al vento, o in luoghi freschi, e si dice: \textit{Pigliar
    una scarmana}, o \textit{scarmanare}. È forse specie di quel male che i medici chiamano
  Pleuritide, ed è comunemente chiamato mal di petto. Qui intende Affaticati
  dal viaggio, in maniera che l'anelito se li rendea difficile, e però non potevano
  camminar più.
\item[CANTAVA che pareva un Rosignuolo] Il Rosignuolo, Uccelletto noto, da i
  Latini detto philomela, ha il più bello, e gagliardo cantare di qualsivoglia Vccelletto,
  e per questo quand'uno canta bene, lo paragoniamo al Rusignuolo.
\item[VOCE tremolante] Voce, che tremava per cagione della paura; Si come i \textit{trilli}
  eran fatti per timore, e si potevano dire più tosto tremoli, o interrompimenti
  di canto cagionati dalla paura, che veramente \textit{Trilli}, che sono un riperquotimento
  di voce musicale nel medesimo tuono. Horazio disse: \textit{Cantu tremulo}.
\item[IN quilio] Secondo che mi disse il Signor Nigetti, fra i musici del nostro secolo
  il Maestro; la voce \textit{quilio} significa un cantare in voce non sua, come se uno
  havesse voce di basso, e cantasse di soprano; sì che s'intende, che Floriano cantava
  per la paura in voce falsa, e non sua naturale, che i Latini secondo Cic. lib.
  3. de Orat. la dicevano \textit{Vocula falsa}. E Titinio appresso Sesto disse \textit{Succrotilla vocula}.
\item[ANDAR alle stelle col canto] Cantar in tuono alto. Se ben qui par che voglia
dire, \textit{se n'andava in gloria}, cioè cantava con gran soddisfazione, e gusto; poi
che soggiugne \textit{in visibilio} che appresso di molti de' nostri vuol dire Andarsene in
estasi, e perdere i sentimenti per il gran gusto, Matteo Franzesi nel Cap. del suo
viaggio da Roma a Spoleti dice.
\begin{verse}
\backspace Vedea passar con torvo supercilio
Qualche Satrapo tronfio, ed appoggiato
Al tappeto, n'andava in visibilio.
\end{verse}
Vergilio Egl. 5. disse: \textit{Voces ad Sydera iactare}.
Ed ottavo Aen. \textit{Effundere voces ad athera}.
\item[TIRAVA innanzi il volontario esilio] Continovava il viaggio, che egli medesimo
  s'era eletto, esiliandosi dalla propria casa.
\item[FAR la zolfa] Detto scherzoso, che signisi a Cantare, far musica, ed è composto
  di tre note musicali, la, sol, fa. Il Signor Salvador Rosa in una sua bella
  Satira parlando della musica dice,
\begin{verse}
\backspace Quanto gira la terra a tondo a tondo,
Luogo alcuno non v'è che di schiamazzi
E di zolfe non sia pieno, e fecondo.
\end{verse}
\item[PER b molle] Il b molle è chiave musicale, o segnatura di semituono; Ma
  qui dicendo \textit{far la zolfa per b molle}, si serve della voce \textit{molle} per intendere:
  ammollare la bocca, cioè bere. E così scherzando sopra alla musica, ed havendo
  detto, che Floriano cantava; soggiugne, che volevaa seguitare a cantare anche
  nell'osteria, \textit{ma per b molle}, ed intende Vuol bere.
\end{description}

\section{Stanza XXXI \& XXXII.}
\begin{ottave}
\flagverse{31}A Campi, hora spiantato alla radice\\
Dominava in quei tempi Stordilano,\\
Se ben Turpino scrive, ed altri dice,\\
Ch'ei regnasse in un luogo più lontano,\\
Hebbe una figlia detta Doralice,\\
C'havea un'occhio c'uccidea il Cristiano,\\
Ma quel che più tirava la brigata\\
È l'esser sola, e ricca sfondolata.
\end{ottave}

\begin{ottave}
\flagverse{32}Com'io dissi, Florian nella Cittade\\
Entrò per rinfrescarsi, e toccar bomba,\\
Ma il gran fraftuono, ch'in quelle contrade\\
D'armi, di bestie, e d'huomini rimbomba,\\
Il sentir su pe i canti delle strade\\
Tutt'a cavallo risuonar la tromba,\\
Ed il voler saperne la cagione,\\
Lo fecero mutar d'opinione.
\end{ottave}

Il Poeta finge Città Regia il Castello di Campi, luogo vicino a Firenze, che
hoggi ha poca forma di Castello, per esser distrutto, e dice che già vi regnava
Stordilano, che hebbe una bellissima Figliuola nominata Doralice, la quale per
esser sola, e ricchissima, era da molti bramata in moglie. E perché questa non
sia creduta la stessa, che quella che l'Ariosto fa Figliuola di Stordilano Re di
Granata dice: \textit{Se ben Turpino scrive, ed altri} (cioè  Ariosto) \textit{dice, ch'ei regnasse
in un luogo più lontano}, cioè in Granata.

Floriano dunque, il quale era entrato in Campi solamente per pigliare un poco
di riposo, e rinfrescarsi, e andarsene, sentendo tanti strepiti d'armi, e romori
di tamburi, si risolve di trattenersi alquanto per intenderne la cagione.

\begin{description}
\item[HAVEA un occhio c'uccidea il Cristiano] Havea così begli occhi, che facevano
  innamorare ognuno. Questo detto vien forse dalla comune opinione di quel serpente
  da i latini detto \textit{Regulus}, e da i Greci, e da noi chiamato \textit{Basilisco}, il quale
  col solo sguardo avvelena, ed ammazza coloro, che egli mira. E molti Poeti
  nostrali per lodare l'occhio di bella donna hanno detto: \textit{Occhio di Basilisco},
  intendendo, che han forza di metter nel cuore il veleno d'amore. Apul.
  \textit{morsicanstibus oculis}.

\item[TIRAVA la brigata] Lusingava, incitava, allettava il popolo a desiderarla.

\item[RICCA sfondolata] Ricca senza fondo: Ricchissima. Diciamo \textit{Ricco in fondo},
  \textit{senza fondo}, \textit{sfondato}, o \textit{sfondolato}, per denotare una ricchezza,
  senza numero, o misura.

\item[RINFRESCARSI] Cioè reficiarsi col riposo, e col cibo. I Latini pure dicevano
  tal volta rinfrescarsi per ristorarsi, trovandosi \textit{refrigeratus} in vece di \textit{refocillatus}.

\item[TOCCAR bomba] Arrivare in un luogo e dimorarvi poco. Questo detto è
  tolto da un giuoco fanciullesco detto \textit{birri e ladri}, il quale fanno in questa maniera.
  S'uniscono molti Fanciulli, e tirate le sorti a chi di loro debba esser birro,
  chi ladro, quelli che sono eletti birri si mettono in mezzo della stanza, o piazza
  dove s'ha da fare il giuoco, e ciascuno de i ladri piglia il suo posto, il quale è
  già stato consegnato per immune; e questo luogo da essi è chiamato \textit{bomba}, che i
  latini dicevano \textit{meta} in questo medesimo giuoco usato ancora da i loro ragazzi, e
  da quelli de i Greci, se bene in qualcosa differentemente. Questi ladri vanno
  scorrendo da un luogo all' altro, e i birri procurano di pigliargli, ed i ladri,
  quando si veggono stracchi, corrono a trovare un di quei luoghi immuni detto
  \textit{bomba}, dove stando, sono franchi, ed i birri non possono pigliargli, e si guadagna,
  o si perde il premio stabilito, secondo che son convenuti d' esser presi, o non
  presi in tante gite; ed il ladro preso ( continovandosi il giuoco ) diventa birro,
  ed il birro, che ha preso diventa ladro. E perché nel toccar bomba si trattengono
  pero diciamo toccar bomba per esprimere arrivare in un luogo, e partirsene presto.
  E questa voce \textit{bomba} vien dal Greco \textit{bombeo}, che vuol dire Strepitare,
  o far suono, (donde \textit{rimbombare}) è da quel romore, che fanno i ragazzi con
  la voce, e con le mani per far conoscere che toccano il luogo immune, questo
  luogo è chiamato bomba. Diciamo \textit{tornare a bomba} che significa \textit{tornare al primo
  discorso}. Vedi sotto C. 8, stan. 15.

\item[FRASTVONO] Fracasso, Strepito, romore confuso, quasi dica fuor di tuono.

\item[CANTO] Cioè l' angolo che fanno le case a capo a una strada che volti in
  un'altra; detto così secondo alcuni, dal Greco \textit{Canthos}, che vuol dire Angolo
  dell'occhio, o dal canto, che nello sboccar delle strade in su le cantonate soleva
  farsi dagli antichi, come si cava da Verg. Egl. 3.
  \begin{verse}
    Non tu in trivijs indocte solebas
    Stridenti miserum stipula disperdere carmen?
  \end{verse}

Ma è detto dai Greco \textit{camptin}, che vuol dire Piegare.
\item[TUTTI a cavallo] Così chiamano i Soldati quella suonata di tromba, che fa intendere
  a i medesimi il montar' a cavallo, la quale par che esprima; \textit{Tutti a cavallo}.
  Costume tolto da i Latini, che per significare il suono della tromba dicevano
  secondo Servio, ed Ennio \textit{Taratantara}.
  \begin{verse}
    At tuba terribili sonitu taratantara dixit.
  \end{verse}
\end{description}

\section{Stanza XXXIIL}
\begin{ottave}
\flagverse{33}Era già scavalcato ad una ostessa,\\
Per far, sì com'ei fece, un conticino,\\
Ne altro hebbe che pane, e capra lessa,\\
Che fitra anche gli fu per mannerino.\\
Bevve al pozo una nuova manomessa,\\
Perch'il vinaio havea finito il vino;\\
Fece conto, e pagò ben volentieri\\
Poi chiese il fin di tanti Strombettieri.
\end{ottave}

\begin{ottave}
\flagverse{34}Ella rispose: E come; E non lo fai?\\
Se per Campi non è altro discorso,\\
Che havendoil Re una figlia, c'hoggi mai\\
Abbraccerebbe un'huom prima c'un'orso;\\
E perché reda ell'è bell', e d'assai,\\
Di pretendenti havendo un gran concorso,\\
Bandire ha fatto, acciò nessun si lagni,\\
Ch'in giostra chi la vuol se la guadagni.
\end{ottave}

\begin{ottave}
\flagverse{35}Ma c'occorre ch'in ciò più mi distenda,\\
Mentre la cosa è tanto divulgata?\\
Però lasciami andar, ch'io ho faccenda\\
Havendo sopra un'altra tavolata.\\
Dice Florian, che ai suoi negozzi attenda,\\
Scusandosi d'haverla scioperata\\
E rimessa la briglia al suo giannetto,\\
Come un pardo saltovvi su di netto.
\end{ottave}

Floriano essendo scavalcato a un'osteria, dopo che hebbe mangiato, e pagato
intese dalla padrona dell'osteria, che quei romori di trombe si facevano
perché il Re voleva maritare la Figliuola a quel Cavaliere, che meglio si portasse
la giostra; onde Floriano montò subito a cavallo per andare a veder questa festa,
\begin{description}
  \item[FARE un conticino] Così usiamo dire per farsi intendere copertamente Andar a
mangiare all'osteria.
\item[FITTO gli fu] Gli fa fatto credere. Gli fu dato ad intendere che e' fusse
Mannerino. Il verbo ficcare usato in questi termini serve per esprimere, che
quella tal cosa fu data per maggior prezzo di quel che ella valeva, o per di miglior
qualita, che ella non era. Vien da ficcar carote, che vedremo sotto questo Cant.
stan. 70. e Cant. 6, stan. 68. Lat. \textit{imponere alicui}.
\item[MANNERINO] Specie d'agnelli castrati, che nella nostra Toscana è ottima
  nel Territorio, e contado di Pistoia, ed è carne squisita al contrario della capra,
  che è la peggiore, che si mangi, ed in particolare cotta a lesso.
\item[MANOMESSA] Quando all'Oste arriva portatogli dalla montagna il vino
  primo cavato dalla botte si dice: \textit{l'oste ha havuto la manomessa}, Ed i Fiorentini,
  che son di buon gusto, o più tosto ghiotti nel bere, lo pigliano più volentieri,
  quando è vino di manomessa, non tanto per la curiosita di gustare quel nuovo
  vino, quanto perché non piacendo loro le fondate, hanno caro di bere del primo,
  che esce della botte, onde pare che il Poeta voglia intendere, che Floriano se
  bene bevve acqua hebbe nondimeno gusto, perché era nuova manomessa, ma in
  effetto gli da la burla dicendosi che \textit{bevve una manomessa nuova} cioè insolita, non
  essendo solito, ne costume, che si manometta il pozzo, se non per le bestie.
\item[VINAIO] Cioè colui che nell'osterie dà il vino. Per maggior intelligenza di
  questo è necessario sapere, che nell'Osterie di Firenze stanno due maestri, e tengono
  garzoni differenziati; Uno di questi maestri è il padrone principale ed in
  lui dice l'Osteria, e questo si chiama il Vinaio; altro è maestro anch'egli, ma
  solamente della Cucina, della quale paga un tanto il mese di pigione al Vinaio,
  dal quale può esser mandato via. Ho voluto dir questo, perché so che a i Forestieri
  è di non poca confusione questa distinzione, perché si fanno far il conto da
  uno, e pensando d'haver finito; gli sopraggiugne poi il secondo Oste, che fa loro
  il conto della Cucina, e cresce la somma del primo conto fatto dal Vinaio.
\item[FECE conto] \  Domandò quanto dovea pagare. Trattandosi d'osterie \textit{Far conto}
  s'intende Haver finito di mangiare.
\item[STROMBETTIERI] \  Intende il romore, che fa il suono delle trombe.
\item[ABBRACCEREBBE un huom prima c'un'orso] \makebox[8pt]{} Così diciamo d'una Fanciulla,
che sia in età da maritarsi, e che sia bella, grande, e ben formata, intendendo
che sia in eta da bramar l'huomo, e da distinguerlo da un'orso, o da non fuggirlo,
come farebbe all'orso. Virg. \textit{Iam matura viro, plenis \& nubilis annis}.
\item[D'ASSAI] Valente, contrario di Dappoco: pare che suoni lo stesso che in latino
  \textit{praestans}.
\item[REDA] Vedi sopra in questo Canto stan. 12. Qui è preso nel suo proprio significato
  d'herede, o successore nelle facultà; e vuol dire che essendo ella Figliuola
  unica del Re, dovea hereditare tutto quello che egli possedeva.
\item[TAVOLATE] Così chiamano li nostri Osti tutti coloro, che vanno a mangiare
  alle tavole delle loro osterie, tanto se fusse un solo per tavola, quanto
  se fussero più, pur che seggano a mangiare a tavola.
\item[SCIOPERATA] Levata dal lavoro, o dall'opera. Vedi sopra C. 1. st. 29.
\item[GIANNETTO] Intende cavallo. Sendo i giannetti specie di cavalli, che vengono
  di Spagna del paese d'Asturia, e perciò dai Latini detti \textit{Asturcones}.
\item[PARDO] Il Gatto pardo è animal noto, come è anche nota la di tui feroce
  agilità, e destrezza; e però appresso di noi è in uso questa comparazione quando
  vogliamo intender l'agilità di vita d'alcuno. Vedi sopra C. 1. stan.~11, \textit{Le scale
  corre lesto come un gatto}.
\end{description}
\section{Stanza XXXVI — XXXVIII}

\begin{ottave}
\flagverse{36}Tocca di sproni, e vanne, e giunge in piazza\\
Dov'egli ha inteso che s'ha far la giostra,\\
Che per vedere il popol vi s'ammazza,\\
E appunto i Cavalier facean la mostra.\\
Sedeva il Re presente la Ragaza,\\
Che quanto adorna, e bella si dimostra,\\
Tanto è confusa havendo a haver consorte,\\
Non a suo mo, ma qual vorrà la sorte.
\end{ottave}

\begin{ottave}
\flagverse{37}Floriano in contemplar faccia sì bella,\\
Dove quel crudo balestrier d'amore\\
Tira frecciate, come la rovella,\\
Sentissi anch'esso traforare il core,\\
E com'huomo di marmo, in su la sella\\
Restò perplesso, e pieno di stupore,\\
Scorgendo Amor, le Grazie, e in un raccolto\\
Le Trombe, e il non plus ultra d'un bel volto.
\end{ottave}

\begin{ottave}
\flagverse{38}Po' far! (dicea) che bella creatura !\\
Quell' Ostessa da vero havea ragione, \\
Perch'ella è bella fuor d'ogni misura,\\
Per me non saprei darle eccezione.\\
Capperi può ben dir d'haver ventura\\
Quello a cui tocca così buon boccone;\\
Ma s'ella s'ha da vincer con la lancia,\\
Hoggi è quando ci arrischio anch'io la pancia
\end{ottave}

Floriano giunto in piazza veduta Doralice così bella se ne invaghisce, e risolve
però di tentare la fortuna, e cimentare la sua persona per avventurare il conseguirla
per moglie.

\begin{description}
\item[Il Popol vi s'ammazza] V'è tanto popolo per veder quella giostra, che s'ammazzano
  l'un l'altro per la strettezza. Hiperbole usatissima in questo proposito
  per esprimere la gran calca, o quantità di popolo.
\item[FANNO la mostra] Quando i Cavalieri, o soldati, o altre genti, che devono
  fare qualche operazione guerriera (ancor che finta) avanti di cominciare a operare
  compariscono in ordinanza questo si dice far la mostra.
\item[LA Ragazza] Intende Doralice figliuola del Re.
\item[A SVO mo] Secondo il suo gusto. Quel \textit{mo} vuol dir modo, usandosi da noi,
come da i Latini, e da i Greci la figura Apocope, che leva l'ultime sillabe alle
parole, e da noi alle seguenti particolarmente; \textit{Modo}, \textit{meglio}, \textit{fede}, \textit{voglio}, \textit{vedi},
\textit{frate}, \textit{santo}, \textit{piede}, ec. Che diciamo: \textit{mo}, \textit{me'}, \textit{fè}, \textit{vo'}, \textit{vè}, \textit{fra}, \textit{san}, \textit{pié}. Ho voluto
notar queste, perché spesso nel nostro parlare ci vagliamo di questa figura, e
si troverà ancora spesso usata nella presente Opera, come habbiamo accennato
ancora sopra C. 1. stan. 10.
\item[TIRA frecciate come la rovella] Tira dardi, e frecce in quantità. Di questo
  termine \textit{come la rovella}, \textit{come la rabbia}, \textit{come il canchero}, ci serviamo per esprimere
  quantità grande, o  vero operazione violenta in superlativo grado; come per
esempio \textit{Il tale corre fortissimo}, \textit{il tale perquote gagliardamente} diremmo \textit{Il tale corre
come la rovella}, \textit{rabbia} o \textit{canchero}, o \textit{perquote come}, ec, E si deduce la comparazione
dalla violenza, con la quale opera il male della rabbia, o del canchero. La
voce \textit{rovella}, o rovello, credo inventata dalle donnicciuole per non profferire la
parola rabbia, come si dice \textit{cappita} in vece di \textit{canchero}, E se bene hanno del furbesco, son tuttavia, molto usate, e l'usò il Malatesti in alcune sue ottave.
\begin{verse}
 Da poi ch'io ho servito per zimbello,
 E sono andato trenta mesi aioni
 Gridando per la rabbia, e pel rovello
 Come fa il Gatto quand'ha i pedignoni ec,
\end{verse}
Ed habbiamo il verbo \textit{arrovellare}, e l'addiettivo \textit{arrovellato}. In somma in
questo luogo dicendo \textit{Tira frecciate come, la rovella} intende, che Doralice con le sue
 gran bellezze faceva innamorare ognuno, che la vedeva.

\item[LE Grazie] I Poeti fingono, che le grazie sieno tre figlie di Giove nominate
  Aglaia, Eufrosine, e Thalìa. \textit{Aglaos} in Greco val per splendido, Eufrosine, ilarità,
  allegrezza, e Thalìa, verdeggiante. sì che dicendo \textit{si scorge in quel volto le
    grazia} vien' a dire: Si conosce in lei splendidezza, allegrezza, e freschezza, cioè
  gioventù sana.
\item[RACCOLTO in uno] Unito in un solo luogo, Termine latino, usato alle volte
  anche da noi in questo proposito.
\item[LE Trombe] Nella più stimata carta de' Ganellini, o Minchiate è effigiata la
  Fama con due trombe alla bocca, e da questa tal carta si chiama le Trombe; E
  per esser questa la superiore a tutte l'altre carte quando si dice: \textit{La tal cosa è le
    trombe} s'intende, che questa tal cosa sia la meglio, che si trovi nel suo genere.
  Ed è detto assai usato per esprimere l'eccellenza d'una cosa, ed ha la forza del
  superlativo.
\item[NON plus ultra] È noto il motto delle colonne d'Hercole, che vuol dire:
  \textit{Non si vadia più avanti}, E noi ce ne serviamo nelle congiunture simili alla presente,
  che s'intende; non si può andar più là, cioè non si può avanzare, o superare
  tal bellezza, o vero non si può far più bella. Esprime anche questo termine
  un superlativo,

\item[PUO' fare] E' termine d'ammirazione,o stupore quasi diciamo: Può mai fare
  il Cielo, o la natura una cosa tanto bella, e perfetta come questa?
\item[CAPPERI?] Ancor questo è termine d'ammirazione; e si dice ancora \textit{cappita},
\textit{canchita}, \textit{canchigna} forse per non dir canchero: Voci inventate dalle donne, come
habbiamo accennato poco sopra alla voce \textit{rovella}. Consuona col latino \textit{Papae}, che
noi diciamo \textit{Pà!} e col latino \textit{babae}, che noi diciamo, \textit{o babbo !} E la parola \textit{capperi},
che tanto in Greco, che in Latino vuol dire il \textit{cappero} frutto noto, serviva anche
a' medesimi per termine d'ammirazione, o giuratorio, come si vede in Laerzio
nella vita di Zenone. \textit{Sed, \& per capparim iurabat, sicut Socrates per canem}, ec. Lo
stesso riferisce Alex.\ ab Alex.\ dier.\ gen.\ lib.\ 5.\ cap.\ 10. Il Lalli nella sua En.\ trau.\
C.\ 1.\ stan.\ 85.
\begin{verse}
Capperi disse Enea, come sì tosto
Fatt' ha sì gran Città questa Signora!
\end{verse}
\item[A CHI tocca così buon boccone] Chi havrà così buona sorte. Chi havrà per moglie
  così bella, e ricca Giovane.
\item[CI arrischio anch'io la pancia] Ci avventuro anch'io la vita.
\end{description}

\section{Stanza XXXIX}
\begin{ottave}
\flagverse{39}O per tutt' hoggi beccomi su moglie \\
Nobile, ricca, e bella; o veramente \\
Vi lascio l'ossa; s'ella coglie, coglie \\
Se nò a patire: O Cesare, o niente.\\
Ciò detto salta in campo, e un'asta toglie,\\
Intruppandosi là dov'ei già sente,\\
C'appunto il Re sollecita, e commette,\\
Che pe' i primi si tirin le bruschette.
\end{ottave}

Risoluto Floriano di provarsi in questa giostra si fa innanzi, e piglia una lancia.
Qui bisogna supporre, che Floriano, e gli altri Cavalieri fussero armati di
dosso, come è necessario, che sieno i Cavalieri, che giostrano a corpo a corpo.

\begin{description}
\item[BECCOMI su moglie] Questo verbo beccare ha signiticato di rubare, guadagnare, o acquiftare,
  Gio. della Casa nel Capitolo in lode del martello d'amore dice
\begin{verse}
So che sapete del ladro sottile,
C'a Giove fe la barba già di stoppa,
Quando gli beccò fu l'esca, e il fucile.
\end{verse}

E però usato per lo più scherzando in occasione di maritaggi, come appunto
nel presente luogo, E si dice \textit{Il tale pigliò moglie, e becca su una buona dote}. E lo
scherzo nasce dal verbo \textit{beccare}, che è noto quel che significhi trattandoli d'ammogliati.

\item[S'ELLA coglie, coglie] S'io m'appongo, sarà bene. S'io vincerò l'havrò indovinata,
  e sarò felice, \textit{Se no a patire}, Se non m'appongo, sarà disgrazia, havrò
  pazienza. In somma con questi due detti vuol mostrare, che Floriano ha l'animo
  accomodato a tutto quel che sia per succedere, o male, o bene che sia.

\item[O Cesare, o niente] \textit{Aut Caesar, aut Nihil}, O morire, o esser qualcosa di garbo.
  Questa sentenza latina si profferisce da noi corrottamente, O Ceseri, o Niccolò, ed esprime
  \textit{Aut Rex, aut asinus} de i Greci, cioè uno de due estremi.

\item[SI tirin le buschette] Si tirino le sorti. Credo che si chiamino bruschette, e non
buschette, o forse in ambedue i modi; che è un giuoco da Fanciulli, e si fa con
pigliare tante fila di paglia, o altra materia simile, quanti sono coloro, che hanno
a concorrere al premio proposto, e quel filo, che tira il premio, si fa o più
lungo, o più corto de gli altri; detti fili s'accomodano fra due assi, o in mano
in modo, che non si veda se non una delle due testate di essi, per le quali testate
ciascuno de' Ragazzi cava fuori il suo, e quello che tira il più lungo, o il più
corto, secondo che è destinato, conseguisce il premio proposto; Questo giuoco
serve ancora ai Ragazzi per fare le divisioni ne i loro giuochi Fanciulleschi, come
farebbe ne i Birri, e Ladri detto sopra in questo C. stan. 32. alla voce Bomba, che
allora pigliano tanti fili, quanti sono i Ragazzi, la metà lunghi, e la meta corti,
e cavandoli da loro a uno per volta detti fili; quelli, che hanno i lunghi, vanno
da una banda, e quelli de' corti dall'altra; e così serve a loro, come serve nel
presente luogo, per un modo di tirar le sorti. E da questi bruscoli, o fili di paglia
mi do a credere, che si dica \textit{bruschette}; e che \textit{buschette} sia quel giuoco, che si
con certi pezzetti di mazza rifessa, e che si tirano, come i dadi, con altro
nome dette \textit{le buffe}. Vedi sotto C.~11. stan. 42.
\end{description}

\section{Stanza XXXX \& XXXXI}
\begin{ottave}
\flagverse{40}Come volontaroso Floriano,\\
Senza chieder licenza, o cosa alcuna,\\
Si fece innanzi, e postavi la mano\\
Di trarne la più lunga hebbe fortuna,\\
Poco dopo il Marchese di Soffiano\\
Simile a quella anch'egli ne trasse una\\
Ond'essi, come pria fu destinato,\\
Furono i primi a correr lo steccato.
\end{ottave}

\begin{ottave}
\flagverse{41}Piglian del campo, e al cenno del trombetta\\
Si vanno incontro con la lancia in resta;\\
Il Marchese a Florian l'havea diretta;\\
Per chiapparlo nel mezzo della testa;\\
Ma quei, ch'e furbo, a un tempo fa civetta,\\
E aggiusta lui, dicendo: Assaggia questa,\\
Perché gli diede sì spietata botta\\
Ch'egli andò giù come una pera cotta.
\end{ottave}

Floriano prese una di dette Bruschette, ed una ne prese il Marchese di Soffiano;
e questi due furono i primi a correre la lancia, nel qual' incontro il Marchese rimase
abbattuto. \textit{Marchese di Soffiano}, È nome a caso, e fa Marchesato una contrada,
o villa vicina a Firenze detta Soffiano.
\begin{description}
\item[CHIAPPARE] Val per colpire.

\item[FURBO] Se ben la voce furbo deriva dal latino \textit{Fur}, che vuol dir Ladro, tuttavia
  ce ne serviamo per esprimere un'huomo scellerato, e che habbia ogni sorta di vizio,
  come s'è detto sopra in questo C. stan. 2. Ed ancora per denotare un'huomo
  aftuto, e che sappia il conto suo, come segue nel presente luogo.

\item[FA CIVETTA] Abbassa la testa. Viene dal giuoco di civetta, che da i giovanotti
  si fa in questa maniera. S'accordano tre, ed uno di loro, al quale è toccato
  in sorte, si pone in mezzo a gli altri due, i quali s'ingegnano di cavargli il
  berrettino di testa con le percosse della mano; e quando egli tocca terra con le
  mani, non puo esser percosso; e però hora alzandosi, hora abbassandosi; tira
  guando all'uno, e quand'all'altro di gran mostaccioni; dura il giuoco fintanto
  che da uno delli due gli sia fatta cascare con un colpo la berretta dalla testa, che
  allora perde il premio proposto, e lo vince colui, che gliel'ha fatta cascare, il
  quale (seguitandosi il giuoco) va nel mezzo in luogo del primo. Tal giuoco si
  fa a tempo di suono, e piglia il nome dalla Civetta uccello, che per buscare il
  vitto scherza con gli uccelletti alzando, ed abbassando la testa, come appunto fa
  colui, che sta nel mezzo. E da questo poi \textit{far civetta} s'intende Abbassare il capo.
  Da Scops, che è un'uccello notturno del genere delle Civette. Era appresso i Greci
  una sorta di giuoco, o passatempo detto \textit{Scopias}, nel quale veniva contraffatto a
  tempo di ballo il muoversi in giro, e l'alzare, e l'abbassare della testa di quell'uccello;
  onde ne fu formato il verbo \textit{Scoptein} irridere, che appresso i Greci vale,
  quel che appresso noi Toscani, Uccellare. V. Giulio Polluce l. 4. cap. 14.

\item[AGGIUSTA lui] Aggiustar uno, s'intende Fargli il suo dovere, e trattare uno
  come egli merita, Lat. \textit{concinare}. Vuol dire ancora conciar male uno, come s'intende
  nel presente luogo, e sotto C.~11. stan. 50. E per altro vuol dire Saldare, o pagare
  un debito. Lat. \textit{pariare}.

\item[BOTTA] Colpo, o percossa. E questa voce \textit{botta} per altro vuol dire una specie
  di Rospo. Lat. \textit{rubeta}.

\item[ANDÒ giù com'una pera cotta] \makebox[1em]{} Cascò giù facilmente, ed a piombo, come fanno
  le pere cotte dal Sole, che cascano facilmente dall'albero; o forse come le
  cotte al fuoco, che son facilissime a andar giù in corpo quando si mangiano.
  Plauto disse: \textit{Tam crebri ad terram decidunt ut pyra}; da che si deduce che s'intenda
  delle pere, le quali cascano dall'albero,
\end{description}

\section{Stanza XXXXII.}
\begin{ottave}
\flagverse{42}In quanto a Sposa, homai questo è ascolto; \\
S'ei toccò terra, ancor la voglia sputi:\\
Così Florian dicea; ne stette molto\\
Ch'il secondo ne viene a spron battuti,\\
Che mette lui per morto, anzi sepolto,\\
Ma il giovane, che dà di quei saluti,\\
Gli mostra in avviarlo per le poste\\
L'error di chi fa i conti senza l'Oste.
\end{ottave}

Comparve il secondo Cavaliere il quale si dava a credere d'haver già morto
Floriano; ma questo col buttarlo a terra, gli fece conoscere quanto s'era ingannato.

\begin{description}
\item[È ASCOLTO] È licenziato. I ragazi, che vanno alle squole, quando sono
stati sentiti leggere dal Maestro si dicono \textit{ascolti}, e s'intendono licenziati: e così
questo Cavaliere essendo passato per le mani del Maestro, che è Floriano, si può
dire \textit{ascolto}, e licenziato dalla Sposa.
\item[TOCCAR terra, e sputar la voglia] Dicono le donne, che quando son pregne,
  venendo loro voglia di qualche cosa, se in quello stante si toccano con le proprie
  mani in alcuna parte del corpo, quivi nasca alla creatura un segno simile a quella
  tal cosa desiderata; e i segni poi chiamano voglie; e che per sfuggire che
  la creatura non nasca con tali segni, o voglie, il rimedio sia, che la Donna pregna,
  quando le viene tal desiderio, tocchi subito terra con la mano, e sputi dicendo
  \textit{A terra vadia}. E però il Poeta, seguitando questa opinione, dice, che se
  il Marchese ha toccato terra per liberarsi dalla voglia della Dama, è necessario
  ancora che egli sputi, a voler che il rimedio sia fatto compitamente, Tal detto
  \textit{sputar la voglia}, è assai vulgato per intender uno, che habbia gran desiderio d'una
  tal cosa, che sia a lui impossibile a conseguire. Vedi Plin. lib. 28.c. 4.

\item[A SPRON battuti] A tutta carriera; Velocemente. Fran. Sacc. Novella \textit{mihi}
31. \textit{E così salito a cavallo n'ando a spron battuti al Palazzo de' Signori}.

\item[LO mette per morto, anzi sepolto] Intende; che questo secondo Cavaliero non
solo credeva di havere a uccidere Floriano; ma gli pareva già d'haverlo ucciso.
Esprime la gran presunzione, che havea di sé stesso questo Cavaliero, e la poca
stima, che faceva di Floriano.

\item[DI quei saluti] Intende di quelle percosse.

\item[FAR il conto senza l'Oste] Stabilire per fatta una cosa, alla quale deve intervenire,
  e concorrere anche la volontà d'un'altro. Dove è l'interesse del compagno,
  si può metter in sicura la propria volontà, ma non quella del compagno.
\end{description}

\section{Stanza XXXXIII.}
\begin{ottave}
\flagverse{43}Comparso il terzo, in testa della lizza\\
S'affronta seco, e passalo fuor fuora;\\
Soggiunge il quarto ed egli te l'infizza\\
Sbudella il quinto, e fredda il sesto ancora\\
All'altro manda il settimo indirizza;\\
L'ottavo, e il nono appresso investe, e fora;\\
E così a tutti con suo vanto, e fama\\
Cavò di testa il ruzzo della Dama.
\end{ottave}

In questa ottava l'Autore narra la vittoria, che hebbe Floriano di sette Cavalieri,
e descrive la lor perdita in sette modi di dire diversi; il primo \textit{lo passa fuor
fuora}, il secondo \textit{l'infizza} (si dovrebbe dire infilza ma non solo perché gli è permessa
questa licenza per causa della rima, quanto anche perché per i più si dice infizza,
e non infilza, s'è fatto lecito dirlo anch'egli) il terzo \textit{lo passa fuor fuori}, il quarto
\textit{lo fredda}, il quinto \textit{l'indirizza all'altro mondo}; il sesto \textit{l'investe}, ed il settimo
\textit{lo fora}. E questi sette modi di dire havendo quasi tutti lo stesso significato d'ammazzare
danno l'occasione d'ammirar l'artifizio del Poeta in mostrate la fecondità
della nostra lingua Fiorentina.

\begin{description}
\item[LIZZA] Che si dice anche Nizza. Vuol dir linea; ma da noi s'intende quel
tavolato, o muro, rasente al quale corrono i Cavalieri le lance al Saracino.
\item[CAVÒ di testa il ruzzo della Dama] Fece uscir di testa il desiderio della dama.
  La voce ruzzo, che dal verbo ruzzare vuol dir Baie, usata in questi termini significa
  prurito, umore, desiderio, ec, sì che dicendosi. \textit{Il tale ha questo ruzzo in
  testa}, vuol dire il tale ha questa voglio, questo humore, ec. Il Laica nov. mihi
  8. dice. \textit{Deliberarono di dargli così fatta gastigatura, che gli uscisse per sempre
  l'humore, e il ruzzo di testa}.
\end{description}

\section{Stanza XXXXIV.}
\begin{ottave}
\flagverse{44}Il Re si rallegrò con Floriano;\\
Sceso di sedia poi con la Figliuola\\
Le fece allor' allor toccar la mano,\\
Come nel Bando havea dato parola;\\
Ond'ogni altro ne fu mandato sano;\\
Ed ei nelle dolcezze infino a gola\\
Bem pasciuto, servito, e ringraziato\\
Rimase quivi a goder il Papato.
\end{ottave}

Il Re fece toccar da Floriano la mano alla Figliuolo, e gliela diede per moglie,
licenziando ogni altro pretendente, e Floriano rimase quivi a godere
queste sue felicità.
\begin{description}
\item[TOCCAR la mano] È lo stesso in questo caso, che che diciamo \textit{impalmare},
o far \textit{l'impalmamento} dal toccamento, che si fa della palma della mano dagli sposi;
che è il primo atto che si faccia per lo stabilimento del contratto del matrimonio,
Vedi sotto C, 12. stan. 50. '
\item[MANDATO sano] Cioè licenziato, ed escluso. Il verbo: \textit{valeo}, che significa
  Star sano, e usato da i latini anche per licenziarsi: \textit{parentibus vale dixit}, ed il
  simile facciamo noi, come si vede nel presente luogo., che diciamo \textit{Mandar sani}
  in vece di licenziargli. Anzi il medesimo verbo \textit{valeo} è tal volta usato da noi per
  intendere Addio, cioè licenziarsi. Il Vai in una sua frottola (se ben pedantesca)
  lo mostra dicendo.
\begin{verse}
  Hore liete,
  Iam vatlete. valete.
  Iam valete amati serculi;
  E tu vale,
  O sodale,
  Che maneggi i miei liberculi.
\end{verse}
Il nostro Poeta sotto C. 6, stan. 18.
\begin{verse}
  Restò la donna, ed ei le disse vale.
  \end{verse}

\item[NELLE dolcezze infino a gola] Immerso nei piaceri, e ne i gusti, sotto C. 4.
stan. 42. dice \textit{esser ne guai a gola}.

\item[GODERE il Papato] Goder le felicità concedutegli dal Cielo.
\end{description}

\section{Stanza XXXXV. — XXXXVIII.}
\begin{ottave}
\flagverse{45}Tre dì suonaro a festa le campane,\\
Ed altrettanti si bandì il lavoro,\\
E il Suocero, che meglio era del pane,\\
Vn' huom discreto, ed un coppa d'oro,\\
Faceva con gli Sposi a scaldamane,\\
Tal'hora a Mona luna, e Guancial d'oro,\\
E fece a' Paggi recitare a mente\\
Rosana, e la Regina d'Oriente.
\end{ottave}

\begin{ottave}
\flagverse{46}L'andar il giorno in piazza ai Burattini,\\
Ed agli Zanni furon le lor gite;\\
Ogni sera facevansi festini\\
Di giuoco, e di ballar veglie bandite;\\
E chi non era in gambe, ne in quattrini\\
Da trinciarle, e da fare ite, e venite,\\
Dicea novele, o stavale a ascoltare,\\
Faceva al Mazzolino, o alle Comare.
\end{ottave}

\begin{ottave}
\flagverse{47}Altri più là vedevansi confondere\\
A quel giuoco chiamato gli Spropositi,\\
Che quei ch'esce di tema nel rispondere\\
Convien ch'il subito depositi,\\
Ad altri piace più Capanniscondere,\\
Hann' altri varij humor, varij propositi,\\
Perché ognuno a un mo non è composto,\\
Però chi la vuol lessa, e chi arrosto.
\end{ottave}

\begin{ottave}
\flagverse{48}Chi fa le Merenducce in sul bavaglio;\\
Chi con amico fa a Stacciabburatta\\
Chi all'Altalena, e chi a Beccalaglio;\\
Va quello a Predellucce, un s'acculatta;\\
Per tutti in somma sempre vi fu taglio\\
Di star lieto così in barba di gatta,\\
E tra Floriano, il Re, e la Figliuola\\
Mai fu che dir n' un' anno una parola.
\end{ottave}

In queste quattro ottave il Poeta narra le feste, ed allegrie, che si fecero in
Campi per lo sposalizio di Doralice con Floriano; le quali feste fa che non trascendano
il genio puerile per continovare a scrivere una novella per i Fanciulli.
\begin{description}
\item[ERA meglio che il pane] Era un' huomo buonissimo, un' huomo che si
accordava a ogni cosa, appunto come è il pane, che s'accorda, ed unisce con
tutte le vivande, almeno appresso a i Fiorentini. In questo proposito i Greci
dissero, \textit{Columba mitior}.
\item[VNA coppa d'oro] Uno al quale non sia da apporre alcun difetto, \textit{omni exceptione
  maior}. Credo che si dica \textit{coppa d'oro}, per intendere oro coppellato, o di coppella,
  cioè raffinato, che Coppella si dice quello strumento, col quale si riduce
  l'oro alla sua vera purità, e perfezione; e \textit{Coppa} vuol dir bicchiere, o altro
  vaso simile, donde poi \textit{Sottocoppa} quella tazza, sopr'alla quale si portano i bicchieri,
  dando da bere, e \textit{Coppiere} quel che porta da bere al Signore.
\item[SCALDAMANE] Quattro, o più s'accordano, e mette ciascuno ordinatamente
  le mani sopra quelle del compagno, e poi vanno cavando per ordine quella
  mano, che è in fondo, e mettendola di sopra all'altre mani, e con quello
  modo; e confricazione pretendono scaldarsele; e però tale operazione è detta
  Scaldamane; ed è giuoco Fanciullesco, che ha la sua pena per chi erra cavando la
  mano, quando non tocca a lui.
\item[MONA luna] S' accordano molti Fanciulli, e tirano le sorti a chi di loro habbia
  a domandar consiglio a Mona luna, e quello a cui tocca vien segregato dalla
  conversazione, e serrato in una stanza, acciò che non possa intendere chi sia,
  quello di loro, che, resti eletto in Mona luna, della qual Mona luna si fa l'elezione
  fra gli altri, che restano dopo che colui è serrato. Eletta che è Mona luna,
  si mettono tutti a sedere in fila, e chiamano colui, che è serrato, acciò che
  venga a domandar il consiglio a Mona luna, Questo tale se ne viene, e domanda
  il consiglio a uno di quet ragazzi, quale egli crede, che sia stato eletto in Mona
  luna, e se s'abbatte a trovarlo, ha vinto; se no, quel tale, a cui ha domandato
  il consiglio gli risponde; io non son Mona luna, ma sta più giù, o più su, secondo
  che veramente è posto quel tale, che è Mona luna; ed il domandante perde
  il premio proposto, ed è di nuovo riserrato nella stanza per tanto, che dai
  Fanciulli sia creata un'altra Mona luna, alla quale egli torna a domandar consiglio,
  e così seguita fin a che una volta s'apponga, ed allora vince; e quello che
  è Mona luna perde il premio, e vien riserrato nella stanza, diventando colui, che
  deve domandare, e quello che s'appose, s'intruppa fra gli altri ragazzi. Il domandante
  richiede fino a quattro volte il consiglio, e può perder quattro premi,
  e poi fimescola fra gli altri ragazzi, esente però da dover più esser domandante,
  se non nel caso, che fatto Mona luna, egli perdesse, e sempre ritorna a
  creare nuova Mona luna, e si deputa nuovo domandante, quando il primo s'apponga,
  o habbia domandato, quattro volte il consiglio, la qual funzione, come
  detto, non può esser forzato a fare, se non quattro volte: ed i premj si adunano,
  e si distribuiscono poi fra di loro ripartitamente, e dal rendergli poi a di
  chi sono, cavano un'altro passatempo, come diremo. Da questo giuoco viene
  il proverbio \textit{Più su sta Mona luna}, che significa Nella tal cosa è misterio più
  importante di quel che altri si pensa.

Nota che tanto questo giuoco, quanto ogni altro, che troveremo nella presente
Opera s'altera, e diversifica secondo li gusti, e convenzioni puerili;
e non mi riprendere se tu ne havessi nella tua puerizia fatti, o veduti fare
alcuni, o tutti diversamente da quello, che io gli descrivo.

\item[GVANCIAL d'oro] Questo pure è giuoco Fanciullesco, quale è fatto così:
  S'adunano più Fanciulli, ed uno si mette a sedere sopra a una seggiola, ed un'altro
  se li pone inginocchioni avanti, e posa il suo capo in grembo a quel che
  siede, il quale gli chiude gli occhi con le mani, acciò che non possa vedere chi
  sia colui, che lo percosse in una mano, che egli si tiene dietro sopr' alle reni, dovendolo
  egli indovinare; e calui che gli serra gli occhi, dopo che questo tale è
  stato percosso gli dice : Chi t'ha percosso? ed egli risponde: \textit{Ficoseccho}; e l'altro
  replica: \textit{Menamelo qua per un'orecchio}. Ed allora quello si rizza, e va a pigliar
  colui, che egli crede il percussore, e se s'appone, ha vinto, e pone il percussore
  in luogo suo, e  li fa dare il premio in mano a quello che siede, e se non s'appone
  perde il premio, quale consegna, al detto sedente, e ritorna al luogo di prima
  per continuare; fin tanto che s'appone, ed alla quarta volta si fa nuova elezzione,
  come sopra a Mona luna. Questo mi par di poter credere, che sia quel
  gioco, che i Greci chiamavano Collabismo riferito dal Buleng.\footnote{Jules-César Boulenger, Loudun 1558- Cahors 1628, storico e gesuita. } de lud. vet.\footnote{De Ludis Privatis ac Domesticis Veterum, Lyon 1627.} cap. 37.
  qual giuoco da quel \textit{Propheriza: quis te percussit?} detto per disprezzo da i Giudei a
  Giesù Cristo sig.\ nostro, si può argumentare, che fusse anco appresso a i Latini.
\item[ROSANA, e la Regina d'Oriente] Sono due Leggende, o Rappresentazioni notissime,
  per esser cantate giornalmente da ogni donnicciuola.
\item[BVRATTINI] Intende quei Figurini di legno, che son fatti muover da uno, che
  a tal effetto s'asconde in un castelletto di legna coperto di panno; e gli fa operare
  mettendo egli sopra alle punte delle dita, e ad un certo suo fischio gli fa parlare.
\item[ZANNI] Per Zanni, che s'intehde servo sciocco Lombardo, qui intende ogni
  sorta di Bagattellieri, che fanno il buffone per le piazze.
\item[FESTINI di giuoco, ec] Quando s'adunano in una casa più Dame, e Cavalieri
  per giuocare insieme, o per ballare nella prima parte della notte, dice fare un
  \textit{Festino}, o \textit{Veglia}. E se bene veglia strettamente presa, pare che significhi più trattenimento
  di ballo, che di giuoco, tuttavia la pigliamo per intendere ogni sorta
  di trattenimento, o di Giuoco, o di Ballo, o di qualsivoglia altra cosa, nella
  quale si spendano le prime hore della notte, dicendosi: \textit{Noi facemmo la veglia a
    studiare, a ballare, a cantare, ec}. Ma volendo pigliare queste due voci nel suo proprio
  significato; \textit{Festino}, S'intende adunanza di persone nobili, sia per ballare,
  o per giuocare in quelle hore della notte; e \textit{Veglia} s'intende d'ogni sorta di persone
  ordinarie; E si come s'avvilirebbe dicendo: \textit{Io fui alla veglia nel Palazzo
  del Principe} così pare, che si burlerebbe dicendo: \textit{Fui al festino in casa un Battilano},
  Quando si dice \textit{Festino pubblico}, o \textit{Veglia bandita} s'intende \textit{Festino}, o \textit{Veglia} a porta
  aperta, dove può andare ognuno. Vedi sotto: C. 9 stan. 51. e Cant. 10. stan. 28.
\item[NON era in gambe; ne in quattrini] Non si sentiva gagliardo da ballare, e non
haveva monete da poter giuocare.
\item[DA trinciarle] Intende da far capriole, cioè saltare. Vedi sotto C, 7, stan. 23.
\item[DA fare ite, e venite] Cioè giuocare. Quando si giuoca, e perdendo si paga
  la posta volta per volta, o si risquote quando ella si vince, diciamo \textit{fare ite, e venite},
  e s'intende pagare il denaro subito perduta la posta; e riceverlo nello stesso
  modo vincendo; ed è il contrario del detto \textit{Fare a tu me gli hai}; che significa giuocare
  in su la fede, o a credenza.
\item[MAZZOLINO] Ancor questo è trattenimento da Fanciulli, e si fa in tal guisa.
Più ragazzi si adunano insieme, e si piglino il nome d'un fiore per ciascuno,
e di questi fiori un di loro, che è il Giardiniere compone un mazzo, e poi dice:
Questo mazzo non sta bene per causa della Viola; e colui, che ha preso il nome
della Viola deve risponder subito: Dalla Viola non viene, ma sì ben dal Giglio, o
altro fiore, che a lui verrà nella mente; e se non risponde subito, o vero se nomina
un fiore che non sia in quel mazzo, perde un premio, il quale si dà al
Giardiniere. E così vanno seguitando fino a che il Giardinere habbia in mano
tanti premj da potere alla fine del giuoco distribuirne almeno uno per ciascuno di
quei ragazzi, che sono nel giuoco; ed il Giardiniere è sottoposto anch'egli alla
perdita del premio, perché se un fiore darà la colpa a lui, e che egli non risponda
subito, e nomini un Fiore, che non sia nel mazzo; perde come gli altri, e il
suo premio va dato in mano a colui, che l'ha fatto errare; ma come in deposito,
perché alla fine del Giuoco va poi con gli altri distribuito dal Giardiniero,
il quale non lo può però dare a se medesimo; E questi premj si domandano \textit{pegni},
e di questi intende il Poeta dove dice: \textit{Convien ch' il pegno subito depositi}.

Finito il Giuoco il Giardiniere distribuisce ripartitamente e pegni pigliandone
ancora per se. Tali pegni poi sono da coloro, che gli hanno dal Giardiniere havuti,
restituiti a i proprj padroni, i quali, se li rivogliono, devon fare una cosa secondo
il gusto di colui, al quale e toccato in sorte il detto pegno; E questo dicono
\textit{far la penitenza}, la quale se egli non fa, il pegno resta in mano a colui, al quale
è toccato, e però questi pegni devono esser di qualche valore, acciò che i padroni
habbian caro di riavergli. Alle volte fanno questo giuoco i Giovanetti di maggiore
età, e riducono questi pegni a moneta, quale depositano ogni volta, che
perdono in mano a un depositario, e se ne servono per far merende, ec, tal giuoco
è poco dissimile a quello, che facevano i Greci detto Basilinda riferito da Giulio
Polluce tab.\ 9.\ C.\ 7.\ e dove noi diciamo Giardiniere essi dicevano Re, come
facevano anche i Latini, e ciò si deduce da Hor. Ep. pr. lib. pr.
\begin{verse}
  \makebox[3em]{\dotfill} At pueri ludentes, Rex eris, aiunt,
  Si recte facies, hic murus aheneus esto, ec.
  Roscia, dic fodes, melior lex, an puerorum
  Naema? quae Regnum recte facientibus offert.
\end{verse}

Se bene potrebbe dirsi, che Orazio non intenda di questo giuoco particolarmente,
perché in tutti li giuochi Fanciulleschi tanto i Greci, che i Latini chiamavano
Re colui, che vinceva, ed asino quello che perdeva; ma perché nel giuoco presente
era fatto Giardiniere (o diciamolo Re) quello che in altri giuochi era rimasto
superiore a tutti, però non m'anlontano da interpretare Orazio, ed applicare
questo suo luogo al presente proposito, nel quale, se il Re errava diventava
l'asino, e Re si faceva colui, che havea fatto errare, o tenendosi il conto di
chi di loro haveva meno errato, quello alla fine era il Re, e quello che più volte
haveva errato era l'Afino, o Re Mida. Vedi il Meursio \textit{de Ludis veterum}. Gli
Spartani similmente per Legge di Licurgo, secondo che riferisce Plutarco nella
vita del medesimo, ai Ragazzi di più di sett'anni, proponevano come Principe
il più savio tra loro, che soprantendesse a' loro giuochi, e Fanciulleschi esercizzj.
\item[ALLE comare] Questo giuoco è trattenimenco di Fanciullette, e lo fanno così:
Mettono una di loro in un letto con un bamboccio fatto di cenci, e fingendo che
questa habbia partorito, le fanno ricever le visite da altre Fanciullette con far
quelle cirimonie, ed accompagnature, che si costumano in occasione di vere parturienti.

Tal giuoco era usato ancora dalle Fanciullette Greche secondo Giulio Pol.lib.9.c.7;
ma in vece d'una Parturiente fingevano una Sposa; e lo dicevano \textit{Phittamelia}.
Qual giuoco fanno pure ancora le nostre Fanciulline, e lo chiamano \textit{far' alle Zie}.
Non ha questo giuoco delle Comare, o Zie altro fine, che di passare il giorno in
quelle loro cirimonie, e ricevimenti, ne i quali alle volte si consuma quello, che
le Fanciullette hanno havuto per merendare.

\item[GLI spropositi] E lo stesso in sustanza, che quello del mazzolino, se non che
  dove in quello si finge un Giardiniere; in questo i Ragazzi s'adattano a qualsivoglia
  altra cosa, con pigliarsi quei nomi, che attengono a quella tal cosa; per
  esempio: Faranno il giuoco sopra il pane; il Maestro sarà il Fornaio, e questo farà
  quello che nel Mazzolino fa il Giardiniere; uno farà la farina, uno l'acqua, uno
  il forno, ed altre cose attenenti alla construttura, e perfezione del pane; Il
  Fornaio dirà: Questo pane non è buono per causa della Farina; quello che ha
  il nome della Farina, deve risponder subito: Dalla farina non viene, ma dall'acqua,
  o da altra cosa che gli venga in mente, attenente al pane, e che sia fra
  loro Ragazzi; e se non risponde presto, o non da la colpa a qualche cosa, il nome
  della quale non sia in quella adunanza, o non sia attenente al pane, perde, e
  deposita il pegno; e si fa nel resto per appunto come nel giuoco del Mazzolino:
  E questo giuoco universale è forse quello, che habbiamo detto sopra, che facevano
  i Greci detto Basilinda, E da noi si chiama \textit{il giuoco de gli Spropositi},  perché
  dovendo quei Ragazzi risponder presto, attribuiscono al pane cose spropositatissime,
  e che non hanno che far punto col pane, o sua bontà, oltre a non esser
  il nome di quella tal cosa in veruno di quei Ragazzi. E quello vuol dire \textit{Uscir di
  tema}.

  Habbiamo un'altro modo di far questo giuoco, ed è così: Mettonsi più persone
  a sedere in giro, e ciascuno dice al compagno in uno orecchio una parola, o
  due al più, e finito il giro, ciascuno ordinatamente dice forte quella parola, che
  gli e stata detta dal vicino, e volendone comporre il periodo si sentono gli Spropositi,
  che risultano da quelle parole; e si da la pena a colui, che ne è stata la
  cagione.
\item[CAPO a niscondere] Vno si mette col capo in grembo a un'altro, che gli tura
  gli occhi, ed un'altro, o più si nascondono, e nascosti danno cenno, e colui che
  haveva gli occhi serrati si rizza, e va cercando di coloro, che sono nascosti, e
  trovandone uno basta per liberarsi da tornare in grembo a colui, dove mette
  quello, che ha trovato, e questo perde il premio proposto, e il trovatore va a
  nascondersi; ma se non trova il nascosto in tante gite, o in tanto tempo, quanto
  sono convenuti, perde il premio, e ritorna a star con gli occhi chiusi come prima;
  e seguita così fino a quattro volte, perdendo quattro premj, come s'è detto
  sopra a \textit{Mona luna}, ed i premj poi si distribuiscono come si fa al giuoco del
  Mazzolino, E quello star con gli occhi serrati si dice star sotto, che i Greci in un simil
  giuoco dicevano \textit{catamyein}, Lat. \textit{connivere}. E colui che è stato sotto quattro
  volte, e non ha mai trovato il nascosto, e per consegucnza perduti i quattro premj,
  occupa il luogo di colui, che teneva sotto, e questo s'intruppa con gli altri Ragazzi,
  fra i quali si tira la sorte a chi dee star sotto, o nascondersi. E così seguitano
  tanto, che si riducano tutti liberi; perché quello che ha pagati li quattro
  premj nel modo suddetto, ed ha occupato il luogo di tenere gli altri sotto, come
  ne vien cavato nella maniera accennata, resta fuori del giuoco, del quale solo
  attende la fine per conseguire anch'egli la sua parte de i premj da distribuirsi. Era
  ancor questo giuoco appresso a i Greci, e lo chiamavano \textit{Apodidrascinda} secondo
  Giulio Polluce lib. 9. c. 7., ma diversificava alquanto; Ed in questo giuoco
  pure il vincente era detto il Re, ed il maggior perdente l'Asino. Vedi il Buleng.
  de lud. Graec. cap. 22. ed il Meursio in verbo \textit{Apodidrascinda}. Simile a questo
  era ancora il giuoco detto da' Greci \textit{Myinda}.

\item[OGNVNO a un mo non è composto] In questo proverbio sentenzioso habbiamo
  ancor noi come i Latini più modi di dire, come: \textit{Le nature son diverse}. \textit{Tanti
    huomini tante berrette}, o \textit{tanti cervelli}, \textit{Tutte non possono esser a un modo}, \textit{Chi la
    vuole a lesso, e chi a rosto}, e molti altri; e ne i Latini si trova. \textit{Quot homines tot
    sententiae}, \textit{Suus cuique mos}, \textit{Trahit sua quemque voluptas}. \textit{Non omnes eadem mirantur,
    amantque}, ed altri infiniti, e tutti con lo stesso significato.
\item[FAR le merenducce] I nostri Stovigliai in alcune Fiere, che si fanno in Firenze
  il giorno della festività di San Simone, ed in quello di S. Martino conducono
  gran quantità di stoviglie piccolissime, come piatti, tegami, pentole, ed ogni altra
  specie di arnesi,  vasellami da cucina, che da essi si fabbricano di terra. Di
  queste si provveggono li nostri Fanciulli per quanto vien loro permesso dalla loro
  borsa, e da queste vien poi loro l'occasione di far le \textit{Merenducce}, perché havendo
  altre masserizie adeguate, come tavole, sgabelli, bicchieri, salviette,  simili,
  imbandiscono una mensa, accordandosi più Fanciulletti, e Fanciulline a portare quello,
  che è dato loro per merenda, ed accomodando tutto in piccole particelle,
  le distribuiscono in quei piattellini, figurando di fare un Banchetto, e mettono
  a sedere a quella tavolina li loro Bambocci; E queste son da loro chiamate
  \textit{Merenducce}, delle quali parla il Poeta, e le quali erano usate ancora dalle Fanciulline
  antiche in occasione del suddetto appellato \textit{Phitrameliae}, come si cava
  dal Meursio, dal Soutero, e dal Bulengero.
\item[BAVAGLIO] Salvietta, o Tovagliolino da Bambini, che si lega al collo con
due cordelline, o nastri, detto così dalla bava, che sopra vi casca dalla bocca de
bambini; i Latini pure secondo l'Onomastico lo dicono \textit{pectorale salivarium}, e con
questi \textit{Bavagli} come lor proprj arnesi apparecchiano le loro piccole tavole quando
fanno le \textit{Merenducce}, e si mangiano quelle particelle distribuite in quei piattellini,
come s'è detto sopra. E di queste \textit{Merenducce} parla il Poeta.
\item[STACCIABBVRATTA] Due seggono incontro l'uno all'altro, e si pigliano
  per le mani, e tirandosi innanzi, e indietro; come si fa dello staccio abburattando
  la farina, vanno cantando una lor frottola, che dice.
  \begin{verse}
    Staccia abburatta
    Martin della gatta
    La gatta andò pel vino, ec.
  \end{verse}

E questo è trastullo usato dalle Balie per acquietare i Bambini di quella età, che
 appena si reggono in piedi.
\item[ALTALENA] Passatempo da Fanciulli; Legano due funi al palco, o vero a
due alberi, e le fanno calare a doppio fino presso a terra un braccio, e sopra di
esse funi accomodano un'asse, sopr'alla quale si pone uno, o più a sedere, e fatto dare
il moto a detta asse vanno cantando alcune canzoni con un'aria aggiustata
al tempo dell'ondeggiamento di quell'asse, e questa l' Æora de' Greci, dai Latini
detta \textit{Oscillatio}, ed altre, volte \textit{Petaurum pensile}, e noi la diciamo \textit{Altalena} dal
Latino \textit{Tollenon}, che vuol dir quella Macchina di legno, con,la quale si cava l'acqua
de i pozzi (come si vede in Plin. lib. 19, c. 4. \textit{Vel Tollenonum haustu rigandos})
da noi detta \textit{Mazzacavallo}. Vedi sotto C. 6. stan. 86. E questo perché facevano
l'Altalena, come la fanno talvolta anche li nostri Fanciulli con incrocicchiare
una trave sopr'all'altra, e ponendosi uno o più ragazzi per testata della trave,
che è di sopra, la fanno alzare, e abbassare a foggia di \textit{Mazzacavallo}. Di questa
parla il Bulenger, de lud. vet. c.~11. Questa \textit{Altalena}, in alcuni luoghi di Toscana
è detta \textit{biciancole}.
\item[BECCALAGLIO] E' un giuoco simile alla mosca cieca detta sopra. C. 1. stan.
40. ne vi è altra differenza, che dove in quello si da, con un panno avvolto, o altra
cosa simile, in questo si da con la mano piacevolmente una sola volta da colui, che
bendò gli occhi a quel, che sta sotto, ed il bendato in vece di dare,  affanna di
pigliare un di coloro, che in quella stanza sono del giuoco, e colui che resterà
preso, deve bendarsi in luogo del bendato, e perde il pegno, e premio, ed il primo
bendato resta libero, e s'intruppa fra quelli, che hanno a esser presi, e si fa
come sopra nel giuoco di Guancial d'oro. Si dice \textit{Beccalaglio} perché questo tale
bendato vien condotto in mezzo della stanza, o piazza, dove s'ha da fare il giuoco;
e colui che lo bendò, e che quivi l'ha condotto gli dice; \textit{Che sei tu venuto a
  fare in piazza?} Ed egli risponde; \textit{A beccar l'aglio}, E quello dandogli leggiermente
con le mani sur' una spalla soggiugne: \textit{O beccati codesto}. Dopo la qual funzione
il bendato s'affatica, di pigliar uno per metterlo in suo luogo. I Greci appellavano
questo giuoco \textit{Chytrinda} da pentola che in Greco, si dice \textit{Chytra}, e lo facevano
nella stessa maniera; ma in vece di bendare gli occhi, mettevano a colui, o
fingevasi, ch'egli tenesse colla sinistra una pentola in capo, e girandogli intorno
lo solleticavano, o percotevano; onde, se egli rivoltandosi, prendeva chi gli
tirava, il preso rimaneva in cambio suo a essere quel della pentola. I Latini lo
dicevono \textit{ludus ollarius}.

Simile a questo era un'altro giuoco usato dalle Ragazze Greche, detto \textit{Chelichelona},
nel quale, messa a sedere quella, a cui davano nome di Chelona, che vuol
dire Testuggine, le dicevano: \textit{Chelichelona quid facis in medio?} e quella rispondeva:
\textit{Lanam texo, \& filum milesium} con quel che segue riferito dal Buleng. de
lud. vet. cap. 41.

Nel giuoco poi della \textit{Chytrinda}, ovvero, \textit{ludus ollarius} dicevano: \textit{Quis ollam?}
e chi teneva la pentola rispondeva: \textit{Ego Midas}, e s'affannava non di pigliare un
di coloro, ma di toccarlo co i piedi, e quel tale così tocco perdeva, e si metteva
la pentola in capo; E perché (come s'è detto sopra) i Greci havevano per costume
di chiamare Re il vincitore, ed asino il perditore, però questo tale, che
havea la pentola in capo si appellava \textit{Mida}, cioè \textit{Re asino}, Vedi Giulio Polluce
lib. 9. c. 7. ed il Buleng. de Lud. Vet, c, 17.

\item[ANDAR a predellucce] Due si pigliano per i polsi d'ambedue le mani l'uno
con l'altro in croce, e formano come una seggiola, e un'altro vi siede sopra, e
questo si dice \textit{andar' a predellucce}. Da i Greci s'usava un giuoco detto \textit{In Cotyla},
ed era il portare uno in su le spalle, e reggerlo, tenendo le di lui ginocchia nelle
palme delle mani voltate dietro alla persona, e detto \textit{In Cotyla}, cioè \textit{nella
ciotola}, o cavo della mano. Ma questo credo che sia un'altro giuoco, che noi
diciamo \textit{a cavalluccio}, che vedremo sotto C. 3. stan. 30. tanto più che i Greci secondo
lo stesso Polluce chiamano questo giuoco detto \textit{In Cotyla}, per altro nome
\textit{Hippada} dal verbo \textit{Hippazin}, cavalcare. E questo se bene è giuoco, tuttavia è
specie di pena per quei, che portano per haver perduto ad altri de' suddetti giuochi.

\item[ACCVLATTARE] È passatempo da Ragazzi, ma è specie di pena, e di tormento
  dovuto a colui che è acculattato. Quattro ragazzi pigliano uno per le
  braccia, e per i piedi, e formandone un quadrato, lo sollevano, e gli fanno battere
  il culo in terra tante volte, quanto merita il suo delitto, o perdita, che ha
  fatto in altri giuochi, come sopra. E questo si dice \textit{acculattare}, che in altro
  significato vedemmo sopra C. 1. stan. 7. Gli Spagnuoli chiamano l'Acculattare
  \textit{mantear}, perché mettono colui che si ha da acculateare in una coperta, o mantello, e
  tenendola da quattro capi, lo sbalzano in alto, e lo fanno ricadere in essa, e noi
  lo diciamo \textit{dar la coperta}.

\item[Vi fu caglio per tutti] Vi fu da dar soddisfazione a tutti. Ognuno hebbe in che
  impiegarsi. Traslato da' Sarti, che dicono in questa roba ci è taglio per un'Abito,
  o per due, ec. per intendere, ci e tanta roba, che si può fare un'Abito, o due, ec.

\item[STAR in barba di Gatta] o \textit{di Micio}, come si disse sopra in questo C. stan. 28.
  annotazione alla voce \textit{sbigottito}, Pare che questo detto possa venire dall'antica
  superstizione degli Egizzj, i quali credendosi, che il Gatto fusse consegrato alla
  Dea Iside, che era la loro Deità maggiore, non solo nutrivano con grandissima
  cura, e splendidezza questo animale, ma secondo Pierio Valeriano reputavano
  degno di morte colui, che ne ammazzasse, o facesse loro oltraggio. E riferisce
  Alex.\ ab Alex.\ dier.\ Gen.\ lib.\ 3.\ cap.\ 7.\ e lib.\ 6.\ c.\ 14.\ che quando moriva un Gatto,
  i medesimi Egizzj per contrassegno di dolore si radevano le ciglia,e poi mettendo
  addosso al morto gatto sale, ed aromati, e coprendolo con un panno bianco lo
  seppellivano, facendoli talvolta sepolcri notabili, tanta era la stima che ne facevano.
\end{description}

\section{Stanza XXXXIX \& L}

\begin{ottave}
\flagverse{49}Mai fu tra lor fin qui nulla di guasto,\\
Se non che Florian volto ale cacce,\\
Havendone più volte tocco un tasto,\\
E sentendosi dar sempre cartacce,\\
Dispose al fin di non voler più pasto,\\
Ne curando lor preghi, ne minacce\\
Fece invitar da i soliti Bidelli\\
Per l'altro dì i Piacevoli, e i Piattelli.\\
\end{ottave}

\begin{ottave}
\flagverse{50}Bench'il Suocero allora, e la Consorte\\
Maledicesser questo suo motivo,\\
Dicendogli che la fuor delle porte\\
Un' Orco v'è sì perfido, e cattivo,\\
Che perseguita l'huomo infino a morte,\\
E che l'ingoierebbe vivo vivo;\\
Con genti, ed armi uscì su l'aurora\\
Gridando: Andianne, andianne, eccola fuora.
\end{ottave}

Non hebbero (come s'è detto) questi Sposi mai occasione d'addirarsi, se non
che Floriano inclinato alla caccia si risolvette andarvi a dispetto della Moglie,
e del Suocero.

\begin{description}
\item[NON fu nulla di guasto] Non furono tra loro mai rotture; cioè non s'adirarono
  mai; e, come si dice, non s'ingrossarono i sangui.
\item[HAVENDONE toccato un tasto] Havendo di ciò domandato alla sfuggita, o
  discorsone con brevità. Tratto da i tasti del Cimbalo, o vero Organo strumenti
  musicali.
\item[DAR cartacce] Non rispondere secondo il gusto di chi richiede; Traslato dal
  giuoco di minchiate, nel quale si dicono cartacce quelle che non contano, e sono
  di niun valore. Vedi sotto C, 8, stan. 61.
\item[DAR pasto] Trattenere uno con scuse, o chiacchiere. E il latino \textit{verba dare};
  \textit{spelactare}. E si dice così, perché il polmone degli animali (che da noi si dice pasto) stracca
  colui, che lo mangia, ma non lo sazia. Si dice anche dar pasto, quando uno, che
  fs giuocar bene a un tal giuoco, finge di saper poco, e si lascia vincer da principio,
  a fine d'indurre il semplice a far grosse poste per vincergli assai.
\item[BIDELLO] Donzello, o Servitore d'Università, o d'Accademia, come sarebbe
  quel Donzello, che serve allo Studio di Pisa, o ad altri simili. E questo
  nome di Bidello secondo l'Autore delle Notizie Ecclesiastiche è corrotto da \textit{Pedullus},
  perché questo Uffiziale, (dice egli) che nell'Accademie, e negli Studj
  pubblici haveva cura d'eseguire le commissioni appartenenti allo studio, soleva
  portare in mano un bastone chiamato \textit{Pedo}; Quantungue altri (soggiunge il medesimo)
  tirino la sua etimologia dalla parola Sassonica \textit{Bydell}, che vuol dire il
  Banditore.

  Ma io credo che il nome \textit{Bidello} sia tolto da \textit{Betulla}, che è quell'albero, del
  quale si facevano le verghe per i fasci, che anticamente portavano i Littori
  d'avanti a i Magistrati del popolo Romano, e che da questo portare i fasci di
  verghe di Betulla, sia poi venuto il nome di Bidello a tali serventi di Università,
  i quali fanno figura di Littori, e nello studio di Pisa portano ancora una grossa
  mazza d'argento (significante gli antichi fasci) quando vanno in funzioni pubbliche
  avanti al Collegio de i Dottori. Alex, ab Alex, dier. Gen, lib. 1. c. 17. in
  fine, dice così.
  \textit{Quodque fascibus, quos praeferebant Lictores, betullas virgas maxime commodas duxere,
    itaque ex illorum virgis tum proper candorem tum propter tenuitarem publicos
    fasces, qui magisiratibus praeirent, effecere}. E Plinio lib. 6. c. 18. \textit{Gaudet frigidis
    sorbus, \& magis Betulla; Gallica haec arbor, mirabilis candore atque tenuitate, terribilis
    Magistratuum virgis}. Lo stesso attesta Polid. Verg. lib. 4. c. 3.
\item[PIACEVOLI, e Piattelli] Sono in Firenze due conversazioni di cacciatori, le
  quali andando alle cacce gareggiano fra loro a chi faccia maggior preda, e quella
  che rimane superiore, tornando, suole entrare nella Città trionfante con fuochi,
  carri, ed altro; e l'una si dice la Compagnia de' \textit{Piacevoli}, e l'altra de' \textit{Piattelli};
  ciascuna ha la sua stanza entro alla quale s'adunano. gli Ufiziali, e Serenti,
  ed Altri; e questi son quelli de' quali dice il Poeta, e chiama i loro serventi
  Bidelli.
\item[VN'Orco] Questa è una bestia immaginaria inventata dalle Balie per far paura
  ai bambini, figurandola uno animale specie di Fata, nimico dei bambini cattivi,
  ed il Poeta, che non s'allontana mai dal genio puerile, mostra che il suocero
  Stordilano voleva indurre nel genero Floriano il timore per farlo astenere da
  andare a caccia, con dirgli che fuori della porta v'era l'Orco, che ingoiava gli
  huomini: Questo nome però viene dall'antica superstizione de i Gentili, i quali
  chiamavano Orco l'Inferno Virg. AEn. lib. 6. \textit{Primisque in faucibus orci}. Ed intendevano
  per Orco anche Plutone, quasi \textit{urgus, sive Uragus ab urgendo} perché egli
  sforza, e spinge tutti alla morte\footnote{Pluto sic dictus, non ab urgendo, ut quidam volunt, sed a Graeco $O\upsilon\rho\alpha\gamma o\varsigma$ dicitur, hoc est, qui in acie extremam agminis partem ducit. Unde non invenuste ad Ditem trausfertur, qui postremum humanae fabulae actum excipit.

Hofmann J. Lexicon universale. 1698. }; e perciò dalle madri, e nutrici per far paura
  alli lor bambini si dice che l'Orco porta via: il che pure vien da i Gentili, che
  pigliando Orco per la morte, lo chiamavano Inesorabile, e rapace. Orazio
  Ode 18. lib. 2,\begin{verse}
    Nulla certior tamen
    Rapacis Orci fine destinata.
  \end{verse}
\item[GRIDANDO andianne andianne ec] Così vanno gridando i cacciatori suddetti
  la mattina avanti giorno per svegliare i compagni. Lo stesso, che \textit{Alò Alò}; ovvero
  \textit{Alon} dal Franzese \textit{Allons}.
\end{description}

\section{stanza LI — LV}

\begin{ottave}
\flagverse{51}Senza veder ne anche un'animale\\
Frugò, bussò, girò più di tre miglia;\\
Pur vedde un tratto correr un Cignale\\
Feroce, grande, e grosso a meraviglia,\\
Ond'ei, che il dì dovea capitar male\\
Si mosse a seguitarlo a tutta briglia,\\
Non essendo informato ch'in quel Porco\\
Si trasformava quel ghiotton dell'Orco.
\end{ottave}

\begin{ottave}
\flagverse{52}Che a posta presa havea quella sembianza,\\
E gli passò fuggendo allor d'avanti\\
Per traviarlo solo con speranza\\
D'haver a far di lui più boccon santi;\\
Così guidollo fino alla sua stanza\\
Dov'ei pensò di porgli addosso i guanti;\\
Poi non gli parve tempo, perché i cani\\
Havrian più tosto lui mandato a brani.
\end{ottave}

\begin{ottave}
\flagverse{53}Però volendo andare in sul sicuro\\
Non a perdita più che manifesta,\\
Perché a roder toglieva un'osso duro\\
Mentre non lo chiappasse testa testa;\\
Gli sparì d'occhio, e fece un tempo scuro\\
Per incanto levar, vento, e tempesta,\\
E gragnuola sì grossa comparire,\\
Che havrebbe infranto non so che mi dire.
\end{ottave}

\begin{ottave}
\flagverse{54}Il cacciator, che quivi era in farsetto,\\
E dal sudore omai tutto una broda,\\
Havendo un vestituccio di dobretto,\\
Ed un cappel di brucioli alla moda,\\
Per non pigliare al vento un mal di petto,\\
O altro, perché il Prete non ne goda,\\
Non trovando altra casa in quel salvatico,\\
Che quella grotta, insaccavi da pratico.
\end{ottave}

\begin{ottave}
\flagverse{55}A tal gragnuola, a venti così fieri\\
C'ogni cosa mandavano in rovina,\\
Tal freddo fu che tutti quei quartieri\\
Se n'andanano in diaccio,  in gelatina,\\
Ed ei ch'era vestito di leggieri,\\
E mai meglio facea la furfantina,\\
Non più cercava capriolo, o damma,\\
Ma da far, s'ei poteva, un po di fiamma.
\end{ottave}

Floriano scorse molta campagna, e cercò buon pezzo, e non trovò mai nulla,
se non che pur vedde un grosso Cignale, al quale si messe dietro co i suoi cani,
non sapendo, che era l'Orco trasformatosi in quel cignale per pigliar Floriano;
dalla vista del quale sparì, e per via de' suoi incanti fece venire una gran
pioggia, e tempesta, la quale obbligò Floriano a ricovrarsi in una grotta, che era
quivi fra quelle macchie, nella quale entrato, si messe a cercare se trovava modo
di fare un po di fuoco.

\begin{description}
\item[FRVGÒ] Cioè cerco minutamente frugando per le siepi con i cani, e bussando
  con le pertiche per tutto.
\item[DOVEA capitar male] Doveva haver disgrazie. Doveva rovinare, E il Lat.
  \textit{Perdo, perire},
\item[A TUTTA briglia] A tutto corso senza punto fermarsi, come fa il cavallo
  quando se gli lascia liberamente la briglia. \textit{Laxatis habenis}.

\item[GHIOTTONE] Epiteto solito darsi a un huomo maligno, e di genio cattivo,
e suona quasi lo stesso, che Briccone, furbo, vizioso, scellerato.

\item[PIV boccon santi] Più buon bocconi. La voce santi in casi simili significa perfezione
  in generale. Vedi sotto C, 3, stan. 8.

\item[PORRE i guanti a dosso] Piglia guanti per mani, e vuol dire Pigliarlo. Habbiamo
  il verbo \textit{agguantare}, cioè pigliare. Guanto dal Germ. Hendt, mano.

\item[ANDARE in sul sicuro] Andar senza paura. Mettersi a fare un negozio con
  sicurezza di non esser'impedito, e che riesca secondo l'intento.

\item[TORRE a rodere un'osso duro] Pigliare a fare una cosa difficile.

\item[CHIAPPARE] Qui val per ritrovare, e sopra in questo C. stan. 41. per perquotere;
  ed il suo proprio significato è Pigliare; dal Lat, \textit{capere}.

\item[TESTA testa] Cioè a solo a solo. \textit{Remotis arbitris}, Diciamo anche a
  quattr'occhi.

\item[GRAGNVOLA] Grandine, che è gocciola d'acqua congelata nell'aria per
  forza di freddo, e di vento, e si fa di vapore freddo, e umido stropicciato nelle
  parti interiori del nugolo. \textit{La pioggia} nasce da vapori freddi, e umidi adunati
  ne i nugoli. \textit{La neve} è impressione generata di freddo, e d'umido; e questo freddo
  è minore di quello, col quale dalla pioggia vien generata la grandine, ed ha in
  se qualche parte di caldo. \textit{La rugiada} è generata di freddo, e di umido non rappreso,
  e questa congelandosi nell'aria diventa la \textit{brinata}. Ho voluto, benché fuor
  di proposito, notare l'origine de i sopraddetti accidenti dell'aria, perché da
  questa s'intendano i loro nomi; in qualche parte d'Italia per avventura differenti.

\item[HAVREBBE infranta non fo che mi dire] Havrebbe schiacciata, o diciamo anche
  ammaccata qualsivoglia cosa per dura che fusse; Non so immaginarmi, ne
  dire cosa tanto dura, che ella non l'havesse infranta. Questo termine \textit{non so che
    mi dire} usato nella forma, che si vede nel caso presente, significa quel che s'è detto;
  ma per altro l'usiamo anche per denotare di non havere, o saper trovar
  modo di rimediare a qualche accidente, per esempio: \textit{Io non so che mi dire, se il
    tale vuol far male i fatti suoi}.

\item[IN farsetto] Vestito leggiermente. Farsetto hoggi intendiamo ogni sorta d'abito
  leggieri, e disinvolto, che sopr'alla camicia si porta sotto gli altri abiti, come
  sarebbe camiciuola, o giubbone, ec.

\item[TVTTO una broda di sudore] Tutto molle dal sudore; Sudatissimo per la fatica
  del viaggio violento.

\item[DOBRETTO] Intendiamo una specie di tela di Francia fatta di lino, e bambagia
  (che è il cotone filato). Si dice anche \textit{Dobletto} da \textit{duplex}, perché nel tesserlo, è fatto
  di doppia orditura, e riempitura. Così \textit{dobbla} e \textit{dobbra} dissero gli antichi.

\item[BRUCIOLI] Quelle sottili strisce, che il Legnaiolo cava da qualsivoglia legno
  lavorandolo con la pialla, si dicono \textit{brucioli}, forse dalla similitudine de' brucioli,
  bachi; e da questi si dicono \textit{cappeli di brucioli} quelli, che son composti, ed intessuti
  di strisce d'un'erba particolare, nello stesso modo, che si fa con la paglia, alla
  similitudine, e larghezza della quale sono ridotte le dette strisce.

\item[ALLA moda] Cioè alla foggia che usa; la quale era nel tempo, che l'Autore
  compose la presente Opera, che i cappelli havevano piccola falda. sì che non
  tanto per esser di brucioli, quanto per esser piccolo, era poco atto a difendere
  dal acqua. Si dice \textit{alla moda} quasi \textit{all'usanza}, che \textit{è modo}, cioè adesso,Fr, alla moda.

\item[MAL di petto] Così chiamiamo volgarmente quell'infermità, che i Medici
  dicono Pleuritide.

\item[PERCHÉ il Prete non ne goda] Cioè per non morire, e così far che il Prete
  non goda il guadagno della cera del funerale.

\item[QVEI quartieri] Intendi per quelle campagne, per quei contorni. Che per altro
  noi Fiorentini per \textit{quartiere} intendiamo una delle quattro parti, nelle quali è
  divisa la nostra Città. E \textit{quartiere} in lingua militare significa Habitazione e dar
  \textit{quartiere al nimico} significa salvargli la vita, e farlo prigione.

\item[INSACCAVI da pratico] V'entra dentro come se egli, per esservi entrato altre
  volte, sapesse la strada, e vi fusse pratico. Se bene \textit{huomo pratico} usato nella
  maniera, che è qui, vuol dire huomo savio, e da saper pigliar compenso in ogni occasione.

\item[GELATINA] Vivanda nota fatta per lo più col brodo di carne di porco cotta
  in aceto, e poi congelato; Ma qui per \textit{Gelatina} intende che l'acqua s'andava
  congelando sopra il terreno, e fa \textit{Gelatina} sinonimo di \textit{Diaccio}, come fa D. inf. 32.

\item[FAR la Furfantina] Si trova una specie di Bianti, i quali per muover le persone
  pie a far loro elemosina, dopo haver bevuta buona quantità di generoso vino,
  ne i tempi più freddi si distendono mezzi ignudi nelle strade più frequentate, e
  tremando fingono di morirsi dal freddo, e questo lor tremare si dice \textit{far la Furfantina},
  cioè fare il giuoco che fanno questi furfanti, ch'è poi passato in dettato,
  che significa, e comunemente s'intende Tremare.

\item[MA meglio] Benissimo. Già mai si trovò chi facesse meglio. Quel \textit{ma} vuol
  dir mai; la figura apocope.

\item[DAMMA] È lo stesso, che Daino specie di capron salvatico. Lat. \textit{dama} D. Inf. 4.
\begin{verse}
  Sì sì starebbe un'cane infra due dame, ec.
\end{verse}
\end{description}

\section{Stanza LVI.}

\begin{ottave}
\flagverse{56}Trovò fucile, ed esca, e legni vari,\\
Ond'un buon fuoco in un cantone accese,\\
E in su due sassi posti per alari,\\
Sopr'un'altro sedendo i pié distese,\\
Così con tutti commodi a c\ellipsis{18pt} pari,\\
Dopo una lieta, il crogiolo si prese,\\
Essendosi a far quivi accomodato,\\
Mentre pioveva, come quei da Prato.
\end{ottave}

Floriano havendo trovato in quella grotta comodità d'accendere il Fuoco,
l'accese, e vi s'accomodò a scaldarsi, aspettando che intanto cessasse la pioggia.

\begin{description}
\item[FVCILE] Intendiamo quello strumento d'acciaio, del quale ci serviamo per
  battere nella pietra focaia ad effetto di cavarne il fuoco; detto \textit{Fucile} da fuoco,
  quasi focaio, o focile. Che però dissesi anche \textit{Focile}.

\item[ESCA] Quel fungo, o sia cuoio corto conciato col salnitro, che facilmente
  piglia fuoco, e serve per tener sopra alla pietra quando in essa si batte per trarne
  il fuoco, da i Latini detta \textit{fomes}. La qual voce, se ben per translato significa
  incitamento, o stimolo, che noi pure diciamo fomite, nondimeno era intesa per ogni
  cosa facile a pigliare quel fuoco, che Vergilio appella \textit{Semina flammae abstrusa
    in venis silicis}. Sì come noi, ancora diciamo \textit{Esca} ogni sorte di cibo d'animali,
  pure dal latino \textit{Esca}, che vuol dir cibo, ed incendiamo ancora questa materia,
  che è atta a pigliare subito il fuoco, quasi sia il cibo del fuoco; anzi a questa
  non diamo altro nome, che \textit{d'esca}, e dicendosi \textit{Esca} assolutamente, e senza
  aggiunta, s'intende solamente questo cuoio cotto, o fungo conciati con salnitro.

\item[ALARI] Sono due Ferri, o Sassi, che si tengono nei focolare, perché mantengano
  sospese le legne, acciò che più facilmente ardano. È voce rimastaci dal
  Latino \textit{lares}, la qual voce spesse volte era presa per fuoco, come si può dedurre
  da Ovid. 1. fast. 18.
\begin{verse}
  Omnis habet geminas hinc, atque hinc ianua frontes,
  E quibus haec Populum spectat, \& illa Larem.
\end{verse}

E da Colum. lib, 11, cap, 1. de Villico. \textit{Consuescat rusticus circa larem Domini,
focumque familiarem semper epulari}. Il Sipontino dice così: \textit{Lares Dij erant apud
Gentiles, \& colebantur domi, focusque illis sacer erat, unde vulgus focum focolare appellat
quasi laris focum}. Molti in vece di dire \textit{alari} dicon \textit{arali}, o sia corrottamente,
o pure, perché gli piglino da \textit{Ara}, intendendo strumenti da mettere in
su l'altare per sostenere le legne per il fuoco de i sacrifizzj, e così fanno che sia
ben detto tanto \textit{arali}, che \textit{alari}.

\item[A C. pari] Agiatamente si dice anche \textit{A pié pari}. Vedi sopra Cant. pr. stan.
82. Lasca Novella 4. lib. 2, \textit{Serviti delle buone vivande, che voi sapere bene acconce,
e stagionate se ne stessero a pié pari}. Si dice anche \textit{a gambe larghe}. Vedi. sotto C. 9.
stan. 32. Ed in molti altri modi, che tutti mostrano la spensierata agiatezza d'uno.

\item[DOP' una lieta] Dopo una fiamma.  Diciamo \textit{lieta} una fiamma chiara, senza
  fumo, e che presto passa detta \textit{lieta} da \textit{laetitia}, come anche \textit{baldoria}, da \textit{baldore} (cioè
  baldanza) voce antica. Gli Spagnuoli similmente dicono \textit{alegron}, un fuoco d'allegria.
  Vedi sopra C. 1, stan. 4. O forse si dice \textit{lieta} dalla parola \textit{lietamente}, che
  appresso ai nostri Contadini vuol dire \textit{prestamense}, cioè cosa, che passa prestamente.

\item[PIGLIARE il Crogiolo] \textit{Stagionarsi}. Quando son formati i bichieri, ed altri
  vasi di vetro, gli mettono così caldi in un fornelletto, che a tal fine è sopr'alla
  Fornace, da i Vetrai chiamato Camera, dove è un caldo moderato, e quivi gli
  lasciano stagionare, e freddare a poco a poco, conducendoli con un ferro alla
  bocca del detto Fornello per da basso, dove non si sente più caldo, il che da essi si
  dice \textit{dar la tempra}, \textit{temperare}, o \textit{dar il Crogiolo}, o \textit{Crogiolare}. E di qui parlando
  dell'huomo intendiamo \textit{pigliare il Crogiolo}, quando dopo una fiamma egli continova
  a stare attorno al fuoco, fino che sia tutto incenerito. E da questo verbo
  \textit{Crogiolare} piglia, o ha l'origine, il \textit{Crogiuolo}, che è quel vasetto di terra
  cotta, il quale serve per mettervi dentro a liquefare, o fondere i metalli nella Fornace,
  detto corrottamente \textit{Coreggiuolo}.

\item[FAR come quei da Prato] Proverbio vulgatissimo, che significa Lasciar piovere.
I Popoli della Città di Prato, che è suddita, e vicina a dieci miglia a Firenze,
nel tempo, che i Fiorentini si reggevano a Repubblica, domandarono licenza di
poter fare una Fiera il dì 8 di Settembre, (la qual Fiera si continova fino al presente
in detto giorno) e per tal' effetto. mandarono Ambasciadori alli SS.\ Priori
di libertà, da i quali fa loro conceduta la domandata licenza, con questo che
pagassero una certa somma di denaro. Accordato il negozio gli Ambasciadori si
partirono; Ma essendo nell'uscir del Palazzo, sovvenne loro, che se in tal giorno
fusse piovuto, non havrebbono potuto far la Fiera, e nondimeno sarebbe loro
convenuto pagare il danaro accordato; onde per assicurar quello punto tornarono
indietro, ed entrati di nuovo da i SS.\ Priori, uno di essi ambasciadori senz'altre
parole disse: Signori, se e' piovesse? Al che uno de' Signori subito rispose:
Lasciate piovere. E di qui nacque questo proverbio \textit{Far come quei da Prato}, che
significa Lasciar piovere.
\end{description}

\section{Stanza LVII — LVIII.}

\begin{ottave}
\flagverse{57}L'Orco fra tanto con mille atti, e scorci,\\
Affacciatosi all'uscio, ch'era aperto,\\
Pregò Florian con quel grugnin da Porci\\
Tutto quanto di fango ricoperto,\\
Che (perch'ella veniva giù con gli orci)\\
Ricever o volesse un po al coperto,\\
Ritrovandosi fuora scalzo, e ignudo\\
A sì gran pioggia, e a tempo così crudo
\end{ottave}

\begin{ottave}
\flagverse{58}Hebbe il giovane allora un gran contento\\
D'haver di nuovo quel bestion veduto,\\
E facendogli addossa assegnamento,\\
Quasi in un pugno già l'havesse havuto,\\
Rispose: Volentieri; entrate drento,\\
Venite, che voi siare il ben vennto,\\
Che dopo il fuggir voi l'umido, e il gielo,\\
Fate a me, ch'ero sol, servizio a Cielo.
\end{ottave}

Mentre Fioriano stava a scaldarsi; l'Orco s'affacciò alla bocca della grotta
senz'haver mutata la figura di Cignale, e pregò Florian, che lo lasciasse entrare;
Ei gli risponde, che entri allegramente, e che ne riceve servizio, perché
essendo solo, ha cara un poca di Compagnia.

Non si maravigli il lettore, che un Cignale parli; e si ricordi, che e una Novella
per i Fanciullini, e che queste cose seguivano.
\begin{verse}
 Al tempo, che volavano i pennati,
 Tutte le cose sapevan parlare;
\end{verse}
Secondo, che dice quel che descrive la guerra di Carnovale con Madonna Quaresima.
Apul. As.1. 2. \textit{Parietes locuturos, boves, \& id genus pecora dictura praesagium.}

\begin{description}
\item[GRVGNO] S'intende la faccia del Porco, da \textit{grunnitus}, che è lo stridere del
Porco. \textit{Grugnino} è detto per vezzi, ma qui è ironico, e per derisione \textit{Guardate
bella faccettina}, o \textit{bel grugnino}, o \textit{bel grugno}, quando vogliamo intendere una
brutta faccia. E si dice \textit{haver il grugno}, dell'huomo quando è in collera, donde ingrugnare
per entrar in collera. Vedi sotto C. 8. stan. 61. e \textit{sgrugnoni} si dicono le
pugna date nen viso.

\item[ELLA vien giù con gli orci] Cioè piove gagliardamente, quasi dica: Ogni gocciola
  è di tanta acqua; quanta ne cade a dar la volta a un'Orcio, che ne sia pieno.
  Si dice anche \textit{Ella viene a bigonce}, a \textit{catinelle}, ec, tutte iperboli per denotare, che
piova gagliardamente. Vedi sotto C. 10. stan. 20.

\item[FACENDOGLI addosso assegnamento] \makebox[1em]{} Disegnando quello, che voleva far di
  quasi fusse già in suo potere, e dominio, come esprime il Poeta medesimo dicendo:
  \textit{Quasi in un pugno già l'havesse havuto}.
\item[FAR servizio a Cielo] Far un servizio, o favore accettissimo, o grandissimo.
\end{description}
\section{Stanza LIX - LXIII.}

\begin{ottave}
\flagverse{59}Si eh! (soggiunse l'Orco) fate motto!\\
Voler ch'io entri dove son due cani!\\
Credi tu pur ch'io sia così merlotto!\\
Se non gli cansi ci verrò domani.\\
S'altro, dice il garzon, non c'è di rotto\\
Due picche te gli vo' legar lontani,\\
E preso allora il suo guinzaglio in mano\\
Legò in un canto Tebero, e Giordano.
\end{ottave}

\begin{ottave}
\flagverse{60}Poi disse: Hor via venite alla sicura.\\
Rispose l'Orco: Io non verrò ne anco,\\
Guarda la gamba! perch' io ho paura\\
Di quella striscia, ch'io ti veggo al fianco,\\
Allor Florian cavossi la cintura,\\
Ed impiattò la spada sott' un banco,\\
Disse l'Orco: (vedutala riporre)\\
Io ti ringrazierei; ma non occorre.
\end{ottave}

\begin{ottave}
\flagverse{61}E lasciata la forma di quel verro,\\
Presa l'antica,e mostruosa faccia,\\
Con due catene saltò là di ferro,\\
E lo legò pel colle, e per le braccia,\\
Dicendo: Cacciatar tu hai pres'erro,\\
Perché credendo di far preda in caccia,\\
All fin non hai fatt'altro ch'una vescia,\\
Ment'il tutto è seguito alla rovescia.
\end{ottave}

\begin{ottave}
\flagverse{62}Rimasto ci sei tu, come tu vedi\\
Senza bisogno haver di testimoni,\\
E perché con levrieri, e cani, e spiedi\\
Far me volevi in pezzi, ed in bocconi;\\
Così perch'ella vadia pe' suoi piedi\\
Farassi a te, ne leva più ne pani,\\
Acciò che, procurando l'altrui danno,\\
Per te ritrovi il male, ed il malanno.
\end{ottave}

\begin{ottave}
\flagverse{63}Ed io c'hebbi mai sempre un tale scopo\\
D'accarezzar ognun, benché nimico,\\
Come la Gatta, quando ha preso il topo,\\
Che, se ben' è tra lor quell' odio antico,\\
Scherzando con esso alquanto, e poco dopo\\
Te lo sgranocchia come un beccafico,\\
Così perché più a filo tu mi metta\\
Voglio far' io, e poi darti la stretta.
\end{ottave}

L'Orco alla cortese offerta risponde, che ha paura de' cani, e della spada; e
Floriano lega quelli in un canto, e ripon questa sotto un banco; Allora l'Orco
si scuopre, ed entrato nella caverna prese Floriano, ed incatenollo.
\begin{description}
\item[SÌ eh?] E un termine, del quale ci serviamo per dimostrare che habbiamo, conosciuto
  l'inganno, o cattivo trattamento, che alcuno ci habbia fatto, o habbia in animo
  di farci, quasi dica: \textit{Così eh vorresti ch'io facessi?} o vero \textit{Così mi
  tratti eh?}

\item[FATE motto] Proferito col primo 'o' stretto. Vuol dire ascoltate, sentite.
  Fate motto a me; ed usato nella forma che è nel presente luogo, ha forza d'esclamazione,
  e vale per un certo modo di domandar consiglio, quando ci detta una
  cosa, che sia impossibile a farsi, o a credersi, quasi chiamiamo altra gente, che ci
  consigli se questa tal cosa sia da farsi, o da credersi; e che senta lo sproposito che
  ci è stato detto. Dirò per esempio; \textit{Costui dice che ha trent'anni, e Sono più di cinquanta
  ch' ei nacque}; Fate motto! Cioè udite sproposito; O vero giudicate, se ciò può essere.

\item[SIA così merlotto] Cioè sia così semplice, così minchione, così privo di senno.

\item[CI verrò domani] Detto ironico, che significa Non ci verro mai. Questo \textit{Domani}
  è il Domani eterno di quell'Oste, che haveva scritto sopr'alla sua bottega
  \textit{Doman si dà a credenza, e hoggi no}. Che l'hoggi era sempre, e il Domani havea
  sempre a venire. Berni \textit{A rivederci alle Calende Greche}, preso da Svet. in Aug. c. 87.

\item[DUE picche] Detto indeterminato, se ben pare determinato, e significa molto
  lontani, e non per appunto la lunghezza di due picche ma forse assai più, e forse
  assai meno.

\item[GVINZAGLIO] È quella corda, o striscia di quoio, con che si tengono i levrieri
  a lassa; e da molti è preso per ogni sorte di legame, derivandolo dal verbo
  latino \textit{vincio}, come \textit{vincastro}, \textit{vinciglia}, ec. ma strettamente guinzaglio,
  s'intende solo la corda, o quoio, col qual si tiene il levriero alla lassa, sebene
  da qualcuno è inteso ancora per quel legame, col quale s'accoppiano insieme
  i bracchi, o altri cani da caccia, Lat. \textit{copula}.

\item[GVARDA la gamba!] Il Cielo me ne liberi, Il Cielo mi guardi, che io sia per
  far questo. In Firenze nella Corte della Mercanzia, che è il Tribunale dove si
  fanno l'esecuzioni Civili, sono alcuni Donzelli, i quali si chiamano Toccatori.
  Questi dopo che in una causa si son fatti tutti gli atti, e si vuol venire all' esecuzione
  personale, vanno ad avvisare il debitore, che se egli non pagherà in termine
  di ventiquattro hore sara condotto in carcere; e senza tale atto, che si dice
  Toccare, o fare il tocco, non si si può con Cittadini Fiorentini venire a detta esecuzione
  personale. Tali Toccatori anticamente per esser conosciuti portavano
  una calza d'un colore, ed una d'un'altro, onde nel passare che facevano fra le
  Botteghe, e per i luoghi più frequentati i ragazzi gridavano: \textit{Guarda la gamba};
  affin che chi era in grado d'esser toccato si potesse fuggire, e guardarsi, non potendo
  i Toccatori far tale azione ne i luoghi immuni; e si dice Toccare perché
  non serve, che costoro avvisino con la voce il detto debitore, ma devono formalmente
  toccarlo con la mano. E da questo è venuto il modo di dire.
  \textit{Guarda la gamba}; che significa mi guarderò, o fuggirò di far tal cosa. Il Lalli
  nell' En. trav. lib, pr. stan. 67. si serve di questo detto nel medesimo proposito.
  \begin{verse}
    Venere allor rispose; Honor Celeste
    Guarda la gamba! usurpare io non voglio.
  \end{verse}

\item[IMPIATTARE] Nascondere, e si dice di materiali; e non pare che
  suonerebbe bene il dire Impiattare la verità, la virtù, ec. Vedi sopra C, 1. stan.
  75. Il Poeta se ne serve sotto C. 19. stan. 5. parlando dell'Aurora; ma la considera
  come donna, e corporea, come si considera il Sole, la Luna, e le Stelle,
  delle quali si dice \textit{Impiattarsi}, o \textit{rimpiattarsi} dietro a i nugoli, o dietro le
  montagne. Petr. Canz. 9. \textit{E lei non stringi che s'appiatta, e fugge}.

\item[BANCO] Vuol dir la Tavola, sopra alla quale si posano le vivande per mangiare:
  se bene \textit{Banco} ha molti altri significati.

\item[IO ti ringrazierei, ma non occorre] Cirimonia che si usa con chi ci habbia fatto
  un favore a rovescio, o vero ce l'habbia fatto quando non occorreva, o quando
  havevamo gia fatto da per noi quel che speravamo da lui; o che di sua cortesia ci
  faccia un favore del quale non havevamo bisogno; ed è lo stesso che dire \textit{Io t'ho
  negli orecchi}, \textit{Io t'ho stoppato}, e simili.

\item[VERRO] Porco maschio senza castrare. Dal Latino \textit{verres}.

\item[TV hai preso erro] Tu hai fatto errore. È detto hoggi poco usato fuor che nel contado.

\item[FARE una veglia] Non conchiudere. Non adempire il suo intento, come
fanno coloro, che andando a tirare con l'archibuso mettono nella canna minor
quantità di polvere di quella richiesta, e scaricando poi non colgono, e fanno
uno scoppio così debole, che a pena si sente, e tale scoppio di dice \textit{vescia}. Si dice
ancora \textit{vescia} una specie di fungo; E vescia dicono le donne un racconto de fatti
d'altri donde \textit{vesciona}, e \textit{vesciaia} una donna, che ridice tutto quello che sente
discorrere.

\item[NE leva più, ne poni] Non aggiungere, e non levare. Cioè sarai trattato
  ugualmente, o per appunto come volevi trattar me \textit{Nec addas, ned adimas}. E
  Dante Parad. C. 30.
  \begin{verse}
    Presso, e lontano lì ne pon, ne leva.
  \end{verse}

\item[IL male, ed il malanno] Il male, e peggio ch' il male.

\item[SGRANOCCHIA] Mangia con l'ossa, e con ogni cosa; ed il Poeta medesimo
  lo dichiara, dicendo: come un beccafico, i quali uccelletti da i più si mangiano
  senza buttar via l'ossa. E \textit{sgranocchiare} se ben s'usa alle volte ne i casi come il
  presente, non lo trovo usato se non per esprimere il romore, che fa coi denti in
  romper quell'ossa colui che le mangia, il qual romore è simile a quello che fa il
  ranocchio quando canta.

\item[HEBBI un certo scopo] Hebbi un certo fine, un certo genio, un certo riguardo.
  La voce \textit{scopo} vien dal Greco \textit{scopos}, che tanto appresso a Greci quanto ai
  Latini, ed appresso a noi vuol dir Berzaglio, e per metafora significa quel fine,
  al quale tende, ed è diretta la nostra mente nelle nostre operazioni, per lo più
  in bene; che non stimerei si potesse dire senza riprensione. \textit{Scopo di rubare}. Si
  dice anche \textit{haver mira}, il qual termine è per avventura più generico, dicendosi
  \textit{haver mira di far bene}, ed \textit{haver mira di far male}.

\item[METTERE a filo] Far venir gran voglia, Traslato dal coltello, ed altri ferri
  taglienti, i quali quando sono ben' arruotati (che si dice \textit{messi in filo}, o \textit{affilati})
  tagliano meglio.

\item[DAR la Stretta] Vuol dire opprimere uno. Ma qui è preso nel suo vero significato
  di stringere, ed intende stringere co i denti, cioè mangiare.
\end{description}

\section{Stanza LXIV.}

\begin{ottave}
\flagverse{64}Così spogliollo tutto ignudo nato,\\
E veduto ch'egli era una segrenna,\\
Idest asciutto, e ben condizionato,\\
Snello, lesto, e leggier com' una penna,\\
Lo racchiuse, e lo tenne soggiornato,\\
Perch' ei facesse un po miglior cotenna,\\
Però che a guisa poi di mettiloro\\
Voleva dar di Zanna al suo lavoro.
\end{ottave}

L'Orco spogliò Floriano per mangiarselo, e vedutolo così magro risolvé di
non toccarlo, ma lasciarlo stare tanto che ingrassasse, e poi mangiarselo.

\begin{description}
\item[IGNVDO nato] Cioè ignudo, come quando ei nacque. Diciamo così per intender
  uno, che non habbia in dosso ne pure una minima parte di vestimento, ed
  ha la stessa forza che dire \textit{Ignudo ignudo}, che per la ragione della replica, vuol
  dire Ignudissimo, o Affatto ignudo.

\item[SEGRENNA] Quella voce, usata per lo più dalle donnicciuole, vale per
  esprimere una persona magra, sparuta, e di non buon colore, che i Latini, tolto
  dai Greco, dicono \textit{Monogrammus}; ed il Poeta medesimo la dichiara dicendo:
  \textit{Idest asciutto}, che \textit{huomo asciutto} intendiamo huomo magro; ond'io mi credo che
  \textit{segrenna} venga da \textit{segaligno} che vuol dire Animale magro e di temperamento non
  atto a ingrassare. Diciamo ancora \textit{mummia}, che sono quei Cadaveri secchi nel
  mare d'Etiopia, o ne i sepolcri dell'Egitto: come vedremo sotto C. 6. stan. 52.
  per intendere Huomo soverchiamente magro. Diciamo \textit{Segrenna} a una donna
  magra, dispettosa, maligna, incontentabile, e che non approva, ne loda: mai
  l'operazione d'altrui.

\item[BEN condizionato] Questo termine, se ben pare riempitura del verso, o (come
  diciamo) borra, non è così, ma è pure che quando si vuole intender un magro,
  habbiamo questo dettato vulgatissimo \textit{Asciutto, e ben condizionato}, tolto forse
  da quello che son soliti dire i mercanti, \textit{la tal mercanzia ci è comparsa asciutta,
    e ben condizionata}, per avvisare il Corrispondente della diligenza del Latore, o
  Condotttiero.

\item[SNELLO, lesto, leggier come una penna] Queste tre voci nel presente luego Sono sinonimi
  significando, ed esprimendo tutte la poca carne che haveva addosso Floriano,
  e che era al maggior segno magro. E la voce \textit{snello} forse origine dal
  Tedesco \textit{Sknel}, che vuol dir Veloce.

\item[LO tenne soggiornato] Lo trattava bene di mangiare. Gli faceva buone spese.
  Che \textit{soggiornare uno} vuol dire Spender il tempo in ben custodire, governare, e
  ristorare uno con quello che occorra, e s'usa questo termine per lo più, trattandosi
  di bestiami, e perciò appropriatamente detto in questo luogo, perché, se
  ben Floriano era huomo, era nondimeno trattato dall'Orco come beitia da ingrassare.

\item[FACESSE miglior cotenna] Ingrassasse. Per intendere uno assai grasso diciamo:
  \textit{Egli ha buona cotenna}; traslato da i porci, la pelle de i quali si dice propriamente
  \textit{cotenna}, che dell'huomo si dice \textit{cotenna} solamente la pelle del capo, o per disprezzo,
  e per intendere un' huomo Zotico, che si dice \textit{huomo di grossa cotenna},
  o \textit{Cotennone}, o \textit{Coticone},

\item[A GVISA di mettiloro, Volea dar di zanna al suo lavoro] Coloro che indorano i
  legnami si chiamano \textit{Metti l'oro}, ed in una parola sola \textit{Mettilori}. Questi per
  brunire, o dar il lustro a i loro lavori si servono de i denti più lunghi, o diciamo
  maestre di cane, di lupo, o d'altro animale simile, (i quali denti chiamiamo \textit{zanne},
  o \textit{sanne} come vedremo sotto C. 7. stan. 54.) e tal lavorare dicono \textit{zannare}, o
  \textit{dar di zanna}. Ma qui \textit{dar di zanna} s'intende il naturale adoperar de i denti, che è
  mangiare; e scherzando con l'equivoco dice che l'Orco voleva \textit{dar di zanna al
    suo lavoro}, cioè mangiarsi Floriano, che era il suo lavoro, che egli havea fatto
  pigliandolo, ed ingrassandolo.
\end{description}

\section{Stanza LXV. \& LXVI.}

\begin{ottave}
\flagverse{65}Amadigi c' andava per diporto\\
Due volte il giorno almeno a rivedere\\
La fonte, e la mortella, che nell'orto\\
Lasciò Florian per tante sue preghiere;\\
Trovato il cesto spelacchiato, e smorto,\\
E l'acque basse puzzolenti, e nere,\\
Qui (dice) Fratel mio noi siam sul curro\\
D'andar a far un ballo in campo azzurro.
\end{ottave}

\begin{ottave}
\flagverse{66}E piangendo diceva; O Tato mio,\\
Se tu muori, che ver sarà pur troppo,\\
S'ha a dire anche di me, telo dich'io,\\
Itibus, come disse P\ellipsis{18pt} Pioppo,\\
Così, senza dir pure al Padre addio,\\
Monta sour' un cavallo, e di galoppo\\
Vscì d' Ugnano molto ben' armato,\\
E seco un cane alano havea fatato.
\end{ottave}

In questo tempo Amadigi s'accorse dalla fonte, e dalla mortella, che Floriano
era in pericolo, e perciò montato a cavallo bene armato, e con un grosso
cane incantato, andò a cercar di lui.
\begin{description}
\item[SPELACCHIATO] Pelato in qua, e in la, cioè parte delle foglie cascate, e
  parte no. Spelacchiato s'intende un'huomo, che stia male a sanità, ed a roba, e
  sia mal vestito per la sua povertà.
\item[SMORTO] S'intende che non ha il suo natural colore buono.
\item[SIAM sul curro] Siamo in procinto; siamo all' ordine; siamo vicini, \textit{Curro}
  son pezzi di quali si metton sotto alle pietre, o ad altre cose gravi per
  facilitargli il moto quando si strascicano, dai Latini detti \textit{Palangae}.
\item[FAR un ballo in campo azzurro] Vuol dire Esser' impiceato; perché \textit{campo
  azzurro} s'intende il campo, che fa l'aria, il quale è azzurro, e colui, che è impiccato
  movendo le gambe, pare che balli in aria, Per maggiore intelligenza la
  voce \textit{campo} pittorescamente parlando, vuol dire quel luogo, che avanza in un
  quadro fuori delle figure, ed altro che vi sia dipinto, come si dice una insegna
  entrovi un lione in campo azzurro. Ed i medesimi Pittori ne cavano il verbo
  \textit{campire}, ché vuol dire Dare il colore, del quale ha da essere il campo.
\item[TATO] Vuol di Fratello. È parola usata dalle Balie per insegnar parlare a i
  Bambini, come Babbo in vece di Padre, Mamma, Bombo, e simili, che per esser
  parole labiali tornano più facili a proferirsi. Furono usate anche dai Latini
  come si vede in Marz. lib, 1. 95.
  \begin{verse}
    Mammas, atque tatas habet Aphra, sed ipsa tatarum
    Dici, \& mammarum maxima mamma potest.
  \end{verse}

Vedi sotto C. 3. stan. 13., e C. 4. stan. 5.

\item[TE lo dich'io] Vale per Te lo giuro; Ti assicuro. Vedi Oraz.\ lib.\ 2.\ Ode 17.
dove parlando con Mecenate infermo, dice:
\begin{verse}
  Ab te meae si partem animae rapit
  Maturior vis, quid moror altera?
\end{verse}

Con quel che segue simile al presente lamento, che fa Amadigi per il Fratello,
che Orazio fa per Mecenate.

\item[ITIBUS come disse P\ellipsis{18pt} Pioppo]\footnote{``Prete Pioppo''} Significa s'ha dire anche di me: gli è morto.
  Questo P\ellipsis{18pt} Pioppo era uno, che havea poca amicizia con Prisciano\footnote{Priscianus Caesariensis, Cesarea 512 - dopo il 527, Grammatico - linguista.}, e
  non ostante sempre slatinava, e fra l'altre quando voleva dire il tale è morto diceva
  Itibus, e intendeva Egli è ito. E da questo suo detto diciamo \textit{Come disse
    P\ellipsis{18pt} Pioppo}, E s'intende il tale è morto.

\item[DIR' addio] Intendiamo quel saluto, che si fa nel pigliar congedo, o licenziarsi
  da uno, ed è lo stesso, che il Latino \textit{Vale}, usato da noi ancora come dicemmo
  sopra, e vedremo sotto C. 6 stan. 18.

\item[GALOPPO] Corso di cavallo, da i Latini detto \textit{cursus gradarius}, che è in
  mezzo tra il trottare, e il correre. Forse meglio \textit{gualoppo} secondo Dante Inf.
  Cant. 22.
  \begin{verse}
    \makebox[10em]{\dotfill} di rintoppo
    A gli altri disse a lui, se tu ti cali
    Io non ti verrò dietro di gualoppo.
  \end{verse}
\item[CANE Allano] Cane grosso per caccia da Cignali, e simili animali feroci, ed è
  maggiore, più fiero, e più gagliardo del Mastino.
\end{description}

\section{Stanza LXVII \& LXVIII.}

\begin{ottave}
\flagverse{67}E cavalcando con la guida, e scorta\\
Del suo fedele, ed incantato Alano,\\
Ch'innanzi gli facea per la più corta\\
La strada per lo monte, e per lo piano;\\
A Campi giunse, dove in su la porta\\
la morte si  leggea di Floriano,\\
Che perché fu creduta da ognuno,\\
Era la Corte, e tutto Campi a bruno.
\end{ottave}

\begin{ottave}
\flagverse{68}L'apparir d'Amadigi agli abitanti\\
Raddolcì l'agro de i lor mesti visi,\\
Che per la somiglianza a tutti quanti\\
Parve il lor Re creduto a' Campi Elisi,\\
Perciò per buscar mance, e paraguanti\\
Andaron molti a darne al Re gli avvis,\\
Altri alla figlia, ed ambi a questi tali\\
Perciò promesser mille bei regali.
\end{ottave}

Amadigi arrivò a Campi, dove dal bruno, che vedde addosso a gli abitatori
conobbe, che era morto il lor Principe; subito che costoro veddero Amadigi,
credettero ch'i fusse Floriano, e perciò molti corsero a darne avviso al Re, e
a Doralice.

\begin{description}
\item[ERA la Corte, e tutto Campi a bruno] Cioè i Cortigiani, e gli abitanti di Cam-
i erano velliti di nero in: segno di mestizia, per la morte del Re Floriano. Petr. Canz. 5.
\begin{verse}
  E vedrai nella morte de' Mariti
  Tutte vestite a brun le donne Perse
\end{verse}

Da alcuni si dice \textit{vestire a lutto}, o \textit{a scorruccio}. Ma credo che essi habbiano
accattate queste voci da i moderni Romani.

\item[AGRO dei lor mesti visi] Viso agro vuol dir Malinconico; e si dice \textit{agro} perché
  uno, che habbia havuto qualche disgusto; suol mostrarlo nella faccia con increspar
  la fronte, e fare altri gesti appunto come fa uno, che mangi cose aspre,
  acide, o agre. E però dice \textit{Raddolcì l'agro dei lor mesti visi}, che significa di
  melancolici, gli fece ritornare allegri.

\item[CREDUTO a i Campi Elisi] Creduto nell'altro mondo; creduto morto, che
i Campi Elisi dalla superstiziosa Gentilità erano creduti il Paradiso. Vedi sotto
C. 6. stan. 32. '

\item[PARAGYANTO] Mancia, o regalo. \textit{Paraguanto}, \textit{dono}, \textit{regale}, \textit{mancia}
  appresso di noi si possono dir sinonimi; E se bene molti vogliono che \textit{mancia},
  e \textit{paraguanto} si dica quello, che dal Superiore si da all'inferiore; e \textit{dono} e \textit{regalo} si
  dica quello, che dall'inferiore si da al superiore (che in questo caso non si direbbe mancia)
  o dall'uguale, all'uguale, nondimeno nel buon parlar familiare si piglia
  uno per l'altro, ne s'osserva tanta strettezza, ed il nostro Poeta pure si
  vede nel presente luogo, che non osserva questa distinzione come poco, o punto
  necessaria.
\end{description}
\section{Stanza LXIX.}

\begin{ottave}
\flagverse{69}Doralice brittande a tai novelle\\
A rinfronzirsi andossene allo specchio,\\
Si messe il grembinl bianco e le pianelle\\
Il vezzo al collo, e i ciondoli all'orecchio,\\
E non potendo più nella pelle\\
Saltò fuor di palazzo innanzi al vecchio,\\
Ed incontro correndo al suo cognato,\\
Ecco Florian (dicea) risuscitato.
\end{ottave}

Doralice sentita questa nuova si raffazzonò, e subito corse incontro al suo cognato
Amadigi, credendolo Floriano suo marito.

\begin{description}
\item[BRILLANDO]. Giubbilando. \textit{Brillo} si dice uno che sia allegro per haver beuuto
  molto vino. Vedi sotto C. 6. stan. 35. ed è il primo grado di briaco dicendosi in
  augumento \textit{Brillo}, \textit{cotto}, \textit{briaco}, \textit{spolpato}, Molti vogliono, che questa voce \textit{brillare}
  venga da \textit{birillo} specie di gioia, e che brillare significhi scintillando tremolare,
  appunto come fa il \textit{birillo}, e come fanno coloro, che sono sommamente allegri,
  ©che habbiano soverchiamente bevuto.

\item[RINFRONZIRSI] Raffazzonarsi, abbellirsi, aggiustarsi la persona tolto dal
  Latino \textit{refrondescere}, che vuol dir quando gli alberi si vestono di nuove frondi,
  le quali nell'antico Fior. si dicevano fronze. Terenz. in Heaut.
  \begin{verse}
    \makebox[4em]{\dotfill} Et nosti mores mulierum;
    Dum moliuntur, \& comuntur, annus est.
  \end{verse}
  Cioè si rinfronziscono (dice l'espositore Landino) s'accomodano, ed acconciano
  la testa.

\item[CIONDOLI all'orecchio] Orecchini. Quelle gioie, che le donne portano pendenti
  all'orecchio, Latino \textit{Inaures}. Da noi chiamati pendenti, e per scherzo
ciondoli.

\item[VEZZO] Quell'ornamento di gioie, che le Donne portano al collo.

\item[PIANELLE] Specie di scarpa, che cuopre solamente la parte dinanzi del piede,
  da i Latini dette \textit{sandalia}, E con dette gioie adornandola, mostra il Poeta quale
  possa essere una Regina di Campi, che non eccede il lusso d'una pulita contadina
  de i Contorni di Firenze.

\item[NON può star nella pelle] Non può aspettare, perché l'allegrezza le ha cagionata
  una inquietudine tale, quale vogliono havere tutti coloro, che dovendo conseguir
  qualcosa di lor gusto, ogni hora d'indugio stimano mille. A questo
  si può applicare quell' \textit{In fermento totus est} de i Latini, che pare che esprima
  quella inquietudine, che suol cagionare l'ira; Lasca Novella 5. \textit{Sì che per la
    passione, e per la rabbia non poteva star nelle cuoia}.

\item[COGNATO] | Latini per cognazione intendevano ogni sorta di parentela. Ma
  noi per \textit{cognato} intendiamo un Fratello di nostra moglie, o un marito d'una
  sorella di nostra moglie, o un marito di nostra Sorella, e nello stesso modo respettivè
  il Fratello del marito, si dice cognato, come intende nel presente luogo.

\item[INNANZI al vecchio] Cioè prima che uscisse di casa il Re suo padre, intendendosi
  comunemente Padre quando in questi termini si dice il vecchio, ancor che
  talvolta il Padre sia giovane.
\end{description}

\section{Stanza LXX — LXXIV.}

\begin{ottave}
\flagverse{70}Noi vi facevam morto; o giudicate,\\
Se la carota c'era stata fitta!\\
Pur noi ci rallegriam, che voi tornate\\
A consolar la vostra gent'afflitta,\\
Domandar non occorre come state,\\
Perché v' havete buona soprascritta,\\
E siate grasso, e tondo com'un porco\\
Per le carezze fattevi dall'Orco.
\end{ottave}

\begin{ottave}
\flagverse{71}M'immagino così perch' io non v'ero:\\
Tu sai com' ell' andò, che fusti in caso,\\
So ben, che mi dirai, che non fu vero\\
Ma la bugia ti corre su pel naso,\\
Hor basta. Tu ritorni sano, e intero\\
(C'a pezzi tu dovevi esser rimaso)\\
Per la Dio grazia, e sua particolare,\\
Perché tel' ha voluta risparmiare.
\end{ottave}

\begin{ottave}
\flagverse{72}Dunque s'ei fa così gli è necessario,\\
Ch'ei non sia là quel furbo ch'un lo tiene,\\
Anzi tutto il revescio, ed il contrario\\
Mentr'egli tratta i forestier si bene.\\
Ed io, che già havea sul calendario,\\
Gli voglio in quanto a me tutto il mio bene,\\
Perch'ei non t'ingoiò; Se ben da un lato\\
Ti stava bene, havendolo cercato.
\end{ottave}

\begin{ottave}
\flagverse{73}Così nel mezzo a tutta la pancaccia,\\
Ch'è quivi corsa, e forma un giro tondo,\\
La sua caponeria gli butta in faccia,\\
E quel ch'ei ne cavò po poi in quel fondo\\
Già che (dicea) con l'andar' a caccia\\
Ai dispetto di tutto quanto il mondo\\
Cavasti, senza far alcun guadagno\\
Due occhi a te, per trarne uno al compagno.
\end{ottave}

\begin{ottave}
\flagverse{74}Mio padre te lo disse fuor de denti,\\
Ed io pur te lo dissi a buona cera\\
Non una volta, ma diciotto, o venti\\
Che l'Orco ti faria quatche billera;\\
Ma tu volesti fare a gli scredenti,\\
Perché te ne struggei come la cera,\\
E quasi un rischio tal fusse una lappola\\
Volesti andarvi, e desti nella trappola.
\end{ottave}

In queste cinque ottave mostra, che Doralice ingannata dalla somiglianza,
che haveva Amadigi con Floriano, gli fa un discorso di congratulazione mescolata
con rimproveri, col quale il Poeta esprime assai bene il costume delle nostre
Femmine in simili casi; tacendo che dal principio del discorso, che è la congratulazione,
lo tratti del Voi, e quando viene a' rimproveri lo tratti del Tu.
\begin{description}
\item[SE La carota c'era stata fitta] \makebox[1em]{} Ficcar carote vuol dire quand'uno inventando
  qualche novella, o trovato, lo racconta poi per non suo, acciò che più agevolmente
  gli sia creduto; sì che Doralice vuol dire; guardate s'ella c'era stata data
  a credere. Vedi sotto Can. 6. stan. 67. e 68. Mattio Franzesi nel Capitolo sopr'alla
  Corte dice:
  \begin{verse}
    Chiama piantar carote il popolaccio
    Quel che diciamo: Mostrar nero per bianco
    Per distrigarsi da qualunque impaccio
  \end{verse}

  E per tutto il medesimo Capitolo discorrendo sopra questo detto, mostra che
  habbiamo anche il verbo \textit{Carotare}, e \textit{Carotiere}, quello che ficca carote. Il Lalli
  En. Tr. lib. 2. stan. 2.
  \begin{verse}
    Egli che ben conobbe al primo tratto
    Ch'era in un campo da piantar carote.
  \end{verse}

  Si dice \textit{Piantar carote}, perché questa pianta fa grossa radice, e cresce assai nei
  terreni dolci, e teneri, ed uno facile a credere si dice \textit{Homo dolce, e tenero}.

\item[VOI havete buona soprascritta] La faccia suol esser dimostratrice delle passioni
  interne, e però dicendosi \textit{haver buona soprascritta} s'intende haver buona sanità, come
  dichiara il Poeta medesimo dicendo; \textit{Non occorre domandarvi come voi state, perché
  si conosce dalla buona soprascritta}, cioè la sembianza, la buona cera, ed aria
  del volto ci dice, che vai state bene. E così la voce \textit{soprascritta}, che vuol dire
  Inscrizione, che si fa alle lettere, ci serve per intender quanto sopra s'è detto.

\item[LA bugia vi corre su pel naso] Tu dai colore. Tu ti muti di colore in viso, perché
  tu hai detto una falsità, \textit{Tui oculi declarant}, Lo Scoliaste di Teocrito spiegando
  quei versi dell'Iditio 12. che in Latino furono così tradotti: \textit{Verum ego te
  laudans, formose, haud mentiar umquam, Nec tenui gravis innascetur pustula nari};
  dice così. Vuol dire, che nel lodarti, io non mentirò, non mi nascerà sopra
  al naso la bugia; poiché alcuni sogliono chiamare certe bollicine bianche, che
  vengono su pel naso, bugie: e colui che le aveva, era notato, come bugiardo.
  Fin qui lo Scoliaste.

\item[RISPARMIARE] O \textit{rispiarmare}. Vale per perdonare. Qui s'intende l'Orco
  che non ha voluto far male alcuno.
\item[HAVER uno sul calendario] Havere a noia, o vero odiar' uno.
\item[QUANTO a me gli vo tutto il mio bene] Per quanto s'aspetta a me gli porto
  tutto quell'affetto, che si può portare; l'amo di tutto cuore.
\item[TI stava bene] E' lo stesso che Ti stava il dovere. Tornava bene, che l'Orco
 t'havesse ingoiato, perché t'haverebbe fatto quello che tu meritavi.
\item[PANCACCIA] Così si chiama da noi quel luogo dove si ragunano i novellisti
  per darsi le nuove l'un l'altro, ed ha questo nome di Pancaccia, perché nel tempo
  di state questi tali si radunavano già per sentire il fresco vicino alla Chiesa
  Cattedrale, sedendo sopra a un muricciuolo coperto di tavoloni, o panconi, e
  da questi prese il nome di Pancaccia. E da questa \textit{pancaccia}, \textit{Pancaccieri}, e \textit{Pancacciai}
  intendiamo quei perdigiorni, che stanno oziofamente ragionando de i fatti
  d'altri, ed in questo senso è preso nel presente luogo, che dicendo \textit{quei della pancaccia},
  intende una quantità di questi Crocchioni. Vedi sotto C. 6. stan. 69. Canti
  Carnascialeschi, \textit{Chi vuol udir bugie, o novellacce Venga ascolar costoro; che
  si stan tutto il dì su le pancacce}.

\item[GLI butta in faccia la sua caponeria] Gli rimprovera la sua ostinazione.

\item[QVEL ch' ei ne cavò po poi in quel fondo] Quel ch'ei guadagnò, ed acquistò alla
  fine delle fini, o in ultimo degli ultimi. Tanto servirebbe dir \textit{po poi} senz'aggiugnervi
  \textit{in quel fondo}, ma così è il nostro costume in simili casi per dar maggior
  emfasi, quasi dica una fine più la delle fini, Vedi sotto C. 8. stan. 51.

\item[CAVAR due occhi a te per trarne uno al compagno] Detto vulgatissimo, che ci
  serve per esprimere \textit{Far a se molto male, per farne pochissimo al nimico}.

\item[FVOR de' denti] Apertamente; chiaramente è il Lat. \textit{Eloqui}, ed è il contrario
  di parlar fra denti, o a mezza bocca, che significa non si lasciare intendere, forse
  e il \textit{Mussitare} de i Latini.

\item[A BVONA cera] Con allegra faccia; cioè non sopraffatto da collera, o altra
  passione, ma con animo riposato; diciamo anche \textit{sul sodo}, \textit{sul serio} tolto lat
  Lat. \textit{Serio admonere}. Il Lalli Eo. Te. C. 4. stan. 103.
  \begin{verse}
    Prega, scongiura, e dille a buona cera.
  \end{verse}

\item[BILLERA] Burla nociva,o se non cattiva del tutto, almeno che non piace;
  voce corrotta da \textit{Villera} voce antica che vuol dir Villania.

\item[TE NE struggei come la cera] Il verbo struggersi, che vuol dine Liquefarsi, serve
  a noi per farsi intendere d'uno che ardentemente desideri qualcosa. Il Lalli
  En. Tr. C. 4. stan. 109. disse.
  \begin{verse}
    Che se ne strugge come le candele.
  \end{verse}

\item[LAPPOLA] Cosa da non stimarsi. L'erba da nostri contadini chiamata \textit{Lappola}
  fa un seme pieno d'acute spine, ma fragili; E però dicendosi: \textit{non lo stimo una
    Lappola}, s'intende non lo stimo punto, e s'usa per lo più trattandosi di bravura,
  e valore, alludendo a quell'armatura di spine, che ha la Lappola, le quali se
  ben son molte, ed acute, non hanno con tutto ciò forza d'offendere, per esser
  fragilissime.

\item[DESTI nella Trappola] V'incappasti, Vi rimanesti preso. \textit{In laqueum incidisti}.
  \textit{Trappola} intendiamo ogni sorte d'artifizio, che si trova per pigliare animali
  tanto di terra, quanto d'aria, e d'acqua, donde \textit{Trappolare} val per Ingannare.
  Ma \textit{Trappola} strettamente presa s'intende un'artifizio per pigliare i topi,
  ed una specie di rete da pescare ha il solo nome di \textit{Trappola}.

  Si dice \textit{Trappola da quattrini}, per intendere Invenzioni per fare spendere.
\end{description}

\section{Stanza LXXV — LXXIX}

\begin{ottave}
\flagverse{75}Amadigi alla donna mai rispose,\\
E fece il sordo ad ogni suo quesito,\\
Ma si ben' attingea da queste cose\\
Quanto a Florian potea esser seguito,\\
E venne immaginandosi e s'appose,\\
Ch'ella fusse sua Moglie, ei suo Marito,\\
E ch'egli essendo tutto lui maniato\\
Fusse pel suo Fratel da ognun cambiato.
\end{ottave}

\begin{ottave}
\flagverse{76}Ma perch' ei non credea veder mai l'hora\\
D' haver il suo Fratello a salvamento,\\
Dà un ganghero a tutti, e torna fuora\\
Dietro al suo can veloce come il vento;\\
Ne era un trar di mano andato ancora\\
A caccia all'Orco ch' ei vi dette drento\\
Come il Fratel vedendo un bel cignale,\\
Ma non fu quanto lui dolce di fale.
\end{ottave}

\begin{ottave}
\flagverse{77}Che seguitollo anch'ei per quelle strade\\
Dond'ei conduce l'huomo alla sua tana,\\
Ove mentre diluvia, e dal Ciel cade\\
E broda, e ceci, il Cristianello intana.\\
Ed egli tanto poi lo persuade\\
Ch'ei lega i cani, e posa durlindana,\\
Havendo havuto innanzi la lezione,\\
Si stette sempre mai sodo al macchione.
\end{ottave}

\begin{ottave}
\flagverse{78}E quando l'Orco poi venne anc'a lui\\
A dar parole con quei tempi strani,\\
Ed all'uscio facea Pin da Montui\\
Affin che l'arme e i cani egli allontani\\
Ei disse: Su piccin piglia colui,\\
E chiappata la spada con due mani\\
Si lanciò fuora, e quivi a più non posso\\
Gli cominciò a menar le man pel dosso.
\end{ottave}

\begin{ottave}
\flagverse{79}E mentre ch'or di punta, ed hor di taglio \\
Di gran finestre fa, di lunghe strisce\\
Più prefto che non va strale a berzaglio\\
Il can s'avventa anch' egli, e ribadisce.\\
Tal che tutto forato come un vaglio\\
Il pover'Orco al fin cade, e basisce,\\
E lì tra quelle rupi, e quelle macchie\\
Rimase a far banchetto alle Cornacchie.
\end{ottave}

Amadigi argumentò dal discorso di Doralice, che ella fusse Moglie di Floriano,
e compreso quanto poteva esser' avvenuto al medesimo; e però senza dar altra
risposta dette addietro, ed uscito di Campi, fu dal Cane guidato alla tana
dell'Orco, il quale fu da lui con aiuto del suo cane, ammazzato.
\begin{description}
\item[MAI] Questo avverbio che significa In alcun tempo serve anche per negativa,
  come è nel presente luogo, e come l'usò più volte il Boccaccio ed in specie Nov.
  73. \textit{Mai frate il Diavol ti ci reca ec.} E Nov. 54. \textit{Che mai ad animo riposato si sarebbe
  potuto ritrovare}, e Nov. 77. \textit{Mai di ciò che hora mi parli dubitai}, Matteo Villani
  lib, 8. cap. 39. \textit{I Perugini mai si vollero dichiarare}, ed in molti altri luoghi del
  Boccaccio, del Passavanti\footnote{Jacopo Passavanti, Firenze, 1302 circa – Firenze, 15 giugno 1357, scrittore, architetto e religioso.}, e d'altri Scrittori del buon secolo si trova usato per negativa.
  Ho voluto dir ciò in questo luogo per toccare la difesa dell'Autore dalla
  critica datagli d'haver usato questa voce \textit{Mai} per negativa senza l'aggiunta della
  particella \textit{ne}, o \textit{non}, e senza correlazione alla negativa anteposta nel medesimo
  periodo, e che tanto vale il dire \textit{Io non farò mai questo}, quanto il dire \textit{Io mai
    farò questo}, E mi rimetto all'uso, ed al \textit{TORTO, E DIRITTO} del P. Bartoli, per
  la difesa di questa opinione.
\item[FECE il sordo] Finse di non sentire.
\item[ATTINGEA da queste cose] IL verbo \textit{attingere} o \textit{attignere}, che è il Latino
  \textit{attingere} per arrivare a un luogo, o a un fine; \textit{Metam attingere}: da noi è preso
  ed usato come il verbo \textit{haurio}, che vuol dir Cavar l'acqua da i pozzi, che noi diciamo
  attignere, ed in significato di \textit{Comprendere}, \textit{vedere}, \textit{udire},
  \textit{oculis \& auribus haudire}. E nel significato di \textit{Comprendere} è preso nel presente luogo.
\item[S'APPOSE] Verbo neutro che val per indovinare: Ed attivo vuol dire Dar
  la colpa a uno. \textit{Io m'apposi di chi haveva fatto il male, e però l'apposi a lui}. Io
  m'indovinai chi fusse stato quello che havea fatto il male, e però ne diedi la
  colpa a lui.

\item[TVTTO lui maniato] Come lui per appunto: Similissimo a lui: \textit{Fatto a capello},
  che vedemmo sopra in questo C. stan. 19. Lasca Nov. 7. dice: \textit{Il qual fantoccio
  vestito de' panni del Pedagogo, tutto maniato parea lui}. Io credo che sia parola
  corrotta da \textit{miniato} cioè diligentemente dipinto, o forse corrottamente derivato
  dai Latino barbaro \textit{Emanatus}, tanto simile a lui, che pare \textit{emanatus ab illo}.

\item[NON credea di veder mai l'hora] Amadigi havea così gran desiderio di vedere
  il suo Fratello libero, che dubitava non fusse per arrivar mai quell'hora, ed ogni
  momento, gli pareva un'anno.
\item[DÀ un ganghero] Dà volta addietro. Ganghero diciamo uno strumento per
  uso d'affibbiar le vesti, fatto di filo di ferro, o d'altro metallo, il quale è fatto
  in forma d'uncino, e da quella rivolta, che egli fa, \textit{dare il ganghero} intendiamo
  tornar indietro. \textit{Retrorsum vela dare}. Dare il ganghero, diciamo quando la lepre
  fuggendo avanti al cane, torna indietro, e lascia correr il cane, che portato
  dalla velocità non si può ritenere, e voltarsi subito come fa essa, che in tanto piglia
  campo in maniera ch'ella scampa, dal che diciamo \textit{Far lepre vecchia} per intender
  \textit{tornar indietro}. Vedi sotto C. 10. stan. 23.

\item[NON fu si dolce di sale] Non fu sì credulo: Sì minchione: Sì sciocco. Una
  vivanda poco salata si dice \textit{dolce di sale}, cioè sciocca. Donde esser senza sale, o
  non haver sale in zucca vuol dire Huomo sciocco, senza giudizio, senza cervello.
  Sale chiamiamo l'arguzie, e detti ingegnosi. Vedi sotto C. 8. stan. 26. Diciamo
  \textit{il tale è dolce}, e senza l'aggiunta \textit{di sale} intendiamo è corrivo, credulo
  minchione, e senza giudizio; e per coprire più questo detto, usano molti dire
  \textit{Lupinaio} (che vuol dir colui che vendendo per Firenze Lupini va gridando \textit{dolci
    dolci}) per intendere \textit{Costui è dolce}. Qui dunque vuol dire, che Amadigi non fu
  corrivo quanto era stato il Fratello a credere all'Orco. Bocc. Gior. 4. n. 2, \textit{Madonna
  Zucca al vento, la quale era anzi che nò un poco dolce di sale}, Lasca Nov. 2.
  \textit{E perché egli era nato in Domenica, non sendo la gabella del sale aperta, tenne sempre
  molto bene del dolce}.

\item[TANA] Caverna, grotta, buca. Donde \textit{intanare}, entrar nella \textit{tana}.

\item[BRODA, e ceci] Intendi acqua, e gragnuola. Fu un ragazzo ghiotto delle
  civaie, per il quale suo padre (per mortificare questa sua gola) ordinò, che nella
  sua scodella non si mettesse altro, che il puro brodo de' ceci, o d'altre civaie
  respettivamente, onde il povero ragazzo vedendo gli altri con le scodelle piene
  di legumi si disperava. Ed essendosene andato un giorno in camera mentre pioveva
  se ne stava alla finestra gridando \textit{acqua, e gragnuola}, e questo per la rabbia,
  che haveva, che si stagionassero i legumi per gli altri, e non per lui. Sentì il
  padre questo suo gridare, che gli disse: perché preghi il Cielo a mandar la grandine,
  cosa tanto nociva? L'astuto ragazzo per scampar la furia subito rispose:
  Padre mio io non ho mai desiderato, o pregato male per nessuno, e se io pregavo
  che insieme con l'acqua venisse anche della grandine, ho voluto intendere, che
  il Cielo vi mettesse una volta in testa di farmi dare con tanta broda una volta
  anche de' ceci, che di questi intendevo quando dicevo gragnuola. Il Padre rise
  dell'astuzia, e dette ordine, che per l'avvenire fusse trattato, come gli altri.
  E da questo intendiamo acqua e gragnuola, quando diciamo broda, e ceci.

\item[CRISTIANELLO] E' detto d'avvilimento, e significa Huomo dappoco, o di
  poca fortuna, o di piccola figura; che i Latini dicono \textit{homuncio}, e noi talvolta
  in questo senso diciamo \textit{Homicciuolo}.

\item[DURLINDANA] Intende la spada,e piglia questa denominazione dalla famosa
  spada d'Orlando Paladino, la quale da i Poeti hebbe il nome di \textit{Durlindana},
  o \textit{Durindana}.

\item[HAVENDO havuto innanzi la lezione] Essendo stato prima informato; avvisato,
  instruito: Cioè havendo compreso dal discorso di Daralice, che questo era
  quell'Orco, che ingannava.

\item[STAR sodo al Macchione] Intendiamo non condescendere alle richieste, o non
  si lasciar lusingare dall'esortazioni di alcuno. Questo detto viene da quegli uccelletti,
  che stanno per le macchie, dove si tendono le ragne, i quali, per essere
  stati altre volte molestati, hanno imparato, che quello scacciargli col battere la
  macchia era di lor poco danno stando fermi, però non si muovono a ogni romore,
  e questi si dicono \textit{star sodi al Macchione}, Di tali uccelli si dice anche
  \textit{accivettati}, Vedi sotto C. 9, stan. 22.

\item[FACEA Pin da Montui] Cioè facea capolino, che vuol dir quel che accennammo
  sopra C. 1, stan. 7. Questo detto viene da una canzonetta, o villanella,
  che dice.
  \begin{verse}
    Pin da Montui, Fa capolino
    Dreto è Menghino, E Mon con lui, ec.
  \end{verse}

  Plauto disse \textit{Ex insidijs clanculum aucupari}.

\item[SU piccino] È modo di incitare il cane contro a uno, È l'\textit{irritare}, o \textit{immittere}
  de i Latini, che noi diciamo anche \textit{ammettere}. Vedi sotto C.~11. stan.~29. si dice
  anche \textit{aissare} verbo originato da quel suono, che fa la voce dicendosi: \textit{su su}; O
  dalla parola \textit{iza} voce antica, che vuol dire Ira, dalla quale habbiamo il verbo
  \textit{aizzare}, o \textit{adizzare}, o \textit{aissare}, Dan, Inf. C.27.
  \begin{verse}
    Dicendo, issa ten va: più non t'aizzo.
  \end{verse}

\item[A PIÙ non posso] Con ogni maggior potere; Quasi dica con animo di seguitare
  a far quella tal cosa fino ache non sara stanco, e non possa più.

\item[MENAR le man pel dosso] Adoperar le mani nella persona d'uno, cioè Perquoterlo.
  La voce \textit{dosso} dal Latino \textit{dorsum}, da noi s'intende per tutto il torso
  dell'huomo, parendo che s'eccettuino da molti il capo, le braccia, e le gambe.
  Lasca lib. 1. Nov. 7. \textit{Non contento di ricercargli col bastone le braccia, e le gambe,
  volle ancora con esso ritrovargli tutto il dosso}.

\item[GRAN finestre, e lunghe strisce] Gran ferite di punta e di taglio \textit{Punctim, \&
  caesim}, disse Vegezio. Dice strisce per la similitudine che ha una lunga ferita di
  taglio con la striscia, e lo fa per esprimere che eran ben lunghe, come dice \textit{finestre}
  quelle di punta perché s'intenda, che eran larghe.

\item[AVVENTARSI] Spingersi, gettarsi, o andar velocemente, o con impeto
  alla volta d'uno, che i Latini dicono \textit{irruere}.

\item[RIBADIRE] Ribattere. Quando si mette un chiodo dentro a una tavola, e che la
  punta di esso chiodo passa dall'altra parte, la detta punta si piega, e si riconficca
  perché il chiodo faccia l'effetto d'una legatura; e per far questo, uno batte in
  su la punta del chiodo, e l'altro tiene a riscontro in sul capo del chiodo un ferro;
  e questo si dice \textit{ribadire}; e però perquotendo Amadigi da una parte, e il cane
  mordendo dall'altra l'Autore per esprimer questo atto si serve del verbo \textit{ribadire}
  usato da molti ed in questi termini, ed anche per replicare.

\item[FORATO come un vaglio] Havevano fatto nella persona dell'Orco più buchi,
  e tagli che non ha un vaglio, strumento col quale si separa il grano dall' immondizie,
  detto dal Latino \textit{Vannus}, e talvolta Crivello dal Latino \textit{Cribrum}, e
  \textit{Cribellum}, voce usata dall'Agricoltore Palladio. Questa comparazione era usata anche
  da i Latini trovandosi in Plauto \textit{Carnificum cribrum}, parlando di un servo che
  era stato mal concio dalle bastonate.

\item[BASISCE] Muore. Questo verbo ha forse l'origine dalla Greca voce \textit{Basis},
  che vuol dire \textit{incessus}, e che intendiamo \textit{il tale se n'andò}, per il tale mori, che diciamo
  \textit{basì}: vedi l'Ottava 82. seguente, e da questo verbo deriva la voce \textit{basto}, che
  vuol dir huomo senza sentimento, e quasi morto. Messer Gio: della Casa nel
  Capitolo del Martello d'Amore dice.
  \begin{verse}
    Perché ti guardi torto la Signora;
    Parti haver le budella in un catino,
    E doventi bafito allora allora.
  \end{verse}

Vedi sotto C. 6. stan. 97.
\end{description}

\section{Stanza LXXX — LXXXII.}

\begin{ottave}
\flagverse{80}Amadigi dipoi fece pulito,\\
Perché trovato havendo il suo Fratello\\
Con una barba lunga da Romito,\\
E più lordo, e più unto d'un panello,\\
Lavatolo, e rimessogli il vestito,\\
Ch' era ancor quivi tutto in un fardello,\\
Lo ricondusse a Campi, ove la Moglie\\
Di lui già pregna, appunto avea le doglie.
\end{ottave}

\begin{ottave}
\flagverse{81}Corse la Levatrice, ed in effetto\\
Fra mille hoimè, se' soldi, e doglien hora,\\
Partorigli una bella piscialletto\\
Che fusti tu, poi detta Celidora,\\
E maritata al Re, come s'è detto,\\
Di Malmantil del qual tu sei Signora;\\
Ne sei, e ne sarai, io lo raffibbio,\\
Se ben non puoi per hor dir come il nibbio.
\end{ottave}

\begin{ottave}
\flagverse{82}Ma presto come lui potrai dir mio.\\
Hor senti pur: Basito Perlone\\
Anco Amadigi subito tuo Zio\\
Venne a tor donna, e n'hebbe un bel garzone,\\
Che Baldo fu chiamato, e quel son' io,\\
Che poi cresciuto detto son Baldone.\\
Hor eccoti dal primo al terzo grado\\
Narrato tutto il nostro parentado.
\end{ottave}

Amadigi trovato il Fratello Floriano lo rivestì e lo ricondusse a Campi dove
Doralice partorì Celidora; e d'Amadigi nacque Baldone. E con terminare il
racconto, termina il Poeta il secondo Cantare.
\begin{description}
\item[FECE pulito] Fece il negozio aggiustatamente, e come andava fatto.

\item[BARBA da romito] Barba lunga, e incolta, che tale per lo più suol esser la
barba de i Romiti.

\item[LORDO] Sudicio schifo. Dal latino \textit{Luridus}, che vuol dir Livido, quasi \textit{per
  lorum cussum, \& lividum factum}. E questo epiteto s'adatta non solamente all'huomo,
  ma ancora ad ogni materiale, o strumento, sopra il quale sia schifezza.

\item[PANELLO] Così chiamiamo un viluppo di cenci intinti nell'olio, sego, o
  altra materia oleacea, e bituminosa il quale serve per abbruciare in far luminarie
  in occasione di pubbliche feste, ed allegrezze in luoghi eminenti, e dominati
  da i venti, a' quali questi resistono.  Dal Greco Panos, che val lo stesso. Varchi
  stor. lib.~11. \textit{Si fecero per tutto feste, ma la sera non s'arsero panelli per difetto d'olio}.

\item[LEVATRICE]. Raccoglitrice. Quella che raccoglie, e leva la Creatura dalla parturiente
  da i Latina detta \textit{obstetrix}, ed in alcuni luoghi detta Mammana.

\item[HOIMÉ] Voce, che esprime afflizione d'animo, e di corpo, che i Latini
  dicevano \textit{hei mihi}, e noi forse l'habbiamo dal Greco \textit{hoi moi}. E quell'aggiunta
  \textit{Sei Soldi} e \textit{doglien' hora} è posta per scherzo, e per burlare chi talvolta si duole, o si
  rammarica, o fa lezzj senza cagione, o per dolori leggieri, che noi diciamo,
  \textit{fare il monello}, e non è riempitura intentata dal Poeta, ma è pur così in uso, dicendosi
  a questo tale: O pover' huomo! \textit{Aimé! sei soldi, e dogliene hora}; e si nomina
  una somma di monete per haver occasione di dire \textit{dogliene}, che è il verbo
  \textit{dare}, ed in questa occasione si dice, perché ha similitudine con la voce \textit{doglia}.

\item[PISCIALLETTO] Una bambina. Quando una donne partorisce una Femmina,
  niuna di quelle donne che sono attorno alla parturiente le vuol dar la
  nuova, che ella sia femmina, ma perché pure al fine ella lo deve sapere, per non
  profferire la parola femmina dicono: Una \textit{Piscialletto}, \textit{Una come me}, e simili. E
  da questo noi habbiamo \textit{far' un bambina}, che vuol dir Fare un'errore.

\item[LO rafibbio] Lo replico.

\item[NON puoi dir come il nibbio] Cioè non puoi dir Mio. Il Nibbio uccello rapace
  non fa altro canto, ne si sente da lui altra voce, che un certo fischio, o strido,
  che par che suoni \textit{mio mio}, e da questo per avventura i Latini lo dicon \textit{Miluus},
  g1i Spagnuoli \textit{Milano}, e i Francesi \textit{Milan}; E noi da questa sua voce volendo
  esprimere, che una cosa sia veramente mia, dichiamo: \textit{Posso dire come il nibbio}, cioè Mio;
  l'autore lo dichiara nel primo verso dell'ottava seguente dicendo: \textit{Ma presto
  come lui potrai dir mio}.

\item[BASITO] Vedi l'ottava 79, antecedente.

\item[ZIO] Fratello del padre, o della madre, o marito d'una sorella del padre, o
  della madre: Qui è fratello del padre.

\item[VN bel garzone] Cioe un figliuol maschio. E qui il Poeta seguita a mostrare il
  costume delle nostre donne accennato nell'ottava antecedente, che quando il parto
  è di maschio, ognuna di loro vorrebbe esser la prima a darne la nuova, e
  danno alla creatura sempre qualche epiteto, come \textit{un bel garzone}, \textit{un bel giovane},
  \textit{un garbato fantoccione}, \textit{un bamboccione d'importanza}. Vedi sopra in questo C.
  stan. 19. ma quando è femmina, tutte le assitenti ammutoliscono, o quando pur' al
  fine lo dicano, danno alla creatura epiteti d'avvilimento, come \textit{Piscialletto},
  \textit{Pisciacchera}, \textit{una sguaiatuccia}, e simili, come habbiamo detto poco sopra.
\item[IL nostro parentado] La nostra Genealogia: In che modo noi siamo parenti.
\end{description}

\section*{FINE DEL SECONDO CANTARE.}
\chapter{Terzo Cantare}

\begin{argomento}
Vengon d'Arno a seconda i legni Sardi,
Sbarcan le genti, e vanno a Malmantile,
Ma per vari accidenti i più gagliardi
Non fan quel tanto, che di guerra è stile.
Arma i suoi Bertinella, alza stendardi,
E mostra in debol corpo alma virile.
Nascon grandi scompigli in quella piazza,
E ognun si fugge in veder Martinazza.
\end{argomento}

\section{Stanza I \& II.}

\begin{ottave}
\flagverse{1}UN che sia avvezzo a starsene a sedere\\
Senza far nulla con le mani in mano,\\
E lautamente può mangiare, e bere, \\
E in festa, e giuoco viver lieto, e sano, \\
Se gli son rotte l'uova nel paniere, \\
Considerate se gli pare strano, \\
Ed io lo credo; c' a un' affronto tale \\
Al certo ognun l'intenderebbe male.
\end{ottave}

\begin{ottave}
\flagverse{2}E pur chi vive, sta sempre soggetto\\
A ber qualche sciroppo che dispiace,\\
Perché al Mondo non è nulla di netto,\\
E non si può mangiar boccone in pace,\\
Hor ne vedremo in Malmantil l'effetto,\\
Che immerso nei piacer vivendo a brace,\\
Non pensa che patir ne dee la pena,\\
E che fra poco s' ha mutare scena.
\end{ottave}

Il Poeta volendo trattare dell'assalto dato a Malmantile, e del disturbo, che
è per apportare l'esercito di Baldone a quelli spensierati, che sono nella Terra,
introduce il presente Cantare con una reflessione, che sia un gran disturbo a coloro,
i quali standosene co i loro commodi, e senza un minimo pensiero, si veggano
sopraggiugnere chi gli privi di questi loro agi; mentre simili accidenti sarebbono
di gran disgusto, e noia anche a coloro, che non stessero con tutti i lor
commodi; perché niuno, o bene, o male, che gli stia, vuol mai ricordarsi, che
tutti siamo sottoposti alle disgrazie, e che nel mondo non si dà felicità perfetta.

\begin{description}

\item[STARSENE con le mani in mano] A cintola, o in seno. Si dice d'uno, che
sia tutto dato in preda all'ozio, ed alla poltroneria, e che non vuol lavorare.
Vn accidioso, nighittoso, o scioperato. I Greci, e Latini dissero: \textit{In choenice
sedere}: \textit{de homine ocioso, \& desidioso}.

\item[GVASTAR l'uova nel paniere] Guastare i disegni altrui, Traslato dal guastar l'uova
  nel nidio, dove son dalla chioccia covate. Vedi Esopo Favola dell'Aquila, e dello
  Scarafaggio. È il \textit{covatum frangere} de i Latini.

\item[SE gli pare strano] Se gli par duro, e difficile a soffrire. Vedi sopra Cant. 2.
stan. 21.,  ed il proprio significato è di \textit{strano}. Stravagante, o forestiero, o non
del nostro parentado; valendocene in tutti questi, ed altri significati, come segue
ne i Latini della voce \textit{extraneus}.

\item[AFFRONTO] Significa Aggressione, assalto, abboccamento. Vedi sopra
Cant. 1. stan. 29. ma si piglia ancora per Sopruso, come è preso nel presente
luogo.

\item[BERE una sciroppo, che dispiaccia] Sopportar per forza una cosa, che sia di
  disgusto, che in Latino: si disse: \textit{Calicem bibere}; perché \textit{Calix} era una specie
  di bicchiere, col quale gli antichi bevevano caldo, come appunto si bevono gli sciroppi;
  e lo facevano ancor' essi per medicamento; e per conseguenza era tal bevanda,
  come a noi, per lo più, di poco gusto.

\item[NEL Mondo non è nulla di netto] Il Mondo non ha felicità perfetta. \textit{Unicuique
  dedit vitium natura creato}.

\item[VIVER a brace] Viver' a caso, senza regola, o considerazione. Ha forse
  questo detto origine dalla misura, che si fa della brace, che per esser cosa vile, e
  di poco prezzo si misura inconsideratamente senza guardare a darne un poca più
  o un poca meno. Da questo poi habbiamo \textit{sbraciare} veduto sopra Cant. 2. stan.
  10, che significa Consumare il suo inconsideratamente.

\item[MVTARE scena] Mutar faccia, o stato, mutar maniera di vivere, Traslato
  dalle prospettive, dove si recitano le commedie, quali prospettive sono da noi
  vulgarmente chiamate Scene.
\end{description}

\section{Stanza III \& IV.}

\begin{ottave}
\flagverse{3}Era in quei tempi la, quando i Geloni \\
Tornano a chiuder l'osterie de' cani,\\
E talun, che si spaccia i millioni \\
Manda al presto il tabì pe' panni lani;\\
Ed era appunto l'ora, ch' i Crocchioni \\
Si calano all assedio de' caldani; \\
Ed escon con le canne, e co' i randelli \\
I ragazzi a pigliare i pipistrelli.
\end{ottave}

\begin{ottave}
\flagverse{4}Quand in terra  l'armata con la scorta\\
Del gran Baldone a Malmantil s'invia,\\
Ond' un famiglio nel serrar la porta\\
Sentì rumoreggiar tanta genia.\\
Un vecchio era quest'huom di vista corta,\\
Che l'erre ogni hor perdeva all'osteria,\\
Tal che tra il bere, e l'esser ben d'età\\
non ci vedeva più da terza in là.
\end{ottave}

Descrive la stagione, che correva; quando la soldatesca sbarcò in terra, e s'avviò
verso Malmantile sotto la condotta di Baldone; e dice che era sul finire dell'Autunno,
poiché cominciava a diacciare, ed i ricchi finti mandavano a impegnare i vestiti da
state per risquoter quelli da inverno; costume assai usato da coloro,
che sfoggiano in vestire quantunque sieno poverissimi, e questi intendi
\textit{ricchi finti, che si spacciano i millioni}, che si suol dire; \textit{Mezzettin non risquote
  Pantalone}, e s'intende, che gli abiti da state non vagliono tanto, che impegnandoli
possano risquotere quei da inverno, come appunto è l'abito povero di Mezzettino
servo sciocco in commedia, e l'abito ricco di Pantalone vecchio in Commedia.
Narra parimente l'hora appunto che era, quando costoro s'accostarono a Malmantile,
e dice, che fu su l'annottare, che è quell'ora, su la quale i Crocchioni
si mettono nelle botteghe intorno a un caldano per passar la veglia. In tale stagione,
e fu quest'ora adunque arrivarono i soldati, condotti da Baldone, sotto
Malmantile, ed un famiglio nel serrar la porta gli scoperse più al romore, che
perché gli vedesse, essendo egli poco men che cieco.

\begin{description}
\item[GELONI] Intende freddi grandi, che fanno gelare, o addiacciare. Detto
equivoco da Geloni Popoli di Scitia, quali popoli pare che voglia dire, che sieno
coloro, che tornano a chiudere l'osterie de' cani. Le quali diciamo alcune buche nel
terreno della nostra Città cagionate dal mancamento delle lastre, le quali buche
nel tempo dell'inverno stanno piene d'acqua, e vulgarmente s'appellano pozze;
ma son chiamate Osterie de' cani, perché a queste vanno i cani a bere, e quando
vengono i diacci (che sono questi Geloni) ancor'esse addiacciano, e così restano
sode, e chiuse in modo che i cni non vi possono bere, e però dice, che i Geloni
tornano a chiuder l'osterie de' cani.

\item[TALUN che si spaccia i millioni] Uno che dà a creder d'esser ricchissimo Diciamo
  \textit{millantare} o \textit{smillantare}, come si vedrà sotto C.~11. stan.~49. d'uno che si
  spacci, o si vanti di ricco, di nobile, di dotto, ec. che da i Latini si dice: \textit{Sese
    iactare}. E questi tali si dicono \textit{Homines gloriosi, thrasones} per smillantatori tanto
  di ricchezze, quanto d'ogni altra cosa.

\item[PRESTO] Luogo pubblico, dove si pigliano in presto denari, con dare in pegno, e pagare g'interessi del denaro.

\item[TABÍ] È una specie di drappo leggieri di seta; E Dicendo: \textit{Manda al presto
  il tabí pe i panni lani}, intende Manda a impegnare l'abito da state per risquoter
  quello da verno.

\item[CROCCHIONI] Chiacchieroni, Cicaloni. Intendi certi perdigiorni, che si
  confinano a sedere in una bottega senza far'altro, che cicalare, il che si dice
  \textit{crocchiare}, o \textit{star'a crocchio}, donde poi \textit{Crocchioni}. Vedi sopra C. 1. stan. 41.

\item[SI calano] Cioè se ne vanno. Detto da gli uccelli, che in su quell'ora si calano
  a i lor pollai per dormire.

\item[CALDANO] Intendiamo quel vaso di rame, o di ferro, o di terra, o di altro
  materiale, che è usato per tenervi dentro brace, o carboni accesi per scaldarsi,
  e questo intende nel presente luogo; che per altro, \textit{Caldano} appellano i
  fornai quella stanza, o volticciuola, che hanno sopra il forno.

\item[PIPISTRELLO] Che si dice anche Vispistrello, o Vipistrello dal Latino Vespertilio,
  è il topo alato, animale notturno notissimo, come ancora è nota la caccia,
  che fanno i ragazzi del medesimo con brandire una canna, al fischio e sibilo,
  della quale egli vola, e da essa vien percosso, e fatto cadere a terra sbalordito;
  e perché alla detta caccia tanto serve una canna, che un bastone, però dice:
  \textit{con le canne, e co' i randelli}, cioè bastoni.

\item[FAMIGLIO] Qui intendi Birro guardia della porta.

\item[GENIA] Dal Grec. \textit{Genea}. Generazione. E vuol dire Gente vile, abbietta,
  e sciagurata: Sinonimo di gentaglia, genticciuola, ec.

\item[PERDER l'erre] Imbriacarsi: perché i briachi stentano a profferire la lettera R
  per avere la lingua legata dal troppo bere.

\item[Non ci vedeva più da terza in là] Se gli faceva buio, o notte a Terza, che è
quasi il principio del giorno, sì che si può dire, che costui fusse sempre al buio,
o non vedesse punto in tutto il giorno. È detto assai vulgato per intender uno
debole di vista, come intende nel presente luogo. Vedi spra C. 1. stan. 9. E forse
vuol intendere Uno di coloro, che perdono la vista alla levata del sole, e la
riacquistano quando il sole va sotto.
\end{description}

\section{Stanza V. — VII.}

\begin{ottave}
\flagverse{5}Per questo mette mano alla scarsella,\\
Ov' ha più ciarpe assai d' un rigattiere,\\
Perché vi tiene infin la faverella, \\
Che la mattina mette sul brachiere; \\
Come suol far chi giuoca a cruscherella, \\
Due hore andò alla cerca intere intere, \\
E poi ne trasse in mezzo a due fagotti \\
Un par a occhiali affumicati, e rotti.
\end{ottave}

\begin{ottave}
\flagverse{6}I quali sopra il naso a Petronciano\\
Con la sua flemma pose a cavalcioni;\\
Tal che meglio scoperfe di lontano\\
Esser di gente armata più squadroni.\\
Spaurito di ciè, cala pian piano,\\
Per non dar nella scala i pedignoni;\\
E giunto a basso lagrima, e singozza,\\
Gridando quanto mai n'ha nella strozza.
\end{ottave}

\begin{ottave}
\flagverse{7}Dicendo forte, perché ognun l'intenda: \\
All'armi all'armi, suonisi a martello,\\
Si lasci il giuoco, il ballo, e la merenda,\\
E serrinsi le porte a chiavistello,\\
Perché quaggiù nel piano è la tregenda,\\
Che ne viene alla volta del Castello;\\
E se non ci serriamo, o facciam testa,\\
Mentre balliamo vuol suonare a festa.
\end{ottave}

Il detto famiglio scoperse col mettersi gli occhiali, che era gente armata, e
per questo si messe a gridare; all'armi.
\begin{description}
\item[SCARSELLA] Tasca, Vedi sopra C. 2. stan. 8,

\item[CIARPE] Intendi robe vili, stracci, bazzecole, che i Latini dissero \textit{Scruta};
ed in altro senso \textit{Ciarpa} vedi sotto C. 5. stan. 33.

\item[RIGATTIERE] Rivenditore d'ogni sorta masserizie, ed arnesi da i Latini
detto \textit{Propola} dal Greco; ed a noi viene da rigaglie, che intendiamo robe diverse
di poco prezzo, ed avanzumi usati. L'Autore assomiglia la tasca di costui a una
bottega di Rigattiere, perché queste per lo più son ripiene di diversi arnesi, fra i
quali e talvolta difficile ritrovarvi una cosa, quand'altri la voglia.

\item[FAVERELLA] Fave macinate, ed impastate con acqua. E di questa si fanno
  torte cotte nel forno, che si dicono ancora Macco forse dal Grec, \textit{Matto}. Lat. \textit{pinso},
  Tale \textit{Faverella} dicono, che sia lenitivo a i dolori d'allentatura, ed habbia virtù
  d'assodar quelle parti; e però dice, che costui \textit{la mette in sul brachiere}, che è quella
  fasciatura, che s'applica all'estremità del ventre per sostenere gl'intestini.

\item[CRUSCHERELLA] È giuoco da Fanciulli. Fanno in sur' una tavola un monticello
  di Crusca, e vi mettono dentro quelle crazie, o quattrini, che vogliono
  giuocare, e mescolando poi bene, si fanno da uno del giuoco, a ciò deputato,
  tanti monticelli di detta crusca, quanti sono i giuocatori, i quali (lasciando da
  quello, che ha fatto i monti, perché deve esser l'ultimo a pigliare il monticello)
  tirano le sorti a chi debba esser il primo a pigliare uno di detti monti, e
  ciascuno nel monte, che gli è toccato va cercando de i denari, che la fortuana,
  v'habbia fatti restare. Stimo, che questo giuoco fusse usato ancora da i Fanciulli
  Latini, perché si trova \textit{Ludere furfure}, Ed a questa ricerca, che fanno i ragazzi
  del denaro assomiglia quello, che  il famiglio per trovare gli occhiali.

\item[FAGOTTI], Involti, o fardelli piccoli. Il Francese ancora dice \textit{Fagots}.

\item[PETRONCIANO] e Petonciano Specie di pomo simile alla mandragora, o
forse specie di Mandragora; e di color paonazzo lucente, nasce d'una pianta
simile alla Zucchetta, e sta appiccato al gambo con un poco di guscio come la
ghianda, alla quale s' affo a figura; ed in alcuni luoghi d'italia
si appella Marignano. A questo \textit{Petronciano} s'allomiglia comunemente, e da tutti
un naso di straordinaria grofiezza, e di colore rosso livido, come vuol che
s' intenda', che havesse questo famiglio.

\item[CAVALCIONI] Vuol dire una gamba da una parte, e l'altra dall'altra,
come si sta in sul cavallo, e come stanno gli occhiali sopra il naso, uno specchio
da una parte, l'altro dall'altra.

\item[PIAN piano] Cioè adagio adagio, bel bello: Adagissimo. La voce piano
aggiunta al verbo fare, ed al verbo andare significa quel, che hel presente luogo,
cioè Adagio, e con diligenza, che i Latini dicono placide incedere; ed aggiunta al
verbo parlare significa parlar con voce bassa, \textit{Submissa voce}.

\item[PEDIGNONI] Specie d'infermità, che viene ne i piedi, e nelle mani per lo
  troppo freddo dai Latini detti \textit{Perniones}.

\item[SIGNOZZARE] O singozzare, o singhiozzare. E' un moto del setto transverso,
  o mediastino, cagionato da soverchia votezza, o ripienezza; ma per similitudine
  significa anche sospirare vehementemente con pianto, come significa nel
  presente luogo. I Latini ancora se ne servivano nel primo significato, e nel secondo;
  \textit{Singultus}, \& \textit{singultire}, \& \textit{singultibus ingemere}.

\item[GRIDA quanto mai n'ha nella strozza] Grida quanto può più, e quanto può
resister la gola. Che \textit{strozza} vuol dire La canna della gola, altrimenti detta
\textit{Gorgozzule}. I Latini pure dicevano \textit{in gutture exclamare}, E da questa voce
\textit{strozza} viene strozzare, che vuol dire Strangolare.

Dante Inf. C. 7. \begin{verse}Quest' inno si gorgoglia nella strozza.\end{verse}
E. C. 28,  \begin{verse}Con la lingua tagliata nella strozza.\end{verse}

\item[SVONISI a martello] Si suonino le campane a rintocchi, che si dice anche: \textit{A
corr' homo}.

\item[TREGENDA] Moltitudine, e quantità di gente. Dalle persone semplici si
  crede, che vadano fuori la notte anime dannate, ed altri spiriti per spaurire la
  gente, e queste chiamano la \textit{Tregenda}. Tal' opinione se bene è di persone semplici,
  e idiote, nondimeno pare che venga seguitata da S. Agostino, poiché nel
  lib. 4. de Civit. Dei dice. \textit{Lamiae dicuntur animae hominum depravata, \& in malis
  vitae meritis maculosae, quae a corpore separatae terriculamenta sunt mortalibus}: nel presente
  luogo è intesa per moltitudine di gente.

\item[SVONARE] Il verbo suonare si piglia talvolta in vece del verbo percuotere,
  e però ne nasce l'equivoco del \textit{suonare mentre coloro ballano}, che vuol dire
  perquotergli, se ben pare, che voglia dire suonare alloro ballo: Ed in ciò imitiamo i
  Latini, che hanno il verbo \textit{pulsare}, che vuol dir perquotere, e vuol dire anche
  suonare ogni sorta di strumento musicale, e le campane; ed il suonatore si dice
  \textit{pulsator}.

\end{description}

\section{Stanza VIII. \& IX.}

\begin{ottave}
\flagverse{8}In quel che costui fa questa stampita,\\
E che ne i gusti ognun si balocca,\\
L'armata finalmente è comparita\\
Già presso a tiro all'alta Biccicocca.\\
Quivi si vede una progenie ardita,\\
Che si confida nelle sante nocca,\\
E se ne viene all'erta lemme lemme\\
Col Batthil Toffie tutto Biliemme.
\end{ottave}

\begin{ottave}
\flagverse{9}Tra questi guitti ancora sono assai,\\
Oltre a Marchesi, Principi, e Signori;\\
Huomin di conto, e grossi bottegai,\\
Banchieri, Setaiuoli, e Battilori,\\
Lanaiuoli, Orefici, e Merciai,\\
Notai, Legisti, Medici, e Dottori,\\
In somma quivi son gente, e brigate\\
D' ogni sorta; chiedete, e domandate.
\end{ottave}

Mentre il suddetto vecchio andava gridando, e che non ostante questo, coloro,
che erano in Malmantile seguitavano a darsi bel tempo, l'armata arrivò
presso le mura; Il Poeta narra la qualità di questi soldati.

\begin{description}
  \item[STAMPITA] Vuol dir suonata, o cantata, Bocc. Nov. 97. \textit{Con una sua viola
    suonò alcuna stampita}.  Varchi stor. lib. 10. Malatesta andò in persona sopra il bastion
    e di S. Miniato con tutti li suoi suanatori, e dopo più lunghe strombettate, e stampite,
    ec. Ma qui intende romore, e cicalamento odioso, che è il senso, nel quale oggi
    per lo più è presa da noi questa parola, ed ha lo stesso significato che \textit{bordello},
    \textit{chiasso}, \textit{musica}, e simili, presi pure metaforicamente, il che vedremo altrove.

\item[BALOCCARSI] Trastullarsi, Perder'il tempo, e trartenersi in cose di poco
  momento, o trastulli da ragazzi, de i quali è proprio il verbo \textit{baloccarsi}, o \textit{balocco};
  e forse è sincopato daf verbo \textit{Badaluccare}, e \textit{Badalucco}; Vedi sotto C. 6, stan. 32.

\item[BICCICOCCA] Diciamo anche \textit{Bicocca}. Varchi stor. lib. 15. \textit{gli furono portate
  le chiavi di non so che Bicocca}; Vuol dir fortezza piccola, e di poca considerazione
  posta in luogo eminente, come appunto è Malmantile, il quale con questa
  sola parola \textit{Biccicocca}, il Poeta benissimo descrive; perché per Biccicocca volgarmente
  intendiamo un Casolare, o castelluccio posto in luogo eminente, ma
  da farne poca stima. Lasca Nov. 3. \textit{Salita che hebbe con non poca difficultà quell'alpestre
  Montagna, credeva entrare in un bel castello, ma riguardando all' intorno, vedde
  che era una Biccicocca più per refugio di capre, che per ricetto di soldati}.

\item[SI confida nelle sante nocca] Ha la sua fidanza nelle pugna. E l'epiteto \textit{sante} è
  messo per esprimere il modo del parlare de i Battilani: Se bene e usato dalla gente
  anche più civile per interider perfezione come vedemmo sopra C. 2. stan.\ 52.
  E qui è benissimo posto, perché \textit{sanctus} vuol dir determinato, o stabilito, sendo
  sincopato da \textit{sancitus}, e le pugna sono s'armi stabilite, e proprie de' Battilani.
  Che per \textit{nocca}, che sono i nodelli delle dita, s'intende tutta la mano serrata, che
  in questo pugno, ed in questo più che in altra maniera si scorgono le nocca.

\item[LEMME lemme] È della medesima natura, ed ha lo stesso significato di pian
  piano detto sopra in questo C. stan. 6., ma è termine restato ne i Battilani, o se
  pure è usato da altri sarà detto \textit{lieme lieme}, che viene dal Latino \textit{leviter}, o \textit{leve}, e
  significa leggiermente, o dal Toscano Lieve, che vuol dir Leggieri.

\item[BATTI, e Tessi] Battilani, che son coloro, che conciano la lana, e Tessi
  quelli che la tessono.

\item[TUTTO Biliemme] Chiamiamo Biliemme quell'ultime contrade della Città
  di Firenze, dove abita questa sorta di gente, la quale veramente, benché
  nata, ed allevata in Firenze, è affatto differente da gli altri Fiorentini ne i costumi,
  e nel parlare; farebbe leggi a suo modo; mangia d' ogni sorta sporcizie,
  come gatti, cani, pesce, e carne fetida; beve ogni sorta di vino sregolatissimamente,
  come afferma il nostro Poeta sotto in questo C. stan. 60. dicendo: \textit{Gente
  che a bere è peggio delle spugne}. In somma è un Popolo da se, che noi chiamiamo
  gli \textit{Unti}, il \textit{Batti}, o \textit{Biliemme}, la qual voce serve ancora per esprimere la più vil
  plebe, come è nel presente luogo.

\item[GVITTI] Guidoni, plebei, sudici, sporchi, e sordidi. E' parola che ha del
  Napoletano, se bene il Varchi stor. lib. 10. se ne serve anch' egli per esprimere
  un' hvomo d'animo vile, dicendo: \textit{Egli era tanto d'animo guitto, e tanto meschino,
  che usava dire: Chi non va a bottega è ladro}.

\item[HVOMINI di conto] Huomini di stima; huomini riguardevoli. Translato
  forse dal giuoco delle Minchiate, nel qual giuoco si stimano, ed apprezzano
  solamente le carte, che contano, le quali son quelle, che vedremo sotto C. 8. stan.
  61. Si dice \textit{Il tale conta} per intendere; il tale è huomo adoperato, o e buono a
  qualcosa.

\item[BATTILORI] Mercanti d'oro filato. \textit{Banchieri} Mercanti di cambio, che
  si dicono Negozianti. \textit{Setaiuoli} Mercanti di drappi, e di seta, \textit{Lanaiuoli}
  Mercanti di pannine, e Lana. \textit{Orefici} Mercanti d'oro, e d'argeato sodo.
  \textit{Merciai} Coloro, che vendono nastri, seta, telerie, ed altre merci simili. E tutti
  questi suddetti in generale si chiamano Mercanti, o mercatanti.

\item[BRIGATE] Quantità di gente, Vedi sopra C, 1. stan.\ 2.

\item[D'ogni sorta, chiedete, e domandate] Cioè domandate, ed eleggete pure, che
  sorta di gente volete, che la troverete fra costoro; perché vi è d'ogni specie
  di persone.

\end{description}

\section{Stanza X. \& XI.}

\begin{ottave}
\flagverse{10}Sul Colle compartisce questa gente\\
Amostante con tutti gli Vfiziali; \\
Tra' quali un grasso v'è convalescente, \\
c' haveva preso il dì, tre serviziali; \\
E appunto al corpo far' allor si sente \\
L'operazione, e dar dolor bestiali, \\
Tal che gridando senz' alcun conforto \\
In terra si butta come per morto.\\
\end{ottave}

\begin{ottave}
\flagverse{11}Il nome di costui, dice Turpino,\\
Fu Paride Garani, e il legno prese,\\
Perch' ei voleva darne un rivellino\\
A un suo nimico traditor Francese,\\
Che per condurlo a seguitar Calvino\\
Lo tira pe' capelli al suo paese,\\
E per fuggirne a i passi la gabella,\\
Lo bolla, marchia, e tutta lo suggella.
\end{ottave}

Ii Generale Amostante distribuisce sul colle di Malmantile i Soldati, fra i
quali era Paride Garani, che havendo preso un gran vacuatorio sentiva dolori acerbissimi,
e però si rammaricava. Il nostro Poeta per accredirare questa opera,
come fece il Pulci nel suo Morgante, e Ariosto nel Furioso, le da anche
egli il fondamento della storia; allegando l'autorità di Turpino, come fece anche
sopra C. 2, stan. 31. e da quello che scrive Turpino, cava che costui havea
nome Paride Garani, il quale havea preso il legno per dare una quantità di legnate
a un suo nimico Francese, che per condurlo a seguitar Calvino, lo voleva
tirare pe i capelli in Francia, e per risparmiarne la gabella d'haveva già marchiato,
e bollato, e sigillato. E scherzando l'Autore con questi equivoci, vuol
dite che Paride prese il Legno santo per medicarsi del mal Franzese.

\begin{description}
\item[PRESE il legno] Cioè bevve il decotto di Legno Santo per medicare il Mal
Franzese; se ben par che voglia dire, prese un pezzo di legno per bastonare quel
suo nimico Francese.

\item[DARE un rivellino] Dare una quantità di legnate. Rivellino e una specie di
  fortificazione, che si suol fare d'avanti alle porte delle Città, o fra le cortine
  delle Fortezze, così detto forse perché \textit{revellitur a linea}, o perché \textit{revellat
    hostium vim}, e da questa rivolta nelle cortine, o dal quasi rivoltarsi egli al nimico habbiamo
  il presente translato, che ci serve per esprimere, Rivoltarsi a uno con
  gran quantità di bastonate, bravate, riprensioni, ec, E dicendosi assolutamente
  e senz'aggiunta: \textit{Gli fece un rivellino}, s'intende \textit{Gli fece una solenne bravata}, o
  \textit{buona passata}, o \textit{gran rabbuffo}; E dare un rivellino, s'intende dar quantità di percosse.

\item[RIDURLO a seguitar Calvino] Par che voglia dire ridurlo a seguitare la setta
  di Calvino Eretico, e vuol dire, che per farlo divenir calvo, questo suo mal
  Francese lo tira per i capelli, e glieli fa cascare.

\item[LO bolla, marchia, e tutto lo suggella] Fa bullette, marchia, e suggella. E vuol
  dire che questo suo mal Francese gli havea cagionato bolle, croste, e lividi; che
  il verbo suggellare vuol dire Far de i lividi nel viso a uno con le percosse, i quali
  noi chiamiamo Pesche: I Latini in questo senso dissero; \textit{suggillare}. Vedi sotto
  C. 6, stan. 54. metaforico da \textit{suggellare} che vuol dire imprimere in cera, ostia, e
  simili nelle lettere, ec. e si dice anche \textit{sigillare} Dant, Purg. C. 7.
  \begin{verse}
    La sua impronta quand'ella sigilla.
  \end{verse}
  E suggellare Dante Purg. C. 10. \textit{Come figura in cera si suggella}. E Canto 33.
  \textit{Ed io sì come cera da suggello}.
\end{description}

\section{Stanza XII. \& XIII.}
\begin{ottave}
\flagverse{12}Disse Amostante, visto il caso strano, \\
A Noferi di casa Scaccianoce: \\
Per Ser Lion Magin da Ravignano, \\
Ch' il venga a medicar, corri veloce; \\
Io dico lui, perché ce n' è una mano, \\
Ch' infilza le ricette a occhio, e croce,\\
O fa sopr' all' infermo una bottega, \\
E poi il più delle volte lo ripiega.
\end{ottave}

\begin{ottave}
\flagverse{13}Gloria cerca Lion, più che moneta,\\
Però ch' ei bada al giuoco, e fa progresso;\\
Per l' acqua in Pindo andò come Poeta,\\
Ond' agl' infermi dà le pappe a lesso.\\
Gli è quel che attende a predicar dieta\\
E farebbe a mangiar con l' interesso;\\
Ma perché già tu n'hai più d'uno indizio,\\
Va via, perché l'indugio piglia vizio.
\end{ottave}

Amostante veduto lo stravagante accidente, ordinò a Noferi Scaccianoce (che
vuol dir Francesco Cionacci\footnote{Francesco Cionacci, 1633-1714, ``Accademico Apatista''}) che andasse per Ser Lion Magin da Ravignano
(che vuol dire Giovann' Andrea Moniglia\footnote{Giovanni Andrea Moniglia, Firenze 1624 - Prato 1700, medico, autore di teatro, librettista, Accademico della Crusca.}) e facesse venire lui medesimo, che è
un valent'huomo, e non come qualcuno, che non sa dove s'habbia la testa, ed
in vece di medicare un'infermo il più delle volte l'ammazza con le sue spropositate
ricette, ed è di quelli, de i quali si può dire.
\begin{verse}
  His, \& si tenebras palpant, est facta potestas,
  Extenuandi agros, hominesque impune necandi.
\end{verse}

Il che non si può dire di Lione, che procura più d'acquistar gloria che oro.
Egli è Poeta, e però non è maraviglia, se andando egli per l'acqua al fonte di
Parnaso dia poi molte pappe con l'acqua a gli ammalati. L'Autore dice così,
perché in una sua leggieri infermità non volle questo medico, che gli pigliasse
medicamento alcuno, ma lo volle curare con la sola dieta, facendoli mangiar
sera, e mattina pappe; e però dice; \textit{Attende a predicar dieta, E farebbe a mangiar
con l'interesso}; perché veramente in quel tempo Lione essendo giovanotto
sano e robusto, mangiava assai. Questo Lione non era stato nominato dall'Autore
nel primo componimento della presente sua Opera, benché suo amicissimo,
havendo solamente nominato quel medicastro, che dice gli spropositi, che vedremo
poco appresso, ma dopo la suddetta infermità, per vendicarsi graziosamente
dell'haverlo tenuto tanto a dieta ce lo volle mettere. Hor tornando a cammino.
Il Generale dopo haver dato a Noferi molti contrassegni, affinché conoscesse
questo medico, manda a cercarne.

\begin{description}
\item[CE n' è una mano] Ce ne son molti. Termine che vien dal Latino. Verg. 4.
En, \textit{Iuvenum manus emicat ardens}.

\item[INFILZA le ricette a occhio, e croce] Si dice anche a occhio, e voce: Fa le
  ricette senza regola, considerazione, o fondamento. Opera senza scuola, o riprova,
  E' termine meccanico.

\item[FAR una bottega sopra uno infermo] Far allungare il male per cavarne maggior
  guadagno. E questo termine s'usa in qualsivoglia negozio, del quale uno procuri
  di prolungar la spedizione per buscar più denaro.

\item[RIPIEGARE uno] Intendiamo Far morir uno, Vedi sotto C. 10. stan. 4.

\item[BADAR al giuoco] Attender con applicazione a quella professione, che uno
fa, o a quel negozio, che ha fra mano, e si dice anche Badare a bottega. Vedi
sopra C. 1. stan. 62. questo verbo \textit{badare} in altri significati.

\item[PAPPA] Cioè pane bollito nell'acqua; o in altro liquore. E' di quelle parole
  inventate dalle Balie per facilitare il parlare a i bambini, come babbo, mamma,
  e simili. I Latini dissero, \textit{pappare}, e i Greci pure dicevano \textit{Pappa} se bene
  in altro senso, volendo esprimere il Padre, il Babbo, Vedi sopra C. 2. stan. 66.
  E sotto C, 4. stan.\ 5 e 12.

\item[ATTENDE a predicar dieta] Sempre dice che si mangi poco; che questo intende
  per far dieta. Se bene appresso a' Medici \textit{dieta} vuol dire regola di vita universale.
  Dieta si dice congresso di gran personaggi per trattare negozzi gravissimi,
  come si dice Dieta il Congresso de i Priacipi Elettori all' Elezione dell'Imperatore.

\item[FAREBBE a mangiar con l'interesso] Mangerebbe sempre di giorno, e di notte,
  come fanno i cambi, o usure, che mangiano dì, e notte, mentre che il tempo
  fa crescer la somma degl'interessi. L'usura in Ebreo dicesi morso.

\item[L'INDVGIO piglia vizio] L'indugiare, o trattenersi è pericoloso di cagionare
  qualche danno, o far perder la congiuntura di conseguir l'intento. \textit{Mora trahit
  damnum}.
\end{description}
\section{Stanza XIV.}
\begin{ottave}
\flagverse{14}Noferi vanne, e sente dir ch' egli era \\
Con un compagno, entrato in un fattoio, \\
Ov' egli ha per lanterna, essendo sera,\\
L'orinal fitto sopra a un schizzatoio,\\
E di fogli distesa una gran fiera,\\
Ha bell', e ritto quivi il suo scrittoio,\\
Si che presto lo trova, e in su l'entrata\\
Dell unto studio gli fa l'ambasciata.
\end{ottave}

Noferi trova il Medico nel Fattoio da olio, che quivi era il suo studio, e gli
fa l'ambasciata.

\begin{description}
\item[FATTOIO] Quella stanza, dove è la macine per infragnere l'olive, e lo
  strettoio, ed altri ordinghi per cavar l'olio dalle medesime olive. Vien dal Latino
  \textit{Olei factorium}.

\item[ORINALE] Vaso di vetro o d'altra materia, nel quale s'orina, da i Latini
  detto \textit{matula}, \textit{vas urinarium}, e \textit{scaphium}, donde i Sanesi chiamano scafarda,
  o scanfarda quella catinella, che a tale effetto usano le donne.

\item[SCHIZZATOIO] È una grossa canna di stagno, o d' altro metallo, con la
  quale si danno i serviziali agl' infermi. Vedi sotto C. 10. stan. 4.

\item[DISTESA una fiera di fogli]  Sparsa una quantità di fogli. Dice \textit{fiera} per la
  similitudine, che haveva quella distesa di fogli con le \textit{fiere}, o mercati, che alcune
  volte all'anno si fanno in Firenze, nelle quali per le piazze si veggono moltissime,
  e diverse mercanziuole, disegni, leggende, ed altri arnesi confusamente.
  Latino \textit{Nundinae}, abbiamo forse questa voce \textit{fiera} dal Latino \textit{forum}, che era inteso
  per la piazza dove si facevano le fiere o mercati, o pure dal Latino \textit{feriae}.

\item[HA bello, e ritto] Ha con facilità aggiustato il suo scrittoio; che la voce bello,
  in questi termini altro non vuol dire, che Ormai, o di già, e serve per emfasi,
  e per denotare la franchezza in terminare una operazione: Si dice \textit{rizzare una
  bottega}, \textit{rizzare un negozio} per dar principio a un negozio.

\item[VNTO studio] Si chiama studio quella stanza, nella quale uno sta a studiare;
  e perché questo Medico haveva deputata per suo studio la stanza del fattoio, lo
  chiama \textit{studio unto}, perché tali stanze sono, o verisimilmente devono essere unte.
\end{description}
\section{Stanza XV \& XVI.}
\begin{ottave}
\flagverse{15}Ei c'alla cura esser chiamato intende\\
Risponde haver' allora altro che fare, \\
Per c'una sua commedia ivi distende \\
Intitolata il Console di Mare,\\
E che se opra sua colà s'attende,\\
Un buon suggetto quivi suo scolare,\\
Di già sperimentato, ed in sua vece\\
Havria mandato lui; e così fece.
\end{ottave}

\begin{ottave}
\flagverse{16}Era quest'huomo un certo Medicastro,\\
C' al dottorato suo se piover fieno\\
E perch' ei vi patì spese, e disastro,\\
E stato sempre grosso con Galeno;\\
E giunto là: Vo far (disse) un' impiastro,\\
Onde s' il mal venisse da veleno\\
Presto vedremo; in tanto egli si spogli,\\
E siami dato calamaio, e fogli.
\end{ottave}

Sentendo Lione d'esser chiamato a medicare, risponde, che per allora non
può venire, ma che manderà un suo scolare valent'huomo. Costui era un gran
bue, e però giunto dove era l'infermo, cominciò subito con gli spropositi.

\begin{description}
\item[CONSOLE di mare] Questa fu una Commedia intitolata \textit{La Serva nobile}\footnote{La Serva Nobile, Musica: Domenico Anglesi (161? - 1674), Libretto: Giovanni Andrea Moniglia, Prima rappresentazione: Firenze Teatro della Pergola, 1660.}, nella
  quale è introdotto per l'Eroe un Console di Mare in Pisa, onde molti la chiamano
  il \textit{Console di mare}, ancor che il titolo stampato in fronte di essa sia, \textit{La
    Serva nobile}, e fu composta dal medesimo Lione, e recitata in musica con grandi
  Apparati d' ordine del Serenissimo Principe Cardinal Gio: Carlo nel suo bellissimo
  Teatro fabbricato allora di nuovo. Ed il nostro Poeta nella presente ottava
  vuol mostrare la poca applicazione, che Lione haveva in quei tempi alla medicina,
  come giovane, se ben per altro dotto; e che poi voltatosi a tale studio ha
  saputo acquistarsi la fama, che ha acquistato, e meritare una delle prime Cattedre
  dello studio di Pisa, e di servire attualmente al Serenissimo Gran Duca per Medico.

\item[MEDICASTRO] Medico di poca scienza, o (come diremo) salvatico,

\item[FE piover fieno nel suo dottorato] Quando si sente uno, che vaole spacciarsi per
  huomo dotto, e dal parlare si fa conoscer per uno ignorante, si suol dire quando
  ci parla \textit{Tirate giù del fieno} intendendosi: Per dargli a questo bue che parla. Sì
  che dicendo che \textit{nell'addottorarsi costui, piovve fieno}, intende che costui fu conosciuto
  per un solennissimo bue; e però venne gran quantità di fieno senz' esser chiesto,
  che diciamo: \textit{La roba ci piove} per intendere vien roba in abbondanza, senza
  chiederla.

\item[È STATO sempre grosso con Galeno] Esser grosso con uno vuol dire essere in
  collera, o esser adirato con uno; sì che dicendo, che costui \textit{è stato sempre grosso
    con Galeno}, perché l'haveva disastrato, e fatto penare, s'intende era adirato
  seco; e però non lo guardava mai, e conseguentemente non havea pratica con
  Galeno, e non sapeva quel che egli dicesse, sì che in sustanza vuol dire un grandissimo
  ignorante nella Medicina.

\item[VELENO] Questa parola ha due significati: uno proprio che è tossico, e l'altro
  improprio, che è fetore. Il primo è quello, che s' intende nel presente luogo,
  il secondo si vedrà nell' Ottava seguente.
\end{description}

\section{STANZA XVII.}
\begin{ottave}
\flagverse{17}Mentre è spogliato, per la pestilenza,\\
Ch' egli esala, si vede ognun fuggire,\\
Pervenne una zaffata a Sua Eccellenza,\\
Che fu per farlo quasi che svenire; \\
Confermata però la sua credenza\\
Rivolto a i circostanti prese a dire:\\
Questo è veleno, e ben di quel profondo,\\
Sentite voi ch' egli avvelena il Mondo?
\end{ottave}

Mentre che Paride si spogliava ognuno per lo gran fetore cominciò a fuggire,
onde il sig.\ Medico, che sente ancor' egli l'orrendo fetore, si confermò nel credere,
che fusse veleno, perché avvelenava.

\begin{description}
\item[PESTILENZA] Intendi fetore grandissimo. E si serve della parola \textit{pestilenza},
  per la parola \textit{veleno} presa in significato di puzzo, o fetore, e per altro
  \textit{Pestilenza} vuol dire mal contagioso.

\item[ZAFFATA] Parte del vapore di quel puzzo, portato dal moto dell' aria.
  E si dice anche \textit{zaffata} d' ogni liquore per intendere \textit{spruzzaglia} d'ogni liquore.
  Franco. Sacc. num. 136. \textit{L'orina gli andò sul Cappuccio, e nel viso, ed alcune zaffate in
  bocca}.

\item[AS. Ecc.] Questo titolo benché non sia così conveniente a' Medici, nondimeno
  è usato dalla nostra plebe in vece dell' Eccellentissimo, e l'Autore lo dà a questo
  medico per derisione.

\item[PROFONDO] Per traslato significa Grandemente, smoderato, o perfettissimo,
  come usavano anche i Latini.

\item[AVVELENA] Rende puzzolente. Ecco la voce \textit{veleno}, ed \textit{avvelenare}
  presa nel secondo senso detto di sopra di \textit{puzzo}, o \textit{fetore}; E l'equivoco, che da
  ciò ne nasce, serve a questo Medico per farsi stimar dotto mostrando conoscere,
  che questo è veramente \textit{veleno}, perché egli avvelena, che vuol dire far putire, ed egli
  lo piglia in significato d'attossicare, e Veleno in significato di tossico, Vedi sotto in
  questo C. stan. 54. la voce lezzo.
\end{description}
\section{Stanza XVUL}
\begin{ottave}
\flagverse{18}Rispose il general, commosso a sdegno:\\
Come veleno? o corpo di mia vita!\\
E dove è il vostro naso, e il vostro ingegno? \\
Lo vedrebbe il mio bue, ch'egli ha l'uscita.\\
A ciò soggiunse il Medico: Buon segno,\\
Segno che la natura invigorita\\
A' morbi repugnante, adesso questo\\
A nostri nasi manda sì molesto.
\end{ottave}

Il Generale s'adira, e dice: Che non havete odorato da sentir questo puzzo,
ne ingegno da conoscere, che egli ha l'uscita! Al che replica il Medico: Questo
è buon segno, perché la natura havendo preso vigore, come quella, che repugna
a i morbi, espelle ora questo morbo, e lo manda ai nostri nasi. Per intender
bene lo sproposito, che fa dire a questo Medico, è necessario sapere, che
la parola \textit{morbo} ha due significati, il primo è infermità, e dicendo \textit{repugnante a i
morbi} intende all'infermità; ed il secondo è \textit{fetore} o \textit{puzzo}; e dicendo \textit{manda a'
nostri nasi questo morbo} intende Manda questo fetore. Ed il buon medico, che stima
che \textit{natura morbo repugnans} voglia dire repugni al puzzo, cava la conseguenza,
che il sentir questo puzzo sia buon segno, perché la natura scacciando il puzzo,
dal corpo dell'infermo, lo manda a i nasi de' circostanti, e così va scemando
il morbo al pazziente.

\begin{description}
\item[LO vedrebbe il mio bue] Lo vedrebbe uno, che non havesse punto di giudizio.

\item[USCITA] Stemperamento di Corpo, Soccorrenza; da' Latini con voce Greca
  detta \textit{Diarrhoea}.

\item[BVON segno] L'Autore mostra in questa Ottava il modo, col quale soglion
  parlare i Medici ignoranti per accreditarsi appresso agl' idioti, dando ragioni
  spropositate, e inducendo aforismi improprj, pur che lusinghino il pazziente con
  una certa apparenza di sperar bene, come fanno gli Zingani, e i Montambanchi.
\end{description}
\section{Stanza XIX.}
\begin{ottave}
\flagverse{19}Vedendo poi, ch'il flusso raccappella \\
(Come quelle c'ha in zucca poco sale) \\
Comincia a gridar: Guardia, la padella; \\
E (quasi fusse quivi uno spedale) \\
Chiamagli astanti, gl'infermieri appella,\\
Il cerusico chiede, e lo Speziale,\\
E venuto l'inchiostro, al fin si mette\\
A scriver una risma di ricette.
\end{ottave}

L'Eccellentissimo Medico vedendo, che il corpo faceva nuova operazione, cominciò
a chiamar la Guardia, che portasse la padella, pensando che quelle parole
havessero virtù di fermare il flusso, havendole sentite dire negli Spedali in
occasioni simili, e però credendo esser nello Spedale chiamava gli Astanti, ec. e
poi si messe a scriver una gran ricetta.

\begin{description}
\item[RACCAPPELLA] Opera di nuovo. Reitera, Replica. Raccappellare si dice
  quando coloro, che stringono l' olive per cavarne l'olio, o le vinacce per cavarne
  il vino, dopo haver dato qualche stretta, allentano lo strettoio, e nelle
  gabbie mettono nuove olive, o nuova vinaccia sopr'all'altra, che v' era prima.
  Alcuni dicono \textit{rincoppellare}, traendolo dalle coppelle de' purgatori d'oro, nelle
  quali rimettono più volte lo stesso metallo per raffinarlo, il che dicono \textit{rincoppellare}.

\item[HAVER poco sale in zucca] Haver poco cervello, poco giudizio. Bocc.n.2,
g. 4. \textit{Per porre la sua belezza innanzi ad ogn'altra, sì come quella che haveva poco sale
in zucca}. Vedi sopra C. 1. stan. 73. e sotto C. 4. stan. 15.

\item[GVARDIA, la padella] Questo e un detto, che s' usa, quando si sente, che
  altri faccia romore per di sotto per causa dell'uscita del vento, e si dice così, perché
  gl' infermi, che sono negli spedali, quand' hanno bisogno di votare il ventre,
  chiamano colui, che è di guardia, che porti la \textit{padella}, che è un vaso di rame,
  ec, il quale è adattato in maniera da potersi mettere, in caso di bisogno, nel
  letto sotto all' infermo, acciò che possa fare il fatto suo, senza muoversi dal
  letto.

\item[STANTi] o \textit{Astanti}, Son coloro, che assistono al servizio degl'infermi, come
  vedemmo sopra C. 1. stan. 48. Lat. \textit{adsantes}.

\item[INFERMIERE] Chiamano negli spedali \textit{Infermiere} colui, il quale invigila
  che gl' infermi sieno messi a letto, quando son condotti allo spedale, e gli piglia
  nota per fargli visitare dal Medico, e gli registra al libro degli entrati, e de gli
  usciti, ed al libro de' morti.

\item[CERVSICO] Quello che medica le ferite, piaghe, ed altri mali esterni, che
  richieggono opera manuale, e cava sangue, ec, detto ancora con voce Greca
  usata da' Latini \textit{Chirurgo}.

\item[LISMA] o \textit{risma}, Diciamo un fagotto, o balletta di carta, che sarà di circa
  500.\ fogli. Dal Gr. \textit{arithmos}. Qui però è detto iperbolico, e per mostrare, che
  questo Medico scrivesse assai, non che veramente consumasse una Lisma di carta.
\end{description}

\section{Stanza XX.}

\begin{ottave}
\flagverse{20}Dove diceva (dopo millioni \\
Di scropoli, di dramme, e libbre tante) \\
Che già, che questo mal par che cagioni\\
Stemperamento forte, umor piccante,\\
Per temperarlo; Recipe in bocconi \\
Colla, gomma, mel, chiara, e diagrante, \\
Quindici libbre in una volta sola \\
Di sangue se gli tragga dalla gola;
\end{ottave}

\begin{ottave}
\flagverse{21}Accio che tiri per canal diverso \\
L'umor che tende al centro, \textit{ut omne grave}\\
Che se durasse troppo a far tal verso \\
Dir potrebbe l'infermo: Addio fave. \\
Poi tengasi due dì capo riverso \\
Legato per i piedi a una trave;\\
Se questo non facesse giovamento, \\
Composto gli faremo un'argomento.
\end{ottave}

\begin{ottave}
\flagverse{22}Peré presto bollir farere a sodo\\
Un'agnello, o capretto in un pignatto;\\
N' un' altro vaso nelle stesso modo\\
Un lupo per infin che sia disfatto;\\
Poi fare un servizial col primo brodo,\\
E col secondo un' altro ne sia fatto;\\
Farà questa ricetta operazzione\\
Senz' alcun dubbio, ed eccola ragione
\end{ottave}

\begin{ottave}
\flagverse{23}Questi animali essendo per natura\\
Nimici, come i ladri del Bargello,\\
Ritrovandosi quivi per per ventura,\\
Il lupo correra dietro all'agnello;\\
L'agnello, che del lupo havrà paura\\
Ritirandosi andra per il budello;\\
Così va in su la roba, e si rassoda,\\
E i due contrarj fan, ch'il terzo goda.
\end{ottave}

In queste sue ricette mostra l'Eccellentissimo Medico la sua goffaggine con
proporre farmachi, e rimedj spropositati, come è quello de i due brodi di lupo,
e d'agnello, e quello del tenere il pazziente appiccato al palco per li piedi col
capo all'ingiù.

\begin{description}
\item[MILLIONE] È un numero determinato di dieci centinaia di migliaia, ma qui
è preso per indeterminato; come succede spesso, che per esprimer, grandissima
quantità di cose, si dice E' un millione delle tali cose, ancor che sieno molte meno,
ed alle volte molte più. Così i Latini in questo senso \textit{sexcenta}, \textit{centum milia},
e Greci \textit{myria}, cioè diecimila.

\item[STEMPERAMENTO forte] Stemperare vuol dir Ammollire, o liquefare, e
  nel ventre di costui era sollevamento d'umori, e stemperamento di materie forti,
  cioè acide, e di umori piccanti. Gli epiteti di forte, e piccante son epiteti
  convenienti al vino, dicendosi vino forte quello, che comincia a diventare aceto
  ed in molti luoghi d'Italia si dice Vin forte, il vino gagliardo, o grande; e vino
  piccante quello che in beverlo fa frizzare le labbra, e la lingua. Questo Eccellentissimo
  Medico però intende quel \textit{forte} per acido, e per grande, e gagliardo;
  E piccante dal verbo \textit{piccare}, che vuol dir Pugnere, Offendere, che si dice anche
  \textit{dar nel naso}. Vedi sotto C. 7. stan. 59. l'Eccellentissino cava l'argumento, che
  questi umori sieno piccanti, perché danno nel naso col loro fetore: Ora per rassodare,
  e coagulare tal stemperamento vuole il prelibato Medico, che si dia al
  pazziente a bere gran quantità di \textit{colla, miele, gommma, chiara d'uovo, e diagrante},
  le cose nella somma, e quantità, che egli pone se s'incorporassero, in
  grandissima quantità d'acqua e sarebbono atte a coagulare, e seccare un lago;
  e se vi havesse aggiunto gesso, e matton pesto havrebbe dato una ricerta da stoppare
  quante rotture si possano mai trovare ne i vivai.

\item[DIAGRANTE] Specie di gomma, o colla, che serve per incollare i drappi
  ne i rovesci de i ricami, o per altre cose simili.

\item[SE li tragga 15. libbre di sangue per la gola] E cavandosi 15. libbre di sangue
  dalla vena della gola del pazziente; e legandolo per i piedi al palco col capo
  all' ingiù (che questo vuol dir caporiverso) pretende il Medico, che la roba sia
  per mutar viaggio, se vorrà condursi al suo centro, che non è più nel luogo,
  dove era prima, ma stante la positura del corpo è diventato suo centro il capo.

\item[CONTINOVASSE a far tal verso] Continovasse a fare nella medesima forma,
  o maniera, Vedi sotto C. 7. stan. 1.

\item[ADDIO fave] Significa Noi siamo spacciati; Noi siam finiti; Siam morti, Fu
  un Villano nel contado d' Imola d' ingegno più tosto grosso che no, il quale haveva
  un bellissimo campo di fave, e nel mezzo di esso era un gran ciriegio carico
  di ciriege. A tal Ciriegio haveva il villano fatta una fortissima prunata, perché
  le ciriege gli fussero colte; e vantandosi di questa sua diligenza, fu sentito
  da un Cieco suo amico, il quale gli disse: Con tutti li tuoi pruni io vi salirò, e
  se non lo faccio, voglio perdere dodici lire, ch' io mi ritrovo, ed il villano replicò:
  Se tu non pigli la scala, o vero non porti il forcone, o altro per levare
  i pruni io voglio giuocare questo campo di fave, e che tu non vi sali. il Cieco
  si contentò, e così convennero. L'astuto Cieco si coperse tutta la vita con buone
  pelli di bue, e così armato passando per mezzo de i pruni senza sentir puntura,
  alcuna, salì sopra il ciriegio. Il villano, veduto questo, tardi accortosi della sua
  balordaggine, piangendo il suo danno gridava: \textit{Addio fave}, cioè io ho perduto le
  fave, Vedi il Cornazzano\footnote{Antonio Cornazzano (Piacenza, 1430 circa – Ferrara, tra il 1483 e il 1484) scrittore e poeta} Novella 10. dove troverai questa favola non travestita,
  e meglio espressa.

\item[TRAVE] Legno grosso, e lungo, che s'adatta a reggere i palchi.

\item[ARGOMENTO] E' lo stesso, che Serviziale, o Cristero detto sopra in questo
  C. stan. 10. e 12. E qui torna bene, perché vuol medicarlo per via d' argumenti
  logici, ma di conseguenze spropositate.

\item[BOLLIRNE a sodo] Cioè bollire molto tempo, e gagliardamente.

\item[BRODO] Decotto di carne. Acqua ingrassata con carne. Se ben la parola
  brodo è comune a ogni sorta di decotto, o minestra, ancorché non di carne.

\item[I DVE contrarj fan che il terzo goda] \textit{Inter duos litigantes tertius gaudet}. Con
  questo argumento, e con questa sen\-ten\-za, e con altre ragioni da squartati, pretende
  l'Eccel\-lentissimo d' haver trovato il modo di fermare il flusso.

\end{description}

\section{Stanza XXIV \& XXV.}

\begin{ottave}
\flagverse{24}Ciò detto rivoltessi al mormorio\\
Di quell'ambrette, ov' a mestar si pose; \\
E, perch' elle sapevan di stantio,\\
Teneva al naso un mazzolin di rose. \\
Soggiunse poi: Costui vuol dirci addio, \\
Che queste flemme putride, e viscose \\
Mostran, che ben' affetto a gli artolani \\
Ei vuol' ire a ingrassare i Petronciani.
\end{ottave}

\begin{ottave}
\flagverse{25}In quel che questo capo d'assivolo\\
Ne dice ogni or dell'altra una più bella,\\
Tosello Gianni, il quale è un buon figliuolo\\
Mosso a pietà, con una sua coltella\\
Tagliate havea le rame d'un querciuolo,\\
Sopr' alle quali a foggia di barella\\
Fu Paride da certi Contadini\\
Portato a' suoi poder quivi vicini.
\end{ottave}

L'Eccellentissimo Dottore, dopo haver fatte le suddette belle ordinazioni, si
mette a stuzzicare quella materia, e da quel puzzo fa pronostico, che il pazziente
sia per morire; e l'argumento, che egli fa di tal morte non è dissimile dalle
ricette. In canto Tosello Gianni accomodò una barella, sopr' alla quale Paride
fu posto, e portato da certi contadini ad una villetta de' Signori Parigi vicina a
Malmantile in luogo detto Santo Romolo; nella qual Villa trovandosi l'Autore
concepì nella mente il far la presente Opera, come dicemmo sopra nel Proemio.

\begin{description}
\item[AMBRETTA] Così chiamiamo guanti, ed altre pelli conciate con odore
d'ambra. Ma qui intende, ironicamente parlando, quella materia fetida.

\item[SAPEVA di stantio] Haveva cattivo odore. Quando una materia per la lunghezza
  del tempo ha cominciato a perdere la sua perfezione, si dice \textit{stantia}; che
  se sia carne, o pesce, non da troppo buono odore; e queste si dice \textit{puzzo di stantio},
  La qual voce viene da stanziare lungo tempo, ed è il Latino \textit{obsoletus}. Vedi
  sotto in questo C. stan. 54.

\item[VUOL dirci addio] Se ne vuol'andare. Ci vuol lasciare, cioè vuol morire.

\item[FLEMMA] Vmor freddo, e umido che i Medici chiamano in Pituita, e comunemente
  si dice hemma dal Greco.

\item[VUOL' andare a ingrassare i Petronciani] Vuol andare a ingrassare gli orti col
  suo corpo, facendoli sotterrare; e piglia \textit{Petronciani} (che vedemmo in
  questo C. stan. 6, quello che sieno) per tutto l'orto. E nota che per autenticare la
  castroneria di questo Medico, l'Autore gli fa dedurre il pronostico della morte
  di Paride dal credere, che il suo corpo sia già corrotto, e ridottosi tutto in quel
  la terza putrida sustanza, ed in conseguenza atto, ed il caso a ingrassare i terreni;
  E vuol dire, che Paride morrà: Dicendosi vulgarmente per intender questo
  \textit{Il tale andò a ingrassare i cavoli}, cioè il tale morì.

\item[CAPO d'assiuolo] A uno ignorante si dice Capo di Bue, Capo di Castrone,
  Capo d'assivolo, e simili, L' \textit{assiuolo} è un'uccello in tutto simile alla Civetta, se
  non che ha sopra il capo, alcune penne ritte, che sembrano corna.

\item[TOSELLO Gianni] Agostino Nelli Gentil' huomo Fiorentino buon letterato,
  e veramente huomo da bene, Che intendiamo \textit{buon figliuolo}.

\item[COLTELLA] Specie di scimitarra, Arme che s'usa portare, quando si va a  caccia.

\item[BARELLA] Arnese fatto di tavole, che ha quattro manichi, serve per portar
  sassi, e altri pesi in due persone; qui intende una barella da portare i corpi
  d'huomini infermi, o morti, che è simile alle bare, o cataletti co i quali si soglion
  portare detti corpi, e da Bara e chiamata barella. Vedi sotto in questo C. stan. 44.

\end{description}

\section{Stanza XXVI.}

\begin{ottave}
\flagverse{26}Fu del Garani ascritto successore \\
Puccio Lamoni anch'ei grand'ingegnere, \\
Bravissimo Guerrier saggio Dottore, \\
Cortigiano, Mercante, e Taverniere, \\
Dicon ch' ei nacque al tempo delle more,\\
Per ch'egli è di pel bruno, e membra nere,\\
Hor qua di Cartagena eletto Duce,\\
Il fior de' Mammaganuccoli conduce.
\end{ottave}

Al Garani fu dato per successore Puccio Lamoni, il quale è Paolo Minucci.
Il Poeta dice che costui era ingegnere, e Mercante; ma tali attributi gli sono finti,
perché io posso giurare, che egli non sa ne dell'una, ne dell'altra professione.
Lo chiama guerriero, e questo perché detto Puccio fece una campagna
nell'esercito Pollacco in Prussia, seguitando quella Real Corte, alla quale era
stato inviato dal Serenissimo Principe Mattias di Toscana alla Maestà del Re Gio:
Casimiro. E perché detto Puccio godé per molti anni, e fino che S.A. visse,
l'honore di servire all' A.S. in qualità di Segretario, però dice che era Cortigiano.
Dice che è Dottore perché veramente egli e addottorato in Legge, se bene
per l'applicazione alla corte, non esercitò tale professione. Lo chiama Taverniere,
perché spesso lo vedeva entrare nell'Osterie, e trattare con Osti, il che
seguiva perché egli vendeva loro del vino raccolto nei suoi beni, e gli conveniva
lasciarsi rivedere spesso per risquoterne il prezzo. Dice che si vocifera, che
gli nascesse al tempo delle more, Perch'egli è di pel bruno, e membra nere, essendo
egli così in effetto: E facendolo Duca di Cartagena dice, che egli conduce \textit{il
fiore de' Mammagnuccoli}, cioè i migliori, e più valorosi \textit{Mammagnuccoli}. Questi
Mammagnuccoli erano una conversazione di galant' huomini, i quali facevano
professione di sapere il conto loro in ogni cosa, e particolarmente nel giuocare,
e spendere bene il lor danaro, e d'essere il fiore della reale, ed onorata
scapigliatura. Havevano un loro capo, che si chiamava Abate, dal quale erano
galtigati, quando facevano qualche errore o nel giuocare, o nello spendere, ma
però tutto era in galanteria. Le loro adunanze si facevano in casa l'Abate, dove
si giuocava a giuochi più di spasso, che di vizio, e si facevano altre allegrie,
di cene, merende, ed altri passatempi. Costoro erano tutte persone serie, e
quiete, e della più riguardevole Civiltà, e perciò era la lor conversazione molto
bramata, onde era numerosissima; Se bene non era ammesso a quella veruno, che
non havesse provata prima la sua dabbenaggine, e non fusse stato riconosciuto
dal Abate, e da altri suoi Consiglieri meritevole d'essere ammesso. Fra costoro
era detto Puccio, e perché egli era forse de' più affezionati, i1 Poeta lo fa loro
Condottiero, e per la stima che faceva di lui nel giuoco delle Minchiate, era solito
chiamarlo il Re delle carte; perciò lo fa Duca di Cartagena, ed è ancora appropriato,
perché detto Puccio per esser di faccia bruna, ha qualche sembianza,
ed aria di Spagnuolo; oltre che nel tempo, che l'Autore lo aggiunse a questa sua
Opera, il detto Puccio, era stato destinato dalla Maestà del Re Gio: Casimiro
per suo Segretario dell' Ambasciata di Spagna.

\section{Stanza XXVII.}

\begin{ottave}
\flagverse{27}L' Armata havea tra gli altri un Cappellano\\
Dottor, ma il suo saper fu buccia buccia,\\
Pero ch' egli studiò col fiasco in mano,\\
Ed era più buffon d' una Bertuccia,\\
Faceva da Pittor, da Tiziano;\\
Ma quant'ei fece mai n'andava a gruccia,\\
Hebbe una Chiesa, e quivi a bisca aperta\\
Si giuocò fino i soldi dell'offerta.
\end{ottave}

\begin{ottave}
\flagverse{28}Franconia si domanda Ingannavini, \\
E fu pregato come il più valente, \\
Perch' egli sapea leggere i Latini, \\
A far quattro parole a quella gente, \\
Egli c' havea in casa it Coltellini,\\
Già fatta una lezione, e falla a mente, \\
Subito accetta, e siede in alto solio\\
Senza mettervi su ne sal, ne olio.
\end{ottave}

Fra gli altri Cappellani, che erano nell' Armata, era un Dottore, ma di poca
scienza; perché il suo studiare era stato il darsi bel tempo. Fu scolare dell'Autore
nella pittura, ma imparò poco, e se bene si presumeva di saper molto, non
fece mai cosa, che non fusse stroppiata. Fu Rettore della Chiesa di Petriolo;
Villaggio vicino a Firenze circa due miglia, e perché egli era huomo allegro, e
di conversazione, dice che egli si giuocò fino i soldi dell'offerta, ed intende che consomava
tutte le sue entrate in allegrie. Il suo nome era Franconio Ingannavini,
cioè Giovannantonio Francini. A questo dunque, come al più dotto fu fatta instanza,
che facesse un poco di discorso a quei Soldati, ed'egli che haveva un
tempo fa recitata una lezione nell'Accademia del Coltellini, e l'haveva ancora
a memoria, si contentò di fare quanto gli era stato imposto, e senza mettere più
tempo in mezzo montò in pulpito.
\begin{description}
\item[BUCCIA buccia] Leggiermente. Cioè sapeva poco; non haveva gran fondamento;
  che si dice anche \textit{in pelle in pelle}. Vedi sotto C, 8, stan.58. ed i Latini
  dissero \textit{superficie tenus}.

\item[PIÙ buffone d' una bertuccia] Huomo arguto, allegro, e faceto. \textit{Buffone} diciamo
  colui, che tiene il popolo allegramente con facezie, e moti, è il Latino
  \textit{Scurra}, Vedi sotto C, 11. stan. 42. E \textit{Bertuccia} diciamo la scimmia.

\item[TIZIANO] Pittore celeberrimo. E con dire \textit{facea da Tiziano}; intende per
  antonomasia, che egli si presumeva d'esser il più valente Pittore del Mondo.

\item[QVANT' ei facea, n'andava a gruccia] Tutto quel che egli faceva, era stroppiato,
  cioè mal fatto, mal dipinto, Vedi sotto C.~11. stan.~41.

\item[BISCA] Luogo pubblico, dove è permesso giuocare a ognuno; \textit{E giuocare a
  bisca aperta}, vuol dire Giuocar sempre, e senza riguardo alcuno.

\item[IL Coltellini] Questo è il Signor Agostino Coltellini\footnote{Agostino Coltellini (Firenze, 17 aprile 1613 – Firenze, 26 agosto 1693), accademico e letterato. } Avvocato Fiorentino huomo
  dotto, ed amatore de i Letterati, il quale in molte opere composte da lui si
  chiama col nome anagrammatico Ostilio Contalgeni. In casa di esso si raguna
  l'Accademia degli Apatisti da esso fondata, nella quale si fanno discorsi Accademici,
  ed altri esercizzj virtuosi. Mirabile per haver saputo far durare per lo spazio
  di cinquanta, e più anni la detta Accademia, sempre in florido, cosa insolita
  a' nostri secoli in questa Città. Interveniva spesso in detta Accademia questo
  Francini, ed alle volte vi faceva qualche lezione; nelle quali mostrò i suoi dotti
  ed eruditi talenti, e se bene l'Autore dice che il suo sapere fu \textit{buccia buccia}, e sotto
  lo chiama huomo senza fondamento, non è però, che egli fusse tale, anzi fra
  gli huomini de' nostri tempi non era dei secondi in dottrina non meno sagra,
  che profana; ed era veramente Dottore di legge.

\item[SENZA mettervi su ne sal, ne olio] Presto, subito, senza replicare, o metter
  difficultà, \textit{Nulla interposita mora}. Fu un tale, che tornato la sera a casa, disse al
  suo servitore: Fammi una insalata, e fa presto, ch' io sono aspettato, e non
  voglio mangiare altro che quella; fa presto. dico. Il servitore presa l'insalata
  senza condire la portò in tavola al padrone; il quale ciò visto lo scridò; Ma il
  servitore rispose; Signore per servirvi presto, non vi ho messo su ne sale, ne olio.
  E da questa goffaggine del servitore viene il presente detto, che significa Fare una
  cosa subito, e senza considerazione.
\end{description}

\section{Stanza XXIX.}
\begin{ottave}
\flagverse{29}Sale in Bigoncia con due torce a vento, \\
Acciò lo vegga ognun pro tribunali, \\
Ove, mostrar volendo il suo talento, \\
Fece un discorso, e fece cose tali, \\
Che ben si scorse in lui quel fondamento,\\
Che diede alla sua casa Giorgio Scali,\\
E piacque sì, che tutti di concordia\\
Si messero a gridar: misericordia,
\end{ottave}

Il Poeta continuando, a voler mostrare, che Franconio fusse di poco valore,
e che però il discorso da lui fatto fusse scimunito, e senza alcun fondamento, lo
burla, e dice che piacque tanto, che il popolo, si messe a gridar \textit{misericordia}; del
qual termine ci serviamo per mostrare, che qualche cosa ci sia venuta a fastidio,
come per esempio. \textit{Ei durò tanto a discorrer, che misericordia}, \textit{Disse tante
  scioccherie, che misericordia}, \textit{Oh misericordia, quanto volete voi durare?} Quali dica,
habbiate misericordia, e compassione di noi, e non ci tediate più,

\begin{description}
\item[BIGONCIA] È un vaso di legno, del quale si servono i Contadini in tempo
  di vendemmia per pigiarvi dentro l'uva, prima di metterla nel tino, e ce ne serviamo
  anche in altre occorrenze, come di portar' acque, e simili.

Il Bini nel Capitolo del Pilo\footnote{Capitolo 29 dal libro ``Le terze rime de messer Giovanni della Casa, di messer Bino, ed altri'', pubblicato per Curtio Navo, senza imprimatur, nel 1532.} dice:
\begin{verse}
  Vuo dir, che se ben' ella il pil mi desse,
  Ed oprassi (non ch'altro) una bigoncia,
  Ognun direbbe, che ben fatto havesse.
\end{verse}
 E perché questo vaso detto Bigoncia è molto simile a una cattedra tonda, però
da molti tal Cattedra si chiama \textit{bigoncia}, come anche tutte l'altre cattedre. Il
Davanzati\footnote{Bernardo Davanzati, 1529 - 1606.} nel suo Cornelio Tacito postille al 2. libro num. 18. dice: \textit{Arringavano
i nostri antichi al popolo in piazza, in ringhiera, e nei Consigli in bigoncia, che
era un pergamo in terra a foggia di bigoncia}.

\item[TORCE a vento] Torce grosse che si fanno di funi di cotone filato attorte per
servirsene a far lume la notte per le strade; e si dicono \textit{a vento}, perché resistono
al vento; e a distinzione di quelle, che si fanno a Venezia, che per esser gentili
si spengono a ogni poco di vento. E \textit{Torcia}, che da i Latini e detta \textit{funalia},
\textit{funalium}, viene a noi dal Francese \textit{Torche}.

\item[CHE diede alla sua casa Giorgo Scali] Giorgio Scali\footnote{Giorgio degli Scali, uomo politico associato alla Rivolta dei Ciompi, 1378, fu arrestato il 16 gennaio 1382 e giustiziato il giorno seguente.} fu in Firenze un riputatissimo
  Cittadino Popolano, il quale nelle dissenzioni, che seguirono a suo tempo
  fra i nobili, e Popolani di Firenze, si fece capo di questa parte, con promessa, e
  speranza d' esser sollevato a cose maggiori, cioè all' assoluto dominio di Firenze,
  e benché per altro accortissimo, e prudentissimo, lasciatosi portare dal dolce desiderio
  di dominare, si fidò nelle vane promesse della instabil plebe, con la quale
  parendogli haver forze bastanti per conseguire l'intento, s' accinfe all' opera;
  ma nel più bello il popolo, o spaventato, o pentito l'abbandonò, ond' egli
  venuto in potere del Governo fu decapitato: E da lui e detto il Proverbio: \textit{Far
  come Giorgio Scali}, che vuol dir Pigliare a far' una cosa senza fondamento, che i
  Latini con similitudine della Scrittura, dissero \textit{Scipione arundineo inniti}, Di questo
  caso di Giorgio Scali parlano tutti gli Storici, che scriveno le cose di
  Firenze di quei tempi, ed il Nerli fra gli altri aggiunge, che allora cominciò questo
  proverbio.
\end{description}

\section{Stanza XXX \& XXXI.}

\begin{ottave}
\flagverse{30}Il tema fu di questa sua lezione, \\
Quand' Enea già fuor del suo pollaio \\
Faceva andar in fregola Didone, \\
Com' una gatta bigia di Gennaio;\\
E che se i Greci ascosi in quel ronzone \\
In Troia fuoco diedero al pagliaio, \\
E in man a Enea posero il lembuccio, \\
Ond' ei fuggi col padre a cavalluccio;
\end{ottave}

\begin{ottave}
\flagverse{31}Così, dicea, la vostra, e mia Regina\\
Qui viva, e sana, e della buona voglia,\\
Cacciata fu dal empia concubina\\
Tre dita anch' ella fuor di questa soglia;\\
Però s'un tanto ardire, e tal rapina\\
Parvi, e' adesso gastigar  si voglia,\\
V'havete il modo senza ch'io lo dica.\\
Io ha finito. Il Ciel vi benedica.
\end{ottave}

Il tema del discorso, che fece Franconio, fu quando Enea esseno fuggito da
Troia fece innamorar Didone, 'ed assomigliando Celidora cacciata di Malmantile
ad Enea scappato da Troia, esorta quei soldati a gastigar l'ardire di Bertinella,
e rimettere Celidora nel suo stato, già che hanno il modo.

\begin{description}
\item[POLLAIO] Si dice da noi quella stanza, nella quale stanno, e dormono i polli:
  E chiamiamo pollaio quelle selve, o macchie, dove la sera vanno gli uccelli
  a dormire; Ma qui intende per translato la nostra Casa, Patria, o luogo, dove
  siamo soliti abitare.

\item[ANDARE in fregola] Dicemmo quel che significhi sopra-C. 1. stan. 25, Ma
  che Didone fusse innamorata d' Enea, come favoleggia Vergilio, è falsità, perché
  oltre che Didone fu così casta, che vedendosi violentata da Iarba Re di Mauritania
  a rimaritarsi seco, volle più tosto da se stessa uccidersi, che offendere il
  suo morto marito Sicheo con nuovi sponsali'; È anche vero, che non potette
  seguire il detto innamoramento, perché Enea fu 360. anni prima di Didone; Tal
  verità si cava da diversi Autori, e si scorge in Darete Frigio', e Ditti Cretense,
  che scrissero la vera Storia dell'eccidio di Troia. Che il nostro Dante poi seguiti
  questa bugia di Vergilio, dicendo nell' Inf. C. 5.
  \begin{verse}
    L' altr' è colei, che s'ancise amorosa,
    E roppe fede al cener di Sicheo.
  \end{verse}
  Non è meraviglia, perché Dante s'era eletto per suo Maestro, e guida Vergilio.
  Che Enea fusse tanto tempo avanti a Didone, si deduce anche dal sapersi, che
  Didone fuggendo l'insidie di Pigmalione suo fratello, che per desiderio di tesoro
  le haveva ammazzato il marito Sicheo, come pure accenna Dante, Purg. C. 20.
  \begin{verse}
    Noi ripetiam Pigmalione allotta,
    Cui traditore, e ladro, e patricida
    Fece la voglia sua dell oro ghiotta.
  \end{verse}
  Portandosene il tesoro in Affrica, chiese a quegli abitatori tanto di terreno
  quanto poteva circondare una pelle di toro, e l' ottenne; Ed astutamente tagliò la
  detta pelle in strisce così sottili, che abbracciò con esse tanto terreno, che vi
  edificò Cartagine, il che fu dopo 70. anni della edificazione di Roma, \textit{la quale fu
    edificata circa 300, anni dopo la morte d' Enea}, Sant' Agostino disse in difesa di
  Didone, che quando Vergilio non fusse stato dannato per altro, meritava l'Inferno
  per questa falsità cotanto pregiudiciale alla ripucazione di Didone, la quale
  difende ancora Ausonio col seguente Epigramma tradotto dal Greco.

  {\centering Ad Didus Imaginem CXI.\\}
  \begin{verse}
\backspace Illa ego sum Dido, vultu quam consipicis hospes,
Assimilata modis pulcraque mirificis:
\backspace Talis eram, sed non Maro quam mihi finxit erat mens,
Vita nec incestis Laeta cupidinibus.
\backspace Namque nec AEneas vidit me Troius unquam,
 Nec Lybiam advenit Classibus Iliacis;
\backspace Sed furias fugiens, atque arma procacis Iarbae
Servavi, fateor, morte pudicitiam
\backspace Pectore transfixo, castos quod pertulit enses,
Non furor, aut laso crudus amore dolor.
\backspace Sic cecidisse iuvat; Vixi sine vulnere fama;
Vita virum, positis moenibus oppetii.
\backspace Invida cur in me stimulasti musa Maronem,
Fingeret ut nostrae damna pudicitia ?
\backspace Vos magis Historicis lettores credite de me,
 Quam qui furta Deum concubitusque canunt;
\backspace Falsidici Vates, temerant qui carmine verum,
 Humanisque Deos assimilant virijs,
\end{verse}

\item[GATTA bigia] È quella, che noi chiamiamo Soriana, che è un misto di color
  bigio, e lionato serpato di nero, qual colore soriano si dice solamente di Gatti,
  onde io argumento, che i primi gatti di questo colore venissero a noi di Soria,
  come vennero alcuni anni addietro quelli del colore del topo portati da Pietro
  della Valle dalla Persia, e però da molti chiamati Persianini. Vedi sotto C. 9.
  stan. 19.

\item[RONZONE] Con la 'Z' cruda\footnote{sorda - \t{ts}} vuol dir Cavallo stallone, o per la monta, da
  i Latini detto \textit{equus admissarius}; e per ronzone, ronzine, o rozza intendiamo
  cavallo cattivo, Ronzone con la 'z' dolce\footnote{sonora - \t{dz}} vuol dire una specie di Moscone, o
  tafano. Qui l'Autore intende quel cavallo di legno fabbricato da i Greci per ingannare
  i Troiani come dice Vergilio. In alcuni Testi si trova scritto \textit{cassone} invece
  di \textit{ronzone}, ma nel mio, che è di mano dell'Autore, è scritto \textit{ronzone}.

\item[PAGLIAIO] È proprio quel cumulo, o massa di paglia, che si fa da i Contadini
  dopo haver battuto il grano, per lo più avanti alle case; ma dicendosi dar
  fuoco al pagliaio, s' intende Dar fuoco alla Casa.

\item[PORRE il lembo]; o il lembuccio \textit{in mano}, Significa Mandar via uno; E questo,
  perché quand' altri vuol mandar via uno di qualche luogo senza parlare, gli fa
  il ferraiuolo addosso, e gli mette un lembo di esso (che \textit{lembo} vuol dire
  Una parte dell'estremità del ferraiuolo, o d'altro abito, o veste simile) nelle
  mani; e da questo colui s'accorge d' esser licenziato, essendo notissimo, che
  questo detto \textit{Pigliare, o dare il lembo} significa Esser licenziato; Tratto dai maestri
  delle botteghe, i quali, volendo licenziare un garzone, gli dicono: piglia il
  lembo; piglia il cencio, ec. e intendono Vattene.

\item[A CAVALLUCCIO] Cioè in su le spalle. E noi diciamo portare \textit{a cavalluccio}
  da un giuoco, che fanno i nostri ragazzi in questa forma. Vno mette il capo fra
  le gambe all' altro per di dietro, e sollevatolo così da terra, lo porta fra le spalle,
  e il collo, e per questo si dice, \textit{a cavalluccio}. I ragazzi Greci, che pure lo
  facevano lo dicevano \textit{in cotyla}, perché facevano porre le ginocchia del portato
  sopr' alle palme delle mani del portatore rivoltate dietro alle reni, ed il portato
  non accavalciava le gambe al collo, come fanno i nostri, ma con le braccia s'atteneva
  al collo del portatore; e lo dicevano \textit{in cotyla} dalla palma, o cavo della
  mano di colui, che portava, come si cava dal Buleng.\ de lud.\ vet.\ cap.\ 20.\ e
  da Cel.\ Rodig.\ lect.\ antiq.\ lib.27.\ cap.27. E questo era più tosto, che giuoco,
  una pena data a quei fanciulli, che haveano perso a qualche altro dei loro
  giochi, che habbiamo accennati sopra nel 2.\ Cantare. E si come erano varj i
  modi, con li quali portavano, così erano diversi i nomi, che davano a questo
  giuoco; perché si trova chiamato \textit{Cubesinda}, ed \textit{Hippas}, si come si vede in Giulio
  Polluce lib. 9. c. 7. Che questo giuoco fusse usato anche dai Latini, si può dedurre
  da Vergilio En.\ lib.\ 2.\ il quale dice che Enea portò il Vecchio Anchise suo
  padre in su le spalle in tal maniera.
  \begin{verse}
    Ergo age chare pater cervici imponere nostrae
    Ipse sibibo humeris, nec melabor iste gravabit.
\end{verse}

\item[DELLA buona voglia] Intendiamo sano, allegro, e con buona speranza. Il
  Lalli En. Trau. lib. 1, stan. 51. disse:
  \begin{verse}
    Stanne, diletta mia, di buona voglia.
  \end{verse}

Parafrasando Vergilio, dove dice: \textit{Parce metu}, E noi diremmo: Non dubitare.

\item[FUOR di questa soglia] Cioè fuori di Malmantile, Piglia la soglia, che è la
  parte di sotto della porta, per tutto Malmantile; o intende soglia per soglio
  reale.
\end{description}

\section{Stanza XXXII \& XXXIII.}

\begin{ottave}
\flagverse{32}Poiché da esso inanimite furo\\
Le schiere, si portaron a i lor posti,\\
E già sdraiato ognun lasso, e maturo\\
In grembo al sonno gli occhi haveva posti,\\
Quand'a un tratto le trombe, ed il tamburo \\
Roppe i riposi, e i sonni appena imposti;\\
Ma svanì presto così gran fracasso, \\
Ch'il fiato a i trombettier scappò da basso.
\end{ottave}

\begin{ottave}
\flagverse{33}E questo cagionò, che incollorito\\
Il Generale di cotanta fretta,\\
Con occhi torvi minacciò col dito,\\
Mostrando voler farne aspra vendetta\\
Seguì c'un' Ufizial suo favorito,\\
Che più d'ogn'altre meno se l'aspetta\\
Toccò la corda con i suoi intermedi\\
De' tamburini, e trombettieri a piedi
\end{ottave}

Dopo che Franconio hebbe dato animo a i soldati ognuno andò a quartiere,
e già tutti stracchi s'erano addormentati, quando in un subito fu dato nelle
trombe, e ne i tamburi, che fecero svegliare tutta la soldatesca; ma questo romore
presto cessò, perché i trombettieri, e tamburini lasciarono star di sonar per la
paura, che hebbero del Generale, il qualee entrato in collera di così gran fretta
giurò di voler gastigar colui, che era stato il capo di al sollevamento, e lo mandó
ad effetto, facendo dare la corda a uno Vfiziale suo favorito, che non se lo
sarebbe mai aspettato, e gli fece mettere i tamburini, e i trombettieri a piedi.
\begin{description}
\item[SDRAIATO] Disteso con comodità. Voce usata da noi per esprimere la
  consolazione, che sente uno, che sia stanco a distendersi con comodità e spensieratamente.
  Vedi sotto C. 6. stan.26. E non crederei d'errare, se dicessi \textit{sdraiato}
di Cerbero, parafrasando Vergilio; dove dice
\begin{verse}
  \makebox[4em]{\dotfill} Atque immania terga resoluit
  fusus humi, totoque ingens extenditur antro.
\end{verse}

\item[A VN tratto] In un subito. E questo termine \textit{a un tratto} significa anche tutti
  due, o più alla volta, e si può intender, che le trombe, ed i tamburi, cioè uno,
  e gli altri svegliassero.

\item[CASCÒ il fiato da basso a' trombettieri] Cascare il fiato vuol dire Haver paura,
  o timore; onde con questo dire intende, che i trombettieri hebbero paura del
  Generale, e perciò lasciarono di sonare; non perché veramente perdessero, o
  uscisse loro il fiato dalle parti da basso.

\item[INCOLLORITO] Adirato. Entrato in collora.

\item[OCCHIO torvo] Frase latina; usata da noi, e significa, e mostra l'ira che
  uno habbia; e dicendosi: il tale mi guarda con mal' occhio, o con occhi torti,
  s'intende il tale è adirato meco: \textit{Haec autem toruitas a taurorum ferocia dicitur}.

\item[MINACCIÒ col dito], Coloro che vogliono gastigare qualche delitto, o vendicarsi
  d' alcuna ingiuria, sogliono brandire il dito indice verso quel tale, che
  vogliono gastigare, e tal brandimento si dice \textit{minacciare} dal Latino Minari, o
  \textit{minitari}.

\item[CHE più d' ogni altro meno se l'aspetta] Per esser questo soldato amico, e molto
  in grazia al Generale; non havrebbe mai creduto, che egli l'hauesse a gastigare,

\item[TOCCÒ la corda] In Firenze danno la corda legando il paziente per le mani
  legate insieme dietro alle reni; e per quelle appiccate a un grosso canapo, che
  passa per una carrucola, tirano il paziente in su, lasciandolo leggiermente scorrer
  in giù, e poi ritirandolo in su tante volte, a quante è condennato, e questo diciamo:
  \textit{dare tratti di corda}. Qual tormento da i nostri antichi era detto \textit{dar la
    colla}, o \textit{collare}, e noi diciamo: \textit{dare la corda}. Soggiunge poi: \textit{Co' suoi intermedj di
    tamburini, e trombettieri a' piedi}; cioè con tutto quello che ci andava; il che era,
  che i tamburini, e i trombettieri, i quali erano stati complici a tal delitto, stessero
  quivi a pié di lui assistenti a vedere eseguire la giustizia, come si costuma,
  quando molti sono complici d' un delitto, per lo quale vien gastigato severamente
  il capo principale, e gli altri complici ricevono minor gastigo, ed assstono a
  vedere il gastigo del loro principale. Io però non sono lontano dal credere, che
  il Poeta per sostenere questa sua Opera sempre in su le burle, habbia voluto
  intendere, che i tamburini, e trombettieri fussero effettivamente legati a i piedi di
  colui, che era tirato su, e voglia mostrare con questo il costume, che si tiene in
  Firenze di legare a' piedi di tali pazienti qualche cosa, che significhi il delitto da
  lui commesso, acciò che il popolo comprenda la cagione di quel martirio, come
  per esempio: a un fornaio, che habbia fatto il pane cattivo, o di minor peso del
  dovuto, faranno legare a' piedi un filo di pane, e così gli daranno la corda: e mi
  lascio indurre a creder, che il Poeta habbia voluro intender questo, dal vedere,
  che egli nell'Ottava seguente dice: \textit{alla corda vuole che sia attaccato così}: i qual
  detto pare che esprima, che il paziente debba toccare la fune co'i trombetti, e
  tamburini legatigli a i piedi.
\end{description}
\section{Stanza XXXIV \& XXXV.}

\begin{ottave}
\flagverse{34}Alla corda così vuol che s' attacchi,\\
Perché d'arbitrio, e senza consigliarsi,\\
Facea venir all'armi, allor che stracchi\\
Bisogno havean più di riposarsi,\\
Ed eran mezzi morti; e come bracchi,\\
Givano ansando inordinati e sparsi,\\
E con un fuor di lingue, e orrenda vista\\
Soffiavan, ch'io ho stoppato un Alchimista.
\end{ottave}

\begin{ottave}
\flagverse{35}Amostante non solo era sdegnato,\\
Che di suo capo, e propria cortesia\\
Senza lasciar, che l'huom riabbia il fiato,\\
Ei volesse attaccar la batteria;\\
Ma perché seco havea concertato,\\
Ch'egli stesso, che sa d'astrologia,\\
Vuol prima, ch'il nimico si tambussi,\\
Veder ch'in Gielo sien benigni influssi.
\end{ottave}

ll Generale fece dar la corda a quell' Vifiziale non solo, perché egli s'era
preso l'arbitrio di far dar' all' armi senza il suo consenso ma ancora perché era
uscito fuori del concertato, il quale era di osservare prima di muovere il campo,
se le stelle presagivano buona, o trista sorte. E qui il lettore si ricordi, che si
sta in su le burle, e sappia, che l'Autore non stimava che l'astrologia arrivasse
a tanta precognizione, ma si bene, che Habeant sua sydera lites, come dicono i
legisti.
\begin{description}
\item[D'ARBITRIO, e propria cortesia] Suonano lo stesso; ed ambedue significano
Di suo capriccio, o volontà.

\item[ANSARE] È quell'impeto, o romore, che fa il respiro, quando si ripiglia
il fiato (che noi pure dal Latino diciamo \textit{anelare}) e viene a \textit{Ansima} Gr. \textit{Asthma}.

\item[BRACCO] Cane per uso di caccia, il quale quando è stracco respira con gran
veemenza, e tiene la lingua fuori; E se bene fanno così tutte le specie di cani, è
nostro solito far questa comparazione solamente ai bracchi, perché questi veramente
sono più sottoposti a straccarsi;  percio che stimolati dal naturale desiderio
di trovar preda, fanno maggiore, e più violento viaggio che gli altri cani.
Persio Sat, 1. \textit{Nec linguae quantum sitiat canis Appula tantum}.

\item[ORRENDA vista] Vista spaventevole; che tale è il veder un'huomo con la
bocca aperta, e con la lingua fuori, perché per lo più restano in questa forma
gl' impiccati.

\item[SOFFIAVAN ch'io ho stoppato un' Alchimista] Alchimisti son coloro, che soffiano
nel fuoco per trovar l'oro, e senza nominare Alchimista, col solo dire: \textit{il tale
soffia} s'intende, è Alchimista, Se bene s' intende anche Fa la spia, come
accennammo sopra C 1, stan. 37. anzi dicendosi \textit{Il tal fa l'Alchimista}, s'intende il
tale fa la spia, e tutto è fondato sul verbo soffiare, che significa \textit{Far la spia}.

\item[IO ho stoppato] Significa io stimo meno, o io non stimo punto il soffiare, che
fanno gli Alchimisti in paragone di quello, che soffiavano questi soldati. Ha lo
stesso significato, che il termine ne disgrado detto sopra C. 1. stan. 51. e che vedremo sotto C. 6. stan. 61.

\item[TAMBVSSARE] Perquotere, dar delle fusse. È parola oggi propria de i
macellari, che dicono Tambussare quando bastonano le bestie morte e gonfiate,
perché la pelle si spicchi bene dalla carne, e dicono anche Tamburare, come vedremo
sotto C.\ 11.\ stan\ 26. E tutto ha Origine dal tamburo, perché il romore,
che fa esso, s'assomiglia al romore, che fanno i macellari.
\end{description}

\section{Stanza XXXVI. — XXXIX.}

\begin{ottave}
\flagverse{36}Homai la Fama, che riporta a volo\\
D'ogn' intorno nuove, e le gazzette,\\
Sparge per Malmantil, che armato stuolo\\
Vien per tagliare a tutti le calzette,\\
Già molti impauriti, e in preda al duolo\\
Non più co i nastri legan le scarpette,\\
Ma con buone, e saldissime minuge,\\
Perché stien forti ad un \textit{rumores fuge}.
\end{ottave}

\begin{ottave}
\flagverse{37}In tal confusione, in quel vilume,\\
All' udir quei lamenti, e quegli affanni\\
A molti ch' eran già dentr' alle piume\\
Lo sbucar fuori parve allor mill anni:\\
Chi per vestirsi riaccende il lume,\\
Però ch' al buio non ritrova i panni,\\
Chi nudo scappa fuori, e non fa stima,\\
Che dietro gli sia fatto lima lima.
\end{ottave}

\begin{ottave}
\flagverse{38}Perché s'egli ha camicia, o brache, o vesta,\\
Non bada che gli facciano il baccano;\\
Ben si del triste avviso afflitto resta,\\
Onde più d' un poi giuoca di lontano,\\
Chi torna indietro a fasciarli la testa,\\
E chi si tinge con il zafferano,\\
Chi dice, c' una doglia sergli è presa,\\
Per non haver a ire a far difesa.
\end{ottave}

\begin{ottave}
\flagverse{39}Altri, che fugge anch' ei simil burrasca,\\
Finge l' infermo, e vanne allo spedale,\\
E benché sano ei sia come una lasca,\\
Col medico s' intende, e col speziale,\\
Perché all'uno, ed all'altro empie la tasca,\\
Acciò gli faccia fede ch' egli ha male;\\
Ed essi questo, e quel scrivon malato,\\
E chi più da, do fan di già spacciato.
\end{ottave}

Sparso per Malmantile l'avviso dell'arrivo di detta Soldatesca, gli abitatori
di quel luogo s' accinsero più al fuggire, che al difendersi. Narra il Poeta diversi
effetti di tale spavento, e le varie scuse, ed invenzioni, che trovano coloro
per non haver ad andare alla difesa della muraglia.

\begin{description}
\item[GAZZETTE] Novelle, Avvisi, Carte d'avvisi. E \textit{gazzetta} diciamo anche
  la crazia, Veneziana.

\item[TAGLIAR le calzette] Tagliar le gambe. E s' intende, dare delle ferite in
  qualsisia luogo del corpo, se ben le calzette non vestono se non le gambe: Come
  diciamo anche rompere la testa, ed intendiamo Ferire il nimico in quelle parti
  del corpo che ci verrà fatto. E diciamo \textit{fiaccar le braccia a uno con le bastonate}, se
  bene in ogni altra parte gli daremo che nelle braccia.

\item[NASTRO] È una specie di tela, o benda che non eccede la larghezza d' un
  sesto di braccio, e serve per legare, o fasciare;  da i Latini però detto, \textit{Vitta}, ed
  in alcuni luoghi d'Italia detto fettuccia.

\item[MINVGE] Corde da strumenti musicali come Tiorbe, Liuti, ec. fatte di budella
  di bestie; e pero Dante Inf. c. 28. per intender budella disse.
  \begin{verse}
    Tra le gambe pendevan le minugia.
  \end{verse}

  Dice che non si sono legate le scarpe coi nastri, ma con le minuge, perché sono
  più sode, e da resister più; ed è costume usatissimo il dire: \textit{Il tale s'era legato
  le scarpe bene, o con le minuge}, per intendere Correva forte, o volava: fuggendo
  i pericoli, che ciò intende con quella sentenza, \textit{Rumores fuge}.

\item[CONFUSIONE, e vilume] Sono in questo luogo quasi sinonimi havendo lo
  stesso significato di Viluppo, imbroglio, ec.

\item[DENTRO alle piume] Cioè nel letto.

\item[FAR lima lima] Beffare, dileggiare. è un modo proprio da Fasciulli, i quali
  quando vogliono dar la burla a uno, si fregano il dito indice sopra l'indice
  dell' altra mano a guisa di coloro che limano, e voltandosi verso colui, che voglion
  burlare dicono. \textit{Lima, lima}. Vedi sotto C. 9. stan. 66. annot.

\item[NON bada] Non cura; Non osserva, Non gl' importa'; Il verbo \textit{badare}, che
  vuol dire osservare, ha più significati, come Attendere, continovare, usare diligenza,
  curare, stimare, ec. Bada a tuoi negozzi. Bada a andare, Bada a chi
  viene. In somma ha la forza del Latino \textit{Curare}, \textit{Vacare}: si dice: \textit{Tener uno a
    bada}, per intender Trattenerlo. Star a bada d'uno: per intendere Stare aspettando
  l' opera, i favori ec. d' uno.

\item[BRACHE] Calzoni. Brache da noi propriamente si dicono quei calzoni larghi,
  che usano i Soldati a piede Tedeschi guardie del Serenissimo Gran Duca, ed
  i Paggi nobili. E si dicono talvolta Brache quei calzoni che si portano di sotto,
  chiamati ancora Mutande; Vedi sotto C.6, stan. 20.

\item[FAR il baccano] Qui vuol dir beffare, dileggiare con fischiate, o strida, o simili;
  ed il suo significato proprio è Fare strepito, far romore e viene da Bacchanalia.

\item[GIUOCA di lontano] Cioè non s'accosta: ed è lo stesso che \textit{starsene alla larga},
  che vedremo nell' ottava seguente.

\item[BVRRASCA] S'intende propriamente il travaglio del mare; ma lo pigliano
  per ogni sorta di sturbamento, o pericolo. Forse meglio borrasca da \textit{Boreas}.

\item[SPEZIALE] Colui che manipola, e vende medicamenti; e però da i Latini
  detto \textit{Pharmacopola}; ed altrimenti \textit{Aromatarius} da \textit{aromata}, e noi lo diciamo
  \textit{Speziale} da spezierie, come si trova anche in Latino Barbaro \textit{speciarius}.

\item[TASCA] Scarsella, che è un sacchetto appiccato a i calzoni, o altre vesti per
  uso di tenervi dentro quello, che occorra alla giornata, e particolarmente danari;
  è il Latino \textit{marsupium}. Ed \textit{empier le tasche a uno}, vuol dire Dargli molto
  danaro.

\item[LO fanno spacciato] Cioè dicono, che egli è in grado di morte. Intende il Poeta,
  che i Medici regolando le attestazioni delle infermità con le somme de i danari,
  che erano lor date, facevano fede esser in grado di morte quello, che più
  ne dava; e quel che ne dava pochi attestavano, che era leggiermente infermo.
\end{description}

\section{Stanza XXXX.}
\begin{ottave}
\flagverse{40}Sì che con queste finte, e con quest' arte \\
Costor, c'usan la tazza, e non la targa, \\
Servir volendo a Bacco, e non a Marte, \\
Che non fa sangue, ma vuol che si sparga, \\
D' uno stesso voler la maggior parte\\
Trovan la via di sparsene alla larga,\\
Ed il restante non sì astuto, e scaltro\\
Comparisce, perch' ei non può far altro.
\end{ottave}

Questi abitanti di Malmantile con tali scuse, ed invenzioni cercano di sottrarsi
dall' andare alla guerra, e solo vi va chi non ha danari, ne invenzioni da liberarsene.
\begin{description}
\item[TARGA] Brocchiero, Scudo, Rotella. Intende, che son più avvezai a bere
  che a guerreggiare, ed hanno più genio con Bacco \textit{Re} del vino, che non
  hanno con Marte \textit{Re} delle guerre; perché quello fa nascere nel corpo il sangue,
  e questo lo fa disperdere.

\item[STARSENE alla larga] Significa non s impacciare d' una cosa, ed è lo stesso
che giuocar di lontano, che vedermmo nell' Ottava antecedente.

\item[ASTVTO, e scaltro] Sinonimi di sagace, ed accorto, Huomo, che fa il conto
  suo. Ma per maggior intelligenza di queste parole \textit{Astuto}, e \textit{scaltro}, \textit{sagace}, ed
  \textit{accorto} è da sapere che, se bene ce ne serviamo per sinonimi, tuttavia ci è
  qualche differenza; particolarmente fra \textit{sagace}, ed \textit{astuto}; perché l'arti, che dalla
  sagacità s' adoprano, non meritano biafimo, per non esser se non avvedimenti
  sottili, ma schietti, reali, e senza fraude, o inganni: E l' astuzia oltre alle
  suddette lodevoli arti si serve anche delle menzogne, fraudi, e falsità, e d' altre
  cose indegne d'animo nobile. E però \textit{Scaltro, ed accorto} pare che meglio s' adattino
  per sinonimi a \textit{sagace}, che ad \textit{astuto}, al quale più proprio sinonimo sarebbe
  Malizioso, o tristo, o furbo; quando però la voce furbo e presa in senso d'huomo,
  che sa il conto suo; Ma, come ho detto, nel comun parlar civile non
  usiamo così esatta diligenza, e puntualità; ma pigliamo l'uno per l'altro.
\end{description}\
\section{Stanza XXXXI \& XXXXII.}

\begin{ottave}
\flagverse{41}Mentr' in piazza si fa nobil comparsa, \\
Anch' in Palazzo armata la Regina\\
Con una treccia avvolta, e l'altra sparsa,\\
Corre alla Malmantilica rovina; \\
Benché ne i passi poi vada più scarsa,\\
Perché all' uscio da via mai s' avvicina; \\
Da sette volte in su già s'è condotta \\
Fino alla soglia; ma quel sasso scotta.
\end{ottave}

\begin{ottave}
\flagverse{42}Viltà l'arretra, honor di poi l'invita\\
A cimentar la sua bravura in guerra,\\
L'esorta l'una a conservar la vita,\\
L' altro a difender quanto può la Terra.\\
Pur fatto conto di morir vestita\\
Voltossi a bere, e divenuta sgherra\\
(Però che Bacco ogni timor dilegua)\\
Dice: O de'miei: chi mi vuol ben mi segua.
\end{ottave}

Mentre che la men codarda gente si raguna in piazza, anche la Regina Bertinella
al romore, nuova Semiramide con i capelli non ancora finiti d' aggiustare,
corre a difender Malmantile; ma non con tanto ardire, perché questa nostra
Semiramide non s' arrischiò così subito a passare la porta della Casa; ma si fermò
in quella, sospesa, e travagliata da due gran passioni Poltroneria, ed Honore,
che quella l'esorta a starsene; e questo, obbliga ad andare, Al fine lasciatasi
persuadere dall' Honore prese animo, ed esortò i suoi a seguirla.

\begin{description}
\item[TRECCIA] I capelli delle donne si chiamano \textit{trecce}, perché per lo più sogliono
  le donne far due parti de i lor capelli, e ciascuna di quelle suddividere in tre
  altre parti, ed intesserle in terzo, il che si dice \textit{treccia}; E Bertinella stava così
  Intrecciandole, quando sentì il romore, per lo che lasciato il lavoro corse con
  una parte intrecciata, e l'altra no, come dicono, che facesse Semiramide, quando
  senti il pericolo, che sovrastava a Babillonia.

\item[MA la soglia scotta] Quando uno o per debiti, o per delitti sta ritirato in casa,
  o in Chiesa, diciamo: \textit{Non esce, perché la soglia scotra}; cioè se egli uscisse di
  casa, o di Chiesa, sarebbe fatto prigione: ed a Bertinella \textit{scotta quella soglia},
  perché se uscisse di quella, pericolerebbe di toccarne.

\item[VILTA] Qui vale per poltroneria, o codardia.

\item[MORIR vestito] S'intende di coloro, che sono ammazzati, i quali muoiono
  con le vesti in dosso, e però dicendo che fa conto di morir vestita, s' intende che
  ella ha risoluto d' andar a farsi ammazzare.

\item[SGHERRA] Brava, Animosa; fatta così dal vino, che leva di testa ogni timore.
  Bacco da i Latini fu detto \textit{Liber}, perché libera l'huomo da i pensieri noiosi,
  e però dice ogni pensier dilegua, ed il Chiabrera\footnote{Gabriello Chiabrera (Savona, 18 giugno 1552 – Savona, 14 ottobre 1638) poeta e drammaturgo, quasi moderno Pindaro. } disse.
  \begin{verse}
    Beviamo, e diansi al vento
    I torbidi pensieri.
  \end{verse}

  Seneca de Tranquillit. disse; \textit{Nonnunquam ad ebrietatem veniendum, non ut mergat
    nos, sed ut deprimat curas \& elevat enim curas, \& ab imo animum movet, \& ut
    morbis quibusdam, ita trisitiae medetur}, Di questa regola si servsempre il Galasso
  Generale dell'Imperadore Ferdinando 2., il quale mai si portò ad alcuno consiglio
  di guerra, ne si messe ad impresa alcuna importante, se prima non aveva
  molto bevuto. E Bertinella imita questo gran guerriero.
\end{description}
\section{Stanza XXXXIII \& XXXXIV.}

\begin{ottave}
\flagverse{43}Dietro a suoi passi mettesi in cammino\\
Maria Ciliegia illustre damigella; \\
Tutto lieto la segue il Ballerino,\\
Che canta il titutrendo falalella. \\
Va Meo col paggio, Zoppica Masino,\\
Corre il Masselli, e il Capitan Santella.\\
Molti, e molt'altri amici la seguiro,\\
E più Mercanti c'hanno havuto il giro.
\end{ottave}

\begin{ottave}
\flagverse{44}La segue Piaccianteo suo servo, ed Aio,\\
C'in gola tutto quanto il suo si caccia,\\
Le cacchiatelle mangia col cucchiaio,\\
Ed è la distruzion della vernaccia.\\
Già misurò le doppie con lo staio,\\
Finito poi che fu quella bonanaccia,\\
Portò per il contagio la barella,\\
Ed hora in Corte serve a Bertinella.
\end{ottave}

Alle voci, ed ordini di Bertinella obbedirono diversi suoi seguaci Birboni, e
Matti.

\begin{description}
\item[MARIA Ciliegia] Fu una Donna creduta pazza, la quale andava per Firenze
ricevendo elemosina senza domandarla, Costei con una flemma, e gravità non
ordinaria discorrendo sempre da per se, diceva belle, e sensate sentenze; la onde
da molti non era stimata pazza, ma uguale a Diogene, che abitava nella botte;
e per tale azione farebbe stato riputato matto, se non havesse lasciato così
belle sentenze, e dogmi, come appunto fece questa madonna Maria, i detti della
quale, o parte di essi sono stati raccolti da un buon letterato, che forse una
volta gli darà alle stampe: Come Diogene, anch' essa non si curava di casa, ma
dormiva nelle strade sotto qualche portico o loggia, e perciò portava seco sempre
un granatino per spazzare quel luogo, dove si metteva a dormire, ed una
spazzola per spazzolarsi la veste, la quale benché poverissima, era nondimeno
molto pulita, e se bene piena di toppe, assai bella per esservi le medesime toppe
messe forse anche senza bisogno, con vago, ed aggiustato ordine. Nella suddetta
sua sporta haveva ancora qualche biancheria, e molte volte un laveggio, o caldanetto
pieno di fuoco, nel quale, passeggiando per le strade, andava quocendo
le sue vivande; sotto la gonnella haveva più sacchetti, entro dei quali riponeva la
pentola, e piatti per suo uso, e quello che le avanzava a' suoi mangiari. Haveva
sorelle, e nipoti i quali si trattavano comodamente, ed habitavano in una
buona casotta, che era di detta madonna Maria, dove ella alle volte andava per
mutarsi; ma non volle mai fermarvisi, ne dormirvi ancor che pregata, e forzata
anche da' detti suoi parenti a volere star con loro. Buscava molti denari con
li quali comprava quello, che parcamente le bisognava, ed ogni Sabato sera dava
per l' amor di Dio tutto quello, che le avanzava, e per lo più a povere Monache,
dove alle volte portò anche fino a dieci scudi, Domandata da alcuno di
qualche parere, non rispondeva; ma seguitando il suo solito chiacchierare, prima,
che quel tale is partisse da lei restava appagato con qualche sentenza, o motto,
che ella diceva a proposito del quesito. Per esempio: Una mattina, sendo
ella  sotto le logge d' avanti al Tempio della Santissima Annonziata, un giovane
netto le domandò, se ella credeva, che la sua moglie bella, da madonna Maria
molto ben conosciuta, fusse honesta; ma glielo disse con la più sporca maniera,
che dir si potesse. Madonna Maria senza alzar la testa, o dar segno d' attenzione
al quesito del giovane, seguitando il suo discorso, che faceva del poco rispetto,
si portava alle Chiese; dopo molte chiacchiere disse; Vedete voi questo
giovane sboccato, il poco rispetto, ch' ei porta alla Chiesa? La sua moglie è
bella, e la prese, che ella era onesta; ma che può ella havere imparato da lui, se
non il modo di diventare altrimenti? ed hora io ho, che ella sia diventata; perché
ogni geloso è becco. E seguitò il suo cicaleccio, entrando in diversi altri gineprai,
come era solita; e così chiacchierando tutto il giorno dalla mattina alla
sera, buscava molti denari. Costei morì, e si trovò nella sua sporta una borsetta,
nella quale era una ricevuta di cinquanta scudi, dati a certe Monache con
obbligo di far dire una messa il mese all' altare della Santis. Nunziata per l'anima
sua; dal che si cava argumento che ella non fusse pazza.

\item[FALALELLA] Così e chiamato un contadino tristo, il quale non havendo voglia
  di lavorare, s'è dato a chiedere elemosina; e per far venire le donnicciuole
  alle finestre, e cavar loro di mano robe, e danari, va per le strade cantando alcune
  sue ottave amorose, e ad ogni due versi fa intercalare con la voce dicendo
  \textit{Falarera tututrendo}, con che si persuade d' imitar il suono del Chitarrino; ed
  all'ultimo dell' Ottave, al medesimo suono della voce, si mette a ballare, e per questo
  il Poeta lo chiama il Ballerino; e poi va attorno chiedendo la limosina.

\item[MEO] Era uno scemo di cervello provvisionato dal Palazzo; e perché egli
  son si reggeva bene in piedi, pero andava sempre appoggiato a un ragazzo; e
  perciò dice: \textit{Va Meo col paggio}.

\item[MASINO] Era uno stroppiato nelle gambe, e nelle braccia; il quale era anch' egli
  provvisionato dal Palazzo per quella sua figura cotanto contraffatta da
  gli stroppi.

\item[MASSELLI] Era un matto, o creduto tale, provvisionato pure dal Palazzo.
  Costui haveva in mente tutte le feste del anno, e quali Ofizzj, e commemorazioni
  dovean farsi da i Preti giorno per giorno. Sapeva in oltre, quali erano quei
  Rettori, e Curati di Chiese, tanto in Firenze, che nel Contado, i quali nelle
  feste trattavano bene, o male ai loro desinari; e da essi si lasciava in tali giorni
  rivedere; e mangiava, e beveva tanto, che è impossibile a crederlo anche da chi
  l'ha più volte veduto. Era soprannaturale nel digerire, e s' è veduto smaltire
  gran quantità di roba, si può dire impossibile, come sarebbe un gran piatto di
  carta straccia bollita in brodo di bue, e condita a guisa di maccheroni; altre volte
  bisso, e tela d' olanda nella stessa forma, e questo in breve tempo, e senza
  difficultà, o dolori. Il Poeta dice; \textit{Corre il Masselli}, perché veramente costui,
  benché decrepito, era di gamba velocissima. Haveva il Sereniss. Gran Duca dato
  per servitore al Masselli un giovanotto gagliardo, perché lo seguitasse per tutto
  dove egli andava, e osservasse tutte le sue azioni, senza mai contradirgli, o
  impedirlo, ed ogni sera riportasse quanto il Masselli haveva fatto in quel giorno.
  Quando il Masselli riceveva alcun disgusto da costui, non s' alterava seco, mas
  si metteva la via fra gambe, e senza mai fermarsi, o voltarsi ne meno a dietro,
  non la guardava a camminare di buonissimo passo 25., o trenta miglia con grandissimo
  travaglio, e rabbia del servitore, che non poteva, ne doveva distorlo, e
  conveniva, che lo seguitasse; onde andava molto cauto in strapazzarlo (come
  sul principio del suo servire faceva fino a bastonarlo) non tanto per paura del
  gastigo da S. A. S. minacciatogli, quanto per il timore, che il Masselli per vendetta
  non viaggiasse.

\item[CAPITAN Santella] \makebox[3pt]{} Questo fu un soldato della Banda di Pistoia, il quale dette
  la volta al cervello (o così finse) perché gli fu rubata la moglie da chi ne poteva
  più di lui. Costui venne in Firenze, e vi dimorò qualche tempo, facendo diverse
  pazzie; ma perché fu conosciuto, che sotto questa sua finta pazzia si nascondeva
  una gran tristizia, fu mandato forzatamente in Candia al servizio de' SS.
  Veneziani, donde non è più tornato.

\item[MERCANTI, c'hanno havuto il giro] Cioè gente impazzata. Si serve della
  parola Giro per intendere il girare del cervello, che vuol dire Impazzare, non
  per il Giro de' Mercanti, che si dice, quando un Banchiere tiene in mano il denaro
  di tutta la Piazza; il che in Firenze tocca a fare una volta per uno a tutti li
  banchieri, o negozianti più grossi per tanti mesi; il che e fatto per comodità de'
  mercanti; e dicesi: avere il Banco giro.

\item[PIACCIANTEO] Fu un Fiorentino di così vili natali, che non si sa trovare
  la casata, ne il vero nome suo, essendo sempre stato inteso col solo soprannome
  di Piaccianteo. Costui dalli parenti suoi fu lasciato assai comodo, ma come quello,
  che era dedito alla crapula, consumò in breve tempo tutto lo stato suo, ed
  a pena haveva dato principio a provare  le miserie della poverta, e gli stenti, che
  la Fortuna di nuovo lo sollevò facendoli redare da un suo congiunto una somma
  considerabile di doppie; e però il Poeta dice: \textit{Già misurò le doppie con lo staio}. A
  queste ancora il buon Piaccianteo diede presto fine, pensando d' haver ad avverare
  il sentenzioso proverbio, che dice; \textit{A uno scialacquatore non mancaron mai
    denari}. Ma s' ingannò, perché ridotto in estrema poverta, e non sapendo far
  mestiero alcuno, si ridusse a portare quella barella, con la quale si portavano gli
  ammorbati al Lazzeretto nel tempo, che fu la Peste in Firenze, e fin che durò
  tal contagio campò di cotesta sua fatica; finita poi la peste viveva di quel che
  buscava con far servizj alle meretrici; e però il Poeta lo fa servitore di Bertinella,
  e suo Aio, e direttore. \textit{Piaccianteo} voce che ha dell' antico \textit{Piacentiero}.

\item[MANGIAR le cacchiatelle col cucchiaio] Iperbole usatissima per intendere un
  gran mangiatore, Cacchiatella, E' una specie di pane finissimo fatto alla foggia
  ed alla grandezza d' una pera bugiarda; onde con questa iperbole, intendiamo
  che pigli in bocca in una volta tante di queste cacchiatelle, quante piglierebbe
  delle fragole, o piselli, o altra cosa simile, e così viene a essere iperbole doppia,
  perché il cucchiaio comune è capace a fatica d' una sola cacchiatella, e la bocca
  dell' huomo difficilmente riceve una sola cacchiatella per volta: e però intendi,
  che mangiava le cacchiatelle in grandissima quantità, e senza numerarle, come
  non si numerano le fragole, ec, che si pigliano col cucchiaio.

\item[È LA distruzione della Vernaccia] \makebox[3pt]{} È gran bevitore. Vernaccia è una specie di vino
  bianco, ma l'Autore per Vernaccia intende ogni sorta di vino.

\item[MISURÒ le doppie con lo staio] Haveva gran denari. Iperbole usata per intender
  un gran ricco; e ci viene dal Latino \textit{Modio pecuniam metitur}.

\item[BONACCIA] Significa placidezza di mare; ma noi la pigliamo anche per
  sorta di bene stare, e di buona fortuna, come e intesa a presente luogo.

\item[BARELLA] Specie di veicolo simile alla bara, o feretro, col quale si portano
  a sotterrare; ma questa che serviva per pertare gli ammorbati era
  coperta sopra con cerchiate, e tela incerata a foggia di calsa tonda di sopra, come
  i tamburi da viaggio.
\end{description}
\section{Stanza XXXXV — L}

\begin{ottave}
\flagverse{45}Comanda la padrona ch' egli scenda,\\
E stia già fuori con gli orecchi attenti\\
Fra quelle schiere, fin ch'ei non intenda\\
A che fine son là cotante genti;\\
Ma quegli, al qual non piace tal faccenda,\\
Se la trimpella, e passa ai complimenti,\\
E, perché a' fichi il corpo serbar vuole,\\
Prorompe in queste, o simili parole.
\end{ottave}

\begin{ottave}
\flagverse{46}Alta Regina, perché d'Obbedire\\
Più d'ogni altro a' tuoi cenni mi dò vanto,\\
Colà n'andro, ma (come si suol dire)\\
Come la serpe, quando và all'incanto;\\
Non ch'io fugga il pericol di morire,\\
Perch'io fo buon per una volta tanto;\\
Ma perché, s'io mi parto, non ti resta\\
Un huom, che sappia, dov'egli ha la testa.
\end{ottave}

\begin{ottave}
\flagverse{47}Non ti sdegnar, s'io dico il mio pensiero,\\
Che possibil non è ch'io taccia o finga,\\
E, se n'andasse il collo, sempr'il vero\\
Son per dirti, e chi l'ha per mal, si cinga.\\
Ti servirò di cor vero, e sincero\\
Senz'interesse d'un puntal di stringa,\\
E non come in tua Corte sono alcuni\\
Adulator, che fanno Meo Raguni.
\end{ottave}

\begin{ottave}
\flagverse{48}Io dunque che non voglio esser de' loro,\\
Ma tengo l'adular pessimo vizio,\\
Soggiungo, e dico, per ridurla a oro,\\
Che mal distribuito è questo ufizio,\\
E che non può passar con tuo decoro;\\
Poiché mostrando non haver giudizio,\\
Un tuo Aio ne mandi a far la spia\\
Quasi d'huomin tu havessi carestia.
\end{ottave}

\begin{ottave}
\flagverse{49}Manda manda a spiar qualche Arfasatto,\\
O un di quei, che piscian nel Cortile,\\
Questo farà il mestier, come va fatto\\
Senza sospetto dar nel Campo ostile:\\
Ostile dico, mentre costa in fatto,\\
Che cinto ha d'armi tutto Malmantile,\\
Tal gente si puo dire a noi contraria,\\
Perché non vien quassù per pigliar' aria.
\end{ottave}

\begin{ottave}
\flagverse{50}E perch' ei non vorrebbe uscir del covo\\
Soggiunge dopo queste altre ragioni;\\
Ma quella, che conosce il pel nell'uovo,\\
S'accorge ben, che son tutte invenzioni;\\
Però senza più dirglielo di nuovo\\
Lo manda fuori a furia di spintoni,\\
E, mentr'ei pur volea imbrogliar la Spagna\\
Gli fa l'uscio serrar su le calcagna.
\end{ottave}

Bertinella vuol mandar Piaccianteo nel Campo di Baldone a spiare; ma egli,
che non vorrebbe andare, adduce mille scuse; quali non gli sono ammesse, ed è
cacciato fuori di Malmantile a furia di spinte.

\begin{description}
\item[TRIMPELLARE] Intendiamo quel suonare adagio, e tentoni la chitarra,
  liuto, o altro strumento simile, che fanno coloro, che imparano a suonare: e
  da questo per \textit{trimpellare}, o \textit{trimpellarsela}, intendiamo indugiare, o trattenersi
  senza profitto, \textit{tempellare} che diciamo anche \textit{metterla sul liuto}, o metterla in musica,
  e suona quasi lo stesso che.

\item[SE la passa in complimenti] Che significa Perder il tempo in vane cirimonie; e
  senza toccare la sustanza del negozio.

\item[VUOL serbare il corpo a i fichi] Vuol veder di viver, quanto ei può, e non
mettersi a rischio d' essere ammazzato.

\item[OBBEDIRE a tuoi cenni mi dò vanto] Professo d'esser' il più obbidiente servitore
  che tu habbia, e di sapere intenderti anche a i cenni.

\item[COME la serpe quando va all'incanto] Cioè mal volentieri, e forzatamente.
  \textit{Volens nolenti animo}, Omero. Il Lalli En. Tr. C. 2. stan. 32. dice
  \begin{verse}
    Come la biscia all' odioso incanto.
  \end{verse}

\item[FO buon per una volta tanto] Posso morire una sol volta, Quando si giuoca il
  danaro, che s' ha in tavola, allora che uno ha perduta quella porzione, che haveva,
  cava di tasca nuovo danaro, o vero dice: \textit{fo buono}, cioè prometto per uno
  scudo, o per due, secondo che gli pare; e s'intende, che non vuol passare quella
  somma, per la quale ha fatto buono, cioè promesso, Per esempio io fo buono
  per uno scudo, l'avversario invita di due, io tengo la posta, ma non posso
  vincere, ne perder più che uno scudo, perché non fo buono di più.

\item[SE n'andasse il colle] Se bene io sapessi, che ci fusse pena la vita. \textit{Neque si
  securim in manibus tenens aliquis cervici esset incursurus meae, conticerem}.

\item[CHI l'ha per mal, si cinga] Non m'importa, che altri l'habbia per male, e
  si cinga pur la spada, ch'io son pronto a rispondergli. Nel primo testo di mano
  dell'Autore dice \textit{si scinga}, e vuol dire si levi pur da lato la spada, perché a ogni
  modo io non voglio far quistion seco. L'Autore, che sapeva, che in tutti due
  i modi si dice, stimo forse meglio detto \textit{si cinga}, perché nel secondo, che pure è
  di sua mano, dice \textit{si cinga}.

\item[SENZ'interesse d'un puntal di stringa] Non voglio da te cosa alcuna, ancor
  che minima. Suona lo stesso che \textit{un puntal d'aghetto}, che vedemmo sopra C. 2.
  stan, 10. e che il Lat. \textit{Ne ligulam quidem}.

\item[FANNO Meo Raguni] Cioè ragunano danari. La forza sta nella voce \textit{raguni}
  che se ben pare, che sia il cognome di Meo, è il verbo ragunare, che significa
  mettere insieme, e \textit{Meo} e preso in vece di \textit{meus, mea, meum}, e vuol dire Meo
  raguni \textit{marsupio}, cioè raguni alla mia tasca.

\item[E TENGO l'adular pessimo vizio] Non è dubbio, che l'adulazione è vizio esecrando,
  e perciò Dante mette gli adulatori nell'Inferno gastigati con quella severa
  pena, che si legge al C. 18, dell'Inf. Cicerone nel suo lib. de Officiis parla
  de gli adulatori così: \textit{His denique temporibus cavendum est, ne assentatoribus patefaciamius
  aures, neve adulari nos sinamus, in quo falli facile est; tales enim nos putamus,
  ut iure laudemur, ex quo innumerabilia nascuntur peccata, cum homines inflati opinionibus
  turpiter irridentur, \& in maximis versantur erroribus}. Diogene Cinico domandato
  qual bestia mordesse più ferocemente rispose; Nelle salvatiche il detrattore,
  nelle domestiche l'adulatore, perché con le sue false lodi ti conduce alle
  rovine; Ed aggiungeva; che le parole composte non per aprire il vero, ma per
  compiacere, sono un capresto melato. Si potrebbono addurre infiniti detti di
  gravissimi Autori, ma si lascia di farlo, perché non torna affatto al proposito, e
  si rimette il lettore a Plutarco nel suo libro \textit{de dignoscendo amico ab adulatore}.

\item[PER ridurla a oro] Per ridurla alla perfezione del discorso, Per venire alla
  conchiusione. Vedi sotto C. 8. stan. 1.

\item[COME se tu havessi carestia a huomini] Come se ti mancassero huomini di spirito.
  Ancora appresso di noi quando si dice: \textit{Il tale è un huomo} s'intende huomo
  buono a qualcosa, seguitando il detto di Diogene \textit{Hominem quaero}. Nella scrittura
  \textit{Confortamini, \& viri estote}. Omero, \textit{Viri estote}.

\item[ARFASATTO] Huomo vile, mal fatto, scimunito, e da poco; che i Latini
  dicono \textit{Vappa}, \textit{Cerdo}, e simili, come si vede in Plauto da noi in questo proposito
  citato C. 6. stan. 98. E questo nome d'Arfasatto viene da \textit{Arfaxaed}
  della scrittura sagra, che nel barbaro secolo non essendo dal volgo inteso, fu
  reso per uno Babbaleo, o Babbano.

\item[DI quei che pisciano nel Cortile] Pisciar nel Cortile vuol dire Far la spia, e questo,
  perché coloro, che fanno la spia, essendo veduti entrare, e uscire del Palazzo
  della Giustizia, hanno qualche rossore, e però essendo veduti da alcuno
  lor conoscente, si fermano nel cortile di detto palazzo a pisciare per scusa. Si
  può anche dire, che il verbo \textit{pisciare} sia preso in significato di buttar fuori, ed intendere
  che \textit{piscino}, cioè buttino fuora quello che sanno nel Cortile della Giustizia,
  ove è la Cancelleria del Bargello, nella quale le spie portano le denunzie.
  Si può anche far reflessione, che detto Cortile sta sempre pieno di Sbirri, i quali
  son' anche per lo più spie, e vi sono due pisciatoi spessissimo adoprati da loro,
  ed intendere, che venga da questo il detto Pisciar nel Cortile. Ma sia come esser
  si voglia, l'effetto è, che \textit{pisciar nel Cortile} s'intende comunemente, Far la spia.

\item[CAMPO ostile] Campo nimico, Dice che è campo ostile, perché osta; e fa
  nascere il bisticcio dalla parola \textit{ostile}, e dalla parola \textit{costa}, la quale nel parlare
  pare che dica \textit{che osta}, che vuol dire s'oppone, e fa ostacolo, facendola di due
  dizioni, cioè \textit{che}, ed \textit{osta}, quando è d'una sola, cioè \textit{costa} dal verbo \textit{costare}, che
  vuol dire Esser manifesto. Modo usato da Franc. Barbarino ne' Mottetti\footnote{forse Barbarino (Barberini), Manfredo Lupo
Sec. XVI. Compositore italiano nato probabilmente a Correggio, Reggio Emilia, prima metà del XVI. Attivo fra la Svizzera e la Baviera.}.

\item[NON vengon quassù per pigliar' aria] Vengon per altro fine, che per andare a
  spasso, o pigliare aria. Detto usatissimo per intendere uno, che vada sotto altri
  pretesti in qualche luogo, e sia poi per negozio importante, e per cavar utile
  da quella gita; che i latini dissero: \textit{Non sine ratione lupus ad urbem}. E noi pure
  diciamo: \textit{Questa cosa non è fatta sine quare}. Vedi sotto C. 4. stan.~11.

\item[CONOSCE il pel nell'uovo] E' sagace, e astuto, e fa considerare ogni minuzia:
  forse è quello, che i Latini dissero: \textit{Ventura per dioptram prospicit}.

\item[A furia di spintoni] Con quantità grande, e spessa di spinte, che tale è la forza
  della parola \textit{furia} in questi termini forse dal Greco \textit{Phora}, che vuol dir' abbondanza,
  o moltitudine, Vedi sotto C. 9, stan. 49.

\item[IMBROGLIAR la Spagna] Quand'uno s'affatica con chiacchiere fuor di proposito
  per divertire uno dal principiato discorso, per non gli dire quel che egli
  vorrebbe sapere, o non fare quel che egli è imposto diciamo; \textit{Egli imbroglia la
  Spagna}.

\item[SERRAR l'uscio in su se calcagna] Vuol dir Serrar'uno fuori della porta. \textit{È il
  contrario di dare dell'imposta sul mostaccio}, che vedremo sotto C. 10. stan. 27., che
  vuol dir proibire l'ingresso a uno che venga per entrare; e quello vuol dire Obbligar uno a uscire.
\end{description}

\section{Stanza LI.}

\begin{ottave}
\flagverse{51}Sperante resta alla Regina intorno\\
Spianator di pan tondo riformato; \\
Gridan le spalle sue remo, e Livorno, \\
Ed ha un C\ellipsis{18pt} che pare un vicinato;\\
La pala nella destra tien del forno,\\
Nella sinistra un bel teglion marmato\\
In cambio di rotella, che gli guarda\\
Da i colpi il magazzin della mostarda.
\end{ottave}

\begin{ottave}
\flagverse{52}De i Rovinati anch' ei passò la barca,\\
Perché la gola, il giuoco, e il ben vestire\\
Gli haveano il pane, la farina, e l'arca\\
In fumo fatto andar come elisire,\\
Tal che, cantando poi, come il Petrarca,\\
Amore io fallo, e veggio il mio fallire,\\
Al giuoco del barone, e alla bassetta\\
Giocava, apparecchiando alla Crocetta.
\end{ottave}

\begin{ottave}
\flagverse{53}Fu dalle dame amato in generale,\\
(Io dico dalle prime della pezza)\\
Poi Bertinella stavane sì male, \\
Ch' ella fece per lui del ben bellezza, \\
Perché spesa la rola, e concia male, \\
Fatta più bolsa d'una pera mezza, \\
Potea di notte, quanto a mezzo giorno, \\
Andar sicura per la fava al forno.
\end{ottave}

\begin{ottave}
\flagverse{54}Ma poi venuta quasi per suo mezzo\\
A porsi sopr'al capo la Corona,\\
E lasciati di già gli stenti, e il lezzo\\
Profumata si sta nella pasciona,\\
N'impazza affatto, e non lo vede a mezzo,\\
E pospostane lei, c'è la padrona,\\
e Martinazza ch'è la Salamistra,\\
Sperante sempre va in capo di listra.
\end{ottave}

\begin{ottave}
\flagverse{55}Hor perch'egli è di nidio, e navicello,\\
E forte, e sodo come un torrione,\\
Gli dà l' ufizio, e titol di Bargello\\
Con la solita sua provvisione,\\
Perché s'in questo caso alcun ribello\\
Si scuopre, facil sia, farlo prigione,\\
Acciò sul letto poi di Balocchino\\
Se gli faccia serrare il nottolino
\end{ottave}

Partito Piacciantco resta appresso Bertinella Sperante; questo era Fornaio assai
comodo; ma tra il suo mandar male, e tra l'essergli stata fatta serrar la bottega,
si ridusse anch'egli malissimo, e nondimeno non usciva mai di casa le meretrici,
dalle quali veramente cavava il vitto, perché essendo bell'huomo era da
esse amato, e se ne servivano per bravo, e per ogni occorrenza loro: E per questo
il Poeta lo fa consigliero, e Bargello di Bertinella.
\begin{description}
\item[SPERANTE] Così veramente haveva nome costui, e faceva il mestiero del
  Fornaio, e però dice \textit{Spianator di pan tondo}: E lo dice riformato, perché fu proibito
  a quei tempi il fare il pan tondo (che così si chiama il più nobil pane, che si
  faccia in Firenze per il pubblico)\footnote{Pan tondo Ducale, prodotto e commercializzato esclusivamente dai ``Forni dell'Abbondanza'', e dagli Appaltatori del Pan Ducale.  Era di fior di farina, e destinato ad un mercato ristretto.  Era proibito produrre pane ordinario in forma che potesse confondersi con il Pan Ducale.} in riguardo dell'appalto, che fu preso di questa
  sorta pane; e però gli convenne serrare la bottega. Ci è però anche lo scherzo
  dell'equivoco, perché \textit{spianatore di pane} vuol dire Colui che fa il pane, ma
  significa ancora uno, che mangi molto pane. Vedi sotto C. 6. stan. 47. Sì che
  si può intendere gran mangiatore di pan tondo, ma riformato; cioè che non ne
  può più mangiar tanto, per non havere il modo da comprarlo. \textit{Riformato} è termine
  militare, e s'intende quel soldato, che è privato della carica, la quale havea;
  che si chiama poi \textit{Ufiziale riformato}.

\item[GRIDAN le spalle sue remo, e Livorno] Ha spalle così grandi, che son desiderate
  a Livorno per mettere a un remo di galera. Questo \textit{gridare, ec}, è un modo di
  dire, che ha lo stesso significato, che \textit{Chiamar di là da' monti}. Visto sopra C. 1. stan. 59.

\item[Un C\ellipsis{18pt} che pare un vicinato]. Ha un c\ellipsis{18pt}\footnote{``Culo''} grande quanto una contrada.
  Iperbole usatissima per denotare un \textit{sedere} estremamente grande, e per vicinato
  intendiamo una contrada.

\item[TEGLIA marmata] Coperchio fatto di marmo minutamente pesto, e terra,
  col quale, sendo infuocato, si cuoprono le teglie, o tegami per rotolare le vivande:
  ed è forse il Latino \textit{clibanus}; che per altro vuol dire armatura fatta di cuoio
  cotto, se crediamo a Pietro Ulloa Vita di Carlo V\footnote{Può trattarsi di una svista, e riferirsi ad Alfonso Ulloa, e del suo ``Vita dell'Invittissimo Imperatore Carlo V'', ``nuovamente mandata in luce'' nel 1560. Pagina 36.}.

\item[IL magazzino della mostarda] Cioè il ventre. \textit{Mostarda} è uno intingolo fatto
di mosto cotto, e senapa, ec. ma qui è presa (come da molti) per quella roba,
che sta nel ventre per qualche similitudine che ha quell'escremento col colore
della mostarda, e \textit{magazzino} diciamo una stanza destinata a riporvi, e
conservarvi, ec. Spagna. almazèn.

\item[PASSO' la barca de' rovinati] È nel numero de' poveri.

\item[ARCA] Voce latina, che vuol dir Cassa in generale, ma noi intendiamo specialmente
  quella gran madia, entro alla quale i  Fornai tengono il pane cotto, o la farina.

\item[FATTO andar' in fumo d'elisire] Fatto andar male senz'alcun frutto appunto
  come fa l'elixire, che lasciato in un vaso aperto svapora, e si disperde.

\item[AL Barone, e alla Bassetta] Sono due giuochi noti, i primo di dadi, e l'altro
  di carte; ma qui scherzando vuol dire, che era divenuto \textit{Barone}, cioè mal vestito,
  guidone, e ridotto al basso, che vuol dire Impoverito; traslato dalla botte,
  che si dice \textit{esser' al basso} quando il vino che v'è dentro è alla fine, e che la botte è
  quasi vota.

\item[APPARECCHIA alla crocetta] Vuol dir non haver da mangiare. \textit{Far degli
  sbavigli} significa non haver da mangiare. Vedi sotto C. 4. stan. ultima. Ed essendo
  costume di molti nello sbavigliare\footnote{forma arcaica per ``sbadigliare''.} farsi la croce col dito pollice incontro
  alle fauci, pero \textit{far le crocette} intendiamo stare a bocca aperta, e vota, che in
  sustanza vuol dire non haver da mangiare, Qui il Poeta rende il detto più oscuro,
  più coperto dicendo \textit{apparecchia alla crocetta}, che è un Convento di Monache,
  nel qual luogo par che voglia dire, che costui desini, e ceni: che questo significa
  il verbo apparecchiare, quando è messo assolutamente, e senza aggiunta.

\item[PRIME dela pezza] E' lo stesso che di prima Classe, o passar per la maggiore
  detto sopra C. 1. stan. 6.

\item[STAVANE male] Tribolava per l'amore, che gli portava, Era grandemente
  innamorata di lui, Latino \textit{deperibat}.

\item[FECE del ben bellezza] Cioè spese, e consumò, quanto ella havea, Havendo
  consumato tutto il suo bene, le rimase solo la bellezza, o vero fece bellezza, ed
  allegria d'ogni suo havere. E' quel \textit{Proterviam facere}, che vedemmo sopra C. 1, stan. 4.

\item[BOLSA] Mal sana per troppa umidità, e ripienezza. E perché questi tali
  \textit{bolsi} soglion esser per lo più ripieni di carne liquida, e di colore fra il verde, e il
  giallo, gli paragoniamo a una pera troppo matura, o fracida, che questo vuol
  dire pera mezza. Virg. \textit{mitia poma}; cioè \textit{maturi}.

\item[POTEVA andar sicura, ec] Questo si dice d'una donna vecchia, e brutta, intendendo,
  che ella è sicura di non esser rapita.

\item[LEZZO] Puzzo, Fetore, Propriamente \textit{lezzo} e un' odore che dispiace, il
  quale non nasce da corpo corrotto, come è quel puzzo, che nasce da una carne
  troppo frolla, o altra cosa marcia, o fracida, che si dice stantia; ma è odore
  naturale, o procede da sudore, o da altra evaporazione, che getta un corpo,
  benché non sia corrotto, onde quello che si sente dal becco, e dalla capra vivi, si
  dice lezzo, e quella che si sente da i medesimi quando son morti, e corrotti si dice
  puzzo o fetore, o sito di stantio. Vedi sopra in questo C, stan. 24. Questo
  \textit{lezzo}, così d. da \textit{olezzo}, è proprio quello, che i L. dicono \textit{Virus}. Noi diciamo \textit{puzzo}, \textit{lezzo},
  \textit{veleno}, \textit{morbo}, \textit{fetore}, \textit{sito}, e simili pigliando l'uno per l'altro, anzi tanto l'uno,
  che l'altro è vocabolo di mezzo, perché tutti si possono intender per buono odore,
  come si cava da Caio Iurisconsulto: \textit{Qui igitur} ( dice egli ) \textit{venenum dicit debet
  adijcere utrum bonum, an malum}. E Statio lib. 2. Sylvarum: \textit{Atque omne benigni
    Virus, odoriferis Arabum; quod crescit in aruis}, Noi ancora diciamo: \textit{sento sito}, e
  \textit{puzzo di muschio}; \textit{sa di muschio ch'egli avvelena}. \textit{Gli ammorba d'ambra}, \textit{sa di
    zibetto ch' egli attoffica}, ec.

\item[PASCIONA] Intende Comodità, e abbondanza d' ogni cosa necessaria al vitto,
  se ben \textit{pasciona} vuol propriamente dire Il pascolo delle bestie.

\item[N'IMPAZZA affatto] È di tal maniera innamorata di lui, che ha perduto
  il cervello. L, \textit{efflictim}, \textit{perdite amat}.

\item[NON lo vede a mezzo] Non gode la vista di lui alla metà di quello, che vorrebbe;
  termine, col quale s'esprime l'affetto grandissimo, che uno porta a
  un'altro, \textit{Non veder più avanti; ne più qua, ne più là}; usò il Bocc.

\item[SALAMISTRA] Maestra di sala. Ma noi intendiamo una donna saccente,
  dottoressa, affannona, e simili, ma per derisione, diciamo Madonna Salamistra.
  Qui intende direttrice del governo; e la chiama Salamistra pur per derisione.

\item[VA in capo di listra] Cioè toltone Bertinella, e Martinazza egli è il il padrone, o
  il primo huomo che sia in Malmantile.

\item[È DI nidio] E' tristo, E' astuto fino dalla culla. \textit{Ab incunabulis vaferrimus}.
Noi pigliamo questo detto da gli uccelli cavati dal nidio, ed allevati, che per
l'uccellatura son sempre migliori, che i presicci.

\item[NAVICELLO] Vuol dir huomo lesto, e che sa tutte le furberie, che diciamo:
  \textit{sa navigare a tutti i venti}. Ha lo stesso significato che esser di nidio.

\item[IL letto di balocchino] S'intende le forche. Da un tale detto Balocchino, che
  fu impiccato in Firenze al Canto alle rondini per ladro di bestie, delle quali fu
  Sensale, e si chiamò anche il Parola. Vedi sotto C. 6. stan. 67.

\item[SERRARE il nottolino] Vuol dire strozzare: intendendosi per Nottolino\footnote{Nottola, più spesso nottolino: elemento di serratura.} quella
  parte della canna della gola, che vulgarmente chiamiamo \textit{gorgozzule}, e questo per
  la similitudine, che ha nell'andare in giù, e in su, quando s'inghiottisce, all'andare
  in giù, e in su delle nottole da serrar porte, ec.
\end{description}

\section{Stanza LVI.}
\begin{ottave}
\flagverse{56}Fa in tanto nel Castel toccar la cassa, \\
E inalberar l'insegna del Carroccio, \\
E comandante elegge della massa \\
Il nobil Cavalier Maso di Coccio, \\
Ch' in fretta alla rassegna se ne passa\\
Con le schiere pero fatte a babboccio,\\
Che ad una ad una accomoda, e dispone\\
Sotto sua guida, e sotto suo campione.
\end{ottave}

Bertinella fa toccar tamburo, e inalberar l'insegna generale, e dichiara generale
della sua gente Maso di Coccio, il quale subito si mette a far la rassegna,
ed accomoda tutti i soldati sotto i suoi Capitani, e Comandanti.

\begin{description}
\item[CARROCCIO] Questo era anticamente un gran Carro di figura quadrata, sopra
  il quale s'inalberava appiccata a una grande antenna l'insegna Generale
  della Signoria di Firenze, e si metteva fuori in occasione di trionfi, o quando i
  Fiorentini uscivano in campagna alla guerra con esercito formato, ed è forse lo
  stesso Carro, e della stessa figura, e grandezza quello, sopra il quale si porta oggi
  il Palio di S, Gio; Bauita.

\item[MASO di Coccio] Tommaso di Coccio fu un Pescivendolo huomo fiero, e di
  gran seguito di suoi uguali, a i quali egli in tutte l'occasioni di feste, cacce, ed
  altre cose simili comandava come a' suoi servitori, ed era benissimo ubbidito da
  chi per genio, ed affetto, e da chi per timore, e però il Poeta lo fa Generale de'
  soldati di Bertinella, che son tutti di condizione simile a lui, come vedremo.
  Lo dice \textit{nobil Cavaliereo}, perché in Firenze egli era conosciuto, e nominato più che
  qualsivoglia gran Cavaliero.

\item[A BABBOCCIO] In confuso, a caso, e senza considerazione.
\end{description}
\section{Stanza LVII.}

\begin{ottave}
\flagverse{57}Si primo è il Furba nobile stradiere, \\
Che non giuoca alla buona, e meno a' goffi, \\
A noccioli bensì si fa valere, \\
Perch' ei da bene i buffi, e meglio i soffi. \\
Il secondo è il Vecchina il gran Barbiere,\\
Che vuol ch'ogni hor si trinchi, e si sbasoffi,\\
E dove a mensa metter può la mano,\\
Si fa la festa di San Gimignano.
\end{ottave}
Al Poeta mette in questa rassegna una mano di plebei noti per qualche loro
azione o buona, o cattiva, e gli nomina con i loro soprannomi. Il primo è il
Furba stradiere, cioè uno di coloro, che alle porte della Città cercano i passeggieri
se hanno roba da gabella, i quali pizzicano di spia; ma questo Furbo era
anche in effetto spia. Il secondo e il Vecchina Barbiere.

\begin{description}

\item[ALLA buona, ed a goffi] Sono due giuochi di carte assai noti: ma con dir così
intende, che costui non era ne buono, cioè semplice, ne goffo, cioè corrivo.

\item[A' NOCCIOLI ben sì] Già che il Poeta porge la congiuntura di narrare, qual
sia appresso a i nostri Ragazzi il giuoco de' noccioli, ed in quante maniere si
faccia, il Lettore si contenterà, che io spieghi con un poco di digressione i modi
co' quali si trastullano i nostri Ragazzi a questo giuoco de' noccioli, e non
si sdegnerà di volgere gli occhi a leggere il discorso di quei trattenimenti, a'quali,
non sdegnò di volger l'animo, ed impiegar l'opera un Cesare Augusto, secondo
che riferisce Svetonio Tranq. riportato, e considerato da Alex. ab Alex, dier.
Gen. lib. 3. cap. 24. e ricordandosi che tutta quest'Opera è fatta per i Fanciulli
più che per quelle persone, che già \textit{reliquerunt nuces}, havra la bontà di concedere,
se non per necessaria, almeno per non affatto fuori di proposito tal digressione
Dico dunque che il giuoco, che fanno i nostri Ragazzi co' noccioli
di pesca (costumato anche da i ragazzi Greci, e Latini, che lo dicevano ludus
ocellatarum, secondo il Buleng, de Lud. veterum, \& Alex. ab Alex. dier. gen. lib. 3.
cap. 21, le di cui parole poco appresso riporteremo) è usato in molte maniere;
ma specialmente giuocano, \textit{a Cavalca}, \textit{alle Caselle}, \textit{alla Serpe}, \textit{a Ripiglino}, \textit{a Sbrescia},
\textit{a Cavare}, \textit{a Sbricchi quanti}, \textit{a Truccino}, ed \textit{alle Buche}. Di tali giuochi, e
di ciascuno di essi narreremo il modo, che tengono a esercitargli, e diremo quali
sieno simili, o gli stessi, che erano usati da gli antichi.

\item[A cavalca] S' accordano due o più, e tirano sopra un piano i noccioli a un
per uno, e tanti ne seguitano a tirare, quanto stieno a far salire sopr' agli altri
tirati un nocciolo, che sopra vi resti, e si regga senza toccare altro che noccioli;
e colui che ha tirato il nocciolo rimasto sopra, vince, e leva via tutti i noccioli
tirati. Lo dicono a Cavalca da quel cavalcare, che fa il nocciolo sopr' a gli altri.

\item[ALLE Caselle] o \textit{Capannelle}. Mettono sopra ad un piano tre noccioli in triangolo,
  e sopra di essi un'altro nocciolo, e questa massa dicono \textit{casella}, o \textit{capannella}
e fatto di éffe il numero tra loro convenuto, ed allontanatisi nella distanza
concordata, tirano in dette Caselle un' altro nocciolo, e colui che tira, e coglie,
vince tutte quelle caselle, che fa cascare col colpo. Questo fu usato ancora da
gli antichi, e dicevano \textit{Ludere Castello nucum} secondo il Buleng. C. 8. Queste caselle
vengono descritte da Ovidio in Nuce in quei versi:\textit{Qutuor in nucibus non
amplius, alea tota est, Cum sibi suppositis additur una tribus},

\item[ALLA serpe] Fanno una di dette caselle, la quale figura il capo della serpe, e
da quella fanno partire un filare di noccioli, che figura il resto del corpo della
serpe, e poi vi tirano dentro con un' altro nocciolo, e chi fa col tiro scappare
uno, o più noccioli del tutto fuori del detto filare, vince tutti li noccioli, che
sono dalla rottura in giù verso la coda di detta serpe, e durano così, fino a che
sia rovinata da un di loro queila casella, che figura il capo della serpe. Questo
pure era usato da i Greci, e Latini, e forse facevano co' noccioli altre figure,
come si cava dal Buleng. Cap. 8, dove si vede, che in vece della serpe, facevano
co i noccioli un triangolo equilatere, o [come dice egli] il delta $\Delta$ de' Greci.

\item[A RIPIGLINO] Pigliano quella quantità di noccioli, che convengono, e tirandogli
all'aria gli ripigliano con la parte della mano opposta alla palma, e se
in tal' atto sopr' alla mano non resta alcun nocciolo,colui perde la gita, e tira
colui, che segue; e così si va seguitando fino che resti sopra detto luogo della
mano qualche nocciolo, e questo al quale e rimasto il nocciolo,dee di quivi tirarlo
all' aria, e ripigliarlo con la palma, e non lo ripigliando perde la gita: se ne
restasse più d'uno sopra alla mano, può colui farne scalare quanti gli piace pur
che ne resti uno; che se non restasse, perde la gita. Ripigliato il nocciolo la
seconda volta, deve costui tirarlo all'aria, ed in quel mentre pigliare uno, o più
de i noccioli cascati, e con essi in mano ripigliar per aria quello che tirò, e non
seguendo, posa i noccioli presi, e perde la gita; e se ne ha pigliati qualcheduno
senza fare errori, restano suoi, e si seguita il giuoco fino a che sieno levati tutti,
Giulio Polluce lib. 9.c. 7. mostra che facessero questo giuoco ancora li Greci, e lo
dissero \textit{Pentalitha}, perché usassero di farlo con un numero determinato di cinque
sassolini, o aliossi.

\item[SBRESCIA] È lo stesso, che ripiglino, se non che nella terza ripigliata devonsi
ripigliare quei noccioli, che cascarono in terra la seconda volta non a
uno, o due per volta, ma tutti a un tratto (il che si dice fare sbrescia) e
lasciandovene pur' uno, o cascandogliene, perde la gita, e così fiva seguitando, fin che
uno pulitamente gli raccolga tutti.

\item[CAVARE] Infilano un nocciolo con una setola di crine di cavallo, alla
qual setola ridotta in forma di campanella, o anelletto legano uno spago, di poi
segnato un circolo in terra, vi mettono i noccioli, che son d'accordo, e colui,
al quale è toccato in sorte, deve, girando in ruota con quello spago il nocciolo
infilato, a tal girare, buttar con esso nocciolo fuori del circolo uno, o più noccioli
di quelli, che son dentro al circolo, e vince quelli, che cava, e se col nocciolo
che gira, tocca terra, perde la gita; ma guadagna i noccioli cavati, e dà il nocciolo
da girare a un' altro. E così si va seguitando fino a che sien cavati tutti i
noccioli, Similmente nel giuoco detto da' Greci \textit{Eis amillan} descrivevano un
cerchio, dentro 'l quale però si doveva buttare l'aliosso in maniera, che vi rimanesse,
e non uscisse di detto cerchio. Appresso di noi anche negli Alioffi si fa a cavare.
Canti alcialeschi; \textit{Perch' al cavare un' eliosso brutto, ec.}

\item[SBRICCHI quanti] Occultano dentro al pugno, o dentro ad ambe le mani
  quella quantità di noccioli, che vogliono, poi domandando ad altri, che indovinino
  il numero de' noccioli occultati, ed indovinandolo vince tutto, se no; deve
  dare quel numero di noccioli, che ha detto di più, o di meno; E questo si fa
  una volta per uno, dovendo il primo, che domandò far' anch' egli domandare,
  e così si va continuando i giuoco. Questo \textit{sbricchi quanti} è lo stesso, che pari, o
  caffo, nel quale si domanda, se il numero è pari, o caffo, e chi s'appone vince
  tutti li noccioli occultati; se no, perde altrettanta somma. I Latini dissero: \textit{ludere par impar}.
  I Greci \textit{artiazein}, Di questo giuoco parla Giulio Polluce sopra
  citato, ed il Meursio \textit{de ludis veterum}, i quali mostrano, che si faceva, come
  pure oggi si fa con i danari, e con altra materia, come mandorle, e simili, atta
  a potersi accomodare dentro alle mani, Ovidio in Nuce. \textit{Est etiam par sit
    numerus qui dicat, an impar. Ut divinatas auferat augur opes}.

\item[A TRUCCINO] Uno tira un nocciolo in terra, e l' altro tira un nocciolo a
  quello, che è in terra, e cogliendolo vince, se no, quello, che tirò in terra il
  primo, raccoglie il suo nocciolo, e lo tira a quello, che tirò l'avversario, e così
  continovano, e chi coglie vince il nocciolo che coglie, o quello che sieno convenuti.
  È simile al giuoco detto da Greci \textit{Streptinda}.

\item[ALLE buche] Fanno diverse buche in terra in giro, formandone come una
  rosa, nelle quali tirano i noccioli, e colui vince, che entra in una di dette buche,
  quella somma, che e prezzata quella buca, nella quale entrò il suo nocciolo: per
  esempio le buche sono sette, la prima che è volta verso donde si tira, che è la più
  facile a entrarvi non fa vincere, non essendo tassata in cosa alcuna, e da i nostri
  ragazzi è detta la buca del Nisio (forse da nihil) E dell'altre una vince tre, una
  quattro, ec. E perciò ho detto, che vince chi v'entra quanto è prezzata la buca,
  e poi va con gli altri ad aiutar condurre il nocciolo nella buca a colui, che al primo
  tiro non v'entrò, e spingendolo di dove e alla volta delle buche col dito indice
  (che dicono limare). Ovidio \textit{Aut pronas digito bisve semelve petit} o col buffare,
  o col soffiare nel nocciolo, (e la differenza da buffare a soffiare vedremo
  poco appresso) nel che adoprano ogni arte per difficultare all'avversario il condurre
  il nocciolo dentro alle dette buche; E così facendo a una volta per uno a
  limare, buffare, o soffiare, colui vince, che ha fortuna di condurre il nocciolo
  dentro a una di dette buche, ancor che il nocciolo sia degli avversarj. Simile
  al fare alle buche è quel d'Ovidio. \textit{Vas quoque saepe cavum spatio distante locatur, In
    quod missa levinux cadat una manu}. Fanno questo giuoco ancora con una palla, e
  giuocano danari, come vedremo sotto C. 8. stan.\ 69. alla voce Aliosso. Ed è simile
  quello che i Greci, secondo Giulio Poll. lib. 9. c. 7. chiamana \textit{Aphetinda}: e secondo
  il Meursio de Lud. Graec. alla voce \textit{Aphetinda}, \& alla voce \textit{Amilla}, ed il
  Buleng. cap. 14. e 40. Se bene tanto nell'\textit{Aphetinda}, quanto in quello, che si chiamava
  \textit{Eis amillan}; tiravano in un circolo, e non nelle buche. Alla buca bensì
  tiravano in quell'altro detto \textit{Tropa}, che corrispondeva a questo nostro. Conchiudo
  dunque, che la maggior parte di detti giuochi erano usati anche da gli antichi;
  E se ben pare, che si servissero delle noci, io non son lontano dal credere,
  che la parola Nuces voglia dire ogni sorta di nocciolo, e mi fondo in Plinio
  lib. 15. cap. 21., dove mette in dubbio, se le noci in quei primi tempi fussero
  ancora arrivate in Italia; ed oltre a questo trovo ne i Latini \textit{Iuglans}, per noce,
  ed ardirei però affermare, che ancor' essi adoperassero noccioli di pesca, o pure
  (come fanno anche i ragazzi de' nostri tempi) alle volte noci, ed alle volte noccioli
  di pesca, seguitando Alex. ab Alex. lib. 3. c. 21., che dice così: \textit{Memini doctos
  viros super nucibus ocellatis eiusmodi, quae essent, ancipitem diu cogicationem duxisse,
  variaque in opinione versari, \& alios nuces avellanas, alios amygdalas putare,
  neque satis ratam sententiam ferre super Tranquilli verbis, quibus Augustum laxandi
  animi causa cum pueris facie liberali ocellatis nucibus lusisse dicit. Quod vere nos sentimus,
  \& probabilius putamus id est: Eiusmodi nuces ocellatas nucleos, quos in persicis
  pomis sitos inspicimus dicamus esse, quibus persaepe ludere nostrates pueros hodie videmus
  dictasque ocellatas propter ocellos, \& foramina, quibus muniuntur undique, neque de
  ansyedala, aut avellana, sicut error habet, sed de persicorum ossibus, quibus  tunc ludebatur,
  \& nunc frequens puerorum ludus est, intelligi convenire credimus explorata,
  \& non ambiguae sententiae fore}. Dalle quali parole s' intende, che anticamente ancora
  si giuocava a questo giuoco de' Noccioli, Ovidio de Nuce, corrobora questa
  verità, e mostra che havessero molti de' suddetti giuochi, o poco dissimili. E
  Marziale attesta, che erano gli stessi genj ne i fanciulli de' suoi tempi, che in quelli
  d'oggidì, e che il portare in tasca noccioli causava a quelli delle mazzate, come
  segue ne i nostri, dicendo:
  \begin{verse}
    Alea parva nuces, \& non damnosa videtur;
    Saepe tamen pueris abstulit illa nates
    \verseprefix{Et altrove.}Iam tristis nucibus puer relictis
    \verseprefix{Ed Horatio.}Postquam te talos, Aule, nucesque
    Ferre sinu laxo vidi, ec.
  \end{verse}
  Sono dunque, e furono sempre puerili tutti li suddetti giuochi; e perciò noi habbiamo
  un detto di disprezzo; \textit{Va a giuoca a' noccioli}, che significa Tu non hai maggior
  giudizio di quel che habbia un fanciullo: Qual detto era usato da i Latini
  pure, come si cava da Persio Sat. 1.
  \begin{verse}
    Et nucibus facimus queacumque relictis
  \end{verse}
  E dicevano \textit{reliquit nuces} d'uno, che dalla puerizia passava a maneggiar cose serie;
  Dal che potrebbe argumentarsi, che 11 Poeta dicendo, che il Furba giuoca
  bene a i noccioli, intendesse, che egli fusse huomo di poco giudizio, e che
  \textit{nucibus imcumbat}; Ma si conosce, che non intende questo, perché prima disse,
  \textit{Non giuoca alla buona ne a i goffi}, significando che non era ne buono ne goffo, ed
  ora col dire, che egli \textit{giuoca bene a' noccioli, perché da bene i buffi, e meglio i soffi},
  vuol dir fa ben la spia, che \textit{buffare}, e \textit{soffiare} vuol dir Far la spia. Vedi sopra
  C. 1, stan. 37.
\item[BUFFI, e soffi] Buffo è un soffiare non continuato, ma fatto a un tratto, come
  si farebbe a sputare, o a profferire la parola \textit{buffi}, donde \textit{bufera}, o \textit{bufea} un gran
  nodo di vento, che passa presto. \textit{Soffio} è un soffiare con la bocca tanto quanto si
  può durare senza ripigliare il fiato, e ciò dico per mostrare la differenza che è
  fra \textit{buffo}, e \textit{soffio}; che per altro sò che \textit{soffio} è generico, e comprende ogni sorta di
  rompimento d'aria fatto col fiato di che che sia, dicendosi \textit{soffiare}, quel fiato, o
  vento, che manda fuori il mantice, \textit{soffiare} si dicono i Venti, ec. Vedi sopra
  C. 1. stan. 39, la voce \textit{rabbuffo}.

\item[IL Vecchina] Era un barbiere così chiamato, il quale ogni sera andava ricercando
  per l'osterie le conversazioni, che erano a cena, e trovandone di suoi amici,
  con varie chiacchiere poco a poco senz'essere invitato si metteva a sedere,
  e mangiava, e beveva quanto più poteva, ed al far de' conti sen' andava senza
  pagare, e quello gli era comportato, perché faceva il buffone; Procurava, che
  le conversazioni di cene si facessero in bottega sua, dove apparecchiava, e provvedeva
  assai pulitamente, e bene, e con spesa aggiustata faceva star bene, e avanzava
  tanta roba per se da viver più giorni, e però dice \textit{Vuol che ogn' hor si trinchi}
  (che dal Tedesco \textit{trinchen} vuol dir bere) \textit{e si sbasoffi}, cioè si mangi assai, donde:
  \textit{basoffione} un che mangia assai: Queste voci \textit{basoffia}, e \textit{basoffione} sono in uso
  appresso alla plebe più bassa, ed i più civili l'adoprano per scherzo, per intendere uno
  soverchiamente grasso, e che mangi molte minestre, le quali si dicono \textit{basoffie} dal
  Latino \textit{vas offae}, cioè Vaso pieno di minestra.

\item[SI fa la festa di San Gimignano] San Gimignano è una grossa Terra del Dominio
  Fiorentino nel Vescovado Volterrano; e la principale, e più solenne festa,
  che si faccia in questa Terra è di Santa Fine, la qual Santa fu di quel luogo: E
  dicendosi \textit{far la festa di S. Gimignano} s'intende si fa fine; e qui vuole esprimere,
  che questo Barbiere dava fine a ogni cosa, che veniva in su la mensa.

\end{description}
\section{Stanza LVIII}
\begin{ottave}
  \flagverse{58}Dalle fredde acque il Mula i fanti approda \\
A spiaggia militar fra fronde, e frasche, \\
Ha nobil bardatura tinta in broda \\
Di cedri, e di ciriege d' amarasche, \\
Co i pescatori al Mula hora s'accoda\\
Dommeo  Treccon de ghiozzi, e delle lasche;\\
Pericol pallerino anch' ei ne mette\\
Dugento suoi armati di racchette
\end{ottave}

\begin{description}
\item[IL mula dalle fredde acque] Fu uno che nel tempo di state vendeva l'acque diacciate
  così soprannominato. Pare che questo Mula sia un gran sig.\ di lontani paesi
  e vicino al Mar gelato, di dove approdi alla spiaggia del mare; ma \textit{approda}, cioè
  s' accosta al restante dell' armata di Bertinella. Dice \textit{fra frondi}, e \textit{frasche}, perché
  questi tali venditori d'acque diacciate sogliono per allettamento ornare le loro
  di verzure, fiori, e frasche.

\item[S' ACCODA] Seguita, o vien dietro immediatamente. Quasi \textit{ad caudam ire}.
  Noi usiamo questo verbo per le bestie da soma, che seguitando in viaggio l'una
  l'altra, viene alla prima legata la seconda, alla seconda la terza, ec, con
  la cavezza alla groppa dell'antecedente, e così chi seguita va con la testa vicina
  alla coda di essa, e questo si dice accodare, benissimo usato qui dal Poeta,
  per il Mula, sendo che a i muli più, che ad ogni altra bestia segue questo accodare.

\item[DOMMEO] È una parola sola, e dovrebbe dire \textit{Dommeone}, che così era
  chiamato un venditore di pesce, e salumi, il quale era amato da tutti i ghiotti
  di Firenze, perché vendeva sempre il miglior pesce, che venisse in mercato, ed i
  giorni di grasso haveva sempre qualche galanteria, o ghiottornia singolare. E
  però lo chiama \textit{treccone}, che vuol dire Rivendugliolo, cioè rivenditore di cose
  commestibili di poco prezzo (che si dice anche barullo) forse dal Latino \textit{tricae},
  bagattelle, cose di poca stima, e di vil pregio; Marziale, \textit{Sunt apinae, tricaeque,
    \& si quid vilius istis}. Dice di \textit{ghiozzi}, e di \textit{lasche} (due specie di pesce note) non
  per intendere, che vendesse solamente questi, ma per mostrare, che vendeva
  pesce in generale.

\item[PERICOLO] Questo fu un tale Alessandso Violani detto Pericolo antonominato
  per il suo gran valore nell'abbaco, come diremo sotto C.~11. stan.~41. E
  perché egli era anche bravissimo giuocatore di Palla a corda, e tenne gran tempo
  a fitto una di quelle stanze dove si giuoca a tal giuoco, lo fa venire con gente
  armate di \textit{racchette}, o \textit{lacchette}, che sono mestole, con le quali si giuoca alla palla
  a corda, e sono composte d'un cerchio di legno col manico, ed il vano è ripieno
  d'una rete fatta di grossa minugia: per \textit{lacchetta} intendiamo anche la coscia
  di dietro del porco, e del castrato; Non so già se la \textit{lacchetta} da giuocare pigli il
  nome da questa, o questa da quella, so ben che si chiamano così l'une, e l'altre
  per la similitudine, che è fra di loro della figura. Questa da giuocare era da i
  Latini detta \textit{reticulum} da quella rete, della quale è composta, come si cava da
  Ovidio: \textit{Reticuloque pilae leves fundantur aperto}. Vedi sotto C. 6. stan. 34. alla
  parola \textit{Pillotta}.
\end{description}

\section{STANZA LIX, STANZA L}
\begin{ottave}
\flagverse{59}Melicche quoco all'ordine s'appresta, \\
Per giannettina ha in mano uno stidione, \\
Ed un pasticcio per visiera in testa \\
Con pennacchio di penne di cappone \\
Un candido grembiul per sopravvesta \\
Gli adorna il c\ellipsis{18pt} e l'uno, e l'altro arnione, \\
Vina zana è il suo scudo, e nell'armata \\
Conduce tutta Norcia, e la vallata.
\end{ottave}

\begin{ottave}
\flagverse{60}L'unto Sgaruglia con frittelle a iosa\\
Alla squadra de Quochi hora soggiugne\\
Quella de' Battilani assai famosa,\\
Gente che a bere e peggio delle spugne,\\
A cui battien (diceva) la calcosa,\\
Ch'affeddeddieci là dove si giugne\\
Noi non habbiamo a scardassar più lana,\\
Ma s'ha a far sempre la lalunediana.
\end{ottave}

Segue Melicche Zanaiuolo di Mercato vecchio, uno di coloro, de' quali ci serviamo
per mandare a casa le robe commestibili, che si comprano in Mercato
vecchio, e ci servono ancora per Quochi. Costoro son per lo più della Vallata
e Cantoni Svizzeri, e dimorando in Firenze soglion far camerata co i Norcini,
che vendono i tartufi, e per questo dice che egli conduce Norcia, e la Vallata. E
perché egli era hvomo pulitissimo, gli fa per sopravvesta un grembiule candido,
come veramente egli sempre portava.

\begin{description}
\item[GIANNETTA] onde \textit{Giannettina}; specie d'arme in asta, nella guerra usata
da gli alfieri. \textit{Gineta} in Spagn. è una piccola lancia; corsesca.

\item[PENNACCHIO] S'intende una quantità di penne di Struzzolo; ma costui
  l'havea di Cappone come trofeo di Quoco.

\item[ZANA] Specie di paniere senza manico composto di strisce di legno gentile,
  e da tale Zana costoro son detti \textit{Zanaioli}. Di questi tali il Poeta fa Capitano Melicche,
  perché in vero egli era riverito da essi, come quelli che nel loro paese
  l'havevano veduto esercitare Cariche riguardevoli, e sapevano, che era de i più
  reputati della sua patria, dalla quale era in quei tempi bandito.

\item[SGARUGLIA] Fu un Battilano assai celebre, e fra i suoi pari Capopolo, e
  da costui quando in commedia e stato introdotto il Battilano l'hanno nominato
  Sgaruglia. Questi condece la schiera de' Battilani, che dice \textit{famosa}, e scherzando
  con l'equivoco, vuol dire Affamata, da Fame, e non da Fama.

\item[FRITTELLE] Così chiamiamo una vivanda fatta di pasta quasi liquida fritta
  nell'olio da i Latini detta \textit{Artolaganus}; e sì come essi mescolavano con detta pasta
  latte, ed altro, così noi pure vi mettiamo delle mele affettate, uva fecca,
  latte, riso, erbe, ed altro secondo i gusti. I nostri contadini nel tempo, che fanno
  l'olio costumano di far molte di tali frittelle, indotti a ciò da havere olio in
  abbondanza, e ne danno anche a i vicini, e parenti; sono però soliti coloro, che
  vanno a veder lavorare, chiedere le frittelle, ed i lavoranti con poca grazia, e
  meno discrezione spruzzano l'olio addosso a quel tale dicendo: Eccoti le frittelle.
  E da questo forse per \textit{frittelle} intendiamo macchie, che vuol dire Ogni segno, o
  tintura, che sia nella superficie d'un corpo diversa dal proprio colore di quel tal
  corpo, come segue, quando l'olio casca sopra ad un panno. Ed il Poeta dicendo,
  che costui \textit{havea molte frittelle}, intende, che egli era assai unto, come sempre
  sono i Battilani per il continuo maneggiare olio, e lane unte.

\item[A IOSA] In quantità grande. Diciamo nel medesimo signifitato \textit{a cafisso}, \textit{in
  chiocca}, \textit{a biscia}, \textit{a fusone}, voce usata da Giovanni Villani, a similitudine della
  Franzese \textit{A foison}, cioè con effusione, senza risparmio, \textit{a furore}, \textit{a precipizio}, \textit{a
  bizzeffe}, \textit{a Isonne}, e simili. Che se bene son modi bassi, nondimeno sono tulvolta
  usati anche fra la gente civile. E questo a \textit{Iosa} credo sia parola corrotta, e che
  dovesse dire a \textit{chiosa}, che significa quelle cappelle, che hanno le bullette, e
  ogni piccola piastra di piombo, di rame, o d'ottone ridotta tonda, e simili
  alle nostre monete, delle quali chiose i nostri ragazzi si servono per giuocare alla
  trottola in vece di monete, e però \textit{chiosa} s'intende per moneta di niuf valore:
  Il Persiani disse:
  \begin{verse}
    Ma s'in tasca non ho pure una chiosa
    A mantenermi, in tanto quae pars est ?
  \end{verse}
  Si che dicendosi: Della tal mercanzia ve n'era a \textit{Iosa}, o a \textit{chiosa} s'intende,
  che di quella mercanzia ve n'era così grande abbondanza, e per questo era a così
  vil prezzo, che se n'haveva fino per una chiosa. Il Berni nel suo Capitolo in lode
  de' Ghiozzi disse:
  \begin{verse}
    Segue da'questo un' altra disciplina,
    Che havend' ingegno, e del cervello a iosa,
    Bisogna che v' habbiate gran dottrina.
  \end{verse}
  Il Domenithi in lode della Zuppa.
  \begin{verse}
    E quinci vien, ch' ella si suol gradire
    Da chi ha cervello, ed intelletto a iosa.
  \end{verse}
Questa voce \textit{chiosa} per similitudine significa ancora le Croste delle bolle, E vuol
anche dire Esposizione, o comento, forse dal latino greco Glossa. Dante num.~2.
Purg. C.~11.
  \begin{verse}
    E serbolo a chiosar con altro resto,
  \end{verse}
  E nel'Inf C.25.disse \textit{Faranno sì che tu porrai chiosarlo},

  Il Varchi nel Capitolo dell'uova sode dice:
  \begin{verse}
    E s'io fussi Dottor, consiglierei
    Che sopra questo si dovese fare
    Leggi, e statuti, e poi gli chioserei.
  \end{verse}

\item[PEGGIO delle spugne] Succia il vino più che non farebbe uaa spugna; cioè
  beve assaissimo, come veramente fanno i Battilani, i quali chi sieno, dicemmo sopra
  in questo C. stan. 8.

\item[BATTER la Calcosa] Frafe Furbesca, che vuol dir batter la strada, camminare;
  e questo parlar furbesco è praticato assai da questa sorta di gente.

\item[AFFEDDEDDIECI] Giuro proprio de' Battilani profferito come è scritto in
  una sola parola con due ff, e quattro d. Quando i Battilani hanno gran lavori
  e sono molte persone a lavorare, hanno ogni dieci huomini un Sopracciò, che
  chiamano il Capo dieci, che è da loro ubbidito, e stimato, e però giurando a
  fe del Dieci, intendendo di costui, stimano di fare un giuramento solenne.
  Credo nondimeno che dicano a fe de Dieci per non dire a fe di Dio, come pure
  dicono per Dianora, Corpo di Dianora per la medesima ragione.

\item[SCARDASSAR la lana] Cioè pettinare la lana con quei pettini, che chiamano
  Cardi, perché hanno i denti torti, e simili a quelli spuntni, che hanno le
  foglie, il fusto, ed il fiore dell'erba detta cardo, del qual fiore quando è secco
  si servono per pettinare, ed unire il pelo de i panni, e però lo dicono cardare,
  ed è il latino \textit{carminare}. Vedi sotto C. 7. stan. 37.

\item[FAR la lunediana] Appresso a i battilani significa non lavorare; e questo, perché
  nel tempo, che l'arte della lana lavorava, costoro guadagnavano assai, ed
  erano pagati dalli loro maestri il lunedì, dove gli altri manifattori sono pagati il
  sabato, e però questo giorno del lunedì, essendo per loro giorne d'allegria stante
  la riscossione, era da essi solennizzato, e non volevano lavorare, (ma stando in festa)
  a consumare in bere, ed in mangiare quel denaro, che havevano riscosso,
  e questa loro solennità chiamavano \textit{Lunediana}, cd alle volte \textit{Lunigiana} ed era
  da essi tal festa così osservata, che tra loro era la seguente cantilena,
  \begin{verse}
    Chi non fa la lunediana,
    E' un gran figlio di puttana.
  \end{verse}
  Ed oltre a questa ce n' è un' altra che dice:
  \begin{verse}
    Il Venerdì de Beccai,
    Il Sabato de gli Ebrei,
    La Domenica de' Cristiani,
    E il lunedì de i Battilani.
  \end{verse}

  Sì che dicendo \textit{lunediana} s'intende festa, come si vede nel presente luogo
  che Sgaruglia dicendo \textit{s'ha a far sempre la Lunediana, ec}, intende hada esser sempre
  festa. Questo nome di Lunediana resta ancor' hoggi, ma come che i Battilani
  sono pochi, ed i lavori meno, convien loro per forza stare alle volte le Settimane
  intere senza lavorare, e così non è messa troppo in uso detta solennità,
  anzi hanno di grazia, lavorare anche il lunedi.
\end{description}
\section{Stanza LXI.}
\begin{ottave}
  \flagverse{61}Conchino di Melone ecco s' affaccia, \\
Che l'Offerta tenendo de gli allori\\
Col fine, e saldo d'un buon prò vi faccia \\
Ha dato un frego a tutti i debitori, \\
Che tutti allegri, e rubicondi in faccia\\
Cantando una canzone a quattro cori,\\
Di gran coltelli, e di taglieri armati,\\
Si son per amor suo fatti soldati.
\end{ottave}

Segue \textit{Conchino di Melone}, il quale si conduce dictro una mano de' suoi debitori,
che si son fatti soldati per la cortesia, che ha fatto loro di scancellare a tutti
il debito, che havevano seco. Fu costui già quoco d'Osterie, e per esser molto
grasso, e di statura piccolo fu chiamato Conchino, gli venne voglia di diventar
maestro, onde prese sopra di se un'Osteria detta \textit{gli allori}, dove subito hebbe
molti bottegai, ma tutti a credenza, per lo che presto fallì; e non trovando modo
di risquotere un soldo gli venne rabbia, ed abbruciò i libri per non haver di
più quella passione di vedere scritti i suoi denari, e non gli potere spendere. E
questo intende dicendo \textit{col fine, e saldo d' un buon pro vi facia ha dato frego a tutti i debitori}.

\begin{description}
\item[S'AFFACCIA] Si fa innanzi. L'Autore si serve di questo verbo afacciarsi,
  per denotare, che costui havea la faccia larga; scherzo assai praticato con uno,
  che habbia gran ceffo dicendosegli affacciatevi, facciami favore, facciami buon viso,
  e simili.

\item[TAGLIERE] Intendiamo un'arnese da cucina, fatto di legno, tondo a foggia
  di piatto per uso d' affettare sopra di esso carne, e per triturarla con quei \textit{gran
    coltelli}, e farne polpette, o altri battuti. I Tedeschi usano in molti luoghi i piatti
  da tavola fatti di legno, e gli chiamano \textit{Talier} con voce venuta d'Italia, come
  si può credere; già che i nostri antichi i piattelli, o tondini dal tagliarvi su le
  vivande, domandavano \textit{taglieri}, onde il proverbio. \textit{Due ghiotti a un tagliere}, cioè
  \textit{a uno stesso piatto}. Trovasi questa voce nella antica lingua Gallese, o Francesca;
  e dicevano \textit{tailleor}; come leggesi in un' antichissimo libro in quella lingua, dal Lat.
  volgarizzato, appellato del Conquista della terra Santa di Gerusalemme, il quale si è
  ritrovato essere di Guglielmo Arcivescovo di Tiro; e si conserva nella preziosissima
  libreria di Manoscritti del Sereniss. Gran Duca, appresso alla Chiesa, e Collegiata
  di S. Lorenzo. Il passo tutto voltato in Toscano dice così; La dentro (in
  Cesarea) fu trovato un vasello di pietra verde, e chiara assai di troppo gran
  beltà, fatto così, come un tagliere\footnote{Si riferisce al ``Sacro Catino'', ora ritenuto manufatto islamico in vetro di color verde smeraldo, del IX-X secolo.}. Li Genovesi pensarono, che ciò fusse uno
  smeraldo. Perciò lo prenderono a lor parte, del guadagno della Città per troppo
  gran somma d'avere. Portaronnelo in lor Città, e l'appesero nella Mastra
  Chiesa, ove egli è ancora. L'huomo vi mette la cenere, che si prende il primo
  giorno di Quarefima, e si mostra altresì come ricchissima cosa. Perché e' dicono
  veracemente, ch'egli è di smeraldo. Nel margine vi è questa postilla in nostra
  lingua. Quando, e dove e' Genovesi guadagnano el \textit{catino} di smeraldo, che tengono
  ancor'oggi nel monte di S. Giorgio, e credesi, che sia \textit{il piatto}, dove mangiò
  Cristo Giesù alla gran cena.
\end{description}
\section{Stanza LXII.}
\begin{ottave}
\flagverse{62}Scarnecchia che di guerra è un ver compendio,\\
L'Eroe degli arcibravi, e dico poco,\\
A cui dovrebbe dar piatto, e stipendio\\
Chiunque governa in qualsivoglia loco,\\
Perché quando seguisse qualche incendio\\
Ei fa il rimedio per guarir dal fuoco,\\
Mena gente avanzata a mitre, e gogne,\\
Da vender fiabe, chiacchiere, e menzogne.
\end{ottave}

\begin{ottave}
\flagverse{63}Rosaccio con altissime parole\\
Movendo il pié racconta, c'a pigione,\\
Fa per quel mese dar la casa al Sole,\\
E nel zodiaco alloga lo Scorpione;\\
Cosi sballando simil ciance, e fole\\
Si tira dietro un nugol di persone,\\
Fa per impresa in mezzo all intervallo\\
Di due sue corna un globo di cristallo.
\end{ottave}

Seguita \textit{Scarnecchia}. Questo fu un Montambanco o Ciarlatano, il quale vendeva
unguento per medicare scottature, e montava in palco sempre in abito da
Coviello col nome di Capitano Scarnecchia, e faceva una mano di braverie a
fine di ragunate il popolo, e però l'Autore lo dice \textit{compendio di guerra, ed Eroe
de li arcibravi}. E perché è Ciarlatano, lo fa capo di Monelli, e gente avanzata
alla berlina, e che è buona a vender bugie, come per lo più sono i Montanbanchi.
Dice che doverebbe esser provvisionato\footnote{``provvisionato'' è chi riceve sussidio pubblico nella Toscana granducale.}, perché ha il rimedio di liberare
dal fuoco le case, che abbruciassero, e scherza, burlando l'unguento, che vendeva
detto Scarnecchia buono a guarire le scottature in un corpo humano, facendolo
buono a rimediare a gl'incendj.

\begin{description}
\item[MITRA, o Mitera] Diciamo quel foglio, che a foggia di corona si mette in
  capo a coloro, che per delitti son frustati, o mandati in su l'asino. Vedi sotto
  C. 6. stan. 50 e C, 12. stan. 19.

\item[GOGNA] \makebox[4pt]{} È lo stesso che Berlina detto sopra C.\ 2.\ stan.\ 15. I Latini la dicono
  \textit{Numellae}, se ben questa era più tosto una specie di ceppi da serrare i piedi, onde
  forse meglio con Plauto, e con Lucilio la chiameremo \textit{collare}.

\item[FIABE, e menzogne] Sinonimi, che significano Bugie. \textit{Fiaba} da \textit{fabula}, e
  \textit{menzogna} dal verbo \textit{mentior}.

  Dopo li suddetti vien \textit{Rosaccio}, il quale conduce seco una gran mano di persone
  tirate dalle sue chiacchiere. Costui fu uno de i più superbi ciarloni, che sia
  mai stato nella Ciarlataneria, e spacciavasi per Astrologo. Non montava in
  banco, ma stava a cavallo allato a una tavola elevata, sopr' alla quale posava
  una faragine di cartapecore di privilegi havuti (diceva egli) per il suo valore
  da i maggiori Potentati della Cristianità, qualche scheretro di gatto, o cane, una
  sfera d'ottone, tre corni neri lunghi, all'uno de' quali era appeso un pezzo di
  calamita, all'altro una palla di limpidissimo Cristallo di Monte, ed al terzo un
  corno, che egli diceva essere d'Unicorno. Vendeva una sua mestura da lui chiamata
  con vocabolo Greco \textit{Nepenthes}, che diceva esser buona a tutte l'infermità
  conforme al medicamento d'Elena chiamato con queste medesimo nome di \textit{Nepenthes}
  (cioè di contrario al dolore) dal Poeta nel 4. dell'Ulissea, ed a chi la comprava
  donava un'anelletto d'osso, che spacciava per ottimo al dolor di testa,
  per esser fatto di dente di Cavallo marino. Diceva havere, imparata l'astrologia
  da un gran Mattematico, ed Astrologo suo Zio nominate Gioseppe Rosaccio,
  che predisse (vantava egli) la rovina della palla della Cupola del Duomo
  di Firenze molto tempo avanti, che cella seguisse. In somma con le ciarle, e
  fandonie ragunava sempre, che montava a cavallo, infinite persone, e pigliava
  buone somme di danari; Il Poeta lo fa condottiere di questa gente adunata con
  le chiacchiere, e gli fa fare per impresa quei tre suoi corni suddetti con la palla
  di cristallo.

\item[ALTISSIME parole] Chiama parole altissime quelle di Rosaccio; perché egli
  sempre discorreva di pianeti, di stelle, e d'altre cose celesti come mostra l'Autore
  con dire, che egli ha affittata la casa al Sole, e messo lo Scorpione nel Zodiaco.
  Senza ironia Dante Inf. 4. chiamò Virgilio; l'altissimo poeta. E poco appresso:
  Così vidi adunar la bella scola Di quel Signor dell'altissimo canto, Ove il Landino:
  Altissimo canto chiama la Poesia, la quale in ottimo, e ornatissimo canto di versi abbraccia
  tutte le dottrine, e massime la Teologia, imperoché i primi Poeti furono Teologi.

\item[SBALLARE] Vuol Propriamente dire disfar le balle, ma ci serve anche per
  esprimere uno che racconti molte, e molte cose più vicine alla bugia, che alla
  verità, ed è il medesimo, che \textit{schiantare}, che vedremo sotto C. 10. stan. 66. Questa
  voce \textit{sballare} in altro significato vedremo sotto C.~11.\ stan.~4.

\item[CIANCE, e fole] Sinonimi;  e l'ultimo è Sincope di favole; ed intendiamo
  chiacchiere lontane dal vero. Petrarca \textit{Sogni d'infermi, e fole di Romanzi}. Il
  Mauro in biasimo dell'Onore\footnote{Capitolo in Dishonor del Honore, al Prior di Iesi.} disse:
  \begin{verse}
    Hor vi dich'io, che le son tutte fole,
    Tutti argumenti da ingannar gli sciocchi,
    Le cose che consistono in parole.
  \end{verse}
  Il Persiani in una sua canzone dice:
  \begin{verse}
    Se con tagliare o fole
    Vo pagar di bravara.
  \end{verse}
  Ottavio Ferrari nelle sue Origini deduce le parole \textit{Ciance}, e \textit{Cianciare} da
  \textit{Cantiones}, \textit{Cantionare}. Il Bocc. Nov. 61. quando disse \textit{La landa di donna Matelda, e
  cotali altri ciancioni} volle dire senza dubbio \textit{canzoni}, le quali (perché erano molto
  in pregio le Provenziali, o le fatte su l'arie di Provenza, come si vede da alcune
  intitolazioni di Lande antiche) chiama come per istrazio, e contraffacendo
  in questo, sì come in molti altri luoghi, la pronunzia delle lingue straniere;
  \textit{ciancioni}; Scherzando anche nel medesimo tempo sull'altro significato, cioè di \textit{ciancia},

\item[VN nugolo di persone] Questa voce nugolo per Quantità grande, è assai usata
  dai noi, e l'usò il nostro Poeta sopra C.\ 1.\ stan.\ 50.
  Così Giuvenale Sat.\ 13.\ imitando
  in ciò Omero; chiamò la moltitudine delle combattenti grù, \textit{nubem sonoram}.

\end{description}

\section{Stanza LXIV.}
\begin{ottave}
\flagverse{64}Sopr' un letto ricchissimo fiorito \\
Portar: Pippo si fa del Castiglione, \\
Ove coperte sta tutto vestito,\\
Ch'in tal modo lo scalda al suo padrone; \\
E pur, s'in arme ei non fu gran perito,\\
Guerrier comodo almen nel padiglione.\\
Questo impera dal morbido piumaccio\\
A quelli del mestier di Michelaccio.
\end{ottave}
Seguita Pippo del Castiglioni portato in un ricco letto, di dove comanda a i soldati,
che son tutta gente senza voglia di lavorare. Costui era il più grazioso, e
faceto umore, che sia mai stato in Firenze, e si chiamò Pippo del Castiglioni, perché
servì lungo tempo a i SS.\ di Casa Castiglioni con fedeltà indicibile, e però da'
medesimi SS, amato a segno, che non ostante le burle, che in diversi tempi, ed
occasioni faceva a essi SS.\ non potettero mai mandarlo via, perché, se lo licenziavano,
egli trovava sempre vaghe invenzioni per non sen' andare, come fra le
molte fu questa: Il sig.\ Cavalier Vieri da Castiglione, al quale per ordinario
serviva, lo licenziò con queste parole: \textit{Sgombrami di Casa}. Pippo andato in Piazza
chiamò quattro Carrettai, e condottigli con le loro carrette d' avanti alla
porta dell'abitazione di essi SS.\ in su l'ora, che il sig.\ Cavalier Vieri soleva tornare
a desinare, ordinò loro, che, se il medesimo sig.\ Cavaliere gli domandasse
quello, che facevano quivi, gli rispondessero, che ve gli haveva mandati Pippo; si
come seguì ed il Sig, Cav.\ disse: che ha da far Pippo delle carrette? Ed egli a
queste parole scappato di dietro a una di esse carrette, rispose: Sgombrare, come
VS.\ Illustriss.\ m'ha comandato; Onde il Sig, Cav.\ ridendo della faceta interpretazione
del suo comandamento lo richiamò in casa, e pagati i carrettai gli licenziò.

\begin{description}
\item[IN un letto riechissimo fiorito] Il medesimo Sig.\ Cav, una sera comandò a Pippo
  che facesse, che il letto fusse caldo, quando egli tornava a dormire, che sarebbe
  stato assai di notte. Pippo si scordò di mettere il caldanetto nel letto, onde,
  tornato il Padrone, e volendo andare a dormire, Pippo si trovò imbrogliato,
  perché stante l'ora tardissima non vi era modo di trovar fuoco; ricorse però alle
  solite astuzie, e questa fu, che egli per la parte di dietro del letto v' entrò
  così vestito com' egli era, ed il padrone, credendo che egli andasse movendo
  lo scaldaletto, si spogliò da per se per non lo scioperare, e spogliato andò alla
  volta del letto, e disse: Cava il fuoco, ed alzata la cortina per entrare nel letto,
  vedde Pippo, che sollevata alquanto la testa disse: Signore il letto non è ancora
  caldo a bastanza. Il sig.\ Cavaliere vedutolo così, e conoscendo l'umore della
  bestia senz' alterarsi lo fece uscire, e toltasela in pace entrò nel letto così come
  era. E per alludere a questa, facezia il Poeta fa venir Pippo portato in un ricchissimo letto.

\item[PIVMACCIO] Guanciale lungo quanto la larghezza del letto; della grossezza
  d' un sacco ordinario da grano, ed è ripieno di piume, e però è detto \textit{Piumaccio}.
  Qui per piumaccio intende tutto il letto.

\item[QUELLI del mestiero di Michelaccio] Gente, che non ha voglia di lavorare,
  che il mestiero di Michelaccio dicono, che era mangiare, bere, e andar a spasso.
\end{description}
  Qui pure bisogna, che il Lettore si contenti ch' io faccia un poco di digressione
  per narrare alcune delle facezie del detto Pippo, meritando la graziosa sagacità
  di questo huomo, che si spenda qualche di tempo in sentire le di lui arguzie,
  il quale è vissuto fino a pochi mesi addietro d' età di 85.\ anni sempre con
  la medesima bizzarria, salvo che, dove prima frequentava molto l'osterie per
  trovar le conversazioni, che gli pagavano lo scotto, (perché mai haveva un
  quattrino, dando egli tutto quello che guadagnava alli suoi vecchi Padre e Madre,
  alli quali continovò d'ubbidire come un fanciullo fino all'età sua di sopra
  75.\ anni, che essi passando cento anni, morirono) dopo la morte del Padre frequentò
  più le Chiese pregando S.D.M.\ per la salute del Sereniss.\ G.\ Duca, dal
  quale godè fino, che visse, onorata provisione per il buon servizio reso alla Serenissima
  Casa.

  Essendo una volta il medesimo sig.\ Cav.\ Vieri al Poggio a Caiano (villa del
  Sereniss.\ G.\ Duca) a servire il Sereniss.\ Sig.\ Principe Card.\ Gio.\ Carlo, mandò
  Pippo a Firenze la vigilia del Santiss.\ Natale ordinandogli, che si facesse-dare dal
  sarto un suo vestito nuovo, e lo portasse al Poggio, e l'ordine, che gli diede fu
  con queste parole: \textit{Va a Firenze, e fatti dare dal sarto il mio vestito, e portalo}.
  Ubbidì Pippo, e la sera medesima tornò col detto vestito del padrone in dosso, ed
  entrato in Chiesa dove era tutta la Corte per udir la Messa (mancandovi
  sig.\ Cav.\ Vieri, che se ne stava in camera aspettando il vestito per metterselo) fu
  veduto da tutti i Cortigiani, e da tutti li Sereniss.\ Principi che quivi erano, ed
  il sig.\ Principe Card.\ Gio.\ Carlo gli disse: sig.\ Filippo che cosa è questa? Voi
  siate molto nobile ? Ed egli rispose: Sereniss.\ queste son grazie che mi fa il mio
  Padrone. E S.A.Rev.\ immaginandosi di come stava il fatto si rallegrò con
  Pippo, il quale fatte, più spasseggiate per la Chiesa sen'andò alle stanze del suo
  Padrone, che vedutolo con quell'abito in dosso lo sgridò dicendo, \textit{Briccone, che
    siam fratelli?} Rispole Pippo: \textit{Perché sig.?} Replicò il sig.\ Cav.\ \textit{Che furfanteria,
    è la tua mettersi il mio vestito?} Mi maraviglio di V.S.\ Illustriss.\ (soggiunse Pippo)
  non me l'ha ella donato? Come donato! (disse il Sig.Cav.) Ti par' egli abito da
  par tuo? Sig.\ sì che mi pare, e mi sta benissimo; E V.S.\ Illustriss.\ medesima
  m'ha detto, che io me lo faccia dare dal sarto, e lo porti, ed ecco ch'io l'ubbidisco,
  già tutta la Corte ha saputo questa generosità di V.S.Illustriss, e si sono
  rallegrati meco del regalo, che V.S.Illustriss.\ mi ha fatto in questa solennità.
  Il Sig.\ Cav.\ conoscendo, che non era suo decoro il mettersi quel vestito, che era
  stato veduto in dosso al suo servitore, stimò bene il quietarsi, e fargliene un regalo,
  per non poter far' altro; E cosi Pippo si godè quell'abito, che per la sua
  ricchezza era decente a un Principe.

Era grande amico di Pippo il Prete Fantacci oggi vivente Rettore della Chiesa
di Varlungo fuori di Firenze circa un miglio, il qual Prete è stato sempre huomo
assai faceto, e piacevole; e fra esso, e Pippo son seguite diverse graziose burle
e fra l'altre il Fantacci disegnò una volta di fare star Pippo senza cena, e necessitarlo
a dormire all'aria; e per questo l'invito ad andare alla sua Chiesa a Cena
quella sera appunto, che il Prete havea fermato d'essere a cena nella Villa de' SS.
Bonsi quivi vicina; e ad effetto, che gli riuscisse il disegno haveva ordinato alla
serva che andasse a dormire a casa una sua parente, e detto al Contadino, che
era presso alla Chiesa, che, se fusse accaduta cosa alcuna attenente alla cura, mandasse
al Prete di Rovezzano, Chiesa vicinissima a quella di Varlungo. Pippo chiesta,
ed ottenuta licenza dal suo padrone, la sera al serrare delle porte della Città, se
n'andò a Varlungo, e trovata serrata la porta della Casa del Prete, dopo haver
molto picchiato, conosciuto, che non era veruno in casa, disperato s'accostò alla
casa di quel Contadino, che haveva l'ordine di mandare la gente a Rovezzano.
e da esso intese, che il Prete era andato a cena fuor di cura, e gli ordini che havea
lasciato. Pippo accortosi molto bene, che il Prete l'haveva burlato, volle
rendergli la pariglia, e perciò fare trovata una scala a pivoli, con essa montò sopra
il tetto della chiesa, e quivi portata buona quantità di paglia, ed altro ciarpame
combustibile, e raro, gli dette fuoco, ed andato alle funi delle campane
si messe a suonare a rintocchi. Il Prete Fantacci, che era.poco lontano sentendo
suonare a martello, st affacciò a una finestra per sentire, che cosa fusse quella,
e veduto il fuoco sopr'alla sua Chiesa, tutto spaventato lascio la cena, e l'allegria,
e corse alla volta della sua casa, nella quale subito entrò per vedere dove
era il fuoco, e rimediarvi con l'aiuto d'una parte de' SS. Commensali, e con
una quantità di contadini, che già erano quivi concorsi con zappe, e pali per rovinare,
e tagliare dove bisognasse. Pippo intanto sceso dal tetto se n'andò
ad arno, e si fermò a cena da un tal Bonini mugnaio suo grande amico, bastandogli
d'havere sturbata l'allegria, nella quale era il Prete, il quale girato e sotto
e sopra per tutta la casa, e non havendo trovato ne meno segno di fuoco,
fece visitare il tetto della Chiesa, e trovò la paglia, che era finita d'ardere, e
vista la scala appoggiata alla muraglia, s'accorse che era stata una contraburla
di Pippo, tanto più che il contadino detto di sopra disse haverlo veduto poco
prima, e perciò sopportandosela in pazzienza, tornò a cenare, dove non mancarono
le minchionature, e barzellette, che furono da quei SS. della conversazione
dette al Prete.

Commesse una volta Pippo non fo che mancamento, per lo quale il Padrone
volle mortificarlo col mandarlo in carcere, onde gli fece dare (come è solito)
un biglietto, acciò lo portasse al Segretario del Magistrato degli Otto, qual viglietto
diceva, che fusse ritenuto il Latore in segrete fino a nuovo ordine. Pippo
prese il viglietto, e indovinatosi del contenuto, e parendogli duro havere a stare
in prigione in tempo di Carnevale, e sapendo, che il non portare il viglietto era
delitto da galera, andava mulinando come potesse salvare la capra, e i cavoli,
quando la fortuna, nell' andar' egli come la serpe all'incanto, gli fece capitare
innanzai un Tedesco giovanetto servitore di livrea del medesimo sig.\ Cav. Vieri suo
Padrone, alla volta del qual Tedesco andato Pippo, quali bravando disse: il
Padrone è in collera, che tu sei stato tanto a venire', perché voleva che tu portassi
questa lettera al Sig: Segretario de gli Otto, e perché è negozio di fretta,
mandava me; se bene, ho da fare assai fu in Palazzo; pigliala, e va via correndo.
Il buon Tedesco non pensando alla malizia porto la lettera, in esecuzione
degli ordini della quale il Tedesco latore fu ritenuto in carcere, e fu risposto
che S.A.S. era restata ubbidita. Pippo il dopo desinare'del medesimo giorno, in
vesti da donna, e senza maschera con le sue proprie basette, e barba se ne passeggiava
il corso delle maschere, havendo d' attorno un popolo infinito. S'abbatté
a vedere questo tumulto il Sereniss. G, Duca, che passava in carrozza per
quella strada, onde spedì uno staffiere per intendere che cosa fusse. Lo staffiere tornò,
dicendo che era Pippo del Cattiglioni in maschera da donna. Ma S.A.S.
che già sapeva del viglietto, replicò: non può essere, onde il Caporale de gli staffieri
andò da per se, e tornò replicando esser veramente Pippo nel modo, che haveva
detto lo staffiere; in tanto S.A.S., s'accostò, e Pippo che gli andava incontro,
ed haveva osseruato, che S.A.S. haveva mandato due volte a veder chi egli era,
fattole una grandissima riverenza disse: \textit{Sereniss, io son io, io son'io, perché il Tedesco
m' ha fatto il servizio di portar la lettera lui; Finalmente conosco hora più che mai che
chi si fa ben volere può sperar sempre questi, e maggiori servizzj}. Il Sereniss. G. Duca
rise dell'astuzia, e ordinò che fusse scarcerato il Tedesco.

Il Sig. Cav. Bernardo fratello del sig.\ Cav. Vieri haveva presa la seconda moglie.
Questa dama volendo esser servita da Pippo per bracciere, perché egli era
huomo d'età, e vestiva di nero, e non con la livrea, come gli altri servitori di
quella Casa, pregò il suo sig.\ Consorte, che lo chiedesse al fratello, perché servisse
a lei, il sig.\ Cavaliere Vieri gli compiacque, se bene con poco suo gusto,
perché era avvezzo con lui, che fuori di quelle sui bizzarrie lo serviva raramente,
e con meno gusto di Pippo, che non avvezzo a servir dame gli pareva duro haversi
ad avvezzare in sua vecchiaia, e mal volentieri lasciava il suo padrone, la
discretezza del quale non sperava trovare in chi che sia; onde pregò la Signora,
che lo volesse lasciare al servizio, che era solito; ma la Signora non volle mai
mutarsi di proposito; per lo che Pippo si gettò alle invenzioni per liberarsene
con riputazione, e con operare, che la Signora lo licenziasse, senza che egli
commettesse mancamento. Chiamò dunque a se alouni ragazzi, e distribuiti fra essi
alcuni pochi soldi, impose loro, che quando lo vedevano con la padrona, s'accordassero
tutti a gridare Pippo, Pippo, Ecco Pippo, e gli facessero il bordello dietro.
I ragazzi invitati al loro giuoco, e che havrebbono dato qualcosa a lui per havere
occasione di far quel chiasso, appena lo veddero uscir di casa, dando il braccio
alla Padrona, che cominciarono a strepitare, e ragunarono quivi quanta gente
era in quei contorni, e Pippo savio, senza mutarsi in faccia seguitava a dare il
braccio alla Signora, la quale vergognandosi, che il suo servitore fusse lo scherzo
Popolo, e che eggli fusse trattato come un pubblico buffone, s'affrettò di
giugnere in Chiesa, pensando, che quivi almeno dovesse fermarsi il baccano,
ma, se cessò il romore, non finì il tumulto, perché quei ragazzi standosi tutti
attorno, non gridavano per rispetto della Chiesa, ma erano cagione, che tutto il
popolo guardasse verso quella parte; per lo che la Signora per liberarsi ordinò a
Pippo, che andasse a casa, e mandasse un'altro servitore, e tornata poi a casa
le parve mill'anni render Pippo a chi gliel' havea conceduto; E così egli ritornò
al primo servizio, sicuro, che alla Signora non farebbe mai più venuta voglia di
farsi servire da lui.

Haveva il sig.\ Cav. Vieri una bella cagna da Fermo, la quale diede in cura a
Pippo dicendogli: Tien conto di questa cagna, ed avverti a non la smarrire,
perché se la smarrisci non ti aspettare altra licenza. Prese Pippo la cura della cagna,
e col trattarla bene l'avvezzò a fare mille giuochi, e se la rese così affezionata,
che era impossibile, che egli la smarrisse. Avvenne, che Pippo fu invitato
a una festa, che si dovea fare in un luogo poco lontano da Firenze, dove
era per trattenersi almeno tre giorni, onde chiese al padrone licenzia per a quel
tempo; ma non l'ottenne, Pippo senza mostrar di ciò disgusto, la mattina avanti
alla vigilia di detta festa comparve in casa senza la cagna, ed il sig.\ Cav. domandò
dov' ell'era. Pippo disse quasi piangendo: Sig.\ io non lo so,, quando io
fui vicino a case mia iersera ella cominciò a fuggire, e per molto, che io le corressi
dietro chiamandola, non fu possibile farla tornare, ne arrivarla. Replicò il
Sig.\ Cavaliere; Tu sai i patti; però va a fare i fatti tuoi, e non haver' ardire di
mettere il piede in casa nostra senza la cagna. Pippo fingendo un dirottissimo
pianto sen' usci di casa, e andò alla festa, alla quale era stato invitato, e passati
alcuni giorni in grandissima allegria se ne tornò a Firenze, e andato fuori della
porta alla Croce da uno Ortolano suo amico, al quale haveva lasciata la cagna,
se la prese, e l'infangò tutta, e le insanguinò l'ugna, acciò paresse spedata, e legatala
con una corda: la condusse al padrone, il quale veduto Pippo con la cagna
gli disse: Dove l'hai trovata? In Casentino, Illustriss. Sig., e non ci voleva altri
che me per trovare il luogo dov'ell'era fitta. Il sig.\ Cav. credette quanto disse
Pippo, il quale con tale invenzione godè la soddisfazione, che bramava. E tanto
basti per un saggio delle facezie di Pippo, il di cui intero nome, e cognome
era Filippo Bussi.

\section{Stanza LXV. \& LXVI.}
\begin{ottave}
\flagverse{65}A gire a Batistone adesse tocca\\
Gran gigante da Cigoli di quelli,\\
Che vanno a corre i ceci con la brocca\\
E batton con le pertiche i baccelli:\\
Per sue bellezze amore ha sempre in cocca\\
Per ferir Dame i dardi, ed e quadrelli,\\
Fa il Cavaliere nelle cavalcate,\\
E va spesso furiero alle nerbate.
\end{ottave}

\begin{ottave}
\flagverse{66}Cento suggetti egli ha della sua classe\\
Anch'eglino Pigmei distorti, e brutti,\\
Fanti che nacquer nelle magne basse,\\
Mi se ben son piccini, vi son tutti,\\
Mangian spinaci, arruffan le matasse,\\
Ed ha più vizzj ognun, di sei Margutti,\\
Cosa è questa che va per il suo dritto,\\
Che non è in corpo storto animo dritto
\end{ottave}

Segue Batistone Nano con una gran quantità di compagni uguali a lui; ma, se
bene son così piccoli, son tutti viziosissimi, e non possono essere altrimenti,
perché in un corpo mal fatto, di rado si trova anima ben composta.

\begin{description}
\item[BATISTONE] Questo fu un Nano levato da guardare le pecore, e condotto
a servire il Serenissimo Principe Mattias di Toscana, dove insuperbitosi, si messe
in sul posto di bello; e facendo lo spasimato di tutte le Dame, e però il Poeta dice:
\textit{Per sue bellezze Amore ha sempre in cocca Per ferir Dame i dardi, ed i quadrelli}, ed
arrivò a segno questa sua inclinazione alle dame, che per potere liberamente praticare
con esse, si contentò che il suo Serenissimo Padrone lo facesse castrare,
come seguì, ma però in burla, e stette nelle mani di Maestro Agnolo Santerelli
Castratore circa un mese, sempre credendo d'essere stato castrato: E perché
egli, non ostante che fusse di statura piccolissima imparò assai bene a cavalcare,
e maneggiare ogni cavallo aggiustatamente, supplendo con la mano a quello, in
che gli mancavano le gambe, era solito ancor egli andare nelle cavalcate dei
i Cavalieri, e però dice: \textit{Fa il Cavaliero nelle cavalcate}. Ma perché questa sorta
di Caramogi e assai sottoposta alle mazzate del padrone, ed egli ne haveva la
sua parte, però il Poeta dice; \textit{Va spesso Foriero alle mazzate}. Questo Nano dopo
la morte del Serenissimo Principe Mattias servi al Serenissimo Gran Duca in qualità
pure di Nano, ma esercitava anche la cucina segreta di S.A.S., nel qual mestiero
s'era fatto peritissimo, per lo che oltre alla buona provvisione e stipendio,
buscava gran mance; ma la Fortuna l'abbandonò in sul buono, perché essendosi
egli innamorato d'una bellissima giovane sua pari di natali, la prese per moglie,
ed in pochi giorni morì. Lo chiama \textit{Gigante da Cigoli}, e che era uno di
\textit{quelli che colgono i ceci con fa brocca}, come si fa de i fichi, e che \textit{battono i baccelli con
la pertica}, come si fa delle noci, non potendo arrivargli altrimenti. Di questo
Gigante da Cigoli, in una collinetta vicina a S.Minkato al Tedesco, si fra
le donnicciuole, una Iperbolica cantilena antica, la quale dice:
\begin{verse}
E d'una punta d' ago
Ne facea pugnale, e spada,
E di quello che gli avanzava
Ne faceva uno spuntoncin,
\end{verse}

E continova questa cantilena con altre iperboli retrograde simili per esprimere
la picciolezza di questo Gigante da Cigoli; e di qui e in uso comune il dire
Gigante da Cigoli a un Nano, che i Latini dissero \textit{Pumilio}, e noi diciamo anche
\textit{Pedina}, similitudine tratta dal giuoco della dama; \textit{Scricciolo} da un'uccello
piccolissimo di questo nome, \textit{Pimmeo} dalla voce Greca \textit{Pygmaios}, che significa
dell'altezza d'un pugno. I Greci dicevano \textit{Nanus, Pusillus quantus Molo}, ed altre
volte \textit{gutta}; ed un Pedante lo chiamo \textit{Titivillitium Scarabei umbra}. Famiano
Strada nelle sue Prolusioni, parlando d' un Nano dice: \textit{Fungino hic genere est,
  capite se totum tegit}, Ed altrove, pure nello stesso proposito dice: \textit{Hominus indicium,
Somninm hominis, salillum animae}.

\item[BROCCA] Voce, che viene dal Greco \textit{Brochos} secondo il Monosino, e secondo
  altri dal Greco \textit{Prochoos}; il che e più verisimile, essendo questo vaso da
  acqua, e quello vaso da vino; e vuol dire un vaso di terra per uso di portare
  acqua, e però detto \textit{Aydria}, e noi lo chiamiamo brocca;, Chiamasi brocca
  ancora uno strumento fatto di canna rifessa in più parti; se quali allargate, e rintessute
  con salci, formano come una piramide a rovescio, e di tale strumento
  fermato in cima a una pertica, ci serviamo per corre i fichi, quando non si possono
  arrivar con le mani; e di questa brocca dice nel presente luogo.

\item[FVRIERO] Si dice colui, che va innanzi a preparare gli alloggi nel viaggiare
  che fa un' Esercito, o altra gente in buon numero. Lat. \textit{metator mansionum}.
  In Latino barbaro dicesi \textit{fodrarius} da \textit{fodrum} voce che vien dal Germanico, la
  quale in buon Latino si direbbe \textit{alimentum}, \textit{pabulum}, \textit{annona}; Onde \textit{Foraggio}, e
  \textit{Foraggiare}, \textit{Provisione di guerra}, e \textit{provvedere l'esercito}. Tutto ciò si osservò dal
  Ferrari nelle Origini alle voci \textit{Foraggio}, e \textit{Foriere}, Ma erra quando piglia \textit{Friere
    dello spedale}, che si trova in Gio: Villani lib. 8. c. 95. per accorciato da \textit{Foriere},
  quasi sia \textit{Provisor hospitij} poiché quivi, si come appresso al Bocc. Nov, 92. significa
  \textit{frate} dal Franzese \textit{frere} cone si domandano anche oggi i Cavalieri di
  Malta. Qui si serve della voce \textit{Furiero} per intender \textit{furia} che suona quantità,
  come dicemmo sopra in questo Cant. stan. 50. e vuol intendere, che questo Nano
  spesso toccava qualche furia, cioè quantità di nerbate. Vedi sotto C, 9. stan. 49.

\item[PIMMEI] Erano popoli nani, che habitavano nell'ultime parti dell' Indie,
  i quali crescevano fino all' altezza al più d' un braccio, e le loro mogli di cinque
  anni partorivano, ed otto erano vecchie. Di questi fa menzione Plinio lib.~4.
  cap.~11. ove dice i barbari chiamarli Cathizi, e lib. 7. cap. 2. Costoro per esser
  così piccoli erano infestati, e rapiti dalle Gru, onde per difendersi andavano
  armati di frecce; e cavalcando sopra alle capre in grandissime schiere a guastare
  i loro nidi, e romper loro l'uova. Di questi parla Giuvenale sat. 13. dicendo.
  \begin{verse}
    Ad subitas Thracum volucres, nubemque sonoram
    Pygmaeus parvis currit bellator in armis.
    Mox impar hosti raptusque per aera curvis
    Unguibus a saeva fertur grue: Si videas hoc
    Gentibus in nostris, risu quatiare, sed illic,
    Quamquam eadem alssidue spectemur, praelia ridet
    Nemo, ubi tota cohors pede non est altior uno.
  \end{verse}

\item[NELLE magne basse] Intende che sono di statura bassa, se ben par che dica
  sieno nati nella bassa Alemagna. Lat. \textit{Germania inferior},

\item[SE bene e' son piccini, vi son tutti] Benché piccoli hanno malizia quanto un
  grande. \textit{Tydeus corpore, animo vero Hercules}; da Omero, il quale descrive Tideo
  il padre di Diomede piccolo si di statura, ma gagliardo.

\item[MARGVTTE] Che Nano fusse costui, e quanto sagace, e scellerato, vedilo
  nel Pulci nel suo Poema intitolato il Morgante ? Questo nome di Margutte forse
  fu finto dal Pulci a similitudine di \textit{Margite}, Personaggio famoso per la sua scempiataggine,
  il quale fa il suggetto d'un intero Poema burlesco di Omero; e ciò
  poté avere imparato il Pulci dal suo dotto amico messer Agnolo da Montepulciano.

\item[NON è in corpo storto anima dritta] Non è in corpo mal fatto, animo ben
  composto, giusto, e che tiri al buono; che tanto significa la voce dritto in questo
  luogo. Si dice anche: \textit{Un segnato da Dio, non fu mai buono}: (alludendo per
  avventura a Caino, Gen. c. 4. vers. 15.: quali che quel tale sia in un certo modo
  contrassegnato, affine, che ognuno, che lo vede si guardi) qual sentenza è
  praticata comunemente, e si vede da i seguenti versi maccheronici.
  \begin{verse}
    Nulla fides gobbis, \& parum credite zoppis,
    Si guercius bonus est, inter miracula scribe.
  \end{verse}
  Un' altro Poeta in questo proposito disse: \textit{Chiude un' anima bigia un corpo nero}.
  Che huomo bigio intendiamo huomo cattivo, di poca coscienza, e manco religione.
  Marziale. \textit{Crine ruber, niger ore, brevis pede, lumine laesus Rem magnam
    praestas Zoile, si bonus es}. Quel Tersite, che quanto sconcio di viso, e scontraffatto
  nel corpo, altrettanto era brutto nell'animo, e di costumi orgogliosi, e insopportabili;
  vien descritto da Omero al 2. dell' Iliade ( secondo la traduzione di
  Pietro la Badessa Messinese, stampata in Padova l'anno 1564.)
  \begin{verse}
    Lusco a' un' occhio, e d' un pié zoppo, e stretto
    Negli omeri, che gobbi ha infin' al collo;
    Aguzzo il capo, e 'l capel crespo, e raro,
    Sucido, e ner, lentiginoso, e marcio,
  \end{verse}
\end{description}

\section{Stanza LXVII.}

\begin{ottave}
\flagverse{67}Piena di sudiciume, e di strambelli \\
Gran gente mena qua Palamidone, \\
Ch'il giorno vanne a Carpi, ed a Borselli, \\
E la notte al Bargel porta il Lancione, \\
Maestro de' Bianti, e de' Monelli,\\
E veste la corazza da bastone,\\
Perch'egli quant'ogni altro suo allievo\\
È tutto il dì figura di rilievo.
\end{ottave}

Palamidone conduce seco una quantità di birboni, stracciati, e sudici come
era lui. Questo fu un guidone mezzo matto, ma tutto tristo, ed al maggior segno
birbone, il quale faceva servizio a' carcerati, e perché continovamente
brontolava, dicendo di pazze scioccherie, haveva sempre dietro una gran quantità di
ragazzi che lo facevano stizzire. La notte per guadagnar qualcosa portava dietro
al Capitano, o Caporale de' Birri un'arme in asta solita portarsi dalla Famiglia
del bargello, quando la notte va facendo la guardia, la quale arme è da noi
detta \textit{lancione}. Ma che egli rubasse non posso crederlo, perché assolutamente non
havea tanto giudizio, e stimo che il Poeta dica questo nel presente luogo, e altrove
per descriverlo per uno di quei furfanti, de' quali si può credere ogni ribalderia.
Palamidone e accrescitivo di \textit{Palamides}, Eroe noto nella guerra Troiana,
secondo la pronunzia Greca più moderna dicesi \textit{Palamide}, e non \textit{Palamedes}; onde
è fatto il soprannome di \textit{Palamidone}; che significa un lungo e sottile, come un
palo, una persona grande di statura.

\begin{description}
\item[ANDARE A Carpi, ed a Borselli] Carpi è un Principato in Italia notissimo; e
  Borselli è un luogo sul Fiorentino, e scherzando con questi due nomi \textit{Carpi} intendiamo
  carpire, cioè rubare, ed a \textit{Borselli}, cioè alle borse per rubare. Aristofane
  Poeta Greco nella Commedia intitolata i Cavalieri, citato dal Monosini
  nel \textit{Flos Italicae linguae}, (ove egli tocca la maniera di parlare Fiorentina; \textit{E'
    piglierebbe per San Giovanni}, usata anche dal nostro Poeta;) dice così: \textit{manus in
    Actolis habet}.  Vuol dire: \textit{sempre chiede, ed è apparecchiato a pigliare}; scherzando sul
  nome di certi popoli chiamati \textit{Aetoli}, per  l'allusione che ha questa voce alla parola
  \textit{atein} che significa chiedere.

\item[PORTARE il Lancione al Bargello] Questo mestiero solito farli da birro novizio,
  lo faceva alle volte Palamidone. come s'è detto.

\item[BIANTI] Si trova una specie di Bricconi, e Vagabondi che vanno buscando
  danari con invenzioni, come si vede da un libretto intitolato \textit{Sferza de' Bianti, ec,}
  E si dicono anche Monelli; se ben veramente per monelli intendiamo quei poveri,
  che si fingono stroppiati, malati, impiagati, o morti dal freddo per muover
  le persone a far loro elemosine, donde poi diciamo \textit{far il monello} quel ragazzo,
  che havendo toccate leggiermente delle busse dal Maestro, o da altri, mette a
  sogquadro il vicinato con le strida per mostrare d' essere stato dalle busse stropiato,
  ed in vero non ha mal nessuno, che si dice anche \textit{far marina}: vedi sopra
  C. 1. stan. 37. alla voce \textit{soffiano}, e sotto C. 4. stan. 8. Di questi intende il Persiani
  nei seguenti versi.
  \begin{verse}
    \backspace Signor non so se voi sapere il bando
    Di chiuder tutti dentro a' Mendicanti
    Mascalzon, vagabondi, e malestanti.
    \backspace Che vanno per le strade mendicando,
    Io che sono in arnese tanto male
    Mi ritrovo in grandissimo viluppo;
    Temo esser preso in vece d'un Gaiuppo,
    E finir la mia vita allo Spedale.
  \end{verse}

\item[VESTE la corazza da bastone] E' armato a bastonate, veste un' armatura da
  difenderlo dalle bastonate; s'intende che è sottoposto a toccare spesso delle
  bastonate.

\item[RILEVARE] Intendiamo buscare, conseguire, ottenere. Petr. Canz. 22.
  \begin{verse}
    Il sempre sospirar nulla rilieva.
  \end{verse}

  Onde se bene \textit{figura di rilievo} vuol dire statua di marmo, o di altro materiale,
  noi incendiamo \textit{rilevare}, cioè \textit{buscare} e qui intende \textit{buscar mazzate}. Il verbo
  rilevare piglia questo significato da rilievo, che sono gli avanzi delle mense de' grandi,
  quali avanzi si buscano per lo più da coloro che servono a tavola, donde diciamo
  Viver di rilievi che vuol dir Campare d' avanzi. Vedi sotto C., 5. stan. 47.
  Franco Sacch. Nov. 154. \textit{Quando la crostata fu mangiata tutta, senza far rilievo ne
    meno de' topi}, \textit{Rilevare} vuol dir Quello esprimere che fanno delle parole i ragazzi,
  quando imparano a compitare.

\section{STANZA LXVIII.}

\begin{ottave}
\flagverse{68}Comparisce fra tanto un carro in piazza\\
Da Farfarel tirato, e Barbariccia \\
Ubbidiente al cenno della mazza\\
Soda, nocchinta, ruvida, e mafficcia. \\
Con che la formidabil Martinazza\\
A lor, ch'è ch'è, le costole stropiccia,\\
E quei Demonj in forma di Camozza\\
Van tirando a battuta la carrozza.
\end{ottave}

In tanto, che si fa la mostra de' soldati di Malmantile comparisce in piazza
un carro tirato da due Demonj in forma di capra salvatica, che questo vuol dir
\textit{camozza}, la quale per lo più si trova ne i monti del Tirolo. Plin, lib. 12.cap.37
la chiama \textit{Rupicapra}. I nostri antichi dissero \textit{Stambecco}, il Lat. \textit{ibex}.

\item[FARFARELLO, e Barbariccia], Nomi di due Demonj dal nostro Poeta cavati
  da Dante, del significato de' quali nomi vedi gli Spositori sopra il medesimo
  Dante.

\item[NOCCHIVTA] Piena di nocchi, che sono quei piccioli rilevati come bolle,
  i quali si veggono per lo più ne i bastoni di pruno, di sorbo, ec, che gli rendono
  ruvidi, e li chiamamo ancora \textit{nodi}, come fanno i Latini.

\item[MASSICCE] Intendiamo tutte quelle cose, che dal peso mostrano esser fatte
  di materia stabile, e solida, e non vote, o vane, o in altra maniera fragili, o
  deboli.

\item[CH'è ch'è] Ad ora ad ora, di quando in quando, spesso.

\item[STROPICCIARE] Fregar qualcola con panno, o altro, ed i Latini \textit{Perfricare}.
  Forse è corrotto da \textit{stoppicciare}, che pare si dovesse dire, da stoppa, o stoppaccio,
  con che per lo più si stropicciano gli arnesi per liberargli dalla polvere. Ma
  \textit{stropicciar le costole a uno} vuol dire \textit{Bastonare uno}.

\item[TIRANO la carrozza a battuta] Non a battuta di musica, ma a battuta della
  mazza, con la quale Martinazza la bastona.
\end{description}

\section{STANZA LXIX — LXXI.}
\begin{ottave}
\flagverse{69}Costei è quella strega maliarda,\\
Che manda i cavallucci a Tentennino,\\
Ed egli un punto a comparir non tarda\\
Quand'ella fa lo staccio, o il pentolino,\\
Come quand'ella si unge, e s'inzavarda\\
Tutt'ignuda nel canto del cammino,\\
Per andar col Barbuto sotto il mento\\
Con la granata accesa a Benevento.
\end{ottave}

\begin{ottave}
\flagverse{70}Ove la notte al noce eran concorse\\
Tutte le Streghe anch'esse sul caprone,\\
I Diavoli col Bau, le Biliorse\\
A ballare, e cantare, e far tempone;\\
Ma quando presso al dì l'ora trascorse\\
Fa di mestieri battere il taccone\\
Come a costei, ch'or viensene di punta,\\
E in su quel carro nel Castello è giunta.
\end{ottave}

\begin{ottave}
\flagverse{71}E la cagion si è, ch'ella ne vada\\
Adesso a casa tutta in caccia, e in furia,\\
L'haver veduto dentro alla guastada\\
Un segno, che le ha data cattiv'uria;\\
Perché vi scorse una sanguigna spada,\\
C'alla sua patria minacciava ingiuria;\\
Perciè, se nulla fusse di quel regno,\\
Ne viene anch'essa a dar il suo disegno.
\end{ottave}

Martinazza è una di quelle streghe, le quali costringono il Diavolo con fare
lo staccio\footnote{stàccio s.m., arnese da cucina, simile al colino. da setaccio, per sincope.}, e il pentolino, e con ungersi per farsi portare a Benevento al congresso
de' Diavoli sotto il noce: Questa Martinazza adesso, si fa riportare furiosamente
da quei Demonj a Malmantile, perché ha veduto nella caraffa una spada
sanguigna, che le presagisce la caduta di Malmantile, onde vi si vuol trovare
ancor'essa per dare il suo aiuto. Questo nome di Martinazza è nome a caso; E
quella strega, e stregherie son tutte dal Poeta dette per accennare l'opinione d'alcune
donnicciuole, le quali portate dall'illusioni diaboliche, si danno a credere
d'havere effettivo commerzio col Diavolo.
\begin{description}
\item[STREGA] Vedi sopra C, 2. stan, 11. Viene da \textit{strix} uccello notturno così
  detto a \textit{stridendo}, secondo Ovid, fast. 6.
  \begin{verse}
    Est illis strigibus nomen, sed nominis huius,
    Causa, quod horrenda stridere nocte solent.
  \end{verse}
E questo uccello (che forse era l'Arpia, ma Plinio, dice, che non si sa qual si
fosse) credevano gli antichi più superstiziosi, che rapisse i bambini dalle culle:
\textit{Et ab huius avis nocumento striges Latini appellabant mulieres puellos fascinantes suo contactu}.
E di qui ancor noi le chiamiamo streghe, che tanto vale quanto \textit{maliarde}
da far malie, fattucchierie, ed incantesimi, e però chiamate ancora \textit{Veneficae}.

\item[MANDARE un cavallucio] Mandare una citazione, cioè chimare uno in
  giudizio criminale con polizza. E queste polizze de' Giudizzj Criminali in Firenze
  si dicono cavallucci a differenza di quelle de' giudizzj Civili, che si chiamano
  Citazioni; e questo nelle polizze criminali è stampata l'impresa, o
  contrassegno del Magistrato criminale, che e un' Huomo a cavallo armato; qual
  contrassegno è chiamato comunemente Cavalluccio.

\item[TENTENNINO] Nome dato dalle nostre donne al Demonio per non lo chiamare
  Diavolo; quali tentatore; col qual nome è nominato presso San Matteo
  Cap, Vers. 3.

\item[FA lo staccio, e il pentolino] Favoleggiano, che quelle donne Maliarde, e Streghe,
  che habbiamo detto, sappiano fare diversi incantesimi per ritrovare cose
  perdute, e per ottenere altri loro intenti, e fra questi incantesimi \textit{fare lo staccio}
  o \textit{il Pentolino}, o \textit{la caraffa}; sì che dicendo \textit{Fa lo staccio, e il pentolino} intende
  fa incantesimi. Quei che indovinano per via di staccio sono detti dai Greci \textit{Coscinomanteis}.

\item[COME quand'ella s'unge, e s'inzavarda] Inzavardare, è uno impiaftrare con
  materia morbida, e viscosa, atta a distendere come il lardo. Il Poeta seguita,
  la vana, e superstiziosa opinione, che queste tali donne vadano ogni tanti
  giorni al congresso de' Diavoli sotto il Noce di Benevento: \textit{Ove la notte al noce
    eran concorse}; al qual luogo dicono esser portate dal Diavolo in forma di caprone,
  che questo intende \textit{il Barbuto sotto al mento}, e cavate dalle loro case per la gola
  del cammino (e però dice \textit{nel canto del cammino}) dal medesimo diavolo forzato a
  far tal funzione da quegli untumi, che dice essersi messa addosso la medesima
  donna; la quale poi a detto congresso \textit{fa tempone}, cioè si da buon tempo; si piglia
  tutti quei piaceri, che le vengono in fantasia quella notte; Ma sul far del giorno
  le convien partire, e il Diavolo in un baleno la riporta al suo paese. Tale opinione
  hanno simili scimunite; ed o sia per effetto di matrice, o pure per opra
  del Diavolo, che per illusione faccia loro apparir per vere tutte quelle scioccherie,
  che esse si fingono nella testa, l'effetto è, che esse si credono d'esser'andate
  veramente a Benevento, ed essere state riportate dal Demonio al loro paese,
  quando effettivamente non si sono mosse del letto.

\item[GRANATA] È un mazzetto di scope, o d' altra cosa simile, che s' adopra
  spazzare, e ripulire le stanze. E con queste granate accese in mano dicono,
  che tali streghe vadano cavalcando sopra un Caprone al detto Noce di Benevento.

\item[BAU, e Biliorse] Questi nomi bau, biliorse, orco, befana, versiera, e altri
  simili, sono tutti inventati dalle Balie per spaventare i bambini, e rendergli ubbidienti,
  persuadendo loro, che questi sieno spiriti infernali, e però il Poeta numera
  fra i Diavoli il Bau, e le Biliorse, per accomodarsi alla capacita de' Fanciulli,
  per li quali professa d'haver composta la presente opera. Vedi sopra C. 2, stan. 50.
  I Greci il cembalo per chetare i bambini dicono \textit{Catabau}.

\item[FAR tempone] Darsi bel tempo; Stare allegramente, pigliandosi tutti quei
  gusti che uno può, e sa pigliarsi, che diciamo anche \textit{sguazzare}, \textit{trionfare}, \textit{far
buona cera}, \textit{Genio indulgere}, \textit{litare Genio}, dissero i Latini. La Compagnia della
Lesina insegnando, in qual luogo si deva pigliare la casa per risparmiare, dice:
\textit{Vorriano le nostre case esser in una quasi dall' altre separata contrada, lontana da vie, e
piazze pubbliche, dove all' occasioni si festeggi, e si faccia trebbi, e tempone}.

\item[BATTER il taccone] È lo stesso, che \textit{batter la calcosa}, detto sopra questo
C. stan. 60, cioè camminar via; andarsene. Si dice anche \textit{battersela}; E \textit{taccone} si
dice il suolo della scarpa, cioè quella parte, che posa in terra. In questo senso trovasi
nei Latini \textit{solum vertere}.

\item[VENIR di punta] Venir con velocita, a dirittura; che diciamo anche \textit{venir di
vela}. Vedi sotto C. 6, stan. 10, Credo sia originato dalle barche, le quali si dice
\textit{venir di punta} quando vengono a dirittura senza volteggiare.

\item[IN caccia, e in furia] Cioè in fretta, frettolosamente, e con furia, come fanno
  coloro, che son cacciati; che però diciamo; \textit{Corre, che par che'egli habbia i birri
dietro}, \textit{Incedit quasi in fugam versus}.

\item[GVASTADA] Specie di vaso di vetro per uso di conservarvi liquori, ed è lo
  stesso, che caraffa dai Latini detta \textit{Phiala}, L'Autore disse sopra nell'ottava
  antecedente, che Martinazza era solita fare lo \textit{Staccio}, e il \textit{Pentolino}, e qui dice
  la \textit{Guastada}; queste maliarde, e streghe empiono di superstiziosi liquori una caraffa,
  o guastada, e facendovi mirar dentro da un fanciullo innocente, gli fanno
  dire di vedervi dentro quel che hanno desiderio di sapere, e tutto per ingannare
  le persone semplici, e cavar loro denari di mano. Questo indovinare per via d'acqua,
  fu anticamente presso i Persiani, e da' Greci si chiama \textit{Hydromantia}. Da
  questo habbiamo un detto \textit{Gli ha il diavolo nell'ampolla} per intendere: Costui indovina
  ogni cosa.

\item[CATTIV' uria] Cattivo augurio. Questa voce Vria corrotta da augurio usata
  per lo più dalle donnicciuole, detta senza aggiunta di cattiva, o buona, s'intende
  cosa, che non piaccia. \textit{La tal cosa mi dà uria}: e s'intende mi dà fastidio,
  mi da impedimento, mi da noia; da che si può credere che sia usata in vece di
  uggia, che pure vuol dir noia, fastidio, impedimento, ec. o forse in vece d'\textit{ubbia},
  che suona lo stesso, che \textit{uggia}, o forse in vece d'\textit{ombra}, che è il medesimo,
  quando vale per impedimento, \textit{la tal cosa mi dà ombra}, per \textit{la tal cosa mi dà noia}, ec.
  Sì che \textit{uria}, \textit{uggia}, \textit{ubbia}, ed \textit{ombra} suonano tutte lo stesso; \textit{uría}, e \textit{ubbía} sono
  usate per lo più dalle donne, e l' altre son più comuni. Si potrebbe anche dire
  secondo il Monosino, che la voce \textit{uria} venisse dal greco \textit{vria}, che suona vento
  prospero, e che sì come habbiamo per costume di dire buona, o o cattiva \textit{sorte},
  quantunque \textit{sorte} significhi assolutamente bene, e felicità; così habbiamo per costume
  di dire buona, o cattiva \textit{Vria}, quantunque \textit{Vria} significhi sempre felicità, secondo il
  Greco \textit{Vria}. Nello stesso modo, benché presso i Francesi \textit{heur} significhi sorte, felicità;
  voce a loro derivata similmente dal Latino \textit{augurium}; dicono \textit{bonheur}, e
  \textit{malheur}, quali \textit{buona}, e \textit{cattiva uria}, cioè buona, e mala ventura; e però volendoci
  servir bene di questa parola Uria, come vocabolo di mezzo, dovremmo aggiungerci
  buona, o cattiva, e non dirla assolutamente, e senza detta aggiunta,
  come habbiamo accennato, che molti se ne servono; ma l'uso ci libera da tali
  astruse stiracchiature. '

\item[SE nulla fusse] Per tutto quel che potesse succedere, Se accadesse qualche disgrazia.
  I Latini in un simil modo per isfuggire il cattivo augurio, e non nominare
  cosa infausta, come è la morte, dicevano: \textit{Si quid patiar}. \textit{Si quid mihi
    humanitus acciderit}, Se Dio facesse altro di me, con tutto ciò, ec.h

\item[NE viene anch' essa a dare il suo disegno] Con queste parole mostra l'Autore
  quanta gelosia haveva Martinazza di non perdere l'autorita, che teneva sopr' a
  Malmantile, ed il sospetto di non esser levata dal grado di Salamistra, che
  godeva, come accennammo sopra in questo C, stan, 54.
\end{description}

\section{Stanza LXXII. — LXXIV.}

\begin{ottave}
\flagverse{72}Fuggì tutta la gente spaventata\\
All'apparir dell'orrido spettacolo,\\
La piazza fu in un' attimo spazzata,\\
Pur un non vi rimase per miracolo,\\
Così correndo ognuno all'impazzata\\
Si fé l'un l'altro alla carriera ostacolo;\\
Chi dà un'urton, quell'altro dà un tracollo,\\
Chi batte il capo, e chi si rompe il collo.
\end{ottave}

\begin{ottave}
\flagverse{73}Figuriamci vedere un sacco pieno\\
Di zucche, o di popon sopr' a un giumento,\\
Che rottasi la corda, in un baleno\\
Ruzzolan tutti fuor sul pavimento,\\
E nell'urtarsi batton sul terreno:\\
Chi si perquota, e chi s'infranga drento\\
Chi si sbucci in un sasso, e chi s'intrida,\\
Ed un altro in due parti si divida.
\end{ottave}

\begin{ottave}
\flagverse{74}Così fa quella razza di coniglio, \\
Che nel fuggir la vista di quel cocchio \\
Chi se rompe la bocca, o fende un ciglio\\
E chi si torce un piede, e chi un ginocchio; \\
A tal che in veder quello scompiglio,\\
Io ho ben preso (dice) qui lo scrocchio,\\
Mentre a costor così comparir volli:\\
Sapeva pur chi erano i miei polli,
\end{ottave}

Il Poeta descrive assai vagamente il timore, e lo spavento, che eatro addosso a
quei di Malmantile per la vista del Carro di Martinazza, la quale vedendo coloro
così spaventati, si pente d'esser quivi arrivata in quella guisa.

\begin{description}
\item[IN un attimo] In un momento. Corrotto da atomo. Si dice anche \textit{In un baleno} ,come
nell' ottava 73. seguente. \textit{In un batter d'occhio}. V. sotto C, 10. stan, 42. dal Lat.
\textit{Ictu oculi}. \textit{En atomo} dissero i Greci. Dante Inf. C. 22. \textit{Subito, e spesso a guisa di baleno}.

\item[NON ve ne rimase un per miracolo] Fuggiron tutti, che non ve ne restò pur'
  uno. Tanto esprimeva se havesse detto: \textit{Non ve ne restò pur' uno}, Ma col dire
  \textit{miracolo} da maggior' emfasi, e seguita l'uso; e vuol dire sarebbe stato creduto miracolo
  se un solo vi fusse restato.

\item[ALL'impazzata] A caso; Come fanno i pazzi, cio senza considerar quello
  che facevano, o dove essi andavano. È il latino \textit{perperam}.

\item[URTONE] Percossa che si dà con tutta la vita in un' altra persona, o in un
  muro, o altrove, ed è lo stesso, che Spinta, ne vi so fare altra differenza se
  non che \textit{Urtare} vuol dir percuotere a caso, ed è il Latino \textit{offendere}; e \textit{Spingere}
  vuol dir Mandar uno innanzi, o indietro con violenza, ed è il latino \textit{impellere};
  Ma nondimeno \textit{urtone}, \textit{spinta} si pigliano l' uno per l'altro, se bene non si direbbe
  Dare una spinta in un muro, o altra cosa immobile, che fatta mobile come
  farebbe un muro sciolto per farlo rovinare, si direbbe Dare una spinta. A
  un'albero quasi reciso da piede per atterrarlo, si direbbe Dar la spinta per farlo
  cadere, ec.

\item[TRACOLLO] Accennamento di cadere. \textit{Extra collum pedis ire}; o pure detto
  così quasi \textit{Tracrello}. Vocabolario della Crusca. Tracollato addiettivo da tracollare,
  che vale lasciar' andar giù il capo per sonno, o simile accidente.

\item[GIUMENTO] Si dice propriamente l'asino  benché s'intenda anche ogni bestiaccia
  da soma. Così presso i Latini: Quello che in S. Gio, cap. 12, è chiamato
  \textit{pullus afinae}, in S. Matteo cap, 21, è detto \textit{pullus filius subiugalis}, \textit{Puledro},
  \textit{figliuolo della giumenta}.

\item[RVZZOLARE] Girare per terra; che diciamo anche Rotolare.

\item[INFRANGERSI] Sflagellarsi, ammaccarsi, disfarsi. Vedi sotto C. 4. stan.
76. C.11, stan. 12.

\item[RAZZA di Coniglio] Gente timida, e codarda. Si dice \textit{poltrone come un coniglio},
  perché questo animale, che è specie di lepre; come quella è timidissimo.

\item[PIGLIAR lo scrocchio] Ingannarsi, Far' errore. Lo sono stato a cena con voi,
  credendo di star bene, ma ho preso lo scrocchio; cioè mi sono ingannato, perché
  sono stato male. Il proprio significato della parola, scrocchio è quando uno per
  trovar danari, piglia a credenza una mercanzia per venticinque scudi, la quale
  non ne vale venti, e poi la vende a quindici, e questo si dice pigliar lo scrocchio.
  Plauto disse: \textit{Emere coeca, vendere oculata die}. Vedi sotto C. 6. stan. 60. E da
  questo, quando noi facciamo una cosa, che non ci torna poi bene, ne in nostro
  utile, e gusto, ma più tosto ci è di danno, si dice \textit{pigliar lo scrocchio}.

\item[SAPEVO chi erano i miei polli] Sapevo di che qualità eran costoro, è il Latino
  \textit{Cognosco oves meas}.
\end{description}

\section{Stanza LXXV. \& LXXVI.}
\begin{ottave}
\flagverse{75}Scese dal carro poi per impedire \\
Così gran fuga, e rovinosa fola;\\
Ma quei viè più si studiano a fuggire, \\
E mostra ognun se rotte ha in pié le suola, \\
Chi finalmente, come si suol dire \\
Chi corre corre, ma chi fugge vola, \\
Ond'ella, ben che adopri ogni potere, \\
Vede che fara tordo a rimanere.
\end{ottave}

\begin{ottave}
\flagverse{76}Perciò si ferma strambasciata, e stracca,\\
Ritorna indietro, ed un de' suoi caproni\\
Dalla Carretta subito distacca,\\
E gli si lancia addossa a cavalcioni;\\
Così correndo tutta si rinsacca,\\
Perché quel Diavol vanne balzelloni;\\
Pur (dicendo: arri là, carne cattiva)\\
Lo fruga sì, ch'al fin la ciurma arriva.
\end{ottave}

Martinazza scese dal carro per fermar quella gente, che fuggiva, e si messe a
correr lor dietro, ma allora sì, che coloro fuggivano, onde ella montata sopr'a
uno di quei caproni al fine gli arrivò. E qui termina il terzo Cantare.

\begin{description}
\item[FOLA] Quantità di popolo, che furiosamente corre a qualche luogo; Traslato
  da i Cavalieri, che giostrano, che dopo, che si sono soddisfatti li concorrenti
  a uno per volta a giostrare, in ultimo corrono al Saracino (così chiamano una
  mezza figura, o busto, di Moro, o Saracino, fatta di legno, e fitta in un palo)
  corrono dico al Saracino tutti in truppa, uno però dopo l'altro, e questo dicono
  \textit{far la fola}, In Latino potrebbe dirsi: \textit{exerceri ad palum}. Vegezio de re militari
  lib. 1. cap. 14. \textit{Tyro, qui cum clava exercetur ad palum, bastilia quoque ponderis
  gravioris, quam vera futura sunt iacula, adversus illum palum tamquam adversus hominem
  iactare compellitur}. E si dice \textit{fola}, o \textit{folata} d'uccelli, di popolo, ec, per
  intender di cose che velocemcate si muovono.in quantità, e presto finiscono. \textit{Folata}
  di vento, Studiare \textit{a folate}. Lavorar a folate, ec, Forse meglio \textit{folla}, che significa
  quel che i Latini dicono \textit{Magna hominum vis, vel turba, aut summa frequentia
    hominum}, Sì come noi dal calcare le strade, che fa il popolo e dallo esser calcati,
  e stretti, diciamo Una molticudine numerosa di gente, una gran \textit{calca}: così i
  Franzeei nella lor lingua la dicono \textit{foule}, cioè folla dal verbo \textit{fouler}, calpestare,
  calcare. Da \textit{folla} abbiamo fatto \textit{Affollarsi}, e \textit{Folto}, denso, calcato; Onde
  \textit{Afoltarsi}, \textit{far furia}, \textit{far pressa}: lo stesso quasi che \textit{Affollarsi} tutto derivando per
  avventura dal Latino \textit{follis}, nel quale sta l'aria serrata in modo, che più non ve ne può
  capire.

\item[STVDIANDOSI] Il verbo \textit{studiarsi} per affaticarsi a far presto, o spedire
  una cosa, che diciamo anche menar le mani. Per esempio: studiatevi, perché il
  tempo è breve, e non finirete, se non fate presto. Qui intende s'affaticavano a
  fuggire. \textit{Operi instare}: al che s'adatterebbe i verbo \textit{incumbo}, \textit{laboro}, ed anche
  \textit{studeo}, e questo dal Greco \textit{speudo}, \textit{affrettarsi}. Nel Salmo: \textit{Domine ad adivandum,
    me festina}. \textit{Signore Iddio, studiati d'aiutarmi}. Orazio.  \textit{Sic festinanti semper
    locupletior obstat}, \textit{a colui che si studia d'arricchire il più ricco dà impaccio}.

\item[MOSTRAR le suola delle scarpe] Corser velocemente; perché così s'alzano
  assai i piedi, e si mostrano le suola delle scarpe. I Greci pure dicevano in questo
  proposito \textit{Cavum pedis ostendere}, Si dice anche \textit{Battere il taccone}, che vedemmo
  sopra in questo C, stan. 79.

\item[CHI corre corre, ma chi fugge vola] Detto sentenzioso, che significa, che molto
  più forte corre quello, che è perseguitato, che non corre colui, che lo perseguita,
  perché la paura gli mette l'ali a' piedi, e per questo dice \textit{Chi fugge vola}.
  Vergilio disse: \textit{Pedibus timor addidit alas}, e Dante Inf.C. 22.
  \begin{verse}
    E poco valse, che l' ali al sospetto,
    Non potero avanzar. \makebox[3em]{\dotfill}
  \end{verse}
  Intendendo, che il gran timore, che hebbe del Demonio quel dannato, lo fece
  esser più veloce, che l' ali di quel Demonio, che gli correva dietro. Della parola
  \textit{Fugit} spiegantissima della velocità appresso Vergilio, vedi Seneca Epist, 108.

\item[FARE tordo a rimanere] Cioè rimarra a dietro, e non arrivera quella canaglia.
  Il giuoco de' tordi ha qualche similitudine con l'Amilla de' Greci, \textit{Quia de
    certo iactu inter ludentes certemen est}, come dice il Buleng. de Ludis Veterum cap.
  14., e la gara si dice in Greco \textit{amilla}. Nell'Amilla si tirava una palla dentro a
  un segno, o circolo, e colui perdeva, la di cui palla usciva, o non entrava nel
  circolo. Nel tordo non si fa ne segno, ne circolo, ma si tira una piccola palla,
  (da noi a distinzione dell'altre palle detta \textit{grillo}, come vedremo sotto C. 6. stan. 22)
  e colui, che la tira dice: \textit{A passare}, cioè a passare con la palla il detto
  grillo, o a rimanere, cioè restar con la detta palla di qua dal detto grillo; così
  tirando ciascuno, s'ingegna di passare, o rimanere il più vicino a detto grillo, che
  egli può; perché chi meno lo passa, o meno addietro gli rimane vince la posta,
  ed a quelli, che non passano, o non rimangono, quando devon rimanere, o passare,
  vince il doppio, e questi perdenti si chiamano Tordi, e sono di tre sorte,
  perché tre sono i casi del tiro; cioè Tordo a passare è quello, che passa di là dal
  grillo quando deve rimanere. Tordo a rimanere quello che rimane di qua dal
  grillo, quando deve passare. E Tordo semplicemente si dice quello, la di cui palla
  resta in dirittura del grillo per banda, e questo da alcuni si fa che non vinca, ne perda,
  da alcuni, che perda solo la metà degli altri tordi, se è più lontano dal grillo
  di quello che vince, e se è più vicino non perde; da alcuni gli è permesso ritirare
  fino a tre volte, quando però sempre resti in dette tre volte nella medesima
  dirittura del grillo; e quando non passi, o non rimanga perde una sola posta: e
  sempre s'intenda passata, o rimasta la palla quando fra essa, e il grillo possa
  interporsi un filo in squadro, se però non 1o tocchi non per banda, ma per quella
  parte, dove ha da rimanere, o restare; e tutto si fa secondo le convenzioni, e
  patti. Questo giuoco per lo più è usato da' ragazzi, o dagl'infimi bottegai di
  Firenze; i quali nei giorni delle feste, uscendo dalla Città per andar' a pigliar'
  aria nel camminare giuocano a questo giuoco, e segnano i danari di mano in mano
  a chi perde, e quando n' hanno segnati tanti, che servan loro per comprar da
  bere, e da mangiare, si fermano alla prima Osteria, e quivi ognuno paga quella
  quantità di danaro, che ha perduto. Hor tornando a proposito dice, che Martinazza
  \textit{farà tordo a rimanere}, ed intende, che rimarra a dietro, e non arriverà
  quella ciurma.

\item[STRAMBASCIATA] Affannata; Oppressa dall'ambascia, che è una certa
  difficultà di respirare cagionata dalla violente fatica nel correre, che muove
  soprabbondanza d'alito. Dante Inf. C.24. \textit{E però leva sù; vinci l'ambascia}. Di
  qui per avventura \textit{Ambasciadore}, che piglia a fare \textit{ambascia}, cioè viaggio per andare
  a quel Personaggio, o Città, a cui egli è inviato.

\item[SI lancia] Si getta; cioè con un salto montò prestamente a cavalcioni al caprone.
\item[SI rinsacca] Assomiglia Martinazza (che cavalcata in ful suo Caprone corre)
  a quando s'empie un sacco di roba leggieri, la quale si mandi giù con fatica, e per
  stivarla, ed empier bene il sacco, questo s'alza, e s'abbassa squotendolo, e così
  faceva Martinazza a cavallo in sul Caprone, il quale faceva a lei questo effetto
  andando \textit{balzelloni}, cioè a salti, come è il proprio correr delle capre. Questa
  voce \textit{balzelloni} viene da \textit{balzellare}, che lo diciamo il saltellar delle lepri nel tempo
  di Maggio, e Giugno, che elle sono in amore, e la caccia che in tal tempo si fa
  si dice andare al \textit{balzello}. Del cavalcare la bestia nera, e cornuta V. Bocc. 8.9.

\item[ARRI là] Cammina là, Va là. Termine stimolatorio usato per asini, e muli,
  ec, dai vetturali. È ben vero, che vedendosi uno a Cavallo, che vi stia su sconciamente,
  si suol dire per derider colui \textit{Arri là} quasi diciamo va a cavalca un asino,
  e portato da questo uso l'Autore fa dire a Marcinazza \textit{Arri là}. Il Monosini
  lo fa venire dal Greco \textit{Errhe}, cioè, \textit{va via}.

\item[CARNE cattiva] Animale vituperoso. Diciamo \textit{carne cattiva}, o \textit{cattivo pezzo
  di carne} ancora a quegli huomini, che sono di genio sciagurato, e maligno. Onde
  si dice quasi in proverbio, e per ironia di chi sia magro, o piccolo di persona,
  ma sia maligno, e astuto, e come si dice ne' suoi panni e' vi sia tutto, \textit{Egli è come in
  Stornello, poca carne, e cattiva}, E qui si può anche dire, che l'Autore la chiami
  \textit{carne cattiva}, perché era capra, che fra le carni, che si mangiano, è la più cattiva.

\item[CIURMA] Dal Lat. \textit{turma}. Si dice propriamente degli Schiavi remiganti di galera:
  Ma si Piglia ancora per quantità di gentaglia, e qui intende di quella canaglia,
  che fuggiva. Vedi sotto C. 5. stan. 16., e C.~11, stan. 16.
\end{description}

\section*{FINE DEL TERZO CANTARE.}
\chapter{Quarto Cantare}

\begin{argomento}
I guerrier di Baldon son mal disposti
Perché la fame in campo gli travaglia;
Il fendesi, e Perlon lasciano i posti,
Non vedendo arrivar la vettovaglia.
Psiche non tiene i suoi pensieri ascosti
A Calagrillo Cavalier di vaglia,
Che promette aiutar la damigella,
E poscia ascolta una gentil novella.
\end{argomento}

\section{Stanza I. — IV.}
\begin{ottave}
\flagverse{1}\textit{Omnia vincit amor}: dice un Testo,\\
E un'altro disse, e dette più nel segno:\\
Fames Amorem superat. E questo\\
È certo, e approva ognun c'ha un po d'ingegno\\
Perché quantunque Amor sia sì molesto,\\
Che tutti i Martorelli del suo Regno\\
Dicano ogn'ora; Ahi lasso, io moro, io pero,\\
E non si trova mai, che ciò sia vero,
\end{ottave}

\begin{ottave}
\flagverse{2}Non ha che far niente con la fame,\\
Che fa da vero, pur ch' ella ci arrivi;\\
Posson gli amanti star senza le dame\\
I mesi, e gli anni, e mantenersi vivi;\\
Ma se due dì del consueto strame\\
I poveracci mai rimangon privi,\\
Ei basta, che de fatto andar gli vedi\\
A porre il capo dove il Nonno ha i piedi.
\end{ottave}

\begin{ottave}
\flagverse{3}Tal che si vien da questi effetti in chiaro,\\
Che d'Amore, la fame e più potente,\\
Ond'è c'ognun di lui più questa ha cara,\\
E quand'alle sue hore ei non la sente\\
Lamentasi, e gli pare ostico, e amaro;\\
Perciò riceve torto dalla gente,\\
Mentre ciascun la cerca, e la desia,\\
E s'ella viene, vuol mandarla via.
\end{ottave}

\begin{ottave}
\flagverse{4}Anzi la scaccia, come un'animale\\
Sul buon del desinare, e della cena,\\
Per questo ella talor, che l'ha per male,\\
Più non gli torna; ovver per maggior pena\\
In corpo gli entra in modo, e nel canale,\\
Che non l'empierebbe Arno con la piena,\\
Come vedremo, c'a Perlone ha fatto,\\
C'a questo conto grida come un matto.
\end{ottave}

Il nostro Poeta riflettendo, che nel presente Cantare gli convien descrivere la
fame, che era nel campo di Baldone, per non esservi ancora comparsa la munizione
di bocca, s'introduce col provare, che la fame è superiore ad Amore, quantunque
la maggior parte degli huomini, seguitando Vergilio Egl. 10. dove cantò:
\begin{verse}
  Omnia vincit amor; \& nos cedamus amori.
\end{verse}
dica che Amore sia più potente,e superi qualsivoglia passione. E dopo haver
provata questa sua intenzione, si maraviglia per qual causa la Fame, essendo
più potente, e più stimabile, e desiderabile, che non è Amore, habbia poi ad essere
scacciata nella maniera, che ognun procura di fare; considera però, che ella
habbia ragione di vendicarsi di tal disprezzo, e con l'andarsene in sul più bello
del mangiare, o col venir troppo, quando non si ha che mangiare, come vuol
mostrare ch'è seguito a Perlone.

\begin{description}
\item[MARTORELLI del regno d'Amore] Innamorati, travagliati, martirizzati da Amore.

\item[AHI lasso] Interposizione, che denota dolore. Quasi dica son lasso, e stanco
dal dolore, dal travaglio, ec. È il Lat. \textit{heu}, \textit{hei mihi}. Francese \textit{Helas}.

\item[NON ha che far niente] Non c'è luogo da far comparazione. Non è nulla,
rispetto alla fame.

\item[STRAME] Si dice il fieno, paglia, o altro simile che si dà per vitto alle bestie:
  Ma qui lo piglia per cibo degli huomini, come è scherzoso costume; e diciamo
  \textit{strameggiare}, quando uno va trattenendosi col mangiare alquanto, aspettando
  che venga in tavola la vivanda per desinare, o per la cena, che si dice
  \textit{sbocconcellare}. Vedi sotto C, 7. stan. 10,

\item[POVERACCIO] Epiteto che esprime la compassione, che s'ha della disgrazia
  di colui, il quale si nomina. Vale per infelice, disgraziato, ec.

\item[PORRE il capo dove il Nonno hai piedi] Farsi sotterrare. Morire. Nella Scrittura
  si dice; \textit{Apponi ad patres suos}.

\item[RICEVE torto] Non se le fa il giusto: Non se le fa il dovere, \textit{Torto} è il contrario
  di \textit{diritto}. E significano questo Giusto; e torto Ingiusto, come vedemmo sopra
  C. 3. stan. 66. \textit{Non è in corpo storto animo dritto}.

\item[ANIMALE] E' nome generico, che significa ogni' specie di vivente; Ma è
  costume pigliarlo in specie, e per \textit{animale} intender solamente le bestie, donde segue
  poi che dicendosi animale a un huomo s'intende un huomo senza ragione, o
  giudizio, in somma un huomo bestia. Bocc.n.79, dice: \textit{Conoscendo questo medico
  esser un'animale}, Vedi sotto in questo C. stan. 51. Cic. Nonne vides, bellua?

\item[IL canale] Cioè il canal del cibo, che è la gola: il \textit{condotto de' bocconi}, che
  così vien descritto in lingua furbesca dalla plebe Fiorentina.

\item[NON empierebbe Arno con la piena] Non l'empierebbe Ano, quando per le
  pioggie vien grosso. Iperbole usata per intender'uno, che non si sazzi mai, ingordo
  tanto del cibo, quanto dei denari, che i latini dissero \textit{Dolium inexplebile}
  d'un huomo, \textit{quem eos non nutriet, illum nec AEgyptius}. Empiti Arnaccio: dicesi
  per dispetto a uno, che non si trova mai sazio; modo basso.
\end{description}

\section{Stanza V. \& VI.}

\begin{ottave}
\flagverse{5}Desta l'Aurora omai dal letto scappa,\\
E cava fuor le pezze di bucato,\\
Poi batte il fuoco, e quocer fa la pappa\\
Per il giorno bambin c'allora è nato;\\
E Febo ch'è il Compar già con la cappa,\\
E con un bel vestito di broccato,\\
C'a nolo egli ha pigliato dall'Ebreo,\\
Tutto splendente viensene al Corteo.
\end{ottave}

\begin{ottave}
\flagverse{6}Ne per ancora l'Ugnanesi genti\\
Hanno veduto comparire in scena\\
La materia che dà il portante ai denti,\\
E rende al corpo nutrimento, e lena;\\
Perciò molti ne stanno mal contenti,\\
Che son'usi a tener la pancia piena,\\
E ben si scorge a una mestizia tale,\\
Che la mastican tutti più che male.
\end{ottave}

Il nostro Poeta (come habbiamo detto altrove) hebbe notizia da Saluadore
Rosa d'un libro Napoletano intitolato \textsc{Lo Cvnto de li Cvnti}, ed in
comporre l'aggiunta alla presente opera se ne valse, cavandone qualche pensiero,
o concetto, come vedremo; e questo è quello della presente descrizione della levata
del Sole. Dice dunque che \textit{svegliata l'Aurora, esce del letto, e cava fuora le
pezze bianche di bucato}; il che allude alla chiarezza che apporta l'Alba. Di poi
\textit{accende il fuoco, e fa quocer la pappa per darla al Giorno bambino che allora è nato}.
E per questo fuoco intende quell'albore che si vede all'apparir dell'Aurora, il
quale va crescendo, e piglia un colore gialliccio per lo vicino apparir del Sole;
e però dice che \textit{Febo viene con  abito di broccato d'oro tutto splendente al Corteo del
giorno bambino}. E così intende che alla levata del Sole i Soldati di Baldone non
ancora havuta la provvisione per vivere, onde sono in collora, e particolarmente
molti di loro, che sono assuefatti a star sempre col ventre pieno.
\begin{description}

\item[PEZZE di bucate] Pezze bianche pulite perché sono di bacato, cioè non adoprate
  dopo che furono imbucatate; ed intende quei panni lini, che servono per fasciare, ed
  involtare i bambini.
\item[BATTE il fuoco] Accende il fuoco. Così diciamo, quando per accendere il
  fuoco si batte nella pietra focaia, se ben non si batte il fuoco, ma la pietra.
  Vergilio nel 6, dell' En, dice.
  \begin{verse}
    \makebox[3em]{\dotfill} quaerit pars femina flammae
    Abstrusa in venis silicis \makebox[4.5em]{\dotfill}
  \end{verse}
\item[PAPPA] Pane bollito in acqua; è la vivanda solita darsi a i bambini quando
  s'allattano, e cominciano a balbettare, e si dice \textit{pappa} perché essendo la lettera,
  'P' puramente labiale, è facile a profferirsi come sono le lettere B, M. e
  però ne i bambini si trova maggiore attitudine a profferir queste, che l'altre
  consonanti, sì che più facilmente profferiscono \textit{babbo}, \textit{mamma}, \textit{pappa}, \textit{bombo}
  che \textit{padre}, \textit{madre}, \textit{minestra}, \textit{bere}, onde le balie si servano di queste parole per
  facilitare. la loquela a i bambini. Tal costume era forse anche negli antichi Romani,
  come si cava da Varrone, (nel libro intitolato Catone, Ovvero dell' allevare
  i figliuoli) che per \textit{Papas} intende quello, che intendiamo noi Toscani per
  \textit{Pappa} e da Persio, che nella Satira 3. disse
  \begin{verse}
    Et similis Regum pueris pappare minutum.
  \end{verse}

  I Greci pure  per i loro bambini si servivano come noi, e come i Latini, di
  voci di due sillabe con raddoppiarne la prima sillaba, per maggiore agevolezza
  del rilevare la parola. Di queste parole bambinesche ne troveremo molte nella
  presente Opera, usate dal Poeta per scherzo, o per accomodarsi alla qualità di
  colui che farà parlare, e non perché sieno in uso altrimenti. Vedi sotto in questo
  Cant, stan,12. dove dice d'un bambino che impara a parlare.

\item[BROCCATO] È una specie di drappo fatto a fiori, e s'intende Drappo tessuto con oro.

\item[A NOLO egli ha pigliato dall'Ebreo] Dice che il Sole ha pigliato a nolo il suo
  splendente abito, per significare che lo rende la sera, come lo restituiscono coloro,
  che pigliano gli abiti a nolo per un giorno; ed intendere che il Sole ascondendosi
  la sera alla nostra vista, lascia quell'abito risplendente, che s'era messo
  la mattina.

\item[CORTEO] Corteggio. Codazzo di donne, ec. che accompagnano una donna
  quando va a marito, o un bambino portato a Battesimo.

\item[UGNANESI genti] I soldati del Duca d'Ugnano; costume de i soldati d'appellar
  esercito dal nome del Generale, come Vaimaresi dal generale Vaimar\footnote{Si riferisce all'esercito del Duca di Weimar — forse per il suo ruolo nella fase iniziale della guerra Franco-Spagnola del 1635-1659.}, ec.

\item[COMPARIRE in scena] Venire in pubblico. Vedi sopra C. 1. stan. 2.

\item[LA materia che da il portante a' denti] La materia, che fa muovere i denti,
  cioè la roba da mangiare; si dice anche Da far ballare il mento. Vedi sotto in
  wuesto C, stan. 23. E \textit{portante} si dice una specie d'andare di cavalli. Il Lalli
  Tr. C, 3. stan. 58. dice.
  \begin{verse}
    Per dare il lor portante ai denti asciutti.
  \end{verse}

\item[LENA] Vedi sopra C. 1. stan. 2.

\item[LA masticavan male] L' intendevano male, la sopportavano mal volentieri.
  È solito quando si pensa a qualche cosa fissamente, e con applicazione il masticare,
  onde Persio delle composizioni ben pensate disse: \textit{Remorsum sapium unquem}:
  E tal \textit{masticare così} pensando si dice anche \textit{ruminare}, o \textit{digrumare}, che è quel
  masticare che fanno gli animali del pié fesso perciò detti \textit{ruminantia} da i Latini.
  Vedi sotto C. 6. stan. 5. Qui fa bell' effetto ' equivoco del verbo \textit{masticar male},
  che pare che voglia dire \textit{l'intendevano male}, e vuol poi dire che masticavano male,
  perché non mangiavano, non havendo che mangiare.
\end{description}

\section{Stanza VII. — IX.}

\begin{ottave}
\flagverse{7}E tra costoro un certo girellaio,\\
Che per l'asciutto va su i fuscellini,\\
Male in arnese, e indosso porta un saio\\
Che fu sin del Romito de Pulcini.\\
Ci è chi vuol dir ch'ei dorma n'un granaio\\
Per c'ha il mazzocchio pien di farfallini\\
È matto in somma, pur potrebbe ancora\\
Un dì guarirne, perché il mal dà in fuora.\\
\end{ottave}

\begin{ottave}
\flagverse{8}E, perch' ei non havea tutti i suoi mesi,\\
Fu il prima ad esclamare, e far marina\\
Forte gridando: Ohimè ch'io vado a Scesi\\
Pel mal che viene in bocca alla gallina,\\
Onde Eravano, e Don Andrea Fendesi\\
C'abbruciavano insieme una fascina;\\
E per cibare i lor ventri di struzzoli,\\
Cercavan per le tasche de' minuzzoli,
\end{ottave}

\begin{ottave}
\flagverse{9}Mentre di gagnolar già mai non resta\\
Colui ch'è senza numero ne rulli,\\
Anzi rinforza col gridare a testa, \\
Lasciano il fuoco, e i vani lor trastulli, \\
E per vedere il fin di questa festa\\
Se ne van discorrendo grulli grulli\\
Del bisogno ch'essi han ch'il vitto giunga\\
Perché sentono omai sonar la lunga.
\end{ottave}

Fra li suddetti soldati affamati l'Autore pone se medesimo descrivendo la sua
perfona, e genio; e dice che egli fu il primo a gridare per la fame, e per questo
Eravano, e Don Andrea Fendesi ancor essi affamati s'accostarono a lui per sentir
la cagione di quelle strida,

Nota che il Poeta divide il periodo nelle due ottave, ottava, e nona, di che è stato
da qualcheduno criticato d' errore, ma pero senza ragione, non adducendo
regola poetica, la a quale vieti il poterlo fare, come habbiamo detto altrove.

\begin{description}
\item[GIRELLAIO] Huomo stravagante. Huomo che gira, s'intende huomo inconsiderato, e
  che fa scioccaggini, e pazzie.

\item[ANDAR per l' ascivtto] Signi esser magro, e con poca carne addosso.
Vedi sopra Ca: stan. 68.

\item[VA in su i fuscellini] Ha gambe così sottili, che rassembrano due fuscelli; termine
  usatissimo da noi in questo proposito; che diciamo, Camminare su fuscelli.

\item[MAL in arnese] Mal vestito: Mal' all'ordine di sanità, d'abito, ec. Lalli
  En. tr, lib, 1. stan. 34.
  \begin{verse}
    Con sette navi Enea che gli avanzaro
    Qui si condusse assai male in arnese.
  \end{verse}
  Lodovico Dolce in lode dello sputo dice.
  \begin{verse}
    Eccomi qui per raccontarne cento,
    Ben ch' io non sia d' accordo col cervello,
    E malagiato in arnese mi sento.
  \end{verse}
  Il Persiani scrivendo al Serenissimo Principe D. Lorenzo dice.
  \begin{verse}
    Io, che sono in arnese tanto male,
    Mi ritrovo in grandissimo viluppo,
    Temo esser preso in vece d' un galuppo,
    E finir la mia vita allo Spedale.
  \end{verse}
  Franco Sacchetti Nov, 122. \textit{Il Saccardo era guarito, e stava bene in arnese}. Bocc.
  g.2.n.8. \textit{Partitosi assai povero, e mal' in arnese da colui, col quale lungamente era stato}.

\item[DEL Romito de' Pulcini\ Questo fu uno che abitava poco lontano da
  Malmantile, e teneva vita eremitica, vestendo di lendinella a foggia di Francescano
  scalzo; Da costui prese il nome di Romito quel luogo vicino a Malmantile
  che dicemmo sopra C. 1. stan. 70. E perché egli oltre al procacciarsi il vitto con
  chiedere elemosina s'aiutava ancora col nutrire nella sua abitazione buon numero
  di Polli per vender l'uova, fu nominato il \textit{Romito de Pulcini}. Quando l'Autore
  compose la presente Opera, detto Romito era morto di gran tempo prima,
  e però dice che il saio che egli haveva addosso fu fino del detto Romito, volendo
  inferire che era gran tempo, che quell'abito era fatto, ed in conseguenza oltre
  all'esser vile per essere stato d'un povero Romito, era ancora lacero, e consumato
  dal tempo.

\item[SAIO] Gonnelletto, o casacca, o simile parte d' abito da huomo; dal Latino
  \textit{Sagum}. Il Varchi stor. fior. lib 9, E sotto il Lucco chi porta un saio, chi una gabbanella,
  o altra vesticciola di panno chiamata casacca.

\item[DICONO ch' ei dorma in un granaio] L'Autore medesimo lo dichiara, seguitando
  perché \textit{ha il mazzocchio pien di farfallini}, se uno dorme, o si trattiene in
  un granaio, si suol'empiere di quei farfallini che stanno fra il grano; e quando
  diciamo: Il tale ha de' farfallini, o delle farfalle, intendiamo E' mezzo matto; e
  di cervello volante, o instabile. E per \textit{mazzocchio} intendiamo il capo, perché
  mazzocchio era una parte del Cappuccio, che già portavano i Fiorentini, secondo che
  dice il Varchi nelle sue storie Fiorentine lib. 9. \textit{Il Cappuccio} (dice egli)
  \textit{ha tre parti, cioè il mazzocchio, il quale e un cerchio di borra, che gira, e fascia intorno
    intorno alla testa, e di sopra, soppannato di nero di rovescio, copre tutto il capo}. Si
  dice oggi corrottamente \textit{mazzucco}, e così havea detto l'Autore, ma havendo il
  medesimo a dipingere uno dell'antico Magistrato di Firenze, mi domandò come
  era veramente l'abito Civile antico, ed io gli feci vedere questo luogo del Varchi,
  onde egli poi mutò, e disse mazzocchio per quanto vedo dal suo secondo
  originale, che e appresso di me.

\item[IL male dà in fuora] Quando il male da in fuora, cioè manda alla cute
  l'interna malignità, suol' essere indizio di salute; costui essendo infermo di
  pazzia, il dare in fuora di tale infermità è il far pazzie; e però il Poeta dice, che
  potrebbe guarirne, perché il mal da in fuora, cioè spera ch'ei guarisca, perché
  fa molte pazzie, che è lo sfogo del suo male, ed il suo dare in fuora.

\item[NON ha tutti i suoi mesi] È spropositato. Non ha l'intera perfezione del cervello.
  Non è stato tutti a nove i meli nel ventre di sua madre a perfezionare il
  cervello. In somma vuol dire Non ha giudizio; è scemo.

\item[FAR marina] Diciamo far marina coloro, che fingendosi stroppiati, ed impiagati
  gridano, e si rammaricano per farsi creder tali; che tanto vale in questo
  proposito \textit{Marinare}, o \textit{far Marina}, quanto rammaricarsi, o dolersi di cosa, che
  dispiaccia, ma per lo più s'intende di coloro, che fingono; come per esempio
  lo scolare battuto dal maestro, si dice far marina, quando fingendo che il maestro
  gli faccia gran male, piange, e stride a più non posso; che di dice anche fare il
  monello, Vedi sopra C, 3. itaa. 67.

\item[VADO a Scesi] Quando diciamo; Il tale è andato a Scesi, intendiamo è morto,
  se ben pare che diciamo è andato alla Citta di Scesi, o Assisi, perché il verbo
  scendere ci serve per intendere morire, Virg. \textit{facilis descensus}.

\item[PEL mal, che viene in bocca alla gallina] Il male che viene in bocca gallina
  da noi è detto \textit{pipita} dal Lat. \textit{pituita}, E perché fra da gente bassa in vece di
  dire \textit{appetito} si dice \textit{appipito}, pero cavano questo detto: \textit{Il tale ha il mal che viene
    in bocca alla gallina}, cioè \textit{la pipita}, e intendeno \textit{appipito}, cioè fame. E questo intende
  il Poeta nel presente luogo con questo detto piebeo.

\item[ERAVANO] Cioè Averano Seminetti. \textit{Don Andrea Fendesi}. Ferdinando
Mendes.

\item[FASCINA] Fascetto di legne; \textit{Ed abbraciare insieme una fascina}, vale star al fuoco
  a scaldarsi, e spender ciascuno la sua porzione nelle legne; E vuol dir anco copertamente
  andare all'osteria, Oraz. \textit{Ligna super foco large reponens}.

\item[STRVZZOLO] Vccello noto, il quale mangia così voracemente, che inghiottisce
  fino il ferro, Dicendosi \textit{ventre di struzzolo} s'intende Ventre insaziabile.
  Plin. degli struzzoli. \textit{Concoquendi sine delectu devoratu mira natura}.

\item[MINVZZOLI] Quei minuti fragmenti, che cascano dal pane, quando si
  spezza. E quest'atto di cercare i minuzzoli nelle tasche, esprime uno che habbia
  grandissima fame.

\item[GAGNOLARE] Voce corrotea da cagnolare, che è il guaire, che fanno i
  cagnolini quando hanno bisogno della poppa. Se per avventura non lo derivassimo
  dal verbo Latino \textit{gannire}, che signitica Rammaricarsi con parole non
  affatto intese mescolate con sospiri, e singulti, che è quelio, che nel presente
  luogo vuol dir gagnolare.

\item[È SENZA numero ne i rulli] È matto. Nel giuoco de rulli si pigliano sedici,
  o più, o meno rocchetti di legno, ciascuno de i quali ha il suo numero, eccetto
  che uno, il quale si chiama il Matto: E però dicendosi: \textit{il tale è il senza numero
    fra i rulli}, s'intende è il rocchetto, che è senza numero, cioè il matto. Questi
  rocchetti si chiamano \textit{rulli}, perché rizzati in terra in ordinanza col detto Matto
  nel mezzo, vi si tira dentro con un Zoccolo di legno grave tondo di figura piramidale,
  il quale si chiama rullo, e il giuoco si domanda \textit{A' Rulli}, ed alle volte
  \textit{a' rocchetti}; E chi più ne fa cadere con quel tiro vince. Si costuma anche tirare
  con una palla di legno.

\item[RINFORZA] Cioè Cresce lo stridere, o il guaire. L. \textit{ingeminat}. Si raddoppia.

\item[GRIDARE a testa] Gridar quanto più si può. Si dice anche \textit{gridare a corr'huomo},
  o quant'uno n'ha nella \textit{strozza}; nella \textit{canna}; o \textit{nella gola}. Vedi sopra C. 3. stan 6.

\item[TRASTVLLI] Trattenimenti. È voce da Fanciulli, e qui vuol esprimere,
  che fussero veramente trastulli da bambini, perché aggiunge l'epiteto vani, come
  era veramente il cercare de i minuzzoli nelle tasche.

\item[PER vedere il fine di quella festa] Per vedere in che haveva a terminare, o a
  che fine fusse fatto quel romore. Quando un discorso, o un suono, o un cantare,
  o altro romore comincia a venirci a fastidio diciamo: \textit{Quando finirà questa
  festa}; questa \textit{musica}; questo \textit{chiasso}; questo \textit{bordello}; questo \textit{baccano}; questo \textit{moscaio?}
  e simili. Vedi sotto C. 9. stan. 51. e C. 10. stan. 53.

\item[GRVLLO] Intendiamo uno melancolico, sbattuto da cattivi effetti, e non affatto
  sano, che si dice anche Acquacchiato; E tal voce è presa forse dalla Grue uccello
  (Sp. grulla) che quando sta fermo posa un sol piede, e tiene l'ale basse in maniera,
  che pare un pollo ammalato; che però tal pollo, ed ogni altro uccello
  così ammalato si dice \textit{grullo}, o \textit{che porta i frasconi}, Vedi sotto C.10, stan. 20.

\item[SENTONO suonar la lunga] Quando il Prete per invitare i popoli alla Messa
  suona la campana, e dura lungo tempo, in contado dicono \textit{suonar la lunga}. E
  da questo durate lungo tempo dicendosi: il tale sente suonar la lunga, s'intende
  ha fame per esser lungo tempo, che non ha mangiato. E per significar più
  copertamente diciamo: Egli ha quella del Carmine, s'intende la lunga, perché nella
  Chiesa del Carmine di Firenze, avanti si dica la prima messa, suonano una campana
  per un grande spazio di tempo, e questo suonamento si dice da tutti \textit{la lunga
  del Carmine}.
\end{description}

\section{Stanza X. --- XII.}

\begin{ottave}
\flagverse{10}Così domandan chi sia quei ch'esclama,\\
E mette grida, ed urli sì bestiali !\\
Gli è detto; Questo è un tale, che si chiama\\
Perlone dipintor de' miei stivali,\\
Un huom c'al mondo s'acquista gran fama\\
Nel far de' ceffautti pe' boccali,\\
E con gl'industri, e dotti suoi pennelli\\
Suo nome eterno fa negli sgabelli.
\end{ottave}

\begin{ottave}
\flagverse{11}Si trova in basso stato, anzi meschino,\\
Ma ben che il furbo ne manceggi pochi,\\
Giuocherebbe in su pettini da lino,\\
Che un'ora non può viver ch'ei non giuochi.\\
Ma s'ei vincesse un dì pur'un quattrino\\
In vero si potrebbon fare 'e fuochi,\\
Perché giocando sempre giorno, e notte,\\
Farebbe a perder con le tasche rotte.
\end{ottave}

\begin{ottave}
\flagverse{12}Giuocossi un suo fratel già la sua parte;\\
Suo padre fu del giuoco anch'egli amico,\\
Però natura qui n'incaca l'arte\\
Havendo hereditato un genio antico.\\
Costui teneva in man prima le carte,\\
Che legato gli fusse anco il bellico,\\
E pria che mamma, babbo, pappa, e poppe\\
Chiamò spade, baston, danari, e coppe.
\end{ottave}

Costoro intesero, che colui, il quale così gridava era Perlone, cioè Perlone
Zipoli, che vuol dire Lorenzo Lippi Autore della presente Opera; e fa che venga
descritto per uno sfortunato, ed ostinato giocatore.

\begin{description}
\item[METTE strida, ed urli bestiali] Stride, ed urla gagliardamente. Dice \textit{bestiali},
perché lo stridere è proprio del porco ferito, ed \textit{urlare} è proprio della volpe,
cane, e lupo; se ben ce ne serviamo anche per l'huomo in questi casi.

\item[DIPINTORE de' miei stivali] Pittore dappoco. È termine comune per coloro,
  che sanno poco in qualsivoglia scienza, o arte. Vedi sotto C. 6. stan. 106.
  E \textit{stivale} diciamo un huomo goffo, e di poco giudizio. \textit{Stivali} diciamo quella
  scarpa, che cuopre tutta la gamba, e s'usa per cavalcare. Ma di i pittori dappoco
  si dice \textit{Pittor da sgabelli, da boccali, da colombaie, ec.} come si vede nella presente
  ottava, che dice: \textit{Fa de' ceffautti ne i boccali}, \textit{E con gl'industri suoi pennelli,
  eterna il suo nome negli sgabelli}. Ma perché questa sua modestia, ed humilità non
  sia di pregiudizio al merito di così gran valent' huomo, replico, che egli fu Pittore
  riputatissimo, come le belle opere sue chiaramente testificano, e come mostrerà
  il sig.\ Filippo Baldinucci, se mandera alle stampe la sua Genealogia de'
  Pittori, Opera degna d'ammirazione si per le belle notizie, che si hanno in essa,
  e si ancora per sapersi, che questo erudito huomo l'ha ritrovate, e messe insieme
  in brevissimo tempo rubato alli tanti riguardevoli affari, che per pubblico
  benefizio lo tengono continovamente occupato.

\item[CEFFAUTTI] Voce composta delle note Musicali \textit{Ce fa, ut}, e non ha significato
  veruno, se non che mostrandosi di dire la chiave del \textit{Ci sol fa ut}, s'esprime
  \textit{Ceffo}, che si piglia per viso, o faccia, se bene appresso di noi \textit{ceffo} vale per muso
  di cane, o grifo di porco, E quantunque venga forse dal Greco \textit{Cephali}, che vuol
  dir Capo, onde anche i Latini, chiamano \textit{Cephalea} un certo dolor di testa, e che
  in Franz.\ \textit{chef} sia \textit{capo}; nondimeno noi non ce ne serviamo se non per ischerzo, e
  per intendere \textit{una faccia brutta, e fatta male}; e però l'Autore, volendo che s'intenda,
  che Perlone dipigne male, chiama \textit{ceffi} quelle facce, che egli dipigne, che
  per altro parlando pittorescamente chiamerebbe Teste.

\item[BOCCALE] È una misura fatta di terra cotta invetriata capace della metà
  d'un fiasco Fiorentino, ma intendiamo ogni sorta di vaso sia più piccolo, o più
  grande, che sia però di questa materia, e figura. E perché questi boccali da Vasellai,
  che gli fabbricano in Montelupo sono dipinti malissimo, e senza un minimo
  disegno, però a uno, che dipinga male si dice \textit{Pittor da Boccali}, o \textit{Pittore da
    Montelupo}.

\item[BASSO stato, anzi meschino] Povero mendico; Poverissimo.

\item[FVRBO] Propriamente ladro dal latino. \textit{fur}, ed è parola ingiuriosissima
  tutavia si piglia per astuto, sagace, scaltrito, e che sa il conto suo: Qui vuol dir vizioso,
  perché ha il vizio del giuoco, \textit{Fur a furuo i[dest] nigro dictus}, Papias\footnote{Papìa il Lombardo, autore del primo dizionario moderno: \textit{Elementarium doctrinae rudimentum} 1040-1060 circa. Papias è greco bizantino, \textit{Precettore}.}.

\item[NE maneggi pochi] Intendi: maneggi pochi danari. Non gli venga alle mani
  gran quantità di danari.

\item[GIOCHEREBBE su i pettini da lino] Intendiamo uno, che giocherebbe con
  ogni maggiore scomodo, come farebbe, s'egli stesse a sedere in su i pettini da lino,
  che son composti d'acutissime punte di ferro.

\item[SI potrebbon fare i fuochi] Si potrebbono fare i fuochi in segno d'allegrezza,
  come d'una cosa insolita. Detto usatissimo, quando succede qualcosa di nostro
  gusto, che siamo stati buon pezzo aspettandola; Che si dice anche \textit{Suonare a
  doppio}, Vedi sotto C, 6, stan. 107.

\item[FAREBBE a perder con le tasche rotte] Perderebbe sempre: Farebbe a gara a
  chi perde più con le tasche rotte, quantunque queste perdano tutti li danari, che
  in esse si mettono.

\item[INCACARE] Disprezzare: La natura non sa grado, e non ha obbligo \textit{all'arte},
  non essendo stato opera dell'arte, che egli giuochi, ma effetto della natura,
  che l'ha prodotto con questo vizio di giuocare. Dan. Pur. C. 10. disse:
  \begin{verse}
    Ma la natura gli haverebbe a scorno.
  \end{verse}

\item[VN genio] Vedi sopra C, 1, stan. 31.

\item[PRIMA che gli fusse legato il bellico] Subito ch'egli usci del ventre della madre.
  \textit{Bellico}. Diciamo quella parte del corpo, d'onde è preso il nostro primo alimento
  nel ventre della madre; la qual parte nel venire al mondo è legata dalle nutrici.
  E ciò serva per dichiarazione del presente detto.

\item[BABBO, Mamma, Pappa, e Poppe] Sono delle prime parole, che si profferiscono
  dai bambini, come s'è detto sopra in questo C. stan. 5. Ma questo Perlone
  prima \textit{spade, baston, denari, e coppe}, che sono li quattro segni differenti
  figurati nelle carte da giuocare, che si appellano semi, come vedremo sotto C. 8.
  stan. 6. E qui gliele fa dire per mostrare, che prima d'ogni altra cosa questo
  Perlone chiamò il giuoco, e che venne fuora con cotesto genio naturale di giuocare.
\end{description}

\section{Stanza XIII. --- XV.}

\begin{ottave}
\flagverse{13}Ma perché voi sappiate il personaggio, \\
Che ciò racconta, è il Franco Vicerosa,\\
Cavaliero, del qual non è il più Saggio; \\
Scrittor subblime in verso, quanto in prosa; \\
Dipinge, ne può farsi da vantaggio \\
Generalmente in qualsivoglia cosa: \\
Vince nel Canto i musici più rari, \\
E nel portare occhiali non ha pari.
\end{ottave}

\begin{ottave}
\flagverse{14}È suo amico, ed è pur seco adesso\\
Salvo Rosata un huom della sua tacca,\\
Però che anch'ei s'abbevera in Permesso,\\
E Pittor passa chiunque tele imbiacca;\\
Tratta d'ogni scienza, ut ex professo,\\
E in palco fa sì ben Coviel Patacca,\\
Che sempre ch'ei si muove, o ch'ei favella\\
Fa proprio sgangherarti le mascella.
\end{ottave}

\begin{ottave}
\flagverse{15}Hor perché Franco, ed egli ogni maniera \\
Proccuran sempre di piacere altrui, \\
Di Pertone dan conto, e, dov'egli era, \\
Di conserva n'andar con gli altri dui, \\
Là dove minchionando un po la fiera\\
Il Franco disse lor: Questo è colui\\
Ch'in zucca non ha punto, anzi ragionasi\\
D'appiccargli alla testa un'appigionasi.
\end{ottave}

Acciò che si sappia chi è colui, che da tal notizia di Perlone, dice; che egli
haveva nome \textit{Franco Vicerosa}, cioè Francesco Rovai Cavaliere dotto, Poeta,
Musico, Pittore, e veramente dotato di quelle buone qualità, e virtù, che dice
il Poeta, e che stanno benissimo in suo pari, come testificano alcune poche sue
Poesie stampate dopo la di lui morte, che non sono anche le migliori, che egli facesse
Dice \textit{che nel portare occhiali non ha pari}, perché haveva naso aquilino assai grande.
Con esso è \textit{Salvo Rosata}, cioè Salvador Rosa huomo anch'egli dotto, e Pittore
eccellente, il cui valore e notissimo, mostrandolo a bastanza le di lui stimatissime
Opere; e quanto valesse nella Poesia si conoscerebbe da alcune Satire da lui fatte,
le quali si spera vedere una volta alla stampa. Questo era amicissimo dell'Autore,
e fu causa, che egli tirasse avanti la presente Opera, persuadendoli, che
era per godere l'aggradimento universale, e gli dette anche notizia de lo Cunto
degli Cunti pubblicato in quei tempi.  Saluator Rosa recitava da Napoletano
in commedia mirabilmente, e si faceva chiamare Coviello Patacca. Questo
Franco Vicerosa, e Salvo Rosata insegnarono dunque ad Eravano, ed al Fendesi
chi, e dove era Perlone.

\begin{description}
\item[AVOMO della sua tacca] Huomo simile a lui. Uniformi di genio. \textit{Questa Tuca}
  detta anche taglia è un pezzo di legnetto fesso in due parti per lo lungo, il
  quale serve per libro di conti a coloro, che non sanno leggere, in questa forma
  Uniscono dette due parti di legnetto, e nella parte più spianata fanno alcune
  tacche, o segni col coltello, i quali segni denotano il numero delle cose prese a
  credenza, o dei danari, che si devono, o de i lavori fatti, ec. Ed un pezzo
  esso legno rimane appresso al creditore, e l'altro appresso al debitore: e quando
  si voglion dar nuovi danari, o segnare nuovi lavori, s'uniscono detti legnetti, e
  vi si fanno i segni che occorrono; E volendo aggiustare i conti si numerano i segni,
  e si vede la quantità del debito, o credito: ne vi può nascere inganno, perché
  se in una delle dette parti di legnetto fara fatto un segno di più, non si può
  far nell'altra, perché non riscontrerà, se il debitore, e creditore non si concedono
  scambievolmente detti pezzetti. Era in uso questa maniera di tener conti
  anco appresso ai Latini, che tal legnetto, che noi appelliamo \textit{Taglia}, o \textit{tacca},
  la dicevano tessera: \textit{Suam uterque tesseram habet; ratio constat}. Havevano ancora
  un'altra \textit{taglia}, che chiamavano \textit{Tessera hospitalis}, la quale serviva per riconoscere
  gli amici, e corrispondenti di diversi paesi, serbando ciascuno il pezzo del
  legnetto; il quale si lasciava anche a gli Eredi; E quando andava uno nel paese
  dell'altro portava la parte del legnetto; e unendolo si dava a conoscere per ospite;
  e però detti legnetti erano custoditi diligentemente. Questo pure si cava da
  Plauto in Pen. \textit{Ego sum ipsus, quem tu quaeris. P, hem quid ego audio? Antidamae
    gnatum esse. P. Si ita est, Tesseram me conferre hospitalem Si vis eccam
    attuli}, Donde havevano poi, \textit{Tesseram frangere hospitalem}, che significa \textit{Violare Ius
  hospitii}. Dal che si cava, che \textit{homo eiusdem tesserae}, sia lo stesso, che huomo della
  medesima taglia, che significa delli stessi genj, e corrispondente. Di quo habbbiamo
  il verbo \textit{attaccare}, che vuol dire Unire due materiali insieme, Ed il verbo
  \textit{attagliare}, che vuol dire Esser uniti di genio. Ricord, Mal. Stor. Fior. cap. 87.
  dice: \textit{Lucca, Pistoia, e Volterra feciono taglia co' Fiorentini},  e s'intende, si collegarono,
  e fecero lega; E si trova ne gli antichi nostri Storici spesso Taglia per
  lega.

\item[PASSA chiungue tele imbiacca] Supera ogni Pittore.

\item[FA sgangherar le mascella] Fa ridere sregolatamente, che è, quel Risu quatere\footnote{Risu quatere aliquem}
  che dicemmo sopra C. 3. stan. 66. alla voce Pimmei. E veramente questo Rosa
  ne gli anni suoi più giovenili, che dimorò in Firenze recitava  (come habbiamo
  detto) questa parte di Napoletano così bene, che si può dire, che egli sia stato il
  Maestro in far questo Personaggio.

\item[ANDAR di conserva] Andare insieme. Detto Marinaresco, che ha questo significato.

\item[MINCHIONANDO la fiera] E' il latino \textit{derideo}, E tanto vale il verbo minchionare,
  che Co\ellipsis{24pt}\footnote{``Coglionare''?} Che non si dice per essere sporco, ed usato da genti vili.
  Quell'aggiunta di \textit{fiera} è solita mettervisi, ma non so già a qual fine, perché
  tanto suona il solo verbo \textit{minchionare}, se non che potrebbe dirsi \textit{Minchionare la fiera}
  esser detto da coloro, che non avendo voglia di comprare passeggiano per
  le fiere domandando del prezzo di questa, o di quella cosa, e non offerendo niente,
  o pochissimo; e stanno a vedere, e osservare chi compra. E venuto poi a
  significare il \textit{Minchionare} assolutamente, e si dice ancora \textit{Minchionare la Mattea}.
  Vedi sotto C. 7. stan. 15. E pur qui ancora senza l'aggiunta di \textit{Mattea} suona
  \textit{burlare}.

\item[IN zucca non ha punto] cioè punto di sale, e s'intende: Non ha cervello in testa,
  Vedi sopra C, 1, Man. 53. Il Mauro in lode della Caccia dice:
  \begin{verse}
    Ed io, che sono un buom materiale,
    Tencando ciò ben mostrerei ch'io fusse
    Da dovero una Zucca senza sale.
  \end{verse}
  Catullo di Quinzia disse:
  \begin{verse}
    Nulla in tam magno est corpore mica salis.
  \end{verse}

\item[ATTACCARGLI alla testa un'appigionasi] Essendo la sua testa vota; per mostrare,
  che ella si può affittare si discorre d'appiccargli l'appigionasi, che così chiamo
  quella cartella, in cui sta scritto a lettere grandi APPIGIONASI, e s'appicca
  sopr'alle porte delle case disabitate, affin che si conosca, che quella è casa
  da affittarsi, o appigionarsi, appunto come dice, che era la testa di Perlone, che
  per esser vota di cervello, era in grado da potersi affittare, o appigionare. In
  alcuni luoghi d'Italia conservano l'uso antico, scrivendo in L. \textit{Est locanda}.
\end{description}
\section{Stanza XVI \& XVII}

\begin{ottave}
\flagverse{16}Spiacque il suo male ad ambi tanto tanto, \\
E mentre ei piange, che si getta via, \\
Il pietoso Eravan pianse al suo pianto \\
Verbigrazia per fargli compagnia; \\
Poi tutto lieto postosegli accanto \\
Per cavarlo di quella frenesia, \\
Di quelle strida, e pianto sì dirotto, \\
Che fa per nulla il bietolon mal cotto.
\end{ottave}

\begin{ottave}
\flagverse{17}Se forse dice; tu sei stato offeso,\\
Che fai tu della spada il mio piloto?\\
A che tenere al fianco questo peso\\
Per startene a man giunte come un boto?\\
S'al corpo alcun dolor t'havesse poi\\
Gli è qua chi vende l'olio dello Scoto;\\
Se t'hai bisogno d'oro io ti fo fede,\\
Che qualsivoglia Banca te lo crede.
\end{ottave}

A costoro dispiacque molto il male di Perlone, ed Eravano dopo haver compianta
questa sua disgrazia, si messe a consolarlo, è ad esaminarlo strettamente
per sapere la cagione di sì gran suo pianto.

\begin{description}

\item[BIETOLONE mal cotto] Huomo sciocco insipido, svenevole, appunto come è
  la bietola: Marzial. 13. \textit{Ut sapiunt fatuae fabrorum prandia beta}, voce \textit{Bietola},
  che viene dal Latino \textit{beta}, che vuol dire una specie d'erbaggio, tanto nel
  nostro idioma, quanto nel Greco, e nel Latino serve ancora per esprimere un'huomo
  sciocco, ed insipido. Laerzio nelle vita di Diogene Cinico dice così:
  \textit{Circumstantibus se adolescentibus est dicentibus: Caveamus ne mordeat nos: Bono inquit
  estote animo filioli, carnis enim betis non vescitur}. Plin. lib. 20. cap. 22. mostra, che
  i mariti volendo dire villania alle mogli dicevano loro \textit{bliteae}, raccogliendolo dalle
  commedie di Menandro; e si legge in quelle di Plauto, intendendo una cosa
  sciocca, e che non è buona a nulla; E come noi da \textit{bietola} caviamo il verbo \textit{sbietolare},
  che vuol dire Scioccamente piangere. (Vedi sotto C. 7. stan. 93.) e \textit{imbietolire},
  che vuol dire Commoversi, o effeminarsi. (Vedi sotto C. 9. stan. 57.) così
  gli antichi havevano \textit{betizare}, che ha lo stesso, o poco differente significato.

  \textit{Bietolone} dunque suona lo stesso, che Scimunito; ma con l'aggiunta di mal cotto
  vuol dire Scimunitissimo, perché la bietola cotta poco, dicono, sia più insipida
  della cruda.

\item[Piloto] Si chiama colui, che governa la Nave: dagli antichi Toscani detto
  \textit{Pedotto} forse dal L. \textit{pedes} preso per remi, come appresso Plauto \textit{navales pedes},
  o per funi da nave, come appresso altri. Ma questa voce Piloto ci serve per esprimere
  un'huomo da poco, poltrone, irressoluto, e flemmatico, ed in questo senso
  è preso nel presente luogo, Vien forse in tal caso dal Lat. \textit{plotus}, che vuol dir
  huomo, che per havere i piedi troppo piatti, e contraffatti cammina male. Vedi
  sotto C. 6, stan. 90.

\item[A CHE portare] A che fine portare? Che occorre che tu porti? \textit{Ad quid hoc
facis? Ad quid venisti?} nel Greco dice \textit{eph' hoo}, cioè per l'appunto \textit{A che?}

\item[STARSENE a man giuite come un boto] Boti chiamiamo quei Fantocci, o statue,
  che si mettono attorno all'Immagini miracolose per contrassegni di grazie
  ricevute, e però si dovrebbe dir \textit{Voti}, ma per iscambiamento di lettera si dice Boti.
  Berni in biafimo d' un' huomo brutto.
  \begin{verse}
    \makebox[3em]{\dotfill}Fugge da' ceraioli
    Acciò che non lo vendan per un boto.
  \end{verse}
  Che anticamente detti Fantocci si facevano di cera, e per lo più con le mani
  giunte in atto d'orare, e per questo dice \textit{starsene a man giunte come un boto}, che
  s'intende d'uno, che non sappia, o non voglia operare, e muover le mani per
  lavorare; e vuol'inferire, Che fai tu delle mani, e della spada, che tu non l'adoperi
  a vendicarti, se v'è stata fatta ingiuria? Mons. della Casa Galateo. Fo
  boto per modo di dirlo sempre.

\item[LO Scoto] intende di quel Ciarlatano, che vendeva Lattovarj, ed olj contro
  a veleni detto lo Scoto.

\item[TE lo crede] Scherza con l'equivoco, dicendo \textit{ogni banca te lo crede}, cioè ogni
  banca ti crede, che tu habbia bisogno dell'oro, e pare che voglia: Ogni
  banca ti fiderà, o presterà l'oro.
\end{description}

\section{Stanza XVIII. --- XXII.}

\begin{ottave}
\flagverse{18}Dopo Eravano poi nessun fu muto,\\
Ch'ognun gli volle fare il suo discorso\\
Offerendo di dargli ancora Aiuto,\\
Mentre dicesse quanto gli era occorso;\\
Ond'ei che havrebbe caro esser tenuto\\
D'haver più tosto col cervello scorso\\
Alzando il viso in loro gli occhi affisa,\\
E sospirando parla in questa guisa.
\end{ottave}

\begin{ottave}
\flagverse{19}Non v'è rimedio amici alla mia sorte; \\
Il tutto è vano, già che la sentenza \\
È stabilita in Ciel della mia morte, \\
Che vuol ch'io muoia, e muoia in mia presenza,\\
Già l'alma stivalata in su le porte\\
Omai dimostra s'esser di partenza.\\
Già con il corpo tutti i sentimenti\\
Le cirimanie fanno, e i complimenti
\end{ottave}

\begin{ottave}
\flagverse{20}Mutar devo mestier s'avvien ch'io muoia,\\
Il soldato cioè nel ciabattino,\\
Però che mi convien tirar le quoia\\
Per gir con esse a rincalzare il pino;\\
Un'altra cosa ancor mi dà gran noia,\\
Ed è che sotto son come un cammino,\\
E là dinnanzi a Minos, e agli altri Giudici\\
Rappresentar mi devo co pié sudici
\end{ottave}

\begin{ottave}
\flagverse{21}Ma ecco omai l'hora fatale è giunta, \\
Ch'io lasci il mio terrestre cordovano; \\
Già già la morte corre che par' unta \\
Verso di me con la gran falce in mano; \\
Spinge ella il ferro nel bel sen di punta, \\
Ond'io mancar mi sento a mano a mano: \\
Però lo spirto, e il corpo in un fardello \\
Tiro fuor della vita, e vo all'avello.
\end{ottave}

\begin{ottave}
\flagverse{22}Hormai di vita son uscito, e pure\\
Non trovo al mio penar quiete, o conforto,\\
O Cielo Mondo, o Giove, o creature\\
Dite, s'udiste mai così gran torto?\\
Se Morte è fin di tutte le sciagure,\\
Come allupar mi sento ancor che morto?\\
E come, dove ognuno esce di guai,\\
Mi s'aguzza il mulino più che mai?
\end{ottave}

Anche gli altri dopo Eravano gli offersero il loro aiuto, ed egli fingendosi pazzo
comincia a dire una mano di scioccherie, e mostrando di creder d'esser morto,
si maraviglia, che \textit{mors, qua omnia soluit} non gli habbia levato l'appetito
di cibarsi.

\begin{description}
\item[HAVERE scorso col cervello] Esser' impazzato. Haver dato la volta al cervello.
  Metafora tolta dall'orivolo a ruote, che si dice guasto, quando le ruote
  scorrendo escono del lor moto regolato.

\item[AFFISSAR gli occhi in uno] Guardare senza punto movere gli occhi; atto da
  pazzo di quella specie, che domandano Maniaci.

\item[ALLA mia sorte] Di quel che m'ha da succedere. Questa voce \textit{sorte} appresso
  di noi si piglia in diversi significati, come seguiva anche appresso a i Latini, da i
  quali si diceva \textit{fors} ogni avvenimento di Fortuna. Cic.lib.2. de Divinatione. \textit{Quid
    enim sors est? idem propemodum, quod micare, quod talos iacere, quod tesseras}, ed in
  questo senso è preso nel presente luogo. Si dice tirar le sorti, per intender quel
  \textit{super vestem meam miserunt fortes} dell'Evangelista.

  La pigliavano per carica, o incumbenza, secondo Livio: \textit{Si id gravaretur
    facere, quod non suae fortis id negotium esset}.

  La pigliavano per stirpe, secondo Ovid. 6. fast.
  \begin{verse}
    \backspace Si genus aspicitur, Saturnum prima parentem
    Feci, Saturni sors ego prima fui.
  \end{verse}

  La dicevano anche il capitale, e quello che noi pure diciamo sorte principale;
  Plaut. Most, \textit{Quatuor quadraginta illi debentur mina, Et sors, \& foenus DA, tantum est}.

  Altre volte pigliavano \textit{sors, pro iudicio} secondo Verg. 6. Aneid.
  \begin{verse}
    Nec vero hae sine sorte data, sine iudice sedes,
  \end{verse}

  Perché (secondo Servio) non s'udivano le cause \textit{nisi per sortem ordinate, nam,
  quo tempore causae agebantur, conveniebant omnes, \& ex sorte dierum ordinem accipiebant,
  quo post trigesimum diem causas suas exequerentur.}

  Dicevano sorte gli Oracoli, o risposte, o le polizze sopra alle quali si scrivevano
  le risposte. Val. lib. 1. \textit{Cuius rei exploranda gratia legati ad Delphicum oraculum,
  retulerunt: praecipi sortibus, ut aquam eius lacus emissam per agros diffunderent}. Virg.
  in questo senso disse: \textit{Lycie sortes}. Appresso noi ancora (come ho accennato)
  \textit{sorte} si piglia per fortuna, o destino, per condizione, stato, o essenza. E diciamo
  toccare in sorte, che significa ottenere la benefiziata, quando s'estraggono
  le polizze, che è quel \textit{mittere sortes}; e se bene in significato di fortuna vogliono
  alcuni, che si debba dire \textit{sorte}, ed in significato di qualità, o condizione \textit{sorta}
  hoggi (almeno nel parlar familiare, e Civile) non trovo, che s'usi tal distinzione,
  ma sento usare alcune volte l'una per l'altra indifferentemente.

\item[CIABATTINO] Uno che raccomoda scarpe rotte; da \textit{ciabatta}, che vuol dire
  Scarpa vecchia, e scarpa all'Appostolica, che sono quelle, che oggi usano i
  Cappuccini. In molti luoghi de' contorni Fiorentini chiamano Ciabattini ancora
  quelli, che fanno di nuovo; che noi chiamiamo Calzolai, in Ispagnuolo detti
  similmente \textit{zapateros}; e questo nome di \textit{Ciabatta} viene secondo alcuni da \textit{Clavata},
  cioè scarpa ferrata con chiodi; quali son quelle che usano i contadini, e i cacciatori.

\item[TIRAR le quoia] Havendo detto, che \textit{di soldato doveva diventar Ciabattino}, dà
  la ragione perché; ed è questa, che gli convien tirar le quoia, come fanno i Ciabattini,
  e i Calzolai, che tirano i quoi per condurgli a quella misura, che vogliono,
  delle quali quoia dice, che si dee servire per \textit{rincalzare il pino}, cioè far le
  scarpe al pino. Nota che lo scherzo dell'equivoco, nasce da \textit{tirar le quoia}, che
  vuol dir Morire, e \textit{rincalzar con esse il pino}, che vuol dire Farsi sotterrare a pié
  del pino, e così alzandogli la terra attorno rincalzarlo, che questo vuol dire rincalzare
  un'albero. Osserva ancora, che facendolo parlar da pazzo vuol, che
  coloro credano, che egli habbia concepito nel cervello questo sproposito d'haver
  a fare le scarpe a i pini; perché quando un Calzolaio dice; Io calzo il tale, s'intende
  \textit{Io gli fo le scarpe}. Plut. in Dem, \textit{E calzandosi dicea}, Il Gr, \textit{crepidas subligant}.

\item[SOTTO, son come un cammino] Sono schifo, ed ho le carni sudice, come è un
  cammino, dove si fa il fuoco, Comparazione usatifiima particolarmente dalle
  donne.

\item[MINOS, e gli altri Giudici] I Giudici dell' Inferno secondo le favole degli antichi
  Poeti, e della Gentilità sono tre, cioè Minos figliuolo di Giove, e di Europa,
  che fu Re di Candia, Eaco, che fu figliuolo di Giove, e d'Egina, che fu
  Re d'un Isola già detta Enopia, la quale egli poi dalla madre chiamo Egina, e
  Radamanto, che fu figliuolo di Giove, e d'Europa, che fu Re di Licia. Questi
  Re, perché furono severi amatori della giustizia, dicono i detti Poeti, che Plutone
  gli eleggesse per Giudici dell'Inferno, affinché esaminassero l'anime, ed
  assegnassero loro le pene, che meritavano, e da quello, che di loro scrive Verg.
  Aen. 6. si può comprender il lor preciso, e particolar ofizio, che di Minos dice:
  \begin{verse}
    \backspace Quaesitor Miinos urnam movet, ille silentum
    Conciliumque vocat, vitas, \& crimina discit.
  \end{verse}
  E di Radamento dice;
  \begin{verse}
    \backspace Gnosius haec Rhadamanthus habet durissima Regna,
    Castigatque, auditque dolos, subigitque fateri.
  \end{verse}
  D' Eaco parla Ovidio così;
  \begin{verse}
    \makebox[10em]{\dotfill} Tuasque
    AEacus in penas ingeniosus erit.
  \end{verse}

  E conchiude il Poeta, che uno di questi Giudici esamini, l'altro giudichi, il terzo
  mandi ad esecuzione. Se ben Dante nel 5. dell'Inferno dice:
  \begin{verse}
    Stavvi Minosse orribilmente, e ringhia,
    Esamina le colpe nell'entrata,
    Giudica, e manda secondo ch'avvinghia.
  \end{verse}

\item[CORDOVANO] Specie di quoio da fare scarpe, la concia del quale fu forse
  inventata in Cordova, e perciò tali quoi chiamansi propriamente cordovani, e son
  pelli di Castroni, o d'altri animali, ma qui intende pelle humana, e dicendo
  \textit{lasci il mio terrestre cordovano} intende io muoia, come intendon quelli, che dicono
  \textit{Terrestre salma, Terrena spoglia}, e simili, Cunto de li Cun. \textit{Pesto, e concio per cordonano}.

\item[CORRE che pare unta] Corre velocemente; comparazione dalle carrucole, o
  pulegge, o altre simili, le quali quando sono unte con olio, sapone, o altro,
  scorrono velocemente.

\item[FALCE] Strumento, col quale si sega il fieno; e col quale spesso si vede dipinta
  la morte con essa in mano.

\item[GUAI] Travagli, sventure, sciagure, afflizioni. Vedi sopra C. 1, stan. 28.

\item[ALLUPARE] Haver gran fame, perché dicono, che il lupo sempre habbia
  gran fame; quindi il volgo chiama Male della Lupa quello di coloro, che sempre
  mangerebbono, perché da loro vien prestissimo smaltito il cibo con pochissimo
  nutrimento, ed è quella infermità, che i Medici chiamano Fame canina.
  Vedi sotto C. 5. stan.\ 61. E da questo male chiamato della Lupa diciamo \textit{allupare}
  uno che habbia gran fame.

\item[AGUZZARE il mulino] Far venire, o crescere l'appetito: perché aguzzare
  la macine del mulino vuol dire Metterla in taglio in maniera, che si renda più
  ingorda. Vedi sotto C. 7. stan. 31.
\end{description}
\section{Stanza XXIII. --- XXXVIII.}

\begin{ottave}
\flagverse{23}Va a dir che qua si trovi pane, o vino\\
O altro da insegnar ballare al mento;\\
Se non si fa la cena di Salvino,\\
Quanto a mangiar non c'è assegnamento,\\
O ser Isac, o Abramo, o Iacodino,\\
Quando v'havete a ire al monumento,\\
Voi l'intendete, che nel cataletto\\
Con voi portate il pane, ed il fiaschetto,
\end{ottave}

\begin{ottave}
\flagverse{24}Orbè compagni? olà dal cimitero,\\
S'il Ciel danari, e sanità vi dia\\
Empiete il buzzo a un morto forestiero,\\
O insegnateli almeno un'osteria;\\
Se ben voi fate qui sempre di nero,\\
Perché di carne havete carestia:\\
È tale l'appetito che mi scanna\\
Ch'un Diavol cotto ancor mi parrà manna.
\end{ottave}

\begin{ottave}
\flagverse{25}Se ben non c'è da far cantare un cieco,\\
Di questa spada all'oste fo un presente,\\
C'ad ogni mo, da poi ch'ella sta meco,\\
Mai batté colpo, o volle far niente;\\
Per una zuppa dolla ancor di greco.\\
Ma che gracchio io? Qui nessun mi sente.\\
Che fo? s'i morti son di pietà privi\\
Meglio sarà ch'io torni a star tra i vivi.
\end{ottave}

\begin{ottave}
\flagverse{26}Qui tacque, e per fuggir la via si prese\\
Facendo sempre il Nanni, ed il corrivo,\\
Perch'egli è un di quei matti alla Sanese,\\
C'han sempre mescolato del cattivo;\\
Per haver campo a scorrer il paese\\
Ne fece poi di quelle con l'ulivo\\
Mostrando ogn'hor più dar nelle girelle,\\
E tutto fece per salvar la pelle.
\end{ottave}

\begin{ottave}
\flagverse{27}Perché uno, ch'il soldato a far s'è messo,\\
Mentre dal campo fugge, e si travia,\\
Sendo trovato, vien senza processo\\
Caldo caldo mandato in piccardia;\\
Però s'ei parte non vuol far lo stesso,\\
Ma che lo scusi, e salvi la pazzia,\\
Onde minchion minchion facendo il matto,\\
Se ne scantona, che non par suo fatto.
\end{ottave}

\begin{ottave}
\flagverse{28}Il Fendesi a scappare anch'ei fu lesto\\
Con gli altri tre correndo a rompicollo,\\
Volendo risicar prima un capresto,\\
E morir con la stomaco satollo,\\
Che restar quivi a menarsi l'a\ellipsis{24pt},\\
Ed allungare a quella foggia il collo.\\
Il danno certe è sempre da fuggire,\\
S'egli avvien peggio poi, non c'è che dire.
\end{ottave}

Perlone seguitando a dire spropositi per esser tenuto matto si parte, e per salvar
la vita continovò a fare delle sciocchezze, sapendo, che un soldato che scappa
dal campo, e si parte senza licenza è reo di morte, ed il Fendefi, e gli altri
scamparono anch' essi.

\begin{description}
\item[VA a dir che qua si trovi] È vanità il credere, o dire che qua si trovi; s'inganna
  chi crede che qua si trovi.

\item[INSEGNAR ballare al mento] Mangiare. E' lo stesso che Dar il portante a'
  denti detto sopra in questo C, stan. 6.

\item[FAR la cena di Salvino] Andare a letto senza cena; che la cena di Salvino era
  Pisciare, e andare a letto.

\item[O SER Isac, o Abramo, o Iacodino] Intende tutti gli Ebrei, e seguitando l'opinione
  del volgo, il quale crede, che quando gli Ebrei seppelliscono i loro morti
  mettano lore appresso del pane, e del vino dice: \textit{Voi l'intendete che morendo
    portate con voi il pane,e il vino}, poiché nel mondo di qua non si trova ne da
  mangiare, ne da bere.

\item[CATALETTO] Quella barella, entro alla quale si portano i morti al sepolcro,
  che i Latini dicevano \textit{feretrum}. Voce composta di \textit{Letto}, e \textit{Cata} preposiz. Gr.\footnote{dal greco $\kappa\alpha\tau\alpha$ cioè "giù, in basso, sotto"}

\item[ORBÈ, olà, alò] E simili; sono voci, e termini usati per farsi sentire da chi è
  alquanto lontano; come fa il Latiao \textit{heus}, Orbè, e fatto da Ora bene; Or beat
  Latino \textit{age vero}; \textit{Alò} dal Fr. \textit{allons}; andianne.

\item[CIMITERO] Piazza, nella quale si fanno i sepolcri per li morti, Voce che
  viene dal Greco \textit{coemasthae}, che suona dormire, riposarsi. Onde \textit{coemeterion}, è lo
  stesso, che Dormentorio. Quindi i Cretensi chiamavano Cimeterio una casa
  pubblica, la quale serviva per alloggiare i pellegrini. Vedi sotto C. 7. stan. 27.

\item[S'IL Ciel danari, e sanità vi dia] Dice questo sproposito per accrescere in
  coloro la credenza, che egli sia matto, sapendo bene che i morti non hanno bisogno
  di sanità, ne si curano di denari.

\item[BUZZO] Intendi il ventre dell'huomo, da busto che s'intende tutta quella
  parte del corpo humano, che è dal collo al pettignone, senza le braccia.

\item[FAR di nero] Mangiar di magro. I venerdì, sabati, Quaresima, ed altre vigilie
  si chiamano giorni neri, quasi giorni di lutto destinati alla penitenza, ed il
  Poeta scherzando con l'equivoco del nero, col quale è solito farsi l'apparato a'
  morti, par che voglia dire non mangiate mai carne, perché soggiunge \textit{di carne
  havete carestia}, e par che intenda non havete carne da mangiare, e vuol dire
  non havere carne in su l'ossa, perché i morti in breve tempo restano puri scheletri
  senza carne.

\item[APPETITO che mi scanna] Fame così grande, che mi fa morire, che mi fa
  perder la canna della gola; che scannare uno, vuol dir Tagliarli la canna della
  gola. Cunto de li Cunti Giorn. 1.\textit{Se la necessità non la scannava}.

\item[MI parrà manna] Mi parrà buonissima; come parve, e fu a gli Ebrei la Manna,
  che mandò loro Dio nel deserto, che ricevendola esclamavano \textit{Manu}, cioè
  Che è questo? onde sortì il nome.

\item[NON ho da far cantare un cieco] Non ho ne meno un quattrino da darlo a un
  cieco, perché canti un' Orazione.

\item[IN ogni mò] Per: a ogni modo. È termine assai usato in Firenze in diversi sensi,
  perché, o significa disprezzo, come nel presente luogo, \textit{Voglio dar via la spada,
  perché ad ogni modo non battè mai colpo}, cioè perché io non la stimo per non haver
  ella mai lavorato. O significa necessità di fare, o non fare una cosa per esempio,
  \textit{si può far quanto si vuole, che ad ogni modo s'ha da morire}. Significa contentarsi di
  quello, che uno ha conseguito: \textit{Io ho guadagnato poco, ma ad ogni modo io mi
    contento}. Significa Ostinazione. \textit{So che la tal cosa mi può nuocere, ma la voglio
    fare ad ogni modo}. Vedi sopra Can. 1. stan. 3. il termine; \textit{suo danno}, che par che
  habbia correlazione al termine, A ogni modo, V.gr. \textit{Se io ho perduta la tal
    cosa, suo danno; ad ogni modo io non me ne servivo}, E quel \textit{mo} per modo è
  la figura apocope da noi molto usata come vedremo altrove.

\item[MAI batté colpo] Diciamo: \textit{il tale non batté mai colpo} per intendere, il tale
  non lavora mai, e qui intende, che la spada di Perlone nelle sue mani non lavorò mai.

\item[ZUPPA] Pane intinto nel vino, o in altro liquore. Forse meglio \textit{Suppa}, Franco
  Sacc, Nov. 86. \textit{E fatta la suppa con le spezie, subito porta in tavola il ventre, e la
    suppa}. Stimo che venga dal Tedesco \textit{Suppen}, che vuol dir Brodo di carne, o d'altro,
  che si quoca lesso. In questo senso una sorta di minestra chiamiamo \textit{zuppa
    Lombarda}, Vedi sopra C. 2. stan. 7. Ma l'uso ha introdotto il dir corrottamente
  zuppa, e da molti inzuppa; come zolfa, e zezzo, e zinfonia in vece di solfa,
  sezzo, sinfonia, e simili.\footnote{Vedi anche ``berzaglio'', o la parola ``verzicola'', che in altri testi appare come ``versicola''. }

\item[GRACCHIARE] Discorrer senza proposito, o profitto. Da Graccio Latino
  \textit{gracculus}. Il tale mi chiese dieci scudi in presto, ma io lo lasciai gracchiare. Vedi
  sotto C, 7. stan. 59. e C. 8. stan. 65.

\item[FAR il nanni, ed il il corrivo] Fingersi corrivo, goffo, semplice, basèo.

\item[MATTI alla Sanese] Si dice \textit{Sanesi Matti}, ma in effetto son più sagaci degli altri,
  e però dice \textit{Matti alla Sanese, c'han sempre mescolato del cattivo}; cioè dell'astuto,
  del sagace, ed ingegnoso.

\item[NE fece di quelle con l'ulivo] Fece delle scioccherie grandissime. In alcune solennità,
  suole la generosa pietà del Sereniss. G. Duca liberare dalle carceri alcuni
  debitori con pagare il loro debito, o parte di esso; e questi tali vanno processionalmente
  a render grazie a Dio al Tempio della Santiss. Annonziata, o di S. Gio: Batista;
  e quelli che hanno pagato tutto il debito, e sono affatto liberi portano
  in mano un ramo d' olivo a distinzione di quelli, che per non haver pagato
  tutto il debito, ma parte di esso devono tornare in carcere, i quali non hanno
  l'olivo in mano, ma son legati. Da questo ramo d'ulivo, che in tal congiuntura
  denota pagamento intero, credo che sia nato il dettato; La tal cosa e con l'ulivo,
  che significa cosa grande nello stesso modo, che i Latini dissero \textit{palmaris}, ed
  esprime un'azione ardita, che diciamo anche \textit{marchiana}; \textit{da pigliar con le molle},
  ec, come s'intende qui, che vuol dire, che questo fece cose grandi, ed ardite.

\item[DAR nelle girelle] Impazzire. Vedi sopra C. 3. stan, 43., e sotto C.9. stan.10.

\item[MANDATO in Piccardia caldo caldo] Impiccato subito preso senza far processo:
  \textit{Caldo caldo} subito, e prima che la cosa si raffreddi. \textit{Piccardia},
  \textit{in ipso ardore criminis}, Provincia della Francia, serve, scherzando con la
  similitudine della parola, per intendere \textit{impiccare}. I Latini pure havevano
  un termine coperto per fare intendere impiccare, che era \textit{literam longam facere},
  come si vede in Plauto; il che ha data occasione a molti letterati di discorrere per
  chiarire qual fusse questa lettera  e Celio Rod. leet, Ant: lib. 10. cap. 8. conchiude,
  che fusse il T maiuscolo, che è simile alla forca, che facevano i Latini.
  Noi ancora diciamo: \textit{Andare a Lungone} che è un Porto in Toscana; \textit{Andar a Fuligno},
  cioè a \textit{fune}, e \textit{legno}; \textit{Dar de' calci al vento}: \textit{Ballar in campo azzurro} sopra
  C.2. stan. 65. \textit{Ballar nel Paretaio del Nemi}, sotto c. 6. stan. 50. E tutti significano
  Esser impiccato.

\item[MINCHIONE] Da minchia detto sopra in questo C. stan. 15.\footnote{Qui il Minucci sembra non voler insistere con le Minchiate, gioco delle carte descritto al C. 8. stan. 61., dove il \textit{Matto} offre proprio l'opportunità di scantonare \textit{che non par suo fatto}, proprio come di seguito.}

\item[SE ne scantona, che non par suo fatto] Se ne va via e non pare che faccia
  questo per andarsene, È forse quell'\textit{Agere se} di Ter. in Andr.

\item[CORRER a rompicollo] Correr velocemente; e a precipizio senza considerare
  la strada buona, o cattiva.

\item[ARRISCHIARE un capresto] Avventurare a essere impiccato. Corre più tosto
  il rischio d'andare in su le forche, che quello di morir di fame.

\item[MENARSI l'A\ellipsis{24pt}] Perder il tempo senza far nulla. Se vuoi intender bene
  questo detto, leggi discorso d'Anibal Caro in difesa di Ser' Agresto\footnote{Menarsi l'Agresto: \textit{Far cosa di poca riputazione, per non aver da far altro}. È possibile che il Minucci non volesse davvero indicare un ``discorso in difesa di Ser' Agresto'', pseudonimo di Annibal Caro, ma semplicemente cercasse l'occasione di menzionare la parola mancante, senza cadere nella volgarità.}.
\end{description}

\section{STANZA XXIX. --- XXXI.}
\begin{ottave}
\flagverse{29}Lasciam costoro, e vadan pure avanti \\
Cercando il vitto lì per quel contorno,\\
Che se fame gli caccia, e' son poi fanti\\
Da battersi ben ben seco in un forno: \\
Perché d'un gran guerrier convien ch'io canti\\
Mezzo impaniato, perch'egli ha d'intorno \\
Vna donna straniera in veste bruna,\\
Che s'affligge, e si duol della fortuna.
\end{ottave}

\begin{ottave}
\flagverse{30}Calagrillo è il guerriero, e via pian piano\\
Cavalcando ne va con festa, e gioia,\\
Ognor tenendo il chirarrino in mano,\\
Perché il viaggio non gli venga a noia,\\
È bravo sì, ma poi buon pastricciano,\\
E' farebbe servizio infino al Boia,\\
Venga chi vuol, a tutti dà orecchio,\\
Se bene fusse il Bratti Ferravecchio.
\end{ottave}

\begin{ottave}
\flagverse{31}Poiché bella è colei che si dispera \\
Sempre piangendo senz' alcun ritegno, \\
E vanne, come io dissi, in cioppa nera \\
Per dimostrar di sua mestizia il segno,\\
Perciò con viso arcigno, e brutta cera\\
Par un Ebreo c'habbia perduto il pegno,\\
E di quanto l'affligge, e la travaglia,\\
Calagrillo il Campion quivi ragguaglia.
\end{ottave}

Il Poeta lascia il discorso di quegli affamati, e si mette a narrare la favola travestita
di Psiche; la quale chiede aiuto a Calagrillo, che è Carlo Galli Capitano
di Cavalli, gli racconta i suoi travagli.

\begin{description}
\item[SON fanti] S'intende son huomini c'hanno cuore, e spirito da fare quella
  tal cosa; e da pigliare ogni risoluzione.

\item[DA battersi ben ben seco in un forno] Da combatter con la fame anche dentro a
  un forno pien di pane, e mangiandoselo, vincerla, e farla fuggire.

\item[MEZZO impaniato] Imbrogliato; Intrigato: Traslato da gli uccelli, che havendo
  toccata la pania\footnote{Pà-nia, s.f., Materia appiccicosa ricavata dalle bacche del vischio, usata per catturare piccoli uccelli. Sinonimo di vischio. ``Impaniato'' quindi sinonimo di ``invischiato''.}, volano sì, ma con difficultà per l'impedimento, che dà
  loro la pania, che hanno sopra alle penne.

\item[BVON papricciano] Huomo dolce, grossolano, huomo alla buona. \textit{Pastricciano}
  è specie di Pastinaca. Il detto antico e Buon pasticcione, cioè di buona pasta.
  \textit{Placidus tanquam aqua silens}.

\item[BATTI feravecchio] Molti vogliono, che si dica il Bratti ferravecchio, il quale
  fu un huomo facultoso, ma di cattiva fama: Costui lasciò poi tutto il suo havere
  a una Confraternita di secolari intitolata in S. Gioseppe, perché delle rendite
  se ne dessero tante elemosine, come segue fino al dì d'hoggi; ma a me pare,
  Che meglio sia dire il \textit{Batti}; perché il \textit{Batti}, cioè i \textit{Battilani}, quando noo possono
  più lavorare non sapendo far altra arte, si mettono a fare il rivendirore di
  cenci, e ferri vecchi, e dall'andar gridando per la Città \textit{Chi ha ferri vecchi}, hanno
  acquistato il nome di Ferravecchio. E perché queste sono vilissime persone, ed
  alle quali si ha poco riguardo;  quando vogliamo esprimere, che uno sia di mansueta,
  ed umil natura, e indifferente con tutti, sogliamo qualificarlo con questo
  termine. \textit{Saluta, o farebbe servizio, anche al Batti ferravecchio}. Che se dicesse il
  \textit{Bratti} calzerebbe tanto bene; perché finalmente il \textit{Bratti}, fu persona di qualche
  riguardo, e Civiltà. \textit{Imbratta} soprannome trovasi nel Bocc.\footnote{Guccio Imbratta, anche detto Guccio Balena, o Guccio Porco, fante di Frate Cipolla nella novella 10 della giornata sesta. Compare in chiusura della novella 7 della giornata quarta.}

\item[PSICHE] È nota la favola di \textit{Psiche}, descritta maravigliosamente da Apuleio,
  la quale il Poeta incastra in questa sua Opera, e l'immaschera assai aggiustatamente.

\item[VISO arcigno] Viso aspro, che denota dolore, o altra passione travagliosa. Lat.\ \textit{Torva facies}.

\item[BRUTTA cera] Haver brutta, o cattiva cera vuol dire Faccia, che dal suo
  cattivo colore indichi poca sanità, o grave disgusto, che travagliando l'animo,
  il corpo, E \textit{brutta cera} vuol dir ancora Fisonomia cattiva.

\item[PARE un'Ebreo c'habbia perduto il pegno] Quand'uno per qualche disgusto mostra
  faccia malinconica ci serviamo di questo detto, perché o sia vero, o sia nostra
  opinione, rarissimi sono gli Ebrei, che habbiano faccia allegra; ma un' Ebreo
  che habbia perduto il pegno aggiunge melanconia a malenconia, e però
  mostra faccia deformatissima.
\end{description}

\section{Stanza XXXII. --- XXXIIII.}

\begin{ottave}
\flagverse{32}Signore (incominciò) devi sapere,\\
Ch'io hebbi un bel marito, ma perch'io \\
Dissi chi egli era contro al suo volere,\\
Già per sett' anni n' ho pagato il fio; \\
Perché egli allor per farmela vedere \\
Stizzato meco sen' andò con Dio \\
In luogo; che a volerlo ritrovare \\
La carca ci volea da navigare.
\end{ottave}

\begin{ottave}
\flagverse{33}E quando poi io l'ho bell', e trovato,\\
Martinazza, ch'è sempre lo Scompiglia,\\
Fa sì che pur di nuovo m'è scappato,\\
Ed in mia vece all'amor suo s'appiglia,\\
Tal ch'io rimango cacciator sgraziato;\\
Scuopro la lepre, e un'altro poi la piglia.\\
Ti dico questo; perché havrei voluto\\
Che tu mi dessi a raccatrarlo aiuto.
\end{ottave}

\begin{ottave}
\flagverse{34}El le promette, e giura, ch'il marito \\
Le renderà, però non si sgomenti,\\
E se non basterà quel che ha smarrito, \\
Quattro, e sei bisognando, e dieci,e venti.\\
Ed ella lo ringrazia, e del seguito\\
Di tante sue fatiche, e patimenti\\
(Fatta più lieta per le sue promesse)\\
Così da capo a raccontar si messe.
\end{ottave}

Psiche espone a Calagrillo il suo bisogno, e lo richiede d'aiuto; Ei glielo promette,
ed ella fatta allegra per tal promessa, incominciò a discorrere, narrando
tutte le fatiche, e disagi patiti da lei in ricercare del Marito.

\begin{description}
\item[N'HO pagato il fio] N'ho pagato la pena; è il Lat. \textit{poenas dare}. Fio è voce
  Fiorentina antica, che vuol dir \textit{feudo}. Gio. Villani lib. 5. cap. 1. \textit{Scomunicò
  Federigo, ed assolvette tutti li suoi Baroni da fio, e saramento}, ec, ma da noi hoggi non
  usata se non nel senso suddetto; nel quale anche l'usò Dante Purg, C, 10.
  \begin{verse}
    Di tal superbia qui si paga il fio.
  \end{verse}

\item[CI voleva la carta da Navicare] Era impossibil ritrovar quel luogo senz'haver
  la carta da navicare, o la bussola.

\item[L'HO bell'e trovato] L'ho già trovato. Vedi sopra C. 3. stan. 14. la forza di
  questo addiettivo \textit{bello} in questi termini.

\item[M' HA scartato] M' ha rifiutato. Traslato dal giuoco delle carte, che quando
  una carta, che habbiamo in mano non fa per noi, la buttiamo sopr'al monte
  delle carte; il che si dice scartare, vedi sotto C. 8. stan. 6. alla voce Minchiate.

\item[A RACCATTARLO] Cioè ritrovarlo, riaverlo, ricuperarlo. Il proprio
  significato di raccattare è Ragunare, mettere insieme. Vedi sotto C. 10. stan. 37.

\item[NON si sgomenti] Non si perda d' animo, non si sbigottisca. Petr. 42. 4.
  \begin{verse}
    E fol della memoria mi sgomento.
  \end{verse}
  Dante nel Purg. C. 14. in significato attivo. '
  \begin{verse}
    Cacciator di quei lupi in su la riva
    Del fiero fiume, e tutti gli sgomenta.
  \end{verse}

\item[SMARRIRE] È un certo perdere con speranza di ritrovare. Dan. Inf, C, 1.
  \begin{verse}
    Che la diritta via era smarrita
  \end{verse}

\item[QUATTRO, sei, e dieci, e venti] Scherza facendo, che Calagrillo prometta
  più di quel ch'è richiesto, come fanno tutti i bravazaoni, e in tanto mostra, che
  a una bella donna non mancano mariti.
\end{description}

\section{Stanza XXXV --- XXXIX}

\begin{ottave}
\flagverse{35}Cupido è la mia cara compagnia,\\
Ricco garzon, se ben la carne ha ignuda,\\
Anzi non è, t' ho detto una bugia,\\
Perch'ei non mi vol più cotta, ne cruda,\\
Ma senti pure, e nota in cortesia:\\
Quando la madre sua ch'era la Druda\\
Del Fiero Marte, idest la Dea d'amore\\
Gravida fu di questo traditore;
\end{ottave}

\begin{ottave}
\flagverse{36}Perch' una trippa havea, che conveniva,\\
Che dale cigne homai le fusse retta,\\
Cagion ch'in Cipro mai di casa usciva,\\
Se non con i braccieri,  ed in Seggetta,\\
Pur sempre con gran gente, e comitiva,\\
Com' a Regina; com' ell'è, s'aspetta,\\
I paggi ha dietro, e gli staffier dinanzi,\\
E dagl'inlati due filar di Lanzi.
\end{ottave}

\begin{ottave}
\flagverse{37}Essendo così fuori una mattina \\
Per suoi negozzi, e pubbliche faccende,\\
Urtò per caso una Vacca Trentina,\\
E tocca a pena, in terra la distende;\\
Ond' ella dopo un'alta rammanzina,\\
Perch'una lingua ell'ha che taglia, e fende:\\
Va, che tu faccia, quando ne sia otta\\
Un figliuol (dice) in forma a una botta
\end{ottave}

\begin{ottave}
\flagverse{38}E così fu ch'in vece d'un bel figlio\\
Di suo gusto, e di tutti i Terrazzani\\
Un rospo fece come un pan di miglio,\\
C'havrebbe fatto stomacare i cani;\\
Che poi cresciuto, fecesi consiglio\\
Di dargli un po di moglie, ma i mezzani\\
Non trovaron mai donna, ne fanciulla,\\
Che saper ne volesse, o sentir nulla.
\end{ottave}

\begin{ottave}
\flagverse{39}Se non ch'i miei maggiori finalmente\\
Mio padre ch'il bisogno ne lo scanna,\\
Con un mio Zio ch' andava pezziente,\\
E un mio fratello anch'ei povero in canna,\\
Sperando tutti tre d' ungere il dente,\\
E dire: O corpo mio fatti capanna,\\
E riparare ad ogni lor disastro,\\
Me gli offeriro; e fecesi l'impiastro.
\end{ottave}

Racconta Psiche a Calagrillo la dolorosa storia, e facendosi dalla nascita di
Cupido dice, che nacque in forma di rospo per la maladizione d' una vecchia, e
che poi cresciuto fu a lei dato per marito.

\begin{description}
\item[NON mi vuol cotta, ne cruda] Ne a lesso, ne a rosto. Non mi vuol più in
  maniera nessuna. Il Lalli En. Tr. lib. 2. stan. 42. dice:
  \begin{verse}
    Non gli volle annasar crudi, ne cotti
  \end{verse}
\item[DRUDA] Innamorata, tanto in bene quanto in male; perché si dice amante,
  innamorato, damo, non sempre in significato disonesto.\\
  Dan. Par. C. 12. \textit{Dentro vi nacque l'amoroso Drudo}\\
  \makebox[30pt]{}\textit{Della fede Cristiana il S. Atleta.} Parla di S. Domenico.

  Se bene nel presente luogo s'intende Meretrice, concubina.

\item[CIGNE] Sono striscie di quoio, o d'altra materia adattate a sostenere, e
  tenere insieme qualsivoglia cosa, dette cigne, da cignere.

\item[BRACCIERI] Coloro, sopr' alle braccia de' quali con una mano s'appoggiano
  le Dame andando a piedi per la Città.

\item[DAGL'inlati] Dalle bande, da i lati. Idiotismo usato assai \textit{in lati} per lati.

\item[LANZI] Così chiamiamo i soldati Tedeschi della guardia pedestre del Seren.
  G. Duca. Vedi sopra C. 1, stan. 52.

\item[VACCA Trentina] Così chiamiamo certe donnicciuole poco honeste, sfacciate,
  ed ardite, che non portano rispetto a veruno; e credo che si dica così per
  la similitudine, che hanno con le Vacche di Trento, le quali per esser' avvezze a
  sempre per le campagne del Tirolo, sono falvatiche, e feroci.

\item[RAMMANZINA] È lo stesso, che rammanzo detto sopra C.1.st. 52., e che
  rabbuffo nel med. C. st 39. Da alcuno è definita così: Riprensione fatta con parole
  minaccevoli, e ingiuriose. Forse dalle dicerie de' romanzi.

\item[HA una lingua che taglia, e fende] Ha una cattiva lingua, che dice ogni sorta
  male senza rispetto, o riguardo alcuno, che \textit{lacera l'altrui riputazione}.

\item[HAVREBBE fatto stomacare i cani] Così sporco, e nefando, che havrebbe
  provocato il vomito fino a i cani per la sua schifezza. In questo senso i Latini
  pure si servivano del verbo \textit{stomachari}.

\item[DARGLI un po di moglie] La voce \textit{poco} è usata da noi in diverse maniere; o
  declinabile, che significa quantità, come \textit{dategli un poco di carne}; o indeclinabile
  per avverbio; come andare un poco a Roma; Dategli un po di moglie, e serve per
  emfasi al discorso, e non per quantità, potendosi dire \textit{andate a Roma}: \textit{Dategli
    moglie}, che tanto esprime senza la voce \textit{poco}, la quale però nel presente luogo non
  è ripienezza, o (come diciamo) borra; ma è così detto per mostrarne l'uso,
  che appresso di noi e frequentissimo, ma nel caso come il presente è tanto usato,
  che non pare si possa dire altrimenti. Quel po per poco è la figura apocope
  usatissima da noi in questa, ed in altre voci enunciate sopra C. 1. stan. 36.

\item[MEZZANO] Sensali. Coloro che sono mediatori a conchiudere ogni sorta
  d'affare.

\item[IL bisogno ne lo scanna] È poverissimo; muore di necessità; la voce \textit{scannare}
  s'usa da noi per esprimere un soverchio desiderio di qualsivoglia cosa, se bene il
  suo più proprio è della fame, come s'è veduto sopra in questo C. stan. 24.

\item[PEZZIENTE] Povero, che chiede limosina. Deriva dal Latino \textit{petere} onde,
  povero pezziente vuol dir \textit{pauper petens eleemosinam}; ed è lo stesso che \textit{povero
    in canna}, quasi ignudo come una canna; altri vogliono, che quello \textit{incanna} sia una sola
  parola, e voglia dire \textit{incannatore}: Che quando un' huomo si mette a incannare,
  è segno, che è miserabile, perché il guadagno dell'incannare è infelicissimo.
  Il Varchi Stor. Fior. lib. 12. \textit{Perderono tutto quello, che in molt'anni havevano raggruzzolato,
    e diventarono poveri in canna}. Franco Sacc. Nov. 181. \textit{Voi altri Astrologi,
    per guardar sempre il Cielo, perdete la Terra, e siete sempre poveri in canna}.

\item[UNGER il dente] Mangiar roba, che unga il dente come carne, ec. e non
  sempre pane, come son necessitati fare i mendichi; e vuol dire Far miglior vita,
  mangiar un po meglio.

\item[DIRE al corpo: fatti capanna] Haver tanto da mangiare, che gli convenga
  pregare il Cielo, ia diventare il suo corpo capace quanto una stanza da
  riporre il fieno (che questo vuol dir Capanna) per haver luogo dove riporre
  tanta roba. Usiam questo termine quando veggiamo uno avvezzo a vivere miseramente,
  e che si trovi poi a un banchetto lautissimo.

\item[SI fece l'impiastro] Cioè s'accordo, si conchiuse il negozio.
\end{description}

\section{Stanza XXXX \& XXXXI}
\begin{ottave}
\flagverse{40}Fu volentier la scritta stabilita, \\
Io dico sol da lor, che fan pensiero \\
Di non havere a dimenar le dita,\\
Ma ben di diventar lupo cerviero;\\
E, perché e' son bugiardi per la vita,\\
Dimostrano a me poi il bianco pel nero\\
Dicendomi, che m'hanno fatta sposa\\
D'un giovanetto, ch'è sì bella cosa.
\end{ottave}

\begin{ottave}
\flagverse{41}Soggiunsero di lui mill' altre bozze,\\
Ma quando da me poi lo veddi in faccia\\
Con quella forma, e membra così sozze,\\
Pensate voi se mi cascò le braccia,\\
Anzi nel giorno proprio delle nozze,\\
C'a darmi ognun venia il buon prò vi faccia,\\
Ogni volta con mio maggior dolore\\
Sentivo darmi una stoccata al cuore.
\end{ottave}

Psiche continova il racconto, e dice, che finalmente fu conchiuso il parentado
fra lei, e il Rospo figliuolo di Venere.

\begin{description}
\item[STABILITA la scritta] Fermato, e conchiuso il contratto del Matrimonio,
  che appresso di noi si dice La scritta del parentado.

\item[NON havere a dimenar le dita]Cioè haver a viver senza liavorare, senza durar fatica.

\item[DIVENTAR lupo Cerviero] Divorare, mangiar voracemente, come fa il Lupo
  cerviero\footnote{Lupo Cerviero: lupo che dà la caccia ai cervi. Altro nome della Lince.}. Plin. 1.8.c.22, de \textit{Lupis} dice così: \textit{Sunt in eo genere qui Cervarii vocantur,
  qualem e Gallia Pompeij Magni arena spectatum diximus, huic quamvis in
  fame mandenti si respexit, oblivionem cibi surrepere aiunt, digressumque quaerere aliud}.
  E da tale agonia di mangiare s'assomiglia un huomo, che mangi voracemente,
  ad un lupo cerviero.

\item[BOZZE] Intendi bugie, fandonie, trovati non veri, finzioni, e simili.
  Quando non vogliamo credere qualche novità, che ci sia raccontata diciamo:
  \textit{Io l'ho per bozza}. Traslato da i Pittori, che dicono \textit{bozze}, e \textit{abbozzare} quelle
  prime pennellate, che danno in una tela, e gli Scultori quei primi colpi, che
  danno in un marmo, o altro; i quali additano un non so che del vero, che vi
  faranno col finirle. Vedi sotto C. 7. stan, 5.

\item[MI cascò le braccia] M'abbandonai; mi perdei d'animo; mi sgomentai.
\end{description}

\section{Stanza XXXXII --- XXXXIV}

\begin{ottave}
\flagverse{42}Non lo volevo; pur mi v' arrecai \\
Veduto havendo ogni partito vinto:\\
Ma perché non è il Diavol sempre mai\\
Cotanto brutto com' egli è dipinto,\\
Quand'io più credo a gola esser nei guai\\
Ecco al mio cuore ogni travaglio estinto,\\
Vedendo  ch'ei lasciò, send'a quattr'occhj;\\
La forma delle bozze, e de' ranocchi.
\end{ottave}

\begin{ottave}
\flagverse{43}E molto ben divenne un bel garzone,\\
Che m'accolse con molta cortesia,\\
Ma subito mi fa commissione,\\
Ch'io non ne parli mai a chi che sia;\\
Perch'io sarò, parlandone, cagione,\\
Ch'ei si lavi le mani de fatti mia,\\
E per ne men sentirmi nominare\\
Si vada vivo vivo a farsi sotterrare.
\end{ottave}

\begin{ottave}
\flagverse{44}E perché quivi ancora havrà paura\\
Ch'io non vada a sturbargli il suo riposo,\\
Havrà sopr' ad un monte sepoltura,\\
Che mai si vedde il più precipitoso,\\
Ed alto poi così fuor di misura,\\
Che non v'andrebbe il Bartoli ingegnoso;\\
Oltre che innanzi ch'io vi possa giugnere\\
Ci vuol del buono, e ci sarà da ugnere.
\end{ottave}

Cupido si mostra a Psiche in forma d'un bel giovane, lasciata la fozza figura del
rospo, ed a lei fa comandamento, che di ciò in maniera alcuna non parli, perché altrimenti
facendo; sarà cagione, che egli la lasci, e se ne vada in luogo da non poter
esser più trovato.

\begin{description}
\item[MI v'arrecai] Condescesi; acconsentj, mi v'accomodai; vedi in questo Can,
stan. 80. preso per accomodarsi col corpo; e qui è preso per accomodarsi con
l'animo.

\item[VISTO il partito vinto] Veduto che la cosa haveva a andare in quella guisa.
  La voce \textit{partito} ha diversi significati: perché vuol dire Scrutinio, che noi
  corrottamente diciamo \textit{squittino}. Vedi sotto Can. 6. stan. 109., e di qui \textit{Visto il partito
  vinto}, vuol dire Visto, che il negozio era stabilito così, perché quando il partito
  è vinto, il negozio s'intende stabilito. \textit{Metter il cervello a partito}, significa
  metter in dubbio uno se deva fare, o non fare una tal cosa. \textit{Donna di partito} vuol
  dir meretrice. Si piglia in vece \textit{d'accordo}, \textit{patto}, \textit{baratto}, o \textit{condizione}. Io vendo
  una cosa col tal partito, ec.  Significa \textit{risoluzione}, o \textit{determinazione}. Io ho preso
  partito d'andarmene. Significa \textit{termine}, \textit{pericolo}, Il tale si condusse a mal partito,
  cioè a \textit{cattivo termine}, o a \textit{pericolo di vita}, o \textit{povertà}. Ci serve per esprimer
  \textit{maniera}, \textit{modo}: lo non vi verrò a partito alcuno. Significa \textit{rimedio}, \textit{espediente}.
  Presero per partito di segargli la gamba, ec.

\item[IL Diavol non è brutto com' egli è dipinto] Il Male non è poi sempre tanto, quanto vien raccontato.

\item[NE' guai a gola] Immerso nelle disgrazie. Vedi sopra C. 2. stan. 44. il suo contrario.

\item[A QUATTR'occhi] A solo a solo, \textit{Remotis arbitris}.

\item[SI lavi le man de' fatti mia] Non voglia saper più nulla dime. Tratto dall'antico,
  come si vede in Pilato, che col lavarsi le mani pretese di non haver, che fare
  della Sentenza data contro al nostro sig.\ Giesù Cristo. Il Lalli Eneid. Trau.
  C.4. stan. 92.
  \begin{verse}
    E mi lavo le man de fatti tuoi
  \end{verse}

\item[IL Bartoli ingegnoso] Il Bartoli\footnote{potrebbe riferirsi a Cosimo Bartoli, (Firenze, 20 dicembre 1503 – Firenze, 25 ottobre 1572), che tradusse in lingua toscana il trattato De Architectura di Leon Battista Alberti.}, che ha stampato un trattato dell'architettura,
  perché dice ingegnoso cioè ingegniere, che appresso di noi vuol dire Architetto;
  e non Bartolo legista\footnote{Bartolo da Sassoferrato (Sassoferrato, 1314 – Perugia, 13 luglio 1357) giurista.} (come si trova in alcuni testi, dove dice Bartolo, e non il
  Bartoli) perché trattandosi di salire un luogo erto può giovar più i sapere
  d'un'Architetto, che quello d'un Legista.

\item[CI vuol del buono] Ci sarà molto da faticare, o da spendere, o da camminare,
  o simili, servendoci questo termine per intender tutto quello ci possa esser necessario
  in uno affare, secondo la subietta materia, come per esempio: A scriver
  la presente Opera ci vuol del buono, e s'intende ci vuol molto tempo, molta
  fatica, molti fogli, ec. ed è lo stesso che \textit{ci sarà da ugnere}. Il che viene dal
  medicare i feriti, e però per lo più s'usa in cose di poco gusto, e fastidiose, per esempio:
  Il tale ammazzò uno, vuol haver da ugnere, cioè vuol haver molti travagli,
  spese, difficultà, ec. ad aggiustare il negozio. Il Mureto lib. 9. cap.13. Var.
  lect. disse: \textit{Non parva \& pauca sed multa \& magna ad hoc efficiendum requiruntur}.

\end{description}
\section{STANZA XXXXV.}
\begin{ottave}
\flagverse{45}Poi ch' una strada troverò nel piano,\\
Che veder non si può già mai la peggio,\\
Poi giunto a pié del monte alpestre, e strane\\
Con due uncini arrampicar mi deggio\\
Menando all'erta hor l'una, hor l'altra mano,\\
Come colui, che nuota di spasseggio,\\
Ed anche andar con flemma, e con giudizio\\
S' io non me ne vogl'ire in precipizio.
\end{ottave}

\begin{ottave}
\flagverse{46}Scosceso è il monte in somma, e dirupato,\\
E il viaggio lunghissimo, e diferto,\\
Così disse Cupido smascherato,\\
Dopo cioè ch'ei mi si fu scoperto;\\
Ond'io promessi di non dir mai fiato,\\
E che prima la morte havria sofferto,\\
Che trasgredir d'un punto in fatti, o in detti\\
A suoi gusti, a suoi cenni, a suoi precetti.
\end{ottave}

Cupido accenna a Psiche parte delle fatiche, e travagli, che ella havrà nell'andare
a ricercarlo; e Psiche gli promette di non dir mai nulla a nessuno.

\begin{description}
\item[VNCINI] Strumenti di ferro adunchi, ed aguzzi, servono per appiccarsi a
  gualcosa, e si fanno anche di legno per uso di corre frutti, e per altre occorrenze
  rustiche.

\item[RAMPICARE] E proprio dei gatti, e d'altri animali simili, che salgono
  su per gli alberi, appiccandosi co' rampi, cioè con l'ugna delle zampe. Vedi
  sotto in questo C. stan. 68. E ci serviamo del verbo rampicare per esprimere un
  che salga in qualche luogo difficile, ancor che lo faccia senza rampicare. Vedi
  sotto C. 9, stan. 25.

\item[NVOTA di spasseggio] Nuotare di spasseggio diciamo quand' uno essendo tutto
  nell' acqua dalla testa in fuori, cava fuora di essa un braccio per volta ordinatamente,
  battendolo sopra all'acqua per romperla, e spingersi avanti.

\item[NON dir fiato, e non fiatare] È lo stesso che non parlare. Vedi sotto C.6. st. 12.
  Si dice anche \textit{non alitare}. \textit{Non far verbo}, Berni Orland.
  \begin{verse}
    E senza più fiatar mi stava chiotto.
  \end{verse} Vedi sopra C, 1. stan. 10.

\item[GVSTI, cenni, precetti] In questo luogo hanno tutti tre lo stesso significato
  di comandamento. Considerandosi \textit{gusto} per il meno stimato, \textit{cenno} nel secondo
  luogo, e \textit{precetto} per lo più stimato, denotando dominio.
\end{description}
\section{Stanza XXXXVII --- XXXXVIII.}

\begin{ottave}
\flagverse{47}Ne tal cosa a persona haurei scoperta,\\
Perché tuttavia la gente sciocca\\
Ridea del rospo, e davami la berta;\\
Ed io, che quand'ella mi venne in cocca,\\
Non so tener un cocomero all'erta,\\
Mi lasciai finalmente uscir di bocca,\\
Che quel non era un rospo, ma in effetto\\
Un grazioso, e vago giovanetto.
\end{ottave}

\begin{ottave}
\flagverse{48}E che, se lo vedesson poi la notte\\
Quand'in camera meco s'è serrato,\\
E getta via la scorza delle botte\\
Ch'un sole proprio par sputato,\\
Le male lingue forse starian chiotte\\
Che sì de' fatti altrui si danno piato,\\
Però che non si può tirar un peto\\
Ch'il comento non veglian fargli dreto.
\end{ottave}

Vinta Psiche dalla collera, che le venne per esser burlata dall'altre donne,
scoperse il segreto, E nota che l'Autore mostra il costume delle nostre femmine,
e quelle di tutto il mondo, le quali obligate a narrar qualche loro mancamento;
si fanno dalla lontana, e cercano di persuadere d' haverlo commesso, necessitate, e
forzate da' maggiori mancamenti d' altri.
\begin{description}
\item[DAVANMI la berta] Mi davano la burla, mi beffavano, mi minchionavano.
  Berta si dice quel ceppo, col quale, impernato sopra i pali, si fanno le palizzate
  ne i fiumi, battendo sopra i pali per via di corde, o manichetti, che sono in
  detto ceppo. E il Latino irridere, Raccontano le nostre donne, che quel sagace
  villano nominato Campriano, del quale diremo sotto C, 11. stan. 48. essendo venuto
  in mano della giustizia per le sue cattive opere fu condennato a esser messo
  in un sacco, e buttato in mare; In esecuzione di che fu messo dentro al sacco, e
  consegnato a i famigli, che lo buttassero in mare. Nell'andar costoro ad eseguire
  gli imposti furono per strada assaliti da alcuni masnadieri, i quali si crederono,
  che in quel sacco fusse roba di valore; onde i famigli per scampar la vita
  lasciato ivi il sacco con Campriano, si fuggirono. Campriano piangendo
  si doleva della sua disgrazia, il che sentito da uno di quei masnadieri gli domandò
  perché piangeva, ed a qual fine era stato messo in quel sacco. Il sagace Campriano
  gli rispose; Io piango di quel, che altri gioirebbe, ed è, che questi SS.
  voglion darmi per moglie Berta unica figliola del Re nostro, ed io non la voglio,
  conoscendomi inabile a tanto grado, per esser' un povero villano. E perché essi
  dicono, che se ella non si marita a me, l'oracolo ha detto, che questo Regno andrà
  sottosopra, m'hanno messo in questo sacco per condurmi a farmela pigliar
  per forza; e questa è la causa del mio pianto. Il masnadiero credendo alle parole
  di costui, si concertò con i Compagni d'andar'esso a pigliare questa buona
  fortuna, e ripartirla con essi: onde fattosi mettere dentro al sacco da Campriano
  che non restava di pregarlo a volergli far del bene quando fusse poi Re,
  fece allontanare i compagni, e serratolo entro al sacco, stette aspettando, che
  ritornassero coloro, i quali non stettero molto a comparire con nuova gente, e
  veduto quivi il sacco abbandonato, lo ripresero, ed essendo vicini alla riva del
  mare, ve lo precipitarono, e così sposarono a Berta il balordo mafnadiero. E
  di qui venne \textit{dar la berta}, o \textit{la figliuola del Re}, che vuol dir \textit{burlare}, \textit{minchionare},
  come habbiamo accennato. Si dice anche \textit{dar la madre d' Orlando}, perché da
  alcuni si crede, che la madre d'Orlando Paladino havesse nome \textit{Berta}.

\item[QUAND' ella mi viene in cocca] Quando mi viene in proposito di dire. E si dice
  anche \textit{ella mi viene in cocca} per intendere quand' io entro in collora, come
  s'intende nel presente luogo. E cocca diciamo quella tacca la quale e nella freccia
  per adattarla in fu la corda dell'arco\footnote{Da cui evidentemente \textit{incoccare} e \textit{scoccare}.} da i Latini detta \textit{Crena}, donde poi diciamo
  \textit{cruna}, quella tacca, o fessura, che è nella parte opposta alla punta dell'ago da
  cucire, dal Gr. \textit{Acocche}; \textit{estremità acuta}, Dan, Inf. C. 12.
  \begin{verse}
    Chiron prese lo strale, e con la cocce
    Fece la barba indietro alle mascello
  \end{verse}

\item[NON so tenere un cocomero all'erta] non posso far di meno di non la dire. Si
  fa questa comparazione al cocomero, perché essendo questo di, figura sferica, e
  liscio, facilmente ruotolando può scorrer giù per un'erta, o monte, e facilmente
  può esser anche tenuto fermo; onde molto ben si dice Non fa tener un  cocomero
  all'erta d'uno che sia facile a palesare un segreto, che con ugual facilità
  potria tacerlo.

\item[PRETTO sputato] Similissimo a lui: per appunto come lui, e senza alterazione
  alcuna come è il vino pretto, cioè senz'alterazione d'acqua, o d'altro. E
  quella aggiunta di sputato si toglie da coloro, che pigliano le misure col filo, come
  muratori, e legniaioli, i quali in qualche occasione per andar giusti, e' per appunto
  sogliono tirare il filo, e sputandovi sopra lasciano cascar lo sputo nella
  parte, che gli è sotto, e da quello conoscono se il lavoro e per appunto.

\item[CHIOTTE] Chete. Voce Fiorentina, ma poco usata fuor di scherzo, se bene,
  come poco sopra s'è visto, l'usò il Berni nell'Orlando. \textit{E senza più fiatar ne stava chiotto}.

\item[SI danno piato de' fatti d'altri] Gli danno pensiero; Gli sono a cuore i fatti d'altri.
  Si metterebbero a litigare per i fatti d' altri; Che \textit{Piato} vuol' dir \textit{litigio}.
  Vedi sotto C. 7. stan, 27.

\item[NON si può tirar un peto ec.] Non si può far una cosa benché minima, che il
  popolo non vi voglia far sopra i suoi discorsi.
\end{description}

\section{STANZA IL --- LIII}

\begin{ottave}
\flagverse{49}Le ciglia inarca, e tien la bocca stretta\\
Chiunque da me tal maraviglia ascolta;\\
Ma quel ch'importa a sordo non fu detta,\\
Che Vener, ch'ogni cosa havea ricolta,\\
Per veder s'ell'è vera, o barzelletta,\\
Poiché a dormire ognun sel' era colta,\\
Entra in camera, e vien pian, Piano al letto,\\
E trova il tutto appunto come ho detto.
\end{ottave}

\begin{ottave}
\flagverse{50}E nel vedere in terra quella sposiglia,\\
Che per celarsi al mondo il giorno adopra,\\
Di levargliela via le venne voglia, \\
Acciò con essa più non si ricuopra: \\
Così la prende, e poi fuor della soglia \\
Fa un gran fuoco, e ve la getta sopra, \\
Ne mai di lì si volle partir Venere \\
Infin che non la vedde fatta cenere.
\end{ottave}

\begin{ottave}
\flagverse{51}Fu questa la cagion d'ogni mio male,\\
Perché quando Cupido poi si desta\\
Si stropiccia un po gli occhi, e dal guanciale\\
Per levarsi dal letto alza la testa,\\
E và per rivestirsi da animale,\\
Ne trovando la solita sua vesta\\
Si volta verso di me, si morde il dito,\\
E nello stesso tempo fu sparito.
\end{ottave}

\begin{ottave}
\flagverse{52}Non ti vo dir com'io restassi allora,\\
Che mi fovvenne subito di quando\\
Il primo dì mi si svelò, c'ancora\\
Mi fece l espertissimo comando,\\
Ch'in alcun tempo io non la dessi fuora,\\
Ed io son' ita sciocca, a farne un bando,\\
E poi mi pare strano, e mi scontroco,\\
S'egli è in valigia, ed ha comprato il porco.
\end{ottave}

\begin{ottave}
\flagverse{53}Sospesa per un pezzo io me ne stetti, \\
Chi io aspettavo pur ch' ei ritornasse; \\
A cercarne per casa poi mi detti \\
Per le stanze di sopra, e per le basse; \\
Guardo su pel cammin, giro in su i tetti, \\
Apro gli armarj, e a scostar le casse,\\
Ne trovandolo mai, al fin mi muovo\\
Per non fermarmi fin ch'io non lo trovo.
\end{ottave}

Il segreto palesato da Psiche, venne all'orecchie di Venere, la quale quando
Cupido dormiva gli abbruciò la veste da rospo; il che veduto Cupido la mattina
se ne fuggì, e Psiche si messe a cercar di lui.

\begin{description}
\item[NON fu detta a sordo] Fu detta a chi ne fece capitale, a chi importava saperlo.

\item[OGNI cosa havea raccolto] Haveva sentito, e inteso ogni cosa.

\item[BARZELLETTA] Cosa non vera, ma detta per scherzo. E si dice Barzellettare uno,
  che discorra burlando, e scherzando.

\item[PIAN piano] Questo termine, che vuol dire Adagio adagio, significa ancora
  (come nel presente luogo) Senza far punto strepito, o romore.

\item[GUANCIALE] Piccolo piumaccio, sopra il quale si posa la guancia, quando
  si sta nel letto, detto \textit{guanciale} da guancia, come in diversi luoghi, è detto
  \textit{origliera} da orecchio.

\item[RIVESTIRI] Rivestirsi da rospo. Ecco la voce generica animale, che noi
  usiamo le, come accennammo sopra in questo C. stan. 4.

\item[NON  ti vo dire] È lo stesso termine, \textit{che pensate voi}, visto sopra in questo C.
  stan 41. Ed esprime Non voglio dirlo, perché da per voi vel' immaginerete; Vedi sotto la stan. 76.

\item[NON la dessi fuora] Non la manifestassi, ed io n' ho fatto un bando; ed io, ho
  pubblicata per tutto. \textit{Non modo tubam, sed etiam praeconem adhibui}.

\item[MI scontorco] Scontorcersi e proprio delle serpi ferite; e parlandosi d'huomini
  s'intende un certo atto, che denota dolore per qualche disgusto, o travaglio insopportabile.

\item[È IN valigia] È in collora, in ira; Nel bugnolone, nel gabbione, e simili,
  che in moltissimi ne habbiamo in questo significato.

\item[COMPRAR il porco] Significa andarsene; ed è come l'interpetrazione di \textit{svignare},
  quasi voglia dire \textit{suinam}, cioè \textit{suillam emere}, o che più tosto sia detto
  \textit{svignare} quasi \textit{scappar via dalla vigna, e fuggirsene}, come quei che son colti a
  cogliere, o mangiare uva nell'altrui vigna. Diciamo \textit{battere il taccone}, \textit{battersela},
  \textit{corsela}, e che se ben son voci, che hanno del furbesco, sono però comunemente
  usate, e sempre intese in questo senso. Vedi sotto C.~11. stan.~11.

\end{description}
\section{Stanza LIV --- LVIII}

\begin{ottave}
\flagverse{54}Scappo di casa, e via vò sola sola,\\
Ne son lontana ancora una giornata,\\
C'io sento dire: Aspettami figliuola,\\
Mi volgo, e dietro veggomi una Fata,\\
E perch'ella mi diede una nocciuola,\\
Quest'è meglio, diss'io, d'una sassata,\\
Di ciò ridendo un'altra sua compagna\\
Mi pose in mano anch'ella una castagna.
\end{ottave}

\begin{ottave}
\flagverse{55}Ed io, c'allora harei mangiato i sassi\\
M'accomodai per darvi su di morso,\\
Ma fummi detto ch'io non la stiacciassi,\\
S'un gran bisogno non mi fusse occorso.\\
Vergognata di ciò con gli occhi bassi\\
Il termine aspettai del lor discorso,\\
Poi fatte le mie scuse, e rese ad ambe \\
Mille grazie, le lascio, e dolla a gambe.
\end{ottave}

\begin{ottave}
\flagverse{56}Ripongo la nocciuola, e la castagna,\\
E rimetto le gambe in sul lavoro\\
Per una lunga, e sterile campagna\\
Disabitata più che lo Smannoro;\\
Dopo cinque anni giunta a una montagna,\\
Mi si fe innanzi un grande, e orribil toro,\\
Che ha le corna, e i più tutti d'acciaio,\\
E tira che correbbe nel danaio.
\end{ottave}

\begin{ottave}
\flagverse{57}E come Cavalier ch'al saracino\\
Corre per carnovale, o altra festa,\\
Verso di me ne viene a capo chino\\
Con la sua lancia biforcata in testa,\\
Io già con le budella in un catino\\
Addio dicevo al Mondo, addio chi resta.\\
Addio Cupido dove tu ti sia,\\
A rivederci ormai in pellicceria.
\end{ottave}

\begin{ottave}
\flagverse{58}O Mamma mia, che pena, e che spavento \\
Hebbe allor questa mezza donnicciuola? \\
Tremavo giusto come giunco al vento, \\
Che quivi mi trovavo inerme, e sola; \\
Pur come volle il Ciele io mi rammento\\
Del dono delle Fate, e la nocciuola\\
Presa per caso presto sur' un sasso\\
La scaglio, ella si rompe, e n'esce un masso.
\end{ottave}

Messasi in viaggio Psiche s'imbatté in due Fate, dall'una delle quali hebbe
una nocciuola, e dall'altra una castagna, e le dissero, che non le stiacciasse, se
non a un gran bisogno. Dopo cingue anni di cammino per un deferto arrivò a
pié d'una montagna, dove le venne incontro un toro con le corna d'acciaio;
dal quale spaventata Psiche stiacciò la nocciuola, e ne nacque un masso.

\begin{description}
\item[FATA] Fate sono donne indovine dette secondo alcuni dal Greco \textit{Phatis}, che
  suona Donna indovina, e quelle forse che i Latini co' Greci chiamano \textit{sibille},
  ma dalle nostre Balie nel contare le novelle a i fanciulli son prese per donne di
  buon genio, e che fanno servizio al prossimo con le loro azioni, e son contrarie
  all'Orco, al Bau, e alle Befane, che sono nimici de' bambini, a i quali queste
  sempre fanno servizio, ed il Poeta, col regalo, che fa lor fare a Psiche, mostra
  questa verità. Da gli antichi furono anche chiamate Ninfe, e Dee, e l'Ariosto
  nel suo Furioso ciò afferma, dicendo:
  \begin{verse}
    Queste c'hor Fate, da gli antichi furo
    Chiamate Ninfe, e Dee con più bel nome.
  \end{verse}
  Di queste Fate discorre  l'Autore sotto, nel Canto settimo, ed è credibile, che
  questa voce Fate venga dal Latino \textit{Fata fatorum}, che Dan, Inf, C, 9, disse le fata.
  \textit{Che giova nelle fata dar di cozzo?}

\item[QUESTO è meglio a una sassata] Quando si riceve da uno qualche regalo di poco
  valore, si dice per scherzo: \textit{Questo è meglio d'una sassata}, o vero \textit{d'un calcio di
    mosca}: volendosi inferire, che da quello, al nocivo, o al nulla vi è poca differenza.
  Plau. in Tr. disse \textit{Melius est quam deterrimum}.

\item[ALLOTTA haurei mangiati i sassi]. Allora havevo così gran fame, che haurei
  mangiata qualsivoglia cosa, ancor che dura quanto un sasso. Io crederei, che il vestitore
  di questa favola havesse seguitato i compositori de' Palmerini, degli Amadis,
  ed altri Cavalieri erranti, che mai in tanti viaggi, che fanno lor fare, pur'
  una volta si trova, che in campagna mangiassero; ma il sentir, che Psiche discorre
  di mangiare, e che fu levata dond'ell'era, perché non vi morisse di fame,
  mi fa credere diversamente, cioè che in questo suo iungo viaggio le Fate le empiessero
  il corpo, che ella non sen' avvedesse.

\item[SCHIACCIARE], Corrottamente diciamo anche \textit{stiacciare}, vuol dir Rompere,
  o infragnere, ed è proprio di quelle cose, che hanno guscio, come noci, mandorle,
  uova, e simili.

\item[DOLLA a gambe] Comincio a camminare; è lo stesso che \textit{rimetto le gambe in
  lavoro}, che è nell'ottava 56. seguente. Il Lall. En. Tr. C. 2, stan. 33.
  \begin{verse}
    Quand'io la diedi a gambe,e dentro a un fosso
  \end{verse}
  Lasca Nov. 6. \textit{Temendo, che colui non gli uscisse dietro, s'uscì di casa prestamente, e
    la dette a gambe, e per la fretta si scordò di serrar l'uscio}. I Lat. pure dissero \textit{conijcere
    se in pedes}.

\item[LO smannoro] Così è detta una gran pianura posta poco lontana per di sotto alla
  Città di Firenze, la quale dura più miglia per ogni verso, senza mai trovarsi
  una casa, se bene è tutta coltivata. Si dovrebbe dire \textit{Ormannoro} dalla famiglia
  antica degli Ormanni, la quale era già padrona di tutte quelle pianure, che si dicevano
  \textit{Campi Ormannorum}.

\item[TIRA che correbbe in un denaio] Tira così aggiustatamente, che egli correbbe
  in ogni piccolo berzaglio, come è un denaro, che è la quarta parte del quattrino
  Fiorentino, con altro nome detto picciolo, ed un giulio ne vale 160.

\item[SARACINO] Così chiamiamo quella statua, o fantoccio di legno, che figura
  un Cavaliero armato, al quale (come a berzaglio) corrono i Cavalieri le lance;
  E si dice anche \textit{Buratto}, che è un' altra sorta di berzaglio (il quale si mette
  in vece del Saracino) ed è una mezza figura secondo alcuni, che nella sinistra.
  tiene lo scudo, nella destra la spada, o bastone; la quale se non è colpita nel petto,
  girando si rivolta, e percuote colui, che fallì.\footnote{Vedi pure al cantare precedente, ottava 75, la voce ``FOLA''.}

\item[LANCIA biforcata] Intende le corna del Toro.

\item[CON le budella in un catino] Mi credeva già morta; Mi credeva già essere stata
  sbudellata dal Toro. Luigi Groto Cieco d'Adria, in una sua lettera al Petr.
  dice: Quei cani con il loro bau bau ci fecero parere d'havere le budella in un
  catino.  E \textit{Catino} Intendiamo un vaso di terra, o d'altra materia per servizio di
  Cucina, e per uso di lavar piatti, ec.

\item[A RIVEDERCI in pellicceria] A rivederci fra i morti. Questo è il comiato,
  che noi finghiamo, che si diano le volpi  l'una con l'altra, perché sapendo, che
  devon esser'ammazzate, e le lor pelli vendute, dicono alli lor figli, quando da
  esse si separano: \textit{A rivederci in pellicceria}, che così si chiama in Firenze quella
  strada, nella quale sono le botteghe di coloro, che comprano, e vendono pelli
  di animali per foderare abiti, ec. ed in mano di costoro, o tardi, o per tempo
  sanno che devon capitare.

\item[O MAMMA mia] O mia madre. Esclamazione di spavento, e di timore,
  usata propriamente da' fanciullini, quasi dica: O mia madre soccorretemi in
  questo pericolo.

\item[DONNICCIVOLA] Vuol dir Donna di spirito minore di quel che converrebbe
  al suo naturale, da i Latini detta \textit{Muliercula}. Sì che mezza donnicciuola vuol
  dir Donna quasi da nulla, e senza spirito.

\item[GIUNCO] Specie di virgulto, che nasce in luoghi padulosi\footnote{sic: ``padulosi''.}, del quale si servono
  i Villani per legare i stralci teneri delle viti, ec.

\item[MASSO] S'intende un sasso grande. Questi nostri scarpellini chiamano il
  masso La cava delle pietre.
\end{description}

\section{ STANZA LIX. STANZA LX.}

\begin{ottave}
\flagverse{59}Tal pietra per di fuora è calamita,\\
E ripiena di fuoco artifiziato, \\
Hormai arriva il Toro, ed alla vita \\
Con un lancio mi vien tutto infuriato, \\
Ma perché dietro al masso ero fuggita \\
Il ribaldo riman quivi scaciato,\\
Ch'in esso dando la ferrata testa \\
da qulla calamita affisso resta.
\end{ottave}

\begin{ottave}
\flagverse{60}Sfavilla il masso al batter dell'acciaro,\\
E dà fuoco al rigiro ch'è nascosto,\\
Ed egli a' razzi ch' allor ne scapparo\\
Un colpo fatto haver vede a suo costo,\\
Perché non vi fu scampo, ne riparo,\\
Ch'ei fra le fiamme non si muoia arrosto,\\
Ed io scansato il fuoco, e ogni altro affronto,\\
Lieta mi parto, o tire innanzi il conto.
\end{ottave}

Il detto sasso era per di fuori calamita, e dentro era fuoco lavorate, onde il
Toro perquotendovi con le corna ch'erano d'acciaio vi rimasero appiccate, e
da quella percossa nacque il fuoco, il quale s'appiceò all'ordigno, ed abbruciò
il Toro. Psiche libera da questo incontro seguitò il suo viaggio.
\begin{description}
\item[CALAMITA] È la pietra simpatica del ferro, o forse madre, dai Latini
detta \textit{Magnes}. Vedi sotto C. 8, stan. 45. e 66.

\item[FUOCO artifiziato] Vuol dire ogni forma di composizione fatta con polvere
(che diciamo Da archibuso) tanto per guerra, quanto per feste.

\item[RIMANE scaciato] Riman burlato. È lo stesso, che \textit{rimaner con un palmo di
  naso}, che vedremo sotto C. 6, stan. 5.

\item[RIGIRO] Intende l'ordigno di fuoco lavorato, che è composto dentro al
  masso.

\item[RAZZI] Raggi di fuoco o del Sole, o d'altro scintillante. Ma dicendo assolutamente
  razzi, intendiamo quei fuochi artifiziati, che si fanno in occasione
  di feste con polvere d'archibuso constipata, e benissimo legata entro alla carta,
  ridotta come pezzi di canna,

\item[TIRO innanzi il conto] Seguito il mio viaggio, Vedi sotto C. 6. stan. 16.
  Tanto serviva \textit{tiro innanzi}, e senza mettervi \textit{il conto} suonava il medesimo,
  ma l'uso nato da quei, che tengono libri di debitori, e creditori ci obliga a dir così.
\end{description}

\section{ STANZA LXI --- LXVI}

\begin{ottave}
\flagverse{61}Più là ritrovo un grand' uccel grifone,\\
E topi assai, che giran come pazzi,\\
Perch'egli entrato in lor conversazione\\
Gli becca, grafia, e ne fa mille strazzj,\\
Di lor mi venne gran compassione,\\
E vo per ovviar, ch'ei, non gli ammazzi,\\
Ma quei mi sente al moto, e in pié si rizza,\\
E per cavarsi, vien con me la stizza.\\
\end{ottave}

\begin{ottave}
\flagverse{62}Questo animate ha il busto di cavallo\\
Di bue la coda, e in fu le spalle ha l'ale,\\
Il capo, e il collo giusto come il gallo,\\
E i pié di nibbio vero, e naturale,\\
Gli artigli di fortissimo metallo\\
Grandi grossi, e adunchi in modo tale\\
Che non vedesti quando leggi, o scrivi,\\
Mai de tuoi di più bei interrogativi.
\end{ottave}

\begin{ottave}
\flagverse{63}Son' appuntati poi c'a far più acuto\\
Un'ago altrui darebbe delle brighe,\\
Tal che, s'al viso fussimi venuto\\
Con essi, mi lasciava assai più righe\\
D'un libro di maestro di liuto,\\
Ed una stamperia di falsarighe,\\
Con farmi a liste come le gratelle\\
Da quocervi le triglie, e le sardelle.
\end{ottave}

\begin{ottave}
\flagverse{64}Hor per tornare. In quel ch'io ho timore\\
Ch' il mio grifo sia scherzo del grifone\\
La castagna ch'io in tasca caccio fuore\\
La rompo, e n'esce subito un Lione,\\
Che mi scemò non poco il batticuore\\
Perch'egli in mia difesa a lui s'oppone,\\
E mostrogli hor con l'ugna, ed hor co' denti\\
In che mo si gastigan gli insolenti.
\end{ottave}

\begin{ottave}
\flagverse{65}L'uccello anch'egli, che non ha paura\\
Gli rende molto ben tre per pan per coppia,\\
Ma quel che haver del suo nulla sicura\\
Il contraccambio subito raddoppia,\\
E ben ch'ei voglia star seco alla dura\\
L'afferra, e stringe tanto ch'egli scoppia\\
Di poi garbatamente gli riesca\\
Gli stinchi su i nodelli, e me gli reca.
\end{ottave}

\begin{ottave}
\flagverse{66}Metto uno strido, e mi ritiro in dreto\\
Io ch'ho paura allor ch'ei non m'ingoi,\\
Ma quegli ch'è un Lione il più discreteo,\\
Che mai vedesse il mundo prima, o poi,\\
Ciò conoscendo tutto mansueto\\
Gli lascia in terra, e va pe' fatti suoi,\\
Ed io gli prendo allora, essendo certa\\
D'averne a haver bisogno in sì grand'erta
\end{ottave}

\begin{ottave}
\flagverse{67}Lá dove non si può tenere i piedi,
Ma bisogna che l'huom vada carponi,\\\
Perciò con quegli uncini poi mi diedi\\
A costeggiar il monte brancoloni,\\
E convenne talor farsi da piedi\\
Battendo giù di grandi stramazzoni,\\
Perché non v'è dove fermar il passo:\\
Cagion che spesso mi trovai da basso.
\end{ottave}

Psiche superato il pericolo del Toro s'imbatte in un' uccello Grifone, che havea
l'ugna d'acciaio, onde roppe la castagna, e n'usci un Lione, che la difese
da quello uccello, e tagliandogli gli artigli, li portò a lei, la quale gli prese, e
con essi attaccandosi all'erto monte, cominciò a salirvi.

\begin{description}
\item[TOPI che girano come pazzi] Sorci, che vanno in'qua e in la correndo senza
  saper dove determinatamente, appunto come fanno i pazzi.

\item[CAVARSI la stizza. Sfogar la collora, la rabbia, l'ira.

\item[NIBBIO] Uccello di rapina noto. Qui descrive il Grifone, e lo fa mezzo cavallo, e
  mezzo uccello, e con la coda di bue, e se bene da i pi e descritto mezzo lione,
  e mezzo uccello, e nimico mortale de' cavalli, come si deduce da Verg.
  Eg.8. \textit{Iungantur iam Gryphes Equis}, tuttavia non fa errore a comporlo di che bestie
  gli è piaciuto,  perché questo mostruoso animale in ogni maniera che sia è
  del tutto favoloso, secondo Plinio lib, 10. c.44. \textit{Pegasos} (dice egli) \textit{equino capite
    volucres, \& Gryphes aurita aduncitate rostri fabulosos reor, illos in Scythia, hos in
    AEthiopia}.

\item[INTERROGATIVO] È un contrassegno d'ortografia, il quale si pone in fine
  de' periodi, che conchiudono interrogare, o richiedere, e perciò è detto Punto
  interrogativo. E perché tal contrassegno è di figura simile a un'uncino, però a
  questo assomigliamo gli artigli degli uccelli, come fa qui il Poeta, assomigliandogli
  a quelli del grifone.

\item[LIBRO di maestro di liuto] Intendi libro da musica,, che son pieni di righe,
  affine di icrivervi sopra le note musicali.

\item[FALSARIGHE] Carte rigate, e lineate di nero, le quali si mettono sotto al
foglio, sopr'al quale si scrive, affine di far i versi diritti, ed uguali camminando
sopra quel segno, che dalla falsariga per trasparenza si vede sopra il foglio, ove
si scrive.

\item[LISTE] Qui vale per striscette di ferro, con le quali son composte le gratelle
  strumenti da cucina, che servon per mettervi sopra il pesce, o altro a quocere
  arrosto. E con tutte queste similitudini intende, che se l'uccelio havesse messo
  gli artighi addosso a Psiche, l'haverebbe malamente graffiata, e segnata.

\item[GRIFO] Vuol dir Faccia di porco, o simili; e s'intende alle volte: la faccia
  dell'huomo, ma per scherzo, o per disprezzo; e qui il Poeta se ne serve per far
  bisticcio di Grivo, e Grifone.

\item[BATTICUORE] Paura, timore. Da quella frequenza di battere, che fa il
  polmone dalla parte del cuore, quando si ha qualche spavento: I Latini pure dicevano
  \textit{animi, vel cordis percussio}.

\item[INSOLENTE] Arrogante, fastidioso, petulante. Uno che tratta, e procede fuori del dovere.

\item[GLI rende tre pani per coppia] Gli rende più del suo dovere, perché a render
  tre pani per i due, che è la coppia, si rende la meta più del dovere: E con questo
  modo di dire s'intende, che uno si difenda da un' altro con parole, e con fatti
  sempre con vantaggio, che diciamo anche \textit{render pane per focaccia}.

\item[NON si cura haver niente di suo] Intendi Non vuol'esser da lui superato.

\item[AFFERRARE] Abbrancare, pigliare stretto; \textit{Vi apprehensum detinere}.
\item[NODELLI] Intendi la congiuntura delle gambe co' piedi.
\item[ANDAR carponi] Camminar co' piedi, e con le mani per terra, ed è lo
  stesso, che \textit{Andar brancolone}, che si vede nel verso seguente; se non che questo
  vuol dir Salire adoperando le mani, e i piedi; e \textit{carponi} è camminare alla piana
  con le mani, e co' piedi, Dante Inf. C, 26. descrivendo una simil salita dice:
  \begin{verse}
    \backspace E proseguende la folinga via
    Tra le schegge, e tra i rocchi dello scoglio
    Il pié senza la man non si spedia.
  \end{verse}
\item[STRAMAZZONI] Intendi Cascate; che per altro ramazzone intendono gli
  schermitori una specie di taglio.
\end{description}

\section{Stanza LXVIII --- LXXI.}

\begin{ottave}
\flagverse{68}Tutti quei topi via ne vengon ratti,\\
E furon per mangiarmi dalla festa,\\
Però che dalle granfie io gli ho sottratti\\
Di quella bestia a lor tanto molesta;\\
Così vò rampicando come i gatti\\
Sull'aspro monte dietro alla lor pesta,\\
Sopportando fatiche, stenti, e guai,\\
E fame, e fete quanto si può mai.
\end{ottave}

\begin{ottave}
\flagverse{69}Pur finalmente in capo a due altri anni\\
Giungemmo al luogo tanto desiato;\\
Ma non finiron qui mica gli affanni,\\
Perché di muro il tutto è circondato;\\
E qui s'aggiugne ancor male a malanni,\\
Ch'io trovo l'uscio, ma'l trovo diacciato;\\
Pensa s'allor mi venne la rapina,\\
E s'io dicevo della Violina.
\end{ottave}

\begin{ottave}
\flagverse{70}Hora tu sentirai ch'il dare aiuto\\
A tutti quanti sempre si conviene,\\
Perché già mai quel tempo s'è perduto,\\
Che s'è impiegato in far' altrui del bene,\\
Non dico sol all'huom, ma anco a un bruto,\\
Che forse immondo, e inutile si tiene,\\
E che tu non lo stimi anche una chiosa,\\
Però che ognuno è buono a qualche cosa.\\
\end{ottave}

\begin{ottave}
\flagverse{71}Se tu giovi al compagno, allor tu fai\\
(Quasi gli presti roba) un capitale,\\
Anzi talor per poco, che gli dai\\
Ti rende più sei volte che non vale.\\
Ma non si dee ciò pretender mai,\\
Perch'ell è cosa, che starebbe male;\\
Questo è un censo il quale a chi lo prende\\
Richieder non si può s'ei non lo rende.
\end{ottave}

I topi, che Psiche liberò dagli artigli del Grifone la seguitarono facendole
gran festa, e con quella compagnia in capo a due altri anni arrivé Psiche al luogo
dove era Cupido, che era un recinto di mura, dentro al quale non si poteva
passare se non per una sola porta, e questa era serrata.

\begin{description}
\item[VENGONO ratti] Vengono velocemente dal Latino \textit{rapidus}, D. Infer. C. 21.
  \begin{verse}
    Perch'io mi mossi, ed a lui venni ratto
  \end{verse}
  Ed habbiamo rattezza, per prestezza, o velocità. Varch, Stor. lib. 4. \textit{In quel mezzo il
  sig.\ Sciarra Colonna partissi con gran rattezza da Roma}.

\item[FAR festa a uno] Rallegrarsi con uno. Ricevere, o trovar uno con atti di
  amorevolezza, e cortesia; Che nelle bestie si conosce tal rallegramento da i gesti,
  come nel cane dal dimenar della coda, ne i gatti dal fregarsi addosso a uno,
  ed altri animali dal moto degli orecchi, come forse si conosceva in quei topi. Il
  Lat. adulari fanno venire alcuni da \textit{ad, \& ura}, che in Greco significa coda quasi sia
  \textit{cauda adblandiri}.

\item[RAMPICANDO] Intendi salire appiccandosi con gli artigli del Grifone, come fanno i gatti.
  Viene da \textit{rampi} che s'intende ugne di gatto, lione, tigre, e simili.
  Si dice anche \textit{inerpicare} da erpico strumento rustico da romper le terre.
  Mattio Franzefi sopra alle maschere dice:
  \begin{verse}
    \backspace Non vi crediate, che qualunque saglie
    Havesse da un posta tanto ardire,
    Ch'inerpicasse sopra alle muraglie
  \end{verse}
  Ma oggi corrottamente si dice \textit{innarpicare}, e \textit{annarpicare}. Vedi sotto Can. 9. stan. 25. e 28.

\item[DIETRO alla lor pesta] seguitando le lor pedate.

\item[MICA] È una particella riempitiva in compagnia della negazione per emfasi
  del discorso, appunto come i Latini dicono ne quidem » se bene \& diferente dal
  Latino, perché non s' usera per affermativa, io voglio mica, come essi dicono ero
  quitem volo, sì che se bene e per emfafi ha però qualche parte del negativo, quasi
  diciamo: Io now voglio ne pur' una mica, che vuol dir minuzolo di pane, o granello
  di sale. IL Petr. Son..91. We mica trove il mio ardente defio.

\item[AFFANNI] \makebox[1pt]{}  Dolori di cuore, che fanno quasi venire in angoscia, Petrar.\ son.~11.
  \begin{verse}
    Se la mia vita dal aspro tormento
    Si puo tanto schermire, e dagli affanni.
  \end{verse}

\item[AGGIUNGER male a malanni] Al male accrescer male, e peggio.

\item[USCIO diacciato] \makebox[1pt]{} Cioè porta serrata. Vedi sopra C.~3.\ stan.~3.

\item[MI venne la rapina] Mi venne rabbia, collora, o stizza. Rapina vuol dire rubamento
  violento: quindi uccello di rapina; ma dalle nostre donne è presa in
  cambio di \textit{rabbia}, per sfuggir di dire \textit{rabbia} creduta parola peccaminosa, e dicono
  \textit{arrapinare}, \textit{arrapinato}, per \textit{arrabbiare}, ed \textit{arrabbiato}.

\item[DICEVO della violina] Dicevo del male fra me medesimo, perché le cose non
  andavano a mio modo. Questo so che significa \textit{Dir della violina}, non so già da
  che habbia origine questo dettato, che è lo stesso che \textit{Dir l'orazione della bertuccia}.

\item[NON lo stimi una Chiosa] Non lo stimi punto. Vedi sopra C. 3. stan. 60, alla voce \textit{iosa}.

\item[FAR un capitale] Metter insieme una somma considerabile di denaro per haverlo
  pronto a ogni suo bisogno: Si dice anche \textit{far un'assegnamento}.

\item[CENSO] La natura del censo, è che colui, il quale presta danari a censo, non
  può richieder la somma principale, che egli dà, ma solo i frutti d'essa; può ben
  colui che gli piglia render la medesima somma principale a ogni suo piacimento,
  e colui, che la diede è forzato a riceverla, come dice il Poeta assomigliando colui,
  che fa il piacere a un'altro, a uno che dia a censo, e dice, che colui che
  fa il piacere non dee, ne può pretender la ricompensa, ma la può bene sperare,
  e ne vive creditore: Che perciò ben dice Seneca \textit{de Beneficijs} lib.3.c.14, \textit{Vide etiam
    atque etiem cui des, nulla actio erit, nulla repetitio}, B lib, 4.cap. 39. \textit{Alia conditio
    est in credito, alia in beneficio}.

\end{description}

\section{STANZA LXXII --- LXXV}

\begin{ottave}
\flagverse{72}Guarda s'ell'è così; Io per la mia\\
Pietà di prender di quei topi cura,\\
Da lor vinta restai di cortesia,\\
E n'hebbi la pariglia con l'usura,\\
Però ch'in questa zezza ricadia,\\
Ch'io ho d'haver trovata clausura,\\
Eglino tutti sul cancel saliro,\\
E si fermaro, ove è la toppa, in giro.
\end{ottave}

\begin{ottave}
\flagverse{73}E gli denti appiccando a quel legname, \\
Come s'in bocca havessero un trapano, \\
Presto presto vi fecero un forame \\
Da porre il fiasco, e vender il trebbiano,\\
Tal ch'in terra cascando ogni serrame\\
Spalanco l'uscio di mia propria mano,\\
E passo dentro, e resto pur confusa,\\
Perch' ancor quivi è un'altra porta chiusa.
\end{ottave}

\begin{ottave}
\flagverse{74}Ma parve giusto come bere un'uovo\\
A i topi il farvi il consueto foro,\\
E dopo questa a un'altra, e poi di nuovo\\
Infino a sette fanno quel lavoro;\\
Quando fra i verdi mirti io mi ritrovo,\\
Che fan corona a una cassa d'oro,\\
Ch'è a pié d'un Tempio, c'è dipinto a graffio\\
E a prima faccia tien quest'epitaffio.
\end{ottave}

\begin{ottave}
\flagverse{75}Cupido Amor, che tanti ha sbolzonato\\
Berzaglio qui si giace della morte,\\
Ei ch'era fuoco, il naso hora ha gelato,\\
Se i cuor legò, prigione è in queste porte.\\
Hallo trafitto, morto, e sotterrato\\
Quella Cicala della sua consorte,\\
Ne sorgerà, se pria colma di pianto\\
Non sarà l'urna, che gli è qui da canto.
\end{ottave}

I Topi suddetti rimunerarono Psiche, perché rodendo fino a sette porte, che
erano in quel Serraglio, fecero cascare i serrami, e Psiche entrata dentro, trovò il
sepolcro d'Amore, e dall'Inscrizione, che in esso era, comprese quello, che le
restava da fare.
\begin{description}
\item[HEBBI la pariglia] Hebbi il contraccambio. È il Latino \textit{Par pars referre}.
Pariglia intendiamo due cose uguali nel giuoco di Carte, o dadi, come due sei,
due assi, due figure, ec, e di tal voce non ci serviamo se non nel giuoco, o nel
caso del presente luogo di render contraccambio sì in bene, come in male. Vedi
sotto C. 6, stan. 69. Io l'ho per voce Spagnuola, ed il Varchi nella stor. lib. 8.
l'usò in un certo modo come straniera dicendo: \textit{Dopo essersi vendicati, ed haver
renduto il contraccambio, o, come si suol dire, la pariglia}.

\item[CON l'usura] Col frutto. Cioè mi contraccambiarono, facendo maggior servizio
  a me, che non havevo io fatto a loro.

\item[ZEZZA] Ultima. E' voce antica hoggi poco usata fuor che nel contado.
Vedi sopra C. 2, stan. 2, Si trova anche \textit{sezza}, \textit{sezzaia}, o \textit{zezzaia}.

\item[RICADIA] Noia, travaglio, avversità, molestia, o simili che vengono dopo
  a un'altro disgusto; da \textit{ricadia}, che è quando uno infermo già quasi sanato, viene
  a riammalarsi, o per lo mal governo, o per altro. Nella storia di Semifonte
  Trattato terzo. \textit{Con li loro misfatti, dando alli Fiorentini non poca ricadia}. Franc.
  Sac. Nov. 98. \textit{Che ricadia è questa di questi porci?}

\item[CANCELLO] Intende il legname, che chiude una porta: ma propriamente
  \textit{cancello} diciamo una chiusura di porta fatta di stecconi, o strisce di legno, o di
  ferro separate l'una dall'altra a guisa di gabbia.

\item[TOPPA] Intendiamo quella piastra di ferro, sopr'alla quale son fabbricati gl'ingegni
  della serratura, detta assolutamente, o senza aggiunta, perché per altro
  Toppa si dice ogni pezzo di panno, legno, quoio, ferro, ec. che s' adatti a
  rotture di cose di sua qualità, ec.

\item[TRAPANO] È uno strumento specie di fucchiello, col quale si forano materiali
  duri come pietre, e metalli, ec. Dal Greco \textit{Trypanon}.

\item[DA porre il fiasco] Coloro che vendono il vino a fiaschi, appiccano un fiasco
  sopr'alla porta della loro casa, come dicemmo sopra C. 1. stan. 76, ed oltre a
  questo hanno per lo più nella porta, o nel muro una finestrella, per la quale danno
  fuora il fiasco, che vendono; a questa finestrella assomiglia il foro fatto da i
  topi; e se bene dice \textit{da vendere il trebbiano} pigliando questa specie di vino per tutte
  le specie di vino, intende esser questo tale sfondato simile a quello, che si fa
  nelle porte per vendere il vino.

\item[SPALANCARE] Aprire largamente, quanto si può.

\item[PARVE come bere un'uovo] Fu cosa facilissima, come è il bere un'uovo: i Greci
  pure dissero in questo proposito \textit{Quo pacto quis ovum sorberet}, e trovasi questa
  frase presso Ateneo.

\item[DIPINTO a graffio] Dipingere a graffio, sgraffio, o graffito, è un'imprimer
figure, ec. con un ferro acuto nell'intonacatura fresca de' muri con detto ferro,
che si chiama graffio, forse dall'antico \textit{graphium}, che era lo stilo di ferro, col
quale scrivevano.

\item[BOLZONARE] o \textit{sbolzonare}. Sacttare, frecciare, da bolzone specie di freccia
  Mattio Franzesi sopra alla boria dice:
  \begin{verse}
    \backspace Di qui Amore accorto balestriere
    Bolzona qualche giovane galante
  \end{verse}

\item[HA il naso gelato] Ha il naso freddo. Pigliando la parte per il tutto, vuol dire,
  che Cupido è freddo, cioè morto.

\item[CICALA] Animale noto; ma qui si dice una, che chiacchierando assai, non
  può ne sa tener segreta cosa alcuna; e degli huomini diciamo \textit{Cicaloni}. Appresso
  i Greci \textit{cicala} non suona male, poiché alle cicale sono da essi rassomigliati in più
  d'un luogo i Poeti per il continovo cantare, che fanno, e questi, e quelle. E
  questo nostro Poeta graziosamente chiamò Musa la cicala sopra C. 1, stan, 2.
\end{description}

\section{Stanza LXXVI --- LXXX}}

\begin{ottave}
\flagverse{76}Non ti vuo dire adesso sin quel caso\\
Mi divennero gli occhi due fontane,\\
E feci come chi s'è rotto il naso,\\
Che versa il sangue, e corre al lavamane;\\
Cors'io a pianger a quel vaso\\
Durando a lagrimar sei settimane,\\
E, per haver quel più voglia di piagnere,\\
Mi diedi pugna sì ch'io m'ebbi a infragnere.
\end{ottave}

\begin{ottave}
\flagverse{77}Quand io veddi ch' egli era poco meno\\
In su ch'all'orlo, ed esser a buon porto,\\
Volli innanzi ch'e fusse affatto pieno,\\
E ch'il marito mio fusse risorto.\\
Lavarmi il viso, e rassettarmi il seno,\\
Acciò sì lorda non m'havesse scorto;\\
Perciò mi parto, e corro, se in quel monte\\
Per avventura fusse qualche fonte.
\end{ottave}

\begin{ottave}
\flagverse{78}In quel ch'io m'allontano com'io dico,\\
Martinazza, che era in Stregheria,\\
Passò di là portata dal nimico,\\
Che non porette star per altra via;\\
E perché sempre fu suo modo antico\\
Di far pertutto a alcun qualche angheria;\\
Lesse il pitaffio, squadro l'urna, e tenne,\\
Che lì fusse da farne una solenne,
\end{ottave}

\begin{ottave}
\flagverse{79}Se qua, dice fra se, Cupido dorme,\\
Vuo risvegliarlo per veder un tratto\\
S'egli è come si dice, e se conforme\\
A quel che dai Pittori vien ritratto\\
Se ben chi lo fa bello, e chi deforme,\\
Basta mi chiarirò com'egli e fatto;\\
Per questo ad empier mettesi quel vaso,\\
A cui poco mancava ad esser raso.
\end{ottave}

\begin{ottave}
\flagverse{80}Con l'animo di pianger vi s'arreca, \\
Ma ponza ponza, lagrima non getta,\\
Si prova a far cipiglio, e bocca bieca, \\
Ne men queta è però buona ricetta; \\
Al fin si pone a un fumo, che l'accieca\\
Sì che per forza a pianger è costretta,\\
Onde la pila in mezzo quarto d'ora\\
Restò colma, e Cupido scappò fuora.
\end{ottave}

In ordine al Cartello havendo Psiche con le sue lagrime quasi piena l'urna,
andò a lavarsi il viso, e raccomodarsi la testa; Intanto Martinazza arrivò al
sepolcro, e con le lagrime sue finì d'empier l'urna, e Cupido usci dal Sepolcro.

\begin{description}
\item[NON ti vo dire] Questo termine serve per esprimere. \textit{Da te puoi ben sapere
  questa cosa meglio di quello che io sapessi dirti}; o vero \textit{so che tu hai da per te tanto spirito da
  giudicar come io rimanessi, senza che io te lo dica}, Suona lo stesso che \textit{pensa tu}, \textit{giudica tu},
  \textit{dica tu}, \textit{tu puoi sapere}, ec. Vedi sopra in questo C, stan. 41. stan. 52, e stan. 59.
  Simile è quello: Non domandar, se Durlindana taglia.

\item[LAVAMANE] È uno strumento di legno, o d' altro, che con tre piedi forma
  come una piramide in triangolo equilatere, e sopra esso si pola la catinella, o
  altro vaso per lavarsi le mani.

\item[ERA poco meno che all'orlo] Era quasi pieno. L' acqua arrivava quasi all'estremità del vaso:
  che questo vuol dire \textit{orlo}, che viene dal latino \textit{ora}, che significa
  l'estremità di qualsivoglia cosa.

\item[LORDO] Schifo, intriso. Dal latino \textit{Luridus}.

\item[VA in stregheria] Dicemmmo sopra C.2. stan. 11, donde derivi tal nome di Strega,
  ed al C. 3. stan. 69, dicemmo esser fama, che tali Streghe vadano la notte a
  cavallo in sul caprone a Benevento al congresso de' diavoli. E questo: intende dicendo
  \textit{Andare in Stregheria portata dal nimico}, che vuol dire il Demonio, in forma
  di Caprone. Che queste donnicciuolucce credute Streghe vadano in sul Caprone
  a Benevento è opinione vulgata, e molti di cervello debole l'hanno per
  indubitata, e le medesime Streghe se lo credono, perché il Diavolo con illusioni
  fa loro apparir per vera questa falsità; Ma la graziosa sagacita d'un Superiore
  ne fece chiarire tutti i dubbj in questa forma.

  Fu condotta alle carceri una di queste tali inquisita di maliarda, ed il Giudice
  dopo molte esame havendo trovato, che veramente costei era una donna, che si
  credeva far malie, stregar bambini, ed altre scioccherie, ma in effetto non v'era
  cosa di conclusione, o di proposito, risolvette di gastigarla per la mala intenzione,
  ed in tanto soddisfare alla propria curiosità. Fattala però venire a sé l'interrogò
  se andava ancor' ella a Benevento, rispose che sì, onde egli le disse: Io vi
  voglio perdonare se voi andrete questa notte a Benevento, e domattina mi racconterete
  quanto vi sarà successo. Bisogna che mi diate la libertà, replicò la donna,
  acciò io possa nella mia stanza fare i miei scongiuri, e le mie unzioni; il
  Giudice gliela concedette con questo che voleva dargli da cena insieme con un
  compagno: il che accettò la donna, bastandole esser fuori di quel luogo, dove il
  Diavolo non poteva capitare. Andata dunque a casa cenò con il detto compagno,
  che era un giovanotto ortolano, e con un'altro giovane, che la donna si
  contentò che egli conducesse, e bevuto abbondantemente come era il suo costume
  in tali sere di viaggio, lasciati i commensali a tavola sen' entrò nella solita
  camera, e quivi spogliatasi, senza serrar la porta, ne le finestre della medesima
  camera (che tale è l'ordine del Diavolo) s'unse con più forte di bitumi puzzolenti,
  e postati a diacere in sul letto, subito s'addormentò; I due compagni, così
  instruiti, entrarono in camera, e legarono la donna per le braccia, e gambe alle
  quattro cantonate del letto, e benissimo la strinsero con funi, e si messero a chiamarla
  con altissime voci, ma come fusse morta non faceva moto, ne dava segno
  alcuno di sentire, onde i detti cominciarono a martirizzarla bruciandole hora
  una poppa, hora una coscia, e finalmente così l'impiagarono in diverse parti del
  corpo, e le arsero fino alla cotenna la metà della chioma; Cominciando a venire
  il giorno la donna con sospiri, e lamenti diede segno di svegliarsi, onde i detti
  le sciolsero i legami, ed uno di loro andò per una seggetta, e l'altro la rivestì
  tutta sbalordita e dal sonno, e molto più da i martorj; giunta la seggetta, in essa
  la portarono al Giudice, il quale l'interrogò se era stata a Benevento, ed ella
  rispose che sì, ma che haveva patito gran travagli, ed era stata bastonata con
  verghe di ferro infuocate, e strascinata, e legata per le braccia, e per le gambe,
  era stata riportata dal suo Caprone, che nel lasciarla le haveva abbruciate con la
  granata mezze le trecce, e questo perché ella haveva ubbidito al Giudice, e che
  si sentiva morire dal gran dolore delle piaghe. Il Giudice ordinò, che subito fusse
  medicata, come seguì; ed intanto disse alla donna: Io v' ho fatto scottare, e
  battere per gastigo del tuo errore, e perché tu conosca, che non altrimenti a
  Benevento, ma in casa tua hai ricevuto questi travagli, e ti risolva a lasciar queste
  false credenze; che se lo farai, io ti perdonerò. Da questo bel modo di gastigare
  cavò l'arguto Giudice quella verità, che appresso lui era certissima.

\item[NON potette star per altra via] Non potette essere in altra maniera, perché
  Martinazza non havrebbe mai potuto salire su quel monte; se non ve l'havesse
  portata il Diavolo.

\item[ANGERIA] Violenza, dispiacere, sopruso. Viene dal Latino greco \textit{Angaria},
  che suona \textit{coactio}. Varchi Stor. Fior. lib. 2. \textit{E perché i Fiorentini nuovi tributi,
    ed angherie ritrovare havevano}.

\item[SQUADRÒ] Guardò diligentemente, ed accuratamente. Vedi sopra C. 1. stan. 32.

\item[FARNE una solenne] Fare un'angheria delle maggiori, che si possano fare.
  La voce \textit{solenne} è da noi spesso usata in vece di grandissimo, ed è tolta da i riti
  della Chiesa, che si dicono feste solenni, le maggiori feste, che seguono nell'Anno.
  Così \textit{hieros}, cioè sagro, presso i Greci, e \textit{sacer} presso i Latini vale talvolta
  grandissimo, \textit{Anchora sacra}, \textit{Morbus sacer}, è lo stesso, che \textit{Anchora maior},
  \textit{Morbus maior}. E Virgilio quando disse; \textit{Auri sacra fames}, per avventura intese
  grandissima.

\item[VIEN ritratto] Vien dipinto. Se il dipinto è come il vero. Dice: \textit{chi lo fa
  bello, e chi deforme}, per intendere, che i pittori da pochi soldi lo dipingono male.

\item[AD esser raso] Ad esser pieno affatto. Viene dal misurare il grano con lo
  staio, che per dare, e ricevere il dovere s'empie lo staio, e quando è pieno si
  striscia sopra con un bastone, e si fa cascare quel grano, che è sopr'alla bocca
  dello staio, e questo si dice \textit{radere}, e tal bastone si dice \textit{rasiera}, e lo staio così pieno
  si dice \textit{raso}, cioè pieno per appunto fino all'orlo della bocca.

\item[VI s'arreca] Vi s'accomoda con positura del corpo; sopra in questo C. stan.
  42, s'arrecò con l'animo.

\item[PONZA ponza] Ponzare è una forza che si fa in se medesimo, ritenendo il
  fiato, quasi riducendo tutto lo sforzo in un punto, come fanno le donne, quando
  mandano fuora il parto. Questo \textit{ponzare} è corrotto dal buon Toscano,
  \textit{pontare}, come si vede dal Petrarca, che dice:
  \begin{verse}
    Io riconobbi a guisa huom che ponta
  \end{verse}
  L'Espositore dice \textit{idest che spinga}. Vedi l'Alunno fabr, num, 609. la voce \textit{pontare}.
  Ed il termine \textit{ponza ponza} serve per esprimere uno, che assai lavorando, conchiuda
  poco; che si dice anche \textit{tresca tresca}. \textit{Ticche ticche}, \textit{Ienneinne}, che vedremo
  sotto C. 5, stan. 51, \textit{In vanum laborare}. Se bene qui si può intendere, che Martinazza
  moltissimo ponzasse.
\item[CIPIGLIO] È uno increspamento della fronte fatto in giù alla volta degli
  occhi, ed è una guardatura d'uno adirato, o d'uno estremamente superbo, quasi
  \textit{piglio del ciglio}. Gli antichi, come Dante dissero \textit{Piglio} la guardatura.

\item[BOCCA bieca] Bocca storta. La voce \textit{bieco} Latino \textit{obliquus}, è usata assai da i
  Legnaioli per intendere l'inegualità d'un legno, e dicono \textit{sbiecare} quando lo
  pareggiano, e fanno uguale.

\item[PILA] È proprio quel sodo, sopra il quale posano gli archi de i ponti. Ma
  si piglia anche per quel vaso grande di pietra, nel quale si mette l'acqua per abbeverare
  le bestie, o per altro uso simile; in somma per pila intendiamo ogni vaso
  di pietra che tenga, o riceva acqua.
\end{description}

\section{STANZA LXXXI, STANZA LXXXII}

\begin{ottave}
\flagverse{81}Quand'ella verso lui volte le ciglia,\\
E vedde quella sua bella, figura \\
Disposta, e graziosa a meraviglia, \\
Che più non si può far n' una pittura, \\
Gli s'avventa di subito, e lo piglia, \\
E, senza ricercar della cattura, \\
Da' suoi staffieri tenebrosi, e bui\\
Portar se ne fa via con esso lui.
\end{ottave}

\begin{ottave}
\flagverse{82}Fermossi a Malmantile, e per marito\\
Lo volle, e già le nozze han celebrate.\\
Come sai tu (dirai) tutto il seguito?\\
Lo so, che me lo dissero le Fate,\\
Quelle, che mi donar quel che hai sentito\\
Ch'in due Aquile essendo trasformate,\\
Perché lassù facea degli sbavigli,\\
M'han trasportata qua ne i loro artigli.
\end{ottave}

Martinazza porta via Cupido, ed in Malmantile lo piglia per marito; Così
havevano raccontato a Psiche le Fate, le quali trasformate in due Aquile l'havevano
portata via da quel monte co' loro artigli. E qui finisce il quarto Cantare.

\begin{description}
\item[CATTURA] Si dice quella somma di danaro, che si dà a i birri quand'hanno
  pigliato uno; e si dice anche cattura quella polizza, e ordine che si dà alli
  sbirri perché piglino uno. Di qui il Poeta cava lo scherzo dicendo, che Martinazza
  piglio Cupido senz'haver l'ordine della cattura, e lo portò via, e non aspettò,
  che le fusse dato il denaro della cattura, che havea fatta di lui.

  \item[FACEA degli sbavigli] Si dovrebbe dire \textit{sbadigi}. Dan. Inf. C. 45.
  \begin{verse}
    Anzi co' pié fermati sbadigliava
    Pur come sonno, o febbre l'assalisse
  \end{verse}
  Ma hoggi si dice \textit{sbavigli}, e \textit{sbavigliare}; che un'aprimento di bocca, ripigliando
  il fiato, e poi mandandolo fuora, il che per lo più è cagionato dal sonno, da
  pensieri, da tristizia o malinconia, o da altro rincrescimento, perché lo sbaviglio
  nasce da vapori grossi, e frigidi generati nello stomaco da ozio, e da pigrizia,
  i quali salgono alla bocca per la via del cibo, e spargonfi per le mascella, e
  la natura bramosa di mandargli fuora, alita con aperta bocca, il che da i Latini
  si dice \textit{oscitare}. \textit{Fare degli sbavigli}. Significa non haver roba da mangiare, ne
  altro da recrearsi al bisogno, ed habbiamo una rima, che dice:
  \begin{verse}
    \backspace Chi sbaviglia non può mentire
    O egli ha sete, o egli ha fame, o e' vuol dormire.
  \end{verse}
  Sicché la povera Psiche stando in quel luogo, dove non era da mangiare, ne da
  bere, haveva occasione di sbavigliare non potendo cavarsi la fame, ne la sete.

\item[ARTIGLI] Dal Latino \textit{articuli}. Zampe degli uccelli, o altri animali ditati.
  Qui intende le mani delle Fate, le quali convertite in Aquile, havevano artigli in
  vece di mani. Se bene diciamo talvolta artigli le mani dell'huomo. Bocc. Canz.
  alla Nov. 6.
  \begin{verse}
    \backspace Amor, s'io posse uscir de' tuoi artigli,
    A pena creder posso,
    Che alcun altro uncin mai più mi pigli.
  \end{verse}
\end{description}
\section*{FINE DEL QVARTO CANTARE.}

\chapter{Quinto Cantare}

\begin{argomento}
  Vuol con gl'incanti dar la Maga aita
  In Malmantile al popolo assediato,
  Ma dagli spirti è così mal servita,
  Che tra i nimici e il suo saper beffato;
  Vien Calagrillo, e a duellar l'invita,
  E l'invito è da lei tosto accettato.
  Il Fendefi, e altri due com'è usanza,
  Sparir di Piaccianteo fan la pietanza.
\end{argomento}

\section{STANZA I \& II}

\begin{ottave}
\flagverse{1}E si trova talun, ch'è sì capone,\\
Ch'ad una cosa, che si tocca, e vede,\\
E che di più gli afferman le persone,\\
Vuol' esser ostinato, e non la crede.\\
Un'altro è poi sì tondo, e sì minchione,\\
Che se le beve tutte, e a ognun dà fede;\\
E ci son' huomin tanto babbuassi,\\
Che crederebbon, c'un'asin volassi.
\end{ottave}

\begin{ottave}
\flagverse{2}Gli estremi non fur mai degni di lode:\\
Ci vuol la via di mezzo, e chi ha cervello,\\
Se vere, o false novitadi egli ode\\
A crederle al compagno va bel bello:\\
Le crede, s'elle son fondate, e sode,\\
Ma s'elle star non possone a martello,\\
Non le gabella mica di leggieri,\\
Come fa il Duca a certi messaggieri.
\end{ottave}

Volendo il Poeta nel presente Cantare narrar l'inavvertenza de' due Diavoli
mandati da Martinazza per far diloggiar Baldone, e lo scambiamento delle palle,
per lo quale apparvero a Baldone diversamente da quello, che dovevano, il
che fu causa, che egli non prestò fede alle loro parole, s'introduce col dire:
Che l'esser' huomo testardo, e capone non è bene, ma che non è però anche
bene l'esser così credulo, che si dia fede a tutto quello, che si sente dire, onde è
degno di lode colui, che sa pigliare la via del mezzo, dando credito a quelle cose,
le quali egli conosce haver fondamento di verità, come fece Baldone alli due
messaggieri di Martinazza.

\begin{description}
\item[CAPONE] Testardo. Huomo ostinato nella sua opinione. In Latino pure
  potrebbersi chiamare questi cali \textit{Capitones}; da noi altrimenti \textit{Caparbi}.
\item[TONDO] Huomo groffolano; semplice, facile, credulo, ec, Epiteto, che si
  dà a i panni lani, che si dice \textit{tondi}, quando sono grossi, contrario i \textit{fini}. E così
  diciamo \textit{huomo fine}, che è il contrario d'\textit{huomo tondo}, Lasca Novella 2. \textit{Mariotto
  fu huomo di sì grossa pasta, e così tondo di pelo, che in quattr'anni di squola non
  potette mai imparare l'Abbiccì}. Vedi sotto C, 6, stan. 80.
\item[MINCHIONE] Semplice. Vedi sopra C, 4. stan. 15.
\item[SE le beve tutte] Crede tutto quello, ch'ei sente dir.
\item[BABBVASSI] Ignoranti, huomini di cervello grosso. Vedi sotto C.6. st. 80.
\item[CREDEREBBON ch'un asin volasse] Per esprimer'uno, che crederebbe etiam
  le cose impossibili a credersi, ci serviamo di questo detto. In Empoli in un dì
  solenne dell'anno, fanno una antica festa, o rappresentazione di far volare l'Asino:
  Quindi è, che nel Capitolo in lode dell'Asino, che va colle rime del Berni, si
  dice:
  \begin{verse}
    \backspace Ben mostran gli Empolesi aver cervello,
    Quanto conviensi ad ogn' huomo da bene;
    Che l'Asin diventar fanno un uccello.
  \end{verse}
\item[NON può stare a martello] Non corrisponde al vero. Tratto dat Cimento
  dell'argento, che quando non stà, cioè non resiste al Martello, non è vero argento.
  I Latini pure direbbero in questo proposito \textit{non est aurum igni probatum}.
\item[NON le gabella] Non le passa per vere. Non le crede: dal \textit{Passaggio}, ovvero
  \textit{Gabella} delle porte, o de' passi; onde il verbo \textit{Gabellare}, per ammettere, e approvare
  una cosa per buona, e per vera. \textit{Mica} particella riempitiva a maggior
  enfafi della negativa, come già, e mai, ec, \textit{Io non vuò mai, che si dica}. \textit{Io non vuò già,
  che si dica}, \textit{Io non vuò mica, che si dica}. Vedi sopra C. 4, stan. 69.
\end{description}

\section{STANZA III --- VII.}

\begin{ottave}
\flagverse{3}Ma, perché chi m'ascolta intenda bene; \\
Tornar' a Martinazza mi bisogna, \\
La qual dianzi lasciai, se vi sovviene, \\
Ch'in sul Caprinfernal, pigra carogna, \\
Quel popolaccio ha aggiunto, e lo ritiene \\
Dal fuggir via con tanta sua vergogna, \\
Perché quando per lei la raffigura,\\
Rallenta il corso, e piscia la paura.
\end{ottave}

\begin{ottave}
\flagverse{4}E quivi con l'affanno in sulla pena\\
Tutto lamenti, con doglienze, e strida,\\
Tremando forte, come una vermena,\\
La prega, perch'in lei molto confida,\\
E perch'addosso giunta gli è la piena,\\
E lì tra lor non è capo, ne guida,\\
A far in mo, se si può far di manco,\\
Ch'ei non s'abbia a cacciar la spada al fianco.
\end{ottave}

\begin{ottave}
\flagverse{5}Ella risponde allor, ch'è di parere,\\
Ch'il pigliar l'arme faccia di mestiero,\\
Che per la Patria par che sia dovere\\
Il farsi bravo, e diventar guerriero,\\
Se ben fra tanto vuol un po vedere\\
S'ella con Gambastorta, e Baconero\\
Trovar potesse il modo, che costoro\\
Vadan a far il bravo a casa loro.
\end{ottave}

\begin{ottave}
\flagverse{6}Ciò dette balxain casa, € cold dentro
Per aenerfi dispogliafi in capelti,
E cacciatasi addosso quant' unguento
Haveva ne! [uci feridi alberetli,
Vin gran circolo fa nel pavimento,
E con un uafo in man,scriteiye Cartelli,
Borbortando parole tutravia,
Che ne men si direbbono in Turchia,
\end{ottave}

\begin{ottave}
\flagverse{7}Fa un salto a pié pari in mezzo al segno, \\
E quivi havendo all ordine ogni cosa,\\
Per mandar ad effetto il suo disegno \\
Grida così con voce strepitosa:\\
O colaggiù dal sotteraneo Regno\\
Cornuti mostri, e gente spaventosa,\\
Filigginosi abitator di Dite,\\
Badate a me; le mie parole udite.
\end{ottave}

Torna adesso a Martinazza, la quale sopra nel C.~3.\ stan.~76, lasciò, che montata
a cavalcioni in sul Caprone, haveva arrivato quel popolo, che fuggiva per
la paura, ma riconosciutala, la prega a dar' aiuto a Malmantile, e far, che essi
non habbiano a combatter, se si può. Ella dice, che stima necessario il combattere,
ma che intanto vuol vedere, se gli riesce cacciar via il nimico per altre,
strade, e vassene in casa a fare i suoi incantesimi a questo effetto.

\begin{description}
\item[CAPRINFERNALE] Due dizioni come ridotte in una significante Caprone d'Inferno:
Ed intende quel Diavolo in forma di Capra sopr' al quale era cavalcata
Martinazza, e sopra il quale si favoleggia, che vadano le Streghe a Benevento,
come s'è notato sopra C.~3. stan. 69.

\item[CAROGNA] Vuol dire Cadavero d'huomo, o di bestia. Cavalcanti stor.
fior. lib.3. cap. 2. dice: \textit{Se volete veder quanto la lor perfidia si distese contro al sangue
de' nostri maggiori, cercate i Conventi de' Frati, e troveretegli pieni di corpora, e di
carogne de i vostri antichi}. Da questo dire del Cavalcanti m'induco a credere, che
la voce \textit{Carogna} significhi cadavero d'huomo ammazzato con ferite, e straziato,
e che però ci serviamo di tal voce per intendere una bestia piena di mascalcie, e
guidaleschi, e stimo con Pier Vettori nelle Varie lezioni, che venga da \textit{Charonia},
che intendevano già le voragini del fuoco, che in diverse parti del mondo si trovano,
e le dicevano \textit{Charonia} da Caronte, perché la superstiziosa Gentilità stimava,
che tali vorapini fussero bocche d'Inferno, e che per quelle s'andasse da
Caronte; E perché hanno sempre puzzo orrendo, che procede da acque sulfuree,
da questo cominciarono a chiamare \textit{Charonia} tutte quelle cose, che grandemente
putivano; E noi seguitando gli antichi diciamo \textit{Carogna} a tutte le cose,
che putono, come fanno le bestiaccie guidalescose, e le morte. Dicigmo \textit{Carogna}
anche un'huomo, che habbia cattivi sentimenti, perché un'azione mal fatta si
suol dire \textit{Questa pute; o non ha buono odore}.

Gli Ateniesi chiamavano \textit{Charonia} quella porta del Pretorio, o Palagio del Potestà,
per la quale uscivano coloro, che erano condotti al supplizio, secondo
che riferisce Giulio Polluce nell'Onomastico, e Alex.\ ab Alex.\ lib.4.\ c.16.\ e Cel.\
Rod.\ lect.\ antiq.\ lib.4.\ c.8, e lib.17.\ c.9. Tolta la derivazione di tal voce pure
da Caronte, che conduce l'anime al supplizio, passandole in barca, e si dice \textit{mandar'
uno a Caronte} per intender mandar uno alla morte.

\item[PISCIA la paura] Ripiglia animo. Non ha più paura. Dopo che i cani si
sono azzuffati sogliono pisciare; e comunemente dalla plebe si dice che pisciano
la paura; e da questo diciamo \textit{pisciar la paura}, quand'uno spaventato, o impaurito,
perde quel timore.

\item[L'AFFANNO in sulla pena] Era aggiunto alla pena, che hebbe per la paura
  l'affanno cagionato dal correre. Vedi la voce Affanno sopra C.4. stan.\ 69.

\item[VERMENA] Un sottile, e giovane ramo d'una pianta si dice Vermena dal
  Latino \textit{Vimen}. Quel passo di Vegezio; de re militari lib.\ 1.\ cap.\ 11. \textit{Quemadmodum
    ad scuta viminea, vel ad palos antiqui exercebant tyrones}: L'antico volgarizzatore
  traduce così. \textit{Come a scudi fatti di vermene, o pali, si provavano i Cavalieri}.

\item[GLI giunta addosso la piena] Sono accadute loro tutte le maggiori disgrazie, e
  piena è presa nel senso detto sopra C. 1. stan. 84.

\item[A FAR in mo che non s'habbia a metter la spada al fianco] Far in modo che il
  negozio s'aggiusti, senz'havere ad adoprare le armi, che si dice \textit{Aggiustarla
    la spada nel fodero}.

\item[SE si può far di manco] Se la necessità non forzi a fare in questa maniera.

\item[GAMBASTORTA, e Baconero] Nomi di Diavoli inventati qui dal  Poeta,
  nello stesso modo, che inventati furono i nomi di \textit{Barbariccia},
  e \textit{Farfarello}, e simili.

\item[BALZA in casa] Va velocemente in casa. \textit{Balzare} propriamente si dice quel
  saltare, che fa la palla, o pallone perquotendo in terra, Vedi sopra C.2. stan.15.

\item[SPOGLiASI in capelli] Si spoglia ignuda, e scioglie le trecce de i capelli, così
  vuol intender il Poeta, se bene si serve del detto \textit{spogliarsi in capelli}, che significa
  adoperare ogni suo sapere, e tutta l'applicazione per fare una tal cosa; per intendere
  ancora, che Martinazza s'era tutta applicata a far, che Baldone per
  via d'incanto diloggi da Malmantile,

\item[CACCIANDOSI addosso] Mettendosi addosso, E se bene il verbo \textit{cacciare} vuol
  dir intromettere con violenza, noi lo pigliamo in senso di mettere, come si vede
  nell'Ottava antecedente \textit{cacciar la spada} per metter la spada.

\item[ALBERELLI] Vasi di terra, o di vetro, entro a' quali si conservano
  Unguenti, e cose simili; e son forse quei vasi, che i Latini chiamano \textit{alveoli}, e
  pigliano il nome da questi.

\item[BORBOTTARE] È un certo parlar fra i denti poco inteso da chi l'ascolta,
  che diciamo anche \textit{brontolare}, E' il Latino \textit{submurmurare}. \textit{Borboryttein} appresso i
  Greci è il \textit{romoreggiare}, o \textit{mormorare che fanno le budella}: Verbi formati dal suono
  stesso naturale.

\item[A PIÉ pari] Cioè a piedi giunti insieme. Questa voce pari, che per altro
  vuol dire \textit{ugualità di numero}, ed il suo contrario è \textit{dispari} (che diciamo \textit{caffo}) che
  i Latini dicono \textit{par, \& impar}, serve ancora per denotare ugualità di misura di un
  corpo, come qui, che s'intende, che un piede non era ne più innanzi, ne più
  indietro dell'altro. Si dice \textit{esser pari} quando uno s'è vendicato con un'altro, o
  ha pagato tutto quello che doveva, E ancora : esser pari e pagati. \textit{Andar pari},
  quando non si pende per nessun verso. \textit{Strada pari} per strada spianata. In somma
  l'adopriamo in tutte quelle cose, dove entri ugualità.

\item[FILIGGINOSI] Affumicati. Tinti da fumo, come sono i cammini, che son
  neri per la filiggine, che è composta di fumo, e d'umido. Lat. \textit{fuliginosi}.

\item[BADATE a me] Attendete a me; Osservate le mie parole, e state attenti a quel c'io dico.
\end{description}
\section{Stanza VIII \& IX}
\begin{ottave}
\flagverse{8}Vi prego, vi scongiuro, e vi comando\\
Per la forza, e virtù ai questi incanti,\\
Per quest'acqua, che a gocce in terra spando\\
Dagli occhi distillata degli amanti,\\
per questa carta ov'è stampato il bando\\
Di questa porcheria de' guardanfanti,\\
Che di portar le donne han per costume,\\
Ricettacol di pulci, e sudiciume.
\end{ottave}

\begin{ottave}
\flagverse{9}Per gl'imbrogli vi chiamo, e l'invenzioni,\\
Che ritrova il Legista, ed il Notaio,\\
Quando per pelar meglio i buon pippioni\\
Gli aggira, che ne anco un'arcolaio;\\
Horsu, pezzi di sacchi di carboni,\\
Per quei ladri del Sarto, e del Mugnaio,\\
Che ti voglion rubar a tuo dispetto,\\
Uscite fuor, venite al mio cospetto.
\end{ottave}

Martinazza con diversi scongiuri chiama gli Spiriti infernali, per servirsene
a far diloggiar Baldone da Malmantile: E l'Autore mostra il disprezzo, che egli
fa degl'incantefimi, facendo che Martinazza costringa i demonj con le cose ridicole,
che egli mette in queste due Ottave.

\begin{description}
\item[SCONGIVRARE] Questo verbo t da noi usato per inteddere Esorcizzare, cioè
  costringere il Diavolo per via di giuramenti di formule sacre dette per questo
  Esorcismi, cioè scongiuri; e comunemente è preso in questo senso, ed anche più
  largamente si tira, come qui, alla maniera d'invocare gli spiriti, usata da'Maghi,
  se bene il suo proprio significato è domandare, o chiedere con grande ardenza,
  ed è in augumento del verbo pregare dicendosi. vi prego, vi supplico, vi scongiuro.
  Latino \textit{obsecro}, \textit{obtestor}.

\item[PORCHERIA] Si dice non solamente un'atto sporco, ed illecito, ma ancora
  una materia schifa, sporca, e brutta, o mal fatta. Come per esempio: \textit{Il tale
  fece un'orazione, che riuscì una bella porcheria}, \textit{La vostra mercanzia non hebbe esito,
    perché fu stimata una porcheria}: \textit{I Libri di quel Mercante furono abbruciati, perché
  eran pieni di partite false, e d'altre porcherie}. Varchi nelle storie Fiorentine dice:
  \textit{Era appunto sparsa in Firenze l'usanza d'andar in zazzera, e mantello, che era una
    bella porcheria}. « Questa voce \textit{Porcheria} significante disprezzo potrebbe venire dal
  Latino \textit{porcaria}, che vuol dir l'utero delle Vacche, o delle Troie, dopo che hanno
  partorito, o per dirla colle parole di Plinio lib.11. Cap. 37. \textit{Vulvam partu edito},
  e tali vulve, particolarmente quando non avevano condotto il parto, ma si erano
  sconciate, dagli antichi Romani erano mangiate per una cosa singulare, dove
  la \textit{Porcaria} non la mangiavano tanto voientieri, forse per esser cosa più schifa. Era
  chiamata porcaria in un certo modo per disprezzo, e così ha portato a
  nol il ignificato, che ritiene di disprezzo, ed abbominazione. Ma la più semplice
  origine è da porco animale immondo; e così detta \textit{porcheria}, cosa da porci,
  come furfanteria, cosa da furfanti, e simili.

\item[GVARDANFANTE] È uno strumento composto di cerchi di filo di ferro in
  tondo; il quale portano le donne Spagnuole, e circonda loro la cintura sotto le
  vesti, le quali fa gonfiare: E lo dicono \textit{guardanfante}, perché egli difende dalle
  percosse l'infante, cioè la creatura, che hanno le donne pregne dentro all'utero.
  Perché questa foggia di vestire, che havevano cominciata ad usare le donne di
  Firenze, conosciuta presto per spropositatamente dispendiosa, e scomoda, s'andava
  a poco a poco disusando, il Poeta in questo Incantesimo di Martinazza pone
  il Bando, cioè l'esilio, e proibizione di tale usanza.

\item[PIPPIONI] Piccioni. S'intende gente semplice, e corriva, come appunto
  sono i pippioni, \textit{colambarum pulli}, colombi giovani. \textit{E pelare un pippione} vuol dire
  Cavar danari di mano al corrivo.

\item[ARCOLAIO] Strumento, sopr' al quale s'adattano le matasse, d'accia o
  d'altra materia per incannarle, o aggomitolarle col girare, il che è assai veloce,
  ed è un moto perpetuo, e però dice \textit{aggira che ne anche un'arcolaio} intendendo aggira
  bene, ed assai:  ed aggirare in questo luogo vuol dire ingannare; donde \textit{aggiratore},
  ingannatore, Così \textit{Bindolo}, si prende per huomo aggiratore; e \textit{Abbindolare}
  per girare, cioè non si rinvenire col cervello. L. \textit{delirare}, o pure per Aggirare,
  Ingannare; Latino \textit{circumvenire}.
\end{description}
\section{Stanza X. --- XII.}

\begin{ottave}
\flagverse{10}Tutto l'Inferno a così gran parole\\
Vien sibilando, e intorno le saltella,\\
Come dall'alba al tramontar del sole,\\
Fa quel, ch'è morfo dalla Tarantella,\\
Domandale Pluton quel ch'ella vuole,\\
Che stridendo ogni dì lo dicervella,\\
E lui, c'hor mai ha dato nelle vecchie,\\
Fa ire in giù, e in su come le secchie.
\end{ottave}

\begin{ottave}
\flagverse{11}Ed a far ch'ei si pigli quella stracca \\
Senza cagion, gli par ch'ell'abbia il torto, \\
Perché dalla profonda sua baracca \\
A Malmantil non è la via dell'orto. \\
Corpo! (dic'ella, ed al Celon l'attacca) \\
A venir infin qui tu sarai morto!\\
Ma senti il mio Pluton, non t'adirare, \\
Che venir non t' ho fatto sine quare, \\
\end{ottave}

\begin{ottave}
\flagverse{12}Ma perché tu mi voglia far piacere\\
Di darmi Baconero, e Gambastorta,\\
Perch'io mi vuò dell'opra lor valere\\
In cosa che mi preme, e che m'importa.\\
Plutone allor quei due fa rimanere,\\
E la strada si piglia della porta,\\
Seguito da i sugi sudditi, che tutti\\
Posson fondar la Compagnia de' brutti.
\end{ottave}

Agli scongiuri di Martinazza le comparisce avanti Plutone con molti Diavoli,
ed ella gli chiede Baconero, e Gambastorta. Ei le lascia quivi li detti due demoni,
e con gli altri se ne torna all'Inferno.

\begin{description}
\item[SIBILANDO] Soffiando,filchiando. E' voce Jatina, che ritiene il suo.
10. Verg, En. 11, edrrettis horret fquamis, © fibilat ore. Iptendiamo
mente il filchiare de i serpenti.

SALTELLA. Fa spelfi, e piccoli salti; \& il faltar delle Rane, Vedi
6. stan. 37,

\item[MORSO dalla Tarantella] Per la Calavria, e Puglia dicono si trovi un piccolo
  ragno detto Tarantola, o Tarantella, il quale nato \textit{ex putri} scappa dalle fessure
  della terra in tempo di state. Questo mordendo un'huomo, gli mette addosso
  una infermità specie di rabbia, che lo forza a ballare continovamente dalla
  levata, al tramontar del sole, ne prova quiete, se non quando sente sonare con
  chitarra o con altro strumento simile, un'aria detta perciò la Tarantella, al qual
  suono questo tale attarantato si affatica a ballare tanto, che stracco casca come
  morto; e stato in questo svenimento qualche hora, si rizza, e cessa di ballare, restando
  sano per qualche giorno: E perché in quel paese si trovano molti infettati
  da tal veleno, vi sono anche molti, che fanno il mestiero del sonare, e son
  pagati dall'attarantato. Dicono, che tale infermità duri quanto dura la vita di
  quell'insetto che morsicò l'attarantato, la quale dicono, che non passi tre anni;
  e vi sono però huomini a posta pagati da quei Comuni, i quali vanno cercando
  questi animalucci per ammazzargli per universal benefizio, e ne hanno un tanto
  per tarantola, rassegnandola a un Rettore a ciò deputato. Dicono in oltre, che
  questo tale morsicato provi la detta infermità ogni anno per un mese poco più, o
  poco meno intorno a' quei giorni, ne' quali fu morsicato, che sarà intorno al Sol
  leone, e che se ne trovino di quelli che la provino ogni mese per qualche giorno.
  Si chiama \textit{Tarantola}, o \textit{Tarantella} dalla Città di Taranto, nel cui territorio forse
  più frequentemente si trova. Il Lalli nell'En.\ Tr.\ lib.\ 1, stan.\ 22. dice,
  \begin{verse}
    Enea quantunque bravo anch' ei tremante
    Morso dalla Tarantola parea,
  \end{verse}

\item[LO dicervella] Gl'introna la testa con le strida: lo sbalordisce; lo fa assordare
  con le stride.

\item[HA dato nelle vecchie] E' invecchiato, s' intende uno che si tratti da vecchio; ancor che non lo sia.

\item[SECCHIA] Vaso di rame, col quale si cava l'acqua dai pozzi. Vedi sotto
  C.7. stan. 3. Ed il detto \textit{far come le secchie} senz'altra aggiunta, significa andar in
  giù, e in su, appunto come fanno le secchie infunate nella Carrucola.

\item[BARACCA] Intende abitazione, Che \textit{baracca} vuol propriamente dire quel
  luogo, che s'eleggono i soldati in campagna per loro abitazione, nel quale fanno
  un ricinto, e capannello di frasche, o d' altro,col quale si difendono dal
  sole, e dalle acque. Viene dal verbo barrare, che vuol dir Circondare, o accerchiare.
  Si dice anche \textit{trabacca}, o corrottamente, o pure \textit{eo quod trabibus constructa sit}.

\item[NON è la via dell' orto] Questo dettato significa; la via è lunghissima, e disastrosa,
  perché perr ordinario dall'orto alla casa non è più lungo viaggio, che cavare
  un piede fuori della porta, la quale di casa esce nell' orto, essendo per lo
  più nella Città gli orti appiccati alle case.

\item[CORPO! ed al Celon s'attacca] Vuol dir Corpo del Cielo. Si dice Corpo del
  mondo, Corpo del diavolo, ec, Ma quando uno passa più là bestemmiando le
  Deità, diciamo: \textit{Ei l'attacca al Celone} per intendere, egli entra nel Cielo, cioè
  bastemmia i Numi Celesti; E per render più oscuro questo detto, ci serviamo della
  vace \textit{Celone}, che vuol dir quel panno, che si mette sopr' alla tavola da mensa
  avanti di distendervi sopra la tovaglia.

\item[TU sarai morto] Detto ironico per mostrar la poca stima, che si fa della fatica,
  che habbia durata uno a nostro pro, ed il poco grado, che gli sen' habbia, massime
  quando quel tale ne fa grande ostentazione.

\item[NON sine quare] Voci latine usate nel suo significato; e dicesi: \textit{Non sine quare
  lupus at urbem}, e significa non senza qualche fine, o cagione. Franco Sacch.
  Nov.2. \textit{Gli venne voglia d'andar a trovare il Re Adovardo, e non sine quare, perché
    egli havea sentito molto lodarlo}.

\item[POSSON fondar la Compagnia de' brutti] Sono tutti bruttissimi. Habbiamo in
  Firenze un'Accademia, o Compagnia detta de' Brutti, la quale si raguna ogni
  anno il giorno di Befana (che così si dice il giorno dell'Epifania) ed in un lautissimo,
  e stravagante simposio si crea il Console nuovo per un'anno, e si appella
  il Fondatore, e si fa sempre il più brutto. E di questa intende il nostro
  Poeta.

\end{description}
\section{STANZA XIII --- XVI.}
\begin{ottave}
\flagverse{13}Lascian Plutone, e corron dalla Druda\\
I due spirti, aspettando il suo decreto,\\
Ed ella allor che fa da Cecco suda\\
Per far sì che Baldon dia volta a dreto,\\
Ed anche se si può ch'ei vada a Buda,\\
Gli prega, che le dien qualche segreto\\
Da far Senz'altre guerre, ovver contese,\\
Che quelle genti sfrattino il paese.
\end{ottave}

\begin{ottave}
\flagverse{14}Io ho (diec un di lor) bell'e trovato\\
Un invenzion, che ci verrà ben fatto,\\
Perché il Duca Baldone è innamorato\\
Della Geva di Corte, e ne va matto,\\
Ma la furba lo tiene ammartellato,\\
E a due tavole dar vorrebbe a un tratto,\\
Tenendo il piè in due staffe, amando lui,\\
E parimente il Duca di Montui.
\end{ottave}

\begin{ottave}
\flagverse{15}Però se non finghiam ch' egli le scriva\\
Ch'il suo rivale (adesso ch'egli ha inteso\\
Ch'ei s'è partito) con la gente arriva,\\
Per volergliela su levar di peso,\\
E che se proprio è ver, che per lei viva\\
(Com'ei spesso giurò) d'amore acceso,\\
E se gli è cara lo dimostri, e prenda,\\
Ed armi, e bravi, e corra, e la difenda.
\end{ottave}

\begin{ottave}
\flagverse{16}Vedrai ch'il Duca torna allota allotta\\
Correndo a casa, come un saettone\\
Con quanta ciurma, ch'egli ha qua condotta,\\
Per voler ammazzar bestie, e persone.\\
Hor dunque tu che sei saputa, e dotta,\\
Che non la cedi manco a Cicerone,\\
Scrivi la carta, che tu sai che noi\\
Sian tutti un monte d'asini, e di buoi.
\end{ottave}

I Diavoli trovano l'invenzione di far diloggiar Baldone da Malmantile, e
questa è fargli intendere, che la Geva sua dama è in pericolo d'esser rapita, e
dicono a Martinazza, che scriva la lettera.

\begin{description}
\item[DRVDA] Innamorata, amante, ec, se bene non sempre si piglia in significato
  disonestoso; Qui intende dama di Plutone, che era Martinazza, che, come
  strega, haveva lui per innamorato.

\item[FA da Cecco suda] S' affanna,s' affatica. Scherza con questo nome \textit{Cecco suda},
  perché quand'uno s'affatica, e s'affanna senza proposito, mostrando di far gran
  cose diciamo: \textit{Il tale suda}. Di questa natura era quel Cortigiano descritto dal
  Berni nelle Rime. \textit{Ser Cecco non può star senza la Corte, Ne la Corte può star
    Ser Cecco}.

\item[VADA a Buda] Vada via per non tornar più. Proverbio nato dalla guerra,
  che già fece il Turco contro Lodovico Re d'Ungheria quando acquisto Buda circa
  l'anno 1626., che vi morirono quasi tutti li Cristiani, che vi andarono, ed
  il medesimo Re; E però da quel tempo in qua dicendosi:\textit{Il tale è andato a Buda},
  s'intende è andato via per non ritornar più, o vero è morto, ed ha il medesimo
  senso, e per la medesima cagione; \textit{Il tale e andato a Scio}. \textit{È andato a Patraffo}, scherzo
  sulla Città d'Acaia famosa per il martirio di S. Andrea, come se si dicesse in
  Latino: \textit{ivit Patras}; e sulla frase usata dalla scrittura, sopra quei che muoiono,
  e si seppelliscono, quasi dica; È andato \textit{ad patres suos}.

\item[SFRATTINO, o sbrattino il paese]. Ripuliscano il paese, cioè se ne vadano.

\item[DAR' a due tavole a un tratto] Far due negozzi in uno stesso tempo. Tratto
  dal giuoco di sbaraglino, nel quale con un sol tiro, si dia due, e tre tavole, o
  girelle. Si dice anche: \textit{far' un viaggio, e due servizj}. Vedi sotto C.~6, stan.~7.

\item[TENERE il piede in due staffe] Attendere a due partiti. \textit{Unum eligere, \& alterum
  non dimittere}. Tacito \textit{Diversas spes spectare}.

\item[MONTUI] Villaggio vicino a Firenze. Dovrebbe dirsi Mont'Vghi dalla famiglia
  degli Ughi antichissima Fiorentina. Ricordano Malespini nella Stor. Fiorentina
  cap 32. Il sesto compagno ebbe nome Ugo, questi anche fue nobilissimo
  gentiluomo Romano, e di questo discesono gli Ughi, e per innanzi il poggio,
  che oggi si chiama Montughi, s'è chiamato per loro. Lo stesso conferma Gio,
  Villani lib.~4. cap.~11.

\item[ALLOTTA allotta] Allora allora; Subito subito. \textit{Nulla interposita morula}.

\item[SAETTONE] Specie di Serpe, detto così, perché forse vada veloce come
  una saetta, e credo sia il \textit{coluber} dei Latini.

\item[CIURMA] Propriamente vuol dire Remiganti di galera: Ma qui è presa per
  soldatesca, come si trova anche presa in più Storie Fiorentine antiche, e sopra
  C.3. stan.76., e sotto C. 11. stan. 76. dal Latino \textit{turma}, se bene propriamente si
diceva di soldati a cavallo.

\item[VUOL ammazzar bestie, e persone] Vuol disertare il paese. Quando vogliamo
  esprimer uno, che vanti di voler far gran bravure, e non lo giudichiamo atto a
  farne veruna, diciamo \textit{Vuol ammazzar bestie, e persone}. Ed in tal senso di derisione
  è preso nel presente luogo. Il Berni nelle rime congiunse queste due voci curiosamente
  allor che disse: \textit{Con un mondo di bestie, e di persone}.

\item[SEI saputa] Sei dotta; sei scientifica. Donna \textit{saputa}, \textit{sacciuta}, \textit{saccente} vuol
  dir di Una donna, che in tutte le cose vuol far da maestra. Colla stessa figura di
  \textit{saputo} per saccente, dicesi \textit{Avvertito}, \textit{Accorto}, \textit{Avvisato}, e dagli antichi \textit{Sentito}
  per huomo  che avverta, e che s'accorga delle cose, e che stia sull'avviso, e
  simili. Il participio passivo in forza di attivo.

\item[SIAMO una mana d'asini, e di buoi] Siamo tanti ignoranti. Per lo più a queste
  due bestie ed al Castrone assomigliamo coloro, che non hanno scienza alcuna.
  Se bene l'Autore sapeva, che il Demonio possiede tutte le scienze, che così
  suona il suo Greco nome \textit{Daemon}, cioè sapiente; e noi d'uno che sappia eccellentemente
  qualche cosa dichiamo; Egli è un Demonio; nondimeno ha voluto,
  che questi due Diavoli si dichiarino ignoranti, acciò che si creda più facilmente
  l'errore, che fecero di scambiare le palle, come vedremo.
\end{description}
\section{Stanza XVII --- XXI.}
\begin{ottave}
\flagverse{17}Non ti dò contro, rispond'ella, a questo, \\
Ed h gusto chee voi vi conoschiate: \\
Hor sù, dice il Demonio, scrivi presto\\
Due parole in tal genere aggiustate: \\
sì, dic'ella; ma vedi, io mi protesto, \\
Ch'io non portai mai lettere, o imbasciate. \\
Scrivi, soggiunge quei, che quanto al porta \\
Eccomi lesto qui con Gambastorta.
\end{ottave}

\begin{ottave}
\flagverse{18}E per dar al negozio più colore,\\
In forma voglio ir' io d' una comare\\
Della sua Geva detta Monafiore\\
Confidente del Duca in ogni affare;\\
Gambastorta verrà da servitore,\\
Che mostri di venirmi a accompagnare,\\
E già per questo ho fatte far di cera\\
Due palle, una ch'è bianca, e l'altra nera.
\end{ottave}

\begin{ottave}
\flagverse{19}Quand'un tien questa nera in una branca,\\
Di subito d'un'huom prende figura; \\
E s'ei vi chiude quell'altra ch'è bianca,\\
In femmina si muta, e trasfigura. \\
Sì che riguarda ben s'altro ci manca, \\
E distendi mai più questa scrittura; \\
Ch'il mio compagno, ed io qua per viaggio \\
Ci muterem l'effigie, e il personaggio.
\end{ottave}

\begin{ottave}
\flagverse{20}La nera a lui darò ch'altrui lo faccia\\
Parere un huom di venerando aspetto;\\
La bianca terrò io, che membra, e braccia\\
Della donna mi dia, che già t'ho detto.\\
La Strega qui gli dice, ch'ei si taccia;\\
Perch ella scrive, e guasto le ha un concetto\\
Ma lo scancella, e mettelo in postilla.\\
Così piega la carta, e la sigilla.
\end{ottave}

\begin{ottave}
\flagverse{21}Le fa la soprascritta, e poi finisce\\
A piè d'un ghirigoro in propria mano,\\
E con essa quel diavolo spedisce\\
Alla volta del Principe di Ugnano;\\
Là dove l'uno e l'altro comparisce\\
Con una delle dette palle in mano,\\
Credendo l'un rappresentar la Fiore,\\
E l'altro il Servo, ma sono in errore.
\end{ottave}

Martinazza scrive la lettera a Baldone in nome della Geva, e i diavoli pigliano
la medesima lettera per portarla un di loro trasformato in Mona Fiore, e
l'altro in un Servo per via di due palle, e se ne vanno così da Baldone. Ma per
havere scambiate le palle, chi dovea apparire la Fiore, appare il Servo, e furono
scoperti.

\begin{description}
\item[NON portai mai lettere, o imbasciate] La maggiore offela, che si possa fare a
  certe donnicciuole, è il dir loro; porta lettere; porta imbasciate; fa servizzi, porta
  polli (detto credo io dal Franzese Poulet, che significa letterino d'amore, quasi
  portatrice di lettere amorose) perché vuol dire Ruffiana: E però madonna
  Martinazza, che non vuole quest'offesa addosso si dichiara, che non è donna da
  portar lettere, o ambasciate, cioè da far la ruffiana.

\item[ECCOMI lesto] Eccomi pronto: Eccomi all'ordine. \textit{Lesto} in questo
  luogo vuol dir disinvolto, e senza imbarazzi.

\item[DAR colore al negozio] Far' apparir per vero quel che è incerto; Dargli verisimilitudine.
  Questo fanno appresso i Rettorici quei, che da loro sono chiamati
  Colori. Givvenale dice: \textit{dic, Quinctiliane, colorem}.

\item[COMARE] Quella che tiene la creatura al Battesimo. E qui il Poeta osserva
  il costume, che in simili amori per lo più la Balia, la Comare sono mezzane, e
  portano le parole,

\item[MONA] È parola sincopata da Madonna, ed è il titolo che si da comunemente
  alle donne d'infima plebe dicendosi in diminuzione Signora, Madonna,
  Monna, come Signore, Messere, Sere. Ma perché Monna oltre al significato di
  Bertuccia, ha ancora altro significato osceno (almeno in lingua Veneziana) noi
  per sfuggir l'equivoco; hoggi costumiamo dire \textit{Mona} e non Monna:

\item[MAI più] Hormai, Cioè finiscila una volta: È termine dimostrative d'una
  certa impazienza, e si dice: \textit{Oh mai più}: ed è il latino \textit{tandem aliquando}: e si
  confà con l'imperativo, dicendosi: \textit{Oh mai più: finitela}.

\item[POSTILLA] Nel nostro idioma ha diversi significati; perché, o vuol dire
(figuratamente secondo Dante) immagine d'un'oggetto, che ritorni alla nostra
veduta da un vetro, o dall'acqua chiara, Dan, Par. C. 30,
\begin{verse}
  \backspace Quali per vetri trasparenti, e tersi,
  O ver per acque nitide, e tranquille,
  Non sì profonde, ch'i fondi sien persi;
  \backspace Tornan de nostri visi le postille
  Debili si, che perla in bianca fronte
  Non vien men tosto alle nostre pupille.
\end{verse}

O vuol dire annotazioni, o glossa, che i Latini dicono expositio. O si piglia per
breve scrittura aggiunta, ed è composta di due dizioni \textit{post}, \& \textit{illa}. Quasi dica,
\textit{Postilla verba}, cioè dopo quelle parole, scrivi, o aggiungi questo, e questo. E
da queste annotazioni, glose, o aggiunte hoggi per \textit{postilla} intendiamo anche
la margine del libro, cioè quel bianco che si lascia di sotto, e di sopra, e dalle
bande del foglio scrivendo, o stampando: sì che scrivere in possilla vuol dire scriver
in detta margine; e s'intende ogni aggiunta, che si faccia al testo scritto, o
stampato in qualsivoglia luogo della carta o sia di sotto, o di sopra, o dalle bande
fuori de i versi ordinati, e regolati; ed in questo modo, e luogo, dice che
scrisse Martinazza.

\item[GHIRIGORO] È un tratteggio di penna usato per lo più nelle soprascritte
elle lettere, come mostra il Poeta nel presente luogo, che faccia Martinazza.
\textit{Ghirigoro} da' nostri antichi era detto in volgare il nome Latino di Gregorio;
onde \textit{Papa Ghirigoro} trovasi sempre costantemente scritto nel Malespini, e nel
Villani; come era la lingua di quel tempo. Ma qui Ghirigoro apparisce per avventura
dal girare, e rigirare della penna così detto. E le parole \textit{In propria mano}
s'usano nelle soprascritte di quelle lettere, le quali si mandano a uno, che sia nel
medesimo luogo, o Città, o vero poco lontano da colui che scrive.
\end{description}

\section{STANZA XXII --- XXXVI.}
\begin{ottave}
\flagverse{22}Che Baconero il quale è un'avventato,\\
Nel dar la palla all'altro di nascosto,\\
Senza guardarla prima havea scambiato,\\
E preso un granchio, e fatto un grand'arrosto,\\
Perciò quand'a Baldone egli è arrivato\\
Dice cose dal ver troppo discosto,\\
Mentr'egli afferma d'esser donna, e sembra\\
Huomo alla barba, all'abito, e alle membra.
\end{ottave}

\begin{ottave}
\flagverse{23}E Gambastorta anch'ei balordo, e stolto,\\
Mentr'apparir si crede un'huom dabbene,\\
Alla favella, alla presenza, e al volto\\
Per Una fa servizzj ognun la tiene.\\
Il foglio intanto il Duca havea lor tolto,\\
E veduto lo scritto, e quel contiene;\\
Resta certo di quanto era indovino,\\
Ch'i furbi vorrian farlo Calandrino.
\end{ottave}

\begin{ottave}
\flagverse{24}E poiché gli hanno detto, che la Geva\\
A lui gli manda con quel foglio a posta\\
Ma prima che da loro lo riceva\\
Hann'ordine d'haverne la risposta;\\
E soggiunto, che mentr'ella scrivea,\\
Gettava gocciolon di questa posta,\\
Per il trambusto grande ch'ella ha havuto,\\
Come potrà sentir dal contenuto.
\end{ottave}

\begin{ottave}
\flagverse{25}Egli è (dic'egli) un gran parabolano,\\
Chi dice ch'ell'ha scritto la presente,\\
Quand'ella non pigliò mai penna in mano,\\
E so di certo ch'ella n'è innocente.\\
Che poi tu sia la Geva, ch'in Ugnano\\
A me fu molto nota, e confidente,\\
E tu sia huom, a dirla in coscienza,\\
A me non pare, e nego conseguenza.
\end{ottave}

\begin{ottave}
\flagverse{26}I buon non compagni a una risposta tale\\
Guardandosi in viso, in quel sendosi accorti,\\
Ch'egli hanno equivocato, e fatto male;\\
Restan quivi allibbiti, e merzi morti,\\
Ed alle gambe havendo messo l'ale\\
Fuggon ch'e par ch'il diavol se li porti\\
Con una solennissima fischiata\\
Di Baldone, e di tutta la brigata.
\end{ottave}

Giunti quei Diavoli da Baldone, credendosi rappresentare uno la Fiore e
l'altro il Servo, non essendo accorti d'havere scambiate le palle, fecero la loro
ambasciata: Ma Baldone, compreso, che questa era una furberia, non tanto da
ciò, quanto dall'essergli noto, che la Geva non sapeva scrivere, se gli levò dinanzi
con una gran quantità di fischiate.
\begin{description}
\item[AVVENTATO] Uno che opera senza considerazione, e furiosamente. Huomo
  inconsiderato, e precipitoso; dal ivo Latino \textit{adventare} in significato
  d'avvenirsi, cioè imbattersi in una cosa con velocità, e con furia.

\item[DI nascosto] E lo stesso, che Di soppiatto detto sopra C.~1. stan. 75.

\item[PIGLIAR un granchio] Pigliare errore; Intender una cosa per un'altra. Si dice
  \textit{pigliare un granchio a secco} quando uno nel picchiar qualche materiale, scambiandolo,
  si batte il martello sopr'alle dita, o si serra le dita fra due materiali: e da questo
  errore intendiamo poi far un'errore, quando diciamo \textit{pigliare un granchio}. Berni
  \textit{Che Virgilio ha preso Un granciporro}.

\item[FAR un arrosto] Far un'errore. E' lo stesso che \textit{pigliar un granchio}. Viene
  per avventura dal verbo \textit{arrostarsi}, che vuol dir affaticarsi spropositatamente, e
  furiosamente; e le cose fatte in furia non si fanno mai bene.

\item[BALORODO, e stolto] Sinonimi che significano Huomo senza giudizio. La voce
  stolto è pura latina, e balordo è lo stesso che in Lat. \textit{bardus}.

\item[UNA FA servizzj] Come s'è detto sopra s'intende una Ruffiana.

\item[VOGLION farlo Calandrino] Calandrino, secondo che dice il Boccaccio nelle
  sue Novelle, fu un'huomo tanto credulo, che gli fu dato ad intendere fino, che
  egli era pregno, e però da costui diciamo \textit{Tu mi vuoi far Calandrino} per intendere
  \textit{Tu mi vuoi far credere quel che io so, che non è vero}. Si dice anche \textit{far
    Cappellino}, da uno de' nostri tempi della natura di Calandrino.

\item[HANNO ordine d'haverne la risposta] Il Poeta per maggiormente esprimere la
  castronaggine di costoro, fa che chieggano la risposta prima di presentar la proposta.

\item[GETTAVA goccioloni di questa posta] Lagrimava gagliardamente. Il termine
  \textit{Di questa posta} significa grossezza; \textit{erano pere di questa posta}, cioè pere grossissime,
  e si suppone, che colui, il quale dice così, accompagni il parlare col gesto
  delle mani dimostrante la grossezza di quella tal cosa. Si dice anche \textit{tanto fatte};
  \textit{tanto grosse}, come vedremo sotto C. 10, stan. 17. 18. e 36.

\item[TRAMBVSTO] Travaglio, rimescolamento, sollevamento d'animo per causa di disgrazie.

\item[PARABOLANO] Bugiardo; chiacchierone; spropositato: Da Parabola, cioè
  similitudiae, o Racconto; ne' Capitoli di Carlo il Calvo si legge. \textit{Parabolaverunt,
    simul, \& consideraverunt}. Parlarono insieme, \textit{Du Fresne} alla V. \textit{Parabola}.

\item[SO ch'ella n'è innocente] Intende; io so ch'ella non sa scrivere. Per esprimere
  uno che non habbia ne pure una minima notizia d'una tal cosa diciamo:
  \textit{Il tale non ha peccato alcuno nella tal cosa o è innocente della tal cosa}.

\item[NEGO conseguenza] Nego il tutto: perché negando la conseguenza, si viene
a negare implicitamente tutto l'argomento, e così tutto il discorso.

\item[ALLIBBITI] Confusi, sbalorditi per un subito timore, o vergogna, e perciò
  diventati di colore smorto, e gialliccio, come, seccandosi, diventano le potature
  degli olivi, che si chiamano \textit{libbie}, dalla qual voce viene \textit{allibbito}, e \textit{allibbire}.
  Vedi il Vocabolario della \textit{Crusca} alla Voce \textit{Allibbìre}. Il Varchi Stor. Fior. lib, 10.
  \textit{Niuno udì tal cattiva nuova, il quale incontinente (quasi le fusse venuta meno la terra
  sotto a' piedi) non allibbisse}.

\item[FISCHIATA] Romore di voci, fischi, urli, battimenti di mani, e d'altro,
  che si fa dietro a uno per dargli la burla. Far le fischiate a uno, quel che i Lat.
  dissero \textit{exsibilare}.
\end{description}

\section{Stanza XXVII. --- XXX.}
\begin{ottave}
\flagverse{27}Adesso a Calagrillo me ne torno
Con la dolente Psiche ognor d'attorno, \\
Ch' ad ggni quattro passi fa un lamento.\\
Ha camminato tutto quanto il giorno,\\
E domandato cento volte, e cento\\
La via di Malmantile, e similmente\\
Di Martinazza, e se v'è di presente.
\end{ottave}

\begin{ottave}
\flagverse{28}Dè in un, ch'al fin la mette per la via \\
Con dirle, che quest'orrida Befana, \\
Che già d'un tozzo haveva, carestia \\
E stava come l'erba porcellana, \\
In hoggi ha di gran soldi in sua balia, \\
Ed ha una casa come una Dogana, \\
E nella Corte è in grado, e giunta a segno, \\
Ch'ell'è il \textit{totum continens} del Regno.
\end{ottave}

\begin{ottave}
\flagverse{29}Che la padrona il tutto le comparte\\
Come s'in Malmantil sien due Regine,\\
Anzi il bando si manda da sua parte,\\
Perch'ella soffia il naso alle galline,\\
Così poi c'hebbe dato libro, e carte,\\
Entra nell'un viè un, che non ha fine,\\
Costui, che quivi s'è posso a bottega\\
A legger sopra il libro della Strega,
\end{ottave}

\begin{ottave}
\flagverse{30}Quest'altro, che non cerca da costui\\
Di questi cingue soldi, havendo fretta,\\
Poi ch'egli ha inteso quel che fa per lui,\\
Sprona il cavallo tutto a un tempo, e sbietta,\\
La donna che trovare il suo, colui\\
Di giorno in giorno per tal mezzo aspetta,\\
Per non lo perder d'occhio, e ch'ei le manchi,\\
Segue la starna, e gli va sempre a i fianchi
\end{ottave}



Torna il Poeta a parlar di Calagrillo; il quale camminando Psiche s'imbatte
in uno, che le dà avviso di dove sia Martinazza.
\begin{description}
\item[MARCHIARE] Si dice marciare, che vuol dir camminare. Voce Francese,
  ma già fatta Italiana. Vedi sopra C. 1, stan. 43. E più accosto alla pronunzia,
  Oltramontana, dicesi anche \textit{Marciare}, forse da \textit{Marcia}, contrada, pace, cammino
  \textit{danesmarce} disse il Villani la Danimarca; cioè \textit{Danese Contrada}.

\item[BEFANA] Intendiamo Donna brutta, malfatta. Vedi sotto C. 8. stan, 30.
e C.9, stan. 1.

\item[TOZZO] S intende pezzo di pane. \textit{Haver carestia d'un tozzo}, Vuol dire esser mendico, pezzente.


\item[STAVA come la porcellana] Cioè terra terra, come l'erba porcellana, che serpeggia
  per terra, e non alza mai virgulti; detta \textit{porcellana} dal Latino \textit{Portulaca}.
  E questo detto significa Uno che sia in povero stato, e non habbia modo di sollevarsi,
  che i Latini pure dicevano \textit{humi jacere}.

\item[IN sua balìa] In suo potere, e dominio. Balìa e voce fatta venire dal Monosini
  dalla Greca \textit{Buleia}, che suona lo stesso che \textit{Bulo}, cioè consiglio, Parlamento,
  Senato. A noi suona Potestà, giurisdizione, autorità, e quel che i Latini dicevano,
  \textit{potestas imperium}. Dan. Purg, C. 1.
  \begin{verse}
    \backspace Ed hora intende mostrar quegli spirti,
    Che purgan se, sotto la sua balìa.
    \verseprefix{Petr.C.36.}Mentre ch'il corpo è vivo,
    Hai tu il freno in balia de' pensier tuoi.
  \end{verse}

\item[HA una casa come una Dogana] Cioè piena di robe, come sono le Dogane piene di mercanzie.

\item[IL Bando va da parte sua] Cioè, ella comanda.
\item[SOFFIA il naso alle galline] Ella fa tutte le faccende. E questi tre modi di dire
  \textit{Totum continens del Regno}, \textit{Bando va da parte sua}, e \textit{soffia il naso alle galline} hanno tutti
  lo stesso significato; ma di questo ci serviamo per lo più per derisione, per intendere d'uno
  che habbia ambizione d'esser creduto gran ministro, ed habbia i maggiori maneggi
  d'un governo, e non sia vero; che per ischerzo direbbesi anche \textit{Arcifanfano}.
  En.Tr.l.4.st.15.\textit{Soprattutto a Giunon, che del far razza È detta l'arcifanfana, e 'l fac todo}.

\item[DAR libro, e carte] Dare esatta notizia d'alcuno. Viene da'coloro, i quali
  havendo debito co' Magiftrati, son mandati in esazione a i Ministri forensi, alli
  quali Ministri i Magistrati mandano il contrassegno del libro, nel quale è scritto
  il debito di quel tale, il nome, e casato di esso, l'origine, e somma del debito,
  ed a quante carte è la sua partita: E questo si dice \textit{dar libro, e carte}, che passano
  in proverbio, significa Dar notizia chiara, ed esatta d'alcuno; o palesare chi
  habbia fatta un'azione per altro occulta.

\item[ENTRA nell'un viè uno] Fa un discorso da non uscirne mai, come avverrebbe
  se uno volesse seguitare \textit{Un vie uno fa uno}, \textit{due viè due fa quattra}, ec, che s'anderebbe
  nell'infinito. Dice il Varchi nel suo Ercolano, che in questo senso si dice
  \textit{Cantar la canzone dell'uccellino}. Con tal dettato s'esprime un chiacchierone,
  che cicalando, faccia molte digressioni spropositate per allungare il suo cicalamento
  con racconti assai sconvenevoli, che si dice; \textit{Entrare in un ginepraio}, \textit{saltare
  di palo in frasca}.

\item[S'È messo a bottega] S'è preso per arte, per suo mestiero, o negozio. Quando
  uno fa qualche operazione con tutta applicazione, ed attenzione, e con dimostrazione
  di voler durare assai, diciamo; \textit{Costui s'è messo a bottega}.

\item[LEGGER sul libro d'alcuno] Narrar le azioni, qualità, e stato d'alcuno.

\item[NON cerca questi cinque soldi] Non cerca, non gl'importa, non proccura
  sapere questa cosa. Quand'altri fa un discorso, e fa una digressione senza tornar
  più al primo proposito, se li dice: \textit{Voi pagherete la pena de' cingue soldi}. Vedi sotto
  C.~8. stan. 15. E però dicendo: \textit{Non cerco questi cingue soldi},s'intende; non mi
  curo di guadagnar questa pena de' cingue soldi, con obligarti a seguitare il principiato
  discorso.

\item[SBIETTA] Scappa via presto, Vedi sotto C. 7. stan. 87.

\item[IL suo colui] Il suo amante, cioè Cupido.

\item[PER non lo perder d'occhio] Perché non le esca di vista. Per non lo smarrire.

\item[SEGVITA la starna] Quand'uno seguita un'altro per haver da lui qualche
  favore, diciamo: \textit{Ei seguita la starna}. E si dice la starna, e non altro uccello,
  perché queste si vincono col seguitarle, osservandole dove si posano, e straccandole nei loro voli.
\end{description}
\section{Stanza XXXI ---- XXXV}


\begin{ottave}
\flagverse{31}Quando al Castello al fin sono arrivati,\\
Là dove altrui assordano l'orecchie\\
Gli strepiti dell'armi, e de' soldati,\\
Che d'ogn'intorno son più delle pecchie;\\
Domandan soldo, ed a Baldon guidati,\\
Che havendo del guerrier notizie vecchie,\\
Gli va incontro, l'accoglie, e riverisce,\\
Ed egli a lui con l'armi s'offerisce.
\end{ottave}

\begin{ottave}
\flagverse{32}Ma piacciati, soggiunse, ch'io ti preghi\\
Per questa donna rimaner servito,\\
Che questo ferro pria per lei s'impieghi\\
Per conto qua d'un certo suo marito.\\
A tanto Cavalier nulla si nieghi,\\
Risponde a ciò Baldon tutto complito,\\
Tu sei padrone; fa ciò che tu vuoi,\\
Non ci van cirimonie fra di noi.
\end{ottave}

\begin{ottave}
\flagverse{33}Ti servirò di scriverti alla banca,\\
E in tanto per adesso io ti consegno\\
Il gonfalon di questa ciarpa bianca,\\
Che tra le schiere è il nostro contrassegno;\\
Tal che libero il passo, e scala franca\\
Haurai per dar' effetto al tuo disegno;\\
Che non so qual si sia, ne lo domando;\\
Pero va pur ch'io resto al tuo comando.
\end{ottave}

\begin{ottave}
\flagverse{34}Ei lo ringrazia, E gito più da presso,\\
Ove sta chiuso di Psiche il bel Sole,\\
Ad essa dice: In quanto al tuo interesso,\\
Fin qui non t'ho servito, e me ne duole,\\
Che tu non pensi, havendoti promesso,\\
Ch'io faccia fango delle mie parole,
E ch'il mio indugio, e il non risolver nulla\\
Sia stato un voler darti erba trastulla.
\end{ottave}

\begin{ottave}
\flagverse{35}O ver ch'io me la metta in sul liuto,\\
O ti voglia tener l'oche in pastura,\\
Come quel che ci vada ritenuto\\
Per mancanza di cuore, e per paura,\\
Perché si come havrai date veduto,\\
Non ho fin qui trovata congiuntura\\
Di chi m'indirizzasse qua al Castello,\\
Per poterne cavar cappa, o mantello.
\end{ottave}


Calagrillo con Psiche arriva al Campo, e chiede soldo: Baldone l'accetta, e
gli da licenza d'andare a servir Psiche, con la quale avviandosi verso Malmantile,
Calagrillo si scusa di non' haver prima servita.

\begin{description}
\item[SCRIVER alla banca] Arrolare uno per soldato: Banca diciamo quel luogo
  dove sono scritti i soldati, e dove son loro pagatii denari degli stipendj.

\item[GONFALONE] Vuol propriamente dire vessillo; ma si piglia per ogni sorta
  d'insegna, Vedi il Vossio \textit{de vitijs sermonis} lib, 1. ove di questa voce.

\item[CIARPA] E' una legaccia di drappo, che dai soldati si cinge come la cintura
  della spada. E per altro \textit{ciarpa} vuol dire quel che accennammo sopra Cant.
  3. stan. 5. Franzese \textit{Escharpe}.

\item[SCALA franca] Franchigia; Libertà d'andare, o stare. Passo libero.

\item[FAR fango delle sue parole] Disprezzare la parola data e non osservar le promesse.

\item[DAR erba trastulla] \textit{Metterla sul liuto}; \textit{mandar l'oche in pastura} hanno tutti tre
  lo stesso significato, che è trattener' uno con chiacchiere. Lat. \textit{verba dare}.
\end{description}
\section{Stanza XXXVI. --- XXXVIII.}

\begin{ottave}
\flagverse{36}Risponde Psiche a questa diceria:\\
Io non entro Signore in questi meriti,\\
Non ho parlato mai, ne che tu sia, \\
Tardo, o spedito, o ver che tu ti periti,\\
Quel che tu fai, tutt'è tua cortesia,\\
Per tal l'accetto, e 'l Ciel te lo rimeriti,\\
Con darti in vita honor, fama, e ricchezza,\\
Sanità dopo morte, ed allegrezza.
\end{ottave}

\begin{ottave}
\flagverse{37}Sta quieta, le dic'egli, e ti conforta, \\
Ch'io voglio adesso dar fuoco al vespaio, \\
Così col Corno, il quale al colle porta,\\
Chiama la guardia, o vero il portinaio, \\
Non è sì presto il gatto in su la porta, \\
Quand'ei sente la voce del beccaio; \\
Quanto veloce a questo suon la Ronda \\
Sopr' alle mura accostasi alla sponda.
\end{ottave}

\begin{ottave}
\flagverse{38}Un par d'occhiacci orlati di savore\\
Così addosso a un tratto gli squaderna,
Che par quand'il Faina alle sei hore\\
In faccia mi spalanca la lanterna,\\
E mediante un certo pizzicore,\\
Ch'ei sente al collo, i pizzicotti alterna,\\
Ond'alle dita egli ha fatti i ditali\\
D'intorno a innumerabili mortali.
\end{ottave}


Psiche rende grazie a Calagrillo della carità, che le promette, e facendo le lor
cirimonie, s'accostano al Castello, dove Calagrillo, sonando il Corno, chiama
la sentinella, la quale subito s'affaccia alle sponde delle mura.

\begin{description}
\item[DICERIA] Vuol dire Ragionamento, Discorso, Orazione: ma hoggi questa
  voce è usata per lo più per intendere Ragionamento stucchevole, e odioso per la
  lunghezza.

\item[NON entro in questi meriti] Non parlo di queste cose. Ma questo detto ha
  una certa forza d'esprimere: io non ardisco d'entrar tanto in la col discorso; maniera,
  che viene dal solersi dire; il merito della lite, o della causa, cioè l'importanza
  del fatto.

\item[SANITA, ed allegrezza dopo morte] E detto giocoso, perché un corpo morto
  non può haver sanità, ne allegrezza, ne altre passioni. Ma si potrebbe anche
  dire, che questa donna, parlando iperbolico, voglia dire che egli viva sano, ed
  allegro sempre eziam dopo morte, il che è imposhibile, come è imposibile viver
  mill'anni, e pure si dice: vi prego mille anni di vita, \textit{Sanità} è un'augurio, che
  corrisponde at Greco \textit{hygiainein}, cioè \textit{star sano}, che metteva innanzi alle sue epistole
  Pittagora devotissimo della sanità; \textit{Allegrezza} corrisponde a quel saluto, che
  in principio esprimevano i Greci comunemente nelle lor lettere, perché dove i Latini
  pongono \textit{Salutem dicit}, essi scrivevano \textit{Chairein}, cioè come tradusse Orazio
  in una sua Epistola \textit{Gaudere}, volendo dire, Il tale al tale desidera \textit{allegrezza},
  siccome in quell'altro modo usato da Pittagora: il tale al tale, desidera \textit{sanità}.

\item[DAR fuoco al vespaio] Violentare a uscir fuora uno, che sia dentro; come segue,
  quando si da fuoco a un vespaio, che le vespe son forzate dal fuaco a scappar
  fuori. Vedi Omero lib, 16, dell' Iliade,

\item[LA voce del Beccaio] Vanno per Firenze alcuni Beccai, o Macellari yendendo
  carne per dare a' gatti, e fanno certe lor voci così ben conosciute da i medesimi
  gatti, soliti havere la carne, che appena costoro hanno aperta la bocca, che i
  gatti sono in sulla porta. A questi gatti assomiglia la guardia di Malmantile, che
  a pena sentito il suono del corno s'affaccia alla muraglia. Delle voci, e de' versi
  che fanno i venditori, che vanno attorno per invitare il compratore, Seneca ep.
  56. \textit{Iam libarij varias exclamationes, \& botularinm, \& crustularium, \& omnes popinarum
    institores, mercem sua quadam, \& insignita modulatione vendentes}.

\item[RONDA] Si dice quel Soldato di guardia, che rigira, e passeggi per la muraglia
  della fortezza, visitando la Sentinella, detta così dall'andare in volta, e
  come i Franzefi dicono, \textit{aller en rond}.

\item[SPONDA] Parapetto della muraglia; Quel pezzo di muro, che avanza alle
  muraglie sopra il terreno del terrapieno, e si dice \textit{sponda} quel muretto, o spalletta,
  che avanza sopra il terreno, a i pozzi, a' fiumt, ec.

\item[ORLATI di savore] Circondati di cispa per la similitudine, che ha con la cispa
  il savore secco. E \textit{savore} è uno intingolo fatto di noci, e pane pesto, e liquefatto
  con agresto. E \textit{cispa} diciamo quell'umor crasso, che si condensa intorno alle palpebre,
  e su i peli degli occhi.

\item[COSÌ a un tratto gli squaderna gli occhi addosso]Subito fissa sopra di lui gli occhi
  ben'aperti. E questo verbo \textit{squadernare} s'usa per divolgare, manifestare, ec.
  Dan. Par. C.33.
  \begin{verse}
    Ciò che per l'universo si sqaderna
  \end{verse}

\item[FAINA] Celebre Luogotenente di Birri così chiamato per soprannome.

\item[SPALANCARE] Aprir quanto si può una porta, un' armario, e simili: levare
  la palanca, cioè il palo, che tiene in alcune porte fermato tutta, o una
  banda della porta; aprire affatto. Vedi sotto C. 6. stan. 43.

\item[ZILOTTO] È uno stringimento, che si fa in qualche parte del corpo,
  pigliando la pelle col dito indice, e stringendola col dito pollice; e così faceva
  costui intorno al collo, \textit{alternando i pizzicotti}, cioè facendoli hor con l'una, hor
  con l'altra mano per pigliare i pidocchi, che sono quegli \textit{innumerabili mortali, che
    col loro gli hanno fatti i ditali}, cioè ricoperte le dita; Che \textit{ditale} intendiamo
  parte del guanto, che cuopre il dito.
\end{description}

\section{STANZA XXXIX. \& XXXX.}

\begin{ottave}
\flagverse{39}Non tanto s'abburatta per la rogna, \\
E pe' bruscol, che vanno alla goletta, \\
Quanto che dir non può quel che bisogna \\
Ch'ei tartaglia, e scilingua anche a bacchetta, \\
Qual il quartuccio le bruciate fogna, \\
Ne senza quattro scosse altrui le getta, \\
Tal si dibatte, e a vite fa la gola \\
Ogni volta ch'ei manda fuor parola.
\end{ottave}

\begin{ottave}
\flagverse{40}Bu bu, bu bu comincia, ch'il buon giorno\\
Vorrebbe dar al Cavalier, ch'ei tiene\\
Il Corrier, mediante il suon del Corno,\\
Del popol d'Israel ch'or va, hor viene;\\
Van le parole a balzi, e per istorno\\
Prima c'al segno voglian colpir bene;\\
Pur pinse tanto, che gli venne detto;\\
Buon dì Corrier, che nuova c'è di Ghetto.
\end{ottave}


Descrive il Poeta la guardia, la quale havendo creduto che Calagrillo fusse
un' Ebreo, lo saluta come tale.

\begin{description}
\item[S'ABBURATTA] Si dimena: Si dibatte. Abburattare propriamente vuol
dire Separare la farina dalla crusca con lo staccio.

\item[BRUSCOLI che vanno alla goletta] Intende i pidocchi, che vanno alla gola.
  \textit{Goletta} intendiamo l'estremità dell'abito da huomo intorno alla gola. Ed il
  Poeta copre questo detto con l'equivoco di \textit{Goletta}, fortezza in Barberia, e con la
  voce \textit{bruscoli}, che sono minutissime particelle di legno, o paglia, o simili, ed egli
  intende pidocchi.

\item[TARTAGLIARE] Intoppare nel profferir le parole; pronunziar con difficultà,
  e \textit{scilinguare} vuol dir Balbettare.

\item[A BACCHEITA] Comandare a bacchetta vuol dire Comandare assolutamente
  e dispoticamente in ogni congiuntura, come Re, o Capitano, che porti
  scettro, mazza, o bastone di comando, e di qui intendesi, che costui tartagliava,
  e scilinguava ogni lettera.

\item[QUARTUCCIO] Misura Fiorentina capace della sessantaquattresima parte
  dello staio\footnote{Staio, usato per misurare volumi aridi, come i cereali. Uno staio equivale a circa 24.3ℓ, la sua sessantaquattresima parte è circa 380ml.  Lo stesso nome Quartuccio si riferisce anche alla centosessantesima parte del barile da vino, circa 45.6ℓ. In questo caso un quartuccio equivale a circa 280ml. Il peso è approssimativamente uguale nei due casi. }, e per lo più è un vaso di legno.

\item[BRVCIATE] Marroni cotti arrosto in padella, o in forno, o sotto la brace.

\item[FOGNARE] Fogna vuol dire quel vacuo fatto ad arte sotto terra per dove
  passa l'acqua, e si conduce scolando al fiume dal Lat, \textit{fovea}: E di qui \textit{fognare la
    misura} vuol dir metter la roba nella misura in maniera, che apparisca piena,
  ma dentro vi sieno molti vacui, come facilmente segue nel quartuccio, entro al
  quale non si possono stivare i marroni, i quali per esser di figura rotonda non
  riempiono lo spazio, ma fanno naturalmente, che rimangano fra l'uno, e l'altro
  molti vacui nella misura; la quale poi, volendoli votare, è necessario squotere;
  perché s'affrontano nell'uscire, e soqquadrano alla bocca del quartuccio
  in maniera, che non potriano scappar fuori, se non si squotesse il vaso, ed uscendo,
  fanno un romore simile a uno che tartagli, le di cui parole pare, che non
  possano uscir di bocca, se egli non si squote, dibatte, o storce; e quell'intervallo
  che egli mette fra una parola, e l'altra lo figura il vacuo che sta fra un marrone,
  e l'altro. E questo intende col dire \textit{qual il quartuccio le bruciate fogna}, cioè
  fogna le parole con interuallo di tempo, e non di luogo.

\item[FAR la gola a vite] Storcer la gola. Vedi sopra C. 2, stan. 9.

\item[PER storno] Si dice quel ritornare indietro, che fa la palla che ha percosso
  nella parte opposta dove è stata tirata o sia muro, o sia altro, ed è termine proprio
  del giuoco delle pallottole, e s'intende quand'uno tira per accostarsi al segno
  per via di detto storno, e non direttamente: E così indirettamente uscivano
  di bocca a costui le parole. In somma vuol dire, che egli impuntava nel parlare,
  tartagliava, e parlava a salti.

\item[GHETTO] Così chiamiamo il Serraglio, nel quale stanno in Firenze, ed in
  altre Città gli Ebrei: E perché questi hanno nome di tener di mano a stregherie,
  però dice che il Corriere di quel luogo è solito spessoo andare a Malmantile a trovar
  la flregha Martinazza. \textit{Ghetto} e voce Caldea, che significa libello di repudia;
  onde noi diciamo \textit{Ghetto} per intender luogo di gente segregata, e repudiata
  dal commercio degli altri huomini. Gli Ebrei quando vogliono dire loro
  mogli, che le gastigheranno col repudiarle dicono; \textit{Ti manderó al Ghet}.
\end{description}

\section{STANZA XXXXL STANZA XXXKIL}

\begin{ottave}
\flagverse{41}Rispose l'altro, tal parola udita:\\
D'esser corriere già negar. non posso,\\
Perch'io l'ho corsa a far questa salita,\\
Ma quanto al Ghetto io non la voglio addosso; \\
Non ho che far con gente Ifraelita;\\
Ben ti farà il mio brando il cappel roffo,\\
E col darti sul viso un soprammano \\
D'Ebreo farà mutarti in Siciliano.\\
\end{ottave}

\begin{ottave}
\flagverse{42}Ma che vo il tempo qui buttando via\\
In disputar con matti, e con buffoni?\\
Il trattar teco credomi che sia\\
Come a' Birri contar le sue ragioni;\\
Ne dissi mal, perch' hai fisonomia\\
D'un di color, che ciuffan pe' calzoni,\\
E l' esser tu costì, par ch'ella quadri,\\
Ch' i Birri sempre van dove son ladri.
\end{ottave}

\begin{ottave}
\flagverse{43}Benché voi siate come cani, e gatti,\\
Ch'essi non han con voi gran simpatia,\\
Perché peggio de' diavol sete fatti,\\
Usando nel pigliar più tirannia;\\
Dell'alma sola quei son soddisfatti,\\
Ma voi col corpo la portate via.\\
Hor basta, se tra voi tant'odio corre.\\
Meglio a i lor danni ti potrò disporre.
\end{ottave}

\begin{ottave}
\flagverse{44}Hor dunque tu, che sei così pietoso,\\
Che pigli i ladri, acciò Mastro Bastiano\\
Sul letto a tre colonne almo riposo\\
Dia lor del tanto lavorar di mano,\\
Perch'a qualunque ladro il più famoso\\
Martinazza di rubar non cede un grano,\\
Che non uccella a pispole, ma toglie\\
Cupido a questa donna, ch'è sua moglie.
\end{ottave}

\begin{ottave}
\flagverse{45}Lo stesso devi oprar, c'a lei sia fatto;\\
Mentr'a costei non renda il suo Consorte.\\
A cui (perch'ei consente in tal baratto)\\
Questa potrebbe far le fusa torte;\\
Ed ei si cerca esstr mandato Hn tratto\\
Sull'asin con due rocche dalla Corte,\\
Sì che, se tu nol sai, ti rappresento,\\
Cè un difordine qui ne puo far cento,
\end{ottave}

\begin{ottave}
\flagverse{46}Però se voi adesso, a cui s'aspetta,\\
Costà non impiccate questa Troia,\\
Io stesso vuò pigliarmi questa detta,\\
E farle il Birro, e in sulle forche il Boia,\\
Mentre però Cupido non rimetta:\\
Ma se lo rende non vi do più noia,\\
Va dunque, e narra a lei quanto t'ho detto,\\
Ch'io qui t'attendo, e la risposta aspetto.
\end{ottave}


S'adira Calagrillo, che colui l'habbia preso in cambio del Corriere degli Ebrei,
e lo minaccia di rompergli la testa, e sfregiarlo; e dopo havergli detto molti improperj,
gli ordina, che da sua parte avvisi Martinazza, che renda Cupido; altrimenti
glielo farà render per forza.
\begin{description}
\item[L'HO corsa] Ho fatta questa cosa senza considerazione. Quand'altri fa qualche
  risoluzione, che non riesce poi buona, diciamo: \textit{Ei l'ha corsa} dall'armeggiare,
  e dal correre la giostra. Similmente diciamo; \textit{Fare una carriera}. Qui fa
  giuoco la voce corsa, che è cosa da Corrieri.

\item[NON la voglio addosso] Non la voglio sopportare. Si dice anche \textit{non la voglio in
  sul giubbone}.

\item[GENTE Israelita] Intende Ebrei: Popolo d'Israel.

\item[IL cappello rosso] Gli Ebrei in Firenze portano per contrassegno il Cappello
  rosso. Il Poeta dice, farò ben' io diventare Ebreo te col farti il cappello rosso col
  sangue. E poi d'Ebreo ti farò diventar Siciliano tagliandoti il viso, ed intende
  quel Siciliano Montambanco, che per accreditare il suo Olio da Ferite si faceva
  gran tagli nella persona, e con esso se le medicava.

\item[SOPRAMMANO] Quel colpo, che si dà con spada, o bastone, cominciando
  da alto, e calando a basso. Vedi sotto C, 10. stan. 52.

\item[BUFFONE] Uno che fa professione di trattener la brigata con facezie.

\item[DIR le sue ragioni a Birri] Raccomandarsi a chi non può, e non vuol far
  servizio; anzi ha caro il tuo male. Vuol anche dire discorrer con uno, che non
  bada a quel che tu dica; o vero buttar le parole al vento, Plauto disse nel Pseudolo:
  \textit{apud novercam queri}.

\item[CIUFFAN pe' calzoni] Cioè i Birri; i quali pigliano pe' calzoni. Il verbo
  \textit{ciuffare} ha del furbesco; e vuol dir Pigliar con presa stabile, e buona, come è
  quella che si fa, pigliando uno per il ciuffo, cioè pe' capelli. Petrarca. \textit{Le man
    l'avess'io avvolte entro a' capegli}.

\item[ESSER come cani, e gatti] Esser poco d'accordo, o poco uniti, anzi sempre
  nimici, come naturalmente sono i cani, e i gatti.

\item[NON ha gran simpatia\ La voce \textit{simpathia} Greca fatta Toscana significa inclinazione
  scambievole, o similitudine di genio, di voleri, e d'affetti.

\item[MAESTRO Bastiano] Intende il Boia, che allora così havea nome, e prima
  era stato Maestro Biagino. Vedi sotto C, 6. stan. 56.

\item[LETTO a tre colonne] Cioè le forche, le quali veramente son tre colonne con
  una stanga sopra a traverso, ed in molti luoghi sono in triangolo.

\item[LAVORAR di mano] Vuol dir rubare. scherza dicendo, che il Maestro,
  (cioè il Boia) perché essi ricevano qualche riposo da tanto lavorare (cioè rubare)
  gli mette in su 'l letto a tre colonne (cioè in sulle forche) ed in sustanza buol dire
  gl'impicca, perché son ladri. E Calagrillo, seguitando l'equivoco del riposo,
  dice alla guardia, che se ella ha punto di pietà, e discrezione, dovrebbe dar questo
  riposo in sul letto di tre colonne a Martinazza per il suo tanto lavorare, cioè
  impiccarla, perché è ladra. I Latini pure per dir copertamente rubare dissero:
  \textit{manu sinistra uti} secondo Catullo in Asinium.
  \begin{verse}
    Marrucine Asini, manu sinistra
    Non belle uteris in ioco, atque vino;
    Tollis lintea negligentiorum.
  \end{verse}
  E per dire copertamente Impiccar'uno, dicevano; \textit{literam longam facere}, come habbiamo
  notato altrove\footnote{Quarto Cantare, Ottava 27.}.

\item[NON cede un grano] Non cede punto. Che grano si può dire una particella
  inconsiderabile del peso\footnote{Misura per orefici, un grano toscano equivale a poco meno di 50mg}, poiché 24. grani fanno un danaro, 24. denari fanno
  l'oncia, e 12. once fanno la libbra\footnote{Una libbra di Firenze era circa 339.5g}.

\item[NON uccella a pispole] Non si cura di conseguir cose di poco momento, come
  è fra gli uccelii la pispola. I Latini dissero \textit{Non captat muscas}.

\item[FAR le fusa sorte] Far le corna. Vuol dir quand'una donna si mescola con
  altri huomini, che col suo marito. Il Burchiello Poeta capriccioco, il quale va
  sotto nome d'Accademico Fiorentino incerto, nella Raccolta delle Rime Piacevoli
  del Berni, Casa, ec.
  \begin{verse}
    Non ti fidar di femmina, ch'è usa
    A far le fusa torte al suo marito.
  \end{verse}
  Il Berni nel suo primo capitolo dell'orto dice:
  \begin{verse}
    E finalmente non fara mai fusa
    Donna alcuna per lui torte al marito.
  \end{verse}
  Si dice \textit{fusa torte} per intender copertamente Corna.

\item[MANDATO con due rocche in sull'asino] È costume in Firenze, al gastigo del
  delitto del pigliar più d'una moglie, aggiugnere una dimostrazione obbrobriosa,
  che è il far' andar' per la Città il delinquente legato sopra ad un'asino, con una
  mitra di foglio in capo, ed a cintola due, o più rocche inconocchiate, che significano
  le due, o più mogli.

\item[QUESTA troia] Questa porca. Epiteto vituperosissimo nelle donne, perché
  vuol dire Laida meretrice:: nell'huomo non è tanto ingiurioso il dirgli porco.

\item[MI vuò pigliar questa detta] Vuò pigliarmi l'assunto di far questa cosa. \textit{Star della
  detta} vuol dire prometter per un'altro, o star mallevadore, cioè di far una tal
  cosa, se non la farà quello, che è principalmente obbligato. \textit{Comprar una detta}
  vuol dir comprar un'avviamento, un credito, ec. \textit{Detta} è dal plurale Latino
  \textit{Debita}.

\end{description}
\section{STANZA XXXXVIIL}
\begin{ottave}
\flagverse{47}La Ronda, che far lite non si cura,\\
E vuol riguardar l'armi dalle tacche,\\
Quantunque ad alto sia sopr'alle mura\\
Molto lontana, e già in salvummeffacche,\\
Non vuol teners mai tanto sicura,\\
Che rilevar non possa delle pacche; \\
Però veduto havendo il Ciel turbato\\
Tace, ch'ei par un porcellin grattato.
\end{ottave}

\begin{ottave}
\flagverse{48}Lascia la sentinella, e caracolla\\
Giù pel castello, dando questa nuova,\\
E benché il Maggioringo della bolla\\
Gli habbia promesso, mentre ch'ei si mova\\
Di fargli porre a' piedi la cipolla,\\
Cercando della morte in bella prova\\
Vuol avvisar di ciò Mona Cosoffiola,\\
Ch'è per basire a questa battisoffiola.
\end{ottave}

La Guardia, che è un vero poltrone, sentendo le bravate ai Calagrillo, zitto
zitto si parte, e tremando va a dare questa nuova a Martinazza.

\begin{description}
\item[RIGUARDA l'armi dalle tacche] Non vuol cavar fuori la spada, per non la
  guastare. Intendi che costui era un codardo, perché per dir copertamente poltrone
  a un soldato, se gli dice: \textit{Rispiarma foderi}.

\item[IN salvummeffacche] Parole latine corrotte, e ridotte in una, usato assai dalla
  plebe ignorante per intendere Andare in salvo, ed è il Latino \textit{ad asylum confugere}.

\item[RILEVAR delle pacche] Buscare, o toccar delle ferite, che questo intendiamo
  \textit{pacche}, ma è detto plebeo. Il Vocabolista Bolognese dice che \textit{pacca} significa percossa
  gagliarda. La forza di questo verbo rilevare vedemmo sopra C. 3. stan. 67.
  Il Varchi stor. Fiorent. lib.6, dice \textit{Il figliuolo del quale nominato Lorenzo, rilevò una
    ferita}.

\item[HAVENDO veduto il Ciel turbato] Havendo conosciuto, che costui era in collora.
  Si dice anche \textit{la marina turba}.

\item[TACE che pare un porcellin grattato] Similitudine assai usata per intender uno,
  che non risponda alle grida d'un'altro o per paura, o per riverenza, o per la
  coscienza macchiata, o per altro; e si fa la comparazione al porco, perché il
  porco che stride, grattandolo si quieta, ed i porcai gli rendono maneggiabili col
  grattargli.

\item[CARACOLLA] Il verbo \textit{caracollare} vuol propriamente dire Volteggiare col
  cavallo, ma non ostante qui torna assai bene per esprimere, che costui per la
  paura andasse girando per il castello, non gli parendo trovare luogo sicuro. E
  però anche in uso \textit{caracollare} per camminare a piede, volteggiando d'una strada in
  un'altra, e diciamo \textit{far un caracollo} per intendere una girata. Viene dalla voce
  Spagnuola \textit{caracol}, che vuol dire \textit{chiocciola}.

\item[IL maggioringo della bolla] Termine della lingua furbesca, che in Firenze vuol
  il Fiscale;  ma s'intende per il Superiore in quegli affari di che si tratta. Vale,
  il Maggiore della Città, chiamata in quella lingua \textit{Bolla} dal Greco \textit{Polis} e
  barbaricamente \textit{Polla}.

\item[FARGLI mettere a' piedi la cipolla] Fargli troncar la testa, e mettergliela a
  piedi: come si costuma in Firenze quando, il cadavero del giustiziato dee stare
  esposto per qualche ora al pubblico; che gli mettono la testa a i piedi.

\item[È PER basire] È per transire, per svenirsi, per morirsi. Vedi sopra Cant. 2, stan. 79.

\item[MONA Cosoffiola] Nome usato per intender una donna faccendiera, affannona, o
  sudatora. Sebbene \textit{Cosoffiola} (secondo il Varchi nel suo Ercolano alla voce
  Battisoffiola) è lo stesso che battisoffiola, e significano affanno, paura, rimescolamento
  grande, ma breve, che cagioni battimento di cuore, o frequenza d'alito,
  il che si dice soffiare: Franco Sacc. Nov. 44. \textit{M'hai data così gran battisoffia, che
    io non sarò mai più lieto, e forse me ne morrò}. Non credo che sia lontano da questo,
  quello che diciamo \textit{soprassalto al cuore}; lo stesso che batticuore, affanno cagionato
  per paura, o dolore improvviso dagli Spagnuoli detto \textit{sobresalto}, nel Franz. \textit{sursaut},
  Corn, Tacito lib. 5. dice: \textit{Exterritae sunt acri magis quam diuturno timore}. Ed il nostro
  Davanzati parafrasando queste parole dice \textit{hebbero battisoffia}.
\end{description}
\section{Stanza IL. \& L.}

\begin{ottave}
\flagverse{49}Ella insieme le schiere ha già ridotte \\
Di genti, che non vagliono un pistacchio, \\
Cioè di quelle, a cui fece la notte \\
Col suo carro sì grande spauracchio, \\
Ed hor quivi parare, e dar le botte \\
Insegna lor, che non ne san biracchio \\
Ma quand'innanzi a lei costui, si ferma\\
Così tremante, la cavò di scherma.
\end{ottave}

\begin{ottave}
\flagverse{50}Mentre del fatto poi le dà contezza\\
Con quell'ambascia, e lingua di frullone\\
Fa (perché nulla mai si raccapezza)\\
Chi lo sente morir di passione;\\
Ma quella, c'a sentirlo è forse avvezza,\\
L'intende un po così per discrezione.\\
E qui finiscon le lezion di guerra,\\
Perch'ella non dà più ne in Ciel ne in terra.
\end{ottave}


Martinazza stava appunto instruendo quei soldati, che s'eran fuggiti per paura
de' suoi Caproni, quando arrivò un la sentinella con l'ambasciata di Calagrillo,
che la turbò tutta, ond'ella lasciò stare il dar lezione.
\begin{description}
\item[NON vagliono un pistacchio] Non son buoni a nulla. Si dice un pistacchio, un
lupino, una lisca; una sorba, una lappola, un pelo, un baiocco, un bagattino,
un picciolo, un zero, un'ette, un fico, cica, un iota, una chiarabaldana, un
puntal di stringha, o d'aghetto, una succiola, un soldo, un quattrino, un corno;
tutti per esprimer la poca stima, che si faccia d'uno, o d'alcuna cosa. E si
dice anche  non lo stimo il cavolo a merenda. Latino \textit{cicum}, \textit{titivillitium}.

\item[SPAVRACCHIO] Significa quel che accennammo sopra C. pr. stan. 40. E di
lì si dice fare spauracchio a uno per intendere spaventar uno, o mettergli paura
con fatti, o con parole.

\item[NON ne san biracchio] Non ne sanno nulla. Si dice anche straccio, brano, o
brandello, e simili.

\item[CAVARE un di scherma] Vuol dire far perder il filo del discorso a uno, ed è
  lo stesso che cavar di tema. Ma qui vuol dir' anche far lasciare star di schermire,
  e torna bene, perché Martinazza lasciò la scherma, ed uscì di tema, e di
  proposito per l'ira, che le cagionò l'ambasciata fattale in nome di Calagrillo.

\item[AMBASCIA] Affanno, o difficile respirazione d'alito, Fran. Sacc. N.139.
\textit{Tosto colui di chi erano stati, sen' andò con l'ambascia della morte a ripigliarli}.

\item[LINGUA di Frullone] Cioè che parla a salti, o a intoppi, come è il rumore,
  che fa il frullone, che è quell'ordingo, col quale, per via d'una ruota dentata, si
  separa la farina dalla crusca\footnote{il Frullone è un dispositivo meccanico per abburattare (da buratto ‘setaccio’) le farine. Attestato per la prima volta nel 1550 come “pulcherrimum instrumentum”, in un’opera di Girolamo Cardano, innovò il processo di macinazione dei grani e la panificazione.}.

\item[NON raccapezza nulla] Non intende nulla. Vedi sotto C. 6, stan. 101.

\item[L'INTENDE per discrezione] Quando per altro ci è noto un negozio, e che
  taluno ce lo racconti confusamente, o lo scriva con cattivi, e non intelligibili caratteri,
  sentito, o letto da noi, sogliamo dire; \textit{l'habbiamo inteso per discrezione},
  cioè habbiamo havuto la discrezione di non gli far ripetere il discorso, o di farlo
  di nuovo scrivere, già che per qualche informazione, che havevamo di quel fatto,
  intendevamo quel discorso, o scritto.

\item[NON dà ne in Ciel, ne in terra] È fuori di se. Non sa quel che ella si faccia.
  \textit{Non tocca ne ciel, ne terra}; dissero anche i Greci in questo proposito; e l'usa Luciano
  nel \textit{Pseudamante}, o vogliam dire \textit{Falso indovino}.
\end{description}
\section{Stanza LI --- LIII.}
\begin{ottave}
\flagverse{51}Tutta in un tempo vedesi cambiare\\
L'amante ingelosita Martinazza,\\
Hor ora è bianca, come il mio collare\\
Hor bigia, or gialla, or rossa, or paonazza.\\
Hor più rossa del c\ellipsis{18pt} d'uno scolare\\
Dopo ch'egli ha toccata una spogliazza;\\
In somma ella ha in sul viso più colori\\
Ch'in una bottega non han cento Pittori.
\end{ottave}

\begin{ottave}
\flagverse{52}Rabbiosa, il capo versa il ciel tentenna,\\
Quasi col piede il pavimento sfonda,\\
Hor si gratta le chiappe, hor la cotenna,\\
Hor dice al messaggiero che risponda,\\
Hor lo richiama mentr'egli è in Chiarenna,\\
Grida, e minaccia, e par che si confonda,\\
Mille disegni entro al pensier racchiude\\
Ienne inne, e nulla mai conchiude.
\end{ottave}

\begin{ottave}
\flagverse{53}Il guardo al fine in terra havendo fisso\\
N'un vasto mare ondegga di pensieri,\\
E lagrime diluvia sopra il viso\\
Grosse come sonagli da sparvieri,\\
Che lavandole il collo lordo, e intrifo\\
Laghi formano in sen di pozzi neri;\\
Al fin tornata in se, con la gonnella\\
S'asciuga, e al messaggier così favella.
\end{ottave}


Narra gli accidenti, ed i moti diversi cagionati in Martinazza dall'ambasciata
di Calagrillo, ed in fine Martinazza s'accinge a dar la risposta. L'Autore
descrive Martinazza per una solenne sgualdrina poiché dice, che è così grande il
sudiciume che ella ha addosso, che le lagrime che le cascano dagli occhi fanno
parerle nel collo tanti laghi di pozzi neri, cioè di cessi, i quali ella s'asciuga
con la veste.

\begin{description}
\item[BIANCA come il mio collare] Diventa bianca comie un panno curato, E queste
  mutazioni di colore son proprie d'uno che habbia l'animo alterato sì in male,
  come in bene, perché la palidezza, e sbiancamento denota sollevamento d'animo
  non essendo altro, che un mancamento di sangue, il quale per la paura se
  ne fugge al cuore, e lascia le vene del volto; ed il rosso denota ira perché questa
  cagiona ribollimento di sangue intorno al cuore, che scorre per tutte le vene,
  ma apparisce più nella faccia, perché quivi sono molte vene intercutanee, o vogliamo
  dire in pelle, che facilmente lo scuoprono; ed lo stesso effetto viene parimente
  dalla vergogna, la quale però si dice anche erubescenza.

\item[DOPO ch' egli ha toccata una spogliazza] Dopo che egli è stato frustato in sul
c\ellipsis{18pt} dal maestro. Spogliazza quali \textit{expoliatio}, spogliagione si dice quando il
Maestro fa cavare i calzoni a uno scolare, e mettendolo sopr'alle spalle d'un'altro,
gli dà con la sferza in sul c\ellipsis{18pt}. E quando gli dà nella stessa forma, ma
senza i mandar giù i calzoni si dice dare una mula, o un cavallo. A questo
c\ellipsis{18pt} frustato assomiglia l'Autore il viso di Martinazza quando le diventa rosso.
Una simile spogliazza, quasi come a ragazzo insolente, o minacciata là nel secondo
dell'Iliade a quel brutto mostaccio di Tersite, a cui Omero (secondo la
traduzione Latina ad verbum del Gifanio) fa dire da Ulisse: \textit{Ne posthac Ulyssi
  caput humeris adsit, \&c. Si non ego te comprehensum, \& charis vestibus exutum Pallioque,
  \& tunica, quae pudenda contegunt, Flentem veloces ad naves dimisero, Caedens e
  concione duris verberibus}\footnote{esistono differenti traduzioni dello stesso passo.}.

\item[TENTENNA il capo inverso il Cielo] Dimena la testa verlo il Cielo. Atto che
  si fa da molti quando accade loro cosa di poco gusto, quasi vogliano minacciare
  il Cielo perché cagiona loro quella tal disgrazia: i Latini dissero; \textit{caput quatere}.

\item[SFONDA il pavimento col piede] Per la collora batte i piedi in terra così fortemente,
  che fa quasi rovinare il Palco. Properzio. \textit{Et crepitum dubio suscitet ira pede}.

\item[SI gratta le chiappe, e la cotenna] Si gratta le natiche, il capo, che è un atto
  solito farsi per lo più dalle donne quando succede loro qualche disgrazia. Per \textit{cotenna}
  s'intende il capo, perché la pelle del capo dell' huomo si dice cotenna; se ben
  vuol dire la pelle del porco, ed impropriamente si dice la pelle d'ogni animale
  vedi sopra C. 2. stan. 64. ed in ciò noi ci conformiamo co' Latini, che dicono \textit{cutis}
  la pelle del capo dell'huomo, e dicono anche \textit{cutem detrahere} per scorticare qualsivoglia
  pelle, il proprio vocabolo della quale è \textit{pellis}.

\item[QLVAND' egli e in chiarenna] Quand' egli è molto lontano. \textit{In oras longinquas},
  e da questo noi diciamo: \textit{Quand'egli e in orinci}. Usato dal Davanzati nel Tacito.

\item[IENNE inne] Di questo termine ci serviamo per esprimere uno che s'affanni
  di operare, e non conchiuda. Viene da quello stento che fanno i ragazzi quando
  imparano a compitare; quasi dica compita compita, e mai rileva, ed ha lo stesso
  significato, e forza che \textit{ponza ponza} detto sopra C, 4. stan. 80.

\item[SONAGLI da sparvieri] Intende lagrime grosse come sono i sonagli, che s'appiccano
  a i piedi degli sparvieri; comparazione iperbolica, ma assai usata per intender
  grosse lagrime. AEn.11. \textit{It lacrymans guttis. humectat grandibus ora}.
  \textit{Sonagli}, e \textit{campanelli} chiamiamo quelle gallozzole, che fa l'acqua quando piove, cadendo sopra
  i rigagnogli; o altrimenti nello scorrere.

\item[POZZI NERI] Bottini. Quei luoghi sorterranei, entro a' quali si getta ogni
  sorta d'immondizia; ma propriamente \textit{pozzo nero} è bottino, o fogna smaltitoia
  del cesso, a differenza di quella degli acquai.
\end{description}
\section{STANZA LIV --- LVI. }
\begin{ottave}
\flagverse{54}Torna, e rispondi a questo Scalzagatto, \\
Che si crede ingoiar con le parole: \\
Ch'io non so quel ch'ei dica, e s'egli è matto \\
Non ci posso far' altro, e me ne duole, \\
Poi circa alla domanda, ch' egli ha fatto; \\
Che gli darò Cupido, e ciò ch'e' vuole, \\
Se con la spada in mano, o ver con l'asta\\
Prima di guadagnarlo, il cor gli basta.
\end{ottave}

\begin{ottave}
\flagverse{55}Però s'in questo mentre umor non varia,\\
Domani al far del dì facciami motto,\\
E s'io gli farò dar le gambe all'aria,\\
Quella sua landra ha da pagar lo scotto,\\
Mia se la sorte fosse a me contraria\\
Vuol c'a me tocchi andar col capo rotto,\\
Prenda Cupido allor, ch'io le prometto\\
Lasciarglielo segnato, e benedetto.
\end{ottave}

\begin{ottave}
\flagverse{56}Ciò detto parte, e quei ch'era huomo esperto \\
(Essendo stato Cavallaro, e Messo) \\
Al Cavaliere ad unguem fa il referto \\
Di quel che Martinazza gli ha commesso;\\
Ed in viso vedendolo scoperto,\\
Quest' ha bisogno dice d'un buon letto,\\
Perch'egli è duro, e non punto pupillo,\\
Lo conosco bensì, gli è Calagrillo.
\end{ottave}


Martinazza manda a dire a Calagrillo, che gli dara Cupido, s'ei lo guadagnerà
con l'armi; ma se ella vince, vuol Psiche: la ronda porta l'ambasciata, e
riconosce Calagrillo.

\begin{description}
\item[SCALZAGATTO] Huomo vile, Guidone.

\item[CREDE ingoiar con le parole] Crede farci paura con le chiacchiere. E si dice:
  \textit{Mangiar vivo uno con le parole}.

\item[S'IN questo mentre umor non varia] Se fra tanto non si muta d'opinione.

\item[LANDRA] Sgualdrina, donna di bordello, ed intende Psiche; Landra è epiteto
  conveniente alle più infami, e laide meretrici, quasi \textit{latrina}, che la fogna,
  e ricettacolo di tutte le schiferie.

\item[HA da pagar lo scotto] Ha da pagare la pena. Pagar lo scotto vuol dire pagar
  all'oste quello, che s'è mangiato, pagar la sua porzione, la sua quota; Terenzio
  disse \textit{symbolam dare}. Ma qui intende il Latino \textit{poenas luere}, Dan. Purg. C.~30.
  \begin{verse}
    \backspace L'alto fato di Dio sarebbe rotto
    Se Lete si passasse, e tal vivanda
    Fusse gustata senz'alcuno scotto
    \backspace Di pentimento, che lagrime spanda.
  \end{verse}
\item[ANDAR col capo rotto] Andar con la peggio; cioè ch'io perdessi il duello.

\item[SEGNATO, e benedetto] Liberamente, e senz'eccezione alcuna. Fran.Sacc. Nou.
  104. \textit{Vattene ogni hora pur Segnato, e benedetto}. Esprime un dar via qualcosa, o
  uno volentieri, e con anime di non rivolerlo; Un licenziare affatto.
Virc. Egl. \textit{longumque vale, vale, inquit Iola}.

\item[CAVALLARO] È un famiglio, che porta le citazioni criminali mandate da
  i Ministri forensi, chiamato \textit{Cavallaro}, perché stante il largo dominio, e giurisdizione,
  che ha il suo tribunale, e necessario che vada a cavallo; \textit{Il Messo} è quello
  che porta le citazioni civili pure de i Ministri forensi, e fa i gravamenti, ec. e
  non va a cavallo, perché non gli occorrono lunghe gite, come al Cavallaro; a
  Roma si domanda \textit{Cursore}; nome simile al \textit{Viator}, col quale era disegnato dagli antichi
  Romani il donzello, o fante pubblico.

\item[AD unguem] Per appunto\footnote{Esattamente, precisamente, magari ``spaccando il capello''.}. Frase latina usata assai da noi.

\item[FA il referto] Riferisce. Frase curiale, che vuol dire quando il Cavallaro, o
  Messo havendo data la citazione, riferisce in atti d'haverla data, che dicono anche
  \textit{fare il rapporto}. E l'Autore si serve di questa frase (per altro non usata in
  questi termini) perché ha detto, che questa Guardia era stato Cavallaro, e Messo.

\item[EGLI ha bisogno d'un buon lesso] E' carne dura, e però ha bisogno di bollire
  assai nell'acqua. È detto vulgato per esprimere un'huomo, che sa il conto suo,
  forte, gagliardo, e difficile a superarsi, che diciamo: \textit{Osso duro} per esempio; Il
  tale ha tolto a rodere un'osso duro.

\item[NON è pupillo] Non ha bisogno di Tutori, suona lo stesso che \textit{ha bisogno a un
  buon lesso}, se bene \textit{non è pupillo} si ristringe a saper fare i fatti suoi, ed \textit{ha bisogno
  d'un buon lesso} esprime saper fare i fatti suoi, ed esser bravo, e valente in ogni
  cosa.
\end{description}

\section{STANZA LVII. --- LXII.}
\begin{ottave}
\flagverse{57}Ma qui la dama, e Calagrille resti,\\
Quest'altro giorno rivedremogli poi. \\
Il passo meco hora ciascuno appresti \\
Per giunger il Fendesi, e gli altri duoi, \\
Che seguitaron come voi intendesti\\
Perlon, che sen'andò pe' fatti suoi,\\
Che troveremgli, se venir volete\\
Più presto assai di quel che vi credete.
\end{ottave}

\begin{ottave}
\flagverse{58}Che giò giò se ne vanno giù pel piano \\
Sbattuti com'io dissi dalla fame; \\
Ma non son iti ancora un trar di mano \\
Che senton razzolar fra certo strame; \\
Perciò con l'armi subito alla mano \\
Corron dicendo: Qui c'è del bestiame, \\
Sì che quando crediamo di tirar minze, \\
Il corpo forse caverem di grinze.
\end{ottave}

\begin{ottave}
\flagverse{59}Cursosi quei che fusse di vedere \\
Dentr'a una stalla inabitata entraro, \\
E vedder, ch'era un'huom posto a giacere \\
Sopr'alla paglia a guisa di somaro; \\
Accanto havea da mangiare, e bere, \\
E gli occhi distiliava in pianto amaro, \\
E tra i disgusti, e il vin ch'era squisito \\
Pareva in viso un gambero arrostico.
\end{ottave}

\begin{ottave}
\flagverse{60}Questo è il Piaccianteo già subblimato\\
Al grado honoratissimo di spia,\\
Quel che per soddisfar tanto al palato\\
Ha fatto in quattro dì Fillide mia,\\
E lì con la sua spada s'è impiattato,\\
Dell'honor della quale ha gelosia,\\
Che havendola fanciulla mantenuta\\
Non gli par ben ch'ignuda sia tenuta.
\end{ottave}

\begin{ottave}
\flagverse{61}Ma perché un huom più vil mai fe natura,\\
Si pente esser'entrato in tal capanna,\\
Però che a starvi solo egli ha paura,\\
Che non lo porti via la Trentancanna,\\
E perché tutto il giorno quant'ei dura,\\
Egli ha il mal della lupa, che lo scanna,\\
Non va mai fuor s'a cintola non porta\\
L'asciolver col suo fiasco nella sporta.
\end{ottave}

\begin{ottave}
\flagverse{62}Ovungne egli è, d'untumi fa un bagordo, \\
Ch'ognor la gola gli fa lappe lappe; \\
Strega le botti di lor sangue ingordo, \\
E le sustanze usurpa delle pappe; \\
Aggira il beccafico, e pela il tordo,\\
E a poveri cappon ruba le cappe,\\
E prega il Ciel, che faccia che gli agnelli\\
Quanti le melagrane, habbian granelli.
\end{ottave}


L'Autore torna a parlare di Perlone, e degli altri, che lasciò sopra C.4. stan.28.,
i quali per la fame s'andavano ailontanando dal Campo, e narra, che costoro
trovarono in una Capanna quel Piaccianteo, che fu da Bertinella mandato
fuori a spiare, come vedemmo sopra C. 3. stan. 45. il quale haveva seco da mangiare,
e da bere. Nella presente Ottava 62. descrive assai vagamente la ghiottornia
di Piaccianteo.

\begin{description}
\item[GIÒ giò] Adagio adagio. È la figura \textit{aphaeresis}.
\item[RAZZOLARE] Fregare, raspare, fragare; ec. Qui vuol dir quel romore
  che fa la paglia, o cosa simile, quando è maneggiata in massa.

\item[STRAME] Paglia, fieno, o altra materia simile per cibo delle bestie. Vedi
  sopra C, 4. stan. 2.

\item[TIRAR minze] Vuol dite stentare. Ma s'intende morire: Si dice milza, ma
  il Poeta si serve della licenza, e seguita intanto i più che dicono; \textit{minza} e non
  \textit{milza}.

\item[CAVARE il corpo di grinze] Mangiare assai, che in questa maniera gonfiando
  il ventre, si levano le grize al corpo. Plauto disse \textit{ventrem distendere}. Virg.
  Georg. \textit{distendunt nectare cellas}, cioè empiono.

\item[PAREVA un gambero arrostito] Era rosso in viso come sono i gamberi fritti:
  similitudine assai usata per esprimere un rosso in viso, per il soverchio vino
  bevuto.

\item[HA fatto Fillide mia] Ha finito, ha consumato, o mandato male tutto il suo
  havere. E' detto ianadattico \textit{Filide} per fine, Ma per avventura ha la sua origine
  da Fillide figliuola di Licurgo Re de i Traci, la quale s'innamorò di Demofonte
  figliuolo di Tefeo, e di Fedra, quando nel tornare dalla guerra di Persia
  essendo stato spinto da i venti contrarj nel Regno di Tracia, fu da Fillide ricevuto
  con segni di grande amorevolezza; ma egli senza riguardo a i benefizzi da
  essa ricevuti, sen'andò; per lo che Fillide disperata s'impiccò. Da questa disperata
  morte di Fillide, quando diciamo \textit{far Fillide}, intendiamo finir la vita, e
  finire la roba.

\item[IMPIATTATO] Nascosto, Vedi sopra C. 2. stan. 60.

\item[DELL'honor della quale ha gelosia] Ha gelosia dell'honor della sua spada, perché
  havendola tenuta sempre fanciulla, cioè vergine (che s'intende non mai
  adperata) stima poco honefto il lasciarla vedere ignuda, come è veramente poco
  onesto a una vergine lasciarsi vedere ignuda. E con tali scherzi vuol dire, che
  costui era codardo, e vile, e di poco animo, ed uno di coloro che \textit{umbram
    suam metuunt}.

\item[TRENTANCANNA] Una bestia ch'ingoia o tracanna trenta per volta;
  ed è una di quelle larve immaginarie inventate dalle Balie per far paura a i bambini,
  come bau, befana, e simili dette altrove.

\item[IL male della Lupa] È inteso da noi per una infermità, che fa stare il paziente
  in continua fame, ed i Medici la chiamano \textit{fame canina}.

\item[CHE lo scanna] È un termine che significa grandezza di passione, ed ha forza
  d'avanzare ll superlativo, perché dicendosi, \textit{Ha una fame, una sete, un desiderio,
    ec. che lo scanna}, s'intende fame, sete, o desiderio grandissimo, e più, Vedi sopra C.4. stan.24.

\item[ASCIOLVERE] Solvere il digiuno; sdigiunarsi, fare colazione. Vedi sopra
  C.1. stan. 35. ma qui è preso per mangiamento in generale, cioè per la materia
  da mangiare.

\item[UNTUMI] Intende roba da mangiare, che sia unta, come polli, carne,
  pesce, ec.

\item[BAGORDO] Bagordare, o far bagordo vuol dir Giostrare, giuocar d'armi,
  far conviti, ed ogni altra sorta d'adunanza festiva, ancorché non d'armi. E
  potrebbe dirsi scherzando bagordo, quasi \textit{vagus ordo}, confusione ordinata; onde
  da quel numero di gente in confuso, la quale interuiene a tali bagordi, pigliamo
  poi \textit{bagordo} per commistione di varie cose, come nel presente luogo, che intende
  mescolanza d'untumi. Vedi sotto C. 6. stan. 2. Del resto \textit{Bagordo} viene da \textit{Bigordo},
  che vuol dire \textit{Asta}. E Bigordare trovasi presso gli antichi; per correr la
  lancia, Fazio degli Uberti nel Dittamondo al Canto 32,
  \begin{verse}
    \backspace Giovani bigordare alli chintani,
    E gran tornei, e una, e altra Giostra
    Farsi veder con giuochi nuovi, e strani.
  \end{verse}
  Poi si disse \textit{Bagordo}, e \textit{Bagordare}; e si trassero queste voci a significare ogni sorta
  di stravizio, e di ricreazione. Che Bigordo voglia dire \textit{Asta}, ci è l'esempio di
  Giovanni Villani lib. 7. rubric. 132. \textit{E recossi palio di drappo ad oro sopra capo
    Messer Amerigo di Nerbona portato sopra bigordi per più Cavalieri}. Folgore
  da San Gimignano\footnote{Folgóre da San Gimignano, pseudonimo di Giacomo di Michele o Jacopo di Michele secondo fonti diverse (San Gimignano, 1270 – San Gimignano, 1332), poeta comico-realistica, uno stile basso con tematiche laiche e mondane. } Rimatore antico citato dai Conte Ubaldini nelle Annotazioni a
  Messer Francesco da Barberino: \textit{E rompere, e ficcar bigardi, e lance}.

\item[LA gola gli fa lappe lappe] Signitica desiderar ardentemente di mangiare. Voci
  nate dal suono che fa il palato con la lingua, e con le labbra quand'uno biascia senza
  havere nulla in bocca, che è segno di fame, qual suono pare che dica lappe
  lappe; donde poi il verbo allampare, che vuol dire haver gran fame. Così \textit{Lapto}
  in Greco, che è lo stesso, che \textit{Lambo} in Latino, è fatto dal medesimo suono.

\item[STREGA le botti] Stregare vuol dir succiare il sangue, perché dicono, che le
  Streghe succiano il sangue a i bambini; e però dicendo \textit{strega de botti} intende succia
  il sangue delle botti, che è il vino, del quale è \textit{ingordo}, cioè avidissimo.

\item[VSVRPA le sustanze dele pappe] Divora la carne, che è la sostanza del brodo,
  del quale si fanno le pappe.

\item[AGGIRA il beccafico, e pela il tordo] Aggirare, e pelare, metaforicamente parlando,
  significa ingannar'uno, e cavargli da dosso danari, come habbiamo accennato
  sopra in questo C. stan. 9. Il Poeta scherzando piglia detti due verbi nel lor
  vero senso, ed intende girar nello spiede i beccafichi, e pelare i tordi per quocergli,
  e mangiarsegli.

\item[LEVA le cappe ai capponi] Cioè divora la pelle de' capponi.

\item[E PREGA il Ciel che faccia, che gli agnelli, ec.] \makebox[1pt]{} Dove giù agnelli hanno solamente
  due granelli, (cioè testicoli) vorrebbe, che ne havessero quanti n'hanno le
  melagrane. E così descrive un solenne ghiotto; e crapulone. Similmente un certo
  Filofleno solenne mangiatore, siccome riferisce Aristotile lib.\ 3.\ delle Morali
  indirizzate a Nicomaco, cap.\ 10.\ desiderava d'avere il collo più lungo d'una
  grue supponendo, che così fusse per essere il gusto maggiore.
\end{description}
\section{STANZA LXIII. --- LXVI.}
\begin{ottave}
\flagverse{63}Vedenda quivi comparir repente \\
L'infolite armi, sbigottisce il ghiotto,\\
E dal timor ch'egli ha di tanta gente \\
Trema da capo a pié, si piscia sotto: \\
Con tutto ciò digruma allegramente, \\
E spesso spesso bacia il suo barlotto, \\
E acciò stremata non gli sia la vita \\
Non dice men: degnate, o a ber gli invita.
\end{ottave}

\begin{ottave}
\flagverse{64}Ma i Cavalier famosi a quel plebeo,\\
Che non profferì lor della rovella,\\
Furon per insegnare il Galateo\\
Con battergli già in terra una mascella, \\
Chi sei? (diss' un di loro) e Piaccianteo, \\
Ch'è un pover huom, risponde, e in quella Cella \\
Molt' anni in astinenza ha consumati \\
Per penitenza de' suoi gran peccatei.
\end{ottave}

\begin{ottave}
\flagverse{65}E quei soggiunge: Mi rallegro, e godo\\
Che voi facciate bene, e vi son schiavo;\\
Ma s'il patire è fatto a questo modo,\\
Penitente di voi non è più bravo,\\
Tal ch'io per me vi mando a corpo sodo\\
Non nel settimo Ciel, ma nell'ottavo,\\
Donde ai mondani, e a me che sono il capo,\\
Pisciar potrete a vostra posta in capo.
\end{ottave}

\begin{ottave}
\flagverse{66}Ma perch al certo Vostra Reverenza,\\
Ch'è stenuata, come un Carnovale,\\
Havrà fatta fin' hor tant' astinenza,\\
Che basti a soddisfar a ogni gran male,\\
Hor puo lasciar a noi tal penitenza,\\
Acciò baciam la terra del boccale,\\
Per più mondi accostarsi a quest avanzi\\
Delle reliquie, ch'ell'ha qui dinanzi.
\end{ottave}


Piaccianteo vedendo comparir coloro armati, hebb'un grande spavento, ma
non per questo abbandono ii mangiare, anzi si studiava più per il timore, che
haveva, che coloro non gli stremassero la provvisione. Domandato poi, chi egli
era, rispose esser uno, che faceva penitenza de' suoi peccati in quella cella con digiuni,
e astinenze: Dalla qual risposta accortisi, che egli era un birbone, uno di loro
scherzando sopr'al digiunare, gli dice, che lasci un po fare il medesimo digiuno,
ed astinenza ancora a loro.
\begin{description}
\item[SBIGOTTISCE] Spaurisce. Si perde d'animo. Vedi sopra C.2. stan.28. Dan. Inf.C.22.
  \begin{verse}
    Così mi fece sbigottir lo Mastro,
    Quand'i gli vidi sì turbar la fronte.
  \end{verse}

\item[GHIOTTO] Goloso; Avido di mangiar del buono. Lat. \textit{gluto}.
\item[SI piscia sotto] Vuol dire haver gran paura. Vedi sopra in questo C. stan. 3.
\item[DIGRUMARE] Intendi mangiare; se bene digrumare è il masticare, che fanno
  le bestie dal pié fesso, che si dice anche ruminare dal Latino, che però chiama
  \textit{ruminantia} le dette bestie, come habbiamo accennato sopra C. 4. stan. 6.5 e vedremo
  sotto, C, 6. stan. 5.
\item[BACIA il barlotto] Beve. Barlotto è un vaso di legno di figura simile al barile,
  ma è assai minore, perché sarà di tenuta o più, o meno fino a dieci fiaschi, che tenendo
  dieci fiaschi si chiama mezzo barile. Qui pero non intende strettamente questa specie
  di barlotto, ma un vaso da vino portatile addosso, comunque si sia o di vetro, o di
  terra, o una Zucca, anzi stimo che intenda più tosto di terra, perché più giù
  dice \textit{baciamo la terra del boccale}.
\item[STREMARE] Vale scemare, sminuire, quasi ridurre allo stremo.
\item[DEGNATE] È un modo di dire usato da coloro che mangiano all'osteria,
  quando passa intorno alla loro tavola alcun loro conoscente, e dicono: \textit{degnate},
  cioè degnatevi di bere. E perché è termine usatissimo dalla plebe, il Poeta fa
  che costoro si maraviglino, che Piaccianteo non l'usi, e fa prendere argumento,
  che egli non l'usi per paura, che non sia accettato l'invito, e scematagli la
  provvisione.
\item[CAVALIERi famosi] Cavalieri illustri, e di fama. Ma qui \textit{famoso} non deriva
  da fama, ma allude a fame, e vuol dir Cavalieri affamati.
\item[PLEBEO] Vuol dire huomo di Plebe; ma ce ne serviamo anche per intendere
  huomo infame, senza honore, e senza creanza. Qui se ne serve per contrapposto
  di Cavalieri famosi, e vuol dire, che si come quelli erano famosi, cioè affamati,
  costui era infame, cioè senza fame, perché havea ben mangiato.
\item[NON profferì della rovella] Non offeri nulla; usandosi spesso il verbo \textit{profferire},
  In vece del verbo \textit{offerire}; e la parola \textit{della rovella} è posta a maggior' emfasi per
  esprimere non offerì nulla, ne meno una cosa nociva.

\item[INSEGNARE il Gatateo] Insegnare le creanze, i buoni termini. Galateo è intitolata
  un' Operetta di Monsignor Gio. della Casa, la quale insegna le buone
  creanze.

\item[BATTERGLI giù una mascella] Dargli un taglio nel viso, e fargli cadere una
  ganascia.

\item[IO vi son schiavo] Vi son servitore. E' un detto usato, quando alcuno faccia,
  bella azione, che meriti lode, per esempio Il tale fece una bellissima Orazione;
  io gli son schiavo. I Caporali nella vita di Mecenate dice,
  \begin{verse}
    E si legge ch'Augusto un dì gli disse:
    Capitan Mecenate io vi son schiavo.
  \end{verse}

\item[NELL'ottave Ciclo] L'Autore tenendo l'opinione, che i Cieli sieno otto dice,
  che costui merita d'andare nell'ottavo, cioè nel supremo; perché ha fatta tanta
  penitenza, che merita il sovrano posto nel Cielo.

\item[MONDANI] Intende peccatori. Coloro che sono dediti a i piaceri mondani.
\item[STENVATO come un Carnovale] Magro, come un Carnovale: comparazione
  ironica, che vuol dire Grassissimo, come si figura il Carnevale.
\item[BACIAMO la terra dei boccale] Baciar la terra è un'atto, che si fa dalle persone
  divote per umiltà, Ma costui sostenendo l'equivoco del far penitenza, dopo
  haver detto, che gli piace il modo del digiunare, che fa Piaccianteo, dice che
  vuol ancor'egli far'un'atto d'uiilta con baciar la terra, ma però quella del
  boccale, cioè bere. \textit{Boccale} è un vaso di terra capace della meta d'un fiasco, ma
  si piglia per tutti li vasi di terra a quella foggia, ancorché maggiori, e di tenuta
  di un fiasco anche più.

\item[PER accostarsi più mondi] Per accostarsi più puri, havenod fatto l'atto di penitenza,
  e d'umiltà con baciar la terra.

\item[RELIQV1E] Avanzi, fragmenti; e scherzando sempre con la bontà, e perfezione
  del penitente, par che pigli \textit{reliquie} nel senfo speciale, che l'intendiamo
  noi, cioè ossa, ed altri fragmenti di Santi, ed ei vuol poi dire gli avanzi del di
  lui mangiamento. Latino \textit{mensae relique}. Ed in quest'ottava l'equivoco è sostenuto
  da costui in mostrare a Piaccianteo di credere, che egli fusse un penitente,
  che stesse quivi per fare astinenza, come haveva detto; e per indurlo a contentarsi,
  che essi ancora s'accomodino con lui a far la penitenza nella stessa maniera,
  che faceva egli.
\end{description}

\section{STANZA LXVII. --- LXVIII.}
\begin{ottave}
\flagverse{67}Qual madre, che ripara il suo figliuolo, \\
Ch'è sopraggiunta da mordaci cani, \\
Ei cuopre tutto con il ferraiuolo,\\
Ed eglino gli danno in su le mani; \\
E col lazo del Piccaro Spagnuolo, \\
Che dalla mensa vuol tutti lontani, \\
Acciò poi a tal cose non arrivi, \\
Con due calci lo fan levar di quivi.
\end{ottave}

\begin{ottave}
\flagverse{68}Così fan carità di più rigaglie\\
Oltr' ad un'Oca grossa arciraggiunta;\\
Ma vedendo più là fra quelle paglie\\
D'un pezzo d'arme luccicar la punta,\\
E del giaco scappare alcune maglie\\
Da quella sua casacca unta, e bisunta,\\
Insospettiron, com'un'altra volta\\
Potrà sentir chi volentier m'ascolta.
\end{ottave}


Piaccianteo vedendo, che costoro s'accostavano per torgli la roba, cerca di
salvarla, coprendola col ferraiolo, ma essi con una mano di calci l'allontanarono,
e d'accordo si messero a mangiare: Ma intanto, osseruato, che egli era armato,
presero sospetto, e fecero quello, che sentiremo sotto nel C. 8. stan. 60.

\begin{description}
\item[RIPARARE] Rimediare. Val per difendere. Ed in questa comparazione
  imita Dante Infer. C. 23. che dice:
  \begin{verse}
    Come la madre, ch' al romore è desta,
    E vedo preso a se le fiamme accese,
    \backspace Che prende il figlio, e fugge, e non s'arresta,
    Havendo più di lui, che di se cura;
    Tanto che solo una camicia vesta.
  \end{verse}

\item[FERRAIVOLO] Mantello. Un panno ridotto tondo, e adattato a coprire
  tutta la persona sopra agli altri abiti, mettendolo in su le spalle.

\item[LAZO del Piccaro Spagnuolo] Gli zingari, quando s'abbattono nel corrivo;
  per truffarlo, e rubargli qualcosa, che gli habbiano veduta, trovano diverse invenzioni,
  come di farlo ballare, o cantar con loro, o fargli mettere in capo
  qualche ordingo, che gli occupi la vista, o con fargli metter il capo in un'armario
  a vedere il Mondo nuovo, e molt'altre invenzioni per distrarlo, ed haver
  comodità di rubargli quel che hanno disegnato, mentr'egli astratto da tali operazioni
  non bada a quel che gli facciano d'attorno; come spesso veggiamo seguire
  in commedia, che il servo astuto, per truffare il servo stolto si vale di simili
  astuzie. E questo si dice \textit{il lazo del Piccaro Spagnuolo}, cioè invenzione dello Spagnuolo
  furbo.  Donde poi \textit{lazo}, \textit{lazeggiare} significa qualunque azione, che facciano
  i Comici per esprimere il lor pensiero. E \textit{lazo}, che in Spagnuolo significa
  \textit{laccio}, si prende da noi per quel che i Latini direbbero \textit{captio}, \textit{sophisma}, \textit{commentum},
  \textit{technae}, \textit{versuria}, \textit{fallacia}, \textit{artes}, \textit{doli}, Ed in questo significato va profferito con
  la 'z' dolce, e non cruda, ed aspra, perché con la cruda significa sapore aspro,
  ed astringente, come quel della prugna, della sorba mal matura, e simili, che i
  medici dicono \textit{acido}; Dante Inf, C. 15,
  \begin{verse}
    Ed è ragion, che là tra i lazzi sorbi
    Si disconvien fruttare il dolce fico
  \end{verse}
  La lazzeruola\footnote{\textit{Malpighia emarginata} DC., pianta arbustiva tropicale, originaria del Nuovo Mondo. Non ha molta diffusione in Italia, si chiama da noi con i suoi nomi spagnoli di \textit{acerola}, \textit{manzanita}, \textit{semeruco}, o anche \textit{ciliegia delle Barbados}.
  }, perché è frutta di sapore, \textit{lazzo}, cioè \textit{acido} dicesi da gli Spagnuoli
  \textit{azerola} quasi dal Lat. diminutivo \textit{acidula}.

\item[FAR carità] Fra i Bacchettoni s'intende mangiare insieme. E tra gli antichi
  Cristiani, i conviti, che si facevano a' Poveri; di limofine, si domandavano \textit{Agapae},
  cioè \textit{Caritadi}. E \textit{Pietanza}, voce conservatasi tra' Frati, e tra le Monache,
  significa piatto, o mangiare offerto dalla pietè, e carità de' benefattori; non
  significando altro \textit{Pietanza}, che \textit{Pietà}. Il Beato Fra Iacopone: \textit{Vorria trovar
    alcuno, Che avesse pietanza De lo mio cor afflitto}.
\item[ARCI raggiunta] \makebox[1pt]{} Grassissima. Uccello soprammodo grasso si dice raggiunto.
\item[LUCCICARE] Risplendere; Rilucere. Viene da Lucciola.
\item[CASACCA] Parte d'abito da huomo, che copre la persona da mezza la pancia
  in su fino al collo. Così \textit{Casula} in Latino; se bene altra sorta di veste, diversa
  dalla Casacca, fu detta così, perché copre tutta la persona a guisa, che fa la
  casa, se crediamo a Isidoro nel lib. 19. delli Origini, al cap. 24.
\end{description}
\section*{FINE DEL QVINTO CANTARE.}

\chapter{Sesto Cantare}
\begin{argomento}
Nel tenebroso centro della Terra,
Ove regna Plutone entra la Strega,
E vuol che seco per finir la guerra
Di Malmantile entri l'Inferno in lega.
Fanno concilio i mostri di sotterra,
Ove ciascun buone ragioni allega;
Certa al fin le promette l'assistenza,
Rend' ella grazie, e fa di lì partenza.
\end{argomento}

\section{Stanza I --- III}

\begin{ottave}
\flagverse{1}Miser chi mal' oprando si confida:\\
Far' alla peggio, e ch'ella ben gli vada,\\
Perché chi piglia il vizio per sua guida,\\
Va contrappelo alla diritta strada.\\
E benché qualche tempo ei sguazzi, e rida\\
Col vento in poppa in quel che più gli aggrada,\\
E' vien poi l'ora, ch'ei n'ha a render conto,\\
E far del tutto, dondola, ch' io sconto.
\end{ottave}

\begin{ottave}
\flagverse{2}Di chi credi Lettor tu qui ch'io tratti?\\
Tratto di Martinazza iniqua Strega,\\
C'ha più peccati, che non è de' fatti,\\
E pel Demonio ogni ben far rinnega;\\
Di darsi a lui già seco ha fatto i patti,\\
Acciò ne' suoi bagordi la protega,\\
Ma state pur; perché tard, e per tempo\\
Lo sconterà; da ultim' è buon tempos.
\end{ottave}

\begin{ottave}
\flagverse{3}Non si pensi d'haverne a uscir netta;\\
S'inrighi pur col Diavol, ch'io le dico,\\
Se forse haver da lui gran cose aspetta,\\
Che nulla dar le può, ch'egli è mendico,\\
E quand' ei possa, non se lo prometta,\\
Perch'ei, che sempre fu nostro nimico,\\
We può di ben verun vederci ricchi,\\
Vna fune daralle, che l'impicchi.
\end{ottave}

Il Poeta havendo pensiero di narrar la gita, che fece Martinazza al Regno di
Plutone per muoverio ad aiutarlo a diloggiar Baldone da Malmantile, ed a
gastigare Gambattorta, e Baconero, fa l'introduzione al presente Cantare con
una riflessione morale ponderando, che quei, che opera male, non può sperare
d'haver mai bene, e principiando come l'Ariosto C. 6.
\begin{verse}
  Miser chi mal' oprando si confida
\end{verse}
Conchiude, che Martinazza, la quale non fa se non sciagurataggini, e s'è data
al Diavolo, non può sperar d'haver a haver bene, perché il Diavolo è nimico
del genere umano, e non può vedergli ben veruno.

\begin{description}
\item[FAR alla peggio] Far' ogni male senza riguardo alcuno.
\item[VA contrappelo] Non va per il verso buono. Va al contrario di quello, che
  deve fare per andar per la diritta via. Sen. epist. 122. \textit{Omnia vitia contra naturam
    pugnant, omnia debitum ordinem deferunt; hoc est luxuriae propositum gaudere perversis:
    nec tantum discedere a recto, sed quam longissime abire; deinde etiam e contrario stare}.
  Si dice anche andare a ritroso dal Latino \textit{retrorsum}. Dan. Purg. C. 10, in simil
  proposito dice.
  \begin{verse}
    O superbi Cristian miseri, e lassi,
    Che della vista della mente infermi
    Fidanza havete ne i ritrosi passi.
  \end{verse}
 E la metafora d'andar contrappelo è tolta da i pezzi di panno, o di pelle pelosa,
 che in cucirle insieme s'osserva, che il pelo vada tutto per un verlo, acciocché
 non si confacciano. A tastar un panno, o pelle pelosa per il verso, che va il
 pelo, torna più facile, e non si trova resistenza alcuna, come a andar contro a
 pelo.

\item[SGUAZZI] Goda allegramente.

\item[COL vento in poppa] Secondo che ei desidera: Come succede quando si ha il
  vento in poppa della nave: e significa \textit{i negozzj vanno bene}. I Greci pure dissero
  \textit{Secundo vento navigare}.

\item[DONDOLA ch'io sconto] Vuol dire sconterà il buon tempo, che ella si è data,
  provando altrettanti disgusti, E' detto usato dalla Plebe, nella quale e nato; essendo
  lato detto da un macellaro, a cui era stata rubata in più volte gran quantità
  di carne, ed essendo fatto ritrovato il ladro, fu impiccato, ed il macellaro
  vedutolo appeso alle forche disse: \textit{Dondola, ch'io sconto}; intendendo a vederti
  dondolare Sconto il debito, che hai meco per la carne rubatami. Dondolare, è
  lo stesso che ciondolare, come appunto fa l'impiccato; e tal Verbo \textit{dondolare}
  piglia il nome da quel don don, che fa il suono delle Campane. E da quello medesimo
  suono, che faceva quel tanto rinomato vaso dell'Oracolo di Giove, che
  era in Dodona Città dell'Epiro, stima, e con molta ragione, derivarsi il nome
  di Dodona Abramo Berkelio Olandese\footnote{Abraham van Berkel, 1639-1686, figura poco nota dell'illuminismo radicale olandese, contemporaneo del Lippi e Minucci.} nelle Osservazioni al Frammento dell'Opera
  originale di Stefano de Urbibus\footnote{Stefano di Bisanzio, di lui si ignorano i dati biografici, comunque con tutta probabilità era un grammatico costantinopolitano vissuto nel VI secolo. Fra le sue opere, pervenuteci in forma piuttosto frammentaria, un dizionario geografico in 50 o 60 volumi.}. \textit{Dondolare}, o \textit{dondolarsela} vuol dire Starsene
  a sedere senza far nulla, di dove \textit{Dondolone} vuol dire un perdigiorno. Quindi
  un moderno Poeta intendendo di questi tali disse:
  \begin{verse}
    Voi dal notturno al mattutin crepuscolo
    Vi dondolate, e fate a tu me gli hai,
    Ne conchindete o proponete mai,
    Se non rovine al popolo minuscolo.
  \end{verse}

\item[C'HA più peccati, che non è de' fatti] Ha più peccati ella sola, che non sono
  quelli, che sono stati fatti, o commessi da tutto il mondo insieme infino a ora.

\item[BAGORDI] Festeggiamenti. Vedi sopra C. 5, stan. 62.

\item[TARDI, o per tempo] Diciamo anche \textit{Tardi, o accio} (cioè avaccio, parola,
  antica, rimasa in contado, che vale tosto) o vero \textit{tardi, o avale}; che dissero
  ancora gli antichi \textit{aguale}; cioè ora, in questo punto; vuol dire; questo seguirà una
  volta o presto, o tardi. Lat. \textit{serius, ocyus}.

\item[DA ultimo è buon tempo] Da ultimo verra il sereno. \textit{Post nubila Phoebus}. Qui è
  detto ironico, perché significa, che da ultimo per Martinazza verrà il tempo cattivo,
  cioè sarà gastigata del suo mal fare.

\item[INTRIGARSI] Vuol dire impacciarsi, o interessarsi: e vuol dir' anche imbrogliare,
  o mescolar una cosa con un' altra in maniera di confonderle, donde \textit{intrigo}
  per imbroglio.

\item[UNA fune daralle, che l'impicchi] Quand'altri ci ha mal serviti, per mostrargli,
  che non merita rimunerazione, si suol dire; Gli vuò dare un par di corna;
  Un par di funi, o una fune, che l'impicchi.
\end{description}
\section{STANZA IV. STANZA}

\begin{ottave}
\flagverse{4}Horsù tiriamo innanzi, ch'io ho finito,\\
Perch' a questi discorsi le persone\\
Non mi dicesser: Questo scimunito\\
Vuol farci qualche predica o sermone,\\
Attenti dungue. Già v'havete udito\\
L'incanto, ch'ella fece a petizione\\
Di quei del luogo, c'hebbero concetto\\
Scacciarne il Duca; ma svanì l'effeto.
\end{ottave}

\begin{ottave}
\flagverse{5}Ella ch'in tanto havuto havea sentore,\\
Che quei due spirti sciocci ed inesperti\\
Havean dinanzi a lui fatto l'errore,\\
Sì che da esso furono scoperti;\\
Se la digruma, che ne va il suo honore,\\
Mentre gli accordi fatti, ed i concerti\\
Riusciti alla fin tutte panzane,\\
Con un palmo di naso ne rimane.
\end{ottave}



Il Poeta lasciando da parte la moralità, viene al racconto, e torna alla memoria
del Lettore l'incanto fatto da Martinazza per cacciare il Duca, che non
hebbe effetto, per lo che ella è in collera, perché le pare di perdere di quella
stima, nella quale era tenuta dai popoli, e soldati di Malmantile.

\begin{description}
\item[SCIMVNITO] Sciocco, scempiato. Vedi sopra C. 1. stan. 17.
\item[SVANÌ l'effetto] Non riusci l'effetto: il negozio andò in fumo. I Lat. pure
  dissero \textit{Evanuit}, \& \textit{evanescere}.
\item[SE la digruma] Seco stessa la pensa, e masticandola non la può inghiottire,
  cioè non la può sofferire. E si dice digrumare, e ruminare, e dagli antichi fu detto
  \textit{rugumare}, onde forse è fatto digrumare; (che e il rodere che fanno le bestie dal
  pié fesso, come vedemmo sopra C.4. stan. 6. e C. 5. stan. 63.) perché uno, a cui
  succeda cosa di poco suo gusto, suole per lo più stando pensoso masticare, o biasciare
  appunto come fanno dette bestie quando digrumano, al che per avventura
  ebbe riguardo Omero in quel verso tradotto da Cicerone.
  \begin{verse}
    Ipse suum cor edens, hominum vestigia vitans.
  \end{verse}
  quasi che chi maninconico rumina, e biascia masticandola male; mostri di
  beccarsi il cuore.
\item[RIVSCITI tutti panzane] Son riusciti tutte vanità, tutte chiacchiere. \textit{Che dar
  panzaze, bubbole, chiacchiere, ec}, vuol dir promettere, e non mantenere, che si
  dice \textit{inzampognare}, \textit{infinocchiare}, ed è il Lat, \textit{Verba dare}.
\item[RIMANE con un palmo di naso] Riman burlata, beffata, Il Lalli En, tr. lib. 1.
stan. 11. dice.
\begin{verse}
  Ed io son per restar in queste caso
  Con sei palmi lunghiffini di naso.
\end{verse}
\end{description}

\section{Stanza VI. \& VII.}

\begin{ottave}
\flagverse{6}Ma non si sbigottisce già per questo,\\
Che vuol cansar quell'armi dalle mura,\\
Ai Diavoli, da' quali hebbe il suo resto,\\
E che gliel'hanno fatta di figura,\\
Vuol, dopo il far, che rompano un capresto\\
Squartare, e poi ridurre in limatura,\\
Perché non fu mai can, che la mordesse,\\
Che del suo pelo un tratto non volesse.
\end{ottave}

\begin{ottave}
\flagverse{7}Basta, ch'ella sel'è legata al dito,\\
E l'ha presa co' denti, e ser'affanni;\\
Tal c'andarsene in Dite ha stabilito,\\
Perché ne vuol veder quanta la canna,\\
Ed oprar, che Baldon resti chiarito\\
C'ambisce in Malmiantil sedere a scranna;\\
Hor mentre a questa volta s'indirizzi,\\
Potrà far un viaggio a due servizj.
\end{ottave}


Martinazza non si perde d'animo, e vuole in ogni maniera scacciar l'esercito
di Baldone da Malmantile. Risolve però d'andare all'inferno in persona a trovar
Plutone, per ottener da lui il gastigo di quei due diavoli, che fecero l'errore,
ed un nuovo modo di far diloggiar Baldone da Malmantile.
\begin{description}
\item[NON si sbigottisce]Non si perde d'animo; Non si sgomenta. Vedi sopra C. 2.
  stan. 28. e C. 5. stan. 63.
\item[HEBBE il suo resto] Hebbe finito di conoscergli. Hebbe viflo quanto essi valevano.
  Si dice \textit{Tu m'hai dato il mio resto}: \textit{Tu m'hai pieno}: \textit{Son sazio}, \textit{son stufo di te},
  per intendere Non mi varrò mai più dell'opera tua.
\item[GLIEL'hanno fatta di figura] Le hanno fatto una ingiuria grandissima, una
  solennissima burla. Tratto dal giuoco di primiera, quando uno havendo buon,
  punto, ed essendo per vincer la posta, un'altro con figura fa una primiera, e gli
  leva la posta.
\item[ROMPANO un capresto] Restino impiccati. Chiamano capresto quella cordicella
  sottile, che il Boia lega al collo a coloro, che egli impicca, la quale dicono,
  che morto il paziente si rompa; e però dice rompano un capresto; detto
  usatissimo per intendere farsi impiccare.
\item[RIDURRE in limatura] Ridurre in minutissimi pezzi. Limatura si dicono quei
  fragmenti che cascano dal ferro, o altro metallo, quand'altri lo lima.
\item[NON mi morse mai cane, ch'io non volessi del suo pelo] Nessuno mi fece mai ingiuria,
  che non mi volessi vendicare. Nessuno mi morse, che io non lo rimordessi.
  Dicono che il pelo del cane sia medicamento alle morsicature fatte dal medesimo
  cane. Vedi sotto C. 9. stan. 58. E da questo rimedio ha origine il presente
  dettato; che i latini dissero \textit{Nemo impune abiit, qui me ausus sit laedere},
\item[SEL' è legata aj dito] Ne ha presa memoria per vendicarsi. Sogliono molti per
  haver memoria di qualche negozio, che devano fare, legarsi un filo intorno a dito;
  il che ha dato origine al presente dettato. Il Lalli En. Tr. Can. 2, stan. 25, dice:
  \begin{verse}
    Sel' attaccò, come suol dirsi, al dito.
  \end{verse}
  Nel Deuteronomio al sesto, \textit{Eruntque verba haec, quae ego praecipio tibi hodie in corde
  tuo: \& narrabis ea filijs tuis., \& meditaberis sedens in domo tua, \& ambulans in itinere,
  dormiens atque consurgens: \& ligabis quasi signum in manu tua}. E sono al cap. 11.
  \textit{Ponite haec verba mea in cordibus, \& animis vestris, \& suspendite ea pro signo in manibus}.
  Fra Giordano Predicatore antico Domenicano; nel Vocabolario della
  Crusca alla Voce \textit{Filateria}. Le filaterie si erano una carta, ove erano scritti i
  comandamenti della Legge, e portavanla intorno al braccio apertamente. E quivi
  va spiegando, cred'io, il passo di San Matteo cap. 23. \textit{Dilatant enim phylacteria
    sua}. È voce Greca; da \textit{phylattein}, guardare, custodire, significante certe strisce
  di quoio, o di cartapecora, che gli Ebrei si legano al braccio per tenere maggiormente
  a memoria i passi della Scrittura, che quivi sono notati, le quali da loro si
  domandano: \textit{Tephilim}.
\item[L'HA presa co' denti] S'è adirata grandemente, e s'è messa in animo di vendicarsi.
  Vuol impiegare ogni suo studio per vendicarsi. Sogliono i calzolai per far
  venire il quoio a quel segno che loro bisogna; tirarlo co' denti, e di qui nasce il
  presente termine, che esprime uno, che si sia preso a cuore di fare un negozio,
  e che voglia impiegare ogni suo talento per conchiuderlo.
\item[SE n'affanna] Se l'è presa a cuore. N'ha premura. Se ne dà pena, e pensiero.
\item[IN Dite] Dite, secondo il favoloso creder de i Gentili è lo stesso, che Plutone,
  l'uno, e l'altro nome significando ricchezze delle quali, perché si cavano di sotterra,
  facevano Custode, e Padrone quel loro Dio sorterraneo; ma qui si piglia Dite
  per la Città, e per il Regno di Dite.
\item[NE vuol veder quanto la canna] Cioè quanto tira, o è lunga la canna da misurare;
  e s'intende vederla per la minuta, e quanto si può, e fare ogni sforzo per
  arrivare al suo intento.
\item[RESTI chiarito] Resti sgarito: Scaponito. Vedi sopra C. 1. stan. 1.
\item[SEDERE a scranna] Vuol dire comandare; esser padrone. \textit{Scranna}, o
  (come diciamo noi) ciscranna, è una specie di seggiola da i Latini detta \textit{sella plicatilis}.
  Dante Purg. C, 19. dice:
\begin{verse}
  \backspace Hor chi sei tu che vuoi sedere a scranna
  Per vindicar da lungi venti miglia
  Con la veduta corta d'una spanna ?
\end{verse}
Buratto nell'Apologia contro al Castelvetro dice: \textit{Non habbiate tanto cervelle, che
  basti, se ben volete sedere a scranna per giudicare gli altri}.
\item[FAR un viaggio a due servizzj] Che dichiamo anche: \textit{Fare un viaggio, e due
  servizzj}). Con un medesimo viaggio far due negozzj, che è impetrar da Plutone
  il gastigo di quei due diavoli, e lo sfratto di Baldone. Ne i Latini si trova in
  questo senso \textit{Duos parietes de eadem fidelia dealbare}. E si dice anche \textit{Dare a due
    tavole a un tratto}, Vedi sopra C. 3. stan. 14.
\end{description}
\section{Stanza VIII. --- X.}
\begin{ottave}
\flagverse{8}Giù da Mammone andar vuole in persona, \\
Che più non è dover, ch'ella pretenda, \\
Che sua bravicornissima corona \\
Salga a suo conto a ogni poco, e scenda, \\
Chieder grazie, e dar brighe non consuona, \\
E chi ha bisogno, si suol dir, s'arrenda, \\
Per questo a lei tocca a pigliar la strada, \\
Per c'alla fin convien, che chi vuol vada.
\end{ottave}

\begin{ottave}
\flagverse{9}Perciò s'accontia, e va tutta pulita\\
Col drappo in capo, e col ventaglio in mano\\
A cercar chi l'informi della gita;\\
Ne meglio sa, che Giulio Padovano,\\
Che l'ha su per le punta delle ditta,\\
E più di Dante, e più del Mantovano,\\
Perch'eglino vi furon di passaggio,\\
E questo ogni tre di vi fa un viaggio.
\end{ottave}

\begin{ottave}
\flagverse{10}Onde a trovarle andata via di vela\\
Domanda (perchè in Dite andar presume)\\
Che luoghi v'è, che gente, e che loquela\\
Ed ei di tutto le dà conto, e lume;\\
E poi per abbondare in cautela,\\
Volendola servire infino al fiume,\\
Le porge un fardellin piccolo, e poco\\
Di robe, che laggiù le faran giuoco.
\end{ottave}

Martinazza risolve d'andare in persona a trovar Plutone, considerando, che
non è dovere, che questo Re per lei a ogni poco si scomodi; e però sapendo, che
Giulio Padovano è più informato d'ogni altro della strada dell'Inferno, se ne va
a pigliar da lui informazione, e della gita, e dei costumi di quei paesi; ed egli
l'istruisce, e per servirla meglio la vuol accompagnare fino al fiume Acheronte,
ed intanto le dà un fardellino di robe, che laggiù verranno a bisogno.
\begin{description}
\item[BRAVICORNISSIMA corona] Epiteto, e titolo composto dall'Autore a Plutone.
  Il Lalli En. Tr. lib. 1. stan. 16. parlando d'Eolo Re de' Venti dice:
  \begin{verse}
    Dunque poi che Giunone alla presenza
    Di sua Real ventosità fu giunta.
  \end{verse}
\item[MAMMONE] Da Mammona; parola usata nell'Evangelio. Alcuni espositori
  della Sacra Scrittura vogliono, che Mammona sia voce Caldea, e significhi
  \textit{opes}, ed altri che sia voce Siriaca, e significhi quello, che in Greco significa
  \textit{Plutos}, che è \textit{divitiae}, sì che concordano, e tanto è a dir Mammone, che
  Demonio, ovvero Plutone, che qui s'intende per il Re dell'Inferno. Viene
  dalla radice Ebrea \textit{Taman}, che propriamente significa \textit{nascondere}, \textit{riporre}, e
  per così dire \textit{intanare}; onde si fece \textit{Matmon}, e alla Siriaca \textit{Matmona}, cioè \textit{ricchezze
  nascoste}, o vogliam dire \textit{tesoro}. Mammona poi venne a dirsi per più agevolezza di pronunzia.
\item[DAR brighe] Dare scommodi, dar molestie. La voce briga significa operazioni
  scommode, faticose, e noiose.
\item[CHI ha bisogno s'arrenda] Chi ha bisogno non sia superbo, ma si pieghi a raccomandarsi,
  e pregare; Che il verbo \textit{arrendersi} val per cedere piegarsi, o condescendere.
\item[CHI vuol vada] Chi vuol ottenere una cosa vada a chiederla da per se, ed il
  proverbio dice \textit{Chi non vuol mandi, e chi vuol vada da se}. Che diciamo anche \textit{Non
    è più bel messo, che se stesso}, o vero, \textit{Chi va lecca, E chi sta si secca}.
\item[ACCONCIARSI] Rinfronzirsi, raffazzonarsi. Vedi sopra C.\ 2. stan.\ 69.
\item[DRAPPO] Dicendosi drappo assolutamente s'intende drappo da donna, che
  è una striscia di taffettà, o d'ermisino larga fino a due braccia, e lunga fino a quattro,
  la quale dalle donne Fiorentine di condizione ordinaria è portata in capo,
  o alle spalle quando vanno fuori di Casa. In Venezia \textit{drappo} significa ogni sorta
  di vestimento, sì come presso i Toscani antichi scrittori. Vedi sotto C.7.stan.22.
\item[VENTAGLIO] Strumento noto usato dalle donne la state per farsi vento.
\item[L'INFORMI della gita] Le insegni la strada, che conduce all'inferno.
\item[GIVLIO Padovano] Io veramente non ho saputo ritrovare chi sia questo Giulio
  Padovano\footnote{Potrebbe anche trattarsi di Giulio Cesare Casseri (1552-1616). Sebbene originario di Piacenza, si stabilì a Padova, dove studiò ed insegnò anatomia. Autore delle \textit{Tabulae anatomicae}, pubblicato nel 1627 a Venezia, principale trattato di anatomia del secolo diciassettesimo, magari noto al Lippi per le 97 tavole anatomiche. }, se forse non ha inteso di Giulio Hygino scrittore d' Astronomia.
  Ma costui fu liberto, o vogliam dire schiavo affrancato d'Augusto; condotto da
  lui ragazzo d'Alessandria, secondo che alcuni vogliono; i quali perciò lo stimano
  Alessandrino; o pure di nazione Spagnuolo, secondo la testimonianza di Svetonio
  nel Libro \textit{de illustribus Grammaticis}.
\item[L'HA su per le punte delle dita] La sa benissimo; Latino \textit{in numerato habet}.
  Aldo Manuzio nella dedicatoria di Giuvenale disse: \textit{Quando eas tenebas memoria, quam
    digitos unguesque tuos}, Cicerone nella Orazione contra Cecilio intitolata \textit{Divinatio}:
  \textit{Quid cum accusationis tuae membra dividere ceeperit, \& in digitis suis singulas partes
  causae constituere ? Quid, cum unumquodque transigere, expedire, absolvere?}
\item[DANTE, e il Mantovano] Dante Poeta Fiorentino; e Vergilio, il quale Dante
  finge, che fusse sua guida all'Inferno, e pero dice: \textit{Eglino vi furon di passaggio}.
\item[OGNI tre dì] Questo modo di dire, se bene è determinate, significa spesso spesso,
  o a ogni poco indeterminatamente.
\item[ANDAR via di vela] Andar via velocemente, e a dirittura, come fa la nave
  quando va a vela.
\item[PER abbondare in cautela] Cioè per servirla bene. Diciamo \textit{abbondar in cautela}
  quando uno fa più di quel che sia richiesto, o più di quel che sia necessario; per
  esempio. Io darò dieci scudi a uno, perché mi compri una mercanzia, la quale so
  che non vale così gran somma; ma per assicurarmi del caso, che valesse non
  più, li do due altri scudi \textit{per abbondare in cautela}, cioè per andar cautelato, e in
  sul sicuro, che non gli manchi denaro, se ella valesse più. Qui però vuol dire
  Abbondare, ed eccedere in cortesia nel servirla.
\item[LE faranno giuoco] \makebox[8pt]{} Le torneranno a proposito. Le verranno a bisogno. Le
  saranno d'utile.
\end{description}
\section{STANZA XI. --- XV.}

\begin{ottave}
\flagverse{11}Così la Maga se ne va con esso,\\
Che l'introduce in una bella via \\
Tutta fiorita, sì che al primo ingresso \\
Par proprio un Paradiso, un'allegria; \\
Ma non più presto l'huom il pié v'ha messo \\
Ch'ella diventa un'altra mercanzia \\
Per i gran morsi, e le punture acerbe, \\
Che fanno i serpi ascosi fra quell'erbe.
\end{ottave}

\begin{ottave}
\flagverse{12}Entravi Martinazza, e sente un tratto\\
Due, o tre morsi a pié dove calpesta,\\
Perciè bestemmia, che non par suo fatto,\\
E dice: O Giulio mio, che cosa è questa?\\
Ed ei ridendo allora come un matto;\\
Non è nulla (rispose) vien pur lesta;\\
Che pensi tu ch'io sia privilegiato?\\
Anch'io mi sento mordere, e non fiato.
\end{ottave}

\begin{ottave}
\flagverse{13}Questa è la via, che mena a casa calda,\\
Perch'ella è allegra, o almeno ella ci pare,\\
Perché a martello poi non istà salda;\\
La scorre ognor gente di mal'affare,\\
Le serpi sono ogni opera ribalda\\
Ch'ella ci fa, le quali a lungo andare\\
Di quanto ha fatto, scavallato, e scorso\\
Ci fa sentir al cuor qualche rimorso.\\
\end{ottave}

\begin{ottave}
\flagverse{14}Ma se ravvista un tratto del suo fallo\\
Bada a tirar innanzi alla balorda,\\
Perch'il vizio rifiglia, e mette il tallo\\
Vien sempre più a aggravarsi in su la corda,\\
Ul male invecchia al fine, e vi fa il callo\\
Sì che venga un Serpente pure, e morda,\\
Ch'ei non sente ne meno anc'un ribrezzo,\\
Così peggio che mai la dà pel mezzo.
\end{ottave}

\begin{ottave}
\flagverse{15}Nella neve si fa lo stesso giuoco,\\
che l'huom sul primo diacciasi le dita,\\
Poi quel gran gelo par che manchi un poco,\\
E sempre più nell'agitar la vita;\\
Al fine ei si riscalda come un fuoco\\
Sì che non la farebbe mai finita,\\
Ne gli darebbe punto di spavento\\
Quand'ei v'havesse ancora a dormir drento.
\end{ottave}

Martinazza se ne va con Giulio, il quale la conduce per una strada, che al
primo ingresso pare una bella cosa, ma presto si conoice, ch'ell'è altrimenti per li
morsi che danno i serpi ascosi infra quell'erbe; Giulio mostra a Martinazza,
che questa strada, che guida all'Inferno è facile, e gustosa, e se bene e ripiena
di malanni, non son sentiti ne conosciuti da quelli, che la camminano, perché
vi si sono assuefatti; appunto come fanno coloro, che mettono le mani nella neve,
che a principio la toccano fredda, e col seguitare a maneggiarla, par loro che ella sia calda.
\begin{description}
\item[PARE un Paradiso] Pare una cosa tanto allegra, e vaga, che più non si può
  fare. Telemaco figliuol d'Ulisse nel quarto dell'Ulissea, arrivato in Sparta; nel
  considerare attentamente la ricchezza, e l'ampiezza del Regio Palazzo di Menelao,
  prorompe in quella esclamazione: \textit{Tal dentro è del gran Giove il gran Palazzo}.
\item[DIVENTA un altra mercanzia] Diventa un' altra cosa. Usiamo dir \textit{mercanzia}
  per esprimere ogni sorta di cosa ancor che incorporea, come \textit{lo studiare è una certa mercanzia},
  ec.
\item[NON par suo fatto] Non par che faccia quella tal cosa. Vedi sopra Can. 4.
  stan. 16.
\item[CASA Calda] Intende l'Inferno. Il Lalli En. Tr. parafrasando \textit{facilis
  descensus Averni} ec. dice:
  \begin{verse}
    \makebox[8em]{\dotfill} Enea mio bello,
    A casa Calda si va presto presto;
    Ma ritornar in su, questo è il bordello.
  \end{verse}
\item[NON è nulla] Queste due negative secondo la buona regola doverebbono affermare,
  ma è nostro idiotismo tanto inveterato, che l'uso ci libera dall'errore, se
  ce ne serviamo in questo modo per negativa. Appresso i Greci due negative, o
  affermano, ma negano maggiormente, ed è maniera, siccome appresso
  noi, così appresso loro usatissima.
\item[NON sta a martello] Non regge alla prova. Non è com'ella pare. Metafora
  tolta dal Cimento dell' oro. Vedi sopra C. 5, stan. 2.
\item[A LUNGO andare] Col tempo. In processo di tempo; Se continoverai lungo tempo.
\item[SCAVALLATO] Cioè datasi ogni sorta di bel tempo. Si dice anche \textit{scorrer
  la cavallina}. Virg. 3. Georg. \textit{Scilicet ante omnes furor est insignis equarum, Et
  mentem Venus ipsa dedit}. E poi: \textit{illas ducit amor trans Gangara, transeque sonantem}, \&c.
  Vedi fopra C. 1. stan. 66.
\item[QUALCHE rimorro] Senton rimorder la coscienza per gli errori commessi.
\item[ALLA balorda] Senza considerazione.
\item[METTE il tallo] Tallisce, fa nuove messe. Vuol dire: un vizio ne genera
  molti. Tallo è parola venuta a noi dalla lingua Greca, che significa germoglio,
  usata ancora dagli agricoltori Latini.
\item[VIENE a aggravarsi in su la corda] Vien più che mai a crescere il male; perché
  quando uno tocca il martirio della corda, e s'aggrava in su la medesima corda,
  fa crescere il dolore; Ed altrimenti \textit{aggravarsi in su la corda} vuol dire quando uno
  esaminato in su la corda dice cose, che fanno crescere l'indizio, che egli habbia
  commesso un delitto.
\item[FA il callo] Vi s'assuefà. \textit{Et ab assuetis non fit passio}, dice, che non sente
  ne meno un ribrezzo.
\item[RIBREZZO] Che vuol dire capriccio di febbre; cioè quel tremore, o brivido,
  che si sente prima, che entri la febbre. Latino \textit{rigor}. Il Cavalcanti Stor. Fior.
  lib.2.cap.21, dice: \textit{Antipatro di Sidonia in quel giorno, che egli nacque, ogn'anno
    gli arrivava qualche ribrezzo di febbre, e tanto continua, che un'anno gli rinvestì in
    mortale accidente}. Ma Dante nell'Inf. C. 17 mostra, che si dicesse riprezzo.
  \begin{verse}
    \backspace Qual è colui c'ha sì presso il riprezzo
    Della quartana, c'ha già l'ugna smorte,
    E trema tutto pur guardando il rezzo.
  \end{verse}
  E al C.32.dice:
  \begin{verse}
    \backspace Poscia vedd'io mille visi cagnazzi
    Fatti per freddo, onde mi vien riprezzo,
    E verrà sempre de i gelati guazzi.
  \end{verse}
  Ma noi lo pigliamo anche (come e preso nel presente luogo) per ogni leggiero
  sollevamento d'animo, o spavento, o per un semplicissimo dolore. Ed alle volte
  per fastidio, o travaglio per esempio \textit{Il tale commesse quel mancamento; ne vuole
    haver de' ribrezzi}. Vedi sotto C. 11, stan. 2.
\item[LA dà pel mezzo] Fa tutto quello, che gli vien volontà senza riguardo alcuno.
  È dedotto da quelli; che in tempo di pioggia camminando per la Città vanno
  per il mezzo della strada, e non si guardano dall'ammollarsi per l'acqua
  caduta, che scorre pel mezzo, e per quella che vien dal Cielo.
\end{description}
\section{Stnaza XVI \& XVII.}
\begin{ottave}
\flagverse{16}Hor tu m'hai inteso: rasserena il volto, \\
Che tu vedrai tirando innanzi il conto \\
(Perché di qui a poco non c'è molto) \\
Che delle serpi non farai più conto,\\
Ma dimmi, c'ha' tu fatto del rinvolto?\\
L'ho qui, dic'ella, sempre lesto, e pronto: \\
Sta ben, soggiunge Giulio, adungue corri,\\
Perché qui non è tempo da por porri.
\end{ottave}

\begin{ottave}
\flagverse{17}Resta, dic'ella, omai ch'io ti ringrazio\\
Dell'instruzion, ch'appunto andrò seguendo:\\
Promissio boni viri est obligatio,\\
Dic'egli; T'ho promesso, e però intendo\\
Ancor seguirti questo po di spazio,\\
E quivi con un tibi me commendo,\\
All' in qua ripigliando il mio cammino\\
Ti lascio, come io dissi al colonnino.
\end{ottave}


Giulio esorta Martinazza a non haver paura, ed a camminare; ed ella lo ringrazia
dell'instruzione datale, e lo prega a partire, ed egli ricusa di farlo, perché
le ha promesso di accompagnaria infino al fiume Acheronte.
\begin{description}
\item[DI qui a poco non c'è molto] Questo termine giocoso e usato per esprimere
  \textit{fra pochissimo tempo}.
\item[TIRANDO innanzi il conto] Seguitando il suo viaggio, È termine mercantile
  che vuol dir portare un conto avanti da un libro a un'altro, o da una carta a
  un'altra nel medesimo libro. Donde poi \textit{tirar innanzi il conto} vuol dir
  Camminare avanti. Vedi sopra C. 4, stan. 60.
\item[NON è tempo a por porri] Non è tempo da perdere. Non è da indugiare.
  Quando si pongono i porri, sono così fottili, che richiedono molto tempo a
  porgli, e da questo habbiamo il presente proverbio, che si dice anche \textit{Non è
    tempo da dar fieno a oche}.
\item[PROMISSIO boni viri est obligatio] Sentenza latina, che vuol dire un'huomo
  da bene è obligato a mantener la parola, ed osservare quel che ha promesso.
\item[CON un tibi me commendo] Detto latino, che suona con un mi raccomando a te;
  cioè con salutarti. Quando diciamo: \textit{Addio}, ci s'intende; vi raccomando.
  Saluto di congedo, Catullo: \textit{Commendo tibi me}.

\item[TI lascio al colonnino] Ti abbandono. \textit{Lasciar al colonnino} vuol dire lasciar uno
  nel pericolo, perché \textit{colonnino} intendiamo quella colonnetta di legno traforata,
  la quale è davanti alle forche, e vi legano i malfattori quando gli strozzano.

\end{description}

\section{Stanza XVIII --- XXI.}
\begin{ottave}
\flagverse{18}Ed essa allora abbassa il capo, e tocca,\\
Se ben de' serpi ell'ha qualche paura;\\
Pur via zampetta, e fatto del cuor rocca,\\
Va calcando la strada alla sicura,\\
Sè ch'ella non si sente aprir la bocca,\\
Perché non è più morsa, o non lo cura:\\
Giunti alla fine al gran fiume infernale,\\
Restò la donna, ed ei le disse: Vale.
\end{ottave}

\begin{ottave}
\flagverse{19}Questo è il famoso fiume d'Acherorte,\\
Ove s'imbarca ognun, che quivi arriva,\\
S'affaccia anch'essa, ma il nocchiere Caronte,\\
Da poi che tratto ognuno hebbe da riva,\\
Sta in dietro, grida a lei con torva fronte,\\
Che quà non passa mai anima viva;\\
Ond'ella, messi fuor certi baiocchi,\\
Gli getta un po di polvere negli occhi.
\end{ottave}

\begin{ottave}
\flagverse{20}Ed egli, che da essa hebbe il sapone,\\
E che si trovò lì come il ranacchio,\\
Preso dalla medesima al boccone,\\
Mentr'ella saltò in barca, chiusdfe l'occhio.\\
La Strega fra quell'anime si pone,\\
Quai con le brache son fino al ginocchio,\\
Dovendo a' Soprassindaci di Dite\\
Presentar de' lor libri le partite.
\end{ottave}

\begin{ottave}
\flagverse{21}Piangendo, come quando uno ha partito\\
Le cipolle fortissime malige:\\
Passan quel fiume, e poi quel di Cocito;\\
Ultimamente la palude stige,\\
Che a Dite inonda tutto il circuito,\\
E in se racchiude furbi, e anime bige,\\
Ove Caronte al fin sendo arrivato\\
Sbarcò tutti; ed ognun fu licenziato.
\end{ottave}



Martinazza seguita il suo viaggio, e non fa più stima delle morsicature de i
serpi, ed arrivati al fiume d' Acheronte', Giulio si licenzia dalla donna, la quale
s'accostò per entrar nella barca; ma Caronte lo sgridò dicendo, che non poteva
entrarvi, ond' ella gli diede un poco di mancia, ed ei finse di non la vedere entrar
in barca, dove ella si mescolò con gli altri, e fu condotta all' altra riva, e
quivi con essi sbarcata.
\begin{description}
\item[TOCCA] Si dice \textit{tocca il cocchio}, e significa: cammina innanzi. Vedi sopra
  C. 1. stan. 41.
\item[ZAMPETTA] Muove le gambe: Cammina. Zampettare si dice propriamente
  de' bambini quando cominciano a imparare a andare.
\item[NON si sente aprir bocca] Non si sente parlare. Sono infiniti i modi, che habbiamo
  per esprimer il silenzio d'uno, come \textit{star zitto; non fiatare; non far verbo;
    ammutolire; star chiotto, lasciar la lingua al beccaio, haver visto il lupo; diventare
    Arpocrate} ec.
\item[GLI disse] Vale. Gli disse Addio.
\item[ACHERONTE] I fiumi dell'Inferno da i Gentili si dicevano quattro, e che
  nascessero dalle lagrime de' mortali, per lo stato de' quali figura Dante la statua,
  che vedde in sogno Nabucdonosor, che havea la testa d'oro, le braccia, e
  petto d'argento, il corpo fino alle cosce di rame, le gambe di ferro, ed il destro
  piede di terra cotta; da questa dice che scaturiscono le dette lagrime, le quali
  formano li detti quattro fiumi Infernali, e così la descrive nell'Inf. C. 14.
  \begin{verse}
    \backspace    Dentro dal monte sta dritto un gran veglio,
    Che tien volte le Spalle in ver Damiata,
    E Roma guarda si come suo speglio,
    \backspace    La sua testa è di fin' oro formata,
    E puro argento son le braccia, e il petto,
    Poi è di rame fino alla forcata.
    \backspace    Da indi in giuso è tutto ferro eletto,
    Salvo che 'l destro piede è terra cotta,
    E sta in su quel più, ch'in su l'altro, eretto.
  \end{verse}

Il primo dunque di detti fiumi e Acheronte, che ia un certo modo significa
privazione d' allegrezza;da Acheronte nasce Stige, che significa cosa dispiacevole,
odiosa, quale e il Dolore; perché questo ne viene dopo la privazione dell'allegrezza,
Il terzo è Flegetunte, che signfica pensiero ardente travaglioso. E da
questi tre fiumi si genera il quarto, che e Cocito stagno,o fiume del lamento, e del
pianto. Questa favolosa opinione de' Gentili tocca Dante nell' Inf, C, 14. seguitando
i sopraddetti versi.
\begin{verse}
  \backspace Ciascuna parte, fuor che l'oro è rotta
  D' una fessura, che lagrime goccia,
  Le quali accolte foran questa grotta,
  \backspace Lor corso in questa valle si diroccia:
  Fanno Acheronte, Stige, e Flegetonta;
  Poi sen va già per questa stretta doccia.
  \backspace Infin là dove più non si dismonta,
  Fanno Cocito, e qual sia quello stagno
  Tu'l vederai, però qui non si conta.
\end{verse}

\item[CARONTE] Notissimo barcarolo dell' Inferno. Vedi sopra C. 2, stan.24.
\item[HEBBE tratto ognun da riva] Hebbe levate d'in su la riva tutte l'anime, imbarcandole.
\item[TORVA fronte] È latino usato da noi; E vuol dire Viso burbero, aspro, agro,
  arcigno. 
\item[ANIMA viva] Intendi huomo, che non sia morto. Virg, 6. En. \textit{Corpora
  vina nefas Stygia vectare carina}. Sa bene il nostro Poeta, che anime sono immortali,
  ma seguita il costume d' intendere huomo vivente, quando diciamo anima
  viva (Genesi cap.2, Et factus est homo im animam viventem) ed imita Dante Inf.
  C. 3. che dice:
  \begin{verse}
    E tu che sei costì anima viva,
    Partiti da codesti, che son morti.
  \end{verse}
  Il Lalli En, Tr. C. 3. stan. 16.
  \begin{verse}
    E non v' è mai entrata anima viva.
  \end{verse}
\item[GLI gettd un po di poluere negli occhi] Gli decte un po di mancia, I Latini pure
  dissero: \textit{Pulverem oculis offundere}. E s'intende dar mance per corrompere il giusto,
  quasi diciamo: \textit{Abbagliare gli occhi del giudice con l'oro, acciocché non vegga
    la giustizia}. 
\item[HEBBE il sapone] Era stato subornato, e corrotto con la mancia; Gli erano state
  insaponate le carrucole (che vuol dire Tirar' uno al nostro volere, e renderlo
  facile a quel che noi bramiamo, e fare che non strida contro di noi) con
  dargli la mancia; come con l'insaponare una carrucola, o una ruota si facilita
  il veicolo, e si fa, che non strida. Ed è lo stesso che \textit{gettar la poluere negli occhi}
  detto poco sopra. Dicesi anche: \textit{Ugner le mani}. Bocc, Nov.6. \textit{Il buono huomo per
 certi mezzani gli fece ugner le mani}.
\item[PRESO al boccone, come il ranocchio] Obbligato a tacere, per havere havuta la
  mancia. È lo stesso che li suddetti due modi di dire, cioè \textit{Havere il sapone}, e
  \textit{Havere la polvere negli occhi}. Qui non vorrei che il Lettore credesse, che il Poeta
  avesse opinione, che i regali potessero corrompere i demonj, se ben la sentenza
  portata da Ovidio dice \textit{munera (crede mibi) placant hominesque deosque}, ma sapesse haver'
  egli detto così, per mostrare che l'oro arriva a corromper quelli, che ne meno si
  crederebbe, e che meno dovriano lasciarsi arrivar dall'oro, e finalmente ha voluto esprimere
  la possanza, che hanno i regali di far conseguire ciò che si vuole.
  \textit{Ombia enim per pecuniam facta sunt}. Si racconta di Filippo Macedone, che havendo
  fatto riconoscere una fortezza, ed essendogli riferito, che era impossibile il
  pigliarla, domandasse agli sploratori, se vi era modo di farvi andare un' asino carico d'oro,
  volendo inferire, che dove non potevano l'armi, sarebbe arrivato l'oro;
  \textit{Auri Sacra fames, quid non mortalia pectora cogis?} E Orazio. \textit{Aurum per
    medios ire satellites, Et perrumpere amat saxa potentius Ictu fulmineo}.
\item[CHIUSE l'occhio] Finse di non vedere. È il latino connivere, Vedi sotto C. 10. stan. 5.
\item[CON le brache fino al ginocchio] Il proverbio \textit{Caschar le brache} è il medesimo,
 che \textit{cascar le braccia}; che vuol dir perdersi d'animo, Omero: \textit{Animus in pedes decidit}.
 Cascò il cuore; cascò l'animo a' piedi, Onde dicendo, che costoro \textit{havevano le brache
 fino al ginocchio}, intende che eran loro cascate affatto, cioè erano
del tutto perduti d'animo, perché doveano render conto delle loro azioni. Vedi sotto C. 9. stan. 24.
\item[SOPRASSINDACI] Così chiamiamo noi quel Magistrato, che ha l'autorità di riveder
  conti a tutti i Magistrati, Ofiziali, e Ministri del dominio Fiorentino.
\item[CIPOLLA maligia] Specie di cipolla da mangiare, che è fortissima, e fa venir le
  lagrime a tagliarla, e maneggiaria; Bocc. gior. 8.n.2, \textit{E talora un mazzetto di
    cipolle malige, o di Scalogni}. Il Lalli En, Tr. C. 3.
  \begin{verse}
    Così dicea, e tutto iil volto molle
    Havea di pianto, come se schiacciato
    Vi fusse sopra il sugo di cipolle.
  \end{verse}
\item[COCITO] Vedi sopra alla stan. 19, alla parola sAcheronte, e quivi troverai
  ancora quel che sia la Palude Stige, della quale vedi anche sotto in questo Cant,
\item[GENTI bige] Genti scellerate, e da non se ne fidare. Per comporre il color
310 i Pittori mescolano tutti i colori, e lo chiamano il coler dell' asino; e però
lic | huomo bigio s' intende uno, che ha tutti i vizzj. Va moderno Poeta.,
€¢ notammo spra C. 3. stan. 66. dific parlando d' uno di questi tali, che era
moro.
\begin{verse}
  Chiude un' anima bigia un corpo nero,
  \end{verse}
L'origine di questa parola bigio in questo significato stimo, che nasca da questo.
Eranoin in Firenze Ne' secoli passati tre fazioni, l'una de' Fautori di Fr. Girolamo
Savonarola, la quale era detta de' Piagnoni, l'altra de' conttarj a detto Fr. Gite: |»
Jamo chiamata gli e4rrabbiati, o Compagnacci; e sta'di loro ¢1 teo'nimi- = |
ci, e discordi, faluo che univano nel? esser contrarj alla terza fazione;, ¢
de' fautori de' Medici, la quale era detta de' Padefehi, i quali non conue
ne con l'una, ne con l'altra fazione. Di questi che inclinavano alla |
Patieschi talvolta alcuno per suoi fini particolari s' univa o con Puna, o con!
tra delle prime due, ma era ricevuto con sospetto, che non fale)
ro deliberazioni, e pero dicevano: Won e da fidarsi di lore', Son Bigi. E
ucftowforfe ha havuto origine questa voce bigio in significato' 4' huomo da
se ne fidare, Vedi la Relazione di Firenze del Foscari, e il Nardi nell
Florentine fib, 2.:
\end{description}
\section{STANZA XXIL}
{22}Gli entrar dovendo in Dite,e faltn,e gira,
Che par quando mi barbera latrottola,
endar non vi vorrebbe, e si ritira
Grattandosi belande la collottola;
Pur finalmente forza ve la tira,
Come fa il peso al grillo,una pallottola;
Così ne van quell' anime nefande
Chi dal piccin tirata,e chi dal grande,

{23}Perch gli è offa,e pelle,
Ch'ei par proprio il ritratto dello
STANZA XKV,

Per la gran calca nel passar le porze sì che quand! ei si fence il tow
Connenne a ognino andarne cosa piena, Lerche la fame quivi ne c
Ma la Strega non hebbe tanta sorte, Liingoxxa, che ne manco non gli tocca
Che trenla il can, che quivi Pain catena; Ne di qua, ne di la gih per 3
E perché per tre bocche abbaia forte, Ma fubico gli venne il,,
Ella dice: Ti dia la Maddalena; Ond'ei sallunga in terra afar lai

Chil papavero,e il loglioch

Faria dormir un' orso, non
ZA X\&KVI '
Sdraiata dorme; e riss com wh tf,

E in tatotrovail pane,e in pexsiltaglia,
E in tre gole ch' egli apresgliene scaglia,
STAN

'Hor mentre fa il fonnifero il suo corso,
La donna che più la facea la [corta Lerno da bore fa versa la porta,
(Peroccht havea timor di qualche morfo) E poi (benchrella fusse alquanto [rect
Vedendo che la bestia, come morta Da unacor[aye in Dite anchrellait
L' anime rimafte attorno alla Città di Dite mostrano co' gesti,
lentieri vadano dentro alla Città; ma i loro peccati @ forza ve le
anime nell entrar della porta fecero così gran calca, che la Strega nof
star con esse, € tanto più, che ell' hcbbe 'paura di Cerbero'; 'onde'pel
fene gli gett del pane fatto col fonnifero; per lo che il cane' si ad
ella entro nella porta. E quiil nostro Poeta imita Verg.nel 6,
dare a Cerbero dalla Sibilla ana fliacciata'col fonnifero, e nelle pr
23. 24.25. parafrafa,si può dire,i seguenti versi del medesimo
Cerberus hse ingens larratn rege i
Personat, aduerfo recubans insmanis in dntro
Cui vates borreve videns iam colle colnbris 22



SESTO'CANTARE:

2 9 o medicatam frugibus ofam
Aes, Obijcit; ille fame rabida tria guttura pander
9 Corripit obietam, atque immania terga refoluit
5 Bafus humi, torque ingens extenditur antro,
Il verbo barberare è usato da' nofiri fanciulli per intender quan-
gira a salti, enon va unita per cagione dell' esser mal contrappela-

LA, Strumento;del quale si servono i ragazzi per giuocare, ed è un
foggia di piramide, che finisce in una punta di ferro, Vedi sopra
si fa girare avvoltandola con uno spago,¢ poi scagliando 3 ter-

3.

14, tirando con velocita a se la mano, alla quale e legato detto spago.
GRATTANDOS! ta collottola, Grattandosi il capo nella parte di dietro dai
tini detta cerwix. E questo e un' atto solito farsi per lo piti dalle donne, e da'
ndo hanno qualche di(grazia, o gran disgusto. Vedi sopra Cant. 3.

VDO. Vale piangendo: perché sc bene il belare e proprio delle peco-
li, e viene dalla voce, che fanno tali bestie, che suona be be, ce ne
lamo anche per esprimere il pianto dell' huomo, ma per derisione; donde si
belone, pecorune a uno che pianga assai. Un moderno Poeta disse;
Hor ch' è per te finita la pasciona,
Gf Che fai che tu non beli, o pecorona?

GRILLO. E' un verme piccolo volatile noto: Ma trattandosi di pallottole
Grilles intende que!la piccola palla, che si tira per segno nel giocare alle pallot-
tole,0 alle pialtrelle,omurelle, Vedi sotto in questo C, stan.34, e C.9.staa.17.
PALLOTTOLA, Intende quelle palle di legno, che servono per giuocare., j
nelle quali sono tre contrappefi di piombo, per via de' quali si fanno fare alies ?

F operazioni, e voltamenti, che si vuole, ' uno di questi si chiama. i
¢atetia, V altro il grande; ed il terzo il piccino, ed il Poeta, aflomigliando quell'

=

ASSES EES:

ge  *himea queffe pallottole, dice, che ancor' esse son forzate a entrar nell' inferno \
dal,¢ chi dal grande, cioè chi da i peccati piccoli, e chi da i grandi.

ye! Quantità grande di popolo; folla.

5 ANDAR con la piena, Andar co' più; andar in truppa con tutte quelle anime,

m lie piema per similitudine significa inondazione, o furia di popolo. Virg. Georg,
i lane falitantum rotis vomit adibus undam. Andar con la piena significa ancora se-
,g  Biitat Popinione comune; andar co' più.
ee AL Cane che quivi fa in Catena, Cerbero cane con tre tefte,due delle quali stan-
4  nolempre svegliate. Hercole lo lego, ed i) nostro Poeta imitando Vergilio co-
' me s'€ detto, lo fa addormentare col/pane alloppiato.
4  TTdiala Maddalena, Possa tu cfler impiccato. Dicevafi porta di Caronte da-
m gli Atenicfi quella porta del Palagio del Podestà, dond”uscivano coloro, che an-
;  davano alle forche, come accennammo /opra C. 5. tan. 3. e noi diciamo Ti dia
ta Maddalena, da quella Campana, che e nella torre del Batgello, la quale suo-
ma, <n va alle'forche, e si chiama la A¢addalena, perché con tal no-
mee » per esser la Cappella di quel Palazzo sotto i ticolo di'S, Maria

GLIE.

.

oT

264 ' - MALMANTILE:
GLIENE scaglia, Gliene tira da lontano; Glien' av
ra nen segli volle accostare.,
HAVEKIA mangiato Salerno, Havrebbe mangiato i fat, Ve
disse; fume rabida. E si trova Satylnm voraret, che batylum chi

piecra, che si divord Saturno, sDEDy
SER faccents, Si dice: Ser faccenti, o Barbaffori quasi Valvafori,
dajc,a coloro, che tutte le cose fanno, e dicono magiftralmente, ¢
degli altri; E' però detto scherzoso, e per burlareuno. Qui intende:
natori dell' Inferno. E' parola derivata dall' anti.o verbo /accio, per,

Lopio'g > vA
PER il mal gaverno, Per il poco mangiare, che gli danno... Nell
Governare le gailine; cioè dar loro da mangiare. Similmente i La
soidati pigliavano un poco di rinfresco,dicevano; corpora curare,
Governare gli uliyi disse Pier Vettori,cioè concimargli;quasi questo sia
Si servtto che tien ! anima co' denti, Si macilente, e magro,.che p
lerebhe ? anima, se non la riteneflc con lo stringer i denti. Giobbe pe
se medesimo emaciato, e confunto. Pelli meae, consumptis carnibus,
meu, ce 3
EGLI è ossa, e pelle. Non ha carne addosso: E' magrissimo, Plauto d
questo proposito Offa, atque pellis. E Dant. Purg. C.23. dice;
Wegli occhi era ciascuna oscura,@ CAVA
Pallida nella faccia, e tanto scema,
Che dal! offa ta pelie s* informava,
SPENTO, S intende al maggior segno magro.
LA fame lo scanna, Muore di fame. Vedi sopra C, 4. stan. 24.
CANNA., Intendi la canna della gola, la quale si dice canna per.
ne, che ha il gargarozzo con la canna, Dan, Inf,C, 28, f
Restato a riguardar per meraviglia she, *
Con gli altri, mnanzé agli altri aprì la canna, cnt
Onde Scannare, sgozzare: tracannare., ingollare, pha
GLI wiene il sonno in cocca, Cioè nell' eftremita delle palpebre, che vengono
chiudersi, Gli vien voglia grandissima di dormire. one
S* ALLVNG A in terra, Si distende in terra, Lmmania terga refeluit Fufus hima
totoque ingens extenditur antro; dice Verg. com' habbiamo accennato sopra.
A FAR (a nanna, A doimite. Termine insegnato dalle Balie it
imparano a parlare, per esser più facile a dir nanna, che dormire, Lala N.
Non lajcio mai certi detti che haveva imparato da bambino,chiamando pappo il
vino bombo, i quattrini dindi, e quando voleva andare a dormire, diceva
ta nanna. | Launi fimimente !'addormentarsi de'bambini alla Ninna
tilena delle Balie, da lor detta Ladus, e da' Greci Wynnini; dicevano La
MENT RE il fonnifero fa il uo corso, Ul fonnifero fa la sua Operazione +
PAP AVERO,¢ Loglio, li papavero è quell' erba, il feme, ed eftratto
quale compone l'oppio, o fonnifero; ed ii loglio € un' erba, che nasce si
ni, il feme della quale mangiandolo, dicono, che faccia sbalordire, ¢
no. £ da questi mali effetti del Loglio habbiamo un proverbio, che

Lad



oo CANTARE?.
significa Io non son balordo.
sopra C,3.stan. 32. /draiarsi è il verbo recumbere; E Ver-
u pacule recubans fub tegmine fagi; stimo che intenda sdra~
te ne stai all! ombra d' uno spazioso faggio. E nota,
, che vuol dir largo, o spazioso, ¢€ staco cavato il verbo
I eas » e passare il tempo senza pensieri, il che chia-

tifmo assai usato. *
. Ronfare: Quel romore, che si fa da molti nel respirare dormen-:

ae;
da bette. Vuol dire accostarsi. Perché le doghe, e l'altre parti
'botte son lavorate in modo, che si compaginano,, ed unilcono,

STANZA XXVIIL.
S' ell è come voi dite a questo modo:
(Si le risponde ) andate pur madonna,
Perch' altrimenti c* entrerebbe il frudo,
E voi fharesti in gogna alla colonna,
Horsit correte pria che freddi il brodo

dice) ola, che roba e quella?
i ( dic' ella) nel forame,

non ho qui roba da gabella, Che la Regina poi farebbe donna
— Se non un po a' alors' a Proferpina Da farci per la frizza, e pel rovello
Porto, perch' ella fa la gelatina. Buttar' a' più la forma del cappella,

za havea sotto alcune rame d'alloro; e da i gabellieri le fu doman-
+ ma essa con dire, che era per servizio di Proferpina, si libera,
a del gabellicre. 11 Poeta imita Vergilio, il quale fa che Enea d'or-

Sibilla porti a Proferpina il ramo di quell' albero con le foglie d' oro,
Come si vede al lib. 6, dell' Enetde.

—

Latet arbore opaca
 Aureus, \& folijs,\& lento vimine ramu
iis Iunini Tnferna dittus facer,
HELL. Quella cosa mala; cioè la spia.
> (A della fame. Ha grandissiima fame, perché non guadagna denari
omprar roba per mangiare. Quando i mestieri non lavorano si dice: i /egna-
, alzolai, ec, arrabbian della fame, cioè non hanno da lavorare.
erai il forame. Per befiar' uno, che dandosi a creder d' haver fatto
gno a [pele, e dispetto nostro, e non tha fatto, diciamo: Tx ti
. Qui vuol dire: tu credevi di haver guadagnato il quarto, che
sca alle spie, ma non e stato vero.
»1W.A, Fu figlivola di Giove, e di Cerere, la quale fingono gli an-
» che essendo un giorno a corre i fiori,futfe rapita da Plutone Re del?
» 5 fatta sua moglie: Ma Cerere non potendo comportare, che la figliuo-
imanefle appresso ai rattore; supplicd Giove, che volefic levarla dall'lnterno,
eae le, pur che ella non havesse preso cibo alcuno; Ma havendo
pina mangiato alcuni granelli di Melagrai:a non potette ulcire; Cerere di
0 supplicò, e stimolò tanto Giove, che ottenne, che Prosperina steile sei
'dell' anno nell' Inferno con Pintone, e sei mefi con la madre in Cielo. EB
. con



366 MALMANTILE o

così Praferpina ref sei mefi in Cielo, dove e chiamata Luna', @ fel

ferno, dove e chiamata Proferpina, ed in Terra è chiamata Diana,
£

triplicata efienza Verg. disse:

Tergeminamque Hecaten, tria Virginit ora Diane,
E perché la Luna sei mefi dell' anno cresce, e sei mei scema, perd'i
till finfono, che ella stesse sci mefi in Cielo, € sei mefi nell' Inferno,

+ teeth die

goer

no splenda in Terra, ed è detta Diana. A questa finzione allude Dany

Ada mon cinquanta volte sia raccefa
La facia dela Donna che qus regge ¢
GELATINA, Brodo fatto con la carne di porco, € rapprelo; e si fa
' i)

brodo di pesce. Vedi sopra C. 2. stan.

2. stan. 15.

STIZZA, Ira, Vedi sopra C. 2, stan. 78, al termine fw piccine, Era
velo, collora, e simili, i potiono dir finonimi di stizza, quando è presa in
senso; che per altro diciamo fizza Una specie di lebbra, che vien€

ad altre bestie.

S:AREBBE derma. Questo termine significa Havrebbe animo: Si farebbe
to, ardirebbe, non la guarderebbe, ed ha lo stesso significato, che Son | i

detto sopra C, 4, stan. 29.

BVTT AR 4: pie la forma del cappello, Cioè buttar la testa a i piedi j) Teoncare

il capo, che è la forma del Cappello.
STANZA XXIK

La Maga senza dir più da vantaggio,
Metr'egli a/petta un po di miacia,e intuona;
Ripiglia prontamente il suo viaggio
E incontra Nepo già da Galatrona,
C” havendo dato Id di se buon faggio,

Jn ongi e favoritoye per la buona,

Perché Breuffe in oltre a' premi, e lode

L: ha di pin fatto Diavolo a due code,
s tA NZA XXX,

Hor che gli arriva ail' improvifo addosso
4 venir della maga ch'e il suo cuore,
Lui Mago pur tagliatole a fus dosso
Le [pedisce per suo trattenitore,
Alentr'il petardo col cannon pi grosso
Sentefi fargli frrepitoso onore,
Cavalier Nepo, com' io dissi dianzi,
Col riverirla se gis affaccia innanzi,

C'ENTREREBBE il frodo, Ci farebbe la pena d' haver frodata;
nifestata la roba, per non pagare il dazio, o gabella. |
LN gogna. Alla berlina, che e quel gastigo vituperoso, che dicemhmo:

veh a

ira

it

STANZA XXXL
E perché 4 Benevento est com luk,
Com! ei di lei,bavuto havea
Won prima si riveggon ch' D
Rifanno il parentado,e o a
Tra i diavoli poi vin nei '
E percht Martinazza v' e novitidy
E non intende il gracidar che @ fam,
L interprete fa egli, e il torcimanit
STANZA XXXIL
Per via informa, ele da molti
Diufanza,¢ lyoghi,e intanto di bua e
Lo guida ai fortumati Campi Elifi
Dove si mangta,e beve hi we
E tra quei rofolaceè, € fior
Si peat tong far diquatcroeaaty

Chi un baloccose chi ut z
Che li non e un negorio per | if



CANTARE

PV paar ar XXXIV.
 Quivi si fa al palione,¢ alla pilerta,
| Parte ne giuoca al Suffise alle Murelle,
Conte carie a Primiera un'alerafroca
A confartini giuoca,¢ le ciambelle,
etri fanno a Ci è alla lotta;
indovinells.,¢ chi novelle; (gio
7 lie fiorienn'altrounramo aun fag-
ths cagliata, ¢cou esse canta maggio.
A XMXXV,
Altri piglia, o dispenfa del tabacco
Altre piglia le mosche, un' altro grilli
E cates quanti in quei traftulli immer se
Sirengonail tenor,(t vanno aiversi,
Za iL Ȣ-s?incontrd ia Nepo da Galatrona molto
o da Plutone', il quale per fare onore a Martinazza da lui tantoramata.,
i ptrartenitore, apendo cheerano amici-, Così dunque.ac-
a Nepo sche de faceva J' interprese, perché ella non intendeva il
voli, se ne passd ne i Regni-bui; edi) primo luogo, che ved-
; ono. Campi Gli;,¢ quati il Poeca descrive ripieai di quei trattenimen-
; e fanciuileschi., che (on-soliti facli dai bottegai più vili per le festivica
1 fubucbani,come sono le Ville degli Strozzi, Pucci, e Gerini, dove
f pola per godere allegramente, e seaz' un peofiero al mondo quel-
fa, che concede la campagna-, ¢-sospendere alquaato i pensicri aviosi del 1

|, | hvorare

wy

IL AMANCLA. Vedi sopra C, 2, an, 68.
bh ANTVON ARE, Vuol dive dac priacipio.al canto; Ma qui significa chiedere
y/ —SON inetti, o cennila.maacia;.¢ ci serve per intendere domandare con ceani, o
o i quaifivoglia cosa: per esempio: Ll talc insuona, vorrebbe andar' a cena,
a ta bostega, ec.
o 'aiatrova', uvuno nel contado di Galatrona luogo nel Valdarno di
gt »9. conpolueci simpatiche, o.con altro medicava tutte le ferite, e 1
yli buomini,, come di beflic,senza vedere il paziente ma folo'ia fu le» \
y@ = pezzebagnate nel faaguc di ello, o sopra un panuo, che havetfe toccato lo Rrop~
IL PiOI per le-betic in qualsivoglialor malore, pigliava la loro cavezza', o bri-
ge ia » e sopra quelli diceva alcune parole, e le medicava; e per que-
'i sua lica superflizione da molti fu stimato stregone, come lo stima il Poe~
i seers. ioe eonosciuto con Martinazza a Benevento, e che era mago
afuo-doflo.
è DAR bran fore di se.Pacfi conolcere con le sue azioni-per huomo di garbo,.
wt = o prudenre, o-virtuolo.
eo 'ER (a-buona,. S? intende,t per la-buona: strada; e vuol dire.. B' in buono
yf Mato; Gitira innanzi bene..

|, BREPSSE.,. Intende Plucone; ed \& lo stesso',.che la: Bilisrfa, colla' qual voce»
inno paura le Balic a' bambini, furfe dal Lat. \&rebus, originato così; Erehufe,,
le. Me uv Lie

—



 I er
oR

oMALMAN ore *

268

addosso una' lucertola 'con'due code sia fortunatissimo in:
larmente nel gitloco, ¢'percid vuol dire, che questo' o atiffi
grandemente privilegiato da Plutones percht haveva le due code
GLl wrriva addosso, Cioè sopraggiunge inaspettatamente a Plutone''t
Martinazza tanto amata da lui.) > nImAS a8: /
T AGLIAT OLE E'fuo dof, Fatto per appunto come lei, che havi mede
nj, ed inclinazioni, che ha lei. Traslato da' gli abiti, che si dicono sagtiati 4
doffe quando tornano bene in doflo. e 2S arand on
TRATTENITORE, Si dice quel Cortigiano, che vien di a ferniret
Ambatciatore, o altro forestiero, che sia ricevuto, e spefato'dalla Cortew >
PET ARDO, Specie d' artiglieria nota, che serve per'buteare a terra, 1¢ por
te delle Città. In Latino fa detea da Famiano Strada con' voce Greta
Pyloclaptrum, Quasi Spezzaporta. op i» 2001y(4 ob oat
RIE ANNO il parentado, et amicizia, Quando due amici stati 'lango!
Jontani l'uno dal' altro senza vedersi, si ritrovano insieme, e fanno le:
diciamo; Rifare il parentado, e l'amicixia, NOVICE Tt
VB novizia, Non v'è pratica, perché non v' è mai fata in qu
hospes, € noi lo traslatiamo ad uno, cheè nuovo, e non praticato in
affare. Lat. nonus, rudis, i ages oe
GRACIDARE, E' proptio delle ranocchie,'ma'qui intende il
voli, che forse se lo figura come quello delle ranocchie'.' Dan. Inf.
E come a gracidar si sia la rana,
INTERPRETE, e Turcimanno. Si poslono dit finonimi,se non che Znterprete¢
propriamente quello, che esplica i fenfi delle parole, e Turcimanno \& ¢
parla in vece di coluj, che non intende il hnguaggio,riportando le parole; che»
sente dire nella lingua dell' uno, e dell' altro respettivamente, Da alcuni dicesi
Dragomanno dalla voce Greca Dragomenos, che significa Znterprete usata da' Greti
Orientali de* tempi baffi; da Tbargum, che in Levante significa interprerazione,
Thirghem in Caldeo vale ¢/porre jesplcare, e da questa radice e detta a
Thargum la Parafrafi Caldea della Scrittura. Ma hoggi Turcimanne da i più q
tende ruffiano da quel portare le parole. BH
DI buon trotro. Paaiainaids di buon passo. Trotto diciamo una specied' an-
dare del Cavallo, che è fra il passo ordinario, ed il correre, ed è il latino ie:
cajare. =.
eal Elifi., Bil creduto Paradifo de i Gentili. Vedi sopra C,2. stan. 68:
e4 BERTOLOTTO. Senza pensare al pagamento, che si dice anche 4 Vf
a Youne; a Scrocco; a Salicone, Vedi sopra C, 1ftan.77, e sotto C, 7, stan. 5.
ROSOLACCH, e foralife. Specie di vilissimi fiori Aloette A
PAR di quattro, ed' orto, Seiben par-che voglia dire giuocare inuitando di
quattro, e d' otto; tutra via s' intende fMarsene senza far nulla, che si dice::
'ar ateco mec, dondolarfel4, farea tn me gli hai, ondg un nostro Poeta moderno —



H ae '
SESTO CANTARE:

ative. errno al mattarin crepuscolo
Weddin — me gli bai,
nathan 'oponete, a concludete mai,ec,
oe vine Trattenimento «Da Badalueco, che vol dire pro-
: ia.» o leggicre combattimento. Latino velitario, e figurata-
» o trattenimento piacevole. Ma la parola balocco, o balocarsi \&
bambini; e nel contado è preso per indugiare.
' grandissimo, quasi dica spazioso tanto quanto un' occhio ¢
pO +
UETT£,. Diminutivo di mucchio, che vuol dir quantità di cose riftret-
» quasi monticelletti, Latino cumuli  acerni; o Così mucchietti di gente
s d otto, o dieci tare riftrette insieme. Dan. lof, C, 27.
| B di Pranceschi fangainoso mucchio
i  » Sorte le branche verdi si ritrova..
pure il mondo in carbonata.. Diventi carbone, e abbruci pure il Mondo,
i, e vadia sottosopra il Mondo.
un fastidio di niente.. Non vuol sentir noia, o pigliarsi pensiere
che si vuole, o dibene, o di male,
. Ballare senz' ordine, o regola. Vien forse da Ballunchiare
»» chefe bene è parola non usata,pur |'usò il Boccaccio Nou. 72. pe:
ballo di contadini.:
Strozzini.Gli Strozzini,come habbiamo d., una villa de'SS.Strozzi po.
a da Firenze,così detta.Si come.il Prato del Pucci, e del Gerini sono due
aburbane.de' SS. Marchefi Pucci, e Gerini; a' quali luoghi, suole l'eftate
plebe Fiorentina'a spassarsi, con far merende, balli, ed altro, che le tor-
o,come dice il Poeta nelle presenti Ottave.
pallone  e alla pillotta, 1) pallone e una grossa palla da giuocare fata di
€tipiena di vento, alla quale si di con il braccio armato d' un bracciale
nO: ela pillotta e una palla piccola pure ripiena di vento, e se le da con
a di legno. Questi giuochi di palla, sono antichi, perché secondo Pli-
% 59. furono troyati da un certo Pytho. Herodoto lib,.1, riportato da
slid. Verg. lib. 2. cap. 13. dice, che l'inventafiero i Lidi. Alea verd teffe-
» farumgue ludos, \& pila, cateraque luforia recreandi animt gratia inventa,
a» preter quam talaria, Lydi populi Afi omnium primi, cxcogitavere \&c. Ac-
-» qui Lydos ciufmodi aleatorias artes non tam voluptatis, quam.compendij,gra-
» Un excogitafic idem Herodotus tradit, nam cum gravitate annone patria tem.
» pore Atydis Manis Regis filij premeretur, fic famem confolari foiebant, alte-
 £0 quidem die cibum fumentes, altero ludis operam dantes; atque hoc modo
, -% inediam folantes vixere annis duodeuiginti, E da' popoli Lyds alcuni voglio-
D0, siccome è Ifidoro nelle Origini, che venga la parola Ludus, o Ludius, che
lo stesso, che Iftrione. E ognun fa, che i Lidi dal' Afia pafiarono in Italia, ¢
popolarono l'Etruria, ovvero Toscana; E da loro i Latini le cirimonie facce,.
dudi, che si domandavano scenici particolarmente appreero; EB Hifer in lin.
ica, onde e detto Jfrioni significava in Latjno Ledio siccome dice Tito
3 Poi questo nome /udus significante a cae spettacolo attenente, o far.
a. m2 to



aye MADMANTILBES ¢

to per canfa di religione, si stele a significare: in generale 0
ib 1, e Suida dicono, che Anagallide Gramatica diCorfl i
mento della faltazione a palla,-cioè del gi alla palla at
a Naafica figliuola d' Alcinoo Re di' Corsu'y wolendo fare questa
il vanto d' una tale invenzione a/una sua paefana, \& veramente Naufica:
» Del reflo Di

trodotta fola tra ' Eroine da Omeru a giuocare alla palla

attribuisce quest' invenzione a' Sicionj, e Hippafo altro Autore citato da,
a'Lacedemoni,come ache tutti gli altri corporali esercizzj-E che-futie mol
to dagli Spartani, o Lacedemoni lo mottra Properzio in quel vero
veloct fallit per brachia iatiu, delY Elegia che cominicia,) Atuita tna\, t
vamur inra paleftre, Dal che si viene in chiar, che il giuoco della;palla sia ante
chissimo, e si può credere col Soutero de Jud, Veterum libs 3) Ci 14. e! id,
Verg. lib. 2. cap. 13. che questa'variazione d' origini proceda dall'havere havuto
gli antichi diverse ipecie di-paila, i come habbiamo noi', e che gli accennati ia-
ventori habbiano ciascuno 'inventata la sua species perché:se noi habbiamoiil pale
» lone; i Latini havevano, ipfe follis, pila, \& ipfis genus; conftarque V

3» to inflata. Habbiamo la pillotca', \& est ib follicuius, pilay 6 ipfa parva, \&
»» similicer conttat aluca vento inflata. Simile a questae la palia i

in-vece d' esser ripiena di vento,°¢ ripiena di borra'; Ja' qual palla hoggi per lo
più € usata da i contadini, o questa havevano anche gli antichi e la dicevano 2-

fa paganica, F 3 Syaapedt
Marz. lib. 14. Hacyqna difficilis turget paganica plama,—. bd
Folle minus laxa est, o minus artta pila; ee °

Habbiamo la palla simile alla bonciana', ma assai minore ¥ che chiamiamo pala
desina, che pure ? havevano anche secondo alewni i Latin, o la dicevano Pila
fixentina, perché forse nel pacle Fierentino si lavoratiero le miglioris Habbiamo
la palla facta di cenci impentita, che i Latini pure havevano, e- la chiamavand
co' Greci Phannida, o vero Harpafium, pesche te ne servivano per far il gvoto
y» da noi detto il Calcio secondo 1] Sipontino, che dice, Harpattumy più genu
x» elt; grofhor, quam pila paganica, tenuior, quam follis; E panno fere fit,
aliquando ex pelley lana, «omentove impletur, Non repercutitur, sed cum,
» multi fint Judentes in duas partes'divifi, ita ut utrique \&-regione fibi inwicem
oy Oppositi fine, ad fuos quilque transmittere pilam constur, quam aduerlari) co
y» Hantur arripere; Alarpajtum diem a Giseco Aarpayin, quod estirapere, quia
3» proietam pilam mulu fimul conantur ariipere, sed ob cam causam inuicem
> profternuntur, ve
Marz. lib. 7. ep. 31. Won harpasta vagus pulnerulenta rapis, A aie
Habbiamo la paiia a'corda, che terue per giuocare con'la-raechertasnelle Manze
fabricate per tale effetto; ed'etli havevano pram trigomalemscost decta 'non perché
futie di figura triangolarejma perché era triangolast la stanzajeove conefa
cavano, e per dare a questa pallayfi servivano del rericixd, che è JotieNOy chet
racchetta, o laccherta,come accennammo sopra C.3. tan, 58. -Di questa lacche
ta parla Ovid. lib. 3. 1.4 M
Reticutoque più leves fundantur aperto, - nt
Nec, mfi quam tollas, ulla movenda pila off.

s

BB aeFF FEES SEH TF ROPSCFARRP xf RE MASTS SOW HEELS Sees.
SESTO CANTARE.

tebe wit DARE Dawe
tepidum dextra, levaque trigonem.
tichi wfafie la palla-ripiena di borra od' altro pelo, si cava
; tino riportato qui sopra, e dal nome di efa, perché
» che sia decta Pia dal pelo,col quale è ripiena;.se bene altri vo-
wenga dal Greco Pefeo:, idest equo, perché € di figura sferica, che è
ogni parte, o pure ( il che € più: verilimile) dal pee rh cioè
ibrata, e sbalzaca, e perciò anche in Greco, si come in Toscana è det-
Dionifodoro antico gramatico, dove nel testo dell' Viiflea co-
leggevafi Spheran, col qual nome chiamano i Greci da paila; si di-
i scriveile Patian. come per chiosa, e interpetrazione della voce d'Ome-
questo vien riferito da Euftazio, che sopra quel Poeta il gran comento
, Che i Greci ancora havessero motte specie di palle,si può dedurre non solo
cfere stati inucatati i giuochi di palia nel tempo, che fiorivano i Greci, e
| flo di oro la Spheromachia, l'Amilla, ed altre specie di giua-
ileriti da Giulio Polluce, e dal Bulengero; ma da quello., che scrive
ino lib, 20, C. 14. dove dice, che fra i Greci giuocavano alla palla
huomini, che le donne; e ciò cava da Homero. Si trova in oltre, che
'Siracufano giuocava alla palla, ed alla pillotta per ricuperar le forze.
ex ab Alex. dier. gen. lib. 3... 21. £ si può credere, che si come noi habbiamo
'diverse palle, e diversi modi di giuocar con esse, così non mancassero a loro ancora
l'invenzioni per soddisfart.
| AL soffi,£! un giuoco solito farsi per lo più da ragazz' in questa maniera. S'uni-
-eono dues più ragazzi,¢ pigliano una pietra, € polatala per ritto in terra vi
c ra quel danaro, che son conucnuti di giuocare, ed allontanatifi in,
liftanza, che sono-d'accordo, tirano una jaftra per uno ordinatamenie
q i¢tra ritta;sopr'alla quale sono i denari,¢ che si chiama il Suffije se que.
A 'Wien colpico, e fatto cadere,i danari, che cascano,sono di colui, la lattra.
del quale ha fatto cascare il fut, se però sono più vicini alla sua Jaftra, che al
- fal moneta, cheè più vicina al fufli,te gli rimette sopra., e quello a,
cui) ira, e seguitano,come sopra, tanto che la moneta mefia sopra al sul
“tei finita'di ievare ne) modo, che s'è detto. Da questo giuoco: habbiamo un,
Proverbio che dice Efer w/usfi, il che significa esser queliberzaglio, dove ognuno
“tira,cioè sopra il quale devon cadere tutte le burle,-c tore le minchionature..
 Questo giuoco \& forse lo stesso, che da' Greci era detto Epbedri/mo, feconge Giu-
Tio Polluce, il Buieng. c. 48., ed il Meurs, de lud. Graecor,, te bene non giuoca-
~Vano denari, ma colui, che non butcava in terra ai futh,portava a cavalluccio
» quello, che to bucrava,il quale gli turava gli occhi colle mani, finche (enza errare
bb portale alla Jafira, o pietra, che si chiamava diores, cloe Adeta o Confine,e
-facevarquello, che comandava il vincitore, il quale in questi loro giuoch era,
hiamato'Re, ed il perditore era detto Mida, 0. vero Asino, come habbiamo vi-
“flo altrove.
 | MPRELLE. E? giuoco simile alle pallottole, se non che in vece di palle ado-
oo eee laftrucce, ed un piccolo sasso per grillo, e tal giuoco si dice anche pia-
eae wee «®



27 MALMANTILE? 4

PRIMIERA, Giuoco noto, che si fa con le carte. = =)
FROTT A, Flotta,, o fiotta. Vuol dire quantità di gente unite i

muove; dal Latino fluttus, Virg.Georg. dane Salucantum toris vomit edibus andi,

Varchi Stor.lib. 15. Z vedendo sopra a un monticello non molto
frotta di. contadins. DEIN
CIAMBELLE, e confortini, Sono specie di pafie fatte col zucchero.
uova,e queste son poe a vender da alcuni pi pel contado, dove si
¢ raddotti, che in Città; e questi portan seco anche le carte per giocare,:
quali hanno diverse invenzioni di giuochi, come la mora, il tocco, ec. E
venditori quando giuocano, danno in vece di danari quei confortin', e ci
se perdono; e se vincono,ricevono danari. L, circali, ernfiula. /
CIVETT-A, Quel giuoco fanciullescho, che dicemmo sopra C. 2, stan. 41.
INDOVINELL1, Latino griphi,enigmata;Quello che in latino dal greco i
enigma, noi circoscrivendolo diremmo detto oscure, e disseile a i
E la voce enigma s'¢ fatta Toscana, e l'usiamo come l'usò il Malatesti nellay
sua Sfinge, Vedi eras 8. stan. 26, a vege!
CANT A Haggio, Nel principio di Maggio sogliono le Ragazze plebe
di Firenze, o del Contado sbeceane scomeni foe Pr) eee ¢
di joro-in mano un ramo d' albero adornato di fiori andar cantando ere
diverse canzonette per l'allegria del nuovo maggio, e per buscar mance da
ro, che si pigliano i) passatempo di farle cantare al suono d' uno strumento.
cembolo, che è un' assicella ridotta in cerchio, e fondata di cartapecora da una
parte fola a guisa di tamburo. Questo costume di rallegrarsi il Maggio viene dal'
antico, e si trova, che appresso i Romani Xalendis, Nowis, @ Zdsbus maij Lari
Deo facra ficbant afello panibus coronato, Quindi forse ancora Maggio si chiamavdl
mese de gli Afini, che per altro fu detto men/is hilaritatis, Che nel mefedi Mag:
gio si faceffero allegric forse più di quello, che comportafie ' onefta, e lavert-
condia, ne fanno fede gl'Imperatori Arcadio, e Onorio nella loro Costituzion
inferita. da Giustiniano nel Codice lib. 11. 45. de maiuma, la quale era una alle
gria, che si faceva per il Maggio secondo che spiega Suida. Da questo mele quel
ramo d albero, che i contadini piantano la notte di Calen di Maggio avantiall
uicio. delle loro innamorate, si chiama Adaio; Questo costume d? appiceare:
maio alla casa della Dama e riferito come proprio anche della Francia da Mat
ziale d' Aluergna ne' suoi Arresti d' Amore, all! Arresto quinto, il quale scritt
re fiori nel 1400. Qual luogo Benedetto Curzio comentando dice; Prima d
Maij menfis invenes pluribus lndis, ac iocis fefe exercere confueverunt, arborem sapean
mero deportantes, ac in loco publico, aut etiam aute alicnius egreci virt januam 5
frequensiits amica fores pl vel promifinis adamantil
Signijs, atque emblemucibus.

isarote
BRANCO., Quantità di popolo indeterminata; ma si dice più di bestie; com?

branchi di polli di pecore,di buoi, di aGni, ec, Vedi in questo C.  Ortawa
seguente. 4 sacl
HA mofo Oste a sacca. Cioè mangiato, e bevuto quanto l'Oste vi haveva-»
nel modo » e con quella furia, che segue nel dare il sacco a una Cittas ae
sopra:

(EN-

MEZZL brilli, AMez2i briachi, Brillo vuol dir briaco allegro. Vedi \&
2, stan. 69. At

Bee coc eee e2@Heekse ewer es) =!

eseF SF AS Soa



273
a bacco, Una villantella che si canta per incitare
Aidlors eg MeL gave

»\\) Faceiam brindis a bacco, o vi:
questa,va il bicchiere attorno, ed ognuno' beve,intuonando prima.
però dice mentre ice 3 cioè mentre il bicchiere va a tor-
/perché tal-costume è usatissimo in simili allegrie, però il Poeta, che s' in-
mostrar, che quivi si sta in felte, e in giuoco, dice che facevano brindi
jot cantavano'bevendo. I Latini dicevano Propinare, cioè prebibere dal
tim, che suona lo stesso che il far brindis, ed usavano anchressi questo
bere in giro', che dicevano ia orbem bibere, \& circumferebant feyphum
¢d essi pure cantavano in tale occasione di bere; come scrive Dione, che
e Roi aC l'quando al banch che fece
bevve a un bicchiere, che li fu porto da una bella femmina.
'brindsfi. Se ben pare che venga dal Tedesco pringen, percht volendo
) a*nazione bere, 'ed inuitare il compagno,suol dire: Zk Mellan-
'y che vuol dire' /o ve lo presento; e questo già facevano, perché quel vino,
havevano'a bere restasse benedetto dal Compagno, il quale foleva rispondere
nges, che vuol dire Dio lo benedica. Tuttavia il Lalli nella sua Moscheide
“61. graziosamente gli da origine dalla Città di Brindis, dove chi va ad
f è da ogni vessazione curiale tanto Criminale, che Civile, onde a.
 faevbrindisi par che sinviti uno ad andar ad abitar quella Cited, cioè a lasciar a
¥ 'parole del Lalli son-queste: i
|. Brindisi bella s io m appongo al vero,
Date son messi i brindisi in usanza,
Quafil buom dica; Lascia ogni pensiero;
Beviamo allegri, e rinfreschiam la panga 5
E se pui il creditor duro, e fevero
Ci fa da' birri apparecobiar la franzAy
Brindisi habbiamo, Brindisi diletta,
Che quanto più si bee, vie pss n' alletta;
paglie,o /pilli. E' un giuoco da' fanciulli, che si fa così: Pigliane
due corte fila di paglia, e posandole sopra un piano liscio vanno
'4spingendole con le dita tanto, che uno di detti spilli, o fli cavalchi l'altro,e quel-
lo, che resta di sopra vince, giuoco così detto dal Ter?, cioè tog/i, regi. In La-
tind tudere aciculis, E perché questo giuoco-¢ di niuna, o poca conchiufione, hab-
- biamoil proverbio s Fare 4 tes? con gli spillerti.; che significa affaticarsi, e per-
'Mere il tempo senz' utile, o profitto; ed esprime ancora Far una cofacon fordido

S' aiutano l'un l'altro, e's' accordano.
Ww XXXVI.

me ZA
| “Za donna refia litrafecolata,. Per tutta la Cutta vien falutata,
o) Fedendo quanto bene ognun si spalfa; E infinle stanghe,e ogni forcon s'abbassa,
i he Nepo l'ha di già infor mata, Ed ella hor qua,bur la voltando inchini,
ragiona di lorma guarda,e pala: Pare nna bandernola da cammini.

STAN:

ST vengone il tenor, si vanno a' versi.
* STAN



a MALMANTILE::
STANZAXXXVAL. STANZA KEI.
è

Pera che tutti quanti queit Demoni, Percio pafjano in casa o

Per vederla, n'uscian di quelle grote, Firatoconta Stree il Reda
'Ronzando con un brance di moscioni,. Le da la ben vennta,e vento
Che saggirin d'actorno.a una-botte, dt Le ipieaie sete: nar d
Saleellam per de strade,e fui balconi, Ella per conseguir ogni suo jutento:
Comal piover a' agosto fan le borte, hf "

See

E stan. vedendo [ue fembianze belie,, bail
Voei altese fiochese suon di man con elle, -grazia anchiet di dar i
STANZA XXXVIIL, STANZA KXKK Og,
Così fra quel diabolico rombarzoa Sta pur, dic' ey cont anime (ato, g
La fhrega se ne va con lo stregone » C' a servirts mo mo vuo dar di piglia,,
Sic alla fine arrivanoa Palayro lo.già,come tu fai, baveo impranate;,
La-dove s' abboccaron con Plucone, Ma il tutto.d andato poi in iscompiglia

Ma. perché tra di laro entré-nel mare Horfu: fra poco adsunerdil fensta,. ¢
Scwwccamente il eALandragora buffone, E sopra questo si farò confizka,.0
Chiin quel cailoquio fesigran fraftuono, eAicio Baldon batta ig ritirates,
Che finalmente ognuno uscs dt tnono-, E tu reftt consenta,, econfolata,. >

Martinazza resta maravigliata, che costoro stieno cos: allegramente, e pak
fando pel mezzo a una infinita di Demonj, che tutti la riveralcono., giunle com

Nepo a Palazzo, dove se le fece incoptro Plutone, che la condutie dentro\,,e>

quivi havendole essa detto il suo bisogno, Plutone se peomess di confolarla..

REST A trafecolara, Resta.maravigliata: Strabilisce. Vedi sopraC, 1, st. 28.
ST ANG A. Pezz0 di travicelio,.cioè un legno grosso.pii d' un bastone.
FOKCONE.. E'un' afta'di legno sopra, alla quale e adattato un tridente di

ferro, e serve per uso delle Malle.

INC HINO. Vedi sopra C..1, stan. 34:

BANDERVOL A da Cammini, Bandecuola vuol dir piccola bandiera, o pen

noncello, che è quel pezzetto.di drappo, che già portavano:j Cavalleggieri appie-

cato vicino alla punta dellatancia a guisa di bandiera; ed a guisa di questa im

Firenze se.ne vedono fatte,.di lama di ferro potte in più. emincati luoghi delle

case, come sono le pergamene, dond' esce il fumo dei cammini, equelte servo

mo-per far conoscere i venti col lor girare, € voltard in sul. ferro:,.nel quale sono
jnfilate, e bilicate; ed a queste allomiglia Martinazza, af

RUNZANDO. Ronzare si dice propriamente delle mosche; e però.dice Ce

me fanno i moscioni, che sono-quei piccoli- vermi alati, che nascono-dal vino.
COME fanno le botte-al piover a' Agosto.. S'¢ veduto dalla sperienaa, che la piog-

già di state,cascando nella poluere scaldata dal Sole inuigorisce le rane,.o bore
nate di poco,,se bene molti. hanno creduto, che le faccia nascere quell" acqua,coo

quel Sole; il che è falfo, perché prefe (ubito scappate dalla poluere si son trovate

Exé SE vo ESE ZZ mERZ SHE aes TLS E

col ventricolo pieno-d' erba,. Ma sia come si voglia,basta che atal”acqua

gono-faliar, ma:d'.un salto debole,.¢ fico, spon come il Poeta: sesptl-
“mere, che faltaflero quei Diavoli. Va Posta faccto Piorcntinodel¢rivend a
* sual. cavadi Mancii-ia. unsuo Sonetta-dice:; i

Seae


H SESTO CANTARE. 275

ro Si fivergognan che passan di norte,
seat oe oe ae feet
Pe " sides \ Trottando, e faltellando come bette,
OCT alte, e foche, e suon di man con elle Così canto Dante Taf. C. 3.
Ke @”. Intendi frida, e'colui, che continova 'a gridare,afhoca per l'affati-
"cam 10 dell" aspera artetia, sì che il secondo nasce dal primo. E /uon di man,
im elle'; cide con quelle voci accompagnano il romore; che fanno co: batter le»

| “ROMBAZZO, Rombazzo vien dal verbo rombare, che vuol dir ronzare,o frul-
Tare, che @ quel romore, che fa perl' aria una cosa lanciata con yiolenza, e si
'Piglia per ogni sorta di Arepito', o fracasso. I Varchi Stor.lib. 10. in questo me-
¢ signiticato dice bombagze voce formata dal fnono,nella stessa maniera, che
; )formd bembus: Torma eAimatloness implernnt cornwa bombas, perché dice
lunge prombertare, e spampite fatve con invredibile Bombazzo, quasi in tal
Mn paffero e nimici. Ma |'Autore oon Storia di Semifonte dice al trattato 4,
! atone la T erra;allotra fenritosi per quelli della Città il rombazxu, E V'ulo
puechecl obbiighi adire romibazo v

| le nel mazzo, S'accompagnò con loro, Che diciamo; incrn/carsi, fic»
è ien'dal giudco del mazzolino detco sopra C. 2. stan. 46.

' a apora', Costui era un buffone, o più costo un matto di Corte, che
)  chiacehierava (empre', e senza propotito, o conchiufione.

j KeVIO. Voce latina fata di rado in Firenze,e vuol dire Ragionamen-

LSS" CO

to, che fanno insieme due, o più persone. Corrisponde alla Greca Dialogos, che

ifica indo la: parola Znrerlocutio, dilcorlo che si tiene fra due,o più persone;
dai Pranzefi detto Emtrerien quati Trarrenimento,

VSCH di ruono, Perde¢ il flo del ragionamento,si dice anche:a/cir di tema, fmar-

rite argumento, il proposito. Vedi sopra C. 2. stan. 47. E' pref la similicudine

3 feherzando fal doppio significato della parola scordar/f |a quale tan-

tOVi dice' d* un' huomo, che non si ricordi piti di quel che ha proposto di dire;

'uno stromento, che non sia in corde, e non sia temperato al giutta

O@ vnc, che non canti ginflo, e fuor del legitimo tuono, il che si dice

TIRATISs de banda. Condottisi in un' altra parte della stanza, feparatifi, o
allontanatifi da quel congresso.
CHE vento? bat [pinta in quelle bande, Qual cagione l'ha moffa a andar in quel
0". S.

 TRABALLARE. E' quell ondeggiamento, che fa uno, quando non può fo-

: in piedi, Mattio Franzefi in lode della Posta dice,
cd Chi domanda per nome la cavala,

Chr eels ha sentito dir, ch' è favorita
Wy Poi partendo chi trotra,e chi trabala,
Qui vuol dire, che Malmantile era in pericolo di cadere, cioè esser preso da Bal-
done, Diciamo in questo senso anche baienare, barcollare, \n certe rime mano-
scritte nella' Libreria di S, Lorenzo, si dice d' un cotto, che barcollava: Es'¢
Yalena, @ non batena a (ecco. Qui si nea sul doppio significato di balenare.
? i a Mo

5 FS = BERS SRE SA

*

=

276 MALMANTILE o
40! md, Adesso adesso. BE' il latino, omb:
Firenze. L' usò più volte Dante nel suo Poema,si co
re altre parole Lombarde; B il Bocce. Nov. 32. 4@ vi
Jaca della donna, ch'era Veneziana. prey
DRO' di piglio. Dard di mano, cioè comincerd.
ficaya quasi quel, che 1 Latini dissero Expilare; i Franzefi p
dier nel sangue, e nell' aver di piglio, E.'l suo cont ) Fazio.
Poema che fece in terza Rima, ove e introdotto Solino a dettare a.
di ecografia, e del mondo; che perciò lo intitold Dita mundi,
andò; dice così al canto 142. ( Parla del Saladino ) i
a Costui per (un francherza, e gran consiglio,
Tolfe la terra fanta a' Cristiani
Vincendo quegli, e dando lor di pe lion SNe
FLAVEO imprunato. Havevo ordinato il rimedio, Vien da quell' imp
che dicemmo sopra C, 3. stan. 21, Addio fave. i
HOR sik, Termine esortativo,e conclusivo,e diciamo nello stesso senso. 01
quasi Or via, Latino Eia age, Vedi sotto C, 12, stan. 47. Diciamo + hor fu
diciamo bac ipfa hora furge,\& hoc factas, aga
BATT Ala ritirata, Sene vada da Malmantile, Batter la ritirata
0] tamburo si fa quella fonata, per la quale i soldati intendono doversi
¢ lasciar ! impresa. Gio, Villani ciò disse; fonare (a ritirara; quali
il Franzele, retraite..
STANZA XXXXL STANZA XX
lo ti ringraxio st, ma non mi place Dico di Gambasporta il tuo v
Percio ( gli rispond' ella ) di manieray
Ch'ionon voglia pivlinr laspada'lgiaco,
Ch in bagnela son più di quel ch'io era;
Così con quei due [pirti bavend» ilbaco
Sagginnge (per c' alor vuol far Japera)
to Vho con quei briccon furfanti indegni, Ches'egl adaffe un polafruftain ve
C' hanno frurbato tutti i miei disegni,  Imparerebbon per nn? aitra volt
STANZA XXXXIII,

E di quel pallerin di Baconero 4
Che ib oo | giuoco con “nt i
Scambiando it color bianco per
Error, che nol farebbe anc' un cava
Ma e'vit che gli strapazzanoil.

See

7

a

a oi

a

Risponde il Re; Facciam quanto ti piace, Non penso di restar già contumace, —
Ma ti verranno a chieder perdonanzA, S'io non ti servo,perch'ioso a fidanta. '
Sì che ru puoi con offi far la pace Dunque ti Lascio,e fone al to piacere;
Pero racquiera,e vane alla tua fiaza, Fatti servir da questo Cavaliere.,

Martinazza ringrazia Piutone, e dolendosi del danno cagionatoli da si:
florta, e Baconero lo prega a gastigargli: Plutone l'esorta a placarsi,¢
che andranno a chiederle perdono dell' errore; e fatte con essa sue cirimonie
rimanda alle stanze. a '

WON vogtia pigiar la pada, e ilgiaco, Non voglia armare contro di loro per
yendicarmi. 4 =i eee
SONO in bugnola, Sono in collera, Bugnola si chiama un' arnese fatto dicot —
doni di paglia, entro al quale si conferua grano, biade, ec, da i Latini detta
cumera, Bfidice cfler' in byguola, nel bygnolone, in valigia, nel gabbione ee

eS.

=

asget


SESTO CANTARE: 277

ee in cOllera. E tutte queste manicre vogliono esprimere il gonfiare,

un fa per l'infiammazione della bile commofla. Orazio Bile wmer iecar; dove
vaveva detto: mexm iecur vrere bilis, Ovidio ne' Balti, Intumuit Luna, cioè
onfidzentro in valigia. Gli Spagnuoli similmente dicono, embotijar/e.

| HAVENDO il baco, Havendo ira. Traslato da i cani, i quali quando hanno

: \& n certo baco nella lingua per di sotto, par che sieno (empre adirati, ed il simile,

segue ne i Montoni,quand' hanno il baco, o tarlo dentro alle corna,
'AR la pera. Anticamente s'abbruciavano i corpi morti sopr' ad un monte

; = » qual monte quando era accefo, chiamavano ?yra. Lall. En, Tr, lib. 5.
teil
ae: Già l'alta pira di Didone ardea,
pate £ vibrava lontan fiamme,e faville.
'Edda questo credo, che venga il nostro far /a pera, € che s'intenda anche am-
na 'uno,quasi si dica: 40 vagtio far /a pira al tale, S' intende anche far /a [pia

'AR fallo, Far' errore. E' termine del giuoco di palla: e però il Poeta se ne
ue'; perché l'errore fu fatto con le palle. Properzio lib.3. we pila veloces fal-
lit per brachia iattus.

NOL farebie ance un cavallo, Error groffissimo, e che non lo farebbe anche
una bestia; e si dice un cavailo, perché questo animale pare, che habbia discorso, e
giudizio pi che ogni altro animale. I Greci di sippos, che vuol dire cavallo, se
ne per una particella, che aggiunta a' nomi, importa grandezza. Hip-
pomarathram id @ il finocchio faluatico, e Aippomyrmeces, certe formiche,
che passano di grandezza l'ordinarie, e comuni. Onde errore,o sproposito da.
cavalliè mn' error grande. O pure si dice così, perché sia degno di cavallo, cioè
i  di gastigo, qual si suol dare nelle scuole a' fanciulli.

STRAPAZZANO il mestiero, Cioè nell' operare, non considerano quel che

f — eANDASSE la fruspa in volta. Se la frufla andasse attorno. Se fussero di quan-

~ doing bastonati, fruftati.

\ NON penso di restar contumace, Termine di cirimonia, che significa:non penso
di commenter mancamento. La voce contumace e Latina; però il Lettore si può
soddisfare circa i suoi significati,

FO 4 fidanza, Confido, che per tua cortesia non l'haurai per male, e mi scu-

A ferai; termine usato fra gli amici intrinfechi,¢ si dice anche; Fo 4 sicurta,

SONO al tno piacere. Termine usato da' superiori con gl inferiori, in yece di

fF aetgeps Coenen, torent N

ia 'avaliero, Intende Nepo,

J STANZA XXXKXIV.

( ses mena allora alle sue spanve, Ove gli orsi facendo alcune danze

f Cha parameti havean di quoi darn ats Dan la vivanda, e da lavar le mani;
; Ricamati di signoli, e di stianze, Volati al cibo al fin,come gli affori,

| Efepeans di via de Pelacani Sembrano 4 foe fo due toccateri
= Nn 2 STAN.

278
STANZA XXXXV.
Fioritaè la tovaglia, e le faluetce
Di verdi pugnitopi, e di froppioni,
Saldate con la pece,e in piega frrette,
Infra le chiappe frate de' Demanj.
Nepo fra tanto a mavinar si mette 5
E cheto cheto fa di gran boccont;
Osservando Caton, ch' intese il gioco,
uando disse: in connito parla poco.
STANZA XXXXVL
Fa Martinazza un bel menar di mani;
Ma più cheilvitregli occhi al finfi pasce,
E quel pro falie, che fab erba a'cani,
Che il pan le buca, e sloga le ganasce,
Perché resce vi son come trapans,
Ne manco se ne pro levar con Pasce.
Crudveilcarnaggio,e si tirante,e duro,
Che non viene apuntareipiedi al muro,

Prexioft liputvl seve ne foaè i
Portati ciascheduno in sua guafiada
Essendovi aqua fortes inchifro bus
Di quel proprio, c'adopera to Spada,
Ella che quivi frar voleva in tnano, ~
E non cambiar, partendosi,la strada,
Perché i gran vini al cerebro le danno
Ben ben Vannacqua con agrestojeranno,

STANZA IL

E fatte due tirate da Tedesco
La tazza butta via subito sm terra;
Lero ch'ell'è di morto un teschio fresco,
Che suona, e tre di fa n'ando fotterra,

Nepo, che mai alzi vif da defeo,
E intorno.aibuon boccotiratohaaterra;
Anch'egli al fine dato a tutto il

La bocca follevo dal fiero pasto. '

Nepo conduce Martinazza alle sue stanze,dove era imbazidita la mensa, e fu-
bito si mettono a mangiare. L'Autore descrive la qualita de i j dell'
imbandimento, de i trattamenti, € de i cibi, tutto appropriate a uno apparta

miento, e banchetto da Diavoli.

QVOF humani, Pelli a' huomini. Se ben quoio vuol dir pelle di bestia conciat'
si piglia ancora per pelle d' huomo, come s' è veduto sopra C, 4. stan, 20, € com

lo prefe il Ruspoli dicendo:

Un certo ch' in full offa ha tecco il quoio;

SIGNOLI, Specie d' apostema nella cute;da i Medici detti Puruneull.,

STIANZE. Quelle croste, che fa nella pelle la rogna; o altre bolle; dai
Latini dette erafe. Varchi Stor. Fior.lib.1.4, G4 trewarono roso dello fomaco quant
un giutio con una fianga nera sopr' a quel roso.

SAPEAN di via de' Pecalani, Puzzavano di bestia morta di più giorni, La
via de' Pelacani si dice in Firenze quella; dove son le conce delle pelli, nella»
quale e sempre un puzzo orrendo cagionatoye dalle conce, € dalla corruzione di

quelle carni

VOLAT I al cibè come gli aftori, Entrati a tavola veloceiente. Avventacifi al
cibo, come fa l'afore, il quale, benché habbia il cibo a fuc dominio,vi s\avven-

ta,¢ lo divora con rapacica grandissima
SEMBRANO 4 solo a fol due Toccatori,

Dicemnmo sopra C. 2. stan. 66. quel che

* sieno i Toccatori, Questi sono solamente due, e volendo andare a cena all'osteria
son foreati andar da lor due foli, che le conversazioni de' galaathuomini oon

li

ee Fa ESOS ee Oe Ren eee se. Se

Rey



2
7
e
e

=

>

=

SE Tyr

,
'
j
7

f

it

RET TE in piega, Le faluctte, e tovaglie si piegano in diverse maniere, ¢
si fa loro pigliare la figura, che si vuole, col tenerle così picgate Mrette in un tor-
0, ofirettoio fatto a posta per tal' effetto, in vece del quale strettoio, quelte
0 state strette fra le natiche de i Demonj; e ciò dice. per esprimere, che fon

Deseo
 ANTESE il giuoco... Sapeva,come era conueniente fare, quando disse: Pauca in
loguere.
 FAun bel menar di mani, Si tadia; 8 affatica a mangiare, Vedi sopra C. 1.

LE f il pro, che fal erba a cani, Non le fa pro. Quando i cani mangiano ler.

REST E. Quei fili sottilissimi, che stanno appiccati alla spiga del grano, dell'
orzo, e della fegale; dal Lat, aris.

TRAPANO., Specie di succhiello, o foratoio atto.a bucar pietre, ferro, ed
i altra materia per dura che sia, e s' adopra facendolo girare com una corda,
loi ? habbiamo dai'Greco Trypanon, Vedi sopra C. 4. stan. 73.
NON se ne puo levar cont' ace, E' così duro, che ci vuol l'asce alevarne un

iT NON viene,a puntar i piedi al muro, Non se ne può strappare a fare ogni mag-

\ BAR lo spiano a casa a altri, Mangiare a casa d' altri (enza spendere. Vedi

\o, 3. stan. 51. Questo detto viene dallo spiano del grano, che vien dato
dal Magiltearo dell” Abbondanza a i Fornai per fmaltire il vecchio, che si ritrova
hei magazzini pubblici, e da questo rifinimento /pianare, o far lo spiano a casa d'
altri intendiamo rifinire, o consumare quello, che colui ha di commestibiic in,

ECASEO barca, e pan Bartolommeo, Precetto della squola de' ghiotti, che vuol
dire Mangiar la midolla del cacio, e la corteccia del pane.

FREMERE. EB' voce latina, che conferua appretio noi lo stesso significato:
'Verg, 1. Bn, Cuntti fimul ore fremebant, E altrove descrivendo il Furore; fremir
borridus ore cryento,

BRANO, Pezzo dicarne (forse dal Latino membrana ) o altro strappato
- violenza, e si dice sbranare; e sbranalo, Vedi lopra C, 2. stan. 52. mandato
abrani.

CIBREO, Guazzetto fatto di colli, e ventrigli di poli, 2zinueal. Può essere
Originata questa parola dalla Latina Gigeria. Feito Gramatico: Gigeriaex muiti

js ape.
MAGNANO, Quali machinarius fabbricatore di ferti minuti, e di piccoli ia-
i: Begui



280.
gegni,come chiavi, toppe; a distinzione di Fabbro, che

ine Zappe, vanghe, ec, e del Manescalco, che fabbrica ferri

ci¢ i Magnani son sempre tinti di nero, il Poeta dice che il cil

lero interiori,per esprimere, che era nero. ate sh SOR G
VENTRIGLIO, Ventricolo degli uccelli; in altri luoghi oe theta “i

STRIGOLI. Diciamo quella membrana, o rete gratia; che sta
budella degli animali. i

ACQVA alle mule', BE! un detto di gente baffla, che significa date

GVASTADA, Vasewto di vetro corpacciuto, e col collo lungo, e fire
serve per lo pia tenervi l'acqua per annacquare il vino, quando si beve.
antichi dissero Jngaifara. I) Canini la fa venire dalSiriaco Ga/far, che' v:
stesso. Potrebbe anche comodamente dedursi dal Greco Gaffer, che v:
corpo; e così Guaftada esser detca dalla figura corpacciuta, nello: stesso
punto che Grassa voce Siciliana usata dal Boccaccio neile Novelle in ]
te viene, si come molte della Sicilia, dalla Greca Ga/ira, un. poco tt
Iettere; la quale significa as vaso che habbia pancia,

LO Spada, Valerio Spada celeberrimo Macftro di scrivere, huomo si
¢ che non resta addietro a veruno nella galanteria del tratteggiare con
mano, e frappeggiare, e far paefi con la penna; come d' intagliare in
bulino, e acqua forte, fu amicissimo dell'Autore, e suo scolare nel
vive ancora; € ben che d' eta sopra settant' anni,indefessamente lavora p
nare il suo nome.
VOLENDO star in tuone, Volendo star' in cervello, e noms' imb
CAMBIAR la strada, Quando vogliamo dir copertamente a uno, Tu sei bria-
co;diciamo. 7' bai fmarrita la Strada, e pero intende;non si vuole imbriacarey
KANNO. Acqua passata per cenere, detta anche 4/eia, dal Lat, dixivinm, I
dottitfimo Ferrari nelle Origini della lingua Italiana, dice così; Ranno; lixsvia
Vude vox ortum trabat, omnibus vestigijs indagara battenus fefellit, Chi fa, che non —
si origini dalla voce Greca Xhanis, che significa, Milla, gocciola, perché il sann0
stilla a gocciola a gocciola da que! vaso, che perciò diceli Colatos ? (ee

ia del vinos4

SEP FRE SSS pec he es ee:

a:

F,

4 ef ert Fz Ez.6 =F

FATTE due tirate da Tedesco, Fatte due gran bevute. Manda gli
Latini dicono: pocula obducere;i Franz, avaller,

SVONA. Di questo verbo fonare ci serviamo per intender copertamente

mere.

l'MAL alzé vifa dal desco., Stette empre attento alla roba, che era in tavolas+
Termine usato per intendere uno, che a tavola mangi con avidita, e non pigli div —
vertimento di sorta alcuna; E de/co se ben vuol propriamente dire la tavola,dove
si sta a mangiare, onde il dettato: Chi non mangia al desco Ha mangiato di,
oggi e poco inteso per altro, che per quel iegno, sopr' al quale i macellari tagliae
uo la carne, e per quel banco, a) quale nelic Confraternite, o compagnie de'
colari ficde il Giovernatore,
TIR ATO ha aterra ai buon bocconi, Ha mangiato assai de' buon bocconi 5
lo fico, che menar le mani detto sopra. aM

La bocca follevs dat fiero pasto. Verlo di Dante Inf, C, 33. Lascid star dimane
giar quell' orride vivande,. ioe

- STAN-



SESTO CANTARE 281

on SUBTAN ZA. STANZA LI.
Lasciatii voti, e i piatti scemi Spargon le rame in varia architettara
in anno al giardino pieno di emente Scheretri bianchi, e rosse anotomie,

Di berlines di mitere,e di remi, Gi aborst,i mostriei gobbi in fu le mura
Edi strumenti da castrar da gente; Forman spailiere in lnogo di lumic;

ifede in me2x0 sl paretaio del Nemi Dugna, di denti, e simil' ffatura z

, D'un pergolato,il quale a ogni corrente Lnfeliciate son tutte le vie;
i con quattro braccia di cavexra, Non bel fepoteroanicchia il fore butta
. Penoloni, che sono nna bellexra. Del continuo morchia, e colla strutta.

ee STANZA LIL
sono abbroffolite, e feure Sui dadi i torsi nebili feulture

ie del mar venute della rena, (Perch'in rovina il tutto iitempo mena)

'intorno intorno in varie positure Rispaurati sono, e rifarcité

d | Sep rem leggiadra scena, Da vere, e fresche tefte divanditi,

lito che ro di mangiare, Nepo condufie Martinazza nel giardino. Qui
icipia a descrivere un giardinu da Diavoli mostrandolo ripieno di tutti quei
Malanni, e disgrazic, che-alla giornata accadono a i mortali.
 LASCLAT Li bicchier vori, e+ piacti scemi. Havendo bevuto,e mangiato quan.

piaciuto.
> GLARDINO. Luogo dove si piantano fiori, ed altre delizie similida i Latini
detto Florarinm, fen pomarinm. Vicne questa voce dal Tedesco Garten, e questo
dal Latino bortus, secondo il Ferrari, ii quale biafima il Perionio, che la fa ve.
nire dal Greco ardevein, innafiare, seguitaco in ciò dal Monofini. Ma tanto que-
glinella sua lingua Francese,quanto questi nella nostra Toscana,sono troppo appassionati
acl far venire le voci dal Greco yilche non \& sempre vero, ch' elle»

ee ee

'vengano.

 SERLINA. Gogna. Vedi sopra C. 2. stan 15.,¢ C, 3. stan. 62,
HITERA, B) quel berrettone, o cartoccio di foglio; che dalla Giustizia si

fa meteere in testa a coloro » che sono fruftati in full asino. Vedi sotto Can, 12.

Bt

KP.

» 4h Pareraio del Nemi. Intendiamo le forche, perch queste son fituate in un
campo, che era, e forse è ancora della famiglia de' Nemi, e lo diciamo Pareta-
40 per coprire il detto.. Li Pareraio € un boschetto fatto per uccellare a fringuelli,
ed altri uccelletti simili nominato Pareraio dalle retiyche s' adoprano a tal caccia,
le small fichiamano parere. Vedi sopra C. 4. stan,27. al termine mandatoin Pic.
tardia,

~ PERGOLATO., Le viti che sostenute in aria da pali, e pertiche, formano co-
Me Una coperta, o tetto si dicono pergole » O pergolati, come dicono anche i La-

Se

 CORRENTE. E10 stesso che travicello » cio® un legno lungo,grosso più d'un
> € 8' adatea a formare, e sostenere i palchi, e vetti delle case,

| 46 a¥LZZA. § intende quella fune, con la quale si legano per il capo le be-
-..

| fit ye però € detta cavezza-quasi capo, e il Poeta la chiama così, perché è lega.
, 2 per il-collo, ecapo degi' impiccati a quei correnti, e gli chiama Penzoli, per-
-Sh€ gli figura grappoli d'vua pendenti a questa pergola,;
, BRA: Shek;



282 MALMANTILE- 2
SP-ARGON le rame. Gli alberi che sono in questo.giarditio distendon
rami in diverse mani¢re; ma in-vece d' alberi sono scheletri.
tomie. Scheletro, o scheretro diciamo tutta l'offatura dtumco
ogni altro animaie,ripulita dalle carni, e rimessainfieme con
105; e4notomia chiamiamo il corpo d'un' huomo, ed” altro'ani I
mostra tutti li nerui'y wulcoliy e vene, chefoho foro Jaipelley: soe
SPALLIERE, Quelle piante, ed alberhy che si fannovdutendere fu perte,
ra con i rami, come limoni, e fufini, ec, si dicono spallieré se qui:pig
mie per ogni specie di pomi d' agrumi, dice, che in vece di tali pomi
questi alberi a spalliera gli aborti., i miofiti €s gobli.< ai Se
INSELICIAT E. Seliciavo dal Latino filices diciamo up lafiricd fatto
ma strettamente, intendiamo quei lastrichi fatti-di plete: piccotidinies, o- tan
glion fare ne i viali de i giardini a foggia di Mofaico con pietreyperd maggiori di ji
quelle del mo/aico,e minort atial di quelle degli acciortolatiy e sono diary colo- |

ri in maniera che (ene formano figure,¢c. Come col Mofaico., Binovece di gi
fic pictruzze,dice che (on fatte d' ugna, drdendi, e d' aitreoflature minute.)
a MORC HLA, Intendiamo la fondata dell'oli0 dai Latuno-amarca, o gets dat
fr, aan. 4 2
CABBROSTOLITE, Abbronzate. e<bbrofolire propriamente viohdire qu
abbruciamento che si fa agli uccelli pelati, agcio fiabbrucino quei pr 'the
non si son potuti levare con le mani; ma qui vuol diretince dal fuoco ¢on un
leggieri abbronzamento ¥ che diciamo: abbractacchiate -3 egiggdted
MV MMIE. Sono cadaveri d' huomini che-hanno la-carne appiceata W
full' ofia seccatavi sopra da balfami, bitumi,ied aromati, come son cOlpl,
che si trovano sotto le Piramndi di Bgicco, 1 quali sono di persone peincipaliy che
gli Egizzj havevano per costume di riewpucre di baifami, ed aromaci, fate
gli con strette strisce di tela', odi drappo com murabilitfima maestria, e pt
hi insieme con qualche Idoletto fatto di metaijo dentro a una cafla, che sate
se

VAN

'

uy

Ni

iy

i

:

faccia d' huomo; così gli riponevano sotto quelie piramidi, dove non \&
cevano; ma si feccava la carne, e si riduceva tanto queila, che l'offo come

trito; per lo che si ono-conferuati quet corp: fino a1 tempi nottei, ed f

ne trovano, Polid. Verg. de Rer, imuen, lib. 3. 6, 10. riferisce-con te seguentipe |

3» tole il modo di questo fotrerrare i cadaveri degh Egizzj:Agypuj Hatin mor hg

>» tuo homine ferro incurto cerebrum per-nares educebant, jocam iiusmedt

we is expl; deinde lapide A chiopico circa-ilia i 'a

>» bant,atque illac omnem alucum protrahebaat » \& ubi repurgaverant, rorfam lig

»» Odoribus contafis:refarciebaat., ide iterum Contucbine. Vbi hae fecitlent,fa- toy

>» liebant nitro adulto feptuaginta dics, nam diutius (alire non ticebav; quib

y»» exactis Cadaver findone inuoluebant gummi iilinentes; Ho deinde.

\&

fy

A

w

I

9» Pitiqui ligneam hominis efigiem faciebant, in qua inferebane,

>» lumque ita reponebant; Eeid, ut arbitror,ica facticabant ju eo acto” >

»» ta cadavera diucurnius incorrupta servarent. ae
Altri cadaveri secchi ci vengono pure dagli Egizzj,i quali corpi |

teriori, e-tutto, secco, e come impietrito; e ono iciza farciature; equell a

corpi d' huomini, che dal vento sono staui fotterrati vivinella rena, € quivi com



tidal' tar della rena, Di

chi, ma p:

SESTO CANTARE:
fertiatifi forse per causa de' venti meridionali, e però il nostro Poeta dice: Fenn-
queste Mummie si servono i Medici per diversi farma-
a particolarmente per la Triaca, La'voce Mummiaé Araba; e il Voffio

tira da Atam, che'in Arabesco vuol dire, cera (de vitijs Sermonis lib, 2. cap.
la cera e '! miele faculta conferuatrice; e della cera si servivano gli

per mantenere i cadaveri secondo Brodoto, Jib, 1. Ma la pece mescolata
bitume, era forse quella materia, per quel che apparisce,con la quale
gli Egizzj condivano tali corpi, la quale in Latino greco dicono Pi

289

magi
?

DI, Intende quelle bafi, sopr' alle quali son posate le statue.

R51, Intende torsi d' huomini, che pittorescamente parlando vuol dire il
senza testa, e braccia, e cosce Latino truncus; e questi dice, che sono
ilareiti; cioè raccomodati, rappezzati, riftaurati con havervi mefie in vece

- delle lor tefte già consumate dal tempo, altre tefte nuove, e fresche di banditi;

uol e meta » che alle volte si veggono al Palazzo della Giustizia, eo.
sopr'alle forche elposte alla vista del popolo, essendo fate tagliate di poco tem.

po ai maifar
/ STANZA Lil

Inter '8 quadri di cipolle
Snip i fior Youyatee nariche;
Soma teiccioni, i signoli,e te bolle,

Le posleme, ta rigna, e le volatiche.

Vil mal Pricefe entrante alle midolle,
CW feminaro dalle male pratithe,

'Teticberi, le rabbie, e gli altri mali,
- | Che vi mandano gli Offi,e è Vettnrali,

malfartori bandit, « per frelshe

STANZA LIV.

Pescheinsu gli occhi fonui arzurre,egialle,

I marchi, che fiorir debbon le spalle

Ai tagliaborfe, e ladri ancor scolari;

Le piaghe a masse, 4 pererecci a bulle,

wend ventose,¢ gonghe in più filars, o
 e il fior di rofolia, € più rofoni

D' ortefica, vainolo,e pedignoni.

Cu re for per chi gli porta pari,

o ita a descrivere i) giardino dell' Inferno', ed in queste duc ottave narra
i - quelche contengono gli spartimenti. è
<QVADRI di cipolfe. Intende quelli spartimenti, che si fanno in terra ne i giar-
Me\gquali si pongono le cipolle de' fiori. Latino areole, puluini,
» PRA foglie, e-natiche. Dice così per mostrare, che questi mali vengono nella
carne ef mente, e pigliando natiche per tutta la pelle dell' huomo, dice che
fra quelle foplie nascono questi mali in fu le natiche, intendendo la pelle, e per-
ché anche la maggior parte de' medesimi mali per lo più viene in fu le natiche,
'come laogo pib carnoso.;
CHE vi mandano gli Offi, e i Vetturali, Questa sorta di gente ha per costume
itnprecar sempre male, come venga la rabbia, il canchero, la pefte, o firnili,
» PESCHE in fu gli occhi, Qucei-lividi, che vengono attorno agli occhi, quando
sono stati percossi da pugna, o da altri, e sono di colore azzurriccio, € intorno
: 7  Dar le pesche: i Latini dicono /uggillare alquem, vedi sopra C, 3,
'1., che noi purt diciaino anche figilli tali lividi,¢ diciamo anche: figillare un'

uno, *
 GLI sfregi fior per chi gli porta pari, Gli sfregi son fiori, che anno bene in sul
v ) di coloro che portan as tani, cioè fanno bene il raffiano, che portar i pollé
woo) dir fareil rufhano dalla voce pouler Francese che vuol dir, viglierro amoroso,
diciamo porta poulers. Qo CUAR.

SSS HSS ETE SaaS er



290 MALMANTILE

MARCH, Tntende quei segni, che dalla giuftizia si fanno
droncelli, gai per esser giovanettinon sono capaci della pena
Higmata, Vedi opra C. 2. st, 3. alla voce sberlefe. 6

PLAGHE a masse, peterecci a balle, Piaghe, e peterecci ins rand
ma, Nell' uso diciamo anche Patereccio; e Panareccio dal Greco, usato ai
da' Latini Paronychia, postema che si forma alla radice dell' ugna, che i
chiamano Redivias., o Reduvias yp ea

GONGHE. Intendiamo gavine infermita che viene nel collo, e quei t t
che son talvolta /pine ventose, perché diciamo haver le gonghe os ee e

venga apparentemente nella pelle della gola sotto le ganalce, Latino tonfilla,glan
dule faucium, t VE
Ma perché non paia che io voglia fare un trattato di chirurgia, i
splicazione di questi mali; tanto più che io stimo, che faranno intesi per
Italia, nella quale son chiamati nell' iftessa, pore differente mamiera,
intelligenza dell' opera serve sapere, che in questo giardino sono tutte
ta, che vengono agli huomini esteriormente, le quali il Poeta vuol mostrares,
che si generano nell' Inferno, come sentina di tutti i mali. 10.04
 LV. STANZALVL,.

Alla ragnaia al fin si son condatti
Di fils da toccar la margheritey »
Ove de' tordi cala, ede' merlottt
Alla ritrosa quantità infinitay

ai Lae

ae

Si maraviglia, si fupisce, epanta
Martinazza in veder si vaght fiori,
Erimirando hor questa,hor quella pianta
Non fol pasce la vista in quei colori,

Ata confortar si sente tutta quanta
Alla fragranza di sh grats odori,
E ai non corne non pxo far di meno

Che son poi da Biagin pelati,¢ corti
Sgozzando de' più frolli ana partita,
Altrane/quartaye quellach'e puafrefea

Be ssh ee eh

Vu bel maxxetto, che le adorni il seno,

Nello Stidione infilza alla Turchesca,
NZA LVI.

= BEG ES.

Veduto il tutto, Nepo la conduce

Chi per la pizzicata, che produce
A! bagno, on ogni schinvoy e galeotto )

| dl uazo,fa tragedie sn sul.capporta
Opra qualeofa: Un fa le calze,un cuce, Un mangia,un fofia nella verrinala,
etltri vende acquavite,altré il biscotto, Vin trema in sentir dir:fuor camiciuels,
Martinazza resta maravigliata, e si fupisce, e rimirando tutte quelle piante,
paice la villa, e soddisfa all' odorato con quella fuave fragranza, ne può non fa-
re un mazzo di quei fiori galanti per adornarsene il seno. Visto il giardino, Ne
po la conduce alla ragnaia, dipoi al Bagno, dove stanno i galeorti » descritto ¢-
me è appunto quello di Livorno cirea operazioni, che fanno i galeotti, |
SP ANT ARS!, Dallo Spagnuolo e/pantar/e. Vuol dire efttemamente mart
vigliarsi, e si dice in augumento maravicliarsi, frabilirft, spantarsi, che è il verbo ui
§paventarsi fincopato. Habbiamo l'addicttivo/panro, che significa eftremament —
marayigliolo. Ma forsc e da Spandere, quali voglia dire largo a ie
de, ampio, e in confeguenza maraviglioso. E di Spamse addicttivo, del, Ay
Spandere cen' ¢l'esempio in Messer Cino, Quando ha per gli occhi (un poem \&;
fad 7 wap Sah
4 VN bel marzetto, che le adorni il seno, Bello ornamento del send d' una few.
* pa havervi croste, rogna, e simili galanictic, delle quali potcya esser fy
~

UGA

Serer,



SESTO CANTARE)

Poeta scherza per esprimere la laidezza di Martinazza.
BE' una felua, o macchia folta posta per lo più lungo i rivi, per
si pe eee sospefa a due Mili, e questa rete si chiama ragna
'a imitazione di quei veli, che fanno i ragni per pigliare le mosche,
ragne. Pietro Angelo da Barga nel suo Poema della caccia de-
celli: Hos caffes, has ipfa plagas, bac retia quondam Ante alias omnes telamt
ere dotta Innenit dixitque suo de nomine Arachne, E da questa rete ragna si
nna ea » Ove Gi tende per pigliar tordi, beccafichi, ec.
'teccar la margherita, Cio quelle Aanghe,sopr' alle quali si da il mar-
lla Corda, che questo vuol dir toccar /a margherita.
DI, merlotti, Vuol dir merli giovani, ma dicendosi merlotto, o Tordo
stintende Huomo semplice, corrivo, che cala; che si lascia pigliare.

3

st. 59.
g Gabola fatta a foggia d' una trappola da topi, con la quale per
certo Ordigno si pigliano vivi gli uccelli, detta così per esser la parte, da
eferrare rivolta in dictro. Vedi sopra in questo C. st, 1. alla voce con-
Qui per ritrosa intende Carcere.
; » Maestro Biagio, o biagino vuol dire il Boia, che così havea no-
me, quando l'Autore compose le presenti Qrtave; ed a questo fucceffle Maestro
Baltianodetto sopra C. s. tt. 44.;
0 + Poco gli manca a essere stantio; s' intende animale morto di più
giorni, Vedi sopra C.3. stan. 24. la voce stantio,
INELLARE alla Turchesca, Cioeimpalare,
BAGNO. Così chiamiamo quel ferraglio, entro al quale si tengono gli schia:
vi, €coloro', che per delirti son condennati alla galera, detti pero Galeotti, i
quali dimorando quivi,fanno i meftieri enunciati dal Poeta, che si serve della vo-
 e bagne per l'equivoco,il quale fa credere, che in questo giardino sia ancora il
g0 da bagnarsi per mostrarlo ripieno d'ogni delizia;come il Paretaio, e la ra-
-  E.questo ferraglio di galcotti credo, che si dica bago, perché in esso quei
iquenti purgano i loro misfacti, come con l'acqua del bagno si purgano le»
'delle membra. Gagno si disse ancora un luogo simile, Li Pulci nel Mor-
pante: Dife Aforganre allora: ia son nel gagno De' diavoli,
» PIZZICAT A, Specie di confezione minutissima, ma per la similitudine della
figura di essa confezione, e per il senso del verbo pizzicareintendiamo ( come
qui §' intende ) pidocchi.; pe
FA ie in sul cappotto, Ammazza pidocchi in sul cappotto, che e quella,
fo tche portano gli schiavi, o galcotti, remiganu, ed ogni altro mari-
; detto, siccome Cappa, 4 capiendo, perché piglia, e cuopre tutta la vita,
- SOSPIAR nella verriola. Cio bere, perché bevendo si fofha, o: respira col na-
s0 nella vetriola  cioè nel vetro. Detto che ha del parlar furbesco.. Vetrivea er-
s herba parietaria detta daalcuni. 11 Monofini lib. 9. Indicare.yo-
Aen muito vino se ingurgitafle, dicimus. Egis ha toccato ben la verrigla,
Vesrivola est herba insestoribus notissima, de qua Petrus Cre/centius lib, 6. ¢, ult, pocula
itrea vulgo fiune, q
. do f Auzzino vuol bastonare un galeotto per qualche
Oo 2z suo

od
see ce



292

suo mancamento suol dire fuor camicinola, intendendo, che

ha da esser bastonato; e però dice: Chi trema in (emir dir

trema per il timore delle bastonate. i
CAMICIVOLA. Bun piccolo farletto di panno lino, b2

che secondo la Ragione si sotto gli altri abitisopra alla Camicia

dersi dal freddo, come 10 detto sopra alla voce Farfetto: gli

chiamano gix/ecca, vod awieirsele

STANZA LVIIL TAN:

Vanno più innanzi a'gridi,ed a'romori
Che fanno i rei legati alla catena
Ove a ciascun secondo i snoi error
Datoe il gafligo, e la dovura pena,
esi primi che for due Proceuratori
Cavar si vede tl sangue d' ogni vena,
E questo lor avvien, perché ambidui
Furon mignatte delle borfe altrui, Con

STANZA LX, Oy

Quei, dice Nepo,t il Re degli xfurai, 1 gran se gli marcy dentro a'

Che pel guadagno scortico il pidocchio, Che nol vendea se non vaiea

Un fernizio ad alcun non fece mai, Così fece det ged hor
Se non col pegno, e dandoli lo scrocchio GP intarla il dosso,e da'fiohfe fh
Passano avanti a vedere i delinquenti legati alla catena, e gait er lo

falli. I primi sono due Causidici, ed il secondo è un' Viuraio, ti

secondo il merito. » Seah

PROCCVRATORS, Agitatori di liti. Cautidici tanto Civilijche criminali

MIGNATTE. Sanguifughe. Quei vermi acquaticijde i quali si servono 1 Ce»

rufici per cavar sangue; e perché si dice, che i danari sono il secondo fangut

però esser mignatta delle borfe alerni vuol dir Succhiare, roe ¢avar il de i

altrui borfe, come fa la mignatta fucchiando, e cavando il sangue dalle vent»,

diciamo mignarta, o mignella a uno, che € stretto del suo, e volenti sig:
quello d' altri: A questi tali può quadyare ciò, che disse Orazio. Lon milfura ch
tem nif plena cruoris birudo, a:

V-AGLIARSI, Intendi dimenarsi come fa uno, che habbia rogna, o altro pe

la vita, che si dimena, e scontorce per grattarsi il prudore; o pizzicore conl'a

bito, che ha in dosso, e fa con la vita un moto simile a.quello, che fa und, che

vagli il grano. sae
TONCHI, Forse dal Latino sondere pre(o per mictere Ȣ divorare, Sond vet

mi piccoli, o infetti, che si generano nelle fave, pilelli, ed in altri i

 votano i granelli rodendoli; da i Latini detti Curcudiones. Virg, 1 z

pulatque ingentem farris acervum Curculio, es visage

TIGNVOLE. Bachi simili ma si generano ne i pani, e fogi impaftari; dai
latini detti Tee, Di queste ne nascono ancora dal grano, e si chiamano prt

noli. peas ah
™ MOSCIONT. Quei moscherini, che nascono dal vino, che dicemmo sopra in
questo C, stan. 37. oe

Son

Se Fern Ee eae SE Sr FEKETE



SESTO CANTARE: 293
So eae eet si generano nel legno, e lo rodono; da i Jatiai

er) eet.
| RARE « Intende quei farfallini, che si generano nel grano. Pyrau/fa,cou
4 areca sono app: = farfalle più grandi, le quali papain oe al
lume, e vis'abbruciano. Di queste disse il Petrarca. Semplicetta farfalla al lume

 “COCCIOLE. Piccoli tumoretti, o enfiature cagionate da' morsi danimal
come zanzare, bruchi,¢ simili.:

'S8RANF, Rotture; Scorticature. Vedi sopra in questo C. stan. 47.
PER ristoro. Per ricompenfa. Dan, Par. C. 5.,

Gabino Dunque che render puoffi per ristore?

Equife ben pare, che il nostro Poeta voglia dire »per ristoramento, o alleggeri-

ia de i-teavagli,¢ pene, nondimeno è tutto il contrario, perché @ parlare -

.
P| ' e vuol dire; oltre a gli altri travagli ha di più, che lo flagellano,¢ pelta-
foe dese pieno di feudi d' oro'. Questa voce. rifore vien dal verbo ri~

ie derivante dal verbo reffawrare, ed ha quasi, lo: stesso. significato,
non che questo vuol dire Acconciare, o raffettar case, ed altri materiali; ¢

i dir Ricompentare, o rifar danni,
ia Lo,4 Sy Nae ppi aunacordicella; i dendosi per
Py mbello quel facchetto-pieno di segatura,0 di cenci, che adoprano i ragazzi
'Perquotere i contadini,come dicemmo sopra C, 1. tt. 59. Zimbello detto, cred'
10, quasi cennelio, civè piccol segno, argumentandolo dallo Spagnuolo, che il
chia five '

o Ub Re degli nfurai. Wmaggiore usuraio del mondo. Detto che viene da i Gre-
| Gi yiquali chiamavano Re,quello che avanzava,superava, e vinceva gli altri ne i

en: 9

i) 29h giuochi fanciulieschi; ed Asino quel che perdevay come habbiamo detto altro-
Vebs iy):

Yi, SCORT-1CO? il pidocchio. Significa esser avido del denaro, e far' ogni maggior

a per guadagnare; si dice scorticar if piducchio, per vender la pelle, € con

¢ Planto 6 pod dire, Vel unguinm prafegmana colligere.: 6

, DAR be screcchio, Prestar danari a usura »ed in vece di dar. denari effettivi,dar
aoe vaglia dieci oe venti. Vedi sopra C, 3. st, 74. ¢d è la più efecranda

'2, che si trovi, e forse la più praticata.
; MARCIRE, Intendiamo infradiciare, corrompersi, Dal Latino marceres;
|

SE non valeva un' occhio, Se non si vendeva caro,¢ a prezzo rigorofidimo: Non
vit cosa più cara dell occhio. Onde Catullo. Ni te pins oculis meis amarem
INT ARLARE, Esser mangiato da i tarli, o tignuole, che i Latini dicevano:
Cariem sentire,

E PESTO dai fui soldi, Inscanto dalle percofle di oa facchetto pieno delle
ae monete. Vuol mostrar in somma il nostro Poeta, che
= Per qua quss peceat, per bec torquetur.

STAN.



294 MALMANTILE?@

STANZA LXL oe STANZA
Va! altro ad un balcon balla, e coruetta, Dice la maga questoe
Ch un diavol con tasferzaacentocorde uand ella i
Chrun grad'occhio dibue ctascihainverta, | Cofkui ha fates quale
Prima gli da certe picchiate forde, Par non fo nulla,e no
Con una spinta a baffo poi to getta Domandaa
dn cert' acque bitumose, elorde, ° Tal penaa chifi debbagda
Chee' n' esce poi sch' ione disgradogli orci, Ed et che per servirla è 9)
O peggiv d'un Norcin mula wines ' Prontamente così le da risposa,
STAN ZA LKIL + SD
Quei fu Zerbino, ed! amorose dardo Ma dell' occhiate sue ben più '
Mostrando,il cuor ferito,€ manomesse, Hor sentene il riverbero', ¢:
Credeva il mio fantocciocon un sguarde E com'.ci già pensd far alle da
Di (briciolar tuttoil femmineo feffo; Dalla finestra è tratto in

Quel che segue e uno che peced d' ambizione di bello'y¢ lindo, e credeva'
la sua bellezza di far' innamorar tutte le dame, ed hora'riceve la a
suo peccato, p 94> being
CORVETT A, Salta. Cornettare \& un certo faltellar de*eavalli 5dal Laci eure
uari, Spagnuolo corwar; piegare, innarcare, torcere» EB questo'
appropriato in questo juogo per esprimer i) moto, che faceva costui, il,
evitare le sferzate,era necessario che falcella(se a tempoy ed in quella'
to, che fa il cavallo, quando coruetta.: >> SRD
VN grand? cchio di bue ciascuna ha in vette, Pone in vetta, cioè nella cima di
queste corde, tocchio del'bue, e non d' altro animale; perché bovis ocnle oculorum
pulchritudo, \& nitor fiemscatur, e trovatene l'esempio in Omero, dal quale
Giunone è chiamata boopis, cioè bovinos oculos habens o vero Dea dagli occhi grandi,
¢ perciò maeflosa. E costui doveva esser gaftigato con la bellezza degli occhi;
perché con la pretesa bellezza de' suoi occhi, haveva egli 10 AO
PICCHIATE forde. Picchiate, e percofle gagliarde, Percofle » che facciano
molto male, e non paia che lo facciano; servendoci in questo caso la voce ford
per la voce occw/to,.come si dice ricco sordo, per ricco non palefe, o non' cond:
sciuto. Ie
LE disgrado, Quel che vaglia questo termine vedi sopra C. 3. stan, 37. al ter
mine ho froppato. AMY
ORC/0. Che cosa sieno orcj. Vedi sopra'C. 1. st. 7. Qui intende orci da olia,
che son sempre schifi. a
NORCINO mula de' porci, Coloro che in Firenze ammazzano i porci, € così
morti gli portano sopr' alle spalle alle botteghe de' Macellari,sono per lo più del
paefe di Norcia, e pero gli chiama mule Norcine, cioè portacors da Wercia e O-
storo son sempre tutti unti di gratio di porco, lorditimi, e schifi di sangue.
QVEST Ac ariosa. Questa € cosa grande, ardua, e-che arreca:stupore; otra
ordinaria, e stravagante,¢ che non si può credere, me
NON ono far giudizio. Cioè giudizio temerario,e falfo; Maniera da Ipocriti,
¢ faifi bacchettoni scrupolosi, chelp
ZERSINI, Così chiamiamo quei giovani, che persuadendosi d' esser belli, faa
Z no

=

Ce et ee ee ee ij

aL
= aie oi

Sef asl 2 \&



- STO CANTARE 295

 vanno lindi credendosi di far innamorare ognuno con la lor

quel dohenae l'Ariosto nel Furioso deferive per il più belio, ¢

¢ di quel tempo. E si dice anche Mirtillo;nome cavato'dal Gua-
fins Vedi sotto C, 10, stan, 30,

10 il. cuor ferito, e manomesso aeasmerslo gacdds Facendo da inna-

2. Nibbiaceo, Vecellaccio, ec. tutti servono per intendere un
cimunito.

aR re in minutifiimi pezzi, o-ridurre in. bricioli,ed in-

morir di 'patimo, e disfarsi per amor di lui-tutte le dame.
0. +) Sinonimi che fignificano:li riperquotimenti, che fan-
i del Sole, o il fuoco nella parte oppostaa quella,dove direttamente
i Chimuci dicono; Fuoco di riverbero., o di riflefo.. Qui inten-
'costui con quelle fruftate piene d' occhi, ha il gaftigo dell' occhiate amo-

egli nel mondo dava alle donne.
Reta is ake dame. Cioè si come egli pensd che le dame cascasse-
la sua bellezza, ( il che appresso di noi vuol dir farle morice
re), così egli e buttato da qusi balconi entro al litame, per maggior
0 efti. tali sono schizzinosi ne poslono. vedersi addosso un,
che guafti la loro attillatura,¢lindura,,

] STANZA LXV.

; ANZA LXIV.
an ch' e legato, e che gli e posto Qui Nepo scuopre la di lui magagna,
berrettin baffo a tagliere, Mostrando ch' ei fu nobile, e ben nato,
-colpe colpo da discosto E sempr'hebbe il Pedante alle caleagna;
fra. gliene facadere. Cc ontuttocio voll' esser mal creato;
Misero sia quivi immoto, e tosto Perché se e fulfe statoil Re di Spagna,
do gli occhi ai colpi dell'arciere, Mi cappello a nellun mai s' è cavato;
muave Punto, ochinaorizza, Pero s ei fu villano, hora il maestro
€.42 cultello che è infizra, GU! insegna le creanze col baleftro.
STANZA LXVI.
4 par comune usanza, Se ¢' faltan la granata,addio Creanza,
4.risponde al Galatrons; ae ch? e° fien nati nella Falterona,

noi Fanciusi un pocon offer uanz A, Ata per la loro afinita Superba,
eyes: il. ahi bastona. Son poi fuggiti più che la mal erba;

“« Dattro. che segue è uno, che nel Mondo non volle mai imparare.i buoni costu-
Bi. non si volle mai cavar il cappello di testa per riveric nefluno,per grande
j 'ch ¢gli fusse, onde gli avviene il gafligo, che si dice nelle presenti ottave; E
Masoasaa dice a Nepo, che hoggi di questa sorta mal creati e pieno il Mondo.

pies (TINO a tagliere. Berretta bafla e piatta,nella quale non si vede la

del capo, come sono /e coppole Napoletave,
eat Qgni volta ch' ci tira. Vedi sopra C. 2. stan. 57.
« Sta duro; Sta faldo; Sta fermo; Non si muove.

(RCIERE. Colui che tira con la balestra « -4rciere in molti luoghi del nostro
do s' intende il Caprone, o Becco. Lat, aries.
(AG AGN.A. Mancamento, difetto. E parlandosi d' huomini s' inane tan-
'animo, che di corpo, Dante Jaf. C. 33. dice. OGe-



age 'MALMANTILE”
O Genovefi huomini diversi 08 OND
“D egni costume, e pien d! ogni magagnò
Lalli En. Trau, ston erga i j up reeves J
¥ i bE pias va

Ogni trattate conte ogni magagna |

Magagna in Lat, barb, \& detta Mahaminm, Swan) Franz. Aabain,

¢ vuol dire propriamente mutilazione di membra,¢ si stende a significare « i

no,¢ detrimento. Vedi Du Frefne nel Gloffario alla Parola Aabamiam,
BEN nato, Nato di nobili, ed honefti parenti,

HEBBE sempre it: Pedante alle calcagna, Hebe sempre il Maahre a te 4

gl' insegnava i buoni costumi, e termini, cet Sar
MAL creat, Senza creanza, Vao-che non fai buoni termini o costumi,
VILLANO. Contadino', Stintende uno scortele, e mal creato. Planco ra
merum, intende un' huomo ruftico, senza civilta, fenga galanteria, un pre
villano, Catullo, Peeniruris, \& inficeviarum, [1 contrario di vidlane®,,
SE faltan la granata, Se essi cscono di (orto la cura del padre;'¢ del macho,
Si dice faltar la granata; quand? uno esce de' pupilli che ini diero's ema
re ex Spbebis, Dicono che quando uno e arruolato per birro,debba far
mee a fare il noviziato,¢ fnito questo tempo gli faceian fare una cirion
faltare topr'a una granata, che gli mettono d'avanti in terra,e che fatta questa azione
refti libero dal noviziato, ed.in un certo modo esca de' pupilli; e da que r
monia (che se non è vera, e assai vulgata ) credo 10, che habbia origine il pre
sente detto, F 'hae?
PAIONO nati nella Falterona, Paiono nati in luoghi incolti,e difabigati,come
sono le montagne della Falterona in Casentino, dove poche creanze im-
pararsi, non essendo in quei luoghi con chi praticare, se non con pecore,€ por
ci, Ci ferniamo però di questo termine per esprimere un' huomo incivile, e foz"
zo, eche tratti da villano; come e quercubus, aut faxis natus, ae
SON fuggiti più che a malerba. Nefluno gli vuol praticare. Sono sfuggiti 42
tutti. Malerba intendiamo l'ortica erba nora, la quale \& sfuggita da tuttl » per

ché pugne.
STANZA LXVIIL STANZA LXVIIL

Ma chi è quel, o hai denti di cignale Ora per queste sue finzioni eterne,
E lingua così lunga, e moffruosa ? Chi egli bebbe sempre nella mercathrs,
Si vede, che son fuor del naturale Lucciole dands a creder per lanternt,
A me paion radici, o simil cosa. Sharbata gli han la lingnaye denratirs
Wepo rispose; Quello e un Sensale Main bocca havids pos di gran cavertty
Che si i act il Parola, ma la glossa Perché non datur vacuum jn naturhy
Huom di fandonie, dice, e di bugie, Glibanno a mifferio in quelle fhanze we
Perché in esse fondo le fenferie. Composto denti, e lingua di carate «

Segue un Sen(ale, il quale e gaftigato delle bugie, che anna cavato
la lingua, e identi, ed in quella vece meflovi delle carore. Ji Poeta si serve dell!
affioma Peripatetico! Won datur vacuum in natura, col quale ingende che fusse ae
cefiario riempier quei voti,cagionati dal' eftrazione della lingua, € deati » Ae
scherza, sapendo bene auch' egli, che quei medesimi voti erano già ripienl
aria. >) ” ae;

S82 rennrer® BRfs2-nPhee

~

See



'fon mediatori afar vender una mercanzia:
vin Firenzeun'Sensale di bestie, huomo sccl-
oy che per le sue farberie:fu impiccato a: forche erette a possa per
0 ee eealns 3 ed è lo stesso, che quegli che fu,
chino detto | 3. fhe 55.
OIE ye bagi s aiclonaoe dal vero, e sono si può dir finonimi, se
Ml dir chiacchi vana, e bagia propri vuol dire attesta-

fenferia. S' intende,quando uno di questi Senfali fa vender qualcosa,e

per lanterne, Dar a creder una cosa per un'altra, 1 Lalli

| Lucciole qui rimiro per lanterne,
£4, Bi quel vermicello alato, che di notte riluce da i latini detto Ci-
Nottiluca; da'Tedeschi animaletto di'S. Giovanni, e da' Greci Lampyris dal
e fai egiare nelle tenebre, come egli fa;\¢ /anterna \& quello arne-
; 'quale-si porta il lume la-notte ferrato da talco, offo., o vetro per di-

jo dal vento; ed è voce.pura latina,

4°) Specie di radica, Latino /iser, Mail proverbio Pisntar, o fecar
ca dare a creder bugie. Latino imponere alicui, Onde Impostura, e
febene si dice in più grave significato. Vedi sopra C. 2, st. 70. Dice
yperché vi son messe tali carote,è non solamente per riempiere i
perdar il gaftigo a'costui delle tante carote, che esso haveva piantate,
era in'vita', facendogli haver sempre dentro alla bocca effettive, e natu-

a. 'ANZA LXIX, STANZA LXX,
See volta ha la facia, Vedi colui',¢ al colle ha un' orinale;
“Bute diavol legnainolo in sul groppone Cieco, rattratto lacero', e piagato ?
— Gli ascia itlegname,fega,ed ipialliaccia, Ei fu Governator a uno spedale,
(Ste servir per suo pancone; Ow ei non volle mai pur un malato,
a fu; c'alla pancaccia Ora per pena ogni dolore', e male,
aglian le legne addossv alle persone, Che gl infermi y' haurebbono portato
tener sa lingua in briglia ( Mentr' alia barba lor pappo st bene )
ender. lapariglia, Sopr' al uo corpo tutto Guanto trene,
'il gaftigo dato-a* Mormoratori, ed a quelli, che, essendo Aati Sopranten-
| Spedali,non hanno havuto carita; ma solo hanno attefo a crapulare per
Manodion 3 che dovevan fomminiftrare a' poveri, ed infermi.
edi

VE; Codrione. Le parti di dietro del'huomo fra le reni, € le nati-
fowo C, 10; st. 50. LU Persiani disse,
“\ Ceascun teme, e si caca nelle brache
Tn vederus appiccato sul groppane
t ® “Lo frocco da scannar le pastinache, -
si cava che e usato, ma per lo più in scherzo,- Viene secondo il Ferrari dal

Orrhopyginm, che significa lo stesso. H
ARE, Tagliar con l'asce, che è uno strumento da legnaiuoli noto,chia-
Pp mandolo o

. =.
a ooo



Tar ae

298 MALMANTILE o

mandolo così anche i Latini, che lo dicono ¢4/cia. Ifidoro neilé Origint lib, 19)
6.19. Ascia ab affulis dita quas-a ligno eximit yenius diminutionm nomen est asciole
( forse accetta ) Est autem manubrio brevijex aduerfa parte referens vel, i
Jexm, vel canatum, vel bicorne rafirum, Vitruvio difie Asciare Lib. VI. ¢. 2. Suma.
tur Ascia, o quemadmodum materia ( Qui invende il ego; che gli Spagnuoli dal
Latino chiamano, madera ) dolatur, fic calx lacn macerata ascierur, Am,
[MPlALLaccLa, Qui la rima forse ha necefitato  Autore a servith di
questo verbo impiatiacciare in vece del verbo piallare, che vuol nee
-gnami con:la pialla come intende qui, ed il verbo impiallacchare vuol dire tito
prire un legname con piallacci ( fefftles lamina, famine pratenues \e disse Plinio
fond fottilifene afircelle di noce, con le quali si cuopre altro legname più vilei
far cafle, tavole, ed altro, nella forma che si fa con ' ebano, granatiglia, ed
tri legnami nobili. Plinio discorrendo di legnami, de' quali gli antichi si servi-
vano per impiallacciare lib. 17. 43. Que i laminas fecantur, quorumque Z
weftiatur alia materies, pracipua funt cedrus, terebinthus, etc, E poco 2
prima origo lnxuria, arborem alia imtegi, o viliores ligno pretiofiures cortice fier; B
PO, Lvcugitate funt, \& ligni brattea, nec fatis, Capere tingi animalinm cornua.
dentes fecari, liguumgue ebore dsspingui, mox operiri =
LALLA, Chiamano i Legnaiuoli quello strumento di legno', che ha un ferro
incaflato, col quale aflottigliano, appt » pulil 9 ed addiri: ile
gnami, da i Latini,secondo molti,detto Dolabra, ma forse con qualche
'Vn' antico Grammatico pat che la confonda coll' alcia.. Dolare fabri' Co
ascia ledere, Si legge in Colum, lib. 3. Qua falce amputari non possunt, dette dole
bra abradito,il che pare che voglia dire più tosto accetta, o pennsto, ovanga;
che pialla: E corrobora questa opinione il medesimo Colum. lib. 4. ¢-24, serven
dofene in diminutivo; Semper circa crus dolabella dimovenila 'ef revra, clot Inter.
no al cansbo della vite e da levare (a terra con una dccettina, 1) Calepino tiene sche
da pialia si dica runcina, e porta ' autorita di Plinio lib. 16. cap. 42. edd éncitare
runcinarum raptus, ove pare, che descriva appunto l'operazione della yf
per infino l arricciolinamento de' trucioli: Tutto il testo dice così: Br ad quack
que libeat intoftina opera aptissima ( parla dell' abeto ) five Graco, five Campanty,
ficulo fabricae artis genere spettabilis, ramentorum crinibus pampinate ae
be se voluens ad incitates runcinarum raptus, Ma io ardisco conteaddirgli ot
Y autorita d' Hermolao che dice: Runcine [unt maiores ferra, quibus fabri materis-
rij fecant arborum moles fubiettis canterijs, Sì che non la pialla, ma fa fega grande,
'che adoperano i Marangoni per ricidere i legnami, adattandoli sopra quel C4
Valletti, che noi chiamiamo canreo ( dal Latino cantherins, cioè cabalus: -e più
volgarmente pietiche, i quali sono one di due correnti inchiavardati ial
a guila di cefoie (che propriamente si dicono pietiche ) e d'un' aleco pezzo'di cor
rente, che si merte a traverso alle pictiche (e questo si dice Canteo) € a
così un triangolo vi adattano per via di piuoli il legno da fegarsi, Runcare è tet
'mine d' agricoltura, 'che vuol dir propriamente tor via, onde se ne formd per
yeotura la parola antica Latina averruncare, cioè avertere; e se ne Iddio
wAverruncus detto Così, perché ab'eo precari folent, ut pericula avertat; si come dice
Watrronc., E in proposico d' agricultura (ene fabbricarono le parole aie.

fF EP Re ewe

a
wt



SESTO CANTARE: 299

Ronconé yle quali significano strumenti da nettare.i campijda rimondare fructi, ¢
i cinne raat Plinio lib. 18, ¢./21. Siligioem gfe striven. sfemen, shepelea

accato  farrita 11 iB « Runcatio, cum feres in articulo off, evulfis inu-
. joo eoremng apes radicem indicat, Segetemque discernit a cespite. EB Catone cap.
even cremarique; ie che più tosto Runcina parrebbe, che avel-
 fead essere la roncola, o cosa simile, che la fega, o la pialla. Ma forse non.
tanto. il Calepino, quanto anche il Vocabolario della Crusca dal levar via, ¢
a faellere,¢ ripulire ( che questo significa, come s'è vitto il verbo Runcare ) hanno
dato il nome di rancina alla pialla),perché clla pulisce, appiana, e leva il fover-
da! legnami.. Tuttavia anche per questa ragione la dires do/abra, perché si-
questa ancora pulilce, e rade, come dice Colum, nel juogo sopra cita-
sia. come esser si voglia,poco fa sal rth nostram, bastandoci intendere, che

[ene quello strumento da legnaiuoli; che habbiamo accenaato,
 PANCONE. Chiamano i degnainoli quella loro panca grotia, sopra la quale
ilegnami per lavorargli, deta pancone, perché ¢€ fatta d' un paycone
aoe un' afie grossa-circa un. quarto di braccio » che sono asse da rifen-

ALLA pancaccia Così si chiama quel luogo dove in Firenze si tiene il croc.
discorre de' fatti d'altri, e delle nuove. Vedi sopra C. 2. st. 73, E per-
ildir male del prossimo si dice Tagliar le legne addosso.a uno. Latino famam.,
icnius lacerare prosciadere, pero.a costoro vien dato il gaftigo adeguato, con
Halab loro addosso il legname essertivamente.
TENER (a lingua in briglia, Parlar consideratamente, e con riguardo, e si di-
ce anche: Tener la lingua a freuo.
ghtNDeR la pariglia. Render il contraccambio. Pariglia vuol dire una cosa,
può dividersi in due parti yguali; come nel numero due si può far' uno,¢ uno.
£ 'to render pariglia vuol dir render ugual contraccambio.. Vedi sopra C, 4. st,
ma pwr pari referre de' Lat. Dan. nel Parad. C. 26. dice:
x Perch' io lo veggio nef verace /pegtio,
yh « ~ Che fa di se pareglie L altre cose
E nulla fece tui di se pareglio.
Hoggi però in questo senso,¢ maniera, che si serve Dante di questa voce pare-
44 non mi pare, che si usi, se non da' Franzefi 3 che dicono pareil,

ALLA barba loro. A spele loro, Questo termine esprime Pigliare, o consuma-
re una cosa d' altri contro al gusto » € volontà del padrone di essa; o a dispetto,e
208 del medesimo.

AP PO. Cioè mangid, Donde Pappolone uno che mangia assai che vedem

FER SSLA SES SCES CLESSSE RE

Hi

laedinee 1 6.

4 eee NZ A LEXL

5 'Chia, coltui, ¢' habbiamo a. dirimpetto Che non ne pag mai un maladetta 5

è (Dice la donna) a cui guegli animali Tenne gran posto, se spele bestiali;
 Sharban con le tanaglieilcuor del petto? Ma pai per soddssfare ei non hauria

fw risponde:.Questo e un di ques tali, Voluto men trovargli per la via,

a a

; Pp z STAN-

|
|


300 MALMANTILE o
STANZA LKXILe o ooo) oo SDANZA\LERMD
Colni, e hail viso peffo, eit otto... Riferva il. muragche c' ni
'Da quei due spire in femimils spoglie ».. Donne, che se a eH
Hus vile fu, ma bifeaiuoloxe ghiogtoy' >  D* arir giviellare, e luce

Che si volle cavar tutte le voglies\ Dar ile... al mario in i
Ocni sera tornava a casa cote, she Hor le superbe pierre; t
E dava col baston cena alla moglie; Alla lor liberta fanno il

Hor finti quella Peffa quei demoni, o Pero che tanto erandi, e tant!

Sopra di lui fan trionfar baton...» ».. | Chan fatto per lor carcere'
Termina la mostra delle:pene date'a idelinquenti con tre forte
il primo è dato a coloro, che non vollero mai pagare i loro debiti. Iba,
dato a i crapuloni strapazzatoridella mogtiex ll terza \& quello dato alle do
ambiziose,e vane. ag janguu vik > > sitll
TANAGLIE, Strumento di ferro fatto a foggia divcefoia, e serve V
chiodi da i legni,, ec. Da i latini detto forcipes, ig Mie ta hea'
NON ne pag un matadetto. Non volle mai pagareun debito, Non pagd mii
un quattrino di debito. L' epiteto ma/adetto ha la forza d'un becco d'un
no decto sopra C, ¢. f..68. ) vast tye
TENNE gran posto, Si trattd alla grande), e'fece spele bepiali', cioè grant
inconsiderate. Lat, immanes.: 23 elit Held eg?
NON hauerebbe volute crovargli per la via, Quand'anche egli' havesse trovato per
la strada il denaro, del quale era debitore;non havrebbe ad ogni modo' pagato il
suo debito. Questo termine ci serve per esprimere', che nefluna cosa hat
potuto muoverlo dal suo proposito, e fargli venir'voglia di pagate. ~~
PESTO, Infeanto, ed ammaccato dalle-battonate, che gli danno quel De-
moni finti la sua moglie. E questo vuol dire trionfar 'bastoni..) |» t
AYO M vile, Qui vuol dire huomo di bala condizione. *
RISC AWOLO. Huomo che pratica le bische «| Bische diciama quei raddotti
pubblici, dove si giuoca a carte, e a dadi; nome forfe'venuto dal verbo bi/ear-
sare, che vuol dir Mandar male sproposicatamente il fao havere: e corriff
al Latino prodigere. L' usò Dante nell' Inferno C, 10,
Biscarca ye fonde te sue faculeade, ag A
GHIOTTO. Huomo,a cuipiace mangiar del buono'. Vedi sopra C. e
DAV-A col baston cena alla moglie » In vece di portar da-cena alla moglie; la be
stonava, Costume assai usato dalla gente' d' infima plebe, imbriacarsi all'osterie,
¢ non pensar' a mandare da cena a casa alla moglie, e così-briachi'tornare @/¢a-
fa, e perché la povera moglie si duole d' eer digidna', bastonarla
DAR del c,,. in fal laferone, Quand? un mercante fallisce; diciamo + vitalelt
dato ile...sul laftrone. Beanetto Latini nel/Patafio-disse Dar det Ad wy
Questo proverbio è nato'da un'proverbio antico, che era 'in Firenze; chee *t
i quali fallivano, o rifiutavano-l' eredita del padre', “andavano 'nel mezzo:
Mereatunuovo ( luogo dove si ragunano i mercanti pers iare)e\gniei era,
¢d € ancora una gran laftra di marmo tonda, che si chiama il carrotcia( pere evi
\& poita per segno,dove si fermava il carroccio, sopra il quale s' inalberava Pinte
gua generale de' Piorentini, quando andavano alla guerra ) e sopra —
a:

Shy

Bese e see we OE pm ERE ESSE SS ee EE

OPS SSEREG EES a oe



jor

. awifta del popolo, che nell' hora, che si doveva fare tal
10; \& questo atto assicurava la loro persona dalle mo-
'di debito s:ne potevano i creditori moleltare se non la roba, la
deva ceduta tutta a favore de i Creditori, non essendo per questo
(0 il debitore a pagaie ultra vires, essendo questo come un cedo bonis del
'dus. Così questra laftra alle persone de' falliti, che a quellarifug-
era come una Ara, o vogliam dire altare, o luogo facro, o afilo, o 1
già, che dall' esser prefi gli assicurava, e questo, perché cflendo dedicata
pubblico di sostenere il solenne:carro, e la tanto famosa infegna della
endeva per questo riguardo franchi, ed immuai coloro, che col se-
 prendevanne folennemente, e con cirimonia il posscsso. Di qui dar
laftrone vuol dir-fallire. Edi qui pure,quand'uno casca, e batte il c\ellipsis{18pt}
lle Jaftre diciamo < // tale ha rifiutato il padre, Fallire ancora dichiamo
i pole: Ef rale U' ha infilate; che corrisponde al Latino decoxit,
TTON/, Sono il latino dateres detto sopra C,1. st. 67. £ fare, o dare il
» Vuol dir fare a uno qualche danno grave; e qui vuol dire; sono il lor '

,¢ pena.

TANZA LXXIV. STANZA LXXV.

in orecchiche mi par che e suont Dice la Maga; Vo venir anch' io,

po tabellaccio del Senato, Perch'il veder pin altro non m' importa,

| Sichee' mi fa meftier chrior abbandoni, Ed in questa Cittd così a bacio

\ LiPeri cht io non-voglio esser' appuntaro; A diria mi par @ esser mezza morta;

i v ch reStavano i lion, Vaglia trattar col Ke d'un fatto mio,

Ma non posso venir chro son chiamato, Ed andarmene poi per la più corta

Ed ecco appunto Diavoli co' lucchi; Ed ei le dice in burla; Se ru parti,
Peri lascia ch'io corra,e m'imbacucchi, Vaviain un'ora,etorna poi intrequarti,

. li suddetti gaftighi dati a i delinquenti, Nepo sentenda la Campana

del Senato si licenzia dalla Strega, ma dovendo esser' anch' ella nel Senato per

Bat Re, dice volerlo seguir sin quivi, di dove spedita se ne yuo] andare per

''*

STAR i orecchie. Ascoltare con attenzione. e4uribus arrettis aufeulrare,

o TABELL-AC C/O. Così è chiamata da molti la Campana del palazzo del Po-

dest ( hoggi del Bargello, la quale \& detta la Maddalena,come vedemmo sopra

ein questo C, stan. 23. ) forse dal latino T-abeltiones, che vuol dir Notai, i quali di-

Moravano, e tenevano i lor banchi dentro, ed attorno al detto palazzo,ragunan-

~d0vifi al suono di detta campana, la quale hoggi e detta anche /a furba, perché

lori d'alcune feftc, non suona, se non per esecuzioni criminali di tefte, e forche,

 €/a nowe per mostrar l'hora, che non si può più portare armi; o pure è così

-detta dal suono oscuro,¢ malinconico, o che almanco rappresenta cosa mefta,
'come il suono delle tabelle ne' giorni Santi.

4 NON celia essere appuntato, Coloro che'fon de! Consiglio del Dugento, e d'al-

“tri Magiftrati di Firenze se non vanno al detto Consiglio,quando si raguna a suo-

: erence condanaati in certa somma di danaro; e questo diciamo

2YCCO, Br la sopravvelta,o mantello Curiale di Firenze, ed era anticamente
nay

V abito 4

:

302 MALMANTILE: >

l'abito civile ordinario; e perch questo haveva già un
metteva in doflo detto lucco, si doveva dire imbacuccarsi. Va
141, Subito fu preso; e imbacuccata col cappuecio, fu condotto alle carci
C. 11, st.22. a Pou
e4 B.AC/0, Campagna, dove batte poco il Sole, che diciamo Al rezi
uggia. Vedi sopraC, 3. st. 71. alla voce Vria, e sotto C, 9. st. 44, e c, 10,
1 contadini in vece di dire: luogo o piaggia volta a mezzo giorno, dicono:
Jatio, ed in vece di dire: volta a tramontana, o a settentrione dicono: ab
o a paggino che e i) contrario di folatio, Credo venga dal Latino
si come natio da natixus. Da molti si dice meriggio quel luogo,dove
no i raggi del Sole per interposizione di che che sia,¢ ( pare a prima'
troppo lodevolmente, perché meriggio da meridies vuol dit mezzo giorno,.
do appunto i raggi del Sole sono più: quocentije però andare al meriggio p
be che volefle dir più tosto andare a scaldarsi a' raggi del Sole di mezzo
che andar all' ombra per difendersi da i raggi del Sole. Per corroborazion
questo idiotifmo, si uova in Autore approvato per buon Scrittor Toscano
vollero fare il viaggio di notte per lo gran freado, ma si bene in full' ora met
allora cheil Sole con i suoi raggs havesse addolcito i rigori hiemati.Maquestitalifid
dono conl'uso, e potrebbe dirsi anche colla ragione, perché meriggio nel si
to di luogo ombroso, e difeso dal Sole,è lo fiefio, che juogo da passare IL ore
del mezzo di, la quale cosa i Latini dicevano meridiari, Catullo. dube adte
aeridiathm, Ora dal meriggiare, cioè fare all' ombra nell' ore calde è
meriggio, ¢, da meriggio, rezzo.. Va in un' ora, e torna pai in tre g.
€ uno scherzo usato aliai fra gente bassa,ed intende Va hora in uno,cioè
¢ torna poi divi(o in tre quarti; fij impiccato; se ben pare che voglia dire: Va»
in un guarto d' ora, ritorna in tre quarti. Cirimonia da Diavoli,
STANZA LKXVI, STANZA LXXVIL —
Tun vuoi gli rispos' ella,sempre il chiafo; Ed ella per oferta così magna hah
Wel Con/figlio così ne va con esso Ringraziamenti fattigli abarellay.
Ove ciascun l'honora, e dalle il passo, Dice,c'hor mai sbrartar vuol la capagnly
Sbirciandola un po meglio,e più da preffa, E tornar a dar nuove a Bertinella, —

Ella baciando tl manto a Satanaffo Pluton le dd ticenza,el parse
Fino alla porta ye [i se ne (gabellay

Lopregad' ofsernar quanto ha promesso,

Ei.ghe! conferma, e perché stia sicura, Ond' ellain Dite aun Vetturin saccopty
Per la Palude Stige glielo giura. Che la rimeni a casa per la posta *
La Maga così scherzando, e burlando con Nepo se ne va con esso in Confi-

glio, dove ognuao l'honora.. Fa riverenza a Piutone, e lo prega a ma

quanto le ha promesso; Eigliclo giura foleanem:a:2, ¢1 accompagnatala fino
aiia porta del Consiglio la liceazia, ed ella va a cercar d' un Veccuriag, che la

riconduca per la polta a Casa,, 8
Tu vuci si chiasso, Tu yuoi la burla. Tu scherzi. Chiaio nel proprio e 3

strezta, vicolo Lat, vicus quali erano le strade di Ro oa aatica, edel pri

cerchio in Firenze. Gio, Vill, 10, 29. S apprefe fuoco ix Firense in Borgo S.

Appostolo nel Clafso tra' Bonciani ye gli Acciainoli,B pecché in queste straducole abl-

tavano taluoita donne di mal aftare, Chuatio e detto forse da Vicus Vicario, Ha-

sa

a2EH@ Ze O- Se LED



FS Sue

SESS SEAS a”

==

ESET ER EC SEE 8 Se

a

an SESTO CANTARE je

ga; in buon Latino Vicinia ) venne a significare Posribolo, e perché in tali difo-
nefti Inoghi si fa gran baccano,e ff scherza, e si burla senza rispetto; perciò
' iglia per burla,,per ischerzo. Se bene e molto verisimile, che in questo

hielo
ultimo significato di strepico, e di baccano, quale fanno quelli, che licenzioia-
-menze tratcano,¢ burlano, venga dal Latino de' tempi baifi; che il suono di

» o degli organi, e degli altri strumenti domandavano C/aficum,

tutte le campane
che i buoni Litini dicevano della rromba, a cui son succedute le campane. Lt

lo dice Glas,

| SBIRCLAN'DOLA, Guardandola bene. Vedi sopra C, 1. stan. 9.

9.
 GLIELO ginra per la Pande Stige. Giuramento folenne, ed inuiolabile degli
Dei secondo la falsa credenza de i Gentili, come si cava da Omero in più luoghi
del Lliade, e da Verg Zn. lib. 6.:

: Stygiamque paludem,

ea
>t: Dif cuins inrare timent, \& fallere numen.
« la ragione, per la quale questo sia giuramento folenne,secondo Servio,¢ questa

4» Styx'meerorem significat, Dij autem lati (uat femper; ergo qui meerorem non
y» featiunt rant per tristitiam, que res eft sue macure contearia; ideo Lufiu-
per execrationem habent. L' altra ragione ¢, perché havendo Vitto.

iuola di Scige aiutati gli Dei nella guerra contro ai Gigaati Ticani, Gio-

ve per rimuneracla, volle che coloro, che giuravano per Suge di ici madreo,
fussero privi del nettare delli Dei, e non osservavano il giuramento. \& queste
“sole furono finte, e credute di Stige, perché (econdo Teofrasto questo Suge era
un fonte in Arcadia, le cui acque, e pesci erano velenosi per la di ini eltrema
frigidita; e di questa acqua dice Plin, lib. 30. cap. 16. che Aatipatro voleife da-
re ad Alctiandro Magno, quando volle avvelenarlo per consiglio d' Arifotile.
yy Vogulas tantim mularum repertas, neque ullam aliam materiam, que non,
a» percoderewur a veneno Stygis aque, cum id dandum Alexandro Magno Anti-
a» pater mitteret, memoria dignum eft, magna Aristotelis infamia excogiratum.
A barella. ln quantità geande, Si dice a balle a masse, a facca, ec. sono
pero modi bafli, e più tosto scherzosi, e s' usano parlando tanto di cose corporee,

quanto incorporce,

\ SBRATTAR la campagna, Andarlene: Sbrattare propriamente significa net-
tare 50 ipulire, contrario d' /mbrattare; sì che sbrattare él paefe vuol dire ripu-
dice il 9 o per confeguenza andarsene da quel luogo.

SENE (gabella. La lascia; Sisbriga; si libera, e filicenzia da lei. Dedotto
dalla Gabella, che si paga, perché, come è pagato il dazio, o gabellad' una.
Mereanzia, si dice sgabellata, e così si spedisce, e manda via.

BITE «. Qui la Città di Plutone, detta così da divitie, le  ci vengono tut-
tedi sotto terra. I Latini chiamarono Due, quel che con Greco yocabolo dice-
vano altrimenti P/zsone, che vuol dire il medesimo, € significa il ricco Lddio,
Addio detie ricchezze, come s' è veduto sopra.

WET TVRLNO, Coini che presta cavallia nolo, o a vettura.

STAN.

'
4
|


304 MALMANTILE: —
3 STANZA LXXVIILA
I Re fatta con tei la dipartenza | Salirò alla,
Al falon del Consiglio se ne torna,
Onde ciascuno alla eo
Alza il Civite, e abbassa gli le corna,
Plutone licenziata la Maga se ne torna in
sua refidenza si prepara a discorrere.
FATTE le dipartenze. Licenziatifi scambievolmente.
ALZ A il Civile, Alza le natiche. Civile € una prospetti
fentancte abitazione di Città; contraria a quella, che-si dice of
pagna. I Latini simil ue entrate principaliin;/
di quelli che venivano dalla piazza,o dal mercato; l'altra di coloro., che si
geva che venifiero di lontant pacfi, o di fuori dalla Città; La prima ente:
diceva a foro, l'altra 4 peregre, siccome riferisce Vitruvio. Noi per questo —
miamo Foro la parte in Paccia della scena, Lin eRe
RAGNI, Quci veli che fanno i ragni.. Narrano le favole degli antichi
li, che in Lidia fa una femmina detta Arachne nata in contado di bafla
quale fu così valorosa nel ricamare, ed in ogni sorta d' artifizio di tela ¢!
che non solo superava tutte l'altre femmine, ma hebbe ardire di co;
la Dea Pallade; onde Pallade superata, e vinta da lei, per dispetto le;
Javoro,¢ la converti in Aragno verme, che è quell' infetto che si
veli per pigliar le mosche da noi chiamato, ragno, o raguarelo. Ovid,
tam, Dante nel Purg. C. 12. rocca questa favola, -
O folle-Aragne, si vedeva io te
Già mezz? aragna triste in [u gli straceè 2
Dell' opera, che mal per te si se. bie è
DR APPELLONI, Così chiamiamo quei pezzi di drappo i quali ee
no pendenti al cielo de i baldacchini, e delle refidenze de i Prinejpi;,
rano le Chiese, ec. Varchi St. Fio. lib. 14. Ed al vano dela Cupola era tirate in fu
Le funi wr belissimo ortangolo di drappelloni. Matt, Villani lib. 9,.cap. 43 deferiven~
do le nobili efequie fatte nella fepoltura dei Cavaliere Messer Biordo degli 4
uni. E sopra nha xn drappo a oro con drappelloni pendenti coll' arme del peice.
comune,e di parte Quelfa',¢ degli Vbertiné. Tali drappelloni coll' arme si
appiccati in gran numero nella insigne Chiesa Collegiata di S. Lorenzo un
giorno dell' anno, per memoria di antichi benefattori, ee
SPVT-A un ciabattine. Quando uno per soprabbondanza di catarto ha difficulta
in spurgarsi, fogliamo dire: gli ha un ciabattino giis per la gola ye doy
Sputa un ciabattino, — a ee oe ¥ - cme se nel Lal
oni, Coll' occhiaia lnvida toffire, e spurar far, te 2 r
Spee Fan STANZA LXXIX. noaniaie mM
Spiegar volendo poi quanto gli occorre, Onde nui siam quaggit in. fondo di
Comincia il suo proemio intal maniera; Gente, a cui si fa notte avanti fer
Voi che di sopra al Sole in queste forre Voich' in malizia,in ogni frode,e
CadcSts meco all' arta oscara,¢ nera, Siateè Adacftri di color che a)

ug

Ce eR eek ai kB



+

SESTO CANTARE, 305
STANZA LXXXIL
Cominci il primo: Dite, Mdalebranche,
r e  Quel che e'vi par che qui v'adafe fatto,
'bazzicar taverne, e chialfi Levato i Tocco, e sollevace t anche

5 agnun di voi st bravo,edotto,
ea vo.
ib pincrife

1 Alor quel Diavol n' un medefmotratto
a

Un capitombol fa sopr' alle panche,

I a un famiglio a' Otto; Efalta ae nel mero com! un gatto
aunque, benche pare Cittadini Ma perch'il Lucco s'appicco a nn chiodo,
el vieupero ingeg ns peregrini ' Si ric e, e parla a questa modo:
TANZA EX xr, STANZA LXXXIII,
tusti'in correfia O Re, cus splende in mano il gran forcone
Da Martinazxa nofira confidente, Sil Cappello speriale ha quel segreto,
 Poithe Baldone ancor cerca ogm via Col qual si fa scornare un pedignone,

Dh entrar in Malmansiscon tantagente, fo ? ho da far tornar un' buomo a dreto.

ar-ch?eglisbandi, e trucchi via So gidche qualche debito ha Baldone
| Adbope Mctitentensy z che e lo vuol pagare in (ul tappeto, '
pe ere [opra questo il suo parere Perciò manda Pedino (a in campagna,
. 'the o! ci fusse da tencre', Ch' ei giuocherd di posta di Calcagna,
: io de i Diavoli fr: composo dall'Autore; dopo che egli ottenne
meat » nell' esercitare il quale conobbe l'autorita, che si usurpano i Can-
: +s anon hoe be » metce per Cancelliere di questo Consiglio un Ciappellet-
celliers che fun notaio scellerato, secondo che riferisce il Boccaccio nelle sue Novel-
leye bcontraddica a tutto quello', che vien proposto. I nomi di questi
colt pis son cavati da Dante nel suo Inferno; e sappia il Lettore, che li spro-
he dicono,son poco lontani da quelli, che l'Autore setiva dire nel areating
rid iperfonaggi che finge in questi Diavoli sono simili alli suoi Colleghi,
ed egli medesimo in leggermi questo Canto mi diceva; il tal Diavolo è simile al
tal mio Coliega ye il tale, al tale; e mi parvero appropriati benissimo; non sti-
mo già bene nominargli. Ma tornando a proposito dico, che Plutone volendo

fens re de' suoi Senatori, fatta una breve orazione nella quale inferiice
un ver | Petrarca Gente, 4 cui st fa notte avanti sera, ed uno da Dante Siese i
Mathri di color che fanno, ordina a Malebranche il dire, quel che egli farebbe
per mandar via Kaldone da Malmantile,ed egli,fatte prime sue diaboliche cirimo-
nie dice che il suo pensiero farebbe di farlo citare alla Mercanzia da qualche
suo creditore, salea:

FORRA, Valle lunga, e stretta posta fra poggi alti, onde poco dominata dal

3o però ben detto forra il pacfe infernale dove non batte mai fole.

 GENTE 4 cui si fa notte avanti sera, Con questo yerfo del Petrarca, ? Autore

 intende che costoro son sempre di notte, cioc al buio.

, BABLV ASSO, Huomo senza giudizio, icimunito. L' origine sua \& scura;
forse da Valwaffor parola feudale, dalla quale e fatto anche Barbafforo, lo Netio
che » o-dortoraccio; faccente; e che si da scioccamente ad intendere di
sapere o pure da Bwaccio peggiorativo di bue. Vedi sopraC. 5. stan. 1. Ul Bini

in lode del Malfrancese dice.
Qq Eri-

Mi =p.



TPE

305 MALMANTILE | 3
Erispondendo a certi Habbuaffi, scea 2c bia
Che voglion dir, che questa malattia i eS ntn se
Tatto il corpo ci florpi, eck fracali. Ah eat RRR

Ed il Molza in lode de' fichi: My
Hor fa tut argumento, habbuaffo. ono AE
TONDO pii che ? O ds Giotto. Huomo tondo vuol dire huomo grosso d inge-
gno, ed ignorante, come s'è accennato sopra C. 5. stan. 1, sì che più rondo dell''O
: Grotto vuol dire ignoranuthmo, e più, perché lO, che fece Giotto Pitore fu
tonditfimo, secondo che riferisce Giorgio Vatari nella vita di esso Giotto,
BALZICARE, Praticare; Converlare: Bocc. Giorn. 9. Nov. 5. £ vatrene nel
la cosa dela paglia, ch' e sh miglior nego che ci sia, perciocché non vi bazzica mai per
“ote. 1

ona.

CHTASS, Bordelli, lupanari, luoghi, e contrade, nellequali habitano le
meretrici, come era in Firenze il Chiaffo de' Buoi, e il luogo, dove ora € il Ghet-
to, detto anticamente Chiafo \& perché in tali moghi usa di fare fracatio, e rumore
disonesto; di qui forse e che chia/so, e bordedio si prende ancora per tumult die
fordinato, insolente, e lascivo..: swash

'PIV' cattiuo di tre afi, Affo si dice il numero-uno-de i dadi, che e i)
numero, e per confeguenza nel più è il peggiore che vi sia tirando tre dadi,)
questo il presente termine significa cattivissimo:, che vale per aftutiffiime, ed è lo
stesso che Più trsffo a' un famuglio a' Orto, che pur vuol dire fagacissimo,eche fail
conto suo, Famigio a' Oro. B' uno de' Birri del Magistrato degli Orto di Bali
di Firenze, che e il Magistrato Criminale; e perché si sappone che costoro fap-
piano tutte le furberi¢e, però si dice: Il tale e pis triffo a' un famiglio d' Onto, per
esprimere; e huomo fagacissimo. 1 Greci dissero Cantharo. afturior, che qui
Cantharo fu un'oste d'Atene astutissimo. 4/*m in antico Latino voleva dire,
sotto, fens accompagnatura; onde chi cantava  senza strumento che L/accompa-
gnaile, si diceva costui: canere affa voce, Di qui può esser venuta la voce Afoe
Kespare in affo, cioè esser lasciato solo, se bene altri gli aflegnano altra origine: 0
pure da «fino che così chiamavano ne' dadi /' nita i Greci, dicendola Ones. I
nostro Proverbio: O a/s0, o/ei i Greci dicevano, o diciotto, orre. O sre sei, ore
afi. 'Giulio Polluce lib, 9, al cap, di giuochi fanciulleschi, e de' trattenimenti de

1i antichi. A
PAZZO Cittadino, Questo epiteto si suol dare 4 colore che fanno sutte le tor elt
4 casa, e senza considerarione; ed € lo stesso che dire ux cernellaccio,

SBAND-RE, Disfare le bande, cioè licenziare i Soldati.

TKYCCHI via, Se ne vada. E' modo baflo, cavato forse dalla parola Ze
ruck Tedesca profferita da i Lanzi, quando con le loro alabarde fanno allonta-
nare il popolo; O forse dal giuoco del Trucco, che si dice truccare, o trncciart
la palla, quando cogliendola con un' altra palla si manda via dal luogo', dove

era; dal frequentativo Latino tra/fare usato da Catullo. ' ai

TOCCO. Con il primo o largo; Specie di berrettone, che anticamente ulava
in Firenze in yece di cappello. Varch. Stor, lib, 11. Cow le calze foppannate: a le
jerra bianca,¢ le berrette, o vero tocchi di colore rosso.:

SOLLIFATE ? anche. Alzati i fianchi, cioè rizzatosi da (edere, —

jicia

ek=s \& re:

BEERS Ee Seek Sec s-

Se

Le fF Pe coor Fae

=
=

Sar erree



ore cubito,

Dan. Inf. canto 34,
8

7 la quale

zioni civil

STANZA LXXXIV.

Pluton diede con tutti una rifata,

\fiantar fino il brachiere,

RB discegi: Va via bestia mcancara
Com' entra celafsedia il dare,e havere?
Segualalero che vien dela pancata,

Rizzato Barbariccia da sedere
Sichina,ementre abbassa gli la chioma

Alea le pe,e mostra sf bel di Roma,

STANZA LXXXV.

Poi  intirizzaye dice in rauco suono:

- Se non si leva dalle fquadre tl capo,
Quale e Baldone,e non si da nel buono,
Mai si verrd di tal negorio a capo,

| Dove, se manca lui quanti vi sono,
Restari come molche senza capo,

- A poco a poco, a truppe, e alla sfilata
Partendoyn breve disfaran o areata,

——— a

Qqz

SESTO CANTARE; 307

parte del corpo, che è fra il fianco ye la coscia, da Ancon greco

ire gomito;¢ si piglia per ogni (ora di piegatura, come lo mostra il

Città d? Ancona così detta dal gomito, che faquivi la spiaggia; Pli-

pio libs 3. caps 13. La iifders colonia Ancona apposica promontorio Cumero in ipfo
q se if

«o Quando noi fursmo la dove la coscia
Gait sis Ss volge appunto sul grosso dell' anche.
Edi qui sciancato'è un zoppo, che habbia mancamento in tal luogo. Vedi
sotto C. 11. stan. 40. B il Latino Coxendices.
ty PITOMBOLO B' quando uno, posando il capo in terra, volta sopr' a
quello tutta la vita, Vedi sotto C, 7. st. 20.
| ORB, cui plende in mano sl gran forcone. Fingono che Nettunno Re del mare
atello di Plutone usi in vece di scettro una forca con, tre punte, e però detta
in realta è una fiocina da pescatori, Latino fu/cina, e Plutone
tun Bidente, cioè forca con due pente; Equesto \& il gran forcone.
er Speziale.B uno Speziale in Fireaze, che fa per insegna un cappelio.
aiubeaowe - Enfiagione che viene ne i piedi',.¢ nelle mani per causa del
r Latino Pernio. Vedi 2 C, zt. 6.
- LOamal pagare in sul t: « La vuol pagar per via di Corte, con tutte le fo-
temipebe, non vuol oa Saede non feglt mandano j birti a gravarlo, o cattu-
en dice che Baldone gimecherd di calcagna, cioè fuggira per la paura
teller preso per debito, quando vedrà Pedino, che così si chiamava uno già bir-
ro della Mercanzia » che èil Magistrato, per yia de] quale si mandano l'efecu-

¥.
re + Subito. Latino e vestigio. Traslato dal giuoco di Tan » che si dice
dar di posa — si da alla palla prima, che tocchi terra. Vedi
e Ly

sotto C. 7,st.92.
TANZA LXXXVL
Circa il pigliarlo, ionoal' ho, eglit fallo:
Facciam conto ch'in braco alla pastura
Vin toro sia costui o xn cavalo;
Tiriamgl addosso qualche accappiatura
Legata innanzia un bel mazzacavalla
Collocato in castel presso alle mura,
Ond' ei si levi un tratto all'aria, e pai
Si tiri dentro,e dove piace a noi,
STANZA LXXXVIL
Buono, rispose il Re,non mi dispiace;
Ma il Cancellier di subito riprefe:
Sia detto,o Senator,con voffra pace,
Tant! oltre il poter nostro non 8 esse/e,
Li tutto (aria nulo, ef foggiace
Ad esser condennato nelle pefe,
Ed io farei stimato anc' un Marforio;
econfentir a Kn! atto perentoria,
STAN-



we
308

ae NZA ieee
Perché sempre de ire i

y slrapacs 4 ete 5m rAgion®y
Pei Sella è in morayvienfi a un! inibitay beg upalerele
E non giovando, alla comminazione
Ch' in pena caschi delle forche a vita,
E se la parte innova lefione,

Aller puo condennarsi, havende ofate
Di far causa pendente un' attentato >
Plutone, ridendo con gli altri della coveaaaeen

secondo, che viene nea pancata,nominata Barbariccia, cheidica i)
e questo propone che si tiri un laccio a Baldone € per vid d'un
s'alzi, e si porti dove più piacera; ma ciò non ea C
de Piutone ordina al ter2o nominato Calcabrinal 3 dica il suo parere
fiui si rizza,, e fa riverenza al Re per far il discorso, meer

ti Ottave.
' SCHLANT ARE, Denes » spezzare detto da. Splenare, BB
lo, che fidisse sopra C, 3, st. 5. A

BESTIA incancara, Così diciamo per esprimere n*huomo feo 9
traslato da quelle beflie, che alle volte conducono. con loro.i Mont
quali essi fanno far molti giuochi, e dicono che tali: bestie-sieno:
operino per vie diaboliche. Si dice be/tia smcantataa und di poca confi
ed avvedimento, come il Lalli En, Trau. C. 20. 56.
Così gridammo, e con.la propria appa
Ci dessimo in sul pie bestie incantare
COAL entra con l'assedio, Significa come s' accorda, o che i che re
l'afledio,
IL bel di Roma, Così diciamo per intender apertamente c\ellipsis{18pt} 5
Roma intende il Colosseo, da noi corrottamente detto Culifeo,
SINTIRIZZA, Si vizza,si distende in fu la — 'EB' un' atto,¢l:
ta una certa superbia, e prefunzione di se stesso, ed \& quella prefopopea' py ches:
dicemmo sopra C, 1, st. 72. 5 a eae
NON si verra a capo dé tal negorio, ec. Non si conchiudera», 0: terminera if nt
gozio. ne woagiher
REST ATI come mosche fenxa capo, Cioè senza oe direzione '0g
Senza sapere che cosa havere a fare., o risoluere: i infect fo
capo, s' aggirano inutilmente, strascicando il sane di bey
dove.
ALLA sfilata, Senza ordine; confulamente, e senza andare in ila,
nanza: Sbandati ' so29
S' 10 non V ha, ezli éfalle. Yo son sicuro di pigliarlo. Seionom lo-p
per errore, E' specie di giuramento vantatorio, come: appease
sotto C..8. stan. 72. \& mio danne che vedremo C, 10, stan. 49.
ACC APPLATVRA, Una fune accomodata,, € —- cay
do, che —, ibqual nodo firdice-cappio scorfoio.,

page e st FER w Gees = FL TFAF EE

=>



- SESTO'CANTARE: 309

MALZAC AVALLO', B un corrente, o pertica grossa congegnata per tra-
——-yerfo, come: acavallo 'un legno ritto; la quale's' alza-da-una parte
'con tirare a la parte | » E questo ordingo e usato assai ne i piani di

Firenze per cavar I" dai i. [ Latini lo dissero rolenonem a toliendo,
dete Smiles quella awbiid, della quale si servivano i nostri antichi a
Acagliar pictre:chiamata Azangano.. Livio dice: ariere Tollenonibus Inbramenta
i iy aur fa » sp vobuffos incuriebant, sta hina milirare
fien descritta da Vegezio così; Tollenc dicitur, quoties una trabs in terram praatee
a ry cui in fummo verrice alia transuers4 trabs longior, dimensa medietate, 6on~
neititur, €o —— 5 at si unum capue-deprefjeris, alina erigatur, L' antico Vol-
“garizzamento lralenoè detto, quando una trave alta si ficca in terra, alia quale nel
J una altra trave più lunga per lo traverso, enel meyxo misurata, si com-
«mete in tal modo che se! uno capo si china, l altro in aito si leva. Da questa voce
-alvaleno ( Lat. toileno)si dice ? e4italena giuoco, che i ragazzi fanno con due travi
* incrociate, e bilicate l'una sopr' all' altra a foggia di Mazzacavallo. Vedi sopra
Ga, stan. 48, Mattio Franzefi contro alle sberrettate dice.
6 Biggetnslo- Ma chi trovalfe il modo a bilicalle,
ee
o

Ma» 2 Sarebbe un [chifanoia, e faria bene
vail, Van contrappefo d! un mazzacavallo,
SIA detto con vostra pace. Perdonatemi; s'io v' offendo in dirlo, Non vi adi-
wvioffendete, io lo dico. Frafe de' Latini Pace rua hoc dicam, Nell'
igen di Quinto Catulo, Pace mibi liceat, Caleftes, dicere veffra. Adortali
sue pulerior se Deo, Che Annibal Caro nel primo Sonctto delle sue Rime vol-
CO olfimi j e *icontra a lei mi parue oscuro, Santi Nums del Ciel, con vofira pace
—— LO vieme', che dianzi era si bello.
(2 —--—« BSSER condennari nelle spese'. Cioè buttar via sa fatica, e il denaro, oleum, \&
«Opera perdere. Ma propriamente ¢ffer condannato nelle /pefe vuol dire, quando
=“ UNO!Per aver litigaco una cosa ingiulta, e dal giudice condannato a rifar cute
le spese all avveriario; e però questo Cancellicre dice, che noa vuole acconfenti-
8a tale'atto-per'essere ingiusto, e da efer condannato nelle spese.
Ss imato'un Adarforio, Sarei stimato un' huomo senza sentimento,o giu-
dizio; come @ la'starva di Marforio in Roma,
' ATTO fruffratorio  Awo vano, fatto senza proposito, E questo termine,
come tutti gli altri-delle (eguenti stanze 88. e 89. son termini Curiali jche veaca-
do dal latino', ed essendo praticati in cucti li Tribunali d' Italia non-dubito, che

,
,  farannointefi da'ognuno; però ne tralalcio la spiegazione.
, ~ STANZA LXXXX. STANZA LXXXX1L
E poi ha fatte riverenze in chiocca Aa in vece di quel cappio da beltresca,
C0 fuivi più Lindi-a pianta di pattona, Ch'è il:toffice de ladri, si prouuegga
Si soffia it naso ye [parzafi la bocca, Pua bilancia, o rete per La pesca;
- Epostain equilibrio la persona Con una lnnga fune, che la regea 5
o Come quel che si pensa dar' in.brocca E perch' sl fatto meglio ci riesca
Tutto sfromato dice: Alta Corona, Si ringa tutta, accio che non si veega 5
Circa Pordingo, pur si merra in opra; Einverra quanto ell' apre, ivi fispanda,
o Perch*ioconcorrose affermo quanto sopra, Fino-ch' +l porco vengane alla ghianda,
Bate tit: STAN-

OO EE L

310

STANZALXXXXL...,
Perché 8 e muovyon l'armi,di ragione
(Se dal capo l esercita e condotto )
Annan a tutti marcerd Baldone,
E quand' ei giunga,ed ha la rete sotto,
Fate che lefie allor fien più persone
A farla tirar fu con l'avannotto,
Operando in maniera,ch' egli infacchi
tn lnogo, ove si vede il fole a scacchi, Lodando il fa
S.T.AN, ZA \ LXSXXIVi, copie
'Qui, dice il Re, si da fempreinbudella, Gli ha sempre più ritorte cl
'Siche mi cascan le braccia, ef ovaiay Mace' non locredes'einonvaal
Mentre cofini a ogni cosa appeila,; i
E co' /uoi punti mena il can per l' aia;
li terzo Diavolo, che e Calcabrina, dopo haver fatta rive
mano di smorfie, come fanno certi Oratori affettati, dice, che app
cavallo, ma che in vece del cappio scorfoio piglierebbe una rete da
il Cancelliere s' oppone; onde Plurone sgridando il medesimo Canc
al quarto Diavoilo, che e Cappelluccio, che dica il suo parere. at
IN chiocca, In quantità grande, in abbondanza, un diluvio di rive!
PATTONA. Specie di pane fatto di farina di castagne, che per ¢
più di figura lunga, s' aflomiglia a un piede mal fatto di un' huomo,
da, Prolusione Plautina prima dice: Qui enim pedibus fant planis ploti:
che piede di parcona si può dir plotus dalla voce Latina Plautus, che fig
fo; e questa dal Greco Plarys lato, largo; donde noi a tali huomini,
i piedi malfatti-diciamo Pileri. Vedi sopra C. 4. st. 17. li Franzese dice P:
Spagnuolo Pata il suolo del più di bue, gatto, oca, e simili; dail Gr, Parei
vuol dire battere col pié; calpeftare; calcare; EB Patdn similmente in
2 il contadino, che porta le scarpe grandi, e grosse, e rozzameate fa
trebbe anche esser detta Partona, in un certo modo quasi Pafona, cioè
pata grea s perché¢ quella a similitudine d' un pines groffolano,e
'Pattume dilie Ser Brunetto nel Pataffio quello, che oggi dichiamo 2.
spaccatura ye mescnglio di cose fracide; e ClO pure cred' io, dal Greco
peltare. Ed sl pattume vien rammuricando, Il che ha qualche simili 0
Patrons, cosa fordida, e vile, e di brutto colore, s Greci ( per dire anche q
lo sterco, perché si scarica i) ventre lungi dalla strada comunale, che dal?
strada batcuta si dice Pates; dissero dpoparema, il che può aver dato origine al
arole Pattume,¢ Pastona,, Gli dice findi, ma per ironia, che in, vece d'
picde ben fatto, \& attillato, vuol dir piede sconcio,¢ mal fatto. Lindo
la venuta a noi modernamente di Spagna; e \& come /enda in quella lingua Vi
da) Latino /emita, e linde da) Latino mite; così indo credo che sia d i
mito, cioè limitato, aggiuftato, ben afletto, composto. Da Lindo diciamo
che Allindarsi,e Allindirsi Sp. alindarfe, '3 eng ela
si /offia it naso, e spaxafi la bocca. Eipurga il nao, e spura, e con la lin
netta identi, che sono.quei lezz), che fanno moiti Ocarori, come porre in



SESTO CANTARE: gu

brie ta persona; cioè dopo haver dimenato in qua, e in la il corpo,fermarsi in po-
fitura intir, come ha detto nell' Octava antecedente, che sono tutte smor-

fie, che denotano nell' Oratore una sciocca superbia, e prefunzione di se stesso;

ed il Poeta lo tocca col verso che segue, dicendo: Come quello che se pensa dare in,
b che vuol dire, @ima di haver trovata l'inucnzione buona, e d' haver im-

; cioè dato nel segno.

O sfrontaro. Arditamente, sfacciatamente. I) Franzese similmente ¢f-

ERT ESCA, o Bertresca, o belrresca; E' una specie di cateratta, ches' alza,
abbassa, e serve per riparo di guerra in fu le torri, ein fu le mura fra un
, ¢l'altro; o così si dice ogni luogo, sopr' al quale si salga con pericolo
ecipizio. Di qui viene il verbo berre/care, o bertrescare usato da molti per
ndere Armeggiare, o affaticarsi intorno a un lavoro, e non trovar la via a
hes i per berte/es intende la forea; per similitudine delle berte/che, le quali
i di legname, che si ponevano in alto. Gio, Villani lib. 9. 114. Per
@ il porto era tutto impalizzato, e incatenato e@ di sopra di erosso legname imber-
+ Queste bertesche, o torri di legname alzate fu le mura dovcano servire
cose a gettar pictre, onde forse € la parola pertrechor, che significa,
pre i Spagnuoli munizioni, e ripari da guerra, cioè le nostre berre/che, det~
ta forse così da echar las pedras.
- BILANCTA. Specie di rete da pescare, detta così per esser a foggia di bilan-
sia; strumento, col quale si pefa la roba.
a ella apre. Cioè quanv' ella allarga per ogni verso.
"FINO a ch' il porce vengane alla ghianda, Fino a che venga a dare nella trappo-
la; ficali al zinkello. Esintende fino a che Baldone andando alla volta di Mal-
antile dia nella rete fuddetca.
\ SIENO Iefle. Sc bene leflo vuol dir Agile. Vedi sopra C. 1, st, 11. Tuttavia
far leffo vuol dire star pronto, all! ordine, o preparato.
~ AFANNOTTO. Pesce piccolissimo. Voce corrotta da Vguannotto, o Vw
annolto, che significa pesce nato quell' anno: perché gann0,0 unguanno vuol
ir quel anno, se bene usato solo nel contado,¢ |l'Autore se ne servc in bocca,
@un contadino sotto C. 10. st. 35:1 Latini dicevano Hornus, ed hornotinus una
colad'tn? anno. Il Poeta da nome d' avannorto a Baldone, percht dovea esser
preso con la bilancia, che € la rete, con la quale si pigliano gli avannotti.
IN lnogo, ove si vezga il Sole a scacchi, Cie in prigione; perché le finestre fer-

a

—

we

=
=

a

ye 'tate della prigione, battendovi i raggi del Sole, fanno a figura dello scacchiere,
g! nel luogo dove termina il loro sbattimento, o ombra dei ferri. Da queste fine-
oo r te, Ograre di ferro delle prigioni, si formo 1] verbo e4egratighare usaco
if dal Boce. Nov. 85. Tw m' hai aggratighato il cuore colla rua ribeba', Clo€ imprigiona

to col suono della tua rideca, come oggi diremmo: e da Brunetto nel Patafiio.
TVTT At fava. Tutta è una stessa cosa. Sol est Apollo, ipfe pollo Sol, Di-
Geil Cornazzano Nov. 11. che fu una Signora, la quale volendo riprender co+
Potomee il mario, perché lasciando lei andava dalle Meretrici, gli fece un
utidimo desinare, ogni vivanda era condita, e ripiena di fave con diversi stra-
'Vaganti ma delicati fapori. 1) marito le domandava; Che cosa e questa? ed el.
we la

ee!

io.


gin MALMAN TILE?
la rispondeva; Fava, E quest' altra? Fava. In somma
gaor marito sceglieve quanto volete, perché sattae fava;: egli.
guta, e faceta riprenfione della lie, mut vita, conoscer 3
na ail' altra non può esser' altra differenza, che quella che nasce da un for
sfrenato appetito. E di qui poi venne il dettato 7 ase e fava che significa \&|
anne daxke Meee a eee so of OO OM
/L Cipolla, Autore noto, che ha scritto.in Criminale.
a Plutone, che se bene quivi, e/cln/a ogni ragione Civile s* attende
Tuttavia gli Autori criminali non approvano quell' operazione
rimette dicendo; Se tu lo comandi,io non ho che replicare, € conc
anche tu lo voletli far' impiccare, e squartare; che questo iatende / i
lo squarto. Tole 4
7 ad in budella, Non si conchiude cosa di buono, Questo proverbio.
copertamente: Far come il cane de/ peducciaio, e s' intende dare.in budella. |
esprime discorrer' assai, e conchiuder poco, ed è lo stesso che dar.in cenci
MI cascano le braccia ye? ovaia, Mi perdo.d' animo affatto.. Si dice.
cuore, le braccia, le brache, il fegato, il fiato, eda moltis' ovaia peri
pertamente è testicoli, e tutti hanno lo stesso signiticato, dl perdersi d' animo.
qui accoppiandone due, cioè /e braccia, e /' ovaia, esprime perdersi affatod! a
nimo. Latino ovaria, che si sono sCoperte ultimamente nelle donne, dagli
erano creduti, e detti 1 loro telticoli. % si
AOGNI cosa appella. Non 'è cosa che Nia a suo modo,, da: difficulta a ogni
cosa,a ogni cosa ha che dire; e non se ne sta, e non fen' acquieta,
appeharsi termine legale. Toa
CO! suoi punti mena il can per aia, Co' suoi punti legali, e con le difficalta 5
che oppone, manda in lungo le cose senza venire a conciufione alcuna. e4it
vien dal latino area, e vuol dir quel pezzo di terra spianata, ed accomodata per
battervi, e mandarvi sopra il grano, e biade, d
ALA più ritorte, che faftella. Ha più ripieghi, e compenfi, che non a
cidenti, che faccedono, Ovvero egli trova subito riparo a ogni accula «
si dicono-quei legami fatti di vinciglie d'alberi, con i quali si legano i falci di
legne, € di sieno, o d' altro, detti ritorte, perché quella vinciglia si attorce per
renderla maneggiabile, e flessibile a fine d' adattarla a legare. Dan. Inf, o. 19:
Che spezzate bavertan ritorte,¢ frrambe, et
El non lo crede, Questo termine significa Tu non ti vuoi emendare; e si dice
Won crede al Santo, /e non fa miracoli; cioè non crede d' haver a esser gaftigatoyin
che ei non prova il gaftigo. Qui dice se ei non va a degnaia, cio se egli non \& le
gnato, e bastonato: Legnaia e un borghetto vicino a Firenze, ed il nome di
gnaia ci scrue per esprimere legnate, o bastonate. Vedi sotto C, 11, st, 116 gr
tar /a tigna., Dove fimettono diversi modi di dire per intendere Bastonar un
CAPPVCCIO, Il Varchi Stor, Fiorentina lib. 9. dice; 11 Cappucgio ha te
>» parti: il Mazzocchio, che e un cerchio di borra coperto di pango, che
»» facia d' attorno alla testa, € di sopra, foppannato dentro di rovescio
x» tito i) capo. La 1 opsia € quella, che pendendo in fu le spalle, difende
»> guancia finiftra. Li Becchetto e una strilcia doppia del medesimo pane

oa
a
rm

BSB eee Fa.

Pe ee ed

=

BRE eBas FS:



SESTO CANTARE Re
ra si, in fir la spalla, e bene spafio s' avvolge al collo, e» %
eller più destri, e più, intorno alla tela, ec. EB
che già portavano le persone civili 5.¢ del quale parla il
st. 7. alla voce Adaxxocchio., ¥
STANZA LAXAXV.
che direi,0 Sire, Perch eil ha, detto.con si texfo dire,
te ch? io dica mi vien detto, Ghiioffoper dir che mais' uds tal detto;
np non ofa, ch' io non ho che dire, Pero dico.ch' a dir non mi dd il cuore 5
ir quate qui quel? altro ha detto; Elascio dire a un' altro dicitore,
ecio,, che è il quarto diavolo, fatee sue cirimonic, fa un dilcorso fen-
¢, come si vede nella presente Octava tutta di scherzo sopra il yer-
le non richiede spicgazione, ma solo rifleifione al grazioso, ed in.

STANZA XCVIIL
Valeati, dice il Re, spropositato;
S? alcuna cosa qui non bas proposta
Come vuoi tu buaccio che'l Senato
Yada in Cancelleria per a risposta?
Par sento,rispond' ei ych' in Mdagiftrato
Così dir s* hi ed io l'ho detto apposta;
Mas ioviscadolerxo,e alcun m'incolpa
hiandellino. Dica Baciapile, Drerrore in questo,iomeneredo in colpa,
ANZA XCVIL, STANZA XCIX,
Non occorre brunir co i labbriisassi,
Dice Plutone, ofsaccia senza polpe,
E fare il torcicollo, e ovunque pafi
Semmar discipline,e dir tue colpe,
Ch to foyche chi per lepre tt compraffi,
Havrebbe almen tre quarti dell.
un in mexco,bacia terra,ein fine Pera va a fieds,¢ segua il Tiritera;
7 Auago piovon discipline.. E queis' affeteaye parla intal maniera +
rende Cappelluccio, ed in tanto il quinto Diavolo, che e Libicoc-
re sbocear' Arao in Malmaatile, qual consiglio e riprovato co.
ile; Oade Plutone ordina al festa Diavoio, che e Baciapile,il propor.
-¢questi dice, che vadano in Cancelleria per la ri/potta, che € lo stesso che
Vi
VE

Bao SO pero Plutone lo fgrida, ed ordina al Tirirera che e il settino

10 dica, ed eglis' accinge a parlare.

INE. Quel che significhi diceamo sopra C, 3, st.27. E il Latino fearra,

shine. Un poco poco. E qui, clicndo deteo ironico signitica; e un,

spazio da Arno a Maimantile..

'ASEO, Balordo, melcato, stupido, bafofo, A questa voce allude la Pran-

Smarrite, confnfo, quasi sbafito. B far il bafeo vuol dir finger di nou in-

3 ersi huomo senza giudizio, dal verbo ha/ire vilto sopra C, 2, stan.
Reflo che far /4 carta di masino, o la gatta morta, vio sopra C. 1, st. 19.

? Ipocrifia, E' un SH ipocrito. La voce Ipocrito yi dal

o reco

>
?
+



THe

314 MALMANTILE §

Greco Hypocrinephai, che faona contraffare; ¢l'Ipocrifia si difinisce Vina calli.
da, ed afluta palliazione del vizio occulta; perché Ipocrito si chiama colui, che
essendo uno scellerato, nondimeno nell' abito; negli atti,e:

d' eficr buono, es' affatica di parere quel che egli nonè,¢ ep rmer iamente J rin
ta significa commediante, iitiens ~S. 'Aposbad nel Sermone da 'enerdi dopo lan
s» Domenica della Quinquagefima. Hypocrita Greco fermone fiailator ie

>» pretatur, qui, dum intus malus fit, bonum se palam ostendit.
»» faifum, crifn vero mdicium (onat. Nomen autem hypocrite translacum eft a
3» specie eorum, qui eae tecta facie inceduat, distinguentes vuleum ccerulco,
x) hivcogue colore, \& cozteris pigmentis, habentes fimulacra oris lintea gypfata,
yy» \& vario colore distineta, nonnumquaim colja, \& manus creta' Z
yy utad personx colorem peruenirent, \& populum, dum in ludis agerent, falle.
3» reat, modo in specie viri, modo in forma feminz, \& reliquis preeftigijs. I
sy Berni nell Orlando contra gl' Ipocriti, Won han'da fare lemaschere a ——
i, Questi sciagurati sono di tre forte. La prima è di coloro, che fingono |
cospetto degli huomini d' esser pieni di religione, ed internameare sono ateifli,
La seconda è di coloro, che fanno del bene non moffi dalla virtù, o dall' amore
del bene, ma per esser creduti buoni. La terza è di coloro, che dimostrano di
non esser buoni, perché altri credano, che eglino fien buoni da vero, enon,
ipocriti, In questo Diavolo si scorgono tutte tre queste specie d' Ipocriti, che
appresso di noi sono lo stesso, che Bacchettoni; detto sopra C, 2, stan, 1. Dante
nell' Inf. C, 23. parlando di loro dice: ih
Laggils trovammo una gente dipinta;
Che gira attoruo assai con lenti palpi,
Piangendo, e nel fembiante franca, e vinta; Lai
E qui dite; i/o /morte, cioè faccia pallida, e scolorita; e'dice*che pioveno
/cigline per intender uno di tali Bacchettoni falfi o diciamo Ipotrito. B sotto
nell' ottava 99. seguente dice, Seminar discipline, che ha lo stesso senso. Bs' usa
assai il servirli di questi due termini per esprimere: B? passato per questa strada
un bacchettone. Veramente questi tali infami non ia(siane di valersi di tute le
forte d' apparenze, ed io ne conofeo uno della prima specie d' Ipocriti, che tro-
vandosi in una pubblica adunanza, ia cavarsi ii fazzoletto di talea lascid cadere
una disciplina a vista d' ognuno; ed essendogli detto, che avvertifi', che gli era
cascato non fo che dalla tasca, egli raceogliendola 'disse: Non @ mia roba;'
son così buono s che io adopri tali arnesi. Di/ciplina chiamiamo quella sferza- 5
che le persone veramente buone adoprano a battersi per far penitenza; così
dall'ammunire, ovvero gaftigare-il corpo, per renderlo servo ubbidieate al fu0
Signore, e ben disciplinato; cioè instrutto del suo dovere, che è la fummilfione
alia ragioné. L! uso frequente della disciplina cominciò in Tolcana y'¢ si diffule
per tutta Italia,e si ereflero Compagnie de' Disciplinanti,o Batcati 1 aring' 1460,
Sigonius de Regno tralia. i bx ats i
SPROPOSIT ATO, Vino che non fa, ne dice éofa a ptoposite. =.
BV ACCIO. Ignorantaccio. Che si dice anche edfiraccio, Ci 4
dual, bue di panno, Vedi sopra C, 3. stan. 49. la voce arfafarco 1 Lz:
havevano diverse voci, che esprimevano questo stessojcome si vede in rid

BEG ec kf eke? oe tS eS eee

arr =

u
tty

Bee



SESTO CANTARE:; 2

. Sc, 1, dove dice.Qui ubique unt, quifuere, quique futu-
ed Gesdepucblici-) fod, fang:; hard', Bdeani Buccones, Solu
'ante eo flultitia; \& moribus indoetis, \& Terent. in Heaut. 5.
haram rerum conuenit que sunt diea in flultum:, caudex,

plumbeus.
4 plea. B' quello, che i Latini dicono wltro, confultd, ovvero dedita
ioe non per errore, Oo inconsideratamente.
angolezzo. Il verbo scandolezzo portato dal Greco al Latino, e dal Lati-
 noanoi, ha significato d' inciampare, e d' adirarsi come vedemmo sopra C. 1.
- stan. 56. e se gli da anche il significato di quelle parole Si oculus tuus fandalizat te
te, come è nel presente luogo » che preso in significato attivo vuol dire: Se io vi
dé occasione di far errore: se io vi sono cagione d' inciampo; ff ribi offenfioni /um;
2 | afero, per esempio, fo credeva, che il tale fulfe huome da bene, mail fen-
pai, che ecli da a usura, mba scandolezrato, cio fatto mutare il concetto, che

 BRYNIR e@ labbri i fai. Brunire, parlandosi di materiali fodi come ferro,
ilo, oro, ec, vuol dire Dar il lustro, e però intende qui dar il lustro ai sassi co
labbri, baciandoli spesso, atto, che si fa da i Cristiani devoti per segno d'u-

y sopra C. 2. stan. 9. disse dar il lustro a' marmi co i ginocchi.
ACCLA Jenza polpe. Carne cattiva ame quando si compra la carne, che
sia con molto offo, si dice: Vi e poco del buono; e da questo dicendosi a un
-huomo o/sa senza carne s' intende tristo, ribaldo, o scellerato.

CHI ti comprasse per lepre, havrebbe almeno tre quarti di volpe, Chi ti credesse
semplice, troverebbe poi in te tre quarti almeno di malizioso, o furbo. In La-

tino fidirebbe: Pro fimplci columba, afluta vulpes. In tutta questa Ottava narra
«| -Moltedi quelle azioni be fanno gl' Ipocriti, e Bacchettoni falfi.

jp. aS STANZA

i La che sono un! infano, eignaro ogni hora, Finché lo spirto sporti al foro fora,

Bb erche saper fupir non voglio, o vaglio, Dond' ti fa i peti,e puted oglio,e d'aclio,
oy 'al Duca,percht a' muri ei mora Accio ' accia fu? aspo doppo atdoppi
it Tofoin tefha si dia Jie meglioun maglio, La Parca,eil porco con la fhoppa Poppi:
@  Wiiritera, che e il fettimo Diavolo propone che si dia in sul capo a Baldone,
i €s'ammazzi. 1 Poeta lo fa parlare in bifticcio a imitazione del Pulci nel suo
f Morgante lib. 23. che dice. ° 3

La casa cosa parea bretta,¢ brutta
ah è V inca dal vento /a natta, e la notte,

eF

goo Stilla di flelle, ¢? a tecto era tutta,

BB. Mapnifrnte E fuina, e fuena di botto una borte.
BH Pere havea pure,¢ qualche frasta frutta,
io Del pane a pena ne deste a tai dotte

Poseia che pesci, e lasche prefe all' esca,

Y Lt Metta allorra alla frasca fu frefea.
MAGLIO, Dal Lat. malleus. Martello grande di legno per uso di battere i

.

ie

- Setchi alle botti, o per ammazzare i buoi, o per altri lavor: di legname, nei
squali richicggano

percufioni gagliarde, e gravi.
Rr z

SPOR-



316

SPORTARE. Avanzare in fuora, come avanzano le gronde de i tetti fuori
dclle muraglie delle case; donde Sporti quelle aggiunte che son fatte alle case
fuori del muro maestro, e rette da' beccatelli, sorgozzoni, o colonne, (in Latino
Afeniana, che il Filandro sopra Vitruvio definisce prorette proiet pergula
dicate a Menio, Oc.) € qui vuol dire = feapps, o esca fuori lo io EBT

PETO. Quel romore che fa il vento stappando all'huomo dalle parti di basso.
Lat. peditus.

ASPO. E' un bastoncello con due traverse in croce contrapposte, e distanti
alquanto l'wna dall' altra, topra vi qualei raguna il filo/per ridurlo in'
ane dal' ane pare, nape pelyeapeuaaaeal Gnindolo onde Agenina

PARCHE, Le tee donne appellate Clra, wAcropo ye Licheft, e dere
quia nemsni parcunt, five quod parce,@ pene avare vicnm eribuant. La Gi
ilimava, che queste futicro Figliuole dell' Erebo., e deta Notte., se
Natura Deor, e secondo altri, che faflero Fighie di Demogorgone; €
figuratiero le tre cose necefiarie all' hnomo, cioè il nascere, ii vivere
re; dicendo che una di loro detta Cioto fila, cheè il nascere, ta se ) detta
trope annaspa, che è il vivere, la terza detta Lachefi taghia il t ce il mo
re. Le chiamarono anche Nona, Decima, e Morte. 1. a Ee i

STANZA Cl. STANZA CHL —
Ben tu puxxs ai pazro ch' e um pexro, Lonon fo se Baton fornia, 0} iy
Disse Pluton, bestiaccia, per bifticcio, Perché ei vuol,
Perch' io per me non fo, ne raccaperro Famate i conti 5e conta,
Quel che tu voglia dir neltuocapriceio, Wel zerolho frat unaye:
Ata non son Re, s'io non'te ne divezRo,
E perché tu non tent grattaticcio,
Mentre stima non fai delle bravate,
Quest altra volta le (aran pectiate. Tremande andranne come
STANZA Cll. STANZA CIV.
via seguite: Sui lo Scamonea Ola, dove siam nos ( dice P
Si rizza, in vrfo tutto infauguinato, Eche Yi bite chtio
Perch' ei,ch'eun fastidioso,appio havea Daro benvio fat a 4
Fatto a graffi con un, che "ha en allato, Si calle fiele iv cnderd it bade

Pero con la bifunta sua geornea, Guarda quel'thetu di barone
La qual traluce come Ciel frellato, E va piit'tespo ye col 'aki
Sich'ella un' Argo par fatto alla macchia, Sta ne i vermini-, e parla con gindiciy
Sinettayal Res' inchina,ecostgracchia; Che per min se ti privo dell' nfixis.
Plutone dopo haver riprefo il Tiritera, comanda, che 'dica Scamonea ottav
Diavolo;'il quale da anch'-egli un consiglio spropositato,¢ con parole eel
onde Piutone lo fgrida,minacciandolo di levarglivia degnira Senatoria, o
non s' avvezza a parlare con termini oneiti,'¢ rilpettosi. Poe «
BISTICCIO, E: \a figura'che i Greci dicono Parecheff yed \& quando si
due parole che hanno lo stesso', o poco differente suono-, diverso
come si vede nell' antecedente ottava 100. ene i due primi veri
101. Detto Bificcio quali Dificcio dal Latino greco Disti¢hnm; rela
che Biforto \& fatto dal Lar, diPortus; Biffemto, dal Lat,

£0

Oat FS OSes ten renzEesa

Be RE fee a



 SESTIO CANTARE;? 317
»Ci0e maltrattare,efimili,. Imperciocché i primi bifticci, de' quali ci
gli Efempi'yconfiftevano in Distici, o eel dire coppie di versi

lia stessa vocesla quale significava duc cose diverse,secondo che o piuilar-

b stretta, o intera, o'dimezzata fiyprofferiva. Fra Guittone d' Arezzo,

' Poeti'Antichi di. Mons, Allacci, tutta una Canzone va tessendo

i diparole ed' quella che si trovaa carte 385..nella Licenza,

qual Canzone dice cosi

ny yedo,
Sen' ake mido,

aera Edi, che prefofo,

iihies tex i912 cio vuol di tornar fo,
la in'primo luogo vale ad banc ipfam hor am,siccome adesso vale ad boc ipfur
| secondo:luogo %d ¢/savuol dire ad ¢/sa mia donnaya les, 1 fait eda,
coll secondoymeido,L, me dedo, || primo fo vuol dir/ono verbo. I fecon-
-Ne'fonoefempi in Bindo Bonichi, ed in Francesco da Bzrberino.
raccapegzo.. Non fo'ridurre'a capo: Non rinucrgo: Non rinucngo:

vo: Non intendo. \

C70. Qui vuol dire opinione, o pensiero. Vedi sopra C. 1, st, 21.
'fon Re, Laicio d'etier Re, E' termine giuratorio che esprime Tanto
“€vero che iovho fatta, o farò la tal cosa., quanto è vero cheio sono quale io fo-
0 Non (60 Padre ui Telemaco, cioè non sono. Viifie feio non ti feufto; Dit

f 4 Terfite presso a'Omero.

ui te ne dwezzo, \S' io non ti fo lasciar questo vizio,-0 questo tuo modo
-diteattate. E> il contrario d' avvezzare. Vengono da Vizio sens avvitiare pec
eallietare a'un vizio difuxiare per liberare da un vizio. E questi due verbi
attivi, che neutri hanoo sempre lo iteilo significato. Diciamo per esempio
'Phateeces I del tabasco', cickiessersi afiwefatto a pigliarne.
tem gratcaritcio.,. Twaon fai Rina de i piccoligaftighi; Tu non temi
; enon cri le riprensioni.. Nelle Raccolce de' Greci trovafiun certo
ico', che voltato.in 'Latino suona così:
sls (6a Tncus maxima non timer Strepitus.
Egrattaticcio intendiamo grattatura, che leggicrmente offende la cute.
PECCIATE, Petcofie nella peccia, Caici nei ventre. Termine baffo, e più
toffo feherzoso. Peccia lo istesso, che pancia, se\bene della parte, che è dallo ito.

Maco al none Peccia pare più verso lo stomaco, Pancia più verso il petti-
“Btione, Questa'è dal Latino pancices; inictini; quella forse dallo Spagnuolo pe-
Latino pettus, onde Rimpecciare.

| BISUNT A giornea.. Velte atiai unta.. 'BE per giornea's' intende la sopravveRe
 Mdei soldati', che i Latini dicono Ch/amydem ye Lispiglia per veste d' aurorita,
 donde habbiamo un proverbio che dice.
AP PIBBIARS! la giornea,, Che Gignifica prefumersi molto di se medesimo. 11
'Bn,. 102, parlando di Didone dice:

Come



pugs. -IMALMANTILED®

Came Diana allor che xscirne acacia,
Lungo ? Exrota, o pure in Cinto (nales damipcab
Fratutte U' altre la giornea s allaccia

E suol parer fra le sue Ninfe un fole
Il Forti, parlando della Prammatica delle donne al caps mibi 242.
parole da i libri pubblici di questa Città,dice: Wen porevane portare
o mantello o altro vestito sparato, ne maniche sparateso tagliate per il lunga
cia. Donde si deduce, che questa era yna sopravvelte, oizimarra aperta t
dinanzi, usata anche dagli huomini,di conto nelle case..Ma da noi hoggi-
glia per toga, o veste curiale, che chiamiamo Jucco, e nel. presente 1uogo |

dir questo, *

RALVCE, Traspare; E s' intende, che era piena di buchi, perel
giunge pare un' Argo fatto alla macchia, cio s' aflomiglia a un' Argo malfa
Argo fu quei pastore, che havea cento occhi., e fu lasciato.da Gi
dia d' lo figliuola d' Inaco convertita da Giove in vacca; ed a sen
miglia i buchi, che erano nella veste di Scamonea.« Plautoy se i
chiamé casa illustre quella, per la quale per essere il tetto rotto, si vedeva il Cite
lo. Quel che voglia dire aipingere ala macchia, vedilo sopra C. 1. st. 69, dove
vedrai anche il significato di gracchiare. ew

PRAT IC A, Intendiamo Confulta, o Congreffo di Confultori dallo Spagnud-
lo Platica ragionamenio, discorso, donde Praticare um megezio vuol dir,
o maneggiare un negozio. Varchi St. Fior. lib. 14. Ragunafi la eraser
ro, che per esser la Città ferma, non faceva bisogno fare altra spefa. wa
volo credo, che intenda furbar la noffra pratica, cioè dar diflurbo a 2.3
nostra amica, perché haver xna pratica si dice quand' uno ha,o fitiene qualche» | a,
donna, o innamorata: e corrobora questa opinione il sapere, che Baldone noms Uy
flurbava il Consiglio de' Diavoli, ne Ji loro congressi, o pratiche, ma

=

Martinazza con aflediar Malmantile, “ae AG
L! HO nei zero. L' ho nel forame: Non lo stimo. Zero è la figura tonda: 'ay
Abbaco detta forse da Giro, la quale forma le decine, e per similitudine s* inten Pie
de il torame, e ci serviamo di questa parola per coprire il detto sporcost' hols | hi
c\ellipsis{18pt}, usatissimo fra la gente bafia in questo significato di disprezzo; equitormas | %
bene, perché dice con rasta la sua aritmetica, cioè abbaco, io/'bo nel zero, che® | ity
figura di aritmetica. 4

BACCHIO, Baltone, o pertica dal Latino baculus. E felleticare qui intende
perquotere; e parla ironico, perché le bastonate sono contrarie del folletico,

NON fara in gramatica, Non fara difhcile, e che ci voglia grande stadio.
Gramatica preflo gli antichi volea dire /ingua Latina, come quella', per Ji
la quale ci bisognava lo studio della gramauca. E perciò la Greca anticajoyvero I
Ellinica, e litterale,.che si conferna folamence nelie scrivure; a differenzadelld |\,¢
volgare, e moderna, la quale oggi si parla, corrotta da quell' antica, e fichiama =
Komeca, cioè Greca de' sempi baffi, ne' quali i Greci non più tennero il lor antico
nome di Aellines, ma per gl Imperatori Romani, che in Oriente avevan trasfe-
rito ! imperio Romei comincjaronfi a nominare; quella Greca antica, dico, tr-
vasi chiamata gramatica greca; perché gli Odierni Greci per apprenderla my

sogno

2



SESTO CANTARE: 319
fognò di gramatica, si come noi per imparare la Latina. Nel principio dell' an-
tico ss enero ee delle vite di Plutarco si legge. Qui comincia la
di Plutarco, la quale fne traslatata di gramatica greca in voleare greco in Rodi,
B perché la Grammatica'è cosa spinosa, e difficile; per questo il dichiarare,¢
re l'intelligenza di qualche fatto, o questione oscura, e imbrogliata di-
sgramaticare
RACHE piene, Per la'paura si movera loro il ventre, e s' empicranno le
» Vedi sopra C. 1. st. 43.
sT1CO., Uno che difiicilmente ha il benefizio del corpo.
ME paralitica, Cioè tutta tremante come sono | paralitici.
VE sia noi? Dove credi tu d' ellere ? Termine che significa Porta rispetto
jerfone sed al nega dovetu sei, Alcflandro sentendosi recitare da uno, che
distesa la Storia de' suoi fatti, una narrazione lontana dal vero; disse allo
|5, Evdove eramo noi allora? quali dicefle: Che non ti ricordi; che io v' era
5? Altre volte significa: Che non hai giudizio? per esempio T dai cexto
tale, che non ha haver 50,, dove siam noi? cioè dove siam noi col cerucllo?
E si? Termine usato per indurre timore, ed ha del giuratorio; E che si,
? quali dica: Giuro che si; ch' io ti zombero, se tu nox parli meglio. Si
per fare star a segno i fanciulli, E che si, che io vengo cofid, e vi sferzo,
Sidice anche, Vale o giuochiamo, o stiamo a vedere, che io visferze ? Vin Poeta
moderno se-ne servi per giochiamo, dicendo:,
'ahscas E che si, padron mio, ch'io m' indovine
SD ei Del voffro andar girando la cagione ?
SCORRETT ACCIO. Huomo scorretto diciamo colui, che senza rispetto al-
Gund dice parole'sporche' ed oscene, ed indecenti in ogni laogo.
ZOMBARE. Perquotere. Bil Latino Verberare, Dal faono. Così Typro de?
 Greci sche vuol dire verbero, e verbo fatto dal suono; onde ne nacque Typanon,
=. we yi Tamburo; dal quale abbiam fatto noi Tamburare, e T ambuffare;
i ympanum, Zombare. Appresso i Greci bombes e il rombo, o romore delle
Pappresso i Latini bvmexs è il suono che fa il corno. Appresso di noi Bom-
berda ¢\detta dal gran rimbombo neilo spararsi;; € così tutte queste lingue si sono
accordate, contraffacendo il suono medesimo, che da cose concave ulcendo, ¢
rigitando, e:ampliandosi perwene all' orecchio.
\&/MBOMBO. Rifaonamento, l'Eco, cioè quel suono che resta alquanto
ug romore, e maffime ne i luoghi cavernosi. Dante Inf..C, 16,
Già era il loco, ove s' udia il rimbombo
by i Dell' acqua che cadea nel? altro giro
ac hi ' Simil a quel che  arnie fanno rombo,
A VA col calzar del piombe, C: ina adagio; e fd nelle tue op
4 diy Governati con prudenza, Lat. A¢arura ler, Dante Par, C. 13,
yg 2 3 E questo ti sia sempre pivmbo a* piedi
4 = Per farti muover lento come buom laffo,
o
i

RADDA Ed al si, ed al mo, che tu non vedi



320 MALIMAN TIE BS @
(STANZA CV.: Z
S* alza Scorpione alloraye wiere da ef
ate it Corno orvibile, proposhoy
Che gli eserciti dive in fuga ha, melo.
Conforme ferive, ¢,accerta, 2 Arioffo.
Si rallegra Pluton,e dice; Adesso
Naw ci fara: dal Cancelliere opposhe
Perché ci calza bene, € certo questa
Cosa del corno a mevarper la testa.
STA
Vuoi forse darci qualche eceezsone,?
Stiamo in decretis; diy peta vestito;
Va ben, risponde il Sere, ch ex proponey
Cosa, che non deprava ordines9 Ti0% ognun.
Fatta che hebbe Plutone la.bravata a Scamonea,si riazo Scorpione
volo, e propole, che si pigliaiie. i| Cormo.d: Afiolfo,, il che piacque a.
per questo si volioal Cancelliece domandandoli,se ci havewa difhicul
provo; Onde Plutone ordino, che si.facesse il partito.. F
SOGGHIGNARE.. Mottrare, o far segno di ridere quali dafu
ere per segno di. dif

bene in:sua forza \& il latino fubridere » ed è un certo,
zo, o di poca stima, che altri faccia di, qualcosa 5. e si, chiamayrifo
cioè non puro, non vero; ma, fate.;

JO non son qui per candeliiere. 1o.non son qui solamente per far
devo dire ancor'io i mio. parere, quando occorra.

DOTTOR de' mei strvali. Termine di disprezzo, e vuol dire
Vedifopra C, 4, tt 10. Aewe

PET O-vestite « Che cosa sia peto, vedemmo nell' ottava roo, d
quando.il vento esce dalle parti da baflo accompagnato con qualcosa altro,
ce peto vestito. Eda questo il Lettore può comprendere quel che significhi,,

SONATE un doppio, Quand' altri dopo molte cose mal fatte ne fauna bent
dal medesimo solita farsi di rado., o vero dopo, che uno habbia terminata
faccenda con grande stenta, ed in molto tempo, diciamo.: Sonate wm ¢i0
tutte le campane per l'allegrezza di questa cosa insolita » o della rerminagion
di questa faccenda, che si pensava non -havesse a esser terminata may) |

F-AR il partite, Fas.loferutinio, che noi volgarmente diciamo far lo /gxitine}

o fquittinare.
STANZA CVIIIL, STANZA CIX.

Vanno le fave attorno, edi lupini-, Vauno i danyelti ognun dalla sua banehy
E sentefi fiuonato,¢ fuor di chiave Ma perché ne ricevan: Be
Alle panche gridar: Tavolaccsné, * Che pis neffuna ardy a it.Re comand
Raccogliete pel numero,¢ le fave Se mon vuolyche a pier popolo si sferti
Pigliate in man; che questi cittadini, Di nuovo attorno s boffoli si manda
Che in simil Luogo far dourianfulgrave Da vincersi il partite pe' due rere
Rendano( il capo havendo pien ds baie ) E cercate alla fin tutte le panchty
Male i partivi, e mangian le civaie. Fu vinto non ostante cana:

te

.

RPS R RE SRA HE GRE ERE REESE

Bess ~~



| j donzelli vanno raccogliendo i vor!
ti in contrario fu vinto, che si

lone da Malmantile. E qui ter
~ Vedil Ariofo nei suo Orlando furioso., che lo finge un

i fyono fugava la gente.
fave arr edi Inpini. E' costuine in Firefze, come era anche
di fare i partiti, o (quittini con fave~, e Jupini; e pero havendo il Poe-
uto, che nel ConGglio grande di Firenze chiamato il Consiglio dei Dugen-
yhel quale inte ono centinaia, e cehtinaia di persone ( come in questo
Consiglio de' Diavoli e necessario, che intervenissero sopra 300, Demonj, mentre
to voti non impedivano il yincere il partito) i Tavolaccint, Donzelli van-
F endo le fave, ed i lupini a coloro, che devon rendere i) partico, fa
'il medesimo costume nel presente consiglio de' Diavoli, dove dice che si fen-
idare spuonato, e fuor di chiave, cioè in voce, che non intuona, e non accorda,
) procede, perché essendo più d'uno, ed in diverse parti della stanza a,
impotfibile che s' accordino nel tuono, come anche perché dette yoci
ite'fra tanta gente, che bisbiglia., il che le rend ottule, ed offulcate.
VOLACCINO, Servo,0.Donzello di Magistrato; così detto secondo al-
r abellio detto fopta in questo C. st. 74., ma io credo, che i Tavolaccini,
che sono un 'numero determinato, e differenti dagli altri Donzelli, sieno quelli
che al rempo della Repubbiicha stavano sempre in palazzo, e servivano alla ta-
] vola de' S5. ciascuno il fu', e due n' haveva il Gonfalonicre, e si dicevano Ta-
 — volaccini dal servire alle Tavole; e che habbiano conferuato il nome, si come si
conferua ancora J" usizio, essendo costoro obbligati a andare a servire alle tavole
eo in palazzo del Sereniss. G, Duca in occasione di Forefticri, o di Spolalizzj, ec.
ma per altro aprono ogni mattina, e ferrano ogni (era le Porte della Città. °
» RACCOG LIE le fave per il numero. A fine di saper con facilica, quanti fieao
coloro, che rendono il voto, il Tavolaccino pigia in mano'da ciascuno una fa-
va, o poi si contano., ed indicano il numero de ivotanu, equelto si dice
c i numero, E pigliano le fave in mano, enon nel boflolo, per aflicu-
-rarsi che non vi sia chi ne metta pi d'una, ed alteri il numero,
STAR sul grave. Tener il decoro, la gravita. Star favio,
HA il capo pien di baie. Sempre vuole icherzare.
RENDER jl parrito, E' quel dare, o mecter la fava, o lupino nel boflolo, che
si dice: dare il voto.
4 PIEN popolo, In prefenza, ed a vilta di tutto il popolo.
“BOSSOLO. Quel vaso, nel quale si metiono i voti dagli Ateniefi detto Camus,

4 Vedi sopra C, 1. st. 37.

am

e

\section*{FINE DEL SESTO CANTARE.}
\end{document}

J i

si

h ot

; pw

i Ss \ SET-



SUSE le a
oe

a

®

= =

ARGOMENTO.
Paride dop' haver molto bevuto
Entra,' andar' al campo, in frenefia, ae
E come il sonno havea pel ber perduto,;
3 Perde nel gir di notte anche la via: te
Cade in un fosso, onde a donargli aiuto ath:
. Corron le Fate,¢ gli nfan cortesia; eee
Vien condotto sn un' antro, e per diporte e
La froria gli è narrata di Magorto.:

Sepepapeapeaye pase

joinaatuaiaa
?

Beis -gFf essete fer

STANZA L STANZAII
V Ino tempera te disse Catone, Perché se quel s marrage ne einuecebidy Me
Perché si dee berue a modo,eaverso, ed è burlato il tempo di [un vithy =| ii

E non come cosa qualche trincone, Almen Sent ilfapor di quel cb'el

Che, giorno, e notte sempre fa un verso; E tien la faccia rossa ye colorita

Ond! et si quoce,e perciò ei va aGirone, Buvlar anche si fa chi va alia.

La fauola divien dell' universo, E infacea senza gusto a

E vede poi morendo in tempo breve Che'jo tien sepre bol/oyein man del

Chie ver sche chi più beve mance beve. Aquall we a Pe morir di tified.
lL

Bre aa Ee

STANZA I STANZAIV,
S? il troppo vino fa, che 'hnom foggiace Pero sia chi si mae egli, è um dappoce
etal' error di tanto pregiudizio; Chi imbotta al sayy come gli
Chi non ne beve,e quedo,a cui no piace, 5S! avvegzi a ber del vinoa peeve,
Aquesto cite dunque ha un gra giudizio; Chiei fa che Vacqua fa marcire si pally

Anzi che nd, sia detto con Sua pace, Aa com' io dico si vuol berne pipe;
Per c ogni eftremo finalmente e virio, Basta ogni. volta cingue, 6 se

E se di biasmo e degno Punose taltro Perch' egli è poi nocivo il ersncar tame,
Questo ha il vataggio al mio parer stz'altro, Com' udirere adesso in queste C
Volendo il Poeta narrare in questo Canto l'accidente occorso a Paride
ni, per haver troppo bevuto, s' introduce col riflettere, che siccome e male
molto vino, così che sia anche male il bere solamente acqua; e 2 che
dovendosi eleggere uno dei due mali, sia meglio eleggere quello del ber vi,
ma pero regolatamente, et MO-



SETTIMO CANTARE: 323
AMODO,¢ a verso. Regolatamente, E' il latino vulgato: modis, o formi
dmipblein co: i
 TRINCONE.. Uno che beva afflai. Da Trinchen Tedesco bere, tirar giù.
i sopra C. 1, st. 6. Si dice anche pecchiare nella presente Ottava teraa, quasi

'facciare il vino come fanno le pecchie, ( cioè l' api che fanno il miele, così der-
; ee me le quali fueciano il dolce da i fiori, ed i vini bianchi geac-
'tolit edaldetto verbo pecchiare si dice pecchione a uno, che beve assai; e pecchione
ichiama ua' ape faluatica,e maggiore dell'altre, che fuccia il miele prodotto dall'

api da' Latini chiamato fucus.. Virg, gnauum fucos pecus a rabepibas arcent,
dice ciencare nella presente Qttava quarta. Vedi il Landino esposizione a Di.
nf, C, 9, alla parola cionca nel verso Che fol per pena ha la /peranza cionca, do-
dice, che cionco è parola Lombarda, e significa moxxo, ma cioncarc in Fiorentino
mnifica difordinatamente bere; Sì che questi tre verbi trincare, pecchiare, e cioncare
fanno lo stesso significato, e se bene hanno del foreftiero, wuttavia sono usati in

o

na

ot.
"SEMPRE fa un'verso. Sempre fa la medesima cosa. Diciamo Verso il canto
pce “4 Verso del rusignuolo, Verso del fringuello. E da tal verso viene
trato.

'WA AsGirone. Huomo, che gira; intendiamo pazzo. E però servendoci della
voce Girone, che e un Villaggio vicino a Firenze, copertamente intendiamo
uno che fa delle pazzie, come s' intende nel presente luogo.

DIVIEN la favola dell universo, E' burlato da wtti. 4 ore est omni populo, It
Lalli Ha, Tr. C. 4. 2. 78.

Son fatta ime la favola del mondo

I) Pett, Ata ben veggio or, si come al popol tutto Favela fui gran tempo. Tibullo lib,
1, Ne turpis fabula iam. Nella scrittura. Et fattus /um illis in parabolam,

CHI più beve manco beve, Cioè, chi troppo beve s' ammaia,e muore,e così vive
poco, € per confeguenza beve manco, cive dura a bere manco tempo di colui,
che beve poco. Marz, lib. 6. Jmmodicis brevis est atas, o rara fenettus, che da
noi poi si dice in proverbio Poco cs vive chi troppo sparecchia. A similitudine di que-
flo si dice: Chi più fudia, manco findia,

OGNTeftremo è vizio. Ogni eftremo e male. Ogni troppo è troppo. Questa
sentenza wfiamo dirla // troppo, e il poco Guasta id gineco, Al che pare, che facciano
molto a proposito i seguenti versi di Orazio.

Eft modus sn rebus, funt certi denique fines,
nos ultra, citraque nequit confiftere rectum,
E Terenzio mettendo in Latino una sentenza d' un favio della Grecia disse; We

ee a es ee ee

quid nimis,

SENZ' altro, Assolutamente; senz' alcun dubbio. Latino fant, procul dubio

VA ala secchia, Beve acqua. Secchia diciamo quel valo, col quale si cava
— da i pozzi dal Latino firwla. Vedi sopra C. 5. st. 10,
: ACCA, Per Gmilitudine diciamo facco al ventre dell' huomo;quindi dnfac
tare vuol dir Mandar giù nel ventre. Pulci Morg. C. 19. st. 137.
sot E mangia, e beve, e infacca per due verri
Peril contratio/acar in iipagubelo è trarre, —— fuori,

8 2

i ee

scl



324 MALMANTILE: >

S¢lelT A. Che non ha fapore alcuno. Deh kagions ia, SR
BOLSO. Vedi — C. 3. st. 53. Grasso non naturale, con di direlpi-
ro. Cavallo bolfo i Franzefi dicono pou/if dal pullare, cioè | a f
la lena affannata. Lucano lib. 4. Pettora rauca gerunt, qua creber anbelitus urge
Ex defetta gravis longé trabit ila pulfus. r aivye
LN man del Fifico, Col medico sempre attorno; cioè fempreinfermo, o
CAL imborta al porzo. Chi beve sempre acqua. E' lo stesso che dafaceare
to sopra. o aged
ANIM ALE. Intende animale irrazionale. Se bene la voce animale \& generi-
ca,¢ comprende sotto di se anche l'huomo, noi ce ne serviamo per speciale, in~
tendendo solamente le bestie, sì che dicendosi a un' huomo T sei un! animale, in.
tendiamo Tx sei una bestia; Vat srragionevole, ae
Ss? AVEZZI, 8 afuefaccia., Vedi sopra C, 6, st, 101. Nth > Aa
FA marcirei pali, Vuol dire:il vino si guafla annacquandolo, quasi dica;
infradiciare i pali, che reggono le viti, che producono il vino; o se fara
infradiciare il vino, che nasce dalle viti., che sono più deboli de i pali, mentre
son da essi sostenute, Dichiamo anche per biafimare l'ulo dell' acqua: 2? acqua
rovina è ponti: quasi s' abbia a intendere + O pepfate, se non royinera gli foma-
chi deglt huomini, che sono più deboli ! > KE
BOCC ALE. E una milura capace della meta d' un fiasco Fiorentino, Dice
cinque o sei boccali per scherzo,sapendo bene, che ogni maggiore bevitore non

bevera mai Gi gran quantità in una volta, van
STANZA V. STANZA. Vide
Omai ferra gli ordinghi, e le ciabatte E Paride, ¢' anch'egli si ritrova
Chiunque lavora,e vive in sul travaglio, A corpo voto in quelle e hie}!

E difilato a cena se la batte

A cafayo dove più gli viene il taglio.
Chi dal compagno aufo il dente sbatte,
Tanti ne vaatavernach'è unbarbaglio,
Parte alla bufea,e infin,pur che firoda,
Per tutto e buona fhanza, on'altri goda,

D Amor chiarito figliod' una Lous,
Che sualiziar gli ha fatto le bufecchie,
Dice al villan:Va a coprarmi delve
Ecco sei gink, sonne ben parecelne 5
Piglia del pane ye sopra tutto areca
Buon vino fai\ non. qualche cerboueca,

STANZA VII, Abe

Eset' avanza poi qualche quattrino,

Spendilo in cacio, non mi portar reffor
Meller fine, rispose tl Conradino,
lo torva, 8°10. ne trove, ancor corefto.

E partendo gli ride? occhioliney
Sperando haver a far un po dagrefe;
Aa, facendo i snoi conts per la via,
S' accorgeche e' non v' e da far calia.

Deferive afai vagamente il venir della notte; fu la quale ora Paride affalito
dalla fame comanda a Mco suo contadino, che vada a comprar roba da-ma
giare,¢ da bere, e per tale effetto gli da sei giuli, con ordine che gli spenda

i 5G

tuttl. e by. soi 92
ORDINGHT, Intende ogni sorta d' arneGi, ingegni, machines firemsentiod'
Javorare. Diciamo anche Ordigni; anzi gli anuchi non difero algrimenti

CIABATTE,, Vuol dir propriamente scarpe vecchie, e quelle scarpe all! Ap *

stolica, che usano i frati scalzi, ma s' intende anche i frammento di ma-
tcriali di coloro che lavorano', e per ogni sorta di mafieriziuole veechies'¢ 6
fumate, che i Latiat diconoscrara., Z WWE

Ay

ZeRm =

— t

Bee essezeane28=

z

Fae BF ee SSR est avEeESeF®.



SETTIMO CANTARE: 325
 VIVE in faultravaglio, Latino manribus viitum quaritat, Campa delle /uabpaccia.

'Travagliare in lingua Francese vuol dir lavorarc, ed in Firenze pure è usato in.
6 denbo diegndns: cosa ben travagliata in vece di ben ieleaans edi qui si di-
in vece di viver col lavoro, o con le sue fatiche, cioè di quel che si
alavorare. Petr. C, 3.
es ynque animale alberga in Terra,
Se non se alquanti.c' hanno in odio il Sole,
: Tempo da travagliare è,quantoe il giorno:
Ma poi che 'l Ciel accende le (ue Hele,
Ree rteit. 9 * Qual tornaa casa equal s' annida in felua,
ve % portent Per aver posa almen infin all' alia.
es ben per altro travagliare yuo! dire esser' anguftiato da infermita, o da altro.

  DIFILATO. A diriuura: Latino re%a. Con preftezza, e senza fermarsi.
ze serve anche sotto in questo C. st, 63. Varchi Stor. Fior, lib. 9.

non prima giunto a Firenze che andandofene difilato, fenra pur cavarsi giù

i SE la bate..Se ne va via. E' termine assai usato fra la gente bafla per espri-
ib yia, o partirsi in fretta, ed ha del furbesco batsere /a caleofa, cioè
: utter la trade, andar via, camminare, donde /frada battuta vuol dire strada.,
a 2 camminata, o strada di passo. Latino via trita, Lucrezio Avia Pie-
i' Tidum peragro loca, nullins ante Trita fol, 1\ Petcarcha disse: Ogni segnato calle,
—- Prove contrario alla tranquilla vita.
DOVE gli viene il tagio. Dove gli torna più comodo. Vedi sopra C. 2. st. 48.
' ', Senza. spendere. E' detto plebea... Si scrivono da i Magiftrati di Si.
 renee: di,commiflioni ai Ministri forensi, le quali da coloro, che le chieg-
g0n0, ele presentang; si pagano.a i Magiftrati, che le fanno, ed a i Ministri,
I¢ le gicevono; e quando non sono chielte, ma son fatte, e mandate per pro.
'Prio interefle di quel Magi(trato, che le fa,non vi e spefa alcuna, e pero afiaché
tall lettere, le quali non si pagano,si potiano distinguer da quelle, che si pagano,
fetivono nella soprascritta ¢x ofitio, ma l'abbreviano scrivendo ex Viso, ed i ta-
volaccini, o donzelli, che le confegnano non leggono se non ex fo, e distin-
Buono queste due specie di lettere, dando a quelle, che si pagano il nome di let-
tere col diritto, cioè con la dovuta.spefa, ¢d all' altre il nome.dell' Hf, cioè fen-
za spela + Edi qui e nato questo detto a Vo, che vuol dir senza spela, e serve in
gni uccafione.

ckseee=

*~

A SBATTE if dente. Cioè mangia. 3

J E un barbaglio. Son tanti, che fanno abbagliare; Non se ne pué raccorre il

f conto senza sbagliare; o abbarbagliarsi., cioè errare; dal Parpaglione, che disse-
70 gli antichi alla Provenzale;cioè dal Latino papilio;farfalla;di cui e noto lerrare

® — intorno al jume.

è CILLA busca, Cercando sua ventura.. Bufeare.. Vuol dir Acquiftare, otte-

iy ere, puadagnare.. E dalla Spagnuola ha/car yenuta a noi questa voce infieme

) SON molte altre negli nltimi cempi.

i Si reda, Simangi. Sc bene rodere si dice de' topi,de' tarli, e simili. Per tutto

y Pbuena anza on' alsri gode, Voi bonum, ibe patria. Dove si tla bene, quello è

' 42)" buoa

—E

326 ' MALMANTILE *o
buon pacfe; E per ogni patfe, e buona Stanza, Disse come in proverbio il Pe.

trarca. ' Sp heist tity
CAT APECCHIE, (ntendiamo luoghi orridi, inculti', € dil. Mattio
Franzefi in lode delle gotte: Alor per uscir di queffe carapecchie, N
do che pecchra \& fatto da apes, apecala, o apicu/a così verifimilmente carapere
puo dedursi da apex apichlus, che vuol dire piccola fommitdy e cara preposizione —
Greca, la quale dice un certo ordine', o € aggiunta per maggior forza, come si
vede nelle parole, Carafalco, Cataictto, Caruno', che dissero gli antichi per =
Scheduno,¢ simili. > te
CALARITO. Aggiuftato Vedi sopra C. 1. stan. 1.Vuol dir che Amore l'ha
= accomodato, perché s' era pieno di mal di chiafio, come si disse one
in. 11. Oe ie
LOVA. Lorda; Poltrona. E' parola d'ingiuria a tina donna. E*voce stra-
niera; e vuol dir Lupa; che similmeate gli Spagnuoli dicono soba; e si
maeretrice. Gio. Vill, lib. 1. cap. 25. parlando di Romulo'; e Remo allevati da
una Lupa dice: Questa Laurenza era bella, e di suo corpo guadagnava come
ce,€ pero dai vicini era chiamata Lupa; onde si dice furono nutricati da lupa 3 il che
cavo egli da Livio lib. 1. /une qui Laurentiam valgato corpore lupam voratam inter
Stores putent: inde locum fabula, \& miraculo datum, +:
SVALIG/ ARE. Cavar della valigia. Qui intende; gli ha fatto'confamarei
denari, perché ha/ecche se bene si dicono i ventricini del porco Boce. giù
Noy. 10. Dove le femmine vanno in xoccoli [u pe i monti rivestendos pores delle lor bu
Jfecchie medesime noi le pigliamo per tasche, o borfe, nelle quali i tengono ida
nari. E /wasgiare propriamencte intendiamo, quando i Jadri digtrada rubano @>
uno tutto quello, che egli ha addosso; e lo pigliamo per finonimo di /accheggiare;
'PARECC HIE, Numero indeterminato che esprime, Molti; dal Lat.
que, secondo alcuni: Volgarizzamento di Palladio manoferitto; Nel mele di
Marzo al cap. de ficu. Si metta sotto alle barbe parecchie pietre, *
CEKSONEC.A, Vino fradicio. L' Accademico Fiorentino incerto, cos! n0-
minato in una Raccolta di Rime piacevoli, che dicemmo altroye essere il Bare
chielio, descrivendo un cattivo vino dice, Te et

Staccio non pafferebbe ne framigna
Tiant' è morchiato,¢ con la feccia miffo;
Sciroppo mi par ber, ma non di vigna; Ly
Chi ne beve non ghigna, ej
Ch' egh: è ciprigno, e cerboneca fina; ' b
Chindendo gli occhi, mi par medicina,

Brunetto Latini nel suo Pataffio disse Cerbonea.
Wel ver quest' e pur nuova Cerbones Hem
Forse si dovrebbe dir cerconeca, derivando questa voce da cercone che vuol dit
Fea ease at si dice cercone dal circolare, che fa il vino quando da la volta s §
fig ¥ ac SAM



" Gulbinges verso;
Hciibiekh Rife, \& argutis, quiddam promifit ocellis,

agreffo. Avanzare; ma intende d' avanzo illecito, come farebbe, quan-

acomprare roba, dice havere spelo più di quello, che ha spelo,

quell' avanzo. Vien da i cantadini, che per rubare al padrone piglia~

f ey va non matura, ( che si chiama agreffo ) e ne fanno fugo, e lo veadono.

j

SETTIMO

CANTARE: 327

QUESSER fine, Vuol dir Messer si, Ma dice Messer fine, perché fa parlare a

jun contadino: noffri fic rire toquuntur.

t occhioline. Vuol dir si rallegra. Ul rider dell' occhio forse accennd

le cravaglic de la vita,

\ lo termine ha lo stesso significato anche in Napoli, come si cava da lo Cun-
to deli Cunti di Gianalefio Abbattutis gior. 1. Cunto 8. dove dice; A¢oPrannole
kefrifolesco' li quale maritattero turte L autre fizlic, restannole pure agresta pe' gliottere

IN v' e da far calia. Non y' è da far avanzi. Calta si dicono queirimafu-
a hares argeato, che nel lavorarlo cadono, e si dicono calia quali calo
a ?

rz o dell' argento, che ridotto poi in proverbio esprime ogni sorta di pic-
it Solo avanzo..
o STANZA VIIL STANZAIX.

Perch'egl e tardi,, ed ha voglia di cena,
Poi c ogni cosa ha bell' e preparato,

st v4;il pane,e ilcacto,es! vin rocac-
ep fatto un guazcabuglio nella [porta, Si frrugge, e si consuma per la pena 5
un » Le quattro lire slazzera ye si lpaccia. Che ti non torna il messo,ne il mandate;
Lialerol asperta agloria,e insis la porta Ma quand' ei vedde con la sporta piena

. eee seh av ognor s! affaccta, Giunger al fine il suo gatto frugaso:
of EB per anticipare, il fuoco accende, O ringraziato, dice, sia Minoffe,
ae — Lavai bicchicri,e fa L altre faccende, Ch' una volta le furon buane mose.
ie | bast S.T.AN.Z.AoX,

Chiappa le, robe, e mentre ch' ei balecca Sbhocconcellandointanto,il fiasco shocca,
at In quocer t vova,e tl cacioch' e fupendo; Econ due man alzatolo bevendo,
ih Sente venirsi 2 acquolina in bocca, Dice al villan, che nominate è Meso:

E far lagola come un faliscendo,

Hlorsit ti fo briccone, addio, io beo.

-M.Comtadino mandato da Paride a provveder la roba, andò all' Oite per fbri-
arli, compro il tutto. Paride in canto stava aspettandolo con grande anfieti;
¢ fubico. giunto, egli mefie a quocer l'uova,¢ il cacio, e in ranto viato dal' im-
-pazienza, e dalla fame comincio a mangiar del pane, ed a bere.
» PER la pi corta, Vuol dir per la rada più corta; ma qui intendi per sbri-
garsi più presto. -
.» PROCACCIA. Provvede. Vuol propriamente dire cercar di trovare una co-
fa, e trovarla; Lat. per/equi \& affequi, esprimendosi con questo solo verbo pro-
eacciare la diligenza, che s' usa in cercare, e andare a caccia d' una cosa, e la
fortuna, che s' ha di trovare quel che si cerca; onde poi molti dicono: buon pro-
| €aecino uno che s' ingegna per ogni maniera di guadagnare.
|. GY AZZ ABVGLIO. Mescolanza, mescuglio, 11 Casa acl suo Capitolo del
Martello di amore dice;

Nox

habe Ve > pay


328 MALMANTILE! o
Non eva donna rica yo poverina yo)
Si facea d! ogni cosa un guarrabuglio —
Ogni fhanza eva camera, e cucina, |
Mattio Franzefi nel suo viaggio di Venezia dice:
Ear a una tavolata allegra cera, «
. Edi var} discorsi un guazxabuglio >
Il Lasca Nov, 10, Versarono aceto, vino', olio, sale, e farina, @ fecero un euat
lio il macgior del mondo, Dal che si cava, che questa voce esprime mescolanza.
di cose maceriali, ed anche di non materiali; Voce composta di Guazzare', ch
 dibattere cosa liquida, e di Boelire: quali da una Ricetta che dica »Guarzs,e
Bolli; fattone Guarzabuglio.:
LIRA, E' una moneta Fiorentina, che vale un giulio e mezzo, detto anche
Cofime, perché il nostro G, Duca Cofimo l'inventd, e fa il primo', che' bat
in Firenze questa moneta.: 33 Sepia
SLAZZERA, Cava, conta, mette fuora', fa venir fyora'a forza', E*
furbesca, se bene assai usata. gree
S1spaccia, Sisbriga: Si spedisce.;
L' ASPETT Aa gloria* L' aspetta con gran desiderio, con pazienza efrema,
Si dice anche a/pettare a bocea aperta.. Larus bians. wine
HA bell' e preparato, Ha di già mef' all' ordine. Vedi sopra C. 3... 14 =
NON torna ne il Messo, ne it Mandato, Non torna lui, e non manda alcunoa
dir quel che sia di lui. Diciamo anche 4 ho mandate il.corno, dal coruo, che man-
dd Noe fuori dell' arca, il quale 'non tornd mai. pee
GATTO frugato, Così son chiamati per ischerzo da i ragazzi i contadini.
Carus in Latino è cauto, aftuto; € con queste nome chiamafi anche il Gatto anl-
male.notas il\quale.quando.¢ fato frugato con pertiche, o con bastoni, non fa
altro, che volgersi spaurito, € che guatare; onde vogliono alcuni, che abbia il
nome. Così i) contadino, quando scende alla Città, Dante Purg. 26,) o
Non altriments flupido si turba o }
Lo montanaro, e rimirando ammuta, ae
Quando Hx 9 e faluatico s inurba, ANH
VINA volta furon buone mie, Vana volta ci tornd, Questo detto wfatissimo in
ucfto significato, vien da coloro, che stando a veder correre al palio per lo grat
desiderio, che hanno di vedere arrivare i cavalli, spesso gridano; Eccagl se bea
veramente non sono; ma pure al fine vengono, ed allora dicono, Queffe Jie iy.
buone moffe. 4 che pafiato in proverbio; significa a terminazione di ¢
¢vento, o negozio. ite Ty
St balocca. Si trattiene. Si dice anche: Par' a bada, o badaluccare; Bi YOO Wl

usata per ibambini. Vedi sopra o. 6. st. 32. he i, \&

} = 2 ak

SPP eee SF

rescezrr=

STVPENDO. Buonissimo. Vedi sopra C. 6. \&. 35. Cosa maravigliola, hy
perfetta, che induce stupore, ae
VENIR l'acquolsna in bocca. Si sente consumar dal? appetito, o per te

soprabbonda la faliva in bocca, la qual faliva e caula che /e gola gli fac bi
Saliscendo, perché il gorgozzule gli va ingid, e insu per inghiottir quell' umid by
E faliscendo \& una strilcia di ferro, che s' adatta a lerrar le porte, apeaaia liy



'di pane,¢ mangia.

01, Onofrio; ed altri infiniti.
Tl fobriccone. Ti

'quidice Briccone per brindift,
 STANZA XI.

Così per celia cominciando a bere,
d un forfo,e dagliens il secondo;
Fesigche dal vedere, e non vedere,

Ei ditde'al vino totalmente fondo;
tavola di poi messo 4 sedere,
| Lasciato it. ixfee voto sopra il tondo,
Viltoffi a dieci pan da Adeo provvsffi,
Eis th momento fece repuliffs.
STANZA XII.
4 i pan dotto,e uinginlio di formaggio
» Non glittoccaron l'ngola,e s'inghiotte
Due par diferque d'uova,e da vataggio,
| Potdice; Meo spilla quella botte,
Chet'hai per Popre,e dami il vino afaggio,
To vib feafera anch' io far le mie lotte,
| Ben th'io sia bere, fis ripieno,e /uentri,
Percht mi par,ch' una latcata e entri,
STANZA XIII.
URufticoche dar del suo non usa:
- Non saper, dice, dove sia il fucchiella,
~ Che per casa non v* \& fhoppa ne fufa,
E che quel non e vin,ma acquerello.
Civuol, risponde Paride,altra fenfa,
By itty » di canna fa un cannello,
-E, in fa ha borte posso a capo chino,
~ Con est, pel cocchiume fuccia il vino.

SASS Ek

eee SE Ae

SETTIMO CANTARE:

con'alzarla, ed abbassarla. In questo significato'diciamo ancora:
I » vedi sopra C, 5. stan. 62.

VCELLANDO. Diciamo sbocconceilare,
compagni a mensa, o che sia portata

SBOCC A il fiasco. Stura il fiasco, e fquotendolo butta fuora il vino, che è nel-
per arlo dall' immondizie, o fiore, che vi pos' essete.
hot Bartolomeo » Ela figura Apherefis speffo usata da noi ne i nomi
me Cecco per Francesco fatto da Cesco ( che trovafi nel Decamerone
cioè Francesca ) Menico per Domenico; cos! Lippo, Stagio, Coppo,
, Noferi, accorciarono i nostri antichi da Filippo, Anaftagio, Iacopo,

329

and' uno, mentre aspett2,
roba in tavola, piglia de'

Ti fo brindisi. Questo e quel modo di parlare, che dicono Za.
» come accennammo sopra C, 1. st. 28, al termine:uscir del seminato; ¢

STANZA XIV.

E perché e buono,e non di quello,il quale
E nato in fu la schiena de' ranocchi,
A Meo, che più tosto a Carnovale,
Che per Vopre lo ferba,e/ce degli occhi,
E bada a dire; Ovvia, vi farò male,
Ma quegliche non vuol ch'ei Pinfinocchi,
Edè la parte sua furbo, e cattivo,

Gli risponde: Ob tu sei caritativo.
STANZA XV.

= *

Lasciami hie la bocca ascintta,
Che diavol penst tn poisch'ia ne bea?
lo poppo poppo, ma il cannel non butta,
Risponde etieo: Po far la nostra Dea,
Che sei buttasse, la berefti tutta,
O' diferezione s'e' cen' e minuzzolo;
Paride beve,e poi gl da lo spruzolo,

STA I

Non vi fo dir se Meo sllor tarocca:

Ma l'altro, che del vin fu stpre chiotte 5
Di nnovo appicca al suo canel la bocca,
E lascia brontolare, e tira sotto,

Ma tanto esclama,prega,dagli,e tocca,
Chrei lascia al fin diber già mexxo cotro,
Dicendo ch'ei non vuol ch' il vin lo quoca,
eta che chi lo trove non era un' oca,

if Patide in barla in burla bevendo, votd il fiasco, e poi si mangid dieci pani,
Prova, il cacio provveduto da Meco, il quale egli prego, che gli defle a laggio

botte, e Meo adduce diverse scule per non glielo dare; onde
; re

Paride



330 MALMANTILE

Paride fatto un bocciuolo di canna si messe a fucciare il vino per
= s —S cui duole il <a seatats suo 4!
ere; ma egli seguita, e per farlo più arrabbiare gli sbruffa
torna a bere. Al fine già fazio, laid star di Croniede
buona cosa, e che l'Inventore fu un gran valent' huomo; ma
ber più, per non' imbriacare. 2 ety X iy
PER celia, Voce usatitiima in Firenze, per denotare buria, feberze.
una giovane Commediante, la quale era di genio scherzolo, e burlesco, e face

la parte della serva; e si domandava Celia.
| Ui Persiani. Ji tno canto e più dolce d' una auelia; ea
Ma feufami, se reco io fo la Celia.

DAGLIENE un forse @c, Cioè bevi un poco, e poi un' altro p
la quantità di vino, o d' altro liquore, che si può bere senza ripigliar
Latino /orbere. ' Mee
FAs) che dal vedere, e non vedere. La cosa andò in maniera, che it
mento; in un batter d' ecchio. /n stfu oculi.
DIEDE fondo al vine, Cioè vord il fiasco. Fini il vino. Dar fondo a un
fa vuol dir consumare affatto, Termine marinavesco; e si dice dar fondo
la nave si ferma in porto, finito il viaggio.;
TONDO, Così chiamiamo quel piatto spianato di stagno, o d' altra}
pra il quale in tavola si posano i bicchieri.
FECE repulifti, Fini; ripuli, consumo ogni cosa, ne volle veder la
inine baffo, e usato dalla plebe.;
NON gli toccaron ? ugola. Non gli scemarono } appetito. Quando a tn gran-
de affamato si da poco cibo, diciamo: Won gli ha toccato ? ugola, e ancora + Now
elt ha roccato un dente,e proverbialmente: E ffata una fava in bocca all orso.
non palatum rigat. Vgola si dice quella particella carnosa, che pende fra le faucl
per uso di formar conucnientemente la voce. Latino wa, columella, =
SERQVA, Numero di dodici, ma si dice d' vova, di pere ye simili, che}
altro si dice dozzina. cay
SPILLA la bore. Buca la botte. Spillare si dice da spillo, che è quel
to, col quale si bucano le botti, e questo forse dal Latino /picu/am, o pure
Spinula, Crescenzio lib. 4. c. 41. chiama /pina fecaria, € '| suo antico —
zatore, spina fecciaia, la cannella posta nel fondo de' yafi da vino, per
uscire la feccia. $
OPERE, Coloro che aiutano lavorare a i contadini, ricevendo il p
Ic loro fatiche giorno per giorno si dicono opere, 6 opre. In Latino fimi
opera si dicono | lavoranti. + Om
VVO far le mie lorte, Voglio far le mie forze. Voglio pigliarmi tutte lo
disfazioni possibili, Diciamo; ilsale vuol troppe lorte, troppe invenie, troppi
troppe cirimonie: quand' uno in far' un'-operazione la vuol far con ogni
ancor che superfluo, e non necessario. a
SVENT RI. Scoppi per lo troppo mangiare, e bere. ”
VINA lartata centri, Ci stia bene una lattata. Diciamo: fare uma lat
do dopo che s' è mangiato,, e bevuto beac, si fa venir in tavola nu

Pow eaenw Re BB OSS. fo. ee ee

tetas Hess:
Stes *

= F


SETTIMO CANTARE:? 331

nuovi bicchieri puliti. Che per altro /atrara \& una bevanda fatta con zucchero,
-orz0, e femi di popone, che benissimo pefti, e liquefatti con acqua gli fannd

Pe

yt passare per stamigna, la quale si da per lo più a' febbricitanti per rinfrescare: ed
. t PY pat gr i pad. hese! abbisto pot il nome di /attata ot suddetto nuovo
bere f » come che vogliano intendere, che questo secondo bere non fia
 $propositato, ne per gola, ma per rinfrescare l'ardore del vino bevuto, come fa
alla febbre la datara, la quale diciamo più comuncmente orzata.
- S¥CCAIELLO, Diminutivo di /ucchio, che vale lo Neflo. Strumento d' ac:
ciaio per uso di bucar legnami: e il Latino Terebra.
NON ha froppa, ve fufa, 1 villano per non dar bere, trova scufa di non poter
metter fa cannella alla botce, perché non ha stoppa da avvoltare in fulla cannel-
la per adattarla al buco della botte, ne meno può bucarla, perché non ha fula
da turare il buco dello spillo, delli quali fui ( che per altro servono alle donne
Se sopra il filo, quando filano a rocca ) ci serviamo per turare simili
gi bucht yperché per esser ben tondi, e di figura piramidale,ferran bene ogni buco,

A di pi r scula, che quello non e vino, ma acquerello, che è la lavatura

ai
io

gi delle vinacce, e serve per bevanda de i contadini, da molti detto vinello, e da
2 altri mezzingo, e da i Latini Lorea, o Lora. Ma Paride, che molto ben conosce,

che: oat sono tutte invenzioni, gli dice: C+ vxol altra feufa, ed intende; Non
~ - Mailerrd per questo di far quel che io ho in animo, cioè di bere.

COCCHIV AE. Quel turacciolo di legno, col quale si tura la buca di sopra
sh della botte; e si chiama così anche la stessa buca. | Latini lo dicono do/ij opercu-
him.”

ie SVCCIARE. Attrarre a se l' umido, o fugo. Dal Latino /were.
a NATO in fu le schiene de' ranecchj, Nato ne i pantani,dove stanno i ranocchi,
j@  chenoné vin buono.
i  ESCE degli occhs. Non può vederlo consumare + lo da mal volentieri, Gli

»duole il veder consumar quel vino, quanto gli dorrebbe il perdere il lume degli
@ Occhi. Detto assai usato in simile proposito.
2 NON vuol che ? infinocchi, Non vuol che con le chiacchiere lo ritenga dal bere
gi — 'Tnfinvccbiare \& lo stesso, che dar panzane, bubbole, o chiacchiere ed è il Latino Ver-
a) ha dare, 1 Lalli En. Tr. C. 4. st. 107. dice.

: - Per ch' il parlar di lei non l'infinocchi,

b OHTY sei caritativo Tu hai la gran pieta dime. E' detto scherzoso, usato in
simili congiunture, e si dice + 7% hai caritd pelosa, o la caritd di mona Candida, che
Pe biascicava i confetti agli ammalati per levar loro la fatica,
is NON fo se tu minchioni la mattea. Non s0 se ww burli, Vedi sopra C. 4. st. 15.

 Che pensi tu mai ch' io ne bea? Quanto pensi tu, ch' io al fine ne beva. Altro~
i ve habbi: detto di questa particella mai, che altre volte afferma, altre volte
6 hega, ed altre volte significa tempo, come qui, che vuol dire, quanto pensi tu,
4 Piblesnihiximene boda; In Latino dircbbelt, Waid demim cenfer?

40 poppo Poppe « Cioè io attendo a fucciare, ma io tiro fu poco vino, perché il

Cannello ne da 3
of PVO! far la nena Dea, Esclamazione, o giuramento di contadini; quasi yo-
a lendo significare /a Dea Pales. Virg. 3. ark) Te quoque magna Pales Oc, <
4 12 Te 2 SE

va Se

—

——

332 MALMANTILE a2,,
SE @ cen'? minwzob, Se cen' \& punto. Se ei cen' è pur un poto, Ser:
to Latini nel Pataffio. lovee for pata |) me i
GLA da lo spruRzole. Gli sputa il vino nel vilo a minute sille. Sprags j
ciamo quando comincia a piovere minutamente, onde Spragzaglia osservd il Vet-
tori dirsi da' contadini una piccola quantità di ope per similicudine
7 -AROCC A. Entra in collora; arrabbia.. Voce usata in Firenze, e in
Lombardia. Francesco Negri nel suo Taflo in lingua Bolognefe, portan:
quello il verso d' un' argumento, che dice 4/ Re si turba alla novela rea, pari
4 Re al sente,¢ e minza a taruccar,
SRONTOLARE. E un rammaricarsi, o dolersi di qualche soprufo, o finifiro
vvenimento con parole non affatto esprefle, ma confule, e male articolate, ¢
fra i denti, che si dice anche bofenchiare; ( Nella Valdinievole meer
to il calabrone ) Viene per avventura dal Greco Bronean, che vuol dir P
Virg. in quel verso, ove nomina i Ciclopi affaccendati a lavorare il ferro, ¢:
mini nella fucina di Vulcano. Bronte/que, Sterope/que o nudus membra,
I primo nome lo cava dal tuono, il secondo dal folgore, il terzo dall' ancudine,
¢ dal fuoco, ~
TIRA sotto, Attende, continova, (eguita a fare quella tal cosa. Si9
DAGLI, e tocca, Questo termine significa, fa, e rifa la tal cosa, ovvero pre:
£2, e riprega; e si dice Dagli, picchia, € rocca. Ovvero Dagli, toca, valle
martella, i
MEZZO cotto. Quasi briaco. Vedi sopra C. 6. st, 35.
CHE lo trovo non era un'oca. Chi lo trovd non era huomo senza cervello, ma
un valent' huomo. Ceruel d' oca, o capo d' oca vuol dir huomo di poco giudi-

zio.
STANZA XVII, STANZA XIX,

Wit rn

Che scufa non gli pare haver, che vaglia,
Che non gli fea a viltade attribuito;
Così ribeve un colpettino, e in cambio
D' andar. 4 lett s'arma,e piglia Pabio,

STANZA XVIIL

Senza lume, ne luce wia spulerza,

E corre al buio, che ne anche il vento
Non ha para mica della brezza,
Perch' egli ha in corpo chi lavora dreto;
Per la mora si ben si scandolexa,

Che dando il ¢,,interraacgni mometo,
Quanto più casca,e nella mema pesca,
Tanto pin sentech'elt'? molle, Puta '

Poiche dai cibo,e da quel vin che fmaglia, Dopo cht ei fu cascato ye vicascatly
Si sente tutto quanto ingazzullito, Ler non sentir quel mollee frescoacert,
Risolue ritornar alla bactaglia, Che'lvino,e quato diatiavea i
Donde innocentemente s'è partite, Opra di dentro si, ma non di fuera,

Giiito al mulin dal mezzin giis sbracciate
Si sciaguatta i calzoni in quella gore
Per dopo nella casa di quel loco
Farfegli tutti rasciugar al sotto,
STANZA XX,
Mentre si china dando ilc,., a leva;
Ei fece us capitombolo nelacqua -
Ona' avvien ch'una voltacilacquabew
Sopra delvinyche maiper
Quanto di buon si ¢; cheisei vplen
Lavar i pauni,jlcorpo anche risciacgisy
E divien lacqua si feremexegiallay
Gh' i pesci vengon sutti quanti a gall

—

Be GBB PRR E S3SsS> Pe

=

ee en EC!. Be SE we 2 are



SETTIMO CANTARE 333
BiwbineeDAN ZA XXL Hoi

tutte a lui son note, Lotanto si conduce fra le ruote,
ne per muotar bene sl Romano;. Che fan girando macinare il grano
ei corpo, confie fa le gore, Ben fen' avvede, e già mette a entrata
Anna/pa cof piede,e con la mano, Di macinarsi,e fare una fRiacciata,

wide sentendosi inuigorito risoluette di ritornare al campo; e così (enz' altro
si mefie in viaggio, ma fendosi infangato, volle lavare i calzoni in una go-
 4, €vicalcd dentro, e se bene egli fapeva nuotare, e s' affaticava per ulcir dell”
'Acqua, tuttavia conobbe, che portava pericolo d' entrar sotto le ruote del muli-
no, e restarvi infranto, se non gli accadeva quello, che sentiremo appresso.
» VINO che fmaglia, Vino potente, e generolo. Si dice /magliare, perché il vi-
 no nel mescersi nel bicchiere lascia nella superficie una stummia, che fa certe cose
q come maglic » le quali il vino gencroso rode, e consuma subito; e questo disfac
que ie si dice /magliare, e quando non le disfa, e segno, che ha poco spi-
ya ito. Edi quiicicchi hanno un detto: Laloccom! io, o vommene ? ed intendono
¢0si didomandar al compagno alluminato, il quale ha mesciuto nel bicchiere, se
quella fummia se ne va, o si trattiene, ed in confeguenza s'il vino e buono, o
tattivo.,, Lasca Nov, 4. fecero uno scotto regio con quel vino, che /magliava,
AINGAZZVLLITO. Forse meglio ingazzurlito. Vuol dir rinuigorito, rin-
» o rallegrato di quella allegrezza, che mette addosso il buon vino.
i dice entrar in xurlo, o in zurro, corrottamente da rvzzo, e questo dal Latino

\&

ruere,

ANNOCENTEMENTE  e partito, Dice innocentemente, perché in vero Pa-
tide non haveva errato a partirsi dal campo, poiché n' era stato cavato da colo-
TO, chelo portavano via infermo, come s'è detto sopra C. 3. st. 25.

YN colpettino. Un' altra volta, Un' altro poco. 1 Franzefi similmente dicono
per esempio; boire encore un cowp. Bere un' altra volta. Provarsi a bere un' altro
poco. Ad è traslato dal provarsi in giostra.

RIGLIAR ? ambio, Andarsene. Voce corrotta da ambulo latino, che vuol dir
andare, o pur vien da amb io specie d' andatura di cavallo, con altro nome detto
Portame., pecché per esprimere andavfene diciamo Pigliare il portante,

SENZA lume, ne luce, Afiatto albuio. Senza lume terreno, e senza splen-
dor celeite..

SPFLEZZ.A. Va via furiosamentc. Parmi che possa venire da spulare il gra-
no, che i) vento furiosamente porta via la pula, cioè i gusci del grano; o da

pigliare il puleggio detto sopra C. 1. st. 80. *
 “OTA, Terra inguppata nell' acqua, e ridotta quasi liquida. Così appresso
i Franzefi moire \& il Latino dus, madidus, e quel che noi diremmo mole,
pA MEMMA, o melma. Quella terra, che nel fondo:de' fiumi, soffi, laghi, ¢,
ae » tidotta liquida, che la diciamo anche bellerta per me/metta Latino Limu
  Verifimilmente dal Greco Adigma, che vuol dire miffura.
: a. Meflo in corpo. Detto plebeo. Vedi sopra la voce Gubbia-
mC, 1, st 36.
DA'mexxo in gilt sbracciato. Così dice per scherzo, sapendo bene che sbracciato
Significa,quand' upo tirando la manica in fu fino al gomito,lascia ignuda quella, i
' parte

ea

SSBB SE”



,

334

arte del braccio, e non quand' uno si cava i calzoni', co!

aride, il che si dice sbracato; ma-l'Autore si serve della voce
tendere spogliato; enon è-vero che habbia a dire sbracato
corretto, non solo perché l'originale di mano dell'Autore, che
ed in un suo primo sbozzo dice sbracciato, ma anche perché sed
mezzo ix giit ' intenderebbe che ei si fusse tirato fu i calzoni fino a
enon che se gli fusse affatto cavati, come era necessario, che,
voleva lavargli. ¢
SCLAGV-ATT ARE, Dimenare un panno, o altro simile nell' acqua.
GORA. Vuol dire un canale d' acqua, che corre, e propriamente s*
quella fofia, per la quale si conduce |*acqua a i mulini per macinare
ali fofie, o gore si fanno a quei mulini, che sono in fa' rivi,o p
quali e scacfita d' acqua, non essendo necessaric a i fiumi reali, nei:
servi abbondanza d' acqua, basta un sostegno, o steccaia ( che no
scaia ) che volti l'acqua al mulino, e serva per Cosa, chet ae
alla quale si raguna tutta l'acqua, che porta la gora. Gli antichi finiv
voci in Ora non folumeute quelle, che aveano simili udi se col Lat, co
le quattro tempora, come ancor oggi diciamo; mia anche le Bergora,
le Campora; E simili. Onde il Sannazzaro nelie Ecloghe della sua A
se licenza di dire Pratora per Prats: Gc. Si pote dunque dare benissimo:
quef' acque così ragunate essi chiamaflero Lacora dal Lat. Jacus, € poi
a staccare la voce, e dirsi La gora. Da i Jatini si trova esser tali, o si
d' acqua chiamati Euripi, e dVili, ma credo che fussero iperboliche a
come si può dedurre da Cic, 2. de legibus, dove dice; Dattus ag
Wilos, Enriposque vocant quis non irriferit ? E veramente e cosa da rid
Euripus è nno stretto di Mare,ove è il fluflo, e reflufio; Ed il Nilo e de'
ri fiumi del Mondo; E queste son fofle semplici, e laghetti, che gli
mani fecero correre infino di vino in occasione di feste; e da ciò piglio
to, che gli adulatori per piacere a' Signori, le chiamassero Wii, ed

DANDO ile... aleva, Cioeaizando ile....ed abbafiando il ca:

FECE un capitombolo, Rivolto il corpo sul capo sottosopra; fece un
capo, rivoltandosi sottosopra. Vedi C. 6. stan. 84. '

e4 GALLA, Nella superficie dell' acqua. Dai verbo galleggiare.,
origine da galle, che sono quelle leggierissime palle, che nascono dalle
donde /eggieri, com' una galla,

JL Romano, Fu uno Stufaiolo, che insegnava nuotare alla gioventh Fiorentit
MOLTO annajpa. Annaspare vuol dir mettere il filato sopr' all' alpo”
durre i) filo in matafle, edipanare, Lat, geomerare,afhne d' adattarlo
re, dal Greco ana/pan, che vale retrabere, revellere, E da questo qu
perde molto tempo a far qualche operazione,¢ non conchiude cofadi bi

ciamo annaspare. Qui vuol dire, che egli muoveva i piedi,.¢ le mani,

ve le maui colui che annaspa;¢ si puo anche intendere che armeggiava |

naspava molto, e conchiudeva poco. eal
GLA mette aentrata di far una stiacciata, Già tien per certo d' havere@

infranto dalle ruote del mulino, I caflieri, ed ogn' altro che tenga libri

%

Tg eee ee ae oe ee

es ES ee


SETTIMO CANTARE: 335
: trata,¢ uscita,mette a entrata, quando ha ricevuto il denaro; € da questo noi

mo Tien per certo, o ha già per ricevuta quella tal cosa.
Stearn e a ccsatabuperr
» che il meschin gra si prefume Ognun si tenga pure il suo parere;
andar a far la cena alle scanae 's O quelle, o altre, a me non fa farina,
¢ una porta,e in chiaro (ume Bastini per adesso di sapere,
ise 2 iar conocchie, Che queste non son bestie da dozrina;
Che le Naiadi Ninfe di quel fume, E, 8' ella non m' e feata data a bere,
—— Coronate di giunchi,¢ di Elle son Fate c' han virtù divina,
—— Corrono ad aintarlo infin c a viva E che sia il vero, fede ve ne faccia
La dove il di riluce) in falnoarrina, Li Garani scampato dalla fliaccia.
STANZA XXIII. STANZA XXV.
— Bvede all ombra di falcigne frasche 11 quale così molle, e sbraculato
io Fra le più brave musiche acquavle MU cadavero par di Mona becca,
\&s == Parte di loro al suon di bergama/che, Ch' essendo stato allor difatterrata,
we » ¢fefhe ragliar le capriole, Hlabbia fatto alla morte una cilecca;

ei Chitien che queste Ninfe fien le lasche
[Chile firene, ed altri le cagznole;
4  Lonon fo chi di lor dia pin nel buono,
i Ble lascio nel grado, ch' elle sono,

is Ah % $

jd Male Fate, che specie son di pesce,

ys Edbimoilcorpo aftar nell'acqua avvezzo
st Piiche 2 a bagnate a lor rincresce,

—, cos: fradicio merz0;

Si [quote,¢ trema si,ch' io ho fopparo
Per San Giovanni il carro della Zecca,
Ementr' ei ff debatee, e il capo scroiia,
UI pavimento,¢ i circofhanti ammolla,

TANZA XXxVI.

Percio lo spoglian; ma perché riesce,

Queido un vuol far più prespo,fhar un pexro,
Pertrattenerlo(mentr'hor questa,hor quella
L) asciuga) una conto queita novella,

Paride stava con timor d' affogare, fu foccorso da alcune Ninfe', le

messero a spogliarlo, ed intanto una di loro contd la novella, che vedremo

of

;:

3 lo cavarono dell' acqua, e lo conduffero alle loro stanze, dove dette Ninfe
af

MESCHINO, lafelice; Povero. E voce, che denota commiserazione.
ge ANDAR a far la cena de' ranocchi, Cioè affogare, annegare, e così diven-

tar cibo:de' ranocchi.

yi CONOCCH/z, Pennecchi in falla rocca, che sono quei rinvolti di lino, o
è Jana, o altra materia simile, che le donne per filarla accomodano in fulla'rocca
to da esse usato per filare; Voce corrotta da cannocchie, secondo il Fer-

rati, perché-le rocche per lo più sono di canna; Il Voflio la fa venire dal Lar.

, Golus; quaGi Rorpiata da colwcula.
i

~DRAPPL, Cioè quei drappi da donna, che dicemmo sopra C. 6. stan. 9.
5 ~~ CAMPEGGIAR conocchie, Sappotto che le muca di quelle stanze fussono bian-

che,ogni cosa di

alfivoglia colore vi si discerne ben sopra, e però ( servendoti

; del verbo pittoresco campeggiare ) intende; si distinguevano sopr'a quel bianco i

,  drappi,

¢ fuentolavano, e le rocche appiccate alle muraglie,

GIFNCO, Pianta, o virgulto noto, che nasce vicino all' acque, ed in luoghi
| Umidi, e padulosi, e non fa foglie, ne tronchi;ma salti,come paglia,lisci, (en-
E 2a nodi, se non uno ia vette, dove nalce il feme. E per questo habbiamo un,

pro-



336 | MALMANTILE * o
proverbio, che dice: Cercar if nodo-in sul ginnco; Lat, sodum in scpe ghree
significa cercar le difficalta, dove elle non sono. V2 POT en

PANNOCCHIE, Spighe, che si producono dalle canne, dalla ei

panico, \&c, dal Latino Panicu/a, voce usata 'da Plinio, ove tratta
Carerum gracilsas nodis dispintta leui faftigio renmatur in Caxmina,
la coma,; HeeO SN

SALCIGNE frafehe, Frondi di falcio albero noto, che nasce; € vien più
roso in luoghi padulosi, Lat. frondes /aligna. a

eAL suon di bergamasche. Chiamiamo Bergamasca' un*ballo compone' rato
di salti, e capriole, e-però dice guinre, e fespe rrgliar le-capriale; vest

CHZZVOLE, Sono certi animaletti neri, che vivodo sae:
ti pancia, e coda, e col tempo diventano ranocchie, € met 1¢ gambe,
cascado loro la coda,mutano colore di nero in verde macchiatoje
mo la mestola da muratori; Lac, trudla, e che It Abate Baldo da Vebi
zionario sopra Vitruvio dice al suo paefe chiamarsi Cacchiara,

LE lascio nel grado ch elle sono, Sieno chi elle si voglionojio non do'
nome, che un' altro; perché ciò zon fa fartmarcio' non m' importa; eT
proposito mio. E qui l'Autore mostra d' haver notizia delle diverse
Gentili circa alle Ninfe, le quali tutti concordano esser Figliuole
¢ conchiudono che le piii fussero Deita aquatiche; le quali Deita noi”

retiamo, che ficno diversi effetti, che produce Pumidita, E che parte
infe sieno de i prati, parte de' boschi, parte de i monti, e con diver
Nereidi, Napee, Oreadi, ec,: '

NON son bestre da dozzina, Non son bestie ordinarie, e da farne poca Mima:
Diciamo cosa da dozzina, o dozzinale, quella, che e lontana dalla perfe
che @ lavorata con poca diligenza.

S' ELLA non m' e frata data a bere, S ella non m' è stata data a credere] *

FATE. Vedi sopra C, 4. stan. 54. '

STIACCTA, Si dice quella trappola, che si tende con le laftre a i topi,ed ag
uccelli, così detta, perché nel cadere addosso all' animale, lo stiaccia.
SBRACVLATO. Senza brache, e senza calzoni. 2
C-ADAVERO di Mona Checca, Si suole in Firenze nel giorno della commen
razione di tutti i morti,ne i fotterranei della Bafilica di S. Lorenzo, che fond il
fepoltuario, esporre uno scheletro di morto con veli in tea', ed altri:
menti, e questo da i ragazzi è detto Atona Checca; cioè Madonna Fr.
questo nome poi comuncmente s' usa per esprimere uno sbattuto, ed -afflitto dale
la fame, dal freddo, e da altro stento. Aristofane portato in Latino dice: AF
bil a Charephonte disser. ww
FARE nna cilecca, o scilecca, Far una burla; cioè finger di voler 1
fa, e poi non la fare; Sicché vuol dire: habbia finto d' esser morto', \&
sia stato vero. Habbia gabbato la morte. Diciamo anche pare uit m b
rato. Ll Bini nel secondo Capitolo dell' orto dice: 2 og ae
' Ho una vaca, ma ell! ha una pecca 1 EY
D! un certo suo turacciol benedetto,
C' ogni volta mi fa qualche cilecca,

REF FS ee 8a Ea ee

2.

Er

eee ae oe ee a oe



ee
1) TP Sr TIMOsCuNTUReE 337
20.

So fips. Qui bao stesso significato, ché mb difetadé detto sopra. 1. 0.
34. e c. 6, stan. 61. er altro havere ffoppato uno, vuol dire H2-
negli orecchi, ec. per esempio. Tu mi hai fatto il servizio tanto tardi, che
ho havuto più bisogno, e però io ¢* be fropparo.
lella Zecca, 1) giorno di S. Giovanbatista e la maggior solennità, che
ri in Firenze per esser del Santo Avvocatu, € Protettore della Città, ed
oP ome agifteati di Firenze, e tutte le Terre, e Castella fubordina-
; inio fanno la citimonia dell' offerta al Tempio dedicate al detto Santo,
fra gli altri il Magistrato della Zecca offerisce un gran Carro trionfale in figu-
piramidale alto circa 20. braccia, e nella fommita di esso Carro è un' huomo
tutto coperto di peli, legato.con func a un palo di ferro alto circa un brac-
cio e mezzo, che formando in cima un mezzo circolo gli fascia lo flomaco,dove
f detto huomo,acciò non cachi, il quale rappresenta San Giovanni nel
to. E perché tal:Carro nell essere strascicato brandisce, e squote, però
che e nella cima del Carros' agica grandemente ancor' egli; Ed il Poeca
huomo intende dicendo, che Paride si squote più del Carro della Zecca,
Colui, che è sopra detto Carro*
CB yO ineresce. Vuol dir venire a noia, o a fastidio, ed 8 il Latino
Bocce, gior.5. Nov. 6. lo fari s} 5 che lavedrai tanto, che ella ti increscerd.
gnilica haver dispiacere,c' una cosa sia fatta,, o non fatta. Bocce. Nov. detta.
Ma di cidyche facto, glincrebbe, Significa compassionare uno, come nel pre-
ee eis questo C, stan. 50. Significa'ancora haver dispiacere inten-
dendosi esserinelle Fate maggiore la compassione, che havevano di Paride per ve-
derlo così mal condotto, che non era il disgusto d' esser bagnate; E sono questi
due significati tanto prossimi; che spesso col solo verbo rincrescere s* esprime.»
Punoe Valtro, come fegae qui, e nel Petr. Son. 44.
dP ote Onde il lasciare el' 4/pettar m' incresce,
Rs a intendere mi pefa, mi dispiace il lasciare, e mi viene a noia l'al-
pettare, Li Persiani nella lettera al sig.\ Principe D.Lor. disse:
Ml mio bifegno ho gra detto a parecehi
(art 's) EB ciascun se ne duole, e gli rincresce
4Omexo. Coml', ¢, stretta, e con una fola,z, che fa alpro ( per-
ché:con Pelarga., e con due zete, che fanno dolce, secondo l'opinione del dot-
4. Carlo Dati, vuol dire meta ) significa bagnato assai; e la voce fradi.
¢io che wuol dire corrotto, qui signitica inzuppato d' acqua. La voce mezo vuol
dire una:cosa tenera per esser troppo matura, come farebbe una mela o pera, ec.
vedi sopra C, 3; stan. $3.0 una cosa intenerita per haver inzuppato molto umi-
do una spugna intinta nell' acqua, e questo e il senso de! presente
hiogo:.. Adezo \& dal Lat, mitis per maturo; ed € il contrario di acerbo, che così
chiamiamo la:frutca'n6 per anco matura.V olgarizzamento antico di Palladio,nel
mese di Gennaio; 'tit, 15. Serbanfi le forbe:, se si colgano dure; ec. e ivi comin-
)  Glanfigimmerzare. Ul Lat, dice: wbi mitescere caperint, —

2
.
"

= BBtesei E

3
a
7

=

xe x vy STAN.

o

'
i

338 MALMANTILE ~~

STANZA XXVIL
Fro un tratto una dama,eun Cavaliero
Moglie,e Marito in buono,e ricco sPato,
Che fatti vecchi, contr' ognipenfiero 5
Dopo a' haver qualche anno 'higare
La grinza pelle con il cimitero
Conuenne loro al fin perdere sl piato'y
E senz' appello haver a far proposito
Di dar per ficw tal' offa in deposito,
STANZA XXVIIL
Lusciaron due Figliuoli è più compliti
Che'l mondahavesse mai sule /ue sceney
Perch' essi havevan tutti srequifiti
Donati aungalat'hnomo,e un bom dabbene;
Aggiunto che di solds eran.gremui,
(Che questo infomac quel che vale,e tiene)
Stavan d' accordo, in pace,ed inamore,
Er eran pane ye cacio, anima,e core,

gidi non usa più.
PIATO ye piatire.

Perciocche il nostro
gis habent vigorem,

v

potevano piatire per La lor

La Fata principié a contare la novella (la quale ¢,tolta'da:lo Cunto deli
ti gior. 4. Cunto 9, e gior. 5. Cunto 9, ) e dice: Furon già unadama, e ut
valicro marito, e moglie, i quali venendo a morte lasciarono due Figlit my
costumati, e ricchi, i quali s' amavano grandemente l'un l'altro. Qui
fa una digressione, e considera, che questo modo di trattacfi fra i Si

Lite, o litigare d' avanti a' tribunali, detto dal Latybar+
baro placitum per lite,¢ placitare;la qual voce ritengono bella e intera i Veneaia-
ni, Placitum è il decreto, sentenza del giudice, o Magifttato, e quel che i Fran-
z¢si dicono 4rreffo secondo il Budeo da arefeein, che in Greco vuol dire placere.
Ne' Senatuscon(ulti, ovvero Decreti, e Sentenze. del Senato di Roma ulayand
questa formula: Sexatui placere \&c. come si ricava da Cicerone Filippica 3. € 5+
Nell' Ordinanze Regie in Francia si legge sempre in fine: Car tel est nostre plaifin
iacere è tale. EB nella legge si dice; che Principam le

enne poi da' Latini bafii a tirarsi questa parola'a se
il proceffo della lite medesima., si come anche éudiciwm significa Ja' sentenxa jt
4a lite medesima, che fa nascere la sentenza. Piatire lo Spagnuolo dice;
Franzese plaider; wutti dall' iftessa fonte Latina, Il Doni.nel suo Cancelli
Sempre ne piati la rovina va innanzi, e chi piatisce ha quant' ei vysle il
Ed il Varchi Sc, Fior, lib, 14. Erano affegnate le canfe delle pouere
ertd.. E poco appresso, dice; Perché
care quel piato al terxo posefore. Ed in quest ultimi versi della presente:
27, dice metaforicamente, che.a costoro già fatti vecchi dopo haver fatta

apt eh g
Ce fare i e
ca è nece

E fr as de cei osre

ll contrario costor di chi io favelle
1 quai di cortesia furon due spe

E trattavan ciascun da buon Prac
5S' haurebbon

il
pene chee
bifegnsos. mi

SS = 61 PERERA EES eeeeeeee

rar lungo tempo la loro carne a i (epolcri., conuenne morire, e fart

Il proverbio piatire i cimiseri vuol dire Esser d' eta cadente, che Luciano portal
in Latino dice: lterum pedem fepulero, o vero in cymba Charontis habere; C

noi pure diciamo; Havere il più fu la bara o vero il più nella foffa, —

GA


SETTIMO'CAN'TPARE. 339
GALANT" nemo yed huomo dabbene. Si posion dir finonimi; ma Mrettamente
galant' buomo vuol dire huomo di garbo, e come dicono i Franzeli, ones" uomo;
oltre acid amorevole, ed alla mano, ed huomo dabbene vuol dire huomo di co-
»huomo d' anima, e che fa opere buone. Spagn, hombre de bien. L' uno
l'altro comprendono i Greci colla fola parola Caoscagathos. Calos ignitica.,
adig eh buono, da bene. “y
GR. 4. Ripieni. Bil latino Spifus.. Denfus... E qui vuol dire havevano

pieno di frutti, un luogo pieno di mosche, o simili; perché tal voce 4 do-
usare in quelle occasioni, nelle quali cade la similitudine del proprio di
Hy « Greto-vuol dire terreno'ghiaioso, e pieno di faili, come sogliono ri-
S tive de i nostri tinmi, scolata che e l'acqua piovana, quali rive però
og — chi )Greto, come greto d'arno, greto di mugnone, ec, Ora Grero addict.

fies no di danari; se bene e detto improprio,perché gremito s' intende un'

dice i) Vocabolario della Crusca, /o diciamo in significato di speo; forse daila
titudine [pela de' fussi de' greti;e diciamo anche in questo significats Gremito. Quan.
ame,inclinerei a credere, che Gremity dal dirli propriamente degli alberi,quan-
'ono 'picni di fiors, o  carichi di frutta, venitic da Greminm perciocche if
quella parte, che suole empicrsi di tali cose. Gli antichi Volgarizza-
che i Latins dissero diteus eth traduilero greto; laonde potrebbe ad alcuno
 questa ia fatta da quella. Seneca epilt, 15; Liles repersi en littore cal-
oie ao aouisiem amet delettant c Panciulli \& sthcesanc in cose di
piccol pregio, si come.tono pictre, che l'huomo truova nel viaggio, € uel gresv
del mare, e ne' fiumi. Palladio nc] Gennaio tit. 14. favellando della lattuga.
Candida fieri putancur, si fiuminis arena, vel litoris frequenter [pargatur in medias,
rm E posiono diventare bianche se entra loro, e intra le loro foglie spefle volte si
spatga rena del fiume, o del greto, Qnde a dire gremito di soldi s' invenderebbe,
= hi sopra il vestito,o sopr' alia persona sparfo gran numero di soldi,
i somMeenemito di mo/che 8 intende haver molce mosche addosso, € non nella tafea,
i" Oinicafla.; Tuttavia, sc bene. improprio, è alle volte usato, come qui,
}, ESSER pane, e cacio, anima,e cuore, Andar' uniti, e d' accordo in ogni ope-
A razione, Bene conmeniunt, o in una fede morantur,.
wht iy oterra al Sole, Se hanno mafierizie, o poderi; per esprimere,
6 uno che. i peep roba diciamo: // tale ha quattro cenct, o se ha beni stabili
in terreni: Egli ha della terra al Sole,
» SLAMO di si perfida cottoia, Siamo così iniqui, e di mal animo, Quei legu.
Mi, che per moito che si tengano al fuoco non si quocono, ne inteaeriscono
Mai, si dicono di cattiva cotteia  € però con dire huomo di catriva cottoia, §? in-
tende di genio maligno,¢ difficile a persuadersi al bene. Gr, ateramon,
~ ESSER al dumcine, Vuol dire eller in cftremo di vita; € vicue dall' uso, che è
nello Spedale di S, Maria Nuova di mettere uo piccolo lume a un Crocititio al
letto di coloro, che sono agonizzanti. Si dice an.ora; cfier alla candela,
) NON gli fovverrebbon d' un tnpine, Non gli darebbono un minimo aiuto. Sov.
yenire neutro vuol dir covalent. Non mi (ovviene, quando fu questlo. Non mi
eieanam fugquesto. Lat, mentem /ubire in mentem venire, fuccurrere, Fr,
(e fovvenir, 2

Vv2 " eHOz.

oe

Seba thasa

STEED

et

=—s=

o



340 MALMANTILE) |

1
HOZZORECC HI. Huomo scelleratoyed infame: EB questo,perché
fattori, che per la tenera eta sono elenti dalla ordinari: i
fiizia contraflegnati, come dicemmo sopra C.2. stan. 3. e C. 6. stan, $4. e fra
gli altri contrafiegni uno è il mozzar loro una parte degli arecchi,. \. mye
. LOKT AR acqua per orecchi, Fare a uno watt i servizzi i

HAVBEBBON volute indovinare. Questo termine esprime la
che uno ha in servir l'altro, e compiacerli in tutto Veen.
STANZA XAXL sT ZA XXXL
Essendo un giorno infiemo a um conuito, E tutti quei che feggon quiviamensa
uad'appanto agurzato hinoil mulino, Lfervi, i circostanti,ed f
4 mangian con buonissima appetite, i
4Von focome il maggtore dette Nardino
Nell' affettar il pan taglioffi um dito,
51 ch' egli infanguind st tovagtioline
E parwegli si bello a quel mo intrifo,

Ch ei si pose 4 guardarlo fifo fife. Lsangue:
STANZA XxXIl, STANZA XXXIV,
E resta a seder li tutsa infenfato, Che gli par di veder ymentre in

CL! ei par di legno anch'er come la fedia, i for ver
Luo is ( tanto nel wifo è dilavato ) qualche Dea di Ciele
Con la tovaglia i simili in commedia Composta colafsis di rose, ¢, Z
E mirando quel panno infanguinato E si gli piace:,.¢ tanta ght
Hor mai tant! allegria mutain tragedia, Che finalmente mentre ch'
AMeatre nel pin bel suon delle scodelle Una moglie d' un tal componimene
Si vede — riposar le mascelle, 'Non fad de i fuci di mai più contente,
Edendo gli faddetti giovani a un conuito, Nardino, che era.il maggiore,afiet-
tando il pane, si tagliò un dito, ed infanguinò il tovagliolino, e nel mirae quel
bel rosso in sul bianco, s' innamorò in maniera, che si propose di non haver mai
a restar consolato, s'ei non piglava una moglie compotta di quel colore del
tovagliolino insanguinato.

CONMITO. Desinare, o cena splendida, Dal latino Consivixm, o c
da Conuitare nel senso che gli Spagnuoli pigliano il loro. Combidar, per fuaicare,
¢ nel quale il prefe il Boccaccio, che difie, Commie « mangiare:,. E, Conuirati alt
ravole, ii B
AGVZ ATO il muline.. All' ordine con la fame per mangiare.. Così eratta lt
similitudine dal mulino; dicesi Adacinarea due palmenté, c10e mulini; di ¢
preftezza, o voracita maftica da amendue i lath aun. tranoy
itanza 22. > e ORE ong B
APPETITO. Vuol dir. appetenza, e desiderio im generaley»ma
detto assolutamente,¢ sen2? aggiunta,vuol dir Fame o voglia 4 0) *
giare. Vedi sopra C. 4. tt. 8, e mal che viene in bocca allagalina, > 2 sb Oi
TOV-AG LIOLINO. Quasi piccola tovaglia. Quel pezzo di panne line
tiene avanti,quando si mangia etiendo a mensa, Boccaccio disse
lo dichiamo anche falwietta dalla voce Spagnuola. Servillera, perché serve mol
al minificro, e al scruizio della tavola, hae Be = HERR,

og

\&eFRESELLCFE - Pee stsiz=

Tr

gg E 2 se FZ.

=

2 ge (Ha


SETTIMO CANTARE: 34

INT RISO » La poluere } o altra materia simile stemperata con liquore, come
e:farii Wa si dice:imtrifo, e intridere Ma significa ancora imbrat-

| tao, [porcato, ec. come significa in questo |
PISO fifo. 'Senza batter' oechio, wha greta attenzione: dntentis, inconni-

eculis. 1 Greci dicono in una parola e4/cardamytti, che @ lo stesso che:

s o irca'y

8 28S Cash vedessio fifo,
BSB NG Come Amor dolcemente gli governa
Seles Sol un giorno da presso,

2m 8 Senza volger giamai rota superna,'

Bp%b-on pers | We pensassi a' altrui, ne di me Stess,

 Purcdiwg sci El batter eli occhi miei non fusse [peffo.

- DILAV. idito. Smorto. Si dice dilavato ogni colore, che non

' “ATO; Impallidi

-ariva alla perfezione della sua essenza: come rosso dé/avato si dice un color roto,
'che sia più sbiancato,e pity'chiaro del vero roflo. Latino difurus.

PPO far con la tovaglia i simili in commedia, Intende ch' egli e bianco appunto
come? latovaglias Latino' non oxnm: fic ouo simile, I due simili ® un fuggetto di

', come quello de Menechmi di Plauto, a molti vi hanno scherzato,per-
 secondo d' intrecci.
A, Specie di tela lina fatta a un' opera, che si chiama renfa, detta così

dalla Citta'di Rein Francia. Così 'Perpignano sorta di panno dalla Città della.
Navarra di questo nome. e4razzi dalla Città d' Arras in Fiandra':¢ Dxagio al

) Boccaccio si diceva un panno, che veniva di Dovay Città di-Fiandra.,
che Gio; Villani secondo l'uso de' suoi tempi, chiama Doagio. Latino Duacum.
Baldacchino j deappo di Levante; da Habbillonia, che i Levantini chiamano 24-
g4ady inoftrivr antichi Ha/dacco, Gio: Villani |. 7. E'messo fuori delta Città, sopra
(a sua persona umricco palio di Baldacchini di seta ed' oro,

LENZA,olenfa, Lat, tinea, filum piscatorinm, detta così quasi dal Latino
linea, Quella'cordicella fatta di crini di cavallo, o di feta cruda, con la quale
filegnitiamo da petcare:» Franco Sacchetti Nov. 163. Egli haves preso l'allumi-
white ale lente acscandole con 200, Fiorini a' ord, Lalca Nov. 166. Fau'nn pescatore

i puceote i e/cando con lami, e con lenze,

* Wid oer. Acetta. Pezo di tela ia larghezza det suo essere, € lungheza

4d libitum; come un telo di lenzuolo, 6 di paramento sdrucito in tutta la lun-
ta di esso lenzuolo, o paramento, Diciamo ref da pane quella rovaglietra,

O.triscia'di panno lino, con la quale si cuopre il pane in fu l'afle', Qui intende

iktovaglinolo. Te/econ I", e slargo usato da alcuni in Poesia, vuol dire il dar-

do. Lat. telam..

“'GLfvaapelo, Glivaa genio. Se gli confa: e secondo il suo gusto; él) op-

posto:d” contrappelo detto sopra C. 6, itan. r. }

"2 a ete mine XKXV,. 2
da figura nel pensiero, E come chet la vegga daddovero
ieee paket, eas Divoto se le inchina, € le favelld,
Co' suci capelli-d' oro, e t* occhio nero, E le promette,  egli haurd moneta,

“Che più ne men la matturing frella. Di pagarie la fiera al? Lmproneca,
: STAN.

ig”



34%
E vuol mandarle il cuore in un pasticcio,
Perch' sila se ne serva a colarione;
E gli s' interna si coral capriccio
E tanto (ene va ip contemplarione s:

Nardino s' immagina, e si compone nel pensigro una'
parendogli d' haverla veramente avanti a gliocchi., le parla, e se
le dona il cuore; ed in questa guisa s' inaamora ardentemente d' una b
maginaria. sie eye Han HOR ee IPE

PRES C.A. Trattandosi d' huomo s' intende Vino dipoca eta; ed h
donna freschi s' intende fani, gagliardj 5. di buona cera,quantunque
grave. Virg. cruda deo, viridi/que fenetius. Frefeo Secondo il Ferrari

ney

Origine dal Latino vire/cens. La marturina Hella. Virg. Qualisie
fer undis,; to

PAG ARLE la fiera all' Improneta, Pagarle un regalo all
giorno di S, Luca 18, d' Oucbre all' Impruncta', la quale'è una Chiesa
tana da Firenze, celebre,¢ frequentata per un' immagine miracolosa
stima vergine, che e quivi; la quale in tempo di calamita, e di
portata folennemente a Firenze; e nella venuta di questa Immagine f
una Lauda in una Raccolta antica di Laude spiriwali,

E SE gli imerna si cotal capriccio, Gli si ficca nel cervello, o-gli
mente questo capriccio, fantasia, opinione. Vedi sopra C, 15M. a4
S' INMAMORA come un miccio., S' innamora come un' asino
mente, perché l'asino e ostinatissimo,e capone. 5 itt

STANZA XXXVII STANZA X
Cos} a credenza infacca nel frngnuolo,
Ma da un catoegliharagion davidere,
Che s'egli -veryc'eAmor vuol esser solo,
Rivale non e qui con chi contendere,
Ma Brunettoilfratel, che n'hagra duolo,
Poich'ilsuo male alcun non pus 'copredere.
Tien per la prima un' ottima ricetta 5
'Di rimandarlo a casa in una feggetta,
STANZA da
Ei che vagheggia fott' alle lenzuola Replica quello, e feccafi la.gola:
i gentil volto,e le dorate chiome, Lo fruga, tira, e chiamalo per nome
Ne anche gli risponde una parola, Ed ei pianta una vignaye nulla
Won che gli voglia dir ne chene come, Pur tanto Caltrofa,ch' ei si rifente,
Così Nardino sianamora ardentemente senza saper di chi. Brunetto fu)
tello lo fece portare a casa, dove lo meflero in sul letto, e vennero,
Speziali a vifitarlo, ma non conoscevano ne meno essi il di lui male; onde
netto si mefie a pregarlo, che gli dicesse quel che egli havea; e Nardino!
la sua contemplazione non rispondeva; pure alla fine vinto da tanti
fratello parlo nella maniera, che vedremo nell' Octave seguenti.
eA CREDENZ A, Vuol dire, quando si compra qualche mercanzia,



See aS SELB RTE SE \&

ASE

=~

SETTIMO'CANTARE. 343
si sborsa il danaro allora, as garlo in altro tempo. Ma qui vuol
dire feniza proposito, o senza son *

mento. fH Varchi nel Cap. dell' vova fede.
o) Chiba fquadrato ben la quintefenza,
“. 9) Dite ch' ella non ha color neffuno-,
o 5 Bebe quel giallo v e posto a credenza.
pTr@ng) Rir67." > ° '
Contro di noi bravavano a credenza.
Questa maniera è corrispondente al graris de' Latini. Perfecuti funt me gratis, La
version Greca dice dorean; in dono, cioè di lor cortesia, senza che io il meritassi.
INS ACCA nel frugnuolo: S' innamora; Se bene entrar nel frugnolo vuol dire
anche entrar' in collera. Frugnuolo è  Janterna; con la quale si va di notte
a cacciaagli vecelli,ed'a pescare; ed è parola corrotta da fornuolo, perché tal
trnaessendo simile alla bocca d'un forno, così e chiamata.
EGLI ha ragion da vendere Gli avanza della ragione. Ha grandissima ragione.
\ SEGGETT-.A; Seggiola portatile con due stanghe. Vedi sopra C. 1. st. 48.
 GOMITO, La congiuntura del braccio dalla parte di fuori, dove si piega a.
mezzo il braccio,, dal Latino cubirs.

VAGHEGGIA, Fa all'amore, amoreggia, con desiderio d' avere la cosa amata,
Yagguarda., come difie il Buti cittadino, e Lettore Pifano nella sua lettura sopra
Dante, Vedi sotto C. ro. st. 44. Dan, Purg. C. 16.
aT A Esce di mano a [ui, che la vaghegvia,

\ S.* Prima che sia a guisa di fanciula.
Enel Parad) 10. | Eli comincia a vagheggiar nell' arte Di quel Macfire.
Fazio degli Vberti nel Dictamundi; canto 143.
ale Efe @' udirlo proprio ti vaghegsi.
(cioè feivagho; ardentemente desideri ) E canto 144.
we Bios va pur, che quanto priego', e chieggio
Al fommo bene, e fol, che tosto sia

2 Vaeay Wel paefe, ch' i bramo, e ch* i vagheggio.
cioè desidero, ne son vago; col quale io fo all' amore; ea cui mi pare un' ora,
mille anni di ritornare » Vagheggiare il Ferrari deduce dal Latino vistare, frequen-
ter ae, citaa ptoposico i versi di Lucrezio lib, 1. che descrivono Marte, che

Venere. uae
—— in gremium qui fepe tuum se
'yO Reycit aterno devinitus vulnere 'amoris,
Arque ita suspiciens tereti cervice reposta,
Pascit amore avides inbians in te Dea vifus,
O:pure view da Vago, avido; perché chi e avido di godere la cosa amata, va at-
torno percercarla, e fi'rigira come farfalla intorno al lume della bellezza di
quella, Dante in un suo Sonetto.
. To son se vago dela bella luce
Degli occhi traditor, che m' anno occifo,
Che la dov' io son morto, e son derifo,
La gran vagherza pur mi riconduce.
NE che, ne come. Intendi, che non solo non gli volle dire ne il male, ne la,
Caul@ di efig, ma ne meno volle parlare, SEC-

'a

344

MA LMDANMTILE o

SECC-ASI la gola., Se glia icequanie fauci per. isemnan eb Li strc d
PIANT A una vigna.,. Non bada, 0-non attende.a, ice «Che noi

diciamo anche far orecchte di mercante » che è:'

titi, che lif

propongono, attento solo al, suo vantaggio., irc 57 'Ear conte che

L Imperatore o far conto, che uno cants.. Per il conteario schi parla
non bada, o non vuol badare, dicesi Predicare al defer

C. 10, st,

bere.
Studio iaktabat inani,

46. Jn Latino pyre.trovanGi molt: detti in questo
Vento loqui., Surdo canere, Frufira 3 velin wannm cantare, cum pisce,
Aliam rem agere » Oc, Virg. Ech 2, tbi bec sanouiteeli

@ gente,
10, Predicare:
significato, come:
Lire

z ve è

SZrifene. Cioè si sloeglia da quella applicazicne o filamin unis sili

STANZA X STANZA XXEXIL o
Dicendo; Fratel mio, se nae mi vuoi Kedi jSoggsnnfest' altro, och' io m' adirey'
Quel benyche tu dicei volermi a faced, 0. O\par. Rpiuernred etme
Non mi dar noia,va pe' fatti.tuor, Hai tus quiftione? hai tu qualche rigire
Lerche ii mio mal non è male da biacca, Tx me 0s 4 dire in tutte le manicre,
Al quale ad ogni mo trovar non puoi  Lardin rispose, dopo on ne

Un rimedia, che vaglia una patacca, Tu sei importuno oi pbanmal

Perch'eglie e stravagante, ed alla moday,
Che non se ne rinuien Caposne coda
STANZA XXxXX1L.

Brunetto udito il cafoje quanto e' sia

Ma da chiio devo, iro eccomi prow; 4
Così guivi-di tutto fa unragcontes.)

STANZA XXXXUL ]k

E conoscenda, c' a ridurlo in sesto

Ji suo cordogtio,anch' eidolente refea, Ci vuol'alero che il medico,oilbarbiert,
Se ben per fargli cnor mostra allegria Vifi spenda la visa,e vadailrefios. |
AMa(com'io dico)dentro e chi la pefia Vuol rimediarvi in tutte le ae

Perch' in veder si gran malinconia,
Ed un umor si fifo nella testa,

E sav Ff risolue pr
D andar girando il i meal kh
Jn quanto a lui gli par che la fucchielli, Di trovargli una mogtic di suo gfe,
Per terminare il giuoco a! pagrerelli, Com' ei gliel' ha dipinta ginsto gino,
Fratel mio, se veramente.tu,mi porti quell' affetto, che ww-dici, lasciami Mary |
¢ non mi dir più altro, perché ad ogni modo tw pm rimediare al mio mal
che è grandithmo, Brunctto di nuovo lo prega, onde Nardino, vii
importunita gli racconta tutto il caso, e Brunetto, se bene dentro ha
travaglio facea buon viso, e datogli animo si risolué d' andar girando il Mondo
per veder di trovare una donna Jecondo il gusto di Nardino,, e cavarlo di quella '
frenefia.
Una cfortazione,, e richiefta simile a quella, che fa Brunetto a Nardino, fail
Ma scherone allo Gnocco meek saper fa. di hui affiizione come si vede ne i
versi dello Stef
Atto pr. Sc. pr. » ae riporto qui, perché il Letore veda, che a un' Let
rerato ( come era lo Stetonio ) non si dildice alle volre lasciare gli fludj più fer
per le bizzarric fanciullesche., ¢-spero, che non fara Ailgnee queita poca di digrel-
ne.



SETTIMO'CANTARE, 34y
6G NOCCHVS, ET MACCHERO:?
Ga ne Mundo traviare venivi,
pay ” Cur non tum morui, cum primim lucis in aura
- » Sborsavit genitrix ? Cur me disgratia emper
49 Perfeguitat manigolda fenem? Cur ladra placerum
gy, Abftulis', \& cunctis caricas me feeva malannis ?;
7 Quando finalmentum dabitur misura teavai ?
x9 Quando refinabis streghissima filia streghae?
» Dum me-pensabam biancain reposare vechiezam,
x Mille diabolicis straziorque, creporque ruinis.
» Vh me mefehinum | Poterit quis ferre focorfum ?
'Ma. 55 Appuntum Gnoccum video. Quid brontolas? ola !
gy Fronte malinconica, quid tecum, Gnocche, favellas ?
 Sigy-Deh poverome, pares viridas magnafle lucertas;
> 5, Tam demagratus, tam difuenutus apares.
gy Testa dolet forfan ? Sciatica ? Fiftula ? peius ?
»» Ag potius placidam flurbant penseria mentem ?
y» Dig mihi quzfo tuam scannat quid, Gnocche, coradam 2?
3» Vade viam, Macherone, tuam. Pradele, fogare

a ~~ 4, Mevolo', nec quidquam poteris fuccurrere Gnocco.
i Ma,, Ohimé! cur sprezas fradelli verba pregantis ?
o 3» Quis scit ? parlando passabit forte dolorus,

| gy, Praefertim-caro dum palefatur amico,

Ga 4, Deh nolis, quxfo, nolis mihi rumpere testam:
ea 5, Deh lafia me star; fum pienus; vade bonhoram,
p » Nec des impaccium, quoniam mihi crescis afannum.
wt | May »y Deh possar mundus ! Tortum mihi facis adeffum.

io ' »» Cur mihi, Gnocche, tuum non vis sfogare'Jamentum ?
ie x» Sum pro te chi 16: prafum dic, quafo, travaium'.
eh Gn.,, Pur ibi: Vade tuum, cancar ! tu vade viagginm.

". » Me miserum ! ad mundum veni trascinare cordam;
oe x Mancum nonne malum fuerat non nascere, vel i
i 9 Nascere debebam, plus prattum nascere fungus,

ib 9» Quam malé stentando scontentus vivere femper 5
più » Omnibus \& giornis centum morire fiatis ?

Ma. 5. Maide ! Cordoglio sciappas, \& spernis aitam?
ry » Vadis \& ad guilam matt, Lanzique briachi ?
ia a» Infuper, \& fdegnas, si quis tua vulnera-curat?

Gn, !
> a »» O bellum tempus, Machero, poca(que facendas !
oi > Otmnes'confilium femper dare novimus'aleti

i yy Sed fibi medesimis nolunt procurare parerum.

a - ay Bene dicit.vaigi proverbium: Ducere danzam,

i » 'nuces OMnes, qui fedent, bactere norunt,
ys Cum funt ad terram. Me lafits dico, malhoram.
Ma,  » Ah Zucarine meus, meus, ah Gnocchine, galantus,
a Xx

»» Quid

346

Gn.

Ma.

Ga,
Ma,

Ma.

MALMANTILE 4)

9 Quid facies hosti, fidedegnaris amico?) ) — wanes,
yo Cur mihi nafoonsdisy pbc vulnera
»» Non ego partibo, nifi contes ante' marezam.

x» Su, fradelle, euam crep: oaconien ieee raconta.
x» Non parlas ? Deh butta fora, meschine, venenum;
» Dic mihi, que-carpunt faftidia tristia mentem, ) 5)
5» Que lacerant cuce, que te fulpiria rumpunt?.
>» Nonue recordaris tirictos nos esse parentes ? HPiy
>» Eft tua mamma mee carnalis,Gnocche, sorella,
x» Atque ego nawura si,non caenalis, amore
3» Sum tibi fradeiius plus quam carnalis: aitam,

3» Quam potero tibi, Gnocche, dabo,; fac denique provam eu

»» Nam abi porto beaum, nec me fradelle licenties. ai
x» Namque amo te plus quam me stessum » ane
»» Dicito cunea mihi, nec ce meschine' fa
»» Confilium forfan potero tibi dare galantum. \&

2» Quid turbulentus guardas ? fu butta deh foras;

»» Eia, valent' homus; non finghiottire bilognat;

"5 Valneris ascosti nunquam medicina trovatur; '5 7)
x» Atsborsando foras fanatur fepe dolorus; 2
»» Fiftula, qua tumuit, totos corrumperet artus, itt
» Ni lancetta viam barbieri lefta taiaret,
3, Sufum, Gnocche valens » cordolia dire comenza. 4b

»» O fortuna mihi nimium traversa tapino,
sy Que mihi per forzam non firappas ventre magonem;
x» Eft-ne poshbilum, quod non sborsare fiatum,

» Unam nec potero gambam distendere voltam ?
x» Sum desperatus: volo me impiccare da verum;
x» Cerne, mei, Machero, cavezam porto fomari.
» Impiccare ? mai. Non impiccare te,non.non;
a» Mattelcis; costat troppum impiccare: nieatum

» Tu facies. Guardes gambam ! impiccare ?,Diavol !

» Et te, meque fimul piccares, Gnocche « Gas 'orlanauting
” Maidé » quis tantum milzam tibi rodit afannus?

x» Dic, faporite meus, que te fuentura chiapavit ?

») Sime impiccabo, cunctos scappabo, travaios.
>» Pur iJluc; iftam mattezam manda malhoram.

», Sola meum stentum poterit sbandire,caveza\. Mt bed
>» Ab nimium certé te stessum, Gnocche', fafinas: ) )
5 Mancum donna timet, mancum se donna sgomentat: a
»y Ne facias cofam talem pazelcis adeflum,;

»» Incidis ia brafam cupiens evitare padellam, te
>» Qui fugiens damnum, foccorsum a Morte rechiedisy «

»> Qua nullum maius damoum reperitur inorbe, >)
» Dicas, quid peius furca maginare poteftur?, 5 «4

FF pe pe

=



SETTIMO CANTARE;

> Nonne vides furcas ipfos odiare (afinos,

» Millantas furcas meritant qui mille fiatis ?

x» Forse putas bellam cofam piccare feltefium ?

» Nullos audifti, nullos nec, Gnocche, latrone

93 Esse volenterum piccatos., Canchere | robbam

x Perdere, poderos,, filios, atque moieram

»» Possumus; at contum non muttit perdere vitam |

xy Parlemusd' altro, bona notte; porge cavezam,

» Fac sennum matti, caveas non talopram.

» Si sennum matti facerem, mattiffimus essem; '
y» Sum deliberatus cannam truncare una volta;

2» Nec parles; quoniam mandas tua verba Patraflum;
x» Et liquidas tentas accoglicre retibus auras;

347

 5» Dextra orecchia bibit, sed versat lava parolas;

y» Surdo verba canis; oleum fimul opera perdis.

»» Qui pro te robbam propriam, vitamque gitarem,

»» Pocum stimo malum pro te gittare parolas.

»» Indarnum gracchias, indarnum dico, va viam.

» Litera vis tandem ficri longissima ? Ga, Certum.

2 Et godis cortum laqueo difrumpere collum ?

» Audis, Ma. Et tandem cornacchis essere pastum ?

» Sentis. Ma. Bavofam buccam torquere ? Gn, Cofiaum;
2» Et tralunatos oculos mostrare ? Gn, Davanzum.

» Lucentem faciem, lucentia bracchia, fula

» Viscera, contradam totam peftare fetore,

2 Et vitiare diem vitiato viscere letum ?

»» Sinum; si dico, finum; volo rumpere cannam.

» Heu ipfis fugiende lupis, buttande fofatis,

» Terribilis stratiande modis, privande facrato.
» Denique penserus nullus te, Gnocche, tuorum

yy Tangit ? Cui laffas pupillos, paze, chiatinos ?

x» Cui robbam ? cui confortem ? miserofque parentes ?

» Teque finalmentum ? Case qui scribitur heres ?

»» Vis proprias carnes tecum mandare Patrafium ?

»» Vis proprios natos panem cattare per uscios,

2 Disperios pueros pitocorum more per urbes 2

x» Et post de fora veniet qua fama da verum ?

»» Gloria que Caf laflatur? Respice tandem:
2» Teque, ruofque fimul, misere miserere fameia 5
» Et miserere tui, qui proijciere fofato,

x Indignum facro corpus recoprire tereno..

2 Forfan ad Stygias ibis ? (eu forfan Acheum

»» Ibis ad Infernum ? pensa, pover' home, to feetos;
2 Pensala, dico, benum: facile eft calare dcoiium,
»» Sed montare super cancar; stentare bisognat;

Xx 2 >> Sed

sj

348

Ga,
Ma.
Ga,
Ga,

Ma.
Ma.

Ma.

Gn.

Sum contentus; abi, grarum sed fiascum, —
' 4 Nam stio cesta 6 rampas beulaon 6 a 4

MALMANTILE o).

j» Sed nec stentando brutto scapulabis ab Oreo. »

»» Horfus tornemus ca(as; fu, Gnocche: cavezam

>» Case mitte tue. Pensas piccare? bel opram; «

»» Essere non vellem, Veneto pro boia teforo,

» At tw, te stessum si piccas, boia farabis.

» Ah tibi, ne quel, tibi fis ne boia medemo, ~~

2 Et qui pro centum mundis non essere velles;

sv Bflere pro nihilo nolis. pec porge,
ico

pocum, ff l'3
Forfitan ipfa dies (aldabie, Gnocche, fericam, — è
Dura remolicfeune paicis,\& tempore forbay 99
2 Nespula dura die mivescunt, nespula dura
»» Guarda mo-, si Gnocchi poterit mitescere noia
Tu bene cicalas; dottorus, \& esse videris::
Sed cicala purem; gicttas nam carmina faxis.. \ «
Al facias i » Gnocche, pl >
»» Extremumque mihi praftes, care Gnocche; favorem.
2» Quem nam ? dil, Ma, ura; facies, quod certe domando
»» Dummodo fare queam, fabo, sta fupra parolam, —
Et potes, \& legrus facies. Gn. Dic ergo,quid opta
Eft mihi botazus vinetti, Gnocche, rubentis,
xs Quod difamoratis posset rubare coradam,
»» Illius humore taze cum plena plaaura cht,
Saltitat, \& brillat, brillando lumina frezaty
Et rubor in vitro liquefatti more rubini,
»» Ac dicto citius spumat; hune inde dileguat
» Puri sbottigliata meri vis fernida, quali
»» Cum soffiat Boreas, nubes sfrattare per aura
»» Cernitur, \& Calum laté purgare ferenum.
ay Sat scio,, si nafum preettabis ad ante bicherum',
»» Optabis fieri totum te, Gnocche, nafonem; $ ”
»» Piccantum retinet pulerum, garbumque galantum, o
2» Quod refucitaret mortos. De hoc, questo,pochertum «
»» Gustes, ante tuum claudas quam tofte fiaum',: sie
xy Atque mei hoc portes extremi pignus amoris, f
x Vis rechem chi 16? Gn, Reches, sed frettola paffym. - P
» Nigotta proderic, cum fim piccandus adi ys ta
»» Auamen hanc lafles, dum torno, Gnocche, cavezam, «
» Ne te gire viam tua tantum spafima cogant,
»» Et fine gustando vinum, morire, galantum',

VOLER bene a facea. Portar granditfimo affetto.. BE' f
Va pe' fatti tuoi. Cioè vattene,e bada a te, Res tuas ribi babero;

riti anticamente alle mogli, quando secondo le Romane
Vedi sopra C, 5. st. 57. 7 —

ah
8S Geta


SETTIMO:\CANTARE: 349

| NON 2 mal da bideca', Non è male ordinario,e che firifani con pocdrimedio,
perché la Biacca; che ¢un biahco cavato dal piombo, ¢d è adoprato da i Pitio-
ri, serve anche per fare un' unguento buono.a poco altro, che ad alleggerire il
ng Lp erat però dicendosi.: Won e mat da biacca, s' inten-
de, Manatee Sri aia:

 CHE waglia una patacca, Che,vaglia nulla. Che patacca è moneta, che. in,
Firenze non vale, Paracomé una moneta di ramevufata in Portogallo, che vale
guattrini, Cos} noi d*una cosa da noi tenuta in poco pregio, diciamo. Wo

ale un soldo, Nonne darei un soldo See

ALLA moda. Vuol dite all usanza, come vedemmo sopra C. 2. st. 54. mr
in questo luogo vuol dire stravagante, o nuovo, e non più sentito, o vilto, e del
tutto infolito; Diciamo: cervello alia moda per significare cervello fravagance, o
fantaftico; dal mutar che si fa tutto giorno dela moda nel veltire.
— WON si rinniene ne capo, ne coda. Non frritrova, ne il-principio, ne la fine di
wm, yeofa. Non'fi fa, non's' intende, o non si ritrova come la cosa si tlia, Vee
—- baput, nec pedes, disse Cic, EB' traslato dalle matasse del filo, e si dice anche Now
ritrov, che @ il principio della matafla.
ih) «| AL tu quiftione? Antendiamo havere inimicizic.

| HAL eu qualche rigiro? Har ta qualche innamorata ? Che la voce rigiro usata.,
] 'come nel presente luogo,vuol dir Pratica di donne per vizio; che per altro,rigiro

'significa Ripiego, dicendosi: 1] tale fa molte faccende, percht egli ha molui ri-
giré, cive neigh ed oe di vendere la sua roba. Alle volte Gi piglia per

« Vedi sopra C.4. st. 60,
. DENT RO è chi |e pefhe' Quand' uno si sforza di mostrarsi nel vil allegro, ed

'ha travagli da star malinconico,diciamo; Ei fa beon viso, ma dentro è chi la pefa,

hes dentro sta in altra guisa. Kifus in ore, fletus in corde, Virg. Spem vulti fimu-
lat, premit altum corde dolorem.

5 Heaiore fie in teha, Pensiero, o fantasia ostinata. Vedi sopra C. 1. ff. 10,
PAR ch ei la fnechieli. Egli sta fra i) st, e il no di fare una tal cosa, che direm-

mo Irrefoluto. Dante Inf. 8.

Padi age Che'lsi,e'l no nel capo mi tenzona,

Traslato dal giuoco delle carte, che firdice fucchielare quando si tira fu la carta,

'adagio adagio 3 if che pure è traslato dal bucar col fucchiello, che e una azione

simile al tirar fu la carta. Qui vuol dire, Pare che questa sua fiflazione lo voglia.

adagio adagio fare impazzire,¢ ridurlo a i Pazzerelli, che e lo spedale,dove si
mettono i pazzi.
RIDVRLO in feffo. Ridurlo alla giufta misura; Raggiuftarlo, rimetterlo in,

buon' essere: fargli ritornare il giudizio.. Vedi sopra C. 1. st. 15.

SU spenda la Vita, € vada il reffo. Si spenda ta vita, e la toba. Tratto dal giuo-
0, le ff faole scommettere, e dire. Yada il resto; fo del reffo, Bquic det.
to per figura; perché quando è andata la vita, che e la più cara cosa, che noi

ha iamo, par che non ci refti quasi altro da buttar via.
g 6 "O cinffo, Perappunto. Ela replica ha la solita forza di superlativo.

—— Catullo.. Af igis magis increbrefeunt. Nell'Ebraico Azeod, che vuole dice af/ai, rol

ot, raddoppiato vuol dire afaiffine,moltijimo.,

|

STAN-



359
STANZA XLIVs.0%
Percio d abiti, e soldi si provvede, |
E da buone [peranze alfno Nardino.

Esce di casa,e mettefi in cammino,

Shirciandofempreinqua,e inlafevede >

Donna di viso bianco, echermifino;
E se ei ne incontra mai di quella tinta,
Vuol poi chiarirsi,vellae verayo finta,

STANZA

Di modo cl ei non vuol restarvi colto
Ma fharvi lesto, e rivederla bene
E per questo una [pugna seco ha tolto 5
E sempre in molle accanto se la tiene,

 Suegetto, che gis occorra farne prove,
Brunetto date buone speranze.al suo fratello,montd a cavallo; ed
0 un' huomo a piedi, fen' andò cercando d* una donna bianca', e rofladi
naturalmente, e sapendo che tutte le donne hoggi si lisciano, haveva
spugna bagnata,per far con quella la prova,se il colore era finto,04
per molto, che egli cercafle,non trove mai donna, nella quale occorrefie far tal
Prova, perché si conosceva senza farla, che twtte eran tinte, e-lisciate. Quello
colore finto, che chiamiamo liscio, o belletto,si dice anche frco-, che |
buona a tignere i pani; da i Latini detea faces, e l'intendevano
si per questo lilcio, o belletto. Plaut. Moft. 4.118. Vetule edentula,
corporis fuco occultant. EB di qui i Latini per fuco intendono una sorta d
che ricopre con artifizio un mancamento in una mercanzia, ec, onde;
cere,

pk
S8IRCLANDO, Guardando attentamente. Vedi sopraC, 1.f9,
CHER4MISINO. Roso di Chermisi, o Cremesi, E' il roflo porporino, chef
fa col sangue di certi vermi chiamati con voce Spagnuola Cocciniglia dal Latino
coccineus color, colore di grana, colore vermuelio; ed \& il più nobile, ed a 0
te, che si trovi, ne mai perde il suo colore:e da questo nel presente luoge inte
de rosso naturale a perfezione, e che non perde, come farebbe il finto +
o Karmes in Arabico vuol dire grana.. Latino coccum,(econdo lo Scaligero eferti
tazione 325.

D1 quetia tinta. Di quel colore. E' termine pittoresco, cotumandosi
dire: La tale ha una carn i

di carne.
VVOL chiarirsi, Vuole accertarsi.

MALIMW/AN TILE o 9%

4 STANZAKLV. | '
— \

> Che non si mini o si taftri le quai;
Epreso un bud cavalloyeunbuomoapiedey 00 B
Cc

ragione nella quale seno beile tinte, per intendere Belli

'

Pas.

“Wella pare il ritratto

Quattro dita vi lascia fu di loiay
Oe
Chrella par proprio un' Angiolin di
XLV" a pot
Con che passan le sopraiil volte,
Vedrà: s'il color o se vii

Aa gira girasin fasti ei non yitroyh,

i=

S
sa

fucum

a eee

aachil
ea

NON si minij. Non si tinga, Minio \& specie di color rossa cavato ¢
8

\> adit
ae 1 si:
no; € miniare € una specie di dipignere con finidimi colori sopra cose e
me cartapecora, ec,

S/ lustri le quoia. Si \isci la pelle.

MOST ACCIO infrigno. Vilo grinzolo,

refroigne.

o cresposo,o rinfrignato of
ANCROIA, L' Ancroia è finta una donna brava in un Poema



SETTIMO.CANTARE - 351

~ Regina Ancroia; e perché questo Poema è degli antichi, che si trovino nella.

 lingua nostra, mi do a credere, che quando si dice l'Ancroia,' intenda una vec-
chia. di Berni, descrivendo la sua serva in un Sonetto dice.
' Lobo per cameriera mia l' eAncroia,
. Madre di Ferrak, Zsa di Morgante,
y w  )  ehicavola maggior dell' Amostante,
phe Baliadel Turco, e fuocera del boia
 Ma pnd esser ancora, che queita voce Ancroia sia un' addiettivo, che venga da
i@, che vuol dire Zotico, e duro dai Lat. corium quali inquoito, fatto duro, co-
il quoio. )
thee Col pugno gli percosse I epa croia,

ae Da questa voce croio habbiamo il verbo tncroiare, che vuol dire aggrinzare,.¢d

= =
ee

ce
a

SRSLERESE

renee per intender pelle grinza, e fecca, e indurita, come € quel -

vecchie, alle quali però si dice per scherzo Aduna incroia, che nel parlare
'Pultima lettera di 44ona confonde, e mangia la prima d' ixcroia, viene a
ancroia, che vuol dir vecchia grinzola. Jncroiato si dice un qaoio, che per
stato preflo al fuoco, sia divenuto duro, e grinzoso, ed il simile una carca-
abbruciacchiata. Si dice incroiaro anche un panno divenuto sodo per gli
mi', e lordure; ma di questo'è più proprio incorezzato, dal Lat. corryia. Il
bolifta Bolognefe dice, che Ancroia signitica vecchia, che va crollando il ca-
po, e che viene dal Greco Craein che vuol dir croilare. Ma venga donde si vo-
glia, basta che appresso di noi vaol dir Donna vecchia, e brutta, ed im questo
Aenlo e presa nel presente luogo.
 LOLA, Sudiciume. Terra stemperata con acqua, e ridotta liquida, che con
altro nome chiamiamo mota. Qui vuol dir quelle materie, che si mettono in sul
vilo le donne, le quali s' imbellettano, Voce fatta per avventura dal L, i/luvies..
. IMPIAST RA, S' unge con materic bituminose,¢ viscose come è l unguento.
STVCC.A, Stucco è quella composizione di geffo, e colla,¢ d' altre materie
4tenaci, che serve per riturar feflurejo magagne ne i legnami. E facco è una spe-
cic di gesso, o terra, o altra composizione, con che si tanno le figure di rilievo,
i per fucco intende quelle materie, che le dunne si mettono sopra il viso per
la faccia, e turarsi le margini del vaiolo,,o altre cicatrici; che il
verbo fuccare vuol dire intafare, cioè riempiere i buchi, e ragguagliare una.
I¢; donde gli Orefici dicono fluccare, quando con una certa loro lima
detta lima stucca, spianano i lavori d' argento. Sewccare vuol dire ancora quan-
-do un cibo ci apporta naufea, o i discorsi d' alcuno ci vengono a fastidio.
. ViN* Angioline di Lucca.. A Lucca fabbricano certi figurini di cera, di geffo,
Od' altra materia,a' quali dopo formati danno il-colore di carne con un rotio lu-
Sirante; per questo d' una donna lisciaca diciamo; Pare un' Angiolino di Lucca.
Gosi iGreci, che le belle persone afiomigliano alic statue ben fatte, le chiamano
Agalmata,e Properzio,ditie che il colurito del viso della sua donna era giutto co-
»me quello, che si scorgeva nelle pitture del famoso Pittore Apelle. Quals Apelici
<¢f color in tabulis. in un' Bpigramma Greco una faccia impellettata, e lilciata,
con clegante bifliccio vien detta Profopeion, con Profopon, cioè maschera'y o non
faccia « Vedi Cel, Rod. Leet, antig. lib. 29. C. 7.

NON



55%
NON vuol resparni colto., Non ins rimanare q
ST ARV1 lesto. Stare'accorto, O.avvertito, 9) o 96
G/RA gira, Cammina in diversi luoghi; cammina
IN fatti, E' lo stesso, che in somma, o in pen? L,
STANZA XLVIL..
Dopo che tanto a ricercare e itoy
Che i calli alc,. ha fattoinfulafellay
Giunfe una sera al luego d'un Romito,
C' arestar L innita nella fan Cella, quel delle ¢
A lui parne toccar il Ciel col dito Di che speffa ciascun
(Per non haver a iar fuori ala stella) Stettero a croschio ii
4M pafjar dentro, ed egli,e il scraitore,
Ringragiando idbuon buom di val favore,
STANZA XLVIILL
Vestia di bigio il Vecchio Mdacslente,
Facendo penitenca per Adacone Dice chi sia echo di cafael
E perch' ei fu nell accattar frequente, Non per fa conta,mad! un si
Per nome si chiamo fra Pigolone. Del quale infino alt' Tad
Costui,( com' io diceva ) allegramente Perché gli pare uscito die
dn Cella raccetto le lr perfoue, Non fifas ei fifia più carne,
Spoglia il cavalio,e gli trito la paglia; Così piangendo in far di cid'm
Sul desco por distele la tovaglia, Per la mmuta conragl la.
Capito Brunetto una sera alla Cella d' un Romito, dove esseada
tato, stando a tavola raccontd al Romito ibcafo del Fratello, dicendo
fuora per far servizio al medesimo suo Fratello,
TOCC AR il Ciel col dito, Conseguir l'imposfibile. Ap
ST AR alla fella. Dormire all' aria; a ciclo scoperto; alla fella diana, Lat
ub dio, NNR
MACILENTE. Mal sano; Cioè magro per lo stento, e giallo i
ione.
: EV frequente nell' accattare, Duc tefti di mano dell'Autore dicono uno
te, edè) ultimo; ¢l'altro servente, equesto e la prima bozza, e se i
¢ Iraltro può stare, io pighierei ? ultimo, perché in fultanza vuol dire che
€ra attento,¢ diligente nell' accattare, e sempre chiedeva, che da,
importunita, s acquiftd il nome di fra Pigolone che così chiamiamo
sempre chieggono, e che mostrando una certa ingordigia di roba,si
pre dello stato loro. Pigolare €il verso de' puicini, che beccano. Lat.
Spagn. piar dal fare pio pio, che così e il lor verso. ovat
DESCO, Tavola, sopra la quale si pongono le vivande, quando si
dal Lat. discus, che e pierra rotenda, o laftra da scagliarsi, Vedi sopra!
TVTTO accatrato. Ogni cosa havuta per limofina.
FIORITO quanto un Adaggio. Fiorititimo;percht ii mese di 'Maggio' la
ne de i fiori; O pure perché queili, che vanuo a cantar maggio,porta
ad aealbire tutto pieno di-diverti tiori, il qual camo @ albero ch
Bio » ° maio. Diciamo: vizo foecisay quando o per ctier ab ton



SETTIMO°CANTARE, 333
per altro mancamentoj il vino dosi nel bicchieresha 'nella superficie minu-
tissimi frammenti d' una cerca specie di muffa'bidrica; che è il panno, che si fa
dal vino, equ 'chiamano fort; sì che quis" intende, che il vino era vicino al
fondo dell ', o havea altro mancamento, che' produce la detta muffa; se be-
I 'chevoglia dire Vino isquisito; perché /ro itoeattribuco di perfezione in tut-
4 syeccetto che nelvino, che l'esser fiorito è segno d' imperfezione.
' centuna bette. Questo numero centuna, benché sia determinato,
dee | t per indeterminato; e vuol dire Cavato da infinite botti di coloro,
}haveyan dato per limofina, E questo pure è imperfezione del vino, che
perde lo spirito., e la bontà in tanti travafamenti, e mescolamenti,,
 STETTERO a crocchio. Stettero chiacchicrando.. Vedi sopra C, 1, st. gt., €
Bietemin così detto dallo strepito, che si fa ridendo, e chiacchicrando
ielle conversazioni di trattenimento, perciò dette Crocchi, Dal romore similmen-
 teedal faono che rendono, sono dette da' Prancefi Cloches le Campane. Così
i — 'saccordano nel rappresentare con l'arte i semplici suoni inartico-
jt lati che sono un' inalterabil linguaggio della natura. '
ed batte dove il dente duoie, Si dilcorre empre volentieri di quelle cose,
j@ dove hala passione, o sia di gusto, o di disgutto.
' a il campanello. Parlaya sempre lui, Questo detto viene da i Magiftra-
/
Cd

Bet
*

tidi ¢, ne i quali uno dei Colleghi si chiama il Proposto, e questo sempre
Sa aj litiganti, e chiama, e licenzia dall' udienze, ed i compagni
' a cheti; e questo Proposto tiene allato alla sua seggiola un campa-
nellowE da questo, quand' uno in una conversazione sempre parla lui, diciamo:
yp Bi tleneil caimpanetio.,
APINCRESCE fino all anima' Gli ho grandissima compassione; Vedi sopra
in questo o; st, 26. Mi'dilpiace, mi pefa. Dante Inf. 6,
se RDS “Mi pefa'st, ch' a lagrimar m' innita,
DU Greco dice Achthomai, mi dolgo; e 10 Spagnuolo similmente pe/ame. Onde quel
che'in Toscano si dice' dare il mi dispiace, esso dice, dar ef pefame: La stessa forza
ha MMe, AP inere/ee, quali mibi merave/cir, secondo il Ferrari; mi grava, e»
BS Pope nage Amore e pefoscominciò Dante una Canzone. £' m' incre/ce di
” Z PMO. Sus '
WON fap ei ff fra carne'; o pesce. Non fa quel ch' ei si sia. Nonè in cervelio,
Non ha' vO conoscimento. Awevo pesce dicevano gli antichi un' biomo /Pra-

Eee,

ae

wm math
ip! neces ANZA LL STANZA LIL
ee Sta Pigolone attenro a collo rorto Egli ha un giardino posso in un bel piano,
i Ad 'ascaltarlo s€ poi ch' egli ha finito; Ch' e ognor frorito,e verde tutto quato;
“a: F igliuol,risponde a lui, datti conforto, Giardiniero non v' t, ne Ortolano,
oy + Bfappi, che th sei nato vestito, Che a entrarvi nefjin pus darsi vanto,
» Che qui? Pbbnom falnatico Magorto, Da per se lo lavora di sua mano,
a Ch'e un bestione, un diavol travestito, E da se (0 fondo per via a! incanto,
o \& “Che se te lo vedessi vb eglie pir brutto! Con una casa bella di frupore,
ye Balke a suo tempo conterotti it rxtto, Che vi potrebbe ar  Imperadare.
5 4 vy STAN:

354 MALMANTILE

STANZA LIIL
Ma io ti uno dar' adesso un? abbozzata
Lui presto presso della sua figura.
Ei nacque a' un Folletto, ed' nna Fata
A Fiefol n' una buca delle mura,
ed è si brutto, poiche (a brigata
Solo al suo nome crepa di paura;
O questo e il caso a por fra i nocentini
ed far manciar la pappa a quesbabini,
STANZA LIV,
Oltre ch' ei pute come una carogna
Ede pin nero della mezza norte,
Ha il ceffo d'Orfoye it collo diC arogna,
Ed una pancia » come una gran borte
Va in sui balefirs, ed ha bocea di fogna
Da dar ripiego a un tin di mele cotte
Zanne ha di porco,e naso di csvetta,
Che piscia in bocca,e del continuo getta,

STANZA

ea lasciando per hor  altre da parte,
Cocomeri vi son di certa raza,
Che chi ne puo haver uno,e poi la parte,
Vi trova una bellissima ragazza,

Pigolone incefo il bisogno di Brunetto, gli da animo con dirgli, che
huomo faluatico ha quivi un' orto,dove son cocomeri, che tagliandoli n'esce suo-
ra una bella fanciuila, la quale chiede da bere, ma sce feglida., ella
Deicrive ancora in queste quattro Orta ve la qualita di
SE/ naro vestito, Hai havuto buona fortuna, o qu wd
questo termine per esprimere,quand' uno desiderando qualcosa difficile a,
s abbatte accidentalmente a trovarla per appunto,,come ci la desideraya, eda
proposito del suo bisogno. Dicono te Levatrici., che talvolta nascono ban
con una certa spoglia sopr' alia pelle, la quale spoglia non si leva loro subito
ti, ma si lascia, e casca poi da per se in processo di giorni.; eral creatur
si dice vara vefiira ed € preso per augurio di felicita di quella tal.

ha dato origine al presente dettato.

VN diavol travestito. Vin diavolo immascherato da huomo; intende un'

brutto, quanto il Diavolo.
BELLA di flup
yvede;
VO!

I Pittori dicono Abbozzare

Jerto e Gianni Schicchi, dice che i P

STANZALVL. —
Dell' ofa poi ne fa si ce

ere. Bellissima mirabilis vifu. Tanto bella, che fa flupire
ma per venire la voce /tupore dal latino,può ognuno intendere il suo}
IG LIO darti un' abborzata, Cioè ti vo;

Y > quelle prime pennellate, che danno in.una tela;
trove, dove voglion fare una pittura. Vedi sopra C. i uy

FOLLETTO. Uno di quelli spiriti infernali, che dicono che stieno per Hari
1) Ferrari nell' Origini alla Voce Fuile,citando Dante Inf, 30, essi disse, quelf

z

E della pelle ne fa maccheroni, =

Diente in Jomma v't, che vada ma
Sreche Brunetto figliuol mio, tn femti,
Ch' egli è un cattivo,ed orrido ammale,
Hora torniamo a suci scompartimemi,
Ove son frutte buone quanto il
Vaghe piante, bes fiori, ed altre vole
Com' to ti potrei dir maraviglose,
LVI, AYE
Che per esser aftuta la sua parte,
Dirattiche tu gli tpia una fun rare,
A un di quei fonti li si chiari,e freddi,
Ma se la ferni, a Lucca ti i

efto Mi i
ota. cadena

ni

glio descrivere alquanto, 0
st, 41.

?olletti sono la/civ) genij ac Lemures rifu a
domos implentes,. = E.



SETTIMO CANTARE. 335
FAT A. Vedi sopra C, 4. st. 45. i
eA FIESOL o una bucadelle mura, A Ficlole si veggono ancora alcune reliquie

delle mura di antica Città, ed in ci frammenti di muraglie fra l'altre si
 vede una gran buca di »od'altra cosa simile, la quale dalle donnicciuole è
- ereduta, ed è dataac ai fanciuili per abitazione delle Fate, ¢. pero vol-
ms @idetta /a buca delle Face. E questa è quella buca, nella quale dice
-» che Magorto era nato d' un Folletto, ed' una Fara. Angelo Poliziano
~ al titolo Lamia dice: Vicinus quoque adbuc Fefulano ruscnlo meo lucens Son-
ticuins est, fecreta in umbra delite/cens, ubi fedem esse nunc quoque Lamiarnm narrant

a imuliercule « Questa credo sia quella caverna, che-oggi si chiama /a fonte fotterra
¢,  luogo orrido  e spaventevole, ma sempre pieno di limpidissima, e freschiffim
lt NS TEV ' 2

ry “SNocewr 17 « Cioè quei ragazzi, che s' allevano nello Spedale degl'Innocen-
so erensa +5: ton

ca ~ CAF AR mangiar la pappa a quei bambini, Così diciamo d' un' huomo, o donna
nil a e brutti, quaGi che sieno come il Bau, la Befana, e simili larue in-
ga  uentate dalle Balic per render i bambini ubbidienti, e fare che per il timore man-
- Cai wa. Vedi sopra C. g. st. 3. E.questo putire da i Latini era espreffo
0 co} paragone, perché dicevano vixum cadaver. 11 Monofini.

ut nero della mezza notte. Negritimo, piii nero del buio,

 VAin fui balefri » Ha le gambe fottili, e torte come sono i baleftri, compa-
'fazione vulgata), sendoci una cantilena di Balie, che dice.
° Ben ne venga Mignamau,

Saif oe j Cha le gambe a baleftrucci.
O81. 3 e Sbilenco, dicesi chi ha le gambe torte; e ancora Aver le bilie;
tratta la similitudine da certi legni torti, o randelli, co” quali i vetiurali legano
itetto, e arrandellano le fome; da loro dette bilie.
 BOCCA di fogna. Alla bocca delle fogne maestre, o principali, che ricevo-
'no acqua delle strade,quando piove, e la conducono nel fiume d' Arno, è figu-
rato up mascherone di pietra, il quale ingoia l'acqua ed oga' altra sporci-
zia., € di queste intende il Poeta; e da questo diciamo: Bocca di fogna a uno, che
mangia, ed ingoia ogni sorta di cibo, se bene sporco, senza distinzione, o ri-
yaleuno. Latino bedwo, gurges. Queste fogne in altri luoghi d' Italia sono
lette Chiaviche dal Latino Cloaca.

DA dar ripiego., Cioè dove entrerebbono tante mele cotte, quante n' entrereb-
be in un sina, che quel gran valo di Iegno, entro al quale si mette l'uva pigia-
$a bollire per farne vino. rare
», ANNE, Denti: Propriamentes' intende di quei denti Junghi, che hanno i
Signali, i lupi,i.cani, ec. che noi li chiamiamo anche denti Adac/tri; o Adaeftre.
Vedi £ aes, st. 64. Borle \& meglio dir /anne, ed è più conforme all' origine,
Onde /ubfannare buriarsi d' uno ridendo, in maniera, che tutti identi, come di-
$e il Boce. si poteflero trarre; mostrando le sanne. Daa. Inf, C, 6,

Quando ci scorfe Cerbero il gran vermo,
Le bocche aperfe,¢ —— le fanne.
yz

SER GF BRLLSES week

Se

it
4

eC,



356 MALMANTILE?D*)2

e c.22. E Ciriatto, a cui di bocca nscia > Vo
D' ogni parte una Janna come 4 porcoy eres Pepe)

Gli fa sentir-comel' nie sdrucia. » 8
NASO, che pisciain bocce, Cioè naso aquilino' che ha la a
la bocca, e pare che vi colidencro. vb~orn saokay
BERLING ACCIO « i Giovedi geaflo, chee l'ultimo giovedt

detto Serlingacciv da Berlingare, che vuol dire bere, e mangiare',

mente, come si fa in quel giorno: e così Magorto, quando pi a
faceva conto, che quel giorno fusse il Berlingaccio, toleani:

menti, pappalecchi, e Gorxeviglie,daligodere, Latino ga

ennizzandolo con
c wifare, conic si
antico Glofiario, onde lo Spagnualo gozar., godere pel nostro gavayzare
ti finonimi, che voglion dir ghiotcornic Bocc, g. 8.n. 2. Sé cue
pis volte insieme fecero corzovighe sec. << i8927 toil?:
MIGLLACCIO, Sangue di porco, o d' altro animale mefeolato'con
farina', e poi frittoynella padella a uso di frictata'da alcuni* Latiot
chus; se bene questa era una composizione-dicacio, e falamevdal
che vuol dir cacio, etarichas, che vuol dir falame. 1
STVZZIC ADENTI, Nettadenti: Sottilissimi,ed acuti stecchi
d' offo, o d' altra maceria per uso di nettare i denti + Latino
BYONI quanto il sale. Saporitissimi.. Vaaevivanda con molto sale'
rita, che vuol dire il contrario di sciocca, oinGipida ye: senza Yale 7 ¢
faporito è meglio al gulto, che Pinfipido,'¢ pero per faporito i i
¢ dicendosi; buoni quanto il sale, s' intendefaporiciimi, cioè gustofissimi

fapore. e
TCOCOMERO. Specie di mellone acquoso di fapore'dolce, che nella
stagione calda per rinfrescarsi. In moiti luoghi d' Italia chiama
così la chiama il Mattiolo, e dice che era incognita a iLatini, se bene si crova
cuckmis, ma intendono il cetriuolo, che pure in alcuni luoghi si chiama eeeome™
Anguria, dice il Perrari, e detta quasi cucumus anguinens, o così questo nome
che era proprio del cecriuolo,per mancanza di vocabolo fu tratto a
fructo, che noi Toscani chiamiamo cocomero. i
e4 LVCC Ati riveddi, Questo detto significa Non la vedrai più,
Buoni da Lucca nel suo teforo de“Proverbi dice, che havendo un
Lucchefe veduto un Gentilhuomo Pifano a Lucca,usò seco cortesia
desinare a casa sua, dove condotto, fu trattato con ogni sorta <r
titofi il Pifano, e ritornaro alla patria,avvenne che fra poco tempo'
andò a Pila, dove paruegli conucnevole vilitare il Pifano suddetto: T
pero alla casa di esso, dopo haver molte volte buflatosal fine sa ffaccit
o gli disse che non lo conosceva; onde il Lucchete disse: 2: e4 Lucta ci-vede
Pifa ti conobbi, o con questo fiticenzid. Così forive un Lavcchefe:, ma tt
voltano il proverbio dicendo: 4 Pifativedat pea Lacca si i
grato,¢ scortefe quello da Luccaje:non quello da Pita') Seibene il
era ne Lucchefe ne Pifano nella sua En. Te. C. 3. st. 4. dice:;
E dicon spefo altrnis Ti veddi a Lucca,

—nw ese E-epeEE PBB TL ete



SETTIMO CANTARE! 337

STANZA LIX.
Efe en ean
Dirad, che tu buon Cavalier non sia,
entre conforme all' oblige non vft
Servitie con le Dame, e cortesia.
ea lascia dire,e tien gli orecchichinfi,
Non ti piccar di cio, sia pure al quia,
Gracchi a sua postayty non le dar bere,
Accio non fugga;e poi ti sia il dovere.

Con questa, che fara farta a pennello, Vientene dunque meco,e sia in ceruelo.
| Come te cerchi, lenerai dal cuore Cammina piano, e fa poco romare,
—  Ogni dogti i affanno al tuo fratello, Chefee' ci sentea forte, scuopreil cane,
16 --Edioten' entro già mallevadore. No occor' altro;noi habbiam fatto ilpane,

ite seguita:a narrar la favola del Cocomero, ed instcuito Brunetto di co-
a' ome contenere, perché la fanciulla non gli scappi, s' avvia con esso alla
volta del giardino di Magorto. -
s far conto @ haveria vifea. Ti puoi dare a credere d' haverla veduta qua.
4: i a.vedere, perché non la rivedrai più. 2
Al uno stivale. Refterai beffato, Retterai uno scimunito. Vedi sopra
yg  (Geqfero, LGreci dissero Bagas con/fieyti, da un tale detto Baga, o pure Bagna
a} 'nome da Eunuco; che fu un' huomo infipidissimo; Donde poi noi diciamo Bag-
sg) 60» o Baggiano, a ua' huomo scimunito se non forse da Ha/eo,, e da Habbano; 0
da Bageiano sorta di fave maggjore dell' aitre.
of va di forche, edi moine, Vina quantità grandissima di finte carezze,€
'Nezzi; i Latini dissero blandicie, Ed in questo proposito tanto è dire far le forche,
5  dexai, quanto mome, significando tutte tre una sorta di Jufinghe fatte con
wit Bti,ocon parole, e sono quasi lo stesso che adulazione; perché ancor le»
dt “nine sec, son atti, gesti, e discorsi, i quali contengono, se noa falfe lodi, co-
»meccontienc ? adulazione, almeno falfe dimostrazioni d' affetto affine di com-
eo. jiacere ye di acquiftar ia grazia di colui, a cui si parla,¢ queste son proprie di
ie di femmine, e l'adulazione e conuentente ad ogat sorta di persone,
ma è sempre indizio d' animo vile, ed esseminato. Ll Landino nell' esposizione
a Dante Inf, C. 18. dice, che gli adulatori in lingua Fiorentina si dicono moinieri;

f=

¥ »Ma questa voce non si dicendo in oggi, ac avendo autorita di Scrittore nell' an.
si tico, mi fa credere, che il Landino la derivaile a capriccio daila voce Fiorenti-
i na Meive non trovando parola corrispondente alla Latina ddulatores, I) Cala

nel Galateo volendo mettere in volgare il Latino ada/ari, lo esprelie colla paro-

SSL.

» la Piaggiare, L Bini in lode del mal Francese dice:
uhangl lo non roppi già mai; ne carfiiancia 5
Machi mi va con si fatse moine,
Vorrei porergli sfondolar la pancia.
 La Stor. di Semifonte Trattato 4. Quand! altri ha ofefo un fupremo, non e da si~
< darsi di lui', ne delle sue aftute moine,¢ Lusinghe.,
-  NON+i piccare. Non v' offendere none' adirace; Non cntrare in gara; Non
ats ti

o!

=

a

ae
mI cL

358 MALMANTILE 4) |

ti stimare ingiuriato. Vedi sopra C, 3. stan, 20. Tanto il Franzese quan:
to lo Spagnuolo Picar voglion dire Pugnere; forse da Picca;

colina

wale Omero appella nyttein, cioè pungere. Vino piccame \& que
iecda Ȣ che punga, lee they cP amma
bio; Tienle caro. ll Persiani Tesan taeda bea
Va menati l agresto 5:
Ceruellaccio peftato per Lambiceo ?

Che 'l tuo mordente ha trove poco appicco.

Di questo iv non mi picco “ct

Che s* io non ho la nobilta a bigonce, yet

Mi basta ds non esser a' undici once, (cioè bastardo) —

PICC ARS!, Vuol dir anche persuadersi, o darfia creder d' etfer eccellente
in una cosa, come piccarsi di bravo, di bello, di dotto, ec, e vale quanto esser am
bizioso, o haver ambizione. sx pee

SSE £8 ® Ss ome

ST-A al quia, Sta sodo: Non badare a quel che ella dice; enon tilasciares | tf
suolgere, o persuadere a darle da bere. Dante. State contenti, wmanagenits, |v,
al quia, ' hive ow

GRACCHI a sua posta. Gridi, cicali, eflami pure quanv'ella vuole; lasciala | yy
dire, lasciala cantare. Quand' uno vuol qualcosa da un' altro, ed ' ty
mandarglieia,¢ colui non glicla vuol dare,suol replicare a i detti di: R
chia, gracchia; quasi dica: Tanto mi muove il tuo dire, quanto il geacchiar) | ti,
d' una cornacchia. Vedi sotto C, 8, stan. 64. far Fup

TJ (tia il dovere, Ti succeda quel che w meriti. aa

SAKA fatta a pennelo, Cioè fara similissima, ed appunto come c

T” entro Mallevadore, Te ne assicuro. Ti fo sicurta, che leverai u
Fratello questa frenefia. Adadevadore e il Latino Fdeinffor, quasi afidarore, afi
curatore; detto Maiievadore secondo il Menagio, dal /evare in alto la.
segno d' assicurazione, Lo Spagnuolo lo chiama Fiador, la qual voce in
co Vo)garizzamento Toscano manoscritto delle Vite di Plutarco tra
lingua Aragonefe, refid senza interpretazione insieme con alcune altre, il)
guiva in queste tali traduzioni, o per vezzo del traduttore, o per i
gine, o perché non ne sapesse pi la. Caro mon volle il diposito, ma fiette t
tutti, wah:

NOL habbiam fatto il pane, Noi habbiam dato nel laccio. Noi i
vuro la disgrazia senza rimedio. Diciamo ancora; Voi habbiam fritto, Vou
fouo C. 8. stan. 54. sega

STANZA LXI. STANZA LXIL o
Zitti dunque, nefjun parts, o risponda + A casa lo strascina ete lo fica |

eAndiamo che e's' ha a ir poco lontano,
Così va innanzi, e l'altro lo seconda,
Oikjernitor lo segue anch' ei Piano piano,
Ma quel Demonio, che va, sempr' inroda,
Gli sente, e gli vuol vincer della mano,
Perché gli asperta,e il vecchioc'alla fiepe
Vien primoghiappa/,come dir: pepe.

7s facta,e conlacorda ve

E fatto questo a un canapol'

Che vien dal palco oueea a verta
E per pigliar il resto della critthy
Ejce poi fuora, ma ncl fate'
Che quand ei prefe q '
Ad aspettarlo havute

SS ee. ew SE FF ee eR Keke eee



SETTIMO CANTARE: 359

Soot a Selb SRRSTANZA LXE
| Edoggimai si trovano in franchigia, Sfogarsi intende,¢ a quella veste bigia
ene Vuole un po meglio scardalfar le lane,
Kabel: june, en'érantoin valigia Percio /u verso il bofeo col pennato
Che ne manco daria la pace 4 un cane; A tagliar un Quercinal va difilato.

Pigolone esortando i compagni a far romore,s'avvia con essi verso il giar-
dino, ma appena giun(ero alla fiepe, che Magorto gli senti, e prefe il Vecchio,
che era op vicino alla detta fiepe, e condottolo a casa lo ferro in un facco,¢

palco, tornd per pigliare il reito, ma non gli trovando, fen' andò

| alco 5
al hosco per fare un buon battone, col quale haveva ia animo di bastonare Pi.

 2ITTL, Cheti, Vedi sopra C. 1. stan. 10.

LO seconda, Gli va dietco: Lo seguita, Petr. Canz. 8.

pies Ed un gran vecchio il secondava appresso.

 EB spelfo in ronda. Gira per orto facendo la guardia. Ronda dal Lat, retun-

dus; dal quale è fatto il Franzese Rond ritondo.

 GLI enel vincer della mano, Vuole esser pis diligente, e più lesto di loro; gi
wool prevenire. E traslato da quei giuochi di dadi, ec, ne 1 quali il punto ugua-
le nonè pace, ma vince quello, che e il primo a tirare; per esempio, io fond il
primo a tirare, e scuopro sei; tira il secondo, e parimente scuopre sei, e se be-
neil punto e uguale, vinco io, che sono stato il primo a tirare; e questo si dice
Vincer della mano, perch colui » che e il primo a tirare,si dice haver la mano.
tanto basta ai nostro proposito, f€ bene moiti altri giuochi di carte danno questo
Privilegio alla mano.

s SIEPE, Chiudenda, o riparo fatto di pruni, ed' altri sterpi agli orti, eda
icampi, E' voce latina. Franco Sacc. Nov. 83. E giungende dove era la vigna,
qucftaera molto affoffara, e con una buona fiepe.

CHLARPA fu, come di pepe. Piglia subito,¢ senza contrasto, o fatica alcu-
na. Credo, che questo detcato sia corrotto,¢ che si debba dire: Come dir: pepe,
che è facilidimo a profferirsi, come tutto labiale,¢ di sillaba raddoppiata; e che
da questa facilita si cavi il fgniticaco di facilita in dire 50 fare una tal cosa, per-
ché'a dire; 'Come di pepe non ci fo trovar significato, o sale alcuno. Chiappare
dal pecaaere. Da Arripere fece il Bocce. Arrapare, Nella Lettera del medefi-
mo scrittay a Messer Francesco Priore di Santo Appostolo, E fimalmente can
più largo parlare scrivi, che io non doveva così subito il partire, anzi la fuga dal tuo
Mecenate arrapare, Volle esprimere il Lat. fugam arripere con dare a quel verbo
una terminazione Toscana. Così rappare abbiamo fermato da extra, € rapere.

STRASCIN ARE. Strascicare un materiale per terra senza sollevarlo,o por-
lo sopra veicoli. Lat, Trabere.

FICC-ARE, Vuol dir mettere una cosa in un recipiente con violenza dal La-
tino figere,

 CRICC-A, § intende conversazione, o compagnia di più persone: metaforico
da quei giuochi di carte, ne i quali tre figure uguail insieme si chiamano cricea,
come tre Re, tre Dame, o tre Fanti.

» | AVRIANO banuto del bue. Haurebbono havuto poco giudizio, poco avve-
be

.

———

360

SI trovano in franchigia. Si ttovano in sicuro, in luogo, dove n

refi; che franchigia intendefi un luogo immune per pri

tincipi, Lat, asy/m, che pure alcuai Toscani dico alte
ine eT

mosi di yoci nuove,dallo Spagnuolo dicono amparo,
RIMANE un bel minchone. Riman buriato, riman beffato. weno
stan. 15. si dice ancora reffare uno fivaie sopra in questoC, spo
E in valigia, Erin collera. Si dice anche im bigencia yin' nel
nel gabbione, ec, come habbiamo notato sopra C. 6. tans 41. \&
un' arnese di quoio, entroal quale si mettono cose necefiarie per la
fona, quando si viaggia, e's' adatta in fulla groppa dei cavalo, e quelli
vanno a piedi la portano in su le reni, ma questa propriamenfe si dice
NON darebbe la pace a un cane. Non darebbe la pace a Veruino; cioè
stizza, o collera, che egli ha, che se gli venisse avanti un' amico,
be come nimico, perché la.rabbia gli ha fatto perdere il conoscimento, Si dice
xn cane, o non un' altro animale, perché l'alo nostroè di dire + Wow
do guardi in viso; Non ha cane che gli voglia bene; non ha cane che lo foccorra 6p ai
t, € questo perché il cane e timbolo della fedelra', ne-si trova animale pill
liare, ed amico dell' huomo, che il cane; e pero dovendosi pigliare un'
vicino all' humanita, e profiimo al ragionevole; nel prelente luogo 5
i sopraddetti proverbi, pigliamo il cane. ta
SFOG ARS/ intende. Si vuol cavar la rabbia. Vuole sfogar ¥ ira;
all' ira, come si fa del fuoco, del fummo, che gli si da apertura,
VVOLE un po meglio scardaffar la lana A quetia veste bigia, Scardaflar'
vuol dir battere, e pettinar la lana; con denti di fil di ferro a i an.
che cara: ( dalla similitudine del cardo erba spinosa ) raffinare la lana, accioeeht
si posia fiare. Vedi sopra C, 3. stan. 60. e per metafora significa baflonare ind;
¢ però qui dicendo, vole scardaffare, ec. intende Vuol battonare ue
torna bene l'equivoco, perché par che voglia dire rilavorare,¢ di;
re la lana, con la quale e fatta la veste di Pigoione. Li Pulci nel Morgantes: ”
Adattera it bartaglio ancor dal Cielo ee
In qualche modo a scardaffargli il pelo, a
PENNATO, Coitcllone adunco, il quale serve per potar le viti, app
forte così da quella cresta, o penna tagjicnte, che ha nella parte di
nio Marcello alla Voce Bipennis dice così: Bipennis manifefium ef id we
utraque parte fir acutum, Nam nonnulli gubernaculorum partes tenuores ad D
mulitudinem pinnas vocant eleganter, Pennato ancora è epiteto, che e stato'
Latino a' volatili.. Onde tcherzando sull — » ditie 1) Boece, Gi ]
18. / vidi volare i pennati, cosa incredibile a chi non gli avesse veduti, EB n0i'
a raccontare gualche novella, per renderla più credibue, factiamo
segnito nell' antico assai, quando gli huomini eram più semplici', \&
che volavano i pennati, Palladio de Re ruftica tit. 43. discorrendo de'
deContadini vi nomina è pennati,e gli chiama falces a rergo acutas, atque laiitl,
DIFILATO. E jo fietlo che Andar di vela,di filo, addirinura,
C. 6. stan, 10, Vedi sopra in questo C, stan. 5.

ob”

MALMANTILE!D 04%

eaEFRS rere \&F Peerae

oe. pers er kro 2e028 825 5



Bot
STAN ZA LXV.

Ed ei le corde alfacco aun tratto feialte,
£ fatto quel meschino uscirne fuore,
Che lo ringraria, e bacta mille volte,

el cht del vecchio. E fa un falto poi per quell' amore,

chiufo in quel sacco iltrova pohe >. > Vi merce il can cin e guarda le ricolte,

oe. 4 mal por! Dandogti aint, ed ezli se il servitore,

Poi con i piatri ye pie vasi di terra

Due fiaschi di vin rojo, e lariferra,

LxVL
Quando Magorto in già viene a ricifa
Con una fhanga in man cotanto fara,
fesse crspands delle rifa Perchee gli par mill' anni con quel tronco
wove con quegli altri firimpiatta; Difar vedere altrui ch' ei non è monco,

0, che stava naicosto a osservare, veduco partirii Magorto, corse alla

9 e trovato il vecchio nel sacco jo cavo., e vi mefie dentro il cane con

di terra, e duc falchi di vino, e rattaccatolo come stava prima si na.

oo vedde venir Magorto con una grande stanga in mano.
felice, ' paroia di commiserazione,come meschino,¢ simili.

YANDOS! 4 mal porto, Trovandoli a cattivi termini.

arrucolada pozzo, Carrucola e una catiecta di legno, e tal volta di fer-

alla quale e impernata una gircila scanalaca, e (ope'a tal girella s'a-

, o catena per tirar fu pefi con facilita, e questa carrucola si tiene co-

ente appiccata al pozzo per tirar fu acqua, ed il moto, che fa cal girella

ta cagiona per lo più strepito, al quale il Poeta atiomiglia i sospiri,
; 'igolone.
ae SFA fais, per quell amore. E' un detto faceto, col quale s' esprime la gran-
a, e contento d'alcuno: E tal detto viene da quzi Ciechi, che per
i Popolo fanno nelle piazze giocolare i cani, e fra gli altri giuochi gli

s e al bastone con dire: fa un falto per amor d' un pane, ed il cane tutto

» o per il contrario dicendogli; /alta per uaa mano di bastonate, il ca-

ein atto di mordere, e non saita; ed il termine per qucil' amore figati-

lazione, O in riguardo; come Lo fo la tal cosa per amor tuo, s! in-

bh tende Io la fo in riguardo, o a contemplazione tua per l'amore ch' 10 ti porwo,

 SERATT-A, Vedi sopra C. 5. stan. 13.

flere f delle rifa. Rider gagliardamente. Rider come fece Margutte, che

baenpp:s secondo che favoleggia il Pulci nel suo Morgante; Ll' verbo

a altro vuol dire allentarsi gi' inceltini, vale anche quaato /eoppiare,

parities pur si dice: Scoppiare,¢ morire dalle rifa, Ed \& quel re quati che

“habbiamo decto sopra C, 3. ttan. 65. Li Pulci nella Beca dice:

Petty wit ' Ta fet nel letto, e crepi dake rifa.

st enone Sitorna a nascondere. Vedi sopra C. 20, stan. 60. e sotto C.9.

bis he fa cht ek s* appiartd miffer gli denti.

ia era i emi a Trattaco —— dice: Quejte cose ho cavate da un six

bro

oraeeaal



362 MALMANTILE —

bro del Comune, che fu impiattato da uno de' Buonhuomini,¢
4 ricifa, Senz' intermidione; senza fermarsi, a p
difilato detto poco sopra Octava 63. antecedente. I) Pulein
ES io mi metto a cantar a ricifa, a
COT ANTO fata, Grossa in questa guisa. Vedi sopra C. 5.
stan. 36. Tam
Par veder, ch' ei non è monco, Far conoscere ch' egli ha le mani; 0
non ha mancamento alle braccia. Jonco vuol dir uno che ha manco
tutte due le mani. Lat. A¢ancas,
STANZA LXVIIL.
errriva in casa, ¢fbracciafi,.€ si mette Ed ei, ch' e 1 fusse furie non vi
( Serrato V uscio ) con il sue randello Che infin.ch' ei non fisfoga
Sopr' aquel [acco afar le sue venderte, Sta intato il vecchio all'uscio,
Suonanao Zuant'ei pao sodo a martello, Ad origliare per udir qualedfa
Ll Romito che stava-ale velette, E sente dire: O lecca
Perché? nscio hadi fuora il chiavistello Carne feantia, barba pi
Andi ( benché tremando,e con spavento Ribaldo, Santinfizza, €
Che havea di lus) e ve lo ferro drento, C'aquel d' altri pon cingue, el
STANZA LXIx, 1

Guardate qui la gatra di Masino, Ma quel' hai toltoa me,
Che riprendeva il virio, ed il peccato, Won dubitar ti costera fa
+Se il monello-ha le man fatte a uncino Che tante volte al pozzova'
Per gire a [erafignar pel vicinato? Ch' ella vi lascia il manicog

Magorto, arrivato a casa, si messe a bastonar quel facco, credendo che vi
fusse dentro Pigolone.; Ma questo-essendo uscito di-casa mefle il ci
di fuori alla porta, e fermatosi alquanto quivi, senti che Magorto mn
facco gli diceva una mano d' improperj. win ti
SBKACCLARS!, Vuol dire Denudarfi'il braccio da mezzo in giù te |
mano come accennammo sopra in questo C. stan. 19, B sbracciarsi; ee
mente parlando vuol dire Impiegare ogni sua forza, diligenza, ed mol
in un' affare. Lat, mamibus, pedibu/que eniti. want 1
SVONANDO a martelio. Cioè bastonando. Suonar' a martello si <<, m
do la campana suona a rintocchi., come fa il martello full ancudine, ii che i | %
quando si vuol ragunare il popolo per li bisogni della Città. Il verbo fumaretil | \&
Latino puifo, vale appretio di noi, come appresso i Latini per suonare, e per |
perquotere. Vedi sopra C. 3. stan. 7. aie
ST AVA alle velette., Stava osservando. Veletta, o vedetta diciamo
to, che sta in fusse mura d' una Città, o Fortezza a far la guardia d
munemente/entinells., edil lwogo dove sta detto soldato si dice velerra
Sumo che sia trasiato da i Marinari, che tengono la detta guardia
albero della nave, e dicono metter l'huomo aila vela, o veletta forse
piccola vela, che sia in quel luogo. Tarcagnotta Stor. lib, 5. p. 3. 7
Partitosi pero il Priore Strozzi da Marfilia con 2 3. Galere, ed una g
welette in mare lo venne ad sacontrare. Dal che ficava che si chi:
-gune barche, le quali camminino avanti a una armata con huo


SETTIMO CANTARE. 363

le, opure da vedere vederta'e poi corrottamente veletta. Si come da /pecio anti-
¢ Latino significante lo veggio » si fece /pecula luogo eminente che signo-

reggi molto paele. Ma sia come si sia basta il sapere, che stare alle velette vuoi
dire Stare a osservare.
| Bin fale furie, E'colmo d' ira.
ORIGLIARE, Star in orecchi, Star a sentire, e vedere con attenzione, edi
iy! costo.Pranzele oreillier. Spagn, otear forse dal Gr, Ora,orecchie, che i Fiaa-
fini spiega:/piare, eguardare da suogo aito, come fanno le sentinelle.
 PEVERADA, Brodo di carne, o d' altro, E /ecca peverada vuol dir Brodaio,
se Beiniignifica porco, perché il porco mangia volentieri ogni sorta di broda...

 War, St, Fior. lib. 14. dice: Gli diede una mineffrina bolita, cotta in peverada di

- pollo. Detta Pewerada dal Penere, cioè dal pepe, che per dar (apore si metteva.

fa le mineftre, come fu da altri dottamente osservato.

CARNE stantia, Carnaccia vecchia, e frolla. Vedi sopra C. 3. stan. 24. e $4.
ye  SAKBA piattolosa, Termine ingiurioso per un vecchio;¢ vuol dire barba schi-
i 2, epiena di pidocchi, e d' altre lordure,

SANTINFIZZ A. \pocrito; de i quali a bastanza s' è detto altrove; EB per
yy  satinfieza's' intendono certi Torcicolli, che stanno tutto il giorno d' avanti a
 una immagine d' un Santo, perché si creda che essi facciano orazione.
yi, GABBADEL. Rinncgato. Uno che gabba, cioè inganna le Deita, adoran-
fio, Oggi una, e domani un' altra, rinnegando la prima. Se bene Deus non ir-
yal 'Tetur. Si dice ancora Gabbafanti. §
ay, PON cingue, e teva sei, Vuol dire Tu sci ladro; perché ponendo cinque dita
we della mano, fai il numero di sei con aggiugnere alle cinque dita la roba, che
sath porti via. Plauto disse: Trism literarum Homo, cioè tees Habbiamo diversi
modi di dire copertamente Ladro, come Sgrafignare. Havere le mani a oncim,
: che si vedono nella presente Otttava 69. Bespemmar con le mani, Andar aCarpi,
¢ 64 Borfelli., Par il Lanzo ( che in lingua lanadattica vuol dire Ladro ) gixocar, 0
 lavorar di mano,¢ Gimili.
i 'è LAgatia di Mafiro, Questa fingeva d' esser morta,¢ nen era,e però vuol
è " dire huomo finto. Huomo che fa il semplice,e non è. Lat, Lepus dormiens, Te-
nere gli occhi aperti, haver L occhio, ed aprir l'occhio vuol dire andar cauto nell'
Operare: e perché tanto la lepre, che il gatto tengono gli occhi aperti anche dor-
mendo, servono a i Latini, ed a noi per esprimer un' huomo vigilante, cd ay-
yeduto, e che mostri di non essere. Vedi sopra C, 1. stan. 19.

MONELLO. Così chiamiamo quei guidoni, che per Firenze battono mari-

 Ma, comes'è detto sopra C. 4. stan. 8. Siccome Guidone di nome proprio fiè
fatto appellativo, così forse anche Monello, in principio diminutivo dt Adone,
accorciato dal nome proprio di Simone è venuto a significare una tal razza di

persone.
'. ASSASSINO. Vuol dir ladro di strada, ma quiè detto in vece di furbo,o
' »¢ può anche intendersi ladro di strada,
NON dubisar ti coferd falato. Sta sicuro, che ti ha da costare assai, o che ne

-pagherai un gran fio.
— TANTO va la secchia al pozzo, ec, Tante volte si torna a fare un male, che
Seri tey ' Zz

2 una
i
i ~
"

ha ee a

364 MALMANTILE o
una volta vi si riman colto. Una volta' fa per molte; e diciamo ancora; Tate ) wij

volte va la gatta al lardo, che unavolta vi lascia la zampa\, ec

violantium malus est, ed orecchie della secchia diciamo quelle due | tL

rate, nelle quali e infilato il manico di efia (ecchia. sear Ne at
S 4

TANZA LXXx. STANZA LXXL ai
Poi sente, ch'egli dopo una gran bibbia Ben ch ci creda finua '
D ingiurie dd nel facco una percuffa, Tira di nuovo, eda vicino
Che rurte le froviglie /perra,e tribbia, Ed il suo cane acchiappa i
Ech eidiceva; Horsuglihorottol ofa; Che fa-urliche van nell
E che di nuovo un' altro ne rafibbia., » Ona' egli fiupefarto afjai ne
E che ( facendo il vin la terra rofja ) Dicendo: Qui è quand iomi
Soggiunge:O quanto/ague banelievene! Se nce' il sangue egli ha di gi
Quella ghicttene, a me, beeva bene. Come a gridar puo egli
Seguitando Magorto a dire ingiurie, da una bastonata in sul facco, €
i piatti, e fa verlare il vino, e credendolo il sangue di Pigolone resta
to, che ne posia haver tanto; € replicando un' aitra batlonata, ae
po il cane; 11 quale cominciò a urlare, ed ei credendo, che fuifero strida di
lone, strabilisce e non retta capace, che egli pola haver più forza di 7
ara

ae

PoetEstEe

frida, mentre ha versato tutto ul sangue.
DOP PO una gran bibbia, Dopo una lunga diceria, o filaftrocca;

Dopo haver dette tante ingiurie, che farebbono un gran libro, da Biblia Greco
Latino, che vuol dir br:; E se bene la voce Bibbia oggi comunemente e istela
per il libro della Sagra Scrittura, cuctavia noi la pigliamo ancora ne i cascome
il presente nel detto sensodi libro, o di lettera, o di discorso lungo, come pate
che la pigliaficro gli antichi secondo Herodoto lib, 1. dove dice > Alarpagum
clufife, leporis ventri biblion.ad Cyrum, Se bene qui e Viguerro, letrera, Dal po
ma d' Omero intitolato l'Lliade, il quale è d' una prodigiosa quantità di vert,
come quelli, che ascendono al numero di quindicimila (etcecento oreantatre; ut
gran moltitudine di cose, o di parole, dissero i Latini 4ias, o Hiades, Propeaid
41.2, clegia 1.

oe rRs oe ST

Tune vero longas condimus Iliadas,
Seu quicquad fecit, fine est quodcumqne locura
Ataxima de nibilo nascitur bifforia, ne

RAFFIBEIA, Replica. 'Irasiaco dal congingner con fibbia bottoni,
il che si dice -4fibbiare, Vedi sopra C, 2. tt. 81.

STOVIGL/A, Intendiamo ogni sorta di piatti, e vasellami di terra per nfo di
cucina. Ll Ferrari, Seovigue, Fittiia, vafenla, \& frivola. Vandena, to
comperi. 1o stimo che sia parola florpiata dalla Latina. Veenfilia, Crefe, 12:12.
E molti altei arnesi,e fovg 4 di bilogno. Pallad. volgarizzato lib, 1. tity 6 Faber
da far terramenti, e dilegname, e di /ovigli da vino, da Javorare, eda
Questo ultimo non è nel Launo, ed è aggiunto nella traduzione per impiegate
voce Mowtgli, 7 3

TX/BSLARE, Propriamente vuol dire Batter il grano in fulltaia dab Latino
Tribula tribule, o tribulum tribuli, che vuol dire una specie di carro, ¢ol già
fquoceva il grano in fit!" ata, come si cava da Colum. 'lib, 2, cap. ie

@uifa eo PEEP ~ se Ee oe eT



SETTIMO CANTARE. 365

'unt adijcere Tribulum, \& trabam posis, e Varr. lib. 1.C, 25. E'/picisin area

cM tivwencis iunitis', o tribula. B questo dal Greco eribein peftare, trita-

. Latino terere, o da thlibein schiacciare, dal qual verbo viene il Latino trib.

travaglio  dettoanche da' Santi Padri prefura, '

£. Questo termine significa A mio giudizio; Secondo me. Secondo il
4/0 intendimento; e per ie si dice replicatamente 4 mé a me, Quan-

» cl0è per quanto io giudico i Franzefi Quant' a moi, 1 Greci similmente

» cioè secondo me, secondo il mio giudizio.

DE haver finita la fefia. Crede haver terminato il negozio, cioè d' haver'

Pigolone, Similitudine trata dalla folenaita, colla quale son facti

i, che si giuftiziano.

CHIAPP A. Coglie: perché se bene «cchiappare vuol dir pigliare uno con
¢ violenza, ci serve er esprimere colpir bene. Latino Certo ietw a/-

Spagnuolo, acertar, Vedi C. 2. st. 41.

EF ATTO. Rimasto stupido per la meraviglia grande. Latino ob/upe-

STANZA LXXII. STANZA LXXIIL
in questo mentre:col suo fante Perch'ei del certain quanto a contentarla
 Haven di già scorrendo pel giardino Non ci ha ne meno un minimopenfiero,

it pritrovato, e quelle piante E pero quante volte ella ne parla,
we coles, che chede sl suo Nardino, Mura discorso, ela riduce al zero;
vot! tha trata fuor belle galante, Ma perch'ellae mozzinaye con laciarla
tif be mon si vedde mai il più bel fennino, Le Afonache trarria del Monaftero,
Econ un sno bocchin da sciorre aghetts Vedeyche s*ella bada troppo a dire
i  Chiede da ber ma non già fel' asperti, Si lascerebbe forse connertire,
" av ' STANZA LXAlV.
wil) Peri per non cadere in questo errore E ch'ei ne venga ch' ei l'aspetta fuore,
st | Lapiglianun tratto,efe la portain frrada, eAccio con essi anch' egli se ne vada,
yl Ed ai vecchio fa dir pel servitore, Che i non vuol lasciarlo nelle pefte,
* Che'pit tempo non è di frar' a bada, 44a condurlo al pacfe alle lor felke.

“Mentre che Magorto si fludia a bastonare, il favio Brunetto col servitore era
andatoinell' orto, ed havea trovato il Cocomero, e tagliatolo n' era ulcita la
fanciulla 'che egli cercava, la quale si mefle a pregario, che egli l'empictic la
tazza, maei non volle contentarla, anzi la prefe, e la porto in strada, e man-
dO i teruidore a chiamar Pigolone per condurlo seco alle nozze di Nardino.
a ANTE, Si dice i) servitore; dall' intero infanre,si come in Latino Puer signi-
" fica ferno, da noi detto anche garzone, se ben Fante però comunemente vuol dire

" - soldaro'a piede, perché ne' tempi dell' Imperio baflo, che la milizia cominciò a ri- |
of a tarsi più per la cavalleria, che per la soldatesca a piede; il pedone si venne as |

'ttimare come ministro., e servitore del Cavaliere; e perciò fu detto fanre,
| SENNINO. E' una parola, che si dice per vezzi a una femmina bella, favia,
~¢ pulita, e che operi cen giudizio con senno,¢ con puntualita. Latino scita pue/~
la,feitula. z
~~ BOCCA dat feiorre agherti., Così diciamo di quelle femmine, le quali per parer
“belle tengono la bocca ferrata, e ridotta forzatamente pi Mretta del. suo nau.
ss rale;



366 MALMANTILE o

sale; ne muovono i labbri di come se gli sono accomodati allo specc
par proprio, che habbiano la bocca accomodata a feiorre un ng
Aghetto è quello, che vedemmo sopra C.2.f. 10,
'NON se? aspetti. Non lo speri. Cioè non asperti, che le dia bere « |
gnuolo ¢/perar e lo stesso, che a/pettare. bes pe a My
LA riduce al vero, La riduce al nulla; Zero quella figura d'abbaco, che
se stessa non rileva numero alcuno, ed accompagnata, forma le decine, e eile
per esprimer  nuda, '
eHOZZINA, Huomo aftuto, triflo, e che fa il conto fino, mas'inte
genio maligno. Latino Vulpis reliquie. Questa voce vien forse da orecehi m
che così son segnati quei furbi, che meriterebbono le forche, ma perla
eta non ne son capaci, sopra C. 6. st. 54., ed in questo C, st. 30. Ȣ credo
perché diciamo Azoyzorecchi in vece di mozzina nello stesso significato,
TRARRIA le eAtonache dal eAionaftero, Conseguirebbe l'impossibile con fa
sua induftria, periuafiva » ed cloquenza. Diogene disse: Oratio non ex ani

Sfn2i* 8

proficiscens, sed ad gratiam composita meleus est laquens, quod [cilicer blandé x
ens hominem ingulet. té
NON è tempo di star' a bade, Non \& tempo di trattenersi. Non v' étempods | \&
erdere.; R
LASCTAR' uno nelle pefte. Abbandonar' uno nel pericolo, Vino fa' R
folenza, o mala creanza, e per non esser percoflo fugge viaye lascia i Ra
€ questo si dice /a/ciar nelle pefte, cioè nelle pedate, o nella strada, che è
mancamenti ha fabbricato ai pericolo,colui che è fuggito;si pronunzia cont ay
ma c stretta a differenza di pefte infermita, che si pronunzia con l'è lagaies | 4
pero questa rima ha un a di falsità, ma tollerabile, ed¢ ammessa. \&
STANZA L\&XV. STANZA LXXVL
Così di la poi ructi fer partita; Brunetto si ridea di Pigolone mm LA
Ma piis dogni altro allegra la faciullay Perch' ei parea nel viso un fico vittty | %i
Perché non prima fu dell' orto uscita E menaua a due cambe di ee pth
Crognt incanto,agnt vogliain lei s'anulla, Com' egli havesse hauuto i Birri drt; aay
Anzi ai lor preghi in sul caval, Salita, E la donna diceva: Grambracont, hey
Che la duri; ed il vecchio manfuttt, | 4

Senza più ragionar di ber, ne nulla,
Va sipreinnazs ag altri un trar di mano Che si vedeua fatto il lor xsmbello:
Fiera, e bizzarra come un Capitano. Dagli pur (rispondea) ch'eglie fafitl. | "sp
Vicita che fu sa fanciulia dell' orto cessd incantefimo, e la voglia del bere4y
anzicon la maggior' allegria del mondo monté a cavallo scherzando, € moe | \&
teggiando il vecchio, il quale era ancor pailido per lo spavento havuto, i)
"RIZZARKO. Wuol dir lracondo, Suzzofo, o cosa simile, secondo chelule | \&
rono gliantichi, Ma si piglia anche per spirito(o, e vivace, come è !
presente luogo. In Spagnuolo Zixarro significa uno che vada bello, e fupecbo nel Ge
veitire. B similmente roba bizarra, che 1 Pranzcfi direbbero bigearree, ie) 4
roba, cioè vette bellissima, varia, e pomposa, donde poi da noi si prende Bare |
ro per capriccioso, strano, stravagante. eae Bk
FICO vieto. Fico annebbiato, o afato. Un fico, il quale al colore, e tene
rezza par maturo, non è, ma dalla nebbia è ridotto giallo, come se fusse ma:



SETTIMO CANTARE? 367
furo: comparazione, che esprime assai bene la faccia gialla, e grinza di Pigolo-
ne. El'epiteto Viero e proprio della carne falata, lardo, burro, e olio, quando

. per essere stantij, e corrotti mutano il colore, l'odore, ed il fapore.
, MENAR di spadone 4 due gambe. Fuggire; Correre. Spadone a due mani si
quella pada più grande delle spade comuni ordinarie, la quale s' adopra
-ambe ie mani, e per derisione di coloro, che, vantandosi di bravi, all' occa-
poi fuggono, col solo dire; meno di Spadone, o gioco di spadone, s' intende a

ye gambe, che vuol dir Fuggi. Vedi sotto C. 10, st. 3.
COM egli havefe havuto s Birri arety, Detto usato per esprimere, che uno
corra velocemente

GIAMBRACONE, che a duriDubito, che voi non fiate per durare a cammina-
re. Giambracone fu un mato, che sempre andava gridando: Che /a duri, e»
| però quando noi veggiamo, che uno faccia un' Operazione con grande attenzio.
'Re,€ che noi dubitiamo, che egli non sia per durare fogliamo dire Giambracone,

an) o (enza dire, che /a ders intendiamo; piaccta al Cielo, che egli continovi, € così \& Co-

inteso.
BATT O il loro Zimbello. Divenuto lo scherzo. Zimbello,oltre al significato,
i @ o sopra C, 1. st, 59.,vuol dire aacora quell' ucceilo, che si lega per
un piede allato al hoschetto de' paretai, o altri luoghi, dove si tende per pigliare
ig uecelli, che tirandosi quella cordicella, che ha legata al piece si fa suolazzare
Per incitare gli altri uccelli a calarsi. Latino amis illex, € dallo strapazzo, che
ry tale uccello riceve diciamo Zimbellouno quando e burlato, beffato, e strapazato
ad da tutti; nel qual senso e preso nel presente luogo; e sotto C. 9. st. 66.
7 — DAGLI ch' egli¢ faffedo. Dag, ch' ci lo merita. Olleruifi che 1 verbo Dare
nei cafi come i presente,vale per continovare, seguitare, durare, ec. e con dire
rin solamente dages icnz' altra aggiunta s' intende /eguita; ma s'aggiunge ch' egli e faf-
Selle per una certa vaghezza, e per un genio,¢ naturale inclinazione, che han-
N01 Fiorentini'd: paciar per proverbio, metafore, comparazioni, o similitudini;
i

- € forse e aggiunto per confondere,ed oscurare il detto,perché dare al fafedo vuol
se dir perquoterio, e nov vuoi dic seguitare. Habbiamo due specie di tordi, cioè
: botraces ye fafjedi, 1 primi son meno aftuti, e più facili a lasciarsi pigliare, i fecon-
| di sono più aftuti, e ad ogni poco di romore scappano, pero quando la notte col
 "s frugnuolo si scuoprono, si dice dagli con la ramata, che qucsto e fa/sello, che alpet-
«' ta poco. In fuftanza nel presente luogo vuol dire continuate, o Seguitare, a burlar~

i mi, beffarmi, e strapazarmi, ch' io lo merico, Da questa aftutezza del faffello si di-
a si S¢ fafsello a un' huomo, che sa il conto suo, ed esercita il suo sapere a vantaggio,
my" pretendendo sempre più del giuflo, e del dovere, avido di guadagnare, € tenace
* el suo più del conueniente.
w STANZA LXXVIL

us Così feberzando, com io dico,in briglia Percio dopo haver fatte molte miglia,
i Ne vanno Lenya mai sentirsi franchi, E che tor parue un tratto d'esserfrachi,
) Esempr' ognun pin calda se la piglia, Tutts affannati per st lunga via,
wo. Percheilcimor glispinge,e /pronai fachi; D'accordo si fermaro a un'Osteria,

il
ty
 STAN.



MALMANT DLE) S G0 —

368
STANZA Laie au a
Dove il padron che intende fare. pasto >» Ben. ”, '
Trovagran vroba per yi garbato, 'Guamioiinfa ha a0,
Chreitien che afar no habia rroppornaspo, E che quella »
Mae? non fache enon hanno desinate; Che's,

Brunetto con la sua compagnia seguita allegramente 'il suo vi
do per il timore, che hanno di Magorto, ma sti ia 7
un'Osteria, dove mangiaron più di quello, che il padrone non: z

SC HERZARE in briglia, Questo detto, che significa uno, che stando|
faculta, e d'ogni commodo, non ostante si duole dello stato fuoy éd
anche per intender'uno, che stia allegramente,¢ scherzando fenzae
che egli è in grandissimo pericolo;¢ così s'intende nel presente luoga,.

scherzano senza pen(are al pericolo,nel quale sono » che Mas arrivi
dosso, da-chanhp, pea

OGNVNO se la piglia più calda,~Ognuno se ne piglia maggior
sto pigliarsela caida i Franzefi esprimono col verbo chalsir, e noi cal
dal Lat. calere; Boccaccio nel Poema in ottava 'rima intitolate il

de' fatti di Tefeo 1, 2.
Oude li se nuova vifion vedere; s OSes i
Perché di ritornar li fu in calere. 2 Mey
E appresso. Vici d' Atene, ne li fu in calere, ae
D' Ipolita amor dolce, e pudico. me
Spiegd la forza di questo verbo il Petrarca quando disse } ee

We dentro sento, ne di fuor gran caldo;
Che fa come una spiegazione de' due versi immediate precedenti:.
Ne del volgo mi cal', nedi fortuna; oy gaialah
Ne di me molto me 'di cosa vile, ome
GLI parne d' esser franchi, Parue loro d' esser in sicuro, ed esser liber da Mo

orto. hoped
OO ARE 4 pasto. Si dice quando l'Oste senza prezzare cosa per cosa di quell

che mette ia tavola vuole un tanto per persona, e mette in tavola quello yee

are a lui.
f NON habbiano a far troppo guasto. Non habbiano.a mangiar molto, Le.

aPEFER

feo incognito dice.
lo ero fario, e non sei troppo guasto,
Il Berni in lode delle pesche +
Dioscoride, Plinio ye Tecfrasto
Lon hanno scritto delle pesche bene
Lerché non ne facevan troppo guasto,
Cioè non ne mangiavano molte, perché 'non gli piacevano.
V? \& rimasto, L' ha fgarrata. E' rimasto ingannato,, come chi
trappola. ssf
LON vi resta fate. Non vi resta nulla. Vedi sopra in questo o, stan. 7}
Mattio Franzefi contr' alle sberrettate dice;; cal
“Hed
Ae

FREER

we

ss,


SETTIMO CAN TARE 369

A cavarsela, e metter più di cento
> *Folte per hora, +l che non serve a fiato,
va dietro alla casserta, Cioè non si gaadagna, ma più tosto si perde,
TANZA LXXIX. - “STANZA LXXX,
sntante | frracco S' 16 percossi quel vecchio marivolo
wil randello a quel partito, Com' ha io fatto, disse, un canicidio?
'ciolte,ed apertohavedo omai quel/acco Sa ch' io lo prefi ye la ferras qui solo,
sencinar la carne del Romito, Che gnun porea vedermi,o dar fastidio,
Ed in quel cambio vistovi il suo bracco Won fo s' o sono il Grasso Legnaiuolo
coceh, vetri macolo,¢ bafito, A queste metamorfofi d' Ovidio,
fa miaravigliato in una forma Che sono in ver meranigliose, e frane

Ct ei non fa s'ei sia desto,os'ei si dorma, Poi cnn Romito: mi dinenta un cane,
; STANZA LXXXL
e'| povera Melampo Lo ho una rabbia addosso ch'io avvam:
Che | nette gua tencé oui Jaci; Con quel veechiaccio barba d'Olo sae

Chi più farò la guardia al mio bel capo Ch al certo fatto m' ha così bel giuoco;
defio, che t' hai cliinfe le lanterne? Che dubbio! metcerei le man nel fuoco,
eo Magorto'dal bastonar quel facco lo spiccd dal palco, ed apertolo vi
il suorcane; ¢€ restando maravigliato, fuppone che sia stato Pigolo-
li habbia fatta questa burla.
ere In quella guisa; in quella forma, in quella maniera,
kntendi frammenti di piatti, pentole, ed altri vasi di terra,
Pe.Badro, giuntatore. E' voce Napoletana,'ma già facta Fioren-
tina, A 4 7
CHE gine pores darmi fastidio, Che niuno poteva impedirmi, La voce gaxno
per nino, hogei @ usata solo da 1 nostri contadini,
NON fos' io sono 11 Grafo Legnainolo, Non sos' io mi sia diventato ur' altro, il
'Graflo Legnaiuolo fa vn Fiorentino, il quale fu tanto semplice, che gli fu dato
@ credere, che non era più lui ma diventato un' altro  e per questo tale fu mefio
! e alloppiato, e fatto dormire quando si rifenti, s* accordé a paga-
Te le tpee je le cancellature per il precefo delitto, del quale fu affoluto, benché
havesse Confeffato'd' haverlo commefio come nuovo perlonaggio, e pagd il dena-
10 un fratello di quello, che il Graflo si credeva d' essere, e duro in questa cre-
  denza qualche temipo; e fin che li suoi veri parenti lo fecero riconoscersi, e ritor.
share: che egli era. La Novella pare a me, è stampata dietro alle cento No.
vellea dell' edizione de' Giunti. Da costui digiamo i Grass Legnaiuolo per
intendere un' huomo semplicissimo., e facile a creder ogni-cosa, bench' ei sappia
non esser vera, ed esser' imposiibile, che ella sia. Si dice ancora Calandrino, ¢
Cap,,comie aecennammo sopra Cy 5, st23. t,
VE Romito mi dineneaun cane, Se bene intende, cheil Romito era diventato un
'caN';/perché nel! facco trove il cane, vi haveva meflo i) Romito, si potrebbe
'anche 'che intendefie parergli gran metamorfosi, che un Romito, cioè ua'
i bene jdiventi un cane, cioè nno scellerato. i a
- MAL chitfe-ie lanterne'; Hai chiufi gli occhi;-ed.intende sei morto,, Chiamanfi
Anche gli Occhi sccicanré'in lingua furbesca; € Così li chiamo in un verio del suo
fio Brunctto Lauini Macftro di Dante. Aaa 10

an

+225 28

wi tie Ea =



370 MALMANTILE o\&|

10 ho una rabbia addosse ch' io avvampo, Latino Jn fermento totus si
collora, un' ira grandissima. vvampare significa abbruciare leg)
cfempio; Un panno bianco accostato a una fiamma s' infuocola,e piglia
si dice arfo, o abbronzato, o avvampato. »  Sie

BARBA d' Oloferne. Barbaccia. E' nota la Storia facra di Iuditta,
la testa ad Oloferne. Nel pepeeiones detta storia, li Pittori per far
Oloferne per un' huomo crudele, dipingono la di lui testa tagliata brate:
barba lunga, folta, e rabbuffata; e da questo il dire a uno barba a' Olofer
giurioso, perché suona anche lo stesso, che resta d' impiceato, “

\&8\& set ise

wr

METTEREL la mano nel fuoco. Mi par d' esser così certo di gaefta cosa, cheio | »,
la giurerei con metter la mano nel fuoco. Uno de' giudizzi, che chiamavano' wi
vini, appreflu i Safloni era la prova, che faceva il reo, via del suo > 1
nendo in mano ferro infocato. E le solennità, colle quali si veniva a qu -
va, sono descritte puntualmente dietro all' Istoria clica di Polidoro Vi j "

TANZA LXXAIL STANZA LEXXIV be
Oimé le mie stoviglie, e il vin di Chianti Ma perch' ei vede quivi le a Ti

Chrio tolfi in dar la caccia aun vetturale Volte al giardino,e poi versolavity | Gy

eA cagion di quel tristo Graffiafanti CheBrunetto,equeglialtri: tu

in um tempo e versato, e ito male, Quando v'entraro,e quando andarovia R

Giuroal Ciel ch'io non una ch'ei se ne vati, Infospettito, lascia andar il Frate, Pe

E,s'ei non vola, puo far capitale Ed entra nel giardino,e a, i

Chr io voglia ritrovarlo,e s'es c' incappa Scorge quel suo cocomero dit an

Che mi venga (a rabbia s'ei mi scappa, Ch'e frato il fargli un fregio sopr w

STANZA LXXXIIL STANZA LXXXV, | 'y
Lo troverò bensì, perch' io vue ire Poiché levata gli han quella figliualay | x

Qua intorno per veder s'io lorintraccio; Chiin essa(cons' io ho dette) si trouwvs, |»

Cos} corre alla porta per uscire, Per la stizza non puo formar )

Ma cei nd puofarlo,perché e' v's il chianaccio Si soraffia, barre s denti, efa z

Lo fquote, e shatte per volerlo ape eS E spalancando poi tanto di gola dey

Edhor v'attacca 'uno,hor altro braccio; Verla,befemmia il Ciel, z ne

Noiato al fine vanne,e corre ad alto, Dicendo; QO Macomettoe ¢

E dai balconi in frrada fa un falto, Che si facciano al mondo i")

STANZA LXXXVL mime | i
fa quanto a te chi ti pisciasse addosso Sapro ben' io a costor far shy. «thy \,

So ben che th non ne farefti cao; Credilo pur, percht, se si da il case h

Ma io che da miei di mai bevvi grosso, (Che si dara feny'altro) chia, eT i

E le mosche levar mi so dal naso 4o me gli vue di posta ingoiar vit, t

Seguita Magorto a dolersi della sua di(grazia; poi fata risoluzione d'è ty
cercar del Romito, falta dalla finestra in strada, dove vedute alcune. kj
fo il giardino, infospettito lasciò il pensiero d' andar cercando di Pi bry
ne va alla volta del giardino  e quivi accortosi del ratto della fanci u
di voler trovare coloro, che gli hanno fatto questo torto, e di, volergli tu \&
goiar vivi. Nota che il nostro Poeta in qn ottava 84. e stato cri dy
ché s'¢ servito della voce ia in tutte tre le rime, ma tal fottigliezza fb iy
tosto chiamare ignoranaa, perché se bene \& sempre la stessa voce, “—— 7 j

SETTIMO CANTARE: 371

sempre diverso significato, perché la prima significa strada; la seconda significa,
altrove, o moto da un luogo a un' altro, e la terza significa modo, guila, ma-
~niera, ec, E di simili rime troverai altrove in questa Opera, e sempre le vedrai

 lodevoli per l'artifizio, più tosto, che biafimevoli per ta poca avvertenza.

- AOL. Esclamazione, che esprime disgulto, o dolore. Latino Hei mibi;
- CHIANTT, E' una regione in 'Toleana dove nasce vino buonissimo. E Vettu-

 intendiamo colui, che sopra alle bestie conduce vino, ed altre robe da un,
ogo all'altro; a differenza di Vetvurino, che presta, ed accompagna caval-
a lettighe, ec, a i Viaggianti. Vedi sopra C. 6, tt. 37.
| DAR fa caccia, Correr dietro a uno. E propriamente si dice Dar /a caccias,
 quando i birri corron dietro a uno per pigliarlo.
giù = GRAFFIASANT!, Bacchettone, lpocrito, E' lo stesso, che Santinfizza det-
igi to sopra in questo C, st. 68.
più PVO? far capitale, Può esser certo. Qufta voce Capitale significa lo stato, o
Ill faftanze d' uno: tale ha 10, m, fendi di capirale. Significa aflegnamento. Chi
yt del mio fn capitale detto sopra C. 2. st. 7. Significa forte principale. Latino Sors,

i detta
i

,

yah

qs dat Greci cephalaion, cioè caput; dagli Spagnuoli candal, che corrisponde»
niall nostro Capitale, e Candaloso dicono colui, che ha gran capitale, cioè grandi
we = fallanze. U/ rale ha havuto la sentenza contro, ed è Stato condennato nelle [pefe, ed 4

are cento fendi di frutti, e mille di capitale. Significa quello vedremo sotto C. 8,

u@ 1.65. Qui significa può credere, può esser sicuro.
jah «SET c' inciappa. S' ci mi da nelle mani. Se, c' incoglie. S' egli casca ne' mici
ei) Meguati

i) UI venga la rabbia, Giuramento imprecativo contro se stesso. Giuro di voler

yj far latalcofa,, e se non la fo, mi fottopongo a ogni maggior tormento.

ait 8" 10 to rintraccio, Traccia significa orma, o veltigio; onde tracciare vuol dir

«at 'seguitare le pedate, e per confeguenza qui intende: Se io lo ritrovo; Traccia si

iti dice quella strada, che fa il cane per la passata della lepre, o d' altro animale

im fiutando; viene questo verbo rintracciare, che vuol dir Ritrovare, e rraccia-

jus) ecetcare, Latino vesticare. x

ynt o CHLAFACCIO. Elo stesso, che chiavistello detto sopra C. r. st. 69. che i Sa

se nefi dicono pestio dal Latino pefu/us. 11 Conte Vgolino preflo Dante Inf. 33. Ed
io fent) chiavar ? nscio di sotto all' orribile torre; cioè mettere il chiavaccio.

rit «= A QVELL Avia. A quella foggia. Inquellaguisa.

ull PARGLI uno sfregio in sul viso. Fargli una ingiuria ignominiosa, si come sono

iii Bl sftegi. Vedi sopra o. 2. st. 3.e c. 6. tt. 54.

PAlabava. Intendi ha gran rabbia. Latino fromachatur, Che bava quell'
que. UmMOre viscoso, che da per se stesso casca dalla bocca come schiuma, come si vede
eS se i cani arrabbiati, donde e presa la presente metafora. Si dice ancora: Afi
: fs venir (a bava'; di chi mi fa entrare in collora, e di chi noia forte,

La ML Ciel minaccia', e brava. Sgrida, e minaccia il Ciclo. Vedi sopra C. 5. st
ff Gx, che dice Rabbiosa,il capo verso sl Ciel tentenna, che è quel minacciare il Cielo.
2 'Di questo verbo bravare, che vien dal Provenzale il Varchi ne fa un lungo discor-
ut fonel suo Hercolano, e lo giudica molto esprimente il latino obrargare. Catullo.
ty:
6

è Aaa z Gel:

ane

Ate

| 372 MALMANTILE >

t+
] Gellins audierat, patruum obiurgare. falerb § 5 SOV si Dayne
| St quis delitias oe. 9 aut factret, et)
| TANTO di gla; Gola assai larga. Vedi fortoC, 16. st, 18, '
| ce tanto usata in questi termini, è tote Fig i
NON ne farefti caso,quand' uno ti pisciafe addosso, Non ti
j non t' importerebbe quand' uno ti pilciafle addosso; ed intend: Sei
ne, e codardo, che fopporterefti qualsivoglia grandissima ingiuria
ne. Un'antico Poeta per voler esprimere uno scellerato,¢ ingiuri
| ria di suo padres dice:patrios mincerit in cineres, B Pittagora in uno de'
boli per dinotare il rispetto, che si dee portare alla Divinica,
non si pisci in faecia al Sole.: itty
NON bevvi groso, Non fopportai mai ingiuria alcuna. Ber
Non la guarda così per la minuta, ma m5 8 ogni ingiuria senza
ne, fingendo non fen' avvedere. Tratto dal bere le medicine, le quali
faporano, ma si mandano giù a occhi chiufi, » 0: Soa sos Say
| MI fo levar le moscbe a' intorno al naso ~ Mi s0,vendicare dellvingiuri
cilitd, Omero nell' liiade La preftezza, colla quale un Dio fa tornare indietroi
colpi avvelenati contro a un' Eroe compara al cacciare d' una mofea, che fa la
Madre dal corpo del suo figliualo. o J oenetige
FAR ilc,..rosso auno, Galtigar' uno.. Tratto da i Pedanti, i quali )
i ragazzi perquotendola in sul ¢.. 5 e gliclo fanno rosso con'le)
sopra C, 4. st. §1. i: 1 a 0b if Settee SIR
oe. Subito, Viene dal giuoco di palla, che si dice Dar di nr.

si da di primo tempo, cioè avanti, che la pallatocchi terra «Lavinond seftigio«
INGOLARE, E' lo stesso, che ingollare detto sopra Cas st: 63, e vuol die mat:
dar la roba gil: nello stomaco. ' ol Owe
STANZA LXXXVIL, STANZA LXXXIX:
Ma dove col ceruel fon' io trascorfor Quel detia Cella del Romite eiil
Più buesdi me non e sotto le frelle, Ove trovaede if ibrvepuenia
Perchinnanzi ch'io habbia pref Porfo Intana dentro, e non wi scongeniing)
Vio (come si suol dir) vender ta pelle, Fruga,erifrugninquaye iia aea
Fatei ci voglion qua, perch' il discorsa Sgomina cia che v'èdafommoywinty
Fuor che ai Senfali non frutto covelle, M14 tutto in vanoyondeghial,
E mal per chi ha tepo,e tempo aspettas Sen esce con le man pieae ns yente
Che mitre piscia il can,ia lepre sbierta, Ma dieci volte più dimal talento,
STANZA LXXXVIIL STAN ZAULX Xia ly
E peri prima, che 4 vila a gamba « Entra neh bafeayeogni,
Vaa fuga mi fuanin divconcerto E in fammea ne cored pax
A casa Pigolon vuogt ix-di gamba s | LB wedde', fanz ain '

Che vi [ard coi compliciidel certo, vorrde pigiato esser bidval far decent.
ares fice ribinfe

Così conchinfo, correch? diff gamba, 4 «| Onde nelifine alf

N

Ne np bp cw @ eee FZ cee Fees SF ee hese K

E come un bracco taper-quel deferto Che pur mol vendicar segtandiane

Tutti quanti quei benghi a uno 4 uno Così v' ar rivera. po' pai in
Cercando.s'ei vi scmopee,o sentecaloxne 1 Seveifuffe( di i ne

K aia M



SETDEMO CANTARE: 373
° STANZA LXXXXIL;
Poiche Brunetro, e le sue camerate

Pagaron L offe,( il quale assai contefe,
Perché le gole lar. difabnare
Gli eran parute meeety Spese)-

». Partiron,, e poi dopo-altre fermate,,
Sie SUnapdeledee

i di quanto have 10 E giunto a casa,ringrazianda il Cielo
Ve più, ne manco ne segui  effetto. Entra in fala,¢ di posea fa un belo.
yi STANZA LXXXXIUL
Trovan Nardino acor di male oppreffo,
E sbietolar lo veggono ancor Lui,
L' Alhante, che porgevali horxata
saper. 9 me men per cui, >, Purine faceva lafua quattrinata
Magorto lai lamenti, e si,mette a ceecar di coloro., che gli havevano ru-
t la Figliuola Ȣ nen gli trovando nella Cella del Romito, ne in alcun altro
ricorfe-a gl'.incanti; co i quali costrinfe tutti della casa di Brunerto a pian-
3 onde Brunetto con i compagni arrivato a casa subito cominciò, ed
icompagni a piangere.;
fon' ia scorfo coi cervello' Che armegg' io? Che giro.io? Che frenetich' io?
SD pWVOMe fetta le frelle it più buedi me... Ao sono il maggiore ignorante che fa nel
Mondo Vedi sopra C. 6. stan. 98. Sarr /a Luna; i Petrarca..,Arda.s o mora,o
dangnifer sun pin gentile Stato del mo non e forte la Luna, '
thsi Bi: da.pelle dell' orso. prima di pigliarlo. Fax aflegnamento fopna una cosa,
che ancora non s'è conseguita, ed è anche molto dubbio/o.Ji conseguirla..Essen-
ido anati cre Giovani per ammazzare \un' orso'si| quale faceva molto danno,

prima che atrivafiero.al luogo dove soleva trovarsi l'orso, si fermarono a un'
Bieria ed havendo assai ben mangiato, dissero all'Oste., che lo paghercbbono
son denaci: del danatiy a, che haure bbano dato loro le, Comunia per V'orso,
che walevano ammageare, e d.amapifi verlo dove stavala ficra, subito, che la
veddero fidiedero a fuggire, e uno di loro fali sopra ad un' albero.s l'altro scap-
- PP Viayyediiltcrzo fu lypraggisnto dall' Oxlo ei quale bavendolelg,caceiato fot-
~todtinfran bene beng sdi'por gli.accoflo digrife ail' orecchio,, ed intanto quel
meschino se neftava come monto, senza muoyerf punto; e perché J' orso naw.
cralmente (secondo. dicona alcuni.) quando: ee ede, che.! apimale da lui. afaltato
sia morto, non gli da più fastidio, credendo che costui fatie morto,fen! andò,7¢
'€Gluiideyd si, ed ay vind vero la Città cutto mal coacio.. Quella,.che,era fa
 litodimaulimalbero:iccle-s ed.accompagnatoli coneflo, gli domandé quel che,gli
 havesseidetto.? orso nell' orecchia, sd egli rispole = Mijha detto, che io non mi
“fidi più difimilicompagni come sei G28 Che 19. now veoda.a, pelle.dell or(o. se
jeteemeete ho preso,.: B.da questa.novella-habbiamg il prefeuce proverbio, che
idictlanche: Vender t' uscelio in fu la frascas L Geesi ditiero: Anrequam pisce

BSELEAL:

smuriam misces.. J.
- MAI frutd covelle. Non fa d utile alcuno, Covelle \& voce romagnuola.e vuol
dire. Qualcosa, E' poco usata nok Fiogcatino fuor sie da qualche.consadino. Il
TANS:: valore

ERLETELLE

=

=



374 MALMANTILE o

valore di questa voce è assai copiosamente espreffo dal Copetta inun f
sopra il non covelle, Nel Decameron trovati Cavelle per lo stesso
Lat. quod velles.

epee ae!
E' mal per chi ha tempo, € tempo a/petta, che mentre, ec, Mal fa colui, che ha
do l'occasione pronta perde il tempo, e non la piglia,perché mentre si
J occasione fugge: E' noto il verso: Fronre capillaca post i
verbo sbiettare l'habbiamo anche sopra C. 5, stan. 30, Adentre il can piscia,
se ne va. 1 Latini ditfero Semper nocuit diferre pararis; secondo Lucano, di¢
forse Dante nell' Inf. C. 28. disse:: 7
Questi scacciato il dubitar sommerse
In Cefare affermando, che il fornite
Sempre col danno L) attender foferfe.
PRIMA che a viola a gamba, ec. Intende prima che d' accordo se ne
Viola a gamba e il baflo di viola, Fuga è specie di fonata a capriccio,
vuol dir Suonata concertata con diversi Rrumenti, ec. Econ questi
tende quel che s'è accennato. }
INT ANA, Entra dentro. Si serve di questo verbo anche sotto
25. se bene e improprio; perché vuol dire Entrare in una tana, o buca
rebbe intanare una volpe, un taflo, un granchio, ec, cutcavia e pur
to come nel presente luogo. ei
AMO. Niuno. Dal Lat nemo. Voce oggi usata solo dai contadiai ell
nostro Poeta se ne serve anche sotto C, ro. stin. 37. in bocca d' un
SGOMINA, Si dice anche (gombinare,( contrario di combinare,

piare, unire ) e vuol dir mettre in confulione, o sortosopra tutto | che si

maneggia. Lat. perturbare,.
DA fommo aime, Frafe latina, che significa Da capo a piedi: Dalla fomal-
ta della casa, fino a i fondamenti di essa: Petrarca Trionfo della Fama, ene
Onde da imo Perdusse al fommo t edificio fanto, + ae
LE man piene di vento. Cioè senz' haver trovato, o conchiufo nulla. Nellie
Scrittura. Et nibil invenerunt in manibus (nis; che diciamo ancora Con le troment
facco Ter, disse Infetta re. ee |
DI mal talento, 1 collera,e con volontà di far del male, e di vendicatl:
Varchi Stor. lib. 4. Erano verso i nobis di malissimo talento, ne altro per manvweit:
gli alpettavano, che quel che avvenne. BY frale usaca dai Boccaccio. =)
NE cerco per mars e monti, Questo detto iperboiico è uiacutimo per esprimett
Ne cercd da per tutto; Viene dal Latino. ee
SENZA metteria in forse, Senza dubicar più. Senza metterla in dubbio. Dd
mettere in forse fece Dante il verbo inforfare, seguivato in ciò dal Petrarca +
IL pigiato esser (ui al far de* conti. A consideraria bene} offefo', e-beffaroem
solamente lui. Quattro giuocano infieme', tre vincono, ed un di loro folamet”
te perde; questo tale si dice è/ pigiato, cio® quello, che ha gli altri addosso, et
cui si spreme il denaro. Bs' intende in ogui caso, che la disgrazia tocchia ue
solo della conversazione, e tutti gli altri habbiano (oddiienioasl “, outile dal
danno di lui.: Ae:
POPOL in quel fondo, Vedi sopra C, 2, tian,,3.



y '"SETTIMO CANTARE.: 375

 BeANNO avanga. Vanno secondo il desiderio. Ex animi eins sententia ille cr
unt. Noil habbtamo da i Contadini, che quando si rende loro facile il lavo-
'la terra con la dicono; 4 lavoro va 4 vanga, cioè bene,¢ come si de-
« Bvangad strumento ruftico fatto a foggia di pala,ma di ferro più
» o più acuta, del quale i contadini si servono per rivoltolar la terran.

edi sopra C. 6, stan. 69, al verbo impiallacciare., Columella lib, 3, la chiama do-
ira,¢ perché questo nome vuol dire più tosto la piaila, forse Columella inten-
ee flrumento usato a suoi tempi, che faceva sopra alla terra l'effetto che
t pialla sopra il legno, ( come e hoggi la marra scopaiola, della quale si
serUono i contadini per ripulire,¢ radere i boschi di scope per disporgli alla semen-
 ta della fegale ) perché,se volefie dire la vanga,havrebbe detto acuta dolabra fodi-
%,¢ non abradito: E la vanga si trova bipalinm, in Varrone: /d priss bipalio

STVMMIA di furfanti, Scelleratissimi, ex omni vitiorum colluvione concreti.

i 'Stammia,scbiuma, o spuma, \& quello escremento, che nel bollire una pentola.
* piena di carne, e di acqua manda alla superficie, il quale si butta via, perch ¢

th Imm ia; onde fummia di furfanti, 11 peggio, che sia nella furfanteria. i

hay! ( difabicata. Lat. gurges, Così diciamo di colore, che sempre mangia-

yi —— si veggono fazzi.

“an paruti cari per le spee. Exa parfo all'Ofte, che costoro havessero man-

ot 9 troppo. D' uno che sia buono a poco, e mangi assai, e che vada a servire
od | }; Beli e caro per le spese; e intendefisfe gli da più del dovere,¢ di que! che
wr en sua abilita a dargli solamente mangiare, senza dargli danari per prov-
a ¢. li Lalli nella sua En, Ir. C. 2. stan. 130.

Sit Non vaglio un pel; son caro per le spese.:
Dit! DI posta fa un belo Subito comincia a piangere a tn Vedi sotto C.9, st.21.
ig) o SSHETOLARE., Cio piangere. Vedi sopra C. 4. stan, 16.

AST ANTE, Intende colui, che aifisse al servizio di Nardino infermo, 4fan-
oiil 34 si dicono quei Serventi, che affistono a servire gl' infermi negli Spedali,¢ questi
it Aoglion esser chiamati dalle persone comode ad aififtere alli loro infermi, e pe-

10 qui lo chiama col nome 4' 4/fante, (upponendolo uno di questi tali.
in ORZ ATA, Bevanda rivfrescativa fatta di feme di popone, orzo, e zucche-
yo 79 denissimo peiti e liquefatti con acqua,¢ passati per stamigna, si da per lo più
ra febbricitanti;detta anche /arrara come habbiamo veduto sopra in questo C. st. 12.
NE faceva la Sua quattrinata, Cie faceva la sua parte del pianto,

Ae STANZA XCIV. STANZA XCV.

oe /ardin vede colei bell' ye verzosa Mettere pur così le mani innanz

ro Com' appunto ? haueva nel pensiero, ( Rispond' ella) Signer per non cadere,

i E dices Benuenuta la mia [pola, AMentre,temendo ch' io non mici feanzi,

1) Ko 9 tt piacere a se da Cavaliero. Specorare si bench? è un piacere:

we 4a voi piangere ? ditemi una cosa Ch io mi vi levi, ditems, dinanzi,
Bs Koi ci venite a malincorpo, e ¢' vero? Che voi non mi porete pix vedere 5)
RD. Non vogliate risponder che ¢' non sia, Senza darmilaburla,ch' io m' acquicta,

" Perché

bé vei mi direfti una bugia.

E Senza replicar do voita a dreto.

STAN-

+


376

STANZA XCVIV

Ne fofopra la man non volterciy
j Che Pandarese lo far mi fontitt nia,
| Eben c'al mondo t6\sia come gli Bbreiy
| Che non han terra fer mayo patriaalcuna
i Andrd pensardo incanto a farki rie
Per veder di trovar miglior fortuna',
ct fond » come diceva Afona Berta: vr

t Chi non mi vuol segn'e che non mi merta, - Pero non vogliar
? STANZA XCVHL: %

Ella soggiunge, ed Egli ribadi/ee j
| Ella non cede, ed ei risponde a thiond.; o ” ogmoraincafa, fuora,
Pur gliacquiera Brunetto,e al fin glsunisce,  ( Perch sempre si fnmena,
Sicché 2 un o altro chiedefi perdond} « o) Hantioa tener agli ovcbi |
Nardino vede la Fanciulla; e la trova per appunto'comie fel" era'
|; ma vilto che ella pi angeva le dice, che dubita, che ella' sia venuta mi
ed ella gli risponde, che dubita, che più tosto egh non la riceva vo!
pra questo seguitavano a contraftare, ma Brunetto al finé gli raj
tutto questo ognuno seguitava a piangere. i ¢
j VOI ci venite a matincorpo, Voici venite malvolentieri,'¢ con
| soddisfazione;cetra' flomaco,cOtra voglia,fatrone'taa fola parol: come
METT ETE te mani innanzi, Questo certnine ci ferae per esprimete
accufa un' altro di qualche mancamento, del quale merita dj esset 4
per esempio: I ragazai dello Spedale degi*imhdceati, i quali si
sieno tutti bastardi, in occasione di contraftare'con alcri ragazzi,
giuria che dicano a quelli ¢, 7 /ei bastardo, pecche non sia detto a”
fio si dice: Aderrer le mani sananzé + o vi si aggiugne anche; per
prevertere,occupare. -
NON mi ci fanzi. Non mi fermi in que(ta Casa per sempre.'
SPECORATE. Piangere. Diciamo be/are per piangere per la
che ha cobbelar degli agnelli, € delle pecore certo planco Jango', che!
rei bambini, come accenuammo sopra C. 6, ttn, 22. e'da questo
Specorare in vece di belare, e s intende piangere.; nist
St ben ch' è un piacere. Tanto bene, che € un gusto a sentirvi, e vederti
NON ne volterei la mano sortosopra, In questa cosa io fouo imdife;
poco ny importa il faria,o non farla, Vicne da i Latini che discvano
Ne manum quidem verterem, + tye
ESSER come gis Ebrei, Cioè non haver luogo che sia\suo propri
ra il Poeta medesimo dicendo > Wun ho terra jermayche intende terta'y
abitazione fermata, e stabilita per-lei, che per altro Lerra* termat
paefe, che non Ifota di mare', Lar, continens', + syageael
VOI vi levate in barca, Vou cntrate in colicra, Vedi\sopra Ce
dice anche abarcare, e 2 iracondo, o vero facile al? iray chek
» derocholor\& dette da NOt Aacmo as poca levacurascioe che'ci'vudl poco |
rein collcra.

~ ida non per
Co

a
3


SETTIM/O CANTARE; 377
Qui vuol dire fofferenza, o pazzienza, che per altro' Flemma
accennammo C. 3. stan ages % ho
4, Iraconda, Vedi sopra C, 1, Aan, 29, Alewni critici hanno
dra questa refs; giudicandola rimafaifa in'tignardo-dell' 5s, dolce di
'studardi ai/perte/a.,¢ dell', 0, Jargo diquelle, e fretto di queste,
voglio quictare, e difendere il holtro Poeta col Riscelli, o con
'non mi fon'voluto pigliar la briga di vedergli come non necessaria,
bem loro un' esempio d' Autore claifico il quale dice.
Hag. Lb 9.30 La verginellae simile alia rosa ~
9, Nei bel giardin /u ta nativa spina,
Mentre fola,¢ sicura si riposa
Ne gregge, ne\pastore se le aunicina
, Li aura fuave, el' alba rugiadosa, ec,
Eg con questo esempio.( il quale sia per regola, o per licenza) di fal-
ire il nostro Poeta, e quietargli, aucor per J' altre, che hanno osservatese sopra
stan, 13. Rosa  prola, e cosa; e foro in questo C, stan. 103. Sposa, cosa,

BADIRE, Ribattere, conficcare
Vedi sopra'C, 2. tan. 79.
DE 4 two, Rilponde aggiuttatamente, ed a proposito di quel, che si
verbum audit, tale dicit. Si dice anche Rispondere per le Rime. La
iilitudine € tracta dalla Musica; la seconda dalla Poesia; B allude al co-
he de' Poeti che indirizzando /' uno all' altro Sonetti,e¢ proponendosi que-
levano,e le scioglievano in altra eguale composizione teffuta delle
t eliine rime, il qual costume venuto dall' antico, si mantiene anche in oggi.
sul - Sl fmacica, e B cosa, Si manda escremeati dal naso,¢ lagrime dagli occhi
oi M 'piaato, che/moccicare vuol dire mandar fuori mocei, che € quello
 eleremento del ceruelio, che esce dal naso detto da i Latini mens.
PEZZVOL A. Pazzoletto, o Moccichino; ed \& quel pezzo di panno lino, che

dall' altra parte un chiodo, Vale per re-

i si lo di se per uso di nettarsi i naso, —
sl) py, SEANZA XCIX, STANZA o,
fs wa in un continuo pianto, E veduto ch' ell! e tra buona gente,
; om - ees: 3 i \&
| Phangona i fer wise piangon gls animali, Moglie a! un ricco, € nobil Baccalare,
i Onde th guazzo per terraetale,e tanto, Eche Sok le puo mancar niente,
iB Chee Portan tutti quanti gli frinali. Per ch' elinein unacafacome un mare,
, iamoa Magorto, che fra tanto Won vi fo dir e ei gongola, e ne ente
a Per saper quel che sia di questi tai, Contento grande, € guffo singolare,
il Ez dowe la jua figlia si rnrovi, De-modo ch'ei si pente,affligge,e duole
a = Hes fato ab confuse incanti nuovi. ao — ha fatto ye rifarcir lo vuole,
RS ae abled. 5) o STANZ yt
Pi un fae cogno, E poi che dentro più non ne puo porre 5
\& Sumnonipencace, a nae foot; Biphedi pte "Luo aspetto e molto brutto
lage e 0,enecomincia a corre, Si lata, vipilisce ye raffaxcona
a ~ Darando fin che hebbe pieno tuto; E rimbeuilce tutta ta persona,
0 SS AAMERabs he ig

Bbb STAN-



378 MALMANTILE; =.

. STANZA CIL
E presa addosso poi quella sua cassa,. Che al suo ver
CL? e tanto grave,ch' ei vi crepa sotto, Mirando in rif
St metic la wa, @ presto se ne passa.;  Eversad pomiin
Ow e la figlia s e il flebile raddotso Poi st

Mentre che costoro piangono, Magorto per via-de'
\& la Figliuola, snaolcendonche ella è bene aipentt si. %
folue di regalare gli sposi d' una quantità grande di pomi d' co + A008;
to, € Così fece, ed ap arrivo oo in casa degli sposi tutti cessarono di pia
GV.ALZO. Luogo pieno,d'acqua, dove fiipossa guazzare, cioè pall
picde scnza navilio, che noi dal Jatino diciamo, vadoy o gaado; onde ilk
Vaaa così detto perché quel luogo dicevai Yada Volaterrana's e guadare x
¢ passare: Ma i piglia ancora per ogni grande ammollamento, che si fa
neile case, o altrave in sul suoloy come € preso nel presente ye: aque
calo viene da guazza, la quale cade dal Cielo, altrimenti detta 4 at
prvina come gelata disse Dante dal Lat, gelu; e non da guazzare il flume; Sefore
ie non voleflimo pigliarlo per parlare iperbolico, come e  adoperare
per passar-tal molle, che è in quella lanza. 0) oop oe
8ACC-ALARE. Huomo di stima. Uno dei principali del paefe
anche Barbafforo, Baccalare da Baccalaureus si dice colui, che nelle
acquiftaco un grado proffimo al Dottorato, o Maestrato detto altriments I
ziato; il che usa nelle Fraterie, e corrottamente lo dicono Saccelliere
grado si ritrovava anche nell' ordine della Cavalleria.:
E una casa come un mare. Cioè sempre piena di zoba ed abbonda
bene, si come il mare, che e immenfo, detto perciò. da Omero atrygi ih
non ha fin, ne fondo, Si dice anche Una casa come una Dogana, a
GONGOLA. Greco cancharei » Giubbila:-Si rallegra 2-5i
certa allegrezza interna. B' voce usata assai dalla piebe.: Re
KIS-AKCIRE, Ristorare; Rifare il danno,0 ricompen(argli qd havergli tenn.
ti tanto in pianto. E per altro questo verbo ri/arcire vuol dir raffer
visto sopra,C, 6. stan. 52,
cOGAO, E' una misura,immaginaria di vino, che contiene dieci barill
quale corrottamente si dice Cento; Deriva.dal Lat. congims. Onde Bigonet §
da un Lat, bicongins; a Pistoia perciò dette più prossimamente all' origine Bite
Gio, Villani lib. 8, rubr, 116. Valle lo fraio del grano in Firenze solds 8. Leagan del
mofto in certe parti meno di (oldi 40, Ma qui è preso, come \& coflume », per une
certa sorte di cafla, o più tosto cesta fatta, e contesta di strilce dal
corbelli, ma è di foggia Junga,.ed ha il coperchio, come hanno le, c
S/rafazzona. Si ripulilce; Si rinfronzilce. V.cdi ops. Cy 2 stange
si rifa; si rimette in fazione, in abito; fala Pre tecr yi la bella
nicra. Gli autichi dal Provenaale dissero agenzare, cioè \&, '
ce Gente usata dagli antichi Toscani ancora per Gentile, Br
voi, donna gente M' ha preso.amor, non è.già maraviglia. L
il senno, e li gents coraggi.. I Beato lacopong disse che la 2
geaz@, clot non rilciacqua, come spiegd alcuno, ma rafazzuna, ri
'



SETTIMO CANTARE: 37
sotto pet lo fverthio peso j ed'il verbo crepare, che»
, come vedemmo et: stan. 18.qui¢ nel suo vero
ire, perché quella gran fatica può cagionare l'allentamento.
0€ si cava di Se.; fila errcor tact propriamente il pile
essendo il nostro cappello più tosto il pera/us.
Fras ) STANZA'CIV.
2) Eperché qualfinoglia-donniccinola,
2 Lpauen: Porta la dote, ed il corredo appresso,
digni cosa,.——*. Acciacch' in quella casa la Figliuola
Sdegho*toritmenté ha spento; ~~ ~Polfa mostrar a baker qualche regresso,
<1 SS Ne che gli abbin a aner quelcalcioingola
“C' un piccolo ne anche v' habia messo,
'La vuol dotar conforme al grado loro
pss 'Con quel gran-monte di bei pomi a! ora,
SM Evopny shaves Ps ANN FAs +P;.
llor brillando con Brunette o Edegli poi al fin com ogni afferto
gr axiejefanaratanccoglitza; °°\° Riker? rutti, è volle far partenza,
inato un grande ye bel banchetto” > Ledandosi del furto del Romita,
“ar le nozze in sua prefenra, 'Che si grand' allezrezza ha partorito;
orto si fa. conolcere per il padre della, Sposa, ed aificurando Pigolonc, ¢
i perdonato, ed' haver guito', che segua quel parentando, colti-
vl lla caffa piena di pemi d'oro. Si fanno però di nuovo gli spon.
il banc! ¢¢ Magorto se ne corna al suo paefe, dando molte lodi a,
per esser'egli stato autore di così gran 'conteato. E qui con la fine de]-
la'raccontata dalle Pate a Paride termina il fettimo cantare,
  AMAN vote, Senza nulla in mano: cioè si mariti(enza dare dote alcuna;
we} «= CORREDO, Quegli arnesi, abiti, ed altre robe j'che fi'danno alle Femmine,
 Oltre alla dote, quando si maritano, che i Giureconfulti digono Parapherna dal
yep Greco Para, che vuol dire oltre, e pherna, che vuol dir dote,
giù HAVER regrefio Termine legale, che vuol dire haver azione di domandare,
 Sontro\a tino, per rifarsi del pagato ad un' altro; Vedi sotto C, 8. st. 42, E co-
jj MUunemente significa un certo ardire, ed autoritd sopra ad una persona, o sopra
i i suoi beni ed effetti: 1 rale gli ha preso regresso addefvo, per intendere ha preso
yg) atdire sopra di lui.

a gli abbin a haver quel calcio ia gola, Non habbiano a poter rinfacciar-

us om; “ oe non v' habbia portato nulla: Noa habbiano a ha.
'caufa di conculcarla, '

| ELANDO. Giubbilando, Vedi sopra C. 2. st. 69.

4 CACCOGLIENZE. Vedi sopra C. 1. st. 34.

le mgxe. Cioè di nuovo si fecero gli sponfali, e folennemente ff
di sposi.

' 'Gohan

FINE DEL SETTIMO CANTARE:
Bbb 2 OTTA-

tk


\&3 ARGOMENTO.
oY
D' un! avventura grande è po

OTTAVO.CAN
a; J 3
SERS]
Dalle sue Fate Paride vestita. ous. a ae
Vede lagalleria di quel? albergo;\.\ a
Seco eer eenin

Ond' ei pigliavicenza, e voltaiLcergo.,

a

5 Vien Piaccianteo condotto al Generale,.... »

ey Che non glivolle far ne ben, nemale,

BAP 25 CEPR CUE MFI CED
N75 h
ENA AAACN |
STANZA STANZA THM. Ta
Orrei, che mi dicesse un di coffora, La notte, disse,¢nn wafo dit
Che giofran tusta notte per le vie:, Che versa affronti, rifichi, € iy.
Che gusto v' ¢,perch' a ridurla a ora Pero, he nel:fiuo sempo shucan fuora kg
Non v'è guadagno, e son tutte parries Tutes i ribald ydadrt, e rompicali;

Poiche ( lasciando, che enon e decoro ) Onde sia ben'riporsi di buon' boa, !
L? aria cagiona cento malattie, E dene esempio Lhuom pig is al iy
Mille disgrazie possono accadere, Che  unds loro al piis vale. at T
Mille malanni, Diauoli, e Versiere, E pria ch' il. fol sr amoncé se ripome thy
STANZA IL. STANZA LWe be
Sapete, che e' s'inciampa, ¢.che e' si cafen, Edegli, che at un. mondo assai più vale t
Si puo in cambio a' un' altro esser'offefo, Sta fuori tutta notse,o dineth, Pa
O dar mun, fet' bai monete intasca, £ gira al bmo come un' ani 4
C alleggerir ti voglia di quel peso, hie
Maca: qual ma si pudcorrer burrasca, t
Però vi ginro, chiio non ho'mai inteso ny
La fin di questi tali, € tengo a mente de
Quel c'wn tratto mi disse un buom valete, to modo, che non ve da le
STANZA V. » DSR hebeR Cy
Perché le son tutte cose provate, Come al Garani quand! a gal K
E vere, che non v'? spina ne offo, e-4ndato era la norte n
E non si tronan poi sempre le Fates Che, mentie oi by
Che vengano 4 leuarti il mal da dosso, Da esse ebbe un sanor di \&



2

OTTAVO CANTARE 381

Poeta ifeguitare a narrare quanto avvenne a Paride s' introduce col
che nocumento sia ' andar fuori di notte, e che però sia cosa da,
, dente il'non considerare quanti pericoli si possono correre; Ed
nigliando la notte al Vaso di Pandora conchiude, che si dovrebbe imparar
i polli, che vanno a dormir subito, chee' s'è riposto il fole, e così sfuggire
te le disgrazie, perché non si trova sempre chi liberi dal male, come avvenne
a Paride, che dalle Fate fu liberato dal pericolo di morte.
 GIOSTRARE. O armeggiare. Mctaforicamente s' intende andar girando, o
passeggiando senza saper dove, o senza fine determinato, che si dice anche anda-
» Oagironi.
ARIDVRLA « oro. Per ridurla alla. conchiufione. Vedi sopra C, 3.st. 43.
e malanni Diavoli,e Versiere. E' un modo di dire assai usato in simili
aioe per esprimere possono avvenire tutte le forte di disgrazic.
VEKSIERA, orl infernale, che dalle nottre donnicciuole e intesa per una
fla moglie del Diavolo. Forse viene dal Latino Ver/usia, che vuol dir
'malizia; e si dice Versiera un ragazzo malizioso, fastidioso, e insolente, ma
Spill veri » che venga dal Latino aduer/arins,col quale nome e disegnato il
7 nella scrittura., ddnerfarins nofter diabolus, Petearca.
5 1; sì che anendo le reti indarno tefe,
U1 mio duro avversario se ne scorni,
Da aduerfarius nelio stesso modo, che 1 Francesi fecero aduerfaire, così i nostri an-
: iz i's Auuerfiere ? anuerfiere, e poi finaimente /a Versiera. Ll Beato lacopone da

i

if

Hl canto 62.

ih “A Lo nemico ingannatore

oe o8aiiin::; * Anerfier de la Signore.

xs) Eecant2r. | Fata gli anerfere venire,

8 ' Chel degian accompaguare,

an Nell? uso dicesi Far la Versiera, fare il Diauolo, e peggio,

ib  INCTAMP.ARE, B il latino ofendere. Vedi sopra C. 1. st. 13.

i OT ASC.4. Quella facchetta, che si porta comunemeate appiccara agli abiti per
nfo diyportar roba necellaria alla giornata, come denari, e simili da' Latini detta

it Pera, o Zona. 8

/ @ALLEGGERIRE di quel peso. Cio portar via i denari, e cos} alleggerirlo del

', pelo, edella noia, che per quello gli veniva.::

a MANC A in che mo, Cioè sono infiniu i modi. Il termine mance in questo ca-

oe a0 sufato ironicawente, perché s' intende: Vou mancano s modi.

a  CORRER burrasca, E termine Marinaresco, che significa Correr pericolo, ed
in questo Ggnificato e preso comunemente, se bene berrasca vuol propriamente

dire sollevamento di mare per il cattivo temporale di venti, ec.

VASO di Pandora, E) nota la fayoladi Pandora, la quale fauna Femmina.,

( che Giove fece fabbricare da Vulcano, e darle.in dono di ciascuno degli Dei je

' parti, affine di farne innamorar Prometco,.¢d indurlo ad aprire un va-
7 fo pieno di tutti i mali, che Giove haveva dato alla medesima, che lo donaffea
jf Promotco » che vuol dire Prevvidente; che antivede, per vendicarf dell' ingiu-
, tia da eifo fattogli quando rubo u fuoco celefte, ma nonl'havendo Prometeo
oe voluto

rie =
*

382 MALMANTILE o

voluto accettare, lo prefe Epimeteo suo fratello, 4
fatto, il quale l'aperfe., e vennero fuori tuttii mali, che
questo e il vaso, che il Poeta intende nel presente luogo, ¢'
ni nel secondo capitolo della pefte dicendo:
To leffi già \& un vaso di Pandora,
Che n' era drento il canchero, e la febbre,
E mille morti, che n' usciron fuora

Orazio lib, 1, Ode 3.
Post ignem atheria domo
Subduftum, macies,\& nova febrium
Terris incubuie cobors,
La favola, e raccontata da Esiodo. è
RISICO., Riftio, o rifico dal verbo arrificarf?, arrischiarsi,o d
vuol dire Esporsi al cimento, o avventurarsi a qualche icolo. In'
Risco significa, rups pricipizio, luogo pericoloso. Cie, se bene mi f
quam in diffictle, \& scopuloso loco verser, rificoso. “he a;
TRACOLLI, Da tracollare: altrimenti barcollare, che è fm
il Latino metare, o ritubare; e qui vuol dir Disgrazia,, o pericolo, by
ROMPICOLLI. Huomini; che consigliano, o inducono altri a far 2 |

Latino in omnem audaciam proieeti. A a Cu
TSTONE. Moneta Fiorentina, che vale tre giuli,o paoli, 9 | 4
VAL più a! un mondo., Questa iperbole significa non vi e prezzo, Ta

Star discoffo un mondo, disse il Bronzino nelle rime burlelche; cio Me

Spario. u

CERCAR di Frignuccio, Cercar le disgrazie. Andar incontro a' pore
Frignuccio dalle nostre donnicciuole è preso per il Diavolo, e diciamo ane ny
cercar il male come i eMedicr, I Latini in questo proposito dissero; Camarinam me \ 4
xere da una pianta,.4a quale ha le foglie così fetenti, che movendole, @tocta | Mi

dole lasciano un puzzo terribile: o forse da una palude detta Camarina! do
cina al castello detto Camarina in Sicilia, la qual palude, perché cagi \
detto Castello la pefte, i pacfani domandarono ad Apollo, se era bene Q
re detta palude, e l'Oracolo rispole: Camarinam non esse mouendam, I
fatto poco conto di detta risposta,vollero feccarla, e n' hebbero il gaft q

i nimici passando per quella palude già fecca, entrarono nel Castello, €
“TW bella proua. A pola; e l'addicttivo beds s*usa in questi ca pes eal
IN bella proua. a; e l'addiettivo s'ula in i cal '
r seaen un Gpuinivonmal dica in prouffima. Vedi sopra se
nell' uso: L' ho bell' e fatea questa, o quella cosa; cioè l'ho fatta fa
terminata, fornita. x
CHI cerca trona, Detto sentenziolo, che significa, che colui, che
al male, merita che gli succeda:
NON 0' e spina, ne offo.. E' negozio spianato. E cosa liscia, Non'
bitare, non ci e da incontrare difficulta alcuna i
AGAMBE alzate, Cioè col capo all' ingit'. Si dice anche ve a
are, Vs0 questa frafe Agambe alzare. Ser Brunetto Latini maestro di D

tase n= ~sxse.



-OTTAVO CANTARE, 3 83

ovvero Capitoli pieni di gerghi, e di vocaboli Fiorentini; e vole spie-
'atto di chi iemwomnla in terra per iscaticare il ventre. Zvidi a eae al

ecanteee: con riverenza, cacava ) che questo vuol dire torrire in

ested col, ut

“EGGLAV A con la morte. Faceva conto di morire. Temeva di morire

nel mulino.
STANZA VI. STANZA IX.
efto vuol pur ch' io di lui difeorra, Circa questo,pensiero elle non hanno,
Onde di nuono a i fatti suoi ritorno. Ne di fare altre spese, come accade
Le Ninfe, eb? il vedean barter la borra Ad ogni galant' huomo.a capo a anno
Tutte gli son co' panni caldi attorno, D acconci,taffe, laftrichi di trade:
B già tra loro par che si difeorra UL vito,e il freddo non puo far lor dino,
Di fargli dare una scaldata in forno, Perch il tetto, che scorre,e mai non cade

ta perché questo in danno suo rifulta LZ? Inverno fui pilaftri di coralto
+ volleil/uo parere anch'ei inCfulta, Si ferma,e forma un palco di criftallo.
:S8TANZA VIL. STANZA X.
ino di non farn? altro; ond' esse Di Stare il Sole giu ne' suci quartieri
rivestire a [pefe loro; Non puo col frugnolone haver l'ingre/io,
7 icia nuova una gli messe, Tal ch' elle stanno bene, e volentieri,
« C'ba dal colloe da man trina e lavoro, E gedono un pacifico pussesso,

Pr altra il ginbbone yn'altra le bracheffe Paride intanto infra tazze,e bicchieri,
"un ricco,¢ nobil quoio a' oro, E di più forte vini, e fructe appresso,

« Fit altra gli ravvia la capeliera, Con è/se ritrovandosi in cantina, —
J\&€ ette il benduccto,e la montiera. Valle provarne almeno una trentina,
oe STANZA VILL STANZA XI,
Alpalfe poi lo menan per la mano Ve per questo alterato egli ne restay

4 ta lor bella abitazione, Ovengach'egli¢ avverzoin dlemagna,

Ma poi più buona,benché sia in patano, Oc' a salvar quel vin faccia la testa,

ea pagar non hanno la pigione, Ed in quel cambio dia nelle calcagna;

¢,un negoriv odio/o, e frrano Ragio, che quadra bene,e quedaye questa,
: quell' insolente del padrone Perch' ei non urta mai chi l'accepagna,

oad na a casa,e co si poca graria Ma sipre in tuono,e dritto com un fufo

Chiedeilfemeftrech'ei nov'è una crazia, Con esse per le scale torna fufo,

-olox gli pic's STANZA XIL

nv! Ow egli entrato in una bella fala, Di li poi falgon sopr' 4 un' altra sala
6 Ch ella sia l'Accademia si figura, Di baston congegnati infra due mura,

' he vi son aratolo, e la pala Donde, arpicando come fan le gatre

ti d Strumenti da Siudiar ? agricoltara, Vanno a passar per certe cateratte.,

, DiParide dunque vuol seguitare a discorrer il Rocta, e dice, che conoscendo
i ke Ninfe,, che eglifentiva un gran freddo, volevano metterlo a rasciugare, e

 Hilcaldarsi in un forno.,ma.egli non volle, onde esse gli fecero un vestito nuovo

' spefe nella maniera,.che viene espresso, in guetta Stanza fettima; Di poi
Jomenarono a vedere la loro abitazione, ed in cantina dove bewve assai;e.nony
-danno per le ragioni; che adduce il Poeta; e di cantina salirono alle

0

2 er



22,

384 MALMANTILE?

BATTER [a borra, Iorendiamo Tremare, e battere i dei
do: E si dice così per la similitudine, che ha tal b
che si fa della borra:la quale e speci¢ di lana triturata col coltello e |
empiere i basti delle bettie da foma, ec. e per liberar devta borra dalla)
si mette sopra a un' asse forata con piccoli, e spessi fori, o filbatte
di corde adattate a questo effetto; e questo battere fa uno strep: ct
che similiiudine col batter de i denti, che faccia uno tremante per ¢:
do, ec, Si dice anche batter la Diana; tremare tutto, stando allt aria, a€
scoperto; Latino /ib dio... Vedi sotto C. 9. tt..6. os 8 Coe Raa

BRACHESSE, Brache, caizoni, Voce Veneziana taluolea wfata anc

On ' '
QVOF d' oro, Pelli di bestie conciate, e dorate', servono per
ze in vece di drappi. ' at

GLI ranua la capellicra, Gli pettina la zazzera yO chioma, 3

benaa. Striscia di panno lino bianca, che s' appicca pendente alla
cintola de i bambini, perché si posiano con essa nettare ij malo, -

MONTIERA, Specie di berretta usata dai bambini. Dallo §;
tera, berrettino,

PANT ANO. Palude', che diciamo anche padule, luogo pieno:d?
ma, che renda il terreno inzuppato, riducendolo come fango, da i
detto Palus, paludis,,

PIGIONE, Cioè quel denaro, che si paga per fitto d'una cosa; E
con termini proprj tro si dice quel danaro, che'fi paga per poderi, et
pigione si dice quel denaro, che si paga per Case, o botteghe, dicendo
botteghe, o casamenti: Ed appigionare case,e botteghe. Di queste si dice
ma dei terreni mai si direbbe appigionare. Pigione dal Latino
forse da fexdum, fio, e questo dal Latino fides, ¥

STRANO., Stravagante. Qui intende noioso, odioso, fastidiolo..
frrano dal Latino extraneus ritiene anche appretio di noi il significato di
ro, o lontano dal parentado nostro. Vi/o ffrano, vuol dir vilo arcigno
o cruccioso; viso rane vuol diranche faccia macilente, e pallida,”

SEMEST RE, Numero di sei mefi; ma intendi il denaro, che si
pigione di sei mefi. Le

T ASSE, e laftrichi di strade. Spe(e, che occorrono farsi alla giornata
ro-, che posleggono case in Firenze; che /africhi, intende quella spela
partice fra i padroni delle case per raflettamento, e Jaftricamento
della Città.

TETTO, che sempre scorre,¢ mai non cade., Abitano sotto acqua 51
il loro retto', che sempre scorre, e mai non cade," 1 abt
PILAST Ri di Coralie, Pilattri si dicono quelle colonne fatte di
altri safi, per foftener volte, Latino pile. B percht'il corallo naso
finge, che questo tetto f regga loprai pilaftri di coralloje vaol di
werno s' agghiaccia l'acqua, e fiferma. 1 boy poner
NON pro col frugnolone haver ? ingreffo. Non può il Sole trama
netrare i suoi raggi fotco l'acqua, Fragnojone da Frugnuolo detto

2.2. -Seeeepeee seers eee =

 =

BRipnmaes..



OTTAVO CANTARE. 385

ae 'RATO. Commoffo, o perturbato da qualsisia accidente. Ed alterato
dal vino vuol dir Briaco. Onde gli Alterati Accademici già famosi in Firenze
Bee ate

o per Impre(a un Tino; in cui Gi pigiava l'vua, e ogni Accademico usa-
per impresa particolare cose attenenti a vino; si come quella della Crusca,
le succedé, usa per impresa tutte cose attenenti a grano.
ACCIA a salvar la tea. Non offenda cot suoi fumi la testa, perché e vino
- Detto scherzoso tratto da quelli, che giuocando di scherma non fanno
gioco, ma pattuiscono di salvare la testa, cioè non si colpire nella te(ta.
GION che quadra benese quellaye questa. Tanto può esser per questa ragione,
per quella, che egli non Sa rimatto alterato dal tanto bere.

NON urta chi o accompagna, ma è sempre in tnono. Non barcolla come fanno i
riachi, e non da spinte a chi è seco, ma sta in cervello, e va dritto.

| ARATOLO.. Si dice anche aratro dal Latino. EB erato si trova nell' antico
“Volgarizzamento di Palladio:; donde e fatto il diminutivo drarolo. Strumento

quale i villani'rompono la terra, facendolo tirar da i buoi.

'PIC ANDÒ, Bi il verbo arrampicare fiacopato, e vuol dire il falire, che

oi gatti sopr' a ua' albero, o simili, e viene da rampicone, che è un ferro
inde Die, che usano i marinari per pigliare, e fermar le navi. Latino
ro  harpagonis; da che noi pure lo diciamo anche arpazone, e arpagonare,
CAT ARATTE. Et voce latuna, che vien dalla Greca catarrbattes, con la
intendiamo ancora quelle buche fatte ne i palchi,per le quali si pafia di for-
entrarein luoghi superiori con scala a pioli, come farebbe falire per di
faltetto + E per lo piij tali cateratte s' usano per entrar nelle colambaic;
a sorta era la cateratta, che dice in questo lyogo.

TANZA XIII, STANZA XV.
4 qui la Mula vuol ch' io mi dischiari Horsi per ch' io non caschi nella pena
Circa il deferiner queste loro Stanze, De cingue solds; ecco ritorna a bomba
Che stio vi pongo addobbi un poordinari, A Brache d'or, che nel (alire arrena
Non per dir bugie, ne ffranaganze; Per quella seala, che va fu per tromba,
o Peréhiile Ninfe han solo i necessari, Perché se bene ei fail Adagia da Siena
wv ie moderne usAnre, Gli è difadatto, e pefa chregls spiomba,

i
Per infeonare'a noi c habbian le borie E con le Ninfe a correr non puo porsi,

4 adri',¢ letti d'ore,e tante lorie, Maffime liche v' e un falir da Orsi,
atl STANZA XIV. STANZA XVI
i; Ch ognun vuol far il Principealdid'eggi, Elle di già, com' to diceua adesso

rl | Seben chi la volesse rivedere, Vicite son di sopra a stanze nuone,
Melti firvegvon far grandezze,e sfoeei, eApettando, che facia anch'ei Listesso,
cae,

specchto poi col rigattiere: C' appunto com il cambero si muone;
a sa lnfsobgrande, egiaregnainsns poggi, Onde connien poi loro andar per esso,
 Efon nelle capanne le portiere + Ed aiutarlo fin, che piacque a Gioue,

4 | Bera icannelt infin qualsivoglia unto Che quasi manganato,e per strettuio

Had fuck fhiperti, e seggiole di punto. Passasse ad alto il Caualier di quoio,
Pr l'Autore di voler dire la yerita, prega il Lettore a non pigliare
wzione, se in descrivere le maficrizie delle Ninfe metterad addobbi, ed ar-
Refi un poco ordinarj, perché in eftctto ead così; e da questo piglia occasione di
tye Lee biafi-

S



386 MALMANTILE

biaiimare il Info, che è oggi in Firenze. Di poi tornando
che le Ninfe salirono alle stanze di sopra, doye con gran fa
de, il quale chiama il Cavalier di quoio, perché era v
demo. shee
ADDOBS!, Masserizie, ed arnesi per uso, ed ornamento
verbo addebbare, che vuol dire Adornare. Du Frefne nel Gloffari
dig Latinitatis. addobbare, armis inffruere, militare cingulum alic
confetka ex adoptare, quod qui aliquem armis instruit, ac militem ne
modo adopter in filixm, sì che Addobbare secondo questo autore vi
solennità del vettire i Cavalieri,
ZORIA, Aibagia. Vanagioria. » oe
SFUGGI. Vlanze fontuote canto di vestire, quanto d' addobbamenti
fatti con spiendideza, € più del consueto; Donde si dice fare sfoggio, 0
quando i trutti fanyo quantità grandissima di frutte, o quando chi
più del solito; ed in somma s' intende d' ogni operazione, che esca del
o cel naturale; come si dice frutta sfoggrata quella, che eccede img
beilezza, e supera l'altre fructe della sua specie. EB la forza della
venendo da seggia, cioè ulanza, el solito, antepostavi l', s, vuol dir.
foggia, cice tuor del solito, e del consueto. Gio, Villani quel che noi
foggi, chiamna difordinati ornamenti lib. 9. ¢, 245.5 e lib. 10. Cap.t
mo autore lib, 12, cap. 4. £ mon e da Lasciare as fare memoria d' una,
mintazion d' abito, che ci recaro di nuovo i Franceschi. EB poco foro,
natura siamo dijposti moi vani cittadint alle mutazioni de' nuomi abiti
trafare. Sfoggio dunque vale fuori di foggia., cloè dellafuzione, o VO
y.cniera' di fare ordinaria, e usitata; che il Villani comes'è villo
sformata mutazione a' abito; e disordinati, e sconuenenoli, e disonesti, ef
mienti ye nuoni, e iffrani abiti,:
CHI (a volesse rinedere. Cioè chi la volefle bene efaminare, o rice!
maniera questi cali possano fare Gimili sfoggi + 'i
SONO a specchio. Hanno debito. Traslato da coloro, che hanno d
decime, che si pagano al Principe, i quali \&i dice esser'.a specchio, p
notati a un libro, che si chiama lo specchio.; Qui dicendo.; sono
ghattiere, da:due colpi, uno che coltoro, che fanno tante borie non
gate, ¢J''altroy che questi loro sfoggi sono di robe usate, € vedute
poiché l'ha prefé'dal rigattiere, che vuol dire Vao, che vende mafleri
ed abiti usati. Vedi sopra C. 3. st 5.
POKTIERA, Paramento di drappo, o d'altro, che serve per
porte delle Ranze nelle case Civili. Da alcuni detta in Latino velum ada
TRA icanneili. Vuoldire fra la gente più vile; perché fra i cannelli.
mo fra i tefitori di lana, che son gente d' infima plebe, ed.¢ lo stesso
qualfineglia unto; perché questi tal: maneggiando sempre lane unte
sempre unti; e qui aggiungendo al detto fra i canned, il si
intende, che fino i Batulani, che fra gli unti sono i più vili »fanno le!
SEGGIOLE di punto, Cioè seggiole ricamate, o trapuntate di
mo: Panto Vaghere,0 punto Franzce, 3

BABeRew eft gFen2n2 se 8s =

=

a=
Se =z

SURFER


*' OTTAVO CANTIARE: 387
CASCAR nella pena de' cingue soldi. Quand' altri nel discorso fa una digrettio-
ne, €non torna mai a) primo proposito, gli diciamo: Voi cascherere nella pert.
de' cingxe soldi, 1 Varchi nel suo Hercolano pariando di questa pena dice: E chi
cominciate alcun ragionamento,e pot eutrato in un' altro, non si ricordaa prit di

nave 4 bomba 2 fornire il primo, pagava già, secondo teftimonio dal Burchiello, ni.

te

'a

4
wail
a

offo, 1 g 'o non valeua per aunentura in quel rempo più di quei cingue soldi, che
a ced Nae quali Lacie vegghiamo, che ST rarchs si serve del detto
Tornare « Bomba per tornare a segno, o al proposito del primo discorsa, come fa
il nostro Autore nel presente luogo. L' Ariofto Satira prima dice;

= Ma perché i cingue soldi da pagarte,

t Tx che leggi, non ho, ritornar vuglio

7 La mia favola, donde ella si parte.

 eARREN A, Intoppa; Si ferma; Non seguita il viaggio. Traslato dalle na-

Viquando si fermano, perché tuccano il lette dell' acqua, che si dice arrenare, 0

incagliare, De 1 gual) verbi ci serviamo per esprimere non tanto il fermarsi in un

'Wiaggio, quanto il fermarsi in un dilcorso, o nel proseguimento di qualsivoglia.
'aaione, negozio. Latino hnerere.
 PAil mangia da Siena, Fa il bravo. Fa il valoroso. Il Mangia da Siena è
\ di metailo atiai grande, la quale è posta sopra la Torre dell' orivolo
del Comune di quella Città, la qual figura dicono, che sia il fimulacro d@' uno an-
tico huomo bravo detto ii Mangia; Ma io son d' opinione, che ella sia il fimu-

lacro di qualche antico Podesta di Siena, e che habbia acquiftato il nome di 423-

54 da qualche inferizione, che havesse appresso, la qual dicesse Il eA/agna di Sio-
WAS COR i) ALagnifico di Siena, che s' intendeva già il Podesta: Ma sia com ef-

fer fivoglia, a noi basta sapere, che questo detto serve per in tender con derifio-
ne un bravo, o valente; quasi voglia mangiare le persone, e ingoiarle,

DISADATTO, Contrario d' Atto, destro, agile, ec, Uno che duri gran,
fatica a maneggiarsi,o muoversi per la gravezza, o per altro accidente, Sciar-
feancora e contrario di arto, e significa uno, che fa male, o negligentemente
quel 'ch' e' fa; poco pulito nelle sue faccende, e nella persona,

CON le Noha correr non puo porsi, Non può gareggiare con le Ninfe a chi più
corre. Interide, che le Ninte al sicuro lo superercbbeno nel corso;

VP Bun falir da Orsi. V' \& cattivo, o difhcil falire. L' Orso è un' animale, che
f ben, ir goffo, e difadatto, nondimena e assai destro, e facilmente fale anche
in ionghi inaccefibili; donde noi habbiamo: Efer come L' Orsa, cioè Sofuse destro,

Ui Berni nel Cap, al Fracaftoro dice:
Shy. Conniene ivi lasciar t" xfato corsa,
«ta £ falir fs per una certa [cala,
i Dove hauria rotto il colle ogni destr' Orso,
'Ostiero nell Iliade al nono chiama una rupe 50 balza Aigitips, cio8 dalle capre abe
“Vandonata; o questo medesimo nome di Leeips danno gli antichi a una Città dell'
Afola di Cefaionia, € ua' aitra dell' Epico. Noi diciamo di Jaoghi simili erti ri-
'Pidi, o scosceli: Won vi falirebbero le capre, le ee Virgilio nell' Egloghe dide:
repe. Quella montagna altissima nell' India; fu'la quale fu il primo Ale(-
fandro Magno a falire, fu detta da' Greci eornos, cioè senza uccelii, quasi mon.
oksal Cec 2 tagna

ca i


388 MALMANTILE

tagna da non potersi ne anche da chi avesse l'ale formontare
S7 muove come il gambero. Cioè va all' indictro, Wepam
MANG.ANATO, lnfranto; Mangano ( dal Greco mage:
na, con la quale si distendono, e si-da il juftco.a i panni, ¢
fare a forza di rulli sotto un gravissimo peso, e tal panno, 0
si dice poi manganato. E Mangano come s'accennd sopra Cw.
na militare della quale i nostri antichi si servivano per scagliar'p
fiediate,'¢ con essa scagli anche | ini, che dicevano poi
ganati, cioè sflagellati, e pelti dalla percossa; e così si potrebbe inten
ride; ma perché soggiunge passaro per frretrow, che è un' altra machina, ¢

ue per stringer ulive, ec., o per mettere in piega a panni, si vede, che
quel mangano da panni.
Ss ZA XVIL
UN un Dormentorio grande, ma diverso,
Ove ciascuna in proprio ha la fuaceltay
Che sia com' io dir per questo verso,
( Se non erra Turpin, che ne favella )
Vana fanga a mez aria tuna travorfo, ¥
Dow' alla tien se calze,e la gonnella, w
i penzol delle forbe, e del trebbiano, ag
E quel che più le par di mano in mano; fa
TANZA XVIIL. ae
Più giù da banda un tavolin si vede, er
Che sa i tre/polifa la mnna nanna, Sxpenfa, che vi ina
E sia spatiiera al muro, ove si vede Ada trova im ozso tutti gli 1
Via fiuvia di giunchi,e fottil'canna, E i piatei ripulisi come sp fai
Evvi una madia zoppa da un piede, Teglie, e padelle, inutile a
E il filatoio con la sua ciscranna, Star'appiccare al muro per gli si
Won v'è letti, se non un per micliaio, Ed anche son per sparut pilt a et
'Che tutte quante dormono al pagliaio, Perché il gattoa dormir vedein, | dy
: STANZA XX14, sth ng
Ond? egli offefo molto se me tiene, ») (Gliaccanan ch'ei vedrà fel ta
Ch' una mantita per la golatocca; Ed ei ghignando allor più noms (i
Ma quelle, che s' avveggon molto bene, E con esse ne va di compng §
Chregli bal'arme diSienarpreffain bocca Per ultimo a veder la Gi 7 hk
Descrive nelle presenti Octave il dormentorio delle INinfe,e ledorowmafieria® | Yu
Arrivano ee cucina, dove Paride a (aol e de pr ta
arata 'cosa alcuna 'per mangiare; Ma ie Ninfe lo quictano'con dirgli, ch e
ey ada eiare} 0d de lo'condi a veder la Galleria, Pe
'DiVERSO, Differente., o 'dissimile aghi-altri Dormentorj, perché: «
Celle non 'fon fatte di muraglia, ma son tutte in-una grande stanza-y §
vile con stanghe app al:palco ciondolot foa r
quali poneng jo ciascuna'le sue 'robe, e panni le 'fa servire per muro
.così vengono 'fortnate'le Celle:. 'Si può anche dire, yche la voce
dae significati il primo,'che vuol dire diferente ( e gquesto f a


OTTAVO CANTARE 389

'messo per contrapposto, come la-tal cosa e diversa dalla tale ) il scondo quando
po 'ate che vuol dire strano, o stravagante, il Poeta lo piglia ia
Recornto significato. — lo piglid Dante Inf. C. 7,
 Entrammo g pen via diver[a, Oc,
Cavaleamti nelle sue storie lib, 12 parlando di Cammillo quando ifefe il
lio dice:, Non guardo all' ingiusto cacciamento, ma con grandiffi-
i yy mo esercito corse alla dife(a della patria, e liberolla da così diversa fortuna.
7 es, Ricordano Malesp. Stor, Fior. cap. 80. dice: E ciò fu per l'invidia della Si-
> ae i » che non era.al loro volere, e fu diversa, ed aspra guerra. Vedi. fo-
int 2, stan. 3.
bis Beene del trebbiano. Che cosa intendiamo per penzolo vedemmo sopra C.
6. stan. 50. e Trebbiano € specie d' uva bianca, ma.qui e preso in generale eee
(IL ogni sorta.d'uva,\che's' appicca nelle stanze per ferbare all' Inverno.
 DI mano in mano. Di tempo in tempo. Lat. Deinceps, che s' intende fuccelfiue
neat:  Cic, 7, Ep. Fam, disse De manu in manum. Dan. Par. 6, dice:
jas E [otto t' ombradelle Sacre penne
we \Governo il mondo li di mano.in mano.
yin
ye
si
yh
ue
ae
lf

. Ed  detto figuratamente-dal far passaggio una cosa dalla mano d' uno nella.
'mano dell' altro.. Dal giuoco. detto Lampade dromia, nel quale colu aveva il
-vanto che va una fiaccola accefa correndo., e così bella, e accefa la confe-
gnava.a chi aveva.a correre dopo di lui; disse Lucr. lib, 2. Augescunt alva LEnbes 9
lia minuuntur, Inque breni [patio musantur fecla animantum, Et quasi curfores vite
Po eamema 9 Gi0e succede t'.uno vomo all' altro, l.nne vinente all' altro di mano

0 roc, Dal Lat. tripus, odes, E un pezzo di legno, o ceppo., in cui
a son fite tre mazze, fope' alle quali posando., serve;per fohence tavole, e deschi,
oe da i Latinidetto Trapecophorus.s'quali mensam ferens.

' | PAlaninnananna. Non Ma forte in terra, ma dimena o:per l'inegua lita de-
ri N 'te tre mazzc, o del suolo,,.o per altro mancamento; e diciamo far /a ninna manna
Od eda,quel dimenare;che si fa-della,culla.de ibambini, 'quando dallebalie si procu-

ache dormano, che si dice ZVmnare, spetche per lo più sogliono accompagnare

4 -talmoto.con una lor cantilena, che dice Ninna nanna il mivbambino. Vedi sopra
iS 'Cae Renaes. Questo dimenare si dice anche:ewlare pur dalla Quila de' bambini..
, SPALLIER A, Quella'parte della seggiola, alla quale:s' appoggiano le spalle
mA Aedendos |B per /paliiere intendiamo quelle nuragli¢ 5 alle-quali sono.appoggiate

spianted'agrami, ec. come's'¢ detto sopra\C. 6, stan.'51. Questo artitizio-di
Farele:mura:coile piante-dicesi-da alcuni in Lat.-opus topiarium.. Equi nee
'quel-'muro parato di stuoie tatte di giunchi, o-canne paluftri, che fourasta.alla.
oat »sopr' alla: quale dice;che fedevano le Ninfe, ¢serve,per spallicra alla.me~

J

3 STVOLA, B il Latino Storeache conlerua appresso noi il suo:significato.,

il HUADIA, Dal Latino maetra,i| qual pure e Geeco;.ed una cafla sadatrata
it sopra-quattro:piedi, dentro alla quaic si lavora la patta per far-il pane; La dice
3 Zoppa.da'un'piede perché le: mancava., o crarottouno diguefli piedi.. Zoppa si-
'il siete den tee cain tavon della vecchierella Bayside la;presso rida



390 “ MALMANTILE

lib. 8. delle Trasformaziuni; ma ella la fece stare pari con me!
to; mensam fuccintta, tremenfque Ponit anus; mensa sed erat pes ter
sia parem fecit, i, 2/1 ihe
FILATO/O. Strumento col quale per via d' una gran ruota si fila Jan:
napa, ec, e si fanno le'funi. 1) OL HR
CISCRANNA. Specie di seggiola come accennammo sopra C,
DORMONO «i pagliaio. Cioè dormono in fu la paglia.
HVOMO alia buona, Huomo schietto, fincero,e senza malizia; Huo

za cirimonie, e nimico del luffo, e delle boric fine fuco, o fallacijs, | Ve

sornm, ed Hxomo posirixo intendiamo uno, che non fa sfoggi nel veltire, ¢
ogni cosa si tratca senza lufflo. SO
SENTITOSI allegare i dents. Vuol dite sentitosi (timolare dalla golae dal
desiderio di mangiare; se bene allegare i dents vuol dire quando i deat pert
matfticata qualcola acida, o agra. coine 'il limone:, ec. s*iavormentileono, e i
sente una certa difsculta nel mafticare.Ma usandosi come nel pretente iuogo,vu0
dir venir voglia di mangiare.:
TEGLIA. Specie di tegame fatto di rame stagnato per di dentro, serve pe
quocervi torte, e migliacci, ec. (| Monofini lo fa venire dal Greco Telia, a
gual voce tra l'altre cose signitica l'a/se da pane, e"| turacciolo,o coperchio del fum
maiuolo, o vogliam dire di quel canale, che gli antichi, in vece di cammino,ave
vano per servizio di cucina, buono solo a ricevere,¢ porcar via it A
dicendolo molt Tegehia, e gli antichi in particolare, mi muovo a
venga più costo dal verbo Latino Tegere. Queste teglie hanno nell'
ta una campanella di ferro per comodità d' appiccarla, e le padelle hanno un
anclio in cima al magico per il medesimno essetco'; € questi fond gli orecchi de'qua-
li parla il Poeta dicendo: Stanno appiccate al muro per els orecchi., Ovidio lid. \&
Metam, erat aluens illic Taginexs, dura clauo fuspenfus z anfa, hia
TORN/ZE, Parlando di gatei s' intende quel ronfare che fanno; perché e \&
mile a quel romore, che fa il tornio quando gira. + Aba
TOCC A una mentita per la cosa, Dar una mentita per la gola a uno e quando
se gli dice, che egli afferma il failo, ed è grandissima ingiuria, e che muove al
¢ pero il Poeta scherzando dice, che\Paride si adira per l'offela, che ri
quella mentita per la gola, cioè di quel tupposto che vi fusse roba per la golasy
che fu falfo, WS EE
Li arme di Siena impressa in bocca, L' arme di Siena è una Lupa, ed il mal dé
la lupa e inteso comunemente per una infermita, che fa stare il pazziente in col
tinova fame; onde quando vogliamo intendere; il tale ha gran fame diciamd:
Egil ha il mat della iupa, e pis copertamente Egli ha ? arme adi Senay es'
la lupa, cioè la fame. Vedi sopra C. 3. stan. 22. Kh
VEDKA',  il corpo teene, Cioè mangiera, e bera. Detto assai afato”
gente di vil condizione. » 1. 3a
GHIGNANDO., Ridendo leggiermente. Lat. fubridere,
GALLERIA, Cosvin voce straniera chiamiamo aicune Manze piene
nate di-galanterie, ¢di-cose singolari, e maraviglioe 3 quali ttaaze
son dee Pmacorheca dal Greco Pmax, che suona tabula pita, o theca

oe

oe. >Eeerec-se.= ez ER PTE



orre al
STANZA XXIL

fb Principi ritvatti ye di Patrizzi,
 Originali farti già in Fiarenga

4) Da quel, che gis vendea sotto gl' usizri,

» Ed euns dello feeffo una Sibilla

 Eduna bella Cittadina in villa.

Bt STANZA XXIIL

me

SE aa

tapelPa fole, e sgabelli
intorna inalzan sopra al piano
 Statue eccellents di quet Prafiteli
BH Già shalt danno il moto in. Settignano,
Ce * Buonarrnoti, e i Donated

Caer

— Aquel baffa ritseva di lor mano

z

ORTAVO CANTARE. jor
erste: Sper altro Galletia voce militare e specie di fortificazione.
xX

TANZA XkXIV.

sì che que? opre, che non hanno pari,
Quantoi (uddetti quadri,c' han del vago
Non si polfon pagar mai con danari,
Perché son giore, che non hanno pago;
Vao scaffale v' e di libri vari,

Ch' eran La libreria di Simon Mago,

C' abbellita di scorie, e di romanzi

Fu pot venduta lor dal Pocauanzi,
STANZA XXV.

Exni un tomo fra gli altri scrittoa penne,
C' ame par bello, e piace fine fine,
One si legge in carta di cotenna
Tradotte le librettine in feftine,

E che Gateno, e il medico eduicenna
In musica mettean le medicine;

ww \&

Leo, s' sl corpo sempre a chi le pigli

GC as paari scalzi pur si vede ancora
' Gorgheggia,¢ canta,nonè meraxiglist,

| Sw t arco della porta per di fuora,
+3 9A da principio'a descrivere la Galleria delle Pate, e narra la bellezza
4 aicune pitture, e statue non dissimili dal refto delle maflerizie, per esser' opra
ade ad pilicimuniti: Artefici, (e bene scherzando gli efaita sopra i più eccellenti
Macitri, Oitre alle picture ve anche wo foaffale pieno di libri dei medelima yaio-
IE ye. » che sono te pitture, e scolture.
FRONTESPIZZ/. Vedi sotto C. 9. stan. 15. i
MAIOLIC 4. Specie di piatti, ed altri valeilami di terra, la quale meglio,
Che im aites iuogh: si Javora oggi in Faenza;¢ questa terra è detta maiolica dall
Mola di Adsiorica, o Adaiorca dove già si fabbricava; €l'lfola che diciamo oggi
Maiorca già si diceva Maiolica,, come si vede in Gio; Villani lib. 4. cap. 30. WVe-
54 anni ds Cristo 1117. Qui Pifani fectono nna grande armata di Galees e Navi, ed
ndaronofopr' ali' Hola di Adavolica., B che iaquelta lola si fabbricafiero tali va-
lami si deduce, non solo da] nome, che ritengono di Maiolica, ma anche dal
Vedersi nelle fabbriche antiche di Pifa 4 etparticolarmente nelle facciate delle
Chiele murati di tai piatti come per trofeo, e memorie delle vittorie havute da
i Pilani contro ai Maiorchin: 5
VNA bella Cutadina in vila, Era già in Firenze un Pittore da pochi soldi, il
quale faceva ritratci di Principi, di donne Fiorentine in abito da Villa, e da Cit-
ta, de Sibuie yle Mules ec, -¢ tucto.così malfatto, che non eran comprate tali
picture se non da genti di contado,¢ per vilidimo prezzo. Dette pitture si ven-
devano sotto le Logge, che sono d' avantia quelle stanze, dove si radunano i Ma-
giltraui di Firenze,c questo luogo si dice sotto gli Vfizei, e per una bella Cittadina in
Villa, € una Sibilia intende di queste belle pitture.

D1 quei Prafitelli, Di quelli Scultori valorosi, e celcbri, come fu Prafiteles;
/parla però ironicamence, e per derisione. Praffirelle detto poeticamente come
Annibaile, Eetorre, e simili per ia rima,in vece di Praffitele, Annibale, Ben '

“hoo 0

=SERS BQ SSESHER ES

A

2
Se

=

a= EE



392 MALMANTILE™
Così i Latini raddoppiarono la Lat. in Relligio, x Relient a ¢

la legge del verso. 4
CHE a i sassi i. daensit mined Settignano. Dar il moto ai fafti,
si vuol dire Formar figure di pictra: Virg. vines ducene de marmore
Settignano Borgo vicino a Firenze abitano quasi tutti scarpellini
fabbricano poco altro che stipitt, sCaglioni Primi if
che di case, ec, talvolta Javorano anche delle figure » ma per lo
le suddette pitture; e pero il Posta scherzando dice, dannoi moto
che voglia dire animano i fafi,, fabbricando statue, che peiolowive
de, che danno il moto ai (atli, cioè gli muovono, ed è s I
quali sono ia.quei monti di Settignano, lyogo detto così quali
dere, o posieilione della casa Seprimia, antica Romana, siccome
della Perronia, e altri molu iuoghi dello Stato, che risepgane anon
padroni, nobili Cittadini dell antica Roma,

QLVEL basso ritievo di tor mano, Se, Perché fir' c d
erano queste statue » porta l'esempio d' una figura, che è: nell irchi
ste della Chiesa di 3. Paolo de i Carmelitani Sealzi, che è-una

afio rilievo, la quale rappresenta, o almeno dovrebbe rap,
lo, maè lavorata così maravigliolamente male,.ches'è rela
sua storpiataggine; ed €-compagna delle (tupende pitture del Pamo
Zannino da Campugnano. Intendendo dunque al omen Boera
tre figure, che le sono attorno fatte della medesima maniera vuol
che si vedevano in quella Galleria eran maligimo fatte,

NON hanno pago. Non hanno prezzo: E? parlareironico, e-wuol
hanno prezzo, clot non s' apprezzano, non si fimano, non vaglion A

SC.AFFALE. Armadio aperto fatto.a palchetti per uso di tener libri. |
nome di Sehapha, e di Scapbos si dicono in Greco molti arnesi,e stramenti, 1
tutti'o concavi, o- scavati per uso di tener roba, dal verbo scapresm
re cauare,scanare, Onde scaffaie, arnese, che ha varie capacita, €
ne' quali si ordinano,¢ si pongono i libri Lat. platens armarium

SHMON Mago, Fu lt Autore, e capo de'. Simoniaci; essendo-
che tentafie di comprar da 5, Piero i beni.Sacri, e Spirituali, come si
acti degli Apostoli. \& che cosa sia Mago, Vedi sopra C, 1. stan. 20,5

POC AVANZI, Fu un Libraio Fiorentino così detto, ii quale nel'
 Autore compose la presente Opera era ridotto in poverta, € vendeva'
che leggende.

CART A di cotenna. Intende Cartapecora.

LIBRETTINE, Quel libretto, che insegna conolcere le figure dell"
¢ le prime regole del medesimo. Il Burchicilo..Vedilo andar,¢h e* par
tine, cioè e tanto magro, secco ye ee C' pare una signrad
tini un macilente, efienuato » e deforme nelio stesso modo.
grammo, ioe delineato solamente,¢ fattovi il (olo,¢ puro din

o colorito.
MEDICINA. Quando si dice semplicemente medicins da noi?
 bevanda folutiva, che si beve'con la preparazione, oe ¢
ta prima con alcuni sciloppi, cc.



OTTAVO CANTARE., 393

| GORGHEGGIARE, E} termine musico da i Lat. detto Vibrifare, ed \& un tril-
lo di voce fatto con la gola, al quale in un certo modo € simile quel romore, che
fa nel.corpo il vento, o altra sollevazione d' umori cagionata dalla medicina,
ed il Poeta ii » di questo romore, che fa il corpo dice, che il pazziente>
può far di meno-di non cantar così, poicht Galena, cd Avicenna haveyano
in musica tali medicine

b STANZA XXVI,
ave n't in rima, che la Sfinge e detto Perch' ci, chefa, chee Sale bebbe concetto,
| Seelta d Enigmi, che non hanno nguali, Acciò che i versi suct sieno immorcali,
Perch! agnune e distinto in um fonetto, £ i vermi dell'obblio non dien or noia
we Che if Poeta ha ripien tutto di fali; Porgli fra fale,e inchioftro in falamoa.

Bra questi libri delle Fate si crova anche la Sfinge, che è una scelta d' Indoyi-
i distinsi ciascuno in un fonetto, opera del sig.\ Antonio Malatefti; la qua-
Ieil nostra Poeta ( facendo di essa quella stima che merita ) non haverebbe metia
ion le, se il medesimo Malatelti non ! havesse forzato a farlo,com-
lo egli medesimo la presente Ortava nog alterata punto dal nostro Poeta.
fale Opera conticae ( come habbiamo detto ) Indovinelli, il Malateiti
il nome di Sfioge, che fu un Maitro appresso a Tebe, Figliuolo ( secondo
no )del Gigante Titone, e di Echidna, che significa Vipera; e Fratel carnale,
il, della spaventola Gorgone, del Can Cerbero, del Serpente
di pi tefle chiamato Idra, e di pi altri mostri e animalacci, il qual mostro di-
4 -tmorava in-un monte contiguo a Tebe sopr' ad uno scoglio vicino alla strada, ed
| a chiunque passava proponeva wx dubbie[ che i Greci dicono evigma, i Latini
nt 'uphas pure dal Greco; e noi indoninello come s'è detto sopra C. 6, stan. 34.]e
; Leqlioa ace lo scioglieva, il mostro improvvifamente lo pigliava,e}' uccide-
'i va. Agcadde, che Edipo figlio di Laio Re di Tebe fu quivi mandato, ed il Mo-
, fico gli propote: Qual' era quell'Animale, che da principio andava con quattro
“ piedi, poi con due, ed in ultimo con tre = Edipo rispole, questo esser ' huomo,
"i, che da bambino va carponi con le mani, e co1 piedi, € così con quattro piedi,
se poi rittoin fa due piedi, ed in vecchiaia con tre, perché va col bastone; E con
tal folygione vinfe il mostro, che percio si mori. Pe
 RIPLENO di foi. Ripicno di belli, ed arguti pensieri. I Latini ancora chia-
vif mavang falil' arguzie, trovandosi in Orazio.. Nofri proaui Plautinos landanere
fale, Giulto Liptio Aatig. lect. dicit se amare elegantes Plauti fates, Lucano: Non
ie Solici ifere fates. Tor. in Eun, Qui baber falem, qui in te edt, intende scienza, fa-
“if 'pere. Ma qui.' Autore scherzando con l'equivoco del fale dice: Che il Mala-
teftisil, (a che cosa e il fale,¢ che cifecti partoriica [ perché egli era guar-
dane azzini del Sale di Firenze] 'ia messo de i fali.ne i suoi fonecti, per
1 fr loro falamoia con } inchioftro, athaghé i suoi versi si conferuino, e si
mw?  difendano da i tarli della dimenticanza, sapendo, che il fale conferua, o difen-
we ic ins; e le composizioni si conferuano da i vermi dell' obbiio con,
g@  Icriverle, € questo si fa con ".inchioftro, e pero lo chiama falamoia, I Latini
cone la la Murra, del che noi componghiamo la voce /alamoia, quali
falis mursa, l'iachiottro da Monsignor Ciampoli fu chiamato dal con(eruare s¢
Orie € i noint degli huomini Bai/amo della fama,
t Ddd STAN.

“

a



394 MALMANTILE o

STANZA XXVIL
Altri Poemi poi vi sono ancora, E uncerto Mal
£d hanno caparrato alla Condotta Ecco subito bell! ¢
Grillo ilGiambarda, Ipolito,e Dianora Le Deecol Babi, chel ha
1 ferre Dormienti, e Donna Isotta; i Z
Narra che moir' altri Poemi sono in detto scaffale, e mette t
frottole composte da' Cicchi per le donnicciuole, e per i fanciulli.
genie dice, che fara ancora la presente fuaopera, '
sNC AP ARR ATO, Data la caparra cioè dato danari innanzi per fert
mercanzia per conto proprio. ( Voce formata, dice il Perrari, da cape a
Qui vuol dire che hanno chicfto lu MALMANTILE, Gili antichi d
rare da Arra, caparra. "
ALLA Condotta, Così \& chiamata a Firenze una strada, nella quale
botteghe i Librai, e alcuni Stampatori, ed è così appellata, perché
stima strada haono i magazzini coloro, che tengono 1 muli per lao
mercanzie a Roma, a Bologna ed altrove..
MESSE in rotta le Dee col Bambi. 11 Bambi era uno, che vendeva
maggio, ec. che noi chiamiamo Pizzicagnoli. Dice che le Ninfe sono p
car lite con detto Bambi, perché esso impedira, che elle non habbiano il B
di MALMANTILE,, volendolo egli per farne alle accinghe tance ¢:
per inuoltar falumi. Ed in fuftanza vuol dire, che la prefenve sua Opera'

2 2 eee.

r=

na per vendere a peso per carta al pizzicagnolo; che così diciamo
che un libro non habbia in se di buono altro che la carta.. E qui se
dice questo per sua umilta, e modeflia [ non essendo la sua Opera da
pelo per carca j tuttavia, non sapendo che la mia penna dovea farle meritare
tal fine, fece buon pronoftico, € non dubito, che havera dato nel segno I
Lalli nella sua Franceide C. 4, stan. 21. Si servi di
E le cartacce lor servono al fine
Per avvolger U acciughe e le Tonine,

STANZA XXVIII.
Bovvi anch' un libro ds segreti, il quale
Gioua a chi legge, e insegna di bei tratti
Ed infra' altre a far che le cicale
Cantsn senza che'l corpo se le gratti,
Ea far ch' i tordi magri con? occhiale
Guardandogli divengan tanto fatti,
Deferive pos moltissimi rimedi
Per chi parisce de i calli de' piedi,

STANZA XXX.

Perchi la donna come altera, e vana
Sopr' agli sfoges ognor pen[a,e vaneggia,
ae cht el?” abi un ceffo di befana
Pomposa,e riceavuol che ogni la veggia;

questa medesima frale. ee

STANZA XXIXi
S? io vi narrassi tutto il
Costui, direfti, ha it cera j

Pur vuo! contarnen' una folamentt R
Chie vera, ne crediare eb io sarfilh "i
Racconta a! una tal parturientt:
Ch' una carrenea seen faeae, \&
E ch' una voglia fu, che bawen bavwsy b
Ed io lo crederé senza dispura.
Percio colei bebbe la voplit;
Della grandezza dell' I
eanceeioeadeg robe ik i
Le girelle vorrian, ebe'tfa 1
E è


rt

a

SB SSE CRELESE EGE

SEES SRSA Seth

i

Ma hafti circa i libri quanto ho detto,

OTTAVO CANTARE. 393
“STANZA XXXL

ed qualch' error novoglio far fuggerte,
 Perch'ioche negli Pudi non m'imbrog lio, Che pur eroppin' ho fatti for' al foglio,
eee altri non ho letto E pot perché fom tanti,¢ tanti i tomi,

Lorfe i fatti lor saper non voglio, Che ne anche fo dir d'unterzo: non:,
eos il racconto de i libri, che sono nello scaffale,¢ narraado un favoloso

=,  iperbolico parto, fa una leggieri fatira contro al luo delle donne.

 10 sfarfaili. lo aggiunga al vero: Io m' avvantaggi acl racconto. Dalla far.
falla, che gira e s' avvolge or qua, or la, e detto sfarfallare.

 ¥NA vglia fu, Che cosa sia voglia in questo proposito. Vedi sopraC. 2. st. 42.
— ALTIERA,e vana. Altiero, si può dir finonimo di superbo, pigliandosi
spesso ' uno per l'altro; se bene a/tiero si dice colui, che per grandezza d' animo
non riguarda,¢ non applica a cose vili, anzi dimostra vers di quelle una cerca
schifezza generola, e senza vizio,¢/uperbo si dice colui, che per vizio, e per

apriccio spropositato disprezza tutti, e tutte te cose indifferentemente, e senza
Thasoee alcuna. Qui, dicendo a/tera intende piena di prefunzione di se stel-
fa, che e lo stesso che /uperbo; e Vana dedita alle vanita, o vanagloriosa, boria.
fa, li Petrarca distingue queste due voci, dicendo nella Caaz, 22,

costs + Ch' in vista vada altiera, e difdegnosa,
Non superba, e ritrosa.

 BEF ANA, Significa Donna malfatta: perché befana diciamo un fantoccio fat-
todicenci, che si suole da alcuni mettere alle tinestre il giorno dell' Epifania, il

jale da Epifania e detto. corrottamente il giorno di Befana. Vedi sorta C. 9,

Si

I,

TREGG/A. Intende carrozza. Se ben tregeia è un veicolo ruftico senza ruo-
te per uso di portar paglia, e legne, ec. facendolo tirar strasciconi da i buoi.
Servio sopra quel verso di Virg. 1. Georg. Tribulaque, traheaque, o iniquo ponde-
re rafirs dice così. Traha genus vebiculs dittum a trahendo; nam non haber rotas,
edè la nostra Treggia. F:

4L sangue tira, L? inclinazione, o genio le spinge, le forza, Intende che le»

irelle, che le donne hanno in testa, havendo simpatia coal' altre girelle, fanno
Seiderare alle donne quelle della carrozza.

NON m! imbroglio negli fudi. Cioè; non attendo agli spudi; nan ho che fare con,
loro; non mi intrometto di fiudiare; nan me ne impaccio,

PUR troppi n' ho fasti sul foglio. Per modettia intende; Pur trappi sono gli er-

rori che ho fatti nel comporre la presente Storia.

STANZA XxXU,
Però seguiam con Paride le Dee
A veder cose belle, e Strauaganti;
E prima tronerem di gran miscee,
 Corpi di Mummie,ed ofa di Giganti;
 Her in corpo a pesce due galee,
Tmpietrive com turti i naujganti,
eps 9 li quali esse han per tradizione
Ci

fur fatti del gingerol di Nerone.
Larti del gingguol di Ner Hid

STANZA XXXIIL

Chinfe nel vaso poi vedrem le cotte
C' bebbe quel Vecchio Chioccia di Sileng,
El asta che fu, dicon, di Nembrotte
Con che voile infilzar 2 Arcobateno;
Benché si creda più di Don Chisciotte,
E veramente non puo far di meno,
Perché in vetta nel mezzo della lama
V' è scritto Dulcineach' erafuadama,

2 STAN.

396 MALMANTILE

STANZA XXXIV.
Pende dal palco un secco gran Serpente,
Che uh al Cocodrilo s' assomiglia,
E dicon che la coda folamenre
Per laliighezza arrina a cingne miglia;
A1a quel che più curioso di niente
E' certo, è una grandissima conchiglia, /
Ouxe fra minuta alga, e poca rena Chi vi dipana fa quant'
Sta congelaro un' uouo di Balena, C” al fin @ ogni gomitol si
Lasciato il raceonto de' libri, torna l'Autore a narrar le cose mai
singolari, che sono in questa Galleria, E perché in tali Gallerie i proc
le fa di riporvi cose flravaganti, ed ancicagliec ragguardevoli, e molte da
ne fingono per accreditare il luogo, € pero 11 nostro Poeta mette anche
mano di cose iperboliche, come sono due galec impictrite in corpo 4 |
€ favolose, come un vaso pieno di gotte, ec, Vedi Liaciano nell' fitoria
ove delcrive terre, ed huomini in corpo'a una'balena; B Efiodo, ove
il vaso di Pandora, ove erano tutti i malori, e tutti i malaoni,
AUSCEE. Intendiamo bazzecole, mafieriziuole, ed arnesi vecchi di
prezzo, che habbiano del curioso; metcuglio di bagattelle, di curiosita ¥
AV MME, Vedi sopra C, 6. stan. 52. i
GWVGGIOLO di Nerone, Habbiamo un 'nostro detto, che è: Meron
ginggiole, che serve per esprimere; 4 fortuna mi s' artranerfa; Ml Diaual
disce l'efecuzione del mio pensiero, E viene non da Nerone iimperadore,
contadino chiamato Neri, il quale stava sopra un giuggiolo, osservando
che entravano in casa sua pee rubare, e toftoro accortifi a' esser;
mostrare che gli volevano fare una burla, ¢. non rubare: gli ditiero; 4b WV
\w sei in sul giuggiolo, intendendo: Noi t' havevamo ben veduro. E del lepname
di questo giuggiolo dice, che eran fatte le due alee impietrite incorpo.al pele.

VECCHIO chioccia, Vecchio malandato. ' uno, che sia alquanto infermo de
ciamo chiocciare; dalla chioccia, gallina vecchia,e spelata, che cova i puleitl,
come il malato cova il letto; e !Autore chiama Suleao vecebio chioccia
Icno Pedante, ed Aio di Bacco si faceva portare topra aun' asino, C
mezzo infermo; ed i Gentili dicevano, che egli si trattava in questa forma, per-
ché essendo egli il maestro di Bacco, il quale € numerato fra gli Dei el
amici delle comodità, e del piacere, era gitilto, che fudde un' huomo di tuttil
suoi comodi. ar tag

VOLLE infilzar  Arcobaleno, Volle infilzar \Areo-Celefte; che il chia
mavano Iride, e la dicevano insieme co' Greci. Atmbasciatrice degli x

wee

4B. 5.
: Frinde Colo mifit Saturnia Tino,
Ed il nostro Poeta dicesche Nembrotce vole injfilgar \& Arcobitteno
fu quello; che Pe eer si pensd di voler guerreppiar col Cielo ed.
to fabbricd la famosa Torre'di Babel, cioè della confusione. ar
DIN Chifeiorre, Che in nostra lingua voreebbe dire: Di
mile, Fu un Giteadino-delaMuntia, il quate havendo letti molti

pA seop ok FLFR e un lere2t sere res e-2

ae



OTTAVO CANTARE 397
valleria, cio? Amadis di Gaula, Palmerino d' Oliva, ec. s' imbriacd, eddinuaghi
dei meftiero di Cavaliere Brrace di tal maniera, che si messe ad immitare le azioni
di detti Cavalieri » facendosi armare con quelle cirimonie, che eran soliti fare
“quei; anch' egli a cercare l'avventure, come graziosamente rac-
conta 26 Michel Ceruates'nel suo D6 Chisciotte,il quale fu molto bene tradotco
nostro volgare da Lorenzo Franciofini da Castel Fiorentino, assai benemerito
' a Spagnuola; (1 aggiunta, o secondo libro del qual racconto' voglio-
 no, che sia stato composto da Carlo V. Imperatore ) E perché i Cavalieri Erranti
Ron erano stimati veri Cavalieri, se non havevano l'innamorata, però questo
@ Don Chisciotte si finfe ancor egli la sua, che fu Dulcinea del Tobofo; E da questa
ae il nostro Poeta prova scherzofamente, che questa Atta fusse più tafo di
| Don Chilciotte, perché nella lama', che era.in cima alla detta asta v' era (eritvo
'Daulcinea, ed intende, che questo ferro era doice, cioè di cattiva tempera.
FN gran Serpente, Questa iperbole del Serpente e posta qui ad immitazionc, o

iat per dir meglio, in derisione di coloro, che scrivono le Storie d' Etiopia, che>
wi di -esservi tali Serpenti, che ingoiano un Ceruio, o un Bue intero per volta

 €sono di lunghezza di piii di trenta piedi; E che M. Attilio Regulo nella prima
til > oda ai Cartaginefi ne uccidefle uno in Affrica preflo al fiume Bagra-
it che era lungo 120, piedi.

MANTICE, o mantaco. Vedi sopra C. 1. stan. 55.
si) = SARCOL AIO, Steumento fatto di canne rifefie, o stecche dilegno, sopra il
wai 'wales' adatta la matafla per comodità di dipanarla, o incangarla come s'è det~
wi WifoprarC. 5, stan. 9. E dipanare € raccorre il filo,formandone una palla per co-
shi imetterlo in opera, e tal palla si dice gomitolo dal Latino glomerare, e+
i Soma che il gomirolo, che a Roma ancora si dice glomero.
4) STANZA XXXVI. STANZA XXXVUL
se Van Sfera bellissima si vede, S'.in Grecia fatra fu la criftullina,
nis o (Ch sopr' aun ben tornito piediffallo, E questa di vesciche vien da Troia,
ae 'Che per ginfiezza tutze l' akre eccede, Che a Fiefol fu portara a Catilina
ail on farte di legno, o di metalios ue norte ch' ei ee verso Piftoia,
es pure, e fotrerrifi Archimede Ch' ei non giunfe ne anc! alla mattina,

i Con lla sua, ch'ei fece ai Criftalla, Chet arate wi tio le quoia,
am % Che bisogna guardarla,e /parsi addietra Sicché due Capitan sue camerate
; ie “ “Per'timor di non romper qualche verro, La presero ye la diedero alle Fate,
2 STANZA XXXVIL STANZA XXXIX,
Che questa, che con ogni diligenza Mentre s ammira così-bel lauoro

Di purgate vesciche fu commilfa,

E vi si fanno fu cento argumenti,

iv Se perdisgraia, o per inavvertenza Paride guarda 5¢ vede una di loro
ae Perquote ocade,ell' e Sempre la fiefa; Canarsi un' occhiolaparrncta,esdenti,
E sel criftalio ha in se larrasparenza, E dargli aun' altra,perch'inturto ilcoro

 LA vescica al Diafano s' appresta, Delle. Naiadi ch' ini fom presenti,
if --Edè\un corpoyche giammai non varia, O fuora (che pur anche son parecchi )
eo E-quel si cangia ognor secondo t' aria, Ha fol quei detrynn'ecchio,e due cernecebi

Se.

STAN-


398 MALMANTILE. o

STANZA XXXX.
Pero ch' elle son cieche, e vecchie tutte,
E loro i denti son di bocca nsctti,
Ma ni per questo ell' appariscon brute,
Ch' ell! hanno volti bel, e coloriti,
E se mangiar non possoncarne, efrutte
Elle s' aintan con de' pambolliti
Perché quei denti,come gli occhi,eiricc
Non hanno pin virtit, che fom posticci.
STANZA XXXXIL
Così per offernar le lor vicende
lucha ch' io dico se gli caua adcffo, Cedendo ogni ragione,¢ ogni
Già ritornata dalle sue faccende, Perch' inqueff'oraa è
Perch' il portagli pin non le e permeffa, La fronte escape,erife
Descrive una Sfera fatta di vesciche di Porco, e mostra, che sia
re di quella di Crifallo, che fece Archimede Siracufano, perché e più f
più sicura. Mentre che Paride stava mirando, e dilcorrendo sopra ilb
della Sfera di vesciche, una delle Ninfe si cavo la Parrucca sun' Occhio, 1
¢ dette il tutto a un' altra, perché così e l' ordine fra loro, Qui pate, che:
alle Lamie, Donne, o Larne per dir m lio, che con carezze allettatrici
stimate da' superftiziosi Gentili mangiarsi i bambini; le quali fea cutre
no un' occhio solo, e quello usavano a viceada hor questa her quella,se
deicrive Angelo Poliziano lib. 3. tit, Lamia, che dice: Lamia h
excmptiles, hoc eft quos fibi eximunt, detrahuntque cum libuit, curl
y» cum libuit refumunt, atque affgunt; alice vero ctiam dentibus utuntur eque
y» exemptilibus, quos noete non aliter reponunt, quam togam, ficut ha CO
>> mam (uam illam dependuiam, \& cincinnos, \&c. Sed lamia hac quoties do
egreditur oculos fuos fibi afhgit, vagatur per fora ee plateas, \&c, domum ye
»» ro cum revenic, in ipfo fatim limine demit illos fbi oculos, abijcitque in le
»» culos; ita femper domi cca, foris oculata,,
PIEDIST ALLO. Bi quceila pictra, che e sotto al dado, sopra il quale pola
colonna: e qui e preso per tutta la bafe, che regge questa sua Sfeca, comet prt:
fo comunemente. aia
VADA, efotterrifi eArchimede. E'oscurata la gloria d' Archimede; Quan?
uno fa un'operazione meglio d' un' altro diciamo al superato; T ti pwoi ire ari:
porre,oafotterrare. Intendendo; Tu hai perduto cutto il credico, o la stima
che e quella senza la quale uno è tra gli huomini come morto; i che
che non si dee più far taata Mima della Sfera d' Archimede fatta di cri
ché questa facta di veiciche l'ha tuperata. 2
DATvoia, Non dalla Città di Troia, come pare, che ogi dice
Troia femmina del porco, delle cui vesciche era formata guetta sfera
V1 tira e quia. Vimori. Vedi sopra C, 4. tt. 20, Qui cocea la con
nione, che Catilina famoso capo di congiura descritto da Salustio mori
floia «
Vi fanno cento argumenti. Cioè discorrono assai sopra questa Sfera. —

xy



OTTAVO CANTARE. 399

Ml PARRVCO-A, Voce straniera fattz nostrale, e vuol dire Zazzera, dichioma
®@ finta, che diciamo: Zazzera posticcia dal Francele. Perroxque, chioma. Potreb-
» be forse dirsi in Latino capidamentum.
- CERNECCHI, Capelli pendenti alla testa; qui intende quella parrucca,o ca-
peli postice:; se ben cernecebi Gi dicono quei foli capelli, che pendono dalle tem-
* pie agli orecchi con altro nome dette faceagore, che i Latini, secondo i) Polizia-
no nel luogo sopra citato dicevano cincinnos, E noi diciamo cincinns quei ciondo-
 lidi pelo, che sogliono haver i capretti, ed i Becchi foro la gola, i quali hanno
st qualche similitudine con questi capelli, che noi chiamiamo cernecchi.
PAN bollite, Pappa fatta di pane, bollito in acqua.
 MASC-ALCIA, Magagna; Difetto; mancamento. E' lo stesso, che guida-
" Ielco, ma questo si dice solo nelle bettie, e mascalcia, che farebbe veramente so-
Todelle bestie, ' usGiamo anche per gli huomuni, e talvolta per i materiali, Vie
'un' antico libro Toscano intitolato Libro d+ ea/calcra, che @ dell' arte del ma-
nescalco, de re veterinaria,
DA quella via 01a quella via, Subito. Senza metter tempo in mezzo. Latino

=

SEE

ill extemplo, e vestigio. Se bene si potrebbe intendere ancora per In quella maniera;

ind in quella guisa, come ¢inteso sopraC. 7.0.84. 0

na oN ogni regreffo. Cede ogni azione; ogni autorita. Vedi sopra C. 7.st.104.

giù. RIFERRAR (a bocca, Intende rumettere i denti. Bocca sferrata si dice a uno,

em =the habbia meno i denti dinanzi dal ferrare le beltic, e rimeteer loro i chiodi a'

il pied, si sono sferrate.

mw STANZA XXXXIIL STANZA XXXXIV.

0 Pitna di cibi intanco nna credenza Credilo a me ch' egli è del elorioso
pal Wit pari pari aperta spalancata, Pero qua dentroyvia,distendi il braccio,
8 OB fatta da vicin la rinerenza Che troverai del buono, e del gustoso,
ye Parole pronunzio di quespa daca: Se tu volessi ben del Castagnaccio,
pi ° Caualier, se tu voi far penitenza, Paride fece un po del vergognoso,
on 'Ein parte a noi piacere,e cosa grata Hla nel veder le bombole nei ghiaccio,
440 munizion da caricar la canna, Mando presto dabanda la vergogna,
gl E poi da bere un vin ch'è una manna. E fece come i Ciechi da Bologna,

oF STANZA XXXXV.

Levategli poi'via la calamita Sicch? in quanto ad bauer taglioo ferita
ost Di quel buon vino,e maffime del bianco Jn altra parte era sicuro,¢ franco,
ipo Gli fararon le Dee tutta /a vita Poi dangli un brando con la sua cintura,
gil  WDalla baferra in fror del laro manco, E del trattarlo  intavolatura,
ye Mentre stavano guardando le suddette galanterie,comparue und credenza aper-
i ta piena di roba da mangiare, e da bere, ed inuiid Paride a soddisfarsi; egli dopo

haver fatto alquanto lo Riiakian'> mangid, e bevve; Terminato il mangiare se

46  'Ninfe lo fatarono,rendendogli impenctrabile tutta la persona, eccetto che la ha-
'fetta mancina + Qui il Poeta immuta l'Autore, che favoleggia Orlando impene-
o@  trabile in tutta la persona, eccetto che nelle piante de' piedi.
@ | CREDENZA. Così chiamiamo un' armadio, entro al quale si ripongono, ¢
tonferuano gli arnesi, ed avanzi della mensa; il quale armario si dice ancora,
4 tredenziera  perché quei bicchicri vaii, e baciji d' argento, ec. che si mettono al-
Ic

400 MALMAN TILES o
le tavole de' Grandi per servizio; 0. per apparato della
diti tutti insieme, si dicono credenza, e i si rij
vriano riporre in detto armadio, che però lo chiaa
tino eroacns re

SPALANC AT A. Affatto aperta. Vedi sopra'C. gf. 38, Pal
cato diciamoa la chindenda\, o riparo fatto con è pali, a un >
vuol dir Senza palanca, e per conleguenza totalmente aperto, o
tegno, o inipedimento. ei è

PAROLE di questa data, Parole (mili a\queste, o di
ta, la quale si attende moltissimo nel gioco delle'carte', per ef
chiate; Onde si dice: Ha farrauna buona,o una cattiva data,

SE tu vuoi far penitenza, Se tw vuoi mangiare. Termine usato per:
inuitar' uno a desinare, o cenar connoi, guafi ditiamo:venite a digi;
ché la nofira mensa e povera, e (eatlardi cibi. Sidicc.amcoralfar carisa 5,00

s'è visto sopra C. 5. st. 68, otis verb 9!
HO munirione per caricar la canna. Ho roba da mangiare\, eda
care la canna della gola., e non quella dell'archibuso <2 5, Gate
VN vin, ch' una manna, Vino isquisitissimo, che tale si legge fusse!
che mando Dio nel deferto al Popolo cletta., Vedi ferro Cip.st58..Ma
stranicra, ma fatta nofirale, che significa una brina:condenfata: tenera'y'
detta cos} dall' Ebraico, Azanbm; ioe Quid of bee: come: si dice nell?
16. Poiché maravigliati gli Ebrei di questo nuovo, e faporolo cibo 5
uno all' altro; Che e ciò, che no? mangiamo ? Dalquesta dolcezza viene '
nostro detto. 1 Latini dicevano in questo proposito dowis Velar..
EGLI¢ det glorioso. I Battilani chiamano vino glorioso il vino pga 4
roso, e buonissimo, e dicono groliefe in vece di gloriose; cioè valor lo
vaalle stelle. In certe Prose Toscane antiche, delle quali alcune si ritrovanom
nuicritte nella Libreria di S, Lorenzo date fuora dal Doni, vi e una leteera amt:
rosa, nella quale e accennato Amore con dire: Quel gloriose; titolo dato

da' nofiri Battilani al vino; e veramente Amore non imbriaca meno di

si faccia il vino il pith glorioso, a
VIA. Questo termine serve per follecitare, o incitare uno. Latino Eia ae,
CAST AGNI AC C/0.Pane fatto di farina di Castagne: qui vuol

per opera d' incanti quella credenza dava tutto quello, che uno fapeva

itis g

tern gab

rare.: a
FECE il vergenoso. Finfe dinon si ardire a mangiare. Mostrava vergognitt
d' accettar l'invito, che gli faceva quella credenaa. ig
BOMBOLE. Vali di vetro, i quali servono per mettere il vino in
ghiaccio, o neve, detti così ( secondo alcuni ) dai suono, che fanno-nel |
fuori il vino, che par che suoni bombof. Al Rotenano' vuole, che i Latini
da tal suono le diceflero amphore bilbina; ma può anch' essere, che agi ie
così da bombo voce.puerile, che vuol dir bevanda, decta così dal fueno, —
COME i Ciechi da Bolugna, Si da loro un toldo, perché cominci

c bisogna poi dargliene duc, perché si chetino. Ci iciue per espri:

i
mt

SBER ote

=
=

Ask RESS Pa SeETLE

T~s-e

>



f

rad
ike

i
wb 1

-

4

4
4
of

4

a 4
 Pigliual
| Ainqud franortese thcome,e ilquddo,e il doue,

Escciam

Ciechi da Bi

re.
~ STANZA XXXXVl'
'Eperche if tempo ormai era rrascorso,
jarlo dowean di quini altroue
Prima in sua lode fatto un bel discorso,

( dissero ) quanto t° e occorso

Py ae tutto per appunto,

Ot è gus ra nofira giunto,

oc SPAN ZA KLE X Vil.

Akcibsuvada incontro a un'aunentura

< d prod' un pover huomo questa notte;
Quelle è un tal cognominato il Tura,
CH in Parion confiaua le pilotte.

0 Eta tebellenze un mofiro di natura,

Sicebe tutse /e donne n' eran corte 5

Elasciando i roccberti, ed i cannelli

2 Per-lui chee cht e facevano a capelli,

“STANZA XXXXVIIL

Non ch? eine desse loro accafione,

— Come qualcbe narcify inesbertato,
C'una caffia, che e' vegga a un verone

?0/ «far lo spasimato;

' Anis un diqueie'al Addo ta 4pigione,

«A bioscio nel vestire, e sciammannato,
o' addosso i panni ognor tutti mincfira

0) Tirati gli parean dalla finefira,

“OTTAVO CANTARE,

magliana a Adarte,al Soleje aGiove,

gor

 ptegare a far'una tal.cosa mostrando non voler farla, e bisogna
che refti di farla, Orazio.:
» Omnibus hoc vitinum est cantoribus, inter amico
0) Wt munquam inducant animum cantare rogati,
drinffi numquam defiftant,
Sid > da Ferrara, o da Milano, 1 Latini in questo proposito
'ditlero Arabicus Tibicen. Qui intende, che Paride si fece pregare a mangiare,'¢
se e poi non si trovava 11 modo, che egli restafie.

LAMIT A,B! \a pictra Adagnes, la quale ha proprieta d' attrarre il ferro,
punto ha il vino di tirare a se Paride, ed è fra etio, ed' il yino la stessa,
»cheè fra jacalamita, e il ferro. Vedi sopra C, 5. st. 59. B sotto ing
— questo C, tt. 66,
| | Di trattario ? intavolatura', L? instruzione di come si debba adoprar quella spa-
t+ Intavolatura e scritwura, che per yia di note, e di numeci regola la mano del

STANZA IL,
Ed esseeran capone; ma chiarite;
el fin lasciando quel fuocnor di fmalto,
Fecer come la Volpe a quella vite
C* hauea si bell' Una ye tanto ad alto,
Che dopo mille prone, anzé infinite
Arrinar non potendoni col falto,
Gli è mé,dilfe,chio cerchi altra pastura,
Che quespa da ogni mo non è matura,
STANZA L,
Così non la faldo era Martinarza,
La qual non vitrouado anch'elja attacco,
Poicht gran tempo andata ne fu parra
Hanedo if rerzo,e il quarto,e ognuno firac
Codurreun giorno fecelo alla mazza, sco
E per via d' un che le teneua il facco
Aunezro a tofar pecore, ed agnelli,
Mentr' ci dormina gli taglios capelli,
NZA LI.

Quei capelii c un rempo hanea chiamati

Del suo falcto mortal funi, e risorte,
Le bioae chiome wh Dio,queicriniaurati
Che ricoprivan rante piazze morte,
Onde scoperti furo s trincerati

Onc il nimico si facea si forte;

Perché (per quanto un Autore accenna)
Lo rimondavon fino alla cotenna,

ale fate dopo haver lodato Paride per bravo, per bello, e per valoro(o gli dif-
fero sche IL havevan fatto capitar quivi, perché egli andatie a liberar il Tora,
quale loda ironicamente, e dice, che tutte le donne erano innamorate di iui; ma

Ece

accor-



qo2 MALMANTILE (9

accortefi, chenon corri leva a niffuna, lo lasciarono, ¢:

egli non volle mai corrisponderle, haveva fattagli la malia
ottave seguenti.

1 DE
AVVENTVRA. 1 Romanzatori Spagnuoli in quei loro Amadis di Gaul
Palmerini d' Oliva chiamavano ayventure ( awenturas) quegli i
ne i quali s'imbattevano i Cavalieri Ecranti, e però il nostro:
creato il Cavalier di Quoio,vuol, che ancor' egli sia fimato i

che vada a provare l'avventura di liberare il Tura dall' incantefimo. 1
similmente dissero aduentures, Bi nostri Toscani ancora, sentendosi:
termine cavalleresco, chiamarono gli accidenti, che accadevano

davan loro materia di fare prodezze dawenture. L' Alamanni nel
principio.

3 teal

Narrero di Girone l'alte auwenture. a ad

E da ciò il Boce, Tels. lib. 5. disse: 5 aabiitly
eettersi in aunentura:. ee

Ma non li parne via ben ben sicura. poe

Pero non se ne mife in auuentira, sung

4L Tura, Costui era un pover' -huomo, che gonfiava le pillotte in] }
che in Firenze è la strada, dove si giuoca alla pillotea detta così da
perché in essa anticamente haveano le botteghe coloro, che lavoravano:
mi, o pure ( il che forse e più verisimile ) quasi Ripar Regio Ripe Roine
tale strada sbocca sul Passeggio di Lung' Arno; [n Roma ancora vi e la
da di Parione detta similmente così detta quali Rione a Ripa, Regio Riper
pure è così chiamata, quasi Parte di Rione; Pars regionis, come mi vien riferito
leggersi in alcune Carte, o Contratti. E perché veramente costui era brut
di faccia, ed haveva la zazzera avviluppata', elorda, lo chiama moffro
va in bellegza, ed intende Deforme, se ben par, che voglia dire, di belleam
sopranaturali. aise
PILLOTT A, Specie di palla da giuocare, Vedi sopra C. 6. st. 34.
WV' ERAN corte. Erano abbruciate dal fuoco d' amore per lui Virg, Mir
infelix Dido: dice briache del suo amore, cs' intende innamoratissime di Iai. Lt
ebrie amore. Piauto nel eAilit glorifo,o Soldato,al quale da nome di
cioè di Abbartitore di Torri,e di Città; 0, came noi diremmo Tagliacantoiyo
Spacca Montagne; fa dirgli da e4rtorrage, cioè in nostra lingue Sparapane Pata
to, suo adulatore; che tutte le donne sono di lui fieramente innamorate » Le
tibi ego dicam; quod omnes mortales sciunt; Pyrgapolinicem te unum in tere rn
Virtute, \& forma, \& fattis muittiffimus? Amant te omnes mulieres,neque herce i
ria, Qui fis tam puicher. Ed egli sprezzatore altero di tali amori iange !
Jamente la sua disgrazia, beccandosi fu queste lodi; dell esser hwoWd, |
da fare innamorare di lui tutto il Mondo, WVimis est miseria
pimis. '

LASCIANDO i roccherti, ed i'cannelli. Lasciando star di lavorare.
prefe tanto forte l'amore; e tanto le teneva fisse nell' amoroso penk
non potevano più atcendere a' loro usati lavori. Quando Didone si
'rata d' Enca., non tirava innanzi gli edifizj.,.¢ le fabbriche della sua

preaereseei n



OTTAVO CANTARE:

Virgilio ebbe a dire: pexdemt opera interrupta, minaque Adiroxam ingentes) come
che era occupata da più possente pensiero. Co! presente detto di lasciare
droccherti, 4 canneilé, s' intende questo, perché le donne a' infima plebs (che ta-

1 epeweneenye > che erano l'innamorate di costui ) per lo più non hanno
lavoro

» che ? incannace, e teffere, a' quali lavori s adoprano i Rocchetti (che

son legnetti tondi forati per lungo, e servono per ragunarvi sopra la feta, ed

',

oo LU BRE

AaeTEEs

TS A A Se ee, ae

altro filo: ed i Cauneii, che sono pezzuoli di canna tagliata fra un nodo, e
Paltro, dai Latini però detti ixternodia, e servono per lo medesimo effetto d' a-
dunarvi sopra la feta, ec. per adattaria a telsere; I che si dice incannare.
hone èch'è, Ad oraad ora; Di momento in momento. Vedi sopra Can. 3.

 PACEVANO a i cape, Si perquotevano, S'azzuffavano. Quando due donne
combattono tra di loro diciamo fare «4 capelli; perché il lor perquotersi, e per lo
 più) pigliarsi 1 una l'altra per i capelli.
 CVFFIA. Berretta a foggia di facchetto, entro alla quale le donne \& ferrano
icapelli in testa, e quando noi diciamo nel modo, che e detto ne! presente luogo
tna ¢xfia, un ciapperone, e simili arnesi usati dalle donne, intendiamo una Don-
na, Così dal portare lancia, o barbuta; i toldati medesimi si chiamavano Lance,
¢ Barbute, come ficava da Matteo Villani, 11. 81. e Erodoto volendo dire, che
pera si ritrovavano avere in. piedi ottomila soldati, che portavano rotel.
90 brocchiere; disse ottaci/chilian aspida, cioè scudi militari,o rotelle ottomila.
VEKONE. Latino moenianum, podium, pergula, e in Greco secondo alcuni Pe-
ribolas da peribaliern abbracciare, circondare, che i Francesi dicono enxironner.
Propriamente vuol dire andito, o terrazzo scoperto': Qui credo, che habbia a dir
Baleone,¢ non Verone. Verone \& detto quasi girone, cioè giro, dall' andarvi sopra
¢rigirare, eAndito, che e lo stesso par facto da e4ndare. Latino ambulatio,
EGLiea Pigione al mondo. Così diciamo d' un' huomo spensierato, sciatto,sen-
2a considerazione, e che vive a caso, che si dice anche Auomo a bioscio, sciaman:
nato ( cioè male ammannato,male all' ordine ) e che i panu gli paiono tirati addossa
finefira,B cd questi quattro modi di dire |'Autore delcrive l'attilatezza del Tu-
ra;del.refto, parlando secondo moralita, ognuno dovrebbe stare in questo mon-
do, come a pigione; perché la nostra propria casa è nel Cielo. E nel Salmo 118,
4ncola ego fum in terra, i) Greco dice Parcecos, e alcuni Salteri dicevano, come ri-
ferisce S. Agostino sopra i Salmi, inquilinus, cioè pigionale. F
CAPONE, Ostinato Latino. Pertinax. Pertiwax.
FAR come (a voipe alla Vite. La Volpe dopo haver molto faltato, e dopo essersi
i per arrivare un grappolo d' vua, e non l'havendo potuto arci-
vare disse: La voglio lasciare stare, perché ad ogni modo ella non e matura..
Può aver data occasione a questa novelletca quella d' Efopo, della Volpe, € del
Pruno; in cui la Volpe, che voleva falire una fiepe, mi fuppongo, per mangiar
Tuva, della quale è ghiottidiima, pensando di troyare il Pruno buon' amico, «.-
sto ingannata del suo pensiero; poiché attaccandovili refto intaccata, e ! appoggio
le fu ferita, e volendola poi disputar con lui, ebbe il-torto: E questo detto
ei serve per esprimere uno, che habbia usata ogni poslibil diligenza per conseguite
una tal cosa,¢ non ' havendo potuta ottenere, 9 habbia abbandonata i' im-
Ece 2 presa

ae
F
7



404 MALMANTILE o

presa come imposiibile, o sia quella tal cosa fata data a un' al
vanti di non l'haver voluta, perché:non era buona, o non!
diciamo: farsi honore d una cosa n

COSÌ' non fa falde eartinazza, Cos\ non fini, o termind ?
nazza la quale non troxando attaeco, cioè non trovando luogo di f
suo amore ver(o il Tura, del quale andò paz ca, cioè sterte innamoratiti

CONDFRRE uno alla mazza, Tradir' uno: Condurre uno con inga
finghe in mano de' suoi nimici, o della giuftizia, o in qualche altro
come si suol dire; al macedo, Latino /n infidias ducere. re

TENER il facco, Tener di mano. Aiutare a cometter un delitto. Ha
un proverbio sentenzioso, che dice: Tanto ne va a chi ruba, quanto a
Jucco, che esprime Agentes, \& confentientes pari pena puniuntur. B diciamo anche
Tenersi il facco l'un t altro; che esprime il detto di Teren. Tradere operas mutna

FEY NI, e ritorte del suo fascio mortale. Metafora amorosa + Si ne
ritorte tengono unite più legne in un falcio, o faftello, così i capelli del Tura,
quasi funi, € ritorte tengono unita col corpo |'anima, cioè tengono in
Amanti del medesimo Tura, E riorre dicemmo, che cosa sieno sopra C.

PLAZZE morte, Si dicono i luoght vacanti de i soldati; per esempio
no € pagato per cento soldati, e non ne ha f€ non novanta; quei dieci
cento, che mancano si dicono piazze morte. Ma qui intende quelle pi
lasciano le margini, o cicatrici de i mali, che vengono nel capo; sopr”
li non natcono capelli. “|

1 TRINCIERAT 1, V luoghi, dove erano le trinciere. Intende,
gliargli i capelli si sono scoperti quei Inoghi, i quali con- elle: margini
una campagna piena di trincicre. done tl nimico si facena forte s clot di 0
devano i pidocchi. we tage
TRINCIERA, oT rincea, EB' un' alzamento di terreno: condotto a foggia d
baltione, nel ricinto del quale dimorano i soldati per difenderti dallartiglierigg®
de i nimici. Franzese rrenchee, cioè tagtiata, yene
LO rimondaron fino alla cotenna, Gili tagliarono i capelli fino rafente la pelle.
Rimondare vuol dir Tagliare a un' albero i rami: B curenna's invende solo lap
le del porco, ma quando si tratta del capo s' intende anche quella dell huom
Vedi sopra C. 5. st, 52. * 4a?
STANZA LIL STANZA LALO
E cos) Aartinaz2a hebbe il suo fine E questo Lupo raggirar si vede 4 2
Volendo vendicarsi per tal via, Intorno a un montnoso cafameme—
Pero sche buona parte di quel crine,
Ch' alcun non few avvedde, leppo vidy
E fabbriconne al Tura le rove
Con una potentissima malia,
Che revifrata in Dite al pratocollo
fa un Lupo rapace trasfor mola,

4

NS ne ears ESR ene

D' una Genteycheymentre mice ilpitl

r

yor b

Zexv7ek


OTTAVO CANTARE 403,
o» STANZAEIV. 5 STANZA LV;
d vanne,e perché tu non facia Eli la prende con il libro insieme,
— Qualche marronyma vegas arar drittoy Dicendo, che varraffi dell! avviso,
- Acco tal: ero si disfacci

f Pi disfaccia, E.ched' incano ye dianoli non teme,
- Percht feattado un pel tu baurefti. ifritto, Perché eglirehuom, che fa mostrar ilvifo,

Yi he questo libro qui faccia per faccia Si parte,e per c'al capo andar glipreme,
hl ordine, ei modo si ritrova [critto, da due parti vorrebbe efer divifo;
ipa Portalo teco, e.accio che rm discerna, Pur vuol servirle, perché si figura
Perch! egli e buio to questa laterna, Che non ci vada gran manifattura,
i Metono: STANZA LVL
wit poi nel suo cervello, Ricerca nel suo maftro scartabello
ia Che sa quel luego a bambera s'inuia Di quei pacfi la Geografia,;
| Potrebbe andar a Roma per eMugello, Aa quel(per quato noi potré coprendere)
oPerchvei non si rinnien dow' ei si sia, Non si vorria da lui lasciar' intendere,

Hi
id ~~ Martinazza:hebbe il suo intento, perché presa buona parte de i capelli del Tu-
sf 4 con essi gli fece una malia, che lo trasformd in lupo, e lo confind in un mon.
ati tevicino:a Maimaatile. Finito questo racconto le Fate licenziaron Paride,¢ gli
r diedero un libro, dove era scritto il modo da tenersi per disfar quell' incanto, ed
we una lanterna per farsi lume; e Paride si parti con risoluzione di sbrigar questa
we faccenda prima d'andare al Campo.
 LEPPO' via, Portd via di nalcofto. Il verbo /eppare ci serve per esprimere
svelocita 'hell' andar via o nel levar via qualcosa.
4 MALIA. IncanteGimo, fattucchieria, stregoneria. i
1” o PROTOCOLLO. Libro pubblico tenuto da i Notai per scrivervi sopra i contratti,
e testamenti, e così e inteso da noi; se ben protocode vuol dire libro da registrarvi
sopra, che che sia. 11 Berni fonetto in biafimo d' una mula dice:
4 > — E troppo sia diginna
ie ts srry? Cht il prorecollo memoria non fanne.
Perché veramente Prorocoo.è un libretto,sopra il quale 4 segnano, e registrano
r brevemente le cose, per diflefderne poi scrittura più largamente, ed autentica.
i  Msnce: deteo così quali, primo libra incallaro, e legato. Liber ex glutine compattus,
o in guematta referuntur, Ma il nostro Poeta jo piglia nel senso, che oggi usiamo
di libroida Notai, e intende che Martinazza haveva fatto contratto cal Dia-
mM volo di questa malia; il qual contratto era già mefio al libro del Notaio del Dia-
it volo yeper questo detta malia era autenticata, e non si poteva alterare, perché
' era passata per mano di Notaio, e regiftraca al suo protocollo.
  - CASAMENTO montnoso, Intende il Castello di Montelupo, che ogg? quali
 — distrutco siperd più colto Cafotare, che Ca/iello, e lo dice montuoso, perché \&
sopra un monte come lo mostra il nome medesimo, E nota, che ancor qui il no-
® fico Poeta vaimitandoi Romazatori Spagnuoli, che fanno parlare oscuramente,
| e come gli Oracoli quei loro:Alchifi, Zirtee, Wrgatide, cc, incantatori.
« MENT-RE move il più sopra alia terra v' e rinuolea drento: Le reliquie di questo
/ — Castelio sono abirate da persone, che fabbricano valellami di terra, come pen-
tole, boceali, ec, quali si fabbricano per via d' una ruota, la quale va moffa cot

piedi, e fa efsctto del tornio, e perché in muover desta ruota, € fabbricare il
it valo,

aiie



oh MALMANTILE
vaso, la terra schizza addosso a chi lavora però dice Ademtre shane il più sopra alla

terra v' è rinvolta drento,
FAR' un marrone; Far' un error grandissimo; wx crrorome,
e4¢KAK aritto, Operar giuftamente, Non fare errori. Tolto dal Bifo
ciamo ancora, rigar diritto. 4a) aA
SCATT ANDÒ un pelo. Se tu uscifi punto dell instruzione, che tu'
tare, o scoccare, si dice della freccia quando seappa dalla cocca, t
di qui € tolta la metafora, o forse dall'orivolo'a ruote. See
TV hauerefti fritto. \\ Proverbio dice: Come difela Tinca ai Ti
altra aggiunta s' intende: oi habbiam feito, Qui intende tu hanrelti
tu haurelti rovinato questo negozio, EB' lo steilo che: Noi habbiam
ne detto sopra C. 7. st. 60. ow
HVOM che fa mostrar il vifa. Huomo ardito, e che non fee cen
ABAMBERA, Acaflo. Latino Jrconfultd. Vien forse da e è
vuol dir ragazzuolo,(enza giudizio, B il ragazzo in alcunt luoghi chiamato Bar
berottolo. Dicesianche. 4 fanfera. 2 6a
ANDAR 4 Roma per Adugello, Far' una strada al tutto contraria, come \&
rebbe andar da Firenze a Roma, e pigliar la flrada per il Mugello, che € dirt
tamente contraria. oa
NON si rinuiene, Cioè non riconosce in che parte ci si sia, e non fa quel chid
si debba fare. aed
MAST RO scartabello. Tntende quel libro, che gli haveano dato naa

è il suo maestro, e direttore; Questa voce scartabello, e corrotta da C.
anticamente era intesa per un libro di stima; come mostra il Dortissimo 5:
ditissimo sig.\ Francesco Redi nelle annotazioni al suo bellissimo Ditiramboac
18. Gli Spagnuoli chiamano Careape/ una scrittura continuata nel foglio fen
voltarlo, comes' usa negli editti; dal' essere cred' io, non ripiegata, come ifr
gli, ma stefa, come una pelle; o perché Gi distendeflero tali forte di seriteure a0.
3

in carte ordinarie, ma in pelli, ovvero in cartapecore.

” we an ee ae Le Lae eeasS we

STANZALVIL STANZA LVIIL ~~
Fu Paride persona letterata, Ma benche la lettura sia fantafticay
Che già udiato havea più d'un faleero, A un che, si pu dir non fa niente 5
AAa pei, non ne volendo più fonata, E 0? altro di virtit non ha se
Alla squola fiudi di Prete Pero, Che pelle pelle 1 Alfabeto a mentt,
Pera s' ei non ne intende boccicara Tanto la biascia, strologaye rimafice
EB? da scufarlo; e poi, per dire il vero, C' 4 compito leggendo finalmente —
Lettere, ed armi van di rado unite 41 funto apprende,e fra Lalere sue ciate
Per ¢' han di precedenza eterna ite. Ripone il libroye sprona i le foarpe.
. STANZA LIX, otra
Cos} commina, e a quel Castello arriva, Aa perche' non e tempo cb'
Passa dentro, lo gira,¢ si fiupysce, Quanto col Tura a Paride
Che quixi non si vede anima vina Con buona gratia vofra Fare pasft »

Perc'aquell ora in casa ognun poltrisce, Per difinir di Piaccanseo la catfte



» Jee cold

OTTAVO CANTARE; 407
STANZA LX.

Che da.quei tristi, com! io dissi ananti Di poi gli stessi fel cacciaro innanzi
| ( Fatto mentre pappana affegnamento Ginfto come un Villano in fu il giumeto,
 Diinfaccarsi per lor -quei boccon fanti ) Pungolandolo, come un' animale
 Tocco de è pie nell' Crsenal del vento; Fin, che lo spinfer dove e il Generale,

. Deferive le qualita di Paride, e dice, che egli era letterato, perché havea let-

@ to pidd' un faltero,, che e quel libricciuolo, contenente alcuni Salmi; che si da a

Aeggere a' ragazzi quand' hanno imparato a conoscer le lettere dell' Abbicc:; E
Tamuiodke, samedi che vm fapeva troppo leggere; e dice, che non ¢
da far meraviglia di questo, perché l'armi, e le lettere mai furon d' accordo,¢
»però egli, che era armigero, era scufabile, se non era letterato; con tutto ciò
'compitando leffe in quel libro, ed intese quel ch' ei doveva fare; ed arrivato al
-Cafamento montuoso trovd che ognuno dormiva. E quil'Autore lascia il parlac
i lui, e torna a parlar di Piaccianteo, che lasciò sopra nel fine del Canto 5. e>
dice, che a-furia di calci e pungolate fu da coloro condotto doy' era il Generale,
o NON ne volendo sapere pin suonata. Non volendo pii: sentirne discorrere di tare
tuna tal cosa, e qui intende non volendo più studiare.
LA squola ds Pretepero, \nsegnava dimenticare.
o NON intende boccicata, Non ne intende punto. Non conosce a pena le lette-
“res perché boccicata stimo che venga da abbiccs, ae dica non fa lt Abbicci, che
pt oma, che con i Greci ancor noi diciamo diphabero, el' usa il nostro Poeta,
sente Oxtava 58. Procopio nella Storia segreta narrando l'ignoranza di
- Giuttino imperadore, che poi si adottd Giuftiniano; dice che egli era Analfubeto,
»tioé sche non fapeva l'abbicci; ac scrivere il suo nome.
» PELLE pete. Superficialmente. \&' lo stesso che baccia buccia detto sopra C.

se stans27, b 4
o BIASCIARE. Mafticare senza denti; cioè con la lingua,¢ col palato. Qui
'lntende quello studiare, che fanno i fanciulli, quando imparano a leggere, che

“prima di rilevare, o profferic la parola, che leggono, la compitano [utte voces,
'con la bocca il medesimo gesto, che fa uno, che biascia; e lo steilo vuoi

dire quel rimaftica, ec, e stroluga intend: cerca d indovinare quel che dica queila

scrittura.

. + Leggere a compito, e quello accoppiar le lettere, e sillabe, che

fannoi fanciuili, quando commciano a imparare a leggere, il che si dice compi-

tare. cio contare a una a una le lettere, per poi sommarle, per così dire, in

“ana parola; il che si dice, rileware,

~ CLARPE, Bazzecole, Vedi sopra C. 3. stan. 5.

SPRONAR te scarpe. Detto afato per burlar' uno, che viaggi a picdi

ANIMA vina. Ancor sopra C. 6. stan. 19, si serve di questo detvo assai usate
“anol, se ben si fa che l'anima sempre vive;e qui vuol dire, che tutti dormivano.

POLT RIRE, Dormire. Vien da Poltro, che vuol dir letto; circa che vedi
sotto C, 9. itan. 39.

FA-CIAM panfa, Riposiamoci: o fermiamoci. Frafe Latina venuta dal
Greco, usata anco da noi, i quali da Paufa abbiamo fatto Poa, eda Pan/are
'usato pure da' Latini de' tempi bali, 'y/are~

; ae.



408 MALMAONTILE TO
ARSENAL del vento. Ripostiglio deViyento § cioè il
dire una stanza, entro-alla quale \&i fabbricanoi Dante laf. <
unle neil Arzana de' Keneviani, » 6 6) e
Mu hoggi si dice:ax/enale, e credo che sia parola corrott. 3
@rx manalis, ia quale origine viene approvata-dal Ferrari.)
PYNGOLARE; Stimolare. Pangolo ¢-quel-bastone con tuna' punta a
@ acciaio in'cima j\del quale si erupnoi congadihi per re
camminino; Lat. fimulus 7B questo si dicepangalare',: caiy iss%
STANZA LXA vogcl now pS STPPARNg a
etppunto rl Generale a faris' eposto. * 'Cofteroal fine fegli

a

Alle minchiace,¢d e cof ridicola. la) Rersdingli del prigion
Jl vederlo ingrugnato, e mal disposo. Ma e¢ possom predicar
Perchigli e fata morta una verzicila, o. Perch'eglich'e' mele
Le carte ha dato mal, non ha risposto, o. Oo “Eiperde una gram

E poi-dt non-contare anco pericola E gliene duole} e now
Sendoscopertobaner di più nna carta yi.. Lornonddrerta,e:

Perch dtrado, quando rubay foatee, 00° Pletofamente fa qucfte si
-Costoro, che conducevano Piaccianted parrivarono al Ge eye
va giuocando alle Minchiate, ma percheegli-+haveva fatto-unaln a
perdeva, e però rain colicra, in-vece d afeoitare quel che essi dice
se a dolersi della Fortuna -; come sentiremo appresso, D i nonedyy

MINCHIATE,£ un giuoco assai noto detto anche 7%
Germini » Ma perché è poco usato fuori della nostra Toscana,o d
mente da quel che uGiamoi noi, per intelligenza delle presente Ottava è
necessario sapersi, che il giuco delle Minchiate si fa nella maniera che
E' composto questo giuoco di novantasette carte, delle quali 56. d
racce, € 40. si dicond'Tarocchi', ed una, che si dice ilmatto: 1é carte
in quactro specie, che si dicono femi, che in quattordici sono efigiati De
da Galeonto Marzio diconGi essere pani antichi contadineschi ) 10 1.4: Coppe
14. Spade, ed in 14. Bastoni, ed in ciasCuna carta di que(ti fea cominciad
(che si dice afo ) fino a dieci','¢:nell' undecima e figuraro un Bante,
Cavallo, nella 13. una Regina, € nella 14. un Re, e pucte queste carte
fuor che i Re si dicon cartacce, Le 40. si dicono Germmis o Tarocabiy o
voce Tarocchi vuole il Monosino che venga dal Greco: Etaroohi; quah
egli con ' Alciato, demotunrur fodales s1li'y gus cibi causarad dufum-com
quella voce non fo »che sia; fo bene, che Afereroi, e Hetaroiwuol dire
da questa voce diminuita all' usanza: latina si può-etiere fatto 4
Compagnont, Germini forse da Gemini segno celette, che ¢-fra Ta
è il maggiore. In queste carte di Tarocchi (onw efigiati diverti
Segni celetti, e ciascuna ha il suo numero da uno fino a 35,,¢0"
no a 40. non hanno numero, ma. si distingue dalla figuea am
maggioranza, che è in questo ordine Strela, Luna, Sole, Afonds eT)
éla idre, e farebbe il numero 4o. L' allegoria ¢,iche siccome le)
vinte di juce dalla Luna,¢ la Luna dal Sole, così il Mondo e maggi
ela Fama figurata colle Trombe, vale più che ii Moado;

ale
a

REF FSSRSETFESTLRR SLE EESS EE

SRE

“PRSPEREVF eRe SBBEZE


OTTAVO CANTARE; 409

'I huomo n' è ulcito, vive in esso per fama, quando ha fatte azioni glo-
. li Petrarca similmente ne Trionfi te come un giuoco, perché Amore € fu-
daila Castita, la Castita dalla Morte, la Morte dalla aw > e la Fama
| Diviniva, la quale eternamente regna. Non e numerata ne anche la carta
ma vi è impressa la figura d' un AZarro, e questa si confa con ogni carta,e»
ogni auacco, ed è superata, da ogni carta, ma non muor mal, cioè non
i umat nel monte dell' avversario, il quale riceve ia cambio dei detto Maco
altra cartaccia da quello, che detce il aa ¢, ( alla fine del giusco questv,
'dette il Matto, non ha mai preso carte all' avversario, conuiene che gli dia
(0, On havendo altra carta da dare im sua vece, e questo è il caso, nel
si perde ii matto; Di tali Tarocchi altri si chiamano mobili perché conjano
chi gli ha tn mano vince quei punti, che etli vagliono ) altri ignobili, per=
snon'contand. Nobili ono Ve, due, tre, quattro,e cingue, che la Carta

%
del Fao conta cingue,¢ l'altre quattro contano tre per ciascuaa. Ui numero 10.
:

43.20. € 28. tino ai 35. inclufive contano cinque per ciascuaa, e l'ultime cingue
| Guutanio dieci per ciascuna,e si chiamano sd, [i Matto conta cinque, ed ogni
Re conta cinqute, e ono aacor' eifi fra ie carte nobili, s1 numero 29 non conta
se non quando è in verzicola, che allora conta cingue, ed una voita meno delle»
'compagne felpetcivamente; Delic dette carte nobyli si formano le Verzicole, che
Ordini, e (egucnze almeno di fre carte uguuli, come tre Re, o quattro Re;
: di tre carte andanti, come xe, due, o tre, quattro, e cingue, o compote, co-
i bet 'hie wo, 13.¢28. Vo, matto, e quaranta, che sono le Trombe, Dieci 20. € 30.
tq OVEFO 20 30.¢ go. E queste verzicole vanno mostrate prima, che si cominci il
giuoco,¢ meife ia tavola, il che si dice acenfare la Verzicola, Con tutte le verzi-
i 'evi si confa il matto, e conta doppiamente, o triplicatamente come fanno l'al-
a “tre, che sono in verzicola,la quale efilte senza matto, € non fa mai yerzicola se
i 'non nell' une, matto, e trombe, Di queste carte di verzicola si conta il numero
“che vagliono, tre volte, quando pero l'avversario non ve la guafti ammazando-
a “Vene una Carta, o pir, con carte superiori, che in questo caso quelie, che resta-
f 'DO; coMtano due voite, se però non restano in (eguenza di tre, per esempio: Io
a. “'moltrO 4 principio del giuoco 32. 33. 34.¢ 35. se mi mi muore il 33.0 il 34. cho
os rompone la seguenza di cre,la verzicola e guaflata, e quelle, che vi restano con-
tano foiametice due volte per una, ma se mi muore il 32. o il 35. vi resta la se.
Suchza di tre,€ per confeguenza e verzicola, e contano il lor valore tre volte»
ciafeheduna. Mf Atato, come s'è detto, non fa seguenza, ma couta sempre
è 1i suo valore due voice, o tre secondo, che conta la verzicola 9 guasta, o falua-
a ta} o quando s' ha più d' una verzicola, con tutte va i Adarto, ma una fol volta
i) conta tre ed il refto conta due; € questo s' intende delle verzicole accufate,/e
mottrace, prima, che si conunci 1! giuoco, perché quelle fatte gon le carte am-
" mazzate agh avversarj, come farebbe; se havendo io il 32. ed il 33. ammazzaii
ai' avversario ti) 31, o ii 34. ho fatta la yerzicola, e queita conta duc volte,
Quando e ammazzata alcuna delle carte nobili, ciascuno avversario fegna a co-
'lav, a cui € thata morca canti segni, o punt:, quanti ae yaleva quella tal carca;
Ccvetto però di quelie, che sono state mostracte in verzicola, delle quali, sendo
Auiazeate non si gaa cosa alcuna (. a da quello, che per privil:gio non
3 f giuo-



4r0 MALMAN TILE

» giuoca ) perché tali segni vengono dagli avyerfarj guadagnati
del valore di essa verzicola, che dovria contar tre volte,
ed il 29, morendo la verzicola, dove esso eatrava, conta solo cing
carte poi, le quali si dicono carte ignobili, e cartacce non contand
mazzano tal volta le aabili, che coa;ano comei tarocchi dal aunero 6,
amwmazzaa tutti i piccin, cioè l'1.2. 3.4. e 5. dal 14. in fu am nazzano
il tredici, e dal 21. in fa ammazzano anche il 20, ed ogni tarocea
Re ) ma servono per rigirare i giuoco; il qual giuoco appreilo di noi non
non in quattro persone al più, ed allora si danao 21, Carta per ci
do si giuoca in due, o ia tre, se ne danno 25. E giocandosi in quai
primo che seguita dopo quello, che ha mescolace le carte in fala mano
dice bauer (4 mano] ha la faculta di non giuocare, e paga fegni trenta
che nel giuoco pigita ' uitima carta, e questo che piglia l'ultima |
dice far f' ultima ) guadagna a ciascuno ai gueili, che hanno giuocato
Colui, che non giuoca guadagna ancor' egli de i morti, cioè segna,
lore della carta a colui,al quale e amimazzata deta carta. Se
giuoca, il secondo ha la faculta di non giuocare pagando go. segni,se
ca il 3. ha derta faculea pagando go. fegal, se il 3. giuoca paifa la faci
che paga 60, segni come sopra. Ma se il giuoco e solamente tn tre
ci questa faculta di non giodare.

Me/colate che sono le carte, quello de i giocatori, che è a mano
quello, che ha mescolato,n' alza una parte,e se v'è volta nel fondo di quel
del mazzo, che gli resta in mano una delle carte aobili, o un tarocco dal
27. inclufive, 12 piglia, e seguita a pigliarle fino a che aoa vi trova und
ignobile: Quello, che ha mescolate le carte dopo haverne date a ciale

PSPS PPT

se stesso dodici la prima girata, e tredici la seconda, e scoperta a %
carta la feuopre anche a se medesimo, e poi guarda quella, che segue,¢! a
se fara carta nobile, o tarocco dal 21,al 27. e seguica a pigliarne come +

gu-sto si dice rubare, e queste carte, che si rubaao,¢ si scuoprono, sendo a
guadagnano a colui,a chi si sCoprono,o che le ruba, tanti segni,
gliono; e coloro, che le rubano € neceffario, che scartino; ciae si levino:
altrettante carce a loro elezione, quaate ne hanao rubate per ridurre le
al numero adeguato a quello de i compagni; e chi non scarta, o per
dente di carte mal contate, si trova da uitimo con più carte, o con.
avversarj per pena del suo errore non conta i puati, che vagliono le fu
ma se ne va a monte; Colui, che da le carte, se ne da più, o meno d
stabilito, paga 20. puati a ciascuno degli avversarj, e chi se ne trova
più, e deve scartare quelle, che ha di più; ma non può far vacanza
deve rimanere di quel feme,, che egli scarta; Se ne ha meno, la deve
monte a sua elezione, ma senza vederla per di dentro, cioè chieder la qu
o la festa, ec. di quelle, che sono nel monte, € quello, che mescoid le
si dice far /e carte ) fattele alzare gli da quella, che ha chiesto,
Cominciafi il giuoco dal mostrar le verzicole, che uno ha in mano
mo dopo quello, che ha mescolate le carte in fu la mano destra,
quna carta, ( il che si dice dare ) quegli altri, che seguono devon dare

4

OTTAVO CANTARE. 4it

mo feme', se ne hanno; € non ne havendo devono dar tarocco, e quello si dice
nes » E dando del medesimo feme si dice ri/pondere. Chi non risponde,
ed | reece, feme, che è stato messo in tavola, paga un sessanta punti
ciascuno, e lia carta nobile, che havesse ammazzato; per esempio il

Toes ai di danari, ed il econdo benché habbia denari in mano, da.un

acco sopra il Re, e l ammazza; scoperto di haver in mano denari, rende
acolui dichiera, e paga agli avversarj fessanta punti per ciacuno, come
MY gE deo. Ogni tarccco piglia tutti i emi, e fra lor taroccht il maggior numero

-piglia il minore, ed i matto non piglia mai', e non è preso, se non nel Calo dec
| todi sopra. Così si seguita dando le carte yeu il primo a dare e quello che piglia
; Tecarte date; ed ognuno si fludia di pigliare all' avversario le carte, che conta-
—-HO€ quando s'¢ finito di dare tutte le carte, che s' hanno in mano ciascuno con-
b “tale carte, che ha prefe', ed havendone di più delle sue 25. segna a chi l'ha me-
ge MO taati punti, quante sono le carte che ha di più, dipoi conta i suoi onort, cod
l'il valore delle carte nobili, e verzicole, che si trova in esse sue carte, e fegna
all' avversario tanti punti, quanti con li suoi onori conta più di efl> 5 ed ogni
F et si mecte da bai.da un fegno, il quale si chiama un fefanca, € que'ti
Sfane valutavano secondo il concordato. E tanto mi pare che basti per facili-
tare l'intelligenza delle presenti ottave a chi nonfulle pratico del giwoco delle
Minchiate, che usiamo noi Toscani, che è assai differente da quello, che con le
'medesime carte usano quelli dala Liguria, che lo dicono Gsmellini; perch Adin~
¢hiate in quei paefi è parola oscena. Da questo'giuoco vengono molte manicre
dire; come essere il matro fra tarecchi; entrare in tutte le verzicole; Essere le»
Trombe, carracce; Contare; non contare; e simili.
\ INGRVGNATO. In collera. Chi s' adira, o entra in collera suol mostrarlo
con la Mutazione di volto, torcendo la bocca, o increspando la fronte, com,
atti simili, che si dice anche far mufo, e far grugno, o ingrugnare. Vedi sopra C,
2, Man. 57, Lasca Nov. 10 Ata Beco non la porendo [gozzare se ne fRaua ingrugnato
 anki che no, Viceli anche portare tener broncto; imbroncrare,. Nonio Marcelio an~
tico Gramatico. Bronci /unt producto ore, \& dentibus prominentibus.
DS AMMAZZ AT A una Verzicola, Ammarzare,rubare,foartare, dar mal le
artenon contare, verzicola, non rispondere, sessanti, ec. lepgi queiche habbiamo
~detto qui sopra alla voce Afinchiare. read
» HVOMO roto. Huomo collerico. Lat. praceps in ira, che si dice ancora in

yb + mn ena huomo precipitolo.

E

BE

RUEBRERLER SSeS.

,¢ o NON ci puo fear sotto, Non la può foffrire. Lat. /ustinere, pati.
gi | LOR nem da retra', Non bada, o non attende a que! che est dicono. Non da
. Lat. mon faciem accomodat aurem. Dar vetta in altro senso dissero gli
“f antichi nelle cose di guerra, per quello che i Latini dissero, imperum /ustinere,
yy » GAGNOLARE. Rammaricarsi. Vedi sopra C. 4. stan. 9.
n.% STANZA LXIIL
| Che t* ho io fatto mai fortuna ria Lucho non si farebbe anch' in Turchia,
/ Chee? bas con me si grand' inmicizia; Lie proprio un'impierade un'inginftizia;
, Mentre tu mi fai perder tuttauia Vedi, non lo negar che tu? kai meco;
5 Che @ nem mi tocca pureadir:Galizia? E poi fen'. aunedrebbe Nanni cieco,
2 Es £2 STAN.

ee Low

o SSR SS aes eee

a

412 MALMANTILE

STANZA LXIV,
eMa, se volubil sei quanto /degnosa
Facciam la pace, manda via lo fdegno;
E [e tu sei de' miseri pietosa, a
Danne,col farmi vincer, qualche segno, Si 3} 5 ma basta po
2» Fu il vincer sempre mai lodeual cosa, O Baccellacero
a» Vincafi per fortuna 50 per ingerno,

ee FFE CEES SP rsa ererersEa

Percio de' danni miei refhando faria, Capitale\ Sarthe
La Fortuna mi sia non la Disgraria, Se tu nd voi più

STANZA
E così finiran tanti [chiamazri
Dichiamar la Fortuna,ei giuochi inginffi, — Ov'io ritrowo ognor exttit
Che mentre vi ti ficchie vit ammarzi Per forza al giuoco mi ric e
T4 [pendi, e paghi il Boiacherifrupi, Appunto, come il ferre-a cal
1i Generale si duole della Fortuna perché gli e contraria, e lo fa
pre: la prega a volersi mutare, ed essergli una volta favorevole: o 0
sto C. 15. stan. 1. dice Px i vincere, ec, Ma poi accorgendoli, che il suo
è inutile, riprende se medesimo, del vizio, che ha di giocare, ma conok
l'ammonizioni non sono abili a farlo desistere dal giuocare..
WON mi tocca a dir; Galizia, Non ho punto il conte mio. I
de della Galea disse:
E se non ne facean tanto romore
Non saria lor toccato a dir: Galizia 3
Tanta gente » andaua per amore. ae
Ed il Persiani dolendosi, che un suo fratello era più lefto, epi aflute
disse;
E prima: 1 mio fratello è una ciuftizia,
Che mi riuede moito bene il pelo,
1 credeu' eljer furbo, e giuro al Ciele
Che seco non mi tocca a dir; Galizia, ket
Da quiefto che dice il Persiani può,chi legge,comprendere il vero
sto detto. 3
NON si farebb! ancl' in Turchia, Non si farebbe in luogo veruno 5
foaa del mondo, se ben fusse il maggior nostro nimico, come ¢id Treo
sopra C. 5. stan. 6. i, Suagtod Cane
SEN' avvedrebbe Nanni cieco, Lo conoscerebbe uno, che non havesse
Lo vedrebbe un Cieco,come era Nanni. li Proverbio dice: éome:
cieco, e (enz' altra aggiunta s' intende, vedere, perché questo Nanni
va sempre; vedere, Si dice anche semplicemente Vamnicreca., o 8!
defimo. Si dice anche: Le vedrebbe Cimabues vibe ncn ciecos 05\che
>

eed

vt

occhi di panno, detto h » venendo da
tura in Firenze, non perché eghi fusse cicco » ma \& voieva denota
fusse nato al mondo cieco, vive affatto al buio del disegno. 1
THM.

444A che gracchia io? Ma che sto io a ciarlare in vano. @



OTTAVO CANTARE, 43

re della Cornacchia yo del graccio, quasi Lat. graceutare, Ma ci serve per clpri-
un cicalare senza mento, senza frutto, oal vento, Vedi sopraC. 1.
staa, 69. C. 4. stan. 25..e c, 7. Rao. 59. Ser Brunetto Latini nel Patathio; in quel

-weelo: Adi aific, 10 non fo.y ch' aurem cornacclie ? volle dire in gergo; alludendo
eal faono della cornacchia; Che auremo per il ores di domani, Lat.cras,

a

 DISDETT 4. Dilgrazia. Maia fortuna. B' ii contrario di Desta, che vuol
 dir buona forcuaa nei Eee, Oinaltro. Sp. defdicha L, malum fatum,mala fors.
 FINCER la posta. Guadagnare quello, che va in giuoco. Vedi sotto in questo

wp o. stan. 7..¢ vuol dire vincere una volta foia.
— PORRE 4 Caualiere, Rimaner superiore, Caxaliere si chiama quella Torretta,
4 nelic Fortezzeavanza sopra a tutte le muraglic della medcfima fortceza;¢ di

Essere yo fiare a Canalere, vuol dire Esscr superiore, o avanzare il compa-

- gno, Varcit Stor, lib, 9. Zara questa parte delle mura di qua d' Arno non banendo

wile Me monti, ne colli sopraccap!, non puo dal di sopra, (come si dice)a canalicre essere offefa.

rd  BACCELLACC I/O, Scimunito, Sciocco; Infeafato. Auguilo Imperadore»
all” diceva bacelus. €

'isp -Lorfe fogna pere, Ognuno Gi figura di goder quel ch' ei vorrebbe, ognuno fo-

: ch'er bramna. Virg. ed. 8. 4% qui amant ipfi fioi somma fingunt. Vedi [0-
 pra C, 2. sta, 7. B per qual caula ti dica / oro, e non altri aaunali, Vedi C. 1.
ca 31. Teocrito ditie; Omnis canis panem fomniat, ec.
| ) @APIT ALE, Questo termine oltr'a i signincati, che dicemmo sopra C. 7.
Mian, 82, protterito nel modo, che e nel presente uogo, ha la forza del Latino
Fiinam o yuoi dire piaccia a Dio, che non sia per essere,¢ che non segua, in
contrario.

y SCHLAMAZZO. Romore, Strepito. Traslato dalle galline, il gridar delle
quali Gi dice ichiumazzare, Ll Vocaboiitta Bolognele dice, che 1 verbo schia~
-Mbazzare significa Kiciamare io darao, dal Verbo Greco Sciamocheo, che vale

} mare cum umbra, Ma e yvanita; perché schiamazzo vien dal Latino exc/smatio,
V1 ficchi evi  ammazzs, ha questo caso son quaGi Sinonimi, € fignificano
— immergerti, o applicacti cutto a una cosa, A "4
bp PAGE boia che +i frajti, Spend per haver danno. Teognide disse: Sibé

Oe sph vineula cudie.

bp LABRICCINO del Paonazei, Intende carte da giocare, perché già un tale de'

te Paonazzi fabbricava dewte carte.

APPYNTO come ti ferro 4 calamita, Per simpatia, come fa la calamita al fer-

i ro; Beeausito detta da Franzeli simaat, cive Pica amante.

df SA ANGA LXVIL STANZA LXVILL

is E Sard. ver, ch' so habbia a star feggetto Datemi dungue un marzo in sula tefia
. | a una cosa, che mi da tormento? Vedere; eccoms qui ch io non mi muaitay

St

a

Come tormento ? oibo \ s' 10 ci ho dilette.

» St un intanco per lui vine scontenta.

O per fido giuocaccto | e maledetto
Cin e ha trouate,e me, chetifrequente,

Ne voi farete cosa men che bonefta,
Se dal.giocar, morendo, io mi rimond,
Soc! ogni di farebbe questa festa,

C! altro diletto, che giocar mon proKo,

i Tu non cs bai colpa tu, 4 me il gaftiga Ed a giuocare omai son tanto avenge
a — 2 poicht cou te m' intrig? he'd pentirms non giana £4 Se



414 MALMANTILE o
STANZA LXIX.
LD usare ogni sapere, ogni mi
Non vale a far mi cotro al gioco,
Imperocch' io t ha fitto si nell' ofa
C? amos mio mal qual afferato inferno,
E forse giochero dentr' alla fofia,
Che forse? diciam pur:tengo per fermo;
E se trouar le carte ini non pufio,
Fari, ( pur chee si eiecbi} all aliofiso. + I quarti anro,oo'far
Seguita il Generale a lamentarsi, e combattendo in lui la voglia
con la ragione, e con la conucuieuza', prega gli amici, che Pai
ché vede, che non c'è altro modo; che egli si rimanga di'giocare}
d' esser certo d' havere a giocare anche dopo morte, e che alla fepoltura
dare con le carte da giocare nel feretro nella maniera,, che esprime
va 70. *
b7z0" « Questa voce ha diversi significati, perché ce ne serviamo
come nel presente luogo: per dimostrazione'di naufea, come oii 5
e questa ? (orto C, 10, tt. 23. per riprenfione, o difapprovazione: Oibe.
cosa,ed esprime il latino Kab, \& espace, E gue) che i Greci distero e4ib
ciamo anche: aibo, eibo, e tbo, 4 Oe
SCONTENTO, Scontclato, disguftaro » La \ettera', sy aggiunta
pio di nomi, verbi, ec, ha nel pariar nofire la forza, che appresso
particella i» privativa di Circa di che vedi il Varchi nell' Hercola
de alla particella ex.
MAZZO, Quei martellone di legno, che adoprano i. Macellati
la tela a' buoi, donde mazznola queiia, che a Roma adoprano per
i malfattori. Si dice anche mageio, nia questo € propriamenteq
prano i bottai a cerchiar le botti, Dal Latino malleus. 18
FARE schermo contro al gioco, Difendersi, o riposarti dat non gioeare.
dal verbo schermire, che vuoi dire Eiercitarsi per imparare a difenderhi
il qual viene dal Germano be/chirmen, siccome vuole il Voto. Dan,
O Grscopo dicea da Sant' Andrea, La ie
Che t' è viovato ds me fare schermo?,
i Petr, Son, 17. Ch' i non son forte ad aspettar la luce
Di questa donna, e non fo fare sebermo
Di luoghi tenebrosi,e a' hore tarde ? i rue
L' HO fitto nell ofa. Ho un desiderio di giocare internatissimo;
giovane innamorato difie, Georg. lib. 3. Quid ivvenis magnum cui
ignem Durus amor? Bil Petrarca.: Dee ai
eee

E ricercami le midolle yet ofa, o ~ A
AMO il mio mal qual ajerato infermo. Come brama: il febbricitante di bee
che gli e nocivo, così bramo io di giocare, che mi e dannoso, she?
e4ALIOSSO. Come habbiamo detto sopra C, 1. st. 9. tutti li gi i
da i Latini si dicono alea: da che io deduco, che questa voce Aliso
Latino alea, \& of, e significhi, come in efictto significa ofo da gu



pe

Ste

:

it
r|

OTTAVOCANTARE. i

i l'aftragalo.de i.Greci., Dicesi ancora Catrioffo; quai. gasdy
uct otio bared et gambe didietro di tutti Pen o
on e nell' agnell

> Pagnello, bue, ec, che negli animali d' ugna sode, come il
ec, o ditate come il Lione, ec. non si trova, eccetto, che nell' Alicor-
o Pol. Virg, lib. 2, cap. 13.. e Dianel Soutero de Aleatoribus lib, primo
+» Buleng. de lud, Veter. c, 58. ed è un' offetto di figura quadrilunga da
concavo,¢ dall' altra conueflo: Nel mezzo del concavo apparisce un
co,, ed il conuetio., che è la parte opposta al concavo, forma: in cia-
luefiancate duc piccoli buchi; nelle testate del fianco al concavo,¢
flo | due superficie quasi piana, se non-che in una si vede un segno come
»¢ hell' altra un segno come un 8., € queste duc parti quando l'Alioffo si
in tavola sono le più difficili a rimanere scoperte, perché ono di più dificil
del concavo, e del conveffo, e l'altre due fiancate non restano mai (co,

f wee perché niuna per la sua rotondita può posare «<I nostri ragazai dell'infima
 plebe,nel giuocare con quest' offo s' adaitano a quei segni, servcndofene per nu-

 Miero.con fare il concavo il numero 4nd, il conuctfo farina, cioè naila, per effler
qusito 11 più facile arimanerescoperto, la parte dove e il segno 8. vince otto
tiene la figura di quci numero;>¢ da' Greci quello numero, di otto negli
chiamato Srefichora,s¢ la parte dove è il. segno.S, vinca dodici,) perché
haf 'a quasi di libra, che si divide in 12. parti; o secondo, che conuengono,
ado, o variando questo giuoco, secondo i patti:(B l'usano dettira-
dalla Pafqua di Refurrezione ( nel qual tempo s' ammazzano gli agaclli,
mpe de' quali si trovano.questi offi } fino a che vengono le pelche, ed al-
lato.' Auuofo.,.¢ giuocano ai nocciolisne i modi detti sopra C, 3. st. 37.
qual giuoco durauo.a giuocare, fino a che stiacciati i noccioli vendono l'ani-
me di ef aghi spcaziali., che fara per tutto Octobre in circa  e da questo tempo
fino a Quaresima giuocano alla ruila, o alle buche com la palla di legno nel
Che si difie opra C, 3... 57.; e per tuttaJa Quarefima giocano alla trot-
« E così distribuscono 1 loro trattenimenti per tutto l'anno., Ma tornando
all' Aliofse; appretio agli antichi Romani era usato dagli huomini più fenfati, ed
in diverse maniere; e fra l'altre il concavo.era chiamato Cane, o canicula forse
da. fiella lucida, che si vede nella bocca del Gane Celefte;, stella cattiva, ¢
malefica; € colui, che tirando faceva apparire detto lato,, posava in tavola due
denari, o quello.che erauo conuenuti fra loro i giocatorl, ed era cattivo, onde
Pecfio dif. Damnosa Canicula quantum Kaderet, la parte oppolta,a, detta era
ohiamata Venus stella benigna, e benelica; e significava il num, fer Latino Serio,
da noi detto Size, nel giuoco dello Sbaraglino, quasi Seino da' Greci chiamato
 Hexites,¢ chi tirando scopriva questa Venere guadagnava [ei, e tutto ello,
che haveyano polazo in tavola coloro,, che havevano scoperto Cane, o Canico-
la, Giulio Poiluce lub. 9. dice, che da ipill, il Sei era chiamato C0,.¢ il Cane,
| Ovverol'Aflo; Chio: e che in questo lor talo non havevano, ne il duc, ne il cia-
gue. Con queiio offo giocavano tanto i Greci, quanto 1 Laci in altre maniere,
~ o fino con sei, e oti offi per volta, ma a me balla haver accennata la suddetta»
“per testimonio, che anticamente ancora era in uso questo-giuoco; e tralalcio di
harrare J' altre manicre che son molie, perché non fa a proposito noltro ee

SE



o0ti('(“' Cz AT

se il Lettore ne faffe curioso legga Polid. Verg. lib. 2.
Aleatoribus lib, pr, cap. 29, Buieng, de lud, Vet. Gap ye
rum gen. lib, 3. Cap, 21. Ho decto, che questo Aljotio ogg
zi, ed il nostro Autore ci addita quetla verita, faccndo r
prrché si giochi, all' altofso, Se trovar le carte ivi mon posso; e intend: V
sempre, e f€ non troverd carte,giuocherd all' a/io/so, quantungue
fagazzi, pur ch' io soddisfaccia al vizioso genio, che ho di giocare.

VAN co libri, ec, A' Dottori, quando portati alla fepoltura j

mettere nel feretro,o bara i librised a i Cavalieri la (pada al fiance f

dice, che fara fatto a lui, che per far conolcere, che meiitre ville era |

re, gli faranno una ghirlanda di quei fiort, che sono itmpreifi nelle

veste fara ricamata di picche, e di cuori, e sotto la testa giù c

mattoni; ed in questa maniera hawia anch' egli actorno tucei quattro'

sono impretfi nelle carte da giocare a primicra. + a ee
Far sn quarto @ germini, Giocare in quaccro alle minchiace, Vedi fop

aS etek

questo C. st. 61. 5 00? th Se Ta
STANZA LXXI. STANZA LXX th
Volea seguir, ma tutti della lanza Amoltance ch e buow ai bei
Gli dieron fu la voce con il dire, E por da bene,acor chia a
Che il perdere e comune,e fhar' nfanza, Dt questo suo viocar, don't si he
E perde una miseria ds tre lire, i p ow
Pero si qusets pure,e habbsa speranta Dicendo \&° a impiccarie hon,,
C' an giorno la difdetta ha da finire, L! bauer femp.icemienteunpo dm o
'Pero che i tempi variabili sono, Ma quand snch ezti havesse ivan Ga
E dopo il triffo n' ha a venire il buono, Del far la [pia non se we fa pr by
STANZA LXXIl, STANZA Lax th
Intanto gli mostraron il Prigione Ed al prigion preterito imperferto” oy
Che fort' il manto deit lpoerifia Rinolto con le carve im man l'itty \ >
In carwa, dicendo, in divozione Già fattofele porre a dirimpette \&,
Faceva lo scultore, idest la spia; i giocar a' nna crazia la tay
Berd, perch' im essetro egli è un euidone Ouver si metra fuor in [i a |
L  impicchi s* ei vuoi far opera pi: Un teftoncino, e sia guerra finitay | fy
Serragli pur, dicean, la gola, e poi, Così lo prega, lo sconginra,e inpatt |
S' ci dice più nulla, apponlo a noi, Bada pur sempre « mescolar leo
Voleva il Generale contiouare il sao lamento, ma 1 circoftanti lo \&
tare confolandolo,¢ mostranuogli, ch' ei si faceva (corgere a far tanto ke ¥
per una perdita disi pochi foidi: Intanto gli prelentarono Piaccianted
it, che lo facesse impiccare, perché egli era Spia; Ma il Generale buonhio® )

jo fece liberare, dicendo, che un poco d' indizic non era bastance a
care, ed oltre a questo del fara spia non se ne fa ne meno procetllo 5
che se s' havessero a fare impiccare tutte le spic ci farebbe facceada, |
medesimo Generale inuita Piaccianteo a giocar seco di poco,¢ (olo per'
Nei che il Poeta esprime il vizio internaco di giuocare, che era
ché nello stesso tempo, che determina di non voler mai più gi i
tersi a giocare fino con un vil prigione, con /' anticta, che muitea Ma



OTTAVO CANTARE. 417
\der sempre a mescolar le carte; come fanno coloro, che punti dal gino-
per haver perduto, vorrebbono pur trovare con chi giocare per ricattarsi.
'LI dieron fu la voce, Lo fecero chetare. Latino.: Vocem alicui comprimere,.
CDE una miseria di tre lire, Perde poco. La voce miseria, che per altro
ifica infelicita, o avarizia, usata in questi termini serve per avvilire; e pero
ime qui una somma di niuna considerazione.
i SOTTO a manto d@' Ipocrifia, Sotto feula, sotto pretesto, sotto coperta di far

teh

: - BACEVA ta feultore, Cie faceva l'alcoltatore, e non lo statuario, ed inten.
de, Stava alla scolta, cio fava ascoltando 1 ditcorsi d' altri per ridirgli; e con
| termine equivoco viene a dir copertamente Far /a /pia, come dichiara il

medesimo
-G71DONE. Furfante, Huomo d' infima plebe (enza riputazione, Vedi sopra

Gr, 63. '
AP. a noi, Mlins crimen affinge nobis, See' fa più la spia, gaftiga noi.
4 'Tiathcuriamo » OP entriamo mallevadori, che e' non fara più la (pia, Elo
[ll stesso', che mo danno, che vedremo [otto C, 11, st. 49. cioè mio sia it danno, se non
'jail Segue tort, Come iv dico,

 HVO.MO di buona passa. Huomo di buona natura, Latino Oleo tranquiltior.
i Plauets in Poenuio', dra hunc canem faciam pibi-oleo tranquilliorem, farò stare zitto,

| 60m' olio,
yet = — DOP ci fr enasta, Dove egli pecca. Con che egli varia la sua buona natura.
4 ~ DEL far la spia non se ne fa proce(so, Gaftigar uno senza far proceffo vuol dir

iio sommariamente. Latino indica canfsa, o più tolto, de plano, cioè
ein nea ita di giudizio, senza sedere a banco di ragione, o come si dice an-
4 che volgarmente pro tribunals; ma qui par che voglia due', che le spie noa solo

we non si gaitigano, ma ne anche se ne fa proceffo. %

yal. PRIGION preterito imperfetto. La voce preterito, che suona passato, qui vuok

wie dir, che il prigione era dictro al Generale; e la voce smperfetro denota Vimperfe-
zione, e vighiaccheria di Piaccianteo.

uli. “
 FN teffoncino. Teflone e una moneta, che vale tre paoli, e da molti in occa-
il fione di giuoco si dice Vm re/toncino, per intendere giochiamo solo un teftone,¢ fi

ai S774 fimta, coe non si giuochi più.;;

BADA a mescolare ve carte, Con questa azione di badare ( cioè continovare ) a
mecolar le carve inuitando colui a giocare esprime, come habbiamo detto, la
ye BFAD vVoglia, che il Generale ha di giocare,
; “STANZA LidY, STANZA LXXVIL
i Queeli che compracerto non gis cofka, Duraro a battagliar forse tre hore,
£ vede bauerl'hauuta a buon mercato; Poi la levaron quasi che del pari;

4 Li inusto tiene, e regge a ogns poita, Se non ch' il General fu vincitore

'i Ben ch'ei non habbia un bagattino atiato, Di certa po di somma di danari,

= E dice, al più faremo una batospa E perché gli domanda, e fa. Sealpore,:

4 Kuddei mi vincaye voglia esser pagato, Quei, che gli spese in cene,e in desinari,

“p Li rapa sangue non si pnd cavare Lon bauer ( dice) manco affegnamento
Ne far due cose, perdere, e pagare « Tal ct Amoftance resta al fallimento,

a tele: gre en



ge MALMANTILE =

Piaccianteo actetta l'innito, e mefiiGi.a giocare il
@ alquanti denari; ma perché Piancianteo non ne haveva
grit Così fa la Fortuna, quando perleguita un giuocatore
o|amente quando oon vi è modo d ciler pagato
\item[L'HA havuta a buon mercato] Ha (campato un gran perict
non ha havuta quella pena, o gafligo., che egli conolceva c
TIENE L inuto, Accetta l'invito,¢ s' accorda a giocare,..
REGGE a ogni posta, Posta ( trattandosi di giuoco ) vuol dir
danaro ? che 1 giuocatori concordano, che corra volta per, eal
si dice inuirare, e reggere 4 ogni posta, 8' intende tenere tutti gl' inuil
BAGATTINO. La quarta parte del quattrino Fiorentino, con altro none
detto Picciolo, Latino We obolum quidem, Voce, è moneta Veneziana,
FARE una batofta, Combattere, e questionare con parole, ec. Latino,
cari, ed habbiamo ancora ii verbo barofare, per combattere prope
ria di Semifonte trattato quarto, Non havendo tanta.geate, che
Terra batoftare, E più sotto. Hor dt qud, hor di la si baoftafe., j
NON si pus cavar di rapa sangue. Non i può cavare una cosa di, wee
\&. Latino. gum è pumice postalare. Plauto. Nam tu aquam e pumice non pr

stulas, qui ipfus fitiat. iva
LA levaron quasi del pari, Cis' intende /a scrittura; Non vi corle qa iene,

cioè si vinfe, e si perdé poco. mitidiy
FA scalpore; Fa romore; Contende alzando la voce. 5 Oe

NON hauer manco afegnamento. Non haver danari, ne modo da trovames.
Ela voce manco in questi termini ha la forza del Latino, nec etiam, ome
quidem, che noi pure diciamo, ne pare, ne meno, ne ance. lo credo,

Ce corrotta da ne anco.

REST A al fallimento, Resta con quel credito da non tlguoter mai pt

fallito s' intcnde colui, che non ha denasi, ne aflegnamenti,

FINE DELL' OITTAVO CANTARE,

ae



: ARGOMENTO
" Ginnti i rinfrescbi, e inusgorito il Campo

ie

Qe

a
Corre all' affalto, e segue aspra baruffa;

~ Malmantil quaft e preso, ond' al un scampo %,
f Chiama all' accordo, e termina la rufa, [ae
i Chi tratta più di guerra hor trova inceampo, >

a = Perché nell' allegrezze ognun si tnffa,
5 Faffi in Corte il conuito, e poi, dal vino
. Riscaldati quei Principi, il feltine. Ol
«ERPS EARS Pb Pape Pape ce ere? 7
: Nie
as 48
STANZAI. STANZAILIL
ye Aguerra, ch' in Latino e detta bello sì che e' mi par ben tondo,ed un corriva,
Parbrutta ame in volgar per sei Befane, Chi pus fear bene in casa allezro,e sano,
Non cr altro se e comincia quel bordello E lascia il proprio per ? appellativo
Di quell' artiglierie, che fom mal fane, Cercando miglior pan, che que! di grano,
| Eche enon v'è da metter' in caffello; Cen' un' altra ancorch'io non arrixe 5
E Slenti poi per altro com' un cane Ch' e quell'assalir un con Parmi in mano,
Sere' un quattrino,e pien di vitupero Che non fol non m' ha fatto viliania,
( Ditelo vei, se questo e un bel meftiero, Ma, che mai viddi in viso in vita mia,
STANZAIL  STANZA IV,
E pur la gente corre,¢ vi s' accampa Florsit cerchi chi vuot bartagliae risse,
- Ognit per farfiua'huomo,e acquiftar gradi, E si chiarifeaye prow: un po le chiare,
~ Quasi degli buomms cosa sia la frampa Che s' io. credessi farmi un altro Viifse
Mentr! il canarne l'ossa avvien aradi, L'armi,percioné m'hano ainzapognare,
LA gli buomin si disfanno,e chi ne scapa Ognuno ha il suo capriccio, come disse
14 tirato diciatto con tre dadi, Quel Lanzoghe volea farft impiccare,
E pria ch' ei ginnga a esser Caporale 'Pero mi quiero, ma perc' bora brama
CHangierd certo, più d'un fraiodi fale, Atoftrarus il vero;attenti,e cominciamo,

,Per introduzione de! presente Cantare, nel quale il Poeta vuol deferiver | af.
Ito dato. a Malmantie, si serve della dimostrazione, che la guerra (ia una brut.

ta cosa, e che pero habbiano poco giudizio coloro, che vt vauno; perché se be»
nei Latini la chi o Bello ( il che secondo alcunt facevano per aatifrali, cig'

Gee 2 pec



420 MALMANTILE

per una figura di parlare contraria a » che s'intende, c
bosco, che € senza luce; Parce le, che memine proctnt
guerra, che non ha in se cosa alcuna di bello, egli nondimeno
tissima, e ripiena di pericoli, come farebbe a dire i colpi 3
abbondante di patimenti, e stenti come farebbe il non haver, che
non haver mai denari; onde un Poeta per ispiegar la bruttezza di
Lelia horrida bella, Oltre a questo @ contro alle ragioni della
gnar l'armi a danno di chi mai ci fece ingiuria alcuna, disse un G
lum a beluis dicirur, perché \& cosa darbeftie, Si maraviglia pero
vada volentieri ingannata dalla speranza, che in quella si face
¢ non s' accorgono, che più tosto vi Gi disfanno, e quand' anche g
ci vuol degli anni prima, che uno conseguisca i minori gradi della o
la guerra Vx fol ne premia, € un million ne ammiazra. Conchivde p
vo di giudizio colui, che potendo stare a casa sua con ogni commodo,
trigarsi con la guerra, e che quanto a se, quand' anche fusse certo d
ventare il maggior' huomo del mondo, non si lascera mai lusingare da
ranze: Ma perché egli fa, che ognuno può far di se a suo modo, sosp
scorrer più de i mali, che nascono dalla guerra, e s' accinge a;
con deferivere l'affalto dato a Malmantile dall' esercito di Baldone.
IN volgare, Cioè a parlare chiaro, fuor di gramatica. '
BRVTT A per jei befane, Befane come dicemmo si C..8, st. 30. vu
Panioccio fatto di cenci,e di qui per Befana intendiamo non solamente
na brutta, e mal fatta; Ma le Balie si servono della voce befana per i
una di quelle Larue, che nuocono a i bambini, come il Baw, er.;'¢ gli p
no, che ci sia la Befana cattiva,¢ la buona, e che venga nellecale perk
del cammino del focolare, e però la notte avanti al giorno dell' Bpifania,
Gio, Villani lib. 7, e I nostro popolo anc' oggi chiama Befania, onde;
mente vien questo nome di Befana, come s'\& detto sopra, fanno che i
appicchino le calze a i cammini, perché le dette Befane gliel' empiano di
buona, o cattiva, secondo, che essi sono stati 6 buoni, o cattivi ze tali
buone, o eattive si figurano sempre brutte; onde bratro per sei befane vuol dit
eftremamente brutto. J Filofofi icolaftici per esprimer più la, che i
dicono M2 «fo, dando alle qualita gradi fino in otto, e volgarmente per elprimt
lo Reflo si dice Sei, come di sei corre, ec, se bene e un termine, che ha
furbelco, Cscala per sei putte, e simili. Ll Ferrari cavando la definizione
na dal Politi Autor Sanese la descrive così: Larwale fimulacrum, ——
nia puerss terriculamentum Suspenditur; unde nomen invenit, B foggiunfe w
mulieres deformes Befane dicuntur larua illa turpiores. Dice finalmente, che i Frar
cel dicono Tphanie dal Greco Tbeophama, cioè Apparizione.d' iddio.
nocte danno ad intendere le superftiziot(e,¢ ignoranti femmine a' semplici
li, che seguano molte cose fuor dell' ordine della natura., miracoloic,
per esser la vigilia della festa de' Magi, né sanno, che con questo nt ¢
Persiani, ond' ebbe origine, eran chiamati i Savi, e intendenti
Natura, delle Stelle, e de! Ciclo. ia 3 NM

at

EPS \&

eee FF FaTRRSE

oe: esx EB >i

WEL bordello, La voce bordello, che propriamente vuol ie i igo



NONO CANTARE. qat

blico dove abitano:le meretrici., e presa da noi in più fenfi, come per frepito, 0

per una cosa flucchevole:,¢ noiosa, come è presa nel presente luogo, e altri la

iglian inteoder Difficulta, o fatica »comela prefe il Lalli nella sua Ea.tr.

le paroled: Verg: Hoc opus bic labor,:

enn Ene aio bello 5

8 et Cafacalda si va presto presto;

}) gameboeene| 2-1 | Ma vitornar in fu, quefioe il bordello,

aa "0 è da mettere in Castelo. Specie di pariar Janadattico, del quale par-

2 a. sopra C, 1. st, 29. alla voce /eminato, es' intende Non v' è da mettere in
~,

» che significa poi; non v' e reba da mertere sm corpo, cioè non y'è da man-

'. In furbesco; Non v'è da smorfire; Non-v' e da empiere il fuflo, che così

, dicesiil corpo nello stesso modo, che in Greco volgare si dice Cormi da literale
a Corner, che vuol dire Fusso,o Ceppo, Latino /ipes, candex.

| ~ STENT A come uncane, Patilci, ed hai careftia delle cose necessarie.al vivere.

: eo della Caccia lib. 5. Ergo age duro dffuescant vittu catuli, Si dice frentar

bracco, quando uno per la sua poverta ha male il modo di provvedersi il

we

ie mee

-PIENO ai vitupero. Pieno di pidocchi, rogna, ed altre tattere, e porcherie
4 i indivitbil della soldzvefea yi chet dice anche: Pieno “4s Bobbio, dal

— Latinovepprobrinm, ebbrobrio ) e Peno di fastidio; del refto Vitupere significa infamia

bye vergegna, Bocc, Nov. 63.
in ET.. Abi vitupero del guasto mondo
ptt T] medesimo Boccaccio nella Teleide lib. 1.

BOs Abi vitupero della gente Achiva,

ee Omero,¢ Epimenide citato da', Paolo diticro in questo senso mala probra, cioè
id vituperosi.
o Per farff haomo. Per diventar un' huomo valoroso: Che essere un huomo, 6
sit an'huome, serve apprelso di noi per intender quello, che intendeva Diogene,
of ES diceva 1 Aominem quero, diccfi Esser un' huomo Givven. f wis efe aliquis,
ie scrittara Confortamini, \& essere robuffi, Omero, Viri effore, \& forte cor fumite,
af VCHivescampa. Scampare vuol dire fuggire, scappare, o liberarsi da un peri-
PI colo + € qui intende chi eicè vivo, o avanza alla guerra, Scampare, quali u/cire
J dal.campo; dalla battaglia. !
| o MA twraterdiciotto con tre dadi. Ha havuto la maggior fortuna, che si possa,
haere,/perché il cum, 18, ¢)il maggiore, che si potia fare con tre adi. 1.Gre~
J cl pure ond eae dicevano: Ter /ex iattare, come si ricava da Giulio
| Pollucesnell? Onomattico. Sy aah it
Bi CAPORALE. Capo di fquadra, che fra gli Vfiziali e il minor grado, che si
j dia nella milizia, Caporati disser gli antichi per Principale,Latino Capitalis. Gio,
Villani 1. 28. parlando di Roma dice:

ee Fu caporale regno di se medesima
— Biib. 12.89. eA tutte le caporali Citia a' tralia.
La voce è formata dall' antico plurale 'Capora', come Campora, Borgora', e

simili. °°:
- MANGERA pri: duno fraio di fale, Significa-consumesa molto tempo, perch
x molto

—

4uz MALMANTILE™
molto tempo ci vuole aun' huomo solo a consumare 'uno f

chi, quando volevano significare-un 'tempo lungo; dicevano com
che sled da mangiare a d' un -moggio di sale, Cicerone de Ami
que illud est, quod vulgo dicitur wultos modior 'falis fimul edendos esse
nus expletiim fit, Questamaniera proverbiale pure in. pro
usata da Piurarco nel libro della multiplicita degli amici » Si può
che inghiottira più d' un boccone amaro 5: e di poco suo:gusto.
con troppo fale si dice amara; e pero mangiando molto faleman
amaro. ' ' +.
TONDO, e-corrivo.. Si poslon dir-finonimi; e il primo signific
fo, ed inGpido, ed il secondo, che:si dice\anche: Corribo., huomo leg
cile a creder' ogni cosa. Latino credu/xs:.. 1\Napolerani dicono ¢
minchionare, burlare.,.¢ dar. pasto'a uno; sopra-C..6..f, 80, disse.«
tondi più dell' O di Giotto, chesuona loyftessa, Tonsa fimiimente: pre
}i vale balordo, dappoco., semplice, goffoxCunto degli cunti?
Bue.:
LASCIAR il proprio per  appellative. Maniera di dire tratta dalla
in cui si danno nomi di due forte, alcuni chiamati propri, altri appell
dire; Lasciar il certo per It incerto, Bar come il Cand' Biopo ci
che haveva in bocca,per pigliar quella,della quale vedeva,lo shattimento 4
qua, che gli pareva magguore, e lo stefto significato ha; Cercarmmighor
grano, Eliodo Poeta Greco: Folle e colui, che lascia andar le cose facili
¢ con certa speme segue le pin difficili, e lomrane..\4\ pene
10 non arrino, Cioè lo noa comprendo + lo non arrivo col mio giudizio a it
tendere. In lingua furbesca.. foo» ammasco s non redo  cive non piglio, nonae
zanno, non comprendo. Lat. non affeguor. iru

ESPs SGP SSE

ee ee rset

VILLANIA, [ngiuria,Soprufo, mal termine s LG

S? io credessi farmi un nuouo Viiffe ec, 8' io credefGi di diventare il maggior hut
mo del mondo. Diciamo Va nuoxe Orlando. L Greci Alter Hercules, 'gh

SI chiarisca col pronar le chiare, S' accerti di que(ta cosa con provare le feri
perché chiara intendiamo quell' albume deli' uova, i quale s' adopra a medicit
le ferite, vedi sopra C, 1. stan. 60. ed il Poeta servendosi del verbo ehrarive che
vuol dire (caponire, o (gannare,€ della voce chiare fa nascere lo (cherad.
 INZAMPOGNARE., Ingannar con infinghe. Lat, Verba dare:ed è 10 hielo
che iatinocchiare detto sopra C, 7, stan. 14. Dalla natura del suono, e della Me
fica.; incancatrice delle meoti degli nomin.. Fra tutti gli trumenti però. que d
fiato, levano più di (efto, e pare, che percuotano l'anima più gagit ¢
onde furono, ad esclufione degli altri,usati nelle battaglie, nelle quali facevad
meitieri tor via da cuori l'appreafione del pericolo., e infonderni la a
speranza. Noi habbiamo un Proverbio. Far come i Pifferi di montagna (|
nator di piffero strumento di fiato contadine(co.) che andarono per piferare
rono pifferats. Volcano minchionare gli altri coi, darne, ¢. furonc.
col toccaine. Fare uno cornamu/a appresso il Pulci, ¢'] Burchiclioe
inzampognare verbo facto da stampogna strumento di fiato rufticale, così a
Symphonia, della qual voce servcndosi Daniello al cap. 3. nell' Litoria

cn x aie



 ciulli 5

= a

ae

S

S=etis \&. BEx es a i EO

, NONO CANTARE.

43

¢ narrando che essi non attefero-punto il cenno., che per comando Regio
si dava, @ adorare la Statua, col suono di tromba, di cevera, di finfonia:, e di

ees ae suoni; sg si può dire [ siami lecito qui dt servicmi. di questa baga ma-
inzampognare,come giù altri. Tromper in Vranz,

=» e pur dal Latino carmina,

we

be Dis LEAN ZAWV..
2 aurora ye come diligente
azza le stelle in Cielo, ¢fapulito,
eae ffi alla finestra d! Oriente
Evora l orinal del suo Marito,
. Ma perché il Carretton ricco, e lucente

. Acciocch'ei non la vegga/cociase/ciarta,
eee: amegerneved ei iirimpiatta.
: STANZA:
Quande il vutto easone ' (rinfresco '
St che,chi hauea col mafticar dinieto,
pe oma iecamente il corpo al desco,
E come si /uot dir ) riebbe il pero;
ae Hi General, che tutta notte al fresco
nda con? Afirolabio innanzi,e indreto,
Batrendo la Diana in sul lunario
Hanea fatto di Stelle un calendario,

~ Edi nostro Autore dice =

. Già muone il Sole,ed ella U' ha sentiza,.

as = forse a corno, o tromba de' ciurmatori: E Charmer Ancantayes >

UGNFNO ha il suo capriccio. Virg. Quifque fuas patimur manes, Ognuno ha je
: fantalic, Vo Lanzo, essendo riprelo, perché faceva cose da esser impiccato,
ve Che folerce tire » lasciatte far a ie, percht ho ancor ie mie pelle capricce, BE chi
ha Lanzw, Vedi sopra C, 1. stan. 52. e c. 4. stan, 36.

STANZA VII.
seaienaat era anch' egli riuedere
Tutto quanto aggrez.rato al pappalecco,
Done per hauer meglio it suo doxere
Fece in principioun bel murare afecca:
Quando fu pieno,al fin chiese da bere,
E poi ch' egli hebbe in molle polto ilbecco:
Fighnoli, 3 4iffe, omai venutat l'hora,
Ch' e' si tratta d' hanerla acauar fuora.
STANZA VIIiL.

S' a mensa ognun di voi tanto s' affolta,
Atangia per quattro,e bewerpoi per fecte 5
Che par proprio che sia giuntoa ricolta,
Anrich'egls bablia afar le fuevendette,
Tat ch' io pensai vedern' anc' una volta
La tonaglia ingoiare,¢ le faluette,
Ed bebbi un tratto anche di me paura,
Per una spalla dauola sicura,

“I nostro Poeca de(crivendo la levata del Sole imita Daate nel Purg, C.2.dove,
descrivendo anch' egli il parcir dell” Aurora dice:

65 bid Sicche le bianche,¢ le vermiglie.guance
La doue io era, dela bella Aurora,
Per troppa etade dieninan rance.

Accio ch' ei non la vegea sconcia ye [ciatta,
Manda giù impannata,¢ si rimpiatta.

Ed intendono Vaoy et Alero, che quei colore, 1: quale appariva nell' Oriz-
lente per caula dell' Aurora, era quai (parito; ed in iu queit' hora comparue la
munizione da bocca, edi soldati i rinfrescarono. Dopo di che 1) Generale det-
'We principio a far l'orazione per inanimire i Soldau j quaic Orazione militace si
Soutiene nelle presenti stanze fettima, e ottava, e nelle quaccro segueati.

ere de fielle in Cielo, e fa pulico,. L' Aurora coi ivy ipicndore, offusca

quelio



4z4 MALMANTILE ~

quello delle Stelle, € così le leva dai Cielo e lo fgombra;
VOT AP orinas del uo Mdarito, Cioè del vecchio Titone favol
la Aurora, Virg, Tithont crovenm tinquens Aurora cubile. D
cubina di Titon antico già s* imbiancaua al balzo a' Oriente Puor delle
dolce amico, Qui pero descrive Aurora nei suo primo app
la parola  imbiancana. Li noitro Poeta poi-, per 'Vorsmale e J
tende quella rugiada, la quale caica sopr' alla terra cicca'l apparic
la qual' hora l'Alba, o Aurora si perde, pero dice Adanda gin o impan
rimpiatta, cioè ferra le tinciire, es' asconde J “
SCONCLA, e feiatta. Si potion dir Sinonimi.. Se bene /oon cia
mente dire una Donna, che non si sia ancora accomodata icapelli
quale accomodaiento di capelit dice Accunciatu ase feiatea vaold
scompotta, e che habbta gu abiti male adattati, e agguitati
sconcio e più generica, che nome la voce /sarto, core:
tine. Znconcinnus, inbonestus, wdecens, incompositus',
1M? ANNAT A, Così chiawiiamo questeiat di legno sportellatt
tono alle fineilre per chiuderle con carta, tela 50 vetsr, che vi si
fendersi dai treddo,o dal Soe, \& mandar giù  émpannaca vuoldir se
tclio di gueito telaio, e chiuder la fineitra; perché per lo più deceit:
aggiu(tati in maniera, che per aprire, e Chiudere s'\ alzano, ed. abbul
diciamo tar fu, e manaar gin. 6
SJ rimpiatta, S? a(conde. Vedi sopra C, 7. stam. 66.
HAVE A col mafwar dimcto, A chi era vietato i mangiare t
havevano ) traslato da 1 Magittract di Firenze, Re' quaii ti dice baxer
non poter conseguirgli, e aver proibizione per qualche tempo di et
jut, che v' habbra parenu, oche gi habbia efereman di corto, Oo) per  @
givni ttabilite dalle leggi. Dan. Purg, C. 14. one
Lav' e meffier ds conforto Diniero, asthe
Negli Statuti Fiorentini diceti barbaramecate Dewerum ou itl
LIET AMENT E, Vuol dire-Allegramente da lito; se bene i noltti Contile
pi dicono /eramenre in vece di prettamente; e forse qui i Autore lo Cee
fio tenlo; perché si può credere, che 1 soldatis' accoftafiero \& mangiare:
gramente, e prestamente. Li Lat edacer donde e venutu il Poscano Allegri s*
1 Frangele Alaigre ( che più mostra la iua origine ) vale pronto, H
E /efo per avventura puo eiler fatto da servs ae
AP POGGLARE il corpo al desco. Si dice anche di chi rifeuote danari o prove
fione da banco, o luogo pubbico. Cie accoltarti alla menia per mangiare.
RIEBBE wf pero, Svritociiia + Ripreie forza sok pero quello tia, vedi
6. ttan, 107. Del riavere i) peto vedi una curiosa noveilettain Giovannt:
te,detto Gioviano Poatano,ne! Diaiogo iniacolato earenio p
cipio. Del maic che:fa al vento caccaiuly, o del beue, che neit:
cice, se ne legge un'epig) Greco di Nv » melita 3 1
dire Fiorita Kaccolta de' medetunt bpigramun 51 quaic cradon ave
suona così. Peditus occadst muitos incinjus in aluo; Lipiojts batoo,
Seriat,@ occidie rurfum si peditus; ergo Regibus auguftis qui

Fz ER THELIST.

Be

eee 2 baw ere



cue NONOCANTARE ~- — 45

BATTENDO 14 Diana in sul lunario, Tremando dal freddo per essere thao
all' aria a considerar le stelle. Batrer la Diana, Vuol dir battere il tamburo all'
pparir del giorno, quando si vede la Stella mattutina, ovvero Stella Diana, cioè

del di. Ma per mecafora intendiamo battere i denti per il freddo, che di- ae

mo anche barter la bora, Vedi sopra C. 8. stan. 6, >, a

 TVTTO aggrezzato, Intirizzato per il freddo; Affiderato; Agghiacciato;;

sghiadato; morto di freddo. sggrinzato truovafi nell' antico per secco, e
liato di carne, quali sogliono restare i morti ( appellati perciò da Greci /i-

res, ci0è privi d' umidore, secondo che vuole Pjutarco nel libro intitolaro 'J
inal sia de' due più profitrenole; ! acqua, o pure i fuoco,¢ quali si veggono essere

is mie structe, smunte, e secche. da Aggrinzaro forte e nato Aggrizzato, p
| PAPPALECCO, Antende al mangiamento in generale: che per altro Pappa-

 decco se - leccornia, ghiottornia ( Franzcle; friandife ) come habbiamo veduto

1C.7. stan. 55.
i hes Os niece il suo donere, ec, Mostra che il Generale, essendo affamato,
yi aifolratle anch' egli a mangiare, acciocché gli toccasse la sua parte; intenden-
j ' do che mangio assai prima di bere Tee murare a secco, vuol dir murare senza
eaicina o alcro bicume, ma con i foii safi, e trateandosi di mangtare vuol cir
jot Mangiare senza bere. Nell' antico facevano la parte a mangiare, e a cia~
feheduno toccava la sua; il Juffo poi levd questa usanza; dice Plucarco nelle Que-
 stioni Conviviali lib, 2. g. 10.
; MESSE il becco in molie, Vuol dit bere, pigliandosi la voce becco, che vuol dir
re il rofteo degli uccellr, per la bocea deli huomo, Questo detto merrer il becco in
molle Gguinca anche parlare, aprir la bocca. Gli Spagnuoli la faccia dell' humo

dicot roffro da quella degli uccelli.
i 'S' afolta'. S? atfacica con furia, e con vehemenza.
im STA Gitmo 4 ricotra, Cioè ch' e' G sia nell' abbondanza maggiore, come si fup-
pone che e' si sia nel tempo, che si faono le raccolte: Se forse nua voletim» dire,
che costoro mangiando facevano uno sparecchiare simile a quello, che tanao co-
loro che fegano 1 grano, ec.

PAR cbt egli habbia a far le sue vendette. Quand' altri mangia,¢ beve assai,o
fa quaififia operazione fen' iatermiiione, riposo, o rispiarmo, ci serviano di
questo'detto, assomigliando quel tale a uno, che per vendicarsi portato dall' ira
Opert veementemente.

PER una spalla davola sicura.M'era entrato così gran timore, che non mangiaflero
anche me, che d'accordo havrei daca una delle mie spalle per confecuarim: 1 ceito,

STANZA IX. STANZA xX.
Redeamus ad rem; Se ( come ho detto') Che quasi fui per dar nelle girelle,
Qua fufte al ber infer mie al magiar fani, Perché dopo ch' i punti della Luna

Eco+ coltelli sn man, (Pandoui a petto, Hebbs deferitti, e che extse'le Helle
| Runfeiste si brani (parapant, fic Haneuo rassegnate ad una ad una
bli battaglia vedervi ancora aspetto Trouo [marrite bauer le Gallinelle:
| Con la spada così menar le mant, Ma dopo è, ch' io mi dauo alla fortund,
Ona ib aimico vino, ed abbartuto Che fra le elle fiffe, efral' erranti,
NNe sia, come franotre ho preveduto « Won vedenone anche i Mercatanti,
VR Hhh <2 Ska

CRRREBALERE EMASE.

=



26 MALMANTILE

STANZA XI.

M€a dissi poi da me, che poco importa
Se quel branco di Polli non si troua
Ani che questo a noi risparrio apporta,
Peroche magian molto,e non fann' nova;

E [e ne anche alcuna Stella ho scorta
De! Mercatanti, qui creder mi gioua,
Che e'fieno in fierayo vero al lor viaggio,
Per laViaLatrea a mercatar formaggio, Essi cerchin la roba, e mo
Seguita il Generale la sua orazione militare, con la quale dopo hai
suoi Soldati di bravi nella maniera, che si vede, termina suo
che si vada ad affaitare il nimico, perché spera, che sieno per h;
tuna per le ragioni, che dice, con le quali da un poco di bur! ara
FVSTE al bere infermi, al mangiar fani, Bevelte, e mang te assai,
gi' Intermi per lo più vorrcbbono sempre bere, ed i fani mangiano
cassai.:
ST-ANDOV1 a petio co' coltelli in mano. Par che voglia dire,
fronte per far alle coltellate » ed intende, che stayano a mensa uno
altro co' coltelli in mano per tagliar pane, e c,, ec.
SPAR AP ANT, Così diciamo per derisione a un bravazzone, e qui ton
ne, perché questi soldati mangiavano gran quantità di pane, 4 '
PIÙ per dar nelle gireile. Fui per dare la volta al cerucllo. Vedi sopraC.t.
GALLINELLE, Quelle sette Stelle, che si veggono fra il Tauro, ef
dette Pleradi; in Lat. Vergilie, Il comento d' Arato Latino. Pleiades 4 plartit
te Graci vocant, Latini eo guod Vere exoriantur Vergilias dunt. Aicum dil
Pleiades sieno nominati, quasi Plefiades cio che si Ranno accoflo,per.
ci le chiamaton anche B try, cioè Grappol d' uva,¢ noi Galinelley p
piccole,¢ in un mucchio. Lt Vberti nel Dittamondo.
Poi disse: guarda nella frome a quelle,
Le qua' da' fani 'Pliadi [on dette,
E che i volear le chiaman Gallinelle, 4
AU! dauo alla fortuna, Mi tribolayo: Mi disperavo: Si dice an
alle freghe, al diauolo, alla versiera, alle bertucce, a' cani e simili,
fortuna: tratto per avventura, da' Marinari, quando disperati, ab
in braccio alla borra(ca; la quale da' nofiri Toscani fortuna di mare 5¢
folutamente vien detta. Il Petrarca s' era dato in un certo, modo alla
quando,descrivendo il suo stato infelice diceva. a wi
Fra si contrari venti in frale barca.
Ui trouo in alto mar senza gouerno,.
E poi. Ch' ia mede/mo non fo quel ch' io
MERE AT ANT1. Le tre stelle del cingold @
Tauro, così dette perché sono infeme, e paion compagne,
ragione. Adercatante dicevano gli antichi quel che noi. oggi p
-reante. L' arte de' Mercatanti nella nostra Città ancora al,
servato l'antico nome.,: % '

SREERGERE

2RERSE

ee ae



NONO CANTARE. 27

 BRANCO 4i polli.Latende le Gallinelle dette di sopra.ll Ferrarialla voce Branca
dice in fondo: Branco eream pro grege.Vin branco-di pecore.Vaa mano di pecore.
Mon n pro mulritudine, ec, Manus autem est branca, ut alibi anumaduerfurm,
REDER mi vious che fien per la Via Latcea, ec. Scherzando con queiti aoint di
clot Gallinelle, e Mercatanti discorce di esse, come se quelle fussero galline,
che son difatili,perché mangiano, e non fanno uova,¢ che questi Mer-
i non eran nel Cicio, percèé erano andati a provvederd di formaggio
Via Lactea, la quale egli fuppoae di latte, e che pero vi sia il formaggio a
Mercato; e conchiade, che ancor questi sono difutili, perché fond intenti
ente a' guadagni, e non si curano di gloria di guerre; e pero che e bene, che
. questi non Gi trovino ia Cielo, perché torna a ior favore, e pero si poila
8 “ entrar' in guerra con buono augurio. Ridicole confeguenze altrologiche, con le
'quali mottra la poca stima, che egli fa dell' Astrologia come di cosa frivola,e vana,
— Fra laren, 8 quel circolo bianco, che divide da una parte all' altra l'Oriz-
"-zonté, edi nose i vede 1m Ciclo la meta, il quale dicono tia formato di miaucil
fime fielle; Da molti è cniamato /a va Romana, Dan. nel Parad, C. 14. la chia-
| m0 Galafia, dalla voce Greca, colla quale questo yalibul cercnio del Ciclo si caia-
Ma Galaxsas, cive laccco,
| Come distinta da minori in maggi
ee Lum biancheggia tras pols del mondo,
a Galafiass, che fa dubbiar ben Sages,
SON boti; Son huoauni di gesso, e di Aucco; che s'intende huomini buo-
ot at ia yilolidi; Lat. frpites, caudices. Vedi sopra C. 4. tan. 17. e sotto C.
ws Tt, fap, 41. Similicudine tratta da quell' immagini, che appicca nelic Chicle chi
ge 8 botato. In ispagayoio Sore e (puatato, che ha il caglio morto, Lat, hebes,
age tt Oftde boro de ingenso vale huomo d' ingegno poco vivace; ouylo.
se | DANNO te ferste con (a penna, Cioè terilcono sella borla, quando scrivono
Te partite in debico a uao. EB verameute le partite in debita sono ferite, perché
GidiceL denars sono it secondo sangue, i) quale con tali ferite si cava d' addosso al
Proilimo, Così i dice volgarmcnte Tarare ana frecesa, calui, che chiede a uo' al-
tro in danari,vedi topra ( 2.¢ insdguinarti chi comincia a toccar guattrini,
sh) Dl dar foro, Deve dare, coe divicae lor dzbitore, e per l'equivoco inten-
de deve Perquocergli; e da cio cava la coalegueuza, che noa fiea buont per la
Suerra, poiche se cia piantaav una partica ( snteadi dispongono una parte, una
 quaama di Soidati Jognuno gli dee dare (taccadi perquocere tali Soldati ) e
j gueilt che da tutti ac coccane, boo son buoui per la guerra, Psancare una par-
Ma Cinferire, o descrivere nel Giorudle, o uubro di uegozio uaa parte, o arcico-
lo, capo di (crittura, che da dcbuo, € credito a chi s' alpetia; 1 che si dice
anche decendere una parsita, decendere uno debore ye creanwe, toric dal Latino
recerfere, deiccivere, regiltrare.
STANZA XIIL

| Non prima fabili l'andare in GMErTA, Com un bratcod uccelliil quale in terra
Che vede/ts pie presto ch' 10 nol dico Sts calato a beccar grano, o paniva;
Vitleuaiena, «ur trattoyun ferra ferray Va che si muons basta, che quct solo
Ed ir correnas contr' Alil inimne. £4 fuoice pyuare a tats nw volo,
è: Hhh z STAN:

zat,



428 MALMANTILE™
STANZA XIV.
J coraggioft al primo, che si moffe,
Gli altri (già fendo meglio [ui piccenali )
Non poterono star più alle moffe,
Ma corsero ancor lor come Terzuoli 5
Giunti di Malmantile in fu le fose,
Drizrate al muro assai feale a pinuoli
i falirvi renewano una baia
Com' andar pe' piccions in colombaia.
STANZA
Gli fiipits, le foglie, e gli architraui
A quclto efecto essendo già (murati
Per via di curri, dargani, e ditrani
Gli hanevan su le mura strascinati, Faceano un venga addossoat
Stabuito d' entrare in guerra,¢ dar U affalto a Malmantile i più ¢
rono i primi a muoverdi, e gli altri meno coraggiosi (eguicarono. \&
Dante, che nei Purg. C, 2, dice:
Come quando cogliendo o biada, o loglio
J colombi adunati alla paffura
Quieti senza mostrar U usato orgoglio 5
Se cosa appar ond' essi habbian paura
Subitamente lasciano fRar ? esca
Perché affaliti son da maggior cura,
Arrivati dungue alle mura di Malmantile, credendosi di trovar fac
s' ingannarono, perché quei di sopra gagliardamente si difendevano
altro. Qui e da considerare, che se bene Capitelliye Srontespizzs son me
shitettura, il Poeta (cherzando con l'equivoco.di capi, e fronti, e serve
verbo Pampare nel senso, che lo pigliano i Legnaiuoli, ec, che dicen
C. 1, tt. 8., vuol die, che tali merii pictre, ed altro devano sopra 1 2
alic fronti dei soldati, e gli stampavane, cive gli facevano di quei-
chiamano stampe, ed in fuftanza vuol dire, che rompevano tefle,¢
suono, che rendono i corpi battuti fecero i Greci il lor verbo typrein,
re; da questo verbo ne venne Typus voce pur Greca accettata da'|
una forma imprefia, o cavata fuori col battere: Se ne fece ancora 7}
tamburo, che Omero più conforme all' origine disse Tympanon seguito
Catullo nel Poema Gailiambico. Noi abbiamo voci da riferire a queste' \
come farebbe Stampa, Stampita, Stampare, Stampanare, Ma in pro
fiampe fatte sul moftaccic d' un' antico Giucatore di pugna, evvi un
gramma del Greco Lucilio, che in nostra lingua voltato dice Così;
2 un vaglio, Appollofane, il tua capo,
O qual fu mai pin traforato arnese,
Son tane di formiche 90r dritte, or torte,
E par, che con bizzarre, e varie nore
Un Lirico eccellente il Lidio v' abbia
Inravolate sopra, ol Frigio canto.

esceftie?

jie te te en ee i a.


NONO CANTARE,
6 Or franco vibra il minacevol pugno
 Ecombarci pur liero in duro arringo 5
 Che se colpo novelio a te discende,:
Quel ch' ai riscoffo, aurai, ma non già nuond
et Capir nel capo tuo potra ferita,
PIP preso chrio nol dico,, Preftitiino consumaron manco tempo a far tal cosa,di
silo, che io consumo a dirlo. Latino dicto citsus.
“N lena leua, un ferra ferra, Quando vogliamo intendere, che una gran quan:
: di popolo adunata in qualche luogo si sia partita in un subito, e velocemente
ia one di questo.detto 5B signiticano quasi lo stesso, se non che l'ultimo ef-

» quando uno è da altri incaizato a correr, ec, vedi sopra C. 1, st. 63. e»

ke
- f hail

pero nel p luogo si potrebbe anche dere, che i primi volon-
 tarj, ed 1 secondi forzati dalla riputazione. 11 Varchi Stor, lib. 2. dice: Pa /ubir
| Wegridato: armi armi, lena lena, ferra serra, ec, Dal che si cava, che questo detto
tog significhi Leva la roba di sopr' alle;moftce delle botteghe, e ferrale come (eguiva
at | Firenze nelle sollevaziont di popolo, e che ii medesimo detto sia poi facto co-

Mune a oga: sorta di tumulto, e per ¢sprimer un moto turioso di quaatita di po-

4
Ll

| AR correnda. Andar correndo. Il verbo ire venendo dal Latino, vale appresso
di aot quaato il verbo anaare, ma ci serviamo solo dell' Intinito ire, del partici-
Pi9 ito, o solo, o accompagnato col verbo essere, e dell' Lmperfetto ina, ixano,
che si dice poi, giva, e giwano, Nella vita di Cosa di Rienzo (critta in lingua Ro-

Mana antica trovali jio, e seffero, e simili, che i Toscani cangiando l'[ coafonan-
foi *ealpra nella doice lettera si dicono gio, cioè andò, € gifero, cioè andaflero.
wi fimiimente prende alcuai tempi, come farebbe i presenti di tutti i modi,
'i dai verbo Vado, io vo; ancorche Dante viatle forefticramente, edadi per Vada;
gg o 0i0 cofretto dalia rima.

» ST ANDÒ mestio in fui piccinoli, Essendo pi gagliardi nelle gambe; e questo
gi AVVeniva, perché havevano mangiato. £ piccinols, che è il gambo delle fruttes.
g Latino pedicutus, e pref comunemente in questo caso per le gambe dell' huomo,
ia NON porersero Rar falc alie moffe. Non potettero contenersi, che non corref-
a fero. Toho da j Wavalli Bacbari, i quali corrono a i palj, che essendo tenuti per

lo freno dai loro Stallon: al luogo donde a) suono della tromba deeono partirsi,
7 che si dice le moffe ( Latino carceres ) molte volte scappano, prima che sia dato i)
' detto segno,e questo si dice non far ferme aie moffe, che poi passato in proverbio
! non haver pazzicnza, o lofferenza, ma per il gran desiderio d' arriva-
i Tea Uo luogo, partirsi prima del dovere; ed esprime quella inquietudine, che uno
, hanelitaspercar, che segua una tal cosa da iui ansiosamente bramata. Del Ca-

vallo generoso Virg. Georg. 3.
Stare loco nescit, micat auribus,si tremit artus,
Colettumgue premens volvit sub naribus ignem,

CORSERO come terzxoli, Corsero con la stessa velocita,con ia quale vola alla.
preda il terzuoio (pecie di falcone. Perché così sia detto rende ta ragione il Tua-
No de re accipitraria lib. 1, edtrque ad co. cum tres foetu enitatur eodem Predones gene~
rosa parens mas kitinus imo despectus letto incet appellatur, \& inde Tertius,

SCA-



4jo MALMANTYPLE \&
SCALE 4 pivoli. Scale fabricate di due corredti «
glioni sono pivoli ficcati fra 'uho ¥'¢ l'altrore C
fine in distanza uguale a riscontro, ovvero'i detti f
o stecche, o regoli di legno conficcati in deeti correntt Mampati
riscontro. B pinole, ( Latino clanicx/a, civxt cavicchio; ovvero
de ogni pezzo di-bastone adattato a porerd mettere in un buco,
TENEV ANO una baia, Stimavand cof: facile;*Stima
burla, ec. Latino mage, Ii Ferrari dice poter venire questa voce da
iflar' a bada, in ozio, Latino wataré, o O01 i
COLOA18 ALE, Quelle stanze fabricate per lo più nelle form
per uso de i colombi, € nelle quali'wascono i piceionit) «>
FEC ERO parergli altro suono, Fecero lor conotcere, che |
ment.. voy
ewERLI, Qvei picco}i murelli'; in distanza uguale'y ned quali per!
mioano te muraghie delle Città, ¢servond per: parapetti'. ad soldati,
per difesa della muraglia; così dette quali. murnlesdice il Berrari; fume
primes parus murs,Dichiamo @una-cosa;che ancora abbia delle dific
rarsi,e che non Gi fiano per anco spuntate: £ ci è de merio, cioè non è elpy
to il cutto, Ci rea ancora qualche parte da abbattere 2 Vedi sotto o 12)
ISSO fatro. Subito. Due voci Latine corrotte, e ridotte Toscane,
loro lo stesso signincato.
DISEATTO (e reftuggini, Infrante le Teftuggini animali Terreftri,
che hanno la coccia, o guscio durissimo da alcuni'detti Tartaruche
he, da altri bezzache ( dal bezzicare,.ch' elle fanno raspando in terra
atinl Tefudmes, E § potriaanche dire, che ? Autore intendetle di qu
razions da guerra, che usavano gli antichi dette Te/udines; nelle gi;
no foo alle mura, reggendosi fulie spalie gli uni gli altri, e aiutandofia m
tarui sopra, coperti turu di feudi, € terran iteme per ripararsi da' colpi, che
si (cagliavano per di sopra; E questa operazione s' addimandava refixggine spe
ché stavano col capo, € 'colla vita dentro agli icudi, come stanno le
(in Lp. torragas in Beanz. ortaes ) dentro alle loro scodelle 3 le quali )
dette da' quei dello stato di Muang, come racconta il Ferrari bi/se fo
bijce (codellaie, perché anno 1i capo di bilcia, e stanno rinchiufe cone i
della; Onde potrebbenfi dire, dom:porte, come un' antico Poeta chiamé le chien
de. Autione famoso ceteratore e fatto parlare da Pacuvio così, delcrive
tetluggine con que' versi portati da Cicerone de divin, ub. 2. Q@madrapes 1am
da, agreftis, bumilis, aspera, Capite breui, cernice anguina, adjpettu trad
ruche,¢ BR2uhe, sovo voci usate dai Caro ne' Mauiaccint; e il Veneziaiol
chiama Gv/ane dal Gr. Chelonei, da noi si dicono anche butte seodellaic,
BAST le NO Seré, Celebre, e nouttime scrittore d' archucuura.
EbDIF/Z/0, Preto largamente s' inteuce Ogni sorta di faborica, €
ma preso ttrettamente vuol dir faia, ec, Case, ed altre niuraghe, |
ades, @ facio; ed in questo andiamo uniti co' Latini, che per earfien
no ogni sorta di scrittura. Gio, Villani t, 128. Pauose/f ad ascdin, 00, o
difici, e per cane per forza ebbe, Li lib, del conquiito, Per joraa a

gE Es PSs SEES. =

'> ie P

= Fo



NONO CANTARE.

1. Capiteli, e frontejpizi,, Columnarnm capitula, o fronts bespitii, >
(ATT H Srglie.s ¢. aui, Stipi sono le pietre de i tianchi,¢ foglie quel-
a parneey quelle dilopra, che tutte insieme formano una por-
a» Suipice dal Latino :pes.. Architrave; quasi trave principa-

: « Quei ruotoli di legno, che servono per facilitare lo strascico de i pei;
atini li ditiero Palange, Vedi sopra C. 2, st. 65. Dichiamo: mertere une fal exr-
Spiguerlo a poco.a poco, e condurlo doicemente a fare alcuna cosa, La
Voce viene probabilmente dal Latino baiudare; questo aggiuttar' un corpo
}a un' altro in maniera, che quello lo porti con sicurezza. E la seconda
| Latino xmbdicus, cioè punto ne) mezzo, Bilicare quali ponere in umbitica,
ARGANO. Strumento, che servc pct tirar fu pefi in alto, che da huomini è
" moflo in giro per via di leve. Alcuni Latini lo dicono Sucu/e, i Greci oniffi, cioè
 Afineli:, e questo \& V argano,secondo il Filandro, cum axe iacente, quello pui cum
axe ereite, dice che in Latino e Ergeta, cioè macchina da lavoro; donde, o da
voce(lecondo il Baido sopra Vitruvio)è fatta la noltra Argano,
MSADATT 1. Scommodi; Non atti a esser portati, o Arascicati.
MC ATI, Meili in bilico-,-0. equilibrio., Latino Jibratis.. Diciamo.bilico
ofitura d' un corpo sopra ad un' altro in maniera, che posando quasi in un
non penda, o aggravi più da un lato, che dall' alo. L nostri Scarpellini
 dicono baggiclare per biluare. i
it. BOTT O porto, Si dice. Ch' è cb' € 5 colpo colpe sec. e 8' intende Spefiime volte

PAR* un venga, Tirar roba da alto a batio sopra auno, che sia foo.

“a ay STANZALXVI. STANZA XVIIL
a Le Donne anch? esse corron co' figlinoli, Chi, perché già non piglin l imbeccata
f i 2 dy che troxan, gettan dalle muray Cuopre i capi con tegoli, e mattoni,
o con la conca, o vaso da vinolt Chi verssa giù bollente la rannata,

a 9» Pighia a qualcun del capo la. misura; Che pela i vifie porta via i bordoni,
a8 Profuma il piscio i panni, ei ferraixoli Nei? olto un'altra intigne la granata,
yet Ne guardan vc v'é penail far bruttura, E fal asperges sopra i morioni,
ps Chi tira gi: unjastrone alic cerned, Altre buttan le casse,accio i soldaté
ie Che se ewe orili serva per murella, Partir si debban, poiché son cafjati,

ie ooNarraiil Poeta la difesa, che facevano queidi Malmantile, e descrive diverle
we" Operazioni militari adeguate alla composizione burie(ca di cutta. opera.
CONCA, Valo grande fatto di terra cotta, entro al quale si fanno i bucati
Ke ASO da viyoli, Sono vatetti di terra cotta simili alle conche, ma piccoli, en.
| 80a! quali Gpongono vivoli, cd altre pianterelle d' erbe, o fiori. Dice che.con
v — gucfi pigliano la misura a.ijcapi, perché hanno il vacuo capace della tetta d? unt
Td huomo; al quale quando i Cappellat voglion pigliare la misura della testa, metto-
u ~—-'NO in capo un tappelio; € ceftaco di Malmanzile per pigliar tal milura, in vece
sso un cappelio., mettevano-un valoda vivoli: e cosìscherzando intende, che ti-
@ — ravano (ule tefte a i soldati di Baldone i deni vali.;
@ \SEvi dipenail far brurture', Se\vi e pena il fare sporcizie; Dice che tirano fino
Dorina, e non guardano,-se. ciò sia proibito,: e con questo dire, accenna il co-
ef flume, che e in Firenze a” affiggere alle muraglic dove non si vuole, che fien fat.
r te



432 MALMANTILE

te sporcizie, certe tavolette di pietra, nelle quali \& scritto il
flrato degli Otto, che proibisce, e mette la pena a'chi fa
niuno si posia pretendere ignoranza; Ed intende anche di
¢ grave pena, che è in Firenze a buttare dalle finestre nel
torno a' quali dispone anche la ragion comune, come si vede
De his, qui deiecerine, vel effuderint, ' '
SE v' ¢grilli, Sopra nel C, 6, st. 22. dicemmo, che grille si cl
cosa palla, che si tira per segno, giucando alle palloctole; ed all
strelle, qual giuoco dicemmo come ti facia sopra ia detto C.6.t,
rché tirandosi, or qua, or la alla ventura, o alla volontà
a il falto del grillo, che dopo un breve falteilare si ferma, e-poi
-dicesi ancora Lecco, quali i/ex eMurelle chiamanfi anco
nelle sue Rime. orate
Ch' io do sempre nel lecco alle murelle OP R
dal Toscano antico eora, che e lo stesso, che il Latino Moles }ép
si dice di pictre. A'awer la resta piena di griili s' intende uno, che ha capric
vaganti; ¢d il Poeta scherzando'con questo equivoco di' grille dice
quelle laftre a' grilli, che sono neile tette di'coloro, come se piocatietd
strelle, o murelle. Dal pazzo similmente,¢ curioso faito del grillo son detti
icapricci, e fantasie stravaganu, che faltano in capo, e per così dire
PIGLIAR' un' imbeccata, Infreddare: B diciamo ancora: Pigliare df mitt
caffrone, perché il beccd, ed il castrone hanno una tal raucedine, che
pre, che cofiano, appwato come fanno gl infreddati.
Té£GOL/, Pezzi di terra cotta adattati a coprire i tetti delle case.

ap

HlAe.
: RANNAT 4, Liscia forte; che è quell acqua bollita'con cenere; ¢
dalla conca, quando si fanno i bucati. Lacino /ixininm,
BORLON/, Inteudiamo quelle penne, che non del cutto spun
scorgono dentro alla pelie degli uccelli, e per similitudine intendiamo il)
spunta nella faccia degli huomini « way
FAI alperges con la granata, Diciam far ? asperges quando con spugha
tra cosa si (pruzza acqua, o altro liquore,.a minute stilie; la qual cosa il
chiama e4/pergere, qui dice, che spruzzavan' Olio con le pranate;
aiciato un mazzo di scope, o d' altro simile adattato per (pazzare,)

stanze.

SOLDAT! caffati. S' intendono quelli, che sono stati pri
la milizia, perché cafare vuol dire cancedare: Ed il Poeta
guivoco di <afaté, cioè percotli dalle cafie', dice, che se son
nou dal Campo, perché non son più nel numero de” float,

SLANZA XIX,

Vi? altro con un gatto vwol la berta, Ed il primo ch' et trova
Legato il cala,ond' es fra quei.d'Vgnano - Che dou'ticbiappar
Sguawnialugna, econ la bocca aperta

Griaa ina/prio in sue parlar Soriano

oF

ee ee ee ee eT oe eS Ol ee

=~ aw



NONO CANTARE: 433

arnt) re Bie XX, e
Miagola, e soffia it gatto, es' arronciglia,
Ed Gite endian heeree
janes quel che oa " trattopigla
Beli è miracol poi se pite gli scappa;
thie oat peter tee cos riglia,
jie Lo tira fu con qualche bella cappa,
a «Ci qustcheciarpayo qualche pinacchiera,
ye  Ecosi gli riesce di far fiera,
ame cool (STANZA XXL
due Quand una volta lasciale calare
ib oi iaers al buffo di Grazian Molletto,
Che fu;di posta per ispiritare,
«Quel pelliccion vedendo intorno al petto,
we Le bestia intanto falta, e dal coliare
'hoe "

=

Tutto prima gli straccia un bel gigisetto,
fet  Dipos si lanciaye al capo se gli ferra,
ebst  sì che il cappelio gli mando per terra,

STANZA XXIL
Non.sa Grarian, che Diauol si sia quello:
Pur tanto fac' al fine ei se ne sbriga,
Ea aiza il vif per farne un maceilo,
¢Ha vedendo il rigiro, e ch'ei s' intriga
Con dame, vuol canarsi di cappello;
Ma perch' il micio gis ha tolto La briga,
La Dama accsuetrata, anzi civetta
Lo burla, che gli è corsa la berretta,
STANZA XXIILL
Ed ei, che da colei punger si sente,
Onde al naso lo stronzolo gli fale,
Perde il rispetto,¢ quiui si rifente
Con dirgli, Atona merda, e ogni male,
Vain questo al aria ungraromar digeete,
Che 4 terra scende a masse dalle scale
Fiaccate,erotte ach'elfe dagli /prazrolt
'Di pierre, c' ancor grattana § cocuzzoli,

oa Continova il Poeta a narrare gli accidenti, che (eguono nell' aflalto di Mal-
ie mantile, e dopo haver detcritto una Donna, la quale con un gatto legato a un
" i miazzacavallo andava levando rcba da dosso a quetlo, e a quello, come segue a
ol Graziano Molletto ( che e il sig.\ Conte Lorenzo Magalotti ceicbre per aobilta,
HF 'e dottrina ) dice che le scale degli AGalitori furon rotte dagli Allediati: e con i
r sassi, e con altro, che tiraco di sopra alle mura, dava ancora addosso a i soldati.
at IL (a berta, Vuol la burla ( vedi sopra C, 4. st. 47..) onde shertare, lo stef-
4 fo, che beffare. [i Davanzati ped dite Swerrare nella (un traduzione di Tacito.
mY Corte poesie senza antore, che fuertavano le sue crudeid. Se bene in questo luogo si
; poirebbe intender per berta quello strumento, che serve per ficcare i pali ne i
ea pfiumi nel far le fleccaie, che e un gran ceppo di legno ferrato, il quale infilato in
“ln pernio, o ago di ferro confitto sopr' alia testa d' un palo, s'alza per via di fu-

ni, e si lascia ca(care sopr' alla testa del detto palo ( già fitto in terra) per fario

sf andar pita drento. E perché in questa medesima guila faceva Colci coi gatto,in-

yo teade, che defie così /a berra, eruendosi del mazzacavaljo, che appretio gli an-
ti"  tichi era usato per arnese militare, come s'è toccato sopra C.6. st. 86. In propo-
i)" fito di Berta per Bxrla, il Ferrari dice così: ognuno poi la creda, come gli pare
4 f verisimile, Dopo aver detto, che que' delio stato di Milano chiamano Berta
8 ta Gazzera, e ciò dal balbettare,ch' ella fa; foggiugne; (aoniam autem fanne,
gil! At que irrifionis [pecies est aliena verba imitando reperere,inde Berta pro Inda,ae derisione
gi accipitur, o fare una berta illudere, \& decipere. O pure finalmente e forte più
credibile, che venga questa maniera di dire dalla novella raccontata sopra nelle
Annotazioni alla St, 47. del quarto Cantare,
d\& —. SGVAINA I agna, Cava fuori' ugna, che tiene alcofte dentro alla pelle, la
we bed gli serve per guaina, ed il Poeta scherza, dicendo /guaina U' ugna. lock que
gnano

Wo + Appropriando benissimo uns, a Vgnano.
yd 4NASPRITO. Incollorito, meflo in ira, in stizza, in rabbla. Latino exa/~
lii IN

peratis,

i



54

IN parlar Soriano, Cioè oer gatti in ling
si dice quello, che ha la pelle di color lionato serpato d
ché si dia in altri animali, o in panai, non si dice foriana; se
perché i gatti di tal colore fien venuti di Soria, come ai
di Persia quelli di color di topo portati da Pietro della Valle,
chiamati Persiani, o per Persianini. 'one

DISERT A, Cioè ttroppia; concia male. Guasta.

VVOL ievarne il brano, Brano dal Latino barbaro mn.
il pezzo, Vedi sopra C. 6. st. 47.

MIAGVLARE, o ignaulare. Bi ii

I gridar de i gatti; + il soffiare dic
quello strepito, che fanno aprendo la gola, quando fond in rabb

S' ARRONCIGLIA., si torce in sse stesso, come fa la serpe quan
viene da ronca, roncola, ronciglia; specie d' arme; o più” 2
agricoltori, ed è fata come una spada, ma è torta in cima a guisa d
serve per eflirpare i pruni: o pure da Ronciglio, usato' da' Dante per graf
fauto a uso d' uncino. i "

E MIRACOL 8 egli scappa, E cosa soprannaturale, o imposiibile,
degli artigli, £1 Petrarca. soe eee
E cio, ch' in me non era

Mi pareua un miracolo in altrui
cioè una cosa, che non potefie stare. '

LO tiene in brigla, Cioè 10 maneggia bene, facendolo operat

CLARPA, Dal Franzese e/charpe, banda, bandiera.. Quel draj
tano i soldati cinto:de' soldati era proprio il cintolo, onde cinguoie fol
dalla milizia, Vedi sopra C. 5. st. 33. 5) ie 7

FAR fiera, Buscar, o acquiftar roba = per esempio ends pirando per
torni, e chi gli dette pane, cht voua, chi una cosa, e chi un' altra tanto,
Satta un poco di fiera, se ne tornd, mn.

D1 posta, Subito: Di primo tempo. Vedi sopra C. 7. st. 92. BY
giuoco di palla, che si dice dar ai posta quando si da alla palla, prima
terra, ed è il Latino ilico, e vefigio, Gli antichi dillero: Di colpo,
fo, che di Borto. 7

FV per spiritare. Hebbe un grandissimo spavento, o paura.

GIGLIETTO. Specie di trina con punte; così detta, perch ha
col giglio.

Avr. Cioè gnell' ordigno, col quale la donna alza, ed ab
Vedi sopra C. 4. st. 69. Se bene \& può ree la voce rigire nel
mo sopra C. 7, st. 41,, ed intender, che Graziano, alzando il ca
giro, cioè la donna, e dedurre questa opinione da quel, che soggiung
Vedendo, che s' intriga con Dame,,

ACCWETT AT A, Afiuta; Sagace. Tolto dagli uccelletti,
civertats, quando havendo altre volte veduta la civetta sono dit
non si la(ciano lusingare a volarle attorno, come fanno quelli
mai più veduta. ae

eANZL cinetta. Più toto troppo ardita, e sfacciata. Si dice'

, eel ee8 TR

e— lL eBeuwe ete

— 7 ate

= ~ - — =



x A juomo da poco, però con tale equivoco

NONO CANTARE: 435

vane troppo ardita nel trattar con gli huomini, quasi faccia con essi, come la
corerasss gi uccelletti, che cerca con gli suoi gesti di tirargli a se. Vedi for-
to in questo C, st. 60, E Plin, lib. 10. cap. 17.;
CHE gli e corsa la berresta; Che il gatto.ha fatto preda, e gli ha portaro yia il
ppello.. Ma perché, La/ciarficorrer, pee via la berretta, vuol dice Elicre
mentando G h diveherch iandnban tone
raziano womo da poco dal veder, che si lascia rubare, € portar
via il cappello, gli an burla; di che egli s' adira, perché si sente fete
if r¢ dall' etiere burlato da que(ta donna,
-, GLI fale lo frronzufo ai naso. Derro sporco, che significa entrare in collera, ma
= poco usato, dicendosi più tolto fair la muffa, o la fenapa, o la moftarda, o it
herimo, ec. Vedi sopra C. x, st. 39. Bil Lalli En, Trau, C, 2. st, 65,
Waapn 6 Airs Corebo un tale strazio,e tanto,
i Con la moffarda al naso,e nol comporta,
AGli Ebrgi.colla fiessa voce significano, ¢'/ na/o, e ira, perciocché par, che qui-
¥iclla particolarmente rifegga, siccome disse Teocrito + acris bits ad nafum fedet,
Onde noi dichiamo Arric¢iare il naso per isdegnarsi; simile in parte quel che dicevano
gli antichi Leware il miffo. La voce Ebrei fie Aph, in Siriaco Apha; onde
. itorcec: e venuta la nostra 4fa, colla quale a ete una cosa fomi-
gliaptitima alle vampe dell' ira; cioè un vapore, e yn caldo fallidioso, e affan-

HO!»
t 'sop SLrifenre. S'adira: Entra in collera, perché e burlato,
pjat 'A merda, Detio ingiurioso usato fra le donne di vil condizione, e del-
Ta voce mona vedi sopra C, 5. tt, 18.1 Lagini similmente (asum, conum, frerquili-

me,

. FLACCATE. Spezate, Fiaccare \& verbo proprio per esprimer, quando un le-
£00, o altro. materiale si rompe in mezzo per fouerchio pelo, Latino fari/cere.,
 springs. Donde poi bxeme fiacco vuol dir huomo affaticato, e stracco; se bene \&

ver) imile sche venga dal Latino faces, faccidus, dichiamo, fiaccare |e braccia

A uno, clive infragnerglicle, e romperglicle colle bastonate.

SPKVZZOLARE. Vedi sopra C, 7. st. 15. E qui è detto ironico, ed intende

f Bingge pict 7 '
V2ZZOLO. Latino vertex, cacumen. La parte di sopra del capo dissefi an-
she. Zwecolo, siccome da Cocuzza de' Napoletani ( Latino cacarbita ) e si dice an-
Gora. comiznole, se bene questo e proprio delle fommita de' tetti, e de' camumini;
dal Latino cudmen quali culminnlum «
ares ST NZA XXIV, STANZA XXV.
Chi con, chi per banda,.¢ chi fupino Quantungue il.campo annaffi tal rugiada,
i se ne viene, e fa certe cascate, Come le zucche, annarpican le [cale,
Che manco ie farehbe un' Arlecchino, Onde più a' xno in già versala strada
, Quand in commedia fa le sue scalare; Fa pur di nnono un bel [alto mortaic;
sì che, stinnanzi fecero il fantino, M44, piché ammonti ne traboechije cada,
Le brache in fasti glieran pui cascate, Sardonello [2a forte, e in alto fale,
> B infranes, e pefti andando gis nel foffo E trai mimici al fine a lor mal grado

Mette [u il piede,e agli altri ye Uguado
2 PAN.

, Hanatolere a quchlo nuove scate adddfe,,
geet ii



436 MALMANTILE™
STANZA XXVI.
Chi vidde in un pollaio, ove fisrona | *
Un numero di polli senza fine
Tra lor cascar qualche pokafira nuona,'
Che roft addoss' elt ha gullie galline
Ciascun per far di lei l'ultima'prona 5
Eye aa folelapariea athee, 1
Che la difende, e da beccar (e porta Ma Eravan, che
Stroppiata rimarrebbe, e forse morta, Aiuto a un cempo,ed
Rotte le scale coloro, che erano sopra di esse cascarono nel fofla
r0 corpi furon polate nuove scale, in fa le quali intrepidamente
neilo falto sul muro, e scel nella Terra, dove fu da mojti di quei
falito: Ma Eravano, che lo vedde in pericolo d' esser ammazzato
¢gli dentro a dargli aiuto.

BOCCONIL. Dittefo in terra, o altrove con la pancia, e faccia ve
no, Lat. pronus contrario di Sapino, fusse reni; Lat. fupinus ye Per,
la doppia posicura che resta, diversa dall' una, e dal' altra, la diciamo 4 x
Per franco,¢ Per latv, Lat. in latus. Bocconi  detto colla stessa forma, che!
nocchioni, Brancoloni, Saltelloni, e simile, che si -dicono anche Boccone
vhione, ec, anzi questa ultima maniera è l' usata dagli Autori antichi Ti

eARLECCHINO. Va secondo Zanni, cioè un servo.semplice in
Così nominato, il quale faceva assai bene le scalate, che son quei giuoc
Ai suol fare detto Zanni in commedia con una scala a pivoli, sopra alla
affaticandosi di voler falire, casca in diverse manicre. f

FECERXO il fantine, Pecero il bravo, l' ardito, il coraggioso, Si
gura. Egli e fantino cioè persona, da fare questo, e altro, Fantino di
faate. Lat, infans, cioè Ragazzino usato dagli antichi in generale, @
oggi a un significato particolare. Chiamando noi fantini quei R i, ¢
pr' a cavalli spogliati corrono al palio, Si dice anche fare if Baiardina, da
lardo celebre Cavatlo di Rinaldo Paladino, così detto dal suo mantello,
yea essere Baio accefa.:

GLI eran cascate le brache. Gli era entrata la paura addosso
animo. Vedi sopra C, 6. stan. 20, Lat. aninsum desponderant,

ANNAF SI tal rugiada, Annaffiare vuol dire Ammollare, o af
giada vuol dire quel che accennammo sopra C. 2. stan. 55. alla voce gr
Ma qui da nome di ragiada a quelle pietre ec, che buttavan già gli

-dnnafiare detto da Adacqware, che si dice anche /anacquare,e Annacquare,
Ui duc ultimi verbi diconfi propriamente del remperare coll acqua il vino;
equare propriamente e dare [ acqua alle piante. + la
INARPICARE, Aggrapparsi, forse dal Gr. herpein chet in
Pere, reptare, Salire in alco, appiccandosi con le mani, € co' piedi,
no i gatti. Si dice anche rampicare sopra C. 4, lan. 68. ed-«
vedremo nella seguente ottava 28,:
SALTO mortale. Chiamano i Giocolatori falto mortale,quando
tecra Con le Mani', o con alcro faltano, voltandy la persona fo}

> eran

WEITAF.

wee Se peg RRFESZLTE=



NONO CANTARE, 437

verifimilmente facevano coloro, che ca(cavano, o erono gittati da alto 'a batfo.
) TRABOCCARE, Intende precipitare, o cascare da alto a baflo, rompersi
la bocca; andar colla bocca per terra. E se bene il proprio significato di trabuc-
“care è quando mettendosi in un vaso maggior quantica di liquore, o d' altro, di
PS yche possa capire', casca dalla bocca del vaso quel, che vi e di più; onde per
figura si dice'un Trabocco di sangue, ec, tuttavia si piglia ancora in senso di calca-
te. Traboceo ne i vizzi, ec).
hie = ROMPE il guado. Apre la strada, o il passo. Ovid. de arte amandi,comandando
'ex che si rompa il guado per via di viglietto, dice: Cera vadum tenter, Guado vuol
s dir quel luogo ne i fiumi,per dove si può passare senza navilio, che si dice guada-
ve; Eda questo guadare, o rompere il guado s' intende aprirsi il passo in qual.
“Voglia occasione, o congiuntura.. Parrebbe che fletie meglio vado dal Latino
mis » siccome si dice ancora volgarmente il porto di Yada, dal Lat. Wada VYo-
 taterrana; perch così Gi fuggi V equivoco di guado (pecie di tintara, ma
ivell quelli stitichi, i quali si vergognano, che la nostra lingua sia aiutata dalla sua+
frit madre Latina,non ci concorrerebbono, e darebbono una turbativa a chil' usaiic.
hist = MANDAK 4 Purafso. Par morire; E perché significa il medesimo che man-
aoe » o 4 Scio credo che derivi da i foccorsi maadati in diverse occafoni,
| “tempi ai detti tre Juoghi, da i quali non essendo tornato veruno di quelli, che
al —andarono, quando si vedeva mancare uno in paefe, si cominciafle a dire. Eel
stl e andato a Buda, a Scio,0 4 Patrasso; per intendere egli € andato in luogo, don-
de non tornera mai più, duc, unde negat redire quemquam; e s' intende egli è
i = Morto. Vedi sopra C. 5. itan. 13.
j TIRAR l'ainxolo. Vuol dir morire, dalle cunvulfioni della persona, che pa-
§\&  tilcono quei, che si muoiono, Aixslo è (pecie di rete da pigliare uccelli. E la for-
2a, che fa ' uccellatore nel tirare l' aiuoio, o simil sorta di rete, e deferitta da
id Petro de Angelis da Barga in que' versi +
0! Tum vero innitens pedibus confurgit,\& omne
Intendens neruos magno trabit impete funem.
4 ZO scorge debito. 1.0 vede in pericolo di morte. '
STANZA XXVIII. STANZA XXIX,
1 ' Chinmgue è 'n Castelle allor pien di paura - Auitiene a lor ne pri, ne meno un' iota
% Corré per far o auanti et prit non vada, Com' ai fancinlli, quando per la via,
Fan la tura at rigagnol con la mota,

«RB memrtil vuol rispinger dalle mura,
' “\ Ch altri più la 2 arrampica non bada; El! acqua ne comincia a portar via,
| | itr db ouniare anco di gua proccura Che,mentr' affodan quixi ov'ellaé vota,
| Main fete lui ghit ged farts la frrada, Essa distende altrone la corsia,
, E se riparan la, prt qua fracassa,

| E a cogs intorno tanto il popol cresce,
C" ogni riparo innalido riesce. Tal ch' ella rompe,e a lor dispetto pa/sa,

«\ [Soldati di Baldoné superate tutte le difficuita, finalmente entrarono in Mal-
o mantile, éd il Poeta paragonando questa cacrata ad un' acqua corrente, che rom-
 pe, € passa ogni oftacolo, che le 4 pari avanti, esprime I" inutil difela, che fan-
“no i Terrazzani.
ARRAMPLARE. E' lo stesso che inarpicare detto poco sopra, ed è il Latino
Perreptare.
VN

™"

a


438  MiALLIMIAN TLE Bot

VN ita, Un niente, detto sopra C. 1» stan. 18.)

RIG AG NOLO. Diminutivé di ome, Piccolo riva,
è proprio per intendere da parte più: bafla, che e nel 0
di Firenze per dove scotre l'acqua's che piove, efic 7
intende nel presente luogo 4 € ¢' aacenide comuacmente s che un
rigo, o rio diremmo rixolo o ra/celloy dewro così da Riuiceday la
presso alcuno antico. Se bene Dante nell' Inf. C, 1g.dices Ed
Sente rigagno, ec. ed intende quel fiatnieell@, o rivos il, 0
nali. Li Varchi Stor. Fior, libro 13. Commiciarono ad nscar fuara
e che i rigagnoli correuano, ele ve erat piene di motayedifarge rt
Nov. 16. 4 rigagnolo della qual via corre, chepare un fiumicclian |

MOT A, 'Lerra ben inzuppata acl? acqua. Ai Percariz, Lupum
 immora, Per intelligenza della \iuddetta comparazione e ince
i ragazai dell' 1afima piebe di ae 'sogliono per loro pa
dopo la pioggia (corre l'acqua per detti rigagnoh pigliate del
ond ae come un Danian opposte ai corso dell' non
passaggio al fume, e questa chiamano la twra.; ma fiocome d'
quel iuogo sempre va crescendo,)così 0. per 10 pelo, rompe
bondanza traboccando la superay e pada via noa oltaace dri
v' appiichine, come dice il Poeta. Qunero nell' Aliads ib, a 5s,

De! Troiani fereci allagranturbay...
dt folgorante eApollo andanainnanze
Tenendo in mano il preziufo fondo:

Ei degls Achini il muro aterra Sefe;
Ne coffogli fatica, appunto.come
Lungo il mare il fanciulfacoll arena,
Che poich¢ fabbricato ha per. suo gsoco
Va gentil fanciullecsco alto lanoro;
Colle mani, e co' pie scherzando il guasta,

A lor dijpetto,, Contro alor voglia. Lat. ijs innitis, Il Boce, disse
Per di(peto. A Dante prima, e poi al Petrarca ia uecedlica della rl
il servirli della parola De/picto accordandosi in ciò, siccome ima
col dialetto. Provenzale, o Francelco. Virg, ecl. 2. Despectus tibi Jum ne
queris, Tu m' hai in dispetto,ne ti cale il sapere,chi io mi sia, Confiache
la strada, che è.per il mezzo della galera; onde que) groilo Canaone.
diceli Cannone di corsia, S' intende ancora per la correate dell' acqua..

FeSlFaer- ae

eet

it Se OR Peas aw

STANZA Xxx, opqas ae a
Già tutti son di sopr' alla muraglia, Celidora a due man 4
Che la circonda un lunge terrapieno; Che ne-anche un vi
Già si fiorisce in si crudel barcagha Tanti fil d'erba gol

Di sanguinacci la gran madre il seno: Lane' buomini così


4
NONO CANTARE:  439.
o- - STAN ZAXXEL oo. STANZA XXX.
ee, jth Amiffame —. — Adafa di Coccio a questo,e quel comand,
Da toccatori fan col brandispocco, Ed all'un dane,e aun'altronepromerte,
d h Lacompagnia del Furbainnanci mada,

“Pere che della morte almen Ceffane,

'Se non prigion si fa chi è da lor tocce, Che refti ai fianchia Batiston commette
AIP incontro ritrovafi Sperante; Com Pippoyil quale (Pa dal' altrabanda,
WA) + Che fa menando (a sua pata, il fiocco, Ma egli imretreguardia poi si mette,
Wh E se già le fuftanze ha difipace, E mentr'ognun favanza agloriasmtente
a Ei fiede a gambe larghe,¢ si fa vento.

Hor mand'a male gli buomini a palate,
+ Essendo già wtci i Soldati di Baldone faliti sopr' alla muraglia, e padati oclla
PS di dentro si mettono alla difesa, Sinarra la bravura di Celidora, di
y edi Amostante, s' accenna 1l valor di-Sperante, !a diligenza di Mafo
S eraccc pane wragtoe ut Coot cies A
La gran madre si se i sanguinacci 11 seno » Ci terra s'asperge di sangue:
88 Ounero nell” Lliade (petisind « =:
pm 8 di sangue la terra intrifa corre.
® La Gran madre per la Terra intese if Petrarca nel Trionfo della Morte.
elf SEG ORY O ciechs, if tanto affaticar che giova?
jeu 08 Tutti tornate alla gran madre antica 5
pone E'L nome vostro appena si ritrova.
  TOCCATOR?, Vedi sopra C. 2. tan. 60.¢ C. 6. stan. 44. 3
 “ BRANDISTOCCO.. Specie a' armein asta; simile alla picca, ma l'asta più
corta, ed i ferro più' largo, ¢'pily lungo, che non e quel della picca; e credo
venga dal Tedesco froch, che vuol dir battone, € brando che da' Pocti Eroici mo-
derat si prende per Iipada, e significhi Spada in sul bastone.. Stocco e dal Greco
Felechot Lat. Pipes, candex, da cui è facta anche la voce feecco,  perciocché pri-
ma per battersi si adoprarono le-mazze, e poi si venne a ferri;( Orazio Serm.
1.1. Sat, 3. Vaguibus o pugnis dem fuftibus, atque ita porro Pugnabant armis » que
'pelt fabricaverat nfus i nomi potleduti già dall'arme di legno, furono ereditati
'dalle arme di ferro, che a quelle succederono. Onde Stocco, che in Germanico è
baitone, a nOi significa /pada corta, e floccata ia ferita, che si da con quella. Brand
* jn Saflonico e riz one, o fuoco; onde Brandispoccbi poterouo essere cio che Virgi-
“tio lib. 7.¢ 11. chiaina /fipires, o /udes pranffas, ovvero obuftas cioè bastoni, 0

mazze appuntate col fuoco. 3
' CESSANTE. Si dice quel debitore, che essendo stato toccato da i toccatori

“può esser fatto prigione dopo le 24. hore da che è lato toccato, ( del quale ato
me rt e. (a 60. e C. 6. stan. 44.) ed il Poeta scherzando coll'

'Paclammo sopra o.
egnivoco toccare, cide esser percosso; dice che quello, che da costoro è tocco di-
viene almeno Cefante della moree, se non prigione, ed intende che quello, che da
costoro è ferito o muore; o resta vicino al morire, com” è proto ad andar in

Prigione colui che e tocco « ' <

FAR il focco, Fioccare vuol dir quando nevica gagliardamente, € da questo
diciamo fare il fioceo per esprimere un' abbondanza di che che Ga, per esempio si
fa ii fioece delli uccelli, o de' pesci, o de' denari, ec. si direbbe a uno, che pigliaf.

se molti uccellt, molei pesci, o molti danari, ¢¢. \& così nel preteate luogo inten-
de

%

a



440 MALMANTILBE

de che Sperante ammazzafle molti huomini con
il vello della lana Lat. foccus., Si trae anche come's' ¢detto
ve, che Marziale appella tacitarum vellera aquarum, La
in abbondanza, si dice Fioceare; e stendefi anche r
aver dewo di Mcnelao: Poco dicea, ma bene, viene a dire d'
Atandaua fuor diluvi di parole, '
Come allor che di verno ilnembo fiocca,.
E fu pe' monti nena a! ogn' intarnos dohlgioee
MANDAR male a palate. Vuol dire mandar male il fay
gamente ed inconsideratamente. E qui ii Rocta ia Spe
vendo havuto per costume di mandar male ii tuo a 0
l'antica ulanza di mandar male a palate ancora gli huomini, e d
con quella sua pala, concia male moltihuomini, '
A chine dd, ¢achi ne promette. Diciamo così d'uno insolente
che tutto il giorno facia risse, perquotendo quand' uno .< quand"!
con questo dettato il Poeta deferive la,natura. di Malo di Coccio, il
s'è detto sopra al suo luogo ) era huomo di conversazione,¢ nelie tel
ordi, ne 1 quali si trovava, foieva vOler (empre sopraftare gli aluri
¢ ca Cf farsi ubbidire con le grida, e tainolta con ie butie,
 gambe larghe. S' esprime con questo termine la commeaita, e (pe
ginc,con la quale uno ficce a pigliarh riposo; (e si dimotira un pimuo ¢
sare, ed amico dell' ozio, e delja pigrizia) che si dice: Stare iw Rane
C, 3. stan. 72, € C. 3. stan. 1, 60m s¢ mani in mano; Con ie mani in cintola, —
STANZA -XXXiLL STANZA AARIV,
Amostante alt incontro un nuoko eAarte.  Vedendo i Terrazzanigbe stannoin fa
Senbra fra tutti anants alia testata, Che il nimico ad S[padeye gioca
Lo segue Pao C orbi da una parte s Ler non far Mole 1H fab MALCON KG i
E aa quest aitra Egeno alta franceta, Ritsranfi, e non sengon più
Vengonsi in tanto a mescolar le carte Ma speron ben ( moftcanaoa
E vien /pade,ebaston per ogni armata, Denari,e coppe)indurghs a far p
Ectidam puche,e 4 gsmocar none leflo a) si
Vs perde ia figkra, e fa acl r¢sto. Speaiscon, che pario in
eile preicnu due otrave il Poeta dopo haver lodato per vaiorolo
seguicato dai Corbi, e da Egeno, icherza in sull' equivaco del giuoco, \&
sucne rai as/corso dai proverbio « Vengonsi a mescolar le carte,( che
€ \¢ ne Locca, O se ne 1iceve, Come vedremo sotto C. 10, Ble
auibedue 1 campi vanno ( cioè s' adoprano ) /pade, e ha/tom, e che chi
che ( ive urta nelle picche ) perde /a figura (che € una di quelle carte, nell
Ji sono efhgiaui ques fantocci, che ne 1 giuochi di daia tono te carte,
cive perde la propria perlona,e fa del refto ( cioè muore ). £ Terr
in fors, C1U¢ hanno i lor punto in fiori, ( ed intende tanao ip
Bria ) vedende che 41 nimico ad /pade ( cioè adopra ic ipade). Per non,
+ maitom: ( cloe per non fare un monte di mori in iu 4 mattoni, e ¥
fui terreno.) ff rtir ano da chore ( cide lasciano J' ardire,) me tengon
Vuoi dike HU VoOguon più giuocare, ed intends non voglion più

gs gp emer Ee ae PP ee EsP soo eee eee FEE



« NONO CANTARE;

ano di ridurgli a far partite, cioè accordarli, mostrandogli

44t
i ddwari', e coppe, cine

 ofterendo loro dell oro: E pee questo mandano al Campo un' Ambasciadore,

che parld nella maniera che se
 STANZA XxXxy.

 Spida Signori ? armi ognun sospenda,

Ache far questa guerra aspra,e mortalel
Fermi per grazia; più non si contenda,
Per c! alsrimenti vi farete male.
Fate che la cagion aimen s' intenda,

| Ca cherichedi a questo mo.non vale

F

ni

it

ee

| Bchi pretende venga con le buone,
Che dara glifard soddisfarione,

'ntiremo nelle seguenti ottave.

“STANZA XXXVI.

Con queiyche dona per amor non s' nf4
4n tal modo ta forza,e la rapina,
Chiedere,imperciocch? giammai ricufa
Ui ginfto, ed st douer la mia Regina,
No entraron mai moschein bocca chinfa,
E con chi tace qua non s indonina ?
Poss' egli accomodarla con danayi?
Dungue parlace, e vengafi ai ripari,

 L Ambaiciadore de 1 Terrazani espone la sua ambasciata, e chiedendo tregua,
-elolpenfione d' armi conchiyde che la Regina di Malmanule e pronta a dar loro
fodistazione, pero domandino, che faranno esauditi.
|. SPiDA, Questa è una parola usata da j ragazzi ne i loro giuochi fanciul-
ye non hay ( ch'io sappia ) significato nefluno universalmente, ma nel
modo, che se ne servono i ragazz: signitica sospenfione di giuoco, o permuffione
@ eleacarsi per alquanto da efio senza pregiudizio, appunto come si fa con la fo-
spenfione d' armi in occasione di distide o particolari, o generali, ond' io crede-
rei che si potefle dire, che questa voce /pida fusse corrotta da ssida, o disfida, I
- Fagazai si servono di queita voce così, per esempio. Wel giuoco de' birri, e ladri
detto sopra C, 2. staa. 32, quand' uno occa bomba o per qualche sua faccenda
on attenente al giuoco, vuol partire,per assicurarsi dal' esser catturato dice;
Spida, E con queita parola s'intende per lui fatta sospenfione di giuoco: E quan-
do il ragazzo, che è Ggnore del giuoco dice Spida s' intende sospenfione generale.
Ed il Poeta » che si ricorda che egli scrive una Novella per i fanciulli s' accomo-
daa i termini da loro praticati,ed intesi, facendo servirsi a questo Ambasciadore
della voce Spida per farsi intendere che vorrebbe sospenfion d? armi.

Cae hericheli + Chetamente; occultamente, senza parlare. Varchi St. Fior.
lib, 15. Per Ze case si facenano delle ragunate a chetichelli.

WON vale. Questo pure e termine fanciullesco, se ben talvolta usato anche
dagli huomini d' eta, e significa Non è dovere, Non conuiene, Non sta bene,
ec. Preso per avvenitura dal giuoco, in cui chi scommette dice per esempio; Va-
dedi tanto ? E quegli che non accetta dice: Non vale, cioè non fo buona questa
erate 90 pure quando si fa contra le leggi del giuoco, si dice similmente

NON entraron mai mosche in bocea chiufa, Chi non chiede, non conseguisce;
chi non parla non. inteso. Lo Stefonio nella sua Gnoccheide ato primo sce-
a prima dice,

hee pak Vulneris alcofti nunquam medicina paratur,

£ viene a fonar lo stesso che con chi tace, qua non ' indoxina, Plauto nel Pseu-
dolo Att. 1, se. x. ove introduce lo [chiavo, che così parla al suo giovane Padro-
Ae Innamorato,

Kkk Si



PrViot ovoy

E poi conchiiile:
ina fuggire i litigi.

dice Così

“STANZA AdXxVIL
A quel sl General,c' ha un pod' ingegno
Kusene il colpo,e in dietro si discofta
Che si fer mina i suoi, aipoi fa segno,
' Pala parola, e manda gente 4 posta,
Ne bado molto a fargli har a segno,
Chela materia si trove disposta;
Crascun a! amie le parci ferte faldo,
pC? ognun cerca fuggire il ranno caido.
STANZA XxxXvVill,
ch della pelle ha punto panto cara y'
ch che von vorrebbe esser nccifo
'empre de feiarre di fuggir procexra
~ BYe mai c entra, ha caro esser ditsfo,
' Ben ch} ei. mostré non, baner pasra
S? in quel Cimento lo guardate in vif
Lisciato lo vedrete d' un belletto
* Composto di giuncate, e di brodetto,

* Ordiaa i) Generale, che si fermi il combatteré, e trova i'Sol
dieatidivai » perché a ognuno piace il vivere; e sia uno'coraggiol
mai essere, al cimento poi non haura careftia di timiore. Fermato gue'
battere, Chi era ferito s' andò a far medicare e ah

PASS AR parola, \& termine militare, che significa far fapete |
Capitano' per tucco l'esercico con dirlo a uno, che'lo dica a un?
vada seguitands fiaché lo sappia ognuno senza che si faccia
Qi. Gli aatichi Capicani fa

fiziali fubordinati wa
si conteneva l'ordine di cio, ch

'di yoo! ior cl icvar qiaao da ij i

aes ie Hic Sen ae

eee UM AN TY Le Mi

v hominum parsi vem
6 °)\Nees te rogandi, o eileen ye x
Nunc quoniam id fieri non poteft
| Me fubiget, ut ve rogitem'; Fjord
Eloquere ut quod ego ne/cio, id tet '
“PVOS S? egis actomviarla con danari. Ci è egli modod
trovVar rant denaro, che aggiuitj questa ca

*''Dungue parlare, Quest' ultimo verlo pat tolto: di oda g
1, ove Teti patia al'iuo Figliuolo addolarato;

Parla; sis Wb habs sit digi by beds, ue

Tener la'cofd Wath tua'beentt ascofa'y mins aA

eAiciocché tu ae thee a
TA

oe

sa Pee

Dewo uli

1h te ada,

0G ease
Bao i a
ade

SE FLSFFSRPSR oS Staete

A
Sien: Projo brau,
Se mai vengono a
Crediare che elo fan
Perec' a rutei viene il bi
Ech ela palferebban
Se lo potesser far con tor
snithewsate i a quella opiniiie
Di veder Cuanro viner fa
STANZA XXWK
E questi che badauane ax
in Malmanty, 8 accorfe
Che que; none meflier
Pero si contetaron dell”
Gai tagle alcuno impi
Hitri rimette braccia,e ¢
Altri da capo a
Echi si fa uae ed

SEF

pest

a
. =.

ih
nye
aaa

- o = \& wo Z.

f es haat 1803 ae Udsarntare Teffera. Amminiand'



eee -

NONOPQANTA RE

} 88
'Siliodtalico,, eee etn ee ff
pases temmalumeaee €.con ording,.0 de Da by eat re
ea - oa Leib se sittin

ST ROV Oar niaseria,disposta. Veove. prontezza d! ubbidire » perché cialcur

- inclinava a lasciare il combattere. Sante eT ae
\ \ EVGGARE it ranto valde  Buggire i pericoli,o le fatiche, ~
| HA care eferdinifoHa caro che-qualcuno entri di mezzo,.¢ impedi(ea i
tocombatreresiche questo vuoldire diwidere una quiftione. Lac, pugnam dir e.
elLilcio Lateadiamo tutte quelle mefture,, con le quali aicune»
sper parce-bellefi lisciang. ta faccia, che diciaino imbelietrarfe: decto [:con-
do aleuaisda wRerlerra. cioè. melmay fango. In Franaefe il (cio dicesi Farò, onde
ciog unbratwace, e 44re ana farda, e una fardaca, il che figuratameate>
- Sluergognare uno.con mato pungente in pubblico, che alccimenti dice si; dur /a
 Ceretata, E. dare una cenciata fudices, ccacta dal costume de' Ragazzi Fiorentini,
che il'di di yuezza Quarefima, quando ( per usare un loro idvotismo ) si (ega la
eal cioè viene ad.ctlere partita per mezzo quella Stagione di penitenga;
Peete ior abufo,ednfolenza batcono el viso alla gente grotlolana, o fenipiice
pd al COntado.cenci intinti.nell'tnchiottro, o in altro fudiciuine.. Branco Saccheci
disse Dane ca Fare, e dare nna zapare, per offeadere coa marto. Vedi sopra, a,

ae Pilla 45.0: base wid Ca Te ya
jit «OM Ne ATA, Latte rapprefo,, e (errato in fogli¢ di farfara con giynchi,, e»
Gdecta ginncata, la.quale mescolaca con broderro, che e mincitra, fata d'
Wlovauidette liquide con brodo, o acqua,¢ agrelto, o fugo.di limone,,. farebbe
un color¢ fra ij, giallo 5,¢ il bianco, appunto come diventa ia faccia di coloro, che
\& i da subito timore, sink, 1
ASN AD/ERL, Huomini sanguinarij: Da Mafnada, che vuol dire truppa.
l'di Soldatic: what, militum manus Ma per lo piii intendiamo compaguia di ajiaii-
at poeaid Aieada.
TIRARLA

fuori, Cio.cavar fuori la spada per combattere. Virg. vagina.

VOISN Gx

aetkbEiks =

TREE

RESEE

er aay;.
= SATEICVORE. Eccessiva paura, e spavento. Dicesi solo dal frequente bat-
'eres che si fence dalla parte del cuore in uno, che habbia timore. Se bene af but-
ter del cuore e indizio ancora d' altre pastioni, che futte anno quivi lor, seggio;
“eae gran defio, congiunto colla speranza di vicino conseguimeato del defi.
rato bene, la quale pero dai timore, non è mai io tuto disgiunca,:
sualptelten'arctben 4 Jeegiert, Paciimente lascerebbono (tire dt far quella quittio.
'Re. ln un frammcnto di Storia, Fioreaciaa manoscritca, che dame oa tisfa di -
i ncarvi il principio si legge: 5, Gli difero un monte di villagia, ¢
ond 'ingiurie, ma.il Cattellano, che era di uci Soldat, che avg iano canioin
dt ight doula Cavalleria, se la palso di leggiers,¢ la Ciaaiogli gracchiare,
sgnattendeva.a star. deatro;.ed a i suoi Suldaci, che lo pregavag» a ulcire, e dare
vs, addosso.al nimico,, rispondeva; Lo noa vogity ultirs, percaé nog voglio cae
Se CEDER guurs [a vivere un poltrone. Con questo termine descriviamo 490, che
yuo brighe, ac faciche, o.pensiert, a¢ meno f yuule esporce.2 rifthi, o.pe~
SS Ree: ye oe

MSS ERE E

ae
ar eet

444 MALMANTILE

ricoli di sorta alcuna.. Il Ferrario seguitando il Salmafio nel |
le che la voce poltrone venga da Police trunco, dicendo che:
andare alla guerra si trova che si troncassero a posta da lor
dito grosso; B dovea essere usata tanto questa furfanteria, ¢
tali il soprannome, e furono appellati Azurci secondo che
Cellino lib, 15. il che.volea dire poltreni; poiché Murcia pret
mava la Dea dell' oziofita, e della poltroneria, Origine et
non la credo vera, stimando che la voce polerone venga pill: sto da
poledro, ( come alcuni spiegano quel be/fie poltre di Dante Purg. )
Poltrone a.uno, che non vuole, o non può durar fatica, appu 0
dro, il quale non è ancora atto alla fatica. Ovvero da poltro, che
secondo 1 Landino sopra quel patio di Dante Inf. 24. che dice

Hor mai conuien che tu così ti spoirre,

Disse it maestro; che feggendo in pinma

in fama non si vien, ne sotto coltre.

Donde poltroni gli huomini pigri  e dormiglioti, dice il

zione di questo patfo.

PREG Sk FS oye = oe

— meftiero da abborracciare, E' cosa da farsi consideratan t
caso,
LMPLAST R ARSI con le chiare, Medicarsi con le chiare d' uovo le ae
di sopra in questo C, stan. 4 A a Re
PARSI aar de' punti in sul cefs, Ricucired tagli, che ha nel viso,: quale cae 9 pe
ma cefo, perché guatto da i tagli, non merita nome di faccia. Cefe o Fran a
se € parola nobile, che significa Capo, come alcuai vogliono, dal Gr. gi grps mH
nol e parola di dispregio, e significa vifaccio brutto. ae 'a
STANZA XXxxl. STANZA XXKX ui
Baldane in questo per la più sicura * Et essi andaron con la lor patente tp
Due gran Dottori atrattamentiinuia, Di poter dire ye fare, € alto ¢
Lun Fitfolan Branducci che proccura Lor camerata fa tra? a
D' haver se non po in Pifa,oin Paxia, Che gli seguia curioso per. =
<ilmeno in refettorio una lettura, Baldino Filippucei lor yy
ZL! altro è Meinforcon da Scarperia, Huom, che più tosto canta py
ChefeVbuom vine per mangiar vi ginro, Crescer volea come gli altri appa e
Ch' ei vuol campar mill anni del sicuro, 3 44a si pent),quand'a e
. STANZA XXXXIL. STANZA XXXXIV, 9 \&
Calfandro Cala Cheleri fra tanto Son alti gli altri due fuor di mifar «
Del Duca allora il primo Segretaria Ond! ei nel me? o camm
“ 7° loro un discorso di quel tanto Refha aduggiato sv hed)
evan dire al lo aunerfario Ne men pro crescer pit
Cacciatof, Giosieiae: ar
Escorso turto if suo vocabolario

Scriffe in manierayefeceun tale Spoglio,
Che mese un mar diCruscain mexico feglio,



NONO CANTARE: 445

PRES os | HROMOVE TH ANZ AX X "BV. lov
ella pure alor quiui's'inchina, Purche il nome confervi di Regina,
Dando a ciafennoi fut debiti riroli., Luando per t annenire altras' intitoli,
Econ essi ferme IL altra mattina Che questons le nieghin, chiede al mato.
| Mdiscorrere, e far patti, e capitoli, Wel resto por da loro il foglio bianco,

manda suoi Amba(ciadori a Bertinella, i quali con essa fermarono di
stabilire i capitoli della pace per la matuna seguente, promettendo la medesima
| Bertinella d' acconfentire a tutto,pur che le retti il titolo di Regina.
DE gran Dettors, Dice due grandi, perché veramente erono ambedue di. sta~
a ce alta, ed un solo di essi era veramente Dottore, cioè Ficlolano Branducci,
ai che e Frdncesco Baldovini giovane dotto, e spiritoso; ma perché nel tempo, che
i fu composta la pretente Opera era assai difapplicato, pero lo motteggia, dicendo,
che egli proccura d' havere una lettura in un refettorio, se egli non la può otte-
 Berein Pifa; o in Pavia. Ma non voglio già io lasciar nelle menti di chilegge-
 fala presente Opera l'imprefiione', che questo Baidovini fulie lettore da' Retet-
fod t0rj, € pero dico, che le (ue beile, ed erudite composizioni lo fecero conolcere»
infin in Parigi, dove essendu fate fenuite in diverse Accademie dall' Em. Sig.
ym Card, Chigi tino di la lo fece chiamare a Roma, e lo diede per Segr. all' Em. Sig.
» Cardinal Nini, la qual carica eghi esercito pi anni molto Jodevoimente; ma
kit Beceilitato dalla poca buona sanità, che godeva in quel clima, se ne tornd alla
| patria, dove essendo stato prowvilto d' una Pieve, quivi se ne vive godendo mag-
b,@ Blor quiere, e miglior faluce, che non godeva a Roma. i
él MELN forcon da Scarperia, Pierfrance(co Mainardi grandissimo di statara, ma
G8 ware dottore. Questo per esser,si può dire,un colotio, ed in sul fiore della gio~
veotl thangiava ati,¢ però il Poeta dice, che se 1 mangiare fa campare, ¢gli
(Ill Per viver molto tempo. L'iperbole di mile anni (e bene \& di numero determi-
'ge ato; si piglia per indeterminaco, e signitica lunghissimo tempo.
I * CASS ANDRO Cheieri, Cive il sig.\ Alessandro de' Cerchi Cavaliere, e Sena-
we tore Fiorentino Segretairo della Sereni(s. Granduchessa, e però ii Poeta lo fa pri-
mo Segretario del Duca. E perché veramente egli € un Gentilhuomo di gutto
"i isquisito, e d' una cloquenza aggiuftacissima, dice, che con la direzione del Boc-
sil caccio (le cuj opere regolano la lingua Fiorentina per esser' egli il nostro Cicero-
Ne ) ¢scorrends il suo Vocabulario ( cive il Vocabolario della Crusca ) messe um mare
di crufea in mezzo fostio, e (cherzando l'Autore con l'equivoco di Crusca buccia.
uv del grad, ee CRVSCA Accademia Fiorentina, intende, che questo'Caflandro se-
id 'ce un diflefo compotto di parole approvate dalla medesima Accademia della,
», 'Crufea, nella quale si fa proteifione di pariare, € scriver pulitamente la vera
“| lingua Fiorentina.
7 PER far un diffefo di quello, che doveano dire, Cioè per metter loro in scritto
I Iattruzione di come doveano'contenerai in trattar 'accordo,si come si faa tutti
gli Ambasciadori,e plenipotenziari, che si mandano da' Principi, Repubbliche ec,
 FAR to spoglio a! un libro, Mercantilmente's' intende copiare le partitede' i de-
 bitori; e per altro s'intende quando si cavano da un libro quei concetti, tentenze,
'parole, delle quali ci voguamo servire in far qualche composizione.
POTER dire,¢ fare, e alto, e bao, Potcr negoziare, e conciudere a lor gu-
O

d)
i

e,
flo', € velonta, the ih
dicono: Peni; j
patentee Bi

libero.

LALDINO Filippucci, Filippo Baldintcci d
e questo intende il Poeta dicendo Huomo', che canta'ben
¢reicera più, perché egli e duggiato da quei due huomini lunghi
e Meio, de' quali egli lo dice'parevte, non perché vera

eg ee ee

e accomodarsi alla rima. Questo¢
jamo detto sopra nel Proemro. ~
* LVYOGO

5 STANZA XXXXVIL
Eperché ore già finian del giorno
Siconfuled, che fulfe fatrafera
, Percio tutti alle spanze fer ritorno
Com! un fatto digatti, fuor di Schiera,
I Cittadini Pavan @ ogn! intorno

* Welle radesfu i cantize alla fronciera, Che non si
Bicivcgh' ognun secondo il suo porere Gis teiehnzs Gene Bl
o 5 foreftieri in ala dia quartiere, Sti Mab spefa dicey men Wid

a ST AN-Z/A*REXRWVNA ome DAVIN
o"Del Principe a' Vgnan pot si domanaa, Poeperre ner

“\  perché la labarda anch' egls appoog
* 'Staffer attorno a rivercar si manaa;

un facco, a quait
LA quarticre »
fied a ME i

ee

uae

BG

anggiaco, Vuol dir luogd, dove nonatt
Pinterposizione di muraglic } o d” altro, EY Gail doghile pian 00
tate, € con poco vigore, e i dicona auggiare; da Yggia » ombra,;
TENNE un mexro miglio di pace. 'Per mbitrar', che queni t
haveano le gambe lunghe, si servc di queste"iperbole'd? un imezzo mi
DA loro il fogito bianco, Apptova tutto quello'; che essi conchi
loro Jil foglio, bianco firmato di tua mano,acctocche vi ferivano lee
capitoli della pace, come più piacera loro, 'Che e lo stesso, chedit
in voi in tuto, e pertutto, In questo senso dific il Petrarca ». my

"Chi Lhabbia racceteato, e chil' alloggi; x
Etiendosi già fatta (era ciascuno sbandd, €d i Terrazami tt
sex dar' alloggio a | soldati di Baldone. Bertinella iawn Pala x

¢d il Generale, 1 quali accctcarono Pinvito. Si'cered deiDuca per co

'ch' eli in Palazzo, dove-bnalmente egli venne dopo qualche di

o che non voleva parursi dalla iocanda, nella quale s' era accomodato..

COME un facco ds Gatts, Cr0e lenz' Ordine, o'regola, ma con!

~ tende, che ifoldau sbandarono, chi io qua, chivin Jay come

Gi dja! andare.
rova aliogyio, Dar

aan ta
a i

os MBA

aa

: wa
STANZA XXEK
Grants a palarro Bi
In Amospame eC
E-wuol che (gli odj mai:
Stien seco, ma ciafe
» Puer' finalmence ne i preg

Se, es 8. SERS PES EL EETE RBPRERPS SR

S” era decniarovoue
Priaichiei n'wferf
Nand per:

dort ~ 1
quarticre significara
ae Swan grote hk ey



em, Sista 30a sobre ipaaie
dA 12. epill. 33. quidem,
r Fer ime —m sed egoa egh, ur eee - Croe noo
wesmercnen gliteci croppe cirimonie. E appresso. Pall pot C. Ca~
» Hlorum ego vix attigi penulam; ramen remanferunt Dichia=;
e ferraiyo'o jinuitare uo. aitaseawate » o pregario a voler 'rima-
co noi. £ ta/ciarsi tirare pel ferrainole, e non accettarc l'invito » € ari
Koa > '
CH! vs disagio, Quand' altri e invitato a un conuito aed
teatro. datalcuno.y.per licenziarsi da chi lo tratticne ta full' ora del ¢o.

s te la-causa speria quale ei i parte, suol fernirsi di qu:flo ate
al eons (a, non dia aifaeio: cioè se 10 son caula, che egli (peade, aun e dovere 5
'disagio-col tarmi aspettare.
“ ee ~Andar a mangiar a casa d' altri senza spendere...:
operat ferraiuolo, o ¢appa.s perché in vece di quello ia porcano sul-
i:Alabardieri + i quali in occasione 4' avere aire a tavola  s¢ ne, spa-
ae appoggiuala-aila parece, e perdo.con quest) decto intendiamo. Posare ra
ior (ad! aters5c.quivi mangiare, se bene Pe/are tl ferrainolo.s' ay
“4 ancora un giovane, che non ha provisione, ma serve in uo banco,, o 'in who ff.
2ibegravissy bastandogl d' edereimpiegato, e d” abuuart per poter goder€ col
oe
MWAMBRA locanda. Incendiamo reli Alberghi, o vero Osterie, che danno, da
 dOrmice a vforetticri.
SERA nce wiare. a era nome eed Havea eletto quel lyogo per' Abto
Fipotor, exis Wiens t
VOLLE mille Porei. Vole iacpdofiaith di citimonie,¢ lusinghe: ed. e io. neiio hc
'chevwererdetto: itopra'< Com fran Bche Janene, così dewto dal Latido vente c1oe
di corpo, e gi fl
“WCODAZZO\, Intende seguito di gente “dictto.« Warchi Stor, Fior, lib. I2.Faé
al Primt Cittadini eli fecero codazrodietro, accompagnandolo, eraccompagnandolo gaila
we ius Cufanl Palarrxo; comes' ei fusseril padrone di Firenze,

hat

Ltd fate

oh “WHSPANZA thy STANZA L...,
A cena (perché il giorne in questo loco dn cambio di guarir dell' appetica ~~

a: * Lblebbertvairra faccenda le brigate, Facenano un collo come nna. Giz est

; 8 arta cucinave intorno al foco ) Se vien frictate, og un Sana accinits,
wt Senses furia ds friteate', Che per aria chi puofe.la scaraffa;
od e nem ipresba si, ma duran poco, Si riduffero in brene a tal partita.,
3 \Che-uppena farte ellveran già ingoiate, C' ogms volta faceanoa rufa raffac,
a Presse gente a rauolaera molta', tn ultimo seguendo Bertinella. >»
gi sR, We" miangiawan dueye tre per wolta, L! andanano @ cauar.dela padella.
gf oWDelerivetarcena fatta'da' Bertinella a i Foreftieriy la. aleconfiflettga, in,
pt fritcate » mangiate con fa fiiria, che egli dice: passo Reale, e cirimonie conue-
if se a una Regina di Malmantile.

iin fueria di fritrare, Beitvate in quantità; 3 Waa gran quantica di Fricta.
sopra C, 3. st. 50. EXIT:

eet



.

448 aan F EDR
PRITT APA SEE viv eda! factard WOVa bE
felid'pddella' asfoge ia aveortah,ielde mene a
125 appresso 'atirort baslerebe dine, petcheirgioy
sce Sen eal: as tra ng “
GIRAFF-A, 'Avimale quadeupede § ikqualess se bene
fidema,€ s citaaaiea Dencaaneg eine toy -havil €onouid
a'quello del' Cammello'ylegambe'dinanai abo i quelled
coda j ed è del colore meuctia®,- che q
i Latini lo dicono' Camelopardalis, cioè bela Yeheticne! I
'Pantera, Pannoil-coo comenine ewafnd inwndealiangas Lio eel
interpretare, che non' fifazialleroy" perchemmeareare | dial
cibo con gran'deiiderio';Latino-¥ehiare } 0) chesaliuagatiero ene
beta

a

per vedere donde, e quandowenivanolle Feiecace ena
refize'a tempo tuo fa menzione'ibPolizignò-nelie® pellance, » Gitiog

Scaligero' simil dit questo:
ail Esercitazionie 209. nutn 3s OVedice "hei Persiani Girmafa P. f
E Abts il BOM Gina Parse Omit » Ha" eH) bho mee
o STAPA accinite, Sravarateetito'y Teo} oprepataco sidal Laci
-didiatho stavalattento', <u'all' ordiné cones, tnleMtaro.chigmaw,

ho tifato i ahtivo's particolarmenté dation Villanty*s sempre in

spele sei ptovvedere'danatir: "Ora /peritintratctared! Origine softy
nendosi il danaro a fructo, la Corte' prititipale 9 siccome da"Greciy dalla
detta Capo scost-da nO1'fi ehiamo Capitalc; e Fondo! ancora, dai tei idere.y 6
la petunia data a intereties a:pensa' di fondo, e»pedere!, orpotieth
ta'; Che'perd:' nftra, come geactata dai danaro \y-the! ayprincipi
Greti Chiamarono Torr stioeParre, 1 Latiniyemes siqua nig
fu Ud Varrdne', e da Norio Marctlio Oticrpabome apiraies p
posito'; ff disse la forte' pquafipecinia capitales principal ndan
che'da questa pechaiirpolta 12/a%phincipio s hevenivd poirdngu

da' Holtri anticht Crvaren, voce che finulmentestrovatiun Gio «
la) éhé i Franzefi didero chewanee,'cioe rendita envratayda Chef, capo.Ora
cinire, che anche dillero, Cixamgare, e lo stesso, che Provvedere ti
<cidé \& chiesact, aflegnar fondi's*¢ ludghi da rischotere; foraire ye:

See. > ea ae PERFEPE RSET ERE RRR E

rnito, nogeiLefto » sircensp
. oP OPP Dee oes ai
y uP Via Con firia, come si-fardellescara

atrornd Peiitrelehh Voce alle vdice usacar; enn Jaycredo

i rofto 'fied' per bi 'iaS* a la asad
Pi Tn ape. Si dice' ido sono più gente d' act
Gialcuno  affatina con preftezza's € (eti2"Ordine, O-regola dip
'egli pud'dic Shae Sad repair med, toa? inciutlese

i e da notare 'Poeta | ' i
Pin pane sopraveiene fiipro
fritearemifttvie? dalle macenier Unica feu

ve



|

!
!

Stanchi di mangiar, non sazz}

Finito

BPA ficsass-

STANZAL
'at anna
Tal musica fini po poi in quel fondo;
Ma perché dopo cena sl vin lauora
Facean parzie le 'ior del mondo,

| Fra' akre Bertinella, e Celidora
inganancieree per burla un bale tando,
 Eapooa

4 0. entrouni altra brigata
Tal che si fece poi veglia formata.

sien STANZA LIL

'Fano poi com' è  usanka
Moite candele intorno alla muraglia,

 Lesplendor delle quali in quella franca
E sale, e tanto, chelagente abbaglia,

+ he distinte si vedeva in danza

bt meglio capriole intreccia, e taglia
Wannaccio in tanto [opr' alla spinetta 2
S' era mefioa xappar la Spagnoletta.

NONO CANTARE.

Z rel taano gestive insane discadess lnnetira nazioneda
orit quali dicono, che i Fiorentini fanno je frittate d'un' uova !'una per rilparmiare;
 \& però dices che durano poco, e per questo ce ne vogliono molte pi + sì che per
sta ragione non è vero, che si facciano sottili per risparmiare, essendo certo,
he tanto. 3¢ tanto unto si con/uma a far' una frittata d'un' uovo [olo,quan-
wm to a farne una-di sci; onde si viene a consumare cinque volte più, perché una
- fristata di sei uova faziera tre persone, e fet frittate d' un' uovo l'una.non sazic-
un' huomo solo. sì che non di fordidi, ma di ghiotti in questo partico-
potion esser tatiati i Fiorentini, che fanno ie frittate di poche uova l'una,
inché sieno più cotte, e più gustofe. Di questa verita si puo chiarire, chi non
erede, con fare a quattro persone due frittate di sei uova l'una, e vedrà, che
eranno fatica a finirle » come le finiranno ben presto quattr' altri, a'quait fa
dieno dicci anche di due uova l'una, purché ben cotte, e questi si ridurrando
a rufa raffa, ed a rubarle anche dalla padella, come facevano coioro di
tile, Raffa raffa \& lo stesso, che il Latino rape, rape, dal Latino rapere,
 fifece rabare, e si poté ancora formare, rappare, come il Boccaccio in una sua
'manolcritta da fugam arripere, formd Arrapare, o dillero la fuga.
r « Leppare, voce della lingua furbesca puo venire di qui, o più toflo da
vare, significando portar via con preftezza, La figura è la medesima, come
Tose dice Prometter Roma, e toma, per avvcatura dallo Spag. tomar; quali;
E piglia, ch' 10 la fo già un, e tela dd. Tre agiole,¢ barugule. L. naga, varie,

mgé. Daa rufa è facto gure; scompigliare.

449
ei detrat-

STANZA LIID

Un gobbo [no compagno un tal delfino
C' alle borfe. più rofto, che nel mare
Tempesta induce; prefe un violino,
Che fonando parea pien di zanzare,
Intanto un ben dipinto mefolina
Si porge in mano a quei ch'ha dainitare,
Et Ygnanefe, al quale il balle tocca
Sciorina a Kertinella in fusse nocca,

STANZA LIV.

2' grave il colpo,¢ gingne in modo tale,
Che quanto piglia tanta pelle sbuccia:
La Danna, bench fentasi far male
Senx' alterarsi in burla se la fugcia,
No vol parer ma infel'ha poi per male,
E dice l' orazion della bertuccia
Sorride, ma nel fin par che riesca
tn un rider più tosto alla Tedesca.

» che ebbero di cenare i Conuitati cominciarono a ballare così in burla,

Ma crescendo il popolo riusci poi veglia formata. Così per lo più segue fra lay

dalla quale nel tempo di Carnevale, dopo le cene solite farsi

x i, si da ne i suoni, e cominciano a ballare fra di loro pa-

Ren, e fenvefi da chi patia per le Se e da i viciui vi concorre altro Boge
it: 1 e



ayo MALMANTILE

e si fa vera veglia di ballo, come segui fra questi connitati
quali essendo toccato a fare da mai 'del batto alla messola
egli inuité Bertinella, perquotendola co! messolino
che le sbuccid le nocca, di ché la donni's'adirò,se bea non ta
ballo alla mefola si costuma in queste veglie per introdu
lo, che è eletto Maestro rocca con que! meftolino le mania
vita al ballo, e poi tocca le mani ad alcrertanti huomini, ¢q
vitate vanno a ballare, e nel ballare il Maeltro da il me!
ella va con esso a toccare tanti huomini, e tante donne, € così
tri usano questo ballo con fare, che il Maestro tocchi ante:
lato che hanno alquanto fra di loro, vanno senza meftola a
mini come e solito, e si seguita senza adoprar più la'mefola',” Q
si dice batlo alla messola, si ta anche colla pezzuola, o Oy
lando si getta a quello, che si vuole inuitare, e così di mano in
chiamato Ballo alla pexruola, 6
ST ANCH di mangiare, non faxxj. Stanchi dal? affaticarsi a maflicar pi
ma non già fatolli, perché havevano mangiato poca roba. Ll Petrarca nel T
fo d' Amore, nel principio:; ne
Sranco già di mirar, non fagio ancora,
Giuvenale Sat. 4. ragionando di Meffalina moglie di Claudio
Et laffata viris, nondum fatiata recessit.
TAL misura fini po poi in quel fondo, Alla fine delle fini tal' opet
nd: Pur una volta fini. Latino ad extremum, tandem, aliquando,
C, 4. st. 9. in questo C, st. 1, alla voce Bordello, € sotto C. 10,
ne po pot, ec, Vedi sopra C. 2, st. 73.

sR SERS TSE RPESEE

=
=

Ha SPR a=

a W

iL vin laxora, 1\ vino opera,fa la sua operazione con dar” alla testaye '
briacare. Del suo lavoro, € della sua operazione si può dire quel che difie} ka
delle pecchie. Ferner opus. i ty

B ALLO tondo, Specie di ballo, che si fa, pigliando più persone per! »
¢ formando così di tutti loro un circolo, ch' è forse Latino Choreas m
nostri Toscani detto Carolare. ee Ye

VEGLIA formata. Veglia vera, e folenne con tutte 'le formalita, i 4
Vedi sopra C. 2, st, 46. dove teoverai Jutrecciare, e tagliar capriole, \& ie 4
st A

23. q
Nunn acco. Questo fu un tale nominato Giovanni, € si diceva
cio per la sua (ciattezza, e spensicrataggine [ poicht fo nome \&
del vero nome Giovanni; sopra il qual nome è da vedi tole
della' Casa ]; Questo insegnava fonare la chitarra 4/ed if
pochissimo come quello, che non haveva cognizione cna della
rd dice epee 4a spagnoletta ( specie di danza ) aflomigliando il
cato delle dita in fu lo Arumento, a uno, che zappi: e Spinerra
balo, o Bonaccordo,,
VN gobbo. Intende il gobbo Trafedi, il quale faceva p

violino, ma fonava assai male, e per questo iI Poeta dice: ch
@i xanxare, aflomigliando il fonar di lui al ronzare delle



d NONO CANTARE,
'It! mipiccoli alati, co acutidimo pungiglione, Questo Gobbo servl alla Sere-
oleemmt aioe. quaita di Nano, e per le sue facete manicre piacque
" salia Serentis, Arciduchessa Anna d'Austria, chg o condufle con se, quando an-
do dove entro tanto in grazia al Serenils, Arciduca Ferdinando Car-
Jodi lei marito, che  arricchi non solo con li suoi gro fipendj, € molto più
con I regaii', ma ancora con 4 denari, che questo generoso Principe si lasciava.
da efio nel Bins delle mane » nel quale il Trafedi era aftutidimo, e face-
 'Ya grosse,potte, perché fapeva, che perdendo S, A, S.non voleva eller pagata,
lige se vinceva era pagato puuwwalmente. E per questo il Poeta dice, che ip un di
Wh quei Delfini, che'predicono rempesta alse borfe, come vogliono; che il pelce Delfino
ica la tempelta nel Mare, e perché questo pesce pare, che sia gobbo, però
i ) per coltyine chiamar Lojfini, + gobbi, Mori poi questo Trafedi, e la-
jit scid mece.ie sue faculta a una donna di camera della Sereniss. Arciduchessa, della
Co qual donna haveva tatco scmpre¢ da innamorato, con patto, che si maritafle con
un Fiorentino suo amuco, che era in Insprug, come segui.
1 MESTOLINO, Cucchiaio di jegno per uso di cucina: Diminutivo di 4zefo-
4, la quale in Lombardia chiamano 44¢/cosa, dal mescolare,
Ada inuitare. k4a da chiamare ai bailo, '
—— SCWWRINA, Chog batte gagliardamente, Il proprio di sciorinare \& quando si
get ort > abit: di paano fuori delle casse ne i tempi di State, e si disten-
 dono per targlt pigliac aria, batcendogli con (curisci,( che dichiamo camari dal
pot Greev camaces) donde scamarare si dice questo battere, per cavargli la poluere,
st o Per liberacgis dalle cigauole - E da questo scamatare, o perquotere j panni, ec.
igel Pigliamo il verbo sciorinaré per perquotere, E sciorinarf? intendiamo uno, che per
 A gran caldo Gi leyi gli abiti daddotia; Dal Latino ara detta poi ora coll' o lar-
f £9, quale Gi fence, quando.ia plebe de' ragazzi con sua antica canzone grida al-
sath le matchere u carnovale efiora Ter, in Adelph, Accipiunds, o muffitanda in iyria
adalescentium est. L' huomo se ladeve fucciare. Quivi Donato, Adafitare enim,
pe
4

4

Proprit'ef? difimulandi canfatacere. E Sopra. eHufficanda; Patienda, consideranda
cum filentio, Gc, e dal sao diminutivo non usato orina, cioè auretra, ne riufei il
verbo Sciorinarsi, che e lo stelio, che se dicetle,con Latino barbaro, e ridico-
fo exawrinare. Netia Valdiaicvole dicono; scfobacare quando exopacare, cavare
i day'. opaco,;
IN buria se la fuccia, La comporta come fatta in ischerzo; dal fucciare-, che
"| si fa, quando si feate grave dolore; tirando a se il fiato,
| NeMivuel parere, mat' ha poi per male. Non vorrebbe, ch' e' si conoscesse;
mane ha veramente havuato diigulto. Virg. premit alcum corde dolorem,
DICE Porazione della bertuccia, Dice de] male borbottando, o brottolando
sotto voce, e così facendo con la bocca quei getti, che fa la bertnceia, o scimmia,
“quando@in rabbia, che pare, che elJa borbouti, e discorra dentxo a i denti; che
-diciamo comunemente, che ella dica orazioni. —;
| RISO alla Tedesca, Rifus fardonicus. Kifo finto,¢ che par più tosto pianto.
In lingua Tedesca ridere si dice Jache; ond' io credo, che il noflro Autore, che
“haveva qualche cognizione di quella lingua per essere stato alquanto tempo ia la-
sprug » habbia detto ri/o alla Tedesca ° non perché Bertinella ridetie, come fanno
12 i Te-



st¢
pero pla
argla
Fase bine ) Che fiand' similt 1
meazione.
STANZA' ae ai
Al Det veramebie pare Bratig? 2G Ya beffii ?
vse "babii bya A onde or Bhreded 00 ee arvlnesy z

Perch gli par a' haverle dato prano, Ci morde in qualche part
ernei d haverla tocca a malo envoy “" Ech) se
Ma quando sanguinar vedde la mano,” “ts ee
Io mi difdico, disse, e me ne penta,?\ "Faia
Finalmente to ho tl diauol nelle braccia, uel mespolino
E [ono,¢ faro sempre una bestiacci@ ha vette!

STANZA LVR?
Wer carargliene pena, è 'Biri
nonfacome,al paren h
Dror

Sl e WSRoERRER

ae
Rin arap in Canberit ih fablerro 2 « o 1009 Syaadermapuoraee:
2” aaanai più TPinig ane Se raceme ie Casaliadonma,

: STANZA LVIL: STANZA Lk
He Principe'a quel oriad ) Wigule? emairep 'LLG ridsa\ Dortma ator come'
3 'a foggiradrd ¥2 Wictrtdto here', 192% «Bldipolaize mance,

= call tro 'du hhh VeAbite dhe JOU 2 « 28 IRE scowe/l ah aldorne gal
Co amore in tui vuol far le sue vendette, —- OUR GNMEGC cated fareRiifwe

Ui quel vive fhiattin combean picebio', "0 « Ds iene yx

erkriwet:

es

“CG abiriih aiiplerofkdW maiekirE\2) 6 OG RLU enareeinura pei

IL mefolina, o quei, che glienc dette Di non mostrar in ranto 8
«NB per BP laa bdr qa Ol 19 LY pPERRE OS Gece vel wsuforat medic
SO Po igeres ia terrain Cente rnild pores |. 129 UD anguenre che teyfan

“0: Bj doe 'G mara vighias che ta Donna*faccit st gran/laniento pparendyy
Osporer haverle® ad maida (anpucuccortifi,, che ib male
2 Se G7 QUEP EY pli HOU eredcvaYy-ripreh de se Reffo |) 2 si metre ivo!
WY CON Medica Me HE biedtlediAratite si feuop! namiorato 0
OCP ARS, enande torino.' Rifentirts'¢ dolerf tanta) 1c! ol amie le
>. Ona ABD Read, “ANfatiday AppehalNon glipard' haverla quai
Sreata ye da Stevicate; e Srenravee dal Liarino settentarey come owii ¢
z ati Cie, Si adPAtcic, ALE wir msiferdque fuftento..10 MeHor} cioè paul
yea thiala pera mi Condued's @ ati'reggo'.° Non folametite dicta
Sich ea oial caeueanae ” | @ mala faricay
C'ferto'y Batinovint' ech ytenid} catkanten\ B fitcometi dies 464
Bebe » cioè grandissima. Ho auuta una buona malartiey
1
Ou

ee ae on ane:

imal¥orza'; pochissimo,.»vsn wed wy


'

ere a
icp. divenfamnenne 4 ne

races ne aa Ow at

sor we! iy 51
nym ORG Shots ovo

vale CMOS en no Mle ghinibizzef.
a RS nme S'S on ioeaeemaneenenie gee

wh ib sen oh
+ CANom artes adenine
a “od medesimo in lode dellt,Vimor.malancolico,..,. A drow Seqeeay Bw '9 sConUL AE

611 » Bvan fuggendo ogns altra compagnia, ) 2 ASWA bE
aaa SOL op Ae Cd ghizibizein 98 concerti, e 4 CARTE, 96 sui grktis WY
yo viens o, Lheecompagman pur sempre.vada.. a8 iA sce s\c08 DAN
j Story Bior-liba.t5e.dige < acca, LAA AER, Seonpes ghicibiczand
Plow dy Dihoeae'D sua | ee
@bArvcca il-ticchio,.. Gliwien, questa volonts ».pen eG @ gape o afoul dal
« Branzcle,%ix.,. mosca caninay;,Sumili, ma Aatnabate Penie, bvalilla.s.¢ Al
adaliailillo.»che ¢,una-molca pungentitima » che infelta 4 i da noi ia
i coenaadad pacerba faransy quo tora aermorndeyert peta NB
mda S 4 APS A
ae Relacanantciocanstiye! dolore. che; prova uo pazeiente,, quap-
See una fericafirmettelale, accto, o,altra.cosa, simile.4.che, moktihica, e>
Corrodede. partielle de'iquali carpi acri, e mordags fembragg..al, ate a
Buila difrecoje feriicanms oPRAZRHO. —.sisshwo9 90. 9\ml)o 5
RAR un tira 4 an Suniendettar uo mal Ceri: c0 a che a iactia a
UNOswiaw A oie ar re we-sfhow vow wh srsivsusily stls,, in
h\STAACCLA comme 4 pleebiawsE grand a.collera Bg i
' sofehiacciare Ggnificabatiene identi per la collera, —- per a ae; ed ha
piquetto: Gignificato fenz! aggivagerus come vom picrhio ma,tal Gmilitnding s ay
questo. uccelioiha propriesa: naturale dij batter, sceau cere
rofted.in fu sramiideg|t aibert per; fueginsdefarmu e sliggual ce
concbellittima giz »che;¢ queiasiMope haysr, molgo., eet 2
'¢ ville uscir le formiche si diflende some morte sopra, quel amo,» €, Ca'
ladingua g, che éJunga, e .carnola.,¢ quella.distends opera il, medesimo a an 2c
ose formighe, vi vanao sopra.per.palcerti, e quando.al Picchio, pare di haveruene
——— abaftanea), ura ate taolinguayeddngoia, aDa.quest '0 uccelio deco in
» Gea Oryscalaptes » 0198: Pictinatere di quencery € InnLaty pics li.¢.formaso,probabil-
ovamente il, verbo Picchiare,cioè. batierese. chi batted demtlperila stizea,paresche face
lorfiedle romore,ca.tdenuyche fa ak prcchio cal becco », Plasto spel, pro-
Seeremersaniice srond Sai S108 OH,. ai )
= MANDA giù Trinigante,eAacomisto,. Bestepymia >maledice tua tal Be,

WAKA

=

adil SeEEELEe dpe



a

454 MALAEAN TILE Oe
¢ suoi falfi Profeti ¥ pn eee
colle maladizioni, coprecesteats e bestemmie oe
GV AIRE, RawmaricarS, eoeee aie: i'
gagnolare. Vedi sopra-C, 4. stan. vventura da wagire ee
guaina; perché i cani quando ne ae tocche,fanno um mug
gito de' bambitii'. 'Si può anche dire, che venga da 1 ase i
rammaricarff dell' huomo. 1 wales Now, 2 bn R
comincia  ffridere, e Puaire., a he wl
METTE 4 foqquadro, Solleva, e mette axofgr tutti i vi
re, Soqquadro \& voce usata dat muratori'y eee Ȣ simili, e v1
squadro, che e quando per accidente d*
mancamento un pelo tirato, o strafeiaatonon può fare” ib suo corlo,
rd cagiona, che giù steomenti del veicolo, o treno facciano si ito at
per lo sforzo, ed affaticamento yche riceyono, eda Yale
drare, e mettere a fogquadro iv vecedi Rordirecobromorey) \&
/MBLETOLIRE, Commuoyerti } Intensrire « Vedi sopra C.
tini pure in vece di /anguere, dicevano'volzarmente ne! sane
eficr cenero., e moscio, pigliando la similitudine das real ¢
signitica erbageio 20 ortaggio; Auguito Imperadore formé una 5
rola, e dilie Serizare pigliando ia similudine dalle bietule, ~per vi
languids'; non iftar bene. Vedi Suetonio'nella Vita d Augulto » Ove:
voci,¢ maniere particolari, che questo Principe ulavaynel. par
Celio Rodigino lib. 15. c10. Now similmente, diciamo! fauna
si, illanguidirsi per il aah d'amore, B Bretolone pre a hu
mii fatta;
BESTIA scimunita, Spend spropositato senza jmenlitnaiasiiya -
zio affatto. Lisca Nov. 2, dts perché. ellaera ponera, a queste se
torre senza dote, ec, Scimunito; sciacco, Scimunito'é lo stesso che wren
Lat. incaftigarns, Gr. acolafes, che not riceve'lammoniziani;) €
fictti, monitoribus aff ah E perché questi, o simili a loro fogliono essere
ale il giovane deloritto da Orazio, Sublimis cupidusque,o amara reli
nix; E qual'é quei, che difvaol cio, che volle: come disse Dante nf
ro nell' Fliade al terzo libto; Delle giowani genti rigogliofe Sempre per:
tere menti; cioè per dirla volgarmente hanno il cervello sopra Jab:
@ che Scimwunito', che di sua natura yale Non ammonito, non riprefo 5
stigato, o che non vuol essere amimonito, ne riprefo, ne galtigato; ¢
rio, € mentecatti fanno; venga' a 'signiticare /eiocco, e haomo dt
to, L' esempio del Bocce. nel Filocolo lib, 4. dove: parlando come
Il tno diletto e dimorar ne! vani occhi delle foimunte femmine, pwd elle
voglia dire ancora licenziofe, immodette, intemperanti, e non
ze solamente,
RAGNATELO. Ragno, infetto noto, Dicono che perm
dej cane si piglia del suo pelo, e fiipone sopr' alla parte offela,
HO sopra C.6 stan. 6.¢ che il ragno, e 'o scorpione aumpa
foper a la piaga che hahao faita coi loro morfo,suaino il pazziene

Page

eFEEEE

 REEE

ea

Se Ss See Se


'*
NONO CANTARE: 455
necredendo chest pezzi delmeftolino, habbiano la stessa virtù; lega sopralia se-
rita, che ha fatta col meftolino a-Bertinella, idetti pezzi Maforle Baldone:, co-
me Soldato bravo » haveva notizia della jancia,con la quale Achille feci Telefo,
ee nea sehen havea detto J' Oracolo, i, Qua
. iabit medebirur, Donde Dante afer. C, 31, disse:, '
lo) loi Cosnod! toche foiena la lancia,

he 14, 0D! Achille,¢ del fu padre esser cagione

tHe Prima di trista,e poi di buona mancia,
| -\Bierede; che il meftolino habbia la medesima virtù della detta lancia.

>

Buk

qt ALAN del Cielo » Quali che Adanna def Cielo, ¢s' intende orto rimedio per
at fanar male,»come fu ottimo rimedia per liberar.daila fame 11 popojo eleno
wiytt inane che. Dio giù mando nel deferto.. diFirenzuola in lode del iegno fante
io, 3 > <oSy
se) sbiaib shoizwe2 S\& uno'non mangia, s' un non si riposa,
lags i Osha il fegato guafto, ole budella,
Rab > Bgli è a man del Crelo a ogni cosa.

** Nota!che in:questo detto la parola:4¢an:non vuol dir mano, non, essendo pa-
Ola figurata'per apocope, ma nell*intera sua efieaza Adem, che così si trovan
scritta nelSacro Tefto quella, che Dio mando al suo.Popolo (che noi poi chia~
jamO manna )¢tal man si dice nella Sapienza al capo 16. che havetle ogni buon
x vien chiamata quivi Paze approncato, e apprestato dal Cielo fenya fatica
o pero iniqucito detto credo che fr debba intender e Zanna, e non mano per si-
se uba cosa ottima in ogni gencre; e-che ciò sia vero, quando sopravvie-
he a*yao® qualcosa di suo gulto,suoi dire: ' wa manna,e non mano: e se uno
ricercaté | se per un su6 conuito una tal vivanda gli piacera prisponde farò Atan-
adScome si Vede 'fopra G, 8. stan. 43. Se bene potrebbe anche dirsi», che colla
feta parola Gi aljndetle a due signincati, e a quello che ora di sopra si è detto,
WMtan; cioè manna, e dian, cioè mano, E ALano de! cielo potrebbe parer det-
ta'Colla medcfima forma', con cui diciamo di qualche rimedio., o medicamenio
cfitace Kyi e (Paro la man di Dio, il che coceisponde a ciò»; che dice Piutarco
fOnumM Conuiuialiam lib. 4. quacit.1.)cheun certo Filone medico,aicuni me-
'Witdinenti Reali, così decti perché erano da Re ) enon da Poveri, o per essere
i*! fepreti di Ré jo per la loro eccellenza; e che dal (occor(o potente, che se ne ri-
; ceveva, erano-chiainati /exipbarmaca, appclld com-particolare.appellazione
mani degl Idaij.
jd) WPREGLAT-A, e neva, Intrifa, sporcata, tinta, Da i venti, che portanan via le
i mmelecine Bal gran vento, che per le parti da baflo gli usciva dal corpo accom-
ip 'pagnato da qualche altra cosa; la-quale ricoprendo le'mele che sono quella par-
ce più eafnola delle:cosce, che forma il sedere } ” alconde alia vita o costin un,
w? Cert modo fe' porta via; sì che il Poeta Meoppiando quel verlo, che dice. «Dai
md 'venti, che Portanan via le vele, intende, che la Camicia di Baldone era tinta dallo
z)
6
è

RELILE wtuateale

'sterco ':
SQVADERNA fuori, Cava fuori de i calzoni,¢ la distende. Morg. Le chiap-
/quaderno con rinerenza, Dante Par. 33. Cio che per o uninerfo si /quaderna.

! tele, ciò che e sciolto, e s(parfo per l'universo, prendendo la fimulicudine da'

J libri sciolti, e squadernati. DR



'436 MALMANTILE |

DIRGLI manco che messere, ec. Dirgli iurie
tia dissero i Lat, ed il Lalli Bitar kon by eHloee Lich
è Teitt m' ha detco peggio che messere. 6)
Molti dicono + Asessere él' asino: ond' io stimo:che dic
che meficre s' intenda, l'ingiurid più che se gli havesse
Comico Fiorentino nella Moglie Ato 4. sc, 10. in-derisione del t
dice: Si; Adefsere e  safine, che va nel mezzo. Quali dica:
quando passa per le strade gli fa largo, eva nel mezzo,
BEL vedere, 1 bel di Roma ¥' intende it Colofico ycheinoi
Ciamo Culilco; eda questo per belmadere, Obel di Kama iptei
che Bertinella pericolava di mostrare alzando le gambe.
Bellofguardo, son nomi di juoghi, e ville nobilidime nel Fic
vato, e donde si scorge molto, e bel paefe.
eHEDICO da fucciole, Medico spropositato,e dipoca:scienza,
mo i marroni cotti col guscio nell' acqua, e preadono tal nome dal /ucciare 5§
fanno i ragazzi per trarne senza aprir wutto 1 gu(cio, la pasta, che vi \& dent
E perché questo cibo e vilissimo; pero foros iamo da si i
nulla. I Latini dissero bomo manct cioè di niua pregio,
fico; per Naucum intendendo il Gufer, o buccia di quaifivoglia cof
la, che si butta via, e non buona a nulla,
LE fa veder le lucciole, Le fa pianger per il dolore, Quando uno}
tale, che gli muova. le lagrime, pare al pazziente di veder per ari
14 di minutissime stelle, simili alle lucciole, il che e cagionato dall'
lagrime, e che pafiando sopra alle pupille offende, ed altera la virtù v va

Oe STANZA LXL STANZA LXUL,

Non dimostra la taccia così mefta S' impiccherebbe, ma dall' altro:
Quel ragagxo scolar,quel cauczzmola, Ei va pos retinente,e¢

Allor che motti giorni e (ato festa
E che finita poi quella vignuala,
Ji matadetto tempo ecco s' appresta,
Ch' e's ha di nuouo atornar alla/quola,
We si gualta belando si la bocca
uuand il matftro col bajion to chiecca, Gli vada in (u le forche
STANZA LXiL STANZA Lxl
Qrante cambiate in viso,¢ mal contento, Poiche '1 cundotto delle pat
Adefo pare il pouero Baldone, S' ha da ferrar(dic' egli,
\ o ha nna stizna,ch'ei si rode drento, Perché si ia leva alle sue.
Per non bauer ceruel, ne discrizione,
Che benc' altrni la morte dia [pauento,
Se e' non fusse che e'c'è condenvariong
Achis ammarza pena della vita,
Con una fune baurebbela finita,

Con quella mane' alei dif

un mutmameess ofa. sealers a2 ESEG~0FE

oe Se


pas sro LIDUE

8 ois! me nyo 34 fan ne'iipauni,
Ps em wre - dif oe vs onngoy eRereheymensre th' I ami, ella v anuifea
'chap yh Ob bu) Chomsany, u lite
a ye fooppia dali th Gi Sent habbia on acgquauite.
intovaci) Poetara Sabeshens cepaeha cheba baleen
ps9 sorim re eran moped Da questo a srcorgentof Bering
di.lei 3-4 h

a iste ine goticiod si sbaseni tmodh 1b i od LE": sy sais: Nas. aoe}
ianbopkematendonstetniens faney Olaltra sorta di)logame, con
eaeioees ed<altre:bestiefimili + Evcaves.t; filidice ancora,

fa merce) collooa: malfastori pquandog)' i =
Oy: 6. Saatgo! aah Eda emadl noiidiciamora un rapazzo-malig

 lioiywenn\ LiVai facendorparlare tn Pedantesdice 2a (09 11309 isos to att?
doped d jwoda, Hee, Seana \& osjuy 1k1gGs esm2) 901813 Iq issagayi Onnst
ab omtetharion Seieababtanitioher ants omiiiiv s odio olaup dioieg ¥
of Mey aba so, O folerro-vrifar ciferory PO Over owsd Orstid imaed bellow
OR iiteade 4a sesete st O, eyed |i ccaaiate marae) 19g e Colt
' A quella id nines ehocien en quel-pdatemps,
itp aeaast iid? eavpolonannavOaiee:
ivcen se sar bebe vignnola', l'ela dnrage pes inten:
credo che sia pata He Tulnadaeeibe deseo opera

mad C.9, nf eke an la) Petite Tn: Eictirhnadutemne 'a a' Buoh

cL aa

uh fa già'an val Sse da 'Panzano, ib quale havendo din' fola pic
ue co) Poe facevz a ex faye barilittivino; ede ae,
on rei hi achiorbailt, ed haVeva 2*OB Hi Torte frattesyichie tro -
7 AO' al non eglnop it si meow rubsiidoltuva,

mee ore Ha" mvae e sempre dieeva";' 'Chie'ratdoplievaer bani bofa
nefla; rem POscoPLe®, thie per fuor bifog ni" tai vende iadetenvigng ye
fre do pia Fitoperta della derta'vight', HoH potevarabare 'come
ee: or Salis fo S*aFri(chiava a imbdttare ance witio's per fo” che
dom abdate alii fo? amici 'da CHe*procedeva' phe eel" err 'vino 5
edali Prisporideva, che era fina la vignuota ae a teiccoe dice: il
può eifer che” venga it i decacos ee ae a) “wip rehea
ov mang an =q oon

rig Pgh Syeepoda balie Roca Pane o.°6. stan.
nares @¢ fo stessd', titi duc verb} stei'dal Tao's IP BalerNov, 7,
ane A ractomandana ' a phi poteres; e coloré antendtnand a vbinctirts 'chi di

pinseebed ls sha olehnon aol,
te una fine baurebbela fin ita: Havrebbe fiaito ee 'fub" CRN aziO' ton-im-

TANTO, o quanto, Termine, che significa piccola quantità, ed 2 lo feffo.
che par un poco; alquanto, Petrarca. E tu, se taxto, o quanto.d' Amor senti,

4 un sopractieni', Parca waa folpenfione, un preety di foptatrenere;
' 'Profiiagato il termine. Mm coNn-

bie —



453 MALMANTILE >

CONDOTTO delle pappardelle\, Cioè la cannaidella gola,
del cibo detto da' Greci Ocfophages, e da noi scherzosamente él condotto de! ho
che risponde alla parola Greea significante il porta cibo, o il Port i
piglia pappardelle, che sono lasagne corte nel brodo di carne per ogni cibo,
ti chiamano pappardelle la ricotta stempcrata con acqua rola), eu ova, a
¢€ poi fritta a toggia di frittelle.

TLR AR le quoia, Signitica morire, come dicemmo LC. 4. 20,
scherza, moitrando, che per la legge del Taglione si gattigar le gu
( civé la pelle ) dei Duca per haver egli commesso un delitto nel
nella, rompendogli quella della mano, € seguita lo (cherzo dicendo, «
morire in /« tre degus  [ che vuol dire in sule forche ] perché con un
col meftolino J fece la decta ferita nella mano di Bertinella; e di più
Ballerino a vento (che vuol dire ballerino da qulla ) per mostrare:
egli commefio ' errore bailando, farebbe gaftigaco con esser fatto mori
do, come pare che muoia colui che e impiceato, Vedi sopra\C, 2, st
re un ballo in campo aczurro; che e lo stesso, che Tirar de' calci a Ronaio,
vento Borea, o Tramontano, Quel che sopra dice: in /u tre legni per i

forche; è simile a quel di Plauto, che volemdo intender Far, cioè ladro, difle
trinm literarum homo, vel
FACENDO il Nanni. Facendo il goffo. Fingendo di non badare, oofferm
re,Vedi sopra C, 4. stan, 26. Mostrando di non s'accorger di quel che faceva Bal-
done, facendo le viste di non vedere. *
SCOPPLA dalle rifa, Ride secgolatamente. Vedi C. 3, stan. 66, alla yo
Pimmei, e C, 7. stan. 66. 0

\&
Been SSB SRS Ewe ow o=

PER l'allegrezza non puo fear nei panni. Si rallegra geandemente.
capir nella pelle. Per il gran gusto Gi rallegra tanto, che non trova qui
di sopra C, 2. @aa, 69, Piatone acl Carmide, poco dopo al principio, volend
esprimere una gran paiione di piacere, e di gioia fa dire a Socrate, \&
più in me flefo. i o ae

cANDARE in fumo ad acquauite, Risolucre in nvila. Suanire. Lat, 4
re. Sidice anche in tu:no d' elisire, Od' eferuite, sopra C, 3. han. 52.

STANZA LXVL STANZA LXVIL-
Atentre Baldon qual fempluerto uccello, 4a ridan pure, e faccian ci
Coast d! tntorno alla cinetta armeegia Per ch' ci vuol far orecehieds Merci,

Lo burtino te genti, Amor ta,
C” ad ogni mo farò fido,¢
Come talor 3! abbrucia ico
“i garto al fuoco, e frau

ed tuti guint serve per zimbello,
Senzache mai vi badi, o fen' annecgia
Ogun lo burla, e dice; Pelle vello;
Crafexn dice la sua,ciascun motteggia,, «

Beato chi pu bella te la feianta Baldon già fenve tl fuocose
E pot leuanfi crosci dell ottanta, Aa com un pan di ”
; STANZA LXVIH. 6

ne ot See wa

E cos} wa,per ca principio Amore, Ma nel getrarla allor
Par bella cosa, efembra ginsto ginfto Perché riftringeye ride
Vira pera cotogna, il cui colore, Ecosi Amor, al primae ani
Odor » saper aslesia,¢ piace al gxfboy C! allerta, e piace 54



lal

aa,t
ie
ib,

NONO CANTARE. 459

STANZA LXIX,

Ed agli cht impaniato, € 4 qualche segno ta lasciamla per hor cbt io'fo diferno,

 Credeil suo amor da lei esser gradiro, Che quefho canto refti qui finito,
 Altero vanne, e fhima a' esser degno, Perché dife un Dottor da Paleftrina

——— Diinvidia più che d'effer mostro a dito, Breuis oratio penetra in cantina,

. era così fit di la, che faceva mille me-
lenfaggini, per le quali era da ogauno burlato, ed egli Fingeva di non se n' ac-

c » o continovava a fare (cioccherie ostiaato in quell' Amore, come tal

volta @ un gatto ostinato a stare intorno al fuoco, ancorché si feata abbruciare.

4 Poeta adsauglia Amore alle pere cotogne, le quali dilettano con l'odore, col

colore, ¢daano gusto nel mangiarle, ma si dura poi fatica a digerirle, € diven-

do che Baldone si reputava più degno d' esser invidiato, che compatito, termina

il nono Cantare.
| CWETT A, Vedi sopra in questo C. stan. 22.

SERVE per zimbello. Servc per scherzo di tutti. O pure per allettatore degli
altriamanti a venire ad amar la sia Dama. Ii Malatefti parlando in persona d
un villano mandato d' oggi in domani, e burlato dalla sua Dama, disse;
' Da poi ch io ho fernito per zimbello,

E son andato trenta mefi aiont

Gridando per la rabbia', e pel ronello
vibr od Come fa il gatte quando ha i pedignoni
ud id « Alla mia Betta ho pur dato? anello, ec, 7
DICE: vello vello, Termine, che fenifica Derisione', quasi dica; guarda, >
guarda lo feiocco, il pazzo,o simili, ed è lo stesso che Esser moffrato a dito per de-
rifione, che vedremo appresso nell' ottava 69. e che far lima lima dittro a uno vi-
sto sopra C. 3. stan. 37.

MOTT EGGIARE. Burlare, o beffare copertamente uno con detti acuti, e+
mordaci. 1 Greci di C diare uno; noi p biarlo, egiarlo, Da
motto, parla; che si piglia anche dagli antichi per sentenza, o concetto, o det-
to intero; B Azorsetto, cine breve detto, e sentenziofo, come son quelli intitolati
Motterti ne' documenti d' amore di mefler Francesco da Barberino. Asutire, loqui
disse Sesto, foggiugnendo l'autorita d' Ennio nel Drama intitolato Telefo. 2a.
am missive piebero piaculum eft, EB ttimato un delitto a ud plebeo il far motto,cioè
aprir bocea, e parlare: onde Azertegesare non è altro, che parlare con qualche.
bel dettoy @acuto. Dal Greco Azythos viene il Latino murire, €'| noltro Adorze,
Ui Casa)però nel Galateo col definire i Motti /pectal pronrezza, e leggiadria, ed
oftano movimento a' animo; pare che in un certo modo lo faccia venire, O pure
scherzaquafi, che venga'da A4oto, movimento.

BEAT O chi più belo te ta frranta, 'BE' lodaco colui, che la dice più bella in bef-
famento di Baldone; ci serviamo dell' cpiteto bearo per felice, avventurato,
fortunato.,'efimili (come se ne serve il Poeta anche sopra C. 1. st. 29. come nel
presente:luogo-, cheesprime, Fanno a gara a chi più bene lo burla: Latino Cer-

sare conuitijs ) Petr. i
NED Beato venir men che 'n lor (Rhos:
Me più caro il morir, che viner fenxa;
= Mmm 2 Le
\& ca

q



460 MALMANT ELE:

LEV AN crosci dell! ottanta., Si ride fmoderatamente. La vt
quel bollore gagliardo, che fa la pentola, Fee era 9 Op
€ si dice croscrare dal suono +. ik gal verbo
Dan, Inf. C. 24,

O giuftizia di Dio quant at
Che carai colpi per venderta crefeia oi

Tl termine dedil otrawta significa squifitezza., o p
ne logico a to: o forse dalle, ralce specie dipannine; le quali
tanta paiole sono a buonidimo grado di perfezione 50 finezza..

Ck ALECC!, o cicalices. Dilcorsi faytida più persone insieme
priamente dire Discorsi dell! azioni, ed snteredi altrui con.
di bene: éd intended per lo più, Cigalamenti fatti dadonn
digiorni, novellieri; per questo quando si sente Ree nuova
dice € un cicaleccio, o una cicalata. >

FARK orecchie di mercante. Finger di non ascoltare 2 o nan.
che altri ti discorra. E propriamente s'intende far oregchie di mercante coll,
che essendo richiesto di qualcosa, o ripreso d' 4leun vizio non
richieste, o non si emenda agli avvertimenti, o riprensioni... Si dice piantare me
vyna lopra C. 7. It. 39. Far conto, chee passi l. Leperadore. ne to. si

COSTERECCT, intendi le Costole: Li costato..

EVN certo imbroglio, E' un certo negozio imbrogliato, is difficile, cele
mo anche ana cosa così fatta, intendendo una cosa. che pon ha eo del banat
del giusto, dell' onefto o del fattibile. ons,

WEL gettarla, Dicono, che la pera cotogna viloinga il venton-a coed stil
mangia, e lo rifecchi rendendolo stiticho, e però dive;Vel.gerranla da dolore se
più lotto dice; Nel fine ti vogtio, nello smaitirla si man. ia fuori
mu dica le ti riesce così di gusto come pel principio s:cioèiquando lama

41d impaniato, E' rimatto preso alla pania, come rimane-il pettiroflo
do la Civetta, intende s' è innamorato 4moris yorte dmplicitus, aK or
parazione, che ha fatta sopra dicendo,
etientre Baldon qual, semplicerta angela 2

". Così d intorna alla Civetta armecoia..
Quando uno ha male grave, da non ne potere ( non iisimene err
dichiamo; £g/i ha impaniato, eq o¢ eam

ALTERO vanne, Vedi sopra C. 8. st. 30, Qui-vuol dire gout,
mando, che questo amore lo renda degno d' eGere invidiato per haver
bene, come stima l'amore.di Bertinella, che d' eles ¢ompatito del
d' cllersi innamorato di costei. B così si da.a,credere digodere ogni
sapendo, che come disse Erodoto nel libro intjtolaca, Talia 5-2 meglio
diato, che compatito; la quale sentenza colle essi parole appunta, a
fa l'usò Erodoto, dichiamo noi comunemence tutto giorno; E.chee ji: ue
ce Pindaro nella Raccolta morale dello Stobea eHMiglian Minvidiak F,
le quali sentenze dalla nostra plebe ridotte in una Cantilena Fiore
Così e sa sincoomate

Meglio e invidia fop| tare h

Che di se compajfion dare,



NONO CANTARE. 451

 DOTTOR! di Paleftrina, Se ioffapeti, che Catone havesse detto. Brevis ora

Caios crederei, 'che volefle dir di lui, perché fu originario di Tusculo,

di Prafeai »eche havette pigliato Palefrina, cioè l'antico Prenelte per Fra-

7 € S'i0" fapeti » che un montambanco, il quale si faceva chiamare il Dotto-
redi Paleftrina, e faceva da Attrologo fusse solito dire tal sentenza, stimerci, che
ee questo, Ma intenda di chi egli vuole, basta che con questa fencenza
dai opps ha voluto significare, che i difeort brevi piacciono inating ai

2 icantinieri, ( perché ne' suoi Originali trovo una volta im excints,

'ra volta i in cantina ) ed in fultanza intende, che ancora gi' idioti amano,e>

ei eee idiscorsi brevi.
fo
ime i

nt FINE DEL NONO CANTARE.

DECIMO CANTARE,
Peeabasdlasibabastiasdbarls 8

ARGOMENTO,
Per far la Adaga col Rival quiftione
Va, ma in vederlo pot le spalie volta,
E, con lui dietro,
Ove e la gente per balare accolta,
Del Lupo in traccia Paride si pone,
Ui trova,e'l prende con induftria molta, we

ugge nel falone,

i E uccifo quel, da fine alf avventura,

STANZA I.
wanti ci stan. che vestono armatura
0° Dartor di feberme, e ingoiator di fquole
4 ditminedaces » che fanno altrui paura,
o Premar la Terrase [paventare tl Sole;
' o BE ratcontande ognor qualche branura
f

o

Sempre ogn'un cone parole;
St fda sl caso di venire all' ergo,
Labial om! olia, poi voltano «| tergo.

STANZA

tpien mostra in zucca bauer del Sale,
hb ee jon [fanio sempre fugge ta guiftione,
Anxi veder facendo quanto ei vale
odMebpicare al bisogna di spadone,

| Ed wu tal guisa è liberate il Tura,

| pene Reps ep pe geste eer a

we

STANZA II,

Mae son da compatir fee fanno errore,
benché non fembri mancamento questo,
Se chi 4 menar le man nonglidailcuore
In quel cambio a menare è piedi è leo,
Ob mi direte: Vanne del tuo bonore

Si, ma un po di vergogna pala presto,
Helio è dir: Un Poltron qui si fugvi,
ee: qui fermofi un bravo,e si mori.

L,

E che ( chi a neffun vorria far male )
Sa ritirarsi dalt' occaftune,

E Senza pagar tafteso chi lo medichi
La campo, che ai ni re se Fee

«dh theme

i eee}


462 MALMANTILE.

STANZA TV.
Ma voi, che di question fate bottega
Credendo immortalarvi; e che vi giova
Far la spada ogni di com! una fega, imparate
E porni a rischise far ogni gran prova, eg
Il nostro Poeta volendo deferivere nel presente Cantare la di
lagrillo a Martinazza, per la paura, e poltroneria della
segui, s' introduce con dire, che quei Bravazzoni,ed Amm
pre discorrono di far rissle, e quiftioni, quando si vien poi ai
ratamente, e loda il lor pensiero, contiderando, che 1
la vita, che far fermo, ed esser' ammazzato per il vano pretesto di rij
eche non può esser biafimato colui, che non havendo cuore a menar |
mena in quel. cambio i piedi, e fa intanto un' azione degna di lode, fug;
male. Conchiude al fine, che tali bravi, che cercano d*immortalarai
ro bravure, e smargiafferie s' ingannino, perché dopo la lor morte:
ur minima menzione di loro: Giù esorta pero ad imparare da i
DOT TORI di scherme,e Ingoiatori di (quole.. Cioè che fanno da mae!
ma, e che si prefumono di saper tenere in mano la spada meglio di chi
da nelle squole di scherma. Ma qui scherzando.con l'equivoco di (quola'
che cofioro son bravi mangiatori, poiché ingetano /e /axole, che fo
ne fatto di farina mescolata con anici, ed € chiamato squola,
figura d' uno strumento,col quale si tefe,detto corrottamente /guola
dixs, come vuole il Ferrari; ed è quella cafletta fatta a foggia di na
ro chiamata anche navicella)entro alla quale s' adatta il cannello pieno dil
passarlo a riempier l'ordito: Si dovrebbe dire (paola, ma l'uso ha
la notizia di tal voce. Dan. Inf. C. 20,
Vedi le triffe, che lasciaron U ago
La (puola,e il fufo, e fecersi indovine,
E nel Purgatorio Can. 31.
E, tirandosi me dietro, fen giva.
Sour! esso ? acgua liene come pola. ?
FANTONSACC!/, Huomaccioni; Huomini di statura grande; ma dicendol
Fantonacei §' intende in un certo modo grardi,e poleroni,o difutili. B dict:
Galeonaces, @Uanizoldacei, ec, Omero nell' Liiade lib, 3. introduce Extore,
del male a Paride suo fratello. £ tra gli altri mali, che gli dice, unoedi
marlo, Eidos ariffe, cioè un bel fantone, d'ottime fattezze; o come meer
significando la bellezza del corpo,disgiunta daila virtù dell' animo;un
un Dongelione, o come dice qui il noitro Poeta; un Fantonaccio, cio? che!
mostra, ma e poco buono a auila, *
AMMAZLAR con le parole, Legiones difflare spiritu,come disse Pl
dato millantatore. Pretender di farsi stimare, e temere col dilcorrer
ritie, quiftioni, ammazzamenti, e con esercitar sempre con chi fil
arrogante superiorita. Di-questi parla Famiano Strada Jib, 2, Pro
Gloriofi ifti duces. Det homsnumque contemprores, \& gut se atijs faci
Calo minitabundi gre 'p ATLis, Guam profil d 08

DR Pweg er ge ep roeae: =. wa

Sener



DJECIMO CANTARE: 493.
tini chiamano milites gloriofos, questi vantatori poltroni, de i quali intende il Poe.
ta nel presente luogo,e se ne dichiara col dire: Se view mas il ca/o di venire all'ergo,
~ ifica, se vien mai il caso d' haver ad adoprar l'armi, non parlano più, ¢
fuggono, che € quell' abijcere Clypexm de i Latini.

VN poco di vergogna passa presto. Quel poco di roflore, che si ha per una cosa
mal fatta fuanisce, essi disperde: Seatenza usata, e praticata da coloro,
che fanno poca stima della riputazione.

(i MEG LIO e dire: Vin Poltron qui si fuggi, ec. Buona sentenza,¢ vera, e prati+
jig cata da coloro, che bramano pe tofto vivere con poca riputazione, che glorio-

gi famente morite; il che bene esprime il detto Latino Vir fugiens denuo pugnabit.

m Der, che s'era srmato, ed havea fatto (Crivere nel suo scudo a caratteri
iamt d' oro BON FORT VN\& vantandosi di voler-far gran bravure, se egli entra
è,g Va in guerra; quando si venne al combattere, buttd via lo (cudo, e si fuggi, ed
misit a. coloro, che lo taflavano poi di codardo disse: Vir qui fugie, ruxfus redinregra-
nme bit pralinm, indicans ueilins Patria fugere, quam pralio mori, mortuus enim non pi.
sen grat (che noi diciamo: / morti non fan pin guerra; ) at qui falurem quefiuit in fuga,
poet pote/? sm multis pralijs patria u/ui efe. Tuttavia anche appresso gli Antichi era vitu-
dda Peroso questo tuggire; e si trova, che 1 Lacedemoni bandirono Archiloco sola-
digi mente, perché havea scritto, che era meglio abijcere clypeum, quam interire,
jue a del fale im <ueca, Kaver giudizio. Vedi sopra C. 4. st. 15. e C. 8. st,
wi, o CHOCAR di spadone, Par che voglia dire, che questo tale si difenda con gio-
jgad care di spadone a due mani, ma intende, che gioca di spadone a due gambe.,
yal Slot fugge: motteggiamento usatissimo verlo coloro, che fuggono per paura il
ie dite Ginora ben di /padone, e \enza dite a due gambe s' intende fuggi. Vedi sopra

| C, 7.0.76. Giuocar di spadone si usa ancora di dire in proposito d' una casa, che

sia igauda, e (pogliata di maflerizie; in questa maniera. Vi si puo giuocare di [pa-
done, ciaé Non vi e cosa alcuna, che possa arrestare, o impedire questo esercizio,
che ha bisogno di iuogo largo, e difimbarazzato.
TaSTE, Vedi sopra C. 1. st 60. Talte fila, che si mettono nelle ferite, dette
così dal taflare, che fanno la lunghezza, e larghezza di quelle. Latini panicidi
ai Vulnerary, lineamenta, i
g DAR campo, che si predichi di ivi, Dac' occasione, che si discorra di lui con
wm) lode. £1 ver! predicare usato in questi termini figaifica Far' cn:omj, o lodare,
| Quand' uno fa qualche azione bella, e di cia si pavoneggia, (ogliamo dire in de-
Be 2 Chese ne predich,
PAR botreca di quiftioni, Viuer di risse. Haver care le risse per guadagnares.
E tanto questo detto quanto far da spada come una fega, cioè intaccaria nel far qui-
fione, come è intaccata, o denotata una fega ) sono detti deriforj a tali Bravaz-
zoni,¢ Tagliacantoni.
LA morte vi si piega, Voi morite, e dopo la vostra morte non si discorre più
de! vostri gran fatti, e si perde la unemoria delle voitre azioni » e vanne del pari

la bravura, e la codardia » Quell' importuno, che per la via facra s'avvid dictro

a Orazio, enon lo voleva lasciare; domandatorda lui, se avava netiuno de'fuoi,

che' aspettafiero a caia; pee maggior suo dolore gli rilpose: Omues compo/ui,(a-

no accomodati, la morte gli ha ripicgaui tucci, Sa) th ee SUN



44

Colei c ha fatto buio
Paga di sogni i debiti a ciafiuno, .  (Benche si
Quella, che dianzi tolfe al di la vitay Per fuggir
Cagion, che tutto il mondo porta ae Comincea a
Descrive con vaga maniera in quest' Ottava V apparir
con equivoci; uae far buio vuol dit Consumar tutto il suo Sed '
tendedo della notte)vuol dire ha oscurato: e se ha confamato | !
¢ fallita » e non prod pagare i suoi debitife non con i
ricca se non di sogni;e pagar di fogns vuol dir pagar di moneta
non pagare, Vedi sopra C. 2. st. 7. fugge dungue la notte per
giona non solamente, perché è fallita, ma ancora i ella te
sia fatta la spia, che ella poco diana. uccife il giorno perché la
oscurita uccide il giorno ) per la qual morte tutto i) mondopi ee
dir, che per tutto il mondo la notte e buio, enter: bruno,e €0
te di gualche nostro conginen i se bene ella non dovrebbe temere di tal!
zione, perché Si chinde gli ocehi a, che fgets on off.
re, finger di non sapere; e il eos connivere., Vedi sopra C, 6.8 vit
vuol dire che si chiudano effettivameate gli occhi, perché og ne
fuggir I iba c' ha le calze gialle, per fuggix V Alba, A e spia del gi
che ha le calze gialle, perché il primo albore del giorno è i colore frail
€ giallo, e così s' accomoda all' equivoco delle calze gialle shee
ze il contraflegno delle spie, o de i toccatori come accenn sopra C
stan. 60. 03 99g
COMINCTA a ragionar dt far le balle + Comincia a ragionare, o r
partenza, che questo intendjamo quando diciamo: 4 rale fa te balle

fa colligar:
TANZA VI, STANZA VII
E denna » che di quci balletti Jf aftidita poi da ranto fran) +
Sarebbe in corte tutto il condimento, Suvi mulinelli, forge cL

Ler ch' in un tempo fol con i calcerti £ data nna Seofferra come i fae
Ballandosuona al par d' ogni strumento, La laciachiede, britdospi 7
Lupo cena per degni suci risperti Perché il mmico all? alba de' Ta
Prefe dag altri un canto in pagamento, Vuol trucidare in singolar
E sopra un pagliericcio angufto, e fod Ed a fargli servixio, più
Fino ad hora s'è cotta nel | suo brodo. Vuol ee
STANZA VIL. ANZ

Pero che wel pensar che la mattina i vi intrepid
Entrar in campo dee alla tenzone, Espaccia il Baiardino, eit
Fa ginfto, come quella Nocentina, Chi la fringesse,
C's giorno andar douendo a processione, Pagherebbe qualcosa ay

Occhio non chinde, e tuttania mulina, Ma tutto questo

(ZRF, BORED ER RESEP RL Bae. ELLER ew eee

Tanto che ud capoell' bacome unceftone; La faccia tofta 7
Così la Strega in cella solitaria Sperando
eAtrende afar mille caspelli in aria, Chie! non fen,

101 Sig



DECIMOCANTARE. 465

-'Martinazza, che farebbe stata la perfezione di quella veglia, se ne ritiro in
camera, e possafi in sul letto stava pensando alla battaglia, che doveva fare con
jagrillo, ed alla fine, se ben veramente non farebbe voluta andare a combat-
ere, finge coraggio per non esser cae codarda, ed in sul far del giorno chie-
le sue armi, (perando pure, che habbia a succeder qualcosa; che impedilca, ¢
® sia causa che non segua il detto duello.

SAREBBE fata ii condimento, Cioè Carebbe stata la perfezione di quei bali,
 di quell' allegria. Così quando sopraggiugne qualche persona gradita in una con-
" jone, si dice per ilcherzo, Venir ella, come il cacto fu maccheroni, come lo
- xuechero in fusse fragote, o fusse vinande; valendo con queste batie similitudini si-
gaificare ciò che più nobilmente fidirebbe. Essere ella il condimento della con-
tm ucriazione, e non vi mancare altro per renderla gustofa, faporita, e perfera.
hued SVON-A al par d' ogni strumento, Ghediio vogliamo dir copertamente, che una
wet cosa pute diciamo: La talcofa suona, Vedi sopra C. 6 stan: 49, ed il Poeta cava
da ciò lo scherzo dell' equivoco, mostrando di dire che Martinazza suoni d' ogni
mit strumento, ed intende che le putano assai i piedi, poiche dice, che ella /uona co'
'mj ¢alcetti, che sono scarpini di panno lino, che si portano in piedi in su la carne fot-
shay to te calze; e si dicono cascerti ancora quelle scarpe di quoio forcile, senza suolo,
gum ma con la fola piantella, che usano i ballerini, e che ulavano già l¢ nostre donne
ga di portare sopr' alla calza quando portavano le pantofole.
ott  PIGLIAR un canto in Pagamento. Significa Andarsene. I debitori, che volen-
rag ticri (cantonano i suoi creditori,si dicono dare un canto in pagamento,cioè fug-
gigi gite il creditore per non pagarlo, e per non avere occasione di trattare con lui:
|. PAGLIERICCIO. E quel gran sacco pieno di paglia, che usiamo tenere in fu
gig Fletti sotto le materasse, detto anche /accone.
wt ~~ 8° 6 cotta nel suo brodo, Non ha havuto veruno d' attorno. Quando alcuno f2
: qualche risoluzione, che non è approvata, o non piace agli altri, e non e da ve-
yi tuno in quella seguitato diciamo; E /* quocerd nel /no brodo, cice senza che altri
vi a \& nulla del suo; o vero Farò come gli (pinaci, e s' intende che si quo-
cono ir brodo
già FA come quella Nocentina, Nello Spedale deg!*Innocenti di Firenze (che è quel
“4 nel guale s' allevano i nati per lo più di copula iliecita, si come accennam-
i Te sopra C, 1. stan, 85. ) stanno riferrate molte Fanciulle, che noi chiamiamo
a Mocentine le quali non escon fuori se non una voita ! anno, che è la mattina,
, della vigilia'di San Gio: Batista, che vanno per la Città procethionalmente; e
Pe ciascuna di loro ha gran desiderio di far tal gita, non vi € aubbio, che
f speranza d' haver a godere si bramata foduistazione, fa, che pare a' ciascuna
 mill' anni, che venga il giorno,¢ che per tal pensiero poco derma la notte avan.
 £1, rivoltando per la mente wweti li modi di comparire atullata, e bene all' ordi-
ne; il che è caula, che la mattina ella ha poi un capo c me un ceffone, cice grof-
o €pieno di confusicni per haver poco dormito, ed affaticaia la mente in quei
Pensieri; € queste son quelle, alle quali il Poeta aflomiglia Martinazza.
MVLINARE + Pensare; Disegnare, andar vagando con la immaginazione;
che diciamo anche: Ghiribizzare. Vedi sopra C. 9 stan. 56. Viene dal Latino
molior » che vuol dir wacchinare,O ne dal volgare Aduino, quali girare coi pen-
aa ficro

te


466 MALMAN TILLEY

ficro come un mulino. Virg, disse spedissimo +| Corde:
che fanno le persone innamorate peulando fidamen
giamente ne diede la descrizione in Didone,
Multa viri virtus animo, multufque
Gentis honos  barent infixs pe'tore vultu
Verbaque, nec placidam membris dat
Tutta la notte va mulinando « E lo stesso, chevaculer. Ho
Quid brexi 'fortes iaculamur auo
multa ?
E' detto ballo scagliarsi col pensiero ora in una cosa ora, inua)
Mattio Franzefi acl Capitolo delle Nuove,
Lasciamo aftroiegare a chi indovina
Per wie di conetiure, e di difeorsi,
E col vernel fantaitica, e mulinay.

HLA il capo come un cefeone. Gli si confonde ik cerucilo, Pai p
do diciamo fa ii capo grosso, 9 se gli ingrossa il capo, intendiamo
de il giudizio: EB Cefone \& un gran paniere fatto di vinciglic dt
te, ed \& capace di mezza (oma, e perché ha la figura a:
queta comparazione. vil

CAST ELLO in aria, Pensieri senza fondamento, ed affegnamenti
nt, e che non poslono riuscire. Laili Ha, Tr. C. 2. st, 2470 ADA

Fra me facea mille Caffelli im aria ode
Ariftofane intitola una sua Commedia, in cui. Gi burla di
Nuuole; e lo fa falire, e passeggiare in aria.y per mostrate, pr !
vana, e senza fondamento la (un filofofia. Noi quando vogliamo dire!
badare a' discorsi terij, e avere il capo altrove,¢ a bagatelle; Dichiamo i
fare a' nuuoli, (e non vuol dire più toflo in lingua Lanadattica: Pen/area milla.

MVLINELLO. E uno strumenco di ferro, che serve per sollevar peli
derivandojo dal verbo malinare detto sopra significa inucnzioni,
ne, disegni, ec,

DATA una scofetta come i cani, S intende, che Martinazza I
veilita, e levandosi dal paglicriccio, fece come fanno 1 Cani, quando,
no, che per lo più si fquotono. A

ALBA de' Tafani, Si dice quell' ora del giorno sche il Bolee
re vigore, nella qual' ora i Tafani sono più vivaci, Tafano. Lati
un verme volatile simile alla vespa nel colore, e nella figura; ma
assai maggiore, ed ha ancor' egli un' acuto pungiglione 5, ficche
de' Tafani s' intende leyarsi di la da mezzo giorno wi) |) say Pr
PAR vegnia uno, Far cortefie, o carcazeasuno, an iL
no affetcate, si dicoao /ezzi, quali iddicia'y o intedtus » come k i
sca Novella 10. Serallegro con Nencio [poso della Ragaread, 6 \&
bene, e le facesse verxi. Col dire.  farls servizia, e pin chee
orecclu sieno i maggiar pezzé,intende, che Martinazza gli fara g
tarlo in pezzi così minuti, che un' orecchio intero sia: 1 mag)
trovi del suo corpo »detto acim per suena un



a
om

DECIMO'CANTARE 467

'SP ACCTA il Baiardino, e il Rodomonre, Si fa Mimar bravo, come favoleggia,
' Ariofto, che fusse il Cavallo di Rinaldo Paladino appellato Baiardo, € quel
¢ Saracino detto Rodomonte. Può anche essere, che far il Baiardino, signifi-

chi far il bravo da un tal Pietro Terraglio soprannomniato Baiardo, che fr un
soldato di-valore, e d'inufitate forze, il quale mori forro Milano militando al

-servizio del Re Francesco di Francia, come narra il Varchi Stor. Fior. lib, 2,
| CHI la fringesse fra uscio, ¢' mare, Chi l efaminatie bene; chi glielo do-

mandafle da solo a solo,

segua, o non vada la posta, o l'invito
tutte le cose, che intenzionate, non s' ¢!

 PAGHEREBBE quatcofa a farne monte, Spenderebbe qualcosa a non far questo
“duello. in ructi i giuochi si dice far monte, quando si reita d' accordo, che non
roposto; e questo e fatto poi comune a.

i(cono: per esempio / tal matrimonio,

he era già conchinfo', ando'poi 4 monte, cioè non si stabili. lo voleva andare a Ro.

with
joie

ma, ma poi ne feci monte, cio non andai.

IN se tien duro, Lo tien segreto in se. Non si confida con veruno,

FA factia tofta,, La faccia fol' esser dimostratice delle interne paffioni; e pe-
ydiciamo; / rale fa faccia toa, intendiamo il tale si sforza di non sco-

iia prit co mutamenti del voito 1 suoi segreti, essendone richieflo, ¢-di non confet-

waco

a

giù
:
3
eo
è
:

i

:

“STANZA X.
Spada,e lancia fra taro un Servo apprefia

i Col perto.a borta in man Laltro galoppa,

'a altro o elmo da coprir la testa
Da distder unalcro,e bracciaye groppa,
Di che coperta in ricca sopranuefta
Par un pulcin rinvolto nella floppa,
Ed allestica in sul cantar del gallo
eitro quivi non resta, che il Canalo

fare itdelino »essendone claminato. Latino frontem perfricit,

STANZA XL

Percio fa comandare a i Barbereschi,
Che lo menin n' un campo di gramigna
Accioech'ei pasca un poco,e si rinfreschi
Perché per altro il poverm digriena.
La marca bebbe del Reeno,es enidalescbi
Gis hanno rifatta quella di Sardigna,
Maglie,e reti ha negli occhi,ode per cena
Vanne a pescar nel lago di Bolfena

B servi di Martinazza le portano l'armi, delle quali armatasi, ordina, che le
sia condotto:il Cavallo, quale il Poeta de(crive per una solennissima Carogna.
“GALOPP A, Cioè Corre', Verbo usato in questo significato,ma però impro-

prio, perché galoppare, o gualappare \& specie di correr di Cavallo; la qual voce
concorrono gli eruditi a farla venire dal Greco calpareia,

GROPPA, Si dice la parte di dietro del cavallo, o simile animale, ma qui in-

tende la schiena di Martinazza.

PARE un pajein rinnolto nella floppa, Quando si vede uno, che non fa portare

l'abito in dotfo, e che pare impaftoiato nel camminare per causa deg!i abbiglia-
menti, che had' attorno, l'aflomigliamo a un pulcino, o pollastrello rinvolto
nella stoppa; e non siamo is ciò dissimili dai Latini, che in questo proposito
didero. Herer ranguam mus in pice.

SVL cantar del gato, All' apparir del giorno, che a talora fogliano per Io più

cantare i Galli Vedi sotto C. rr. st. 5. Orazio.

etd galli cantum con fultor ubi ostia pulfar,
BARBERESC HI, Intende gli Stalioni; se bene Sarbere/chi chiamiamo coloro,
N ai

on 2 i quali

Bikes e, 4


468; MALMANTILE —

t quali cvflodi(cono, e gevernano i Cavalli Barbari, ¢
Poeta gli chianya così per derisione del Cavallo di Martina
Firenze 1 Cavaili, che corrono a i palj della Città, ton
frica, che noi chiamiamo Barberia,
CRAMIGNA, Erba nota buona per pascolo degli Afini più
li, ma a quelio di Martinazza non par poco haver di questa,
zerin digrigna, clue s¢ nou havesse di questa, non havrebbe.
ci serviamo del verbo digrignare per intendere flentar per la fa
nare, e acrocare i denti per non hauer altro, in che ado
canl, ec. che si dice digrigware, quando per la rabbia
Tat Cas.
x Non vedi tu, che digrignano i denti
Econ le cigha ne minacctan anoli?
Ed egliame: Non vno, che cu paventi,
Lascsagli digrignar pure 4 lor senno,,
MARCA, Contraflegno. Es' intende quel fegao, che hannoi

li, o di razza in una coscia, o nel collo, perché da essi si possa
razza sono. Virg. 3. Georg. Continuoque notas, nomina gentis inurunt,
che questo Destriero di Martinazza havea già la Marca del Regno di
sono oggi i migliori) ma che i guidade/chi gue n' haveano mutata in
digna, € non intende dell' Liola di Sardigna, ma di quel luogo fuori
Firenze, dove si scorticano le bestie morte detto la Sardigna, came
pra C, 1. st. 2g., ed intende, che questo Cavallo per li guidaleichi, ed
fetti, che haveva, era buono a mandare in Sardigna allo Scorticatoio
te/co diciamo ogni scorticatura fatta alle Bestie dalle selie, balti, o altro. Mau

Franzcli descrivendo un cavallo fintile a questo disse; wig
Dinanzi ei non e 21d troppo gagliardo; “iy
Ma in sa la scbiena ha qualche curdalescho,
E le spronate mostran, ch' e infingardo, ™

MAGLIE,¢ veri, Così chiamiamo alcuni mancamenti, che vengono si
occhi alle bestie; ed il Poeta servendosi dell' equivoco dice, che con 'quelle ra
può andar a pescare nel Lage dé Bolfena; ed intende, che il cavallo-era bof
dicemmo sopra C, 3. st. 53. » che cosa sia. E così sotto questi equivoci iroa

mente loda il-Cavailo di Martinazza. sagt
STANZA XIl. STANZA XIIL
Hor mentre pajce 1 misero animale, E ti faluta,e tt si raccomanda, —
Eche si fala cerca aclla fella, E per cha inteso, che rm fai duclly
Giunge un Diavol più ner aet caviale Un rotelion di fughero ti manda,
Con un marteile in mano,e una rorella, Spada non già,ma ben gnejto:
Ed un liquor botiente ix un pitale Con una potentissima benanday —
'Ed inchinato a lei cos favella: Ch' 10 ti prefemo emr'a
I Re dell' Infernal Diavoleria Bell! e caiduceta come la.

Con queste trescherelle a te m' innia, edilo [pedal si ad ta medicinal
;: aie



DECIMO CANTARE. 469

. STANZA XIV. STANZA XVI.

Hor fenrj; che qui batte sl fondamento Ma se per non haver buon corridore'
Quand' ih nimico ti verra a ferire Quivi a canfares tu non fulfe leffA,
Va pure innanzi,e non haver spavento, O per altra disgrazia; o per errore
el ferro questa targa a offerire, Ei r'appoggiassi qualche calpo in tej 5

 E tuffo ch' ei la passa per di drento, Vorlio, che tu per sicurtà maggiore

» Sia presto col martello a ribadire, Hor per allor4 ti tracanni quests,

Ma lasciagnene subito alla spada Quale e una bevanda sh squifita,
Peich'egli a se tirando, tu non cada, Che chi Lha in corpo no pua uscir di vita.
Ni STANZA. XV. STANZA xVIL
Face' egli poi con essa quanto vuole Così le fa rngoiar tanty dt micca
| Che pix di punta non puo farts offe/a, D! una colla renace di tal forte,
» Di taglio manco, essendo c' una male Che dove per fortuna ella si scca
Si fata a maneggiar pur troppo pela; wl mondo non è presa la più forte;
Portila dunque per ombrello al Sole, Luefta ( die' egli) Uanima t appicca
» Pere alia resta non gis muona scefa Ben ben col corpo, e s'aitre non e morte
 Edigli( già che queila non e il case) o? una fepararion di que/ts Aussi,
Che  egli ti vuol dar, ti dia di naso, Oxgi timor non hai de' fasti [uci.

» che Martinazza aspetta il suo Cavallo riceve un regalo da Plutone. 5
confiftente in armi jd in. una bevanda per difendersi dalle ferite,¢ dalla morte,
Nota che in questo bel regalo il Poeta immita coloro, che hanno scritto le pro-

 dezze d? Amadis di Gaula, ed altri Romanzatori, i quait, quando il loro Erce
dee esporsi a qualche battaglia pericolofa, fanno sempre, che qualche Mago
“amico di esso Eroe io mandi a regaiare d' armi incantate, o altri difenfivi, ed
inttruziom, '

St fata cerca della fella. Si ta cercando della fella, Dice così per mostrar, che
ae era tanto iniolito ad adoprar la (ella, che non si lapeva più dov'
ella fusse, >

PIÙ ALE. Alberello, o vaso di terra, come dichiara il medesimo Autore nell'
Ottava seguente dicendo; ch io ti presento entr' a questo aiberelio, Se ben Pitale \&
Piopriamente quel va, che si mette centro alle predelle con altro nome detto
tantero..L' uno, e il altro nome dai Greco, quello da Pitharion, piccol valo di
terra, doioiwm; quetio da Cantharos voce usaca anche da' Latini. o significa un
vao lungo, e stretto in fondo. E con manichi, quale e queilo, che si vede cal-
volta figurato in mano a Bacco.

« TRESCHERELLE, Lato trice, Bagatelle; Coferelle di poco prezzo, Ve-
di sotto in questo C, f, 28.

SVGHERO. Pianta aota simile alla Quercia, e fa le ghiande ferotine, e la
faa leggierisfima (corza serve per far lavort da refiitere all' acqua, come farebbe
caiietce per mettervi bomboie di vetro piene di vino, o d' altro per diacciare.

 BELL e calduccia,, Temperatamente calda; e come si da la medicina, che intea-
diamo bevanda folutiva. Vedi sopra C, 8. tt. 25.

CHUVOVERE fiefa, Fer yenire l infreddatura. Scefa diciamo una distillazione,
o catarro, che dalla testa casca nell'altre membra per causa del freddo.

Tl dia di naso. Detto iporco usatisfimo nella Picbaglia ia segno di disprezzo,e

sin-

————————————< =

i

se

472 MALAANTILES @
s intende di nafoine,... che per ricoprire si dice 6
serve = esprimere la poca stima, che si fa della ——

NON fussi lefiaa canfarci. Noa fai presta a-fuggirli,.
Effugere, delinearesy nes lisdab Greco compre ara
detto così quali CG x F

TRACANNI, they bevay logolli i

TANT Adi mica, Vina gran quantità di ininefied = "
tore del Capitolo in lode de' Peducci, parlando:della min Secccea i
E gli ho tutes per cari, non che buoni

Von ostante, che sia chi dica espreffa,

Che tanta micca e cosa da bricconiy
Ser Brunetto Latini servendosi di questa voce nel suo libro vco
tutto di gerghi,¢ vocaboli,¢ proverbi Hinsanwsats 7 intitolaco
che sia antica Cittadina-di Firenze, 1

Non ti darei una mica di beata; ¢
Se bene qui par, che voglia dire un bricivlo, dal tele
tanta si pronunzia col gelto, che accennammo sopra OC. soft.
Luefia pea, e vedremo (orto nell Ortava 18.¢ 36. seguenti

FICC-A. Ficcare vuol dir Mexeré » 0)Cacciar per forza'.

NON è presa (a pix forte, Diciamo fan presax, quando la collay cal
o simili s' appiccano gagliardamente in quet noghi »ne\bquali-sono

L'ANIMA \& appieca, Si ricordi il Lettore, che quella 6
fu le burle, e particolarmente dove si trata diyincanti,ne iquali, q
trava luogo di fare apparir qualche azione spropositata,non lafera
segue in guesta bevanda, la quale dice, che appicca ' Anima al
che egli creda, o voglia periuadere, che ciò possa per incanto farsi
firare la goflaggine di Martinazza, e di coloro j che hanno tanta a
caatelimi, e ne i Demouj,

STANZA XVIIL
Quando la Maga vede un tal presente,~
C' ha in se tanta virtù, tanto valore
Da. morte 4 vita riauer si sente,

Si ringalluzza, e fa tanto di cuore, 'Cusiabe 'hontai i se
E dove fares ita un po.arilente Percio fatracal ronzin ha fell
Nel far con Calagrillo il bellumore, Vi monta sopra, € poi te xomb

Hor e ha la barca assicuraca in porto

Pere! adesso ch' eg ha ratte
Per sette volte almanco lo vuol morte, rddy

Camminerebbe più in
STANZA \&X

Perché ei bada a spudiar declinazioni Pur.grazia del mated
Pin non si pua farlo levare « panca; Tentenna tanto,
Le polizze non Pwo, parca i i frasconi, Chiesvien. done n'
E con lo spalle s*¢ givcaio un' anca; M14 « carinetie il fang

Martinazza inansmita dai regalo mandatole da Piutone, etlendo-
Sole, monta a cavallo, e taaro io fruga con gli sproni,, e col im.
zoppicando pur alia fiac  conduile ai luogo dove havea ote \

si
at

reese erProczsleezeTEt.2f:r2=...



DECIMO CANTARE. 471
ST sente viauer da morteavies.Cioè le passa quel timore, c' havea dvessere
- ammazzata da Calagrillo | x,.
mi. SI ringaliarza. Si caliegea. Lat. Gefire, Si dice ringalluzzarsi, quasi mo-
~ strarli ficro.,.¢4 animoso come fanno 1 Galletci,quando si preparaao per -
“ ter fra oro, @ dopo che hanne combattuco, e vinto. Lucilio 4ib, 8, satyr. dice:

eee Galli nacens cum victor se Gallus bonefte
ee ite o Sufulie in digitos, primore/que erigit ungues.
@ [Lalli En, Te. C.5. dan. a6. ditle 3. Jn guetta nacas anor si ringalluzza. Stor:
di Seumifonte TLratt, 321 Semifoutefi, credendo d'hawer ogni dsfficuied fopita, rinesl-
— bnzxaronfi, @ fidandosi di (un valentia, ec, B pi Lowe dice: Veds quanto noi fama
om 4 iti, e 8 mimici ringalluRrati, eC.
gibi FAltanto-de cuore, Piglia animo, le cresce V' ardire. E il termine Tanto nel fix
infos gail » che diceauno nell' Occava 17, antecedeate ed altrove,.¢ si fuppone
i sho deteo.aicrove:), che colui, che per la faczia la dimottrazioac con la
| Mano accennando la grotiezza, e pants di quella cal cosa, Quei che i.La-
ect ual daimus, vooltci quali sempre dicono coraggia, e cxore.
ia | SAKEBSE a a rilente., Sarcbbe andata adagio. Circospetta ) O rattenuta a4
¥ risoluersi,, )L? havrebbe pensata, o contidcrata. Significa infomma operar coa
tuwore. Leace per lento, siccome Violente per Violento dicesi da alcuat; come
Questo filo, queita corda e fenre,.cloe non tela, non urata. Da Lente si fece Ri-
ing (fn sche noo ti usa se von in questa Maniera: eFadare a rilente, e significa lo
cai stesso, che Lente cioc ientameate., Nello tteflo modo chel'antica voce Diricapo
ail usatardal?-anuco volgatizzatore di Virgilio;c lo iteilo che Dacapo,
PAR ih bed: umore. de dea huomo dell' umore, vuoi dire huomo faceto, e
SFAZ1010,5-come vedenyno fapra C. 1, stan. 1o..¢.58. s*intende anche wao., che
si Voglia: sOpcattarc 1 Compagna-di parole, e di fatti, e c, comes' intende nel. pre-
feute Moga, f
aia AOR eC ha la barca aficarats in porto, Cioè le par d' haver assicurata la vita col
regalo mandatoie da Piuwone
ih nae Vheche racing a | bucats wi fu i terrazzi', Cioè il Sole, che asciuga i panui
“moi deabucati, Dereazzo! »( quali Terrazzo) diciamo quella parce fuperiore >
jul dele case la quale per loipiu edasciata da ana banda aperta', e feoza muro, in
Se vece dei quaie lita, solteacre al tetta- da colonac, e fom fabbricati in questa forma
ww per comodità d' havere idole epercid das Latinidetti Solarimm, eda i Greci
Wil hewocaminus', Cio fornace del Sole,
iM CAM AMNGREBBE più in tre di che in uno, None dubbio., che qualsivoglia
m! — Animaie.camuninerebbe pith ia tee giorni, che.in uao, ma uGamo questo modo
wih di dure per moitrar la fiacchezza d'uao Aaiuale, quasi diciamo: Quel viaggio
che egli na da farein un giorno, 10 farcbb¢e pwd voleauieri in tre giorni, che in
yun foiow ue ¥
ul BADA a fhudiare-declinagioni, Attende,:o-continovaad accennare di -cadere
w” ~— perladebolezza. Declinare's' intende uno, che*etfendo in buono stato:, o  dt fa~
ie hita,o di roba, cominci amancare nell' uuo, O.nell' altra; equi (cherza cont
4) equivoco delle declinaziom de 1 noim 5'ed-insende, che-il cavalo per la deboica
za era fempre per. cascare..
0 Wow


47% MALMANTILE o

NON si pui far lenare 4 panca, Non si può farlo riavei
star ritto: quand' uno è stato lungo tempo afflitto da i difaftri
to per terra, o vero terra terra) € che a poco a poco si va
Comincia a rizzarsi a panca; BE' traslato da 1 Bambini, quando:
dar ritti appoggiandosi alle panche; onde habbiamo un detto per
uno sia più aftuto d' un' altro, che dice: Quando it Diauolo del tale nat
del? altro andaua alle panche, Franc, Sac: Nov. 158. dice: ach 60)
nostra mercanzia, che non ce ne rizzerems più a per questo anno, o
NON pwd le polizze. Non ha tanta forza ch' ei possa portare una po
Latini pure dissero: We folium quidem fuspinet. t

PORT Ai frasconi, ec. Diciamo portare i frasconi uno, che sia alg
mo, traslato dagli uccelli, ne i quali e contraflegno d' infermita,
abbaflate, che paiono bestie cariche di faftella di frasconi. Vedi.,
g. alla voce grado. Qui vuol dir che il Cavallo era infermo, e malandato per lt
vecchiaia. | Lb
CON 10 [palo s* e giuscato un anca, Scherza con l'equivoco del giuoco di
nel 8 quand' uno piglia tante carte, che col lor contare
31. si dice /pallace, o ha baxuto lo /pallo, e perde, sì che intende che il

\&

Martinazza è spallato. von lil
GRAZIA del martello, e degli sproni, Con ' aiuto del martello, che le mand)
Plucone, e degli sproni, cio perquotendolo col martello, epi 1
gli sproni: Diciamo anche mercé del martello, ec, er
S* arranca, Diciamo arrancarsi, quand” uno per qualche difetto non pot
muover le gambe s' affatica per camminare, e/forse e il verbo p
pato. Vi chi lo fa venire da Anca, che è l'offo tra "I fianco, ela coleiay
questa dalla Greca Ancon,colla quale si significa il gomito, e si stende ad
gature, che somigliano quella del gomito, Onde Sciancaro, quasi ex:
pun ha intere, enon senza mancamento l'anche. B Arrancarsi quasi tirarh, 2
straicinarsi distro l'anche. 15h) aga
NE ha da ire il sangue a catinelle, e ha bigonce, Ha da verlari moltissimo far
ue. Vedi sopra C. 2. stan. 57. (perbole usaca quando due Poitroni
ducllo, Vedi sopra C. 1, stan. 62. in altro signiticato. BC, 3. Ran, 29, che ol
sia bigoncia, Quando l'indugio piglia vizio, e-che fa di bisogno la preftezza jl
altro proposito dichiamno. ee ne va il sangue a catinelle, aah
STANZA XXI. STANZA XXIL
Quand! ti Nimico, ch' ius faa disagio Se tu sapessi, come tu non faiy
A tal pigrizia,grida ad alta voce, C' armi son queste pie
Vieni Afinaccia, moniti Santagw Farefte forse il brauo mance,
Cb' so son qui pronto acaricarti anoce, O parlerefti almen-a' altro ling
Ella risponde: A noce? Biagio; Ma già che tn venifti a tno
Fate un popian Barbier ohe'lranoquoce; i
S' altro viso non haivallo a procura,

SeweR.> es > sre

cr

a repo RB re =

atten

o

alee

Lerche codesto non mi fa para. rrotté
Arrivata Martinazza al luogo dove s' haveva a fare il duello.wi tr
¢o Calagrillo, il quale vedendola venire così adagio ia fgrida, ela



SSk EL

8 EE Sei ESei a oak

=
\&

DECIMO CANTARE. 4B
ella gli risponde;che non ha tanta furia, dicendogli ch' ei non' farebbe tante»

bravure, se egli fapefie di che armi ell' e armata),¢ che ella veniva per ammaz-

zarlo.

“STA 4 disagio. Patisce aspettando: Sente incommodo in aspettarla,
 eASINACC/A. Parola ingiuriosa, e benissimo iesire in questo caso a

Martinazza, perché veniva pigramente, come fal' Asino.

. SANT AGIO, Si dice veramente Ser egio; che fu un Medico così nominato,
perché taceva tutte le sue faccende con ogni maggior suo agio,¢ commodiia fino
a tirighare, e ripulire la sua mula, senza muoversi dal letto; ed è passato poi in
verbio, e yuo! dir Huomo di turti i suoi comodi, e tardo nell' operare, che
ju una parola diciamo - Agiato. O forse-dalla voce Toscana, che vuol dire Len-
fecha, Comodird,

A caricarts a noce, Quando il noce è carico di noce, si scarica con le baflonate,
e pero dice, che wuol caricarla alla foggia, che si carica il noce, pec scaricarla
poi-con le percosse

» @LAGIO Biagso. Modo di dire usatissimo, e particolarmente de i Fanciulli,
€ credo che si dica per caula della rima, e del bifticcio, perché per altro il nome
Biagio e superfluo all'espressione, valendo tanto il dir solamente adagio, quanto
adagio Biagio, S¢ bene ci e una favola notissima d' un certo Contadino nominato
Biagio, i quale perché non gli fussero rubati i suoi fichi, se ne stava cutea la not-
te a far loro la guardia; onde alcuni Gioyanotti per levarlo da tal guardia, e
poter a lor gusto corre 1 fichi, fintifi Demonj una notte s' accoftarono al capan-
nettoid) Biagio mentr' era dentro, e discorrendo fra loro di portar via la gente,
ciascuno narrava le sue bravure; ed uno di coltoro disse ad alta voce; Se voglia-
mo fare un' opera buona catriamo nella Capanna, e portiamo via Biagio; Bia-
B10 ciò -udito,scappd dai capannetto tutto pieno di paura gridando Adagio adagio.
o di qui puo forse havere origine il presente dettato Adagio Biagio, o adagio disse

ago,

FAT £ pian Barbiere che'l ranno quoce. Di questo dettato ci serviamo, quando
Ron vogliamo acconscutre che si faccia qualcola in nostro danno.

» COT ESTO vifa non mi fs paura, Quando vogliamo mostrare di non temere
diciamo: Ha tu altro viso?e qui Martinazza dice: Va 4 cerca d! un' altro viso
perché corefho non mi fa paura.

SEVER AGGIO. Invende quella colla fehe le ha fatta bere il Diavolo, 1] Fran-
zele dice bexarage corcispondentemente alla nostra voce.

A tao ma' guar, Cioè a tuoi mali guai; Mal per te, che ci venifti, Ci sci ve-
puto per wrovare il tuo danno, Cusi 44a' passi diceli alcuna volta per cattivi pal-
si; ome 'Piano a ma' passi,

MANDA 1 faggio. Quando si da una piccola porzione di quella mercanzia,
che si vaol yendere:, acciocché il compratore possa riconoscere la qualità di etla
mercanzia si dice; dare, o mandare il faggio. B Martinazza dice a Calagrillo,
che intanto mandi il faggio della sua carne ai vermini, perché fra poco vuol
mandargli nell' avello tutto il corpo.

NON volti portar basto. Non son solita fopportare ingiurie.

Ooo STAN:
414 MALMANTILRS 2
STANZA XXIIL t ]
Horsh, dic' egli, all armiv apparecchia,
E vedrem se farai tante corenne,
o questo suono allor mona Pennecchia

y

Dice fra se: No,no:Non taro Ammenne, \-  E ch* io t° insegni far

Sard meglio qui far da lepre vecchia, Così tn ch' item

E fenva star a dir pur al o... vienne, Milafis a}

Fa proua ( già dilcefa dal destriero ) Ma fa pur quitof

Se le gambe (c\dicon meglio il vero. Bt ual, se eu fu z.
STANZAXXV20 1)

S? al cimento, dic' ella, del duello C
A furta corsi, hor fuggolo qual pefte, Però che dop' al muro f
Pero va ben, che chi non ha cervello Grid egli quanto vuol,
Habbia gambe, e così mena le [este, Che per le grida it Lupose
Mortinazza, vedendo, che Calagrillo non cede alle sue bravate,

che fara meglio per lei non indugiar più a fuggirsene, pero (non si:

cavallo) fmonto, e fuggi così a piede verlo il Castello!
rimproverandole il mancamento, ma essa stimando più il peri

la perdita della riputazione fen' entra in Malinantile, e lo lascia
SE farat tante corenne, Se farai tante bravure. Detto di derisione a wu

vantatore. wR
MONA Pennecchia, Detto derifivo alle donne. Da Pennecchio', ig

priamente si è quella quantità di lino, o lana, o cosa simile, che si

rocca per filarla, detta così quasi pensiculum.. Dal Lat. pensum.
NON tanto ammenne. Non fara così. Ogai parola non vuol risposta Per

io non voglio poi anche fidarmi in tutto di Platone'. Amen \& parola Bl

vale In verita, Per verita. er
FAR da lepre vecchia, Cioè tornare in dietro, La lepre vecchia per'

gnar terreno, quando e seguitaca dal levriero da in dictro, (il cS atto

La un ganchero, Vedi sopra C. 2, stan. 76. ) ed il cane furiofo se

scappa innanzi, e perde l'occasione di pigliarla. L' aftuta maniera

della Lepre è descritta mirabilmente da Eliano nella Storia degli animali'
cap. 14.. are

SENZA dire alc,,., vienne, Andarsene subito, o senza 'merter tempo it

mezzo. II Pulci nel Morgante, £ non è tempo da dire ale.... vienne.
SE le gambe le dicon meglio il vero. Se cilia fara più presto a fuggire

a cavallo, Quando le gambe, braccia, o altre membra fanno bene la

razione diciamo: Le gambe, ec, mi dicono sl vero, cioè non mii fallifeone

mancano sotto., Wee

Cl hauessi detto ulmen Salamelech, Almeno ci' havefii'ta detto.

Turchesca usata da noi per (cherzo; e significa, Pace, o Salurea voi. ~

FARM le feilecche. Betfarmi.. Vedi sopra C, 7. stan. 25, 11 Vor

goefe dice, che Cilecca wien dal Greco Cileo, che wwol dir mulceo far

feilecca far tl contrario di carezne, civ far burle. Ma può essere, che |

licta Gi fece Lezei forca di delicatczzc così Scileccke il contrario, che A

aliettare, e poi burlare,

E intana di riterno

= pe eet S*e2 ce oc We ieee se*8. BS screes.es.es


SSe

DECIMO CANTIARE. 475
WMI lasci a prima giunta in sulle secche. Subito,m') abbandoni + Milasci:senz'

- alcoltarmi..B' lo stesso.che lasciar in Naffo,visto sopra C, 1, an. 79, Si dice an-

che /asciare in Seco; lasciar sulle secche di Barberia. Lat. Syrtos.

AO teco il sarlo. Ho.rabbia teco., perché ilxoder. della rabbia s' assomiglia al
roder del tarlo nel legname:,Per il contrario si dice: auer baco.con una persona,
cioè averci paiione. Petrarca: Afentre che il cuar daeli amorofi vermi fu consumato

TI vegito se tu fussi in gremboa Carlo, Tiarriverd per.tuto, Diciamo; J.
grembo 4 Carlo, cioè Carlo Magno Imperadore, per mostrare che si vuole arri-
vare uno, e vendicarsi in ogni maniera, quand' egli anche si fuggisse fotco la pro-
tezione del più porente, valorofo Principe del. mondo, come fu Carlo Magno;
econ i Latini diciamoanche,in grembo a Gione.. at

| CORRER a furia,, Eo stessos che far una cosa senza considerazione.. Vedi
sopra C. 5. stan, 41. E qui (cherzaintendendo se corse nel venice corre anche
nel tornare in dietro. '
 CHL nan hs cervello habbia gambe. Significa chi non ha havuto giudizio,o me-
moria di pigliare, o fare tutto quello, che egli doveva in uo viaggio, habbia gam-
be, cloe lo faccia in due, o più viaggi, ma qui il Poeta scherza,.¢ motteggian-
do Martinazza si serve del proverbio, per intender, che se ella non hebbe cer-
ucllo ad accettare, e venire al cimento del, duello, habbia hora le gambe per

ire
MENA le feste, Pa speti,¢ lunghi padi, Le (ele, cioè il compasso, s' assomiglia
alle gambe dell' huomo; e pero mexar le feffe s! intende adoprar prelto le
gambe, cioc camminar velocemente, correre.
ANT ANA. Intendi se n' entea nel Castello di Malmantile. /ntanare da Tana;
cava foterranea.
DIET RO ai muro faluns efte, Chi ha un parapetto di muraglia non e dubbio,
che € sicuro dalle stoccate.. B/se dal Lat, \&è, formato all' usanza nostra, de'
li niuna parola intera finisce in confonante. Ii Burchiello nella fine del primo
Sonetto. on funt non funt pisces pro Lombardi. il primo fant va seritto, e letto
funte come qui Efe, acciocché il vero torni. E in quel verso, per dire anche
spe 28' aliude a un vero Racconto, che si trova (critta nelle Craniche de' Pre-
icatori, alla vita di Giovanni da Vercelli Generale.
DALLE grida scampa il Lupo, Detto ulaciflino per mostrar la poca Rima,
che si fa di coloro, che gridano,

STANZA XXVL STANZA XXVIII.
Poich* egli vede in somma che costei, CHartinarza, che teme del suo male,
Alsrimenti non torna, fa i suot conti, Vedendo che 'l nimico se le accaita,
Che fara ben ch' ei vada a trouar Lei, Tre (caglionc'ba la porta,a un tepo fale,
Come faceua Macometto a i montis E gli da nel moftaccio dell! imposta «
E perch' ell ha due gambe, ed egli sei Ds poi dandola a gambe per ie soale,
( Mentre pero di fella ei non i/monts ) Senta dar tempo altempo,apigliar fofta
L arvriuerd:ne primaildestrier punge, Infacca nel falon, la done e il ballo,
GC? all entrar dé Palazzocs te lagsunge. Ed ei la segue foefo da cauallo.
O00 2 STAN-
.


476 MALMAN TIVE) 0

STANZA XXVIIL:
Appunto era seguito in sul feftino, \
(Come interuienein ee
Che due di quei che fannoda xerbine
S' cron per Donne disfidari «morte
L' un foreftiero, e /mentico pel vino he
L' a mi lafera,anch'eicenddoincorte } -
Ha Spada accato il Cortigian,ch'é l'altro,
Ma più per ornamento, che per altro, Alle spalle

ca STANZA XKX.
In quel ch'ei morde i guati,efaquei sees o Che im

Che van de plano all arte del Adirrilto, o

Ech'eglihafempriall'ufeioguoctes:

Dietro alla Serega giunge Calagrillo, Più des pie e

Calagrillo seguitando Martinazza entra con Lei nel is oO
che già fatto giorno ) continovavano a ballare,'¢ mette paura a
larmente a un-zerbiaello, che ¢flendosi sfidato con un suo tivale
fusse quelio, e pero si fuggi codardamente. 3
COME faceva Macometto ai monti, cioè se NON VengZORd Pr aendi si
noi da loro, che così e fama, che dicefle Macomeito, per mofti
miracolo, comando a i monti, che scenueilero gilda iui, “e veer
venivano 'dicesse; Horst: andremo noi da loro.

HA sei gambe, Cioè due sua, e- quattro del Cavallo. 0

GL1 da l'imposta nel moftaccio, Gli ferra la porta in faecia Che T
mo quel legname, che'chiude le porte,'¢ fincitre da: Launo poiter) B
Serrar la porta in faccia:, per intendere operare - fare in modo, the =
vicino alla porta non entri,'¢ ferrar (4 porta tn fa le calcagna, 'intendere
uno fuori di casa., come vedemmo sopra C, 3. st. 50. 'Nenehe serial
T imposta nel viso., o ne i piedi. ae
DANDOLA a gambe, Cominciando a correre. Vedi sopra 0.4
SOST-A. Riposo.. Vien dai verbo /ofare, chee: il. Laune/ ey
re,o fifere,

FESTINO, Trattenimento di giuoco;o di ballo. Vedi ropa Ca
celi Fefino., quasi felta piccola,, come quella, che'fi fa\felle'ca!
delle grandi, che si fanno-nel pubbiico..

TRESC.A, Così-anticamente dicevafiuna speeie di allo dal qual
hoggi Tre/cone specie di'bailo, come vedremo sotto C; 11g. U
Purg. c. '10, la piglia per specie diballo, dicendot
Trescando alzatol' umile §. rosacea
E nel presente logo e presa per adunanza wi. gence', che'
che la piglia il medesimo nell' daf.C.14.

-  Senza'viposo mai eralatrefea
Da trefea; trefeare, ches' intende operaré; e Tre)
telle, che vuol dir cose di poco prezzo, o stima. Vedi a
£ANNO da Zerbino, Fanno dei bello, e del galante,

ao Pe oes SP THE SS E

-

a aie aa Ae



¥

DECIMO CANTARE 477

\ TVTT AY architettura's ec. Vuol dite, che quel tale usava nel veltire ogni ar~
te, € s' aggiuftava con ogni maggior lindura, diligenza,edifegno.
GONFIO, Alticro, e superbo per la sua bellezza, come fa 11 Pavone,. che al

i detto delle persone più semplici,' gonfia perché si stima bello; donde poi pavonez-

giarsi, che vuol dir considerarsi, e vagheggiarsi per bello; E questo verbo <(pri-
'Me quel che vuol dir il Poeta nel presente luogo.

CREDE turar le Dame in Veffunio, Crede far perder 'tutte le Dame per il suo
amore. Crede, che la sua bellezza sia per far' ardere del suo amore 3 e Vefusio \&
il monte del Regno di Napoli, dove sono le voragini di fuoco.. rf
 HA paura del dilnnio, Cioè del diluuto delle percofle, le-quali spengono amor
nel cuore, e ' accendono nelle spalle ma differentissimo.

» VAN deplano all' arte del Mirtilio. Son-douute,-¢ si richiedono all' arte dell' ia-
namorato, da que! Mirtillo introdotto per innamorato dal Guarino avila fus
Tragicommedia incitolata Pafforyfide.

HA gli occhi a' mobi, Bada, oflerua, sta vigilante. E diciamo «' mochi, e non
allfaltre biade di maggior valore, perché essendo i Mochi cibo proprio de i Co-
lombi, sono da' essi prt, che l'alcre danneggiati quando sono di poco seminati,
€ peroé necessario haver J'.occhio, e badare con più attenzione a i mochi, che

Ll alte biade.

[pochi, Detto ironico, 'che significa moltissimi.
ha più cnor dun grille, EB' codardo, non ha animo, Sotto C.11.z9.dice,
Han facte di Leoni, e cnor di scriccioli, Appretlo i Greci per il contrario trovafi
Thymaleon, cioè Cuar dt leone, per vomo valoro(o, forte, cortaggiofo.

FA più capicale de' piedi, che del ferro, Si confida più ne i piedi, che nella sp2-
da; cioè Mimd più ficuca dife(a quella del fuggire, che quella dell' armi: e circa
queita voce capit aie. Vedi sopra C. 7. It. 82. e C. 8. st. 6.

STANZA XX\&K1. STANZA XXXIL
Toffo tornando l' amicizia in parte, Prima, che tra costoro altro ci nasca,

Si viene allarmi, che ciascuna armata
Ciò tien del altra un segno fatto adarte
Per darle atradimento la-pierrara:
'Di qui si viene a mescolar le-carre,
Tal ch' in vederlatante scompigiara,
Rittrandosi a dir badan le Dame:
Baha basta; non più; dentro le lame.

£ che la rabbia affatto entri frat cani,
E ms conuien fattar di palo in frasca,
E ripigliar la Storia. del Garani,

Chre dietro a far che'l Turacirinafia,
eAccio,tornato pot come i Criffiani,
Ad onca della Strega-ogni mattina
Ritorni a vifitar ta' Kegolina

Di questo sollevamento ciascuna-del'e-parti prefe sosperto di tradimento,.e per=
ciò si venne all' armi dentro al medesimo falone.. 'Qui l'Autore.lascia costoro, ¢
torna a Paride Garani, il quale egli latciodopra C. 8. st. 59..

TORNO' ? amicizia inparce. L amicizia si divile; cive ritornd inimicizi

“mMeera*prima. Parre t quella; che i Latini'dicevano parter, 'cioè fetta, fazione;
'onde Parziale, cioè affezionato,difenditore. Quel che sia parte per womo di spa-
da ch' egli era, e non di lettere, lo defini assai bene Farinata degli Vberti ti vec-
'chio, 'pretio.a Gio. Villani |. 12. Volere, e disuolere; € per oltraggi, e grazie ri-
Ceuute,

DAR ta pietrata, Dar colpo mortale; o conclusivo, dare a tradimento la pic-

trata


478 -MALMANTILE \&

trata è ver in quel verso di Plauto; leera manu fere lapidem; panem oftemit
altera, Che risponde anche per appunto al nostro proverbio ane,¢ (a
Sajata.; o% SW
ST viene a mescolar le carte, Si me(cold la zuffa. Vedi sopra C. 9. st, 35.
SCOMPIGLLAT A. Confula. Qui intendi, rottalapace.
LA rabbia, e fra i cani, Così diciamo quando vogliamo esprimere »
s' azzuffano indiftintamente: Ii Latino Xabies inter canes, Ee
SALT AR di palo im frasca, Passar da un discorso ad un' altro assai
dal primo. Far digressione. 11 Monofini dice, che con questa nostra
s' accorda quella de' Latini usata da Tertulliano, De calcaria in carbonariam..
Ma questa s' accorda più con quell' altra,. Dalla padelia nella brace. I |
Tertulliano nel lib, de Carne Chrifti dice così. dgitwr de calcaria, quod:
carbonariam; a Adarcione ad Apellen, i ose
LA regolina, Così chiamano i Ragazzi dell' ipfima Plebe Pornia ube
ga, la quale sta aperca in tempo di Quarefima, ed ivi si vendono frittelle,
Ii, baccala fritto, ed altre forte d' untumi simili, praticata, e frequentata da' ra
gazzi, ed altre genti vilissime, come era il Tura, che spesso v' andava.
STANZA XXXIIL STANZA XXXV.

Paride giunto in mexxo ai cafolari,

Ove messer Morfeo aun tempo solo
Fa dir di sia molts in Pian Giullaré
Strepitando fugeir lo fece a volo,
Sicognun deffo vanne a'fusi affari,

Ed ci, che Star non vuol quivi a piuolo

eAnzi dare al negurio [pedizione,

Domanda di quel luogo infor marione,
STANZA XXXIV.

Un gran Villano, un bnom acta matura

De' Quarantotti li di quel Contado,

Che perché ei non ha troppa [effitura,

ed è profontuofo al quinto grado

Junanzi se gi fece a dirittura,

E concerts suoi inchin da Fraccurrado,

Benevenga disse, Vostra signoria,

E Le buone Calende sl Ciel vs dia,

Jn quanto al Lupo egli e un! animale
Aa che aninial dich? io bue,
Via fiftol ds quei veri, un faci
C' ha fatto per sngenito gran dant,
E già con i forconi, e con le pale
J popoli affilliti rurto mguanno
Quin' oltre gli enno feati tutti riete
Per levar questo marbo da nn

STANZA XXXVL

Ma gli e un fetanalfo foatenato,

Che non teme legami, ne ea.
S? e carpito più voilti, ed ammagliatty
Ed ha ricifo funi tantogrofe,
Le bastonare non gli fanno fate
Chie' navha abriga rocehechethaledt
D' ammayyario co' ferri non c'è viy
Cb' egli e come frncar n' uaa matity

STANZA XXXVIL

La entro a quella felua ei si rappiarra, Che tutti gl' animaliycht ei raccat

* Perch' elia égrande,dirupata ye fitta, Cudfando gli trascina lvirittay
wacciocche nimu.un tratto lo cumbatta,, E chi guatar potesse; io.fopenfierd
Quand egli ha dato a'Socci la sconfitra, Chie' v' habbia fatto a' ofa uns "

Paride entraso ne i Calolari di Montelupo trovo, che tutti dormiyand,|
con strgpitare fece (uegliargli, ed havendo caro di sbrigarsi, proccurd
intormazione da qualcuno delle qualita ed abitazione del Lupo, ¢s' ak W
un Villano Sateapo del paefe, che gliene diede puntual ragguaglio. Ecol dif
fo., che.fa fare a questo Villano, mostra il modo di parlare del cont
KODZCp

tel on oe Ge Cee:

i el i


ee.

tak

Zo
7
a
a
q
i
5
j
:

%

DECIMO CANTARE:? 479

CASOL ARI, Intendiamo più case insieme in campagna scoperte, € spalcate;
qui intende di Montelupo, il quale se bene e Castello, ha più figura di Cafolare
per esser le Case tutte quasi rovinate, e distrutte.

MOREFEO, Favoloso Ministro del Sonno, il quale i Gentili tenevano, che a»
i comandamenti del Sonno suo padrone si trasformafie nella facia, nel passare, €
ne i costumi in qualsivoglia vivente, e però fu scritto: Hominum fittor Morpheus',
bestiarum imirator. Ed altri, Atorpheus, o varijs fingit nova vultibus ora, detto
Morfeo da Morphe, che in Latino vuol dire forma, faccia; onde noi Smorfie,
per brutto arto, o gesto fvenevole, che si facial particolarmente col vio. E
roan in furbe(co; mangiare. Qui dal nostro Poeta Morfeo, e preso per'lo Nef

fo sonno. \

FA dir di si a molti in Pian Ginllari, Fa dormir molti; perché colui, che dor-
me senza posar la testa, l' inchina, e fa con essa il medesimo atto, che fa colui',
il quale con efia accenna di dir di si. In Piaw Gintlaré intende nel letto, che anti-
camente'fi costumava il dire. / vo in' Pian Ginllari per intendere, io voa letio,
o mi pongo gil a dormire: Ma questo detto come oggi poco usaco è ancora poco
inteso. Per altro Pian Gindari \& chiamato un Borghetto di Case nel concorno de'
Vilage di Firenze non troppo distante dalla Città, che anticamente era de'Giul-
Jari cafata Fiorentina. Giullari, e Giulleria, dal Latino iaculares, vuol dir butio-
ne, e buffoneria, o allegria. Vedi il Varchi nei suo Hercolano; ed il medesimo
nelle Stor. Fior. lib. 15. Won gridavan con quella festa, e ginlleria ch' eran soliti.

STKEPIT ANDÒ fuegir lo fece 4 volo. Facendo romore, fece fuggir Morfeo,
cioè sveglid i popoli.

NON vuol far a pivolo, Non vuole star' a disagio aspettando; diciamo: Tener
uno a pivolo, quando lo facciamo aspettar più del dovere, o più di quel che egli
vorrebbe, quasi che egli flia legato alla nostra volontà contro a sua voglia, come
si fanno star legate le bettie a i pinoli, che sono pezzi di bastone, che fitti per le»
mura servono a i Contadini per legarvi le bestie.

DE' Uuarantotte del contado, De i più riputati, e Aimati del paefe; perché il
Quarantotto in Firenze è la dignita Senatoria, la quale e il maggior grado, che
godano i Cittadini Fiorentini.

NON ha feffitura. E' huomo ardito, e libero nel parlare', non ha vergogna, o
-riguardo o timore, che lo ritenga; e s' intende anche Un' huomo, che operi, c
viva inconsideratamente, Sefirwra chiamano le Donne quella filza di puoti radi,
che son solite fare da piedi, o nel mezzo delle loro vesti per farle divenir pib cor-
te, o per aliungarlo con sdrucire detti punti secondo, che torna loro in acconcio
dal Latino /ectura, come vuole il Ferrari, Le Romane moderne la dicono ritrep-
pio, quasi piccol ritiramento della velte, ed è lo stesso, che imbaftitura, che ve-
dremo sotto C, 12. st. 33.

PRESONTVOSO, Più che ardito, e poco men, che impertinente: Uno che
prefume assai di se medesimo, e s' arroga piii di quel ch' ei merita. Un' arrogan-
te. Daa. Purg. C. 11, dice.

Ed e qui perch fu prefontnofo

DA Fraccurrado, Da Fantoccino; da burattino; che intendiamo quei bam-

bocci, che dicemmo sopra o. 2. st. 46, 11 Bini nel Capitolo del Bicchicre <
Kuch



MALMANTILE

Questi perché son grandi, ancor son belli
Sends poca betta senza grandeRra,
\ wei paion Fraccurradi,¢ Spivitellig
Tra' canti Carna(cialeschi vi e un canto intitolato. Canta, ni

Fraccurradi, e Bagattelle, ove sono descritti, i giuachi, che
o giucatori di mano con tali legnetti, e burattini, detti-Frac

LE buone Calende il Ciel vi dia. Virconceda il Cielo, tutti i.
dia ij buon' anno.

SVE di panno, Sciocchissimo ch' io fone, Io ho manco giu
dicenci. Vedi sopra C, 6. st. 98. Lyf

VN fistolo. Le nostre Donnicciuole intendono Demonio, Diavolo. Vi
male maladetto, Bocce, gior. 7, Nou, 6. dufino a tanto, che il fiftolo u
Juo marito. Così detto dal filchiare de' serpenti, a' quali egli e affo

F AC/MALE, Huomo maligno, e da fare cout it
lefactor, Cavalcanti Storia lib, 9. cap, 11. Cerri huomini besti
i quali mai alcun bene fecero, e now hanrebbono saputo farne, huomini faci
futili, n't

PER ingenito. Per naturale instinto, che questo vuol' intender quel Ci

eASSILLIT I, \oucleniti, adirati. L' Affillo è un vermicello volati
alla zanzara, ma più grande, ed ha un forte, e lungo pungiglione
quando il Bue e punto, entra in grandiilima smania,¢ tem eda qu
tadini quando vogliono intendere, che uno è in collera dicono; Eel:
o¢ afiduo, Sula in Firenze ancora questo termine, ma per ischerzo,,
con ammogliati con i quali farebbe termine ingiuriofo, quando non fusse usa
in burla, perché e un dirgli Bxe, ve hfe

¥GVANNO. Quest' anno, Vedi sopra C. 6. st. 92. alla voce auannotte,

SMINOLT RE glienno feati tutti rieto, Qui intorno gli sono stati cust dietro ct
cando di pigliarlo, Enno, e la terza persona del numero plurale dell' indi
del verbo essere, hoggi poco usato in guesta forma fuor, che da i contadini;¢!
uso Dante Parad. C, 13, me

Non per saper lo numero, che enno oot

PER levar questo morbo da tappeto Per levar queita pefte, e questa tribolazion
dal mondo; J sappete serviva già in Firenze per strato ai Supremi
quindi /euare uno da tappero figuibca levario, o privario di quella dignita
quale e posto, che por pafiato in proverbio yuoi dire privare, o levar uno
qualsivoglia luogo, come qui che s' intende levar dal mondo,

SET AN ASSO, Satana; Demonio, dai Latino Saranas,come
nuovo testamento. Appelliamo Saranafo uno, che sia fiero, ¢
scrua di tal jua forza per far del male: e usato però dalle donne contro,
ciulli fieri, e vivaci, 1 quali chiamano anche WVabifi. In Ebraico:
onde il nostro Dante. i? acai

48a

Pape Satan pape fatan aleppe. ) arie

Evuol dire Aduer/arins, Aduer/arins nofter dsabolus, ate
CakPITO. Cioè pigliato con violenza, dal Latino carpere.
i Contadini. 'sili

zeseseEer: |

ae pee ee ee

— a


ee

=

fet

he

ete

RERt E

DECIMOCANTARE: 481

2. Vedi sopra in questo C. st, 18. il termine santo di cuore,
NN giù fanno fata, Non gli fanno male, o danao

'TANTO

 NON? ha 4 briga tocche, che Ube feoe. Subito, che ¢gli ? ha toccate gli pal-
fa il-doiore, non stima 'e percoise. Quando i Cant hanno toccato delle bastona-
te si squotano, e restano di guarite, che e indizio, che non sentono, O non cura~
no più il doiore, e di qui viene questo significato di squotere |e bufic, e ne hab-
biamo il dettato Tw fai come i Cams, es' intende cu (quoti le bufie, che significa
Non le cur:, non le senti, non ne fai thma, ec. Vedi sotto C. 11, tt 44,

MACT A. Con Vi longa. Monte di fatii dal Latino Adaceria,

Sl rimpiatra, Sinaconde, Vedi sopra C. 9, fh. 5,

 dVia40.. iano « Latino nemo. Won sopra C, 7. st. 89.
. £0 combatra. Gli dia noia;! impedisca,,

QVAN DL egis ha dato a' Soccs la sconfitta, Quand' egli ha messo sortosopra, o in
contufione le mandrie', cioè fatti fuggire i bettiami afialtandogli: Che Socciv.s'in-
teade quel beltiame, il quale si da a un Contadino per far' a mezzo del guadagno,
quasi dica a Sccio, cloe a compagnia. L'azione, che nasce dal contratto di So-
Gita, si domanda da' Legifti Azione Pro focio; Ma noi per Seccio intendiamo
una focieta, o compagnia particolare, ovvero una Accomandita di beltiame, che
si.da altrui., perché lo cuftodi(ca, e governi » a mezzo guadagno, e perdita. So-
Zi /poj pure dal Latino Sectas intendiamo quel, che i Latini dissero /edatis iure
Sodalitijs iunetus, o Buon forse dichiamo a colui, che non guasta mai, e che acco-,
da le conversazioni,

CA' ei raccatta, Ch' ei raduna, Ch' ci trova, e piglia,

CIVEF ANDÒ. Cioè¢ pigliando con voracita; rubando.

LU ritza, Cioè in quel luogo li, Termine ruttico, Dal Latino i rea, Qui-
via diritto; in quella dirittura, 0,, come 1 Franceli dicono, en cer endroit,

10. fo pensiere ch' e v? habia fatto a' ofa un cimitero, lo credo ch' ci v' habbia ra-
gunato una gran quantità-d' offa. Che Cimitero diciamo 1] luogo, dove si forter-
Tano imorty. Vedi sopra C. 4. st.2g.¢ C. 7. tt, 27,

STANZA 'AxXVill
Sta Paride afenttrio molto attenta,

Ada pai vedendo quant' ei si prolunga

. Frafe dice; Costui cs ha dato drento
Come quel che vuol far mela ben lunga,
Gli e me troncargli qui il ragionamento

 checio prima, che il ds mi sopraggiunga
40 polfa lasciar l'opera compira,
Peri gis. dice: O via falia finita.

STANZA XXXIX,

Poi ch' egls ha inteso dow' ei possa bartere

e4.un diprefja 4 rinuergare il Tura,
Lell'efer foleo il bofeose a' altre tartere,
Che gli narri costui, saper non cura:
La laterna apre,e il libro,od'alcarattere
Poa, vedendo., dar' una lettura,
Così leggendo senti darsi norma

Di quanto debba fare, in questa forma,

STANZA XXxX,

Vicino al boschereccio Scannatoia
Mentr' il froco di stipa vi riluca,
Palton grosse, Bracctale,¢ Schizzatoia
Co' Gucators a palleggiar conduca;

Ai rumbombar del suo diletto quoia
Toffe vedrà, che 'l Gocciolone sbuca
Keuei ricchi arnesi vago di mirare,
Che già in Firenze lo facean gunfiare >

Sta Paride attento al discorso dei Villano; ma conoscendo ch'egli era entrato
ip on discorio da non finir mai, lo fece chetare, e preso il libro, da esso compic-
BEDS tate;

se quel ch' ci doveva fare.

co.


482 MALMANTILE

COSTVI ci ha dato drento. Costui è entrato in un discorso da non'
fine; e me la vuol far (unga, Cioè vuol far' una lunga diceria,
OVVTA, E' lo steBo, che ors. Latino' Eia age. Termine, che
spedizione. ms " i ast
DOP' ci può battere, Cioè da qual parte egli habbia andare per ir:.
Tura. Tey et
APN dipresso, Alquanto vicino a dove egli sia. Si dice o a ited “tid
vel circa, Dal dirsi per esempio: Furono tanti, quanti io v' ho detto vel cireay
Cit, o in quel torno, haa alle se
RINVEKG ARE. Rinuenire; Ri; Ri iare; Raccapezzare. o ist
ALTKE tattere, Altre zacchere, minuzie, o circoftanze di poca considera- a
vione. Se ben Tattere per (cherzo s' intende una specie di malore, che viene in 4/4
torno al stesso per crescenza di carne. i ?
CARATTERE, La forma, o figura delle letcere dell' Abbiccl. Voce latina
tolta dal Greco Character, ¢d i) Monosino vuol che itia lio dir carartolo, ma si
non fo per qual cagione, se non fusse per allontanarsi dal Latino, che per altro te
non ho letto tai, ne sentito dir carartoso, se non a qualche Villano del tutto ru- ne
fico. eee
SCANNATO10, S) intende il luogo dove s' ammazzano i buoi, edaltrebe>
flie, ma qui intende quella felua, entro alla quale si nascondeva il Tura,¢las Ki
chiama scannatoio, perché quivi il Lupo scannava le bestie,
BRACCIALE, Manica di legno dentata, della quale s' arma il braccio per laf
giocare al pallon grosso. Vedi sopra C. 6. st. 34. any Ta
SCHIZZ AT O/O ( qui intende il piccolo). Strumento d* ottone, o d' altro Ne
metallo fatto a foggia di canna da crilteri, ma assai minore, e serve per metter '
vento in qualunque luogo con violenza, come si faa gonfiar palloni,0 pillotte, wt
o per (chizzar liquori; e 'i maggiore per far serviziali. Latino e/yfer detto così,,

quasi frumento inondante, e lavativo. Vedi sopra C, 3. st. 14. Che

PALLEGGIARE, Dare alla palla, o Pallone, mandindoio, e rimandandolo Che
per tra(tullarsi, e per avviare 1) giuoco; ma non giocare regolatamente, Onde» Ry
quando uno tira ia luogo un neguzio, coll' avviare chi glielo raccomanda, 2 un? Cad
altro, e che quello lo rimanda, al primo, e tutti due si accordono a burlare il Tes,

pover' huomo; si dice + Tra loro se ¢a palieggiane; che in Latino forse i direbbe Giaj
Coludunt, ast SOE?
GOCCIOLONE. Si dice a uno, che ta guardando una cosa con grande atten- Tu

zione, e con desiderio a' ortencria, e propriamente si dice di quelli innamoratt » Beri
che stanno i giorni interi appit d' una casa a guardar la dama, che € alla finellsa, Ng
¢ si coniumano, € si struggono a poco a poco, e per così dire a filia a flilla, o vu
però dice Gocetolone al Tura, e vuol' esprimere, che egli era innamorato di que L
guarnefi. Lucrezio lib, 4. Pariando degl'innamorati. ee fog
Wamque voluptatem prafagit multa cupido, Ri
Hac Venus eft vobis, bine autem est nomen amiorit; - Ca
Hine ila primum Veneris dulcedinis in cor B g

Stilauit gutta, o fucceffit frigida cura, ?
CHE già lo facean gorfiare, La voce gontiare vyol dire Andar superbo, comes oy

4
ra
53,



DECIMO CANTARE.
dicemmo sopra in questo.C. st, 29. sed il Poeta (cherzando con l'equivoco di gon-

fiar

3 ma in effetto vuol

483

Ic pillotte, e palloni, che era il mefticro del Tura, come acccnnammo sopra

Gf st, 47. pare, che voglia dire, che quegli arnesi eran caula, che il Tura (eo

andava sup poi dire, che quegli aracfi eran caula ch'ei

J Sontava le milous » ei palloni, e che egli gonfiava la pancia, buscando per mez-
zo

imi arnesi da comprar roba per empictia.

— STANZA XXXXL
Paride in soofe fatice ubbidisce 5
Accender fa le scope, e intorno al si
 Già questoje quel st spaglia ed alleftisce
 M faa braccialeye si comincia il gimoco;
Al suan del qual! Amico comparisce,
M4 ritenuto, perch' e vede il fuoce,
 Elemento, che vien dall' animale
Fugritaper instinto naturale.
STANZA XXX AIL
NGarani che fava alle-velette,
 Fedendo che'd Compar viene alla cesta,
Che te scope si spengano commerte,
 Edin un tempo a i Giscator da festa:
WD un batrer docchio il giuoco si difmeste
La fipa si sparpaglia ye si calpefta;
Tal che sicuro t' animal ridotto,
Va Paride pian piano, e fa fagotto.
5

STANZA XXXXIII.
Ciò ch'é in giuoco in nn fascio egh ravvia,
E tra gambe la ferada poi si caccia
A tutto strascicanao per la via
Con una fune a otto, o dieci braccia.
Spinto dal genio a quella ghiortornia
Da lunge il Tura scguita la traccia,
Come fa il Gatto dietro alle vivande,
E il Porco a' beveroni,ed alle ghiande.
STANZA XXXXIV.
Vaghecgialo, s'aliunga, xappa,e mugola,
Talor 8 appressa,econlexampe iltoces,
Hor mostra shavigliando aperta l'ugola
Hor per leccarlo appoggiavi la bocca,
Tutto lo fina, lo roniftia,e frugola;
Così mentre il suo cnor givia trabocca
Ej, che non rocea per letizia terra,
Entra nel Borgo, e in gabbia si riferra:

TANZA XXxxv.

Perché Paride fa ferrar le porte,
E poi comanda a un branco di Famigli,
Che quiui farti bauea venir di Corte,
Che di loy mano l' Animal si pieli;

Ma i Birri, che buscar temean la morte
Non voglion accercar simil consigli,
E fan conto ( se ben' ei fa lor cuore )
Che @ passi cutrania 2 Imperadore,

Paride in ordine a quel che trovd scritto nel libro datogli dalle Fate, fece acc
cender il fuoco d' avanti a) bosco, ed attorno vi messe gente a giocare a} pallo-
Ne: a quel romore il Tura ulci dal bosco, ed allora Paride fece un falcio de'brac-
Ciali, pallone, ed altri arnei, e legatolo a una fune lo fece strascicare per la
scada, la qual conduce al Castello di Monte Lupo, dentro al quale i condusse il
Tura, seguitando quegli arnesi, e Paride fece ferrar le porte, ed ordinò ad alcuni
Bi tri, che quivi haveva per questo fatti venire, che lo pigliassero, ma essi
impautiti non vollero accoftarsi.;

C4ALLEST/RE. Metter' all' ordine: Approntare

L) AMICO comparisce. Cioè il Tura esce dal bosco,¢ vien fuora spinto dal gu-
sto di vedere il pallone.

RITENVTO., Renitente; cicé non alla libera, ma con qualche timore per
¢aufa det fuoco:, del quale il Lupo n1cura'mente ha timore.

ST AVA alle velette, Stava osserv'ndo. Vedi sopra C. 7. st. 67. It Burchiello
nella Novella del Medico Bolognef=. e dello Scolar semplice dice: Andando ¢ri-
dando cerci tutta ia casa, e tronarlo non gli fu ordine, onde tratte dalla disperarione si

Ppp 2 parti,

484 MALMANTELE:
parth, e lo Scolare, che staua alle velette Vitorhato in'cafay ec,
. IL Compar viene alia cesta, Cioè Animale vien fuor
zimbello de i braceiali 5'¢ palloni, ec, iy HgOW
DA sia ai Giuocatori, Ha vestardi giocare; Licenzia iG)
agli Scolari vuol dir Licenziar la Squola, e di qui dicendosi dar,
cenziare ogni sorta di lavoro, Daag
IN un barter ad? occhio, Inun momento. 1 Latini pure ditono-Jr%è
SPARPAGLIATE, Spandere contufamente, e ae ee
come si fa della paglia, quanido'si batte, e si spoglia il'grano'. 1 Pulei dite:
Sopr' alle spallela treccinsperpagtia., 0
FA fagotto, Fa un fa(ciode rbracciali, paltomt ec. Par fagotto; e 10!
quasi, che far le balle per bactersela, per andarsene. Latino v4/a colligere. ~
Sl caccia la via fra gambe, Comincia a camminare. Latino + viem,
SEGVIT A la traceia, Seguita, o va dictro allapefta, oalla sed étol-
to dai Bracchi, i quali si dice/egwitar /a traccsa, quando mel cercar della %.
ec. fiutando seguitano quella strada, e quel tratto', per dove ella ha tirato;
per dove e passata: di qui habbiamo il verbo inzracetare-detto sopra C, 7. st,
BEVERON!, Così chiamano i -nostri Comtadini quella bevanda grossa fatta di
crusca, € d'acqua, ec, la-quale danno a i Porci. Vega eh aie
LO vagheggia, 'Lo guarda aftewuofamente.. Sivaledi-questo verbo vaghectiog
per esprimer il gulto, col quale 1) Tura guardava quegli arnesi;:essendo tal ¥
proprio degl' innamorat', Vedi sopra C. 7. st79. (4 aay
MVGOLARE, Buna voce indistinta, e che non finitamuore fra i denti.
ROVISTIARE; Ravoltolare, netter soflopray Forte aeglio-romifia dal verbo
rovistare, che vuol dir Muovere da un luogo all'altro. Ji Pulci., Morgante va
rouiftando ognt cosa.. Hh Wx
PER letizia non tocca terra, Sopra C. 9..st.63, Per V allegrezza non può*star
n¢ i panni, 'che € lo stesso; e figaisca haver'allegrezza', o gufo grandissimo; Si
dice ancora; ma in modo batio, Lacamicianon gli tocca il sedere, Ml Boccaccio
Novella 32. iam
FAMIGLI, Qui sintende Famighi di Giuftizia,cioè Birri;la famiglia debPode-
sta,dal Boccaccio detti fergents, quali ferxientes, siccome'da noisfamigli,cic' fa
FA conto, che pafii 'dmperadere,, Finge di-non intendere, o-di non lentire qu
che si dica\, Detto forse questo dal tempo', quando'era l'dmper: rec
vanni Paleologo-in Firenzeal Conciiio »che per cfersi già tata familiate la un
vista, e forse, mancandogli i danari., non comparendo:così pompola, ne Co
bella compagnia; e appagata anche dalla prima volta in fu, lacuriosita; quan-
do passava per le strade, non doveva far muovere la 'gente come prima, e come
andò egli arrivd; Onde si venne a dire, quasdo uno non si cura di qualche co-
f: Facciam conto, che passi lo Laperadore, t a7%8 one
: ST, AN 2, AoREXKEV End OV
Poiché gran perso ha i porri bapredicate, Senza pin (har a burtear via il fiato,

E che fan conto tuttania cb'eb cantiv, Totti di mano abc. iiguanti,
Pero che.da i Ribaldi gli vien dato: Bisogna, dice, con questa canaglia
Li udienz.a, che da il Papa i furfanti, Far come il Podestdidi

'AN:



DECIMO CANTARE: 485

ZAXXXXVIL o» STANZA XXXXVIIL.
ds caps Si resta il Lupo, e'l Tura buomo diviene;
Ma non pero, che libero ne sia,

ad una delle [ue legacce

a addosso al' Animale C' ambi fone appiccati per le rene
eee a uso di bifacce: Formandoun Leong ha e la Bugia,
r di tal.concia dé cauiale Dice Turpino,e par ch' et dica bene,

Ch' essendo questa si crudel malia,
ina di iupo,ed una d'buomo fembra, Lon erano.a disfaria mai bastani
di sua [pecie oguunna ha le/ue mzbra, » Gli odor birreschi semplici dei guanti,
4)! STANZA IL
opri tal masserizia E Paride, che gra ' chbe notizia
molto pin fatto le mani, Da quel suo libro,si da quint ai cani,
 Percheglincants in man delaGinffizia Perché pin oltre il libro non ispiega,
i fichi- alla nebbia vengon van, 'Ona' et fa conto al fin di tor ia lega,
ide veduto che i Birri non ubbidivano, ed havendo per avvertimento dal
bro datogli dalle Fate, che gl' incanti rimangon vani iv mano della Giuftizia,
sdiede a credere che haveffero tal virtù ancora i guanti dei birri, e per questo
f eae al Caporale, e gli mefle addosso alla bestia, la quale si converti
Induce corpi appiccati insieme, che uno d' huomo,¢ I altrodi lupo. A tal me-
tamorfofi resta Paride stupefatto, e non sapendo che cosa farsi, perché il libro
bon inlegna da vantaggio » risolué di chiamar due fegatori per(eparar It Animal
bruto*dal razionale. In questo mostro il nostro Poeta imita Dante nell' Inf. C.
 25. nella commiftione di que! Serpe con ' anime di quei cingue Cittadini Fioren-
i € la delcrizion di tal mostro comincia al verso: Se tu sei hor Lettore acreder
dente,
 PREDICARE a' porri. Predicare al deferto,, Affaticarli in-vano a esortare
uno.a far bene, che i Latini dissero vento logui; Surdocanere.
 PANNO conto ch' ei cantiB lo stesso, che dar Pandienza che da il Papa ai furfanti
che ia fuiteza vuol dire n6 fare stima delle parole d'un0,0n6 badare a quel ch'es dice.
CAPOXKALE. 'Capo di squadra di birri, Grado che si di anche sia i Soldati.
Vedi sopra'C.'9. stan. 2.
 BAR come il Podestà di Sinigaglia, Cioè comandare, e farda se.. I) Duca di
Calauria Sigifmondo havea aflediato Sinigaglia,nella qual Terra era per Gover-
shatore foltituto da Gio: de Castro, Petruccio Piccolomini; Costui tentd di ab-
' la Terra, dicendo esser. meglio uccello di-campagna, che di gabbia,
¢d a lutaderiva il Podesta, ma i Cictadini featendo questo dissero di volergli get-
“tare dalle fiaeftre se più parlavano d' abbandonare la Città, e vennero tanco in
odio ved in disprezzo de i Cittadini, che. quando comandavano noa erono ubbi-
Giti, edi qui venne il Proverbio: Far.come it Podestà di Sinigaglia, cioè Coman-
dare, e'far da se, Cavalc. Scor.:
 Deeaa + S'intende quei legami, con i quali ff legano le calze, cingendo
ambe.
MSACCE, Così chiamiamo due facchetti appiccati "uno contro all' altto a
'due cigne, i quali si mettono a traver(o ai cavallo, ec. sopra il quale si cavalca.,
'€ servono per porcar robe, come si fa con una valigia, sono appellate —

me

vit

486 MALMANTILE

bis facche, due volte facche, o facche a doppio. Lat, AMdantica Bocce,
nov. 10.5, Haveva Frace Cipolla comandato che bea guardafle, che let
»» (ona ada toccaile le cose sue, e (pectalmeace le sue bilacce nelle q
»» cose rare. B più otto nella medeGa novella. La prima cof che venac
» presa fu la bifaccia, agila quale era la peana. weet
CONZI 4A. Quando Gi dice coacia di guaati s' iateade profumameato,

si dice guanti di coacia di Rona, di Venezia » di Spagna 7 ec. e 8 intend
mati alla foggia di Roma, ec. Qui dice concia di Cauiale, cioè feteatt »
fragore, o feagraaza e Detto ironico., Were ta
LA Sugia. La Bugia Gi figura uaa Femmina con due facce differenti, come
@' orso 0d' huo ny, o di lupo, e d' huomo, come è aci prefeace luogO,
DICE Tarpino, Scherza cone fa sopra C. 2, stan, 31, autorizzando en

te (un Novella com i detti di Turpino, come fa ? Aciofto. lve
MALIa, Iacantefimo. Suregoneria. Vedi sopra C, 8, stan. 52. Donde 44-
liarda una strega. iy
T AL maferizia, Iatende i guanti del birro. cust cee
Sd aicani, S'adica, Quando uno per la stizza grida, e fa altre dimostra+
zioni d' impazzienza, o di rabbia diciamo; Si daa' can, Vedi sopra C. the 10,
STANZA L STANZA LiL...

Per ciò fatti venir due Marangoni, E morta re la dd per cofacertay =
Con tutto quell' ordingo, che s' adopra M4 quel Demonio insieme strappicch,
eA fegare i legnami, edi panconi, E qual porco ferito agolaaperta

ef dinider il Moffro metre in opra;
Mitre la fegaim mero ai dusigropponi
Scorre cosiva il mondo sortosopra
Mediante il rumor de i due parrienti,
Che un fa d' urli el altro dilamenti,
STANZA LL
Pur senza ch' inraccato elit habbia un offo
La [ez infino ail' uitsmo aileefe
Lasciando il Tura libero, ma rosso,
Dietro ds sangue com' un Genone/e;
La Be(tia gli volea tornare addosso,
Ma Paride, che (ubito ? intese
Presa la (pada la cagio pel mexrd
Pensando di madarla un trattoalrezro,

aa
Per dinorarlo forte se gli ficeay - -
Ed eslic! alt' incontro stan. all! eta,
la [a La testa un sopramman gli appicedy
Ch in due parti diuifela di netto
Com' una tefticcinola di capretto.
STANZA Luh
M4 ritornato a penna,¢4calamaio
Pur quello Heffo a Paride si volta,
Che per veder il fin di quel mofeaio
See' fulfe mai possibile una volta,
Mena le man chee pare un Berrettait,
Ed a chius' occhi pur suonas r4ccilta
E dagli, e picchia,risuona se mA 7
4a forbice, t e sempre bella.

Paride fatti venir due Segatori d' asse, fece fegare il Mostro in fu It artasatu-
ra deli' huomo con la beltia, e così gli fepard; Ma la Bestia tentava di
carsi onde Paride caglid la Bestia pel mezzo, ma eifa presto strappiccd, B qui
il nostro Autore immita l'Ariofto nella favola d' Orillo; levata da Vergilioacil
Eneide, che finge un tal' Erillo Re di Paleftrina che haveva tre anime, onde era
necessario tre volte ammazzarlo per finirlo a.

tHARANGONT, Si dicono i Garzoni de i Legnaiuoli che lavorano peropra,
quando in una bottega, e quando in un' altra a tanto il giorno, e non jn
una boticga a falario di tanto il mese; ma qui l'Autore intende Segatori di le-

goami,

a ese

—————

eee

me

=

- +e

DECIMO CANTARE. 487

| guaind 3 €gli ordinghi, che ? adopra, sono la fega a due mani, lima per mettere
} Gags denti, e il cavalletto per adattarvi sopra quel materiale, che i dec (e-
ox. cavalletco si chiama pietiche. Vedi sopra C. 6. stan. 6g. alla voce im-

 PANCONT. Sono afi grosse circa un quinto di braccio, le quali si rifendouo
per farne o assi più sottili, che si dicono panconcelli, o per farne correnti.

GROPPONE. S' intende la parte di poms di tutti gli animali, o bipedi,o

yadrupedi, e lo diciamo ancora codione, ed è propriamente quella parte che re=

fra le natiche, e le reni. Vedi sopra C. 6, stan. 69.

VA sottosopra il mondo, Lo strepito confonde l'universo. I Latini pure dicono
Mundi fumma readit ima, o ima fumma;¢ vuol dire, che jo ttrepo era gran.
'dithmo per le strida del Tura, e per gli urli del Lupo.

 ROSSO come un Genoxeje, \&* in Firenze una Compagnia, o Confraternita di
Secoiari detta de' Genovefi, perché e formata di gente di quella Naziwne s Co-
storo hanno per coitume d' andar proceflionalmente la sera dei Giovedi Santo a
vifitare le Chiese, si battono le reni ignude con mazzi di corde entrovi alcune
ficiie di metailo acute come quelle degli sproni, e queste forando la pelle ne trag-
gono il sangue, il quale bagna loro le reni, ele tigne di roflo; E di questi in-
tende il nostro Poeta nel presente luogo.

. tH ANDARE uno al rezzo. Mandare uno nell' altro mondo, df fresco, cioè
il corpo suo sotto terra. Ammazzar' uno. Rezo, vuol dire un luogo dove non
arrivano i raggi del Sole per interposizione di che che sia, e si dice anche, me-
riggiv, bacio, ombra,¢ uggia. Vedi sopra C. 6. stan. 75 e c. g. stan. 44.

ST.AV-A aif erta, Stava ocnlato; stava avvertito. Erta si dice la talita d'un

BRIO; e are all' erta e termine di caccia, percht la Lepre ha per propria di
for sempre alla volta della fommita de' monti, per non esser così facilmente
arrivata, e pigliando i suoi riposi, scoprir paefe, e minchionarc icani; e pera

in caccia State al? erta s? intende Habbiate l'occhio, osseruate; il che ¢

poi pafiato in dettato comune a ogni cosa.

PN sopramman gli appicca, Gli da un soprammano, che è quel colpo, che si da
¢ spada, bastone, ec. cominciando da alto, e calando a balio. Vedi sopra

5- stan. 41.

D1 netto. S' intende lo taglid pulitamente in un fol colpo,

TESTICCIWOLA, Le telte degli Agnelli, e de i Capretti da noi si chiamano
Teftucinote, e per friggerie si tagliano nel mezzo per lo lyago in duc parti ugua~
li; eda questo taglio afiomiglia quello, che fa Paride alia tetta det Lupo.

4 penna,¢ 4 calamaio, Per ! appunto. Vedi sopra C. 2, stan. 19.

VEDER il fin di quel mofeaio, Veder il tine di quetia cosa noioia. Vedi sopra
C, 4. stan. 9.e c. 9. stan. 51.

MEN A le man ch' ei par ux Berrettaio, Menar le mani dicemmo sopra C, 1, st,
7. quel che significhi, e qui intende che. mcnava le mani con ceierua,come fauna
1 Berrettai, e Cappellai, che nel felcrare i cappelli, o berrette menano le mani
Prelto in riguardo dell' acqua bollente, con ia quale si fa tal lavoro,

SVONA a raccolta, Continova a perquoter a jungo, che così suona la campana;
quando suona a raccolta di popolo per le prediche, ec. ed 1 verbo sonare si

° gailca
+

ee

- — “gk ae
2 488
a
488 MALMANTILE |
gnifica anche perquotere, ¢d e della medesima natura, che il
habbiamo detto altrove. 6 ay eee
DAGLI, picchia, risuona,e marvella. Questo di dire
re uno, che adopri ogni fea induftria, per fare una cosa perf
do più vole le diligeaze. Vedi sopra C. 7. stan, 16. Similitudine,
tratta da' fabbri, quando Javorano il ferro sopra l'incudias; Qui:
d' Orazio incudi reddere versus, mettergli alP incudine, sotto
critica. Cioè efaminargli, rivedergli di nuovo.con somma, rigorofa
diligenza. La nottra maniera; Barrere il ferro quando è caida, ebbe
meate da questa prontezza, e macitria talieme, che si adopra per lavorat
nalmente l'dcudir degli Spagauoli, che vale aixeare, voce ormai si
è fatta dal Latino ddcudere, ciod battere insieme il medesimo ferro.
dichiamo per esempio. La prego a volere accudive « quefke megorio; o si
FORSICE.. Questo termine significa ostinazione,, per esempio. fo 2,
che tu non faccia la tal cosa; e tu forbice, cioè Tu ostinato l'hai voluta |
modo. Dicono che venga da uaa Donna offinata, e capona,, la quale
chieito al Marito un par di Forbice, e non havendoglicie il marito mai:
te,ella ad ogni cosa, che il marito le domandava rispondeva: Forbice;
impazzientato da queita sciocca ostinazione,le proibi il dirlo
più lo diceva, per lo che il marito la baflond, ma. non per: ella se
maneva, ficche egli un giorno sopraffauto dalla collera la gewo in ump
cila fino che potette parlare sempre dite; Forbice, ed in ultimo goa p
valersi della voce, si valfe delle mani cavandolg fuori del' aequa con le
giori alzate ed allargate in figura di forbice,per mottrare che moriva |
oitinazione, e caponeria. Questa novella e vulgatitiima fra le nostre
io ho trovata tra una raccolra di efempi facta da un Buontempr
mano del medesimo tengo fra i miei nianoscritci. 2 e
Lit sempre quella bella; L' e sempre quella medesima. Questo yien da un
co, 1] quale andava accartando,¢ cantava una cerca orazione al suono di un
tarrino, fermandosi alle porte de' suoi benefators i giorni destinati;
venuto a fastidio, do sempre la defima cosa, inci:

quelli, che gli facevano l'elemosina a dirgli, che se non cantava q 'ae
orazione non gli havrebbero dato più nulla, ed egli rispandeva; Pa
se', cht domani ve ne vaglio cantare una bella, Ma pecche il Povererto ai 4
se.non quella, tornaya l'altra mattina, e cantava la steila, laonde i f ”
fattori.accortifi, che il Meschino non ne. fapeva altre compathonaadolo, giù te
cevono. L' è sempre quella bella, ed intendewano l'e tempre quella 1 ig
che e poi venuro in dettato, e significa noi siam sempre-alie medelim a
quanto racconto ancora fra gli scriti del medcfimo Bugnrempi top z
pucato ali' origine del presente dettato. ren “a i
S. TAN ZiAisbl Moni tap 'y¢:
Tal ch' ei si scofta none, e dieci passi, Pervia gli anuenta m
E piglia fato, perch! es pronar vuvle, i
Selavirtude a forte gli giouassi, i;

C* hanno! erbe, le pictre, e le parole;



489
gout STANZA  i
recasse a scorno, Resta in parata', molto gira il cnaré
alle gioftre,e alle quitaney | 'Pimceis pikes anc ielibbienicfie,
we b gli vada incorna, > Merce ch ei fache'l Diauvloe bugiardo,
EB latrartigo' sassi, come un cane; “E quanto en sia furtile,¢ filigrosso;
i, ver ch' e' fusse ! apparir del giorno, oPercia si merte un pezro a bellofguardo,
; L! Ombre,il Bau, ele Befane oCredendoognor che gli faltasse addosso,
Sparyce affatco, e più non si rinede, Aa poi ch' ei vedde omas d' ¢/ser sicuro
Ma Paride per questo non gli crede | = Andò all Ofte, e cauollo di pan duro,
Vedendo Paride, che quel Mostro si rappiccava sempre » e che ci non trovava
'modo di liberarsene per ferite, che glisdette, gli venne'in pensiero, che se era la
Werita, che in herbis, verbir, \& lapidibus stesse la virtù, poteiic essere che alcune
di queste cose havetie virtù di fare sparire, e svaniresl Moltro; e pero preso il
 [xa dove, il quale era pieno di parole, e dliverle erbe, € de i faili ogni cosa tirò
   addosso a quel Mostro, e l'indovinò, perché subito egli sparì, ed il Tura rimase
   libero”, 'Con tutto questo,Paride non si fidando, stette buon pezzo a osservare;
ma veduro, che il Lupo non compariva più si parti, e andò all' olteria a man-

Patt. i
Ors ' fiato, Cioè si riposa,
. MLAEST RO Grillo Contadino, 8 nota la favola'di Grillo Contadino, il quale
per fardispetta @un sue fratello Medico sche non gli volle dar parte d' un tefo-
F0, che infizme! havevano trovato, si fece Medico anch' egli,¢ con i sui forcuna-
a fiti's' acquifto la grazia del suo Re, non solo per havergli: rifanata la
cavandoie una tilca di pesce della gola con ungerle ilc,..., ma ancora
per haver saputo indovinare i segreti del medesimo Re, e chi erano coloro, che
a lui rubato havevano, in somma fece diverse scioccheric, le quali tutte per gli
} spares fidondarono in stima del suo valore, e l'accreditarono per un valoro(o
Medico, e grandissimo Indovino, come si legge nella di lui favolosa vita, o di-
Ciamo spiritola Satira.
WINT ANA} Bruna campanella, che si tien sospefa in aria (oftenuta da una
molla dentro a un canacilo, alla quale per infilarla corrono 4 Cavaiiert con la
Jancia', come fanno anche'al Saracino, che dicemmo sopra C. 4, than, 57. € si di.
Ce ancora Chintana, Varchi Stor, Fior, lib, 15. Fecera metrer della rena a! avanti
al palazze, ed appiccare /a chintana, Dai noltri Ragazzi e detta corrottamente
Timana 9 ed è iatelo quel lor passatempo, che fanno, infilando una zucca fresca
in una corda, e pottala in aria attraverso a una Arada corrono con alle la mano
@ dare in detta zucca, unmitando i Cavalieri, i quali corrono alla quintana, 0
al Saracino, Dice che Paride era avvezzo alle gaintane, e alle gioffre [che nel
Prelence inogo son fitioninu; s¢ ben gioftra's' intende quando i Cavaiieri corrono
a corpo a om 70 al Saracino, e quintana significa quello, che diciaino qui fo-
Pra) perché Paride haveva più aout militato im Spagna, dove haveva cfercitaco
1 jor! gradi della mulizia, e tornato alla Patria tu dal Serenityaio Gran Duca
fatto Governatore deija forcezza veochia di Livorno, ed hunorato del titoio di
Macttro di Capo, I nome tuo era Andrea Parigi, fu fratello d Aifon(o, e di
Paoio detto sopra Papirio Gola, è Figliuolo di Giulio, e fu come custi questi va-
sa = Qqq jen-

-

Se

:

o

i

=i

aa

—

' Ye
*

490 MALMANTILE™
lentissimo Tngegnere, € periti archi Qui
Ferrari cusi. Ludus equeltris,cum diretta in encun fimulachrnn:
gehtat, bala incurritur, Alcunt han detto come Vguecione Pifano.
zionario, che Già così detta dalla quinta parte della piazza yin
tri, come Balfamone sopra Fozio da un certo Quinto inventore 2ed
la vera origine mottra il Pertari essere da Comrus.cioè ee i
punta di ferro; e si raccoglie dabtitolo nel Godice:, de i,
radore chiama questo giuoco con voce Greca Kynranos., In ordi
Chintano, e non Chincana pare, che lo chiamaile, se sha a
ma, Fazio degli Vberti nel Dittamondo.:
Gionani bigordare alli Chintani,
E gran tornei, ed. una, ed altrag
Far si vedea con giuochi nuoui se ferant. -\

jofira '
>

CALAPPOLERIE. Cosa di poca stima: oda farne poco conto i “Apine; '

triceque,¢ buttubata, V. Feito, e ivi sopra lo Scaligere.,
BAV,e Befane, S' intendono quelle Larue inveatate dalle Balie per far paura
ai Bambini, come habbiamo decto sopra C. 2. stan. 50. et
REST A sn parata, Si ferma in guardia, cioè con la spada pronta, ed in posi-
tura comoda a ferire, E' termine da schermitori. yori
MERCE', Con la prima, €;, firetta, ela seconda longa, vuol dir mercede
che profferito al contrario vuol dir mercanzia: Nel modo che:è detta nel pre-
sente luogo, ed in molt' altre occasioni mere vuol dire per causa di ciò: qual di
ca io riconosco tal mercede, tal benefizio da questa cosa, o da i,
ec, ficome Paride riconosce questa mercede, o benefizio di non si fidare del Dia.
volo dal sapere, che quello e bugiardo, ed ingannatore. Questoidetto e lo'ftelio,
che Grazia del marcello, e degli foroni, che vedemmo sopra in que(to C, fran, 20,
1L Diauoloé futtile, e fiia grofo. 11 Diavolo è fagace, ed inganna l'huomo,
facendo il goffo, ed il balordo. * inet
REST Aa bellu (guards, Reled guardando attentamente. Bello fguarde® una
villa poco lontana da Firenze: e per la similitudine che ha questo nome bella/enar-
do con il verbo guardare si piglia in detto significato. pn amaetir
 CAPOLLA di pan duro, Mangid adai. Gli mangid tutto il pane, che haveva
in casa, gliclo rifint. Detto usatissimo per esprimere Aeangiare assai ee,
spy paler

ays Nae

FINE DEL DECIMO CANTARE. eae



|

eee
eh ARGOMENTO, '
St

Cangia le dance in rissa un? accidente, a
iSe

Fuggonfi Bertinella, e Martinacza,

-VNDECIMO CANTARE,

Vien fuor Biancone, e fa morir gran gente;

5 Ma gli Orbi a tui fan poi sentir la mazza, 6%
es Da Celidora, e da Baldon possente 33
ee Mezza defirntta e quella trista razza; th

Taghanfi a pezei in quelle squadre, e in queste '“*
E così in ata? fanfi le feffe. e a ge

2 —
RAPALA AAAS A

om STANZAL STANZA Ik
Chi mi.dard la voce, ele parole ui ci vorria chi scortica L' agnello,
'antia dir la guerra indiavolata; Es al mondoé persona più inumana,
Ond' oggimas dara le barbe al Sole o descriver la frrage,ed il flagello
Bertinella con tutta la sua armata; Che seguir si vedrd di carne humana;
C'alCiel Gagliarde alzando,e Capriole, Ch' io già oni sento, mentre ne favello,
\Farò.verso Volterra la Calata, A tremito venir della quartana.,
. Efe d' amor canto con cetra in mano, E n' ho si gran terror, ch'io vi confefi0,
<Derd col ferro il ve/pro Sicilsano ? Che mai più de'miei di farò quel desso,

Tinoftro Poeta volendo.nel presente Cantare narrar la battaglia seguita ia Mal-
mantile, e le crudcita grandi, che (uccessero nel Palazzo della Regina, dice, che
a fac tale descrizione vorrebbe esser un' huomo sanguinario, quanto è colui, che
scortica giù agnelli; che non si spavencerebbe, come fa egli acl rammentarsi i]
grande firazio, che fu fatto di carne humana in tal batcaglia. Qui immuta Dan-
te-nel principio del C..8. dell' Inf. che dice;

i Chi porrsa mas pur con parole scialte
Dicer del sangue, e delle piaghe a pieno
Ch! 10 hora vidi, per narrar più volte ?
4 mi lingua per certo uerria meno, L
— avventura seguita Vergilio nei 6. deli' Kncid., che dice, imitando pure

°

Qqq 2 Non


492 MALMANTILE
Non mibi'y si Sassen or ag
Omnia penaru ee omina polfem.—

E così rende l'uditore attento,  curioso, col promettere di vol
venimenti così maravigliofi, che non e per trovar parole adegu
ne esprimere. A! > bet: F e

'DARA le barbe al fole, Morira,. E' traslato dalle piante, le qu
cioè si feccano, quando si fuelgono, e si voltano loro le barbe al So

GAGLIARDA, e Calata, Sono-duc specie di danza, ob
scherza con la voce ealata., che vuol dir caduta, oftela, d
ver fatte qui Gagliarde, e capriole fara la calata,, cioè calera verso
comunemente s'intende andar sorterra, cioè morire. Jay +e

DIRA il Vespro Siciliano, Dopo haver cantato versi amosgh ante fj
Siciliano, che s' intende; vedrà, € provera stragi. B' nora la follev ne de
ciliani (orto Gianni di Procida contro a i Francesi nel cempo, che questi ti g
giavano la Sicilia nella qual sollevazione fu il egno, che un determina gi
al suona del Velpro ciascuno si movesse contro a i Francesi, come se
successe grandissima strage di essi Francesi; E da questo è nato il '
Vespro Siciliano; che vuol dir fare steagi,ammazzare. Vedi Gio. Villanil
61.¢ Giachecto Male(pini nella Continuazione della Storia di Ricor
cap. 209, >

Hil festive l'agetib + Sona tial yarenaisinmeeltaie
i quali nel tempo, che sono gli agnelli, vanno per Firenze gridando. Ch
scorticar l'dgnello; per bulcar denari in ammazzare, e scorticare Metti ani
il nostro:Poeta da quello (canaare,\¢:scorticar un' intinica di'effranil,
puta huomini crudeli, e senza pieta, e questa per'accomodarsi abgenioy"e cap.
cita de i fanciulli, che stimano quell' atto una granditlima inumanita,
nando quelle bestiuole innoceati. ny sieht

FLAGELLO. Qui è preso in significato di eopine, farts ee CA

di. Vedi sopra C.1. tt. 45. invaltro signiticato. In Gio. Villani trovafi nel fen ayy
usato qui dal Poeta; F/agello, e Fragelio; come costuma di dire anche aug
piebe Fiorentina, e come dissero i Greci, e si legge ne} tefto Greco dell®; pac
fey

uy

dy

hio, Phragellion per quello, che i Latini dicono Fracetium Omcto
sgrazia,sferza, o 2agello ds Givve vinci Node tibro 12) verlo 397%
831. Attila Re degli. Vani tu soprannominato per quelo, Frage! t
TREMITO dela quartana, Quci brividi, che si-sentono' dal pazavente nell'en-
trare della febbre quartana, i quali sono assai maggiori diquegli, che soglions fie
venire, quand' uno ha qualche spavento}-eperd 'con dives VA tvemiep dela pias edy
sana, intende, che lo (pavento era grandissima,€ fuori dell' ordinario: E «ali tend
brividi, o tremiti vengon' allt huomo:, perthela'patira fringe il cuore; per lo

che il sangue corre tuctovin aiuto di esso;¢percio--membri esteriori, e ie parti te
superficiali, ed cftreme rimangon\fredde; edi steddo facendo riftrit i pori, be
cagiona quel che i Latini dicono rigor » che farizeare i capelli, o pels "€ Cagio~ ni
na il cremito, il quale si domanda capriccio, e rsbrezzo, Vedi C. 6 Gig

MAL più, de' miei di fare quel defo, Spaurisco tanto, che esco



cro prima.

ets DAN ZALHI8.'

be il galio apportator del giorno
La notte nera più d! un Calabrone,

Bil sua buio,e quant'abre eli'ba dintorne
Diognise qualungue grado,e condizione,
| Acid sicuri omai faccian ritorno
\ Gli nccei, cantando il lor falfo bordone,

AIncitr'al Sol,ch'in quespa parte,e in quella
| Fa pel lor gorxo nascer le granella
lead ety

Perché-crafeun » che quini si ritrova,

VNDECIMOCANTARE. 493

“fino a che viverd » non farò mai più allegro, come era mio solito, perché questo
- spavento m' ha fatto mutar compicifione, e temperamento: Non saro piii, quel

STANZA IV.

Quand' infra Dame, e Cavalieri erranti,
C' al trescone in Palazzo eran intentiy
Comprefeun dietro all'alero i duellanti,
Armati tutti due, come fergenti,

-Si shallo il ballo, andar da cantoicanti,
Ele chitarre, ei musics Srumenti
Ai proprj suonatori, e balierini
Divenner rante cnfie,¢ berrestini,

STANZA V.

Si fa pero bifbiglio, e si rinnuou

 Kedendo entrar quell' armi coid dentro, L? odio fra te farion già quasi sperto,
 Subirovdisse: Qui garta cicacca: Che tirando ai rispere: gu la bufa,

: =, £ trama di qualche tradimento, Ruppe la tregua, e rappicce la xufa,

4 iver la levata del Soley e dice, che in fu quell' hora entrarono nella stan-
22, ove si faceva il ballo, Martinazza, e Calagrillo, che la seguitava con l'armi
F inmano:; per lo che si lascid star il baliare, e si venne all' armi., rompendo la

tregua, perché ciascuna delle parti sospetto d' esser tradita, e che questo fusse uno
— militare, come i ditie sopra C, 10, stan.31. dove laicid questi duel-
EL gaily apportaror del giarno sbandina la notte. 1) gallo e solito cantare in full'ap-
pariridel giorno, ¢. però dice ch' egli è apporrasore del giorno, e che da 11 ban-
do alia notte col suo cantare. Somniaque excuffit nuncia lucis aus, disse un Poeta;

Excubitorque diem cantu predixerat ales, canto un' altco, \& erifta spettabilis alta,

Auroram gallus vecat applandentibus als, Disse il Poliziano nel suo Villano.

CALABRONE. E! uva specie d' infetto, o verme alato di figura simile alla
mofta »maatlai più grande, e di colore ncgrissimo, ed ha un jungo, forte, e»
acutissime pungiglione. Con questo nome chiamiamo,ancora il tafano detto fo-
Corot. 8. 1 Greci Prouerbilti ditiero scarabao mgrior, Più nero dello scara-
B10, che e un' altra specie:di mosconaccio. i
4N comro.al Sole, Glivucceili vanno incontro al Sole cantando in ringraziamen-

to delbenefizio, ch' ci fa joro, maturando le biade per loro alimento.

* GOZZO. E! il primo ventre degli uccelli, cloe quella vescica, che hanno ap.

Ppit-del colio, dove si ferma il cibo, che beccano, edi guivia poco a poco si di-

Mtribuilce al ventricolo; e da noi si piglia ancora per la gola dell' huomo, perché

vien da gutrur. re

° CAVALIERI erranti, Così son chiamati quei Cavalieri avventurieri, che son

descritti ne i Romanzi Spagnvoli da loro detti Cauaheros andanes; wa qui inten-

de, che erravano perché stavano ballando aliora, che bilognava combateere.

“| TRESCONE, Specie di ballo, cos detto da Tre/ca balio anuco. Vedi iopra

G. 10. st, 28, Dante Purg. 10.

: ee S=TePiie etre

= ee

a

i



494 MALMAINTILE® (0
Li precedewa al benedetto Vaso

Trescando alzato, o umile S. ane cond
cioè faltando, ballando. M As +
SBALLO'. \\ verbo shallare vuol dire disfare le balle; ma qui

re il balio, In buon Toscano non si direbbe shallare il dar fine al
pis la forza della lettera 5s, aggiunta al principio di verbo, 0
ignificato contrario si come la particella, i», appresso i latini, |
tare, spiantare; grariofo, fgrariaso, ec, ma il Poeta se ne
scherzo di ballare, e sballare, e seguita il bitticcio  dar da canto s canti
figuratamente sbaf/are, per eccedere la verita ne' racconti; e o ¢
numeri di cose con vantaggio, e con caricatura. " *
DIVENT AR casse, e berrettini, ec, Cuffia, come s'e detto sopra C, 8, fh. 48:
una berretta fatta di velo, o di tela.a foggia di facchetio usata dalle |
ferrar dentro i capelli in capo; dice, che gli Prumenti vennero casse ye
perché le chitarre, ed altri strumenti simill corpacciuti, essendo bateuti in
capi di coloro, e per la loro fottigliezza sfondandosi, fecero I effetto
be in sul capo la cuftia, o berrettino, cioè lo ricoperfero, e ferrarono in
E' detto usatitfimo. Ti faro wm berretrino della chitarra, per intendere i
chitarra in fu la cefta. Vina timil frafe venne in capo a Omero nell' Iliade, quan-
do disse, Lapidea indui tunica, per voler dire, Essere tapidato', quasi il ricoprire
uno di faffate, sia uo fargli un vestito di pietre, che gli stia bene alla vita.
GATT A cicova, Ciۏ mifterio foro. Ci e inganno, eum Tras tiled
i Latini. tet ain
TRAMA, Si dice quella feta, ec., che serve per riempiere le a
renza dell' altra, che serve per ordire, che si dice orsoio; che per la più n
si dicono ordito, e ripieno. Dante Parad. C. 17. t soi aged Rl hn,

Poiché racendo si mostro [pedita Tat

L! anima fanta di metter la trama Che

Jn quella tela, ch' io le porsi ordita, (SRE LS Sir

'Ma trama Gi piglia per concerto, ene habbiamo il verbo tramare, cheiwuol dir bag
negoziare copertamente, e forco mano, dilegnare,, concertare, Mraletrami ge ha,

fio affare,ec, Bdicendo: Queffaé trama ds qualche tradimento, intendes/Queho Oe)

@ tradimento concertato. Latino /ute/a doi. Varchi Stor, Fior, lib Cm
d' una conuenzione facta senza saputa d' un terzo dice + Orazio se ne? ada
rugia, senza che il Sig, Gentile fuspicasse non che sapesse cosa alcuna di questa' i
trama di gocciola per intedere specie d apopicsia,quasi una coperta apoplethiaye da 4
questo si potrebbe intendere per rrama, uaa (pecic; e dire questa è specie di qual i
Che tradimento. Storia di Scmifonte Trattat, 3. dice. 4 popolo fa fallewe, e grida tha

na, [uspwcando, che trama ui falfe, contro di lus, speotepecnh aot ¥

BIS BIGLIARE, Dilcorrer in segretor, che si dice anche Far Pith pifft; ij
Pispigliare, che usò Dante Parg. C. 5. Skit ise Bap w
Che si fa cio, che quini si pispielia, “ ¥ they

E si dice pi/pigio., e pispiglio, sorta di cicalamento; e viene da quel fafurrio, che: hi

scatiamo da coloro, che parlano in fegecto.. toggi pia comunemente si diceb® =
Soighiare, bifviglio, e bifriglio, 5 te te
Th Ry

ae ae


na, O rispetto
 STANZAVL
metre man da buon Soldato,
imico ritorna a Bertinella,
f quale in quel punto casco il fiato,
UM fegato, la milza, e le budella,
Vedendo, quando men' hauria pensato,
 Vicire i pefei fuor deta padelia,
 Mtentre la fa venir Adarte vigliacco

Col suo Baldone alle peggio del sacco.

STANZA Vil,

} VNDECIMO CANTARE.
| | TIRANDO giù La buffa.a i rispessi. Non havendo più rispetto, o riguardo al-
cuno. Sxffa intendiamo una berretta, la quale e fatta a

f » € mandata gil cuopre anche tutta la faccia, e i collo: Eda questo
la faccia, mandar gis /a buffa., vuol dire oprare senza riguardo, e scaza

495

foggia di morione, che

STANZA VIIL.

Mentre 8 alcun t' osserva, ella pon mente
Per canfarsi enon esser appostaca 5
Ecco in un tratto vedefi presente
Martinazza la sua confederata,

Che poco dianzi anch' ells fimiimeute

Di man di Calagrillo e scapolata,

E seco vanne in luoghi occulti, e fenré

A fare uncanti, es faliti (congiuri,
STANZA Ix,

eit a o un certo vento non le gusta, Nes quali aiuto ella chiede a Plutone,
Che fa le (pade,e ognor per l'aria sischia, Ed ei comparfo quixi in uno ispante

wiil —- E.grd vedendo che (a morte aggiufta Dice, c' ha fatto a lor riquifizione
yee] Chipi e vuol far det brano,e pin starrischia, Già [pedire un tacche per un gigante
at Bel bello fuigna, e vanne alla rifrufta Qual' è quel famosissime Brancone,
it | Dun luego da salvarsi da tal mischia, Che col bartaglio,ch' era di Morgante,
pt} —- Adtischiayche non gli par di porer credere, Verrd quini tra poco in lor foccorso
- Ee Percio sospira, e non si puo discredere, ef dar picchiate,e' hanno a pelar  orsa,
\& votes: f STANZA X. xi

Ed eccolo ( foggiunfe) ovvé battaglio\ E 8 anuedra,c' al fin piscio nel vaglio,

o desi fo dir ych'il primo,ch' egli accoppa,
Tatra l'armata a irsene in sharaglio
Che la barba penso farci di froppa;

E che al pigliar un Reeno non è loppa;
Cot scaciata abbaffera la cresta
dn veder, che de' suci non campa testa,

Si rappicca la battaglia, e Bertinella essendosi perduta d' anima, per vedere
i ritornato suo nimico., quand' ella pensava d' haverlo tutto dalla sua, e
-temendo di non esser ammazzata in quella Foote » meditava di salvarsi in qual-
4 che ficuco, ed appunto-s' imbatcé in Martinazza scampata da Calagrillo,
J € con essa en' andò in iuogo appartato a fare incanteGimi, per costringer Plutone
F -ad aiutarle; ed Egli comparfo quivi dice, che si fara venire il Gigante Biancone,
il uals in questo dire arrivO quivi, e Piutone rincuora le donne con raccontare
la bravura di flo, dalla quale da loro per distrutta l'armata di Baldone.
LE casca il fate. Si perde d' animo. E soggiungendo: Il fegato, la milza ye,
, te budelia, intende Si perda d' animo affatto
— QFeANDO men fet è pensaro. Quando meno dubitava. Non expettato valvu
ab hoffe culit..

VSCIRE i pesci fuor della padella. Perder quel ches' era acquiftato, e sopra di
che s' era fatto aflegnamento certo, e sicuro.

VENIR alla peegio del facco, Venire al maggior segno di discordia, e di rottu-
ta, Nelle guerre il peggior grado, che sia, €, quando le Città,0l'Armate son
messe a facco; e però dicendosi /e peggio de! facco in peggior grado, e condizio-
ne, che è haver il facco. VL

,

|

ee
496 — MALMANTILE o 7
VIGLIACCO,, Vile, codarda., EB voce spagnuola, vell
significa furbo,¢ furfante, poltrone. i
SEL bella, Con bella maniera, e senza dar o del
antichi disser; bedlamente,manoneinufo..
SVIGNA. Se ne va con preftezza, o fugge. Forfeda questo
viene e omprare if porco, che vuol dite anch' egli Andarsene
fuinam, 010 fuillans emere. ed è usate questo verbo svignare
besco. Vedi sopra C. 4. stan. 51, Si potrebbe anche dire, come pei
erudito, che questo verbo fuignare ligniticaado scappar dalla Vigna, s°:
scappare di foro la Vigna, strumeato o macchina milicare, che serviva
tichi per andare sotto ie muraglic a combatier le Piazze, con le quali”
difeadevano gli atiecianti da i (aii, ed altre cose, che erano: buttace lor
dagli affediati, le quali necetiitavano quelil, che vi erano.coperti a
sotto alle medesime vigne; extra vineam exire, che suona fuignare.
VANNE ala rifrufia, Vuol dice cerca mioutamente, e con diligenza
NON si pus discreaere, Non può non credere. Non può creder, che
a cffer così, e non habbia a eficre altrimenti. Non può capacitarli
SCAPULAT A, Fuggita; Scuppata. 3' intende scampato il pericolo
LACCHE', Ragazzi, cae corrogo appiedi per servizio de' loro
di sopra C. 2, stan. 29. 2 ae
BLANCONE. B' quel coloffo di marmo bianco., fattura dell' Ammannato, il
quale e posto in Firenze nella Piazza dei Gran Duca, dentro a una valea gran-
de, la quale riceve l'acqua da diverse fontane, che scacuriscono da detto: fo
¢ suoi annetii; e se bene rappreicnta Nettugno, e chiamaco da cutta M Biancones — ui,
ai ee ray Vaca; 1 hi
MORGANT E, 11 Pulci in un suo Poema intitolato il Morgante narra'; che

M
questo era un Gigante, 1 quale nog adoprava per coubattere alt': he un Ya
gran battaglio da campana, joe alo tf

PICCALATE ¢' hanno a pelar  orso. Picchiate gagliarde, perché il) pelo dell' Oh
orso essendo difficile a suellere, e pelare » non si fa caicare con' ky
se leggieri, Pelare, wattandosi di muraglie, o pietre vuol dire-space; ol
si, o (crepolare, onde potrebbe dirli hanno a peiare  orso, cioè tare fore yi
rompere l' orso, che Gi dice quel pictronc, che adoprano gli fiyfaiuol FN
lire i piano delle stufe, onde nabbiamo poi menar l'orso.a Atoaan Pre
re ripulir Modana, e Ggnitica mecterii a far una cosa umpolsibue uk

PENSO' farci la barba di Stoppa, S'intcnde; E poi dargh tuoco.
Penso ingaonarci,.¢ por farci ogni maggior danuo, ie
PISCIO' nel vaglio, Blo stesso che far la zuppa nel paniere desto sopra C.
stan.7. E.che cosa sia vaglio, Vedi sopra C, 2. stan, 79. Luciano in ab
co volendo spiegare, che il far bene a' crifti e come un tar la 2upp.
perché 1 benetizzi riceuti (cappano jaro prettissimo dalla memora; 4
buomo cattivo,e sconoicente a una bog forata, che uo quello, che va i met. tes,
te, si ver(a. Plauto nei Pfeudolo, o vogliam dire Bugiardello; 2Vae piuris refert, ="
quam si imbrem in cribrum geras, Corcisponde questa maniera alia noltes (char
nel vagtio, Luciano nei Live dic; come da in cofano forato, o

ey



VNDECIMO CANTARE:

497

)zuppa nel panicre. Playto pure nel Pleudolo, la pertu/um ingerimus. dicta do-

opera ludimus. La favola delle Danaidi ha fatto luog
nifica non e cosa facile. Loppa; che si dice

19 al prouerbio.

NON: ne Detto bafio, che
anche lolla,
anche

ce » gli c levata.

a STANZA XL
Qui tacqueil Diawol,perch' e fatto rece,
“ él aria al capo giù è maligna,
anuerzo 4 star sempre nei fuco,
Vatea alle donne il dietroacafa,e/uigna,
EB lafesaus il Gigante nel sua law,
Che douendo a Baldon grattar latigna,
 Sull ulcio det falon già perwenuto,
edge Hf batraglinje questo fu il faluto,
STANZA Ali,
Sei braccia era ti bascagtio aito, e ds passo,
| Bm injragnena aimen arciotto,o vent,

4; Ma dando fu nei patcormando a baffo
cha  Van trang intatiata, e tre correnti,

; E fece tai frafiuano,e cal fracasso
14 Che shalord? « un tratto i combattenti,
oh OE per pawra, a chi non fu percosso

we |, Non rimase sr quel punto/anguc addosso,

il gulcio, che si leva di sopr' al grano quando si bacte, che si chia-
inche pyle. Lat, apinde secondo Nonio Marcello gramatico. 5

Y SCACIAT A ~ Rimanere scaciato; vuol dir Rimaner buriato, ches' intende
; nd' ugo credendosi conseguite una cosa, e facendolela sua, o non la confe-

 ABBASSERA la cresta. Gli scemera!* umore, o I alterigia, I Galli d' In-
dia, quand' entrano in frenefia, gonfiano,, e cresce loro la cresta, € patleggiano
on una certa intronizzatura, che par (uperbia; ed usciti di quella frenefia, sce~
ma, ed abbaifa loro la creita,¢ di qui vicne il presente dettaco, che significa
readersi umiie, contrario di Rizzar (4 crea,;

STANZA XIII,

Ed infra gli altri Piaccianteo, il quale
S' era schermito bene infizo aliora,
Vedendo un fantoccion si badiale,
Dopo il terror di tanre [pade fuora,
Di quel detto farebbe capuaie,

a9 C' un bel fuggir faina la vita ancora,
444 perché in quae in la v'é mal riscotro,
Vede hauer viso di sentenra coniro..

STANZA XIV,

Poiché non fa tronar modo, ne via
Per nellun verse da [campar laguerra,
Ech' ovis e forza, che chi v'é vi stia,
Pond morto, gettasi gilt in terra,

E ritrouando la botrigleria

Apre t armadia, e dentro vi si ferra,
Con pensiero di fiarni sempre occulto,
Fin che si quiet così gran tumulto,

! Plutone si paite dalle Donne,e la(cia quivi il Gigante Biancone, il quale andò
, alla lanza,dove si faceva ia zuffa, ed arrivato in fu la porta alzd il battaglio,
: per comigciar con esso a perquotere, ma al primo colpo dette in una traye, la.
quale per esser fradicia, si fraca(so insieme con pill correati. Tal colpo spauri

» tutti coloro, che eran quivi, e particolarmente Piacciantco, il quale fino allora

8 era ben difeso, ma per lo spavento, che hebbe dei Gigante,

getto in terra,

s fingendosi morto, ed a poco a poco si condufie all' armadio della bortiglicria '

b nel quale entrato vi si erro.

si ATIOr«0, Divenuto fioco. Uno, che per catarro, o per altro impedi-
} mento aell' aspera arteria ha perduta la chiarezza della voce, li dice rancus,don-
y de rancedine, e reco. Dan, Int. C. 14.:

A, Erendele a colui ch' era già reco, £.
(| Li aria gli è maligna, 1? aria gli nuoce, gli cagiona danno,

1, dietra a casa,¢fuignua. Volta le reni, e se ny « Bil verbo /ujznare,detto
rr GR.

Poco sopra nell' ovtava fectuma,

AT.



MALMANTILE® o
S' initende perquotere. 'Così I intende
. fo direi anche, maio temo, che'ella

ny Won s apparecchi a grattarmi la tigne.

Si dice anche cacciar la mo/ca da deffo, in questo C. stan. 20, !
dajfar la lana, sopra C, 7, stan,63. Adandare a Legnaia,sopra C.
ter Ta poluere, sotto C, 12. stan, 1, E tutti hanno lo stesso signi

INFRAGNERE. Ammaccare, o pigiare una cosa tani
forma., come farebbe Peftare un fico maturo, ec, e il Lat. ¢/
Vedi sopra C, 4. stan. 76, e sotto in questo C, stan. 17.) *

INT ARLAT A, Rofa dai tarli, che sono quei vermi, li
dentro al legname.,, e di ele si nutricono; da i Latim detti rer
'C.6. stan. 59.

FRASTVONO, Fracasso. Sinonimi, che significano Romore, stre

NON gli rimase fangute in deffo. Acbbero così grande spavento, che

mate spirito, Dicono, che a uno, che habbia ha'vuro un granditimo sp
o paura, se in quel punto gli fule tagliata una vena, non gliu
per le ragioni accennate sopra in questo C, stan. 2, d

S' eva Jchermito bene. Cioè,s' ra difeso.. Havea scampato il toccatne

BAD/LALE, Grande, Si dice anche machofo, imperialc, € simili,
'scherzo; e significa grande più del naturale. Kose ee
VN bel fuggir falna la vita ancora, Alla (entenza che dice Un'bel morir tutta le
vita honora, rispondono coloro, che flin:ano più il vivere 5, oe

Sa ney

bony

Vo bei fuggir fainn In vita ancora, 7 ag
V" è mal viscontro, V' è male il modo. Non W'@ buona congiuntura, ~ io
VPEDE baker viso di sentenza contro, Conosce di non'haver ragione, cioè, che il Mt

'ncgozio non è per seguire, com' ei vorrebbe. A tthe? Wy
CAl ve vi sia, Chi ha havuta la disgrazia, se la ianga: E si dice: Obi v'é ui

vi sia, e chs nen 0° è non v" entri, qui però intende; chi in quella stanza vistia, tt

perché non se ne può ustire. Reet eee
BOTTIGIERIA, Armadio,o stanza, ove si tengono V afi da Vino ' a)

¢ servizio della mensa. Voce, che vicn dal Francele Borteille, che e Cl

'fiasco, o altro vaso simile da vino. 4 4

STANZA XV. 7

'Col battaglio di nucuo agile, e presto o già ch' egli non puo tt

Tira il Gigante, e da nella lumiera', etrmeggiar col bat: 'et lento, f k,
Ls qual cadendo fece del suo refto, 'Pero che il toga non ha gran diffanza, fr
Perché i [pense, e rope ciò che v' era; “Cagion ch' ei trowa fempre' mento; a
Hor, 8 eglie in bestia, dicanelo questo, 'Lafeialo andar bawendo pin fidana x
Mentre ch' ei da ne' lumi intal manera, Nelle sue manych' in simile Srumenb hei
E dice che'! Demonio lo jtafila, E piglia quells ciurma abbietra,e sbricia a
Poiché eli fa saltir due colpi in fila. 'eA4 menate, com' anici im camicia yj
' STANZA XVIL. oe ee fg
'Così tutto arrabbiato, come un-cane Talche'l me/chin non mangera più par ad
Piglia un pel coho,¢ feactialo nel muro, Perciò gli amics [uci, a
Di sorta, che disfatto ei ne rimane “We voglion, che il ribaldo,
Som' wie ficaccia piattalo matures Gli andaron alta-vien tush quant

Stet a. ac? ae


VNDECIMOCANTARE. 492
STANZA XVIIL STANZA XIX.

"sion costoro un brance di galletti, E come la mia Serva, quana' in fretta
Quando la state, a tempo di ricolta, Dee fare ilpesce a uovo,e che si caccia,
Antorne a qualche bica units, e spretti Trama due nova einfigmele picchiet:s,
non di loro a berricar s' afolta, Sicche in untempoturte due le(chiaccia

Pere il Gigante fa certi feambierti, Bs che dall' tra e [pinto alia yenderca
Che re ne [uifa quattro,o se per volta; Softien quei due,es' apreneliebraccia;
Insassidico al fin da quel baccano, Poryciacche,pacte insieme quello,e queste;

Si china,ed aggavignane un per mano, Stcche e diwentan prit che pollo pefto.

. Biancone con un coipo fracafia la lumiera, e spegne tutti i lumi. Nota che,
i se bene era di giorno, la lumiera era tuctavia accela, il che spesso aveiene in ta-
lioccafioni di veglie, che i segiivorl distratti dal gusto del ball,fanno mezzo
— senz' avvedersi, che sia pafiata la notte, Ll Gigante in collera lascia il
ttaglio, e comincia a pigliar quella gente, e bacteria per le mura, onde tut-
tian tratto gli corsero addosso, ma egl si difendeva, facendo di loro un gran
-maccilo.
LVMIER A, EB vno strumento, col quale si softengono in aria più lumi acce-
si, che i Latini dicono Lychauchus pensius, luceraiere in aria.
FECE del sue refto. Far dei retto s' intende fipire la roba, la vita, ec. qui dun-
que vuol dire si speafero atfatto 1 lumi. <
B in bestia, B in collera., Dar ne i lumi, vuol dire entrar grandemente'ty col-
Tera, dar nelle (candescenze; ed è lo steilo che dar nelle furie, ed il Poeta (cherza
 con questa metafora di dar ac' lumi, ed intende dare etfettivamente col batcta-
- glio ne i lumi della lumiera.

ail; 4L Dianol to feafiia, 11 Diavolo lo perfeguita; Gli e contrario.

IN fila, Vo doppo l'altro, senz' intramezzo.
ot CARMEGGIARE, Questo metaforicamente significa Aggirarsi, o affaticarsi in
ibs vano; e signitica anche ingaonarsi, per esempio: Tu armeggi, se tu (peri d' ot-

tenere, ec, ma qui e preso anche nel suo proprio signiticato di mineggiar lara;

gli Cnell' altro d' aggirarsi. —

wo) CWRMA. Genraccia vile. Vedi sopra C, 3, stan. 76 e C. g. stan. 16,

ABBIETT A, e sbricia, Sinonimi, che figaincano vilitfima, minurissima gente,

A manate, Da i più si dice menare. Quanti a' entrano in uaa mano; e per la
grandezza della mano del Gigante fuppone il Poeta, che fica moltiimi per vol-
ta, perché dice: came anici sn camicia, che sono anici coperti di 2ucchero, de i
quali con una mano se ae pigliauo le centinaia.

FICO piattole, E' una specie di fico detta così.

NON voglion ch' ci se ne vanti. Lo voglion gattigare, perch' ci non s' habbia a.
gloriare d' haver ammazzato quel loro amico.

». BlC-AQuafi da il Lat. Barbaro apica dal buono -dpex. Così chiamano i Conta-
dini quel monte di grano in paglia a mazzi, da loro così accomodato, affinché
si stagioni, pec poterlo cavar dalla spiga; deta da 1 Latini rrieict congeries. Da
questa voce bica habbiamo il verbo sdbicare per accamulare. Dante laf, C. 9.

Come le rane innanzi alla mmica,
Biscia per l'acqua si dileguan tutte
Per e alla terra ciascuna s abbua, Rrr2z~ BEZ-

SEE CERCA E



500

MALMAN TYLER: 1 v

BEZZIC ARE, MW beccare'de i pollaftrelli si dice bez
FA certi scambietti, Cioè contraccambia le percofie,

ra

Scambietto * termine di ballo, che significa mutanea'
INF AST IDITO da quel baccano, Klicndogli v«

si
sopra C, 4. stan. 9.

Allor Bieco non ha pite fofferenza,
E giura, che di questot: Bacchillone
Von andra al Prete per la penitenza,
Perch'ei vuol, chee' la faccia col bastone;
Ei fui, che di ral arme ban da teenza
Gite ne daran a una fanta ragione'
Così guida i fuvi ciechiyow' e il coloffe,
Accto gli caccin le mosche da defo.
STANZA XXL.
Eglino tutti quini fermi a tiro
Presso.a Biancone aun fiscbioco' bastoni,
Senza tramezzo alcun, senza respiro
We diedero un carpiccio di queé buoni,
Ed egli con un piede alzato in giro
Fa lor sentir, s' egli ha fodii talloni,
E mentre questo passa ye quel rientra,
'Con quel pedino te li chiappa,e /uentra,

'Bieco veduto questo fa vehire-i suoi Ciechi,i quali tutti in giro ini

la importunita. La voce baccano, che significa combat esett
piglia nel senso, che si piglia musica, felta » bordello, '

Quand! ecco rt veccbio Paolino

Ve anti

AGG AVIGNA, Piglia,¢s' intende cinger con la 'mano 'tu

glia, in maniera, che si possa tenere stretto con factiita,
PESCE a' weno, Vova fritte »0 frittata, che dicemmo sopra C. 9
s' intende propriamente la frittata, che dopo eer cotta, 0
ruotolo, pure nella padella; rifritca, e ridorta in figura “di p
ta pesce d'uono, La Compagnia della Lefina dice: La 'consner
antichi, i quali conrenti a' un pesce d' uouo di due woun al più

ClACCHE. Questa parola non ha verun significato', ma solo
no, che fanno l'uova, ed altre cose similt, quando si rompono, edil
ne serve pr esprimer quel bateere, che fa il Gigante di'quei due hi
tr' all' altro, ed immita Dante, che nell' laf. C,32,dice:
LVon hauea pur dall' orlo fatto Crich
E seguita i Latini, che pure 'hanno a finta voce Tax,
come si vede in Plauto in Perla; dove per intender buie dice > Tax
meo. E noi pure diciamo'tach, e pach; anzi le percotie da molti in F
cono pacche, come dice anche il noltro Poeca sopraC, 5. st. 47. Da
ta la parola Fiorentina dcciaceare,, che e lo ttetio, che' Pefeare
dicesi 'Pepe acciaccary; modestamente infranto,e Acciaceo sopi
do uno per così dire calpefta,¢ maktratta un'alero., ”
5 j

3¢ per

la hile elbetae,

a;
STANZA XXN

Aquat fa più cagon,cblT efti,e'
E ( perchegli e bizzarre) bam
Condotti com' ei suole,un par a

OveSalito a Petigion di

Vavol matel,ch'egis' ha de ee

T aftando,owe il Gig

£ darel ccc ieP bocca

SEafi Si =F eo \&

oy
Wes,

ee
a

aie



VNDECIMO CANTARE. yor

affaltano co bastoni, e Paolino falito sopr' a i suoi trampoli metie i) suo
iuolo sopr' alla faccia-di esso Biancone, il quale però s' adira, e beltemmia

i suot falfi Dei. Pah
| BACCHILLONE, o Bacchiglone, E nome d'un fiume, che passa dalla Cita
| Vicenza, in Latino detto Azedoacus minor (econds Fra Leandro Alberti; ed ¢
ida Dante Inferno 15.-ove discorre d' uno, a cui fu permutato il Velco-
irenze in quello di Vicenza, che dal servo de' servi Fu trafmmutato d' Arno
one. Da questo fatto di Messer' Andrea Mozzi, che così si domanda-
Vescovo, o pure dal verso di Dante nacque in Firenze il proverbio; del
fanno teftimonianza il Varchi nell' Ercolano, e il Borghini. Sacare d''e4r-
in Baechilione, aitudendo al saito dal Vescovado di Firenze a quello di Vicen-
y che significa faltar d'un proposivo in un' altro s Saitar ai palo i frafea: Ma
-questa voce Bacchillona aggiunta a huomo significa huomo infipido, e buono 4»
. oe » ancorché di persona grande; e suona lo tteflo, che Gaicone, Palamidonc,
i: » e simili,.¢ credo, che sia il medesimo dire a un! huomo Lacchillone,
scheCaftrone, e che venga da Bacchio, che in alcuni juoghi di Toscana vuol dire
we — agnello,e cos: Bacchi/one voglia dire agnelio grade,cioè Caffrone. O pure viene dal
o | Lat. bacuius,quati Perticone, Scuriscione, O vero \& deo quali Baleceone; che si
»¢€ non fa niente dibuono, ne di ferio.

WON andra al Prete per la penitenza. Questo modo di dire usiamo per fare in-
'tendere, che ci vogliamo vendicare del oprufo, o torto fattoci, o che vogliamo
galligare uno di qualche mancamento commesso; quasi diciamo: lo medesimo
i dard la pena di questo suo fallo, (enza che egli vada per essa al Confefore sed

il Poeta l' e(prime dicendo: Perché vuol, ch' ei la facia col baffane.,
| AIANNO ficenza-di porter tale arme, Cioè hanno permiflione di portare il ha-
it scherza, peso ivciechi portano il bafione per necefita, per farsi lan

QW VINA fanca ragione, Gli daranno le'bastonate,.come vanno date, e quella

pi |  WoCe Sama, se ben pare riempitura per emfali, nondimeno detta in questi termi-
sf ablignifica perfezione, quasi dica divera, e di tutta ragione, e d' intera giufti-
a Zia, che la voce Sanetus fiacopata da Suncitus vuol dire Nabilito, determinato.,

» Nov. 10. £ battnrala adungque d' una fanta regione, cioè.con una folenne ma-
niera; dateglicie delle'buone. Vedi l'Orava 25. seguente..

GLI caccino le''mofebe da defo, Lo battonino. Vedisopra in questo C. st, 11,

SENZA tramezzo, e senza respiro, Senz' intermiffione di tempo, e senza pi-
igliare riposo.

NE dettero un-carpiccio di quei buon, Ne detterouna buona,'¢ gran quantità.
Carpiccio viene dal verbo carpire,-¢ pero vuol dire. manata., o manciata, e cence
Aeruiamo per intender quantità., ma per lo più di bufie, comel'intese ilFiren-
-2uola nell' Afin d' oro + £ poscia, che per nua volta gle x' hebbe dati un-carpiccia de

i

TALLONI + Quella parte del piede, che e tra la noce., e il calcagno,:ma qui
'piglia la parte per cueto il piede. Vien dai Latino Tans. C. 8, st..69.
 PEDINO, Deito ironico., ed.intende gran picde, pedone,

SPER



goz MALMANTILE

SVENTRA. Rompe, spezza, o sfonda il ventre,
attivo, che fventrare neutro ha il figaitca
PAOLINO Cieco. Questo fu un Cieco compo!
zonette, le quali si sentono ancora cantar per Firenze da al
azzi, e per questo il nostro Poeta dice: Fs più canzoni, ch
oeti celeberrimi del nostro secolo. Tali sue canzoni anda'
le piazze, dove per adunare il popolo faceva fare diversi
cani, ed egli medesimo, benché affatto cieco, e decrepito, 2
trampoli di legno a i piedi, Questitrampoli erano duc pertiche, in
ciascuna,delle quali era fitto un pivolo, e sopr'a questi dae pivoli fali
sopr' ad essi i piedi, e foftenendo la persona col rimanente di de
con adattarfele sotto le braccia,camminava con granditima franchezza
poli da' Latini si domandano Graiie, 'ccondo Nonio Marcelle; e quei,
minano su' trampoli, Gratlatores. Feito dice; Grattarores i
ni, qui, ut in faltatione tmitarentur agipanas, adiettis perticis furculas h
que in bis superftances aa similitudinem crurum eins generis gradiebantir
prer difscuteacem confiffendi, Plauto Vinceretis curfuceruas,o gallatorem.
D1 cento scampolt, Tutto rappezzato; che scampole Jiciamo quel pezzo d
no, o drappo, ec, che al mercante avanza d'uua tela quasi pezzo,così
pato, cioè avanzato a far' un' abito 1nccro; e qui intende toppes o pezei
anno. ere a
. (MB ACVCC ARE. S! intendé coprire il capo,¢ ilwifo.. Vedi
si. 73. Varchi Stor. Fior, ub. 1.4 Subso fu preso,¢ smbacnceato col eapp
dotto alle carceri,
Sl scandolezza, S' adira. Vedi sopra C.
di scandolezzare e quel, che dicemmo sopra C.
BREZZA, Vento freddo; Vedi sopra C. 7. st. 18. ue
PAbP AICO. E' un pezzo di drappo incre(pato da una parte, e ridotto quai Ht
in forma di facco, quale portano in capo le donne per difendersi freddo, ed 'afl
oggi lo chiamano anche cufia, Mattio Franzefi in lode delle Malehere dice e » all

£Lvvi un segreto, che a noi dir si puore, vet ~ Yay
Che la mascheraé me' a! un pappafico, si
E pero si vente in van. cufola, e [quote ty
Ed il medesimo in lode della Potta uso il verbo impappaficarft di aay
Chi ale tempse si fascia gli vechiati 'ake
Chi sopr' a i berrettin impappafica, ine
PORCO, Aggiunto a huomo vuol dire Schifo. ps a
0740". Intend, Che schitezza e questa? Vedi sopraC, 8.67 yy
ALLEZZA, Vedi sopra C.-3. st, 64. \& wota, che il verbo allezeare tantoat =”
tivo, quanto neutro ha lo stesso significato,; 3 sur (oy
SA di refe azzurro, Per tigncre in azzurro adoprano i Tintori ere
fetore orrendo, o sia galla, o sia guado', o uno, l'altro infiemes 1M

rimane per qualche rempo in fu la roba tinta, e particolarmence in sul 1in0
pero dice quel cenciaccio fa ai refe azzurro, ed intende. Ha gran fetores'
verbo appeftare ha lo fictio significato, e natura, che ha il verbo 4
di al detto C, 3, st. 54.

bee



STANZA XXIV.
levare intanto hawea Perlone
| La srane dal Gigante roninata;
Abe ancor quini ciondolone,

he la lumiera già tenea legata,
“Ed 4 foggia d! eAriere, o eMontone

7 nla addietro, e dannole l' andata

- Verso quel torvion, che si distese,

> STANZA XxV.
Hor' quando ( perch' egls sbalordito,
~ Etutto intenebrato in terra giace )
 LCieehi più che mai fanno pulito,
 Edegli se le piglia in fanta pace,
OB fra le maxe innolto.a quel partite
Un facco diventato par di brace,
o Eben quel panno al viso gli è dovuto,
—— Dovendosi si-cappuecto aun batturo,

lo stesso significato.

= =.
—

orribili Giganti.

| Col si più voite in bocca del Franzese, Perché quivinon è troppo-buon' aria.

VNDECIMO CANTARE 503

TI vuo' dar l'incenso con le peta. In vece di farti honore, ed incensarti, voglio
sprezzarti, offerendoti cose puzzolenti, come suol'esser il peto, del quale Vedi sopra C.\ 6.\ st.\ 100, Orazio. Vin tu Curtis Iudaeis oppedere?

STANZA XXVI.

Mentre gli rompon Poa, € poi gli fanne

Così t incannucciata co' randelli
E talor, non wedendo ove si danno,
Si tamburan fra lor come vitellt',

Gli altri soldati a gambe se la danno,

Ed ognun dice: alla larga seabells;
Euege la parte amica,e la contrariay

STANZA XXVIII,

Ma reftin pure a rinfrescarle gli orbi,

Con quell? snfalatina di mazzocchi,
Ed et riposi all? ombra di quei forbi,
Che gli grattan la rognaco' lor nocchi
Mentre quivi per far dispetto aicorbi,
Sotto quel cencio tien-coperti gli ovcht 5
Che sugnun parte,ed io mi partoacoray
'Pen tornare a Baldone,¢ Celidora:

~ Con inucazione, e macchina di Perlone, il Gigante e atterrato, ed i Ciechi
'gli vanno tutti addoflo col bastone, ed in questo grado lo lascia il Poeta, e torna
'a dilcorrer di Baldone, e di Celidora.

CIONDOLONE. Una cosa, che sta pendente da alto a baffo [enz' esser ferma
'in verun' altro luogo, che dove è appiccata, come farebbe il battaglio.nella cam-
. » si dice far ciondolone, o ciondojoni dal verbo ciondolare, come dal verbo pen-
Gee si dice pendotoni, o penzoloni; da dondolare., dondoloni, che tutti hanno quali

ARIET E, o montone, Macchine,'0 strumenti bellici antichi, de' quali si servi-
| -vanoiper rovinare le muraglie; Sono notidimi., parlandone tutti gli Storici La-
“tin; ma particolarmente Giulio Cefare ne' suoi comentarj.
quel sorrione. Così è chiamato dal nostro Poeta il Gigante, perché
avanza sopra gli altri huomini., come avanzano i torrioni sopra lemuraglic; ed
anche perché servendosi dell' Ariete, o Montone, lo deve adoperare, non in un'
huomo, ma inuna torre, come è solito adoprarsi simili arnesi. Da questa gi-
'gantesca flatura, per la quale.essi sono affomugliati alle torri; fece Dante il ver-
'ho Torreggiare assai galantemente. Inf. 31. Vorreggiavan dé mezzra ta persona Gli

* COL si del Franzese in becca.'Gridando®: bud, but, che voce dimostrativa di
p| dolore, ed in lingua-Franzese vuol dire si. 4

¥ - SBALORDITO. Siordito, fuori del sentimento'per le percosse ricevute..
¥) o INTENEZRATO, Si pwd dir finonimo di sbalordito: e qui vale per intormen-
i * 'tito dalle percotie. Un fatio, muraglia, o altro simile materiale folido, e dura,

“Ai dice intenebrato, quando, per le peccolic., che se gli danno per romperlo., e 4i-

Bio,



504 MALMANTILBE. | ¥

dotto in termine, che dal suono si conosce, che si comincia.a
F ANNO puiite, Vuol dire Ripulire, ma detto in questi te
da vero, o perfettamente; E' lo stesso, che Fardi buono detto fo
SE le pigia in fdnca pace. Se le piglia con tutta, ed intera quiete. Ci
bastonare, e non si rivolta, ne's'adira. E la voce Santa ha la forza,
detto sopra in questo C, st. 20. ' olay
KINVOLTO fra le mazze. Coloro, che portano la brace a vende
ze, la mettono ne i facchi; e per ammagliarii, e legargli sopra tie,
tatamente gli rinuojtano in alcunc imazze; ed il Poeta schereando dice
gante e simile a uno di questi facchi pieni di brace, perché egli € rinu
mazze,¢ intende di quelle mazze, con le quali i ciechi lo bastonano.
BATTVTO. Chiamiamo Barrasi coloro delice Contraternite fecolari
proceflionalmente vanuo con velti line in dowlo, le quali chiamiamo faccht
figueino vesti di penitenza ) cappe, o velti da bactu,, cioè, che f bane,
si disciplina, ed il capo, e faccia coperta con un cappuccio appiccato a detta
vefle. Ed il Poeta scherzando con l'adicttivo barrute, cio bastonate, e col sa
stantivo barruto, cioè humo di Confraternita, dice, che ai Biancone staya
il Cappuccio, perché era barrato; e per cappuccio piglia quel ferraiuolo, che
lino Cieco havea meflo in capo al Gigante. '
INC ANNVCCIAT A co' randelli. A coloro, che si sompono braccia,gambe,
o cosce, ec, Nel raflettare tal rottura, «fhuche ' off Rando fermo al luogo,ac-

comodato si rappicchi, fanno una fasciatura con pezzi d' assicelle, o stecche, la i
gual fasciatura chiamano / incannucerata, e pero dice, che, havendo rouse | offa U

al Gigante, gli fanno hora l'incannucciata co' randelli, cioè con quei afloat, Dk
0' quali lo perquotono. 37h Une)

s1 tamburano come vitelli. Si bastonano ben bene. Quando i Macellari hanne Ni
ammazzato un Vitello, o Bue, ec. lo gontiano, ed acciocché il vento pall Da
da per tutto faccia spiccare la pelle daija carne, bastonauo la bestia con alcune>

f
mazze, e questo si dice tamburare » o tambu/sare, che vedemmo sopra C. 2. AL34. ie
ed a questo ramburare aflomiiglia le bastonace, che si danno fra loro i Ciechi, whe

wuol dire molte, fode, e spete « Sidice samburare, perché date in quelle pelli di si
Bue, ec. gonfie, fanno il suono simile a quello del tamburo strumento 3 i

E per altro ramburare uno vuol dire quereiario; e questo perché anti

Firenze 4 tenevano in alcuni Juoghi pubbiici de' Magiltrati certe;

hi da chiunque si voleva, erano meile le denunzie legrete, € queste calle C
vano tambari, e da essi tamburare, era il medesimo, che acculare, o quetelare.
Vedi gli Staunti di Firenze al libro intitolato. Ordinamenta snspicia contra Magnare
(citau aicune -voite da Gio, Villani ) al capitolo, ove Gi tratta del mettere nel tam~
buro. ieee
ALLA larga /gabelli'. Allontaaiamoci. Quando dopo la cena si fa balla, 0al-
tro passatempo simile nella medesima stanza, nella quale s'è cenato sche 1 com
mensali si rizzano, e per dar luogo si fanno levar via le tavole, le seggiole, e Blt
sgabelli; ed ogn' altro, che potetic dare impedimento, si suol dire: alla

belli, e s' intende; si levi di mezzo ogui impedimento; il che e in ¢
che significa; facciafi ala, o si taccia largo, ma per lo più s' 1

HOEZBE SE 2E x PRZ



VNDECIMO CANTARE. 505

er ae: ' [
troppo buon aria, Li gon' y'é buono tare; Intendi: v'é pericolo di
- MAZZOCC H!, Così chiamiamo i Talli del radicchio, ne i quali nasce il se-
de iquali si fanno infalate, che sono rinfrescative, ed il Poeta, (cherzan-

'con I equivoco dimazzocchio., che vuol dire: bastone, dice che con questi

1 i taano al Gigante l'infalata per rinftescarlo, ed intende; /e ha/tonare,
SURAT. 1 bastoni de'Ciechi per to più sono di sorbo,o a' altro legnaine simile
chiuto, sodo.,.¢ grave,¢ dicendo 1) Poeta; Si riposf alPombra di guei sorbi, che
i grartan ia rogna co' lor nocehi, intende + si riposi sotto quelle bastonate de i

Ai “@ t
o PER far disperco a i corbi tiem coperti gli ccchi, Per fare Mizza a i corui per la.
» che hanno di non poter beccare, e cavare glivocchi al Gigante, poiché gli
, o difesi col mantello di Paolino cicco,
PANZAXXVIIL, STANZA XXIX,
Che la-nel mezzo a's suoi nimics comba Su via figlixoli; orto buon piceini,

“Di moda, ch? essi foeman per bollire, Faccian di quepti furbiyun tracto,ciccioli,
Che dove i colpi ella indirizza,e pidha, Nim remete di questi spadaccins,
Te sli manda in un subito a dormire, C' alcimento non vaglion poi tre pictiali;

“Che ne meno col fhan della (un tromba

¢ Es in vista.vi paton Paiadini
NCamprian eli fardbbe rifentire

Han facce di Lionije cuor di fericciell;

|B quamo brava, similmente accorta', Efel-gridare,e ilbravar lor v' afforda,
ait Acombatrere i suci così conforta, Ut can chabbaiayraro avyien che morda,
ya ov Deferive laibravaca, e prudenza di Celidora, e riferisce  orazidne da essa fat-

z Pe inanimire i foidati, la quale € veramente appropriata al personaggio, che
ae::

~ ZOMBA, Perquote ». Vedi sopra C, 6. st. 104.
| SCEALAN per bollire, Vuol dire sminuiscono, e quell' aggiunta per bollire, si
'poneiper un costume introdotto da un quoco goffo, e ghiotto, il quale havendo
mMeflo'a quocere Ieile alcune merle, se ne mangid più della meta, e portate il re-
A 'in' gli domando il padrone, che cosa havea fatto dell' altre merle? ed
“il'quoco gli rispose; Sig, sono scemate per bollive, E da questa goffa aftuzia quando
diciamhoe Laral cofaé (cemara per bodire, intendiamo, che una tal cosa e (cemata
assai, senza potersene ritrovare il conto, o sapersi la causa del mancamento.
~~ PIOMBA.. Precipita'; lascia calare, o calcare il colpo.
"LA tromba di Campriano, Questo Campriano fu un contadino aftuto, come s'è
'accennatosfopra C. 4. st. 47.,¢ Come si vede dalla sua fayolofa foria ttampata,
€Ol titolo Storia di C ampriano, 11 quale per'far denati trovd diverse inucnatoni di
gabbare le persone semplici; e fra l'altre quella d* una pentola, che bolliva fen-
(2a fuloco yperché da efio levata, mentre gagliardamente bolliva, € portat®in,
~mezzo a) una stanza, la fece vedere al corrivo, a cui voleva venderla; costui ve-
dmala veramente bollire, senz' haver fuoco avanti, subito se ne inuaghi, ed ac.
» Sordosi di compraria per il prezzo, che convewncro. Giunto poi guetlo tale a,
casa con la pentola, e volendo senza fuoco farla bollire, e non gli riuicendo, si
~ quereld con Campriano, dicendogli, che I ne ingannato; Campriano chiamé
ss la

SU OS eae st ey



506 MALMANTILE

la moglie, e la sgridd, dicendo, che non potev esser 5

cambtata, La donna fingendo un gran timore, con gran la;

per haverla inavvertentemente rotta, glien' havyeva data un' a!

paura, che havea del marito. Di che Campriano mostrand

to, cavo fuori un colrello, e con esso feri la moglie nel petto

ascofa sotto i panni una gran vescica piena di sangue, il quale fg

che uscitie dalla terita factale da Campriano; per la quale fingendo |

fer morta, calcd in terva. LU gonzo si doleva, che Campriano per e C

gicra havetle commedo un delitto così grave; Ma Campriano con facia.

£'i die + Sc ben la donna € morta, 10 1apro rifulcitarla, quando vorro

basta, ch' io suoni questa trombetta; e stimolato dal fempiice a far

piacque » e fonata la tromba, la donna ff rizzo, mostrando di rifalc

il (emplice con grand' instanza chiese la tromba a Campriano, il quale d

te preghicre a gran prezzo gliela vendé: Costui andato a cala prefe o

gridar con la moglic, ed in fine le diede una pugnalata, con la quale

€ poi si mefle a (onar la tromba, ava quella infelice elendo veramente morta,n0a

rifalcitd altrimenti, B per questa caula, e per altre sue sei. aggini fu Cam:
riano condaanato alla morte, che dicemmo sopra C, 4. st. 27, E di questa trom-

Es parla i) Poeta nelprefente luogo. sot then is ERE
SOTTO buon piccini. Esortazione, che si fa a' cani, quando s' incitano,o am- —

mettono contro qualche fiera, come vedemmo sopra C. 2. £.87,; ed il

si softiene sempre in fu le burle, fa che questa Capitanessa esorti, edit

suoi soldati con questi termini da cani.: - 1

cicciold. Frammenti di grasso di porco, che avanzano nel tegame,o altro

va(o, quando i fa lo strutto, o lardo, da alcuni detti ancora dardings, ficche> — jy

vuol dire facciamo di costoro minutissimi pezzi. Ciccio/o diminutivo, che vieoe> (
da Ciccia; la quale nel linguaggio delle Balie, e de” fancimili vale'apprefodinot
Carne; siccome appresso i fanciuili Greci Tria. ei Whi
SPADACCINI, Così si dicono per derilione coloro, che portano laspadas =p,
solo per pompa. juin seemnigadgtit Une
PALADINI, Cioè Conti Palatini, Quegli huomini bravi, evalorolidifran- —j,
cia cantati dal Boiardo, dall' Ariofto, e da altri; e da questi dicen ty

Mena (e mani come un Paladino, intendiamo buome valorofo; poiche t O tay,

do. Cosisappresso gli Antichi,Ercole, e Achille si veniva a chia a
rofo,¢ dicevano: editer Hercules,¢ di Lucio Sicinio Denotato agg
mano bravissimo, riferisce Gellio lib. 2. cap. 11.5 che per la gt: eras Oy
appellato Achilles Romanus, Di questi Conti Paladini, '0 del Palazzo intese il ei

Petrarca nel Trionfo della Fama Cap. 2, ro) Ne
Cingean costus + uci dodici robufti, + ' d } yy

FACCE di Lioni, ecuor di scriccioli, Mostrano d' esser bravi', ed animosit te

codardi. Lo scricciolo essendo il più piccolo uccello, che ti trovi, ha per conie- ».

guenza il cuore piccolissimo, ed huomo di piccol cuore s'i huomo timido,

e codardo. Vedi sopra C. 10. st. 30, Latino pari, o angu(ti anim Mi-

eropsychvs«

4



I ee

VNDECIMO CANTARE.

597

eet. di rado morde:,. Chi fa molte parole, suol far pochi faci. SE

ape

Suol far poche parole.
STANZA XXX.

bb wel ch? Ella da ritto,e da rove/cio,
— Condicende, va fenando a doppio,

Da sul vifaal Cornacchiann marove/cio

Cun mighio si senti lonan sale 3
et =
anc' egli eA cantocneall pio,
| Mail fapor non gusto già de' buon oo
Come chi prefe il suo de' cartoccini,
» STANZA XXXI,
Sperance per di id gran colpi tira
Con quell' “infornapan della sia pala,

We barte in terra,sempre ch' e la gira,

5 shafiti per la fala,
Tal che ciascuno indietro si ritira,
»O per franco schifandolo fa aia,

Bhi l' asperta, come bavete inteso,

' ek elon Ai) ie i pf

Perch Alsicardo, c' al pafsol? attende,
4 gozz0 gli trafora col pugnale
Ete lo manda a far le sue faccende;

ANZA
r ome il fuggir questa volta non gli vale,

proverbio con dire, Caneche murde, non abbaia s' esprimera la

Curzio:, Aleffima queque siumina minimo labuntur sono; ed. anche
Polidoro. Vergilio: Cave rib
lontano il detto di Catone

se flefic sentenze,habbiamo in uso. anche\nel pariar nostro dicendosi:

a @ acque chete, Guardati dail? acque chere; Chi far di farsi vuole;

acane mnto., \& ab aqua filenics A
1 Demiffos.animos, tacitos vitare me-

STANZA XXXIL

eAmostante, che vede tal fiagello

D? se! arme non usata più in battaglia,
erica la spadaye quando vede il bello,
Tira unfenditeein mezoglicla taglia;
Riman brusto Sperante, e per rowcllo
Li refto, che gli auanza all aria scaglia;
Vola il trovone, eil Dianol fack' eicaschi
Sula bottiglierra tra vetri, e fiaschi,
STANZA X\&XLIL

Dalle diacciate bombole,e guaktade

i vino sprigionato bianco ye rosso
Fugge per b asse, e dann felso cade
Giù dow' è Piaccianteo,e dagli addosso:
Ei che nel capo ha sempre stoccht,e spade,
04 quel fresco di (ubita riscosso,
Pensando sia qualche spada, ocoltello,
Si lancia fuora,evia farpa fraselle.,
XXXIV.

Così dal gozz0 venne ogni suo male,.
Per tui fadi, per lui la vita spende;

E vanne al Diavolche di nuouo piccalo,
A ustolare a mensa appie di Tantaio,

-Celidora esortando i suoi a combattere non lascia di menare le mani; Si nac-

tano diversi avvenimenti, e la morte del Cornacchia, e di Piaccianteo.
SVONA a doppio, Intendi perquote incessantemente. Suonare a doppio inten-
do tucte le campane, o la maggior parte dicfle, che sono in un
campanile » fyonano insieme.

Vedi sopra C, 6, st. 107. Sonare per percuotere,,

il Boccaccio Novella 67. E alzato il bastone i cominciò a sonare. Latino

are.

 MANROVESCIO, E} quel colpo, che si da col braccio all' indietro,cioè con la
|p conuefia della mang, e da quella parte con bastone, o altro, che s' habbia

in mano,
ako scepi se h seid Meneoee un miglio, Il romore si senti molto da lontano, Ioke:

roposito.

een rere sopra C, 3. st. 21.
SICLIANDO un fempiterno aloppio « qu 20 alloppiarsi, o pigliar L oppia;

o cor-



. | ee
so8 MALMANTILE 10%
o corrottamente'? alloppie vuol dire addormentarsi da Opi
Sicch€ qui intende, che prefe un sonnoeterno, cioè mori.)
Oui dura quies oculos, \& ferrens unger Somnus; in arernam gli
Dice; che per —— ¥ oppi rein perché It haveva dato.
tempo, per mostrare, che quis peccat,per hac torquetar,,
di Pincha, che per caula dab guacnisoeetipibans F c
zo muore, " see

INFORNAP ANE, Cioè la pala da infornare il pane,
per arme, oh obi
SBASIT!, Morti, VedifopraC.2. f. 79.0 a a
FA ala, Fa largo; fa piazza'. 'Latino Viam prebere 3 win decedere, fun
HA finito il peso. sa fivito di fare quel, che gli era stare ordinaro; ha
compito; € s' intende ha fino ta vita: Metaforico di questa porgione di
che si da alli bactilani dali loro Capodieci di tance libbre@i lana, che
vorare, la qual porzione chiamano un peso, e dicono bauer finito il peso
peafum, quando hanno finito di lavorar quel tanto', che era stato loro daro.
QUANDO vedde il bello, Quando vedde il destro; il tempo a proposito.. >
REST A brute. Kiman bettato, essendogli avvenuto quello,.che egh non s'al-
pettava, nel qual caso il viso resta macchiato di tristezza'y € i.
confusione.. We
SOMBOLA. Vedi sopra C. 8. st. 44.,.
FESSO. Fetura apertura di legname, o d' altra'materia, o
vasi di terra cotta, Latino Rima, ' ' 3
WEL capo frucchi, e spade. Dubita, che tutto quello, che egli sente, sieno ar-
mi per l'immaginazione depravata della paura; per la quale  rifeofo
tremore, che viene per qualche accidente inaspettato; 'che. ci cugioni

per lo spavento, ches' abbia di qualche cosa improwvifa. Vedi fo he
C. st.2. se RitEe '
SARPA. Se neva. E verbo marinarclco. Latino foluir, anchoram vellit. Bog.
l' aggiuata della voce fratello è posta per emfafi, e quali per un giuro "g hl
LO manda a far le fne faccende. Lo spediice. Quis' intende 'ammazzay — te
PIANT ALO a ustolare. Latino ardere, inbiare. Lo mettevaliato a Tantalo 2 i
desiderar ancor' egli il cibo. Ed usuiare è latino; 'quafi dica + re dal ri
desiderio d*haver quella tal cosa, che egli vede. 'Ovidio negli Ai ¢
indomitis ignem exercentibus curis Fertilis, accenfis menfibus arder %
proposito ci feraiamo anche del verbo spirare. Vedi sopra C. 1, A. 31, diciamo ch
anche Vrolare; particolarmente de'cani, che fanno col mofo atte vie
vande, € per così dire le mangiano coal occhi, € col desiderio. ee, \&
TANT ALO. E' nota la favola di Tantaio hglivolo di Gioves-e di Plotenin'2, 7
il quale per far prova del valore degli Dei 'gli convitd, € diede loroim tavola cot. i
to, e spezzaco un suo figliuolo detto Pelope; Ma gli Deis' aftenaero op
cibo, eccecto Cerere, che mangié le (chiene, le quali le furono'poi Fit 1
Dei, che lo fecero rifalcitare, e confinarono all' Inferna T: r be
cendolo patire di concinova fame, e fete, per thaggior suo te
'metcere sopra il flame Ereditaao, che moltra acque doiciifims,a!;.


VNDECIMO CANTARE:; 509
1 felabbra, ma non tanto, che ne possa bere, e sopra alla testa ha un'
albero-carico di frutte bellissime le quali s' allontanano quand' egli s* allunga per
'pigliarle' 41 nostro Poeta, che-ha de(critto Piaccianteo per un' huomo golofo

'y che morendo,egli fara confinato all Inferno, € per questo suo peccato di
ola fara mefio allato a Tantalo a /stare anch' egli, come fa Tantalo,vedendo

ha da faziarsi, e che non possa haverla. Bologninus. i
Tantalus bic etram fitiens potare vetatur,
Ha 'a quod Pelopis Dijs epulanda dedit,
quali Omero nell' 11. dell' Viiflea descrive la pena di Tantalo, tradot-
Latini suonane così:
Stat miser in medio; medijs exardet in undi

Tantalus,\& fruftra circumfert pallidus ora,
Proximus illudit mento circumfluus humor
Et prope Yorantes contingunt corpora grtra,
Et crines,@ barba madent a/pergine crebra;
Dumque undam captar fitienti Tantalus ore

STANZA XXXV.
Era un camer ata un tal Guglieimo,
Cha la labarda,e ifuci calzonia strisce
Virbigonicinolohaincapo in vece d'elmo,
E'tutto il reffo armaro a flocchefisce.
» “Alemnnno è costui Perneiter [celmo,
» Econ quel dir che brava,ed atterrisce,
Sbruffi ferenti (earicando; e rutti
Ln un tempo spaventa,e ammorba tutti,

STANZA XXKVL

Humoremque cavis, fentat tomprendere palmis.
Hen /upito, ben longe fugitura recurfitat unda,

STANZA XXXVIL
Perché voltando il ferro della cappa

Verso Alticardo a vendicar [ amica,
Quei ghetascafaye glittra sotto,e'l chinppa
Con la spada meixo del bekico,
Ond'sl vim pretto in maggior copiascappa,
Che no mesce in tre dil Inferno,e tl Fico,
Ala non va mal, perch'e: caduto allotta,
Hentre boecheegia tutto lo rimborta,

STANZA XXXVIIL

1 Costud a quel ghiortone a tutte Uhore Gira Sperante pegeio a' un mulino

Fu buon compagno a ber la maluagia, Perch'arme alcuna in manpiiend.gl resta,
rer non cadere adesso in qualch'errore, Par trova un tratte un pie d'un tavolino,
yi E far' un torto alla cavaleria, £ Ciro incontra,e gls vuol far la festa,
a] Pur'anco gli vuoi far,mentre chrei muore Aa quei preso di quivi un sharagline,
, Con farsi dar due crocchie, compagnia, Voa casa con esso a ini fain refia,
E non duri molta farica in questo, Perché passando ? offo oltr* alta pelle,
: ~ Chet trove chi [pedilo'e bene,e presto. Nel capo gli raddoppra le cirelle.,
. Seguitando il Poeta a narrare gli accident occorsi in questa zaffa, dice, che

Alticardo ammazz0 Guglielmo Lanzo, che volle seguicare in morte Piaccianteo,
come l'haveva seguitato sempre all oiterie; B Ciro Serbatondi ammazza Spe-
rante, con battergli un tavoliere da giocare a-sbaraglino in fu la testa,
GVGLIELMO Tedesco. Fu questo Ledelco Soldato della Guardia pedestre del
Serenitfimo Gran Duca, la quale e composta d' Alabardicri veltiti a livrea con,
brache larghe fatte a strisce paonazze, e role,¢ si chiamano Linzi. Vedi fo-
pra C. 4. stan. 4. E perché questi non portano ferraiuolo, o cappa, diciamo per
ascherzo ferraiuolo, o cappa quella labarda, che portano in spalla., come vedre-
me



gre MALMA NETLLTW

mo appresso stan. 27. e s'è accennato sopra €. 9. stan. 48..€  r
date, o percofie colla jabarda. Costui era molto amico di-

aiuto a mandar male la roba, e però il Poeta dice, ch' ei lo vuol

in morte 2) La OORT

BIGONCIVOLO.. Diminutivo di bigoncia, detto sopra C, 10, stan. 7
costui con un bigonciuolo, arnese, che per lo pi s' adopra al vino,
che in tutte le (ue operaziont egli haveva l'animo al viao, e con
( che vuol dir pesce bastone, vivanda assai usata dai Tedescht ) per m
alla voglia del vino haveva unita ancora quella del mangiare. Si. ar
ancora, che il Poeta voglia mostrare, che costui era fudicio,.¢ c
in effetto egli era, e come per lo più sono questi Lanzis a caula forse di
pelce, che veramente ha sempre malo odore.

BEKNEIDEK Scelm. Voci Todesche le quali in nostra ipa suonane
cone, scellerato,

ATT ERRISCE, Spaventa. La pronunzia Todesca ha un certo accento,
fa credere, che colui, che parla bravi sempre, € per questa rozzezza di 'al
gua dicono che ella sia propria, ed il caso a comandare eserciti, come la Fran-
ccle a aoe con dame, la Spagnuola al comando politico, ie cuaanaraerey
queste cose Pr

SBRVEFL, BE? quel mandar fuori per bocca il vento jonato in carded
prabbondanza di ae E ratti si ie dire lo stesso, aso che per rasse inten
diamo il puro vento,¢ sbruffo si dice quando il vento vicn fuor del corpo «
no firepito, che non viene il rutto, ma accompagnato con un poco sae;
¢fiendo lo cheafare un mandar fuori di bocca con violenza vino, o altro i
AMMORBA, Fa putire. Vedi sopra in questo Cant. stan, 23. quie pe

significato attivo, cioè appefta; mette la pefte in tutti. '
GHIOTTONE,, Gran go.olo; Gran ghiorto. Intende di Pisesbame a

MALV AGIA, Specie di vino assai noto; ed a noi viene di Vi qui ie
pigliando la specie per il genere, intende che gli fu sempre compo be a tain
sorta di vino.

CROCCHIE; Percofle, Da ereechiare che in significato attiyo vuol dire P
motere

2 SPEDILLO bene ye preso. In poco tempo gli diede buona sp t 7"
ammazzo presto, ed affatto. Questo detto bene, e presso era il mol 3 7
cademia Fiorentina detta de' Rifritti, ed il Poeta se ne serve, p più
fu già di detta Accademia, ed immita un' altro Poeta, che nelj' umprovvila, o \
byona morte d' uno pure di detta Accademia difie; 'aban bar
E per mostrar, come Rifritto ville, pide eh e to

Mor:, come Kifriteo E PRESTO, E BENE, Ca

EEE e il Fico, Sono due Ofterie di 'Eirenze così nomina' die oo i;
Infega

Bosc HEGGIARE. Quel moto, che fanno con aprire, e (errage la bosaia
mandar fuora gli ultimi spiriti coloro, che muoiono 4

LO rimborra, Rimette nella bors 9 lO¢ in corpo, 'ribeve -
che gli era ulcito di corpo.



.

VNDECIMO CANTARE) Sat

GLI vel far la feta, Cide lo vuole finire, lo vuole ammazzare '
GLI fauna casa in testa. Nel giuoco di sareg eee una casa,vaol dire rad-
iar le girelle, o tavole sopr' a uno de' 24. segni, che sono nel tavoliere, cd
i. scherza con questo addoppiar le gireile con dire che batrendogti il ta-
 voliere in cefta gli raddoppia le girelie, che quiui haveva, e così gli fa una case,
- intesta, che haver girelle in testa s' intende tuomo col cerucllo che gira. Vedi

C. 9. stan. 10.
STANZA XXXX.

-) STANZA XXXIX.
Ritvasse già Perlone un certo Marte, Tofelloych in fere.ra ad bxom non cede
Riesce adefio qus tutto garbato,

o Cthaucua il naso da fiurar poponi,

| E perch ei nol pago mai del ritratto, Perch'ei rifana un zoppo da un piede,
Pere fa seco adesso agli fgrugnoni; Cregnor fu quella parte andò seiancato,

| Edieglien' un si forte. ch' in quell' atto Mentre di taglio un sopramanglidiede

Gli si fhianto la firinga de' calzoni, dn quel, che sano havea dall'alrra late,
| Che qual tenda calando alle calcagna Che pareggiolio, ond' ei fu poi di quci
 Scopri scena di bosco, e di campagna, Che dicon: qui¢ mioye qua vorres,

% STANZA XXXXIL
Grazian di sangue in terra ha fatt'un bagno Che vie da un trcbettier di Carla Atagne
Onde gli è ya 4 chi va gin che nnoti; Quando le molfe dar fece ai tremors;
 Afetta un Salta,e xn Birrocolcopagns Toglie ad unl'asta,tl qual fail Paladine
1 E frroppia uneal, che fale erucce aiboti, Se ben con essa fu [parzacammino,

“Seguita a narrare varj accidenti occorsi in quella zutla, e le racconca le bravu-
re di Tofello Gianni, e di Grazian Molletto.
SU ffianto la firinga de' caizoni, Si roppe la stringa, cioè quel legame, che ferra
calzoni in fulia pancia.
TENDLe4. Intende nel presente luogo quella tela, che si mette d' avanti a i
chi »sopra i quali si rappresentano Commedie, affinché cuopra le scene per
Doprine nel dar principio alia Commedia; Lat. /iparinm, e però dice, che i snoi
calzoni. essendogli cascati,, scoperfono scena di bosco, ec, cioè quel, che da loro
'eraycoperto. Caso veramente seguito a Perione, che,per voler ae pagato d'ua
Fitratto., che egli havea'fano a uno, gli conuenne fare alle pugna, ed ia quel
re gli cascarono i calzoni.
SCIANC ATO. Uno, che va zoppo per haver difecto nell' anche, offo princi-
pale delle cosce. Vedi fupra C. 6, stan. 82.
. CHE dicon; quie mio, e qua worrei, Così diciamo di quelli zoppi, che vanno
a gambe larghe per disecco, che habbiano nell' anche, o in ambedue le ginocchia,
€ non posano i piedi in dritto, secondo J' uso comune, ma pare, che vogliaao
can un piede andare in un iuogo ye con' altro in un' altro, e che accennino qui
 mio, €qua vorrei, Di questi tali diciamo ancora Andare a feiacquabarili, perch
fanno lo stesso moto con ia persona, che fa uno, che (ciacqui un barile «

AFFETTA, Taglia da una parte all'altra, come si fa al pane, del quale
propriamente si dice affettare, o far fette.

VN Sata, Si chiamano Salti quei famigli, e donzelli dell' Arte dell' honefta
“(che in Firenze € il Magistrato, al quale son fottoposte le Meretrici ) i quali fanno
ogni sorta d' cfecuzione tanto Civile, quanto Criminaie contro le Meretricé,
t VN

me

i ee


le figure di carta petta:, le
di boto, e d' haver ricevuto ae
cono Bari. Vedi sopra C, 4. tt

sente 'uogo il nostro Poeta,

5 tz MALMANTILE | (¥
VN tal che fale erucce a boti « Intende' uno seultore dappoed, che.
qnaii @ mettono alle Immagini sacre.
razia; e queste figure:co
c. £7. Gruccia è dal Lac, barb:
€ bastone fatto a croce; onde in alcuni Juoghi della '
Far le grucce a una figura, s intende fra i pittori.
stan. 27, Intendi dunque, che costui era scultore stroppiatore dit
fabbricava se non faacecci di carta pella, formati confornie di gi
no di quella bellezza, che può vedere?chi andra nelle! Chiele
miracolofi; e queste figure faceva così male, che le strop
da sapere che /eultor da bori suona fra gli scultori lo stesso, che fra i
Pittor da fgabelli, dewo sopra o, 4. stan, 10, Questo tale ancorché fulie:
¢ nato d' intima plebe,  ttimava un Buonarruot; essi piccavaidi nobile
dice, che yen da un tromberta di Carlo Adagno, quandevle moffe dar fac
Cioè ha origine da un trombettiere,dei
re.i bandi, che dar de mofe a' tremori, vuol dir comandarfo\

ticamente, se bene in deco scherzolo,¢ per derisione, come se ne serve nel pre.
> aa

aria
java affatto e Ino

quale:Carlo Magno i serviva per manda

Ae

SPALZZACAMMENO. Vanno per Firenze alcuni o Marchigiani o Lombardi
con una pertica in spalia gridando: Spazzacammina, acciocthé pia
che essi ripuliscono le cappe, o gole de i cammini dalle filiggine » Vino:
tait era cului, il quale con queli' alta, clod con ta pertica tr
ladino.

STANZA XXXXIL
Tutto tinte ne va Puccio Lamoni
Stoccheggianda nel merzo della Vuffa,
£ in Pippa un tratte da del Castitioni,
Che majcherato ancor tira di buffa;
Ea ci che nel sentir quei farfallont,
Venir più tofta sentefi la musa,
Passandalo pel petto banda banda
Ai far rider le piattole lo manda,
STANZA XXAXKTL
Nanniruffa ha più la pien di ferite,
Pericolo, che fu [copa meftieri,
Fu pailaio, Senfale, etitor di lite;
Srette Bargelioy ed abbaco di xeri
Prefel appalte alfin dell! acquavire:
Ala pris fuaniro i fuvi peafieri,
Lon pite il wana fhillando, ma il cernello
Per mettervi poi il moffo,el'acquerelio,

Continoya a narrar quel, che segue neheombattimento,

mazzamenti,

TVTTO tinte, Vuol dire adirato, ma il Poeta si serve'di
ché detco Puccio è di faccia bruna, come s'€ detto sopra C.

6 OP
STANZA XXXREV) o
Con Duriano ii Purba eccoalle wank o
Di ferro da fradceri i Safe,
Ev altro una paletta tq
E con est a tui cerca,e sbracia
Ma percht quei le fqnete, tome cani
Gi fraricatt fuaf hs chibufo, (4

Chreghi ha a' Monnini,evane

Fatto d ognun polpette
S' 4 tanto mal non se:
Col dar sul grifo-a tui Salue Rofata,
Chef. oui 2
Vuol ch' e facia pere

Cb? essendo presa
Lo spinge fuor

«=. FREER

=a.



wii =

a a ae

Pre Sst = = -

Sate Sb tes eee

VNDECIMO CANTARE 513

 TIRA4di bufa, Fa i buffone. Le buffe, come accennammo sopra C. 2. staa.
2. alla voce bu/chette, sono pezzetti di mazza rifessa, e formano quaai un dado,
se non che hanno te parti piane, ed una conuefia, e si tirano come idadi, fa-
eendo Con esse quei giuochi, che si resta d' accordo con sei, orto, o più di cali

 buffe; e per me stimo, che s' usino, come s' usavano dagli aatichi gli aliogi: ma

: è i e giuoco da fanciulli,percio habbiamo il detto sirar ds buffayche vuol
ire Far cose da fanciulli, ec. da persone di poco giudizio, che poi da questo in
una parola si dice buffone, e far il buffone; che i Latini dicendolo scarra lo delcri-
vono per uno, che rifum ab audientibus caprar, non habita ratione verecundie, aut di-
gnitatis, o così per uno, che non habbia l'intero giudizio da distinguere i tempi,
ee wetl ne le persone, come e per lo più il giudizio d' un fanciullo. UI P.
', Vincenzo Maria Carmelitano Scalzo nel suo viaggio all' [adic Oricntali lib. 4,
¢.26. descrivendo un' uccello detto Buffo [ che è forse.quello che i Launi Bubo,
€ noi chiamiamo Gufo } dice così,, I nottri antichi lo chiamaron Buffo, onde:
y» forse hebbe origine il nome di buffone, poiché è incredibile, quanto questo
a uscello sia inclinato agli scherzi, ed alle burle, con le quali bene (peffo atcer-
y rice di notte, ed inganna la gente.
|. BARFALLONI, Denti spropositati, e sciocchi.

SENT ES! venir la muffa. Si sente venir V' ira; Entra in collera.

LO manda a far rider te piattole, Lo manda a far il buffone nell' altro mondo,
dice /e piartole, perché questi son vermi, che stanao negli aucili, ed hanno oc-
¢afione di rallegrarsi per 11 nuovo cibo che a lor viene dall' andar egli nell' avello,

PERICOLO, che fa Scopameftieri, Si dice Scopameftieri colui, il quale seguita
poco tempo a far un' arte, ma lasciandola stare ne vada a fare un' altra, perché
la prima non gli piaccia,come appunto fece questo Aleilandro Violant detto
Pericolo, nominato sopra C. 3. stan. 58. il quale veramente fece tutti i mefticri
enunciati nella presente Ottava 43. ed in ultimo si diede.a trovare invenzioni di
Mettere appalti; cominciò dal Tabacco, e poi l'Acquavite, i quali senza suo
utile, o pochiffisno conchiufe per altri. Dice, che abbaco di zeri,perché veramen-
te ci fu un grandissimo abbachifta,e per questo havendo saputo trovar degli erro-
ri. contro a' ministri grandi, fu da essi perfeguitato si, che fu mandato in gaiera;
Ma havendo le notizie date da lui fatto al fine (coprir la verita, furono i delin-
Foe gaftigati, ed egli cavato di galera. Dice abbaco; ma percht questo verbo

gnifica ancora star dietro a fare una cosa,¢ non trovare la via a terminarla,
per non haver tanto giudizio, o scienza che a ciò basti, il Poeta piglia tal detto
in questo luogo nell' uno, e nell' altro senso, cioè, che egli fusse veramente gran-
de abbachifta, e che egli abbacasse, cioè armeggiafle col cervello senz' utile e

conchinfione, e però v' aggiunge di zeri, perché, sia pur grande un' abba-
¢hifta quanto si vuole, che mai non rilevera somma alcuna, se non si servira d'
altra hgura che del zero, Cos} in effetto fu costus che con tutto il suo grand' ab-
baco non pes mai far conto, che gli tornafle bene, e con tutte le sue arti, ed
invenzioni si può dire che abbacasse, perché in ultimo si mori quasi di fame,

PIGLIAR ? appatto. Quand' uno col pagare ai Principe una somma convenuta
Piglia ' assunto di provvedere uno Stato d' una mercanzia, e fa proibire che -al-
tri la possa vendere, o fabbricare senza sua licenzia, diciamo pigiare appaito, che
Sil Las, Adonopolinm. Tre MET-


514 MALMANTILE si

MET TERVI il mofto, eI acquerello « Consumarvi tanto le bu
tive fuftanze.. Oleam, \& operam perdere,

FVSO da StradieriChi fiend gli Stradieri dicemmo sopra C. 3.
sto lor fufo e un ferro sottile lungo, ed acuta, col quale forano i
altro a fine di vedere, se vi sia occulrata roba, che paghi gabella.

PALETT A da Caldani, B' una meltoletta di ferro con manico: g0 » che
serve per iftuzzicare i fuoco 'nel caldano,0 focone,il quale, che cosa sia, Vi
C. 3. stanza 3.; ee

SBRACTARE.. Vuol dire iftuzzicar la brace, perché s' acceada, o P'accelas tie
spandere alquanto, e qui dicendo: gf sbracta il mufo, intende, 10 perquote con la

paletta nel viso, e gli¢ lo sCortica. 1. 20.) aga Deke
4 LE squote come fanno icani, Non ttima, Non cura le buffe.. Vedi sopra C, 10, Suan
anza 36. 5 'sun Obey Mec
eARe HIBESO ch' egli ha a' Monnini, Doriano fa morire il Fucba con 'uno: Sino
quei suoi Monnini detti sopra C, 1. @. 44. i quali Monnini ij Poeta insieme cons Nan
ogai altro stimava tanco sciocchi,e odiosi, che credeva fuflono abil a far morire Celido
uno di naufea,; al fen

SQV ARCINA, Spada corta,e larga,altrimenti detta colrellao mezza/pada. Te
POLPETT A... Vivanda nota fatta di carne benissimo bactuta con coltello sed
impaftata con uova, cacio, pan grattaco, fale, spezierie,ecs > an Oh Difer
CERVELLATA, è specie di falficcia fatta di carne, o di ceruelli di poreo La
triturati, ed imbudellati come la falficcia. E dicendo far poiperte, e ceruellaths Tere
4' huomini intende far macello, e strage d' huomini.
CONTADINA., Specie di danza usata nel Carnovale, la quale confifte tutta hag

in forze in questa maniera,, Octo-, o dieci huomini si fermano ritti col im Cel
fieme in giro con le braccia alla coliottola l'uno all' altro; opr' alle. di ha
quciti faigono quattro, o sei», sopra i sei altri tre,¢ soprai tre wao,¢ fatea que- att
ita regolata mafia vanno girando a tempo di suono,, ed in ultimo quello, che € to
cima sopra a tutti, fa un capitombolo sopr' alle spalle di quei tre alla volta delter= e
reno, dove e ripigliato da due, che sono quivi a tale effetto;:nello stesso modo ty
fanno poi i tre,¢ poi i (ei, e dopo questi gli otto, o i dieci fanno iltcapitombolo ile
in terra; e questa dicon far /a tombeiata. EB percht Mato di Coccio.in: for- te
ta di bal'o era Maestro, € però dice, che Salvo Rofara sapendo, bea la re
Contadina, lo fa fare la tombolata gil perla scala. aan Ui
STANZA XAXXXVL STANZA XX¥EKVIL- ti
Palamidone in tanto con la mano, Quasi di viver Bariftone uso, > '
In tasca a Belmaforto andana in volta, Egeno affronta con un prmerwoloy he
Per tirarne la borsa in suw pran piano, E perché quei |" uccedia come nn gifoy i
Per carita che non gli fusse tla; Salea ch' ei pare'un gailestovmapanele ip
Mail buon pensier ch' egli bayrie/ce vano E raito fa cht iL manbeaeenpe, ta
Perch' egli col pugnal se gli rinolta's Manda My
E fa per carizade anch' e che muoiar, E por to pi: the
'extecio fa vita non gli tolga il boa, 'Per dario per un 1



EB passagli un vestir

ee

STANZA XXXXVIIL

Exquei gli duol che'l rinnono quell anno,
- Bfee' si muor vuol che gli paghi il danno,

ae VNDECIMOCANTARE, 515

STANZA XXXXIx,

Romolo infilza “to mezzo al bufto > L' armi Papirio ad un Prandron guadagna,
'tes iyoenunise un canto erafugviasco, Che. fae apiacuhine lo Swillerra;
Efe ne muor con molto suo difeuito, Ma  a parole gli è Spaccomontagna,
«Perché egli haveva a esser aun fiasco; AUP ergo poi riesce Spada fanta,
Tira inun tempo fifo aun bell imbufto, Perchheifactee it al Cel dar lecalcagna,
demmafeo, 'Won una voir dice, ma cinguanta:

Sta[uch'in terra i pari miei non danno
Ed ei risponde: S'io sto (uy mio danno,
L

STANZA
riga il Mula, ePoste degli allori,
Son mandati per sempre a far un sonno,
Miccioge'l Baggina da Strazildo Nori
Sono inviati done andò il lor Nonno,
E nelle parti giù posteriori
Panfiloagginfta Meoyche vendeil tonno,
Tal che s* allor putina, hor chi accofta
Sente che raddoppiata egli ha la posta,

\ Narra' morte d' alcuni disensori di Mal mantile, e le bravure de' Soldati di

a, Se'brami tanto d' intendere i nomi anagrammatici, quanto di sapere
chifieno gli altri. Vedi sopra al C. 1. ed ai C. 3,
STVEO. Sazio. Annoiato. 2

 PENT ERVOLO.. Piccolo file di ferro aeuto, del quale infra gli altri si servo-
no i farti per far buchi agli abiti.

DB aecelta > Lo baria; lo schernisce. Dice come un gufo, cioè come fanno gli
ucceiletth al: gato, che è uno uccello notturno, e simile alla Civetta, ma assai più
grande } chey Latini dicono babenem, donde bubbofone si dice a uno spropositato
chiacehierone; e bubbole i racconti spropositati, e non' veri ( forse da Bubbola uc-
cello, Lat. «pupa. ) In questo uccello detto gufo, o barbagianni, favoleggiano
giù atichy Poeti, che fusse mutaco da Proferpina quell' Ascalafo, che fece la spia
a.Proferpina d' haver ella mangiato la melagrana, il che fu causa, che ella non

¢ ulcir daii' Inferno. Ovid, 5. Met. Questo uccello € forse lo stesso, che quel

Pgeedel quale habbiamo detto sopra in questo C. stan. 42.

~ GALLETTO marzxolo, | galli, che nascono del mese di Marzo, quando poi
fifega il grano son più grandi,e fs gagliardi di quelli, che nascona d' Aprile,
eper questofaicano piii alto alle spighe del grano, onde col dire: Salea come un
galletto marxvalo, s' intende falta gagliardamente.

 LL mal tarenfo, Vuol\ dire huomicciuolo di cattivo animo, che i Latini purer
dicono boma fungini generis.

4VEFETTO. Intendiamo una specie di tavolino; ma quis' intende un colpo,
che si da'col dito di mezzo accomodato a guila di molla a! dito pollice,o ( come
diciamo ) dito geoffo, e poi lasciato (appar con violenza al luogo, dove si vuol
colpire «| Moiti pero per bufferto, o buffertune, intendono.colpo di tutta la mano;
¢ appresso gli muoli Boferada, o Boferon vuol dire moftaccione, guanciata.,

Macon questo huomicciuolo, che non era da pugna, o simili, si può credere,
che intenda veramente pufferro dato con un fol dito.

BAR querciuole, Cioè con le gambe alzate all' aria, € s' intende st ammazza,

-Lnoftri ragazzi dicono far querciuolo, quando no pola le mani, ea testa in,

terra, e manda le gambe all' aria; quaft mostrando qd essere una.pianta, la sc
od Tee "2 a



——

516 MALMANTILE | o

ha,della quale sia il capo, il corpo sia il futto,e i rami le zampe. ho
seguente dice dar /e ca/cagna al Cielo sche vuol dir caduto in terra b Bul
così si mostrano le calcagna al Cielo, e fi'dice anche mandare a gambe | no

FVGG/ASCO. Riurato, fuggitivo. Vao, che per paura de' birri sg
vedere, se non ne i luoghi immuni. we ky

HAVEVA a offer a un fiasco, Croe 8 haveva a trovare a bere i 5
Quando alcuni voglion bere insieme un fiasco di vino, € pagarne i
ii valore per mettere insieme la cricca, dicono Chi vaol essere 4 un fiasco? Mi,
tende chi vuol accordarsi a bere, € pagar cia(cuno la sua parte? BY termiae! Bad
fo, ed usato fra l'infima plebes ate a0

BELL imbuffe. Bella preteaza, Va di coloro, che Manno in fa la ky
quaii non hanno di buono che la prefenza, da 1 Launi soprannominati 4
per metatora, perché /folones si dicono quci bet rami, che noa ab
donde noi diciamo folly a uno che non € buoao se non a far comparla,o v
za,come si dice qui 7 bell' smbuffo, che diciamo ancora wa bel coram Vobis. A
Tulipano, diciamo a uno, che abbia buono aspetco; e poche altre quali Ti,
similitudine del fore così detto, venutoci di Turchia, che va imitando la! hare
¢ la vaghezza della Tulipa, o del turbante Turche(co,ondehailnome, =u

DOMMASCO, Deito così dalla Città di Damatco in Levante. Specie di v
drappo fottile di feta fatto a fior1, o ( come diciamo ) a opera. os baa

RINNOVO! quedl'anno, Se ' era fatto di nuovo quell' aano, Pare che sia foli-
to quando altri si fa un vestito nuovo per li primi giorai, che -adopra havers = nd
giù qualche riguardo di più, come faceva costui, che per esser ii suo vettito nuo- T
vo, l'apprezzava più della propria vita, poiché rinfaccia, e proreiladeldanno
del vestito, e di quello della vita non ne dilcorre, oem oie ¢

StanDROWE, Huomo di Fianiira, Ma perché huomo di Fiandea diciamo j
Fiammingo, la voce Fiandrone ci fertic per esprimere Vino spaccone, éhe si vanti P
di bravo,raccoatando le prodezzc tacte da im fuori di qua, ed uno di quelli, che b
i Latin dicono milires gloriofos, ed in questo senso lo piglia il Poeta nel presente i
luogo, se ben (cherza con l'equivoco; Ed egli stesso lo dichiara dicendoy Che» I
fan Taghiacantons,e lo Smillanta; all' ergo poi riesce Spada fanta, cioè fa da bravo, ha
ma dovendo venire a i fatti, e alia conclufione, riesce una (pada, che non fa mal ¢
veruno, e pero Santa; ed in fultanza un poitrone. Dicesi nell' uso, "i
buona pada; cioè € huomo, che fa bene adoprare la spada. Nel Pianto che't Pe
Carlo Magno nella morte di Rolando da' nostri Poeti detto Orlando, appresso vy
Tarpino Arcivescovo di Rems, e compagno in guerra del medesimo Carlo: 6 die fing
ce. O brachium dextrum corporis mei, barba optima, decus Gallarums, inf hg
Carlo chiama Oriando Spada della giuftizia alludendo alla formidabile spada da ie
Turpino detta durenda, da' duri colpt ch' egli dava con etia da' poeti Darindana, Il
oh wrath rf, o fmill. dich un nostro pi bio in. di,
che dice La fradera del' kiba, che vuol div vantatore di gran cose 50;
re; Equesto perché la stadera dell' Elba; che serve per pefare barche piene ey
ferro, acile sue tacche comincia a contar da/ mille, e seguita s -a migliaia
Tagliacantoni, cioè, che tira gill pezzi di muraglia corrifj | Pyrgo ii
wices di Riawto, Che vorrcbbe dire in noltra Lingua Atrerrasor ty



ee e

4
Ai. si

a
VNDECIMO CANTARE. 537

Lo Smillanea, cioè Smillantatore si esprime dal Greco Thrafon, cioè Audace.,

BHES Ske

ee

Sesh o £F

SPTVSILRS Pee Thr ees CUR ET

Baldanzofo; e dal Latino Adiles gloriofus. E la parolaé fatta da Adidanea, (cher~
'zofamente usato dal Boce. in vece di mille; dandogli la desinenza di quaranta,
cinguanta, e simili; quasi uno non sia contento di dire la semplice parola di mil-
le, ma la voglia go > e far parere la cosa più di quel ch' ell' e in esserto.
'S' io Ho fu, mio danno, Non mi rizzo al certo. Questo termine mio danno usa-
to in questa forma, e specie di giuramento, ed ha la forza del termine appon/o 4
noi, decto sopra C. 8. stan, 72. € 3° io non' ho,egli e fallo,detto sopra C. 6. (tan, 86,
MiCC4O, Così era nominato un garzone della pallaa Corda, che e uno di
coloro i quali stanno nel mezzo della stanza, mentre si gioca, a raccorve la pal
la, e rammentare il giuoco.
BAGGILANA, o Baggina, Eva un Battilano, che in occasione di felte serviva
ai Bawtilant per tamburino.
DOVE anao il lor Monno, Cioè nell' altro Mondo. Vedi sopra C. 4, tan. 2.
NELLE parti posteriori. Cioè nel c....0 come baflamente si dice, nel preterito,
dove dice che e prima putiva, hora pute il doppio, che questo vuoi dire
ha raddoppiato la posta.
e4GGIVST A. B' preso ne) senso medesimo, che è preso sopra C, 2. stan. 41.
CHEO che vende il Tonno, Fu un venditore di peice falato,¢ tali huomini
hanno (empre addosso cattivo odore.

STANZA LI. STANZA LIL
In abito Scarnecchia da Coviello,
Tinta de brace l una,el altra guancia,
EB per sua spada sfodera un fuscelio,
C" al pome a' una bella melarancia,
Rinolto con quest' armi a Sardonello,
Perma, gis dice, guardati la pancia,
Ed enrisponde: ueftoé pensier mio,
z rant un colpo, ete lo manda a Scio.
Gustavo Faibi con un soprammane,
Di nerty il capo fmoccola a Santella
Scaramuccia si muar fotte Erauano,
C' aimazza anche Gaba da Berzighella,
E fuentra quel birbon dell Ortolano,
Che fa il minchion per non pagar gabella,
Ma colto poi vi reffa ad ogni modo,
Mentr' adesso gli va la vita iv frodo,

Descrive l'abito, ed armi di Scarnecchia,
che refto morto da Sardonello;

Eravano ammazza Scaramuccia, Gaban da Berzighella,¢ l'Ortolano.
COVIELLO. Cioè lacoviello maschera, che finge un bravo sciocco Napole-
tano, 'a quale s' aggrotte(ca con fargli i bafi alla Spagauola col nero 41 b ace,es~
PerO dice Tinto di brace? una, el' akira guancia,¢ con armaria d' una spada faca
d' una mazza, che ha in vece di pome una mela, o melarancia, o altra frutra
simile per rendere il personaggio più cidicolo,e così vestiva questo Montambanco,
facendosi.chiamare Scarnecchia... Vedi sopra C, 3. tt. 62, Così Cosa, e Zanni,
personaggi ridicoli di Commedia sono nomi proprj de' loro paefi, donde si fingo~
.no»s accorciati dagl'interi nomi Niccola, e Giovanni; onde va in terra lorigine
di Zanni, che alcuni ingegnofamente hanno tirato dal Latino Sannio, mis.
LO manda-a Scio, Lo manda all' altra vita, ed è lo stelio, e si dice per ia me-
defima ragione, che mandar a Pasraffo,0 a Buda, detto sopra C, 5. st. 134
- SMOCCOLA il capo. Taglia il capo... Smoccolare si dice tagliare il Lucignolo
di una candela, o altro lume per levar quegli escrementi, che fa la fiaccola, che
hiamali f i. » che queiti Spagi sear'



8 MALMANTILE

desfavilar quasi exfavillare; il Vives disse exfungare formando la all
Virg. 1. Georg, Scintillare oleum, \& putres concrescere fungos, ol
SCARAMVCCIA, Vo' aitea maschera, come Scarnecehia - tit
Ourava 51., ma questo era Iftrione, e non Montambanco. i owe Roi
GABAN da Berrighella, Questo pure era Iftrione, ce rappresentava wo |
dt un Romagauolo ttoito. ' = Oe
L'ORTOLANO, Costui fa un yeechio aftuto, che: per ein
dovutali per aicuni delitti commeili, s' era finto ae 1 Del
chion per non pagar gabella, Menandro, Rufticum essete fimulas, tam Par
vi resta colto, cioè viene (caperta questa sua malizia da Bravano, che Per
vita in frodo, a colui, che non volea pagar la gabella,e¢ vuol wae Sin
in vece di frede solamente l'usiamo di dire dalla fraude, che si comm el
pagare la gabella. Ta
STANZA LIL i
e4rmato a priuileo} omai Rofaccio Che piove al
Marte sguaina, e Venere influente, Ond''ci in quel pumoandada Nan
Ma ae Sardonello sul moftaccio Vede le elle, e linac t altrasfera un
Gli fece con la spada un' ascendente, Nel viso ectifia,e dice: Ty
Rofaccio ricoperto di privilegj cava fuora Marte; e Venere; che Pe
tivi influffi, ma Sardonello fece piombare sopr' a di luiun pefimo % Tee
tagliandogli con un soprammano parte del vilo, e del collo, ed un braccio Rani
il qual dolore egli vede le stelle, ed eclifiando l'una, € laltra sfera del Coat
ferrando gli occhi dice: Buona sera, cioè perme, fatto buio, «B Mi
sto Rolaccio si piccava d' Aftrclogo, come s'è detto sopra C, 31M. 63.5 11 Poeta tg
con la presente Otrava descrive la di lui morte con equivoci di termini affrolo- pred
ici. f Lapa
: ARMATO 4 priniteg|. Questo Rofaccio, come ancora gli altri Montamban- ro
chi per accreditare i rimedj, che da essi son dispensati, mostrano una infiuna di iw)
privilegj concefli loro da diversi Principi; e pero ii Poeta lo fa axmato di privi- the
legi. Uontanlald ken

SGVAINA, Virgilio vagina eripit, Sfodera Marte, o Venere; che predicono
rovine; B dice /gnaina, che vuol dir cavar la spada dal fodero, o guaiaa, perché
s*intende, che non haveva alcr' armi offenfive, che Venere, e Marte unfluili
cattivi 'a duaaiead

ASCENDENTE, Termine astrologico, col quale qui intende colpo di taglio, Un

che viene da alto a baflo, piovendo, cioè calando in sul capo, ec,

OCCIDENTE, Intendiamo l'occalo del Sole', maqui intende ocealo, cioè è
morte di Rofaccio, ily oni aalaan ey baatee tm
VEDE le free. Quand' uno fence gran dolore; si dice + Eeli ha veduto le fielle, hi
perché le lagrime, che vengono in (ugli occhi per il dolore, G

la rescazione della luce, che yi batte, una cosa simile a una quantità di mi -

nute stelle in Ciclo, che più volgarmente diciamo veder 'nce i

mo sopra C. 9, st. 60, 5 ma qui si serve di questo, perché.gli

re di farlo morire astrologicamente, i
ECLISS.A. Chiude, cuopre; ficome alla Luna 'restano i

»hajean > x

3



VNDECIMO CANTARE. sip
 dail interposizione della Terra 1 raggi del Sole, quando seguono I ecliffi.
DICE buona sera, Cioè si fa buio per lui, ven donate 10. st. 5. Qui intende
è finito il giorno del mio vivere. Virgilio in-aternum clauduntur lumina noflem, o
i asadostndelcginnanalo » che, havendo:manco un' occhio, e Ji fa ca-
vato l'altro, disse: Buona worte per tutto lo tempo, '

i STANZA LIV. STANZA LV. i
Mein per fiancofentefi percosso Già per la franca il sangue era a tal segue
Datlo stidion del cuciniere Melicche, C” andar vi si potea co' mauicelli —

 Parafiraccio porco grande, e grosso Istrion Vespi tutto furia,e sdegne
Perch' il ghiowso si fa di buone micche; Rinualto ha quivi tl povero Adaffelli,
| Sirivolta eAeino,¢ da al coleffo E col coltel da Pedrolin di legno
Nelda gola ch' egli ha pien di pasticche, Su pel capo eli squotola icapellig,
«Tal che morendo dolcemente il guitto: acleciopratcane poi la lifoa, el Lota...
» Addio cucina dice, ch'iobo frito, Pius bella faccian la conocehia a Cloto.,
ils STANZA LVL
NGatsi,, e Paol Corbi inveleniti A tal ch'i pacfani sbigottiti,
| Quali villan ch' i tronchsyed i rampolli E dal disagio sconquassati, e frolli
 Taglin di marzo ai fratti ed alle viti (Oktre che a' pachi il numero è ridotto)
| Potanda i basts braccia, gambe,e colli; Cominctaron le gambe a tremar sotto,
. Termina con te presenti Ortave il racconto del combattimenco (egaito in Mal-
mantile,, e dice la morte:di Melicché, ¢del.Mailelli., e qui tinisce ' Vadecimo

re

MELICC HE, Vedi sopra C, 3. st. 59, lo chiama Parafiraccio,perché era huo-
Ȣdel continuo havrebbe mangiato: EB questa voce Parafito, che appresso
di noi ha dell'ingiurioso, non era così appresso gli antichi,come si può de-
durre da molti Autori tra'guali Luciano; ma particolarmente da Piutarco, dove
fitrova': Parafitos nontancumappellabant strici adalatores illos, qui apud Dinitum
tmensas wutriuntnr, fedietianm tos,qus ob rem egrecit gestam,publico /umptu in Prytaneo
atebautur Oc, Onde delle Stinche di Firenze, nel capitolo in lode del Debito, il

Bernt; è
Voi fore quel famoso Priranco, è
Ab bower yas Doe renews in grassoin fisoi baront
I popaly che discese' due F efeo,

Exit Atheneo Parafiti olim appelabuntur foci, 7 fideles Pontificum, eAMagiftratiiz,
Ibmedefimo Plutarco.. ¥

PASTICCHE., Specie di confezione fatta col zucchero\muschiato,:ec; e però
dice more 'doicemente, perché ha gli per la gola 11 zucchero, Pa/feca voce Spa-
boas » siccome anche 'Pa(figlia, che vale lo stedo; e sono tutte due diminutivi di
pasta.

GVITTO, Huomo vile, abbietto, fudicio, sporco., e sciatto. Vedi sopra
C, 3.\f..9.è:voce Napoletana, ma usata oggi anche da noi, 'Nella raccolta de'
 Poeti antichi-dell' Aliacci, Pra Guittone-scrivendo un Sonetto, siccome da esso si
raccoglie.a Messere Oneito da Bologna 'Poeta, e amico suo; scherza sul nome di
turer € die, *

— SAS QF Cisasn see

=

Pita

A.. SSe



g20 MALMANTILE | o ¥

Voktre nome, Messere,¢ caro,e onratoj
Lo meo assai ontofo,e vil pensando, =
Ma al vostro non vorrei auercangiato, =
10 ho fritto, Scherza col verbo friggere, che vuol dir Quocere carne,0
padella con lardo, o olio; ed il detto ho fritto, che significa il
in malora. Latino Attum est de me; perij. Vedi sopra C. 8. st. 54, tor
nel presente luogo, perché par che dica; Addio cucina, ti lafio non
più bilogno di te, perché io ho già fritto, ed intende ho finito di vivere.
IST RION Vespi. Pietro Sufini. Questo fu cognato dell'Autore, e giù
grandissimo (pirito, copiofissimo d' invenzioni, come si vede in una
commedie da lui composte, e da altre sue Opere poetiche, B pecige p
fentava in commedia ottimamente tutte le parti, ma in specie quella del se
zanni, ( cioè servo sciocco Lombardo ) che usa armare con un coltello di Tegao
simile a quello,col quale si batte, e si scotola il lino per purgarlo dalla lisca,
perciò chiamafi Scotola; però il Poeta lo fa azzuttare col Masselli, e sc
con quel coltello la zazzera. Dice coltello da Pedrotino, perché con tal
ceva chiamare in commedia detto Sufini nella parte di servo feiocco. Questo mo-
ri giovane poco dopo l'Autore; e con esso si può dire, che in Firenze morifles 4
la moderna arte comica, o almeno la franchezaa, e leggiadria nel maneggiarlag =
SQVOTOLARE. Vuol dire battere il lino. Ma qui intende squotere i capelli
per facilitare a Cloto, una delle tre Parche, il farne la conocchia, aleism
INVELENIT/, Ancrudeliti, inviperiti, inaspriti, incancheriti, arrabbiati
son finonimi per intender' uno, che sopraffatto dalla collera operi of
te, e con ira, in maniera, che non sappia quasi distinguer ch'eififaccias, =
Similitudine presa dal serpente in collera; di cui Virgilio lib, 2, En, tcolentem
tras,\& coerula colla tumentem. wm abean
POT-ANO. Latino amputant,demetunt, obtrancant, tutte similitudini trate
dal' agricoltura. Potare si dice de' traici delle viti, € de' rami degli alberi; ma il
Poeta si ferne di questo verbo per corrisponder' alia similicudine, havendo dewto
quasi viltan ch! e' tronchi, eds rampolli taglin ds Marzo, ec,. sd
SCONGVASSATI, Stanchi, € rovinati walla fatica del combattere.
FROLL], Qui vale per stanchi, ed indebolits, t¢ ben per altro Frode vuol di-
re stantio. Vedi sopra C. 3. st. 55. alla voce Leazo, iahesh
TREMAR le gambe sotto. Vuvi dir haver paura. Virg. Eo. ry.
ae folvuntur frigore membra, Se ben si puo anche intendere, che le §
mente tremafiero per la debolezza, e thancneaza.

FINE DELL' VNDECIMO CANTARE. +

Ze

Berbnanheansr ~ekk dé



BER TEER BEE

HifwMAMALAM

Fea dattacbute

A R GOMENTO,
e A, Montelupo. da Paride 1). nome,
Poi gapigar la Maga,e Biancon vede,
Rimef[a sn. Trano è. Colidera 3 3 e1come.
~ Aarito, al general dd ln fuafede.
a Baldon, che la fortuna ha per.le chieme
Con Calagrille aVgnan rivolgeil picde,
E al suo bel. Regno con Amor va, Psiche
A corre il frutto, delle sue fariche.

ae

speyrepeage

sae PP Regen

STANZA HL
Che sono fratt com! io dissi sopra,

were STANZA I.
Swanco già di vangar tutta mattina

“Abconcadino al fin a va-a rifelnere,
- Te forniar Vopresed-in chiamar la T ina

* Cokmerize guarto,eil petal dell'afoioluere;

( Nella Maga affidatifi) asperranda
Da' Diavoit im lor pro veder quale'eprs;

ea chi-vive a speranta muor acids;

eee tn Caffelle ancor non firifina Perch in Dite son tutti sottosopra,
Phu quei-marei di squotersi la poluere; 'Per non saper dove, come, ne quando
Onde: Badldon quei popoli-di/per de Laftiasse il Cornocapolfo,c! ale (chiere
Tal che a' joldati Malmantileé al verde, Esser tromba dovea nelle carricre.
STANZA Il. TANZAILV

E vase Sta, perché porevan dianzi,

vedean col peggivandar sicuro,

~\ Cederil campo, @ non tirare innanzi

ra Star avoler cozzar col mura:

< E così va, che questi son gli avanei,

Che fafempre colsie'ha is capo duro,

» Che dentro.a feifi reputa un' Oracolo,

Ne crede al Santo,se non fa miracolo,

Di modo, che Plurone omai scornato,

Poiche quel corno pitenon si ricrova,
Pel Proconfolo dice haver pefearo +
Pero connien pensare a invenzion nuova;
Ha innanss ch' ei-risolua col Senato,
Eche'l:foccorso 4 Atalmantil si muova,
Ch'egli habbia.aeffer proprio pot savvisa
Di Meffinail soccorso, v quel di-Pisa,

wey introduce i Poeta in questo Duodecimo Cantare con la rifleilione, che i (ol-
7

vv dati

|



22 IRKTVUA MAL MAN TILE > 1009

dati-di Bertinella non, haurebbono ricevyto,così gran danno\ |
sono accordati,, e non fufione faut in tanta 'tingaiones la.;
in loro per la speranza, che havevano negl'incanti di,Mar
havevano havuto effetto alcuno, 1 Diavoli non feppe:
dove fufie ii Corno d' Aftvlfo., non si ricordando, che. an
quando Affolfo andò per il senno d'Orlando, comedice.|'/
| KANG AKE...Lavorar la. cerca conia vanga... Bipalio

FERALAR l'opre,.Cioè far defiltere dal lavorare eer an
raion Depera. fra. i\contadini.s' intendedlJayoro;, che fa.un'
no, e s' intende.ancora lo Relio huomo.s.che ya.alavorare a
io.ho; chamato due, opere, per iacender due huomint; In questo lavoro ci
dicci opere, per intender dicci giorai di iavoro, ec,

p44 Ting. La Caterina, intende ladonna del Contadino

MEZZO, quarto... Così chiamano i Contadini un gran valo —
foggia. da boccale 5.del.quale si leswo, fespartag da bere ai Javor.
po, e gli danno questo cee perché'e forse di ce een
staio. x ae
PER SL afeigluere 1 cont ania ebaainesior il desinare asciolvere, Seren csnidal
foluere il digiung, dali. sdigunarsi, ¢d.il desinare, lo chismano wien
terzo mangiare Aicono./a cen

eA non si rifina., 'Nanay resta,,
esprima una.op e feng'
Ciog perquoterii,, bastonarsi.. Vedi rae C7. tt. 63. t by

ESSER? al verde., Eji¢r' ajla.fine. 'Tratto dalle candele. di, se ce seieprion dig
son unte di verde nel piede.. Viano nel Magistrato del Sale di.
le tafe dell'Osterie.,,¢ darle.al più esserente., e agl.tempo, che aMtneenapi thd
colissima candela di cera tinta da piede di color verde ognuno puo.otferine, es ida
consumata quella noo può. più veyung offerire sopr' a quell'osteria, ma s'intende P
restata a colui), che ha cfiertoyi maggior prezzo, ovvero non arrivando.lotier. Cm
ta.aldovere, ' Osteria ai suoyo si dubasta un' altro giorno con nuova candelerta', deat
EB digui habbiamo il dewato hs ha che dir,dicada candela e al.verde, che significa ted

on, si fa fine. Ma — che p00 iar

BINED a ET SERRE SSESE

sbrighiamoci, che il cmpo fugge.. E questo eficr' al verde e pafiato in the
per tutte le cose, come.cficr' al verde ot danari, vuol dire esser' alia ka
pari,...Va mpderan Poeta teleth scritto nell'Osteria di Radicofaa Pre
trata » qenm ° Re he age

| Cohanr, p Spebater ridotto alverde.: ee un ie ' hy

. » Gineca, uper ricattarsi, e Sempre perde

COzzAR col muro», Tentac l'impossibile,. Contrattar con chiha pb: forea di re
poi, Clavam, ¢. manu Herculis extorquere. Diceli anche: saree a co4hi co! mre othe
ciuoli. Nell' Ecolcfiafico cap, 3. Ditiori re ne focins fuerss 3 Qu Ep
cabys ad ollam ? Quandoenim se colliferint, confringerur. La Feces Fest
tole nel. fume galleggianti 2 una di rame, l'altra dicerra fa a. e oka
quale viene, Anaae ad. Efopo,¢ troyafi refa in versi Latini gala,
CAPL dur ag te ostinati. Dure pom ebb —o Ne
'ST tacas lends » Amico della sua opinionc,¢. che Li, Gli
uw

r



~~ oN eeenrvc

S88

SSBB ER CEES £5:

DVODECIMDJSIVLTIMO'CANTARE, 8
reat fate', e dit meglio w ogni altro. Huomo di quefd naritat dice de'
= Setpe thea Nim di lapete;e d-etere ungtan” buotao - Baxi.
; edi fe'micltefimo ',¢ pereid ine diviene contumace 3!

PUR Hess ONT I OM 9G; ty
VENOM orede al Sarito';\se*hon fa miracoli, Non crede'; che una cosa pli poma'ii-
teruenire', fe'rion la vede fegitire.Generario prava quarit signum videre. B per lo
più s' usa in' occasione' dammonire, o rinfacciare j'come e nel/pretente hiozo} ll
tale è lato pir volte: avvertito dition contindvare @ fat'quella tale operazione-,
perche'gliene' potrebbe seguir male'; ma' egli ostinato wor erede at Bantoy se nor se
miracoh, cioè non da retca agli aveertimenti; ma! vudl-seguitare?,finvhe la die
ee succeda » 4' Proverbifti Greci mettond un Proverbio,-che dice: Prime
a rem. PURI Toe h) LL Mes bas! Pe TNE Dey 99@ 1951b

CHI vive con (peranzia mior cacando. Detto. sporco » ¢d usato per lo più fa,

genterviles;\c vuol dire -'chiMfi palce di speranza-,'muore di faine'y"ed-in fulliinza

a €*vanitaril'fondarsi nelle speranze. “ai /pe'neratar, wi reer

mats ig 2. 0g

SON tutti sottosopra, Sono in grandissima confusione. '
sI DOVEA fer tromba alle carriere.' Dovea' fare scappat tut? peome facev't il
Corno 4' Aftolfo: e'come fa' scappare dalle motfe i cavaili'barbati'y che edrreno
al palio quella tromba, che suona il banditore, per dare if feghd della [otpperied
SCORN ATO' »» Vuol dif beffato; ma qui 410 (cherzu di /eorWard\, che Vadeiie
senza corna, come era rimafo Piutone fenza'¢orno, cine senza it Corti dA Nbk
fo. Var animale, che abbia perdute; o tronche le cortia, vient ad avere per
del decoro; onde scornato diciamo per beffato. Acheloo 'fiume; 'efleatlogli d2*Er-
colelevato un corno', rimase scornato; e svergonato. Onde Ovidio 9; Met Muh
tas Achelons agrefies, Et! Laverne cornu; meuijs capur abdidlit sndis,  Hivtc tanith abla
ti dommie idibure decoris, Gc} 229 10109 Lb 2h 4 HW) 199 > piri
> SPBSCAR per il Proconfalé. Ho Neff, che durar fatica per impoverite; sean,
CG operam perdere. \\'Proconfolo'é ia Firenze il Magistrato', che soprdutenie 21
dottori,¢ Notai, ed ha la'saa refideriza otto le logge,dove sono giù altri Viizzi,
acll'ultima abitazione versoril fiume d' Arno; il qual fiume per quello spazio',
che e fra l'un ponte, e/' altro} ', 6 almeho efa già fortopotto alla 'giurifdizione
del medesimo Magistrato del Proconfolo'; come pesca'ad elo rifetuata ne' vr ti
poteva pescare senza licenza del detto Magiftracs 3! non vi era-già ditra pena aIfi
contraffacienti, se non la perdita delle reti, e del pesce, che hanno preso, fead
acchiappati in sul fatto; E Pett utes! aie

STAN ZAV E>) > STN ZA VEU.
ae Paride ritorno, OO Ada quegli, e ° obligate si non Witende,
onGhte nellvoffe alla quarta sboccatura;\ Wor vuol phr quanto un capo di spilletto;
« Eperché dal pacscegls ha in quelgiornd E subito ogni cosa indietro vende,
0 Foleo ogni nota', liberando ib Tura; Ringrariande cinscsin det buon' afetto,
8| La\gente quini corre @ intornd E'dwe', che da lor nulla precende;

ed rallegrarfidelia fuabravkrh >) 0! 2) EB Te aiteddisfarle bhnho concerto,
Ne lo ringraxiasewrallegrarsi intenta, °°! Perital niemoria gli fara più griro
Chi gli da chin lt dona z'chi gli-avvina, o: Che it tuogo'Aonteliipa sia chizmaro;

ang, ~:

Vvv 2 STAN-



524

Si si, ch' eli è dover da tutte quanté
Gli fu risposte, ed in un tempo stefo
Li editto pel Caspello fu pe i canti
Per notizia de' Popols fu messory>
Che dinuleato pos di te avanti yo. «\
Fu osservato si, che finoadefo-..- \
Lucho nome confernan quelle mura,
E'l manterranno,fin che'l mondo.duna, ¢

STANZA AX ) f \“ A

E che fuor del Caspella il,popoh proves: «3000 (- ) SERB;

Che ognor ne scappa qualche sfucinata,

Per to più gemse yeh? a peta 31 i loxofexes fer o

Cotantaé rifinita,¢ maltrattata, — Qui pinto innansi stwile sentiva) >

Tornaril Poeca a discorrer di Paride, il quale havendo ridosto il Tura nely

fino staco, haveva liberato quei popoli, i quali per riconofeimento del

ordinarono, che que! luogo si chiamafie daallora avanti Montelupo

torna al campo, e trova ogni cosa murata..
LA quarta sboccarura, Cie ha sbaccato, 'cioe.: manomeflo

vuol dire: ha bevuto tre fiaschi di vino, ec cominciato ibquarto2\Iperbole, che

significa: ha bevuto molto vino, sborcare propriamrnte Qgettare via

vino, che e nel collo del fialco., per purgarlo affaordalll'obia.yec, LiAQpesas!
CHI gli da, chi gli dona, e chi giù avventa iB' detto giecofo nfato per burlare

uno, che figlorij d' essere sspesso:regalato; es) intende; chido ee 1

avventa, cice fafiate, ec. € lo scherzo dell'equivoco't:nelwerbomare, e '
NON enol, quant' un puntale d' agherto, Racufarurto.. Vedisfopra Capt, 10;
RINGR AZIO' del buono affetto. Termine di cirimonia julaci si

ringrazia uno'del regalo, e nello fefid tempo si Ficnfa di rice -dicia- f

mo;non voglio,o non stimo il regalo, servendo, per obligarmiy Pinclinazio- '

ne, che io veggio in voi di farmelo; e questa teftimonianza  chehio dal:voltro
affetto verso di me. ist
eH/ONTE Lupo. Finge, che Montelupo Castelio wicino a
anch' eg) quasi distrutto bavefle nome da quota azionedi:
biamo per tradizione vuigata, che eglilfusseanticamente
stare il Castelio di Capraia luogo allora forte' fituato rincontro
cendo coloro, che.' edificarono: Perdifiragger questa Capra 'Non sci guole altro, che
un Lupo, e perciò lo nominarono Castello Lupo, che. per esser Mopraiun monte si
detto Monte Lupo. Coca bg ED
GLI venne il grille. Gli venne voglia: E' 1o-stesso, che tocedsill

sopra G. 9. st. 56. con Sp bei
STRVGGIMENTO. Un continuo ardente pensiero 50% I

iftruggimento vuol guarire, cioè vuol' adempire questo.fne-desiderio

all armata. Ii Burchiello, fe'ben mi ricorda; Se/piri:d.amo rd

\item[SPARITO ciò che v'era] Non v'era più persona alcuna, ip
Baldone era diloggiato, ed entrato'in Malmanule. 495



“DVODECIMO;ETVLTIMOCANTARE. 525
SEVCINAT A, mola neg Vana gran quantità, Fuciaayyicn dal

che wuol-dic
ani

(O facina @ i

» o luogo dove Gi ri

no mercangic; e
be capire una fucina prela. pec
operasoni

le ic
' et re Bocce, Nov. 2. ee ane eena di diaboliche

igion dire; 4

si erika, 'vuol anche dire il Barccaten de' fabbri o delle fonderie,.ec,
| RIELNIT. « Malconcia,@aaca, finica, sopunatai ¢s.intcnde di sanità,e roba,

or STANZA a
pala, e ne ri/contra un branco,
Preeti lemgean,
'bi dietro fr ascicar fivedeun fianco
; | gli gi

STANZA XL

Chi ha scatole, chi sacchi,¢ chi sieehiee

Di givie di mifoee, dibiancheriay \
Va" altro ha una ranaca di scrittwe 4,
Ch' agli ha @un Pinto della dtercariby

agli senza.adar albaco, £ piange,ch).ei le vede mal sicure,
\& nee Sete egli ha riscoffo; Pero che *l vento gliene porta via;
 Ciascuno hail/uofardel) diguelletre/ibe, Vat altro dopo bauer mille imbarazzi,
a a og si ha potuco beans ee Port' addosso nna gerla di ragazzi,
STANZA XIilL
reimbacuccato Arete feretto Le dine agliocchihantutteilfazzoletto,

a ria > 'eJpelse, Ipesso si szattiene 9

E sgombrane 2 Py rocche,e pergamene

tra ys' elle, le stanno.

Chi'lf il ye chi >

Chi porta nngatto,e La caninainbraccio,

Sono

lex
te vede una gran quantità di gente, che fugge da Maimantile, per (cam-
parila vita, e porta seco, le cose più grate; nel che il Poeta s' accomoda a' gen) di
quelle tali perionc, che fuggono, ed a quello che,per lo più,luol seguire ia simili
Seapets;

at ENC INGO. Se ben significa quantità di polli, o di pecore, o simili, tuttavia

ne serviamo per esprimere ancora quantità d' huomini, Lat. bomnum manus.
Vedi aC, 6, tan. 35;

T ASCIC A dittro on Fucwene Va zoppo., per esser Mroppiato da.un fianco.
HA »ifeoffo Senza aspettare al abate, Glioperarj ordinariamente ri(quotono le
ro mercedi, e prezzidelliloro lavori il giorno del fabato 5. ed il Poeta scherza

col. verbo rifquotere, che vuol dire ricever denari e ce.ne serviamo ancora per
intendere Ricever butle.

GVIDALESCO. Malcalcia; Scorticatura. Vedi sopra C, 10, stan. 11.

TRESCHE. Qui intende bagattelle, bazzecole, arnesi di poco,prezzo; Lar,
trica, Vedi sopra C, 10. stan, 12,

SCATOLA, Lat, cap(ule. Sono caflette con fondo,e coperchio, fatte con,
sottilissime asticelle in varie figure, secondo che richiede la roba, che dentro ay
efie si ripone.

SLANCHER/E, S' intende ogni sorta di panno.lino., come tovaglic, lenzuo-
la, camice, ec.

PLATO, Lite civile » dal Lat, placitum, VedifopraC. 7, stan.27.

MERE ANZI A. Altrimenti Afercatanzia. Così chiamiamo, in Firenze quel
Foro, o Magittrato, al quale si ricorre, per far l'efecuzioni civili,¢ ai we son

fone-



ee

526
fortoposti tutti li Mercanti, ec. il quale ha particolari fat
'MB ARAZZI, Spagnuolo, Embarazes » Roba', th
6 feommodo; ed' aBBHaED il verbo imbarakzare, cht
nefi(, te tina Qanza ¥ ec * » ASSAM vaya
GERLA Da gero Latino','che vuol dire
Ma Voce il nostro Chimentelli nel' Azsr nie?
di bastoni a guisa di gabbia da uceelli
larga je fondato hella' parte più tretca, det
per portare il pane eotto da un luogo all'altro'y adatrandoselo
alle reni; € et eeitind nim
firo Autore nella-eecera alla Serenissima Arciduchessa Claudia,
nelProemio'j dove Wie' Che i Prascica diecro'lina gerbi- tdi farfaallond COR
gran quantità dipropositi) Può bene anche essere Che il' Poeta intend'
mente ger/a, e che voglia dire, ché havessero due }o tre bambini'in u
talé gerle §'per:portari'pilr comodamente's coiné veggiamo tutto ”ll B
parire povere donne della Garfagnatiay e d* altrove, che portino due, 0
gaaai addosso imgerie, 6 altri trabicvolifimill /-)) >) 9
tM BACVEC ATO} Copertd5¢ Ito 'bene, ¢'s* ihtende pi
pert ibcaporn Vedi sopra C)1't, Man22: se bendal Cy 6, stan. "64.
ne serve per intendere Mettersi l'abito addosso, tuttavia e da norare,'
intende il lucco, che e l' abito Curiale "if quale'aiicledmente haveva il
per coprir la testa, e però mietrerfivtal"abito si diceva Pmbackecarff; Si
inbavagliare. Giovanbatista Bufia? asBehedetto 'V archi lettera nona, | ¢
da AMona'coiet, “ed imbavagliatala la conddffero alle Palle se 3
OLE risconera » Cive riconta la moneta 7 per vedere} i '
traruino, vuol dire imbatcersi in-who Pma risconttare libri;' ferirrtite 5
danariy contipecyvuolidit Rivederé je rorha®l Ay
Z HO29! ib. 920074 2» MH299)  tegal cuba
» con te assegnò Wi piabtd', 6'di dolore'
il fazzokeio agli ocehi', Veli topra G..9! stan.48! ahaa
SCOMBRANO « Portan via Seombrare [ quati dal Latind excumiiliré, ton?
trario d? Ingombrare, che ¢'come se fotle dal Lit! ficidmindare] deco
t6, ci serve per intendere. portar le? mafferivie"die ania casa a tn” altra
mo in-vece'del verbo diloegiare', sieggiare 'Biaiken archi
BeASPL rocthe je pergamene' 7 re fruthenti atcenenti
habbiamodenowoped nel o. ye Rad.'>\\E -pertamend intertddid tes a
Carta con'la quale'fermano ia 'condechia' ia (u'l roced "per fadifitarell Blare
la'dicond paielibead pershe per tovpit: ol esser facta di carta pecora,€ he ti

et anche, carta perdaminay i 9
199 9.99 roel vp SAB NON Z AOR 249 Pr cxortert
Entra: Paride al fimdentro alla portarys 9 * Ma quel che mardni¢ha p tt dppor
OOue\g lipar a! entrasdentroun matelles \ -<Si% st' veder tn' pi m Capen
Chr ad ogni palje troua gente morta. Di scope, e di fascine f
1Oiper lo-mer, che (Ps per far fardeliay > ani? i iy:
Oud i IMGselis e edusiiwi wh sil? - otary ate wage "A Ve Loi 4

aw


——— a

BOE Gg CUaio 'Set etree eee

DVODECIMO;EDVETIMOCANTARE: = 527

oo ye STAN ZAXWE oo cen! Singeatte:
arriuato in pragza, Egliftaben, pere una simil raza,
Perchi(domanda) ésigran fuoceaccefor.. C? ha fatto se @ ogni lana un peso,
 Egh érisposto:egheper Martinarra, E' si vorrebbe ( Dia me lo perdoni )
bid v'e dentro,escrine:latopre/o;. Gaftigar a milura di carboni. 7
'aride entra oe! Castello, e vede molta gente morta, o malameate ferita., e+
jartinazza mefia nel fuoco per gaftigo dellc sue stregonerie.
 MACELLO, Beccheria. Luogo dove s'ammazzano le bestie per vitto dell?
mo: E per macedo intendiamo Strage, o difipamento di che che, sia. Qui
Iptende, che a-Paride par d' entrare in una bottega di un macellaro in riguardo
=| molto sangue, che vede (parfo per il Castello. Così quel che dice Dante, che
V go Ciapecta tofle figliuolo d' un beccaio di Parigi., Sccfano Pafquier va interpe-
trando, che abbia voluco dire di un bravo soldato, quale era suo Padre, che per
la @trage che faceva, era riputato, come un maceliaro. '
CHE fea per far fardelio, Lat, vafa collig't, che è vicino a morte;, sta-per an-
darlene da questo mondo. Vedi sopra C, 4, stan, 21.
CAPANNELLY di feope. Piccola capagna, mucchio, monte di (cope., ec, il
eee quando era per l'cffetto, che era fatto, questo, era dat Lacini:detto eon

Inc reca Pyra dal Greco Pyr, vuol dir fucco,e noi pure lo diciamo Pira, Dang
126.: ii
i Chi è in quel fuoco, che vien.si dinife,
eed z Ds sopra, che par furger dalla pira,
iets: Ove Exeocle cul fratel fu mifo. »
SCRIVE; lato preso, Antendi; ha cieito per sc quel luogo - /edem occupauit;ma
Per maggior chiarczza di questo detto, e da sapere, che in Firenze si fanno ogni
@nco tra gli altri quattro mercati, uno per Quartiere, che il primo nel Quar-
Gere, e in fu la piazza di S, Maria Novella il primo giorno di Quarefima, ach
quale Gi vendono Icgumi, feccumi, e frutte. Li secondo nel giorno di SS, Simone
nel Quartiere,, e in fu la piazza di S. Croce, Li terzo la.vigila dituitii Santenel
Quartiere e in fu la piazza di S, Giovanni, acl quale si vendevano oche.; ma
questo € andaco in defuciudine; perché e perduta l''ulanza di regalar l'oca lay
mattina di cucti i Santi. LI quarto nel giorno di San Martino nel Quartiere,.e»
in fu la piazza di S. Spirito. In questo, come nel secondo Gi veadono abiti, pan
Hine, ed ogni sorta d' arncfi, e maflerizic;.¢ come-che acile dette fire concorro
Ho molti mercanti di panni, ed altri artefici d' ogai sorta.,. così alle. volremanca
doro il luogo, dove polarsi, per farvi.ia quel giorno la lox boxtega; onde. piglia-
Ho il luogo qualche giorno avant, e segnano jo spazio dei juogo,, che piguano
con getio 50 altra unta, e vi (crivono in leere cubicali LATO PRESO,, e que-
flo servc per impedire, che altri entrino in quel luogo.: Edi. qui dicendosi; I
tale ha (critto /ato preso in quella cala, ec, intendiamo: quella cala, ec. e per iui,
une gli può esser tolta. Così. dice, che Martinazza scriveva dace pre/o in quel mon=
te di scope, per iagendere,.chc havea tatto in modo, che.qucl fuoco.non le po-
teva esser tolto.... 4g Neds e402 an ic
|fatto a' ogni Jana un peso, Ha commefio ogni sorta di de'ito-senza riguar-
do alcuno. “Si dice anche far d' ogms erba fa/cio, Che in (uitaaza s' Intende un' nue.
mo

|



38 “1 ALMA NIP 1

mo scellerato, di coscienza larga fhe Hon' tetne
giuttizia; che'in Latino' pure si ditebbe, ex guoliber,
mea quella; Aivdum fie-pratum, quod non persranfeie lit

b10 me loperdent, Detto da Ipocriti, perch e in' un' certo
cenza a Dio di fare un peccato impune.1 Latini havevano'una i
che parte simili + Si Dijs«placet', " eee
. GASTIG ARE a misura di-carboni. Dar maggior gattigo di
il detingtente. 11 carbone e fra le più vili/ mercanzie; chef
misura, € per questo non ff guards così: per la minuta in darne
bra, e pero habbiamo questo dettato, che significa: dar' più |
nel Morgante. ef misura di crufea, e dt carboni, + o RE Oa

STANZA XV.; STANZA HVE)!

la quespo., e ognum parla della Strega, i i a
Si sente dire; A voi; largo, Signori,
E un bnomaccion più lungo a' unalega,

Dal Palazzo si vede conaur- fuori, Per esser vogavanti di galere

Poi sopra il Carro, ove Birrenoil leva, Chetal fa d Amoktante
E cinto ( come già gl Lmperadort ) Eperch'egli@un ?
Dialorowmvece, a' uncarton le chioma, Sentengtaro I hanea' nfarey
Va trionfante al Remo, non a Roma, Che Atalmantil non ha legniyne Mare
STAN ZA XWPLY

Perciò, mentre che tutto ignudo nato, Lat confulte it decreto ha renocate, ~
Senonch' egli ha due frasche per brachetra, Sicche di luimndn' ordine 8 >
Sh) bel trofeo si muone, ed è tirato Ed ¢\ Stato spedito un Cancel re
Da quattro canallaccs dacarretta, €on più famigli « farlo-ratzenere o>

. I. Gigante Biancone legato ignudo sopra un carro e condorto fuori di: Palazzo
per esser menazo in Galera; ma quella esecuzione resta sospela, perché Malman+
tile non haveva', ne Mare, ne galere-, Haba 3 sun-

LARGO Signori', Date luogo; Fate ala. I Latini far far largo dicevano Sum
monere, Orazio. Neque confularis Summoner liter. Vedi sopra C, 11. fam

PIV" lungo a una lega, Iperbole usatifiima per esprimere Lunghitiimoy Di
atiche pis /ungo a una picea, 6 LO alae

BIKRENO, Intende birro, e'fi dice'cos) per la. similitadine
con Kirreno, che fu amante d' Olimpia,secondo |" Arioito', dal! snes.
capertamente birro diciamo: lo /poso a' Olimpia, th ial ene

CINTA di cartone (a chioma, A coloro,, che per delitti-son la
frufta, asino, o berlina, fogliono per maggior vilipendio meceereinteta un bet
rettone di foglio', che per-esser a foggia ai mitra-epiicopale lovehiamano milena,
quali' sono 'quelle, colle quali farono:dipinti nelle itira del PalagiodehPorelta
oggi derro del Bargello', 1 seguaci del caceato Duca @ Areael, le
per l'antichita appena si veggono'. VeditopraG, 6. Man, 56, €eque
per cartone, che per altro vuol dire quella carta grotia, che (erie
incartar pauat, cc, r

HAVO MO abandiera, Haomo a caso, inconsiderato » volubil
riofo nelle sue operazioni. +k Saga, Url al

SS ae
DVODECIMO,ETVLTIMOCANTARE, 525
IGNVDO nate. Affaito'igdudo. Vedi sopra o, 2. stan. 64. IL Coloffo ad*noi
; e"mto ignudo; faluo.che ha due frasche per braghertas cioè duc
fogliedi vite-fatte di ferro, o  d' altro metailo dorato, che gli cuoprono. le parti

\& e SESLOU Re Ub =e a
« CAVALLAGCCE da carretta, Coloro., che in Firenze tengono carrette a vet
ra? per-portar mercanzie yed arnesi da un luogo.a un' altro hanno sempre caval-
lacci vecchi, rifiniti, ¢-ai poco valore, e pero dicendosi cavalio da carretta ys"
intende cavailaccio di tal sorta. Qui il Poeta finge, che il Gigance Biancone fal
smelo sopra.avun carro tirato da quattro di questi cavallacci » perché 1l Colosso
detto Biancone sta sopra ad un carro, che si. figura tirato da quattro, Cavalli
anarini,. > ' 2 a
LA vinocate il Processo.. Intendi ha: mutata la sentenza, o decreto della galera
havendo considerato, che non se li poteva dare esecuzione, perché Malmantile
non ha gaiere,ne dominio di mare.; '

» » STANZA XVIIL STANZA XIX,
~Hragazzi infrattanto, che son triffi, E perch! ei.nonha in dosso alcuna vefay
o Aveder cio che fusse, essendo corsi, Lo segnan colpo colpo in modo,tate

Epaich' egli è un prigion,/i fona avvisti, Ch' mmnanzi ch' e finiscan quella feta,
let Bich eglieben legara, e non puo sciarsi, Ne lo fuifaron, e conciaron male;
ly) Unitamente in un balen provuifi E al miteron, che atorre haueainsefa,
Di bucce, di meluzze, rape, etorsi, ( Bench giammaispuntate auefel' aig,)
* Cominciarono a far achi pri tira, Conquei suot merli, che non ban lepeane,
Ed anche non tiranan fuor di mira. Pigliar volo alt aria al fin conuenne,
Narra gli strapazzi, ed infulti, che yengon fatti al Biancone, e con questo
smostra il coflume de i ragazzi Fiorentini, i quali quando un malfattore e condot-
-to per la Città in full' afiao, o metio alia berlina, lo trattano nella forma, che
dice del Biancone, tirandogli torli, cioè gambi di cavoli, bucce di poponi, e si-
“mili immendizie. £ nota che havendo egli.detto, che Biancone haveva Jamice-
»ra, perché il Coloffo detto Biancone ada ha veramente la mitera » fa che i sa-
~gazzi la levino co i faifi di capo-al Gigante Biancone». i
-4N-nn baleno. Subito; In.un batter d' occhio, detto sopra C, 11, stan. 42. Di-
ciamo anche: in men che noo,balena; essendo il baleno, o il Jampo 4. siccome yil
vento,¢'l fulmine cosa velocifiima, Onde noi d' uno yche corra e sparisca.yia
fuggendo, diciamo = £' pare il vento, Ha fatto comenu baleno. Corre, come unit
Yacsta, Pare che"! vento se loporti, Virg. En. |. 5- J,
Primus abit, longeque ante omnia corpora Nifus ont
Emicat,\& ventis, \& fulminis ocyor alis,
Dove quell' Emicat vaic: Scappa fuora,¢ innanzi agli altri, come um lampo, Si
Swede correr la piazza in un baleno,
«»LVON tiran fuer di mira. Colpivano nel luogo, dove segnavano.. Vedi sopra
~C. 1. stan..37. dove troverai co/po colpo, che significa ogai coipo, che ¢' tirana.
Che diciamo anche Zorto bette, Mira e lo stelio che Scopus, voce Greca usata.da'
« Latini,; facta da Scopein, mirare,;
le PkIa Ache finife ques foffa. Primachee' finisse quell? operazione; Si dice
anche + quel/a musica; quel baccano; aes Ȣsimili, Vedi sopra C. ae fh 53+
c xx.; +, tbe

rad

=eSh 2

See CUR SS

=e

~  a ae

MBAS, %
eng

s

530 MALMANTILE

MITERONE a torre. Quel foglio, che per derisione si mette i
fattori detto mitera, come habbiamo accennato poco |
doil capo al delinquente, apparisce a i circostanti una roronda t
la parte di sopra di detto foglio molte volte l'intagliano a guisa d
farsi sopr' alle muraglie delle Città; e così havevano fatto a quelle
e'perd il Poeta scherza con la voce merlo, che è un' uccello note
glia dicendo, che se bene i merli, che haveva in capo Biancone n
mar mefle le penve, e non havevano mai spuatate / ali; tuttavia
vouare,ed intende, che quel Afirerone fu fatto volare dalle bucciate,
che gli tirarono quei ragazai, con le quali glielo levarono di testa,
STANZA AX. STANZA XXIL oy
Paolin Cieco, il qual non ha fuvi pari Ed ci lo donaa Bieco,e a Pasian
Nel fare in piazza giuocolar' i cani, Col carro,e tutcel' altre ap,
E vendea l' operetic, ed è lunari,
E proprio ha genioa spar coi Ciarlatani,
Pens[ato ch' ex farebbe eran denari,
Se quel bestion venisse alle sue mani,
Pere' baurebbe,a moffrarsi,quel Gigante
Pix caica, che non hebbe l Eiefante.
STANZA XX1
Così presa fra se risoluzione,; Subito qui Paolino scende,
| Vain Corte a Bieco,¢ lo conduce fuora; Per trouar qualche st buon.
Gili dice il suo pensiero, € lo dispone Havendolo ferrato fra due ee
etchieder il Gigante a Celidora; Accio non sia veduti da persona, Vey

E Bieco andato a ritronar Baldone Bieco a tenerlo con due altri atendey
Tanto l'infipilla, e allora allora E se lo vede muouer 510 ha;
Ei corre alla cugina, e gliene chiede; Ma egliha fortuna, perch écni grande,
Ed ella volentier elielo concede, Che non gli arrina mancod

fande,
Paolino Cieco ottiene da Celidora in dono il Gigante insieme co! carro, sul quale
era, e sul quale lo condufle a Firenze, e si fermo ia fu la Piazza della Signoria,
havendo chiufo dewto Gigante fra due tende; affinché non fatie venduto, e men-
tre.così stando, Paolino cerca d' una stanza, per metteruelo, e farlo poi vedere
a coloro, che havessero pagato un tanto per uno, come si faceva dell' Biefaate,
fuccetle quel, che sentiremo appretio, * ie
“ ELEF ANTE, ¥u condotto in Firenze più anni ono un' Elefant
il popolo per la curiosica correva in gran numero a vederio sotto ie logge
Signoria ( hoggi detta de' Lanzi, perché quivi € il quartiere de' Trabanti, o fan-
ti della guardia del Serenils, Gran Duca da noi chiamati Lanai') dove fava rin-
chiufo in un tavolato, e si pagavano alcune crazie per entrarvia vederlos ¢
fio animale fingulare ne i noltri paefi, mori in Firenze per lo gra freddy ela
sua pelle ripiena, e lo scheletro nettato, e messo insieme si confervano nella Gal-
leria del Sereniss. Gran Duca. ucoensini aie
INZIPILLO'. Inttigo, stimold, pregd instantemente, e forse voce corrottas —
Sill:

da hbillare, Latino foilare, infufurrare, trovandolt nella' flor

traccaco feume: Di-ninwa miseredenca era stato antore, e nulla male:
date, ta



DVODECIMO,ED VLTIMOCANTARE, = 31

TRAINO. Diciamo quella quantità di roba, che possono strascinare duc buoi,
che i contadini dicono trainare, ed il veicolo chiamano traino, o treggia, La-
tino traba, o trahea, a trahendo, Virg. Georg. 1, Tribulaque, trabeaque, o ini-
que pondere rafiri. Si dice anche sraine una mafura di travi, che contiene quattro
Breccia quadre. Qui intende quel carro, sopra il quale era il Biancone con tutti
phate arnesi, e pigia la voce sraino nel significato della voce rreno usata per

rsi intendere carro, e bagaglio dell' artiglierie; !a qual voce s'accorda 'colla.
Franzese Train. Noi percio la diciamo ora Treno,rappresentando quella prooun-
zia; ora 77a:mo coll' accento fulia prima, non facendo conto della pronunzia
Oltramontana, ma della (crittura. Qui il Poeta dice Traine coll' accento fulla.
penultima; per accomodarsi alla neccitsta della rima. Franco Sacchetti nelle Ri-
me fimiimente pose questa voce nella fine d' un verso,

Per tirar colti piedi un gran traino,

LA Piazza dela Synoria, La Piazza, che hoggi si dice Piazza del Gran Du-
a,¢ si diceva de' Signori, o della Signoria, perché è d' avanti al Palazzo de'
Priori, e Gonfalonicri di Firenze, che si dicevano la Signoria, nella qual Piazza
@ la fuddewta loggia, detta de' Lanzi

CHE non gli arrsva manco alle matande, Cioè non gli arriva ai bellico, perché
mutande chiamiamo propriamente certe piccole brache, le quali si potiany,quan-
do si va a bagnarsi in Arno, per coprire le parti vergognofe, le quali mutagde»
per ordinario cuoprono dai bellico fino al principio della colcia.

STANZA XAlV, STANZA XXV.
Piange Siancone, e chiede altrui mercede, Quei tre yc ognor came cuciti a i fianchi,

E mentre il Fato,e la Fortuna accufa,

Euor delle tende si guardo gira,e vede
« Perseoy'ha in man la testa di Medusa,

E immoto resta li da capo a piede,

Ne più si duol yma tien la bocca chiusa,

Perché col Carro, e tutta la sua muta

De cavallaccs in marmo si tramuta,

Gi favan quivi,accioch'ei nofeappasse,
Privi di senso allora,e freddi,e bianchi
eAnch' eglino si fanna immobil sasso.
Ata perchs'l protungarmi non vi stachi,
Glie me',c' a Malmantile io mene palji,
Ove giù amici Paride ritrova,

E sente,¢' ogni cosa si rinnova,

Ii Gigante Biancone era così grande, che avanzava il capo sopr' alle tende;
nel girare, che egli fece la testa verlo la loggia de' Lanzi, vedde i teschio di Medusa
tenuta in mano da Perseo; per la qual vista rimase immobile, e divenne
sasso tanto lui, quanto il carro, i cavalli, e coloro, che gli erano d' attorno; E
così il Poeta da la sua fine, e si sbriga dal Gigante; di poi ritcorna a discorrer di
quel che si faceva a Malmantile.

PERSEO, ¢' ha in man la testa di Medusa, Questa è una statua di bronzo, la
ae € fiuata sotto un' arco di detta loggia de' Lanzi; opera di Benucnuto Cel-

i; e rappresenta Perseo con la testa di Medusa in mano, verso ia quale stacua,
guarda il Coloffo detto Biancone, percht e di marmo bianco. E nota la fayola
di Perseo figliuolo di Giove, edi Danac, il quale uccile Medusa figliuola di Forco
strupata da Nettunao nel Tempio di Pallade, la quale percio sdegnata convertì
i capelli di Medusa in serpi, e fece che la sua facia faceili diventare di sasso
coloro, che la guardaflero: Ma il detto Perseo havuti da Mercurio gli stivali, ¢
la scimitarra, mentre Medusa dormiva s le taglio ia celta, la quale pot ee

xX 2 mefle



sg IFATHAI OM ES +9 vi h
AS3HL ) più Ofisip MALM he thy; 4 ie on si
miefie nel proprio 'feudo, Di questa favola si servé il! Poeta 4 } 7
gante;dicendo, che per haver' eghi mirato questa tettadé-]
marmo, € così da graziosamente una favoloia origine a questo f
rappresenta Nettenno Dio del Mare', ed! è»posto nella: Piazza' del G
sopr'ad un carro tirato da-quattro cavalli marini nel mezzo a una
quale riceve I acqua', 'che scaturisce davaleuni niechi, o conchiglic
in mano da alcune statue di Tritoni-alte quanto le gamberdel d
or dette stawe stanno attorno:"E queste il Poeta finge', che sieno

mipagni, che dice fargli cucits a i fianchi, e che non gli arrinano a le
dé'; E così viene a conformarsi col gruppo, che si vede di queste ttarue
fo tutto di marmo,

CVCIT 1 ai fanchi, Stretti attorno, come se fussero euciti, Detto uk
per'esprimere uno, che mai si levi-d' attorno a un' altro;€ qui corna bene,|
Ché quelle statue sono così strette attorno aj Coloffo, che paiono cavate:
fo marmo, del quale e cavato il Colosso.

GLle me', Gli è meglioy. Vedi sopra C. 2. st, 10. <a?

PSTANZA XXVEy STANZA XXVHE
Poicht Baldone eAalmantile ha preso, Cos} cercando le grandexe i |

E tutte quelle povere brigate Soe @altrihor feo Ve, ?
Saluopera chi non si fusse arreso ) Onde tornata Celidora, il Lage
mii se ne son ite a gambe akace, De i popoli padrona, e dello Stato
Sitché'da queste havendo al fin coprefo Temendo ancor de'

.

'Pot Bertinella, ch ella l ha infilate; Nuovi Miniffri fa, nuove ve i
Perammazzarsi sfodera un pugrale, Se ben de i primi poco ha da temere
Ada quei,ch'é buono,non le vuol far male, Che tutes ban ripiegate le bandiert i
STANZA XXVIL. STANZA BALK
Clienon fo come gli esce fra le dita, E per eftinguer la memoria i
B/fulta in Strada, che le gabe ha destre, Di Bertinella in ogni gente ye-loco }
Ov" ella a ripigliarlo'é pos /pedita Si levan le sue armi, il suo ritratto,

Tagliato in croce si condanna al fusce'
aE perch'elt habbia a raccorciar, la gita, Un bando va di poi, \& averum patto
Le fa pigliar la via dalle finefire; Neffan ne parti pite punto ye poco
NEMa wa sh, ma poco poi le importa Sotto pena di fear in fu la fume
MT rovaricht amarza,se viginnge morta, Quattro mefi al palarzo del Com
Celidora tornata padrona di Malmantile fa buttar Bertinella
ordina nuovi Magiftrati, e comanda, che non si parli più di Berti

villime'pene. jo faite

Dix'chi dopo di lei fa le mineftre;

ELLAL ha infilate. \ofilar le pentole, vuol dire Esser rovinato
ver finito-, o perduto la roba, e la vita, ec, che di tutto s*in cok
mente. “tale ? ha inflate. Latino decoxit. +20 sett

LE gambe ha dere, Non, che quel pugnale havesse gan
dire } cheetlendo grave, gli fu facile andar' a baffo in strada';
perie 'finestre anche Bertinella da chi fa le minefre, cioè dachi
avichi comand; chee Celidora ritornata padrona di Malmanule.
gacge ae peccato, Ha la pena det suo fallire, e che ha m
whet; F



1 v¢ t E LAN 7
-.  DVODECIMO;EDVLTIMOCANTARE. | 533
' flaver voluto per strade indirette farsi Regina', usirpando queld' altri iio! is > /
be i. icsanlig vogliamosintendere uno, che piocenea oe taper fare Ogni-cosa
meglio degli altri diciamo; M.raleeit Lagi, Che il Lagi fu anticamente un Sen-
icato wv Firenze, che faceva tutti i negozzj della piazza': Si dice
rO per scherzo, e per una certa ironia, e derisione. ho “ogee
* HANNO ripiegato le bandiere, Cioèhanno finito; Son morte, Il Petfiani,
parlando di se medesimo in questo proposito disse + ty
core edi primo tramontano a quest® ascintte ae)
si Be Ditems pure sl requie,¢ il Miserere,:
Perch' so fo vela, e piego le bandiere; '
 E buona notte; a rinederci tutti,
LE fue' arm, Intendi'}*infegne della sua cafata, o stirpe. ue
7 ~ STAR in [u la fune quattro mefi. Now è posibile' star in fa la coda quattro
y hore, non che quattro mefi., ond' io penso, che con questa iperbole voglia iaten-
sia condennato alla morte, alludendo agi' impiccati, che in un certo modo
quando pendono dalle forche a vista del
popolo; st poslono dire stare in sulla corde,

be in fulla fune.
¢ STANZA XXX. “STANZA* XXXIIL
jee | Yr Orarore intanto de' più brani Spiegafi se desea 4 ttn tavolotto
" ACelidura Aaimamue inuia, Vol abito mavi di mezzalana,
a 'Che det Caffelo ad essa da le chiavi, Che infu fianchi appiccato ha per diforto
Evende omaggio con la diceria; Pn lindo rid aief alla Romana;

Ed ella in detti macffofi, e gravi
Pronta risp a tant' Ambasceria;
Inds le chiavi piglia ye nn' altro mazzo
= Wi quelle delle stanze det palazzo.
ae STANZA AXAIL,

E perché gli è un perro, ch' eli' ha voglia
Di riveder, come ad arnesi e pieno;
Del Mamoye d'altri addobbsfi di/poglia,
E comincia a girarlo dal terreno;
4Guardarobi aspetta, ead ogm foglia,

Poi viene un verde nuouo camiciotto
Con bianche imbaftiture alla balana;
E poi due trincterate camicinole,
Che fanno piatza d' arme alle tignuole,
STANZA XXXIV,

Vua Rimarra pur difaianera, ~~
Per dove si fa a' sassi arcisquisita,
Perché gli aliorti, e it banero a spalliera
Pavan la testa, e in giu meza la vita,
Portandola alle

'i; te,o0anna fitra,
\C* ad aprer gli usci patono it baleno; Torre,e comprar si pio roba infinica,
è E subito poi lefto-uno safiere Cb elt" hadue manicon s) badiali,
mn Quand' elta palfa, le alza le portiere. Che è ine quattordici arfenali.
f STANZA XXAIL, STANZA -XXXV,
Ed ella se ne va sicura,e franca, Vina cappa tane bella, e pula
b Sapendo ogms traforo a munadito, Di cotone; se ben vesta indecifo,
ie Perché troppo.non è, ch'ella ne manca, S' ell'¢ di drappo, o pur ringiovanita,
EP abito, fin quando havea mario, Perché non se ie vede pelo in viso,
'0 Scese; )£i70, fali ne mat fu hance s Evvi @ abiti pur copiainfinica,
2 t Sin che non hebbe di veder finite; Mia chi unto, chi roto, e chi ricifo;
2 All' ulssvia si fece in guardaroba Che il tempo guasta tutto; e per marura
% eAprir gli armadi,e cavar fuor la roba, Cosa bella quagzit pala, e non diira,
4 Malmantue manda un suo Ambalciadore, o Depataco a renaer' wbienes:
a Ce.
f.

E


44.534
a Celidora; ¢d ella attualmente, e corporalmen
tutte le stanze del Palazzo, ed in Guardaroba fa la
veramente adeguati a una Regina'di Malmantile.. 3
RENDE a la diceria, Cioè fece una Orazione d'
mone, o Discorso, col quale refe ubbidienza. is. 4
HA voglia di rinedere. Ii Poeta (prime benissimo il genio unit
fire donne, quale è di rivedere tutte le casse, armadi, ec. subito, che
o maritaggio entrano in una casa a loro nuova, ho isch ete
TERRENO., S' intendono qui, secondo l'uso., le prime fanze d' una cal
che sono al piano della frada, Del reo Terrenoé la tetra stessa così,0 così ¢
dizionata. Latino terrenum; folum, ager.» - send, -
PALONO il baieno, Cioè tannopretto, Dante Pars 25. Subito
di baleno. Inf, 22. i2 men, che non bafena, vatiot ”
OGNI traforo. Antendi ogni porta, ognicriuscita,/ogni minima:
4A MENA dito, Sa benitimo. Latino caller, Le sono notissime st
L ABITO' fin quando banea marito, Celidora, come s'è detto sopra C,
Fu moglie del Re di Malmantile, e da lui haveva ereditato i Regno, i)
MAVE, Color wrchino chiaro. Azzurro sbiancato, i
GV ARDINF ANTE. Vedi sopra C. 5. tt. 8. *: geomet
MEZZ ALANA, Tela fatta di lino,è lana, che inuna fola parola si dice
ancora acce//ana, quali accia, e (ana; roba assai da i nostri Contadini.) |
C.AMICIOTTO.. Così chiamano le Contadine,quella'velteda donna, chele
Fiorentine chiamano fortana, Et
CON bianche imbaftiture alla baixana, Costumano le nostre Contadine di fare
nelle loro vesti yicino a terra una cintura con punti di refe bianco in sul nero jun-
ghi, acciocché si veggano da lontano, e queiti punti sostengono una piegatura
fatta nel giro di detta velte per accortarla, e serve a loro per ornamento,0 guat-
nizione, e si danno ad intendere di far creder nuova la medesima — causa
di quella punteggiatura, e che aliora sia uscita delle mani del Sarto; il ee
quando vuole imbaftire,.0 dar priueipio a cucire yo' abito per mettere int 9
eda segno i pezzi, che vuol cucire, e solito fare tal punteggiatura larga, da
questo imbaffire si dice imbaftitura altrimenti feffitura, o ritreppio, Latino /ubfutnr4.
E questo verbo smba/tire servc per intendere ogni cosa principiata,e non perfezio-

nata; come éo ho imba(Pito L' orazione, che debbo recitare, ed in poche ere ”:
che diciamo abbozzare. we

BALZ ANA. Iniendono il giro da piedi della veste; altrove Pideos 'Latino
limbus « LF

TRINCIER AT E.camicinole.. Vuol dit camiciuole consumate dalle tignuoles »
per la similitudine, che e tra una campagna pieaa di trinciere, ed.un panno ple
no d' intignature, che percio apparisce bucato, € trinciato, Vedi sopraC. 8. st
51. E.che cosa sia camiciuola. Vedi sopra C, 6, st: 57, at otwe att

BANNO piagza a arme alle tignuole, Vedi opra Co. 51. -questo medesimo
concetto sopra il capo del Tura; B che sia tignuola al C, 6. st. 54. € Cs 10. (h 12+

ZIMARRA, Abito, che già ulavano portare le Donne Fiorenti all?
altro abito detto /orrana; il quaic da i Latini e detto amiculam, il qual'

' YY



tie
a

“= SSeeresiut

DVODECIMO,EDVLTIMOCANTARE. ~ 535

'veramente assai decorofo, e modeflo, e non come quello, che usano hoggi, del
quale si può dire:con Quinto Curzio lib. 5. Feminarum conniusa inenntinm in prin-
cipio modestus eft habitus, dewde fumma quaque amicula exuunt, panlatimque pudoré
profanant, ad ultimum ima corporum velamenta proyjciunt, Ma tornando a proposi-
to: Questa specie d'abito detto Zimarra haveva intorno al collo un collare gean-
de (che chiamayano bavero ) fatto di tela incollata,e cartone,e ripieno di stecche
d' offo di balena; ed in fu le spalle, dove ha principio il braccio un giretto actor-
no al braccio farto della stessa roba, che il bavero ) qual giretto il nostro Autore
appella aliotti, perch così si chiama, ed alle volte si dice piffagne ) dal quaie
pendeva una manica larga.,¢ grande quanto una buona sporta, la qual manica
non s' imbracciava, ma serviva così pendente per ornamento, e per una certa

“grave accompagnatura; ed oltre a questo dava commodita di riporvi fazzoletto,

Oaltro, che occorretie. Di queste maniche, tali se ne son vedute a' mici giorni,
che farebbono fiate capaci di cinquanta libbre di grano l'una, e più; o però il
Poeta dice, che sono il caso per andare alle nozze, ed ai mercati, perché vi si
può mettere molta roba dentro: E gli-aliorri, e banero difenderebbono da un col-
Po in riguardo della roba, di cui son compolli; E dice /a rea; perché questi ha-
veri, nascondevano dentro di loro tutto 11 capo di chi gli portava; e tali aliorti
si sono veduti, i quali coprivano più di inezzo il braccio.

DOVE si fa ai fafi, Dove si tirano le fafiate; il che segue in Firenze in Mercato
nuovo, dove 1 garzonetti delic butteghe de' Setaioli quindici, o venti giorni
avanti alla Solennica di S, Gio, Batilla fra il mezzodi, e il vespro fanno fra- di
loro alle fafiate, e necetiitano tutti li bottegai di quelle contrade intorno al
Mercato nuovo a star ferrace per quell' ore; e questo fanno per solennizzare la detta
festa quel tempo innanai; e per questa ragione tutte le botteghe, che sono in quel-
la firada, dove tirano i fatfi, hanao la riuscita in aleca strada per di dietro, di
dove entrano i macitri, e lavoranti, senza aprire lo (portello principale, e quivi
attendendo a i lor lavori, laiciauo che i loro ragazzr Gi piglino per quell'ore tale
spasso, anzi ci sono taiuoica de i maeitri, che comandano a1 loro ragazzi, che
vadano a pigliarii, spaveatati da un profetico detto: Guai a Firenze, quando in.
Mercato non si fara ai sassi; V sano di fare a' fai anche in Roma i ragazzi Tra-
fleverini. E fare a' faji, Hgucacaiente s' iotende » Mandar male, rovinarsi, get-
tar via il suo, Latino di/apidare, fare alla peggio, e operare senza giudizio; si
faceva a' sassi ancora in Firenze per accafione d'allegrezze pubbliche, e una fine-
fica di rame traforata fu posta al Palazzo de' Medici,oggi de' Marchefi Riccardi
Per vedere questo spettacolo, come e sato da altri scritto, ed osservato.

ARCISLVISITO, Ui cafissimo, buoniitimo,, attissimo, e più, se più si può E

dire. B' un termine, ches' usa per farsi intendere; più fu, che il superiativo, di-
cendosi buono, più buono, buonissimo, ed arcibuonitime. Ma dicendosi buono,
Migliore, in vece di più buono, e isquisito in vece di buonissimo, che fa.
V effetto del superlativo di buono, non pare che sia ben detto più isquisito, e»
isquifitissimo,facendosi cosi'un superlativo di superlativo; tuttavia per J'ulo inteo-
dotto non farebbe riprefo chi lo facetle; ed io crederei, che fusse meno biatime-
vole dire, arcisquisite, che isquifititfimo, perch non trovo troppo in uso il dire
più isquisito, onde non può s' uso antrodurre isquifitisimo.s che toguirebbe al più
squi.

ae



536, uuie nohace dae aaa
isquisito..;L Latini dicono bonus, melior 9 che: Q d F
byono,, migliore, e i/quifite; ed io conde cic i che Bctedfinn piste ass
miffiaus:, che faonerebbe più isquisito, isquifitissimo, se it I

trova eptimiffimus.. Appretio det nostri Autori Toscani si trova, 1
molto, aijai, e simili a i superlativi, come notammo Coat 17. >} ia r
buona grazia di essi, lo stimo.errore, perché molto, più 5" » Hiufili
faculta di scemare, e non cre(cere il superlativo,. aa

er esempio if tale e Luonissimo, vuol dite il tale è perferramente
iamo molto, certo', che (cemiamo la perfezione di buono 5
molto buono, ma von perfettamente buono, eficado maolte una'
s2.5 € non indeterminata, come e il superlativo: EB — » che
1iguifito, e isquifieissimo, o arcisquisite, hanno presa la vace
tivo da per se, e non come per superlative di buano; il che vi 7

tofna poi all' addigteivo aigliore, che non riseve altrazione, nomdicendof » nondicendof i;
migliore, Be migliorissimo, le hen si dicewmalte miglione, e assai mia
marlod' essenza;. come ia bbiia thes detto s»perché solo, o  affat miglit
men buono, che non fa migeiore assolutamente detto:, se non comparando:
all: altra quale sia.di loro meglio, st Amr
i ZANE, Colore fra il paonazo, e il lionato. 2p
OTONE, Vuoldire bambagia non filata, Manoi per cotone:
sorta dipanao col pelo annodato; come.è la saia rovelcia 50 il rovele [
Hon si dicono corone se non. hanno il pelo aanodato, che allora si dicano di coteney
o actoronati, Dice, che num e certo se sia rowescia, o drappa 5) pceaim
la feta. 2 ellendogli caduto il pelo, per esser logoro je perché:è senza pelo dice

che € riagiouanito; Sicch¢ in fuftanza vyol dire che: era usato, i lal.
R(CISO, Qui vale per intendere consumato nelle piegature d'un di !

epanno 5, per essere stato così piegato lungo tempo; che per altro ri
un legno, o altro materiale tagliato ne] mezzo yed e il contrario' div rife se

nel oy pela per illungo, Vedi sopra o. 11, tan, 36, ricife,
ANZA XXXVI. STANZA X\&K.
Basta es eve qualcosa un po cattind, Due altre 'armadj poi i fur

Che Celidora ha quint abiti, e panni, Che ? anoe tutto
Che al certo (tuttanolta ch' ella vina )
Puofrancamence andar in lq co gli anni,
Ma perché al suo char magnonos'arriua E un' altro di pin tr
'Di certe roppe, scampoli, e foppanni + Bealze ye scarpe ye)
Top Wimpaccio vollese a quella gente, Chea vedersi p er
"Ch edt'ha a' intorno,farne un belpresete, Ve poi'la nidoigi
. STANZA xxxViil st
mui se si parte ed Apre uno Riperto A
2 intagli,e a' arabe/chi ornato,e ricco,
“E trois due cafferse di belletto
Cort! altre di pezrette, e @ orichitco
va il Poeta a narrare glia arnesi,e
hon si parte dallo feh
' faye | Ae a ee

n

ME



Si elcr ibs.

a

SERRE ES o =

SSE ST Pesrsr st i ta.

DVODECIMO,EDVLTIMOCANTARE: 537

contro alle donne, mostra; che se usano il belletto, ed il liscio, hanno anche
bisogno della medicina da rogna, e del rottorio.

VN po cattiua, Quel po vuol dir poco per la figlira Apocope; ed un poco cat-
tiva, trattandosi di abiti, e d' altri materiali, s' intende per lo pit', consumati,
2

vecchi.
TVTT AVOLT A, ch' ella viva, Pub francamente andar in da con eli anm, Pav

che voglia dire, che se Celidora vivera, ha tanti abiti, che le basteranno molti.

anni senza farsene di nuovo; Ma dall' essere gli abiti della detta qualita, si com-
prende, che scherzando vuol dire, che se Celidora vive, invecchiera, percht
andar in Id con gli anni vuol dire invecchiare, come s' accennd sopra C, 2, stan. 2.

(siginines Ritagli, pezzi di panno, o drappo. Scampoli, vedi sopra C. 11;

in. 22.

SOPP:ANNI, Fodere, cioè tele vecchie, che hanno servito per fodere d'abiti.
Scherzando burla la generosita di Celidora, la quale con queste galanti ciarpe,
che son fondacci d' una bottega di rigatticre, o ferravecchio, regala i suoi più:
cari per non apparir meno generosa di Bertincila, che regalo la patcona, come
vedemmo sopra:C. 1. stan. 81. a:

D* oronetro: Par che dica d' oro pulito, e puro, ma intende wetto d' oro, cioè
puro; senz' oro, Equivoco usatissimo in'quelto propotite,

LA miaferizia per la casa. Incendiamo 11 Cariello', o turacciolo del ceffo; e
flo, perché un tale detto Galeno, che andava per Firenze vendendo tali carielli,
gridava shi vuol la masserizia per la casa, in vece di dire, chi vuol Carielli; od
era bene inteso da tutti,

RABESCHI, o Arabeschi, Specie di pittura fatta a fogliami, fiori,
mascheroni., o altro, tutto aggrottelcato, cioc sproporzionato dal naturale, detto cosi,
perché forse tal maniera sia venuta d' Arabia, secondo che si può dedurre
da. Cel. Rodig. Jib. 29. ¢. 5. dove trattando delle Lamie, e delic Sircae, dice;
LaAmmiam vero opera parerga ex Arabia maftichen vocant,

SELLETTO. Liscio. Mestura, con la quale si lisciano, ed imbellettano le
donne « Vedi sopra C. 9. stan, 38.

PEZZETTE. Sano pezzi di tela bambagina tinti col cremisi, e zucchero, ed
altre sono di carta fabbricate in Spagna, e se ne servono le femmine per colorirsi
di rosso la faccia.

ORICHICCO. Gomma di Ciriegio, di Pesco, o di Sufino, ec. della quale si
servono le femmine per lustrarsi la faccia, e per appiccarsi veli in fu la teita.

“PER Jambicco. Adagio adagio scaturendo da piccioli fori fatti nel coperchio
del fiaschetto., come s'ufa dei' acque odorifere. Lambicco e il nao della campa~
na, e d' ogni cappelio per uso di stillare, donde lambiccare, e passar per lambicco,
intende stillare; E lambiccare, o lambiccarsi il cervello, è lo stesso che mulmare,
detto sopra C, 10, stan.7.

ALLERA, Pianta nota, le di cui foglie eruono per cauteri; e così i ceci bian.
chi, li quali per tal effetto erano ia quello (tipo.. Da queste cose vili comprenda
il Lettore, che il Poeta si maaticne sempre in fu gli (cherai, deferivendo una Re-
gina, e Palazzo ricchi di quegli addobbi, che son conuenienti a una beac stant
cOntadina, e decenti alla grandezza d'una Regina di Maimantile,,

% Yyy. STAN,


Sh. MALMAN TILE 1980
STANZA XXXIX,.., ith NZ 3
dun caffon diferro vada REREO, c i i color |
L Quiuitvoua il morto, nia dd vero, s -
Che i diamantiye le givie di gram pregro
Lon v'bano che far nidla,e sono un zero;
Lerche si tratta, che vi. Safe un wero
bi perle, che se ben pendeana in nero
Examsi grosse, che st [parfe vace,;
Ch' ell' eran poca manco d' una. noce; Sun i quartrini, i precioli ye i bateati,
STANZA XXXX STANZA -XXXXIL
D? anells ya! orecchini Vé1h marame} 'Poi ne venixan gli occhidiciueste;
“Tanti gioie!ls pot, ch' e un fracasso; Ma il proseguir più olere fa interrortes,
Perc' alla donwa:

Di medaghe dorate 50, vavindi-rame' a
dir, che" Duca levolea far

Un moggio ne misurano,, @ di palo;
Ala quella e sparr ates, ed nn litame Ond? ella il tatto nelcafjon rimette

Risperto alle monere, che più baffo E riferrato scende giwdi (orto,
Le più belle comparsero del mondo; Oue Baldon ? aspercarn iftinali,
Ch! in faseri poses creffi stanvo al fonda: -» one partir di quini fha'im sul? ali >

: STANZA: EMBKI MO vinnd cow 2

Per e agginftare omas tutte le cose, In punto, @ questo fine aller
Che pin desiderar non si potea in: ier 'bined
Egli, ch' eva per far come le/pose La puliva.per metterie la fellay

LA ritornata s idef? alla Dacea, Licenrioffs costidullasorellay o >
Celidora trova il caffone de'.danari,, e coi tal-occatione i Poeta'
monete Fiorentine eficttive, ed immaginarie.. kn tanto che Celidora va vedendo
queste ricchezze; vien da lei Baldone-suo cugino per liceoziativ) 9
TROFA it morta, Cioè trova il buond. Diciamo rrewar it morta, o fare nit
morto, quand' uno trova ripod qualche gran vallente, o fa in gua-
dagno.. A. P
LON o ha che far nulla, Par che voglia dire non si stimano, vispette al? altre
Givie, che sono in.quet /uege; ma in eisai vuol dire; che quedo non e luoge per toro
cioè non ve ue fond, i b tone Se
Sf trata. Si discorre; Termine assai usato per esprimere una che
s' habbia di qualche cola'; quasi-dica > Si difeorre comunemente, che'
così..
AL marame. Una quantità grandissima. Marame propriamente vuol dire ogni
rifiuto di mercanzia, come quella, che dal mare è gettata a' riva bi i”
tum, Ma quando diciamo marame nel modo; che! è detto: nel eel
intendiamo abbondanza così grandé.d' una cosa, che generi naulea, €
disprezzabile la medesima cosa. Fra i nottci Contadini Gedice
tendefi? avanzo, e rifiuto delle frutte rimatte lord, dopo. la celta', o° vel
delle migliori » noa fo s¢ essi Rroppiano'la nostraparola, o-feonoi Cori
la loro, dico bene che mi pare più fighificante; Amaramejehe J
Fiorentino quello, che questo, che per così dire', ha del Nape
Vedi il Vocabolario della Cru(ca alla voce Cerna',

UN fracasso. È lo stesso che un flagello, un barbaglio detto sopra C. 7. stan. 5.

UN moggio. Il nostro moggio è di staia 24, lo staio è di libbre 50.\ di grano, e
la nostra libbra è once dodici, Ma qui è detto iperbolico, è significa quantità
grandissima.

RISPETTO a questo, A paragone di questo; cioè a paragone delle monete,
che son più basso.

I pesci grossi stanno al fondo, Detto, che significa: Il meglio sta nel fondo.

PIASTRA, È lo Scudo, o Ducato d'argento Fiorentino, che vale lire sette
ed è moneta effettiva. Il Fiorino è moneta immaginaria, e valeva quando più,
e quando meno, essendoci anche il fiorino d'oro, che forse è quello che habbiamo
ancora hoggi d'oro effettivo, e lo chiamiamo zecchino gigliato, ma il fiorino
ne immaginario, ne effettivo appresso di noi non è più in uso, Scudo d'oro
è moneta immaginaria usata da i Mercanti per facilita di scrittura, valutandolo
lire sette, e mezzo, se ben molti per scudo d'oro intendono la mezza doppia.
La Lira moneta d'argento effettiva, e si chiama Cosimo, e vale dodici crazie.
Il Giulio, che si chiama anche Pavolo è moneta d' argento, e vale otto crazie,
Il Carlino pur d'argento effettivo ne vale sci; ed il Testone val due lire; questa
moneta già in Firenze si chiamò Riccio, dall'impronta della testa del Duca
Alessandro de' Medici, che era ricciuta. La mezza piastra e d' argento effettiva,
e vale lire tre, e mezzo. La crazia è moneta d' argento basso, ed è l'ottava
parte del giulio. Il quattrino è moneta di bronzo effettiva, ed è la quinta parte
della crazia. Il soldo moneta immaginaria che vale tre quattrini; ed il battuto
ne vale due: hoggi l'habbiamo ambedue di bronzo effettive. Il quattrino si divide
in quattro denari di bronzo effettivi, ma hoggi non se ne vedono, se non in
occasione di tributi Ecclesiastici, che sono presentati, e son poi resi, perché gli
possano haver un'altr'anno.

OCCHI di Civetta, intende le monete d'oro, come il doblone, che vale lire
quaranta. La doppia, che vale lire venti. La mezza doppia, che vale lire dieci, Il
quarto di doppia, che vale lire cinque. L' ottavo di doppia, che vale lire due, e
mezzo, che tutte sono d'oro effettive. Habbiamo ancora il zecchino, il quale
chiamiamo gigliato, che vale lire dodici, ed è il più purgato, oro che si conij, e
si può dire il nostro unghero. Si trovano ancora de' dobloni di quattro, e cinque,
e di sei doppie l'uno, di conio Fiorentino.

SPAKTIMENT!, Divisioni, feparamenti. Chiamiamo spartimenti quelle,
divisioni di'tereeno, che Gi fanno ne 1 giardini per piantarvi le cipolle da tiori.

ali (partimenti:, se bene sono di diverle figure, si dicono anche quairi. Vedi
pe C,6,-ttan. 63..E per similitudine aiciamo spartimenti te divisioni » che si
trovano ineafiecte, o scatole, come erano queiti delle monere,

VENNERO pris hafere. Intendi Avvisi, o imbalciace, che Staferta appresso
di noi,¢:1o stesso, che Corriere. Sp. efafera.

\ BAR matte'. Elo stesso che abbaccarli con uno e parlargli, Vedi sopra C,
2, stan. 59.,in altro significato, — > ne

STA sm full aii. EP all' ordine per partirsi. SST

. FAR come le spose. Significa ritornare; lo dichiara il Poeta medesimo,dicendo:
Tdest la ritornata; E questo perché già costumavasi, e forse ancora in alcuni Iu

es:

a

, ee ee ee

=e 2 SS Sw

PR 6S we

d-
Koyy 1s ghi

=

s4o
hi@eoRitma, che le si dopo'effere state dicti', o pre:
foie rotniao alla casa paceraa's Fer sephe qui giù

Teniarns dell Achinea. Taupe lo fallone, 'che* cated
che Achinea, o Chinea, intendiamo il cavailo buon rer
éuina specie di cavaili particolare «Sp, bacanea. Franz, bacquenen'y
STANZA XXXXIV, ts “STANZA. Xx
O mai è tempo, cara Celidora,
Ch! inverso li miei [udditi m' apprefi >
Che 'l trattene*mi di vanvargio faora,

Pregsndicar potrebbe a' miei interessi Dite, non ci oi fusse corda,

Pero qui refea tu co! tuoi,sn buon bee, Bifog a Lmeteae epee a
E farti anwe, e rispercar da essi, ( Rispsfe il General) 3 ella 8

Ed in ordine a questo i conviene ee ome t

Fare anche un' altra cosa per tuo bene,
STANZA XXXKXV.
Perché, s' io parte ei »cugina mia,
Non fo 2 se tn ci havraituttiitnsigufti,
Che qui non è neffun., che per te sia,
Mentre forsee poi nuowi difeusti,
Ma voglia il Ciel, ch' io dica la bugia

Ed ogni modo vo', che tut' aggiuiti, 'tipo presto sles of
Per sicurtà con an compatie » Ugquale Vuolotu? parla. a Her =e |

S accafi teco,¢ qucfto, e il Generale, D: mat più si, e daccela in fa

STANZA XXKXVL STANZA AKAM;

LT byei hati difender si da vanto, Ed ella nel sentir, cons eit affrin

Che tn vedi,egli ¢branoquarun Marte, A dar pronta: rispotta atal do

E se finor per noi ha fatto tanto, D' un modesto roffor tutta,

Pifa quel che eifara,s'egli entra aparte,

Orsit  daglt la mano; cana [it ilgnanto;

E voi non ve ne fiate più in disparte,

Casa Latoni, o Amostante nostro

Fareui innanzi, dite il fatto vostro

STANZA L

Degli dunque la mano in mia prejenzas 3 Ma per non recar tedio

E voi, o General, datela a les, Idest a chi ascolta i versi mitiy

Ch io 'voglio prima della mia partenca

Veder solennizzar questi Flimenci. La[cidgliyadiame;

Baldone da per sposa Celidora al Generale Amoitaate Latoni “ai

dopo haver narrato il discorso fatto da Baldone a paliow per indurlaa

tarsi d' haver questo marito, ed i soliti lezzi donneschi farti da 2

dir di si; passa a di(correr d' un' altra sposa, che e Psiche, cone ee i

Lagere ouave.

hai neJunsche per te sia « Non hai nefiyno, she aid

a

'

DVODECIMO,EDVLTIMOCANTARE. 548

OVVTA. Termine che significa spedizione, o incalzamenio a far presto. BE' il
Latino Hia,age. Vedi sopra C. 6, stan. 90. alla voce, horse, aiegh 3
PASS ATE gud. Venite qua. Lat. ade/dum. B: modo di dire, che significa
comandar con imperio,.¢ con (everita, ed ha del bravatorio. R59,
SE vi piace la pannina, Se vi piace la mereanaia, cioè Celidora.
NON ci tenete più in sulla corda. Non ci fate più Aentage,o desiderar la rispo-
sta. Non cé renere piis coll' animp dubbio, e sospese,:
SON bell' e accordato. lo sono, affatto d' accordo; son contentissimo. Vedi fo-
pra C. 3. faa, 14, Questo termine bee, '
TERREL d bauerne di beato, Lo riputerei mia gran felicita, Stimerei d' haver
gran forte, WV' avrei di carti, Mi terrei d' etfer beato, ee
EGL1¢ dower sentir  altca campana. E' cosa giulta sentir I altra parte,
TRANA, Questa voce non havrebbe alcun significaco, se bene e assai usata 5
ma perché pace, che immiti il suono della tromba, quando si da la moffa a i ca~
vali, che corrono al palio; ci serve per esprimer mxovité  /pedi/citi, sbrigati a.
far la tal cosa, Q pure e detto Trana, cioè tra' pur/d tira avanti; dal verbo Tra-
nare, che vale trarre con fatica qualche cosa, e strascinarla.

, ALAL più. Questo termine usato nel modo, che è nella presente Ortava, ci è
familiarissimo, ¢d ha quati lo stesso significato che evvia detto poco sopra, e s'ula
Pua per F altro in occatione di stimolar qualcheduno a spedirsi; ed esprime una
certa impazzienza di colui, che stimola. E' il Lat. ea tandem. Finiscila,. dille
ana volta,

DAG ELA in sanore. Rispondi secondo il nostro desiderio, Quando si vince
una lite, si dice haner la sentenza in sanore..

CUOKIE con (a ghirlanda. Significa morir vergine. A coloro che muoiono in
concetto di vergini, quando si portano al sepolcro, costumasi di porre in testa
una ghirlanda di fiori in segno della loro castità. Qui il Poeta scherza, come è
solito farsi, quando si discorre d' una donna impudica, che Gdice Elba giurate
di morir con (a ghirlanda, ed è detto ironicamence, e per intendere, e//a vual por=
i tare il vanto ye La corona delle donne impudiche, Ma non per questo il Poeta (che
molto ben si ricorda, che Celidora, per essere stata moglie del Re di Malmantile,
non è più da ghirlanda, intende, che Celidora fosse impudica, ma dice così
per ischerzo, e per seguitare il costume della plebe, la quale, quand'uno nomina
sorella, madre, o moglie, suol dire; puttana di me, e simili. Se si parla d'ammogliati
suol dire becco del diavolo, ee. Tal cohtume moitrd il Poeta ancor fo-
praC. 2. stan. 21. dove dicendo: 4 saper quante paia fan tre buoi, foggiugne sfybi-
to Se ben dat padre, ec. e vuole intender padre bue, secondo lo scherzo suddetto:
' Non è pero questo stimato offefa, percht avvien sempre detto per ischerzo; ma
4 ricice bene odiolo, e riaferescevole l'eder.u/aco spesso, ed in ogni congiuntura,
y come è usato fra i più vili, che lo fanno per parer fagaci, e concettof.
¢ Sl riftringe nelle [palle. Cioè 8' accorda, ed accop/ente a quel, che altri dice,
ib
v

aS Sen

o propone. EB' un' arto solito farsi da quelli, che è rimettono, o aderiscono alla
voloata d' uno, per non poter fare alttuncnti, o convinti dalle ragioni, o indo
ti dalla necessità, quasi dicano: Pazienza; Bisogna frarct. Bocc. Giorn, 2, nov, 8,
' Ada pure nelle [palte rifiretto casi quella dagiar a daee setae mole Mas /ihoorre AnGa,
yar 2. se



Sate
Eefubetiesaivolen nos si faceia essert “7
volta della testa, non dimend dictarho >. re;

0 garbate: O così sta'bene Lat, edge, perphtore belle Te
sue ii contento |, che's' ha», he una cofa  succeda secondo chefi desid
APREST Oye male, te cone dafser Meglio'¢ farimale'y¢ pre!
si mai col pensiero “dis voler far benew Chi fa o|,,emale pfiaalineare'
cha facenuy adagio, e bene, mainon conchiude, o-termina'quel'cheha
moidi fare, non si puddin che facciay e veramente nonfa'y e pend nell'c
dei fare e meglio far male, che non fare. '
DATE (a mano Dar ia mado (Latinoviuagere\ dexreras yO la.
nia, che fr faccia negli spontaiizai') e dice impalmare O:far Limpalmamentos,
STANZA Lie oS BAN ZAMLDL oub
Sogwitoical sxb Lvoe gud Phiche bakes,
(olLanSenegee |y. che sn: last frggiafh patra
mand eiskincdrfecon La ging iddaes,,\

ee o al dueilo non volle la gatta; Per eat sa i
quefamnalivnara Medesy) >) sion Lagwale
ieplaeaens ere ot mqe) asBe Pe ep

neater (grades ust \& Biche trt/ud honor ae
ones', aldan pian, Ces ie perdadayy 00% ~ nel e es aero ae we }
ia

STANZA LIL; oralga on nenaaet
Bit won potends bauer Cupide sposoy 08 90 Pereincomintance/m: ets a4 F
hori: Amardai martha tontana >: \\Bacendo com? il'can delParcolano,, o)
Ou Revael,s elapher (can iia Ucanegdlah iG 9% (O'all! infatara now P
Che pur veduto sia da corpo humane: E non pao ines eae
Martinazza haveddo prdiilto, che dovea esser fatta imoriré, eiche per Gupi
do non dovea esser piirfucsspolo,, inttidiosa, che.questo'be ne havetioa epodie dd
alteiy: !-haveva incancatoun-udga igacto per impedire: yoche'altrinon havefe
\ EFOGIFA ratta., Boggiva velotemente, Ratto viene dal ae eee
verbio Fiorentino;.Cbrva-pianosa ratto, corri(pondente ai Latiaoy
GING ADEA, Intend laspada, come s'intende: conunenene wb al
deta dall'impugnarficé tutte cinque le dicasete bene itbastone pure simpugna coeur hur
te cinque de-dita, non si di¢e-cinquadea, pecché questo fipud im}
digch jal che\non Gi — fare delianspads ordinariayy 0fe pur ff
o con difficultà VS RSH A
wuollagstea, a vuolattendete pNomwuol'badarey
Rissmnneiir quel tal:negozio. Hl Berni nell\Orlando,=
« Chey come si suol dir, voglit la gutta, ~~
OVA Aeden B+ uora lacrudela, che eh Medea si
Oza Re de' Colchi:,»versaril fratello Absyreo- opr )
fo, Glauca sua rivale,¢' co yet th suo ne per 4

"ihe 'Vee Rte Sceey cr

jeicodeind Mamie) matto; A Gatto ata

goD

Se peeve fats = \&

-

ie
6

|

DVODECTMO,ED VLTIMOGANTARE '5432

ne fuan'o\pibes inet faitem a <, be aL a oe
do)0-da qualche donni at iftra ye wih won; sfiss alist ctloe

TIRA per dado. “Conia aplageresrnoraands più Bettilenel
la milizia, soldati insieme habbiano commefio qualche delitto 'ca-

pitale, farmorice tn di loro'y,¢ falvar.la:vita a tutti gli altrt, facendo loro tiz
rariila-forte ne s€ però 5.1 orcas dettirdadi, e da-credere, che ace
compagnine tal funzione con, fo! i xe con pianti;.¢ stimo però sche il Poeta
digcndo ztiraper-dade, intenda, toipieay o plange pill di cuore che mai; /eguirae
Piangeress pisces gagardamene yes sie pare, she non heaas here aim > 6 sia -
da principio,

“hssan wage. Esser desiderosa d' una tahoe. « Saiwere vago, che vuol dir be
lo, adarne.yec. Sig iglia\ ancora in questo fendi bramefo, ec. Tiraleé — divbes cir
vuol dire: Zi tale genio', ha gusto di betle burle, e feberzi.

HA già sanoil-pianto. Liha già pianto per perduto. Termine assai usato rims:
Gimili congiunture.. Pianto è quel lamento, che si fa-sopra il morto,decgo:così.dal
batterti, per.dolore il petto.» Latino planitus, rodalia = voce Lavina:hanne fat-
ta Gmiimente i Pranzefi la loro Peainte, seh ats eh

eA LZ AR capanne, eo, Cioè quei monti di scope 9 ec. chevsaveno fatei per \&b-
biuciar Martinazza come fredetto sopra in questo G.t.-3» e queste sonove 2o/e
as Fusco, ie quali dices che sshanno a fare per-hongrdi Jet; \cheper altrovyquan-

do diciamo:: s! banno a fare.cose di ey on st afarcofe, bole; wine.

frofe, e fuori del consueto,

FAR come il cane dell'ortolano, Cioè non volere, o non potere havert una cosa,
ed impedire, che altri l'habbia, come fa il cane dell'orcolano'} che nin,
fuangia-! crbaggio, ¢.non vuole che altet lo. Piglt Canis in Praesepi + Provetbio
nlato da bucianoy

eT AN Ze: bItbb

“iquid 6 of STANZA LV M
tio, \e Bsiche bebbtrogeuife Cos 'byes: affanni ¥¢ le fatiche 0!) Ob
< WDE extte quello eb' è feglito'ie Corre; » Soffente per rant' anni, e lafri ee

~Gbda il teiogoappinte now si farprecifoy

Risrovatosi, Amore; ed egli, e Priche

AtRena si fainsaprer'tattele porte; » Rappattumato fu dai cavalieri 30
se Amanro crofeiar fenve/i wrgran rife, >: Onde foordats deli"ingiurie amicbe, a
atest obie peg gio; poi fwonsr; ma forte Eriuniti più che volentieri:
\& Aeafivmare.di-pefe sr aboccanti, + vad regp sposi fero i bactabaffi yoo 9
i obSemea sunosceriehi rece, Contantics' bq Restando \# parte diver foe je (pai
STANZA LV. STANZA shVTR cor 6
“Gir per peniescate ognnn presto addirizza, (i Gluntis cialdeni pots e fare i bile,
Che dal timor gli si arricciane è peli. Ml Duca diede al fin 2 ultimo Addie' -
Ma C alagrillo aitiero, ¢-pien di fiicxa “E Jubiro conagni suo vassallo
ib oGem talus frrifeia fa colps cradels, =: dnnerfa Venano Spiele it-pendio
wi Wa per le stance fende,taglia.e infizza, E Catagrilio ix groppa al-fud <aualld
jlut 444 mon cliappa, se vende' logences cil' o \Preforoon Pficle it Raretrare Dia,)

~ien Rar tde inns e i¢ok fucertibroinranty,.
E il Diavol cacciaye manda vialincato,

o8e\Gupido per opra dij Pakide Aixicrova je per inekzo di-quei Cavalieri'

“3uVO

'¢Aashrei pars); eintefolildor 20
Gb ricomdusse 'ali' Amoroso. a


344
con Psiche, si fanno le feste delio spolalizio di
lo di loa Bache con iain beioenta ~ dy,
lo accompagna Psiche se Regno d',
€ROSCLAR an ria. Rider gagliatdamenre- Vedi
GRAV, traboce: Gravi:più del giusto pelo,
on delle nee ee on se ne feeuc per
¢ seguita, chi recé contanti ( che termine proprio
= intender, chi dava se heeds '
e4ADDIKIZZ 4Ciok va via. Fugge per la più hota tema

STKISCLA; Intendila spada, come intese (apra:C. 2: st, 60. ° ae

CHIAPPA, Coglig s ritrova, perquote s¢aipilce. Vedi sopra C7.

RAGNATELI, Ragni,piccolt vermis o inferti nati... Vedi sopra.
Le stanze piens di ragnateli significa vote dogui.altra fa. Siauimente |
volendo dire il borficchio voto, dite; Plexys facculius est arancaram,

RAP PATTV MATL. Intendiamo rappacificati. Da molti si dice
ge di pace donde: O vincere, o patrare, clo' pareggiare; far pace» ae
gredo venga questo verbo rappatramare, il quale e atlai usato, mala) 4
da pochi fuori della plebe.::

CIALDONI, Specie di pasta confetta, contorta sottile come l'ostie, ed attorta,
e ridotta come un grosso cannello di canna

STANZA LVIILED VETIMA.,
Finito è il nostro scherzo: hor facciam festa,
Perché la Storia mia non va più avanti,
Sicché da fare adesso altro non resta,
Se non ch'io riverisca gli ascoltanti.
Ond'io perciò cavandomi di testa
Mi v'inchino, e ringrazio tutti quanti;
Stretta la foglia sia, larga la via:
Dite la vostra, ch'i' ho detto la mia.

SCHERZO, Qui vale per trattenimento, Latino lusus, Sogliono i nostri
Contadini, quando fanno le loro veglie di ballo, dopo che hanno un pezzo ballato;
introdurre qualche intermedio, rappresentazione, o giocolamento di forze, o
altro, q questo chiamano lo scherzo, che per lo più finisce in burlar qualche
semplice, e dar'occasione di ridere, e questo tale è poi anche detto lo scherzo, e così
l'intendiamo comunemente, ed il nostro Poeta molto bea l'esprime servendosene
nella sua lettera alla Sereniss. Arciduchessa Claudia d'Austria, riportata sopra
nel Proemio, dicendo: Contentandomi io, che la mia Leggenda, come nata da scherzo,
mi faccia scherzo alle genti.

FATE festa, Cioè siate licenziati, Vedi sopra C. 10, st. 42.

Nota, amorevole Lettore, che il Poeta per terminare la presente sua Oera,
ringraziando con questa ultima Ottava gli uditori, si serve della chiusa inventata,
ed usata dalle donnicciuole, quand' hanno raccontata una novella; cioè
Stretta la foglia sia, larga la via;
Dite la Vostra, ch'io ho detto la mia,

E conchiude, che ha contata una Novella, come diede intenzione sul principio
di quest'Opera. Ed io pure me ne servo per incitare altri a dir qualcosa
meglio di quello, che habbia fatt'io, non so s'io mi dica nel dichiarare, o pure
confondere, ed intrigare quello che nella presente Opera ho stimato poco
intelligibile fuori della Città di Firenze, e prego il discreto Lettore a compatir
me, che per ubbidire ho pigliato a far' un volo superiore alle mie forze, ed a
contentarsi di biasimar me solo, e non quei, che mi comando, perché habbia
fatto errore nell'elezione. E fo punto.

FINE DEL XILEDVLTIMOCANTARE.

è; MEE LY Big

I Molto Rev, Sig Gio; Domenico
cia di riconoleeté con ogni di
. Opera fouo il Titolo di Malmant
Zipolt, vi sia cov alecuna, che ¢
~ Cattolica, eda' buoni 'Costumi 4
» Maggio 1686.; =e aoe '

Niccolo Castellam Vic. Cen. Fiorent, dam Ry ia

si
Mluftriss. e Rev. Sig, g
Ho attentamente letto Oe cor Operetta al
le Racqusftato di Perlone Zipoli, insieme con le fae note
spiegazioni, e per non avervi trovato cosa, ne
alla Santa Fede Cattolica,ed a' buoni costumi,
mano mi soscrivo. Firenze 20, Settem. 1686,
Gro. Domenico del Bruno en Sac, Ti

Attesa la sopraddetta relazione si stampi > osservati gli ordini
soliti, Data z0.Settemb, 1686. > Z;
Niccold Casteliani Vic. Ge

I Molto Rev. Padre Lettore Dolci Minor Otfrvante Conf i
tore del Sant'Vfizio di Firenze legga attentamente la
presente Opera di, Perlone Zipoli,. intitolata Malmantile
Raquistato, e ritrovandovi cosa repugnante alla Sat
Cattolica, e buoni costumi, riferisca, Dal 9, Vfizio.
Firenze 17, Ottobre 1686.::
Fr, Francesco Agostino Gambaroua Min,
Del S, Vyizso.

Reverendiss. Padre,:
Ho rivista, e ben considerata l'Opera intitolata A
le di Perlone Zipoli, e per non esslervi cosa repu
-aggiunte, stimo possa
D'Ogni Santi li 24. Febbr. 1686, Pe
Fr, Bragio Dolei Ain, Offer. Conf, del S. j

Attenta presata relazione.
Imprimatur;

Fr. Ces. Pallarvicinus Ordimis Min, Convent, Vie, Ge
S. Off. Florentia.

Ruberto Pandolfini Senat. e Aud. di S, A. S.

Stel 3% oy rm i

BT 2Mh ood ww

LocalWords:  havrei Sinigaglia habbia habbiamo crazia crazie donnicciuole
% LocalWords:  Celidora Bertinella soggiugne Conchiude giuoco giuochi Acciò
% LocalWords:  Malmantile bracciuoli Amadigi haver havevano alli pilorci ei

% LocalWords:  huomo dicesi nidio sustanza Franzesi perquote rovella havrà
% LocalWords:  Malatesti fussero sieno Doralice misterio donnicciuola fusse
% LocalWords:  giuocare perditore minchiate maraviglia cerviero Martinazza

\clearpage

{%
  \kern -2em\fontshape{sc}\normalsize \centering
\textls[180]{\Large Di M. Francesco Berni}\\ \kern 2pt
\textls[360]{\normalsize alla corte}\\\kern 7pt
\textls[240]{\Large del duca Alessandro a Pisa} \\ \kern 2em
}

\begin{verse} \large
  \backspace Non mandate Sonetti, ma prugnuoli,
  Cacasangue vi venga a tutti quanti,
  Qualche buon pesce per questi dì Santi,
  E poi capi di latte negli orciuoli.\\

  \backspace Se non altro de' talli di Vivuoli,
  Sappiam, che siate spasimati amanti,
  E per amor vivete in doglia, e 'n pianti,
  E fate versi come Lusignuoli.\\

  \backspace Ma noi del sospirare, e del lamento
  Non ci pasciam, né ne pigliam diletto:
  Perocché l'uno è acqua, e l'altro è vento.\\

  \backspace Poi quando vogliam leggere un Sonetto,
  Il Petrarca, e'l Burchiel n'han più di cento,
  Che ragionan d'amori, e di dispetto.\\

  \makebox[6em]{} Concludendo in effetto
  Che noi farem la vita alla divisa,
  \makebox[1em]{} Se noi stiamo a Firenze, e voi a Pisa.
\end{verse}

\clearpage
% LocalWords:  Gambastorta Calagrillo ianadattico primiera capresto
