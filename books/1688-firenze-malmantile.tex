\documentclass[12pt,a5paper]{book}
\usepackage[utf8]{inputenc}
\usepackage[T1]{fontenc}
\usepackage[italian]{babel}
\usepackage{changepage}

\usepackage{etoolbox}
\apptocmd{\thebibliography}{\setlength{\itemsep}{-2pt}}{}{}

\usepackage{tikz}
\usetikzlibrary{decorations.shapes,shapes.geometric}

% avoid orphans and widows, allow for (a lot of) letter spacing.
\usepackage[defaultlines=2,all]{nowidow}
\usepackage[tracking]{microtype}
\sloppy

% how to format and space chapter titles
\usepackage{titlesec}
\titleformat{\chapter}[display]
            {\huge\bfseries\scshape}
            {\vspace{-1.5em}}
            {0pt}
            {}
\titleformat{\section}[display]
            {\large\bfseries}
            {\vspace{-1.5em}}
            {0pt}
            {}
\titleformat{\subsection}[display]
            {\normalfont\fontsize{12}{14}}
            {\vspace{-1em}}
            {0pt}
            {\centering}
% I like this font!
\usepackage{tgbonum}
\renewcommand{\rmdefault}{qbk}
\usepackage{lettrine}
\usepackage[left=13mm,top=11mm,right=13mm,bottom=13mm]{geometry}

\newcommand{\backspace}{\-\kern -1.0em}

\makeatletter
\renewcommand{\@makefntext}[1]{%
  \setlength{\parindent}{0pt}%
  \begin{list}{}{\setlength{\labelwidth}{12pt}%
    \setlength{\leftmargin}{\labelwidth}%
    \setlength{\labelsep}{2pt}%
    \setlength{\itemsep}{0pt}%
    \setlength{\parsep}{0pt}%
    \setlength{\topsep}{0pt}%
    \footnotesize}%
  \item[\@thefnmark\hfil]{#1}% @makefnmark
  \end{list}%
}
\makeatother

\renewcommand{\negthinspace}{\hspace{-3pt}}

\newcommand*\sepline{%
  \kern 3pt \hrule \kern 2pt
}

\title{%
  \kern -2em\fontshape{sc}\normalsize
\textls[180]{\Huge MALMANTILE}\\
\textls[360]{\normalsize RACQVISTATO.}\\\kern 8pt
{\LARGE POEMA}\\
\textls[240]{\Large DI PERLONE ZIPOLI}\\
{\small CON LE NOTE DI PVCCIO LAMONI.}
}

\author{%
{\normalsize DEDICATO}\\
\textls[220]{ALLA GLORIOSA MEMORIA}\\
\textls[-20]{\footnotesize Del Sereniss. e Rerverendiss. sig.\ Principe Card.}\\
\textls[320]{\textsc{\Huge LEOPOLDO}}\\
\textsc{\LARGE de' medici}\\
\textsc{e}\\
\textsc{risegnato alla protezione}\\
\textsc{del}\\
\textls[-20]{\footnotesize Sereniss. e Reverendiss. Sig Principe Card.}\\
\textls[40]{\textsc{\Huge FRANC. MARIA}}\\
\textsc{\LARGE nipote di s.a.r.}
}

\date{%
\vfill\scriptsize
\textls[120]{\small\scshape In Firenze}\\
\sepline{}
\textls[-30]{\scriptsize Nella Stamperia di S.A.S.\ alla Condotta.\ 1688.\ \textit{Con lic.\ de Super.}}\\
\textls[120]{E PRIVILEGIO}\\
\textls[60]{Ad istanza di Niccolò Taglini.}
}

\newcommand{\flagverse}[1]{\vspace{6pt}\hspace{-8pt}\makebox[26pt]{\fontshape{sc}\footnotesize \hfill{}#1\hspace{4pt}}}

\newenvironment{signature}{

\kern 1em

\hfill \begin{minipage}{4.5cm}\centering}
{\end{minipage}}

\newenvironment{poesia}{%
  \kern -2em
  \setlength{\parindent}{-1em}%
  \setlength{\parskip}{8pt}%
  \begin{adjustwidth}{5em}{}\-

  }
               {\end{adjustwidth}}
\renewenvironment{verse}{%
 \itshape\setlength{\parindent}{30pt}
  \setlength{\parskip}{0pt}
  \obeylines}
               {}
\newenvironment{ottave}{%
  \noindent\hspace{36pt}\begin{minipage}{10cm}
  \setlength{\parindent}{-18pt}
  \setlength{\parskip}{6pt}
}
               {\end{minipage}\\

               \kern 7pt}

\newenvironment{argomento}{%
  \section*{\textsc{Argomento}}\hspace{36pt}\begin{minipage}{10cm}
  \setlength{\parindent}{0pt}
  \setlength{\parskip}{-2pt}
  \obeylines}
               {\end{minipage}\vspace{10pt}}

\newcommand{\stanzadash}{\rule[2pt]{54pt}{1pt}}
\newcommand{\markstanzablock}[1]{\item[\stanzadash] \textbf{#1} \stanzadash}
\newcommand{\makestanzalabel}[1]{\textit{\textbf{#1}}}
\renewenvironment{description}
                 {\begin{list}{}{%
                       \setlength{\labelsep}{4pt}
                       \setlength{\labelwidth}{8pt}
                       \setlength{\topsep}{0pt}
                       \setlength{\parsep}{0pt}
                       \setlength{\parskip}{0pt}
                       \setlength{\itemsep}{0pt}
                       \setlength{\leftmargin}{12pt}
                       \setlength{\itemindent}{0pt}
                       \let\makelabel=\makestanzalabel}\small}
                 {\end{list}}

\begin{document}
\raggedbottom

\pagenumbering{gobble}
\maketitle
\pagenumbering{roman}

\-

\vspace{6em}

{\centering\Large
{\footnotesize\textsc{al sereniss., e rev.\ sig.\ il sig.\ principe card.}}\\
\textls[44]{\huge FRANCESCO MARIA}\\
\textls[100]{\LARGE DE' MEDICI.}\\
\kern 2em}
Il
Sereniss. e Reverendiss. Principe Cardinale
Leopoldo de' Medici Zio di V.A.R.\ Principe
di quelle rare, ed ammirabili qualità,
che hanno fatto stupire tutto il Mondo,
fino da i più teneri anni dell'A.V.R.\
conobbe, che in lei dovea continuare quello
splendore, che hanno accresciuto alla
sua Sereniss. Casa le stimabili doti di
V.A.R; E per questo, siccome giudicò, che l'A.V.R.\ gli
dovesse succedere nelle virtù, e nella dignità, così volle, che
ella fusse anche erede della sua singolar Libreria. In questa,
havea l'A.S.Rev.\ destinato, che dovesse ottenere il luogo la
presente Opera di Perlone Zipoli, a cui S, A. R, m'onorò
comandarmi, ch'io facessi alcune note, grazia compartitami
(siami lecito il dirlo) forse con qualche scapito del
prudentissimo giudizio di S.A.R.; Ed havendo io ubbidito nella
miglior forma, che havevo saputo, già si pensava alla stampa,
quando i Fati invidiosi tentarono di privarla di così pregiato
onore: e sarebbe loro riuscito, se la somma prudenza di
quel gloriosissimo Principe non havesse a i medesimi impedito
il corso, con prepararle il rimedio nel rifugio alla protezione
di V.A.R.

Se ne vien però il povero Malmantile a' piedi di V.A.R.
umilmente supplicando la sua benignità a volersi degnare di
riceverlo nella sua grazia, e, come erede obbligato;
riverentemente convenendola al Tribunale della sua generosità,
perché gli faccia godere la giustizia, concedendogli il luogo
stabilitogli, acciò egli possa dirsi veramente rifatto dalle
rovine cagionategli da tante sue disgrazie, e da tanti suoi
sinistri avvenimenti: Ed io piglio l'ardire d'accompagnare
queste preci, che egli porge a V.A.R., come quello, che
conosco d'haverlo con la mia penna costituito in grado d'haver
maggiormente bisogno dell'autorevol patrocinio di V.A.Rev.\
alla quale intanto umilissimamente inchinato bacio
ossequiosissimamente la Sacra Porpora.

Di V.A.Rev.

\begin{signature}
Vmilissimo Servidore\\
Puccio Lamoni
\end{signature}




\clearpage
{\centering\Large
{\footnotesize\textit{Al Sereniss.\ Rev.\ Sig.\ il Sig.\ Principe Cardinale}}\\
\textls[44]{\LARGE LEOPOLDO DE' MEDICI}\\
       {\large PADRONE CLEMENTISSIMO.}\\
       {\normalsize PVCCIO LAMONI.}\\
       \kern 2em}

SERENISS. E REVERENDISS. SIG.

MENTRE stavo meditando d'ubbidire a i cenni stimatissimi
di V.A.Rev.\ col far le Note alla presente Leggenda di
Perlone Zipoli, mi cadde sotto l'occhio un sonetto del
Burchiello, nel quale havendo osservato, dove dice:
 Non sunt, non sunt pisces pro Lombardis,
mi saltò il ticchio d'esser' il Lupo nella favola, cioè che questo verso
m'avvertisse, che la faccenda da V.A.Rev.\ impostami non fusse
carne da' miei denti, ond'io havevo già quasi pensato di far conto, che
passasse l'Imperadore: Ma considerando poi, che farebbe stato errore in
gramatica, e da pigliar con le molle, il far'orecchie di mercante a i
riveritissimi comandamenti di V.A.R.\ ho risoluto di non metterla più in
musica, o in sul liuto, ne mandarla d'oggi in domani, dando erba
trastulla, e menando il can per l'aia, ma (venendo a dirittura a i ferri)
non tener più questo cocomero in corpo, e così cavarne cappa, o mantello
più per eseguire gli ordini di chi può comandare a bacchetta, che
perché io resti persuaso d'haver forze sufficienti a portar sí grave soma;
E quantunque io sappia, che havrei fatto molto meglio a lasciar la lingua
al beccaio, perché così havrei sfuggito il farmi dar la quadra, o la
madre d'Orlando, e sonar dietro le padelle da coloro, che si pigliano
gl'impacci del Russo, e ficcando il naso per tutto, fanno poi le Scalee
di S. Ambrogio, come quelli, che havendo mangiato noci, apporrebbono
al sale, senza considerare che ognun può fare della sua pasta
gnocchi, e che [come disse colui, che s'impiccò] ognuno ha i suoi
capricci; tuttavia ho voluto (legando l'asino dov'è piaciuto al padrone)
dare a conoscere che V.A.R.\ non farà, come il Podestà di Sinigaglia;
Se poi ad alcune di questi tali rincresce, mettasi a sedere, e, se non gli
piace, la sputi o mi rincari il fitto; e se dirà, che in fare alla presente
Opera le Note comandatemi, io non habbia preso il panno pel verso,
ma più tosto fatti de' marroni, e pigliato de' granchi a secco, lo lascerò
ragliare; perché son sicuro, che non mi farà baciare il chiavistello, ne
Pigliare il puleggio dalla casa mia; ne mi può accusare di delitto da
farmi mettere in Domo Petri fra i due Apostoli, o da farmi meritare d' esser'
ammazzato con una lancia da pazzo; E se l'indiscretezza di questi tali
mi condannerà per gli errori, che troveranno nelle Note fatte da me, la
mia ignoranza m'assolverà. Non ne ho saputa più: ho soddisfatto al
debito d'ubbidire, e mi quieto col detto di Donatello: Piglia un legno,
e fann'un tu. Mi fara forse detto: Tu porti frasconi a Vallombrosa,
cavoli a Legnaia, ed acqua in mare, e vai contrappelo alla buona
strada a comparire avanti a un Principe così erudito con questi tuoi
scritti; ed io a lettere d'appigionasi, e di scatola, senza saltare in sulla
bica, o entrar nel gabbione, rispondo a costoro, i quali fanno tanto il
Cecco suda, che portano ben loro le mosche in Puglia, e i Coccodrilli
in Egitto, e dandomi il mio resto, hanno trovato il modo d'intisichire,
senza però dirmi cosa, che io non sappia; perché conosco-ancor io il
pane da sassi, la Treggea dalla gragnuola, e le cornacchie dalle cicale; e
sapendo quanto il mio cavallo può correre, sarei venuto di male
gambe, e quasi come la serpe all'incanto, a metter questo cembolo in
colombaia; se non mi fusse noto, che colui, che è avvezzo a mangiar
sempre starne, desidera talora carne di Storno, e non fussi certo, che
la somma prudenza di V. A. R, (conoscendo, che il pruno non produce
limoni, e che dalla botte non esce mai, se non di quello che v'è
dentro, che parimente è impossibile, che il Gufo faccia il verso del
Rusignuolo) non è per isdegnare di ricevere le baie di Perlone Zipoli con
l'abito da villa messo loro in dosso dalla mia zucca, poco atta a
rappresentar l'impresa degli Accademici Intronanti, perché le manca il
Meliora Latent.

Supplico però l'impareggiabile umanità di V.A.R. a voler restar
servita di far conoscere a questi tali, che io ho legato il Cavallo a
buona caviglia, con fare degne queste mie insipidezze d'un benigno suo
sguardo; non perché lo meritino per se stesse, ma perché bensì conviene
alla continuazione di quel generoso aggradimento, col quale si compiacque
ricevere in vita dell'Autore il medesimo Malmantile. Il quale
se con le mie ciarle haverà fortuna di comparire in pubblico, godendo
sí pregiato favore, si potrà dire, nato vestito, ed io cascherò in piè
come i gatti, e mi pioverà il cacio in su i maccheroni: E così con
haver'immitato il cane di Butrione, non havrò timore di coloro, che passano
per la maggiore; perché sapendo essi, che l'Aquile non fanno guerra co'
Ranocchi, sdegneranno abbassarsi tanto con la loro critica, mettendo le
mani in si vil pasta; e quegli Aristarchi, i quali non contano, e non
hanno voce in capitolo, per haver poco di quel che il bue ha troppo, e
che sono come monete stronzate, o come i cavalli di regno; non saranno
causa, che io alzi i mazzi; ne mi faranno venire la muffa, o il moscherino
col loro gracchiare; perché oltre all'essere scritto pe' boccali, che il
Cieco non può giudicare de' colori, si sa ancora, che raglio d'asino
non entrò mai in Cielo, che però conoscend'io, che essi son per fare,
Come colui, che tosa il porco, non gli stimo il cavolo a merenda, e gli
ho dove si da al bossolo da spezzie, e dove si soffiano le noci; Sicché si
possono andar' a riporre a lor posta, e fare un mazzo de' loro salci.  E se
bene dice il proverbio, che la carne di Lodola va a Piacenza a ognuno;
io non mi curo, che me ne sia data, anzi per non mangiarne, son
contento far sempre di nero, purché non mi dieno di bianco questi Correttori
delle stampe, che tiranneggiando le lettere, perché si stimano il
Secento, cercano i fichi in vetta, e 'l nodo in sul giunco. Ma se poi mi
vorranno pure strazziare, io gli assicuro, che e' non hanno a mangiare il
cavolo co' ciechi, quantunque io non sia tanto addietro con l'usanza,
che io voglia mai far credere a haver cattivi vicini, o sia di natura
d'ungermi gli stivali a mia posta. Mi mandino, pure: all'Vccellatoio
quanto a lor piace, e mi facciano anche dietro lima lima, non faranno
però causa, che io faccia come Chele Masi, perché me la farebbono di
figura, e mi scotterrebbe troppo; se bene mi persuado, che ancor'essi
non fussero per uscirne netti; e che fusse per succeder loro il mangiar
noci col mallo, e far come i Pifferi di montagna, poiché, se essi si stimano
piccioni di Gorgona, ed io non son di Valdistrulla; perché sono uscito
di dentini ed ho rasciutto il bellico, e per questo so ancor'io quante
paia fanno tre buoi; onde a dirmi cattivo cattivo, la farà fra Baiante, e
Ferrante, perché io son d'una natura, che non posso ber grosso, e mi so
levar le mosche d'intorno al naso, ne mi morse mai cane, che io non
volessi del suo pelo, massimamente quando m'è saltato il capriccio di
voler la gatta, e badare a bottega, giuocando per la pentola; e s'io me
la son mai legate al dito, o l'ho presa co' denti, n'ho voluto vedere
quanto la canna; perché non mi suol morire la lingua in bocca, ed ho
tagliato lo scilinguagnolo, ne m'è piaciuto mai portar barbazzale, e so
lasciar la squola d'Arpocrate, quando è tempo, ed in particolare con
quei tali che, son più tondi dell'O di Giotto, e che stimando una stessa
cosa il chiacchierare, che il condennare, non sanno portare altre ragioni,
che quel maladetto \textit{non si può}.

Ma perché non paia ch'io saltando di palo in frasca voglia dar panzane
a V.A.R.\ e che questa mia lettera sia il vicolo di mona Sandra, conchiudo,
tornando a bomba, che stimerò d'haver toccato il Ciel col dito,
e tirato diciotto con tre dadi, se potrò conoscere, che l'A.V.R.\ resti
servita di credere, che in questa parte io l'habbia: ubbidita giusta mia
possa, come riverentemente la supplico a degnarsi di far apparire con l'onore
di nuovi suoi comandamenti. Mentre facendo la festa di S. Gimignano
umilissimamente inchinato bacio ossequiosissimamente a V.A.R.\
la Sacra Porpora.

\clearpage
\noindent\textsc{\centering
\textls[180]{\large al cvrioso e discreto lettore}\\
{\large pvccio lamoni.}\\
\kern 0.5em}

La presente Opera di Perlone Zipoli si manda alle stampe, per soddisfare
alla curiosità di molti, che bramosi di pigliarsi il passatempo di leggerla
ne hanno fatta instanza. E perché in alcuni detti, e proverbi usati
in Firenze, de' quali si serve il nostro Autore, possa esser' intesa anche da
color, che lontani dalla nostra Toscana, non hanno la vera cognizione del valore,
e senso di essi, vi ho aggiunto alcune note, con le quali se non ho appieno
soddisfatto, mi basta, che havrò forse data occasione col mio cicalare, che
venga ad altri voglia di meglio discorrere. Tu intanto ricordati, che questa è
una novella; e così ti accomoderai a compatire, se alle volte mi son fatto
lecito di dare qualche spiegazione favolosa. So, che havrai la bontà di sbandir la
censura, e ti tornerà commodo, perché facendo altrimenti havresti troppo da
fare, poche, o forse niuna essendo di quelle cose, che ho scritto, che non la
meritino con un nuovo foglio, e per questo non te ne prego: ti prego bene, se sei
Fiorentino, a legger' il Testo, e non le Note, perché queste non son fatte per te,
che, meglio di quel ch'io habbia scritto, intendi la forza de i detti, che ho
preteso dichiarare,

Dovrei notare gli Autori, a i quali son ricorso per tirare a fine la presente
fatica, ma perché gli bo nominati in tutti quei luoghi, dove è convenuto valermi
della loro autorità, tralascio di farlo; non voglio già tralasciare di confessar
l'obbligo, che queste mie Note, ed io habbiamo all'Eccell.\ e dottissimo Sig.\
Gio.\ Cosimo Villifranchi, ed agli Eruditiss.\ SS.\ Anton Casto, e Sig.\ Francesco
Maria Bellini, i quali m'hanno onorato di più erudite notizie; ed in ultima
attestar la fortuna che hanno havuto questi miei scritti di passar sotto l'occhio
dell'Ecc.\ Sig.\ Abate Anton Maria Salvini\footnote{Anton Maria Salvini, Firenze 1653 - Firenze 1729. Grecista, con Antonio Maria Biscioni, 1674-1756 figura sulla copertina delle edizioni 1731 e 1750 del Malmantile.} il quale non solamente s'è contentato
d'emendar molti miei errori, ma d'ingagliardire ancora le mie debolezze con non
poche sue bellissime erudizioni, a segno, che ha fatto nascere in me una speranza,
che sia per esser ricevuta volentieri questa mia Opera, e d'haver guadagnato
non poco appresso al Mondo letterato, per haver dato occasione a questo dottissimo
huomo d'esercitare la sua stimabilissima penna, i tratti della quale, come
non ho dubbio che nobilmente risplenderanno dentro all'oscurità della mia, così
son certo, che saranno da tutti benissimo ravvisati: Ne confesso però al
medesimo il mio debito, e ne porto al pubblico questa attestazione, perché si sappia
che quello, che sarà riconosciuto per non mio, non è latrocinio, ma regalo
fattomi da questo, e da altri huomini dotti per loro generosità, e per sollevar
Perlone dal discredito, che haveriano fatto meritare a questa sua Opera i miei scritti.\\
Lettore, vivi felice.

\clearpage

{\centering\Large
\textls[244]{\LARGE PROEMIO.}\\
\kern 1em}

Lorenzo Lippi\footnote{Lorenzo Lippi, Firenze 1606 - Firenze 1665, pittore. ``Perlone Zipoli'', poeta, scrittore.} (che in Anagramma nella presente Opera si chiama Perlone
Zipoli ) è stato ne i tempi nostri Pittore non poco celebre, come testificano
molte, e molte sue fatiche. Ciò lo fece meritare d' esser chiamato dalla
Sereniss. Arciduchessa Claudia d'Austria\footnote{Claudia de' Medici, Firenze 1604 - Innsbruck 1648. Reggente del Tirolo dalla morte del secondo marito Leopoldo d'Asburgo nel 1632 alla maggiore età del figlio Ferdinando Carlo nel 1646.} per valersi dell'opera sua a Inspruk,
dove dette principio a questa da lui chiamata Leggenda delle due Regine di
Malmantile, e la dedicò alla medesima Sereniss.\ Arciduchessa Claudia. Haveva però
l'Autore concepita nell'animo suo quest'Opera qualche anno prima, e nel
tempo, che essendo in Villa de' SS, Parigi a S. Romolo nell'andar per quelle campagne
a diporto, vedde le muraglie di Malmantile; ed haveva discorso questo
suo pensiero col sig.\ Filippo Baldinucci\footnote{Filippo Baldinucci, Firenze 1624 - Firenze 1696. Storico dell'arte, politico e pittore, ``Baldino Filippucci''.}, dal quale poi nel tessimento del Poema
hebbe, come da persona erudita ( che tale lo dichiara la sua bell'Opera mandata
da esso alla luce intitolata Notizie de i Professori del disegno) non piccolo aiuto
in proposito della lingua, e d'altro, e particolarmente nei descrivere il Consiglio
de i Diavoli nel Canto sesto.

Tal composizione fece egli a solo fine di mettere in rima alcune novelle, le
quali dalle donnicciuole sono per divertimento raccontate a i bambini, e di sfogare
la sua bizzarra fantasia, inserendovi una gran quantità di nostri proverbi, ed
una mano di detti, e Fiorentinismi più usati ne i discorsi famigliari, sforzandosi di
parlare, se non al tutto Bocaccevole, almeno in quella maniera, che si costuma
oggi in Firenze dalle persone Civili, ed ha sfuggito per quanto ha potuto quelle
parole rancide, alle quali vanno incontro tal'uni, che per spacciarsi huomini
letterati, non sanno fare un discorso, se non vi mettono, guari, chente, e simili
parole, che per essere state usate dal Boccaccio\footnote{Giovanni Boccaccio, Certaldo 1313 - Certaldo 1375.}, essi credono, che dieno l'intero
condimento alli loro insipidi ragionamenti, e stimano, che quello sia il vero parlar
Fiorentino, che non è inteso, se non da i lor pari, e non s'accorgono, che
in tal guisa parlando, si rendono scherzo di chiunque gli sente, come bene attesta
questa verità il Lasca\footnote{Anton Francesco Grazzini detto il Lasca, Firenze 1505 - Firenze 1584} in quel suo Sonetto sopra l'Opere del Berni\footnote{Francesco Berni, Lamporecchio 1497 - Firenze 1535. ``che dice le cose sue semplicemente, e non affetta il favellar toscano''.}, dicendo:
\begin{verse}
\hspace{-1em}Non offende gli orecchi della gente
Con le lascivie del parlar Toscano,
Vaquanco, guari, mai sempre, e sovente
\end{verse}
Ed Antonio Abbati\footnote{Antonio Abati, Gubbio inizio secolo XVII - Senigallia 1667} dice
\begin{verse}
\hspace{-1em}Peggio non ho, che quel sentir parlare
Con tanti quinci,e quindi, e, ec.
\end{verse}
Anzi in questa parte l'unica intenzione del nostro Poeta è stata di far conoscere
la facilità, e pienezza del parlar nostro, e \textit{Cogliendo della lingua materna il più
bel fiore}, mostrare, che ancora ad uno, che non ha (come'appunto, era egli)
altra eloquenza, o poca più di quella, che gli dettò la natura, non è impossibile
il parlar bene. Questo, ed altri fini dell'Autore s'argumentano dalla seguente
Dedicatoria, che egli stesso scrisse alla Sereniss.\ Arciduchessa Claudia, la quale
lettera io pongo qui per confonder coloro, che pur vorrebbono fargli dire quel
che mai il nostro Poeta hebbe in pensiero.

\begin{adjustwidth}{1.5em}{}
  \itshape
Ati figliolo di Creso Re di Libia (se è vero, che io non ne so più la, e la vendo,
come io l'ho compra) vedendo il padre in pericolo, isso fatto cavò fuora
il limbello, e disse le sue sillabe, come un Tullio; Tutto il rovescio dovrebbe
fare il pesce pastinaca senza capo, e senza coda della mia Leggenda a mal tempo,
ch'io mando a V.A.S.\ perché vedendo ella quel dolce intingolo di quel
fantoccio di suo padre in procinto d'esser mandato all'Vccellatoio, e quasi ridotto
alla porta co' saffi, e che gli sien suonate dietro le padelle, anzi fra il
tocca, e non tocca di scior Pallino, potrebbe a sua posta far' un mizzo de' suoi
salci, e farsi ricucire la bocca per non haver più occasione di formar verbo.

Ma perché si compiace V.A.S.\ di volerne una secchiatina, benché questa mia
Leggenda non fusse degna di fiutare eziam i luoghi privati, verrà di gala col suo
ricadioso cicaleccio, che si strascica dietro una gerla di farfalloni, a farne una
stampita anche ne i Palazzi reali, perché ella è una prosontuosina da darle del
Voi; Ond'io conoscendo nella temerità di essa l'ubbidienza dovuta de iure a i
riveriti suoi cenni, gli è giuoco forza, voglia il mondo, o no, che ella si metta
giù a bottega a sfogare la fisima de' suoi fantastichi ghiribizzi, contentandomi
io, che ella, come nata da scherzo, mi faccia scherzo alle genti. Compatisca
dunque l'A.V.S.\ questa sconciatura partorita nel tempo, che io do
festa a i pennelli, mentr'ella non apprezzando un'ette gli applausi volgari, riceverà
per grazia sterminata, e per arcisbardellatissimo favore, se queste baie
riusciranno di qualche valezzo nel cospetto di V.A.S.\ alla quale profondamente
inchinandomi, con ogni debita rivereaza bacio la Veste.
\end{adjustwidth}

Da questa lettera adunque si viene in non piccola cognizione de i sentimenti
dell'Autore nel comporre la presente Opera; La quale fu da esso presso che
terminata in Inspruch, e dedicata come ho detto alla Sereniss.\ Arciduchessa
Claudia; Ma essendo S.A.S.\ in quei medesimi tempi passata all'altra vita,
convenne all'Autore tornare alla Patria, dove fu questa sua Novella veduta da diversi
amici suoi, fra i quali dal sig.\ Romolo Bertini Servidore del Sereniss Principe
Cardinale Leopoldo de' Medici\footnote{Leopoldo de' Medici, Firenze 1617 - Firenze 1675, cardinale dal 1668.}, e molto accetto per l'ottime sue qualità, virtù,
e dottrina, e da esso hebbe S.A.R.\ la prima notizia della presente Opera, e fino
da allora mostrò l'A.S.R.\ non piccola inclinazione, che si pubblicasse, e se
tralasciò di comandarne la stampa, fu, perché sentì dal medesimo Bertini, che
l'Autore pensava d'accrescerla.

Fu veduta ancora dal sig.\ Francesco Rovai\footnote{Francesco Rovai, 1605-1647. ``Franco Vicerosa''}, e dal sig.\ Antonio Malatesti\footnote{Antonio Malatesti, Firenze 1610 - Firenze 1672. ``Amostante Latoni''.};
ambi Poeti nel lor genere Eccellentitfimi, dal sig.\ Salvador Rosa\footnote{Salvator Rosa, Napoli 1615 - Roma 1673. ``Salvo Rosata''} non men celebre
nella Poesia, che nella pittura, e dal quale il Lippi hebbe notizia Dello Cunto
de li Cunti\footnote{Pubblicato da Adriana Basile fra gli anni 1634-1636.} di Gianalesio Abbattutis\footnote{Giovan Battista Basile, Giugliano di Napoli 1566 - Giugliano 1632.}, di dove l'Autore cavò poi alcune novelle,
che si trovano in quest'Opera: La quale in somma fu veduta da molt'altri eruditi
ingegni; e fu il Lippi da essi consigliato, e poco meno, che forzato a metterla
alla stampa, con persuaderlo, che meritava la pubblicazione: ma ricusò egli
sempre di far tal passo, conoscendo molto bene, che colui, che stampa l'Opere
sue, s'espone ad un certissimo pericolo, per una incerta gloria, e massime nel
presente secolo, che vi è maggiore abbondanza di spropositati, e mordaci Satirici,
quali con invidioso livore lacerano le fatiche altrui, che di Censori discreti, i
quali con dotti avvertimenti n'emendino gli errori.

Dalle grandi instanze fattegli dagli amici suddetti, che egli stampasse questa
sua Novella, insospettito il Lippi, che il libro di detta sua composizione non gli
fusse levato, e contro a sua voglia stampato, andava molto circospetto, non lo
lasciando in luogo, dove fusse sottoposto a tal caso; Ma essendo una volta andato
in villa de' SS. Susini suoi cognati, e di quivi alla villa del sig.\ Don Antonio de'
Medici\footnote{forse Anton Francesco de' Medici, 1618-1659, frate dell'ordine dei Cappuccini}; dove havendo portato il detto libro per passare, leggendolo, la veglia,
la notte, mentre egli durmiva, il sig.\ Piovano Gualfreducci, ed il sig.\ Tommaso
Fioretti con l'assistenza del medesimo sig.\ D. Antonio sciolsero il detto libro, e
fra tutte due lo copiarono e la mattina lo rilegarono, e lo raccomodarono in
maniera, che egli non s'accorse del virtuoso furto. Questa copia capitò poi in
mano a Paolo Minucci\footnote{Paolo Minucci, Firenze 1606 - Radda 1695. ``Puccio Lamoni''}, il quale facendo al Lippi la solita instanza di metterlo alla
stampa, ed egli ricusando, gli disse il Minucci, che l'havrebbe egli fatto stampare;
¢ replicando il Lippi, che se ne contentava, se vi era modo, il Minucci
col mostrargli la detta copia scoperse il furto, e fece conoscere la possibilità, che
havea di farlo stampare, S'alterò non poco il Lippi veduto questo, ma come
huommo virtuoso, ed onorato volle, che la vendetta di tal disgusto fusse il costituire
il Minucci, ed ogni altro in grado di non si curar più di stampar quell'Opera;
questo fu con aggiugner'ad essa alcuni episodj, ed altro, in maniera, che in
breve tempo la ridufle da fette piccoli canti, che ell' era, alli dodici, che è la
presente; e perché non gli avvenisse di questa, come gli era accaduto della prima
teneva l'originale di essa in modo riserrato, e ristretto, che non lasciava vederlo
ne meno all'aria, e poco altro poteva haversene, che sentirne recitar da lui
qualche Ortava alla spezzata, ed il Minucci più d'ogni altro haveva questo favore
da lui, perché col fargli sentire l'augumento, che dava a quest Opera, stimava
di fare scemare nel Minucci la volontà di stamparla, e conseguir l'intento,
che s'era prefisso, ma ne seguì tutto il contrario, perché havendo il Minucci
sparso fra gli amici, che il Lippi riduceva la sua Opera in stato ragguardevole,
pervenne questa notizia all'orecchie del Sereniss.\ sig.\ Principe Card.\ Carlo de' Medici\footnote{Carlo de' Medici, 1595-1666.}
Decano del Sa.\ Collegio, e S.A.R.\ curiosa di veder quest'Opera comandò
al Minucci, che operasse d'appagare tal sua curiosità. Il Minucci manifestati al
Lippi i sentimenti dell'A.S.R.\ esortò a non contraddire di ricever l'onore
che S.A.R\ gustava di fargli; ed egli conoscendo, che mal poteva negare d'ubbidire
a tanto Principe, per il quale (come fratello della Sereniss, Arciduchessa.
Claudia) riteneva congiunto al debito di suddito un genio non ordinario di servirlo,
e persuafo pure una volta; che il pubblicar detta Opera non gli poteva
apportar se non lode, condescese a lasciarne pigliar copia per S.A.R.\ la quale si
piacque di dar dimostrazione del suo benigno aggradimento con atti non piccoli
della sua solita generosità, e verso il Lippi, e verso il Minucci, che ne fece
la copia, perché così volle il Lippi, o per spaventar il Minucci con la gran macchina,
che appariva, e così levarlo dal pensiero di pigliarsi questa fatica, ed
addormentare intanto nel sig.\ Principe Card.\ la volontà d'haverlo (come disse il
medesimo Lippi) o pure, perché quella copia non capitasse in mano ad altri, che
del medesimo Minucci, del quale si fidava, e per sua bontà, e perché haveva
anche veduto, che di quella copia, che teneva detto Minucci della prima Opera,
non s'era mai saputo cosa alcuna, perché esso Minucci l'haveva sempre occulata,
e negata a ognuno d'haverla, Ma quel'ultima copia sendo in mano del
detto Sereniss.\ sig.\ Card.\ Decano, accrebbe nei SS.\ suoi Cortigiani la curiosità
d'haverla, e cosè per diverse vie ne trassero una copia. Da questa poi se ne sono
sparse infinite; ma perché l'Autore sopravvisse qualche poco di tempo, e sempre
accrebbe, o moderò qualcosa, ed oltre a questo, perché la poca avvertenza di
coloro, che hanno copiato, ha causato, che si trovino molte copie, e difettofe,
o guafte, il Minucci riputandosi in un certo modo cagione di questo disordine risolvette
per rimediarvi, di supplicare il Sereniss.\ Principe Leopoldo (allora non
Cardinale, al quale dall'Autore stesso fu quest'Opera dedicata, dopo la morte
della Sereniss.\ Arciduchessa Claudia) di permettergli il mandare la detta Opera
alla stampa, per rinnovare la memoria de] già defunto Lippi\footnote{Siamo quindi fra il 1665 ed il 1668.}, e S.A.\ glielo
concedette, con obbligo però, che gli facesse alcune Note, ed esplicazioni; E così
contento l'universale, che desiderava tal pubblicazione, e diede al Minucci il
gastigo d'esscre stato causa del suddetto disordine, ed al Lippi la soddisfazione\footnote{postuma}
dovutagli dal Minucci per la violenza fattagli, con obbligare il medesimo Minucci
a sottoporre ancor'egli i suoi scritti a quei danni, che dalle stampe ne
risultano; Sentenza veramente giusta, come appoggiata al fondamento della pena del
Taglione, ma troppo severa nell'arbitrio per la gran disparità, che è fra la vaga
Opera del Lippi, e l'insipide chiacchiere del Minucci, sopr'alle quali, e non
sopra gli scritti del Lippi si fermeranno, e poseranno tutti gli Aristarchi; con
tutto questo non ha il Minucci voluto intentare appello, anzi, sendosi accinto
subito a dare esecuzione alla sentenza, ha aggiunto all'Opera le Note comandate,
con le quali ha egli preteso d'operare, che fuori di Firenze, e della nostra
Toscana, e per tutta Italia possano esser meglio intese molte parole, detti, frasi,
e proverbj, che si trovano nell'Opera, forse non intesi del tutto altrove, che in
Firenze; e prega il Lettore a compatire, se non sia da esso soddisfarto appieno, e
ricordargli, che non è stata mente del Minucci il portare l'etimoiogia delle parole,
frasi, e proverbj, ma d'esplicargli in maniera, che possano esser'intesi anche
fuori di Firenze, ed habbia il medesimo Lettore la discretezza di riflettere, che
molti Fiorentinismi sono in uso, nati dal puro caso, senza un minimo
fondamento, o ragione, perché si dicano, e che;
\begin{verse}
Non omnium, quae a maioribus nostris scripta, aut dicta sunt, ratio reddi potest.\footnote{Adattato da Tommaso d'Aquino, Summa Theologiae, Q.\ 95, Art.\ 2. ``Sed non omnium quae a maioribus lege statuta sunt, ratio reddi potest, ut iurisperitus dicit.''}
\end{verse}


\clearpage
\noindent\textsc{\centering\large
\textls[240]{\Huge MALMANTILE}\\
{\small DISFATTO}\\\kern 6pt
\textls[360]{\LARGE ENIGMA}\\\kern 4pt
{\normalsize DEL SIG.\ ANTONIO MALATESTI.}\\
\kern 1em}

\begin{poesia}
Ov'è l'Etruria indomita, e infeconda,\\
Già fui per molti figli e ricco, e bello,\\
Or c'una fascia a pena mi circonda,\\
Povero, brutto, e vil non son più quello.

M'hanno gli amici più che 'l vento, e l'onde\\
Levate l'ossa, e toltomi il cappello,\\
E fino il nome par che corrisponda;\\
Vna mala tovaglia, o un mal mantello.

Così ridotto trovomi a mal porto,\\
Col corpo voto, e senz'un membro intero,\\
E pur con tuttociò non mi sconforto;

Anzi ora godo, e farmi eterno spero,\\
Mentre in Flora un' Augel per suo diporto,\\
Cantando in burla, mi rifà da vero.
\end{poesia}


\chapter{Primo Cantare}
\pagenumbering{arabic}

PRIMO Cantare. Ecco che il nostro Poeta mantiene l'intenzione
data di pubblicare una Leggenda,e non un Poema, mentre mette
sopra ogni Canto l'inscrizione, che si vede in diverse leggende
dove in vece di dire Canto 1., e Canto 2, ec. come usano nei
Poemi Italiani, egli dice Primo Cantare, e così seguita fino all'ultimo,
volendo per la sua modestia esser chiamato Compositore
di Leggende, non Autore di Poemi, ed in uno stesso tempo
con bell'arte difendersi dalle censure di chi lo tacciasse di non aver'osservate le
regole del comporre i Poemi, sapendosi, che a queste non sono sottoposti i
compositori di Leggende.

\begin{argomento}
Marte sdegnato perché il Mondo è in pace
Corre, e da letto fa levar la suora,
E in finto aspetto, e con parlar mendace
Mandala a svegliar l'ire in Celidora,
Fa la mostra de' suoi Baldone andare
Indi all imbarco non frappon dimora,
E per via narra con che modo indegno
a occupate avea il suo Regno.
\end{argomento}

Gli Argomenti a tutti li Canti di quest'Opera sono di Amostante Latoni, cioè
Antonio Malatesti, fatti di comandamento del Sereniss.\ Principe Cardin.\
Leopoldo de' Medici.

\section{Stanza I}

\begin{ottave}
\flagverse{1}Canto lo stocco, e 'l batticul di maglia,\\
Onde Baldon sotto guerriero arnese,\\
Movendo a Malmantil' aspra battaglia\\
Fece prove da scrivern' al paese,\\
Per chiarir Bertinella, e la canaglia\\
Che fu seco al delitto in crimen lesa\\
Del far' a Celidora sua cugina,\\
Per cansarla del Regno, una pedina,
\end{ottave}

Mostra l'Autore in questa sua introduzione, che egli vuol descriver da Guerra
fatta da Baldone in aiuto, e difesa di Celidora, e vuol persuadere, che se ben
dice \textit{aspra battaglia} fu una guerra di nulla, e però seguita: \textit{fece prove da
scrivern'al paese}, del qual detto ci serviamo per derisione, quando altri ha fatta
una azione da lui stimata grande, e bella, che in effetto non è poi tale, anzi è
tutta il contrario, e si dice:  \textit{Hai fatto assai, scrivi al paese}.

\begin{description}
\item[BATTICVLO di maglia] Intende il Giaco, arme difensiva di dosso, cioè una
camiciuola composta di maglie di ferro, ed è la lorica ansulata, che usavano gli
antichi. E se bene \textit{batticulo di maglia} non è veramente buon Fiorentino, nondimeno
è spesso usato, ma per giuoco, ed è comunemente inteso per il Giaco, e si dice
così, perché coprendo quest'arme le parti di dietro, nel moto che fa colui, che
l'ha in dosso, batte in quella parte; come si dice Picchiapetto quel Gioiello, che le
donne usano portare al collo pendente sul petto.

\item[MALMANTILE] E' un Castello antico vicino a Firenze circa dieci miglia,
  oggi del tutto rovinato, e distrutto, ne vi si vede altro che lé muraglie Castellane.

\item[CHIARIRE] Questo verbo, che oltre a gli altri significati, vuol dire Far conoscere
  l'errore, o Render capace; nel presente luogo vuol dice Scaponire, o
  Sgarire: \textit{Il tale mi faceva l'huomo addosso, gli ho dato una buona quantità di pugna, e l'ho
chiarito}; cioè con questo l'ho reso capace, e fattogli conoscere la stima, che io fo
  di lui, e quella che egli deve far di me. Questo verbo è traslato dal verbo Chiarire,
  che è Purificare ogni liquore torbido, e contaminato da materie crasse.

\item[CANAGLIA] Gente vile, ed abietta, che tali saranno, come vedremo, i soldati
  di Bertinella, i quali il Poeta mette Huomini d'infima plebe, che Cicerone
  chiama Imi subsellij homines. Il Sig\ Francesco Maria Bellini in alcune sue bellissime
  reflessioni, che si è contentato fare sopr'alla prsente Opera, ponderando la
  parola Canaglia dice, che l'allungamento delle parole in \textit{aglia} sta Oggi in
  Toscana un certo avvilimento, e disprezzo del subietto, e s'usi solo in cose vili, e
  plebee, e però si dica de' Birri sbirraglia; della Plebe. Plebaglia, e gentaglia; de i
Fanciulli, e popolo infimo Spruzaglia, (metaforico da spruzolo, acqua minuta)
e che questo sia antichissimo Latino, sia di neutro plurale, del quale si servirono i
Latini per comprender l'appartenenze della cosa, della quale parlavano, v.g.\
delle cose appartenenti alle navi dicevono Navalia; alla Cacina Popinalia, e molt'altri,
è corrotto da noi con l'aggiunta della lettera G.

\item[IN crimen lesa] È delitto di lesa Maestà cacciare una Regina del
suo Regno.

\item[FAR' una pedina] Si dice Fare una pedina a uno allora che procurando questo
  tale di conseguire cosa di suo gusto, ed essendo vicino a ottenerla, un'altro, a
  cui haveva confidato tal negozio; gliela leva su. Viene dal giuoco di Scacchi, dicendosi
  propriamente: Dare scacco di pedina.

In oltre, chi è pratico del giuoco di Scacchi sa, che quando s'è perduta la
Regina, si procura di racquistarla con far' arrivare una pedina al posto dove
stava la Regina dell'avversario al principio del giuoco, e così intendere, che Celidora
priva del Regno conveniva, che sotto nome di Pedina tornasse a ricuperarlo,
se voleva esser detta Regina.

Si potrebbe anche dire, che il nostro Poeta seguitando il costume che habbiamo
di chiamar Dame le Signore grandi, e Pedine le donne d'infima plebe, habbia inteso,
che Bertinella, togliendo il Regno a Celidora, l'habbia cavata del nome di
Dama, per haverla ridotta in grado miserabile, le habbia fatto meritare il nome
di Pedina; ma l'esser' il nome, di Celidora nel terzo caso, e non nel secondo, o
nel quarto; fa languire questa riflessione.
\end{description}

\section{Stanza II}
\begin{ottave}
\flagverse{2}O Musa, che ti metti al sol di state\\
Sopr' un palo a cantar con si gran lena, \\
Che d'ogn'intorno assordi le brigate,\\
E finalmenre scappi per a schiena;\\
S'anch'io sopr'alle picche dell'armate\\
Volto a Febo con te venga in iscena,\\
Acciò ch'io possa correr questa Lancia,\\
Dammi la voce, e grattami la pancia.
\end{ottave}

Quest'Ottava ha poco bisogno di spiegazione vedendosi chiaro, che il Poeta,
invoca per sua Musa la Cicala, e così dà a conoscere, che egli vuole scrivere affatto
mostrando, che per fare una composizione come egli ha in animo,
e per descrivere una guerra qual fu quella di Malmantile, gli basta haver
chiacchiere.

Si potrebbe anche dire, che il Poeta sapendo che non si trova, che le Muse habbiano
dato mai alcuno aiuto effettivo, ed evidente, come dette la Cicala a Eunomo
Locrense Suonatore nella disputa, che hebbe con Aristono, supplendo con
la voce al mancamento della corda strappata, come si legge in Strabone lib. 6.
voglia, come fece Eunomo, far più capitale della Cicala, che d'altre Muse:
E può anch'essere, che egli invochi la Cicala, perché stimi più nobili delle Muse le
Cicale per esser queste più riguardevoli, come nate avanti alle Muse (secondo la
favolosa credulità de' Gentili) d'Huomini, li quali per lo gran gusto, che hebbero
del cantare, furono in cicale convertiti, come si cava da Celio Rodigino lib.\
17.\ cap.\ 6.\ le cui parole sono queste: \textit{Fertur enim hosce homines fuisse ante Musas;
natis deinde Musis, cantumque monstrato, illorum nomnullos voluptare cantus usque adeo
delinitos fuisse, ut canentes cibum, potumque negligerent, imprudenterque perirent; ex
quibus deinde cicadarum genuss sit propagatum, ec,}

Dice il Doni nella sua Zucca, che tutti li Poeti hanno la loro Cicala, e che
questa serva loro per Fama publicando le loro Poesie, onde il nostro Poeta seguitando
l'opinione del Doni invoca la Cicala destinata al suo servizio, perché gli
faccia questo di pubblicare le sue Poesie.

\begin{description}
\item[PALO] Pertica, Bastone di legno, che si mette per sostegno alle viti, ed altri
  arbuscelli simili.

\item[LENA] Significa quello, che i Latini dicono \textit{respiratio}, cioè quieto, e
  tranquillo
  anelito, il che mentre è nell'Huomo, egli si mantiene senza difficultà, nelle
  forze: ma la troppa fatica di corpo, o di mente spesso fa affannare tal Lena,
  però che uno, che s'eserciti assai senza posarsi, appunto come fa la Cicala col
  suo cantare senza riposo, si dice Haver gran Lena.

  Dante Inf.\ C.\ 1.\ \begin{verse}E come quel che con lena affannata, ec.\end{verse}

  Al Canto 24.\ \begin{verse}La Lena m'era dal polmon si sì smunta, ec.\end{verse}

  Vedi sotto C.\ 4.\ stanza 6.

  Varchi\footnote{Benedetto Varchi (Firenze, 19 marzo 1503 – Firenze, 18 dicembre 1565), umanista, scrittore e storico.}  stor.\ lib.\ 5. \begin{verse}Essendo egli di pochissimo spirito,
    e di gentilissima Lena\end{verse}

  Franco Sacc.\ Nov.\ 127.  \begin{verse}Alla fine perdendo questi ciechi
      la Lena per essersi molto bene mazzicati, ec.\end{verse}

  I Latini con la voce \textit{Vis}, e con la voce \textit{robur} esprimevano questa Lena.

\item[VENIRE in scena] Comparire in pubblico,  vedi sotto C.\ 4.\ stan.\ 6.

\item[CORRER questa lancia] Tirar' a fine quest'Opera.

\item[GRATTAMI la pancia] Col grattare il corpo alla Cicala, ti fa che ella canti,
  la Cicala a grattare il corpo a lui, acciò che'egli canti. Quand'altri
  sa qualcosa, ed è duro a manifestarla, si dice; \textit{Grattagli la pancia, che egli
    canterà},
  cioè interrogalo, ed esaminalo bene, che egli dirà tutto quello, che tu
vuoi; si che il senso di questo detto \textit{Grattare il corpo a uno}, è Incitarlo a discorrere.
Vedi sotto C.\ 2.\ stan.\ 8.

\end{description}

\section{Stanza III \& IV}

\begin{ottave}
\flagverse{3}Alcun forse dirà ch'io non so cica, \\
E ch'io farei 'l meglio a starmi zitto, \\
Suo danno; innanezi pur, chi vuol dir dica, \\
Fo io per questo qualche gran delitto? \\
S'io dirò male, il Ciel la benedica; \\
A chi non piace, mi rincari il fitto: \\
Non so, se se la sanno questi sciocchi,\\
Ch'ognun può far della sua pasta gnocchi.

\flagverse{4}Mi basta sol che Vostra Altezza accetta\\
D'onorarmi d'udir questa mia storia\\
Scritta così come la penna getta,\\
Per fuggir l'ozio, e non per cercar gloria;\\
Se non le gusta, quando l'avrà letta\\
Tornerà bene il farne una baldoria:\\
Che le daranno almen qualche diletto\\
Le Monachine, quando vanno a letto.
\end{ottave}

In queste due Ottave l'Autore piglia a difender se medesimo dalle male lingue,
e mostra, che poco gl'importa l'esser lodato, o biasimato in questa sua Opera, e
che, non essendo obbligato a veruno, vuol soddisfare a se medesimo, ed al suo capriccio;
e però dice: \textit{S'io dirò male il Ciel la benedica}, che significa Vadia il negozio,
come e' vuole, che non m'importa. E seguita \textit{A chi non piace mi rincari il
  fitto}, volendo mostrare, che per non essere obbligato a render conto ad alcuno delle
sue azioni, non teme d'esser ripreso, o di ricever danno; e soggiugne: \textit{Ognun
  può far della sua pasta gnocchi}, cioè ogni huomo libero puo fare del suo, a suo modo.
Conchiude in somma, che egli vuol dar gusto a se medesimo, e lasciar dire
chi vuol dire, bastandogli, che S.A., cioè il Sereniss.\ Principe Card.\ Leopoldo de'
Medici, a cui dedica l'Opera, si contenti di riceverla, e d'udirla, \textit{scritta come
  la penna getta}, cioè composta non ad altro fine, che di spassarsi; ne si cura
d'acquistar gloria per tal composizione, anzi supplica S.A.\ ad abbruciarla quando
l'haverà letta, che riceverà qualche gusto dal veder' \textit{andare a letto le Monachine}. E
per Monachine intende quello, che intendono i nostri Fanciullini, cioè quelle piccole
scintille, che, nell'incenerirsi la carta, a poco a poco si spengono, e facendo
un certo moto, pare che si dileguino, sembrando tante Monache, le quali col
loro lume in mano scorrano per il dormentorio, andando a letto.

\begin{description}
\item[CICA] Niente. Anzi vuoi dire (se si può) Manco di niente, dicendosi in
  diminuzione \textit{Poco, niente, Cica}. Viene dal latino \textit{Cicum}, che vuol dir Quel velo,
  che si trova nelle melagrane per divisione de' suoi granelli, che per esser così sottile,
  e di niun valore, serviva ai Latuini per dimostrare la poca stima, che facevano
  d'una cola, dicendo: \textit{Ne Cicum quidem dederim}, ec. e noi diciamo in questo
  proposito \textit{lappola, lisca, ec.}
\item[ZITTO] Quieto. \textit{Stare zitto} vuol dire Non parlare, Viene dal cenno. \textit{Zi},
che si suol fare, quando senza parlare si vuol fare intendere a uno, o più, che
quietino, come facevano ancora i Latini, che per accennare ad altri, che si
quietasse profferivano le due consonanti S.T.

\item[GNOCCO] È una specie di Pane gramolato, mescolato con anici; e questa
pasta fra le nobili è la più vile: Il proverbio \textit{Ognun può far della sua pasta gnocchi}
significa ognuno ha il libero arbitrio, ed esprime quello, che i Latini dissero:
\textit{Unusquisque in re sua moderator, \& arbiter, ec}.
\item[SUO danno] Non m'importa, Non stimo questa cosa. E diremmo; \textit{io so che la
tal cosa m'è nociva, suo danno io la voglio non ostante ec}, Esprime Io la voglio, se
bene mi può nuocere, ec. Vedi sotto C.\ 4.\ stan.\ 26.\ al yermine \textit{In ogni modo}.
\item[RINCARARE] Accrescere il prezzo. E questo detto Rincarare il fitto usato in
  questi termini significa: Non fo stima, ne temo le male lingue, perché non mi
  possono far danno.
\item[FITTO] Pigione, Canone, cioè Quel danaro, che si paga annualmente per
una Casa, o Podere, o altri beni, che si posleggono d' altri con pagargit un tan-
'to lvanno. Locarionis canones,
\item[BALDORIA] Fiamma accesa in materia secca, e rara, come paglia, e simili,
  che presto s'accende, e presto finisce; detta forse \textit{Baldoria} da Baldore, O
  Baldanza, che vuol dire Allegrezza: quindi \textit{Lieta} significa poi Baldoria, come vedremo
  sotto C.\ 2.\ stan.\ 56. Diciamo anche \textit{Far baldoria}, quando altri spende
  allegramente, e si da bel tempo consumando tutto il suo havere; il qual detto vien
  forse da un religioso costume, che era fra gli Antichi, che delle vivande sagre
non si lasciassero avanzi, ma quello che avanzava s'abbruciasse; il qual rito
si cava dai Precetti di Moisè in proposito del'Agnello Pasquale. Questa specie
di Sacrifizio fu usata anche da i Gentili Romani, e la dicevano: \textit{Proterviam
  facere}, che vuol dire Far'una fiamma, o baldoria; E pigliavano ancor'essi \textit{proterviam
facere} nel senso detto sopra di consumare, e mandar male il suo, come si cava
da Macrob. lib.\ 6.\ Saturnal.\ 2., dove si legge, che Catone motteggiando un tal
Albidio, che haveva consumato tutto il suo havere, e solo gli era rimasta una Casa,
la quale gli abbruciò, disse: \textit{Proterviam fecit, propterea quod ea, quae comesse
non potuerit, quasi combussisset.}
\end{description}

\section{Stanza V}

\begin{ottave}
\flagverse{5}Offerta gliel'haveo già, lo confesso,\\
Ma sommen'anche poi morse le mani, \\
Perch'il filo non va ne ben, ne presso, \\
E versi v'è ch'il Ciel ne scampi i cani:\\
Ma poi ch'ella la vuole, e io l'ho promesso\\
Non vo mandarla più d'oggi in domani,\\
Che chi promette, e poi non ta mantiene,\\
Si sa, l'anima sua non va mai bene.
\end{ottave}

Mostra l'Autore, che la convenienza per haver'egli promessa a S.A.R.' quest'Opera,
l'obbliga a mantenere la parola, quantunque egli conosca, che non sia cosa
d'esser veduta da S.A.R., e per questo s'è morso le mani, cioè pentito
grandemente d'haverla promessa, perché vede che la tesstura dell'opera non sta
ne bene, ne presso a bene, e vi son versi \textit{che il Ciel ne scampi i cani}, cioè così
stroppiati, che tanto male non ne vorrebbe vedere, ne meno a un cane.
Ed il verbo \textit{scampare} attivo, come è in questo luogo, significa Liberare. Ma
conchiude poi, che già che S.A.R.\ la vuole, non sta bene che egli la mandi più in
lunga da hoggi in domani, ma è dovere osservar la promessa; al che fare s'accigne
adesso, non solo per questa convenienza, ma ancora per il timore della pena meritata
da colui \textit{che promette, e non mantiene} la quale è che \textit{L'anima sua non va
  mai bene}. Sentenza usatissima da i nostri Fanciulli; e viene dall'antico, poiché
l'usavano ancora i fanciulli greci secondo il Monosino Fior. Ital. linguae lib. 3.9.109.
dove cava dal Greco le seguenti parole: \textit{Nos autem dicimus id, quod solent pueri:
quae recte data sunt non licere rursus eripi}: Che suona lo stesso che: \textit{Chi da, e ritoglie
il Diavol lo ricoglie}, che vale lo stesso che: \textit{Chi promette, e non  mantiene L'anima
sua non va mai bene}.

\section{Stanza VI}

\begin{ottave}
\flagverse{6}Ma che? si come ad un che sempre ingolla\\
Del ben di Dio, e trinca del migliore, \\
Il vin di Brozzi, un pane, e una cipolla\\
Talor per uno scherzo tocca il cuore; \\
Così la vostr'Idea di già satolla\\
Di quei libron, che van per la maggiore,\\
Fore potrà, sentendosi svogliata,\\
Far di quest'anche qualche corpacciata.
\end{ottave}

Ripiglia animo il Poeta; e spera che S.A.R.\ sia per contentarsi di leggere
questa sua Opera, se non per altro, almeno per distrarsi dagli studj più serij, e
considera, che si come colui, che e solito far vita lautissima, havea talvolta gusto di
mangiare un pane, e una cipolla; e ber vino da niente, così chi è solito legger
libri più sensati, talora averà non poco gusto a legger libri di baie, e facezie.

\begin{description}
\item[INGOLLARE] Vuol dit Mangiat presto, ed inghiottire senza mafticare.:
Stuf più il verbo Ingoiaresefendo il verb» ingodare usato nel Contado, se bene &
forse meno barbaro che #ygoiare, perché e più proffimo alla sua latina origines,
che è la proposizione Zr, e guia, ed in questa appunto 'inghiortita la leteera 'L. se-
condo la stretta' pronanzia comane Toscana, € mutato in I ferrato 50 confonante
si dice comunemente Ingoiare: Così dice il sig.\ Francefeo Maria Bellini.

\item[DEL ben di Dio] Delle più buone vivande; che i Latini dicevano \textit{Jovis nectar},
e noi diciamo \textit{latte di gallina}, che vedremo in questo Cant, stanza'64.

\item[TRINCARE] Bere assai; Voce che viene dal Tedesco; e diciamo \textit{Trinca}, o
  \textit{Trincone}, uno che beva sregolatamente; Vedi sotto Cant. 7, stanza 1.

\item[DEL migliore] S'intende quel che vuol dire, ma il senso più astruso puro
  Fiorentino è, che gli Osti di Firenze vendono sempre due specie di vino rosso, uno
di poco prezzo, che lo dicono Vino di sotto, o di bassa, perché viene da' luoghi
di sotto a Firenze, dove fanno Vini deboli, e leggieri; e l'altro di maggior prezzo,
che lo dicono vino di sopra, o de migliore; e di questo intende il Poeta.

\item[TOCCARE il cuore] Dar soddisfazione intera: Quando altri mangia con gusto,
  e si conosce, che quella vivanda gli fa pro, diciamo: \textit{Le tal vivanda gli ha
  toccato il cuore}.

\item[SATOLLO] Sazio, Ripieno. Dal latino \textit{satur}. Qui vale per Stracco di leggere.

\item[BROZZI] È un di quei luoghi sotto Firenze, dove nasce il detto vino debole.
Vedi sotto in questo Cant.\ stanza 47.

\item[PER scherzo] Intendi non per fame, o sete; ma per stravizio, o tornagusto.
E' voce Tedesca, e là pur suona lo stesso

\item[ANDAR per la maggiore] Esser della prima ' fle: Traslato da i Magitteati
dell Arti della Città di Firenze, delle quali 5: ena: 'che sono
Giudici, e Notai; Cambio; Mer 5 Lana 5 Seta; Speziati, i
se paflano a Cavalleria, Alere Minori, che art eenan *) Quota eee
non paflano, 0: ra non pafiavano aca 'quando 'in
ze si dice, // ale va per: 'delle:




maggiore ss Sete 'una

are Arti, ed' della cap sw claffe, Come s' intende ie laogo's
\item[SVOGLIATO] Senz' appetito: senza puto di mungeyo

eae opie..

\item[FAR una corpacciata] Saziarsi. Empier benissimo il corpo =
corpacciata, gu altri legge, ree ° fa altra cosa'
te'fa una volta.
\end{description}

\section{Stanza VII \& VIII}

\begin{ottave}
\flagverse{7}Già dalle guerre le Provincie stanche,  \\
Non sol più non venivano a battaglia,   \\
Ma fur banditi gli archi, e l'armi bianche, \\
Ed etiam il portar un fil di paglia \\
Vedeansi i bravi acculattar le panche \\
E sol menar le man fu la tovaglia;      \\
Quando Marte dal Ciel fa capolino,    \\
Come il topo dall'orcio, al marzolino

\flagverse{8}Che d'haverlo non v'è ne via ne modo,  \\
Se dentr'ad un mar d'olio non si tuffa, \\
E reputa il padron degno d'un nodo,   \\
Che lo lascia indurire, e far la muffa. \\
Così Marte, che vede l'armi a un chiodo \\
Tutt'appiccate malamente sbuffa,      \\
Che metter non vi possa su le zampe    \\
E che la ruggin v'habbia a far le stampe.
\end{ottave}

Il Poeta dà principio all'Opera, descrivendo lo stato, in che erano le cose del
Mondo, e dice, che tutto era in pace, ne si usava più arme di sorta alcuna; ed i
bravi, ed huomini armigeri acculattavano le panche, cioè Stavano oziosi, e menavano
le mani solo in su la tovaglia, che viene a dire Attendevano solamente a mangiare.
E qui scherza con l'equivoco del menar le mani, che vuol dir Combattere, vedi
sotto C. 10, stan. 2,, e trattandosi del mangiare vuol dir Mangiare assai, e presto,
vedi sotto C. 6, stan. 46. Marte però s'adira, che non s'adoprino più l'armi.
L'Autore assomiglia Marte quando s'affaccia al Cielo, ad un topo, che s'affacci
alla bocca d'un'orcio pieno di cacio, e d'olio, che s'adira per veder tal cacio
abbandonato dal padrone, e di non poterlo arrivare, se egli non entra in detto
olio.

\begin{description}
\item[ARMI bianche] Spada, e pugnale, ed eggi altra sorta d'Armi, a distinzion
  dell'Armi da fuoco.
\item[PANCA] Arnese noto fatto di legname per uso di sedere, e possono starvi
più in una volta; detto da i Latini \textit{subsellium}, e viene dalla voce Latina
\textit{Planca}, che significa Assamenti, e tavolati piani.

\item[ACCULATTARE le panche] Significa (siccome habbiam detto) Starsene
senza far cosa alcuna, e spensierato. Ter.\ in An.\ disse \textit{Oscitantes} di coloro, che
stanno in questa maniera, quasi dica. \textit{Stanno sbavigliando}, che noi diciamo:
\textit{Starsene con le mani in mano}, o \textit{Fare a tu me gli hai}, o \textit{Dondelarsela}, e simili, che
tutti ci servono per Per esprimere \textit{Perder' il tempo in vano}, ed è quello che i Latini
dissero; \textit{Manum habere sub pallio}.

\item[TOVAGLIA] Quel panno lino che si distende, sopr'alla mensa da i Latini
  detto Mantile, e noi l'habbiamo forse da Toralia, che erano i panni, che
  \textit{circumponebantur in toris discumbentium}, ec.
\item[MENAR le mani] Quando è posto assolutamente, vuol dire Far quistione,
E con aggiunta, vuol dire Affrettarsi al lavoro, che sara aggiunto; e si usa dire
Mena le mani a correre, d'uno che corra assai, Mena le mani a leggere d'uno
che legga presto, ed in somma d'ogni Operazione humana, ancorche non fatta
con le mani, e qui vuol dire Mangiar prsto, ed il simile sotto C. 6. stan. 46.
\item[FAR capolino] Guardar di soppiatto. Quand'altri procura di vedere, senza
esser veduto, suole asconder la persona dietro a un muro, o altro, e cavar fuori
tanta testa, che l'occhio scuopra quel ch'ei vuol vedere, e questo si dice \textit{Far
capolino}. Sotto C. 2. stan. 78. dice \textit{Fa pan da Montui}, che è lo stesso.
\item[ORCIO] Vaso grande di terra, per uso di conferuar' olio, vino, ed altri
  liquori, si come per conservarvi, ed ugnervi il cacio.

\item[MARZOLINO] Specie di cacio tondo fatto a piramide, e'col manico nel
fondo dalla parte più grossa; chiamato Marzolino,perché si comincia a farlo nel.
mese di Marzo, ed € il miglior cacio, che si faccia nei nostri pacfi. E nel
presente luogo, se ben dice \textit{Marzolino}, intende ogni sorte di cacio.

\item[DEGNO di nodo] Cioè merita la forca per l'errore che fa a non mangiare
quel Marzolino, lasciandolo andar male.

\item[TUTTE l'armi appiccate a un chiodo] Dicendosi: tale ha appiccate l'armi
all'arpione, al chiodo, s'intende: Il tale ha abbandonate l'armi, cioè Lasciato
d'essere armigero. Ciò viene dagli antichi gladiatori, i quali quando dal popolo,
col porger loro una bacchetta erano assoluti, e liberati dal far più il gladiatore,
solevano dedicar l'armi ad Ercole, appiccandole nel di lui Tempio, come
ci mostra Orazio lib. 1. ep. 1.
\begin{verse}
\makebox[12em]\dotfill} Veianius armis.
Herculis ad postem fixis, latet abditus agro.
\end{verse}
Et lib. 3, ode 26.
\begin{verse}
Vixi puellis nuper iduneus,
Et militavi, non sine gloria;
Nunc arma, defunttumqnue belle
Barbiton hic paries habebit.
\end{verse}

\item[SBVFFARE] Dar segni d'ira. Sbuffare è quel soffiare, che suol fare per lo
più uno, che sia in collera, Traslato forse da i cavalli: E si dice Sbuffare,
quando altri adirato si duole, e in uno stesso tempo minaccia con parole.

Dante Inferno C. 18,: Ud.,
\begin{verse}
\backspace Quindi sentiamo gente che si nicchia
Nell'altra bolgia, e che col muso sbuffi,
E se medesima con le palme picchia,
\end{verse}

Viene da Buffo specie di soffio, che vedremo sotto C. 3. stan. 57.

\item[CHE la ruggin v'habbia a far le Stampe] La ruggine, rodendo il ferro, vi fa
  sopra certe impressioni simili a quelle, le quali con acqua forte si fanno nel rame
  per Stampare, e pero le dice Stampe.
\end{description}

\section{Stanza IX}
\begin{ottave}
\flagverse{9}Sbircia di qua di là per le Cittadi, \\
Ne altre guerre, o gran Campion discerne, \\
Che battagie di giuoco a carte, e a dadi, \\
E Stomachi d'Orlandi alle taverne, \\
Si volta, e dà un'occhiata ne' contadi \\
Che già nutrivan nimicizie ererne     \\
E non vede i Villan far più quistione    \\
In fuor che con la roba del Padrone.
\end{ottave}

Marte, riguardando bene per le Città, vede solamente guerre di giuoco, e
gente valorosa, e brava nel mangiare. Voltatosi poi ne i Contadi, che eran già
pieni di nimicizie, e risse, vede, che dai Villani non si fa altra guerra, che
che fanno con la roba del Padrone.

\begin{description}
\item[SBIRCIA] Sbirciare vuol propriamente dire Socchindere gli occhi, acciò che
  l'angolo della vista, fatto più acuto, possa osservare con più facilità una
  minuzia, Se bene si piglia ancora per Guardar per banda, a fine di non essere
  osservato, come fanno spesso gli amanti; movendo la pupilla alla volta dell'angolo
  esterno dell'occhio, con quel muscolo, che per tal cagione da' Medici si chiama
  amatorio; E questo \textit{Sbirciare}, o \textit{Bircio}, e \textit{Sbircio} ha forse l'etimologia dal Latino
  \textit{hirquus}, che Vuol dir l'angolo dell'Occhio. Verg. Egl. 3. \textit{Transversa tuentibus
  hirquis}; la qual parola vuol Servio, che abbia origine da \textit{hircus}, essendo che
  questi animali infuriati per la libidine guardano obliquamente, e torto le capre,
  che amano.

  È pero vero, che il nome Bircio, o Sbircio si dice non solamente di chi ha gli
  occhi scompagnati, ma generalmente ancora di chi ha qualsivoglia sorta d'imperfezione
  agli occhi, essendo noi in questo non differenti da i Latini, appresso
  i quali se ben \textit{luscus} vuol propriamente dire Vno, che ha solo un'occhio, come
  si vede in Giovenale Sat.\ X.\ che parlando di Annibale dice: \textit{Cum Getula ducem
  gestaret bellua lufcum}; che il Petrar.\ disse: \textit{Sour' un grande elefante un Duce losco}.
  E Cic. de orat. \textit{Hic luscus familiaris mens Catus Sentius} :

  \textit{Lusciosus} vuol dire
  Quello, che ha la vista corta, come si può dedurre da Varrone lib. 8. disciplin.\

  \textit{Strabo} Quello che ha gli occhi torti, da noi chiamato Guercio. Cic.\ 1.\ de
  Nat.\ Deor.\ \textit{Et quos insigni nota Strabones, aut Paetos esse arbitramur};che Paetus significa
Uno che abbia gli occhi leggiermente abbassati, che noi lo diremmo Luschetto.
Porfirione annot.\ ad Horat.\ lib.\ 1.\ Sermonum Sat.\ 3. \textit{Paeti proprie dicuntur, quorum
  huc, atque illuc oculi velociter vertuntur}, ec,

Coclites Quelli, che son nati ciechi
da un'occhio. Plau.\ in Cur.\ \textit{Unocule salve; ex Coclitum prosapia te esse arbitror ec}.

Lucini; Quelli che hanno ambedue gli occhi piccoli Plin. lib. 10, cap. 37. \textit{Ab
ijsdem qui alter lumine orbi nascerentur coclites vocant, \& quibus parvi utrisque ocelli,
lucini vocantur}, ec.

Nyctilopes Quelli di vista così debole, che non veggono se non
quando splende il Sole. Plin. lib. 8. cap. 50. \textit{Si caprinum iecur vescantur, restitui
  vespertinam aciem his, quos Nyctilopas vocant}, ec.

Non ostante, appresso molti queste
differenze si confondono, pigliando spesso l'uno per l'altro; così appresso noi
si confondono i nomi Guercio, Bircio, Orbo, Lusco, e simili, ec, accomodandogli
spesso a qualsivoglia imperfezione degli occhi, come vedremo sotto in questo
Cant. stan.\ 37\ che Orbo, vuol dire Affatto cieco, cioè Oculis Orbatus, e stan. 66.
vuol dire Lusco.

\item[CHE a battaglia di giuoco, e a carte, e a dadi] Non vede nel Mondo altre risse
che di giuoco, nel quale egli non ha che fare. Perché torna non affatto fuor di
proposito una riflessione sopra la voce latina \textit{Alea}, e la voce \textit{Talus}: si contenti
il Lettore, che io faccia un poca di digressione. Sono molti de' moderni Latini,
che si servono della parola \textit{Alea} per intendere la carta da giuocare; ma forse
pigliano equivoco, se vogliamo credere a Polidoro Vergilio, al Meursio, al
Soutero, a Raffaello Volterrano, ed altri, che hanno trattato de i giuochi antichi,
i quali la chiamano \textit{charta lusoria}; \& \textit{Alea} chiamano Ogni specie di giuoco
di Fortuna, se forse quei tali non volessero sostenere la loro opinione con dire,
che quando la voce alea è presa in genere generalissimo; allora significhi ogni specie
di giuoco di fortuna: ma presa in genere speciale, significhi la carta da giuocarel
nel che mi rimetto alla prudenza del Saggio Lettore. So bene che fino il
giuoco de' noccioli era detto Alea, come si cava da Marziale.
\begin{verse}
  Alea parva nuces, \& non damnosa videtur,
Saepe tamen pueris abstulit illa nates, ec.\end{verse}
Altra volta la presero per Fortuna, secondo Livio lib.\ 37.\ che parlando d'Antioco
il quale volle più tosto guerra, che pace co i Romani per le dure condizioni,
che gli offerivano, dices, \textit{Nihil ea moverunt regem, tutam fore belli aleam
ratum; quando perinde ac victo iam sibi leges dicerentur},ec, E Colum.\ in Praefat.\ lib.\
1.\ dice \textit{Maris, \& negotiationis alea}. Pare che errino ancora, coloro, che
pigliano la voce \textit{Talus} per intendere il Dado, perché veramente il Dado si dice tessera,
e \textit{talus} vuol dire il Tallone, cioè Quel'osso, che è sopra il calcagno del piede,
donde si dice veste talare, la veste lunga infino a i piedi; E questa voce \textit{Talus},
trattandosi di strumento per giuocare e l'astragalo Greco, che è quello che i nostri
ragazzi chiamano aliosso; ma questo è forse minore equivoco, poiché tal'osso
finalmente viene usato in cambio di dado, servendosi per numeri di quelle macchie,
o segni, che naturalmente sono in dett'osso, come più largamente diremo
sotto C.\ 8.\ stan.\ 69. Gioviano Pontano nel suo Dialogo di Caronte distingue questo
aliosso dal dado, dicendo; \textit{Atque ego numquam talis lusi, nec tesseris}. Lo stesso fa
il Gellio lib.\ 1.\ Cap.\ 20.\ che dice \textit{Talus cubus non est, cubus .n, est figura ex omni latere quadrata, tessera sex lateribus constat}. Marziale pure nel lib.14.ep, 15. mostra
tal differenza, dicendo: \textit{Non sum talorum numero par, tessera dum sit Maior quam
talis alea saepe mihi} ec. Tal differenza si deduce anche da Cicer.\ lib.\ 2.\ de Divinat.\
\textit{Quid .n. fors est ? idem propemodum, quod micare, quod talos iacere, quod tesseras}.

E tanto basti per rispondere a quei che biasimarono l'haver noi messo per esplicare
le presenti due voci Carte, e dadi il latino Charta Luforia, \& Tessera, che per altro
non importava al caso nostro questa digressione, e torna più a proposito il sapere,
che tali giuochi tanto di dadi, quanto di carte, dice Platone in Pedro, che
fussero inventati da un tal Theut Dio de gli Egizzj. \textit{Daemoni autem ipsi nomen Theut,
hunc primum numerum, \& computationem numerorum, Geometriam, Astronomiam,
talorum denique, alearumque ludos audivi}, ec. Raffaello Volterrano, e Celio Calcag.
de Ludo Talario, e Tesserario, dicono, che questi giuochi fussero trovati da Palamede
nel campo Greco sotto Troia, e però gli domanda, \textit{Palamedis alea}; si come
fa il Soutero; Ma Isidoro lib. 8. Originum, concorda bensì, che havessero origine
nel detto Campo Greco, ma da un Soldato, che havea nome Alea, e che
da lui il giuoco prese il nome d'alea, Herodoto lib. 1. riportato da Polid. Verg. lib.
2. cap. 13. dice, che l'inventassero i Lidi per le cause che si diranao sotto C. 6.
stan. 34.

\item[STOMACHI d'Orlando] Dicendosi: \textit{Il tale è buono stomaco}, o vero. \textit{È uno
stomaco d'Orlando}, ec. s'intende, il tale è coraggioso, e bravo; Qui pero valendosi
dell'equivoco di \textit{Buono stomaco}, che vuol dir \textit{Gran mangiatore}, intende Gente
brava nei mangiare,:

\item[DAR un'occhiata] Intendiamo: Guardar' alla sfuggita.
\item[FAR quistione] Far contesa, disputa, rissa; ma dicendosi assolutamente
  senz' aggiunta: Far quistione, s'intende: Combatter con le spade, ec.
\end{description}

\section{Stanza X}
\begin{ottave}
\flagverse{10}Ond'ei ch'in testa quell' umor s'è fitto,\\
Che l'huom si scrocchi pur giusta sua possa;\\
Senza picchiar, ne altro, giu sconfitto.\\
L'uscio a Bellona manda in una scossa;\\
Niun fiata perciò, non sent'un zitto,\\
Perch'ella dorme, e appunto è in su la grossa,\\
Poiché la sera havea la buona donna\\
Cenato fuora, e preso un po di nonna.
\end{ottave}

Marte risolve d'unirsi con la sorella Bellona a fine di mettere scompigli nel
mondo, e andato a trovarla, la vede in letto a dormire briaca ancora della sera
passata.

\begin{description}
\item[UMORE]
Questa voce, che per altro significa materia umida, e liquida, e
parlandosi d'animali significa Flemma, collera, malinconia, ec, viene spesso da
noi presa per Fantasia, o pensiero come nel presente luogo, che dicendo: S'è
fisso quel'umore in testa, vuol dire ha stabilito, ha fermato il pensiero, ha risoluto.
La pigliamo ancora per Desiderio. Bartolomeo Cerretani stor. nell'anno
1502. dice: \textit{Si senti che l'umore di Piero de' Medici, di tornare in Firenze non
era spento, ec, Ma Papa Alessandro, desiderando fare il Valentino suo figliuolo Signore
di Toscana, si volle anch'egli valere di questo umore de' Medici}, ec, Diciamo Bell'umore
Uno che ha fantasie graziose. Vedi sotto in questo C., stan. 58. Si dice Far' il
bell'umore Uano, che vuol far da bravo, e da ardito. \textit{Il tale volle fare il bell'umore
col salire sopra quell'albero, e cascò}, ec. Donde habbiamo Umorista, che significa
Uno di cervello instabile, ed inquieto. \textit{Haver grand'umore} vuol dir' esser superbo,
ed haver gran pretensioni di se medesimo.

\item[CHE l'huom si crocchi] Che l'huomo si perquota. Il verbo crocchiare del quale
  ci serviamo alle volte per il verbo cicalare; come si vedrà in questo Cant.\ stan.\
  4., o C.\ 3.\ stan.\ 3., e che vuol' anche dire Quel suono, che fa un vaso di terra
  cotta fesso, come Pentola, o altro vaso simile; ci serve anche nel significato di
  dar busse, e questo intende nel presente luogo: propriamente Quel cantare, che
  fa la gallina chioccia, quando ha i pulcini.
\item[GIUSTA sua possa] Per quanto egli può; Frase antica latina \textit{iuxta meum posse}, ec.
\item[FIATARE] Significa parlare. Vedi sotto C, 6. stan, 12.
\item[È in su la grossa] È in sul buono del dormire. Dorme profondamente. Traslato
  dal baco da seta, il quale quando dorme per la 3.\ volta, che è il suo dormire
  più gagliardo; si dice: \textit{È nella grossa}.
\item[NON sente un zitto] Non sente verun rumore, cioè ne pur' un di quei cenni,
  \textit{zi} che dicemmo sopra questo Cant.\ stan\ 3. Il Varchi stor.\ lib.\ 6.\ dice: \textit{Con avvertir che ne cenni, ne zitti, ne atti brutti si facessero}.
\item[CENAR fuora] Intendiamo Cenar in conversazione fuor di casa propria.
\item[PIGLIAR la monna] Imbriacarsi. Ci sono più specie di briachi, fra' quali son
  quelli, che si dicono cotti monne, che son coloro, che per lo troppo vino bevuto,
  danno nelle buffonerie, e saltano, e chiacchierano spropositatamente, facendo
  mille altre pazzie, e poi s'addormentano; e si dicono ancora \textit{Cotti nonne},
  o \textit{pigliar la monna}. E questo è il nome generico, il quale comprende tutte le
  specie di briachi, di che parleremo sotto C. 2. stan.\ 69. In questo C.\ stan.\ 77.\
  dice. \textit{S'imbriacaron come tante monne} dal che deduci, che si può dire:
  \textit{Prese la nonna}, e \textit{prese la monna}, che in ambedue maniere ha
  lo stesso significato,

\end{description}

\section{Stanza XI}
\begin{ottave}
  \flagverse{11}Le scale corre lesto com'un gatto,\\
  poi dal salotto in camera trapassa,\\
  E vede sopr'a un letto mal rifatto\\
  ch'ell'è rinvolta in una materassa;\\
  Sta cheto cheto, e con due man dipiatto\\
  Batte la spada sopr'ad una cassa,\\
  La qual s'aperse, ed ivi vistevi drento\\
  Robe manesche, a tutte fece vento.
  \end{ottave}


Bellona non ostante ogni romore,  che faccia Marte, non si sveglia, ed egi
ruba alcune cose, le quali trovò ivi in una cassa. Esprime il Poeta il genio furibondo
di Marte, e la natura del Soldato, che è sempre dedita al rubare.
Esprime ancora la briachezza di Bellona, dicendo, che ella dormiva \textit{rinvolta
nelle materasse sopra un letto mal rifatto}; il che mostra, che quando Bellona andò a
dormire era in grado, che non sapeva distinguere le coperte dalle materasse.

\begin{description}
\item[LESTO come un gatto] La voce lesto, che viene dal Latino \textit{sublestus}, che vuol
dir Leggieri, frivolo, e debole, appresso di noi significa Pronto, agile, e destro;
E questa comparazione \textit{Lesto, come un gatto}; da noi è usatissima per esprimere la
grande agilità d'uno. Vedi sotto C.\ 2.\ stan.\ 35.

\item[SALOTTO] Intendiamo Piccola sala, cioè un ricetto prima che s'entri nella
principal sala.

\item[MATERASSA] Arnese da letto, quello che si dice in Latino Greco Anaclinterium
  a distinzione di \textit{culcitra plumea}, che noi diciamo \textit{Coltrice}; essendo la
  materassa un sacco largo quanto è il letto, e ripieno di lana, ed impuntito nel
  mezzo.

\item[Chero cheto] Quietissimo. Nota che la replica d'una stessa voce, appresso di noi,
  ha la forza del superlativo.

\item[DI piatto] Cioè per lo largo della spada.

\item[MANESCO] Uno che sia, diciamo noi, delle mani, cioè pranto, ed inclinato
  a perguotere, ed no che sia inclinato a rubare. Qui però vuol dire Robe
  atte, e comode a esser portate via. Roba manesca intendiamo Roba, che ci sia
  prenta, e comoda a valersene.

\item[FECE vento a tutte] Portò via ogni cosa. Rubò ogni cosa. Che questo intendiamo
  quando diciamo; Far vento a una cosa.
\end{description}

\section{Stanza XII}

\begin{ottave}
  \flagverse{12}Ma non fa sì, che la sorella sbuchi,\\
Di modo ch'ei la chiama, e li fa fretta;\\
La solletica, e dice: Ovvia fuor bruchi:\\
Lo Spedalingo vuol rifar le letta,\\
S'allunga, e si rivolta, come i ciuchi:\\
Ella ch'ancor del vin ha la spranghetta,\\
E, fatto un chiocciolin su l'altro lato,\\
Le vien di nuovo l'asino legato.
\end{ottave}

Con tutto che Marte faccia ogni diligenza perché Bellona si svegli, solleticandola,
e gridando, che è hora di levarsi, non trova modo di farla destare; anzi,
essendosi ella alquanto sollevata per causa di que' romori, s'allunga, e si rivolta,
poi si rannicchia, e di nuovo si addormenta, perché il vino la tiene oppressa.
Ed è bella espressione d'uno, che dorma con gran gusto, e volentieri; perché
questo tale, sentendo strepito, si risveglia alquanto, e facendo, per lo più, le
operazioni, e moti descritti nella presente ottava, seguita a dormire.
\begin{description}
\item[SBUCARE] Intende svegliarsi, e levarsi; Uscir da quella buca, la quale si fa
  nelle materasse col peso della persona.
\item[FAR fretta a uno] S'intende Stimolar' uno a far presto.
\item[SOLLETICARE] Stuzzicare leggiermente uno in alcuna di quelle parti del
  corpo, le quali, toccate così, incitano a ridere, Viene dal verbo \textit{Sollicito},
  \textit{sollicitas}, quanto val per Tentare.
\item[FUOR bruchi] Dalla voce Bruco habbiamo il verbo \textit{Brucare}, che vuol dir Levar
  le foglie a gli alberi, e per metafora vuol dire \textit{Andar via}, onde quando diciamo:
  \textit{Il tale sbrucò}, intendiamo, Andò via, ed, il simile intendiamo nel dire
  \textit{Fuor bruchi}, cioè andate via. Luigi Pulci Bec.\ \textit{Ognun brucò, che,
    l'era la tregenda}, Onde qui s'intende \textit{Escì, dal letto}. Detto, usatissimo in
  questo proposito.
\item[LO Spedalingo vuol rifar le letta] Questo detto significa, È hora tarda, e da
  levarsi dal letto; ed ha origine da gli spedali, ne i quali si raccettano i Pellegrini;
  dove, quando è hora di levarsi, e che i poveri, e i Pellegrini seguitano a stare
  nel letto, lo Spedalingo, cioè il Guardiano, o Sopracciò dello Spedale suole
  per svegliargli gridare: \textit{S'hanno a rifar le letta}.
\item[CIUCO] Asino giovane, ò poledro. Forse dal latino \textit{Cicur}, che par che
  voglia dire Bestia addomefticata, ed agevole.
\item[HA la spranghetta] o \textit{stanghetta}. Quel duolo di testa, ed inquietudine, che
  si sente la mattina, quando, la sera avanti s'è troppo bevuto, e poco quella notte
  dormito, per lo qual duolo pare, che il capo sia sprangato, o legato con spranghetta,
  o stanghetta. Che così si chiama ogni verga di ferro, o regolo di legno,
  che unisca due materiali insieme; come si dice porta sprangata, una porta, in
  mezzo alle di cui imposte sia conficcato a traverso un regolo di legno, affinché
  dette imposte non si possano aprire, E stanghetta pure si dice quel ferro, che serra
  insieme l'imposte de gli usci, il quale s'apre, e serra con la chiave facendolo
  scorrere in certi anelli, come il chiavistello, dal quale è differente, perché il
  chiavistello non si può, o almeno non è in uso aprir con la chiave.
\item[FATTO un chiocciolino] Cioè Rannicchiatasi, o raggruppatasi quasi in figura
  di chiocciola, come sono quelle focattole, o stiacciate, che fanno le nostre donne
  per i Bambini, le quali chiamano chiocciolini, perché gli fanno a figura di chiocciola;
  e come vediamo, che nel dormire fa per lo più il cane.
\item[LEGAR l'asino] Addormentarsi, Detto, che viene da i Villani vetturali, che
  essendo per strada soprappresi dal sonno, legano l'asino, e s'addormentano nel
  luogo, dove gli piglia il sonno. E col dire: \textit{Il tale ha legato} senza l'aggiunta
  \textit{d'asino}, s'intende; \textit{Il tale s'è addormentato}. Francho Sacchetti\footnote{Franco Sacchetti (Ragusa di Dalmazia, 1332 – San Miniato, 1400), letterato. Visse principalmente nella Firenze del XIV secolo. È oggi ricordato soprattutto per la sua raccolta Trecentonovelle. } nov.\ 171.\ dice:
  \textit{Essendo Gulfo entrato nel letto, quando fu per legar l'asino, il compagno cominciò col
    mantaco a soffiare}. Bocc.\ gior.\ 4.\ nov.\ 9.\ \textit{Di che la donna spaventata,
    per svegliarlo cominciò a prenderlo per lo naso, e tirarlo per la barba, ma tutto
    era nulla, perché egli haveva a buona caviglia legato l'asino}. ec.
\end{description}
\section{Stanza XIII}
\begin{ottave}
  \flagverse{13}O corna disse il Re degli Smargiaffi,\\
E intanto le coperte havendo preso\\
Le ne tira lontan cinquanta passi,\\
Ma in terra anch' egli si trovò disteso;\\
O che per la gran furia egli inciampassi,\\
O ch' elle fusson di soverchio peso,\\
Basta ch' ei batte il ceffo, e che gli torna\\
In testa la bestemmia delle corna.
\end{ottave}

Incollerito Marte leva le coperte a Bellona, e le butta in terra, dove cascò
ancor' egli, e batté il capo, e si fece un bernoccolo, o tumore nella testa, quali
tumoretti da molti per scherzo son chiamati Corna per esser nel luogo, dove nascono
le corna a gli animali.

\begin{description}
\item[DICE bestemmia delle corna] e' piglia la voce bestemmia non nel suo proprio
significato di attribuire, o levare empiamente alla Divinità quello che se le conviene,
ma nel significato di maladizione, o imprecazione, come è preso
tal volta nella nostra Toscana, ed in altre parti d' Italia, e specialmente in
Napoli, dove \textit{iastemiare} è inteso comunemente  per Maledire. E qui dicendo:
\textit{Torna in testa a lui la bestemmia delle corna} intende: Quell'imprecazione che
haveva fatta, venne addosso a lui, e viene a dire Si fece un corno nella testa, cioè
uno di quei bernoccoli, o tumoretti, che per esser nella testa scherzosamente si
chiamano Corna.
\item[SMARGIASSO] Huomo bravo. Armigero. Ma però l'usiamo per derisione,
  e per intendere Un'huomo fuor dei limiti della ragione, e della prudenza,
  ed uno di quei petulanti, e minacciosi, che pretendono di spaventar ognuno con
  la lor pretesa bravura.
\item[CINQUANTA passi] Lontano assai, Detto iperbolico usato spesso anche in
  piccolissime distanze.
\item[INCIAMPARE] Dar co i piedi in qualcosa nel camminare: è il Latino \textit{offendere}.

\item[SOVERCHIO peso] Peso grande, peso fuor di misura, Petr.\ Canz.\ 17.
\begin{verse}
Altri ch'io stesso, e il desiar soverchio,
\end{verse}
E certo che le coperte eran di grandissimo peso, perché Bellona si serviva per
coperte delle materasse, come s'è detto sopra.
\item[BASTA] Termine conclusivo usatissimo da Noi, quasi diciamo: \textit{È a sufficienza},
  e si dice anche \textit{A bastanza}, dal verbo \textit{Bastare}, che è il latino
  \textit{sufficit}. I Latini dicevano \textit{Bat, Sat est}. Plau.\ nel Penuo si servì
  della voce \textit{Bat}, senza aggiunta di \textit{Sat est}, ed i Giosatori di esso
  dicono: \textit{Bat vox, qua utimur cum quempiam iubemus tacere}.
\item[CEFFO] Vuol dir propriamente il muso del cane, del porco, o simili, ma
  si dice anche del Viso, o faccia dell'huomo, ma per lo più in derisione, e per
  intendere una faccia brutta, e mal fatta. Vedi sotto C.\ 4.\ stan.\ 10.
\end{description}

\section{Stanza XIV.}

\begin{ottave}
  \flagverse{14}Ella svegliata allora escì del Nidio,\\
E dicendo ch'in ciò gli sta il dovere,\\
E ch'ei non ha ne garbo, ne mitidio,\\
Non si può dalle risa ritenete,\\
Cosa ch' a Marte diede gran fastidio,\\
Ma perch'ei non vuol darlo a divedere,\\
Si rizza, e froda il colpo che gli duole,\\
Poi dice che vuol dirle due parole.
\end{ottave}

Per l'insolenze di Marte, Bellona finalmente si sveglia, e dà la burla a Marte
perché egli è cascato, e Marte fingendo non sentire la percossa si rizza, e dice a
Bellona, che vuole alquanto discorrerle.
\begin{description}
\item[USCIR del nidio] Uscir del letto: quale chiama Nidio per la similitudine, che
  ha nelle materasse quel luogo, dove s'è dormito, col Nidio, entro al quale covano
  gli uccelli..
\item[GLI fra il dovere] Gli è intervenuto quel ch' ei meritava. \textit{Dovere},
  \textit{giusto}, e \textit{giustizia}, sono sinonimi.
\item[NON ha garbo] Non ha accuratezza. Per intelligenza di questa parola \textit{Garbo}
è da sapere che erano in Firenze due luoghi principali,  dove già si fabbricavano
panni lani d'ogni sorta, uno detto S.\ Martino da una Chiesa, che quivi è dedicata
a detto Santo, e l'altro si domandava il \textit{Garbo}, quali nomi di strade si
conservano fino al presente. Nel detto il Garbo si fabbricavano le pannine di
tutta perfezione; e quelle che si fabbricavano in S.\ Martino erano sempre
d'inferiore condizione, onde venne in uso il dire: La tal cosa è del Garbo,
volendo denotare la perfezione di quella tal cosa. E dalle robe venne alle persone, e si
cominciò a dire: Huomo di garbo, huomo, che ha garbo, ec. intendendo d'uno
che operi bene, e con accuratezza. Cosi dice il Monosino Flor.\ It.\ linguae
alla parola Garbo. E noi diciamo ancora in questo Senso: \textit{Non ha ne Garbo, ne
S.\ Martino},
\item[MITIDIO] Giudizio; ordine; Parola corrotta da metodo.
\item[NON si può dalle risa ritenere] Non può far di non ridere.
\item[DAR fastidio] Dar noia; dar disgusto.
\item[NON vuol darlo a divedere] Non vuol farlo conoscere. L'aggiunta della particella,
  di, al verbo vedere s'usa solo in questo caso per esprimere, far capace,
  o render bene informato.
\item[FRODARE] È noto il suo significato, venendo dal Latino \textit{fraudare}, che vuol
  dire Ingannare; Ma noi lo pigliamo ancora per Occultare, o non manifestare,
  come è preso nel presente luogo; ed è traslato da quel \textit{frodare}:, che vuol dire
  Nascondere qualche roba alla porta della Città, o alla Dogana per fraudare la
  Gabella con il non pagarla, che si dice \textit{Far frodo} Vedi sotto C.\ 6.\ stan.\ 28.
\end{description}

\section{Stanza XV}
\begin{ottave}
  \flagverse{15}Dì pur: la Dea risponde, ch'io ascolto;\\
Hai tu finito ancora? Ovvia, dì presto:\\
Ma prima di quei panni fa un rinvolto,\\
E gettalo in sul letto ch' io mi vesto.\\
Quello non sol; ma quanto haveva tolto\\
Di quella cassa, ei rende, e mette in sesto,\\
E postosi a seder su la predella,\\
Con gravità dipoi così favella.
\end{ottave}

Descrive assai bene il genio inquieto, e furibondo di Bellona, mentre mostra
l'ardenza, con la quale ella stimola Marte a dir quanto gli occorra, interrogandolo
se egli ha finito, quando sa che non ha ancora cominciato, ed in uno stesso
tempo gli comanda, che rimetta le coperte in sul letto: Ubbidisce Marte, e
s'accomoda a sedere per dar principio al discorso, che sentiremo.

\begin{description}
\item[FAR un rinvolto] È lo stesso che Affardellare, abballinare, o far balle,

\item[METTERE in sesto] Accomodare; aggiustare. E in Latino \textit{aptare}, e da
\textit{Metter in sesto} diciamo \textit{Rassettare}, o \textit{metter in assetto}. Varchi Storia libro 8.
\textit{Havendovi dì, e notte lavorato per mettere il Salone in assetto}. L'Autore della
storia de' Piacevoli, e Piattelli lib.\ 2.\ dice \textit{Non pareva possibile distender la
  fila,allogare i lasci, e dar sesto al tutto, e pure ben tosto si vedde mettere ogni
  cosa in assetto}.

\item[PREDELLA] Qui intende Quella seggiola fatta a cassetta, la quale si tiene
vicina al letto per l'occorrenze del corpo; che per altro questa voce \textit{predella} ha
molti significati, chiamandosi predella ancora quell'arnese sopra il quale si posano
le donne quando partoriscono; Predella si dice quello scaglione di legno, sopra
il quale sta il Sacerdote quando celebra Messa; e quella seggiola dove siede
il Sacerdote quando in Chiesa ascolta le Confessioni detta altrimenti Confessionale.
Predella pure è detta quella parte della briglia, che si tiene in mano, come si
cava dal Landino esposizione a Dante nel Purg.\ C.\ 6.
\begin{verse}
  Guarda com'essa fiera è fatta fella,
  Per non esser corretta dagli sproni,
  Poi che ponesti man alla predella.
\end{verse}
\item[FAVELLARE] S'intende Ragionare, discorrere; Strettamente vuol dire
  Parlar con ordine, e massime quando è contrapposto agli verbi Cicalare, gracchiare,
  chiacchierare, e simili. \textit{Il tale non chiacchierava ne cicalava, ma favellava
    e discorreva}. Cioè parlava con fondamento, regolatamente, e seriamente.
\end{description}

\section{Stanza XVI}
\begin{ottave}
  \flagverse{16}Sirocchia, male nuove; poi ch' in Terra\\
Veggiam ch'all'armi più nessuno attende,\\
Onde il nostro mestiero, idest la guerra,\\
Che sta in sul taglio, non fa più faccende;\\
Sai, che la Morte ne molesta, e serra,\\
Che la sua stregua anch'ella ne pretende,\\
E se non se li dà soddisfazione,\\
La ci farà marcir n' una prigione.
\end{ottave}

Marte in questo suo discorso mostra alla sorella la necessità, che ambedue hanno
che si faccia guerra, per il bisogno, che hanno di guadagnare almen tanto da
pagare il dazio alla morte, acciò che ella non gli faccia metter prigioni, e quivi
morire, se non le pagano detto tributo.

\begin{description}
\item[SIROCCHIA] Sorella. Parola Fiorentina; ma oggi poco in uso. Dante nel
  Purg. C.-4, e Canto 21.; 4
  \begin{verse}
    Che se Pigrizia fusse sua Sirocchia, ec.
    L'anima sua ch'è tua, e mia sirocchia, ec.
  \end{verse}
\item[STA in sul taglio] Due specie di Mercanti di drappi, o diciamo Setaiuoli sono
  in Firenze. I primi fabbricano drappi per mandargli fuor di Stato, o per vendergli
  a merciai di Firenze a pezze intere; i secondi fabbricano, e vendono in
  Firenze a braccia, o diciamo a minuto, e questi si chiamano \textit{Setaiuoli, che stanno
    in sul taglio}, Marte dice alla Sorella, che la loro arte, che sta in sul taglio non
  lavora più, ed il Poeta scherza con l'equivoco di Tagliar drappi, e tagliar huomini;
  e che di questa lor'Arte di taglio vuole la morte, che essi paghino il dazio, dando
  alla medesima tanti morti l'anno; onde se la guerra non lavora, non possono
  pagar questo tributo.
\item[SERRARE] O far serra a uno, Affrettare, stimolare, violentare uno. Vedi sotto
C. 9. stanza 13.
\item[STREGUA] Intendi quel dazio, che devono alla morte. La voce stregua, che
vuol dir Porzione dovuta, vien forse dal Latino strena, che significa mancia.
Varchi Stor. lib. 10, \textit{In alcune cose vanno quei tali rispettati, ma in molte più devono
andare alla medesima stregua, e ragguaglio degli altri}, ec.
\item[DAR soddisfazione]. Soddisfare, Adempire ogni sorte di convenienza, o di
  debito che uno habbia con un'altro: Ma strettamente s'intende Pagar quel danaro,
  del quale uno è debitore.
\item[CI fara marcir n'una prigione] Ci fara star tanto in carcere, che noi vi moriremo
  di stento; V'infradiceremo.
\end{description}

\section{Stanza XVII}
\begin{ottave}
  \flagverse{17}Bisogna qui pigliar qualche partito,\\
Se noi non vogliam' ir nella malora\\
Ed un ce n'è ch' è buono arcisquisito,\\
Qual'è, che si risvegli Celidora\\
C'ha dato un tuffo nelle scimunito,\\
Mentre di Malmantil si trova fuora,\\
E passandola sempre in piagnistei,\\
Pigra si sta, come non tocchi a lei.
\end{ottave}

Seguitando Marte il suo discorso, propone che si ponga in animo a Celidora
già cacciata da Malmantile, di risolversi alla vendetta, e così far nascere la
guerra; per rimediare a' lor bisogni.
\begin{description}
\item[PIGLIAR partito], Risolversi a pigliar qualche modo di rimediare.
\item[ANDAR nella malora] Intendi Andare in prigione per questo debito. E il
  latino \textit{In malam Crucem abire}.
\item[ARCISQUISITO] A buono, diciamo in augumento; buono, più buono, buonissimo,
  ed in luogo di buonissimo diciamo anche squisito, facendolo superlativo
  di buono e cosi non, dovrebbe patire agumento; tuttavia si dice Squisito, più
  squisito, squisitissimo, o arcisquisito, imitando forse i Latini, che da \textit{optimus}
  superlativo di \textit{bonus}, hanno, \textit{optimissimus}, Si trova anche nelli Scrittori antichi della lingua nostra.
  L'accrescimento al superlativo, il Bocc.\ nov.\ 19.\ dice \textit{Così santissima donna}, E nov.\ 60. \textit{Così ottimo parlatore}, ec, Gio.\ Villani lib. 12, cap, 104, dice: \textit{Rimase in più pessimo stato}, ed al lib. 7, cap. 100, \textit{La quale era della maggiore di S.\ Gio.\ ed era molto
    fortissima} e cap. 101. \textit{A pié delle Montagne dette Pirre molto altissime}, e questo
  Autore l'usò sempre, che gli venne occasione d'esprimer un gran superlativo; ma
  da i moderni non pare, che sia molto usato, e con ragione, perché con l'aggiunta
  di molto, così, più, o simili, il superlativo che ha la natura del suo nome, riceve
  moderazione, e più tosto scema, e torna indietro della sua essenza;; e così volendo
  dire, che una Montagna sia altissima con Aggiungervi il \textit{molto}, \textit{così}, o \textit{assai}, si
  viene a dire che la Montagna sia alquanto alta, e non in tutto alta, o altissima
  ricevendo in questa maniera il superlativo limitazione, e non agumento. Salustio
  disse \textit{multo pulcherrimam} quando riporta il discorso fatto da Catone Uticense
  a Cesare in proposito della congiura di Catilina.

  La particella arci, che vien dal Greco archos,    che significa Superiore, s'usa
  anche da i moderni pen esprimere (se si, può) di là o più su del superlativo, ed il
  nostro Poeta l'usa anche nel Cant, 12. stan. 34    ma appresso di me anche questa
  particella arci aggiunta al superiativo fa l'effetto    che l'altre dette sopra di
  moderare, e non accrescere, ec.
\item[RISVEGLIARE] Non dal sonno, ma dalla Pigrizia.
\item[HA dato un tuffo nello scimunito] Ha fatta una azione da sciocca, e da stolta,
  Metaforico da i vintori, i quali volendo, che la seta, o altro, pigli il colore,
  l'intingono nel bagno di quel tal colore tante volte, quante par loro che serva.
  E questo dicono \textit{Dare un tuffo}, o \textit{più tuffi}. E dicendoti \textit{Il tale ha datoun tuffo nello scimunito}
  S'intende che quel tale habbia fatta un'azione da scimunito, non però
  che egli sia del tutto scimunito. Questo termine \textit{dar' un tuffo} può forse anche
  venire da coloro, che affogano, i quali prima di morire tornano alla superficie
  dell'acqua due, o tre volte, il che diciamo: \textit{Dare i tuffi}; e che, s'intenda è prossimo
  essere del tutto scimunito, come è vicino a esser del tutto morto colui, che da i
  tuffi nell'acqua. La voce \textit{scimunito} credo che sia composta di due dizioni, cioè
  \textit{scemo}, (che vuol dir' uno che habbia manco giudizio di quel che si conviene) e
  unito, e venga a dire \textit{unitamente scemo}, cioè scemo ugualmente, o del pari, o in
  tutte le parti a un modo, che conchiude affatto sciocco, e insensato.
\item[Si trova fuor di Malmantile] È priva di Malmantile perché le è stato tolto da
  Bertinella, o se ne trova effettivamente fuora. Diciamo: \textit{Io son fuora di tal
    pensiero} per intendere: io non ho più questo pensiero.
\item[PAGNISTEI] Singulti, solpiri mescolati con pianti. Voce da donnicciuole,
Vedi sotto C. 2 stan. 23.
\item[COME non tocchi a lei] Cioè come l'interesse in questo negozio non sia, o
  S'aspetti a lei, ma ad un'altro.

\end{description}
\section{Stanza XVIII}
\begin{ottave}
  \flagverse{18}Ma come quella, pare a me, che aspetta,\\
Che le piovano in bocca le lasagne,\\
Senza pensar un' Iota alla vendetta\\
La sua disgrazia maledice, e piagne; \\
Hor mentre ch'ella in arme non si metta\\
Per racquistar lo scettro, e sue campagne;\\
Molto male per noi andra il negozio,\\
Che muoiam di mattana,e crepiam d'ozio.
\end{ottave}

Marte pone in considerazione a Bellona, che se non trovano il modo di far risolver
Celidora ad armar gente per racquistar il suo stato di Malmantile, il negozio
andra mal per loro, che non hanno faccende.

\begin{description}
\item[CHE le piovano in bocca te lasagne] Vuol del bene, e non vuol durar fatica a
domandarlo: come per esempio uno che ha gran fame, si lascia più tosto finire
da quella, che chiedere il cibo dovutogli, ma aspetta che il cibo gli corra in
bocca da se. Costume di Cuccagna.
\item[LASAGNE] Specie di pasta tirata, ed assottigliata come un velo.
\item[UN Iota] Piccola lettera dell'Alfabeto Greco, e si piglia per esprimer il \textit{niente}.
\item[MORIR di mattana] Morir di malinconia; quasi dica: È così grande la malinconia,
  che mi nasce dall'ozio, che mi fa divenir matto, e morire. Viene da
  \textit{macto}, \textit{mactas}, e forse prima si diceva: Perire di morte mattana, ec. che era una
  occasione speciale, che si faceva da gli Aruspicj nell'immolar le Vittime, le quali
  sventravano vive, e così morivano a poco a poco crudelmente; La onde i Latini
  aggiungono sempre a questo verbo la parola morte o supplicio, come si vede
  in Cicerone, che dice \textit{Morte mactavit}, \& \textit{supplicio mactari}.
\item[CREPARE] Questo verbo Crepare, che significa Quando un legname si spacca,
  o fende da per se: significa ancora Morire a stento, ed in questo senso è preso
  nel presente luogo, o forse e preso nel senso d'Allentare, che vuol dire Quando
  a uno per la soverchia fatica cascano gl'intestini, e voglia Ironicamente
  parlando, che s'intenda; è così grande la fatica, che duriamo, che ci fa allentare.
\end{description}
\section{Stanza XIX \& XX}
\begin{ottave}
  \flagverse{19}Chi sa? forse costei se ne sta cheta\\
Perch' ella vede esser legata corta,\\
Che s'ell'havesse un dì gente, e moneta\\
Tu la vedresti uscir di gatta morta;\\
Ma qui Baldon farà dall'A alla zeta\\
(So quel chi dico, quando dico torta)\\
Ritrova tu costei, sta seco in tuono,\\
Che quant'al resto anch'io farò di buono.

\flagverse{20}Vattene dunque, e in abito di mago,\\
Dopo il formar gran circoli, e figure\\
Conchiadi, e dille che tu sei presago,\\
Che presto finiran le sue sciagure,\\
E quel tuo corazzon pelle di drago\\
Imbottito d'insulti e di bravure\\
Mettile in dosso, che vedrala poi\\
Far lo spavaido più, che tu non vuoi.
\end{ottave}

Marte facendo riflessione che se Celidora havesse chi la soccorresse, ed
aiutasse, ella si muoverebbe a procurare di racquistare lo stato, perciò ordina a
Bellona, che la vadia a trovare, e la rincuori con dirle, che presto riavera il suo
stato, e le metta addosso l'usbergo incantato.

\begin{description}
\item[CHI sa?] Questo termine significa che la tal cosa può essere, o non può essere,
  quasi dica: Chi è colui, che sa di sicuro, che la cosa sia, o non sia così?

\item[È legata corta] Cioè non ha forze bastanti a far quello, che ella  vorrebbe.
  Traslato dal cavallo, asino, mulo, o simili, i quali quando son fieri, e bizzarri si
  legano dovungue si sia con la cavezza corta, affinché non offendano chi va loro
  d'attorno.

\item[VSCIR di gatta morta] Farsi vivo, dimostrarsi fiero. \textit{Far la gatta morta} vuol
dir Simulare. Il Lalli En. Trav, Cant. 2. stan. 12. parlando dsl Cavallo Troiano
dice:
\begin{verse}
  e stanno i Greci ascosti in questo legno,
  e v'attendono a far la gatta morta.
\end{verse}
I Latini dissero \textit{lepus dormiens}, E noi diciamo anche \textit{far la gatta di
  Masino}. Vedi sotto C. 7. stan. 69.

\item[FARÀ dall'A alla zeta] Farà puntualmente quanto bisogna. Farà il tutto.
  L'A, e la Z. sono il principio, e il fine del nostro Abbicci, onde con questo
  termine intendiamo \textit{Sarà fatto il tutto}, come appunto appresso i Greci Alpha, \&
Omega; che è lo stesso che \textit{Capite ad calcem} de' Latini.
\item[SO quel ch'io, dico, quando dico torta] So benissimo come sta questo negozio,
  Esprime \textit{m'intend'io}, Il Pulci nel suo Morgante fa dire a quello scellerato di
  Margutte.
  \begin{verse}
    Io credo nella torta, e nel Tortello:
    Sò quel, ch'io dico, quand'io dico torta,
  \end{verse}
E vuol dire M'intend'io, quel ch'io voglio dire, e quello ch'io intenda per
torta.
\item[STA seco in tuono] Sta seco unita; Va d'accordo seco. Traslato dalla Musica.
\item[FARÒ di buono] Negozierò da vero. Farò quanto bisogna. Quando uno
giuoca di danari si dice \textit{Far di buono}, che vuol poi dire Operar con attenzione; il
chee non si fa quando non si giuoca di buono, non ponendosi attenzione quando
si giuoca da burla.
\item[ABITO da Mago] Non hanno i Maghi abito particolare, ma il Poeta se lo
  figura in quella guisa, che ha veduto in commedia, cioè veste lunga, gran barba,
  e la verga in mano. E \textit{Mago} è voce Persiana, che significa \textit{Sapiens}, e
  quello che i Greci dicono Filosofo. E di questa sorte Filofofi furono quelli Magi, che
  andarono ad adorare Giesù bambino. Ma perché Zoroaste fu anch'egli uno di
  tali Filosofi detti Magi, e secondo Plin. lib. 30. cap. 1. fu inventore dell'Arte
  dell'incantare, però tal arte è detta Magia, e coloro, che l'esercitano son chiamati
  Magi. Tasso Gerusal. C, 10. stan, 29.
  \begin{verse}
    Son detto Ismeno, i Siri appellan Mago,
    Ma che dell'arti incognite son vago.
  \end{verse}
  E perché quest'arte, secondo Polid. Verg. lib. 1. cap. 33. è di sei specie, cioè
  Negromanzia, Geomanzia, Chiromanzia, Piromanzia, Aeromanzia, Hydromanzia,
  però questi Magi son detti ancora Negromanti, ec, Vedi sotto Cant, 2. stan. 5.
\item[SCIAGURA] Questa voce  parrebbe che significasse Scelleraggine, o
  Sciagurataggine si piglia da noi per Disgrazia.  Boccaccio Novella 36. \textit{La
    storia del mio ardire, e della mia sciagura vi racconti} E N. 43. \textit{E della sua sciagura
  dolendosi}. I Latini pure dicevano \textit{Scelus}, e se ne servivano nello stesso modo, che
 facciamo noi per intendere Disgrazia. Plaut. in Capt. \textit{Maior potitus hostium est,
   quod hoc est scelus? Quasi in orbitatem liberos produxerim}. Ter. in Eun. \textit{Neque quemquam
 esse ego hominem arbitror, cui magis bonae Felicitates omnes adversae sint. P. Quid
hoc est sceleris?} Il medesimo significato ha la voce latina — che a noi ha la
voce Sciagurato.

\item[CORAZZONE] Corazza grande, Armatura di petto, e schiene; dal latino
\textit{Thorax}, si dice anche Petto a botta, perché è a figura d'una botta, o perché si
presume, che regga a una botta d'archibuso.

\item[IMBOTTITO] Ripieno, e trapuntato non di cotone, o altro simile, \textit{ma d'insulti
  e di bravure}, che vuol'intendere Incantato, come vedremo appresso nell'ottava 27.

\item[SPAVALDO] Huomo avventato; Huomo inconsiderato, Dal latino \textit{supervalidus}
  Soverchiamente ardito, e quasi temerario, e tutto impertinente.
\end{description}

\section{Stanza XXI \& XXII}
\begin{ottave}
  \flagverse{21}Bellona c'ha il medesimo capriccio\\
Di far braciuole, va col sarrocchino.\\
Con il bordone, e un bel barbon posticcio,\\
Sembrando un venerabil pellegrino;\\
E fatto di parole un gran pastriccio\\
Esser dicendo astrologo, e indovino,\\
Che vien di quel discosto più lontano\\
La ventura le fa sopr'alla mano;

\flagverse{22}Ove doppo mostrato ogni accidente\\
Di tutta la sue vita pel passato,\\
Seggiunge, che per via d'un suo parente\\
In breve tempo riavrà lo stato;\\
Però si metta in arme, ch'un presente\\
Le fa d'um panceron, che ancorché usato\\
Ripara i colpi ben per eccellenza,\\
E poi piglia da lei grata licenza
\end{ottave}

Bellona va a trovar Celidora, e fingendosi Astrologo, le dice molte cose occorsele
per il passato, per accreditarsi; poi le predice, che fra poco tempo ella
riavrà il suo Stato, però si metta in armi; e le dona la corazza incantata, e si
parte.
\begin{description}
  \item[CAPRICCIO] E Pensiero, fantasia, volontà., come intende anche sotto C. 6,
stan. 101. E per altro \textit{capriccio} significa quello, che i Latini dicono \textit{orrore}, che è
quando i peli s'arricciano; il che segue o per lo freddo, o per qualche subito spavento,
o ne i casi di febbre, come s'intende sotto C. 6. stan. 14. e C, 20. stan. 2.
Donde poi habbiamo il verbo \textit{accapricciare}, che vuol dire Havere spavento. Dante
Inf. C22.
\begin{verse}
  Lo viddi, ed anche il cor men' accapriccia
\end{verse}

\item[BRACIUOLE] Si dicono quelle fette, o strisce di carne di porco, o d'altro
  animale, che sono così tagliate per cuocerle sopr'alla bracie, e però dette
  \textit{braciuole}, Ma qui intende fette d'huomini, e vuol dire che Bellona havea la
  medesima volontà di far guerra, che haveva Marte.
\item[SARROCCHINO] È un collarone di cuoio, il quale adattato al collo cuopre
tutte le spalle, e buona parte delle braccia, e petto a foggia di Manteiio, ed è
usato da i Pellegrini, che vanno a piede a visitare i luoghi santi; E questi tali
sono da noi chiamati Pellegrini corrottamente da Peregrini; la qual voce è latina, e
ritiene appresso di noi gli stessi significati di singolare, e grazioso, ed anco di
forestiero, \textit{Peregrinus in domo patris mei}, Petrarca Can. 12.
\begin{verse}
  Mosse una Peliegrina il mio cor vano
\end{verse}
Et intende, che una graziosa, e bella donna mosse il suo cuore. E la detta voce
Sarrocchino credo, che venga da San Rocco il quale portava forse questa parte
d'abito, quando andò peregrinando il Mondo.
\item[BORDONO] È nome particolare, e proprio di quel bastone, che portano i
Pellegrini.
\item[PASTRICCIO] Massa confusa di diverse robe. Qui vuol dire quantità di
  parole mal' ordinate.
\item[DAL discosto più lontano] Più lontano della lontananza stessa, come diremmo:
Vero più del vero, o della stessa verità.
\item[FAR la ventura] Strolagare. Sono alcune donnicciuole originarie d'Egitto,
le quali in Toscana vengono il più delle volte di Sicilia, e si chiamano Zingane.
Queste, dando a creder d'esser perite di chiromanzia per buscar denari, vanno
considerando i lineamenti delle mani alle persone, e palesano (dicono esse) le cose
passate, e predicono le future: E perché discorrono artifiziosamente con certi
lor generali sempre di bene; esse chiamano, ed anche da tutti noi vien detta questa
operazione; \textit{Far la ventura}, o \textit{la buona ventura}.
\item[PARENTE] Intendiamo ogni sorte d'affini, o consanguinei in qualsisia grado;
  così è inteso nel presente luogo, che vuol dire Baldone cugino di Celidora.
Così l'intese Dante nel Parad. C.6., e il Petr. Son. 191. E se bene strettamente
vuol dire il genitore, venendo dal latino \textit{Parens}, e usato da noi in tal senso
assai di rado, e forse non mai fuor che nel numero del più, come l'uso Dante Inf.
Cant. 1.
\begin{verse}
  \makebox[8em]{\dotfill} Homo già fui
E li parenti miei furon Lombardi,
Mantovani per Patria ambi dui,
\end{verse}
Ed il Petr. Canz. 29.:
\begin{verse}
Madre benigna, e pia,
Che cuopri l'uno, e l'altro mio parente,
\end{verse}
\item[PANCERONE] Intende quella gran corazza detta sopra in questo C. stan 20.
\item[ANCORCHÉ usato] Adoperato, Vecchio, Antico.
\item[PIGLIAR buona licenza] Pigliar commiato, Licenziarsi da uno per andarsene.
  E quell'epiteto di \textit{buona}, o \textit{grata} s'aggiugne per esprimere, che quel
  tale parte con buona grazia dell'altro, e con il di lui consenso, e non forzato,
  o scacciato.
\end{description}

\section{Stanza XXIII \& XXIV}
\begin{ottave}
  \flagverse{23}Già il termine d'un anno era trascorso,\\
  Che Celidora havea perduto il Regno;\\
  Quando non pur le spiacque il caso occorso,\\
  Ma volle un tratto ancor mostrarne segno,\\
  Perciò richiesto ai convicin soccorso,\\
  Che un piacer fatto non havrian col pegno,\\
  e tenevano il lor tanto in rispiarmo,\\
  ch'egli era giusto, come leccar marmo.

  \flagverse{24}Fece spallucce a Calcinaia, e a Signa,\\
  Ma la pania al suo solito non tenne,\\
  Perché terren non v'era da por vigna;\\
  Calò nel piano, e ad Arno se ne venne,\\
  Ove Baldon facea nella Sardigna\\
  Vele spiegare, e inalberar' antenne,\\
  Fermato havendo lì come buon sito\\
  D'armati legni un numero infinito.
\end{ottave}

L'Autore toccando la finta storia della perdita dello Stato di Celidora, dice,
che era già passato un'anno, quando la medesima cominciò ad haver pensiero di
ricuperarlo, e per ciò fare, richiese soccorso a diversi vicini, ma senza frutto; la
onde si risolvé di venirsene verso Firenze, e trovò in su la riva d'Arno in un
luogo detto Sardigna Baldone con una buona armata.
\begin{description}
\item[UN tratto] Una volta, La voce tratto ha molti significati dicendosi \textit{tratti di
  fune}, Quello scarrucolamento, che si da a i delinquenti' nel martirio della corda.
  \textit{Tirar i tratti}, diciamo Quelli ultimi moti, che fanno i moribondi nell'esalar lo
  spirito. \textit{Tratto} si dice in vece di estratto, cavato, o dedotto, ec, \textit{Tratto} val per
  distanza, dicendoli tratto di tempo, tratto di via, e simili, \textit{Tratto} di cortesia per
  Atto di cortesia, \textit{Tratto} per maniera, Ed in questo luogo significa Finalmente, ed è
  il latino \textit{tandem aliquando}.

\item[VN piacer fatto non haurian col pegno] Ss lees Vacs chemo
a yeruno:, eziam se li fufle,daco-il pegno ia'mano.

\item[TENER il.suo in rispiarme] Venere il fao ate,econ riguada s > taal dicono
r isparmio 2 ri/parmiare,,

\item[GIVSTO] Questo termine significa Perl? appunto.

\item[ERA come leccar marmo] Bravana ogni ist per. appanto;come vaniti
Tecear' il marmo.

\item[FECE spallucce] Si raccomandd., Questo detto seas dai poverelli', che per.
muovere a compaflione in domandando-l'elemosina, fanno tutte le fmorfic,» e»
gclti, che fanno, e podiono, e fra gli altriil pia comune i Fare /pallucce 5 Siok
StringerJe spalle alla, volea del collo..

\item[LA pania non tenne] Non fece cosa di buono, cioè non hebbe ainto da, colora,
dy quali lo sperava; intendendosi con questo dettato, che quel tale, che fu richie-
flo, von adempi il volere di chi lo richiefe; cite diciamo ancora: Vax.ha trovata
appicco. 1 Latiai pure ia questo proposico ditiero Evannerunt infidia, Rania inten-
diamo il visco, col quale si pigliano gli uccelliy, B diciamo dom tenere quando 5
© per il molle; o per altro la pania non appicca, ne li prendgne son) ae LA

\item[AL suo solito] Secondo il suo costume, Dice al suo solito per dimostrare, che
in quei paesi era da sperar poco bene al solito, perché mon v'è terreno da por viene,
che vuol dire: Non è da far fondamento so da sperare da loro favore alcuno, e
scherza con l'equivoco del parre vigne, perché veramente quei paesi non hanno
terreni buoni a poryite-viti..

\item[CALO' nel piano] Scefe;ne) plano, perch', Calvinaia., e Signaifono ncaa
cOllinette vicin¢ ad Arno...

\item[OVE Baldon facea nella Sardigna] L'Autore, che vuol sempre stare in su le
  burle, e servirsi dello scherzo degli equivoci, fa che Celidora trovi Baldone nella
  Sardigna; e pare che voglia dire l'isola di Sardigna, ed intende di un luogo fuori
  delle mura di Firenze in fa la riva d'Arno, così detto per il fetore, che quivi
  sempre si sente a causa delle bestie del piè tondo, che morte si fanno in quel luogo
  scorticare: e tal nome viene dai Latini; che chiamavano; Sardinia. quei luoghi,
  li quali per li mali odori sono sottoposti all'infezione dell'aria, come è l'isola
  di Sardigna, la quale per havere da Settentrione monti altissimi, che le
  impediscono i venti, è sempre di cattiva aria, e sottoposta alla pestilenza. Di qui
  ancora li nostri Medici hanno dato il nome di Sardigna a quel luogo, nello Spedale
  di Santa Maria Nuova di dove si mettono gli infermi più fetenti
  per piaghe, o altro simile. In detta riva d'Arno chiamata \textit{Sardigna}, si fermano,
  e scaricano, e si ricaricano, i Navili, che da Livorno vengono a Firenze su
  per lo fiume d'Arno, e tali legni, che quivi son sempre in gran numero, finge
  che sieno l'armata di Baldone. Su questa riva, come s'è detto sono gli scorticamenti
  delle bestiacce morte, e però dice, \textit{che vi era buon sito}, e si serve di questa
  voce \textit{sito} per \textit{posto} ed in effetto vuol dire Puzzo, o Mal'odore, che scaturisce da
  quelle Carogne, e la parola \textit{sito}, che vuol dire l'uno e l'altro, fa nascere un bello
  scherzo.  Quello medesimo scherzo può farsi anche nel Latino, perché dicono
  \textit{Situm casprorum} secondo Ces. de bello Gallico, ed intendono ancora puzzo secondo
  Plin. lib. 21, \textit{Pessimum esse Crocum, quod situm redolet}.
\end{description}

\section{Stanza XV \& XXVI}
\begin{ottave}
  \flagverse{25}Costui quando Bellona fu inviata\\
A Celidora, come già s'intese,\\
Da Marte haveva havuta una fardata,\\
Che lo tenne balordo più d'un mese,\\
E gli messe una voglia sbardellata\\
Di far battaglia, e mille belle imprese;\\
Ond'egli entrato in fregola sì fatta\\
Fece toccar tamburo a spada tratta.

\flagverse{26}Poi che'pedoni egli hebbe, e gente in sella\\
Tanta ch'al fin si chiama soddisfatto,\\
Render volendo il Regno alla Sorella,\\
E farle far bandiera di ricatto,\\
Destinò muover guerra a Bertinella,\\
Ch'a lei già dato havea la scacco matto;\\
Cosè con quell'armata, e quei disegni\\
In Arno messe i sopradderti legni.
\end{ottave}

Marte era stato a trovar Baldone, conforme haveva detto alla Sorella, e l'haveva
fatto rifolvere a mettersi in arme per aiutare Celidora, e rimetterla nello
Stato; e perciò con questa gente a tal fine s'era imbarcato.

\begin{description}
\item[FARDATA] Percossa data con un pannaccio intinto in sporcizia; perché
  farda vuol dire sornacchio, che è Un grande sputo catarroso. Vedi sotto in questo
  Cant. stanza 47. E s'intende ancora per Vna quantità di sporcizia bituminosa,
  che tirata in qualche luogo s'appicchi, e s'interni in quel luogo dove è buttata,
  come farebbe una manata di fango, o altro simile buttato in un muro; Dal
  che per metafora intende in questo luogo per Vn colpo, che s'appicchi, e s'interni,
  quella persuasione, che Marte haveva fatto a Baldone di far guerra.
\item[BALORDO] Questa voce che vuol dir Inavvertito, Smemorato, che è il
  latino \textit{mente captus}, ci serve per intendere D'uno, che per qualche accidente
  occorsogli, resti sopraffatto, e non sappia a qual partito appigliarsi, per rimediare
  al danno che da quello accidente gli resulta, e si dice anche \textit{Sbalordito},
  \textit{Stordito}. Vedi sotto C. 11, stan. 25.
\item[SBARDELLATO] Una cosa che eccede i termini del naturale, ed in un certo
  modo avanza il superlativo, perché si dice: Grande, più grande, grandissimo, e
  Sbardellato; è però parola bassa, e poco usata; È forse meglio Disorbicante, o Immoderato,
  che suonano lo stesso. L'Autore del Capitolo in lode de' peducci dice.
  \begin{verse}
    Io sto cinque hore del giorno in mercato
    A pascer gli occhi di sì bell'oggerto,
    E ne cavo un piacere sbardellato,
  \end{verse}
\item[FREGOLA] Voglia grande. Onde vuol dire \textit{Entrata in fregola sì fatta} intende
  Essendogli venuta così gran voglia. È traslato dai pesci, che si dice \textit{Andare in
    fregolo}, quando s'adunano molti insieme per la generazione; ed è il latino \textit{libido},
  o \textit{cupido}, E diciamo \textit{In Fregola} I gatti, quando sono in amore. Vedi sotto Cant.
  3. stan. 30.
\item[TOCCAR tamburo] Vuol dir Suonare il tamburo, ma s'intende Arruolare
  Soldati, il che si dice anche \textit{Batter la cassa} Vedi sotto C. 3 stan. 56.
\item[A spada tratta] Incessantemente, senza riposo, Senza intermissione,  senza levar mano.
\item[FAR bandiera di ricatto] Ricattarsi, Vendicarsi. Questa voce Ricatto, che
  vien dal verbo Ricatcarsi, il quale vuol propriamente dire Liberarsi di schiavitudine,
  da noi è presa per Vendicarsi, e Far venddetta, ed è il Latino \textit{par pari
    referre}. Il dettato \textit{Far bandiera di ricatto} stimo che venga dal costume dei Corsari,
  li quali, quando pigliano qualche legno, che stimino d'essere in grado da esser
  ricattato, v'inalborano una bandiera bianca, con la quale, danno cenno alle
  Terre vicine se lo vogliono ricattare; il che se voglion fare, corrispondono con
  alzar bardiera dello stesso colore; e questo dicono Metter bandiera di ricatto.
\item[DATO havea lo scacco matto] Le havea fatto questo danno, o cagionata questa
  rovina. Il giuoco delli scacchi è antico, e fu usato prima da i Greci, che
  ora lo dicono \textit{Zatrici}, e poi seguitato da i Latini, che lo dissero \textit{Ludus
    latrunculorum}. A questo giuoco si da fine quando e fatto prigione il Re, e si dice allora
  scacco matto; onde qui vuol dice, che Celidora havea toccato Scaccomatto, havendo
  perduto il suo Regno: E s'allarga quello detto a tutto quello, che ad altri
  succeda di gran perdita, o di grave danno.
\section{Stanza XXVII}

\begin{ottave}
  \flagverse{27}Ov'anco in breve Celidora arriva\\
Con armi in dosso, ed altro da far fette,\\
Perché una volta al fin fattasi viva\\
Ha risoluto far le sue vendette;\\
Che l'usbergo incantato della diva\\
L'ha fatto diventar l'Ammazzasette,\\
Ed alle risse incitala talmente,\\
Ch'ella pizzica poi dell'insolente.
\end{ottave}


Celidora arriva all'armata di Baldone nella Sardigna, e quivi comincia a mostrare
gli effetti della Corazza incantata.
\begin{description}
\item[ARME da far fette] Intende la spada, e vuol dire che era larga, ed abile a
far fette.
\item[FATTASI viva] Rifentitasi, e fattasi ardita., E lo stesso che P7cir di-garra
morta detto sopra in questo Cant. fan, 19.:,
\item[USBERGO] Cioè quella Gran corazza di pelle di drago: detta sopra, la quale il
  Poeta qui dichiara, che ha inteso, \textit{incantata} quando ha detto sopra \textit{imbottita
  d'insulti, e di bravure} alla stan. 20.
\item[AMMAZZA fette] Contano le donne una novella per trattenimento de'Fanciulli;
  e per accomodarsi alla loro capacità, dicono::, Fu una volta un bel giovanetto
  in Garfagnana detto Nanni, il quale per la sua mendicità dormiva in una
  capanna da fieno; quivi essendo egli un giorno per riposarsi, e ripararsi dal caldo,
  si messe a pigliar le mosche, e ne haveva ammazzate sette, quando comparve
  quivi una bella Fata, e gli disse; che se le donava quelle sette mosche per cibare
  una sua passera, l'havrebbe fatto ricco. Gliele concesse egli più che volentieri;
  ond'ella innamorata di questa sua cortese prontezza lo prese per la mano,
  e lo condusse alla sua caverna, dove rivestitolo, e datogli danari, ed armi, gli
  pose in testa un'elmo, o berretta in cui era scritto a lettere d'oro: Ammazzasette;
  e lo mando al Campo de' Pisani, i quali in quel tempo. con l'aiuto de Franzesi
  guerreggiavano co i Fiorentini. Arrivato Nanni a detto Campo, chiese
  soldo a i Pisani, e domandatogli del nome rispose: Io mio chiamo Nanni, e per haver
  io solo in un giorno ammazzato sette, ho per soprannome: \textit{Ammazzasette}. Fu per
  questo, e per esser' anche ben formato, con buon soldo, e con non minore stima
  accettato. Essendo poi fra pochi giorni in una scaramuccia morta il Capo
  delle truppe Franzesi, e volendone essi fare un altro, erano fra di loro in gran
  differenza, perché essendone proposti diversi, coloro, a' quali non piacevano. i
  Soggetti proposti, gridavano Nani, Nani, onde i Soldati Italiani, che credettero,
  che dicessero Nanni, Nanni, e che havessero creato lui: cominciarono a
  gridar Nanni, Nanni; viva Nanni; e così a voce di popolo Nanni detto l'Ammazzasette
  restò eletto capo di dette truppe, e divenne ricco, si come gli haveva,
  promesso la Fata. E di questo intende il Poeta, volendo mostrare, che Celidora
  era divenuta brava, quanto questo Ammazzafette, il quale non fece maggior
  bravura, che ammazzar quelle sette mosche, si come ne anche Celidora
  non fece maggior bravura, che affettar quei Cavoli, che vedremo nell'ottava
  29. seguente.
\item[ALLE risse incitala talmente, ch'ella pizica d'insolente] Bellona le fa venir voglia
  così grande di far risse, che ella vien poi a noia, e si rende odiosa con i suoi
  modi impertinenti. 11 verbo \textit{Pizicare} vuol dire Cominciare a essere, o Esseres
  alquanto. \textit{Il tale è stato tanto tempo in Firenze, ch'ei pizica di Fiorentino}, Lo trovo
  anche usato da i Bolognesi in questo senso, e l'usò Francesco Negri\footnote{Giovanni Francesco Negri, Bologna 1593 - Bologna 1659, pittore} nel suo Tasso
  in lingua Bolognese Cant. 1, stan. \makebox[1em]{} dove \begin{verse}El pizigava di sei ann' ch'i Tramuntan\end{verse},
ec. per intendere, Era già presso a sei anni, ec.

\item[INSOLENTE] Si dice colui che dà fastidio, e noia a ognuno, e che si rende
odioso a tutti con le sue azioni impertinenti.
\end{description}
\section{Stanza XXVIII \& XXIX.}

\begin{ottave}
  \flagverse{28}Non così tosto al campo si conduce,\\
Come la suora vuol del Dio Soldato,\\
La Marfisa di nuovo posta il luce,\\
Ch'ell'esce affatto fuor del serminato;\\
E col brando che taglia, com'ei cuce,\\
Da far proprio morire un disperato,\\
Vuol trucidar' ognuno, ognun vuol morto,\\
E guai a quello, che la guarda torto,

\flagverse{29}Se guarda, è dispettosa, e impertinente,\\
 E sempre vuol che sia la sua di sopra;\\
 Talor' affronta per la via la gente\\
 Cercando liti, quasi franchi  l'opra:\\
 Ne venga (dice) pur chi vuol niente,\\
 Però che, chi mi da che far mi sciopra;\\
 Giunta in quest' in un campo pien di cavoli\\
 N' affetto tanti, che Beati Pavoli.
\end{ottave}

Descrive il Poeta una brava spropositata, e impertinente, per mostrare in Celidora
gli effetti dell'incantata Corazza; e con queste azioni, che le fa fare, dipigne
al vivo uno di questi spacconi, e ammazzatori, che noi diciamo che Campano
di fegati d'huomini, e son poi il ritratto della poltroneria, e sfogano la
lor bravura come fa Celidora, in un campo di Cavoli.
\begin{description}
\item[COME la suora vuol del Dio soldato] Come vuol la sorella di Marte, Bellona,
per opra della quale Celidora e capitata a quel campo.
\item[MARFISA] Donna guerriera nota, favoleggiata dall'Ariosto, e però la dice:
  \textit{di nuovo posta in luce}, ed intende una Marfisa moderna fatta brava da Bellona,
  cioè Celidora.
\item[USCIR del seminato affatto] Perder' il senno del tutto, Impazire. Quando altri
  per un grandissimo contento si railegra più del dovuto, diciamo: \textit{Il tale impazisce
    per l'allegrezza}; e così intende di Celidora, non che veramente sia impazita.
  I Latini hanno il verbo \textit{delirare}, che vuol dire Impazire, ed è metaforico dal
  bifolco, sendo composto dalla preposizione \textit{De}, che suona \textit{extra}, \& \textit{lirare},
  che vuol dir Fare i solchi nel campo con l'aratro; e con questo sol verbo
  \textit{delirare} intendono \textit{extra liram incedere}, dove noi diciamo Uicir del seminato, che
  è lo stesso che \textit{extra liram incedere}, o \textit{delirare}, del qual verbo ci ferviamo ancor
  noi nel medesimo senso, come si vede in Dante. Inf. c. 11.
  \begin{verse}
    Ed egli a me; perché tanto delira
    Hoggi l'ingeguo suo da quel che suole.
  \end{verse}
  E si dice anche deliro uno, che sia fuori del senno, Dan. Par. C. 1.
  \begin{verse}
    Che madre fa sopr' al figliuol deliro,
  \end{verse}

Alcuni vogliono, che.questo verbo \textit{Delirare} venga dal Greco, \textit{Lirin}, che vuol
dir scioccheggiare. Diciamo nel medesimo significato Uscire del seminario, E questo
forse deriva dal Latino \textit{Seminarium}, che secondo Colum, lib, 1. de arboribus
c. 1. 3. vuol dir quel luogo, nel quale si seminano le piante per trapiantarle, il che
quando segue, la pianta cavata dal detto \textit{Seminario} resta come un pesce fuor dell'acqua,
e piantata poi ripigiia il vigore, quando ha cominciato ad attaccarsi nella
nuova terra; e da quello, dicendosi huomo fuori del Seminario, s'intende Huomo
sbalordito. Si dice ancora \textit{fuori del secolo}, e habbiamo \textit{strasecolato}, ed il verbo
\textit{Strasecolare}, Vedi sotto Cant, 6, stan. 36. pur tutto a questo proposito. Ma si questo,
come gli altri suddetti termini, con \textit{tutto che possano credersi l'accennate
derivazioni, io stimo che intanto s'usino in questo proposito, in quanto hanno il
principio della parola, che somiglia quello della parola senno}; e che si dica fuori
del \textit{Seminato}, \textit{Seminario}, o \textit{Secolo} in vece di dire Fuori del \textit{senno}. E questa specie
di parlare, che è specie di parlar furbetto, è molto usato in Firenze per scherzo,
e lo dicono parlare Ianadattico, il qual parlare riesce assai grazioso, quando è maneggiato
da persone spiritose, perché talvolta con parole, che non hanno che
fare con quella materia, della quale si discorre, vien descritta per allusioni, ò
per metafore, ò altrimenti quella tal cosa, della quale si parla. Per esempio: Ad
un Priore, il quale a tre mogli, che haveva havuto, non hebbe mai figliuoli, ed
havea nome Antonio, dicevano \textit{Priapo annebbiato}. Ad un Proposto. che havea
nome Girolamo, ed era lungo, secco, e di colore olivastro, dicevano; \textit{Prosciutto girato}.
Di questo parlar' Ianadattico si serve sotto C, 9. stan. 1.
\item[TAGLIA come ei cuce] Tanto è buono a tagliare, quanto buono a cucire, che
  vuol dir: non taglia. Detto usatissimo per intender Ogni sorte di coltello, o arme,
  o forbice, che per la ruggine, o altro non sieno atte a tagliare.
\item[FAR morire un disperato] Dicono che le ferite fatte con i ferri rugginosi, ò
  intaccati, sieno pericolose di cagionare spasimo, e perciò quando si vede un coltello,
  o arme di tal sorte, si suol dire \textit{Farebbe morire uno disperato}, cioè di dolori eccessivi,
  o di spasimo, E tale era la spada, o brando di Celidora.
\item[GUAI a quello] Male, o gran disgrazia avverrebbe a colui, che la guardasse
torto. E il Latino \textit{Vae illi}.
\item[GUARDA torto] Quand' uno non è molto nostro amico, diciamo: \textit{Il tale non
  mi vede con buon'occhio}; O vero \textit{mi guarda torto}, Che i Latini pure dicono \textit{Non
  rectis aspicere oculis}.
\item[DISPETTOSO] Huomo altero, e che disprezza, ognuno, e d'ogni piccola,
cosa s' adira.
\item[IMPERTINENTE] Uno che vuol più del suo dovere, o del giusto, o più di
quel che gli s'appartiene.
\item[VUOL che la sua, stia sempre di sopra] Vuol sempre haver ragione, che si dice
anche Soprastante. E questi tre modi cioè \textit{Dispettoso}, \textit{Impertinente}, \textit{Soprastante}
si posson dire Sinonimi, e significanti Huomo d'una certa imperiosa arroganza, o
superbia, compagna indivisibile di tutti gli Sgherri, e bravanzoni a credenza.
\item[AFFRONTARE] Vuol propriamente dire Assaltare il nemico, ma si piglia
  ancora per Andar' incontro, o affacciarsi a uno per parlargli, e così è preso nel
  presente luogo, per intendere che Celidora cercava spropositatamente l'occasione
  di far quistione, e tutto per descriverla simile a i detti bravi di parole.
\item[CHI mi da che far mi sciopra] Dovrebbe dire Mi sciopera, secondo che da
  alcuni troppo delicati, e punto considerati ne fu avvertito il Poeta, ma la figura
  Sincope (ammessa fra i Latini) Verg. 5. AEn. dice \textit{gubernaclo} in vece di \textit{gubernaculo}
  da noi è accettata anche nella prosa, ed adoprata comunemente in molte voci,
  particolarmente in questa, dicendosi più pesso \textit{Opra}, \textit{Adoprare}, \textit{Scioprare}, che
  \textit{Opera}, \textit{Adoperare}, e \textit{Scioperare}, lo libera da questa censura. E questo termine \textit{Chi
  mi da che far mi sciopra} è proprio di certi Taglia cantoni, che voglion con esso
  mostrare che chi dà loro occasione di far questione gli \textit{sciopera}, cioè li leva dal
  farne un'altra, che han in mano, e li leva da un lavoro per impiegargli in un'altro simile.
\item[N'AFFETTÒ tanti, che Beati Pavoli] Ne tagliò in fette grandissimo numero.
  Quando vogliamo beffare un bravazzone codardo, sogliamo dire: \textit{Gran
  danno che farebbe costui in un'orto di cavoli, o di raduchi}, E quel detto \textit{Beati Pavoli},
  ha origine da un Montanbanco, il quale vendeva il rimedio contro a' veleni con
  dichiarazione di voler donare (come effettivamente donava) la pietra di S.Paolo
  a tutti coloro, che havevano nome Paolo, onde infiniti plebei per buscar quella
  pietra dicevano di haver nome Paolo; sicché egli cominciò ad esclamare
  O quanti Paoli, o quanti Paoli. E perché quelli, che ottenevano quella pietra
  si tenevano fortunati per haver havuto il regalo, ne nacque il dettato. \textit{Son più
    che non furono i Paoli Beati}, che vuol dire, furon moltissimi; Che la voce \textit{Beati} in
  questo caso è sinonimo della voce \textit{felice}, o fortunato, \textit{Beato voi che siete ricco}, per
  Felice, o Fortunato voi, che siete ricco.
\end{description}
\section{Stanza XXX}
\begin{ottave}
  \flagverse{30}Così piena di fumi, ed umor bravi\\
 Che te l'hanno cavata di Calende,\\
 Rivolge l'occhio al popol delle navi,\\
 Là dove Brescia romoreggia, e splende,\\
 E va per infilzarne sette ottavi:\\
 Ma nel pensar di poi, che se gli offende\\
 Far non porrebbe lor, se non mal giuoco;\\
 Gli vuol lasciar campare un'altro poco,
\end{ottave}

 Celidora facendo queste sue bizzarrie, vede la gente di Baldone, ed essendosi
inferocita in quei cavoli, gli vien voglia di far io stesso in quelle genti, ma si
rattien di farlo per non dar loro disgusto, e per lasciargli campare un'altro poco.
\begin{description}
\item[PIENA di fumi, che te l' hanno cavata di Calende] Mostra il Poeta, che
  Celidora sia poco meno, che briaca in questa sua bravura, i fumi della quale le
  habbiano offuscato il cervello, come fanno i fumi del vino a chi troppo beve, che
  questo intende dicendo l'hanno \textit{cavata di calende}, ed è quelio che i Latini dicono
  \textit{extra callem esse}, ed io credo che da questo Latino \textit{callem} venga la corruttela di calende; e per parlare Ianadattico detto sopra in questo C. stan. 28. si voglia dir \textit{cavata
  del calle} per intendere (come facevano i latini) Cavata di Cervello.
\item[BRESCIA romoreggia, e splende] Si sente romor d'armi, e si vedono risplender
  le medesime. A Brescia si fabbricano buone, e belle armi, e però il Poeta
  pigliando La Città per L'armi, che in quella si fabbricano, seguita l'uso nostro, che
  è di dire \textit{Il tale ha tutto Brescia addosso}, per intendere \textit{Ha molt'armi addosso}.
\end{description}

\section{Stanza XXXI \& XXXII}
\begin{ottave}
\flagverse{31}Al fin, deposto un'animo sì fiero,\\
In genio cangia a poco a poco l'ira,\\
E' come un'orsacchin, c'a pié d'un pero\\
A bocca aperta i pomi suoi rimira;\\
Ferma impalata quivi com' un cero\\
Fissando in loro il sguardo, sviene, e spira,\\
Ne può viver al fin se non domanda\\
Ove l'armata vada, e chi comanda.

\flagverse{32}S'abbocca appunto con Baldone steffo,\\
E sentendo ch'  egli ha tal gente fatte\\
Per rimeiter in sesto, ed in possesso\\
Una Cugina sua ch'è per le fratte,\\
Ben ben lo squadra, e dice: Egli è pur desso!\\
Or su ch'io casco in piè, come le gatte,\\
Ed esclama di poi: quest'è un'azione,\\
Che veramente è degna di Baldone.
\end{ottave}

Celidora pero appiacevolitasi, si ferma a guardar con gusto grandissimo quei
Soldati, e domanda di chi è l'Armata, e chi la comanda; e s'abbatte a domandarlo
a Baldone, il quale gli dice, che ha fatto quella gente per aiutare una sua
cugina, ond'ella riconosciuto Baldone, si rallegra, e dice: veramente questa è
un'azione degna di Baldone.
\begin{description}
\item[CANGIA l'ira in genio] Cioè dove prima haveva l'animo d'infilarne sett'ottavi,
  adesso comincia ad haver genio con loro, ed a portargli affetto. Questa
  voce genio se ben non pare che Toscanamente significhi cosa alcuna, nondimeno
  è molto usata dicendosi \textit{Huomo di buon genio}, o \textit{di cattivo genio} per intendere Huomo
  di buona, o cattiva indole, o inclinazione. \textit{Haver genio con uno} È lo stesso
  che Haver simpatia con uno. Appresso i Latini pure se ben genio non si distingue,
  va dall'anima ragionevole, e molti lo pigliassero spesso per Lares; altri per gli
  Dei Penati, altri per il Dio del piacere, altri per li quattro elementi, altri per li
  dodici segni del Zodiaco, altri per lo Dio che faceva nascere y ed altri per diverse
  altre cose; tuttavia essi pure se ne servivano per intendere inclinazione, come
  ci mostra Plauto in Truculento 1, 2. \begin{verse}cum genijs suis belligerare, ec. idem quod defraudare genium.\end{verse}
\item[COME un'orsacchino a piè d'un pero] Si dice L' orso sogna pere; Leva le peres
ecco l' orso, Dal-che si cava, che questo animale sia molto ghiotto delle pere; il
be anche atiefta Vincenzo Martelli nel suo Capitolo in lode delle menzognes
jicendo: \begin{verse}
  Oggi a voi più ch' ad altri si conviene,
  Benché noi siam tant' orsi a queste pere, ec.
\end{verse}
E si dice che in rimirarle gioisca tutto per la sola speranza di conseguirle; e
perciò l'Autore assomiglia Celidora a un picciolo Orso a pie d'un pero, perché in
veder quella gente, la quale ella spera che sia per lei, si rallegra, gode, e brilla,
come fa l'orso stando a pié del pero, vagheggiando le pere.
\item[FERMA impalata quivi come un cero] Per esprimere la stpidità nella quale si
trova Celidora nel vedere quei Soldati, l'Autore dopo haver detto che \textit{stava a
bocca aperta come fra l'orso a pié del pero}, soggiunge \textit{che ella stava impalata, come un
cero}, cioè ritta ritta, e fermata nel posto, come stavano quelle torrette, fatte di
carta, o di panno, o di tavole, che la mattina di S. Gio.\ mettevano li nostri
antichi attorno alla piazza del Tempio di S. Gio. Batista, entro alle quali stava
un'huomo, che le moveva, e queste le domandavano \textit{ceri} secondo che dice Goro
Dati\footnote{Gregorio Dati, 1362-1435, mercante fiorentino.} nei suoi discorsi Storici lib. 6. in fine. Hoggi in vece di tali torrette portano
in due, dello Spedale del Bigallo; sopr' alle spalle processionalmente, uno sgabellone,
sopr' al quale è fermato un gran cero fatto di legno, per sfuggire il pericolo
di romperlo sendo di cera, e faranno 26. o vero trenta ceri, che manda
detto Spedale per tributo al detto Tempio di S. Gio. Batista. Si può anche dedurre
questa similitudine da quei poveri Cristiani, i quali da i Turchi sono impalati,
che verisimilmente stanno intirizzati, e come l'Autore vuol che s'intenda,
che stesse Celidora.
\item[SVIENE, e spira] Svenire vuol dir Perdere i sentimenti, e Spirare vuol dire
Esalar l'anima, sicché si possono dir quasi sinonimi, ma in questo luogo il verbo
\textit{spirare} significa \textit{Ustolare}, che vuol dir Guardare con desiderio di conseguire,
come fa uno che havendo grandissima fame, stia a vedere un che mangi, ed habbia
d'avanti molte vivande; Vedi sotto C. 14, stan. 34.
\item[ABBOCCARSI] Trovarsi, o abbattersi in uno per parlargli. \textit{Io non son ben'
informato di questo negozio, ma m'abboccherò col tale, che m'informerà}.
\item[E' per le fratte] È rovinato. È per la mala. Quello che i latini dissero \textit{De
eo actum est}. \textit{Fratta}. S'intende Borroncello, o Macchia, che suol render' aspro
un paese, e vien dal Greco Frattin che suona Far siepe.

\item[BEN ben lo squadra] Lo guarda benissimo, che la forza della replica è di far
  nascere il superlativo, come accennammo sopra in questo C. stan. 11. Ed il verbo
  squadrare, che vuol dir Misurar con la squadra, significa Considerare, e
  Guardare un' oggetto minutamente, e con diligenza.
\item[CASCARE in pie come i gatti] Ottener da un male, o da un cattivo accidente,
  un bene impensato; che i latini dissero \textit{excidere extra mala},
\end{description}

\section{Stanza XXXIII \& XXXIV}
\begin{ottave}
  \flagverse{33}Maravigliato allora il Sir d'Ugnano,\\
  E chi sei (disse) tu, che sai il mio nome?\\
  Io ti conosco già di lunga mano,\\
  (Ella rispose) e acciò tu sappia il come,\\
  Celidora son'io del Re Fioriano\\
  Fratello d'Amadigi di Belpome,\\
  E con tutto, che già sien' anni Domini\\
  Ch'io non ti viddi, so come ti nomini.

  \flagverse{34}S'ell'è (dic' ei) così noi siam cugini,\\
  E subito si fan cento accoglienze,\\
  Ed ella a lui ne vende mill' inchini,\\
  Egli altrettante a lei fa riverenze,\\
  Così fanno talor due fantoccini\\
  Al suon di cornamusa per Firenze,\\
  Che luna incontro all'altro andar si vede\\
  Mosso da un fil, che tien, chi suona, al piede.
\end{ottave}

Baldone, e Celidora si riconoscono per cugini, e G fanno molte accoglienze.
\begin{description}
\item[CONOSCER di lunga mano] Conoscer di gran tempo. \textit{Lunga mano d'anni}
tanto suona quanto Lunga serie d'anni, o gran quantità d'anni, che diciamo
anche \textit{È un gran pezzo ch'io ti conosco}.
\item[BALDONE, Celidora, e Amadigi] sono nomi a caso:, ma l'\textit{Infante Floriano} è
anagrammatico, da \textit{Raffaello Fantoni}.
\item[SON' anni Domini] Son' anni infiniti. Sono tanti anni, quanti sono dalla nascita
  di Nostro Signore che diciamo Anno Domini. iperbole usatisima in Firenze.
\item[ACCOGLIENZA] Ricevimento con amorevolezza, e cortesia, e con una
  certa dimostrazione d'affetto, che s'usa verso le persone grate. Vien dal Latino
  \textit{Colere}, che esprime Amar con riverenza, ed honore.
\item[INCHINO] È lo stesso che \textit{riverenza} facendosi con abbassar la testa, e piegare
  le ginocchia, ed è proprio delle Donne; \textit{Riverenza} si fa con abbassar la testa,
  e piegandosi un sol ginocchio si manda l'altra gamba addietro a foggia di
  genuflessione, ed è propria degli huomini, come si vede nel presente luogo, che
dice,\begin{verse}
Ella a lui ne rende mille inchini;
Egli altrettante a lei fa riverenze,
\end{verse}
\item[COSÌ fanno talor due fantoccini] Suol' andar per Firenze un contadind, suonando
  una cornamusa, e porta alcune figurine di legno, che hanno le congiunture
  delle membra mastiettate, e contrappesate con piombo in modo, che si
  muovono per ogni verso; queste infilza per lo petto in una sottilissima corda da
  chitarra, o diciamo minugia, la quale da una parte lega ad uno de' suoi ginocchi,
  e dall'altra ad una tavoletta posta in terra a tal fine, e col muovere quella gamba,
  alla quale è legata la corda; fa, che quelle due figurine infilzatevi ballano al tempo
  del suono della cornamusa. Intesa dunque questa operazione, che fanno i due figurini,
  s'intende ancora come facessero fra di loro questi due parenti.
\item[CORNAMUSA] Zampogna doppia, composta d'un basso perpetuo, e di un
  soprano, che canta le note come gli altri Zufoli, e si da il fiato ad ambedue con
  un sacco di quoio, da colui che suona, ripieno di vento: col soffiare in un piccolo
  cannello animellato; ed il suonatore premendo col braccio il detto sacco da il
  fiato a dette due Zampogne.
\end{description}

\section{Stanza XXXV}
\begin{ottave}
  \flagverse{35}Poi che le fratellanze, e i complimenti\\
Furon finiti, a lei fece Baldone\\
Quivi portar un po di sciacquadenti,\\
O volete chiamarla colezione,\\
Hor mentre, ch' ella scuffia a due palmenti\\
Pigliando un pan di sedici a boccone,\\
Si muove il campo, e sott'alla sua insegna\\
Ciascun passa per ordine a rassegna.
\end{ottave}

Dopo finite le cirimonie Baldone fa portar da bere, e da mangiare, e mentre
che Celidora mangia, si fa la mostra de' Soldati.
\begin{description}
\item[FAR le fratellanze] È tratto dall'uso che nelle nostre Compagnie, ò Confraternite
  di secolari, nelle quali a i tempi determinati si vanno tutti ad abbracciare
  l'uno con l'altro; e questa azione dicono \textit{Far le fratellanze}, E da questo
dunque intendi dopo finiti gli abbracciamenti e le cirimonie.
\item[SCIACQVADENTI] Quel che significhi lo dichiara il Poeta medesimo dicendo;
  \textit{O volete chiamarla colazione}. Che vuol dire parcamente cibarsi fuor del desinare,
  e della cena, e viene dal Latino \textit{collectio prandij vel coenae}. Ma siccome son
  diversi li pasti che si fanno in Firenze, così son diversi li nomi che loro danno. Il
  primo mangiare che si fa fra l'alba, e il mezzo giorno si chiama \textit{Asciolvere}, ed
  alle volte colazione. Quello, che si fa a mezzo giorno fi chiama \textit{desinare}. Quello
  che si fa tra 'l mezzo giorno, e la sera si dice \textit{Merenda} quali \textit{meridie edenda}.
  Quello della sera si dice cena, ed allora che per il digiuno la sera si mangia poco
  si dice colazione; E la voce \textit{sciacquadenti} vuol veramente dire Quando si mangia
  qualche poco, per bere con gusto.
\item[SCUFFIARE] Mangiar con ingordigia, o divorare. È voce Fiorentina, ma
hoggi usata solo per scherzo, e vien forse da \textit{Scuffina} che è una raspa, o lima da
legno detta così, perché adoprandola leva molto legno per volta, e per questo è
chiamata anche \textit{ingordina}.
\item[A due palmenti] Da ambedue ganasce: Traslato dal Molino, che si dice
  \textit{Macinare a due palmenti} quando Due ruote lavorano; che \textit{palmento} vuol dire
  tutta la macchina, che fa macinare, dicendosi Molino \textit{d'un palmento}, o di \textit{due
    palmenti}, quando Un molino ha una, o due macini. E stimo che si dica \textit{Palmento},
  quasi Palamento, perché le ruote, che fanno andar le macine son composte di
  tavole a foggia di pale per prender l'acqua, che le fa girare.
\item[UN pan di sedici, ec] Con questa iperbole esprime l'ingordigia di Celidora;
  perché per altro un pane di sedici de' nostri quattrini malamente si può consumare
  anche con sedici bocconi, intendendo \textit{Boccone} quella quantità, che l'huomo
  può pigliar dentro alla bocca in una volta.
\item[PASSAR a rassegna] Quando i Soldati si portano avanti, al loro Capitano,
  e fanno scrivere il lor nome si dice \textit{Passar a rassegna}. E qui Baldone come supremo
  Capitano per fare honore alla cugina, Fa la rassegna, nominando, però solamente
  gli Ufiziali prinicipali; il che pare che più propriamente si dica \textit{Dare}, o
  \textit{far la mostra}, Vedi sotto C, 2. stan. 36.
\end{description}

\section{Stanza XXXVI}

\begin{ottave}
  \flagverse{36}E per il primo viensene in campagna\\
Pappolone il Marchese di Gubbiano,\\
Colui, che nel conflitto della Magna\\
Estinse il Gallo, e seppellì il Germano;\\
È la sua schiera numerosa, e magna,\\
E perch'egli è Soldato veterano,\\
Ha nell'insegna una tagliente spada,\\
Ch'è in pegno all'osteria di mezza strada
\end{ottave}

L'Autore in questa sua Opera mette una mano d'amici suoi sotto nomi anagrammatici,
la maggior parte de' quali è nominata in questa mostra, che Baldone
fa dell'esercito, descrivendone alcuni con qualche loro azione, ò con un'epilogo
della loro vita oltre all'Anagramma. Il primo che viene in mostra e Pappolone,
cioè \textit{Paolo Pepi} anagramma proprio, perché questo gentilhuomo era giovanotto
grande di persona, e grasso, e mangiava assai; e per questo il Poeta lo
dice \textit{Pappolone}, che vuol dir gran mangiatore. Vedi sotto C. 6. stan. 70.,  e lo fa
\textit{Marchese di Gubbiano}, che è un Castello; e Ingubbiare (detto però plebeo) significa
Empier il ventre. Dice \textit{nel conflitto della Magna}, cioè Nel mangiare, se ben
par che voglia dire in una sanguinosa battaglia seguita in Alemagna. Estinse il
Gallo, e seppellì il Germano; par che dica ammazò Francesi, e Tedeschi, ma vuol
dire ch'ei mangiò galli, e germani; e gli fa fare per insegna una spada impegnata
all'oste di mezza strada, che è un'osteria fuor di Firenze un miglio, e così
mostra, che ogni fine di questo tale era il mangiare.
\section{Stanza XXXVII}
\begin{ottave}
  \flagverse{37}Bieco de Crepi Duca d'Orbatello\\
Mena il suo terzo c'ha il veder nel tatto,\\
Cioè perch'ei da un occhio sta a sportello,\\
Soldati ha preso c'hanno chiuso afatto,\\
Son l'armi loro, il bossolo,e il randello,\\
Non tiran paga, reggonsi d'accatto,\\
Soffiano, son di calca, e borsaiuoli,\\
E nimici mortal de' muricciuoli.
\end{ottave}


Segue dopo Pappolone \textit{Bieco de Crepi}, cioò Piero de Becci huomo di faccia non
troppo bella, con occhi biechi, e lusco, e però il Poeta con l'equivoco \textit{d'orbo},
che vuol dir mezzo cieco, come vedemmo sopra in questo Cant. stanza 9., lo fa
\textit{Duca d'Orbatello}, e dice, che vedendo egli alquanto, ha preso per Soldati gente,
che è affatto cieca, avverando il detto. \textit{Beati Monoculi in terra caecorum}. Hanno
questi soldati il bossolo, e il bastone, non tirano paga, ma vivono di limosine,
son tutti spie, ladri, monelli, e nimici de' muricciuoli.
\item[UN terzo] Numero di soldati comandati da pil capitani, e dal Colonnello;che
i Latini dicevano /egionem, ed il Colonnello forse era Tribunus,
\item[MENARE] Condurre, Ma qui fla proprio il verbo Menare secondo il pro-
verbio che dice: Solo tciechi si menano, —~;
\item[HA il veder nel tatto] U ciechi non hanno altra vista, che il tatto, el odorato
nelle cose corporee, e materiali; e 1' udito nell' incorporee.
\item[STA a spertello] Intende mezzo cieco. Metafora tolta da quelle botteghe; le
gualt quando non è fefta intera, e comandata stanno mezze aperte, che si dices
Star' a sported, perché aprono folo quella parte del legname, che si chiama se
tello; e seguita la metafora dicendo: Su/dati ha preso channo.chiufo affatto: cioè s0-
no affatto ciechi. Varchi stor. Hior. lib. 11. dice: Won si tennero le botteghe Aperte,
ne a sportello, ma chinfe affatto, 4 j:
\item[BOSSOLO] E' quel valoa foggia di calice, col quale si raccolgono i voti ne-
gli Squittini. Vedi sotto Cant. 6.stan, 109., e per la similivudine intendiamo quel
valo di latta, di rame, d' ottone, o d' aitra materia, che e usato da i ciechi per
ricevervi l'clemofine, ay
\item[RANDELLO] Intende Quel baflone, che adoprano i ciechi per farfila stra-
da. Se ben randello s'mmtende un Pezzo ci bastone grosso quanto quello de'ciechi,
ma assai pil corto, che s' adopra per firingere le legature delie baile, che però
tale operazione si dice edrrandeliare.
\item[REGGONSI d'accatto] 1 verbo Reggersi in questo laogo, ed in questi termini
vuol dir Cavar il guadagno per mantenersi: M tale si regge cal far' il farto, Cive
vive col guadagno, che cava dal far' il farto, ec. 4
\item[SOFFIARE] }n lingua furbesca vuol dir Far la spia, se bene & inteso comune-
mente. Ed il Poeta parlando di cicchi, i quali hanno per costume di parlar fur-
beko, & serve di questa, ed altre lor parole, come E//er di calea', che vuol dir
Huomo da far qualfivoglia furfanteria, e viene dalla voce Calcagno, che in lin-
gua furbesca vuol dir Moneillo, cioè /adro di calca nella quale entrano per rubar

'ic borfe, e di qui si dicono Borsatolt, e Faglia borfe. Vedi sotto C. 6. stan. 64.
\item[NIMICT de' muriceinoli] Chiamiamo muricciuoli quel pezzo di muro, che avan-
za sopr'a terra attorno alle case; d' altezza d'un braccio pil', o meno,e di
simile largheaza; fatto, o per wlo di sedere, © per difefa de i fondameati. Di
guefti sono nimici i ciechi, perché spesso vi Pp jotono dentro co"! piedi, ingan-
nati dal sentir al vifo, ed alle mani l'aria libera, il che fa lor credere, che non
potia esservi impedimento veruno anche in terra. nth
\end{description}
\section{Stanza XXXVIII}
\begin{ottave}
  \flagverse{38}La strada i più si fanno col bastone,\\
Altri la guida segue d'un suo cane,\\
Chi canta a più d'un'uscio un'Orazione,\\
E fa scorci di bocca, e voci strane;\\
Chi suona il ribecchin, chi il colascione;\\
Così tutti si van buscando il pane.\\
Han per insegna il diavol de' Tarocchi,\\
Che vuol tentar un forno pien di gnocchi.
\end{ottave}

Descrive il modo del marciare di questi ciechi, e fa lor fare quei gesti, ed operazioni,
che son soliti fare andando a cercare elemosine, Dice che \textit{I più si fanno
la strada col bastone; altri si fanno guidare a un cane, ed altri vanno cantando Orazioni
a pié d'un'uscio}; E questi son ciechi stipendiati dalle persone pie, acciocché ogni
giorno, o ogni settimana vadano alle case delle medesime persone a cantare un'orazione
avanti al loro uscio, dove per esser sentiti fanno voci strane, cioè Gridano
forte, e fanno \textit{brutti scorci di bocca}; E questo avvien loro perché, per lo più,
li ciechi oltre alla loro cecità, sogliono havere altri stroppi nella faccia. Molti
suonano il ribechino, cioè il violino, altri il \textit{Colascione}: questo strumento (che da
i più è detto corrottamente \textit{Ganascione}) E' un corpo, come quello della tiorba,
con manico lungo, con due sole corde, il quale si suona con un pezzo di suolo
da scarpa, che volgarmente si dice taccone; E perciò tale strumento è detto anche
Tiorba a Taccone da Filippo Scrutendio da Scafato\footnote{Felippo Sgruttendio de Scafato. Ignoto, forse anagramma di persona reale, in vita nel 1646.}, il quale così intitola il
suo grazioso Canzoniero Napoletano. Alcuni furbi per \textit{colascione} intendono la
forca, perché ancora a questo s'adoprano due corde, la grossa, e la sottile, come
alla forca. Questi ciechi suonatori soglion sempre andar vendendo qualche
Orazione, o Rappresentazione, o altre Leggende, e così tutti si vanno buscando
il pane, cioè guadagnano da vivere. E volendo il Poeta mostrare quanto la gente
di questo terzo sia affamata, le da per insegna un diavolo, che tenta un forno
pieno di gnocchi; e mostra che sia sempre intenta a procacciarsi il vitto con ogni
sorta d'invenzione, che il verbo tentare significa Procurare, o Provarsi di far
una tal cosa, e si deduce, che questo diavolo \textit{tentasse}, cioè si provasse a rubar da
quel forno il pane, che vi era dentro. E per \textit{gnocco} intende Ogni sorte di pane;
Se bene \textit{gnocco} & quella specie di pane, che dicemmo sopra in questo C. stan. 3.
\begin{description}
\item[SCORCI di bocca, e voci strane] Voci strane, e bocche diverse dal naturale;
perché se bene la voce \textit{scorcio} è termine di prospettiva, che mostra la figura esser
resa capace della terza dimensione del corpo; s'intende anche per positura di corpo,
o parte d'esso diversa dal naturale.
\item[TAROCCHI] Carte, con le quali si giuoca alle Minchiate. Vedi sotto C.8. stan.
  61. in una delle quali carte al num. 14. è effigiato un Diavolo; e questo dice, che
  \textit{tenta il forno pien di gnocchi}. Il nostro Poeta haveva dato a questi Ciechi l'impresa
  del Buio, come si vede in alcuni suoi sbozi, che diceva.
  \begin{verse}
    Hanno un' impresa, dove Bieco mette
    Il buio che a svegliar va le Civette.
  \end{verse}
\end{description}

\section{Stanza XXXIX, XXXX, XXXXI}

\begin{ottave}
  \flagverse{39}Dietro al Duca, c'ognun guarda a traverso\\
Vanno cantando l'aria di Scappino,\\
Ma non giunsero al fin del terzo verso,\\
Che venuto alla donna il moscherino,\\
Fatto a Bieco un rabbuffo a modo, e verso,\\
Gli disse: S'io v'alloggio dimmi Nino,\\
Perch'io non veddi mai in vita mia\\
Pigliar i ciechi fuor c'all'osteria
\end{ottave}

\begin{ottave}
  \flagverse{40}Signora, rispos'egli, benché cieca,\\
Fu però sempre simil gente sgherra;\\
Con quel batocchio zomba a moscacieca\\
Senza riguardo, come dar' in terra;\\
Sort'ogni colpo intrepida s'arreca,\\
Che non vede i perigli della guerra:\\
E' cieca è ver, ma pur il pan pepato\\
E' più forte, se d'occhi egli è privato,

  \flagverse{41}Ovvia (diss'ella) tocca innanzi il cocchio,\\
E se costoro a guerreggiar son'atti\\
Tienteli pure, e non mi star' a crocchio,\\
Mentre gli è tempo qui di far di fatti.\\
Va dunque o forte, e invitto bercilocchio,\\
Che i nimici da te saran disfatti,\\
Perch' in veder la tua bella figura\\
Cascan morti, senz'altro, di paura.
\end{ottave}

Questi ciechi andavano dietro a Bieco cantando l'aria di Scappino, (che e una
canzonetta, la quale cantavano i ciechi in Piazza del G. Duca, quando l'Autore
principiò la presente opera) ma Celidora adirata di ciò, dice a Bieco, che
non vuol tal gente, ed egli rispose, che se bene eran ciechi eran però fieri, che
il non vedere i pericoli gli rendeva arditi, e forti, come appunto è il pan pepato,
che è più forte, quando non ha occhi; ond'ella gli dice, che se gli tenga, e vada
allegramente, che ella ha speranza di cavar frutto da lui solo senza loro, perché
stima, che il nimico sia per cascar, morto subito, che vedrà il suo brutto viso.
\begin{description}
\item[GVARDA a traverso] Uno che ha gli occhi scompagnati, come haveva Bieco
  diciamo Guardare a traverso. Vedi sopra in questo Cant. stan. 9. \textit{Transversa tuentibus
  hirquis}, Virg. Egl. 3.
\item[VENUTO alla donna il moscherino] La donna, cioè Celidora, s'adirò. Si dice
  \textit{Venire il moscherino al naso}, perché si trovano alcune piccole mosche, le quali
  volando, talvolta entrano nel naso altrui, e toccando quella parte così sensitiva,
  danno grande alterazione, e mettono l'huomo in una subita impazzienza, e stizza.
  Si dice ancora \textit{Venir la senapa}, o \textit{la Mostarda al naso}, perché nel mangiar la
  mostarda (che e un'intingolo fatto di senapa, e mosto cotto) quando è ben carica di senapa,
  viene al naso un certo pizicore, che forza a, lagrimare. Si dice
  anche \textit{Venir la muffa}, o altri puzi odiosi, e sporchi, come si dice sotto C. 4. stan.
  23. E tutti significano Venir collera.
\item[FATTO un rabbuffo] Bravato. Fare un rabbuffo, o Rabbuffare vuol dire Riprender
  uno con minacce, o Spaventarlo con asprezza di parole. Il Landino nell'esposizione
  a Dante C. 7. dell'Inferno alla parola Buffa, e Rabbuffare dice: \textit{Ma
  proprio Buffa è vento, onde diciamo Buffettare chi getta vento, per bocca,e Sbuffare, quando
  con suono di parole, o a dir meglio Con ventose, ed enfiate parole alcuno minaccia.
  Di qui diciamo Rabbuffare, Conturbare e muover le cose dell'ordine loro, e scompigliarle
  e chiamiamo Rabbuffo, quando Con parole conturbiamo, e Scompigliamo la mente d' uno}.
  Vedi sotto C. 3. stan. 57, la voce \textit{Buffi}.
\item[A modo, e a verso] Con tutta perfezione. B il latino \textit{modis, \& formis}.
\item[DIMMI Nino] Dimmi pazzo, e senza Cervello, come fu Nino, il quale per lo
  grande amore, che portava a Semiramide sua Meretrice o moglie, le concesse
  che per un giorno ella fusse assoluta Regina, ed ella in quel giorno lo fece ammazzare,
  e si confermò Regina per sempre, come si legge in Plutarco in Serm.
  Amator.
\item[PIGLIAR i ciechi fuor c'all'osteria] Quand' uno vince assai, sogliamo dirgli: \textit{Si
  torrà i ciechi}, e s'intende \textit{all'osteria}. E questo perché si suppone, che quel
  tale, che vince per l'abbondanza del denaro venutogli in mano fenza
  fatica, sia per spenderlo profusamente in pigliarsi tutti li suoi gusti fino con
  l'andare a cena all'osteria, e chiamare alla sua mensa a suonare alcuni ciechi, i
  quali in su l'hora del mangiare vanno girando, per l'ofterie a tale effetto, e questi
  sono i Ciechi, li quali Celidora dice haver veduto pigliare all'osterie.
\item[SGHERRO] Bravo. Ammazzatore; Tagliacantoni. Vedi sotto, Cant. 3. stan. 42.
\item[BATOCCHIO] Quel bastone, col quale si fanno la strada i ciechi si chiama
\textit{Batocchio} dal batterlo in terra, che fanno i ciechi, per farsi riconoscere per quel
battere da gli altri ciechi. E però vuol dire anche il Battaglio delle Campane.
\item[ZOMBA] Perquote, bastona. Vedi sotto C. 6. stan. 104., e C. 11. stan. 28.
\item[MOSCA cieca] Il giuoco detto Mosca cieca è trattenimento da Fanciulli, che
  deriva dall'antico, e si diceva \textit{Musca aenea}, e si faceva nel modo, che usano
  hoggi, che è in questa maniera.

  Tirano le sorti fra più ragazzi a chi debba bendarsi gli occhi, (che in questo
  giuoco dicono Star sotto) ed a quello, a cui tocca, sono bendati gli occhi in modo,
  che non possa vedere, e poi con uno sciugatoio, o altro panno avvolto, che ciascuno
  tiene in mano, si danno da gli altri delle percosse a colui, che è sotto, ed egli
  così alla cieca va rivoltandosi, e quello che egli arriva con la percossa deve bendarsi
  in vece del, percuziente, il quale si leva la benda, e va fra gli altri a percuotere
  il nuovo bendato; Quello, al quale di mano in mano tocca a star sotto, mena
  senza riguardo, colpi spietati, sì perché commosso da tanti colpi vorrebbe
  vendicarsi, sì anche perché, cogliendo, il colpo sia in modo da non poter'
  esser negato, procurando ognuno di non toccarne, e d'occultarla, se può,
  quando l'ha toccata, per non haver' a stare in quel martirio, in che è colui, che
  sta sotto. E però dice \textit{Zomba a mosca cieca senza riguardo come dare in terra}.
  Si dice \textit{mazzate da ciechi} per intendere Percosse spietate.

\item[IL Pan pepato è più forte se d'occhi egli è privato] Si suole in Firenze per la sesta
  di tutti i Santi fare un certo pane che da noi si dice \textit{Pan pepato}, il quale è
  composto di sapa, aceto, farina, pepe, ed altri aromati, e mescolanvi pezzetti di
  bucce di poponi, zucche, cedri, e d'aranci conditi in zucchero, o miele, li quali
  pezzetti, quando il pane si taglia, restano nella tagliatura a similitudine d'occhi,
  e perciò da i nostri Fanciulli son chiamati Occhi; E cavandosi dal pane tali
  occhi, che sono dolci, il pane resta \textit{più forte}, cioè più acido; ed il Poeta si serve
  della parola Forte in significato di Gagliardo, dicendo che i ciechi sendo senz'occhi
  son più forti, ed intende gagliardi, scherzando con questo equivoco di forte.
\item[TIRA innanzi il cocchio] Seguita il tuo viaggio, e tanto s'intenderebbe a dir
solamente \textit{tira innanzi} senza porvi l'aggiuata \textit{Cocchio}, ma il Poeta ve lo pone
per seguitar l'uso Fiorentino.
\item[STAR a crocchio] Il verbo \textit{Crocchiare}, e la frase \textit{stare a crocchio}
  significano Cicalare, o Ciarlare di cosa di poco frutto, o importanza per finire il giorno.
  Onde questi tali si dicono \textit{Crocchioni}, \textit{Cicaloni}, \textit{Perdigiorni}, e simili.
  Vedi sotto Cant. 3. stan. 5. Questo verbo \textit{Crocchiare} serve anche per intendere Dar
  delle buffe. Vedi sopra in questo Cant. stan. 10.
\item[BERCILOCCHIO] Epiteto composto dal Poeta, che vuol dir Bircio di che
sopra in questo Cant. stan. 9.
\end{description}
\section{Stanza XXXXII \& XXXXIII}
\begin{ottave}
  \flagverse{42}Ne Segue intanto Romolo Carmari
Cavalier di valore, e di gran fama;
Ma sfortunato, perché coi danari
Giuocando egli ha perduta anco la dama.
Con le pillole date a suoi erarj
L'affetto evacuò l'Arpia ch'egli ama.
Tal che senz'un quattrino ammartellato
Alla guerra ne va per disperato.

  \flagverse{43}Dop'un'insegna nera che v'è drento,
Cupido morto con i suoi piagnoni
Marciar si vede un grosso Reggimento,
Ch'egli ha d'innumerabili tritoni,
Al cui arrivo ugnun per lo spavento
Si rincantuccia, ed empiesi i calzoni,
E da lontano infin dugento leghe
S'addoppiano i ferrami alle botteghe.
\end{ottave}


Segue \textit{Romolo Carmari}, Questo fu un Fiorentino, del quale non stimo bene scioglier
l'anagrammma, e dirne il nome. Questo Gentilhuomo havendo durato un
gran tempo a godere una sua Meretrice, e spesovi molto danaro, o gli fu tolta,
o ella non lo volle più perché egli abbandonò lo spendere; come è proprio di
simili donne; e ciò esprime il Poeta in quei due versi.
\begin{verse}
Con le pillole date a suoi erarj
L'affetto evacuò l'Arpia ch'egli ama.
\end{verse}
I quali versi suonano: L'havergli fatta votar la borsa fece disperdere l'amore,
che ella fingeva di portargli, Onde egli disperato, se ne va alla guerra; e
mostra questo suo spento amore nell'insegna, che egli porta, in cui è dipinto
Cupido morto, che ha d'attorno i suoi piagnoni. E perché questo Signore era
nel vestire positivo, e senza boria alcuna, anzi più tosto abbietto, il Poeta fa,
che egli conduca un reggimento di gente mal vestita, e questi huomini chiama
\textit{Tritoni}, perché Huomo trito, o Tritone tanto vale appresso di noi quanto dire
Huomo mal vestito; E questa gente per esser così mal vestita e stimata una schiera
di Monelli, e di Ladri, e perciò è causa, che s'accrescano i serrami alle botteghe,
e che ognuno fugga per la paura, che ha di loro.
\begin{description}
\item[DAMA] Vuol dir Donna nobile, venendo dal Greco \textit{Damar}, secondo alcuni;
e suona Signora dal Francese Dame, Madame, cioè Signora, mia Signora; ma
si piglia anche per l'amata, come è preso nel presente luogo.
\item[CON le pillole date a suoi erarj] Con l'evacuatorio dato alla sua borsa, cioè con
avergli fatti finire i danari mandò via dal suo corpo la bile amorosa, cio' lasciò
d'amarlo.
\item[L'Arpia] Intende Meretrice, ed esprime una donna rapace, come sono le
  Meretrici (che Arpia in Greco suona come Rapace) e quali sono figurate
  Arpie, che i Poeti fingono esser tre, Aello, Ocipete, e Celeno; e le
  fanno figlie di Nettunno, e della Terra; altri figlie di Thaumante, ed
  Elettra, altri d'altre Deità; basta che se ne servivano per esprimer l'avarizia.
  Vergil. 3. AEn.
  \begin{verse}
Tristius haud illis monstrum, nec sævior ulla
Pestis et ira deum Stygiis sese extulit undis,
Virginei volucrum vultus, foedissima ventris
Proluvies, uncaeque manus, et pallida semper
Ora fame.
  \end{verse}
  E Dante nell'Inf. Cant, 13. seguitando Vergilio dice
  \begin{verse}
    \backspace Quivi le brutte Arpie lor nido fanno,
    Che cacciar dalle Strofade i Troiani
    Con tristo annunzio di futuro danno.
    \backspace Spalle hanno alate, colli, e visi humani;
    Piè con artigli, e pennuto il gran ventre;
    Fanno lamenti su gli alberi, strani.
  \end{verse}
  Questo nome d'Arpia dette a una Meretrice anche il Coppetta nel suo Capitolo
  in biasimo della Signora Ortenzia Greca dicendo
  \begin{verse}
  Arpie crudeli, infide, inique, e ladre
  da venire a fastidio a mille Rome
  Voi, la vostra fantesca, e vostra madre.
  \end{verse}
\item[AMMARTELLATO] Haver martello, o esser' ammartellato vuol dire
Quand'uno innamorato ha gelosia della cosa amata, ovvero ha qualche sdegno
con la medesima. Il Firenzuola nel suo Capitolo in lode del legno santo, chiama
pazzia l'esser'ammartellato dicendo:
\begin{verse}
\backspace Hor nuovamente vi dico che cava
Di fastidio un, che crepi di martello,
Guarda se questa è un'opera brava.
\backspace E s'i pazzi volesson provar quello,
E conoscesson la lor malattia,
Tutti ritornerebbono in cervello;
C'altro non è il martel c'una Pazzia.
\end{verse}
\item[PER disperato] La disperazione è una soverchia inquietudine, cagionata da
  grave disgusto, la quale ci leva affatto il dominio di noi medesimi.
\item[PIAGNONI] Trova spesso nelle storie Fiorentine questo nome Piagnoni, che
vuol dir Coloro che seguitavano la parte di F. Girolamo Savonarola; ma qui vuol
dir Quegli huomini, che si mettono a i mortori de i gran personaggi attorno al
cadavero, tutti coperti di nero, e con lunghi veli, ed in mano hanno uno stendardo,
o pennoncello di taffettà nero: E si dicono Piagnoni, dal piagnere che
dovrebbon fare per la morte di quel tale.
\item[MARCIARE] È il muoversi degli eserciti. Voce restata a noi dal Francese;
  e da molti si dice Marchiare, perché questi tali, vedendola scritta con l'aspirazione,
  la pronunziano all'Italiana, non si curando di riflettere che il C-H suona
  sci, e non chi.
\item[REGGIMENTO] Quantità di Soldati comandata da più Capitani, e dal Colonnello;
  e forse lo stesso, che Terzo detto sopra in questo C. stan. 37.
\item[TRITONI] Sono Dei, o Mostri Marini, i quali si dipingono ignudi, o al
più coperti d'aliga, e di qui gli huomini mal vestiti si chiamano da noi Tritoni,
quasi huomini triti, che suona Huomini vili, ed abbietti. Vedi sotto in questo
Cant. stan. 86.
\item[INCANTUCCIARSI] Nascondersi, o mettersi per i canti per non esser
veduto.
\item[EMPIESTI i calzoni] Per la paura, se li move il corpo, e gli empie le brache.
Questo detto esprime, che Quei Tritoni facevano gran paura a chi gli vedeva, non
che veramente se gli empiessero i calzoni.
\item[S'ADDOPPIANO i serrami alle botteghe] Per afficurarsi da costoro, che sono stimati
  tanti ladri, in gran tratto di paese rinforzano le serrature alle botteghe. E qui
  l'Autore dice tutto quello, che egli può, per mostrar costoro affatto birboni, e
  vera canaglia.
\end{description}

\section{Stanza XXXXIV}
\begin{ottave}
  \flagverse{44}Hor comparisce Dorian da Grilli,\\
che nella guerra e così buon soggetto,\\
Che metterebbe gli Ettori, e gli Achilli,\\
E quanti son di loro in un calcetto:\\
Scrive sonetti, canta ognor di Filli,\\
E' buon compagno, piacegli il vin pretto,\\
Rubato, per insegna, ha nel Casino\\
Il quattro delle coppe c'ha il monnino.
\end{ottave}

Segue nella mostra Doriano da Grilli che è Lionardo Giraldi. Questo gentilhuomo
fu bellissimo humore, molto dedito alla poesia burlesca, buon discorritore,
ed huomo di conversazione; e perché egli haveva per costume il dar de Monnini,
il Poeta gli fa fare per impresa Una carta da giuocare, nella quale in mezzo a
un quattro di coppe è figurato un Monnino\footnote{La bertuccia, nel mazzo delle Minchiate.}.

\begin{description}
\item[METTERE uno in un calcetto] Confondere uno, Superar' uno nel sapere, o
  nel valore, e ridurlo tanto avvilito, che si vorrebbe nasconder dentro a un calcetto,
  vilissima, e piccola parte dell'abito dell'huomo, come quella che non
  cuopre se non il piede, Questo Doriano veramente non fu mai soldato, se ben
  l'Autore dice, che egli è \textit{buon soggetto nella guerra}; ma dice così di lui, perché
  essendo egli di sua conversazione, lo sentiva spesso discorrer delle guerre con gran
  fondamento mostrandosene assai pratico.
\item[VIN pretto] Vino puro, e senza commistione d'acqua, o d'altro; e sentendosi
  in più luoghi del nostro Contado chiamarlo \textit{vino puretto}, non son lontano da
  credere, che la voce \textit{pretto} sia o figurata, o corrotta da \textit{puretto}.
\item[CASINO] Intendi quella Casa nella quale la nobil gioventi Fiorentina s'aduna
  per giuocare,
\item[MONNINO] Le carte de' Ganellini, o Minchiate hanno in se effigiate quattro
  cose diverse, che una parte hanno spade, una parte bastoni, una parte danari,
  ed una parte coppe, e tutte quattro queste specie di carte comingiano da
  uno fino a 14. Nella carta del quattro di coppe in mezzo è figurata una bertuccia
  a sedere, la qual bertuccia da noi è detta \textit{Monnino}. E questa dice il Poeta, che
  è l'insegna di Doriano; perché egli solito di dare i \textit{Monnini}, che vuol dire,
  Quand'uno parlando con un'altro, questo lo forza a dir qualche parola, che rimi
  con un'altra, che a quel tale dispiaccia; per esempio Doriano disse ad un Cherico:
  \textit{Non fu mai gelatina senza \makebox[1.5em]{\dotfill}} E qui si fermò fingendo non si ricordare
  della parola che finiva il verso; ed il Cherico, il quale ben sapeva la sentenza
  gliela suggerì dicendo: \textit{senz'alloro}, e Dorian soggiunse: \textit{Voi siete il maggior bue
  che vada in coro}. E questo si dice dare i \textit{Monnini}.
\end{description}


\section{Stanza XXXXV \& XXXXVI}
\begin{ottave}
  \flagverse{45}Fra Ciro Serbatondi il Sir di Gello\\
Che in Pindo a Mona Clio sostiene il braccio,\\
Egeno de Brodetti, e Sardonello,\\
Vasari, ch'è padron di Butinaccio,\\
Conducon tanta gente ch'è un flagello\\
Da far che le pagnotte habbiano spaccio,\\
Di cui (perch'il mestar diletta a ognuno)\\
Si pigliano il comando a un dì per uno.

\flagverse{46}Di foglio per impresa un bel Cartone\\
Insieme con la pasta egli hanno messo,\\
Dei lor Fantocci, i quali da Perlone\\
Soglion copiare, o disegnar dal gesso,\\
Nel mezzo v'han dipinto d'invenzione\\
L'impresa lor, nella quale hanno espresso\\
Su le tre hore il venticel rovaio\\
C'ha spento il lanternone a un bruciataio.
\end{ottave}

Seguitano tre gentilhuomini scolari dell'Autore; uno è Fra \textit{Ciro Serbatondi},
che vuol dire \textit{Cristofano Berardi}, quale fa Sir di Gello perché ha forse una sua
villa così detta. Dice che \textit{sostiene il braccio, a Mona Clio}, perché egli è huomo
letterato. L'altro è \textit{Egeno de Brodetti}, che vuol dir \textit{Benedetto Gori}. Il terzo è
\textit{Sardonello Vasari}, che vuol dire \textit{Alessandro Valori}, il quale fa Sig. di Botinaccio,
perché ancor'egli ha una Villa così detta. Conducono questi molta gente, la
quale comandano vicendevolmente a un giorno per uno, e perché si conosca che
sono stati tutti tre scolari dell'Autore, fa lor fare una bandiera de i fogli di quei
disegni, che hanno fatto in squola sua; Ma perché questi attesero più alle lettere,
che alla pittura, però non fecero altro acquisto in essa, che quanto bastava per
una certa infarinatura, e per saperne discorrere; egli volendo mostrare questo
lor poco profitto, fa che di lor propria invenzione ritraggano nella detta lor
bandiera una cosa invisibile, come appunto è il Vento.

\begin{description}
  \item[È un flagello] Questo termine significa Infinità, ed Abbondanza grandissima,
ed esprime un numero indeterminato. Vien, forse dai Latino, che tal volta
significa Quantità immensa. Martial. lib. 2. 30. \textit{Et cuius laxas arca flagellat opes},
parlando d'uno che havea gran quantità di danari,
\item[CHE le pagnotte habbiano spaccio] Che s'esiti, che si consumi molto pane. E pagnotta
  se bene non è voce Fiorentina, è nondimeno spesso usata.
\item[MESTARE] Qui val Ministrare, Comandare.
\item[CARTONE] I pittori chiamano Cartone Quella carta grande fatta di più
  fogli, sopr'alla quale fanno il modello di qualche grand'opera, che devono dipignere
  nel muro a fresco, o a tempera, o vero per tessere arazzi.
\item[FANTOCCI] Figure mal fatte. \textit{Pittor da Fantocci} s'intende Pittore da poco,
  appunto come da questa loro impresa vuol l'Autore, che si argomenti che fussero
  questi Signori.
\item[DAL gesso] Cioè dalle figure fatte di gesso. I pittori hanno per costume di
  chiamare dette figure di rilevo, (delle quali si servono per disegnare) col solo
  nome di gesso, senza dir figure, o statue, come si vede nel presente luogo, che
  dice disegnar dal gesso.
\item[LANTERNONE] Arnese noto, che serve a portarvi dentro il lume, e difenderlo dal vento.
\item[BRUCIATAIO] Colui che vende marroni arrostiti alla fiamma, o nel forno,
  che noi chiamiamo Bruciate, donde Bruciataio,
\end{description}

\section{Stanza XXXXVII}

\begin{ottave}
  \flagverse{47}Nanni Russa del Braccio, ed Alticardo\\
Conduce quei di Brozzi, e di Quaracchi\\
Che, perché bevon quel lor vin gagliardo,\\
Le strade allagan tutte co i fornacchi,\\
Hanno a comune un lor vecchio stendardo\\
Da farne a corvi tanti spauracchi,\\
E dentro per impresa v'hanno posto\\
Gli spiragli del di di Ferragosto.
\end{ottave}

Seguitano due altri Gentilhuomini Nanni \textit{Russa del Braccio}, che vuol dire
\textit{Alessandro Brunaccini} ed \textit{Alticardo} che vuol dice \textit{Carlo Dati}; a quali
fa condurre le genti di Brozzi, e di Quaracchi, due luoghi vicini a Firenze, ne i quali nasce
vino debolissimo, e però dice che questi soldati son mal sani; e pieni di catarro,
perché bevono quei vini deboli,  (che egli ironicamente parlando, chiama gagliardi)
che per la loro debolezza danno prima alle gambe, che alla testa.
E perché tali infermi pare che si rihabbiano, e piglino qualche vigore, quando si
trovano all'allegrie; perché fa loro portare una insegna nella quale sono espressi
alcuni di quei bagordi, gozzoviglie, ed allegrie, che già si facevano \textit{il dì di
  Ferragosto}, che s'intende il dì primo d'Agosto, venendo questa voce da Feriare
agosto, e per intelligenza di questo è da sapere, che anticamente solevansi cele
brar le ferie Augustali con grandi allegrie; e ciò si faceva forse, perché essendo
gli huomini nel maggior fervore della state, erano necessitati dal gran caldo a stare
allegramente, perché l'allegria e il primo rimedio della squola Salernitana:
\textit{Haec tria: mens hilaris, requies, moderata diaeta}. Essendo dunque molto pericoloso in quei
tempi d'infermarsi, e perciò molti giorni infausti allora si notavano dagli Egizj,
essendo vicino al Sirio, o Canicula da tutti detta pestifera, come ci mostra Stazio
lib, 1. Silvar, \textit{Illum nec calido latravit Sirius astro}, E' necessario riposarsi, bere, e
mangiare, e stare allegramente; al che consiglia nelle sue Odi Orazio più volte;
Ed habbiamo una cantilena assai praticata, che dice.
\begin{verse}
Quando sol est in Leone,
Bonum vinum cum mellone,
Et agrestum cum pipione.
\end{verse}
E perché veramente il fervore del Sol Leone, o Sirio, e allora nel maggior colmo,
sono le stagioni molto calde; e peggiori, che in tutto l'anno; onde appresso
a' Greci ancora si facevano molte allegrie, e sacrifizzj a segno, che appresso
gli Attniesi secondo alcuni il mese d'Agosto acquistò il nome d'\textit{Hecatombaeon}. Tal feste,
ed allegrie si facevano già a Firenze non solo per la detta ragione, ma ancora per
causa di alcune vittorie ottenute da i Fiorentini in quei primi giorni d'Agosto, e se
ne conserva ancora il costume, ma non si fanno tante feste, quante già si facevano,
poiché solamente si fa correr al Palio alcuni Asini: Sì che s'argumenta, che
il nostro Poeta intenda, che in questa insegna, o stendardo fusse rappresentato il
palio de gli asini, mentre dice spiragli del dì di Ferragosto, che vuol dire un poca
di memoria delle gran feste, che già si facevano in quei giorni.
\begin{description}
\item[SORNACCHIO] Sputo grosso, e catarroso, detto anche farda, Vedi sopra in
questo C. stan. 25. Monsignor della Casa nel suo Galateo dice; \textit{Di soffiamenti di
naso sporcamente, di tirar sornacchi, e sputamenti}.
\item[SPAVRACCHIO] Così chiamiamo quei pannacci, che sopra ad un palo, pertica,
  o albero si mettono per li campi a fine di spaurire i colombi, ed altri uccelli,
  Vedi sotto C. 5. stan. 49.
\item[SPIRAGLIO] Vuol dir fessura in muro, o in tetto, o imposte di usci, o di
finestre, per la quale, trapela l'aria, o lo splendore, che i Latimi dissero \textit{rima}.
In questo luogo però è inteso metaforicamente per Piccola notizia, come è assai
in uso, e forse non lontano da i Latini, che dissero \textit{Spiraculum tantum ius rei ad
me venit} per intendere io ho havuta di ciò qualche notizia,
\end{description}

\section{Stanza XXXVIII}
\begin{ottave}
  \flagverse{48}Gustavo Falbi Cavalier di petto\\
Con Doge Paol Corbi hor n'incammina\\
Gl'Incurabili tutti, e il Lazzeretto;\\
Gente, che uscia di far la quarantina.\\
Van molti a grucce, in seggiola, e nel letto,\\
Perché non sono ancor netta farina;\\
Fan per impresa in un lenzuol che sventola\\
Un Pappino rampante a una pentola.
\end{ottave}

Seguono \textit{Gustavo Falbi}, cioè \textit{Ugo Stufa} Senatore Fiorentino, e lo chiama
\textit{Cavalier di petto}, perché ha la Croce in petto essendo Bali della Religione di
S.Stefano; E l'altro è \textit{Doge Paol Corbi}, che vuol dire \textit{Cavalier Iacopo del Borgo}.
A questi due gentilhuomini fa condurre una quantità di convalescenti, e di stroppiati,
per mostrare, che essi nel tempo; che l'Autore componeva la presente Opera
non erano d'intera sanità per qualche poca d'ipocondria, che gli molestava, e
fa però lor fare per impresa un Servo dello spedale di S.Maria Nuova con le
mani alzate a una pentola.
\begin{description}
\item[INCVRABILI] Così si chiama in Firenze uno Spedale, nel quale vanno a curarsi
  i Maifranzesati.
\item[LAZZERETTO] Luogo, o Spedale in cui si mettono gli huomiai, e robe
  sospette di peste per far lor fare la quarantina, e renderle praticabili, che \textit{Far la
    quarantina} vuol dire Star riserrato in uno di questi luoghi quaranta, o più, o meno
  giorni per spurgar il sospetto d'infezione. E questo nome Lazzeretto viene
  da Lazzero risuscitato da N. Sig. Giesù Cristo, quando era di già fetente il di lui
  corpo.
\item[GRUCCIA] Specie di bastone per gli stroppiati, sopra una teftata del quale
  essendo confitto un legnetto fatto a guisa di mezza luna, si sostiene il corpo mettendo
  detta mezza luna sotto il braccio, e l'altra testata del bastone in terra; e
  perché questo bastone è simile a una croce mi par di poter credere, che la voce
  Gruccia sia corrotta dal Latino \textit{scipio cruciatus},
\item[NON son netta farina] Non sono schietti, non sono affatto sani.
\item[LENZUOL, che sventola] Costoro in vece di bandiera, usano un lenzuolo, e
  ciò per mostrare, che tutte le loro cose sono da spedali; in esso lenzuolo è dipinto
  un'Astante, o Servo dello spedale di S. Maria Nuova, rampante a una pentola,
  cioè con le mani alzate a una pentola, che è in alto; a similitudine del Lione, il
  quale quando si trova dipinto ritto con le branche dinanzi alzate a qualche cosa,
  si dice Rampante. Franco Sacchetti Nov. 133, \textit{Ed hebbero ritrovato per cimiero un
  mezzo orso con le zampe rilevate, e rampanti}.
\end{description}

\section{Stanza IL \& L}

\begin{ottave}\flagverse{49}Bel Masotto Ammirato anch' egli passa\\
Lindo garzon d' ogni virtù dotato,\\
Che può, de' soldi havendo nella cassa\\
Pisciar a letto, e dire : io son sudato;\\
Ma per l'ipocondria, che lo tartassa,\\
Ei si dà a creder d'essere Ammalato;\\
Ma è mangia, beve, e dorme il suo bisogno,\\
Ch'è fino a vespro, e poi si leva in sogno,

\flagverse{50}Con lo scenario in mano, e il mondo fuora\\
Va innanzi a nobil suoi commilitoni,\\
Pancrazio, Pedrolino, e Leonora\\
Lo seguon con un nugol d'Istrioni,\\
C'hanno una insegna non finita ancora,\\
Perché Anton Dei con tutti i suoi garzoni,\\
Incambio di sbrigar quella faccenda,\\
È ito al Ponte a Greve a una merenda.
\end{ottave}

Passa Belmasotto Ammirato, che è Mattias Bartolommei Marchese giovane di bell''aspetto,
ricco, e letterato; il quale fu un tempo, che si persuadeva d'haver tutti
i mali. E perché questo Cavaliere si diletta di comporre commedie, e volentieri
recita in esse lui medesimo, ed appunto nel tempo, che l'Autore accrebbe la presente
Opera, havea detto Signore messa insieme una conversazione di giovani nobili,
che recitavano all'improvviso; però lo fa capo di nobili commedianti, e
gli da uno stendardo non ancor finito, perché \textit{Antonio Dei} ricamatore (e questo
è il vero suo nome, cognome, e professione) in cambio di finirglielo, era andato
a un'allegria al Ponte a Greve, luogo poco lontano da Firenze. Caso seguito
al detto Sig.\ Marchese Bartolommei, che aspettando alcuni abiti per una commedia
, che si dovea far la sera, il Dei in vece di finirgli sen'era andato con tutti
i garzoni della sua bottega fuori di Firenze.

\begin{description}
\item[HAVENDO de soldi nella cassa] Essendo ricco: Non gli mancando denari
\item[PISCIAR a letto, e dire: lo son sudato] E' proverbio assai vulgato, che significa.
Può fare a suo modo, che, o male, o bene che egli faccia, gli è sempre
ascritto a bene; E s'intende d'uno, che sia ricco, e fortunato.
\item[LEVARSI in sogno] Levarsi più presto dell' ere solita di levarsi, quasi dica
S'é levato di notte, sognado esser'hora di levarsi,e qui Autore intende, che a questo
Cavaliere il mezzo giorno, alla quale hora cominciava a destarsi, serviva per aurora,
\item[SCENARIO] È un foglio, sopr'al quale son descritti i recitanti, le scene della
commedia, la quale si dee recitare, ec. i luoghi, per i quali volta per volta devono
uscire in palco i recitanti, afinché quel tale, che assiste gli possa fare uscire
aggiustatamente, ed a i tempi debiti. Tal foglio si domanda anche \textit{Mandafuora}, se
bene il \textit{Mandafuora} è alquanto differente dallo \textit{Scenario}, perché questo s'appicca
al muro dietro alle scene affinché ciascuno recitante lo possa da se stesso vedere,
ed il Mandafuora è tenuto in mano da colui, il quale invigila, che l'opera sia,
recitata ordinatamente; ma tuttavia, come ho detto, s'intende, e si piglia spesso
l'uno, per l'altro.
\item[PANCRAZIO, Pedrolino, e Leonora] Nomi di recitanti nella suddetta conversazione.
\item[NUGOLO a' Istrioni] Gran quantità di commedianti. Questa voce \textit{nugolo}, che
nel presente luogo significa numero infinito, si usa più propriamente parlando di
volatili, perché questi volando gran numero insieme, come farebbono storni,
colombi, ec.\ occupano il sole, ed oscurano l'aria, appunto come fa il \textit{nugolo}. La voce
Istrioni è latina, tolta dall'antico Toscano, come dice Polid. Verg. lik.3-cap.14.
le cui parole son queste. \textit{Et quia Hister Fusco verbo ludus vocabatur, ideo nomen histrionibus
est inditum}, ec. Ma hoggi ce ne serviamo per nome speciale, chiamando
Istrioni solamente i commedianti, che recitano per prezzo.
\item[GARZONI] Intende lavoranti; se ben Garzone vuol dir propriamente Giovane
  scapolo, e senza moglie, come si vede nell'ottava antecedente lindo garzone;
Tuttavia s'intende anche Servitore, o lavorante, che stia a salario in botteghe
di qualfivoglia mestiero.
\item[MERENDA] Specie di mangiare, che si fa tra mezzo giorno, e sera. Vedi
  sopra in questo C, stan. 35,
\end{description}
\section{Stanza LI ... LVI}

\begin{ottave}
  \flagverse{51}Don Panfilo Pilori move il passo\\
Che, tra che per usanza mai sta cheto,\\
Hor ch'ei fa moto fa si gran fracasso,\\
Ch'io ne disgrado il Diavol n'un canneto,\\
Assorda il mondo più d'agn'altro il grasso\\
Papirio Gola, c'appunto gli è dreto,\\
Il qual vestì di lungo, e fu guerriero,\\
Perocché poco gli fruttava il Clero
\end{ottave}
\begin{ottave}
  \flagverse{52}E n'ha fatto con esso de rammanzi,\\
C'un po' di campanile non gli alloga,\\
E questa è la cagion, che là tra i lanzi\\
Da soldato n'andò in Oga Magoza;\\
Ne quivi essendo men tirato innanzi,\\
Posò la spada, e ripigliò la toga,\\
E per lo meglio si risolse al fine\\
Tornar' a casa a queste stiacciatine.
\end{ottave}
\begin{ottave}
  \flagverse{53}Al che tra molti commodi s'arroge;\\
Quel ber del vin; ch'è troppo cosa ghiotta,\\
Qua birre, qua salcraut, qua cervoge,\\
A casa mia dicea, del vin s'imbotta,\\
Però finianla; cedant arma togae:\\
Io non la voglio, in quanto a me, più cotta;\\
Guerreggi pur chi vuol, s'ammazzi ognuno,\\
Ch'io per me non ho stizza con nissuno.
\end{ottave}
\begin{ottave}
  \flagverse{54}Così rinunzia l'armi a Giove, e stima\\
D'esser il più lieto huom che calchi terra,\\
Pensa stato mutar, cangiando clima, \\
Ma trovata l'Italia tutta in guerra,\\
E forzato ferrarsi, più che prima;\\
Ecco il giudizio human come spesso erra\\
Crede tornar fra gente quiete, e gaie,\\
E fugge l'acqua sotto le grondaie.
\end{ottave}
\begin{ottave}
  \flagverse{55}Tra don Panfilo, e lui uno squadrone\\
Dal Pontadera aspettano, e da Vico,\\
Che parte per la via vanno a Vignone,\\
E parte fanne un sonno a piè d'un fico,\\
Costoro empion di rena un lor soffione,\\
E quando sono a fronte all'inimico,\\
Gliela schizzan nel viso, ed in quel mentre\\
Gli piglian gli altri la misura al ventre.
\end{ottave}
\begin{ottave}
  \flagverse{56}L'insegna di costoro è un Montambanco,\\
C'ha di già dato alli suoi vasi il prezzo,\\
E detto che son buoni al mal del fianco,\\
E strolagato, e chiacchierato un pezzo,\\
Ma trovandosi alfin sudato, e stanco,\\
E non havendo ancor toccato un bezzo,\\
Si scandolezza, ed entra in grande smania,\\
Poi dice, che si parte per Germania.
\end{ottave}

Segue Don Panfilo Piloti, che è Ipolito Pandolfini gran chiacchierone, e Papirio
Gola, che e Paolo Parigi, il quale ne i suoi primi anni vestì abito da Prete (che
questo intende col dire \textit{vestì di lungo}) ma poi lo posò, e sen'andò in Alemagna,
alla guerra vedendo, che quell'abito non gli era di frutto; Visto poi, che anche
quel mestiero non gli fruttava, tornò alla patria, e ripigliò l'abito. Ma trovato,
che ancora l'Italia era sottosopra per causa della guerra del Duca di Parma, fu
forzato dal debito di suddito, e dalla convenienza della provvisione, a tornare
alla guerra in servizio del Sereniss.\ Gran Duca, e a lasciar di nuovo l'abito da
Prete. Finita detta guerra il medesimo Paolo Parigi si rimette l'abito, e fattosi
Sacerdote, morì poi Rettore della Chiesa di S.\ Angelo a Vicchio. Questo Paolo
Parigi fu figliuolo di Giulio, e fratello d'Alfonso ambedue Architetti celebri,
come fu ancor'egli, ed Andrea altro suo fratello, che fu Maestro di campo, e
nominato dal nostro Poeta Paride Gurani sotto nel C. 3. stan, 10.

I suddetti due conducono genti dal Pontadera, e da Vico, (Terre vicine a Pisa)
le quali genti dice il Poeta, che \textit{l'aspettano}, perché venendo di lontano per la
stanchezza del viaggio s'erano fermate per la strada a riposarsi; E per mostrare,
che questo \textit{Papirio} era grand'ingegnere, fa che questa gente habbia per arme
un'ordigno per facilitare la distruzione del nimico, il quale e un mantrice pieno
di rena, e per alludere al genio vagabondo di Papirio, ed alle chiacchiere
di Don Panfilo, figura nella loro insegna un Montambanco, che sono genti
chiacchierone, (e però detti anche \textit{Ciarlatani}) e che non hanno patria ferma,
sendo oggi in Firenze, e domani altrove, secondo che gli porta la speranza del
guadagno.

\begin{description}
\item[FRACASSO] Strepito, romore; Vien dal latino Frangere, che vuol dir
  Rompere, e veramente il significato proprio di fracasso e quel romore, che procede
  da frattura, o spezzamento di materiali; se bene si piglia per ogni sorte di
  strepito. Dan. Inf. C. 9.
  \begin{verse}
    già venia fu per le torbide onde
    Un fracasso d'un suon pien di spavento.
  \end{verse}
  E nel Purg. Cant, 14,
  \begin{verse}
    ecco l'alzra con si gran fracasso
  \end{verse}
  Dove l'espositore Landini dice, che Fracaffo vien dal verbo frangere.
\item[NE disgrado il Diavol n'un canneto] Farebbe manco romore il Diavolo in un
  postime di canne. Si figura il diavolo, per lo più, un'huomo con le corna, con
  l'ali, e co i piedi di gallo; onde si dice un \textit{Diavol n'un canneto}, perché si suppone,
  che passando il detto diavolo dentro a un postime di canne, pigli con le corna,
  con l'ali, e con gli artigli le canne, le quali scappando dalle dette corna;
  ali, ed artigli a guisa di molla, perquotono nell'altre canne, che per esser vote
  fanno strepito, e rimbombo non piccolo. Quand'uno s'affatica per conseguir
  qualcosa diciamo: \textit{Il tale ha fatto il diavolo per haver la tal cosa}, e s'intende \textit{ha
  fatto il diavol n'un canneto}, cioè gran romore, Il termine; \textit{Ne disgrado} Vuol dire
  lo stimo manco: lo levo il luogo, o grado: per esempio \textit{Il tale compone versi Latini
  così bene, che io ne disgrado Vergilio}, cioè io stimo, che questo tale habbia tolto
  il luogo a Vergilio, e faccia meglio di lui. Vedi sotto Cant, 3. stan. 34. C. 6.
stan. 61.¢ C. 7, stan. 25.

\item[RAMMANZO] Far un rammanzo, o rammanzina vuol dire, Riprender' uno,
  con minacce; e suona lo stesso, che far' un rabbuffo, o Rabbuffare detto sopra in
  questo C. stan. 39.
\item[NON gli alloga un po' di campanile] Piglia la parte per il tutto, e vuol dire Non
gli fa conseguire una Chiesa.
\item[LANZI] Così chiamiamo i Soldati a piedi guardie del Sereniss. Gran Duca, i
quali son tutti Alabardieri Tedeschi: E pero dicendo: \textit{Andò fra i Lanzi} intende
Andò fra i Tedeschi, cioè in Alemagna; la voce Lanzi e Todesca lasciataci da
loro medesimi, che in salutarsi sogliono chiamarsi \textit{Lantzman}, che suona Paesano;
e \textit{Lanzchnect} vuol dir soldato a piede, e per questo gli Scrittori Fiorentini si
servono della voce \textit{Lanzichenecchi}, per intendere Soldati Alemanni a piede. Ed
il Varchi storie Fiorentine lib. 2, dice così: \textit{Quanto più s'avvicinavano i Lanzi, che
così per maggior brevità gli chiameremo da qui avanti, e non Lanzichenecchi, ec}.
\item[OGA magoga] Quand' uno va lontano dalla sua patria, dicono le nottre donne,
  \textit{Gli è andato in Oga magoga}, Ed intendono gli è andato a casa maladetta, nel
  qual senso è preso anche nella sacra scrittura; e S. Gio; nell'Apocalisse al 20,
  dice \textit{Og magog, \& congregabit eos in praelium}. Ed al cap. 7. dice \textit{In
    dispersionem gentium}, e si trova anche in altri libri della Sac. Bibbia. Vedi Angel. Mons.
  Fio. Ital. linguae alla parola oga magoga. Dicono ancora \textit{Gaga magoga}. E forse
  intendono dei Regno di Goaga in Affrica. Il Vocabolista Bolognese dice, che Og fu
  gigante d'Astarotte Rede Baraniti, della creazione del Mondo 2492, contro al
  popolo d'Israel ne i campi d'Edrai, ove fu destrutto con tutto il suo esercito, e
  cinquanta Città; e che di qui venne il significato Andare in dispersione, e in fumo.
  o a casa del Diavolo, essendo interpetrato Og magog, per il Diavolo. Sin qui
  il Vocabolista. Gli antichi secondo Plinio chiamavano Magog la Città d'Edessa,
  (che Strabone dice, che è l'istessa, che Hierapoli) dove era il celebre Tempio
  della Dea Atergatide detta la Dea Siria, e dove gli Ebrei vissero in cattivita, onde
  da questo dicendosi Andare in Magog, per gli Ebrei era lo stesso che dire:
  Andar' in servitù. Gio: Villani Stor. Fior. lib. 5. Cap. 29. dice: \textit{Le genti, che si
  chiamano Tartari uscirono dalle Montagne di Gog Magog chiamate in latino monti di
  Belgen}. Conchiudo dunque, che non dire \textit{andò in Oga Magoga}. Significa Andò
  in paesi lontanissimi, e di pericolo: ed è quasi lo stesso, che dice \textit{Andò a Buda},
  che vedremo sotto Cant. 5. stan. 13.
\item[TIRATO innanzi] Avanzato a gradi, a dignità, a utili, ec.
\item[TOGA] Vuol dir propriamente abito da Dottori, ma si piglia bene spesso per
l'abito da Prete, come è presa in questo luogo.
\item[TORNAR a casa a queste stiacciatine] Tornare a goder'i comodi della propria
  casa, che si dice anche: Tornare al Pentolino, che i latini dissero: \textit{Redire ad
    pristina Praesepia}. Stiacciatina è diminutivo di Stiacciata, la quale è specie di pane, che
  dopo lievito si stiaccia con le mani per farlo più sottile, affin che si quoca più presto,
  e faccia minor midolla.
\item[S'arroge] ll verbo Arrogere vuol dire aggiugnere. Al che \textit{s'arroge}; al che
  s'aggiugne, e vuol dire; Ci è anche di più. Il Lasca Nov.~5.
\begin{verse}
  E così per non arroger peggio al male, si stava quieta, ec,
\end{verse}
Petr. Canz, 9.
\begin{verse}Eduolmi, c' ogni giorno arrage al danno.
\end{verse}

\item[COSA ghiotta] Cola desiderabile, cosa appetitosa; che \textit{ghiotto} si dice Vno avido
  di mangiar del buono; e viene da \textit{indulgere gutturi}.

\item[SAL craut] Cavolo salato. Voce, e vivanda Tedesca.

\item[BIRRA] o \textit{Cervogia}, Bevanda, che s'usa in Alemagna, ed in altri paesi,
dove è poco Vino; ed è composta di biade, acqua, e fiori di luppoli; ed è lo
stesso \textit{Birra}, che \textit{Cervogia}, e questa ultima è dal Latino.

\item[IMBOTTARE] Metter nella botte. Se bene qui si potrebbe intendere Bere,
costumandosi dire: \textit{Io non imbotto acqua}, in vece di dire: Io non bevo acqua, si
come è inteso sotto C, 7. stan. 4.

\item[NON la voglio più cotta] Per la mia parte mi basta così,ne mi curo di meglio.
Sum presenti Catone contentus, dilic Auguito.
\item[STIZZA] Ira, collera; e vale anche per Inimicizia.

\item[FERRARSI] Intende Armarsi. È detto scherzoso, perché Ferrare, senza dir
più s'intende mettere i ferri all'unghie de' piedi de' cavalli, muli, ed altre
bestie.

\item[GENTI gaie] Genti allegre, ricche, e abbondanti d'ogni comodo, e quiete;
che la voce Gaio è forse sincopata da Gandio.

\item[GRONDAIE] Quel cascare, che fa l'acqua da i tetti, quando piove; e si
dice Grondaia da Gronde, che sono quelle tegole più larghe, le quali son poste
nell'estremità de' tetti. Ed il Proverbio \textit{Fuggir l'acqua sotto le grondaie} vuol dire;
Procurar di fuggire un pericolo, e andarli incontro, che è quello forse, che i Latini
intesero col dire \textit{Incidit in Scyllam cupiens vitare Charybdim}.

\item[ANDARE a Vignone] Andar nelle vigne altrui a corre l'uva; e si dice così
per rendere il detto oscuro, mostrandosi d'intendere d'Avignone in Francia, o
del Bagno di Vignone, che è nello Stato di Siena.

\item[SOFFIONE] Quel piccolo Mantaco, o Mantice, del quale comunemente ci
  serviamo per soffiar nei fuoco, usandolo a mano.

\item[SCHIZARE] Qui è verbo attivo, e vuol dice: Gli gettano con violenza nel
  viso quella che è dentro al soffione.

\item[MONTANBANCO] Uno di coloro che vendono i rimedj nelle pubbliche piazze,
  detti \textit{Montambanchi} dal montare sopra i banchi quando vogliono vendere;
  e detti anche \textit{Ciarlatani} dalle gran ciarle, che sogliono fare.

\item[TOCCATO un bezzo] Preso, o buscato un quattrino. \textit{Bezzo} è moneta, e
  Parola Veneziana, ma usiamo, se non la moneta, almeno la voce \textit{bezo} ancor noi
  per intender Denari in generale.
\item[SI scandolezza] In questo luogo, ed in questi termini significa Adirarsi, e
  mostrar con le parole, e con gli atti la collera, che uno ha. Vedi sotto C. 11. stan.
  23. Verbo che viene dal Greco \textit{scandalizesthai} che suona, a loro, come a noi
  Offendersi, o adirarsi d'una cosa.

\item[ENTRAR in smania] Entrar in grandissima collera; che Smania è una soverchia
  inquietudine, cagionata da febbre, o da eccessivo caldo, o da soverchio
  amore, la quale riduce l'huomo quasi insano, e furioso.
\end{description}

\section{Stanza LVII \& LVIII}
\begin{ottave}
  \flagverse{57}Huomini bravi quanto sia la morte\\
Scandicci n'ha mandati, e Marignolle,\\
Gente, che si può dir che habbia del forte,\\
Poi ch'ella ammazza gli agli e le cipolle,\\
Sue lance i pali son, targhe le sporte,\\
Airchiusi le man, le palle zolle,\\
Va ben di mira, e colpo colpo imbreccia,\\
Maffime quand'altrui vuol dar la freccia,

\flagverse{58}Vien comandata da Strazildo Nori,\\
Ch'è Chimico, Poeta, e Cavaliere,\\
Ed è quel, ch' in un quadro co i colori\\
Fece quei fichi, che divenner pere.\\
E perché questo è il Re de bell'humori,\\
Per dimostrar quanto gli piaccia il bere; \\
Ha per impresa un Lanzo a due brachette;\\
Ch'il molle insegna trar dalle mezzette.
\end{ottave}

Seguita la gente di Scandicci, e di Marignolle, Ville vicine a Firenze, dove
nascono Cipolle, Agli, ed altri fortumi simili in grande abbondanza. Questa
gente dice che è \textit{brava quanto la morte, perché ella ammazza gli agli, e le cipolle, e
si può dire che habbia del forte}, E pare che intenda che ella superi in fortezza, e
bravura gli agli: E vuol poi dire, che ha molti fortumi, ed Ammazza, cioè Fa
mazzi delle cipolle, e degli agli. E perché questi contadini habitando intorno
a Firenze praticano molto la Città, dove è occasione di spendere più che nel
contado, dice l'Autore, che son genti che \textit{danno la freccia}, che vuol dir Chieder
denari in presto; e par ch' ei voglia intendere che son bravi tiratori di freccia,
e d' archibuso. Son comandati da \textit{Strazzildo Nori}, cioè Rinaldo Strozzi Cavaliere
di S. Stefano; ed è quello, che in squola dell'Autore volendo dipignere
alcuni fichi non trovò mai il modo di fare, che non paressero pere. Questo fu
un geatilhuomo di grandissimo garbo, faceto, allegro, e spiritoso, e buon bevitore;
e perciò gli fa fare per impresa un Lanzo, che vota una mezzetta di vino,
e gli fa comandare questa gente, perché fu poi P...... in vicinanza dei
lor paesi.

\begin{description}
\item[SPORTA] Specie di paniere fatto di giunchi, ed ha due manichi; serve per
portarvi dentro erbaggi, ed altro, che si provvede in piazza giornalmente per il
Vitto.
\item[ZOLLA] Gleba, pezzo di terra sollevata nel lavorare i campi, Vedi sotto
in questo Canto stan, 82.
\item[COLPO colpo] A ogni colpo. Intendi: sempre ch' ei tira; colpisce, che la forza
  della replica e di far nascer il superlativo.
\item[IMBRECCIA] Forse meglio \textit{imbercia}; E Significa Pigliar di mira; donde
  \textit{imberciatore} colui che fa professione di tirar d'archibuso; e par che venga da
  sbirciare, e bircio, che è guardar con occhi socchiusi, come dicemmo sopra in
  questo C, stan. 9. e come s'usa a tirar con l'archibuso. Ma puo anche essere che
  venga da breccia che vuol dir Quelle rotture che vengon fatte nelle muraglie
  dall'artiglierie, e si dica imbrecciare per colpire, si come intende nel presente
  luogo pigliando colpire in senso di conseguir l'intento.
\item[DAR la freccia] Come habbiamo accennato, vuol dire Chieder denari in presto;
  e s'intende Vno che habbia poco modo, e minor voglia di rendergli. Gli
  antichi Etiopi, e gli abitatori di Maiorca, ec. non solevano dar mangiare alli
  loro figliuoli, se questi con le frecce non facevano cascare dallo stile, o albero
  il cibo, che vi era posto, ond'io stimo, che questo frecciar per vivere habbia dato
  origine al presente detto. Vedi Alex. ab Alex. dier. gen. lib. 2. c. 25. Il Monosino
  dice, che questo \textit{frecciare} habbia origine dal Latino \textit{ferire} che appresso
  loro haveva il medesimo significato, e lo cava da Teren. in princ. Phormionis:
  \textit{Porro autem Geta Ferietur alio munere ubi hera pepererit}. Diciamo; i denari sono il
  secondo sangue; dar ferita cava il sangue, come il dar frecciate, cava il sangue;
  e per questo dicendo \textit{dar freccia} intendiamo Dar freccia alla borsa, e cavare questo
  secondo sangue, che è il danaro.
\item[BELLUMORE], Huomo allegro, faceto, ec. vedi sopra in questo C. stan. 10.
Quando diciamo, Il tale è Re della tal cosa; intendiamo Vale in superlativo
grado in quella tal cosa; onde \textit{Re de belli humori} vuol dire Grandissimo bell'humore.
Significato che viene da i Greci, i quali chiamavano Re colui, che nei
giuochi fanciulleschi vinceva, e superava gli altri, ed Asino, o Mida era chiamato
colui che perdeva; il che più diffusamente vedremo nel 2. Canto.
\item[LANZO a due brachette] Lanzo dicemmo sopra, che vuol dir soldato Tedesco
  a piede; ma qui vuol che s'intenda uno proprio di quelli della guardia del Serenissimo
  Gran Duca; dicendo a due brachette, perché questi tali Lanzi vanno vetiti
  a livrea, con un paro di brache larghe, fatte a strisce, come son quelle delli
  Svizeri del Papa in Roma, e come quelle de' Trabanti dell'Imperatore.
\item[INSEGNA trarre il molle dalle mezzette] Insegna col suo bere, come si fa a votare
  i vasi pieni di vino, Che \textit{mezzetta} è un vaso fatto di terra invetriata, che
  serve per misurare il vino, ed è capace della quarta parte d'un fiasco Fiorentino.
\end{description}

\section{Stanza LIX \& LX}
\begin{ottave}
  \flagverse{59}Morbido Gatti, Henrigo Vincifedi\\
A far venir innanzi ecco son pronti\\
I fanti, che ne dà il Ponte a Rifredi,\\
Che mille sono annoverati, e conti.\\
Han certi Santambarchi fino a piedi,\\
Che chiaman' il zimbel di là da monti,\\
E paion con la spada in su le polpe\\
Un che facia lo strascico alla volpe.
\end{ottave}

\begin{ottave}
\flagverse{60}Nell'insegna han ritratto u' huom canuto,\\
Che troppo havendo il crin (per osser vecchio)\\
Fioccoso, e lungo, un fanciullino astuto\\
Dietro gli grida: Gli abbrucia il pennecchio.\\
Da questa schiera qui s'è provveduto\\
Gran ceste piene d' huova, e di capecchio\\
Con fasce, pezze, e taste accomodate\\
Per farsi alle ferite le chiarate.
\end{ottave}

Passa l'ultima truppa di Soldati, la quale è composta d'huomini dal Ponte a
Rifredi, che è un luogo vicino a Firenze. Costoro son comandati da \textit{Morbido
  Gatti}, cioè \textit{Migiotto Bardi}, e da \textit{Henrigo Vincifedi}, che è \textit{Vincenzio
  Sederighi}, due gentilhuomini già scolari dell'Autore: E perché questi si pigliavano gusto di
ragionare spesso con un tal Dottor Cupers, glielo fa fare per impresa.

A Questo Dottor Cupers negli ultimi anni della sua vita, che durò sopra ottanta
anni, entrò in frenesia d'esser bello, e si persuadeva che ogni donna s'innamorasse
di lui, e lo volesse per marito, e però andava lindo, e con la chioma
folta, e lunga, e ben coltivata; ma canutissima: onde i ragazzi quando passava
per le strade gli gridavano dietro: Guarda il Pennecchio, gli abbrucia il Pennecchio,
intendendo di detta sua chioma, e lo facevano adirare, e maggiormente
impazire. E perché li contadini del Ponte a Rifredi si danno a credere d' haver
maggior Civiltà degli altri contadini per esser nati, ed allevati, si può dire, nei
Borghi di Firenze, ed intorno alla Petraia, e Castello, Ville spesso habitate
da Principi della Serenissima Casa, perciò per lo più vengono alla Città col
ferraniuolo, o santambarco, che sono le Toghe de i Barbassori, e Dottori
del Contado; e per questo il Poeta dice \textit{Han certi Santambarchi fino a piedi, Che
  chiamano il Zimbel di là da' monti}, cioè incitano i ragazzi a dar loro delle Zimbellate.
E per esser questa l'ultima schiera fa, che ella conduca seco il bagaglio
de i medicamenti per l'Esercito.
\begin{description}
\item[SANTAMBARCO] Specie d'abito, o sopravveste, o diciamo mantello
usato da i nostri contadini per difendersi dall'acqua, e dal freddo; ed è composto
di due larghe strisce di panno cucite in forma di croce con una buca in mezzo,
per la quale passano il capo, e vengono coperti da una parte di detto panno le
schiene, e il petto, e dall'altra le braccia, e i fianchi, Si dovrebbe dire \textit{Salta in
barco}, e così dice Mattio Franzefi nel Capitolo del suo viaggio da Roma a
Spoleto.
\begin{verse}
\backspace Gli osti, c'a profferir mai non son parchi
Volean ch'io scavalcassii a sì mal tempo,
E m'offerivan fuoco, e Saltambarchi.
\end{verse}

Ed è forse meglio detto \textit{Saltambarco}; perché questo abito è composto in tal
forma; che tiene tutta la persona difesa dal freddo, e non l'impedisce il saltare
i fossi, e passare i barchi. Ma si dice \textit{Santambarco} perché così lo chiamano i contadini
che se ne servono, ed è lor abito proprio.
\item[CHIAMAR una cosa di là da i monti] Questo termine significa Meritare una
  cosa grandemente, come per esempio \textit{Il tale è così insolente, ch'ei chiama le bastonate
  di là da i monti}.
\item[ZIMBELLO] In questo luogo intende un sacchetto pieno di crusca;
o di cenci, o di segatura, legato a una cordicella lunga circa due braccia,
col quale i fattorini delle botteghe de setaiuoli nel tempo del Carnevale, quando
passano i contadini per quei luoghi, dove sono le botteghe de i setaiuoli, uno di
loro perquote il contadino; e mentre questo si volta per veder chi ha percosso,
gli altri ragazzi lo perquotono dall'altra banda: E questo per lo più vien fatto a
certi contadini, che se ne vengono in Firenze intronizzati, e in sul grave, come
appunto fanno quei del Ponte a Rifredi. E per altro la voce Zimbello ha il significato,
che vedremo sotto C. 7. stan. 76.
\item[FAR Io strascico alla Volpe] E' una specie di caccia, che si fa alla Volpe, pigliando
  un pezzo di carnaccia fetida, che legata a una corda si va strascicando per
  terra; per far venir la Volpe al fetore di essa Carne; ed il Poeta assomiglia il portar
  della spada di questi Contadini a questa corda, dicendo che stava pendente
  \textit{in su le polpe} (cioè dietro alle gambe, che così chiamiamo cotesta parte) appunto
  come sta la fune di colui, che fa lo strascico alla Volpe.
\item[PENNECCHIO] Qui è preso per chioma, ò Zazzera, come habbiamo accennato
  sopra, metaforico da quell'involto di lino, stoppa, lana, o altra materia
  simile, che adattano le donne sopr'alla rocca per filare, il quale involto si dice
  Pennecchio.
\item[QUESTA schiera qui] La voce \textit{qui} è superflua, bastando per farsi intendere il
dir solamente \textit{da questa Regina} senza aggiungere la particella \textit{qui}: Ma non per
questo il nostro Poeta ha fatto errore, havendo seguitato il nostro Fiorentinismo
usatissimo. Dicendosi comunemente (forse a maggior' emfasi) \textit{Questo negozio qui},
\textit{questa cosa che è qui}, e simili; e la particella \textit{qui} esprime \textit{il negozio, del quale ragioniamo presentemente}, \textit{Questa cosa, la quale habbiamo fra le mani}: Anzi stimo, che
l'habbia fatto ad arte, e per mostrare questo nostro modo di dire, (forse riprensibile)
del quale non mi pare, che in tutta l'Opera si sia servito mai più; quantunque
non gli sieno mancate l'occasioni; E se bene nell'Ottava 65. seguente,
pare, che l'usi nel medesimo modo, osservisi, che quivi è termine dimostrativo
necessario, e non riempitivo, operando che s'intenda di quella Cugina, che è lì
presente, e non d'altra, come si potrebbe intendere, se non vi mettesse la particella
\textit{qui}.
\item[CESTA] Intendiamo un gran paniere, che fa mezza soma di bestia, ed è contesto
  d'assicelle di castagno, o d'altro legname a foggia di cassa, per uso di portare
  da un paese all'altro uova, vino in fiaschi, ed altre cose frangibili; e per lo
  più son fabbricati due attaccati l'uno all'altro con quattro legni gagliardi aggiustati
  in maniera da adattarsi sopra i basti a traverso alla bestia, in modo che tengono
  equilibrate, e ferme dette due ceste anche senza legarle. Se ne fabbricano
  ancora della stessa forma, e materia sciolte, cioè senza i detti quattro legni, e
  queste s'adattano, e fermano in su i basti con le funi, come si fa i Cestoni, che
  sono ancor'essi panieroni di mezza soma fatti di vinciglie di castagno, o altro albero
  intessute, de i quali si parla sotto C.\ 10.\ stan.\ 7.
\item[CAPECCHIO] La pettinatura, cioè quella stoppa più grossa, che si cava dal
  lino sodo la prima volta, che si pettina detta capecchio, perché si cava dai due
  capi del lino, cioè barbe, e cime, le quali sono più ripiene d'immondezze, e di
  filo morto, e inutile.
\item[FAR la chiarata] Il primo medicamento, che si faccia alle ferite è l'albume,
o chiara d'huovo, entro alla qual chiara s'intigne il capecchio, e si pone sopra
alle ferite; E questo si dice \textit{far la chiarata},
\end{description}

\section{Stanza LXI}

\begin{ottave}
\flagverse{61}E' general di tutta quella mandra\\
Amostante Laton Poeta insigne\\
Canta improvviso, come una calandra,\\
Stampa gli enigmi, strolaga, e dipigne.\\
Lasciò gran tempo fa le polpe in Fiandra,\\
Mentre si dava il sacco a certe vigne, \\
Fortuna, che l'havea matto provato\\
Volle, ch' ei diventasse anche spolpato.
\end{ottave}

Generale di tutto questo esercito e Amostante Latoniy, cioè \textit{Antonio Malatesti}
Poeta celebre per molte sue opere, ma specialmente per quella Sfinge, la quale,
come vedremo sotto C. 8. stan. 26. è una scelta d'enigmi in sonetti, de' quali se
ben la stampa ne fa goder pochi, se ne sperava numero maggiore, volendone
egli pubblicare 400. scelti da una infinità, che ne ha composti; ma la di lui morte
seguita poco tempo fa, ci priva per ora di questa consolazione. Ne gli anni suoi
giovenili cantò all'improvviso molto lodatamente, si dilettò d'Astrologia, e nel
disegno fu scolare dell'Autore, e suo amicissimo, come mostra, facendolo capo,
e saperiore di tutti gli amici suoi, che nomina in questo esercito. E perché questo
Amostante era di corpo adulto, ed havea le gambe sottili, dice, che \textit{lasciò le polpe
in Fiandra}, e che \textit{la Fortuna che l'havea provato matto}, volle che egli diventasse
anche \textit{spolpato}, cioè senza polpe; ma aggiunto alla voce \textit{matto} vuol dire
\textit{matto affatto}; non che Amostante fusse affatto privo di cervello; che la voce
\textit{matto} appresso di noi significa ancora Allegro, Faceto, e simili, nel qual senso è presa
nel presente luogo; e però vuol dire, che Amostante era huomo facetissimo.
\begin{description}
\item[MANDRIA] Vuol dire Una gran quantità di bestie; ma qui intende Grani
  quantità d'huomini. Mandra è voce Greca, che suona Spelonca, e luogo, entro
  al quale le pecore s'adunano all'ombra, ma la pigliavano anche per la greggia
  medesima, e da essa dissero Archimandrita il governatore della greggia.
  Dante pure prese \textit{Mandria} per quantità di huomini, nel Purg. C. 3.
  \begin{verse}
    Sì vidd' io muovere, e venir la testa
    Di quella Mandria fortunata allotta,
    Pudica in faccia, e nell'andare onesta,
  \end{verse}
\item[CANTA improvviso] È costume in Firenze al tempo de i gran caldi la notte
  cantare dell'ottave all'improvviso, mentre ne i luoghi più aperti della Città si
  va pigliando il fresco; e perché in tal'esercizio valeva molto il Malatesti; il Poeta
  l'assomiglia alla Calandra uccello di bellissimo cantare.
\item[ENIGMI] Indovinelli. Voce Latinogreca. Vedi sotto C.6, stan.34.c C.8 stan. 26.
\item[LASCIO' le polpe in Fiandra] Non è, che Amostante fusse mai stato in
  Fiandra; ma, perché lo fa generale di questo esercito, è dovere, che egli mostri,
  che Amostante ha vedute, e provate altre guerre, e che egli si sia trovato a
  dar de' sacchi, ne i quali ha lasciate le polpe delle gambe, il che serve per accreditarlo,
  poiché si come ad un soldato gli stroppj, e le cicatrici son di gloria, così
  ad Amostante era di gloria  haver perduto le polpe delle gambe nelle guerre di
  Fiandra; ma il vero è, che quand'uno hale gambe sottili, diciamondi lui: \textit{Egli
  ha lasciato le polpe in Fiandra}: ed il Poeta con questo equivoco, che accredita
  Amostante, vuol dire, che egli haveva le gambe sottili; e seguita con l'altro
  equivoco di \textit{matto spolpato}, che significa, come s'è detto,  matto del tutto, e
  vuol che s'intenda \textit{senza polpe affatto}. E la voce polpa, che significa ogni pezzo, o
  quantità di carne, che sia senz'osso, da noi si piglia per le polpe delle gambe,
  quando è detta assolutamente. (Vedi l'ottava 59. antecedente; E sotto al C.6.
  stan. 99. dice \textit{ossccia senza polpe}, che s'intende tutta la carne di quel'corpo) e
  significa pure \textit{Matto spacciato}.
\section{Stanza LXII}

\begin{ottave}
\flagverse{62}Passati tutti con baule, e spada\\
Serransi in barca, come le sardelle;\\
Gli affretta il Duca, e chi lo tiene a bada,\\
O ferma un passo; guai alla sua pelle,\\
Ch'ei lo bistratta, e come che ne vada\\
Giù la vinaccia, e il sangue a catinelle,\\
E ben che lesto ciaschedun rimiri,\\
Non gli dà tanto tempo ch'ei respiri.
\end{ottave}

Dopo fatta la mostra se n'entra la soldatesca nelle barche con ogni suo arnese,
e Baldone affretta all'imbarco i soldati.
\begin{description}
\item[BAVLE] Intendiamo ogni sorte di cassetta, valigia, o tamburo, che facilmente
  si possa adattare in su la groppa d'un cavallo, mentre si viaggia. Viene
  dal verbo \textit{baiulo}, e l'allarghiamo ad ogni sorta di cassa portatile in su le some, ec.
  Qui intende quell'involto, che portano i soldati sopr'alle reni per lor proprio
  bagaglio, detto altrimenti zaino.
\item[SERRANSI, come le sardelle] Si serrano strettissimi appunto, come stanno le
  sardelle ne i cestoni, quando da Livorno son portate a Firenze, o nei bariglioni,
  quando ci vengono salate. Comparazione assai usata per intendere stetti, e
  serrati insieme, che in voce marinaresca si dice stivati.
\item[TENERE a bada] Trattenere uno. Varchi stor, lib, 4. \textit{Conoscevano, che erano
tutte cose finte, e solo per tenere a bada trovate}, Viene dal Verbo \textit{Badare}, che ha
  molti significati. \textit{Badare} al negozio per \textit{Attendere al negozio}. Significa
  Indugiare, o perder il tempo, come è inteso nel presente luogo, che dice \textit{tiene a bada},
  ed intende, Chi gli è causa d'indugio, o gli fa perder tempo; il Petrarca Son.23.
  \begin{verse}
    Consolate lei dunque, che ancor bada.
  \end{verse}
 Cioè aspetta la venuta del Pontefice, e perde tempo. Significa ancora \textit{continuare}, o
 \textit{seguitare} a far una cosa, Vedi sotto C.1, stan. 20. Significa \textit{Osservare} C.9.
 stan. 28.  Significa \textit{Disprezzare}, \textit{non curare}, per esempio; \textit{Io non bado al tuo gridare}. Intende
\textit{io non stimo, o non curo il tuo gridare}, Da questo \textit{badare}, o \textit{bada} habbiamo \textit{badalone}
che vuol dire Vn' huomo perdigiorno, e che non sa, e non vuol far nulla.
\item[GVAI alla sua pelle] Mal per lui. Vedi sopra in questo C.\ stan.\ 28,
\item[BISTRATTARE] Trattar male, Strapazzare, o Stranare.
\item[VA giù la vinaccia] È necessario far presto per sfuggire il danno, che si patisce
  e che si teme più grave dall'indugio. Quando il mosto, cioè il liquore cavato
  dall'uva, il quale è nel tino, ha bollito a bastanza; perde il vigore, e non
  può più sostenere a galla, cioè nella sua superficie, la vinaccia (che così si chiamano
  i raspi, e bucce dell'uve) onde la lascia cascare in fondo, ed incorporandosi
  con essa di nuovo, si guasta; E questo si dice \textit{andar giù la vinaccia}; che
  poi passato in proverbio significa Quel che habbiamo detto.
\item[NE va il sangue a catinelle] Ne va molto del mia. Per intender, che Un'indugio
  apporta grave dispendio, ci serviamo di questo detto; e si dice anche: \textit{a bigonce}.
  Vedi sotto C.\ 10.\ stan.\ 20.
\item[LESTO] Qui vuol dir Pronto, ed all'ordine.
\item[NON gli da tempo che respiri] Non gli lascia ripigliare il fiato. Questo detto
  esprime un grande affrettamento, o incalzamento.

\end{description}


\section{Stanza LXIII \& LXIV.}
\begin{ottave}
\flagverse{63}Perciò imbarcati tutti in un momento,\\
Poi che Baldon facea così gran serra,\\
Si spiegaron l'insegne, e vele al vento,\\
Quando le Navi si spiccar da terra;\\
Ed egli allora entrò in ragionamento\\
Di quel che lo spingeva a far tal guerra;\\
Ma per contarla più distesa, e piana,\\
Incominciò così dalla lontana.
\end{ottave}

\begin{ottave}
\flagverse{64}Risiede Malmantil sour' un poggetto,\\
E chiungue verso lui volta le ciglia\\
Dice, ch'i fondatori hebber concetto\\
Di fabricar l'ottava meraviglia,\\
L'ampio paese poi, ch'egli ha soggetto\\
Non si sa, vuo giuocare, a mille miglia;\\
V'è l'aria buona azzurre oltramarina,\\
E non vi manca latte di gallina.
\end{ottave}

Fatta la mostra, ed imbarcate in brevissimo tempo le soldatesche, si partirono
le Navi dal lido e fecero vela spiegando le loro insegne. Intanto Baldone dà
principio a narrare la causa, che lo muove a far la guerra di Malmantile, e comincia
dal descrivere la situazione, qualità, e dominio.

\begin{description}
  \item[FAR serra] Affrettare. In alzare. Vedi sotto C. 9, stan. 13.
\item[CONTARLA difesa, e plana] Intendi, Raccontarla puntualmente, e con
tutte le circostanze,
\item[NON si sa uno giuocare a mille miglia] Io giuoco, che non si trova chi sappia,
o possa giudicare a mille miglia, quanto paese gli è suggetto; perché è così gran
paese, che mille miglia non si considerano, essendo parvità di numero, e di materia
in riguardo del tutto, che gli è suggetto. E questa voce \textit{suggetto}, che vuol
dir \textit{sottoposto}, s'intende Situato sotto, e non sottoposto al dominio di Malmantile,
che per esser Posto nella sommità d'un poggetto, ha d'attorno molta pianura,
e colline sottoposte, cioè più basse di lui; se ben par, che voglia dire, che
Malmantile ha dominio immenso.

\item[ARIA azzurra oltramarina] I pittori dicono buon'aria quella, la quale e colorita
  con l'azzurro oltramarino, perché questo non perde mai il colore, come
  perde l'indaco, e lo smalto; ma è però anche vero, che quando l'aria si vede di
  colore azzurro, come è il buono oltramarino, è segno, che è purgata da ogni
  imperfezione di nebbia, o d'altri maligni vapori, e per conseguenza e aria buona;
  il Poeta però dice, che a Malmantile è aria azzurra oltramarina per intendere,
  che a Malmantile è aria, che dura sempre azzurra, come fa quella colorita
  con l'oltramarino, cioè sempre buonissima. E \textit{L'oltramarino} è quel colore, che si
  cava dalla pietra detta Lapislazzuli.
\item[NON vi manca latte di gallina] Vi sono tutte le cose squisite, è abondante d'ogni
  bene. Detto antico, si come si cava da Strabone lib, 14., dove discorrendo delle
  campagne di Samo dice, che erano così fertili, che si diceva comunemente,
  che producessero fino il latte di gallina, cioè quelle cose, che e impossibile, ch'altrove
  si trovino, come è il latte di gallina. \textit{Samus}, dice egli, \textit{feracissima, unde
laudantes non dubitant illud ei proverbium accommodare, quod ferat etiam Gallinae
lac}, ec.
\end{description}

\section{Stanza LXV \& LXVI}
\begin{ottave}
  \flagverse{65}Il Re di questo Regno giunto a morte\\
La mia Cugina qui, che fu sua Donna\\
(Non havendo figliuoli, o altri in Corte\\
Propinqui più) lasciò donna, e Madonna:\\
Ma come volle la sua trista sorte,\\
Un certo diavol d'una Mona Cionna\\
Figliuola d'un guidone ignudo, e scalzo\\
Ne venne presso a farie dar lo sbalzo.
\end{ottave}

\begin{ottave}
  \flagverse{66}Gobba, e zoppa è costei, e mancina,\\
Ha il gozzo, e da due sfregi il vifo guasto,\\
Scorse in Firenze ognor la cavallina\\
Ne i lupanari con gran pompa, e fasto,\\
E perché ossequij havea sera, e mattina,\\
E il titol di Signora a tutto pasto,\\
Fatta arrogante, al fine alzò il pensiero\\
A voler questi onori da dovero.
\end{ottave}

Narra Baldone, che il Re di Malmantile instituì Celidora erede del Regno, e
che questo le fu usurpato da Bertinella, la quale descrive per una donna tutta
contraffatta, e la mostra una vera sgualdrina: ed imita Dante nel Purg. C.19.
che dice.;
\begin{verse}
  Mi venne in sogno wna femmina balba,
  Con gli occhi guerci, e sopra i piè distorta,
  Con le man monche, e di colore scialba.
\end{verse}

Qui è da considerare, che i tanti difetti da Baldone attribuiti a Bertinella,
realmente in lei non fussero, perché, ed egli non se ne farebbe innamorato, come
si dice sotto nel Cant. 9., ed ella non havrebbe havuto tanti altri amanti; Ma
Baldone non l'havendo mai veduta, e volendo concitar contro di lei odio di
quei soldati, che lo seguivano, per istigargli ad andar più volentieri alla ricuperazione
di Malmantile, la rappresenta loro una donna così nefanda.

\begin{description}
\item[SVA donna] Sua moglie, Se bene i Poeti dicendo La mia donna, o La sua
  donna, intendono l'amata.
\item[LASCIO' donna, e madonna] Termine notariesco, e curiale, che significa Padrona
  assoluta. Sincopato di Domina.
\item[VN certo Diavolo] Si dice così quando vogliamo esprimere uno, che è cagione
  di qualche nostra disgrazia: per esempio: \textit{Il negozio andava bene, ma un certo diavolo
    d'un Sensale con le sue chiacchiere lo rovinò} quasi dica \textit{Il diavolo, che guastò
    questo negozio, fu un Sensale}.
\item[MONA Cionna] È un detto di disprezzo, che significa Donna da poco in
ogni operazione: ed il senso della voce Mona, Vedrai sotto C. 5. stan. 18.
\item[GUIDONE] Intendiamo huomo vilissimo, abietto, senza roba, e senza creanza,
  o riputazione.
\item[DAR lo sbalzo] Mandar via; Scacciare.
\item[ORBO]. In questo luogo vuol dir Uno, che vede poco, che noi chiamiamo
  lusco, se bene il suo vero senso è di cieco affatto. Vedi sopra in questo C. stan.
  9. alla voce sbirciare.
\item[MANCINO] Uno che per assuefazione ha maggior forza, ed attitudine nella
  mano sinistra, che nella destra; E perché questo tale si può dire difettoso;
  perciò huomo mancino, vuol dire Huomo non buono; ed in questo senso è preso
  nel presente luogo. E però voce che ha del furbesco. Se ne servì il Lalli nella
  sua En. trav. nel C.2. stan. 40, dicendo,
  \begin{verse}
    Perch' io non fui mai orbo, ne mancino.
  \end{verse}
Ed al C, 4. stan. 67.
  \begin{verse}
    E riuscito in somma un buom mancino,
    Vna delle più vili creature
    C' habbia sto mondo; e pazzo da catena;
  \end{verse}
\item[HA il gozzo] È parola nota, venendo dal latino guttur: Ma qui vuol dire
un gonfio, o scrofa, che vien nella gola, che i medici, che scrivono di simil
male pongono al trattato il titolo de \textit{Boccijs}.

\item[SFREGIO] Cicatrice di taglio nel viso. Ed una donna sfregiata è numerata
  fra le infami, e per la deformità del volto, e per la causa, per la quale si suppone,
  che le sia stato fatto. Vedi sotto C, 2. stan. 3. dove si mostra esser tali sfregi
  vituperosi anche negli huomini, ed al C, 6. stan. 54.

\item[SCORRER la cavallina] Pighiarsi tutti li suoi gusti liberamente, e senza riguardo
  alcuno. \textit{Havere scorsa la cavallina ne i lupanari}, vuoi dir, che era meretrice
  vecchia, ed avanzata ai bordelli, e lupanari. Gli antichi Egizj, quando volevano
  esprimere la sfacciataggine meretricia, figuravano una cavalla senza freno;
  il furore della quale nelle cose Veneree esprime Vergilio 3, Georg. dicendo.
  \begin{verse}
    Scilicet ante omnes furor est insignis equarum.
  \end{verse}
\item[IL titol di Signora a tutto pasto] Cioè continovamente era chiamata Signora.
  Termine usatissimo per intender voglia cosa, che si faccia molto, e continovatamente.
  Il Mauro nel Capitolo in lode della Torniella dice.
  \begin{verse}
    E ragionò di voi a tutto pasto
  \end{verse}
\item[DA dovero] Per debito, Per giustizia, Per merito. Intendi che volle proccurar
  d'havere stato, o signoria per meritare il titolo di signora, ec. ed osserva che quel
  \textit{da dovere} non è la voce \textit{vero} con l'aggiunta della sillaba do, ma è il nome
  \textit{dovere} messo in uso di dirlo così correttamente in casi simili a questo, e per
  esprimere una cosa di dovere o doverosa, e dovuta, e giusta.
\end{description}

\section{Stanza LXVII \& LXVIII}
\begin{ottave}
\flagverse{67}Così la mira ad alto havendo messa\\
A suoi Frustamattoni un dì ricorsa,\\
Bramar dice una grazia, e che in essa\\
Non si tratta di scorporo di borsa;\\
Ma, perché aspira a farsi Principessa,\\
Desidera da loro esser soccorsa\\
Col loro aiuto, volendo, e consiglio,\\
Provar, s'a Malmantil può dar di piglio,

\flagverse{68}Pronto è ciascuno, e vuol tra mille stocchi\\
Esporre il ventre, e come un Paladino,\\
Che per servire a Dame, tali allocchi\\
Cercan l'occasion col fuscellino;\\
Ma non si parli, o tratti di baiocchi,\\
Perché non hanno un becco d'un quattrino;\\
E credon, promettendo Roma, e Toma,\\
Di spacciar l'oro della bionda chioma.
\end{ottave}

Bertinella havendo fatta la suddetta risoluzione, richiese li suoi amanti, che
la volessero aiutare a farsi Principessa con impadronirsi di Malmantile, ed i suoi
Drudi s'esibiscono a servirla, perché sentono di non haver a spendere, il che è
cercato da tutti coloro, i quali con simil donne pretendono di passar per belli,
che è una delle tre specie di persone, che voglion queste femmine d'intorno, cioè
Il bello per sua propria sodisfazione. Il bravo per farsi rispettare. Ed il ricco
minchione, o corrivo, per cavar danari da lui, per campare se medelime, ed i
primi due, Il Persiani dice,
\begin{verse}
Il bravo, ed il corrivo, ed il valente.
Nella mia Mea fallifee
Questo antico dettato
Per c' al bravo, ed al bel non apparisce,
Ma sol vorrebbe il suo minchione allato.
\end{verse}

\begin{description}
\item[PORRE ad alto la mira] Aspirare a cose grandi. Mira si dice quel segno, che
  è nella canna dell'archibuso, o nelle balestre, nel quale s'affissa l'occhio per aggiustare
  il colpo al berzaglio. E di qui \textit{Porre la mira a una cosa} s'intende \textit{Volgere
    il pensiero}, o \textit{aspirare a una cosa}.
\item[FRVSTAMATTONI] Si dicono Quelli, che giornalmente vanno in una
  casa, o bottega, e non vi spendono mai un soldo, o vi portano utile alcuno,
  E si dicono Frustamattoni, perché non son d'altro giovamento, che frustare,
  cioè spazzare, e ripulire con le scarpe i mattoni; i quali son quelle lastre fatte di
  terra cotta, con le quali si lastricano i pavimenti delle stanze, da i Latini detti
  \textit{Lateres}.
\item[SCORPORO di borsa] Spendere. Scorporare vuol dit Estrarre da una massa, o
  da un corpo, o quantità di roba, o una porzione di essa.
\item[DAR di piglio] In questo luogo vuol dir Pigliare, impadronirsi; ed alle volte
  vuol dir Principiare come sotto C.6, stan 60.
\item[ESPORRE il ventre a mille stocchi] Vanti d'innamorati d'andare  soli contro
  a un'esercito intero, come i Poeti favoleggiano, che facessero i Paladini, che
  sono quei dodici Conti di Palazzo, ordinati da Carlo Magno per combattere
  contro a i nimici della S, Fede Cattolica, che furono detti \textit{Comites Palatini}, cioè
  Compagni nel Palazzo, che sono forse gli odierni Pari di Francia: the noi poi
  corrottamente chiamiamo Paladini, e con questa voce intendiattio. Haomé bravo.
\item[ALLOCCO] Specie d' uccello con il capo cornuto, come l'assiuolo, ma è
  più grande, e di colore lionato, con occhi grandi, e lucenti, È animal goffo,
  e se bene vive di rapina, tuttavia è tanto poltrone, che per cibarsi aspetta di pigliare
  gli uccelli, quando gli vanno scherzando attorno, tratti dalla di lui goffaggine;
  e quando se li avvicinano, non con rapacità, ma con flemma, e gravità
  non ordinaria gli prende col rostro, o con gli artigli; E da questa goffaggine
  nel far all'amore, ed aspettare gli uccelli, per Allocco intendiamo Uno, che
  se ne stia perdendo il giorno in vagheggiar Dame senza profitto, ed è lo stesso
  che \textit{Frustamattoni}, \textit{Colombi di gesso}, e simili.
  Con questo nome di \textit{Allocco} in molte parti d'Italia è chiamata ancora la Civetta,
  e credo, perché è di figura, se ben più piccola; simile a quella dell'Allocco, e
  vive con le medesime arti.
\item[CERCAR col fuscelino] Cercar minutamente, e con diligenza; \textit{Il tale cerca le
busse col fuscellino} vuol dire; Il tale fa tutto quel che egli può, per esser percosso,
  o per toccarne. Questo detto vien da quei ragazzi dell'infima plebe, i quali dopo
  che è venuta in Firenze una gran pioggia, che habbia fatta correr l'acqua
  per la Città, vanno cercando per le strade vicine alle gran fogne, che portano in
  Arno, se trovano fra le commettiture delle lastre delle strade spilli, chiodi, ed
  altre cose simili portate, e lasciate quivi dall'acque correnti; e per far ciò si
  servono d'uno stecco, o fuscelletto di scopa, o d'altro, col quale vanno rifrugando
  i fessi di dette commettiture, e perché così gran diligenze son troppe al
  poco utile, n'è nato il suddetto proverbio, che ha l'acceanato senso, ed è lo
  stesso che chiamar' una cosa di la da i monti, detto sopra in questo C, stan, 19.
\item[BAIOCCO]. E parola, e moneta romana, la qual parola è talvolta usata da
  noi per intender Danari, come qui, che dicendo \textit{Non si parli di baiocchi} intende
  \textit{Non si parli di danari}, cioè di Spendere.
\item[NON hanno un becco d'un quattrino] Non hanno pure un denaro, e quella parola
  Becco si mette a maggiore espressione, quasi dica Non hanno ne pure un sol
  \textit{quattrino becco}; cioè cattiyo, e non il caso a spendersi; Se non volessimo dire,
  che venisse questo detto dall'antica moneta Romana di rame; nella quale era impresso
  da una banda il volto di Giano con le corna, e dall'altra un rostro di nave,
  e che il dire; Un becco d'un quattrino sia lo stesso, che dire, ne anche la
  parte d'un quattrino, cioè  la faccia di Giano, che è cornuta.
\item[PROMETTE Roma e Toma] Promette cose grandissime, e che da persona
  alcuna non si possono mantenere, o osservare; i Latini dissero \textit{Maria, Montes polliceri},
  La voce toma non so che habbia nel nostro idioma significato alcuno, e stimo;
  che sia usata in questo detto per darle la rima con la parola Roma; Se forse
  non fusse il verbo spagnuolos tomar, che vuol dir torre, o pigliare, ed intendersi
  \textit{Ti prometto Roma}, (che è a dir tutto il mondo) \textit{e tu toma}, cioè piglia quel che
  ti piace. Lasca Nov. 8. \textit{Però non restava, di sollecitarla promettendole Roma, e toma,
    come se egli fusse il primo Principe del mondo}.
\end{description}

\section{Stanza LXIX, LXX \& LXXI}

\begin{ottave}
  \flagverse{69}Era tra molti suoi più fidi amanti\\
Vn ciarlon, che però detto è il Cornacchia,\\
Ed è di quei pittor, ch' i viandanti\\
Con lo stioppo dipingono alla macchia;\\
E perché nella lingua ha il suo in contanti,\\
Molto si vanta, assai presume, e gracchia;\\
E finalmente colorisce, e tratta\\
Questo negozio, come cosa fatta.
\end{ottave}

\begin{ottave}
  \flagverse{70}Scrive un viglietto poi segretamente\\
Ad un compagno suo capobandito,\\
Dicendo, che veduta la presente,\\
Il suo bagaglio subito ammannito,\\
Di notte tempo meni la sua gente\\
A Rimaggio alla Svolta del Romito;\\
Ma vada alla spezzata, e pe i tragetti,\\
E senza pensar' altro ivi l'aspetti.
\end{ottave}

\begin{ottave}
  \flagverse{71}Andò la carta, e quei c'hebbe l'intesa,\\
Come quel ch' invitato era al suo giuoco\\
Andonne, e guidò seco a quell'impresa \\
Cent'huomin con le lor bocche di fuoco,\\
Quivi il Cornacchia, e quella buona spesa\\
Di Bertinella giunsero fra poco,\\
Anch'eglino con grossa, e folta schiera\\
D'una gente da bosco, e da riviera.
\end{ottave}

Fra questi suoi più fedeli amanti era un tale detto il Cornacchia. Costui era
uno con tal soprannome; perché havea la voce d'un suono simile al gracchiare
della cornacchia, ed era un solennissimo briccone, e ladro, e spia. Questo da a
Bertinella il negozio per fatto, e s'ammannisce a far la sorpresa di Maimantile;
con scrivere ad un capo di ladri da strada suo corrispondente, che si conduca a
Rimaggio con le sue genti con armi, e panni, e l'aspetti alla Svolta del Romito,
che è una contrada in vicinanza di Malmantile. Eseguì l'amico, giunse
con cento huomini ben' armati nel luogo ordinatogli: fra poco vi arrivò ancora
il Cornacchia con Bertinella, con grande schiera di bravi furbi, che questo intende
\textit{gente da bosco, e da riviera}; che i Latini dissero \textit{homines omnium horarum}.

\begin{description}
\item[CIARLONE] Vno, che chiacchiera assai, L'Autore intende, che chiacchierava
  assai alla giustizia, cioè faceva la spia, e perciò detto Cornacchia, che è uccello
  di cattivo augurio; perché il suo ciarlare era di danno al prossimo. Ed in
  vero costui, mentre visse, fu sempre chiamato il Cornacchia, o per questa causa,
  o per quella che habbiamo accennato sopra.
\item[DIPINGERE alla macchia] Dipinger un Ritratto senz'haver d'avanti l'originale,
  ma col solo haverlo veduto. E l'Autore però intende, che egli era ladro di strada,
  e pigliando la voce macchia nei suo vero senso di selva densa, dice,
  che alla macchia ritraeva i viandanti con lo stioppo, ed intende Assaltava la gente alla
  strada con l'archibuso per rubarla, Questa però è finzione, perché il Cornacchia,
  se hebbe la malizia, non hebbe già tanto cuore di far' il ladro di strada, e l'Autore
  lo finge tale per mostrare, che egli era un furbo da far qualsivoglia sciagurataggine.
\item[HA nella lingua il suo in contanti] Vuol dire eloquente, pronto di lingua.
\item[VANTARSI] Promettersi molto di se medesimo, Esaltar le proprie opere,
è il Latino \textit{Iactare}.
\item[GRACCHIARE] Cicalare con poco fondamento, Vedi sotto C. 4. stan 29.
  C. 7. stan. 9, e C. 8. stan. 65. Ma perché costui è chiamato Cornacchia, il Poeta si
  serve del verbo gracchiare per esprimer il cicalar di esso.
\item[COLORIRE] Metafora assai usata, e vuol dire discorrer d'una cosa con aggiustatezza,
  con termini proprj, e con colori rettorici per persuadere, e fare
  apparir vera quella tal cosa, della quale si discorre.

\item[VIGLIETTO] o \textit{biglietto}. Vuol dir lettera; Ma strettamente significa quella
lettera, che si manda in luoghi vicini, come da una casa all'altra, dentro alla
medesima Città, o Terra. Voce che forse viene dal Francese \textit{Poulet}, che vuol dir
lettera, amorosa, o da \textit{Billet}, Vedi sotto C. 6. stan. 54.

\item[BAGAGLAIO] Quelle some, che si conducono appresso gli eserciti per utile, e
  comodo dell'armata, o dietro qualsivoglia viaggiante per servizio della propria
  persona; si dicono \textit{Bagaglio}, forse dal Francese \textit{Bagage}; o dal verbo Bainlare,
  che val Portare, come habbiamo osservato sopra in questo C. stan. 62. alla voce
  Baule, ed è quel che i latini dicevano \textit{impedimenta}.

\item[AMMANNIRE] Metter'all'ordine, Allestire, approntare; quasi dica \textit{ad
  manus habere}. Dante Purg. C. 23.  \begin{verse}
    Di quel ch'il Ciel veloce loro ammannna,
\end{verse} ed al C. 29.\begin{verse}La virtù, c' a ragion discorso ammanna.\end{verse}

\item[ALLA spezzata] A pochi insieme per volta, non in squadre o truppe formate.
  Si dice anche \textit{Alla sfilata}, Vedi sotto C. 6. stan. 85. ed è il \textit{diminutim}
  dei latini.

\item[PE i tragetti] Per le balze, per luoghi, e strade non praticate; e il puro Latino
  \textit{Traiectus}.

\item[HAVER l'intesa] Rimaner d'accordo. Haver l'instruzione di come si debba contenere.
\item[INVITAR uno al suo giuoco] Chiamar' uno a fare una cosa, che sia di suo genio,
  e gusto. I Latini dissero \textit{Musas hortari ut canant}, ec.

\item[BOCCHE di fuoco] Intendiamo Ogni arme da fuoco, atta a portarsi addosso,
  come Moschetti, archibusi, pistole, e simili.
\item[BVONA spesa] Huomo astuto, e scaltrito, e suona lo stesso, che Tristo,
  e Volpe vecchia.
\end{description}

\section{Stanza LXXIL \& LXXIIT.}
\begin{ottave}
  \flagverse{69}Dopo ch' insieme tutti fur costoro\\
Si fece de' più degni una semblea,\\
Del come discorrendo fra di loro\\
Sorprender' il Castello si dovea,\\
Ond'il Cornacchia in mezzo al concistoro\\
Rizzato in pié con gran prosopopea,\\
Ed una toccatina di cappello,\\
In tal modo cavò fuora il limbello.
\end{ottave}

\begin{ottave}
  \flagverse{69}Io so c'a un'ignorante, a un'idiota\\
L'esser il primo a favellar non tocaa;\\
Ma perdonate a questa zucca vota,\\
Signori, s'io vi rompo l'huovo in bocca;\\
Scricchiola sempre la più trista ruota,\\
Così la lingua mia più rozza, e sciocca\\
V'infastidisce, è ver ma v'assicura,\\
Che Malmantile è nostro a dirittura.
\end{ottave}

Ragunati costoro insieme, quei più degni si ristrinsero a consiglio, per fermar
il modo, che si doveva tener per sorprender Malmantile, ed il Cornacchia, fatte
sue cirimonie, comincia a mostrare il modo certo di pigliare detto Malmantile.
\begin{description}
  \item[PRESOPOPEA] Questa voce, che vien dal Greco Prosopopea compostasdi
due dizioni \textit{Prosopon}, che suona \textit{personam} (ed a noi Personaggio) e poeeo, che
suona \textit{facto}, se bene è una figura con la quale fingesi un perlonaggio, come
farebbe introdurre una cosa inanimata, che parli con una animata, \& è contra, tuttavia
noi ce ne serviamo per intender una certa superbia, arroganza, fasto, o
presunzione di se medesimo, dimostrata con gli atti; di che vedi sorto C.6. stan. 85.
Ed in tal senso, secondo il Monosino era pigliata ancora da i Greci. Si dice
da noi anche sussiego, derivando la voce dallo Spagnuolo.
\item[VNA toccarina di cappello] Atto che esprime detta Prosopopea.
\item[CAVÒ fuora il limbello] Cominciò a parlare. Limbelli; Si dicono quei pezzi
  di pelle di bestia, che dalle dette pelli tagliano i Conciatori, donde poi
  \textit{limbellucci} i ritagli delle pelli più sottili, come di cartapecora, che servono per far
  colla da Pittori. E perché tali \textit{limbelli}, quando son freschi; ed umidi sono simili alle
  lingue, perciò per \textit{limbello} intendiamo lingua; e però detto scherzoso, come si
  vede, che l'usò il nostro Autore anche sopra in quella sua lettera alla Sereniss.
  Arciduchessa, riportata da me nel Proemio. \textit{Cavò fuora il limbello, e disse le sue
    Sillabe, come un Tullio}, ec.
\item[IGNORANTE, \& idiota] Sono Sinonimi, ne vi si fa alcuna differenza, se
  bene strettamente \textit{Ignorante} vuol dire uno, che non sa nulla, e \textit{Idiota} par che si
  convenga a coloro, che non hanno cognizione di lettere.
\item[ZVCCA] S'intende il capo dell'huomo per la similitudine, e Zucca vera vuol
  però dire testa senza cervello, che si dice \textit{vota di sale}, o non haver sale in zucca.
  E questo perché è solito nelle cucine tenere il sale in una Zucca secca appesa al
  muro del Cammino. Vedi sotto Can. 4. stan. 15. I Latini pure dicevano \textit{sale} per
  giudizio, e trovasi in Catullo.
  \begin{verse}
    Nulla in tam magno corpore mica salis
  \end{verse}
Vedi sotto C. 8. stan. 26., e Marziale C. 7.
  \begin{verse}
    Nullaque mica salis, nec amari fellis in illis
  \end{verse}
\item[ROMPER l'huovo in bocca] Torre la parola di bocca a uno, ciò è Dire che
  doveva, o voleva dire un'altro. Terenzio disse \textit{Bolus ereptus e faucibus est}.
\item[SCRICCHIOLARE] Stridere, strepitare. S'intende quel romore, che fa
  nel muoversi un legno fortemente stretto, o aggravato da altro legno, o materiale
  duro; come appunto segue nelle ruote da carro. Ed il proverbio: \textit{Sempre
    Scricchiola la peggio ruota del carro}, Significa \textit{Il più sciocco della conversazione, vuol
    sempre parlare}, Detto antico, e vien dal Latino, che dice \textit{semper deterior
    vehiculi rota perstrepit}, ec.
\item[A DIRITTVRA] Cioè assolutamente, sicuramente, e senza difficultà aleuna,

\end{description}

\section{Stanza LXXIV.}
\begin{ottave}
  \flagverse{74}Credete a me: Ciascun si stia nascosto\\
In queste macchie, in questi boschi intorno\\
Ed io da voi fra tanto mi discosto,\\
Ne questa notte farò più ritorno.\\
Rivedremci colà doman sul posto,\\
Perché vicino al tramontar del giorno\\
Vi farò cenno, hor voi ponete mente,\\
E poi venite via allegramente.
\end{ottave}

\begin{ottave}
  \flagverse{75}Parte il Cornacchia, e corre presto presto\\
Da certi suoi amici contadini,\\
Da' quali le lor bestie piglia in presto\\
E carica più some di buon vini,\\
E di soppiatto, come fante lesto\\
Cavò di tasca certi cartoccini\\
Pieni d'alloppio, e dentro al vin li pone\\
Quello impepando, senza discrezione.
\end{ottave}

\begin{ottave}
\flagverse{76}Così carreggia, e giunto a Malmantile\\
All'aprir della porta la mattina\\
Scarica in piazza il vino, ed un barile\\
A regalar ne manda alla Regina.\\
Poi vende il resto a prezzo tanto vile,\\
C'ognun ne compra, e in fin che n'ha in cantina\\
Per rivenderlo altrui, il fiasco attacca,\\
Si cala al buon mercato, a quella macca
\end{ottave}

\begin{ottave}
\flagverse{77}Due, o tre fiaschi davane a quattrino,\\
Ed a' poveri davalo a Isonne,\\
Tal che tutti tuffandosi a quel vino\\
S'imbriacaron come tante monne,\\
E subito dal grande al piccolino\\
Tanto de gli huomin, quanto delle donne\\
Cascaro in sonnolenza sì gagliarda,\\
Che desti non gli havrebbe una bombarda.
\end{ottave}

Cornacchia instruisce i compagni di quello devon fare, e si parte, e va da,
certi contadini suoi amici, da' quali piglia le lor bestie in presto, e lo carica di
vino alloppiato, quale porta in Malmantile, e lo vende così a buon mercato, che
Ognuno ne comprò, e bevvero tanto, che tutti s'imbriacarono, e si messero a dormire

\begin{description}
  \item[PRESTO presto] Prestissimo: per la replica d'una stessa parola, che ha forza di
superlativo, come habbiamo detto altrove.
\item[DI soppiatto] Di nascosto. Vien dal verbo impiattare, che vuol dir Nascondere
  una cosa corporea, come s'è detto altrove.
\item[FANTE lesto] Huom sagace, astuto, e che sa il conto suo.
\item[CARTOCCINO] Diminutivo di Cartoccio, che è una piegatura di foglio, fatta
a Piramide usata da gli speziali per mettervi dentro zucchero, pepe, ed altro simile.
\item[ALLOPPIO] Specie di sonnifero composto di sugo di papavero, coagulato,
secco, e polverizzato, e d'altri ingredienti; e si chiama \textit{oppio}.
\item[CARREGGIARE] Venendo da carro dovrebbe intendersi solamente per Camminar
  col carro, o traghettar robe col carro, ma ci serve per lo più per intender
  ogni sorte d'andare, o camminare, a piede, o a cavallo, conducendo o non
  conducendo roba.
\item[BARILE] Vaso di legno per uso di portarvi olio, vino, ed ogni altro liquore
  simile, ed è la misura comune del vino, capace di 20. fiaschi, e quello da olio
  di 16 fiaschi. Tali vasi son composti, ed aggiustati in maniera da adattarne due
  per volta addosso a una bestia da soma.
\item[ATTACCA il fiasco] Coloro, i quali in Firenze vendono il vino a fiaschi alla
  propria casa, attaccano per segno di ciò sopr'alla porta un fiasco, acciò che il
  popolo vegga il luogo, dove si vende il vino: e pero quando si dice \textit{Il tale ha oggi
  attaccato il fiasco}, s' intende, \textit{il Tale oggi ha cominciato a vendere il vino a fiaschi}.

\item[SI cala a buon mercato] Si lascia persuadere dal prezzo vile a comprare. È
  traslato da gli uccelli, che si calano alla vista della preda.
\item[MACCA] Abbondanza grande. Vien forse dal Latino Mactus, che s'intende
  abbondanza grande, quasi \textit{Magis auctus}. Plau, milit, 4.22. \textit{Macte amare}. E
  si trova \textit{Puer macte virtute}; giovanetto virtuosissimo. Dice il Vocabolista
  Bolognese, che macco vuol dir' abbondanza, che induce disprezo, e così è vero nel
  parlar nostro, che si dice \textit{smaccare} per intender Vituperare, o screditare.
\item[A Isonne] Per niente. Senza spesa, È detto plebeo, ed è usato per lo più tra
  i battilani, i quali hanno per tradizione, che Isonne fusse già un'huomo de' loro,
  il quale mangiava tanto volentieri a spese d'altri, che essendo morto, e seppellito
  già di qualche mese, scappasse dell'avello al discorso, che da alcuni si faceva
  di voler dar mangiare a tutti i Battilani per tre giorni, senza che spendessero,
  Costui havea due fratelli l'uno detto Salicone, e l'altro lo Scrocchina, e però
  \textit{scroccare} mangiare a \textit{Salicone}, a \textit{Scrocco}, e a \textit{Isonne} significano tutti Mangiar senza
  spendere, che Terenzio disse \textit{Asymbolum} composto dalla proposizione A, che
  suona Senza, e \textit{symbolum}, che vale quota, o scotto, e significa senza denari; E si
  come ne i Latini questo \textit{Asymbolum}, fu usato da i parassiti, e guatteri, così il nostro
  \textit{Isonne}, è usato dalla plebaglia, fra la quale è nato.

  Può anch' essere, che questo detto \textit{Isonne} venga da un Iiogo poco fuori di Firenze
  detto \textit{Isonne}, dove anticamente andavano a desinare aicune volte l'anno
  molti battilani, senza spendere, non perché veramente non spendessero, ma perché
  il denaro, che si spendeva in quel desinare, era di mance fatte per le Pasque,
  S. Giovanni, e Carnevale, che messo in una lor corbona, si serbava, e distribuiva
  per questi desinari; e può essere, che questi battilani dessero tal nome
  \textit{Isonne} a quel luogo dove andavano a far questi lor desinari, chiamati da loro
  \textit{desinari a Isonne}; ma sia come si voglia, basta che appresso noi il termine \textit{Isonne} è
inteso per Senza spesa.
\item[TVFFANDOSI] Tuffarsi a una cosa, significa Pigliare, o fare assai una tal cosa.
\item[S'imbriacaron come tante monne] Vedi quel che s'è detto sopra in questo C. stan. 10.
\end{description}

\section{Stanza LXXVIIL}
\begin{ottave}
  \flagverse{78}Quando il Cornacchia vedde il suo disegno\\
Già riuscito, andò sopr'alle mura,\\
Ed ai compagni fece il detto segno,\\
Che bene havendo al tutto posto cura,\\
Saliro al poggio senz'alcun ritegno,\\
Senza sospetto haver, senza paura\\
Dietro al Cornacchia lor guidone, e scorta\\
Dentro al Castello entraron per la porta
\end{ottave}

\begin{ottave}
\flagverse{79}E perc' ognun dormiva, come un Tasso,\\
La donna fece farne una funata,\\
E condursegli a piedi a baciar basso,\\
E renderle il tributo ognun pro rata,\\
A Celidora poi restata in Nasso,\\
Cioè da' suoi vassalli rinnegata,\\
Già che tutti voltato havean mantello,\\
Comandò che baciasse il chiavistello.
\end{ottave}

\begin{ottave}
\flagverse{80}Ell'ubbidì, temendo, ancor di peggio,\\
E ben che fusse un pezzo in la di notte,\\
Il pigliarsene subito il puleggio\\
Vn zucchero le parve di tre cotte.\\
Così finito il solito corteggio\\
Con due strambelli, e un par di scarpe rotte\\
Triffa, e strascina poi per la boccolica\\
Un tozzo mendicava all'accattolica
\end{ottave}

I Compagni di Bertinella veduto il segno dato dal Cornacchia, andatono a
Malmantile, ed entrati dentro, e trovati tutti a dormire gli legarono, e gli condussero
a render ubbidienza a Bertinella, la quale comandò a Celidora, che uscisse
del Castello, ed ellam tutta mal' all'ordine se n'andò, benché fusse assai di notte,
e si condusse a mendicare il vitto.
\begin{description}
  \item[GVIDONE, e scorta] Guidone s'intende Colui che guida; e Scorta è quello che
mostra la strada; ma la voce \textit{Guidone} è forse per scherzo presa dall'Autore nel
senso, che sopra stan. 65. e sotto al Cant, 8. stan.~72.
\item[FAR una funata] Legar con una fune più persone: Quando molti insieme
  commettono un delitto, si suol dire: \textit{Se vengono i birri, voglion far la bella funata}.
  Non perché crediamo, che vogliano effettivamente legargli tutti a una fune, ma
  intendiamo, \textit{Vogliono farne molti prigioni}, e così intendi nel presente luogo.
\item[BACIAR basso] Cioè inchinarsi a baciar i piedi in segno di vassallaggio.
\item[RIMANERE in Nasso] Dai più si dice \textit{rimanere in Asso}, e ciò segue per
  corruzione nella pronunzia, che tanto suona \textit{rimanere in asso} che \textit{rimanere in Nasso}
  come si dovrebbe dire, e significa abbandonato, senza aiuto, e senza consiglio;
  Ed è derivato dalla favola d'Arianna abbandonata da Teseo nell'Isola di Nasso;
  E si dice anche rimanere in su le secche di Barberia, il che corrobora che si debba
  dire \textit{in Nasso}, e non in asso che non ha verun senso, o allegoria. Vedi sotto
  C.\ 10.\ stan.~2.
\item[VOLTAR mantello] Rinnegare. Ribellarsi; andar da un partito all'altro. Il
  Lalli En. trav. C. 2, stan.~39.
  \begin{verse}
    Hor che mi lice di voltar mantello
  \end{verse}

\item[BACIARE il chiavistello] Andarsene senza speranza di tornare. Usiamo questo
  detto per esprimere che non si vuole, che quel tale, che è stato per li suoi
  mali portamenti scacciato d'una tal casa, viva con la speranza di ritornarvi, e
  pero si potrebbe dir con Vergilio \textit{Supremum vale dixit}.
\item[CHIAVISTELLO] Serratura da porte, o finestre, che confiste in un ferro
  lungo, il quale fa la sua operazione, passando per diversi anelli pur di ferro
  adattati nel legname; ed è il Latino \textit{vectis}.
\item[PIGLIAR il puleggio] Andar via. Pigliar il cammino, E' frase marinaresca, ma
  però usata comunemente in questi termini d'andar via presto. Dante Par. C. 23.
\begin{verse}
  \backspace Non è puleggio da piccola barca
  Quel che fendendo va l'ardita prora
  Ne da nocchier, c' a se medesmo parca.
\end{verse}

Da questa voce Puleggio viene \textit{spulezzare}, che vedremo sotto C. 7. stan. 18, che
pure significa Andar via. Forse si potrebbe dir anche \textit{prueggiare} verbo pur
marinaresco, che significa Andar via bel bello.

Vincenzio Tanara nella sua Economia del Cittadino in villa Lib. 6. trattando
dell'erba \textit{Puleggio} dice, che sparsa in luogo dove sieno pulci ha virtù di
scacciarle; onde può essere che da questo effetto dell' erba \textit{Puleggio} venga il presente
dettato. Da \textit{puleggio} forse anche vengono \textit{Pulegge}, che sono quelle piccole
girelle, che si congegnano, ne i legni per facilitare i veicoli, come farebbe dentro a i
regoli da piede alle scene, o prospettive da commedie per renderle più facili a
strascicarsi dentro a i canali in occasione di mutazione delle medesime scene.
\item[UN suechera le parne di tre cotte] Le parve d' haverla a buon mercato: le parve
  d'haver fortuna grandissima, perché s'aspettava malto peggio. Lo Zucchero
  di tre cotte fatte bene si stima che sia il miglior grado di perfezione, della
  quale sono tre i gradi. secondo il detto \textit{omne trinum est perfectum}. Ed i Franzesi
  denominano il superlativo col tre, cioè buono, for buono, e tre buono\footnote{bien, \textit{fort} bien, \textit{très} bien}, per
  buono, molto buono, buonissimo,:

\item[STRAMBELLE] Vesti vecchie, e stracciate. Vedi sotto C, 3., stan.~65.
\item[UN tozzo] Detto così assolutamente senz' altra aggiunta vuol dire un pezzo di pane.
  E \textit{frustum panis}, che usò Dante nel Parad. C. 6. \textit{Mendicando sua vita a frusto a frusto}.
\item[TRISTA, e strascina] Huomo tristo vuol dire Huomo mal vestito, e Strascino
  suona quasi lo stesso, perché Strascini chiamiamo alcuni huomini, i quali vanno
  comprando carne fuori della Città, e l'introducono in Firenze occultamente per
  rubarne la gabella, e perché costoro son sempre unti, sudici, e stracciati, perciò
  dicendosi \textit{Strascino} intendiamo mal' all'ordine di vestito, ec.
\item[BOCCOLICA, e accattolica] Sono due parole dette per scherzo, e per la similitudine
  che hanno con Bocca, e con Accattare, e per parlare Ianadattico, non
  sono però fuori dell'uso della gente più Civile, la quale spesso si serve di parole
  latine a quel proposito, che le pare che facciano giuoco stroppiandole, e interpretandole
  a lor modo, come le presenti \textit{Boccolica}, e \textit{accattolica} che l'una vuol dir Bocca,
  e l'altra Accattare, e così intendesi che Celidora accattava per mangiare. Tal'uso
  d'allusione scherzosa era pur'anche appresso ai Latini trovandosi \textit{Ab Ilio nunquam
  recedis}, che par che voglia dire tu non ti parti mai dalla Città di Troia, e
  s'intende poi; tu non abbandoni mai l'Ilo intestino, cioè sempre mangi.
\item[MENDICARE] Vuol dire durar fatica a conseguire. \textit{Il tale mendica le parole},
  cioe Dura fatica a parlare; ma il suo significato più inteso è Chiedere elemosina,
  Dante Parad. C. 6.
  \begin{verse}
    \backspace Indi partissi povero, e vetusto,
    E s'il mondo sapesse il cor ch' egli hebbe,
    Mendicando sua vita a frusto a frusto, ec.
  \end{verse}
\end{description}

\section{Stanza LXXXXI, LXXXII \& LXXXIII}
\begin{ottave}
\flagverse{81}In tanto Bertinella del Reame\\
Garbatamente fecesi padrona,\\
E de' villaggi, e d'ogni suo bestiame\\
Prese il possesso in petto, ed in persona\\
Poi per letizia cavalieri, e dame\\
Regalò di confetti, e di pattona;\\
E segue ogn'anno di mandarne attorno,\\
Per la dolce memoria di quel giorno.
\end{ottave}

\begin{ottave}
\flagverse{82}Tosto che ci hebbe fitto il capo, volle\\
C'ognun serrasse il traffico, e il negozio,\\
Donando a ciascheduno entrate, e zolle,\\
Acciò se la passasse da buon sozio,\\
Ed allegro, a piè pari, ed in panciolle\\
Senza briga vivesse in pace, e in ozio,\\
Ognun vi s' arvecò di buona gana,\\
Che la poca fatica a tutti è sana,
\end{ottave}

\begin{ottave}
\flagverse{83}Così mai sempre in feste, ed in convito\\
Tirano innanzi questi spensierati;\\
Ne moverebbon per far nulla, un dito,\\
Ben ch' ei credesson d' esser' impiccati;\\
Non teme della Corte, chi e fallito,\\
Che tutti i giorni a lor son feriati;\\
Non v'e giustizia, ne il bargel va fuora,\\
Se non per gastigar chiunche lavora.
\end{ottave}


Sbandita Celidora dal regno, Bertinella prese l'attual possesso di tutto lo stato,
e per acquistarsi la benevolenza de' sudditi cominciò dal regalare le dame, e
cavalieri, con regali degni della vilissima condizione di se medesima, ed appropriati
alle qualità de' Cavalieri, e Dame di Malmantile; poi con feste, ed allegrie
per contentare il popolo, e con levare i Ministri della giustizia tanto odiosi alla
plebaglia, e con fare altri ordini che si leggono nelle presenti ottave.

\begin{description}
\item[IN petto, ed in persona] Attualmente, e corporalmente. \textit{Animo \& corpore}.
\item[PATTONA] Torta, o pane fatto di farina di castagne, con altro nome
  detto \textit{polenda}, dal Latino \textit{Polenta}, che era vivanda fatta di farina d'orzo con
  altre polveri odorifere secondo Varrone. È vivanda vilissima appresso di noi; e
  da questa sua viltà habbiamo un detto di disprezzo, che è; \textit{Mangiapattona},
  \textit{Mangiapolenda} a un huomo vile, e buono a poco. Qual detto usò Plauto chiamando
  questi tali \textit{Pultiphagj}; ma il disprezzo non nasceva dalla viltà della \textit{polenta},
  (che era finalmente il cibo comune anche per le persone di garbo, e generalmente
  mangiando questa sorte vivanda i Romani vissero lungo tempo, Vedi Plin.
  lib. 18. cap. 8.) nasceva bene dall'intendersi con tal detto un huomo buon'a
  poc'altro, che a mangiare, e come noi diciamo \textit{Sparapani}, \textit{Votamadie},e simili

\item[V'hebbe fitto il capo] Se n'era'impadronita: N'haveva preso l'attual possesso;
  perché essendo il capo la più nobile, e principal parte della persona, noi diciamo
  \textit{Ficcare il capo in un luogo} per intendere Entrare in un luogo, e pigliarne il
  possesso personalmente.
\item[TRAFFICO] e negozio. Sinonimi, se bene \textit{traffico} par, che si ristringa all'arti
  manuali; onde con dire \textit{Traffico}, e \textit{negozio} intende non lavorare, ne
  mercanteggiare, o negoziare.

\item[ZOLLA] È il Latino gleba, che vuol dire Pezzo, o massa di terra smossa,
  come s'è accennato sopra in questo C. stan. 57., ma qui pigliando la parte per il
  tutto, intende terreni fruttiferi: \textit{Il tale ha delle zolle}, comunemente s'intende
  Ha de' terreni.

\item[SOZIO] Dal latino \textit{Socius}. Compagno \textit{Viver da buon sozio} vuol dir Viver
  da buon compagno, alla reale, ed alla schietta. E questa voce Sozio non so che
  sia usata se non in questo caso, e con l'aggiunta di \textit{buono}, o \textit{malo}: dicendosi
  Il tale è buon sozioxe, o \textit{non è mal sozio}, per intendere E' galant'huomo.

\item[A piè pari, ed in panciolle] S'usa questo detto per esprimere Un huomo poltrone,
  che non voglia far'altro, che godere i suoi comodi, e la voce \textit{panciolle}
  è composta di due parole, cioè \textit{pancia}, ed \textit{olle}, e suona pancia di pentola, la quale
  col posar pari, e con quella sua gran pancia è il vero ritratto della: comodita, e
  poltroneria. Il Bronz. nel Cap. in lode della Galea dice.
  \begin{verse}
    \backspace Guarì, ma in capo al giuoco, come volle
    Il Ciela, ne fu tratto il poverino,
    E fu privato di stare in panciolle.
  \end{verse}

\item[BRIGA] Noia, fastidio, fatica. Qui è preso per faccenda, o pensiero d'operare.

\item[DI buona gana] Molto volentieri. È detto spagnuolo, e la voce gana è usata da
  noi per intender Voglia, o gusto grande. \textit{Il tale mangia di gana}; \textit{Lavora di gana}, ec,

\item[SCIOPERATO] Uno che non ha, e non vuole haver faccende. Vedi sopra,
  stan. 29. Scioperati s'intendono quei Cittadini, che senza arte, o impiego vivono
  con le loro entrate.
\item[CORTE] Intendi la Corte della giustizia da i Latini \textit{detta Curia} a differenza
  di \textit{Aula}; e vuol dire Ministri della giustizia.
\item[FALLITO] Uno che negoziando ha fatto così gran debito, che non ha
  possibilità di pagarlo. E il latino \textit{decoctus, qui fallit creditores, ipsumque fefellere negacia}.
\item[TUTTI i giorni son feriati] Sempre è festa per loro; Feriato s'intende quel giorno,
  nel quale ancor che lavorativo non si tien da i Magiftrati ragione, e non si
  possono fare esecuzioni civili contra a i debitari, e questo intende dicendo \textit{Non
  teme della corte, chi è fallito}, perché è feriato, e non può esser menato prigione.
\end{description}

\section{Stanza LXXXIV}

\begin{ottave}
\flagverse{84}Ma s'io non erro il tempo è già vicino,\\
Che n'ha a venir la piena de' disturbi, \\
Mentre doman per far un buon bottino\\
Andremo a dar'addosso a questi furbi.\\
Così panno sarà di Casentino,\\
Ne se lamenti alcuno, o si sconturbi;\\
Che che nuoce al compagno in fatti, o in detti\\
Deve saper che; Chi la fa l'aspetti,
\end{ottave}

Baldone, havendo fatto il detto raccanto della cacciata di Celidora, dice
sperare, che sia vicino il tempo, nel quale faranno gastigati coloro, che hanno sorpreso
Malmantile, perché il giorno futuro vuol andare a dar loro addosso.
\begin{description}
\item[HA da venir la piena de' disturbi] Ha da venir grandissima quantità di disgusti a
sturbare i loro commodi. E \textit{Piena} diciamo quando Arno, o altro Fiume cresce
per le pioggie.
\item[SARA' panno di Casentino] Casentino è una Regione in Toscana, dove si fabbrica
  una specie di panni, che bagnati scemano di lunghezza, e larghezza perché
  rientrano. E da questo detto \textit{sarà panno di Casentino}, intendiamo Rientrerà,
  cioè tu hai fatto a me questo, ed io farò a te il simile, cioè Mi vendicherò.
\item[CHI la fa, aspetti] Chi fa un torto al compagno, aspetti pure d'esser contraccambiato.
  Il Petr. disse;
  \begin{verse}
    Chi si prende diletto di far frode,
    Non si dee lamentar s'altri l'inganna,
  \end{verse}
E questi due versi posson servire per dichiarazione delli quattro ultimi della
presente ottava.
\end{description}
\section{Stanza: LXXXV.}
\begin{ottave}
\flagverse{85}Qui racque il Duca; e subito rattacca,\\
Col dire alla cugina in voce bassa\\
Che, perch'egli ha la bocca asciutta, e stracca\\
Il soggiunger a lei qualcosa lascia\\
Non ho che dir (gli rispond'ella) un hacca,\\
Oltre che la sarebbe carne grassa,\\
Dì più tosto, in che mo noi siam parenti,\\
Ch'io non paia a costor de gl'Innocenti;
\end{ottave}

\begin{ottave}
 \flagverse{86}Ed io che non ne ho gran cognizione,\\
E sempre me ne sono stata a detta \\
(Che tutta la mia gente andò al cassone,\\
Come tu sai ch'io ero fanciulletta:)\\
T'udirò volentieri. Allor Baldone\\
Soggiunse: Or or ti servo, e a tanta fretta,\\
Perché non gli moria la lingua in bocca,\\
Ricominciò quest'altra filastrocca.
\end{ottave}

Baldone termina il discorso, e volto a Celidora le dice, che ella soggiunga,
se ha di più; ed essa dicendo, che non ha che soggiugnere lo prega a narrare, in
che modo sieno parenti: E Baldone s'accinge a contentarla. E qui termina il
nostro Poeta il suo primo Cantare.

\item[NON ho che dire un hacca] L' H vogliono, che non sia lettera, ma semplice
  aspirazione, e però dicendosi \textit{Non ho che dire un hacca}, è lo stesso che dire: \textit{Non
  ho che dir nulla}.

\item[SAREBBE carne grassa] Stuccherei il popolo; Mi renderei odiosa. Il Lasca
  Nov. 4. dice: \textit{E poi io non vorrei anche tanto infastidirlo, che egli m'havesse a dire,
    che io fussi carne grassa}. La carne grafia suole a i più che la mangiano cagionare
  nausea; il che diciamo stuccare.

\item[CH' io non paia costor de gl'Innocenti] Che costoro non pensino, che io sia
  bastarda, o senza parenti. In Firenze lo spedale de gl'Innocenti si chiama quello,
  nel quale si mettono ad allevare i bambini, per lo più, nati di congiunzioni
  illecite, i quali corrottamente chiamiamo \textit{Nocentini}. Vedi sotto Cant.\ 10.\ stan.~7.

\item[ME ne sono stata a detta] Non ho cercato di saperne più là; ma ho creduto quel
  che m'è stato detto, o raccontato.

\item[LA mia gente andò al cassone] Mio padre, mia madre, e tutti gli altri miei parenti
  morirono; che per mia gente in questo luogo, ed in questi termini s'intende
  Miei parenti, e non altri.

\item[A tanta fretta] Subito, Prestissimo.
\item[NON gli moria la lingua in bocca] Era loquace, eloquente. Havea facilità a
  parlare. È lo stesso che \textit{Havere il suo in contanti nella lingua} come s'accennò
  sopra stan. 69.
\item[FILASTROCCA] Serie di parole, e per lo più s'intende d'un discorso male
  ordinato, e proprio del racconto, che talora fanno le balie a' Fanciulli in quelle
  lor novelle, come appunto è questa che narra Baldone, che l'Autore oltre all'haverla
  sentita forse raccontare alle sue donne, quando era fanciullino,
  ha tratta dallo Cunto degli Cunti di Gianalesio Abbattutis.
\end{description}
\section*{FINE DEL PRIMO CANTARE}


\chapter{Secondo Cantare}

\begin{argomento}
De i due gran figli del Signor d'Ugnano
Prodigioso il natal narra Baldone;
Come s'acquista moglie Floriano,
E vien dall'Orco poi fatto prigione.
Come Amadigi libera il germano;
E il mostro spaventoso a terra pone,
E dice al fin, che l'un di questi dui
Fu padre a Celidora, e l'altro a lui.
\end{argomento}

\section{Stanza I.}
\begin{ottave}
 \flagverse{1}Era in Ugnano il Duca Perione, \\
Che sempr'all'Altarin fidecommisso \\
Faveva notte, e di tanta orazione, \\
E tante carità, ch'era un subbisso.\\
Ne per altro era tutto bacchettone,\\
Che per un suo pensiero eterno, e fisso\\
D'haver prole, perché della sua schiatta\\
Non v'era, morto lui, ne can, ne gatta.
\end{ottave}

Il Duca Baldone dà principio alla narrativa del parentado, che passa fra lui,
e Celidora, come havea promesso  nell'antecedente Cantare, e dice; Che fu già
in Ugnano il Duca Perione, il quale faceva molte opere pie per disporre il Cielo
a concedergli prole. La favola del nascimento di questi figliuoli trovasi nello
Cunto degli Cunti di Gianalesio Abbattutis Giorn. 1. Cunto 9. ll nostro Poeta
pero non la cavò di quivi; ma la narrò, come l'haveva sentita contare alle sue
donne, quando era fanciullo; e questo è certo, perché questa era nel suo primo
Poema fatto molto prima, che il Basile Autore dello Cunto de li Cunti la stampasse,

\begin{description}
\item[ALTARINO] Così chiamiamo un' inginocchiatoio a foggia d' altare, il quale
  per lo più si tiene allato al letto per inginocchiarsi, e fare orazione.
\item[STAR fidecommisso in un luogo] è detto iperbolico, che significa Star moltissimo
in un luogo; che qui vuol dire Stava sempre, o non si levava mai dall'Altarino;
che s'intende faceva orazioni infinite.

\item[TANTE carità ch era un subisso] Carità, ed elemosine infinite. Per denotare
  una quantità indicibile usiamo dire: \textit{Son tanti, che è un subisso}, \textit{un fracasso}, \textit{un flagello},
  e simili. Questa voce \textit{subbisso} vien forse dal Greco \textit{abyssos}, che significa voragine,
o smisurata profondità d'acque, come suona ancora nel nostro idioma,
donde \textit{subissare} Andar nel profondo, quasi dica \textit{sub abysso}.

\item[BACCHETTONI] Così chiamiamo noi certi colli torti, e graffiasanti, che
  stimano peccato il portare un fiore in mano, e credono poi di far'un'atto meritorio
  a dare a usura; con altro nome chiamati Ipocriti, cioè Pseudobeati; huomini
  da bene per interesse, e per gabbare il compagno; e sono insomma coloro,
  de' quali Giovenale disse: \textit{Qui Curios simulant, \& Bacchanalia vivunt}. E diciamo
  \textit{Bacchettone}, quali \textit{Va chetone}, perché questa Canaglia, che studia di simulare la
  bontà, per arrivare a suoi fini, è simile all'acque profonde, che vanno chete,
  delle quali parlandé Q. Curzio dice: \textit{Altissima quaeque flumina minimo labuntur sono}.
  E come queste acque son sempre di pericolo, così li \textit{bacchettoni} nella loro taciturnità
  occultano il malo animo, che hanno contro al prossimo. Il costume di costoro
  tocca Orazio lib. 1. Ep. 17. dicendo che son devoti di Laverna Dea de
  ladri.
  \begin{verse}
    Labra movens, metuens audiri; Pulchra Laverna,
    Da mihi fallere; da iustum, sanctumque videri.
  \end{verse}

Di questa voce \textit{Bacchettoni} si serve anche il Tassoni nella sua Secchia. \textit{Nimico
natural de' Bacchettoni}. Ed un dottissimo de' nostri tempi, il quale fa un
discorso poetico sopra a costoro, lo termina con dire \textit{Furfante, e bacchetton suona
il medesimo}, Vedi sotto C. 6. stan. 97. dove si dice esser lo stesso \textit{Bacchettoni}, che
\textit{Ipocriti}, i quali S. Matteo chiamò \textit{similes sepulchris dealbatis}; il Berni nel'Orlando
disse. \textit{O agghiacciati dentro, e di fuor caldi}, \textit{In sepolcri dipinti gente morta}.

Giovenale aggiunge al detto di sopra.
\begin{verse}
  Fronti nulla fides; quis enim non vicus abundat
  Tristibus obscoenis ? castigas turpia, cum sis
  Inter Socraticos notissima fossa Cinaedos.
  \end{verse}

Di questi tali parla in diversi luoghi la Sacra Scrittura detestando tal vizio, come
abominevole, ma per brevità tralascio di riportarlo, contentandomi di chiudere
col detto dell'Evangelilta \textit{Atendite a falsis prophetis, qui veniunt in vestimentis
ovium, intrinsecus vero sunt lupi rapaces} e rimetter il Lettore a quello, che scrive
S. Matteo Evangelista al Cap. 6. 15.23.

Tale era appunto questo Perione, che faceva le dette Opere pie, non perché
veramente fusse buono, ma perché con esse pretendeva d'estorcer dal Cielo la
grazia d'haver figliuoli.

\item[SCHIATTA] Stirpe, Prosapia, famiglia.

\item[NON v'era, ne can ne gatta] Non vi rimaneva pur'uno. Plauto disse: \textit{Ne
  musca quidem domi est}, Del qual detto si servì quel servo dell'Imperator Domiziano
  che domandato, se Domiziano era solo in camera, rispose: \textit{Ne musca
  quidem est}, Perché Domiziano stava là dentro ammazzando le mosche. Ter.
  disse: \textit{Ne Sannione quidem relicto}.
\end{description}

\section{Stanza II.}
\begin{ottave}
 \flagverse{2}Così durò gran tempo, ma da zezzo,\\
Vedendo ch' ei non era esaudito\\
Essendo omai con gli anni in là un pezzo, \\
A mangiar cominciò del pan pentito;\\
E quant'ei far solea posto in disprezzo\\
Senza voler più dar del profferito,\\
Gettatosi all'avaro, ed al furfante\\
Cambiò la diadema in un turbante.
\end{ottave}


Continuò gran tempo Perione a far le narrate opere pie, ma veduto ch'ei non
era esaudito, e ch'ei non haveva figliuoli, e trovandosi già vecchio, perché veramente
egli era un di quei Bacchettoni furbi, che habbiamo detto sopra, e che
faceva bene solamente per interesse, si pentì d'haver fatto tante elemosine, ed
altro bene, e mutò costume.
\begin{description}
  \item[DA zezzo] Da ultimo. Forse meglio \textit{sezo}, venendo dal Latino \textit{secius} opposto
di \textit{ocius}. Vedi sotto C. 4. stan. 72.
\item[ESSENDO un pezzo in là con gli anni] Essendo grave d'età. Havendo molti
  anni. Vedi sotto C. 12, stan. 36.
\item[MANGIAR del pan pentito] Cioè si duole, si pente d' haver fatto del bene; ed
è quel \textit{facti poenitere} di Cicerone,
\item[POSTO in disprezzo quanto far solea] Cioè lasciando stare di fare elemosine, e
orazioni, ed altre opere pie come solea fare.
\item[SENZA voler dar del profferito] Senza voler dare più niente; e ne meno quello,
  che havea promesso, o proferto.
\item[GETTATOSI all'avaro] Divenuto avaro per elezione, o diremmo A posta.
\item[FVRFANTE] Vuol dir furbo scellerato, e ladro, e simili venendo dal latino
  barbaro \textit{foris faciens}, operante fuori del dovere, ma si piglia anche per Spilorcio,
  ed avaro, come è preso nel presente luogo.
\item[CAMBIO' la diadema in un turbante] Di Santo divenne Turco, che Diadema
appresso di noi vuol dire quell'ornamento, ò corona di splendori, che si vede
dipinto attorno alla testa de' Santi. Dice che cambio la diadema, che meritava
come Santo, in un turbante, cioè cappello da Turco, non che veramente si mettesse
il Turbante, ma intende, che d'huomo da bene diventò tutto il contrario.
\end{description}
\section{Stanza III}
\begin{ottave}
\flagverse{3}Di poi tutto diverso, e mal disposto\\
In modo degli Dei faceasi beffe,\\
Che s'egli udia trattarne, havria più tosto\\
Voluto sul mostaccio uno sberleffo;\\
La moglie un miglio si tenea discosto,\\
E dov'ei dava a' poveri a bizzeffe,\\
Quando picchiavan poi dalla finestra,\\
Facea lor dar il pan con la balestra.
\end{ottave}

Divenuto Perione tutto diverso da quel che era, come s'è detto, cominciò
anche a non stimar più gli Dei, anzi gli strapazava in modo, che havrebbe voluto
più tosto un sfregio sul viso, che sentirgli nominare; sbandì la moglie, ed in
vece di dar limosine a i poveri gli bastonava.
\begin{description}
\item[DIVERSO] Cioè differente da quel ch'era prima. Se ben questa voce diverso
  significa ancora stravagante. Vedi sotto C. 8. stan. 17. ed in questo senso la piglia
  Franco Sacchetti Nov. 29, E questa natura pare a me, che fusse delle strane, e diverse
  che trovar si potessero. E Nov. 78. \textit{Ed era un'huomo malizioso, reo, e di
    diversa natura}.
\item[FACEASI beffe] Si burlava. Non faceva stima. E il latino \textit{flocci facere}.

\item[SBERLEFFE] Taglio, o sfregio, che i Latini dissero stigma; \textit{Rigido signata
 stigmate fronte}. E perché gli sfregi in sul vifo sono cosa ignominiosa, come s'è
  detto sopra C. 1. stan. 66. da ciò si deduce che Perione havria più tosto sopportata
  ogni grande ingiuria, ed ignominia, che sentir nominare gli Dei. Il Coppetta
  nel Cap. in lode della sig. Ortenzia piglia la voce \textit{sberleffe} in significato di burlare
  uno, con oltraggi, e punture, che hoggi da molti si dice Fare uno scappeneo.
\begin{verse}
Allor l'amico in mezzo a i dolor miei
Mi fece uno sberleffe di velluto,
E mi fece arrossir dal capo a piei.
\end{verse}
E più sotto nel medesimo capitolo lo stesso mostra, che habbiamo anco il verbo
sberleffare dicendo.
\begin{verse}
E col rider di grazia andate piano,
Che non è per infermi util conforto,
E chi vuol sberleffar, sberleffi in vano.
\end{verse}

L'origine da questa voce \textit{sberleffe} vien forse da \textit{Berlina} in questo modo:

Si suole alle volte, dopo haver tenuto in Berlina i ladroncelli, segnargli in
qualche parte del corpo con un ferro infuocato, acciò che fieno dalla Giuitizia
riconosciuti, se altra volta per commessi delitti li tornassero nelle mani. E di
questi segni vedremo sotto C. 6. stan. 54. Ciò si costumava ancora appresso gli
antichi Romani ne i servi fuggitivi, e gli segnavano nella fronte come si cava da
Aulonio Epig. 15. che parlando di un servo nominato Pergamo dice.
\begin{verse}
  \backspace Iam segnis scriptor, quam lentus, Pergame, cursor
  Fugisti, \& primo captus es in stadio; 
  \backspace Ergo notas scripto tolerasti Pergame vultu,
  Et quas neglexit dextera, frons patitur.
\end{verse}

Et aggiungesi alla voce \textit{berlina} quella finale \textit{effe}, da quella lettera maiuscola F,
che è il segno, o marchio, col quale si marchiano i detti delinquenti. Che cosa
sia berlina. Vedi sotto in questo C. stan 15.

\item[MOSTACCIO] Faccia, Volto, ec.

\item[TENEA la moglie discosto un miglio] Tenea la moglie lontana da se, intendi non
  volea più commerzio con la moglie. Lat; \textit{secubabat}.

\item[DARE a Bizzeffe] Dare, o donare largamente. Questa voce, che è composta
  dal latino \textit{bis, \& effe}, cioè due volte, f, vuol dir pienamente, largamente,
  abondantemente, e simili; Quando il sommo Magistrato Romano intendeva
  fare ad un supplicante la grazia senza limitazione, ma pienamente faceva il rescritto
  sotto al memoriale, che diceva \textit{Fiat Fiat}, che poi per brevità costumarono
  di dimostrare questa pieneza di grazia con segnare i memoriali con sole due
  effe, onde quello che conseguiva tal grazia diceva: Io ho havuta la grazia a \textit{bis
    effe}, cioè due volte ff che s'intende grazia intera, e piena, al costrario di quella
  limitata, che era con una sola effe aggiontavi la limitazione, o condizione
  con la quale il Magistrato havea conceduta la grazia. E' da questo \textit{bis effe} s'è poi
  corrottamente introdotto il dir Bizzeffe, che ha il signiticato, che habbiamo
  detto. Nella storia di Semifonte scritta sopra 300 anni sono, si legge al trattato
  terzo. \textit{La Terra di Semifonte era piena di torri merlate, e piombatoie, e di Torricelle
    a bizzeffe}.
\item[DARE il pan con la balestra] Vuol dice strapazare. Fare in maniera, che il
  benefizio sia di disgusto a chi lo riceve. Deriva forse dall'uso, che era in Firenze
  avanti che usasse andar a caccia con l'archibuso, di tenere al suo servizio huomini
  a posta i quali con qualche fsalvaticina mantenessero le mense de i grandi, e
  questo esercizio essendo d'utile, ma assai laborioso, può haver data origine a
  questo Proverbio \textit{dare il pan con la balestra}, cioè accompagnato da fatica, e disagio
  grandissimo. Ma nel presente luogo intende che effettivamente facesse tirare
  balestrate a i poveri.

Si dice ancora in questo proposito. \textit{Porger il pane con la spada}, e ciò forse deriva
da quello, che fece Dionisio Tiranno a un tal Democle Filosofo, il quale
(perché adulando eccedeva in lodare le grandezze di quello stato di Dionisio)
egli fece sedere ad una mensa ripiena delle più esquisite vivande, che per un banchetto
reale inventar si potessero; e fece attaccare per il manico ad una setola
pendente con la punta sopr'alla sua testa, una spada sfoderata, la quale veduta
dal Fitosofo, gli cagionò così grande spavento, che egli non potè se non con molta
paura, e con poco gusto pigliare di quei cibi. Di costui parla Orazio Od.\
pr. lib.~3.
\begin{verse}
 Districtus ensis cui super impia
 Cervice pendet, non siculae dapes
 Dulcem elaborabunt saporem.
 \end{verse}

Si dice ancora, a questo proposito, \textit{dare il par col bastone} che ha origine da
quel che fece il Piovano Arlotto; il quale per gastigar l'indiscretezza d'alcuni
cacciatori, che gli havevano lasciato in casa un branco di cani, quando a questi
dava il pane, l'accompagnava con una mano di bastonate, onde i poveri cani
s'erano assuefatti quando vedevano il pane a fuggire; per lo che divennero cotanto
magri, che a pena si reggevano in piedi. Ritornati i cacciatori per li loro
cani, vedutigli così sfatti si dolevano del Piovano; ma egli preso in mano il solito
bastone, tirò loro in terra alcuni pezzi di pane, ed i cani ricordevoli di come
era solito passare il negozio, in vece d'accostarsi al pane fuggivano, onde il
Pidovano si scusò co i cacciatori dicendo: Come volete che ingrassino, se quando
io do loro il pane,  fuggono come vedere ? E da questa facezia venne questo
proverbio \textit{dar il pan col bastone}, che significa mostrar di voler far del bene a uno,
e fargli del male. Seneca ci fa veder questo modo di dire anche appresso a i Latini,
raccontando il detto di Fabio per soprannome Verrucoso, che il piacere
fatto da persona zotica, e con maniera salvatica chiamava \textit{Panem lapidosum}, che
è appropriato al nostro detto \textit{Dare il pane, e la sassata}.
\item[BALESTRA] Strumento, o arme da caccia, col quale si scagliano palle di
terra secca, nella guisa che si fa delle frecce; e serve per ammazzare uccelletti.
È composta d'un'arco d'acciaio accomodato in cima a un'asta, o legno torto,
dentro al quale sono adattati altri ordinghi di ferro per facilitare l'operazione.
Viene dall'antica ballista arme guerriera, che dicevano ballista forse dal Greco
\textit{ballein}, che significa scagliare.
\end{description}

\section{Stanza IV.}
\begin{ottave}
\flagverse{4}La plebe, i grandi, ed ogni lor ministro\\
Ch'il Duca così buono havean provato,\\
Mentre fu scudo ad ogni lor sinistro\\
Ed in lor pro sarebbesi sparato,\\
Vedutolo così mutar registro,\\
E diventar un turco rinnegato,\\
Eran talmente d'animo cattivo,\\
Che l'havrebbon voluto ingoiar vivo.
\end{ottave}

Per questa mutazione del Duca di buono in cattivo, li suoi sudditi, che prima
l'amavano, cominciarono a portargli odio, e bramargli ogni male.
\begin{description}
  \item[SI sarebbe sparato in lor pro] Havrebbe fatto loro ogni favore immaginabile.
Havrebbe messa, e spesa la propria vita a benefizio loro, e la  voce pro è un sustantivo
che significa giovamento, utile, ec. dal latino \textit{prodest}.
\item[MUTAR regifiro] Mutar maniera di fare., Registro diciamo quell'.ordine di
ferri, il quale & negli Organi strumenti musicali, con ciafeuno de' quali ferri al-
zandolo, o abbaflandolo si da, o leva il fiato a quelle canne, le quali si vuol,
che suonino o nd, ad esserto di far mutar voce all' organo, il che si dice smuear
regifire, che pafiato poi in proverbio significa Mutar maniera, o- modo di
in qualfivoguia cosa. Vedi sotto C8. stan, 52. alla voce protocollo Regi/tro in
altro significato.
\item[INGOIARE] Trangugiare. Mandar gil in corpo una cosa senza anche ma-
flicarla, che si dice anche sngullare. Vedi sotto C. 1. stan. 6.
\end{description}

\section{Stanza V.}
\begin{ottave}
\flagverse{5}Avvenne, che già inteso un Negromante \\
C'un'huom com'era quei sì giusto, e magno,\\ 
Faceva novita sì stravagante, \\
Vn'atto volle far da buon compagno;\\ 
E per ridurlo all'opre buone, e sante\\
Non per speranza di verun guadagno\\
Fintosi un baro, a dargli ando l'assalto,\\
Un po di ben chiedendo per sant'alto.
\end{ottave}


Stando le cose ne i suddetti termini, Va tal mago, inteso che un -huomo
bene come era Perione s' era cangiato in così cattivo, volle fare un' acto da hus
mo da bene, cercando di rimettere Periane nelia*buona firada, e però fintoft
un' accactone, andò.a chiedergli l'elemosina per amor di Dio.

WEGROMANTE. Flo fleffo che Mago: Se bene Negromante venendo da
negromanzia s' intende colui, che per mortuos vaticinarur, che è una delle sei spe>
cie di Magi detti sopra C, 1, stanza 20.5 tuttavia da noi si piglia per nome geng-
rico, e per intendere ogni specie di ae ye di magia. i:

BARO. Biante. Accattone fallo.» ien forse dal Greco Barijs, Bareos., che
suona molefus, importuno, sfrontato, come appunto sono questi tali; e se beng
questa parola ha del furbesco pure s' usa comunemente, ¢' usd il Varchi St. Fior,
lib. 11, Ed in segno, che lo rifiurava,¢ non gli oreduea pil, havendolo per baro af
giumatore,arfe i suoi libri, i

PER Sant' alto. Cioè per Dio. E,parlar furbesco y il quale forse &noto fuori
della nostra Tuscana, come inventato da Vagabondi, Monelli e tianti per non
esser intesi, se non da i lor pati, e poi fattofi familiare a mole' altri, a.segn
che ne-è fatto, stampato il vocabolario.. Si dice anche parlare im gexgo,ed in lingwe

furfantina, come Cl mostra il Vaichi Sr, Fior, lib. 15. Apparifeono più lertere scritte
non ins cifra, main gerga a uso ds lingua furfantina moairo mse « I nostro Poeta. |
ferye di tal parlare nella persona di questo Biante perché, come ho detto; si
huomini son soliti pariar in Pa ao i; eoaipouis ponte'

Rispose Perione -. Pratel mia, « See bai bisognos che posso fare iedns,
se ru te lo credeffi tat? inganm 5 Che son Frafaxia sche rifaccia':
Tu anoi ch' io doni per ? amar di Dio, B che penst sche qus ci sia la cava?

Ne [ai ch' s0 pigliorei per San Giovanni, Non e più sompo che Lerta filava

'Aila richiefta del Mago Perione non si muove a far limolina., anaid = che»

pigtersDhe anch' egli qualcofa, e che e palato quel tempo che egli dava via
1b tua., ¥ e

PIGLIEREL per San Giovauni. §. Gio; Batifta ¢1l Santo provenore seen

atta



oh

;












SECONDO CANTARE, vhs

Città di Firenze,e perciò il giorno della sua fefta e grandememe folennizzato 5 ed
in quel giorno son sicuri nella Città finovi banditi capitali, ficché gli Sbirri non,
Hidnipislanatins + Daquesto è nato' equivoco Proverbio; Pigiterebbe il dé
di San,Giovannir, 0. per San Giovanni, che: vuol dice Piglierebbe anche quel di,
nel quale ne meno i birri pigliano,; ¢s' intende pigdieredbe, cioè accetcerebbe tutto

che gli futle-dato ih-ogai occaiione y ed in ogni tempo. B lo scherzo & nel
verbo pigliare che vuol dir Far cattura, o Catturare,e vuol dire anche Accettare,
© ricevere, come s'intende in questo proverbio; che esprime; Lo piglicrei, ed
accetterei sempre, e non darei mai.

CHE son Fraffazio, Raccontano una favola d' una donna non troppo hone-
sta, la quale havendo commerzio con un tai' huomo detto Fraffazio, fu con esso
una volta trovata dal marico; ed essendo ella altrettauto fagace, quanto il ma-
rito semplice:, e di cervello grosso, gli dicde facilmente a credere, che colui era
un' huomo da bene 5 che andava rifacendo i danni a chiunque occorreva qualche
disgrazia, e che l'haveva chiamato in casa affinché le ricompraffe una faa con-
ca, la quales' era rotta,¢ che appunto gli narrava questo suo danao; foggiun~
gendo; B come, Marito mio! Non conoscete dunque Fraffazio? Il buoa ma-
rito se la bevve, e così la donna scampé la furia, E da questa favola, quando &
dice: fer Fraffazid, vuol dit: » Esser colui che [pende il suo per folevar t aitrui mi-
Seriesye che risa. i danni come dice il nostro poeta.

CHE pensi, che qua ci sia la cava, Pensi che io habbia la cava de' danari, cioè

+ Torna bene a questo detto quel che si trova in Saluftio; Cen/es me vi-
em ararij praftare. Non e pero che cava voglia dire la Zecca, ma si piglia per
questa nel presente detto ( da noi usatiimo tn questo proposito ) perché si sup;
ne 5 ed è verifimile che la Zecca, come luogo dove si batte la moneta, ne fais
abondante, come sono abondanti le cave di quelle cose, che da esse eftraggonfi.

NON e più ib cempo che berta flava. Non € più il tempo, che le cose andavano
come si bramava. 1 tempi son mutati. Pipino Re di Francia per mezzo di suoi
Ambasciadori sposd Berta dal Gran pi figiwuoia di Filippo Re d' Vagheria, las
quaic havendo saputo\, che questo suo Sposo era brutto, © nano, malvolentieri
s? accomodava a dare H conienfo; ma pure, vinta dalla riverenza dovuta ai pa-
dre, condescefe, Arrivata in Prancia, lasciandosi governare dal giovenil senti-
mento;richiefe Elifetta di Maganza un fegretaria ( la quale 4'Vagheria,dove era
naca del Conte Guglielmo di Maganza ribello di Prancia,(e ne vemiva-con Berta a
Pacigi ) che voleile, fingendosi la sua persona, in sua vece sposarsi con Pipino
il quale,¢ pera somighianza, che era fra lor due, '¢ per non haver Pipino mai 4
veduta Berta, non' havrebbe assolutamente riconosciuta, Bliferta da ere
si mostro renitente; ma persuala poi da Grifone,¢ Spinardo di Maganza sui
parenti, condescele a i voleri di 3, B.così arrivatia Parigi, Elifecta si spo-

80 con Pipino in vece di Berta. La quai Berta in tanto di consiglio di detti due

Maganzefi s' era ritirata in ludgo vicino a Parigi, con pensiero fermato cons

-decti Maganzefi di quindi occultamente partir, e tornarfene alla patria com

Aaiuto de' medesimi; ma questi la cradirono, perché in vece di servirla alla vol-

ta della patria sua, ' inniarono ad-un bosco, con urdine a quelli, che la con-

'ducevano, che l'uccidetlero: Mu coitoro mou a picea, in veced' sagen Ihe
bi spo-






p MALMANTILE

spogliarono, e legatala ad un' albero la lasciarono in preda alla Fortuna, e tor-
narono a i Maganzefi, dicendo che l'haveano uccifa 1 Maganzefi per occulta~
re si atroce delitto fecera morire tutti quei ficarj, havendo prima anche-d' arri~
vare a Parigi fatte ritorpare in Vagheria tutte le dame, ed\altre: personenons
complici, ne confapeyoli di si geande scelleraggine 6} cs 2th aawmiss
Berta intanto, che se ne stava così legata:dolendoGi 5 ¢:lamentandosi fu sentita
da un tal Lamberto Cacciatore dei Re Pipino; Costui seguitando la voce ficcon-
duffle dove stava Berta legata all' albero., e scioltala., alla propria.casa la con=
dufle, e la consegno alla moglic yeftendola d' abiti vili, e conformijalla 'posibili-
ta di lui, ed alla poyera condiziane, della: quale Berta disse dessere 2» Quivi
stette Berta circa cingue anni 5 nci qual tempo guadagad molti: denari di) filare}
ed altri lavori,, che insieme con:le figliuole di Lamberto facevacs: Avvenne una
giorno, che essendo Pipino a caccia si condutic folo alla Cala di Lamberto, ove
veduta Berta s' inuaght di lei 5.¢ con essa si:congiunfe sopra.ad-un suo carro 5 nel
qual congiungimento fu gencrato Carlo, coskdetto dal amedefimo Carlo. dn ta»
le occasione Berta scoperfe a Pipino ii tradimento. de i Maganzeft narrandoli
tutto il seguito; perloché Pipino fece abbruciare Elifecta, cd una mano di Mas
ganzefi, e rimefie nel trono Berta, 3h bh fob eomearr« plidhoss
Da questa favolosa foria nacque ilproverbio; Wom è più. ikvempo che Berta fie
lava, Cioè non e pil il tempo che Berta flava neile feive filando., e ricamandoy,
che-significa; Le cose son mutate, ' i Pas X\

Di questo dettorfi servi Berta moglie d' Arrigo 1V,Imperatore, come si vede
nello Scardeonio Monameata Patavina lib. 3. Cialie 1g. de Berta ex Montagna~
no, le di cui parole son queite. Ademonatur ia iifdem Pacavinis eAnnalibns celebris
fama Berteex Vico Montagnani, qus quidem fait ruflicano genere, fed moribus certe
perquam nobilis CO animo perguam generosa, ach
Hee enim tempore Henrici IV, Imperatoris, cum eius uxor, Berta & ipfa muncupa-

ta, Pacavij moraretur, vel cinfdem force nominis similitudine, vel propria generositas
te animi allecta, obrulit ei dono filum tenuiffimum 5 quod-eleganter [amer neverat mar
nin 5, in Vrbem venale detulerat Quod munus Regina bilari vulew accepit 3:
cum, cognoviffet nomen, OF animum mulicris, cam indignam cenfuit, xt vitam inopem
Samineo colo amplius fuspineret fuam,, Dato icague filo procuratori suo, inber ad Pagnm
Monragnani frarim proficisci, ubi mufier habicabat,& pro referends gratia tor terra
ingeraei ex publica adferihi, quantum [pacij filum dono datum extenfum. comprehen-
dere,@ cwrcumdare posset, Quod.cum catere mulieres vidifsent, ilico Berta exemple
attulerunt »& ip[e filum, quod Regina dono darent. At ipfa renuens id ab alijs acci
pere percanté re/pondir, Pertranfye tempus, dum Berta filabat. 3 sree
Gliantichi digevano Non ef amplins atas Cyclopum,ed in. moluc.a)tremaniersficame
Ancor noi diciamo: E finita dacuccagna,o la vignuoladVen e piri cempo ai Bartolommery
ec. Cont quali, ed alpri detti intendiamo Non:si godono pil quelic felicitache già:



si godevano. STANZA. VIL 2
» Signor ( foggiunfe il Adago) mi [a male Hor bapa; Chi del miofac.
Di veder, ¢' un si gran limofiniere, ( Difs egli) fa la xappa nel '
E4 huom tanto benigno, e liberale Pero va in pace tu ca' tuoi bifeget 5
Caduto sia nel mal del miferere.. Pevche per me tx mangerai de' her's



"igi




SECONDO CANTARE: 3

ll fegromante vedendosi cacciar via con tal risposta; replicd, che gli dispia-
eeya,;ch' ei fuile diventato avaro. E Perione li foggiuafe, ch' ci non sperafle da
lui faffidio alcuno..

CADVTO net wial del miferere. Divenuto mifero, cioè avaro, tenace, che se
bene il mal del Miferere è una infermita mortale; Noi ci serviamo della voces
AMiferere nella forma che habbiamo detto sopra:C. 1. stan. 80. della voce boceo/i-
ea; per intender mifere, che nel presente luogo vuol dire avaro; e così è inteso
comunemente, se bene la voce 44/ere propriamente vuol dire infelice 3

FAR capitate. Par' aflegnamento; o sperare nell' aiuto d' alcuno. Vedi sotto
C. 7. stan. 82. Questa voce capitals € dedotta da capirario capitationis, che era una
tafla, otributo, che determinavati ix capita Leaner per aflegnamento; e pro-

tk \priamente capitale del Principe, come ¢€ forse la Decima, che pagano hogei i
nofter contadini, chepure si dice decima in fu latefta.
| RANIERE B.un valo intesluto, e composto:di fili.di vetrice, o-d'\altra spe-
| cie d' albero,.0 di fottilissime strisce di legno in-figure:,¢ forme varie, in tuttes
i le\quali che tieno., ha fempresil manico; che senza manico si chiama corbello,o
' panicra 5 € servono Perporcar frutte, o altra.che sia; detto paniere, o panicra
torfe es » perehé gli antichi tenevano il pane in tal forte di cefta in mezzo
alic menfe, e perciò dat Lacini detto Patarium.
\ Fed Kila ruppanelipanere. Questo proverbio dice;
Chifal alsrni meftiere
ring ta es eeude. en
JE cosìdichiara iluo significato., quale ¢: Che colui, il quale si mette a fare
tuna.cosa,, che non fa) fare, non fara nulla di buono; ed in fultanza vuol dire; Af-
anne wane. Ovid. libs )2.
» Kique liquor rari fub pondere cribri
© Ede forse:meglio dir Juppa, che 2xppa yenendo dal verbo /uppurare, che vuol
dire attrarre ? umido; o da Suppen r Tedeseo.- Vedi sotto C. 4. stan. 25. Ma lufo
isiobligaa dir zup
ok tn pace, aan pfamo. dire y quando:mandiamo via i poveri, che accatta>”
oj |) asd.in un certo, modo Pjauto in milit. dicendo Pax, abi,
MANGERAT de sogni. Mangerai cose immaginarie, clog non mangerai.
ee ele Capitolo della paverta dice.:
of can ls vn Chevsfacciara ralor non si vergogni
| lo permetta ye faccia male,
Latin i on » che non cake viver di segni,
if  A Latin Latini pure bayevan Gmil modo di-dire, come si vede in Givyenale Sat. 6.
ia Qualiacumaue voles ude fomnia vendunt.

E.coloro, che map veaeie accents dana cosa,sogliono fognarla; perché

Saag non e il sogno, che
Vn'

ha corvette
. 9. introduce un, Pastore, che raccontando le sue felicica

Poss ideo quecumyue Tolent in notte videri
Ln [omnis, vim magnam ovium — capellas.



'La onde, Teocrito,
sGosi ragiona:


4 MALMANTIDE

Er anco noto Nonio, che appresso gli antichi Roniani, il verbo ve/cer signifi-
cava vedere: Prius quam infans esserytni oculi facinus vefeuntir cioè videnr;come noi
Pure diciamo; Mangiar un con gli occhi, quando altri guarda uno:con grande ac-
tenzione; ¢diciamo anche:. Dar paffo acti octhi. Dan. Par. Ci 27,08

Efa natura, ed arte le pasture
Da pigliar occhi: 2 ¥

Si che dicendo mangerai de fagni, si può anche intendere, Ti faxieruijeifeddisfa.
rai con dar pasto a gli occhi; o della vifea; che e ho tictioche non mangerai. Vedi
sotto C6. fan, 55. che er la vista.







STANZA. V4ill, STANZA IX)

Comer replico quei.) se.e' si.cicalay E non barteva la mia fine altrove 0%
Che tu darefti via fin ta gouneliay C'adhaver prima ch' io ferraffi li ocehi
Vedendomi [pedato, e per.la mala dn ricompenfa un d:, piacendo a Giove',
Petra haver' jt eranchio.alla fearfella? Della miadonnaquattr'sfeimarmocchi,
foi che tu grarti ij corpe alla cicala Ma finalmene dopo mille prove 3
(Life it Duca) ia levi qucfta cannella Didar' il lustro a marmi coi ginocchi',
ser quel ch'io ti diro, percht se già Tenendo giochi in molte,e il colloavite,
Donai-, non era tuttn carita. E Le nocca cot perro sempre in lite;

STAN ZA) Kina ere ' ars

Lot bebbi bianca a femmine,ed a mafebi, Perché. po poi(difs' io)2li¢ me'chtiocaschi
Ond' io sbraciar volendo a bel diletto, Dalle finestre prima, che dal tetto;
Mi risoluei levar quel vin da fiaschi, - Bil cavarmii di mano aiteffo un pelo,

vE aomdar pile quant'un pinrabd'agherco, Sarebbe un voler dare un pug hioin Cielo,

1, Mago mofira di-non ipoter eredere, che havendo Perione nome'di liberalif-
firio, non s' habbia a muover' a compatlione di Iwi 5 e Perione vinto dal” impor-
tunira di costur, gli dice, che fu già liberale'per disporre'il Cielo 'a concedergli
fig\iuoli; ma perché eg? non era stato efaudito, lascid di far più limofine, ed
hora era impotlibile cavargli di mano'un picciolo.,

Sf cicula, Cioè si dice; Si dilcorre. Il verbo cicalare-usato in'quefi termini
«sprime discorso di-cosa incerta, che frdice anco bucinare y oburicare', E si dice:
la tal cosa non fu poi vera;ma fu una cicalataycioè sene parlo; ma non'é poi stata

vera. “ terigacih;
'DAREST 1 via fin'la ¢onnella « DareRi via fino al'proprio veRivo; datetti via
tutto il'tuo havere.. E se bene gonnellas' intende una specie'd' abito da donna, in
qucito proverbio diventa nome generico per ogni forte d' abito.
SPEDATO,, Cie co' piedi laceri dal viaggio. ' vere
PER la wala. Cioè per la mala via'; e's' intende thal condotto di fawita,e mal'
all' ordine di vestito, e feaza danari '. tiowk wie? 3 mS as
HAVER il granchio alla fearfella'. ChiatniathorGrianebio, © ee ndspetie di
malattiadi spafimo, la quale quando viene alle mani impedifee'il maneggiare le
dita; E da questa quando diciatho W rule bait eranchio alla forfella intendiai
non può adoperare lé tani incorno alla borsa, che'vuol dire; & pigro acavar de.
nari della borsa, cioè, adire: e tenace,, o avaro., ed uno, de' quali parlando
Marziale dice.: aie



ys








SECONDO CANTARE. #8
tpi. a Bitigat'y& podagra Diodorus, Flave', taborant;
ne nh ' Sed mil Patrono porrigit; bac Chiragra est.: j
tx. | Enoi pure diciamo di questi tali; Haver la gorta alle mani, Haver i pedignoni

alle mani; Haver le mani acgranchiate; farebbe a pagar co' monchi,
SCARSELLA, Intendiamo ogni forte di tasca, o borsa di danari, come si
vede forte C, 3. stan. 5., se bene scarfella e propriamente una borfetta di quoio
if Con ferrature di ferro fatta alla foggia delle Carniere da'cacciatori; fa qual forte
ei di borsa usava già in Firenze portarsi da tutti legata a cintola..
GRATT AR' il corpo alla cicala, Incitar' uno a discorrere. Vedi sopra Cant.
primo fan. 2.1 Latini pure difflero in questo propolito Crcadam ala comprehendere.
LEV AR la cannella, Defiftere di fare una tal cosa. Traslato 'dalla botte, alla

hi fileva la cannella,quando è finito il vino, che era in essa, K cannella inten-
t) amo quel legnetto tondo forato per lungo, che si adatta al fondo della bottes
bi, per cavarne il vino,la quale da i Latini con voce Greca si dice epiffominm, Si dice

anche in questo proposito.
i LEVAK A! vino da fiaschi, come vedremo appresso.
; PRIMA ch io ferraffi eli ochi. Prima ché io mor:ffi.
MARMOCC HE, Ragazzi. Queita voce marmocchio in significato di fanciullo,
viene da marmo, alla pulitezza,e liscio del quale s' 9flomiglia il liscio, e pulitezza
del volto de i fanciulli, e delle fanciullette. Or. Od.-19. lib. 1.
h Vrit me Glycere nitor
Splendentis "Paria marmore purius.
DAR il lustro @ warnii'eo' i ginocchi. Cioè Naya tanto tempo, € così speffo in,
ginocchioni, che il lungo fregare con le ginocchia faceva divenir lucentt i mar-
, ini', sopra i quali s'inginocchiava.
1) FENENDO gli occhi in molle. Cio' lagrimando, e cos) tenendo gli occhi in
molle nelic lagrime. 2
COLLO a'vite. Collo torto, come fanno i Bacchettoni. Si dice a vite per si-
militudine, essendo /a ve uno strumento; il quale serve per ferrar' un materiale
con l'altro, che peg essere attorcigliato come /a vite pianta, che produce l'uva,
sn <r? nome', e si dice anche marti e chiocciola: quello dal torcere, col
' > fa 'operazior ione; e questa per la similitudine, cHe ha la sua figura con
 il'gufeio della chiocciola a.
BLE nocea cal petto sempre in lire, Cioè dandosi delle pugna net petto; il
che mostra che le nvcca fiego in lite col petto, mentre non ceflano di pei quoterlo.
E nocca'intendiamo nodelii delle dita. Vedi sotto C, 3. stan. 8., ¢C. 9. Itan. 54.
In somma i} Poeta con queste quattro maniere di dire, ioe Dar' il laitro a' marini
0", ginocchi § 'Tenere chi in male, Haver il colle a vite; e le nocea sempre in lite»
se eee Pie meer” orande; e descrive assai bene un' Hipo-
a; ao € Al» "
ALO F bebbi Slee pens ha da conseguire per via d'eftrazioae “
di-polizze ( come si fa al lotto ) sono scritte Wolaniedele polizze premiate,¢ l'al-
tre son bianche; e chi ha una polizza bianca,non conseguilee premio alcuno. E =
di qui viene:il detto 40 / ho bavuta bianca, che è fatto comune, e per intender di;.
futte quelle cose, che si tenta di conseguire, e non si conseguiscono.
ene c K2 SBRA-





tt
Mar


96 /MAEMANTILE

S8RACIARE, Vuol propriamente dire, allargare, e follevare la brace a fine,
che meglio s' accenda, € renda più calore; ma per metafora intendiamo spender
prodigamente, e largamente, come.s' intende nel presente luogo, e sotto Cant, 3.

an. 2. is

A bel diletto., A posta; o per gusto, ma senza buon fine, e utile 5 e si dice an-
che a bello fiudio., a bella posta, a bella prova., che wuiti si posion Pigliare in questo
senso., Se bene alcune volte significano quel, che i latini dissero dedita opera es»
maffime quando non v'é l' aggiunta di be//a, che in questo.calo e detto ironica~
mente, ed ha forza d' csprimere Liafimevole y come per elempio Veramente tu bat
atta wna bella cosa, cide w hai fatto una cosa biafimevole,.¢ che fla male « 4irg,
Egreciam vero landem, & spolia ampla reportas.

NON darei quanto un puntal d' agherto, Liaghetto una cordicella fatta di.feta,
© d'altro, che serve per afhbbiar le velti, e adattarle alla persona, alla qual. cor-
dicella è solito fare una punta di sottil lamina d ottone, o d'.altro metallo,,¢
gucfte punte si dicono panrali, e di queste punte fen' hanno due,.0.tre per un»
quattrino;¢ da questa vilta serve,ilpresente detto per esprimere; Non dares wiente,
ne meno una cosa, che non val nulla. Che 4 latini difiero fra' altre molto, /7-
tiofam nucem non dederim. E noi pure diciamo we fico fecce, um dnpino-, e Gmili.
Vedi sotto C, 3. stan. 8.

LEV AR il vin da faschi, 1 senso metaforico è lo stesso, che levar la cannella
detto poco sopra stan. 8,

- PO poi, Alla fine. All ultimo de gli ultimi.. Opera anco.in questo detto la,
forza della replica, che induce superlativo, Vedi sorto in questo C, stan. 73,

GL' è me ch' io caschi dalle finestre prima che da} tetra, Nel male & il meglio,' ¢-
leggere il meno. Intende; egli e meglio, che io lasci flare di dare il mio. che se-
guitare, e darlo via tutto, cioè mi conteati di questo danno,, e non lo faccia
maggiore col seguitare a profondere il mio. E que! me per meglio e la figuras
Apocope da noi spesso usata; e l'uso Dante più volte, ma notabilmente nel C.
32, dell' Inferno, che l'usd nel principio del periodo..., aceatijaat

Me fofte fhare qui pecore, o zebe. sda ene t nos
Ma di questa figura Apocope, e come | usiamo, vedi sotto in questo C., stan, 36.

CAVARMI di mano un pelo, Conleguir da. me cosa alcuna, ancor, che di nian
valore. -; slob ned

SAREBBE un voler dare un pugno in Cielo, Sarebbe un yoler tentar,unacofas
impotbile, Facilius Calum digito actingeres, s>










eng ANZA Xt. RIAN Zh AM, Se

Che pagherefti ( disse f editro da te nov alpettar ch' io. chiedds;\

dm “here
Se cio fulle ( rispe/e Perion « Rerehisamanciique abiesta altri mivedas
Mocor eis nod yh fabtia tins dfegoa, \ deboine, vale jd li von
Eraluoglia appiccara habbiaaltarpiones rf fra se ru brami a day
to ti vorrei donar mez" il f pil regno dopo regoverniy m!
Sogginn[e quei: Non vo pur'uns sane 2a ling y
4a solamente La tia 4 Cua cuor th portin a. i

13 Siglo th s D389. 3G



* STAN










ae per = en

SECON DO CANTARE, 77
S T.A.N)Z.A XH.
Ed ordina di poi, che se ne quoca.» « Presa che thaglie fateoit becca allvca,
> Ga terza parte in circa arrofto,e leffa, Che subito ch'in corpo se eacfa,
» (Chim tutti modi è buona)edan'un poca Senza che tu pits altro le apparecehi,
nds nquel modo 4 mangiar alla Ducheffa; Dotrela pregna infin [opr'a sii th
1 mago s' efibisce a dare a Perione il modo, che la sua moglieimpregni;

Perione glidice che se ciò segue li vuol donar t mezzo il suo regno; ed il ak
ricusando il tutto y.da.a Perione laricetta dellAsino marino per impregnar la.

maoglic..
CHE: pagherefti ? Quando veggiamo uno, che fommamente brama di sapere,o
a otenete una cosa; per mostrare, che € in nostra potefta l'adempire il suo defi-
ae »fogliamo dire; Che pagherafti? Che /pendere/ti ? Quanto darefti? © simili,
ares fied » o.diceffi la tal cosa ?
| EGONVE »\ Maliardo, Mago, Negromante, ec, Viene dal.latino, feeon,
| doh oer Murcto nelle sue varie lezioni lib, 12, c.19. emendando un jug-
di, Plauto nelle B « Longurs eft Stris ion... Strigas
( dice egli ) vocabant mulieres y quas etiam noitn velare arbicrabantnn » codemgue.sode
rer: bamines.maleficos, 5 quorum vocabuiorum vulgus in Italiansitur, Vedi soto
3: 69.
, L0.non ve pite.disegne. Yo non ho pila speranza d' ottenere questa cosa. N'ho
affatto levato l'animo, ° il pensiero e
= ARRIGUCARE la: ong te arpione.. Haver lasciata la voglia y o i} desiderio
Eee lo. che e-dppicear' al chido vio sopra C, 1. fan. 8, Eque-
se procede, da i voti-, che anticamente facevanoii Gentili,





jn Tempio; i.quali non & potevano levare di dove eran polti,

DS coupe ticels in uso comune, © profano.
0 RINE, E una specie di chiodo uncinato per uso di regger l'imposle delle
parte netinetire,, girando, quelle sopra di edi. Da i Latiai decti Cardizes.

Se 40 pear Eas Non, yoglio danari. Crazia & delle più vili monete

rESOtO pas essendo, ? ortava parte del giulio.

sn: 2 e maflime dalla gente vile per cA ceaense Nona

sects ta cosa.










a servitore agli huomini xirmofi, edi gar- i
> I tale» & un' hanna egpiesode il.detto di Diogene

diamo huomo dotto, virwolo, e di tutta perfezione.

serine de ale PAutore si serve anche acliottava, 7, ante-

sedente.), seaoecs is aserstaone dun discorso, e paflaggio ad
'oro sonia d ee 3 Ha bastanza quanto. habbiamodetta,per
'gonchiudere iLcome., ndo; o pe 90 non farelaral cof... —~;
« REDA. Cioè lyecettion ide ° danende fighuoli..: Seer: reday
iL tale ha hanuto.nn figh E egies Vinee iorentina »: sprepufata x€
solamente per i contadi; d intendono anche i Peet Lic beitic.

MOSCA, Bind hat be venditori di pesce, che vivevuno al tempo,

che l'Autore compole quef?

) GLI è farto il becco aro bs Deneeid &conchiulo, che i Latini diflera: JZ
' ales. s) Lailt nella sia Bh. Hee. 2. than. 64. ae “xe


78 MAEMANTILE:
We vanno tuiti: il warcie hora-figineca,
Non v2 rimedio;. E fatto il becco all' oca "

Dice Francesco Cieco da Ferrara nel suo Poema intitolato il Mambriano(Ope-
ra nota per esser l' origine, ed- antefatto dell' Orlando innamorato, Poema del
Boiardo, ed in conseguenza dell' Orlando fariofo di Lodovico Ariofto) al'Can-
to secondo 5 che ibe & SORE
» Pugia nel Regno di Cipri un Re chiamato Li¢anoro il quale' havea una fola
y» figuola nominata 'Alcenia, la quale amande egli-al pari di se stesso 5 vole fat
»» pere, se buona, o ria fortuna ella fule per havere 3 fatti però chiamare*alcuni
»» Astrologi fece fare la nativita alla. medcfima (un figliuola; © tutti concordaro-
x» HO, che ella farebbe prima flaca madre, che moglic; 'Onde il Re per evitare
»» il presagito vitupero, fece fabbricare un giardino contiguo al feo palazo rea>
»» le, e dentro al detto giardino edificd una fortifima', ed altissima Torte cons
x» Molte stanze, e con tucte le' comodita, ma senza finestra aleuna, che riuscif-
» se fuori della Torre: Dentro a questa mefié la figlia'con alcune Matfone,e
xy Damigelle, aficurandosi dell*ingrefio della medesima » non solamente col te:
»» Herne cgli proprio le chiavi della porta, ma con haver deputate accuratissime,
»» eraddoppiate guardie di soldati intorno, ed alla: porta della*torre, ed alles
2» mura del giardino; ne altri cotrava nella torre, che una fola'donna, della
» quale il Re si fidava, © le dava la chiave ogni volta, che a lei occorreva an-

»9 dare alla Torre con provvifioni di vitto, od' altro. bias * O33eik
>»  In questo tempo mori un tal Co, Gio: di Famagufta huomo ricchiffimo';- ed
29 alquanto parente dg] Re, © Jalcid érede delle fue'immente faculta Caffandro
23 unico suo figliuolo; Questo giovane fece fabbricar wn palazo fonwofissimo, ia

x» cui teneva corte bandita con tanta splendidezza, che fino al medesimo Re ven-

»» ne voglia d' andarvi, e lo mefie ad effetto / Andatovi dunque fu dal giovaned
» inuitato a cena, ed il Re accetto l'inuito, credendo fargli conofer, che non
»3 era in grado di banchettare decentemente un Re all improvviso. Ma tutto il
»» Contrario avyveane, perch il Re fu così ben servito', e di vivande,¢ di mufi-
sy che, ed' ogni altra cosa coaveniente ad un banchetto regio, che gli parve++
by che Cattinio havefle maggior possanza,' che non haveva égli; onde “eS
») ciò ad havergli inuidia, ed a pensare come' porefle 'morti carlo'; Haven
pera veduto sopra ad una maravigliofa fonte, che era nel giardino, un- motto
ay Che diceva Omnia per. pecuniam fatta funt. Si volto a Caflandro,'¢ difie+

3» Motto € troppo prefontuolo, efiendoci molte eofe, che non si posion fare col
» danaro, AJ che rispose Caflandro: Sire, lo ho posto quivi quel motto | per-
3x ché mi son sempre creduto 5 che il denaro apra la frada anche al? les
31 e fino a hora mi € rivscito, come appunto m: (Gn figurato,Horfu (replied il! RE)
yy Già che tida il cuore-di porer fare gni'cofa col'dcnard, io' ti' do 'tempo

»» anno a procurare per le trade, che vorrai y'di godere la inia'fighuola,che io
sy tengo pella torre guardata,come tu saise (@ dentro a questo me quae
eleva



x t0, fara toa moglie; quando no, la tua testa paghera la pena. EB

» il Re, perché essendo entrato in sospetto della potenea di Catlandro 4
y sotto qualche pretesto levarfelo d' avanti-. 7 A
y»» I povero Caflandro ritnatto sbalordico da tal-proposta,meditaya



<4









SECON DOC AINT AIR E, 99

y» bando dalla patria, “quando Euripide sua Balia.; fapata lacagion® de! fao di-
» sgusto gli disse, che si confolafse., perché ella haveva un sao nipots dotazo di
'97 Così grande ingegno y che afsoltitamente gli havrebbe aperta la strada all' in.
9) \grefso' nella Torrey =.) ee
x» Questo nipote della: Balia,Ewripide fabbricdunOca di legname 5 grande,
y tanto, che potelse agiatamenté asconderfelesin corpo \un':huomo y chev' en-
»» trava,"eutciva perdi forte Pali, e\per via discerti ordinghi facevafare a tal'
35 Oca tutte loperazioni;e'moti-, come se fulse fata'viva, edera del-tuttoper-
99 fettary senon che lemancava il becco. Cafsandra fece sparger voce', che era
syeandato! in lontani-paefi»; ed: intanto 'havendo: fatta portare: occultamente la
+9) detta Ocatin-un luogo-remoto:;¢entro nella medesimay ed/Buripide sia Batia io
“¥; abito: moresco la guidava | sfingendo di venirdal Cairo dove'era verimente >
y nata, ed allevata detta Euripide ) e"parlando \in-quelladingua ben? intesa a1
Ҥ) 'Gafsandro,'toccava con:una 'bacchetta |l'Oca 5 cd'eva ih concerto', che Cal-
~gy fandro iper via di certe Zampogne facefse cantar Ivoca. L' aftutasBalia, ac-
y» cennate a a l'operazioni dell' Oca, andava dicendo, che'a-volerla vedose
$y operar colegalanti,:¢ maravigliofe, bisognava spendere; ¢'perd il popolo,
»» mefsa insieme buona' fomma'di monete,la diede alla Balia plaquale fece tare
1p alrOca diver belle'operazioni. i
+4) Arrive la fama di quel? Oca al? orecchic del Re, \¢ della' Reginay ond»
«39 'fattala'venire a se, dopo haveria veduta operare', regalata Euripide,la man-
3 darono! ad: Alcenia toro: figliuola per farle-pigliar qualche spalso, € diverti-
«Jy mento hei giuochi dell' Ova; la quate condotca nella Porre;il negozio andò ia
“59 maniera,-che-per via deteattati della Batia, Cafsandro'nelto farein camera
«59)'d" Alcona afedtoinquell'Oca:; si godé Alcenin-, efi diedero.la fede di Spot.
'fs! Batro-quests,Cafiandro accomods'all' Oca il beceo; econ la Bulia'ascofto nell'
y» Oca fen' usci della torre, e presentacafi laBalia'con l'Oca dv avanti al Re, ed
§5)° alla Regina' per domiandar' licenea; i Re'difwe: Quest' Oca-hail becco, evpri-
» ma non l'havea? £ la Balia rispose: “Non se le'era meso » perché non eras
x» ancor fatto: e Vostra Machatenga aimemoria-quel che ora ha detto,
«3° Fra pochi giorni(pird il termine', dentro-abquale\Caffandro doveva haver
“Sy” goduta Alcenia,, onde il'Re se lo fece condurre-avanti., e Caffandro disse; Si-
“> te'V. Mvfacciarvenize Euripide mia Batia, I) Re lo-compiacquey'ecomparla
© Sys “Buripide'com l'Oca, fu'dalRe subito riconosciura'yed ella girdifse:°V. Me Gi
“yj 'Tivordi che @ fatto if becoo all' ea; e fatta quivi condurre Oca fece en-
a» trarvi dentro Cafsandro, e lo fece fare le solite operazioni, acciò che il Re
vj, 'Conolcefse »che quella era la steSa Oca,-che 'in quella stefsamaniera era di-
“3, morata più giormi con Alcenia nella' Torre: onde 1! Re conosciuta l aftuzia di
Sy, 'Cafsan,¢ fay recifameateril fatto, e che Alcenia era gravida, ed
“49 havea data la fede di sposaia Cafsandroyiconfermé i) matrimonio per ofseruar
la parola,contentandosi di cedere alla disposizione delcfato'; Eda questa trave-
~* Nita trasformazione diGiove in)Gigno & nato il'proverbio: E furto il becco
-alt' Oca; che significa ( come habbiamo detto ) il negozio è fatto,-0 perfezic-
snato. Questa,-o simile novella leggefi in quelle di Giovanni detto il-Peeoroas.



STAN-




2 ou



80 2M AIM ANTIAUE BD 3 2
STANZA XIV.: " STANZA sKVeb-
O ques ( disse i Duca) è veramente. Benche fuffe costuicom! una pinay).
Da pighar con le malle; Ch unfamaras << Tanto targo, y errs af
Possa col cuore ingravidar la gente; Per non balzar.un trarto alla 2
Vedi non ti fan finta:, io non la paro «; 1 pefextori wennera in pacfes \)
For sis il prowar stom ha acoftarmente,| Così pefeando Lingo ta marina;
E quando mi costaffe ancoben cara Leche benedett' aofino si-prefe m
Vo farlay per veder, se ciò riesce; E il cnor non bel bacina snargentato,
Pero si mandi al mar per queste pesce. 4 suon di pive al 'Duca fu portate,

I) Duca sentendo che il cuor d',un' Asino- marino eraatto\asingravidarila:mo-
glie, si ride-del mago; ma tuttavia era cos ernie deere shaver figliuoli,
che volle p ze do|, che i p gediedi si-
nalmente lo prefero, e portarono il cuore alsDuca gi 139) nupvaile ho sara

E DA pighar con le male. Buna grofia minchioneria 5 ésna0,speapotito. gran-
dissimo. elle intendiamo quello fleumentordi ferro eg pypnitor anamrsan a
boni ardenti:, ec.

VEDI, Questo termine ha del giuratorioy quali dica: oe fede mia: onescin ne
to credo, Credi a me che tu fai male, ec, Vedi sotto: CyB, Nano 63.

NON Ia paro. Non la credo. Tratto dalla Rissa 'so-Maila giuoco dicdadiael
quale side tino tien la polta dice'; Parodi y e nomilastenendo dice Won la pare.
LARGO come-una pina. Si dice largo com' una. Ss verde,la quale strettiflima,

¢ ben ferrata; Comparazione ironica, perché ) fargo vuol dir liberale, ed
huomo frerta "vuol dire avaro »etenace; Si che fendo agi pina, verde, strettidiima,
comparandosi un huomo a questa; sintende trettissimo;.cioè tenacissimo, ava-
rissimo, che i Latini dissero Laro facrificar; che suona;,Glié divoto della folaga,
la quale perché @ di natura vorace y servivaa iJatini per siprimeresieileome
avido del denaro, ¢-lo dicevano Larus hians,,

JGNORANTE. Voo che non fa.) Vedi sopra C. 1. han. 73., Ma wale ancora
per ingrato y Zotico y villano, €'poco amorevole ed in questo luogaé oo vinta fen-
fo, nekqualeé sempre » oper lo: pill, preso-nel contado..)

PER non balvare. Cio' per non andare... Si costumaidire belesre pee. andare '
o cadere injcole di disguito, come balsar infermo in un letta,, baizare in una prigion
He y€ee Non si direbbe balzare ainn banchetros e simili; -Per non balzarein wna pri
kiove j quanti noi fiamo, fara necefjario che altri di noi balzino in ae yedaieri si
Latuino in Chiefa, Difie ! Autore, che seride la vitadi quei tre si ladei Fio~
rentini.

BERLIN.A, Euna specie di Moment 40 gaftigo pcheifi daca i ladroncelii
re loro eS 1h a ae alate
an iu pape i frequentati ye quivifila ae ' in
folome tals = Quel firamento si ebiama anvora Gogna.. itMeah fo110.C. 3.







sear eae tisuuist 1

VENNERO inpacfe. Cioè coat earn 5 fidafeiaron, trovare..«) ean ri-
trovamento di cose ascofe 3; Ed © lo stesso.che venire in foena deco nt.
1, stan, 2,

QUESTO benedetto Afine, L? epiteto benedetto in tali occaloal a 'dir' nag






Fae

SECONDO CANTARE, Sa

to bramatu. To cerco del tale, del quale ha grandiffio bisogno }¢ questo bene-
detto huomo non si trova ' a & chimneys t

BACINO, o bacile» EB' un piatto d' nto 5/0' d'altro metallo'grande'pit:
della solita mifura de i piatti da cavola.,¢ serve propriamente per ricever l' acqua,
che fda alle manv alle tavole de'grandi, se ben s'adopta anche in molte altre occa-
fioni, e per altri effetti. 4 i mE

PIVA, Dicemmo, che cosa-sia sopra C. 1, stan, 34. alla voce cornamusa, T
contadini sogliono per il maggio andare attorno cantando, e suonando la Cor-
namusa, ad effetto di ragunar denari per far'con essi regalo.a qualche !uogo pio,
¢ ricevono l'elemosine, che vengono lor fatte in un bacino, ed in un' altro'por-
tano quel tal regalo, che voglion fare; o vero l'appendono ad.un ramo d'alloro,
aaltro albero:, ¢.dicono.questa:lor gita., andare acantar maggio, 'Tal costume»
tocca il nostro. Autore:con questo modo? di. portdre if cuore dell' Asino mavino al
Duca.

STANZA XVI. STANZA XVIL
Ed.egli prefoil prelibaro Cuore', Allor vedefti partorire il letto

Lodiede a! Cuatvoyalqual metre boicofse, Vn tenero: ye verzofo lertuccin';

Si fece una trippaccia la maggiore Di qual) armadio fece uzo Ripetto,
OCT sdilde' wachmai veduta fofse 3 La seggiola di ldun seggioline 3

Le robe, e mafferizic a quell? odore Latavola figiio un bet buffetro.,
Anely elle divencaron tutte grofe:y xi. >. 9 Hacalfawn vago, @ piccol cafretrino,
« Ein pocasompo.4 un' otta tutte quate. Bildefirenncanteretto mando fuore,)
» Becer A! sccorda,il pargeletto.infante',. C ana bocchina havea tutta fapore.

STANZA XVILL

4, Cuoco anchegli poi non fu minchione,: Ch! in far vivande faporite, e buone 5

¢ bxce witofi n'un fiance, Fu fubso fqnifite ye molto franco,
i vedde prima uscirne uno fidione; Ein quel ch' il padre stettefopr' aparto.,
; oe Guatterinoingrembiulbianco, -  CavindinCorte, s lhi,e al terzoye al quarto
ol dette-il Cuore, al Cuoco, il. quale nel cucinarlo ingravidd, si come an-
cora tutti gli arnesi, e mafferizi¢, che ne sentirono.!,odore »,¢d ana medesima-
hora Partarirondaisios sig 4 oom of 3 A

 Quivorreiry, leetore si ricordaffe che il Poeta, nel comporre ued? Ope-t
ra ha, Pe pA pce cei queldeinonslieseied alte iat n't:

contate ai Fanciulli (come habbiamo decto ) e che pera sta dentroa' termini di»

quelle favole y le qualt com: per lo più invenrate 5 ¢composte da quelle meded-

me doanice  superare la capacita di queitey ne: di-quelli, e si
contentaffe di non, inasione nel seosin dan tuna, cosa tanto favolo.
fa, e faori del, natur. e ih far parcorire le:mallerizi¢; ed! oftervare j che

Oe. » doreo, nel suo Cuntorde:li- Guati

ancora, Gio, E pur fashuomo

ha pais queffa, ed altre novelle fimuli; a folo oggetto di cratteneee.li. picci-
illi 2 COME €Bhi dict.) 9! son on asl sane) + Goons, oveMe KAO
 LRELIBATO » Vuol dire una cosa gutofa 5.0 singolare,, ma significa ancora
eo eperermenta narrata, o deta avanti, come e nel presente luogo, che Signi



fica il suddetto, o accennato cuore; ed habbiamo anche il verbo preibare Dan.
 Purg. Cant, 10, ' a
“WEL L toy
-




82 MALMANTILE

Hor ti rimanclettor soprail tua banco
Dietro pensando a cio, che si preliba.

4 di de nati, Non nacque mai veruno, che vedeffe un ventre maggior di quel-
lo., che haveva il cuoco «E un-termine:, che aroplifica la voce mai; V.giNeflu-
no di quelli » che sono stati al mondo, mai vedde, cc. Py? bominum memoriam.

A un' ota, A uno stesso tempo; a una medesima hora.. Vsandosi da noispefio
la voce ora in vece d' hora:: adefta in vece a' allora, Che orta éegli? ia vece di che
hora e egli? ne

FECER a! accordo il pargolettoinfante, S'accordarono'a partorire a un' hora.
medesima.

LETTVCECINO., Intende piccolo lettuccio, Ma lettucciointendiamo una gran
cassa, la quale per di dietro ha wna spallicra,e dalle teftate i bracciuoli, sopr' alla
quale e solito tenersi uno Mraptnto, e serve per riposo, e per dormirvi sopras
dopo desinare.

ARMADIO ec, Atnefe di legno per riporvi ogni forte di roba,il quale per lo
più si tiene affido, o aveosto al muro, e si apre come le porte, ed ha dentro 'di-
versi palchetti,, o catlette; e per fiperro qui intende piccolo armadio.

SVFFETTO, Intende piccola tavola.

DEST RO. Quello che diciamo anco luogo Comune, ed è quello, dove si va
a scaricare il ventre..

CANTERETTO.. Piccolo Cantero, e questo &un vaso di terra, o di rames
o @ aitra materia, il quale si mette dentro: alle predele 'per recipiente all' uso
suddetto, chiamato così per esser per lo pil) di figura: simile a quel bicchicre che
i Latini chiamavano Cantharas.

VINA bocchina havea tutta fapore. Il Poeta scherza, fapendosi bene, che simil
forte d' —- suol' esser sempre fetida, e però dice che eraeurto fapore, cioè fape-
va di qualcofa.

AUINCHIONE, Vuol dir semplice, corrivo: Ma qui vuol dire uno, che non,
fa meno di quello, che fanno gli altri v. g. Se tx pigli della tal cosa, non weglio esser
Minchione, ne vag tia pigliar' anch' io, iva ae

SC HIDIONE, o fridione, BE -questo ultimo & pil comune; 'Vuol dire quello
firumento da cucina, nel quale s infiiza la' Carne, o Vecelli, per care arrofto,

GV ATT ERINO, Ditinutivo di'Guatcero, chet colui, che serve d' aiuto al
cuoce. Qui intende piccolo cuoco.

GREMBIV LE, Bun panno, col quale si cinge la persona sotto lo omaco
per difendere il vestiro da' g)i untumi; decto così 94a 'regicoreminm; ed in altri
luoghi d' Italia Senale quia fimum tegic;-¢ moltt Zomule da Bimie,

MOLTO franco, La voce franco, che vudl dir sees eer
— un'huomo ardito, coraggiofo, pratico 5.0 to 5) ne
nel sente luogo. au30 & stiem Oug ROL odie: Ne ola
SOPRA parte - Quel tempo, che le donne stanno nel letto:dopo'haver parto-

i cagionati loro-dal'parto pei ef ca
: uae



ti. AYIb50




rito, per riaversi da gli feoncerti
parto. i



Ou sty
STAN-

.- lt i





|


SECONDO CANTARE.

STANZA XIX. STANZA XX.

La Ducheffa ch' il cuore havea inghiorrito,  Crebbera insieme, ed all' adolefeenza
Cotto ch' ci fu com ogni circoftanza, Peruenuti mangiaro il pane affatto;

. Anch' ella con gran guffo del marito Nel far fanta,nel far la riverenza,
Stampo due Bamboccioni d'importanta; Hebbero il corpa 4 meraviglia adatto:
Grazie,e bellexe baveano in infinito, Tra lor yon fu mai Inte,0 differenza,
E coss grande, e tanta somiglianra, Ma di accordo voleanfi un ben matto;
T ant eran fatti uguali,ed a capelle, L! Infante Floriane uno hebbe nome,
Che non si diffinguea questo da quello. E quell' altraeAmadigi di Belpome.
La Dachessa pure partori due bellissimi figliuoli, tanto simili di fatteze, che

non si distinguevano l'uno dail' altro, Questicrebbero, e furono allevati con,

buona creanza, e fra di loro cordialmente s'amarono. Vno di efi hebbe nome
P:-Jnfante Floriano, che vuol dire Raffaello Fantoni, e l'altro Amadigi di Bel-
pome; E queito è nome a caso.

ST AMPO' due bamboccioni a importanza. Pastori due bellissimi figliuoli, e che
havevano tutte Ie condizioni, e parti desiderabili; B nota che il termine d'impor-
tanza usatissimo da noi ia simili occasioni, vale in questo caso quanto il termine
di garbo, e per esprimere una tal quale perfezione del fubietto. Li Lalli Za. Tr,
C. 1, stan. 54. dice.

E produrra, se ben non senza duale,
Due garbars bambucee a xn party fal.

ef capelo. Perl appunto. E il latino.ad wxguem. Termine usato da coloro,

che si regolano col filo nello fquadrare, come sono i muratori, ec. E vuol dire

non vicorre la grofieza d' un capello dall' uno all' altro; ma si usa in ogni con-

giuntura di paragonare, o milurare una cosa conl'altra, non folo in quantità,

coine Ho riscontrate i denari, ¢tornano a capello; ma auche nella qualita come nel

caso nostro, che s' intende: erano uguali di mole di corpo, e simili di fatteze.

MANGLAR il pane affatto. Mangiar bene, e senza far rofumi, o tozi; mas
significa huomo di buon pasto. Vedi sotto C. 8. fan. 56.

FAR sand, E10 stesso, che far la riverenza; ma éun termine, che e proprio
dei bambini,, ido comingiano a imparare.a andare, che quel lor muoversi
timidamente e detto dalle balie far /auta, o pure è, quando fanno la riverenza

-baciando altrui da. mano; ed¢ così detto eae fanita, cioè fare faluce; falu-
tare« 'Diciamo insegnare al Bue far fanta per intendere: Zufegnar le feienze, oi ter

saini ciyili aun' bueme xatico, villano, e di difficile apprenfiu

83





SJ wolenano ne ben marto, S 8! 20 fuile - Equel
termine A4attus, del quale habbiamo dexto sopra C. 1. stan. 76.
STANZA Xx STANZA XXIL
drrivati che furono ambiduai Di modo che fdegnato, come ho detto,
A conoscer bomai il pan da' faff, Ch'il Duca per La sua spilorceria
 Efaper quante paia fan sre busi, Og hor vie piie tenevalo a freccherts,
Se ben dal padre havean de gli spaffi, Vn di fri ed andar via,
Vedendosi già grandi impwcatoi Ma tdcquelo per fare il gioco netto,
Ed a soldi tenusi baffi baffi, Fuor ch'al frarelioyal qual n'unaofferia
Oftico gli pareva, e molto strano, Difse(veduto havendo 4 un fiafeo ilfide)

Bd ia particolare a Fioriano.

Volerjene ramingo andar pel mondo,






84 MALMANTILE>

Cresciuti quefii due Giovani, ed arrivati a condscer il:ben.dal nialé, vedendosi
così grandi pareva lor malagevole il hon haver denari 5 perché il padre perlaifua
spuorceria non gliene davaydi che'pid d'Amadigi sentiva disgufto Piorianosonde
si rifoiuette dsandatovia 5 e perché: ' adempimento di tal faa risoluzione non gli
folle unpedito, non ne parld.ad alcuno, fuori che al fratello Amadigi,
CONOSCER il pan da faffi;\efaper quante paia fan tre buoi, Significano lo stel
fo, cive conoscere il ben dal male. Hor, disse, Novit quid dient era tupinis Si
dice ancora in questo proposito Sapere 4 quanti di e San Biagw, E questo'denoha
origine da.un costume antico, il quale era in Pirenze, che i ragazi fattori delle
bosteghe d' arte di feta:, che son fituate nel Mercato Nuovo vicino alla' Chiefa di
5. Biagio, havevano dicenza, paflato il di della fella di elo Santo ( che fax
sebbe alli 2, di Febbraio, e se ne fa alli 3. per causa della Purificazione, il che
ha daso occasione di ulare questo dettato ) di fare alle falace, € pigliarsi ogai
forte di paflatempo in alcune hore del giorno, ed abbaadonare la bottega per in<
fico a. tucto il giorno di Carnovale; e per questa causa era quel giorno tanto defi-
derato da i ragazi, che fapevano benitfimo il di, che si tolennizzava la deta
f€sia; onde colui, che non fapeva tal giorno, era fra i ragazai ripucato un bag=
&<0, e che non havendo notizia delle cose del mondo ( giudicata da loroquelta
una delle più importanti ) non fufle persona abile, e di tanto°giudizio da saper
fare i fatt suoi. E questo proverbio s' è fatto poi comune a tutti gli huomini per
intendere un' huomo sceruellato, melenfo,e buono a poco. Il Lasca Nov, 4. dices
La Stheggia yed t-Pilneca\, che. Sapevano a due once, quanto colut pefava, ed a quanti
dit San Biagio.
SE ben dal padre havean-de gli spafi, Se bene il padre dava loro de gli avverti-
menti., e paflatempi. Nota che per scherzare il nostro Poeta, subito che ha det-
to duoi seguita dal padre, e questo fa per coccare quel costume burlesco; il'qualeé
iu Firenze (ma pero fra gente:bafla ) che quando uno nomina bae, beccos o ca-
firone,Valtro dira di tuo padre, edicendo vacca,dira di twa madre, e simili, Vedi
sotto C, 12. stan. 49. annot.al termine wmorirecon la grillanda
GRANDI impiccatai. Proibiscono le leggi Y-impiccare chi non paffa 18/anni;
¢ di quinel diciamo.grandi impiceatoi, cioè abili a esser' impiccat, per antender
quelin,y che pafiano la decca eta dir8. anni. $ ”
st SOLDIL tenmti baffi bai, Tenuti con pochi denari, Traslato dail' acque,delle
guali quando ne son poche nei laghi, pozzi, o fiumi, si dice bafe. Vedi sotto in
quetlo C, stan. 61. € parlando d' uno che habbia pochi denari G dice: iL acgue
Jon bafe si come intese colui con quel suo motto ZL' acque Jon baffle, et! ache banno
gran fete, cioè Alle gran veglie i danari son ne « 2:
SOLDO. Vale per intender danari,riccheza..E soldo moneta immaginaria
(hoggiin Firenze eftettiva di bronzo)che vale tre de nostri quattrini;Spetio usiamo
questo termine per una certa.generalita: Il tale.ha de' soldi,de' quatcrini,dell'oro,
Rer intendere € ricco» nonche habbia quantità di soldi, di quattrini,.0 dro ef-
fectivamente yma molti-ne vale il uo stato; Equi intende-Monete «-—
. OSTICO., Spiacevole, Malagevole, lnfopportabile. E il Latino J he
vale per cosa da nimico.
STLANO. Qui ha lo stesso significato a' ofico. Vedi sorto C, 3, stan. ae

. oz ' Te tro









SECONDO' CANITARE. 85)

altro vuol dire stravaginte da eatranens. E molti dicono' rate ajuno che habbia
cattiva cera ye perinfermita sia mal condorto. + } Hal
-SPILORCERTAS Sordidezza, Avarizia. lo credo che questa parola venga da
Pilorci, che i pellicciai chiamatio ao ritagli di pelle,' che non esserido-buoni as
metter' in opera ygli-riducono'in spazzatura, la quale poi veadono per governa-
rei terrenijse li dica /pilorcio quali huomo vile,ed abietto-quato sono questi pilorei.
“ TENER' uno aftecchetto, Bare flar'a segno, '0 far patire uno di quello; che»
églicha bisognio; come'non'lo taftiar mangiare 'quanto ei vorrebbe; o haver de'
danari quanti bramerebbe e Quand' uno per lascarfezza di danari vive mifera+
mente si suol dite: Atele ( difende?, si febermisce, ec) ond' io ton son lontano dat
credere, che questo termine sia corrotto, ¢.che*fildovelie dire'a focbherro da stac-
cheggiare, che è l'iftetlo che schermirG, e può significare Eifere scarfo, o haver.
bisogno di denari.
' VEDVTO il foudo a un fiafeo, Dopo haver bevuto un fiasco di'vino;'e così ha-
ver veduto il fondo'di dencro/del fiaico; ed in fuftanza qui-vuol dire; Dopo ha
ver bevuato molto'bene; ovassai. a

ANDAR ramingo pel mondo, Andarsene errante. Ramingo vien da ramo,¢
si dice Ramingo de gli vecelliidi Rapina, come esprime dl Crescenzio nel Cap, 3.
della *bonca degli Sparuicri lib. 18. con le seguenti parole: Si chiama nidiace, v
wero che di nidio uscito di ramo in ramo va seguitandola madre,e pero fichiama Ramingo,

Ed alli sparuicri & danno tre nomi, cioè Widiace, che & quello, che e cavato di
nidio., ¢dallevato., amingo quello che uscito di Nidio non fa gran volate; e>
Grifagno quello', che già patiato I" anoo ha mutato alla Campagna. Ma queste
aoateat 'noitro:, bastandoci, che a' similitudine*ditali uccelli,, dicefi
Andar ramingo coli; che hora va in un luogo', bora's* incammina ip un' altro,
senza sapere politivamente, dove egli vogha andare, '
&. STANZA XXII

Anadigi 4 distoris tutto un giorno Tn vnoiir difse se verote vain un forno:
Sr arrabbio, s aggird com'un Paleo; E dopo un grande, è lungo piagnifteo;
Ma perché quanto peu eli ava interno HHorsn-vanne(difs' egli)io men' accordo,
Egii ord piu'oftinatod uno Ebreo, Ma lasciami dite qualche ricordo.

» Amadigi sentita questa risoluzione del fratcllo, molto s'affaticd per distornelo;
ma veduto'che per la di lui oftinazione s' affaticava in vano, concorse con lui,
con questo però che gli lasciafle qualche ricordo'di se,
 P-ALEO Cosivchiamiamo una specie dt erba', che nasceintorno alle lagune.
Ma diciamo anche Palcouno strumento di legno, che serve per traftullo, e giuo-
co de' ragazzi, il quale e di figura piramidale al' ingiti; e nella teftata, che viene
+di sopra ha'ua manichecto condo:, il quale.avvoltato con uno spago, o cordiccl-
las' infila in uo' atlicella,bucata,e tirandosi quello spago fifaolta, ed i Paleo feap-
»pa dal buco dell' aificella', © va per terra girando,portato dail'ampulso di quelio
 'spago. Tale dtrumento da i-Latinié detto Tu*ho forse dala figura piramidales.
+ WVerg. 7. Aneid. Cex quondam torto volitans fub verbere turbo, T ibull, Nam

ue aSOry
“at per plana citus fola verbere turbo, Dance nel Paradifo C, 18. a
Ed al nome del alto Uaccabeo
Vidi moversi un' altro roteanda, =
E letizsa era ferza del paleo « i EG


-
86 MALMANTILE

E dice,così, perché a tale strumento si fa continovare il girare perquotendolo
con una sferza, dopo che egli ha hayuto il primo moto, ed.impulfo dal suddetto
spago. Ed il proverbio aggirarsi come ux paleo vuol dire affaticarsi assai, e conchiu-
der poco; che i Latini pure difleru Trochi in morem circumagi, perché dicon Tro-'
chus tanto il paleo, che la trottola, portandolo dal Greco Treches, che vuol dir
ruota, o altro strumento che giri. Vedi sotto C, 6, stan,22. E forse ancheJa yo-
ce latina T-«rbe significa tanto il paleo. che la trottola, perché Turbo vuol dire
ogni cosa che habbia figura Piramidale, a rovescio, cioè il largo di sopra,.¢ da
piede acuta, come appunto e il Paleo, e la Trottola; se bene non sono lo stesso
come ci teftifica una certa cantilena assai praticata fra i ragazi, che dice,

E il Criffiano non e gindeo.,
E la trottola, non e paleo,
E paleo non e trottola, ec, q

PIV? oftinato d' uno Ebreo, Oltinatitfimo, che non si trova nazione più oftinata
nella sua legge, che quella. de gli Ebrei, che pero ha meritato)il titolo 5 che le da
la fanta Chicfa di pertidi, Cino da Piltoia, O vei, che fere wer me sigindes: cioè

erfidi,
. VA in nn forno, Va dove tu vuoi. E specie d' imprecazione, che suol far' uno
vinto dall'impazienza, E si suo] dire anche in questo proposita: Vain malorayva
al diavolo, va in galea,¢ smili, Abi in malam erucemse Plaut, Epid, Ato ts se.2-

ditle; Atala iff usmodi mihi amicos furna merfos, quam fa
XIV. Ss

STANZA
Allor per fadisfarjo Flariano,
Accio che più tener non Labbiain ponte,
Con un basten fatato, c' hayea in mano
Tocco la Terra,e fece uscirne un fate »
E disse: Quindi poi ben che lontano
Vedrai sto vivo,o s'ia sono a Caroute;
Perché quef'acquagzn' or di pina inpito
In che grado so faro diratti appunto,
STANZA XXV.
Sal corso di que? acgua porra cura,
Tutto il carfo vedrai di vita mia;
AMentr' ella e chiara, criftallinase pura,

foro,
TAN ZA XXVI.
Cio dette in capo il berrettin fiferra,
Aerte man,chiude gl occhi,e frrige i deti
E da si forte una imbroccata in terray
Ch' il ferro entrovvi fino ai fornimiti.
La quel che i grills,e + bachi di forterra
Sgombrano tutti i loro i.
Pullula fuori un cefto di mertella,
E di nuovo Florian così favella
STANZA XXVIL
Fratel mio caro, questa Piauta ancora
Com' io la paffi ti dara epee ao
Cioè mentr'ell'e verde,anch' io allara

Di pur ch'io vwva. in feftaged allegria; Son vivefresco,e verde com' un' aglio;
Ed all incantro, se torbida, e feura 5 E quand' ella appaffisce, efi coloray

Ch ella mi vacome dices la Cia; Anch'io lagui/ivod ho qualche travaglio,

Ma k gusapcdadlenste Sern il corso, In somma sell' fecea, leva i moceoli.,

Di ch'io sia itoa veder baliar LOrfo. Per farmidire ilcantoinfscarpezoccoli.

Ficri wo per contentare il fratelio, toccd la terra con un bastone incantato,

; che haveva in mano, e ne fece.nascere una fonte, e disse che dalla mutazione di
quell' acque haverebbe egli conosciuto lo stato, nel egli f tovafle. Dip

mefle mano alla spada, e con ¢ffa buco la terra,.¢ scappo tuori. anor-

tella; E mostrd ad Amadigi, come egli si davea contencec in conascere ancora.
; da questa mortella, in che grado egli si trovatic. i



<= e pts act a |


SECONDO CANTARE: |

87

TENERE in ponte. Tener un sospefo, o irrefoluto. I Latini pure dissero: 2

pdetinere; e però ttimo, che questo nostro detto venga dall' uso antico de'
omani, che nell' clezione de i Magiftrati chiamavano Pontes quelle piccole ta~

yole, sopr' alle quali eran posate le paniere dei voti; di che fa menzione Cic. 1.

Rhet. Pontes diffurbar, Ciftas deijcit; e canto Ravano incerti, e sospefi coloro, che

devano; quanto le cefte de i voti stavano sopra i detti Ponti; E' pero di-
cendo: Ego /um super pontes, vaol dire il mio Voto è ancora nelle Cefte, o coper-

to, e per conseguenza io sono folpefo, ed incerto di que} che habbia a esser di

me, Eci serve poi questo detto Tener' uno in ponte per esprimere; trattener' uno

con le-speranze, o con altro secondo il fubictto. '

SONO a Caronte, Son morto. Son fra l'anime', le'quali paffano la Barca di
Caronte, che secondo la faifa credulita de'Gentili era il Navaleftro,il quale con-
duceva | anime de i morti con la Barca alla Città di Dite. Vedi sotto C. 6. stan.
19: & feqq.

COME dicea la Cia, Miva male, e peggio. Che questo voleva inferire una
tal Ciay © Seia Fruttaiola con un detto sporco da lei molto usato.

SON itoa veder ballar tOrfo. Anche questo 'detto significa son morto.

IN cape itberrettin si ferra yec, Con guefti due versi esprime uno, ches' accin-
ga a fare un' operazione'; nella quale sia necessario ular molta forza', perché-ia
efi: mostra quelle azioni, che per lo più son solite farsi in simili congiunture.

METTE mano, Quando ditiamo assolutamente meteer mano; intendiamo met-
ter mand allrarmi. Diffringere enfem.

iene a via; mae it '

'qui comm pare i proposito il norarewuna la generale portata dal
varehi 'nel fo Hercolano; civt che la lease tain hel frei qual-
ia dizione ne} nostro parlare ha la forza di privazione, come ai

Latin la particela m ha forza di negativa, come doftus, indottus, ec. Ed appref-

fo di noi eaitare y fealuare ec, Ha però questa regola anch' essa le sue eccezioni,

come sbilordite vitol dirbatordo, € non vuol dire senza balordaggine; T urbare,stur-
bare, diffeobare, che faonano'lo fieflo con l'aggiunta, che senza. Talvoltas
anitor' s* aggiunge alla 'deta yS), la particelia a, e particolarmente quando la,

eee ens vocale, come amare, difamare'; interessato, difintere/-
Salto 5 o ORES.

* CESTO', Intendiamo pianta di virgulto, o & erba, come Cefto di lattuga, di
mortella-,-ee.' Se bene de e virgulti si dice anche Pianta, come si vede nella pre-
sente ottava 27.Fratel mio caro queffa Pianta ancora, Viene dal latino Ce/pes, e noi
pure diciamo' Cespugtio. lo stimo, che pianta sia nome generico, poiché serve,
per tutti li vegetabill, dicendosi Pianta di prezemolo, pianta di grano, e pianta
di oe, €¢. € noni direbbe di tutti cefto', ne cespuglio.

'RDE come un' Aglio, Vn bel verde si paragona ad un' Aglio, perché questo

ha le sue frondi di bellitiimo color verde,€ che f) mantengono ver=.

di, e segno di sua perfezione » E però dic Ui tale e verde come un' aglio, s'ia+ 4

tende; e di fanita perfecta' cruda Deo, viridifque fenettus, Horat, Dumques

virent genna, Questa bmi si piglia da tutte le piante, la fanita delle quali

8' argumenta dail' esser ben verdi, che dimostra non havere esse patito, ne essere
12






MALMANTILE?

in grado di feccarfi'. Edialle volte s' intende uno di-mala'fanita quando fi'dices
verde come. un' aglio, mas' intende non la frescheza, che denota il verde delitaglioy «
mail colore, che efiendo verde nella faccia dell' huomo denota pocayfanitas 10
LEV.A i moceoli per farmi dire il canto in fearpe,e xoceoli,Compra la cerayper far-y
mi il funerale:: che moccoleyuol:dire ogni piccola candeladi-cera ye quit prefol
per ogni forte dicandele di cera'. B quel farmé dire il canto fearpe Zoceoli & detto.,

gioco(o usato fra 1 nostriContadini; 1] cual.detto non è forse senza fondamento
neaffatro: improprio, che posia haver origine dalla diligenza, che si pone nel fae,
che i morti quando son portatialla sepoliura habbiano y se sono huemini un parr
di scarpe nuove,e se son donne un par di pianelie,o zaccoli puovi; eRveco/e\e-unay
scarpa col fondo di legno, che serve pen difendere i piedi dall'acqua, che¢perterra.
Ss o s hovel

TANZA: XXVIII: TAIN ZiA>-X SAX. >
Poi che queste parole hebbe finite, è Eth prima giorno fece tama via y\ 9.09 0h
Dal suo caro eAmadigi si licenza y Chi suoi Lacché spedati, e conci male
A qual rimafe tutto, sbigortito, Sirimafero 5 Punoall' offeria,
Pero che gli dolea la sua partenza, c Ent altro fearmanato allo spedale;
Quand' in feha Florian di già falito 9», Ona' ¢i più non havende compagnias'.
Senza gran doble, o lester di credenza Se bene accanto havea spadase pugnale;
Andonne abenefizio di natura i 'Per non haver paurain andarfoloy
Con dug ferni cercande faa ventura,, » Cantava ch' ci pareva unrofignnelo,
exgh onssiber wid AD CAL MMe obyteD oir a \ '
Così muove canzoni ogn' hor cantando Onde ai timori al fin dato di bando.
Con una voce tremolante in quilio, on Tirava innanzi il voiontarwefilio 5,
E qualche trillettin.di quando in quado E ginntova Campi, li fermar si volle
Alle fielie n' andava ye in vifibilio: A bere, e far la Rolfe per bi malle, -»

Floriano si parte dal fratello Amadigi, il,quale ne rimafe aiflitto. Lalcio per,
la flrada i Lacché stracchi,ed egli folo si condufle a Campi, dove si fermé a bere.
SSIGOTTITO, Afflino; perduro d'animo. I Latini dissero «daimo deietius,
Quand' uno sia allegramente diciamo: Il tale fa in.gote, 5.0 fha in barba di micio.,
Vedi in questo C, stan. 48. Si che uno che non stia allegramente si dice vom /ta im,
gore, non sia in barba di micio;.E però non farebbe gran fatto, che questa voces
shigettito venifie dallo Spagnuolo bjeorses 5 che vuol dir bafette y,¢ che per-ia lette-,
ra, $, che aggiunta al principio d' uha parola ha forza di privazione (come,
habbiamo detto poco sopra ) significatie senza bigorres, che vuol dir senza balette,
cioè non in barba,, non allegrameate: 0.forse sbigottico,quasi sbattuto...
(od BENEFIZ10 di natura. A caso; dove la Fortuna lo.guidava... »
LACCHE', Servitori, che corrono.a pié:; e per lo) pil fonowwagazzi
yanetti. Vedi sotto.C. 11. stan. 9. 51k HObHat
SPEDATT., In questo caso non vuol dir Senza,

" eftanchi dal viaggio. Be it by AV Svan: sets ah
Scakddanart tna pecie d'infermitd, che viene. c:
'caldau per violente fatica, o viaggio









che dope essersi foverchiamente Hi orate
freddano o col bere.0,conJo stare al vento, o in luogii frcscht;¢ si dice +) és
gliar una fearmana, o fearmanare.«.E forse specie di quel maics che i medici chia~
mano Pleusitide,edé comunemente chiamato wal di petto.Qui sotcndh, ABBEY






SECONDO CANTARE: 89

dal viaggio, in maniera che l'anelito se li rendea difficile, e però 'non poteva-
no camminar pil.

CANT AVA che pareva un Rofignuolo. 1) Rofignuolo, Vecelletto noto, da i
Latini detto phifomela, ha il più bello, e gagliardo cantare di qualfivoglia Vccel-
letto, e per questo quand' uno canta bene, lo paragoniamo al Rufignuolo.

VOCE tremolante. Voce, che tremava per cagione della paura; Si come'i tril-
Ki eran fatti per timore, e si potevano dire più tosto tremoli, o interrompimen-
ti di canto cagionati dalla paura', che veramente 77id che sono un riperquoti-
mento di voce musicale nel medesimo tuono. Horazio disse: Cantu tremulo.

LN quilio,.. Secondo che mi disse il Signor Nigetti, fra i mufici del nostro seco-
lo il Maeftro; la voce quilio significa un cantare in voce non sua, come se uno
havefle voce di bafio, e cantafle di soprano; Si che s' intende, che Floriano can-
tava per la paura in voce falfa, enon sua naturale, che i Latini secondo Cic. lib.
3.de Orat. la dicevano Vocula fal/a.E Titinio appresso Sesto ditle Succrotilia vocula,

ANDAR alle feelle col canto, Cantar in tuono alto. Se ben qui par che voglia
dire, fen'.andaya in gloria, cioè cantava con gran soddisfazione, e gusto; poi
che foggiugne én vifibiie che appresso di molti de' nostri vuol dire Andarsene in,
eftafi, e perderei sentimenti per il gran guflo, Matteo Franzefi nel Cap, del suo
viaggio da Roma a Spoleti dice.

Vedea pafsar con toruo supercilio
Qualche Sarrapo tronfio, ed appoggiato
il tappeto, n° andava in vifibilio.

Vergilio Egl..5. ditle: Voces ad Sydera iattare,

Ed ottavo Ma. Effundere voces ad athera,

TLR AVA innanzi il volontario efilio, Continovava il viaggio, che egli medefi-
mo s' era eletto,cfiliandosi dalla propria casa.

BAR la zolfa: Detto scherzolo, che signifi a Cantare, far musica, ed e com-
posto di tre note musicali, la, fol, fa. Ll Signor Salvador Rofa in una sua bella
Satira parlando della musica dice,

Quanta gira la terra a tondo a tondo,
Lago alcuno non v' e che di schiamarzi
.) «Edi xolfe non sia pieno, e secondo,

PERS mole. Ib molle e chiave musicale, o segnatura di femituono; Mas
qui dicendo far /a xolfa per b molle, si serve della voce mulle per intendere: am-
mollare la bocca, cioè bere, E così scherzando sopra alla musica, ed havendo
detto, che Floriano cantava; foggiugne,, che voicva seguitare a cantare anche
nell' ofteria, ma per b molle, ed intende Vuol bere.;

STANZA XXXL STANZA XXKIL
A Campi, hora spiantato alla radice Com' io diffi, Florian nella Cittade
Dominava in ques i Storditano, Entré per rinfrescarsi,e toccar bomb,
Se ben Turpine ferive', ed altri dice, — Mail gra fraftuono,cb in quelle cotrade
Ch' ei regnaffe in we luogo piss lontanos —— —- D'arini,di bestie,e d'haomini rumbomba;
Hebbe una figlia detta Doralice, Al sentir fu pei canti delie rade
C'bavea un'occhioc'uccides il Criftiano, Tute a cavalio risuonar la tromba 5

Ma quel che pin tirava la brigasa «Ed il voler saperne la cagione,
El cffer fola ye ricca sfondolata, M = Lo fecero mutar a' opinione.




p

as

he

i) MALMANTILE

1) Poeta finge Città Regia il Castello di Campi, luogo vicino'a Firenze y che
hoggi ha poca forma di Castello, per esser distrutto, e dice che già vi regnava
Stordilano, che hebbe una bellidima Figliuola nominata Doralice, la quale per
etier fola, e ricchissima, era da moiti bramata in moglie. E perché questa non
sia creduta la stessa, che quella che l' Ariofto fa Figliuola di Stordilano Re di
Granata dice: Se ben Turpino ferive, ed altri ( cioè  Ariofto ) dive, ch' ei regna/-
Se in un luogo più lontano, coe in Granata. \

Floriano dunqgue, il quale era entrato in Campi solamente per pigliare un po-
co di riposo, e rinfrescarsi, e andarfene, sentendo tanui strepiti d' armi, € ro-
mori di tamburi, si risolue di trattenersi alquanto per intenderne la.cagione.

HiVEA un octbio c' ucvidea il Criftiano, Havea così begli occhi, che facevano
innamorare ognuno. Questo detto vien forte dalla comune opinione di quel serpente da i latini detto Regulus, e da i Greci, e da noi chiamato Bafilisco, 11 qua-
le col folo fguardo avvelena, ed ammazza coloro, che egli mira. E moiti Poeti
nostrali per ledare l'occhio di bella donna hanno detto: Occhio di Bafilisco, in-
tendendo, che han forza di metcer nel cuore il veleno d'amore. Apul. morficans
tibus oculis,

TIRAVA la brizgata, Lufingava, incitava, allettava il popolo a desiderarla -

RICCEA sfondolata, Ricca senza fondo: Ricchiflima. Diciamo Ricco in son-
do, senza fondo, sfondato, o sfondolato, per denotare una ricchezza. senza nume-
ro,omifura.

RINFRESC ARS, Cioè reficiarsi col riposo, e col cibo. I Latini pure dice-
vano tal volta-rinfre/car/i per ristorarsi,trovandosi refrigeratus in vece di refociliarus,

TOCC AR bomba, Arrivare in un luogo e dimorarvi poco. Questo detto &
tolto da un giuoco fanciullesco detto birri e ladri, il quale fanno in questa manic-
ra. S'uniscono molti Fanciulli, etirate le forti a chi di loro debba esser birro,
chi ladro, quelli che ono eletti birri si mettono in mezzo della stanza, o piazza
dove s' ha da fare il giuoco, e ciascuno de i ladri piglia il suo posto, il quale &
già stato consegnato per immune; e questo luogo da essi è chiamato bomba, che i
latini dicevano mera in questo medesimo giuoco usato ancora da i loro ragazzi, ¢
da quelli de i Greci, se beac in qualcofa differentemente. Questi ladri vanno
scorrendo da un luogo all' altro, e i birri procurano di pigliargli, ed i ladri,
quando si veggono stracchi, corrono a trovare un di quei luoghi immuni detto
bomba, dove ttando, sono franchi, ed i birri non poslono pigliargli, e si guada-

gna, o si perde il premio stabilito,secondo che son convenuti d' esser prefi,onon

prefi in tante gite; ed il ladro preso ( continovandosi il givoco ) diventa birro,
ed Ml birro, che ha preso diventa ladro. E perché nel toccar bomba si trattengo-
no 3 pero diciamo toccar bomba per ¢sprimere arrivare in un luogo, e par-
tirfene presto. E questa voce bomba vien dal Greco bombeo, che vuol dire Strepita-
ore, o far suono, ( donde rimbombare ) è da quel romore, che fanno i ragazzi con
la voce, e con le mani per far conolcere che toccano i} luogo immune, questo
luogo è¢ chiamato bomba. Diciamo tornare a bomba che significa ternare al primo
discorso-, Vedi sotto C. 8, stan. 15. ws i

FRASTVONO. Fracasso, Strepito,romore confufo, quasi dica fuor di tuono.

CANTO. Cioè l' angolo che fanno le case.a capo a una strada che eae

Q — van"al-






SECONDO CANTARE: gt

ian' altra; detto così secondo alcuni, dal Greco Canthos, che vuol dire Angolo
dell occhio, o dal canto, che nello sboccar delle strade in su le cantonate folcva

farsi dagli antichi, come si cava da V

. Egl. 3.

Won tu in trevijs indotte folebas
Stridenti miferum stipula disperdere carmen ?

Ma è detto dai Greco camptin, che vuol dire Piegare.

TVTT 14 cavalo, Così chiamano i Soldati quella suonata di tromba, che fa in-
tendere a i medesimi il montar'.a cavallo, la quale par che esprima; Tati a ca-
valle. Costume tolto da i Latini, che per significare il suono della tromba dice-
vano secondo Servio, ed Ennio Taratantara.

At tuba terribili fonitu taratamara dixit.

STANZA XXXIIL

Era gta feavaicato ad una Oftefa,
Per far, si com' ei fece, un conticino,
Ne altro bebe che pane,e capra leffa,
Che fitra anche gii fu per mannerino
Bevve al pore una nuova manomelja,
Perch' il vinaiohavea finito il vino;
Fece conto, e pag ben volentieri
Poi chiefe il fin ditanti Strombettieri,

s

TANZA

Ma c' occorre ch' in cio più mi distenda,
Mentre la cosa è tanto dinulgata ?
'Pero lasciami andar,ch' ioho faccenda
Havendo sopra un' altra tavolata

STANZA XXXIV.

Ella rispose: E come; E non lo fai?

Se per Campi non è altro discorso,

Che havendoil Re una figliaye' hoggi mai

eAbbraccerebbe un' hué prima Cun or fo;

E percht reda ell' e bell', e d' assai,

Di pretendenti bavendoungrancocorso y

Bandire ha fatto, acid nefun si lagni,

Chin gioftra chi la vuol fela guadagni,

XXXV.

Dicé Florian che ai suoi negozzi atteda,

Scufandosi d' haverla feieperara

E rimeffa la brigha al suo giannetto,

Come un pardo faltovvi /u di netto,

Floriano essendo scavaleato a un' ofteria, dopo che hebbe mangiato, e pa-
gato intese dalla padrona dell' ofteria, che quei romori di trombe si facevano
perché il Re voleva maritare la Figliuola a quel Cavaliere, che meglio si portafle
1p gioftra; onde Floriano monté subito a cavallo per andare a veder questa fefta,

F ARE un conticino, Così usiamo dire per farsi intendere copertamente Andar a
mangiare all' ofteria.

FITTO gli fu, Gli fa fatto credere. Gli fu dato ad intendere che ¢' fufles $
Mannerino., Il verbo ficeare usato in questi termini serve per esprimere, che:
quellatalcofa fudata per maggior prezzo di quel che ella valeva,o per di miglior eS

+ ualita, che ella non era. Vien da ficcar carote, che vedremo sotto questo Cant,
s jan. 70, € Cant. 6, stan. 68. Lat. imponere alicui.
i MAN. a ie d' agnelli castrati, che nel!a nostra Toscana è ottima
nel Territorio., econtado di Piftoia, ed ¢-carne (quifica al contrario della ca-
pra, chee la pepgione; che si mangi, ed.in particolare cotta a leffo.

MANOMESS A, Quando all' Ofte arriva portatogli dalla montagna il vino
primo cavato dalla,botte si dice: 2 offe ha bausto la manomefa, Ed i Fiorentini, ':

othe son di buon gusto,o più tosto ghiotti nel bere, lo pighano più volentieri,

quando è vino di » non tanto per la curiosita di gustare quel nuovo

vino, quanto perché non piacendo loro le fondate,hanno caro di bere del primo,

che esce della botte, onde pare che il = voglia intendere, che Speene fes
14 2 ene

“a4




or MALMANTILE

bene bevve acqua hebbe nondimeno gusto,, perché era nuova manomeffa,, maia
essecto gli da la burla dicendosi che bevve una manomefa nuova cioè infolita, nons
efiendo solito, ne costume, che si manometta il pozzo, se non per le:beltie.
VIN AIO, Cioè colui che nell' ofterie da il viao. Per maggior intelligenza di
questo è necessario sapere', che nell'Ofteric di Firenze stanno due maeltri, e ten-
gono garzoni differenziati; Vno di questi macftri e il padrone principale ed in
lui dice l'Ofteria, e questo si chiama il Vinaio; altro e macltro anch' egli, ma
solamente della Cucina, della quale paga un tanto il mesc di pigione al Vinaio,
dal quale può etier mandato via. Ho voluto dir questo, perché s0 che a i Fore-
flieri è di non poca confufione questa distinzione, perché si fanno'far il conto da
uno, e peafanao d' haver finito; gli sopraggiugne poi il secondo Olte, che fa lo~
ro il conto della Cucina, e cresce la somma del primo conto fatto dal Vinaio.
FECE conto. Domandd quanto dovea pagare.,. Trattandosi d' ofterie Far con
to stintende Haver finito di mangiare.
ST ROMBETT /ER/, Intende il romore, che fa il suono delle trombe.
ABBRACCEREL BE un huom prima c' un' orso, Così diciamo d' una Fanciulla,
che sia in eta da.maritarsi, e che sia bella, grande » e ben formata, intendendo
che sia in eta da bramar ? huomo., e da distinguerlo.da un' orso 5 o da 'non fug-
girlo, come farebbe all' orso. Virg, am maruraviro, plenis © nubilisvannis,
*" D' -dSS-AL, Valente,contrario di Dappoco: pare che suonio stesso chein la-
tino preffans.. f
REDA, Vedi sopra in questo Canto stan, 12. Quié preso nel suo Poe si-
gnificato d' herede,o fuccetiore nelle faculta; e vuol dire che essendo ella Figliuo-
la unica del Re, dovea hereditare tutto quello:che ¢gli posledeva., vise
TeMVOLAT £, Così chiamano li nostri Ofti tutti coloro, che 'vanno a 'man-
giare alle tavole delle loro ofterie, canto se fuffe un folo per tavola, quanto
se fussero più, pur che feggano a mangiare a tavola..:.
SCIOPERAT A. Levata dal lavoro, o dall'opera.. Vedi sopra C, 1. st. 29,
GIANNETTO., 'lntende cavallo, Sendoi giannetti specie di cavalli», che ven-
gono di Spagna del paefe d' Afturia., e perciO dai Lavin detti4spurcones ) -
'P-ARDO, I Gatto pardo¢ animal noto, come e anche nota la di tui feroces
agilita, e destrezza; e pero-appresso di noié in uso questa 'cemparazione quando
vogliamo intender l'agilita di vita d*alcuno3 Vedi sopra C. 1, stan..11, Le /oale
corre lefto come un gatto.. vita ol

STANZA XXXVL STANZA XXXVIL
Tocca di/proni,e vanne,e giunge in pi Floriano in comemplar facciass) bells
'Dovietlibiatatese che riafer lagiofira, > Dave quel evade, bale de

Che per vedere il ws AY Rib y
B apeen | Cova fattanie asia.”
Sedevail Re presentela Ragaca,
Che quanto adorna,e bella si dimostra,
Tanto è confufabavido ahaver coforte,
Won afuo mo ma qual verra ta forte.,


Le ee eee

SECONDO CANTARE.

i è STANZA XXXVIIL,
Po far'\(dicea) che bella creatura | Capperipud ben dir d haver ventura
nell' Offeffa da vero havea ragione, Quelloia cui tocca cos) buon boccone;
Perch' ella e bella fuor d'ognimifura Ma's ellas' ha da vincer con la lancia,
Per me non faprei darle eccerione. Hoggie quadoci arrischio ach'o la picia
Floriano giunto in Jenne veduta Doralice così bella se ne inuaghilce, e risol-
ue pero di tentare la fortuna, e cimentare la sua persona per avventurare i) con-
seguirla per moglie.
LL Popol vis ammazza, V'é tanto popolo per veder quella gioftra, che s' ani-
mazzano l'un l'altro per la strettezza. Hiperbole usatitlim' in queilo proposito
per esprimere la gran calca, o quantità di popolo. i
F.ANNO la mostra,, Quando i Cavalieri, o soldati, o altre genti 5 che devono
fare qualche operazione guerricra (ancor che finta ) avantidi.cominciare a ops-
rare compariscono in ordinanza questo si dice far /a.mofrra,
LA Ragazza. Intende Doralice figliuola del Re.
2A. SVO mo. Secondo il suo gusto. Quel me vuol dir modo, usandosi da noi,
come da ii Latini, e da iGreci la figura Apocope, che leva l'nltime sillabe alle»
role, e da noi alle seguentiparticolarmente; Afodo., meglio fede y vagliv 5 vedi,
Saendaate » piede,ec. Che diciamo + mo, me, se, vo'. ve,fra, fan, più. Howo-
luto-notar queste, perché spesso nel nostro parlare ci vagliamo di questa figura,'¢
fitrovera ancora spesso usata nella presente Opera, come habbiamo.accennato
ancora sopra C. 1. stan. 10,
TIRA frecciate.come la rovella, Tira dardi,e frecce in quantità. Di questo
termine come la rovella, come la rabbia,.come il canchero, ci serviamo per.csprime-
re quantità grande 5.0 vero operazione wiolenta infuperlativo grado; come per
esempio Mtale.corre fortissimo, il tale perquote gagliar ate diremmo.// tale.corre
la rovella, rabbia o canchero, o perquore come., ec, E si.deduce la. comparazio-
Rian violenza., con la:quale opera il male della rabbia, o del.canchero.. Las
evoce fovela:, O.rovello s credo inventata dalle donnicciuole per.non profferire la
arola 'rabbia,.come'fi dicecappira in vece.dicanchero, EB se bene hanno del fur-
ten » son tuttavia, molto:wlate:, ¢.]' usd il Malateftiin,alcune. sue ottave,
: Da poi ch'.io,a servito per rimbelloy
i) \Befonovandato.trenta mefi aioni
| Gridando per larabbia., e pel rovello x 4
8 Come fail Gatto quand' hai pediononi ec, 'eis
Ed habbiamo il verbo. Mare, e}'.addietti lato. 'In formma'in.que~
» flo luogo dicendo Tira frecciate come, /a,rovedaiintende,.che |Doralice.con le sues
 gcan bellezze faceva ianamorare ognuno, che la vedeva,
LE Grazie, | Poeti-fingono, che le grazie-sieno tre figlie di Giove nominates
Ve Aglaia, Eufrofine,, e Thalia. Ag/aos\in Greco val per splendido, Eufrofine, ila-

93

ne







rita, allegrezza, e Thalia, verdeggiante. Si che dicendo 7 feorge in quel.volto les wn
 grave vien'.a dire: Si sce:in lei splendidezza., alleg: » e freschezza, cioè
gioventi fana. <

RACGOLTO in uno. Vnito in un folo'luogo, Termine latino, usato.alle.vol-
ste anche da noi in questo proposito, 3




94 MALMANTILE

LE trombe. Nella più stimata carta de' Ganellini, o Minchiate è effigiatala3
Fama con due trombe alla bocca, e da questa tal carta si chiama le Trombe; &
per esser questa la superiore a tutte l'altre carte quando si dice:\ La tal cofart les
rrombe s' intende, che questa tal cosa sia la meglio, che si trovi nel suo genere
Ed è detto assai usato per esprimere l'eccellenza d' una cosa, ed ha la forza del
superlativo. ?

NON plus ultra, E, noto il motto delle colonne d' Hercole, che vuol dire;
Won si vadia più avant:; E noi ce ne serviamo nelle congiuncure simili alla pre-
sente, che s' intende; non si può andar pil la, cioè non si può avanzare,o fupe-
rare tal bellezza, o vero non si può far più bella. Esprime anche questo termine
un superlativo,

PVO! fare, E' termine d' ammirazione,o flupore quasi diciamo: Può mai fare
il Cielo, o la natura una cosa tanto bella, e perfetta come questa ? L:

CAPPER/? Ancor questo e termine d' ammirazione; e si dice ancora cappita,
canchita, canchigna forse per.non dir canchero: Voci inventate dalle donne;come
habbiamo accennato poco sopra alla voce reve/la, Consuona col latino Pape, che
noi diciamo 4 ! e col latino babe, che noi diciamo, o habbo | E la parola capperi,
che tanto in Greco, che in Latino vuol dire il capper frutto noto, serviva anche
a' medesimi per termine d' ammirazione, o giuratorio, come si vede in Laerzio
nella vita di Zenone. Sed, @ per capparim iurabat, ficut Socrates per canem,ec. LO
stesso riferisce Alex, ab Alex. dier. gen. lib, 5. cap. 10. I Lalli nella faa En, wrau.
C, 1, stan. 85. 2

Capper disse Enea, come si tofto
Fatt' ha si gran Città questa Signora \:
ef CHI toca così buon boccone. Chi haura così buona forte. Chi haura per mo-
glie così bella, e ricca Giovane.
Cl arrischio anch' io la pancia, Ci avventuro anch' io la vita.
TANZA XXXUX

O per tute' hoggi beccomi sue moglie Ciò detto falta incampo,e un' aftatoglie,
Nobile, ricca,¢ bella; o veramente Intruppandosi ld dov' ei già sente,
Vi lafeto Loffa; s* ella cogkie, coglie C' appunto il ReYolleciea, e commette,
Se x0 a patires O Cefare, o niente. Che pe' è prims si tirin le bruschette.

 Risoluto Fioriano di provarsi in questa gioftra si fa innanzi', e piglia una lan.
cia. Qui bisogna fupporre, che Fioriano, e gli altri Cavalieri futicro armati di
dosso, come € necetiario, che sieno i Cavalieri, che gioftrano a corpo a corpo.

BECCOMI fu moglic, Questo-verbo beccare ha signiticato di rubare, guada-
“— © acquiftare, Gio, della Casa nel Capitolo in lode del martello d' amore
ice::: dase,

So che fapete del ladro fottile, 4 oa
C' 4 Giove se la harba già ai stoppay ie
: Quando glibecco fut esca, eilfucile. ae Si

E però afato per lo pi scherzando in occasione di maritaggi, come appunto

nel presente luogo, EB si dice M tale piclio moglie, e becca fu una buona dore., Elo

aan nace dal verbo beccare', che € novo quel che signitichi trattandoli d' am-
mogliati. a: t
+ SE



SMES






SECONDO CANTARE. 95

©) S* ELLA coglie, coglie.S' io m' appongo,fara bene. S' io vincerd l'haurd indo.
winata, e sard felice, Se no 4 patire, Se non m' ayponee » fara disgrazia, haurd
pazienza. In somma con que i due detti yuo! mostrare, che Floriano ha l'animo
accomodato a tutto quel che sia per succedere, o male, o bene che sia.

O Cefare,oniente, Aut Cafar, aut Nihil, O morire y o esser qualcofa di gar-
ho. Questa sentenza latina si profferisce da noi corrottamente, O Ceferi,o Nic-
colo, ed esprime Aut Rex, ant afinus de i Greci, cioè uno de due eftremi.

STL tirin le buscherte. Si tirino le forti. Credo che si chiamino bruscherte,¢ non
buschette, o forse in ambedue i modi; che e un giuoco da Fanciulli, e si fa con,
pigliare tante fila di paglia, o altra materia simile, quanti sono coloto, che han-
no. a concorrere al premio proposto, e quel filo, che tira il premio, si fa o pilt
lungo, o più corto de gli altri; detti fili s' accomodano fra due afi, o in mano
in modo, che non Gi veda se non una delle due teftate di essi, per le quali teflace

ciascuno de' Ragazzi cava fuori il suo, e quello che tira il più lungo, o il pil
corto, sc do che è defti » conseguisce il premio proposto; Questo giuoco
serve ancora ai Ragazzi per fare le divisioni ne i loro giuochi Panciulleschi,come
ofarebbe ne i Birri,e Ladri detto sopra in questo C. stan. 32. aila voce Bomba, che
allora pigliano tant fili, quanti sono i Ragazzi, la meta lunghi, e la meta cor-
ti, e:cavandoli da loro a uno per volta detti fili; quelli, che hanno i lunghi,van-
no da una banda, e quelli de' corti dall' altra; e così serve a loro, come serve nel
presente lnogo, per un modo di tirar le forti. E da questi bruscoli, o fili di pa-
ia mi do a credere, che si dica brascherte; e che bu/cherte sia quel giuoco, che si
con certi pezzetti di mazza rifefla, e che fitirano, come 1 dadi, con altro
nome dette /e buffe. Vedi sotto C, 11. stan, 42.
STANZA XXXXI, STANZA XXXKIL
Come volontarofo Floriano, Piglian del campo,e al cenno del trombetta
Senza cbieder licenza, o cosa alcuna, Sivannoincontro con la lancia in resta; £
Si fece innanzi, e postavi la mano Ii Marchefe a Florian t' havea diretta;
j Di trarne (a pin langa hebbe fortuna, Per chiapparto nel merxo della testa;
Poo dopo il Adarchefe. di Soffiano Ma quei,ch't furbo, aun teposacivetta,
a... Simile a quella anch' egli ne traffe una E aggiufta lui,disendo: Afjaggiaquefia,
Ond' essi, come priaifn destinato,



male abbattuto. AZarche/e di Soffano, E nome a caso, e fa Marchefato una con-,:
trada,o villa vicina a Firenze detta Soffiano. 'f,

CHIAPP ARE, Val per colpire.

FVRBO, Se ben la voce furbo deriva dal latino Fur, che vuol dir Ladro, tut-
tavia ce ne serviamo per esprimere un' huomo scellerato, e che habbia ogni for-
 tadivizio, come s' è.detto sopra in-questo C. stan. 2. Ed ancora per denotare un'

huomo aftuto, e che sappia il conto suo, come segue nel presente luogo »
ee FACIVETT A. Abbaiia la tella.. Viene dal giuoco di eivetta, che da i già-
vanotti si fa in questa maniera. S' accordano tre; ¢d uno di loro, al quale @ toc-
cato in forte, si pone in mezzo a gli altri duc, i quali s' ingegnauo di a i}
' erEsh

Perché gli diede si spierata borta '
. Furono i primi a correr lo freccato, Ch' egli andò exit come una pera cotta,
2 e questi due furono i primi a correre la lancia, n¢! qual' inconcro il Marchefe ri-
ida
york

Floriano prese una di dette Bruschette,ed una ne prefe il Marchese di Soffiano;
j
}

by.








96 MALMANTILE

berrettino di testa con le percoffe della mano; € —— egli tocca terra con le
mani, non puo esser percoflo; e però hora alzandosi, hora abbaflandosi; tiras
guando all' uno, e quand' all' altro di gran moftaccioni; dura il giuoco:
che da uno delli due gli sia fatta cascare con un colpo Ia berretta dalla testa, che
allora perde il premio proposto, e lo vince colui, che gliel' ha fatta cascare, il
wale ( seguitandosi il giuoco ) va nel mezzo in luogo del primo. Tal-giuoco si

rs a tempo di suono, e piglia il nome dalla Civetta yccello, che per bulcare if
vitto scherza con gli uccelletti alzando, ed abbatiando la testa, come appunto fa
colui, che fla nel mezzo.” E da questo poi far civerra s'intende Abbassareil capo.
Da Scops, che € un'uccello notturno de! genere delle Civette. Era appresso i Greci
una sorta di giuoco, o paflatempo detto Scopias, nel quale veniva contratfatto a
tempo di balio il muoversi in giro, e l'alzare, e¢ l'abbassare della testa di quell'
uccello; onde ne fu formato il verbo Scoprein irridere, che appresso i Greci vale',
quel che appresso noi Toscani, Vecellare. V. Giulio Polluce |. 4. cap. 14.

AGGWST A iwi. Aggiultar uno, s' intende Bargli il fao dovere, e trattare uno
come eglimerita, Lat.coxcinare, Vuol dire ancora conciar male uno,come s'intende
nel presente luogo, e sotto C. 11.stan. 50. E per altro vuol dire Saldareso pagare
un debito. Lat. pariare.

BOTT A. Colpo, o percofla. E questa voce bortayper altro vuol dire una spe-
cie di Rospo. Lat. rabera.

ANDÒ gis com' una pera corta, Calcd già facilmente,ed a piombo,come fanno
Ie pere cotte dal Sole, che cascano facilmente dall' albero; o forse come le
cotte.al fuoco, che son facilissime a andar gilt in corpo quando si mangiano
Plauto disse: T'am crebri ad terram decidunt ut. pyra'; da che si deduce che s'intea-
da delle pere, le quali cascano dail' albero,

STANZA XXXKXIL J

tn quanto a Sposa, homai questoe ascoito; Che mette lui per morto, anzi fepolto,
S?ei rocco terra, ancor la voglia spati: Ma il giovane, che da di quei faluti
Così Florian dicea; ne Sette molto Gli mostra in avviarlo per le

» CH il secondo ne viene alpren battnti, LL error di chi fai conti fenzal Ofte
Comparue il secondo Cavaliere il quale si dava a credere d' haver già morto

Floriano; ma questo col burtarlo.a terra, gli fece conoscere quanto s' era ingan-
nato.
£ ASCOLTO. EB licenziato. 1 ragazi, che vanno alle fquole, quando sono
stati sentiti leggere dal Maeltro si dicono a/eolrs, e s' intendono licenziati: e così
iefto: Cavaliere efiendo paflato per le mani del Maeftro, che e Floriano, si può
ire a/colto,¢ licenziato dalla Sposa.. e
TOCCAR terra, e [putar la veglia, Dicono le donne, che quando son pregne,

» venendo loro voglia di qualche cosa, se in quello ante G toccano con le proprie

mani in alcuna parte di » quivi nasca alla creatura un segno simile a quel-
la tal cosa desiderata; e 1 segni poi chiamano vogiic; e che per sfuggire che
la creatura non nasca con tali segni 5 o voglies il rimedio sia, che la Donna pre-—
goa, quando le viene tal desiderio, tocchi subito terra con la mano, e4puti di-
cendo:. 4 terra vadia. B però il Poeta, seguitando questa opinione, dice, che se>
il Marchese ha toccato terra per liberarsi dalla vogiia della Dama, e necessario
anco-






SECONDO CIANTARE. '97
ancora 'chelegli sputi, a voler che il rimedio sia fatto' compitamente, |'Tal detto
| sputar la voglia, & assai vulgato per intender' unosche habbia gran desiderio d'una

-bilcofa; jche-sia-a Iii imposiibile.a-conseguire » Vedi Plin, lib. 28.c. a

\) (A SPROW baceati,, Acvatta carriera; Velocemente! Fran. Sacc, Novella 'mibi
3%. E così falito a cavatlen', ando'd sprom battnti al Palace de! Signori.

) LO. meite per marto, anzi fepolto. Intende; che questd secondo \Cavaliero non
folo credeva di havere auccidere Floriano; ma git pareva'gia d' haverlo uccifo..
Esprime la gran prefunzione, che havea di s¢ stesso muri Cavaliero., e la poca
stima, che faceva di Fioriano.

D1 quei faluti, Invende di quelle percoffe,

PAR il conto senza Ofte. Stabilire per fattayna cosa, alla quale deve inter-
uenire, e concorrere anche la volontà d' un' altro. Doveé T interesse del com-
paguo, si può metter ig sicura la propria yolonta, ma non: quella del compagno.

STAN ZA XXXKUL

—

Comparfo il terzo, in te/a della lizza All' altto manta il fettimo indirizza;
S? affronta seco, e paffalo fuor fupria Liortavose il none appresso inveffe,e fora;

. Soggiunge ilguarteed coli tel infirza. E cotna tutti con suo vant, e fama
Shudella il quinto, efreddailsestoacora Cave di teftail ruzxo-della Dama.

In questa otta va It Autorenactala vittoria', che hebbe Floriano di fette'Ca-
valient ye descrive la lor, perdita'in fette modi di diré diversi; il primo'lo pafsa fuor
fuora,il secondo, <a(si dovrebbe dire infilzazma non folo perché gli f rmel-
fa oot tea per. ¢aulaidella.rima,quanto anche petché per i più si dice infizza,

. enon #85¢\fatto lecito dirlo anch' egli) ib terz0 do pafsa fuor fuori, i) quar-
toua fredda's il quinto Jindirizzaall' altro mondo; il seto Pinnefte, edvil. settimo
dofora, EB 2 quell fette  i dire havendo quasi eutti lo -stesso significato d' am-

TT





' dana l'artifizio de) Poeta-in mostrate la fecon-
. ascenien. lingua jenn
4. Che si dice anche Nizza. Vuol dir linea; ma da noi s'intende quel








tayolato,"o/muro, rafente al quale corrono i Cavalieri le lance al Siracino.
CAVO! ditelea ilrnx20 della Dama, Fete ulcir'di testa il desiderio della dama.
Eaormenes » che dal.verbo razcare vuol dir Baie, wfata in questi. termini si-
Prucio, — sidefiderio, ec, si che dicendosi. Wi fale ha queffa ruzzo in

esmeld ale ha ono +e » gucfto liumore 5c, il Laica nov, mihi
8. dices... così fates gafhigatura, ee rhe re fener £08:
M07 ey 8 wren ta

wn T: A NZA XXXXIV.
AM Re firal 3 con Plaine Ond' ogni-altro ne fu mandato fano;
ny, Seefe di iconsaPiglnola | o>) (Bde nelle ze infino a gola





a se fae ieinteaiaagee 4 moi | ho. > Bem pastinc, servite, e ringrasiato
. Gome nel Bando havea date parola;. ~ Rimsfe quivi a goder il Papaty.
he Re fesse da Foca iano alla ee as er mo-
2 nr -Bioriano rimaie quivi a godere que
ta fisop 'eh Siitins 10

TOCC AR la mano. Elo stesso in were 'calo che che diciamo impalmare,
AL depeisoee tat toccamento, chefi iad Sins nama gry
N

3;



rc


ee
98 MALMANTILE?,

si;che il primo atto che si faccia per lo stabilimento del contratto del matrimo-
nio, Vedi sotto C, 12. stan. 50. '

CH ANDAT O fana, Cioè licenziato, ed esclufo. Il verbo: valeo, (che signifi-
ca Star fano, e usato da i latini anche per licenziarsi: parentibus vale\dixit, ed il
simile facciamo noi, come si vede nel presente luogo., che diciamo Mandar fani
in vece di licenziargli. Anzi 11 medesimo verbo vaieo è tal volta usato danoi per
intendere Addio, cioè licenziarsi. 11 Vai in una faa fiortola ( se ben points)
lo mostra dicendo.

Hore liete,

Tam vatlete. t
dam valete amati ferculi; y q
Etnnale, gyoesse,
O fedale, 7
Che maneggi i miei liberculi; i

U1 nostro Poeta sotto\C. 6, stan. 18.
Refti la donna, ed er le disse vale



oar:

WELLE dolcezze infinoa gola, Immerlo nei piaceri,e ne igufti, sotto C. 4.

fan. 42. dice esser ne guai agola.

GODERE il Papato. Goder le felicita concedutegli dal Cielo e

STANZA XXKKXV.
Tre di suonaro a felta le campane',
Ed altrertanti si band? il lavoro,
Eil Suocero., che meglioera del pane,
Viu' huom discreto ed un coppa d'oro,
Faceva con cli Sposi a fealdamane,
Talhord'a Mona luna,e Guancial d'oro,
E fece a' Paggi recitare a mente
Rofana,e la Regina d' Oriente,
STANZA XXXXVI,

LD andar il giorno in piazza ai Buratriniy
Ed agli Zanni furon'le tor gite;\
Ogni fera facevanfi fepini
Di oe » e di bakar veg lic bandite;
E chi non eraingambe, nein quattrini
Da trinciarle,e da fare ite, e venite,
Dicea novele, o stavale a ascoltare,
Faceva al Maxxolino, o alle Comare,

In queste quattro ottave il Poeta narra le fele, ed allegric, che si fecero ing
Campi per lo sposalizio di Doralice con Floriano; le quali fefte:fa'che non tra~
scendano eo nds pucrile per continovare a scrivere una novella per i Fanciulli.

meglio che il pane. Era un' huomo buonitfiaio, un' huomo che si

accordava a ogni cosa, appunto come è il pane, che s'accorda, ed unilce con,
tutte le vivande, almeno appresso a i Fiorentini. In questo proposito i Greci

. dissero, Columba mitior. f

VAN-A coppa a ora, Vn0 »l quale non sia da apporre alcunrdiferto, omni exce-

ERA un



STANZA XXXXVIL
Altri più la vedevanfi confondere
A quel giuoco chiamato gli Spropositi,
Che quei ch' esce di tema nel rispondere
Connien ch' il Subio depositi 5
Aa altri piace più Capanniscondere,
Hani? altri vary) humor yvarij propositi y
Perché ognuno aun mo none composto
Pero chi la vuol leffa, e chi arrofto,
STANZA XXXXVIIL.°
Chi fale Merenducce in sul bavaglo;
Chi con amico fa a Stacciabburatta
Chival! Altalena, e chi a Beccalaglio;
Va quello a Predellucce,un s' acculatta;
Per tutti in somma sempre vi fu taglia
Di fhar lieto cos} in barba di gatta,
LE tra Floriano, il Re, ela Figliuola
Mai fu che dir n' unt anno una parola,



pe








 $ adunano pil Fanciulli, ed uno,

SECONDO CANTARE: 99

eptiont maior, Credorche si-dica coppa d' are, per intendere oro coppellato; o di

ella, cioè raffinato, che Coppelia si dice quello strumento, col quale si ri-
duce l'oro alla sua vera purita, e perfezione;¢ Coppa vuol dir bicchiere, o altro
valo simile, donde poi Sortocoppa quella tazza, sopr' alla quale si portano i bic-
chieri, dando da bere ye Coppiere quel che porta da bere al Signore.

SC ALDAMANE. Quattro, o più s! accordano, e sates Sac ordinata-
mente le. mani sopra del.com »€\poi vanno cavando per ordine quel~
la mano, che sehen e meets sopra all' altre sane © con quelio
modo; €.confricazionepretendono scaldarfele; e però tale operazione e detta,
Scaldamane; ed ¢.giuoco/Fanciullesco, che ha la sua pena' per chierra cavando la
mano, quando non tocca a lui. at

MONA luna... S' accordano molti Fanciulli,e tirano le forti.a chi di loro hab-
bia a domandar consiglio a Mona luna, ¢.quello.a cui tocca vien fegregato dalla
conversazione, e ferrato in una stanza,.accioO.che non possa intendere chi sia,
quello di loro, che, refti elerco in Mona luna, della qual Mona Juna si fa l'ele~
zione fra gli altri, che restano dopo che coluié ferrato. Bletta:che e Mona lu-
na; si mettono, tutti.a sedere.in fila, e chiamano colui, che è ferraco, \acciò. che
venga a domandar il.consiglio a.Mona,luna.,, Questo tale se ne viene,\¢.doman-
da il consiglio a uno di quet ragazzi, quale egli crede, che sia slato eletto in Mo-
na luna, e se s' abbatte a trovarlo,ha vinto;se nd; quel tale, a cui ha domanda-
to il consiglio gli risponde; lo non son Mona luna, ma fla più gil, o più fu, se-
condo che.veramente.¢. polto quel tale, che @ Mona luna; ed il domandante per-
de il premio proposto, ed e di nuovo riferrato nella stanza per tanto, che dai
Fancuulli sia creata un' altra Mona luna, alla, quale egli torna a domandar confi.
glio, e così seguita fin a che una volta s'apponga, ed allora vince; e quello.che
= Mona luna perde i) premio,, e vien riferrato nella stanza,diventando colui, che
deve domandare, e quello che s' appose,s' intruppa fra gli altri ragazai.. 1 do-
mandante richiede fino,a quattro volte il consiglio, e può perder quattro prevj,
¢ poi fimescola fra gli altri ragazzi, efente però da dover pil esser domandan-
te, se non nel caso,, che fatto Mona luna, egli perdeffe, e sempre fitorna as
ercare nuova Mona luna, e si deputa nuovo domandante, quando il primo s'ap-
ponga » o habbia domandato, quattro volte il consiglio, la qual fuazione, come

detto » non può esser forzato a fare, se non quattro volte: edi premj si adu-
nano, e feditribuiscone poi fra di loro riparticamente, e dal rendergli poi a di
chi sono, cavano un' altro paflatempo, come diremo, Da questo giuoco viene
il proverbio Più fu fa Afona luna, che significa Nella tal cosa è mifterio pil im-
portante di quel che altri si pensa. 7

Nota che. taato questo giuoco, quanto ogni.altro, che troveremo nella pre-

sente Operas' altera,.7 e diversifica secondo li gusti, e conucazioni pue-
— nili;¢.noa mi Tipe ne haveffi nella tua puerizia Eatti » o veduii fare

alcuat, o tutti diversamente da quello, che io gli descrivo *
GV ANCLAL @ oro, Questo eure € giuoco Panciullesco, quale € fatto così:
mette.a sedere sopra.a una seggiola, ed un'
altro se li pone inginocchioni avanti, e pola il suo capo in grembo a quel che>
ficde, il quale gli chinde gli occhi con le mel » acciò che non possa vedere chi
. 2. ha

Oe ees 6
be







100 MALMANTILE 52 if

sia colui, che lo percoffe in unaimano »che egti si tiene dietro \sopr' alle ren; doz
vendolo egli indovinare;.¢'calaiche gli fertagli o¢chi:, dopo \che questotale ¢
flato percosso glidice 2 Chi sha percofa? edegli risponde: Ficefeccho y ef altro
replica: Aderamelo qua per un' orecchio, Ed allora quello fi'tizza',¢ va @ pigtiar
colui, che egiterede it perdutiore y¢ (e $” appone, ha vinto y e ponetil percutfo-
re in lnogo suo;¢ li fa.dare il premio in mano a quello chic.fiede's @ se hom s' ap-
pone perde ih premio.; quale. coniegna, ai detto fedente, e ritertia al udgo di 'pri-
ma per continuare; fin tanto.che s*appone,ed alla quarta vol 'si fa huova'clez-
zione,come sopra a>Monayiona; 'Guetto mi par di potcrcredere j\che sia quel
gioco; che i Greci.chigmavano Coldubi/mo siferivo dal Baleng. de lad.veteap.37:
gual giuoco da quel Propheriza: quis te percufit 7 detto per disprezzo da Giada a
Giesti Crifto sig.\ noft'o, si pudvarguinentare, che fafle anco appresso a-i Latini.
ROSANA,¢ la Regina @Oriente. Sono duc Leggende,o Rappresentaziont n0-
Usfime, per esser cantate giornalmente da ogni donniccinola ngeBh ONO?
BVRATT IN1, litehde quei Figurini di: legho, che for fatti: muover da une, che
a cal effetto s' ascondé in un castelletto di legna coperto di' pannd; € gli fa operas
re.mettendofegli sopra alle punte delle dita,e cd un certo uo fifehio git fa parlare,
ZANNI, Per Zanni, che's'intehde servo feioceo Lombardo, qui intendé ogni
sorta di Bagateellieri, che fanno ilibuffone per le piazze ys) 8.
FEST INI di giuocojec, Quando's' adanano'in wna casa ae Dame', e Cavalieri
per giuocare insieme, o per ballare nella 'prima parte della norte, dice fare un
Feftine, o Veglia. E te bene veglia strettamente presa, pare che significhi più trat-
teaimento di-ballo, ch¢ di giuoco, tuttavia la pigliamo per intendere ogni sorta
di teattenimento, o di Giuaco',' 0! di Ballo yo di qualfivoglia altra cosa', nellas
wale si spendano Ie prime hore della notte, dicendosi: Aoi facemmo la deglia &
dudiare » 4 ballare, a cantare, ec, Ma volendo pigliare queste due voci nel suo pro-
prio significato; Feitino, S' intende adunanza di persone nobili, sia per ballare,
© per giuocare in quelle hore della notee; e Feglia ¥ intende d” ogni sorta di per-
fone ordinarie; E si come s' avvilirebbe ve: fo fui alla veglia nel Palazzo
del Principe così pare, che si burlerebbe dic*ndO: Fué al se/Pino im cafd'uh Battilano,
Quando si dice Feffino'pubblico, o Vee liw bardica s intende Feffino, o Peglia & por-
ta aperta,'dove pud' andare ognuno e Vedi sotto: G@ 9 stan. 51. e Cant. ro,

stan, 28. ' i &
NON era in gembe; ne it quaterin' | Now si sentiva'gagliardo da ballare,.¢ non
stani23:

haveva monete da poter giuocare. © '

DA trinciarle, Intende da far capriole, cide fattare'. Vedi sotto C, 7, stan.

DA fare ite, e venite, Cio givocare. Quando fi'giudea, € rdendo si paga
la posta volta per rota 50 rlguot quia la vine dca oO fare ire',
nite 5 ¢s' intende pagare il subito perdata la posta'; e riceverio'nello'
niodo vincendo; ed e il contrario del detto Parenti ae-gli bai; che figaifica
care in fu la fede, o a-credenza, RF 2 OF Die gTo AND Vib AE er

| MAZZOLINO'. Ancor 'queito'® trateehitienco da Panchull', e si fa in tal gui-

fa. Pia i adwnano inlieme;e si piglino'i nome & tn fiore per ciascuno,
¢ di questi fori un di loro, che @ il' Giatdiniere compone un mazzo', € poi dice:
Questo mazzo non fa bene per causa della Viola; € colui, che ha'prefo a













ice

ies



















\

SECONDO CANTARE: ror

delta: Viola deve risponder fabito: Dalla Viola non viene,ma si ben dal Giglio, o
altro fiore, che' a' lui verra nella mente'; e se'non risponde subito', o vero se no-
mina un'fiore') che non sia in quel mazzo, perde un premio, i) quale si da at
Giardifiiere; 'E così vannio seguitando fino a che il Giardinere habbia in mano
tanti. premj da potere alla fine del giuoco distribuirne almeno uno per ciascuno di
quei ragazzi, che sono nel giuoco; ed il Giardiaiere e fottopotto anch' egii alla
perdita del premio', perché f€ un fiore dara !a colpa a Ini, e che egli non rispon-
da'fubito, e nomini un Fiore, 'che non sia nel mazzo'; perde come gli altri, e it
fio premio va dato in mano a colui, che I" ha fatto errare; ma core in deposi-
to', perché 'alla' fine' del Giuoco va poi con gii altri distribuito 'dal Giardiniero
il quale'non lo può però dare a se medesimo; E questi premj ff domandano pegvi,
edi questi intende il Poeta dove dice: Convien ch' if pegno fubiro deposit!.

Finito il Giuoco i} Giardinicre distribuisce ripartitamente e pegui pigliandone
anebra per se.Tali pegni poi sono da coloro, che gli hanno dal Giardiniere havuti,
reftituiti a i proprj padroni, i quali, se li rivogliono', devon fare una cosa fecon-
do il gusto'di colui), al quale e toccato in forte il detto pegno; E questo dicono
far ta penitenea, Ya quale se egli non fa, il pegno refla in mano a colui', al quale
€ roceato'; e però questi pegni devono esser di qualche valore, accié che i padro-
ni habbian caro di riavergii. Alle volte fanno questo giuoco iGiovanetti di mag-
giore eta ) € riducorio questi pegni a moneta, quale depositano ogni volta, che

'in'mano a un'deposirario, e se ne fernono per far merende, ec, fal giuo-
eo peo diffimile a quello, che facevano i Greci dctto Bafilinda riferito da Giu-
lio 'Polltice tab:'9. '¢. 7..¢ dove noi dicianio Giardini¢re essi dicevano Re, comes
facevano anche i Latini, e ciò si deduce da Hor. Ep. pr. ub, pr.

3 ——— At pweri ludentes, Rex evis', ainnt,
Si rette facies', bic murns abenens essa, ce.
'Rofeia, die fodes 5 melior lex, an puerorum

: WNama ? que Regnum rette facientibus ofert.

Se bene potrebbe dirsi, che Orazio tion imtenida di questo giuoco particolarmen-
te,perché in tutti li givochi Fanciulleschi tanto i Greci, che i Latini chiamavano
Re colui', che vinceva, ed asino quello che perdeva; ma perché nel giuoco pre-
sente era farto Giardiniere (0 diciamiolo Re Wghetto » che in altri giuochi era ri-
mafto superiore a tutci, però sion m' ajlontano da interpretare Orazio, ed appli-
'care 3 suo 1ndgo al presente proposito, nel quale, se it Re errava diventava
YP asino, e Re si faceva colui, che havea fatto errare, o tenendosi il conto di
'chi di loro haveva meno errato, quello alla fine era il Re,¢ quello che più volte
haveva errato era l'Afino', o Re Mida. Vedi il Meurfio de Ludis vererum, Gli
Spartani similmente per Legge di Licutgo, econo che riferifte Plutarco'nellas
vita del medesimo', ai Ragazzi di pil di fer' anni, ei come Principe
il più favio tra loro, che sopranrendetie a' loro giuochi, e Fanciulleschi esercizzj.
~ “ALLE comare. 6 giuoco è trattenimenco di Fanciullette,e 10 fanno così:

Mettono una di loro in un letto con un barrboccio fatto di celici, e flngendo;che

“guefta habbia parrorito, le fanno ricever le vifite da-altré Fanciullette con far

quelle cirimonie, ed accompagnature, che f coflumano in occasione di vere par-

tucienti. Nk,: 3
t i tal


102 - “MALMANTILE

Tal giuoco era usatoancora dalle Fanciullette Greche secondo Giulio Pol.lib.9,¢i7;
ma in-vece d' una Parturiente fingevano una Sposa; e lo dicevano Phitramelia.
Qual giuoco fanno pure ancora le nofire Fanciulline, e lo chiamano far' alle Zie
Non ha questo giuoco delle Comare, o Zie altro fine, che di paflare il giorno in
quelle loro tirimonie, e ricevimenti, ne i quali alle volte si consuma quello, che
Ie Fanciullette hanno havuto per merendare.

GLI spropositi, E lo stesso in fultanza, che quello del mazzolino, se non che
dove in quello si finge un Giardiniere; in questo 1 Ragazzi s'adattano a qualfivo-
glia altra cosa, con pigliarsi quei nomi, che attengono a quella tal cosa;
¢lempio: Faranno il giuoco sopra il pane;il Macfiro fara il Fornaio, e gquesto fara
quello che nel Mazzolino fa il Giardiniere; uno fara la farina, uno l'acqua, uno
il forno, ed altre cose attenenti alla conftructura, e perfezione del pane; IL
Fornaio dira: Questo pane non è buono per caula della Farina; quello che has
il nome della Farina, deve risponder subito: Dalla farina non viene, ma dail'
acqua, o da altra cosa che gli venga in mente, atteneote al pane, e che sia frais
loro Ragazzi; e se non risponde presto,, o non da la colpa a qualche cosa, il no=
me della quale non sia in quella adunanza,o non sia attenente al pane, perde 5 ¢
deposica il pegno; e si fa nel refto per appunto come nel giuoco del Mazzolino:
E questo giuoco universale è forse quello, che habbiamo detto sopra, che face-
vano i Greci detto Bafiinda, E da noi si chiama il giuoco de gi S; » perché
dovendo quei Ragazzi risponder presto, attribuiscono al pane cose spropositanf-
fime,¢ che non hanno che far punto col pane, o sua bontà, oltre a non esser
il nome di quella tal cosa in veruno di quei Ragazzi. £ quello vuol dire VJew di
tema,

Habbiamo un' altro modo di far questo giuoco, ed è così: Mettonfi più per-
fone a sedere in giro, e ciascuno dice al compagno in uno orecchio una parola,o
due al pil, e finito il giro, ciascuno ordinatamente dice force quella parola, che
gli e fata detta dal vicino, e volendone comporre il periodo si sentono gli Spro-
politi, che rifultano-da quelle parole; e fida la pena a colui, che ne @ stata las
cagione.

re 4 niscondere. Vino si mette col capo in grembo a un' altro, che gli tura
gli occhi, ed un' altro, o più si nascondono, € nascofti danno cenno, € colui sche
haveva gli occhi ferrati si rizza, e va cercando di coloro, che sono nalcofti, e»
trovandone uno basta per liberarsi da tornare in grembo a colui, dove mettes
quello, che ha trovato, e questo perde il premio propotto, e il trovatore va as
nascondersi; ma se non trova il nascofto in tante gite, o in tanto tempo, quan-
to sono convenuti,perde il premio, e ritorna a flar con gli occhi chiuli come pri-
ma; e seguita così fino a quattro volte, perdendo quattro premj, come sé detto
sopra a Aduna luna, ed i premj poi Gi distribuiscono come si fa al giuoco del A¢az~
zalina, E quello far con gli occhi ferrati si dice far sotto, che i Greci in un Gmil
giuoco dicevano catamyerm, Lat. connivere. E coiui che è stato fowto quattro vol-
te,¢ non ha mai trovato il nalcofto, e per consegucnza perduti i quattro premj,
occupa il luogo di colui, che teneva fowo, € questo s-iniruppa con gii altri Ra-
gazzi,fra i quali si tira la forte a chi dee Mar foro, o nascondersi; E così segui-



tano canto y che si riducano tutti liberi; perché quello che ha pagati li quatcro~

prem)



3
5
&










SECONDO CANTARE. 103

© premj nel modo suddetto, ed ha occupato il luogo di tenere gli altri sotto, come
ne vien cavato nella maniera accennata, resta fuori del giuoco, del quale folo
attende la fine per conseguire anch' egli la sua parte de 1 premj da distribuirsi, Era
“ancor questo giuoco appresso a 1 Greci, ¢lo chiamavano -Apodidra/cinda secon-
do Giulio Polluce lib. 9. c. 7. 5 ma diversificava alquanto; Ed in questo giuoco
pure il vincente era detto il Re, ed il maggior perdente ? Asino. Vedi il Buleng.
de lud. Grace, cap. 22. éd il Meurfo in verbo e4podidrascinda. Simile a quetio
era ancora il giuoco detto da' Greci Myinda,

OGNVNO 4 un mo non è composto. In questo proverbio sentenziofo habbiamo
f ancor noi come i Latini pi modi di dire, come: Le nature son diverse. Tanti
)— huomini tante berrette, o tanti cervelli, Tutte non possono esser aun modo, Chi la
te 4 leffe ye chia rofto,¢ molti altri; ene i Latini si trova. Quot homines tor fen.
tentie,, Suns cuique mos, Trabit sua quemque ia « Won omnes cadem mirantar,

amantque, ed altri infiniti, e tutti con lo stesso significato.
PAR le merenducce. \ nostri Stovigliai in alcune Fiere, che si fanno in Firenze
il giorno della festivita di San Simone, ed in quello di $, Martino conducono





; gran quantità di stoviglic piccolitfime,come piatti, tegami, pentole, ed ogni al-,

tra specie di arnesi,¢ vasellami da cucina, che da essi si fabbricano di terra. Di
; este si provveggono li nostri Fanciulli per quanto vien loro permefio dalla loro
~ borsa, e da queste vien poi loro l'occasione di far le AZerenducce, percht haven.
-sdo altre'mafierizie adeguate, come tavole, sgabelli', bicchieri, faluiette,¢ fimi-
'li); imbandiscono una mensa, accordandosi più Fanciulletti, e Fanciuiline a por-
tare quello', che e dato loro per merenda, ed accomodando tutto in piccole par-
ticelle, le distribuiscono in quei piattellini,figarando di fare un Banchetto, e met-
*tono a sedere a quella tavolina li loro Bambocci; E queste fonda loro chiamate
© Merenanece,delie quali parla ij Poeta, e le quali erano usate ancora dalle Panciul-
line antiche in occasione del suddetto appellato Phitramelie » come si ca~
va dal Meurfio, dal Soutero, e dal Buleagero.
BAV-AGLIO, Saiuietta » o Tovagliolino da Bambini, che si lega al collo con
due cordelline, o nastri, detto così dalla bava, che sopra vi casca dalla bocca de
~bambini; i Latini pure secondo 1l'Onomastico lo dicono pe'torale falivarium,e con
A mee Bavazlé come lor proprj arnesi apparecchiano Ie loro piccole tavole quan-
~ do fanno le A4erenducce, e Gi mangiano quelle particelle distribuite in quei piattel -
“Aini3-come s*è detto sopra. EB di queste Aderenducee parla il Poeta.
ST ACCIABBVRATT A, Due seggono incontro l'uno all' altro, e si pigliano
per le mani, e tirandosi innanzi, e indietro; come si fa dello staccio abburattan-
do la farina, vanno cantando una lor frottola, che dice.
} Staccia abburacta
» Martin della gatta:
La gatta andò pel vino, ec,
E questo è trastullo usato dalle Balie per acquietare i Bambini di quella eta, che
 appena si reggono in piedi 8 |
\\ ALT ALENA, Paflacempo da Fanciulli; Legano due funi al palco,o vero a
* due al beri, e le fanno calare a doppio fino pretio a terraun braccio, e sopra di
» est funi accomodano un'afle, sopr' alia quale si pone uno, o pits a sedere, ve
; are






ea | ee


104 MALMANTILE

dare il moto a detta affe vanno cantando alcune canzoni con, tin! aria aggiustata
al tempo dell' ondeggiamento di quell' affe, e questa ¢l' £ora de', Greciy,dai La-
tini detta Oscillatio, ed alore, yolte. Petaurnm penjile, ¢noila diciamo Alealena dal
Latino Todewen, che vuol dir quella Macchina di legno.,,con,la quale si cava.Jtac-
qua de i pozzi ( come si vede in Plin, lib, 19, c, 4. Vel Tollenonum haufkn rigandos)
da noi detta Mazacavallo. Vedi sotto C. 6. staan. 86.,E questo perché faceyano
l'Altalena, come la fanno talvolta anche li nostri, Fanciuili con iacrocicchiare
una trave sopr' all' altra, e ponendosi ugo o pil ragazzi per teftata della trave,
che e di sopra,la fanno alzare, e abbailare a foggia di Atayzacavallo. Diguestas
parla il Bulenger, de Ind. vet, c, 11. Questa ditaiera, in aicuni luoghi di, Tofeana
€ detta bictancole, 5,
BECCALAGLIO. E' un givoco simile alla mosca cieca detta sopra. Cox, stan,
40. ne vi è altra differenza, che dove in quello si da,con yn panno.avwolto, 9 altra
cosa simile,in gquelto si da con la mano piacevolmeate una fola, volta da.colui, che
bendé gli occhi a qucl, che sta sotto, ed il bendato in wece di dare, 5 affanna di
pigliare un di coloro, che in quella stanza sono del giuoco, e colui che reftera
preso,deve bendarsi in luogo del bendato, e perde,il pegno,.0 premio.,ed.il pri-
mo bendato resta libero, e s' intruppa fra quelli, che hanno.a eden prefi sie si fa
come sopra nel giuoco di Guancia] d'oro, Sidice Becealaglia,. questo tale
bendato vien condotro in mezzo della stanza, 0, piazza, doye s! hada fare il giuo-
co; e colui che lo bendo, e che quivi l'ha condotto gli dice; Che fei tu-venuto\a.n





fare in piagza? Ed egli risponde; 4 beccar /'aelio, E quello,dandogli leggiermen-
'te See fur' una spalla foggingne:, O beccati codefto, Dopo. la qual fungio-
ne il bendato s' affatica,di pigliar uno per metterlo in suo,luogo. 1 Greci appel-

lavano questo pernen Claeiede da pencola che in Greco, si dice Chysray edo fage-
vano nella stessa maniera; ma in vece di, bendare gli,occhi,mettevano.a colui, 0
fingevaG, ch' egli tenefle colla finiftra una pentola in capo, e girandogli intorno
lo [olleticavano, o percotevano; onde, se egli rivoltandosi, prendeva chi gli
tirava; il preso rimaneva in cambio suo a.essere quel della pentola, 4 Latini lo
dicevond tidus ollarins s 4 &, aillapre>- Su
Simile.a questo era un'altro giuoco usato dalle Ragazze Greche, detto, Cheliche-
Jona, vel quale, messa,a sedere quella, a cui dayano nome di Chelona x chevuol
dire Teftuggine; Ie dicevano: Chelichelona quid facis ix medio ?. quella risponde-
va: Lanam sexo,@ filum miltfium con quel che segue riferito dal-Buleng. de>
Jud. vet. cap. 41. F ae Tere
Nel giuoco poi della Chyrrinda, ovvero, ludus ollarins dicevano: Quis ollam ?
¢ chi teneva la pentola rispondeva: Ego Adidas, cs! affannava non di pigliare un
di coloro, ma di toccarlo co i piedi, e quel tale casi, tocco perdeya, e si metteva
la pentola in capo; E perché ( comes' e detto sopra ) i Greci havevano per co-
flume di chiamare Re il vincitore, ¢d asino il perditore., pecd questo tale, che

havea la pentola in capo si. a Adida, clot Re efino, Vedi Giulio Polluy —

ce lib. 9. c. 7. ed il Buleng. de Lud. Vet, c, 17.

ANDAR a predellucce, Duc si pigliaag peri pol a' ambedue le: st tno!

con l'altro in croce, e formano come una seggiola, e un' altro vi fiede: 3¢

questo si dice andar' 4 predellucce. Das feis=is-ninne un guinea ae aekans*>,
ae ae






+

&

4




SECONDO CANTARE: 105

ed era il portare uno in fu le spalle, e reggerlo, tenendo Ie di lui ginocchia nelle
paime delle mani voltate dietro alla persona, e detto Zz Coryla, cic nella,
siotola, © cavo della mano. Ma questo credo che sia un' altro giuoco, che noi
diciamo 4 cavalluccio, che vedremo sotto C. 3. stan. 30. tanto più che i Greci se-
condo Io stesso Polluce chiamano questo giuoco detto 4x Cory/a, per altro nomes
Hippada dal verbo Hippazin, cavalcare. E questo se bene € giuoco, tuttayia &
specie di pena per quei, che portano per haver perduto ad altri de' suddetti giuochi.

ACCVLATT ARE, BE' pafiaiempo da Ragazzi, ma è specic di pena, e di tor-
mento dovuto a colui che € acculattato. Quattro ragazzi pigliano uno per les
braccia, e per i piedi, e formandone un quadrato, lo follevano, e gli fanno bat-
tere 11 culo in terra tante volte, quanto merita il suo delitto, o perdita, che ha.
fatto in altri giuochi, come sopra. E questo si dice acculattare, che in altro si-
guificato vedemmo sopra C. 1. stan. 7. Gli Spagnuoli chiamano l'Acculattares

“mantear,perché mettono colui che si ha da acculateare in una coperta,o mantello,¢
tenendola da quattro capi, lo sbaizano in alto, € lo fanno ricadere in essa, e noi
lo diciamo dar la coperta,

V1 fu caglio per tutti, Vi fa da dar soddisfazione a tutti. Ognuno hebbe in che
impiegarsi. Traslato da' Sarti, che dicono:in ete roba ci è raglio per un' Abi-
to,0 per due,ec, per intendere,ci e tanta rcba, che si può fare un' Abito, o due, ec.

ST AR in barbadi Gatta, o di Micio, come si disse sopra in questo C. stan. 28.
annotazione alla voce sbigortito, Pare che oo detto possa venire dall'anticas
superitizione degli Egizz}, i quali credendosi, che il Gatto fufle consegrato alla

fide:, che era ia loro Deita maggiore, non folo nutrivano con granditiima
cura, e splendidezza questo animale, ma secondo Pierio Valeriano reputavano
degno di morte colui, che ne ammazzafic, o faceffe loro oltraggio. E riferisce
Alcx,ab Alex, dier, Gen; lib. 3, cap.7. e lib. 6. ¢. 14. che quando moriva un Gat.
to,i medesimi Egiz2j per contraficgno di dolore firadevanole ciglia,e poi metren-
do addosso al morto gatto fale, ed aromati, e coprendolo con un panno bianco lo
feppellivano, facendoli talvolta fepolcri notabil;tanta era la flima che ne facevano.

XXXXIX. STANZA L.

Mai fu tra lor fin qui nulta di guafto, Bench" if Suocero altura, e la Conforte
Se non che Florian volto ale cacce, Maledicefrer questo [uo motivo

. -Hatvendane più volte tocco un taffo, Dicendogli che la fuor delle porte

 E sentendosi dar sempre cartacce, Va' Orco v' è st perfido, e cattive y

Dispose al fin di nan'veler pi pasto, Che perfegutta ? huomo infino a marte,
We curando lor preghi, ne minacce E che lt ingoierebbe vivo vivo;
Fece innitar da i faliti Bidelti Con gentt, ed armi usct sul axrora
Per Paltra di i Piacevoli,¢ i Piattelli, Gridado: Andiane,adiane,eccolafuora.

Non hebbero ( come s' e detto ) questi Sposi mai occasione d” addirarsi, se non
che Floriano inclinato alla caccia si risoluette andarvi a dispetto della Moglie,

¢del Suocero:.

NON fu nulla di guafto « Non furono tra loro mai rotture; cioènon s' adira-
Tono mai; ¢, come si dice 5 non s' ingroflarono i fangui.
 HAVENLONE toveato un vasta, Havendo di ciò domandato alla sfuggita, 0

 discorfone con brevita. Tratto da i tafti del Cimbalo, o'vero Organo strumenti

ae

musicali. oO DAR

*

3%,
106 'MALMANTILE

DeAR cartacce, Non rispondere secondo il gusto di chi richiede; Traslato dal
giuoco di minchiate, nel quale si dicono cartacce quelle che non contano.s:¢ fo-
no di niun valore. Vedi sotto C, 8, stan. 61.

DAR pasto, Trattenere uno con scufe, o chiacchicre. E il latino verba dare; /pe
laitare. E si dice così,perché il polmone degli animali(che da noi si dice pafo)stracca
colui, che lo mangia, ma non lo fazia. Si dice anche dar paffo, quando uno, che
fa giuocar bene a un tal giuoco,finge di saper poco, e si lascia vincer da princi-
pio, a fine d' indurre il semplice a far grosse potte per vincergli assai.

SIDELLO, Donzello, o Servitore d' Vniversita, o d' Accademia, come fa-
rebbe quel Donzello, che serve allo Studio di Pifa, o ad altri simili. E questo
nome di Bidello secondo }' Autore delle Notizie Ecciefiaftiche e corrotto da Pe-
dullus, perché questo Viiziale, ( dice egli ) che nell' Accademie, e negli Studj

pubbiici haveva cura d' efeguire le commiffioni appartenenti allo studio, folevas
portare in mano un bastone chiamato Pedo; Quantungue altri ( soggiunge il me-
see) tirino la sua etimologia dalla parola Saffonica Bydell, che vuol dire il
anditore,

Ma io.credo che il nome Bidello sia tolto da Berul/a, che & quell' albero, del
quale si facevano le verghe per i fasci, che anticamente portavano 4 -Littori
d' avaati.a i Magifirati del popolo Romano, e che da questo portare i fasci di
verghe di Betulla, sia poi venuto il nome di Bidello a tali serventi di Vaiversita,
i quali faono figura di Littori, € nello studio di Pifa portano ancora una grofia
mazza d' argento ( significante gli antichi fasci ) quando vanno in funzioni pub-
bliche avanti al Collegio de i Dottori.. Alex, ab Alcx, dier. Gen, lib. 12.17. in
fine, dice così.

Quodque fascibus, quos praferebant Lictores, betullas virgas maximt commodas dis-
sere, itague ex illorum virgis tum proper candorem tum propter tennitarem publices
Lasces, qui magifiratibus prairent, efecere. E Plinio lib.6. c. 18. Gander frigidis for-
bus,& magis Berulla; Gallica bac arbor, mirabilis candore arque tenuitace, terribilis
Maziftratuum virgis. Lo stesso attefta Polid. Verg. lib. 4. c. 3. 1

oa » e Piattelli, Sono in Firenze due conaerfazioni di cacciatori ~

wali andando alle cacce gareggiano fra loro a chi faccia. maggior predayequeila,
ane rimane superiore, nanos suole entrare nella Città teionfante cebaoth 5
carri, ed altro; e l'una si dice la Compagnia de' Piacevolé,¢ Valtra de' Piatrelli;
¢ ciascuna ha la sua fanza catro alla quale s' adunano. gli Vfiziali, e Servcati,
¢ Altci; e questi son quelli de' quali dice il Poeta, e chiama i loro serventi
idelli.

VN' Orco. Questa e una beltia immaginaria inventata dalle Balie per far paura
ai bambini, figuraadola uno animale specie di-Pata, nimico dei bambini catti-
vi, ¢d il Poeta, che noms” allontana mai dal genio'pucrile, mostra che ibfuoce-
ro Stordilano voleva indurre nel. Floriana ihtimore per farlo aftenere dais
andare a caccia, con dirgli che fuori della porta v' era l''Orco, ers,
huomini: Questo nome però viene dall' antica saperftizione de i yi quali
chiamavano Orce l'Inferno Virg. a. lib. 6, Priemifque in faucibus orci, Bd inten-
devano per Orco anche Plutone, quali wrgus, five Kragus ab urgende egli
sforza, e Spinge rutti alla morte; e perciò dalle madri, e nutrici per' Peas
ial

> Skee? pp seeth seh ce etre,.. x




SECONDO CANTARE: 107

alli lor bambiai si dice che ' Orco porta via: il che pure vien da i Gentili, che
igliando Orco per la morte, lo chiamavano Ineforabile,-¢ rapace. Orazio
Bae 18. lib. 2, Nulla certior tamen
Rapacis Orci fine destinata,
~ GRID ANDÒ andianne andianne., ec, Così vanno gridando i cacciatori faddetti
la mattina avanti giorno per svegliare i compagni. Lo stesso, che Alo Alo; ovs
vero lon dal Pranaele eAilons.
STANZA LL





Senza veder ne anche.un' animale
Frugo, bufso y gird pi di tre miglia;
Pur vedde un tratto correr un Cignale
» Ferace, grande, e grosso a meraviglia,
STANZA LIL
Che a posta presa havea quella fembianza,
E glk pafso fuggendo atlor a! avanti
Per traviarlo folo con speranza
D: haver a far di lus piit boccon fanti;
Così guidollo fino alla [un spanra
» Dov'ei pense di porgli addossoi guanti;
~ Poi nan gli parue tempo, perché i cani
» blauersan piit tafto lui mandate abrani,
STANZA if,
Pero.volends andave in sul sicura
Won a perdira più che manifefia,
Perché a reder roglieva un' offo duro
~ M€entre non to chiappaffe testa testa;
Glisparid' vcchio,e fece un tempo [euro
Per incanto levar, vento, e tempefta,
£. uolash gros comparire,
Cc Dearehte ph ans may che mi dire,

usioriang scorfe moita campagna, e ce

Ond'ei, cht il di dovea capitar male
Si moffe a seguitarloa tutta briglia,
Won essendo infor mato ch'in quel Porce
Si trasformava quel ghiotton dell Orco,
STANZA LIV.

A cacciator, che quivi erain farfetto,
E dal sudore omai tutto una broda,
Havendo un vestituccio di dobretto,
Ed nn cappel di brucioli alla moda,
“Per non pigliare al ventoun mal di petto,
O altro', perché il Prete non ne goda,
Won trovado attra cafain quelfainatico,
Che quellagrotta, infaccavi da pratico.

STANZA. LV.

Atal gragnizola, a venti così fieri
C* ogni cosa mandavano in rovina,
Tal freddo fu che tutti quei quartieri
Sen! andananoin diaccio,¢ in gelatina,
Ed ci ch' era vestito di lercieri,
E mai meglio facea la furfantina,
Won più cercava capriole, o damma,
414 dafar,s' ei poreva,nn po di famma,



rcd buon pezzo, e non trove mai nulla,

se non che pur vedde un grotio Cignale, a] quale si mefle dietro co i suoi cani,
* non sapendo, che era l'Orco trasformatofi in quel cignale per pigliar Fio.

riano dalla vitta lc spari, ¢

pioggia, ctempefta, la-quale obbligo Si

t via de' uoi incanti fece venire una gran
loriano a ricovrarsi in una grotta, che era

vi fra quelle macchie, nella quale entrato, si meile a cercare se trovava modo
deme un po tiesto 4
£.

AVGO.. Cioè cerco:minutamente
do con le pertiche per tutto... ~

'frugando per le fiepi con i cani, e buffan-

DOVEA capitar mate. Doveva haver disgeazic. Doveva rovinare, E il Lat.

-— Perdsyperire,
. cd TVTT Abrigiia, A quto corso

.. GHIOTTONE. Epiteto solito da

«



f  senza punto fermarsi, come fa il cavallo
quando se gii lascia ee « Laxatis babenis
( ) 'aun' huomo maligno, e di genio cattivo,
€luona quasi lo itetio, che Briccone, furbo, viziolo, scellerato.
+ O2

py


108 MALMANTILE

'PIV boccon fanti., Più buon bocconi. La voce fanti in cafi simili significa per-
fezione in generale. Vedi sotto C, 3, stan. 8.

PORRE iguanti a deffo. Piglia guanti per mani, e vuol dire Pigliarlo.; Hab-
biamo il verbo agguantare, cioè pigliare », Guanto dal Germ, Hend:, mano.

ANDARE in sul sicuro, Andar senza paura. Mettersi a fare un negozio con
sicurezza di non esser'impedito, e che riesca secondo l'intento.

TORRE a rodere ut? offo duro, Pigliare a fare una cosa difficile.

CHIAbP ARE, Qui val per ritrovare,e sopra in questo C. stan. 41.per perquo-
tere; ed il suo propeie significato e Pigliare; dal Lat, capere.

TEST A testa, Cioè a folo a folo. Remoris arbitris, Diciamo anche a quat-
tr' occhi.:

GRAGNVOLA, Grandine, che è gocciola d' acqua congelata nell' aria per
forza di freddo, e di vento, e si fa di vapore freddo, e umido stropicciato nelle
parti interiori.del nugolo, La pioggia nalce da vapori freddi, e umidi adunati
ne i nugoli, La xeve e impreffione generata di freddo, e d' umido; e questo fred-
do @ minore di qucllo,col quale dalla pioggia vien generata la grandine, ed ha in
se qualche parte di caldo. La rugiada e gencrata di freddo, e di umido non rap-
preso, e questa congelandosi nell' aria diveata la brinata « Ho voiuto,benché fuor
di proposito, notare l'origine de i sopraddetti accidenti dell' aria, perché da.
questa s'intendano i loro nomi; in qualche parte d'Italia per ayvencura differenti.

HAVREBBE infranta non fo che mi dire, Havrebbe schiacciata, o diciamo an-
che ammaccata qualfivoglia cosa per dura che fufle; Non fo immaginarmi, ne
dire cosa tanto dura, che ella non l'havefie infranta. Questo termine 2on fo che
mi dire usato nella forma, che si vede nel caso presente, significa quel che s' e det-
to; ma per altro.' usiamo anche per denotare di non havere, o saper trovar
modo di rimediare a qualche accidente » per esempio: Lo non fa che mi dire, se it
tale vuol far male i fatti suoi, ©

IN farfetto, Veltito leggiermente. Farfetto hoggi intendiamo ogni sorta d'a-
bito leggieri, e difinuolto, che sopr' alla camicia si porta sotto gli altri abiti, co-
me farebbe camiciuola, o giubbone, ec. “

TVTTO una broda di fudore. Tutto molie dal fudore; Sudatissimo per la fati-
'ca del viaggio violento. i

DOBRETT O.Intendiamo una specie di tela di Francia fatta dilino,e bambagia
(che è il cotone filato ), Sidice anche Dob/erro da duplex,perché nel tefferio,e fatto
di doppia orditura, e riempitura. Così. dobb/a © dobbra dissero gli antichi.

BRYCIOLI. Quelle sottili strisce, che il Legnaiolo cava da qualfivoglia legno
lavorandolo con Ia pialla, si dicono bracioli, forse dalla similitudine de' brucioli y
bachi;e da questi si diconocappelié di brucioli quelli, che son composti;ed intesiuti di
stcisce d' un' erba particolare, nello stcflo modo, che si fa con la paglia y alla
similicudine, e larghessa della quale sono ridotte le dette strisce.

e4LLA moda. Cioè alla foggia che usa; la quale era nel tempo, che l'Autore
compote la presente Opera 5 che i cappelli havevano piccola falda. Si che non
tanto per esser di brucioli., quanto per esser piccolo, era poco atro a difendere»
dai acqua.. Si dice alla mods quali all' usanza, che e modo,cioè adesso,Pr, alla moda,

ei AL di petto, Così.chiamiamo volgarmente quell' infermita., che 1 Medici
Ficena Pletritide. PER.








LR Sree Te ga ere ne MRS OA MMeare | ice nared

SECONDO CANTARE: 10g

» PERCHÉ il Prete non ne goda, Cioè per non'morire, € così far che il Pretes
non goda il guadagno della cera del funerale.

QVEI quartieri. lotendi per quelle campagne, per quei contorni. Che per al-
tro noi Fiorentini per guartsere intendiamo una delle quattro'parti, nelle quali ¢
divisa la nostra Città. E guartiere in lingua militare significa Habitazione e dar
quartiere al nimico significa faluargli la vita, e farlo prigione.

1NSACCAVI da pratico. V' entra dentro come se egli,per esservi entrato altre
volte, fapefle la strada, e vi fufle pratico.. Se bene huomo pratico usato nella ma-
ty che e qui, vuol dire huomo favio, e da saper pigliar compenfo in ogni oc-
ea

GELATINA, Vivanda nota fatta per lo più col brodo di carne di porco cot-
ta in aceto', © poi congelato; Ma qui per Ge/atina intende che l'acqua s' andava
congelando sopra il terreno, e fa Gelarina finonimo di Diaccio,come fa D, inf. 32.

PAR la Furfantina, Si ova una specie di Bianti, i quali per muover le per-
fone pie'a far loro elemosina, dopo haver bevuca buona quantità di gencrofo vi-
no,ne i tempi più freddi si diflendono mezzi ignudi nelle strade-più frequentate, €
tremando fingono di morirsi dal fieddo, e questo lor tremare si dice far /a Pur-
fant ina, cio' fare it giuoco-che fanno questi furfanti,ch' e poi paflato in dettato,
che significa,, e comunemente s' intende Tremare.
~ MA meglio, Benitimo., Già mai si trove chi facefle meglio. Quel ma vuol
dir mai; la figura apocope.

'DAA A1 A.E' \o stesso, che Daino specie di capron faluatico.Lat, dama D. Inf. 4.

Sh si farebbe un'cane infra due dame, ec.
STANZA LVI.

Trove fucile,ed esca,¢ legni var}, Così con tutti commodi ae... pari,
Ondun buon fuoco in uncantone accefe, Dopo una lieta, ilcrogiolo si prefe 5:
E in fu due faffi postt per alari, Essendosi a far quivi Siu >:
Sopr'un' altro fedendo i più difhefe. Mentre pioveva, come quei da Prato i

Bloriano: havendo trovato' ia 'quella grotta comodita d' accendere il Fuoco,
P-accefe:, e vis*accomodd a scaldarsi, alpectando che intanto ceflafie la pioggia.
FVCILE. Intendiamo quello strumento d' acciaio, del quale ci serviamo per
battere nella pietra focaia ad esserto di cavarae il fuoco; detto Fucileda fuoco,
quasi fecaio, o facile. Che per difiefi anche Focile.
£SC.A, Quel fango, o sia cuoio corto conciato'col falnitro, che facilmente»





iglia fuoco., e serve per tener sopra alla pictra quando in essa si batte per trarne
i oa 3 dai Latini detta fomes. La qual rete,fot ben per translato fighifica inci-
tamento., © flimolo, che noi pure diciamo fomite, nondimeno era intesa per
ogni cosa facile a pigliare quel fuoco, che Vergilio-appella seers eth-
Sirufa in venis filicis < Si come noi, ancora diciamo:E/caogni forte di cibo d' ani-
mali, pure dab latino £/ca.,. che vuol dir'cibo,'ed incendiamo ancora questa ma-
teria, che e atta a pigliare subito il fuoco, quasi sia il cibo del fuoco; anai a que-

sia non diamo altro nome, che a' ¢/ca, e dicendosi £/ea assolutamente, e senzas 3

Aggiunta, s' intende solamente-questo cuoio cotto, © fungo conciati con falnitro. = ih

“ALAR!, Sono due Ferri 5 o Safi, che si tengono nei focolare', perché man-

i 'tengano folpele le legne, acid che pil facilmente ardano. £' voce ease i
ue


“110  *MALMANTILE

Latino /ares, la qual voce spefle volte era presa per fuoce ».come si può dedutre
da Ovid. 1. faft. 18. 'i % ve

Omnis haber gemings hincsarque bine ianua frontes

E quibus hec Populum /pettar 5 @ ila Larem.

Eda Colum. lib, 11, cap, 1..de Villico.,.Con/uescat rufticus. circa larem Damini,
focumque familiarem femper epularé. 1| Sipontina dice così:) Lares Di erant apud
Gentiles, & colebantur domi, focu/que illis [acer erat, unde vulgus focum focolare ap-
pellat quasi laris focum. Molti in vece di dire 4lare dicon arali, o sia corrotta~
mente, o pure, perché gli piglino da era, intendendo strumenti da mettere in
fu l'altare per foftenere le legne per il fuoco de i facrifizzj, e così fanno che fias
ben detto tanto arali, che alari.; 'oe )

AC, pari. Agiatamente si dice anche 4 pie pari.. Vedisopra;Cant. pr, stan.
82. Lasca Novella 4. lib. 2, Servsti delle buone vivande:y che voi sapere bene acconces
e fragionatg se ne frettero a pic pari, Si dice anche agambe larghe. Vedi.sotto.C. 9.
stan. 32. Ed in miolti altri movi, che tucti maftrano la spenfierata agiateaza duno,

DOP' una liera. Dopo una famma.  Diciamo Aer una fiamma chiara, senza
fumo, e che presto paflia detta diera da setitsa, come anche baldoria,da baldore(cio®
baldanza ) voce antica.. Gli Spagnuoli similmente dicono alegro, un fugco dal
legria. Vedi sopra C. 1, stan. 4.0 fore si dice seta selngeae Gieramente, che ap>
preGo ai nostri Contadini vuol dire presamense, cioè cosa, che passa prestamente.

PIGLIARE il Cregioio. Stagionarsi, Quando son format i bi chieri, ed altri
vali di vetro, gli mettono così caldi in un fornelletto, che a tal fine e sopr' alla
Fornace, dai ord chiamato Camera, dove e un.caldo. moderato, e quivi gli
lasciano flagionare, e freddare a poco.a poce, conducendoli con un ferro alla
bocca del detto Fornello per da batio,dove non si sente più.caldo, il che da edi G
dice dar asempra, temperare 5 o dar il Crogiolo., o Cragiolare. E. di qui parlando
dell' hyomo intendiamo piglare i/ Cregiolo, quando dopo una fiamma egli conti-
nova a fare attorno al fuoco, fino che sia tutto incenerito. E da questo verbo
Crogiolare piglia, o ha l'origine, il Gregivele sche e quel valetto. di terra cot-
ta, il aale serve per mextervi de ' @ liguefare, 0.fondere i metaili nella Fors
nacc,detto corrottamente Corergimalo, is gabe BAN

FAR come queida Pax Proverbio vulgatissimo, che. significa Lasciar piovere;
1 Fopoli della Città di Prato., che ¢duddita, e vicina a. dicei miglia a Fircnzes '
nel- tempo,\che i Fiorentini fisreggevano.a Repubblica, domandarono licenza di
poter fare una Fiera jl di 8, di Settembre, ( la qual Fiera Gi continova fino al pre-
sente in detto giorno ) e per tal' efietto. mandaropo Ambalciadori alli SS. Priort
di libeeta, da 1 quali fa loro conceduca la domandata licenza ».con Neincliag:
pagaticro una certa s di denaro. Accordato ib aegozio gli Amb; a
partirono; Ma ¢flendo nell ulcir del Palazzo, fowvenoe loro, che sein talgior-
no fufie piovuco, non haurebbono potuto far la Fiera, e nondimeno farebbe loro
convenuco pagare il danaro accordato; onde per aflicurar quello punto tornaros
no indictro, cd entrati di nuoyoida i SS. Priori, uno di ch ambafeiadon feng
altre parole disse: Signori, sse ¢'pioveile? Alicheuno.de'Signori iybita eispole. =
Lasciate piovere, E di qui nacque questo proverbio Far come quel da.Prato', che
signitica Lalciar pioveres 1 ' ie 6 Asal Eas

3




Le a et eT et ee ee



=

SECOND'IO CANTARE. =

STANZA LVIL

LZ' Orco fratantocon mike atri,e feorci,

etffacciatofi all nuscio, ch' era aperto,
Prego Florian con quelgrugninda Porci
Tutro quanto di fango ricoperto,

Che ( perch'ella veniva gin con gli orci)
Ricever o voleffe un po ul coperto,
'Ritrovande/i fuora fealzo, e ignudo
A sigran pioggia,e a tempo tosh trudo

TIL
STANZA LVII1,
Hebbel giovane allora un eran contento

Dhaver di nuovo quel bestion veduto,

E favendogli addossa affegnamento,
Luafi in wn pugno già Phaveffe hanuto,
Rispose: Volentieri; entrate drenjo,
Venite, che voi frare il ben vennto,
Che dopo ilfugeir voi Uamite, eit Zielo
Fate a me; ch' ero fol, fernizioaCiels.



Mentre Fioriano flava a fealdarsi; 1' Orco s' affatcio alla bocca della grotta
senz' haver mutata la figura'di Cignale, e pregd Florian, che !o ldsciatle entra-
re; Eiglirisponde, che entriallegramente, e che ne riceve servizio, perché
essendo folo,ha cara un poca di Compagnia.

. Non si maravigli il lettore, che un Cinaate parli; e si ricordi, che e una No-
vella per i Fanciullini, e che queste cose seguivano.
i Al tempo, che volavano + pennati,
: Tutie'le cose fapevan parlare;
Secondo, che dice quel che descrive la guerra di Carnovale con Madonna Que-
fefima « Apill. As.) i2. Parietes locuturos,boues,o id genus pecora dittura presagint.

GRVGNO. S' intende ia faccia del Porco 5 da grannitus, che e lo stridere del
Porco. Grugnino e detto per vezzi, ma qui e ironico, e per derisione Guardate
bela faccettina, o bel grugnino', o bel. grugno, quando yogliamo jitendere una
brutta faccia 5 EB si dice haver i/.gruero,dell* huomo quando e incollera, donde ix-
gragnare por entrar in coliera'. Vedi sotto C. 8. stan. 61. e /erugoni si dicono le
pugha dace net vil. '

ELLA vith gi ton gli orci. Cio' piove Ne di¢a: Ogai goccib-
lad di tanta acqua; quanta rie cade a dar la volta a un' Orcio, che ne sia piend.
Sidice anche Zila viene a bigonce, a catinelle, ec, tutte iperboli per denotare, che
piova gagliardamente. Vedi sotto C. 10. stan, 20.

FALENDOGLI addosso usiegnamento, Disegnando quello, che yoleva far di
aa i# in fito potere Ȣ dominio, come esprime il Potta medesimo di-

: Quafiin ne già l'hawessé haunto, ' vt
BAR i naan fun servizio, 6 favote accettissime, '6 gratidissimo,
STANZA LIX.) STA EX.;
Poi disse > Hor vin venité Alta ficnra..
Rispose ? Oreo: lo non Verrd ne arco,

Credi tu pur ch? io sia così merlotto! Guarda la gamba | perch' ia he panra
Se non glicanfi ci verrd domani, | ~ Di quella lrifeiagh iwrlveggo ut fiico,
5° altro,dice il garzon,non ¢°è di rotto Allor Florian la cintura',
Die pieche te gli vo" legar lontani', Ed impiarth la [pada ott' un banco,
© Eprefo allora il [uo guinziiglio in mano' Diffe'l' Oreo: ( dedutald riporre ¥

© Lagi in un canto T ehero y¢Giordane, “lo ti ringracierei; ma non accor.

STAN-
i ens

112
STANZA LXL
E lasciata la forma di quel verro,

Presa l' antica,e mostruosa facia,
Con due catene salto la di ferro,
E lo lego pel colle, e per le braccia,
Dicendo: C acciatar tu bai pres' erro,
Perché credendo di far predaincaccia,
All fin non bai fart' altroch unavescia,
Ment' il tutto e seguito alla rovescia.

'MALMANTILE



STANZA LXIL

Rimafto ci fei tu, come tu vedi
Senza bisogno haver di teftimoni,
E perché con leurieri, ¢.cami se spiedi
Far me volevi in peri, ed in bocconi;
Coss perch' ella vadia pe' fusi piedé
Faraffi ate, ne leva piit ne pani,
Accio che, procurando IL altrui danno,
Ler te ritrovi il male, ed s1:malanno,

STANZA. LXIIL

Ed io c* hebbi mai sempre un tale scopo
D' accarezzar ognun, benché nimico,
Come la Gatta,quando ha preso il topo,
Che, se ben' e tra lor quell' odio antico,

Scherzandocon esso alquanto,e poco dopo
Te lo sgranocchia come un beccafico y
Così perché piit a fila tu mi metta
Veglio far' io, e poi darti la firetta.

L' Orco alla cortefe ofierta risponde, che ha paura de' cani, e della spada; e»
Floriano lega quelli in un canto, € ripon questa fetto un banco; Allora l'Orco
si scuopre, ed entrato nella caverna prefe Floriano, ed incatenollo.

S/ch? E un termine, de} quale ci serviamo per dimostrare che habbiamo,co-
nosciuto l'inganno, o cattivo trattamento., che alcuno ci habbia fatto, o hab-
bia in animo di farci, quasi dica: Cos} eb vorrefti.ch' iofaceffi? o vero Così mi
tratti eh ?

FATE motto, Proferito col primo,o, stretto.,. Vuol dire ascoltate, sentite..
Fate motto a me; ed usato nella forma che è nel presente luogo,ha forza d'escla-
mazione, e vale per un certo modo di domandar consiglio, quando ci detta una
cosa, che sia imposfibile a farsi, o a credersi, quasi chiamiamo altra gente, che ci
consigli se questa tal cosa sia da farsi, o da.credersi; e che fenta lo sproposito che
cié stato deito. Dird per esempio; Cofui dice che ha trent' anni ye Sono pin di cin-
quanta ch' ¢i nacque; Fate motto! Cio' udite sproposito; O vero giudicate, se»
ciò può essere.: 3 è:

SLA così merlotto. Cink sia così semplice, così minchione., così privo di fenno.

Ci verré domani, Detto ironico, che significa Non ci verro mai. Questo De-
mani e if Domani eterno di quell' Oite, che hayeya seritto sopr' alla sua bottega
Doman si daa credenza,¢ heggi no, Ghe l'hoggi era sempre;,¢ il Domani havea
sempre a venire.Berni 4 rivederci alie Calenae Greche,prcio da Such. in Aug. c. 87.

DYE picche.. Detto indetecminaro, se ben pare determinato,, e significa molto
lontani, e non per appuaro la lunghezza di due picche ma forse aGiai più, e for-
se assai meno. ) 4

GVINZ AGLIO, Si quella corda, o striscia di gor » con che si tengono. i le-
vrieri a lafla;e da molti è preso per ogni force dj legame, derivandolo.dal verbo
latino wincio, come vincafire, ynciglia, ec. ma strettamente guinzagiio »\ 0. vinza-
gis 3 intende folo la corda, o quoio,col qe si tiene al Jevriero alla lalla; se»

ene da qualcuno e inteso ancora per quel Jegame, col quale $\accoppiano in-
fieme i bracchi, o altri cani da caccia, Lat. copula. ':

GVARDA la gamba! 1 Cielo me ne liberi, Ll Cielo mi guardi, che io sia per,
far questo. In Firenze nella Corte della Mercanzia, che ¢il Lee dovefi

anno

¢:






pet
ti

iF



SECONDO CANTARE. 313

fanno ¥ esecuzioni Civili,sono alcuni Donzelli, i quali si chiamano Toccatori.
'Questi dopo che in una causa si son fatti tuces gliatti, e si vuol venire all' esecu-
zione personale, vanno ad avvisare il debitore, che se-egli non paghera in te:mi-
ne:di ventiquattro hore sara condotto in carcere; e fenzatale atto, che si dice»
Toccare,o fare il tocco, non si si può con Cittadini Fiorentini:venire a detta cfe-
cuzione personale. Tali Togcatori anticamente pet esser conosciuti portavano
una calza d'un colore,ed nad' un' altro, onde nel paflare che facevano fra le
Botteghe,¢ peri i noghi pil gepeeass ixagazzi gridavano: Guarda la gamba;
affin che chi era in grado d'esser toccato si porefic fuggire, e guardarsi, non po-
tendo i Toccatori far tale azione ne i luoghi iumuni; e si dice Toccare perch
non serve, che costoro avvisino con la voce il detto debitore, ma devono for-
malmente toccarlo con la mano, E da questo è venuto il modo di dire.
Guarda la gamba; che significa mi guarderd, o fuggira di far tal cosa. [1 Lalli
nell' En. trav. lib, pr. stan. 67. si serve di questo detco ne] medesimo proposito.
Venere allor rispose; Honor Celefie
Guarda la garuba | usurpare io non vegtio,

IMPIATT ARE, Naicondere, e si dice di materiali; € non pare che
suonerebbe bene il dire Impiattare la verita, 4a virtù, ec. Vedi sopra C, 1. stan,
75 +41 Poeta fene servetotto C. 19. stan. 5. parlando dell'Aurora; ma la conside-
ra. come donna 5 ¢: corporea, come si considera il Sole, la Luna »tle Stelle,
delle-quali si dice Lmpiatcarfe 9 o rimpiattarf: dictro a i nugoli, o dietro le monta-
ce ' f oo lei-non fering: che s appiatta,¢ fugge.

dir la Tayola, sopra alla quale si posano le vivande per man-

giare: ae bene:Banco ha: molti altri significati.

0. dO thringraxierei, vanon occorre; Cirimonia che si usa con.chi ci habbia fatto
sun favore a rovescin,.o vero ce l'habbia fatto quando noy occorreva,o quando
hawevamo'gia fattoda per'noi quel-che speravams da lui; o che difua cortesia ci

faccia un Tavares del quale non havevamo bisogno; ed e lo fieflo che dire 4 "ho

smegli orecchi, ee e fimnili 5

40F2 Porco maichio senza castrare. Dal Latino verres.

TV has Pree ero, elaine stete E — hogei poco usato fy che il

t orstm'c
BARE wpa vefeia “og conchiudere Non 'adempire il Yao intento', come.
aoe diquella i Sten fm r Sees mettono nella canna minor
gu richieda, e fearicando poi non ono, e fantio
uno feoppio. ychea pena 6 sente, e tale sopeenen rd Si dite
ancora ve/ia una:specie di Sr B ve/cie dicono le donne un racconto de fatti
' e velciai donna > che ridice tutto quello che fate

cor 2
“* 5:0 Devine pi ov gibi ion “dpgtungere;'¢ non levare. Cio' farai trattato
'cepa eae oe cae E
e 2iath ie Mian ia woo

Tinie, edi walense pais, e io ch' il male.
sos hggigamels diag cot, 'econ ogni cosa; ed i Posta mete

y




114 MALMANTILE;

mo lo dichiara,dicendo: come um beccafico, i quali uccelletti da i pil si mangiano
senza buttar via  ofla.. E /eranocchiare se ben s' usa alle volte ne i cafi come il
presente, non lo trovo usato se non per esprimere il romore:, che fa coi denti in
romper quell' offa colui che le mangia, il qual romore è simile a quello.che fa il
ranocchio quando canta.

HEBBI un certo feopo, Hebbi un certo fine, un certo genio, un certo riguar-
do» La voce scopo vien dal Greco scopos, che tanto appresso a Greci quanto ai
Latini, ed appresso a noi vuol dir Berzaglio, e ggr metafora significa quel fine,
al quale tende, ed e diretta la nostra mente nelle nostre operazioni, per lo pil
in bene; che non flimerei si potefle dire senza riprenfione. Scopo di rubare. Si
dice anche haver mira, il qual termine e per avventura pil generico, dicendosi
haver mira di far bene,¢d haver mira di far male.

METTERE a filo, Bar venir gran voglia, Traslato dal coltello, ed altri ferri
tagiienti, i quali quando sono ben' arruotati ( che si dice meffi in filo, o affilati)
tagliano meglio.

DAR (a Sretta, Vuol dire opprimere uno. Ma qui aia nel suo vero si-
gnificato di stringere, ed intende stringere co i denti, ci i

mangiare.
STANZA LXIV.
Così spogkollo tutto ignudo nato, Lo racchiufe, e lo tenne foggiornato 5
E veduto ch' egli era una fegrenna, Perch' ei facefe un po miglior corenna,
Adeit asciutto,¢ ben condizionato, Però che a guisa pos di mettiloro
Snello, lefto, e leggier com' una penna, Voleva dar di Zanna al suo lavoro,

L' Orco spoglid Floriano per mangiarfelo, e vedutolo così magro risolvé di
'non toctarlo, ma lasciarlo flare tanto che ingraflafle ye poi mangiarfelo. 4

JGNVDO rato, Cioè ignudo, come quando ei nacque. Diciamo così per in-
tender uno, che non habbia in dosso ne pure una minima parte di vestimento, ed
ha la fiefla forza che dire Zenudoignude,, che per la ragione della replica', vuol
dire Ignudidimo, '0 Affatto igaudo.

SEGRENNA. Quella voce, usata per lo pilt dalle donnicciuole, vale per

'esprimere una persona magra,sparuta,¢ di non buon colore, che i Latini, tol-

to dai Greco, dicono Afonogrammus; ed il Poeta medesimo la dichiara dicendo;
Tdeft asciutto, che bxomo asciutto intendiamo huomo magro; ond' io mi credo che
JSegrenna veaga da fegaligno che vuol dire Animale magro edi temperamento non
atto a ingratlare. Diciamo ancora mummia, che sono quei Cadaveri secchi nel
mare.d' Etiopia, o ne i fepolcri dell' Egitto: come vedremo sotto C. 6. stan. 52,
per intendere Huomo foverchiamente magro. Diciamo Segrenna a una donnas
'magra, dispettola, maligna, incontentabile, e che non approva, ne loda: mai
l'operazione @' alerui inv

fs delkg Sid RO NGS THINY'D a: ' 2
BEN condizsonato, Questo termine, se ben pare riempitura del verso.y-0( ¢o-
me diciamo )borra, non € così ma e pure che quando si vuole intender un ma-
gre » habbiamo questo dettato vulgatissimo scintto ye ben condizionato; xolto for-
se da quello che Ernie denne a >
eben condizionata, per-avvisare il Corrispondente della diligenza de) Latore, o
'Condottticro.. vba Sate

SNELLO leitosleggier come wna penna Queste tre yori nel presente luego Sono f-
ri nonimi



ie


SECONDO CANTARE: 4s

nonime significando, ed esprimendo tutte la poca carne che haveva addosso Plo-
riano, e che era al maggior segno magro. & la voce /ve//aha forse origine dal
Tedesco Skye!, che vuol dir Veloce.

ZO tenne foggiornaco. Lo trattava bene di mangiare. Gli faceva buone pele.
Che /oggiornare uno vuol dire Spender il tempovin ben cultodire, governare, eo
riftorare uno con quello che occorra', e s' usa questo termine per lo pil, trattan-
dosi di bestiami, e perciò appropriatamente detto in questo Juogo, perché, se
ben Floriano era huomo, era gpndimeno trattato dali' Orco come beitia da in-
grafface.

F ACESSE miglior coteyna. \ngraflafic. Per intendere uno assai graffo diciamo:
Egli ha buona cotenna; trasiato da i porci, la pelle de i quali si dice propriamenic
cotenna, che dell huomo si dice corenna solamente la pelle del capo 5:0 per di-
sprezzo, e per intendere un' huomo Zotico, che si dice huomo:digrofsa corenna,
o Cotennone, o Coticone,

AAGVIS A di mettiloro, Volea dar di zanna al suo lavoro, Coloro che indorano i
legnami si chiamano Azeri 1 ore, ed in una parola sola Azettilori,, Questi per
brunire, o dar il Iuftro a i loro lavori si servono de identi pil lunghi,0 diciamo
maettre di cane, di lupo,o d' altro animale simile, (i quali denti chiamiamo <az-
ne, o fanne come vedremo sotto C, 7. stan. 54. ):¢ tal lavorare dicono xannare,o
dar di zanna, Ma qui dar di cannas' intende il naturale adoperar de i denti, che &
mangiare; e scherzando con l'equivoco dice che l' Orca voleva dar di anna al
suo lavoro, cioè mangiarsi Floriano, che era il suo lavoroy che egli havea fatto pi:
Bliandolo, ed ingraflandolo.

STANZA. LXV.

STANZA LXV}1.
— Amadigi c andava per diporto

Due volte il giorno almeno a rivedere

E piangendo diceva; O.T ato mio,
Se tu muori, che ver [ata par troppo 5

La fonte, e la mortella., che nell orto S' hava dire anche dimes tele dich iog

Lascio Florian per tante [we preghiere; Ttibus, come difse BP.... Pioppo,
Trovato il cefto spelacchiato,e¢ smorta y Così, senza.dir pure al Padre addio,
. El acque baffe purzolenti,e nere Adonta four' un cavalo, e di galoppo

Qui(dice)Fratel mio noi fiam sul curra

» Diandar a far un balloincapo azzurra,

Vici & Vanano molto ben' armato,
E feca.un cane alany havea fatata.

In questo tempo Amadigi s' accorse dalla fonte, e dalla mortella, che Floria-

cane incantato., and6.a gercar di lui,

he

ny ipa,
a no era in pericolo, € percié montato a cavallo. bene armato, e con un grosso
os

si partend. Spelacchis

SPELACCHIATQ, Pelato in

BP qua,¢ in la, cioz parte delle faglic cascate, ¢
s' intende un' huomo ' che stia male a fanita, ed a roba oe

sia mal vestito per la sua poyerta. 4
soe SMORTO, S' intende che nonha il suo natural colore buono.
'ah  SLA sul curro, Siamo in procinto; fiamo all' ordine; fiamo vicini, Cxrro
yo = son pezzi di quali G metton sotto alle pietre,o ad altre cose gravi per
rm facilitargli il moto quando si frascicano, dai Latin detti Palange.
rey EAR un ballo ix ¢

' ' azzurro. Vuol dire Esser' impiceato; perché campo az.

i Rurros' intende il campo, che fa l'aria, il quale e azzurro, e colui, che e im-

0 f j 'Piscato movendo le gambe, pare she palit in aria, Per maggiore inteliigen2a la
dais 2

voce
1


|



116 MALMANTILE

voce campo pieeeluiemencients 5 wuol dire quel luogo, che avanza in. uns
quadro fuori delle figure, ed-altra che. vi fia'dipinto » come si dice una insegna»:
entrovi un lione in campo azzurro. Ed i medesimi Pittori ne cavano' il verbo
campire, ché vuol dire Dare il colore, de) quale:ha da essere il campo.

7 ATO. Vuol di Fratello, B' parola usata dalle Balie per insegnar parlare a i
Bambini, come Habbo.in vece di Padre, Mamma, Bombo, e simili, che per ef
fer parole labiali tornano più facili a proferirsi. Furono usate anche dai Latini
come si vedein Marz.lib, £. 95.

Aammas, atque tatas habet Aphra, fea ipa tatarnm
Diti, & mammarnm maxima mamma poteft.

Vedi sotto C. 3. stan. rz., e C. 4. stan. 5.

TK lodich'io. Vale per Te logiuro; Ti afficuro., Vedi Oraz. lib: 2, Ode 17.
dove parlando con Mecenate infermo, dice:

Ab te mex si partem anima rapit
AMatiirior vis, quid moror alters?

Con quel.che segue simile al presente lamento, che fa Amadigi per if Fratello,
che: Orazio fayper Mecenate:.

1T [BV S-come disse P..«. Pioppo, Significa sha dire anche dime: gli mor-
to. Questo P..... Pioppo.era 'uno, che havea poca amicizia con Prisciano., e
non oftaote sempre slatinava', e fra l'altre quando voleva dire i) tale e morto di-
ceva fibas, © intendeva Egli ¢ito. EB da questo fuordetto diciamo Come disse
P., 2. %Pioppo, B-s*intende il tale & morto,. ost th

Dik' addio, Intendiamo quel faluto,, che si fa nel pigliar congedo,o licenziarsi
da uno., ed e lo stesso, cite i Latino Yale, usato da noi ancora come dicemmo
sopra, e vedremo sotto-C, 6ttan. 18. ¥ '

GALOPPO:, Corso divcavalio,ida i Latinhdetto'earfus gradarixs, che & in
mezzo tra il trottare, e il correre. Forse meglio gualoppe secondo Dante Inf.

Cant, 22, Bs
di rintoppo
A gli altri disse a lui, se tu ts cale e
lo non tiverro dietro di guatoppo,
CANE eAllano, Cane grosso per caccia da Cignali,e simili animali feroci, ed &
maggiore, pil fiero, e pir gagliardo del Mattino. re
STANZA LKV



It STANZA oe
E cavalcando con la guida, e foorta L'apparir a! Amadigi agti abit
Dil sue fadeie,eul hicescars lame j Raddoler? agro dei lor mefti- vif
~ Chinnanzi gli facea per la più corta. Che per la somiglianza atucti quanti
La fhrada per lo monte, e per lo piano; iapahio dapenaion @ Campi Elifi,
A Campi giunfe:, dove in su la porta o 2 mance, © paraguantt
rapa Leggenidi, Stan's ' davon moltva darne al Re gli avvisy 4

- Che perchi fucreduta dwognuno, —§ ——-—' Alrrs alia figlia, ed ambi a quest
KraleCeveyovthiie Cohpoabionss oPRercia Cae è
Amadigi ee » dove dal bruno, che vedde addosso' a gli abitatori

conobbe, che era mortoiitlor Principe; subito che costoro veddero Amadigi,

credettero ch' i fulle Florianos e'peccid molti corsero a darne avyviloal Re, ¢

a Doralice. oie ERA








SECON DOCANTARE. fry

ERA laCorte; e ratroC ampica bruno. Cioè i Cortigiani, e gli abitanti di Cam-

i erano velliti di nero in: segno di mestizia, per la morte del Re Floriana. Pecr.
4a; E.vedrai nella morte de eAMariti
Tutte vefiite a brun le donne Perfe

© Da aleuni'fi dice wefire #turto y o a feorrucciv.\ Ma credo che essi habbiano ac-
¢actate queste voci da i moderni Romani. t

AGRO dei lor mefti vifi. Viforagro vuol dir Malinconico; @ si dice agro perch
ung, che habbia hauato qualche disgutto; fuob mostrarlo nelia faccia con incre:
spav la fronte, e fare:altri gefti appunto come fa uno, che mangi cose aspre- 5
acide,oagre. E però dice Raddoler ? agro dei lor mefti vist, che significadi me-
lancolici, gli fece ricornare-allegri:. Ad

CREDIT O «i Campi Elifi, Creduto nellvaltro mondo:;¢reduto morto., che
eens Eiifi dalla superftiziofa Geatilita erano creduti-ii Paradifo.. Vedi sotto

. 6. tam. 32. '

PARAGYANTO, Mancia,o regalo. Paraguanto, dono, iregale, mancia ap-
pretio dinoi si possono dir finonimi; E se bene molti vogliono-che: manvia ye pa-
raguanto si dica quello, che dal Superiore si da all”inferiore; © donoieregalo G
dica quello, che dal' inferiore si da al superiore (che-in-questo caso now si dircb-
be mancia ) o dali'uguale, all' uguale, nondimeno nel buon parlar familiare si pi-
glia uno per l'altro, nes osserva tanta strettezza, ed il nostro Poeta pure si
vede nel presente luego, che non oficrua questa distinzione come poco, o punto

c Orin ta STANZA LHIX
Doralice brittande a tai-sovelle Enon fear pil nella pelle
<A rinfronzirsi ardoffene allo specchio, Salts fuor dipatarre innanzi al vecchio,

Sb mefse il grembinl bianco le pianche Ed invontro correnda sil: sia cognate;

UI veRxo ab collo,e i ciondoli all'erecchio, Ecco Florian ( dicea ) rifucirato,

Dordlice sentieaquesta nuova si raffazzond, e subito-corse incontro al suo co-
gnato: Amadigi, credendolo Floriano suo marito'.

BRILL ANDÒ. Giubbilando.; Brille si dice uno che sia allegro per haver beuuto
molto vino. Vedi sotto C. 6. stan. 35. ed è il primo grado di briaco dicendosi in
agugumento Brivo scorto, briaco, spolpato, Molti voghiono,, che questa voce brilla~
re venga da'biril/è tpecie di gioia', e che brillare significhi scintillando-tremolare,
appunto come fa il biril/e, e come fanno coloro, che sono fonmmamente allegri,
©che habbiano foverchiamente beuuto.:

RINF RONZIRSI, Ratfazzonarsi, abbellirsi, aggiufarsi la persona tolto dal
Latino refrondefeere, che vuol dir quando gli alberi si vestono disnuove frondi,
le quali nell' antico Fior, si dicevano fronze. Terenz. in Heaur.

: —— Et noffi mores mulierum; s

¥

> % Dammolinntur,@ comunrur, annus off,
+ Cioè si rinfronziscono f dice l'espositore Landino js" accomodano, ed accon-
iano la oom ee accu ae og Hh

~ CIONDOLI al? io, ini. le gioi portano
“denti all' orecchic, Latino Zaawres Bis aor cama pendent per cere

ciondoli.






118 MALMANTIEE

VEZZO. Quell'ornamento di gioie, che le Donne portano alicollo

PIANELLE, Specie di scarpa, che cuopre solamente la parte dinanai del pie+
ce,da i Latini dette fandalia, E,con dette gioic adornandola,mostra il Poeta qua-
le possa essere una Regina di Campi', che non eccede il Inflo d' una pulita con-
tadina de i Contorni di Firenze.

NON pus frar nella pele. Non ped aspettare, perché l'allegrezza le ha:cagio-
nata una inquictudine tale, quale /ogliono havere tutti coloro, che dovendo con-
seguir qualcofa di lor guflo, ogni kbra d' indugio. stimano mille. A questo
si può applicare quell' 4 Sermento torus est de i Latini, che pare che espri-
ma quella inquictudine, che suol cagionare l''ira; Lasca Novella 5. Si che per la
paffione, e per la rabbia non poteva [rar nelle cuoia,

COGN-ATO. | Latini per cognazione intendevano ogni sorta di parentela.Ma
noi per cognaro intendiamo un Fratello di noflra moglie youn marito d! una fo-
rella di nostra mogli¢, o un marito di notira Sorella, e nello stesso modo respet=
tive il Fratello del marito, si dice cognato, come intende nel presente luogo s

INNANZI al vecchio, Cioè prima che ulcifie di casa il Re suo padre, inten-
dendosi comunemente Padre quando in questi termini si dice il sage



talvolta il Padre sia giovane,
STANZA LXX.

Noi vi facevam morto; o gindicate
Selacarotac era fata fital
Pur noi ci rallegriam, che voi tornate
A confolar la vostra gent' afflitta,
Domandar non vcorre come state 5
Perthé v' havete buona soprascritta,
E fiate graffo, e tondo com xn parco
Per le carezxe fattevi dail' Orca,

STANZA LXXL

AM immagine così perch' io non vera:
Tu fat com' ell' andò, che fufti in caso,
So ben, che mi dirai, che non fu vero
Ma la bugia ti corre fu pel nafo,
Hor basta, Tx ritorni fano, e intero,
(C' a pezzi tu dovevi esser rimafo )
Per la Dio grazia,¢ sua particolare,
Perché tel' ha voiuta risparmiare,

STANZA

Mio padre te lo disse fuor de denti,
Ed io pur te lo aiffi a buona cera
Lon una volta, ma diciotto, o-venti
Che l' Orce ti faria quatche billera;

io,ancor che

STANZA LXXII,
Dunque s ei fa così gli è neceffaria,
Gh'ei non sia la quelfurboch'unlotiene,
Anzi tutto iLrevescie, ed il contrario
Mentr' egli tratta i foreftier si bene g
(Ed io, che già havea sul calendario,
Gli voglioinquato.ametuttoil miobene,
Perch'ei non t ingoio; Se ben da wn lato
Ti flava bene, havendola cereato,
STANZA LXXIIL
Cast nel mezzo a tutta la pancaccia y
Ch'é quivi corsa,e forma un giro tonde,
La sua caponeria gli batta in facia,
E quel ch'ei ne cavo po poi ingquelfonde
Già che (dicea.) con l'andar' a caccia
Ai disperto.di tutto quanto il mondo
Cavafi,fenra far alcun guadagna
Die occhi ate,per trarne una alcopagno,
LXXIy.
Ma tu volefti fare a gli feredenti,
Perché te ye firuggei come la cera y
E quasi un rischio tal fuffe una lappola
Voleffi andaxxi, e defti nella trappola.

» In queste cingue ottave mostra,, che Doralice ingannata dalla somiglianza,

che haveva Amadigi con Floriano,gli fa un dicorso di congratulazione mefeola-
ta con rimproveri, col quale il Poeta esprime assai bene il costume delle nostre»
Eemmine in simili caf; tacendo che.dal principio del discorso, che e la congra-

tula-




SECONDO CANTARE., 119

tulazione, lo tratti del Voi, e quando viene a' rimproveri lo tratti del Tu.
SE La carotac' era spata fitta. Ficcar carote vuol dire quand' uno inveatando
qualche novella, o trovato,lo racconta poi per non suo,, acciò che pil agevol-
mente gli sia creduto; fiche Doralice vuol dire; guardates' ella c' era stata data
a credere. Vedi sotto Can. 6. stan, 67. e 68. Mattio Franzefi nel Capitolo sopr'
alla Corte dice:
: 'Chiama piantar carote il popolacciv
Quel che diciamo: Adossrar nero per bianco
Per distrigarsi da quaiunque impaccio
E per tutto il medesimo Capitolo discorrendo sopra questo detto, mostra che
thabbiamo anche iliverbo Carerare 5 e Carotiere, quello che ficca carote. LU Lalli
En. Tr. lib. 2. stan. 2.
Egli che ben conobbe al primo tratto
Ch! era in un campo da piantar carote
Si dice Piantar carote, perché questa pianta fa grossa radice, ecresce assai nei
terreni dolci, e teneri, ed uno facile a credere si dice Homo dolce ye tenero.:
VOL havete buona sopraferitra, La faccia suol' esser dimostratrice delle paffiont
interne, e però dicendosi haver buona sopra/critta's' intende haxer biota fanitd,co-
me dichiara il Poeta medesimo dicendo; Von occorre domandarni come voi feate, per
whi ficondsce dalla buona soprascritra, cioè la fembianza, la buona cera, ¢d aria.
del. volto ci dice, che vai state bene. E cosila voce sopra/critra, che vuol dires
Inscrizione, che si fa alle lettere, ci serve per intender quanto sopra s'é dot-

to.
LA bugis vi corre fu pel nafo. Tu daicolore. Tu timuti-dicoloré in vifo, per-
ché tu hai -detto una falfita, Twi oculi declarant, Lo Scoliafte di Teocrito spic-
gando-quei versi dell' Iditio.12. che in Latino furono così tradotei: Verim ego te
« laudans yformofe; baud mentiar umquam, Nec tenni gravis innascetur puftulanari;
sdice così,.Vuol dire; che tiekdodarti, io non mentird,. non mi nascera sopra.,
al nafo la bugia; poiché alcuni sogliono chiamare certe bollicine bianche, che»

vengono fu pel nafo', bugie: c\colui che leaveva, era natato, come bugiardo.
'Fin qui lo Scoliafte.

RISPARMIARE 4 0'ri[pinrmare, Vale:per petdonare.. Quis intende 1' Orco
~che non ha voluto far male alcuno.. '
HAVER uno sul.calendario, Havere a noia, o'vero odiar' uno.
QUANT O w me gli vo turto il miobene.. Pee quanto s' aspetta ame gli porto
tutto quell' affetto, che si può portare; l'amo di tutto cuore.
TI fava bene. E' \o stesso che Ti flava il dovere.. Tornava bene,-che 1' Orco
» t havefle ingoiato, perché t' haverebbe fatto quello che tu meritavi.
PANC.ACCIA, Così si chiama da noi quel luogo dove si ragunanoi novelli-
si per darfirle nuove PunsVaitroyed ha questo nome di Pancaecia, perché nel tem-
di Rate questi rali si radunavano già per sentire il frescowicino alla Chielas
» Cattedrale, fedendo sopra aun muricciualo coperto di tavoloni, o panconi', e>
«da questi prefeil nome di-Puncaccia. Eda questa pancaccia, Pancaccieriye Pancac-
vai intendiamo quei perdigiorni., che stanno oziofamente ragionando de i -fatti
daltci, ed in questo senso e preso nel presente luogo, che dicendo quei della pan-

ALLIS y


120 MALMANTILE

caccia, intende una quantità di questi Crocchioni.. Vedi sotto C.6: stan. 69. Can-
ti Carnascialeschi,.@hi.vxol udir bugie, o movellacce Venga ascolar coffuro; sare
Sf fian wntia it dd fs te pancacce,

GOLA butrarin faccia La sua caponeria., Gli rimprovera la sua oftinazione »

VEL ch' e ne cavd po poi in quel fonda, Quel ch' ei guadagnd, pert ona
fine delle fini, o in ultimo degli ultimi. Tanto servirebbe dir po: senz' aggiu-
gnervi ix quel fonda, ma così & il nostro costume in simili cafi per dar maggior
emfafi, quasi dica una fine più la delle fini, Vedi forgo C. 8. fan. 51.

CAV-AR due occhi a te per trarne uno al compagno,, Detto vulgatitiimo, che ci
serve per esprimere Far 4 se molto male, per farne  packissime al nimica,

FVOR de' denti,, Apertamente; chiaramente¢ il Lat. Eoqui, ed &il contrario
di parlar fra denti, o a mezza bocca, che significa non si laiciaré intendere, for-
se e il Atuffitare de i Latini.

et BVONA cera. Con allegea faccia; cioè non sopraffatto da collera,o altra
paflione, ma con apimo ripolato; diciamo anche sul fodo, sul ferie tolto par
Serio, admonere. li Lalli Eo. Te. C. 4. stan. 103.

Prega y Seonginra, e dille a buona cera,

AILLERA, Burla-nociva,o snon cattiva del tuto, aimeno che non piace;
voce corrotta da Wil/era voce antica che vuol dic Villania,, «2%

TE WE frnegei come ta cera. liverbo fruggersi, che vuol dine Ligutart serve a noi per farsi dered' uno che ard ae » Ue Lali
En, Tr. C,.4, fan, 109.-disse. 5 i d¢ Motsiss

Che se ne firugge come le candele.

LAPPOLA, Cofada non timarf. L'erba da nostricontadini'chiamata Lap.
pola fa un feme pieno d'acute spine, ma fragili; EB però dicendosi: nom do stime tna
Lappola,s' intende non lo stimo punto,¢ s' usa per.lo più trattandosi divbravura,
¢valore, alludendo a quell' armatura di spine y chehala La » le qualifes
pon son saraee 89 aeety Sananne. se belesemme te ag

agilissime..

DEST I nella Trappola. V' incappatti, Vi rimaneRti gee Te dnquewtn inci
si. Trappoja intendiamo ogni forte d' antifizio, che si trova per pigiiare ani.
mali tanto di terra, quanto d' arias ed? acqua, donde 7'rappalare valyper Ingan-
nare. Ma 7 rappo/a strettamente presa s' intende un' aria per per eerste
ed una esicdia rete da pescare ha ae di 7



Su egsd Tr. da.quaterini y Pe Invenzioni:per e fare i
ANZA TXKV a STANZA L XVL

ai eee 97 Ma perch' ei non credea veder mai l'hera
E “as il fordo ad: ogni suo quefice »  DY haver il.suo Fratello a faluamento y
da fiben' attingea.da questecofe >... Daun ganghere a tuctiy.e rorua fuora
price 4 Florian porea eer seguita ys -Dietra al sua can veloce'come it vento;

immaginandosi es apposey  Neera un trar di mano andacaancera

vicemne Atoglic, ci [uo Mariea, ed caccia al! Orcoch' ei vi dertedrento
Bch' egli essendo tutto lui maniate.....  «Come il Fratel vedendo un bebsignale,
Fulje pel suo Fratel dacgnuncambiato, si se wale quante lui dolce di fale.

ab b>

STAN-

“RMS i ie aa ae

q


yh
ity



—

SECONDO CANTARE: — 120
~~ ASTANZA LXXVIL STANZA LXXVIII.
Che segnitollo anch' ei per quelle strade EiquandotOrco poi venue anc!ia lui
Dond' ei conducel huomo allafuatana, et dar parole com quei tempi firani,
Ove rmentre diluvia,edal Ciel cade > 'Bd \alluseio fatea Pin da Montus
OB broda ye cect, Criftianello intana., Affin chet' arme  ¢4 caniegli allonrani
OcEd-egli tanto pai lo persuade Eidiffe: Suipiccin pighacald,
oCw ef lege ivcani:, e pif durlindana, E chiapparata [pada con dueimani
Stavende havuto innanzi la lezione, Si lancto fara, equivi a più non peffe
Si fvetcefempre mai fodo al macchione. Gli comiacio aimenarile man pel defo.
; a S THAW ZA LXXIX.
E mentre chor di punta, ed hor di tagtio \\ ~ Tal-che tatto forato come nn vaglio
Di gran finefire fa, di Linghe Witee.; Ut power: Orco\al fin cade,  bafifee,
Pike preftc che non va firale aberzactio Edheraquelte raps, e quelle: macchie
~\Mbcan savicuta anch' eg lize ribadisce; Rimafe a far banchetro alle Cornacchie,

Amadigi/argumentd dal discorso di Doralice', che-ella fulle Mogli¢di Floria-
nO.,,.¢ compre/o quanto poteva esser' avyenvto al medesimo; e però senza dar al-
tra' risposta dette addietro, ed uscico di Campi; fu'dal Cane guidato alla tana
dell' Orco |, il.quaic.fa da Juiicon ajuto' del suo cane', ammazzato.

44.4f, Questo avvcrbio che significa In alcun tempo serve anche per negativa,
come & nel prefenicJuogo 5 e come l'usd pity volte il Boccaccio ed'in specie Nov,

. Mai frate il Dinvel tivcirecay ccs B Nov. 54. Che mai ad animo-riposato si fareb~
gee witrevare yo Nov, 77. Adai di ciò che hora mi parti dubitai, Matteo Villani
lib, 8. cap. 39. / Perugini mas si vollero dichiarare, ed in molti altri luoghi de) Boc-

caccio, del Paflavanci 5 e d' altri Scrittori del buon secolo si trova usato per ne-
gativa. Ho voluto dir ciò in questo luogo per toccare'la difefa dell' Autore dalla
critica datagli d' haver usato questa voce 44si per negativa senza l”aggiunta della
particella we, © om, e senza correlazione alla negativa anteposta nel medesimo
periodo:, e che tanto vale il dire /o now farò mai questo, quanto il dire. lo nai fas
xo.questo, E.mi rimetto all' uso yed al TORTO,, E D/R/TTO del P. Bartoli,per
la difefa:di 88 a ee ai
i sFECE it fordo,..Finfe di-non sentire. s
WAT TINGEA da quefie cose. MW verbo attingere o attignere, che & il Latin'
me eens un Juogo 5 0.4 un fine; Azeram attingere: da noi € preloy
ed usato come il verbo aurio, che vuol dir Cavar V'aequa da i pozzi, che noi di-
ciamo attignere y ed in significato di Comprendere, vedere', udire, ocults © auribus
haarire « EB nel signi di cen € preso nel presente luogo.
8' APPOSE. Verborncutro che-val per indovinare: Ed attivo vuol dire Dar
la i Bund « Jom apposi di chi baveva fatto il male, e però lappost a lai. \o
a ist oe havea fatto il male, € pero ne diedi las
adei.> onc aa! o5 Yb chsh ee Y
 TVYTTO lui maniato, Come lui per appunto: Similissimo a ui: Fatto a cay
pelle, che vedemmo sopra in questo C, stan. 19. Lacy Nov. 7. dice: M1 qual fan-
taccio veststo de' panns del Pedacoga,tutto maniato parea Ini, lo credo che sia parola

Sorrotta da mizvaro cioè diligentemente dipinto, o forse corrottamente derivato
dai Latino barbaro Emanaeus, tanto simile a lui, che pare emanatus ab illo,
3 2 NON



ae ak ee
se MALMANTILE

NON credea di veder mai ? hora. Amadigi havea così gran desiderio'di vedere
il svo Fratello libero, che dubitava non fade per arrivar mai quell' hora, ed ogni'
momento, gli pareva un' anno, 2
' un ganghero, Da volta addietro. Ganghero diciamo uno st per
uso d' affibbiar le velti, fatto di filo di ferro, o d' altro metallo, il quale € fatto
in forma d' uncino, eda quella rivolta, che egli fa, dare il ganghero intendiamo
tornar indietro. 'Retrorfum vela dare. Dare il Banghero » diciamo quando la lepre
fuggendo avanti al cane, torna indietro, e lascia correr il cane, che portato
dalla velocita non si può ritenere, e yoltarsi subito come fa efla, che in tanto pi-
glia campo in maniera ch'.ella (Campa, dal che diciamo Far lepre veechia per in-
tender tornat indietro. Vedi sorto C. 10. stan. 23. '
NON fu si doice di fale. Non fusi credulo: Si minchione: Si sciocco. Viaas
vivanda poco falata si dice dulce di fale, cioè sciocca. Donde esser senza fale, 0
non haver fale in zucea vuol dire Huomo sciocco, senza giudizio, senza ceruel-
lo. Safechiamiamo IL arguzie, e detti ingegnofi. Vedi otto C, 8. stan. 26. Di-
ciamo if tate e dolce,¢ senza l'aggiunta di fale intendiamo è corrivo, creduloy
minchione, e senza giudizio; e per coprire più questo detto, usano molti dire»
Lupinaio (che vuol dir colui che vendendo per Firenze Lupini va gridando dolcé
dolci ) per intendere Cofui e dolce, Qui dunque vuol dire, che Amadigi non fu
corrivo quantojera stato il Fratello a credere all' Orco. Boce.Gior. 4. ns 2, A444.
donna Zucca al vento, la quale era anzi che nd a. dolce di fale, Latca Nov. 2,
Experché egli era nato in Domenica, non fendo la gabelia det fale aperta§ senne fempres
molto bene del dolce.
TANA, Caverna, grotta, buca. Donde intanare, entrar nella tana,
BRODA,ececi. lntendi acqua ye gragouola. Fu un ragazzo ghiotto*delle
civaic, per il quale fo padre ( per mortificare geen sua gola ) ordind, che nel-
la sua feodella non si metteffe altro, che il puro brodo de'ceci, o d'altre civaie
respettivamente, onde il peer ragazzo vedendo gli altri con le fscodelle piene»
di legumi si disperava,, Ed essendofene andato un giorno in camera mentre pio-
veva se ne fava alla finettra gridando acqua, e gragnuola, € questo per la ia,
che haveva, che si stagionatlero i legumi per gli altri, enon per lui. Senti il
padre questo uo gridare, che gli disse: perché preghi il Cielo a mandar Ia grandi-
ne, cosa tanto nociva? L'aftuto ragazzo per scampar la furia subito rispose:
Padre mio io non ho mai desiderato, o pregato male per nefluno, e se io prega+
vo che insieme con l'acqua veniffe anche della grandine, ho voluto intendere, che
il Ciclo vi metteffe una volta in testa di farmi dare con tanta broda una voltas
anche de'\ceci, che dixquesti intendevo quando dicevo gragnuola. Ii Padre rife
dell' aftuzia 5. dette ordine, che per 1' avvenire fuffe trattato., come gli altri.
E da questo,intendiamo ai » e gragnuola, quando diciamo broda, e cect.
CRISTIANELLO. E' detto d' avvilimento, e significa Huomo dappoco,0 di
'a fortuna 5 © di piccola figura; che i Latini dicono bomuncio, e nob talvolta
in questo senso diciamo Homicciuolo. ae
DIRIND-ANA, Intende la spada,e piglia questa denominazione dalla famo-
sa spada d' Orlando Pajadino, la quale da i Poeti hebbe il nome di Durlindana,
0 Duyindana y '
W #iA-












SECONDO CANTARE, 123
4 HAVENDO havuto innanzi la lezione, Essendo stato prima informato; ayvila-
t0,, instruito.: Cioè havendo comprefo dal discorso di Daralice-, che questo era
quell' Orco, che ingannava. 3

STAR fodo al Macchione. Intendiamo non condescendcre alle richieRe, o'non
Gi lasciar lufingare dall' esortazioni di alcuno. Questo detto viene da quegli-uc-
celletti, che stanno per le ma¢chie, dove si tendono le ragne, i quali, per essere
Mati altre volte molefati, hanno imparato, che quello scacciargii col battere la
macchia era di lor poco danno stando fermi, però ncn si muovono a ogni romo-
re,¢ questi si dicono Par fodi al Mactbione, Di tali uccelli si dice anche accivereati,
Veditotto C, 9, stan. 22,

FACEA Pin da Montui, Cioè facea capolino, che vuol dir quel che accen-
sd sopra C, 1, stan. 7. Questo detto viene da una canzonetta, o villanelia,
che dice. A

Pin da Montui, Fa capolino
Dreto è Menyhino, E Mon con lus, ec,
Plauto disse Ex infidijs clanculum ancupari.

SP-piccino. B* modo di incitare il cane contro a uno, El irritare, oimmittere
de i Latini, che noi diciamo anche ammetcere. Vedi sotto C, 11. stan. 29. si di-
ee anche ai/sare verbo originato da quel fyono, che fa la voce dicendosi: /u /a; 0
dalla parola Ra voce antica » che vuol dire Ira, dalla quale habbiamo il verbo
aizzare 50 adizzare, o aiffare, Dan, Inf. C.27,

Dicendo, fa ten va: più non? aizxo,

A PI non posa. Con ogni maggior potere; Quasi dica con animo di seguita-
rea far quella tal cosa fino ache non sara stanco, € non possa pil.

MEN AR le man pel deffo, Adoperar le mani nella persona d' uno, cioè-Per-
quoterlo. La voce dof dal Latino dor/um, da aoi s' intende per tuto il torfo
dell' huomo, ches' eccettuino da molti il capo, le braccia, e le gambe,
Lasca lib. 1. Nov..7. Non contento di ricercargli col bastene le braccia y e le gambes',
volle ancora con esso ritrovargli tutto il dosso,

GRAN fineltre, e lunghe frrifee. Gran ferite di punta edi taglio Punitim, C-
tefim, disse Vegezio. Dice frri/ce per la similitudine che ha una hunga ferita di
taglio con la striscia, e lo fa per esprimere che eran ben lunghe » come dice fize-

requelle di punta ppctche s'intenda, che eran larghe. 3
CAVVENT ARS!, Spingerti, gettarsi,o andar velocemente » o con impeto
alla volta d* uno, che i Latini dicono irruere.

R16 ADIRE, RibattereQuando si mette un chiodo dentro a una tavola,¢ che la
punta di cflo chiodo pafla dall' altra parte,la detta punta si piega, e si riconficcas
perce il chiodo facia V effetto d' una Jegawura; e per far Qacfio, uno baie in

u la punta del chiodo 5 eV' altro tiene a riscontro in sul capo'delchiodo un fer~
£0 5 € questo si dice ribadire; e però Amadigi da una parte, e il ca~
ne mordendo dallt altra l'Autore per esprimer questo atto si serve del verbo yba-
dive usato da molti ed in questi termini, ed anche per licare,

FORATO come un vaglio. Havevano fatto nella persona dell' Orco più buchi,
€ tagli che non ha un vaglio, firu le si fepara il grano dall' immon-

: in vaglio, col
dizi¢, decto dal Latino Yannns + etal Cavell dal Latin Crsiram ye rie
2 lum




¥G



124 MALMANTILE

Jum, voce usata dall' Agricoltore Palladio:, Questa-comparazione' era: usitaan.
che da i Latini trovandosi in Plauto Carnificum cribrumspariando diua fetvo, che
era Nato mal concio dalle bastonate.

BASISCE. Muore. Questo verbovha forfel' origine dalla Greca voce Bafis =
che vuol dire incefxs, e che intendiamoiil rale se n' andé, peril tale mori, che di-
ciamo basi; vedi.l'Orcava 82, seguente,¢ da questo verbo deriva la voce basto, che
vuol dir huomo senza sentimento.,.¢ quasi morto. Meller Gio; della CaQnel

Capitolo del Martello:d' Amore 'dice.

Perché ti guardi torto'la Signora'; t
Parti haver le budelia in un catino,
E doventi bafiro alloraallora.

Vedi sotto C. 6. fan. 97.
STANZA LXXX

Amadigi dipoi fece pulito,

Perché trovato havendoil suo Fratello

Con una barba lunge da Romito,

E più lordo, epiit unto d'un panello,

Lavatolo, e rimeffugli il vestito,

Ch' era ancor quivituttoin unfardello,

Lo ricondufse a Campi, ove la Adoglie

Di lui già pregna,appunto waren lee le doglie.

Ata presto come luiporrai dir mio.
Hor senti pur: Bafite Periine
eAnco eAmadizi subito tuo Ziv

Vine ator donaye n' hebbe ut bel garzone,

NZA

STANZA LXXXI
Corse la Levatrice, ed in essetro
\Framille boime,fe' soldi,e doglien hora,
Partorigli una bella pi/cialletto
Che fufti tu, poi'derta Celidora,
E maritaca al Re, come s' e detto',
Di Maimantil del qual tufei Signora;
Ne fei, e ne farai, io lo raffibbio,
'Sebennen puoi per hor dircomeitmibbio,
LXXX rs:
Che Baldo fu chiamato je quel fon' io,
Che poi crefeiuto detto son Baldone '
Hor eccotisdal primoval terzo grade
Narrato tutto il nostro parentado.

Amadigi trovato: il Fratello Floriano:lo: rivesth; e lo-riconduffe a Canglt dove
Doralice partori Celidora; e d' Amadigi nacque Baldone.. E. con. terminare il
racconto,termina il Poeta il secondo Cantare's

FECE pulizo, Feceil negozio aggiuftatamente, come: andavSoihadls

BARBA da romita, Barba lunga,¢ incalta:, che-tale perio pu eer

barba:de i Romiti.,

LORDO. Sudicio Ihifo. Dal tno Lari sche wiol dir Livido 2 quali' 'per

lorum cuffum, © livid E

Be P
mo, ma ancora ad ogni più 7o stcumento,sopra il q
chiamiamo un:

PANELLO,. Così

nati da iwenti,a! squetti refiftano,
flor. lib. eee

parturiente da i Latini'd
HOLME. Voce' che esprime:
diceyano bei mibi 'ye noi forte

'habbiamardal-Greco hoi moi
SG [aldi 5 e dogtion' bara pot. aS 2 burlare chi'taiuolew: oe

E... Raccogliteice;. sams ooh, ries! la. Creatura dalla



'alPhuo-
quale sia schifezza.

viluppo'di cenci intinti-nell' olio'; fego i °
altra materia oleacea, ¢: bituminofa. gail.
narie in occasione webises iene

tron Tome ergy
eminent?, e domi-
hin i val lo stesso. Vatchi

-difetto'd? clio'.



'animo. ype di a che i
E quell'








SECONDO CANTARE. 125

firathihiirica, © farlezzj senza cagione yo per dolori leggieri, che noivdiciam>,

Fareil monelio', enone riempituca intentata dal/Poeta, ma è pur cos! in uso, di-
cendosi a questo tale: O pover' huomo! dime | fei soldi, e dogliene bora; e si no~
shimano Toth di monete.per haver occatione di dire dagtiene, che & il verbo
dare, ed in questa occasione si dice, perché ha similitudine con la voce doglia.

PISCIALLEFTO } Wnabathbina', Quando una donne partorisce una Fem-
mina, nitda diguelle donne che sono attorno alla. parturlenterlé yuo) dar las
nuova, che ella sia femmina, ma perché pure al fine ella lo deve sapere, per non
profferire la parola femmina dicono: Vaaipiscialletro;-Vnacome'me ye finili, E
da questo' noi habbiamo far” un bambina, che vuol dir Fare un' errore.

LO rafibbio. Lo replico.

NON puoi dir come it nibbio, Cioè non puoi dir Mio. Il Nibbio uccello rapace
non fa altro canto, ne si sente da lui altra voce, che un certo filchio gio strido 5
sche par che suoni mio mio, e da questo per avventura i Latini lo diconAdiluus,o1i

gauolieA4ilano, e i Francesi Azilan; E noi da questa sua voce volendo espri-
mere, che una cosa fia' veramente mia,dichiamo: Posso dire come il nibbia,cioè Mio;
l'autoré lo dichiara nel primo verso dell' ottava seguente dicendo > eta prefte
come lui potrai dir mio,

BASITO, Vedil ottava 79, antecedente.

Z/O. Fratello del padre, o della madre, o-matito'd' una sorella del padre, 0
della madre: 'Quié fratello del. padre'.

VN betgarzone. Cioeun figiiuol maschio, Equi il Poeta seguita.a mostrare if
costume delle nostre donne accennato nell' ottava antecedente, che quando il par-
to édi maschio joghunia di loro vorrebbe esser la prima a: darne lal nfova, es
dannosalia:creatura sempre qualche epiteto,, come ma bel garzone.s ninbel giovane,
un.garbate fantoccione, um bamboccione a' importanza. Vedi sopra in questo C. stan.
491 ma quandoé fematina-, tutte: le afiiftenti ammutoliscono, o quando pur' al
t ine lovdicano®, dannovalla creatura epiteti d' avvilimento, come 2i/cialletto, Pi

sciatcbera', zm ¢uaintuccia,, © simili, come habbiamo detto poco.sopra.
« 4B nostroparemeado pla poltca Genealogia: In che modo noi fiamo parenti «

\section*{FINE DEL SECONDO CANTARE.}
\end{document}
sod opita 4

~ -RS

aa.

i




INARA TA
Ce Paes aetealaet
TERZO CANTARE,

al 70.
'22 ARGOMENTO. §
5 Vengon d Arno 2 [econda i legni Sardi, ?
Sharcan le genti, e vanno a Malmantile,
& she
ote Nascon grandi scompigli in quella piazza,
e335 E ognun si fugge in veder Adartinazza,
cuve SPL Cte ws cure
PRE IAT

Ma per vari accidenti i più gagliards
STANZA TL STANZA-IL





Non fan quel tanto, che di guerra è file.
Arma i suoi Bertinelia, alza flendardi,
E mostra in debol corpo alma virile, R

Nhe sia avoerxo a frarfeneafedere E pur chi vive y sta sempre to

| V Senzafar nullacon le mani in mano, va ber a Lincs '
E lantamente puo mangiare, e bere, Perché al Mondo non è nulla di netto 5
E in fofha,e giuoco viver litte, e fano, E non si puo mangiar boccone in pace 5
Se gli son rotte nova nel paniere, Hor ne vedremo in Malmantil Veffetto,
Considerate se gli pare firano 5 Che immerso nei piacer vivendo abrace,
Ed to lo credo; ¢' a un' affronto tale Non pensa che patir ne dee la pena,
e4 certo ognun l'intenderebbe male. E che fra poco s' ha mutare feena,

Ii Poeta volendo trattare dell' aflalto dato a Malmantile, e del disturbo, che
@ per apportare l'elercito di Baldone a quelli spenfierati, che sono nelia Terra,
introduce il presente Cantare con nna reflefione 5 che sia un gran disturbo a co-

loro, i quali standofene coi loro

commodi, e senza un minimo pensiero, si v

gano sopraggingnere chi gli privi di.
rebbono di gran disguito, pelt an

ucfti loro agi; mentre simili accidenti fa-
a coloro, che\non steficro con tutti i lor

commodi; perché niuno, o bene,
tutti fiamo fottoposti alle dilgrazic

omale, che gli stia, vuol mai ricordarsi, che
,¢che nel mondo non si aa felicica perfecra.

ST ARSENE con le mani in mano, A cintola, o in feno. Si dice d' uno, che
sia tutto dato in preda all' ozio, ed alla poltroneria, e che non vuol Jayorare.
Viv accidiofo, nighittofo, o scioperato, 1 Greci, e Latini ditlero: Jn choenices
fearre: de bomine ociofo, & desidiofo. '3

GVAST AR ? wove nel paniere, Guaftare i disegni altrui, Traslato. guaftar

a are y nuova

in we




TERZO\CANTARE.- 127

Ituova nel nidio.; dove son dalla chioccia covate. Vedi Efopo Favola dell'Aqui-
la; edello Scarafaggio. E il conatum frangere de i Latini.

SE gli pare frrano. Se gli par.duro', e difficile a foffrire. Vedi sopra Cant, 2.
stan,'21. 5 ed il proprio ¢diffrano. Stravagante, o foreftiero 5 o non
del nostro parentado; valendocene in tutti questi, ed altri significati, come segue

| ne i Latini della voce extraneus,

; AFFRONT O. Significa Aggreffione, affalto,, abbaccamento.. Vedi sopras
Cant,1. stan.29. ma si piglia: ancora per Soprufe, come e preso nel presente»
luogo. '

BERE unasciroppo, che dispiaccia, Sopportar per forza una cosa, che sia di
disgutto 5 che in Latino: si disse: Calicem bibere; perché Calix era una specie di

) bicchiere, col quale gli antichi beveyano caldo, come appunto si bevono gli sci-

s roppi; e lofacevano ancor' essi per: medicamento; e per conseguenza era tal be-
+yanda', come.a noi, per lo più, di poco gusto,
° WEL eAiindo non è nulla ds netto. 11 Mondo non ha felicita perfetta. Vricuigue
dedit vitium natura creato,

VIVER a brace. Viver'acafo, senza regola, o considerazione. Ha forse-

efto. detto origine dalla mifura, che si fa della brace, che per esser cosa vile, ¢

i poco prezzo si mifura inconsideratamente senza guardarea darne un poca pill,

©um\poca meno. Da questo poi habbiamo /braciare veduto sopra Cant. 2, stan.

10, che significa Consumare il suo inconsideratamente.

MVT ARE foena, Mutar faccia, o stato, mutar maniera di vivere, Traslato





dalle prosp 3 dove si le die » quali prolp sono da noi
vu Igarmente chiamate Scene.
' STANZA IIL, STANZAUIV.
Era in quei tempi la, quando i Geloni Quand in rerra  armata con la foorta
Tornano a chinder l'offerie de' cani, Del gran Baldone a Malmantilsinuist,:
'E talun, che si [paccia i mitiioni Ond' un famiglio nel ferrar la porta
~ | Manda al prefta il tabi pe panni lani; Senti rumoreggiar tanta genia.
Ed era appunto l oraych' i Crocchioni Vn vecchiocraquest hnom di viffa corta,
Si calano ull affedio de' caldani; Che L erre ogni hor perdeva all offeria,
» Ed escon con le canne,e co! i randelli . Tal che trail bere, el esser ben a! eta
* Tragazzi a pigliare 5 pipiftrelli. Won ci vedeva più da terza in la.:
Descrive la flagione, che correva; quando la soldatesca sbarcd in terra, e s'av q



vid verso Malmantile sotto la condotta di Baldone; e dice che era sul finire dell'
Autunno, poiché cominciava a diacciare, ed i ricchi finti mandavano a impe-
gnare i veltiti da state per rifquoter quelli da inverno; costume assai usato da co-
loro, che sfoggiano in vestire quantunque sieno poverissimi, e questi intendi
vicchi finti, che si pacciano i milliont »che si suol dire; Adezzettin non rifquate Pan-
talone, cs' intende, che gli abiti da state non vagliono tanto, che impegnandoli
possano rifquotere quei da inucrno', come appunto è }abito povero di Mezzetti-
no servo sciocco in commedia,¢ l'abito ricco di Pantalone vecchio in Commedia.
Narra parimente l'hora appunto che era, quando costoro s'accoftarono'a Mal-
mantile, e dice, che fu fa l' annortare, che € quell' ora, fa la quale i Crocchioni
si mettono nelle botteghe intorno a un caldano per paflar la veglia. In tale fla-

gione




we
o

se BE



TERZO CANTARE: “ag
Fiiova nel nidio, dove fori dalla chioctia covate. Vedi Bfopo Favola dell'Aqui-
: dello faggio. E i} conatum frangere de i Latini.



- SE gli pare firano. Se gli par duro, e difficile a sofftire. Vedi sopra Cant. 2.
stan. 1 eli propdo i 'ato e difrano, Stravagante, o foreficra,Onon

ado; valendocene in cutti questi, ed altri significati, come segue
ne i Latini della voce extraneus.

AFFRONTO, Significa Aggreffione, aflalto, abboccamento, Vedi sopra.

Cant. 1. stan. 29. ma si piglia ancora per Sopru/o, come è preso nel presente>
luogo.
NRERE uno » che dispiaccia. Sopportar per forza una cosa, che sia di
disgutto, che in Latino si disse: Calicem bibere; perché Calix era una specic di
bicchiere, col quale gli antichi bevevano caldo, come appunto si bevono gli sci-
roppi; e lo facevano ancor' essi per medicamento; e per conseguenza era tal be-
vanda, come a noi, per lo pil, di poco gulto,

WEL eMondo non è nulla ds netto, 11 Mondo non ha felicita perfetta. Yricuize
dedit vitium natura creato, e

VIVER a brace. Viver'acalo, fenzaregola, o considerazione. Ha forse-
we detto a dalla mifura, che si fa della brace, che per eter cosa vile, ¢

i poco prezzo si mifura inconsideratamente senza guardare a darne un poca pilt,
© un poca meno. Da questo poi habbiamo /lraciare yeduto sopra Cant, 2. stan.
10, che significa Consumare il suo inconsideratamente.:

MVT ARE feena. Mutar facia, o stato, mutar maniera di vivere. Traslato

dalle prospettive, dove si recitano le commedie, quali prospettive sono da noi
vu Igarmente chiamate Scene.

TANZA IIL STANZA IV.

Era in quei tempi la quando i Gelont Quand in terra armata con la scorta

Tornano a chinder l'offerie de' cani, Del gran Baldone a Malmantilsinuit,
Etalun, che si spaccia i millions Ond' un famiglio nel ferrar la porta
Mazda al prefta il rabi pe panni lani; Senti rumoreggiar tanta genia,
Bd era appunto l orach' i Crocchioni Vn vecchiocra quest huors di vista corta,
Si calano all affedio de caldani; Che l'erre ogni hor perdeva all offeria,
Ed tscon con le canne,e co' i randelli Tal che trail bere, el' esser ben d' eta
Tragazzia Piglares pipisprells A Von ci vedeva pile da terza in la,

Descrive la flagione, che correva; quando la soldatesca sbarcd in terrae s'av-
vid verso Malmantile sotto la condotta di Baldone; € dice.che era sul finire dell'
Autunno, poiché cominciava a diacciare, ed i ricchi finti mandavano a impe-
gnare i veltiti da stare per rifquoter quelli da inverno; costume assai usato da co-
loro, che sfoggiano in vestire quantungue sieno poverissimi, e questi intendi
ricchi finti, che [i [pacciano i millioni, che si fucl dire: Adezzettin non rifquote Pan-
talone, e s' intende, che gli abiti da fate non vagliono tanto, che impegnandoli
possano rifquotere quei da inverno, come appunto è l'abito povero di Mezzetti-
no servo fselocco in commedia,e l'abito ricco di Pantalone vecchio in Commedia.
Narra parimente l'hora appunto che era, quando costoro s*accofarono a Mal-
mantile, e dice, che fu fu l'annottare, che e quell' ora, fa la quale i Crocchiogi
si m:ttono nelle botteghe miorno'2 un caldano per paffar la veglia, Ia raie ta.

prone






ye

TERZO CANTARBE; 129
gab pane: ete. 7 ' = 39 re notte a Terza, che è
uafi il principio « = ie si ire, che i fuffe sempre al buio,
© non vedefle punto entro il giorno. E*detto assai vulgato per intender uno

 debole di yifla, come intende nel presente luogo. Vedi spra C. 1. stan. 9. E for-

se vuol intendere Vno di coloro, che perdono la vista alla levata del fole, e las
eee fole ya sotto.
“ SSPTANZA' V. STANZA VI.
Per questo mette mano alla scarfella 1 quali sopra il nafo a Petronciano
Ow ha più ciarpe assai a un rigattiere, Con la [un flemma pose a cavalcioni;
Perché vi tiene infin la faverella, Tal che meglio [coperfe di lontano
Che la mattina mette sul brachiere; Esser di gente armata più [quadroni.

Come [uol far chi giuoca a cruscherella,

Spaurito di cio, cala pian piano,

Due Andò alla cerca intere inrere, Per non dar nella scala i pedignont;
E poi ne traffe in mezzo a due fagorti E giunto a bajo lagrima, e lingrra y
Vi par a occhiali affumicati,e rotti. Gridando quanto mai n'a nella strowza,
STANZA VIL
Dicendo forte, percht ognun  intenda: Perch quaggik nel piano è la tregenda,

Al} armi all armi, uonifi a marcello
Ss lasci il giuoco,il ball, e la merenda,
E ferrinfi le porte a chiavistello,

Che ne viene alla volta del Caffelio;
E se non ci ferriamo,o facciam tefla,
Mentre balliamo vuol suonare a fefta:

Il detto famiglio s se col mettersi gli occhiali, che era gente armata, es
questo si mefle a gridare; allt armi,

SCARSELLA, Tasca, Vedi sopra C. 2. stan. 8,

CIARPE, Intendi robe vili, firacci, bazzecole, che i Latini dissero Scruta.;
ed in altro senso Ciarpa vedi sotto C. 5. stan. 33.

RIGATTIERE, Rivenditore d' ogni sorta mafferizie, ed arnesi da i Latini
detto Propola dal Greco; ed a noi viene da rigaglic, che intendiamo robe diverse
di poco prezzo, ed avanzumi pfati. L' Autore assomiglia la tasca di costui a una
bottega di Rigattjere, perché queste per lo più son ripiene di diversi arnesi, fra i
quali e caluoita difficile ritroyarvi una cosa, quand' altri la voglia,

PAVERELLA, Fave macinate, ed impaftate con acqua. E di questa si fanno
torte cotte nel forno, che si dicong ancora AZacco forse dal Gree, AMatto.Lat, pinfo,
'Tale Favereda dicono, che sia lenitivo a i dolori d' ajlentatura, ed habbia yirti
d aflodar quelle parti; e però dice, che costui /a mette in sul brachiere, che & quel-
la fasciatura, che s* applica afl' eftremica del ventre per foftenere gl' inceftini,

CRYSCHERELLA. E? giuoco da Fanciulli. Fanno in fur' una tayola yn mon-
ticello di Crusca, e yi mettono dentro quelle crazie, o quattrini » che vogliono
giuocare, e mescolando poi bene, si fanno da uno del giuoco, a ciò depucato,
tanti monticelli di detta crysca, quanti sono i giuocatori, i quali ( lasciando da

quello, che ha fatto i monti, perché deve esser ' ultimo a pigliare il monti-
cello') tirano le forti a chi debba esser il primo a pigliare uno di detti monti, ¢
ciascuno nel monte, che gli è toccato va cercando de i denari, che la fortuaa,
v habbia fatti reflare. Stimo, che questo giuoco faffe nfato ancora da i Fanciul-
li Latini, perché si trova Ludere furfure, Ed a questa ricerca, che fanno i ra-
gazzi del denaro aflomiglia quello, che  il famiglio per trovare gli oc-
chiali PA.

































le m:
4"





















130 MALMANTILE; >
FAGOTT/, Inuolti, o fardelli piccoli» I ncorg
PET RONCIANO, ¢, pi ae ee

forse specie di Mandragora; ¢

simile alla Zucchetta, e fla appicca }

ghianda, alla quale s' affo a figuea;

si appella Atarignano, A questo Perroncians s' allomiglia ¢

ti un nafo di firaordinaria grofiezza, e di colore rosso livido.,

s' intenda', che havefle questo famiglio. wend
e4C AVALCIONT, Vuol dire una gamba da una p:

come si sta in sul cavallo, e come,stanno gli occhiali &

da una parte, ¢' altro dall' altra.. ee a
PLAN piano. Cioè adagio adagio, bel bello: Adaaie. »VOCe p

aggiunta al verbo fare, ed al verbo andare significa quel, che hel prefen te Ju

cioè Adagio, e con diligenza,, che i Latini dicono placide incedere; ed aggiu
verbo parlare signitica parlar con voce bafla, Submifva voce.
PEDIGNONT, Specie d' infermita, che viene,ne i piedi, e nell

troppo freddo dai Latinidetti Perwiones, A pecan svelte
S/GNOZZ ARE,0 fingozzare, o finghiozzare. E' un moto del

fuer (0, o mediattino, cagionato da foverchia -yotezza,o ripienezza;

litudine significa anche sospirare vehementemente con pianto, come

prelente luogo, | Latini ancora s ne servivano nel primo significato, ¢

condo; Singultus-, & fingultire, © fingultibus ingemere.: etee
GRIDA quanto mai n' ha neda Hrozzas Grida quanto pnd pilseg

refifter la gola. Che frozza vuol dire La canna della gola, altrin

Gorgozzule. 1 Latini pure di¢evano in gusture exclamare, £ da questa

vicne strozzare, che yyol dire Strangolare sh

Dinte Inf. C. 7. Quef? inno si gorgoglia nella frozza, |

gE, C28, Con Ia lingua tazliaca nella ferozza
SVONISI a martello. Si suonino le campane a rintocchi, che si

corr' homo.:
TREGENDA, Moltitudine, e quantità di gente, Dalle perfo

crede, che vadano fuori la notte anime dannate, ed altri spiriti per

gente, e queste chiamano la Tregenda. Tal' opinione se bene ei Dp

plici, e idiote, nondimeno pare che venga seguitata da S. Agostino

lib. 4. de Civit, Dei dice. Lamia dicuntur anima hominum depravata,

te meritis macufofa, qua a corpore feparate terriculamenta funt mortalibs

sente lu go è intesa per moltitudine di gente.: )

SVONARE.. li verbo suonare si piglia talvolta in vece del verbo |

e¢ però ne nate l'equivoco del suonare mentre colora ballano y che vuol,; K
tergli, se ben pare, che voglia dire suonare alloro ballo:, Ed in cid'

Latini, che hanno il verbo py//are, che vuol dir perquotere y & 4
suonare ogni sorta di frumento musicale, e le campano; ed il
pelaror..









TERZO CANT ARE: 131

f ava <o STANZA! IX.

Tn qm A Tra quespi paittl acer a fore afsai,

© EB che ne cup ogh 'baloce “Ole d Marthefi, Principi, e Siznori;

Cad omparita 2S Fyamin' di Conto,e eP offi botregai,

a 1s eee mooyct "Wana Siralbe, e Battilori 5

, Quivi le una progenie ardita Lanaiiiol, Ovefici', e Merciai

' a ye erstiey phe ee ti Norai, Legisti, Medici, e Deter 5

“OES ne'viene all'erta lemme lemme “In somma quivi son gente, e brigate
Col Batthil Toffie tutto Biliemme. D' ogni forva; chiedere, e domandare,
be bd fiuddetto vecchio andava ptidandé, e che non ostante questo, colo-

fo, erano in 1 if€seguicavano: a darsi bel cempo, l'armata arrivd

Panera: ¥



a
z
=
3




soe

=

Bese

thufa; 1} Poeea'natra la' qualica di questi soldati.

'A Vuol dir adnate, o'cantata, Bocce. Nov. 97. Cox una sua viola
fronr ia Pampire  Vatchi itor. lib. 10, ALdarespa andò in persona fopi a il bastio-
ne di S, Miniato con tutti li [uoi fuanatori, e dopo piit lunghe Rrombertate, e flampite,
ec. Ma gui iatende romore:, € cicalamento odiofo, che e il senso, nel quale oggi
per lo pire tela da'noi gueft# parola, ed ha lo stesso significato che bordeiio,
d Paar Le Binill red'p ¢iictaforicamente, il' che vedremo altrove.
ip ALOCCARST:"Prettullar®, Perder”il tempo', € trartenerl in cose di poco
momento, o traftalli da ragazzi, dé i quali è proprio il verbo baleccarft, o balocco;
¢ forse & fincopato daf verbo Baddlaceare, e Badahicta; Vedi sotto C. 6, Nan. 32,
Spr ewncOEMEee.» Dictatho Wich Bicacea”™” Vareli MOF. 1B: 15. 214 fureno pore
tare' le'tbiavi di non fo! che Biticea'; Vuol dir forcezza piccola', e di poca considera-
zione posta in luogo eminente, come appunto e Malmantile, il quale con glie-
fla fola patold Biccieocea', il Poeta benitliio descrive; percht per Bitcicocea vol.
garmente intendiamo un Cafolaré', o castelluccio poito in luogo eminente, ma
da:farne poca flima'. Lasca Nov. 3. Salita che hebbe ton non poca diffiultd quell al-
pehre Montagna, credeka entrare in un bel caspello, ma riguardando all' inturno, ved-
de che era ina Biecicoece piit per vefugio di capre, che per ricetto di soldati,

'ST confide nelle fante'nocca. Ha la sua fidanza nelle pugha. EV' epiteto unre &
meffo per esprimere if odo'de} parlare de i Battilani: Se bene e usato dalla gen-
te anche'piil civile per interider perfezione come vedemmo sopra C. 2. stan, 52.
E quié benissimo posto';' perché /anttxs vuol dir determinato, © fabilito, fendo
fincopato da fancitas, e le pugna sono s' armi stabilite, e proprie de' Battilani,
Che' per mocea'; che foo 1 nodelli delle dita, s intende tutta la mano feriata, che

in questo più, che in altra maniera si scorgono le nocca',

* della medesima natura, ed ha lo fiefio significato di pian
piario dette in questo C: stan. 6.; ma e termine restato nee i Battilani, o feo
we @/usate da altri (ara detto lieme lieme, che viene dal Latino ieviter, o leve,e
' 'leggierttiente; o dal'Tescano Lieve, che vuol dit Leggieri.

°BATTL, e Teffi. Bauilani, che son coloro, che cdnciano la lana , e Tefi
'quelli che lateffono, 8) 8 es cerns ' $
0 TETTOBiliemme. Chiamiamo Biliemme quell' ultime contrade della Città
di Firenze, dove abita questa sorta di gente, la quale veramente, benché
'nata, ed allevata in Pirenze,è affatto onary da gli altri Fiorentini - i'co-

: 2 lumi,

2.
=

as





SARA KLEEN.
dh
e
2
5



Digitiesenipuar


a



132 MALMANTILE

flumi, e nel parlare; farebbe leggi a suo modo; mangia d' ogni sorta spo
come gatti, cani, pesce, e carne fetida; beve ogni (orta di vino |
mente, come afferma il nostro Poeta sotto in questo C. stan. 60. dicendc
che a bere e peggio delle spugne. In somma è un Popolo da se, che noi chia
gli Vasi, il Batti, o Biliemme, \a qual voce serve ancora per esprimere la p
plebe, come è nel presente luogo. ak
GVITT/, Guidoni, ace » fudici, sporchi, e fordidi. E' parola che;
Napoletano, se bene il Varchi stor. lib, 10. € ne serve anch' egli per esprimeres
un' hvomo d' animo vile, dicendo: Eli era tanto d' animo guitto, e tanto mefebings
che usava dire: Chi non va a bortega e ladro. (rape
HVOMINT di como. Huomini di stima; huomini riguardevoli.. Trans
forse dal giuoco delle Minchiate, nel “ giuoco si stimano, ed
Jamente le carte, che contano, le quali son quelle, che vedremo sotto C. 8.f
61, Si dice 2 tale conta per intendere; il tale e huomo adoperato, o e buonoas —
walcofa. ah
a BATTILORI, Mercanti d' oro filato. Banchieri. Mercanti di cambio, che
si dicono Negozianti. Serainoli Mercanti di drappi, e di eta, Lanaiuols. |
canti di pannine,¢ Lana. Orefici. Mercanti d' oro, e d' argeato fodo. Alem
ciai, Coloro, che vendono naftri, feta, telerie, ed altre merci simili,
questi suddetti in generale si chiamano Mercanti, o mercatanti.
BRIGATE. Quantità di gente, Vedi sopra C, 1. stan, 2, 3
D ogni sorta, chiedete, e domandate. Cioè domandate, ed eleggete '
sorta di gente volete, che la troverete fra costoro; perché vi è d' pmete







mur te







mete

persone. Hs
STANZA X. STANZA XL

Sul Colle compartisce questa gente 1 nome di costui, dice Turpin
Amostante con tutti gli Vfriali; Fu Paride Garani, e il legno prefty
Tra' quali un graffo v' e conualescente, Perch' ei voleva darne un rivelling —
©' haveva preso il di, tre serviziali; A un [uo nimico traditor France,
E appunto al corpo far' allor si sente Che per condurlo a seguitar Galina
L operazione, dar dolor bestiali 5 Lo tira pe' capelli al fuopacley — *
Tal che gridando fenz: alcun conforto E per fuggirne ai paffi lagabellay
In terra si butte come per morto « Lo bolla, marchia, e tutta lo fuggella,

Ii Generale Amoftaate distribuisce sul colle di Malmantile i Soldati, fra iqua-
li era Paride Garani, che havendo preso un gran vacuatorio sentiva duloci acer
biissimi, e però G rammaricava. Il nostro Poeta per accredirare questa ope
ra, come fece il Pulci nel suo Morgante, e ' Ariofto nel Furiofo, le da anche
egli il fondamento della storia; allegando l'autorita di Turpino., come fece an
che sopra C, 2, stan. 31. e da quello che scrive Turpino, cava che costui havea
nomie Paride Gatani, il quale havea pre(o il legno per dare una quantità di le
gnate a un (uo nimico Francese, cheyper condurlo a segnitar Calvino,s lo yoleva
tiraré pe i capelli in Francia, e per risparmiarne la an  haveva già mat
chiato, e bollato, e figillato. E scherzando l'Autore con. questi. equivori, wl
dite che Paride prefe il Legno fanto per medicarsi del mal Franzefe. |,

PRESE id legno, Cio bevye il decorto di Legno Santo pet medicare il Mal

t r: 'ran-








TERZO CANTARE? 133
Franzefe; se ben par che voglia dire, prefe un pezzo di legno per bastonare quel
Sa:

DARE un rivellino, Dare una quantità di legnate. Rivellino e una specie di
fortificazione, che si suol fare d' avanti alle porte delle Città, o fra le cortines
delle Fortezze, così detto forse perché revellitur a linea, o perché revellat hoftium
vim; e da questa rivolta nelle cortine, o dal quasi fivolarh cal al nimico hab-
biamo il presente translato, che ci serve per esprimere, Rivoltarsi a uno cons
gran quantità di bast » bravate, riprenfioni, ec, E dicendosi aflol
¢ (enz' aggiunta: Gui fece un rivelline, s' intende Gli fece uma folenne bravata,o buo-
na pafsata,o gran rabbuffo; E dare un rivellino,s' intende dar quantità di percofle.

RIDVELO a seguitar Calvino. Par che voglia dire ridurlo a seguitare la fetta
di Calvino Eretico, e vuol dire, che per farlo divenir calvo, questo suo mal
Francese lo tira per i capelli, e glieli fa cascare.

£0 balla, marchia, e tutto lo fuggela, Fa bullette, marchia, e fuggella. E vuol

dire che ee suo mal Francese gli havea cagionato bolle, crofte, e lividi; che

il verbo fuggellare vuol dire Far de i lividi nel vifo a uno con le percofie, i qua=

li noi chiamiamo Pesche: 1 Latini in questo senso dissero; /uzgiliare. Vedi ford

C 6, stan. 54. metaforico da /uggellare che vuol dire imprimere in cera, oftia, ¢
simili nelle lettere, ec. e si dice anche /igi//are Dant, Purg. C. 7.

La sua impronta quand' ella figila.
E fuggellare Dante Purg. C. 10, Come figura in cera si fuggella, E Canto 3}.

Ed io si come cera da fuggello.
STANZA XIL STANZA XIIL

Dife Amostante, viffo il caso forano, Gloria cerca Lion, piit che moneta,
© Noferi di casa Scaccianoce: Pero ch' ei bada al giuoco,efa progreffo;
Per Ser Lion Magin da Ravignano, Per l' acqua in Pindo andocome Poera,
Ch' sk venga a medicar, corri veloce; Ondt agl infermi da le pappe a lefso.
do dico lus, perché ce n' e una mano, Gis è quel che attende a predicar dieta
Ch infilza le ricette a occhio,e croce y E farebbe a mangiar con L' interesso;
O fa sopr' al! inferme una bottega, Ma perché già tu n'hai pits d'uno indizio,
E pos il più delie volte lo ripiega. Va via, perché l'indugio piglia vivio,

Amostante veduto lo firavagante accidente, ordind a Noferi Scaccianoce (che
vuol dir Francesco Cionacci ) che andafle per Ser Lion Magin da Ravignano
(che vuol dire Giovann' Andrea Moniglia ) e facefle venire lui medesimo, che &
un valent' huomo, € non come quaicuno, che non fa dove s' habbia la testa, ed
in vece di medicare un' infermo 1! pil delle volte  ammazza con le sue spropos.
tate ricette, ed e di quelli, de i quali si può dire.

His, & si tenebras pep ant, off facta poreftas y

Extenuandi agros, bomine/que impuné necandi,
. [che non si può dire di Lione, che procura pil d' acquiftar gloria che oro.
Egli e Poeta, e pero non e maraviglia, se andando egli per J' acqua al fonte di
Parnalo dia poi molte pappe con  acqua'a gli ammalati. L' Autore dice così,
perché in una sua leggieri infermica non voile questo medico, che e¢gli pigliaties
amedicamento alcuno, ma lo volle curare con ia fola dicta, facendoli mangiar
fera, e mattina pappe; e però dice; sstende a preduwar dicta, E farebbe a man-

già






ee





*

134 MALMAN TILE T
iar con U interefo; perché veramente itil quel tempo Lione
sano e robufto, mangiava assai. Questo Lione non era sta
tore nel primo componimento de! te sia Oper: 28 suo
havendo solamente eat quel 1 ream ad aie lf
dremo pocoappresso,ia dopo la fudderta infermita,per ven
dell nalce whet to a Tieta ce lo volle ae 'Hor tornandd
no. 1] Generale dopo haver dato a Noferi molti contrafiegni'
scefle questo medico, manda'a'cercarne. © © ssi

CE n' è una mano, Ce ne son molti. Termine'che vien dal Latino ©
En, /unenum manus emicat ardens. aes -

INFILZ A le ricette a occhio, ecroce, Si'dice anche'a Occhio V0
ricette senza regola, considerazione, o fondamento. Opera senza f
prova, E' termine meccanico, - « eee ee ee
FAR una bortega sopra uno infermo, Far allungare il male per cavarne'®@
guadagno, E questo termine s' usa in qualfiyoglia negozio, del quale uno pra
ri di prolungar la spedizione per buscar pity denaro. nse ae
RIPIEG ARE uno. \ncendiamo Far morir uno, Vedi sotto C, 10
BADAR al giuoco, Attender con applicazione a quella profeifioné
fa, o a quel negozio-, che ha fra mano, e si dice anche Badare-a dott
sopra C, 1, stan, 62. questo verbo badare in altri significati, z

PAPPA, Cioè panc bollito nell' acqua; o'in altro liquore. E' dig
le inventate dalle Balie per facilitare il parlare a i bambini, come B
ma,¢ simili. I Latini difflero, pappare, e i Greci pure dicevano ?. eb
in altro senso, volendo-esprimere il Padre, i) Habbo, Vedi sopra C, 2, stan!
E sotto C, 4. stan, $e 12.:: bad ovale

ATTENDE 2 predicar dicta, Sempre dice che si mangi poco; che q
tende per far diera. Se bene appresso a' Medici diera vuol dire regola
versal¢. Dieta si dice congreflo di gran personaggi per trattare n
mi, come si dice Dieta il Congreflo de i Priacipi Elettori all' Ele.

eratore.:

F AREBBE a mangiar con l'interefso, Mangerebbe sempre di giorno, ¢di
te, come fanao i-cambi, o usure', che mangiano di, 'notte, mentre che: a
po fa crescer la somma degl'intereti. L' usura in Ebreo dicefi mor/>', ie 'one

L LN DVGIO piglia vizio. L' indugiare, o trattenerff e pericolofo'di cagio
qualche danno, o far perder la congiuntura di conseguir l'intento. Adoré





















damnum, 7%
STANZA XIV. o e Tai
Noferi vanne, e sente dir ch? egli era Bedi foglirdiste/a una gran eras
"Con un compagno, entratoin nn fattoio, Ha bell", e ritto quivi il [uo se '
Ow egli ha per lanterna, efsende feras\ ~ Siche prejfo lo trovi, eid
Li orinal fitto sopra a-un [chirxatoio, ©. Dell uned Sndio gli facta ]
Noferi trova il Medico nel Fartoio da olio, che quivi era il suo studig? 3




fa l'ambaftiata.:
FATTO/O, Quella lanza, dove & la macine per infragnere l'olive 5

ma
firettoio, ed altri ordinghi per cavar l'olio dalle medesime olive « en it

tino Oles fattorinm. Ree






















gi?

ae
SS



TERZO CANTARE,. 135
elfednehtetse 0.0 altea gaapaveris » ocl quale s!orina, da i Lati-.

ma csc ane » donde i Sanefi chiamano scafarda '
piers effetto usanole donne.

Beis canna di stagno, o d' altro metallo, con la

sf agi' infermi. Vedi forro C10. fan. 4. >
ames ae Sparfa una quantità di fogli. ee era per la si-
quella diltefa di fogli con le ere, o mercati, che alcune

y tan ei eae nelle quail per le piazze si veggono moltiii-
ee v dilegnt » leggende, ed. altri arnest confufamente..,
ra;





 abbiamo forse questa voce fiera dal Latino forwm, che era intclo
a dove si facevano le fire o mercati, o pure dal Latino ferie,.

l ¢ritto si Hacon facilita aggiuftaco il (uo scrittoio; che la voce bello,
terthini ale

in o non vuol dire, che Ormai,o di già,,¢ serve per cmfali,
¢ per denotare la franchezza in terminare una opcrazione: Si dice riccare unde
bortega., rizzare wo negorio per dar principio a un negozio..

VNTO si: Oe chiama fiudio quella stanza., nella quale uno faa Gutters?:

epee Medico haveva depurata per suo (tudio la stanza del fattoio, lo






IN? en au stanze sono, o verifimilmente.devono essere uate.
' NZA X STANZA XVI,
writ chiamato Brae Era quest' huomo.un certo Medicaftro y
y (ponde ide haver' allora altro che fare, C? al dottorato. Luo se piover fienos
nen RR 6 upa.sua commedia ing distende Eperch' ei. vi pati (pefe., e difaftro
* datizalaca. U,Confole di.Adare y E frato sempre grofjo-con Galena;
“Eche/e? opra Jua cold s* attende Egiuntola: Vofar(aife)ua' impisftro >
Pee feast qari sue scolare Onde s' it mat venifJe ds velen
> nd (persmentata ed in Sua vece Presto vedrema; in tanto egli se [pogli,
ia mandato lis; e casi fece. E fiami dato aduenties efogli.

'Scntendo Lione d' esser chiamato a medicare » tisponde, che per allora nons
;wenire, mache. mandcra un (uo (colare valent' haomo, Costui era un gran
— O.giuata doye eral infermo., cominciò subito con. gli (propositi.
COM OLE di mare, Questa tu una Commedia intitolata La Serie nobile, nel-
Jaqualc e introdotto per l'Broe.un Confoic di Mare in Pifa,onde molti la chia-
mano il Con/ole di mare, ancor che il titolo stampato in fronte,di cla sia,La Ser-
na nobiles e tt composta dal medesimo Lione,, e recitata. in) mulica: con grandi
Apparati d' ordine del Serenifsimo Principe Cardinal, Gio; Carlo nel. suo belliti~
mo Teatro fabbricato allora.di nuovo, Ed il nostro Poeta nella presente ottava
vuol mostrare la poca applicazione, che Lione haveva in quei tempi alla medi-
cina,, come giovane, s¢ bea per altro dotto; e che poi voltacofia tale studio ha
saputo acquiflarsi la fama, che ha acquiftato, € meritare una delle prime Catee-
dre dello itudio di Bila, e di feryire attualmente'al Serenifsimo Gran Duca per

CUEDICASTRO. Medico di poca scienza,.0.s come diremo:) faluatico,

SE piover fiera nel suo dottoraro,, Quando si sente uno » che vaole spacciach per
huomo dotto, e dal parlare si fa conoscer per. uno igaorante, si suol dire quaa-
do ci parla Tirate già del sieno intendendoG: Per darg.a queito bus che pacia.. Si

whe










~ Galeno, e non fapeva quel che egli dicelfe, fiche in fuftanza vuol dir un




136 MALMANTILE

che dicendo che nel addottorarsi costui, piovve fiena,intende che costui si
to per un folennissimo bue; e però venne gran quantità di sieno senz'
sto, che diciamo; La roba ci piove per intendere vien roba in abbon

chiederla.;.
E' STATO sempre grosse con Galeno; Esser grosso con uno vuol dire

collera, o esser adirato con uno; Si che dicendo, che costui e Pato fempr
can Galene, perché l'haveva difaftrato,¢ fatto penare, s" intendeera:
seco; e però non lo guardava mai, e conseguentemente non havea p







dissimo ignorante nella Medicina. 2

VELENO. Questa parola ha due significati: uno proprio che & toffico,
tro improprio, che e fetore. IL primo e quello, che s' intende nel presente
go, il secondo si vedrà nell' Ottava seguente. ae
STANZA XVIL eae
Confermata pero la sua credenza '
Rivolto at eeapagl oe a dire bed |








Mentre e spogliato, per ta peftilenza,
Ch' egli efala, si vede ognun fuggire,

Pernenne una zaffata a Sua Eccellenza, uefto e veleno,e ben di quel profonde,
Che fu per farlo quasi che fuenire; Ses voi ch' egli avvelena it

Mentre che Paride si spogliava ognuno per lo gran fetore cominciò a i
onde il sig.\ Medico, che sente ancor' egli l'orrendo fetore, si confermed nel cre
dere, che fufle veleno, percht avvelenava. ie

PESTILENZA, Intendi fetore grandiftimo. E si serve della parola pe
zy per la parola veleno presa in significato di pyzzo, o fetore, e per altro Peft:
lenza vuol dire mal contagiofo,: ? -

Z AFF-AT A. Parte del vapore di quel puzzo, portato dal moto dell! arias:
E si dice anche 7afaea d! ogni liquore per intendere /przzzagiia d' ogni liquore «
Franco. Sace, num, 136, L'orina gli andò sul Cappuccia,e nel vifo,ed alcune rafate io
bocca, coe
4S, Ecc. Questo titolo benché non sia così conueniente a' Medici,nondimeno
 usato dalla nostra plebe in vece dell' Eccellentissimo, el' Autore lo daa
medico per derisione. f

PROFONDO., Per traslato significa Grandemente, fmoderato, o perfettifi-
mo, come usavano anche i Latini. F

AVVELENA. Rende puzzolente. Ecco la voce veleno, ed err
fa nel secondo senso detto di sopra di pazzo, oferore; El" equivoco, che
ciò ne nasce, serve a questo Medico per farsi stimar dotto mostrando conolcere
che questo e veramente veleno,perché egli avvelena, che vuol dire far uutire,ed!

lo piglia in significato d'attofficare,c Veleno in significato di toflico, Vedi sotto in

questo C, stan. 54. la voce lezzo. t
“STANZA XVUL 5
Rispose il general, commoffo 4 Sdegno + A cio foggiunfe il Medico: Buon segney
Come veleno ? o corpo di mia vita | Segno che la natura inuigorita
Edoveeil Mefroaahauedtre ingegno? A' morbi repugnante, adeffo queste
Lovedrebbeil miabuesch'egl bal'nscica, ef nostri nafi manda si moleflo.

Ui Generale s' adira, e dice; Che non hayete odorato da sentir questo puzZ0»






TERZO CANTARE, 37

—- conolcere, che egli ha l'ulcita! Alche.replica il Medico: Que-
foe segno, perché la natura havendo preso vigore, come quella, che re-

pugna ai morbi, espelle ora questo morbo, e lo manda ai nostri nafi. Per in-
- Render fito, fons direa questo Medico, è:necessario sapere, che
“lay significati, il primo ¢.iofermita, ¢-dicendo repygnante aii

morbi intende all! infermita; ed il secondo è fetore 0. puzz0;-¢ dicendo manda a'
nostri nafi queffo morbo intende Manda questo fetore, Ea il buon medico, che sti-
mas che natura morbo repugnans. voglia dire repugni al puzzo, cava la conseguen-
za, che il sentir questo puzzo sia buon segno, perché la. natura scacciando il puz-
20, dal corpo dell' infermo, lo manda a i nafi de' circoftanti, e così va scemando
il morbo al iente..

 £0. vedr mio bye. Lo vedrebbe uno, che non haveffe punto di giudizio.

YSCIT A, Stemperamento di Corpo, Soccorrenza; da' Latiai con voce Greca
detta Diarrhoea.

SVON segna. L' Autore.mostra in questa Ottava il modo, col quale soglion,
parla i Medici ignoranti per accreditarsi appresso agl' idioti, dando ragioni
(propositate, e inducendo ott improprj; pur che lufinghino il pazziente con
una eerta apparenza-di sperar bene, come fanno gli Zingani, e i Montamban-

chi.
wot iaDaceamy § eeoS DANZA XIX.

Vedendo poi y chtil flufso raccappella Chiamagli aspati, cbinfermieri appella,
(Come quelle ctha in zucca poco fale ) Ui Cerafico chiede, e lo Speriale,
Comincia a gridar:Guardia,lapadella; E veuuto Linchioftra, al fin si mette

BE ( quasi fufse quivi.nno spedale ) A feriver una rifma di ricette.
L leatiflimo Medico vedendo, che il corpo faceva nuova operazione,co-

mincida chiamarla Guardia, che portaffe la padella, pensando che quelle pa-
role havessero virtii di fermare il flutfo, havendole sentite dire negli Spedali in,
occasioni simili,¢ però credendo esser ne/lo Spedale chiamava gli Aftanti, ec. ¢
poi si messe a feriver una gran ricetta.

RACCAPPELLA, Opera di nuovo. Reitera, Replica. Raccappellare si di-
ce quando coloro, che stringonol' olive per cavarne I lio, o le vinacce per ca-
varng il vino, dopo haver dato qualche siretta, allentano lo strettoio, e nelle>
gabbie mettono nuove olive, o nuova vinaccia sopr' all' altra, che v' era prima.
Alcuni dicono rincoppedlare, tracndolo dalle coppelle de' purgatori d' oro, nelies
quali rimetcono pits volte lo stesso metallo per rathaarlo,il che dicono rincoppeliare.

HAVER poco fale in xucca, Haver poco cervello, poco giudizio. Boce.n.2,
Be 4: Per porre la fun belezza innanzi ad ogn' altra, si come quella che haveva poco fa
4¢ in xucea,, Vedi sopra C, 1, stan. 73. e sotto C. 4. tan. 15.

GV-ARDLIA, la padella, Questo e un detto, che s' usa, quando si sente, che
altri facciaromore per di sorto per causa dell' u/cita del vento, e si dice così, per-

gl infermi, che sono negli (pedali, quand' hanno bisogno di vorare il.ven-
tre, chiamano colui > che è di guardia, che porti la padel/a., che € un valo di ra~
me, ¢¢, il quale e adattato in maniera-da potersi mettere,in cao di bisogno,nel
0 sotto all' infermo, acciò che possa fare il fatto, suo.; tenza muoversi dal
etto. 3 Bibi 2% oo ui wha

7 ee

s “sTan-










138 MALMANTILE

ST ANT?, o Manti, Son coloro, che affifiono al servizio deg!'
2 me vedemmo sopra C. 1, stan. 48. Lat. «d/antes. i628

INFERMIERE, Chiamano negii spedali Znfermiere colui. 5
che gl? infermi sieno mefhi a ietto, quando son condotti allo spe p
nota per fargli vifitare dal Medico, e gli regiftra al libro degli entrati, ¢
usciti, ed al libro de' morti. (andi 04

CERVSICO. Quello che medica le ferite, piaghe, ed altri ma!
richieggonc opera manuale, e cava langue, ec, detto ancora con voce!
usata da' Latini Chirurgo.

LISMA, ori/ma, Diciamo un fagotto, o balletta di carta, che
a 500, fogli DalGr.arithmos, Qui però è detto iperbolico, e per mofir
quello Medico (crivetie assai,non che veramente consumaffe una Liima di

TANZA Xx STANZA XX
Dove diceva ( dopo millioni Peré presto boliir farere a fodo
Ds feropoli, ai drammeye libbre tante) Voit agnello,o caprette in um pi



















Che già, che questo mal par che cagiont QV un' altro vaso nelle stesso
Stemperamento forte, umor piccante y Vn lupo per infin che sia disfe
Per temperarlo; Recipe in bocconi Poi fare un servizial col prii
Colla, gomma, mel,chiara,e diagrante, E col secondo un' altro ne sia fai

Quindici libbre in una volta fola Farò quefea ricerca operazsone
















Di sangue se gli tragga dalla gola; Senx' alcun dubia, ed eccola rag
STANZA XXL
Accio che tiri per canal diverso Questi animali efsendo per nat
L'umor che tende al cetro, ut One grave Limici, come i tadré
Che se duraffe troppo.a far ral verso Ritrovandosi quivi per

Dir potreble dinfermo: Addio fave. 1 lupo correra dietro all'

Pot tengafi due di capo riverso Lagnelio, che del lupo baurd

Legato per i piedi a unatrave 5 Ritirandosi andra per ibd

Se questo non facefse giovamento, Così va in fu la roba y &.

'Compote gli faremout argomento, E i due contrarj fan, cht. zodd,

To queste sue ricette mostra l'Eccellentissimo Medico la sua g ine COs
proporre farmachi, e rimedj spropositati, come € quello de i due brodidi lupo, —
ed' agnello, e quello del tenere il pazzicate appiccato al palco per i a

. ca











'capo all' ingid, b Ph Sayy sky
eu/LLIONE, E! un numero determinato di dieci centinaia di migl
è preso per indeterminato; come succede spesso, che per esprimer, u
quantità di cose., si dice E' un millione delle tali cose, ancor che sieno mol
no, ed aile volte molte pil. Così i Latini in questo senso fexcenra;
4 Greci myria., cio' diecimila. ou), isbn aN
STEMPERAMENT O forte. Stemperare vuol:dic Ammollire;10
nel ventre di coflui era follevamento d' umori, e stemperamentodi
ti, clot acide, e diumori piccanti.. Gli epireti di forte, ¢) piccante fon'
convenienti al yino,dicendosi vino forte quello, che comincia a diventare ace
ved in molti lugg hi d' Italia si dice Vin forte,il vino:gagliardo, o grande
Bectl: mee















piccante quello che in beverlo fa friazare le Jabra, e la lingua. Questo Bé






TERZO CANTARE: 539
ti?  - lentissimo Medico /però intende quel forte per acido, e per grande, e gagliardo;
v4 E piccante dal ieee % che vual EE Duguere % Offendere che si dice anche
ph dar nel nafo » Vedi sotto C; 7, stan, 59. I Eccellentiflino cava l'argumento, che
ee i umori fieno'piccanti, perché danno nel nafo col loro fetore; Ora per ra(-
Oe] fodare, e coagulare'tal flemperamento vuole il prelibato Medico, che si dia al d
pazziente a bere gran quantità di col/a, miele, gomnia, chiara d wow, e diagran-
tit te, le cose nella ens antita, che egli pone ses incorporaffero, in

ase

Ge grandifiima® quantità: d'.a e fsresbous atte a coagulate, e feccare un Iago;

¢ (e vi havefle aggiunto gefio, e matton pefto-havrebbe dato una ricerta da ilop-
ait pare quante'rorture si possano mai troyare ne i vivai.
rat \ DIAGRANTE, Specie di gomma, o colla, che serve per incollare i drappi
aa ne i'rovesci de i-ricami', o per altee cose simili.
1. SE li-tragga 15. libre di sangue'per la gola, BE cavandosi 15. libbre di sangues
dalla vena della gola del pazziente; e legandolo per i piedi al palco.col capo

Po all" ingid (che questo vuol dir caporiverso ) preteude il Medico y che ia roba sia,
ya per mutar viaggio, se vorra condursi al suo centro, che non & più nel luogo,
fen dove era prima, ma stante la positura del corpo è diventato suo centro il capo,
a CONTINOVASSE afar tal verso, Continovafic a fare nella medesima forma,

'se o maniera, Vedi sotto C. 7. stan..1.

AD DIO fave, Significa Noi fiamo (pacciati; Noi fiam finiti; Siam morti, Fa
et un Villano ne! contado d' Imola d' ingegao più tosto grosso che no, il quale ha-
i veva un bellissimo campo di fave, e nel mezzo di esso era un gran ciriegio carico
o di ciriege. A tal Ciriegio haveva il villano fatta una fortidima prunata, perché
is pops soe gli fussero colte; e vantandosi di questa sua diligenza, fu sentito
. da un Cieco suo amico, il quale glidiffe: Con tutti li tuoi pruni io vi falird, e:
sf se non lo faccio, voglio perdere dodici lire, ch' io mi ritrovo + ed il villano re-
de
eh



plicd: Setu non pigli la scala, o vero non porti il forcone, o altro per levare

1 pruni io voglio giuocare questo campo di faye, e che tu non vi (ali. il Cieco

si contentd; e così.conuennero. L' aftuto Cieco si coperfe tutta la vita con buone

ie pellidi bue, e così armato paffando per mezzo de i pruni senza sentir puntura,
y alcuna, fali sopra il ciriegio. Li villano, veduto questo, tardi accortofi della sua
g* ee » piangendo il suo danno gridava: Addio fave, cioè io ho perduto le

uoh fave, Vedi il Cornazzano Novella 10. dove troverai questa fayola non travetti.
i ta, e meglio espreffa.

TRAVE. Legno grosso,¢ lungo, che s' adatta a reggere i palchi. z
inet ARGOMENTO., E'lo stesso, che Serviziale, o Crificro detto sopra in questo

lt C. stan. 10. e 12. E.quitorna bene, perché vuol medicarjo per via d' argumenti
si ——Jogici yma di canfeguenze sproposirate.
ao BOLLIRKE a fodo, Cioè bollire molto tempo, e gagliardamente,

BRODO.. Decorto di carne. Acqua ingraflata con carne.. Se ben la parola
PA brodo è comune a ogni sorta di decotto, o mineftra, aucorché non di carne.
ie 1 DVE contrar} fan che it terzo goda, Inter duos stigantes tertius gaudet, Con que-
oe flo argumento, e con queta sentenza, e con altre ragioni da fquartati, pretende
cA F Eccellentissimo d' haver trovato il:modo di fermare i Hluilo.;

S2 STAN-




140 MALMANTILE




STANZA XXIV; » STANZA XKV,

Cio detto rivolteffi al mormorio In quel che questo t

Di quell' ambrette, ov' a meftar si pose; We dice ogni or delt'

E, perch' elle fapevan di stantw, TofelloGrani, ilquale tu

Teneva al nafo un mazzolin di rofe. Mofo a pierd, con una fun

Soggiunfe poi: Costui vuol dirci addio, Tagliace havea aun

Che quefie flemme putride, e viscofe Sopr' alte quali a foggia di

Mostran, che ben' affetto agli artolani Fu Paride da certi Conradini






Ei vnol' ire a ingraffare i Petronciani, Portato a' suci poder quivt vicini.

L' Eccellentissimo Dottore, dopo haver fatte le suddette belle ordinazioni
mette'a Muzzicare quella materia,¢ da quel puzzo fa pronoftico, che il
te sia per morire; e ' argumento, che egli fa-di cal morte non ¢€ didimile.
ricette. In canto Tofeilo Gianni accomodé una barella, sopr' alia qual
fu posto, € portato da certi contadini ad una villetca de' Signori —
Malmantile in luogo detto Santo Romolo; nella qual Villa trov:
concepi nella mente il far la presente Opera, come dicemmo sopra ne)

wIMBRETT- A. Così chiamiamo guanti, ed altre pelli conciate con
d' ambra.. Ma qui intende, ironicamente parlando, quella materia fetida,

SAPEVA di stantio, Haveva cattivo odore. Quando una materia per
ghezza del tempo ha cominciato a perdere la sua perfezione,si dice /antia; che
se sia carne, o pesce, non da troppo buono edore; e queste si dice payee
tio, La qual voce viene da stanziare lungo tempo, ed e il Latino



sotto in questo C, stan. 54. sith,
VVOL dirci addio, Sc ne vuol' andare. Ci yuo) lasciare, cioè prire.
FLEMMe4. Vmor freddo, e umido che i Medici chiamano in

munemente si dice hemma dal Greco, reid

VVOL! andare 4 ingrafare i Petronciani, Vuol andare a ingraffare gli orti col
suo corpo, facendoli forterrare; e piglia Perronciani (che vedemmo it
tio.C, fan, 6, quello che sieno ) per tutto lorto. E nota che per care la
castroneria di questo Medico, ' Autore gli fa dedurre il. pronoftico della morte
di Paride dal credere, che il suo corpo sia già corrotto, e ridortofi tutto in quel
la terza putrida fuftanza, ed in conseguenza.atto, ed il'calo.a it i
ni; E vuol dire, che Paride morra: Digendosi vulgarmente per intender que
sto U tale andò a ingraffare i cavoli, cio' il tale mori. ¥ oiioen at

CAPO a affiuele, A-uno ignorante si dice Capo di Bue, Capo di Casteones
Capo d' ativolo,¢ simili, ZL' afixolo è un' uccello in tutto simile alla' Civetta, se
non che ha sopra il.capo, alcune penne ritte, che fembrano cornas ©* atobigt

TOSELLO Gianni, Agostino Nelli Gentil' huomo Fiorentino buon:
¢ veramente huomo da bene, Che intendiamo baor figlinale: » SANA GE

COLTELLA. Specie di scimitarra, Arme ches! usaiportare 4 va



a caccia.. 3 0
BARELLA, Aruefe fatto di tavole » che ha quattro manichi 5 serve pet por
tar faili, e altri pefi in due persone; qui intende una barelia da porearesane
d huomiai inferm), o morti,, che è Gimile alle bares o-catalettico i quali si 10
glion postare detti corpi, e da Bara e chiamata baredla.. Vedi oan”
dane 54. SEAN





li ee a


TERZO CANTARE. 14t
to STANZA XXVI.
Fu del Garani ascritto fucceffare Dicon ch' ei nacque al tempo delle more,
 Puccio Lamoni anch'ei grad' ingegnere, Per ch'eglié di pel brunoye membra neve;
Bravissime Guerrier faggio Datrore, Hor qua di Cartagena eletto Duce

Cortigiano, ante, ¢-Tanerniere, i fior de' Adammaganuccoli conduce.
Al Garani fu dato\ per fucceffore Puccio Lamoni, il quale &, Paolo. Minucci.
Il Poeta dice che costui era ingeenere, e Adercante;.ma tali attributi gli (no fin-
ti, perché io, poslo giurare., che egli non fa ne dell' una, ne dell' altra profettio-
ne. Loichiama guerriero, e questo perché detto Puccio fece una campagna.
nell' esercito Pollacco in Pruifia,seguitando quella Real Corte, alla quale era
fiato inuiato dal Sereniflimo Principe Mattias di Toscana alla Maefta del Re Gio:
Cafimiro. B perché detto Puccio godé per melti anni, e fino che S, A, visse,
l'honore di servire. all' A, S. in qualita di Segretario, però dice che era Corti-
giano. Dice che e Dostere perché veramente egli e addottorato in Legge, sc be-
ne per l'applicazione alla corte.5.non esercitò tale professione. Lo chiama Ta;
verniere., perché spesso lo vedeva entrare nell' Olteric, e trattare con Ofti, il che
seguiva perché egli vendeva loro del vino raccolto nei fui beni, € gli conucni-
va lasciarsi rivedere spesso per risquoterne il prezzo. Dice che si vocifera, che
gli nascefse al tempo delle more, Perch' egli è di pel bruno, e membra nere, eficndo
li così in effetto: E facendolo Duca di Cartagena dice, che egli conduce if
ore de' Mammagnuccoli,cioè i migliori, e più valorofi Mammagnuccoli, Questi
M gnuceoli erano una fazione di galant' huomiai, i quali f
profeifione di sapere il conto loro in ogni cosa, e particolarmente nel giuocare,
© pendere bene ildor-danaro,.¢ d' cflere il fiore della reale, ed onorata
apialanite + Havevano un loro capo, che si chiamava Abate, dal quale erano
galtigati 5 quando facevano qualche crrore o nel giuocare,, o nello spendere, ma
però tutto era in galanteria. Le loro adenanze si facevano in cala l'Abate, do-
ve si giuocava a giuochi più di (paflo., che di vizio, e si facevano altre allegrie,
dicene, merende:s ed altrispaflatempi. Costoro erano tutte persone serie, es
quiste se:della pidriguardevole Civilta, e perciò era la lor conversazione molto
bramata,, onde era-pumerolissima; Se bene non era ammefio a quella veruno, che
non haveile provata prima la sua dabbenaggine, e non fuffe stato riconosciuto
dal Abate, e da altri (uoi Consiglicri -meritevole d' essere ammeflo. Fra costoro
era detto Puccio, e perché egli era forse de' più affezionati, i1 Poeta.lo fa loro
Condotticro., e per la stima che faceva di lui nel giuoco delle Minchiate, era fo-
lito chiamarlo il Re:delle carte; perciò lo fa Duca di Cartagena, ed e ancora ap-
propriato, perché detto Puccio per esser di faccia bruna, ha qualche fembianza,
ed ariadi Spagnuolo; oltre che nel tempo, che l'Autore lo aggiunfe a questa sua
Opera, il detto Puccio, era flato destinato dalla Maefta del Re Gio; Cafimiro
per (uo Segretario dell' Amba(ciata di Spagna.
STANZA XXVIL



L' Armatahaveatragli altri unCappellano Faceva da Pittor, da Tiziano;
Dottormailfnofaper fu buccia a Maquat'ei fece main'adava agrnccia y
'Pero ch' egli Pudio col fafeo in mano, Hebbeuna Chicfa, e quiviabiscaaperta
Ed era più bafon.a' una Bertuccia » Si giuoce fnoi soldi dell' hee ©

Pee






142 MALMANTILE: &
STANZA XXVITL







Franconia si domanda Ingannavini, Lelie havea in casa it
E fu a come il pie valente, Già fatta una lerione,e falla a
Perch' eghs fapea leggere i Latini, Subito accetta y € fiede in alto
A far quattro parole a quella gente', Senta mettervi fu ne fal',

Fra gli altri Cappellani, che erano nell' Armata, era un Dottore y ma dig
scienza; perché il suo studiare era stato il darsi bel tempo. Fu feolare d
re nella pittura, ma impard poco, e se bene fipprefumeva diaper'
fece mai cosa, che non fufle stroppiata. Fu Rettore della Chiefa di Petriolo;
Villaggio vicino a Firenze circa due miglia,¢ perché egli era huomo allegto,
di conversazione, dice che egli ff gidscd fino i soldi dell? Sa} ed intende che co
fomava tutte le sue entrate in allegrie. I suo nome era Franconio th
cioè Giovannantonio Francini, A questo dunque » come al più dotto fu fatta'
stanza, che facefle un poco di discorso a quei Soldati, ed'egli che “haveva' ums
tempo fa recitata una lezione nell' Accademia del Coltellini 5 e 1 a ace

¥ j Wil

a memoria, si content® di fare quanto gli era stato imposto, e senza
tempo in mezzo montd in pulpito.

BYCCTA buccia, Leggicrmente. Cicé sapeya poco; non haveva gran fonda
mento; che si dice anche s# pelle in pelle. Vedi sotto C, 8, stan.58. edi
dissero superficie renus. we ri

PUL buffone ad una bertuecia. Huomo arguto yallegro, © facetoyBaffune die
ciamo colui, che tiene il popolo allegramente con facezie, e moti, &
Scurra, Vedi sotto C, 11. stan. 42. E Sertuccia diciamo la scimmia,

TIZIANO. Pittore celeberrimo. Econ dire facea da Tiziano; intende pet
antonomafia, che egli si prefumeva d' esser il pi valente Pittore del Mondo.

QVANT ' ¢i facea,n' andava a gruccia, Tutto quel che egli faceva rap
piato, cioè mal fatto, mal dipinto, Vedi sotto C. 11. stan. 41. mete

BISC-A. Luogo pubblico, dove & permefio giuocare a ognuno; Egiwecsre#
bisca aperta, vuol dire Giuocar sempre, e senza riguardo alcuno. re

JL Coltellini, Questo & il Signor Agofino Coltellini Avvocato Fiorentino huo-
mo dotto, ed amatore de i Letterati, il quale in molte opere com, da tui si
chiama col nome anagrammatico Oftilio Contalgeni. In casa di eflo @raguilas
l'Accademia degli Apatifti da esso fondata, nella quale si fannoidifeorfi Acad
mici,ed altri esercizzj virtuofi: Mirabile per haver saputo far durare per lo (pa
zio di cinquanta, € più anni la detta Accademia, (empre in florido, cosa inl
lita a' nostri secoli in questa Città. Lntcrueniva spefio in detta Accademia quell
Francini, ed alle volte vi faceva qualche lezione; nelle qualimofro i suol
ed eruditi talenti 5 e f¢ bene l'Autore dice che il (uo sapere fu buecia it, OM
to lo-chiama huomo (caza forndamento, non è però, che egli fulle tale 5
gli huomini de' nostri tempi non era dei secondi in dottrina non meno '
che profana; ed era veramente Dottore di legge. aim

SENZA mettervi fu ne fal,ne olio, Presto, iubito, senza replicare, «o' mettet
dificulta, Nulla interposira mora. Fu un tale, che tornato la feraa cafa'y difeal

suo servitore: Fammt una infalata, e fa presto, ch' io forv aspertato, e noms
yoglio mangiare altro che quella; fa preito. dico. Ll servitore eee






i
ai



TERZO CANTARE. 143
senza condire la portd in tavola al padrone'; il quale ciò visto lo fgridd; Ma il
servitore rispose; Signore per servirvi presto, non vi ho meffo fu ne fale, ne olio,
E da questa goffaggine del servitore viene il presente detto, che significa Fare una





cosa subito 5 e senza considerazione.
+ neue toh Seca Se TAAN ZA.XEIX,
Sale in Bigoncia com due torce a vento, Che ben si (corse in Ini quel fondamito,
eicio lo wegga ognun pro tribunali, Che diede alla [un casa Giorgio Seali,
Ove, mostrar volendo il [uo talento, E piacque si, che tutti di concordia
Fece un discorso, e fece cose tali, Si meffero a gridar: mifericordia,

Il Poeta continuando, a voler mostrare, che Franconio fufle di poco valore,
¢ che però il discorso da Jui fatto futle scimunito., e senza alcun fondamento, lo
burla, e dice che piacque tanto, che il popolo, si messe a gridar mifericurdia; del
qual termine ci serviamo per mostrare, che qualche cosa ci sia venuta a fastidio,
come per esempio. Ei duré tanto a discorrer, che mifericordia, Disse tante feiocche-
rie, che mifericordia, Ob mifericordia,, quanto volete voi durare? Quali dica, hab-
biate mifericordia, e compaffione di noi, e non ci tediare più,

BIGONCLA. Eun valo di legno, del quale si servono 1 Contadini in tempo
di vendemmia per pigiarvi dentro ' uva, prima di metterla nel tino, e ce ne serviamo anche in altre occorrenze, come di portar' acque, e simili,

» Hi Bini nel Capitolo del Pilo dice;
Viua dir, che se ben' ella il più mi deffe,
ind Ed opraffi,( non ch' altro) una bigoncia,
srobe; Ognun direbbe, che ben fatto haveffe.
eo Epperche fo. vaso demo Bigoncia e molto simile a una cattedra tonda,però
da moiti tai Cactedra: si chiama bigoncia, come anche tutte l'alere cattedre. Il
Davanzati ne} suo Cornelio. Tacito postilie al 2. libro num. 18. dice: Arringa-
vans i nostri antichi al popoloin piacza, in ringhtera, e nei Consigls in bigoncia, che
era un pergamo in terra a fogcia di bigomia.

TORCE a vento. Torce grofie che si fanno di funi di cotone filato attorte per
servirfene a far lume la notte per le Arade; e si dicono 4 vento, perché refiftono
alivento;¢ a-distinzione di quelle, che si fanno a Venezia, che per esser gentili
si spengono a ogni poco di vento. E Torcia, che da i Latini e detta fusalia, fu-
natinm, viene'a noidal Francese Terche

CHE diede alla [un casa Giorgw Scali., Giorgio Scali fu in Firenze an riputatif-
stimo Cittadino Popolano, it quale nelle diflenzioni, che seguirono a suo tempo
fra i nobili, e Popoiani di Firenze, si fece capo di questa parte, con promessa, e
speranza d! esser follevato a cose maggiori, cioè all' affoluto dominio di Firenze,
ebenché per altro accortiffino, e prudentiflimo, lasciatofi portare dal dolce de-
fiderio di domiaare, si fido nelle vane promesse della instabil plebe, con la qua-

lep id haver forze bastanti per conseguire 1' intento, s' accinfe all' ope-
ra; ma nel pil bello il popolo,.o spaventato., o pentito.l'abbandond, ond' egli
venuto in potere-del Governo fu decapitato: Eda Ini e detto il Proverbio: Far
come Giorgio Seali, che vuol dir Pigliare a far' una cosa senza fondamento, che i
Latini con similitudine della Scrittura., dissero Scipione arundineo inniti, Di que-
flo calo di Giorgio Scali parlano tutti gli Storici, che scriveno le cose di Firen-
2



var”
144 MADMANTOELSST 9
ze digquei tempi, ed il Nerli fra gli altriaggiunge, che allora'¢omincid
proverbio.; fancy Sutahogwies

STANZA XXX, i
Li tema fu di quefia sua lezione, Così, dicea, la vofirase mia,

Quand' Enea già fuor del suo pollaio SV Quis viva, e fanaye della 2
Paceta andar in'fregola Didone, » » Cacciara fu dal empia Mm








Com! una gatta bigia di Gennaro;~ » Tredira ancl ella fuor ¢

E che se i Greci ascoft in quel ronzone Pere sun rantoardire', etal 1

In Troia fuoce diedero al pagliaio 5 Parvi, @ adefo gaftigar si

Ein man a Enea posero il lembuccio, ¥ bavete il modo senza cht a

Ond' ¢i fuggi col padre a cavalinccio; to ha finito, 1) Ciel vi benedica
Il tema del discorso,, che fece Franconioy fu quando Enea essend:Soggiand
Troia fece innamorar Didone, 'ed aflomigliando:Celidora cacciata di Malman-
tile ad Enea scappato da Troia, esorta quei soldati a gafligar Pardire di Bente
nella, € rimettere Celidora nel suo ttato, già:che hanno il modow
POLLAIO, Si dice da noi quella:stanza, nella quale anno, edo
li: E chiamiamo pollaio quelle felue, o macchie, dove la sera vanno gli uece
a dormire; Ma qui intende per translato la nofira Casa, Patria yorluogo, dove
fiamo soliti abitare. > dog Gat
ANDARE in fregela, Dicemmo quel che significhi sopra-C. 1. stan. 25, Mas
che Didone fulle innamorata d' Enea, come favoleggia Vergilio, &
ché oltre che Didone fu così casta, che vedendosi violentata da Iarba f
ritania a rimaritarsi seco, volle più tosto da se-stessa uccidersi, che il
suo morto marito Sicheo con nuovi sponfali'; EB' anche vero, chene
seguire il detto innamoramento, perché Enea fu 360. anni prima di



verita si cava da diversi Autori, e si (corge in Darete Frigio', e Ditti a
che (crifflero la vera Storia dell' eccidio di Troia. Che il nostro ic
ti gusita bugia-di Vergilio, dicendo nell' Inf, C, 5. “aie a
Li altr' e colei, che s'ancife amorofa, y me
E roppe fede al cener di Sicheo, se gantiel 2

Non è meraviglia, perch Dantes' era eletto per suo Maeftro,  guida Vergilid.
Che Enea fude tanto tempo avanti a Didone, si deduce anche songs
Didone fuggendo l'infidie di Pigmalione suo fratello, che per desiderio i
le haveva ammazzato il marito Sicheo, come pure accenna' Dantes Parg, C.20.
Noi ripetiam Pigmatione allotta, onda
Cui traditore, e ladro ye patricida toe illid
Fece la voglia sua dell oro ghiorta:, ». Seip 5
Portandofene il teforo in Affrica, chiefe a quegli abicatori tantovdi tert d
to poteva circondare una pelle di toro, cl' ottenne; Bd altutamente
detta pelle in firisce così fottili, che abbraccid con esse tanto terreno,
fico Cartagine, il che fu dopo 70. anni della edificazione di Roma » 7
edificata cirta 300, anni dopo la morte a' Enea, Sant' Agostinodifein di It
done, che quando Vergilio non fufle stato dannato per altro, i i
no per questa falfita cotanto pregiudiciale alla ripucazione di Didone sl gl?
difende ancora Aufonio col seguente Epigramma tradotto:dal Grecow o ©



Ad








TERZO CAINT/AA RE:

145






2 S.\Ad Didus Tmaginem CXI.
Dida 5 nis guiase con/pici '
Seances (ase modis ipleagen aise sy

9») Talis cram, fed non Adaro quam mibi finsit erat men:,

|. Vita nec inceftis Leta cupidinibus. ©

Namque nec dneas vidit me Troius unquam,

dig Neo Lybiam aduenit Clafibus Miacts;

L fens 5 atghe arma procacis Larbe

94 > vmorte pudicitiam

nificco 5 caStos quod pertulie enfes
5 wut lasa crudus amore dolor 2 *



nA ot Vita virnm, posiris meenibus oppetiy..
Jnnida cur in me stimulafti musa eMaronem,
Fingeret ut nostra damna pudiciria ?
Vos magis Hiftoricis lettores credice de me,
SL Quan qui fart Denn eonoubitufqne canunt;
bates Vates, temerant qui caPmine vernm,
: Humani[que Deos affiemilanc vibijs,

GATT A bigia Er quella, che noi chiamiamio'Soriana, che è un mifto di co-
lor bigio, € lionato ferpato di neroy-qual colore foriano ff dice solamente di Gat-
ti, onde io argumento, che'i primi' gatti di questo colore veniffero a noi di So-
ria, come vennero alcuni anni addietro quelli del colore del topo portati da Pie-
tro della Vaile dalla Persiaye petd da molti chiamati Persianini,. Vedi sorto C. 9.
stan. 19. 9) oe; a:

RONZONE; Conia jz;-¢ruda vuol dir Cavallo stallone', o per la monta, da
i Latinidetto eguus admiffarins':¢ per ronzone,-ronzine, o rozza jntendiamo
eavallo cattivo, Ronzone'con la, z, dolce vuol dire una specie di Moscone 50
tafano. Qui} Autore intende quel cavallo di legno fabbricato da j Greci per in-
gannare i Troiani come dice Vergilio. In. alcum Tefti si trova scritto caffone in»
vecedi rontone,. ma nel mio, che e di mano dell'Autore, è seritto ronzone,
PAGLIAIO, BE” proprio quel cumulo, o massa di paglia, che si fa dat Conta
dini dopo haver battuto il grano, per lo più avanti alie case; ma dicendosi dar
fuoco al pagliaio, s' intende Dat fuoco alla Cala.
PORRE il'lembo'; o él lembuccio in mano, Significa Mandar via uno; E questo,
perché quand' altri vuol mandar via' uno di qualche luogo senza parlare, gli fas
il ferraiuolo addosso, e gli mette un lembo di eflo ( che /embo vuol dires
'na parte dell'estcemità del ferraiuolo, o d'altro abito, d yefte simile ) nelles

« Sic eecidiffe imvat; Viet ine vninere fama;

mani; e da questo 'colui's aeoees d? esser licenziato, efiendo notissimo, che»
uclto detto Pigtiare', o-dare il lembo significa Eller licenziato; Tratto dai mac-
ri delle bore #'i quali, volendo licenziare un garzone, gli dicono: piglia il
lembo; piglia il'cencio, ec. e intendono Vattene,
ef CAFALIVECIO.. Cioè in fw le spalle. B ndidiciamo portare a cavalluccio
da un giuoco, che fanno i nostri ragazzi in'questa forma. Vino mette il capo fri
le gambe all' altro per di dictro, e Sue così da terra, lo porta fra le spal-

le,



=. ay


146 1 MALMANTLELBE
'ce, cil collo, e per questo fii dice,a cavalinecios, Tir
cevano lo diceyano è coryla. fac Y
sopr' alle palme delle mani del portatore rivoltate dietro.

non accavalciava le gambe al collo, come fanno.i nostri, «
teneva al collo del portatore; € lo dicevano ix cotyla io
pt







mano di colui, che portava, come si cava dal Buleng..
da Cel, Rodig, le, rig lib. 27..cap.27. E questo.era.
una pena data a quei fanciuili, che baveano perfo
giochi, che habbiamo accennati sopra nel 2. Cantare »..B.
modi, con li quali portavano 5 così erano diversi i,nomi,.che dava
giuoco; perché si trova chiamato Cabefinda,ed Hippas, si come si ved
Polluce lib. 9,. 7. Che questo giuoco fufle,usato anche dai L
re da Vergilio En. lib. 2. il quale-dice » che Enea. portd il Vi
padre in fu le spalle in tal muaniers Lawn seta se N

Ergo age chare pater ceruici imponere

Ipfe [ibibo bumeris, nec anidodhearaen A ikea
allegro,.¢.con buona





















DELLA buona vozlia, Intendiamo fano.y,

Lalli En, Trau, lib. 1, stan. s1..diffle. >

 Stanne, diletta.mia, dibuana.

Parafrafando Vergilio, dove dice: Parce merx, E noidiremmo: 4

EVOR di questa foglia, Cioè fuori di Malmantile, Pigliala s

parte di sotto della porta, per tutto Malmantile; 0. intende foglia

reale.. aria. dgis: B4ine 5 Oa:
STANZA XXX Mica civ STANZA X

Poiche da esso inanimite furo caviond ince
Leschiere, si portaron a itor posti,

E già fdraiaco ognun laffo,e maturo

Ingremboalfonnogliocchihavevapeffiy Mojtrando wwoler farne /
Quanda un trattoletrombe,ed il taburo Segui c-un' Vfirial [uo







Reppe i riposf, eifonni appena imposti; 'Che più d'agn'altre men

Safi, presto cos) aoe fracasso, Tocco la redone 4 suoi intern

Ch'il fiatonitrombettier feappo da baffe, De' tamburini ye tro bert
Dopo che Franconio hebbe dato animo a i soldati ognuno andé,

¢ già tutti stracchi s'erano addormentati, quando in un subito fu dato:

be, e ne i tamburi, che fecero svegliare tutta la soldatefea; ma.

presto celsd, perché i trombettieri, e tamburini.lasciarono star di fonar

paura, che hebbero del Generale, il we entrato in collera di così gran

giurd di voler gaftigar colui, che era flato il capo di al follevamento, € 10:

dd ad effetto, facendo dare la corda a uno Vfiziale suo favorito, ch

farebbe mai aspettato, e gli fece mettere i tamburini, ¢i i

' SDRALATO, Diltelo con comodita. Voce da,

confolazione, che'fente uno, che sia stanco a distendersi

eratamente. Vedi sotto C. 6. stan.26. E non crederei

di Cerbero, paraftalando Vergilio; dove dice





Loo his 1D. Oe a co si bs > te os ee Cs




}
i
i
a
i
A
@
ai
y
-



TERZO CANTARE: 147
6  teqce immania tered refoluit
ue a Fufies bumi, toroque ingens extenditur antro. xq

'A Vivato, In un subito. E questo termine @ an rrarto fighified anche tutti
due,0 più alla volta, e si può intender, che le trombe, ed i tamburi, cioè uno,

eee fiato da baffo at trombettieri, Cafeare il fiato vuol dire Haver paura,
o timore; onde con questo dire intende, che i trombetticri hebbero paura' det
een. di fonare; non perch veramente perdeficio, o
ulciffe loro il fiato dalle parti da baffo.

YNCOLLORITO... Adirato-. Entrato in collora.

OCCHIO true, Frase latina; usata da'noi, e significay e mostra lira che»
uno habbia;'¢dicendosi: 11 tale mi guarda 'con mal' occhio, o con occhi torti,
s'intende il tale @ adirato meco - Hec autem toruitas a taurorum ferocia dicitur,

MIN ACCIO! col dito', Coloro che vogliono gaftigare qualche delitto, o ven-
dicarsi d' alcuna ingiuria, sogliono brandire il dito indice verso quel tale, che
vogliono gaftigare, e tal brandimento si dice minacciare dal Latino Afinari, o wi-
nitari, sur:

CHE più d' ogni altro meno se l'aspetta. Per esser questo soldato amico, e molto
in grazia al Generale; non havrebbe mai ¢reduto, che egli l'haueffe a gaftigare,
TOCCO! locorda, In' Birenze danno la corda legando il paziente per le mani
legate insieme dietro alle reni; e per quelle ran @ un grosso canapo, che
paflaper-una carrucola, tirano il p in fu, lasciandolo leggier: feor-
ter in git, e poi ritirandolo in fu'tante volte, a quante e condennato, e questo di-
ciamo; dare rratti di corda. Qual tormento da“1 nostri antichi era detto dar /a,
volla, o collare, ©'noi diciamo: dare la corda. Soggiunge poi: Co' /uoi intermed) di
tamburini, e trombettieri a' piedi; cio con tutto quello che ci andava; il che era,
che i tamburini, e i trombettieri, i quali erano stati complici a tal delitto, stel-
Bp ys! Hr lui affiflenti a vedere efeguire la piuftizia, come si cofluma,
quando molti (ono complici d' un delitto, per lo quale vien gaftigato feveramen.
- te il capoy 'ipale, € gli altri complici ricevono minor galtigo,, ed adiftono a
\\wedere iligaftigo del 'loro principale, Io però non sono Jontano dal credere, che
il Poeta per foftenere 'a faa Opera sempre in si le burle., habbia -yoluto in-
5 che i camburini; e trombettieri fussero essectivamente legati a i piedi di
coini, che era tirato sa, e voglia mostrare con questo il costume, che si tiene ia
Firenze di legare.a' piedi di tali pazienti qualche cosa, che significhi il delitto da
Jui commefio, acid che il popolo comprenda la cagione di quel martirio, come
per efempia:;a un fornaio, che habbia fatto il pane cattivo, o di minor pelo del
dovuto, faranno legare a' piedi un filo di panc, e così gli daranno la corda:e mi
la(cio indurre a cia 5 che il Poeta habbia voluro intender questo, dal vedere,
che: egli: nel? Orrava seguente dice; alla corda onole che sia attaccato così: i qual
detto pare che esprima, che il paziente debba toccare la fune co'i trombetti, ¢
tamburini legatigli a i piedi,:





T2 STAN-




















148: methathethadhalet ities

STANZA Pepe ake:
Alla corda così vol ches' attacchi 5 vstene
Perché a arbitrio ye senza consigtiarfiy aed di
Facea venir all! armtiallor che
Bisogno havean più di riposarsi,
Ed eran mexxs morti; e come bracchi 5
Givano anfando inordinati  esparsi,)
E con un fuor di lingue,e orrendavista -

Sofiavan,ch'ioho spp un Alchimifia.

ll Generale fece dar la corda a quell' Vifiziale non | 20 flo, perch
fol' arbitrio di far dar' all' armi (enga il suo consenso ma \ancora
u(cito fuori del concertato, il quale era di osservare prima dim;
se le flelle presagivano buona,o trista forte, B qui il lettore fir
sta in fale burle, e sappia, che l'Autore non flimayva che I aft
a tanta precognizione, ma si bene, che idabeaue faa Sepnarig sites

ifti, injilisa 6

D ARBIT RIO xe propria eerrefia Saonahe lo stesso; ed  ambesie si
Di suo capriccio, o volontà...) “th
cANS ARE, E} quell' impeto 5 o.romore, che fa il relpiro 24
il - ( che noi pure dal Latino diciamoanelare ) e vient a

aCCO, Cane per uso di caccia, il quale quando è stracco re

ae, e tiene la Jingua fuori; B se bene fanno così tutte le sp
nostro solito far questa comparazione /olamente ai bracchi
meate sono pil fortopostia fracear hij i percio che Rimolati a
di trovar preda, fanno maggiore; e più violento viaggio che
fio Sat, 1. Nec lingue quantum fitiat canis Appula tantum.
ORRENDA vista. Vista spaventevole;, che tale € il veder vat
bocca aperta, e con la lingua fuori, perch Pee: lo pil restano in
gl' impiccatiy. 9 a

SOFELAV-AN ch' io-ho spoppare we Alchimifta. Alchimiti.fon'
fiano nel fuoco per trovarl'oro, e senza nominare Alchimiftay
sale fofia s' intende,, e Alchimifta, Se.bene s' intende, e Bada
cennammo sopra c 1, Nan. 37. anzidicendoli Ji cal. fag, Me bimiffed
tale fa la spia, e tutto è fondato sul verbo fofliare » che signitica Far

10 ho foppato Significa io repre meno, 0.io non 'timo punto il
fanno gli Alchimifti in paragone di-quello, che foffavano questi
stesso significato, che il termine ne ferede dss pra Ga, 1068 ns
mo fotro C, 6. stan. 61, ate

TAMBVSS-ARE., Rerquotere y aie dle bl Biparola gei pr
macellari, che dicond Tiamibaffare qu: se
perché la pelle si spiochi bene salislivaats © ae anc
dremo sotto C. 11, stan. 26, E tutto ha Origine dal.ta





che fa esso,s' aflomiglia al romore, che fanno i macellari 5 '


ase

SAAR RERERAEEORE ERGEE CE Set

-

TERZO CANTARE- 149

STANZA XXXVI.
Hiomai la Bama, che riporta a volo
| Diagn' ii nnove, elegazzette,
 Spargeper Ma ilche ar mato fiuolo
View per tagliare a tutti te calzetre 5
Gig molti impauriti, ein preda al duolo
Non piie co i naftri legan le foarperte,
Macon buone,, e faldissime minuge,
Perché stien forti ad wn cumores tage.
STANZA XXXVIL
In tal confufione, in quel vilume y
| All udir queilamenti, e quegli affanni
4 molti cht eran già dentr' alle piume
Lo shucar fuori parue allor mill anni:
Chi per vestirsi riaccende il lume,
Pera ch' al buio non ritrova i panni 5
«Chi nudo foappa fuori,¢ non fa stima,
Che dietro gli sia fatto lima lima.
Sparfo per

ntile 1" avvilo dell'

STANZA XXXVIIL

 Rerché s'egli ha camicia, o brache,o ve/P4,

Non bada che gli facciano il baccano;,
Ben si del sriffe avuifo afjlitto refha
Onde più a' un poi giuoca di lontano 5
Chi torna indierro a fasciarl la testa y
E chi si tinge con il raferano,
Chi dice, e una dogisa fegii e pref »
Per non haver aire a far dife/a.
STANZA XXXIX,
Altri, che fugge anch' ci simil burrasca,
Finge l' wnferamo, e vanne allo spedate,
E benchi [ano ei sia come una lasca
Col medico s' intende, € col /peziale,
Perché alt'uno, edall'altrocpielatasca,
Accio gli faccia fede ch' egli ha male;
Ed essi quefho, e quel servvon malato,
E chi piis da, do fan dt già /paccsato.

-Maimai arrivo di detta Soldatesca, gli abitatori
s' acciafero pil al fuggire, che al difendersi. Narra il Poeta diver-

-diquel lnogo,

Sen tale spavento, e le varie scule, ed invenzioni, che rrovano coloro
sper nion hayer adandare alla difefa della muraglia,.

“ GAZZETTE., Novelle, Avid, Carte d'-avvili. Egazcetra diciamo anche

-lacrazia,Veneziana.

TAGLIAR ie calzetre. Tagliar le gambe. Es' intende, dare delle ferite in.
pore del-corpo, se ben le.ca/zere non vestono se non le gambe: Come
jamo anche rompere ia tetla, ed intendiamo Ferire il nimico in quelle parti
idel-corpojeherci. verra farto.. E diciamo fiaccar le braccia a uno con le bastonate, se

bene:in ogni altra parce

giidaremo che nelle braccia...

NaST RO. Et una specie di tela, o benda che non eccede la larghezza d' un
feltordi-braccio., eferve'per legare, o falciare sda i Latini però detto, Vitra, ed

in alcuai uogbi d' dralia derco fersuccia.

MINVGE,. Corde da strumenti musicalicome Tiorbe, Liuti, ec. fatte di bu-
<dellaidi bestie;'¢ pero Dante Jaf. c. 28. per intender budella disse,
wich ili \Tralegambe pendevan le minugia, i
-\) Diceche non 6 fonodegare de scarpe coi naftrt,.ma con le minuge, perché fo-
no pil fode, e da refifter pill; Ed e costume nfatissimo il dire: // tale sera dega-
16 le fearpe bene', o.conile minuge y per intendece Correva forte, o volava: fuggen-

«do i pericoli, cheicidiintende con

quella sentenza;, Rumores fuge »

- CONEYSIONE,,¢ vilume., Sono in questo uogo quaG finonimt havendo lo
stesso significato di Viluppo, imbroglio, ec.

DENT RO alle piume

Ji quando -vogliono

FAR lima lim Define, dicggiate
AR lima lima, Beffare, dileggiare. |
dar la-burla AUN 5

un modo proprio da Fasciulli, i.qua-
si fregano al dixo indice fapra l'indice

« dell' alta mano.a guifa'dicoloro cheJimano.s ¢iNoltanioG verlocolui, che ve-

va







150 MALMANTILE a

glion burlare dicono. Zima, lima. Vedi sotto C. 9. stan. 66. annot; ©
WON bada. Non cura; Non offerva, Non gl' importa'; Il verb
vuol dire oflervare, ha più significati, come Attendere,
ligenza, curare, stimare, ec. Bada a tuoi negozzi. Badaa
viene. In somma hala forza del Latino Cwrare, Vacare: si dice: 7
bade, per intender Trattenerlo. Star a bada d'uno: per intendere
do I opera, i favori ec. d' uno. ae FF
BRACHE, Calzoni. Brache da noi propriamente si dicono qu
ghi, che usano i Soldati a piede Tedeschi guardie del Serenissimo Gri
1 Paggi nobili. B si dicono taluoita Brache quei calzoni che si
chiamati ancora Mutande; Vedi sotto C.6, stan.z0, ©
FAR il baccano, Qui vuol dir beffare, dileggiare con fischiate'yo st
mili; ed il suo significato proprio è Fare strepito, far romore e viene:
nalia, ve
GIVOC A di tontano, Cioè non s' accofta: ed è lo stesso'che Parfene
che vedremo nell' ottava seguente cook ta
BVRRASC A. S intende propriamente il travaglio del mare; ma 10:
per ogni sorta di sturbamento, o pericolo. Forse meglio borrafea:


























SPEZIALE. Colui che manipola, e vende medicament; e però
detto Pharmacopota; ed altrimenti -dromatarius da aromata ye noi lo
giale da spezicric, come si trova anche in Latino aeenaae
TASCA, Scarlella, che e un facchetto appiccato ai 3
uso di tenervi dentro quello, che occorra alla giornata, e particolarm
ri; ¢il Latino mar/upinm, Ed empier le tasche a uno, vuol tie Dargli
naro. BN
LO fanno spacciato, Cioè dicono, che egli è in grado di morte,
ta, che i Medici regolando le atteftazioni delle infermita con le fomm
nari, che erano lor date, facevano fede esser in grado di morte q
ne dava; e quel che ne dava ri atteftavano, che era leggi¢rment
STAN xX. 7
Si che con queste finte, e con quef arte D' uno fheffo voler La maggior parte
Coffor,c! usan la taxxa,e non latarga, Trovan la'via di sparfene'all vd n
Servir volendoa Bacco,enona Adarte, Ed il restante non si afbute 7
Che non fa sangueyma vnol che fisparga, Compari/ce,perché ei nom puo far alt
Questi abitanti di Malmantile con tali feufe, ed invenzioni cercano di for
si dail' andare alla guerra, ¢-folo vi'va chi non ha danari, ne da!
berarfene. ' 5 ae (eae
TARGA, Brocchiero,Scudo, Rotella'. Intende 5 che son più avvezai a
re che a guerreggiare, ed-hanno più genio con Bacco 'Re del vi ~
hanno con Marte Xe delle guerre; percné quello fa nascere nel carpo il
equetto lo fa disperdere. ' 1b coreakant
ST ARSENE aa larga, Significa non s impacciare d' una cosa, ed è.
che gixecar di lontano, che vedermmo nell' Ottava antecedente. © 5
eASTVTO,¢ fealtro, Sinonimi di fagace, ed accorto, Huomo, che fa il con-
to suo. Ma per maggior intelligenza di queste parole tute 5 e /eakro, ee

















ds)
eile

“

er ihe



'TERZO CANTARE} 352

td dccorto & da sapere che, se bene ce ne serviamo per finonimi, tuttavia ci ¢
qualche differenza; particolarmente fra,/agace, ed a/Puto; perché l'arti, che dal-
la fagacita s' adoprano, non meritano biafimo, per non esser se non avvedimen-
tivortili, ma schietti, reali, e senza fraude, o ingaani: El afuzia oltre alles
suddette lodevoli arti si serve anche delle menzogne, fraudi,¢ falfira, e d' altre
cose poy scarslepgs nobile. E però Scaltro, ed accorto pare che meglio s' adat-
tino per finonimi a fagace, che ad a/furo, al quale più proprio finonimo farcbbe
Maliziofo, o tristo, o furbo; quando però la voce furbo e presa in fealo.d' huo-
mo, che fail conto suo; Ma, come ho detto, ncl comun parlar civile nots
usiamo così efacta com »¢ puntualita; ma pigliamo l'uno per l'altro.

eee men Xl, ree' XXXXIL

lentr? in piarza si fa nobii ar fa 5 Vilta  arretra, honor ds poil innita

Auch in Palagzo Meda tans A cimentar la sua ee in guerra,
Con unatreccia avvolta,e Paltra/par/4, L cfortal' una a conseruar la vita,
Corre alla Mdaimantilica rovina; L' altroa difender quanto puolaT err.
Benche ne i paffi ape vada più fearfa, Pur fatto conto di morir vestica
Perchi all' uscioda via mais' avvicina; Voltoffi a bere, e divenuta sgherra

+» Da ferte volte in fu già s' e condotta ( Pero che Bacco ogni timer dilegua )
Fino alla fogha; ma quel faffo scotta. Dice:O de'mici:cht mi vualben mi segus.

« Mente che la men codarda gente si raguna in piazza, anche la Regina Berti-
nella. al romore, nuova Semiramide con i capelli non ancora finiti d' aggiufta-
xe, correa difender Mabmantile; ma non con tanto ardire, perché questa nostra
Semiramide non s' arrischid così subito a paffare la porta della Casa; ma si fermd
in quella, sospefa,¢ travagliata da due gran paffioni Poltroneria, ed Honore,
che quella J' esorta a starfene; e gquesto Y obbliga ad andare, Al fine lasciataG
persuadere dall' Honore prefe animo, ed esorto i suoi a seguirla.
go MT RECCLA. 1 capelli delle donne si chiamano #recce, perché per lo pil foglio-
no le donne far due parti de i lor capelli, e ciascuna di quelle fuddividere in tres
altre parti, ed inteslerle in terz0, il che si dice treccia; B Bertinella stava così
Antrecciandole, quando senti il romore, per lo che lasciato il lavoro corse cons
una. parte intrecciata,¢ l'alcra nd, comic dicono, che faceffe Semiramide, quan-
do senti il pericolo, che fouraftava a Babillonia.

MA la foglia scotta, Quando uno o per debiti, o per delitti fla ritiratoin ca-
fa, o in Chiela, diciamo: Nor esce, perché la faglia feotra; cioè se egli ulcitie di
cala., o di Chiefa, farebbe fatto prigione: ed a Hertinclla scorra quella foglia,
perché se ulciffe di quella, pericolerebbe di toccarac.

VILT A', Qui vale per poltroneria, o codardia.

MORIR vefiite., §' mtende di coloro, che sono ammazzati, i quali muoiono
con le vesti in doflo, e però dicendo che fa conto.di morir vestita, s' intende che
ella ha risoluto d' andar a farsi ammazzare.

SGHERRA, Braya, Animofa; fatta così dal vino, che leva di tefla ogni timo-
re. Bacco dai a fu detto ibe, rch liber l'huomo da i pensieri noiofi,
€ pero dice ogni pensier dilegua, ed il Chiabreca disse.

ia ssi oe 9 e dianfi al vento
4 vorbidi pensicré,





192 MALMANTICE! 27

Seneca de Tranquillit. disse; 2vonnunguam ad ebrieracemv
gat nos fed ut deprimat curas = Elevat enim curas,@ ab imo:
norbis quibufdam, ita trifitie medetur, Di tr
Generale del? Imperadore Ferdinando 2., il quale mai si
glio di — » ne si mefse ad impresa alcuna importante,
molto bevuto. E Bertinclla imita questo gran guerriero }

STANZA XXXXIIL ' STANZA XX
Dietro a suoi paffi metteft in' cammino Piaccianteo uo se
“Maria Ciliegia illuftre damigella; t
Tutto lieto la segue il Ballerino, ci a
Che canta il titutrendo falalella. Ed e la diffrurion










Va Meo col paggio, Zoppica Atasino, Già miftro le doppie con la

Corre il Maffelli,e il Capitan Santella, Finiro poi che fu quella bona

Molti,e molt' altri amici la seguiro, o omtagit

E pie Mercanti channo havuto il giro, Ed hora in Corte ferne a

Alle voci, ed ordini di Bertinella obbedirono diversi suoi seguaei'
Matti. ses Vie

MARIA Ciliegia, Fu una Donna creduta pazza, la quale andava'
ricevendo elemosina senza domandarla, Costei con una flemma,
ordinaria difeorrendo sempre da per f€, diceva belle, e fenfate

de da molti non era flimata pazza, ma uguale a Diogene, che abitava
te; e per tale azione farebbe stato riputato matto, se non have
belle Eiietac Ȣ dogmi, come appunto fece questa madonna Mai
la quale, o parte di essi sono staui raccolti da un buon fetterato 5
volta gli dara alle Rampe: Come Diogene,anch' essa non si curava
dormiva nelle frad¢ sotto qualche portico o loggia, e perciò porta'
pre un granatino per spazzare quel luogo, dove si metteva a do
spazzola per spazzolarsi la veste, la quale benché poverifiima, era
molto pulita, e se bene piena di toppe, atiai bella per efservi le n
mefie forte anche senza bisogno, con vago, ed aggiuftato ordine.
ta sua sporta haveva ancora qualche biancheria, e molte volte un
danetto pieno di fuoco, nel quale, pa/seggiando per le rade, ai
le sue vivande; sotto la gonnella haveva pil facchetti, entro ©
pentola, e piatti per suo uso, e quello che le avanzava a' suoi man
va forelle, e nipoti i quali si trattavano comodamente, ed -habitas
buona caforta, che era di detta madonna M aria, dove ella alle volte
mutarsi; ma non volle mai fermaruifi, ne dormirvi ancar che prega
ta anche da' detti suoi parenti-a volere flar con loro, Bu(cava mo!
li quali comprava quello, che parcamente le bisognava, ed ogni
va per ? amor di Dio tutto quello, che le avanzava, e per lo pi
nache, dove alle volte port® anche fino a dieci fendi, Domandata'
qualche parere, non rispondeva; ma seguitando il suo solito chiaceh
ma, che quel tale i partifse-da lei reflava appagato con qualche fencenaa,
to, che ella diceva a proposito del quefito. Per esempio: Vna mattina,
clia (otto le logge d' avanti al Tempio della Santissima Annonziata, un
























Bmnwceocacow =~




TERZO CANTARE.: 153
wie netto le domandd,se ella credeva, che la sua moglie bella, da madonna Maria
we, molto ben conosciuta, fufle honefta; ma gliclo disse con la più sporca maniera,
: che dir si le. Madonna Maria senza alzar la testa, o dar segno d' attenzio-

ne al del giovane, seguitando il suo discorso, che faceva del poco rilpet-

to, si portava alle Chiele; dopo molte chiacchiere disse; Vedete voi questo
oe sboccato, il poco rispetto, ch' ei porta alla Chicfa? La sua moglic &

la, e la prefe, che ella era onefta; ma che puo ella havere imparato da lui, se
non il modo di diventare altrimenti? ed hora io ho, che ella sia diventata; perché
ogni gelofo e becco. E seguitd il suo cicaleccio, entrando in diversi altri gine-
prai, come era solita; e così chiacchierando tutto il giorno dalla matting alla
fera, buscava molti denari. Costci mori, e si trové nella sua sporta una borfet-
ta, nella quale era una ricevuta di cinquanta sCudi, dati a certe Monache con,
obbligo di far dire una meffa il mese all' alrare della Santis, Nunziata per l'ani-
ma sua 3 dil che fl-cava-argumcato.. che.clla non fufle pazza. |!) |

FAL. LA. Così e chiamato un contadino tristo,il quale non hayendo vo-
glia di lavorare, s'é dato a chiedere clemofina; e per far venire le donnicciuole
alle finestre, e cavar loro di mano robe, e danari, va per le strade cantando al-
cune sue ottave amorofe, e ad ogni due versi fa intercalare con la voce dicendo
Falarera tututrendo, con che si persuade d' imitar il suono del Chitarriao; ed all'
ultimo dell' Ottave,al medesimo suono della voce, si mette a ballare, e per que-
fio i) Poeta lo chiama il Ballerino; e poi va attorno chiedendo la limofina.

4420. Era uno scemo di ceruello pro vnSAe dal Palazzo; e perché egli
son si reggeva benc in piedi, pero andava sempre appoggiato a un ragazzo; e+
Patent Palos ct peers.

SRELEEER ELSSEREPES Shs

AL ASINO. Exa uno stroppiato nelle gambe, e nelle braccia; il quale era an-
ia' i q

ch' egli provvifionato dal Palazzo per quella sua figura cotanto contraffatta das
ale gli te og s
% MASSELLI, Era un mato 20 creduto tale, provvifionato pure dal Palazzo.

Coffui haveva in mente tutte le fefte del anno, e quali Ofizzj, e commemora-
zioni dovean farsi da i Preti giorno per giorno. Sapeya in oltre,quali erano quei
Rettori, e Curati di Chiefe, tanto in Firenze, che nel Contado, i quali nelleo
fefte trattavano bene, © male ai loro desinari; e da essi si Jalciaya in tali giorni
rivedere; e mangiava, e beveva tanto, che è imposiibile a crederlo anche da chi
I ha pil volte veduto. Era soprannaturale nel digerire, e s' è veduto fmaltires

SN

ae

we eerie vt: ¥.
gran quantità di roba, si puo dire impotlibile, come farcbbe un gran piatto di
= catia teacsiA bollita in brodo di bue, e condita a guisa di maccheroni; fee vol-
è te biffo, e tela d' olanda nella fiefla forma, e questo in breve tempo, e fenzas
i”  difficulta, o dolori, Li Poeta dice; Corre il eiaffelis, perché veramente costui,
“ benché decrepito, era di gamba velocissima. Haveva ui Serenifs. Gran Duca da-
i to per servitore al Maflelli un giovanotto gagliardo, perché lo seguitafle per tut-
i to dove egli andava, e osservatic tutte le suc azioni, senza mai contradirgli, 0
ee impedirlo, ed ogni fera riportafle quanto il Matielli haveva fatto in quel giorno,
4 Quando il Mafielli riceveva alcun diigulto da costui, non s' alterava seco, mas»
wy si metteva la via fra gambe, e senza mai fermarsi, o voltarsi ae meno a dietro »
i

non la guardava a camminare di buonuilimo palio 25., o trenta miglia con gran-
digimo

aS!






154 MALMANTILE | 4

dissimo travaglio, e rabbia del servitore, che non poteva, ne do
conveniva, che lo seguitaffe; onde andava molto cauto in strapazzarlo
sul principio del suo servire faceva fino a baflonarlo ) non tanto per
gaftigo da S, A, S. minacciatogli, quanto per il timore, che il Mafie
detta non viaggiaffe.
CAPIT AN Santella, Questo fu un soldato della Banda di Piftoia,il g
la volta al cervello ( o così finfe ) perché gli fu rubata la moglie da chi ne}
più di lui. Costui venne in Firenze, e vi dimord qualche tempo,
se pazzie; ma perché fu conosciuto, che sotto questa sua finta pazzia si
deva una gran tristizia, fu mandato forzatamente in Candia al servizio
Veneziani, donde non è pil tornato. s
© MERCANTI, e hanno havato il giro, Cioè gente impazzata. Si serve
parola Giro per intendere il girare del cervello, che vuol dire lmpazzare
per il Giro de' Mercanti, che si dice, quando un Banchiere tiene in mai
naro di tutta la Piazza; il che in Firenze tocca a fare una volta per uno
banchieri, o negozianti più grossi per tanti mefi; il che e fatto per co
mercanti; e dicefi: vere il Banco giro.
P/ACCIANT EO, Fu un Fiorentino di cos) vili natali, che non si fa
la cafata, ne il vero nome suo, essendo sempre stato inteso col folo fop
di Piaccianteo. Costui dalli parenti suoi fu lasciato assai comodo, ma
lo, che era dedito alla crapula, consumd in breve tempo tutto lo flato
a pena haveva dato principio Serer le milerie della poverta, e gli
la Fortuna di nuovo lo follevé facendoli redare da un suo congiunto
considerabile di doppie; e però il Poeta dice: Già mi/uro le doppie con Ia fh
queste ancora il buon Piaccianteo diede presto fine, pensando d' haver
rare il sentenziofo proverbio, che dice; e4 uno scialacquatore non ma
denari, Mas' ingannd, perché ridotto in eftrema poverta, e non
meftiero alcuno, si riduffe a portare quella barella, con la quale fip
ammorbati al Lazzeretto nel tempo, che fu la Pefte in Firenze, €
tal contagio campé di cotefta sua fatica; finita poi la pelle viveva
buscava con far servizj alle meretrici; e però il Poeta lo fa servitore
la, e suo Aio, e direttore. 'Piaccianteo voce che ha dell' antico Peacent
MANGTAR le cacchiatelle col cucchiaio. \perbole usatissima per i
gran mangiatore, Cacchiate/la, E' una specie di pane finissimo fatto
ed alla grandezga d' una pera bugiarda; onde con questa iperbole, int
che pigli in bocca in una volta cante di queste cacchiatelle, quante pigli
delle fragole, o pifelli, o altra cosa simile, e così viene a ¢flere ipafbae
perché il cucchiaio comune è capace a fatica d' una fola cacchiatella
ca dell' huomo difficilmente riceve una fola cacchiatella per volta:
di, che mangiava le cacchiatelle in grandissima quantità, e senza m
me non si numerano le fragole, ec, che si pigliano col cucchiaio.
ELA distrugione della Vernaccia, B' gran bevitore. Vernaccia & una
no bianco, mal' Autore per Vernaccia intende ogni sorta di vino.
AUSVRO' Ie doppie con lo spaio, Haveva gran denari. Ipetbole fata |
der un gran ricco; e ci viene dal Latina Agdio pecuniams metitur












e






























Bh

EEN

Tae

eae

as

TERZO CANTARE. 155
4 SON. V.ACCIA. Significa placidezza di mare; ma noi la pigliamo anche per
ogat sorta di bene (lare, e di buona fortuna, come e intesa a presente oe '
“ BARELLA, Specie di veicolo Gmile alla bara, o feretro, col quale si porta-



ert:

a fotterrare; ma questa che serviva per pertare gli ammorbati era

no
Coperta sopra con cerchiate, e tela incerata a foggia di calsa tonda di sopra, co-

me i tamburi da via,

Comanda la padrona ch' egli scenda,
E sia gik fuori com gli ovecchi attenti
Fra (nth fheesfe ch'ei non intends
| Ache fine son Id cutante genti;
Ma queglial qual no piace tal facceda,
» Sela trimpelia, af pillein complimenti,
E, perché at fichi il corpo ferbar vnoley
Ekin crf in queste, o simili parole,
STANZA XXKXV1,
Alea Regina, perché a Obbedire
Più d ogni altro a' tuoi cenni mi do vito,
Cold n' andro, ma (come si fuel dire )
Come ta ferpe, quando va all incanto;
Won cbt 10 fusga il pericol di morire,
 Perché so fo buon per una volta tanto;
Ma perché,s*io mi parto, non ti refea
Vet buom, che sappia,dov'eglibalarefta,
STANZA XXXXVIL
Non ti fdegnar, s'io dico ul mio pensiero,
at jibil non è ch' io tacciaofinga,
E,(e w andaffe it collo, sempr' il vere
Son per dirti,e chi Pha per mal,ficinga,
Ti fernird di cor vero, e fincero
Senz? interesse d' un puntal di stringa y
E non. come in tua Corte ono aleuni
Adulator, che fanno Adeo Raguni,



STANZA XXXXVIIL
lo dunque che non voglio esser de' loro,
ea tengo l'adular peffimo vizio,
Soggiungo,e dico, per ridurla a oro,
Che mal distribuita e questo usixioy
E che non pus palar con tuo decora;
Peicht mostranda non haver gindizio,
Vn tuo dio ne mandi a far la spia
Quali d' huomin tu havelfi careftia.
STANZA IL,
Manda manda a spiar qualche Arfafatto,
O un di quei, che piscian nel Curtile 5
uefio farò il meftier, come va fatto
Senza sospetto dar nel Campo oftile:
Oftile dico, mentre costa in fatto
Che cinto ha d'armi tutto Adaimantile,
Tal gente si puo dire a noi contraria,
Berche nevien quafsh per pigliar' aria,
TAN f

E perch' ¢i non vorrebbe uscir del cova
Soggiunge dopo queste altre ragioni;
Ma quella, che conosce sl pel nell'vove,
S' accorge ben, che son tutte invenzioni;
Pero [enza pin dirglielo di nuovo
Lo mands fers afuria di [pintoni,
Eynitr'ei pur volea vabrogliar laSpagna
Gli fal uscioferrar fu le calcagna.

Bertinella vuol mandar Piaccianteo nel Campo di Baldone a [piare; ma egli,
che non vorrebbe andare, adduce mille scufe; quali non gli sono ammelse, ed &
cacciato fuori di Malmantile a furia di spinte,

 TRIMPELLARE, Intendiamo quel suonare adagio, e tentoni la chitarra,
liuto, © altro strumento simile, che fanno coloro, che imparano a suonare:

da questo per tri

0 si fen.

er trimpellare, o BIA S
Za profitto,rempeliare che diciamo anche metteria fuk into, o metterla in musica,

€ suona quasi lo stesso che,

SE (a paffa in complimenti, Che significa Perder il tempo in yane cirimonie; ¢

senza toccare la fuftanza dei negozio.

VVOL [erbare il corpo ai fichi.. Vuol veder di viver, quanto ci può, e nony

mettersi a rischio d' efscre ammazzato,



Vi VR.



a
q
%

-
. = ee

—



















156 MALMANTILE

OBBEDIRE a tuoi cenni mi do vanto, Profelso defer' il più o
re che tu habbia, e di sapere intenderti anche aicenni. —
COME Ia ferpe quando va all' incanto, Cicé mal volentieri,
Folens nolenti animo, Omero, Ii Lalli En. Tr. C. 2. stan. 32. dice
Come la biscia all' odiofo incanto:
FO buon rr una volta tanto, Poslo morire una fol volta, Quando |
danaro, ches' ha in tavola, allora che uno ha perduta quella porai
veva, cava di tasca nuovo danaro, o vero dice: fo Laon, cioè pi
scudo, o per due, secondo che gli pare; e s' intende, che non vuol
la (omma, per la quale ha fatto buono, cioè promesso, Per esempio
no per uno scudo, ' avvertario inuita di due, io tengo la posta, ma
vincere, ne perder più che uno feudo, perché non fo Fiat di pil.
SE w' andaffe il colle, Se bene io fapeth, che ci fuffe pena la vita, WV
curim in manibus tenens aliquis ceruici esset incurfurus mes, conticerem. —
CHIP ha per mal, si cinga, Non m' importa, che altri l'habbia
si cinga pur la spada, ch' io son pronto a rispondergli. Nel primo
dell' Autore dice si cinga, e vuol dire si levi pur da lato la spada,
modo io non voglio far quiftion seco. L' Autore, che fapeva, che
i modi si dice, stimo fore meglio detto /i cinga, perché nel secondo
di sua mano, dice f cinga.
SENZ! interesse d' un puntal di springa, Non voglio da te cosa alt
che minima: Suona lo stesso che um puntal d' agherto, che vedemmo
stan, 10. e che il Lat. We ligulam quidem.
FANNO Meo Raguni, Cio' ragunano danari. La forza fa nella ¥
che se ben pare, che sia il cognome di Meo, è il verbo ragunare, ¢l
meitere insieme,¢ 44eo € prelo in vece di meus, mea, meum, © vuol '
raguni marfupio, cioè raguni alla mia tasca,
ETENGO I adular pefimo vizio, Non è dubbio, che l'adulazione'}
trando, e perciò Dante mette gli adulatori nell' Inferno gaftigati cont
vera pena, che si legge al C, 18, dell' Inf. Cicerone nel suo lib. de
de gli adulatori così: lis denique temporthus cavendum est, ne afs
ciamius anres, neve adulari mos finamus, in quo falli facile est; tales enim
wt inre laudemur, ex quo innumeratilia nascuntur peccata, cum homines
nibus turpiter irridentur, © i maximis versantur erroribus. Di:
mandato qual bestia mordeffe più ferocemente rispose; Nelle faluatiche
tore, nelle domeftiche l' aduiatore, perché con le sue falfe jodi ti
rovine; Bd aggiungeva; che'le parole composte non per aprire il ¥
compiacere 5 sono un capresto melato. Si potrcbbono addurre
gravitfimi Autori, ma si lascia di farlo, perché non torna affatto al pi
si rimette il lettore a Plutarco nel suo libro de digno/cendo amico ab
PER ridurla a oro, Per ridurla alla perfezione del difeorfo; Per
conchinfione. Vedi sotto C. 8. stan, ys
COME se tu havefi carefia a huomini. Come se ti mantaflero hu
'to. Ancora appresso di noi quando si dice: W/ tale oa
buono a quaicofa, seguitando il detto di Diogene Alomintem quare
x2, Corsorramini » & virr ofoe, Omero, Viri gfe.






















TERZO CANTARE: 157

@RPASATTO. Huomo vile, mal fatto, scimunito, e da poco; che i Lati-
ni dicono Vappa, Cerdo, e simili, come si vede in Plauto da noi in questo pro-
> citato C. 6. stan. 98. E questo nome d' Arfafatto viene da Arfaxad
della scrittura fagra, che nel barbaro secolo non essendo dal volgo inicio, fu
reso per und Habbaleo,o Habbano. —-

Dl quei che pisciano nel Cortile. Pisciar nel Cortile vuol dire Far la spia, e que-
flo, perch coloro, che fanno la spia, cfiendo veduti entrare, e uscire del Pa-
Tazzo della Giuflizia, hawno qualche roflore, e però essendo veduti da alcuno
lor conoscente, si fermano nel cortile di detto palazzo a pisciare per scufa. Si
può anche dire, che il verbo pi/ciare sia preso in significato di buttar fuori, ed in-
tendere che pifeino, cioè buttino fuora quello che sanno nel Cortile della Giutti-
zia, ove è la Cancelleria del Bargello, nella quale le spie portano le denunzie.
Si può anche far refleffione, che detto Cortile fla sempre pieno di Sbirri, i qua-
Ji fon' anche per lo più spic, e vi sono due pisciatoi spefissimo adoprati da loro,
ed intendere, che venga da questo il detto Pisciar nel Cortile. Ma fiacome esser
Gi voglia, l'effetto ¢, che pisciar nel Corriles' intende comunemente, Far la spia.

CAMPO oftile. Campo nimico, Dice che e campo oftile, perché ofta; e fa.
na(cere il bifticcio dalla parola offile, e dalla parola cosa, la quale nel parlare>
pare che dica che v/a, che vuol dire s' oppone, e fa oftacolo, facendola di duc
dizioni, cioè che, ed offa, quando è Puna sola, cioè coffe dal verbo cofare 5 cho
vuol dire Esser manisesto. Modo usato da Franc, Harbarino ac' Mottetti,

NON vengon quafsh per pigliar' aria. Vengon per altro fine, che per andare a
spatio, o pigliare aria. Detto usatissimo per intendere uno, che vada sotto al-
ti pera in qualche luogo, e sia poi per negozio importante, e per cavar uti-
le da quella gita; che i latini dissero: Won fine ratione lupus ad urbem. E noi pu-
Fediciamo: OQusfta cosa non e fatta fine quare. Vedi sorto C. g. stan. 11,

CONOSCE ii pel nell vove. E' fagace, e aftuto, e fa considerare ogni minuzia:
forse è quello, che i Latinidissero: Ventura per dioptram pro/picit,

A furia di spintoni, Con quantità grande, e spelsa di spiate, che tale è la for-
za della parola faria in questi termini forse dal Greco Phura, che vuol dir' abbon-
danza, © moltitudine, Vedi orto C. 9, stan. 49.

LMBROGLIAR la Spagna, Quand'uno s' aftatica con chiacchiere fuor di pro-
posico per divertire uno dal priacipiato discorso, per non gli dire quel che egli
vorr sapere, o non fare quel che cgli e imposto diciamo; Egdé tmbreglia la,
Spagna,

PERRAR 2 uscio in Ju te calcagna, Vuol dir Scrrar'uno fuori della porta. E? il
contrario di dare dell impo/ta sul moftaccia, che vedremo sotto C. 10, stan. 27., che
vuol dir proibire l'ingretio a uno che venga per entrare; € quello vuol dire Ob-

biigar uno a uscire.
STANZA LL
Sperance resta alla Regina intorne La pala nella defira tien del forne
Spianator di pan tondo riformato; Wella finiftra un bel teglion marmato
Gridan elle remo,e Livorno, dn cambio di rotella, chagli guarda
Ed ha un Co. che pare up vicinate 5 Da j colpe il magaxrin della mofrarda,

STAN-





























158 MALMANTILE ~
STANZA LIL '4 STAN
De i Rovinati anch' ei pafso la barca,;
Perché la gola,il giuoco,e il ben vestire
Gii baveano il pane, la farina,e Parca
Jn fumo fatto andar come elifire 5
Tal che,cantando poi,come il Petrarcay
a2 CAmore io fallo, € vecgio il mio fallire,
Ail giuoco del barone, e alla baffetta
Giocava,apparecchiando alla Crocetta,
STANZA LIII,
Fu dalle dame amato in generale,
(4 dico dalle prime della perza)

Poi Bertinella ffavane si male, Gi dal usizio,e
Ch' ella fece per ini del ben bellezza, Crn la solita faa p
Perché [pefa la rola, e concra male, Perch sin questo caso\aleun
Fatta più bolfa a! una pera TUERZA y Siscnopre, facil sia, farie pri

Potea dt notte, quanto a mezzo giorno, dccid sul letto poi di B. «ch
eAndar sicura per la fava al forno. Se gli facia ferrare il m
Partito Piacciantco resta appresso Bertinelia Sperante; questo era B
fai comodo; ma tra il suo mandar male >» ctra l'effergli stata fara
tega, si ridufle anch' egli malissimo, e nondimeno non usciva tai di
retrici, dalle quali veramente cavaya il yitto, perché essendo bell” hi
esse amato, e se ne servivano per bravo, e per ogni occorrenza loro:
sto il Poeta lo fa consiglicro, e Bargello di Bertinella,

SPERANTE. Così veramente haveva nome costui, e faceya il
Fornaio, e però dice Spianator di pan tondo: E lo dice riformato, p
bito a quei rempi il fare i tondo (che così si chiama ij pil n
faccia in Firenze per il pubblico ) in rgnardo dell' appalto, che fu |
sta sorta pane; e però gli conuenhe ferrare la bottega, Ci è però.anel
zo dell' equivoco, perch /pianarore di pane vuol dire Colui che fa il
significa ancora uno, che mangi molto pane. Vedi sotto C. 6, sta
si può intendere gran mangiatore di pan tondo, ma riformato;
può pil mangiar tanto, per non havere il modo da comprarlo
mine militare, e s' intende quel soldato, che è privato della
vea; che si chiama poi Vfziale riformato, ' 2 14g

GRID AN le spalle sue remo, e Livorno, Ha spalle così grandi,
'rate a Livorno per mettere a un remo di galera. Questogridare, ec, &
dire, che ha lo stesso significato, che Chiamar di id da'monti, Visto sopra (

Van C..,. che pare un vicinats. Haun C..:. grande quanto uaa
Tperbole usatissima per denotare un sedere eflremamente grande j-¢
intendiamo una contrada, - AS: usa

TEGLIA marmata, Coperchio fatto di marmo minutamente pefto,, e ter
col quale, sendo infuocato, si cuoprono le teglie, © tey er rololare le
de: edé forse il Latino clibanus; che per altro yuo) d a
cotto, se crediamo a Pietro Viloa Vita di Carlo V.












































TERZO CANTARE: 159
IL midgaezino della moffarda. Cioè il ventre. AsoParda & uno intingolo fatto
y ofto cotto, e fenapa 2 ¢¢. ma qui e presa ( come da molti ) per quella roba.,
a che fla nel ventre per qualche similitudine che ha quell' escremento col colores
della e magazzimo diciamo una stanza destinata a riporvi,¢ conseruar.
vi Spagna. almazen.

— PASSO' ta barca de' rovinati. EF nel numero de' poveri.
is ARCA, Voce latina, che vuol dir Caffa in generale, ma noi intendiamo spe-
at





cialmente quella gran madia, entro alla quale  Fornai tengono il pane cotto, 0

FATTO andar' in fumo d' elifire. Fatto andar male senz' alcun frutto appun-
to come fa l'elixire, che lasciato in un vaso aperto fuapora, e si disperde.
nem AL Barone, e alla Bafferra, Sono due giuochi noti, i primo di dadi, e l'altro
16% — di carte; ma qui scherzando vuol dire, che era divenuto Barone, cio' mal velti-
i to, guidone, € ridotto al baffo, che vuol dire Impoverito; traslato dalla botte,
van che si dice efer' al bao quando il vino che v' e dentro è alla fine, e che la botte ¢

| quasi vota.
. APP ARECCHIA alla crocetta, Vuol dir non haver da mangiare. Far degli
(ee — sbavigli significa non haver da mangiare. Vedi sotto C. 4. stan. ultima. Ed ef-



wil' sendo i¢ di molti nello sbavigliare farsi la croce col dito pollice incontro }
uk! alle fauci, pero far le crocette intendiamo stare a bocca aperta, e vota, che in fu-
ii@ — stanza vuol dire non haver da mangiare, Qui il Poeta rende il detto più oscuro,

1? = € pill coperto dicendo apparecchia alla crocetta, che & un Conuento di Monache,
iB nel qual uogo par che voglia dire, che costui defini, e ceni: che questo significa
il verbo apparecchiare, quando e meffo assolutamente, e senza aggiunta.
ie PRIME dela pexxa. E' lo feifo che di prima Claffe, o paffar per la maggiore
detto sopra C, 1. stan. 6. p
ST AVANE male, Tribolava per |! amore, che gli portava, Era grandemen-
4 te innamorata di lui, Latino deperibar.
a FECE del ben bellezza. Cioè spefe, e consumd, quanto ella havea, Havendo
4 consumato tutto il suo bene, le rimate folo la bellezza, o vero fece bellezza, ed
nf? ene ogni suo havere. E' quel Procerusam facere, che vedemmo sopra C, 1,
iat D,
2 BOLS A. Mal fana per troppa umidita, e ripienezza. E perché questi tali
1i@ elf soglion esser per lo più ripieni di carne liquida, e di colore fra il verde, e il
giallo, gli paragoniamo a una pera troppo matura, o fracida, che questo vuol
ol dire pera mezza. Virg. mitia poma; cioè maturi,

we POT EVA andar sicura, ec. Questo si dice d' una donna vecchia, € brutta, in-
jc tendendo, che ella e sicura di non esser rapita.
a LEZZO, Puzzo., Fetore, Propriamente /ezzo e un' odore che dispiace, il

pi quale non nasce da corpo corrotto, come e quel puzzo,, che nasce da una carne
troppo frolla, o altra cosa marcia, o fracida, che si dice stantia; ma e odores
yt Raturale, © procede da fudore  o da altra evaporazione, che getra un corpo,
nt beaché non sia corrotio, onde quello che si sente dal becco, e dalla capra vivi, si
ya!  dice lezzo, e quella che si sente ca i medesimi quando son morti, € corrotil si di
| ce puzzo o fetore, o fito di stantio. Vedi sopra in gucfto C, fan. 24. Queite
a.


——

|
4






16° MALMANTILE,,

Jezzo,così dda olezxo,¢proprio quello, chei L.dicono Virus,Not
veleno, morbo, ferore,, he 1 e simili pigliando l'uno per l'al

che l'altro e vocabolo di mezzo, perché tutti si poslono
re, come si cava da Caio Lurisconfiulto: Qui igitur ( dice egli )
ber adijcere utrum bonum, an malum. E Statio lib, 2, Syluarum:Atque
Virus, odoriferis eArabum; quod crescit in aruis, Noi ancora diciamo
purzo di muschio; [a dé mnschia ch' egli avvelena. Gls ammorba d'ambra
to ch' egli attoffica, ec. sue

PASCIONA. Intende Comoditi,c abbondanza d' ogni cosa neceffari
to, se ben pa/ciona vuol propriamente dire Il pascolo delle bestie.

N'IMP.AZZA affatto. E! di tal maniera innamorata di lui, che ha
ilcervello. L, efflittim, perdite amat,

NON (a vede a mezzo. Non gode la vista di lui alla meta di quello,
rebbe; termine, col quale s' clprime l affetto grandissimo, che uno
un' altro, Won veder più avanti; ne più qua, ne pin la; usd il Boce,

SALAMISTRA, Maeltra di fala. Ma iol intendiamo una donna
dottorefla, affannona, e simili, ma per derisione, diciamo ALadonna Sa
Qui intende direttrice del governo; e la chiama Sa/amiffra pur per di

V-A in capo as liftra, Cioè toltone Bertinella, e Martinazza eglié il
il primo huomo che sia in Malmantile,

£ DI nidio, E' tristo, E' aftuto fino dalla culla., e4 incunabulis

Noi pigliamo questo detto da gli uccelli cavati dal nidio, ed allevati
V uccellatura son sempre migliori, che i preficel,

NAVICELLO, Vuol dir huomo lefto, e che fa tutte le furberie', che
sa navigare a tuttii venti. Ha lo stesso significato che esser di nidio.

JL letto di balocchino, 5' intende le forche. Da un tale detto Balo
fu impiccato in Firenze al Canto alle rondini per ladro di bestie, delle
Senfale, e si chiamd anche il Parola. Vedi sotto C, 6, stan. 67.

" SERRARE il nottolino. Vuol dire strozzare: intendendosi per Nottolit
parte della canna della gola, che vulgarmente chiamiamo gorgorzule, €Qe
la similirudine, che ha nell' andare in gii,e in fu,quando ' ing hiotti ce


























in git, e in fn delle nowole da ferrar porte, ec. 2
STANZA LVL ee
Fa in tanto nel Castel toccar la cafsa, Ch' in fretta alla rassegna se
” Einalberar U insegna del Carreccio, Con le schiere pero fatte a bal
E comandante elegge della malfa Che ad una ad una
U nobil Cavalier Mafo di Caccio, Sotto /un guida,e [orto sua t

Bertinella fa toccar tamburo, e inalberar l'iniegna generale, e d
nerale della sua gente Malo di Coccio, il quale subico si metee a far la
ed accomoda tutti i soldati sotto i suoi Capitani, e Comandanti.

CARROCCIO. Questo era anticamente un gran Carro di figura qu:

ra il quale s' inalberava appiccata a una grande antenna ! insegna
della Signoria di Firenze, e si metteva fuori in occasione di trionfi, o |
Fiorenuini uscivano in campagna alla guerra con cfercito formato, ed è
flefio Carro, e della stetla Sgura,e grandezza quello, sopra il quale si,
ii Palio di S, Gio; Bauita, MA







OS eee a a a

Ss peer
ty

E PELLESE

WREARE BEALEE

TERZOQCANTARE, 161

. &€ASO di Coccio, Tommalo di Coccio fu un Pescivendolo huomo ficro, e di
gran seguito di snoi uguali, a i quali egli in tutte I occasioni di fefte, cacce, ed
altre cose simili comandaya come a' suoi servitori, ed era benissimo ubbidito da
chi.per genio, ed affetto, e da chi per timore, e però il Poeta lo fa Generale de'
soldati di Bertinella, che son tutti di condizione simile a !ui, come vedremo.
Lo dice mubil Cavaliere, perché in Firenze egli era conolciuto,e nominato più che
qualfivoglia gran Cavaliero.

4A HABBOCCIO. In confulo,.a caso, e senza considerazione.
STANZA LVI

Si primoé il Purba nobile Bradiere, 1 \fateude il Vecchina il gran Barbiere,
Che non giuoca alla buona,e meno a gofi, Che vnol chrogni hor fitrinchi,e si sbafof,
A noccioli bensì si fa valere E dove 4 mensa metter puo la mano,

,

Perch ci da benet buff,¢ meglio i Sofi. Si fa la fefta di San Gimignano,

Al Poeta mette in questa raficgna una mano di piebei noti per qualche loro
azione © buona, © cattiva, e gli nomina con i loro soprannomi. Ii primo è il
Furba firadiere, cioè uno di coloro, che alle porte della Città cercano i patseg-
gicri se hanno reba da gabella, i quali pizzicano di spia; ma questo Furbo era
anche in effetto spia. li secondo e il Vecchina Barbiere.

ALLA buona, ed a gofi. Sono due giuochi di carte afsai noti: ma con dir così
intende, che costui non era ne buono, cioè semplice, ne goffo, cioè corrivo.

A NOCCIOLL hen sit, Già che il Poeta porge la congiuntura di narrare, qual
sia appeelso a inoftri Ragazzi il giuoco de' noccioli, ed in quante maniere si
faccia,, il Leteore si contentera, che io spieghi con un poco di digreffione i mo-

i 5.€¢ i si traftullano i nostri Ragazzi a questo giuoco de' noccioli, e non
si idegnera di volgere gli occhi a leggere il discorso di quei trattenimenti, a'qua-
Ji,non idegnd.di volger l'animo, ed impiegar l'opera un Cefare Augufto, fecon-
do che riferisce Suctonio Trang. riportato, e considerato da Alex. ab Alex, dicr.
Gen. lib. 3, cap. 24. e ricordandosi che tutta quest Opera è fatta per i Fanciul-
Hale » che.per quelle persone, che già reliquerunt nuces, haura la bontà di con-

»fenon per nece(saria, almeno per non affatto fuori di proposito tal digressione
« Dicodunque-che il giuoco, che fanno i nostri Ragazzi co' noccioli
di ( Cofiumato anche da 1 ragazzi Greci, e¢ Latini, che lo dicevano ladus
acellatarum, secondo i| Buleng, de Lud. vererum, & Alex. ab Alex. dier. gen, lib. 3.
cap. 21, ade di cui parole poco apprefso riporteremo ) e usato in molte maniere;
ma specialmente giuocano, 4 Cavalea, alle Cafelle, alla Serpe, a Ripiglino, a Shree
Seid, 4 Cavare,.aShricchi quanti, aTruccino, ed alle Buche. Di tali giuochi,e
ai ciascuno di edi narreremo ii modo, che tengono a esercitargli, e diremo qua-
li Geno simili, 0,gli Mei, che erano usati da gli antichi, |

A cavaica. S' accordano due o più, e tirano sopra un piano i noccioli a un,
per uno, e tanti ne seguitano a tirare, ane stieno a far falire sopr' agli altri
trati un nocciolo che sopra vi refti, e si regga senza toccare altro che noccioii;
€ cojui che ha tirato il nocciolo rimafto sopra, vince, e leva via tutti i noccioli
tirati. Lo dicono a Cavalea da quel cavalcare, che fa il nocciolo sopr' a gli altri,

ALLE Caselle, o Capannelle.. Mettono sopra ad un piano tre noccioli in trian-
golo, e sopra dieii.un' alteo nocciolo, e rs maiia dicono Ca/ells., o Capan.

nella;






| 2 9

* Giulio Polluce lib. 9.c. 7. mostra che faceflero questo giuoco ancora

 al quale & toccato in forte,deve,girando in rnota con quello'





















162 MALMANTILE™

nella;¢ fatto di éffe il numero tra loro conuenuto, ed
concordata, tirano in dette Caselle un' altro nocciolo 5 e colui cl
vince cutte quelle caselle, che fa cascare col colpo. Questo fu wi
gli antichi, e dicevano Ludere Caffello nucum secondo il Buleng. C.
selle vengono descritte da Ovidio in Nuce in quei veri M
amplins, alea tora off 5 Cum fibi fuppositis additur una tribus,
ALLA ferpe. Fanno una di dette caselle, la quale figura po d
da quella fanno partire un filare di noccioli, che figura il refto del corp
ferpe, e poi vi trrano dentro con un' altro nocciolo, € chi fa col tiro
uno, © più noccioli del tutto fuori del detto filare, vince tutti lino
sono dalla rotwura in git verso la coda di decta ferpe, e durano così, fino ache
sia rovinata da un di loro queila casella, che figura il capo della ferpe.
pure era usato da i Greci, e Latini, e forse facevano co' noccioli altre fig
come si cava dal Buleng. Cap. 8,, dove si vede, che in vece della fei
co i noccidli un triangolo equilatere, o [ come dice egli } il delta &
ARIPIGLINO, Pigliano quella quantità di noccioli, che conuel
randogli all' aria gli ripigliano con la parte della mano opposta alla
in tal' atto sopr' alla mano non resta alcun nocciolo,colui perde la gita, '
colut, che segue; e così si va seguitando fino che refti sopra detto luo
mano qualche nocciolo, e questo al quale e rimafto il nocciolo,dee di qui
lo all' aria, e ripigliarlo con la palma, e¢ non lo ripigliando perde la git
reflafle pi d' uno sopra alla mano, può colui farne scalare quanti:
che ne refti uno; che se non restafle, perde la gita. Ripigliato il
conda volta, deve costui tirarlo all' aria, ed in quel mentre pigliare
de i noccioli cascati, e con essi in mano ripigliar per aria quello che
seguendo, posa i noccioli prefi, e perde la gita; e se ne ha pigliati
senza fare errori, restano suoi, e si (eguita il giuoco fino a che fiend

dissero Pentalitha, perché ulafiero di farlo con un numero det
faflolini, o aliofi.
SBRESCIA, E lo stesso, che ripiglino, se non che nella

vonfi ripigliare quei noccioli, che cascarono in terra la seconda volta

uno, o due per volta, ma tutti a wn tratto s il che si dice fare sb

dovene pur' uno, o cascandogliene, perde la gita, e così fiva seguitando,sia
uno pulitamente gli raccolga tutti. Sd

CAVARE, Infilano un nocciolo con una fetola di crine di 'eavall
ual fetola ridotta in forma di campanella', o anelletto legano uno'!
segnato un circolo in terra, vi mettono i noccioli, che son d' accord





filato,a tal girare,buttar con esso nocciolo fuori del circolo uno y © pil
di quelli y che son dentro al circoloy e vince quelli, che cava je
the gira, tocca terra, perde la gita; ma guadagna i noccioli eavati, eda
ciolo da girare a un' altro. E così si va seguitando fino a ¢he fien
noccioli, Similmente nel giuoco detto da' Greci Eis amillan delerit
¢ehio, dentro 'l quale però si doveva buttarel' aliosso.in mani







ian ite deen cna:

2 ante

ASL Be 2 62g & 2 eo pop
i
i
a
y
it
=
:
i'
re
:
'
of
i

a

as

nll

ad

vi

&

“4h!

va
a
yi

v
è
a

%
9

wiceeimidi —

TERZO CANTARE. ¥63
se,¢ non ulcisse di detto cerchio. Appresso di noi anche negli Alioffi si fa aca.
vare, Canti alcialeschi; Perch' al cavare un' elioffe bruito, ec,

 SBRICCHI quanti, Occultano dentro al pugno, o dentro ad-ambe le mani
ita ioli, che vogliono, poi domandando ad altri, che indo-
vinino ni e'noccicli occultati, ed indovinandolo vince tutto, se no; de-
ve dare quel numero di moccioli, che ha detto di pil, o di meno; E questo si fa
una uno, dovendo il primo, che domando.far' anch' egli domandare,
¢ cosifi va continuando i giuoco.. Questo sbricchi quanti & lo stesso, che pari, o
casto, nel si domanda, se il numero è pari, o caffo, e chis' appone vince
tutti li noceioli occultati; se no, perde altretcanta somma. I Latini dissero: /u-
dere par impar. LGreci artiazcin, Di questo giuoco parla Giulio Polluce sopra
citato, ed il Meurfio de /adts vererum, i quali mostrano, che si faceva, comes
pure oggi si facon i danari, econ altra materia, come mandorie, e simili, at-
ta a:potersi accomodare dentro alle mani, Ovidio in Nuce. Ef etiam par fit nu-
merus qui dicat.,.animpar Vt divinatas axferat augur opes.

A TKYCCINO. Vno tira un nocciolo in terra, ¢l' altro tira un nocciolo a,
quello, che @ in terra, e cogliendolo vince, se no; quello, che tird in terra il
primo, raccoglie il suo nocciolo, e lo tira a quello, che tird ! avversario, e così
continovano., e chi coglie vince il nocciolo che coglie, o quello che sieno conue-
nuti. Ex simile al giuoco detto da'Greci Streprinda.

ALLE buche. Fanno diverse buche in terra in giro, formandone come unas
rofa, nelle quali tirano i noccioli, e colui vince, che entra in una di dette buche,
quella somma, che e prezzata quella buca,nella quale entrd il suo nocciolo: per
esempio le buche sono fette, la prima che e volta verso donde si tira, che è la pil
facile a entrarvi non fa vincere,non efiendo ta(sata in cosa alcuna, e da i aoftri
fagazzié detta la buca del Niffo ( forse da wibil ) E dell' altre una vince tre, una
quattroyec. EB perciò ho detto, che vince chi v' entra quanto è preazata la buca,
€ poi va.con gli altri ad aiutar condurre il nocciolo nella buca a colui, che al pri-
mo tiro non v' entro, € spingendolo di dove e alla volta delle buche col dito in-
dice ( che dicono limare ). Ovidio ut pronas di

igizo bifue femelue ee > o col buf-
fare, o col soffiare nel nocciolo,¢ e la differenza da buffare a fo

fiare vedremo
poranepeeliy ).nel che adoprano ogni arte per difficultare all' avversario il con-
jurre il nogciolo dentro alle dette buche; E così facendo a una volta per uno a
limare, buffare, o soffiare, colui vince, che ha fortuna di condurre il nocciolo
dentro a una di dette buche, ancor che il nacciolo sia degli avver(arj. Similes
al fare alle buche & quel d' Qvidiq. Vas quoque fape canum spatio diffance locatur, In
quod. milla levinnx cadat una manx, Banno questo giuoco ancora con una palla, ¢
giuocano danari, come vedremo sotto C, 8. itan, 69. alia voce Alinfo. Edé fimi-
le quello che i Greci, secondo Giulio Poll, lib. 9. c. 7. chiamana pherinda: © se~
condo il Meurfio de Lud, Grae, alla voce Apherinda, & alla voce milla, ed il
Buleng. cap. 14. € go, Se bene tanto nell' dpberinda quanto in quello, che si chia-
mava Eis amillan; tiravano ia un circolo, e non nelle buche. Alla buca bens}
tiravano in quelltaltro detto Tropa, che corrispondeva a questo nottro. Conchiu-
do we » che la maggior parte di detti giuochi erano usati anche da gli an-
tichi; & se ben pare, che si servitiero delic ae > 10 non soa lontano dal crede-

2 te,




we






ey MALMANTILE ©

re, che la parola Nwces voglia dire ogni sorta di nocciolo,
lib. 15. cap. 21., dove mette in dubbio, se ie noci in:
ancora arrivate in Italia; ed oltre a questo trovone i gla
ed ardirei pero affermare, che ancor' essi adoperafsero noccioli di p
(come fanno anche i ragazzi de' nostri teaspi ) alle volte noci, ¢
cioli di pesca, seguitando Alex. ab Alex. lib, 3. c. 21.5 che dice
Gos viros super nucibus ocellatis einfmodi, qus essent, ancipitem ditt cogicationt
Se, variaque in opinione versari y © alias nuces avellanas, alios amygdalas pa
neque fatis ratam sententiam ferre super Tranquilli verbis, quibus Ang ds
animi canfa cum pueris facie liberali ocellatis nucibus lnfiffe dict.
mus, & probabilins putamns id ef: Einfmodi nuces ocellatas nucleos 5
pomis fitos inspicimus dicamus esse, quibus perfape Iudere nostrares pueros d
dittafque ocellatas propter ocellos, & foramina, quibus muniuntur undiqne
ansyedalas, aut avellana, ficut error haber 5 fed de persicorum offibus, quibus |
debatur 4°& nunc frequens puerorum Indus est, intelligi conuenire credumus
© non umbigua fenrentie fore. Dalle quali parole s' intende, che
cora si giuocava a questo giuoco de' Noccioli, Ovidio de Nuce,ct
verita, € mostra che havefsero molti de' suddetti giuochi, o poco d
Marziale attefta, che erano gli stess genj ne i fanciulli de' suoi tempi, ct
d' oggidi, e che il portare in tasca noccioli causava a quelli delle maz;
segue ne i noltri, dicendo; y 4
Alea parna nuces,& non damnofa videtur;
Sape ramen pueris abjpulit illa nares
Ec altrove. Zam triffis nuctbus puer relictis
Ed Horatio, ?offquam te talos, Aule, nucefque
Ferre finu laxo vidi y ec,
Sono dunque, e furono sempre puerili tutti li suddetti giuochi; e
biamo un detto di disprezzo; Va 4 giuoca a'noccioli, che significa Tuo
gior giudizio di quel che habbia un fanciullo: Qual detto era usato
pure, come si cava da Persio Sat. s.
Et nucibus facimus quscumque relittis i
E dicevano reliquit nuces d'uno, che dalla puerizia paflava a mani
rie; Dal che potrebbe argumentarsi, che 11 Poeta dicendo, che
ca bene a i noccioli, intendefle, che egli fufle huomo di poco giudizio, e cher q
nucibus imcumbat; Ma si conosce, che non intende.questo, perché prima
Non ginsca alla buona ne 4 goff, significando che non era ne buono ne
ora col dire, che egli giuoca bene a' noccioli, percheda bene i buffi, ¢
vuol — ben la spia, che baffare, e fofiare vuol dir Bar la spia
C. 1, stan. 37. 5 «
BVEPI fh. Buffo e un fofiare non continuato, ma fatto:a un tratto - |
si farebbe a sputare, o a profferire la parola buff, donde buferd., o bufea un grat
nodo dj vento, che paffa presto, Sofiodun soffiare con la bocca:tanto quanto
ud durare senza ripigliare il fiato, e ciò dico per mostrare la differenza'
Ee buffo, e foffa; che per altro sd che fof & generico, e comprende og
sompimento d' aria fatto col fiato di che che sia, dicendosi /ofiare y
































a

Sreiteetes

Sat

3 3%

Se

cARE

aee

- TERZO CANTARE;: 165

vento, Che manda fuori il mantice, /offare si dicono i Venti, ec. Vedi sopras
C. 1. stan. 39, la voce rabbuffo,

1L Vecchina, Era tin barbiere così chiamato, il quale ogni fera andava ricer.
candoiper Posterie le conversazioni, che erano a cena, e trovandone di suoi ami-
ci, con varie'chiacchiere poco a poco senz' essere inuitato si metteva a sedere,
© mangiava', € beveva quanto più poteva, ed al far de' conti fen' andava senza
feeds era comportato, è faceva il buffone; Procurava, che

conversazioni di cene si faceffero in a sia, dove apparecchiava, e prov-
vedeva assai pulicamente, e bene, e con spela aggiuftata faceva star bene,e avan-
zava tanca roba per' se da viver più giorni, e però dice Vuol che ogn' hor si trinchi

che dal Tedesco rinchen vuol dir bere ) e /7 sbafoff, cioè si mangt assai, donde:

ve un che mangia assai: Queste voci ha/ofia, e ha/ofione sono in ulb appref-

fo alla plebe più bata, edi più civili ! adoprano per (cherzo, per intendere uno

foverchiamente graflo, e che mangi molre mineftre, le quali si dicono ha/offe dal
Latino vas oft, cioè Valo pieno di mineftra.

St fala fefta di San Gimignano. San Gimignano è una grofsa Terra del Domi-
nio Fiorentino nel Vescovado Volterrano; e la principale, e più folenne fella,
che si faccia in questa Terca & di Santa Fine, la qual Santa fu di quel luogo: E
dicendosi far la feftn-di S, Gimignano' intende si fa fine; € qui ale esprimeres,
che questo Barbitre dava fine a ogni cosa, che veniva in fu la mensa.

' 3 TA

NZA LVIIL
Dalle freddeacqie il Mulaifanti approda Co i pescatoré al Mula hora # accoda
A. id snilitar fra fronde,e frasche, DimeoT receon de ghiozzi,e delle lasche;
Ata nobil bardarura tina in broda Pericol pallerino ancl? ei ne mette
Divcedri edi ciriege d' amarasche, Dugento suoi armati di raccherte

4L mula dalle fredde acque, Pu uno che nel tempo di state vendeva l'acque diac-
kanes a Pare che questo Mula sia un gran sig.\ di lontani paeGi
evicino al Mar gelato, di dove approdi alla spiaggia del mare; ma approda,cioè
s' accoffa alvreftante dell' armata di Bertinella. Dice fra frondi, e fra/che,perché
questitali veaditori d' acque diacciate sogliono per all ornare le loro

di verzure, fiori, e frasche.

8' ACCOD A. Seguita, o vien dietro immediatamente. Quasi ad caudam ire,
Noi wliamo questo verbo per 1e bestic da foma, che seguitando in viaggio Yuna.
I altra viene alla prima legata la seconda, alla seconda !a terza, ¢c, cons
la cavezza alla groppa dell' antecedente, e così chi seguita va con la testa vici-
na alla coda di efla, e questo si dice accodare, benitiimo usato qui dal Poeta,
per il Mula, fendo che a i muli pil, che ad ogni altra bestia segue questo acco-
dare.

DOMMEO. EF' una parola fola, e dovrcbbe dire Dommeone, che così cras
chiamato un venditore di pesce, e falumi, il quale era amato da rutti i ghiotti
di Firenze, perché vendeva sempre il miglior pesce,, che veaiffe in mercato, ed i
giorni di geaflo haveva sempre qualche ee » © ghiortornia singolare. £
pero lo chiama treccone, che vuol dire Rivendugliolo, cioè rivendicore di cose»
commettibili di poco prezzo [ che si dice anche barnllo } forse dal Latino ¢rice,
bagattelle, cose di poca stima, e di vil pregio; Marziale, Sunt aping, triceque y

of









166 MALMANTIEB ¢

& si quid vilius iis, Dice di ghiozzi, e di lafebe ( duc specie di '
per intendere, che yendefie sokamente questi, ma per mostrare
pesce in generale. phi thad t
PERICOLO. Quefio fu un tale Alcflandso Violani detto.
nato per il suo gran valore nell' abbaco, come diremo foro.C, 4
perché egli era anche bravissimo giuocatore di Palla a corda,e 2
po a fitto una di quelle Rtanze dove si giuoca a tal giuoco, if 1
armate di racchette, o daccheste, che sono meftolescon le quali si giuo
a corda, e sono composte d! un cerchio di legno col manico, ¢d il ya
no d' una rete fatta di grofia minugia: per /accherea intendiamo anche.
di dietro del porco, e del castrato; Non (0 già se la /acchetea da giuocare,
nome da guefta, o questa da quella, so ben che si chiamano cogil' une 5 ¢
er la similitudine, che è fra di loro della figura. Questa da gi r



tini detta reticu/um da quella rete, della quale € composta, come si ¢a.
Ovidio: Reticwlogue pile leves fundantur aperto. Vedi sotto C. et
at





viamo per mapdare a casa le robe commeftibili, che si comprano in
vecchio, e ci servono ancera per Quochi. Cofloro son per lp più
e Cantoni Suizzeri, e dimorando in Firenze soglion far camerata co i
che vendono i tartufi, e per questo dice che egli conduce Norcia, e la Vallatas ®
perché egli era hvomo pulitissimo,gli fa per soprayyefta un grembiule candido »
come veramente cgil sempre portava. Z SIWOET
GIANNETT A, onde Giannettina; specie d' arme in asta, nella guerra wat
da gii alfieri, Ginera in Spago. e una piccola lancia; corsesca. jo amet
PENNACCHIO, S' intende una quantità di penne di Struzzolo; ma costul
I havea di Cappone come trofeo di Googe. 5 BOE
Z#N4. Specie di panicre senza manico composto di strisce di Jegno gentile
eda tale Zana costoro son detti Zanaro/s. Di questi tali il Poeta fa Capitano!
licche, perché in vero egli era riverito da efi, coe. quelli che nel loro. ?
} havevano veduto esercitare Cariche riguardevoli, e fapevano,, che era d
reputati delja sua patria, dalla quale era in quei see «dhe
SGARVGLIA. Fu un Battilano assai celebre, e fra i (uoi pari Capopolo, ©
da coflui quando in commedia e stato introdetro il Battilano }' hanno:

rola Pijlotta. ol
; STANZA LIX, STANZA L

Melicche quoco all' ordine s° appresta, L' unto Sgaruglia con frittelle a tof
Per giannettina bain mano unoftidione, Alla [quadra de Qaochi-hora fogging
Ed un pasticcio per vifiera in testa Lucha de' Battilani assai
Con pennacchio di penne di cappone Genre che a bere e peggio
Vn candido grembinl per sopravvcfta A cut battiers (diceva, )
Gii adornailc..,¢l'nno,el'altroarnione, Ch' affeddeddieci la dove si git
Vina zana è il [uo scudo, e nel? armata Noi non habbiamoa se i
Conduce tutta Norcia, e la vallata, 4Ma-s ha a far sempre la ¥
Segue Melicche Zanaiuolo di Mercato vecchio, uno di coloro, de' quali (¢



Sgaruglia. Questi condnce la schiera de' Battilani, che dice famo/«, ¢! ee
wi

gf



do con l'equivoco, vuol dire Attamata, da Fame, e non $4 Sensis
 =






BeSakeswFe gs

=a

TERZO CANTARE. 164

| PRITTELLE. Cosichiamiamo una vivanda fatta di pasta quasi liquida feieta
nell' olio da i Latini detta 4rro/aganus; e si come essi mescolavano con detta pa-
sia latte, ed altro, così noi pure vi mettiamo delle mele affettate, uva feccas,
latte, rifo, erbe, ed altro secondo i gusti. 1 nostri contadini nel tempo, che fan-
no ¥ olio costumano di far molre di tali frittelle, indotti a ciò da havere olio ia
-abbondanza, e ne danno anche a i vicini, e parenti; sono però soliti coloro, che
'vanno a veder lavorare, chiedere le frictelle, ed i lavoranti con poca grazia, ¢
“meno diferezione spruzzano Polio addosso a quel tale dicendo: Eccoti le frittelle.
-E da questo forse per frirte/le intendiamo macchie, che vuol dire Ogni fegao, o
“tintura, che sia nella superticie d' un corpo diversa dal proprio colore di quel tal

corpo, come ieee > quando l'olio casca sopra ad un punno. Ed il Poeta dicen-
“do, che costui molte frittelle, intende, che egli era alfai unto, come sempre
*sono i Bactilani per il continuo maneggiare olio, e lane unte,

A IOSA, In quantità grande. Diciamo nel medesimo signifitato a cafifo,in.
chiocca,a biftia, a fufone, voce usata da Giovanni Villani, a similitudine della.
Franzele 4 foifon, cioè con effutione, senza risparmio, 4 furore, 4 precipizio, a»

“bi > 4 Wome, e simili, Che se bene son modi baifi, nondimeno sono tuluolta
usati anche fra la gente civile. E questo a 4o/a credo sia parola corrotta, e che
doveffe dire a chiofa, che significa quelle cappelle, che hanno le bullette, ¢J
'ogni piccola piaftra di piombo, di rame, o d? ottone ridotta tonda, e simil »
'alle nostre monete, delle yuali chiofe i nostri ragazzi si servono per giuocare alla
«trotrola ta vece di monete, e però chio/a s' intende per moneta di niua valore;
Hi Persiani disse:

ail © * Ma vin tafea non ho pure wia chiofa

A mantenermi, in tanto qua pars est ?

Siche dicendosi: Della tal mercanzia ue n' era a Fof4, o a chio/a s intender y
the di quella mercanzia ve n'era così grande abbondanza, e per questo era a così
vil prezzo, che se n' haveva fino per una chiofa, Ii Berni nel suo Capitolo ia le-
de de' Ghiozzi dilic - 3 i

Segue da'que/to un' altra disciplina,
Che havend' ingerno, e del ceruello 4 iofa,
A ' -  Bisogna che v' habbiate gran dottrina,
H Domenithi in lode della Zuppa.
an E'quincs vien, ch' ella si suol gradire
Da chiha ceruello, ed intelletto a iofa,:
'vote vhio/# per similitudine significa ancora le Crofte delle bolle, E vuol

anche dire E(posizione, o comento, forse dal latino greto Glofa.. Dante num,2,
Purg. C, 11,:

E ferbolo # chiofar con altro refto,
E nelInhC.2y.disse Paranno s) the tu porrai chiofarlo,
Hi Varchi nel Capitolo dell' uova sode dice: 3
Es io fuffi Dottor, consiglieret 5
'Che sopra questo si dovelfe fare
“| Leagi, e statuti, e pos oli chioferet.
© PEGGIO delle spugne, Succia id vino più che non farebbe uaa spugna; cide

deve




168 MALMANTILE®S o@

beve affaissimo, come veramente fanno i Battilani, i quali chi f
pra in questo C, stan, 8. ' subi
BATTER la Calcofa. Frafe Furbesca, che vuol dir batter la firad:
€ questo parlar furbesco è praticato assai da ie sorta di gente.
AFFEDDEDDIECT, Giuro proprio de' Battilani profferit
una fola parola con due ff, e quattro d. i Bargilani
¢ sono molte persone a lavorare, hanno ogni dieci huomini un.
chiamano il Capo dieci, che e da loro ubbidito, e Mimato, e per
se del Dicci, intendendo di costui, flimano di fare yn gidramento. fo)
Credo mondimeno che dicano a se de Dieci per non dire a se di Dio,
dicono per Dianora, Corpo di Dianora per la medesima ragione.
SCARD ASS AR /a lana, Cioè pettinare la lana con ini
no Cardi, perché hanno i denti torti, e simili a i
foglie, il fulo, ed il fiore deli erba detta cardo, del qual fiore
si servono per pettinare, ed unire il peio.de i pauni, e pero lo
ed e i) latino car minare. Vedi sotto C. 7. stan. 37.: 3
FAR la lunediana, Appresso a i batiilani significa non lavorare;.¢ que
ché nel tempo, che l'arte della Jana Javorava, costoro guadagnayano. lai, ed
erano pagati dalli loro maeftri il lunedi, dove gli altri oo i
fabato, e però questo giorno del lunedi,cfiendo per loro giorne d'.
la riscoffione, era da essi Jolennizzato, e non voleyano lavorare, (ma fl
fefta) a consumare in bere, ed in mangiare quel denaro, che havevano'
e guefta loro folennita chiamavano Lunediana, cd alle volte Lunigian 15 ed
da essi tal fefta così osservaza, che tra loro era la seguente cantilena,



























Chi non fa la lunediana, on entngaite

E' un gran figlio di puttana, z ae
Ed oltre a questa ce n' è un' altra che dice me

UVenerd: de Beccai, 4 if

Mt fabato de gli Ebrei, r

La Domenica de' Cristiani, TE

E il lunedt de i Battilani, '

Si che dicendo /unediana s' intende fella, come si yede nel presente
che Sgaruglia dicendo »° ha afar Sempre la Lunediana, ¢c, intende hada 7
pre fefla. Questo nome di Lunediana resta ancor' hogsi » ma come che i

'orza flare alle volte le See

Jani sono pochi, ed i lavori meno, conuien loro per
timane intere senza lavorare, e Così non € mefla troppo in plo detta fo
anzi hanno di grazia, lavorare anche il lunedi,

, ' Stren ZA oe



Conchino di Melone ecco s' affaccia, Che tutti allegris¢ rubicondi in
Che ? Offersa tenendo de gli alloré Cantando nua canzone A'
Col fineye aldo d'un buon pro vi faccta Di gran coltellise di ragliers arpa
Ha dato un frego a tutti s debitori, Si son per amor [uo fatti soldati,
ue Conchino di Melone, il quale si.conduce dictro una mano de* (oi dh

ae che si son fatti soldati per la cortesia, che ha fatto loro di sc
ti il debito, che havevano seco, fu eoflui già quoco d' Ofteric, e per eli }





ee. Se ee

ae


“ee Tt. 4
TERIZO CANTARE: 169

BR to gtaffo, edi flatura piccolo fu chiamato Conchino} gli venne voglia Ui diventar

.  macftro, onde prefe sopra di se un' Ofteria detta ii allori, dove subito hebbe»

am = molti i, ma tutti a credenza, per lo che presto falli; e non trovando mo-



do di rifquotere un soldo gli venne rabbia, ed abbrucid i libri per-non haver di
più paflione di vedere scritti i suoi denari, e non gli potere spendere. E
intende dicendo ¢ol fine, ¢/aldo d! un buow provi facia ha dato frego a tutti è

» S*e4 SP ACCIA, Si fa innanzi.. L' Autore si serve di questo verbo afacciarf,
per denotare, che costui havea la faccia larga; (cherzo assai praticato con uno,
cre habbia gran ceffo dicendolegli afacciarevi, facciami favore, facciami buon vifo,
efimili.

TAGLIERE » Intendiamo un' arnese da cucina,fatto di legno,tondo a foggia
di piatto per uso d' affettare sopra di efio carne, e per triturarla con quei gran,
coltelli', e farne polpette, o altri batcuti. I Tedeschi usano in molti luoghi 1 piat-
ti da tavola fatti di legno, e gli chiamano T-alier con voce venuta d'Italia, come
si può eredere; già che i nostri antichi i piatcelli, © tondini dal tagliarvi fu le»
vivande, domandavano taglieri, onde il proverbio-. Due ghiorti a un ragliere,cioè
4 uno fheffa piatto'. Trovali questa voce nella antica lingua Gallefe, o Francesca;
¢ dicevano railfeor; come leggefi in un' antichissimo libro in quella lingua,dal Lat.
volgarizzato, appellato de] Conquijfo della terra Santa di Gerufalemme, i) quale fié
ritrovato essere di Guglielmo Arcive(covo di Tiro; e si conserua nella preziofiti-
ma libreriadi Manoferitti del Serenifs. Gran Duca, appresso alla Chiela, e Col-
legiata di S, Lorenzo. 1) paffo tutto: volrato in Tofeano dice così; La dentrofin

irea') fu'trovato un vasello di pietra verde, e chiara assai di troppo gran.
belta, fatto così, come un tagliere. Li Genovedi pensarono, che ciò fuffe uno
fmeraldo, Perciò lo prenderono a lor parte, de) guadagno della Città per trop-
po gran somma d'avere. Portaronncio in lor Città, e ? appefero nella Maftra
Chica, ove egli'¢ ancora. L' huomo vi mette la cenere, che si prende il primo
giorno di Quarefima., e si mostra altres! come ricchissima cosa, Perché ¢' dicono
veracemente, ch'egli ¢di smeraldo. Nel margine vi e questa postilla in nostras
lingua, ido\, e dove ¢' Genavefi guadagnano el catino di fmeraldo, che ten.
gono ancor'

gio Criflo Giesi alla gran cena.
STANZA LXII.

Scarnecchia che di guerraé un ver copidioy
L! Eroe degli arcibravt, e dico poco,
A cui dovrebbe dar piatto, e stipendio
Chiungue governa in qualfivoglia loco,
Percht quando seguiffe qualche incendio
Ei fa il rimedio per guarir dal fuoco,
Mena gente avanzata a mitre,e gogne,
Da vender Siabeschiacchiere,e menzogne,

i nel monte di S. Giorgio, e credefi, che sia i piatte, dove man-

STANZA LXIII.

Rofaccio con aleissime parole

Movendo il pie yacconta,c 4 pigione 5

Fa per quel mese dar la casa al Sole,

E nel zodiaco alloga lo Scorpione;

Cusi shallando simil ciance, e fole

Si tira dictro-un nugol di persone,

Fa per impresa in mezzo all internallo

Di due sue corna un giobe di crispalio.

Seguita Searnecchia. Questo fu un Montambanco o Ciarlatano, il quale ven-
deva unguento per medicare scottacure, e montava in palco sempre in abito da
Coviello col nome di Capitano Scarnecchia,. faceva una mano di braverie a

' fine











170 MALMANTILE( ¢

fine di ragunate il popolo, e però It Autore lo dice
de li arcibravi, B perché e Ciarlatano, lo, faycapo di Monelli; © ON
alla berlina » e che & buona.a vender bugie, come perlo pi sono
chi. Dice che doverebbe esser provvifionaro,; pecehe hs iene
dal fuoco le case, che abbrucialsero, e s(cherza, burl: Q
deva detto Scarnecchia buono.a-guarire le scostature in godin
dolo buono a rimediare a gl' incendj.

MIT RA, o Mitera », Diciamo)quel:foglio, chera foggiadi coront si
capo a coloro, che per delitti son Tota © mandati in ute 'afing
C. 6, stan, soeC, 12. stan. 19

GOGNA., E' lo stesso che Berlina detto sopra C, 2. stan. ay. I aaa
no Wumelle, se ben-questa era più toNo una specie di ceppi da serrare ip
de forse meglio con Piauto,-e.con Lucilio la chiameremo colfare. 9
FLABE, e menzogne* Sinonimi, che significano Bugie. Fiaba Pe fab
menorna dal verbo mention, 1 ate
Dopo li faddetti vien Rofaecio y il quale conduce seco una. gran mano. i
ne tirate dalle sue chiacchiere. Costui fa ung de i più superbi ciarloni
mai stato nella Ciariataneria, e spacciavafi per Attrologo. Noma
banco, ma stava a cavallo allato,a una tavola elevata, sopr' alla qual
una faragine di cartapecore di privilegi havati s diceva egl ) pee il
da i maggiori Potentati della [eon » qualche seheretro di gatro;0
sfera d' ottone, tre corni neri lunghi, ail uno de' quali era appefe unip
calamita, all' altro una palla di lumpidissimo Criftailo dil Monte, ed
corno, che cgli diceva eliere d'Vnicorno.. Vendeva una fuacmeftura:dailut chia~
mata con vocabolo Greco Wepensbes, che diceva-esser buona a rutte Pi
conforme al medicamento d' Elena chiamato con queste medesimo
penthes ( cio' di contrartoal dolore ) da\ Poeta nel4.dell' Viiflea, ed-a chi lacom-
prava donava un' anelletto d' oflo, che (pacciava' per ottimo aldol fla;
per esser fatto di dente di Cavallo marino, Diceva havere, impa;
già da un gran Mattematico, ed Alirolego suo Zio nominate Gio
cio., che predifig s vantava egli }.la rovina della palla della Cupol
di Firenze molto tempo avancd, che cella seguifie. In somma:con le
fandonie ragunava sempre, che 1 montava a cavallosinfinite persone;@
buone fomme di danari; 11 Poeta lo fa condotticre di questa ge
le chiacchiere, e gli fa fare per impresa quei wre suoi corni persiaier
di criftallo. pocikrutagt

eALT/SSIME parole... Chiama parole altifoime quelled Rofaccio; pete
sempre dilcorseva di plancti, di stelle,  d? alevevcofe:celefti.comeme
tore con dire, che egit ha affttaralacafa al Solose meffole Se
Senza ironia Dante nf..4, chiamo Virgilio; A aleiffiens Poera -By
Così vidi adunar la bela scola Di-guet Signor det atei/jima canto, O
rissime canto chiamala on gale In Otbimo;e ornarissime
cia turte le dottring,e maffimela Teologia,imperochei primi P
- SBALLARE, Vuol Propriamente dire cava
esprimere uno che racconti moire 5 e molte cofecpiut

















¢} Domo









oe ee


AS

SAN

ut

SERRE

Bak

TER ZO'CANTARE: i 71
verita,ed dil medesimo, che//chianrare, che vedremo sotto C. 10. 'stan. 66. Questa
voce sballare in al ficato vedremo forto'C.'11. stan. 4.

~ CLANCE ye fole, nimi}; eP ultimo @ Sincope di favole; ed intendiamo-

chiacehiere lontane dal vero. Petrarca Sogni d'infermi, e fole di Romanzs, li
Mauro jin biafimo'dell' Onore disse: a

DS Har-abdieh® ia y che le son butte-fole,

SS uttiargumenti da ingannar gli feiocchi

HO OL 9 a Le case che confiftone in parole,
Ti Persiani'in'una sua canzone dice oon,

> i Se con ragliare o fole
ity stew 'Ve pagar di-bravara

“Ottavio Pertari nelle sue Origini-dedudele parole Ciance, © Cianciare da Can-
tiones j Cantionate's It Boce. Now: 61. quando disse tla landa di donria Dfatelda', e
corali altri ciancioni volle dire senza dubbio canzoni, le quali ( perehé erano molto
in pregio le Provenziali, o:le fatte fa VY ariedi'Provenza, come si vede da alcu-
neinttolazioni'di Lande antiche*) chiama come' per iftrazio, € contraffacendo
in questo, ficome in molti altri luoghi,la pronunaia delle lingue Mraniere; cian-
ciont'; Scherzando anche nel medesimo tempo sull' altro significato, cioè di ciancta,

VN nugolo di persone. Questa voce nugolo per Quantità grande,è assai usatas
dainoi y el*usdiil nostro Poeta sopra-C, 1. fan. 50. Così Giuvenale Sat. 13. imi-
tando inci Omero'; chiamd la molcitudine delle combattenti griy, nubem fo.





nora ON1 s2
: an iz §$ TAN ZA LXIV.:
Sopr' un lettoriechissime fiorito E pur, vin arme ei non fu gran perito,
“Rartar: Pippa si fa del Caffigtione, Guerrier comodo almen nel padiglione,
Ove coperte fea tutto vefito, - Queffo impera dal morbido piumaccio
| Ch'in tal mado (0 foalda al suo padrone; et quelli del meftier di Michelaccio,

Seguita Pippo det Ca/tiglioni portato in un ricco letto, di dove comanda a i fol.
dati 5 the on ata get ees vo di lavorare. Costui era il più grazioso,'e
faceto umore, che'fia mai stato in Firenze, e i chiamd Pippo del Castiglioni, per-
ché servi lungo tempo a i SS. di Casa Castiglioni con fedelta indicibile, e pero da'
medesimi $$, aniato a/segno', che non ostante le burle, che in diversi tempi, ed
occasioni: faceva.a efi SS, non potettero mai mandarlo via, perch, s¢ lo licenzia-
vano', egli trovava sempre vaghe invenzioni per non fen' andare,' come fra le»
molte fu questa..: Il sig.\ Cavalier Vieri da Castiglione, al quale per ordinario
serviva, lo-licenzid con queste parole: Sgombrami di Casa. Pippo andato in Piaz-

za chiamd prot Carrettai, e condottigli con le loro carrette d' avanti alla,
porta delltabicazione di essi SS. in sy Y ora, che i) sig.\ Cavalier Vieri foleva tor-
nare a desinare, ordino loro, che, se il medesimo sig.\ Cavaliere gli domandaffe

quello, che facevano quivisgli rispondeflero, che ve gli haveva mandati Pippo; si
come

segui ed il Sig, Cay. disse: che hada far'Pippo delle carrette? Ed-egli a
queste parole scappato di dietro a una di esse carrette', rispose: Sgombrare, co-
me VS. Llluftrils. m' ha'comandato; Onde il Sig, Cav. ridendo della faceta in-

a amen => del- sao comandamento lo richiamo in cala, e pagati i carrettai gli

' = Ys IN


. be flato assai di notte. Pippo si scordo di mettere il caldanetto nel let

ip MALMANTILE

LN un letto riechissimo fiorite, HW medesimo Sig, Cay, una fera con
che facefle, che il letto fuffe caldo, quando egii tornava a dormire 5













tornato il Padrone, e volendo andare a dormire, Pippo si trovo:ii
perché stante l'ora tardissima non vi era modo di trovar fuoco; ricorle:
solite afluzie, e questa fu, che egli per la parte di dietro del letto v' ent
tro così vestito com' egli era, ed il padrone, credendo che sli andafle mo
lo scaldaletto, si poglid da per se per non lo scioperare', e spogliato a
volta del letto, e difie: Cava il fuoco, ed alzata-la cortina y
vedde Pippo, che follevata alquanto la testa disse; Signore il letto non
caido a bastanza. Il sig.\ Cavaliere vedutolo così, e conoscendo l'umore:
bestia senz' alterarsi lo fece uscire, e toltafela in pace entrd nel letto così com
era. E per alludere a questa, facezia il Poeta fa venir Pippo portato in un
chiissimo letto.. ors
PiVatACCIO., Guanciale lingo quanto la larghezza del letto; della grok
za d' un facco ordinario da grano, ed e ripieno di piumeye però + Pinns
cio. Qui per piumaccio intende tutto il letto. 1 ah
QELLI del mepiero di Michelaccio, Gente, che non ha voglia di
che il meftiero di Michelaccio dicono, che era mangiare, bere,e,
Qui pure bisogna, che il Lettore si contenti ch' io faccia un poco di.
ne per narrare alcune delle facezie del detto Pippo, meritando: la
cita di questo huomo, che si spenda qualche di tempo in sentire:
guzie, il quale è viffuto fino:a pochivmefi addietro d' cta di 8s. anni se
la medesima bizzarria, fauo che, dove prima frequentava molt
trovar le conversazioni, che gli pagavano lo scotto, ( perché
quattrino,dando egli tutto quello » che guadagnava alli suoi vecchi
dre, alli quali continovo d' ubbidire come un fanciullo fino al” eta sua di sopras
75. anui, che essi paflando cento anni, morirono ) dopo la morte del Pad
quento pil le Chiefe pregando S. D, M. per la falute del. Serenifs.: G. 'Daca » dal
quale gode fino, che wile, onorata proyifione per il buon servizio
nissima Cala. hbase
Essendo una volta il medesimo sig.\ Cav. Vieri al Poggio a Caianot
'Serenifs. G. Duca }.a scruire il Serenifs, Sig, Principe Card, Gioy a
Pippo a Firenze la vigilia del Santifs, Natale ordinandogli, che si facefle-dare dil
farto un suo vestito nuovo.,'¢ lo portaffe al Poggio., <T ordine, cheigli diedef
con.queste parole: Va a Firenze, e fasti dare dal farto il anio vestito ye portale: We.
bidi Pippo > e la fera-medesima tornd col -detto vestiro del padroneun f
entrato in Chicfa do ve era tutta la Corte per-udir la Mefla-(mancandovi
sig.\ Cav. Vieri, che se ne flava in camera aspettandoiil vestitu per
veduto da wttii Cort igiani, e da-tuui li SereniGs. Principi che quivi.
il sig.\ Principe Card. Gio, Carlo gli disse: sig.\ Filippo che:colaé questa? Val
fiate molto nobile ? Ed egli tispole: Screnil..queste son graziesche mi
Padrone. & S. A. Rev. immaginandosi di come stava:il fatto si,
Pippo sil quale fatte, pity, (paiicggiate per la Chiefa. fen.andò alle stami
Padrone;, che vedutolo con quell' abito in dosso lo sgridd dicendo.; Briccone!









See Se ——— ee





eR gS ce a
Se Eatetetarkies

ae.

SShERs

54
ee

RERLRASEE | Et

&
=

SERA!

TERZO CANTARE: 473

Siam fratelli? Rispole Pippo: Perché sig.\ ? Replicd il sig.\ Cav. Che furfanteria.,
¢ la tua mettersi il mio vestito? Mi maraviglio di V. S, Lluftrils. ( foggiunfe Pippo)
non me l'ha ella donato ? Come donato ! (disse il Sig.Cav. ) Ti par' egli abito da

co eae,¢ mi fla benissimo; E V.S. Iiluftri, medesima a
ha detto., che io me lo

cia dare dal sarto, e lo porti, ed ecco ch' io 1" ubbidi-
sco, già tutta la Corte ha saputo questa generosita di V. S, llluftrifs., e si sono
rallegrati meco del:regalo, che V. S, Uluftrifs. mi ha fatto in questa folennita.
Il Sig, Cay. conofeendo, che non era svo decoro il mettersi quel vestito, che era
flato yeduto in doflo al suo servitore, stimd bene il quietarsi, e fargliene un re-
galo, per-non — far' altro; Ecosi Pippo si godé quell' abito, che per la sua
ricchezzacra ite a. un Principe.
Era grande amico di Pippo il Poon Fantacci oggi vivente Rettore della Chiela
di Varlungo fuori di Firenze circa un miglio,il qual Prete è flato sempre huomo
assai faceto, e piacevole; e fra eflo, e Pippo son seguite diverse graziose buries
¢ fra I altre il Fantacci disegno una volta di fare star Pippo senza.cena, e necel-
fitarlo a.dormire all' aria; € per questo l' inuito ad andare alla sua Chiefa a Cena
ella fera appunto, che il Prete havea fermato d'eficre acena nella Villa de' SS,
Pont quivi vicina; e ad effetto, che gli riu(cisse il disegnoshaveva ordinato alla.
serva che andafle a dormire a casa una sua parente, e detto al Contadino, che

on alla Chiefa, che, se.fufle accaduta cosa alcuna attenente alla curayman-
7 'a oe

Prete di Rovezzano,Chiefa vicinissima a quella di Varlungo. Pippo chie-
fla,ed ottenuta licenza dal suo padrone,la (era al ferrare delle porte della Città,se
anvandd.a Varlungo, e trovata ferrata la porta della Casa del Prete,.dopo haver
molto picchiato,conosciuto » che:non era veruno in casa, disperato s'accofd alla
cafa'diquel Contadino., che haveva l' erdine di mandare la gente a Rovezzano.,
eda eflo intese, che il Prete.era andato.a-cena fuor di cura, e gli ordini che ha-
ea lasciato.s, Pippo accortofi molto bene, che il Prete ' haveva burlato, volles
renderglilapariglia, « perciò fare trovata una scala a pivoli, con essa-montd sa-
pra il tetto della chiefa ye.quivi portata buona quantità dijpaglia, ed altro ciar-
= combultibile,¢ raro., gli dette fueco, ed andatoalle funi delle campane

messe a suonare a rintacchi. 11 Prete Fantacci., che era.poco lontano sentendo
suonare a martello, st affaccid\a una fineltra -per sentire, che cosa fufle quella.,
evedutoil fudco sopr' alla sua Chiefa, tutto spaventato lascio la cena, e l'alle-
gria, e-corsealla volta della sua cafas ncljla quale (ubito.entrd per-vedere doves
era il fuoco., e rimediarvi-can.}'aiuto d' una parte de' SS. Commensali, e con,
uina-quantità di contadini., che.già erano quivi.concorsi con zappe, e pali per ro-
winare),¢ tagliare dove bisognafle. Pippo intanto scefo.dal tetto se.' andò
-ad.arno,¢-si fermo a:cena da.un tal Boni mugnaio suo.grande amico,, baltan-
doglid-havere furbata I allegria,nella.quale era.il-Prete, il quale girato e oy
to.,\¢sopra..per tutta la casa.,.c non-havendo trovato ne meno-segno di fuaco.,
fece viltase il tetto della\Chiefa,:¢ trovo:la.paglia, che era.finita d'ardere, es
vitta la-feala appoggiata alla.muraglia., s'.accorse che era Mata una contraburla.,
di-Pippo, tanto ne sche silcontadino, detto di fopia-disse haverlo.yeduto poco
prima, ¢perciò sopportandofela in pazzienza,tornda ceaare, dove non man-
carono le minchionature 5x¢ barzellette y che fucono da quei SS..della conuee(a-
zione dette.al Prete. Cow-
174 MALM ANTIDE 7

Commeffe una volta Pippo non fo che mancamento, per'
volle mortificarlo col mandarlo in carcere, onde gli fece dare
un biglietto, acciò lo portafse al Segretario del Magiftrato
glietto diceva, che fulse ritenuro il Latore in fegrete fino a nuovo
prefe il viglietto, e indovinatofi del contenuto, e parendogli duro
in prigione in tempo di Carnevale, e sapendo, che il'non portare il ¥i
delitto da galera, andava mulinando come potefse faluare la'capra jt
quando la fortuna,nell' andar' egli come la ferpe all" incantosgli fece
nanzai un Tedesco giovanetto servitore di liurea del medesimo sig.\ C
Padrone, alla volta del qual Tedeseo andato Pippo » quali brava:
Padrone e in collera, che tu fei flato tanto a venire', perché vole
taffi questa lettera al Sig: Segretario de gli Omo, e perché én
mandava me; febene, ho da fare afsai fu in Palazzo; pigliala ye
do. Il buon Tedesco non pensando alla malizia porto la lettera\, in
degli ordini- della quale i} Tedesco latore fu ritenuto in carcere', ¢
che S. A,S. era restata ubbidica.. Pippoil dopo desinare'del medefi
vesti da donna, e senza maschera con le sue propric bafette ye barba se
seggiava il corso delle maschere,havendo d' attorno un popolo infinito *
tea vedere questo tumulto i} Sereni(s. G, Duca, che pa(sava:in carroza
la firada, onde (pedi uno flafiere per intendere che cofafulse. Lo
no, dicendo che era Pippo del Cattiglioni in'matchera da: donna 5
che già fapeva del viglictto,replico: non può efsere, ondesil Caporal
fieri andò da per (¢, etornd replicando efser veramente Pippo nel
veva detto lo flaffiere'; in tanto S, A. S.s' accofto,¢ Pippo che gli
tro,ed haveva ofseruato, che S.A.S,haveva mandato due volte a veder chieglies
fattole una grandissima riverenza dilse: Sereni/s, io son io,io fon'io,percht
wm' ha fatto il servizio di portar la lettera lui; Finalmente conosco bora
chi si fa ben volere po [perar sempre questi, e maggiori servizz).
rife dell' afluzia, e ordind che fufse scarcerato 11 Tedesco.
ig. Cav. Bernardo fratello del sig.\ Cav. Vieri haveva presa I
efta dama volendo elser servita da Pippo per bracciere,

uomo ¢' eta, e veltiva'di nero, e non con la liurea',come gli altri'!
quella Casa, prego il suo sig.\ Conforte, che lo chiedelse al; 'atello, perché servisse
a lei, 11 sig.\ Cavaliere Vieri gli compiacque, se bene cop poco 3
perché era avvezzo a anes fuori di quelle i bizzarrie lo se: rar
e con meno gusto di Pippo, che non avvezzo a servir dame gli a
versi ad aveensate in sua vecchiaia 'e mal! volentieri lalchea il suo padroney it
diferecezza del quale non sperava trovare in chi che sia; onde prego la 4
che'lo yolefse lasciare al servizio, che era solito; ma la'Signoranom
mutatfi di proposito; per lo che Pippo sigettd alle invenzioni il
con riputazione, e con operare, che la Signora lo licenzialse,
mettelse mancamento, Chiamd dunque a sse alouni ragazziy e dil
alcuni pochi soldi, impose loro, che quando lo vedevanovcon)!
dafsero tutti a gridare Pippo, Pippo, Ecco Pippo,'¢ glitacelsero
I ragazzi invitati al loro giuoco, e che haurebbono-daco quaicola'
































—. —we oe Row eee





2e 2 ow an =o SS fF ea eB H.-S ese oe

x




5

TER:Z:O0 CANITARE.
nit reloceafione di far quel chiafso, appena lo veddero ulcir di cafaydanto il braccio
alla.Padrona,s che cominciarono a strepitare, e tagunarono quivi quanta gente |
i era in quei contoral'.¢ Pippo favio, feaza mutarsi in facia seguitaya a dare il
sai) bracciovalia Signora,,.la quale vergognandosi, che il suo servicore fae lo scicr
volun 20'del Popdlo., che egit fulse trattaco come un pubblico buifone ys) aftretto, di

giugnere in Chicla, pen(indo,, che quivi almeno dovelse fermarsi il baccano 5),
niayfe.celsd il pelacajion fini il inicia perché quei ragazai standoli cuttiat.,
» non geidavano per rispetto della Chiela, ma erano cagione y che wiro i
pepo guardalse verso quella parte; per lo che la Signora per liberarsi ordind a»
ippo, che andafse a casa,¢ mandaGe un' altro servitore, e tornata poi 4 ca/a
le parue mill'anni render Pippo a chi glicl' havea cunceduco; E così egti ricorna
al primo ferwizio, sicuro, che alla Signora non farebbe mai più venuta yogiia ai,

eum 208 (ernireda lui. yeaah
jade, Havewadl sig.\ Cav. Vieri una bellacagna da Fermo, la quale diede in cura.a
f dicendogli: Tien conto di questa cagna 5 ed avvert a non la fimasrire y

om perché se la finarrifei non ti aspeteare altra licenza. Prefe Pippo la cura della ca-
ga $24» €col trattarla bene avvezed a fare mille giuochi, e se la refe così alic~

| 2onata', che era imposfibile, che egli la smarrife. Avvenne, che Pippo fu in-
vitat® a una fella', che & dovea fare in un iuogo poco lontano da Firenze, dove
era per tratcenceli almeno tre giorni, onde chicle al padrone ticengia per a quel








te



oe tenipo';-ma non ltottenne, Pippo senza moftcar di ciò disguftoy la mattina avan-
| tivalla-wigilia di dewa fefta:comparue in-cala (enza la cagna, ed il sig.\ Cav. do-
J mandi dov' ell'era.. Pippo dilse quasi piangendo:. sig.\ io non1o.s03, quando io
i fubvicino a: case mix ierfera ella cominciò a fuggire, e per hholto., che 40 le core

mf relsi diettd chiamandola, non fa postibile farla cornare y ne arrivarla.. Replicd il
| sig.\ Cavaliere; Tw fai i parti; pero va a fare i faci moi, e non haver' ardire di
od mettere il piede ineala postea (enza la cagna. Pippo fingendo un dirottissimo
4 pianto fen' ulci di casa, e andò alla fefta, alla quale era stato inuitato, e pafsati
®  alcuni giorni in grandidima allegria se ne torad a Firenze 4 e andato fuori della
porta alla Croce da uno Ortolano suo amico, al quale haveva lasciata la cagna,
; se la prele, e I" infango tutta, e le infanguino l'ugaasaccid parelge (pedata, e Ie~
i gatala con una corda: lx condu/se al padrone, il quale veduto Pippo con la cagna
giidibe: Dovel hai trovata? In Casentino,Liuttrits: Sig,, e nomci voleva altri
| cheme:per trovare il-luogo dov' ell' era fitta. I sig.\ Gav. credette quanto dife
Pippo, il quale con tale invenzione gode la soddisfazione, che bramava, E tan.
torbatti ptrun faggio delle facezic di Pippo, il di cni.intero nome, © cognome

oe
SPAN ZA LXV. STANZA LXVI.
Cenro fuggerti egli ha della fuselaffe

inch egtine Pigmei disterti., e brurti
Fanti che nacquer nelle magne baffe,
Mita se ben son piccini, vi son tutti,
Mangian /pindci,arrufian le mataffe.,
Ea ha più viz2j egnunydi fei. dargusti,
Cosa è quota che va per il sue dritto,

Che non è in corpo storto animo drittas

tf girea Bariffone adele rocca
Gran gigante da Cigoli di quelli,
'Che vanno a corre i ceci con la brocca
| Ebattoncon le perriche i baccelli:
Ler sue bellezze amore hasipreincocca
Per ferir Dame i dardijed e guadrelli,
. Fa il Cavaliere nelle cavalcare,
2 va [pel furiero alle nerbare.



|
|








196 MALMANTELBhAT

Segue Batiftone Nano con una gran quantità di compag al
bene son così piccoli, son tutti viziofissimi, e non Segoe 3
ché in un corpo mal fatto, di rado si trova anima ben outa

BATISTONE, Questo fu un Nano levato da guardare le pecore  ¢
a servire il Serenissimo Principe Mattias di Toscana, dove in! 6
in sul posto di bello; e facendo lo spafimato di tutte le Dame, ©
ce: Per sue bellezze Amore ha sempre in cocca Per ferir Damei
arrivO a segno questa sua inclinazione aile dame, che per porere liberas
ticare con esse, si contentd che il suo Serenissimo Padrone lo facefle
come segui, ma però in burla, e stette nelle mani di Maeftro Agnolo”
' Castratore circa un mese, sempre credendo d' essere flato castrato: EB
H egli, non ostante che fufle di statura piccolidima impard afai bene a
| € maneggiare ogni cavallo aggiuftacamente, supplendo con la mano'

che gli mancavano le gambe, era (olito ancor egli andare nelle ca
i Cavalieri, e pero dice: Fa i/ Cavatiero nelle cavaicate. Ma percht quel
( di Caramogi e assai fottoposta alle mazzate del padrone, ed egli ne hai
sua parte, però il Poeta dice; Va spesso Furiero alle mazzate. Questo Ni
! la morte del Sereni(simo Principe Mattias servi al Serenissimo Gran Duca if
lita pure di Nano,ma esercitava anche la cucina fegreta diS. A, S., nel
stiero s' era fatto peritissimo, per lo che oltre alla buona provvifione¢
buscava gran mance; ma la Fortuna l'abbandond ia sul buono,
si egli innamorato d' una bellissima giovane sua pari disnatali 5 la

glic, ed in pochi giorni mori. Lo chiama Gigame da Cigoli Ȣ
quelli che colgono i ceci con fa brocca, come si fa de i fichi, e che base:
(a pertica, come si fa delle noci, non potendo arrivargli altrimenti.
Gigante da Cigoli,in una collinetta vicina a S.Minjato al Tedesco,si
le donnicciuole, una Iperbolica cantilena antica, la quale dice, ) joe

Ed! una punta d' ago rogebiols

Ne facea pugnale, e spada, itis it
E di quello che gli avanzava de
Ne faceva uno spuntoncin,

ae




















E is questa ilena con altre iperboli retrograde simili ae
re la picciolezza di questo Gigante da Cigoli; e di qui e in uso comune il dires
Gigante da Cigoli a un Nano, che i Latini dissero Pumitio, e noi dici ne
Ledina, similitudine tratta dal giuoco della dama; Sericciole da un'
coliflimo di questo nome, Pimmeo dalla voce Greca Pygmaios 5 che' t
dell' altezza d'un pene « 1Greci dicevano Manus, Pujilius quantus Molo y ed ae
tre volte gutta; ed un Pedante lo chiamo Titsviditinm Scarabei umbra. a
Strada nelle sue Prolufioni, parlando d' un Nano dice: Fangino be seem
capite se torum tegit, Ed altrove, pure nello stesso proposito dice; iis #
cin 5 Somninm hominis, falsllum anima. xa

BROCC.A, Voce, che viene dal Greco Brochos secondo il Monosino, e &
condo altri dal Greco Prochoos; il che e più verifimile, essendo questo valo d
acqua, e quello vaso da vino; e vnol dire un vaso di terra per uso di |
acqua, € pero detto Aydria, e noi lo chiamiamo brocea;, Chiamali broc







a <n


AA

SERL

We

zt

SR

Le ' om |

q

TERZO CANTARE,; 177
ancora uno flrumento fatto di canna rifeffa in più parti; se quali allargate,¢ ria-
'teflute con falci, formano comé una piramide'a rovescio, e di tale strumento
fermato in cima a una pertica, ci serviamo per corre i fichi,quando non si potlo-
no arrivar con le mani; € di questa brocca dice nel presente luogo

~ FVRIERO, Si dice colui, che va innanzi a preparare gli alloggi nel viaggia-
re che fa un' Esercito, o altra gente in buon numero. Lat, metacor mwanfionum,
Tn Latino barbaro dicefi fodrarins da fodrum voce che vien dal Germanico, la.
in buon Latino si direbbe'alinentum, pabulum', annona; Onde Foraggio, e
Foraggiare', Provifione di guerra,e provvedere l'esercito. Tuto ciò si offervd dal
Ferrari nelle Origini alle voci Foraggio, e Foriere, Ma erra quando piglia Frie.
re dello /pedale, che si trova in Gio: Villani lib. 8. c. 95. per accorciato da Foric-
re, sia Provifor bespirij poiché quivi, si come appresso al Bocc. Nov, 92. si-
i. Srate dal Pranzefe pees cone si domandano anche oggi i Cavalieri di

alta. Qui si serve della voce Furiero per intender fur:a che suona quantità,
come dicemmo sopra in questo Cant. stan. 50. e vuol intendere, che queito Nano
spesso toccava qualche furia, cioè quantità di nerbate. Vedi sotto C, 9. fan. 49.

PIMMEL » Beano popoli nani, che habitavano nell' ultime parti dell' Indic,
i quali. crescevano fino all' altezza al più d' un braccio, e le loro mogli di cinque
anni partorivano, ed otto erano vecchie. Di eu fa menzione Plinio lib, 4.
cap. 11. ove dice i barbari chiamarli Cathizi, e lib. 7. cap. 2. Cofforo per efscr
così piccoli erano infeftati, € rapiti dalle Gru, onde per difendersi andayano
armati di'frecce; e cavalcando sopra alle capre in granditlime schicre,a guattare
iloro nidi, e romper loro ! uova. Di questi parla Giuvenale fat. 13. dicendo.

Aad fubicas Thracum volucres, nubemque fonoram
Pygmans parnis currit bellator in armis,

Mox impar hofti raprnfque per aera curuis
Vugiibusa fava fertur grue: Si videas hoc
Gentibus in nostris, rifu quatiere, fed illic,
Quamquam eadem alfidue /pettemur, pralia ridet
Nemo, xbirora cobors pede non est altior uno

NELLE magne baffe. Intende che sono di flacura baffa, se ben par che dicas
sieno nati nella bafla Alemagna. Lat. Germania inferior,

SE bene e' son picein? vi son tutti, Benché piccoli hanno malizia quanto un,
grande. Tydeus corpore, animo vero Hercules; da Omero, il quale descrive Tideo
il padre'di Diomede piccolo si di statura, ma gagliardo.

MAKGVT TE, Che Nano fuffe costui, e quanto fagace, e scellerato, vedilo
nel Pulci nel suo Poema intitolato il Morgante ? Questo nome di Azargusre forse
fu finto' dal Puici a similitudine di 4¢ergite, Personaggio famofo per la sua sccm-
piataggine, il quale fa il faggetto d* un intero Poema burlesco di Omero; e ciò
pore avere imparato il Puici da} suo dowo amico meffer Agnolo da Montepul-
ciano. ©
NON è in corpo forts anima dritta, Non & in corpo mal fatto, animo ben,
composto, giusto, e che tiri al buono; che tanto significa la voce dritto in que-
sto luogo. Sidice anche: Vn segnato da Dio, non 4 mai buono: (alludendo per
avventura a Caino, Gen, c. 4. vers. 15.: quali che quel tale sia in un certo mo-

do








178 MALMANTILE ~ %

do contrafflegnato, affine, che ognuno, che lo vede si guardi ) qu:
praticata comunemente, e si vede da i seguenti versi maccheronici

Nulla fides gabbis,& parum credite ruse, s

Si guercius bonus eft, inter miracula feribe, i

Vn' altro Poeta in questo proposito disse: Chiude un' anima bigia
Che huomo bigio intendiamo huomo cattivo, di poca coscienza, ¢
gione, Marziale. Crine ruber, niger ore, brevis pede, inmine lafus Rt
preftas Zoile, si bonus es. Quel Terfite, che quanto sconcio di vilo, e sc
to nel corpo, altrettanto era brutto nell' animo, ¢di costumi
fopportabili; vien descritto da Omero al 2. dell' Iliade ( secondo la tra
Pictro la Badefla Meilinele, stampata in Padova l'anno 1564.)

Lusco a' un' occhio, e a' um pic oppo, e firetto

Wegli omeri, che gobbi ha sfin' al colle;

Aguxro il capo, e'l capel cre/po ye raroy

Sucido ye ner, lentiginofo, € marcio,

ZA LXVII.



Piena di fudiciume,¢ di frambelli eUacfiro de? Bianti, e de' Monell,
Gran gente mena qua Palamidone, E veste la corazza da bastone y
Chril giorno vanne a Carpi,ed a Borfelli, Perch egli quant' egnialtra (uo alte
E la notte al Bargel porta il Lancione, E' tutto il di figura di riliewe,



Palamidone conduce seco una quantità di birboni, stracciati, ¢
era Jui. Questo fu un guidone mezzo matto, ma tutto tristo, ed al
gno birbone, il quale faceva servizio a' carcerati,¢ perché continovamente br
tolava, dicendo di pazze (cioccheric, haveva sempre dietto una gran quantità di
ragazzi che lo facevano stizzire. La notte per guadagnar qualcofa portaya'
tro al Capitano, o Caporale de'Birri un' arme in asta solita portarsi —
glia del bargello, quando la notte va facendo la guardia, la quale arme noi
detta /ancione, Ma che egli rubafle non posso crederlo, perché afflolutamente nom
havea tanto giudizio, e stimo che il Poeta dica questo nel presente luogo, e a
trove per descriverlo per uno di quei furfaati, de' quali si può credere ogai ribal-
deria. Palamidone e accrescitivo di Pa/amudes, Eroe noto nella guerra Hy
secondo la pronunzia Greca pil moderna dicefi Palamide,e non Palamedes onde
è fatto il soprannome di Palamidove; che significa un lungo e sottile, come wie
palo, una persona grande di stacura. =o

eANDARE 4 Carpi, ed a Borfelli, Carpi e un Principato in Italia notitfimo j ¢
Borfeili è un luogo sul Fiorentino, e scherzando con questi due nomi Carpi itt
tendiamo carpire, cioè rubare., ed a 4or/edé, cioè alle borfe per rubare. Ati
stofane Poeta Greco nella Commedia inticolata i Cavalieri, citato dal
nel Flos Ztalica lingwe, ( ove egli tocca la maniera di parlare Fiorentina; eng

ebbe per San Giovanni, usata anche dal nostro Poeta; ) dice così: manus in Attt-
lis hades « Vuol dire: sempre chiede, ed e apparecchiato a pigliare; scherzando fal
nome di certi pork chiamati 4ro/i, per  allusione che ha questa voce alla pa
rola atein che significa chiedere  ate

PORT ARE il Lancione al Bargello. Questo meftiero (olito farli da birro novi2idr
lo faceva alle yolte Palamidone., comes' è deo. 5 he ee

:; BiANTI
















a

reo.
RARER LARP LEE

weERAS

RNLSESE

mA

Be



TERZO CANTARE: 179

BIANTT, Si trova una specie di Bricconi,e Vagabondi che v anno buscando
danari con invenzioni, come si vede da un libretto intitolato Sferza de' Bianti, ec,
E si dicono anche Monelli; (¢ ben veramente per monelli intendiamo quei pove-
ri, che si stroppiati, malati, impiagati, o morti dal freddo per muover
Ie persone a far loro elemosine, donde rab po far it monello quel ragazzo,
che havendo toccate leggiermente delle dal Maeftro, o da altri, metre as
fogquadro il vicinato con le strida per mostrare d' essere stato dalle bufle strop-

iato., ed in vero non ha mal nefiuno, che si dice anche far marina: vedi sopra
. 1. stan. 37. alla voce fofiano, e sotto C. 4. stan. 8. Di questi intende il Persia-
ni nei seguenti versi.
Signor non fo se voi sapere il bando

Di chinder tutti dentro a! Mendicanti

Mascalzon, vagabondi, e maleftanci,

Che vanno per le frrade mendicando,
fo che sono in arnese tanta male

Mi ritrovo in grandissimo viluppo;

Temo esser preso in vece d un Gaiuppo,

E finir la mia vita allo Spedale.

VEST E la coracxa da bastone. E' armato a bastonate, veste un' armatura da,
difenderlo dalle bastonate; $' intende che e fortoposto a toccare spello delle ha-
stonate.

. RILEV-ARE. Intendiamo buscare, conseguire, ottenere. Petr. Canz. 22.
M sempre sospirar nulla rilieva,

Onde (¢ bene figura ai rilievo yuo! dire Ratua di marmo, o di altro materiale,
noi incendiamo rilevare, cioè bafeare e qui intende buscar mazzate. I) verbo ri-
devare piglia questo significato da rilievo, che sono gli avanzi delle menfe de' gran-
di, quali avanzi si buscano per lo più da coloro che servono a tavola, donde di-
ciamo Viver di rilievi che vuol dir Campare d' avanzi. Vedi (otto C., 5. stan. 47.
Franco Sacch. Nov. 154. Quando la croitata fu mangiata tutta, senza far rilievo ne
meno de' topi, Rilevare yuo) dir Quello esprimere che fanno delle parole i ragaz-

zi,g imparano a compitare.

STANZA LXVIII.

Comparisce fra tanto umcarro in piaga Con che la formidabil Martinazza
Da Farfarel tirato, earbariccia A lor, ch' e ch' è, le costole fRropiccia,
Vobidiente al cenno della\eayz4 E quei Demon} in forma di Camozza
Soda, nocchinta,ruvida,e mafficcia. Vin tirando a battuta la carrozza.

In tanto, che si fa la mostra de' soldati di Malmantile comparisce in piazzas
un carro tirato da due Demonj in forma di capra faluatica, che questo vuol dir
camozza, la quale per lo pil si trova ne i monti del Tirolo. Plin, lib. 12.cap.37
la chiama Rapicapra. | nostri antichi dissero Stambecco, il Lat. ibex..

 PARFARELLO, e Barbariceia, Nomi di due Demonj dal nostro Poeta cava-
- da Dante, del significato de' quali nomi vedi gli Spositori sopra il medesimo

ante.

, NOCCAIVT A, Piena di nocchi, che sono quei piccioli rilevati come bolle,
iquali si veggono per lo più ne i. di pruno, di forbo, ec, che gli rendono
2

rnvidi

x.
ita MALMANTILE (5

ruvidi, eli chiamamo ancora wedi, come fanao i Latini.,..
MASSICCE, Intendiamo tutte quelle cose,, che dal.
te di materia stabile, e folida, e non vote, o vane,0 in
deboli, siqech e rralnepeiegig
CHé ch'é, Ad ora ad ora,.di quando in, quando 4 | argh
ST ROPICCIARE, Fregar qualcola con)p altro,ed i Lat
Forse € corrotto da froppicciare, che pare si dove ie an ee
cio, con che per lo pil si fropicciano gli arnesi per liberargli )
Sropicciar le costole a uno vuol dire Bastonare uno... 07 RA
TIRANO (a carrozza a battuta, Nona battuta di musica »maab
mazza, con la quale Martinazza la bastona, » 'eerie
STANZA LXIX, STANZA LX
Costei e queila firega maliarda, Ove la notte al nace eran concorse
Che manda i cavallacci a Tentennino,. < Tutte le Streghe anch'effe sul ca
Ed egliun punto a comparir non tarda Z Liavosi col Bau y le Biliorfe
Quand ella fa lo feaccio,o ul pentolino, A ballare ye.cantare,e far te
Come quand'ella si unge,e s'inzavarda Ma quando presso al di
Tate ignuda nel canto del cammino, Fa di moffi
Ler andar col Barbuto sotto il mento Come a costei, chor vienfene:
Con la granata accefa a Benevento, E in [yu quel carro nelCaftell
STANZA LXKL
E la cagion si ¢, ch' ella ne vada Perché vi,
Adeffoacafa tutta in caccia,e in furia,  C alla [un patria
L' haver veduto dentro alla guaftada Percio, se nulla fuffe di
Vn segno, che le ha data cattiv' uria; We viene anch'effa a dar il
Martinazza è una di quelle fireghe, le quali. costringono il Dia
lo staccio, e il pentolino, e con ungersi per farsi portare a B
greflo de' Diavoli (otto il noce: Questa Martinazza adesso, si fa
famente daquei Demonj a Malmantile, perché ha veduto-nella cara
da sanguigna, che le presagisce la caduta di Malmantile, onde vi fis
ancor' efla per dare il suo aiuto. (Questo nome di Martinazza e nome 3
quella strega, e fregherie son tutte dal Poeta dette pep accennare l'opinioneé
alcune donnicciuole, le quali portate dall' illufioni diaboliche y si danno a ¢rede
re d' havere effettivo commerzio col Diavolo. > tit oo
STREGA. Vedi sopra C, 2. stan, 11. Viene da frix uccello
detto a fridendo, secondo Ovid, fatt..6. aati
Eft illis frigibus nomen, fed naminis bnius, —
Canfa,quod horrenda stridere noite folents
E questo uccello ( che forse era l'Arpia., ma Plinig,dice 5 che non
fofle ) credevano gli antichi più superftiziofi., che rapifié i bambini-
Et ab huius avis nocumento feriges Latini appellabant mulieres puellos'
raflu, E diqui ancor noi le chiamiamo streghe, che le:
da far malic, fattucchierie, ed incantefimi, e però chiamate.ancora J
MANDARE un cavallucio, Magdaceuna citaai sioé chiamal
gindizio criminale con polizza. E queste polizze de Giudizz) Criminal










































R SRSCER Esa ase EE

—
=

wERERRE

TERZO CANTARE. 181

renze si dicono ¢avallucci a differenza di quelle de' gindizzj Civili, che si chiana-
no Citazioni; e questo nelle polizze criminali e stampata l'impresa, o
contraffegno del Magiftrato criminale, che e un' Huomo a cavallo armato; qual
con e chiamato comunemente Cavalluccio.

-TENTENNLNO.Nome dato dalle nostre donne 2] Demonio per non !o chia-
mare Diavolo; quali rexrarore; col qual nome e nominato preflo San Matteo
Cap, Vers. 3.00 t i:

HA lo paccio ye il pentolino, Favoleggiano, che quelle donne Maliarde, e Stre-
ghe, che habbiamo detto, aes fare diversi incantefimi per ritrovare cofes
perdute, © per ottenere altri loro intenti, e fra questi incantefimi fare lo Paccio,
0 it Pentolino, o la caraffa; Si che dicendo Fa /o staccia, e il pentolino intende fa in-
cantefimi. Quei che indoyinano per via di flaccio sono detti dat Greci Co/cino-
mantels.

COME quand' ella s unge, es inzavarda. Inzavardare,¢ uno impiaftrare con
materia morbida, e viscofa, atta a distendere come il lardo, I) Poeta seguita,
la vana., e superftiziofa opinione, che queste tali donne vadano ogni tanti
giorni al congreffo de' Diavoli sotto il Noce di Benevento: Ove da notre a/ noces
eran concorse; al qual luogo dicono esser portate dal Diavolo in forma di caprone,
che questo intende if Barbuto sotto al mento, e cavate dalle loro case per la gola.
del cainmino (¢ però dice nel canto del cammino ) dal medesimo diavolo forzato a
far tal funzione da quegli uatumi, che dice essersi messa addosso la medesima
donna; a quale poi a detto congreflo fa rempone, cioè si da buon tempo; si piglia
tutti quei piaceri, che le vengono in fantasia quella notte; Ma sul far del giorno
Ie conuien partire, e il Diavolo in:un baleno la riporta al suo paefe. Tale opi-
nione hanno simili scimunite; ed o sia per effetto di matrice, o pure per opras
del Diavolo, che per illufione faccia-loro apparir per vere tutte quelle (ciocche-
tie, che esse si fingono nella testa, l'effetto ¢, che esse si credono d' esser' anda-
te veramente a Benevento, ed essere state riportate dal Demonio al loro paele,
quando effettivamente non si sono moffe del letto..

. GRANAT A, Bran mazzetto di scope, od' altra cosa simile, che s' adopras

spazzare,¢ ripulire le stanze. E con queste granace accefe in mano dicono,
che tali streghe vadano cavalcando sopra un Caproneal detto Noce di Benevento.

BAV, e Biliorfe, Questi nomi bau, biliorfe, orco., befana, versiera, e altri
simili, (ono tutti inventati dalle Balie per spaventare i bambini, e rendergli ub-
bidienti,persuadendo loro, che.questi sieno spiriti infernali, e però il Poeta nu-
mera fra i Diavoli il Bau, ¢le Biliorfe,, per accomodarsi alla capacita de' Fan-
ciulli, per li quali profefla-d' haver composta la presente opera. Vedi sopra Cs
2, stan.'50. 1 Greci il cemibalo per chetare 1 bambini dicono Carabax

FAR tempone., Darfibeltempo; Stare allegramente, pigliandosi tutti quei:
poliskhoee può, e si pigliarsi., che diciamo anche /euazxare; trivnfare; far
juonacera; Genioindulgere, litare Genio, dissero i Latini. La Compagnia della
Letina insegnando., in-qual luogo si deva pigliare laccafa per risparmiare, dice:
Vorriano le nostre.case esser in una quasi dall altre feparata comrada, lontana'da vie, ©

piagze pubbliche y dove all? occasiani si fefteogi,¢ si facciatrebbi, e tempone.

BATTER il taccone.. Elo stesso, che barter la caleofa, dewto sopra  questar

2 lan


YY Bate

* trovafi nei Latini /olwm vertere. &






















182

C, stan. 60,,cioè ¢amminar via; andarfene. Si dice an
dice il suolo della scarpa, cioè quella parte, che posa in
- th
VENIR di punta. Venir con velocita, a dirittura; che diciamo:
vela. Vedi sotto C. 6, stan. 10, Credo sia originato dalle barche,
venir di punta quando vengono a dirittura senza volteggiare.
IN caccia, ein furia, Cioè in fretta, frettolofamente, e con furia,
no coloro, che son cacciati; che però diciamo; Corre, che par ch'egit| ha
dietro, Incedit quasi in fugam versus. A
GV AST ADA. Specie di vaso di vetro per uso di conseruarvi
stesso, che caraffa dai Latini detta Phiala, L' Autore disse sopra nell'
antecedente, che Martinazza era (olica fare lo Staccio, e il Penroline, €
la Guaffada; queste maliarde, e fireghe empiono di superftiziofi li
raffa, o guaftada, e facendovi mirar dentro da un fanciullo innoc
dire di vedervi dentro quel che hanno desiderio di sapere, e tutto
le persone semplici, e cavar loro denari di mano. Questo indovinare
acqua, fu anticamente presso i Persiani, e da'Greci si chiama Aydro)
questo habbiamo un detto Gei ha +i diavoio nell' ampolla per intendere
dovina ogni cosa. Z
CATT IV' uria, Cattivo augurio. Questa voce Vria corrotta da
ta per lo più dalle donnicciuole, deta senza aggiunta di cattiva, o bt
tende cosa, che non piaccia. La tal cosa mi dé uria: © s' intende mi
mi da impedimento, mi da noia; da che si può credere che sia
xegia, che pure vuol dir noia, fastidio, impedimento, ec. o forse ia
che suona lo stesso, che xgeia, o forse in vece 4' ombra, che è il
do vale per impedimento, /a tal cosa mi dd ombra, per la tal cosa
Siche Vria, xegia, ubbia, ed ombra faonano tutte lo stesso; Vria, e
usate per lo più dalle donne, e ' altre son pi: comuni. Si potrebbe
secondo il Monosino, che la voce ria veniffe dal gteco Vria, che
prospero, e che si come habbiamo per costume di dire buona, o ¢è
quantungue /orre significhi assolutamente bene,c felicita;così habbiamo.
di dire buona, o cattiva Vria,quantunque Vria significhi (empre feli
Greco Vria, Nello stesso modo, benché prefso i Francesi bear signi
licita; voce a loro derivata similmente da) Latino augyrinm; dicono
malheur, quali buona, e cattiva wria, cioè buona, e mala ventura; €
doci servir bene di questa parola Via, come vocabolo di mezzo, dovr¢!
giungerci buona, o cattiva,¢ non dirla afsolutamente, e senza detta
come habbiamo accennato, che molti se ne servono; ma l'uso ci Lil
aftrufe stiracchiature. '
SE nulla sue, Per tutto quel che potefse succedere, Se accadefse q
grazia. 1 Latini in un simil modo per isfuggire il cattivo augurio, ¢
nare cosa infanita, come e la morte, dicevano: Si quid patiar. Si gi
manitus acciderit, Se Dio facelse altro di me, con tutto ciò, ¢c.
NE viene anch' essa 4 dare il suo disegno. Con queste parole mostra
quanta gelofia haveva Martinazza di non perdere J' autorita, che «














ale

=
=

Se



wo
off
3
o
è
34
po!
4
eo

Y





TERZO CANTARE; 183
Malmantile, ed il sospetto di non efser levata dal grado di Salamiftra, che go-

deva, come accennammo sopra in questo C, stan, it,

STANZA LXXIL STANZA LXXIIL

Fuggh tutta la gente [paventata Figuriamci vedere un [acco pieno
ell' apparir dell' orrido spertacolo, Di zucche,o di popon sopr' aun giumeto,
La praca fu in unt attimo spayzara, Che rottasi la corda, in un baleno
Pur un non vi rimafe per miracolo, Ruzziolan tutti fuor sul pavimento
Così corrende ognuno all imparzata E nell' urtarsi batton sul terreno:

Si se Cun l'altro alla carriera offacolo; Chi si perquota,e chi s'infranga drento

Chi da un'urton,quell'altro da un tracollo, Chiff sbucci in un fafso,e chi sintrida,

Chi batte il capo ye re colle, Ed un altro in due parti si dsvida.
STANZA LXXIV.

Così fa quella razza di coniglio, A tal che in veder quello scompiglio,
Che nel fuggir la vista di quel cocchio obo ben preso( dice) qui lo ferocchia,
Chi se rompe ta bocca yo fende un ciglio Mentre a costor così comparir voli:
E chi si torce un piede,echiunginocchio; Sapeva pur chi erano i mici polli,

1) Poeta descrive assai vagamente il timore, e lo spavento, che eatro addosso a

ei di Malmantile per la vista del Carro di Martinazza, la quale vedendo colo-
ro così spaventati, si pente d' eer quivi arrivata in quella guila.

IN xn atrimo.\n un momento.Corrotto da atomo.Si dice anche nn baleno,come
nell' ottava 73. seguente. Jn un batter d! occhio. V. sotto C, 10. stan, 42. dal Lat.
Etu oculi:En atomo difiero i Greci. Dante Inf, C. 22. Subito,e (peffo 4 guisa di baleno,

NON ve ne rimafe sy miracolo, Fuggiron tutti, che non ve ne restò pur'
uno. Tanto esprimeva se havefie detto: Von ve ne refi pur' uno, Ma col dires
miracolo da maggior' emfafi, e seguita l'afo; e vuol dire farebbe ato creduto mi-
racolo se un folo vi fufle restato.

CALL impazzata., Acalo; Come fanno i pazzi, cio senza considerar gquel-
lo che facevano, o dove essi andavano. #' il latino perperam,

VRTONE. Percofia che si da con tutta la vita in un' altra persona, o in uns
muro, oaltrove, cd lo fleflo, che Spinta, ne vi (0 fare altra differenza (es
non che Vrtare vuol dir percuotere.a caso, ed € il Latino ofendere; © Spingere
vuol dir Mandar uno innanzi, o-indietrocon violenza, ed e il latino émpellere.;
Ma nondimeno-#rtone, ¢/pmta si pighiano |)' ano per l'altro 5 se bene non si di-
rebbe Dare una spinta in un muro,'0 altra cosa immobile, che fatta mobile co-
mie farebbe un muro sciolto per farlo rovinare, si direbbe Dare una spinta. A
= quasi recifo da piede per atterrarlo 5 si direbbe Dar la spinta per farlo
cadere, ec,

TRACOLLO. Accennamento di cadere. Extra collum pedis ire; o pure detto
così quasi Tracrello. Vocabolario della Crusca. Tracollato addiettivo da rracol-
dare 4 che vale lasciar' andar già il capo per sonno, o simile accidente.

GIVMENT®O. Si dice propriamente l'asino  benché s'intenda anche ogni be-
fiiaccia da foma.. Così presso i Latini: Quello che in $, Gio, cap. 12, chiama-
to pullus afine yin S, Matteo cap, 21, & detto pulls filins fubiugalis, Puledro, figline-
do della giumenta,:

RVZZOLARE. Gisare per terra; che diciamo anche Rotolare.

I3-


184 “MALMANOTILE?

INFRANGERSI, Sflagellarsi, ammaccarsi » disfarsi,
76. C.11, stan. 12. i I ti
RAZZ A di Coniglio, Gente timida, e codarda. Si dice poltrone come
giio, perché questo animale, che è specie di lepre; come quella,
PIGLIAR (0 ferecchio., lngannarsi, Far' errore. Lo sono st
credendo di (tar bene, ma ho pre(o lo (crocchio; cioè mi sono it
sono stato male. Il proprio significato della parola, ferecchio &
trovar danari,piglia a credenza una mercanzia per ventici









questo,quando noi facciamo una cosa, che non ci torna poi bene, ne
utile, e gusto, ma più tosto ci e di danno, si dice pigliar fo ferocchio,
S-APEVO chi erano i miei polli
Cognosco oves meas.



STANZA LXXV. STANZA LXXV
Scefe dal carro poi per impedire Percio si ferma strambasciata ye.

Cosigran fuga,¢ rovinofa fola;
Ma quei vit pis si frudiano a fuggire y Dalla Carretta subito di
E mostraognun se rotte hain pie le (uola, Eqgli si lancia addossa ac
Chi finalmente, come si sual dire Così correndo tutra si
Chi corre corre, ma chi fugee vola, Perché quel Diaval vanne
Ond' ella y ben che adopri ogni putere, Pur ( dicendo: arrila 5
Vede che fara tordo 4 rimanere. Lo fruga si,ch' al fin la
Martinazza scefe dal carro per fermar quella gente, che fuggiva,@!
correr lor dietro, ma allora si, che coloro fuggivano, onde ella
a uno di quei caproni al fine gli arrivo. EB qui termina il terzo Cantat
FOLA, Quantità di popolo, che furiofamente corre a qualche luo
to da i Cavalieri, che gioftrano, che dopo, che si (ono soddisfacti li
a uno per volta a giofirare, in ultimo corrono al Saracino ( così
mezza figura, o bufto 3 di Moro, o Saracino, fatta di legno, efi
corrono dico al Saracino tutti in truppa, uno però dopo l'altro,
far la fola, In Latino potrebbe dirsi: exerceri ad palum. Vegezio
lib. 1. cap. 14. Tiyro, qué cum clava exercetur ad palum, bastilia quogue
gravioris y quam vera futura sunt iacula, adversus illum palum tamquam:
minem sattare compellitur, E si dice fola, o folata d' uccelli, di popolo
tender di cose che velocemcate si muovono.in quantità, e presto finile
ta di vento, Studiare a folate. Lavorar a folate,ec, Forse meglio fola y
fica quel che i Latini dicono Adagna hominum vis, vel turba, aut fummafi
bomsnum, Si come noi dal calcare le strade, che fa il popolo.e daliu ee
¢ stretti, diciamo Vna molticudine numerosa di gente, una gran calc
Franzeei nella lor lingua la dicono fowle, cioè fol/a dal verbo fouler,
calcare, Da folla abbiamo fatto Afollarsi, e Folto, denlo, calcato
tarsi, far furia,far preffa: lo stelso quali che -Afollarsi tutto deriva
tura dal Latino follss, nel quale sta l'aria ferrata in modo, che pi
capire. i ' vi







Sie
Vitae










non ne vale venti, e poi la vende quindici, e questo si dice pigliar lo fero
Plauto disse: Emere caca, vendere oculata die. Vedi sotto C. 6, stan. 60. |






















- Sapevo di che qualita eran costoro, @ il

Kitorna indietro, ed un de' si i



a la

TE Ee a SE


SARRERLL ELLA TERRES DEN

~



TERZO GANTARE. 185



 STVDI. SZ, I) verbo fudiarsi er affaticarfia far presto, o spedire
tuna cosa, che diciamo anche menar le ioe. Per esempio: pie 5 nee il
tempo è breve, ¢.non finirete, se non fate presto. Qui intende s' affaticavano a
fuggire infeare: al che s' adatterebbe i verbo szeumbo, labore, ed anche+

Studeo, e questo dal Greco spexda, afrertarsi...Nel,Salmo: Domine ad adivandium,
me feftina. conn Tddio, Tease ad! aiutarmi. eae Sic feftinanti femper lucn-
pletion obfear y a colni che si fiudia @ arricchire il pi riceo da impaccio.

MOST RAR le suola delle fearpe. Corscr velocemente; perché così s' alzano
aGai i piedi, e si mostrano le (uola delle (carpe. I Greci pure dicevano in questo
propolito Canum pedis offendere, Si dice tan Battere iltaccone, che vedemmo
sopra in questo C, stan. 79.

CAL corre corre y machi fuece vole. Detto sentenziolo, che significa, che mol-
to più forte corre quello, che e perfeguitato, che non corre colui, che lu perfe-
guita, perché la. paura gli mete l'alia' piedi,¢ per questo dice Chi fugge vole..
Vergilio dilses Pedibus simor addidit alas,e Dante Inf.C. 22.

E poco walle yches' ali al sospetto, Non potcro avanzar.

Intendendo,.che il gran timore, che hebbe del Demonio quel dannato,|o fece

efser più veloce, chet! ali di quel Demonio, che gli correva dictro. Della pa-

,, rola agit Ipiegantidima della yelocita appresso Vergilio,vedi Seneca Epift, 108.

PARE tordo a rimanere. Cioè rimarra a dietro, e non arrivera quella cana-
glia'. sfl.giuoco de' tordi ha qualche similitudine con ' Amilla de' Greci, guia de
certo iafhu inter dudentes certemen est, come dice il Buleng. de Ludis Veterum cap.
a4 clagara si dice in Greco.amide. Nell' Amilla si tirava una palla dentro as
wa segno, o-circolo,.¢ colui perdeva, la di cui palla usciva, o non entrava nel
ciccolo., Nel cordo non si fa ne segno, ne circolo, ma si tira una piccola palla,
( da noi a distinzione dell' altre palle detta grille, come vedremo sotto C. 6. stan.
pe colui., chelatiradice: 4 pessare, cioè a pafsare con la palla il detto
gsillo, © a rimanere, cioè rear con la detta palla di qua dal detto grillo; così
sirando ciafeuno,s' ingegna di pafsare, o rimanere il pill vicino a detto grillo, che
egli può; perch chi meno lo palsa, o meno addietro gli rimane vince la possa,
¢d.a quelli, che-non pafsano, o non rimangono, quando devon rimanere, o paf-
fare, vince il,doppio,¢ questi perdenti si chiamano Tordi, € sono di tre forte,
perché tre sono i cafi del tiro; cioè Tordo a pafsare è quello, che pafsa di la dal
grillo quando deve rimanere. Tordo.a rimanere quello che rimane di qua dal
grillo,quando deve paGare. E Tordo semplicemenie si dice quelio,la di cui palia
resta in dirittura del.grillo per banda,e questo da alcuni si fa che non vinca,ne per-
da, daaleuni, che perda folo la meta degli altri tordi, se e più lontano dal gril-
lo di oo che vince 5 efe è più vicino non perde; da alcuni gli € permesso riti-
rare fino a tre volte, quando però sempre refti in dette tre volte nella medesima
dirittura del grillo; e quando non paffi, o non rimanga perde una fola posta: ¢
fsempres' intenda pafsata, o rimasta la palla quando fra ¢fsa, e il grillo pola,
interporsi un filo in fquadro,se però non 1o tocchi non per banda, ma per quella
parte,dove hada rimanere, o restare; e tutto si fa secondo le conuenzjoni, e+
atti. Questo giuoco per lo pil e usato da' ragazzi, o dagl' infimi botiegai di

Firenze; i quali nei giorni felte, uscendo dalla Città per andar' a pigliar'
3 Aa aria
%
ar, tie 5:










- Aty I
(a 1 Al
' in z 'hie sa
186 MALMANTIER()
aria nel camminare giuocano a questo giuoco, \
'no a chi perde, e quando n' hanno fegnatitanti, ¢
bere, e da mangiare, si fermano alla prima Ofteria'; €
quantità di danaro, che ha perduto. Hor tornando a pro
Unazza farò tordo arimanere, ed intende, che rimarra (ro >
quella ciurina. ' pt ge aag
STRAMBASCIAT A, Affannata; Opprea dal' ambascia 5:
dificulta di re(pirare cagionata dalla violence fatica nel correre,
prabbondanza d' alito. Dante Inf. C.24. & però leva si; vinci I
qui per avventura e4mba/ciadore, che piglia a fare amba/cia, cio'
dare a quel Personaggio, o Città, a cui eglié inuiatoys ©
S/lancia. Si getta; cioè con un falto monto preftamente a.
rone. o eal
: S/rinfacca, AGomiglia Martinazza (che cavalcata in fal suo Capi
a quando s' empi¢ un facco di roba leggieri,la quale si mandi gil co
stiuarla, ed empier bene il facco, questo s' alza, es' abba(sa
faceva Martinazza a cavallo ia sul Caprone, il quale faceva a lei
andando baielloni, cio' a falti,come e il proprio correr delle capre
ce balzeloni viene da balzellare, che lo diciamo il faltellar delle le
di Maggio, e Giugno, che elle (ono in amore, e la caccia che in talt
si dice andare al ha/zedo. Del cavalcare la bestia nera, e cornuta V.
ARR 1d, Cammina li, Va la. Termine stimolatorio usato ju
ec, dai vetturali. B' ben vero, che vedendosi uno a Cavallo, i;
ciamente, si suol dire per derider colui 4rri /@ quasi diciamo yaa cavalea un' af.
no, e portato da questo uso 1' Autore fa dire a Marcinazza Arré jd. |
lo fa venire dal Greco Errbe, cio, va via, a
CARNE cattiva, Animale vituperoso. Diciamo carne cattiva', o cat
'di carne ancora a quegli huomini, che sono di genio sciagurato,€ + Oe
de si dice quasi in proverbio, e per ironia di chi sia magro, opi di perl
ma sia maligno,e aftuto,e come si dice ne' suoi panni @ vi sia tutto 5” 0
Stornello, poca'carne,e cattiva, Equi si può anche dire, che I Autore la'

carne cattiva, perché era capra, che fra le carni, che si mangiano', ae
. VMs oyet
























ya.
CIVRALA.Dal Lat. turmaSi dice propriamente degli Schiavi
Jera: Ma Gi Piglia ancora per quantità di gentaglia,e qui intende di
glia, che fuggiva. Vedi (otto C. 5. stan. 16.5 € C. 11, stan. 16;

ae

FINE DEL TERZO CANTARE,
' ve

ee
atk










cee
OCANTARE,

ARGOMENTO.

Tguerrier di Baidon son mal disposti
Perché la fame in campo gli travaglia;
Lifendefi,¢ Perlon lasciamo i posti,
Won vedendo arrivar la vetrovagiia.
Pfiche non tiene i fusi pensiers ascofti
et Calagrillo Cavalier di vaglia,
Che promette aiutar la damigella,
E poscia ascolta una gentil novella.



At



STANZA I.
Maia vincit amor: dice un Teftoy
Exun'altro diffese dette piit nel segno:
Fames Amorem superat. E questo
E' certoye approwacgniic'ha un po d'igegno
Perché quantunque mor sia si molefto,
Che tutti i Martorelli del sue Regno
Dicano ogn' ora; Ahi laffo,io moro,to pero,
Enon si trova mai, che cio sia vero,
STANZA IL
Non ha che far niente con la fame y
Che fa da vere, pur ch' ella ci arrivi;
Posson gli amanti star senza le dame
1 mefi, e gli anni se mantenersi vivi;
Ma se due di del consueto frame
1 poveracci mai rimangon privi.,
Ei basta, che de fatto andar gli vedi
© porre il capo dove il Nonnoba i piedi.

SAAS SSS

STANZA IIL
Tal che si vien da questi effetti in chiaro 5
Che d' Amore, la fame e piit potente,
Ond'écognun di lui più questa ha cara,
E quand' alle sue hore ei non la sente
Lamentafiye gli pare oftico, e amaro;
Percto riceve torto dalla gente,
Mentre ciascun la cerca, e la defia y
Es ella viene, vue} mandarla via.
STANZA lV.
eAnzi la scaccia, come un' animale
Sul buon del desinare, e della cena,
Per questo ella talor, che i'ha per male 5
Psu non glitorna;ovver per macgior pena
In corpo gli entra in modo,e nei canale
Che non Lempiercbbe Arno con la piena,
Come vedremo,c' a Perlone ha fatto,
C” a questo conto grida come xn matto,

Il nostro Roeta riflettendo., che nel presente Cantare gli conuien de(crivere las
fame, che era.neb campo di Baldone, per non esservi ancora comparfa la muni+
zione di bocca, s' introduce col provare, che la fame è superiore ad Amore, quan.
tunque la maggior parte degli dupatinl, leapitando Vergilio Eg), 10. dove canto:

è; a2

Omnia
———————————7~E










188 MALMANTILE
Omnia vincit 'amor; @ 'nos cedithnys ante
éica che Amore sia più 56 fup
ver provata questa sua in si maravi;
più potente, e più stimabiley e desiderabile ych
fere scacciata neJla maniera, che ognun procura di
habbia ragione di vendicarsi di tal disprezao, 6 con I" and
de\ mangiate,' col venir troppo 5 Quand6 nof si ha chen
mostrare ch' è seguito a Perlone. oat
MATTORICLS agen nea es ma
4H! laffo. Inverpolizione, che deniota dolore. dica son
da} dolore, dal travaglio, ec. B il Lat. bem, bei mihi, Francele Helas,
NON ha che far niente. Non e ¢ luogo da far comparazione. Non
isperto alla fame.;
TRAME Si dice il sieno, paglia, © altro simile che si dap
fic: Maqui lo piglia per cibo degli huomini, come e scherzoso
ciamo /rameggiare,quando uno va trattenendosi col mangiare alqui
do che venga in tavola la vivanda per desinare, o per la cena, che
concellare, Vedi sotto C, 7. stan. 10,
'POVER ACC/IO, Epiteto che esprime la compaffione, che s*ha
di colui, il quale finomina. Vale per infelice, disgraziato, ec.
PORRE il capo dove il Nonno hai piedi. Farsi fotcertare. Motire.
tura si dite; Appowiad parres fuor. kr eae
RICEVE torto, Non (e le fa il giufto: Non se le fa il dovere, 7%
rio di diritto. E significano questo Giufto; e torto Ingiafto,“cotne
ra C. 3. stan. 66. None sn corpo florto anime dritto. enh
ANIMALE. E' nome generico, che significa ogni' specie di
costume pigliarlo in specie, e per azmale intender solamente le!
gue poi che dicendosi animale a un huomo's”intende un hnuomo
giudizio,in somma un huomo bestia. Bocc.n.79, dice: Conofeendo
efer un' animale, Vedi sotto in questo C. stan. 5 1.°Cic, Wonne vides, J
ZL canale, cine il canal del-cibo, che & la, Zola' + il comidetto adesbarconiy OY
così vien descritto in lifgua*furbesca dalla plebe Fiorentina. ea
NON ! empierebbe Arno con la piena, Non V-empicrebbe.Atnojqiia
pioggie vien grosso. Iperbole usata per intender"nno,vchenon si
gordo tanto del cibo, 'quanto dei denari, che ilatini disseroD
dun huomo, quem eos non nutriet, illum nec-Bgypras.“Empiti
per dispetto a uno, che non si trova mai fazio; modo'baffo.- oe
STANZA V. STAAwNoZ A OVd,
Defta ? Anrora omai dal letto feappa,
E cava fuor'le peyue di bucato,
Poi barre il fuoce,e quocerfaila pappa
Per il giorno bambin c? allora e navo;
E Feboch'é il Compar già con la cappa,
Econ wr bel vestito di broccato,
C a nolo egli ha pigliato dal' Ebyeo,
Tuste splendente vienfene al Corteo,



r


















Z

















fest!
5
“le
¢
ys
yi
i
wo
5
a



wee

QVARTO CANTARE. 189.

»Inoftro Poeta ( come habbiamo detto altrove ) hebbe notizia da Saluadore
Refa d'un libro Napoletano intitolato LO CVNTO DE Li CVNTI, ed in.
comporre l'aggiunta alla presente opera se ne val (e,cavandone qualche peafiero,
© concetto', come vedremo; e questo è quello della presente de(crizione della lc-
vata del Sole. Dice dunque che /uegliata ' turora, esce del letto,e cava fuora le
perze bianche di tucato; il che allude alla chiarezza che apHOrA l'Alba. Di poi
accende il fuoco!, e fa quocer la pappa per darla al Giorno bambino che allora e nato.
E per questo fuoco intende quell' albore che si vede all' apparir dell Aurora, il
va crescendo, e piglia un colore gialliccio per lo vicino apparir del Sole;
e però dice che Febo viene con  abito di broccato d' oro tutto [plendente al Corteo del
warno bambino. E così intende che alla levata del Sole i Soldati di Baldone non.
ino ancora hayuta la provvifione per vivere, onde sono in collora, epartico-
larmentemolti diloro, che sono afluefatti a far sempre col ventre pieno.
PELZE di bucatePezee bianche pulite perché sono di bacato,cioè non adoprate
dopo che furono imbucatate; ed intende quei panni lini, che servono per falcia-
se, ed inuoltare i bambini.
BATTE il fuoco, Accende il fuoco, Così diciamo, quando per accendere. il
fuoco si batte nella pietra focaia, se ben non si batte il fuoco, ma la pictra, Ver-
gilio nel 6, dell' En, dice.



quarit pars femina flamme

Abftrufa in venis filicis ——————

PAPPA, Pane boilito in acqua; è la vivanda solita darfia i bambini quan-
do s' allattano, e cominciano taliittaro » e si dice pappa perché essendo la let-
tera, P..puramente labiale, e facile a profferirsi come sono le lettere B, M. ¢
pero ne ibambini-si-trova maggiore attitudine a profscrir queste, che l'altre
confonanti, si che pitr facilmente profferiscono habbo, mamma, pappa, bombo s
che padre, madre, mineftra, bere, onde le balie si (ervano di queste parole per
facilitare. la loquela.a i baibini, Tal costume era forse anche negli antichi te
mani, come si cava da Varrone, (nel libro ane ee » Ovvero dell' alle-
vare'® figliuoli ) che per Papas intende quello, che intendiamo noi Toscani
Pappa yoda Pein »che oe Satira 3. dite cs sia

Et similis Regum pueris pres minutum.,

I Grecivpute per «i loro bambini 4i seraivano come noi, e come i Latini, di
voci di due sillabe.con raddoppiarae la prima sillaba,, per maggiore agevolezza
del rilevare layparola... Di.queste parole bambinesche ne troveremo molte nella
»presente Opera., usate dal Poeta per scherzo., o per accomodarsi alla qualita di
colui che fara parlare,¢ non perché sieno in afo altrimenti. Vedi focto in questo
Cant, stan,12.dove dice d' un bambino.che impara.a parlare.

BROCC.ATO. Buna specie di drappo fatto.a fiori, es' intends Deappo tel-
futo'con.oro..

- A NOLO eli ha pigliato dal' Ebreo, Dice che il Sole ha pigliato a noloil suo
splendente.abito., per significare che lo.rende la fera,, come lo reftitui cone caio-
ro', che:pigliano gli. abiti.a nolo per.un giorno; ed intendere che il Sole alcon-
dendosi la fera alla nostravista, la(cia quell' abito risplendente, che.s' era mello
a mattina,, Y
3 COR.














196 MALMANTILE |

ao
CORT EO, Corteggio | Codazzo di donne,ec, che gn
quando va a marito, o un bambino portato a
VONANESI genti, | soldati del Duca d' V;
pellar  esercito dal nome del Generale, come Vaimarefi
COMP ARIRE in (cena, Venire in pubblico. Vedi sopra C.
LA materia che da il portante a' denti, La materia, che fam
cioè la roba da mangiare; si dice anche Da far ballare il mento.
uefto C, stan. 23. 2 portance si dicg una specie d' andare di cavalli. Il Lali
fr. C, 3. fan. 58. dice. 1 aE
Per dare il lor pertante ai denti asciutti,
LENA. Vedi sopra C. 1. fan. 2.
EA maiticavan male, L' intendevano male, la fop n
E solito quando si pensa a qualche cosa fifamente, e con applicazio
care, onde Persio delle composizioni ben pensate disse: Remorfum |
Suem: E tal mafticare cos: pensando si dice auche raminare,o dig
mafticare che fanno gli animali del pié feflo perciò detti ruminantia
Vedi sotto C. 6. stan. 5. Qui fa bell' effetto ' equivoco del verbo
che pare che voglia dire / iutendevano male, e vuol poi dire che n
Ic, perché non mangiavano, non havendo che mangiare. i
STANZA VII. STANZA V
E tra costoro un certo girellaia, E, perch' ei non bavea tutti























Che per U' asciutto va fui fufechini, Fu il primaad esclamare, r¢
Male in arnese ye indoffo porta un faio Forte gridando:Obime
Che fu fin del Romito de Pulcini, Pel mal che vienein

Cit chi viel dir ch'ei dorman'ungranaio Onde Eravano,e Dow
Per c'bail mazzocchio pien di farfallini
E' matto in somma,pur potrebbe ancora
Wan di guarirne,percht il mal da in fuora,
STANZA

Mentre di gagnotar già mai non vesta E per vedere il fin dé
Colni ch' è senza numero ne rulli y Se ne van discorrende g
Anxi rinforza col gridare a testa, Del bifegnoch' essi han cb'il
Lasciano il fuoco ye e vani lor traftulli, Perché fentono omai fons
Fra li suddetti soldati affamati l'Autore pone se medesimo descri
erfona, e genio; e dice che egli fu il primo a gridare per la fame,
ravano,¢ Don Andrea Fendefi ancor essi affamati s' accoftarono a
tir la cagione di quelle strida, 3
Nota che il Poeta divide il periodo nelle due ortave,ottava,e nona,di ¢
to da qualcheduno criticato d' errore, ma pero senza ragione, non a
regola poetica, ia a pale vieti il poterio fare, come habbiamo detto
, G/RELLAIO. Huomo firavagante. Huomo che gira,s' intende

hs pre Ȣ che fa scioccaggini, e pazzie.

ANDAR ? ascivtto, Signi esser ro, e con poca

Vedi sopra Ca: stan. 68. % aad celia
VA infu fefeeliim. Ha gambe così fortui, che rafiembrano



























eee

QVARTO CANTARE, 19%

Mine wfatissimo da noi in questo proposito; che diciamo, Camminare fu fulcelii.
 ALAL? in arnese. Mal veltito: Mal' ail' ordine di sanita, d' abito, ec. Lalli
tr, lib, 1. stan. 34.
eben Pcs navi ao che gli avanzaro
Qui si conduffe afai mate in arnese.

ekildiise:Dotee ta inde dello sputo dice.
iAutee  Eccomi qui per raccontarne centey

Ben ch' io non sia d' accordo col ceruello,

E malagiato in arnese ms fento.
Il Persiani sCrivendo al Serenissimo Principe D, Lorenzo dice.

do, che sono in arnese tanto male,



site Mi ritravo in grandifsimo viluppo,

ics Teme efer pref in vece d' un galuppo y

via E finir la mia vita allo Spedale.

4 Franco Sacchetti Nov, 122. // Saccardo era guarito, e stava bene in arnese. Bocce.

walt) 2+0. 8. Partitofi aljai povero,¢ mal' in arnc/eda colui, col quale lungamente crds

ato.
it DEL Romito de' Pulcini. Questo fu uno che abitava poco lontano da Mal-
- mantile, e teneva vita eremitica, vestendo di lendinella a foggia di Francesca- |
yy nefealzo; Da costu prefe il nome di Romito quel luogo vicino a Malmantile
ae che dicemmo sopra C. 1. stan. 70. E perché egli oltre al procacciarsi il vitto con
. chiedere gwinae aiutava ancora col autrire nella sua abitazione buon nume-
ai ro di Polli per vender.' uova, fu nominato il Romito de Lulcini, Quando I Aa-
he © torecompole la presente Opera, detto Romito era morto di gran tempo prima,
f © pero dice che il /aio che eg\i haveva addosso fu fino del detto Romito, volendo
inferire che era gran tempo, che qucli' abito era fatto, ed in conseguenza oltre
, | all'effer vile per eficre-ttaco d' un povero Romito, era ancora lacero, e confa-
oA mato dal tempo.
yil S AIO, Gonnelletto, o cafacca, o simile parte d' abito da huomo; dal Latina
Sagums. WVarchi flor. fior. lib 9, E forte il Lucco chi porta un faio, chi nna gabba-
gl nella, o altra.vesticciola di panno chiamata cafacca.
wis DICONO ch' ei dornsa inun granaio, L' Autore medesimo lo dichiara, segui-
a tando + perché ha il mazzocchio pien di farfallini, se uno dorme, o si trattiene iny
ifet un granaio, si suol' empiere di quei farfallini che stanno fra il grano; e quando
id diciamo: I}taleha de' farfalliai,o delle farfalle,intendiamo E' mezzo matto; ¢
wif — dicetuello volante, o instabile. E per mazzocchio intendiamo il capo, perché
wil mazzocchio era una parte del Cappuccio, che già portavano i Fiorentini, se-
,  "ondo chediceil Varchi nelle sue storie Kiorentine lib. 9, ll Cappuccio s dice egli)-
wt ha tre parti, cioè il mazzocchio,il quale e un cerchio di borra, che gira, e fascia intor-
wt! = no intorno alla refia, e di sopra, foppannato di nero di ravescio, copre tutto ilcapo, Si
got dice a er mazzucco, © Così havea detto l'Autore, ma havendo il
yl - medesimoa dipingere uno dell' antico.Magiflrato di Firenze, mi domandd come
i era veramente l'abito Civile antico, ed 10 gli feci vedere questo Juogo del Var-
a chi, onde egli poi mutd, e disse mazzocchio per quanto vedo dal suo seconds
.  Originale, che e appretio di me~ 4
we moi IL,




}


















192 MA LMA NTILED

IL mate da in fuora, Quando il male da in fuora, cioè 'man
te l'interna malignita, (uol' essere indizio di falute; cofui essendo infer
pazzia, il dare in fuora di tale infermita e il far pazzie; e il Poet
potrebbe guarirne, perché il mal da in fuora, c1oe spera ch' et
olte pazzie, che e lo sfogo del suo male, ed il fuodare in



ha tutti i suoi mefi. BY spropotiraco. Non ha l'iatera pe:
uello. Non è stato tutti a nove i meli nel ventre di sua madre a p
ceruello. In fomima vuol dire Non ha giudizio; ¢scemo. tf
£.4R ma ina. Diciamo far marina coloro, che fingendosi stroppiati, e
piagati gridano, e si rammaricano per farsi creder tali; che tanto vale inigi
propolito Marinare,0 Jar Adarina, quanto rammaricarsi, o dolersi di cosa,
dispiaccia, ma per lo più s' intende di coloro, che fingono; come per
lo (colare battuto dal macitro,si dice far marina, quando fingendo che il
gli faccia gran male, piange, e firide a più non posso; che di dice anche
monello, Vedi sopra C, 3. itaa. 67. ak
VADO a Scefi, Quando diciamo; I tale ¢.andato.a Scefi, intendiamo
to, se ben pare che diciamo è andazo alla Citia di Scefi, o Affi, p
ho scendere ci servc pec intendere morire, Virg. fucilis descenfus.
PEL mal, che viene in bocca alla gallina, M male che viene inbocca t
na da noi € detto pipint dai Lat, peruira, E perché fra da gence baila in}
dire apperito si dive appipio, pero cavano questo detto':. / tale ha stimal
in bocca alla gallina, c10t la pipita, © intendeno appipite, cioè fame. E

tende il Poeta nel presente luago con questo detto piebeo. peo! t
ERAVANO. Cioè Averano Seminctti. Den Andrea Fendefi. Besdinando
Mendes. oat

PASCINA, Fascetto dilegne;Ed abbraciare insieme una fascinayy
ascaldarsi,, € spender ciascuno la sua porzione nelle legne; E vuol dit 0+
pertamente andare all' ofteria, Oraz. Ligna /uper foco /arge reponens. 6
STRVZZOLO. Vecello noto, il quale mangia così voracemente, che it
ghiotcisce fino il ferro, Dicendosi veutre di fruzzolo stintende Ventre i
Vlin. degli struzzoli. Concoguendi fine deleitu devoratu miranatura, agat
AUNVZZ OLLI, Quci minuti fragmenti, che cascano dal pane, quando
spezza. E quest' atio di cercare i minuzzoli nelle tasche,e(prime uno che:habbia
grandissima fame. odes
GAGNOLAKE. Voce corrotea da cagnolare, che & il guaire, chefanno!
cagnolini quando hanno bisogno della poppa. Se per avventura non lo
vatfimo dal verbo Latino gannire, che signitica Rammaricarf con. parole no
affatto intese mescolate con sospiri., e fingulti, che è quelio, che nel presente?
uogo vuol dir gagnolare. =
E SENZA namero ne i rulli, E' matto. Nel giuoco de rulli fipighi v
© pil, o sr eis SORT > a. de ses hail suo
che uno, il quale jama 11 Matto; E però dicendogi: 4 zale e ih fengammme?
frairulli, i uathnide @ il rocchetto, che e senza numero, cio? il mateo «Quel
rocchetti si chiamano radi, perché rizzati in terra in socal oa
a

nel mezzo, vi si tira dentro con un Zoccolo di leguo grave tondo di






QVARTO CANTARE: 193

midale, il quale si chiama rullo, € il giuoco si domanda «'Rutli, ed alle volte
Samer chi pil ne fa erlest i kee tiro vince. Si costuma anche tirare

dil no. wu
SRINPOREA | Ciotctelee lo Grider,

GRIDARE a



'tefl, Gridar quanto più

0 il guaire. L. ingeminat. Si raddoppia,

si può. Si dice anche gridare 4 corr'huo-

mo,0 quant' uno n ha nella frrotea; nelle canna; o' nella gla. Vedi sopra C. 3,
Ranbgjeiq ish si. pasalty

TRASTVLLI. Trattenimenti*. E' voce da Fanciulli, e qui vuol esprimere.,
che futiero veramente traftulli da bambini, perché aggiunge l'epiteto vani, come
era veramente il cercare de i minuzzoli nelle tasche.:

PER vedere il fine di lla feta. Per vedere in che haveva a terminare 7» Oa,
che fine' fuffe (aeoqelbnomice + Quando un discorso, o un suono » o un Cantare,
o altro romore comincia a venirci a fastidio diciamo: Quando finirà questa
Sefta; questa musica 5 questo chiaffo'; questo bordello; questo baccano; queffo mirscaiore
fmili, Vedi forto'C. 9: stan. §1.€€, ro. stan. 53.

GRVLLO., Int

'eadiamo'uho melancolico,sbattuto da cattivi effetti,e non affat-

to fano, che si dice anche Acquacthiato; E tal voce € presa forse dalla Grue uc-
cello (Spyruila)che quando sta fermo posa un fol piede, e tiene Pale baffe in ma-
niera', che pare un pollo ammalato; che pero tal pollo, ed ogni altro uccello
Cost'ammalato fi'dice Zruilo, o che porta i frafeoni, Vedi fotta C10, stan. 20.

SENTONO suonar la lunga, Quando il Prete per

inuitare j popoli alla Meffa,

suona la campana, e' dura Hs tempo, in contado dicono /uanar la lunga. B

da-questo durate lungo tempo

STANZA X.

Così domandan chi sia quei ch' esclama,
E metre grida jd urli st bespiali |
Glié dette; Quefioeun tale, che fichiama
Perlone dipintor de' miei ffivali,
Vahuom @al mondos'acquifta gran fama
Nel far de' ceffantts pe' boccali,

E con gt induspri,¢ dotti suoi pennelli
Suo nome eerno fa negli sgabelli,

icendosi: il tale sente suonar la lunga, s' intende
me per esser lungo tempo, che non ha mangiato. E
pertamente diciamo: Eeli ha quella de! Carmine, s' intende
Chiefa del Carmine di Firenze,avanti si dica la prima meffa
na-per un grande spazio di tempo, e questo suonamento si dice da tu
del Carmine.

Per significar più co.
la lunga, perché nella
,suonano una campa-
ttl fa lange

STANZA XI,

Si trova in bale frato, ani meschino,
Ma ben che il furbo ne mancect pochi,
Giuocherebbe in fw pettini da lino -
Che ur'ora non puo viver ch'ei né iginochi,
Ma £ti vincelfe un di pur'un quatiring
dn vero si potrebbon fare è fuachi 7;
Perch giocando sempre Siorno,e notte >
Farebbe a perder con le tasche rotte,

STANZA XII,

Giuocoffi un suo fratel già la sua parte;
Suo padre fu deleinoco anch'egli amico,
Pero natura qui n incaca l'arte

Havendo itato un genio antico,

Costoro 'intesero, che'colui, il quale cos} gridava era Perione

Costni teneva in man prima le carte >
Che legato gli fuffe anco il belico:
Epriache mamma, habbo,pappa,e Poppe
Chiamp [pade,baston, danari,e coppe,
> cioè Periones

Zipoli, che vuol dive Lorenzo Lippi Autore della presente Opera; e fa che ven.

§a deicritto per uno sfortunato, ed oftinato giocatore.
i Bb

MET.











194 MALMANTILE,
METTE frida, ed urli beffiali, Stride, ed urla gagliards
perché lo fridere € proprio del porco ferito, ed wrlare &
cane, e lupo; s¢ ben ce ne ferniamo anche per l'huomo i
DIPINTORE de' miei stivali. Pittore dappoco. EB'
ro, che sanno poco in qualfivoglia scienza, o arte. V
E frvale diciamo un huomo gotfo, e di poco giudi:
scarpa, che cuopre tutta la gamba, es' usa per ca)
poco si dice Pittor da sgabelli, da boccali, da colombaie, ec. come si
sente ottava, che dice: Fa de' ceffaurti ne + boccali, Econ gl indufirss
eterna il suo nome negli sgabelli. Ma perché questa sua modeftia, ed h
sia di pregiudizio al merito di così gran valent' huomo, repli
tore riputatiflimo, come le belle opere sue chiaramente teltifi
firera il sig.\ Filippo Baldinucci, se mandera alle stampe la faa,
Pictori, Opera degna d' ammirazione si per le belle notizie, che si
fa, esi ancora per sapersi, che questo erudito huomo l'ha ritrovate a
fieme in brevissimo tempo rubato alli tanti riguardevoli affari, che p
benefizio lo tengono continovamente occupato,;
CEFF AVYTT/, Voce composta delle note Musicali Ce fa, wt, e 00)
ficato veruno, se non che mostrandosi di dire la chiave del Cé fol fa at
Ceffo, che si piglia per vifo, o faccia, se bene appresso di noi cefo vi
di cane, o grifo di porco, E quantunque venga forse dal Greco ©
dir Capo, onde anche i Latini, chiamano Cephalea un certo dolor di
in Franz. chef sia capo; nondimeno noi non ce ne serviamo se non peril
per intendere una facia brutta, e fatta male; © perdl Autore 5 ¥'
tenda, che Perlone dipigne male, chiama cefi quelle facce, che egli dipl
per altro parlando pittorescamente chiamerebbe Tefte. sped B,
bocc-dZE, E' una milura fatta di terra cotta invetriata capace deli
dun fiasco a, 3 ne ogni sorta di vaso sia pil par
rande; che sia però di questa materia, e figura. E perché questi boccali da
ai, che gli fabbricano in Montelupo (on dione ted eae se senza un mal
mo dilegno, però a uno, che dipinga male si dice Pitror da Boceali 5 0» Pittoest
eMontelupo. oc et le
BASSO ffato,anzi meschino. Povero mendico; Poverissimo
FVRBO. Propriamente ladro dal latino. fur, ed & parol ingen
tavia si piglia per 4/tuto, (agace, caltrito,¢ che sa il conto suo: Qui vuol
fo, perché ha il vizio del giuoco, Fur a furuo, i, migro dietus, Papiat.
 AVE maneggi pochi, Intendi: maneggi pochi danari. Non gli vengs,
gran quantità di danari. 2
GIOCHEREBBE sx i pettini-da lino. Intendiamo uno, che giod
ogni iore scomodo, come farebbe, s' egli stesse a sedere in fui
no, che son composti d' acutissime punte di ferro., pote Ne
'SI potrebbon fare i fuochi. Si potrebbono fare i fuochi in segno d'
come d' una cosa infolita, Detto usatissimo, quando si qualco
gusto, che fiamo stati buon pezzo aspettandola; Che si dice anche Sa
doppia, Vedi sotto C, 6, Ran. 107. #

































a

QVARTO CANTARE; 195
well PARERBE 4 perder con le tasche rotte. Perderebbe sempre: Farebbe a gara 24
un chi più con'te tasche rotte, quantunque queste perdano tutti li'danari, che

ha ineffe fimettono,

 INCACARE, Disprezzare: La natura non sa grado, e non ha obbligo «/'
pCeh arte, non essendo flato opera dell' arte, che egli giuochi, ma effetto della natura,
éiy che" ha prodotto con questo vizio di giuocare. Dan. Pur. C. ro. disse;
eye Na la natura gli haverebbe a feorno,
wht VN genio, Vedi sopra C, 1, stan. 31.

“e PRIMA che gli fulfe legato il belico, Subito ch' egli ulci del ventre della madre:

j,i Bellico', Diciamo quella parte del corpo, d' onde è preso i) nostro primo alimen-

coyitg £0 nel ventre della madre; la qual eae nel venire al mondo è¢ legata dalle nutri-

jumt@ ci. B ciò serva per dichiarazione del presente detto.

fu Goa SABBO, Mamma, Pappo, e Poppe. Sono delle prime parole, che si profferi-

ceil = scono dai bambini, come s'é detto sopra in questo C. stan.s. Ma questo Perlone
prima /pade, baston, denari, e coppe, che sono li quattro segni differenti

|
— Srasaoe arte da giuocare, che si appellano femi, come vedremo sotto C. 8.
. stan. 6, E qui 4 sn fa dire per mostrare, che prima d' ogni altra cosa questo Per-
gat Jone ona il giuoco, e che venne fuora con cotefto a eee Ms giuocare.
a ZA XIII, A XIV.
jest Ma Toe voi sappiate il personaggio, E' swo amico, ed e pur seco adeffo
a ofa Saluo Rofata un huom della sua tacca,

i cib'racconta,è il Franco Vicerosa,
pi i Cavaliero, del gual non è il | più, Saggio; Pero che anch ei sabbeverain Permefo,
Scrittor fubblime in ver/oyquate in profa; E Pittor paffa chiunque tele imbiacca;
mt Dipinge, ne pus farsi da vantaggio Tratra d ogni ers at ex profeffo,
, e in qualfivoglia cosa: E in paleo fa si ben Coviel Patacca,
cg Vince mel Canto i mufici più rari, Che sempre ch'ei si muove,och'ei favela
E nel portare ecchiali non ha pari. Fa proprio seangherarti le mascella,
' STANZA ¥ Vv.

ap 2
4 Hor percht Pranco, ed egli ogni maniera La dove minchionando un po la fiera
: 'Proceuran sempre ai piacere altrui, Mt Franco disse lor; Queffoé coli
iA Di Pertone dan conto ye, don' egli era, Ch in xucca non ha punto,anziragionaft
Di conserua n' andar con gli altri dui, Diappiccargli alta teffa un'appigionafi.
Acciò che si sappia chi e colui, che da tal notizia di Perlone, dice; i
haveva nome Franco Vicerosa, cioè Francesco Rovai Cavaliere dotto, Poeta,.
7 Mutfico, Pittore, e veramente dotato di quelle buone qualita, e virtù, che dice
' jl Poeta,e che stanno benissimo in suo pari, come teftificano alcune poche sue.
 Poesie stapate dopo Ia di Ini morte, che non sono anche le migliori, che egli facefie
| Dice che nel portare occhiali non ha pari, perché haveva nafo aquilino assai grande.
Con eflo & Saiwo Rofata, cioè Saluador Rofa huomo anch' egli dotto, e Pittores
eccellente, il cui valore e notissimo, mostrandolo a bastanza le di lui stimatissime
Opere; e aoe valeffe nella Poesia si conoscerebbe da alcune Satire da lui fat-
te, le quali ra vedere una volta alla tampa. Questo era amicissimo dell'Au-
tore, € fu causa, che egli tirafle avanti la presente Opera, persuadendoli, che»
era ce godere l'aggradimento universale, e gli dette anche notizia de lo Cunto
degli Cunti pubblicato in quei tempi.  Saluator Rofa recitava da Napo-
B 2 Jetuno

!
che egli
PAS Br FRE, FM S

i

Quel! aggiunta di fera e solita mettervifi, ma non so gt a



196 MALMANTILE |

Ietano in commedia mirabilmente, e firfaceva.chiamare
sto Franco Vicerosa, e Saluo Rofata insegnarono dunq
defi chi, e dove era Perlone. a7,
AVOMO della sua tacca, Huomo simile.a lui. Vniformi di ge
ca detta anche raglia e un pezzo di legnetto feflo in due parti
quale serve per libro di conti a coloro, che non sanno leggere, ini
Vniscono dette due parti di legnetto, € nella parte più spianata f
tacche, o segni col coltello, 1 - segni denotano il numero delle ¢
credenza, o dei danari, che fidevono,, o de i lavori fatti, pezzo:
eflo legno rimane appresso al creditore, e l'altro appresso al debitore:
si voglion dar nuoyi danari, o segnare nuovi lavori, s' uniscone detti
vi si fanno i segni che occorrono; E volendo aggiustare i conti si
gni, e si vede la quantità del debito, o credito: ne vi può nascere i
ché se in una delle dette parti di legnetto fara fatto un segno di più 5 'a
far nell' altra, perché non riscontrera, se il debitore, e creditore non!
dono scambievolmente detti pezzetti. Era in uso questa maniera di
anco appresso ai Latini, che tal Jegnetto, che noi appelliamo Tagia yo
la dicevano tefera;: Swam uterque teferam habet; ratio conftat. Ha}
un' altra taglia, che chiamavano Tefera ho/pitalis, la quale servi
re gli amici, e corrispondenti di diversi pacfi, ferbando ciascuno
goetto; il quale si lasciava anche a gli Eredi; E quando andava
dell' altro portava la parte del legnetto;e unendolo & dava:a conole
te; e pero detti legnetti erano cuftoditi diligentemente. Questo
Plauto in Pen, Ezo fum ipfus, quem tu quaris. P, hem quid ego audio?
gnatum eje. PB. Sé ita ef, Telferam me conferre bospitalem Si vis ceca a
tli, Donde havevano poi, T¢/seram frangere ho/pitalem, che significa
hospity. Dal che si cava, che homo eix/dem tessere, sia lo stesso, che!
medesima taglia, che significa delli stessi genj, e corrispondente. Diguihi>
biamo il verbo attaccare, che vuol dire Vnire due materiali insieme 'ire
ho atagliare, che vuol dire Esser uniti di genio. Ricord, Mal. Sror-Fionapiy
dice: Lucca, Pifoia, e Volterra feciono taglia co' Fiorentini,.¢$' i
garono, © fecerolega; E si trova ne gli antichi noftti Storici &
lega. Om
PASSA chiungue tele imbiacca. Supera ogni Pittore. I co ee
FA sgangherar le mafeella, Fa ridere scegolatamente, che &,quel Rife quae
che dicemmo sopra C, 3. stan. 66. alla voce Pimmei. E veramente (
ne gli anni suoi più giovenili, che dimord in Firenze recitaya.(
detto ) questa parte di Napoletano così bene, che si può. dire, che eglil
Maeftro in far questo Personaggio, wukseuee;
eANDAR di conserva, Andare insieme.. Detto Marinaresco,
significato.. wise oe
BU MINCEUON.ANDÒ (4 fora. E' il latino derideo, E tanto yale ilvebom®
chionare, che CO...... Che non si dice per essere sporco, ed-usato ah Wt


















Se



to suona il folo verbo mixchionare, s¢ non che

pes









QVARTO CANTARE: 197
era, esser,detto da coloro, che non avendo voglia di comprare paffeggiano per
ie fete J del prezzo di questa, o di quella cosa, e non offerendo tea

 te, o pochissimo; e stanno a vedere, e osservare chi compra. E venuto poi a»

ee tines assolutamente, e si dice ancora Adinchionare la Matter..
edi fonto.C, 7, stan. 15. EB pur qui ancora senza l'aggiunta di A¢arrea suonas

i | £W.xxcea non ha punto; cio' punto di fale, €s* intende: Non ha ceruello in te-
nie) © fla, Vedi sopra C, 1, Man. 53. 1] Mauro in lode della Cacia dice:
vVaetss Ed io, che sono un buom materiale,



| pe Cencande cit ben mafirerci cl io false

foots Da dovero una Zucca senza fale.
Catullo di Quinzia disse: 2

to.s mica falis,

= 0, $93 Wudla in tam magno off ci
odpiil ATT ACC ARGLI alla testa un' appigionafi. Efiendo ia sua testa vota;per mo~
firare, che ella si può afficare si discorre a' appiccargli -appigionafi, che così chia-
mo quella cartelia,in cui fla scritco a lettere grandi APPIGIONASI, e s' ap-

miamo que!
icca sopr' alle porte delle case difabitate, affin che si conosca, che quella è cala

mani
aia ha affittarsi, o appigionarsi, appunto come dice, che era la testa di Perlone, che
feaingt per esser yota di ceruello, era in grado da potersi affitcare, o appigionare. Iny
lawl alcuni i d' Italia conferuano l'uso antico, scrivendo in L. Ef locanda,
ania  STANZA XVI. STANZA XVIL
cl Spiscqued (ito male ad ambi tanto tanto, Se forse dice; tu fei frato offefo,
y, E mentre @ piange, che si getta via, Che fai tu della spada il mio piloro?
ch pt 4 pietofa Eravan pianfe al uo pianto e-4 che tenere al fiance questo pefa
as! Verbigrazia per fargii compagnia; Per startene a mangiante come un boto?
ai Poi tutto liero postefecti accanto S' al corpo alcun dolor t° havesse poi
Ler cavalo di quella frencfia, Gli è qua chi vende I olia dello Scoro;
oo Di quelle firida, e pianto si dirotto, Set' bai bisogno a! oro io ti fo fede,
Che quaifivegta Banca te lo crede,

,
gt. Che fa per nulla il bretolon mal corto.
ca  A costoro dispiacque molto il male di Perlone, ed Eravano dopo haver com~-
' planta questa (un 'disgrazia, si mefie a confolarlo, e ad efaminarlo strettamentes
# per sapere la cagione di si gran suo pianto.
Ji BLETOLONE mat corto. Huomo sciocco infipido, fuenevole, appunto come è
la bietola: Marzial. 13. Vt /apiunt fatus fabrorum prandia beta, voce Sie-
tola, che viene dal Latino bera, che vuol dire una specie d' erbaggio, tanto nel
 nostro idioma, quanto nel Greco, e nel Latino serve ancora per esprimere un'
# —huomo seiocco,ed infipido-, Laerzio nelle vita di Diogene Cinico dice così; Cir.
| cumfbantibus se adolescentibus est dicentibus: Caveamus ne mordeat nos: Bono inguit
| essete anima filioli 5 carnis enim betis non vescitur. Plin, lib. 20. cap. 22. mostra, che
i mariti volendo dire villania alle mogli dicevano loro b/irea, raccogliendolo dal-
le commedie di Menandro; e si legge in quelle di Piauto., intendendo una cofas
fsciocea, e che non è buona a nulla; E come noi da bieto/a caviamo il verbo svie-
tolare, che vuol dire Scioccamente piangere.( Vedi sotto C, 7. stan. 93.) e imbie-
solire, che vuol dire Commoversi, o esseminarsi. ( Vedi sotto C. 9. stan. 57. ) così
gli antichi havevano berizare, che ha lo stcflo, o poco difference signiticato.
A Bie-













198 MALMANTILE \— i
Bierolone dunque suona lo stesso, che Scimunito; ma con l'aggiu
vuol dire Scimunitissimo, perché la bietola cotta poco 5
della cruda
en; oe colui, = governa la Nave: dagli an
to Pedorto forse dal L. pedes preso per remi, come appresso Planto
© per funi da mie eolar efecto « Ma questa voce Piloto ci
mere un' huomo da poco, ae » irreffoluto, e flemmatico;ed in
è preso nel presente luogo, Vien forse in tal caso dal Lar. pi n
mo, che per havere i piedi troppo piatti, e contraffatti cammina male.
to C. 6, stan.
4 CHE portare, A che fine portare? Che occorre che tu porti?
facis? Ad quid venifti ? nel Greco dice eph' boo, cio per l'appunto
ST ARSENE 4 man ginite come un boro, Boti chiamiamo quei F.
tue, che si mettono attorno all' Immagini miracolose per 8
ricevute,e però si dovrebbe dir Yori, ma per iscambiamento di lettera si

Berni in biafimo d' un' huomo brutto.
Fuege da' ceraioli
Acciò che non lo vendan per un boto; ce
Che anticamente detti Fantocci si facevano di cera, e per lo pits
giunte in atto d*orare, e per questo dice /rarfenea man giunte come
s' intende d' uno, che non sappia, o non voglia operare, em
lavorare; e vnol' inferire, Che fai tu delle mani, edella spada, che
doperi a vendicarti, se v e stata fatta ingiuria ? Mons, della Cala
boto per modo di dirlo sempre. i
LO Scoro, \ntende di quel Ciarlatano, che vendeva Lattovarj 5
a veleni detto lo Scoto.
TE ls crede, Scherza con l'equivoco, dicendo ogni banca te lo
banca ti crede, che tu +habbia bisogno dell' oro, e pare che voglia
banca ti fidera, o preftera l oro.
STANZA XVIIL
Dopo Eravano poi neffun fu muto,
CP ogaun gli volle fare il uo discorso
Offerendo di dargli ancora Aiuto,
Mentre dicefse quanto gli era occorso;
STANZA XIx.

Won v' è rimedio amici alla mia forte; =
” I. tutto è vano y già che la sentenza MU soldato ciat nel ciabattino
E' stabilita in Ctel.della mia morte y Peri:che miscomuien.,

Che vnol ch'io muoiase muvia in mia presiza,

Già l alma ffivalata in fu le porte
Omai dimostra a? esser di partenza.
Già con il corpo tutti i sentimenti

Le cirimanie fanne 5 €4 complimen





Ed e che sotto son come wn
Eldinnanzi a Minos,cagh
Rappresentar mi devo co,













“di noi si piglia in diversi

QVARTO CANTARE. 199
STANZA XXL STANZA XKXIL
Wa ecco omai L bora fatale e ginnta, Hormai di vita for nscito, e pure
"Chto lasci il mio terreftre cordovano; Lon trove al mio penar quicte,o cefarto,
Già già la morte corre che par' unta O Cielo Mondo, o Giove, o creature
'fo dime con la gran falce in mano; Dite, s' udiffe mai così gran tor to ?
inge ella il ferro nel bel fendi punta, Se Morte e fin di tutte le sciagure,
3 io mancar mi sento a mano amano: Come alupar mi fento ancor che morte?
Pero lo spirto, e il corpo in un fardello Ecome, dove ognuno esce di guai,
Tivo fuor della vita, e vo all' avello, Mi 8 agugxa il mulino pile che mai?

 Anche gli altri dopo Eravano gli offerfero il loro aiuto, ed egli fingendosi paz-
zo\comincia a dire una mano di (cioccherie,e mostrando di creder d' esser morto,
si maraviglia, che mors, qua omnia foluit non gli habbia levato l'appetito di ci-

HAVERE feorfo col ceruello, Esser' impazzato. Haver dato la volta al ceruel:
lo. Metafora tolta dail' orivolo a ruote, che si dice gualto, quando le ractes
scorrendo escono del lor moto regolato.
APFFISS AR gli occhi in uno. Guardare senza punto movere gli occhi; atto da
pazzo di quella specie, che domandano Maniaci.
ALLA mia forte, Di quel che m' ha da succedere. Questa voce forte appresso
Woidcari » come seguiva anche appresso a i Latini, da i
quali si diceva fors ogni avvenimento di Fortuna. Cic.lib.2.de Divinatione. Quid
enim [ors eft? idem propemodum, quod micare, quod talos iacere, quod tefseras, ed in
senso & preso nel presente luogo. Si dice tirar le forti,per intender quel

super vestem meam miferunt fortes dell' Buangelifta.

La pigliavano per carica, o incumbenza, secondo Livio: Si id gravaretur fa-
sere, quod non sue fortis id negotium esset.

La pigliavano per stirpe, secondo Ovid. 6. faft.

a; Si genus aspicitur, Saturnum prima parentem
. Feci, Saturni fors ego prima fui

La dicevano anche il capitale, e quello che noi pure diciamo forte principale;
Plaut. Mott, Quatuor quadraginta illi debentur mina, Et fors,& fenus DA,tantum est.

Altre volte pigliavano /ors, pro ixdicio secondo Verg. 6. Aneid,

Nee vero be fine forte data, fine indice fedes,

Perché ( secondo Servio ) non s' udivano le caule wif per fortem ordinate, nam.,
quo tempore cause agebantur conueniebant omnes, 7 ex forte dierum ordinem accipie~
bant, quo post trigefimum diem causas fuas exequerentur,

Dicevano forte gli Oracoli, o risposte, o le polizze sopra alle quali si (criveva-
no le risposte. Val. lib. 1. Cuins rei exploranda gratia legati ad Delphicnm oraculit,
vetulerunt: scent fortibus, ut aquam eins lacus emifsam per agros diffunderent, Virg.
in questo senso disse: Lycie fortes. Appresso noi aucora ( come ho accennato )
Jorte si piglia per fortuna, o destino, per condizione, flato, o essenza. B dicia-
Mo toccare in forte, che significa ottenere la benefiziata, quando s' eftraggono
Ie polizze, che e quel mittere fortes; e se bene in significato di fortuna vogliono
alcuni, che si debba dire forte, ed in significato di qualita, o condizione fore,,
hoggi ( almeno nel parlar familiare, e Civile ) non trovo, ches' usi tal distinzio-

ne,

ib aa
200 MALMANTILE —

ne, ma fento usare alcune volte ' una per l'altra indifferentemente} ) ~
ClLABATTINO. Vno che raccomoda scarpe rotte; iabatta
re Scarpa vecchia, e scarpa all' Appostolica, che (ono quelle, che:
Cappuccini. In molti luoghi de' contorni Fiorentini chiamano C
ra quelli, che fanno-di nuovo; che noi chiamiamo Calzolai, in Up
similusente zapaceros; € questo nome di Ciabatraviene
cioè (carpa ferrata con chiodi; quali (on quelle che usano i contadi
ciatori. i
TIRAR le quoia, Havendg detto, che di soldate doveva diventar
la ragione perché; ed e questa, che gli conuien tirar le quoia, come fanne
battini, e 1 Calzolai, che tirano i quoi per condurglia quella mifura, e i
no, delle quali quoia dice, che si dee servire per rimcalzare it pino., cio' fark
scarpe al pino. Nota che lo scherzo dell' equivoco, nasce da rirar le quoia,
vuol dir Morire, e rincalzar con efse il pina, che vuol dire Parsi forerrare
del pino, e così alzandogli la terra attorno rincalzarlo., che questo vu
ealzare un' albero. Ofierua ancora, che facendolo parlar da pazzo'
coloro credano, che egli habbia concepito nel cerucllo questo spro 1
a fare le scarpe a i pint; perché quando un Calzolaio dice; Io calzoil ti
tende Io gli fo le (carpe. Plut. in Dem, E ca/zandoft dices, Il Gr, erepidas si
SOTTO, son come un cammino. Sono schifo, ed ho le carni fudice,
cammino, dove si fa il fuoco, Comparazione usatifiima particolarn

donne. W s
MINOS, e gli altri Gindici, 1 Giudici dell' Inferno secondo le
tichi Poeti, e della Gentilita sono tre, cioè Minos figliuolo di Gi 3
a, che fu Re di Candia, Eaco, che fu figliuolo di Giove, ed' }
¢ d'un Ifola già detra Enopia, la quale egli poi dalla madre chiamo Bgi
Radamanto, che fu figliuolo di Giove, e d' Europa, che fu Re di Li
sti Re,perché furono feveri amatori della giuftizia,dicono i detti Poeti
tone gli cleggefle per Giudici dell' Inferno, affinché efaminaffero } anime
aflegnaffero loro le pene, che meritavano, e da quello, che di loro ferive
2a. 6. si può comprender il lor precilo, e particolar ofizio 5 che di
Quafitor Adinos urnam movet, ille filentum
Concilinmque vocat, vitas, & crimina discit,
E di Radamento dice;
Gnofius hae Rhadamanthus habet durissima Regna,
Castigatque, anditque dolos, fubigitque fateri, §
D' Eaco parla Ovidio così;

































ytd






Tualque ae

Eacus in penas ingeniofus erit, td
E conchiude i} Poeta, che uno di questi Giudici efamini, 1" altro giudichidl
zo mandi ad esecuzione. Se ben Dante nel 5. dell' Inferno.dice:
Stavvi Minofse orribilmente, e ringhia, ss
Efamina le colpe ned entrata, aa a
Giadica, e manda secondo ch' avvinghia. oi

CORDOY-ANO. Specie di quoio da fare scarpe, la concia del uae a for






QVARTO CANTARE. zor

inventata in Cordova,e perciò tali quoi chiamanfi riamente cordovani,e son
poll di Castroni podta "re: qui ioteade, pelle humana', e dicendo
tafei ly cordovane intende io muOia, come intendon quelli, che dicono
Terrefire falma;T errena /poglia,e simili,Cunto de li Cun, Peffoe concio per cordonano.
CORRE che xnta., Corre velocemente; comparazione dalle carrucole, ©
puleges » o altre simili, le quali quando sono unte con olio, fapone; 'o altro,

nate ' a os:
PALCE. Strumento,col quale si fega il sieno; e col quale spesso si vede dipin-
ta lamorte com efla in mano. anne A>
GVAl, Travagli, fuenture, sciagure, afflizioni. Vedi sopra'C. 1, stan. 28.
(ALLER ARE, iiceor gran fame', perché dicono, che il lupo sempre habbia
gran fame; quindi il volgo chiama Male della Lupa quello di coloro, che fem-
pre mangerebbono, perché da joro vien preftiftimo fmaltito il cibo con pochiffi-
mo nutrimento, ed e quella infermita, che i Medici chiamano Fame caninaa.
Vedi sotto C. 5. stan, 61. E da questo male chiamato della Lupa diciamé allnpare
uno 'che habbia gran fame -
\AAGVZZARL it mulino, Par-venire s0crescere V appetito: perché aguzzare
da-macine del mulino yuo! dire Metterla in tagiio in maniera, che si renda più
dngorda.. Vedi fatto C. 7. tan. 31.

STANZA XXIII.
Va'a dir che qua si trovi pane, o vino
O altro da insegnar ballare al mento;
1 Se nonifi fain cenaidi Salusno,
» Qaaato a iar non te afeenamito,
oO iepecdineen 70 i ¥
yuihavete a ireal monumento,
Vor l intendete, che nel cataletto
Com voi portate il pane,ed il fiaschetto,
oa STANZA XXIV,
¢ Let pet old dal cimitero,
S' it Ciel danari's e fanita vi dia
Empiete il buzzo aun morto foreftieroy
O insegnateli almena un ofteria;
Se ben voi fare qui sempre dinero,
Perché di carne havete carcftia:
E' tale appetite che mi scanna
Chiun Diavol corto acor mi parra mina,
STANZA XXV.
Se ben non c e da far cantare un cieco
| Diguefta [pada alOfte foun presente,
C'ad ogni mo, da pai cht ella sia meco,
Mai bate colpo, o volle far niente;
Per una xuppa dollaancor di greco,
Mache gracchioia? Qui neffun mi sente.
Che fo? s* i morti son di pietd privi

STANZA XXVI.

Qui racquese per fuggir la via si prefe
Facendo sempre il Nanni,ed il corrive,
Perch'eglie un di quei marti alla Sanefe,
C' han sempre mescolato del cattivo;
Per haver campo a feorrer il pacfe
We fere pol di quelle con l'ulivo
Miffrando egn'bor più dar nelle girelle,
E tutto fece per faluar la pelle.

STANZA XXVIL

Perch uno, ch' it soldato a far #  meffo,
Mentré dal canipo fiigde, e si rravia,
Sendo trovato, vien senza proceffo
Caldo catdo mantdato in piccardia;
Però £ ei parte non wiiol far lo Refs,
Ma che lo feufi, e falui la parria,
Onde minchion minchion factdo itmatto,

Se ne [cantona', che non par suo fatto,
STANZA XXVITL ©

4M Fendefiascappare anch'ei fu leffo

Con gli altri tre correndo a rompicollo,

Volendo rificar prima un capresto,

E maorir con ta fromaco farollo, A

Che refar quiviiamenarsi Pa...

Ed allungare a quella fozgia il cello;

Li danno certa è sempre da fuggire,

Meglofara ch jotorniaftartraivivi, Co S'egli avvien peggio puinonc'e che dire,


202 MALMANTILE

Petlone seguitando a dire i per esser tenuto matto si 3 e per fale
var la vita continovo a fare ie iioctiee ie, sapendo, che un to aaa
pa dal campo, e si parte senza licenza e reo di morte, ed il Fendefi 5 e gli altri
scamparono anch' essi. i Seek

Vda dir che qua si trovi, EB! yanita il credere, o dire che qua si trovi;s' ingan-

na chi crede che qua si trovi. “vy

INSEGNAR baliare al mento. Mangiare. E' lo fleffo che Dar il portante a'
deati detto sopra in questo C, stan. 6. q

FAR la cena di Saluino, Andare a letto senza cena; che la cena di Saluino era
Pisciare, e andare a letto. *

O SER H/ac,o Abramo, o Iecodino; Lptende tutti gli Ebrei, e seguitandd ¥ opi-
nione del volgo, il quale crede, che quando gii Ebrei feppelliscono i loro morti
metiano lore appreiio del pane, e del vino dice; Voi / intendete che morendo por
rate con voi il pane,e il vino, poiché nel mondo di qua non si trova ne da mangia-
re, ne da bere. Wig

CAT ALETTO, Quella barella, entro alla quale si pottano i morti al sepol-
cro, che i Latini dicevano fererrum. Voce composta di Lettoye Cara preposia. Gr.

ORBE, old, alo, E simili; sono voei, e termini usati per farsi (entire da chie
alquanto lontano; come fa il Latiao eas, Orbé, e fatto da Ora bene; Or beat
Latino age vero; Alo dal Fr, ailons; andianne.

CIMITERO, Piazza, nella quale si fanne i sepoleri per li morti, Voce che
viene dal Greco she, che suona dormire y rip » Onde jon, lo
stesso, che Dormentorio. Quindi i Cretenfi chiamavano Cimeterid una calas
pubblica, la quale ferniva per alloggiare i pellegrini. Vedi sotto C, 7. stan. 27.

S°LL Ciel danari, e fanita vi dia, Dice questo sproposito per acerescere in
ro la credenza, che egli sia matto, sapendo bene che i morti non hanno bisogno
di fanita, ne si curano di denari. gt 04

3VZZO. \itendi il ventre dell' huomo, da bufto che s' intende tutta quellas
parte del corpo humano, che è dal collo al pettignone, senza le braccia. -

FAR di nero, Mangiar di magro. I venerdi, abati, Quarefima, ed aitre vi-
gilie si chiamano giorni neri, quaG giorni di lutto destinati alla penitenza, il
Poeta scherzando con l'equivoco del acro, col quale è solito farsi !apparato a'
morti, par che voglia dire non mangiate mai carne, perché soggiunge di carne
havere carefiia, e par che intenda non havete carne da mangiare, e vuol dires
bon havere carne in sa l' ofa, perché i morti in breve tempo restano puri (chele-
tri senza carne. a

e4PPETITO che mi scanna, Fame così grande, che mi fa morire, che mi fa
perder la canna della gola; che scannare uno, vuol dit Tagliarli la canna della
gola. Cunto de li Cunti Giorn, 1.\Se la necefjita non la foannava, 6

MI parr manna, Mi parra buonissima; come paruc, e sua gli Ebrei la Man-

na, che mando loro Dio nel deferto, che ricevendola elclamayano Adanu cio. -

'Che è questo ? onde forti il nome.

WON ho da far cantare un cieco, Non ho ne meno un quattrino da darlo a un —

cieco,perché canti un' Orazione.:
LA ogni me, Per: ogni modo. E' termine assai usato in Birenze in diversi fen,
per:






=

EF ar eae aoe

CaS pc FF ES.

LL ER FEE








QVARTO CANTARE: 203

ereeeaeens come nel presente luogo, Haglio dar via la spad.,
4d ogni modo non baste mai colpo, cine perché io non la stimo per non hayer
i rato, O significa ne ceffita di fare, o non fare una cosa per esempio,
a » che ad ogni modo s' hada morire. Significa contentarsi di
' sche uno ha confeguito: fo. ho guadagnato poco, ma ad ogni modo io mi
contento. Significa Oftinazione. So hate anaes po muocere 5 ma La voglio
fare ad ogni modo. Vedi sopra Can, tr. stan, 3. il termine; /vo danno, che pat che
habbia correlazione al termine, A ogni modo, V.gr. Se '0 ho perduta la tal

' anno; ad ogni modo io non mene servivo, E quel mo per modo &
Ta figura da noi molto usata come vedremo altrove.

tAl batsé colpo. Diciamo + il tale non batte mai colpe per intendere, il tales
ora mai, e qui intende, che la spada di Perlone nelle sue mani non lavo-







 4¥PP A. Pane intinto nel vino, o in altro liquore, Forse meglio Suppa, Fran-
0 Sacc, Nov. 86. E fatea la fuppa con le spexie,subite porta in tavola il ventre, e la,
Lippe. Stimo che venga dal Tedesco /uppen, che vuol dir Brodo di carne,o ¢'al-
tro, che si quoca leflo. In questo fen/o una sorta di mineftra chiamiamo xuppa,
Lombarda., Vedi sopra C.2. stan. 7. Ma l'uso ha introdotto i) dir corrottamen-
tezuppa,¢da molti ineuppa; come 2ulfa,€¥ézz0, e zinfonia in vece di solfa,
C2z0 5 | ia, e simili.
 GRACCHIARE. Dilcorrer senza proposito, o profitto. Da Graccio Latino
us. Utale mi chiefe dieci scudi in presto, ma 10 lo la(ciai gracchiare. Ve-
10 C, 7. stan. 59. e C. 8. stan. 65. '
"il nanni, ed il il corrive, Finger& corrivo, goffo, semplice, baseo.
MATT alla Sanefe. Si dice Sane/i Matti, ma in effetto son pil fagaci degli al-
tri ene Matti alla Sanefeye' han sempre mescolato del cattivo; cioè dell' aflu-
fagace, ed ingegnofo. '
fece di quelle con I ulivo, Fece delle (cioccherie grandissime. In alcune fo-
a » suole la generosa pieta del Serenifs, G. Duca liberare dalle carceri aleu-
icon pagare il loro debito, o parte di eflo;¢ guchti tali vanno procef-
tea render grazie a Dio al Tempio della Santifs. Annonziata, o di S.
10: | 5 e quelli che hanno pagato tutto il debito, e (ono affatto liberi por-
tano in mano un ramo d' olivo a distinzione di quelli, che per non haver pagato








tutto il debito, ma parte di esso devono tornare in carcere, i quali nou hanno
Volivo in mano, ma son legati. Da questo ramo d' ulivo, che ia tal congiuntu-

rm to inter, credo che sia nato il dettato; La tal cosa e com l'uli-
%, che significa cosa grande nello stesso modo, che i Latini dissero pa/maris, ed
f un' azione ardita, che diciamo anche marchiana; da pigliar con le mole,
€,, Come s' intende qui, che vuol dire, che questo fece cose grandi, ed ardite.
DAR nelle girelle, Impazzire. Vedi sopra C. 3. an, 43., e fortoC.9.Man.10,
MANDATO in Piccardia caldo catdo. Impiccato subito preso senza far pro-
Gelso: Calde aldo subito, e prima che la cosa si raffreddi. Piccardias,

“in ipl ardore criminis, Provincia della Francia, serve, scherzando con la.

Amulitudine della parola, per intendere émpiccare. I Latini pure havevano
MN termine coperto per fare ome impiccare 5 che era diteram longam facere,
oy cz come




— 2° Rey: Cena are




204 MALMANTILES / >

come si vede in Plauto; il che ha data occasione-a molti letterati didilcorrere per

chiarire qual fulle questa lettera  e Celio Rod. leet, Ant:

chinde, che fufle il T maiuttolo, che dGmite alla forcay che facevano

Noi ancora diciamo': Andare * Lamia chore un Porto in Toscana' 'te ae

ligno, ciod a fune'y ¢legno; Dar de' calei wl vento: “Ballar incampo:

C2, fan, 65. Ballar 'nel Paretaio web Nemi (00 Cy an, yor B tute:

no Bfer-impiccato. saath t

@UINC HIONE; Da imitichiet detto sopra in questo a stan. 15.

SE ne feantona, che non par fio fatto. Se neva via e ee ean

questo per andarfene, E fete quell' Agere se di Te ih

CORRER a rompicolo. Cortet velocemente; € 3 precip iio senza wonbiderates —

la strada buona, o cattiva,
ARRISCHLARE un capresto, Avventurare a essere impiceato, Corre it

sto il rischio a' andare in fu le forche, che aa di morir di fame.;






Ss oom see oe

ss














MENARSIU A...... Perder il tempo senza far nulla. Se vuoi intender
ne questo.detto, leagi difeor(o d' Anibal Caro in tin di Ser' ee
STANZA XXIX. TANZA 1!
Lasciam costoro, e vadan pure avanti Danae e i guerriero, € via'
oe il vitto Li per quel contornos Cavalcando ne va con fefa, &

Che se fame glicaccia ye "son poi fanté Ognor tenendo tt chirarrino in
Da batrersi ben ben Seco in un forno: Perché il viaggio non ¢1 a
Perched! un era guerrier conit ch' io cati E' bravo sh, ma'poi buon
Ate2z0 impaniata, perch'eglihad'iterno E farebbe servizioinfino
Vasa donna firaniera in veffe. brane, Venga chi vuol, a tute
Ches' afjligge, e si duol della forenna Se bene fuffeit Carat FB svecchit
STANZA XXXL ' a
Poiché bella e colei che si dispera Percio convifo ar, sana cera
Sempre piangendo fens.' alcun ritegno y Par unEbreoe® beth ee
E vanne,come to diffi, in cioppa mera E as quanto l'affligge,e [a trave
Ler dimostrar di sua mespiziail segno, lagrille il Campion quivh
11 Poeta lafeia il discorso di quegli afamati,'¢ si mette 4 narrare Ja' fav
vettita di Pfiche; la quale chiede aiuto 2'Calagrillo, che e Carlo Galli
di Cavalli' 5 gli racconta'i suoi travagli. -
SON fanti, $? intende son huomini c' hanno cuore, € 'pirito da 'fare quella
tal cosa; e da pigliare ogni risolazione.
Dat battersi ben ben seco in'wn forno. Da combattet con la'fame' <>
un. forno pien di pane, © mangiandofelo » vincerla', e facia
MEZZO impaniato., Ubrogtiato; Intrigato: Trastato da | alin













SPLF2F RESTS Een eee EE














vendo toccata la pania, volano'st, ma con Uifficulta per tim
. loro la*pania y.che hanno sopra alle penne.' +e
BVON papricciane. Huomo dolee, groffotano,"huomo alla bubint Pisfriccia:
wo specie di Pastinaca. 1 dettorantico e Buon pasticcione, cioè eee:
Placidus tanquam aqua filens.
BATT I feravecchio, Molti vogliono, che si dica il Bratti ferra'
le fa aa haomo facultofo, ma di cattiva fama: Costui lascid poi tutto



2 meric 3 sana


SSSRew

aS S58 oe

es tS SS

a
3

ee &

QVARTOVCANTARE. 20s.

C di fecolari intitolata in.S, Giofeppe, perché delle renui-
te fine'; come segue fino al. di d' hoggi;:ma.a me pare,
io hia dire il Bares; e il Batti, ciod i Bactilani s quando noo poilo-
no pii rare "altfa arteyy si mettono.a fare. il rivendirore di
cenci,¢ ferri vecchi » e-dall' andar gridando per ia Gitta Chi ha ferri vec)', han-
ito il nome ! divFerraveccine ~ E perohé queste sono vilissime persone aed
lle quali si ha poco riguardo; + quando vogliamo esprimere, che uno sia di maa-






fueta,, ed umil natura, e indifferente.con tutti, fogliamo qualificarlo con questo
termine | Salute' yofarebbe fernixio, anche al Batti ferravecchio, Che se dicette il
calzerebbe tanto bene; pera finalmente il Braeei, fu persona di qual-
che siguardo', © Civilta... Zmbratra soprannome trovali ne) Bocc,
RESACELE E' nota la favola di Ai pedecleriita maravigliofamente da Apulcio,
il Poeta incalftra in questa sua Opera, e l'immaschera assai aggiultaca-

(Pio aren. Vito aspro sche denota dolore, o altra paffione travagliofa Lar,

TMITT Aces, era, Haver brutta, o cattiva:cera vuol dire Faccia, che dal suo
indi¢hi poca fanita, o grave disgufto., che travagliando I animo,
il corpo 5 E bratra cera yuo) dit ancora Fifonomia cattiva.

un' Ebreo c' habia perduto il peguo. Quand' uno per qualche disgufto mo.

i¢a ci serviamo di quefio detto, perché o sia vero, o sia no-

fra opinione'y rarisiimi sono gli Ebrei, che habbiano faccia allegra; ma un' E-

breo che habbia perduto i} pegno aggiunge melanconia.a malenconia, e pero
mostra facia. deformatidima.

SPANZA XXXII STANZA XXXIIL

Fn vg incta ) devi sapere E quando poi io 'bo bell, € trovato,
wr bel marito, ma perch' io Martinarra, chre sempre lo Scompiglia,

Die eis era contro al suo volere Fa si. che pur di nuovo m'é è foappato,
er fet anni n° ho pagato il fio; Ed in mia vece.ali' amor suo s " appigtia,
aallor per farmela vedere Tal ch'io rimange cacciator seraziato;
Stigtate meco fer' andò con Dio Scnoprolalepres.e un' altropoi lapiglia
a lingo; ebea volerle ritrovare Ti dico questo; perchi hanrei voluto
trier volea da ene e Che tu mi deffi a raccatrarle aiuto.
STANZA XXXII.
a ie pomaree ginra sch il marito Edd ella lo ringraziaye del, Lfeguito
' pero non si feomenti. Di sante sue fatiche,¢ parimenti
“Ef ndpteed « quel che ha smarrito, ( Fata più lieta per le sue promeffe )
¢ [ei bisognando,e dieci,e vents. Così da capo.a raccontar si messe.

Psiche espone a Calagrillo il suo bisogno,e lo richiede d'aiuto; Ei glielo pro-
mictte 5 ed ella fatta alleere per tal promessa, incomincio.a discortere » Narrando
tutte le-fatiche 7 e difagi meee -da lei in ricercare del Marito.

WV-HO pagato il fio..N' ho pagato la poms 3 ¢ibLat. penas.dare, Fio & voces
Fiorentina antica, che ant dir fexdo. Gio, Villani lib. 5. cap. 1.Scomumicd Fede-

Afoluette tutti li suoi Baroni da fio, e (aramento, ec, ma da noi hoggi nor.»
ee eneiene 3 nel qualc anche iusd Dante Purg, C, 10, es


206

MALMANTILE |
Di tal superbia qui si paga il fio. >»

Ved os Bey aA

Cl ebleva la carta da Navicare, Bra impomibit ritrovac sgun bag. fone haves,

la carta da navicare, ola buffola.

ZL! HO bell'e srovate, L ho già trovato. Vedi sopra C. 3. fan. 14 a foraa di

questo addiettivo bedo in questi termini.

4M' HA feartato, M' ha rifiutato. Traslato dal giuoco delle carte, 'che ae

do una carta, che habbiamo in mano non fa per noi,la buttiamo s
delle carte; il che si dice scartare, vedi sotto es i
e4 RACCATT-ARLO, Cioè ritcovarlo, riaverlo, ricuperarlo. Il,
significato di raccattare e Ragunare, mettere insieme »:
NON si szomenti, Non si perda d' animo, non si sbigortilca. Petr. ete vila
E fol della memoria mi sgomento.
Dante nel Purg. C. 14. in significato attivo. '
Cacciator di quei lupi in fu la riva 2 e ih




—
. fan, 6. alla voce

Vedi sotto C, 1





i 65

Del fiero finme, e tutti gli sgomenta, * gh
SMARRIRE. E' un certo perdere con speranza di ritrovare. Dan. Inf, C, rE
Che la diritta via era smarrita vide

QVATTRO, fei,¢ dieci., e venti. Scherza facendo,, che Caria proses

più di quel ch' è richiefto., come fanno tutti i bravazaoni,¢ in tanto
a una bella donna non mancano mariti.

STANZA XXXV.
Cupido è la mia cara compagnia,
Riccogarzon,feben la carne ha ignuda,
efnzi non è, t' ho detto una bugia,
Perch'ei no' mi vol piie covta,ne cruda,
Ma senti pure, e nota in cortesia:
Quando la madre [un ch'era la Druda
Del Fiero Atarte,ideff la Dead amore
Gravida fu di queffo traditore;
STANZA XXXVL
Perch una srippa havea, che conueniva,
Che dale cigne bomai le fulle retta,
Cagion ch'in Cipro mai ai cofausc.vs,
Se non con i braccierijed in Seggetta,
Pur sempre co. gran gente, € comitiva y
Com' a Regina; com' elle, 8 aspetta,
i J paesi ha dietro,e gli fraffier dinanzs,
E dagt inlati due filar “2 Lanzi.

Se non ch e miei m è pase
Mio padre ch i i ne lo scanna y
Con un mio Zio ch' andava zientey

i Cathy

TANZA





STANZA XKXXVIL )

Essendo cos: fuori una mattina se
Per suot negorzi se pi
Vito per caso nna Vacca Trenina >

E tocca a pena, in terra la
Ond' ella un alta rammanting >
Perch'unalinguaellhache ragliase fede:
Va, che tw facia, quando ne sia otta
Vn si Leos ice) in forma a wna betta
ZA XXXVIIL
E ae oh? in-vece d'un bel figiia
Di suo gusto, edi tutti i Te ARAB
Vn rospo fece come un pan di miglies
Cc ee fatto flomacare i cant;
Che poi cresciuto, feceft consiglo
Di dargli un po di moglie ma i merzant
Non Pela pee
ae ee ne voleffe, ofentir nulla
Sperando tutti tre d' ungere il dente,
E dire: O corpo mio fatti cs >
E riparare ad ie lor co

Me gli oferiro;e fecefi impiafiro.

E un mio fratello anch'e
eect Psiche a Calagrillo la dolorofa storia, e facendosi dalla nalcita si

Cupido dice, che nacque in forma di rospo per la maladizionc d' una veschia»€ i
chy poi cre(ciuto fu a lei dato per marito.














Sis cS ee ee 6 ee

egg?

(FEZ EEFE SHEET

SEH 6

£.




QVARTO CANTARE: 207
een eas Hida Nealeffo, nea rofto. Non mi vuol pi ia,






yl Tr. lib. 2. fan. 42. dice:
) Nom gli volle annafar cradi, ne cotté
'DI « Innamorata, tanto in bene quanto ia male; perché si dice amante,
4 » damo, non sempre in significato difonefto.
Da C.12z, Dentro vi nacque L amorofo Drude
RIE j: Della fede Criftiana il S, Atleta, Parla diS. Domenico.
Se bene nel presente luogo s' intende Meretrice, concubina.
| CIGNE. Sono firiscie di quoi, od' altra materia adattate a foNenere, e te-
"ere insieme qualfivoglia cosa, dette cigne, da cignere.

 BRACCIERL, Coloro, sopr' alle braccia de' quali con una mano s'appoggia- -

no le Dame andando a piedi per la Città.
 DAGL inlati, Dalle bande, da i lati. Idiotifmo usato assai in /ati per lati.
LANZI, Così chiamiamo ii soldati Tedeschi della guardia pedestre del Seren.
G, Duca. Vedi sopra C. 1, stan. 52.
VACC-A Trensina', Così chiamiamo certte donnicciuole poco honelte, sfac-
- Giate, ed ardite, che non portano rispetto a veruno; € credo che si dica così pez
militudine, che hanno con le Vacche di Trento, le quali per esser' avvezze a
sempre per le campagne del Tirolo, (ono faluatiche,  fecoci.
— Re IANZINA. E*\o stesso, che rammanzo detto sopra C.1.st. 52.,¢ che
mbbuffo nel med. C.t 39. Da alcuno e definita così: Riprentione fatta con parole
; Minacceyoli, e ingiu:iole.. Forse dalle dicerie de' romanzi.
oy yep ieee ae Ȣfende, Ha una cattiva lingua, che dice ogni sorta

|






; 'male senza rispetco, o riguardo alcuno, che lacera  altrui ripurazione.

4 HAVREBBE fatto spomacare i cani, Così sporco, e nefando, che havrebbew

7 provocato il vomito fino a i cani per la sua schifezza. ia questo feafo i Latini
“pure fiferuivano del verbo fomachari.

s  - | DARGLT un po di moglie. La voce poco e usata da noi in diverse maniere; 0

w  declinabile, che significa quantità, come dureg/i un poco di carne; O indeclinabile
Der % io; come andare un poco a Roma; Dategli un po di moglie, e serve per

- eufafial discorso, e non per quantità, potendosi dire andare 4 Roma: Dategli mo-

flit, che tanto esprime schza la voce poco, la quale però nel presente luogo non
tipithezza, o (come diciamo ) borra; ma € così detto per mostrarne l'uso,

. che appresso di noi e frequentissimo, ma nel caso come il presente è tanto usato,

6

*

%

'

w —'tillima da noi in questa, ed in altre voci enunciate sopra C. 1, ttan, 36.

oh | MEZZANT, Senfali. Coloro che sono mediatori a conchiudere ogni sorta,
y afare,

AL bifegno ne lo Jeanna. E? poverissimo;muore di neceffita; la voce feannare,

a Sula da noi per esprimere an foverchio desiderio di qualfivogtia cosa, se bene il
a Myo più proprio è della fame, come s' è veduto sopra in questo C, stan. 24,

' PEZLIENT INTE, Povero, che chicde limofina. Deriva dal Latino perere onde,

h, povere pexriente vuol dir pauper petens cleemofinam; ed € lo stesso che povero in can.

 %4, quasi ignudo come una canna; altri vogliono, che quello ixcanna sia una fo-

iy! Waparola, e voglia dire imcannatore: Che quando un' huomo si metre a incanna-

ee; re





'che non pare si possa dire altrimenti. Quel po per poco € la figura apocope usa~ -






208

re, &segno, che e miferabile
4] Varchi Stor. Fior. lib, 124:
grnrzolato, e diventarone poveri
Jogi, per guardar sempre it Cielo ¢
VNGER il dente; Mangiar roba, che unga il dente come carne
sempre pane, come son neceffitati fare i mendichi; e vuol dire Far
mangiar un po meglio. Wee OE ied dos sana
DIRE al corpo: fatti capanna « Haver tanto dam
pregare il Cielo, ia diventare il suo corpo capace quanto.
riporre il soe ir C.
tanta roba. Viiam questo termine quando veggiamo. uno avvezzo.a
ramente, e che si trovi poi a un banchetto Jautifimd.) 6 obasbe
SI fece limpiafiro, Cioè x accorde, si conchiufe ii negozia.. s
STANZA XXXX. 9 STANZA XXXXE A
Fu volentier la scritta frabilita, Saggiunfers di tui mill aleve bowxey
To dice fol du lor, che fan pensiero Ata quandoda me poi lo
Di non havere a dimenar le dita y:
Ma ben di diventar lupo ceruicra 5
E,, perché e' son bugiardi per la vita. y
Dimoltrano a me pai il bianco pel nero
Dicendomi, che m' hanno fatta sposa Ogni volta con mio maggior dolare.
D! un giovanetto sch! e st-belta cosa. Sentivo darmi ana frvccataalcueres
Psiche continova il racconto, e dice, che finalmente-fu-conchinfo db parenta-
do fra lei, e il Rospo figliuolo di Venere. Z 1 be
ST ABILIT Ala ferieea, Fermato, e conchiufo il contratto del Matrimonio,
che appresso di noi fi'dice La scritta del parentado. oma
NON haverea dimenar le dita.Cio' haveraviver stza iavorarejatan dau
DWENT-AR lupo Ceraiero, Divorare, mangiar yoracemente ycome fail Lupo |
ceruiero. Plin, J. 8.c.22, de Lapis dice così: Sunt in eo genere qui Cornary v0~
cantur, qualens è Gallia Pompei, Atagni arena Spectarum diximus 5 buic ae
fame mandenti si respexit, oblivionem cibi furrepere ainnt digrefumque quarere alind »
£ da tale agonia di-mangiare s' aflomiglia un huomo, che mangi sonaednes,
ad un lupo ceruicro.” 23 ices
BOZZE. Intendi bugie, fandonic, trovati non veri; finziont » )
Quando non vogliamo credere qualche novita, che ci sia raccontatadiciamo:
fot ho per bozza. Traslato da iPitori, che dicono bozze, € abbensare Gace), 1
prime pennellate, che danno in una tela, e gli Sculzori quei primi colpi, che>”
danno in un marmo, O altro; i quali additano un non so che del vero
faranno col finirle.. Vedi sotto C. 7. stan, 5. 3 i
'MI casco le braccia, M' abbandonai; mi perdei d' animo; mi sgome
4 STANZA XXXX1 HGS
Now lo voleva; pur miv' arrecai Quando più
Veduto baxendo i tc

sip este
Ma perché non è il Diaual sempre mai
Cotanto brutto com! eglié dipinto,

































































QVARTO CANTARE: ty

'A XXXXIIL STANZA XXXKIV.

un bel g ' E perché quivi ancora haurd paura

Chrionon vada a stucbargli il (uo riposo,
'\uHlaurd sopr' ad un monte fepoltura,

Che mai si vedde ul più precipiteo,
Ed alto poi così fuor di mifura,
Che non v'andrebbe il Bartoli ingegnofo;
Oltre che innanxi ch'io vi possa gingnere
4 Ci-vuol del buono,e ci [ard da ngnere.
forma d'un bel giovane,ia(ciata la fozza figura del
jalei fa comadamento, che di.ciò in maniera alcuna non parli,perché altri-
ficendo; fara-cagione, che égli la jasci, e f¢ ne vada in luogo da non po-

trovaro.

i. Condescefi; acconfentj, mi v' accomodai; vedi in questo Can,
preso per accomodarsi col corpo;¢ qui e preso per accomodarsi con,































a ee

—

partite vinto. Veduto che la cosa haveva a andare in quella guifas;
dia diversi significati: perché vuol dire Servtinio, che noi corrot-
n0/quitrino, Vedi sotto Can, 6. stan. 109., e di qui Viffe il partite
dire Visto, che il negozio era stabilito così, perché quando il parti-
il negozio §' intende stabilito. Adetter il cernello a partite, significas
metter in dubbio uno se deva fare, o non fare una tal cosa. Donna di partite vuol

ice.'Si piglia'in vece 4° accordo', patro, baratto, o condizione, Io vendo
4 col tal partito, ec, Significa ri/oluzione, o determinazione. Io ho preso

=

Rs. Ps -s =

wm i ine. Significa termine, pericolo, 1] tale ficonduffe a mal parti-

4 MO, Clot a catrive termine, o 4 pericolo di vita, o powertd. Ci serve per espri-

ch, Mer maniera, modo: lo non vi verrd a partito alcuno. Significa rimedio, e/pe-

i? dente. Prefero per partito di fegargli la gamba,cc.

A “AL Diaval non e brurto con egli e dipinto. 11 Maleinon è poi sempre tanto,quan-

ed taro.

f ila gola., Immerfo nelle disgrazie. Vedi sopra C. 2. stan. 44. il suo
; AQVATT R cechi, A folo a folo, Remoris arbitris,

ist 'Sila »de'farté mia', Non voglia saper più nulla dime. Tratto dall'an-

g@ 'ico, Come.si vede in Pilaco, che col lavarfi'le mani pretese di non haver, che fa-

gh tenella Sentenza data contro al nostro sig.\ Giesi Crifto. Ii Lalli Eneid. Trau,

C4. stan. pa.

oy wid E mi lavo le man de fatti tok

s AL Bartel ingegnafo di Bartoli, che ha stampato un trattato dell'architettura,
perch diceingernofo cioè ingegnicre, che appresso'di-noi vuol dire Architecto;
 €non Bartojo legifta(come si trova in altunr tefti,,dove dice Bartolo, enon il

«® Bartoli »»perché tratandosi di falire un luogo erto può giovarpil i sapere d' un'

oe itetto:, 'che quello d' un Legitta. ' SE Stetson

tl Ol val del buono. Ci fara molto da faticare, o da spendere, o da camminare,

, — @simili, servendoci questo termine per — tutto quello ci possa ates necet-
Ss fario




eee

aio MALMANTILE )

fario in uno affare, secondo la fubietta materia, come per esempio: A feriver
la presente Opera ci vuol del buono, € s' intende ci vuol molto tempo, moltas —
fatica, molti fogli, ec. ed è lo stesso che ci farò da mgnere. Ll che viene dal me-
dicare i feriti,e però-per lo più s' usa in cose di poco gutto,e fastidiofe, per efem-

pio: Ll tale amaiazzo uno, vuol haver da ugaere,cioè vuol hayer

molti trava-

gli, (pele, difficulta, ec. ad aggiustare il negozio. 11 Mureto lib. 9. cap.13.Var,

cur,
STANZA XXXXV.

Pos ch' una firada trovero nel piano y
Che veder non si puo già mai la peggiay
Poi giunto apie del more alpeftre, eftrane
Con due uncini arrampicar mi se
Menado alt'erta hor l'una,por U'altra mandy
Come colui, che nnota di spalfecgio y
Ed anche andar con flema,e co gindizio
'S' io non me ne vogl' ire in precipizio,

let, dille; Non parna & panca, fed multa & magna ad hoc efscienduns req

STANZA XXXXVL
Scoscefo è il monte in somma, e dirupate y
Edl viaggio lunghissimo, e diferto, —
Così disse Cupido (mascherato,
Dopo civt ch' ei mi si fu feoperto;
Ond? io promeffi di non dir mai fiato y
E che prima la morte bauria foferts,
Che tra(gredir d'un pittoin fatti in detti
Afuoigufti,alucicenni fu

Cupido accenna a Psiche parte delle fatiche, e travagli, che ella havra nell
andare a ricercarlo; e Psiche gli promette di non dir mai nulla a nefluno,

VNCINI, Strumenti di ferro adunchi, ed aguzzi,, servono per a
gualcofa, e si fanno anche di legno per uso di corre frutti, e per altre occorren-

ze ruftiche.

eats

RAMPIC-ARE. E proprio dei-gatti 5 e d! altri animali simili, che falgono

fu per gli alberi, appiceandosi co' rampi, cioè con l'ugna delle
sotto in questo C, stan. 68. E ci serviamo del verbo rampicare per esprimere uns —

. Vedi

che falga in qualche luogo difficile, ancor chedo faccia senza rampicare.) Vedi

sotto C. 9, stan. 25.

tere
NVOT A di (paffezgio, Nuotare di spaffeggio diciamo quand? uno essendo tut-
to nell' acqua dalla testa in fuori, cava fuora di essa un braccio per volta ordi-
natamente, battendolo sopra all' acqua per romperla, e (pingersi avanti.
NON dir fiato, e non fiatare. B lo stesso che non parlare. Vedi sotto C.6. st
42. Si dice anche non alitare. Non far verbo, Berni Orland. ae
E ferra piit fiatar mi.Pava chiasto, Vedi sopra C, 1. fan. 10.7
GVSTI, cenni, precerti, In questo Inogho hanno tutti tre lo stesso significato
di comandamento. Considerandosi gu/o per il meno stimato, cenno nel
dJuogo, e precetto pet lo pi stimato, denotando dominio.

STANZA XXXXVIL

We tal cosa a persona haurei feopertas
Perché rusravia ta gente soiocca
Riden del rospo, e davami ia berta;
Ed io, che quand'ella mi Venne in coca,
Won fo tener un cocamero all'erta,
MMs lasciai finalmente uscir di bocca,
he quel non era un rospoyma in esserta
Yn Srariofo y ¢-vago giovanesso »

STANZA XXXKVIIL

E che y se lo vedeffon poi la notte

“Quandin camerd messin
Exgetra via la feorxa delle bette
Chun fole proprio par Sputate »
Le male lingwe forse Mparian chiettes
Che si doteiars Se
Pero che now si pus tiraran peo
Chil comento ay veglian fre ree *




sn

S@eenvwris =

anes

ear

ES SsHAewvaesak 7a e FES AF ES

wa



eae




ee

= 3

,

a

“

QVARTO CANTARE: zit

- Vinta Psiche dalla collera, che le venne per esser burlata dal' altre donnes,
Scoperle il fegreto 5 E nota che PAutore mostra il costume delle nostre femmine,
e quelle di tutto il mondo, le quali obligate a narrar qualche loro mancamento;
si fanno dalla lontana, e “ rfaadere d' haverlo cx i xo
forzate da' maggiori mancamenti d' altri.
pe cme berta, Mi davano la burla, mi beffavano, mi minchionava-
no, Berta si dice 3 col l¢,impernato sopra i pali, si fanno le paliz-
7 Patieead barcode fares i pe via di corde se ras ensets » che (as in
detto ceppo. E il Latino irridere, Raccontano le nostre donne, che quel fagace
villano nominato Campriano, del quale diremo sotto C, 11. stan. 48. cffendo ve-
nuto in mano della giuftizia per le suc cattive opere fu condennato a esser mefio
inun facco, e buttato in mare; In cfecuzione di che fu meffo dentro al facco, ¢
oC to a i famigli, che lo buttaflero in mare. Nell' andar costoro ad efegui~
re ini imposti furono per strada affaliti da alcuni mafnadieri,i quali si cre-
derono, che in quel facco fufie roba di valore; onde i famigli per scampar la vi-
ta lalci ivi 11 facco. con Campriano, si fuggirono, Campriano piangendo
fidoleva della sua disgrazia, il che sentito da uno di quei ma(nadicri gli doman-
'db perché piangeva, ed a qual fine era flato meflo in quel facco, Il fagace Cam-
'priano gli rispose; Io piango di quel, che altri gioirebbe, ed &, che questi SS.
voglion rmi per ie Berta unica figliola del Re nostro, ed io non la voglio,
conofeendomi inabile a tanto grado, per ¢fler' un povero villano. E perché efi
dicono, che se ella non si marita a me, l'oracolo ha detto, che questo Regno an-
“dra fortofopra, m' hanno meffo in questo (acco per condurmi a farmela pigliar
'forza;.¢ + oe ela causa del mio pianto. il mafnadiero credendo alle paro-
dicoftui, si concertd con i Compagni d' andar' eflo a pigliare questa buona.
fortuna, e ripartirla con essi: onde fattofi mettere dentro al facco da Cam-
> che non restava di pregarlo a volergli tar del bene quando fufle poi Re,
'feceallontanare i compagni, e ferratolo entro al sacco, stette aspettando, che
titornaflero coloro, i quall non flettero molto a comparire con nuova gente, es
veduto quivi il facco abbandonato, lo riprefero, ed essendo vicini alla riva del
mate, velo precipitarono, e così sposarono a Berta i) balordo mafnadiero. E
di qui venne dar la berta, o /a figliuola del Re, che vuol dir burlare, minchionare.,
come habbiamo accennato. Si dice anche dar /a madre d' Orlando, percht das
alcuni ficrede, che la madre d'Orlando Paladino havefle nome Serta,
— QVAND' ella mi viene in cocca, Quando mi viene in proposito di dire. B si di-
ce anche ella mi viene in cocca per intendere quand' io entro in collora, come s' in-
tende nel presente luogo. E cocca diciamo quella tacca la quale e nella freccia.
per adattarla in fu la corda dell' arco da i Latini detta Crena, donde poi diciamo
eee eae 3 © feflura, che è nella parte ae alla punta dell' ago da
cucire, dal Gr, e-cocche; effremitd acuta, Dan, Inf, C, 12.
a cere, Chiron prefe lo firale, e con la cocce
oh; Fece la barba indietro alle mascello
TUN Etiiee na caccinnre all erie a Dioogee far di meno di non la dire. Si
Eft comparazione al cocomero, perché essendo questo di, figura sferica., e»
lilcio, facilmente ruotolando può scorrer gil re un' erta 5 © monte, e facilmen-
War 2 te

Oh
























212 MALMANTULE) 5.

te può esser anche tenuto fermo; onde molto -ben. si-dice Non fa tenet wn ¢
mero ali' erta d' uno che sia facile a palelare:un fegreto y che co ug
potria tacerlo,: «. claticwn stitrranity
PRETTO sputato. Similissimo a lui: per appunto come jui,e senza ale
ne alcuna come è il vino oe 2 eae —- een 3
juclla aggiunta di /pacato si toglic da coloro, che pigliano:le-m cl
eon ae éninale » i quali in qualche Sesion per andi
punto sogliono tirare il filo 5 e sputandovi sopra lasciano cascar.
Parte, che gli è sotto,e da quello conoscono se il lavoro e per appunio,
CHIOTTE. Chere. Voce Fiorentina, ma poco usata fuor di schet
ne, come poco sopra s'é visto,! usd il Berni nell' Orlando E senza pits
frava chiotto tO Homa} onus
ST danno piato de' fatti d' altri. Gli danno pensiero; Gj sono a cuore
altri, Si mecterebbero a litigare per i fatti d' altri; Che Piato yuo)' d
Vedi sotto C, 7. stan, 27. tb orng
VON si può sirar un peto ec. Non si può far una cosa bench minima,
popolo non vi voglia far sopra i suoi discorsi, Anh sole t
STANZA IL... STANZA Lhioy




Le ciglia inarca, e tien la bocca feretta
Chiunque da me tal maraviglia ascolta;
Ma quel ch' importa afordono fu detta,
Che Vener, ch'ogni cosa havea ricolta,
Per veder s' elle verayo barzelletta,
Poiché a dormire ognun fel' era colta,
Entra in camera,e vien pia, 'Piano al letto,
E trova il tutto appunte come ha detto.














STANZA L
E nel vedere in terra quella Sposigla LVon tivo dir com io restaffi allora Tya,
Che per celarsi al mondoil giorno adopra, Che mi fovvenne subito di a lw
Di levargliela via le venne voglia, 4 primo di mi si fueld, e ancora —
Accs con off più non si ricwopra: Mi fece l espertissimo comando, 2
Così la prendeye poi fuor della faglia Chtin alcun rempo io non ta deffi fora,
Fa un gran fuoco, e ve la getta Sopra 5 Ed io fon' ita scioceaya farne un bands, è
We mai di ti si volle partir Venere = poi mi pare rano,e mi arcoy dy,
Infin che non la vedde fatea cenere, Segli e in valigiased ha copratosiporct. hy
STANZA LIL Uae yeas \
Sospe/a per un pexxo io me ne fretti, Guarda fu pel camin, civo in fai vetti, a
Chi io aspetrave pur ch' ei ritornalfe; etpro cli armarjye a Scoftar le case
A cercarne per A/a poi mi derti We trovanidelo mai, al fin'msi a i
Per le fhanze di soprase per le bafe; Per non fermarmi fin ch'ia no letrove, tay

Il fegreto palefato da siche, venne all' orecchie di Venere » la quale quando
Cupido dormiva gli abbrucié la veste da rospo; il che veduto Cupido la
se ne faggi,¢ Psiche si mede a cercar di lui m

ha,

aca desta a fordo, Fy detta a chi ne fece capitale 4 8 chi importava (a mt
pe Oo « i i




213



|, Haveva sentito', e inteso ogni cosa ?

Cosa non vera,ma detta per scherzo. E si dice Barzellet-
'lando, e scherzando.

fo termine, che vuol dire Adagio adagio, significa ancora

c }) Senza far punto firepito, o romore.

'LE. Piccolo piumaccio, sopra il quale si posa la guancia, quando

; “rn mame guancia, come in diversi luoghi, e detto origdie-

Rea Ie Oiler

 Rivestirfi'da rospo. Ecco la voce generica animale, che noi

le, come accennammo sopra in questo C. stan. 4.

dire. E? lo stesso termine, che pensare voi, vilto sopra in questo C.

one voglio dirlo, perché da per voi vel' immaginerete 5










fuora, Non la manifeftaffi, ed io n' ho fatto un bando; ed io Y ho
tutto. Won modo tubam, fed etiam praconem adhibui,

Scontorcersi e proprio delle ferpi ferite; e parlandosi d' huomini
n certo atto, che denota dolore per qualche disgufto, o travaglio in-




valigia, E' in collora, in ira; Nel bugnolone, nel gabbione, e simili,
ni ne habbiamo in questo significato.
AR it porco. Significa andarfene; ed e come l'interpetrazione di /ui~
3 quasi voglia dire fuinam, cioè fuillam emere, o che più tosto sia detto /ui-
si feappar via dalla viena, e fugcirfene, come quei che sOn colti a coglic-
we uva nell” altrui vigna. Diciamo batrere sf raccone, batterfela, cor-
je se ben son voci, che hanno del furbe(co, sono però comunementes
pre intese in questo senso. Vedi sotto C. 11. stan. 11.
ANZA LiV. STANZA LVI.
¢ via v0 fola fola y Ripongo la noccinola, e la castagna
ancora una giornata y E rimetto le gambe in sul lavoro
tedsperrami figliuolay Per una lunga, e ferile campagna
dietro veggomi una Fata, Difabirata più che lo Smannoro;
mi diede una noccinola, Dupo cinque ani giuntaa una-motagna,
jee lio, difs' io, d' una faffatra M1 si se inmanzi un grade, eorribil toro,
un' altra faa compagna Che ha le corna, ei più tutti d'acciaioy







mano anch'ela una castagna, Etira che correbbe nel danaio.
STANZA LVIIL.
ei mangiato i faffi E come Cavalier ch' al faracino
'accomodai per darvi fu di morfo, Corre per carnovale, o altra feta, »
ch'io non La spiacciaffi, Verso di me we viene 4 capo chino

gran bifagno non mi fuffe occorso Con la (un lancia biforcata in tefba,
ata di ciò con eli occhi baffi Lo già con le budella in un catino
tai del lor discorso, Addio dicevo al Atondo,addiochirefta.



Seufe,e refe ad ambe eAddio Cupido dove tu ti sia,
le lascio, dolla a gambe, A rinederct ormai in pellicceria,
STAN-



Digitizers: By


aig MALMANTILE' ©

STANZA LVIII,. tigeesbiadecs WT ORS
O Mamma mia, che pena, e che (pavento Pur come volle il Ciele io ini rammite
Hebbe allor questa mezxa donniccinala? Del dono delle Fate, e la nocciuola =
Tremavo giufto come giunco al vento, Presa per caso presto sur un sasso
Che quivi mi trovavo inerme, e fola; La scaglio, ella si rope, en'esce un masso,

Medflafi in viaggio Psiche s' imbatté in due Fate, dall' una delle quali wee
una nocciuola, e dal' altra una castagna, e le dissero, che non le stiacciaffe, f
non aun gran bisogno. Dopo cingue anni di cammino per un deferto arrivd a»
pi¢ d' una montagna, dove le venne incontro un toro con le corna d* acciaio;
dal quale spaventata Psiche stiaccid la n occiuola,¢ ne nacque un maflo,

FATA, Fate sono donne indovine dette secondo alcuni dal Greco Phatis che

suona Donna indovina, e quelle forse che i Latini co' Greci chiamano /ibilies
ma dalle nostre Balie nel contare le novelle a i fanciulli son prefe;
buon genio, e che fanno servizio al proffimo con le.loro azioni, € & contrarie
all' Orco, al Bau, e alle Befane, che foo nimici de' bambini, a i quali questes
sempre fanno servizio, ed il Poeta, col regalo, che fa lor fare a Psiche, mostra
guefta verita. Da gli antichi furono anche chiamate Ninfe,¢ Dee, el'
nel suo Puriofo ciò afferma, dicendo: sia
Queste c' hor Fate, da gli antichi furo Agee
Chiamate Ninfe,e Dee con più bel nome, weil
Di queste Fate discorre ' Autore sotto, nel Canto fettimo, ed è credibile, che
questa voce Fate venga dal Latino Fara fatorum, che Dan, Inf, ¢, 9, disse le fata,
Che giova nelle fata dar di cozzo? A
QVESTO e meglio a una falfata, Quando si riceve da uno qualche regalo di po-
co valore, si dice per (cherzo: Queffo e meglio d' una faffata, o vero a' un calcio di
mosca: volendosi inferire, che da quello, al nocivo, o al nulla vi è poca i
za. Plau. in Tr. disse Afelins est quam deterrimum, E
ALLOTT A haurei mangiati i faffi. Allora havevo così gran fame, che haurei
mangiata qualfivoglia cosa,ancor che dura quanto un faflo, lo crederei, che il ve-
stitore di questa favola havefle seguitato i compositori de' Palmerini, degli Ama-
dis, ed altri Cavalieri erranti, che mai in tanti viaggi, che fanno lor fare, par'
una volta si trova, che in campagna mangiaflero; ma il sentir, che Psiche
scorre di mangiare, e che fu levata dond' ¢ll' era, perché non vi moriffe di fame,
mi fa credere diversamente, cioè che in questo suo iungo viaggio le Pare le em-
pieffero il corpo, che clia non fen' avvedetic, %
SCHIACCLARE, Corrottamente diciamo anche fiacciare, vuol dir Rompere,
6 infragnere, ed e proprio di quelle cose, che hanno gu(cio, come noci, man-
dorle, uova, e simili. i
DOLLA agambe. Comincio a camminare; è lo stesso che rimetto le gambe in [
lavoro, che è nell ottava 56. seguente. I La)l. Eo. Tr. C, 2, stan. 33.
mand' so la diedi a gambe,¢ dentro ann foffo
Lasca Nov. 6. Temendo, che colui non gli uscife dietro, s* usch di casa s pleted, è
la dette agambe,e per la fretta si feordo di ferrar l'nscio, 1 Lat. pure dilfero conijcert
Se in pedes.
ZO Smannoro, Così è detta una gran pianura posta poco lontana per —

Es iae e-em ee







PF See Se



SEs eEee eat peeiiak

&
=

SPREE FFG








QVARTO CANTARE: ary
alla Città di Firenze, la quale dura più miglia per ogni verso,senza mai trovarsi

una casa, se bene è tutta coltivata. Si dovrebbe dire Ormannoro dalla famiglia
-antica degli Ormanni, la quale era già padrona di tutte quelle pianure, che si di-
Ormannorum

A che correbbe in un denaio, Tira così aggiuftatamente, che egli correbbes
Piccolo berzaglio, come è un denaro, che è la quarta parte del quattrino
3 con altro nome detto picciolo, ed un giulio ne vale 160.
iCZNO. Così chiamiamo quella statua, o fantoccio di legno, che figura
0. armato, al quale ( come a berzaglio ) corrono i Cavalieri le lance;
dice anche Buratto, che e ua' altra sorta di berzdglio( il quale si mettev
vece del Saracino ) ed e una mezza figura secondo alcuni', che nella fiailtra.
i¢ lo feudo, nella destra la spada, o bastone; la quale se non è colpita nel pet-
si rivolta, e percuote colui, che falll.
Ld biforcata, Intende le corna del Toro.
CON le budelin in un catino. Mi credeva già morta; Mi credeva già essere sia-
'ta sbudellata dal Toro. Luigi Groto Cieco d' Adria, in una sua lettera al Petr.
dice: Quei cani con il loro bau bau ci fecero parere d' havere le budella in un,
“eatino, E Catino Intendiamo un valo di terra, o d' altra materia per servizio di
ucina, e per uso di lavar piatti, ec.
} A RIWEDERCL in pellicceria, A rivederci fra i morti. Questo è il comiato,
ky

es






GREERFE Ge

a che noi finghiamo, che G diano le volpi ' una con I altra, perché (apendo, che
bm devon esser' ammazzate, e le lor peli vendute, dicono alli lor figli, quando da
fle si feparano:. A rivederci in pellicceria, che così si chiama in Reena quella
jp firada,, nella quale (ono le botteghe di coloro, che comprano, e vendono pelli
oi  Gianimali per foderare abiti, ec. ed in mano di costoro, o tardi,o per tempo
g¢ -fanno che devon capitare.
I O MAMMA mia, O mia madre. Esclamazione di spavento, e di timores,
wata propriamente da' fanciullini, quasi dica: O mia madre foccorretemi in
t icalo.
~ SONNICCIVOLA, Vuo0l dir Donna di spirito minore di quel che converreb-
€ al suo naturale, da i Latini detta Aduliercu/a, Siche mezza donnicciuola vuol
dir Donna quasi da aulla, e senza spirito.
WNCO. Specie di virgulto, che nasce in lwoghi padulofi, del quale si servo-
20 i Villani per legare i cralci teneri delle viti, ec.
444550. 8 intende un faflo grande, Questi nostri scarpellini chiamano il





maffo La cava delle pictre.

 STANZA LIX. STANZA LX.

Tal pietra per di fuoraé calamita Sfavilla il maffo al batter dell' acciaro,

| E ripiena di fucco artifiziato y £ da fuoco al rigiro ch' e nascofo,

Hor mai arriva il Toro, ed alla vita Ea egli a ragzs cb allor ne feapparo

f Con un lancio mi vien tutto infuriato y Vincolpo fatto haver vede a suo costo,
Ata dietro al maffe ero fuggita Perch non vi fu feampo, ne riparo,
. re riman quivi scaciato,. Chrei fra le fiamme non si mucia arrofscs
| CW in fo dando ha ferrata testa Ed iofeanfatoilfuoco,e ogni altro arate,

da quclte celamica afifo rea, Lieta mi parto, o tire innanzs 1 conto,
Ni: se sss; rf






216 MALMANTILEY |

Il detto faffo era per di fuori calamita, e dentro era fuoco lavorate, onde il
Toro perquotendovi con le corna ch' erano d'acciaio vi rimafero eo
da quella percofla nacque il fuoco, il quale 's'appiced allt ordigno 5 'abbrucid.
il Toro. Psiche libera da questo incontro seguito il fu viaggio.) ) ae
CALAMIT A.B: \a pictea simpatica del ferro yo forse madre dai L

detta Adagnes. Vedi sotto C, 8, fan. 45. € 66.

oy logeny!

FYOCO artifiziata, Vuol dire ogni forva di composizione fatta con
(che diciamo Da archibuso ) tanto per guerra, quanto per fefley.
RIMANE scaciaro, Riman burlato, E' lo stello, che rimaner con un Z

nafo, che vedremo sotto C. 6, fan. 5.

mafio.

2o0h Hd

RIGIKO. Intend l'ordigno di fuoco lavorato.y che e composto dentro al

v} of sala,

RAZZ, Raggi di fuoco o del Sole, o d*altro (cintillante..Ma dicendovaf
folutamente razz!, intendiamo quei fuochi artifiziati, che-si fanno in Occalione
di fefte con poiuere d' archibulo conttipata,e benisfimo Jegata entro alla —

dotta come pezzi di canna,

TIO innanzi il conto, Seguito il mio viaggio, Vedi sotto C. 6.stan. 16, Fane

to serviva tivo innanzs, © senza mettervi if conto suonava i) medesimo;

nato da quei, che tengono libri di debitori, e creditori ci obliga a dir così,

» STANZA LXIL

Piit la ritrovo un grand' uccel erifone,
E tops assai, che giran.come PARR y
Perch' egli entrato in lor conver/azione
Gli becca,grafiase ne fa mille [Praxxj y
Di lor mi venne gran compalfione,
E vo per ovviar,ch'ei,non gli ammarzi,
Ma quei mi sete al moto,einpic firizza,
E per cavarsi, vien con meta flizza,

STANZA LXIL

Questo animate ha il buffo di cavallo

Di bue la coda,e in fucte spalleha laley

M capo, e it colo ginffo come sl gallo, «

Li pie di nibbio.vero ye maturale 5).
Gii artigl di fortissimo metalle
Grandi groffise.adunchi in modo tale
Che non vedefti quando leges, o ferivi,
Mai de tuoi di pin bei imterrogacini,
STANZ on
Son? att poie'a far pis acuta
' Seecmaige.aedieeie ighe y
Tal che,s' al vifo fuffinaiwenuto
Con essi, mi lasciava assai più righe
D' un sibro di macfiro di linto,,
Ed una flamperia di falfarighe
Con farmia life come le gratelle.
. Da quocervi le trigtie,e le fardelle,.



STANZA LXIVE ©
Hor os tornare, In quel chia hetimere
Ch' il

mio grifo sia scherze

La cafagna ch! io in + ecio fuore

La rompo, en' esce subite un Ltone,*

Che mi scomo non poco tl barvionore»

Perch egli in mia difefa a lui soppone,

E mostrogli hor con Pigna,ed hor ee dati

Jn che mo si gaftigan gli infolenti.”

STANZA Lky,

L' uccelle anch' eel, che non ha pana

Gei rende molto ben tre per

Ada quel che haver del suo '4

Al contraccambio subito
E ben ch' ei owltepauiiae
+ Liafferraye firinge tanto el
Di poi garbatamente gli iesca
Gli flinchi fu's nodelli, e me gli rech:
STANZA LXVL st
AMetto uno firide ye mi ricive imdrete
Loch he paura aller ch'e: mmm ings »
Ata quegii ie




a Che mai vedelje st

Cio conoscenao sutta egal
Gli lascia.in terrase va perfacti fuck >
Ed so gli prendo aliora, efsemdo certs
Diaverne ahaver bisogno in figrad'erta



























ee a

a

Q

3S

——— eS Re







217
STANZA LXVIIL
piedi, Evconuenne talor farsi da piedi
1 \Bawrendo git di erandi flramazzeoni,
mi Perché non ve dove fermar il pafsa +

| morte brancoloni y Cagion che /pefio mi trovai da ha/so,
trato il pericolo del Toro s' imbatte in'un' uccello Grifone, che ha-
@ acciaio, onde roppe la castagna, en' usci un Lione, che la difefe

|» e tagliandogli gli artigit, li porté.a lei, Ia quale gli prefe, ¢
ttaccandosi ail' erto monte, cominciò a falirvi.
'che girano come pazzi, Sorci, che vanno in'qua e in la correndo senza
ve determinatamente, appunto come fanno i pazzi.
RS/ la fizza. Sfogar la collora, la rabbia, I ira.
0... Vecelio di rapina noto. Qui descrive il Grifone, e lo fa mezzo ca-
mezzo uccello, e con la coda di bue, e se bene da i pi e descritto mez-
€ mezzo uccello, e nimico mortale de'.cavalli, come si deduce da Verg.
rantur iam Gryphes Equis,tattavia non fa errore a comporlo di che be-
iuto » perché questo mostruofo animale in ogni maniera che sia è
fayolofo., secondo Plinio lib, 10. ¢. aan Pegafos ( dice egli ) equino capite
Gryphes Aurita aduncitate rofiri fabulofos reor, illos in Scythia, hos in

BRROG-AT IVO, E} un contraflegno d' ortografia, i} quale si pone in fine
che conchiudono sn » Orichiedere, e weet e detto Punto
E:perché tal contrafiegno e di figura simile a un' uncino, pero a
q igliamo gli artigli degli uccelli, come fa qui il Poeta, assomigliando-
pli a quelli del. grifone,
LI baaeftro di linto, Intendi libro da musica,, che son pieni di righe,afi-
se di icrivervi sopra le note musicali.
PALSARIGHE.. Carte rigate, e lineate dinero, le quali si mettono sotto al
al quale si (crive,affine di far i versi diritti,'ed uguali camminando
0, che dalla faifariga per trasparenza si vede sopra il foglio, ove

PS

LIST]:. Qui vale per firiscette di ferro, 2 le ees composte le gratelle
firumenti dacucina, che servon per mettervi sopra 1! pesce, o aliro.a quocere»
arrofto Econ mute queste similitudini intende, che se ucceliot ha vette meta
fli artighi addosso a Psiche, l'havercbbe malamente graffiata, e segnata.
G. Vuol dir Faccia di porco, o simili; e s' intende alle volte: la faccias
3 gan ame per naneeae » o per disprezzo; € qui il Poeta se ne serve per far
bifticcio rh e¢ Grifone,
x “4 VORE ~Panra 3 tumore. Da quella frequenza di battere, che fa il
a dalla ee del — » quando si ha qualche spavento: 1 Latini pure di,
anitah  velcordis percnffio.
INSOLENTE. Acrogante, fastidiolo, petulante,. Vno che tratta., e proce.
~ per copia « Gli rende pib del suo dovere., perché.a 'render
che ¢la coppia,si ms la meta più del dovere: B con questo
€. modo

jiized le
s
Se sna

218 MALMANTILE.¥

modo di dire s' intende, che uno Gi difenda da un' altro con pare
sempre con vantaggio, che diciamo anche render pane per

NON si cura haver niente di sue, Intendi Non vuol' esser da lui fope

e4f PERRARE, Abbrancare, pigliare stretto; 1 apprehenfam
NODELLI, Intendi la congiuntura delle gambe co' piedi, ©
eANDAR carponi. Camminar co' piedi, e con le mani per terra,
fieflo, che Andar brancolone, che si vede nel verso seguente; se non che qu
vuol dir Salire adoperando le mani, e i piedi; e carponi e camminare alla p
con le mani, e co' piedi, Dante Inf. C, 26. descrivendo una simil falita dice
E proseguende la folinga via *
Tra le febegge, e trai rocchi dello feoglia
i più senza la man non si [pedia, ae
STRAMAZZONI. \ntendi Calcate; che per altro #ramazxone intendono gli

schermitori una specie di taglio.
STANZA LXVIIL

Tusei quei topivia ne vengon ratti,

E furon per mangiarmi dalla fefta,,

Pero che dale granfie io gli hofottracti

Di quella bestia a lor tanto molefta;

Così ve rampicando come i gatté

Suil' aspro monte dietro alla lor pefta,

Sopportando fatiche, frenti, e.guai,

E fame, e fete quanto si puo mai,
STANZA LXIxX,

Pur finalmente in capo 4 due altri anni
Giungemmo al luogo tanto defiato;
Ma non finiron qui mica gli afanni,
Perché di muro il tutto e circondato;

EB qui s'aggingne ancor malea malanni,
Chto trove Cuscioyma'l trove diacciato;
Penssa 8 allor mi venne larapina,
Es' 10 dscevo della Violina,



see
STANZA LXX.

Hora tu sentirai ch' il dare aiutca

A tutti quanti sempre si connine y

Percht già mai quel tempo s'? perdate,
Che dicepirws in foretell,
Non dicofet althuom, ma ico a un britd,
Che forse immrondo, e inutile fitiene y
E che tx non lo fRimi anche una chioft,
Pero che ognuno è buono agqualehecifa,
STANZA LXXL
Setu giovi al compagno, allor tu fai
( Quasi gli prefti roba ) un capi
Anyi talor per poco, cheghaai
Ti rende psa fei volte che non vale.
Ma non Fase io pretender mai,
Perché ell è cosa, che starebbe male;
Quefod un censo il quale achiloprende
Rrchieder non si pwd s' ci non to vende,
facendole




1 topi, che Psiche liberd dagli artigli del Grifone la seguitarono
gran fella, e con quella campagaia in capo a due altri anni arrivé Psiche al luo
go dove era Cupido, che era un recinto di mura, dentro al quale non si poteva
palsare se non per una fola porta, e questa era ferrata, oh Reed
VENGONO ratti. Vengono velocemente:dal Latino rapidus, D, Infer, ©. 21.
Perch' io mi moffi, ed a lui venni ratto Ee

4 whe
Ed habbiamo rartezza,per preftezza, o velocita. Varch, Stor. lib. 4. Za ele j
Lae

70 il sig.\ Sciarra Colonna sens ¢on gran rattezza da Roma,;

FAR fefta anno, Ral

egrarsi conuno. Ricevere, © trovar uno con atti di

amorevolezza:, e cortesia; Che nelle beltie si conosce tal rallegramento da i
fj, come nel cane dal dimenar della coda, ne i gatti dal fregarsi u
ed altri animali dal moto degli orecchi, come forse si conosceva in quei topi.

Lat. adulari fanno yenire alcuni da ad,& wra, che in Greco significa coda quasi fit





















































VARTO CANTARE: 219

ndi falire appiccandosi con gli artigli del Grifone, co.
i. Viene a. che s' intende ugne di gatto, lione, tigre, €
anche snerpicare e ico Mtrumento ruftico da romper le terre.
Franzefi sopra alle maschere dice:

| Nom-vi crediate, che qualunque faglie
Haveffe da [un posta tanto ardire,
Chr inerpica/se sopra alle muraglie

si dice inmarpicare,e annarpicare. Vedi sotto Can. 9.stan,

w
'alla lor pefta, Seguitando le lor pedate,
a icella riempitiva in i er emfafi

ir a 'B'
appunto come i Latini dicono ne quidem » se bene & diferente dal
¢ non s' usera per affermativa, io voglio mica, come essi dicono ero
che se bene e per emfafi ha però qualche parte del negativo, quasi
fo now voglio ne pur' una mica, che vuol dir minuzolo di pane, o granel-
Petr. Son..91. We mica trove il mio ardente defio.
Dolori di cuore, che fanno quasi venire in angoscia, Petrar.





Buy Leh >



ris

Se la mia vita dal aspro tormento

Si puo tanto febermire, e dagli afanni,

GER male a malanni. Al male accrescer male, e peggic.

(0 diacciato. Cio' porta ferrata. Vedi sopra C. 3. stan, 3.

la rapina', Mi venne rabbia, collora, o stizza.. Rapiva vuol dire ru-

quindi uccello di rapina; ma dalle nostre donne è presa in

» per sfuggir di dire rabbia creduta parola peccaminofa, e dico.

i rv arrabbiare, ed arrabbiato

bicevo del male fra me medesimo, perché le cose non

mio modo. Questo so che significa Dir della violina, non so già da.
gine questo dettato, che e lo stesso che Dir l'orazione della ber-

ee

i una Chiofa. Non lo stimi punto. Vedi sopra C. 3. stan. 60, alla

capitale. Metter insieme una somma considerabile di denaro per ha-
a ogni suo bisogno: Si dice anche far un' afsegnamento,
$0. La namra del censo, e che colui, il quale presta danari a censo,non
richieder la somma principale, che egli da, ma folo i frutti d' efla; può ben
steel la medesima somma principale a ogni suo piacimento,
la diede @ forzato a riceverla, come dice il Poeta assomigliando co-
ilpiacere a un' altro, a uno che dia a censo, e dice, che colui che
non dee, ne può pretender la ricompenfa, ma la può bene sperare,
creditore: ow ben dice Seneca de Beneficijs \ib.3.c.14, Vide etiam
crit, nulla repetitio, B lib, 4.cap. 39. Alia conditio

beneficio.

SEE










su Se

Ee 2 ' STAN-

citized bylibgie
*
 yorture di cose di sua qualita, ec.














220 “MALMANTILE)

STANZA LXXIL

Guarda s' el? è così; Lo per la mia
Picea di prender di quei topi curay
Da lor vinta respai ds cortela,
E wt hebbi la pariglia con?) usura y
Peri ch' in quefia xxx ricadia,
Ch' io ho d'haver trovata claufura.,
Eglino tutti sul cancel faliro,

E si fermara, ove e la toppa, ingiro,

STANZA LXXIIL i

E gli denti appiccando.a quel legname, Cupido etmor, che tanti ha
Come s' in bocca haveffero un trapano y Berzaglio qui si giace della
Prefeo presto vi fecero un forame Ei chera fuoco,il nafo
Da porre il fiasco,e vender iltrebbiano, i 0
Tal ch' in terra cascando ogni ferrame
Spalanco l' xscio di mia propria mano,
E paffo dentro,¢ refto pur confufa,
Perch' acor qusvie un'altra porta chiusa, Non farò U urna, che glié qui daca
1 Topi suddetti rimunerarono Psiche, perché rodeudo fino a fette porte, che
erano in quel Serraglio,fecero cascare i ferrami, e Psiche entrata dentro,trovd il
sepolcro @ Amore, e dail' Inscrizione, che in esso eras comprefe quello'
restava da fare. t $

HEBBE la parigla. Hebbi il contraccambio.. B' il Latino Par
Pariglia intendiamo due cose uguali nel giuoco di Carte,,0dadi,, come due!
due alli, due figure, ec, e di tal voce non ci serviamo se non nel giuoco 5 o nel
caso del presente luogo di reader contraccambio si.in bene, come in male }
sotto C, 6, stan, 69. Io l'ho per voce Spagauola, ed il Varchi nella. 8.
} usd ia un certo modo come straniera dicendo.: Dopo efersi vendicari
renduto i contraccambio, o y come si suol dire, la pariglia. y

CON ? xfura, Col frutto. Cioè mi contraccambiarono, facendo
vizio a. me, che non havevo io fatto a loro;:

ZEZZA, Vitima. E' voce antica hoggi poco usata fuor che nel,
Vedi sopra C. 2, stan. 2, Si trova anche/egee y/ezzaiay 0xerraid, >>

IC.ADIA, Noia, travaglio y avversita, moleftia, © simili che vengono\
aun' altro dilgutto; da ricadea, che e quando uno infermo già quaGi fanato, vie~
ne a riammaiarsi, o per lo mal governo, o per altro. Nella storia di Semifonte
Trattato terz0. Con li loro misfatti, dando alli Fiorentini non ic
Sac, Noy. 98. Che ricadiaé que/pa di questi porci?.. t rs 19:0).
CANCELLO. lntende il legname, che chiudeama.porta: ma pro}
cancella diciamo una chiufura di porta fatta di stecconi, orstrisce di legno
ferro feparate l'una dal' altra a guia di gabbia.; ora.gve

TOPPA, Intendiamo quella piaftra di ferro, sopr'alla quale son
ingegni della ferratura, detta assolutamente, o senza aggiunta, perché per alt
Toppa G dice ogai pezzo di panne, legno, quoio, ferro, ec. che s' adatti:
































ia


















Be

EST SPSS SSeS Fo BF

Fee





QVARTO CANTARE? a2r
frumento specie di fucchiello, col quale si forano mate.

tre 'metalli, ec. eel odeeaa

@°, Coloro che vendono il vino a
loro » come dicemmo sopra C. 1. stan. 76, ed oltre a,

più nella porta, o nel muro una finestrella,per la quale dan-
vendono'; a'questa'finestrella atfomiglia il foro fatto da i
0 iano pigliando questa specie di vino per tut-
'intende esser questo tale sfondato simile a quello, che si fas

vendere if
dere il vino'.

Ateneo

ca
MOR 2a.20

che Cupido ¢'freddo, cioè morto

schi, appiccano un fialto

'£. Aprire largamente, quanto si può.
bere un novo. Fu cosa facilissima,come e il bere un' uovo: i Gre-
'in questo proposito Oxo patio quis ovum forberet je trovali questa,

0 agrafio. ipingere-a grattio, sgraffio, o graffito,è un' imprimer
anwterisacee nell' ifesthacdties fee(ca de' muri con deteo ferro,
la graffio, forse dal' antico graphium, che era lo filo di ferro, col

(ONARE; o sbolzonare'. Sacttare, frecciare, da bolzone [pecié di frec-
Mattio Franzefi sopra alla boria dice:
Di qui Amore accorto balefriere
\* “Bolzona qualche giovane galante
lato', Ha il nafo freddo. Pighando la parte per il tutto,vuol dire,

A? Animiale'noto; ma qui si dice una, che chiacchierando assai,non

se fa tener
ree

we

Feta cosa alcuna; e degli huomini diciamo Cicaloni, Appresso
'cicala non sona male, poich? alle cicale sono da essi raflomigliati in più

aa peril continovo cantare, che fanno, e questi, e quelle. EB
questo Poeta graziosamente chiamo Musa la cicala sopra C. 1, stan, 2,

STANZA LXXVL
Non tivo dire adeffo sin quel caso
ro gli occhi due fontane,
“E feci'come thi 2 rotto il nafo,
* Che versail fing ne,e corre al lavamane;
“malin oorer 4 quel vaso
Durante a lagrimar fei fertimane,
¥. il pite voglia di piagnere,
inte ceils miebbi iniricobe
Pret aes LXXViL
Quand io ch' egli era poco meno
“Mn fach Muh ape 4 buon porto,
Valli innanzi ch'e fulfe affatro pieno,
Ech il marito mio fuffe riforto.
Lavarmi it vifo ye raffettar mi il feno,
Accio st lorda non m' baveffe scorto;
Percio miparto ye corro,se in quel monte
Per avventura fuffe qualche fonce.



STANZA LXXVIII,
In quel clt io m' allontano com' io dico y
AMiartinaxxa, che era in Stregheria,
l'fio di la portata dal nimico,
Che non porette tar per altra via;
E perché sempre fu [uo modo antico
Di far pertutto aalcun qualche agheria;
Lefie il pitaffio, (quadro Purnaye tenne,
Che li fufse da farne una folenne,
STANZA LXXIx,
Se qua, dice fra se, Cupido dorme,
Vito risvegliarlo per veder un tratto
“ Sregli ty come F dice, e se conforme
eA quel che dai Pittori vien ritrarto *
Se ben chi lo fa belo, e chi deforme,
Basta mi chiarird com' egli e farto;
Per questo ad emprer mettefi quel vaso,
ef cui poco mancava ad efser raso,

Zed bya

























222 MALMANTILE 5)

2 STANZA LXXX..%
Con l'animo di pianger vis arreca, Al fin si pone a
4a ponza pontia, lagrima non getta Si che per forza a pianger è
Si prova a far cipiglio, e bocca bieca, Onde la pila in Ai

Ne men qucfiaé pero buona ricetta; Refto colma, e Cupido se
In ordine al Cartello havendo Psiche con le sue lagrime quasi piena l'urn
andò a Javarsi il vifo, e raccomodarsi la testa; Intanto Martinazza arriyo:
polcro, e con le lagrime sue fini d' empier l' urna, e Cupido usci dal Sepoley
WON ti vo dire, Questo termine (crue per esprimere. Date puoi ben fa
Sta cosa meglio di quelo che io fapeffi dirti; o vero so che tu hai da per te tant
gindicar come io rimaneffi, senza che io te lo dica, Suona lo stesso che pensa
dica tu,tu puoi sapere, ec, Vedi sopra in questo C, stan. 41. stan. 52,5 e fan,
Simile è quello: Non domandar, se Durlindana taglia.  +p ae
LAVAMANE, ¥ uno strumento di legno, o d' altro, che con tre piedi
ma come una piramide in triangolo equilatere, e sopra esso si pola la catinell
altro vaso per Javarsi le mani. ri;
ERA poco meno che ail' rie, Era quasi pieno. L' acqua arrivava
mita del yafo: che questo vuol dire or/o, che viene dal latino.ora y si
¥ eftremita di qualfivoglia cosa.
LORDO., Schifo, intrif©. Dal latino Luridus. eee
VA in fregheria. Dicemmmo sopra C.2. stan. 11, donde derivi tal nome di Stre-
ga, c¢dal C. 3. stan. 69, dicemmo esser fama, che tali Streghe vadano la notte:
cavallo in sul caprone a Benevento al congrelso de' diavoli. E questo: (
cendo Andare in Stregheria portata dal zimico, che vuol dire il Demonio, in for-
ma di Caprone. Che queste donnicciuolucce credure Streghe vadano in sul Cas
prone a Benevento e opinione vulgata,¢ molti di ceruello debole I hanno per
indubitata, e le medesime Streghe se }o credono, perché il Diavolo con illuf
fa loro apparir per vera questa falfita; Mala graziosa fagacita d' um Superiore
ne fece chiarire tutti i dubbj in questa forma. ee
Fu condotta alle carceri una di queste tali inquifita di maliarda, ed il Gindi
dopo molte efame havendo troyato, che veramente costei era una donna, che si
credeva far malic, stregar bambini, ed altre scioccherie, ma in effetto non v'era
cosa di conciufione, o di proposito, risoluette di gaftigarla per la mala i
ne, ed in tanto soddisfare alia propria curiosita. Fattala però venire a sé 'inter~
rogd se andava anccr' ella a Benevento, rispose che si, onde egli le die: Tovi
voglio perdonare se voi andrete questa notte a Benevento, e domattina mi race
conterete quanto vi fara fucceflo. bisogna che mi diate la liberta,replicd la don-
na, acciò io possa nella mia stanza fare i miei scongiuci, e le mie unzioni; il
Giudice gliela concedette con questo che voleya dargli da cena insieme con
compagno: il che accettd la donna, bastandole esser fuori di quel luogo., dove il
Diavolo non poteva capitare. Andata dunque a casa cend con il detto compa:
goo, che era un giovanotto ortolano,¢ con un' altro giovane, che la donna
© che egli conducefle, e beyuto abbond: come era il sao coflu
me in tali fere di viaggio, la(ciati i commensali a tavola fen' entrd nella solitas
camera, € quivi spogliatasi, senza ferrar la porta,ne le finestre della medesimas
camera »





223

# Yordine del Diavolo ) s'unfe con più forte di bitumi puzzolen-
liacere in sul letto, subito s' addormentd; I due compagni, così
camera, e legarono la donna per ecia, e gambe alle
del letto, e benissimo la strinfero con funi, e si meffero a chia-
voc!, ma come fuffe morta non faceva moto, ne dava segno
onde i detti cominciarono a martirizzarla bruciandole hora
hora una costia, e finalmente così l'impiagarono in diverse parti del
arfero fino alla cotenna la meta della chioma; Cominciando a veni-

} donna con sospiri, e lamenti diede segno di suegliarsi, onde i det-

Jegami, ed uno di loro andò per una feggetta, ¢/' altro la rivesti
¢ dal sonno, e molto più da 1 martorj; giunta la feggetta,in efla
tarono al Giudice, il quale l'interrogé s¢ era stata a Benevento, ed ella.
che'si, ma che haveva patito gran travagli, ed era stata bastonata cons

ferro infuocaté, e strascinata, e legata per le braccia, e per le gambe,
riportata dal suo Caprone, che nel la(ciarla le haveva abbruciate con la
fa mezze le trecce, e questo perché ella haveva ubbidito al Giudice, e che
itiva morire dal gran doloredelle piaghe. Il Giudice ordind, che subito sul-
ita, come segui; ed intanto disse alla donna: Io v' ho fatto scottare, «
gaftigo del tuo errore, e perché tu conofea, che non altrimenti a.
yma in casa un hai ricevuto questi travagli, e ti risolua a lasciar que-
Re falfe credenze; che se lo farai, io ti perdonerd. Da questo bel modo di gafii-
cay Pt arguro Giudice quella verita, che appresso Jui era certissima.

NON, far per altra via, Non potette cflere in altra maniera, perché
non havrebbe mai potuto falire su quel monte; se non ve l'havefles
iavolo.

R/A, Violenza, dispiacere,soprufo. Viene dal Latino greco Anga-
4 cuaetio. Varchi Stor. Fior. lib. 2. E perché i Fiorentini nuovi tributi,
ritrovare havevano.
























tuna folenne. Fare un' angheria delle maggiori, che si possano fare.
me & da noi spefio usata in vece di grandissimo, ed è tolra da i riti
t Chiefa, che si dicono fefte folenni, le maggiori fefte, che seguono nell'An-
0. Così bieros, cioè fagro, pretio i Greci, e facer presso i Latini vale talvolta
Brandissimo, e4nchora facra 5 Adorbus facer, & lo stesso, che Anchora maior,
us maior, B, Virgilio quando disse; uri facra fames, per avventura inteses
ma..

VIEN ritratto, Vien dipinto. Se il dipinto 2 come il vero. Dice: chi to fa.
9 e chi deforme, per intendere, che i pittori da pochi soldi lo dipingono
“eAD fer rafo, Ad esser pieno affatro. Viene dal mifarare il grano con lo

Maio, che per dare, e ricevere il dovere s' empie lo staio, e quando è pieno si

- sttifeia sopra con un bastone, e si fa ca(care quel grano,, che è sopr' alla boccas

ha questo si dice radere, e tal bastone si dice rafera, € lo staio così pie-
Oli dice rafe 5 cioè picao per appunto fino all' orlo della bocca. a
F
le









224
VIs' arreca, Vis' accomo
42, s' arrecé con l'ani
PONZ A po
fizto,quasi riducen
mandano fuora il parto
tare, come si vede dal rca, che dice; oie
le riconobbi a usa huom che ponta
L' Espositore dice ideft che ipinga. Vedi 'Alunno fabr, num,
Ed il termine ponza ponza serve per esprimere uno, che aff ora
da poco; che si dice anche tresca tresca. Ticche ticche, denneinne y,
sotto C. 5, stan. 51, Za vanum laborare. Se bene qui Deaheaiers
nazza moltissimo ponzaffe. joke pibale
C/PIGLIO, E! uno increspamento della fronte fatto in git la
occhi, ed è una guardatura d' uno adirato, o d' uno eftremamente fup
piglio del ciglio, Gli antichi, come,Daate dissero Pigtia,la guardatura.
BOCCA bieca, Bocca storta. La voce biero Latino obliguus, e usata assai
Legnaioli per intendere l'incgualita d' un legno, e digono sbiecare quan
reggiano, e fanno uguale.; tahoe
PILA, E' proprio quel fodo, sopra il quale posano gli archi de i
si piglia oat fs quel vaso grande di pictra, nel quale si mette ac
heverare Ie bestie, o per altro uso simile; in (omma per pila intendiamo.
fo di pietra che tenga, o riceva acqua. owasell
STANZA LXXXI, STANZA L
Quand' ella verso lui volte le ciglia, Fermoffi a Adaimantile,
E vedde quella [un ea, figura Lo vole, e già le moze
Disposta  e graziosa a meraviglia, Come fai tn ( dirai ) custo dfeguita?
Cie più are KS 'far n' una pittura, Lo so, che ub ho a le.:
Gli s avventa di subito, e lo pigha, Aeeee mi donar quel
E,senza ricercar della cattura, Chiin due Aquile essendo trasformatt,
Dat suoi frafieri renebrofi, e but Perché lafsi facea degli shavighiy WS
Portar se ne fa via con esso lui, Mt han traportata qua ne hy it Be
Mactinazza porta via Cupido, ed in Malmantile lo piglia per marito; ©
havevano raccontato a Psiche le Fate., le quali trasformate in due A ic Vhave.
yano portata via da quel monte co' loro artigli. E qui finisce il quarto Cantare,
CATTVRA, Si dice quella somma di danaro, che si da a i birri quand' haa:
no pigliato uno; e si dice anche catrura quella polizza, e ordine che si da alli sbir
ri perché piglino uno. Di qui il Poeta cava lo (cherzo dicendo, che Marti
za piglio Cupido feng' haver l'ordine della cattura, e to port via,¢ nona
t0, che le futle dato il denaro della cattura, che havea fattadilui,
PACEA degli sbavigli. Si dovrebbe dire shadigi.Dan. Inf. C. 45.
: Anxi co pie fermati shadighava
ie Pur come sonno,ofebbrel' afjaliffe i
Ma hogei si dice sbavieli, e sbavigiiare; che un' aprimento
do il fiato, e poi mandandolo fuora, i) che per lo pil e cagionay (
penGeri, da tristizia o malinconia, o da altro rincrefscymento,










pee





























'





TO CANTARE. ary

¢ frigidi generati nello stomaco da ozio, e da pigri-

bocca per la via del cibo, e spargonfi per le ma(cella, ¢

gli fuora, alita con aperta bocca, il che da i Latini
ili, Significa non haver roba da mangiare, ne»

> ¢d habbiamo una rima, che dice; I

aviclia non può mentire *

- Ocgliba fete, o egli ha fame, o & vuol dormire,

overa Psiche fando in quel luogo, dove non eva da' mangiare, ne da
Wa Occasione di sbavigliare non potendo cavarsi la fame,ne la fete,

GLI, Dal Latino articnli. Zampe degli uccelli, o altri animali ditati,

le mani delle Pate, le quali convertite in Aquile,havevano artigli in

mani. Se¢ bene diciamo talvolta artigli le mani dell' huomo. Bocc,Canz.

Go? Amor ys io polfe uscir de' tuoi artigh, i

A pena creder possa,

Che alcnn altro uncin mai pile mi pigl,

FINE DEL QVARTO CANTARE;

INTO CANTARE,


























Se
. VERSIE 3
tan' Conme Coane seiee NS
ie ARGOMENTO, ¢
/ Vuol con gl' incanti dar la eAaga aita ?
F Jn Malmantile al popolo afsediato,
'an Ma dagli [pirti è così mal fernita,.
Che trai zimici e il suo saper beffato; -e.
me Vien Calagrilio,e a duellar U inuita y Me
h EP inuito è da lei toffo accettato,
'il H Fendefi, e altri due com' è usanza,
ie Sparsr di Piaccianteo fan (a pieranza, 2
go CR CIID CR BI CUM DBE
SR Sarath 7s
¢ art
ot STANZA L STANZA IL
ip Gli eftremi non fur mai degni di lode:
Ci vuol la via di merzz0,¢ chi ha ceruelle
Se vere, o falfe novitadiesli ode

A crederle al compagno va bel bella:
Le crede, s'ele fom fondate, e fade i
Ma s' elle fhar non possone a martelfo
Won le gabelia mica di leggieri,
Come fa il Duca a certi mefsaggieri,
Ff E Vo-






226

Volendo il Poeta nel presente Cantare: wince iadia vy
mandati da Martinazza per far diloggiar Baldone'y ¢10
le, per lo quale apparvero a)Baldone diversamente da quello
che fu causa, che egli non prestò fede alle loro cmgpcta
Che l'esser' huomo teftardo,.e capone non @ bene;
bene I esser così credulo, che si dia fede a onan ra che si sente d
degno di lode colui, che fa pigliare la via del mezzo, dando credito a' pele
se, le quali-egli conosce haver fondamento di-veriea y come fece Bald
meflaggieri di Martinazza De ES Ie:
CAPONE, Tellardo. Huomo ofitio nella faa opinione. In 'Baca
potrebbersi chiamare questi cali Capirones; da noi altrimenti Capardi,
TONDO. Huomo groffolano; semplice, facile, credulo, ec,
4 ai panni lani, che si dice sends, quando sono grossi, contrario i fini,
diciamo huomo fine, che & il contrario 4' huome rondo, Laica Novella 2
t0 fu hnomo di si i ereffs pafba, € così rondo di pelo, che in quattr? anni ai squola
tette mai imparare t' eAbbicci. Vedi sotto C, 6, stan. 80.
MINC HIONE, Semplice. Vedi sopra C, 4. stan. 15.
SE le beve tutte. Crede tutto quello, ch' ei sente dir.
BAB BV-ASS!, Igooranti, huomini di ceruello grosso'. Vedi frto'!
CREDEREBBON cl? un afin volaffe.. Per esprimer' uno, » che cred
le cose imposfibili a credersi, ci serviamo di questo detto. In Empoli |
lenne dell' anno, fanno una antica fefta, o rappresentazione di ae
no: Quindi è, che nel Capitolo in lode delp “Ate, che va colle rime:
dice:




















Ben mostran gli Empolefi aver cervello,
uanto conkienfi ad ogn' huomo da bene;
Che ? Afin diventar fanno un uccello,
NON pué tare a martello. Non cortilponde al vero, Tratto dat Cimento dell?
argento, che quando non fa, cioè non refifte al Martello, ae i e eee
1 Latini pure direbbero in questo proposito mon ef? aurum ignt prob. 5
NON le gabella. Non le pafla per vere, Non le crede: dal Pelee Ped
Gabeila delle porte, o de' pati; onde il verbo Gabellare, per ammecterey € a]
vare una cosa per buona, e per vera. <dica particella Heer
enfafi della negativa,come già,e mai, ec, Lo non vai mai, che si dica im sis
che si dica, Io non vud micayche si dica, Vedi sopra C. 4, stan. 69. '







STANZA IIL STANZA v. i

Ma, perché chi m' ascolta intenda bene; Ella ahaa allor, ih deb pers

Tornar a Martinazza. mi bisogna, Chit pigliar Panes

La qual dianzilasciat, levi fovviene, Che per ta pet ce i

Ch' in sul Caprinfernal, Pigra earegna, Ai farfibravo ye bee

Quel popolaccio ha aggiuntoye lo ritiene Se ben fra tanto: moana

Dal gear via com tantafuavergogna, S* ellacon Gambaforrn,¢ B

Perché quando per lei larafiguray mode

Railenta il corso, e piscia la paura,





QVINTO CANTARE: 227
AL pibieroin: “STANZA VI.
l 0.4 Ciò dette balxain casa, € cold dentro
0 Per aenerfi dispogliafi in capelti,
) Ecacciatasi addofso quant' unguento
Haveva ne! [uci feridi alberetli,
Vin gran circolo fa nel pavimento,












E con un uafo in man,scriteiye Cartelli,
Borbortando parole tutravia,
Che ne men si direbbono in Turchia,
STANZA VIL
pari in mezzo al feonoy O colaggiie dal forreraneo Regno

wdo all ordine ogni cof, Cornuti mostri, e gente spaventofa
ad effetto il [uo disegno Pilizginosi eee Dees
voce firepito/a: Badace a me; le mie parole udite.
0 a Marcinazza, la quale sopra nel C. 3, fan. 76, la(ciò, che mon-
cioni in sul Caprone, haveva arrivato quel popolo, che fuggiva per
riconosciutala, la prega a dar' aiuto a Malmantile, e far, che essi
iano a-combatter,se fijpud. Bila dice, che Rima neceffario il combat-
che intanto vuol vedere, se gli riesce cacciar via il nimico per altre,
e vattene in casa a fare i suoi incantefimi a questo effetto.
(FERN ALE.Duedizioni.come ridottein una,significante Caprone d'In-
i quel Diavolo in forma divCapra sopr' al quale era cavalcata,
Za j5e il quale si favoleggia, che vadano.le Streghe a Benevento,
sopra C, 3. stan. 69.
Vuol dire Cadavero d' huomo, o di bestia. Cavalcanti flor,
p. 2. dice;. Se volere veder quanto la lor per fidia si disse/e contro al fan-
macgiori, cercate i Connenti de'Frati, e troveretegli pieni di corpora, e di
vi antichi. Da questo dire del Cavalcanti m' indugo a credere, che
na ifichi cadavero d' huomo ammiazzato con ferite, e straziato,
cl ere di tal voce per intendere una bestia piena di mascalcie, ¢
¢ flinio con Pier Vettori nelle Varic lezioni, che venga da Charonia,
ano già le voragini del fuoco,, che in diverse parti del mondo fj tro-
dicevano Charonia da Caronte, perché la superftiziofa Gentilica sti-
yche tali vorapini fussero bocche d' Inferno, e che per quelle s' andafle da
E perché hanno sempre puzzo orrendo, che procede da acque (ulfu-*
flo cominciarono a chiamare Charonia tutte quelle cose, che grande-
vano; E noi seguitando gli antichi diciamo C aragna a tutte le coies,
wutono, come fanno le bestiaccic guidalescofe, e le morte. Dicigmo Caro-
un' huomo, che habbia cattivi sentimenti, perché un' azione mal fat-
dire 4 putes; onon ha buono odore., 3
Atenieli chiamayano Charonia quella porta del Pretorio, o Palagio del Po-
r uscivano cojoro, che erano condorti al supplizio, secondo
iulio Polluce nell' Onomattico, e Alex, ab Alex, lib, 4.c. 16. e Cel.
lect. antiq. ¢, 8, e lib, 47. c. 9. Tolta la derivazione di tal vace pures
ate, che conduce J' anime al hppltsinpatenee in barca, e si dice mane
2 dar®






































228

dar' uno a Cavonte per intender wala uno alia morte. vile
PISCIA la paxra, Ripiglia animo.. Non ha più para.
s no azauffaci sogliono pisciare; e comunemente dalla plebe si dice
li paura; € da questo diciamo pisciar ta paura 5 quand' uno -spaventato
rito,perde quel timore. * Achy
L! ABF-ANNO in fulla pena, Eca aggiunto alla pena, che hebbe per la
) atfanno cagionaro dal correre, Vedi la voce Affanno sopra C.g. stan, 69.
VEKMENA. Va foil, e giovane ramo d' una pianta si dice Vermena
Lavino Himen, Que) patio di Vegezio; de re militart lib, 1. cap. 11. Quemadns
dut ad feuta viminea, vel ad palos antiqui exercebant tyrones: L' antico vo
tore traduce csi. Come a feudi farti di vermene, o paliy tt provavand 4C.
GLI giunta addosso la piena, Sono accadute loro tutte le maggiori:
piena è presa nel senso detto sopra C, 1. stan. 84.
eo FAR in mo che non s' habia a metter la spada al fanco, Far in modd ch
negozio s' aggiufti, scnz' havere ad adoprare le armi, che si dice Agginftarla
la spada nel fodero, i
Sé si puo far ds manco, Se la necefita non forzi a fare in questa maniera.
GAMB ASTORT A, e Baconero. Nomi di Diavoli inventati qui dal' | Poetas }
nelio feflo modo, che inventati furono i nomi di Barbariccia, € Parfarelte
simili.
BALZA in casa, Va velocemente in casa. Zalcare propriamente si ¢
faltare, che fa la palla, o pallone perquotendo in terra, Vedi sopra CG;
SPOGLLASLin capelli, Si spoglia ignuda, e scioglie le trecce dei
vuol intender il Poeta, se bene si scrue del detto /pegliar/i im capes che si
adoperare ogni suo sapere, e tutta l'applicazione per fare una tal cosa; per in-
tendere ancora che Martinazza s'era tutta applicata a fat, che Balser per
via d' incanto diloggi da Malmantile,
€ACCIANDOS! addosso,, Metiendosi addosso, E se bene il verbo ite uo!'
diy intromettere con violenza, noi lo pigliamo in senso di mettere 5 come i vede
nell' Ottava antecedente cacciar la (pada per metter la spadas
ALBERELLI, Vali diterra, Odi vetro, entro a' quali si ccobsladalatan,
guenti, e cose simili; e son forse quei vasi, che i Latini chiamano alweolt
giiano il nome da questi. 2b Pai
BORBOTT ARE, EB' un certo parlar fra i denti poco inteso da chi l'Seana
che diciame anche brontolare, E' il Latino fubmnrmurare. Borboryttein
Greci è il romoreggiare, © mor morare che fanno le budella: Verbi psiiies al rian
fleflo naturale. !
e4 Più pari, Cio' a piedi giunti insieme. Questa voce pari y che per
vuol dire xgwatied di numero, ed il suo contrario e dispari ( che diciamo se
i Latini dicono par, © impar, servc ancora per denotare ugwalita di
corpo » come gui; che s' intende, che un — non era ne pill innanai, er A
indietro dell' altro Si dice efer pari quando uno $'é vendicato con penn
ha pagato tutto quello che doveva, E ancora + esser pari e gat &
quando non si pende per neflun verso. Strada pi ari per. re faut In
wa l'adoptiamo in tutte quelic cose, dove entri aa Peet




































ah
 STANZA VIIL.
p vi scongiuro, € vi coman:
Pokies, € virtù ai questi incanti,

 QVINTO CANTARE:

229

YOST; Affumicati. Tinti da fumo, come sono i cammini, che son
filiggine, che ¢:composta di fumo, e d' umido. Lat. faliginof'.
ATE ame. Attendcte a me; Osservate le mic parole, e state attenti a

STANZA Ix
Per gl' imbrogli vi chiame,e I invenzioni,
Che ritrova il Legifta, ed il Notaio,
Quando per pelar meglio i buon pippiont
Gii aggira, che ne anco un' arcolaio;
Florsu, pexri di Sacchi di carboni,



|porcheria de' guardanfanti Per ques ladri del Sarto,e del Adngnaio,
le donne De per cofume, Che ti voglion rubar a tuo disperto; x
di pulci, e fudiciume. Vycste fuor 7 venite al mio sae.

con diversi (congiuri chiama gli Spiriti infernali, per servirfenes
a far diloggiar Baldone da Malmantile: & l'Autore mostra il disprezzo, che egli
fa degl' incantefimi, facendo che Martinazza ¢oftringa i demonj con le cose ri-
— ditoleé, che egli mette in queste due Orcaves «

, SCONGIVRARE. Queito verbo t da noi usato per inteddere Eforcizzare,ciod

b “il Diavolo per via di giuramenti di formule facre dette per questo
 Elorcifmi, cioè (congiuri; e comunemicnte e preso in questo senso, ed anche pill
atgamente si tira, come qui, alla maniera d' inuocare gli (piritizufata da'Maghi,

se bene il suo proprio significato è¢ domandare, o chiedere con grande ardenza,

! edein to del verbo pregart dicendosi. “i prego, vi supplico, vi feongin.

ORCHERLA. Si dice non solamente un' atto sporco, ed illecito, ma ancora
una matetia schifa, sporca, e brutta, o otal fatta..Come per esempio: 1/ tale
Seve wis crarione, che rinfed una bella porcheria, La vostra mercanzia non bebbe efite,
perché a una porcheria: I Libri di quel Mercante furono abbruciaci, perché
rat Pieni di parrire falfe, a' altre porcherie. Varchi nelle Rorie Fiorentine dice:
Era appunto sparfa in Firenze  usanza a? andar in 2axrera,  mantello, che era una
Ia por « Questa voce Porcheria significante disprezzo potrebbe venire dal
Latino porcaria, che vuol dir I utero delle Vacche, o delle Troie, dopo cht han-
NO partorito, o per dirla colle parole di Plinio lib.11. Cap. 37. Yuluam partu edito,
€tali vulue, particolarmente quaado non avevano condotto il parto, ma si era~
no feonciate, dagli antichi Romani erano manger per una cosa conser
la Porcaria non la mangiavano tanto voientieri,forse per esser cosa più schifa.Era
chiamata porcaria in un certo modo per disprezzo, € così ha portato ay
nol il ignificato, che ritiene di disprezz0, ed abbominazione. Ma la pil fem-
plice origine € da porco animale immondo; € così deta porcheria, cosa da porci,
Some furfanteria, cosa da furfanci,¢ simili.
- GVARDANP ANTE. E? uno strumento compolito di cerchi di filo di fetro in
tondo; il quale portano le donne Spagnuole, © circonda loro fa cintura sotto le
Velli, le quali fa gonfiare: E lo dicono guardanfame, perch egli difende dalles
offe l'infante, cio' la creatura, che hanno le doane prtgne dentro ali'urero,
Perché questa foggia di vestire, che havevano cominciata ad usare Ie donne di

. Firenze,

/ 70. Latino obfecro, obre/for.












Yo!






230
Firenze, conosciuta rae
dava a poco a poco difufa

ne il Bando, cioè l'efilio, e proibizione «

PIPPIONT. Piccioni. S' intende gente semplice,
sono i pippioni, colambarum pulli, colombi giovani. £
Cavar danari di mano al corrivo. i
ARCOLAIO, Strumento, sopr' al quale s' a
@' altra materia per incannarle, o aggomitolarle co
ed e un moto perpetuo, e però dice ageira che ne anc!
gira bene, ed assai: ed aggirare in qucito luogo vuol dire ingannare; ¢
ratore, ingannatore, Coa Bind » si prende per huomo aggiratore; ¢
fare per girare, cioè non si rinuenire col ceruellogL. delirar: \ gg
Ingannare; Latino circumuenire.: Ns

STANZA X.;;
Tutto lt Inferno a cos; gran parele

Vien fibilando, e intorno le faltella

Come dall' alba al tramontar del fole y

Fa quel, ch' è morfo dalla Tarantella, 1 Com

DIRT ANZA), SL K

Ed a far ch'ei si pigli quella spracca Ma perché tu mi voglia far

Senza cagion,gli par ch'ell'abbia iltorto, 'Di darmi Baconero,
Perché dalla Pro 'onda sua baracca Perch'jo mi va dell'
A Malmantil non e la via dell orto. 4n cosa che mi preme, ec
Corpo | ( dic'ellayed al C elon l'attacca) Plutone allor quei due fa?
A venir infin qui tu [arai morto} E la frrada si piglia della p
Ma senti il mio Pluton, non t adirare, Seguito da i sugi sudditi,
Che venir non t' ho fatto fine quare, Posson fondar la C ompagnia.
Agli scongiuri di Martinazza le comparisce avanti Plutone con moltil
ed ella gli chiede Baconero, e Gambaftorta, Eile la(cia quivi
nj,¢con gi altri se ne torna all' Inferno..

SIBILANDO, Soffiando,filchiando. E' voce Jatina, che ritiene il suo.
10. Verg, En. 11, edrrettis horret fquamis, © fibilat ore. Iptendiamo
mente il filchiare de i ferpenti.

SALTELLA. Fa spelfi, e piccoli falti; & il faltar delle Rane, Vedi
6. stan. 37,

MORO dalla Tarantella, Per la Calavria, e Puglia dicono si trovi un
lo ragno detto Tarantola, o Tarantella, il quale nato ex purrs scappa
fure della terra in tempo di state. Geeta mordendo un'huomo,gli mepte ad
una infermita specie di rabbia, che lo forza a ballare contiaovament
vata, al tramontar del fole, ne prova quicte, se non quando sente
chitarra o con altro strumento simile,un' aria detta percio la Tanane
faono questo rale attarantato si affatica a ballare tanto', che flracco
morto; € stato in questo fuenimento qualche hora,si rizza, e ccfla,
stando fano per qualche giorno: E perché in quel paefe si trovano moll


































-QVINTO CANTARE: 231

“Dicono, che tale infermita duri quanto dura la vita di
attarantato, la quale dicono, che non paffi tre anni;
posta pagati da quei Comuni, i quali vanao cercando
gli per universal benefizio, e ne hanno un tanto
un Rettore a ciò deputato. Dicono in oltre, che
ficato provi la detta infermita ogni anno per un mese poco pill,o
torno a' quei giorni,ne' quali fu morficato, che fara intorno al Sol
: trovino di quelli che la provino ogni mese per qualche giorno,
> o Tarantella dalla Città di Taranto, nel cui territorio forse
te si trova. Ti Lalli nell' En. Tr. lib. 1, stan, 22. dice,
| Enea quantunque bravo anch' ei tremante
, Morfo dalla T arantola parea,
Gl'introna la testa con le strida: lo sbalordisce; lo fa aflordare

'vecchie. E' invecchiato, s' intende uno che si tratti da vecchio;

Valo di rame, col quale si caval' acqua dai pozzi. Vedi sotto

an. 3. Ha if decto far come le secchie senz' altra aggiunta, significa andar in
af come fanno le secchie infunate nella Carrucola.

Intende abitazione, Che baracca vuol propriamente dire quel

(0 i soldati in campagna per loro abitazione, nel quale 2a

€ capannello di frasche, o d' altro,col quale si difendono dal

icque. Viene dal verbo barrare, che vuol dir Circondare, o accer-

ice anche frabacca, © corrottamente, © pure ¢0 guid trabibus conftrue












via dell' orto, Questo dettato significa; la via & Junghissima, e difa-
r ordinario dall' orto alla casa non è pil lungo viaggio, che ca~
ri della porta, la quale di casa esce nell' orto, essendo per lo
a Città gli orti appiceati alle case.
PO! ed al Celon ? attacca, Vuol dir Corpo del Cielo. Si dice Corpo del
a del diavolo, ec, Ma quando uno pafia pib ia bestemmiando les
: Bil ateacca al Celone per intendere 5 egli entra nel Ciclo, cio'
jumi Celefti; EB per render più oscuro questo detto,ci serviamo del-
ne, che wuol dir quel panno, che si mette sopr' alla tavola da meafa
di distendervi sopra la tovaglia..
. Detto ironico per mostrar la poca stima, che si fa della Fatica,
durata uno a nostro pro, ¢d il poco grado, che gli fen' habbia,mafii«
tale ne fa grande oftentazionc.
: + Voci latine usate nel suo significato; e dicefi: Won fine qua-
3, e significa non senza qualche fine, o cagione, Franco Sacch,
de Gli venne vogla d' andar a strovare il Re Adovardo, e won fine quare, perché

tree molto lodar|o.

IN fondar la Compagnia de' brutti. Sono'tutti bruttissimi. Habbiamo in.
tun' Accademia, o Compagnia detta de' Brutti, la quale si raguna ogni
giorno di Befana ( che così si dice il giorno dell' Epifania ) ed in un lau-

tulimo,















































age MALMANTILB, 5
tissimo, e stravagante fimposso si crea il |
iftim 8: maf Confininanmmnt

la il Fondatore,¢ si fa sempre il pil brutto.
Poeta.;
STANZA XIII

Lascian Plutone,e corroz dalla Druda
J dug spirti, aspettando il suo decreto,
Ed ella allor che fa da Ceccofuda
Per far si che Baldon dia volta a dreto,
Ed anche se si puo ch' ei vada a Luda,
Gli prega, che le dien qualche fegreto
Da far Jenz' altre guerre, ovver conte/e,
Che quelle genti sfrattino il pacfe.

STANZA X vo

Pers se non finghiam ch' egli le feriva
Ch'il suo rivale (adeffo ch'egli ha inteso
Chrei #¢ partite) con la gente arriva,
Per volergliela (u leyar di pefa,

E che se propria è ver, che per lei viva Hor dunque tu che [ei saputa
( Com' ei /peffo giurd) d' amore accefo, Che non (a cedi manco a Ci
E feglit cara lo dimoffri, e prenda, Scrivi la carta, che tu fai ch
Ed armi,e braviye corra,e la difenda, Sian tutti un monte d'a

I Diayoli rrovano l'inyenzione di far diloggiar Baldone da Mal
guefta € fargli intendere, che la Geva sua dama e in pericolo d' esser
cono a Martinazza, che sCriva la lettera. i

DRVDA, Innamorata, amante, ec, se bene non sempre si piglia
to difoneftofo; Qui intende dama di Plytone, che era Martinazga, che » some
firega, haveva Jui per innamorato. AY

FA da Cecco fuda, S' affanna,s' affatica. Scherza con questo nome:
perché quand' uno s' affatica, e s' affanna (enza proposito, mostrando d
cose diclamo: “tale /uda. Di questa natura era quel Cortigiano descritt
Berni nelle Rime. Ser Cecco non puo far senza la Corte, Ne la Corte puo iar
Ser Cecco,: a oa

VADA4« Buda, Vada via per non tornar pi. Proverbio nato dalla guerra
che già fece il Turco contro Lodovico Re d' V ria quando acquifto Bud
ca l'anno 1626., che vi morirono quaii cutti li Criftiant, che yi andaronos ¢
il medesimo Re; E però da quel tempo in qua dicendosi: // rale e andato 4 Bu
s' intende è andato via per non ritornar pil, o vero € morto, ed ha il mede
senso,e per la medesima cagione; // rale e andato a scio. F andato a Patraffojschet-
zo fulla Città d' Acaia famofa per tl martirio di S. Andrea, come s¢ si dicefleua
Latino: ivir Patras; e fulla frale uiata dalla scrittura, sopra quei che muolon0 »
¢ si feppelliscono, quasi dica; E' andato ad patres /uos è
SFRATTINO, o sbrattino il paefe. Ripuliscano il paefe, cioè (ene vadanO.
DAR' a due tavole'a un tratto, Far due negozzi in uno ste(io tempo. Tratt0
dal giuoco di sbaraglino, nel quale con un fol tro, si dia duc, e tre tavole, 0
girelle. Si dice ie: far' wn Viaggio, e due fernizj. Vedi sorto C, 6, stan. ae.























QVINTO CANTARE, 233
a in due faffe. Attendere a due partiti. Vitumeligere, G alte-
eaiasicinn o. eae ahbe dirsi Mont' Vghi dalla fa-

illaggio vicino a Firenze.

antichissima Fiorentina. Ricordano Malespini nella Stor. Fio-
l no ebbe nome V 60 5 questi anche sue nobilissimo
no sei ql dicefona gli Vghi, e per innanzi il poggio,
'si chiama Montughi, s'é chiamato per loro... Lo stesso conferma Gio,
RE
7 balan. Allora allora; Subito subito... Vulla interposita morula.
INE, Specie di Serpe, detto così, perché forse vada yeloce comes
facta, e credo tuber dei Latini,
y 4A. Propriamente vuol dire Remiganti di galera: Ma qui @ prefa per
, come si trova anche prefa in pib Storie Fiorentine antiche, e sopras
76.5 e fora C. 11. stan. 76. dal Latino tarma, se bene propriamente si
di soldati a cavailo.
OL ammazzar bestie, e persone. Vuol difertare il paefe. Quando vogliamo
ler uno, che vanti di voler far gran brayure, e non lo giudichiamo atto a
me Veruna, diciamo Vol ammazzar be/tie,¢ persone. Ed in tal senfo di derifio-
'DEE preio nel presente luogo. IL Berni nelle rime congiunfe queste due voci cu-
allor che disse: Con wn mondo di bestie, e di perfane,
saputa. Sci dotca; sei sCientifica. Donna fapura, facciura, faccente vuol
di Vna donna, che in tutte le cose vuol far da macftra. Colla stessa figura di
se faccente, diceli e4uuertito, edccorto,edunifato, e dagli antichi Senrito
ee che avverta, e che s' accorga delle cose, e che stia full' avvilo, ¢
t, ll participio paffivo in forza di attivo.
| SLAMO una mana a afini, e di buoi, Siamo tanti ignoranti. Per lo più a que-
fle dug ie ed al Castrone assomigliamo coloro, che non hanno scienza alcu-
ha. S¢ bene  Autore fapeva, che il Demonio possiede tutte le scienze ( che così
a uo Greco nome Daemon, cioè fapiente; e noi d' uno che sappia eccel-
a che cosa dichiamo; Egli e un Demonio; nondimeno ha voluto,
efi due Diavoli si dichiarino ignoranti, acciò che si creda. più facilmence
Vertore, che fecero di scambiare le palle, come vedremo.
STANZA XVII, STANZA XVIIL
Non ti dé contre atispond' ella, a questo, E per dar al negoxio più colore,
Ed che voi vi conoschiate: In forma voglio wr' io d' una gomare
| Her si, dice i Demonia, ferivi prefta Della sua Geva detta Monafiore
parole in tal genere agginftate Confidente del Duca in ogni affare;
' a; ma vedi, io mi prote(to, Gambafforta verra da fernitore,
6 portai mai lettere,o imbasciate, Che mostri di venirmi a accompagnare,
Scriniforgiunge queische quato al porta E già per questo ho fatte far di cera
ito qui con Gambaftorta. Due pallennach'e bianca,e Laltra nera,















——

SS ———

= TaAsS ASD




s
#
si
¥
8
i]
¥
4

Eccoms



Gg STAN.








" i
234 MALMANTILE >
STANZA XIX. shoo eg rah
Quand' un tien quefia neva in una brica, La nera a lui dare c
Di subito a un' huom prende figura; 1

E s'ei vi chinde quellaltra ch' bianca,
dn femmina si muta, e trasfigura. v
Si che riguarda ben 8 altro ci mance, La Strega qui gli dice,
E distendi mai più quefha feriteura; Perch ella ferive,e guaspoleha
Chil mio compagno,ed io qua per viaggio Ma lo [cancella, e mettelo in |
Ci marerem Leffigte, e il personaggio. ° Così fie" la carta, ela figilla

STANZA XkKI. van
Le fa la soprascritra, e poi finisce
A più a' un ghirigere ta propria mano,
E con est quel diavolo /pedisce



















no la medesima lettera per portarla un di loro trasformato in Mona
1 alco ia un Servo per via di due palle, e se ne vanno così da Baldo:
havere feambiate le palle, chi dovea apparire la Fiore, appare il
rono scoperti.:
NON portai mai lettere, o imbasciate, La maggiore offela, che si
certe donnicciuole,¢ il dir loro; porta lettere; porta imbasciate; fa servig:
polli ( detto credo io dal Franzefe Puuler, che significa letterino @' amy
portatrice di lettere amorofe ) perché vuol dire Rufhiana: B però
Martinazza, che non vuole quest' offefa addoflo si dichiara, che non (
portar lettere, o ambasciate, cioè da far la ruffiana. vb
ECCOMI lefto, Eccomi pronto: Eccomi all' ordine. Zefo in ' f
vaol dir difinuolto, e senza imbarazzi. oth SBE!
DAR colore.al negorio, Bar' apparic = vero quel che è incerto; Da:
similitudine. Questo fanno appresso i Rettorici quei, che da loro foro
Colori, Givvenale dice: dic, Quinttiliane, colorem, + yo
COM ARE. Quella che tiene la creawura'al Battefimo. E qui il Poeta
il costume, che in simili amori per lo più la Balia, la Comare sono ne »
portano le parole, yagi
4ONA. E parola fincopata da Madonna, ed è il titolo che si da comunt-
mente alle donne d' infima plebe dicendosi in diminuzione Signora » Madonna»
Monna, come Signore, Meffere, Sere. Ma perché Afonna oltce al significato dl
Bertuccia, ha ancora altro significato osceno (almeno in lingua Veneziana) no!
per sfuggir l'equivoco; hoggi costumiamo dire Agora enon Monona:
ALAT più, Hormai, Cioè finifeila una volta: E' termine dimostrative @
certa impazienza, e si dice: Ob mai piis: ed: il latino eandem aliquande:¢
confa con I imperativo s dicendoGi: Ob mai pis: finitela, OE RB S|
POSTILLA, Nel nostro idioma ha diversi significati; perché 6 vaol }
( figuratamente secondo Dante ) immagine d' un' oggetto, che ritorni alla
weduta da un yetro, o dall' acqua chiara, Dan, Par. C. 30,




















">


=

<= SSaeu SE BS

a

QVINTO CANTARE:






Ls 235
Quali per-verri trasparent?, e terff,
Over per acque nitide, e tranquil,
Wem Wardens « Won si profonde, ch' i fondi fien persi';
6 Lernan de nostri vifi le postiile
Pgs » Debili st, che perla in bianca fronte
at  Won vien men tofto alle nostre pupille.

O viiol dire annotazioni, o glofa, che i Latini dicono expo/tio. O si piglia per

aggiunta, ed.¢ composta di due dizioni po?, & ila. Quali dica,

Postilla verba, cio dopo quelle parole, scrivi, o aggiungi questo, e questo, E

annotazioni, glofe, o aggiunte hoggi per /ofilla intendiamo anche»

del libro, cioè quel bianco che si la(cia di sotto, e di sopra, e dalles

foglio scrivendo, o flampando: Si che scrivere in possilla vuol dire scri-

Margine;¢ s' intende ogni aggiunta, che si faccia al tefto scritto,o

in qualfivoglia luogo della carta © sia di sotto, o di sopra, o dalle ban-

idei versi ordinati, e regolati; ed in questo modo, © luogo, dice che
"Marti

—. E'un tratteggio di penna usato per Io più nelle sopra(crittes

» come mostra il Poeta nel presente luogo, che faccia Martinazza..

Ghirigoro da' nostri antichi era detto in volgare il nome Latino di Gregorio;
Ghirigoro trovafi sempre costantemente scritto nel Malespini, e nel

Villani; come era la lingua di quel tempo. Ma qui Ghirigoro apparilce per av-

Ventura dal girare, e rigirare della penna così detto.. E le parole /n propria mano

8 usago nelle soprafericte di quelle lettcre, le quali si mandano a yno, che sia ne!
luogo, o Città, o vero poco lontano da colui che scrive.






el.

STANZA XXII
Che Baconera ii quale e un' avventato,
Neb dar la palla alt' altro di nascofto,
“Senza grardarla prima havea [cabiato,
a: fattoun grad' arrofto,
i quand' a Baldone egli e arrivata
Dice case dal ver troppo discofto,
i afferma d'effer dina,e stbra
Huamo alla barba all abito,e alle mebra,
STANZA XXIIL

ECambaflorca anch'ei balordo, e frolto,

Mencr'apparir ficrede un' hue dabbene,
Alla Favela, alla prefenza, e al volta
Per Vna fa fernizrj ognun la tiene,

» Afoglio intantoil Duca havea lor tolto,
Eveduto lo scritta, e quel contiene;
Refha certo di quanto eraindovine,
Ch i furbi vorrian farlo Calandrino,



STANZA XXIV.

E poiché gli hanno detto, che la Geva 5
A lui gli manda con quel foglioa posta
Ma prima che da loro lo riceva
Hann! ordine d' haverae la risposta;
E foggiunto y che mentr' ella scrived,
Getrava gocciolon di questa posta,
Per il trabufograndech'cllahahavuto,
Come potra sentir dal contenuto,

STANZA XXV,

Egli e (dic? egli ) um gran parabolano y
Chi dice ch' ell' ha scritto la presente,
Quand'ella no piglio mai penna in mata,
E fo di certo ch' ella n' e innocente
Che poi tu sia la Geva, ch' in Venane
A me fu molto nota, e confidente,

E tu sia buom, 4 dirla in coscienza,
AA me non pare y¢ nego confeguenza,

Gg STAN:




236 MALMANTILE —
STANZA XKVL»~

1 non compagni a una risposta tale - Ed alle

Guardanft in vifo,inquelfendofiaccorti,

Chregli hanno equivocato,e fatto male;

Refhan quivi allibbiti, e merzi morti, 'Di Baldone, e di tutta

Giunti quei Diavoli da Baldone, credendosi rapprefensare uno
V' altro il Servo, non essendo accorti d' havere scambiate le palle,se
ambasciata: Ma Baldone, comprefo, che questa era una furberia, non
ciò, quanto dall' essergli noto, che la Geva non fapeva ferivere 5 se gli cl
nanzi con una gran quantità di filchiate. ocx prcee
tVENT ATO, Vno che opera senza considerazione,e furiofamente.
mo fd 3 € precipitofo; dal ivo Latino ad. in si
to d' avvenirsi, cioè imbattersi in una cosa con velocita »e con furia,
DI nascofte, B \o feo, che Di foppiatto detto sopra C, 1. stan. 75.
PIGLLAR un eros « Pigliare errore;Intender una cosa per un' altra, §
pigtiare un granchio a fecco quando uno nel picchiar qualche materialesh !
si batte il marteilo sopr' alle dita, o si (erra le dita fra due materiali
ecrore intendiamo poi far un'errore, quando diciamo pigtiare un gr Beral +
Che Virgilio ha preso Vn granciporro, "phase:

FAR un arrofto. Far un' errore. E' lo stesso che pigliar un granchio, vad
per avventura dal verbo arrofarsi, che vuol dir affaticarsi spropositatamente
furiofamente; e le cose fatte in furia non si fanno mai bene, a

BALORODO, e folto, Sinonimi che significano Huomo senza giudi a vO~
ce stolto è pura latina, e balordo e lo stesso che in Lat, bardus. aeons )

VNAF A fernixxj. Come's'é detto sopra s' intende una Rufiana, ©

VOG LION farlo Calandrino, Calandrino, secondo che dice il
sue Novelle, fu un' huomo tanto credulo, che gli fu dato ad intendere fino, che
egli era pregno, e però da costui diciamo 7% mi vnoi far Calandrino per intend —
re Tu mi vuoi far credere quel che io fo, che non è vero, Si dice anche far
pedino, da uno de' nostri tempi della natura di Calandrino

HANNO ordine @ haverne la risposta, 1 Poeta per maggiormente a
castronaggine di costoro, fa che chieggano la risposta prima di p a

ropotta. ¢

CETTAVA 'coccioloni di queffa posta, Lagrimava gagliardamente. Il termine —
Di questa posta ignitica grofiezza; erano pere ds questa posta, cioè pere grodidi-
me,¢ si la: » che colui, il quale dice cos,, accompagni il parlare col geft
delic mani dimostrante la groffezza di quella tal cosa', Si dice anche samo fatte;
tanto grosse, come vedremo sotto C. 10, stan. 17. 18. e 36.

TRAMBVSTO, Travaglio, rimescolamento, follevamento
fa di disgrazie. 3: aig

PARABOLANO, Bugiardo; chiacchierone; [propositato: Da Parabola, cide
similitudiae,o Racconto; ne'Capitoli di Carlo il Caluo si legge. Par: itty
fimul, & considerauernnt. Parlarono insieme, Du Frefne alla V. Parabola,

SO ch' ella n' 2 innocente. Intende; io fo ch' ella non fa scrivere. Per esprit
re uno che non habbia ne pure una miaima notizia d' una tal cosa




































d'animo per eae







QVINTO 'CANTARE:

alcxno nella tal cosa o e innocente della tal cosa:

go il tutto: perché negando la confé apn viene a
4 e tutto I arguimento, € così tutto il dif

un subito timore 50 wergognie,e¢ perciò
'solore fmorto y¢ gialliccio, come, feccandosi, diventano le potatu-
che si chiamano éibbie,dalla qual voce viene alubbitoye altibbire, Ve~
rio della Era/ca alla Voce “Alibbire + IL Varchi Stor, Fior. lib, 10.
ova il quale incontinente ( quasi le fuffe venuta meno la terra

237












: ie,
ATA. Romore di yoci, filchi, urli, battimenti di mani, ¢d'altro,
uno per dargli la burla, Far le fischiate.a uno, quel che i Lat.

STANZA XXIX.
Che la padrona il tutto le comparte
Come s'in Malmantil fien due Regine,
 Psiche egnor di attorno, Anzi il bando si manda da sua parte,
ggni quattro palfi fa un Jamento. Perch' ella soffia il nafo alle galline 5
to tutto quanto il giorno » Così poi c bebbe dato libro,¢ carte,
cento volte,¢ cento Entra nell' un vie un, che non ha fine y
Malmantile, e Similmente Costui, che qusvi s' e posso a bottega
tinaza,e fev'é ds presente. A legger sopra il libro della Strega,
TANZA XXVIIL STANZA XXX,
xn, ch' al fin la mette per la via Quest altro, che non cerca da costui
sche quest orrida Befana, Di questi cingue soldi, havendo fretta,
un.toz%0 haveva,careftia Poichegli ha ee quel che fa per luiy
sitet erba porcedana y Spronailcavalle tutto aun tepo,eshietta,
di.gran soldi.in [un balia, La donna che trovare il suo,colui
una cefacome una Degana, Digiorno in giorno per tal mezro aspetta,
Corteeingrado,e giunta a segno, Per no loperder adocchio,ech'ei le machi,
i totum continens del Regno, Segue la frarna,egli nasipresi ifianchi,
| Poeta a parlar di Calagrillo; il quale camminando Psiche s' imbatte»
he le da avvilo di dove sia 'Martinazza.
MARCHIARE, Si dice marciare, che vuol dir camminare. Voce Francele,
ma già fatta Italiana. Vedi sopra C. 1, stan. 43. EB più accofto alla pronunzia,
-Oltramontana, dicefi anche Aerciare, forse da Marcia, contrada, pace, cam-
'Mino dane/marce disse il Villani la Danimarca; cioè Danefe Contrada,
'ANA. Intendiamo Donna brutta, malfatta. Vedi sotto C. 8. stan, 30.
C.9, fan. 1.
T0ZZO. S intende pezzo di pane. Haver careftia d' un toxxo, Vuol dire ef-
mendico, pezzente.
ST AVA come la porceliana. Cioè terra terra, come l' erba porcellana, che fer-
2 per terra, € non alza mai virgulti; detta porcellana dal Latino Portulaca.
questo detto significa Vno che sia in povero stato, e non habbia modo di falle-
-Varsi, che i ooo pure dicevano humi sacere.
sea balia. In (ao potere,e dominio. Balia e voce fatta venire dal Mroh-

















*
BS
5
€









238 MALMANTILE™)
ni dalla Greca Buleia, che suona lo stesso che Bule; cic

Senato. A noi suona Potefta,giurifdizione,autorita,e quel che i Latini
poreflas imperium, Dan. Purg, C. 1. bt







Che pure
Petr.C.36. Afentre ch' il corpe è vivo y i
Hai tu il freno in balia de' pensier tuoi
HA una casa come una Dogana, Cioè piena di robe, come sono le Dog e
ne di mercanzic. set
IL Bando va da parte sua, Cioè, ella comanda'. 7A oP
SOFFLA it nafo alle galline. Ella fa tutte le faccende. E questi tre
Totum continens de/ Regno;Rando va da parte fuaje fofiail nafoallegallineh
lo stesso significato;ma di questo ci serviamo per lo più per derisione,per in
uno che habbia ambizione d'efler creduto gran ministro,ed habbiaim
neggi d'un governo,e non sia vero; che per ischerzo direbbefi anche
En.Tr.l.4.st.15.SopratrnrtoaGinnon, che del far ragza E'detta Varcifanfana,
DAR libro ye carte. Dare fata notizia d'alcuno, 'Viene da'coloro,i
havendo debito co' Magiftrati, son mandati in efazione a i Ministri forenfi
wali Ministri i Magiftrati mandano il contrafegno del libro, nel quale €
il debito di quel tale, il nome, e cafato di eflo, !'origine, e somma delde
ed a quante carte e la un partita: E questo si dice dar libro, e carte,
in proverbio, significa Dar notizia chiara, ed efatta d' alcuno; 0
habbia fatta un' azione per altro occulta. 4
ENTRA nell' un vie uno, Faun discorso da non ulcirne mai, cor
be se uno voleffe seguitare Vn vie uno fa uno, due vit due fa quattra y ec, che sande-
rebbe nell' infinito. Dice il Varchi nel suo Ercolano, che in questo sensorfidice
Cantar la canzone del? uccelline', Con tal dettato s* esprime un chiacchierone'y
che cicalando, faccia 'molte digreffioni spropositate per allungare il suo cicala~
mento con racconti assai sConuenevoli, che si dice; Entrare in un ginepraio
re di palo in frasca, elk
S* £ meffo a bortega, $'& preso per arte, per suo mefticro, o negozio. Quan
do uno fa qualche operazione con tutta applicazione, ed attenzione, e con dime
strazione di voler durare assai, diciamo; Costui sé meffo a bottega, oh
LEGGER sul libro d aicuno. Narrar le azioni, qualita, e stato d'aleuno,
NON cerca questi cinque soldi, Non cerca, non gl' importa, non proccuras
sapere questa cosa. Quand' altri fa un discorso, e fa una digreflione senza tornat”
più al primo proposito, se lidice: Voi pagherete la pena de' cingue soldi. Vedi fot-
to C, 8. stan. 15. E pero dicendo: Non cerco que/ti cingue soldi, § intende;non mi
curo di oes gnefta pena de' cingue soldi, con obligarti a seguitare il prin>
cipiato discorto. ai
SBIETT-A. Scappa via presto, Vedi sotto C. 7. stan. $7, 4
IL suo colui. Ui fao amante, cioè Cupido. Se
PER non Ws perder a' occhio, Perché non le esca di vista, Per non lo fmarrite. —
SEGVIT A la flarna, Quand' uno seguita un' altro per haver da lui qualche
favore, diciamo: Zifeguita la starna. E Gi dice la farna, © non altro uccell0s

























ie

SEPRTA TES ESS

SS “eS wt a

SAS

a

3. stan.s, Franzele.




paletettrorcs:

ne ilo
4 STANZA XXXL
— were i

ipeepet dot ons 3e ar fide,

'inttorna son piie delle pecchie;
eh soldayed a Suis faite udati,
Che havido del guerrier noticievecchie,
 Gliva incontro, Vaccoglie ye riverisce,
t a luicon DY arms 8 offerisce.
TANZA XXXil.
i, fazginnfe, ch? io si preghé
l'donna rimaner fernito,
a questo ferro lei stimpieghi
| Per conto qua Paces — j
t tanto Cavalier nulla si nieghi,

~Risponde acio Baldon tutto complito,
* Tu fei padrone; fa cio che tu vuoi y
Nom ci van cirimonie fra di noi,
Sita he
Over cl? io me La metta in [ul liuto,
“Ot veglia tener Poche in pastura,
"Come quel che ci vada ritenuto

Per mancanza di cuore, v\per paura,




ss QVAN TO CANTARE:
Q seguitarle,ofleruandole dove si posano,e straccanioie

239

STANZA XXXIIL
Ti servsro di feriverti alla banca,
E in tanto per adeffo io ti ceafegne
M1 gonfalon di questaciarpa bianca,
Che tra te schiere? il nostro cetraffegns:
Tal-che libero il paffo, e feala franca
Haurai per dar' effetto al tno disegno;
Che non fo qual si sia, ne lo domando;
Pero va pur ch'io refto al tuo comande,
STANZA XXXIV,

Ei lo'ringrazia, E ito più da preja,
Ove sia chinfo di Psiche il bel Sole,
Ad essa dice:In quanto al tuo interesso,
Fin qui non t'bo servito,e me ne duole,
Che tu non pensi, bavendati prome(fos
Ch' io facciafango delle mie parole,

E ch'il mioindagio, eilnorifoluer nulla
Sia fatoun voler darti erba traspulla,
XXXV.

Perché si come haurai date vedite '
Won ho fin qui trovata congiuntura
Di chi m' indirizzaffe qua al Castello,
Per porerne cavar cappa, o mantelle.

~ lo-con Psiche arriva al Campo, e chiede soldo: Baldone l'accetta, ¢
nza'd' ancare a servir Psiche, con la quale avviandosi verso Malmanti.

da,
£ Calagrillo'fi scufa di non' haver pri

'ima servita.

alla banca, Atrolare uno per soldato: Banca diciamo quel luago
ee soldati, e dove son loro pagatii denari ae fipend}:

7 (LONE. Vuol propriamente dire vefillo; ma
@insegna, Vediril Voilio de vitijs fermonis lib, 1. ove di qui

& piglia per ogni sorta
la voce.

CLARPA, E' una legaccia di drappo, che dai soldan fi'cinge come la cintu-
'ra della spada.. E per altro ciarpa vuol dire quel che accennammo sopra Cans.

Escharpe.

SCALA franca. Franchigia; Liberta d' andare,'o fare. Paffo libero.

I ston fango delle sue parole. Dilprezzare la parola data e non osservar Ie pro-

DAR erba'eraftulla, AMdetterla sul linto; miandar voche in paffara hanno tutti tre
verbie dare,

lo Reffo signific

ato, che € trattener' uno con Soa. Lat.
$1

NZA XX

Risponde Priche a questa'diceria:

| Lo non entro Signore in questi meriti y
Non ho parlato mai, ne che ti sia 2
So o spediro, o ver che tm ti peritis



Quel ke tu (fii, tutte tua cortesia,y
Per tal? accettose'l Ciel te lo rimeriti,
'Co darti invita honor,fama,ericchezza,
Sanita dopo morte 5 ed allegrenza.
STAN-
Se

240 MALMANTILE.
STANZA XXXVIL STANZA XX¥XVHE
Sta quieta, le dic egli, e ti conforta, Van eee @! occhiacci orlati di favore 'a
Chia vegtia adeffa dar fuoco al ve/paio, 03 addolfa & unt eratte ght [qnaderna

Così col Corno, il quale Al colle porta', Che par quand' il Faina alle se bare
lanternay.

Chiama la guardia, o vero it portinaio, In facia mi spalanca la i
Non e si prefeo il gatto in fu la porta y E mediante.un certo pirzz \ ee
Quand es sente la voce del beccaio; Chrei sente al colo, pizzicotti alterna,
Quanto veloce a questo suon la Ronda Ond? alle dita egli ha farsi i digali
Sopr' alle mura accoftasi alla sponda D' imorno a innumerabili mortali, —

Psiche rende grazie a Calagrillo della carita, che le promerte, e facendo le lor
cirimonie, s' accoftano al Castella, dove Calagrilio, (onande il Corna, chiama
la sentinella, la quale subito s'affaccia alle (ponde delle mura, oy

DICERIA, Vuol dire Ragionamento, Discorso, Orazione: ma
voce è usata per lo più per intendere Ragionamento flugchevole, € odi perla
lunghezza, io

NON entro in questi meriti. Non parlo di queste cose. Ma questo detto has
wna certa forza d' e(primere: io.non ardisco d'entrar tanto in la col discorso;mae
niera, che viene dal folersi dire; il merito della lite, o della causa, cioè limpor-
tanza del fatto. re

SANIT A, ed allegrezza dopo morte, E detto giocofo, perché un corpo mor
to non può haver fanita, ne allegrezza,ne altre paflioni. Ma si potrebbe anche
dire, che questa donna, parlando iperbolico, voglia dire che eglt viva fano, ed
allegro sempre eziam dopo morte, il che e imposhibile, come e imposibile viver
mill' anni, € pure si dice: vi prego mille anni di vita, Sanied-è un* augurio sche
corrisponde at Greco hygiainein, cioè far fano, che metteva innanai alle sue ¢pi-
flole Pittagora devotissimo della sanita; degrezza corrisponde a quel fal
in principio espri i Greci co} nelle lor | sperché dove i La
tini pongono Salutem dicit, essi scrivevano Chairein, cioè come tradufle Orazio
in una sua Epiftola Gaxdere, yolendo dire, Ii tale,al tale desidera allegrexea fic
come in quell' altro modo usato da Pittagora: il tale al tale,desidera fanita.

DAR es al a 3 Violentare a ulcir fuora uno, che sia dentro; come se+
gue, quando si da fuoco'a un velpaio, che le velpe son forzate dal tyaco a seap-
par fuori. Vedi Omero lib, 16, dell' Iliade, }

LA voce del Beccaio,. Vanno per Firenze alcuni Beccai, o Macellari yendendo
carne per dare a' gatti, e fanno certe lor voci così ben conoscimee da i ot
gatti, soliti havere la carne, che appena costoro hanno aperta la bocca, che i
gacti sono in fulla porta, A questi gatti assomiglia la guardia di Malmantile, che
a pena sentito il suono del corno s' aftaccia alla muraglia, Delle voci, e de'verli x
che fanno j venditori, che vanno attorno per inuitare il compratore, Senecacp. 14
56. lam libarij varias exclamationes,@& botularinm, & eruftularium »@ omnes pope M
narum infeitores, mercem sua quadam,& infignita modulatione vendentes $ 4),

t
”



SE




= 2.
seg



£ ia eS ezZenre eee

RONDA. Sidice quel Soldato di guardia, che rigita, e pafleggi Ia)
raglia della fortezza, vifitando la Sentinella, detca wa isi andsee incall ee
come i Franzefi dicono, aller en rond, stat

SPONDA, Parapetio della muraglia; Quel pezzo di muro, che avanza alle






Ss SSREREE BS TELA SSA TAs SAT. SS ae

 QVINTO CANTARE: 2gt








z del terrapieno, e si dice /ponda quel muretto, 9 spalice-
ilterreno, a i pozzi, a' fiumt, ec.

f di favure', Circondati di cispa per la similitudine, che ha con la cispa
co; E/avore e uno intingolo fatto di noci, e pane pefto, e liquefatto
+ sec eSnpemeef quell' umor craffo, che si conden(a intorno alle pal-
i sli occhi.

a Eicaaepccinns gli occhi addosso. Subito fifla sopra di lui gli occhi
« Equelto verbo /quadernare s'ula per divolgare, maniichare, ec.

pats Lv 33+
plires,: Cio che per  nniverso si fgnaderna

WN, Celebre Luogotenente di Birri così chiamato per soprannome ¢
ILANCARE. Aprir quanto si può una porta, un' armario, e simili: le-
aca, cioè il palo, che tienc in alcune porte fermato tutta, o unay
slia porta; aprire affatto. Vedi sotto C. 6. stan. 43.
ZILOTTO, #' uno stringimento, che si fa in qualche parte del corpo,
sliando la pelle col dito indice, e stringendola'co) dito pollice; e così facevas

intorno al collo, a/ternando i pizeicorti, cioè facendoli hor con l'una, hor

/mano per pigliare i pidocchi, che sono queghi iznumerabili mortali, che

ue loro gli hanno fatts i ditali, cioè ricoperte le dita; Che ditale inten-
diamo parte del guanto, che cuopre il dito.
— STANZA XXXIX. STANZA XXXX,



Non tanto s* abburatta per la rogna, Bu bis, bu by comincia, ch' il buon ciorno
(Epe brnfeol, che vanno alla goletta, Vorrebbe dar al Cavalier, ch' ei tiene

n dir non può quel che bisogna Ii Corrier, mediante il suon del Corno y

line feiligua ache abacchetta, Del popol d'Israel chor va, hor viene;

Qual ib quartuccio le bruciate fogna, Van le parole a balzi, e per ifforno
Nefenza quattro scofe altrui le getta, Prima cal segno voglian colpir bene;
Talfi dibarte,¢ a vite fa la gola Pur pinfe tanto, che gli venne detto;

volta ch! ei manda fuor parela, Buon di Corrier che nuovac'é diGhetito;

(ctive il Poeta la guardia, la quale havendo creduto che Calagrillo fuffes
un' ', lo faluta come tale.

S'2BBVRATT A. Si dimena: Si dibatte. Abburattare propriamente vuol
dire Separare la farina dalla cru(ca con lo stacciv.
BRYSCOLI che vanno alla goletta, Intende i pidocchi, che vanno alla pola.;
Goletta intendiamo Vestremita dell' abito da Huomo intorno a!la gola. Ed il Poe-
ta copre a detto con l'equivoco di Goerta, fortezza in Barberia, e con las
Voce bru/coli, che sono minutissime particelle di legno, o paglia, o simili, ed egli
TART AGLLARE, Intoppare nel profferir le parole; pronunziar con difficul-
i: e /eilingware vuol dir Balbettace..:
4 BACCHEITA, Ci dare a bacch vuol dire C dare affolut
Mente e dispoticamente in ogni congiuntura, come Re, o Capitano, che porti
scettro, mazza, o Raftone a wf 3 e di qui defi, che costui eile "
¢ (cilinguava ogni lettera.

LFARTYCCIO, Milura Fiorentina capace della arg aacaner partes

D flaio, e per lo più e un vaso di legno, BRP.






5s eee





















242 MALMANTILE) |

BRVCIATE, Marroni cotti arrofto in padella 5-0 in forn, o sotto la
FOGNARE, Fogna vuol dire quel vacuo fatto ad arte |:
paiia l'acqua, e si conduce feolando al fiume dal Lat, fovea:
mifwra vuol dir wetter la roba nella mifura in maniera, che
ma dentro vi sieno molti vacui, come facilmente fegae nel ia,
quale non si posiono flivare i marroni, i quali per cher di figura rotonda non,
ricmpiono lo spazio, ma fanno nacuraimeate, che rimangano fra l'und, ¢!'al-
tro molti vacui nella mifura; la quale poi, volendoli votare, io fquo-
tere; perché s' aftrontano nell' ulcire, € foqquadrano alla bocca. del quartuccio
in maniera, che non potriano (cappar fuort, se non Gi squotesse il vaso,ed uscen.
do, fanno un romore simile a uno che tartagli, le di cui parole pare, che non,
potiano uscir di bocca, se egit non si (quote, dibatte, o Sores equ aae
lo che egii mette fra uaa parola, e l'altra lo figura il vacuo che fla fra un
rone,¢l altro. E quctto intende col dire qual il quartuccio le bruciate fogna', cioè
fogna le parole con interuallo di tempo, e non di luogo.. Daa
EAR la gola a vite. Storcer la gola. Vedi lopra C, 2, stan. 9. atlas
PER frorno, Si dice quel ritornare indietro, che fa la palla che ha
nella parte opposta dove e fata rata o sia muro, © sia altro, ed termine,
prio del giuoco delle pallottole, e s' intende quand! yno tira per accoftarfial se
gno per via di detto florno, e non direttamente: E così indirettamente
di bocca a costui le parole. Ln somma vuol dire, che egli impuncava nel parla-
re, tartagliava, e parlava a falti. valighelitya
GHETTO. Così chiamiamo il Serraglio, nel quale stanno in Firenze, ed in
altre Città gli Ebrei: E perché questi hanno nome di tener di mano afregheric,
pero dice che il Corriere di quel luogo è solito Ipetio andare a Malmantile a to-
var la flregha Martinazza.. Ghetto e voce Caldea, che significa libello di
dia; onde noi diciamo Gherro per intender luogo di gente fegregata, ~~.
dal commercio degli altri huomini. Gli Ebrei quando vogliono dire loro
mogli, che le gaftigheranao col repudiarle dicono.; Ti manderé al Gher. a
STANZA XXXXL STANZA XXXKIL
Rispose t altro, tal parola udita + 41a che vo il tempo qui buttande vid
D' elfer corriere già negar. non possey In disputar con matti., econ buffoni
Perch' io Pho corsa afar questa falitay 4 trattar teco credomi che sia ~~
Ata quato al Ghetto ia ndlavoglioaddosso; Come a' Birri contar le sue ragioni;











Non ho che far con-gente Ifraelita.; We diffi mal, perch hai fifonomia
Benti fara il mio brando ilcappel roffay D' un di color, che cinffan pe' caltonk y
Ecol darti [ul vifo un soprammano El' cffer tu costt, par chvella quadriy

D Ebreo fara mutarts in Siciliano Ch' i Birri sempre van dove son ladri.
STANZA XXXXIIL Joa
Dell! alma fala quei:son soddisfatti'y
Aa, v0i col corpa la portate wa,
Hor bastasfe bs voi tant' odio corres
disporre.



Bench? voi fiste come cani,egatti,
Ch'effi non han com, voi gran simpatia y
Perché peggia de' dtavol fete fatti,
Vande nel pighar pis tiraunis. 5 Mezlio,a-i.lor danni ti:posras


i eh he ke le SE had

}



eae tn i

Sessa BPres &
>

243
STANZA XXXXV.
La frefto-devi oprar 5 &° a tei sia farto;

et cui(perch ei confente in tal baratro
ag porrebbe far le fufa torte;
@i si cerca efstr mandato Hn tratto
Salt' afin con due rocche dalla Corte,
Si che, [tu nol fai, ts rappresento,
C” un difordine qui ne puo far cento,

XXXVI.
t Mentre pero Cupido non rimetta:
non impiccate questa Troia, Ma se la rende non vi do più nota,
vK Pigharmi questa derta Va ditone,enarra a lei quanto tho detto,
i Birro,e in fulle forche il Boia, Chri gui rattendo,e la risposta aspetto,

Sadira Calagrillo, che colui l'habbia preso in cambio del Corriere degli Ebrei,
¢lo minaccia di rompergli ia velta, e sfrepiarlo; e dopo havergli detto molt im.

Oper) » gli ordina, che da sua parte avvisi Martinazza, che renda Cupido; al-
titi fara render per forza.

LHO corsa.. Ho fatta questa cosa senza considerazione. Quand? altri fa qual-
che risoluzione, che non riesce poi buona, di¢iamo: E# l'ha corsa dall' armeg~
gist, e'dalcorrere la gioflra. Similmente diciamo; Fare una carriera. Qui fa
giuoco la voce corsa, che e cosa da Corrieri. si
NON la veglio aitdoffo. Non la voglio fopportare, Si dice anche non /a veglio in
sul giubbone.

Gente Israelita, Intende Ebrei: Popolo d'Israel.

IL cappello rosso. Gli Ebrei in Firenze portano per contrassegno il Cappello
rosso. Il Poeta dice, farò ben' io diventare Ebreo te col farti il cappello rosso col
sangue. E poi d'Ebreo ti farò diventar Siciliano tagliandoti il viso, ed intende
quel Siciliano Montambanco, che per accreditare il suo Olio da Ferite si faceva
agg persona, e con esso se le medicava.

AUMANO. Quel colpo, che si da con spada, o bastone, comincian-
do da alto, e calando a batlo., Vedi sotto C, ro. stan. 52,:

SVEPONE » Vino che'fa profeffione di trattener la brigata'con facezie.
DIR Me sue ragioni a Biri, Raccomandarfia chi non pwd', € non vuol far
servizio'; anzitha caro'il tuo male. Vuol' anche dire discorref con und, che nons
a 'ta dica;o vero buttar le parole al vento, Plauto disse nel Pfeudolo;
spud novercam queri, '. t
CHEFAN pecalzont, Ciov i Bitri; i quali pigliano pe' calzoni. [1 verbo
cisfare hha del furbe(co; €*vuol dir Pigliar con prefa Mabile, e buona, come &
feleste ate, pigliando uno per il ctuffo, cioè pe''capelli. Petrarca', Le san
'4vef? jo avoolte entro a' capegli, fn
“ESSER come vani,egatts, Eller poco d' accordo, o poco uniti, anzi fempres'
fhimi¢i,come naturalmente fonoi cani, e i gatti. 2
NON ha gran simpatia. La voce fimpathia Greca fatta Toscana significa incli:
hazione scambievole, o similitudine di - » di voleri, e d' aftetti,
. amity 5: 5:

eu AE.

Mentr'a tofitinonrendailfnoCoforte.




























ve
244 MALMANTILE | —

MAESTRO Bastiano, Intende il Boia, che allora cos
era stato Mieftro Biagino. Vedi sotto C, 6. tan. 56.
LETTO a tre colonne, Cioè le forche,le quali veramente son
una stanga sopra a traverso, ed in molti luoghi sono, ey:
LAVORAR di mano, Vuol dir rubare. scherza dicendo, «
( cioè il Boia ) perché essi ricevano qualche riposo da tanto lavo!
gil mette in sul letto.a tre colonne pee in (ulle forche ) ed in fuftan t
gl' impicca, perché son ladri, E Calagrillo, seguitando l'equivoco del ri
dice aila guardia, che f¢ ella ha punto di pieta,¢ discrezione, dovrebbe
sto riposo in sul letto di tre colonne a Martinazza per il suo tanto
impiccarla, perché¢ ladra, I Latini pure per dir copertameate
manu finifira uti secondo Catullo in Afinium. ' sey
Marrucine Afini, manu finiftra
Non belle uteris in ioco y atque vino;
Tollis lintea negligentiorum.
E per dire eopertamente Impiccar'uno,dicevano; diteram longam.
bsamo notato altrove.
NON cede un grano: Non cede punto. Che grano si può dire ana
inconsiderabile del pelo, poiché 24. grani fanno un danaro, 24.
V oncia,e 12. once fanno la libbra.
NON uccella a pispole. Non si cura di confeguir cose di poco mon
è fra gli uccelii la pispola. 1 Latini dissero vom capeat mujcas.
FAR le fufa sorte, Far le corna. Vuol dir quand' una donna si Of
altri huomini, che col suo marito. I Burchiello Poeta capricciofo, VAs
sotto nome d' Accademico Fiorentino incerto, nella Raccolta delle Rime Piace-
voli del Berni, Casa, ec, i.
Non ti fidar di femmina, ch' è usa si '
e4 far le fufa torte al suo.marito,
Il Berni nel suo primo capitolo dell' orto dice:
E finalmente non fara mai fufa
Donna alcuna per lui torte al marito; Ky
Si dice fu/a torte per intender copertamente Corna.;
MANDATO con due rocche in full' asino. EB costume in Firenze, al
dclitio del pigliar più d' una moglic,aggiugnere una dimostrazione
che e il far' andar' per la Città il delinquente legato sopra ad un' afino',
mitra di foglio in capo y ed a-cintola due, o pill rocche inconocchiate, che
ficano le due, o più mogli. ses eeh
, QUEST A troia, Questa porca. Epitcto vituperofissimo nelle donnes
ynoldire Laida meretrice:: nell' huomo non € tanto ingiuriofo i). dirgli
Ml x0 pighar questa detta, Vud pigliarmi V'aflunto di far questa
derra vuol dire prometter per un' altro, o star mallevadore, cioè di far'
cosa, fenon la fara quello, che  priacipalmente obbligato. Comprar.wna 4
yuo] dir comprar un' avviamento, un credito, ec, Derta & dai plura
Devitt»

















fucere



\





















245
STANZA XXXXVIIL
Lascia la sentinella, e caracolla
Gin pel castello, dando queste nuova
E benche il Adaggioringo della bulla
Gi habbi est,r ech'ei si mova
: 'Di fargii porre a' piedi la cipolia,
r non possa dele pacche; Cercando della morte in bella prova
to havendo si Ciel turbato Vuol avvisar di cio Mona Cofofiola,
¢i par un porcellin grattato, Ch' è per bafire a questa battifoffola,
» che ¢un vero poltrone, sentendo le bravate ai Calagrillo, zitto
/¢ tremando va a dare questa nuova a Martinazza.
RDA l'armi dalle tacche.. Non vuol cavar fuori la spada, per non la,
:. Intendi che costui era un codardo, perché per dir copertamente pol-
un soldato, se gli dice: Rispiarma foderi.
i Facche. Parole latine corrotte', e ridotte in una, usaro assai dalla
- plebe nte per intendere Andare in faluo,ed è il Latino 4d asylum confugere,
Rl LEV-AR delle pacche.. Bulcare, o toccar delle ferite, che questo intendiamo
hey ma ¢detto plebeo. Li Vocabolifia Bolognefe dice che pane significa per-
agliarda.. La forza di questo verbo rilevare vedemmo sopra C. 3. stan. 67.

hi stor, Fiorent. lib.6, dice 4/ figlixole del quale nominato Lorenzo,rilevo unas
3








IDO veduto il Ciel turbato, Havendo conosciuto, che costui era in col-
lice' anche /4 marina turba,
iB che pare un porcellin grattato. Similitudine assai usata per intender uno,
| nrispenda alle grida d' un' altro o per paura, o per riverenza, o per lao
coscienga macchiata »© per altro; e si fa la comparazione al porco, perché il
Porto che firide,grattandolo si quieta, ed i porcai gli rendono maneggiabili col
Peg
OLLA, Il verbo caracolfare vuol propriamente dire Volteggiare col
na non ofante qui torna assai bene per esprimere, che costui per las
laffle girando per il caflello, non gli parendo trovare luogo sicuro. B
anche in uso caracol/are per camminare a piede,volteggiando d' una strada in
altta,¢ diciamo far un caracel(o per intendere una girata. Viene dalla voce
[ Spagnuola caracol » che ynol dire chiocciola, '
ene deta bolla, Termine della lingua furbesca 5 che in Firenze vuol
il Fiscale 3 ma s' intende per il Superiore in quegli affari di che si tratta, Va-
> il Maggiore della Città, chiamata in quella lingua Bolla dal Greco Polis es
arbaricamente, Polla,
| FARGLI mettere a' piedi (a cipolla,, Fargli troncar la tela, e mettergliela a
i: come si costuma in Firenze quando, il cadavero del giuftiziato dee Mares
elposto per qualche ora al pubblico; che gli mettono la testa.a i piedi.

PER bafire. Bi per tranfire 5 per (ucnirsi 5 per morirsi. Vedi sopra Cant, 2,
NA Cofofiola, Nome usato pet intender una donna faccendiera, aftanno-
Ofudatora. Scbbene Ce/osso/a [ secondo ul Varchi nei suo dircojano all. voce

4: Batti.







|






Rartifoffiola 1810 Reflo che batrifofiola, e significand affanno >)
mento grande, ma breve, che cagioni battimento di cuore, 0
il che si dice fofhare: Franco Sacc, Nov. 44. Ad' bai data eos),
io non [ard mai più litro', @ forse me ne morro. Non credo che fit
quello che dictamo fepraffalro at exore; lo stesso che batticudre y affanno
to per paura,o dolore improvviso dagli Spagauoli detto,/obre/

| 246 MALMANTILE™
i
Corn, Tacito lib. 5. dice: Exterrite unt acri magis qudm dinenrno rimore, Bail n0-

i stro Davanzati parafrafando queste parole dice bebbero batrifoffia, woth

| STANZA IL. STANZ Ay Be

I Ella insieme le schiere ha già ridotte Atentre del farto poi le da context' ake

! Di genti, che non vaglionoun piftacchio, Com quell'ambasciaye ling un di frullone
Cive di quelle, a cus fece la notte Fa ( perché nulia mai fora Se

} Col [uo carve si grande spauracchio, Chi lo sente morir di paffione;
Ed hor quivi parare, e dar le botre Ma quellasc'a fenrirlo> forse avvertAy
Insegna lor, che non ne fan biracchio Li intende un porcost i/ is è
Ma quand innanRi a lei costni, ififerma Equi fnifoom le legion di wera y |
Cos} tremante,la cave di scherma, Perch'ella ni'da-pin:ne inCielneinverra,

Martinazza stava appunto instruendo quej soldati, che s evan -faggiti per pa
ra de' suoi Caproni, quando arrivd un la sentinella con Panbotetara a Gale
grillo, che la tarbd cutta', ond' ella la(ciò flare il darlezione, si
NON vagliono un piffacchio. Non son buoni a nulla. Si dice un piftacchio yun
lupino, una lisca; una forba, una lappola', un pelo yun baiocco, wa baa,
un picciolo, un zero, nn' ctte, un fico, cica.y un iota, una chiarabaldana, ui
puntal di stringha',o'd' aghetto, una fucciola, un soldo:,.an quaterino, un-cor-
no; tutti per e(primer'la poca flima;che si facia d' uno, o d' alcuna cosa, Eft
dice anche: non lo stimo 1) cavolo a merenda, Latino cicwm, titiviliitium ~~
SPAVRACC HIV, Significa quel che accennammo sopra C. pr. stan, 40, Bab
li si dice fare /pauracchio a uno per intendere spaventar uno, o mettergli pauras
con fatti, o con parole.,
NOW ne fan biracchio, Nomnefanno nulla. Si dice anche straccio, brand, 0
brandello, e simili, id
! CAV-ARE un di (cherma, Vuol dire far perder il filo del discorso a uno, ed &
| lo stesso che cavar di'tema.. Ma qui: vuol dir' anche far lasciare far di-
re, e torna bene, perché Martinazza.la(cio la scherma', ed usci di tema, ¢'
proposito per l'ira, che'le cagiono  ambatciata fattale in'nome di Calagrillo.
eAMBASCIA, Attanno, o difficile re(pirazione d' alito, Fran. Sace, N.139
T ofto colui di-chi erano frati., (en' andò con l' ambascia della morte a ripigliarl. oh
LINGY-A di Frulione, Cioè che parla a salti, o a intoppi'y comeeil rumore}
che fa il frullone, che e quell' ordingo, col quale; per via d' una»ruota:
fepara la farina dalla crusca dsmaxy
NON raccaperza nulla, Non intende nulla. Vedi sotto C, 6, stan rot.”
LP INTENDE per discrezione, Quando per' altro ci & noto un-negozio, ¢' che
taluno ce lo racconti confufamente, o lo scriva con cattivi, e non intelligibili'ea-
ratteri, sentito,o letto da noi, fogliamo dire; 2 habbiamo inze/o per diferezionts
cioè habbiamo havuto la discrezione di non gli far ripecere il discorso cacy

=

a



streeceee hag feet

we ei



2A# FEZ















se”

&

= Sas * BS

id

A= 2% BE

SF

 -

* mente dalla vergogna, la quale però si dice anche erubescenza.

— PE He.




247
quel fat-




er qualche: informazione, che havevamo di
orso, © scritto:. i
,we im terra; E? fuori di se, Non fa quel che ella si faccia.
piel 5 we terra; dissero anche 1 Greci in questo proposito; e l'ula Lu-
Plendamante, ovogliam dire F aifo indovino,
ANZA LL STANZA LIL
wedefi cambiare Rabbiofa, it capo versa il crel tentenna,
Quasi col piede il pavimento sfonda,
tor figratrale chsappe,jvor la corenna,
Hor dice al meffaggiero che risponda,
Hor larichiama metr'egli in Chiarcna,
Grida,¢ minaccia,e par che ficunfond,
Hille disegni tro al pensier racchinde
Lenne inne ye nulla mai conchiude,
LULL.
Che lavandole il collo lordo,e intrifa
Laghi formano in fen di poxxi neri;
el fin tornara in se,con la gonnella
Yi come fonagli da (parnicri, S' ascinga,e al meffaggier così favella,
Narra gli accidenti, ed i moti diversi cagionat: in Martinazza dall' ambascia-
tadi | ed in fine Martinazza s' accinge a dar la risposta. L' Autores









deferive.. per-una folenne sgualdrina poiché dice, che & così grande il
'udigitme che-clia ha-addotio., che le Jagrime che le cascano dagli occhi fanno
)arerle nel collo tanti laghi di pozzi neri, cioè di cedi, i quali ella s' alciuga












BIANCA come il mia collare, Diventa bianca comie un panno curato, E que-
te mutazioni di colore son proprie d' uo che habbia l'animo alterato si in ma-
€, comein bene, perché la palidezza, e sbiancamento deaota follevamento d'a-
dimo-non essendo altro-, che un mancamento di (angue, il quale per la paura se
ne fugge al cuore, e lascia le vene del voito; ed il roflo denota ira percht questa

tibollimento di sangue intorno al cuore y che scorre per tucce le venc,
Ma apparisce più nella faccia, perché quivi (ono molte vene intercucance, o vo-
gliamo dire im pelle,.che faciimente lo scuoprono; ¢€ lo: stessoeffetto viene pari-










DOPO ch cgit ha toccata una spogliazza. Dopo che egli è stato fiuftato in sul
©...,dal maeftro. Spogliazza quali expoliatio, spogliagione si dice quando il
Macftro fa cavare i calzoni a uno (colare, e mettendolo sopr' alle palle d' un'
altro, gli dd con la sferza in fulc..... E quando gli da nella (tela forma, ma,
senza i mandar gill i calzoni si dice dare una mula, o un cavallo. A questo
&,.,. fruftato assomiglia |'Autore il vifo di Martinazza quando le diventa rotio,
Vna simile spogliazza, quasi come a ragazzo insolente, © minacciata la nel se-
dell' Liiade a quel brutto moftaccio di Terfite, a cui Omero [ secondo la,
one Latina ad verbum del Gifanio } fa dire da Vlisse: Ne posthac Viyfi
apne bumeris adfit, Gc, Si non ezo te comprehenfum, & charts veflibus exutum Paltio-
que, © tunica, que pudenda contegunt., Flenvem veloces ad naves dimifero, Cadens ¢
Concioue duris verberibus. TEN-














f

eae

EE ————— ==

—ooeee






248 MALMANTILE! 79

TENTENNA il capo innerfo il Cielo, Dimena la testa verlo il
§ fa da molti quando accade loro cosa di-poco gusto, quasi voglia
il Cielo perché cagiona loro quella tal di(grazia: i Latiabdissero; caput 9
SEONDA il pavimento col piede. Per la collora batte i piedi in terra
mente, che fa quasi rovinare il Palco. Properzio. Et «repitwm dubi
ede. tf
ST gratta le chiappe, e la cotenna, Si gratta le natiche, il capo chee
to solito farsi per lo più dalle d6ne quando succede loro qualche disgrazia,Pei
vas' intende il capo, perché la pelle del capo-dell' huomo si dice cotenna;
vuol dire la pelle del porco, ed impropriamente si dice la pelle d' ini
vedi sopra C. 2. stan. 64. ed in ciò noi ci conformiamo co' Latini, che cuit
ta pelle del capo dell' huomo, e dicono anche cutem detrabere per scorticare qual-
fivoglia pelle, il proprio vocabolo della quale è pellis. amd Ae
QLVAND?' egit e in chrarenna, Quand' egii e molto lontano.. Zp oras “
¢ da questo noi diciamo: Quand' ezli e in orinci. Viato dal Davanzati nel.
J ENNE inne, Di questo termine ci serviamo per esprimere uno che L
di operare,e non conchiuda. Viene da quello stento che fanno i ragazzi r
imparano a compitare; quasi dica compita compita, e mai rileva., ed halo fle
fo significato, e forza che ponza ponza detto sopra C, 4. stan. 80, sea
SON*GLI da sparuieri, Intende lagrime grofie come sono i fonagli, che s'ap-
piccano a i picdi degli sparuieri;comparazione ipcrbolica,ma assai 'inten
der grosse lagrime.Ain.1 1.41 lacrymans.guttisghumettat gradibus acon
nels chiamiamo quelle gallozzole, che fa l'acqua quando pioye, cadendo sopra»
i rigagnogli; o altrimenti neilo scorrere., '
2022/1 NER/. Bottini. Quei luoghi forterranei, entro a' quali Anraes
sorta d' jmmondizia; ma propriamente pozzo wero è bottino, o fogna
del ceflo, a differenza di quella degli acquai, i
STANZA LIV, STANZA LV. >
Torna,¢ rispondi a queffo Scaizagatto y Pero s in questo mentre umor non varia y
Che si crede ingoiar con le parole: Domani al far del di facciami mottos









































Ch'io no fo quel ch'ei dica,e s'eglié matto E s'io gli faro dar le gambe all! aria
Won ci posso far' altro ye mene duole, Quella sua landraba da pagar lofeette,
Poi circa alla domanda, ch' egti ha fatto; Mia se la forte fofse a me contraria
Che gli daro Cuptdo,e ciò ch' e vnole, Vuol 'a me tocchia adar col capo roti,
Se con la spada in mano,o ver co laa #renda Cupido allor, ch io le prometto
Prima di guadagnario,il cor gli basta, Lasciarglielo segnato, e benedetto, —
STANZA LVL
Ciò detto partese quei chieralbnomo e/perto Ed in vifo vedendolo scoperta 5
( Essendo lato Cavailaro,e Atefo ) Ruch' ha bifegno dice d'un buon lett y
el Cavaliere ad unguem fa il referto Perch'egli è duro,e non punto pupilles
Di quel che Martinazzagh ha comeff[o, Lo conosco bensì, gli¢ Calagrillas —

Martinazza manda a dire a Calagrillo, che gli dara Cupido, s'¢i lo gl
gnera con! armi; ma se ella vince, vuol Psiche: la ronda porta l'ambaleiata,®
riconosce Calagrillo

SCALZ 4GATTO, Huomo vile, Guidone.
























TO CANTARE. $40
ral eeenesircl neha con Ie chiacchiere'. E si dice:

teu varia Se fra tanto non si muta d' opinione.
ina, donna di bordello, ed intende Psiche; Landra è epi-
pinfami, e Jaide meretrici, quali datrina, che la fogna,

ro. Hada pagare la pena. Pagar lo scotto vuol dire pagar
sé mangiato, pagar la sua porzione, la sua quota; Tercnzio
+ Ma qui intende il Latino penas mere, Dan, Purg. C. 30.
» L) alto fato di Dio farebbe rotto
Se Lece si paffaffe, e tal vivanda
Fuffe gustara fenx' alcuno scotta
pentimento, che lagrime panda,
è « Andar con la peggio; cioè ch' io perdeffi il duello;
ATO,e Back. Liberamente,¢ senz'eccezione alcuna.Fran.Sacc. Nou.
e ogni hora pur Segnato, e benedetto, Esprime un dar via qualcofa, o
uno volentieri, e con anime di non rivolerio; Vn licenziare af-
Peecceit vale,ingust Iola,

. BE' un famiglio, che porta le citazioni criminali mandate da
s Trent Cavallaro, pecché stante il largo dominio, e giurisdi-
il suo tribunale, e necetiario che vada a cavallo; 4 Adess0 € quello
eitazioni i pure de i Ministri forenfi, e fa i gravamenti, ec. es
lo, perché non gli occorrono lunghe gite, come al Cavallaro; a
panda Cxrfore; nome simile al Viator,col quale era dilegnato dagli an-
iil donzello,o fante pubblico.
hem» Per appunto. Frafe latina usata assai da noi.
“CARA cater. Riferilce. Frafe curiale, che vuol dire quando il Cavallaro, o

i data la citazione, riferisce in atti d' haverla data, che dicono an-
rapporto. Ev Autore si serve ee frale ( per altro non usata in,
i) perché ha detto, che questa Guardia era stato Cavallaro,e Meffo,
tbisogno d'un buon lefso, E' carne dura, e pero ha bisogno di bollires
7 Sore paaglonds EB detto vulgato per esprimere un' huomo, che sa il conto suo,
ye difficile a superarsi, che diciamo: O/so date per esempio; Il
“cies @ rodere un' offo duro.
+ Non ha bisogno di Tutori, fiona lo stesso che ha bifagno a un
brian è pupille si rifttinge a saper tare i fatti suoi, ed ha bi/orno
oe e(prime saper fare i fatti suoi, ed esser bravo,e valcate in ogni










'=
:

STANZA LVIL
wi tie dame Calagrille vesti, Che seguitaron come voi inten defi
giorno rivedremg li poi. Periun, che (en' andò pe' fatti suoi,
ekeongeon apprefti Che trovereme ti, se venir valere

Per ginger it 7 Fendefi gli altri duoi, Pin presto afsai di quel che vi credete,

oH STAN.

ee <

Digitized


























”

250 MALMANTILE.
STANZA LVIIL
Che zio cio se ne vanno git pel piano
Shattuti com? io diffi dala fame;
114 non son iti ancoraun trar di mano A 'per soddisfar s.
Che fenton razzolar fra certo firame;  ——-Hla fatto in quattro di Pil





Percio con b armi subito alla mano El: con la sua spada se

Corron dicendo: Qui c'e del bestiame, Delt' honor della quale

Si che quando crediamo ditirar minze, Che havendola fancinila

Ji cor pu forse caverem di grinze, Non gli par ben ch' ienuda

STANZA LIX. STANZA LX

Curtofi quei che sue di vedere Ata perché un huom pix vsl mas,

Dentr' a una fialla inabitata entraro y Si pente esser'entrato in talc

E vedder,ch'e: a un'huom postoagiacere Pero che a fearvi folo egti ha,

Sopr' alia paglia a guisa di fomaro; Che non lo porti via la Trentan

Accanto havea da mangiare,¢ bere y E perché tutto il giorno quant'es

E gli occhi distiliava in pranto amaro, Egli ha il mal della lupa, che

E trai disgufti,e il vin ch' era quifite  Non va mai fuor #4 cinta







Pareva in vifo un gambero arrospico, DL? ascioluar col suo fiasco nell
STANZA LXIL

Ovungne elie, d'untumi fa un bagordo, Aggira il beccafico,¢ pela il
Ch' ognor la gola gli fa lappe lappe; E a poveri cappon ruba le
Strega le botti di lor sangue ingordo, E prega il Cielche fac






E le fuftanze usurpa delle pappe; Quant le melagrane,
L' Autore torna a parlare di Perlone, e degli altri, che Jaicio sopra si
28., i quali per la fame s' andavano ailontanando dal Campo,e¢ narra 5
floro trovarono in una Capanna quel Piaccianteo, che fu da Bertinellan
fuori a spiare, come vedemmo sopra C, 3. stan. 45. il quale haveva seco
giare,e da bere. Nella presente Ottava 62.de(crive assai vagamente la)
nia di Piaccianteo, 3 ere
G/0' cio. Adagio adagio. B' la figura aphere/is. nf ist
RAZZOLARE, Fregare, raspare, fragare; ec. Qui vuol dir quel romor
che fa la paglia, o cosa simile, quando e maneggiata in mafia. ae
STKAME, Paglia, fino, o aitra materia simile per cibo delle bestie +
sopra C, 4. stan. 2. a
TIRAR minxe. Vuol dite stentare. Ma s* intende moriré: Si dice milzi
il Poeta si serve della licenza, e seguita intanto i pil che dicono; minga
milzas
C AVARE il corpo di grinze. Mangiare assai, che in questa maniera gonb
il ventre s si levano le grina¢ al corpo. Plauto disse ventrem diffendere
Georg: distendunt netare cellas, cioè empiono « Paes
PAREV-A un gambero arroftito, Bra rosso in vifo come sono i
ae assai usata per esprimere un rosso in vilo, per il
evuto,
ALA fatto Sillidé mia, Ha finito, ha consumato, o mandato male tutto |
havere. E' detto fanadatsico Filide per fine, Ma per avventura hala fa 0















bands 2
Lia

© eae anges






 -QVINTO CANTARE, 251

figliuola di Licuego Re de i Traci, la quale s! innamoré di Demo-

di Tefeo, e di Fedra, quando nel tornare dalla guerra di Pera
) stato spinto da i-venti contrarj nel Regno di Tracia, fu da Sillide rice.
on segni di grande amorevolezza; ma egli senza riguardo a i benefizzi das
efla ricevuti, fen' andò; per lo che Sillide disperata s' impicco. Da questa dispe-
'rata morte di Sillide, quando diciamo far Fidiide, intendiamo finir la vita, e fini-
re!






. - ¥
 IMPLATT ATO, Nacoto', Vedi sopra C. 2-fan. 60.
DELL! honor detla quale ha gelofia. Ha gelofia dell' honor della sua spada, per:
he havendola tenuta sempre fanciulla, cioè vergine ( che s' intende now mai
perata ) stima poco honeflo il lasciarla vedere ignuda, come è veramente po-
@una vergine lasciarsi vedere ignuda. E con tali scherzi vuol dire, che
codardo, e vile,¢ di poco animo, ed uno di coloro che wmbram fam

'ANC ANNA, Vna beltia ch' ingoia o tracanna trenta per volta;
/è una di queile Jarue immaginarie inventate dalle Balie per far paura a i bam-
'a come bau, befana, e simili dette al trove.

 dL male della Lupa, £ inteso da noi per una infermita, che fa stare il pazien-
te in continua fame, ed i Medici-la chiamano fame canina. F
¢ | CHElofeanna. BE' un termine che significa grandezza di paflione, ed ha forza
1 davanzare jl superlativo, perché dicendosi, Ha ana fame, una fete, un desiderio,

tc. che le feanna, s' intende fame, sete » o desiderio grandissimo, e più, Vedi fo-

ei = praC.4. stan.2z4.
ASCIOLVERE. Solucre il digiuno; sdigiunarsi, fare colazione. Vedi sopra
“ stan. 35. ma qui è preso per mangiamento in generale, cioè per la materia

A

Hi Tad! Intende roba da mangiare, che fiaunta, come polli, carnes,

yg
a ec,
' 2 ME -

. ' 'BAGORDO. Bagordare, o far bagordo vuol dir Gioftrare, giuocar d'armi,

+ far conviti, ed ogni altra sorta d' adunanza feftiva, ancorch¢ non d'armi. E
o potrebbe dirsi (cherzando bagordo, quasi vagus ordo, confufione ordinata; onde
a di gente in confulo » la quale interuiene a tali bagordi, Pigliamo
— ~pol do per commiftione di varie cose, come nel presente luogo, che intende
 mescolanza d' untumi. Vedi sotto C. 6. stan. 2. Del refto Bagordo viene da Zi-
tite vuol dire eda. E Bigordare trovafi presso gli antichi; per corger la



» Fazio degli Vberti nel Dittamondo al Canto 32,
; Giovani bigordare alli chintani,
w hig E gran tornti,e una,e altre Gioftra
Bete 'i Farsi veder con giuochi nuovi,e (rani,
"Poi si disse Bagordo,¢ Bagordare; e si trafiero queste voci a Ggnificare ogni sorta
# 'di stravizio, e di ricreazione. Che Bigordo voglia dice eda, ciel esempio di
Giovanni Villani lib. - rubric. 132. £ recoffi patio di drappu ad oro fupra capo
i Helfer Amerigo di Nerbona portato sopra bigerdi per pix Cavalieri, Eclgo-
è z da San Gimignano Rimatore antico citato dai Conte Vbaldini nelle Annota-
' Meficr Brance(so da Barberi an + Brompere, e ficcar bigards ye lance,
ee LA














23s MALMANTILE) | 5

LA gola gli fa lappe lappe', Sigaitica detiderar ardentem
nate dal cao chet il palato con la linge £0 VES
za havere nulla in bocea, che e segno di » qual suono pare.
lappe; donde poi il verbo al/ampare, che vuol dire haver gran fam
in Greco, che è lo stesso, che Lambo in Latino, & fatto dal medesim

ST REG-A (e botti, Stregare vuol dir fucciare il sangue, perch dicono,
Streghe ficciano il sangue a i bambini; e però dicendo frega de borts ini
cia 1 sangue delle bottt 5 che e il vino y del quale e éxgordo, cioè aviditlimo

VSVRP A le (ustanze dele pappe « Divora la carne, che ¢la foftanza del
del quale si fanno le pappe. 2 nlohetoendt

cAGGIRA il beccafica, e:pela ibtords, Aggirare, € pelare,metafo!
lando, significa ingannar' uno, e cavargli da dosso danari,; come habbia
nato sopra in questo C. stan. g. Li Poeta scherzando pigha decti due verb
vero fenio, ed intende girar nello spiede i beccafichi, e pelare i tordi p
cergli s e mangiarfegli. '

LiVA ie cappe ai capponi, Cioè divora la pelle de' capponi.

E PREG A il Ciel che faccta, che gli agnelli, ec, Dove git agnelli hanno se
te due granelli, (cioè tefticoli ) vorrebbe, che ne haveilero' quanti n' hs i
melagrane. E così descrive un folenne ghiotto; e crapulone. Similmente un Cet
to Filofleno folenne mangiatore 5 siccome 'riferitce Ariftotile lib, 3. delle Morali
indirizzate a Nicomaco, cap, 10. desiderava d' avere il collo più lunge d' unas

'i ate

























grue supponendo, che così fuffe per essere il gusto maggiore, tae:
STANZA LXIiL comanaal L ake
Vedenda quini comparir repente E quei soggiunge: Adi rallegro,e Gedo ~
L' infolite cae shigortisce il ghiotta, Che ee | facciate bene,e vi son febiave;
F dal timor ch' egli ha di tanta gente Ma s' il patire e fattoa yoo y
Trema da capo a pic, si piscia sotto: Penitente di voi non e pin bravo,
Con tutto cio digruma allegramente, Tal ch'io per mevi mando a

E spefjo speffa bacia il suo barlorto 5 Non nel fettime Ciel, ma
E accio frremara non gli sia la vita Donde ai midani,ea meche[owoileapi,
Non dice mendegnateo aber gli invita, Pisciar potrete a vostra polkain capo,
STANZA LXIV. STANZA LXVL
e/a i Cavalier famofi a quel plebeo s Ata perch al certo Vostra Reverenza
Che nou proffer: lor della rovelia, Ch' è frenuata, come-un Carnovale;
Furon per infeguare il Galatea Hanra fatta fin' hor tant' affinentay
Con batrergli gik in terraiimama/cella, Che basti a soddisfar a ogni gran tally
Chi fei? ( difs' un di loro)e Piaccianteo, flor puo lasciar a noi t
Chie xa pover huorispode,ein quella Cella dccio baciam ta terra r
Molt? anni in aftinenza ha consumati Per piit mondi accofharsi aquest avans
Per penitenza de' suoi gran peccate, Delle retiquie sch' ell' ha gi eek
Piaccianteo vedendo comparir coloro armati, hebb'un lef
non per questo abbandono ii mangiare, anzi si ——— peril
lomandato














 haveva, che coloro non gli stremafiero la provvifione.

era,rispote eller uno, che faceva penitenza de'fuoi peccati ia quella cella
caitinenze: Dalla qual risposta accortifi, cheegli era un birbone 5 |

*






NTO CANTARE: 253
Bli-dice, che lasci un po fare il medesimo digiuno,

4/Si perde d'animo. Vedi sopra Gastan8.Dan,
Così mi fece sbigortir lo ALefro,

A'S i gli vidi s} turbar La fronte;
9 Golofo; Avido di mangiar del buono. Lat. ¢éuto;
'nol dire haver gran paura. Vedi sopra in questo C. stan. 3.
, Intendi mangiare; se bene digrumare e il matticare, che fan»
più feflo, che si dice anche ruminare dal Latino, che perdchiama
e dette — come habbiamo accennato sopra C. 4. stan. 6.5 e ve-
9 sorta.C, 6. fan. 5.
ACIA il barlotto, Bove. Barlottoeun vaso di legno di figura simile al barile,
i: hé fara di tenuta o pils,o meno fino a dieci fiaschi, chetenedo
chi si chiama mezzo barile.Qui pero n6 intende strettamente capt specie
t@, ma un vaso da vino portatile addoflo,comunque si sia o di vetro,o di
una Zucca\, anzi stimo che intenda più tosto di terra, perché più git
camo la terra del boccale.
- STREMARE, Vale (cemare, fminuire, quasi ridurre allo stremo.
 DEGNATE, Eun modo di dire usato da coloro che mangiano all' ofteria,
intorno alla loro tavola alcun lwro conoscente, e dicono: deguate,
wi di bere, E perché e termine usatiflimo dalla plebe, il Poeta fa.,
¢ si maraviglino, che Piaccianteo non l'usi,e fa prendere argumento,
ped afi per paura, che non sia accettato Vinuito., e scematagli la.

.
CAVALIERS famoft. Cavalieri illuftri, e di fama. Ma qui famofo non deriva
sma allude a fame, e vuol dir Cavalieri aftamati.

We + Vuel dire -huomo di Plebe; ma ce ne serviamo anche per intende.

te 'infame,senza honore, e senza creanza. Qui se ne serve per contrap.
lieri famofi, e vuo) dire, che si come quelli erano famofi, cioè af.

bul era infame, cioè senza fame, perché havea ben mangiato,

, Von, 2} della rovelia, Non offeri nulla; usandosi spesso il verbo proferire,

In vece-del verbo oferire; e la parola della revella & posta a maggior' emfafi per

tiprimere non offeri nulla, ne meno una cosa nociva.,

~, ANSEGNARE il Gatateo, Insegnare |e creanze, i buoni termini, Galateo. è in-

titolata un' Operetta di Monfigaor Gio. della Casa., la quale insegna le buones























RESleG =





&














Creanze,

. eae ERGLI. 'gilt una mascella, Dargli un tagliovnel vifo, e fargli cadere una
analcia,

40 vi son febiavo, Vi son servitore. E' un detto usato, quando alcuno faccia,
la azione, che meriti lode, per esempio Il tale fece una beliima Orazione;

fo gli son schiavo, I Caporali nella vita di Mecenate:dice 5

H E si legge ch' erugufto un di gli disse:

Gari Capitan Mecenate io vi son [chiava,.

 NELL' ottave Ciclo. L? Autore tenendo l'opinione, che i Cieli ficno otto dice,

Ha: che





aS SELES SEEPS ES

BAG



Digi















254 MALMANTILE™ >

che costui merita d' andare nell' ottavo, cioè nel fup p
penitenza, che merita il fourano posto nel Ciclo.
MONDAN!. Intende peceatori. Coloro che fono'd
dani. i
ST ENV-ATO come un Carnovale., Magro, come un Carnov:
ironica, che vuol dire Graffissimo, come si figura il Carnevale, —
BACLAMO la terra dei boccale, Baciar la terra è un' atto, che si
fone divote per umilta s Ma costui foftenendo l'equivoco del far
haver detto, che gli piace il modo del digiunare, che fa Piaccianteo, d
vuol ancor' egli far' un' atto d' uiilta con baciar la terra, ma
boccale, cioè bere. Bocca/e e un vaso di terra capace della meta d' un
si piglia per tutti li vasi di terra a quella foggia, ancorché maggiori, ¢
ta di un fiasco anche più, t
PER accoftarsi più mondi, Per accoftarsi
nitenza,¢ d' umilta con baciar la terra.
RELIQV1E. Avanzi, fragmenti;¢ scherzando sempre con la bontà
fezione del penitente, par che pigli re/igure nel senfo speciale, che I
noi, cioè offa, ed altri fragmenti di Santi,ed ci vuol poi dire gli avanzid
lui mangiamento. Latino mense relique. Ed in quest ottava l'equivoco:
ficnuto da costui in mostrare a Piaccianteo di credere, che egli fuffe u
te, che flefle quivi per fare aftinenza, come haveva detto; ¢€ per i
tentarsi, che essi ancora 's' accomodino con lui a far la penitenza nell
nicra, che faceva egli.
STANZA LXVIL STANZA LX
Qual madre, che ripara il suo figliuolo Così fam carua di più rigaglie —
Ch e sopragginnta da mordaci cani, Oltr' ad un'Oca grossa ar j
Ei cuopre tutto con il. Serraiuols, Ma vedendo pits la fra quelle re
Ed eglino gt danno in fale mani; Dun perro d'arme luccicar,
E col laza del Piccaro Spagnuolo, E del giaco feappare alcune ma lie f
Che dalla mensa-vnel tutts lontani, Da quella sua cafacca untae
etecio pot a tal cose non arrivi, Infospettiron, com' un' altra volta
Con due caici lo fan levar di quiri, Patra sentir chi volencier m° ascolta,
Piaccianteo vedendo, che costoro s' accoftavano per torgli la roba, cerca di
faluarla,coprendola col ferraiolo, ma essi con una mano di calci V' allontanaro-
no,¢ d'accordo si messero a mangiare: Ma intanto,ofieruato, che egli era at-
mato, prefero sospetto, e fecero quello, che sentiremo fozto nel C. 8. stan. 60,
RIPARARE, Rimediare. Val per difendere. Ed in questa comparazione> —
'imita Dante Infer, C. 23. che dice:
Come la madre, ch' al romore e defta,
E vedo preso a (e le fiamme accefe,
Che prende il figlio, e fuge, e non s'arrefta,
Havendo pri di lui, che di se cura;
Tanto che folo una camicia vesta, -
FERRAIVOLO, Mantello. Vn panno ridotto tondo, e adattato a coprires
tutta la persona sopra agli altri abiet, metcendolo in fy ic spalle.



pil puri, havendo fatto Pau





















p2Ep SRE EE EE








O-CANTARE;: 255

nuolo, Gli zingari, quando s' abbattono nel corrivo;
fa, che gli habbiano vedata, trovano diverse in-
di farlo ballace, o cantar con loro, o fargli mettere in capo
go, che gli'occupi la vista, o con fargli metter il capo in ua' arma-
Eee oe, = ae ed invenzioni per nw ed haver
i rubargli e hanno disegnato, mentr' egli aftratto da tali ope-
a badaa ry gli Ducts attorno; clea (pets veggiamo eat
commedia, che il servo aftuto, per truffare il servo stolto si vale di simili
tic. E questo si dice il ¢axo def Piccaro Spagnuolo, cioè invenzione dello Spa-
0 » Donde poi /azo, dazecgiare significa qualunque azione, che fuc-
oi Comici per ¢sprimere il ior pensiero. E /azo, che in Spagnuolo significa
'prende da noi per quel che i Latini diccbbero capeio, (ophi/ma,commentur,
versuria, fallacia, artes, doli, Ed in questo significato va profferito con,
» €non cruda, ed aspra, perché con la cruda significa fapore aspro,
ate, come que) della prugna, della forba mal macura, e simili, che i
il dicono acide; Dante Inf, C. 15,
ie Ed è ragion, che la trai layzs forbi
ge 7 Si disconuien fruttare il dole fico
Z » perché e frutca di fapore, /azz0, cio' acide dicefi da gli Spagnuoli
quasi dai Lat. diminutivo acidu/a,
'AR carita, Fra i Bacchettoni s' intende mangiare insieme. E tra gli antichi
( iconuiti, che si facevano a' Poveri; di limofine, si domandavano dea-
pat, clot Caritadi, EB Pietanza, voce confervatasi tra' Prati, e tra le Monache,
Piatto, o mangiare offerto dalla picta, e carita de' benefattori; non,
indo altro Pieranza, che Piecd, 1] Beato Fra lacopone: Vorria trovar
skune, Che avefe pictanya De lo mio cor afflitto.
ARCT raggiunta, Grathiiima. Vccello soprammodo grafio si dice raggiunto.

 APCCICARE, Rilpiendere; Rilucere. Viene da Lucciola.
: CASACCA. Parte d' abiro da huomo, che copre la persona da mezza la pan.
vin fy al collo. Così Ca/x/a in Lauino; se bene altra sorta di vette, diver-
fa \Cafacca, fu detta così, perché copre tutta la persona a guisa, che fa la
tala j se crediamo a Ifidoro nel jib, 19, delli Origini, al cap. 24,

FINE DELQVINTOCANTARE,







Dy



S=sSTO


















lll

——

SS OSE See

——





























(ESSE Se ARS th WE
© PA SA a aes oe dP
SESTO CANTAR
Ese WES CSS
ARGOMEN TO, ' 8

5 Nel tenebrofa centro della Terra,
Ove regna Plutone entra la Strega,

oF E vnol che [eco per finir la guerra
Di Malmantile entri f Inferno in leva.
" Fanno concilio i moffridi fotterra,
Ove ciascun buone ragioni allega;
2 Certa al fin le promette ? affiffenza,

Rend' ella grazie, efa di li partenza.

Be secxrige asin
Bia PS A

aoe
STANZA I STANZA IL
Miler chi mat oprando si confida: Di chi creas Letcor tu.qui cht ia tratti®'
Far' alla peggivse ch'elia ben gli vada, Tratto di Adartinazza inigua Seregay
Perché chi pyglia il vizio per jua guida, Cha pin peccati, che non e de' fattiy
Vs contrappelo alla diritta firada. E pel Demonio ogni ben far rinnegay.
E benche qualche repo ei (guarzrijerida Di darsi a lus già seco ha fattod patti, a
Col vetoin poppain quel che pingli aggrada, ecio ne' suci bagordi la ay 'A,
E' vienposl'ora, ch'ei n' ha arender coto, | Ma frate pur; perché rards,e per-tempe iy
E far del tutto 5 dondola, ch' io feonto, Lo sconterd;.da ultima' e buon tempos hy
STANZA LU. Pee ky
Non si penst dhaverne a uscir netra; E quand' ei possa, non se lo prometta, ty
S'inrighi pur col Diavol, ch'io le dico, Perch'ei, che sempre fu nofire wimite y
Se forse haver da iui gran cose a/petta, We puo di ben verun vederci ricchi,
Che nulla dar le puoch'egli e mendico, Vana fune daralle, che L impicchi.

Ji Poeta havendo pensiero di narrar la gita, che fece Martinazza ai Regno di
Plutone per muoverio ad aiutarlo a diloggiar Baldone da Malmantile, ed a g*
stigare Gambattorta, e Baconero, fa ' introduzione al presente Cantare cons
una riflefione morale ponderando, che quei, che opera male, non può sperare
d' haver mai bene, e principiando comel'Ariofto C, 6.

Atifer chi mal! oprande si confida
Conchiude, che Martinazza y la quale nou fa se non sciagurataggini 5 es' édata
al Diavolo, non può sperar d' haver @ hayer bene, perché il Diavolo e es





SESTO CANTARE:

non pt irepral bon paces Hig ak
«10 weet senza riguacdo alcuno.,
va per il verso.buong. Va al contrario di quello, che
da diritta,via.. Sen, epift, 122. Omnia vitia contra naturam
'dinem deferunt; boc off Ingcuria propositum gaudere peruer fis:
dere a reito, fed quam longissime abire; deinde ctiam e contrario flare.
andare a.rigrofo dai.Lating retror/um, Dan, Purg. C,.10, in simil

PPE wii fou
“gy Ofgek Criftian miferi g¢ laff y

) 4 ) (Che della vista della mente infermi
pAb Fidanzahavete nes ritrofi paffi.

@' andar contrappelo e tolta da i pezzi di panno, o di pelle pe-
in cucirle insieme.s' oficrua, che il pelo vada. tutto per un.verlo, ac-
> iano.. A taftar un panno, o pelle pelofa per il verso, che vail
pilfacile, ¢. non si trova refillenza alcuna, come.a andar contro as
By Jatin

j
» Goda allegramente. |
:
|
;
|
|












|
cou vento in poppa. Secondo che ¢j desidera: Come succede quando si ha il
Vento in poppa della pave:-¢ significa e m¢gonzj vanno bene, | Greci pure dissero

vento navigare.

i OLA ch' io sconto, Vuol dire scontera il buon tempo, che ella fié data.,
; alcrewtanti disgutti, E' detto usato dalla Plebe, nella quale e nato; ef
endo lato detto.da un maceliaro, a cui era flata rubata in pil volte gran quan-
ita di Catne 5 ed essendo fatto ritrovato il ladro, fu impi¢cato., ed, il maceliaro
'appelo alle forche disse: Dondola, ch' io feonto; intendendo'a vederti
 dondolare Sconto il debito, che hai meco per la carne rubatami.. Dondolare, &
lo Re ciondolare, come appunto fa l'impiccato; e tal Verbo dondolare
i il nome da quel don don,, che fa il suono delle Campane. E da quello me-
» che faceya quel tanto rinomato vaso dell' Oracolo di Giove, che
rain. Città dell' Epiro, Mima, e con molta ragione, derivarsi il nome
' 41, Dodona Abramo Berkelio Olandefe nelle Osservazioni al Frammento dell' O-

Peta originale di Stefano de Vrbibus. Dondolare, o dondolarfela vuol dire Star-

fenea sedere senza far nulla, di dove Dondolne vuol dire un perdigiorno. Quin-

diua moderng Poeta insendendo di questi tali disse:
Voi dal notturno al mattutin crepuscolo
ah hay Vi dondolate,e¢ fate atu me gli hai,
8st. Soaisconer We conchindete o proponete mai,
Se non rovine al popolo minkscolo.;

© HA più peccari, che non ¢.de' fatti « Ha più peceati-ella fola, che non sono
quelli, che sono flati fatti, o commeffi da tutto il mondo insieme infino a ora,)

SAGORDI, Fefteggiamenti. Vedi sopra C. 5, stan. 62,

TARODI, oper tempo. Diciamo anche Tardi, @ accio ( cioè avaccio, parola,
antica, rimafa in contado, che vale tofto ) o vero} tardi, o avale; che dissero
& ancora gli antichi agwale; cioè ora, in questo punto; vuol dire;questa seguira una
s olta:opreflo,otardi. Lat. /erins, ocyus. 3
i. Mare fh Kk DA '

us SA EE

SaaS TS

'












ase MALMANTILE ”
DA ultimo è buon tempo, Da ultimo verra il ferena Pe
deito ironico, perché significa, che da uttimio per Martinazza'
tivo, cice fara gaftigara del suo mal fare, ©
INT RIGARS?, Vuol dire impacciarG, o interessarsi: e vuol
giiare, o mescolar una cosa con un' altra in-maniera di
go per imbroglio. Bihes ae BR
VIA fane daralle, che? impicchi, Quand' altri ci ha im: niti, pe
gli, che non merita rimuncrazione, si fol dire; Gli vud dare un par
Vn par di funi, o una fune, che impicchi. if = ie
STANZA IV. STANZA











Horsit tiriamo innanzi, ch' io ku finito Ella ch'in tanto havuto havea,
Perch' a questi discorsi le persone Che quei due spirti feiocci ed
Von mi dicefer: Questo feimunito Havean dinanrs a lis fatto U











Virol farct qualche predica ofermone, Si che dat esso furono Scoperti,
edirenti dungue. Già v'havete udito Se la digruma, che ne va il fuo'
L incanto, ch' elia fece a petizione Mentre gli accordi se
Di quei det luego, c' hebbere concerto Rinsciti alla fin tutte,
“Scacciarne il Diuca;ma fuani lefferto, Con un palmo di nafo ne vil
Ii Poeta lasciando da parte la moralita,viene al racconto, e torna alla
tia de] Lettore ' incanto fatto da Martinazza per cacciare il Duca
hebbe effetto, per lo che ella è in collera, 4 le pare di perdere<
ma, nella quale era tenuta dai popoli, e soldati di Malmantile.
SCIMVNITO, Sciocco', scempiato. Vedi sopra C. 1. stan. 17:
SVANL: Posserto, Non riu(ci? effetto: il negozio andò in fumo, 1 Lat.
'dissero Exanuit, & evanescere. 198
SE la digruma, Seco stessa la pensa, e maficandola non la può inj
'cioè rion la può fofierire. E si dice digrumare, e ruminare, e dagli an
'to rugumare, onde forse & fatto digrumare; (che e il rodere che | Ie
ipi¢ tefio, come vedemmo sopra C.g- stan. 6. e C. 5. stan. 63.) perché
succeda cosa di poco suo gulio, suole per lo pil stando pensofo ma
scrare appunto come fanno dette bestic quando digrumano., al che
ebbe riguardo Omero in quel verso tradotto da Cicerone.
Ipfe funm cor edens, hominum vestivia vitans, a
quasi che chi maninconico rumina,'e biascia mafticandola male; 'moftri di
carsi il cuore. %; a
RIVSCIT I tutti panzane., Son riusciti tatte vanita, tutte chiacchiere
anzave, bubbole, chiacchiere, ec, vuol dir promettcre » e NON mantenere, ©
dice inzampognare, infinocchiare, ed' il Lat, Verba dare. 4
RIMANE con tin palrio di nafo, Riman burlata', beffata, I-Lalli En, trl
stan. 11. dice. tet T stkaigit











Ed io son per restar in quefie caps
Con fei palmi lunghiffini di nafo'e









SES TO C'AIN TAR E:

WZARQVEE cad oon il STANZA VU.

se Basta, chella fel e legata al-dizo,

 Etha prefa co' denti,, e fer! affenns;
Tal e andarfene in Dite ha feabilito,
Perch ne vuol veder quanta la canna,
Ed oprar,, che Baldon-refti chiarito
Crambisce in Malmiatilfedereaferana;
Hor mentre a quefia volta 8 indivi,
Potra far un viaggio a due servizj.

non si perde d' animo, e vuole in ogai maniera scac¢iar l'elercito

a Malmantile. Risolue pera d' andare all' inferno ia persona a tro-

-» per ottener da lut il gaftigo di quei due diavoli, che feccro i'errore,

jo modo di far diloggiar Baldone da Malmantile,

shiz eee fiperde d' animo; Non si gomenta..., Vedi sopra C.

8.6 C. 5. stan. 63.

b lifterrefe. Viebbe finito di cono(cergli, Hebbe viflo quanto essi va-

Si dige Ta m' bai dato il mivrefio: Tu m' hai preno: Son fazio, son feufo di

r intendere Now mi varro mai più dell' opera tua.

hanno fatta di figura, Le hanno fatto una ingiyria grandissima, unas

ima buria. Tratio dal giuoco di primicra, quando uno havendo buon,

ed efiendo per vincer la possa, un' altro con figura fa una primiera,e gli









ANNO un capresto. Reftino impiccati. Chiamano capresto quella cor-

jie, che il Boia lega aj cojlo a coloro, che egli impicca, la quale di-

morto il paziente si rompa; e però dice rompano un caprelto; detto

tidimo per intendere farsi impiccare.

DERRE in lowatura, Ridurre in minutissimi pezzi. Limatura si dicong quei
i che cascano dal ferro, o altro metallo, quand' altri lo lima.

i morfe mai cane, ch' 10 non voleffi dei fyo pelo, Nefluno mi fece mai in-

40 non mi voleffi vendicare. Nefluno mi morfe, che io non lo rimor-
si, D che il pelo del cane sia medicamento alle morficature fatte dal me-

'  defimo cane. Vedi sotto C. 9. stan. 58. Eda questo rimedio ha origine il prefen-
te dettato; che i latini dissero Nemo impune abyt, qui me aufus fit ledere,

it SEL' è legara aj dito. Ne ha prefa memoria per vendicarsi. Sogliono molti per

 haver memoria di qualche negozio, che deyano fare,legarsi un filo intorno a} di-

a = i che ha dato origine al presente dettato. Ii Lalli En. 'Lr. Can. 2, stans25,

Tigran Sel' attaccd, come suol dirsi, al dito.
Nel Deuteronomio alsesto, Eruntque verba hac, qua ego precipio tibi hodie in corde
tu: © narrabis ea filijs.tuis., & meditaberis (edens in domo tua, & ambulans in itino=
1, dormiens arque confurgens: Cr ligabis quasi signum in manu tua, B (ono al cap.it,
Ponite hac verba mea in cordibys 5  animis vestris, & fuspendite ea pre sieno in man
 tiibus, Bra Giordano Predi antico Domenicano; nel Vocabolario della.
 Ctusca alia Voce Filareria. Le filaterie si erano una carta, ove erano scritti i co~
Mandamenti della Legge, e portavanla — al braccio apertamente. B quivi

S$4NE2 2 va



Dia


= Bi Asi - SS ee oe

= Se eS

ee



254 MALMANTILE?

va spiegando, cred' io, il paffo di San Matteo cap. 23.) Di
Jua, B voce Greca; da phylattein > puardare, di 1
di quoio, o di cartapecora, che gli Ebrei si legano albraccio'
mente a memoria-i padi della Scrittura, che.quivi sono nota
domandano:Tephilim, wa haba sof eo eseytt -

L A profa covdenti » S' & adirata grandemente,°¢ sé meffa in a
dicarsi. Vuol impiegare ogni suo-stadio per vendicaré icalzolai
venire il quoio a quel fegao che loro bilogna; tirarlo co'.denti 5 di
presente termine, che esprime uno; che si sia preso 'a cuore di' un
e'che vogiia impiegare ogni suo talento:per conchiuderloy

SE »' afanna, Sel & prefaa cuore: N' ha premura * Sene da pena 5
fiero., |





ne,|'uno,¢ l'altro nome significado ricchezze delle quali,perché si cavano di
ra, facevano Cuftode,e Padrone quel loro Dio forterranco; ma qui si piglia
per la Città, e per il Regno di Dite,: aa
Ne vnol veder quanto la canna. Cio' quanto tira, o & lungs la canna da mifu
rare; ¢s' intende vederla per la minuta, e quanto si può, e fare ogni sforzo pet
arrivare al suo intento, °
REST I chiarito. Refti (garito: Scaponitox Vedi sopra C. 1/stan. 12”
SEDERE a feranna, Vuol dire comandare'; esser padrorie s “Scrannay
me diciamo noi ) ci/cranna,è una specie di seggiola da i Latini detta se
Dante Purg. C, 19. dice: ' a cS
Hor chi fei tu che vuoi sedere a feranna a
Per vindicar da lungi venti miglia alee
Con la veduta corta d'una spanna ? Ue aan
Buratto nell'Apologia contro al Castelvetro dice + Non habbiare tanto cermelle, he
baffi, se ben volete (edere a feranna per giudicare gli altri, uy
FAR un viaggio a due servizzj, Che dichiamo anche + Fare un viaggio 5 e dies
servizx). Con un medesimo viaggio far due negozzj, che è impetrar da Plurone
il gaftigo di quei due diavoli, e lo sfratto di/Baldone, Ne i Latini si trova ims
questo senso Duos parieres de eadem fidelia dealbare,E si dice anche Dare a diet
vole 4 un tratto, Vedi sopra C, 3. tanet4. > oe

On



., STANZA VILL STANZA 'IX.
Git da Mammone andar vuolein persona, Perciò s* accontia, e-vie tutta ph y
Che pile non e dover, ch'ella pretenda 5 Col drappoin capo,e vol veraglioin mant
Che sua bravicornissima corona © cercar chi? informi della cite;
Salga a suo conto a veni poco, e scenda, Ne meglio'fa, che Giulio Padovanes
Chieder grariese dar brighe no cesuona, Chet ha fu 'per le punta delle ditt)
E chi ha bisogno,si suol dirys' arrenda, E più ds Dante, e pi del Manrovans
Per questo a lei tocca apigliar la firada, Perch eglitio vi furon di a
'Per calla fin conuien, che chi vuol vada, E questo ogni tre di vi '
, ol vt sain wan?
at oie

t: 3h wath ~ 1,

IN Dite. Dite, secondo il favolosorcreder de i Gentili'é lo stesso see

































SESTO CANTARE: "255
WaAE PCS. S. - VOW ZA)






E poi per abbondare in cautela,
che in Di prefume) Volendola fernire infino al fiume,
che gente, ¢che loquelay Le porge un fardellin piccolo, € poce
ple dd conto,¢ lume; Di robe, che laggth le faran giuoco,

a'rifolue' ae in persona a aan x oe che
y che questo Re per lei a ogni scomodi; e però sapendo, che
Padovano è pi aibithaes d oll alevo' della strada dell' Inferno, se ne va
r da lui informazione, e della gita, e dei costumi di quei paefi; ed egli
ice, e per servirla meglio la vuol accompagnare fino.al fiume Acheron-
intanto le da un fardellino di robe, che laggib verranno.a bisogno'.
VICORNISSIMA corona, Epiteto, e titolo composto dall' Autore a Pla.
gli they Lalli' Bn: Tr. lib. 1: tan. 16. parlando d' Eolo Re de' Venti dice;
ie sh » Dunque poi che Giunone alla prefenza
ro) pis) ies Di sua Real ventofita fu giunta.
'  - MAMMONE, Da Mammon; parola wfata nell' Evangelio. Alcuni esposi-
ue | Sacra Scrittura vogliono, che Mammona sia voce Caldea, e significhi
$5 ed altri che sia voce Siriaca, e significhi quello, che in Greco significa,
» che € divitie, si che concordano, e tanto è a dir Mammone, che»
 Demonio, ovvero Plutone, che qui s'intende per il Re dell' Inferno. Vie-
a ne dalla radice Ebrea Taman, che propriamente significa a/condere, riporre, es
Per Così dire-intanare; onde si fece AZatmon, e alla Siriaca Adatmona, cio' ricchez~
Xt nafeofie, © vogliam dire teforo. Mammona poi venne a dirsi per più agevolez~
za zia
2. + Dare scommodi, dar moleltie. La voce briga significa opera-
2ioni (coirimode, faticofe, e noiofe. eit
yf CHP ha bisogno # arvenda. Chi ha bisogno non sia superbo, ma si pieghia rac.
Ȣpregare; Che il verbo arrendersi val per cedere piegarsi, o con-



' Cc.
ie CHI wolf vada. Chi vuol ottenere una cosa vada a chiederla da per s¢, ed il
et Ndice Chi non vuol manat, e chi vuol vada da se, Che diciamo anche Non i
ie «ibe mee, Che le spesso, o vero, Chi va lecca, E chi fha si fecca, i

ACCONCIARS!. Rinfronzirsi, raffazzonarsi. Vedi sopra'C, 2.stan, 69,
DRAPPO, Dicendoli drappo assolutamente s' intende drappo da-donna s che
una strifeia di taffetta,o d' ermifino Jarga fino a due braccia,e lunga fino aquar-

tro, la quale dalle donne Fiorentine di condizione ordinaria € portata in capo,

Oalle spaile quando vanno fuori di Casa. In Venezia drappo significa ogni sorta

divestimento, si come presso i Toscani antichi scrittori. Vedi sotto C.7.Man.az,

VENT AG LIO, Strumento noto usato dalle donne la state per farsi vento,

ZL INFORMI della gita, Le insegni la strada, che conduce all' iaferno,

GIVLIO Padovano, Io veramente non ho saputo ritrovare chi sia questo Giu-
 lio Padovano, se forse non ha inteso di Giulio Hygino scrittore d' Aitronomia,

Ma costui fu liberto, o vogliam dire schiavo atirancato d' Augufto; condoreo da

lui ra d' Aleflandria, secondo che alcuni vogliono; i quali perciò lo stima-

no Al rino; o pure di nazione Spagauoio, secondo la teltimonianza di Sue~
¢ Maio nel Libro de illuftribus Grammaricrs. L' HA





Sea S KE =










256 MALMANTILE ©

L' HA fu per le punte delle dita, La fa benitfimo; Latino im m
do Manuzio nella dedicatoria di Giuvenaie disse: Quando eas tench
digicas ungue(que twos, Cicerone nella Orazione cont i
uid cum accufationis tua membra dividere ceperit 5 @
canfe conftisnere ? Quid, cum unumquodgue tranfigere, expedire,

DANTE, e il Mantovano, Dante Poeta Fiorentino; e Vergilio
te finge y-che fuffe sua guida all'Inferno, e pero dice: Egeine vs furon

| OGN tre di, Questo modo di dire, se bene € determinate, significa tp
fo, ©.a ogni poco indeterminatameate, eden ah

ANDAR via divela, Andar via velocemente, e a dirittura, come |
quando va a vela. » one

PER abbondare in cautela, Cioè per servirla bene. Diciamo abbondi
quando uno fa più di quel che Ga richiefto, o pil di quel che sia n
elempio. lo dard diect feudi a uno, perché mi compri una mercanzia, la«
fo che non vale così gran somma; ma per aflicurarmi del caso, che valeffe:
ir cautelato'y










pil, li do due altri (cudi per abbondare in caurela, cio per anda
sul sicuro, che non gli manchi denaro,se ella valefe pi. Qui pegd
Abbondare, ed eccedere in corcefia nel servirla. rune
LE faranno giuoco. Le torneranno a proposito., Le verranno a bisogno, Le
faranno d' utile. sod
STANZA XI. STANZA XI

Così la Maga se ne va coneffo, Questat la via, che mena
* Che f introduce in una bella via Perch'ellac allegrayo
Tutta frorita, st che al primo ingreffo Perché 4 martello poi non we
Par proprio un Paradifo, un' allegria; LA feorre ognor gente di mal afarts —
Ma nipin prefol buom il pic v'ha mefo Le ferpi sono ogni opera ribalda y
Cb' ella diventa un' altra mercanzia Chrella ci fale quali a lungo ani
Per i gran morsi,¢ le punture acerbe, Di quanto ha fatto, scavallata, ¢fenfe
Che fanno i ferpi ascofi fra quel! erbe, Ci fa sentir al cuor qualche rumorfa,
STANZA XIL STANZA XIV. 5















Entravi Martinazza,e sente un tratta tMa se ravvisia un tratto del sue,
Dueyo tre morsi a più dove calpefta, Bada a tirar innanzé alla balorday —
Percio befemmia, che non par [uofatta, Perch'il vizio rifiglia,e mecte il tally
E dice:O Giulio mio, che cosa e quefia? Vie stpre pix aaggravar/iinfulacerds,

Ed ei ridendo allora come un matto; Ul male invecchia al fine,e vi fa ilealle 9

Non è nulla ( rispose ) vien pur lefta; Siche venga un Serpente parese '

Che pensi tu ch'io sia privilegiato so Chrei x6 sente ne meno anctun ribregeny oa
bY 19 mi fento mardere,e non fiato. Cos} peggio che mai la da pel mezts

pa kar "STANZA XV, er




lla neve si f4 lo stesso giusco y ' Al fine ei ff rifealda come un fucco
Ebert ment Jul primo dacciafi le dita, Si che non lofarwits mai finita

Poiquelgragelo par che maanchi un poce, We gli darebbe punto di spavento'

E sempre pine nell! Agitar (a vita; Quandei thavelfe acoraa dorami

Martinazza se ne va con Giulio, il quale la conduce per una strada,
primo mgreiio pare una belia cosa, ma presto si conoice, ch'ell'é al

SSLPEE sR Eee Sp Ge eae gis








SESTO CANTARE, zy7

i-€ i i ascofi infra quell' erbe; Giulio mostra a Martinazza,

 strada, che guida all' Inferno è facile, e gustofa, e se bene e ripicna.,
i, non son sentiti ne conosciuti da quelli, che la camminano, perché
afluefatti; appunto come fanno coloro, che mettono le mani nella ne-
i ipio la toccano fredda, e col seguitare a maneggiarla, par loro
PARE un Paradifo., Pare wna-cosa tanto allegra, e vaga-, che più non si può
 fare. Telemaco figliuol d' Viifie nel quarto del' Viiflea, arriyvato in Sparta; nel
 considerare attentamente la ricchezza, e l'ampicaaa del Regio Palazzo di Me-

i — in quella ¢sclamazione: Ta/ dentro e del gran Giove ilgran Pa-
tO,

4 ENT A wn altra mercanzia. Diventa un' altra cosa. \Veiamo dir mercanzia
i ogni sorta di cosa ancor che incorporea, come 40 frudiare sé una cer-

py eC,
par (ue fatto + Non par che faccia quella tal cosa. Vedi sopra Can, 4.
aj stan. 16.
-) CASA Calida, Intende V' Inferno. Il Lalli En. Tr. parafrafando facilis de-

 ferfus Auarni ec. dice:
= Eves mio bello
: A casa Calda si va presto presto;
' | seen 'CHa ritornar infu, quale è il bordello.
] NONE nulla, Queste due negative secondo la buona regola doverebbono afier-

| Mare, ma è nostro idivtifino tanto inveterato, che I” uso-ci libera dall' errore, se
"'@eneseruiamo in questo modo per negativa. Appresso i Greci due negative, 0
. = affermano, ma negano maggiormente, ed e maniera, siccome appref-
40 noi; così appresso loro usatissima.
¢ WNONfaa marcello, Non regge alla prova. Noné-com' ella pare. Metafora
yf 'tolta'dal Cimento dell' oro. Vedi sopra C. 5, stan. 2.
'4 LINGO andare. Col tempo... In procetio di tempo'; Se continoverai lun-

ee. '

fo SCAVALLATO, Cioè datasi ogni sorta di bel tempo. Si dice anche scorrer
i,  latavallna » Virg. 3. Georg. Scilicer ante omnes furor est infignis equarum, Bt men-
io 'tem Venus ipfa dedit. E poi: dilas ducit amor trans Gargar astranfeque fonantem,&c,
»  VedifopraC. x. an..66.:

|) REALCHE vimorfo. Senton rimorder la coscienza'per gli ¢rrori-commefi.,
@ — ALLAbalorda, Senza-considerazione.

METT £ il tao. Talliice, fa nuove mefle, Vuol dire:sun vizio ne genera,
g  Mholti, Tallo @ parola veriuta a noi dalia lingua Greca, che significa germoglio,
et 'usata ancora dagli agricoltori-Latini..

»  VIENE «aggravarsi in fu lacorda. Vien più che mai a crefeere il male; perché
ido uno tocca il martirio della corda, ¢s' aggrava in fu la medesima corda,
~ fa crescere il dolore; 'Ed altrimenti 4geravar/? in fu ta corda vuol dire quando uno
#  sfaminato'in fu la-corda dice-cose, che fanno crescere I"indizio, che egli hab.
y EMS somuneo-un dette. Ang

§ | “PAilecatlo. Vis' afiucka. Er ab afuctis non fit paffio, dice, che-non fen-
si “se pcmeno.un. c. a RL






Foe,

a ee a








258 MALMANTILE,

A/BREZZO, Che vuol dire'capriccio di febbre; cioè quel
che si sente prima, che entri la febbre « Latino rigor avalc
lib,2.cap.21, dice: Antipatro di Sidonia in quel giorno, che egl
gli arrivava qualche ribrezzo di febbre, e tanto continua, s ce
mortale accidente, Ma Dante nell' Inf. C. si
Qual è colic! ha si prefa
Della quartana, ¢' ha già fugna smorte
1 E trema tutto pur guardando il raze.
BalC.za.dice: Pascia vedd' io mille vifi cagnagri;
Patti per fredds, onde mi vien riprenroy

E verrd sempre de i gelati guazzi,
Ma noilo pigliamo anche ( come e pre(o nel,presente luogo ) per ogni leggi
follevamento d' animo, o-spavento, o per un jemplicitiimo dolore, Bdal
te per fattidio, o travaglio per esempio // rale commelfe quel mancamento; ne,
haver de! ribregzs, Vedi sotto C, 11, stlan.2., 9 spkway ¥
La dd pel mezzo, Fa tutto quello, che gli vien yolonta senza riguardo aleuno.
E' dedotto da quelli; che in tempo di pioggia camminando per la Città yanno
per il mezzo della firada, e non si guardano dal' ammollarsi per J' acqua cadu















Se SEES ST ee

=














ta, che scorre pel mezzo, e per quella che vien dal Cielo, <i ied
STANZA XVI, ' STANZA XVIL sp
Hor tu m' hai inteso:rafferena il volto, Resta, dic? ella, omai ch' io ti ringrazia We
Che tu vedrai tirando innanzi il conto Dellinfernzionsch'appiio: li
( Perché di qui a poco non c' è molto ) Promiffio bons viri off obligatiay
Che delle ferpi non farai pin conto, 1 Die egli; Tho promeffoge intends a
Ma dimmi, c' ba' tu fatto del rinwolto 2 Ancor seguirti questo po iay) ha
Lho qui, dic'ella,sempre lefto,e pronto: E quivicon un tibi = "
Sta ben,soggiunge Ginlio,adungue corriy Ail) in qua ripigliands.il mio.cammind



Perché qui non e rempo da por porri, Ti lascio, come io diffial
. Giulio ¢sorta.Martinazza a non haver paura, ed a camminare; ed
grazia dell' instruzione datale, e lo prega a partire, ed egii ricul di farlo, pet
¢hé l¢ ha promesso di accompagnaria infino al fiume-Achcronte.. 3... 4 ki
D1 qui a poco non ¢? e molto, Questo termine giocofo e usato per esprimere r+
ochissime tempo. ' 4: vowed Sag
TIR ANDÒ innanzs il conto, Seguitando ll suo viaggio, EB' termine mereaatilt,
che vuol dir portare un conto avanti da un libro a un' altro, oda unacattaae
un' altra nel medesimo libro, Donde poi tirar innanzé il conto vuol die Cammina
re avanti. Vedi sopra C. 4, stan. do, ot, nls To i
NON è tempo #a por porri, Noné tempo da perdere, Non & da indugiares.
Quando si pongono i porri, sono così fottili, che Hehe
i a Wnt
| big














*Zezesaerea

[Ras

porgti; eda questo habbiamo il presente proverbio, che si dice anche +
tempo da dar fleno aoche. i w naobs
PROMISSLO boni viri est obligatio, Sentenza latina, che vuol dire un':
da bene e obligato a mantener la parola,.¢d osservare quel che ha p: a2
CON un tibi me commendo, Detto latino, che suona con un. mi & a
te; ciog con falutarti.. Quando diciamo: Addio, C1 s' intende; vi.raccomandd «
' ms Sadun






f ut  -gRSTO CANTARE; 19
alut lo; Catullo: Commtenide tibi me.






colonnino, Tial

AANZA XVIIL

ti '4 il capo, @ rocca,

¢ ferpi elt a qualche paura;
2 x tase fatto def enor roccay
Vi otal teense ntl

is

£.



 & Bed

disse: Vale.

til famofo fume d' Acherorte,
ve s'imbarca ognun, che quivi arriva,
| Saffaccia ach'effayma il nocchiere ar ite,
7 che trate ognuxo hebbe dariva,
dietra,grida a lei con toruafrote,
Che quad non pafsa mai anima viva;
~Ond' ella, meffi fuor certi baiocchi,

2° Lascia? ad Colosnine yuo! dir lasciar uno
laren i mo
vanti alle forche, €'vi leg

Ha colonnetta di legno traforata—,
inialfactori 1 gli strozzano.
STANZA XxX,
Ed égli, che da e/a bebbe il fapone,
E che si trove li come tl ranacchia y
Preso dalla wiedefima al boccone,
Menty" ella faite in barca, chinfe locchia;
La Strega sia quell anime si pone,
Luaic nlebrach: son fino al ginocchio y
Duvendo a' Sopraffindagi di Dite
Presentar de' lor libri le partite.
STANZA XXL
Piangendo, come quando yno ha partite
Le cipolle fortissime malige:
Pafsan quel fiume,e poi quel di Cocito;
Vitimamente la palnde frige,
Che a Dite inonda tutto il circuito,
E in se racchide furbi,e anime bige
Ove Caronte al fin [endo arrivata

Gli getta un po di poluere negli occhi, Sharco tutti; ed ognun fu licenziato}

'Martinazza seguita il suo viaggio, e non fa più stima delle morficature de i
sespi yed artivati al fiume d' Acheronte', Ginlio si licenzia dalla donna, la qua-
Te ) per entrar nella barca; ma Caronte lo rie dicendo, che non poteva
eotrarvi, ond' ella gli diede un poco di mancia, ed ei finfe di non la vedere en-
Wat in barca, dove ella si mescold con gli altri, e fu condotta all' altra riva,¢
guivi con essi sbarcata.
oe, « Sidice tocca il cocebio, e significa: cammina innanzi. Vedi sopras

41.

ZAMPETT A. Muove le gambe: Cammina. Zampettare si dice propriamen-
te de'bambini quando cominciano a imparare a andare.

NON si sente aprir bocca, Non si sente parlare. Sono infiniti i modi, che hab-
biamo per esprimer il filenzio d' uno, come far Zitto; non fiatare; non far verbo
mmansolire; star chiopto, lasciar la lingua al beccaio, haver viffo il lupo; diventare Ar»
porrate ec.

GLI difie: Vale. Gli disse Addio -

ACHERONT £. | fivmi dell' Lnferno da i Gentili si dicevano quattro, e che
fialteflero dalle Jagrime de' mortali, per lo Mato de' quali figura Dante la flatua,
the vedde in foguo Nabucdonofor, che havea la telita d' oro, le braccia, es
Vie d'argenio, il sorpo fino alle cosce di rame, le gambe di ferro, ed il defiro

di terra cotta; da questa dice che scaturiscono le dette lagrime, le quali for-
Mano li detti quattro finmi Infernali, e così la deferive nell' Int. C, 14,
Dencro dal monte fra dritto un gran veglio,
Che tien volte le Spalle in U Te;

A% BBt BW BEERS. SS Sak

=

SAFE

E Ra-




260 MALMANTILE

E Roma guarda si come suo [pegtio,
La sua tespa e di firt oro formata y
E puro argento son le braccia, e il, A
Bhd dl nec tcan eee — ah x Tan
Da indi in ginfo¢ tutto ferro eletta, 7 tae
Salno che'! deffro piede è terra cotta y ref 9
E fla in fu quel pinych'in fu U altro, eretto a
Il primo dunque di detti fiumi e Acheronte, che ia un certo modo
privazione d' allegrezza;da Acheronte nasce Stige, che significa cosa dispi
odiofa, quale e il Dolore; perché questo ne viene dopo la privazione dell'
grcaza, Ll rerzo è Flegetunte, che signfica pensiero ardente travagliolo.
questi tre fiumi si genera il quarto, che e Cocito stagno,o fiume del
pianto. Questa favolosa opinione de' Gentili tocca Dante nell' Inf, C, 14.!
tando i sopraddetti versi. om








Ciascuna parte, fuor che l'oro e rotta ity
D' una fefsura, che lagrime goccia 5 ait
Le quali accolte foran questa grotta, one

Lor corso in questa valle si diroccia ae
Fanno Acheronte, Stige, e Flegetonta; ihe
Poi fen va gik per quefea firetta doccia y;

Infin la dove piit non si difmonta, ~ alt
Fanno Cocito, e qual sia quello fhagno
Tu'l vederai, però qui non si conta,





ih

at '
CARONTE. Notissimo barcarolo dell' Inferno. Vedi sopra C. a, stan.2g) |,
HEBBE tratto ognun da rina, Hebbe Jevate d' in fu la riva tutte ! anime, im |
barcandole, hu My
TORVA fronte, B latino usato da noi; E vuol dire Vifo burbero,asp toy
arcigno. Rs,
ANEMA via, Intendi huomo, che non fiamorto. Virg, 6, Em. Corports | (ty
vina nefas Stygia vettare carina. Sa bene il nostro Poeta, che anime sono immors | i
tali, ma seguita il costume d' intendere huomo viveate, quando diciamo animas jij
viva [ Genefi cap,2, Ee fattus eS homo im animam vinentem ] ed imita Dante !
C, 3. che dice; }
E tn che fei cost} anima vina, |
Partiti da codefti, che son morti, |e
Ii Lalli Ea, Tr. C, 3. stan. 16. ea
E non v' è mai entrata anima vina y iy
GLI gettd wn po di poluere negli occhi. Gii decte un po di mancia, I Latini pute jp
difero: Puluerem oculis offundere, Es' intende dar mance per corrompere il git %ir
sto, quasi diciamo: Abbagliare gli occhi del gindice con l'aro 5 accivcche mom ty
instizia. 194
£ pire il fapone. Exa flato fubornato, e corrotto con la mancia; Gli er: y

flare infaponate le carrucole (che vuol dire Tirar' uno al nostro volere, erent
derlo facile a quel che noi bramiamo, e fare che non strida contro di noi ) coma
dargit la mancia; come con! iafaponare nna carrucola, o na ruora si facilites

| y
:















261
cOlO 5 }y che non frida, Ed & 10 stesso che getrar fa poluere negli ecchi
poce Dicefi anche: Vener le mam. Bocc, Nov.6. 2 bucno husmo por
i fece ngner le mani,
come il ranocchio. Obbligato a tacere, per havere havuta la
li faddetti due modi di dire, cio& Havere il fapone, e>
Qui non vorrei che ij Lettore credefle, che il Poeta
'egali potcficro corrompere i demonj, se bea la sentenza
lice munera(crede mibi)placant homine/que deo/que,ma (apefie haver'
-mostrare che l'oro arriva a corromper quelli, che ne meno si
¢ meno dovriano lasciarsi arrivar dail' oro, e finalmente ha vo-
la posianza, che hanno i regali di far confeguire ciò che si vuole.
per pecuniam fattafune, Si racconta di Filippo Macedone, che haven-
9 riconoicere una fortezza', ed cflendogli riferito, che era imposfibile il
la, domandafie agli sploratori,se vi era modo di farvi andare un' asino ca-
volendo inferire, che dove non potevano I armi, farebbe arrivato
5 uri Sacra fames,quid non mortalia pettora cogis ? BE Orazio. aurum per
ire fatellites, Et perrnmpere amat faxa potentius Iitu fulmines,
YSEL oechio. Finfe di non vedere. £7 il latino connivere, Vedi sotto C.





















Rieke fino al ginocchio, 11 proverbio Ca/char Le brache & i) medesimo,
le braccia; che vuol dir perdersi d' animo, Omero: <dnimus in pedes de.

il cnore; cascd l'animo a' piedi, Onde dicendo, che costoro have-
che fino al ginocchio, intende che eran loro cascate affatto, cioè erano
> — @ animo, perché doveano render conto delle loro azioni. Vedi
SINDIACT. Così chiamiamo noi quel Magiftrato, che ha l'autori-
cr icontia tutti i Magiftrati, Ofiziali, e Miniitri del dominio Fioren-









maligia. Specie di cipolla da mangiare, che è fortissima, e fa ve-
ime a tagliarla,¢ maneggiaria; Bocc. gior. 8.n.2, £ talora un mare
cipolle malige, o di Scalogni, 1 Lalii Eo, Tr. C. 3.
Così dicea, e tutto ii volto molle
Haves di pianto, come [e [chiacciato
> ¥ Vi fuffe sopra il fugo di cipolle.
COCITO, Vedi sopra alla stan. 19, alla parola sAcheronte, e quivi troverai
ancora quel che sia la Palude Stige, della quale vedi anche sotto in questo Cant,

3 Ibige; Genti scellerate, e da non se ne fidare. Per comporre il color
310 i Pittori mescolano tutti i colori, e lo chiamano il coler dell' asino; e però
lic | huomo bigio s' intende uno, che ha tutti i vizzj. Va moderno Poeta.,
€¢ notammo spra C. 3. stan. 66. dific parlando d' uno di questi tali, che era












Y; Chinde un' anima bigia un corpo nero,
di questa parola bigio in questo significato stimo, che nasca da questo.
in Firenze pe foci paiisti tre ai, una de' Fautori di Er, Giblin
= Sree ke Savona-

















264 MALMANTILE =~

Savonarola, la quale era detta de' Piagnons, I altea de? conttarj a detto Fr. Gite: |»
Jamo chiamata gli e4rrabbiati, o Compagnacci; e sta'di loro ¢1 teo'nimi- = |
ci, e discordi, faluo che univano nel? esser contrarj alla terza fazione;, ¢
de' fautori de' Medici, la quale era detta de' Padefehi, i quali non conue
ne con l'una, ne con l'altra fazione. Di questi che inclinavano alla |
Patieschi talvolta alcuno per suoi fini particolari s' univa o con Puna, o con!
tra delle prime due, ma era ricevuto con sospetto, che non fale)
ro deliberazioni, e pero dicevano: Won e da fidarsi di lore', Son Bigi. E
ucftowforfe ha havuto origine questa voce bigio in significato' 4' huomo da
se ne fidare, Vedi la Relazione di Firenze del Foscari, e il Nardi nell
Florentine fib, 2.:
STANZA XXIL
Cli entrar dovendo in Dite,e faltn,e gira,
Che par quando mi barbera latrottola,
endar non vi vorrebbe, e si ritira
Grattandosi belande la collottola;
Pur finalmente forza ve la tira,
Come fa il peso al grillo,una pallottola;
Così ne van quell' anime nefande
Chi dal piccin tirata,e chi dal grande,
STANZA XXIIL





















Perch gli è offa,e pelle,
Ch'ei par proprio il ritratto dello
STANZA XKV,

Per la gran calca nel paffar le porze Si che quand! ei si fence il tow
Connenne a ognino andarne cosa piena, Lerche la fame quivi ne c
Ma la Strega non hebbe tanta forte, Liingoxxa, che ne manco non gli tocca
Che trenla il can, che quivi Pain catena; Ne di qua, ne di la gih per 3
E perché per tre bocche abbaia forte, Ma fubico gli venne il,,
Ella dice: Ti dia la Maddalena; Ond'ei sallunga in terra afar lai

Chil papavero,e il loglioch

Faria dormir un' orso, non
ZA X&KVI '
Sdraiata dorme; e riss com wh tf,

E in tatotrovail pane,e in pexsiltaglia,
E in tre gole ch' egti apresgliene feaglia,
STAN





'Hor mentre fa il fonnifero il [uo corso,
La donna che piit la facea la [corta Lerno da bore fa versa la porta y
(Peroccht havea timor di qualche morfo) E poi (benchrella fuffe alquanto [rect
Vedendo che la beffia, come morta Da unacor[aye in Dite anchrellait
L' anime rimafte attorno alla Città di Dite mostrano co' gefti,
lentieri vadano dentro alla Città; ma i loro peccati @ forza ve le
anime nell entrar della porta fecero così gran calca, che la Strega nof
flar con esse, € tanto più, che ell' hcbbe 'paura di Cerbero'; 'onde'pel
fene gli gett del pane fatto col fonnifero; per lo che il cane' si ad
ella entro nella porta. E quiil nostro Poeta imita Verg.nel 6,
dare a Cerbero dalla Sibilla ana fliacciata'col fonnifero, e nelle pr
23. 24.25. parafrafa,si può dire,i seguenti versi del medesimo
Cerberus hse ingens larratn rege i
Personat, aduerfo recubans insmanis in dntro
Cui vates borreve videns iam colle colnbris 22













SESTO'CANTARE:

2 9 © medicatam frugibus ofam
Aes, Obijcit; ille fame rabida tria guttura panders
9 Corripit obietam, atque immania terga refoluit
5 Bafus humi, torque ingens extenditur antro,
Il verbo barberare è usato da' nofiri fanciulli per intender quan-
gira a falti, enon va unita per cagione dell' esser mal contrappela-









LA, Strumento;del quale si servono i ragazzi per giuocare, ed è un
foggia di piramide, che fini(ce in una punta di ferro, Vedi sopras
si fa girare avvoltandola con uno spago,¢ poi feagliando 3 ter-





3.

14, tirando con velocita a se la mano, alla quale e legato detto spago.
GRATTANDOS! ta collottola, Grattandosi il capo nella parte di dietro dai
tini detta cerwix. E questo e un' atto solito farsi per lo piti dalle donne, e da'
ndo hanno qualche di(grazia, o gran disgufto. Vedi sopra Cant. 3.











VDO. Vale piangendo: perché sc bene il belare e proprio delle peco-
li, e viene dalla voce, che fanno tali bestie, che suona be be, ce nes
lamo anche per esprimere il pianto dell' huomo, ma per derisione; donde si
belone, pecorune a uno che pianga assai. Vn moderno Poeta disse;
Hor ch' è per te finita la pasciona,
Gf Che fai che tu non beli, o pecorona?

GRILLO. E' un verme piccolo volatile noto: Ma trattandosi di pallottoles
Grilles intende que!la piccola palla, che si tira per segno nel giocare alle pallot-
tole,0 alle pialtrelle,omurelle, Vedi sotto in questo C, stan.34, e C.9.staa.17.
PALLOTTOLA, \ntende quelle palle di legno, che servono per giuocare., j
nelle quali sono tre contrappefi di piombo, per via de' quali si fanno fare alies ?

F operazioni, e voltamenti, che si vuole, ' uno di questi si chiama. i
¢atetia, V altro il grande; ed il terzo il piccino, ed il Poeta, aflomigliando quell'






=

ASSES EES:

















ge  *himea queffe pallottole, dice, che ancor' esse son forzate a entrar nell' inferno \
dal,¢ chi dal grande, cioè chi da i peccati piccoli, e chi da i grandi.

ye! Quantità grande di popolo; folla.

5 ANDAR con Ia piena, Andar co' più; andar in truppa con tutte quelle anime,

m lie piema per similitudine significa inondazione, o furia di popolo. Virg. Georg,
i lane falitantum rotis vomit adibus undam. Andar con la piena significa ancora se-
,g  Biitat Popinione comune; andar co' pill.
ee AL Cane che quivi fa in Catena, Cerbero cane con tre tefte,due delle quali stan-
4  nolempre svegliate. Hercole lo lego, ed i) nostro Poeta imitando Vergilio co-
' me s'€ detto, lo fa addormentare col/pane alloppiato.
4  TTdiala Maddalena, Possa tu cfler impiccato. Dicevafi porta di Caronte da-
m gli Atenicfi quella porta del Palagio del Podefla, dond”uscivano coloro, che an-
;  davano alle forche, come accennammo /opra C. 5. tan. 3. e noi diciamo Ti dia
ta Maddalena, da quella Campana, che e nella torre del Batgello, la quale suo-
ma, <n va alle'forche, e si chiama la A¢addalena, perché con tal no-
mee » per esser la Cappella di quel Palazzo sotto i ticolo di'S, Maria

GLIE.

.


oT









264 ' - MALMANTILE:
GLIENE feaglia, Gliene tira da lontano; Glien' av
ra nen segli volle accoftare.,
HAVEKIA mangiato Salerno, Havrebbe mangiato i fat, Ve
disse; fume rabida. E si trova Satylnm voraret, che batylum chi

piecra, che si divord Saturno, sDEDy
SER faccents, Si dice: Ser faccenti, o Barbaffori quasi Valvafori,
dajc,a coloro, che tutte le cose fanno, e dicono magiftralmente, ¢
degli altri; E' però detto (cherzofo, e per burlareuno. Qui intende:
natori dell' Inferno. E' parola derivata dail' anti.o verbo /accio, per,

























Lopio'g > vA
PER il mal gaverno, Per il poco mangiare, che gli danno... Nell
Governare le gailine; cioè dar loro da mangiare. Similmente i La
soidati pigliavano un poco di rinfresco,dicevano; corpora curare,
Governare gli uliyi disse Pier Vettori,cioè concimargli;quasi questo sia
Si servtto che tien ! anima co' denti, Si macilente, e magro y.che p
lerebhe ? anima, se non la riteneflc con lo stringer i denti. Giobbe pe
se medesimo emaciato, e confunto. Pelli meae, consumptis carnibus y
meu, ce 3
EGLI è ofa, e pelle. Non ha carne addosso: E' magrissimo, Plauto d
questo proposito Offa, atque pellis. E Dant. Purg. C.23. dice;
Wegli occhi era ciascuna oscura,@ CAVA s
Pallida nella faccia, e tanto scema y
Che dal! offa ta pelie s* informava,
SPENTO, S intende al maggior segno magro.
LA fame lo scanna, Muore di fame. Vedi sopra C, 4. stan. 24.
CANNA., Intendi la canna della gola, la quale si dice canna per.
ne, che ha il gargarozzo con la canna, Dan, Inf,C, 28, f
Restato a riguardar per meraviglia she, *
Con gli altri, mnanzé agli altri apr} la canna s cnt
Onde Scannare, sgozzare: tracannare., ingollare, pha
GLI wiene il sonno in cocca, Cioè nell' eftremita delle palpebre 5 che vengonos
chiudersi, Gli vien voglia grandissima di dormire. one
S* ALLVNG A in terra, Si distende in terra, Lmmania terga refeluit Fufus hima
totoque ingens extenditur antro; dice Verg. com' habbiamo accennato sopra.
A FAR (a nanna, A doimite. Termine insegnato dalle Balie it
imparano a parlare, per esser più facile a dir nanna, che dormire, Lala N.
Non lajcio mai certi detti che haveva imparato da bambino,chiamando pappo il
vino bombo, i quattrini dindi, e quando voleva andare a dormire, diceva
ta nanna. | Launi fimimente !'addormentarsi de'bambini alla Ninna
tilena delle Balie, da lor detta Ladus, e da' Greci Wynnini; dicevano La
MENT RE il fonnifero fa il uo corso, Ul fonnifero fa la sua Operazione +
PAP AVERO,¢ Loglio, li papavero è quell' erba, il feme, ed eftratto
quale compone l'oppio, o fonnifero; ed ii loglio € un' erba, che nasce si
ni, il feme della quale mangiandolo, dicono, che faccia sbalordire, ¢
no. £ da questi mali effetti del Loglio habbiamo un proverbio, che



Lad













oo CANTARE?.
significa Io non son balordo.
sopra C,3.stan. 32. /draiarsi è il verbo recumbere; E Ver-
u pacule recubans fub tegmine fagi; stimo che intenda sdra~
te ne stai all! ombra d' uno spaziofo faggio. E nota,
, che vuol dir largo, o spaziofo, ¢€ staco cavato il verbo
I eas » e paffare il tempo senza pensieri, il che chia-

tifmo assai usato. *
. Ronfare: Quel romore, che si fa da molti nel re(pirare dormen-:







ae;
da bette. Vuol dire accoftarsi. Perché le doghe, e l'altre parti
'botte son lavorate in modo, che si compaginano,, ed unilcono,

STANZA XXVIIL.
S' ell è come voi dite a questo modo:
(Si le risponde ) andate pur madonna,
Perch' altrimenti c* entrerebbe il frudo,
E voi fharefti in gogna alla colonna,
Horsit correte pria che freddi il brodo












dice) ola, che roba e quella?
i ( dic' ella) nel forame,



non ho qui roba da gabella, Che la Regina poi farebbe donna
— Se non un po a' alors' a Proferpina Da farci per la frizza, e pel rovello
Porto, perch' ella fa la gelatina. Buttar' a' più la forma del cappella,

za havea sotto alcune rame d'alloro; e da i gabellieri le fu doman-
+ ma essa con dire, che era per servizio di Proferpina, si libera,
a del gabellicre. 11 Poeta imita Vergilio, il quale fa che Enea d'or-

Sibilla porti a Proferpina il ramo di quell' albero con le foglie d' oro,
Come si vede al lib. 6, dell' Enetde.

—











Latet arbore opaca
$ Aureus, & folijs,& lento vimine ramus
iis Iunini Tnferna dittus facer,
HELL. Quella cosa mala; cio' la (pia.
> (A della fame. Ha grandiffiima fame, perché non guadagna denari
omprar roba per mangiare. Quando i meftieri non lavorano si dice: i /egna-
, alzolai, ec, arrabbian della fame, cioè non hanno da lavorare.
erai il forame. Per befiar' uno, che dandosi a creder d' haver fatto
gno a [pele, e dispetto nostro, e non tha fatto, diciamo: Tx ti
. Qui vuol dire: tu credevi di haver guadagnato il quarto, che
sca alle spie, ma non e flato vero.
»1W.A, Fu figiivola di Giove, e di Cerere, la quale fingono gli an-
» che efiendo un giorno a corre i fiori,futfe rapita da Plutone Re del?
» 5 fatta sua moglie: Ma Cerere non potendo comportare, che la figliuo-
imanefle appresso ai rattore; supplicd Giove, che volefic Jevarla dall'lnterno,
eae le, pur che ella non haveffe preso cibo alcuno; Ma havendo
pina mangiato alcuni granelli di Melagrai:a non potette ulcire; Cerere di
0 supplicò, e stimolò tanto Giove, che ottenne, che Prosperina steile fei
'dell' anno nell' Inferno con Pintone, e fei mefi con la madre in Cielo. EB
. con

















366 MALMANTILE ©



così Praferpina ref sei mefi in Cielo, dove e chiamata Luna', @ fel




ferno, dove e chiamata Proferpina, ed in Terra è chiamata Diana,
£

triplicata efienza Verg. disse:

Tergeminamque Hecaten, tria Virginit ora Diane,
E perché la Luna fei mefi dell' anno cresce, e fei mei feema, perd'i
till finfono, che ella stesse sci mefi in Cielo, € fei mefi nell' Inferno,





+ teeth die

goers



no splenda in Terra, ed e detta Diana. A questa finzione allude Dany

Ada mon cinquanta volte sia raccefa
La facia dela Donna che qus regge ¢
GELATINA, Brodo fatto con la carne di porco, € rapprelo; e si fa
' i)



brodo di pesce. Vedi sopra C. 2. stan.

2. stan. 15.

STIZZA, Ira, Vedi sopra C. 2, stan. 78, al termine fw piccine, Era
velo, collora, e simili, i potiono dir finonimi di stizza, quando è prefa in
senso; che per altro diciamo fizza Vna specie di lebbra, che vien€

ad altre bestie.

S:AREBBE derma. Questo termine significa Havrebbe animo: Si farebbe
to, ardirebbe, non la guarderebbe, ed ha lo stesso significato, che Son | i

detto sopra C, 4, stan. 29.

BVTT AR 4: pie la forma del cappello, Cio' buttar la testa a i piedi j) Teoncare

il capo, che è la forma del Cappello.
STANZA XXIK

La Maga senza dir piit da vantaggio,
Metr'egli a/petta un po di miacia,e intuona;
Ripiglia prontamente il suo viaggio
E incontra Nepo già da Galatrona,
C” havendo dato Id di se buon faggio,

Jn ongi e favoritoye per la buona,

Perché Breuffe in oltre a' premi, e lode

L: ha di pin fatto Diavolo a due code,
s tA NZA XXX,

Hor che gli arriva ail' improvifo addosso
4 venir della maga ch'e il [uo cuore,
Lui Mago pur tagliatole a fus dosso
Le [pedisce per suo trattenitore,
Alentr'il petardo col cannon pi grosso
Sentefi fargli frrepitofo onore,
Cavalier Nepo, com' io diffi dianzi,
Col riverirla se gis affaccia innanzi,

C'ENTREREBBE il frodo, Ci farebbe la pena d' haver frodata;
nifeftata la roba, per non pagare il dazio, o gabella. |
LN gogna. Alla berlina, che e quel gaftigo vituperoso, che dicemhmo:



veh a


















ira

its

STANZA XXXL
E perché 4 Benevento est com luk,
Com! ¢i di lei,bavuto havea
Won prima si riveggon ch' D
Rifanno il parentado,e o a
Tra i diavoli poi vin nei '
E percht Martinazza v' e novitidy
E non intende il gracidar che @ fam,
L interprete fa egli, e il torcimanits
STANZA XXXIL
Per via informa, ele da molti
Diufanza,¢ lyoghi,e intanto di bua es
Lo guida ai fortumati Campi Elifis
Dove si mangta,e beve hi we
E tra quei rofolaceé, € fior
Si peat tong far diquatcroeaaty

Chi un baloccose chi ut z
Che li non e un negorio per | if























CANTARE

PV paar ar XXXIV.
 Quivi si fa al palione,¢ alla pilerta,
| Parte ne giuoca al Suffise alle Murelle,
Conte carie a Primiera un'alerafroca
A confartini giuoca,¢ le ciambelle,
etri fanno a Ci è alla lotta;
indovinells.,¢ chi novelle; (gio
7 lie fiorienn'altrounramo aun fag-
ths cagliata, ¢cou esse canta maggio.
A XMXXV,
Altri pigha, o dispenfa del tabacco
Altre piglia le mosche, un' altro grilli
E cates quanti in quei traftulli immer se
Sirengonail tenor y(t vanno aiversi,
Za iL Ȣ-s?incontrd ia Nepo da Galatrona molto
o da Plutone', il quale per fare onore a Martinazza da Jui tantoramata.,
i ptrartenitore 5 apendo cheerano amici-, Così dunque.ac-
a Nepo sche de faceva J' interprese, perché ella non intendeva il
voli, se ne pafsd ne i Regai-bui; edi) primo Juogo, che ved-
; ono. Campi Gli;,¢ quati il Poeca descrive ripieai di quei trattenimen-
; e fanciuileschi., che (on-soliti facli dai bottegai più vili per le feftivica
1 fubucbani,come sono le Ville degli Strozzi, Pucci, e Gerini, doves
f pola per godere allegramente, e seaz' un peofiero al mondo quel-
fa, che concede la campagna-, ¢-sospendere alquaato i pensicri aviofi del 1

|, | hvorare

wy













IL AMANCLA. Vedi sopra C, 2, an, 68.
bh ANTVON ARE, Vuol dive dac priacipio.al canto; Ma qui significa chiedere
y/ —$ON inetti, © cennila.maacia;.¢ ci serve per intendere domandare con ceani, o
o i quaifivogiia cosa: per esempio: Ll talc insuona, vorrebbe andar' a cena,
a ta bostega, ec.
o 'aiatrova', #uvuno nel contado di Galatrona luogo nel Valdarno di
gt »9. conpolueci simpatiche, o.con altro medicava tutte le ferite, e 1
yl} buomini,, come di beflic,senza vedere il paziente ma folo'ia fu le» \
y@ = pezzebagnate nel faaguc di ello, o sopra un panuo, che havetfe toccato lo Rrop~
IL PiO} per le-betic in qualfivoglialor malore, pigliava la loro cavezza', o bri-
ge ia » e sopra quelli diceva alcune parole, e le medicava; e per que-
'i sua lica superflizione da molti fu stimato stregone, come lo stima il Poe~
i seers. ioe eonofeiuto con Martinazza a Benevento, e che era mago
afuo-doflo.
è DAR bran fore di se.Pacfi conolcere con le sue azioni-per huomo di garbo,.
wt = © prudenre, o-virtuolo.
eo 'ER (a-buona,. S? intende,t per la-buona: firada;\¢ vuol dire.. B' in\ buono
yf Mato; Gitira innanzi bene..

|, BREPSSE.,. Intende Plucone; ed & lo stesso',.che la: Bilisrfa y colla' qual voce»
inno paura le Balic a' bambini, furfe dal Lat. &rebus, originato così; Erehufe, y
le. Me uv Lies

—S
























# I er
oR

oMALMAN ore *



268



addosso una' lucertola 'con'due code sia fortunatissimo in:
larmente nel gitloco, ¢'percid vuol dire, che questo' o atiffi
grandemente privilegiato da Plutones percht haveva le due code }! =)
GLl wrriva addofo, Cio' sopraggiunge inaspettatamente a Plutone''t
Martinazza tanto amata da lui.) > nImAS a8: /
T AGLIAT OLE #'fuo dof, Fatto per appunto come lei, che havi mede
nj, ed inclinazioni, che ha lei. Traslato da' gli abiti 5 che si dicono sagtiati 4
doffe quando tornano bene in doflo. e 2S arand on
TRATTENITORE, Si dice quel Cortigiano, che vien di a ferniret
Ambatciatore, o altro foreftiero, che sia ricevuto, e spefato'dalla Cortew >
PET ARDO, Specie d' artiglieria nota, che serve per'buteare a terra 5 1¢ por
te delle Città. In Latino fa detea da Famiano Strada con' voce Greta
Pyloclaptrum, Quasi Spezzaporta. op i» 2001y(4 ob oats
RIE ANNO il parentado, et amicizia, Quando due amici flati 'lango!
Jontani l'uno dal' altro senza vedersi, si ritrovano insieme, e fanno le:
diciamo; Rifare il parentado, e l'amicixia, NOVICE Tt
VB novizia, Non v'é pratica, perché non v' è mai fata in qu
hospes, € noi lo traslatiamo ad uno, cheé nuovo, e non praticato in
affare. Lat. nonus, rudis, i ages oe
GRACIDARE, E' proptio delle ranocchie,'ma'qui intende il
voli, che forse se lo figura come quello delle ranocchie'.' Dan. Inf.
E come a gracidar si sia la rana,
INTERPRETE, e Turcimanno. Si poslono dit finonimi,se non che Znterprete¢
propriamente quello, che esplica i fenfi delle parole, e Turcimanno & ¢
parla in vece di coluj, che non intende il hnguaggio,riportando le parole; che»
sente dire nella lingua dell' uno, e dell' altro respettivamente, Da alcuni dicefi
Dragomanno dalla voce Greca Dragomenos, che significa Znterprete usata da' Greti
Orientali de* tempi baffi; da Tbargum, che in Levante significa interprerazione,
Thirghem in Caldeo vale ¢/porre jesplcare, e da questa radice e detta a
Thargum la Parafrafi Caldea della Scrittura. Ma hoggi Turcimanne da i pill q
tende ruffiano da quel portare le parole. BH
DI buon trotro. Paaiainaids di buon paffo. Trotto diciamo una specied' an-
dare del Cavallo, che è fra il paffo ordinario, ed il correre, ed è il latino ie:
cajare. =.
eal Elifi., Bil creduto Paradifo de i Gentili. Vedi sopra C,2. stan. 68:
e4 BERTOLOTTO. Senza pensare al pagamento, che si dice anche 4 Vf
a Youne; a Scrocco; a Salicone, Vedi sopra C, 1ftan.77, e sotto C, 7, stan. 5.
ROSOLACCH, e foralife. Specie di vilissimi fiori Aloette A
PAR di quattro, ed' orto, Seiben par-che voglia dire giuocare inuitando di
quattro, e d' otto; tutra via s' intende fMarfene senza far nulla, che si dice::
'ar ateco mec, dondolarfel4, farea tn me gli hai, ondg un nostro Poeta moderno —






















H ae '
SESTO CANTARE:

ative. errno al mattarin crepuscolo
Weddin — me gli bai,
nathan 'oponete, a concludete mai,ec,
oe vine Trattenimento «Da Badalueco, che vol dire pro-
: ia.» © leggicre combattimento. Latino velitario, e figurata-
» © trattenimento piacevole. Ma la parola balocco, © balocarsi &
bambini; e nel contado è preso per indugiare.
' grandissimo, quasi dica spaziofo tanto quanto un' occhio ¢
pO +
UETT£,. Diminutivo di mucchio, che vuol dir quantità di cose riftret-
» quasi monticelletti, Latino cumuli  acerni; © Così mucchietti di gente
s d otto, o dieci tare riftrette insieme. Dan. lof, C, 27.
| B di Pranceschi fangainofo mucchio
i  » Sorte le branche verdi si ritrova..
pure il mondo in carbonata.. Diventi carbone, e abbruci pure il Mondo,
i, e vadia fottofopra il Mondo.
un fastidio di niente.. Non vuol sentir noia, o pigliarsi pensiere
che si vuole, o dibene, o di male,
. Ballare senz' ordine, o regola. Vien forse da Ballunchiare
»» chefe bene è parola non usata,pur |'usd il Boccaccio Nou. 72. pe:
ballo di contadini.:
Strozzini.Gii Strozzini,come habbiamo d., una villa de'SS.Strozzi po.
a da Firenze,così detta.Si come.il Prato del Pucci, e del Gerini sono due
aburbane.de' SS. Marchefi Pucci, e Gerini; a' quali luoghi, suole l'eftates
plebe Fiorentina'a spaflarsi, con far merende, balli, ed altro, che le tor-
o,come dice il Poeta nelle presenti Ottave.
pallone  e alla pillotta, 1) pallone e una grossa palla da giuocare fata di
€tipiena di vento, alla quale si di con il braccio armato d' un bracciale
nO: ela pillotta e una palla piccola pure ripiena di vento, e se le da con
a di legno. Quefii giuochi di palla, sono antichi, perché secondo Pli-
% 59. furono troyati da un certo Pytho. Herodoto lib,.1, riportato da
slid. Verg. lib. 2. cap. 13. dice, che l'inventafiero i Lidi. Alea verd teffe-
» farumgue ludos, & pila, cateraque luforia recreandi animt gratia inventa,
a» preter quam talaria, Lydi populi Afi omnium primi, cxcogitavere &c. Ac-
-» qui Lydos ciufmodi aleatorias artes non tam voluptatis, quam.compendij,gra-
» Un excogitafic idem Herodotus tradit, nam cum gravitate annone patria tem.
» pore Atydis Manis Regis filij premeretur, fic famem confolari foiebant, alte-
$9 £0 quidem die cibum fumentes, altero ludis operam dantes; atque hoc modo
, -% inediam folantes vixere annis duodeuiginti, E da' popoli Lyds alcuni voglio-
D0, siccome è Ifidoro nelle Origini, che venga la parola Ludus, o Ludius, che
lo stesso, che Iftrione. E ognun fa, che i Lidi dal' Afia pafiarono in Italia, ¢
popolarono l'Etruria, ovvero Toscana; E da loro i Latini le cirimonie facce,.
dudi, che si domandavano scenici particolarmente appreero; EB Hifer in lin.
ica, onde e detto Jfrioni significava in Latjno Ledio siccome dice Tito
3 Poi questo nome /udus significante a cae spettacolo attenente, o far.
a. m2 to
































aye MADMANTILBES ¢

to per canfa di religione, si stele a significare: in generale 0
ib 1, e Suida dicono, che Anagallide Gramatica diCorfl i
mento della faltazione a palla,-cioè del gi alla palla at
a Naafica figliuola d' Alcinoo Re di' Corsu'y wolendo fare questa
il vanto d' una tale invenzione a/una faa paefana, & veramente Naufica:
» Del reflo Di

trodotta fola tra ' Eroine da Omeru a givocare alla palla

attribuisce quest' invenzione a' Sicionj, e Hippafo altro Autore citato da,
a'Lacedemoni,come ache tutti gli altri corporali esercizzj-E che-futie mol
to dagli Spartani, o Lacedemoni lo mottra Properzio in quel vero
veloct fallit per brachia iatiu, delY Elegia che cominicia,) Atuita tna\, t
vamur inra paleftre, Dal che si viene in chiar, che il giuoco della;palla sia ante
chiflimo 5 e si può credere col Soutero de Jud, Veterum libs 3) Ci 14. e! id,
Verg. lib. 2. cap. 13. che questa'variazione d' origini proceda dall'havere havuto
gli antichi diverse ipecie di-paila, i come habbiamo noi', e che gli accennati ia-
ventori habbiano ciascuno 'inventata la sua species perché:se noi habbiamoiil pale
» lone; i Latini havevano, ipfe follis, pila, & ipfis genus; conftarque V

3» to inflata. Habbiamo la pillotca', & est ib follicuius, pilay 6 ipfa parva, &
»» similicer conttat aluca vento inflata. Simile a questae la palia i

in-vece d' esser ripiena di vento,°¢ ripiena di borra'; Ja' qual palla hoggi per lo
pil € usata da i contadini, © questa havevano anche gli antichi e la diceyano 2-

fa paganica, F 3 Syaapedt
Marz. lib. 14. Hacyqna difficilis turget paganica plama y—. bd
Folle minus laxa oft, © minus artta pila; ee °

Habbiamo la palla simile alla bonciana', ma assai minore ¥ che chiamiamo pala
desina, che pure ? havevano anche \secondo alewni i Latin, © la dicevano Pila
fixentina, perché forse nel pacle Fierentino si lavoratiero le miglioris Habbiamo
la palla facta di cenci impentita, che i Latini pure havevano, e- la chiamavand
co' Greci Phannida, o vero Harpafium, pesche te ne servivano per far il gvoto
y» da noi detto il Calcio secondo 1] Sipontino, che dice, Harpattumy pile genus
x» elt; grofhor, quam pila paganica, tenuior, quam follis; E panno fere fit,
aliquando ex pelley lana, «omentove impletur, Non repercutitur, fed cum,
» multi fint Judentes in duas partes'divifi, ita ut utrique &-regione fibi inwicems
oy Oppositi fine, ad fuos quilque transmittere pilam constur y quam aduerlari) co
y» Hantur arripere; Alarpajtum diem a Giseco Aarpayin, quod eftirapere, quia
3» proietam pilam mulu fimul conantur ariipere, fed ob cam causam inuicem
> profternuntur, ve
Marz. lib. 7. ep. 31. Won harpasta vagus pulnerulenta rapis, A aie
Habbiamo la paiia a'corda, che terue per giuocare con'la-raechertasnelle Manze
fabricate per tale effetto; ed'etli havevano pram trigomalemscost decta 'non perché
futie di figura triangolarejma perché era triangolast la stanzajeove conefa
cavano, e per dare a questa pallayfi servivano del rericixd y che & JotieNOy chet
racchetta 5 o laccherta,come accennammo sopra C.3. tan, 58. -Di questa lacche
ta parla Ovid. lib. 3. 1.4 M
Reticutoque pile leves fundantur aperto, - nt
Nec, mfi quam tollas, ulla movenda pila off.

ss







BB aeFF FEES SEH TF ROPSCFARRP xf RE MASTS SOW HEELS Sees.
SESTO CANTARE.

tebe wit DARE Dawe
tepidum dextra, levaque trigonem.
tichi wfafie la palla-ripiena di borra od' altro pelo, si cava
; tino riportato qui sopra, e dal nome di efa, perché
» che sia decta Pia dal pelo,col quale è ripiena;.se bene altri vo-
wenga dal Greco Pefeo:, ideft equo, perché € di figura sferica, che &
ogni parte, o pure ( il che € pil: verilimile) dal pee rh cioè
ibrata, e sbalzaca, e perciò anche in Greco, si come in Toscana è det-
Dionifodoro antico gramatico, dove nel tefto dell' Viiflea co-
leggevafi Spheran, col qual nome chiamano i Greci da paila; si di-
i (criveile Patian. come per chiofa, e interpetrazione della voce d'Ome-
questo vien riferito da Euftazio, che sopra quel Poeta il gran comento
, Che i Greci ancora havessero motte specie di palle,si può dedurre non folo
cfere stati inucatati i giuochi di palia nel tempo, che fiorivano i Greci, es
| flo di oro la Spheromachia, l'Amilla, ed altre specie di giua-
ileriti da Giulio Polluce, e dal #ulengero; ma da quello., che scrive
ino lib, 20, C. 14. dove dice, che fra i Greci giuocavano alla pallas
huomini, che le donne; e ciò cava da Homero. Si trova in oltre, che
'Siracufano giuocava alla palla, ed alla pillotta per ricuperar le forze.
ex ab Alex. dier. gen. lib. 3... 21. £ si può credere, che si come noi habbiamo
'diverse palle, e\diverty modi di giuocar con esse, così non mancaffero a loro an-
“coral invenzioni per soddisfart.
| AL soffi,£! wn givoco solito farsi per lo più da ragazz' in questa maniera. S'uni-
-eono dues più ragazzi,¢ pigliano una pietra, € polatala per ritto in terra vi
c ra quel danaro, che son conucnuti di givocare, ed allontanatifi in,
liftanza, che sono-d'accordo, tirano una jaftra per uno ordinatamenie
q i¢tra ritta;sopr'alla quale sono i denari,¢ che si chiama il Suffije se que.
A 'Wien colpico, e fatto cadere,i danari, che cascano,sono di colui, la lattra.
del quale ha fatto cascare il fut, se però sono più vicini alla sua Jaftra, che al
- fal moneta, cheé più vicina al fufli,te gli rimette sopra., e quello a,
cui) ira, e seguitano,come sopra, tanto \che la moneta mefia sopra al sul
“tei finita'di ievare ne) modo, che sé detto. Da questo giuoco: habbiamo un,
Proverbio che dice Efer w/usfi, il che significa esser queliberzagiio, dove ognuno
“tira,cioè sopra il quale devon cadere tutte le burle,-c tore le minchionature..
 Questo giuoco & forse lo stesso, che da' Greci era detto Epbedri/mo, feconge Giu-
Tio Polluce, il Buieng. c. 48., ed il Meurs, de lud. Graecor,, te bene non giuoca-
~Vano denari, ma colui, che non butcava in terra ai futh,portava a cavalluccio
» quello, che to bucrava,il quale gli turava gli occhi colle mani, finche (enza errare
bb portale alla Jafira, o pietra, che si chiamava diores, cloe Adeta o Confine,es
-facevarquello, che comandava il vincitore, il quale in questi loro giuoch era,
hiamato'Re, ed il perditore era detto Mida, 0. vero Asino, come habbiamo vi-
“flo altrove.
 | MPRELLE. E? giuoco simile alle pallottole, se non che in vece di palle ado-
oo eee laftrucce, ed un piccolo faflo per grillo, e tal giuoco si dice anche pia-
eae wee «®




27 MALMANTILE? 4

PRIMIERA, Giuoco noto, che si fa con le carte. = =)
FROTT A, Flotta,, o fiotta. Vuol dire quantità di gente unite i



muove; dal Latino fluttus, Virg.Georg. dane Salucantum toris vomit edibus andi,

Varchi Stor.lib. 15. Z vedendo sopra a un monticello non molto
frotta di. contadins. DEIN
CIAMBELLE, e confortini, Sono specie di pafie fatte col zucchero.
uova,e queste son poe a vender da alcuni pi pel contado, dove si
¢ raddotti, che in Città; e questi portan seco anche le carte per giocare,:
quali hanno diverse invenzioni di giuochi, come la mora, il tocco, ec. E
venditori quando giuocano, danno in vece di danari quei confortin' 5 e cis
se perdono; e se vincono,ricevono danari. L, circali, ernfiula. /
CIVETT-A, Quel giuoco fanciulle(cho, che dicemmo sopra C. 2, stan. 41.
INDOVINELL1, Latino griphi,enigmata;Quello che in latino dal greco i
enigma, noi circoscrivendolo diremmo detto oscure, e diffeile a i
E la voce enigma s'¢ fatta Toscana, e l'usiamo come l'usd il Malatefti nellay
sua Sfinge, Vedi eras 8. stan. 26, a vege!
CANT A Haggio, Nel principio di Maggio sogliono le Ragazze plebes
di Firenze, o del Contado sbeceane scomeni foe Pr) eee ¢
di joro-in mano un ramo d' albero adornato di fiori andar cantando ere
diverse canzonette per l'allegria del nuovo maggio, e per buscar mance da
ro, che si pigliano i) paflacempo di farle cantare al suono d' uno strumento.
cembolo, che è un' afficella ridotta in cerchio, e fondata di cartapecora da una
parte fola a guisa di tamburo. Questo costume di rallegrarsi il Maggio viene dal'
antico, e si trova, che appresso i Romani Xalendis 5 Nowis, @ Zdsbus maij Lari
Deo facra ficbant afello panibus coronato, Quindi forse ancora Maggio si chiamavdl
mese de gli Afini, che per altro fu detto men/is hilaritatis, Che nel mefedi Mag:
gio si faceffero allegric forse pil di quello, che comportafie ' onefta, e lavert-
condia, ne fanno fede gl'Imperatori Arcadio, e Onorio nella loro Costituzion
inferita. da Giustiniano nel Codice lib. 11. 45. de maiuma, la quale era una alle
gria, che si faceva per il Maggio secondo che spiega Suida. Da questo mele quel
ramo d albero, che i contadini piantano la notte di Calen di Maggio avantiall
uicio. delle loro innamorate, si chiama Adaio; Questo costume d? appiceare:
maio alla casa della Dama e riferito come proprio anche della Francia da Mat
ziale d' Aluergna ne' suoi Arrefti d' Amore, all! Arrefto quinto, il quale scritt
re fiori nel 1400. Qual luogo Benedetto Curzio comentando dice; Prima d
Maij menfis invenes pluribus lndis, ac iocis fefe exercere confueverunt, arborem fapean
mero deportantes, ac in loco publico, aut etiam aute alicnius egreci virt januam 5
frequensiits amica fores pl vel promifinis adamantil
Signijs, atque emblemucibus.

isarote
BRANCO., Quantità di popolo indeterminata; ma si dice più di bestie; com?

branchi di polli di pecore,di buoi, di aGni, ec, Vedi in questo C.  Ortawa
seguente. 4 sacl
HA mofo} Ofte a facca. Cioè mangiato, e bevuto quanto I Ofte vi haveva-»
nel modo » e con quella furia, che segue nel dare il sacco a una Cittas ae
sopra:

(EN-

MEZZL brilli, AMez2i briachi, Brillo vuol dir briaco allegro. Vedi &
2, stan. 69. At










Bee coc eee e2@Heekse ewer es) =!



eseF SF AS Soa






273
a bacco, Vna villantella che si canta per incitare
Aidlors eg MeL gave

»\\) Faceiam brindis a bacco, © vi:
questa,va il bicchiere attorno, ed ognuno' beve,intuonando prima.
però dice mentre ice 3 cioè mentre il bicchiere va a tor-
/perché tal-costume è usatissimo in simili allegrie, però il Poeta, che s' in-
mostrar, che quivi si sta in felte, e in giuoco, dice che facevano brindis
jot cantavano'bevendo. I Latini dicevano Propinare, cioè prebibere dal
tim, che suona lo stesso che il far brindis, ed usavano anchreffi questo
bere in giro', che dicevano ia orbem bibere, & circumferebant feyphums
¢d essi pure cantavano in tale occasione di bere; come scrive Dione, che
e Roi aC l'quando al banch che fece
bevve a un bicchiere, che li fu porto da una bella femmina.
'brindsfi. Se ben pare che venga dal Tedesco pringen, percht volendo
) a*nazione bere, 'ed inuitare il compagno,suol dire: Zk Mellan-
'y che vuol dire' /o ve lo presento; e questo già facevano, perché quel vino,
havevano'a bere restaffe benedetto dal Compagao, il quale foleva rispondere
nges, che vuol dire Dio lo benedica. Tuttavia il Lalli nella sua Moscheide
“61. graziosamente gli da origine dalla Città di Brindis, dove chi va ad
f è da ogni veflazione curiale tanto Criminale, che Civile, onde a.
 faevbrindifi par che sinviti uno ad andar ad abitar quella Cited, cioè a lasciar a
¥ 'parole del Lalli son-queste: i
|. \Brindifi bella s io m appongo al vero,
Date son meffi i brindifi in usanza,
Quafil buom dica; Lascia ogni pensiero;
Beviamo allegri, e rinfreschiam la panga 5
E se pui il creditor duro ye fevero
Ci fa da' birri apparecobiar la franzAy
Brindifi habbiamo, Brindifi diletta,
Che quanto più si bee, vie pss n' alletta;
paglie,o /pilli. E' un giuoco da' fanciulli, che si fa così: Pigliane
due corte fila di paglia, e posandole sopra un piano liscio vanno
'4spingendole con le dita tanto, che uno di detti spilli, o fli cavalchi l'altro,e quel-
lo, che resta di sopra vince, giuoco così detto dal Ter?, cio' tog/i, regi. In La-
tind tudere aciculis, E perché questo giuoco-¢ di niuna, o poca conchiufione, hab-
- biamoil proverbio s Fare 4 tes? con gli spillerti.; che significa affaticarG, e per-
'Mere il tempo senz' utile, © profitto; ed esprime ancora Far una cofacon fordido























S' aiutano l'un I altro, e's' accordano.
Ww XXXVI.

me ZA
| “Za donna refia litrafecolata,. Per tutta la Cutta vien falutata,
©) Fedendo quanto bene ognun si spalfa; E infinle stanghe,e ogni forcon s'abbaffa,
i he Nepo I ha di già infor mata, Ed ella hor qua,bur la voltando inchini,
ragiona di lorma guarda,e pala: Pare nna bandernola da cammini.

STAN:

ST vengone il tenor, si vanno a' versi.
* STAN






























a MALMANTILE::
STANZAXXXVAL. STANZA KEI.
è

Pera che tutti quanti queit Demoni y Percio pafjano in casa ©

Per vederla, n'uscian di quelle grote, Firatoconta Stree il Reda
'Ronzando con un brance di moscioni y. Le da la ben vennta,e vento
Che saggirin d'actorno.a una-botte 5 dt Le ipieaie sete: nar ds
Saleellam per de firade,e fui balconi y Elia per confeguir ogni (uo jutento:
Comal piover a' agosto fan le borte, hf "

Sees






E fan, vedendo [ue fembianze belie,, bails
Voei altese fiochese suon di man con elle, -grazia anchiet di dar i
STANZA XXXVIIL, STANZA KXKK Og,
Così fra quel diabolico rombarzoa Sta pur, dic' ey cont anime (ato y g
La fhrega se ne va con lo firegone » C' a servirts mo mo vuo dar di piglia,,
Sic alla fine arrivanoa Palayro lo.già,come tu fai, baveo impranate;,
La-dove s' abboccaron con Plucone, Ma il tutto.d andato poi in iscompigha




Ma. perché tra di laro entré-nel mare Horfu: fra poco adsunerdil fensta y. ¢
Scwwccamente il eALandragora buffone y E sopra questo si farò confizka y.0
Chiin quel cailoquio fesigran fraftuono, eAicio Baldon batta ig ritirates,
Che finalmente ognuno uscs dt tnono-, E tu reftt consenta,, econfolata,. >

Martinazza resta maravigliata, che costoro stieno cos: allegramente 5 e pak
fando pel mezzo a una infinita di Demonj, che-cutti la riveralcono., giunle coms

Nepo a Palazzo, dove se le fece incoptro Plutone, che la condutie dentro\,,e>

quivi havendole essa detto il suo bisogno, Plutone se peomess di confolarla..

REST A trafecolara, Resta.maravigliata: Strabilisce. Vedi sopraC, 1, st. 28.
ST ANG A. Pezz0 di travicelio,.cioè un legno grosso.pii d' un bastone.
FOKCONE.. E'un' afta'di legno sopra, alla quale e adattato un tridente di

ferro, e serve per uso delle Malle.

INC HINO. Vedi sopra C..1, stan. 34:

BANDERVOL A da Cammini, Bandecuola vuol dir piccola bandiera, o pen

noncello, che è quel pezzetto.di drappo, che già portavano:j Cavalleggieri appie-

cato vicino alla punta dellatancia a guisa di bandiera; ed a guisa di questa ims

Firenze se.ne vedono fatte,.di lama di ferro potte in più. emincati luoghi delles

case, come (ono le pergamene, dond' esce il fumo dei cammini, equelte servo

mo-per far conoscere i venti col lor girare, € voltard in sul. ferro: y.nel quale sono
jnfilate, e bilicate; ed a queste allomiglia Martinazza, af

RUNZANDO. Ronzare si dice propriamente delle mosche; e però.dice Ce

me fanno i moscioni, che sono-quei piccoli- vermi alati, che nascono-dal vino.
COME fanno le botte-al piover a' Agosto.. S'¢ veduto dalla sperienaa, che la piog-

già di state,cascando nella poluere scaldata dal Sole inuigorisce le rane,.o bore
nate di poco,,se bene molti. hanno creduto, che le faccia nascere quell" acqua,coo

quel Sole; il che è falfo, perché prefe (ubito scappate dalla poluere si son trovate






























Exé SE vo ESE ZZ mERZ SHE aes TLS ES







col ventricolo pieno-d' erba,. Ma sia come si voglia,basta che atal”acqua

gono-faliar, ma:d'.un salto debole,.¢ fico, spon come il Poeta: sesptl-
“mere yche faltaflero quei Diavoli. Va Posta faccto Piorcntinodel¢rivend a
* sual. cavadi Mancii-ia. unsuo Sonetta-dice:; i



Seae


H SESTO CANTARE. 275



ro Si fivergognan che paffan di norte,
seat oe oe ae feet
Pe " sides \ Trottando, e faltellando come bette,
OCT alte, e foche, e suon di man con elle Così canto Dante Taf. C. 3.
Ke @”. Intendi frida, e'colui, che continova 'a gridare,afhoca per l'affati-
"cam 10 dell" aspera artetia, si che il secondo nasce dal primo. E /uon di man,
im elle'; cide con quelle voci accompagnano il romore; che fanno co: batter le»

| “ROMBAZZO, Rombazzo vien dal verbo rombare, che vuol dir ronzare,o frul-
Tare, che @ quel romore, che fa perl' aria una cosa lanciata con yiolenza, e si
'Piglia per ogni sorta di Arepito', o fracasso. I Varchi Stor.lib. 10. in questo me-
¢ signiticato dice bombagze voce formata dal fnono,nella stessa maniera, che
; )formd bembus: Torma eAimatloness implernnt cornwa bombas, perché dice
lunge prombertare, e spampite fatve con invredibile Bombazzo, quasi in tal
Mn paffero e nimici. Ma |'Autore oon Storia di Semifonte dice al trattato 4,
! atone la T erra;allotra fenritofi per quelli della Città il rombazxu, E V'ulo
puechecl obbiighi adire romibazo v

| le nel mazzo, S'accompagnd con loro, Che diciamo; incrn/carsi, fic»
è ien'dal giudco del mazzolino detco sopra C. 2. stan. 46.

' a apora', Costui era un buffone, © più costo un matto di Corte, che
)  chiacehierava (empre', e senza propotito, o conchiufione.

j KeVIO. Voce latina fata di rado in Firenze,e vuol dire Ragionamen-



LSS" CO



s

to, che fanno insieme due, o più persone. Corrisponde alla Greca Dialogos, che

ifica indo la: parola Znrerlocutio, dilcorlo che si tiene fra due,o più persone;
dai Pranzefi detto Emtrerien quati Trarrenimento,

VSCH di ruono, Perde¢ il flo del ragionamento,si dice anche:a/cir di tema, fmar-

rite argumento, il proposito. Vedi sopra C. 2. stan. 47. E' pref la similicudine

3 feherzando fal doppio significato della parola scordar/f |a quale tan-

tOVi dice' d* un' huomo, che non si ricordi piti di quel che ha proposto di dire;

'uno stromento, che non sia in corde, e non sia temperato al giutta

O@ vnc, che non canti ginflo, e fuor del legitimo tuono, il che si dice

TIRATISs de banda. CondottiG in un' altra parte della stanza, feparatifi, o
allontanatifi da quel congreflo.
CHE vento? bat [pinta in quelle bande, Qual cagione l'ha moffa a andar in quel
0". S.

 TRABALLARE. E' quell ondeggiamento, che fa uno, quando non può fo-

: in piedi, Mattio Franzefi in lode della Posta dice,
cd Chi domanda per nome la cavala,

Chr eels ha sentito dir, ch' è favorita
Wy} Poi partendo chi trotra,e chi trabala,
Qui vuol dire, che Malmantile era in pericolo di cadere, cio' esser preso da Bal-
done, Diciamo in questo senso anche baienare, barcollare, \n certe rime mano-
scritte nella' Libreria di S, Lorenzo, si dice d' un cotto, che barcollava: Es'¢
Yalena, @ non batena a (ecco. Qui si nea sul doppio significato di balenare.
? i a Mo

5 FS = BERS SRE SAS


*





=







276 MALMANTILE ©
40! md, Adeffo adeffo. BE' il latino, omb:
Firenze. L' usd più volte Dante nel suo Poema,si co
re altre parole Lombarde; B il Bocce. Nov. 32. 4@ vi
Jaca della donna, ch'era Veneziana. preys
DRO' di piglio. Dard di mano, cioè comincerd.
ficaya quasi quel, che 1 Latini dissero Expilare; i Franzefi p
dier nel (angue, e nell' aver di piglio, E.'l suo cont ) Fazio.
Poema che fece in terza Rima, ove e introdotto Solino a dettare a.
di ecografia, e del mondo; che perciò lo intitold Dita mundi,
andò; dice così al canto 142. ( Parla del Saladino ) i
a Costui per (un francherza, e gran consiglio y
Tolfe la terra fanta a' Cristiani
Vincendo quegli, e dando lor di pe lion SNe
FLAVEO imprunato. Havevo ordinato il rimedio, Vien da quell' imp
che dicemmo sopra C, 3. fan, 21, Addio fave. i
HOR sik, Termine esortativo,e conclusivo,e diciamo nello stesso senso. 01
quasi Or via, Latino Eia age, Vedi sotto C, 12, stan. 47. Diciamo + hor fu
diciamo bac ipfa hora furge,& hoc factas, aga
BATT Ala ritirata, Sene vada da Malmantile, Batter la ritirata
0] tamburo si fa quella fonata, per la quale i soldati intendono doversi
¢ lasciar ! imprefa. Gio, Villani ciò disse; fonare (a ritirara; quali
il Franzele, retraite..
STANZA XXXXL STANZA XX
lo ti ringraxio st, ma non mi place Dico di Gambasporta il tuo v
Percio ( gli rispond' ella ) di manieray
Ch'ionon voglia pivlinr laspada'lgiaco,
Ch in bagnela son piit di quel ch'io era;
Così con quei due [pirti bavend» ilbaco
Sagginnge (per c' alor vnol far Japera)
to Vho con quei briccon furfanti indegni, Ches'egl adaffe un polafruftain ve
C' hanno frurbato tutti i miei disegni,  Imparerebbon per nn? aitra volts
STANZA XXXXIII,

























E di quel pallerin di Baconero 4
Che ib oo | giuoco con “nt i
Scambiando it color bianco per
Error, che nol farebbe anc' un cava
Ma e'vit che gli firapazzanoil.

See




7











as




a oi

a

Risponde il Re; Facciam quanto ti piace, Non penso di restar già contumace y —
Ma ti verranno a chieder perdonanzA, S'io non ti servo,perch'iofo a fidanta. '
Si che ru puoi con offi far la pace Dunque ti Lascio,e fone al to piacere;
Pero racquiera,e vane alla tua fiaza, Fatti servir da questo Cavaliere.,

Martinazza ringrazia Piutone, e dolendosi del danno cagionatoli da G:
florta, e Baconero lo prega a gaftigargli: Plutone I esorta a placarsi,¢
che andranno a chiederle perdono dell' crrore; e fatte con essa sue cirimonie
rimanda alle stanze. a '

WON vogtia pigiar la pada, e ilgiaco, Non voglia armare contro di loro pet
yendicarmi. 4 =i eee
SONO in bugnola, Sono in collera, Bugnola si chiama un' arnese fatto dicot —
doni di paglia, entro al quale si conferua grano, biade, ec, da i Latini dettas
cumera, Bfidice cfler' in byguola, nel bygnolone, in valigia, nel gabbione ee

eS.



=




asget






SESTO CANTARE: 277

ee in cOllera. E tutte queste manicre vogliono esprimere il gonfiare,

un fa per l'infiammazione della bile commofla. Orazio Bile wmer iecar; dove
vaveva detto: mexm iecur vrere bilis, Ovidio ne' Balti, Intumuit Luna, cioè

onfidzentro in valigia. Gli Spagnuoli similmente dicono, embotijar/e.

| HAVENDO il baco, Havendo ira. Traslato da i cani, i quali quando hanno

: & n certo baco nella lingua per di sotto, par che sieno (empre adirati, ed il simile,








segue ne i Montoni,quand' hanno il baco, o tarlo dentro alle corna,
'AR la pera. Anticamente s'abbruciavano i corpi morti sopr' ad un montes



; = » qual monte quando era accefo, chiamavano ?yra. Lall. En, Tr, lib. 5.
teil
ae: Già l'alta pira di Didone ardea,
pate £ vibrava lontan fiamme,e faville.
'Edda quefio credo, che venga il nostro far /a pera, € che s'intenda anche am-
na 'uno,quasi si dica: 40 vagtio far /a pira al tale, S' intende anche far /a [pia






'AR fallo, Far' errore. E' termine del giuoco di palla: e però il Poeta se ne
ue'; perché l'errore fu fatto con le palle. Properzio lib.3. we pila veloces fal-
lit per brachia iattus.

NOL farebie ance un cavallo, Error groffissimo, e che non lo farebbe anche
una bestia; e si dice wn cavailo, perché questo animale pare, che habbia discorso, e
giudizio pi che ogni altro animale. I Greci di sippos, che vuol dire cavallo, se
ne per una particella, che aggiunta a' nomi, importa grandezza. Hip-
pomarathram id @ il finocchio faluatico, e Aippomyrmeces, certe formiche,
che paffano di grandezza l'ordinarie, e comuni. Onde errore,o sproposito da.
cavallié mn' error grande. O pure si dice così, perché sia degno di cavallo, cioè
i  di gaftigo, qual si suol dare nelle scuole a' fanciulli.

 STRAPAZZ ANO il meftiero, Cioè nell' operare, non considerano quel che



f — eANDASSE Ia fruspa in volta. Se la frufla andaffe attorno. Se fuffero di quan-

~ doing bastonati, fruftati.

\ NON penso di restar contumace, Termine di cirimonia, che significa:non penso
di commenter mancamento. La voce contumace e Latina; però il Lettore si può
soddisfare circa i suoi significati,

FO 4 fidanza, Confido, che per tua cortesia non l'haurai per male, e mi scu-

A ferai; termine usato fra gli amici intrinfechi,¢ si dice anche; Fo 4 sicurta,

 SONO al tno piacere. Termine usato da' superiori con gl inferiori, in yece di

fF aetgeps Coenen, torent N

ia 'avaliero, Intende Nepo,

J STANZA XXXKXIV.

( ses mena allora alle sue spanve, Ove gli orsi facendo alcune danze

f Cha parameti havean di quoi darn ats Dan la vivanda, e da lavar le mani;
; Ricamati di signoli, e di stianze, Volati al cibo al fin,come gli affori,

| Efepeans di via de Pelacani Sembrano 4 foe fo due toccateri
= Nn 2 STAN.








278
STANZA XXXXV.
Fioritaé la tovaglia, e le faluetce
Di verdi pugnitopi, e di froppioni,
Saldate con la pece,e in piega frrette,
Infra le chiappe frate de' Demanj.
Nepo fra tanto a mavinar si mette 5
E cheto cheto fa di gran boccont;
Osservando Caton, ch' intese il gioco,
uando disse: in connito parla poco.
STANZA XXXXVL
Fa Martinazza un bel menar di mani;
Ma più cheilvitregli occhi al finfi pasce,
E quel pro falie, che fab erba a'cani,
Che il pan le buca, e sloga le ganasce,
Perché refee vi son come trapans,
Ne manco se ne pro levar con Pasce.
Crudveilcarnaggio,e si tirante,e duro,
Che non viene apuntareipiedi al muro,



Prexioft liputvl seve ne foaé i
Portati ciascheduno in sua guafiada
Essendovi aqua fortes inchifro buss
Di quel proprio, c'adopera to Spada,
Ella che quivi frar voleva in tnano, ~
E non cambiar, partendosi,la firada,
Perché i gran vini al cerebro le danno
Ben ben Vannacqua con agreftojeranno,

STANZA IL

E fatte due tirate da Tedesco
La taxxa butta via subito sm terra;
Lero ch'ell'é di morto un teschio fresco,
Che suona, e tre di fa n'ando fotterra,

Nepo, che mai alzi vif da defeo,
E intorno.aibuon boccotiratohaaterra;
Anch'egli al fine dato a tutto il

La bocca follevo dal fiero pasto. '

Nepo conduce Martinazza alle sue stanze,dove era imbazidita la mensa, e fu-
bito si mettono a mangiare. L' Autore descrive la qualita de i j dell'
imbandimento, de i trattamenti, € de i cibi, tutto appropriate a uno appartas

miento, e banchetto da Diavoli.

QVOF humani, Pelli a' huomini. Se ben quoio vuol dir pelle di bestia conciat'
si piglia ancora per pelle d' huomo, come s' è veduto sopra C, 4. stan, 20, € coms

lo prefe il Ruspoli dicendo:

Vn certo ch' in full offa ha [ecco il quoio;
SIGNOLI, Specie d' apostema nella cute;da i Medici detti Puruneull.,
STIANZE. Quelle crofte, che fa nella pelle la rogna; o altre bolle; dai
Latini dette erafe. Varchi Stor. Fior.lib.1.4, G4 trewarono rofo dello fomaco quant
un giutio con una fianga nera sopr' a quel rofo.
SAPEAN di via de' Pecalani, Puzzavano di bestia morta di più giorni, Las
via de' Pelacani si dice in Firenze quella; dove son le conce delle pelli, nella»
quale e sempre wn puzzo orrendo cagionatoye dalle conce 5 € dalla corruzione di

quelle carni

VOLAT I al cibé come gli aftori, Entrati a tavola veloceiente. Avventacifi al
cibo, come fa l'afore, il quale, benché habbia il cibo a fuc dominio,vi s\avven-

ta,¢ lo divora con rapacica grandissima
SEMBRANO 4 folo a fol due Toccatori,



Dicemnmo sopra C. 2. stan. 66. quel che

* sieno i Toccatori, Questi sono solamente due, e volendo andare a cena all'ofte-
ria son foreati andar da lor due foli, che le conversazioni de' galaathuomini oon

li



ee Fa ESOS ee Oe Ren eee se. Se



Rey








2
7
e
e

=

>

=

SE Tyr

,
'
j
7
s
f

it



RET TE in piega, Le faluctte, e tovaglie si piegano in diverse maniere, ¢
si fa loro pigliare la figura, che si vuole, col tenerle così picgate Mrette in un tor-
0, ofirettoio fatto a posta per tal' effetto, in vece del quale strettoio, quelte
0 state firette fra le natiche de i Demonj; e ciò dice. per esprimere, che fons

Deseo
 ANTESE il giuoco... Sapeva,come era conueniente fare, quando disse: Pauca in
loguere.
 FAun bel menar di mani, Si tadia; 8 affatica a mangiare, Vedi sopra C. 1.

LE f il pro, che fal erba a cani, Non le fa pro. Quando i cani mangiano ler.

REST E. Quei fili fottilissimi, che stanno appiccati alla spiga del grano, dell'
orzo, e della fegale; dal Lat, aris.
 TRAPANO., Specie di succhiello, o foratoio atto.a bucar pietre 5 ferro, ed
i altra materia per dura che sia, e s' adopra facendolo girare com una corda,
loi ? habbiamo dai'Greco Trypanon, Vedi sopra C. 4. stan. 73.
NON se ne puo levar cont' ace, E' così duro, che ci vuol l'asce alevarne uns

iT NON viene,a puntar i piedi al muro, Non se ne può strappare a fare ogni mag-

\ BAR lo spiano a casa a altri, Mangiare a casa d' altri (enza spendere. Vedi

\©, 3. stan. 51. Questo detto viene dallo spiano del grano, che vien dato
dal Magiltearo dell” Abbondanza a i Fornai per fmaltire il vecchio, che si ritrova
hei magazzini pubblici, e da questo rifinimento /pianare, o far lo spiano a casa d'
altri intendiamo rifinire, o consumare quello, che colui ha di commeftibiic in,

ECASEO barca, e pan Bartolommeo, Precetto della squola de' ghiotti, che vuol
dire Mangiar la midolla del cacio, e la corteccia del pane.

 FREMERE. EB' voce latina, che conferua appretio noi lo fieflo significato:
'Verg, 1. Bn, Cuntti fimul ore fremebant, E altrove descrivendo il Furore; fremir
borridus ore cryento,

BRANO, Pezzo dicarne (forse dal Latino membrana ) o altro strappato
- violenza, e si dice sbranare; e sbranalo, Vedi lopra C, 2. fan, 52. mandato
abrani.

 CIBREO, Guazzetto fatto di colli, e ventrigli di poli, 2zinueal. Può essere
Originata questa parola dalla Latina Gigeria. Feito Gramatico: Gigeriaex muitis

js ape.
MAGNANO, Quali machinarius fabbricatore di ferti minuti, e di piccoli ia-
i: Begui












280.
gegni,come chiavi, toppe; a distinzione di Fabbro, che

ine Zappe, vanghe, ec, e del Manescalco, che fabbrica ferri

ci¢ i Magnani son sempre tinti di nero, il Poeta dice che il cil

lero interiori,per esprimere, che era nero. ate sh SOR G
VENT RIGL/O, Ventricolo degli uccelli; in altri luoghi oe theta “i






STRIGOLI. Diciamo quella membrana, o rete gratia; che sta
budella degli animali. i

AC QV A alle mule', BE! un detto di gente baffla, che significa date

GVAST ADA, Vasewto di vetro corpacciuto, e col collo lungo, e fire
serve per lo pia tenervi l'acqua per annacquare il vino, quando si beve.
antichi dissero Jngaifara. I) Canini la fa venire dalSiriaco Ga/far, che' v:
ficflo. Potrebbe anche comodamente dedursi dal Greco Gaffer, che v:
corpo; e così Guaftada esser detca dalla figura corpacciuta, nello: stesso
punto che Graffa voce Siciliana usata dal Boccaccio neile Novelle in ]
te viene, si come molte della Sicilia, dalla Greca Ga/ira, un. poco tt
Iettere; la quale significa as vaso che habbia pancia,

LO Spada, Valerio Spada celeberrimo Macftro di (crivere, huomo si
¢ che non resta addietro a veruno nella galanteria del tratteggiare con
mano, e frappeggiare, e far paefi con la penna; come d' intagliare in
bulino, e acqua forte, fu amicissimo dell' Autore, e suo scolare nel
vive ancora; € ben che d' eta sopra settant' anni,indefeflamente lavora p
nare il suo nome.
VOLENDO star in tuone, Volendo star' in cervello, e noms' imb
CAMBIAR Ia strada, Quando vogliamo dir copertamente a uno, Tu fei bria-
co;diciamo. 7' bai fmarrita la Strada, e pero intende;non si vuole imbriacarey
KANNO. Acqua passata per cenere, detta anche 4/eia, dal Lat, dixivinm, I
dottitfimo Ferrari nelle Origini della lingua Italiana, dice così; Ranno; lixsvia
Vude vox ortum trabat, omnibus vestigijs indagara battenus fefellit, Chi fa, che non —
si origini dalla voce Greca Xhanis, che significa, Milla, gocciola, perché il sann0
stilla a gocciola a gocciola da que! vaso, che perciò diceli Colatos ? (ee

ia del vinos4











SEP FRE SSS pec he es ee:

a:

F,

4 ef ert Fz Ez.6 =F








FATTE due tirate da Tedesco, Fatte due gran bevute. Manda gii
Latini dicono: pocula obducere;i Franz, avaller,

SVONA. Di questo verbo fonare ci serviamo per intender copertamentes

mere.

l'MAL alzé vifa dal desco., Stette empre attento alla roba, che era in tavolas+
Termine usato per intendere uno, che a tavola mangi con avidita, e non pigli div —
vertimento di sorta alcuna; E de/co se ben vuol propriamente dire la tavola,dove
si fla a mangiare, onde il dettato: Chi non mangia al desco Ha mangiato di,
oggi e poco inteso per altro, che per quel iegno, sopr' al quale i macellari tagliae
uo la carne, e per quel banco, a) quale nelic Confraternite, o compagnie de'
colari ficde il Giovernatore,
TIR ATO ha aterra ai buon bocconi, Ha mangiato assai de' buon bocconi 5
lo fico, che menar le mani detto sopra. aM

La bocca follevs dat fiero pasto. Verlo di Dante Inf, C, 33. Lascid star dimane
giar quell' orride vivande,. ioe

- STAN-
















SESTO CANTARE 281







on SUBTAN ZA. STANZA LI.
Lasciatii voti, e i piatti scemi Spargon le rame in varia architettara
in anno al giardino pieno di emente Scheretri bianchi, e rosse anotomie,



Di berlines di mitere,e di remi, Gi aborst,i mostriei gobbi in fu le mura
Edi firumenti da castrar da gente; Forman spailiere in lnogo di lumic;












ifede in me2x0 sl paretaio del Nemi Dugna, di denti, e simil' ffatura z

, D'un pergolato,il quale a ogni corrente Lnfeliciate son tutte le vie;
i con quattro braccia di cavexra, Non bel fepoteroanicchia il fore butta
. Penoloni, che sono nna bellexra. Del continuo morchia, e colla firutta.

ee STANZA LIL
sono abbroffolite, e feure Sui dadi i torsi nebili feulture

ie del mar venute della rena, (Perch'in rovina il tutto iitempo mena)

'intorno intorno in varie positure Rispaurati sono, e rifarcité




d | Sep rem leggiadra feena, Da vere, e fresche tefte divanditi,

lito che ro di mangiare, Nepo condufie Martinazza nel giardino. Qui
icipia a descrivere un giardinu da Diavoli mostrandolo ripieno di tutti quei
Malanni, e disgrazic, che-alla giornata accadono a i mortali.
 LASCLAT Li bicchier vori, e+ piacti feemi. Havendo bevuto,e mangiato quan.

piaciuto.
> GLARDINO. Luogo dove si piantano fiori, ed altre delizie similida i Latini
detto Florarinm 5 fen pomarinm. Vicne questa voce dal Tedesco Garten, e questo
dal Latino bortus, secondo il Ferrari, ii quale biafima il Perionio, che la fa ve.
nire dal Greco ardevein, innafiare, seguitaco in ciò dal Monofini. Ma tanto que-
glinella sua lingua Francese,quanto questi nella nostra Toscana,sono troppo ap-
Paflionati acl far venire le voci dal Greco yilche non & sempre vero, ch' elle»



ee ee

'vengano.

 SERLINA. Gogna. Vedi sopra C. 2. stan 15.,¢ C, 3. stan. 62,
HITERA, B) quel berrettone, o cartoccio di foglio; che dalla Giuftizia si

fa meteere in testa a coloro » che sono fruftati in full asino. Vedi sotto Can, 12.

Bt

KP.

» 4h Pareraio del Nemi. Intendiamo le forche, perch queste son fituate in uns
campo, che era, e forse è ancora della famiglia de' Nemi, e lo diciamo Pareta-
40 per coprire il detto.. Li Pareraio € un boschetto fatto per uccellare a fringuelli,
ed altri uccelletti simili nominato Pareraio dalle retiyche s' adoprano a tal caccia,
Ie small fichiamano parere. Vedi sopra C. 4. stan,27. al termine mandatoin Pic.
tardia,

~ PERGOLATO., Le viti che foftenute in aria da pali, e pertiche, formano co-
Me Una coperta, o tetto si dicono pergole » O pergolati, come dicono anche i La-

Se

 CORRENTE. E10 stesso che travicello » cio® un legno lungo,grosso più d'un
> € 8' adatea a formare, e foftenere i palchi, e vetti delie case,

| 46 a¥LZZA. § intende quella fune 5 con la quale si legano per il capo le be-
-..

| fit ye però € detta cavezza-quasi capo, e il Poeta la chiama così, perché è lega.
, #2 per il-collo, ecapo degi' impiccati a quei correnti, e gli chiama Penzoli, per-
-Sh€ gli figura grappoli d'vua pendenti a questa pergola,;
, BRA: Shek;













282 MALMANTILE- 2
SP-ARGON le rame. Gli alberi che sono in questo.giarditio distendon
rami in diverse mani¢re; ma in-vece d' alberi (ono scheletri.
tomie. Scheletro, o scheretro diciamo tutta l'offatura dtumco
ogoi altro animaie,ripulita dalle carni, e rimeflainfieme con
105; e4notomia chiamiamo il corpo d'un' huomo, ed” altro'ani I
mostra tutti li nerui'y wulcoliy e vene, chefoho foro Jaipelley: soe
SPALLIERE, Quelle piante, ed alberhy che si fannovdutendere fu perte,
ra con i rami, come limoni, e fufini, ec, si dicono spallieré se qui:pig
mie per ogni specie di pomi d' agrumi, dice, che in vece di tali pomi
questi alberi a spalliera gli aborti., i miofiti €s gobli.< ai Se
INSELICIAT E. Seliciavo dal Latino filices diciamo up lafiricd fatto
ma firettamente, intendiamo quei lastrichi fatti-di plete: piccotidinies 5 o- tan
giion fare ne i viali de i giardini a foggia di Mofaico con pietreyperd maggiori di ji
quelle del mo/aico,e minort atial di quelle degit acciortolatiy e sono diary colo- |

ri in maniera che (ene formano figure,¢c. Come col Mofaico., Binovece di gi
fic pictruzze,dice che (on fatte d' ugna, drdendi, e d' aitreoflature minute.)
a MORC HLA, Intendiamo la fondata dell'oli0 dai Latuno-amarca, © gets dat
fr, aan. 4 2
CABBROSTOLITE, Abbronzate. e<bbrofolire propriamente viohdire qu
abbruciamento che si fa agli uccelli pelati, agcio fiabbrucino quei pr 'the
non si son potuti levare con le mani; ma qui vuol diretince dal fuoco ¢on un
leggieri abbronzamento ¥ che diciamo: abbractacchiate -3 egiggdted
MV MMIE. Sono cadaveri d' huomini che-hanno la-carne appiceata Ws
full' ofia seccatavi sopra da balfami, bitumi,ied aromati, come son cOlpl,
che si trovano sotto le Piramndi di Bgicco, 1 quali (ono di persone peincipaliy che
gli Egizzj havevano per costume di riewpucre di baifami, ed aromaci, fate
gli con strette strisce di tela', odi drappo com murabilitfima maeftria, e pt
hi insieme con qualche Idoletto fatto di metaijo dentro a una cafla, che sate
se

VANS



'

uy

Ni

iy

i

:

faccia d' huomo; così gli riponevano sotto quelie piramidi, dove non &
cevano; ma si feccava la carne, e si riduceva tanto queila, che l'offo come

trito; per lo che si ono-conferuati quet corp: fino a1 tempi nottei, ed f

ne trovano, Polid. Verg. de Rer, imuen, lib. 3. 6, 10. riferisce-con te seguentipe |

3» tole il modo di questo fotrerrare i cadaveri degh Egizzj:Agypuj Hatin mor hg

>» tuo homine ferro incurto cerebrum per-nares educebant, jocam iiusmedt

we is expl; deinde lapide A chiopico circa-ilia i 'a

>» bant,atque illac omnem alucum protrahebaat » & ubi repurgaverant, rorfam lig

»» Odoribus contafis:refarciebaat., ide iterum Contucbine. Vbi hae fecitlent,fa- toy

>» liebant nitro adulto feptuaginta dics, nam diutius (alire non ticebav; quib

y»» exactis Cadaver findone inuoluebant gummi iilinentes; Ho deinde.

&

fy

A

w

I







9» Pitiqui ligneam hominis efigiem faciebant, in qua inferebane y

>» lumque ita reponebant; Eeid, ut arbitror,ica facticabant ju eo acto” >

»» ta cadavera diucurnius incorrupta servarent. ae
Altri cadaveri secchi ci vengono pure dagii Egizzj,i quali corpi |

teriori, e-tutto, fecco, e come impictrito; e ono iciza farciature; equell a

corpi d' huomiai, che dal vento sono staui fotterrati vivinelia rena 5 € quivi com











tidal' tar della rena, Di



chi, ma p:

SESTO CANTARE:
fertiatifi forse per causa de' venti meridionali, e però il nostro Poeta dice: Fenn-
queste Mummie si servono i Medici per diversi farma-
a particolarmente per la Triaca, La'voce Mummiaé Araba; e il Voffio

tira da Atam, che'in Arabesco vuol dire, cera (de vitijs Sermonis lib, 2. cap.
la cera e '! miele faculta conferuatrice; e della cera si servivano gli

per mantenere i cadaveri secondo Brodoto, Jib, 1. Ma la pece mescolata
bitume, era forse quella materia, per quel che apparisce,con la quale
gli Egizzj condivano tali corpi, la quale in Latino greco dicono Pi



289

magi s
?

DI, Intende quelle bafi, sopr' alle quali son posate le statue.

R51, Intende torsi d' huomini, che pittorescamente parlando vuol dire il

fenza'tefla', e'braccia, e cosce Latino truncus; e questi dice, che sono
ilareiti; cioè raccomodati, rappezzati, riftaurati con havervi mefie in veces

- delle lor tefte già consumate dal tempo, altre tefte nuove, e fresche di banditi;

uol e meta » che alle volte si veggono al Palazzo della Giuftizia, eo.
sopr'alle forche elposte alla vista del popolo, essendo fate tagliate di poco tem.

po ai maifar
/ STANZA Lil

Inter '8 quadri di cipolle
Snip i fior Youyatee nariche;
Soma teiccioni, i signoli,e te bolle,

Le posleme, ta rigna, e le volatiche.

Vil mal Pricefe entrante alle midolle,
CW feminaro dalle male pratithe,

'Teticberi, le rabbie 5 e gli altri mali,
- | Che vi mandano gli Offi,e è Vettnrali,



malfartori bandit, « per frelshe

STANZA LIV.

Pescheinsu gli occhi fonui arzurre,egialle,

I marchi, che fiorir debbon le spalle

Ai tagliaborfe, e ladri ancor scolari;

Le piaghe a maffe, 4 pererecci a bulle,

wend ventofe,¢ gonghe in più filars, ©
 e il fior di rofolia, € più rofoni

D' ortefica, vainolo,e pedignoni.

Cu re for per chi gli porta pari,

o ita a descrivere i) giardino dell' Inferno', ed in queste duc ottave narras
i - quelche contengono gli spartimenti. è
<QVADRI di cipolfe. Intende quelli spartimenti, che si fanno in terra ne i giar-
Me\gquali si pongono le cipolle de' fiori. Latino areole, puluini,
» PRA foglie, e-natiche. Dice così per mostrare, che questi mali vengono nella
carne ef mente, e pigliando natiche per tutta la pelle dell' huomo, dice che
fra quelle foplie nascono questi mali in fu le natiche, intendendo la pelle, e per-
ché anche la maggior parte de' medesimi mali per lo più viene in fu le natiche,
'come laogo pib carnofo.;
CHE vi mandano gli Offi 5 e i Vetturali, Questa sorta di gente ha per costumes
itnprecar sempre male, come venga la rabbia, il canchero, la pefte, © firnili,
» PESCHE in fu gli occhi, Qucei-lividi, che vengono attorno agli occhi, quando
sono flati percofti da pugna, o da altri, e sono di colore azzurriccio, € intorno
: 7 $ Dar le pesche: i Latini dicono /uggillare alquem, vedi sopra C, 3,
'1. y che noi purt diciaino anche figilli tali lividi,¢ diciamo anche: figillare un'

uno, *
 GLI sfregi fior per chi gli porta pari, Gli sfregi son fiori, che anno bene in sul
v ) di coloro che portan as tani, cioè fanno bene il raffiano, che portar i pollé
woo) dir fareil rufhano dalla voce pouler Francese che vuol dir 5 vigtierro amorofo,
diciamo porta poulers. Qo CUAR.

SSS HSS ETE SaaS ers








290 MALMANTILE

MARCH, Tntende quei segni, che dalla giuftizia si fanno
droncelli, gai per esser giovanettinon sono capaci della pena
Higmata, Vedi opra C. 2. st, 3. alla voce sberlefe. 6

PLAGHE a maffe, peterecci a baile, Piaghe, e peterecci ins rand
ma, Nell' uso diciamo anche Patereccio; e Panareccio dal Greco, usato ai
da' Latini Paronychia, postema che si forma alla radice dell' ugna, che i
chiamano Redivias., o Reduvias yp eas

GONGHE. Intendiamo gavine infermita che viene nel collo, e quei t t
che son talvolta /pine ventofe, perché diciamo haver le gonghe os ee e

venga apparentemente nella pelle della gola sotto le ganalce, Latino tonfilla,glan
dule faucium, t VE
Ma perché non paia che io voglia fare un trattato di chirurgia, i
splicazione di questi mali; tanto più che io stimo, che faranno intesi per
Italia, nella quale son chiamati nell' iftefla, pore differente mamiera,
intelligenza dell' opera serve sapere, che in questo giardino sono tutte
ta, che vengono agli huomini esteriormente, le quali il Poeta vuol mostrares,
che si generano nell' Inferno, come sentina di tutti i mali. 10.04
} LV. STANZALVL,.

Alla ragnaia al fin si son condatti
Di fils da toccar la margheritey »
Ove de' tordi cala, ede' merlottt
Alla ritrofa quantità infinitay




ai Lae








aes

Si maraviglia, si fupisce, epanta
Martinazza in veder si vaght fiori,
Erimirando hor questa,hor quella pianta
Non fol pasce la vista in quei colori,

Ata confortar si sente tutta quanta
Alla fragranza di sh grats odori y
E ai non corne non pxo far di meno

Che son poi da Biagin pelati,¢ corti
Sgozzando de' più frolli ana partita y
Altrane/quartaye quellach'e puafrefea

Be ssh ee eh

Vu bel maxxetto, che le adorni il feno,

Nello Stidione infilza alla Turchesca,
NZA LVI.

= BEG ES.

Veduto il tutto, Nepo la conduce

Chi per la pizzicata, che produce
A! bagno, on ogni schinvoy e galeotto )

| dl uazo,fa tragedie sn sul.capportas
Opra qualeofa: Vn fa le calze,un cuce, Vn mangia,un fofia nella verrinala y
etltri vende acquavite,altré il biscotto, Vin trema in sentir dir:fuor camiciuels,
Martinazza resta maravigliata, e si fupisce, e rimirando tutte quelle piante,
paice la villa, e soddisfa all' odorato con quella fuave fragranza, ne può non fa-
re un mazzo di quei fiori galanti per adornarfene il seno. Visto il giardino, Ne
po la conduce alla ragnaia, dipoi al Bagno, dove stanno i galeorti » descritto ¢-
me è appunto quello di Livorno cirea operazioni, che fanno i galeotti, |
SP ANT ARS!, Dallo Spagnuolo e/pantar/e. Vuol dire efttemamente mart
vigliarsi, e si dice in augumento maravicliarsi 5 frabilirft, (pantarsi, che & il verbo ui
§paventarsi fincopato. Habbiamo l'addicttivo/panro, che significa eftremament —
marayigliolo. Ma forsc e da Spandere, quali voglia dire largo a ie
de, ampio, e in confeguenza maravigliofo. E di Spamse addicttivo, del, Ay
Spandere cen' ¢1' esempio in Meffer Cino, Quando ha per gli occhi (un poem &;
fad 7 wap Sah
4 VN bel marzetto, che le adorni il feno, Bello ornamento del send d' una few.
* pa havervi crofte, rogna, e simili galanictic, delle quali potcya esser fy
~

UGA



Serer,







SESTO CANTARE)

Poeta scherza per esprimere la laidezza di Martinazza.
BE' una felua, o macchia folta posta per lo più lungo i rivi, per
si pe eee sospefa a due Mili, e questa rete si chiama ragna
'a imitazione di quei veli, che fanno i ragai per pigliare le mosche,
ragne. Pietro Angelo da Barga nel suo Poema della caccia de-
celli: Hos caffes, has ipfa plagas y bac retia quondam Ante alias omnes telamts
ere dotta Innenit dixitque (uo de nomine Arachne, E da questa rete ragna si
nna ea » Ove Gi tende per pigliar tordi, beccafichi, ec.
'teccar la margherita, Cio quelle Aanghe,sopr' alle quali si da il mar-
lla Corda, che questo vuol dir toccar /a margherita.
DI, merlotti, Vuol dir merli giovani, ma dicendosi merlotto, o Tordo
stintende Huomo semplice, corrivo, che cala; che si lascia pigliare.

3




st. 59.
g Gabola fatta a foggia d' una trappola da topi, con la quale per
certo Ordigno si pigliano vivi gli uccelli, detta così per esser la parte, da
eferrare rivolta in dictro. Vedi sopra in questo C. st, 1. alla voce con-
Qui per ritrofa intende Carcere.
; » Maeftro Biagio, o biagino vuol dire il Boia, che così havea no-
me, quando I Autore compose le presenti Qrtave; ed a questo fucceffle Maeftro
Baltianodetto sopra C. s. tt. 44.;
0 + Poco gli manca a essere stantio; s' intende animale morto di più
giorni, Vedi sopra C.3. stan. 24. la voce stantio,
INELLARE alla Turchesca, Cioeimpalare,
BAGNO. Così chiamiamo quel ferraglio, entro al quale si tengono gli schia:
vi, €coloro', che per delirti son condennati alla galera, detti pero Galeotti, i
quali dimorando quivi,fanno i meftieri enunciati dal Poeta, che si serve della vo-
 e bagne per l'equivoco,il quale fa credere, che in questo giardino sia ancora il
g0 da bagnarsi per mostrarlo ripieno d'ogni delizia;come il Paretaio, e la ra-
-  E.questo ferraglio di galcotti credo, che si dica bago, perché in esso quei
iquenti purgano i loro misfacti, come con l'acqua del bagno si purgano le»
'delle membra. Gagno si disse ancora un luogo simile, Li Pulci nel Mor-
pante: Dife Aforganre allora: ia son nel gagno De' diavoli,
» PIZZICAT A, Specie di confezione minutissima, ma per la similitudine della
figura di essa confezione, e per il senso del verbo pizzicareintendiamo ( comes
gui §' intende ) pidocchi.; pe
FA ie in sul cappotto, Ammazza pidocchi in sul cappotto, che e quella,
fo tche portano gli (chiavi, o galcotti, remiganu, ed ogni altro mari-
; detto, siccome Cappa, 4 capiendo, perché piglia, e cuopre tutta la vita,
- SOSPIAR nella verriola. Cio bere, perché bevendo si fofha, o: respira col na-
s0 nelia vetriola  cioè nel vetro. Detto che ha del parlar furbesco.. Vetrivea er-
s herba parietaria detta daalcuni. 11 Monofini lib. 9. Indicare.yo-
Aen muito vino se ingurgitafle, dicimus. Egis ha toccato ben la verrigla,
Vesrivola est herba insestoribus notissima, de qua Petrus Cre/centius lib, 6. ¢, ult, pocula
itrea vulgo fiune, q
. do f Auzzino vuol bastonare un galeotto per qualche
Oo 2z suo











od
see ce






















292

suo mancamento suol dire fuor camicinola, intendendo, che

ha da esser bastonato; e però dice: Chi trema in (emir dir

trema per il timore delle bastonate. i
CAMICIVOLA. Bun piccolo farletto di panno lino, b2

che secondo la Ragione si sotto gli altri abitisopra alla Camicia

dersi dal freddo, come 10 detto sopra alla voce Farfetto: gli

chiamano gix/ecca, vod awieirseles

STANZA LVIIL TAN:

Vanno più innanzi a'gridi,ed a'romori
Che fanno i rei legati alla catena
Ove a ciascun secondo i snoi errors
Datoe il gafligo, e la dovura pena,
esi primi che for due Proceuratori
Cavar si vede tl sangue d' ogni vena,
E questo lor avvien, perché ambidui
Furon mignatte delle borfe altrui, Con

STANZA LX, Oy

Quei, dice Nepo,t il Re degli xfurai, 1 gran se gli marcy dentro a'

Che pel guadagno scortico il pidocchio, Che nol vendea se non vaiea

Vn fernizio ad alcun non fece mai, Così fece det ged hor
Se non col pegno, e dandoli lo (crocchio GP intarla il dosso,e da'fiohfe fh
Paflano avanti a vedere i delinquenti legati alla catena, e gait er lo

falli. I primi sono due Causidici, ed il secondo è un' Viuraio y ti

secondo il merito. » Seah

PROCCVRATORS, Agitatori di liti. Cautidici tanto Civilijche criminali

MIGNATTE. Sanguifughe. Quei vermi acquaticijde i quali si servono 1 Ce»

rufici per cavar sangue; e perché si dice, che i danari sono il secondo fangut

però esser mignatta delle borfe alerni vuol dir Succhiare, roe ¢avar il de i

altrui borfe, come fa la mignatta fucchiando, e cavando il sangue dalle vent»,

diciamo mignarta, o mignella a uno, che € firetto del suo, e volenti sig:
quello d' altri: A questi tali può quadyare ciò, che disse Orazio. Lon milfura ch
tem nif plena cruoris birudo, a:

V-AGLIARSI, Intendi dimenarsi come fa uno, che habbia rogna, © altro pe

la vita, che si dimena, e scontorce per grattarsi il prudore; o pizzicore conl'a

bito, che ha in dosso, e fa con la vita un moto simile a.quello, che fa und, che

vagli il grano. sae
TONCHI, Forse dal Latino sondere pre(o per mictere Ȣ divorare, Sond vet

mi piccoli, o infetti, che si generano nelle fave, pilelli, ed in altri i

 votano i granelli rodendoli; da i Latini detti Curcudiones. Virg, 1 z

pulatque ingentem farris acervum Curculio, es visage

TIGNVOLE. Bachi simili ma si generano ne i pani 5 e fogi impaftari; dai
latini detti Tee, Di queste ne nascono ancora dal grano, e si chiamano prt

noli. peas ah
™ MOSCIONT. Quei moscherini 5 che na(cono dal vino, che dicemmo sopra in
questo C, stan. 37. oe

Son























Se Fern Ee eae SE Sr FEKETE

















SESTO CANTARE: 293
So eae eet si generano nel legno 5 e lo rodono; da i Jatiai

er) eet.
| RARE « Intende quei farfallini, che si generano nel grano. Pyrau/fa,cou
4 areca (ono app: = farfalle pil grandi, le quali papain oe al
lume, e vis'abbruciano. Di queste disse il Petrarca. Semplicetta farfalla al lume



 “COCCIOLE. Piccoli tumoretti, o enfiature cagionate da' morsi danimal
come zanzare, bruchi,¢ simili.:

'S8RANF, Rotture; Scorticature. Vedi sopra in questo C. stan. 47.
PER riftoro. Per ricompenfa. Dan, Par. C. 5.,

Gabino Dunque che render puoffi per riftore?

Equife ben pare, che il nostro Poeta voglia dire »per riftoramento, o alleggeri-

ia de i-teavagli,¢ pene, nondimeno è tutto il contrario, perché @ parlare -

.
P| ' e vuol dire; oltre a gli altri travagli ha di pil, che lo flageliano,¢ pelta-
foe dese pieno di feudi d' oro'. Questa voce. rifore vien dal verbo ri~



ie derivante dal verbo reffawrare, ed ha quasi, lo: stesso. significato,
non che questo vuol dire Acconciare, o raffettar case, ed altri materiali; ¢

i dir Ricompentare, o rifar danni,
ia Lo,4 Sy Nae ppi aunacordicella; i dendosi per
Py mbello quel facchetto-pieno di segatura,0 di cenci, che adoprano i ragazzi
'Perquotere i contadini,come dicemmo sopra C, 1. tt. 59. Zimbello detto, cred'
10, quasi cennelio, civé piccol segno, argumentandolo dallo Spagnuolo, che il
chia five '

© Ub Re degli nfurai. Wmaggiore usuraio del mondo. Detto che viene da i Gre-
| Gi yiquali chiamavano Re,quello che avanzava,superava, e vinceva gli altri ne i

en: 9

i) 29h giuochi fanciulieschi; ed Asino quel che perdevay come habbiamo detto altro-
Vebs iy):

Yi, SCORT-1CO? il pidocchio. Significa esser avido del denaro, e far' ogni maggior

a per guadagnare; si dice scorticar if piducchio, per vender la pelle, € con

¢ Planto 6 pod dire, Vel unguinm prafegmana colligere.: 6

, DAR be ferecchio, Preftar danari a usura »ed in vece di dar. denari effettivi,dar
aoe vaglia dieci oe venti. Vedi sopra C, 3. st, 74. ¢d è la più efecrandas

'2, che si trovi, e forse la più praticata.
; MARCIRE, Intendiamo infradiciare, corrompersi, Dal Latino marceres;
|

SE non valeva un' occhio, Se non si vendeva caro,¢ a prezzo rigorofidimo: Non
vit cosa pil cara dell occhio. Onde Catullo. Ni te pins oculis meis amarem
INT ARLARE, Esser mangiato da i tarli, o tignuole, che i Latini dicevano:
Cariem sentire,

E PESTO dai fui soldi, Inscanto dalle percofle di oa facchetto pieno delle
ae monete. Vuol mostrar in somma il nostro Poeta, che
= Per qua quss peceat, per bec torquetur.

STAN.











294 MALMANTILE?@

STANZA LXL oe STANZAS
Va! altro ad un balcon balla, e coruetta y Dice la maga questoe
Ch un diavol con tasferzaacentocorde uand ella i
Chrun grad'occhio dibue ctascihainverta, | Cofkui ha fates quale
Prima gli da certe picchiate forde y Par non fo nulla,e no
Con una spinta a baffo poi to getta Domandaa
dn cert' acque bitumofe 5 elorde, ° Tal penaa chifi debbagda
Chee' n' esce poi sch' ione disgradogli orci, Ed et che per servirla è 9)
O peggiv d'un Norcin mula wines ' Prontamente così le da risposa,
STAN ZA LKIL + SD
Quei fu Zerbino, ed! amorofe dardo Ma dell' occhiate sue ben più '
Mostrando,il cuor ferito,€ manomeffe, Hor sentene il riverbero', ¢:
Credeva il mio fantocciocon un sguarde E com'.ci già pensd far alle das
Di (briciolar tuttoil femmineo feffo; Dalla finestra è tratto in

Quel che segue e uno che peced d' ambizione di bello'y¢ lindo 5 e credeva'
la sua bellezza di far' innamorar tutte le dame, ed hora'riceve la a
suo peccato, p 94> being
CORVETT A, Salta. Cornettare & un certo faltellar de*eavalli 5dal Laci eure
uari, Spagnuolo corwar; piegare, innarcare, torcere» EB questo'
appropriato in questo juogo per esprimer i) moto, che faceva costui, il,
evitare le sferzate,era necessario che falcella(se a tempoy ed in quella'
to, che fa il cavallo, quando coruetta.: >> SRDS
VN grand? cchio di bue ciascuna ha in vette, Pone in vetta, cioè nella cima di
queste corde, tocchio del'bue, e non d' altro animale; perché bovis ocnle oculorum
pulchritudo, & nitor fiemscatur, e trovatene l'esempio in Omero, dal quale
Giunone è chiamata boopis, cioè bovinos oculos habens o vero Dea dagli occhi grandi,
¢ perciò maeflofa. E costui doveva esser gaftigato con la bellezza degli occhi;
perché con la pretesa bellezza de' suoi occhi, haveva egli 10 AO
PICCHIATE forde. Picchiate, e percofle gagliarde, Percofle » che facciano
molto male, e non paia che lo facciano; servendoci in questo caso la voce fords
per la voce occw/to,.come si dice ricco fordo, per ricco non palefe, o non' cond:
sciuto. Ie
LE disgrado, Quel che vaglia questo termine vedi sopra C. 3. stan, 37. al ter
mine ho froppato. AMY
ORC/0. Che cosa sieno orcj. Vedi sopra'C. 1. st. 7. Qui intende orci da olia,
che son sempre schifi. a
NORCINO mula de' porci, Coloro che in Firenze ammazzano i porci, € così
morti gli portano sopr' alle spalle alle botteghe de' Macellari,(ono per lo più del
paefe di Norcia, e pero gli chiama mule Norcine, cioè portacors da Wercia e O-
storo son sempre tutti unti di gratio di porco, lorditimi, e (chifi di sangue.
QVEST Ac ariofa. Questa € cosa grande, ardua, e-che arreca:stupore; otra
ordinaria, e stravagante,¢ che non si può credere, me
NON ono far gindizio. Cioè giudizio temerario,e falfo; Maniera da Ipocriti,
¢ faifi bacchettoni scrupolofi, chelp
ZERSINI, Così chiamiamo quei giovani, che persuadendosi d' esser belli, faa
Z no



=

Ce et ee ee ee ij

aL
= aie oi

Sef asl 2 &







- STO CANTARE 295

 vanno lindi credendosi di far innamorare ognuno con la lor

quel dohenae PAriofto nel Furiofo deferive per il più belio, ¢

¢ di quel tempo. E si dice anche Mirtillo;nome cavato'dal Gua-
fins Vedi sotto C, 10, stan, 30,

10 il. cuor ferito, e manomeffo aeasmerslo gacdds Facendo da inna-





2. Nibbiaceo 5 Vecellaccio, ec. tutti servono per intendere un
cimunito.

aR re in minutifiimi pezzi, o-ridurre in. bricioli,ed in-

morir di 'patimo, e disfarsi per amor di lui-tutte le dame.
0. +) Sinonimi che figaificano:li riperquotimenti, che fan-
i del Sole, o il fuoco nella parte oppostaa quella,dove direttamente
i Chimuci dicono; Fuoco di riverbero., o di riflefo.. Qui inten-
'coftui con quelle fruftate piene d' occhi, ha il gaftigo dell' occhiate amo-

egli nel mondo dava alle donne.
Reta is ake dame. Cioè si come egli pensd che le dame cascaffe-
la sua bellezza, ( il che appresso di noi vuol dir farle morice
re), così egli e buttato da qusi balconi entro al litame, per maggior
0 efti. tali sono schizzinosi ne poslono. vedersi addosso un,
che guafti la loro attillatura,¢lindura,,

] STANZA LXV.

; ANZA LXIV.
an ch' e legato, e che gli e posto Qui Nepo scuopre la di lui magagna,
berrettin baffo a tagliere, Mostrando ch' ei fu nobile, e ben nato,
-colpe colpo da discofto E sempr'hebbe il Pedante alle caleagna;
fra. gliene facadere. Cc ontuttocio voll' esser mal creato;
Mifero sia quivi immoto, e tofto Perché se e fulfe statoil Re di Spagna,
do gli occhi ai colpi dell'arciere, Mi cappello a nellun mai s' è cavato;
muave Punto, ochinaorizza, Pero s ei fu villano, hora il maeftro
€.42 cultello che & infizra, GU! insegna le creanze col baleftro.
STANZA LXVI.
4 par comune usanza, Se ¢' faltan la granata,addio Creanza,
4.risponde al Galatrons; ae ch? e° fien nati nella Falterona,



































noi Fanciusi un pocon offer uanz A, Ata per la loro afinita Superba,
eyes: il. ahi bastona. Son poi fuggiti più che la mal erba;



“« Dattro. che segue è uno, che nel Mondo non volle mai imparare.i buoni costu-
Bi. non si yolle mai cavar il cappello di testa per riveric nefluno,per grande
j 'ch ¢gli fulle, onde gli avviene il gafligo, che si dice nelle presenti ottave; E
Masoasaa dice a Nepo, che hoggi di questa sorta mal creati e pieno il Mondo.

pies (TINO a tagiiere. Berretta bafla e piatta,nella quale non si vede la

del capo, come sono /e coppole Napoletave,
eat Qgni volta ch' ci tira. Vedi sopra C. 2. stan. 57.
« Sta duro; Sta faldo; Sta fermo; Non si muove.

(RCIERE. Colui che tira con la balestra « -4rciere in molti luoghi del nostro
do s' intende il Caprone, o Becco. Lat, aries.
(AG AGN.A. Mancamento, difetto. E parlandosi d' huomini s' inane tan-
'animo, che di corpo, Dante Jaf. C. 33. dice. OGe-

















age 'MALMANTILE”
O Genovefi huomini diversi 08 OND
“D egni costume, e pien d! ogni magagnd
Lalli En. Trau, ston erga i j up reeves J
¥ i bE pias vas

Ogni trattate conte ogni magagna |

Magagna in Lat, barb, & detta Mahaminm, Swan) Franz. Aabain,

¢ vuol dire propriamente mutilazione di membra,¢ si stende a significare « i

no,¢ detrimento. Vedi Du Frefne nel Gloffario alla Parola Aabamiam,
BEN nato, Nato di nobili, ed honefti parenti,




HEBBE sempre it: Pedante alle calcagna, Hebe sempre il Maahre a te 4

gl' insegnava i buoni costumi, e termini, cet Sar
MAL creat, Senza creanza, Vao-che non fai buoni termini o costumi,
VILLANO. Contadino', Stintende uno scortele, e mal creato. Planco ra
merum, intende un' huomo ruftico, senza civilta, fenga galanteria, un pre
villano, Catullo, Peeniruris, & inficeviarum, [1 contrario di vidlane®,,
SE faltan la granata, Se essi cscono di (orto la cura del padre;'¢ del macho,
Si dice faltar la granata; quand? uno e(ce de' pupilli che ini diero's ema
re ex Spbebis, Dicono che quando uno e arruolato per birro,debba far
mee a fare il noviziato,¢ fnito questo tempo gli faceian fare una cirion
faltare topr'a una granata, che gli mettono d'avanti in terra,e che fatta questa azione
refti libero dal noviziato, ed.in un certo modo esca de' pupilli; e da que r
monia (che se non è vera, e assai vulgata ) credo 10, che habbia origine il pre
sente detto, F 'hae?
PAIONO nati nella Falterona, Paiono nati in luoghi incolti,e difabigati,come
sono le montagne della Falterona in Casentino, dove poche creanze im-
pararsi, non essendo in quei luoghi con chi praticare, se non con pecore,€ por
ci, Ci ferniamo però di questo termine per esprimere un' huomo incivile, e foz"
zo, eche tratti da villano; come e quercubus, aut faxis natus, ae
SON fuggiti più che a malerba. Nefluno gli vuol praticare. Sono sfuggiti 42
tutti. Malerba intendiamo l'ortica erba nora, la quale & sfuggita da tuttl » pet

ché pugne.
STANZA LXVIIL STANZA LXVIIL

Ma chi è quel, o hai denti di cignale Ora per queste sue finzioni eterne,
E lingua così lunga 5 e moffruofa ? Chi egli bebbe sempre nella mercathrs,
Si vede, che son fuor del naturale Lucciole dands a creder per lanternt y
A me paion radici, o simil cosa. Sharbata gli han la lingnaye denratirss
Wepo rispose; Quello e un Senfale Main bocca havids pos di gran cavertty
Che si i act il Parola,ma (a glofa Perché non datur vacuum jn naturhy
Huom di fandonie, dice, e di bugie, Glibanno a mifferio in quelle fhanze we
Perché in esse fondo le fenferie. Composto denti, e lingua di carate «

Segue un Sen(ale, il quale e gaftigato delle bugie, che anna cavato
la lingua, e identi, ed in quella vece meflovi delle carore. Ji Poeta si serve dell!
affioma Peripatetico! Won datur vacuum in natura, col quale ingende che fulle ae
cefiario riempier quei voti,cagionati dal' eftrazione della lingua, € deati » Ae
(cherza, sapendo bene auch' egli, che quei medesimi voti erano già ripienl #
aria. >) ” ae;





S82 rennrer® BRfs2-nPhee

~

See






'fon mediatori afar vender una mercanzia:
vin Birenzeun'Senfale di bestic, huomo sccl-
oy che per le sue farberie:fu impiccato a: forche erette a possa per
0 ee eealns 3 ed è lo stesso, che quegli che fu,
chino detto | 3. fhe 55.
OIE ye bagi s aiclonaoe dal vero, e sono si può dir finonimi, se
Ml dir chiacchi vana, e bagia propri vuol dire attefta-

fenferia. S' intende,quando uno di questi Senfali fa vender qualcofa,e




per lanterne, Dar a creder una cosa per un'altra, 1 Lalli














| Lucciole qui rimiro per lanterne,
£4, Bi quel vermicello alato, che di notte riluce da i latini detto Ci-
Nottiluca; da'Tedeschi animaletto di'S. Giovanni, e da' Greci Lampyris dal
e fai egiare nelle tenebre, come egli fa;\¢ /anterna & quello arne-
; 'quale-si porta il lume la-notte ferrato da talco, offo., o vetro per di-

jo dal vento; ed e voce.pura latina,

4°) Specie di radica, Latino /iser, Mail proverbio Pisntar, o fecar
ca dare a creder bugie. Latino imponere alicui, Onde Impostura, es
febene si dice in pil grave significato. Vedi sopra C. 2, st. 70. Dices
yperché vi son messe tali carote,è non solamente per riempiere i
perdar il gaftigo a'coftui delle tante carote, che esso haveva piantate,
era in'vita', facendogli haver sempre dentro alla bocca effettive, e natu-

a. 'ANZA LXIX, STANZA LXX,
See volta ha la facia, Vedi colui',¢ al colle ha un' orinale;
“Bute diavol legnainolo in sul groppone Cieco, rattratto lacero', e piagato ?
— Gli ascia itlegname,fega,ed ipialliaccia, Ei fu Governator a uno spedale,
(Ste servir per [uo pancone; Ow ei non volle mai pur un malato,
a fu; c'alla pancaccia Ora per pena ogni dolore', e male,
aglian le legne addofsv alle persone, Che gl infermi y' haurebbono portato
tener sa lingua in brigiia ( Mentr' alia barba lor pappo st bene )
ender. lapariglia, Sopr' al uo corpo tutto Guanto trene,
'il gaftigo dato-a* Mormoratori, ed a quelli, che, essendo Aati Sopranten-
| Spedali,non hanno havuto carita; ma solo hanno attefo a crapulare per
Manodion 3 che dovevan fomminiftrare a' poveri, ed infermi.
edi








VE; Codrione. Le parti di dietro del'huomo fra le reni, € le nati-
fowo C, 10; st. 50. LU Persiani disse,
“\ Ceascun teme, e si caca nelle brache
Tn vederus appiccato sul groppane
t ® “Lo frocco da scannar le pastinache, -
si cava che e usato, ma per lo più in scherzo,- Viene secondo i} Ferrari dal






Orrhopyginm, che significa lo stesso. Hs
ARE, Tagliar con l'asce, che è uno strumento da legnaiuoli noto,chia-
Pp mandolo ©

. =.
a ooo








Tar ae

298 MALMANTILE ©

mandolo così anche i Latini, che lo dicono ¢4/cia. Ifidoro neilé Origint lib, 19)
6.19. Ascia ab affulis dita quas-a ligno eximit yenius diminutionm nomen est asciole
( forse accetta ) Est autem manubrio brevijex aduerfa parte referens vel, i
Jexm 5 vel canatum, vel bicorne rafirum, Vitruvio difie Asciare Lib. VI. ¢. 2. Suma.
tur Ascia, © quemadmodum materia ( Qui invende il ego; che gli Spagouoli dal
Latino chiamano, madera ) dolatur, fic calx lacn macerata ascierur, Am,
[MPlALLaccLa, Qui la rima forse ha necefitato  Autore a servith di
questo verbo impiatiacciare in vece del verbo piallare, che vuol nee
-gnami con:la pialla come intende qui 5 ed il verbo impiallacchare vuol dire tito
prire un legname con piallacci ( fefftles lamina, famine pratenues \e disse Plinio}
fond fottilifene afircelle di noce, con le quali si cuopre altro legname più vilei
far cafle, tavole, ed altro, nella forma che si fa con ' ebano, granatiglia, ed
tri legnami nobili. Plinio discorrendo di legnami, de' quali gli antichi si servi-
vano per impiallacciare lib. 17. 43. Que i laminas fecantur, quorumque Z
weftiatur alia materies, pracipua funt cedrus, terebinthus, etc, E poco 2
prima origo lnxuria 5 arborem alia imtegi, © viliores ligno pretiofiures cortice fier; B
PO, Lvcugitate funt, & ligni brattea, nec fatis, Capere tingi animalinm cornua.
dentes fecari, liguumgue ebore dsspingui, mox operiri =
LALLA, Chiamano i Legnaiuoli quello strumento di legno', che ha un ferro
incaflato, col quale aflottigliano, appt » pulil 9 ed addiri: ile
gnami, da i Latini,secondo molti,detto Dolabra, ma forse con qualche
'Vn' antico Grammatico pat che la confonda coll' alcia.. Dolare fabri' Co
ascia ledere, Si legge in Colum, lib. 3. Qua falce amputari non possunt, dette doles
bra abradito,il che pare che voglia dire più tosto accetta, o pennsto, ovanga;
che pialla: E corrobora questa opinione il medesimo Colum. lib. 4. ¢-24, serven
dofene in diminutivo; Semper circa crus dolabella dimovenila 'ef revra, clot Inter.
no al cansbo della vite e da levare (a terra con una dccettina, 1) Calepino tiene sche
da pialia si dica runcina, e porta ' autorita di Plinio lib. 16. cap. 42. edd éncitares
runcinarum raptus, ove pare, che descriva appunto l'operazione della yf
per infino l arricciolinamento de' trucioli: Tutto il tefto dice così: Br ad quack
que libeat intoftina opera aptissima ( parla de}l' abeto ) five Graco, five Campanty,
ficulo fabricae artis genere spettabilis, ramentorum crinibus pampinate ae
be se voluens ad incitates runcinarum raptus, Ma io ardisco conteaddirgli ots
Y autorita d' Hermolao che dice: Runcine [unt maiores ferra, quibus fabri materis-
rij fecant arborum moles fubiettis canterijs, Si che non la pialla, ma fa fega grande,
'che adoperano i Marangoni per ricidere i legnami, adattandoli sopra quel C4
Valletti, che noi chiamiamo canreo ( dal Latino cantherins, cio' cabalus: -e pil
volgarmente pietiche, i quali sono one di due correnti inchiavardati ial
a guila di cefoie (che propriamente si dicono pietiche ) e d'un' aleco pezzo'di cor
rente, che si merte a traverso alle pictiche (¢ queito si dice Canteo ) € a
così un triangolo vi adattano per via di piuoli il legao da fegarsi, Runcare è tet
'mine d' agricoltura, 'che vuol dir propriamente tor via, onde se ne formd per
yeotura la parola antica Latina averruncare, cio' avertere; e se ne Iddio
wAverruncus detto Così, perché ab'eo precari folent, ut pericula avertat; si come dice
Watrronc., E in proposico d' agricultura (ene fabbricarono le parole aie.





fF EP Re ewe

a
wt






SESTO CANTARE: 299

Ronconé yle quali significano strumenti da nettare.i campijda rimondare fructi, ¢
i cinne raat Plinio lib. 18, ¢./21. Siligioem gfe striven. sfemen, shepeleas

accato  farrita 11 iB « Runcatio, cum feres in articulo off, evulfis inu-
. joo eoremng apes radicem indicat, Segetemque discernit a cespite. EB Catone cap.
even cremarique; ie che più tosto Runcina parrebbe, che avel-
 fead essere la roncola, o cosa simile, che la fega, o la pialla. Ma forse non.
tanto. il Calepino, quanto anche il Vocabolario della Crusca dal levar via, ¢
a faellere,¢ ripulire ( che questo significa, comes' e vitto il verbo Runcare ) hanno
dato il nome di rancina alla pialla),perché clla pulisce, appiana, e leva il fover-
da! Jegnami.. Tuttavia anche per questa ragione la dires do/abra, perché si-
questa ancora pulilce, e rade, come dice Colum, nel juogo sopra cita-
sia. come esser si voglia,poco fa sal rth nostram, bastandoci intendere, che

[ene quello strumento da legnaiuoli; che habbiamo accenaato,
 PANCONE. Chiamano i degnainoli quella loro panca grotia, sopra la quale
ilegnami per lavorargli, deta pancone, perché ¢€ fatta d' un paycone
aoe un' afie grossa-circa un. quarto di braccio » che sono affe da rifen-



ALLA pancaccia Così si chiama quel luogo dove in Firenze si tiene il croc.
discorre de' fatti d'altri, e delle nuove. Vedi sopra C. 2. st. 73, E per-
ildir male del prossimo si dice Tagliar le legne addosso.a uno. Latino famam.,
icnius lacerare prosciadere, pero.a costoro vien dato il gaftigo adeguato, cons
Halab loro addosso il legname essertivamente.
TENER (a lingua in brigiia, Parlar consideratamente, e con riguardo, e si di-
ce anche: Tener la lingua a freuo.
ghtNDeR la pariglia. Render il contraccambio. Parigiia vuol dire una cosa,
può dividersi in due parti yguali; come nel numero due si può far' uno,¢ uno.
£ 'to render pariglia vuol dir render ugual contraccambio.. Vedi sopra C, 4. st,
ma pwr pari referre de' Lat. Dan. nel Parad. C. 26. dice:
x Perch' io lo veggio nef verace /pegtio,
yh « ~ Che fa di se pareglie L altre cofes
E nulla fece tui di se pareglio.
Hoggi però in questo senso,¢ maniera, che si serve Dante di questa voce pare-
44 non mi pare, che si usi, se non da' Franzefi 3 che dicono pareil,

ALLA barba loro. A spele loro, Questo termine esprime Pigliare, o consuma-
re una cosa d' altri contro al gusto » € volontà del padrone di efla; o a dispetto,e
208 del medesimo.

AP PO. Cioè mangid, Donde Pappolone uno che mangia assai che vedem

FER SSLA SES SCES CLESSSE RES



Hi

laedinee 1 6.

4 eee NZ A LEXL

5 'Chia, coltui, ¢' habbiamo a. dirimpetto Che non ne pag mai un maladetta 5

è (Dice la donna) a cui guegli animali Tenne gran posto, se spele bestiali;
 Sharban con le tanaglieilcuor del petto? Ma pai per soddssfare ei non hauria

fw risponde:.Questo e un di ques tali, Voluto men trovargli per la via,

a a

; Pp z STAN-







|
|


300 MALMANTILE ©
STANZA LKXILe o ooo) oo SDANZA\LERMD
Colni, e hail vifo peffo, eit otto... Riferva il. muragche c' ni
'Da quei due [pire in femimils spoglie ».. Donne, che se a eH
Hus vile fu, ma bifeaiuoloxe ghiogtoy' >  D* arir giviellare, e luce

Che si volle cavar tutte le voglies\ Dar ile... al mario in i
Ocni fera tornava a casa cote, she Hor le superbe pierre; t
E dava col baiton cena alla moglie; Alla lor liberta fanno il

Hor finti quella Peffa quei demoni, © Pero che tanto erandi, e tant!

Sopra di lui fan trionfar baton...» ».. | Chan fatto per lor carcere'
Termina la mostra delle:pene date'a idelinquenti con tre forte
il primo è dato a coloro, che non vollero mai pagare i loro debiti. Iba,
dato a i crapuloni strapazzatoridella mogtiex ll terza & quello dato alle do
ambiziofe,e vane. ag janguu vik > > sitll
TANAGLIE, Strumento di ferro fatto a foggia divcefoia, e serve V
chiodi da i legni,, ec. Da i latini detto forcipes, ig Mie ta hea's
NON ne pag un matadetto. Non volle mai pagareun debito, Non pagd mii
un quattrino di debito. L' epiteto ma/adetto ha la forza d'un becco d'un
no decto sopra C, ¢. f..68. ) vast tye
TENNE gran posto, Si trattd alla grande), e'fece spele bepiali', cioè grant
inconsiderate. Lat, immanes.: 23 elit Held eg?
NON hauerebbe volute crovargli per la via, Quand'anche egli' havefle trovato per
la strada il denaro, del quale era debitore;non havrebbe ad ogni modo' pagato il
suo debito. Questo termine ci serve per esprimere', che nefluna cosa hat
potuto muoverlo dal suo proposito, e fargli venir'voglia di pagate. ~~
PESTO, Infeanto, ed ammaccato dalle-battonate, che gli danno quel De-
moni finti la sua moglie. E questo vuol dire trionfar 'bastoni..) |» t
AYO M vile, Qui vuol dire huomo di bala condizione. *
RISC AWOLO. Huomo che pratica le bische «| Bische diciama quei raddotti
pubblici, dove si giuoca a carte, e a dadi; nome forfe'venuto dal verbo bi/ear-
sare, che vuol dir Mandar male sproposicatamente il fao havere: e corriff
al Latino prodigere. L' usd Dante nell' Inferno C, 10,
Biscarca ye fonde te sue faculeade, ag A
GHIOTTO. Huomo,a cuipiace mangiar del buono'. Vedi sopra C. es
DAV-A col baston cena alla moglie » In vece di portar da-cena alla moglie; la be
stonava, Costume assai usato dalla gente' d' infima plebe, imbriacarsi all'ofteri¢,
¢ non pensar' a mandare da cena a casa alla moglie, e così-briachi'tornare @/¢a-
fa, e perché la povera moglie si duole d' eer digidna', bastonarlas
DAR del c,,. in fal laferone, Quand? un mercante fallisce; diciamo + vitalelt
dato ile...sul laftrone. Beanetto Latini nel/Patafio-disse Dar det Ad wy
Questo proverbio è nato'da un'proverbio antico, che era 'in Firenze; chee *t
i quali fallivano, © rifiutavano-l' eredita del padre', “andavano 'nel mezzo:
Mereatunuovo ( luogo dove si ragunano i mercanti pers iare)e\gaiei era y
¢d € ancora una gran laftra di marmo tonda, che si chiama i} carrotcia( pere evi
& poita per segno,dove si fermava il carroccio, sopra il quale s' inalberava Pinte
gua generale de' Piorentini, quando andavano alla guerra ) e sopra —
a:

Shy







Bese e see we OE pm ERE ESSE SS ee EE

OPS SSEREG EES a oe






jor

. awifta del popolo, che nell' hora, che si doveva fare tal
10; & questo atto afficurava la loro persona dalle mo-
'di debito s:ne potevano i creditori moleltare se non la roba, las
deva ceduta tutta a favore de i Creditori, non essendo per questo
(0 il debitore a pagaie ultra vires, essendo questo come un cedo bonis del
'dus. Così questra laftra alle persone de' falliti, che a quellarifug-
era come una Ara, © vogiiam dire altare, o luogo facro, o afilo, o 1
già, che dail' esser prefi gli afficurava, e questo, perché cflendo dedicata
pubblico di foftenere 1i folenne:carro, e la tanto famofa infegna della
endeva per questo riguardo franchi, ed immuai coloro, che col se-
 prendevanne folennemente, e con cirimonia il posscfso. Di qui dar
laftrone yuo) dir-fallire. Edi qui pure,quand'uno casca, e batte il c...
lle Jaftre diciamo < // tale ha rifiutato il padre, Fallire ancora dichiamo
i pole: Ef rale U' ha infilate; che corrisponde al Latino decoxit,
TTON/, Sono il latino dateres detto sopra C,1. st. 67. £ fare, o dare il
» Vuol dir fare a uno qualche danno grave; e qui vuol dire; sono il lor '

,¢ pena.

TANZA LXXIV. STANZA LXXV.

in orecchiche mi par che e suont Dice la Maga; Vo venir anch' io,

po tabellaccio del Senato, Perch'il veder pin altro non m' importa,

| Sichee' mi fa meftier chrior abbandoni, Ed in questa Cittd così a bacio

\ LiPeri cht io non-voglio esser' appuntaro; A diria mi par @ esser mezza morta;

i v ch reStavano i lion, Vaglia trattar col Ke d'un fatto mio,

Ma non posso venir chro son chiamato, Ed andarmene poi per la più corta

Ed ecco appunto Diavoli co' lucchi; Ed ei le dice in burla; Se ru parti,
Peri lascia ch'io corra,e m'imbacucchi, Vaviain un'ora,etorna poi intrequarti,

. li suddetti gaftighi dati a i delinquenti, Nepo sentenda la Campanas

del Senato si licenzia dalla Strega, ma dovendo efser' anch' ella nel Senato per

Bat Re, dice volerlo seguir sin quivi, di dove spedita se ne yuo] andare per



















''*















STAR i orecchie. Ascoltare con attenzione. e4uribus arrettis aufeulrare,

© TABELL-AC C/O. Così è chiamata da molti la Campana del palazzo de} Po-

deft ( hoggi del Bargello, la quale & detta la Maddalena,come yedemmo sopras

ein questo C, stan. 23. ) forse dal latino T-abeltiones, che vuol dir Notai, i quali di-

Moravano, e tenevano i lor banchi dentro, ed attorno al detto palazzo,ragunan-

~d0vifi al suono di detta campana, la quale hoggi e detta anche /a furba, perché

lori d'alcune feftc, non suona, se non per esecuzioni criminali di tefte, e forche,

 €/a nowe per mostrar l'hora, che non si può pill portare armi; o pure è così

-detta dal suono oscuro,¢ malinconico, o che almanco rappresenta cosa mefta,
'come il suono delle tabelle ne' giorni Santi.

4 NON celia essere appuntato, Coloro che'fon de! Consiglio del Dugento, e d'al-

“tri Magiftrati di Firenze se non vanno al detto Consigiio,quando si raguna a suo-

: erence condanaati in certa somma di danaro; e questo diciamo









2YCCO, Br la sopravvelta,o mantello Curiale di Firenze, ed era anticamente
nay

V abito 4


:






302 MALMANTILE: >

l'abito civile ordinario; e perch questo haveva già un
metteva in doflo detto lucco, si doveva dire imbacuccarsi. Va
141, Subito fu preso; e imbacuccata col cappuecio, fu condotto alle carci
C. 11, st.22. a Pou
e4 B.AC/0, Campagna, dove batte poco il Sole, che diciamo Al rezi
uggia. Vedi sopraC, 3. st. 71. alla voce Vria, e sotto C, 9. st. 44, ¢C, 10,
1 contadini in vece di dire: luogo o piaggia volta a mezzo giorno, dicono:
Jatio, ed in vece di dire: volta a tramontana, o a fettentrione dicono: ab
© a paggino che e i) contrario di folatio, Credo venga dal Latino
si come natio da natixus. Da molti si dice meriggio quel luogo,dove
no i raggi del Sole per interposizione di che che sia,¢ ( pare a prima'
troppo lodevolmente, perché meriggio da meridies vuol dit mezzo giorno,.
do appunto i raggi del Sole sono più: quocentije però andare al meriggio p
be che volefle dir più tosto andare a scaldarsi a' raggi del Sole di mezzo
che andar all' ombra per difendersi da i raggi del Sole. Per corroborazion
questo idiotifmo, si uova in Autore approvato per buon Scrittor Toscano
vollero fare il viaggio di notte per lo gran freado, ma si bene in full' ora met
allora cheil Sole con i suoi raggs haveffe addolcito i rigori hiemati.Maquestitalifid
dono conl'uso, e potrebbe dirsi anche colla ragione, perché meriggio nel si
to di luogo ombrofo, e difefo dal Sole,è lo fiefio, che juogo da paffare IL ore
del mezzo di, la quale cosa i Latini dicevano meridiari, Catullo. dube adte
aeridiathm, Ora dal meriggiare, cioè fare all' ombra nell' ore calde è
meriggio, ¢, da meriggio, rezzo.. Va in un' ora, e torna pai in tre g.
€ uno (cherzo usato aliai fra gente bafsa,ed intende Va hora in uno,cioè
¢ torna poi divi(o in tre quarti; fij impiccato; se ben pare che voglia dire: Va»
in un guarto d' ora, ritorna in tre quarti. Cirimonia da Diavoli,
STANZA LKXVI, STANZA LXXVIL —
Tun vnoi gli rispos' ella,sempre il chiafo; Ed ella per oferta così magna hah
Wel Con/figtio così ne va con esso Ringraziamenti fattigli abarellay.
Ove ciascun l'honora, e dalle il paffo, Dice,c'hor mai sbrartar vuol la capagnly
Sbirciandola un po meglio,e più da preffa, E tornar a dar nuove a Bertinella, —

Ella baciando tl manto a Satanaffo Pluton le dd ticenza,el parse
Fino alla porta ye [i se ne (gabellay

Lopregad' ofsernar quanto ha promesso,

Ei.ghe! conferma, e perché stia sicura, Ond' ellain Dite aun Vetturin saccopty
Per la Palude Stige glielo giura. Che la rimeni a casa per la posta *
La Maga così scherzando, e burlando con Nepo se ne va con esso in Confi-

glio, dove ognuao l'honora.. Fa riverenza a Piutone, e lo prega a ma

quanto le ha promeffo; Eigliclo giura foleanem:a:2, ¢1 accompagnatala fino
aiia porta del Consiglio la liceazia, ed ella va a cercar d' un Veccuriag, che la

riconduca per la polta a Casa,, 8
Tu vuci si chiafso, Tu yuoi la burla. Tu scherzi. Chiaio nel proprio e 3

strezta, vicolo Lat, vicus quali erano le strade di Ro oa aatica, edel pri

cerchio in Firenze. Gio, Vill, 10, 29. S apprefe fuoco ix Firense in Borgo S.

Appostolo nel Clafso tra' Bonciani ye gli Acciainoli,B pecché in queste straducole abl-

tavano taluoita donne di mal aftare, Chuatio e detto forse da Vicus Vicario, Ha-

sas



























a2EH@ Ze O- Se LED






FS Sue

SESS SEAS a”

==



ESET ER EC SEE 8 Se

a



an SESTO CANTARE jes

ga; in buon Latino Vicinia ) venne a significare Posribolo, e perché in tali difo-
nefti Inoghi si fa gran baccano,e ff scherza, e si burla senza rispetto; perciò
' iglia per burla,,per ischerzo. Se bene e molto verifimile, che in questo

hielo
ultimo significato di strepico, e di baccano, quale fanno quelli, che licenzioia-
-menze tratcano,¢ burlano, venga dal Latino de' tempi baifi; che il suono di

» © degli organi, e degli altri strumenti domandavano C/aficum,

tutte le campane
che i buoni Litini dicevano della rromba, a cui son succedute le campane. Lt

lo dice Glas,

| SBIRCLAN'DOLA, Guardandola bene. Vedi sopra C, 1. stan. 9.

9.
 GLIELO ginra per la Pande Stige. Giuramento folenne, ed inuiolabile degli
Dei secondo la falsa credenza de i Gentili, come si cava da Omero in pil luoghi
del Lliade 5 ¢da Verg Zn. lib. 6.:

: Stygiamque paludem,



ea
>t: Dif cuins inrare timent, & fallere numen.
« la ragione, per la quale questo sia giuramento folenne,secondo Servio,¢ questa

4» Styx'meerorem significat, Dij autem lati (uat femper; ergo qui meerorem non
y» featiunt rant per tristitiam, que res eft sue macure contearia; ideo Lufiu-
per execrationem habent. L' altra ragione ¢, perché havendo Vitto.

iuola di Scige aiutati gli Dei nella guerra contro ai Gigaati Ticani, Gio-

ve per rimuneracla, volle che coloro, che giuravano per Suge di ici madreo,
fussero privi del nettare delli Dei, e non osservavano il giuramento. & queste
“sole furono finte, e credute di Stige, perché (econdo Teofrafto queito Suge era
un fonte in Arcadia, le cui acque, e pesci erano velenofi per la di ini eltremas
frigidita; e di questa acqua dice Plin, lib. 30. cap. 16. che Aatipatro voleife da-
re ad Alctiandro Magno, quando volle avvelenarlo per consiglio d' Arifotile.
yy Vogulas tantim mularum repertas, neque ullam aliam materiam, que non,
a» percoderewur a veneno Stygis aque, cum id dandum Alexandro Magno Anti-
a» pater mitteret, memoria dignum eft, magna Ariftotelis infamia excogiratum.
A barelia. ln quantità geande, Si dice a balle a maffe, a facca, ec. sono
pero modi bafli, e più tosto scherzosi, e s' usano parlando tanto di cose corporee,

quanto incorporce,

\ SBRATTAR la campagna, Andarlene: Sbrattare propriamente significa net-
tare 50 ipulire, contrario d' /mbrattare; si che sbrattare él paefe vuol dire ripu-
dice il 9 © per confeguenza andarfene da quel luogo.

SENE (gabella. La la(cia; Sisbriga; si libera, e filicenzia da lei. Dedotto
dalla Gabella, che si paga, perché, come è pagato il dazio, o gabellad' una.
Mereanzia, si dice sgabellata, e così si spedisce, e manda via.

BITE «. Qui la Città di Plutone, detta così da divitie, le  ci vengono tut-
tedi sotto terra. I Latini chiamarono Due, quel che con Greco yocabolo dice-
vano altrimenti P/zsone, che vuol dire il medesimo, € significa il ricco Lddio,
Addio detie ricchezze, come s' è veduto sopra.

WET TVRLNO, Coini che prefta cavallia nolo, o a vettura.

STAN.



'
4
|





304 MALMANTILE: —
3 STANZA LXXVIILA
I Re fatta con tei la dipartenza | Saliro alla,
Al falon del Consigho fene torna,
Onde ciascuno alla eos
Alza il Civite sé abbaffa gii le corna,
Plutone licenziata la Maga fene torna in
sua refidenza si prepara a discorrere.
FATTE le dipartenze. Licenziatifi (cambievolmente.
ALZ A il Civile, Alza le natiche. Civile € una prospetti
fentancte abitazione di Città; contraria a quella, che-si dice of
pagna. I Latini simil ue entrate principaliin;/
di quelli che venivano dalla piazza,o dal mercato; l'altra di coloro., che si
geva che venifiero di lontant pacfi, o di fuori dalla Città; La prima ente:
diceva a foro, 1' altra 4 peregre, siccome riferisce Vitruvio. Noi per questo —
miamo Foro la parte in Paccia della scena, Lin eRe
RAGNI, Quci veli che fanno i ragni.. Narrano le favole degli antichi
li, che in Lidia fa una femmina detta Arachne nata in contado di bafla
quale fu così valorofa nel ricamare, ed in ogni sorta d' artifizio di tela ¢!
che non folo superava tutte l'altre femmine, ma hebbe ardire di co;
la Dea Pallade; onde Pallade superata, e vinta da lei, per dispetto le;
Javoro,¢ la converti in Aragno verme, che è quell' infetto che si
veli per pigliar le mosche da noi chiamato, ragno, o raguarelo. Ovid,
tam, Dante nel Purg. C. 12. rocca questa favola, -
O folle-Aragne, si vedeva io te
Già mezz? aragna triste in [u gli straceé 2
Dell' opera, che mal per te si se. bie è
DR APPELLONI, Così chiamiamo quei pezzi di drappo i quali ee
no pendenti al cielo de i baldacchini, e delle refidenze de i Prinejpi;,
rano le Chiefe, ec. Varchi St. Fio. lib. 14. Ed al vano dela Cupola era tirate in fu
Le funi wr belissimo ortangolo di drappelloni. Matt, Villani lib. 9,.cap. 43 deferiven~
do le nobili efequie fatte nella fepoltura dei Cavaliere Messer Biordo degli 4
uni. E sopra nha xn drappo a oro con drappelloni pendenti coll' arme del peice}.
comune,e di parte Quelfa',¢ degli Vbertiné. Tali drappelloni coll' arme si
appiccati in gran numero nella infigne Chicfa Collegiata di $, Lorenzo un
giorno dell' anno, per memoria di antichi benefattori, ee
SPVT-A un ciabattine. Quando uno per soprabbondanza di catarto ha difficulta
in spurgarsi, fogliamo dire: gli ha wn ciabattino giis per la gola ye doy
Sputa un ciabattino, — a ee oe ¥ - cme se nel Lal
oni, Coll' o¢chiaia lnvida toffire, e spurar far, te 2 r
Spee Fan STANZA LXXIX. noaniaie mM
Spiegar volendo poi quanto gli occorre, Onde nui fiam quaggit in. fondo di
Comincia il [uo proemio intal maniera; Gente, a cui si fa notte avanti fers
Voi che di sopra al Sole in queste forre Voich' in malizia,in ogni frode,e
CadcSts meco all' arta oscara,¢ nera, Siateé Adacftri di color che a)

































ug

Ce eR eek ai kB








+

SESTO CANTARE, 305
STANZA LXXXIL
Cominci il primo: Dite, Mdalebranche,
r e } Quel che e'vi par che qui v'adafe fatto,
'bazzicar taverne, e chialfi Levato i Tocco, e follevace t anche

5 agnun di voi st bravo,edotto,
ea vo.
ib pincrife

1 Alor quel Diavol n' un medefmotratto
a

Vn capitombol fa sopr' alle panche,





I a un famiglio a' Otto; Efalta ae nel mero com! un gatto s
aunque, benche pare Cittadini Ma perch'il Lucco s'appicco a nn chiodo,
el vieupero ingeg ns peregrini ' Si ric e, e parla a questa modo:
TANZA EX xr, STANZA LXXXIII,
tusti'in correfia O Re, cus splende in mano il gran forcone
Da Martinazxa nofira confidente, Sil Cappello (periale ha quel fegreto y
 Poithe Baldone ancor cerca ogm via Col qual si fa feornare un pedignone,

Dh entrar in Malmansiscon tantagente, fo ? ho da far tornar un' buomo a dreto.

ar-ch?eglisbandi, e trucchi via So gidche qualche debito ha Baldone
| Adbope Mctitentensy z che e lo vuol pagare in (ul tappeto, '
pe ere [opra questo il suo parere Perciò manda Pedino (a in campagna,
. 'the ©! ci fuffe da tencre', Ch' ei giuocherd di posta di Calcagna,
: io de i Diavoli fr: composo dail' Autore; dopo che egli ottenne
meat » nell' esercitare i} quale conobbe l'autorita, che si usurpano i Can-
: +s anon hoe be » metce per Cancelliere di questo Consiglio un Ciappellet-
celliers che fun notaio (cellerato, secondo che riferi(ce il Boccaccio nelle sue Novel-
leye bcontraddica a tutto quello', che vien proposto. I nomi di questi
colt pis son cavati da Dante nel suo Inferno; e sappia il Lettore, che li spro-
he dicono,son poco lontani da quelli, che PAutore setiva dire nel areating
rid iperfonaggi che finge in questi Diavoli sono simili alli suoi Colleghi,
ed egli medesimo in leggermi questo Canto mi diceva; il tal Diavolo è simile al
tal mio Coliega ye il tale, al tale; e mi parvero appropriati benissimo; non sti-
mo già bene nominargli. Ma tornando a proposito dico, che Plutone volendo






fens re de' suoi Senatori, fatta una breve orazione nella quale inferiice
un ver | Petrarca Gente, 4 cui st fa notte avanti fera, ed uno da Dante Siese i
Mathri di color che fanno, ordina a Malebranche il dire, quel che egli farebbes
per mandar via Kaldone da Malmantile,ed egli,fatte prime sue diaboliche cirimo-
nie dice che il suo pensiero farebbe di farlo citare alla Mercanzia da qualche
suo creditore, salea:

FORRA, Valle lunga, e stretta posta fra poggi alti, onde poco dominata dal

3© però ben detto forra il pacfe infernale dove non batte mai fole.

 GENTE 4 cui si fa notte avanti fera, Con questo yerfo del Petrarca, ? Autore

 intende che costoro son sempre di notte, cioc al buio.

, BABLV ASSO, Huomo senza giudizio, icimunito. L' origine sua & scura;
forse da Valwaffor parola feudale, dalla quale e fatto anche Barbafforo, lo Netio
che » o-dortoraccio; faccente; e che si da scioccamente ad intendere di
sapere o pure da Bwaccio peggiorativo di bue. Vedi sopraC. 5. stan. 1. Ul Bini

in lode del Malfrancese dice.
Qq Eri-









Mi =p.






TPE



305 MALMANTILE | 3
Erispondendo a certi Habbuaffi, scea 2c bia
Che voglion dir, che questa malattia i eS ntn se
Tatto il corpo ci florpi, eck fracali. Ah eat RRR



Ed il Molza in lode de' fichi: My
Hor fa tut argumento, habbuaffo. ono AE
TONDO pii che ? O ds Giotto. Huomo tondo vuol dire huomo grosso d inge-
gno, ed ignorante, come s' e accennato sopra C. 5. stan. 1, si che più rondo dell''O
: Grotto vuol dire ignoranuthmo, e pil, perché lO, che fece Giotto Pitore fu
tonditfimo, secondo che riferisce Giorgio Vatari nella vita di eflo Giotto,
BALZICARE, Praticare; Converlare: Bocc. Giorn. 9. Nov. 5. £ vatrene nel
la cosa dela paglia, ch' e sh mighor nego che ci sia, perciocché non vi bazzica mai per
“ote. 1



ona.

CHTASS, Bordelli, lupanari, luoghi, ¢contrade, nellequali habitano les
meretrici, come era in Firenze il Chiaffo de' Buoi, e il luogo, dove ora € il Ghet-
to, detto anticamente Chiafo & perché in tali moghi usa di fare fracatio, e rumo-
re difonefto; di qui forse e che chia/so, e bordedio si prende ancora per tumult die
fordinato, insolente, e lascivo..: swash

'PIV' cattiuo di tre afi, Affo si dice il numero-uno-de i dadi, che e i)
numero, e per confeguenza nel pil è il peggiore che vi sia tirando tre dadi,)
questo il presente termine significa cattiviflimo:, che vale per aftutiffiime, ed e lo
ficflo che Pil trsffo a' un famuglio a' Orto, che pur vuol dire fagacissimo,eche fail
conto suo, Famigio a' Oro. B' uno de' Birri del Magiftrato degli Orto di Bali
di Firenze, che e il Magiftrato Criminale; e perché si sappone che cofloro fap-
piano tutte le furberi¢e, però si dice: Il tale e pis triffo a' un famiglio d' Onto, per
esprimere; e huomo fagacissimo. 1 Greci dissero Cantharo. afturior, che qui
Cantharo fu un' ofte d' Atene aftutissimo. 4/*m in antico Latino voleva dire,
sotto, fens accompagnatura; onde chi cantava  senza strumento che L/accompa-
gnaile, si diceva costui: canere affa voce, Di qui può esser venuta la voce Afoes
Kespare in affo, cioè esser la(ciato folo, se bene altri gli aflegnano altra origine: 0
pure da «fino che così chiamavano ne' dadi /' #nita i Greci, dicendola Ones. I
nostro Proverbio: O a/s0, o/ei i Greci dicevano, o diciotto, orre. O sre fei, ore
afi. 'Giulio Polluce lib, 9, al cap, di giuochi fanciulleschi, e de' trattenimenti de

1i antichi. AS
PAZZO Cittadino, Questo epiteto si suol dare 4 colore che fanno sutte le tor elt
4 casa, e senza considerarione; ed € lo stesso che dire ux cernellaccio,

SBAND-RE, Disfare le bande, cioè licenziare i Soldati.

TKYCCHI via, Se ne vada. E' modo baflo, cavato forse dalla parola Ze
ruck Tedesca profferita da i Lanzi, quando con Ie loro alabarde fanno allonta-
nare il popolo; O forse dal giuoco del Trucco, che si dice truccare, o trncciart
la palla, quando cogliendola con un' altra palla si manda via dal luogo', doves

era; dal frequentativo Latino tra/fare usato da Catullo. ' ai

TOCCO. Con il primo o largo; Specie di berrettone, che anticamente ulava
in Firenze in yece di cappello. Varch. Stor, lib, 11. Cow le calze foppannate: a Ie
jerra bianca,¢ le berrette, o vero tocchi di colore rosso.:

SOLLIFATE ? anche. Alzati i fianchi, cioè rizzatofi da (edere, —

jicias




ek=s & re:

BEERS Ee Seek Sec s-

Se

Le fF Pe coor Fae

=
=

Sar erree












ore cubito,

Dan. Inf. canto 34,
8



7 la quale

zioni civil

STANZA LXXXIV.

Pluton diede con tutti una rifata,

\fiantar fino il brachiere,

RB difeegi: Va via bestia mcancara
Com' entra celafsedia il dare,e havere?
Segualalero che vien dela pancata,

Rizzato Barbariccia da sedere
Sichina,ementre abbafsa gli la chioma

Alea le pe,e mostra sf bel di Roma,

STANZA LXXXV.

Poi # intirizzaye dice in rauco suono:

- Se non si leva dalle fquadre tl capo,
Quale e Baldone,e non si da nel buono,
Mai si verrd di tal negorio a capo,

| Dove y se manca lui quanti vi sono,
Restari come molche senza capo,

- A poco # poco, a truppe, e alla sfilata
Partendoyn breve disfaran o areata,

——— a



Qqz

SESTO CANTARE; 307

parte del corpo, che è fra il fianco ye la coscia, da Ancon greco

ire gomito;¢ si piglia per ogni (ora di piegatura, come lo mostra il

Città d? Ancona così detta dal gomito, che faquivi la spiaggia; Pli-

pio libs 3. caps 13. La iifders colonia Ancona apposica promontorio Cumero in ipfo
q se if

«© Quando noi fursmo la dove la coscia
Gait sis Ss volge appunto sul grofso dell' anche.
Edi qui sciancato'é un zoppo, che habbia mancamento in tal luogo. Vedi
sotto C. 11. stan. 40. B il Latino Coxendices.
ty PITOMBOLO B' quando uno, posando il capo in terra, volta sopr' as
quello tutta la vita, Vedi sotto C, 7. st. 20.
| ORB, cui plende in mano sl gran forcone. Fingono che Nettunno Re del mares
atello di Plutone usi in vece di scettro una forca con, tre punte, e però dettas
in realta è una fiocina da pescatori, Latino fu/cina, e Plutone
tun Bidente, cioè forca con due pente; Equesto & il gran forcone.
er Speziale.B uno Speziale in Fireaze, che fa per insegna un cappelio.
aiubeaowe - Enfiagione che viene ne i piedi',.¢ nelle mani per causa del
r Latino Pernio. Vedi 2 C, zt. 6.
- LOamal pagare in sul t: « La vuol pagar per via di Corte, con tutte le fo-
temipebe, non vuol oa Saede non feglt mandano j birti a gravarlo, o cattu-
en dice che Baldone gimecherd di calcagna, cioè fuggira per la paura
teller preso per debito, quando vedrà Pedino, che così G chiamava uno già bir-
ro della Mercanzia » che éil Magiftrato, per yia de] quale si mandano l'efecu-

¥.
re + Subito. Latino e veffigio. Traslato dal giuoco di Tan » che si dice
dar di posa — si da alla palla prima, che tocchi terra. Vedi
e Ly s

sotto C. 7,st.92.
TANZA LXXXVL
Circa il pigliarlo,# ionoal' ho, eglit fallo:
Facciam conto ch'in braco alla pastura
Vin toro sia costui o xn cavalo;
Tiriamgl addofso qualche accappiatura
Legata innanzia un bel mazzacavalla
Collocato in castel prefso alle mura,
Ond' ei si levi un tratto all'aria, e pai
Si tiri dentro,e dove piace a noi,
STANZA LXXXVIL
Buono, rispose il Re,non mi dispiace;
Ma il Cancellier di subito riprefe:
Sia detto,o Senator,con voffra pace,
Tant! oltre il poter nostro non 8 esse/e,
Li tutto (aria nulo, ef foggiace
Ad efser condennato nelle pefe,
Ed io farei flimato anc' un Marforio;
econfentir a Kn! atto perentoria,
STAN-













we
308

ae NZA ieee
Perché sempre de ire i

y slrapacs 4 ete 5m rAgion®y
Pei Sella è in morayvienfi a un! inibitay beg upalerele
E non giovando, alla comminazione
Ch' in pena caschi delle forche a vita y
E se la parte innova lefione y

Aller puo condennarsi, havende ofate
Di far causa pendente un' attentato >
Plutone, ridendo con gli altri della coveaaaeens

secondo, che viene nea pancata,nominata Barbariccia, cheidica i)
e questo propone che si tiri un laccio a Baldone} € per vid d'un
s'alzi, e G porti dove pil piacera; ma cié:non ea C
de Piutone ordina al ter2o nominato Calcabrinal 3 dica il suo parere
fiui si rizza,, e fa riverenza al Re per far il discorso, meer







ti Ottave.
' SCHLANT ARE, Denes » spezzare detto da. Splenare, BB
lo, che fidisse sopra C, 3, st. 5. A

BESTIA incancara, Così diciamo per e(primere n*huomo feo 9
traslato da quelle beflie, che alle volte conducono. con loro.i Monts
quali essi fanno far molti giuochi, e dicono che tali: bestie-sieno:
operino per vie diaboliche. Si dice be/tia smcantataa und di poca confi
ed avvedimento, come il Lalli En, Trau. C. 20. 56.
Così gridammo, e con.la propria appa
Ci deffimo in sul pie bestie incantare
COAL entra con l'afsedio, Significa come s' accorda, o che i che re
I afledio,
IL bel di Roma, Così diciamo per intender apertamente c... 5
Roma intende il Colofseo, da noi corrottamente detto Culifeo,
SINTIRIZZA, Si vizza,si distende in fu la — 'EB' un' atto,¢l:
ta una certa superbia, e prefunzione di se stesso y ed & quella prefopopea' py ches:
dicemmo sopra C, 1, st. 72. 5 a eae
NON si verra a capo dé tal negorio, ec. Non si conchiudera», 0: terminera if nt
gozio. ne woagiher
REST ATI come mosche fenxa capo, Cioè senza oe direzione '0g
Senza sapere che cosa havere a fare., o risoluere: i infect fo
capo, $' aggirano inutilmente, strascicando il sane di bey
dove.
ALLA sfilata, Senza ordine; confulamente, e senza andare in ila,
nanza: Sbandati ' so29
S' 10 non V ha, ezli éfalle. Yo son sicuro di pigliarlo. Seionom lo-p
per errore, E' specie di giuramento vantatorio, come: appease
sotto C..8. fan. 72. & mio danne che vedremo C, 10, stan. 49.
ACC APPLATVRA, Vna fune accomodata,, € —- cay
do, che — y ibqual nodo firdice-cappio scorfoio.,
















page e st FER w Gees = FL TFAF EE



=>










- SESTO'CANTARE: 309

MALZAC AVALLO', B un corrente, o pertica grossa congegnata per tra-
——-yerfo, come: acavallo 'un legno ritto; la quale's' alza-da-una parte
'con tirare a la parte | » E questo ordingo e usato assai ne i piani di

Firenze per cavar I" dai i. [ Latini lo dissero rolenonem a toliendo,
dete Smiles quella awbiid, della quale si servivano i nostri antichi as
Acagliar pictre:chiamata Azangano.. Livio dice: ariere Tollenonibus Inbramenta
i iy aur fa » sp vobuffos incuriebant, sta hina milirare
fien descritta da Vegezio così; Tollenc dicitur, quoties una trabs in terram praatee
a ry cui in fummo verrice alia transuers4 trabs longior, dimensa medietate, 6on~
neititur, €o —— 5 at si unum capue-deprefjeris, alina erigatur, L' antico Vol-
“garizzamento lralenoé detto, quando una trave alta si ficca in terra, alia quale nel
J una altra trave più lunga per lo traverso, enel meyxo mifurata, si com-
«mete in tal modo che se! wno capo si china, l altro in aito si leva. Da questa voce
-alvaleno ( Lat. toileno)si dice ? e4italena giuoco, che i ragazzi fanno con due travi
* incrociate, e bilicate l'una sopr' all' altra a foggia di Mazzacavallo. Vedi sopra
Ga, stan. 48, Mattio Franzefi contro alle sberrettate dice.
6 Biggetnslo- Ma chi trovalfe il modo a bilicalle,
ee
os



Ma» 2 Sarebbe un [chifanoia, e faria bene
vail, Van contrappefo d! un mazzacavallo,
SIA detto con vostra pace. Perdonatemi; s'io v' offendo in dirlo, Non vi adi-
wvioffendete, io lo dico. Frafe de' Latini Pace rua hoc dicam, Nell'
igen di Quinto Catulo, Pace mibi liceat, Caleftes, dicere veffra. Adortalis
sue pulerior se Deo, Che Annibal Caro nel primo Sonctto delle sue Rime vol-
CO olfimi j e *icontra a lei mi parue oscuro, Santi Nums del Ciel, con vofira paces
—— LO vieme', che dianzi era si bello.
(2 —--—« BSSER condennari nelle spefe'. Cioè buttar via sa fatica, e il denaro, oleum, &
«Opera perdere. Ma propriamente ¢ffer condannato nelle /pefe vuol dire, quando
=“ UNO!Per aver litigaco wna cosa ingiulta, e dal giudice condannato a rifar cute
le spefe all avveriario; e però questo Cancellicre dice, che noa vuole acconfenti-
8a tale'atto-per'essere ingiufto 5 e da efer condannato nelle spefe.
Ss imato'un Adarforio, Sarei stimato un' huomo senza sentimento,o giu-
dizio; come @ la'starva di Marforio in Roma,
' ATTO fruffratorio  Awo vano, fatto senza proposito, E questo termine,
come tutti gli altri-delle (eguenti stanze 88. e 89. son termini Curiali jche veaca-
do dal latino', ed essendo praticati in cucti li Tribunali d' Italia non-dubito, che



,
,  farannointefi da'ognuno; però ne tralalcio la spiegazione.
, ~ STANZA LXXXX. STANZA LXXXX1L
E poi ha fatte riverenze in chiocca Aa in vece di quel cappio da beltresca,
C0 fuivi più Lindi-a pianta di pattona, Ch'é il:toffice de ladri, si prouuegga
Si soffia it nafo ye [parzafi la bocca y Pua bilancia, o rete per La pesca;
- Epostain equilibrio la persona Con una lnnga fune, che la regea 5
© Come quel che si pensa dar' in.brocca E perch' sl fatto meglio ci riesca
Tutto sfromato dice: Alta Corona, Si ringa tutta, accio che non si veega 5
Circa Pordingo, pur si merra in opra; Einverra quanto ell' apre, ivi fispanda,
© Perch*ioconcorrose affermo quatofopra, Fino-ch' +l porco vengane alla ghianda,
Bate tit: STAN-






OO EE Ls












310

STANZALXXXXL...,
Perché 8 e muovyon l'armi,di ragione
(Se dal capo l esercita e condotto )
Annan a tutti marcerd Baldone,
E quand' ¢i giunga,ed ha la rete sotto,
Fate che lefie allor fien più persone
A farla tirar fu con l'avannotto,
Operando in maniera,ch' egli infacchi
tn lnogo, ove si vede il fole a feacchi, Lodando il fa
S.T.AN, ZA \ LXSXXIVi 5 copies
'Qui, dice il Re, si da fempreinbudelia, Gli ha sempre più ritorte cl
'Siche mi cascan le braccia, ef ovaiay Mace' non locredes'einonvaals
Mentre cofini a ogni cosa appeila,; i
E co' /uoi punti mena il can per l' aia;
li terzo Diavolo, che e Calcabrina, dopo haver fatta rive
mano di smorfie, come fanno certi Oratori affettati, dice, che app
cavallo, ma che in vece del cappio scorfoio piglierebbe una rete da
il Cancelliere s' oppone; onde Plurone sgridando il medesimo Canc
al quarto Diavoilo, che e Cappelluccio, che dica il suo parere. at
IN chiocca, In quantità grande, in abbondanza, un diluvio di rive!
PATTONA. Specie di pane fatto di farina di castagne, che per ¢
più di figura lunga, s' aflomiglia a un piede mal fatto di un' huomo,
da, Prolusione Plautina prima dice: Qui enim pedibus fant planis ploti:
che piede di parcona si può dir plotus dalla voce Latina Plautus, che fig
fo; e questa dal Greco Plarys lato, largo; donde noi a tali huomini,
i piedi malfatti-diciamo Pileri. Vedi sopra C. 4. st. 17. li Franzefe dice P:
Spagnuolo Pata il suolo del più di bue, gatto, oca, e simili; dail Gr, Parei
vuol dire battere col pié; calpeftare; calcare; EB Patdn similmente in
2 il contadino, che porta le scarpe grandi, e grosse, e rozzameate fa
trebbe anche esser detta Partona, in un certo modo quasi Pafona, cic
pata grea s perché¢ quella a similitudine d' un pines groffolano,e
'Pattume dilie Ser Brunetto nel Pataffio quello, che oggi dichiamo 2.
spaccatura ye mescnglio di cose fracide; e ClO pure cred' io, dal Greco
peltare. Ed sl pattume vien rammuricando, Il che ha qualche simili 0
Patrons, cosa fordida, e vile, e di brutto colore, s Greci ( per dire anche q
lo sterco, perché si scarica i) ventre lungi dalla strada comunale, che dal?
firada batcuta si dice Pates; dissero dpoparema, il che può aver dato origine al
arole Pattume,¢ Pastona,, Gli dice findi, ma per ironia, che in, vece d'
picde ben fatto, & attillato, vuol dir piede (concio,¢ mal fatto. Lindo
la venuta a noi modernamente di Spagna; e & come /enda in quella lingua Vi
da) Latino /emita, e linde da) Latino mite; così indo credo che sia d i
mito, cio' limitato, aggiuftato, ben afletto, composto. Da Lindo diciamo
che Allindarsi,e Allindirsi Sp. alindarfe, '3 eng ela
$1 /offia it nafo, e spaxafi la bocca. Eipurga il nao, e spura, e con Ia lin
netta identi, che sono.quei lezz), che fanno moiti Ocarori, come porre in








































SESTO CANTARE: gu

brie ta persona; cio' dopo haver dimenato in qua, e in la il corpo,fermarsi in po-
fitura intir }, come ha detto nell' Octava antecedente, che sono tutte smor-

fie, che denotano nell' Oratore una sciocca superbia, e prefunzione di se stesso;

ed il Poeta lo tocca col verso che segue, dicendo: Come quello che se pensa dare in,
b che vuol dire, @ima di haver trovata l'inucnzione buona, e d' haver im-

; cioè dato nel segno.

O sfrontaro. Arditamente, sfacciatamente. I) Franzefe similmente ¢f-

ERT ESCA, o Bertresca, o belrresca; E' una specie di cateratta, ches' alza,
abbaffa, e serve per riparo di guerra in fu le torri, ein fu le mura fra uns
, ¢l'altro; © così si dice ogni luogo, sopr' al quale si falga con pericolo
ecipizio. Di qui viene il verbo berre/care, o bertrescare usato da molti per
ndere Armeggiare, o affaticarsi intorno a un lavoro, e non trovar la via as
hes i per berte/es intende la forea; per similitudine delle berte/che, le quali
i di legname, che si ponevano in alto. Gio, Villani lib. 9. 114. Pers
@ il porto era tutto impalizzato, e incatenato e@ di sopra di erosso legname imber-
+ Queste bertesche, o torri di legname alzate fu le mura dovcano servire
cose a gettar pictre, onde forse € la parola pertrechor, che significa,
pre i Spagnuoli munizioni, e ripari da guerra, cioè le nostre berre/che, det~
ta forse così da echar las pedras.
- BILANCTA. Specie di rete da pescare, detta così per esser a foggia di bilan-
sia; firumento, col quale si pefa la roba.
a ella apre. Cioè quanv' ella allarga per ogni verso.
"FINO a ch' il porce vengane alla ghianda, Fino a che venga a dare nella trappo-
la; ficali al zinkello. Esintende fino a che Baldone andando alla volta di Mal-
antile dia nella rete fuddetca.
\ SIENO Iefle. Sc bene leflo vuol dir Agile. Vedi sopra C. 1, st, 11. Tuttavias
far leffo vuol dire star pronto, all! ordine, o preparato.
~ AFANNOTTO. Pesce piccolissimo. Voce corrotta da Vguannotto, o Vw
annolto 5 che significa pesce nato quell' anno: perché g#ann0,0 wnguanno vuol
ir quel anno, se bene usato folo nel contado,¢ |l'Autore se ne servc in bocca,
@un contadino sotto C. 10. st. 35:1 Latini dicevano Hornus, ed hornotinus unas
colad'tn? anno. Il Poeta da nome d' avannorto a Baldone, percht dovea esser
preso con la bilancia, che € la rete, con la quale si pigliano gli avannotti.
IN lnogo, ove si vezga il Sole a scacchi, Cie in prigione; perché le finestre fer-

a

—

we

=
=

a

ye 'tate della prigione, battendovi i raggi del Sole, fanno a figura dello scacchiere,
g! nel luogo dove termina il loro sbattimento, o ombra dei ferri. Da queste fine-
oo r te, Ograre di ferro delle prigioni, si formo 1] verbo e4egratighare usaco
if dal Boce. Nov. 85. Tw m' hai aggratighato il cuore colla rua ribeba', Clo€ imprigionas



to col suono della tua rideca, come oggi diremmo: e da Brunetto nel Patafiio.
TVTT At fava. Tutta è una stessa cosa. Sol est Apollo, ipfe pollo Sol, Di-
Geil Cornazzano Nov. 11. che fu una Signora, la quale yolendo riprender co+
Potomee il mario, perché la(ciando lei andava dalle Meretrici, gli fece uns
utidimo desinare, ogni vivanda era condita, e ripiena di fave con diversi stra-
'Vaganti ma delicati fapori. 1) marito le domandava; Che cosa e questa? ed el.
we la

ee!



io.






gin MALMAN TILE?
la rispondeva; Fava, E quest' altra? Fava. In somma
gaor marito sceglieve quanto volete, perché sattae fava;: egli.
guta, e faceta riprenfione della lie, mut vita, conoscer 3
na ail' altra non può esser' altra differenza, che quella che nasce da un for
sfrenato appetito. E di qui poi venne il dettato 7 ase e fava che significa &|
anne daxke Meee a eee so of OO OM
/L Cipolla, Autore noto, che ha scritto.in Criminale.
a Plutone, che se bene quivi, e/cln/a ogni ragione Civile s* attende
Tuttavia gli Autori criminali non approvano quell' operazione
rimette dicendo; Se tu lo comandi,io non ho che replicare, € conc
anche tu lo voletli far' impiccare, e squartare; che questo iatende / i
lo squarto. Tole 4
7 ad in budella, Non si conchiude cosa di buono, Questo proverbio.
copertamente: Far come il cane de/ peducciaio, e s' intende dare.in budella. | s
e(prime discorrer' assai, e conchiuder poco, ed e lo stesso che dar.in cenci
MI cascano le braccia ye? ovaia, Mi perdo.d' animo affatto.. Si dice.
cuore, le braccia, le brache, il fegato, il fiato, eda moltis' ovaia peri
pertamente è tefficoli, e tutti hanno lo stesso signiticato, dl perdersi d' animo.
qui accoppiandone due, cioè /e braccia, e /' ovaia, esprime perdersi affatod! a
nimo. Latino ovaria, che si (ono sCoperte ultimamente nelle donne, dagli
erano creduti, e detti 1 loro telticoli. % si
AOGNI cosa appella. Non 'é cosa che Nia a suo modo,, da: difficulta a ogni
cosa,a ogni cosa ha che dire; e non se ne fla, e non fen' acquieta,
appeharsi termine legale. Toa
CO! suoi punti mena il can per aia, Co' suoi punti legali, e con le difficalta 5
che oppone, manda in lungo le cose senza venire a conciufione aleuna. e4its
vien dal latino area, e vuol dir quel pezzo di terra spianata, ed accomodata per
battervi, e mandarvi sopra il grano, e biade, ds
ALA piit ritorte, che faftella. Ha più ripieghi, e compenfi, che non a
cidenti, che faccedono, Ovvero egli trova subito riparo a ogni accula «
si dicono-quei legami fatti di vinciglie d'alberi, coni quali si legano i falci di
legne, € di sieno, o d' altro, detti ritorte, perché quella vinciglia si attorce pet
renderla maneggiabile, e fleffibile a fine d' adattarla a legare. Dan. Inf, ©. 19:
Che spezzate bavertan ritorte,¢ frrambe, et
El non lo crede, Questo termine significa Tu non ti vuoi emendare; e si dices
Won crede al Santo, /e non fa miracoli; cioè non crede d' haver a esser gaftigatoyin
che ei non prova il gaftigo. Qui dice se ei non va a degnaia, cio se egli non & Ie
gnato, e bastonato: Legnaia e un borghetto vicino a Firenze, ed il nome di
gnaia ci scrue per esprimere legnate, o bastonate. Vedi sotto C, 11, st, 116 gr
tar /a tigna., Dove fimettono diversi modi di dire per intendere Bastonar wn
CAPPVCCIO, Il Varchi Stor, Fiorentina lib. 9. dice; 11 Cappucgio ha te
>» parti: il Mazzocchio, che e un cerchio di borra coperto di pango, che
»» facia d' attorno alla tefla, € di sopra, foppannato dentro di rovescio
x» tito i) capo. La 1 opsia € quella, che pendendo in fu Ie spalle, difende
»> guancia finiftra. Li Becchetto e una strilcia doppia del medesimo panes














oa
a
rm

BSB eee Fa.

Pe ee ed

=








BRE eBas FS:









SESTO CANTARE Re s
ra si, in fir la spalla, e bene spafio s' avvolge al collo, e» %
eller pil destri, e più, intorno alla tela, ec. EB
che già portavano le persone civili 5.¢ del quale parla il
st. 7. alla voce Adaxxocchio., ¥
STANZA LAXAXV.
che direi,0 Sire, Perch eil ha, detto.con si texfo dire,
te ch? io dica mi vien detto, Ghiioffoper dir che mais' uds tal detto;
np non ofa, ch' io non ho che dire, Pero dico.ch' a dir non mi dd il cuore 5
ir quate qui quel? altro ha detto; Elascio dire a un' altro dicitore,
ecio,, che è il quarto diavolo, fatee sue cirimonic, fa un dilcorso fen-
¢, come si yede nella presente Octava tutta di scherzo sopra il yer-
le non richiede spicgazione, ma folo rifleifione al grazioso, ed in.

STANZA XCVIIL
Valeati, dice il Re, spropositato;
S? alcuna cosa qui non bas proposta
Come vuoi tu buaccio che'l Senato
Yada in Cancelleria per a risposta?
Par fento,rispond' ¢i ych' in Mdagiftrato
Così dir s* hi ed io l'ho detto apposta;
Mas ioviscadolerxo,e alcun m'incolpa
hiandellino. Dica Baciapile, Drerrore in questo,iomeneredo in colpa,
ANZA XCVIL, STANZA XCIX,
Non occorre brunir co i labbriifaffi,
Dice Plutone, ofsaccia senza polpe,
E fare il torcicollo, e ovunque pafi
Semmar discipline,e dir tue colpe,
Ch to foyche chi per lepre tt compraffi,
Havrebbe almen tre quarti dell.
un in mexco,bacia terra,ein fine Pera va a fieds,¢ segua il Tiritera;
7 Auago piovon discipline.. E queis' affeteaye parla intal maniera +
rende Cappelluccio, ed in tanto il quinto Diavolo, che e Libicoc-
re sbocear' Arao in Malmaatile, qual consiglio e riprovato co.
ile; Oade Plutone ordina al fefta Diavoio, che e Baciapile,il propor.
-¢questi dice, che vadano in Cancelleria per la ri/potta, che € lo stesso che
Vi
VE






































Bao SO pero Plutone lo fgrida, ed ordina al Tirirera che e il settino

10 dica, ed eglis' accinge a parlare.

INE. Quel che significhi diceamo sopra C, 3, st.27. E il Latino fearra,

shine. Vn poco poco. E qui, clicndo deteo ironico signitica; e un,

(pazio da Arno a Maimantile..

'ASEO, Balordo, melcato, stupido, bafofo, A questa voce allude la Pran-

Smarrite, confnfo, quasi sbafito. B far il bafeo vuol dir finger di nou in-

3 ersi huomo senza giudizio, dal verbo ha/ire vilto sopra C, 2, faa.
Reflo che far /4 carta di masino, o la gatta morta, vio sopra C. 1, st. 19.

? Ipocrifia, E' un SH ipocrito. La voce Ipocrito yi dal

© reco

>
?
+






THe

314 MALMANTILE §

Greco Hypocrinephai, che faona contraffare; ¢1' Ipocrifia si difinifee Vina calli.
da, ed afluta palliazione del vizio occulta; perché Ipocrito si chiama colui, che
essendo uno scellerato, nondimeno nell' abito; negli atti,e:

d' eficr buono, es' affatica di parere quel che egli noné,¢ ep rmer iamente J rin
ta significa commediante, iitiens ~S. 'Aposbad nel Sermone da 'enerdi dopo lan
s» Domenica della Quinquagefima. Hypocrita Greco fermone fiailator ie

>» pretatur, qui, dum intus malus fit, bonum se palam oftendit.
»» faifum, crifn vero mdicium (onat. Nomen autem hypocrite translacum eft a
3» specie eorum, qui eae tecta facie inceduat, distinguentes vuleum ccerulco,
x) hivcogue colore, & cozteris pigmentis, habentes fimulacra oris lintea gypfata,
yy» & vario colore distineta, nonnumquaim colja, & manus creta' Z
yy utad personx colorem peruenirent, & populum, dum in ludis agerent, falle.
3» reat, modo in specie viri, modo in forma feminz, & reliquis preeftigijs. I
sy Berni nell Orlando contra gl' Ipocriti, Won han'da fare lemaschere a ——
i, Questi (ciagurati sono di tre forte. La prima è di coloro, che fingono |
cospetto degli huomini d' esser pieni di religione 5 ed internameare sono ateifli,
La seconda è di coloro, che fanno del bene non moffi dalla virtù, o dall' amore
del bene y ma per esser creduti buoni. La terza è di coloro, che dimostrano di
non esser buoni, perché altri credano, che eglino fien buoni da vero, enon,
ipocriti, In questo Diavolo si scorgono tutte tre queste specie d' Ipocriti 5 che
appresso di noi sono lo stesso, che Bacchettoni; detto sopra C, 2, stan, 1. Dante
nell' Inf. C, 23. parlando di loro dice: ih
Laggils trovammo una gente dipinta;
Che gira attoruo assai con lenti palpi,
Piangendo, e nel fembiante franca 5 e vinta; Lai
E gui dite; i/o /morte, cioè faccia pallida, e scolorita; e'dice*che pioveno
/cigline per intender uno di tali Bacchettoni falfi o diciamo Ipotrito. B sotto
nell' ottava 99. seguente dice, Seminar discipline, che ha lo stesso senso. Bs' usa
assai il servirli di questi due termini per esprimere: B? paflato per questa stradas
un bacchettone. Veramente questi tali infami non ia(siane di valersi di tute le
forte d' apparenze, ed io ne conofeo uno della prima specie d' Ipocriti, che tro-
vandosi in una pubblica adunanza, ia cavarG ii fazzoletto di talea lascid cadere
una disciplina a vista d' ogauno; ed essendogli detto, che avvertifi', che gli era
cascato non fo che dalla tasca, egli raceogliendola 'diffe: Non @ mia roba;'
son così buono s che io adopri tali arnesi. Di/ciplina chiamiamo quella sferza- 5
che le persone veramente buone adoprano a battersi per far penitenza; così
dall'ammunire, ovvero gaftigare-il corpo 5 per renderlo servo ubbidieate al fu0
Signore 5 e ben disciplinato; cioè instrutto del suo dovere, che & la fummilfione
alia ragioné. L! uso frequente della disciplina cominciò in Tolcana y'¢ si diffule
per tutta Italia,e si ereflero Compagnie de' Disciplinanti,o Batcati 1 aring' 1460,
Sigonius de Regno tralia. i bx ats i
SPROPOSIT ATO, Vino che non fa, ne dice éofa a ptoposite. =.
BV ACCIO. Ignorantaccio. Che si dice anche edfiraccio y Ci 4
dual, bue di panno, Vedi sopra C, 3. stan. 49. la voce arfafarco 1 Lz:
havevano diverse voci, che esprimevano queito stessojcome si vede in rid










BEG ec kf eke? oe tS eS eee

arr =

u
tty



Bee





SESTO CANTARE:; 2s

. Sc, 1, dove dice.Qui ubique unt, quifuere, quique futu-
ed Gesdepucblici-) fod, fang:; hard', Bdeani Buccones, Solus
'ante eo flultitia; & moribus indoetis, & Terent. in Heaut. 5.
haram rerum conuenit que sunt diea in flultum:, caudex,

plumbeus.
4 plea. B' quello, che i Latini dicono wltro, confultd, ovvero dedita
ioe non per errore, Oo inconsideratamente.
angolezzo. Il verbo scandolezzo portato dal Greco al Latino, e dal Lati-
 noanoi, ha significato d' inciampare, e d' adirarsi come vedemmo sopra C. 1.
- stan. 56. e se gli da anche il significato di quelle parole Si oculus tuus fandalizat te
te, come è nel presente luogo » che preso in significato attivo vuol dire: Se io vi
dé occasione di far errore: se io vi sono cagione d' inciampo; ff ribi offenfioni /um;
2 | afero y per esempio, fo credeva, che il tale fulfe huome da bene, mail fen-
pai, che ecli da a usura, mba feandolezrato, cio fatto mutare il concetto, che














 BRYNIR e@ labbri i fai. Brunire, parlandosi di materiali fodi come ferro,
ilo, oro 5 ec, vuol dire Dar il lustro, e però intende qui dar il lustro ai faffi co
labbri, baciandoli spefio, atto, che si fa da i Cristiani devoti per segno d'u-

y sopra C. 2. fan. 9. disse dar il lustro a' marmi co i ginocchi.
ACCLA Jenza polpe. Carne cattiva ame quando si compra Ia carne, che
sia con molto offo, si dice: Vi e poco del buono; e da questo dicendosi a un
-huomo o/sa senza carne s' intende tristo, ribaldo, o (cellerato.

CHI ti comprafse per lepre, havrebbe almeno tre quarti di volpe, Chi ti credeffes
semplice, troverebbe poi in te tre quarti almeno di maliziofo, o furbo. In La-

tino fidirebbe: Pro fimplci columba, afluta vulpes. In tutta questa Ottava narra
«| -Moltedi quelle azioni be fanno gl' Ipocriti, e Bacchettoni falfi.

jp. aS STANZA

i La che sono un! infano, eignaro ogni hora, Finché lo spirto sporti al foro fora,

Bb erche saper fupir non voglio, o vaglio, Dond' ti fa i peti,e puted oglio,e d'aclio,
oy 'al Duca,percht a' muri ei mora Accio ' accia fu? aspo doppo atdoppi
it Tofoin tefha si dia Jie meglioun maglio, La Parca,eil porco con la fhoppa Poppi:
@  Wiiritera, che e il fettimo Diavolo propone che si dia in sul capo a Baldone,
i €s'ammazzi. 1 Poeta lo fa parlare in bifticcio a imitazione del Pulci nel suo
f Morgante lib. 23. che dice. ° 3

La casa cosa parea bretta,¢ brutta
ah è V inca dal vento /a natta, e la notte y

eF

goo Stilla di flelle, ¢? a tecto era tutta,

BB. Mapnifrnte E fuina, e fuena di botto una borte.
BH Pere havea pure,¢ qualche frasta frutta,
io Del pane a pena ne deste a tai dotte



Poseia che pesci, e lasche prefe all' esca,



Y Lt Metta allorra alla frasca fu frefea.
MAGLIO, Dal Lat. malleus. Martello grande di legno per uso di battere i



.

ie



- Setchi alle botti, o per ammazzare i buoi, o per altri lavor: di legname, nei
squali richicggano

percufioni gagliarde, e gravi.
Rr z

SPOR-









316

SPORTARE. Avanzare in fuora, come avanzano le gronde de i tetti fuori
dclle muraglie delle case; donde Sporti quelle aggiunte che son fatte alle case
fuori del muro maeftro, e rette da' beccatelli, sorgozzoni, o colonne, (in Latino
Afeniana, che il Filandro sopra Vitruvio definisce prorette proiet pergula
dicate a Menio, Oc.) € qui vuol dire = feapps, o esca fuori lo io EBT

PETO. Quel romore che fa il vento stappando all'huomo dalle parti di basso.
Lat. peditus.

ASPO. E' un bastoncello con due traverse in croce contrapposte, e distanti
alquanto l'wna dail' altra, topra vi qualei raguna il filo/per ridurlo in'
ane dal' ane pare, nape pelyeapeuaaaeal Gnindolo onde Agenina

PARCHE, Le tee donne appellate Clra, wAcropo ye Licheft, e dere
quia nemsni parcunt, five quod parce,@ pene avare vicnm eribuant. La Gi
ilimava, che queste futicro Figliuole dell' Erebo., e deta Notte., se
Natura Deor, e secondo altri, che faflero Fighie di Demogorgone; €
figuratiero le tre cose necefiarie all' hnomo, cioè il nascere, ii vivere
re; dicendo che una di loro detta Cioto fila, cheé il nafeere, ta se ) detta
trope annaspa, che è il vivere, la terza detta Lachefi taghia il t ce il mo
re. Le chiamarono anche Nona, Decima, e Morte. 1. a Ee i

STANZA Cl. STANZA CHL —
Ben tu puxxs ai pazro ch' e um pexro y Lonon fo se Baton fornia, 0} iy
Disse Pluton, bestiaccia, per bifticcio, Perché #¢i vuol,
Perch' io per me non fo, ne raccaperro Famate i conti 5e conta,
Quel che tu voglia dir neltuocapriceio, Wel zerolho frat wnaye:
Ata non son Re, s'io non'te ne divezRo,
E perché tu non tent grattaticcio,
Mentre flima non fai delie bravate,
Quest altra volta le (aran pectiate. Tremande andranne come
STANZA Cll. STANZA CIV.
via seguite: Sui lo Scamonea Ola, dove fiam nos ( dice P
Si rizza, in vrfo tutto infauguinato, Eche Yi bite chtio
Perch' ei,ch'eun fastidiofo,appio havea Daro benvio fat a 4
Fatto a graffi con un, che "ha en allato, Si calle fiele iv cnderd it bade

Pero con la bifunta sua geornea, Guarda quel'thetu di barone
La qual traluce come Ciel frellato, E va piit'tespo ye col 'aki
Sich'ella un' Argo par fatto alla macchia, Sta ne i vermini-, e parla con gindiciy
Sinettayal Res' inchina,ecostgracchia; Che per min se ti privo dell' nfixis.
Plutone dopo haver riprefo il Tiritera, comanda, che 'dica Scamonea ottav
Diavolo;'il quale da anch'-egli un consiglio spropositato,¢ con parole eel
onde Piutone lo fgrida,minacciandolo di levarglivia degnira Senatoria, ©
non s' avvezza a parlare con termini oneiti,'¢ rilpettofi. Poe «
BISTICCIO, E: \a figura'che i Greci dicono Parecheff yed & quando si
due parole che hanno lo stesso', o poco differente suono-, diverso
come si vede nell' antecedente ottava 100. ene i due primi veri
101. Detto Bificcio quali Dificcio dal Latino greco Disti¢hnm; rela
che Biforto & fatto dal Lar, diPortus; Biffemto, dal Lat,


















£0

Oat FS OSes ten renzEesa














Be RE fee as










 SESTIO CANTARE;? 317
»Ci0e maltrattare,efimili,. Imperciocché i primi bifticci, de' quali ci
gli Efempi'yconfiftevano in Distici, o eel dire coppie di versi

lia stessa vocesla quale significava duc cose diverse,secondo che o piuilar-

b stretta, o intera, o'dimezzata fiyprofferiva. Fra Guittone d' Arezzo,

' Poeti'Antichi di. Mons, Allacci, tutta una Canzone va tefiendo

i diparole ed' quella che si trovaa carte 385..nelia Licenza,

qual Canzone dice cosis















ny yedo,
Sen' ake mido,

aera Edi, che prefofo,

iihies tex i912 cio vuol di tornar fo,
la in'primo Juogo vale ad banc ipfam hor am,siccome adeffo vale ad boc ipfurs
| secondo:luogo %d ¢/savuol dire ad ¢/sa mia donnaya les, 1 fait eda,
coll secondoymeido,L, me dedo, || primo fo vuol dir/ono verbo. I fecon-
-Ne'fonoefempi in Bindo Bonichi, ed in Francesco da Bzrberino.
raccapegzo.. Non fo'ridurre'a capo: Non rinucrgo: Non rinucngo:

vo: Non intendo. \

C70. Qui vuol dire opinione, o pensiero. Vedi sopra C. 1, st, 21.
'fon Re, Laicio d'etier Re, E' termine giuratorio che esprime Tanto
“€vero che iovho fatta, © farò la tal cosa., quanto è vero cheio sono quale io fo-
#0 Non (60 Padre ui Telemaco, cioè non sono. Viifie feio non ti feufto; Dit

f 4 Terfite presso a'Omero.

ui te ne dwezzo, \S' io non ti fo lasciar questo vizio,-0 questo tuo modo
-diteattate. E> il contrario d' avvezzare. Vengono da Vizio sens avvitiare pec
eallietare a'un vizio difuxiare per liberare da un vizio. E questi due verbi
attivi, che neutri hanoo sempre lo iteilo significato. Diciamo per esempio
'Phateeces I del tabasco', cickiessersi afiwefatto a pigtiarne.
tem gratcaritcio.,. Twaon fai Rina de i piccoligaftighi; Tu non temi
; enon cri le riprenfioni.. Nelle Raccolce de' Greci trovafiun certo
ico', che voltato.in 'Latino suona così:
sls (6a Tncus maxima nom timer Strepitus.
Egrattaticcio intendiamo grattatura, che leggicrmente offende la cute.
PECCIATE, Petcofie nella peccia, Caici nei ventre. Termine baffo, e più
toffo feherzofo. Peccia loiflefio, che pancia, se\bene della parte, che è dallo ito.



























Maco al none Peccia pare più verso lo stomaco, Pancia più verso il petti-
“Btione, Questa'é dal Latino pancices; inictini; quella forse dallo Spagnuolo pe-
Latino pettus, onde Rimpecciare.

| BISUNT A giornea.. Velte atiai wnta.. 'BE per giornea's' intende la sopravveRe
 Mdei soldati', che i Latini dicono Ch/amydem ye Lispigiia per veste d' aurorita,
 donde habbiamo un proverbio che dice.
AP PIBBIARS! la giornea,, Che Gignifica prefumersi molto di se medesimo. 11
'Bn,. 102, parlando di Didone dice:

Come






















pugs. -IMALMANTILED®

Came Diana allor che xscirne acacia ys
Lungo ? Exrota, o pure in Cinto (nales damipcabs
Fratutte U' altre la giornea s allaccia

E suol parer fra le sue Ninfe un fole
Il Forti, parlando della Prammatica delle donne al caps mibi 242.
parole da i libri pubblici di questa Città,dice: Wen porevane portare
© mantello o altro vestito sparato, ne maniche sparateso tagliate per il lunga
cia. Donde si deduce, che questa era yna sopravvelte, oizimarra aperta t
dinanzi, usata anche dagli huomini,di conto nelle case..Ma da noi hoggi-
glia per toga, o veste curiale, che chiamiamo Jucco 5 e nel. presente 1uogo |



dir questo, *

RALVCE, Traspare; E s' intende, che era piena di buchi, perel
giunge pare un' Argo fatto alla macchia, cio s' aflomiglia a un' Argo malfa
Argo fu quei pastore, che havea cento occhi., e fu lasciato.da Gi
dia d' lo figliuola d' Inaco convertita da Giove in vacca; ed a sen
miglia i buchi, che erano nella veste di Scamonea.« Plautoy se i
chiamé casa illuftre quella, per la quale per essere il tetto rotto 5 si vedeva il Cite
lo. Quel che voglia dire aipingere ala macchia, vedilo sopra C. 1. st. 69, doves
vedrai anche il significato di gracchiare. ew

PRAT IC A, Intendiamo Confulta, o Congreffo di Confultori dallo Spagnud-
lo Platica ragionamenio, discorso, donde Praticare um megezio vuol dir,
© maneggiare un negozio. Varchi St. Fior. lib. 14. Ragunafi la eraser
ro, che per esser la Città ferma, non faceva bisogno fare altra spefa. wa
volo credo, che intenda furbar la noffra pratica, cioè dar diflurbo a 2.3
nostra amica, perché haver xna pratica si dice quand' uno ha,o fitiene qualche» | a,
donna, o innamorata: e corrobora questa opinione il sapere, che Baldone noms Uy
flurbava il Consiglio de' Diavoli, ne Ji loro congreffi, o pratiche, ma

=

Martinazza con aflediar Malmantile, “ae AG
L! HO nei zero. L' ho nel forame: Non lo stimo. Zero è la figura tonda: 'ay
Abbaco detta forse da Giro, la quale forma le decine, e per similitudine s* inten Pie
de il torame, e ci serviamo di questa parola per coprire il detto sporcost' hols | hi
c..., usatissimo fra la gente bafia in questo significato di disprezzo; equitormas | %
bene, perché dice con rasta la sua aritmetica, cioè abbaco, io/'bo nel zero, che® | ity
figura di aritmetica. 4

BACCHIO, Baltone, o pertica dal Latino baculus. E felleticare qui intendes
perquotere; e parla ironico, perché le bastonate sono contrarie del folletico,

NON fara in gramatica, Non fara difhcile, e che ci voglia grande stadio.
Gramatica preflo gli antichi volea dire /ingua Latina, come quella', per Ji
la quale ci bisognava lo studio della gramauca. E perciò la Greca anticajoyvero I
Ellinica, e litterale,.che si conferna folamence nelie (crivure; a differenzadelld |\,¢
volgare, e moderna, la quale oggi si parla, corrotta da quell' antica, e fichiama =
Komeca, cioè Greca de' sempi baffi, ne' quali i Greci non più tennero il lor antico
nome di Aellines, ma per gl Imperatori Romani, che in Oriente avevan trasfe-
rito ! imperio Romei comincjaronfi a nominare; quella Greca antica, dico, tr-
vasi chiamata gramatica greca; perché gli Odierni Greci per apprenderla my

sogno

2

















SESTO CANTARE: 319
fognd di gramatica, si come noi per imparare la Latina. Nel principio dell' an-
tico ss enero ee delle vite di Plutarco si legge. Qui comincia la
di Plutarco, la quale fne traslatata di gramatica greca in voleare greco in Rodi,
B perché la Grammatica'é cosa spinofa, e difficile; per questo il dichiarare,¢
re l'intelligenza di qualche fatto, o questione oscura, e imbrogliata di-
sgramaticare
RACHE piene, Per la'paura si movera loro il ventre, e s' empicranno le
» Vedi sopra C. 1. st. 43.
sT1CO., Vno che difiicilmente ha il benefizio del corpo.
ME paralitica, Cioè tutta tremante come sono | paralitici.
VE sia noi? Dove credi tu d' ellere ? Termine che significa Porta rispetto
jerfone sed al nega dovetu fei, Alcflandro sentendosi recitare da uno, che
distesa la Storia de' (uoi fatti, una narrazione lontana dal vero; disse allo
|5, Evdove eramo noi allora? quali dicefle: Che non ti ricordi; che io v' era
5? Altre volte significa: Che non hai gindizio? per elempio T dai cexto
tale, che non ha haver 50,, dove fiam noi? cioè dove fiam noi col cerucllo?
E si? Termine usato per indurre timore, ed ha del giuratorio; E che si,
? quali dica: Giuro che si; ch' io ti zombero, se tu nox parli meglio. Si
per fare flar a segno i fanciulli, E che si, che io vengo cofid, e vi sferzo,
Sidice anche, Vale o giuochiamo, o stiamo a vedere, che io visferze ? Vin Poeta
moderno se-ne servi per giochiamo, dicendo:,
'ahscas E che si, padron mio, ch'io m' indovine
SD ei Del voffro andar girando la cagione ?
SCORRETT ACCIO. Huomo scorretto diciamo colui, che senza rispetto al-
Gund dice parole'sporche' ed oscene, ed indecenti in ogni laogo.
ZOMBARE. Perquotere. Bil Latino Verberare, Dal faono. Così Typro de?
 Greci sche vuol dire verbero, e verbo fatto dal suono; onde ne nacque Typanon,
=. we yi Tamburo; dal quale abbiam fatto noi Tamburare, e T ambuffare;
i ympanum, Zombare. Appresso i Greci bombes e il rombo, o romore delle
Pappresso i Latini bvmexs è il suono che fa il corno. Appresso di noi Bom-
berda ¢\detta dal gran rimbombo neilo (pararsi;; € così tutte queste lingue si (ono
accordate 5 contraffacendo il suono medesimo, che da cose concave ulcendo, ¢
rigitando 5 e:ampliandosi perwene all' orecchio.
&/MBOMBO. Rifaonamento, l'Eco, cio' quel suono che resta alquanto
ug romore 5 e maffime ne i luoghi cavernofi. Dante Inf..C, 16,
Già era il loco, ove s' udia il rimbombo
by i Dell' acqua che cadea nel? altro giro
ac hi ' Simil a quel che  arnie fanno rombo,
A VA col calzar del piombe, C: ina adagio; e fd nelle tue op
4 diy Governati con prudenza, Lat. A¢arura ler, Dante Par, C. 13,
yg 2 3 E questo ti sia sempre pivmbo a* piedi
4 = Per farti muover lento come buom laffo y
o
i








RADDA Ed al si, ed al mo, che tu non vedi

















320 MALIMAN TIE BS @
(STANZA CV.: Z
S* alza Scorpione alloraye wiere da ef
ate it Corno orvibile, proposhoy
Che gli eserciti dive in fuga ha, melo.
Conforme ferive, ¢,accerta, 2 Arioffo.
Si rallegra Pluton,e dice; Adefso
Naw ci fara: dal Cancelliere opposhes
Perché ci calza bene, € certo questa
Cosa del corno a mevarper Ia testa.
STA
Vuoi forse darci qualche eceezsone,?
Stiamo in decretis; diy peta veffito;
Va ben, risponde il Sere, ch ex proponey
Cosa, che non deprava ordines9 Ti0% ognun.
Fatta che hebbe Plutone la.bravata a Scamonea,si riazo Scorpione
volo, e propole, che si pigliaiie. i| Cormo.d: Afiolfo,, il che piacque a.
per questo si volioal Cancelliece domandandoli,se ci havewa difhicul
provo; Onde Plutone ordino, che si.faceffe il partito.. F
SOGGHIGNARE.. Mottrare, o far segno di ridere quali dafu
ere per segno di. dif




bene in:faa forza & il latino fubridere » ed e un certo,
zo, o di poca stima, che altri faccia di, qualcofa 5. e si, chiamayrifo
cioè non puro, non vero; ma, fate.;

JO non son qui per candeliiere. 1o.non son qui solamente per far
devo dire ancor'io i mio. parere, quando occorra.

DOTTOR de' mei firvali. Termine di disprezzo, e vuol dire
Vedifopra C, 4, tt 10. Aewe

PET O-vestite « Che cosa sia peto, vedemmo nell' ottava roo, d
quando.il vento esce dalle parti da baflo accompagnato con qualcofa altro,
ce peto veltito. Eda questo il Lettore può comprendere quel che significhi,,
SONATE un doppio, Quand' altri dopo molte cose mal fatte ne fauna bent
dal medesimo solita farsi di rado., o vero dopo, che uno habbia terminata
faccenda con grande stenta, ed in molto tempo, diciamo.: Sonate wm ¢i0
tutte le campane per l'allegrezza di questa cosa infolita » © della rerminagions
di questa faccenda, che si pensava non -haveffe a esser terminata may) |

F-AR il partite, Fas.loferutinio, che noi volgarmente diciamo far lo /gxitine}




© fquittinare.
STANZA CVIIIL, STANZA CIX.

Vanno le fave attorno, edi lupini-, Vauno i danyelti ognun dalla sua banehy
E sentefi fiuonato,¢ fuor di chiave Ma perché ne ricevan: Be
Alle panche gridar: Tavolaccsné 5 * Che pis neffuna ardy a it.Re comand
Raccogliete pel numero,¢ le fave Se mon vuolyche a pier popolo si sferti
Pigliate in man; che questi cittadini, Di nuovo attorno s boffoli si manda
Che in simil Luogo far dourianfulgrave Da vincersi il partite pe' due rere
Rendano( il capo havendo pien ds baie ) E cercate alla fin tutte le panchty
Male i partivi, e mangian le civaie. Fu vinto non ostante cana:

te

.

RPS R RE SRA HE GRE ERE REESE






















Bess ~~






| j donzelli vanno raccogliendo i vor!
ti in contrario fu vinto, che si

lone da Malmantile. E qui ters
~ Vedil Ariofo nei suo Orlando furiofo., che lo finge uns

i fyono fugava la gente.
fave arr edi Inpini. E' costuine in Firefze, come era anche
di fare i partiti, o (quittini con fave~, e Jupini; e pero havendo il Poe-
uto, che nel ConGglio grande di Firenze chiamato il Consiglio dei Dugen-
yhel quale inte ono centinaia, e cehtinaia di persone ( come in questo
Consiglio de' Diavoli e necessario, che intervenissero sopra 300, Demonj, mentre
to voti non impedivano il yincere il partito) i Tavolaccint, Donzelli van-
F endo le fave, ed i lupini a coloro, che devon rendere i) partico, fas
'il medesimo costume nel presente consiglio de' Diavoli, dove dice che si fen-
idare spuonato, e fuor di chiave, cioè in voce, che non intuona, e non accorda,
) procede y perché efiendo più d'uno, ed in diverse parti della stanza a,
impotfibile che s' accordino nel tuono, come anche perché dette yoci
ite'fra tanta gente, che bisbiglia., il che le rend ottule, ed offulcate.
YOLACCINO, Servo,0.Donzello di Magiftrato; così detto secondo al-
r abellio detto fopta in questo C. st. 74., ma io credo, che i Tavolaccini,
che sono un 'numero determinato, e differenti dagli altri Donzelli, sieno quelli
che al rempo della Repubbiicha stavano sempre in palazzo, e servivano alla ta-
] vola de' $5. ciascuno il fu', e due n' haveva il Gonfalonicre, e si dicevano Ta-
# — volaccini dal servire alle Tavole; e che habbiano conferuato il nome, si come si
conferua ancora J" usizio, essendo costoro obbligati a andare a servire alle tavole
eo in palagzo del Serenifs. G, Duca in occasione di Forefticri, o di Spolalizzj, ec.
ma per altro aprono ogni mattina, e ferrano ogni (era le Porte della Città. °
» RACCOG LIE le fave per il numero. A fine di saper con facilica, quanti fieao
coloro, che rendono i} voto, il Tavolaccino pigia in mano'da ciascuno una fa-
va, © poi si contano., ed indicano il numero de ivotanu, equelto si dice
c i numero, E pigliano le fave in mano, enon nel boflolo, per aflicu-
-rarsi che non vi sia chi ne metta pi d'una, ed alteri il numero,
STAR sul grave. Tener il decoro, la gravita. Star favio,
HA il capo pien di baie. Sempre vuole icherzare.
RENDER jl parrito, E' quel dare, o mecter la fava, o lupino nel boflolo, che
si dice: dare il voto.
4 PIEN popolo, In prefenza, ed a vilta di tutto il popolo.
“BOSSOLO. Quel vaso, nel quale si metiono i voti dagli Ateniefi detto Camus,

























4 Vedi sopra C, 1. st. 37.

ams

es

if FINE DEL SESTO CANTARE.

J i

si

h ot

; pw

i Ss \ SET-





























SUSE le a
oe

a

®

= =

ARGOMENTO.
Paride dop' haver molto bevuto
Entra,' andar' al campo, in frenefia, ae
E come il sonno havea pel ber perduto,;
3 Perde nel gir di notte anche la via: te
Cade in un fofso, onde a donargli aiuto ath:
. Corron le Fate,¢ gli nfan cortesia; eee
Vien condotto sn un' antro, e per diporte e
La froria gli è narrata di Magorto.:

Sepepapeapeaye pases

joinaatuaiaa
?

Beis -gFf essete fer

STANZA L STANZAII
V Ino tempera te disse Catone, Perché se quel s marrage ne einuecebidy Me
Perché si dee berue a modo,eaverso, Ed e burlato il tempo di [un vithy =| ii

E non come cosa qualche trincone, Almen Sent ilfapor di quel cb'el

Che, giorno, e notte sempre fa un verso; E tien la faccia rossa ye colorita

Ond! et si quoce,e percié ei va aGirone, Buvlar anche si fa chi va alia.

La fauola divien dell' universo, E infacea senza gusto a

E vede poi morendo in tempo breve Che'jo tien sepre bol/oyein man del

Chie ver sche chi più beve mance beve. Aquall we a Pe morir di tified.
lL






Bre aa Ee

STANZA I STANZAIV,
S? il troppo vino fa, che 'hnom foggiace Pero sia chi si mae egli, è um dappoce
etal' error di tanto pregiudizio; Chi imbotta al sayy come gli
Chi non ne beve,e quedo,a cui no piace, 5S! avvegzi a ber del vinoa peeve,
Aquesto cite dunque ha un gra gindizio; Chiei fa che Vacqua fa marcire si pally









Anzi che nd, sia detto con Sua pace, Aa com' io dico si vnol berne pipe;
Per c ogni eftremo finalmente e virio, Basta ogni. volta cingue, 6 se

E se di biasmo e degno Punose taltro Perch' egli è poi nocivo il ersncar tame,
Questo ha il vataggio al mio parer stz'altro, Com' udirere adeffo in queste C
Volendo il Poeta narrare in questo Canto l'accidente occorso a Paride
ni, per haver troppo bevuto, s' introduce col riflettere, che siccome e male
molto vino, così che sia anche male il bere solamente acqua; e 2 che
dovendosi eleggere uno dei due mali, sia meglio eleggere quello del ber vi,
ma pero regolatamente, et MO-










SETTIMO CANTARE: 323
AMODO,¢ a verso. Regolatamente, E' il latino vulgato: modis, o formis
dmipblein co: i
 TRINCONE.. Vno che beva afflai. Da Trinchen Tedesco bere, tirar git.
i sopra C. 1, st. 6. Si dice anche pecchiare nella presente Ottava teraa, quasi



'facciare il vino come fanno le pecchie, ( cioè l' api che fanno il miele, così der-
; ee me le quali fueciano il dolce da i fiori, ed i vini bianchi geac-
'tolit edaldetto verbo pecchiare si dice pecchione a uno, che beve assai; e pecchione
ichiama ua' ape faluatica,e maggiore dell'altre, che fuccia il miele prodotto dail'

api da' Latini chiamato fucus.. Virg, gnauum fucos pecus a rabepibas arcent,
dice ciencare nella presente Qttava quarta. Vedi il Landino esposizione a Di.
nf, C, 9, alla parola cionca nel verso Che fol per pena ha la /peranza cionca, do-
dice, che cionco è parola Lombarda, e significa moxxo, ma cioncarc in Fiorentino
mnifica difordinatamente bere; Si che questi tre verbi trincare, pecchiare, e cioncare
fanno lo flefio significato, e se bene hanno del foreftiero, wuttavia sono usati in



os

na

ot.
"SEMPRE fa un'verso. Sempre fa la medesima cosa. Diciamo Ver/o il canto
pce “4 Verso del rufignuolo, Verso del fringucilo. E da tal verso vienes
trato.

'WA AsGirone. Huomo, che gira; intendiamo pazzo. E però servendoci della
voce Girone, che e un Villaggio vicino a Firenze, copertamente intendiamo
uno che fa delle pazzie, come s' intende nel presente luogo.

DIVIEN la favola dell universo, E' burlato da wtti. 4 ore est omni populo, It
Lalli Ha, Tr. C. 4. 2. 78.

Son fatta ime la favola del mondo

I) Pett, Ata ben veggio or, si come al popol tutto Favela fui gran tempo. Tibullo lib,
1, Ne turpis fabula iam. Nella scrittura. Et fattus /um illis in parabolam,

CHI più beve manco beve, Cioè, chi troppo beve s' ammaia,e muore,e così vive
poco, € per confeguenza beve manco, cive dura a bere manco tempo di colui,
che beve poco. Marz, lib. 6. Jmmodicis brevis est atas, © rara fenettus, che das
noi poi si dice in proverbio Poco cs vive chi troppo sparecchia. A similitudine di que-
flo si dice: Chi piie fudia, manco findia,

OGNTeftremo è vizio. Ogni eftremo e male. Ogni troppo è troppo. Questa
sentenza wfiamo dirla // troppo, e il poco Guasta id gineco, Al che pare, che facciano
molto a proposito i seguenti versi di Orazio.

Eft modus sn rebus, funt certi denique fines,
nos ultra, citraque nequit confiftere rectum,
E Terenzio mettendo in Latino una sentenza d' un favio della Grecia disse; We

ee a es ee ee

quid nimis,

SENZ' altro, Assolutamente; senz' alcun dubbio. Latino fant, procul dubio

VA ala secchia, Beve acqua. Secchia diciamo quel valo, col quale si cavas
— da i pozzi dal Latino firwla. Vedi sopra C. 5. st. 10,
: ACCA, Per Gmilitudine diciamo facco al ventre dell' huomo;quindi dnfacs
tare vuol dir Mandar git nel ventre. Pulci Morg. C. 19. st. 137.
sot E mangia, e beve, e infacca per due verri
Peril contratio/acar in iipagubelo è trarre, —— fuori,

8 2

i ee

scl








324 MALMANTILE: >

S¢lelT A. Che non ha fapore alcuno. Deh kagions ia, SR
BOLSO. Vedi — C. 3. st. 53. Graffo non naturale, con di direlpi-
ro. Cavallo bolfo i Franzefi dicono pou/if dal pullare, cioè | a f
la Jena affannata. Lucano lib. 4. Pettora rauca gerunt, qua creber anbelitus urge
Ex defetta gravis longé trabit ila pulfus. r aivye
LN man del Fifico, Col medico sempre attorno; cioè fempreinfermo, ©
CAL imborta al porzo. Chi beve sempre acqua. E' lo stello che dafaceare
to sopra. o aged
ANIM ALE. Intende animale irrazionale. Se bene la voce animale & generi-
ca,¢ comprende sotto di se anche l'huomo, noi ce ne serviamo per speciale, in~
tendendo solamente le bestie, fiche dicendosi a un' huomo T# fei wn! animale, in.
tendiamo Tx fei una bestia; Vat srragionevole, ae
Ss? AVEZZI, 8 afuefaccia., Vedi sopra C, 6, st, 101. Nth > Aa
FA marcirei pali, Vuol dire:il vino si guafla annacquandolo, quasi dica;
infradiciare i pali, che reggono le viti, che producono il vino; o se fara
infradiciare il vino, che nasce dalle viti., che sono più deboli de i pali, mentre
son da essi foftenute, Dichiamo anche per biafimare l'ulo dell' acqua: 2? acqua
rovina è ponti: quasi s' abbia a intendere + O pepfate, se non royinera gli foma-
chi deglt huomini, che sono più deboli ! > KE
BOCC ALE. E una milura capace della meta d' un fiasco Fiorentino, Dice
cinque o fei boccali per scherzo,sapendo bene, che ogni maggiore bevitore non







bevera mai Gi gran quantità in una volta, van
STANZA V. STANZA. Vide
Omai ferra gli ordinghi, e le ciabatte E Paride, ¢' anch'egii si ritrova
Chiunque lavora,e vive in sul travaglio, A corpo voto in quelle e hie}!

E difilato a cena se la batte

A cafayo dove piit gli viene il taglio.
Chi dal compagno aufo il dente sbatte,
Tanti ne vaatavernach'é unbarbaglio,
Parte alla bufea,e infin,pur che firoda,
Per tutto e buona fhanza, on'altri goda,

D Amor chiarito figliod' una Lous,
Che sualiziar gli ha fatto le bufecchie y
Dice al villan:Va a coprarmi delves
Ecco fei gink y sonne ben parecelne 5
Piglia del pane ye sopra tutto areca
Buon vino fai\ non. qualche cerboueca,

STANZA VII, Abe

Eset' avanza poi qualche quattrino y

Spendilo in cacio, non mi portar reffor
Meller fine, rispose tl Conradino,
lo torva, 8°10. ne trove, ancor corefto.

E partendo gli ride? occhioliney
Sperando haver a far un po dagrefe;
Aa, facendo i snoi conts per la via,
S' accorgeche e' non v' e da far calia.



Deferive afai vagamente il venir della notte; fu la quale ora Paride affalito
dalla fame comanda a Mco suo contadino, che vada a comprar roba da-ma
giare,¢ da bere, e per tale effetto gli da fei giuli, con ordine che gli spendas

i 5G

tuttl. e by. soi 92
ORDINGHT, Intende ogni sorta d' arneGi, ingegni, machines firemsentiod'
Javorare. Diciamo anche Ordigni; anzi gli anuchi non difero algrimentis

CIABATTE,, Vuol dir propriamente scarpe vecchie, e quelle scarpe all! Ap *

stolica, che usano i frati fealzi, ma s' intende anche i frammento di ma-
tcriali di coloro che lavorano', e per ogni sorta di mafieriziuole veechies'¢ 6
fumate, che i Latiat diconoscrara., Z WWE

Ay

ZeRm =

— t

Bee essezeane28=

zs


Fae BF ee SSR est avEeESeF®.












SETTIMO CANTARE: 325
 VIVE in faultravaglio, Latino manribus viitum quaritat, Campa delle /uabpaccia.

'Travagliare in lingua Francese vuol dir lavorarc, ed in Firenze pure è usato in.
6 denbo diegndns: cosa ben travagiiata in vece di ben ieleaans edi qui si di-
in vece di viver col lavoro, o con le sue fatiche, cioè di quel che si
alavorare. Petr. C, 3.
es ynque animale alberga in Terra,
Se non se alquanti.c' hanno in odio il Sole,
: Tempo da travagliare è,quantoe il giorno:
Ma poi che 'l Ciel accende le (ue Hele,
Ree rteit. 9 * Qual tornaa casa equal s' annida in felua,
ve % portent Per aver posa almen infin all' alia.
es ben per altro travagiiare yuo! dire esser' anguftiato da infermita, o da altro.

  DIFILATO. A diriuura: Latino re%a. Con preftezza, e senza fermarsi.
ze serve anche sotto in questo C. st, 63. Varchi Stor. Fior, lib. 9.

non prima giunto a Firenze che andandofene difilato, fenra pur cavarsi git

i SE la bate..Se ne va via. E' termine assai usato fra la gente bafla per espri-
ib yia, o partirsi in fretta, ed ha del furbesco batsere /a caleofa, cioè
: utter la trade, andar via, camminare, donde /frada battuta vuol dire strada.,
a 2 camminata, o strada di paflo. Latino via trita, Lucrezio Avia Pie-
i' Tidum peragro loca y nullins ante Trita fol, 1\ Petcarcha disse: Ogni segnato calle,
—- Prove contrario alla tranquilla vita.
DOVE gli viene il tagio. Dove gli torna più comodo. Vedi sopra C. 2. st. 48.
' ', Senza. spendere. E' detto plebea... Si scrivono da i Magiftrati di Si.
 renee: di,commiflioni ai Ministri forenfi, le quali da coloro, che le chieg-
g0n0, ele presentang; si pagano.a i Magiftrati, che le fanno, ed a i Ministri,
I¢ le gicevono; e quando non sono chielte, ma son fatte, e mandate per pro.
'Prio interefle di quel Magi(trato, che le fa,non vi e spefa alcuna, e pero afiaché
tall lettere, le quali non si pagano,si potiano distinguer da quelle, che si pagano,
fetivono nella sopra(critta ¢x ofitio, ma l'abbreviano scrivendo ex Vifo, ed i ta-
volaccini, o donzelli, che le confegnano non leggono se non ex fo, e distin-
Buono queste due specie di lettere, dando a quelle, che si pagano il nome di let-
tere col diritto, cio' con la dovuta.spefa, ¢d all' altre il nome.dell' Hf, cioè fen-
za spela + Edi qui e nato questo detto a Vo, che vuol dir senza spela, e serve in
gai uccafione.

ckseee=

*~

A SBATTE if dente. Cioè mangia. 3

J E un barbaglio. Son tanti, che fanno abbagliare; Non se ne pué raccorre il

f conto senza sbagliare; o abbarbagliarsi., cioè errare; dal Parpaglione, che disse-
70 gli antichi alla Provenzale;cioè dal Latino papilio;farfalla;di cui e noto lerrare

® — intorno al jume.

è CILLA busca, Cercando sua ventura.. Bufeare.. Vuol dir Acquiftare, otte-

iy ere, puadagnare.. E dalla Spagnuola ha/car yenuta a noi questa voce infiemes

) SON molte altre negli nltimi cempi.

i Si reda, Simangi. Sc bene rodere si dice de' topi,de' tarli, e simili. Per tutto

y Pbuena anza on' alsri gode, Voi bonum, ibe patria. Dove si tla bene, quello è

' 42)" buoa




—E

326 ' MALMANTILE *©
buon pacfe; E per ogni patfe, e buona Stanza, Disse come in proverbio il Pe.

trarca. ' Sp heist tity
CAT APECCHIE, (ntendiamo luoghi orridi, inculti', € dil. Mattio
Franzefi in lode delle gotte: Alor per uscir di queffe carapecchie, N
do che pecchra & fatto da apes, apecala, o apicu/a così verifimilmente carapere
puo dedursi da apex apichlus, che vuol dire piccola fommitdy e cara preposizione —
Greca, la quale dice un certo ordine', o € aggiunta pet maggior forza, come si
vede nelle parole, Carafalco, Cataictto y Caruno', che dissero gli antichi per =
Scheduno,¢ simili. > te
CALARITO. Aggiuftato Vedi sopra C. 1. stan. 1.Vuol dir che Amore l'ha
= accomodato, perché s' era pieno di mal di chiafio, come si disse one
in. 11. Oe ie
LOVA. Lorda; Poltrona. E' parola d'ingiuria a tina donna. E*voce fira-
niera; e yuo) dir Lupa; che similmeate gli Spagnuoli dicono soba; e si
maeretrice. Gio. Vill, lib. 1. cap. 25. parlando di Romulo'; e Remo allevati da
una Lupa dice: Questa Laurenza era bella, e di suo corpo guadagnava come
ce,€ pero dai vicini era chiamata Lupa; onde si dice furono nutricati da lupa 3 il che
cavo egli da Livio lib. 1. /une qui Laurentiam valgato corpore lupam voratam inter
Stores putent: inde locum fabula, & miraculo datum, +:
SVALIG/ ARE. Cavar della valigia. Qui intende; gli ha fatto'confamarei
denari, perché ha/ecche se bene si dicono i ventricini del porco Boce. git
Noy. 10. Dove le femmine vanno in xoccoli [u pe i monti rivestendos pores delle lor bu
Jfecchie medesime noi le pigliamo per tasche, o borfe, nelle quali i tengono ida
nari. E /wasgiare propriamencte intendiamo, quando i Jadri digtrada rubano @>
uno tutto quello, che egli ha addosso; e lo pigliamo per finonimo di /accheggiare;
'PARECC HIE, Numero indeterminato che esprime, Molti; dal Lat.
que, secondo alcuni: Volgarizzamento di Palladio manoferitto; Nel mele di
Marzo al cap. de ficu. Si metta sotto alle barbe parecchie pietre, *
CEKSONEC.A, Vino fradicio. L' Accademico Fiorentino incerto 5 cos! n0-
minato in una Raccolta di Rime piacevoli, che dicemmo altroye essere il Bare
chielio, descrivendo un cattivo vino dice, Te et





Staccio non pafferebbe ne framigna
Tiant' è morchiato,¢ con la feccia miffo;
Sciroppo mi par ber, ma non di vigna; Ly
Chi ne beve non ghigna, ej
Ch' egh: è ciprigno, e cerboneca fina; ' b
Chindendo gli occhi, mi par medicina,

Brunetto Latini nel suo Pataffio disse Cerbonea.
Wel ver quest' e pur nuova Cerbones Hem
Forse si dovrebbe dir cerconeca, derivando questa voce da cercone che vuol dit
Fea ease at si dice cercone dal circolare, che fa il vino quando da la volta s §
fig ¥ ac SAM








" Gulbinges verso;
Hciibiekh Rife, & argutis, quiddam promifit ocellis,

agreffo. Avanzare; ma intende d' avanzo illecito, come farebbe, quan-

acomprare roba, dice havere (pelo più di quello, che ha (pelo,

quell' avanzo. Vien da i cantadini, che per rubare al padrone piglia~

f ey va non matura, ( che si chiama agreffo ) e ne fanno fugo, e lo veadono.

j

s

SETTIMO

CANTARE: 327

QUESSER fine, Vuol dir Meffer si, Ma dice Meffer fine, perché fa parlare a

jun contadino: noffri fic rire toquuntur.

t occhioline. Vuol dir si rallegra. Ul rider dell' occhio forse accennd





le cravaglic de la vita,



\ lo termine ha lo stesso significato anche in Napoli, come si cava da lo Cun-
to deli Cunti di Gianalefio Abbattutis gior. 1. Cunto 8. dove dice; A¢oPrannole
kefrifolesco' li quale maritattero turte L autre fizlic, restannole pure agrefta pe' gliottere

IN v' e da far calia. Non y' è da far avanzi. Calta si dicono queirimafu-
a hares argeato, che nel lavorarlo cadono, e si dicono calia quali calo
a ?

rz o dell' argento, che ridotto poi in proverbio esprime ogni sorta di pic-
it Solo avanzo..
o STANZA VIIL STANZAIX.



Perch'egl e tardi,, ed ha voglia di cena,
Poi c ogni cosa ha bell' e preparato,

st v4;il pane,e ilcacto,es! vin rocac-
ep fatto un guazcabuglio nella [porta, Si frrugge, e si consuma per la pena 5
un » Le quattro lire slazzera ye si lpaccia. Che ti non torna il meffo,ne il mandate;
Lialerol asperta agloria,e insis la porta Ma quand' ei vedde con la sporta piena

. eee seh av ognor s! affaccta, Giunger al fine il [uo gatto frugaso:
of EB per anticipare, il fuoco accende, O ringraziato, dice, sia Minoffe,
ae — Lavai bicchicri,e fa L altre faccende, Ch' una volta le furon buane mofe.
ie | bast S.T.AN.Z.AoX,

Chiappa le, robe, e mentre ch' ¢i balecca Sbhocconcellandointanto,il fiasco shocca,
at In quocer t vova,e tl cacioch' e fupendo; Econ due man alzatolo bevendo,
ih Sente venirsi 2 acquolina in bocca y Dice al villan, che nominate è Meso:

E far lagola come un faliscendo,

Hlorsit ti fo briccone, addio, io beo.

-M.Comtadino mandato da Paride a provveder la roba, andò all' Oite per fbri-
arli, compro il tutto. Paride in canto stava aspettandolo con grande anfieti;
¢ fubico. giunto, egli mefie a quocer l'uova,¢ il cacio, e in ranto viato dal' im-
-pazienza, e dalla fame comincio a mangiar del pane, ed a bere.
» PER la pi corta, Vuol dir per la rada più corta; ma qui intendi per sbri-
garsi più presto. -
.» PROCACCIA. Provvede. Vuol propriamente dire cercar di trovare una co-
fa, e trovarla; Lat. per/equi & affequi, e(primendosi con questo folo verbo pro-
eacciare la diligenza, che s' usa in cercare, e andare a caccia d' una cosa, e las
fortuna, che s' ha di trovare quel che si cerca; onde poi molti dicono: buon pro-
| €aecino uno che s' ingegna per ogni maniera di guadagnare.
|. GY AZZ ABVGLIO. Mescolanza, mescuglio, 11 Casa acl suo Capitolo del
Martello di amore dice;

Nox











habe Ve > pay








328 MALMANTILE! ©
Non eva donna rica yo poverina yo)
Si facea d! ogni cosa un guarrabuglio —
Ogni fhanza eva camera, e cucina, |
Mattio Franzefi nel (uo viaggio di Venezia dice:
Ear a una tavolata allegra cera, «
. Edi var} discorsi un guazxabuglio >
Il Lasca Nov, 10, Versarono aceto, vino', olio, fale, e farina, @ fecero un euat
lio il macgior del mondo, Dal che si cava y che questa voce esprime mescolanza.
di cose maceriali, ed anche di non materiali; Voce composta di Guazzare', ch
 dibattere cosa liquida, e di Boelire: quali da una Ricetta che dica »Guarzs,e
Bolli; fattone Guarzabuglio.: s
LIRA, E' una moneta Fiorentina, che vale un giulio e mezzo, detto anche
Cofime, perché il nostro G, Duca Cofimo l'inventd, e fa il primo', che' bat
in Firenze questa moneta.: 33 Sepia
SLAZZERA, Cava, conta, mette fuora', fa venir fyora'a forza', E*
furbesca, se bene assai usata. gree #
S1spaccia, Sisbriga: Si spedisce.;
L' ASPETT Aa gloria* L' a(petta con gran desiderio, con pazienza efrema,
Si dice anche a/pettare a bocea aperta.. Larus bians. wine
HA bell' e preparato, Ha di già mef' all' ordine. Vedi sopra C. 3... 14 =
NON torna ne il Meffo, ne it Mandato, Non torna lui, e non manda alcunoa
dir quel che sia di Jui. Diciamo anche 4 ho mandate il.corno, dal coruo, che man-
dd Noe fuori dell' arca, il quale 'non tornd mai. pee
GATTO frugato, Così son chiamati per ischerzo da i ragazzi i contadini.
Carus in Latino è cauto, aftuto; € con queste nome chiamafi anche il Gatto anl-
male.notas il\quale.quando.¢ fato frugato con pertiche, o con bastoni, non fas
altro, che volgersi spaurito, € che guatare; onde vogliono alcuni, che abbia il
nome. Così i) contadino, quando scende alla Città, Dante Purg. 26,) ©
Non altriments flupido si turba o }
Lo montanaro, e rimirando ammuta, ae
Quando Hx 9 e faluatico s inurba, ANH
VINA volta furon buone mie, Vana volta ci tornd, Questo detto wfatissimo in
ucfto significato, vien da coloro, che stando a veder correre al palio per lo grat
desiderio, che hanno di vedere arrivare i cavalli, spesso gridano; Eccagl se bea
veramente non sono; ma pure al fine vengono, ed allora dicono, Queffe Jie iy.
buone moffe. 4 che pafiato in proverbio; significa a terminazione di ¢
¢vento, o negozio. ite Ty
St balocca. Si trattiene. Si dice anche: Par' a bada, o badaluccare; Bi YOO Wl

usata per ibambini. Vedi sopra ©. 6. st. 32. he i, &

} = 2 ak



SPP eee SF



rescezrr=



STVPENDO. Buonissimo. Vedi sopra C. 6. &. 35. Cosa maravigliola, hy
perfetta, che induce stupore, ae
VENIR l'acquolsna in bocca. Si sente consumar dal? appetito, © per te



soprabbonda la faliva in bocca, la qual faliva e caula che /e gola gli fac bi
Saliscendo, perché il gorgozzule gli va ingid, e insu per inghiottir quell' umid by
E faliscendo & una strilcia di ferro, che s' adatta a lerrar le porte, apeaaia liy













'di pane,¢ mangia.

01, Onofrio; ed altri infiniti.
Tl fobriccone. Ti

'quidice Briccone per brindift,
 STANZA XI.

Così per celia cominciando a bere,
d un forfo,e dagliens il secondo;
Fesigche dal vedere, e non vedere y

Ei ditde'al vino totalmente fondo;
tayola di poi meffo 4 sedere,
| Lafeiato it. ixfee voto sopra il tondo,
Viltoffi a dieci pan da Adeo provvsffi y
Eis th momento fece repuliffs.
STANZA XII.
4 i pan dotto,e uinginlio di formaggio
» Non glittoccaron l'ngola,e s'inghiotte
Due par diferque d'uova,e da vataggio,
| Potdice; Meo spilla quella botte,
Chet'hai per Popre,e dami il vino afaggio,
To vib feafera anch' io far le mie lotte,
| Ben th'io sia bere, fis ripieno,e /uentri,
Percht mi par,ch' una latcata e entri,
STANZA XIII.
URufticoche dar del suo non usa:
- Non saper, dice, dove sia il fucchiella,
~ Che per casa non v* & fhoppa ne fufa,
E che quel non e vin,ma acquerello.
Civuol, risponde Paride,altra fenfa,
By itty » di canna fa un cannello,
-E, in fa ha borte posso a capo chino,
~ Con est, pel cocchiume fuccia il vino.
















SASS Ek









eee SE Ae



SETTIMO CANTARE:

con'alzarla, ed abbaffarla. In questo significato'diciamo ancora:
I » vedi sopra C, 5. stan. 62.

VCELLANDO. Diciamo sbocconceilare,
compagni a mensa, o che sia portata

SBOCC A il fiasco. Stura il fiasco, e fquotendolo butta fuora il vino, che & nel-
per arlo dall' immondizie, o fiore, che vi pos' essete.
hot Bartolomeo » Ela figura Apherefis (peffo usata da noi ne i nomi
me Cecco per Francesco fatto da Cesco ( che trovafi nel Decamerones
cioè Francesca ) Menico per Domenico; cos! Lippo, Stagio, Coppo,
, Noferi, accorciarono i nostri antichi da Filippo, Anaftagio, Iacopo,

329

and' uno, mentre aspett2,
roba in tavola, piglia de'













Ti fo brindifi. Questo e quel modo di parlare, che dicono Za.
» come accennammo sopra C, 1. st. 28, al termine:u(cir del seminato; ¢

STANZA XIV.

E perché e buono,e non di quello,il quale
E nato in fu la schiena de' ranocchi,
A Meo, che più tosto a Carnovale,
Che per Vopre lo ferba,e/ce degli occhi,
E bada a dire; Ovvia, vi farò male,
Ma quegliche non vuol ch'ei Pinfinocchi,
Edé la parte sua furbo, e cattivo,

Gli risponde: Ob tu fei caritativo.
STANZA XV.



= *

Lasciami hie la bocca ascintta,
Che diavol penst tn poisch'ia ne bea?
lo poppo poppo, ma il cannel non butta,
Risponde etieo: Po far la nostra Dea,
Che sei buttaffe, la berefti tutta,
O' diferezione s'e' cen' e minuzzolo;
Paride beve,e poi gl da lo spruzolo,

STA I

Non vi fo dir se Meo sllor tarocca:

Ma l'altro, che del vin fu stpre chiotte 5
Di nnovo appicca al suo canel la bocca,
E lascia brontolare, e tira sotto,

Ma tanto esclama,prega,dagli,e tocca,
Chrei lascia al fin diber già mexxo cotro,
Dicendo ch'ei non vuol ch' il vin lo quoca,
eta che chi lo trove non era un' oca,

if Patide in barla in burla bevendo, votd il fiasco, e poi si mangid dieci pani,
Prova, il cacio provveduto da Meco, il quale egli prego, che gli defle a laggio








botte, e Meo adduce diverse scule per non glielo dare; ondes
; re

Paride











330 MALMANTILE

Paride fatto un bocciuolo di canna si messe a fucciare il vino per
= s —S cui duole il <a seatats suo 4!
ere; ma egli seguita, e per farlo più arrabbiare gli sbruffa
torna a bere. Al fine già fazio, laid star di Croniede
buona cosa, e che l'Inventore fu un gran valent' huomo; ma
ber pil, per non' imbriacare. 2 ety X iy
PER celia, Voce usatitiima in Firenze, per denotare buria, feberze.
una giovane Commediante, la quale era di genio scherzolo, e burlesco, e face

















la parte della serva; e si domandava Celia.
| Ui Persiani. Ji tno canto e più dolce d' una auelia; ea
Ma feufami, se reco io fo la Celia.

DAGLIENE un forse @c, Cioè bevi un poco, e poi un' altro p
la quantità di vino, o d' altro liquore, che si può bere senza ripigliar
Latino /orbere. ' Mee
FAs) che dal vedere, e non vedere. La cosa andò in maniera, che it
mento; in un batter d' ecchio. /n stfu oculi.
DIEDE fondo al vine, Cioè vord il fiasco. Fini il vino. Dar fondo a un
fa vuol dir consumare affatto, Termine marinavesco; e si dice dar fondo
la nave si ferma in porto, finito il viaggio.;
TONDO, Così chiamiamo quel piatto spianato di stagno, o d' altra}
pra il quale in tavola si posano i bicchieri.
FECE repulifti, Fini; ripuli, consumo ogni cosa, ne volle veder la
inine baffo, e usato dalla plebe.;
NON gli toccaron ? ugola. Non gli scemarono } appetito. Quando a tn gran-
de affamato si da poco cibo, diciamo: Won gli ha toccato ? ugola, e ancora + Now
elt ha roccato un dente,e proverbialmente: E ffata una fava in bocca all orso.
non palatum rigat. Vgola si dice quella particella carnofa, che pende fra le faucl
per uso di formar conucnientemente la voce. Latino wa, columella, =
SERQVA, Numero di dodici, ma si dice d' vova, di pere ye simili, che}
altro si dice dozzina. cay
SPILLA la bore. Buca la botte. Spillare si dice da spillo, che & quel
to, col quale si bucano le botti, e questo forse dal Latino /picu/am, o pure
Spinula, Crescenzio lib. 4. c. 41. chiama /pina fecaria, € '| suo antico —
zatore, (pina fecciaia, la cannella posta nel fondo de' yafi da vino, per
uscire la feccia. $
OPERE, Coloro che aiutano lavorare a i contadini, ricevendo il p
Ic loro fatiche giorno per giorno si dicono opere, 6 opre. In Latino fimi
opera si dicono | lavoranti. + Om
VVO far le mie lorte, Voglio far le mie forze. Voglio pigliarmi tutte lo
disfazioni possibili, Diciamo; i/ sale vnol troppe lorte, troppe invenie, troppi
troppe cirimonie: quand' uno in far' un'-operazione la vuol far con ogni
ancor che superfluo, e non necessario. a
SVENT RI. Scoppi per lo troppo mangiare, e bere. ”
VINA lartata centri, Ci stia bene una lattata. Diciamo: fare uma lat
do dopo che s' è mangiato,, e bevuto beac, si fa venir in tavola nu






















Pow eaenw Re BB OSS. fo. ee ee















tetas Hess:
Stes *

= Fs


SETTIMO CANTARE:? 331

nuovi bicchieri puliti. Che per altro /atrara & una bevanda fatta con zucchero,
-orz0, e femi di popone, che benissimo pefti, e liquefatti con acqua gli fannd

Pe

yt paflare per stamigoa, la quale si da per lo più a' febbricitanti per rinfrescare: ed
. t PY pat gr i pad. hese! abbisto pot il nome di /attata ot suddetto nuovo
bere f » come che vogliano intendere, che questo secondo bere non fias
 $propositato, ne per gola, ma per rinfre(care l'ardore del vino bevuto, come fa
alla febbre la datara, la quale diciamo più comuncmente orzata.
- S¥CCAIELLO, Diminutivo di /ucchio, che vale lo Neflo. Strumento d' ac:
ciaio per uso di bucar legnami: e il Latino Terebra.
NON ha froppa, ve fufa, 1 villano per non dar bere, trova (cufa di non poter
metter fa cannelia alla botce, perché non ha stoppa da avvoltare in fulla cannel-
la per adattarla al buco della botte, ne meno può bucarla, perché non ha fulas
da turare il buco dello spillo, delli quali fui ( che per altro servono alle donne
Se sopra il filo, quando filano a rocca ) ci serviamo per turare simili
gi bucht yperché per esser ben tondi, e di figura piramidale,ferran bene ogni buco,

A di pi r scula, che quello non e vino, ma acquerello, che è la lavatura



ai
io





gi delle vinacce, e serve per bevanda de i contadini, da molti detto vinello, e das
2 altri mezzingo, e da i Latini Lorea, o Lora. Ma Paride, che molto ben conosce,

che: oat sono tutte invenzioni, gli dice: C+ vxol altra feufa, ed intende; Non
~ - Mailerrd per questo di far quel che io ho in animo, cioè di bere.

COCCHIV AE. Quel turacciolo di legno, col quale si tura la buca di sopras
sh della botte; e si chiama così anche la stessa buca. | Latini lo dicono do/ij opercu-
him.”

ie SVCCIARE. Attrarre a se l' umido, o fugo. Dal Latino /were.
a NATO in fu le schiene de' ranecchj, Nato ne i pantani,dove stanno i ranocchi,
j@  chenoné vin buono.
i  ESCE degli occhs. Non può vederlo consumare + lo da mal volentieri, Gli

»duole il veder consumar quel vino, quanto gli dorrebbe il perdere il lume degli
@ Occhi. Detto assai usato in simile proposito.
2 NON vnol che ? infinocchi, Non vuol che con le chiacchiere lo ritenga dal bere
gi — 'Tnfinvccbiare & lo stesso, che dar panzane, bubbole, o chiacchiere ed è il Latino Ver-
a) ha dare, 1 Lalli En. Tr. C. 4. st. 107. dice.

: - Per ch' il parlar di lei non l'infinocchi,

b OHTY fei caritativo Tu hai la gran pieta dime. E' detto scherzoso, usato in
simili congiunture, e si dice + 7% hai caritd pelofa, o la caritd di mona Candida, che
Pe biascicava i confetti agli ammalati per levar loro la fatica,
is NON fo se tu minchioni la mattea. Non s0 se ww burli, Vedi sopra C. 4. st. 15.

 Che pensi tu mai ch' io ne bea? Quanto pensi tu, ch' io al fine ne beva. Altro~
i ve habbi: detto di questa particella mai, che altre volte afferma, altre volte
6 hega, ed altre volte significa tempo, come qui, che vuol dire, quanto pensi tu,
4 Piblesnihiximene boda; In Latino dircbbelt, Waid demim cenfer?

40 poppo Poppe « Cioè io attendo a fucciare, ma io tiro fu poco vino, perché il

Cannello ne da 3
of PVO! far la nena Dea, Esclamazione, o giuramento di contadini; quasi yo-
a Jendo significare /a Dea Pales. Virg. 3. ark) Te quoque magna Pales Oc, <
4 12 Te 2 SE


va Se

—

——





332 MALMANTILE a2, y
SE @ cen'? minwzob, Se cen' & punto. Se ei cen' è pur un poto, Ser:
to Latini nel Pataffio. lovee for pata |) me i
GLA da lo spruRzole. Gli sputa il vino nel vilo a minute sille. Sprags j
ciamo quando comincia a piovere minutamente, onde Spragzaglia osservd il Vet-
tori dirsi da' contadini una piccola quantità di ope per similicudines
7 -AROCC A. Entra in collora; arrabbia.. Voce usata in Firenze, e in
Lombardia. Francesco Negri nel suo Taflo in lingua Bolognefe, portan:
quello il verso d' un' argumento, che dice 4/ Re si turba alla novela rea, pari
4 Re al sente,¢ e minza a taruccar,
SRONTOLARE. E un rammaricarsi, o dolersi di qualche soprufo, o finifiro
vvenimento con parole non affatto esprefle, ma confule, e male articolate, ¢
fra i denti, che si dice anche bofenchiare; ( Nella Valdinievole meer
to il calabrone ) Viene per avventura dal Greco Bronean, che vuol dir P
Virg. in quel verso, ove nomina i Ciclopi affaccendati a lavorare il ferro, ¢:
mini nella fucina di Vulcano. Bronte/que, Sterope/que © nudus membra y
I primo nome lo cava dal tuono, il secondo dal folgore, il terzo dall' ancudine,
¢ dal fuoco, ~
TIRA sotto, Attende, continova, (eguita a fare quella tal cosa. Si9
DAGLI, e tocca, Questo termine significa, fa, e rifa la tal cosa, ovvero pre:
£2, e riprega; e si dice Dagii, picchia, € rocca. Ovvero Dagli, toca, valle
martella, i
MEZZO cotto. Quasi briaco. Vedi sopra C. 6. st, 35. s
CHE lo trovo non era un'oca. Chi lo trovd non era huomo senza ceruello, ma
un valent' huomo. Ceruel d' oca, o capo d' oca vuol dir huomo di poco giudi-

zio.
STANZA XVII, STANZA XIX,

Wit rn





Che scufa non gli pare haver, che vaglia,
Che non gli fea a viltade attribuito;
Così ribeve un colpettino, e in cambio
D' andar. 4 lett s'arma,e piglia Pabio,

STANZA XVIIL

Senza lume, ne luce wia spulerza,

E corre al buio, che ne anche il vento
Non ha para mica della brezza,
Perch' egli ha in corpo chi lavora dreto;
Per la mora si ben si (candolexa,

Che dando il ¢,,interraacgni mometo,
Quanto più casca,e nella mema pesca,
Tanto pin sentech'elt'? molle, Puta '

Poiche dai cibo,e da quel vin che fmaglia, Dopo cht ei fu cascato ye vicascatly
Si sente tutto quanto ingazzullito y Ler non sentir quel mollee frescoacert,
Risolue ritornar alla bactaglia, Che'lvino,e quato diatiavea i
Donde innocentemente s' e partite, Opra di dentro si, ma non di fuera,

Giiito al mulin dal mezzin giis sbracciate
Si (ciaguatta i calzoni in quella gore
Per dopo nella casa di quel loco
Farfegli tutti rasciugar al sotto,
STANZA XX,
Mentre si china dando ilc,., a leva;
Ei fece us capitombolo nelacqua -
Ona' avvien ch'una voltacilacquabews
Sopra delvinyche maiper
Quanto di buon si ¢; cheisei vplens
Lavar i pauni,jlcorpo anche rifeiacgisy
E divien lacqua si feremexegiallay
Gh' i pesci vengon sutti quanti a gall

—

Be GBB PRR E S3SsS> Pe

=

ee en EC!. Be SE we 2 ares






SETTIMO CANTARE 333
BiwbineeDAN ZA XXL Hoi







tutte a lui son note, Lotanto si conduce fra le ruote,
ne per muotar bene sl Romano;. Che fan girando macinare il grano
ei corpo, confie fa le gore, Ben fen' avvede, e già mette a entrata
Anna/pa cof piede,e con la mano, Di macinarsi,e fare una fRiacciata,

wide sentendosi inuigorito risoluette di ritornare al campo; e così (enz' altro
si mefie in viaggio, ma fendosi infangato, volle lavare i calzoni in una go-
 4, €vicalcd dentro, e se bene egli fapeva nuotare, e s' affaticava per ulcir dell”
'Acqua, tuttavia conobbe, che portava pericolo d' entrar sotto le ruote del muli-
no, e restarvi infranto, se non gli accadeva quello, che sentiremo appresso.
» VINO che fmaglia, Vino potente, e generolo. Si dice /magliare, perché il vi-
 no nel mescersi nel bicchiere lascia nella superficie una stummia, che fa certe cose
q come maglic » le quali il vino gencrofo rode, e consuma subito; e questo disfac
que ie si dice /magliare, e quando non le disfa, e segno, che ha poco spi-
ya ito. Edi quiicicchi hanno un detto: Laloccom! io, o vommene ? ed intendono
¢0si didomandar al compagno alluminato, il quale ha mesciuto nel bicchiere, se
quella fummia se ne va, o si trattiene, ed in confeguenza s'il vino e buono, o
tattivo.,, Lasca Nov, 4. fecero uno scotto regio con quel vino, che /magliava,
AINGAZZVLLITO. Forse meglio ingazzurlito. Vuol dir rinuigorito, rin-
» o rallegrato di quella allegrezza, che mette addosso il buon vino.
i dice entrar in xurlo, o in zurro 5 corrottamente da rvzzo, e questo dal Latino





&

ruere,

ANNOCENTEMENTE # e partito, Dice innocentemente, perché in vero Pa-
tide non haveva errato a partirsi dal campo, poiché n' era stato cavato da colo-
TO, chelo portavano via infermo, come s'é detto sopra C. 3. st. 25.

YN colpettino. Vn' altra volta, Vn' altro poco. 1 Franzefi similmente dicono
per esempio; boire encore un cowp. Bere un' altra volta. Provarsi a bere un' altro
poco. Ad è traslato dal provarsi in gioftra.

RIGLIAR ? ambio, Andarsene. Voce corrotta da ambulo latino, che vuol dir
andare, o pur vien da amb io specie d' andatura di cavallo, con altro nome detto
Portame., pecché per esprimere andavfene diciamo Pigtiare il portante,

SENZA lume, ne luce, Afiatto albuio. Senza lume terreno, e senza splen-
dor celeite..

SPFLEZZ.A. Va via furiofamentc. Parmi che possa venire da (pulare il gra-
no, che i) vento furiofamente porta via la pula, cioè i gusci del grano; o das

pigliare il puleggio detto sopra C. 1. st. 80. *
 “OTA, Terra inguppata nell' acqua, e ridotta quasi liquida. Così appresso
i Franzefi moire & il Latino dus, madidus, e quel che noi diremmo mole,
pA MEMMA, o melma. Quella terra, che nel fondo:de' fiumi, soffi, laghi, ¢,
ae » tidotta liquida, che la diciamo anche bellerta per me/metta Latino Limus
  Verifimilmente dal Greco Adigma, che vuol dire miffura.
: a. Meflo in corpo. Detto plebeo. Vedi sopra la voce Gubbia-
mC, 1, st 36.
DA'mexxo in gilt sbracciato. Così dice per (cherzo, sapendo bene che sbracciato
Significa,quand' upo tirando la manica in fu fino al gomito,la(cia ignuda quella, i
' parte

ea

SSBB SE”






,









334

arte del braccio, e non quand' uno si cava i calzoni', co!

aride, il che si dice sbracato; ma-l' Autore si serve della voce
tendere spogliato; enon è-vero che habbia a dire sbracato
corretto, non folo perché l'originale di mano dell' Autore, che
ed in un suo primo sbozzo dice sbracciato, ma anche perché fed
mezzo ix giit ' intenderebbe che ei si fufle tirato fu i calzoni fino a
enon che se gli fufle affatto cavati, come era necessario, che,
voleva lavargli. ¢
SCLAGV-ATT ARE, Dimenare un panno, o altro simile nell' acqua.
GORA. Vuol dire un canale d' acqua, che corre, e propriamente s*
quella fofia, per la quale si conduce |*acqua a i mulini per macinare
ali fofie, o gore si fanno a quei mulini, che sono in fa' rivi,o p
quali e feacfita d' acqua, non essendo necessaric a i fiumi reali, nei:
servi abbondanza d' acqua, basta un foftegno, o steccaia ( che no
scaia ) che volti l'acqua al mulino, e serva per Cosa, chet ae
alla quale si raguna tutta I acqua, che porta la gora. Gli antichi finiv
voci in Ora non folumeute quelle, che aveano simili udi se col Lat, co
le quattro tempora, come ancor oggi diciamo; mia anche le Bergora,
le Campora; E simili. Onde il Sannazzaro nelie Ecloghe della sua A
se licenza di dire Pratora per Prats: Gc. Si pote dunque dare benissimo:
quef' acque così ragunate essi chiamaflero Lacora dal Lat. Jacus, € poi
a staccare la voce, e dirsi La gora. Da i Jatini si trova esser tali, o si
d' acqua chiamati Euripi, e dVili, ma credo che fussero iperboliche a
come si può dedurre da Cic, 2. de legibus, dove dice; Dattus ag
Wilos, Enriposque vocant quis non irriferit ? E veramente e cosa da rid
Euripus è nno stretto di Mare,ove è il fluflo, e reflufio; Ed il Nilo e de'
ri fiumi del Mondo; E queste son fofle semplici, e laghetti, che gli
mani fecero correre infino di vino in occasione di fefte; e da ciò piglio
to, che gli adulatori per piacere a' Signori, le chiamaffero Wii, ed

DANDO ile... aleva, Cioeaizando ile....ed abbafiando il ca:

FECE un capitombolo, Rivolto il corpo sul capo fottofopra; fece un
capo, rivoltandosi foctofopra. Vedi C. 6. stan. 84. '

e4 GALLA, Nella superficie dell' acqua. Dai verbo galleggiare.,
origine da galle, che sono quelle leggierissime palle, che nascono dalle
donde /eggieri, com' una galla,

JL Romano, Fu uno Stufaiolo, che insegnava nuotare alla gioventh Fiorentit
MOLTO annajpa. Annaspare vuol dir mettere il filato sopr' all' alpo”
durre i) filo in matafle, edipanare, Lat, geomerare,afhne d' adattarlo
re, dal Greco ana/pan, che vale retrabere, revellere, E da questo qu
perde molto tempo a far qualche operazione,¢ non conchiude cofadi bi

ciamo annaspare. Qui vuol dire, che egli muoveva i piedi,.¢ le mani,

ve le maui colui che annaspa;¢ si puo anche intendere che armeggiava |

naspava molto, e conchiudeva poco. eal
GLA mette aentrata di far una stiacciata, Già tien per certo d' havere@

infranto dalle ruote del mulino, I caflieri, ed ogn' altro che tenga libri

%



















Tg eee ee ae oe ee











es ES ee








SETTIMO CANTARE: 335
: trata,¢ uscita,mette a entrata, quando ha ricevuto il denaro; € da questo noi

mo Tien per certo, o ha già per ricevuta quella tal cosa.
Stearn e a ccsatabuperr
» che il meschin gra si prefume Ognun si tenga pure il [uo parere;
andar a far la cena alle feanae 's O quelle, o altre, a me non fa farina,
¢ una porta,e in chiaro (ume Bastini per adeffo di sapere,
ise 2 iar conocchie, Che queste non son bestie da dozrina;
Che le Naiadi Ninfe di quel fume, E, 8' ella non m' e feata data a bere,
—— Coronate di giunchi,¢ di Elle son Fate c' han virtù divina,
—— Corrono ad aintarlo infin c a viva E che sia il vero, fede ve ne faccia
La dove il di riluce) in falnoarrina, Li Garani feampato dalla fliaccia.
STANZA XXIII. STANZA XXV.
— Bvede all ombra di falcigne frasche 11 quale così molle, e sbraculato
io Fra le più brave mufiche acquavle MU cadavero par di Mona becca,
&s == Parte di loro al suon di bergama/che, Ch' essendo stato allor difatterrata,
we » ¢fefhe ragliar le capriole, Hlabbia fatto alla morte una cilecca;

ei Chitien che queste Ninfe fien le lasche
[Chile firene, ed altri le cagznole;
4  Lonon fo chi di lor dia pin nel buono,
i Ble lascio nel grado, ch' elle sono,

is Ah % $

jd Male Fate, che (pecie son di pesce,

ys Edbimoilcorpo aftar nell'acqua avvezzo
st Piiche 2 a bagnate a lor rincresce,

—, cos: fradicio merz0;

Si [quote,¢ trema si,ch' io ho fopparo
Per San Giovanni il carro della Zecca,
Ementr' ei ff debatee, e il capo feroiia,
UI pavimento,¢ i circofhanti ammolla,

TANZA XXxVI.

Percio lo spoglian; ma perché riesce,

Queido un vuol far pits prespo,fhar un pexro,
Pertrattenerlo(mentr'hor quefia,hor quella
L) asciuga) una conto queita novella,

Paride stava con timor d' affogare, fu foccorso da alcune Ninfe', les

meffero a spogliarlo, ed intanto una di loro contd la novella, che vedremo

of

;:

3 lo cavarono dell' acqua, e lo conduffero alle loro stanze, dove dette Ninfe
af

MESCHINO, lafelice; Povero. E voce, che denota commiferazione.
ge ANDAR a far la cena de' ranocchi, Cioè affogare, annegare, e così diven-

tar cibo:de' ranocchi.

yi CONOCCH/z, Pennecchi in falla rocca, che sono quei rinvolti di lino, o
è Jana, o altra materia simile, che le donne per filarla accomodano in fulla'rocca
to da esse usato per filare; Voce corrotta da cannocchie, secondo il Fer-

rati, perché-le rocche per lo più (ono di canna; Il Voflio la fa venire dal Lar.

, Golus; quaGi Rorpiata da colwcula.
i

~DRAPPL, Cioè quei drappi da donna, che dicemmo sopra C. 6. stan. 9.
5 ~~ CAMPEGGIAR conocchie, Sappotto che le muca di quelle stanze fuffono bian-

che,ogni cosa di

alfivoglia colore vi si discerne ben sopra, e però ( servendoti

; del verbo pittoresco campeggiare ) intende; si distinguevano sopr'a quel bianco i

,  drappi,

¢ fuentolavano, e le rocche appiccate alle muraglie,

GIFNCO, Pianta, o virgulto noto, che nasce vicino all' acque, ed in luoghi
| Umidi, e padulofi, e non fa foglie, ne tronchi;ma falti,come paglia,li(ci, (en-
E 2a nodi, se non uno ia vette, dove nalce il feme. E per questo habbiamo un,



pro-







336 | MALMANTILE * ©
proverbio, che dice: Cercar if nodo-in sul ginnco; Lat, sodum in scpe ghree
significa cercar le difficalta, dove elle non sono. V2 POT en

PANNOCCHIE, Spighe, che si producono dalle canne, dalla ei

panico, &c, dal Latino Panicu/a, voce usata 'da Plinio y ove tratta
Carerum gracilsas nodis dispintta leui faftigio renmatur in Caxmina,
la coma,; HeeO SNS

SALCIGNE frafehe, Frondi di falcio albero noto, che nasce; € vien pil
rofo in luoghi padulofi, Lat. frondes /aligna. a

eAL suon di bergama(che. Chiamiamo Bergama(ca' un*ballo compone' rato
di falti, e capriole, e-però dice guinre, e fespe rrgliar le-capriale; vest

CHZZVOLE, Sono certi animaletti neri, che vivodo sae:
ti pancia, e coda, e col tempo diventano ranocchie, € met 1¢ gambe,
cascado loro la coda,mutano colore di nero in verde macchiatoje
mo la meftola da muratori; Lac, trudla, e che It Abate Baldo da Vebi
zionario sopra Vitruvio dice al suo paefe chiamarsi Cacchiara,

LE lascio nel grado ch elle sono, Sieno chi elle si voglionojio non do'
nome, che un' altro; perché ciò zon fa fartmarcio' non m' importa; eT
proposito mio. E qui l' Autore mostra d' haver notizia delle diverse
Gentili circa alle Ninfe, le quali tutti concordano esser Figliuole
¢ conchiudono che le piii fussero Deita aquatiche; le quali Deita noi”

retiamo, che ficno diversi effetti, che produce Pumidita, E che parte
infe sieno de i prati, parte de' boschi, parte de i monti, e con divers
Nereidi, Napee, Oreadi 5 ec,: '

NON son bestre da dozzina, Non son bestie ordinarie, e da farne poca Mima:
Diciamo cosa da dozzina, o doxxinale, quella, che e lontana dalla perfe
che @ lavorata con poca diligenza. s

S' ELLA non m' e frata data a bere, S ella non m' è stata data a credere] *

FATE. Vedi sopra C, 4. stan. 54. '

STIACCTA, Si dice quella trappola, che si tende con le laftre a i topi,ed ag
uccelli, così detta, perché nel cadere addosso all' animale, lo stiaccia.
SBRACVLATO. Senza brache, e senza calzoni. 2
C-ADAVERO di Mona Checca, Si (uole in Firenze nel giorno della commen
razione di tutti i morti,ne i fotterranei della Bafilica di S. Lorenzo, che fond il
fepoltuario, esporre uno scheletro di morto con veli in tea', ed altri:
menti, e questo da i ragazzi è detto Atona Checca; cioè Madonna Fr.
questo nome poi comuncmente s' usa per esprimere uno sbattuto, ed -afflitto dale
la fame, dal freddo, e da altro stento. Ariftofane portato in Latino dice: AF
bil a Charephonte differ. ww
FARE nna cilecca, o feilecca, Far una burla; cioè finger di voler 1
fa, e poi non la fare; Sicché vuol dire: habbia finto d' esser morto', &
sia stato vero. Habbia gabbato la morte. Diciamo anche pare uit m b
rato. Ll Bini nel secondo Capitolo dell' orto dice: 2 og ae
' Ho una vaca, ma ell! ha una pecca 1 EY
D! un certo suo turacciol benedetto,
C' ogni volta mi fa qualche cilecca,





REF FS ee 8a Ea ee

2.



Er

eee ae oe ee a oe






ee
1) TP Sr TIMOsCuNTUReE 337
20.

So fips. Qui bao fleffo significato, ché mb difetadé detto sopra. 1. 0.
34. ¢C. 6, stan. 61. er altro havere ffoppato uno, vuol dire H2-
negli orecchi, ec. per esempio. Tu mi hai fatto il servizio tanto tardi, che
ho havuto più bisogno, e però io ¢* be fropparo.
lella Zecca, 1) giorno di S. Giovanbatifta e la maggior folennita, che
ri in Firenze per efser del Santo Avvocatu y € Protettore della Città, ed
oP ome agifteati di Firenze, e tutte le Terre, e Castella fubordina-
; inio fanno la citimonia dell' offerta al Tempio dedicate al detto Santo,
fra gli altri il Magiftrato della Zecca offerisce un gran Carro trionfale in figu-
piramidale alto circa 20. braccia, e nella fommita di esso Carro è un' huomo
tutto coperto di peli, legato.con func a un palo di ferro alto circa un brac-
cio e mezzo, che formando in cima un mezzo circolo gli fascia lo flomaco,dove
f detto huomo,acciò non cachi, il quale rappresenta San Giovanni nel
to. E perché tal:Carro nell essere stra(cicato brandisce, e squote, però
che e nella cima del Carros' agica grandemente ancor' egli; Ed il Poeca
huomo intende dicendo, che Paride si squote più del Carro della Zecca,
Colui, che è sopra detto Carro*
CB yO ineresce. Vuol dir venire a noia, o a fastidio, ed 8 il Latino
Bocce, gior.5. Nov. 6. lo fari s} 5 che lavedrai tanto, che ella ti increfeerd.
gnilica haver dispiacere,c' una cosa sia fatta,, o non fatta. Bocce. Nov. detta.
Ma di cidyche facto, glincrebbe, Significa compaflionare uno, come nel pre-
ee eis questo C, stan. 50. Significa'ancora haver dispiacere inten-
dendosi esserinelle Fate maggiore la compaflione, che havevano di Paride per ve-
derlo così mal condotto, che non era il disgufto d' esser bagnate; E sono questi
due significati tanto prossimi; che spesso col folo verbo rincrescere s* e(prime.»
Punoe Valtro, come fegae qui, e nel Petr. Son. 44.
dP ote Onde il lasciare el' 4/pettar m' increfee,
Rs a intendere mi pefa, mi dispiace il lasciare, e mi viene a noia l'al-
pettare, Li Persiani nelia lettera al sig.\ Principe D.Lor. disse:
Ml mio bifegno ho gra detto a parecehi
(art 's) EB ciascun se ne duole, e gli rincresce
4Omexo. Coml', ¢, firetta, e con una fola,z, che fa alpro ( per-
ché:con Pelarga., e con due zete, che fanno dolce, secondo l'opinione del dot-
4. Carlo Dati, vuol dire meta ) significa bagnato assai; e la voce fradi.
¢io che wuol dire corrotto, qui signitica inzuppato d' acqua. La voce mezo vuol
dire una:cosa tenera per esser troppo matura, come farebbe una mela o pera, ec.
vedi sopra C, 3; stan. $3.0 una cosa intenerita per haver inzuppato molto umi-
do una spugna intinta nell' acqua, e questo e il senso de! presentes
hiogo:.. Adezo & dal Lat, mitis per maturo; ed € il contrario di acerbo, che così
chiamiamo la:frutca'n6 per anco matura.V olgarizzamento antico di Palladio,nel
mese di Gennaio; 'tit, 15. Serbanfi le forbe:, se si colgano dure; ec. e ivi comin-
)  Glanfigimmerzare. Ul Lat, dice: wbi mitescere caperint, —


























2
.
"

= BBtesei ES

3
a
7

=

xe x vy STAN.

os




'
i



338 MALMANTILE ~~

STANZA XXVIL
Fro un tratto una dama,eun Cavaliero
Moglie,e Marito in buono,e ricco (Pato,
Che fatti vecchi, contr' ognipenfiero 5
Dopo a' haver qualche anno 'higare
La grinza pelle con il cimitero
Conuenne loro al fin perdere sl piato'y
E senz' appello haver a far proposito
Di dar per ficw tal' offa in deposito,
STANZA XXVIIL
Lusciaron due Figliuoli è più compliti
Che'l mondahaveffe mai sule /ue sceney
Perch' essi havevan tutti srequifiti
Donati aungalat'hnomo,e un bom dabbene;
Aggiunto che di solds eran.gremui,
(Che questo infomac quel che vale,e tiene)
Stavan d' accordo, in pace,ed inamore,
Er eran pane ye cacio, anima,e core,

gidi non usa pil.
PIATO ye piatire.






Perciocche il nostro
gis habent vigorem,

v

potevano piatire per La lor

La Fata principié a contare la novella (la quale ¢,tolta'da:lo Cunto deli
ti gior. 4. Cunto 9, e gior. 5. Cunto 9, ) e dice: Furon già unadama, e ut
valicro marito, e moglie, i quali venendo a morte lasciarono due Figlit my
costumati, e ricchi, i quali s' amavano grandemente l'un l'altro. Qui
fa wna digreffione, e considera, che questo modo di trattacfi fra i Si

Lite, o litigare d' avanti a' tribunali, detto dal Latybar+
baro placitum per lite,¢ placitare;la qual voce ritengono bella e intera i Veneaia-
ni, Placitum è il decreto, sentenza del giudice, o Magifttato, e quel che i Fran-
z¢si dicono 4rreffo secondo il Budeo da arefeein, che in Greco vuol dire placere.
Ne' Senatuscon(ulti, ovvero Decreti, e Sentenze. del Senato di Roma ulayand
guefta formula: Sexatui placere &c. come si ricava da Cicerone Filippica 3. € 5+
Nell' Ordinanze Regie in Francia si legge sempre in fine: Car tel est nostre plaifin
iacere è tale. EB nella legge si dice; che Principam le

enne poi da' Latini bafii a tirarsi questa parola'a se
il proceffo della lite medesima., si come anche éudiciwm significa Ja' sentenxa jt
4a lite medesima, che fa nascere la sentenza. Piatire lo Spagauolo dice;
Franzefe plaider; wutti dail' iftefla fonte Latina, Il Doni.nel suo Cancelli
Sempre ne piati la rovina va innanzi, e chi piatisce ha quant' ei vysle il
Ed il Varchi Sc, Fior, lib, 14. Erano affegnate le canfe delle pouere
ertd.. E poco appresso, dice; Perché
care quel piato al terxo posefore. Ed in quest ultimi versi della presente:
27, dice metaforicamente, che.a costoro già fatti vecchi dopo haver fatta




















apt eh gS
Ce fare i es
ca è neces

E fr as de cei osre

ll contrariocoftor di chi io favelle
1 quai di cortesia furon due spe

E trattavan ciascun da buon Prac
5S' haurebbon














il
pene chee
bifegnsos. mi





SS = 61 PERERA EES eeeeeeee

rar lungo tempo la loro carne a i (epolcri., conuenne morire, e fart



Il proverbio piatire i cimiseri vuol dire Esser d' eta cadente, che Luciano portal
in Latino dice: lterum pedem fepulero, o vero in cymba Charontis habere; Cs

noi pure diciamo; Havere il più fu la bara o vero il più nella foffa, —






GA








SETTIMO'CAN'TPARE. 339
GALANT" nemo yed huomo dabbene. Si posion dir finonimi; ma Mrettamente
galant' buomo vuol dire huomo di garbo, e come dicono i Franzeli, ones" uomo;
oltre acid amorevole, ed alla mano, ed huomo dabbene vuol dire huomo di co-
»huomo d' anima, e che fa opere buone. Spagn, hombre de bien. L' uno
I altro comprendono i Greci colla fola parola Caoscagathos. Calos ignitica.,
adig eh buono, da bene. “y
GR. 4. Ripieni. Bil latino Spifus.. Denfus... E qui vuol dire havevano





pieno di frutti, un luogo pieno di mosche, o simili; perché tal voce 4 do-
usare in quelle occasioni, nelle quali cade la similitudine del proprio di
Hy « Greto-vuol dire terreno'ghiaiofo, e pieno di faili, come sogliono ri-
S tive de i nostri tinmi, scolata che e l'acqua piovana, quali rive però
og — chi )Greto 5 come greto d'arno, greto di mugnone, ec, Ora Grero addict.

fies no di danari; se bene e detto improprio,perché gremito s' intende un'

dice i) Vocabolario della Crusca, /o diciamo in significato di speo; forse daila
titudine [pela de' fuffi de' greti;e diciamo anche in queffo significats Gremito. Quan.
ame,inclinerei a credere, che Gremity dal dirli propriamente degli alberi,quan-
'ono 'picni di fiors 5.0 carichi di frutta, venitic da Greminm perciocche if
quella parte, che suole empicrsi di tali cose. Gli antichi Volgarizza-
che i Latins dissero diteus eth traduilero greto; laonde potrebbe ad alcuno
 questa ia fatta da quella. Seneca epilt, 15; Liles repersi en littore cal-
oie ao aouisiem amet delettant c Panciulli & sthcesanc in cose di
piccol pregio, si come.tono pictre, che l'huomo truova nel viaggio, € uel gresv
del mare, e ne' fiumi. Palladio nc] Gennaio tit. 14. favellando della lattuga.
Candida fieri putancur, si fiuminis arena, vel litoris frequenter [pargatur in medias,
rm E posiono diventare bianche se entra loro, e intra le loro foglie spefle volte si
spatga rena del fiume, o del greto, Qnde a dire gremito di soldi s' invenderebbe,
= hi sopra il vestito,o sopr' alia persona sparfo gran numero di soldi,
i somMeenemito di mo/che 8 intende haver molce mosche addosso, € non nella tafea,
i" Oinicafla.; Tuttavia, sc bene. improprio, è alle volte usato, come qui,
}, ESSER pane, e cacio 5 anima,e cuore, Andar' uniti, e d' accordo in ogni ope-
A razione, Bene conmeniunt, © in una fede morantur,.
wht iy oterra al Sole, Se hanno mafierizie, o poderi; per esprimere,
6 uno che. i peep roba diciamo: // tale ha quattro cenct, © se ha beni stabili
in terreni: Egli ha della terra al Sole,
» SLAMO di si perfida cottoia, Siamo così iniqui, e di mal animo, Quei legu.
Mi, che per moito che si tengano al fuoco non si quocono, ne inteaeri(cono
Mai, si dicono di cattiva cotteia  € però con dire huomo di catriva cottoia, §? in-
tende di genio maligno,¢ difficile a persuadersi al bene. Gr, ateramon,
~ ESSER al dumcine, Vuol dire eller in cftremo di vita; € vicue dail' uso, che &
nello Spedale di S, Maria Nuova di mettere uo piccolo lume a un Crocititio al
letto di coloro, che sono agonizzanti. Si dice an.ora; cfier alla candela,
) NON gli fovverrebbon d' un tnpine, Non gli darebbono un minimo aiuto. Sov.
yenire neutro vuol dir covalent. Non mi (ovviene, quando fu questlo. Non mi
eieanam fugqueito. Lat, mentem /ubire in mentem venire, fuccurrere, Fr,
(e fovvenir 5 2

Vv2 " eHOz.

oe

Seba thasa

STEED



et

=—s=

o




























340 MALMANTILE) |

1
HOZZORECC HI. Huomo scelleratoyed infame: EB questo,perché s
fattori, che per la tenera eta sono elenti dalla ordinari: i
fiizia contraflegnati, come dicemmo sopra C.2. stan. 3. e C. 6. stan, $4. e fra
gli altri contrafiegni uno è il mozzar loro una parte degli arecchi,. \. mye
. LOKT AR acqua per orecchi, Fare a uno watt i servizzi i

HAVBEBBON volute indovinare. Questo termine e(prime la
che uno ha in servir l'altro, e compiacerli in tutto Veen.
STANZA XAXL sT ZA XXXL
Essendo un giorno infiemo a um conuito y E tutti quei che feggon quiviamensa
uad'appanto agurzato hinoil mulino, Lfervi 5 i circoftanti,ed f
4 mangian con buonissima appetite 5 i
4Von focome il maggtore dette Nardino
Nell' affettar il pan taglioffi um dito,
51 ch' egli infanguind st tovagtioline
E parwegli si bello a quel mo intrifo,



Ch ei si pose 4 guardarlo fifo fife. Lsangue:
STANZA XxXIl, STANZA XXXIV,
E resta a seder li tutsa infenfato, Che gli par di veder ymentre in

CL! ei par di legno anch'er come la fedia, i for ver
Luo is ( tanto nel wifo è dilavato ) qualche Dea di Ciele
Con la tovaglia i simili in commedia Composta colafsis di rofe, ¢, Z
E mirando quel panno infanguinato E si gli piace: y.¢ tanta ght
Hor mai tant! allegria mutain tragedia, Che finalmente mentre ch'
AMeatre nel pin bel suon delle scodelle Vna moglie d' un tal componimene
Si vede — riposar le mafeelle, 'Non fad de i fuci di mai pil contente,
Edendo gli faddetti giovani a un conuito, Nardino, che era.il maggiore,afiet-
tando il pane,si taglio un dito, ed infanguind il tovagliolino, e nel mirae quel
bel roflo in sul bianco, s' innamoré in maniera, che si propose di non haver mai
a reflar confolato, s' ci non piglava una moglic compotta di quel colore del 10+
vagliolino infanguinato, 5 nhy
CONMITO. Desinare, o cena splendida, Dal latino Consivixm, o c
da Conuitare nel senso che gli Spagnuoli pigliano il loro. Combidar, per fuaicare,
¢ nel quale il prefe il Boccaccio, che difie, Commie « mangiare:,. E, Conuirati alt
ravole, ii B
AGVZ ATO il muline.. All' ordine con la fame per mangiare.. Così eratta lt
similitudine dal mulino; dicefi Adacinarea due palmenté, c10e mulini; di ¢
preftezza, © voracita maftica da amendue i lath aun. tranoy
itanza 22. > e ORE ong B
APPETITO. Vuol dir. appetenza, e desiderio im generaley»ma
detto assolutamente,¢ sen2? aggiunta,vuol dir Fame © voglia 4 0) *
giare. Vedi sopra C. 4. tt. 8, # mal che viene in bocca allagalina, > 2 sb Oi
TOV-AG LIOLINO. Quasi piccola tovaglia. Quel pezzo di panne line
tiene avanti,quando si mangia etiendo a mensa, Boccaccio disse
lo dichiamo anche falwietta dalla voce Spagnuola. Servillera, perché serve mol
al minificro, e al scruizio della tavola, hae Be = HERR,

og






















#&eFRESELLCFE - Pee stsiz=

Tr

gg #E 2 se FZ.

=

2 ge (Ha








SETTIMO CANTARE: 34

INT RISO » La poluere } © altra materia simile stemperata con liquore, come
e:farii Wa si dice:imtrifo, e intridere Ma significa ancora imbrat-

| tao, [porcato, ec. come significa in questo |
PISO fifo. 'Senza batter' oechio 5 wha greta attenzione: dntentis, inconni-

eculis. 1 Greci dicono in una parola e4/cardamytti, che @ lo stesso che:





s o irca'y

8 28S Cash vedessio fifo,
BSB NG Come Amor dolcemente gli governa
Seles Sol un giorno da prefso,

2m 8 Senza volger giamai rota superna,'

Bp%b-on pers | We pensaffi a' altrui, ne di me Stess,

 Purcdiwg sci El batter eli occhi miei non fuffe [peffo.

- DILAV. idito. Smorto. Si dice dilavato ogni colore, che nons

' “ATO; Impallidi

-ariva alla perfezione della sua essenza: come rosso dé/avato si dice un color roto,
'che sia più sbiancato,e pity'chiaro del vero roflo. Latino difurus.

PPO far con la tovaglia i simili in commedia, Intende ch' egli e bianco appunto
come? latovaglias Latino' non oxnm: fic ouo simile, I due simili ® wn fuggetto di

', come quello de Menechmi di Plauto, a molti vi hanno (cherzato,per-
 secondo d' intrecci.
A, Specie di tela lina fatta a un' opera, che si chiama renfa, detta così

dalla Citta'di Rein Francia. Così 'Perpignano sorta di panno dalla Città della.
Navarra di questo nome. e4razzi dalla Città d' Arras in Fiandra':¢ Dxagio al

) Boccaccio si diceva un panno, che veniva di Dovay Città di-Fiandra.,
che Gio; Villani secondo l'uso de' suoi tempi, chiama Doagio. Latino Duacum.
Baldacchino j deappo di Levante; da Habbillonia, che i Levantini chiamano 24-
g4ady inoftrivrantichi Ha/dacco, Gio: Villani |. 7. E'messo fuori delta Città, sopra
(a sua persona umricco palio di Baldacchini di seta ed' oro,

LENZA,olenfa, Lat, tinea, filum piscatorinm, detta così quasi dal Latino
linea, Quella'cordicelia fatta di crini di cavallo, o di feta cruda, con la quale
filegaitiamo da petcare:» Franco Sacchetti Nov. 163. Egii haves preso l'allumi-
white ale lente acscandole con 200, Fiorini a' ord, Lalca Nov. 166. Fau'nn pescatore

i puceote i e/cando con lami, e con lenze,

* Wid oer. Acetta. Pezo di tela ia larghezza det suo essere, € lungheza

4d libitum; come un telo di lenzuolo, 6 di paramento sdrucito in tutta la lun-
ta#di esso lenzuolo, o paramento, Diciamo ref da pane quella rovaglietra,

O.triscia'di panno lino, con la quale si cuopre i} pane in fu l'afle', Qui intende

iktovaglinolo. Te/econ I", e slargo usato da alcuni in Poesia, vuol dire il dar-

do. Lat. telam..

“'GLfvaapelo, Glivaa genio. Se gli confa: e secondo il suo gusto; él) op-

posto:d” contrappelo detto sopra C. 6, itan. r. }

"2 a ete mine XKXV,. 2
da figura nel pensiero, E come chet la vegga daddovero
ieee paket, eas Divoto se le inchina y € le favelld,
Co' suci capelli-d' oro, e t* occhio nero, E le promette, # egli haurd moneta,

“Che più ne men la matturing frella. Di pagarie la fiera al? Lmproneca,
: STAN.



ig”











34%
E vuol mandarle il cuore in un pasticcio y
Perch' sila se ne serva a colarione;
E gli s' interna si coral capriccio
E tanto (ene va ip contemplarione s:

Nardino s' immagina, e si compone nel pensigro una'
parendogli d' haverla veramente avanti a gliocchi., le parla, e se
Ie dona il cuore; ed in questa guisa s' inaamora ardentemente d' una b
maginaria. sie eye Han HOR ee IPE

PRES C.A. Trattandosi d' huomo s' intende Vino dipoca eta; ed h
donna freschi s' intende fani, gagliardj 5. di buona cera,quantunque
grave. Virg. cruda deo, viridi/que fenetius. Frefeo Secondo il Ferrari












ney



Origine dal Latino vire/cens. La marturina Hella. Virg. Qualisie
fer undis,; to

PAG ARLE Ia fiera all' Improneta, Pagarle un regalo all
giorno di S, Luca 18, d' Oucbre all' Impruncta', la quale'é una Chiefa
tana da Firenze, celebre,¢ frequentata per un' immagine miracolofa
stima vergine, che e quivi; la quale in tempo di calamita, e di
portata folennemente a Firenze; e nella venuta di questa Immagine f
una Lauda in una Raccolta antica di Laude spiriwali,

E SE gli imerna si cotal capriccio, Gli si ficca nel ceruello, o-gli
mente questo capriccio, fantasia, opinione. Vedi sopra C, 15M. a4
S' INMAMORA come un miccio., S' innamora come un' asino
mente, perché l'asino e oftinatissimo,¢capone. 5 itt

STANZA XXXVII STANZA X
Cos} a credenza infacca nel frngnuolo,
Ma da un catoegliharagion davidere,
Che s'egli -veryc'eAmor vuol esser folo,
Rivale non e qui con chi contendere,
Ma Brunettoilfratel, che n'hagra duolo,
Poich'ilsuo male alcun non pus 'copredere.
Tien per la prima un' ottima ricetta 5
'Di rimandarlo a casa in una feggetta,
STANZA da
Ei che vagheggia fott' alle lenzuola Replica quello, e feccafi la.gola:
i gentil volto,e le dorate chiome, Lo fruga, tira, e chiamalo per nome
Ne anche gli risponde una parola 5 Ed ¢i pianta una vignaye nulla
Won che gli voglia dir ne chene come, Pur tanto Caltrofa,ch' ei si rifente,
Così Nardino sianamora ardentemente senza saper di chi. Brunetto fu)
tello lo fece portare a casa, dove lo meflero in sul letto, e vennero,
Speziali a vifitarlo, ma non conoscevano ne meno essi il di lui male; onde
netto si mefie a pregarlo, che gli dicesse quel che egli havea; e Nardino!
la sua contemplazione non rispondeva; pure alla fine vinto da tanti
fratello parlo nella maniera, che vedremo nell' Octave seguenti.
eA CREDENZ A, Vuol dire, quando si compra qualche mercanzia,
































See aS SELB RTE SE &

ASE

=~



SETTIMO'CANTARE. 343
si sborsa il danaro allora, as garlo in altro tempo. Ma qui vuol
dire feniza proposito, o senza son *

mento. fH Varchi nel Cap. dell' vova fede.
©) Chiba fquadrato ben la quintefenza,
“. 9) Dite ch' ella non ha color neffuno-,
© 5 Bebe quel giallo v e posto a credenza.
pTr@ng) Rir67." > ° '
Contro di noi bravavano a credenza.
Questa maniera è corrispondente al graris de' Latini. Perfecuti funt me gratis, La
version Greca dice dorean; in dono, cioè di lor cortesia, senza che io il meritaffi.
INS ACCA nel frugnuolo: S' innamora; Se bene entrar nel frugnolo vuol dire
anche entrar' in collera. Frugnuolo è  Janterna; con la quale si va di notte
a cacciaagli vecelli,ed'a pescare; ed e parola corrotta da fornuolo, perché tal
trnaefiendo simile alla bocca d'un forno, così e chiamata.
EGLI ha ragion da vendere Gli avanza della ragione. Ha grandissima ragione.
\ SEGGETT-.A; Seggiola portatile con due stanghe. Vedi sopra C. 1. st. 48.
 GOMITO, La congiuntura del braccio dalla parte di fuori, dove si piega a.
mezzo il braccio,, dal Latino cubirs.

VAGHEGGIA, Fa all'amore, amoreggia, con desiderio d' avere la cosa amata,
Yagguarda., come difie il Buti cittadino, e Lettore Pifano nella sua lettura sopras
Dante, Vedi sotto C. ro. st. 44. Dan, Purg. C. 16.
aT A Esce di mano a [ui, che la vaghegvia,

\ S.* Prima che sia a guisa di fanciula.
Enel Parad) 10. | Eli comincia a vagheggiar nell' arte Di quel Macfire.
Fazio degli Vberti nel Dictamundi; canto 143.
ale Efe @' udirlo proprio ti vaghegsi.
(cioè feivagho; ardentemente desideri ) E canto 144.
we Bios va pur, che quanto priego', e chieggio
Al fommo bene, e fol, che tofto sia

2 Vaeay Wel paefe, ch' i bramo, e ch* i vagheggio.
cio' desidero, ne son vago; col quale io fo all' amore; ea cui mi pare un' ora,
mille anni di ritornare » Vagheggiare il Ferrari deduce dal Latino vistare, frequen-
ter ae, citaa ptoposico i versi di Lucrezio lib, 1. che descrivono Marte, che

Venere. uae
—— in gremium qui fepe tuum se
'yO Reycit aterno devinitus vulnere 'amoris,
Arque ita suspiciens tereti cervice reposta,
Pascit amore avides inbians in te Dea vifus,
O:pure view da Vago, avido; perché chi e avido di godere la cosa amata, va at-
torno percercarla, e fi'rigira come farfalla intorno al lume della bellezza di
quella, Dante in un suo Sonetto.
. To son se vago dela bella luce
Degli occhi traditor, che m' anno occifo,
Che la dov' io son morto, e son derifo,
La gran vagherza pur mi riconduce.
NE che, ne come. Intendi, che non folo non gli volle dire ne il male, ne la,
Caul@ di efig, ma ne meno yolle parlare, SEC-








'a


344

MA LMDANMTILE ©



SECC-ASI la gola., Se glia icequanie fauci per. isemnan eb Li strc d
PIANT A una vigna.,. Non bada, 0-non attende.a, ice «Che noi

diciamo anche far orecchte di mercante » che &:'

titi, che lif

propongono, attento folo al, suo vantaggio., irc 57 'Ear conte che

L Imperatore o far conto, che uno cants.. Per il conteario schi parla
non bada, o non vuol badare, dicefi Predicare al defer

C. 10, st,

bere.
Studio iaktabat inani,

46. Jn Latino pyre.trovanGi molt: detti in questo
Vento loqui., Surdo canere, Frufira 3 velin wannm cantare y cum pisce,
Aliam rem agere » Oc, Virg. Ech 2, tbi bec sanouiteelis

@ gente,
10 5 Predicare:
significato, come:
Lire

z ve è



SZrifene. Cioè si sloeglia da quella applicazicne o filamin unis sili






STANZA X STANZA XXEXIL os
Dicendo; Fratel mio, se nae mi vuoi Kedi jSoggsnnfest' altro, och' io m' adirey'
Quel benyche tu dicei volermi a faced, 0. O\par. Rpiuernred etme
Non mi dar noia,va pe' fatti.tuor y Hai tus quiftione? hai tu qualche rigire
Lerche ii mio mal non è male da biacca, Tx me 0s 4 dire in tutte le manicre,
Al quale ad ogni mo trovar non puoi  Lardin rispose, dopo on ne

Vn rimedia, che vaglia una patacca, Tu fei importuno oi pbanmal

Perch'eglie e firavagante, ed alla moday,
Che non se ne rinuien Caposne coda s
STANZA XXxXX1L.

Brunetto udito il cafoje quanto e' sia

Ma da chiio devo, iro eccomi prow; 4
Così guivi-di tutto fa unragcontes.)

STANZA XXXXUL ]k

E conoscenda, c' a ridurlo in sesto

Ji suo cordogtio,anch' eidolente refea y Ci vuol'alero che il medico,oilbarbiert,
Se ben per fargli cnor mostra allegria Vifi spenda la visa,e vadailrefios. |
AMa(com'io dico)dentro e chi la pefia Vuol rimediarvi in tutte le ae

Perch' in veder si gran malinconia,
Ed un umor si fifo nella testa,

E sav Ff risolue pr
D andar girando il i meal kh
Jn quanto a lui gli par che la fucchielli, Di trovargli una mogtic di suo gfe,
Per terminare il giuoco a! pagrerelli, Com' ¢i gliel' ha dipinta ginfto gino,
Fratel mio, se veramente.tu,mi porti quell' affetto, che ww-dici, lafeiami Mary |
¢ non mi dir pill altro, perché ad ogni modo tw pm rimediare al mio mal
che & grandithmo, Brunctto di nuovo lo prega, onde Nardino, vii
importunita gli racconta tutto il caso, e Brunetto, se bene dentro ha
travaglio facea buon vifo, e datogli animo si risolué d' andar girando il Mondo
per veder di trovare una donna Jecondo il gusto di Nardino,, e cavarlo di quella '
frenefia.
Vna cfortazione,, e richiefta simile a quella, che fa Brunetto a Nardino, fail
Ma scherone allo Gnocco meek saper fa. di hui affiizione come si vede ne i
versi dello Stef
Atto pr. Sc. pr. » ae riporto qui, perché il Letore veda, che a un' Let
rerato ( come era lo Stetonio ) non si dildice alle volre la(ciare gli fludj pil fer
per le bizzarric fanciulle(che., ¢-spero, che non fara Ailgaee queita poca di digrel-
ne.









SETTIMO'CANTARE, 34y
6G NOCCHVS, ET MACCHERO:?
Ga ne Mundo traviare venivi,
pay ” Cur non tum morui, cum primim lucis in auras
- » Sborsavit genitrix ? Cur me disgratia emper
49 Perfeguitat manigolda fenem? Cur ladra placerum
gy, Abftulis', & cunctis caricas me feeva malannis ?;
7 Quando finalmentum dabitur mifura teavai ?
x9 Quando refinabis streghidima filia streghae?
» Dum me-pensabam biancain reposare vechiezam,
x Mille diabolicis straziorque, creporque ruinis.
» Vh me mefehinum | Poterit quis ferre focorfum ?
'Ma. 55 Appuntum Gnoccum video. Quid brontolas? ola !
gy Fronte malinconica, quid tecum, Gnocche, favellas ?
 Sigy-Deh poverome, pares viridas magnafle lucertas;
> 5, Tam demagratus, tam difuenutus apares.
gy Testa dolet forfan ? Sciatica ? Fiftula ? peius ?
»» Ag potius placidam flurbant penseria mentem ?
y» Dig mihi quzfo tuam (cannat quid, Gnocche, coradam 2?
3» Vade viam, Macherone, tuam. Pradele, fogare

a ~~ 4, Mevolo', nec quidquam poteris fuccurrere Gaocco.
i Ma,, Ohimé! cur sprezas fradelli verba pregantis ?
o 3» Quis scit ? parlando paffabit forte dolorus,

| gy, Praefertim-caro dum palefatur amico,

Ga 4, Deh nolis, quxfo, nolis mihi rumpere teftam:
ea 5, Deh lafia me star; fum pienus; vade bonhoram,
p » Nec des impaccium, quoniam mihi crescis afannum.
wt | May »y Deh possar mundus ! Tortum mihi facis adeffum.

io ' »» Cur mihi, Gnocche, tuum non vis sfogare'Jamentum ?
ie x» Sum pro te chi 16: prafum dic, quafo, travaium'.
eh Gn.,, Pur ibi: Vade tuum, cancar ! tu vade viagginm.



". » Me miferum ! ad mundum veni trascinare cordam;
oe x Mancum nonne malum fuerat non nascere, vel i
i 9 Nascere debebam, plus prattum nascere fungus,

ib 9» Quam malé stentando scontentus vivere femper 5
più » Omnibus & giornis centum morire fiatis ?

Ma. 5. Maide ! Cordoglio (ciappas, & (pernis aitam?
ry » Vadis & ad guilam matt, Lanzique briachi ?
ia a» Infuper, & fdegnas, si quis tua vulnera-curat?

Gn, !
> a »» O bellum tempus, Machero, poca(que facendas !
oi > Otmnes'confilium femper dare novimus'aletis

i yy Sed fibi medesimis nolunt procurare parerum.

a - ay Bene dicit.vaigi proverbium: Ducere danzam,

i » 'nuces OMnes, qui fedent, bactere norunt,
ys Cum funt ad terram. Me lafits dico, malhoram.
Ma,  » Ah Zucarine meus, meus, ah Gaocchine, galantus,
a Xx

»» Quid


346

Gn.

Ma.

Ga,
Ma,

Ma.

MALMANTILE 4)

9 Quid facies hofti, fidedegnaris amico?) ) — wanes,
yo Cur mihi nafoonsdisy pbc vulnera
»» Non ego partibo, nifi contes ante' marezam.

x» Su, fradelle, euam crep: oaconien ieee raconta.
x» Non parlas ? Deh butta fora, meschine, venenum;
» Dic mihi, que-carpunt faftidia tristia mentem, ) 5)
5» Que lacerant cuce 5 que te fulpiria rumpunt?.
>» Nonue recordaris tirictos nos esse parentes ? HPiy
>» Eft tua mamma mee carnalis,Gaocche, sorella,
x» Atque ego nawura si,non caenalis, amore
3» Sum tibi fradeiius plus quam carnalis: aitam,

3» Quam potero tibi, Gnocche, dabo,; fac denique provam eu

»» Nam abi porto beaum, nec me fradelle licenties. ai
x» Namque amo te plus quam me stessum » ane
»» Dicito cunea mihi, nec ce meschine' fa
»» Confilium forfan potero tibi dare galantum. &

2» Quid turbulentus guardas ? fu butta deh foras;

»» Eia, valent' homus; non finghiottire bilognat;

"5 Valneris ascofti nunquam medicina trovatur; '5 7)
x» Atsborsando foras fanatur fepe dolorus; 2
»» Fiftula, qua tumuit, totos corrumperet artus, itt
» Ni lancetta viam barbieri Jefta taiaret, #
3, Sufum, Gnocche valens » cordolia dire comenza. 4b

»» O fortuna mihi nimium traversa tapino,
sy Que mihi per forzam non firappas ventre magonem;
x» Eft-ne poshbilum, quod non sborsare fiatum,

» Vnam nec potero gambam distendere voltam ?
x» Sum desperatus: volo me impiccare da verum;
x» Cerne, mei, Machero, cavezam porto fomari.
» Impiccare ? mai. Non impiccare te,non.non;
a» Mattelcis; costat troppum impiccare: nieatum

» Tu facies. Guardes gambam ! impiccare ?,Diavol !

» Et te, meque fimul piccares, Gnocche « Gas 'orlanauting
” Maidé » quis tantum milzam tibi rodit afannus?

x» Dic, faporite meus, que te fuentura chiapavit ?

») Sime impiccabo, cunctos scappabo, travaios.
>» Pur iJluc; iftam mattezam manda malhoram.

», Sola meum stentum poterit sbandire,caveza\. Mt bed
>» Ab nimium certé te stessum, Gnocche', fafinas: ) )
5 Mancum donna timet, mancum se donna sgomentat: a
»y Ne facias cofam talem pazelcis adeflum,;

»» Incidis ia brafam cupiens evitare padellam, te
>» Qui fugiens damnum, foccorsum a Morte rechiedisy «

»> Qua nullum maius damoum reperitur inorbe, >)
» Dicas, quid peius furca maginare poteftur?, 5 «4










FF pe pe

=








SETTIMO CANTARE;

> Nonne vides furcas ipfos odiare (afinos y

» Millantas furcas meritant qui mille fiatis ?

x» Forse putas bellam cofam piccare feltefium ?

» Nullos audifti, nullos nec, Gnocche, latrones

93 Esse volenterum piccatos., Canchere | robbam

x Perdere, poderos,, filios, atque moieram

»» Possumus; at contum non muttit perdere vitam |

xy Parlemusd' altro, bona notte; porge cavezam,

» Fac fennum matti, caveas non talopram.

» Si fennum matti facerem, mattiffimus essem; '
y» Sum deliberatus cannam truncare una volta;

2» Nec parles; quoniam mandas tua verba Patraflum;
x» Et liquidas tentas accoglicre retibus auras;



347

 5» Dextra orecchia bibit, fed versat lava parolas;

y» Surdo verba canis; oleum fimul opera perdis.

»» Qui pro te robbam propriam, vitamque gitarem,

»» Pocum flimo malum pro te gittare parolas.

»» Indarnum gracchias, indarnum dico, va viam.

» Litera vis tandem ficri longissima ? Ga, Certum.

2 Et godis cortum laqueo difrumpere collum ?

» Audis, Ma. Et tandem cornacchis essere pastum ?

» Sentis. Ma. Bavofam buccam torquere ? Gn, Cofiaum;
2» Et tralunatos oculos mostrare ? Gn, Davanzum.

» Lucentem faciem, lucentia bracchia, fula

» Viscera, contradam totam peftare fetore,

2 Et vitiare diem vitiato viscere letum ?

»» Sinum; si dico, finum; volo rumpere cannam.

» Heu ipfis fugiende lupis, buttande fofatis,

» Terribilis stratiande modis, privande facrato.
» Denique penserus nullus te, Gnocche, tuorum

yy Tangit ? Cui laffas pupillos, paze, chiatinos ?

x» Cui robbam ? cui confortem ? miferofque parentes ?

» Teque finalmentum ? Case qui (cribitur heres ?

»» Vis proprias carnes tecum mandare Patrafium ?

»» Vis proprios natos panem cattare per uscios,

2 Disperios pueros pitocorum more per urbes 2

x» Et post de fora veniet qua fama da verum ?

»» Gloria que Caf laflatur? Respice tandem:
2» Teque, ruofque fimul, mifere miferere fameia 5
» Et miferere tui, qui proijciere fofato,

x Indignum facro corpus recoprire tereno..

2 Forfan ad Stygias ibis ? (eu forfan Acheum

»» Ibis ad Infernum ? pensa, pover' home, to feetos;
2 Pensala, dico, benum: facile eft calare dcoiium,
»» Sed montare super cancar; stentare bisognat;

Xx 2 >> Sed


sj



348

Ga,
Ma.
Ga,
Ga,

Ma.
Ma.

Ma.

Gn.



Sum contentus; abi, grarum fed fiascum, —
' 4 Nam stio cesta 6 rampas beulaon 6 a 4




MALMANTILE ©).

j» Sed nec stentando brutto feapulabis ab Oreo. »

»» Horfus tornemus ca(as; fu, Gnocche: cavezam

>» Case mitte tue. Pensas piccare? bel opram; «

»» Essere non vellem, Veneto pro boia teforo,

» At tw, te stessum si piccas, boia farabis.

» Ah tibi, ne quel, tibi fis ne boia medemo, ~~

2 Et qui pro centum mundis non essere velles;

sv Bflere pro nihilo nolis. pec porge,
ico

pocum, ff l'3
Forfitan ipfa dies (aldabie, Gnocche, fericam, — è
Dura remolicfeune paicis,& tempore forbay 99
2 Nespula dura die mivescunt, nespula dura
»» Guarda mo-, si Gnocchi poterit mitescere noia
Tu bene cicalas; dottorus, & esse videris::
Sed cicala purem; gicttas nam carmina faxis.. \ «
Al facias i » Gnocche, pl >
»» Extremumque mihi praftes, care Gnocche; favorem.
2» Quem nam ? dil, Ma, ura; facies, quod certe domando
»» Dummodo fare queam, fabo, sta fupra parolam, —
Et potes, & legrus facies. Gn. Dic ergo,quid optas
Eft mihi botazus vinetti, Gnocche, rubentis,
xs Quod difamoratis posset rubare coradam,
»» Illius humore taze cum plena plaaura cht,
Saltitat, & brillat, brillando lumina frezaty
Et rubor in vitro liquefatti more rubini,
»» Ac dicto citius spumat; hune inde dileguat
» Puri sbottigliata meri vis fernida, qualis
»» Cum soffiat Boreas, nubes sfrattare per auras
»» Cernitur, & Calum laté purgare ferenum.
ay Sat scio,, si nafum preettabis ad ante bicherum',
»» Optabis fieri totum te, Gaocche, nafonem; $ ”
»» Piccantum retinet pulerum, garbumque galantum, ©
2» Quod refucitaret mortos. De hoc, questo,pochertum «
»» Gustes, ante tuum claudas quam tofte fiaum',: sie
xy Atque mei hoc portes extremi pignus amoris, f
x Vis rechem chi 16? Gn, Reches, fed frettola paffym. - P
» Nigotta proderic, cum fim piccandus adi ys ta
»» Auamen hanc lafles, dum torno, Gnocche, cavezam, «
» Ne te gire viam tua tantum spafima cogant,
»» Et fine gustando vinum, morire, galantum',

































VOLER bene a facea. Portar granditfimo affetto.. BE' f
Va pe' fatti tuoi. Cioè vattene,e bada a te, Res tuas ribi babero;



riti anticamente alle mogli, quando secondo le Romane
Vedi sopra C, 5. st. 57. 7 —





ah
8S Geta












SETTIMO:\CANTARE: 349

| NON 2 mal da bideca', Non è male ordinario,e che firifani con pocdrimedio,
perché la Biacca; che ¢un biahco cavato dal piombo, ¢d è adoprato da i Pitio-
ri, serve anche per fare un' unguento buono.a poco altro, che ad alleggerire il
ng Lp erat però dicendosi.: Won e mat da biacca, s' inten-
de, Manatee Sri aia:

 CHE waglia una patacca, Che,vaglia nulla. Che patacca è moneta, che. in,
Firenze non vale, Paracomé una moneta di ramevufata in Portogallo, che vale
guattrini, Cos} noi d*una cosa da noi tenuta in poco pregio, diciamo. Wo

ale un soldo, Nonne darei un soldo See

ALLA moda. Vuol dite all usanza, come vedemmo sopra C. 2. st. 54. mrs
in questo luogo vuol dire stravagante, o nuovo y e non più sentito, o vilto, e del
tutto infolito; Diciamo: ceruello alia moda per significare ceruello fravagance, ©
fantaftico; dal mutar che si fa tutto giorno dela moda nel veltire.
— WON si rinniene ne capo, ne coda. Non frritrova, ne il-principio, ne la fine di
wm 5 yeofa. Non'fi fa, non's' intende, o non si ritrova come la cosa si tlia, Vee
—- baput, nec pedes, disse Cic, EB' traslato dalle mataffe del filo, e si dice anche Now
ritrov, che @ il principio della matafla.
ih) «| AL tu quiftione? Antendiamo havere inimicizic.

| HAL eu qualche rigiro? Har ta qualche innamorata ? Che la voce rigiro usata.,
] 'come nel presente luogo,vuol dir Pratica di donne per vizio; che per altro,rigiro

'significa Ripiego, dicendosi: 1] tale fa molte faccende, percht egli ha molui ri-
giré, cive neigh ed oe di vendere la sua roba. Alle volte Gi piglia per

« Vedi sopra C.4. st. 60,
. DENT RO è chi |e pefhe' Quand' uno si sforza di mostrarsi nel vil allegro, ed

'ha travagli da star malinconico,diciamo; Ei fa beon vifo, ma dentro è chi la pefa,

hes dentro sta in altra guisa. Kifus in ore y fletus in corde, Virg. Spem vulti fimu-
lat, premit altum corde dolorem.

5 Heaiore fie in teha, Pensiero, o fantasia oftinata. Vedi sopra C. 1. ff. 10,
PAR ch ei la fnechieli. Egli fla fra i) st, e il no di fare una tal cosa, che direm-

mo Irrefoluto. Dante Inf. 8.

Padi age Che'lsi,e'l no nel capo mi tenzona,

Traslato dal giuoco delle carte, che firdice fucchielare quando si tira fu la carta,

'adagio adagio 3 if che pure è traslato dal bucar col fucchiello, che e una aziones

simile al tirar fu la carta. Qui vuol dire, Pare che questa sua fiflazione lo voglia.

adagio adagio fare impazzire,¢ ridurlo a i Pazzerelli, che e lo spedale,dove si
mettono i pazzi.
RIDVRLO in feffo. Ridurlo alla giufta mifura; Raggiuftarlo, rimetterlo in,

buon' essere: fargli ritornare i} giudizio.. Vedi sopra C. 1. st. 15.

SU spenda la Vita, € vada il reffo. Si (penda ta vita, e la toba. Tratto dal giuo-
0, le ff faole feommettere, e dire. Yada il refto; fo del reffo, Bquic det.
to per figura; perché quando è andata la vita, che e la pil cara cosa, che noi

ha iamo, par che non ci refti quasi altro da buttar via.
g 6 "O cinffo, Perappunto. Ela replica ha la solita forza di superlativo.

—— Catullo.. Af igis magis increbrefeunt. Nell'Ebraico Azeod, che vuole dice af/ai, rol

ot, raddoppiato vuol dire afaiffine,moltijimo.,





|



STAN-








359
STANZA XLIVs.0%
Percio d abiti, e soldi si provvede, |
E da buone [peranze alfno Nardino.

Esce di casa,e mettefi in cammino,

Shirciandofempreinqua,e inlafevede >

Donna di vifo bianco, echermifino;
E se ei ne incontra mai di quella tinta,
Vuol poi chiarirsi,vellae verayo finta,

STANZA

Di modo cl ei non vuol restarvi colto
Ma fharvi lefto, e rivederla bene
E per questo una [pugna seco ha tolto 5
E sempre in molle accanto se la tiene,

 Suegetto y che gis occorra farne prove,
Brunetto date buone speranze.al suo fratello,montd a cavallo; ed
0 un' huomo a piedi, fen' andò cercando d* una donna bianca', e rofladi
naturalmente, e sapendo che tutte le donne hoggi si lisciano, haveva
spugna bagnata,per far con quella la prova,se il colore era finto,04
per molto, che egli cercafle,non trove mai donna, nella quale occorrefie far tal
Prova, perché si conosceva senza farla, che twtte eran tinte y e-lisciate. Quello
colore finto, che chiamiamo li(cio, o belletto,si dice anche frco-, che |
buona a tignere i pani; da i Latini detea faces 5 e l'intendevano
si per questo lilcio, o belletto. Plaut. Moft. 4.118. Vetule edentula,
corporis fuco occultant. EB di qui i Latini per fuco intendono una sorta d
che ricopre con artifizio un mancamento in una mercanzia, ec, onde;
cere,

pk
S8IRCLANDO, Guardando attentamente. Vedi sopraC, 1.f9,
CHER4MISINO. Rofo di Chermisi, o Cremesi, E' il roflo porporino, chef
fa col sangue di certi vermi chiamati con voce Spagnuola Cocciniglia dal Latino
coccineus color, colore di grana, colore vermuelio; ed & il più nobile, ed a 0
te, che si trovi, ne mai perde il suo colore:e da questo nel presente luoge inte
de rosso naturale a perfezione, e che non perde, come farebbe il finto +
o Karmes in Arabico vuol dire grana.. Latino coccum,(econdo lo Scaligero eferti
tazione 325.

D1 quetia tinta. Di quel colore. E' termine pittoresco, cotumandosi
dire: La tale ha una carn i

di carne.
VVOL chiarirsi, Vuole accertarsi.

MALIMW/AN TILE © 9%



4 STANZAKLV. | '
— \

> Che non si mini o si taftri le quai;
Epreso un bud cavalloyeunbuomoapiedey 00 B
Cc

ragione nella quale feno beile tinte, per intendere Belli













'

Pas.

“Wella pare il ritratto

Quattro dita vi lascia fu di loiay
Oe
Chrella par proprio un' Angiolin di
XLV" a pots
Con che paffan le sopraiil volte,
Vedrà: s'il color o se vii

Aa gira girasin fasti ei non yitroyh,

i=

Ss
sa





fucum

a eee

aachil
ea

NON si minij. Non si tinga, Minio & specie di color rossa cavato ¢
8

\> adit
ae 1 si:
no; € miniare € una (pecie di dipignere con finidimi colori sopra cose e
me cartapecora, ec,

S/ lustri le quoia. Si \isci la pelle.

MOST ACCIO infrigno. Vilo grinzolo,

refroigne.

© cresposo,o rinfrignato of
ANCROIA, L' Ancroia è finta una donna brava in un Poema








SETTIMO.CANTARE - 351

~ Regina Ancroia; e perché questo Poema è degli antichi, che si trovino nella.

 lingua nostra, mi do a credere, che quando si dice l'Ancroia,' intenda una vec-
chia. di Berni, de(crivendo la sua serva in un Sonetto dice.
' Lobo per cameriera mia l' eAncroia,
. Madre di Ferrak, Zsa di Morgante,
y w  )  ehicavola maggior dell' Amostante,
phe Baliadel Turco, e fuocera del boia
 Ma pnd esser ancora, che queita voce Ancroia sia un' addiettivo, che venga da
i@, che vuol dire Zotico, e duro dai Lat. corium quali inquoito, fatto duro, co-
il quoio. )
thee Col pugno gli percoffe I epa croia,



ae Da questa voce croio habbiamo il verbo tncroiare, che vuol dire aggrinzare,.¢d

= =
ee

ce
a



SRSLERESE



renee per intender pelle grinza, e fecca, e indurita, come € quel -

vecchie, alle quali però si dice per (cherzo Aduna incroia, che nel parlare
'Pultima lettera di 44ona confonde, e mangia la prima d' ixcroia, viene a
ancroia, che vuol dir vecchia grinzola. Jncroiato si dice un qaoio, che per
flato preflo al fuoco, sia divenuto duro, e grinzofo, ed il simile una carca-
abbruciacchiata. Si dice incroiaro anche un panno divenuto sodo per gli
mi', e lordure; ma di questo'é più proprio incorezzato, dal Lat. corryia. Il
bolifta Bolognefe dice, che Ancroia signitica vecchia, che va crollando il ca-
po, ¢che viene dal Greco Craein che vuol dir croilare. Ma venga donde si vo-
glia, basta che appresso di noi vaol dir Donna vecchia, e brutta y ed im questo
Aenlo e prefa nel presente luogo.
 LOLA, Sudiciume. Terra stemperata con acqua, e ridotta liquida, che cons
altro nome chiamiamo mota. Qui vuol dir quelle materie, che si mettono in sul
vilo le donne y le quali s' imbellettano, Voce fatta per avventura dal L, i/luvies..
. IMPIAST RA, S' unge con materic bituminofe,¢ viscofe come è l unguento.
STVCC.A, Stucco è quella composizione di geffo, e colla,¢ d' altre materie
4tenaci, che serve per riturar feflurejo magagne ne i legnami. E facco è una spe-
cic di gesso, o terra, o altra composizione, con che si tanno le figure di rilievo,
i per fucco intende quelle materie, che le dunne si mettono sopra il vifo per
la faccia, e turarsi le margini del vaiolo,,o altre cicatrici; che il
verbo fuccare yuo) dire intafare, cioè riempiere i buchi, e ragguagliare una.
I¢; donde gli Orefici dicono fluccare, quando con una certa loro lima
detta lima stucca, spianano i lavori d' argento. Sewccare vuol dire ancora quan-
-do wn cibo ci apporta naufea, o i discorsi d' alcuno ci vengono a fastidio.
. ViN* Angioline di Lucca.. A Lucca fabbricano certi figurini di cera, di geffo,
Od' altra materia,a' quali dopo formati danno il-colore di carne con un rotio lu-
Sirante; per questo d' una donna lisciaca diciamo; Pare un' Angiolino di Lucca.
Gosi iGreci, che le belle persone afiomigliano alic statue ben fatte, le chiamano
Agalmata,e Properzio,ditie che il colurito del vifo della sua donna era giutto co-
»me quello, che si scorgeva nelle pitture del famofo Pittore Apelle. Quals Apelicis
<¢f color in tabulis. in un' Bpigramma Greco una faccia impellettata, e lilciata,
con clegante bifliccio vien detta Profopeion, con Profopon, cioè maschera'y © non
faccia « Vedi Cel, Rod. Leet, antig. lib. 29. C. 7.

NON







55%
NON vnol resparni colto., Non ins rimanare q
ST ARV1 lefto. Stare'accorto, O.avvertito, 9) © 96
G/RA gira, Cammina in diversi luoghi; cammina
IN fatti, E' lo stesso, che in somma, o in pen? L,
STANZA XLVIL..
Dopo che tanto a ricercare e itoy
Che i calli alc,. ha fattoinfulafellay
Giunfe una fera al luego d'un Romito,
C' areftar L innita nella fan Cella y quel delle ¢
A lui parne toccar il Ciel col dito Di che speffa ciascun
(Per non haver a iar fuori ala stella) Stettero a croschio ii
4M pafjar dentro, ed egli,e il feraitore,
Ringragiando idbuon buom di val favore,
STANZA XLVIILL
Vestia di bigio il Vecchio Mdacslente,
Facendo penitenca per Adacone Dice chi sia echo di cafael
E perch' ei fu nell accattar frequente, Non per fa conta,mad! un si
Per nome si chiamo fra Pigolone. Del quale infino alt' Tad
Costui y( com' io diceva ) allegramente Perché gli pare uscito die
dn Cella raccetto le lr perfoue, Non fifas ei fifia pile carne,
Spoglia il cavalio,e gli trito la paglia; Così piangendo in far di cid'm
Sul desco por distele la tovaglia, Per la mmuta conragl la.
Capito Brunetto una fera alla Cella d' un Romito, dove esseada
tato, stando a tavola raccontd al Romito ibcafo del Fratello, dicendo
fuora per far servizio al medesimo suo Fratello,
TOCC AR il Ciel col dito, Confeguir I imposfibile. Ap
ST AR alla fella. Dormire all' aria; a ciclo scoperto; alla fella diana 5 Lat
ub dio, NNR
MACILENTE. Mal fano; Cioè magro per lo stento, e giallo i
ione.
: EV frequente nell' accattare, Duc tefti di mano dell' Autore dicono uno
te, edé) ultimo; ¢l'altro servente, equesto e la prima bozza, e se i
¢ Iraltro può stare, io pighierei ? ultimo, perché in fultanza vuol dire che
€ra attento,¢ diligente nell' accattare, e sempre chiedeva, che da,
importunita, s acquiftd il nome di fra Pigolone che così chiamiamo
sempre chieggono, e che mostrando una certa ingordigia di roba,si
pre dello stato loro. Pigolare €il verso de' puicini, che beccano. Lat.
Spagn. piar dal fare pio pio, che così e il lor verso. ovat
DESCO, Tavola, sopra la quale si pongono le vivande, quando si
dal Lat. discus, che e pierra rotenda, o laftra da scagliarsi, Vedi sopra!
TVTTO accatrato. Ogni cosa havuta per limofina.
FIORITO quanto un Adaggio. Fiorititimo;percht ii mese di 'Maggio' la s
ne de i fiori; O pure perché queili, che vanuo a cantar maggio,porta
ad aealbire tutto pieno di-diverti tiori, il qual camo @ albero ch
Bio » ° maio. Diciamo: vizo foecisay quando o per ctier ab ton





































SETTIMO°CANTARE, 333
per altro mancamentoj il vino dosi nel bicchieresha 'nella superficie minu-
tissimi frammenti d' una cerca specie di muffa'bidrica; che è il panno, che si fas
dal vino, equ 'chiamano fort; si che quis" intende, che il vino era vicino al
fondo dell ', o havea altro mancamento, che' produce la detta muffa; se be-
I 'chevoglia dire Vino isquisito; perché /ro itoeattribuco di perfezione in tut-
4 syeccetto che nelvino, che l'esser fiorito è segno d' imperfezione.
' centuna bette. Questo numero centuna, benché sia determinato,
dee | t per indeterminato; e vuol dire Cavato da infinite botti di coloro,
}haveyan dato per limofina, E questo pure è imperfezione del vino, che
perde lo (pirito., e la bontà in tanti travafamenti, e mescolamenti,,
 STETTERO a crocchio. Stettero chiacchicrando.. Vedi sopra C, 1, st. gt., €
Bietemin così detto dallo strepito, che si fa ridendo, e chiacchicrando
ielle conversazioni di trattenimento, perciò dette Crocchi, Dal romore similmen-
 teedal faono che rendono, sono dette da' Prancefi Cloches le Campane. Così
i — 'saccordano nel rappresentare con l'arte i semplici suoni inartico-
jt lati che (ono un' inalterabil linguaggio della natura. '
ed batte dove il dente duoie, Si dilcorre empre volentieri di quelle cose,
j@ dove hala paffione', o sia di gusto, o di disgutto.
' a il campanello. Parlaya sempre lui, Questo detto viene da i Magiftra-
/
Cd







Bet
*

tidi ¢, ne i quali uno dei Colleghi si chiama il Proposto, e questo sempre
Sa aj litiganti, e chiama, e licenzia dall' udienze, ed i compagni
' a cheti; e questo Proposto tiene allato alla sua seggiola un campa-
nellowE da quefio, quand' uno in una conversazione sempre parla lui, diciamo:
yp Bi tleneil caimpanetio.,
APINCRESCE fino all anima' Gli ho grandissima compaffione'; Vedi sopras
in questo ©; st, 26. Mi'dilpiace, mi pefa. Dante Inf. 6,
se RDS “Mi pefa'st, ch' a lagrimar m' innita,
DU Greco dice Achthomai, mi dolgo; e 10 Spagnuolo similmente pe/ame. Onde quel
che'in Toscano si dice' dare il mi dispiace, esso dice, dar ef pefame: La stessa forza
ha MMe Y AP inere/ee, quali mibi merave/cir, secondo il Ferrari; mi grava, e»
BS Pope nage Amore e pefoscominciò Dante una Canzone. £' m' incre/ce di
” Z PMO. Sus '
WON fap ei ff fra carne'; o pefee. Non fa quel ch' ei si sia. Noné in cervelio,
Non ha' vO conofeimento. Awevo pefee dicevano gli antichi un' biomo /Pra-

Eee,

ae









wm math
ip! neces ANZA LL STANZA LIL
ee Sta Pigolone attenro a collo rorto Egli ha un giardino posso in un bel piano,
i Ad 'ascaltarlo s€ poi ch' egli ha finito; Ch' e ognor frorito,e verde tutto quato;
“a: F igiiuol,risponde a Ini 5 datti conforto, Giardiniero non v' t, ne Ortolano,
oy + Bfappi, che th fei nato vestito, Che a entrarvi nefjin pus darsi vanto,
» Che qui? Pbbnom falnatico Magorto, Da per se lo lavora di sua mano,
a Ch'e un bestione, un diavol travestito, E da se (0 fondo per via a! incanto,
o & “Che se te lo vedeffi vb eglie pir brutto! Con una casa bella di frupore,
ye Balke a suo tempo conterotti it rxtto, Che vi potrebbe ar  Imperadare.
5 4 vy STAN:









354 MALMANTILE

STANZA LIIL
Ma io ti uno dar' adeffo un? abbozzata
Lui presto presso della sua figura.
Ei nacque a' un Folletto, ed' nna Fata
A Fiefol n' una buca delle mura,
Ed e si brutto, poiche (a brigata
Solo al suo nome crepa di paura;
O questo e il caso a por fra i nocentini
ed far manciar la pappa a quesbabini,
STANZA LIV,
Oltre ch' ei pute come una carogna
Ede pin nero della mezza norte,
Ha il ceffo d'Orfoye it collo diC arogna,
Ed una pancia » come una gran borte
Va in sui balefirs, ed ha bocea di fogna
Da dar ripiego a un tin di mele cotte
Zanne ha di porco,e nafo di csvetta,
Che piscia in bocca,e del continuo getta,

STANZA

ea lasciando per hor  altre da parte,
Cocomeri vi son di certa raza,
Che chi ne puo haver uno,e poi la parte,
Vi trova una bellissima ragazza,

Pigolone incefo il bisogno di Brunetto, gli da animo con dirgli, che
huomo faluatico ha quivi un' orto,dove son cocomeri, che tagliandoli n'esce suo-
ra una bella fanciuila, la quale chiede da bere, ma sce feglida., ella
Deicrive ancora in queste quattro Orta ve la qualita di
SE/ naro vestito, Hai havuto buona fortuna, o qu wd
questo termine per esprimere,quand' uno desiderando qualcofa difficile a y
s abbatte accidentalmente a trovarla per appunto,,come ci la desideraya, eda
propolito del (uo bisogno. Dicono te Levatrici., che talvolta nascono ban
con una certa spoglia sopr' alia pelle, la quale spoglia non si leva loro subitos
ti, ma si lascia, e casca poi da per se in processo di giorni.; eral creatur
si dice vara vefiira ed € preso per augurio di felicita di quella tal.

ha dato origine al presente dettato.

VN diavol travestito. Vin diavolo immascherato da huomo; intende un'

brutto, quanto il Diavolo.
BELLA di flup
yvede;
VO!

I Pittori dicono Abbozzare

Jerto e Gianni Schicchi, dice che i P

STANZALVL. —
Dell' ofa poi ne fa si ce

ere. Bellissima mirabilis vifu. Tanto bella, che fa flupire
ma per venire la voce /tupore dal latino,può ognuno intendere il suo}
IG LIO darti un' abborzata, Cioè ti vo;

Y > quelle prime pennellate, che danno in.una tela;
trove, dove voglion fare una pittura. Vedi sopra C. i uy

FOLLETTO. Vno di quelli spiriti infernali, che dicono che stieno per Hari
1) Ferrari nell' Origini alla Voce Fuile,citando Dante Inf, 30, efi disse, quelf




z





E della pelle ne fa maccheroni, =

Diente in Jomma v't, che vada ma
Sreche Brunetto figliuol mio, tn femti,
Ch' egli è un cattivo,ed orrido ammale,
Hora torniamo a suci scompartimemi,
Ove son frutte buone quanto il
Vaghe piante, bes fiori, ed altre vole
Com' to ti potrei dir maraviglofe,
LVI, AYE
Che per esser aftuta la sua parte,
Dirattiche tu gli tpia una fun rare,
A un di quei fonti li si chiari,e freddi,
Ma se la ferni, a Lucca ti i

efto Mi i
ota. cadena

nis

glio de(crivere alquanto, 0
st, 41.

?olletti sono la/civ) genij ac Lemures rifu aS
domos implentes,. = E.













SETTIMO CANTARE. 335
FAT A. Vedi sopra C, 4. st. 45. i
eA FIESOL o una bucadelie mura, A Ficlole si veggono ancora alcune reliquie

delle mura di antica Città, ed in ci frammenti di muraglie fra l'altre si
 yede una gran buca di »od'altra cosa simile, la quale dalle donnicciuole è
- ereduta, ed è dataac ai fanciuili per abitazione delle Fate, ¢. pero vol-
ms @idetta /a buca delle Face. E questa è quella buca, nella quale dices
-» che Magorto era nato d' un Folletto, ed' una Fara. Angelo Poliziano
~ al titolo Lamia dice: Vicinus quoque adbuc Fefulano ruscnlo meo lucens Son-
ticuins est, fecreta in umbra delite/cens, ubi fedem esse nunc quoque Lamiarnm narrant



a imuliercule « Questa credo sia quella caverna, che-oggi si chiama /a fonte fotterra
¢,  luogo orrido  e (paventevole, ma sempre pieno di limpidissima, e freschiffims
lt NS TEV ' 2

ry “SNocewr 17 « Cioè quei ragazzi, che s' allevano nello Spedale degl'Innocen-
so erensa +5: ton

ca ~ CAF AR mangiar la pappa a quei bambini, Così diciamo d' un' huomo, o donna
nil a e brutti, quaGi che sieno come il Bau, la Befana, e simili larue in-
ga  uentate dalle Balic per render i bambini ubbidienti, e fare che per il timore man-
- Cai wa. Vedi sopra C. g. st. 3. E.questo putire da i Latini era espreffo
0 co} paragone, perché dicevano vixum cadaver. 11 Monofini.

ut nero della mezza notte. Negritimo, piii nero del buio,



 VAin fui balefri » Ha le gambe fottili, e torte come sono i baleftri, compa-
'fazione vulgata), sendoci una cantilena di Balie, che dice.
° Ben ne venga Mignamau,

Saif oe j Cha le gambe a baleftrucci.
O81. 3 e Sbilenco, dicefi chi ha le gambe torte; e ancora Aver le bilie;
tratta la similitudine da certi legni torti, o randelli, co” quali i vetiurali legano
itetto, e arrandeliano le fome; da loro dette bilie.
 BOCCA di fogna. Alla bocca delle fogne maeftre, o principali, che ricevo-
'no acqua delle strade,quando piove, e la conducono nel fiume d' Arno, è figu-
rato up ma(cherone di pietra, il quale ingoia l'acqua ed oga' altra sporci-
zia., € di queste intende il Poeta; e da questo diciamo: Bocca di fogna a uno, che
mangia, ed ingoia ogni sorta di cibo, se bene sporco, senza distinzione, o ri-
yaleuno. Latino bedwo, gurges. Queste fogne in altri luoghi d' Italia sono
lette Chiaviche dal Latino Cloaca.

DA dar ripiego., Cioè dove entrerebbono tante mele cotte, quante n' entrereb-
be in un sina, che quel gran valo di Iegno, entro al quale si mette 1' uva pigia-
$a bollire per farne vino. rare
», ANNE, Denti: Propriamentes' intende di quei denti Junghi, che hanno i
Signali, i lupi,i.cani, ec. che noi li chiamiamo anche denti Adac/tri; o Adaeftre.
Vedi £ aes, st. 64. Borle & meglio dir /anne, ed e pill conforme all' origine,
Onde /ubfannare buriarsi d' uno ridendo, in maniera, che tutti identi, come di-
$e il Boce. si poteflero trarre; mostrando le sanne. Daa. Inf, C, 6,

Quando ci feorfe Cerbero il gran vermo,
Le bocche aperfe,¢ —— le fanne.
yz



SER GF BRLLSES weeks

Se

it
4

eC,












356 MALMANTILE?D*)2

¢C.22. E Ciriatto, a cui di bocca nscia > Vo
D' ogni parte una Janna come 4 porcoy eres Pepe)

Gli fa sentir-comel' nie sdrucia. » 8
NASO, che pifeiain bocce, Cioè nafo aquilino' che ha la a
la bocca, e pare che vi colidencro. vb~orn saokay
BERLING ACCIO « i Giovedi geaflo, chee I ultimo giovedt

detto Serlingacciv da Berlingare, che vuol dire bere, e mangiare',

mente, come si fa in quel giorno: e così Magorto, quando pi a
faceva conto, che quel giorno fufle il Berlingaccio, toleani:

menti, pappalecchi, e Gorxeviglie,daligodere, Latino ga





ennizzandolo con
c wifare, conic si
antico Glofiario, onde lo Spagaualo gozar., godere pel nostro gavayzare
ti finonimi, che voglion dir ghiotcornic Bocc, g. 8.n. 2. Sé cues
pis volte insieme fecero corzovighe sec. << i8927 toil?:
MIGLLACCIO, Sangue di porco, o d' altro animale mefeolato'con
farina', e poi frittoynella padella a uso di frictata'da aleuni* Latiot
chus; se bene questa era una composizione-dicacio, e falamevdal
che vuol dir cacio, etarichas, che vuol dir falame. 1
STVZZIC ADENTI, Nettadenti: Sottilissimi,ed acuti stecchi
d' offo, o d' altra maceria per uso di nettare i denti + Latino
BYONI quanto il fale. Saporitissimi.. Vaaevivanda con molto fale'
rita, che vuol dire il contrario di sciocca, oinGipida ye: senza Yale 7 ¢
faporito è meglio al gulto, che Pinfipido,'¢ pero per faporito i i
¢ dicendosi; buoni quanto il fale, s' intendefaporiciimi, cioè gustofissimi

fapore. es
TCOCOMERO. Specie di mellone acquofo di fapore'dolce, che nella
stagione calda per rinfrescarsi. In moiti luoghi d' Italia chiama
così la chiama il Mattiolo, e dice che era incognita a iLatini, se bene G crovas
cuckmis, ma intendono il cetriuolo, che pure in alcuni luoghi si chiama eeeome™
Anguria, dice il Perrari, e detta quasi cucumus anguinens 5 © così questo nome
che era proprio del cecriuolo,per mancanza di vocabolo fu tratto a
fructo, che noi Toscani chiamiamo cocomero. i
e4 LVCC Ati riveddi, Questo detto significa Non la vedrai più,
Buoni da Lucca nel suo teforo de“Proverbi dice, che havendo un
Lucchefe veduto un Gentilhuomo Pifano a Lucca,usd seco cortesia
desinare a casa sua, dove condotto, fu trattato con ogni sorta <r
titofi il Pifano, e ritornaro alla patria,avvenne che fra poco tempo'
andò a Pila, dove paruegli conucnevole vilitare il Pifano suddetto: Ts
pero alla casa di eflo, dopo haver molte volte buflatosal fine sa ffaccit
© gli disse che nom lo conosceva; onde il Lucchete disse: 2: e4 Lucta ci-vede
Pifa ti conobbi, © con questo fiticenzid. Così forive un Lavcchefe:, ma tt
voltano il proverbio dicendo: 4 Pifativedat pea Lacca si i
grato,¢ scortefe quello da Luccaje:non quello da Pita') Seibene il
era ne Lucchefe ne Pifano nella sua En. Te. C. 3. st. 4. dice:;
E dicon spefo altrnis Ti veddi a Lucca,



































—nw ese E-epeEE PBB TL ete










SETTIMO CANTARE! 337

STANZA LIX.
Efe en ean
Dirad, che tu buon Cavalier non sia,
entre conforme all' oblige non vft
Servitie con le Dame, e cortesia.
ea lascia dire,e tien gli orecchichinfi,
Non ti piccar di cio, sia pure al quia,
Gracchi a sua postayty non le dar bere,
Accio non fugga;e poi ti sia il dovere.

Con questa, che fara farta a pennello, Vientene dunque meco,e sia in ceruelo.
| Come te cerchi, lenerai dal cuore Cammina piano, e fa poco romare,
—  Ogni dogti i affanno al tuo fratello, Chefee' ci sentea forte, scuopreil cane,
16 --Edioten' entro già mallevadore. No occor' altro;noi habbiam fatto ilpane,

ite seguita:a narrar la favola del Cocomero, ed instcuito Brunetto di co-
a' ome contenere, perché la fanciulla non gli scappi, s' avvia con esso alla
volta del giardino di Magorto. -
s far conto @ haveria vifea. Ti puoi dare a credere d' hayerla veduta qua.
4: i a.vedere, perché non la rivedrai pill. 2
Al uno stivale. Refterai beffato, Retterai uno scimunito. Vedi sopra
yg  (Geqfero, LGreci dissero Bagas con/fieyti, da un tale detto Baga, o pure Bagoas
a} 'nome da Eunuco; che fu un' huomo infipidissimo; Donde poi noi diciamo Bag-
sg) 60» © Baggiano, a ua' huomo scimunito se non forse da Ha/eo,, e da Habbano; 0
da Bageiano sorta di fave maggjore dell' aitre.
of va di forche 5 edi moine, Vina quantità grandissima di finte carezze,€
'Nezzi; i Latini dissero blandicie, Ed in questo proposito tanto è dire far le forche,
5 « dexai, quanto mome, significando cutte tre una sorta di Jufinghe fatte con
wit Bti,ocon parole, e sono quasi lo stesso che adulazione; perché ancor le»
dt “nine sec, son atti, gefti, e discorsi, i quali contengono, se noa falfe lodi, co-
»meccontienc ? adulazione, almeno falfe dimostrazioni d' affetto affine di com-
eo. jiacere ye di acquiftar ia grazia di colui, a cui si parla,¢ queste son proprie di
ie di femmine, e l'adulazione e conuentente ad ogat sorta di persone,
ma è sempre indizio d' animo vile, ed esseminato. Ll Landino nell' esposiziones
a Dante Inf, C. 18. dice, che gli adulatori in lingua Fiorentina si dicono moinieri;



f=

¥ »Ma questa voce non si dicendo in oggi, ac avendo autorita di Scrittore nell' an.
si tico, mi fa credere, che il Landino la derivaile a capriccio daila voce Fiorenti-
i na Meive non trovando parola corrispondente alla Latina ddulatores, I) Cala

nel Galateo volendo mettere in volgare il Latino ada/ari, lo esprelie colla paro-

SSL.

» la Piaggiare, L Bini in lode del mal Francese dice:
uhangl lo non roppi già mai; ne carfiiancia 5
Machi mi va con si fatse moine,
Vorrei porergli sfondolar la pancia.
 La Stor. di Semifonte Trattato 4. Quand! altri ha ofefo un fupremo, non e da si~
< darsi di lui', ne delle sue aftute moine,¢ Lufinghe.,
-  NON+i piccare. Non v' offendere none' adirace; Non cntrare in gara; Non
ats ti



©!

=




as

ae
mI cL

358 MALMANTILE 4) |

ti flimare ingiuriato. Vedi sopra C, 3. stan, 20. Tanto il Franzefe quan:
to lo Spagnuolo Picar voglion dire Pugnere; forse da Picca;












colina

wale Omero appella nyttein, cioè pungere. Vino piccame & que
iecda Ȣ che punga, lee they cP amma
bio; Tienle caro. ll Persiani Tesan taeda bea
Va menati l agrefto 5:
Ceruellaccio peftato per Lambiceo ?

Che 'l tuo mordente ha trove poco appicco.

Di questo iv non mi picco “ct

Che s* io non ho la nobilta a bigonce, yet

Mi basta ds non esser a' undici once, (cioè bastardo) —

PICC ARS!, Vuol dir anche persuadersi, o darfia creder d' etfer eccellentes
in una cosa, come piccarsi di bravo, di bello, di dotto, ec, e vale quanto esser am
biziofo, o haver ambizione. sx pee

SSE £8 ® Ss ome



ST-A al quia, Sta sodo: Non badare a quel che ella dice; enon tilafeiares | tf
suolgere, o persuadere a darle da bere. Dante. State contenti, wmanagenits, |v,
al quia, ' hive ow

GRACCHI a sua posta. Gridi, cicali, eflami pure quanv'ella vuole; lasciala | yy
dire, la(ciala cantare. Quand' uno vuol quaicofa da un' altro 5 ed ' ty
mandarglieia,¢ colui non glicla vuol dare,suol replicare a i detti di: Rs
chia, gracchia; quasi dica: Tanto mi muove il tuo dire 5 quanto il geacchiar) | ti,
d' una cornacchia. Vedi sotto C, 8, stan. 64. far Fup

TJ (tia il dovere, Ti succeda quel che w meriti. aa

SAKA fatta a pennelo, Cioè fara similissima, ed appunto come cs

T” entro Mallevadore, Te ne afficuro. Ti fo sicurta, che leverai u
Fratello questa frenefia. Adadevadore e il Latino Fdeinffor, quasi afidarore, afi
curatore; detto Maiievadore secondo il Menagio, dal /evare in alto la.
segno d' afficurazione, Lo Spagnuolo lo chiama Fiador, la qual voce in
co Vo)garizzamento Toscano manoscritto delle Vite di Plutarco tra
lingua Aragonefe, refid senza interpretazione insieme con alcune altre y il)
guiva in gucfte tali traduzioni, o per vezzo del traduttore, o per i
gine, o perché non ne fapefle pi la. Caro mon volle il diposito, ma fiette t
tutti, wah:

NOL habbiam fatto il pane, Noi habbiam dato nel laccio. Noi i
vuro la disgrazia senza rimedio. Diciamo ancora; Voi habbiam fritto, Vou
fouo C. 8. fan, 54. sega

STANZA LXI. STANZA LXIL ©
Zitti dunque 5 nefjun parts, o risponda + A casa lo strascina ete lo fica |







eAndiamo che e's' ha a ir poco lontano,
Così va innanzi, e I altro lo seconda,
Oikjernitor lo segue anch' ei Piano piano,
Ma quel Demonio, che va, sempr' inroda,
Gii sente, e gli vnol vincer della mano,
Perché gli asperta,e il vecchioc'alla fiepe
Vien primoghiappa/,come dir: pepe.

7s facta,e conlacorda ve

E fatto questo a un canapol'

Che vien dal palco oueea a vertas
E per pigtiar il refto della critthy
Ejce poi fuora, ma ncl fate'
Che quand ei prefe q '
Ad aspettarlo havute






SS ee. ew SE FF ee eR Keke eee













SETTIMO CANTARE: 359

Soot a Selb SRRSTANZA LXE
| Edoggimai si trovano in franchigia, Sfogarsi intende,¢ a quella veste bigia
ene Vuole un po meglio feardalfar le lane,
Kabel: june, en'érantoin valigia Percio /u verso il bofeo col pennato
Che ne manco daria la pace 4 un cane; A tagliar un Quercinal va difilato.

Pigolone esortando i compagni a far romore,s'avvia con essi verso il giar-
dino, ma appena giun(ero alla fiepe, che Magorto gli senti, e prefe il Vecchio,
che era op vicino alla detta fiepe, e condottolo a casa lo ferro in un facco,¢

palco, tornd per pigtiare il reito, ma non gli trovando, fen' andò

| alco 5
al hosco per fare un buon battone, col quale haveva ia animo di bastonare Pi.

 2ITTL, Cheti, Vedi sopra C. 1. stan. 10.

LO seconda, Gli va dietco: Lo seguita, Petr. Canz. 8.

pies Ed un gran vecchio il secondava appresso.

 EB spelfo in ronda. Gira per orto facendo la guardia. Ronda dal Lat, retun-

dus; dal quale è fatto il Franzefe Rond ritondo.

 GLI enel vincer della mano, Vuole esser pis diligente, e più lefto di loro; gi
wool prevenire. E traslato da quei givochi di dadi, ec, ne 1 quali il punto ugua-
Ie noné pace, ma vince quello, che e il primo a tirare; per esempio, io fond il
primo a tirare, e scuopro fei; tira il secondo, e parimente scuopre fei, e se be-
neil punto e uguale, vinco io, che sono stato il primo a tirare; e questo si dice
Vincer della mano, perch colui » che e il primo a tirare,si dice haver la mano.
tanto basta ai noftco proposito, f€ bene moiti altri giuochi di carte danno questo
Privilegio alla mano.

s SIEPE, Chiudenda, o riparo fatto di pruni, ed' altri sterpi agli orti, eda
icampi, E' voce latina. Franco Sacc. Nov. 83. E giungende dove era la vigna,
qucftaera molto affoffara, e con una buona fiepe.

CHLARPA fu, come di pepe. Piglia subito,¢ senza contrafto, o fatica alcu-
na. Credo, che questo detcato sia corrotto,¢ che si debba dire: Come dir: pepe,
che è facilidimo a profferirsi, come tutto labiale,¢ di sillaba raddoppiata; e che
da questa facilita si cavi il fgaiticaco di facilita in dire 50 fare una tal cosa, per-
ché'a dire; 'Come di pepe non ci fo trovar significato, o fale alcuno. Chiappare
dal pecaaere. Da Arripere fece il Bocce. Arrapare, Nella Lettera del medefi-
mo scrittay a Meffer Francesco Priore di Santo Appostolo, E fimalmente can
più largo parlare ferivi, che io non doveva così subito il partire, anzi la fuga dal tuo
Mecenate arrapare, Volle esprimere il Lat. fugam arripere con dare a quel verbo
wna terminazione Toscana. Così #rappare abbiamo fermato da extra, € rapere.

STRASCIN ARE. Stra(cicare un materiale per terra senza follevarlo,o por-
lo sopra veicoli. Lat, Trabere.

FICC-ARE, Vuol dir mettere una cosa in un recipiente con violenza dal La-
tino figere,

 CRICC-A, § intende conversazione, o compagnia di più persone: metaforico
da quei giuochi di carte, ne i quali tre figure uguail insieme si chiamano cricea,
come tre Re, tre Dame, o tre Fanti.

» | AVRIANO banuto del bue. Haurebbono havuto poco giudizio, poco avve-
be

.




———

360

SI trovano in franchigia. Si ttovano in sicuro, in luogo, dove n

refi; che franchigia intendefi un luogo immune per pri

tincipi, Lat, asy/m y che pure alcuai Toscani dico alte
ine eT

mofi di yoci nuove,dallo Spagnuolo dicono amparo,
RIMANE un bel minchone. Riman buriato, riman beffato. weno
stan. 15. si dice ancora reffare uno fivaie sopra in questoC, spo
E in valigia, Erin collera. Si dice anche im bigencia yin' nel
nel gabbione, ec, come habbiamo notato sopra C. 6. tans 41. &
un' arnese di quoio, entroal quale si mettono cose necefiarie per la
fona, quando si viaggia, e's' adatta in fulla groppa dei cavalo, e quelli
vanno a piedi la portano in su le reni, ma questa propriamenfe si dice
NON darebbe la pace a un cane. Non darebbe la pace a Veruino; cioè
stizza, o collera, che egli ha, che se gli venisse avanti un' amico,
be come nimico, perché la.rabbia gli ha fatto perdere il conofeimento, Si dice
xn cane, © non un' altro animale, perché l'alo nostroé di dire + Wow
do guardi in vifo; Non ha cane che cli vogiia bene; nom ha cane che lo foccorra 6p ai
t#, € questo perché il cane e timbolo della fedelra', ne-si trova animale pill
liare, ed amico dell' huomo, che il cane; e pero dovendosi pigliare un'
vicino all' humanita, e profiimo al ragionevole; nel prelente luogo 5
i sopraddetti proverbi, pigliamo il cane. ta
SFOG ARS/ intende. Si vuol cavar la rabbia. Vuole sfogar ¥ ira;
all' ira, come si fa del fuoco, del fummo, che gli si da apertura,
VVOLE un po meglio scardaffar la lana A quetia veste bigia, Scardaflar'
vuol dir battere, e pettinar la lana; con denti di fil di ferro a i an.
che cara: ( dalla similitudine del cardo erba spinofa ) raffinare la lana, accioeeht
si posia fiare. Vedi sopra C, 3. stan. 60. e per metafora significa baflonare ind;
¢ però qui dicendo, vole scardaffare, ec. intende Vuol battonare ue
torna bene l'equivoco, perché par che voglia dire rilavorare,¢ di;
re la lana, con la quale e fatta la veste di Pigoione. Li Puici nel Morgantes: ”
Adattera it bartaglio ancor dal Cielo ee
In qualche modo a feardaffargli il pelo, a
PENNATO, Coitcllone adunco, il quale serve per potar le viti 5 app
forte così da quella crefta, © penna tagjicnte, che ha nella parte di
nio Marcello alla Voce Bipennis dice così: Bipennis manifefium ef id we
utraque parte fir acutum, Nam nonnulli gubernaculorum partes tenuores ad D
mulitudinem pinnas vocant eleganter, Pennato ancora è epiteto, che e stato'
Latino a' yolatili.. Onde tcherzando sull — » ditie 1) Boece, Gi ]
18. / vidi volare i pennati y cosa incredibile a chi non gli aveffe veduti, EB n0i'
a raccontare gualche novella, per renderla più credibue, factiamo
segnito nell' antico assai, quando gli huomini eram più semplici', &
che volavano i pennati, Palladio de Re ruftica tit. 43. discorrendo de'
deContadini vi nomina è pennati,e gli chiama falces a rergo acutas, atque laiitl,
DIFILATO. E jo fietlo che Andar di vela,di filo, addirinura,
C. 6. stan, 10, Vedi sopra in questo C, stan. 5.

ob”









MALMANTILE!D 04%

eaEFRS rere &F Peerae

oe. pers er kro 2e028 825 5




Bot
STAN ZA LXV.

Ed ei le corde alfacco aun tratto feialte,
£ fatto quel meschino uscirne fuore,
Che lo ringraria, e bacta mille volte,

el cht del vecchio. E fa un falto poi per quell' amore,

chiufo in quel sacco iltrova pohe >. > Vi merce il can cin e guarda le ricolte y

oe. 4 mal por! Dandogti aint, ed ezli se il servitore,

Poi con i piatri ye pie vasi di terra

Due fiaschi di vin rojo, e lariferra,

LxVL
Quando Magorto in gik viene a ricifa
Con una fhanga in man cotanto fara,
fesse crspands delle rifa Perchee gli par mill' anni con quel tronco
wove con quegli altri firimpiatta; Difar vedere altrui ch' ¢i non è monco,

0, che stava naicofto a osservare, veduco partirii Magorto, corse alla

9 e trovato il vecchio nel sacco jo cavo., e vi mefie dentro il cane con

di terra, e duc falchi di vino, e rattaccatolo come stava prima si na.

oo vedde venir Magorto con una grande stanga in mano.
felice, ' paroia di commilerazione,come meschino,¢ simili.

YANDOS! 4 mal porto, Trovandoli a cattivi termini.

arrucolada pozzo, Carrucola e una catiecta di legno, e tal volta di fer-

alla quale e impernata una gircila scanalaca, e (ope'a tal girella s'a-

, o catena per tirar fu pefi con facilita, e questa carrucola si tiene co-

ente appiccata al pozzo per tirar fu acqua, ed il moto, che fa cal girelia

ta cagiona per lo più strepito, al quale il Poeta atiomiglia i sospiri,
; 'igolone.
ae SFA fais, per quell amore. E' un detto faceto, col quale s' esprime la gran-
a, e contento d'alcuno: E tal detto viene da quzi Ciechi, che per
i Popolo fanno nelle piazze giocolare i cani, e fra gli altri giuochi gli

s e al bastone con dire: fa un falto per amor d' un pane, ed il cane tutto

» © per il contrario dicendogli; /alta per uaa mano di bastonate, il ca-

ein atto di mordere, e non saita; ed il termine per qucil' amore figati-

lazione, O in riguardo; come Lo fo la tal cosa per amor tuo, s! in-

bh tende Io la fo in riguardo, o a contemplazione tua per l'amore ch' 10 ti porwo,

 SERATT-A, Vedi sopra C. 5. stan. 13.

flere f delle rifa. Rider gagliardamente. Rider come fece Margutte, che

baenpp:s secondo che favoleggia il Pulci nel suo Morgante; Ll' verbo

a altro vuol dire allentarsi gi' inceltini, vale anche quaato /eoppiare,

parities pur si dice: Scoppiare,¢ morire dalle rifa, Bd & quel re quati che

“habbiamo decto sopra C, 3. ttan. 65. Li Pulci nella Beca dice:

Petty wit ' Ta fet nel letto, e crepi dake rifa.

st enone Sitorna a nascondere. Vedi sopra C. 20, stan. 60. e sotto C.9.

bis he fa cht ek s* appiartd miffer gli denti.

ia era i emi a Trattaco —— dice: Quejte cose ho cavate da un six

bro
































oraeeaal


































362 MALMANTILE —

bro del Comune, che fu impiattato da uno de' Buonhuomini,¢
4 ricifa, Senz' intermidione; senza fermarsi, a p
difilato detto poco sopra Octava 63. antecedente. I) Pulein
ES io mi metto a cantar a ricifa, a
COT ANTO fata, Grofia in questa guisa. Vedi sopra C. 5.
stan. 36. Tam
Par veder, ch' ei non è monco, Far conoscere ch' egli ha le mani; 0
non ha mancamento alle braccia. Jonco vuol dir uno che ha manco
tutte due le mani. Lat. A¢ancas,
STANZA LXVIIL.
errriva in casa, ¢fbracciafi,.€ si mette Ed ei, ch' e 1 fulle furie non vi
( Serrato V uscio ) con il sue randello Che infin.ch' ei non fisfoga
Sopr' aquel [acco afar le sue venderte, Sta intato il vecchio all'uscio,
Suonanao Zuant'ei pao fodo a martello, Ad origliare per udir qualedfa s
Ll Romito che stava-ale velette, E sente dire: O lecca
Perché? nscio hadi fuora il chiavistello Carne feantia, barba pi
Andi ( benché tremando,e con spavento Ribaldo, Santinfizza, €
Che havea di lus) e ve lo ferro drento, C'aquel d' altri pon cingue, el
STANZA LXIx, 1










Guardate qui la gatra di Masino, Ma quel' hai toltoa me,
Che riprendeva il virio, ed il peccato, Won dubitar ti costera fa
+Se il monello-ha le man fatte a uncino Che tante volte al pozzova'
Per gire a [erafignar pel vicinato? Ch' ella vi lafeia il manicog









Magorto, arrivato a casa, si messe a bastonar quel facco, credendo che vi
fufle dentro Pigolone.; Ma questo-efiendo uscito di-casa mefle il ci
di fuori alla porta, e fermatofi alquanto quivi, senti che Magorto mn
facco gli diceva una mano d' improperj. win ti
SBKACCLARS!, Vuol dire Denudarfi'il braccio da mezzo in git te |
mano come accennammo sopra in questo C. stan. 19, B sbracciarsi; ee
mente parlando vuol dire Impiegare ogni sua forza, diligenza, ed mol
in un' affare. Lat, mamibus, pedibu/que eniti. want 1
SVONANDO a martelio. Cioè bastonando. Suonar' a martello si <<, m
do la campana suona a rintocchi., come fa il martello full ancudine, ii che i | %
quando si vuol ragunare il popolo per li bisogni della Città. Il verbo fumaretil | &
Latino puifo, vale appretio di noi, come appresso i Latini per suonare, e pet |
perquotere. Vedi sopra C. 3. stan. 7. aie
ST AVA alle velette., Stava osservando. Veletta, o vedetta diciamo
to, che sta in fulle mura d' una Città, o Fortezza a far la guardia d
munemente/entinells., edil lwogo dove sta detto soldato si dice velerra
Sumo che sia trasiato da i Marinari, che tengono la detta guardia
albero della nave, e dicono metter l'huomo aila vela, o veletta forse
piccola vela, che sia in quel luogo. Tarcagnotta Stor. lib, 5. p. 3. 7
Partitofi pero il Priore Strozzi da Marfilia con 2 3. Galere, ed una g
welette in mare lo venne ad sacontrare. Dal che ficava che si chi:
-gune barche, le quali camminino avanti a una armata con huo




SETTIMO CANTARE. 363

le, opure da vedere vederta'e poi corrottamente veletta. Si come da /pecio anti-
¢ Latino significante lo veggio » si fece /pecula luogo eminente che signo-

reggi molto paele. Ma sia come si sia basta il sapere, che stare alle velette vuoi
dire Stare a osservare.
| Bin fale furie, E'colmo d' ira.
ORIGLIARE, Star in orecchi, Star a sentire, e vedere con attenzione, edi
iy! costo.Pranzele oreillier. Spagn, otear forse dal Gr, Ora,orecchie, che i Fiaa-
fini spiega:/piare, eguardare da (uogo aito, come fanno le sentinelle.
 PEVERADA, Brodo di carne, o d' altro, E /ecca peverada vuol dir Brodaio,
se Beiniignifica porco, perché il porco mangia volentieri ogni sorta di broda...

 War, St, Fior. lib. 14. dice: Gli diede una mineffrina bolita, cotta in peverada di

- pollo. Detta Pewerada dal Penere, cioè dal pepe, che per dar (apore si metteva.s

fa le mineftre, come fu da altri dottamente osservato.

CARNE stantia, Carnaccia vecchia, e frolla. Vedi sopra C. 3. stan. 24. e $4.
ye  SAKBA piattolofa, Termine ingiuriofo per un vecchio;¢ vuol dire barba schi-
i 2, epiena di pidocchi, e d' altre lordure,

SANTINFIZZ A. \pocrito; de i quali a bastanza s' è detto altrove; EB per
yy  satinfieza's' intendono certi Torcicolli, che stanno tutto il giorno d' avanti a
 una immagine d' un Santo, perché si creda che essi facciano orazione.
yi, GABBADEL. Rinncgato. Vno che gabba, cioè inganna le Deita, adoran-
fio, Oggi una, e domani un' altra, rinnegando la prima. Se bene Deus non ir-
yal 'Tetur. Si dice ancora Gabbafanti. §
ay, PON cingue, e teva fei, Vuol dire Tu (ci ladro; perché ponendo cinque dita
we della mano, fai il numero di fei con aggiugnere alle cinque dita la roba, che
sath porti via. Plauto disse: Trism literarum Homo, cioè tees Habbiamo diversi
modi di dire copertamente Ladro, come Sgrafignare. Havere le mani a oncim,
: che si vedono nella presente Otttava 69. Bespemmar con le mani, Andar aCarpi,
¢ 64 Borfelli., Par il Lanzo ( che in lingua lanadattica vuol dire Ladro ) gixocar, 0
# lavorar di mano,¢ Gimili.
i 'è LAgatia di Mafiro, Questa fingeva d' esser morta,¢ nen era,e però vuol
è " dire huomo finto. Huomo che fa il semplice,e non è. Lat, Lepus dormiens, Te-
nere gli occhi aperti, haver L occhio, ed aprir l'occhio vuol dire andar cauto nell'
Operare: e perché tanto Ia lepre, che il gatto tengono gli occhi aperti anche dor-
mendo, servono a i Latini, ed a noi per esprimer un' huomo vigilante, cd ay-
yeduto, e che mostri di non essere. Vedi sopra C, 1. stan. 19.

MONELLO. Così chiamiamo quei guidoni, che per Firenze battono mari-

 Ma, comes'é detto sopra C. 4. stan. 8. Siccome Guidone di nome proprio fié
fatto appellativo, così forse anche Monello, in principio diminutivo dt Adone,
accorciato dal nome proprio di Simone è venuto a significare una tal razza di




persone.
'. ASSASSINO. Vuol dir ladro di strada, ma quié detto in vece di furbo,o
' »¢ può anche intendersi ladro di strada,
NON dubisar ti coferd falato. Sta sicuro, che ti ha da costare assai, o che ne

-pagherai un gran fio.
— TANTO va la secchia al pozzo, ec, Tante volte si torna a fare un male, che
Seri tey ' Zz

2 una
i
i ~
"

ha ee a




364 MALMANTILE ©
una volta vi si riman colto. Vna volta' fa per molte; e diciamo ancora; Tate ) wij

volte va la gatta al lardo, che unavolta vi lascia la zampa\, ec

violantium malus est, ed orecchie della secchia diciamo quelle due | tL

rate, nelle quali e infilato il manico di efia (ecchia. sear Ne at
S 4

TANZA LXXx. STANZA LXXL ai
Poi sente, ch'egli dopo una gran bibbia Ben ch ci creda finua '
D ingiurie dd nel facco una percuffa, Tira di nuovo, eda vicino
Che rurte le frovigiie /perra,e tribbia, Ed il suo cane acchiappa i
Ech eidiceva; Horsugiihorottol ofa; Che fa-urliche van nell
E che di nuovo un' altro ne rafibbia., » Ona' egli fiupefarto afjai ne
E che ( facendo il vin la terra rofja ) Dicendo: Qui è quand iomi
Soggiunge:O quanto/ague banelievene! Se nce' il sangue egli ha di gi
Quella ghicttene, a me, beeva bene. Come a gridar puo egli
Seguitando Magorto a dire ingiurie, da una bastonata in sul facco, €
i piatti, e fa verlare il vino, e credendolo il sangue di Pigolone resta
to, che ne posia haver tanto; € replicando un' aitra batlonata, ae
po il cane; 11 quale cominciò a urlare, ed ei credendo, che fuifero strida di
lone, strabilisce e non retta capace, che egli pola haver più forza di 7
ara



ae

PoetEstEe

frida, mentre ha versato tutto ul sangue.
DOP PO una gran bibbia, Dopo una lunga diceria, o filaftrocca;

Dopo haver dette tante ingiurie, che farebbono un gran libro, da Biblia Greco
Latino, che vuol dir br:; E se bene la voce Bibbia oggi comunemente e istela
per il libro della Sagra Scrittura, cuctavia noi la pigiiamo ancora ne i cascome
il presente nel detto sensodi libro, o di lettera, © di discorso lungo, come pate
che la pigliaficro gli antichi secondo Herodoto lib, 1. dove dice > Alarpagum
clufife, leporis ventri biblion.ad Cyrum, Se bene qui e Viguerro 5 letrera, Dal po
ma d' Omero intitolato l'Lliade, il quale è d' una prodigiofa quantità di vert,
come quelli, che a(cendono al numero di quindicimila (etcecento oreantatre; ut
gran moltitudine di cose, o di parole, dissero i Latini 4ias, o Hiades, Propeaid
41.2, clegia 1.

oe rRs oe ST

Tune vero longas condimus Iliadas,
Seu quicquad fecit, fine est quodcumqne locura
Ataxima de nibilo nascitur bifforia, ne

RAFFIBEIA, Replica. 'Irasiaco dal congingner con fibbia bottoni,
il che si dice -4fibbiare, Vedi sopra C, 2. tt. 81.

STOVIGL/A, Intendiamo ogai sorta di piatti, e vasellami di terra per nfo di
cucina. Ll Ferrari, Seovigue, Fittiia, vafenla, & frivola. Vandena, to
comperi. 1o stimo che sia parola florpiata dalla Latina. Veenfilia, Crefe, 12:12.
E molti altei arnesi,e fovg 4 di bilogno. Pallad. volgarizzato lib, 1. tity 6 Faber
da far terramenti, e dilegname, e di /ovigli da vino, da Javorare, eda
Questo ultimo non è nel Launo, ed è aggiunto nella traduzione per impiegate
voce Mowtgli, 7 3

TX/BSLARE, Propriamente vuol dire Batter il grano in fulltaia dab Latino
Tribula tribule, o tribulum tribuli, che vuol dire una specie di carro, ¢ol già
fquoceva il grano in fit!" ata, come si cava da Colum. 'lib, 2, cap. ie

@uifa eo PEEP ~ se Ee oe eT












SETTIMO CANTARE. 365

'unt adijcere Tribulum, & trabam posis, e Varr. lib. 1.C, 25. E'/picisin area

cM tivwencis iunitis', © tribula. B questo dal Greco eribein peftare, trita-

. Latino terere, o da thlibein schiacciare, dal qual verbo viene il Latino trib.

travaglio  dettoanche da' Santi Padri prefura, '

£. Questo termine significa A mio giudizio; Secondo me. Secondo il
4/0 intendimento; e per ie si dice replicatamente 4 mé a me, Quan-

» cl0é per quanto io giudico i Franzefi Quant' a moi, 1 Greci similmente

» cioè secondo me, secondo il mio giudizio.

DE haver finita la fefia. Crede haver terminato il negozio, cioè d' haver'

Pigolone, Similitudine trata dalla folenaita, colla quale son facti

i, che si giuftiziano.

CHIAPP A. Coglie: perché se bene «cchiappare vuol dir pigliare uno con
¢ violenza, ci serve er esprimere colpir bene. Latino Certo ietw a/-

Spagnuolo, acertar, Vedi C. 2. st. 41.

EF ATTO. Rimafto stupido per la meraviglia grande. Latino ob/upe-

STANZA LXXII. STANZA LXXIIL
in questo mentre:col suo fante Perch'ei del certain quanto a contentarla
 Haven di già scorrendo pel giardino Non ci ha ne meno un minimopenfiero,



















it pritrovato, e quelle piante E pero quante volte ella ne parla,
we coles, che chede sl suo Nardino, Mura discorso, ela riduce al zero;
vot! tha trata fuor belle galante, Ma perch'ellae mozzinaye con laciarla
tif be mon si vedde mai il più bel fennino, Le Afonache trarria del Monaftero,
Econ un sno bocchin da sciorre aghetts Vedeyche s*ella bada troppo a dire
i  Chiede da ber ma non già fel' asperti, Si lascerebbe forse connertire,
" av ' STANZA LXAlV.
wil) Peri per non cadere in queffo errore E ch'ei ne venga ch' ei l'aspetta fuore,
st | Lapigtianun tratto,efe la portain frrada, eAccio con essi anch' egli se ne vada,
yl Ed ai vecchio fa dir pel servitore, Che i non vuol lasciarlo nelle pefte y
* Che'pit tempo non è di frar' a bada, 44a condurlo al pacfe alle lor felke.

“Mentre che Magorto si fludia a bastonare, il favio Brunetto col servitore eras
andatoinell' orto, ed havea trovato il Cocomero, e tagliatolo n' era ulcita las
fanciulla 'che egii cercava, la quale si mefle a pregario, che egli l'empictic las
tazza, maei non volle contentarla, anzi la prefe, e la porto in firada, e man-
dO i teruidore a chiamar Pigolone per condurlo seco alle nozze di Nardino.
a ANTE, Si dice i) servitore; dail' intero infanre,si come in Latino Puer signi-
" fica ferno, da noi detto anche garzone, se ben Fante però comunemente vuol dire

" - soldaro'a piede, perché ne' tempi dell' Imperio baflo, che la milizia cominciò a ri- |
of a tarsi pil per la cavalleria, che per la soldatesca a piede; il pedone G venne as |

'ttimare come ministro., e servitore del Cavaliere; e perciò fu detto fanre,
| SENNINO. E' una parola, che si dice per vezzi a una femmina bella, favia,
~¢ pulita, e che operi cen giudizio con fenno,¢ con puntualita. Latino scita pue/~
la,feitula. z
~~ BOCCA dat feiorre agherti., Così diciamo di quelle femmine, le quali per parer
“belle tengono la bocca ferrata, e ridotta forzatamente pi Mretta del. suo nau.
ss rale;


















366 MALMANTILE ©

sale; ne muovono i labbri di come se gli sono accomodati allo specc
par proprio, che habbiano la bocca accomodata a feiorre un ng
Aghetto è quello, che vedemmo sopra C.2.f. 10,
'NON se? aspetti. Non lo speri. Cioè non asperti, che le dia bere « |
gnuolo ¢/perar e lo stesso, che a/pettare. bes pe a My
LA riduce al vero, La riduce al nulla; Zero quella figura d'abbaco, che
se stessa non rileva numero alcuno, ed accompagnata, forma le decine, e eile
per esprimer # nuda, '
eHOZZINA, Huomo aftuto, triflo, e che fa il conto fino, mas'inte
genio maligno. Latino Vulpis reliquie. Questa voce vien forse da orecehi m
che così son segnati quei furbi, che meriterebbono le forche, ma perla
eta non ne son capaci, sopra C. 6. st. 54., ed in questo C, st. 30. Ȣ credo
perché diciamo Azoyzorecchi in vece di mozzina nello stesso significato,
TRARRIA le eAtonache dal eAionaftero, Confeguirebbe l'impotlibile con fas
sua induftria, periuafiva » ed cloquenza. Diogene disse: Oratio non ex ani

Sfn2i* 8

proficiscens, fed ad gratiam composita meleus est laquens, quod [cilicer blandé x
ens hominem ingulet. té
NON è tempo di star' a bade, Non & tempo di trattenersi. Non v' étempods | &
erdere.; R
LASCTAR' uno nelle pefte. Abbandonar' uno nel pericolo, Vino fa' R
folenza, o mala creanza, e per non esser percoflo fugge viaye la(cia i Ra
€ questo si dice /a/ciar nelle pefte, cioè nelle pedate, o nella strada, che &
mancamenti ha fabbricato ai pericolo,colui che è fuggito;si pronunzia cont ay
ma c stretta a differenza di pefte infermita, che si pronunzia con l'é lagaies | 4
pero questa rima ha un a di falfita, ma tollerabile, ed¢ ammefla. &
STANZA L&XV. STANZA LXXVL s
Così di la poi ructi fer partita; Brunetto si ridea di Pigolone mm LA
Ma piis dogni altro allegra la faciullay Perch' ei parea nel vifo un fico vittty | %i
Perché non prima fu dell' orto uscita E menaua a due cambe di ee pth
Crognt incanto,agnt vogliain lei s'anulla, Com' egli haveffe hauuto i Birri drt; aay
Anzi ai lor preghi in sul caval, Salita, E la donna diceva: Grambracont, hey
Che la duri; ed il vecchio manfuttt, | 4

Senza più ragionar di ber, ne nulla,
Va sipreinnazs ag altri wn trar di mano Che si vedeua fatto il lor xsmbello:
Fiera, e bizzarra come un Capitano. Dagli pur (rispondea) ch'eglie fafitl. | "sp
Vicita che fu sa fanciulia dell' orto cefsd incantefimo, e la voglia del bere4y
anzicon la maggior' allegria del mondo monté a cavallo scherzando, € moe | &
teggiando il vecchio, il quale era ancor pailido per lo spavento havuto, i)
"RIZZARKO. Wuol dir lracondo, Suzzofo, o cosa simile, secondo chelule | &
rono gliantichi, Ma si piglia anche per spirito(o, e vivace, come è !
presente luogo. In Spagnuolo Zixarro significa uno che vada bello, e fupecbo nel Ge
veitire. B similmente roba bizarra, che 1 Pranzcfi direbbero bigearree, ie) 4
roba, cio' vette bellissima, varia, e pomposa, donde poi da noi si prende Bare |
ro per capricciofo, firano, stravagante. eae Bk
FICO vieto. Fico annebbiato, o afato. Vn fico, il quale al colore, e tene
rezza par maturo, non è, ma dalla nebbia è ridotto giallo, come se fulle ma:






SETTIMO CANTARE? 367
furo: comparazione, che esprime assai bene la faccia gialla, e grinza di Pigolo-
ne. El'epiteto Viero e proprio della carne falata, lardo, burro, e olio, quando

. per essere stantij, e corrotti mutano il colore, l'odore, ed il fapore.
, MENAR di spadone 4 due gambe. Fuggire; Correre. Spadone a due mani si
quella pada pil grande delle spade comuni ordinarie, la quale s' adopra
-ambe ie mani, e per derisione di coloro, che, vantandosi di bravi, all' occa-
poi fuggono, col folo dire; meno di Spadone, o gioco di spadone, s' intende a

ye gambe, che vuol dir Fuggi. Vedi (otto C. 10, st. 3.
COM egli havefe havuto s Birri arety, Detto usato per esprimere, che uno
corra velocemente

GIAMBRACONE, che a duriDubito, che voi non fiate per durare a cammina-
re. Giambracone fu un mato, che sempre andava gridando: Che /a duri, e»
| però quando noi veggiamo, che uno faccia un' Operazione con grande attenzio.
'Re,€ che noi dubitiamo, che egii non sia per durare fogliamo dire Giambracone,

an) © (enza dire, che /a ders intendiamo; piaccta al Cielo, che egli continovi, € così & Co-



inteso.
BATT O il loro Zimbello. Divenuto lo scherzo. Zimbello,oltre al significato,
i @ o sopra C, 1. st, 59.,vuol dire aacora quell' ucceilo, che si lega per
un piede allato al hoschetto de' paretai, o altri luoghi, dove si tende per pigliare
ig uecelli, che tirandosi quella cordicella, che ha legata al piece si fa suolazzares
Per incitare gli altri uccelli a calarsi. Latino amis illex, € dallo strapazzo, che
ry tale uccello riceve diciamo Zimbellouno quando e burlato, beffato, e strapazato
ad da tutti; nel qual senso e preso nel presente luogo; e sotto C. 9. st. 66.
7 — DAGLI ch' egii¢ faffedo. Dag, ch' ci lo merita. Olleruifi che 1 verbo Dare
nei cafi come i presente,vale per continovare, seguitare, durare, ec. e con dire
rin solamente dages icnz' altra aggiunta s' intende /eguita; ma s'aggiunge ch' egli e faf-
Selle per una certa vaghezza, e per un genio,¢ naturale inclinazione, che han-
N01 Fiorentini'd: paciar per proverbio, metafore, comparazioni, o similitudini;
i



- € forse e aggiunto per confondere,ed oscurare il detto,perché dare al fafedo vuol
se dir perquoterio, e nov vuoi dic seguitare. Habbiamo due specie di tordi, cioè
: botraces ye fafjedi 5 1 primi son meno aftuti, e piit facili a la(ciarsi pighare, i fecon-
| di sono più aftuti, e ad ogni poco di romore scappano, pero quando la notte col
# "s frugauolo si scuoprono, si dice dagli con la ramata, che qucfto e fa/sello, che alpet-
«' ta poco. In fuftanza nel presente luogo vuol dire continuate, o Seguitare, a burlar~

i mi, beffarmi, e firapazarmi, ch' io lo merico, Da questa aftutezza del faffello si di-
a si S¢ fafsello a un' huomo, che sa il conto suo, ed esercita il suo sapere a vantaggio,
my" pretendendo sempre pil del giuflo, e del dovere, avido di guadagnare, € tenace
* el suo più del conueniente.
w STANZA LXXVIL

us Così feberzando, com io dico,in brigiia Percio dopo haver fatte molte miglia,
i Ne vanno Lenya mai sentirsi franchi y E che tor parue un tratto d'esserfrachi,
) Esempr' ognun pin calda se la pigtia, Tutts affannati per st lunga via y
wo. Percheilcimor glispinge,e /pronai fachi; D' accordo si fermaro a un' Ofteria,

il
ty
# STAN.






MALMANT DLE) S G0 —









368
STANZA Laie au a
Dove il padron che intende fare. pasto >» Ben. ”, '
Trovagran vroba per yi garbato 5 'Guamioiinfa ha a0,
Chreitien che afar no habia rroppornaspo, E che quella »
Mae? non fache enon hanno desinate; Che's,

Brunetto con la sua compagnia seguita allegramente 'il suo vi
do per il timore, che hanno di Magorto y ma sti ia 7
un' Ofteria, dove mangiaron più di quello, che il padrone non: z

SC HERZARE in briglia, Questo detto, che significa uno, che stando|
faculta, e d'ogni commodo, non ostante G duole dello stato fuoy éd
anche per intender'uno, che stia allegramente,¢ scherzando fenzae
che egli è in grandiflimo pericolo;¢ così s'intende nel presente luoga,.

scherzano senza pen(are al pericolo,nel quale sono » che Mas arrivi
dosso, da-chanhp, peas

OGNVNO se la pigtia pits calda,~Ogauno se ne piglia maggior
sto pigtiarfela caida i Franzefi esprimono col verbo chalsir, e noi cal
dal Lat. calere; Boccaccio nel Poema in ottava 'rima intitolate il



de' fatti di Tefeo 1, 2.
Oude li se nuova vifion vedere; s OSes i
Perché di ritornar li fu in calere. 2 Mey
E appresso. Vici d' Atene, ne li fu in calere, ae
D' Ipolita amor dolce, e pudico. me
Spiegd la forza di questo verbo il Petrarca quando disse } ee

We dentro fento, ne di fuor gran caldo;
Che fa come una spiegazione de' due versi immediate precedenti:.
Ne del volgo mi cal', nedi fortuna; oy gaialah
Ne di me molto me 'di cosa vile, ome
GLI parne d' esser franchi, Parue loro d' esser in sicuro, ed esser liber da Mo

orto. hoped
OO ARE 4 pasto. Si dice quando I Ofte senza prezzare cosa per cosa di quell

che mette ia tavola vuole un tanto per persona, e mette in tavola quello yee

are a lui.
f NON habbiano a far troppo guafto. Non habbiano.a mangiar molto, Le.



aPEFER



feo incognito dice.
lo ero fario, e non fei troppo guafto,
Il Berni in lode delle pesche +
Dioscoride, Plinio ye Tecfrafto
Lon hanno scritto delle pesche bene
Lerché non ne facevan troppo guafto,
Cioè non ne mangiavano molte, perché 'non gli piacevano.
V? & rimafto, L' ha fgarrata. E' rimafto ingannato,, come chi
trappola. ssf
LON vi resta fate. Non vi resta nulla. Vedi sopra in questo ©, stan. 7}
Mattio Franzefi contr' alle sberrettate dice;; cals
“Hed
Ae

FREER

we





ss,






SETTIMO CAN TARE 369

A cavarfela, e metter più di cento
> *Folte per hora, +l che non serve a fiato,
va dietro alla cafserta, Cioè non si gaadagna, ma più tosto si perde,
TANZA LXXIX. - “STANZA LXXX,
sntante | frracco S' 16 percoffi quel vecchio marivolo
wil randello a quel partito, Com' ha io fatto, disse, un canicidio?
'ciolte,ed apertohavedo omai quel/acco Sa ch' io lo prefi ye la ferras qui folo y
sencinar la carne del Romito, Che gnun porea vedermi,o dar fastidio,
Ed in quel cambio vistovi il suo bracco Won fo s' o sono il Graffo Legnaiuolo
coceh 5 vetri macolo,¢ bafito, A queste metamorfofi d' Ovidio,
fa miaravigliato in una forma Che sono in ver meranigliofe, e frane













Ct ei non fa s'ei sia defto,os'ei si dorma, Poi cnn Romito: mi dinenta un cane,
; STANZA LXXXL
e'| povera Melampo Lo ho una rabbia addofso ch'io avvam:
Che | nette gua tencé oui Jaci; Con quel veechiaccio barba d'Olo sae



Chi più farò la guardia al mio bel capo Ch al certo fatto m' ha così bel giuoco;
defio, che t' hai cliinfe le lanterne? Che dubbio! metcerei le man nel fuoco,
eo Magorto'dal bastonar quel facco lo spiccd dal palco, ed apertolo vi
il suorcane; ¢€ restando maravigliato, fuppone che sia stato Pigolo-
li habbia fatta questa burla.
ere In quella guisa; in quella forma, in quella maniera,
kntendi frammenti di piatti, pentole, ed altri vasi di terra,
Pe.Badro, giuntatore. E' voce Napoletana,'ma già facta Fioren-
tina, A 4 7
CHE gine pores darmi fastidio, Che niuno poteva impedirmi, La voce gaxno
per nino, hogei @ usata folo da 1 nostri contadini,
NON fos' io sono 11 Grafo Legnainolo, Non sos' io mi sia diventato ur' altro, il
'Graflo Legnaiuolo fa vn Fiorentino, il quale fu tanto semplice, che gli fu dato
@ credere, che non era più lui ma diventato un' altro  e per questo tale fu mefio
! e alloppiato, e fatto dormire quando si rifenti, s* accordé a paga-
Te le tpee je le cancellature per il precefo delitto, del quale fu affoluto, benché
havefle Confeffato'd' haverlo commefio come nuovo perlonaggio, e pagd il dena-
10 un fratello di quello, che il Graflo si credeva d' essere, e duro in questa cre-
  denza qualche temipo; e fin che li suoi veri parenti lo fecero riconoscersi, e ritor.
share: che egii era. La Novella pare a me, è stampata dietro alle cento No.
vellea dell' edizione de' Giunti. Da costui digiamo i Graff Legnaiuolo per
intendere un' huomo semplicissimo., e facile a creder ogni-cosa, bench' ei sappia
non esser vera, ed esser' imposiibile, che ella sia. Si dice ancora Calandrino, ¢
Cap,,comie aecennammo sopra Cy 5, st23. t,
VE Romito mi dineneaun cane, Se bene intende, cheil Romito era diventato un
'caN';/perché nel! facco trove il cane, vi haveva meflo i) Romito, si potrebbes
'anche 'che intendefie parergli gran metamorfofi, che un Romito y cioè ua'
i bene jdiventi un cane, cioè nno (cellerato. i a
- MAL chitfe-ie lanterne'; Hai chiufi gli occhi;-ed.intende fei morto,, Chiamanfi
Anche gli Occhi sccicanré'in lingua furbesca; € Così li chiamo in un verio del suo
fio Brunctto Lauini Macftro di Dante. Aaa 10

an

+225 28

















wi tie Ea =














370 MALMANTILE ©&|

10 ho una rabbia addosse ch' io avvampo, Latino Jn fermento totus si
collora, un' ira grandissima. vvampare significa abbruciare leg)
cfempio; Vn panno bianco accoftato a una fiamma s' infuocola,e piglia
si dice arfo, o abbronzato, o avvampato. » # Sie

BARBA d' Oloferne. Barbaccia. E' nota la Storia facra di Iuditta,
la testa ad Oloferne. Nel pepeeiones detta storia, li Pittori per far
Oloferne per un' huomo crudele, dipingono la di lui testa tagliata brate:
barba lunga, folta, e rabbuffata; e da questo il dire a uno barba a' Olofer
giuriofo, perché suona anche lo stesso, che resta d' impiceato, “



&8& set ise

wr









METTEREL la mano nel fuoco. Mi par d' esser così certo di gaefta cosa, cheio | »,
la giurerei con metter la mano nel fuoco. Vno de' giudizzi, che chiamavano' wi
vini, appreflu i Safloni era la prova, che faceva il reo, via del suo > 1
nendo in mano ferro infocato. E le folennita, colle quali si veniva a qu -
va, sono descritte puntualmente dietro all' Iftoria clica di Polidoro Vi j "

TANZA LXXAIL STANZA LEXXIV be
Oimé le mie stoviglie, e il vin di Chianti Ma perch' ei vede quivi le a Tis

Chrio tolfi in dar la caccia aun vetturale Volte al giardino,e poi versolavity | Gy

eA cagion di quel tristo Graffiafanti CheBrunetto,equegiialtri: tu

in um tempo e versato, e ito male, Quando v'entraro,e quando andarovia R

Giuroal Ciel ch'io non una ch'ei fene vati, Infospettito, lascia andar il Frate, Pe

E,s'¢i non vola, puo far capitale Ed entra nel giardino,e a y i

Chr io voglia ritrovarlo,e s'es c' incappa Scorge quel suo cocomero dit an

Che mi venga (a rabbia s'ei mi feappa, Ch'e frato il fargli un fregio sopr w

STANZA LXXXIIL STANZA LXXXV, | 'y
Lo troverò bensì, perch' io vue ire Poiché levata gli han quella fighualay | x

Qua intorno per veder s'io lorintraccio; Chiin essa(cons' io ho dette) si trouwvs, |»

Cos} corre alla porta per uscire y Per la stizza non puo formar )

Ma cei nd puofarlo,perché e' v's il chianaccio Si soraffia, barre s denti, efa z

Lo fquote, e shatte per volerlo ape eS E spalancando poi tanto di gola dey

Edhor v'attacca 'uno,hor altro braccio; Verla,befemmia il Ciel, z ne

Noiato al fine vanne,e corre ad alto, Dicendo; QO Macomettoe ¢

E dai balconi in frrada fa un falto, Che si facciano al mondo i")

STANZA LXXXVL mime | i
fa quanto a te chi ti pisciaffe addofso Sapro ben' io a costor far shy. «thy \ y

So ben che th non ne farefti cao; Credilo pur, percht, se si da il case h

Ma io che da miei di mai bevvi grofso, (Che si dara feny'altro) chia, eT is

E le mosche levar mi so dal nafo 4o me gli vue di posta ingoiar vit, t

Seguita Magorto a dolersi della sua di(grazia; poi fata risoluzione d'é ty
cercar del Romito, falta dalla finestra in strada, dove vedute alcune. kj
fo il giardino, infospettito la(ciò il pensiero d' andar cercando di Pi bry
ne va alla volta del giardino  e quivi accortofi del ratto della fanci u
di yoler trovare coloro, che gli hanno fatto questo torto, e di, volergli tu &
goiar vivi. Nota che il nostro Poeta in qn ottava 84. e stato cri dy
ché s'¢ servito della voce ia in tutte tre le rime, ma tal fottigliezza fb iy
tofto chiamare ignoranaa, perché se bene & sempre la stessa voce, “—— 7 j




SETTIMO CANTARE: 371

sempre diverso significato, perché la prima significa strada; la seconda significa,
altrove, o moto da un luogo a un' altro, e la terza significa modo, guila, ma-
~niera, ec, E di simili rime troverai altrove in questa Opera, e sempre le vedrai







 lodevoli per l'artifizio, più tosto, che biafimevoli per ta poca avvertenza.

- AOL. Esclamazione, che esprime disgulto, o dolore. Latino Hei mibi;
- CHIANTT, E' una regione in 'Toleana dove nasce vino buonissimo. E Vettu-

 intendiamo colui, che sopra alle bestic conduce vino, ed altre robe da un,
ogo all'altro; a differenza di Vetvurino, che prefta, ed accompagna caval-
a lettighe, ec, a i Viaggianti. Vedi sopra C. 6, tt. 37.
| DAR fa caccia, Correr dietro a uno. E propriamente si dice Dar /a caccias,
 quando i birri corron dietro a uno per pigliarlo.
git = GRAFFIASANT!, Bacchettone, lpocrito, E' lo stesso, che Santinfizza det-
igi to sopra in questo C, st. 68.
più PVO? far capitale, Può esser certo. Qufta voce Capitale significa lo stato, o
Ill faftanze d' uno: tale ha 10, m, fendi di capirale. Significa aflegnamento. Chi
yt del mio fn capitale detto sopra C. 2. st. 7. Significa forte principale. Latino Sors,

i detta
i



,

yah



qs dat Greci cephalaion, cioè caput; dagli Spagnuoli candal, che corrisponde»
niall nostro Capitale, e Candalofo dicono colui, che ha gran capitale, cioè grandi
we = fallanze. U/ rale ha havuto la sentenza contro, ed e Stato condennato nelle [pefe, ed 4

are cento fendi di frutti, e mille di capitale. Significa quello vedremo sotto C. 8,



u@ 1.65. Qui significa può credere, può esser sicuro.
jah «SET c' inciappa. S' ci mi da nelle mani. Se, c' incoglie. S' egli casca ne' mici
ei) Meguati

i) UI venga la rabbia, Giuramento imprecativo contro se stesso. Giuro di voler

yj far latalcofa,, e se non la fo, mi fottopongo a ogni maggior tormento.

ait 8" 10 to rintraccio, Traccia significa orma, o veltigio; onde tracciare vuol dir

«at 'seguitare le pedate, e per confeguenza qui intende: Se io lo ritrovo; Traccia si

iti dice quella strada, che fa il cane per la paflata della lepre, o d' altro animale

im fiutando; viene questo verbo rintracciare, che vuol dir Ritrovare, e rraccia-

jus) ecetcare, Latino vesticare. x

ynt © CHLAFACCIO. Elo stesso, che chiavistello detto sopra C. r. st. 69. che i Sa

se nefi dicono pestio dal Latino pefu/us. 11 Conte Vgolino preflo Dante Inf. 33. Ed
io fent) chiavar ? nscio di sotto all' orribile torre; cioè mettere il chiavaccio.

rit «= A QVELL Avia. A quella foggia. Inquellaguifa.

ull PARGLI uno sfregio in sul vifo. Fargli una ingiuria ignominiofa, si come sono

iii Bl sftegi. Vedi sopra ©. 2. st. 3.¢C. 6. tt. 54.

PAlabava. Intendi ha gran rabbia. Latino fromachatur, Che bava quell'
que. UmMOre viscofo, che da per se fleflo ca(ca dalla bocca come schiuma, come si vede
eS se i cani arrabbiati, donde e prefa la presente metafora. Si dice ancora: Afi
: fs venir (a bava'; di chi mi fa entrare in collora, e di chi noia forte,

La ML Ciel minaccia', e brava. Sgrida, e minaccia il Ciclo. Vedi sopra C. 5. st
ff Gx, che dice Rabbiofa,il capo verso sl Ciel tentenna, che & quel minacciare il Cielo.
2 'Di questo verbo bravare, che vien dal Provenzale il Varchi ne fa un lungo discor-
ut fonel suo Hercolano, e lo giudica molto esprimente il latino obrargare. Catullo.
ty:
6

è Aaa z Gel:





anes

Ate


| 372 MALMANTILE >

t+
] Gellins audierat, patruum obiurgare. falerb § 5 SOV si Dayne
| St quis delitias oe. 9 aut factret, et)
| TANTO di gla; Gola assai larga. Vedi fortoC, 16. st, 18, '
| ce tanto usata in questi termini, è tote Fig i
NON ne farefti caso,quand' uno ti pisciafe addofso, Non ti
j non t' importerebbe quand' uno ti pilciafle addosso; ed intend: Sei
ne, e codardo, che fopporterefti qualfivoglia grandissima ingiuria
ne. Vn'antico Poeta per voler esprimere uno scellerato,¢ ingiuri
| ria di suo padres dice:patrios mincerit in cineres, B Pittagora in uno de'
boli per dinotare il rispetto, che si dee portare alla Divinica,
non si pisci in faecia al Sole.: itty
NON bevvi grofo, Non fopportai mai ingiuria alcuna. Ber
Non la guarda così per la minuta, ma m5 8 ogni ingiuria senza
ne, fingendo non fen' avvedere. Tratto dal bere le medicine, le quali
faporano, ma si mandano git a occhi chiufi, » 0: Soa sos Say
| MI fo levar le moscbe a' intorno al nafo ~ Mi s0,vendicare dellvingiuri
cilitd, Omero nell' liiade La preftezza, colla quale un Dio fa tornare indietroi
colpi avvelenati contro a un' Eroe compara al cacciare d' una mofea, che fa las
Madre dal corpo del suo figliualo. © J oenetige
FAR ilc,..rosso auno, Galtigar' uno.. Tratto da i Pedanti, i quali )
i ragazzi perquotendola in sul ¢.. 5 e gliclo fanno rosso con'le)
sopra C, 4. st. §1. i: 1 a 0b if Settee SIR
oe. Subito, Viene dal giuoco di palla, che si dice Dar di nr.











si da di primo tempo, cioè avanti, che la pallatocchi terra «Lavinond seftigio«
INGOLARE, E' lo stesso, che ingollare detto sopra Cas st: 63, e vuol die mat:
dar la roba gil: nello stomaco. ' ol Owe
STANZA LXXXVIL, STANZA LXXXIX:
Ma dove col ceruel fon' io trascorfor Quel detia Cella del Romite eiil
Pits buesdi me non e sotto le frelle, Ove trovaede if ibrvepuenia
Perchinnanzi ch'io habbia pref Porfo Intana dentro, e non wi scongeniing)
Vio (come si suol dir) vender ta pelle 5 Fruga,erifrugainquaye iia aea
Fatei ci voglion qua, perch' il discorsa Sgomina cia che v'édafommoywinty
Fuor che ai Senfali non frutto covelle, M14 tutto in vanoyondeghial,
E mal per chi ha tepo,e tempo aspettas Sen esce con le man pieae ns yente s
Che mitre piscia il can,ia lepre sbierta, Ma dieci volte più dimal talento,
STANZA LXXXVIIL STAN ZAULX Xia ly
E peri prima, che 4 vila a gamba « Entra neh bafeayeogni,
Vaa fuga mi fuanin divconcerto E in fammea ne cored pax
A casa Pigolon vuogt ix-di gamba s | LB wedde' 5 fanz ain '

Che vi [ard coi compliciidel certo, vorrde pigiato esser bidval far decent.
ares fice ribinfe

Così conchinfo, correch? diff gamba y 4 «| Onde nelifine alf

N

Ne np bp cw @ eee FZ cee Fees SF ee hese KS

E come un bracco taper-quel deferto Che pur mol vendicar segtandiane

Tutti quanti quei benghi a uno 4 uno Così v' ar rivera. po' pai in
Cercando.s'ei vi scmopee,o sentecaloxne 1 Seveifuffe( di i ne

K aia MS


























SETDEMO CANTARE: 373
° STANZA LXXXXIL;
Poiche Brunetro, e le sue camerate

Pagaron L offe,( il quale assai contefe,
Perché le gole lar. difabnare
Gli eran parute meeety Spefe)-

». Partiron,, e poi dopo-altre fermate y,
Sie SUnapdeledee



i di quanto have 10 E giunto a casa,ringrazianda il Cielo
Ve più 5 ne manco ne segui  effetto. Entra in fala,¢ di posea fa un belo.
yi STANZA LXXXXIUL
Trovan Nardino acor di male oppreffo,
E sbietolar lo veggono ancor Lui,
L' Alhante, che porgevali horxata
saper. 9 me men per cui, >, Purine faceva lafua quattrinata
Magorto lai lamenti, e si,mette a ceecar di coloro., che gli havevano ru-
t la Figliuola Ȣ nen gli trovando nella Cella del Romito, ne in alcun altro
ricorfe-a gl'.incanti; co i quali costrinfe tutti della casa di Brunerto a pian-
3 onde Brunetto con i compagai arrivato a casa subito cominciò, ed
icompagni a piangere.;
fon' ia feorfo coi ceruello' Che armegg' io? Che giro.io? Che frenetich' io?
SD pWVOMe fetta le frelle it più buedi me... Ao sono il maggiore ignorante che fa nel
Mondo Vedi sopra C. 6. stan. 98. Sarr /a Luna; i Petrarca..,Arda.s o mora,o
dangnifer sun pin gentile Stato del mo non e forte la Luna, '
thsi Bi: da.pelle dell' orso. prima di pigliarlo. Fax aflegnamento fopna una cosa,
che ancora non s' e confeguita, ed e anche molto dubbio/o.Ji confeguirla..Essen-
ido anati cre Giovani per ammazzare \un' orso'si| quale faceva molto danno,

prima che atrivafiero.al luogo dove foleva trovarsi l'orfo, si fermarono a un'
Bieria ed havendo assai ben mangiato, ditlero all' Ofte., che lo paghercbbono
son denaci: del danatiy a, che haure bbano dato loro le, Comunia per V'orfo,
\uhe walevano ammageare 5 ¢d.amapifi verlo dove stavala ficra, subito, che las
veddero fidiedero a fuggire, e uno di loro fali sopra ad un' albero.s 1' altro scap-
- PP Viayyediiltcrzo fu lypraggisnto dail' Oxlo ei quale bavendolelg,caceiato fot-
~todtinfran bene beng sdi'por gli.accoflo digrife ail' orecchio,, ed intanto quel
meschino se neftava come monto, senza muoyerf punto; e perché J' orso naw.
cralmente (secondo. dicona alcuni.) quando: ee ede, che.! apimale da Jui. afaltato
sia morto, non gli da più fattidio, credendo che costui fatie morto,fen! andò,7¢
'€Gluiideyd si 5 ed ay vind vero la Città cutto mal coacio.. Quella y.che,era fa
 litodimaulimalbero:iccle-s ed.accompagnatoli coneflo, gli domandé quel che,gli
 havefleidetto.? orso nell' orecchia y sd egii rispole = Mijha detto, che io non mi
“fidi più difimilicompagai come fei G28 Che 19. now veoda.a, pelle.dell or(o. se
jeteemeete ho preso,.: B.da questa.novelia-habbiamg il prefeuce proverbio, che
idictlanche: Vender t' uscelio in fu la frascas L Geesi ditiero: Anrequam pisces













BSELEAL:












smuriam misces.. J.
- MAI frutd covelle. Non fa d utile alcuno, Covelle & voce romagnuola.e vuol
dire. Qualcofa, E' poco usata nok Fiogcatino fuor sie da qualche.consadino. Il
TANS:: valore

ERLETELLE

=

=






374 MALMANTILE ©

valore di questa voce è assai copiofamente espreffo dal Copetta inun f
sopra il non covelle, Nel Decameron trovati Cavelle per lo stesso
Lat. quod velles.

















epee ae!
E' mal per chi ha tempo, € tempo a/petta, che mentre, ec, Mal fa colui, che ha
do l'occasione pronta perde il tempo, e non la piglia,perché mentre si
J occasione fugge: E' noto il verso: Fronre capillaca post i
verbo sbiettare l'habbiamo anche sopra C. 5, stan. 30, Adentre il can piscia,
se ne va. 1 Latini ditfero Semper nocuit diferre pararis; secondo Lucano, di¢
forse Dante nell' Inf. C. 28. disse:: 7
Questi feacciato il dubitar fommerfe
In Cefare affermando, che il fornite
Sempre col danno L) attender foferfe.
PRIMA che a viola a gamba, ec. Intende prima che d' accordo se ne
Viola a gamba e il baflo di viola, Fuga è specie di fonata a capriccio,
vuol dir Suonata concertata con diversi Rrumenti, ec. Econ questi
tende quel che s' e accennato. }
INT ANA, Entra dentro. Si serve di questo verbo anche sotto
25. se bene e improprio; perché vuol dire Entrare in una tana, o buca
rebbe intanare una volpe, un taflo, un granchio, ec, cutcavia e pur
to come nel presente luogo. ei
AMO. Niuno. Dal Lat nemo. Voce oggi usata folo dai contadiai ell
nostro Poeta se ne serve anche sotto C, ro. stin. 37. in bocca d' un
SGOMINA, Si dice anche (gombinare,( contrario di combinare,

piare, unire ) e vuol dir mettre in confulione 5 o fortofopra tutto | che si

maneggia. Lat. perturbare,.
DA fommo aime, Frafe latina, che significa Da capo a piedi: Dalla fomal-
ta della casa, fino a i fondamenti di essa: Petrarca Trionfo della Fama, ene
Onde da imo Perduffe al fommo t edificio fanto, + ae
LE man piene di vento. Cio' senz' haver trovato, o conchiufo nulla. Nellie
Scrittura. Et nibil invenerunt in manibus (nis; che diciamo ancora Con le troment
facco Ter, disse Infetta re. ee |
DI mal talento, 1 collera,e con volontà di far del male y e di vendicatl:
Varchi Stor. lib. 4. Erano verso i nobis di malissimo talento, ne altro per manvweit:
gli alpettavano, che quel che avvenne. BY frale usaca dai Boccaccio. =)
NE cerco per mars e monti, Questo detto iperboiico è uiacutimo per esprimett
Ne cercd da per tutto; Viene dal Latino. ee
SENZA metteria in forse, Senza dubicar più. Senza metterla in dubbio. Dd
mettere in forse fece Dante il verbo inforfare, seguivato in ciò dal Petrarca +
IL pigiato esser (ui al far de* conti. A consideraria bene} offefo', e-beffaroem
solamente lui. Quattro giuocano infieme', tre vincono, ed un di loro folamet”
te perde; questo tale si dice è/ pigiato, cio® quello, che ha gli altri addosso, et
cui G spreme il denaro. Bs' intende in ogui caso, che la disgrazia tocchia ue
folo della conversazione, e tutti gli altri habbiano (oddiienioasl “, outile dal
danno di lui.: Ae:
POPOL in quel fondo, Vedi sopra C, 2, tian,,3.











y '"SETTIMO CANTARE.: 375

 BeANNO avanga. Vanno secondo il desiderio. Ex animi eins sententia ille cr
wnt. Noil habbtamo da i Contadini, che quando si rende loro facile il lavo-
'la terra con la dicono; 4 lavoro va 4 vanga, cioè bene,¢ come si de-
« Bvangad strumento ruftico fatto a foggia di pala,ma di ferro pil
» © pill acuta, del quale i contadini si servono per rivoltolar la terran.

edi sopra C. 6, fan. 69, al verbo impiallacciare., Columella lib, 3, la chiama do-
ira,¢ perché questo nome vuol dire più tosto la piaila, forse Columella inten-
ee flrumento usato a suoi tempi, che faceva sopra alla terra l'effetto che
t pialla sopra il legno, ( come e hoggi la marra scopaiola, della quale si
serUono i contadini per ripulire,¢ radere i boschi di scope per disporgli alla semen-
 ta della fegale ) perché,se volefie dire la vanga,havrebbe detto acuta dolabra fodi-
%,¢ non abradito: E la vanga si trova bipalinm, in Varrone: /d priss bipalio










STVMMIA di furfanti, Scelleratissimi, ex omni vitiorum colluvione concreti.

i 'Stammia,scbiuma, o spuma, & quello escremento, che nel bollire una pentola.
* piena di carne, e di acqua manda alla superficie, il quale si butta via, perch ¢

th Imm ia; onde fummia di furfanti, 11 peggio, che sia nella furfanteria. i

hay! ( difabicata. Lat. gurges, Così diciamo di colore, che sempre mangia-

yi —— si veggono fazzi.

“an paruti cari per le (pee. Exa parfo all'Ofte, che costoro havessero man-

ot 9 troppo. D' uno che sia buono a poco, e mangi assai, e che vada a servire
od | }; Beli e caro per le spefe; e intendefisfe gli da pil del dovere,¢ di que! che
wr en sua abilita a dargli solamente mangiare, senza dargli danari per prov-
a ¢. li Lalli nelia sua En, Ir. C. 2. stan. 130.

Sit Non vagiio un pel; son caro per le spefe.:
Dit! DI posta fa un belo Subito comincia a piangere a tn Vedi sotto C.9, st.21.
ig) © SSHETOLARE., Cio piangere. Vedi sopra C. 4. stan, 16.

AST ANTE, Intende colui, che aifisse al servizio di Nardino infermo, 4fan-
oiil 34 si dicono quei Serventi, che affiftono a servire gl' infermi negli Spedali,¢ guefti
it Aoglion esser chiamati dalle persone comode ad aififtere alli loro infermi, e pe-

10 qui lo chiama col nome 4' 4/fante, (upponendolo uno di questi tali.
in ORZ ATA, Bevanda rivfrescativa fatta di feme di popone, orzo, e zucche-
yo 79 deniflimo peiti e liquefatti con acqua,¢ passati per stamigna, si da per lo più
ra febbricitanti;detta anche /arrara come habbiamo veduto sopra in questo C. st. 12.
NE faceva la Sua quattrinata, Cie faceva la sua parte del pianto,

Ae STANZA XCIV. STANZA XCV.

oe /ardin vede colei bell' ye verzofa Mettere pur così le mani innanzs

ro Com' appunto ? haueva nel pensiero 5 ( Rispond' ella) Signer per non cadere,

i E dices Benuenuta la mia [pola, AMentre,temendo ch' io non mici feanzi,

1) Ko 9 tt piacere a se da Cavaliero. Specorare si bench? è un piacere:

we 4a voi piangere ? ditemi una cosa Ch io mi vi levi, ditems, dinanzi,
Bs Koi ci venite a malincorpo, e ¢' vero? Che voi non mi porete pix vedere 5)
RD. Non vogliate risponder che ¢' non sia, Senza darmilaburla,ch' io m' acquicta,

" Perché

#



bé vei mi direfti una bugia.

E Senza replicar do voita a dreto.

STAN-








+





























376

STANZA XCVIV

Ne fofopra la man non volterciy
j Che Pandarese lo far mi fontitt nia,
| Eben c'al mondo t6\sia come gli Bbreiy
| Che non han terra fer mayo patriaalcuna
i Andrd pensardo incanto a farki ries
Per veder di trovar migtior fortuna',
ct fond » come diceva Afona Berta: vr

t Chi non mi vuol segn'e che non mi merta, - Pero non vogliar
? STANZA XCVHL: %



Ella soggiunge, ed Egli ribadi/ee j
| Ella non cede, ed ei risponde a thiond.; © ” ogmoraincafa 5 fuora y
Pur gliacquiera Brunetto,e al fin glsunifee,  ( Perch sempre si fnmena,
Sicché 2 un o altro chiedefi perdond} « ©) Hantioa tener agli ovcbi |
Nardino vede la Fanciulla; e la trova per appunto'comie fel" era'
|; ma vilto che ella pi angeva le dice, che dubita, che ella' sia venuta mi
ed ella gli risponde, che dubita, che più tosto egh non la riceva vo!
pra questo seguitavano a contraftare, ma Brunetto al finé gli raj
tutto questo ognuno seguitava a piangere. i ¢
j VOI ci venite a matincorpo, Voici venite malvolentieri,'¢ con
| soddisfazione;cetra' flomaco,cOtra voglia,fatrone'taa fola parol: come
METT ETE te mani innanzi, Queito certnine ci ferae per esprimete
accufa un' altro di qualche mancamento, del quale merita dj esset 4
per esempio: I ragazai dello Spedale degi*imhdceati, i quali si
sieno tutti bastardi, in occasione di contraftare'con alcri ragazzi,
giuria che dicano a quelli ¢, 7 /ei bastardo, pecche non sia detto a”
fio si dice: Aderrer le mani sananzé + © vi si aggiugne anche; per
prevertere,occupare. -
NON mi ci fanzi. Non mi fermi in que(ta Casa per sempre.'
SPECORATE. Piangere. Diciamo be/are per piangere per la
che ha cobbelar degti agnelli, € delle pecore certo planco Jango', che!
rei bambini, come accenuammo sopra C. 6, ttn, 22. e'da queito
Specorare in vece di belare, e s intende piangere.; nist
St ben ch' è un piacere. Tanto bene, che € un gusto a sentirvi, e vedertis
NON ne volterei la mano fortofopra, In questa cosa io fouo imdife;
poco ny importa il faria,o non farla, Vicne da i Latini che discvano
Ne manum quidem verterem, + tye
ESSER come gis Ebrei, Cioè non haver luogo che sia\suo propri
ra il Poeta medesimo dicendo > Wun ho terra jermayche intende terta'y
abitazione fermata, e stabilita per-lei, che per altro Lerra* termat
paefe, che non Ifota di mare', Lar, continens', + syageael
VOI vi levate in barca, Vou cntrate in colicra, Vedi\sopra Ce
dice anche abarcare 5 e 2 iracondo, o vero facile al? iray chek
» derocholor& dette da NOt Aacmo as poca levacurascioe che'ci'vudl poco |
rein collcra.

~ ida non per
Co





a
3



SETTIM/O CANTARE; 377
Qui vuol dire fofferenza, o pazzienza, che per altro' Flemmas
accennammo C. 3. stan ages % ho
$4, Iraconda, Vedi sopra C, 1, Aan, 29, Alewni critici hanno
dra questa refs; giudicandola rimafaifa in'tigaardo-dell' 5s, dolce di
'studardi ai/perte/a.,¢ dell', 0, Jargo diquelle, e fretto di queste,
voglio quictare, e difendere il holtro Poeta col Riscelli, o cons
'non mi fon'voluto pigliar la briga di vedergli come non necessaria,
bem loro un' esempio d' Autore claifico il quale dice.
Hag. Lb 9.30 La verginellae simile alia rofa ~
9 5 Nei bel giardin /u ta nativa spina,
Mentre fola,¢ sicura si riposa
Ne gregge, ne\pastore se le aunicina
, Li aura fuave, el' alba rugiadofa, ec,
Eg con questo esempio.( il quale sia per regola, o per licenza) di fal-
ire il nostro Poeta, e quietargii, aucor per J' altre, che hanno osservatese sopra
stan, 13. Rofa  prola, e cosa; e foro in questo C, stan. 103. Sposa, cosa,

BADIRE, Ribattere, conficcare
Vedi sopra'C, 2. tan. 79.
DE 4 two, Rilponde aggiuttatamente, ed a proposito di quel, che si
verbum audit, tale dicit. Si dice anche Rispondere per le Rime. Las
iilitudine € tracta dalla Musica; la seconda dalla Poesia; B allude al co-
he de' Poeti che indirizzando /' uno all' altro Sonetti,e¢ proponendosi que-
levano,e le scioglievano in altra eguale composizione teffuta delle
t eliine rime, il qual costume venuto dall' antico, si mantiene anche in oggi.
sul - Sl fmacica, e B cosa, Si manda escremeati dal nafo,¢ lagrime dagli occhi
oi M 'piaato, che/moccicare vuol dire mandar fuori mocei, che € quello
 eleremento del ceruelio, che esce dal nafo detto da i Latini mens.
PEZZVOL A. Pazzoletto, o Moccichino; ed & quel pezzo di panno lino, che












dall' altra parte un chiodo, Vale per re-








i si lo di se per uso di nettarsi i nafo, —
sl) py, SEANZA XCIX, STANZA ©,
fs wa in un continuo pianto, E veduto ch' ell! e tra buona gente,
; om - ees: 3 i &
| Phangona i fer wise piangon gls animali, Moglie a! un ricco, € nobil Baccalare,
i Onde th guazzo per terraetale,e tanto, Eche Sok le puo mancar niente,
iB Chee Portan tutti quanti gli frinali. Per ch' elinein unacafacome un mare,
, iamoa Magorto, che fra tanto Won vi fo dir e ei gongola, e ne ente
a Per saper quel che sia di questi tai, Contento grande, € guffo singolare,
il Ez dowe la jua figlia si rnrovi, De-modo ch'ei si pente,affligge,e duole
a = Hes fato ab confuse incanti nuovi. ao — ha fatto ye rifarcir lo vuole,
RS ae abled. 5) o STANZ yt
Pi un fae cogno, E poi che dentro più non ne puo porre 5
& Sumnonipencace, a nae foot; Biphedi pte "Luo aspetto e molto brutto
lage e 0,enecomincia a corre, Si lata, vipilisce ye raffaxcona
a ~ Darando fin che hebbe pieno tuto; E rimbeuilce tutta ta persona,
0 SS AAMERabs he ig

Bbb STAN-

































378 MALMANTILE; =.




. STANZA CIL
E prefa addosso poi quella sua cafsa,. Che al suo ver
CL? e tanto grave,ch' ei vi crepa sotto y Mirando in rif
St metic la wa, @ presto fene passa.;  Eversad pomiin
Ow e la figlia s e il flebile raddotso Poi st

Mentre che costoro piangono, Magorto per via-de'
& la Figliuola, snaolcendonche ella è bene aipentt si. %
folue di regalare gli sposi d' una quantità grande di pomi d' co + A008;
to, € Così fece, ed ap arrivo oo in casa degli sposi tutti ceflarono di pia
GV.ALZO. Luogo pieno,d'acqua, dove fiipossa guazzare, cioè pall
picde scnza navilio, che noi dal Jatino diciamo, vadoy o gaado; onde ilk
Vaaa così detto perché quel luogo dicevai Yada Volaterrana's e guadare x
¢ paflare: Ma i piglia ancora per ogni grande ammollamento, che si fa
neile case, o altrave in sul suoloy come € preso nel presente ye: aque
calo viene da guazza, la quale cade dal Cielo, altrimenti detta 4 at
prvina come gelata disse Dante dal Lat, gelu; e non da guazzare il flume; Sefore
ie non voleflimo pigliarlo per parlare iperbolico, come e  adoperare
per paflar-tal molle, che è in quella lanza. 0) oop oe
8ACC-ALARE. Huomo di stima. Vno dei principali del paefe
anche Barbafforo, Baccalare da Baccalaureus si dice colui, che nelle
acquiftaco un grado proffimo al Dottorato, o Maeftrato detto altriments I
ziato; il che usa nelle Fraterie, e corrottamente lo dicono Saccelliere
grado si ritrovava anche nell' ordine della Cavalleria.:
E una casa come un mare. Cioè sempre piena di zoba ed abbonda
bene, si come il mare, che e immenfo, detto perciò. da Omero atrygi ih
non ha fin y ne fondo, Si dice anche Vna casa come una Dogana, a
GONGOLA. Greco cancharei » Giubbila:-Si rallegra 2-5i
certa allegrezza interna. B' voce usata assai dalla piebe.: Re
KIS-AKCIRE, Riftorare; Rifare il danno,0 ricompen(argli qd havergli tenn.
ti tanto in pianto. E per altro questo verbo ri/arcire vuol dir raffer
visto sopra,C, 6. stan. 52,
cOGAO, E' una mifura,immaginaria di vino, che contiene dieci barill
quale corrottamente si dice Cento; Deriva.dal Lat. congims. Onde Bigonet §
da un Lat, bicongins; a Piftoia perciò dette più prossimamente all' origine Bite
Gio, Villani lib. 8, rubr, 116. Valle lo fraio del grano in Firenze solds 8. Leagan del
mofto in certe parti meno di (oldi 40, Ma qui è preso, come & coflume », per une
certa forte di cafla, o più toflo cefta fatta, e contefla di firilce dal
corbelli, ma è di foggia Junga,.ed ha il coperchio, come hanno le, c
S/rafazzona. Si ripulilce; Si rinfronzilce. V.cdi ops. Cy 2 stange
si rifa; si rimette in fazione, in abito; fala Pre tecr yi la bella
nicra. Gli autichi dal Provenaale dissero agenzare, cio' &, '
ce Gente usata dagli antichi Tofeani ancora per Gentile, Br
voi, donna gente M' ha preso.amor, non è.già maraviglia. L
il fenno, e li gents coraggi.. I Beato lacopong disse che la 2
geaz@, clot non rilciacqua, come spiegd alcuno, ma rafazzuna, ri
'





SETTIMO CANTARE: 37
sotto pet lo fverthio peso j ed'il verbo crepare, che»
, come vedemmo et: stan. 18.qui¢ nel suo vero
ire, perché quella gran fatica può cagionare l'allentamento.
0€ si cava di Se.; fila errcor tact propriamente il piles
essendo il nostro cappello più tosto il pera/us.
Fras ) STANZA'CIV.
2) Eperché qualfinoglia-donniccinola,
2 Lpauen: Porta la dote, ed il corredo appresso,
digni cosa,.——*. Acciacch' in quella casa la Figliuola
Sdegho*toritmenté ha spento; ~~ ~Polfa mostrar a baker qualche regresso,
<1 SS Ne che gli abbin a aner quelcalcioingola
“C' un piccolo ne anche v' habia meffo,
'La vuol dotar conforme al grado loro
pss 'Con quel gran-monte di bei pomi a! ora,
SM Evopny shaves Ps ANN FAs +P;.
llor brillando con Brunette © Edegli poi al fin com ogni afferto
gr axiejefanaratanccoglitza; °°\° Riker? rutti, è volle far partenza,
inato un grande ye bel banchetto” > Ledandosi del furto del Romita,
“ar le nozze in sua prefenra, 'Che si grand' allezrezza ha partorito;
orto si fa. conolcere per il padre della, Sposa, ed aificurando Pigolonc, ¢
i perdonato, ed' haver guito', che segua quel parentando, colti-
vl lla caffa piena di pemi d'oro. Si fanno però di nuovo gli spon.
il banc! ¢¢ Magorto se ne corna al suo paefe, dando molte lodi a,
per esser'egli stato autore di così gran 'conteato. E qui con la fine de]-
la'raccontata dalle Pate a Paride termina il fettimo cantare,
  AMAN vote, Senza nulla in mano: cioè si mariti(enza dare dote aleuna;
we} «= CORREDO, Quegli arnesi, abiti, ed altre robe j'che fi'danno alle Femmine,
 Oltre alla dote, quando si maritano, che i Giureconfulti digono Parapherna dal
yep Greco Para, che vuol dire oltre, e pherna, che vuol dir dote,
git HAVER regrefio Termine legale, che vuol dire haver azione di domandare,
 Sontro\a tino, per rifarsi del pagato ad un' altro; Vedi (otto C, 8. st. 42, E co-
jj MUunemente significa un certo ardire, ed autoritd sopra ad una persona, o sopra
i i suoi beni ed effetti: 1 rale gli ha preso regresso addefvo, per intendere ha preso
yg) atdire sopra di Jui.

a gli abbin a haver quel calcio ia gola, Non habbiano a poter rinfacciar-
































us om; “ oe non v' habbia portato nulla: Noa habbiano a ha.
'caufa di conculcarla, '

| ELANDO. Giubbilando, Vedi sopra C. 2. st. 69.

4 CACCOGLIENZE. Vedi sopra C. 1. st. 34.

le mgxe. Cioè di nuovo si fecero gli sponfali, e folennemente ff
di sposi.




' 'Gohan

FINE DEL SETTIMO CANTARE:
Bbb 2 OTTA-






tk












&3 ARGOMENTO.
oY
D' un! avventura grande è po

OTTAVO.CAN
a; J 3
SERS]
Dalle sue Fate Paride vestita. ous. a ae
Vede lagalleria di quel? albergo;\.\ a
Seco eer eenin














Ond' ei pigliavicenza, e voltaiLcergo.,









a

5 Vien Piaccianteo condotto al Generale y.... »

ey Che non glivolle far ne ben, nemale,

BAP 25 CEPR CUE MFI CED
N75 h
ENA AAACN |
STANZA STANZA THM. Ta
Orrei, che mi dicesse un di coffora, La notte, disse y¢nn wafo dit
Che giofran tusta notte per le vie: y Che versa affronti, rifichi, € iy.
Che gusto v' ¢,perch' a ridurla a ora Pero, he nel:fiuo sempo shucan fuora kg
Non v'é guadagno, e son tutte parries Tutes i ribald ydadrt, e rompicali;

Poiche ( lasciando, che enon e decoro ) Onde sia ben'riporsi di buon' boa, !
L? aria cagiona cento malattie, E dene esempio Lhuom pig is al iy
Mille disgrazie possono accadere, Che  unds loro al piis vale. at T
Mille malanni, Diauoli, e Versiere, E pria ch' il. fol sr amoncé se ripome thy
STANZA IL. STANZA LWe be
Sapete, che e' s'inciampa, ¢.che e' si cafen, Edegli, che at un. mondo assai più vale t
Si puo in cambio a' un' altro esser'offefo, Sta fuori tutta notse,o dineth, Pa
O dar mun, fet' bai monete intasca, £ gira al bmo come un' ani 4
C alleggerir ti voglia di quel peso 5 hie
Maca: qual ma si pudcorrer burrasca, t
Però vi ginro, chiio non ho'mai inteso ny
La fin di questi tali, € tengo a mente de
Quel c'wn tratto mi disse un buom valete, to modo, che non ve da le
STANZA V. » DSR hebeR Cy
Perché le son tutte cose provate, Come al Garani quand! a gal K
E vere, che non v'? spina ne offo, e-4ndato era la norte n
E non si tronan poi sempre le Fates Che, mentie oi by
Che vengano 4 leuarti il mal da dosso, Da esse ebbe un fanor di &












2

OTTAVO CANTARE 381

Poeta ifeguitare a narrare quanto avvenne a Paride s' introduce col
che nocumento sia ' andar fuori di notte, e che però sia cosa da,
, dente il'non considerare quanti pericoli si possono correre; Ed
nigliando la notte al Vaso di Pandora conchiude, che si dovrebbe imparar
i polli, che vanno a dormir subito, chee' s'é riposto il fole, e così sfuggires
te le disgrazie, perché non si trova sempre chi liberi dal male, come avvenne
a Paride, che dalle Fate fu liberato dal pericolo di morte.
 GIOSTRARE. O armeggiare. Mctaforicamente s' intende andar girando, o
pafleggiando senza saper dove, o senza fine determinato, che si dice anche anda-
» Oagironi.
ARIDVRLA « oro. Per ridurla alla. conchiufione. Vedi sopra C, 3.st. 43.
e malanni Diavoli,e Versiere. E' un modo di dire assai usato in simili
aioe per esprimere possono avvenire tutte le forte di disgrazic.
VEKSIERA, orl infernale, che dalle nottre donnicciuole e intesa per una
fla moglie del Diavolo. Forse viene dal Latino Ver/usia, che vuol dir
'malizia; e si dice Versiera un ragazzo maliziofo, fastidiofo, e insolente, ma
Spill veri » che venga dal Latino aduer/arins,col quale nome e disegnato il
7 nella scrittura., ddnerfarins nofter diabolus, Petearca.
5 1; Si che anendo le reti indarno tefe,
U1 mio duro avversario se ne scorni,
Da aduerfarius nelio stesso modo, che 1 Francesi fecero aduerfaire, così i nostri an-
: iz i's Auuerfiere ? anuerfiere, e poi finaimente /a Versiera. Ll Beato lacopone da












i





if

Hl canto 62.

ih “A Lo nemico ingannatore

oe o8aiiin::; * Anerfier de la Signore.

xs) Eecant2r. | Fata gli anerfere venire,

8 ' Chel degian accompaguare,

an Nell? uso dicefi Far la Versiera, fare il Diauolo, e peggio,

ib  INCTAMP.ARE, B il latino ofendere. Vedi sopra C. 1. st. 13.

i OT ASC.4. Quella facchetta, che si porta comunemeate appiccara agli abiti per
nfo diyportar roba necellaria alla giornata, come denari, e simili da' Latiai detta

it Pera, o Zona. 8

/ @ALLEGGERIRE di quel peso. Cio portar via i denari, e cos} alleggerirlo del

', pelo, edelia noia, che per quello gli veniva.::

a MANC A in che mo, Cioè sono infiniu i modi. Il termine mance in questo ca-

oe a0 sufato ironicawente, perché s' intende: Vou mancano s modi.

a  CORRER burrasca, E termine Marinare(co, che significa Correr pericolo, ed
in questo Ggnificato e preso comunemente, se bene berrasca vuol propriamentes

dire follevamento di mare per il cattivo temporale di venti, ec.

VASO di Pandora, E) nota la fayoladi Pandora, la quale fauna Femmina.,

( che Giove fece fabbricare da Vulcano, e darle.in dono di ciascuno degli Dei je



' parti, affine di farne innamorar Prometco,.¢d indurlo ad aprire un va-
7 fo pieno di tutti i mali, che Giove haveva dato alla medesima, che lo donaffea
jf Promotco » che vuol dire Prevvidente; che antivede, per vendicarf dell' ingiu-
, tia da eifo fattogli quando rubo u fuoco celefte, ma nonl'havendo Prometeo
oe voluto



rie =
*
















382 MALMANTILE ©

voluto accettare, lo prefe Epimeteo suo fratello, 4
fatto, il quale l'aperfe., e vennero fuori tuttii mali, che
questo e il vaso, che il Poeta intende nel presente luogo, ¢'
ni nel secondo capitolo della pefte dicendo:
To leffi già & un vaso di Pandora,
Che n' era drento il canchero, e la febbre,
E mille morti, che n' usciron fuora




Orazio lib, 1, Ode 3.
Post ignem atheria domo
Subduftum, macies,& nova febrium
Terris incubuie cobors,
La favola, e raccontata da Esiodo. è
RISICO., Riftio, o rifico dal verbo arrificarf?, arrischiarsi,o d
vuol dire Esporsi al cimento, o avventurarsi a qualche icolo. In'
Risco significa, rups pricipizio, luogo pericolofo. Cie, se bene mi f
quam in diffictle, & scopulofo loco verser, rificofo. “he a;
TRACOLLI, Da tracollare: altrimenti barcollare, che & fm
il Latino metare, o ritubare; e qui vuol dir Disgrazia,, o pericolo, by
ROMPICOLLI. Huomini; che consigliano, o inducono altri a far 2 |

Latino in omnem audaciam proieeti. A a Cu
T#STONE. Moneta Fiorentina, che vale tre giuli,o paoli, 9 | 4
VAL più a! un mondo., Questa iperbole significa non vi e prezzo, Ta

Star discoffo un mondo, disse il Bronzino nelle rime burlelche; cio Me

Spario. u

CERCAR di Frignuccio, Cercar le disgrazie. Andar incontro a' pores
Frignuccio dalle nostre donnicciuole è preso per il Diavolo, e diciamo ane ny
cercar il male come i eMedicr, I Latini in questo proposito dissero; Camarinam me \ 4s
xere da una pianta,.4a quale ha le foglie così fetenti, che movendole, @tocta | Mi






dole lasciano un puzzo terribile: o forse da una palude detta Camarina! do
cina al castello detto Camarina in Sicilia, la qual palude, perché cagi \
detto Castello la pefte, i pacfani domandarono ad Apollo, se era bene Q
re detta palude, e l'Oracolo rispole: Camarinam non esse mouendam 5 I
fatto poco conto di detta risposta,vollero feccarla, e n' hebbero il gaft q

i nimici paflando per quella palude già fecca, entrarono nel Castello, €
“TW bella proua. A pola; e I addicttivo beds s*usa in questi ca pes eal
IN bella proua. a; e I addiettivo s'ula in i cal '
r seaen un Gpuinivonmal dica in prouffima. Vedi sopra se
nell' uso: L' ho bell' e fatea questa, o quella cosa; cioè l'ho fatta fa
terminata, fornita. x
CHI cerca trona, Detto sentenziolo, che significa, che colui, che
al male, merita che gli succeda:
NON 0' e spina, ne offo.. E' negozio spianato. E cosa liscia, Non'
bitare, non ci e da incontrare difficulta alcuna i
AGAMBE alzate, Cioè col capo all' ingit'. Si dice anche ve a
are, Vs0 questa frafe Agambe alzare. Ser Brunetto Latini maeftro di D




tase n= ~sxse.










-OTTAVO CANTARE, 3 83

ovvero Capitoli pieni di gerghi, e di vocaboli Fiorentini; e vole spie-
'atto di chi iemwomnla in terra per iscaticare il ventre. Zvidi a eae als

ecanteee: con riverenza, cacava ) che questo vuol dire torrire in

ested col, ut

“EGGLAV A con la morte. Faceva conto di morire. Temeva di morires

nel mulino.
STANZA VI. STANZA IX.
efto vuol pur ch' io di Ini difeorra, Circa questo,pensiero elle non hanno,
Onde di nuono a i fatti suoi ritorno. Ne di fare altre spefe, come accade
Le Ninfe, eb? il vedean barter la borra Ad ogni galant' huomo.a capo a anno
Tutte gli son co' panni caldi attorno, D acconci,taffe, laftrichi di trade:
B già tra loro par che si difeorra UL vito,e il freddo non puo far lor dino,
Di fargli dare una scaldata in forno, Perch il tetto, che feorre,e mai non cade























ta perché questo in danno suo rifulta LZ? Inverno fui pilaftri di coralto
+ volleil/uo parere anch'ei inCfulta, Si ferma,e forma un palco di criftallo.
:S8TANZA VIL. STANZA X.
ino di non farn? altro; ond' esse Di Stare il Sole giu ne' suci quartieri
rivestire a [pefe loro; Non puo col frugnolone haver l'ingre/io,
7 icia nuova una gli messe, Tal ch' elle stanno bene, e volentieri,
« C'ba dal colloe da man trina e lavoro, E gedono un pacifico pufsesso,



Pr altra il ginbbone yn'altra le bracheffe Paride intanto infra tazze,e bicchieri,
"un ricco,¢ nobil quoio a' oro, E di più forte vini, e fructe apprefso,








« Fit altra gli ravvia la capeliera, Con è/se ritrovandosi in cantina, —
J&€ ette il benduccto,e la montiera. Valle provarne almeno una trentina,
oe STANZA VILL STANZA XI,
Alpalfe poi lo menan per la mano Ve per questo alterato egli ne restay

4 ta lor bella abitazione, Ovengach'egli¢ avverzoin dlemagna,

Ma poi pits buona,benché sia in patano, Oc' a faluar quel vin faccia la testa,

ea pagar non hanno la pigione, Ed in quel cambio dia nelle calcagna;

¢,un negoriv odio/o, e frrano Ragio, che quadra bene,e quedaye questa,
: quell' insolente del padrone Perch' ei non urta mai chi l'accepagna,

oad na a casa,e co si poca graria Ma sipre in tuono,e dritto com un fufo

Chiedeilfemeftrech'ei nov'é una crazia, Con efse per le scale torna fufo,

-olox gli pic's STANZA XIL

nv! Ow egli entrato in una bella fala, Di li poi falgon sopr' 4 un' altra sala
6 Ch ella sia l'Accademia si figura, Di baston congegnati infra due mura,

' he vi son aratolo, e la pala Donde, arpicando come fan le gatre

ti d Strumenti da Siudiar ? agricoltara, Vanno a pafsar per certe cateratte.,

, DiParide dunque vuol seguitare a discorrer il Rocta, e dice, che conoscendo
i ke Ninfe,, che eglifentiva un gran freddo 5 volevano metterlo a rasciugare, es

 Hilcaldarsi in un forno.,ma.egli non volle, onde esse gli fecero un vestito nuovo

















' (pefe nella maniera,.che viene espresso, in guetta Stanza fettima; Di poi
Jomenarono a vedere la loro abitazione, ed in cantina dove bewve assai;e.nony
-danno per le ragioni; che adduce il Poeta; e di cantina falirono alles




0



2 er
































22,

384 MALMANTILE?

BATTER [a borra, Iorendiamo Tremare, e battere i dei
do: E si dice così per la similitudine, che ha tal b
che G fa della borra:la quale e speci¢ di lana triturata col coltello e |
empiere i basti delle bettie da foma, ec. e per liberar devta borra dalla)
si mette sopra a un' afle forata con piccoli:, € speffi fori, © filbatte
di corde adattate a questo effetto; e questo battere fa uno firep: ct
che similiiudine col batter de i denti, che faccia uno tremante per ¢:
do, ec, Si dice anche batter la Diana; tremare tutto, stando allt aria, a€
scoperto; Latino /ib dio... Vedi sotto C. 9. tt..6. os 8 Coe Raa

BRACHESSE, Brache, caizoni, Voce Veneziana taluolea wfata anc



On ' '
QVOF d' oro, Pelli di bestic conciate, e dorate', servono per
ze in vece di drappi. ' at

GLI ranua la capellicra, Gli pettina la zazzera yO chioma, 3

benaa. Striscia di panno lino bianca, che s' appicca pendente alla
cintola de i bambini, perché si posiano con essa nettare ij malo, -

MONTIERA, Specie di berretta usata dai bambini. Dallo §;
tera, berrettino,

PANT ANO. Palude', che diciamo anche padule, luogo pieno:d?
ma, che renda il terreno inzuppato, riducendolo come fango, da i
detto Palus, paludis,,

PIGIONE, Cioè quel denaro, che si paga per fitto d'una cosa; E
con termini proprj tro si dice quel danaro, che'fi paga per poderi, et
pigione si dice quel denaro, che si paga per Case, o botteghe, dicendo
botteghe, o casamenti: Ed appigionare case,e botteghe. Di queste si dice
ma dei terreni mai si direbbe appigionare. Pigione dal Latino
forse da fexdum, fio, e questo dal Latino fides, ¥

STRANO., Stravagante. Qui intende noiofo, odiofo, fastidiolo..
frrano dal Latino extraneus ritiene anche appretio di noi il significato di
ro, o lontano dal parentado nostro. Vi/o ffrano, vuol dir vilo arcigno
© crucciofo; vifo rane vuol diranche faccia macilente, e pallida,”

SEMEST RE, Numero di fei mefi; ma intendi il denaro, che si
pigione di fei mefi. Le

T ASSE, e laftrichi di firade. Spe(e, che occorrono farsi alla giornata
ro-, che posleggono case in Firenze; che /africhi, intende quella spela
partice fra i padroni delle case per raflettamento, e Jaftricamento
della Città.

TETTO, che sempre feorre,¢ mai non cade., Abitano sotto acqua 51
il loro retto', che sempre feorre, e mai non cade," 1 abt
PILAST Ri di Coralie, Pilattri si dicono quelle colonne fatte di
altri safi, per foftener volte, Latino pile. B percht'il corallo nafo
finge, che questo tetto f regga loprai pilaftri di coralloje vaol di
werno s' agghiaccia l'acqua, e fiferma. 1 boy poner
NON pro col frugnolone haver ? ingreffo. Non può il Sole trama
netrare i suoi raggi fotco l'acqua, Fragnojone da Frugnuolo detto









2.2. -Seeeepeee seers eee =

 =

BRipnmaes..









OTTAVO CANTARE. 385

ae 'RATO. Commoffo, o perturbato da qualsisia accidente. Ed alterato
dal vino vuol dir Briaco. Onde gli Alterati Accademici già famofi in Firenze
Bee ates





o per Impre(a un Tino; in cui Gi pigiava l'vua, e ogni Accademico usa-
per imprefa particolare cose attenenti a vino; si come quella della Crusca,
le succedé, usa per imprefa tutte cose attenenti a grano.
ACCIA a faluar la tea. Non offenda cot suoi fumi la testa, perché e vino
- Detto (cherzofo tratto da quelli, che giuocando di scherma non fanno
gioco, ma pattuiscono di faluare la testa, cioè non si colpire nella te(ta.
GION che quadra benese quellaye questa. Tanto può esser per questa ragione,
per quella, che egli non Sa rimatto alterato dal tanto bere.

NON urta chi o accompagna, ma è sempre in tnono. Non barcolla come fanno i
riachi, e non da spinte a chi è seco, ma sta in ceruello, e va dritto.

| ARATOLO.. Si dice anche aratro dal Latino. EB erato si trova nell' antico
“Volgarizzamento di Palladio:; donde e fatto il diminutivo drarolo. Strumento

quale i villani'rompono la terra, facendolo tirar da i buoi.

'PIC ANDÒ, Bi il verbo arrampicare fiacopato, e vuol dire il falire, che

oi gatti sopr' a ua' albero, o simili, e viene da rampicone, che & un ferro
inde Die, che usano i marinari per pigliare, e fermar le navi. Latino
ro  harpagonis; da che noi pure lo diciamo anche arpazone, e arpagonare,
CAT ARATTE. Et voce latuna, che vien dalla Greca catarrbattes, con las
intendiamo ancora quelle buche fatte ne i palchi,per le quali si pafia di for-
entrarein luoghi superiori con scala a pioli, come farebbe falire per di
faltetto + E per lo piij tali cateratte s' usano per entrar nelle colambaic;
a sorta era la cateratta, che dice in questo lyogo.
















TANZA XIII, STANZA XV.
4 qui la Mula vnol ch' io mi dischiari Horsi per ch' io non caschi nella pena
Circa il deferiner queste loro Stanze, De cingue solds; ecco ritorna a bomba
Che stio vi pongo addobbi un poordinari, A Brache d'or, che nel (alire arrena
Non per dir bugie, ne ffranaganze; Per quella seala, che va fu per tromba,
© Peréhiile Ninfe han folo i necefsari, Perché se bene ei fail Adagia da Siena
wv ie moderne usAnre, Gli è difadatto, e pefa chregls spiomba,

i
Per infeonare'a noi c habbian le borie E con le Ninfe a correr non puo porsi,





4 adri',¢ letti d'ore,e tante lorie, Maffime liche v' e un falir da Orsi,
atl STANZA XIV. STANZA XVI
i; Ch ognun vuol far il Principealdid'eggi, Elle di già, com' to diceua adesso




rl | Seben chi la volesse rivedere y Vicite son di sopra a stanze nuone,
Melti firvegvon far grandezze,e sfoeei, eApettando, che facia anch'ei Listesso,
cae,







(pecchto poi col rigattiere: C' appunto com il cambero si muone;
a sa lnfsobgrande, egiaregnainsns poggi, Onde connien poi loro andar per efso,
 Efon nelle capanne le portiere + Ed aiutarlo fin, che piacque a Gioue,

4 | Bera icannelt infin qualsivoglia unto Che quasi manganato,e per strettuio

Had fuck fhiperti, e seggiole di punto. Pafsafse ad alto il Caualier di quoio,
Pr I Autore di voler dire la yerita, prega il Lettore a non pigliare
wzione, se in descrivere le maficrizie delle Ninfe metterad addobbi, ed ar-
Refi un poco ordinarj, perché in eftctto ead così; e da questo pigiia occasione di
tye Lee biafi-




SS



























































386 MALMANTILE

biaiimare il Info, che è oggi in Firenze. Di poi tornando
che le Ninfe falirono alle stanze di sopra, doye con gran fa
de, il quale chiama il Cavalier di quoio, perché era v
demo. shee
ADDOBS!, Mafferizie, ed arnesi per uso 5 ed ornamento
verbo addebbare, che vuol dire Adornare. Du Frefne nel Gloffari
dig Latinitatis. addobbare, armis inffruere, militare cingulum alic
confetka ex adoptare, quod qui aliquem armis instruit, ac militem ne
modo adopter in filixm, Si che Addobbare secondo questo autore vi
folennita del vettire i Cavalieri,
ZORIA, Aibagia. Vanagioria. » oe
SFUGGI. Vlanze fontuote canto di vestire, quanto d' addobbamenti
fatti con (piendideza, € pil del consueto; Donde si dice fare sfoggio, 0
quando i trutti fanyo quantità grandissima di frutte, o quando chi
più del solito; ed in somma s' intende d' ogai operazione, che esca del
© cel naturale; come si dice frutta sfoggrata quella, che eccede img
beilezza, e supera l'altre fructe della sua specie. EB la forza della
venendo da seggia, cioè ulanza, el solito, antepostavi l', s, vuol dir.
foggia, cice tuor del solito, e del consueto. Gio, Villani quel che noi
foggi, chiamna difordinati ornamenti lib. 9. ¢, 245.5 e lib. 10. Cap.t
mo autore lib, 12, cap. 4. £ mon e da Lasciare as fare memoria d' una,
mintazion d' abito, che ci recaro di nuovo i France(chi. EB poco foro,
natura fiamo dijposti moi vani cittadint alle mutazioni de' nuomi abiti
trafare. Sfoggio dunque vale fuori di foggia., cloé dellafuzione, o VO
y.cniera' di fare ordinaria, e usitata; che il Villani comes'é villo
sformata mutazione a' abito; e disordinati, e sconuenenoli, e difonefi, ef
mienti ye nuoni, e iffrani abiti,:
CHI (a volefse rinedere. Cio' chi la volefle bene efaminare, o rice!
maniera quefii cali possano fare Gimili sfoggi + 'i
SONO a specchio. Hanno debito. Traslato da coloro, che hanno d
decime, che si pagano al Principe y i quali &i dice esser'.a specchio, p
notati a un libro, che si chiama lo specchio.; Qui dicendo.; sono
ghattiere, da:due colpi, uno che coltoro, che fanno tante borie non
gate, ¢J''altroy che questi loro sfoggi sono di robe usate, € vedute
poiché l'ha prefé'dal rigattiere, che vuol dire Vao, che vende mafleri
ed abiti usati. Vedi sopra C. 3. st 5.
POKTIERA, Paramento di drappo, o d'altro, che serve per
porte delle Ranze nelle case Civili. Da alcuni detta in Latino velum ada
TRA icanneili. Vuoldire fra la gente più vile; perché fra i cannelli.
mo fra i tefitori di lana, che son gente d' infima plebe, ed.¢ lo stesso
qualfineglia unto; perché questi tal: maneggiando sempre lane unte
sempre unti; e qui aggiungendo al detto fra i canned, il si
intende, che fino i Batulani, che fra gli unti sono i più vili »fanno le!
SEGGIOLE di punto, Cioè seggiole ricamate, o trapuntate di
mo: Panto Vaghere,0 punto Franzce, 3





BABeRew eft gFen2n2 se 8s =

=

a=
Se =z

SURFER






*' OTTAVO CANTIARE: 387
CASCAR nella pena de' cingue soldi. Quand' altri nel discorso fa una digrettio-
ne, €non torna mai a) primo proposito, gli diciamo: Voi cascherere nella pert.
de' cingxe soldi, 1 Varchi nel suo Hercolano pariando di questa pena dice: E chi
cominciate alcun ragionamento,e pot eutrato in un' altro, non si ricordaa prit di

nave 4 bomba 2 fornire il primo, pagava già, secondo teftimonio dal Burchiello, ni.

te

'a

4
wail
a



offo, #1 g 'o non valeua per aunentura in quel rempo più di quei cingue soldi, che
a ced Nae quali Lacie vegghiamo, che ST rarchs si serve del detto
Tornare « Bomba per tornare a segno, o al proposito del primo discorsa, come fa
il nostro Autore nel presente luogo. L' Ariofto Satira prima dice;

= Ma perché i cingue soldi da pagarte,

t Tx che leggi, non ho, ritornar vuglio

7 La mia favola, donde ella si parte.

 eARREN A, Intoppa; Si ferma; Non seguita il viaggio. Traslato dalle na-

Viquando si fermano, perché tuccano il lette dell' acqua, che si dice arrenare, 0

incagliare, De 1 gual) verbi ci serviamo per e(primere non tanto il fermarsi in un

'Wiaggio, quanto il fermarsi in un dilcorso, o nel proseguimento di qualfivoglia.s
'aaione, negozio. Latino hnerere.
 PAil mangia da Siena, Fa il bravo. Fa il valorofo. Il Mangia da Siena è
\ di metailo atiai grande, la quale è posta sopra la Torre dell' orivolo
del Comune di quella Città, la qual figura dicono, che sia il fimulacro d@' uno an-
tico huomo bravo detto ii Mangia; Ma io son d' opinione, che ella sia il fimu-

lacro di qualche antico Podefta di Siena, e che habbia acquiftato il nome di 423-

54 da qualche inferizione, che havefle appresso, la qual dicesse Il eA/agna di Sio-
WAS COR i) ALagnifico di Siena, che s' intendeva già il Podefta: Ma sia com ef-

fer fivoglia, a noi basta sapere, che questo detto serve per in tender con derifio-
ne un bravo, o valente; quasi voglia mangiare le persone, e ingoiarle,

DISADATTO, Contrario d' Atto, destro, agile, ec, Vno che duri gran,
fatica a maneggiarsi,o muoversi per la gravezza, o per altro accidente, Sciar-
feancora e contrario di arto, e significa uno, che fa male, o negligentemente
quel 'ch' e' fa; poco pulito nelle sue faccende, e nella persona,

CON Ie Noha correr non puo porsi, Non può gareggiare con le Ninfe a chi pil
corre. Interide, che le Ninte al sicuro lo superercbbeno nel corso;

VP Bun falir da Orsi. V' & cattivo, o difhcil falire. L' Orso è un' animale, che
f ben, ir goffo, e difadatto, nondimena e assai destro, e facilmente fale anche
in ionghi inaccefibili; donde noi habbiamo: Efer come L' Orsa, cioè Sofuse destro,

Ui Berni nel Cap, al Fracaftoro dice:
Shy. Conniene ivi lasciar t" xfato corsa,
«ta £ falir fs per una certa [cala,
i Dove hauria rotto il colle ogni destr' Orso,
'Ostiero nell Iliade al nono chiama una rupe 50 balza Aigitips, cio8 dalle capre abe
“Vandonata; © queito medesimo nome di Leeips danno gli antichi a una Città dell'
Afola di Cefaionia, € ua' aitra dell' Epico. Noi diciamo di Jaoghi simili erti ri-
'Pidi, © feosceli: Won vi falirebbero le capre, le ee Virgilio nell' Egloghe dide:
repe. Quella montagna altidima nell' India; fu'la quale fu il primo Ale(-
fandro Magno a falire, fu detta da' Greci eornos, cioè senza uccelii, quasi mon.
oksal Cec 2 tagna





ca i






































388 MALMANTILE

tagna da non potersi ne anche da chi aveffe l'ale formontare
S7 muove come il gambero. Cio' va all' indictro, Wepam
MANG.ANATO, lnfranto; Mangano ( dal Greco mage:
na, con la quale si distendono 5 e si-da il juftco.a i panni, ¢
fare a forza di rulli sotto un gravissimo peso, e tal panno 5 0
si dice poi manganato. E Mangano come s'accennd sopra Cw.
na militare della quale i nostri antichi si servivano per (cagliar'p
fiediate,'¢ con essa scagli anche | ini, che dicevano poi
ganati, cio' sflagellati, e pelti dalla percofla; e così si potrebbe inten
ride; ma perché soggiunge paffaro per frretrow, che è un' altra machina, ¢














ue per stringer ulive, ec., © per mettere in piega a panni, si vede, che
quel mangano da panni.
Ss ZA XVIL
WN un Dormentorio grande, ma diverso,
Ove ciascuna in proprio ha la fuaceltay
Che sia com' io dir per questo verso,
( Se non erra Turpin, che ne favella )
Vana fanga a mez aria tuna travorfo, ¥
Dow' alla tien se calze,e la gonnella, w
i penzol delle forbe, e del trebbiano, ag
E quel che più le par di mano in mano; fa
TANZA XVIIL. ae
Più git da banda un tavolin si vede, er
Che sa i tre/polifa la mnna nanna, Sxpenfa, che vi ina
E sia spatiiera al muro, ove si vede Ada trova im ozso tutti gli 1
Via fiuvia di giunchi,e fottil'canna, E i piatei ripulisi come sp fai
Evvi una madia zoppa da un piede, Teglie, e padelle, inutile a
E il filatoio con la sua ciscranna, Star'appiccare al muro per gli si
Won v'é letti, se non un per micliaio, Ed anche son per sparut pilt a et
'Che tutte quante dormono al pagliaio, Perché il gattoa dormir vedein, | dy
: STANZA XX14, sth ng
Ond? egli offefo molto se me tiene, ») (Gliaccanan ch'ei vedrà fel ta
Ch' una mantita per la golatocca; Ed ei ghignando allor pits noms (i
Ma quelle, che s' avveggon molto bene, E con esse ne va di compng §
Chregli bal'arme diSienarpreffain bocca Per ultimo a veder la Gi 7 hk
De(crive nelle presenti Octave il dormentorio delle INinfe,e ledorowmafieria® | Yu
Arrivano ee cucina, dove Paride a (aol e de pr ta
arata 'cosa alcuna 'per mangiare; Ma ie Ninfe lo quictano'con dirgli, ch e
ey ada eiare} 0d de lo'condi a veder la Galleria, Pe
'DiVERSO, Differente., o 'diffimile aghi-altri Dormentorj, perché: «
Celle non 'fon fatte di muraglia, ma son tutte in-una grande stanza-y §
vile con stanghe app al:palco ciondolot foa r
quali poneng jo ciascuna'le sue 'robe, e panni le 'fa servire per muro
.così vengono 'fortnate'le Celle:. 'Si può anche dire, yche la voce
dae significati il primo,'che vuol dire diferente ( e gquesto f a









OTTAVO CANTARE 389

'meffo per contrapposto, come la-tal cosa e diversa dalla tale ) il (condo quando
po 'ate che vuol dire strano, o stravagante, il Poeta lo piglia ias
Recornto significato. — lo piglid Dante Inf. C. 7,
 Entrammo g pen via diver[a, Oc,
Cavaleamti nelle sue storie lib, 12 parlando di Cammillo quando ifefe il
lio dice:, Non guardo all' ingiufto cacciamento, ma con grandiffi-
i yy mo esercito corse alla dife(a della patria, e liberolla da così diversa fortuna.
7 es, Ricordano Malesp. Stor, Fior. cap. 80. dice: E ciò fu per l'inuidia della Si-
> ae i » che non era.al loro volere, e fu diversa, ed aspra guerra. Vedi. fo-
int 2, stan. 3.
bis Beene del trebbiano. Che cosa intendiamo per penzolo vedemmo sopra C.
6. stan. 50. e Trebbiano € specie d' uva bianca, ma.qui e preso in generale eee
(IL ogni sorta.d'uva,\che's' appicca nelle stanze per ferbare all' Inverno.
 DI mano in mano. Di tempo in tempo. Lat. Deinceps, che s' intende fuccelfiue
neat:  Cic, 7, Ep. Fam, disse De manu in manum. Dan. Par. 6, dice:
jas E [otto t' ombradelle Sacre penne
we \Governo il mondo li di mano.in mano.
yin
ye
si
yh
ue
ae
lf



. Bd  detto figuratamente-dal far paflaggio una cosa dalla mano d' uno nella.
'mano dell' altro.. Dal giuoco. detto Lampade dromia, nel quale colu aveva il
-vanto che va una fiaccola accefa correndo., e così bella, e accefa la confe-
gnava.a chi aveva.a correre dopo di lui; disse Lucr. lib, 2. Augescunt alva LEnbes 9
lia minuuntur, Inque breni [patio musantur fecla animantum, Et quasi curfores vite
Po eamema 9 Gi0e succede t'.uno vomo all' altro, l.nne vinente all' altro di mano

0 roc, Dal Lat. tripus, odes, E un pezzo di legno, o ceppo., in cui
a son fite tre mazze, fope' alle quali posando., serve;per fohence tavole, e deschi,
oe da i Latinidetto Trapecophorus.s'quali mensam ferens.

' | PAlaninnananna. Non Ma forte in terra, ma dimena o:per l'inegua lita de-
ri N 'te tre mazzc, o del suolo,,.o per altro mancamento; e diciamo far /a ninna manna
Od eda,quel dimenare;che si fa-della,culla.de ibambini, 'quando dallebalie si procu-

ache dormano, che si dice ZVmnare, spetche per lo più sogliono accompagnare

4 -talmoto.con una lor cantilena, che dice Ninna nanna il mivbambino. Vedi sopra
iS 'Cae Renaes. Questo dimenare si dice anche:ewlare pur dalla Quila de' bambini..
, SPALLIER A, Quella'parte della seggiola, alla quale:s' appoggiano le (palle
mA Aedendos |B per /paliiere intendiamo quelle nuragli¢ 5 alle-qualt sono.appoggiate

spianted'agrami, ec. come's'¢ detto sopra\C. 6, stan.'51. Questo artitizio-di
Farele:mura:coile piante-dicefi-da alcuni in Lat.-opus topiarium.. Equi nee
'quel-'muro parato di stuoie tatte di giunchi, o-canne paluftri, che fourasta.alla.
oat »sopr' alla: quale dice;che fedevano le Ninfe, ¢serve,per spallicra alla.me~



J

3 STVOLA, B il Latino Storeache conlerua appresso noi il suo:significato.,

il HUADIA, Dal Latino maetra,i| qual pure e Geeco;.ed una cafla sadatrata
it sopra-quattro:piedi, dentro alla quaic si lavora la patta per far-il pane; La dices
3 Zoppa.da'un'piede perché le: mancava., o crarottouno diguefli piedi.. Zoppa si-
'il siete den tee cain tavon della vecchierella Bayside la;presso rida




390 “ MALMANTILE

lib. 8. delle Trasformaziuni; ma ella la fece stare pari con me!
to; mensam fuccintta, tremenfque Ponit anus; mensa fed erat pes ters
sia parem fecit, i, 2/1 ihe
FILATO/O. Strumento col quale per via d' una gran ruota si fila Jan:
napa, ec, e si fanno le'funi. 1) OL HRS
CISCRANNA. Specie di seggiola come accennammo sopra C,
DORMONO «i pagiiaio. Cio' dormono in fu la paglia.
HVOMO alia buona, Huomo schietto, fincero,e senza malizia; Huo

za cirimonie, e nimico del luffo, e delle boric fine fuco, © fallacijs, | Ve

sornm, ed Hxomo posirixo intendiamo uno, che non fa sfoggi nel veltire, ¢
ogni cosa si tratca senza lufflo. SOS
SENTITOSI allegare i dents. Vuol dite sentitofi (timolare dalla golae dal
desiderio di mangiare; se bene allegare i dents vuol dire quando i deat pert
matfticata qualcola acida, o agra. coine 'il limone:, ec. s*iavormentileono, e i
sente una certa difsculta nel mafticare.Ma usandosi come nel pretente iuogo,vu0
dir venir yoglia di mangiare.:
TEGLIA. Specie di tegame fatto di rame stagnato per di dentro, serve pe
quocervi torte, e migliacci, ec. (| Monofini lo fa venire dal Greco Telia y a
gual voce tra l'altre cose signitica 1' a/se da pane, e"| turacciolo,o coperchio del fum
maiuolo, o vogliam dire di quel canale, che gli antichi, in vece di cammino,ave
vano per servizio di cucina, buono folo a ricevere,¢ porcar via it A
dicendolo molt Tegehia, e gli antichi in particolare, mi muovo a
venga pil costo dal verbo Latino Tegere. Queste teglie hanno nell'
ta una campanelia di ferro per comodita d' appiccarla, e le padelle hanno un
anclio in cima al magico per il medesimno essetco'; € questi fond gli orecchi de'qua-
li parla il Poeta dicendo: Stanno appiccate al muro per els orecchi., Ovidio lid. &
Metam, erat aluens illic Taginexs, dura clauo fuspenfus z anfa, hia
TORN/ZE, Parlando di gatei s' intende quel ronfare che fanno; perché e &
mile a quel romore, che fa il tornio quando gira. + Aba
TOCC A una mentita per la cosa, Dar una mentita per la gola a uno e quando
se gli dice, che egli afferma il failo, ed e grandissima ingiuria, e che muove al
¢ pero il Poeta scherzando dice, che\Paride si adira per I offela, che ri
quella mentita per la gola, cioè di quel tupposto che vi fufle roba per la golasy
che fu falfo, WS EE
Li arme di Siena imprefsa in bocca, L' arme di Siena è una Lupa, ed il mal dé
la lupa e inteso comunemente per una infermita, che fa stare il pazziente in col
tinova fame; onde quando vogliamo intendere; il tale ha gran fame diciamd:
Egil ha il mat della iupa, e pis copertamente Egli ha ? arme adi Senay es'
la lupa, cioè la fame. Vedi sopra C. 3. stan. 22. Kh
VEDKA', # il corpo teene, Cioè mangiera, e bera. Detto assai afato”
gente di vil condizione. » 1. 3a
GHIGNANDO., Ridendo leggiermente. Lat. fubridere,
GALLERIA, Cosvin voce straniera chiamiamo aicune Manze piene
nate di-galanterie, ¢di-cose singolari, e maravigtioe 3 quali ttaazes
son dee Pmacorheca dal Greco Pmax, che suona tabula pita, © theca

oe
















oe. >Eeerec-se.= ez ER PTE







orre al
STANZA XXIL

fb Principi ritvatti ye di Patrizzi,
 Originali farti già in Fiarenga

4) Da quel, che gis vendea sotto gl' usizri,

» Ed euns dello feeffo una Sibilla

 Eduna bella Cittadina in villa.

Bt STANZA XXIIL

me

SE aa

tapelPa fole, e sgabelli
intorna inalzan sopra al piano
 Statue eccellents di quet Prafiteli
BH Già shalt danno il moto in. Settignano,
Ce * Buonarrnoti, e i Donated

Caer

— Aquel baffa ritseva di lor mano

z



ORTAVO CANTARE. jor
erste: Sper altro Galletia voce militare e specie di fortificazione.
xX s

TANZA XkXIV.

Si che que? opre, che non hanno pari,
Quantoi (uddetti quadri,c' han del vago
Non si polfon pagar mai con danari,
Perché son giore 5 che non hanno pago;
Vao feaffale v' e di libri vari,

Ch' eran La libreria di Simon Mago,

C' abbellita di feorie 5 e di romanzi

Fu pot venduta lor dal Pocauanzi,
STANZA XXV.

Exni un tomo fra gli altri scrittoa penne,
C' ame par bello, e piace fine fine,
One si legge in carta di cotenna
Tradotte le librettine in feftine,

E che Gateno, e il medico eduicenna
In musica mettean le medicine;

ww &

Leo, s' sl corpo sempre a chi le piglis

GC as paari scalzi pur si vede ancora
' Gorgheggia,¢ canta,noné meraxiglist,

| Sw t arco della porta per di fuora,
+3 9A da principio'a descrivere la Galleria delle Pate, e narra la bellezza
4 aicune pitture, e statue non diffimili dal refto delle maflerizie, per esser' opra
ade ad pilicimuniti: Artefici 5 (e bene scherzando gli efaita sopra i più eccellenti
Macitri, Oitre alle picture ve anche wo foaffale pieno di libri dei medelima yaio-
IE ye. » che sono te pitture, e scolture.
FRONTESPIZZ/. Vedi sorto C. 9. stan. 15. i
MAIOLIC 4. Specie di piatti, ed altri valeilami di terra, la quale meglio,
Che im aites iuogh: si Javora oggi in Faenza;¢ questa terra è detta maiolica dall
Mola di Adsiorica, o Adaiorca dove già si fabbricava; €1' lfola che diciamo oggi
Maiorca già si diceva Maiolica,, come si vede in Gio; Villani lib. 4. cap. 30. WVe-
54 anni ds Cristo 1117. Gui Pifani fectono nna grande armata di Galees e Navi, ed
ndaronofopr' ali' Hola di Adavolica., B che iaquelta lola si fabbricafiero tali va-
lami si deduce, non folo da] nome, che ritengono di Maiolica, ma anche dal
Vedersi nelle fabbriche antiche di Pifa 4 etparticolarmente nelle facciate delies
Chiele murati di tai piatti come per trofeo, e memorie delle vittorie havute da
i Pilani contro ai Maiorchin: 5
VNA belia Cutadina in vila, Era già in Firenze un Pittore da pochi soldi, il
quale faceva ritratci di Principi, di donne Fiorentine in abito da Villa, e da Cit-
ta, de Sibuie yle Mules ec, -¢ tucto.così malfatto, che non eran comprate tali
picture se non da genti di contado,¢ per vilidimo prezzo. Dette pitture si ven-
devano sotto le Logge, che (ono d' avantia quelle stanze, dove si radunano i Ma-
giltraui di Firenze,c questo luogo si dice sotto gli Vfizei y e per una bella Cittadina in
Villa, € una Sibilia intende di queste belle pitture.

D1 quei Prafitelli, Di quelli Scultori valorofi, e celcbri, come fu Prafiteles;
/parla però ironicamence, e per derisione. Praffirelle detto poeticamente come
Annibaile, Eetorre y e simili per ia rima,in vece di Praffitele, Annibale, Ben '

“hoo 0

=SERS BQ SSESHER ESS

AS

2s
Se

=

a= EE




392 MALMANTILE™
Così i Latini raddoppiarono la Lat. in Relligio, x Relient a ¢

la legge del verso. 4
CHE a i faffi i. daensit mined Settignano. Dar il moto ai fafti,
si vuol dire Formar figure di pictra: Virg. vines ducene de marmore
Settignano Borgo vicino a Firenze abitano quasi cutti fearpellini
fabbricano poco altro che stipitt, (Caglioni Primi if
che di case, ec, talvolta Javorano anche delle figure » ma per lo
le suddette pitture; e pero il Posta (cherzando dice, dannoi moto
che voglia dire animano i fafi,, fabbricando statue, che peiolowive
de, che danno il moto ai (atli, cioè gli muovono, ed e s I
quali sono ia.quei monti di Settignano, lyogo detto così quali
dere, o posieilione della casa Seprimia, antica Romana, siccome
della Perronia, e altri molu iuoghi dello Stato 5 che risepgane anon
padroni, nobili Cittadini dell antica Roma,

QLVEL bafso ritievo di tor mano, Se, Perché fir' c d
erano queste statue » porta I esempio d' una figura y che &: nell irchi
ste della Chiefa di 3. Paolo de i Carmelitani Sealzi, che è-una

afio rilievo, la quale rappresenta, o almeno dovrebbe rap,
lo, maé lavorata così maravigiiolamente male,.ches'é rela
sua storpiataggine; ed €-compagna delle (tupende pitture del Pamo
Zannino da Campugnano. Jntendendo dunque al omen Boera
tre figure, che le sono attorno fatte della medesima maniera vuol
che si vedevano in quella Gaileria eran maligimo fatte,

NON hanno pago. Non hanno prezzo: E? parlareironico, e-wuol
hanno prezzo, clot non s' apprezzano 5 non si fimano, non vaglion A

SC.AFFALE. Armadio aperto fatto.a palchetti per uso di tener libri. |
nome di Sehapha, e di Scapbos si dicono in Greco molti arnesi,e stramenti, 1
tutti'o concavi, o- (cavati per uso di tener roba 5 dal verbo scapresm
re cauare,scanare, Onde scaffaie, arnese y che ha varie capacita 5 €
ne' quali si ordinano,¢ si pongono i libri Lat. platens armarium

SHMON Mago, Fu lt Autore, e capo de'. Simoniaci; essendo-
che tentafie di comprar da 5, Piero i beni.Sacri, e Spirituali, come si
acti degli Apostoli. & che cosa sia Mago, Vedi sopra C, 1. stan. 20,5

POC AVANZI, Fu un Libraio Fiorentino così detto, ii quale nel'
 Autore compose la presente Opera era ridotto in poverta, € vendeva'
che leggende.

CART A di cotenna. Intende Cartapecora.

LIBRETTINE, Quel libretto, che insegna conolcere le figure dell"
¢ le prime regole del medesimo. Il Burchicilo..Vedilo andar,¢h e* par
tine, cioè e tanto magro, fecco ye ee C' pare una signrad
tini un macilente, efienuato » e deforme nelio fieflo modo.
grammo, ioe delineato solamente y¢ fattovi il (olo,¢ puro din

o colorito.
MEDICINA. Quando si dice semplicemente medicins da noi?
 bevanda folutiva, che si beve'con la preparazione, oe ¢
ta prima con alcuni sciloppi, cc.





































OTTAVO CANTARE., 393

| GORGHEGGIARE, E} termine mufico da i Lat. detto Vibrifare, ed & un tril-
lo di voce fatto con la gola, al quale in un certo modo € simile quel romore, che
fa nel.corpo il vento, o altra follevazione d' umori cagionata dalla medicina,
ed il Poeta ii » di questo romore, che fa il corpo dice, che il pazziente>
può far di meno-di non cantar così, poicht Galena, cd Avicenna haveyano
in musica tali medicine




b STANZA XXVI,
ave n't in rima, che la Sfinge e detto Perch' ci, chefa, chee Sale bebbe concetto,
| Seelta d Enigmi, che non hanno nguali, Acciò che i versi suct sieno immorcali,
Perch! agnune e distinto in um fonetto, £ i vermi dell'obblio non dien or noia
we Che if Poeta ha ripien tutto di fali; Porgli fra fale,e inchioftro in falamoa.

Bra questi libri delle Fate si crova anche la Sfinge, che è una scelta d' Indoyi-
i distinsi ciascuno in un fonetto, opera del sig.\ Antonio Malatefti; la qua-
Ieil nostra Poeta ( facendo di efla quella stima che merita ) non haverebbe metia
ion le, se il medesimo Malatelti non ! havefle forzato a farlo,com-
lo egli medesimo la presente Ortava nog alterata punto dal nostro Poeta.
fale Opera conticae ( come habbiamo detto ) Indovinelli, il Malateiti
il nome di Sfioge, che fu un Maitro appresso a Tebe, Figliuolo ( secondo
no )del Gigante Titone, e di Echidna, che significa Vipera; e Fratel carnale,
il, della spaventola Gorgone, del Can Cerbero, del Serpente
di pi tefle chiamato Idra, e di pi altri mostri e animalacci, il qual mostro di-
4 -tmorava in-un monte contiguo a Tebe sopr' ad uno scoglio vicino alla strada, ed
| a chiunque paflava proponeva wx dubbie[ che i Greci dicono evigma, i Latini
nt 'uphas pure dal Greco; e noi indoninello come sé detto sopra C. 6, stan. 34.]e
; Leqlioa ace lo scioglieva, il mostro improvvifamente lo pigliava,e}' uccide-
'i va. Agcadde, che Edipo figlio di Laio Re di Tebe fu quivi mandato, ed il Mo-
, fico gli propote: Qual' era quell'Animale, che da principio andaya con quattro
“ piedi, poi con due, ed in ultimo con tre = Edipo rispole, questo esser ' huomo,
"i, che da bambino va carponi con le mani, e co1 piedi, € così con quattro piedi,
se poi rittoin fa due piedi, ed in vecchiaia con tre, perché va col baitone; E con
tal folygione vinfe il mostro, che percio si mori. Pe
 RIPLENO di foi. Ripicno di belli, ed arguti pensieri. I Latini ancora chia-
vif mavang falil' arguzie, trovandosi in Orazio.. Nofri proaui Plautinos landanere
fale, Giulto Liptio Aatig. lect. dicit se amare elegantes Plauti fates, Lucano: Non
ie Solici ifere fates. Tor. in Eun, Qui baber falem, qui in te edt, intende scienza, fa-
“if 'pere. Ma qui.' Autore scherzando con l'equivoco del fale dice: Che il Mala-
teftisil, (a che cosa e il fale,¢ che cifecti partoriica [ perché egli era guar-
dane azzini del Sale di Firenze] 'ia meffo de i fali.ne i suoi fonecti, per
#1 fr loro falamoia con } inchioftro, athaghé i suoi versi si conferuino, e G
mw?  difendano da i tarli della dimenticanza, sapendo, che il fale conferua, © difen-
we ic ins; e le composizioni si conferuano da i vermi dell' obbiio con,
g@  Icriverle, € queito si fa con ".inchioftro, e pero lo chiama falamoia, I Latini
cone la la Murra, del che noi componghiamo la voce /alamoia, quali
falis mursa, 1' iachiottro da Monsignor Ciampoli fu chiamato dal con(eruare s¢
Orie € i noint degli huomini Bai/amo della fama,
t Ddd STAN.







“

a
























394 MALMANTILE ©

STANZA XXVIL
Altri Poemi poi vi sono ancora, E uncerto Mal
£d hanno caparrato alla Condotta Ecco subito bell! ¢
Grillo ilGiambarda, Ipolito,e Dianora Le Deecol Babi, chel ha
1 ferre Dormienti, e Donna Ifotta; i Z
Narra che moir' altri Poemi sono in detto scaffale, e mette t
frottole composte da' Cicchi per le donnicciuole, e per i fanciulli.
genie dice, che fara ancora la presente fuaopera, '
sNC AP ARR ATO, Data la caparra cioè dato danari innanzi per fert
mercanzia per conto proprio. ( Voce formata, dice il Perrari, da cape a
Qui vuol dire che hanno chicfto lu MALMANTILE, Gili antichi d
rare da Arra, caparra. "
ALLA Condotta, Così & chiamata a Firenze una strada, nella quale
botteghe i Librai, e alcuni Stampatori, ed e così appellata, perché
stima strada haono i magazzini coloro, che tengono 1 muli per lao
mercanzie a Roma, a Bologna ed altrove..
MESSE in rotta le Dee col Bambi. 11 Bambi era uno, che vendeva
maggio, ec. che noi chiamiamo Pizzicagnoli. Dice che le Ninfe sono p
car lite con detto Bambi, perché eflo impedira, che elle non habbiano il B
di MALMANTILE,, volendolo egli per farne alle accinghe tance ¢:
per inuoltar falumi. Ed in fuftanza vuol dire, che la prefenve sua Opera'

2 2 eee.

r=

na per vendere a peso per carta al pizzicagnolo; che così diciamo
che un libro non habbia in se di buono altro che la carta.. E qui se
dice questo per sua umilta, e modeflia [ non essendo la sua Opera da
pelo per carca j tuttavia, non sapendo che la mia penna dovea farle meritare
tal fine, fece buon pronoftico, € non dubito, che havera dato nel segno I
Lalli nella sua Franceide C. 4, stan. 21. Si servi di
E le cartacce lor servono al fine
Per avvolger U acciughe e le Tonine,

STANZA XXVIII.
Bovvi anch' un libro ds fegreti, il quale
Gioua a chi legge, e insegna di bei tratti
Ed infra' altre a far che le cicale
Cantsn senza che'l corpo se le gratti,
Ea far ch' i tordi magri con? occhiale
Guardandogli divengan tanto fatti,
Deferive pos moltissimi rimedi
Per chi parisce de i calli de' piedi,

STANZA XXX.

Perchi la donna come altera, e vana
Sopr' agli sfoges ognor pen[a,e vaneggia,
ae cht el?” abi un ceffo di befana
Pomposa,e riceavuol che ogni la veggia;















questa medesima frale. ee

STANZA XXIXi
S? io vi narraffi tutto il
Costui, direfti, ha it cera j

Pur vuo! contarnen' wna folamentt R
Chie vera, ne crediare eb io sarfilh "i
Racconta a! una tal parturientt:
Ch' una carrenea seen faeae 5 &
E ch' una voglia fu, che bawen bavwsy b
Ed io lo crederé senza dispura.
Percio colei bebbe la voplit;
Della grandezza dell' I
eanceeioeadeg robe ik i
Le girelle vorrian, ebe'tfa 1
E è



rts

a

SB SSE CRELESE EGE

SEES SRSA Seth

i

Ma hafti circa i libri quanto ho detto,



OTTAVO CANTARE. 393
“STANZA XXXL

ed qualch' error novoglio far fuggerte,
 Perch'ioche negli Pudi non m'imbrog lio, Che pur eroppin' ho fatti for' al fogiio,
eee altri non ho letto E pot perché fom tanti,¢ tanti i tomi,

Lorfe i fatti lor saper non voglio y Che ne anche fo dir d'unterzo: nom:,
eos il racconto de i libri, che sono nello (caffale,¢ narraado un favoloso

=,  iperbolico parto, fa una leggieri fatira contro al luo delle donne.

 10 sfarfaili. lo aggiunga al vero: Io m' avvantaggi acl racconto. Dalla far.
falla, che gira e s' avvolge or qua, or la, e detto sfarfallare.

 ¥NA vglia fu, Che cosa sia voglia in questo proposito. Vedi sopraC. 2. st. 42.
— ALTIERA,e vana. Altiero, si può dir finonimo di superbo, pigliandosi
spesso ' uno per I altro; se bene a/tiero si dice colui, che per grandezza d' animo
non riguarda,¢ non applica a cose vili, anzi dimostra vers di quelle una cerca
schifezza generola, e senza vizio,¢/uperbo G dice colui, che per vizio, e per

apriccio spropositato disprezza tutti, e tutte te cose indifferentemente, e senza
Thasoee alcuna. Qui, dicendo a/tera intende piena di prefunzione di se stel-
fa, che e lo stesso che /uperbo; e Vana dedita alle vanita, o vanagloriofa, boria.
fa, li Petrarca distingue queste due voci, dicendo nella Caaz, 22,

costs + Ch' in vista vada altiera, e difdegnofa,
Non superba, e ritrofa.

 BEF ANA, Significa Donna malfatta: perché befana diciamo un fantoccio fat-
todicenci, che si suole da alcuni mettere alle tinestre il giorno dell' Epifania, il

jale da Epifania e detto. corrottamente il giorno di Befana. Vedi sorta C. 9,

Sis

I,

TREGG/A. lntende carrozza. Se ben tregeia è un veicolo ruftico senza ruo-
te per uso di portar paglia, e legne, ec. facendolo tirar strasciconi da i buoi.
Servio sopra quel verso di Virg. 1. Georg. Tribulaque, traheaque, © iniquo ponde-
re rafirs dice così. Traha genus vebiculs dittum a trahendo; nam non haber rotas,
edé la nostra Treggia. F:

4L sangue tira, L? inclinazione, o genio le spinge, le forza, Intende che le»

irelle, che le donne hanno in testa, havendo simpatia coal' altre girelle, fanno
Seiderare alle donne quelle della carrozza.

NON m! imbroglio negli fudi. Cioè; non attendo agli spudi; nan ho che fare con,
loro; nom mi intrometto di fiudiare; nan me ne impaccio,

PUR troppi n' ho fasti sul foglio. Per modettia intende; Pur trappi sono gli er-

rori che ho fatti nel comporre la presente Storia.

STANZA XxXU,
Però seguiam con Paride le Dee
A veder cose belle, e Strauaganti;
E prima tronerem di gran miscee,
 Corpi di Mummie,ed ofa di Giganti;
 Her in corpo a pesce due galee,
Tmpietrive com turti i naujganti y
eps 9 li quali esse han per tradizione
Ci

fur fatti del gingerol di Nerone.
Larti del gingguol di Ner Hid

STANZA XXXIIL

Chinfe nel vaso poi vedrem le cotte
C' bebbe quel Vecchio Chioccia di Sileng,
El asta che fu, dicon, di Nembrotte
Con che voile infilzar 2 Arcobateno;
Benché si creda più di Don Chisciotte,
E veramente non puo far di meno,
Perché in vetta nel mezzo della lama
V' è scritto Dulcineach' erafuadama,

2 STAN.


396 MALMANTILE

STANZA XXXIV.
Pende dal palco un fecco gran Serpente,
Che uh al Cocodrilo s' assomiglia,
E dicon che la coda folamenre
Per laliighezza arrina a cingne miglia;
A1a quel che più curioso di niente
E' certo, è una grandissima conchiglia, /
Ouxe fra minuta alga, e poca rena Chi vi dipana fa quant'
Sta congelaro un' uouo di Balena, C” al fin @ ogni gomitol si
La(ciato il raceonto de' libri, torna l'Autore a narrar le cose mai
singolari, che sono in questa Galleria, E perché in tali Gallerie i proc
le fa di riporvi cose flravaganti, ed ancicagliec ragguardevoli, e molte da
ne fingono per accreditare il luogo, € pero 11 nostro Poera mette anche
mano di cose iperboliche, come sono due galec impictrite in corpo 4 |
€ favolose, come un vaso pieno di gotte, ec, Vedi Liaciano nell' fitoria
ove delcrive terre, ed huomini in corpo'a una'balena; B Efiodo, ove
il vaso di Pandora, ove erano tutti i malori, e tutti i malaoni,
AUSCEE. Intendiamo bazzecole, mafieriziuole, ed arnesi vecchi di
prezzo, che habbiano del curioso; metcuglio di bagattelle, di curiosita ¥
AV MME, Vedi sopra C, 6. fan. 52. i
GWVGGIOLO di Nerone, Habbiamo un 'nostro detto, che è: Meron
ginggiole, che serve per esprimere; 4 fortuna mi s' artranerfa; Ml Diaual
disce l'efecuzione del mio pensiero, E viene non da Nerone iimperadore,
contadino chiamato Neri, il quale stava sopra un giuggiolo, osservando
che entravano in casa sua pee rubare, e toftoro accortifi a' esser;
mostrare che gli volevano fare una burla, ¢. non rubare: gli ditiero; 4b WV
\w fei in sul giuggiolo, intendendo: Noi t' havevamo ben veduro. E del lepname
di questo giuggiolo dice, che eran fatte le due alee impietrite incorpo.al pele.

VECCHIO chioccia, Vecchio malandato. ' uno, che sia alquanto infermo de
ciamo chiocciare; dalla chioccia, gallina vecchia,e spelata, che cova i puleitl,
come il malato cova il letto; e !Autore chiama Suleao vecebio chioccia
Icno Pedante, ed Aio di Bacco si faceva portare topra aun' asino, C
mezzo infermo; ed i Gentili dicevano, che egii si trattava in questa forma, pet-
ché essendo egli il maeftro di Bacco, il quale € numerato fra gli Dei el
amici delle comodita, e del piacere, era gitilto, che fudde un' huomo di tuttil
suoi comodi. ar tag

VOLLE infilzar  Arcobaleno, Volle infilzar \Areo-Celefte; che il chia
mavano Iride, e la dicevano insieme co' Greci. Atmbafeiatrice degii x

wee

4B. 5.
: Frinde Colo mifit Saturnia Tino,
Ed il nostro Poeta dicesche Nembrotce vole injfilgar & Arcobitteno
fu quello; che Pe eer si pensd di voler guerreppiar col Cielo ed.
to fabbricd la famofa Torre'di Babel, cioè della confufione. ar
DIN Chifeiorre, Che in nostra lingua voreebbe dire: Di
mile, Fu un Giteadino-delaMuntia, il quate havendo letti molti































pA seop ok FLFR e wn lere2t sere res e-2s

ae












OTTAVO CANTARE 397
valleria, cio? Amadis di Gaula, Palmerino d' Oliva, ec. s' imbriacd, eddinuaghi
dei meftiero di Cavaliere Brrace di tal maniera, che si messe ad immitare le azioni
di detti Cavalieri » facendosi armare con quelle cirimonie, che eran soliti fare
“quei; anch' egli a cercare l'avventure, come graziosamente rac-
conta 26 Michel Ceruates'nel suo D6 Chi(ciotte,il quale fu molto bene tradotco
nostro volgare da Lorenzo Franciofini da Castel Fiorentino, assai benemerito
' a Spagnuola; (1 aggiunta, o secondo libro del qual racconto' voglio-
 no, che sia flato composto da Carlo V. Imperatore ) E perché i Cavalieri Erranti
Ron erano stimati veri Cavalieri, se non havevano l'innamorata, però questo
@ Don Chisciotte si finfe ancor egli la sua, che fu Dulcinea del Tobofo; E da questas
ae il nostro Poeta prova (cherzofamente, che questa Atta fulle più tafo di
| Don Chilciotte, perché nella lama', che era.in cima alla detta asta v' era (eritvo
'Daulcinea, ed intende, che questo ferro era doice, cioè di cattiva tempera.
FN gran Serpente, Questa iperbole del Serpente e posta qui ad immitazionc, o

iat per dir megho, in derisione di coloro, che scrivono le Storie d' Etiopia, che>
wi di -esservi tali Serpenti, che ingoiano un Ceruio, o un Bue intero per volta

 €sono di lunghezza di piii di trenta piedi; E che M. Attilio Regulo nella prima
til > oda ai Cartaginefi ne uccidefle uno in Affrica preflo al fiume Bagra-
it che era lungo 120, piedi.

MANTICE, o mantaco. Vedi sopra C. 1. stan. 55.
si) = SARCOL AIO, Steumento fatto di canne rifefie, o stecche dilegno, sopra il
wai 'wales' adatta la matafla per comodita di dipanarla, o incangarla come s' e det~
wi WifoprarC. 5, stan. 9. E dipanare € raccorre il filo,formandone una palla per co-
shi imetterlo in opera, e tal palla si dice gomitolo dal Latino glomerare, e+
i Soma che il gomirolo, che a Roma ancora si dice glomero.
4) STANZA XXXVI. STANZA XXXVUL
se Van Sfera bellissima si vede, S'.in Grecia fatra fu la criftullina,
nis © (Ch sopr' aun ben tornito piediffallo, E questa di vesciche vien da Troia,
ae 'Che per ginfiezza tutze l' akre eccede, Che a Fiefol fu portara a Catilina
ail on farte di legno, o di metalios ue norte ch' ei ee verso Piftoia,
es pure, e fotrerrifi Archimede Ch' ei non giunfe ne anc! alla mattina,

i Con lla sua, ch'ei fece ai Criftalla, Chet arate wi tio le quoia,
am % Che bisogna guardarla,e /parsi addietra Sicché due Capitan sue camerate
; ie “ “Per'timor di non romper qualche verro, La prefero ye la diedero alle Fate,
2 STANZA XXXVIL STANZA XXXIX,
Che quefia 5 che con ogni diligenza Mentre s ammira così-bel lauoro

Di purgate vesciche fu commilfa,

E vi si fanno fu cento argumenti,

iv Se perdisgraia, o per inavvertenza Paride guarda 5¢ vede una di loro
ae Perquote ocade,ell' e Sempre la fiefa; Canarsi un' occhiolaparrncta,esdenti,
E sel criftalio ha in se larrasparenza, E dargli aun' altra,perch'inturto ilcoro

 LA vescica al Diafano s' appresta, Delle. Naiadi ch' ini fom presenti,
if --Edé\un corpoyche giammai non varia, O fuora (che pur anche son parecchi )
eo E-quel si cangia ognor secondo t' aria, Ha fol quei detrynn'ecchio,e due cernecebi

Se.



STAN-























398 MALMANTILE. ©

STANZA XXXX.
Pero ch' elle son cieche 5 e vecchie tutte y
E loro i denti son di bocca nsctti,
Ma ni per questo ell' appariscon brute,
Ch' ell! hanno volti bel, e coloriti,
E se mangiar non possoncarne, efrutte
Elle s' aintan con de' pambolliti
Perché quei denti,come gli occhi,eiriccs
Non hanno pin virtit, che fom posticci.
STANZA XXXXIL
Così per offernar le lor vicende
lucha ch' io dico se gli caua adcffo, Cedendo ogni ragione y¢ ogni
Già ritornata dalle sue faccende, Perch' inqueff'oraa è
Perch' il portagli pin non le e permeffa, La fronte escape,erife
Descrive una Sfera fatta di ve(ciche di Porco, e mostra, che sia
re di quella di Crifallo, che fece Archimede Siracufano, perché e più f
più sicura. Mentre che Paride stava mirando, e dilcorrendo sopra ilb
della Sfera di vesciche, una delle Ninfe si cavo la Parrucca sun' Occhio, 1
¢ dette il tutto a un' altra, perché così e l' ordine fra loro, Qui pate, che:
alle Lamie, Donne, o Larne per dir m lio, che con carezze allettatrici
stimate da' superftiziofi Gentili mangiarsi i bambini; le quali fea cutre
no un' occhio folo, e quello usavano a viceada hor questa her quella,se
deicrive Angelo Poliziano lib. 3. tit, Lamia, che dice: Lamia h
excmptiles, hoc eft quos fibi eximunt, detrahuntque cum libuit, curl
y» cum libuit refumunt, atque affgunt; alice vero ctiam dentibus utuntur eque
y» exemptilibus, quos noete non aliter reponunt, quam togam, ficut ha CO
>> mam (uam illam dependuiam, & cincinnos, &c. Sed lamia hac quoties do
egreditur oculos fuos fibi afhgit, vagatur per fora ee plateas, &c, domum ye
»» ro cum revenic, in ipfo fatim limine demit illos fbi oculos, abijcitque in le
»» culos; ita femper domi cca, foris oculata,,
PIEDIST ALLO. Bi quceila pictra, che e sotto al dado, sopra il quale pola
colonna: e qui e preso per tutta la bafe, che regge questa sua Sfeca, comet prt:
fo comunemente. aia
VADA, efotterrifi eArchimede. E'o(curata la gloria d' Archimede; Quan?
uno fa un'operazione meglio d' un' altro diciamo al superato; T# ti pwoi ire ari:
porre,oafotterrare. Intendendo; Tu hai perduto cutto il credico, o la flimas
che e quella senza la quale uno è tra gli huomini come morto; i che
che non si dee più far taata Mima della Sfera d' Archimede fatta di cri
ché questa facta di veiciche l'ha tuperata. 2
DATvoia, Non dalla Città di Troia, come pare, che ogi dice
Troia femmina del porco, delle cui vesciche era formata guetta sfera
V1 tira e quia. Vimori. Vedi sopra C, 4. tt. 20, Qui cocea la con
nione, che Catilina famofo capo di congiura descritco da Salu(tio mori
floia «
Vi fanno cento argumenti. Cio' discorrono assai sopra questa Sfera. —

xy
















OTTAVO CANTARE. 399

Ml PARRVCO-A, Voce straniera fattz nostrale, e vuol dire Zazzera, dichioma
®@ finta, che diciamo: Zazzera posticcia dal Francele. Perroxque, chioma. Potreb-
» be forse dirsi in Latino capidamentum.
- CERNECCHI, Capelli pendenti alla testa; qui intende quella parrucca,o ca-
peli postice:; se ben cernecebi Gi dicono quei foli capelli, che pendono dalle tem-
* pie agli orecchi con altro nome dette faceagore, che i Latini, secondo i) Polizia-
no nel luogo sopra citato dicevano cincinnos, E noi diciamo cincinns quei ciondo-
 lidi pelo, che sogliono haver i capretti, ed i Becchi foro la gola, i quali hanno
st qualche similitudine con questi capelli, che noi chiamiamo cernecchi.
PAN bollite, Pappa fatta di pane, bollito in acqua.
 MASC-ALCIA, Magagna; Difetto; mancamento. E' lo stesso, che guida-
" Ielco, ma questo si dice folo nelle bettie, e mascalcia, che farebbe veramente so-
Todelle bestic, ' usGiamo anche per gli huomuni, e talvolta per i materiali, Vie
'un' antico libro Toscano intitolato Libro d+ e#a/calcra, che @ dell' arte del ma-
nescalco, de re veterinaria,
DA quella via 01a quella via, Subito. Senza metter tempo in mezzo. Latino

=

SEE

ill extemplo, e veffigio. Se bene si potrebbe intendere ancora per In quella maniera;

ind in quella guisa, come ¢inteso sopraC. 7.0.84. 0

na oN ogni regreffo. Cede ogni azione; ogni autorita. Vedi sopra C. 7.st.104.

git. RIFERRAR (a bocca, Intende rumettere i denti. Bocca sferrata si dice a uno,

em =the habbia meno i denti dinanzi dal ferrare le beltic, e rimeteer loro i chiodi a'

il pied, si sono sferrate.

mw STANZA XXXXIIL STANZA XXXXIV.

0 Pitna di cibi intanco nna credenza Credilo a me ch' eglié del eloriofo

pal Wit pari pari aperta spalancata, Pero qua dentroyvia,distendi il braccio,

8 OB fatta da vicin la rinerenza Che trouerai del buono, e del gustofo,

ye Parole pronunzio di quespa daca: Se tu voleffi ben del Castagnaccio,

pi ° Caualier, se tu voi far penitenza, Paride fece un po del vergognofo,

on 'Ein parte a noi piacere,e cosa grata Hla nel veder le bombole nei ghiaccio,
440 munizion da caricar la canna, Mando presto dabanda la vergogna,

gl E poi da bere un vin ch'é una manna. E fece come i Ciechi da Bologna,

oF STANZA XXXXV.

Levategli poi'via la calamita Sicch? in quanto ad bauer taglioo ferita
ost Di quel buon vino,e maffime del bianco Jn altra parte era sicuro,¢ franco,
ipo Gli fararon le Dee tutta /a vita Poi dangli un brando con la sua cintura,
gil  WDalla baferra in fror del laro manco, E del trattarlo  intavolatura,
ye Mentre stavano guardando le suddette galanterie,comparue und credenza aper-
i ta piena di roba da mangiare, e da bere, ed inuiid Paride a soddisfarsi; egli dopo

haver fatto alquanto lo Riiakian'> mangid, e bevve; Terminato il mangiare se

46  'Ninfe lo fatarono,rendendogli impenctrabile tutta la persona, eccetto che la ha-
'fetta mancina + Qui il Poeta immuta l'Autore, che favoleggia Orlando impene-
o@  trabile in tutta la persona, eccetto che nelle piante de' piedi.
@ | CREDENZA. Così chiamiamo un' armadio, entro al quale si ripongono, ¢
tonferuano gli arnesi, ed avanzi della mensa; il quale armario si dice ancora,
4 tredenziera  perché quei bicchicri vaii, e baciji d' argento, ec. che si mettono al-
Ic


400 MALMAN TILES ©
le tavole de' Grandi per servizio; 0. per apparato della
diti tutti insieme, si dicono credenza, e i si rij
vriano riporre in detto armadio, che però lo chiaa
tino eroacns re

SPALANC AT A. Affatto aperta. Vedi sopra'C. gf. 38, Pal
cato diciamoa la chindenda\, o riparo fatto con è pali, a un >
vuol dir Senza palanca, e per conleguenza totalmente aperto, ©
tegno, o inipedimento. ei è

PAROLE di quefia data, Parole (mili a\queste, o di
ta, la quale si attende moltissimo nel gioco delle'carte', per ef
chiate; Onde si dice: Ha farrauna buona,o una cattiva data,

SE tu vuoi far penitenza, Se tw vuoi mangiare. Termine usato per:
inuitar' uno a desinare, o cenar connoi, guafi ditiamo:venite a digi;
ché la nofira mensa e povera, e (eatlardi cibi. Sidicc.amcoralfar carisa 5,00

s'é visto sopra C. 5. st. 68, otis verb 9!
HO munirione per caricar la canna. Ho roba da mangiare\, eda
care la canna della gola., e non quella dell'archibuso <2 5, Gate
VN vin, ch' una manna, Vino isquisitissimo, che tale si legge fulle!
che mando Dio nel deferto al Popolo cletta., Vedi ferro Cip.st58..Ma
stranicra, ma fatta nofirale, che significa una brina:condenfata: tenera'y'
detta cos} dali' Ebraico, Azanbm; ioe Quid of bee: come: si dice nell?
16. Poiché maravigliati gli Ebrei di questo nuovo, e faporolo cibo 5
uno all' altro; Che e ciò, che no? mangiamo ? Dalquesta dolcezza viene '
nostro detto. 1 Latini dicevano in questo propolito dowis Velar..
EGLI¢ det gloriofo. I Battilani chiamano vino gloriofo il vino pga 4
rofo, e buonissimo, e dicono groliefe in vece di gloriofe; cio' valor lo
vaalle stelle. In certe Profe Toscane antiche, delle quali alcune si ritrovanom
nuicritte nella Libreria di S, Lorenzo date fuora dal Doni, vi e una leteera amt:
rofa, nella quale e accennato Amore con dire: Quel gloriofe; titolo dato

da' nofiri Battilani al vino; e veramente Amore non imbriaca meno di

si faccia il vino il pith gloriofo, a
VIA. Questo termine serve per follecitare, © incitare uno. Latino Eia ae,
CAST AGNI AC C/0.Pane fatto di farina di Castagne: qui vuol

per opera d' incanti quella credenza dava tutto quello, che uno fapeva

itis g

tern gab




rare.: a
FECE il vergenofo. Finfe dinon si ardire a mangiare. Mostrava vergogaitt
d' accettar l'inuito, che gli faceva quella credenaa. ig
BOMBOLE. Vali di vetro, i quali servono per mettere il vino in
ghiaccio, o neve, detti così ( secondo alcuni ) dai suono, che fanno-nel |
fuori il vino, che par che suoni bombof. Al Rotenano' vuole, che i Latini
da tal suono le diceflero amphore bilbina; ma può anch' essere, che agi ie
così da bombo voce.puerile 5 che vuol dir bevanda, decta così dal fueno, —
COME i Ciechi da Bolugna, Si da loro un toldo, perché cominci

c bisogna poi dargliene duc, perché si chetino. Ci iciue per espri:














i
mt

SBER ote

=
=

Ask RESS Pa SeETLE



T~s-e

>

















fs

rad
ike

i
wb 1

-

4



4
4
of

4



a 4
 Pigliual
| Ainqud franortese thcome,e ilquddo,e il doue,




Escciam

Ciechi da Bi




re.
~ STANZA XXXXVl'
'Eperche if tempo ormai era rrascorso y
jarlo dowean di quini altroue
Prima in sua lode fatto un bel discorso,

( dissero ) quanto t° e occorso

Py ae tutto per appunto,

Ot è gus ra nofira giunto,

oc SPAN ZA KLE X Vil.

Akcibsuvada incontro a un'aunentura

< d prod' un pover huomo questa notte;
Quelle è un tal cognominato il Tura,
CH in Parion confiaua le pilotte.

0 Eta tebellenze un mofiro di natura,

Sicebe tutse /e donne n' eran corte 5

Elafeiando i roccberti, ed i cannelli

2 Per-lui chee cht e facevano a capelli,

“STANZA XXXXVIIL

Non ch? eine deffe loro accafione,

— Come qualcbe narcify inesbertato,
C'una caffia, che e' vegga a un verone

?0/ «far lo spafimato;

' Anis wn diqueie'al Addo ta 4pigione,

«A bioscio nel vestire, e feiammannato,
©' addosso i panni ognor tutti mincfira

0) Tirati gli parean dalla finefira,

“OTTAVO CANTARE,

maghana a Adarte,al Soleje aGiove,

gor

 ptegare a far'una tal.cosa mostrando non voler farla, e bisogna
che refti di farla, Orazio.:
» Omnibus hoc vitinum est cantoribus, inter amicos
0) Wt munquam inducant animum cantare rogati,
drinffi numquam defiftant,
Sid > da Ferrara, o da Milano, 1 Latini in questo proposito
'ditlero Arabicus Tibicen. Qui intende, che Paride si fece pregare a mangiare,'¢
se e poi non si trovava 11 modo, che egli restafie.

LAMIT A,B! \a pictra Adagnes, la quale ha proprieta d' attrarre il ferro,
punto ha il vino di tirare a se Paride, ed è fra etio, ed' il yino la stetla,
»cheé fra jacalamita, e il ferro. Vedi sopra C, 5. st. 59. B sotto ing
— questo C, tt. 66,
| | Di trattario ? intavolatura', L? instruzione di come si debba adoprar quella spa-
t+ Intavolatura e scritwura, che per yia di note, e di numeci regola la mano del

STANZA IL,
Ed esseeran capone; ma chiarite;
el fin lasciando quel fuocnor di fmalto,
Fecer come la Volpe a quella vite
C* hauea si bell' Una ye tanto ad alto,
Che dopo mille prone, anzé infinite
Arrinar non potendoni col falto,
Glié mé,dilfe,chio cerchi altra pastura,
Che quespa da ogni mo non è matura,
STANZA L,
Così non Ia faldo era Martinarza,
La qual non vitrouado anch'elja attacco,
Poicht gran tempo andata ne fu parra
Hanedo if rerzo,e il quarto,e ognuno firac
Codurreun giorno fecelo alla mazza, (co
E per via d' un che le teneua il facco
Aunezro a tofar pecore, ed agnelli,
Mentr' ci dormina gli taglios capelli,
NZA LI.

Quei capelii c un rempo hanea chiamati

Del suo falcto mortal funi, e risorte,
Le bioae chiome wh Dio,queicriniaurati
Che ricoprivan rante piazze morte,
Onde feoperti furo s trincerati

Onc il nimico si facea si forte;

Perché (per quanto un Autore accenna)
Lo rimondavon fino alla cotenna,

ale fate dopo haver lodato Paride per bravo, per bello, e per valoro(o gli dif-
fero sche IL havevan fatto capitar quivi, perché egli andatie a liberar il Tora,
quale loda ironicamente, e dice, che tutte le donne erano innamorate di iui; ma



Ece

accor-










qo2 MALMANTILE (9

accortefi, chenon corri leva a niffuna, lo lasciarono, ¢:

egli non volle mai corrisponderle, haveva fattagli la malia
ottave seguenti.

1 DE
AVVENTVRA. 1 Romanzatori Spagauoli in quei loro Amadis di Gaul
Palmerini d' Oliva chiamavano ayventure ( awenturas) quegli i
ne i quali s'imbattevano i Cavalieri Ecranti, e però il nostro:
creato il Cavalier di Quoio,vuol, che ancor' egli sia fimato i

che vada a provare l'avventura di liberare il Tura dail' incantefimo. 1
similmente dissero aduentures, Bi nostri Toscani ancora, sentendosi:
termine cavalleresco, chiamarono gli accidenti, che accadevano

davan loro materia di fare prodezze dawenture. L' Alamanni nel
principio.




3 teal

Narrero di Girone l'alte auwenture. a ad

E da ciò il Boce, Tels. lib. 5. disse: 5 aabiitly
eettersi in aunentura:. ee

Ma non li parne via ben ben sicura. poe

Pero non se ne mife in auuentira, sung

4L Tura, Costui era un pover' -huomo, che gonfiava le pillotte in] }
che in Firenze è la strada, dove si giuoca alla pillotea detta così da
perché in efla anticamente haveano le botteghe coloro, che lavoravano:
mi, o pure ( il che forse e più verifimile ) quasi Ripar Regio Ripe Roine
tale strada sbocca sul Pafleggio di Lung' Arno; [n Roma ancora vi e la
da di Parione detta similmente così detta quali Rione a Ripa, Regio Riper
pure è così chiamata, quasi Parte di Rione; Pars regionis, come mi vien riferito
leggersi in alcune Carte, o Contratti. E perché veramente costui era brut
di faccia, ed haveva la zazzera avviluppata', elorda, lo chiama moffro
va in bellegza, ed intende Deforme, se ben par, che voglia dire, di belleam
sopranaturali. aise
PILLOTT A, Specie di palla da giuocare, Vedi sopra C. 6. st. 34.
WV' ERAN corte. Erano abbruciate dal fuoco d' amore per lui Virg, Mir
infelix Dido: dice briache del suo amore, cs' intende innamoratissime di Iai. Lt
ebrie amore. Piauto nel eAilit glorifo,o Soldato,al quale da nome di
cio' di Abbartitore di Torri,e di Città; 0, came noi diremmo Taghacantoiy©
Spacca Montagne; fa dirgli da e4rtorrage, cioè in nostra lingue Sparapane Pata
to, suo adulatore; che tutte le donne sono di lui fieramente innamorate » Le
tibi ego dicam; quod omnes mortales (ciunt; Pyrgapolinicem te unum in tere rn
Virtute 5 & forma, & fattis muittiffimus? Amant te omnes mulieres,neque herce i
ria, Qui fis tam puicher. Ed egli sprezzatore altero di tali amori iange !
Jamente la sua disgrazia, beccandosi fu queste lodi; dell esser hwoWd, |
da fare innamorare di Jui tutto il Mondo, WVimis est miferia
pimis. '

LASCIANDO i roccherti 5 ed i'cannelli. Lasciando star di lavorare.
prefe tanto forte l'amore; e tanto le teneva fisse nell' amorofo penk
non potevano più atcendere a' loro usati lavori. Quando Didone si
'rata d' Enca., non tirava innanzi gli edifizj.,.¢ le fabbriche della faa













preaereseei ns




OTTAVO CANTARE:

Virgilio ebbe a dire: pexdemt opera interrupta, minaque Adiroxam ingentes) come
che era occupata da più possente pensiero. Co! presente detto di lasciare
droccherti, 4 canneilé, s' intende questo, perché le donne a' infima plebs (che ta-

1 epeweneenye > che erano l'innamorate di costui ) per lo più non hanno
lavoro

» che ? incannace, e teffere, a' quali lavori s adoprano i Rocchetti (che

son legnetti tondi forati per lungo, e servono per ragunarvi sopra la feta, ed

',

oo LU BRE

AaeTEEsS

TS A A Se ee, ae

altro filo: ed i Cauneii, che sono pezzuoli di canna tagliata fra un nodo, e
Paltro, dai Latini però detti ixternodia, e servono per lo medesimo effetto d' a-
dunarvi sopra la feta, ec. per adattaria a telsere; I che si dice incannare.
hone éch'é, Ad oraad ora; Di momento in momento. Vedi sopra Can. 3.

 PACEVANO a i cape, Si perquotevano, S'azzuffavano. Quando due donne
combattono tra di loro diciamo fare «4 capelli; perché il lor perquotersi, e per lo
 più) pigliarsi 1 una I altra per i capelli.
 CVFFIA. Berretta a foggia di facchetto, entro alla quale le donne & ferrano
icapelli in testa, e quando noi diciamo nel modo, che e detto ne! presente luogo
tna ¢xfia, un ciapperone, e simili arnesi usati dalle donne, intendiamo una Don-
na, Così dal portare lancia, o barbuta; i toldati medesimi si chiamavano Lance,
¢ Barbute, come ficava da Matteo Villani, 11. 81. e Erodoto volendo dire, che
pera si ritrovavano avere in. piedi ottomila soldati, che portavano rotel.
90 brocchiere; disse ottaci/chilian aspida, cioè scudi militari,o rotelle ottomila.
VEKONE. Latino moenianum, podium, pergula, e in Greco secondo alcuni Pe-
ribolas da peribaliern abbracciare, circondare, che i Francesi dicono enxironner.
Propriamente vuol dire andito, o terrazzo scoperto': Qui credo, che habbia a dir
Baleone,¢ non Verone. Verone & detto quasi girone, cioè giro, dall' andarvi sopra
¢rigirare, eAndito, che e lo stesso par facto da e4ndare. Latino ambulatio,
EGLiea Pigione al mondo. Così diciamo d' un' huomo spensierato, sciatto,sen-
2a considerazione, e che vive a caso, che si dice anche Auomo a bioscio, sciaman:
nato ( cioè male ammannato,male all' ordine ) e che i panu# gli paiono tirati addofsa
finefira,B cd questi quattro modi di dire |'Autore delcrive l'attilatezza del Tu-
ra;del.refto, parlando secondo moralita, ognuno dovrebbe flare in questo mon-
do, come a pigione; perché la nostra propria casa è nel Cielo. E nel Salmo 118,
4ncola ego fum in terra, i) Greco dice Parcecos, e alcuni Salteri dicevano, come ri-
ferisce S. Agostino sopra i Salmi, inquilinus, cioè pigionale. F
CAPONE, Oftinato Latino. Pertinax. Pertiwax.
FAR come (a voipe alla Vite. La Volpe dopo haver molto faltato, e dopo essersi
i per arrivare un grappolo d' vua, e non I havendo potuto arci-
vare disse: La voglio la(ciare flare, perché ad ogni modo ella non e matura..
Può aver data occasione a questa novelletca quella d' Efopo, della Volpe, € del
Pruno; in cui la Volpe, che voleva falire una fiepe, mi fuppongo, per mangiar
Tuva, della quale è ghiottidiima, pensando di troyare il Pruno buon' amico, «.-
sto ingannata del suo pensiero; poiché attaccandovili refto intaccata, e ! appog-
gio le fu ferita, e volendola poi disputar con lui, ebbe il-torto: E questo detio
¢i serve per esprimere uno, che habbia usata ogni poslibil diligenza per confegui-
Te una tal cosa,¢ non ' havendo potuta ottenere, 9 habbia abbandonata i' im-
Ece 2 prefa

ae
F
7




404 MALMANTILE ©

prefa come imposiibile, o sia quella tal cosa fata data a un' al
vanti di non l'haver voluta, perché:non era buona, o non!
diciamo: farsi honore d una cosa n
COSÌ' non fa falde eartinazza y Cos\ non fini, o termind ?
nazza la quale non troxando attaeco, cioè non trovando luogo di f
suo amore ver(o il Tura, del quale andò paz ca,cio' sterte innamoratiti
CONDFRRE uno alla mayza, Tradir' uno: Condarre uno con inga
finghe in mano de' suoi nimici, o della giuftizia, o in qualche altro
come si suol dire; al macedo, Latino /n infidias ducere. re
TENER il facco, Tener di mano. Aiutare a cometter un delitto. Ha
un proverbio sentenziofo, che dice: Tanto ne va a chi ruba, quanto a
Jucco, che esprime Agentes, & confentientes pari pena puniuntur. B diciamo anche
Tenersi il facco l'un t altro; che esprime il detto di Teren. Tradere operas mutnas
FEY NI, e ritorte del suo fascio mortale. Metafora amorofa + Si ne
ritorte tengono unite pill legne in un falcio, o faftello, così i capelli del Tura,
quasi funi, € ritorte tengono unita col corpo |'anima, cioè tengono in
Amanti del medesimo Tura, E riorre dicemmo, che cosa sieno sopra C.
PLAZZE morte, Si dicono i luoght vacanti de i soldati; per esempio
no € pagato per cento soldati, e non ne ha f€ non novanta; quei dieci
cento, che mancano si dicono piazze morte. Ma qui intende quelle pi
la(ciano le margini, o cicatrici de i mali, che vengono nel capo; sopr”
li non natcono capelli. “|
1 TRINCIERAT 1, V luoghi, dove erano le trinciere. Intende,
gliargli i capelli si sono scoperti quei Inoghi, i quali con- elle: margini
una campagna piena di trincicre. done tl nimico si facena forte s clot di 0
devano i pidocchi. we tage
TRINCIERA, oT rincea, EB' un' alzamento di terreno: condotto a foggia d
baltione, nel ricinto del quale dimorano i soldati per difenderti dallartiglierigg®
de i nimici. Franzefe rrenchee, cioè tagtiata, yene
LO rimondaron fino alla cotenna, Gili tagliarono i capelli fino rafente la pelle.
Rimondare vuol dir Tagliare a un' albero i rami: B curenna's invende folo lap
le del porco, ma quando si tratta del capo s' intende anche quella dell huom
Vedi sopra C. 5. st, 52. * 4a?
STANZA LIL STANZA LALO
E cos) Aartinaz2a hebbe il suo fine E questo Lupo raggirar si vede 4 2
Volendo vendicarsi per tal via y Intorno a un montnofo cafameme—
Pero sche buona parte di quel crine,
Ch' aleun non few avvedde, leppo vidy
E fabbriconne al Tura le rove
Con una potentissima malia,
Che revifrata in Dite al pratocollo
fa un Lupo rapace trasfor mola,









4

NS ne ears ESR ene

D' una Genteycheymentre mice ilpitl










r

yor bS



Zexv7ek


OTTAVO CANTARE 403,
o» STANZAEIV. 5 STANZA LV;
d vanne,e perché tu non facia Eli la prende con il libro insieme,
— Qualche marronyma vegas arar drittoy Dicendo, che varraffi dell! avviso y
- Acco tal: ero si disfaccis

f Pi disfaccia y E.ched' incano ye dianoli non teme,
- Percht feattado un pel tu baurefti. ifritto, Perché eglirehuom, che fa mostrar ilvifo,



Yi he questo libro qui faccia per faccia Si parte,e per c'al capo andar glipreme,
hl ordine, ei modo si ritrova [critto, da due parti vorrebbe efer divifo;
ipa Portalo teco, e.accio che rm difeerna, Pur vnol servirle, perché si figura
Perch! egli e buio to questa laterna, Che non ci vada gran manifattura,
i Metono: STANZA LVL
wit poi nel suo ceruello, Ricerca nel [uo maftro feartabello
ia Che sa quel luego a bambera s'inuia Di quei pacfi la Geografia y;
| Potrebbe andar a Roma per eMugello, Aa quel(per quato noi potré coprendere)
oPerchvei non si rinnien dow' ei si sia 5 Non si vorria da lui lasciar' intendere,

Hi
id ~~ Martinazza:hebbe il (uo intento, perché prefa buona parte de i capelli del Tu-
sf 4 con essi gli fece una malia, che lo trasformd in lupo, e lo confind in un mon.
ati tevicino:a Maimaatile. Finito questo racconto le Fate licenziaron Paride,¢ gli
r diedero un libro, dove era scritto il modo da tenersi per disfar quell' incanto, ed
we wna lanterna per farsi lume; e Paride si parti con risoluzione di sbrigar questas
we faccenda prima d'andare al Campo.
 LEPPO' via, Portd via di nalcofto. Il verbo /eppare ci serve per esprimere
svelocita 'hell' andar via o nel levar via quaicofa.
4 MALIA. IncanteGimo, fattucchieria, stregoneria. i
1” © PROTOCOLLO. Libro pubblico tenuto da i Notai per scrivervi sopra i con-
oe trattiyeteftamenti 5 e così e inteso da noi; se ben protocode vuol dire libro da re.
| giftrarvifopra, che che sia. 11 Berni fonetto in biafimo d' una mula dice:
4 > — E troppo sia diginna
ie ts srry? Cht il prorecollo memoria non fanne.
Perché veramente Prorocoo.è un libretto,sopra il quale 4 segnano, e regiftrano
r brevementé le cose, per diflefderne poi scrittura pil largamente, ed autentica.
i  Msnce: deteo così quali 5 primo libra incallaro, e legato. Liber ex glutine compattus y
o in guematta referuntur, Ma il noftco Poeta jo piglia nel senso, che oggi usiamo
di libroida Notai, e intende che Martinazza haveva fatto contratto cal Dia-
mM volo di questa malia; il qual contratto era già mefio al libro del Notaio del Dia-
it volo yeper questo detta malia era autenticata, e non si poteva alterare, perché
' era paflaca per mano di Notaio, e regiftraca al suo protocollo.
#  - CASAMENTO montnofo, Intende il Castello di Montelupo, che ogg? quali
# — distrutco siperd più colto Cafotare, che Ca/iello, e lo dice montuofo, perché &
sopra un monte come lo mostra il nome medesimo, E nota, che ancor qui il no-
® fico Poeta vaimitandoi Romazatori Spagnuoli, che fanno parlare oscuramente,
| e come gli Oracoli quei loro:Alchifi, Zirtee, Wrgatide, cc, incantatori.
« MENT-RE move il più sopra alia terra v' e rinuolea drento: Le reliquie di questo
/ — Castelio sono abirate da persone, che fabbricano valellami di terra 5 come pen-
tole, boceali, ec, quali si fabbricano per via d' una ruota, la quale va moffa cot

piedi, e fa efsctto del tornio, e perché in muover desta ruota, € fabbricare il
it valo,



aiies









oh MALMANTILE
vaso, la terra schizza addosso a chi lavora però dice Ademtre shane il più sopra alla

terra v' è rinvolta drento,
FAR' un marrone; Far' un error grandissimo; wx crrorome,
e4¢KAK aritto, Operar giuftamente, Non fare errori. Tolto dal Bifo
ciamo ancora, rigar diritto. 4a) aA
SCATT ANDÒ un pelo. Se tu uscifi punto dell instruzione 5 che tu'
tare, o scoccare, si dice della freccia quando seappa dalla cocca, t
di gui € tolta la metafora, o forse dall'orivolo'a ruote. See
TV hauerefti fritto. \\ Proverbio dice: Come difela Tinca ai Ti
altra aggiunta s' intende: oi habbiam feito, Qui intende tu hanrelti
tu haurelti rovinato questo negozio, EB' lo steilo che: Noi habbiam
ne detto sopra C. 7. st. 60. ow
HVOM che fa mostrar il vifa. Huomo ardito, e che non fee cen
ABAMBERA, Acaflo. Latino Jrconfultd. Vien forse da e è
vuol dir ragazzuolo,(enza giudizio, B il ragazzo in alcunt luoghi chiamato Bar
berottolo. Dicefianche. 4 fanfera. 2 6a
ANDAR 4 Roma per Adugello, Far' una strada al tutto contraria, come &
rebbe andar da Firenze a Roma, e pigliar la flrada per il Mugello, che € dirt
tamente contraria. oa
NON si rinuiene, Cioè non riconosce in che parte ci si sia, e non fa quel chid
si debba fare. aed
MAST RO scartabello. Tntende quel libro, che gli haveano dato naa









è il suo maeftro, e direttore; Questa voce scartabello, e corrotta da C.
anticamente era intesa per un libro di stima; come mostra il Dortissimo 5:
ditissimo sig.\ Francesco Redi nelle annotazioni al suo bellissimo Ditiramboac
18. Gli Spagnuoli chiamano Careape/ una scrittura continuata nel foglio fens
voltarlo, comes' usa negli editti; dal' essere cred' io, non ripiegata, come ifr
gli, ma stefa, come una pelle; o perché Gi distendeflero tali forte di seriteure a0.
3

in carte ordinarie, ma in pelli, ovvero in cartapecore.



” we an ee ae Le Lae eeasS we

STANZALVIL STANZA LVIIL ~~
Fu Paride persona letterata, Ma benche la lettura sia fantafticay
Che già udiato havea più d'un faleero, A un che, si pu dir non fa niente 5
AAa pei, non ne volendo più fonata, E 0? altro di virtit non ha se
Alla squola fiudi di Prete Pero, Che pelle pelle 1 Alfabeto a mentt,
Pera s' ei non ne intende boccicara Tanto la biascia, strologaye rimafice
EB? da scufarlo; e poi, per dire il vero, C' 4 compito leggendo finalmente —
Lettere, ed armi van di rado unite 41 funto apprende,e fra Lalere sue ciate
Per ¢' han di precedenza eterna ite. Ripone il libroye sprona i le foarpe.
. STANZA LIX, otra
Cos} commina, e a quel Castello arriva, Aa perche' non e tempo cb'
Paffa dentro, lo gira,¢ si fiupysce, Quanto col Tura a Paride
Che quixi non si vede anima vina Con buona gratia vofra Fare pasft »

Perc'aquell ora in casa ognun poltrisce, Per difinir di Piaccanseo la catfte




» Jee cold

OTTAVO CANTARE; 407
STANZA LX.



Che da.quei tristi, com! io diffi ananti Di poi gli feeffi fel cacciaro innanzi
| ( Fatto mentre pappana affegnamento Ginfto come un Villano in fu il giumeto,
 Diinfaccarsi per lor -quei boccon fanti ) Pungolandolo, come un' animale
 Tocco de è pie nell' Crsenal del vento; Fin, che lo spinfer dove e il Generale,

. Deferive le qualita di Paride, e dice, che egli era letterato, perché havea let-

@ to pidd' un faltero,, che e quel libricciuolo, contenente alcuni Salmi; che si da a



Aeggere a' ragazzi quand' hanno imparato a cono(cer le lettere dell' Abbicc:; E
Tamuiodke, samedi che vm fapeva troppo leggere; e dice, che non ¢
da far meravigiia di questo, perché l'armi, e le lettere mai furon d' accordo,¢
»però egli, che era armigero, era scufabile, se non era letterato; con tutto ciò
'compitando leffe in quel libro, ed intese quel ch' ei doveva fare; ed arrivato al
-Cafamento montuofo trovd che ogauno dormiva. E quil' Autore lascia il parlac
i lui, e torna a parlar di Piaccianteo, che la(ciò sopra nel fine del Canto 5. e>
dice, che a-furia di calci e pungolate fu da coloro condotto doy' era il Generale,
© NON ne volendo sapere pin suonata. Non volendo pii: sentirne discorrere di tare
tuna tal cosa, e qui intende non volendo più studiare.
LA squola ds Pretepero, \nsegnava dimenticare.
© NON intende boccicata, Non ne intende punto. Non conosce a pena le lette-
“res perché boccicata stimo che venga da abbiccs, ae dica non fa lt Abbicci, che
pt oma, che con i Greci ancor noi diciamo diphabero, el' usa il nostro Poeta,
sente Oxtava 58. Procopio nelia Storia segreta narrando l'ignoranza di
- Giuttino imperadore, che poi si adottd Giuftiniano; dice che egli era Analfubeto,
»tioé sche non fapeva I abbicci; ac scrivere il suo nome.
» PELLE pete. Superficialmente. &' lo stesso che baccia buccia detto sopra C.

se stans27, b 4
© BIASCIARE. Mafticare senza denti; cioè con la lingua,¢ col palato. Qui
'lntende quello studiare, che fanno i fanciulli, quando imparano a leggere, che

“prima di rilevare, o profferic la parola, che leggono, la compitano [utte voces,
'con la bocca il medesimo gefto, che fa uno, che biascia; e lo steilo vuoi

dire quel rimaftica, ec, e firoluga intend: cerca d indovinare quel che dica queila

scrittura.

. + Leggere a compito, e quello accoppiar le lettere, e sillabe, che

fannoi fanciuili, quando commciano a unparare a leggere, il che si dice compi-

tare. cio contare a una a una Ie lettere, per poi fommarle, per così dire, in

“ana parola; il che si dice, rileware,

~ CLARPE, Bazzecole, Vedi sopra C. 3. stan. 5.

SPRONAR te fearpe. Detto afato per burlar' uno, che viaggi a picdi

ANIMA vina. Ancor sopra C. 6. stan. 19, si serve di questo detvo assai usate
“anol, se ben si fa che l'anima sempre vive;e qui vuol dire, che tutti dormivano.

POLT RIRE, Dormire. Vien da Poltro, che vuol dir letto; circa che vedi
sotto C, 9. itan. 39.

FA-CIAM panfa, Riposiamoci: o fermiamoci. Frafe Latina venuta dal
Greco, usata anco da noi, i quali da Paufa abbiamo fatto Poa, eda Pan/are
'usato pure da' Latini de' tempi bali, 'y/are~

; ae.
















408 MALMAONTILE TO
ARSENAL del vento. Ripostiglio deViyento § cioè il
dire una stanza, entro-alla quale &i fabbricanoi Dante laf. <
unle neil Arzana de' Keneviani, » 6 6) es
Mu hoggi si dice:ax/enale, e credo che sia parola corrott. 3
@rx manalis, ia quale origine viene approvata-dal Ferrari.)
PYNGOLARE; Stimolare. Pangolo ¢-quel-bastone con tuna' punta a
@ acciaio in'cima j\del quale si erupnoi congadihi per re
camminino; Lat. fimulus 7B questo si dicepangalare',: caiy iss%
STANZA LXA vogcl now pS STPPARNg a
etppunto rl Generale a faris' eposto. * 'Cofteroal fine fegli

a




















Alle minchiace,¢d e cof ridicola. la) Rersdingli del prigion
Jl vederlo ingrugnato, e mal disposo. Ma e¢ possom predicar
Perchigli e fata morta una verzicila, ©. Perch'eglich'e' mele
Le carte ha dato mal, non ha risposto, ©. Oo “Eiperde una gram

E poi-dt non-contare anco pericola E gliene duole} e now
Sendoscopertobaner di più nna carta yi.. Lornonddrerta,e:

Perch dtrado, quando rubay foatee, 00° Pletofamente fa qucfte si
-Cofloro, che conducevano Piaccianted parrivarono al Ge eye
va giuocando alle Minchiate, ma percheegli-+haveva fatto-unaln a
perdeva, e però rain colicra, in-vece d afeoitare quel che essi dice
se a dolersi della Fortuna -; come sentiremo appresso, D i nonedyy

MINCHIATE,£ un giuoco assai noto detto anche 7%
Germini » Ma perché è poco usato fuori della nostra Toscana,o d
mente da quel che uGiamoi noi, per intelligenza delle presente Ottava è
necessario sapersi, che il giuco delle Minchiate si fa nella maniera che
E' composto questo giuoco di novantafette carte, delle quali 56. d
racce, € 40. si dicond'Tarocchi', ed una, che si dice i/ matto: 1é carte
in quactro specie, che si dicono femi, che in quattordici sono efigiati De
da Galeonto Marzio diconGi essere pani antichi contadineschi ) 10 1.4: Coppe
14. Spade, ed in 14. Bastoni, ed in cia(Cuna carta di que(ti fea cominciad
(che si dice afo ) fino a dieci','¢:nell' undecima e figuraro un Bante,
Cavallo, nelia 13. una Regina, € nella 14. un Re, e pucte queste carte
fuor che i Re si dicon cartacce, Le 40. si dicono Germmis o Tarocabiy ©
voce Tarocchi vuole il Monosino che venga dal Greco: Etaroohi; quah
egli con ' Alciato, demotunrur fodales s1li'y gus cibi causarad dufum-com
quella voce non fo »che sia; fo bene, che Afereroi, e Hetaroiwuol dire
da questa voce diminuita all' usanza: latina si può-etiere fatto 4
Compagnont, Germini forse da Gemini segno celette, che ¢-fra Ta
è il maggiore. In queste carte di Tarocchi (onw efigiati diverti
Segni celetti, e ciascuna ha il suo numero da uno fino a 35,,¢0"
no a 40. non hanno numero, ma. si distingue dalla figuea am
maggioranza, che è in questo ordine Strela, Luna, Sole, Afonds eT)
éla idre, e farebbe il numero 4o. L' allegoria ¢,iche siccome le)
vinte di juce dalla Luna,¢ la Luna dal Sole, così il Mondo e maggi
ela Fama figurata colle Trombe, vale più che ii Moado;

ale
a














REF FSSRSETFESTLRR SLE EESS EE





SRE

“PRSPEREVF eRe SBBEZE



OTTAVO CANTARE; 409

'I huomo n' è ulcito, vive in esso per fama, quando ha fatte azioni glo-
. li Petrarca similmente ne Trionfi te come un giuoco, perché Amore € fu-
daila Castita, la Castita dalla Morte, la Morte dalla aw > e la Famas
| Diviniva, la quale eternamente regna. Non e numerata ne anche la carta
ma vi è impressa la figura d' un AZarro, e questa si confa con ogni carta,e»
ogai auacco, ed e superata, da ogni carta, ma non muor mal, cioè non
i umat nel monte dell' avversario, il quale riceve ia cambio dei detto Maco
altra cartaccia da quello, che detce il aa ¢, ( alla fine del giusco questv,
'dette il Matto, non ha mai preso carte all' avversario, conuiene che gii dia
(0, On havendo altra carta da dare im sua vece, e questo è il caso, nel
si perde ii matto; Di tali Tarocchi altri si chiamano mobili perché conjano
chi gli ha tn mano vince quei punti, che etli vagliono ) altri ignobili, per=
snon'contand. Nobili ono Ve, due, tre, quattro,e cingue, che la Cartas

%
del Fao conta cingue,¢ l'altre quattro contano tre per ciascuaa. Ui numero 10.
:











43.20. € 28. tino ai 35. inclufive contano cinque per ciascuaa, e l'ultime cingue
| Guutanio dieci per ciascuna,e si chiamano sd, [i Matto conta cinque, ed ogni
Re conta cinqute, e ono aacor' eifi fra ie carte nobili, s1 numero 29 non contas
se non quando è in verzicola, che allora conta cingue, ed una voita meno delle»
'compagae felpetcivamente; Delic dette carte nobyli si formano le Verzicole, che
Ordini, e (egucnze almeno di fre carte uguuli, come tre Re, o quattro Re;
: di tre carte andanti, come xe, due, © tre, quattro, e cingue, o compote, co-
i bet 'hie wo y 13.¢28. Vo, matto, e quaranta, che sono le Trombe, Dieci 20. € 30.
tq OVEFO 20 30.¢ go. E queite verzicole vanno mostrate prima, che si cominci il
giuoco,¢ meife ia tavola, il che si dice acenfare la Verzicola, Con tutte le verzi-
i 'evi si confa il matto, e conta doppiamente, o triplicatamente come fanno l'al-
a “tre, che sono in verzicola,la quale efilte senza matto, € non fa mai yerzicola se
i 'non nell' une, matto, e trombe, Di queite carte di verzicola si conta il numero
“che vagliono, tre volte, quando pero l'avversario non ve la guafti ammazando-
a “Vene wna Carta, o pir, con carte superiori, che in questo caso quelie, che resta-
f 'DO; coMtano due voite, se però non restano in (eguenza di tre, per esempio: Io
a. “'moltrO 4 principio del giuoco 32. 33. 34.¢ 35. se mi mi muore il 33.0 il 34. cho
os rompone la seguenza di cre,la verzicola e guaflata, e quelle, che vi restano con-
tano foiametice due volte per una, ma se mi muore il 32. o il 35. vi resta la se.
Suchza di tre,€ per confeguenza e verzicola, e contano il lor valore tre volte»
ciafeheduna. Mf Atato, come s'é detto, non fa seguenza, ma couta sempre
è 1i (uO valore due voice, o tre secondo, che conta la verzicola 9 guasta, o falua-
a ta} © quando s' ha pill d' una verzicola, con tutte va i Adarto, ma una fol volta
i) conta tre ed il refto conta due; € questo s' intende delle verzicole accufate,/es
mottrace, prima, che si conunci 1! giuoco, perché quelle fatte gon Ie carte am-
" mazzate agh avversarj, come farebbe; se hayendo io il 32. ed il 33. ammazzaii
ai' avversario ti) 31, o ii 34. ho fatta la yerzicola, e queita conta duc volte,
Quando e ammazzata alcuna delle carte nobili, ciascuno avversario fegna a co-
'lav, a cui € thata morca canti segni, o punt:, quanti ae yaleva quella tal carca;
Ccvetto però di quelie, che sono state mostracte in verzicola, delle quali, sendo
Auiazeate non si gaa cosa alcuna (. a da quello, che per privil:gio non
3 f giuo-

























4r0 MALMAN TILES

» giuoca ) perché tali segni vengono dagli avyerfarj guadagnati
del valore di efla verzicola, che dovria contar tre volte,
ed il 29, morendo la verzicola, dove esso eatrava, conta folo cing
carte poi, le quali si dicono carte ignobili, e cartacce non contand
mazzano tal volta le aabili, che coa;ano comei tarocchi dal aunero 6,
amwmazzaa tutti i piccin, cioè l'1.2. 3.4. e 5. dal 14. in fu am nazzano
il tredici, e dal 21. in fa ammazzano anche il 20, ed ogai tarocea
Re ) ma servvno per rigirare i giuoco; il qual giuoco appreilo di noi non
non in quattro persone al più, ed allora si danao 21, Carta per ci
do si giuoca in due, o ia tre, (¢ ne danno 25. E giocandosi in quai
primo che seguita dopo quello, che ha mescolace le carte in fala mano
dice bauer (4 mano] ha la faculta di non giuocare, e paga fegai trenta
che nel giuoco pigita ' uitima carta, e questo che piglia l'ultima |
dice far f' ultima ) guadagna a ciascuno ai gueili, che hanno giuocato
Colui, che non giuoca guadagna ancor' egli de i morti, cioè segna,
lore della carta a colut,al quale e amimazzata deta carta. Se
giuoca, il secondo ha la faculta di non giuocare pagando go. segni,se
ca il 3. ha derta faculea pagando go. fegal, se il 3. giuoca paifa la faci
che paga 60, segni come sopra. Ma se il giuoco e solamente tn tre
ci questa faculta di non giodare.

Me/colate che sono le carte, quello de i giocatori, che è a mano
quello, che ha mescolato,n' alza una parte,e se v'é volta nel fondo di quel
del mazzo, che gli resta in mano una delle carte aobili, o un tarocco dal
27. inclufive, 12 piglia, e seguita a pigiiarle fino a che aoa vi trova und
ignobile: Quello, che ha me(colate le carte dopo haverne date a ciale

PSPS PPTs

se stesso dodici la prima girata, e tredici la seconda, e scoperta a %
carta la feuopre anche a se medesimo, e poi guarda quella, che segue,¢! a
se fara carta nobile, o tarocco dal 21,al 27. e seguica a pigliarne come +

gu-sto si dice rubare, e queste carte, che si rubaao,¢ si scuoprono, sendo a
guadagnano a colui,a chi si sCoprono,o che le ruba, tanti segni,
gliono; e coloro, che le rubano € neceffario, che scartino; ciae si levino:
altrettante carce a loro elezione, quaate ne hanao rubate per ridurre le
al numero adeguato a quello de i compagai; e chi non scarta, o per
dente di carte mal contate, si trova da uitimo con pil carte, o con.
avversarj per pena del suo errore non conta i puati, che vagliono le fu
ma se ne va a monte; Colui, che da le carte, se ne da più, o meno d
flabilito, paga 20. puati a ciascuno degli avversarj, e chi se ne trova
pill, e deve scartare quelle, che ha di più; ma non può far vacanza
deve rimanere di quel feme,, che egli scarta; Se ne ha meno, la deve
monte a sua elezione, ma senza vederla per di dentro, cioè chieder la qu
o la fefta, ec. di quelle, che sono nel monte, € quello, che mescoid le
si dice far /e carte ) fattele alzare gli da quella, che ha chiefto,
Cominciafi il giuoco dal mostrar le verzicole, che uno ha in mano
mo dopo quello, che ha mescolate le carte in fu la mano destra,
quna carta, ( il che si dice dare ) quegli altri, che seguono devon dare


4



OTTAVO CANTARE. 4it

mo feme', se ne hanno; € non ne havendo devono dar tarocco, e quello si dice
nes » E dando del medesimo feme si dice ri/pondere. Chi non risponde,
ed | reece, feme, che è stato meffo in tavola, paga un feflanta punti
ciascuno, e lia carta nobile, che haveffe ammazzato; per esempio il

Toes ai di danari, ed il econdo benché habbia denari in mano, da.un



acco sopra il Re, e l ammazza; scoperto di haver in mano denari, rende
acolui dichiera, e paga agli avversarj feflanta punti per ciacuno, come
MY gE deo. Ogni tarccco piglia tutti i emi, e fra lor taroccht il maggior numero





-pigiia il minore, ed i matto non piglia mai', e non è preso, se non nel Calo dec
| todi sopra. Così si seguita dando le carte yeu il primo a dare e quello che piglia
; Tecarte date; ed ognuno si fludia di pigliare all' avversario le carte, che conta-
—-HO€ quando s'¢ finito di dare tutte le carte, che s' hanno in mano ciascuno con-
b “tale carte, che ha prefe', ed havendone di pil delle sue 25. segna a chi l'ha me-
ge MO taati punti, quante sono le carte che ha di più 5 dipoi conta i suoi onort, cod
l'il valore delle carte nobili, e verzicole, che si trova in esse sue carte, e fegoas
all' avversario tanti punti, quanti con li suoi onori conta più di efl> 5 ed ogni
F et si mecte da bai.da un fegao, il quale si chiama wn fefanca, € que'ti
Sfane valutavano secondo il concordato. E tanto mi pare che basti per facili-
tare l'intelligenza delle presenti ottave a chi nonfulle pratico del giwoco delles
Minchiate, che usiamo noi Toscani, che è assai differente da quello 5 che con le
'medesime carte usano quelli dala Liguria, che lo dicono Gsmellini; perch Adin~
¢hiate in quei paefi è parola oscena. Da questo'giuoco vengono molte manicres
dire; come essere il matro fra tarecchi; entrare in tutte le verzicole; Efsere le»
Trombe 5 carracce; Contare; nom contare; e simili.
\ INGRVGNATO. In collera. Chi s' adira, o entra in collera suol mostrarlo
con la Mutazione di volto, torcendo la bocca, o increspando la fronte, com,
atti simili, che si dice anche far mufo, e far grugno, o ingrugnare. Vedi sopra C,
2, Man. 57, Lasca Nov. 10 Ata Beco non la porendo [gozzare fene fRaua ingrugnato
 anki che no, Viceli anche portare tener broncto; imbroncrare,. Nonio Marcelio an~
tico Gramatico. Bronci /unt producto ore, & dentibus prominentibus.
DS AMMAZZ AT A una Verzicola, Ammarzare,rubare,foartare, dar mal les
artenon contare, verzicola, non rispondere y sessanti, ec. lepgi queiche habbiamo
~detto qui sopra alla voce Afinchiare. read
» HVOMO roto. Huomo collerico. Lat. praceps in ira, che si dice ancora in

yb + mn ena huomo precipitolo.

ES

BE

RUEBRERLER SSeS.

,¢ © NON ci puo fear sotto, Non la può foffrire. Lat. /ustinere, pati.
gi | LOR nem da retra', Non bada, o non attende a que! che est dicono. Non da
. Lat. mon faciem accomodat aurem. Dar vetta in altro senso dissero gli
“f antichi nelle cose di guerra, per quello che i Latini dissero, imperum /ustinere,
yy » GAGNOLARE. Rammaricarsi. Vedi sopra C. 4. stan. 9.
n.% STANZA LXIIL
| Che t* ho io fatto mai fortuna ria Lucho non si farebbe anch' in Turchia,
/ Chee? bas con me si grand' inmicizia; Lie proprio un'impierade un'inginftizia;
, Mentre tu mi fai perder tuttauia Vedi, non lo negar che tu? kai meco;
5 Che @ nem mi tocca pureadir:Galizia? E poi fen'. aunedrebbe Nanni cieco,
2 Es £2 STAN.



ee Low


© SSR SS aes eee

a




























412 MALMANTILE

STANZA LXIV,
eMa, se volubil fei quanto /degnofa
Facciam la pace, manda via lo fdegno;
E [e tu fei de' miferi pietofa, a
Danne,col farmi vincer, qualche segno, Si 3} 5 ma basta po
2» Fu il vincer sempre mai lodeual cosa y O Baccellacero
a» Vincafi per fortuna 50 per ingerno y

ee FFE CEES SP rsa ererersEas

Percio de' danni miei refhando faria, Capitale\ Sarthe
La Fortuna mi sia non la Disgraria, Se tu nd voi pil

STANZA
E così finiran tanti [chiamazri
Dichiamar la Fortuna,ei giuochi inginffi, — Ov'io ritrowo ognor exttit
Che mentre vi ti ficchie vit ammarzi Per forza al giuoco mi ric e
T4 [pendi, e paghi il Boiacherifrupi, Appunto, come il ferre-a cal
1i Generale si duole della Fortuna perché gli e contraria, e lo fa
pre: la prega a volersi mutare, ed essergli una volta favorevole: © 0
sto C. 15. stan. 1. dice Px i vincere, ec, Ma poi accorgendoli, che il suo
è inutile, riprende se medesimo, del vizio, che ha di giocare, ma conok
l'ammonizioni non (ono abili a farlo defiftere dal giuocare..
WON mi tocca a dir; Galizia, Non ho punto il conte mio. I
de della Galea disse:
E se non ne facean tanto romore
Non saria lor toccato a dir: Galizia 3
Tanta gente » andaua per amore. ae
Ed il Persiani dolendosi, che un suo fratello era più lefto, epi aflute
disse;
E prima: 1 mio fratello è una ciuftizia y
Che mi riuede moito bene il pelo,
1 credeu' eljer furbo, e giuro al Ciele
Che seco non mi tocca a dir; Galizia, ket
Da quiefto che dice il Persiani può,chi legge,comprendere il vero
sto detto. 3
NON si farebb! ancl' in Turchia, Non si farebbe in luogo veruno 5
foaa del mondo, (¢ ben fulle il maggior nostro nimico, come ¢id Treo
sopra C. 5. stan. 6. i, Suagtod Cane
SEN' avvedrebbe Nanni cieco, Lo conoscerebbe uno, che non havesse
Lo vedrebbe un Cieco,come era Nanni. li Proverbio dice: éome:
cieco, e (enz' altra aggiunta s' intende, vedere 5 perché questo Nanni
va sempre; vedere, Si dice anche semplicemente Vamnicreca., © 8!
defimo. Si dice anche: Le vedrebbe Cimabues vibe ncn ciecos 05\che
>

eed

vt

occhi di panno, detto h » venendo da
tura in Firenze, non perché eghi fuffe cicco » ma & voieva denota
fufle nato al mondo cieco y vive affatto al buio del disegno. 1
THM.

444A che gracchia io? Ma che sto io a ciarlare in vano. @




OTTAVO CANTARE, 43

re della Cornacchia yo del graccio, quasi Lat. graceutare, Ma ci serve per clpri-
un cicalare senza mento 5 senza frutto, oal vento, Vedi sopraC. 1.
staa, 69. C. 4. stan. 25..¢C, 7. Rao. 59. Ser Brunetto Latini nel Patathio; in quel

-weelo: Adi aific 5 #10 non fo.y ch' aurem cornacclie ? volle dire in gergo; alludendo
eal faono della cornacchia; Che auremo per il ores di domani, Lat.cras,

a

 DISDETT 4. Dilgrazia. Maia fortuna. B' ii contrario di Desta, che vuol
 dir buona forcuaa nei Eee, Oinaltro. Sp. defdicha L, malum fatum,mala fors.
 FINCER la posta. Guadagnare quello, che va in giuoco. Vedi sotto in questo

wp ©. stan. 7..¢ vuol dire vincere una volta foia.
— PORRE 4 Caualiere, Rimaner superiore, Caxaliere si chiama quella Torretta,
4 nelic Fortezzeavanza sopra a tutte le muraglic della medcfima fortceza;¢ di

Essere yo fiare a Canalere, vuol dire Efscr superiore, © avanzare il compa-

- gno, Varcit Stor, lib, 9. Zara questa parte delle mura di qua d' Arno non banendo

wile Me monti, ne colli sopraccap!, non puo dal di sopra, (come si dice)a canalicre essere offefa.

rd  BACCELLACC I/O, Scimunito 5 Sciocco; Infeafato. Auguilo Imperadore»
all” diceva bacelus. €

'isp -Lorfe fogna pere, Ognuno Gi figura di goder quel ch' ei vorrebbe, ogauno fo-

: ch'er bramna. Virg. ed. 8. 4% qui amant ipfi fioi somma fingunt. Vedi [0-
 pra C, 2. fla, 7. B per qual caula ti dica / oro, e non altri aaunali, Vedi C. 1.
ca 31. Teocrito ditie; Omnis canis panem fomniat, ec.
| ) @APIT ALE, Questo termine oltr'a i signincati, che dicemmo sopra C. 7.
Mian, 82, protterito nel modo, che e nel presente uogo, ha la forza del Latino
Fiinam © yuoi dire piaccia a Dio, che non sia per essere,¢ che non segua, in
contrario.

y SCHLAMAZZO. Romore, Strepito. Traslato dalle galline, il gridar delle
quali Gi dice ichiumazzare, Ll Vocaboiitta Bologaele dice, che 1 verbo schia~
-Mbazzare significa Kiciamare io darao, dal Verbo Greco Sciamocheo, che vale

} mare cum umbra, Ma e yvanita; perché schiamazzo vien dal Latino exc/smatio,
V1 ficchi evi  ammazzs, ha queito caso son quaGi Sinonimi, € figaificano
— immergerti, o applicacti cutto a una cosa, A "4
bp PAGE boia che +i frajti, Spend per haver danno. Teognide disse: Sibé

Oe sph vineula cudie.

bp LABRICCINO del Paonazei, Lntende carte da giocare, perché già un tale de'

te Paonazzi fabbricava dewte carte.

APPYNTO come ti ferro 4 calamita, Per simpatia, come fa la calamita al fer-

i ro; Beeausito detta da Franzeli simaat, cive Pica amante.

df SA ANGA LXVIL STANZA LXVILL

is E Sard. ver, ch' so habbia a star feggetto Datemi dungue un marzo in sula tefia
. | a una cosa, che mi da tormento? Vedere; eccoms qui ch io non mi muaitay




St

a

Come tormento ? oibo \ s' 10 ci ho dilette.

» St un intanco per lui vine scontenta.

O per fido giuocaccto | e maledetto
Cin e ha trouate,e me, chetifrequente y

Ne voi farete cosa men che bonefta,
Se dal.giocar, morendo, io mi rimond y
Soc! ogni di farebbe questa fefta,

C! altro diletto, che giocar mon proKo y

i Tu non cs bai colpa tu, 4 me il gaftiga Ed a giuocare omai son tanto avenge
a — 2 poicht cou te m' intrig? he'd pentirms non giana £4 Se







414 MALMANTILE ©
STANZA LXIX.
LD usare ogni sapere, ogni mi
Non vale a far mi cotro al gioco,
Imperocch' io t ha fitto si nell' ofa
C? amos mio mal qual afferato inferno,
E forse giochero dentr' alla fofia,
Che forse? diciam pur:tengo per fermo;
E se trouar le carte ini non pufio y
Fari, ( pur chee si eiecbi} all aliofiso. + I quarti anro,oo'far
Seguita il Generale a lamentarsi, e combattendo in lui la voglia
con la ragione, e con la conucuieuza', prega gli amici 5 che Pai
ché vede, che non c'é altro modo; che egli si rimanga di'giocare}
d' esser certo d' havere a giocare anche dopo morte, e che alla fepoltura
dare con le carte da giocare nel feretro nelia maniera,, che esprime
va 70. *
b7z0" « Questa voce ha diversi significati, perché ce ne serviamo
come nel presente luogo: per dimostrazione'di naufea, come oii 5
e quefia ? (orto C, 10, tt. 23. per riprenfione, © difapprovazione: Oibe.
cosa,ed esprime il latino Kab, & espace, E gue) che i Greci distero e4ib
ciamo anche: aibo, eibo, e tbo, 4 Oe
SCONTENTO, Scontclato, disguftaro » La \ettera', sy aggiunta
pio di nomi, verbi, ec, ha nel pariar nofire la forza, che appresso
particelia i» privativa di Circa di che vedi il Varchi nell' Hercola
de alla particella ex.
MAZZO, Quei martellone di legno, che adoprano i. Macellati
la tela a' buoi, donde mazznola queiia, che a Roma adoprano per
i malfattori. Si dice anche mageio, nia questo € propriamenteq
prano i bottai a cerchiar le botti, Dal Latino malleus. 18
FARE schermo contro al gioco, Difendersi, o riposarti dat non gioeare.
dal verbo schermire, che vuoi dire Eiercitarsi per imparare a difenderhi
il qual viene dal Germano be/chirmen, siccome vuole il Voto. Dan,
O Grscopo dicea da Sant' Andrea, La ie
Che t' è viovato ds me fare schermo? Y
i Petr, Son, 17. Ch' i non son forte ad aspettar la luce
Di questa donna, e non fo fare sebermo
Di luoghi tenebrofi,e a' hore tarde ? i rue
L' HO fitto nell ofa. Ho wn desiderio di giocare internatissimo;
giovane innamorato difie, Georg. lib. 3. Quid ivvenis magnum cui
ignem Durus amor? Bil Petrarca.: Dee ai
eee






















E ricercami le midolle yet ofa, © ~ A
AMO il mio mal qual ajerato infermo. Come brama: il febbricitante di bees
che gli e nocivo, così bramo io di giocare, che mi e dannofo, she?
e4ALIOSSO. Come habbiamo detto sopra C, 1. st. 9. tutti li gi i
da i Latini si dicono alea: da che io deduco, che questa voce Aliso
Latino alea, & of, e significhi, come in efictto significa ofo da gu






















pe

Ste

:

s

it
r|





OTTAVOCANTARE. is

i l'aftragalo.de i.Greci., Dicefi ancora Catrioffo; quai. gasdy
uct otio bared et gambe didietro di tutti Pen o
on e nell' agnell



> Pagnello, bue, ec, che negli animali d' ugna sode, come il
ec, © ditate come il Lione, ec. non si trova, eccetto, che nell' Alicor-
o Pol. Virg, lib. 2, cap. 13.. e Dianel Soutero de Aleatoribus lib, primo
+» Buleng. de lud, Veter. c, 58. ed e un' offetto di figura quadrilunga das
concavo,¢ dall' altra conueflo: Nel mezzo del concavo apparisce un
co,, ed il conuetio., che è la parte opposta al concavo, forma: in cia-
luefiancate duc piccoli buchi; nelle teftate del fianco al concavo,¢
flo | due superficie quasi piana, se non-che in una si vede un segno come
»¢ hell' altra un segno come un 8., € queste duc parti quando l'Alioffo si
in tavola sono le più difficili a rimanere scoperte, perché ono di più dificil
del concavo, e del conveffo, e l'altre due fiancate non restano mai (co,





f wee perché niuna per la sua rotondita può posare «<I nostri ragazai dell'infima
 plebe,nel giuocare con quest' offo s' adaitano a quei segni, servcndofene per nu-

 Miero.con fare il concavo il numero 4nd, il conuctfo farina, cioè naila, per effler
qusito 11 pity facile arimanerescoperto, la parte dove e il segno 8. vince otto
tiene la figura di quci numero;>¢ da' Greci quello numero, di otto negli
chiamato Srefichora,s¢ la parte dove è il. segno.S, vinca dodici,) perché
haf 'a quasi di libra, che si divide in 12. parti; o secondo, che conuengono,
ado, o variando questo giuoco, secondo i patti:(B l'usano dettira-
dalla Pafqua di Refurrezione ( nel qual tempo s' ammazzano gli agaclli,
mpe de' quali si trovano.questi offi } fino a che vengono le pelche, ed al-
lato.' Auuofo.,.¢ giuocano ai nocciolisne i modi detti sopra C, 3. st. 37.
qual giuoco durauo.a giuocare, fino a che stiacciati i noccioli vendono l'ani-
me di ef aghi spcaziali., che fara per tutto Octobre in circa  e da questo tempo
fino a Quaiefima givocano alla ruila, o alle buche com la palla di legno nel
Che si difie opra C, 3... 57.; e per tuttaJa Quarefima giocano alla trot-
« E così distribuscono 1 loro trattenimenti per tutto l'anno., Ma tornando
all' Aliofse; appretio agli antichi Romani era usato dagli huomini più fenfati, ed
in diverse maniere; e fra I altre il concavo.era chiamato Cane, o canicula forse
da. fiella lucida, che si yede nella bocca del Gane Celefte;, stella cattiva, ¢
malefica; € colui, che tirando faceva apparire detto lato,, posava in tavola due
denari, o quello.che erauo conuenuti fra loro i giocatorl, ed era cattivo, onde
Pecfio dif. Damnofa Canicula quantum Kaderet y la parte oppolta,a, detta era
ohiamata Venus stella benigna, e benelica; e significava il num, fer Latino Serio,
da noi detto Size, nel giuoco dello Sbaraglino, quasi Seino da' Greci chiamato
 Hexites,¢ chi tirando scopriva questa Venere guadagnava [ei, e tutto ello,
che haveyano polazo in tavola coloro,, che havevano scoperto Cane, o Canico-
la, Giulio Poiluce lub. 9. dice 5 che da ipill, il Sei era chiamato C0,.¢ il Cane,
| Ovverol'Aflo; Chio: e che in questo lor talo non havevano, ne il duc, ne il cia-
gue. Con queiio offo giocavano tanto i Greci, quanto 1 Laci in altre maniere,
~ © fino con fei, e oti offi per volta, ma a me balla haver accennata la suddetta»
“per teflimonio, che anticamente ancora era in uso questo-giuoco; e tralalcio di
harrare J' altre manicre che son molie, perché non fa a proposito noltro ee






SE



















o0ti('(“' Cz AT

se il Lettore ne faffe curioso legga Polid. Verg. lib. 2.
Aleatoribus lib, pr, cap. 29, Buieng, de lud, Vet. Gap ye
rum gen. lib, 3. Cap, 21. Ho decto, che questo Aljotio ogg
zi, ed il nostro Autore ci addita quetla verita, faccndo r
prrché si giochi, all' altofso, Se trovar le carte ivi mon posso; e intend: V
sempre, e f€ non troverd carte,giuocherd all' a/io/so, quantungue
fagazzi, pur ch' io soddisfaccia al viziofo genio, che ho di giocare.

VAN co libri, ec, A' Dottori, quando portati alla fepoltura j

mettere nel feretro,o bara i librised a i Cavalieri la (pada al fiance f

dice, che fara fatto a lui, che per far conolcere, che meiitre ville era |

re, gli faranno una ghirlanda di quei fiort, che sono itmpreifi nelle

veste fara ricamata di picche, e di cuori, e sotto la testa git c

mattoni; ed in questa maniera hawia anch' egii actorno tucei quattro'

sono impretfi nelle carte da giocare a primicra. + a ee
Far sn quarto @ germini, Giocare in quaccro alle minchiace y Vedi fop

aS etek

























questo C. st. 61. 5 00? th Se Ta
STANZA LXXI. STANZA LXX th
Volea seguir, ma tutti della lanza Amoltance ch e buow ai bei
Gii dieron fu la voce con il dire, E por da bene,acor chia a
Che il perdere e comune,e fhar' nfanza, Dt questo suo viocar, don't si he
E perde una miferia ds tre lire, i p ow
Pero si qusets pure,e habbsa speranta Dicendo &° a impiccarie hon, y
C' an giorno la difdetta ha da finire, L! bauer femp.icemienteunpo dm o
'Pero che i tempi variabili sono, Ma quand snch ezti havefse ivan Ga
E dopo il triffo n' ha a venire il buono, Del far la [pia non se we fa pr by
STANZA LXXIl, STANZA Lax th
Intanto gli mostraron il Prigione Ed al prigion preterito imperferto” oy
Che fort' il manto deit lpoerifia Rinolto con le carve im man l'itty \ >
In carwa, dicendo, in divozione Già fattofele porre a dirimpette &,
Faceva lo scultore, ideft la spia; i giocar a' nna crazia la tay
Berd, perch' im essetro egli è un euidone Ouver si metra fuor in [i a |
L  impicchi s* ei vuoi far opera pi: Vn teftoncino, e sia guerra finitay | fy
Serragli pur, dicean, la gola, e poi y Così lo prega, lo sconginra,e inpatt |
S' ci dice più nulla, apponlo a noi, Bada pur sempre « mescolar leo
Voleva il Generale contiouare il sao lamento, ma 1 circoftanti lo &
tare confolandolo,¢ mostranuogli, ch' ei si faceva (corgere a far tanto ke ¥
per una perdita disi pochi foidi: Intanto gli prelentarono Piaccianted
it, che lo facefle impiccare, perché eglt era Spia; Ma il Generale buonhio® )

jo fece liberare, dicendo, che un poco d' indizic non era bastance a
care, ed oltre a questo del fara spia non se ne fa ne meno procetllo 5
che se s' havessero a fare impiccare tutte le spic ci farebbe facceada, |
medesimo Generale inuita Piaccianteo a giocar seco di poco,¢ (olo per'
Nei che il Poeta esprime il vizio internaco di giuocare, che era
ché nello stesso tempo, che determina di non voler mai più gi i
tersi a giocare fino con un vil prigione, con /' anticta y che muitea Ma









OTTAVO CANTARE. 417
\der sempre a mescolar le carte; come fanno coloro, che punti dal gino-
per haver perduto, vorrebbono pur trovare con chi giocare per ricattarsi.
'LI dieron fu la voce, Lo fecero chetare. Latino.: Vocem alicui comprimere,.
CDE una miferia di tre lire, Perde poco. La voce miferia, che per altro
ifica infelicita, o avarizia, usata in questi termini serve per avvilire; e pero
ime qui una somma di niuna considerazione.
i SOTTO a manto d@' Ipocrifia, Sotto feula, (otto pretesto, sotto coperta di far






teh

: - BACEVA ta feultore, Cie faceva I alcoltatore, e non lo statuario, ed inten.
de, Stava alla feolta, cio fava ascoltando 1 ditcorsi d' altri per ridirgii; e cons
| termine equivoco viene a dir copertamente Far /a /pia, come dichiara il

medesimo
-G71DONE. Furfante, Huomo d' infima plebe (enza riputazione, Vedi sopra




Gr, 63. '
AP. a noi, Mlins crimen affinge nobis, See' fa più la spia, gaftiga noi.
4 'Tiathcuriamo » OP entriamo mallevadori, che e' non fara pil la (pia, Elo
[ll fleflo', che mo danno, che vedremo [otto C, 11, st. 49. cioè mio sia it danno, se non
'jail Segue tort, Come iv dico,

 HVO.MO di buona paffa. Huomo di buona natura, Latino Oleo tranquiltior.
i Plauets in Poenuio', dra hunc canem faciam pibi-oleo tranquilliorem, farò stare zitto,



| 60m' olio,
yet = — DOP ci fr enasta, Dove egli pecca. Con che egli varia la sua buona natura.
4 ~ DEL far la spia non se ne fa proce(so, Gaftigar uno senza far proceffo vuol dir

iio fommariamente. Latino indica canfsa, o più tolto, de plano, cioè
ein nea ita di giudizio, senza sedere a banco di ragione, © come si dice an-
4 che volgarmente pro tribunals; ma qui par che voglia due', che le spie noa folo

we non si gaitigano, ma ne anche se ne fa proceffo. %

yal. PRIGION preterito imperfetto. La voce preterito, che suona passato, qui vuok

wie dir, che il prigione era dictro al Generale; e la voce smperfetro denota Vimperfe-
zione, e vighiaccheria di Piaccianteo.

uli. “
 FN teffoncino. Teflone e una moneta, che vale tre paoli, e da molti in occa-
il fione di giuoco si dice Vm re/toncino, per intendere giochiamo folo un teftone,¢ fis

ai S774 fimta, coe non si giuochi più.;;

BADA a mescolare ve carte, Con questa azione di badare ( cioè continovare ) a
mecolar le carve inuitando colui a giocare esprime, come habbiamo detto, las
ye BFAD vVoglia, che il Generale ha di giocare,
; “STANZA LidY, STANZA LXXVIL
i Queeli che compracerto non gis cofka, Duraro a battagliar forse tre hore,
£ vede bauerl'hauuta a buon mercato; Poi la levaron quasi che del pari;

4 Li inusto tiene, e regge a ogns poita, Se non ch' il General fu vincitore

'i Ben ch'ei non habbia un bagattino atiato, Di certa po di somma di danari,

= E dice, al più faremo una batospa E perché gli domanda, e fa. Sealpore,:

4 Kuddei mi vincaye vogiia esser pagato, Quei, che gli spefe in cene,e in desinari,

“p Li rapa sangue non si pnd cavare Lon bauer ( dice) manco affegnamento
Ne far due cose y perdere, e pagare « Tal ct Amoftance resta al fallimento,

a tele: gre en










ge MALMANTILE =

Piaccianteo actetta l'innito, e mefiiGi.a giocare il
@ alquanti denari; ma perché Piancianteo non ne haveva
grit Così fa la Fortuna, quando perleguita un giuocatore
o|amente quando oon vi è modo d ciler pagato
\item[L'HA havuta a buon mercato] Ha (campato un gran perict
non ha havuta quella pena, o gafligo., che egli conolceva c
TIENE L inuto, Accetta l'inuito,¢ s' accorda a giocare,..
REGGE a ogni posta, Posta ( trattandosi di giuoco ) vuol dir
danaro ? che 1 giuocatori concordano, che corra volta per, eal
si dice inuirare, e reggere 4 ogni posta, 8' intende tenere tutti gl' inuil
BAGATTINO. La quarta parte del quattrino Fiorentino, con altro none
detto Picciolo, Latino We obolum quidem, Voce, è moneta Veneziana,
FARE una batofta, Combattere, e gueftionare con parole, ec. Latino,
cari, ed habbiamo aucora ii verbo barofare, per combattere prope
ria di Semifonte trattato quarto, Non havendo tanta.geate, che
Terra batoftare, E più (otto. Hor dt qud, hor di la si baoftafe., j
NON si pus cavar di rapa sangue. Non i può cavare una cosa di, wee
&. Latino. gum è pumice postalare. Plauto. Nam tu aquam e pumice non pr

stulas, qui ipfus fitiat. iva
LA levaron quasi del pari, Cis' intende /a scrittura; Non vi corle qa iene,

cio' si vinfe, e si perdé poco. mitidiy
FA fealpore; Fa romore; Contende alzando la voce. 5 Oe

NON hauer manco afegnamento. Non haver danari, ne modo da trovames.
Ela voce manco in questi termini ha la forza del Latino, nec etiam, ome
quidem, che noi pure diciamo, ne pare, ne meno, ne ance. lo credo,

Ce corrotta da ne anco.

REST A al fallimento, Refla con quel credito da non tlguoter mai pt

fallito s' intcnde colui, che non ha denasi, ne aflegnamenti,

FINE DELL' OITTAVO CANTARE,



ae












: ARGOMENTO
" Ginnti i rinfrescbi, e inusgorito il Campo



ie

Qe

a
Corre all' affalto, e segue aspra baruffa;

~ Malmantil quaft e preso, ond' al un feampo %,
f Chiama all' accordo, e termina la rufa, [ae
i Chi tratta più di guerra hor trova inceampo, >

a = Perché nell' allegrezze ognun si tnffa,
5 Faffi in Corte il conuito, e poi, dal vino
. Riscaldati quei Principi, il feltine. Ol
«ERPS EARS Pb Pape Pape ce ere? 7
: Nie
as 48
STANZAI. STANZAILIL
ye Aguerra, ch' in Latino e detta bello Si che e' mi par ben tondo,ed un corriva,
Parbrutta ame in volgar per fei Befane, Chi pus fear bene in casa allezro,e fano,
Non cr altro se e comincia quel bordello E lascia il proprio per ? appeliativo
Di quell' artigtierie, che fom mal fane, Cercando miglior pan, che que! di grano,
| Eche enon v'é da metter' in caffello; Cen' un' altra ancorch'io non arrixe 5
E Slenti poi per altro com' un cane Ch' e quell'afsalir un con Parmi in mano,
Sere' un quattrino,e pien di vitupero Che non fol non m' ha fatto viliania,
( Ditelo vei, se questo e un bel meftiero, Ma, che mai viddi in vifo in vita mia,
STANZAIL  STANZA IV,
E pur la gente corre,¢ vi s' accampa Florsit cerchi chi vuot bartagliae rifse,
- Ognit per farfiua'huomo,e acquiftar gradi, E si chiarifeaye prow: un po le chiare,
~ Quasi degli buomms cosa sia la frampa Che s' io. credeffi farmi un altro Viifse
Mentr! il canarne l'ofsa avvien aradi, L'armi,percioné m'hano ainzapognare,
LA gli buomin si disfanno,e chi ne feapa Ognuno ha il [uo capriccio, come difse
#14 tirato diciatto con tre dadi, Quel Lanzoghe volea farft impiccare,
E pria ch' ei ginnga a efser Caporale 'Pero mi quiero, ma perc' bora brama
CHangierd certo, più d'un fraiodi fale, Atoftrarus il vero;attenti,e cominciamo,

,Per introduzione de! presente Cantare, nel quale il Poeta vuol deferiver | af.
Ito dato. a Malmantie, si serve della dimostrazione, che la guerra (ia una brut.

ta cosa, e che pero habbiano poco giudizio coloro, che vt vauno; perché se be»
nei Latini la chi o Bello ( il che secondo alcunt facevano per aatifrali, cig'

Gee 2 pec




























420 MALMANTILE

per una figura di parlare contraria a » che s'intende, c
bosco, che € senza luce; Parce le, che memine proctnt
guerra, che non ha in se cosa aleuna di bello, egli nondimeno
tissima, e ripiena di pericoli, come farebbe a dire i colpi 3
abbondante di patimenti, e stenti come farebbe il non haver, che
non haver mai denari; onde un Poeta per ispiegar la bruttezza di
Lelia horrida bella, Oltre a questo @ contro alle ragioni della
gnar I armi a danno di chi mai ci fece ingiuria alcuna, disse un G
lum a beluis dicirur, perché & cosa darbeftie, Si maraviglia pero
vada volentieri ingannata dalla speranza, che in quella si face
¢ non s' accorgono, che più tosto vi Gi disfanno, e quand' anche g
ci vuol degli anni prima, che uno confeguisca i minori gradi della o
la guerra Vx fol ne premia, € un million ne ammiazra. Conchivde p
vo di giudizio colui, che potendo stare a casa sua con ogni commodo,
trigarsi con la guerra, e che quanto a se, quand' anche fufle certo d
ventare il maggior' huomo del mondo, non si lascera mai lufingare da
ranze: Ma perché egli fa, che ognuno può far di se a suo modo, sosp
scorrer pil de i mali, che nascono dalla guerra, e s' accinge a;
con deferivere l'affalto dato a Malmantile dall' esercito di Baldone.
IN volgare, Cioè a parlare chiaro, fuor di gramatica. '
BRVTT A per jei befane, Befane come dicemmo si C..8, st. 30. vu
Panioccio fatto di cenci,e di qui per Befana intendiamo non solamente
na brutta, e mal fatta; Ma le Balie si servono della voce befana per i
una di quelle Larue, che nuocono a i bambini, come il Baw, er.;'¢ gli p
no, che ci sia la Befana cattiva,¢ la buona, e che venga nellecale perk
del cammino del focolare, e però la notte avanti al giorno dell' Bpifania,
Gio, Villani lib. 7, e I nostro popolo anc' oggi chiama Befania, onde;
mente vien questo nome di Befana, come s'& detto sopra, fanno che i
appicchino le calze a i cammini, perché le dette Befane gliel' empiano di
buona, o cattiva, secondo, che essi sono stati 6 buoni, o cattivi ze tali
buone, o eattive si figurano sempre brutte; onde bratro per fei befane vuol dit
eftremamente brutto. J Filofofi icolaftici per esprimer più la, che i
dicono M2 «fo, dando alle qualita gradi fino in otto, e volgarmente per elprimt
lo Reflo si dice Sei, come di fei corre, ec, se bene e un termine, che ha
furbelco, Cscala per fei putte, e simili. Ll Ferrari cavando la definizione
na dal Politi Aucor Sanefe la descrive così: Larwale fimulacrum, ——
nia puerss terriculamentum Suspenditur; unde nomen invenit, B foggiunfe w
mulieres deformes Befane dicuntur larua illa turpiores. Dice finalmente, che i Frar
cel dicono T#phanie dal Greco Tbeophama, cioè Apparizione.d' iddio.
nocte danno ad intendere le superftiziot(e,¢ ignoranti femmine a' semplici
li, che seguano molte cose fuor dell' ordine della natura., miracoloic,
per esser la vigilia della fefta de' Magi, né sanno, che con questo nt ¢
Persiani, ond' ebbe origine, eran chiamati i Savi, e intendenti
Natura, delle Stelle, e de! Ciclo. ia 3 NM





at

EPS &

eee FF FaTRRSEs

oe: esx EB >is



WEL bordello, La voce bordello, che propriamente vuol ie i igo








NONO CANTARE. qat

blico dove abitano:le meretrici., e prefa da noi in più fenfi, come per frepito 5 0

per una cosa flucchevole:,¢ noiofa, come è prefa nel presente Juogo, e altri la

iglian inteoder Difficulta, o fatica »comela prefe il Lalli nella sua Ea.tr.

le paroled: Verg: Hoc opus bic labor,:

enn Ene aio bello 5

8 et Cafacalda si va presto presto;

}) gameboeene| 2-1 | Ma vitornar in fu, quefioe il bordello,

aa "0 è da mettere in Castelo. Specie di pariar Janadattico, del quale par-

2 a. sopra C, 1. st, 29. alla voce /eminato, es' intende Non v' è da mettere in
~,



» che significa poi; non v' e reba da mertere sm corpo, cioè non y'é da man-

'. In furbesco; Non v'é da smorfire; Non-v' e da empiere il fuflo, che così

, dicefiil corpo nello stesso modo, che in Greco volgare si dice Cormi da literale
a Corner, che vuol dire Fuffo,o Ceppo, Latino /ipes, candex.

| ~ STENT A come uncane, Patilci, ed hai careftia delle cose necessarie.al vivere.

: eo della Caccia lib. 5. Ergo age duro dffuescant vittu catuli, Si dice frentar

bracco, quando uno per la sua poverta ha male il modo di provvedersi il

we

ie mee

-PIENO ai vitupero. Pieno di pidocchi, rogna, ed altre tattere, e porcheries
4 i indivitbil della soldzvefea yi chet dice anche: Pieno “4s Bobbio, dal

— Latinovepprobrinm, ebbrobrio ) e Peno di fastidio; del refto Vitupere significa infamia

bye vergegna, Bocc, Nov. 63.
in ET.. Abi vitupero del guasto mondo
ptt T] medesimo Boccaccio nella Teleide lib. 1.

BOs Abi vitupero della gente Achiva,

ee Omero,¢ Epimenide citato da', Paolo diticro in questo senso mala probra, cio'
id vituperosi.
o Per farff haomo. Per diventar un' huomo valorofo: Che essere un huomo, 6
sit an'huome, serve apprelso di noi per intender quello, che intendeva Diogene,
of ES diceva 1 Aominem quero, diccfi Esser un' huomo Givven. f wis efe aliquis,
ie scrittara Confortamini, & essere robuffi, Omero, Viri effore, & forte cor fumite,
af VCHivescampa. Scampare vuol dire fuggire, scappare, © liberarsi da un peri-
PI colo + € qui intende chi eicé vivo, o avanza alla guerra, Scampare 5 quali u/cire
J dal.campo; dalla battaglia. !
| © MA twraterdiciotto con tre dadi. Ha havuto la maggior fortuna, che si possa,
haere y/perché il cum, 18, ¢)il maggiore, che si potia fare con tre adi. 1.Gre~
J cl pure ond eae dicevano: Ter /ex iattare, come si ricava da Giulio
| Pollucesnell? Onomattico. Sy aah its
Bi CAPORALE. Capo di fquadra, che fra gli Vfiziali e il minor grado, che si
j dia nella milizia, Caporati differ gli antichi per Principale,Latino Capitalis. Gio,
Villani 1. 28. parlando di Roma dice:

ee Fu caporale regno di se medesima
— Biib. 12.89. eA tutte le caporali Citia a' tralia.
La voce è formata dail' antico plurale 'Capora', come Campora, Borgora', es

simili. °°:
- MANGERA pri: duno fraio di fale, Significa-consumesa molto tempo, perch
x molto




—














4uz MALMANTILE™
molto tempo ci vuole aun' huomo folo a consumare 'uno f

chi, quando volevano significare-un 'tempo lungo; dicevano com
che sled da mangiare a d' un -moggio di sale, Cicerone de Ami
que illud est, quod vulgo dicitur wultos modior 'falis fimul edendos efse
nus expletiim fit, Questamaniera proverbiale pure in. pro
usata da Piurarco nel libro della multiplicita degli amici » Si può
che inghiottira pil d' un boccone amaro 5: e di poco suo:guflo.
con troppo fale si dice amara; e pero mangiando molto faleman
amaro. ' ' +.
TONDO, e-corrivo.. Si poslon dir-finonimi; e il primo signific
fo, ed inGpido, ed il secondo 5 che:si dice\anche: Corribo., huomo leg
cile a creder' ogni cosa. Latino credu/xs:.. 1\Napolerani dicono ¢
minchionare, burlare.,.¢ dar. pasto'a uno; sopra-C..6..f, 80, disse.«
tondi più dell' O di Giotto, chesuona loyftefla, Tonsa fimiimente: pre
}i vale balordo, dappoco., semplice, goffoxCunto degli cunti?
Bue.:
LASCIAR il proprio per  appellative. Maniera di dire tratta dalla
in cui si danno nomi di due forte, alcuni chiamati propri, altri appell
dire; Lasciar il certo per It incerto, Bar come il Cand' Biopo ci
che haveva in bocca,per pigliar quella,della quale yedeva,lo shattimento 4
qua, che gli pareva magguore, e lo stefto significato ha; Cercarmmighor
grano, Eliodo Poeta Greco: Folle e colui, che lascia andar le cose facili
¢ con certa speme segue le pin difficili, e lomrane..\4\ pene
10 non arrino, Cioè lo noa comprendo + lo non arrivo col mio giudizio a it
tendere. In lingua furbesca.. foo» ammasco s non redo  cive non piglio, nonae
zanno, non comprendo. Lat. non affeguor. iru

ESPs SGP SSE

ee ee rset



VILLANIA, [ngiuria,Soprufo y mal termine s LG

S? io credeffi farmi un nuouo Viiffe ec, 8' io credefGi di diventare il maggior hut
mo del mondo. Diciamo Va nuoxe Orlando. L Greci Alter Hercules, 'gh

SI chiarisca col pronar le chiare, S' accerti di que(ta cosa con provare le feri
perché chiara intendiamo quell' albume deli' uova, i quale s' adopra a medicit
le ferite, vedi sopra C, 1. stan. 60. ed il Poeta servendosi del verbo ehrarive che
vuol dire (caponire, o (gannare,€ della voce chiare fa nascere lo (cherad.
 INZAMPOGNARE., Ingannar con infinghe. Lat, Verba dare:ed e 10 hielo
che iatinocchiare detto sopra C, 7, stan. 14. Dalla natura del suono, e della Me
fica.; incancatrice delle meoti degli nomin.. Fra tutti gli trumenti però. que d
fiato, levano più di (efto, e pare, che percuotano l'anima più gagit ¢
onde furono, ad esclufione degli altri,usati nelle battaglie, nelle quali facevad
meitieri tor via da cuori l'appreafione del pericolo., e infonderni la a
speranza. Noi habbiamo un Proverbio. Far come i Pifferi di montagna (|
nator di piffero strumento di fiato contadine(co.) che andarono per piferare
rono pifferats. Volcano minchionare gli altri coi, darne, ¢. furonc.
col toccaine. Fare uno cornamu/a appresso i\ Puici, ¢'] Burchiclioe
inzampognare verbo facto da stampogna strumento di fiato rufticale, così a
Symphonia, della qual voce servcndosi Daniello al cap. 3. nell' Litoria






cn x aie




 ciulli 5





= a

ae

Ss

S=etis &. BEx es a i EO

, NONO CANTARE.

43

¢ narrando che efi non attefero-punto il cenno., che per comando Regio
si dava, @ adorare la Statua, col suono di tromba, di cevera, di finfonia:, e di

ees ae suoni; sg si può dire [ fiami lecito qui dt servicmi. di questa baga ma-
inzampognare,come git altri. Tromper in Vranz,

=» e pur dal Latino carmina,

we

be Dis LEAN ZAWV..
2 aurora ye come diligente
azza le stelle in Cielo, ¢fapulito,
eae ffi alla finestra d! Oriente
Evora l orinal del suo Marito,
. Ma perché il Carretton ricco, e lucente

. Acciocch'ei non la vegga/cociase/ciarta,
eee: amegerneved ei iirimpiatta.
: STANZA:
Quande il vutto easone ' (rinfresco '
St che,chi hauea col mafticar dinieto,
pe oma iecamente il corpo al desco y
E come si /uot dir ) riebbe il pero;
ae Hi General, che tutta notte al fresco
nda con? Afirolabio innanzi,e indreto,
Batrendo la Diana in sul lunario
Hanea fatto di Stelle un calendario,

~ Edi nostro Autore dice =



. Già muone il Sole,ed ella U' ha sentiza,.

as = forse a corno, o tromba de' ciurmatori: E Charmer Ancantayes >

UGNFNO ha il suo capriccio. Virg. Quifque fuas patimur manes, Ogauno ha je
: fantalic, Vo Lanzo, essendo riprelo, perché faceva cose da esser impiccato,
ve Che folerce tire » lasciatte far a ie 5 percht ho ancor ie mie pelle capricce, BE chi
ha Lanzw, Vedi sopra C, 1. stan. 52. ¢C. 4. stan, 36.

STANZA VII.
seaienaat era anch' egli riuedere
Tutto quanto aggrez.rato al pappalecco,
Done per hauer meglio it suo doxere
Fece in principioun bel murare afecca:
Quando fu pieno,al fin chiefe da bere,
E poi ch' egti hebbe in molle polto ilbecco:
Fighnoli, 3 4iffe, omai venutat l'hora,
Ch' e' si tratta d' hanerla acauar fuora.
STANZA VIIiL.

S' a mensa ognun di voi tanto s' affolta,
Atangia per quattro,e bewerpoi per fecte 5
Che par proprio che sia giuntoa ricolta,
Anrich'egls bablia afar le fuevendette,
Tat ch' io pensai vedern' anc' una volta
La tonaglia ingoiare,¢ le faluette,
Ed bebbi un tratto anche di me paura,
Per una spalla dauola sicura,

“I nostro Poeca de(crivendo la levata del Sole imita Daate nel Purg, C.2.dove,
descrivendo anch' egli il parcir dell” Aurora dice:

65 bid Sicche le bianche,¢ le vermiglie.guance
La doue io era, dela bella Aurora,
Per troppa etade dieninan rance.

Accio ch' ei non la vegea feoncia ye [ciatta,
Manda git impannata,¢ si rimpiatta.

Bd intendono Vaoy et Alero, che quei colore, 1: quale appariva nell' Oriz-
lente per caula dell' Aurora, era quai (parito; ed in iu queit' hora comparue la
munizione da bocca, edi soldati i rinfrescarono. Dopo di che 1) Generale det-
'We principio a far 1' orazione per inanimire i Soldau j quaic Orazione militace si
Soutiene nelle presenti stanze fettima, e ottava, e nelle quaccro segueati.

ere de fielle in Cielo, e fa pulico,. L' Aurora coi ivy ipicndore, offuscas

quelio






















4z4 MALMANTILE ~

quello delle Stelle, € così le leva dai Cielo e lo fgombra;
VOT AP orinas del uo Mdarito, Cioè del vecchio Titone favol
la Aurora, Virg, Tithont crovenm tinquens Aurora cubile. D
cubina di Titon antico già s* imbiancaua al balzo a' Oriente Puor delle
dolce amico, Qui pero descrive Aurora nei suo primo app
la parola # imbiancana. Li noitro Poeta poi-, per 'Vorsmale e J
tende quella rugiada, la quale caica sopr' alla terra cicca'l apparic
la qual' hora l'Alba, o Aurora si perde 5 pero dice Adanda gin o impan
rimpiatta, cioè ferra le tinciire, es' asconde J “
SCONCLA, e feiatta. Si potion dir Sinonimi.. Se bene /oon cia
mente dire una Donna, che non si sia ancora accomodata icapelli
quale accomodaiento di capelit dice Accunciatu ase feiatea vaold
scompotta, e che habbta gu abiti male adattati, e agguitati
sconcio e pil generica, che nome la voce /sarto 5 core:
tine. Znconcinnus, inbonestus, wdecens, incompositus',
1M? ANNAT A, Così chiawiiamo queiteiat di legao sportellatt
tono alle fineilre per chiuderle con carta, tela 50 vetsr, che vi si
fenderG dai treddo,o dal Soe 5 & mandar git  émpannaca vuoldir se
tclio di gueito telaio, e chiuder la fineitra; perché per lo pile deceit:
aggiu(tati in maniera, che per aprire, e Chiudere s'\ alzano, ed. abbul
diciamo tar fu, e manaar gin. 6
SJ rimpiatta, S? a(conde. Vedi sopra C, 7. stam. 66.
HAVE A col mafwar dimcto, A chi era vietato i mangiare t
havevano ) traslato da 1 Magittract di Firenze y Re' quaii ti dice baxer
non poter confeguirgli, e aver proibizione per qualche tempo di et
jut, che v' habbra parenu, oche gi habbia efereman di corto, Oo) per  @
givni ttabilite dalle Jeggi. Dan. Purg, C. 14. one
Lav' e meffier ds conforto Diniero, asthe
Negli Statuti Fiorentini diceti barbaramecate Dewerum ou itl
LIET AMENT E, Vuol dire-Ailegramente da lito; se bene i noltti Contile
pi dicono /eramenre in vece di prettamente; e forse qui i Autore lo Cee
fio tenlo; perché si può credere, che 1 soldatis' accoftafiero & mangiare:
gramente, e preftamente. Li Lat edacer donde e venutu il Poscano Allegri s*
1 Frangele Alaigre ( che pil mostra la iua origine ) vale pronto, H
E /efo per avventura puo eiler fatto da servs ae
AP POGGLARE il corpo al desco. Si dice anche di chi rifeuote danari o prove
fione da banco, o Juogo pubbico. Cie accoltarti alla menia per mangiare.
RIEBBE wf pero, Svritociiia + Ripreie forza sok pero quello tia, vedi
6. ttan, 107. Del riavere i) peto vedi wna curiosa noveilettain Giovannt:
te,detto Gioviano Poatano,ne! Diaiogo iniacolato earenio p
cipio. Del maic che:fa al vento caccaiuly, © del beue, che neit:
cice 5 se ne legge un'epig) Greco di Nv » melita 3 1
dire Fiorita Kaccolta de' medetunt bpigramun 51 quaic cradon ave
suona così. Peditus occadst muitos incinjus in aluo; Lipiojts batoo,
Seriat y@ occidie rurfum si peditus; ergo Regibus auguftis quis





Fz ER THELIST.

Be

eee 2 baw ere











cue NONOCANTARE ~- — 45

BATTENDO 14 Diana in sul lunario, Tremando dal freddo per essere thao
all' aria a considerar le stelle. Batrer la Diana, Vuol dir battere il tamburo all'
pparir del giorno, quando si vede la Stella mattutina, ovvero Stella Diana, cioè

del di. Ma per mecafora intendiamo battere i denti per il freddo, che di- ae

mo anche barter la bora, Vedi sopra C. 8. fan. 6, >, a

 TVTTO aggrezzato, Intirizzato per il freddo; Affiderato; Agghiacciato;;

sghiadato; morto di freddo. sggrinzato truovafi nell' antico per secco, es
liato di carne, quali sogliono restare i morti ( appellati perciò da Greci /i-

res, ci0é privi d' umidore, secondo che vuole Pjutarco nel libro intitolaro 'J
inal sia de' due più profitrenole; ! acqua, o pure i fuoco,¢ quali si veggono essere



is mie structe, smunte, e secche. da Aggrinzaro forte e nato Aggrizzato, ps
| PAPPALECCO, Antende al mangiamento in generale: che per altro Pappa-

 decco se - leccornia, ghiottornia ( Franzcle; friandife ) come habbiamo veduto

1C.7. stan. 55.
i hes Os niece il suo donere, ec, Mostra che il Generale, essendo affamato,
yi aifolratle anch' egli a mangiare, acciocché gli toccaffe la sua parte; intenden-
j ' do che mangio assai prima di bere Tee murare a fecco, vuol dir murare senza
eaicina o alcro bicume, ma con i foii safi, e trateandosi di mangtare vuol cir
jot Mangiare senza bere. Nell' antico facevano la parte a mangiare, e a cia~
feheduno toccava la sua; il Juffo poi levd questa usanza; dice Plucarco nelle Que-
 stioni Conviviali lib, 2. g. 10.
; MESSE il becco in molie, Vuol dit bere, pigliandosi la voce becco, che vuol dir
re il rofteo degli uccellr, per la bocea deli huomo, Queito detto merrer il becco in
molle Gguinca auche parlare, aprir la bocca. Gli Spagauoli la faccia dell' humo

dicot roffro da quella degli uccelli.
i 'S' afolta'. S? atfacica con furia, e con vehemenza.
im STA Gitmo 4 ricotra, Cioè ch' e' G sia nell' abbondanza maggiore, come si fup-
pone che e' si sia nel tempo, che si faono le raccolte: Se forse nua voletim» dire,
che costoro mangiando facevano uno sparecchiare simile a quello, che tanao co-
loro che fegano 1 grano, ec.

PAR cbt egli habbia a far le sue vendette. Quand' altri mangia,¢ beve assai,o
fa quaififia operazione fen' iatermiiione, riposo, o rispiarmo, ci serviano di
queito'detto, assomigliando quel tale a uno, che per vendicarsi portato dail' ira
Opert veementemente.

PER una spalla davola sicura.M'era entrato così gran timore, che non mangiaflero
anche me, che d'accordo havrei daca una delle mie spalle per confecuarim: 1 ceito,

STANZA IX. STANZA xX.
Redeamus ad rem; Se ( come ho detto') Che quasi fui per dar nelle girelle,
Qua fufte al ber infer mie al magiar fani, Perché dopo ch' i punti della Luna





Eco+ coltelli sn man, (Pandoui a petto y Hebbs deferitti, e che extse'le Helle
| Runfeiste si brani (parapant, fic Haneuo rassegnate ad una ad una
bli battaglia vedervi ancora aspetto Trouo [marrite bauer le Gallinelle:
| Con la spada così menar le mant y Ma dopo è, ch' io mi dauo alla fortund,
Ona ib aimico vino, ed abbartuto Che fra le elle fiffe, efral' erranti,
NNe sia, come franotre ho preveduto « Won vedenone anche i Mercatanti,
VR Hhh <2 Ska

CRRREBALERE EMASE.



=





































26 MALMANTILE

STANZA XI.

M€a diffi poi da me, che poco importa
Se quel branco di Polli non si troua
Ani che questo a noi risparrio apporta y
Peroche magian molto,e non fann' nova;

E [e ne anche alcuna Stella ho scorta
De! Mercatanti, gui creder mi gioua,
Che e'fieno in fierayo vero al lor viaggio,
Per laViaLatrea a mercatar formaggio, Essi cerchin la roba, e mo
Seguita il Generale la sua orazione militare, con la quale dopo hai
suoi Soldati di bravi nella maniera, che si vede, termina suo
che si vada ad affaitare il nimico, perché spera y che sieno per h;
tuna per le ragioni, che dice, con le quali da un poco di bur! ara
FVSTE al bere infermi, al mangiar fani, Bevelte, e mang te assai,
gi' Intermi per lo pil vorrcbbono sempre bere, ed i fani mangiano
cassai.:
ST-ANDOV1 a petio co' coltelli in mano. Par che voglia dire,
fronte per far alle coltellate » ed intende, che flayano a mensa uno
altro co' coltelli in mano per tagliar pane, ¢c,, ec.
SPAR AP ANT, Così diciamo per derisione a un bravazzone, e qui ton
ne, perché questi soldati mangiavano gran quantità di pane, 4 '
PIÙ per dar nelle gireile. Fui per dare la volta al cerucllo. Vedi sopraC.t.
GALLINELLE, Quelle sette Stelle, che si veggono fra il Tauro, ef
dette Pleradi; in Lat. Vergilie, Il comento d' Arato Latino. Pleiades 4 plartits
te Graci vocant, Latini eo guod Vere exoriantur Vergilias dunt. Aicum dil
Pleiades sieno nominati, quasi Plefiades cio che si Ranno accoflo,per.
ci le chiamaton anche B try, cioè Grappol d' uva,¢ noi Galinelley p
piccole,¢ in un mucchio. Lt Vberti nel Dittamondo.
Poi disse: guarda nella frome a quelle y
Le qua' da' fani 'Pliadi [on dette,
E che i volear le chiaman Gallinelle, 4
AU! dauo alla fortuna, Mi tribolayo: Mi disperavo: Si dice an
alle freghe, al diauolo, alla versiera, alle bertucce, a' cani e simili,
fortuna: tratto per avventura, da' Marinari, quando disperati, ab
in braccio alla borra(ca; la quale da' nofiri Toscani fortuna di mare 5¢
folutamente vien detta. Il Petrarca s' era dato in un certo, modo alla
quando,descrivendo il suo stato infelice diceva. a wi
Fra si contrari venti in frale barca.
Ui trouo in alto mar senza gouerno,.
E poi. Ch' ia mede/mo non fo quel ch' io
MERE AT ANT1. Le tre stelle del cingold @
Tauro, così dette perché sono infeme, e paion compagae,
ragione. Adercatante dicevano gli antichi quel che noi. oggi p
-reante. L' arte de' Mercatanti nella nostra Città ancora al,
servato l'antico nome.,: % '

SREERGERE

2RERSES

ee ae














NONO CANTARE. 27

 BRANCO 4i polli.Latende le Gallinelle dette di sopra.ll Ferrarialla voce Branca
dice in fondo: Branco eream pro grege.Vin branco-di pecore.Vaa mano di pecore.
Mon n pro mulritudine, ec, Manus autem est branca, ut alibi anumaduerfurm,
REDER mi vious che fien per la Via Latcea, ec. Scherzando con queiti aoint di
clot Gallinelle, e Mercatanti discorce di esse, come se quelle fussero galline,
che son difatili,perché mangiano, e non fanno uova,¢ che questi Mer-
i non eran nel Cicio, percné erano andati a provvederd di formaggio
Via Lactea y la quale egii fuppoae di latte, e che pero vi sia il formaggio a
Mercato; e conchiade, che ancor questi sono difutili, perché fond intenti
ente a' guadagni, e non si curano di gloria di guerre; e pero che e bene, che
. questi non Gi trovino ia Cielo, perché torna a ior favore, e pero si poilas
8 “ entrar' in guerra con buono augurio. Ridicole confeguenze altrologiche, con le
'quali mottra la poca stima, che egli fa dell' Astrologia come di cosa frivola,e vana,
— Fra laren, 8 quel circolo bianco, che divide da una parte all' altra l'Oriz-
"-zonté, edi nose i vede 1m Ciclo la meta, il quale dicono tia formato di miaucil
fime fielle; Da molti è cniamato /a va Romana, Dan. nel Parad, C. 14. la chia-
| m0 Galafia, dalla voce Greca, colla quale queito yalibul cercnio del Ciclo si caia-
Ma Galaxsas, cive laccco,
| Come distinta da minori in maggi
ee Lum biancheggia tras pols del mondo,
a Galafiass, che fa dubbiar ben Sages,
SON boti; Son huoauni di gesso, e di Aucco; che s'intende huomini buo-
ot at ia yilolidi; Lat. frpites, caudices. Vedi sopra C. 4. tan. 17. e sotto C.
ws Tt, fap, 41. Similicudine tratta da quell' immagini, che appicca nelic Chicle chi
ge 8 botato. In ispagayoio Sore e (puatato, che ha il cagho morto, Lat, hebes,
age tt Oftde boro de ingenso vale huomo d' ingegao poco vivace; ouylo.
se | DANNO te ferste con (a penna, Cioè terilcono sella borla, quando scrivono
Te partite in debico a uao. EB verameute le partite in debita sono ferite, perché
GidiceL denars sono it secondo sangue, i) quale con tali ferite si cava d' addosso al
Proilimo, Così i dice volgarmcnte Tarare ana frecesa, calui, che chiede a uo' al-
tro in danari,vedi topra ( 2.¢ insdguinarti chi comincia a toccar guattrini,
sh) Dl dar foro, Deve dare, coe divicae lor dzbitore, e per l'equivoco inten-
de deve Perquocergli; e da cio cava la coalegueuza, che noa fiea buont per las
Suerra, poiche se cia piantaav una partica ( snteadi dispongono una parte, una
# quaama di Soidati Jogauno gli dee dare (taccadi perquocere tali Soldati ) es
j gueilt che da tutti ac coccane, boo son buoui per la guerra, Psancare wna par-
Ma Cinferire, o descrivere nel Giorudle, o uubro di uegozio uaa parte, o arcico-
lo, capo di (crittura, che da dcbuo, € credito a chi s' alpetia; 1 che si dices
anche decendere una parsita y decendere uno debore ye creanwe, toric dal Latino
recerfere, deiccivere y regiltrare.
STANZA XIIL















| Non prima fabili l'andare in GMErTA y Com un bratcod uccelliil quale in terra
Che vede/ts pie presto ch' 10 nol dico Sts calato a beccar grano, o paniva;
Vitleuaiena, «ur trattoyun ferra ferray Va che si muons basta, che quct folo
Ed ir correnas contr' Alil inimne. £4 fuoice pyuare a tats nw volo,
è: Hhh z STAN:



zat,









































428 MALMANTILE™
STANZA XIV.
J coraggioft al primo, che si moffe,
Gli altri (già fendo meglio [ui piccenali )
Non poterono star più alle moffe,
Ma corsero ancor lor come Terzuoli 5
Giunti di Malmantile in fu le fofe,
Drizrate al muro afsai feale a pinuoli
i falirvi renewano una baia
Com' andar pe' piccions in colombaia.
STANZA
Gli fiipits, le foglie, e gli architraui
A quclto efecto efsendo già (murati
Per via di curri, dargani, e ditrani
Gli hanevan su le mura firascinati, Faceano un venga addofsoat
Stabuito d' entrare in guerra,¢ dar U affalto a Malmantile i più ¢
rono i primi a muoverdi, e gli altri meno coraggiofi (eguicarono. &
Dante, che nei Purg. C, 2, dice:
Come quando cogliendo o biada, o loglio
J colombi adunati alla paffura
Quieti senza mostrar U usato orgoglio 5
Se cosa appar ond' essi habbian paura
Subitamente lasciano fRar ? esca
Perché affaliti son da maggior cura,
Arrivati dungue alle mura di Malmantile, credendosi di trovar fac
s' ingannarono, perché quei di sopra gagliardamente si difendevano
altro. Qui e da considerare, che se bene Capitelliye Srontespizzs son me
shitettura, il Poeta (cherzando con I equivoco.di capi, e fronti, e serve
verbo Pampare nel senso, che lo pigliano i Legnaiuoli, ec, che dicen
C. 1, tt. 8., vuol die, che tali merii pictre, ed altro devano sopra 1 2
alic fronti dei soldati, e gli stampavane, cive gli faceyano di quei-
chiamano stampe, ed in fuftanza vuol dire, che rompevano tefle,¢
suono, che rendono i corpi battuti fecero i Greci il lor verbo typrein,
re; da queito verbo ne venne Typus voce pur Greca accettata da'|
una forma imprefia, o cavata fuori col battere: Se ne fece ancora 7}
tamburo, che Omero pil conforme all' origine disse Tympanon seguito
Catullo nel Poema Gailiambico. Noi abbiamo voci da riferire a queste' \
come farebbe Stampa, Stampita, Stampare, Stampanare, Ma in pro
fiampe fatte sul moftaccic d' un' antico Giucatore di pugna, evvi un
gramma del Greco Lucilio, che in nostra lingua voltato dice Così;
2 un vagho, Appollofane, il tua capo,
O qual fu mai pin traforato arnese,
Son tane di formiche 90r dritte, or torte,
E par, che con bizzarre, e varie nore
Vn Lirico eccellente il Lidio v' abbia
Inravolate sopra, ol Frigio canto.

esceftie?

jie te te en ee i a.





NONO CANTARE,
6 Or franco vibra il minacevol pugno
 Ecombarci pur liero in duro arringo 5
 Che se colpo novelio a te discende,:
Quel ch' ai riscoffo, aurai, ma non già nuond
et Capir nel capo tuo potra ferita,
PIP preso chrio nol dico,, Preftitiino consumaron manco tempo a far tal cosa,di
silo, che io consumo a dirlo. Latino dicto citsus.
“N lena leua, un ferra ferra, Quando vogliamo intendere, che una gran quan:
: di popolo adunata in qualche luogo si sia partita in un subito, e velocemente
ia one di questo.detto 5B signiticano quasi lo stesso, se non che l'ultimo ef-





» quando uno è da altri incaizato a correr, ec, vedi sopra C. 1, st. 63. e»

ke
- f hail

pero nel p luogo si potrebbe anche dere, che i primi volon-
 tarj, ed 1 secondi forzati dalla riputazione. 11 Varchi Stor, lib. 2. dice: Pa /ubir
| Wegridato: armi armi, lena lena, ferra serra, ec, Dal che si cava, che questo detto
tog significhi Leva la roba di sopr' alle;moftce delle botteghe, e ferrale come (eguiva
at | Firenze nelle follevaziont di popolo, e che ii medesimo detto sia poi facto co-

Mune a oga: sorta di tumulto, e per ¢sprimer un moto turiofo di quaatita di po-

4
Ll

| AR correnda. Andar correndo. Il verbo ire venendo dal Latino, vale appresso
di aot quaato il verbo anaare, ma ci serviamo folo dell' Intinito ire, del partici-
Pi9 ito, © folo, o accompagnato col verbo essere, e dell' Lmperfetto ina, ixano,
che si dice poi, giva, e giwano, Nella vita di Cosa di Rienzo (critta in lingua Ro-

Mana antica trovali jio, e seffero, e simili, che i Toscani cangiando l'[ coafonan-
foi *ealpra nella doice lettera G dicono gio, cioè andò, € gifero, cioè andaflero.
wi fimiimente prende alcuai tempi, come farebbe i presenti di tutti i modi,
'i dai verbo Vado, io vo; ancorche Dante viatle forefticramente, edadi per Vada;
gg © 0i0 cofretto dalia rima.

» ST ANDÒ mestio in fui piccinoli, Essendo pi gagliardi nelle gambe; e questo
gi AVVeniva, perché havevano mangiato. £ piccinols, che & il gambo delle fruttes.
g Latino pedicutus, e pref comunemente in questo caso per le gambe dell' huomo,
ia NON porersero Rar falc alie moffe. Non potettero contenersi, che non corref-
a fero. Toho da j Wavalli Bacbari, i quali corrono a i palj, che essendo tenuti per

lo freno dai loro Stallon: al luogo donde a) suono della tromba deeono partirsi,
7 che si dice le moffe ( Latino carceres ) molte volte scappano, prima che sia dato i)
' detto segno,e questo si dice non far ferme aie moffe, che poi paflato in proverbio
! non haver pazzicnza, © lofferenza, ma per il gran desiderio d' arriva-
i Tea Uo luogo, partirsi prima del dovere; ed esprime quella inquietudine, che uno
, hanelitaspercar, che /egua una tal cosa da iui anfiofamente bramata. Del Ca-

vallo generoso Virg. Georg. 3.
Stare loco nescit, micat auribus,G tremit artus,
Colettumgue premens volvit sub naribus ignem,

CORSERO come terzxoli, Corsero con la stessa velocita,con ia quale vola alla.
preda il terzuoio (pecie di falcone. Perché così sia detto rende ta ragione il Tua-
No de re accipitraria lib. 1, edtrque ad co. cum tres foetu enitatur eodem Predones gene~
rofa parens mas kitinus imo despectus letto incet appeliatur y & inde Tertius,

SCA-








4jo MALMANTYPLE &
SCALE 4 pivoli. Scale fabricate di due corredti «
glioni sono pivoli ficcati fra 'uho ¥'¢ I altrore C
fine in distanza uguale a riscontro, ovvero'i detti f
© stecche, © regoli di legno conficcati in deeti correntt Mampati
riscontro. B pinole, ( Latino clanicx/a, civxt cavicchio; ovvero
de ogni pezzo di-bastone adattato a porerd mettere in un buco,
TENEV ANO una baia, Stimavand cof: facile;*Stima
burla, ec. Latino mage, Ii Ferrari dice poter venire questa voce da
iflar' a bada, in ozio, Latino wataré, © O01 i
COLOA18 ALE, Quelle stanze fabricate per lo pity nelle form
per uso de i colombi, € nelle quali'wascono i piceionit) «>
FEC ERO parergli altro suono, Fecero lor conotcere, che |
ment.. voy
ewERLI, Qvei picco}i murelli'; in distanza uguale'y ned quali per!
mioano te muraghie delle Città, ¢servond per: parapetti'. ad soldati,
per difefa della muraglia; così dette quali. murnlesdice il Berrari; fume
primes parus murs,Dichiamo @una-cosa;che ancora abbia delle dific
rarsi,¢che non Gi fiano per anco spuntate: £ ci è de merio, cio' non è elpy
to il cutto, Ci rea ancora qualche parte da abbattere 2 Vedi sotto © 12)
ISSO fatro. Subito. Due voci Latine corrotte, e ridotte Toscane,
loro lo feflo signincato.
DISEATTO (e reftuggini, Infrante le Teftuggini animali Terreftri,
che hanno la coccia, © guscio durissimo da alcuni'detti Tartaruche
he, da altri bezzache ( dal bezzicare,.ch' elle fanno raspando in terra
atinl Tefudmes, E § potriaanche dire, che ? Autore intendetle di qu
razions da guerra, che usavano gli antichi dette Te/udines; nelle gi;
no foo alle mura, reggendosi fulie spalie gli uni gli altri, e aiutandofia m
tarui sopra, coperti turu di feudi, € terran iteme per ripararsi da' colpi, che
si (cagliavano per di sopra; E questa operazione s' addimandava refixggine spe
ché flavano col capo, € 'colla vita dentro agli icudi, come stanno le
(in Lp. torragas in Beanz. ortaes ) dentro aile loro scodelle 3 le quali )
dette da' quei dello stato di Muang, come racconta il Ferrari bi/se fo
bijce (codeliaie, perché anno 1i capo di bilcia, e stanno rinchiufe cone i
della; Onde potrebbenfi dire, dom:porte, come un' antico Poeta chiamé le chien
de. Autione famoso ceteratore e fatto parlare da Pacuvio così, delcrive
tetluggine con que' versi portati da Cicerone de divin, ub. 2. Q@madrapes 1am
da, agreftis, bumilis, aspera, Capite breui, cernice anguina, adjpettu trad
ruche,¢ BR2uhe, sovo voci usate dai Caro ne' Mauiaccint; e i} Veneziaiol
chiama Gv/ane dal Gr. Chelonei, da noi si dicono anche butte seodellaic,
BAST le NO Seré, Celebre, e nouttime scrittore d' archucuura.
EbDIF/Z/0, Preto largamente s' inteuce Ogni sorta di faborica, €
ma preso ttrettamente vuol dir faia, ec, Case, ed altre niuraghe, |
ades, @ facio; ed in queito andiamo uniti co' Latini, che per earfien
no ogni sorta di scrittura. Gio, Villani t, 128. Pauose/f ad ascdin, 00, ©
difici, e per cane per forza ebbe, Li lib, del conquiito, Per joraa a



























gE Es PSs SEES. =



'> ie PS

= Fo









NONO CANTARE.

1. Capiteli, e frontejpizi,, Columnarnm capitula, © fronts bespitii, >
(ATT H Srglie.s ¢. aui, Stipi (ono le pietre de i tianchi y¢ foglie quel-
a parneey quelle dilopra, che tutte insieme formano una por-
a» Suipice dal Latino #:pes.. Architrave; quasi trave principa-

: « Quei ruotoli di legno, che servono per facilitare lo strascico de i pei;
atini li ditiero Palange, Vedi sopra C. 2, st. 65. Dichiamo: mertere une fal exr-
Spiguerlo a poco.a poco, e condurlo doicemente a fare alcuna cosa, La
Voce viene probabilmente dal Latino baiudare; questo aggiuttar' un corpo
}a un' altro in maniera, che quello lo porti con sicurezza. E la seconda
| Latino xmbdicus, cioè punto ne) mezzo, Bilicare quali ponere in umbitica,
ARGANO. Strumento, che servc pct tirar fu pefi in alto, che da huomini è
" moflo in giro per via di leve. Alcuni Latini lo dicono Sucu/e, i Greci oniffi, cioè
 Afineli:, e questo & V argano,secondo il Filandro, cum axe iacente, quello pui cum
axe ereite, dice che in Latino e Ergeta, cioè macchina da lavoro; donde, o da
voce(lecondo i} Baido sopra Vitruvio)è fatta la noltra Argano,
MSADATT 1. Scommodi; Non atti a esser portati, o Arascicati.
MC ATI, Meili in bilico-,-0. equilibrio., Latino Jibratis.. Diciamo.bilico
ofitura d' un corpo sopra ad un' altro in maniera, che posando quasi in un
non penda, o aggravi pil da un lato, che dall' alo. L nostri Scarpellini
 dicono baggiclare per biluare. i
it. BOTT O porto, Si dice. Ch' è cb' € 5 colpo colpe sec. e 8' intende Spefiime volte















PAR* un venga, Tirar roba da alto a batio sopra auno, che sia foo.








“a ay STANZALXVI. STANZA XVIIL
a Le Donne anch? esse corron co' figtinoli y Chi, perché gik non piglin l imbeccata
f i 2 dy che troxan, gettan dalle muray Cuopre i capi con tegoli y e mattoni,
o con la conca, o vaso da vinolt Chi verssa git bollente la rannata,

a 9» Pighia a qualcun del capo la. mifura; Che pela i vifie porta via i bordoni,
a8 Profuma il piscio i panni, ei ferraixoli Nei? olto un'altra intigne la granata,
yet Ne guardan vc v'é penail far bruttura, E fal asperges sopra i morioni,
ps Chi tira gi: unjastrone alic cerned y Altre buttan le caffe,accio i soldaté
ie Che se ewe orili serva per murella, Partir si debban, poiché son cafjati,

ie ooNarraiil Poera la difefa, che facevano queidi Malmantile, e descrive diverle
we" Operazioni militari adeguate alla composizione burie(ca di cutta. opera.
CONCA, Valo grande fatto di terra cotta, entro al quale si fanno i bucati
Ke ASO da viyoli, Sono vatetti di terra cotta simili alle conche, ma piccoli, en.
| 80a! quali Gpongono vivoli, cd altre pianterelle d' erbe, o fiori. Dice che.con
v — gucfi pigliano la mifura a.ijcapi y perché hanno il vacuo capace della tetta d? unt
Td huomo; al quale quando i Cappellat voglion pigliare la mifura della testa, metto-
u# ~—-'NO in capo un tappelio; € ceftaco di Malmanzile per pigliar tal milura, in vece
sso un cappelio., mettevano-un valoda vivoli: e cosìscherzando intende y che ti-
@ — ravano (ule tefte a i soldati di Baldone i deni vali.;
@ \SEvi dipenail far brurture', Se\vi e pena il fare sporcizie; Dice che tirano fino
Dorina, e non guardano,-se. ciò sia proibito,: e con questo dire, accenna i} co-
ef flume, che e in Firenze a” affiggere alle muraglic dove non si vuole, che fien fat.
r te





432 MALMANTILE

te sporcizie, certe tavolette di pietra, nelle quali & scritto il
flrato degli Otto, che proibisce, e mette la pena a'chi fa
niuno si posia pretendere ignoranza; Ed intende anche di
¢ grave pena, che è in Firenze a buttare dalle finestre nel
torno a' quali dispone anche la ragion comune, come si vede
De his, qui deiecerine, vel effuderint, ' '
SE v' ¢grilli, Sopra nel C, 6, st. 22. dicemmo, che grille si cl
cosa palla, che si tira per segno, giucando alle palloctole; ed all
firelle, qual giuoco dicemmo come ti facia sopra ia detto C.6.t,
rché tirandosi, or qua, or la alla ventura, o alla volontà
a il falto del grillo, che dopo un breve falteilare si ferma, e-poi
-dicefi ancora Lecco, quali i/ex eMurelle chiamanfi anco
nelle sue Rime. orate
Ch' io do sempre nel lecco alle murelle OP R
dal Toscano antico e#ora, che e lo stesso, che il Latino Moles }ép
si dice di pictre. A'awer la resta piena di griili s' intende uno, che ha capric
vaganti; ¢d il Poeta scherzando'con questo equivoco di' grille dice
quelle laftre a' grilli, che sono neile tette di'coloro, come se piocatietd
strelle, o murelle. Dal pazzo similmente,¢ curioso faito del grillo son detti
icapricci, e fantasie firavaganu, che faltano in capo, e per così dire
PIG LIAR' un' imbeccata, Infreddare: B diciamo ancora: Pighare df mitt
caffrone, perché il beccd, ed il castrone hanno una tal raucedine, che
pre, che cofiano, appwato come fanno gl infreddati.
Té£GOL/, Pezzi di terra cotta adattati a coprire i tetti delle case.

ap
























HlAe.
: RANNAT 4, Liscia forte; che è quell acqua bollita'con cenere; ¢
dalla conca, quando si fanno i bucati. Lacino /ixininm,
BORLON/, Inteudiamo quelle penne, che non de} cutto spun
scorgono dentro alla pelie degli uccelli, e per similitudine intendiamo il)
spunta nella faccia degli huomini « way
FAI alperges con la granata, Diciam far ? asperges quando con spugha
tra cosa si (pruzza acqua, © altro liquore,.a minute stilie; la qual cosa il
chiama e4/pergere, qui dice, che spruzzavan' Olio con le pranate;
aiciato un mazzo di scope, © d' altro simile adattato per (pazzare,)

stanze.

SOLDAT! caffati. S' intendono quelli, che sono stati pri
la milizia, perché cafare vuol dire cancedare: Ed il Poetas
guivoco di <afaté, cioè percotli dalle cafie', dice, che se son
nou dal Campo, perché non son più nel numero de” float,

SLANZA XIX,



Vi? altro con un gatto vwol la berta, Ed il primo ch' et trova
Legato il cala,ond' es fra quei.d'Vgnano - Che dou'ticbiappar
Sguawnialugna, econ la bocca aperta

Griaa ina/prio in sue parlar Soriano s








oF

ee ee ee ee eT oe eS Ol ee

=~ aw






NONO CANTARE: 433

arnt) re Bie XX, e
Miagola, e soffia it gatto, es' arronciglia y
Ed Gite endian heerees
janes quel che oa " trattopigla
Beli è miracol poi se pite gli feappa;
thie oat peter tee cos riglia y
jie Lo tira fu con qualche bella cappa,
a «Ci qustcheciarpayo qualche pinacchiera,
ye  Ecosi gli riesce di far fiera,
ame cool (STANZA XXL
due Quand una volta lasciale calare
ib oi iaers al buffo di Grazian Molletto,
Che fu;di posta per ispiritare,
«Quel pelliccion vedendo intorno al petto,
we Le bestia intanto falta, e dal coliare
'hoe "

=



Tutto prima gli firaccia un bel gigisetto,
fet  Dipos si lanciaye al capo se gli ferra,
ebst  Si che il cappelio gli mando per terra,

STANZA XXIL
Non.sa Grarian, che Diauol si sia quello:
Pur tanto fac' al fine ei se ne sbriga,
Ea aiza il vif per farne un maceilo,
¢Ha vedendo il rigiro, e ch'ei s' intriga
Con dame, vuol canarsi di cappello;
Ma perch' il micio gis ha tolto La briga,
La Dama accsuetrata, anzi civetta
Lo burla, che gli è corsa la berretta,
STANZA XXIILL
Ed ei, che da colei punger si sente y
Onde al nafo lo fironzolo gli fale,
Perde il rispetto,¢ quiui si rifente
Con dirgli, Atona merda, e ogni male,
Vain questo al aria ungraromar digeete,
Che 4 terra feende a mafse dalle scale
Fiaccate,erotte ach'elfe dagii /prazrolt
'Di pierre, c' ancor grattana § cocuzzoli,

oa Continova il Poeta a narrare gli accidenti, che (eguono nell' aflalto di Mal-
ie mantile, e dopo haver detcritto una Donna, la quale con un gatto legato a uns
" i miazzacavallo andava levando rcba da dosso a quetlo, e a quello, come segue a
ol Graziano Molletto ( che e il sig.\ Conte Lorenzo Magalotti ceicbre per aobilta,
HF 'e dottrina ) dice che le scale degli AGalitori furon rotte dagli Allediati: e con i
r faffi, e con altro, che tiraco di sopra alle mura, dava ancora addosso a i soldati.
at IL (a berta, Vuol la burla ( vedi sopra C, 4. st. 47..) onde shertare, lo stef-
4 fo y che beffare. [i Davanzati ped dite Swerrare nella (un traduzione di Tacito.
mY Corte poesie senza antore, che fuertavano le sue crudeid. Se bene in questo luogo si
; poirebbe intender per berta quello strumento, che serve per ficcare i pali ne i
ea pfiumi nel far le fleccaie, che e un gran ceppo di legno ferrato, il quale infilato in
“ln pernio, o ago di ferro confitto sopr' alia testa d' un palo, s'alza per via di fu-

ni, e si lascia ca(care sopr' alla testa del detto palo ( già fitto in terra) per fario

sf andar pita drento. E perché in questa medesima guila faceva Colci coi gatto,in-

yo teade, che defie così /a berra, eruendosi del mazzacavaljo, che appretio gli an-
ti"  tichi era usato per arnese militare, come s' e toccato sopra C.6. st. 86. In propo-
i)" fito di Berta per Bxrla, il Ferrari dice così: ognuno poi la creda, come gli pare
4 f verifimile, Dopo aver detto, che que' delio flato di Milano chiamano Berta
8 ta Gazzera, e ciò dal balbettare,ch' ella fa; foggiugne; (aoniam autem fanne,
gil! At que irrifionis [pecies est aliena verba imitando reperere,inde Berta pro Inda,ae derisione
gi accipitur, © fare una berta illudere, & decipere. O pure finalmente e forte più
credibile, che venga questa maniera di dire dalla novella raccontata sopra nelle
Annotazioni alla St, 47. del quarto Cantare,
d& —. SGVAINA I agna, Cava fuori' ugna, che tiene alcofte dentro alla pelle, la
we bed gli serve per guaina, ed il Poeta scherza, dicendo /guaina U' ugna. lock ques
gnano

Wo + Appropriando benissimo wns, a Vgnano.
yd 4NASPRITO. Incollorito, meflo in ira, in stizza, in rabbla. Latino exa/~
lii IN

peratis,

i








54

IN parlar Soriano, Cioè oer gatti in ling
si dice quello, che ha la pelle di color lionato ferpato d
ché si dia in altri animali, o in panai, non si dice foriana; se
perché i gatti di tal colore fien venuti di Soria, come ai
di Persia quelli di color di topo portati da Pietro della Vaile,
chiamati Persiani, o per Persianini. 'one

DISERT A, Cioè ttroppia; concia male. Guasta.

VVOL ievarne il brano, Brano dal Latino barbaro mn.
il pezzo, Vedi sopra C. 6. st. 47.

MIAGVLARE, o ignaulare. Bi ii
































I gridar de i gatti; + il'fofiare dic
quello strepito, che fanno aprendo la gola, quando fond in rabb

S' ARRONCIGLIA., si torce in sse stesso, come fa la ferpe quan
viene da ronca, roncola, ronciglia; specie d' arme; o più” 2
agricoltori, ed e fata come una spada, ma è torta in cima a guisa d
serve per eflirpare i pruni: o pure da Ronciglio, usato' da' Dante per graf
fauto a uso d' uncino. i "

E MIRACOL 8 egli feappa, E cosa soprannaturale, © imposiibile,
degli artigli, £1 Petrarca. soe eee
E cio, ch' in me non era

Mi pareua un miracolo in altrui
cioè una cosa, che non potefie stare. '

LO tiene in brigla, Cioè 10 maneggia bene, facendolo operat

CLARPA, Dal Franzefe e/charpe, banda, bandiera.. Quel draj
tano i soldati cinto:de' soldati era proprio il cintolo, onde cinguoie fol
dalla milizia, Vedi sopra C. 5. st. 33. 5) ie 7

FAR fiera, Buscar, o acquiftar roba = per esempio ends pirando per
torni, e chi gli dette pane, cht voua, chi una cosa, e chi un' altra tanto,
Satta un poco di fiera, se ne tornd, mn.

D1 posta, Subito: Di primo tempo. Vedi sopra C. 7. st. 92. BY
giuoco di palla, che si dice dar ai posta quando si da alla palla, prima
terra, ed e il Latino ilico, e vefigio, Gli antichi dillero: Di colpo y
fo, che di Borto. 7

FV per spiritare. Hebbe un grandissimo spavento, o paura.

GIGLIETTO. Specie di trina con punte; così detta, perch ha
col giglio.

Avr. Cioè gnell' ordigno, col quale la donna alza, ed ab
Vedi sopra C. 4. st. 69. Se bene & può ree la voce rigire nel
mo sopra C. 7, st. 41,, ed intender, che Graziano, alzando il ca
giro, cioè la donna, e dedurre questa opinione da quel, che soggiung
Vedendo, che s' intriga con Dame,,

ACCWETT AT A, Afiuta; Sagace. Tolto dagli uccelletti,
civertats, quando havendo altre volte veduta la civetta sono dit
non si la(ciano lufingare a volarle attorno, come fanno quelli
mai pil veduta. ae

eANZL cinetta. Più toto troppo ardita, e sfacciata. Si dice'

, eel ee8 TR

e— lL eBeuwe ete

— 7 ate

= ~ - — =




x A juomo da poco, però con tale equivoco



NONO CANTARE: 435

vane troppo ardita nel trattar con gli huomini, quasi faccia con essi, come la
corerasss gi uccelletti, che cerca con gli (uoi gefti di tirargli a se. Vedi for-
to in questo C, st. 60, E Plin, lib. 10. cap. 17.;
CHE gli e corsa la berresta; Che il gatto.ha fatto preda, e gli ha portaro yia il
ppello.. Ma perché, La/ciarficorrer, pee via la berretta, vuol dice Elicres
mentando G h diveherch iandnban tones
raziano womo da poco dal veder, che si lascia rubare, € portar
via il cappello, gli an burla; di che egli s' adira, perché si sente fete
if r¢ dali' etiere burlato da que(ta donna,
-, GLI fale lo frronzufo ai nafo. Derro sporco, che significa entrare in collera, ma
= poco usato, dicendosi pil tolto fair la muffa, o la fenapa, o la moftarda, o it
herimo, ec. Vedi sopra C. x, st. 39. Bil Lalli En, Trau, C, 2. st, 65,
Waapn 6 Airs Corebo un tale firazio,e tanto,
i Con la moffarda al nafo,e nol comporta,
AGli Ebrgi.colla fiefla voce significano, ¢'/ na/o, e ira, perciocché par, che qui-
¥iclla particolarmente rifegga, siccome disse Teocrito + acris bits ad nafum fedet,
Onde noi dichiamo Arric¢iare il nafo per ifdegnarG; simile in parte quel che dice-
wano gli anuichi Leware il miffo. La voce Ebrei fie Aph, in Siriaco Apha; ondes
. itorcec: e venuta la nostra 4fa, colla quale a ete una cosa fomi-
giiaptitima alle vampe dell' ira; cioè un vapore, e yn caldo fallidiofo, e affan-





HO!»
t 'sop SLrifenre. S'adira: Entra in collera, perché e burlato,
pjat 'A merda, Detio ingiuriofo usato fra le donne di vil condizione, e del-
Ta voce mona vedi sopra C, 5. tt, 18.1 Lagini similmente (asum, conum, frerquili-

me,

. FLACCATE. Spezate, Fiaccare & verbo proprio per esprimer, quando un le-
£00, © altro. materiale si rompe in mezzo per fouerchio pelo, Latino fari/cere.,
 springs. Donde poi bxeme fiacco vuol dir huomo affaticato, e stracco; se bene &

ver) imile sche venga dal Latino faces, faccidus, dichiamo, fiaccare |e braccia

A uno, clive infragnerglicle, e romperglicle colle bastonate.

SPKVZZOLARE. Vedi sopra C, 7. st. 15. E qui è detto ironico, ed intende

f Bingge pict 7 '
V2ZZOLO. Latino vertex, cacumen. La parte di sopra del capo diffefi an-
she. Zwecolo 5 siccome da Cocuzza de' Napoletani ( Latino cacarbita ) e si dice an-
Gora. comiznole, se bene questo e proprio delle fommita de' tetti, e de' camumini;
dal Latino cudmen quali culminnlum «
ares ST NZA XXIV, STANZA XXV.
Chi con, chi per banda,.¢ chi fupino Quantungue il.campo annaffi tal rugiada,
i se ne viene, e fa certe cascate, Come le zucche, annarpican le [cale,
Che manco ie farehbe un' Arlecchino y Onde più a' xno in gik versala firada
, Quand in commedia fa le sue fealare; Fa pur di nnono un bel [alto mortaic;
Si che y stinnanzi fecero il fantino, M44, piché ammonti ne traboechije cada,
Le brache in fasti glieran pui cascate, Sardonello [2a forte, e in alto fale,
> B infranes, e pefti andando gis nel foffo E trai mimici al fine a lor mal grado

Mette [u il piede,e agli altri ye Uguado
2 PAN.

, Hanatolere a quchlo nuove scate adddfe, y
geet ii




436 MALMANTILE™
STANZA XXVI.
Chi vidde in un pollaio, ove fisrona | *
Vn numero di polli senza fine
Tra lor cascar qualche pokafira nuona,'
Che roft addoss' elt ha gullie galline
Ciascun per far di lei l'ultima'prona 5
Eye aa folelapariea athee, 1
Che la difende, e da beccar (e porta Ma Eravan, che
Stroppiata rimarrebbe, e forse morta, Aiuto a un cempo,ed
Rotte le scale coloro, che erano sopra di esse cascarono nel fofla
r0 corpi furon polate nuove scale, in fa le quali intrepidamente
neilo falto sul muro y e feel nella Terra, dove fu da mojti di quei
falito: Ma Eravano, che lo vedde in pericolo d' esser ammazzato
¢gli dentro a dargli aiuto.

BOCCONIL. Dittefo in terra, o altrove con la pancia, e faccia ve
no, Lat. pronus contrario di Sapino, fulle reni; Lat. fupinus ye Per,
la doppia posicura che resta, diversa dall' una, e dal' altra, la diciamo 4 x
Per franco,¢ Per latv, Lat. in latus. Bocconi  detto colla stessa forma, che!
nocchioni, Brancoloni, Saltelloni, e simile 5 che si -dicono anche Boccone
vhione, ec, anzi questa ultima maniera è l' usata dagli Autori antichi Ti

eARLECCHINO. Va secondo Zanni, cioè un servo.semplice in
Così nominato, il quale faceva assai bene le scalate, che son quei giuoc
Ai suol fare detto Zanni in commedia con una scala a pivoli, sopra alla
affaticandosi di voler falire, casca in diverse manicre. f

FECERXO il fantine, Pecero il bravo, l' ardito, il coraggiofo, Si
gura. Egli e fantino cioè persona, da fare queffo,e altro, Fantino di
faate. Lat, infans, cioè Ragazzino usato dagli antichi in generale, @
oggi a un significato particolare. Chiamando noi fantini quei R i, ¢
pr' a cavalli spogliati corrono al palio, Si dice anche fare if Baiardina, da
lardo celebre Cavatlo di Rinaldo Paladino, così detto dal suo mantello y
yea essere Baio accefa.:

GLI eran cascate le brache. G\i era entrata la paura addosso
animo. Vedi sopra C, 6. stan. 20, Lat. aninsum desponderant,

ANNAF SI tal rugiada, Annaffiare vuol dire Ammollare 5 o af
giada vuol dire quel che accennammo sopra C. 2. stan. 55. alla voce gr
Ma qui da nome di ragiada a quelle pietre ec, che buttavan già gli

-dnnafiare detto da Adacqware, che si dice anche /anacquare,e Annacquare y
Ui duc ultimi verbi diconfi propriamente del remperare coll acqua il vino;
equare propriamente e dare [ acqua alle piante. + Ia
INARPICARE, Aggrapparsi, forse dal Gr. herpein chet in
Pere, reptare, Salire in alco, appiccandosi con le mani, € co' piedi y
no i gatti. Si dice anche rampicare sopra C. 4, lan. 68. ed-«
vedremo nella seguente ottava 28,:
SALTO mortale. Chiamano i Giocolatori falto mortale,quando
tecra Con le Mani', o con alcro faltano, voltandy la persona fo}












> eran
























WEITAF.

wee Se peg RRFESZLTE=






NONO CANTARE, 437

verifimilmente facevano coloro, che ca(cavano, o erono gittati da alto 'a batfo.
) TRABOCCARE, Intende precipitare, o cascare da alto a baflo, rompersi
la bocca; andar colla bocca per terra. E se bene il proprio significato di trabuc-
“care è quando mettendosi in un vaso maggior quantica di liquore, o d' altro, di
PS yche possa capire', casca dalla bocca del vaso quel, che vi e di più; onde per
figura si dice'un Trabocco di sangue, ec, tuttavia si piglia ancora in senso di calca-
te. Traboceo ne i vizzi, ec).
hie = ROMPE il guado. Apre \a strada, o il paffo. Ovid. de arte amandi,comandando
'ex che si rompa il guado per via di viglietto, dice: Cera vadum tenter, Guado vuol
s dir quel luogo ne i fiumi,per dove si può paflare senza navilio, che si dice guada-
ve; Eda questo guadare, o rompere il guado s' intende aprirsi il paflo in qual.
“Voglia occasione, o congiuntura.. Parrebbe che fletie meglio vado dal Latino
mis » siccome si dice ancora yolgarmente il porto di Yada, dal Lat. Wada VYo-
 taterrana; perch così Gi fuggi V equivoco di guado (pecie di tintara, mas
ivell quelli stitichi, i quali si vergognano, che la nostra lingua sia aiutata dalla sua+
frit madre Latina,non ci concorrerebbono, e darebbono una turbativa a chil' usaiic.
hist = MANDAK 4 Purafso. Par morire; E perché significa il medesimo che man-
aoe » o 4 Scio credo che derivi da i foccorsi maadati in diverse occafoni,
| “tempi ai detti tre Juoghi, da i quali non essendo tornato veruno di quelli, che
al —andarono, quando si vedeva mancare uno in paefe, si cominciafle a dire. Eel
stl e andato a Buda, a Scio,0 4 Patrafso; per intendere egli € andato in luogo, don-
de non tornera mai più, duc, unde negat redire quemquam; e s' intende egli è
i = Morto. Vedi sopra C. 5. itan. 13.
j TIRAR l'ainxolo. Vuol dir morire, dalle cunvulfioni della persona, che pa-
§&  tilcono quei, che si muoiono, Aixslo è (pecie di rete da pigliare uccelli. E la for-
2a, che fa ' uccellatore nel tirare l' aiuoio, o simil sorta di rete, e deferitta das
id Petro de Angelis da Barga in que' versi +
0! Tum vero innitens pedibus confurgit,& omnes
Intendens neruos magno trabit impete funem.
4 ZO feorge debito. 1.0 vede in pericolo di morte. '
STANZA XXVIII. STANZA XXIX,
1 ' Chinmgue è 'n Castelle allor pien di paura - Auitiene a lor ne pri, ne meno un' iota
% Corré per far © auanti et prit non vada, Com' ai fancinlli, quando per la via,
Fan la tura at rigagnol con la mota,







«RB memrtil vuol rispinger dalle mura,
' “\ Ch altri più la 2 arrampica non bada; El! acqua ne comincia a portar via,
| | itr db ouniare anco di gua proccura Che,mentr' affodan quixi ov'ellaé vota,
| Main fete Ini ghit ged farts la frrada, Essa distende altrone la corsia,
, E se riparan la, prt qua fracafsa,

| E a cogs intorno tanto il popol cresce,
C" ogni riparo innalido riesce. Tal ch' ella rompe,e a lor dispetto pa/sa,

«\ [Soldati di Baldoné superate tutte le difficuita, finalmente entrarono in Mal-
© mantile, éd il Poeta paragonando questa cacrata ad un' acqua corrente, che rom-
 pe, € paffa ogni oftacolo, che le 4 pari avanti, esprime I" inutil difela, che fan-
“no i Terrazzani.
ARRAMPLARE. E' jo ficflo che inarpicare detto poco sopra, ed è il Latino
Perreptare.
VN

™"



a











438  MiALLIMIAN TLE Bot

VN ita, Vn niente, detto sopra C. 1» stan. 18.)

RIG AG NOLO. Diminutivé di ome 5 Piccolo riva,
è proprio per intendere da parte più: bafla, che e nel 0
di Firenze per dove feotre l'acqua's che piove y efic 7
intende nel presente luogo 4 € ¢' aacenide comuacmente s che un
rigo, o rio diremmo rixolo o ra/celloy dewro così da Riuiceday la
presso alcuno antico. Se bene Dante nell' Inf. C, 1g.dices Bd
Sente rigagno, ec. ed intende quel fiatnieell@, o rivos il, 0
nali. Li Varchi Stor. Fior, libro 13. Commiciarono ad nscar fuara
e che i rigagnoli correuano, ele ve erat piene di motayedifarge rt
Nov. 16. 4 rigagnolo della qual via corre, chepare un fiumicclian |

MOT A, 'Lerra ben inzuppata acl? acqua. Ai Percariz, Lupums
 immora, Per intelligenza della \iuddetta comparazione e ince
i ragazai dell' 1afima piebe di ae 'sogliono per loro pa
dopo la pioggia (corre l'acqua per detti rigagaoh pigliate del
ond ae come un Danian opposte ai corso dell' non
paflaggio al fume, e questa chiamano la twra.; ma fiocome d'
quel iuogo sempre va crescendo,)così 0. per 10 pelo, rompe
bondanza traboccando la superay e pada via noa oltaace dri
v' appiichine, come dice il Poeta. Qunero nell' Aliads ib, a 5s,

De! Troiani fereci allagranturbay...
dt folgorante eApollo andanainnanze
Tenendo in mano il preziufo fondo:

Ei degls Achini il muro aterra Sefe;
Ne coffogli fatica, appunto.come
Lungo il mare il fanciulfacoll arena y
Che poich¢ fabbricato ha per. suo gsoco
Va gentil fanciullecsco alto lanoro;
Colle mani, e co' pie scherzando il guasta,

A lor dijpetto,, Contro alor voglia. Lat. ijs innitis, Il Boce, disse
Per di(peto. A Dante prima, e poi al Petrarca ia uecedlica della rl
il servirli della parola De/picto accordandosi in ciò, siccome ima
col dialetto. Provenzale, o Francelco. Virg, ecl. 2. Despectus tibi Jum ne
queris, Tu m' hai in dispetto,ne ti cale il sapere,chi io mi sia, Confiache
la strada, che è.per il mezzo della galera; onde que) groilo Canaone.
diceli Cannone di corsia, S' intende ancora per la correate dell' acqua..







FeSlFaer- aes

eet

it Se OR Peas aw





STANZA Xxx, opqas ae a
Già tutti son di sopr' alla muraglia, Celidora a due man 4
Che la circonda un lunge terrapieno; Che ne-anche un vi
Già si fiorifee in si crudel barcagha Tanti fil d'erba gol

Di fanguinacci la gran madre il feno: Lane' buomini così


4
NONO CANTARE:  439.
o- - STAN ZAXXEL oo. STANZA XXX.
ee, jth Amiffame —. — Adafa di Coccio a questo,e quel comand,
Da toccatori fan col brandispocco, Ed all'un dane,e aun'altronepromerte,
d h Lacompagnia del Furbainnanci mada,

“Pere che della morte almen Ceffane,

'Se non prigion si fa chi è da lor tocce, Che refti ai fianchia Batiston commette
AIP incontro ritrovafi Sperante; Com Pippoyil quale (Pa dal' altrabanda,
WA) + Che fa menando (a sua pata, il fiocco, Ma egli imretreguardia poi si mette,
Wh E se già le fuftanze ha difipace, E mentr'ognun favanza agloriasmtente
a Ei fiede a gambe larghe,¢ si fa vento.




Hor mand'a male gli buomini a palate,
+ Essendo già wtci i Soldati di Baldone faliti sopr' alla muraglia, e padati oclla
PS di dentro si mettono alla difefa, Sinarra la bravura di Celidora, di
y edi Amostante, s' accenna 1l valor di-Sperante, !a diligenza di Mafo
S eraccc pane wragtoe ut Coot cies A
La gran madre si se i fanguinacci 11 feno » Ci terra s'asperge di fanguc:
#88 Ounero nell” Lliade (petisind « =:
pm 8 di sangue la terra intrifa corre.
® La Gran madre per la Terra intese if Petrarca nel Trionfo della Morte.
elf SEG ORY O ciechs 5 if tanto affaticar che giova?
jeu 08 Tutti tornate alla gran madre antica 5
pone E'L nome vostro appena si ritrova.
  TOCCATOR?, Vedi sopra C. 2. tan. 60.¢ C. 6. stan. 44. 3
 “ BRANDISTOCCO.. Specie a' armein asta; simile alla picca, ma l'asta più
corta, ed i ferro più' largo y ¢'pily lungo, che non e quel della picca; e credo
venga dal Tedesco froch, che vuol dir battone, € brando che da' Pocti Eroici mo-
derat si prende per Iipada, e significhi Spada in sul bastone.. Stocco e dal Greco
Felechot Lat. Pipes, candex, da cui è facta anche la voce feecco,  perciocché pri-
ma per battersi si adoprarono le-mazze, e poi si venne a ferri;( Orazio Serm.
1.1. Sat, 3. Vaguibus © pugnis dem fuftibus, atque ita porro Pugnabant armis » que
'pelt fabricaverat nfus i nomi potleduti già dall'arme di legno, furono ereditati
'dalle arme di ferro, che a quelle succederono. Onde Stocco, che in Germanico è
baitone, a nOi significa /pada corta, e floccata ia ferita, che si da con quella. Brand
* jn Saflonico e riz one, o fuoco; onde Brandispoccbi poterouo essere cio che Virgi-
“tio lib. 7.¢ 11. chiaina /fipires, © /udes pranffas, ovvero obuftas cioè bastoni, 0

mazze appuntate col fuoco. 3
' CESSANTE. Si dice quel debitore, che essendo stato toccato da i toccatori

“può esser fatto prigione dopo le 24. hore da che è lato toccato, ( del quale ato
me rt e. (a 60. e C. 6. stan. 44.) ed il Poeta scherzando coll'

'Paclammo sopra ©.
egnivoco toccare, cide esser percoffo; dice che quello, che da costoro è tocco di-
viene almeno Cefante della moree, se non prigione, ed intende che quello, che da
costoro è ferito o muore; o resta vicino al morire, com” è proto ad andar in

Prigione colui che e tocco « ' <

FAR il focco, Fioccare vuol dir quando nevica gagliardamente, € da questo
diciamo fare il fioceo per esprimere un' abbondanza di che che Ga, per elempio si
fa ii fioece delli uccelli, o de' pesci, o de' denari, ec. si direbbe a uno, che pigliaf.

se molti uccellt, molei pesci, o molti danari, ¢¢. & così nel preteate luogo inten-
de

%





a










440 MALMANTILBE

de che Sperante ammazzafle molti huomini con
il vello della lana Lat. foccus., Si trae anche come's' ¢detto
ve, che Marziale appella tacitarum vellera aquarum, La
in abbondanza, si dice Fioceare; e stendefi anche r
aver dewo di Mcnelao: Poco dicea, ma bene, viene a dire d'
Atandaua fuor diluvi di parole 5 '
Come allor che di verno ilnembo fiocca y.
E fu pe' monti nena a! ogn' intarnos dohlgioee
MANDAR male a palate. Vuol dire mandar male il fay
gamente ed inconsideratamente. E qui ii Rocta ia Spe
vendo havuto per costume di mandar male ii tuo a 0
l'antica ulanza di mandar male a palate ancora gli huomini 5 ¢d
con quella sua pala, concia male moltihuomini, '
A chine dd, ¢achi ne promette. Diciamo così d'uno insolente
che tutto il giorno facia risse, perquotendo quand' uno .< quand"!
con questo dettato il Poeta deferive la,natura. di Malo di Coccio, il
s' e detto sopra al suo luogo ) era huomo di conversazione,¢ nelie tel
ordi, ne 1 quali si trovava, foieva vOler (empre sopraftare gli aluri
¢ ca Cf farsi ubbidire con le grida, e tainolta con ie butie,
# gambe larghe. S' esprime con questo termine la commeaita, e (pe
ginc,con la quale uno ficce a pigharh riposo;(¢ si dimotira un pimuo ¢
sare, ed amico dell' ozio, e delja pigrizia ) che si dice: Stare iw Rane
C, 3. stan. 72, € C. 3. stan. 1, 60m s¢ mani in mano; Con ie mani in cintola, —
STANZA -XXXiLL STANZA AARIV,
Amostante alt incontro un nuoko eAarte.  Vedendo i Terrazzanigbe stannoin fa
Senbra fra tutti anants alia testata 5 Che il nimico ad S[padeye gioca
Lo segue Pao C orbi da una parte s Ler non far Mole 1H fab MALCON KG i
E aa quest aitra Egeno alta franceta, Ritsranfi, e non sengon più
Vengonsi in tanto a mescolar le carte Ma speron ben ( moftcanaoas
E vien /pade,ebaston per ogni armata, Denari,e coppe)indurghs a far p
Ectidam puche,e 4 gsmocar none leflo a) si
Vs perde ia figkra, e fa acl r¢sto. Speaiscon, che pario in
eile preicnu due otrave il Poeta dopo haver lodato per vaiorolo
seguicato dai Corbi, e da Egeno, icherza in sull' equivaco del giuoco 5 &
sucne rai as/corso dai proverbio « Vengonsi a mescolar le carte,( che
€ \¢ ne Locca, O se ne 1iceve, Come vedremo sotto C. 10, Ble
auibedue 1 campi vanno ( cioè s' adoprano ) /pade, e ha/tom, e che chi
che ( ive urta nelle picche ) perde /a figura (che € una di quelle carte, nell
Ji sono efhgiaui ques fantocci, che ne 1 giuochi di daia tono te carte,
cive perde la propria perlona,e fa del refto ( cioè muore ). £ Terr
in fors, C1U¢ hanno i lor punto in fiori, ( ed intende tanao ip
Bria ) vedende che 41 nimico ad /pade ( cioè adopra ic ipade). Per non,
+ maitom: ( cloe per non fare un monte di mori in iu 4 mattoni, e ¥
fui terreno.) ff r#tir ano da chore ( cide lasciano J' ardire,) me tengon
Vuoi dike HU VoOguon più giuocare y ed intends non vogiion pil
































gs gp emer Ee ae PP ee EsP soo eee eee FEE




« NONO CANTARE;

ano di ridurgli a far partite, cioè accordarli, mostrandogli



44t
i ddwari', e coppe, cine

 ofterendo loro dell oro: E pee questo mandano al Campo un' Ambasciadore,

che parld nella maniera che se
 STANZA XxXxy.

 Spida Signori ? armi ognun sospenda,

Ache far questa guerra aspra,e mortalel
Fermi per grazia; più non si contenda,
Per c! alsrimenti vi farete male.
Fate che la cagion aimen s' intenda,

| Ca cherichedi a questo mo.non vale

F

ni

it

ee

| Bchi pretende venga con le buone,
Che dara glifard soddisfarione,

'ntiremo nelle seguenti ottave.

“STANZA XXXVI.

Con queiyche dona per amor non s' nf4
4n tal modo ta forza,e la rapina,
Chiedere,imperciocch? giammai ricufa
Ui ginfto, ed st douer la mia Regina,
No entraron mai moschein bocca chinfa,
E con chi tace qua non s indonina ?
Poss' egli accomodarla con danayi?
Dungue parlace, e vengafi ai ripari,

 L Ambaiciadore de 1 Terrazani espone la sua amba(ciata, e chiedendo tregua,
-elolpenfione d' armi conchiyde che la Regina di Malmanule e pronta a dar loro
fodistazione, pero domandino, che faranno efauditi.
|. SPiDA, Questa è una parola usata da j ragazzi ne i loro giuochi fanciul-
ye non hay ( ch'io sappia ) significato nefluno universalmente, ma nel
modo, che se ne servono i ragazz: signitica sospenfione di giuoco, o permuffione
@ eleacarsi per alquanto da efio senza pregiudizio, appunto come si fa con la fo-
spenfione d' armi in occasione di distide o particolari, o generali, ond' io crede-
rei che G potefle dire, che questa voce /pida futle corrotta da ssida, o disfida, I
- Fagazai si servono di queita voce così, per esempio. Wel giuoco de' birri, e ladri
detto sopra C, 2. flaa. 32, quand' uno occa bomba o per qualche sua faccendas
on attenente al giuoco, vuol partire,per afficurarsi dal' esser catturato dice;
Spida, E con queita parola s'intende per lui fatta sospenfione di giuoco: E quan-
do il ragazzo, che & Ggnore del giuoco dice Spida s' intende sospenfione generale.
Ed il Poeta » che si ricorda che egli scrive una Novella per i fanciulli s' accomo-
daa i termini da loro praticati,ed intesi, facendo servirsi a questo Ambasciadore
della voce Spida per farsi intendere che vorrebbe sospenfion d? armi.

Cae hericheli + Chetamente; occultamente, senza parlare. Varchi St. Fior.
lib, 15. Per Ze case si facenano delle ragunate a chetichelli.

WON vale. Questo pure e termine fanciullesco, se ben talvolta usato anche
dagli huomini d' eta, e significa Non è dovere, Non conuiene, Non sta bene,
ec. Preso per avvenitura dal giuoco, in cui chi scommette dice per esempio; Va-
dedi tanto ? E quegli che non accetta dice: Non vale, cioè non fo buona questa
erate 90 pure quando si fa contra le leggi del giuoco, si dice similmentes

NON entraron mai mosche in bocea chiufa, Chi non chiede, non confeguisce;
chi non parla non. inteso. Lo Stefonio nella sua Gnoccheide ato primo sce-
a prima dice,

hee pak Vulneris alcofti nunquam medicina paratur,

£ viene a fonar lo stesso che con chi tace, qua non ' indoxina, Plauto nel Pseu-
dolo Att. 1, se. x. ove introduce lo [chiavo, che così parla al suo giovane Padro-
Ae Innamorato,

Kkk Si






PrViot ovoy



E poi conchiiile:
ina fuggire i litigi.

dice Così

“STANZA AdXxVIL
A quel sl General,c' ha un pod' ingegno
Kusene il colpo,e in dietro si discofta
Che si fer mina i suoi, aipoi fa segno,
' Pala parola, e manda gente 4 posta,
Ne bado molto a fargli har a segno,
Chela materia si trove disposta;
Crascun a! amie le parci ferte faldo,
pC? ognun cerca fuggire il ranno caido.
STANZA XxxXvVill,
ch della pelle ha punto panto cara y'
ch che von vorrebbe esser nccifo
'empre de feiarre di fuggir procexra s
~ BYe mai c entra, ha caro esser ditsfo,
' Ben ch} ei. mostré non, baner pasra
S? in quel Cimento lo guardate in vif
Lisciato lo vedrete d' un belletto
* Composto di giuncate, e di brodetto,




* Ordiaa i) Generale, che si fermi il combatteré, e trova i'Sol
dieatidivai » perché a ogauno piace il vivere; e sia wno'coraggiols
mai essere, al cimento poi non haura careftia di timiore. Fermato gue'
battere, Chi era ferito s' andò a far medicare e ah

PASS AR parola, & termine militare, che significa far fapete |
Capitano' per tucco l'esercico con dirlo a uno, che'lo dica a un?
vada seguitands fiaché lo sappia ogauno senza che si faccia
Qi. Gli aatichi Capicani fa

fiziali fubordinati wa
si conteneva l'ordine di cio, ch

'di yoo! ior cl icvar qiaao da ij i

aes ie Hic Sen aes

eee UM AN TY Le Mi

v hominum parsi vem
6 °)\Nees te rogandi, © eileen ye x
Nunc quoniam id fieri non poteft
| Me fubiget, ut ve rogitem'; Fjord
Eloquere ut quod ego ne/cio, id tet '
“PVOS S? egis actomviarla con danari. Ci è egli modod
trovVar rant denaro, che aggiuitj questa ca

*''Dungue parlare, Quest' ultimo verlo pat tolto: di oda g
1, ove Teti patia al'iuo Figliuolo addolarato;

Parla; sis Wb habs sit digi by beds, ue

Tener la'cofd Wath tua'beentt ascofa'y mins aA

eAiciocché tu ae thee as
TA



oe

sa Pees

Dewo uli







1h te ada,

0G ease
Bao i as
ade





SE FLSFFSRPSR oS Staetes

A
Sien: Projo brau,
Se mai vengono a
Crediare che elo fan
Perec' a rutei viene il bi
Ech ela palferebban
Se lo potefser far con tor
snithewsate i a quella opiniiie
Di veder Cuanro viner fa
STANZA XXWK
E questi che badauane ax
in Malmanty, 8 accorfe
Che que; none meflier
Pero si contetaron dell”
Gai tagle alcuno impi
Hitri rimette braccia,e ¢
Altri da capo a
Echi si fa uae ed

SEF

pest















as
. =.



ih
nye
aaa

- o = & wo Z.




f es haat 1803 ae Udsarntare Teffera. Amminiand'




eee -

NONOPQANTA RE

} #88
'Siliodtalico,, eee etn ee ff
pases temmalumeaee €.con ording,.0 de Da by eat re
ea - oa Leib se sittin

ST ROV Oar niaseria,disposta. Veove. prontezza d! ubbidire » perché cialcur

- inclinava a lasciare il combattere. Sante eT ae
\ \ EVGGARE it ranto valde  Buggire i pericoli,o le fatiche, ~
| HA care eferdinifoHa caro che-qualcuno entri di mezzo,.¢ impedi(ea i
tocombatreresiche queito vuoldire diwidere una quiftione. Lac, pugaam dir e.
elLilcio Lateadiamo tutte quelle mefture,, con le quali aicune»
sper parce-bellefi lisciang. ta faccia 5 che diciaino imbelietrarfe: decto [:con-
do aleuaisda wRerlerra. cio'. melmay fango. In Franaefe il (cio dicefi Farò, onde
ciog unbratwace 5 ¢44re ana farda, e wna fardaca, il che figuratameate>
- Sluergognare uno.con mato pungente in pubblico, che alccimenti dice si; dur /a
 Ceretata 5 E. dare una cenciata fudices, ccacta dal costume de' Ragazzi Fiorentini,
che il'di di yuezza Quarefima, quando ( per usare un loro idvotismo ) si (ega las
eal cioè viene ad.ctlere partita per mezzo quella Stagione di penitenga;
Peete ior abufo,ednfolenza batcono el vifo alla gente grotlolana, o fenipiice
pd al COntado.cenci intinti.nell'tnchiottro, o in altro fudiciuine.. Branco Saccheci
disse Dane ca Fare, e dare nna zapare, per offeadere coa marto. Vedi sopra, a,

ae Pilla 45.0: base wid Ca Te ya
jit «OM Ne ATA, Latte rapprefo,, e (errato in fogli¢ di farfara con giynchi,, e»
Gdecta ginncata, la.quale mescolaca con broderro, che e mincitra, fata d'
Wlovauidette liquide con brodo, o acqua,¢ agrelto, o fugo.di limone,,. farebbe
un color¢ fra ij, giallo 5,¢ il bianco, appunto come diventa ia faccia di coloro, che
& i da subito timore, sink, 1
ASN AD/ERL, Huomini fanguinarij: Da Mafnada, che vuol dire truppa.
l'di Soldatic: what, militum manus Ma per lo piii intendiamo compaguia di ajiaii-
at poeaid Aieada.
TIRARLA

fuori, Cio.cavar fuori la spada per combattere. Virg. vagina.

VOISN Gx



aetkbEiks =

TREES

RESEES

er aay;.
= SATEICVORE. Ecceffiva paura, e spavento. Dicefi folo dal frequente bat-
'eres che si fence dalla parte del cuore in uno, che habbia timore. Se bene af but-
ter del cuore e indizio ancora d' altre pastioni, che futte anno quivi lor, seggio;
“eae gran defio, congiunto colla speranza di vicino confeguimeato del defi.
rato bene, la quale pero dai timore, non è mai io tuto disgiunca,:
sualptelten'arctben 4 Jeegiert, Paciimente lascerebbono (tire dt far quella quittio.
'Re. ln un frammcnto di Storia, Fioreaciaa manoscritca, che dame oa tisfa di -
i ncarvi il principio G legge: 5, Gli difero un monte di villagia, ¢
ond 'ingiurie, ma.il Cattellano, che era di uci Soldat, che avg iano canioin
dt ight doula Cavalleria, se la palso di leggiers,¢ la Ciaaiogii gracchiare,
sgnattendeva.a star. deatro;.ed a i suoi Suldaci, che lo pregavag» a ulcire, e dare
vs, addosso.al nimico,, rispondeva; Lo noa vogity ultirs, percaé nog voglio cae
Se CEDER guurs [a vivere un poltrone. Con questo termine descriviamo 490, che
yuo brighe, ac faciche, o.pensiert, a¢ meno f yuule esporce.2 rifthi, o.pe~
SS Ree: ye oe



MSS ERE ES

ae
ar eet






















444 MALMANTILE

ricoli di sorta alcuna.. Il Ferrario seguitando il Salmafio nel |
le che la voce poltrone venga da Police trunco, dicendo che:
andare alla guerra si trova che si troncaffero a posta da lor
dito grosso; B dovea essere usata tanto questa furfanteria, ¢
tali il soprannome, e furono appellati Azurci secondo che
Cellino lib, 15. il che.volea dire poltreni; poiché Murcia pret
mava la Dea dell' oziofita, e della poltroneria, Origine et
non la credo vera, stimando che la voce polerone venga pill: sto da
poledro, ( come alcuni spiegano quel be/fie poltre di Dante Purg. )
Poltrone a.uno, che non vuole, o non può durar fatica, appu 0
dro, il quale non è ancora atto alla fatica. Ovvero da poltro, che
secondo 1 Landino sopra quel patio di Dante Inf. 24. che dice

Hor mai conuien che tu così ti spoirre,

Disse it maeftro; che feggendo in pinma

in fama non si vien, ne sotto coltre.

Donde poltroni gli huomini pigri  e dormiglioti, dice il

zione di questo patfo.

PREG Sk FS oye = oe

— meftiero da abborracciare, E' cosa da farsi consideratan t
caso,
LMPLAST R ARSI con le chiare, Medicarsi con le chiare d' uovo le ae
di sopra in questo C, stan. 4 A a Re
PARSI aar de' punti in sul cefs, Ricucired tagli, che ha nel vifo,: quale cae 9 pe
ma cefo, perché guatto da i tagli, non merita nome di faccia. Cefe o Fran a
se € parola nobile, che significa Capo, come alcuai vogliono, dal Gr. gi grps mH
nol e parola di dispregio, e significa vifaccio brutto. ae 'a
STANZA XXxxl. STANZA XXKX ui
Baldane in questo per la più sicura * Et essi andaron con la lor patente tp
Due gran Dottori atrattamentiinuia, Di poter dire ye fare, € alto ¢
Lun Fitfolan Branducci che proccura Lor camerata fa tra? a
D' haver se non po in Pifa,oin Paxia, Che gli seguia curioso per. =
<ilmeno in refettorio una lettura 5 Baldino Filippucei lor yy
ZL! altro è Meinforcon da Scarperia, Huom, che più tosto canta py
ChefeVbuom vine per mangiar vi ginro, Crescer volea come gli altri appa e
Ch' ei vuol campar mill anni del sicuro, 3 44a si pent),quand'a e
. STANZA XXXXIL. STANZA XXXXIV, 9 &
Calfandro Cala Cheleri fra tanto Son alti gli altri due fuor di mifar «
Del Duca allora il primo Segretaria Ond! ei nel me? o camm
“ 7° loro un discorso di quel tanto Refha aduggiato sv hed)
evan dire al lo aunerfario Ne men pro crescer pits
Cacciatof, Giosieiae: ar
Escorso turto if [uo vocabolario

Scriffe in manierayefeceun tale Spoglio y
Che mese un mar diCruscain mexico feglio,






NONO CANTARE: 445

PRES os | HROMOVE TH ANZ AX X "BV. lov
ella pure alor quiui's'inchina, Purche il nome confervi di Regina,
Dando a ciafennoi fut debiti riroli., Luando per t annenire altras' intitoli,
Econ essi ferme IL altra mattina Che questons le nieghin, chiede al mato.
| Mdiscorrere, e far patti, e capitoli, Wel resto por da loro il foglio bianco,

manda suoi Amba(ciadori a Bertinelia, i quali con efla fermarono di
flabilire i capitoli della pace per la matuna seguente, promettendo la medesima
| Bertinella d' acconfentire a tutto,pur che le retti il titolo di Regina.
DE gran Dettors, Dice due grandi, perché veramente erono ambedue di. sta~
a ce alta, ed un folo di essi era veramente Dottore, cioè Ficlolano Branducci,
ai che e Frdncesco Baldovini giovane dotto, e spiritofo; ma perché nel tempo, che
i fu composta la pretente Opera era assai difapplicato, pero lo motteggia, dicendo,
che egii proccura d' havere una lettura in un refettorio, se egli non la può otte-
 Berein Pifa; o in Pavia. Ma non voglio già io lasciar nelle menti di chilegge-
 fala presente Opera I imprefiione', che questo Baidovini fulie lettore da' Retet-
fod t0rj, € pero dico, che le (ue beile, ed erudite composizioni lo fecero conolcere»
infin in Parigi, dove essendu fate fenuite in diverse Accademie dall' Em. Sig.
ym Card, Chigi tino di la lo fece chiamare a Roma, e lo diede per Segr. all' Em. Sig.
» Cardinal Nini, la qual carica eghi esercito pi anni molto Jodevoimente; mas
kit Beceilitato dalla poca buona fanita, che godeva in quel clima, se ne tornd allas
| patria, dove efiendo stato prowvilto d' una Pieve, quivi se ne vive godendo mag-
b,@ Blor quiere, e miglior faluce, che non godeva a Roma. i
él MELN forcon da Scarperia, Pierfrance(co Mainardi grandissimo di statara, ma
G8 ware dottore. Questo per esser,si può dire,un colotio, ed in sul fiore della gio~
veotl thangiava ati,¢ però il Poeta dice, che se 1 mangiare fa campare, ¢gli
(Ill Per viver molto tempo. L'iperbole di mile anni (e bene & di numero determi-
'ge ato; si piglia per indeterminaco, e signitica lunghissimo tempo.
I * CASS ANDRO Cheieri, Cive il sig.\ Alessandro de' Cerchi Cavaliere, e Sena-
we tore Fiorentino Segretairo della Sereni(s. Granduchefla, e però ii Poeta lo fa pri-
mo Segretario del Duca. E perché veramente egli € un Gentilhuomo di gutto
"i isquisito, e d' una cloquenza aggiuftaciflima, dice, che con la direzione del Boc-
sil caccio (le cuj opere regolano la lingua Fiorentina per esser' egli il nostro Cicero-
Ne ) ¢feorrends il suo Vocabulario ( cive il Vocabolario della Crusca ) messe um mare
di crufea in mezzo fostio, e (cherzando l'Autore con l'equivoco di Crusca buccia.s
uv del grad, ee CRVSCA Accademia Fiorentina, intende, che questo'Caflandro se-
id 'ce un diflefo compotto di parole approvate dalla medesima Accademia della,
», 'Crufea, nella quale si fa proteifione di pariare, € scriver pulitamente la veras
“| lingua Fiorentina.
7 PER far un diffefo di quello, che doveano dire, Cioè per metter loro in scritto
I Iattruzione di come doveano'contenerai in trattar 'accordo,si come si faa tutti
gli Ambasciadori,e plenipotenziari, che G mandano da' Principi, Repubbliche ec,
 FAR to spoglio a! wn libro, Mercantilmente's' intende copiare le partitede' i de-
 bitori; e per altro s'intende quando si cavano da un libro quei concetti, tentenze,
'parole, delle quali ci voguamo servire in far qualche composizione.
POTER dire,¢ fare, e alto, e bao, Potcr negoziare, e conciudere a lor gu-
Os

d)
i


e,
flo', € velonta, the ih
dicono: Peni; j
patentee Bis

libero.

LALDINO Filippucci, Filippo Baldintcci d
e questo intende il Poeta dicendo Huomo', che canta'ben
¢reicera più, perché egli e duggiato da quei due huomini lunghi
e Meio, de' quali egli lo dice'parevte, non perché vera

eg ee ee

e accomodarsi alla rima. Queito¢
jamo detto sopra nel Proemro. ~
* LVYOGO

5 STANZA XXXXVIL
Eperché ore già finian del giorno
Siconfuled, che fulfe fatrafera s
, Percio tutti alle spanze fer ritorno
Com! un fatto digatti, fuor di Schiera,
I Cittadini Pavan @ ogn! intorno

* Welle radesfu i cantize alla fronciera, Che non si
Bicivcgh' ognun secondo il suo porere Gis teiehnzs Gene Bl
© 5 foreftieri in ala dia quartiere, Sti Mab spefa dicey men Wid

a ST AN-Z/A*REXRWVNA ome DAVIN
©"Del Principe a' Vgnan pot si domanaa, Poeperre ners

“\  perché la labarda anch' egls appoogs
* 'Staffer attorno a rivercar si manaa;





un facco, a quait
LA quarticre »
fied a ME i

ee



uae



BG

anggiaco, Vuol dir luogd, dove nonatt
Pinterposizione di muraglic } o d” altro, EY Gail doghile pian 00
tate, € con poco vigore, e i dicona auggiare; da Yggia » ombra,;
TENNE un mexro miglio di pace. 'Per mbitrar', che queni t
haveano le gambe lunghe, si servc di queste"iperbole'd? un imezzo mi
DA loro il fogito bianco, Apptova tutto quello'; che essi conchi
loro Jil foglio, bianco firmato di tua mano,acctocche vi ferivano lee
capitoli della pace, come più piacera loro, 'Che e lo stesso, chedit
in voi in tuto, e pertutto, In questo senso dific il Petrarca ». my

"Chi Lhabbia racceteato, e chil' alloggi; x
Etiendosi già fatta (era ciascuno sbandd, €d i Terrazami tts
sex dar' alloggio a | soldati di Baldone. Bertinelia iawn Pala x

¢d il Generale, 1 quali accctcarono Pinuito. Si'cered deiDuca per co

'ch' eli in Palazzo, dove-bnalmente egli venne dopo qualche di

© che non voleva parursi dalla iocanda, nella quale s' era accomodato..

COME un facco ds Gatts, Cr0e lenz' Ordine, o'regola 5 ma con!

~ tende, che ifoldau sbandarono, chi io qua y chivin Jay come

Gi dja! andare.
rova aliogyio, Dar







aan ta
a i



























os MBA

aa

: wa
STANZA XXEK
Grants a palarro Bi
In Amospame eC
E-wuol che (gli odj mai:
Stien seco 5 ma ciafe
» Puer' finalmence ne i preg

Se, es 8. SERS PES EL EETE RBPRERPS SR

S” era decniarovoue
Priaichiei n'wferfs
Nand per:













dort ~ 1
quarticre significara
ae Swan grote hk ey









em, Sista 30a sobre ipaaies
dA 12. epill. 33. quidem,
r Fer ime —m fed egoa egh, ur eee - Croe noo
wesmercnen gliteci croppe cirimonie. E appresso. Pall pot C. Ca~
» Hlorum ego vix attigi penulam; ramen remanferunt Dichia=;
e ferraiyo'o jinuitare uo. aitaseawate » © pregario a voler 'rima-
co noi. £ ta/ciarsi tirare pel ferrainole, e non accettarc l'inuito » € ari
Koa > '
CH! vs difagio, Quand' altri e inuitato a un conuito aed
teatro. datalcuno.y.per licenziarsi da chi lo tratticne ta full' ora del ¢o.

s te la-causa speria quale ei i parte, suol fernirsi di qu:flo ates
al eons (a, non dia aifaeio: cioè se 10 son caula, che egli (peade, aun e dovere 5
'difagio-col tarmi aspettare.
“ ee ~Andar a mangiar a casa d' altri senza (pendere...:
operat ferraiuolo, o ¢appa.s perché in vece di quello ia porcano sul-
i:Alabardieri + i quali in occasione 4' avere aire a tavola  s¢ ne, spa-
ae appoggiuala-aila parece 5 e perdo.con quest) decto intendiamo. Posare ra
ior (ad! aters5c.quivi mangiare, se bene Pe/are tl ferrainolo.s' ay
“4 'aucora'un giovane, che non ha provifione, ma serve in uo banco,, o 'in who ff.
2ibegravissy baitandogl d' edereimpiegato, e d” abuuart per poter goder€ col
oe
MWAMBRA locanda. Incendiamo reli Alberghi, o vero Offerie, che danno, das
 dOrmice a vforetticri.
SERA nce wiare. a era nome eed Havea eletto quel lyogo per' Abto
Fipotor, exis Wiens t
VOLLE mille Porei. Vole iacpdofiaith di citimonie y¢ lufinghe: ed. e io. neiio hc
'chevwererdetto: itopra'< Com fran Bche Janene, così dewto dal Latido vente c1oe
di corpo, e gi fl
“WCODAZZO\, Intende seguito di gente “dictto.« Warchi Stor, Fior, lib. I2.Faé
al Primt Cittadini eli fecero codazrodietro, accompagnandolo, eraccompagnandolo gaila
we ius Cufanl Palarrxo; comes' ei fufferil padrone di Firenze,



hat

Ltd fate

oh “WHSPANZA thy STANZA L...,
A cena (perché il giorne in questo loco dn cambio di guarir dell' appetica ~~

a: * Lblebbertvairra faccenda le brigate, Facenano un collo come nna. Giz est

; 8 arta cucinave intorno al foco ) Se vien frictate, og un Sana accinits,
wt Senses furia ds friteate', Che per aria chi puofe.la fearaffa;
od e nem ipresba si 5 ma duran poco, Si riduffero in brene a tal partita.,
3 \Che-uppena farte ellveran già ingoiate, C' ogms volta faceanoa rufa raffac,
a Presse gente a rauolaera molta', tn ultimo seguendo Bertinella. >»
gi sR, We" miangiawan dueye tre per wolta, L! andanano @ cauar.dela padella.
gf oWDelerivetarcena fatta'da' Bertinella a i Foreftieriy la. aleconfiflettga, in,
pt fritcate » mangiate con fa fiiria, che egli dice: paflo Reale, e cirimonie conue-
if se a una Regina di Malmantile.

iin fueria di fritrare, Beitvate in quantità; 3 Waa gran quantica di Fricta.
sopra C, 3. st. 50. EXIT:

eet














.

448 aan F EDR
PRITT APA SEE viv eda! factard WOVa bE
felid'pddella' asfoge ia aveortah,ielde mene a
125 appresso 'atirort baslerebe dine 5 petcheirgioy
sce Sen eal: as tra ng “
GIRAFF-A, 'Avimale quadeupede § ikqualess se bene
fidema,€ s citaaaiea Dencaaneg eine toy -havil €onouid
a'quello del' Cammello'ylegambe'dinanai abo i quelled
coda j ed è del colore meuctia®,- che q
i Latini lo dicono' Camelopardalis y cio' bela Yeheticne! I
'Pantera, Pannoil-coo comenine ewafnd inwndealiangas Lio eel
interpretare 5 che non' fifazialleroy" perchemmeareare | dial
cibo con gran'deiiderio';Latino-¥ehiare } 0) chesaliuagatiero ene
betas










a
















pet vedere donde, e quandowenivanolle Feiecace ena
refize'a tempo tuo fa menzione'ibPolizignd-nelie® pellance 5 » Gitiog



Scaligero' simil dit questo:
ail Esercitazionie 209. nutn 3s OVedice "hei Persiani Girmafa P. f
E Abts il BOM Gina Parse Omit » Ha" eH) bho mee
o STAPA accinite, Sravarateetito'y Teo} oprepataco sidal Laci
-didiatho stavalattento', <u'all' ordiné cones, tnleMtaro.chigmaw,

ho tifato i ahtivo's particolarmenté dation Villanty*s sempre in

spele fei ptovvedere'danatir: "Ora /peritintratctared! Origine softy
nendosi il danaro a fructo, la Corte' prititipale 9 siccome da"Greciy dalla
detta Capo scost-da nO1'fi ehiamo Capitalc; e Fondo! ancora, dai tei idere.y 6
la petunia data a intereties a:pensa' di fondo 5 e»pedere!, orpotieth
ta'; Che'perd:' nftra y come geactata dai danaro \y-the! ayprincipi
Greti Chiamarono Torr stioeParre, 1 Latiniyemes siqua nig
fu Ud Varrdne', e da Norio Marctlio Oticrpabome apiraies p
posito'; ff disse Ia forte' pquafipecinia capitales principal ndan
che'da questa pechaiirpolta 12/a%phincipio s hevenivd poirdngu

da' Holtri anticht Crvaren, voce che finulmentestrovatiun Gio «
la) éhé i Franzefi didero chewanee,'cioe rendita envratayda Chef, capo.Ora
cinire, che anche dillero, Cixamgare, e lo fleflo, che Provvedere ti
<cidé & chiefact, aflegnar fondi's*¢ ludghi da rischotere; foraire ye:





See. > ea ae PERFEPE RSET ERE RRR ES



rnito, nogeiLefto » sircensp
. oP OPP Dee oes ai
y uP Via Con firia, come si-fardellescara

atrornd Peiitrelehh Voce alle vdice usacar; enn Jaycredo

i rofto 'fied' per bi 'iaS* a la asad
Pi Tn ape. Si dice' ido sono più gente d' act
Gialcuno # affatina con preftezza's € (eti2"Ordine 5 O-regola dip
'egli pud'dic Shae Sad repair med, toa? inciutlese

i e da notare 'Poeta | ' i
Pin pane sopraveiene fiipro
fritearemifttvie? dalle macenier Unica feu






ve











|

!
!





Stanchi di mangiar, non sazz}

Finito

BPA ficsass-



STANZAL
'at anna
Tal musica fini po poi in quel fondo;
Ma perché dopo cena sl vin lauora
Facean parzie le 'ior del mondo y

| Fra' akre Bertinella, e Celidora
inganancieree per burla un bale tando,
 Eapooa

4 0. entrouni altra brigata
Tal che si fece poi veglia formata.

sien STANZA LIL

'Fano poi com' è  usanka
Moite candele intorno alla muraglia,

 Lesplendor delle quali in quella franca
E sale, e tanto, chelagente abbaglia,

+ he diffinte si vedeva in danza

bt meglio capriole intreccia, e taglia
Wannaccio in tanto [opr' alla spinetta 2
S' era mefioa xappar la Spagnoletta.

NONO CANTARE.

Z rel taano gestive insane discadess lnnetira nazioneda
orit quali dicono, che i Fiorentini fanno je frittate d'un' uova !'una per rilparmiare;
 & però dices che durano poco, e per questo ce ne vogliono molte pi + si che per
sta ragione non è vero, che si facciano sottili per risparmiare, essendo certo,
he tanto. 3¢ tanto unto si con/uma a far' una frittata d'un' uovo [olo,quan-
wm to a farne una-di sci; onde si viene a consumare cinque volte pill, perché unas
- fristata di sei uova faziera tre persone, e fet frittate d' un' uovo l'una.non sazic-
un' huomo folo. Si che non di fordidi, ma di ghiotti in questo partico-
potion esser tatiati i Fiorentini, che fanno ie frittate di poche uova l'una,
inché sieno più cotte, e più gustofe. Di questa verita si puo chiarire, chi non
erede, con fare a quattro persone due frittate di fei uova l'una, e vedrà, che
eranno fatica a finirle » come le finiranno ben presto quattr' altri, a'quait fa
dieno dicci anche di due uova l'una, purché ben cotte, e questi si ridurrando
a rufa raffa, ed a rubarle anche dalla padella, come facevano coioro di
tile, Raffa raffa & lo stesso, che il Latino rape, rape, dal Latino rapere,
 fifece rabare, e si poté ancora formare, rappare, come il Boccaccio in una sua
'manolcritta da fugam arripere, formd Arrapare, © dillero la fuga.
r « Leppare, voce della lingua furbesca puo venire di qui, o pil toflo da
vare, significando portar via con preftezza, La figura è la medesima, comes
Tose dice Prometter Roma, e toma, per avvcatura dallo Spag. tomar; quali;
E piglia, ch' 10 la fo già un, e tela dd. Tre agiole,¢ barugule. L. naga, varie,

mgé. Daa rufa è facto gure; scompigliare.



449
ei detrat-

STANZA LIID

Vn gobbo [no compagno wn tal delfino
C' alle borfe. più rofto, che nel mare
Tempesta induce; prefe un violino,
Che fonando parea pien di zanzare,
Intanto un ben dipinto mefolina
Si porge in mano a quei ch'ha dainitare,
Et Ygnanefe, al quale il balle tocca
Sciorina a Kertinella in fulle nocca,

STANZA LIV.

2' grave il colpo,¢ gingne in modo tale,
Che quanto piglia tanta pelle sbuccia:
La Danna, bench fentasi far male
Senx' alterarsi in burla se la fugcia,
No vol parer ma infel'ha poi per male,
E dice l' orazion della bertuccia
Sorride, ma nel fin par che riesca
tn un rider più tosto alla Tedesca.

» che ebbero di cenare i Conuitati cominciarono a ballare così in burla,

Ma crescendo il popolo riusci poi veglia formata. Così per lo più segue fra lay

dalla quale nel tempo di Carnevale, dopo le cene solite farsi

x i, si da ne i suoni, e cominciano a ballare fra di loro pa-

Ren, e fenvefi da chi patia per le Se e da i viciui vi concorre altro Boge
it: 1 e















ayo MALMANTILE

e si fa vera veglia di ballo, come segui fra questi connitati
quali essendo toccato a fare da mai 'del batto alla meffola
egli inuité Bertinella, perquotendola co! meffolino
che le sbuccid le nocca, di ché la donni's'adirò,se bea non ta
ballo alla mefola si costuma in queste veglie per introdu
lo, che è eletto Maeftro rocca con que! meftolino le mania
vita al ballo, e poi tocca le mani ad alcrertanti huomini, ¢q
vitate vanno a ballare, e nel ballare il Maeltro da il me!
ella va con esso a toccare tanti huomini, e tante donne, € così
tri usano questo ballo con fare, che il Maeftro tocchi ante:
lato che hanno alquanto fra di loro, vanno senza meftola a
mini come e solito, e si seguita senza adoprar più la'mefola',” Q
si dice batlo alla meffola, si ta anche colla pezzuola, o Oy
lando si getta a quello, che si vuole inuitare, e così di mano in
chiamato Ballo alla pexruola, 6
ST ANCH di mangiare, non faxxj. Stanchi dal? affaticarsi a maflicar pi
ma non già fatolli, perché havevano mangiato poca roba. Ll Petrarca nel T
fo d' Amore, nel principio:; ne
Sranco già di mirar, non fagio ancora,
Giuvenale Sat. 4. ragionando di Meffalina moglie di Claudio
Et laffata viris, nondum fatiata receffit.
TAL mifura fini po poi in quel fondo, Alla fine delle fini tal' opet
nd: Pur una volta fini. Latino ad extremum, tandem, aliquando,
C, 4. st. 9. in questo C, st. 1, alla voce Bordello, € sotto C. 10,
ne po pot, ec, Vedi sopra C. 2, st. 73.

sR SERS TSE RPESEES

=
=

Ha SPR a=









a W

iL vin laxora, 1\ vino opera,fa la sua operazione con dar” alla teflaye '
briacare. Del suo lavoro, € della sua operazione si può dire quel che difie} ka
delle pecchie. Ferner opus. i ty

B ALLO tondo, Specie di ballo, che si fa, pigliando più persone per! »
¢ formando così di tutti loro un circolo, ch' è forse Latino Choreas m
nostri Toscani detto Carolare. ee Ye

VEGLIA formata. Veglia vera, e folenne con tutte 'le formalita, i 4
Vedi sopra C. 2, st, 46. dove teoverai Jutrecciare, e tagliar capriole, & ie 4
st A

23. q
Nunn acco. Questo fu un tale nominato Giovanni, € si diceva
cio per la sua (ciattezza, e spensicrataggine [ poicht fo nome &
del vero nome Giovanni; sopra il qual nome è da vedi tole
della' Casa ]; Questo insegnava fonare la chitarra 4/ed if
pochissimo come quello, che non haveva cognizione cna della
rd dice epee 4a spagnoletta ( specie di danza ) aflomighando il
cato delle dita in fu lo Arumento, a uno, che zappi: e Spinerra
balo, o Bonaccordo,,
VN gobbo. Intende il gobbo Trafedi, il quale faceva p

violino, ma fonava assai male, e per questo iI Poeta dice: ch
@i xanxare, aflomigliando il fonar di lui al ronzare delle















d NONO CANTARE,
'It! mipiccoli alati, co acutidimo pungiglione, Questo Gobbo servl alla Sere-
oleemmt aioe. quaita di Nano, e per le sue facete manicre piacque
" salia Serentis, Arciduchelia Anna d' Auftria, chg o condufle con se, quando an-
do dove entro tanto in grazia al Serenils, Arciduca Ferdinando Car-
Jodi lei marico, che  arricchi non folo con li suoi gro fipendj, € molto piit
con I regaii', ma ancora con 4 denari, che questo generoso Principe si lasciava.
da efio nel Bins delle mane » nel quale il Trafedi era aftutidimo, e face-
 'Ya grosse,potte, perché fapeva, che perdendo S, A, S.non voleva eller pagata,
lige se vinceva era pagato puuwwalmente. E per questo il Poeta dice, che ip un di
Wh quei Delfini, che'predicono rempesta alse borfe, come vogliono; che il pelce Delfino
ica la tempelta nel Mare, e perché questo pesce pare, che sia gobbo, però
i ) per coltyine chiamar Lojfini, + gobbi, Mori poi questo Trafedi, e la-
jit scid mece.ie sue faculta a una donna di camera della Serenifs. Arciducheffa, della
Co qual donna haveva tatco scmpre¢ da innamorato, con patto, che si maritafle con
un Fiorentino suo amuco, che era in Insprug, come segui.
1 MESTOLINO, Cucchiaio di jegno per uso di cucina: Diminutivo di 4zefo-
#4, la quale in Lombardia chiamano 44¢/cosa, dal mescolare,
Ada inuitare. k4a da chiamare ai bailo, '
—— SCWWRINA, Chog batte gagliardamente, Il proprio di sciorinare & quando si
get ort > abit: di paano fuori delle caffe ne i tempi di State, e si disten-
 dono per targlt pigliac aria, batcendogii con (curisci,( che dichiamo camari dal
pot Greev camaces) donde feamarare si dice questo battere, per cavargli la poluere,
st © Per liberacgis dalle cigauole - E da queito scamatare, o perquotere j panni, ec.
igel Pighamo il verbo sciorinaré per perquotere, E sciorinarf? intendiamo uno, che per
 A gran caldo Gi leyi gli abiti daddotia; Dal Latino ara detta poi ora coll' o lar-
f £9, quale Gi fence, quando.ia plebe de' ragazzi con sua antica canzone grida al-
sath Ie matchere u carnovale efiora Ter, in Adelph, Accipiunds, © muffitanda in iyria
adalescentium est. L' huomo se ladeve fucciare. Quivi Donato, Adafitare enim,
pe
4

4














Proprit'ef? difimulandi canfatacere. E Sopra. eHufficanda; Patienda, consideranda
cum filentio, Gc, e dal sao diminutivo non usato orina, cioè auretra, ne riufei il
verbo Sciorinarsi, che e lo stelio, che se dicetle,con Latino barbaro, e ridico-
fo exawrinare. Netia Valdiaicvole dicono; scfobacare quando exopacare, cavares
i day'. opaco,;
IN buria se la fuccia, La comporta come fatta in ischerzo; dal fucciare-, che
"| si fa, quando G feate grave dolore; tirando a se il fiato,
| NeMivuel parere, mat' ha poi per male. Non vorrebbe, ch' e' si conosceffe;
mane ha veramente havuato diigulto. Virg. premit alcum corde dolorem,
DICE Porazione della bertuccia, Dice de] male borbottando, o brottolando
sotto voce, e così facendo con la bocca quei getti, che fa la bertnceia, o scimmia,
“quando@in rabbia, che pare, che elJa borbouti, e discorra dentxo a i denti; che
-diciamo comunemente, che ella dica orazioni. —;
| RISO alla Tedesca, Rifus fardonicus. Kifo finto,¢ che par più tosto pianto.
In lingua Tedesca ridere si dice Jache; ond' io credo, che il noflro Autore, che
“haveva qualche cognizione di quella lingua per essere stato alquanto tempo ia Ia-
sprug » habbia detto ri/o alla Tedesca ° non perché Bertinella ridetie, come fanno
12 i Te-

















st¢
pero pla
argla
Fase bine ) Che fiand' similt 1
meazione.
STANZA' ae ai
Al Det veramebie pare Bratig? 2G Ya beffii ?
vse "babii bya A onde or Bhreded 00 ee arvlnesy z

Perch gii par a' haverle dato prano, Ci morde in qualche part
ernei d haverla tocca a malo envoy “" Ech) se
Ma quando fanguinar vedde la mano,” “ts ee
Io mi difdico, disse, e me ne penta y?\ "Faia
Finalmente to ho tl diauol nelle braccia, uel mespolino
E [ono,¢ faro sempre una bestiacci@ ha vette!

STANZA LVR?
Wer carargliene pena, è 'Biri
nonfacome,al paren h
Dror



Sl e WSRoERRER
















ae
Rin arap in Canberit ih fablerro 2 « © 1009 Syaadermapuoraee:
2” aaanai più TPinig ane Se raceme ie Casaliadonma,

: STANZA LVIL: STANZA Lk
He Principe'a quel oriad ) Wigule? emairep 'LLG ridsa\ Dortma ator come'
3 'a foggiradrd ¥2 Wictrtdto here', 192% «Bldipolaize mance,

= call tro 'du hhh VeAbite dhe JOU 2 « 28 IRE feowe/l ah aldorne gal
Co amore in tui vuol far le sue vendette, —- OUR GNMEGC cated fareRiifwe

Ui quel vive fhiattin combean picebio', "0 « Ds iene yx

erkriwet:



ess

“CG abiriih aiiplerofkdW maiekirE\2) 6 OG RLU enareeinura pei

IL mefolina 5 © quei, che glienc dette Di non mostrar in ranto 8
«NB per BP laa bdr qa Ol 19 LY pPERRE OS Gece vel wsuforat medics
SO Po igeres ia terrain Cente rnild pores |. 129 UD anguenre che teyfan



“0: Bj doe 'G mara vighias che ta Donna*faccit st gran/laniento pparendyy
Osporer haverle® ad maida (anpucuccortifi,, che ib male
2 Se G7 QUEP EY pli HOU eredcvaYy-ripreh de se Reffo |) 2 si metre ivo!
WY CON Medica Me HE biedtlediAratite si feuop! namiorato 0
OCP ARS, enande torino.' Rifentirts'¢ dolerf tanta) 1c! ol amie le
>. Ona ABD Read, “ANfatiday AppehalNon giipard' haverla quai
Sreata ye da Stevicate; e Srenravee dal Liarino fettentarey come owii ¢
z ati Cie, Si adPAtcic, ALE wir msiferdque fuftento..10 MeHor} cioè paul
yea thiala pera mi Condued's @ ati'reggo'.° Non folametite dicta
Sich ea oial caeueanae ” | @ mala faricay
C'ferto'y Batinovint' ech ytenid} catkanten\ B fitcometi dies 464
Bebe » cio' grandissima. Ho auuta una buona malartiey
1
Ou s

ee ae on ane:

imal¥orza'; pochissimo,.»vsn wed wy





'

ere as
icp. divenfamnenne 4 ne

races ne aa Ow at






sor we! iy 51
nym ORG Shots ovo

vale CMOS en no Mle ghinibizzef.
a RS nme S'S on ioeaeemaneenenie gee





wh ib sen oh
+ CANom artes adenine
a “od medesimo in lode dellt,Vimor.malancolico,..,. A drow Seqeeay Bw '9 sConUL AE

611 » Bvan fuggendo ogns altra compagnia, ) 2 ASWA bE
aaa SOL op Ae Cd ghizibizein 98 concerti y e 4 CARTE 5 96 sui grktis WY
yo viens ©, Lheecompagman pur sempre.vada.. a8 iA sce s\c08 DAN
j Story Bior-liba.t5e.dige < acca, LAA AER, Seonpes ghicibiczands
Plow dy Dihoeae'D sua | ee
@bArvcca il-ticchio,.. Giiwien, questa volonts ».pen eG @ gape o afoul dal
« Branzcle,%ix.,. mosca caninay;,Sumili, ma Aatnabate Penie, bvalilla.s.¢ Al
adaliailillo.»che ¢,una-molca pungentitima » che infelta 4 i da noi ia
i coenaadad pacerba faransy quo tora aermorndeyert peta NB
mda S 4 APS AS
ae Relacanantciocanstiye! dolore. che; prova uo pazeiente,, quap-
See una fericafirmettelale, accto, o,altra.cosa, simile.4.che, moktihica, e>
Corrodede. partielle de'iquali carpi acri, e mordags fembragg..al, ate a
Buila difrecoje feriicanms ©PRAZRHO. —.sisshwo9 90. 9\ml)o 5
RAR un tira 4 an Suniendettar uo mal Ceri: c0 a che a iactia a
UNOswiaw A oie ar re we-sfhow vow wh srsivsusily stls,, in
h\STAACCLA comme 4 pleebiawsE grand a.collera Bg i
' sofehiacciare Ggnificabatiene identi per la collera, —- per a ae; ed ha
piquetto: Gignificato fenz! aggivagerus come vom picrhio ma,tal Gmilitnding s ay
questo. uccelioiha propriesa: naturale dij batter, sceau cere
rofted.in fu sramiideg|t aibert per; fueginsdefarmu e sliggual ce
concbellittima giz »che;¢ queiasiMope haysr, molgo., eet 2
'¢ ville uscir le formiche si diflende some morte sopra, quel amo,» €, Ca'
ladingua g, che éJunga 5 ¢.carnola.,¢ quella.distends opera il, medesimo a an 2c
ose formighe, vi vanao sopra.per.palcerti, e quando.al Picchio, pare di haveruenes
——— abaftanea), ura ate taolinguayeddngoia, aDa.quest '0 uccelio deco in
» Gea Oryscalaptes » 0198: Pictinatere di quencery € InnLaty pics li.¢.formaso,probabil-
ovamente il, verbo Picchiare,cioè. batierese. chi batted demtlperila stizea,paresche face
lorfiedle romore,ca.tdenuyche fa ak prcchio cal becco », Plasto spel, pro-
Seeremersaniice srond Sai S108 OH,. ai )
= MANDA git Trinigante,eAacomisto,. Bestepymia >maledice tua tal Be,

WAKA



=

adil SeEEELEe dpe

















a

454 MALAEAN TILE Oe
¢ suoi falfi Profeti ¥ pn eee
colle maladizioni, coprecesteats e bestemmie oe
GV AIRE, RawmaricarS, eoeee aie: i's
gagnolare. Vedi sopra-C, 4. stan. vventura da wagire ee
guaina; perché i cani quando ne ae tocche,fanno um mug
gito de' bambitii'. 'Si può anche dire, che venga da #1 ase i
rammaricarff dell' huomo. 1 wales Now, 2 bn R
comincia # ffridere, e Puaire., a he wl
METTE 4 foqquadro, Solleva, e mette axofgr tutti i vi
re, Soqquadro & voce usata dat muratori'y eee Ȣ simili, e v1
squadro, che e quando per accidente d*
mancamento un pelo tirato, o strafeiaatonon può fare” ib suo corlo,
rd cagiona, che git steomenti del veicolo, o treno facciano si ito at
per lo sforzo, ed affaticamento yche riceyono, eda Yale
drare, e mettere a fogquadro iv vecedi Rordirecobromorey) &
/MBLETOLIRE, Commuoyerti } Intensrire « Vedi sopra C.
tini pure in vece di /anguere, dicevano'volzarmente ne! sane
eficr cenero., e moscio, pigliando la similitudine das real ¢
signitica erbageio 20 ortaggio; Auguito Imperadore formé una 5
rola, e dilie Serizare pigiiando ia similudine dalle bietule, ~per vi
languids'; non iftar bene. Vedi Suetonio'nella Vita d Augulto » Ove:
voci,¢ maniere particolari, che questo Principe ulavaynel. par
Celio Rodigino lib. 15. c10. Now similmente, diciamo! fauna
si, illanguidirsi per il aah d'amore, B Bretolone pre a hu
mii fatta;
BESTIA scimunita, Spend spropositato senza jmenlitnaiasiiya -
zio affatto. Lisca Nov. 2, dts perché. ellaera ponera, a queste se
torre senza dote, ec, Scimunito; sciacco, Scimunito'é lo stesso che wren
Lat. incaftigarns, Gr. acolafes, che not riceve'lammoniziani;) €
fictti, monitoribus aff ah E perché questi, o simili a loro fogiiono essere
ale il giovane deloritto da Orazio y Sublimis cupidusque,o amara reli
nix; E qual'é quei, che difvaol cio, che volle: come disse Dante nfs
ro nell' Fliade al terzo libto; Delle giowani genti rigogliofe Sempre per:
tere menti; cioè per dirla volgarmente hanno il ceruello sopra Jab:
@ che Scimwunito', che di sua natura yale Non ammonito, non riprefo 5
stigato, o che non vuol essere amimonito, ne riprefo, ne galtigato; ¢
rio, € mentecatti fanno; venga' a 'signiticare /eiocco, e haomo dt
to, L' esempio del Bocce. nel Filocolo lib, 4. dove: parlando come
Il tno diletto e dimorar ne! vani occhi delle foimunte femmine, pwd elle
voglia dire ancora licenziofe, immodette, intemperanti, e non
ze solamente,
RAGNATELO. Ragno, infetto noto, Dicono che perm
dej cane si pigiia del (uo pelo, e fiipone sopr' alla parte offela,
HO sopra C.6 stan. 6.¢ che il ragno, e 'o scorpione aumpa
foper a la piaga che hahao faita coi loro morfo,suaino il pazziene



Page

eFEEEES

 REEES














ea



Se Ss See Se



'*
NONO CANTARE: 455
necredendo chest pezzi delmeftolino, habbiano la stessa virtù; lega sopralia se-
rita, che ha fatta col meftolino a-Bertinella, idetti pezzi Maforle Baldone:, co-
me Soldato bravo » haveva notizia della jancia,con la quale Achille feci Telefo,
ee nea sehen havea detto J' Oracolo, i, Qua
. iabit medebirur, Donde Dante afer. C, 31, disse:, '
lo) loi Cosnod! toche foiena la lancia y

he 14 5 0D! Achille y¢ del fu padre esser cagione

tHe Prima di trista,e poi di buona mancia,
| -\Bierede; che il meftolino habbia la medesima virtù della detta lancia.

>

Buk

qt ALAN del Cielo » Quali che Adanna def Cielo, ¢s' intende orto rimedio per
at fanar male,»come fu ottimo rimedia per liberar.daila fame 11 popojo eleno
wiytt inane che. Dio git mando nel deferto.. diFirenzuola in lode del iegno fante
io, 3 > <oSy
se) sbiaib shoizwe2 S& uno'non mangia, s' un non si riposa,
lags i Osha il fegato guafto, ole budella,
Rab > Bgli è a man del Crelo a ogni cosa.

** Nota!che in:questo detto la parola:4¢an:non vuol dir mano, non, essendo pa-
Ola figurata'per apocope, ma nell*intera sua efieaza Adem, che così si trovan
scritta nelSacro Tefto quella, che Dio mando al (uo.Popolo (che noi poi chia~
jamO manna )¢tal man si dice nelia Sapienza al capo 16. che havetle ogni buon
x vien chiamata quivi Paze approncato, e appreftato dal Cielo fenya fatica
© pero iniqucito detto credo che fr debba intender e#Zanna y e non mano per si-
se uba cosa ottima in ogni gencre; e-che ciò sia vero y quando sopravvie-
he a*yao® qualcofa di suo gulto,suoi dire: #' wa manna,e non mano: e se uno
ricercaté | se per un su6 conuito una tal vivanda gli piacera prisponde farò Atan-
adScome si Vede 'fopra G, 8. stan. 43. Se bene potrebbe anche dirsi», che collas
feta parola Gi aljndetle a due signincati, e a quello che ora di sopra si è detto,
WMtan; cioè manna, e dian, cioè mano, E ALano de! cielo potrebbe parer det-
ta'Colla medcfima forma', con cui diciamo di qualche rimedio., o medicamenio
cfitace Kyi e (Paro la man di Dio, il che coceisponde a ciò»; che dice Piutarco
fOnumM Conuiuialiam lib. 4. quacit.1.)cheun certo Filone medico,aicuni me-
'Witdinenti Reali, così decti perché erano da Re ) enon da Poveri, o per essere
i*! fepreti di Ré jo per la loro eccellenza; e che dal (occor(o potente, che se ne ri-
; ceveva y erano-chiainati /exipbarmaca, appclld com-particolare.appellaziones
mani degl Idaij.
jd) WPREGLAT-A, e neva, Intrifa, sporcata, tinta, Da i venti, che portanan via le
i mmelecine Bal gran vento, che per le parti da baflo gli usciva dal corpo accom-
ip 'pagnato da qualche altra cosa; la-quale ricoprendo le'mele che sono quella par-
ce più eafnola delle:cosce, che forma il sedere } ” alconde alia vita © costin un,
w? Cert modo fe' porta via; Si che il Poeta Meoppiando quel verlo 5 che dice. «Dai
md 'venti, che Portanan via le vele, intende, che la Camicia di Baldone era tinta dallo
z)
6
è

RELILE wtuateale

'sterco ':
SQVADERNA fuori, Cava fuori de i calzoni,¢ la distende. Morg. Le chiap-
/quaderno con rinerenza, Dante Par. 33. Cio che per o uninerfo si /quaderna.

! tele, ciò che e sciolto, e s(parfo per l'universo, prendendo la fimulicudine da'

J libri sciolti, e squadernati. DR



















'436 MALMANTILE |

DIRGLI manco che meffere, ec. Dirgli iurie
tia dissero i Lat, ed il Lalli Bitar kon by eHloee Lich
è Teitt m' ha detco peggio che meffere. 6)
Molti dicono + Asessere él' asino: ond' io stimo:che dic
che meficre s' intenda, l'ingiurid più che se gli havefle
Comico Fiorentino nella Moglie Ato 4. sc, 10. in-derisione del t
dice: Si; Adefsere e  safine, che va nel mezzo. Quali dica:
quando paffa per le strade gli fa largo, eva nel mezzo,
BEL vedere, 1 bel di Roma ¥' intende it Colofico ycheinoi
Ciamo Culilco; eda questo per belmadere y Obel di Kama iptei
che Bertinella pericolava di mostrare alzando le gambe.
Bellofguardo, son nomi di juoghi, e ville nobilidime nel Fic
vato, e donde si scorge molto, e bel paefe.
eHEDICO da fucciole, Medico spropositato,e dipoca:scienza,
mo i marroni cotti col guscio nell' acqua, e preadono tal nome dal /ucciare 5§
fanno i ragazzi per trarne senza aprir wutto 1 gu(cio, la pasta, che vi & dent
E perché questo cibo e vilissimo; pero foros iamo da si i
nulla. I Latini dissero bomo manct cioè di niua pregio,
fico; per Naucum intendendo il Gufer, o buccia di quaifivoglia cof
la, che si bucta via, e non buona a aulla,
LE fa veder le luccicle, Le fa pianger per il dolore, Quando uno}
tale, che gli muova. le lagrime, pare al pazziente di veder per ari
14 di minutissime stelle, simili alle lucciole, il che e cagionato dall'

lagrime, e che pafiando sopra alle pupille offende, ed altera la virtù v vas
Oe STANZA LXL STANZA LXUL,

Non dimostra la taccia così mefta S' impiccherebbe, ma dall' altro:
Quel ragagxo scolar,quel cauczzmola y Ei va pos retinente,e¢

Allor che motti giorni e (ato fefta
E che finita poi quella vignuala,
Ji matadetto tempo ecco s' apprefta,
Ch' e's ha di nuouo atornar alla/quola,
We si gualta belando si la bocca
uuand il matftro col bajion to chiecca, Gli vada in (u le forche
STANZA LXiL STANZA Lxl
Qrante cambiate in vifo,¢ mal contento, Poiche '1 cundotto delle pat
Adefo pare il pouero Baldone, S' ha da ferrar(dic' egli,
\ © ha nna stizna,ch'ei si rode drento, Perché si ia leva alle sue.
Per non bauer ceruel, ne discrizione,
Che benc' altrni la morte dia [pauento,
Se e' non fuffe che e'c' e condenvariong
Achis ammarza pena della vita,
Con una fune baurebbela finita,

Con quella mane' alei dif






un mutmameess ofa. sealers a2 ESEG~0FE

oe Se




pas sro LIDUE



8 ois! me nyo 34 fan ne'iipauni,
Ps em wre - dif oe vs onngoy eRereheymensre th' I ami, ella v anuifea
'chap yh Ob bu) Chomsany, u lite
a ye fooppia dali th Gi Sent habbia on acgquauite.
intovaci) Poetara Sabeshens cepaeha cheba baleen
ps9 sorim re eran moped Da quefio a srcorgentof Bering
di.lei 3-4 h



a iste ine goticiod si sbaseni tmodh 1b i od LE": sy sais: Nas. aoe}
ianbopkematendonstetniens faney Olaltea sorta di)logame, con
eaeioees ed<altre:bestiefimili + Evcaves.t; filidice ancora,

fa merce) collooa: malfastori pquandog)' i =
Oy: 6. Saatgo! aah Eda emadl noiidiciamora un rapazzo-malig



 lioiywenn\ LiVai facendorparlare tn Pedantesdice 2a (09 11309 isos to att?
doped d jwoda, Hee, Seana & osjuy 1k1gGs esm2) 901813 Iq issagayi Onnst
ab omtetharion Seieababtanitioher ants omiiiiv s odio olaup dioieg ¥
of Mey aba so, O folerro-vrifar ciferory PO Over owsd Orstid imaed bellow
OR iiteade 4a sesete st O, eyed |i ccaaiate marae) 19g e Colt
' A quella id nines ehocien en quel-pdatemps,
itp aeaast iid? eavpolonannavOaiee:
ivcen se sar bebe vignnola', l'ela dnrage pes inten:
credo che sia pata He Tulnadaeeibe deseo opera

mad C.9, nf eke an la) Petite Tn: Eictirhnadutemne 'a a' Buohs

cL aa






uh fa già'an val Sse da 'Panzano, ib quale havendo din' fola pic
ue co) Poe facevz a ex faye barilittivino; ede ae,
on rei hi achiorbailt, ed haVeva 2*OB Hi Torte frattesyichie tro -
7 AO' al non eglnop it si meow rubsiidoltuva,

mee ore Ha" mvae e sempre dieeva";' 'Chie'ratdoplievaer bani bofa
nefla; rem POscoPLe®, thie per fuor bifog ni" tai vende iadetenvigng yes
fre do pia Fitoperta della derta'vight', HoH potevarabare 'come
ee: or Salis fo S*aFri(chiava a imbdttare ance witio's pet fo” che
dom abdate alii fo? amici 'da CHe*procedeva' phe eel" err 'vino 5
edali Prisporideva, che era fina la vignuota ae a teiccoe dice: il
può eifer che” venga it i decacos ee ae a) “wip reheas
ov mang an =q oon

rig Pgh Syeepoda balie Roca Pane ©.°6. fan.
nares @¢ fo stessd', titi duc verb} stei'dal Tao's IP BalerNov, 7,
ane A ractomandana ' a phi poteres; e coloré antendtnand a vbinctirts 'chi di

pinseebed ls sha olehnon aol,
te wna fine baurebbela fin ita: Havrebbe fiaito ee 'fub" CRN aziO' ton-im-

TANTO, o quanto, Termine, che significa piccola quantità, ed 2 lo feffo.
che par un poco; alquanto, Petrarca. E tu, se taxto, o quanto.d' Amor senti,

4 un sopractieni', Parca waa folpenfione, un preety di foptatrenere;
' 'Profiiagato il termine. Mm coNn-

bie —















453 MALMANTILE >

CONDOTTO delle pappardelle\, Cioè la cannaidella gola,
del cibo detto da' Greci Ocfophages, e da noi scherzosamente él condotto de! ho
che risponde alla parola Greea significante il porta cibo, o i} Port i
piglia pappardelle, che sono lasagne corte nel brodo di carne pet ogni cibo,
ti chiamano pappardelle la ricotta stempcrata con acqua rola), eu ova 5 a
¢€ poi fritta a toggia di frittelle.

TLR AR le quoia, Signitica morire, come dicemmo LC. 4. 20,
scherza, moitrando, che pet la legge del Taglione si gattigar le gu
( civé la pelle ) dei Duca per haver egli commeffo un delitto nel
nella, rompendogli quella della mano, € seguita lo (cherzo dicendo, «
morire in /« tre degus  [ che vuol dire in sule forche ] perché con un
col meftolino J fece la decta ferita nella mano di Bertinella; e di pil
Ballerino a vento (che vuol dire ballerino da qulla ) per mostrare:
egli commefio ' errore bailando, farebbe gaftigaco con esser fatto mori
do, come pare che muoia colui che e impiceato, Vedi sopra\C, 2, st
re un ballo in campo aczurro; che e lo stesso, che Tirar de' calci a Ronaio,
vento Borea, o Tramontano, Quel che sopra dice: in /u tre legni per i

forche; è simile a quel di Plauto, che volemdo intender Far, cioè ladro 5 difles
trinm literarum homo, vel
FACENDO il Nanni. Facendo il goffo. Fingendo di non badare, oofferm
re,Vedi sopra C, 4. stan, 26. Mostrando di non s'accorger di quel che faceva Bal-
done, facendo le viste di non vedere. *
SCOPPLA dalle rifa, Ride secgolatamente. Vedi C. 3, stan. 66, alla yo
Pimmei, e C, 7. stan. 66. 0S

























&
Been SSB SRS Ewe ow o=

PER l'ablegrezza non puo fear nei panni. Si rallegra geandemente.
capir nella pelle. Per il gran gusto Gi rallegra tanto, che non trova qui
di sopra C, 2. @aa, 69, Piatone acl Carmide, poco dopo al principio, volend
esprimere una gran paiione di piacere, e di gioia fa dire a Socrate, &
più in me flefo. i o ae

cANDARE in fumo ad acquauite, Risolucre in nvila. Suanire. Lat, 4
re. Sidice anche in tu:no d' elisire, Od' eferuite, sopra C, 3. han. 52.

STANZA LXVL STANZA LXVIL-
Atentre Baldon qual fempluerto uccello, 4a ridan pure, e faccian ci
Coast d! tntorno alla cinetta armeegia Per ch' ci vuol far orecehieds Merci,





Lo burtino te genti, Amor ta,
C” ad ogni mo farò fido,¢
Come talor 3! abbrucia ico
“i garto al fuoco, e frau

ed tuti guint serve per zimbello,
Senzache mai vi badi, o fen' annecgia
Ogun lo burla, e dice; Pelle vello;
Crafexn dice la sua,ciascun motteggia,, «



Beato chi pu bella te la feianta Baldon già fenve tl fuocose
E pot leuanfi crosci dell ottanta, Aa com un pan di ”
; STANZA LXVIH. 6

ne ot See wa

E cos} wa,per ca principio Amore, Ma nel getrarla allor
Par bella cosa, efembra ginsto ginfto Perché riftringeye rides
Vira pera cotogna, il cui colore y Ecosi Amor, al primae ani
Odor » saper aslesia,¢ piace al gxfboy C! allerta y e piace 54






lal

aa,t
ie
ib,

NONO CANTARE. 459

STANZA LXIX,

Ed agli cht impaniato, € 4 qualche segno ta lasciamla per hor cbt io'fo diferno,

 Credeil suo amor da lei esser gradiro, Che quefho canto refti qui finito,
 Altero vanne, e fhima a' esser degno, Perché dife un Dottor da Paleftrina

——— Diinuidia più che d'effer mostro a dito, Breuis oratio penetra in cantina,

. era così fit di la, che faceva mille me-
lenfaggini 5 per le quali era da ogauno burlato, ed egli Fingeva di non se n' ac-

c » © continovava a fare (cioccherie oftiaato in quell' Amore, come tal

volta @ un gatto oftinato a stare intorno al fuoco, ancorché si feata abbruciare.

4 Poeta adsaugiia Amore alle pere cotogne, le quali dilettano con l'odore, col

colore, ¢daano gusto nel mangiarle, ma si dura poi fatica a digerirle, € diven-

do che Baldone si reputava più degno d' esser inuidiato, che compatito, termina

il nono Cantare.
| CWETT A, Vedi sopra in questo C. stan. 22.

SERVE per zimbello. Servc per scherzo di tutti. O pure per allettatore degli
altriamanti a venire ad amar la sia Dama. Ii Malatefti parlando in persona d
un villano mandato d' oggi in domani, e burlato dalla sua Dama, disse;
' Da poi ch io ho fernito per zimbello,

E son andato trenta mefi aiont

Gridando per la rabbia', e pel ronello
vibr od Come fa il gatte quando ha i pedignoni
ud id « Alla mia Betta ho pur dato? anello, ec, 7
DICE: vello vello, Termine, che fenifica Derisione', quasi dica; guarda, >
guarda lo feiocco, il pazzo,o simili, ed e lo stesso che Esser moffrato a dito per de-
rifione, che vedremo appresso nell' ottava 69. e che far lima lima dittro a uno vi-
sto sopra C. 3. stan. 37.

MOTT EGGIARE. Burlare, o beffare copertamente uno con detti acuti, e+
mordaci. 1 Greci di C diare uno; noi p biarlo 5 egiarlo, Da
motto, parla; che si piglia anche dagli antichi per sentenza, o concetto, o det-
to intero; B Azorsetto, cine breve detto, e sentenziofo, come son quelli intitolati
Motterti ne' documenti d' amore di mefler Francesco da Barberino. Asutire, loqui
disse Sesto, foggiugnendo 1' autorita d' Ennio nel Drama intitolato Telefo. 2a.
am missive piebero piaculum eft, EB ttimato un delitto a ud plebeo il far motto,cioè
aprir bocea, e parlare: onde Azertegesare non è altro, che parlare con qualche.
bel dettoy @acuto. Dal Greco Azythos viene il Latino murire, €'| noltro Adorze,
Ui Casa)però nel Galateo col definire i Motti /pectal pronrezza, e leggiadria, ed
oftano movimento a' animo; pare che in un certo modo lo faccia venire 5 O pures
scherzaquafi, che venga'da A4oto, movimento.

BEAT O chi più belo te ta frranta, 'BE' lodaco colui, che la dice più bella in bef-
famento di Baldone; ci serviamo dell' cpiteto bearo per felice, avventurato,
fortunato.,'efimili (come se ne serve il Poeta anche sopra C. 1. st. 29. come nel
presente:luogo-, cheesprime, Fanno a gara a chi più bene lo burla: Latino Cer-







sare conuitijs ) Petr. i
NED Beato venir men che 'n lor (Rhos:
Me piit caro il morir, che viner fenxa;
= Mmm 2 Le
& ca

q




















460 MALMANT ELE:

LEV AN crosci dell! ottanta., Si ride fmoderatamente. La vt
quel bollore gagiiardo, che fa la pentola, Fee era 9 Op
€ si dice croscrare dal suono +. ik gal verbo
Dan, Inf. C. 24,

O giuftizia di Dio quant ats
Che carai colpi per venderta crefeia oi

Tl termine dedil otrawta significa squifitezza., o ps
ne logico a to: © forse daile, ralce specie dipannine; le quali
tanta paiole sono a buonidimo grado di perfezione 50 finezza..

Ck ALECC!, o cicalices. Dilcorsi faytida pil persone insieme
priamente dire Discorsi dell! azioni, ed snteredi altrui con.
di bene: éd intended per lo più, Cigalamenti fatti dadonn
digiorni, novellieri; per questo quando si sente Ree nuova
dice € un cicaleccio, o una cicalata. >

FARK orceciue dt mercante. Finger di nom ascoltare 2 © nan.
che altri ti discorra. E propriamente s' intende far oregchie di mercante coll,
che efiendo richiefto di qualcofa, o riprefo d' 4leun vizio non
richiefte, o non si emenda agli avvertimenti, o riprenfioni... Si dice piantare me
vyna lopra C. 7. It. 39. Far conto, chee paffi l. Leperadore. ne to. si

COSTERECCT, intendi le Costole: Li costato..

EVN certo imbroglio, E' un certo negozio imbrogliato, is difficile, cele
mo anche ana cosa così fatta, intendendo una cosa. che pon ha eo del banat
del giufto, dell' onefto o del fattibile. ons,

WEL gettarla, Dicono, che la pera cotogna viloinga il venton-a coed stil
mangia, e lo rifecchi rendendolo stiticho, e però dive;Vel.gerranla da dolore se
più lotto dice; Nel fine ti vogtio, nello smaitirla si man. ia fuori
mu dica le ti riesce così di gusto come pel principio s:cioèiquando lama

41d impaniato, E' rimatto preso alla pania, come rimane-il pettiroflo
do la Civetta, intende s' è innamorato 4moris yorte dmplicitus y aK or
parazione, che ha fatta sopra dicendo,
etientre Baldon qual, semplicerta angela 2

". Così d intorna alla Civetta armecoia..
Quando uno ha male grave, da non ne potere ( non iisimene errs
dichiamo; £g/i ha impaniato, eq o¢ eam

ALTERO vanne, Vedi sopra C. 8. st. 30, Qui-vuol dire gout,
mando, che questo amore lo renda degno d' eGere inuidiato per haver
bene, come stima l'amore.di Bertinella, che d' eles ¢ompatito del
d' cllersi innamorato di costei. B così si da.a,credere digodere ogni
sapendo, che come disse Erodoto nel libro intjtolaca, Talia 5-2 meglio
diato, che compatito; la quale sentenza colle efi parole appunta, a
fa l'usd Erodoto, dichiamo noi comunemence tutto giorno; E.chee ji: ue
ce Pindaro nella Raccolta morale dello Stobea eHMiglian Minuidiak F,
le quali sentenze dalla nostra plebe ridotte in una Cantilena Fiores
Così e sa sincoomate

Meglio e inuidia fop| tare h

Che di se compajfion dare,


























NONO CANTARE. 451

 DOTTOR! di Paleftrina, Se ioffapeti, che Catone haveffe detto. Brevis ora

Caios crederei 5 'che volefle dir di lui, perché fu originario di Tusculo,

di Prafeai »eche havette pigliato Palefrina, cioè l'antico Prenelte per Fra-

7 € S'i0" fapeti » che un montambanco, il quale si faceva chiamare il Dotto-
redi Paleftrina, e faceva da Attrologo fufle solito dire tal sentenza, stimerci, che
ee questo, Ma intenda di chi egli vuole, basta che con questa fencenza
dai opps ha voluto significare, che i difeort brevi piacciono inating ai

2 icantinieri, ( perché ne' suoi Originali trovo una volta im excints,

'ra volta i in cantina ) ed in fultanza intende, che ancora gi' idioti amano,e>

ei eee idiscorsi brevi.
fo
ime i





nt FINE DEL NONO CANTARE.

DECIMO CANTARE,
Peeabasdlasibabastiasdbarls 8

ARGOMENTO,
Per far la Adaga col Rival quiftione
Va y ma in vederlo pot le spalie volta,
E, con lui dietro,
Ove e la gente per balare accolta,
Del Lupo in traccia Paride si pone,
Ui trova,e'l prende con induftria molta, we

ugge nel falone,

i E uccifo quel, da fine alf avventura,

STANZA I.
wanti ci fan, che vestono armatura
0° Dartor di feberme, e ingoiator di fquole
4 ditminedaces » che fanno altrui paura,
© Premar la Terrase [paventare tl Sole;
' © BE ratcontande ognor qualche branura
f

os

Sempre ogn'un cone parole;
St fda sl caso di venire all' ergo,
Labial om! olia, poi voltano «| tergo.

STANZA

tpien mofirain zucca bauer del Sale,
hb ee jon [fanio sempre fugge ta guiftione,
Anxi veder facendo quanto ei vale
odMebpicare al bisogna di spadone,

| Ed wu tal guifaé liberate il Tura,

| pene Reps ep pe geste eer aS

we

STANZA II,

Mae son da compatir fee fanno errore,
benché non fembri mancamento questo,
Se chi 4 menar le man nonglidailcuore
In quel cambio a menare è piedi è leo,
Ob mi direte: Vanne del tuo bonore

Si, ma un po di vergogna pala presto,
Helio è dir: Vn Poltron gui si fugvi,
ee: qui fermofi un bravo,e si mori.

L,

E che ( chi a neffun vorria far male )
Sa ritirarsi dalt' occaftune y

E Senza pagar tafteso chi lo medichi
La campo, che ai ni re se Fee

«dh theme



i eee}










462 MALMANTILE.

STANZA TV.
Ma voi, che di question fate bottega
Credendo immortalarvi; e che vi giova
Far la spada ogni di com! una fega y imparate
E porni a rischise far ogni gran prova, eg
Il nostro Poeta volendo deferivere nel presente Cantare la di
lagrillo a Martinazza, per la paura, e poltroneria della
segui, s' introduce con dire, che quei Bravazzoni,ed Amm
pre discorrono di far rissle, e quiftioni, quando si vien poi ai
ratamente, e loda il lor pensiero, contiderando, che 1
la vita, che far fermo, ed esser' ammazzato per il vano pretesto di rij
eche non può esser biafimato colui, che non havendo cuore a menar |
mena in quel. cambio i piedi, e fa intanto un' azione degna di lode, fug;
male. Conchiude al fine, che tali bravi, che cercano d*immortalarai
ro bravure, e smargiafferie s' ingannino, perché dopo la lor morte:
ur minima menzione di loro: Git esorta pero ad imparare da i
DOT TORI di scherme,e Ingoiatori di (quole.. Cioè che fanno da mae!
ma, e che si prefumono di saper tenere in mano la spada meglio di chi
da nelle squole di scherma. Ma qui scherzando.con l'equivoco di (quola'
che cofioro son bravi mangiatori, poiché ingetano /e /axole, che fo
ne fatto di farina mescolata con anici, ed € chiamato squola,
figura d' uno strumento,col quale si tefe,detto corrottamente /guola
dixs, come vuole il Ferrari; ed è quella cafletta fatta a foggia di na
ro chiamata anche navicella)entro alla quale s' adatta il cannello pieno dil
paflarlo a riempier l'ordito: Si dovrebbe dire (paola, ma l'uso ha
la notizia di tal voce. Dan. Inf. C. 20,
Vedi le triffe, che lasciaron U ago
La (puola,e il fufo, e fecersi indovine,
E nel Purgatorio Can. 31.
E, tirandosi me dietro, fen giva.
Sour! esso ? acgua liene come pola. ?
FANTONSACC!/, Huomaccioni; Huomini di statura grande; ma dicendol
Fantonacei §' intende in un certo modo grardi,e poleroni,o difutili. B dict:
Galeonaces, @Uanizoldacei, ec, Omero nell' Liiade lib, 3. introduce Extore,
del male a Paride suo fratello. £ tra gli altri mali, che gli dice, unoedi
marlo, Eidos ariffe, cioè un bel fantone, d'ottime fattezze; o come meer
significando la bellezza del corpo,disgiunta daila virtù dell' animo;un
wn Dongelione, o come dice qui il noitro Poeta; un Fantonaccio, cio? che!
mostra, ma e poco buono a auila, *
AMMAZLAR con le parole, Legiones difflare spiritu,come disse Pl
dato millantatore. Pretender di farsi stimare, e temere col dilcorrer
ritie, quiftioni, ammazzamenti, e con esercitar sempre con chi fil
arrogante superiorita. Di-questi parla Famiano Strada Jib, 2, Pro
Gloriofi ifti duces. Det homsnumque contemprores, & gut se atijs faci
Calo minitabundi gre 'p ATLis, Guam profil d 08





















DR Pweg er ge ep roeae: =. wa

Sener







DJECIMO CANTARE: 493.
tini chiamano milites gloriofos, questi vantatori poltroni, de i quali intende il Poe.
ta nel presente luogo,e se ne dichiara col dire: Se view mas il ca/o di venire all'ergo,
~ ifica, se vien mai il caso d' haver ad adoprar l'armi, non parlano più, ¢
fuggono, che € quell' abijcere Clypexm de i Latini.

VN poco di vergogna paffa presto. Quel poco di roflore, che si ha per una cosa
mal fatta fuanisce, efi disperde: Seatenza usata, e praticata da coloro,
che fanno poca stima della riputazione.

(i MEG LIO e dire: Vin Poltron qui si fuggi, ec. Buona sentenza,¢ vera, e prati+
jig cata da coloro, che bramano pe tofto vivere con poca riputazione, che glorio-

gi famente morite; il che bene esprime il detto Latino Vir fugiens denuo pugnabit.

m Der, che s'era srmato, ed havea fatto (Crivere nel (uo scudo a caratteri
iamt d' oro BON FORT VN& vantandosi di voler-far gran bravure, (¢ egli entra
è,g Va in guerra; quando si venne al combattere, buttd via lo (cudo, e si fuggi, ed
misit a. coloro, che lo taflavano poi di codardo disse: Vir qui fugie, ruxfus redinregra-
nme bit pralinm, indicans ueilins Patria fugere, quam pralio mori, mortuus enim non pi.
sen grat (che noi diciamo: / morti non fan pin guerra; ) at qui falurem quefiuit in fuga,
poet pote/? sm multis pralijs patria u/ui efe. Tuttavia anche appresso gli Antichi era vitu-
dda Peroso questo tuggire; e si trova, che 1 Lacedemoni bandirono Archiloco sola-
digi mente, perché havea scritto, che era meglio abijcere clypeum, quam interire,
jue a del fale im <ueca, Kaver giudizio. Vedi sopra C. 4. st. 15. e C. 8. st,
wi, © CHOCAR di spadone, Par che voglia dire, che questo tale si difenda con gio-
jgad care di spadone a due mani, ma intende, che gioca di spadone a due gambe.,
yal Slot fugge: motteggiamento usatissimo verlo coloro, che fuggono per paura il
ie dite Ginora ben di /padone, e \enza dite a due gambe s' intende fuggi. Vedi sopras

| C, 7.0.76. Giuocar di spadone si usa ancora di dire in proposito d' una casa, che

sia igauda, e (pogliata di maflerizie; in questa maniera. Vi si puo giuocare di [pa-
done, ciaé Non vi e cosa alcuna, che possa arreftare, o impedire questo esercizio,
che ha bisogno di iuogo largo, e difimbarazzato.
T#aSTE, Vedi sopra C. 1. st 60. Talte fila, che G mettono nelle ferite, dette
così dal taflare, che fanno la lunghezza, e larghezza di quelle. Latini panicidi
ai Vulnerary, lineamenta, i
g DAR campo, che si predichi di ivi, Dac' occasione, che si discorra di lui cons
wm) lode. £1 ver! predicare usato in questi termini figaifica Far' cn:omj, o lodare,
| Quand' uno fa qualche azione bella, e di cia G pavoneggia, (ogliamo dire in de-
Be 2 Chese ne predich,
PAR botreca di quiftioni, Viuer di risse. Haver care le risse per guadagnares.
E tanto questo detto quanto far da spada come una fega, cioè intaccaria nel far qui-
fione, come è intaccata, o denotata una fega ) sono detti deriforj a tali Bravaz-
zoni,¢ Tagliacantoni.
LA morte vi si piega, Voi morite, e dopo la vostra morte non si discorre più
de! vostri gran fatti, e si perde la unemoria delle voitre azioni » e vanne del pari

la bravura, e la codardia » Quell' importuno, che per la via facra s'avvid dictro

a Orazio, enon lo voleva lasciare; domandatorda lui, se avava netiuno de'fuoi,

che' aspettafiero a caia; pee maggior suo dolore gli rilpose: Omues compo/ui,(a-

no accomodati, la morte gli ha ripicgaui tucci, Sa) th ee SUN










44

Colei c ha fatto buio
Paga di sogni i debiti a ciafiuno, .  (Benche si
Quella, che dianzi tolfe al di la vitay Per fuggir
Cagion, che tutto il mondo porta ae Comincea a
Descrive con vaga maniera in quest' Ottava V apparir
con equivoci; uae far buio vuol dit Consumar tutto il suo Sed '
tendedo della notte)vuol dire ha oscurato: e se ha confamato | !
¢ fallita » e non prod pagare i suoi debitife non con i
ricca se non di sogni;e pagar di fogns vuol dir pagar di moneta
non pagare, Vedi sopra C. 2. st. 7. fugge dungue la notte per
giona non solamente, perché è fallita, ma ancora i ella te
sia fatta la spia, che ella poco diana. uccife il giorno perché la
oscurita uccide il giorno ) per la qual morte tutto i) mondopi ee
dir, che per tutto il mondo la notte e buio, enter: bruno,e €0
te di gualche nostro conginen i se bene ella non dovrebbe temere di tal!
zione 5 perché Si chinde gli ocehi a, che fgets on off.
re, finger di non sapere; e il eos connivere., Vedi sopra C, 6.8 vit
vuol dire che si chiudano effettivameate gli occhi, perché og ne
fuggir I iba c' ha le calze gialle, per fuggix V Alba, A e spia del gi
che ha le calze gialle, perché il primo albore del giorno è i colore frail
€ giallo, e così s' accomoda all' equivoco delle calze gialle shee
ze il contraflegno delle spie, o de i toccatori come accenn sopra C
fan. 60. 03 99g
COMINCTA a ragionar dt far le balle + Comincia a ragionare, o r
partenza, che questo intendjamo quando diciamo: 4 rale fa te baile

fa colligar:
TANZA VI, STANZA VII
E denna » che di quci balletti Jf aftidita poi da ranto fran) +
Sarebbe in corte tutto il condimento, Suvi mulinelli, forge cL

























Ler ch' in un tempo fol con i calcerti £ data nna Seofferra come i fae
Ballandosuona al par d' ogni firumento, La laciachiede, britdospi 7
Lupo cena per degni suci risperti Perché il mmico all? alba de' Ta
Prefe dag altri un canto in pagamento, Vuol trucidare in singolar
E sopra un pagliericcio angufto y e fod Ed a fargli servixio, pil
Fino ad horas' e cotta nel | suo brodo. Vuol ee
STANZA VIL. ANZ

Pero che wel pensar che la mattina i vi intrepid
Entrar in campo dee alla tenzone, Espaccia il Baiardino, eit
Fa ginfto, come quella Nocentina, Chi la fringeffe,
C's giorno andar douendo a proceffione, Pagherebbe quaicofa ay

Occhio non chinde, e tuttania mulina, Ma tutto questo





(ZRF, BORED ER RESEP RL Bae. ELLER ew eee

Tanto che ud capoell' bacome unceftone; La faccia tofta 7
Così la Strega in cella solitaria Sperando
eAtrende afar mille caspelli in aria, Chie! non fen,



101 Sig








DECIMOCANTARE. 465

-'Martinazza, che farebbe stata la perfezione di quella veglia, se ne ritiro in
camera, e possafi in sul letto stava pensando alla battaglia, che doveva fare con
jagrillo, ed alla fine, se ben veramente non farebbe voluta andare a combat-
ere, finge coraggio per non esser cae codarda, ed in sul far del giorno chie-
le sue armi, (perando pure, che habbia a succeder qualcofa; che impedilca, ¢
® sia causa che non segua il detto duello.
SAREBBE fata ii condimento, Cioè Carebbe stata la perfezione di quei bali,
# di quell' allegria. Così quando sopraggiugne qualche persona gradita in una con-
" jone, si dice per ilcherzo, Venir ella, come il cacto fu maccheroni, come lo
- xuechero in fulle fragote, o fulle vinande; valendo con queste batie similitudini si-
gaificare ciò che più nobilmente fidirebbe. Essere ella il condimento della con-
tm ucriazione, e non vi mancare altro per renderla gustofa, faporita, e perfera.
hued SVON-A al par d' ogni firumento, Ghediio vogliamo dir copertamente, che una
wet cosa pute diciamo: La talcofa suona, Vedi sopra C. 6 stan: 49, ed il Poeta cava
da ciò lo scherzo dell' equivoco, mostrando di dire che Martinazza suoni d' ogni
mit strumento, ed intende che le putano assai i piedi, poiche dice, che ella /uona co'
'mj ¢alcetti, che sono scarpini di panno lino, che si portano in piedi in su la carne fot-
shay to te calze; e si dicono cascerti ancora quelle scarpe di quoio forcile, senza suolo,
gum ma con la fola piantella, che usano i ballerini, e che ulavano già l¢ nostre donne
ga di portare sopr' alla calza quando portavano le pantofole.
ott  PIGLIAR un canto in Pagamento. Significa Andarsene. I debitori, che volen-
rag ticri (cantonano i suoi creditori,si dicono dare un canto in pagamento,cioè fug-
gigi gite il creditore per non pagarlo, e per non avere occasione di trattare con Jui:
|. PAGLIERICCIO. E quel gran sacco pieno di paglia, che usiamo tenere in fu
gig Fletti sotto le materaffe, detto anche /accone.
wt ~~ 8° 6 cotta nel suo brodo, Non ha havuto veruno d' attorno. Quando alcuno f2
: qualche risoluzione, che non è approvata, o non piace agli altri, e non e da ve-
yi tuno in quella seguitato diciamo; E /* quocerd nel /no brodo, cice senza che altri
vi a & nulla del suo; o vero Farò come gli (pinaci, e s' intende che G quo-
cono ir brodo
già FA come quella Nocentina, Nello Spedale deg!*Innocenti di Firenze (che & quel
“4 nel guale s' allevano i nati per lo più di copula iliecita, si come accennam-
i Te sopra C, 1. stan, 85. ) stanno riferrate molte Fanciulle, che noi chiamiamo
a Mocentine le quali non escon fuori se non una voita ! anno, che è la mattina,
, della vigilia'di San Gio: Batifta, che vanno per la Città procethionalmente; es
Pe ciascuna di loro ha gran desiderio di far tal gita, non vi € aubbio, che
f speranza d' haver a godere si bramata foduistazione, fa, che pare a' ciascuna
 mill' anni, che venga il giorno y¢ che per tal pensiero poco derma la notte avan.
 £1, rivoltando per la mente wweti li modi di comparire atullata, e bene all' ordi-
ne; il che è caula, che la mattina ella ha poi un capo c me un ceffone, cice grof-
© €pieno di confuficni per haver poco dormito, ed affaticaia la mente in quei
Pensieri; € guefte son quelle, alle quali il Poeta aflomigiia Martinazza.
MVLINARE + Pensare; Disegnare, andar vagando con la immaginazione;
che diciamo anche: Ghiribizzare. Vedi sopra C. 9 stan. 56. Viene dal Latino
molior » che vuol dir wacchinare,O ne dal volgare Aduino, quali girare coi pen-
aa ficro









te








466 MALMAN TILLEY

ficro come un mulino. Virg, disse spedissimo +| Corde:
che fanno le persone innamorate peulando fidamen
giamente ne diede la descrizione in Didone,
Multa viri virtus animo, multufque
Gentis honos  barent infixs pe'tore vultus
Verbaque, nec placidam membris dat
Tutta la notte va mulinando « E lo stesso, chevaculer. Ho
Quid brexi 'fortes iaculamur auo
multa ?
E' detto ballo scagliarsi col pensiero ora in una cosa ora, inua)
Mattio Franzefi acl Capitolo delle Nuove,
Lasciamo aftroiegare a chi indovina
Per wie di conetiure, e di difeorsi,
E col vernel fantaitica, e mulinay.

HLA il capo come un cefeone. Gli si confonde ik cerucilo, Pai p
do diciamo fa ii capo grosso, 9 se gli ingrofia il capo, intendiamo
de il giudizio: EB Cefone & un gran paniere fatto di vinciglic dt
te, ed & capace di mezza (oma, e perché ha la figura a:
queta comparazione. vil

CAST ELLO in aria, Pensieri senza fondamento, ed affegnamenti
nt, e che non poslono riuscire. Laili Ha, Tr. C. 2. st, 2470 ADA

Fra me facea mille Caffelli im aria ode
Ariftofane intitola una sua Commedia, in cui. Gi burla di
Nuuole; e lo fa falire, e pafleggiare in aria.y per mostrate, pr !
vana, e senza fondamento la (un filofofia. Noi quando vogliamo dire!
badare a' discorsi terij, e avere il capo altrove,¢ a bagatelle; Dichiamo i
fare a' nuuoli, (e non vuol dire più toflo in lingua Lanadattica: Pen/area milla.

MVLINELLO. E uno firumenco di ferro, che serve per follevar peli
derivandojo dal verbo malinare detto sopra significa inucnzioni,
ne, disegni, ec,

DATA una scofetta come i cani, S intende, che Martinazza I
veilita, e levandosi dal paglicriccio, fece come fanno 1 Cani, quando,
no, che per lo più si fquotono. A

ALBA de' Tafani, Si dice quell' ora del giorno sche il Bolee
re vigore, nelia qual' ora i Tafani sono più vivaci, Tafano. Lati
un verme volatile simile alla vespa nel colore, e nella figura; ma
assai maggiore, ed ha ancor' egii un' acuto pungiglione 5, ficche
de' Tafani s' intende leyarsi di la da mezzo giorno wi) |) say Pr
PAR vegnia uno, Far cortefie, o carcazeasuno, an iL
no affetcate, si dicoao /ezzi, quali iddicia'y © intedtus » come k i
sca Novella 10. Serallegro con Nencio [poso della Ragaread y #6 &
bene, e le faceffe verxi. Col dire.  farls servizia, e pin chee
orecclu sieno i maggiar pezzé,intende, che Martinazza gli fara g
tarlo in pezzi così minuti, che un' orecchio intero sia: 1 mag)
trovi del suo corpo »detto acim per suena un






























a
om

DECIMO'CANTARE 467

'SP ACCTA il Baiardino, e il Rodomonre, Si fa Mimar bravo, come favoleggia,
' Ariofto, che fulle il Cavallo di Rinaldo Paladino appellato Baiardo, € quel
¢ Saracino detto Rodomonte. Può anche essere, che far il Baiardino, signifi-

chi far il bravo da un tal Pietro Terragiio soprannomniato Baiardo, che fr uns
soldato di-valore, e d'inufitate forze, il quale mori forro Milano militando al

-servizio del Re Francesco di Prancia, come narra il Varchi Stor. Fior. lib, 2,
| CHI la fringeffe fra uscio, ¢' mare, Chi l efaminatie bene; chi glielo do-

mandafle da folo a folo,

segua, o non vada la posta, o l'inuito
tutte le cose, che intenzionate, non s' ¢!

 PAGHEREBBE quatcofa a farne monte, Spenderebbe qualcofa a non far questo
“duello. in ructi i giuochi si dice far monte, quando si reita d' accordo, che nons
roposto; e questo e fatto poi comune a.

i(cono: per esempio / tal matrimonio,

he era già conchinfo', ando'poi 4 monte, cioè non si stabili. lo voleva andare a Ro.

with
joie

ma, ma poi ne feci monte, cio non andai.

IN se tien duro, Lo tien fegreto in se. Non si confida con veruno,

FA factia tofta,, La faccia fol' esser dimostratice delle interne paffioni; e pe-
ydiciamo; / rale fa faccia toa, intendiamo il tale si sforza di non sco-

iia prit co mutamenti del voito 1 suoi fegreti, essendone richieflo, ¢-di non confet-

waco

a

git
:
3
eo
è
:

i

:

“STANZA X.
Spada,e lancia fra taro un Servo apprefia

i Col perto.a borta in man Laltro galoppa,

'a altro o elmo da coprir la testa
Da distder unalcro,e bracciaye groppa,
Di che coperta in ricca sopranuefta
Par un pulcin rinvolto nella floppa,
Ed allestica in sul cantar del gallo
eitro quivi non resta, che il Canalo

fare itdelino »essendone claminato. Latino frontem perfricit,

STANZA XL

Percio fa comandare a i Barbereschi,
Che lo menin n' un campo di gramigna
Accioech'ei pasca un poco,e si rinfreschi
Perché per altro il poverm digriena.
La marca bebbe del Reeno,es enidalescbi
Gis hanno rifatta quella di Sardigna,
Maglie,e reti ha negli occhi,ode per cena
Vanne a pescar nel lago di Bolfena

B servi di Martinazza le portano l'armi, delle quali armatasi, ordina, che le
sia condotto:i} Cavallo, quale il Poeta de(crive per una folennissima Carogna.
“GALOPP A, Cioè Corre', Verbo usato in questo significato,ma però impro-

prio, perché galoppare, o gualappare & specie di correr di Cavallo; la qual voce
concorrono gli eruditi a farla venire dal Greco calpareia,

GROPPA, Si dice la parte di dietro del cavallo, o simile animale, ma qui in-

tende la schiena di Martinazza.

PARE un pajein rinnolto nelia floppa, Quando si vede uno, che non fa portare

l'abito in dotfo, e che pare impaftoiato nel camminare per causa deg!i abbiglia-
menti, che had' attorno, l'aflomigliamo a un pulcino, o pollastrello rinvolto
nella stoppa; e non fiamo is ciò diffimili dai Latini, che in questo proposito
didero. Herer ranguam mus in pice.

SVL cantar del gato, All' apparir del giorno, che a talora fogliano per Io più

cantare i Galli Vedi sotto C. rr. st. 5. Orazio.

etd galli cantum con fultor ubi oftia pulfar,
BARBERESC HI, lntende gli Stalioni; (¢ bene Sarbere/chi chiamiamo coloro,
N ai

on 2 i quali



Bikes e, 4





468; MALMANTILE —

t quali cvflodi(cono, e gevernano i Cavalli Barbari, ¢
Poeta gli chianya così per derisione del Cavallo di Martina
Firenze 1 Cavaili, che corrono a i palj della Città, ton
frica, che noi chiamiamo Barberia, s
CRAMIGNA, Erba nota buona per pascolo degli Afini pil
li, ma a quelio di Martinazza non par poco haver di questa,
zerin digrigna, clue s¢ nou havefle di questa, non havrebbe.
ci serviamo del verbo digrignare per intendere flentar per la fa
nare, e acrocare i denti per non hauer altro, in che ado
canl, ec. che si dice digrigware, quando per la rabbia
Tat Cas.
x Non vedi tu, che digrignano i denti
Econ le cigha ne minacctan anoli?
Ed egliame: Non vno, che cu paventi y
Lascsagli digrignar pure 4 lor fenno,,
MARCA, Contraflegno. Es' intende quel fegao, che hannoi

li, o di razza in una coscia, o nel collo, perché da essi si possa
razza sono. Virg. 3. Georg. Continuoque notas 5 nomina gentis inurunt,
che questo Destriero di Martinazza havea già la Marca del Regno di
sono oggi i migliori) ma che i guidade/chi gue n' haveano mutata in
digna, € non intende dell' Liola di Sardigna, ma di quel luogo fuori
Firenze, dove si scorticano le bestie morte detto la Sardigna, came
pra C, 1. st. 2g., ed intende, che questo Cavallo per li guidaleichi, ed
fetti, che haveva, era buono a mandare in Sardigna allo Scorticatoio
te/co diciamo ogni scorticatura fatta alle Bestie dalle selie, balti, o altro. Mau

Franzcli descrivendo un cavallo fintile a questo disse; wig
Dinanzi ¢i non e 21d troppo gagliardo; “iy
Ma in sa la scbiena ha qualche curdalescho,
E le spronate mostran, ch' e infingardo, ™

MAGLIE,¢ veri, Così chiamiamo alcuni mancamenti, che vengono si
occhi aile bestie; ed i} Poeta servendosi dell' equivoco dice, che con 'quelle ra
può andar a pescare nel Lage dé Bolfena; ed intende, che il cavallo-era bof
dicemmo sopra C, 3. st. 53. » che cosa sia. E così sotto questi equivoci iroa

mente loda il-Cavailo di Martinazza. sagt
STANZA XIl. STANZA XIIL
Hor mentre pajce 1 mifero animale, E ti faluta,e tt si raccomanda, —
Eche si fala cerca aclla fella, E per cha inteso, che rm fai duclly
Giunge un Diavol più ner aet caviale Vn rotelion di fughero ti manda,
Con un marteile in mano,e una rorella, Spada non già,ma ben gnejto:
Ed un liquor botiente ix un pitale Con una potentissima benanday —
'Ed inchinato a lei cos favella: Ch' 10 ti prefemo emr'a
I Re dell' Infernal Diavoleria Bell! e caiduceta come la.

Con queste trescherelle a te m' innia, edilo [pedal si ad ta medicinal
;: aie
























DECIMO CANTARE. 469

. STANZA XIV. STANZA XVI.

Hor fenrj; che qui batte sl fondamento Ma se per non haver buon corridore'
Quand' ih nimico ti verra a ferire Quivi a canfares tu non fulfe leffA,
Va pure innanzi,e non haver spavento, O per altra disgrazia; o per errore
el ferro questa targa a offerire, Ei r'appoggiaffi qualche calpo in tej 5

 E tuffo ch' ei la paffa per di drento, Vorlio, che tu per sicurtà maggiore

» Sia presto col martello a ribadire, Hor per allor4 ti tracanni quests,

Ma lasciagnene subito alla spada Quale e una bevanda sh squifita,
Peich'egli a se tirando, tu non cada, Che chi Lha in corpo no pua uscir di vita.
Ni STANZA. XV. STANZA xVIL
Face' egli poi con essa quanto vuole Così le fa rngoiar tanty dt micca
| Che pix di punta non puo farts offe/a, D! una colla renace di tal forte,
» Di taglio manco, essendo c' una male Che dove per fortuna ella si scca
Si fata a maneggiar pur troppo pela; wl mondo non è prefa la più forte;
Portila dunque per ombrello al Sole, Luefta ( die' egli) Uanima t appicca
» Pere alia resta non gis muona scefa Ben ben col corpo, e s'aitre non e morte
 Edigli( già che queila non e il case) ©? una fepararion di que/ts Aussi,
Che # egli ti vuol dar, ti dia di nafo, Oxgi timor non hai de' fasti [uci.

» che Martinazza aspetta i} suo Cavallo riceve un regalo da Plutone. 5
confiftente in armi jd in. una bevanda per difendersi dalle ferite,¢ dalla morte,
Nota che in questo bel regalo il Poeta immita coloro, che hanno scritto le pro-

 dezze d? Amadis di Gaula, ed altri Romanzatori, i quait, quando il loro Erces
dee esporsi a qualche battaglia pericolofa, fanno sempre, che qualche Mago
“amico di efso Eroe io mandi a regaiare d' armi incantate, © altri difenfivi, ed
inttruziom, '

St fata cerca della fella. Si ta cercando della fella, Dice così per mostrar, che
ae era tanto iniolito ad adoprar la (ella, che non si lapeva più dov'
ella fufle, >

PIÙ ALE. Alberello, o vaso di terra, come dichiara il medesimo Autore nell'
Ottava seguente dicendo; ch io ti presento entr' a questo aiberelio, Se ben Pitale &
Piopriamente quel va, che si mette centro alle predelle con altro nome detto
tantero..L' uno, e il altro nome dai Greco, quello da Pitharion, piccol valo di
terra, doioiwm; quetio da Cantharos voce usaca anche da' Latini. © significa un
vao lungo, e stretto in fondo. E con manichi 5 quale e queilo, che si vede cal-
volta figurato in mano a Bacco.

« TRESCHERELLE, Lato trice, Bagatelle; Coferelle di poco prezzo, Ve-
di sotto in queito C, f, 28.

SVGHERO. Pianta aota simile alla Quercia, e fa le ghiande ferotine, e las
faa leggierisfima (corza serve per far lavort da refiitere all' acqua, come farebbe
caiietce per mettervi bomboie di vetro piene di vino, o d' altro per diacciare.

 BELL e calduccia,, Temperatamente calda; e come si da la medicina, che intea-
diamo bevanda folutiva. Vedi sopra C, 8. tt. 25.

CHUVOVERE fiefa, Fer yenire l infreddatura. Scefa diciamo una distillazione,
o catarro, che dalla testa casca nell'altre membra per causa del freddo.

Tl dia di nafo. Detto iporco usatisfimo nelia Picbaglia ia segno di disprezzo,e

sin-


————————————< =

i

se

472 MALAANTILES @
s intende di nafoine,... che per ricoprire si dice 6
serve = esprimere la poca stima, che si fa della ——

NON fuffi lefiaa canfarci. Noa fai prefta a-fuggirli,.
Effugere, delinearesy nes lisdab Greco compre ara
detto così quali CG x F

TRACANNI, they bevay logolli i

TANT Adi mica, Vina gran quantità di ininefied = "
tore del Capitolo in lode de' Peducci, parlando:della min Secccea i
E gli ho tutes per cari, non che buoni

Von ostante, che sia chi dica espreffa,

Che tanta micca e cosa da bricconiy
Ser Brunetto Latini servendosi di questa voce nel suo\librovco
tutto di gerghi,¢ vocaboli,¢ proverbi Hinsanwsats 7 intitolaco
che sia antica Cittadina-di Firenze, 1s

Non ti darei una mica di beata; ¢
Se bene qui par, che voglia dire wn bricivlo, dal tele
tanta si pronunzia col gelto, che accennammo sopra OC. soft.
Luefia pea, e vedremo (orto nell Ortava 18.¢ 36. seguenti

FICC-A. Ficcare vuol dir Mexeré » 0)Cacciar per forza'.

NON è prefa (a pix forte, Diciamo fan prefax, quando la collay cal
© simili s' appiccano gagliardamente in quet noghi »ne\bquali-sono

L'ANIMA & appieca, Si ricordi il Lettore, che quella 6
fu le burle, e particolarmente dove G trata diyincanti,ne iquali, q
trava luogo di fare apparir qualche azione spropositata,non lafera
segue in guefla bevanda, la quale dice, che appicca ' Anima al
che egli creda, o voglia periuadere, che ciò possa per incanto farsi
firare la goflaggine di Martinazza, e di coloro j che hanno tanta a
caatelimi, e ne i Demouj,

STANZA XVIIL
Quando la Maga vede un tal presente,~
C' ha in se tanta virtù, tanto valores
Da. morte 4 vita riauer si sente y

Si ringalluzza, e fa tanto di cuore 5 'Cusiabe 'hontai i se
E dove fares ita un po.arilente Percio fatracal ronzin ha fell
Nel far con Calagrillo il bellumore, Vi monta sopra, € poi te xomb








Hor e ha la barca afficuraca in porto

Pere! adeffo ch' eg ha ratte
Per fette volte almanco lo vnol morte, rddy

Camminerebbe più in
STANZA &X

Perché ei bada a spudiar declinazioni Pur.grazia del mated
Pin non si pua farlo levare « panca; Tentenna tanto y
Le polizze non Pwo, parca i i frasconi, Chiesvien. done n'
E con lo spalle s*¢ givcaio un' anca; M14 « carinetie il fang

Martinazza inansmita dai regalo mandatole da Piutone, etlendo-
Sole, monta a cavallo, e taaro io fruga con gli sproni,, e col im.
zoppicando pur alia fiac  conduile ai luogo dove hayea ote \

si
at





















reese erProczsleezeTEt.2f:r2=...




DECIMO CANTARE. 471
ST sente viauer da morteavies.Cioè le pafla quel timore, c' havea dvesseres
- ammazzata da Calagrillo | x,.
mi. SI ringaliarza. Si caliegea. Lat. Gefire, Si dice ringalluzzarsi, quasi mo-
~ firarli ficro.,.¢4 animofo come fanno 1 Galletci,quando si preparaao per -
“ ter fra oro, @ dopo che hanne combattuco, e vinto. Lucilio 4ib, 8, satyr. dice:

eee Galli nacens cum victor se Gallus bonefte
ee ite © Sufulie in digitos, primore/que erigit ungues.
@ [Lalli En, Te. C.5. dan. a6. ditle 3. Jn guetta nacas anor si ringalluzza. Stor:
di Seumifonte TLratt, 321 Semifoutefi, credendo d'hawer ogni dsfficuied fopita, rinesl-
— bnzxaronfi, @ fidandosi di (un valentia y ec, B pi Lowe dice: Veds quanto noi fama
om 4 iti, e 8 mimici ringalluRrati, eC.
gibi FAltanto-de cuore, Pigiia animo, le cresce V' ardire. E il termine Tanto nel fix
infos gail » che diceauno nell' Occava 17, antecedeate ed altrove,.¢ si fuppones
i sho deteo.aicrove:), che colui, che per la faczia la dimottrazioac con las
| Mano accennando la grotiezza, e pants di quella cal cosa, Quei che i.La-
ect ual daimus, vooltci quali sempre dicono coraggia, e cxore.
ia | SAKEBSE a a rilente., Sarcbbe andata adagio. Circospetta ) O rattenuta a4
¥ risoluersi,, )L? havrebbe pensata, o contidcrata. Significa infomma operar coa
tuwore. Leace per lento, siccome Violente per Violento dicefi da alcuat; come
Queito filo, queita corda e fenre,.cloe non tela, non urata. Da Lente si fece Ri-
ing (fn sche noo ti usa se von in questa Maniera: eFadare a rilente, e significa lo
cai stesso, che Lente cioc ientameate., Nello tteflo modo chel'antica voce Diricapo
ail usatardal?-anuco volgatizzatore di Virgilio;c lo iteilo che Dacapo,
PAR ih bed: umore. de dea huomo dell' umore, vuoi dire huomo faceto, es
SFAZ1010,5-come vedenyno fapra C. 1, stan. 1o..¢.58. s*intende anche wao., che
si Vogiia: sOpcattarc 1 Compagna-di parole, e di fatti, ¢c, comes' intende nel. pre-
feute Moga, f
aia AOR eC ha la barca aficarats in porto, Cioè le par d' haver afficurata la vita col
regalo mandatoie da Piuwone
ih nae Vheche racing a | bucats wi fu i terrazzi', Cio' il Sole, che asciuga i panui
“moi deabucati, Dereazzo! »( quali Terrazzo) diciamo quella parce fuperiore >
jul dele case la quale per loipiu edasciata da ana banda aperta', e feoza muro, in
Se vece dei quaie lita, solteacre al tetta- da colonac, e fom fabbricati in questa forma
ww per comodita d' havere idole epercid das Latinidetti Solarimm, eda i Greci
Wil hewocaminus', Cio fornace del Sole,
iM CAM AMNGREBBE più in tre di che in uno, None dubbio., che qualfivoglia
m! — Animaie.camuninerebbe pith ia tee giorni, che.in uao, ma uGamo questo modo
wih di dure per moitrar la fiacchezza d'uao Aaiuale y quasi diciamo: Quel viaggio
che egit na da farein un giorno, 10 farcbb¢e pwd voleauieri in tre giorni, che in
yun foiow ue ¥
ul BADA a fhudiare-declinagioni, Attende,:o-continovaad accennare di -cadere
w” ~— perladebolezza. Declinare's' intende uno, che*etfendo in buono stato: 5.0 dt fa~
ie hita,o di roba, cominci amancare nell' uuo, O.nell' altra; equi (cherza cont
4) equivoco delle declinaziom de 1 noim 5'ed-insende, che-il cavalo per la deboica-
#) 2a era fempre per. cascare..
0 Wow









47% MALMANTILE ©

NON si pui far lenare 4 panca, Non si può farlo riavei
flar ritto: quand' uno è stato lungo tempo afflitto da i difaftri
to per terra, o vero terra terra) € che a poco a poco si va
Comincia a rizzarsi a panca; BE' traslato da 1 Bambini, quando:
dar ritti appoggiandosi alle panche; onde habbiamo un detto per
uno sia pil aftuto d' un' altro, che dice: Quando it Diauolo del tale nat
del? altro andaua alle panche, Franc, Sac: Nov. 158. dice: ach 60)
nostra mercanzia, che non ce ne rizzerems piit a per questo anno, ©
NON pwd le polizze. Non ha tanta forza ch' ei possa portare una po
Latini pure dissero: We folium quidem fuspinet. t

PORT Ai frasconi, ec. Diciamo portare i frasconi uno, che sia alg
mo, traslato dagli uccelli, ne i quali e contraflegno d' infermita Y
abbaflate, che paiono bestie cariche di faftella di frasconi. Vedi. y
g. alla voce grado. Qui vuol dir che il Cavallo era infermo, e malandato pet lt
vecchiaia. | Lb
CON 10 [palo s* e giuscato un anca, Scherza con l'equivoco del giuoco di
nel 8 quand' uno piglia tante carte, che col lor contare
31. si dice /pallace, o ha baxuto lo /pallo, e perde, si che intende che il






&














Martinazza è spallato. von lil
GRAZIA del martello, e degli sproni, Con ' aiuto del martello, che le mand)
Plucone, e degli sproni, cio perquotendolo col martello, epi 1
gli sproni: Diciamo anche mercé del martello, ec, er
S* arranca, Diciamo arrancarsi, quand” uno per qualche difetto non pot
muover le gambe s' affatica per camminare, e/forse e il verbo p
pato. Vi chi lo fa venire da Anca, che è l'offo tra "I fianco, ela coleiay
questa dalla Greca Ancon,colla quale si significa il gomito, e si stende ad
gature, che somigliano quella del gomito, Onde Sciancaro, quasi ex:
pun ha intere, enon senza mancamento l'anche. B Arrancarsi quasi tirarh, 2
firaicinarsi distro l'anche. 15h) aga
NE ha da ire il sangue a catinelle, e ha bigonce, Ha da verlari moltissimo far
ue. Vedi sopra C. 2. stan. 57. (perbole usaca quando due Poitroni
ducllo, Vedi sopra C. 1, stan. 62. in altro signiticato. BC, 3. Ran, 29, che ol
sia bigoncia, Quando l'indugio piglia vizio, e-che fa di bisogno la preftezza jl
altro proposito dichiamno. ee ne va il sangue a catinelle, aah
STANZA XXI. STANZA XXIL
Quand! ti Nimico, ch' ius faa difagio Se tu fapeffi, come tu non faiy
A tal pigrizia,grida ad alta voce, C' armi son queste pie
Vieni Afinaccia, moniti Santagw Farefte forse il brauo mance,
Cb' so son qui pronto acaricarti anoce, O parlerefti almen-a' altro ling
Ella risponde: A noce? Biagio; Ma già che tn venifti a tno
Fate un popian Barbier ohe'lranoquoce; i
S' altro vifo non haivallo a procura,




















SeweR.> es > sre

cr

a repo RB re =

atten

o

alee






Lerche codefto non mi fa para. rrotté
Arrivata Martinazza al luogo dove s' haveva a fare il duello.wi tr
¢o Calagrillo, il quale vedendola venire così adagio ia fgrida y ela






SSk ELS

8 EE Sei ESei a oak

=
&

DECIMO CANTARE. 4B
ella gli risponde;che non ha tanta furia, dicendogli ch' ei non' farebbe tante»

bravure,(¢ egli fapefie di che armi ell' e armata),¢ che ella veniva per ammaz-

zarlo.

“STA 4 difagio. Patisce aspettando: Sente incommodo in aspettarla,
 eASINACC/A. Parola ingiuriofa, e benissimo iesire in questo caso as

Martinazza, perché veniva pigramente, come fal' Asino.

. SANT AGIO, Si dice veramente Ser egio; che fu un Medico così nominato,

perché taceva tucte le sue faccende con ogni maggior suo agio,¢ commodiia fino

a tirighare, e ripulire la faa mula, senza muoversi dal letto; ed e paflato poi in

verbio, e yuo! dir Huomo di turti i suoi comodi, e tardo nell' operare, che

ju una parola diciamo - Agiato. O forse-dalla voce Toscana, che vuol dire Len-
fecha y Comodird,

A caricarts a noce, Quando il noce è carico di noce, si scarica con le baflona-

te, e pero dice, che wuol caricarla alla foggia, che si carica il noce, pec (cari-

Carla poi-con le perco!

se

» @LAGIO Biagso. Modo di dire usatissimo, e particolarmente de i Fanciulli,
€ credo che si dica per caula della rima, e del bifticcio, perché per altro il nome
Biagio e (uper fluo all”'e(preffione, valendo tanto il dir solamente adagio, quanto
adagio Biagio, S¢ bene ci e una favola notissima d' un certo Contadino nominato
Biagio, i quale perché non gli fussero rubati i suoi fichi, se ne stava cutea la not-
te a far loro la guardia; onde alcuni Gioyanotti per levarlo da tal guardia, es
poter a lor gusto corre 1 fichi, fintifi Demonj una notte s' accoftarono al capan-
nettoid) Biagio mentr' era dentro, e discorrendo fra loro di portar via la gente,
ciascuno narrava le sue bravure; ed uno di coltoro disse ad alta voce; Se voglia-
mo fare un' opera buona catriamo nella Capanna, e portiamo via Biagio; Bia-
B10 ciò -udito,scappd dai capannetto tutto pieno di paura gridando Adagio adagio.
o di qui puo forse havere origine il presente dettato Adagio Biagio, o adagio disse

ago,

FAT £ pian Barbiere che'l ranno quoce. Di questo dettato ci serviamo, quando
Ron voghamo acconscutre che si faccia qualcola in nostro danno.

» COT ESTO vifa non mi fs paura, Quando vogliamo mostrare di non temeres
diciamo: Ha tu altro vifo?e qui Martinazza dice: Va 4 cerca d! wn' altro vifo
perché corefho non mi fa paura.

SEVER AGGIO. Invende quella colla fehe Ie ha fatta bere il Diavolo, 1] Fran-
zele dice bexarage corcispondentemente alla nostra voce.

A tao ma' guar, Cioè a tuoi mali guai; Mal per te, che ci venifti, Ci sci ve-
puto per wrovare il tuo danno, Cusi 44a' paffi diceli alcuna volta per cattivi pal-
si; ome 'Piano a ma' paffi,

MANDA 1 faggio. Quando si da una piccola porzione di quella mercanzia,
che si vaol yendere:, acciocché il compratore possa riconoscere la qualica di etla
mercanzia si dice; dare, o mandare il faggio. B Martinazza dice a Calagrillo,
che intanto mandi il faggio della sua carne ai vermini, perché fra poco vuol
mandargii nell' avello tutto il corpo.

NON volti portar basto. Non son solita fopportare ingiurie.

Ooo STAN:
414 MALMANTILRS 2
STANZA XXIIL t ]
Horsh, dic' egli, all armiv apparecchia,
E vedrem se farai tante corenne,
© questo suono allor mona Pennecchia

y

Dice fra se: No,no:Non taro Ammenne, \-  E ch* io t° insegni far

Sard meglio qui far da lepre vecchia, Così tn ch' items

E fenva star a dir pur al ©... vienne, Milafis a}

Fa proua ( già dilcefa dal destriero ) Ma fa pur quitof

Se le gambe (c\dicon meglio il vero. Bt ual, se eu fu z.
STANZAXXV20 1)




S? al cimento, dic' ella, del duello C
A furta corsi, hor fuggolo qual pefte y Però che dop' al muro f
Pero va ben, che chi non ha ceruello Grid egli quanto vnol,
Habbia gambe, e così mena le [este, Che per le grida it Lupose
Mortinazza, vedendo, che Calagrillo non cede alle sue bravate,

che fara meglio per Jei non indugiar pil a fuggirfene, pero (non si:

cavallo) fmonto, e fuggi così a piede verlo il Castello!
rimproverandole il mancamento, ma essa stimando più il peri

la perdita della riputazione fen' entra in Malinantile, e lo lascia
SE farat tante corenne, Se farai tante bravure. Detto di derisione a wu

vantatore. wR
MONA Pennecchia, Detto derifivo alle donne. Da Pennecchio', ig

priamente si è quella quantità di lino, o lana, o cosa simile, che si

rocca per filarla, detta così quasi pensiculum.. Dal Lat. pensum.
NON tanto ammenne. Non fara così. Ogai parola non vuol risposta Per

io non voglio poi anche fidarmi in tutto di Platone'. Amen & parola Bl

vale In verita, Per verita. er
FAR da lepre vecchia, Cioè tornare in dietro, La lepre vecchia per'

gnar terreno, quando e seguitaca dal levriero da in dictro, (il cS atto

La un ganchero, Vedi sopra C. 2, stan. 76. ) ed il cane furiofo se

scappa innanzi, e perde l'occasione di pigliarla. L' aftuta maniera

della Lepre è descritta mirabilmente da Eliano nella Storia degli animali'
cap. 14.. are

SENZA dire alc,,., vienne, Andarsene subito, © senza 'merter tempo it

mezzo. II Pulci nel Morgante, £ non è tempo da dire ale.... vienne.
SE le gambe le dicon meglio il vero. Se cilia fara più presto a fuggire

a cavallo, Quando le gambe, braccia, o altre membra fanno bene la

razione diciamo: Le gambe, ec, mi dicono sl vero, cioè non mii fallifeone

mancano sotto., Wee

Cl haueffi detto ulmen Salamelech, Almeno ci' havefii'ta detto.

Turchesca usata da noi per (cherzo; e significa, Pace, o Salurea voi. ~

FARM le feilecche. Betfarmi.. Vedi sopra C, 7. stan. 25, 11 Vor

goefe dice, che Cilecca wien dal Greco Cileo, che wwol dir mulceo far

feilecca far tl contrario di carezne, civ far burle. Ma può essere, che |

licta Gi fece Lezei forca di delicatczzc così Scileccke il contrario, che A

aliettare 5 e poi burlare,

E intana di riterno





































= pe eet S*e2 ce oc We ieee se*8. BS screes.es.ess


SSeS





DECIMO CANTIARE. 475
WMI lasci a prima giunta in sulle secche. Subito,m') abbandoni + Milasci:senz'

- alcoltarmi..B' lo stesso.che lasciar in Naffo,visto sopra C, 1, an. 79, Si dice an-

che /asciare in Seco; lasciar sulle secche di Barberia. Lat. Syrtos.

AO teco il sarlo. Ho.rabbia teco., perché ilxoder. della rabbia s' assomiglia al
roder del tarlo nel legname:,Per il contrario si dice: auer baco.con wna persona y
cioè averci paiione. Petrarca: Afentre che il cuar daeli amorofi vermi fu consumato

TI vegito se tu fuffi in gremboa Carlo, Tiarriverd per.tuto, Diciamo; J.
grembo 4 Carlo, cioè Carlo Magno Imperadore, per mostrare che si vuole arri-
vare uno, e vendicarsi in ogni maniera, quand' egli anche si fuggisse fotco la pro-
tezione del più porente, valorofo Principe del. mondo, come fu Carlo Magno;
econ i Latini diciamoanche,in grembo a Gione.. at

| CORRER a furia,, Eo stessos che far una cosa senza considerazione.. Vedi
sopra C. 5. stan, 41. E qui (cherzaintendendo se corse nel venice corre anche
nel tornare in dietro. '
 CHL nan hs ceruello habbia gambe. Significa chi non ha havuto giudizio,o me-
moria di pigliare, o fare tutto quello, che egli doveva in uo viaggio, habbia gam-
be, cloe lo faccia in due, o più viaggi, ma qui il Poeta scherza,.¢ motteggian-
do Martinazza si serve del proverbio, per intender, che se ella non hebbe cer-
ucllo ad accettare, e venire al cimento del, duello, habbia hora le gambe per

ire
MENA le fefte, Pa speti,¢ lunghi padi, Le (ele, cioè il compaffo, s' affo-
miglia alle gambe dell' huomo; e pero mexar /e feffe s! intende adoprar prelto les
gambe, cioc camminar velocemente, correre.
ANT ANA. Intendi se n' entea nel Castello di Malmantile. /ntanare da Tana;
cava foterranea.
DIET RO ai muro faluns efte, Chi ha un parapetto di muraglia non e dubbio,
che € sicuro dalle stoccate.. B/se dal Lat, &è, formato all' usanza nostra, de'
li niuna parola intera finisce in confonante. Ii Burchiello nelia fine del primo
Sonetto. on funt non funt pisces pro Lombardi. il primo fant va seritto, e letto
funte come qui Efe, acciocché il vero torni. E in quel verso, per dire anche
spe 28' aliude a un vero Racconto, che si trova (critta nelle Craniche de' Pre-
icatori, alla vita di Giovanni da Vercelli Generale.
DALLE grida scampa il Lupo, Detto ulaciflino per mostrar la poca Rima,
che si fa di coloro, che gridano,

STANZA XXVL STANZA XXVIII.
Poich* egli vede in somma che costei, CHartinarza, che teme del suo male,
Alsrimenti non torna, fa i suot conti, Vedendo che 'l nimico se le accaita,
Che fara ben ch' ei vada a trouar Lei, Tre (caglionc'ba la porta,a un tepo fale,
Come faceua Macometto a i montis E gli da nel moftaccio dell! imposta «
E perch' ell ha due gambe, ed egli fei Ds poi dandola a gambe per ie soale y
( Mentre pero di fella ei non i/monts ) Senta dar tempo altempo,apigliar fofta
L arvriuerd:ne primaildestrier punge, Infacca nel falon, la done e il ballo,
GC? all entrar dé Palazzocs te lagsunge. Ed ei la segue foefo da cauallo.
O00 2 STAN-
.























476 MALMAN TIVE) 0

STANZA XXVIIL:
Appunto era seguito in sul feftino, \
(Come interuienein ee
Che due di quei che fannoda xerbine
S' cron per Donne disfidari «morte
L' un foreftiero, e /mentico pel vino he
L' a mi lafera,anch'eicenddoincorte } -
Ha Spada accato il Cortigian,ch'é l'altro,
Ma piit per ornamento, che per altro, Alle spalle

ca STANZA XKX.
In quel ch'ei morde i guati,efaquei sees © Che im

Che van de plano all arte del Adirrilto, ©

Ech'egtihafempriall'ufeioguoctes:

Dietro alla Serega giunge Calagrillo y Più des pie e

Calagrillo seguitando Martinazza entra con Lei nel is oO
che già fatto giorno ) continovavano a ballare,'¢ mette paura a
larmente a un-zerbiaello, che ¢flendosi sfidato con un suo tivale
fufle quelio, e pero si fuggi codardamente. 3
COME faceva Macometto ai monti, cioè se NON VengZORd Pr aendi si
noi da loro, che così e fama, che dicefle Macomeito, per mofti
miracolo, comando a i monti, che scenueilero gilda iui, “e veer
venivano 'dicefse; Horst: andremo noi da loro.

HA fei gambe, Cioè due sua, e- quattro del Cavallo. 0

GL1 da l'imposta nel moftaccio, Gii ferra la porta in faecia Che T
mo quel legname, che'chiude le porte,'¢ fincitre da: Launo poiter) B
Serrar la porta in faccia:, per intendere operare - fare in modo 5 the =
vicino alla porta non entri,'¢ ferrar (4 porta tn fa le calcagna, 'intendere
uno fuori di casa., come vedemmo sopra C, 3. st. 50. 'Nenehe serial
T imposta nel vifo., o ne i piedi. ae
DANDOLA a gambe, Cominciando a correre. Vedi sopra 0.4
SOST-A. Riposo.. Vien dai verbo /ofare, chee: il. Laune/ ey
re,o fifere,

FESTINO, Trattenimento di giuoco;o di ballo. Vedi ropa Ca
celi Fefino., quasi felta piccola,, come quella, che'fi fa\felle'ca!
delle grandi, che si fanno-nel pubbiico..

TRESC.A, Così-anticamente dicevafiuna speeie di allo dal qual
hoggi Tre/cone specie di'bailo, come vedremo sotto C; 11g. U
Purg. c. '10, la piglia per specie diballo, dicendot
Trescando alzatol' umile §. rosacea
E nel presente logo e prefa per adunanza wi. gence', che'
che la piglia il medesimo nell' daf.C.14.

-  Senza'viposo mai eralatrefea
Da trefea; trefeare, ches' intende operaré; e Tre)
telle, che vuol dir cose di poco prezzo, o stima. Vedi a
£ANNO da Zerbino, Fanno dei bello, e del galante,

ao Pe oes SP THE SS Es

-

a aie aa Ae






¥

DECIMO CANTARE 477

\ TVTT AY architettura's ec. Vuol dite, che quel tale usava nel veltire ogni ar~
te, € s' aggiuftava con ogni maggior lindura, diligenza,edifegno.
GONFIO, Alticro, e superbo per la sua bellezza, come fa 11 Pavone,. che al

i detto delle persone più semplici,' gonfia perché si stima bello; donde poi pavonez-

giarsi, che vuol dir considerarsi, e vagheggiarsi per bello; E questo verbo <(pri-
'Me quel che vuol dir i/ Poeta nel presente luogo.

CREDE turar le Dame in Veffunio, Crede far perder 'tutte le Dame per il suo
amore. Crede, che la sua bellezza sia per far' ardere del suo amore 3 e Vefusio &
il monte del Regno di Napoli, dove sono le voragini di fuoco.. rf
 HA paura del dilnnio, Cioè del diluuto delle percofle, le-quali spengono amor
nel cuore, e ' accendono nelle spalle ma differentidimo.

» VAN deplano all' arte del Mirtilio. Son-douute,-¢ si richiedono all' arte dell' ia-
namorato, da que! Mirtillo introdotto per innamorato dal Guarino avila fuss
Tragicommedia incitolata Pafforyfide.

HA gli occhi a' mobi, Bada, oflerua, sta vigilante. E diciamo «' mochi, e non
allfaltre biade di maggior valore, perché essendo i Mochi cibo proprio de i Co-
lombi, sono da' essi prt, che l'alcre danneggiati quando sono di poco seminati,
€ peroé necessario haver J'.occhio, e badare con piit attenzione a i mochi, che

Ll alte biade.

[pochi, Detto ironico, 'che significa moltissimi.
ha più cnor dun grille, EB' codardo, non ha animo, Sotto C.11.z9.dice,
Han facte di Leoni, e cnor di scriccioli, Appretlo i Greci per il contrario trovafi
Thymaleon, cioè Cuar dt leone, per vomo valoro(o, forte, cortaggiofo.

FA più capicale de' piedi, che del ferro, Si confida pil ne i piedi, che nella sp2-
da; cioè Mimd più ficuca dife(a quella del fuggire, che quella dell' armi: e circas
queita voce capit aie. Vedi sopra C. 7. It. 82. e C. 8. st. 6.

STANZA XX&K1. STANZA XXXIL
Toffo tornando l' amicizia in parte, Prima, che tra costoro altro ci nasca,

Si viene allarmi, che ciascuna armata
Ciò tien del altra un segno fatto adarte
Per darle atradimento la-pierrara:
'Di qui si viene a mescolar le-carre,
Tal ch' in vederlatante scompigiara,
Rittrandosi a dir badan le Dame:
Baha basta; non più; dentro le lame.

£ che la rabbia affatto entri frat cani,
E ms conuien fattar di palo in frasca,
E ripighar la Storia. del Garani,

Chre dietro a far che'l Turacirinafia,
eAccio,tornato pot come i Criffiani,
Ad onca della Strega-ogni mattina
Ritorni a vifitar ta' Kegolina

Di questo follevamento ciascuna-del'e-parti prefe sosperto di tradimento,.e per=
ciò si venne all' armi dentro al medesimo falone.. 'Qui l' Autore.lascia costoro, ¢
torna a Paride Garani, il quale egli latciodopra C. 8. st. 59..

TORNO' ? amicizia inparce. L amicizia si divile; cive ritornd inimicizi

“mMeera*prima. Parre t quella; che i Latini'dicevano parter, 'cioè fetta, fazione;
'onde Parziale, cioè affezionato,difenditore. Quel che sia parte per womo di spa-
da ch' egli era, e non di lettere, lo defini assai bene Farinata degli Vberti ti vec-
'chio, 'pretio.a Gio. Villani |. 12. Volere, e disuolere; € per oltraggi, e grazie ri-
Ceuute,

DAR ta pietrata, Dar colpo mortale; o conclusivo, dare a tradimento la pic-

trata






478 -MALMANTILE &

trata è ver in quel verso di Plauto; leera manu fere lapidem; panem oftemit
altera, Che risponde anche per appunto al nostro proverbio ane y¢ (a
Sajata.; o% SW
ST viene a mescolar le carte, Si me(cold la zuffa. Vedi sopra C. 9. st, 35.
SCOMPIGLLAT A. Confula. Qui intendi, rottalapace.
LA rabbia, e fra i cani, Così diciamo quando yogliamo esprimere »
s' azzuffano indiftintamente: Ii Latino Xabies inter canes, Ee
SALT AR di palo im frasca, Paflar da un discorso ad un' altro assai
dal primo. Far digreffione. 11 Monofini dice, che con questa nostra
s' accorda quella de' Latini usata da Tertulliano, De calcaria in carbonariam..
Ma guefta s' accorda più con quell' altra,. Dalla padelia nella brace. I |
Tertulliano nel lib, de Carne Chrifti dice così. dgitwr de calcaria, quod:
carbonariam; a Adarcione ad Apellen, i ose
LA regolina, Così chiamano i Ragazzi dell' ipfima Plebe Pornia ube
ga, la quale sta aperca in tempo di Quarefima, ed ivi si vendono frittelle,
Ii, baccala fritto, ed altre forte d' untumi simili, praticata, e frequentata da' ra
gazzi, ed altre genti vilissime, come era il Tura, che spesso v' andava.
STANZA XXXIIL STANZA XXXV.



Paride giunto in mexxo ai cafolari,

Ove meffer Morfeo aun tempo folo
Fa dir di sia molts in Pian Giullaré
Strepitando fugeir lo fece a volo,
Sicognun deffo vanne a'fusi affari,

Ed ci, che Star non vuol quivi a piuolo

eAnzi dare al negurio [pedizione,

Domanda di quel luogo infor marione,
STANZA XXXIV.

Vn gran Villano, un bnom acta matura

De' Quarantotti li di quel Contado,

Che perché ei non ha troppa [effitura,

Ed e profontuofo al quinto grado

Junanzi se gi fece a dirittura,

E concerts (uoi inchin da Fraccurrado,

Benevenga disse, Vostra signoria,

E Le buone Calende sl Ciel vs dia,

Jn quanto al Lupo egli e un! animales
Aa che aninial dich? io bue,
Via fiftol ds quei veri, un faci
C' ha fatto per sngenito gran dant,
E già con i forconi, e con le pales
J popoli affilliti rurto mguanno
Quin' oltre gli enno feati tutti riete
Per levar questo marbo da nn

STANZA XXXVL

Ma gli e un fetanalfo foatenato,

Che non teme legami, ne ea.
S? e carpito pits voilti 5 ed ammagliatty
Ed ha ricifo funi tantogrofe,
Le bastonare non gli fanno fates
Chie' navha abriga rocehechethaledt
D' ammayyario co' ferri non c' e viy
Cb' egit e come frncar n' uaa matity



STANZA XXXVIL



La entro a quella felua ei si rappiarra, Che tutti gl' animaliycht ei raccats

* Perch' elia égrande,dirupata ye fitta y Cudfando gli trascina lvirittay
wacciocche nimu.un tratto lo cumbatta,, E chi guatar poteffe; io.fopenfierd
Quand egli ha dato a'Socci la sconfitra, Chie' v' habbia fatto a' ofa uns "

Paride entraso ne i Calolari di Montelupo trovo, che tutti dormiyand,|
con firgpitare fece (uegliargli, ed havendo caro di sbrigarsi, proccurd
intormazione da qualcuno delle qualita ed abitazione del Lupo, ¢s' ak W
un Villano Sateapo del paefe, che gliene diede puntual ragguaglio. Ecol dif
fo., che.fa fare a questo Villano, mostra il modo di parlare del cont
KODZCp










tel on oe Ge Cee:

i el i






ee.

tak



Zo
7
a
a
q
i
5
j
:

%

DECIMO CANTARE:? 479

CASOL ARI, Intendiamo più case insieme in campagna scoperte, € spalcate;
qui intende di Montelupo, il quale se bene e Castello, ha pil figura di Cafolares
per esser le Case cutte quasi rovinate, e distrutte.

MOREFEO, Favoloso Ministro del Sonno, il quale i Gentili tenevano, che a»
i comandamenti del Sonno suo padrone si trasformafie nella facia, nel paflare, €
ne i costumi in qualfivoglia vivente, e però fu scritto: Hominum fittor Morpheus',
bestiarum imirator. Ed altri, Atorpheus, © varijs fingit nova vultibus ora, detto
Morfeo da Morphe, che in Latino vuol dire forma, faccia; onde noi Smorfie,
per brutto arto, o gefto fvenevole, che si facial particolarmente col vio. E
roan in furbe(co; mangiare. Qui dal nostro Poeta Morfeo, e preso per'lo Nef

fo sonno. \

FA dir di si a molti in Pian Ginllari, Fa dormir molti; perché colui, che dor-
me senza posar la tefla, l' inchina, e fa con efla il medesimo atto, che fa colui',
il quale con efia accenna di dir di si. In Piaw Gintlaré intende nel letto, che anti-
camente'fi costumava il dire. / vo in' Pian Ginllari per intendere, io voa letio,
© mi pongo gil a dormire: Ma questo detto come oggi poco usaco è ancora poco
inteso. Per altro Pian Gindari & chiamato un Borghetto di Case nel concorno de'
Vilage di Firenze non troppo distante dalla Città, che anticamente era de'Giul-
Jari cafata Fiorentina. Giullari, e Giulleria, dal Latino iaculares, vuol dir butio-
ne, e buffoneria, o allegria. Vedi il Varchi nei suo Hercolano; ed il medesimo
nelle Stor. Fior. lib. 15. Won gridavan con quella fefta, e ginlleria ch' eran soliti.

STKEPIT ANDÒ fuegir lo fece 4 volo. Facendo romore, fece fuggir Morfeo,
cioè sveglid i popoli.

NON vuol far a pivolo, Non vuole star' a difagio aspettando; diciamo: Tener
uno a pivolo, quando lo facciamo aspettar più del dovere, o pil di quel che egli
vorrebbe, quasi che egli flia legato alla nostra volontà contro a sua voglia, come
si fanno star legate le bettie a i pinoli, che (ono pezzi di bastone, che fitti per le»
mura servono a i Contadini per legarvi le bestie.

DE' Uuarantotte del contado, De i più riputati, e Aimati del paefe; perché il
Quarantotto in Firenze è la dignita Senatoria, la quale e il maggior grado, che
godano i Cittadini Fiorentini.

NON ha feffitura. E' huomo ardito, e libero nel parlare', non ha vergogna, o
-riguardo o timore, che lo ritenga; e s' intende anche Vn' huomo, che operi, c
viva inconsideratamente, Sefirwra chiamano le Donne quella filza di puoti radi,
che son solite fare da piedi, o nel mezzo delle loro vesti per farle divenir pib cor-
te, © per aliungarlo con sdrucire detti punti secondo, che torna loro in acconcio
dal Latino /ectura, come vuole il Ferrari, Le Romane moderne la dicono ritrep-
pio, quasi piccol ritiramento della velte, ed e lo stesso, che imbaftitura, che ve-
dremo sotto C, 12. st. 33.

PRESONTVOSO, Più che ardito, e poco men, che impertinente: Vno che
prefume assai di se medesimo, e s' arroga piii di quel ch' ei merita. Vn' arrogan-
te. Daa. Purg. C. 11, dice.

Bd e qui perch fu prefontnofo

DA Fraccurrado, Da Fantoccino; da burattino; che intendiamo quei bam-

bocci, che dicemmo sopra ©. 2. st. 46, 11 Bini nel Capitolo del Bicchicre <
Kuch











MALMANTILE

Questi perché son grandi, ancor son belli
Sends poca betta senza grandeRra y
\ wei paion Fraccurradi y¢ Spivitellig
Tra' canti Carna(cialeschi vi e un canto intitolato. Canta, ni

Fraccurradi, e Bagattelle, ove sono descritti, i giuachi, che
© giucatori di mano con tali legnetti, e burattini, detti-Frac

LE buone Calende il Ciel vi dia. Virconceda il Cielo, tutti i.
dia ij buon' anno.

SVE di panno, Sciocchissimo ch' io fone, Io ho manco giu
dicenci. Vedi sopra C, 6. st. 98. Lyf

VN fistolo. Le nostre Donnicciuole intendono Demonio, Diavolo. Vi
male maladetto, Bocce, gior. 7, Nou, 6. dufino a tanto, che il fiftolo us
Juo marito. Così detto dal filchiare de' ferpenti, a' quali egli e affo

F AC/MALE, Huomo maligno, e da fare cout it
lefactor, Cavalcanti Storia lib, 9. cap, 11. Cerri huomini besti
i quali mai alcun bene fecero, e now hanrebbono saputo farne y huomini faci
futili, n't

PER ingenito. Per naturale instinto, che questo vuol' intender quel Ci

eASSILLIT I, \oucleniti, adirati. L' Affillo è un vermicello volati
alla zanzara, ma pill grande, ed ha un forte, e lungo pungiglione
quando il Bue e punto, entra in grandiilima smania,¢ tem eda qu
tadini quando vogliono intendere, che uno è in collera dicono; Eel:
o¢ afiduo, Sula in Firenze ancora questo termine, ma per ischerzo, y
con ammogliati con i quali farebbe termine ingiuriofo, quando non fulle usa
in burla, perché e un dirgli Bxe, ve hfe

¥GVANNO. Quest' anno, Vedi sopra C. 6. st. 92. alla voce auannotte,

SMINOLT RE glienno feati tutti rieto, Qui intorno gli sono stati cust dietro ct
cando di pigliarlo, Enno, e la terza persona del numero plurale dell' indi
del verbo essere, hoggi poco usato in guefla forma fuor, che da i contadini;¢!
uso Dante Parad. C, 13, me

Non per saper lo numero, che enno oot

PER levar quefio morbo da tappeto Per levar queita pefte, e questa tribolazion
dal mondo; J sappete serviva già in Firenze per firato ai Supremi
quindi /euare uno da tappero figuibca levario, o privario di quella dignita
quale e posto, che por pafiato in proverbio yuoi dire privare, © levar uno
qualfivoglia luogo, come qui che s' intende levar dal mondo,

SET AN ASSO, Satana; Demonio, dai Latino Saranas,come
nuovo teflamento. Appelliamo Saranafo uno, che sia fiero, ¢
scrua di tal jua forza per far del male: e usato però dalle donne contro,
ciulli fieri, e vivaci, 1 quali chiamano anche WVabifi. In Ebraico:
onde il nostro Dante. i? acai

48a
















Pape Satan pape fatan aleppe. ) aries

Evuol dire Aduer/arins, Aduer/arins nofter dsabolus, ate
CakPITO. Cioè pigiiato con violenza, dal Latino carpere.
i Contadini. 'sili



zeseseEer: |

ae pee ee ee

— a


ees

=

fet

he

ete

RERt ES

DECIMOCANTARE: 481

2. Vedi sopra in questo C. st, 18. il termine santo di cuore,
NN git fanno fata, Non gli fanno male, o danao

'TANTO

 NON? ha 4 briga tocche, che Ube feoe. Subito, che ¢gli ? ha toccate gli pal-
fa il-doiore, non stima 'e percoise. Quando i Cant hanno toccato delle bastona-
te si squotano, e restano di guarite, che e indizio, che non fentono, O non cura~
no più il doiore, e di qui viene questo significato di squotere |e bufic, e ne hab-
biamo il dettato Tw fai come i Cams, es' intende cu (quoti le bufie, che significas
Non le cur:, non le senti, non ne fai thma, ec. Vedi sotto C. 11, tt 44,

MACT A. Con Vi longa. Monte di fatii dal Latino Adaceria,

Sl rimpiatra, Sinaconde, Vedi sopra C. 9, fh. 5,

 dVia40.. iano « Latino nemo. Won sopra C, 7. st. 89.
. £0 combatra. Gli dia noia;! impedisca,,

QVAN DL egis ha dato a' Soccs la feonfitta, Quand' egli ha meffo fortofopra, o in
contufione le mandrie', cioè fatti fuggire i bettiami afialtandogli: Che Socciv.s'in-
teade quel beltiame, il quale si da a un Contadino per far' a mezzo del guadagno,
quasi dica a Sccio, cloe a compagnia. L'azione, che nasce dal contratto di So-
Gita, si domanda da' Legifti Azione Pro focio; Ma noi per Seccio intendiamo
una focieta, o compagnia particolare, ovvero una Accomandita di beltiame, che
si.da altrui., perché lo cuftodi(ca, e governi » a mezzo guadagno, e perdita. So-
Zi /poj pure dal Latino Sectas intendiamo quel, che i Latini dissero /edatis iures
Sodalitijs iunetus, o Buon forse dichiamo a colui, che non guasta mai, e che acco-,
da le conversazioni,

CA' ei raccatta, Ch' ei raduna, Ch' ci trova, e piglia,

CIVEF ANDÒ. Cioè¢ pigiiando con voracita; rubando.

LU ritza, Cioè in quel luogo li, Termine ruttico, Dal Latino #i rea, Qui-
via diritto; in quella dirittura, 0,, come 1 Franceli dicono, en cer endroit,

10. fo pensiere ch' e v? habia fatto a' ofa un cimitero, lo credo ch' ci v' habbia ra-
gunato una gran quantità-d' offa. Che Cimitero diciamo 1] luogo, dove si forter-
Tano imorty. Vedi sopra C. 4. st.2g.¢ C. 7. tt, 27,

STANZA 'AxXVill
Sta Paride afenttrio molto attenta y

Ada pai vedendo quant' ei si prolunga

. Frafe dice; Costui cs ha dato drento
Come quel che vuol far mela ben lunga,
Gli e me troncargli qui il ragionamento

 checio prima, che il ds mi sopraggiunga
40 polfa lasciar l'opera compira,
Peri gis. dice: O via falia finita.

STANZA XXXIX,

Poi ch' egls ha inteso dow' ei possa bartere

e4.un diprefja 4 rinuergare il Tura,
Lell'efer foleo il bofeose a' altre tartere,
Che gli narri costui, saper non cura:
La laterna apre,e il libro,od'alcarattere
Poa, vedendo., dar' una lettura,
Così leggendo senti darsi norma

Di quanto debba fare, in questa forma,

STANZA XXxX,

Vicino al boschereccio Scannatoia
Mentr' il froco di stipa vi riluca,
Palton grosse, Bracctale,¢ Schizzatoia
Co' Gucators a palleggiar conduca;

Ai rumbombar del suo diletto quoia
Toffe vedrà, che 'l Gocciolone sbuca
Keuei ricchi arnesi vago di mirare,
Che già in Firenze lo facean gunfiare >

, Sta Paride attento al discorig dei Villano;ma conoscendo ch' egli era entraco
ip on discorio da non finir mai, lo fece chetare, e preso il libro, da cflo compic-
BEDS tate;

se quel ch' ci doveva fare.

co.






482 MALMANTILE

COSTVI ci ha dato drento. Costui & entrato in un discorso da non'
fine; e me la vuol far (unga, Cioè vuol far' una lunga diceria,
OVVTA, E' \o steBo, che ors. Latino' Eia age. Termine, che
spedizione. ms " i ast
DOP' ci può battere, Cioè da qual parte egli habbia andare per ir:.
Tura. Tey et
APN diprefso, Alquanto vicino a dove egli sia. Si dice o a ited “tid
vel circa, Dal dirsi per esempio: Furono tanti, quanti io v' ho detto vel cireay
Cit, o in quel torno, haa alle se
RINVEKG ARE. Rinuenire; Ri; Ri iare; Raccapezzare. © ist
ALTKE tattere, Altre zacchere, minuzie, © circoftanze di poca considera- a
vione. Se ben Tattere per (cherzo s' intende una specie di malore, che viene in 4/4
torno al stesso per crescenza di carne. i ?
CARATTERE, La forma, o figura delle letcere dell' Abbiccl. Voce latinas
tolta dal Greco Character, ¢d i) Monosino vuol che itia lio dir carartolo, ma si
non fo per qual cagione, se non fufle per allontanarsi dal Latino, che per altro te
non ho letto tai, ne sentito dir carartofo, se non a qualche Villano del tutto ru- ne
fico. eee
SCANNATO10, S) intende il luogo dove s' ammazzano i buoi, edaltrebe>
flie, ma qui intende quella felua, entro alla quale si nascondeva il Tura,¢las Ki
chiama scannatoio, perché quivi il Lupo scannava le bestic,
BRACCIALE, Manica di legno dentata, della quale s' arma il braccio pet laf
giocare al pallon grosso. Vedi sopra C. 6. st. 34. any Ta
SCHIZZ AT O/O ( gui intende il piccolo). Strumento d* ottone, o d' altro Ne
metallo fatto a foggia di canna da crilteri, ma assai minore, e serve per metter '
vento in qualunque luogo con violenza, come si faa gonfiar palloni,0 pillotte, wt
o per (chizzar liquori; e 'i maggiore per far serviziali. Latino e/yfer detto così,,

quasi frumento inondante, e lavativo. Vedi sopra C, 3. st. 14. Che

PALLEGGIARE, Dare alla palla, o Pallone, mandindoio, e rimandandolo Che
per tra(tullarsi, e per avviare 1) giuoco; ma non giocare regolatamente, Onde» Ry
quando uno tira ia luogo un neguzio, coll' avviare chi glielo raccomanda, 2 un? Cad
altro, e che quello lo rimanda, al primo, e tutti due si accordono a burlare il Tes,

pover' huomo; si dice + Tra loro se ¢a palieggiane; che in Latino forse i direbbe Giaj
Coludunt, ast SOE?
GOCCIOLONE. Si dice a uno, che ta guardando una cosa con grande atten- Tu

zione, e con desiderio a' ortencria, e propriamente si dice di quelli innamoratt » Beri
che stanno i giorni interi appit d' una casa a guardar la dama, che € alla finellsa, Ng
¢ si coniumano, € si struggono a poco a poco, e per così dire a filia a flilla, © vu
però dice Gocetolone al Tura, e vuol' esprimere, che egli era innamorato di que L
guarnefi. Lucrezio lib, 4. Pariando degl'innamorati. ee fog
Wamque voluptatem prafagit multa cupido, Ri
Hac Venus eft vobis, bine autem est nomen amiorit; - Ca
Hine ila primum Veneris dulcedinis in cor B g

Stilauit gutta, © fucceffit frigida cura, ?
CHE già lo facean gorfiare, La voce gontiare vyol dire Andar superbo, comes oy

4
ra
53,




DECIMO CANTARE.
dicemmo sopra in questo.C. st, 29. sed il Poeta (cherzando con l'equivoco di gon-

fiar

3 ma in effetto vuol

483

Ic pillotte, e palloni, che era il mefticro del Tura, come acccnnammo sopra

Gf st, 47. pare, che voglia dire, che quegli arnesi eran caula, che il Tura (eo

andava sup poi dire, che quegli aracfi eran caula ch'ei

J Sontava le milous » ¢i palloni, e che egli gonfiava la pancia, buscando per mez-
zo

imi arnesi da comprar roba per empictia.

— STANZA XXXXL
Paride in soofe fatice ubbidisce 5
Accender fa le feope, e intorno al si
 Già questoje quel st spaglia ed alleftisce
 M faa braccialeye si comincia il gimoco;
Al suan del qual! Amico comparifee,
M4 ritenuto, perch' e vede il fuoce,
 Elemento, che vien dali' animale
Fugritaper instinto naturale.
STANZA XXX AIL
NGarani che fava alle-velette,
 Fedendo che'd Compar viene alla cefta,
Che te feope si (pengano commerte y
 Edin un tempo a i Giscator da fefta:
WD un batrer docchio il giuoco si difmeste
La fipa si sparpagha ye si calpefta;
Tal che sicuro t' animal ridotto,
Va Paride pian piano, e fa fagotto.
5

STANZA XXXXIII.
Ciò ch'é in giuoco in nn fascio egh ravvia,
E tra gambe la ferada poi si caccia
A tutto firascicanao per la via
Con una fune a otto, o dieci braccia.
Spinto dal genio a quella ghiortornia
Da lunge il Tura scguita la traccia,
Come fa il Gatto dietro alle vivande,
E il Porco a' beveroni,ed alle ghiande.
STANZA XXXXIV.
Vaghecgialo, s'aliunga, xappa,e mugola,
Talor 8 appressa,econlexampe iltoces,
Hor mostra shavigliando aperta l'ugola
Hor per leccarlo appoggiavi la bocca,
Tutto lo fina, lo roniftia,e frugola;
Così mentre il suo cnor givia trabocca
Ej, che non rocea per letizia terra,
Entra nel Borgo, e in gabbia si riferra:

TANZA XXxxv.

Perché Paride fa ferrar le porte,
E poi comanda a un branco di Famigli,
Che quiui farti bauea venir di Corte,
Che di loy mano l' Animal si pieli;

Ma i Birri, che buscar temean la morte
Non voglion accercar simil consigli,
E fan conto ( se ben' ei fa lor cuore )
Che @ paffi cutrania 2 Imperadore,



Paride in ordine a quel che trovd scritto nel libro datogli dalle Fate, fece acc
cender il fuoco d' avanti a) bosco, ed attorno vi messe gente a giocare a} pallo-
Ne: a quel romore il Tura ulci dal bosco, ed allora Paride fece un falcio de'brac-
Ciali, pallone, ed altri arnei, e legatolo a una fune lo fece strascicare per las
scada, la qual conduce al Castello di Monte Lupo, dentro al quale i conduffe il
Tura, seguitando quegli arnesi, e Paride fece ferrar le porte, ed ordind ad alcuni
Bi tri, che quivi haveva per questo fatti venire, che lo pigliaffero, ma essi impau-
titi non yollero accoftarsi.;

C4ALLEST/RE. Metter' all' ordine: Approntare

L) AMICO comparisce. Cioè il Tura esce dal bosco,¢ vien fuora spinto dal gu-
sto di vedere il pallone.

RITENVTO., Renitente; cicé non alla libera, ma con qualche timore per
¢aufa det fuoco:, del quale il Lupo n1cura'mente ha timore.

ST AVA alle velette, Stava osserv'ndo. Vedi sopra C. 7. st. 67. It Burchiello
nella Novella del Medico Bolognef=. e dello Scolar semplice dice: Andando ¢ri-
dando cerci tutta ia casa, e tronarlo non gli fu ordine, onde tratte dalla disperarione si

Ppp 2 parti,


484 MALMANTELE:
parth, e lo Scolare, che flaua alle velette Vitorhato in'cafay ec,
. IL Compar viene alia cefta, Cioè Animale vien fuor
zimbello de i braceiali 5'¢ palloni 5 ec, iy HgOWs
DA sia ai Giuocatori, Ha vestardi giocare; Licenzia iG)
agli Scolari vuol dir Licenziar la Squola, e di qui dicendosi dar,
cenziare ogni sorta di lavoro, Daag
IN un barter ad? occhio, Inun momento. 1 Latini pure ditono-Jr%è
SPARPAGLIATE, Spandere contufamente, e ae ee
come si fa della paglia, quanido'si batte, e si spoglia il'grano'. 1 Pulei dite:
Sopr' alle spallela treccinsperpagtia., 0
FA fagotto, Fa un fa(ciode rbracciali, paltomt ec. Par fagotto; e 10!
quasi, che far le baile per bacterfela, per andarfene. Latino v4/a colligere. ~
Sl caccia la via fra gambe, Comincia a camminare. Latino + viem,
SEGVIT A Ia traceia, Seguita, o va dictro allapefta, oalla sed étol-
to dai Bracchi, i quali si dice/egwitar /a traccsa, quando mel cercar della %.
ec. fiutando seguitano quella firada, e quel tratto', per dove ella ha tirato;
per dove e paflata: di qui habbiamo il verbo inzracetare-detto sopra C, 7. st,
BEVERON!, Così chiamano i -nostri Comtadini quella bevanda grofia fatta di
crusca, € d'acqua, ec, la-quale danno a i Porci. Vega eh aie
LO vagheggia, 'Lo guarda aftewuofamente.. Sivaledi-questo verbo vaghectiog
per esprimer il gulto, col quale 1) Tura guardava quegli arnesi;:essendo tal ¥
proprio degl' innamorat', Vedi sopra C. 7. st79. (4 aay
MVGOLARE, Buna voce indistinta, e che non finitamuore fra i denti.
ROVISTIARE; Ravoltolare, netter soflopray Forte aeglio-romifia dal verbo
rovistare, che vuol dir Muovere da un ldogo all*aliro. Ji Pulci., Morgante vas
rouiftando ognt cosa.. Hh Wx
PER letizia non tocca terra, Sopra C. 9..st.63, Per V allegrezza nom può*star
n¢ i panni, 'che € lo stesso; e figaisca haver'aliegrezza', o gufo grandissimo; Si
dice ancora; ma in modo batio, Lacamicianon gli tocca il sedere, Ml Boccaccio
Novella 32. iam
FAMIGLI, Qui sintende Famighi di Giuftizia,cio' Birri;la famiglia debPode-
fla,dal Boccaccio detti fergents, quali ferxientes, siccome'da noisfamigli,cic' fa
FA conto y che pafii 'dmperadere,, Finge di-non intendere, o-di nom lentire qu
che si dica\, Detto forse questo dal tempo', quando'era l'dmper: rec
vanni Paleologo-in Firenzeal Conciiio »che per cfersi già tata familiate la un
vista, e forse, mancandogli i danari., non comparendo:così pompola, ne Cos
bella compagnia; e appagata anche dalla prima volta in fu, lacuriosita; quan-
do paflava per le strade, non doveva far muovere la 'gente come prima, e come
andò egli arrivd; Onde si venne a dire, quasdo uno non si cura di qualche co-
f: Facciam conto, che paffi lo Laperadore, t a7%8 one
: ST, AN 2, AoREXKEV End OVS
Poiché gran perso ha i porri bapredicate, Senza pin (har a burtear via il fiato,







E che fan conto tuttania cb'eb cantiv, Totti di mano abc. iiguanti,
Pero che.da i Ribaldi gli vien dato: Bisogna, dice, con quefia canaglia
Li udienz.a, che da il Papa i furfanti, Far come il Podeftdidi

'AN:










DECIMO CANTARE: 485

ZAXXXXVIL ©» STANZA XXXXVIIL.
ds caps Si resta il Lupo, e'l Tura buomo diviene;
Ma non pero, che libero ne sia,








ad una delle [ue legacce




a addosso al' Animale C' ambi fone appiccati per le rene
eee a uso di bifacce: Formandoun Leong ha e la Bugia,
r di tal.concia dé cauiale Dice Turpino,e par ch' et dica bene,



Ch' essendo questa si crudel malia,
ina di iupo,ed una d'buomo fembra, Lon erano.a disfaria mai bastani
di sua [pecie oguunna ha le/ue mzbra, » Gli odor birreschi semplici dei guanti,
4)! STANZA IL
opri tal mafferizia E Paride, che gra ' chbe notizia
molto pin fatto le mani, Da quel suo libro,si da quint ai cani,
 Percheglincants in man delaGinffizia Perché pin oltre il libro non ispiega,
i fichi- alla nebbia vengon van, 'Ona' et fa conto al fin di tor ia lega,
ide veduto che i Birri non ubbidivano, ed havendo per avvertimento dal
bro datogli dalle Fate, che gl' incanti rimangon vani iv mano della Giuftizia,
sdiede a credere che haveffero tal virtù ancora i guanti dei birri, e per questo
f eae al Caporale, e gli mefle addosso alla bestia, la quale si converti
Induce corpi appiccati insieme, che uno d' huomo,¢ I altrodi lupo. A tal me-
tamorfofi resta Paride stupefatto, e non sapendo che cosa farsi, perché il libro
bon inlegna da vantaggio » risolué di chiamar due fegatori per(eparar It Animal
bruto*dal razionale. In questo mostro il nostro Poeta imita Dante nell' Inf. C.
 25. nella commiftione di que! Serpe con ' anime di quei cingue Cittadini Fioren-
i € la delcrizion di tal mostro comincia al verso: Se tu fei hor Lettore acreder
dente,
 PREDICARE #3 porri. Predicare al deferto,, Affaticarli in-vano a esortares
uno.a far bene, che i Latini dissero vento logui; Surdocanere.
 PANNO conto ch' ei cantiB lo stesso, che dar Pandienza che da il Papa ai furfanti
che ia fuiteza vuol dire n6 fare stima delle parole d'un0,0n6 badare a quel ch'es dice.
CAPOXKALE. 'Capo di squadra di birri, Grado che si di anche sia i Soldati.
Vedi sopra'C.'9. stan. 2.
 BAR come il Podestà di Sinigaglia, Cioè comandare, e farda se.. I) Duca di
Calauria Sigifmondo havea aflediato Sinigaglia,nella qual Terra era per Gover-
shatore foltituto da Gio: de Castro, Petruccio Piccolomini; Costui tentd di ab-
' la Terra, dicendo esser. meglio uccello di-campagna, che di gabbia,
¢d a lutaderiva il Podefta, ma i Cictadini featendo questo dissero di volergli get-
“tare dalle fiaeftre se più parlavano d' abbandonare la Città, e vennero tanco in
odio ved in disprezzo de i Cittadini, che. quando comandavano noa erono ubbi-
Giti, edi qui venne il Proverbio: Far.come it Podestà di Sinigaglia, cioè Coman-
dare, e'far da se, Cavalc. Scor.:
 Deeaa + S'intende quei Jegami, con i quali ff legano le calze, cingendo
ambe.
MSACCE, Così chiamiamo due facchetti appiccati "uno contro all' altto a
'due cigne, i quali si mettono a traver(o ai cavallo, ec. sopra il quale si cavalca.,
'€ servono per porcar robe, come si fa con una valigia, (ono appellate —







me



















vit






486 MALMANTILE

bis facche, due volte facche, o facche a doppio. Lat, AMdantica Bocce,
nov. 10.5, Haveva Frace Cipolla comandato che bea guardafle, che let
»» (ona ada toccaile le cose sue, e (pectalmeace le sue bilacce nelle q
»» cose rare. B pil otto nella medeGa novella. La prima cof che venac
» prefa fu la bifaccia, agila quale era la peana. weet
CONZI 4A. Quando Gi dice coacia di guaati s' iateade profumameato,

si dice guanti di coacia di Rona, di Venezia » di Spagna 7 ec. e 8 intend
mati alla foggia di Roma, ec. Qui dice concia di Cauiale y cioè feteatt »
fragore, o feagraaza e Detto ironico., Were ta
LA Sugia. La Bugia Gi figura uaa Femmina con due facce differenti, comes
@' orso 0d' huo ny, o di lupo, e d' huomo, come è aci prefeace luogO,
DICE Tarpino, Scherza cone fa sopra C. 2, stan, 31, autorizzando en

te (un Novella com i detti di Turpino, come fa ? Aciofto. lve
MALIa, Iacantefimo. Suregoneria. Vedi sopra C, 8, stan. 52. Donde 44-
liarda una strega. iy
T AL maferizia, Iatende i guanti del birro. cust cee
Sd aicani, S'adica, Quando uno per la stizza grida, e fa altre dimostra+
zioni d' impazzienza, o di rabbia diciamo; Si daa' can, Vedi sopra C. the 10,
STANZA L STANZA LiL...










Per ciò fatti venir due Marangoni y E morta re la dd per cofacertay =
Con tutto quell' ordingo, che s' adopra M4 quel Demonio insieme firappicch,
eA fegare i legnami, edi panconi, E qual porco ferito agolaaperta

ef dinider il Moffro metre in opra;
Mitre la fegaim mero ai dusigropponi
Scorre cosiva il mondo fortofopra
Mediante il rumor de i due parrienti,
Che un fa d' urli el altro dilamenti,
STANZA LL
Pur senza ch' inraccato elit habbia un offo
La [ez infino ail' uitsmo aileefe
Lasciando il Tura libero, ma rosso,
Dietro ds sangue com' un Genone/e;
La Be(tia gli volea tornare addosso,
Ma Paride, che (ubito ? intese
Prefa la (pada la cagio pel mexrd
Pensando di madarla un trattoalrezro,



aa
Per dinorarlo forte se gli ficeay - -
Ed eslic! alt' incontro stan. all! eta y
la [a Latefta un sopramman gli appicedy
Ch in due parti diuifela di netto
Com' una tefticcinola di capretto.
STANZA Luh
M4 ritornato a penna,¢4calamaio
Pur quello Heffo a Paride si volta y
Che per veder il fin di quel mofeaio
See' fulfe mai possibile una volta y
Mena le man chee pare un Berrettait,
Ed a chius' occhi pur suonas r4ccilta
E dagli, e picchia,risuona se mA 7
4a forbice, t e sempre bella.

Paride fatti venir due Segatori d' affe, fece fegare il Mostro in fu It artasatu-
ra deli' huomo con la beltia, e così gli fepard; Ma la Bestia tentava di
carsi onde Paride caglid la Bestia pel mezzo, ma eifa presto firappiccd 5 B qui
il nostro Autore immita l'Ariofto nella favola d' Orillo; levata da Vergilioacil
Eneide, che finge un tal' Erillo Re di Paleftrina che haveva tre anime, onde era
necessario tre volte ammazzarlo per finirlo a.

tHARANGONT, Si dicono i Garzoni de i Legnaiuoli che lavorano peropra,
quando in una bottega, e quando in un' altra a tanto il giorno, e non jn
una boticga a falario di tanto il mese; ma qui l'Autore intende Segatori di le-



goami,


a ese

—————

eee

me

=

- +e







DECIMO CANTARE. 487

| guaind 3 €gli ordinghi, che ? adopra, sono la fega a due mani, lima per metteres
} Gags denti, e il cavalletto per adattarvi sopra quel materiale, che i dec (e-
ox. cavalletco si chiama pietiche. Vedi sopra C. 6. stan. 6g. alla voce im-

 PANCONT. Sono afi grosse circa un quinto di braccio, le quali si rifendouo
per farne o affi più sottili, che si dicono panconcelli, o per farne correnti.

GROPPONE. S' intende la parte di poms di tutti gii animali, o bipedi,o

yadrupedi, e lo diciamo ancora codione, ed e propriamente quella parte che re=

fra le natiche, e le reni. Vedi sopra C. 6, fan. 69.

VA fottofopra il mondo, Lo strepito confonde l'universo. I Latini pure dicono
Mundi fumma readit ima, © ima fumma;¢ vuol dire, che jo ttrepo era gran.
'dithmo per le firida del Tura, e per gli urli del Lupo.

 ROSSO come un Genoxeje, &* in Firenze una Compagnia, o Confraternita di
Secoiari detta de' Genovefi, perché e formata di gente di quella Naziwne s Co-
storo hanno per coitume d' andar proceflionalmente la fera dei Giovedi Santo as
vifitare le Chiefe, si battono le reni ignude con mazzi di corde entrovi alcune
ficiie di metailo acute come quelle degli sproni, e queste forando la pelle ne trag-
gono il sangue, il quale bagna loro le reni, ele tigne di roflo; E di questi in-
tende il nostro Poeta nel presente luogo.

. tH ANDARE uno al rezzo. Mandare uno nell' altro mondo, df fresco, cioè
il corpo suo sotto terra. Ammazzar' uno. Rezo, vuol dire un luogo dove non
arrivano i raggi del Sole per interposizione di che che sia, e si dice anche, me-
riggiv, bacio, ombra,¢ uggia. Vedi sopra C. 6. stan. 75 ¢C. g. stan. 44.

ST.AV-A aif erta, Stava ocnlato; flava avvertito. Erta si dice la talita d'un

BRIO; e are all' erta e termine di caccia, percht la Lepre ha per propria di
for sempre alla volta della fommita de' monti, per non esser così facilmente
arrivata, e pigliando i suoi riposi, scoprir paefe, e minchionarc icani; e pera

in caccia State al? erta s? intende Habbiate l'occhio, ofieruate; il che ¢

poi pafiato in dettato comune a ogni cosa.

PN sopramman gii appicca, Gli da un soprammano, che è quel colpo, che si da
¢ spada, bastone, ec. cominciando da alto, e calando a balio. Vedi sopras

5- stan. 41.

D1 netto. S' intende lo taglid pulitamente in un fol colpo,

TESTICCIWOLA, Le telte degli Agnelli, e de i Capretti da noi si chiamano
Teftucinote, e per friggerie si tagliano nel mezzo per lo lyago in duc parti ugua~
li; eda questo taglio afiomiglia quello, che fa Paride alia tetta det Lupo.

4 penna,¢ 4 calamaio, Per ! appunto. Vedi sopra C. 2, stan. 19.

VEDER il fin di quel mofeaio, Veder il tine di quetia cosa noioia. Vedi sopras
C, 4. stan. 9.¢C. 9. stan. 51.

MEN A le man ch' ¢i par ux Berrettaio, Menar le mani dicemmo sopra C, 1, st,
7. quel che significhi, e qui intende che. mcnava le mani con ceierua,come fauna
1 Berrettai, e Cappellai, che nel felcrare i cappelli, o berrette menano le mani
Prelto in riguardo dell' acqua bollente, con ia quale si fa tal lavoro,

 SVONA a raccolra, Continova a perquoter a jungo, che così suona la campa-

Ra; quando suona a raccolta di popolo per le prediche, ¢c, ed 1 verbo fonare si

° gailca
+

ee







- — “gk ae
2 488
a
488 MALMANTILE |
gnifica anche perquotere, ¢d e della medesima natura, che il
habbiamo detto altrove. 6 ay eee
DAGLI, picchia, risuona,e marvella. Questo di dire
re uno, che adopri ogni fea induftria, per fare una cosa perf
do più vole le diligeaze. Vedi sopra C. 7. stan, 16. Similitudine,
tratta da' fabbri, quando Javorano il ferro sopra l'incudias; Qui:
d' Orazio incudi reddere versus, mettergli alP incudine, sotto
critica. Cioè efaminargli, rivedergii di nuovo.con somma, rigorofa
diligenza. La nottra maniera; Barrere il ferro quando è caida, ebbe
meate da questa prontezza, e macitria talieme, che si adopra per lavorat
nalmente l'dcudir degli Spagauoli, che vale aixeare, voce ormai si
è fatta dal Latino ddcudere, ciod battere insieme il medesimo ferro.
dichiamo per esempio. La prego a volere accudive « quefke megorio; © si
FORSICE.. Questo termine significa oftinazione,, per elempio. fo 2,
che tu non faccia la tal cosa; e tu forbice, cioè Tu oftinato l'hai voluta |
modo. Dicono che venga da uaa Donna offinata, e capona,, la quale
chieito al Marito un par di Forbice, e non havendogiicie il marito mai:
te,ella ad ogni cosa, che i} marito le domandava rispondeva: Forbice;
impazzientato da queita sciocca oftinazione,le proibi il dirlo
più lo diceva 5 per lo che il marito la baflond, ma. non per: ella se
maneva, ficche egii un giorno sopraffauto dalla collera la gewo in ump
cila fino che potette parlare sempre dite; Forbice, ed in ultimo goa p
valersi della voce, si valfe delle mani cavandolg fuori del' aequa con le
giori alzate ed allargate in figura di forbice,per mottrare che moriva |
oitinazione, e caponeria. Questa novella e vulgatitiima fra le nostre
io ho trovata tra una raccolra di efempi facta da un Buontempr
mano del medesimo tengo fra i miei nianoscritci. 2 eS
Lit sempre quella bella; L' e sempre quella medesima. Questo yien da un
co, 1] quale andava accartando,¢ cantava una cerca orazione al suono di un
tarrino, fermandosi alle porte de' suoi benefators i giorni destinati;
venuto a fastidio, do sempre la defima cosa, inci:


































quelli, che gli facevano l'elemosina a dirgli, che se non cantava q 'ae
orazione non gli havrebbero dato pil nulia, ed egii rispandeva; Pa
se', cht domani ve ne vaglio cantare una bella, Ma pecche il Povererto ai 4
se.non quella, tornaya l'altra mattina, e cantava la steila, laonde i f ”
fattori.accortifi, che il Meschino non ne. fapeva altre compathonaadolo, git te
cevono. L' è sempre quella bella, ed intendewano l'e tempre quella 1 ig
che e poi venuro in detrato, e significa noi fiam sempre-alie medelim a
quanto racconto ancora fra gli scriti del medcfimo Bugnrempi top z
pucato ali' origine del presente dettato. ren “a i
S. TAN ZiAisbl Moni tap 'y¢:
Tal ch' ei si scofta none, e dieci paffi, Pervia gli anuenta m
E piglia fato, perch! es pronar vuvle, i
Selavirtude a forte gli giouaffi, i;

C* hanno! erbe, le pictre, e le parole;








489
gout STANZA  i
recaffe a scorno, Resta in parata', molto gira il cnaré
alle gioftre,e alle quitaney | 'Pimceis pikes anc ielibbienicfie,
we b gli vada incorna, > Merce ch ei fache'l Diauvloe bugiardo,
EB latrartigo' faffi, come un cane; “E quanto en sia furtile,¢ filigrosso;
i, ver ch' e' fufse ! apparir del giorno, oPercia si merte un pezro a bellofguardo,
; L! Ombre,il Bau, ele Befane oCredendoognor che gli faltafse addofso,
Sparyce affatco, e più non si rinede, Aa poi ch' ei vedde omas d' ¢/ser sicuro
Ma Paride per questo non gli crede | = Andò all Ofte, e cauollo di pan duro,
Vedendo Paride, che quel Mostro si rappiccava sempre » e che ci non trovava
'modo di liberarfene per ferite, che glisdette, gli venne'in pensiero, che se era la
Werita 5 che in herbis, verbir, & lapidibus stesse la virtù, poteiic essere che alcune
di queste cose havetie virtù di fare sparire, e svaniresl Moltro; e pero preso il
 [xa dove, il quale era pieno di parole, e dliverle erbe, € de i faili ogni cosa tird
addotio a quel Mostro, e l'indovind, perch subito egli spari, ed il Tura rima-
se libero”, 'Con tutto questo,Paride non si fidando, stette buon pezzo a osservare;
ma veduro, che il Lupo non compariva pil si parti, e andò all' olteria a man-

Patt. i
Ors ' fiato, Cioè si riposa,
. MLAEST RO Grillo Contadino, 8 nota la favola'di Grillo Contadino, il quale
per fardispetta @un sue fratello Medico sche non gli volle dar parte d' un tefo-
F0, che infizme! havevano trovato, si fece Medico anch' egii,¢ con i sui forcuna-
a fiti's' acquifto la grazia del suo Re, non folo per havergli: rifanata las
cavandoie una tilca di pesce della gola con ungerle ilc,..., ma ancora
per haver saputo indovinare i fegreti del medesimo Re, e chi erano coloro, che
a lui rubato havevano, in somma fece diverse scioccheric, le quali tutte per gli
} spares fidondarono in stima del suo valore, e l'accreditarono per un valoro(o
Medico, e grandissimo Indovino, come si legge nella di Jui favolosa vita, o di-
Ciamo spiritola Satira.
WINT ANA} Bruna campanella, che si tien sospefa in aria (oftenuta da una
molla dentro a un canacilo, alla quale per infilarla corrono 4 Cavaiiert con las
Jancia', come fanno anche'al Saracino, che dicemmo sopra C. 4, than, 57. € si di.
Ce ancora Chintana, Varchi Stor, Fior, lib, 15. Fecera metrer della rena a! avanti
al palazze, ed appiccare /a chintana, Dai noltri Ragazzi e detta corrottamentes
Timana 9 ed e iatelo quel lor patiatempo, che fanno, infilando una zucca fresca
in una corda, e pottala in aria attraverso a una Arada corrono con alle la mano
@ dare in detta zucca, unmitando i Cavalieri, i quali corrono alla quintana, 0
al Saracino, Dice che Paride era avvezzo alle gaintane, e alle gioffre [che nel
Prelence inogo son fitioninu; s¢ ben gioftra's' intende quando i Cavaiieri corrono
a corpo a om 70 al Saracino, e quintana significa quello, che diciaino qui fo-
Pra) perché Paride haveva pil aout militato im Spagna, dove haveva cfercitaco
1 jor! gradi della mulizia, e tornato alla Patria tu dal Serenityaio Gran Duca
fatto Governatore deija forcezza veochia di Livorno, ed hunorato del titoio di
Macttro di Capo, I nome tuo era Andrea Parigi, fu fratello d Aifon(o, e di
Paoio detto sopra Papirio Gola, & Figliuolo di Giulio, e fu come custi questi va-
sa = Qqq jen-























-















Se

:

©

i

=i






aa






—

' Ye
*

490 MALMANTILE™
lentissimo Tngegnere, € periti archi Qui
Ferrari cusi. Ludus equeltris,cum diretta in encun fimulachrnn:
gehtat, bala incurritur, Alcunt han detto come Vguecione Pifano.s
zionario, che Già così detta dalla quinta parte della piazza yin
tri, come Balfamone sopra Fozio da un certo Quinto inventore 2ed
la vera origine mottra il Pertari essere da Comrus.cioè ee i
punta di ferro; e si raccoglie dabtitolo nel Godice:, de i Y
radore chiama questo giuoco con voce Greca Kynranos., In ordi
Chintano, e non Chincana pare, che lo chiamaile, se sha a
ma, Fazio degli Vberti nel Dittamondo.:
Gionani bigordare alli Chintani y
E gran tornei, ed. una, ed altrag
Far si vedea con giuochi nuoui se ferant. -\





jofira '
>

CALAPPOLERIE. Cosa di poca stima: oda farne poco conto i “Apine; '

triceque,¢ buttubata, V. Feito, e ivi sopra lo Scaligere.,
BAV,e Befane, S' intendono quelle Larue inveatate dalle Balie per far paura
ai Bambini, come habbiamo decto sopra C. 2. stan. 50. et
REST A sn parata, Si ferma in guardia, cioè con la spada pronta, ed in posi-
tura comoda a ferire, E' termine da schermitori. yori
MERCE', Con la prima, €;, firetta, ela seconda longa, vuol dir mercede
che profferito al contrario vuol dir mercanzia: Nel modo che:è detta nel pre-
sente luogo, ed in molt' altre occasioni mere vuol dire per causa di ciò: qual di
ca io riconosco tal mercede, tal benefizio da questa cosa, o da i,
ec, ficome Paride riconosce questa mercede, o benefizio di non si fidare del Dia.
volo dal sapere, che quello e bugiardo, ed ingannatore. Questoidetto e lo'ftelio,
che Grazia del marcello, e degli foroni, che vedemmo sopra in que(to C, fran, 20,
1L Diauoloé futtile, e fiia grofo. 11 Diavolo & fagace, ed inganna l'huomo,
facendo il goffo, ed il balordo. * inet
REST Aa bellu (guards, Reled guardando attentamente. Bello fguarde® unas
villa poco lontana da Firenze: e per la similitudine che ha questo nome bella/enar-
do con il verbo guardare si piglia in detto significato. pn amaetir
 CAPOLLA di pan duro, Mangid adai. Gii mangid tutto il pane, che haveva
in casa, gliclo rifint. Detto usatissimo per esprimere Aeangiare assai ee,
spy paler

ays Nae

FINE DEL DECIMO CANTARE. eae


































|



eee
eh ARGOMENTO, '
St

Cangia le dance in rifsa un? accidente, aS
iSe

Fuggonfi Bertinella, e Martinacza,

-VNDECIMO CANTARE,

Vien fuor Biancone, e fa morir gran gente;

5 Ma gli Orbi a tui fan poi sentir la mazza, 6%
es Da Celidora, e da Baldon possente 33
ee Mezza defirntta e quella trista razza; th

Taghanfi a pezei in quelle squadre, e in queste '“*
E così in ata? fanfi le feffe. e a ge

2 —
RAPALA AAAS AS

om STANZAL STANZA Ik
Chi mi.dard la voce, ele parole ui ci vorria chi scortica L' agnello,
'antia dir la guerra indiavolata; Es al mondoé persona pil inumana,
Ond' oggimas dara le barbe al Sole © descriver la frrage,ed il flagello
Bertinella con tutta la sua armata; Che seguir si vedrd di carne humana;
C'alCiel Gagliarde alzando,e Capriole, Ch' io già oni fento, mentre ne favello,
\Farò.verso Volterra la Calata, A tremito venir della quartana.,
. Efe d' amor canto con cetra in mano, E n' ho si gran terror, ch'io vi confefi0,
<Derd col ferro il ve/pro Sicilsano ? Che mai più de'miei di farò quel desso,

Tinoftro Poeta volendo.nel presente Cantare narrar la battaglia seguita ia Mal-
mantile, e le crudcita grandi, che (uccessero nel Palazzo della Regina, dice, che
a fac tale descrizione vorrebbe esser un' huomo fanguinario, quanto è colui, che
scortica git agnelli; che non si spavencerebbe, come fa egli acl rammentarsi i]
grande firazio, che fu fatto di carne humana in tal batcagiia. Qui immuta Dan-
te-nel principio del C..8. dell' Inf. che dice;

i Chi porrsa mas pur con parole scialte
Dicer del sangue, e delle piaghe a pieno
Ch! 10 hora vidi, per narrar più volte ?
4 mi lingua per certo uerria meno, L
— avventura seguita Vergilio nei 6. deli' Kncid., che dice, imitando pures

°

Qqq 2 Non














492 MALMANTILE
Non mibi'y si Sassen or ag
Omnia penaru ee omina polfem.—

E così rende l'uditore attento,  curioso, col promettere di vol
venimenti così maravigliofi, che non e per trovar parole adegu
ne esprimere. A! > bet: F e

'DARA Ie barbe al fole, Morira,. E' traslato dalle piante, le qu
cioè si feccano, quando si fuelgono, e si voltano loro le barbe al So

GAGLIARDA, e Calata, Sono-duc specie di danza, ob
scherza con la voce ealata., che vuol dir caduta, oftela, d
ver fatte qui Gagliarde, e capriole fara la calata,, cioè calera verso
comunemente s* intende andar forterra, cioè morire. Jay +e

DIRA il Vespro Siciliano, Dopo haver cantato versi amosgh ante fj
Siciliano, che s' intende; vedrà, € provera stragi. B' nora la follev ne de
ciliani (orto Gianni di Procida contro a i Francesi nel cempo, che questi ti g
giavano la Sicilia nella qual follevazione fu il egno, che un determina gi
al suona del Velpro ciascuno si moveffe contro a i Prancefi, come se
fuccefle granditfima Mrage di essi Franceli; E da questo & nato il '
Vespro Siciliano; che vuol dir fare steagi,ammazzare. Vedi Gio. Villanil
61.¢ Giachecto Male(pini nella Continuazione della Storia di Ricor
cap. 209, >

Hil festive l'agetib + Sona tial yarenaisinmeeltaie
i quali nel tempo, che sono gli agnelli, vanno per Firenze gridando. Ch
scorticar l'dgnello; per bulcar denari in ammazzare, e scorticare Metti ani
il nostro:Poeta da quello (canaare,\¢:scorticar un' intinica di'effranil,
puta huomini crudeli, e senza pieta, e questa per'accomodarsi abgenioy"e cap.
cita de i fanciulli, che stimano quell' atto una granditlima inumanita,
nando quelle bestiuole innoceati. ny sieht

FLAGELLO. Qui è preso in significato di eopine, farts ee CA

















di. Vedi sopra C.1. tt. 45. invaltro signiticato. In Gio. Villani trovafi nel fen ayy
usato qui dal Poeta; F/agello, e Fragelio; come costuma di dire anche aug
piebe Fiorentina, e come dissero i Greci, e si legge ne} tefto Greco dell®; pac
fey

uy

dy




hio, Phragellion per quello, che i Latini dicono Fracetium Omcto
sgrazia,sferza, o 2agello ds Givve vinci Node tibro 12) verlo 397%
831. Attila Re degli. Vani tu soprannominato per quelo, Frage! t
TREMITO dela quartana, Quci brividi, che G-fentono' dal pazavente nell'en-
trare della febbre quartana, i quali sono assai maggiori diquegli 5 che soglions fie
venire, quand' uno ha qualche spavento}-eperd 'con dives VA tvemiep dela pias edy
sana, intende, che lo (pavento era grandissima,€ fuori dell' ordinario: E «ali tend
brividi, o tremiti vengon' allt huomo:, perthela'patira fringe il cuore; per lo




che il sangue corre tuctovin aiuto di eflo;¢percio--membri esteriori, e ie parti te
superficiali, ed cftreme rimangon\fredde; edi steddo facendo riftrit i pori, be
cagiona quel che i Latini dicono rigor » che farizeare i capelli, © pels "€ Cagio~ ni
na il cremito, il quale si domanda capriccio, e rsbrezzo, Vedi C. 6 Gig

MAL più, de' miei di fare quel defo, Spaurisco tanto, che esco










cro prima.

ets DAN ZALHI8.'

be il galio apportator del giorno
La notte nera pits d! un Calabrone,

Bil sua buio,e quant'abre eli'ba dintorne
Diognise qualungue grado,e condizione,
| Acid sicuri omai faccian ritorno
\ Gli nccei, cantando il lor falfo bordone,

AIncitr'al Sol,ch'in quespa parte,e in quella
| Fa pel lor gorxo nascer le granella
lead ety




Perché-crafeun » che quini si ritrova,



VNDECIMOCANTARE. 493

“fino a che viverd » non farò mai più allegro, come era mio solito, perché questo
- spavento m' ha fatto mutar compicifione, e temperamento: Non saro piii, quel

STANZA IV.

Quand' infra Dame, e Cavalieri erranti,
C' al trescone in Palazzo eran intentiy
Comprefeun dietro all'alero i duellanti,
Armati tutti due, come fergenti,

-Si shallo il ballo, andar da cantoicanti,
Ele chitarre, ei mufics Srumenti
Ai proprj suonatori, e balierini
Divenner rante cnfie,¢ berrestini,

STANZA V.

Si fa pero bifbigtio, e si rinnuous

 Kedendo entrar quell' armi coid dentro, L? odio fra te farion già quasi sperto,
 Subirovdiffe: Qui garta cicacca: Che tirando ai rispere: gu la bufa,

: =, £ trama di qualche tradimento, Ruppe la tregua, e rappicce la xufa,

4 iver la Jevata del Soley e dice, che in fu quell' hora entrarono nella stan-
22, ove si faceva il ballo, Martinazza, e Calagrillo, che la seguitava con l'armi
F inmano:; per lo che si lascid flar il baliare, e si venne all' armi., rompendo las

tregua, perché ciascuna delle parti sospetto d' esser tradita, e che questo fufle uno
— militare, come i ditie sopra C, 10, stan.31. dove laicid questi duel-
EL gaily apportaror del giarno sbandina la notte. 1) gallo e solito cantare in full'ap-
pariridel giorno, ¢. però dice ch' eglié apporrasore del giorno, e che da 11 ban-
do alia notte col suo cantare. Somniaque excuffit nuncia lucis aus, disse un Poeta;

Excubitorque diem cantu predixerat ales, canto un' altco, & erifta /pettabilis alta,

Auroram gallus vecat applandentibus als, Disse il Poliziano nel suo Villano.

CALABRONE. E! uva specie d' infetto, o verme alato di figura simile allas
mofta »maatlai pil grande, e di colore ncgriflimo, ed ha un jungo, forte, e»
acutissime pungigiione. Con questo nome chiamiamo,ancora il tafano detto fo-
Corot. 8. 1 Greci Prouerbilti ditiero fearabao mgrior, Più nero dello scara-
B10, che e un' altra specie:di mosconaccio. i
4N comro.al Sole, Giivucceili vanno incontro al Sole cantando in ringraziamen-

to delbenefizio, ch' ci fa joro, maturando le biade per loro alimento.

* GOZZO. E! il primo ventre degli uccelli, cloe quella vescica, che hanno ap.

Ppit-del colio, dove si ferma il cibo, che beccano, edi guivia poco a poco si di-

Mtribuilce al ventricolo; e da noi si piglia ancora per la gola dell' huomo, perché

vien da gutrur. re

° CAVALIERI erranti, Così son chiamati quei Cavalieriavyenturieri, che son

descritti ne i Romanzi Spagnvoli da loro detti Cauaheros andanes; wa qui inten-

de, che erravano perché stavano ballando aliora, che bilognava combateere.

“| TRESCONE, Specie di ballo, cos detto da Tre/ca balio anuco. Vedi iopra

G. 10. st, 28, Dante Purg. 10.



: ee S=TePiie etre

= ee

a

i












494 MALMAINTILE® (0
Li precedewa al benedetto Vaso

Trescando alzato, o umile S. ane cond
cioè faltando, ballando. M As +
SBALLO'. \\ verbo shallare vuol dire disfare le balle; ma qui

re il balio, In buon Toscano non si direbbe shallare il dar fine al
pis la forza della lettera 5s, aggiunta al principio di verbo, 0
ignificato contrario si come la particella, i», appresso i latini, |
tare, spiantare; grariofo, fgrariaso, ec, ma il Poeta se ne s
scherzo di ballare, e sballare, e (eguita il bitticcio # dar da canto s canti
figuratamente sbaf/are, per eccedere la verita ne' racconti; e © ¢
numeri di cose con vantaggio, e con caricatura. " *
DIVENT AR caffe, e berrettini, ec, Cuffia, come s'e detto sopra C, 8, fh. 48:
una berretta fatta di velo, o di tela.a foggia di facchetio usata dalle |
ferrar dentro i capelli in capo; dice, che gli Prumenti vennero caffe ye
perché le chitarre, ed altri strumenti simill corpacciuti, essendo bateuti in
capi di coloro, e per la loro fottigliezza sfondandosi, fecero I effetto
be in sul capo la cuftia, o berrettino, cioè lo ricoperfero, e ferrarono in
E' detto usatitfimo. Ti faro wm berretrino della chitarra, per intendere i
chitarra in fu la cefta. Vina timil frafe venne in capo a Omero nell' Iliade, quan-
do disse, Lapidea indui tunica, per voler dire, Essere tapidato', quasi il ricoprires
uno di faffate, sia uo fargli un vestito di pietre, che gli stia bene alla vita.
GATT A cicova, Ciۏ mifterio foro. Ci e inganno, eum Tras tiled
i Latini. tet ain
TRAMA, Si dice quella feta, ec., che serve per riempiere le a
renza dell' altra, che serve per ordire, che si dice orsoio; che per la più n
si dicono ordito, e ripieno. Dante Parad. C. 17. t soi aged Rl hn,





















Poiché racendo si mostro [pedita Tat

L! anima fanta di metter la trama Che

Jn quella tela, ch' io le porsi ordita, (SRE LS Sir

'Ma trama Gi piglia per concerto, ene habbiamo il verbo tramare, cheiwuol dir bag
negoziare copertamente, e forco mano, dilegnare,, concertare, Mraletrami ge ha,

fio affare,ec, Bdicendo: Queffaé trama ds qualche tradimento, intendes/Queho Oe)






@ tradimento concertato. Latino /ute/a doi. Varchi Stor, Fior, lib Cm
d' una conuenzione facta senza saputa d' un terzo dice + Orazio se ne? ada
rugia, senza che il Sig, Gentile fuspicasse non che sapesse cosa alcuna di questa' i
trama di gocciola per intedere specie d apopicsia,quasi una coperta apoplethiaye da 4
questo si potrebbe intendere per rrama, uaa (pecic; e dire questa è specie di qual i
Che tradimento. Storia di Scmifonte Trattat, 3. dice. 4 popolo fa fallewe 5 e grida tha

na, [uspwcando, che trama ui falfe, contro di lus, speotepecnh aot ¥



BIS BIGLIARE, Dilcorrer in fegretor, che si dice anche Far Pith pifft; ij
Pispigiiare, che usd Dante Parg. C. 5. Skit ise Bap w
Che si fa cio, che quini si pispielia, “ ¥ they

E si dice pi/pigio., e pispigiio, sorta di cicalamento; e viene da quel fafurrio, che: hi

featiamo da coloro, che parlano in fegecto.. toggi pia comunemente si diceb® =
Soighiare, bifvigtio, e bifrigtio, 5 te te
Th Ry



ae ae






na, O rispetto
 STANZAVL
metre man da buon Soldato y
imico ritorna a Bertinella,
f quale in quel punto casco il fiato,
UM fegato, la milza, e le budella,
Vedendo, quando men' hauria pensato,
 Vicire i pefei fuor deta padelia,
 Mtentre la fa venir Adarte vighacco





Col suo Baldone alle peggio del (acco.

STANZA Vil,







} VNDECIMO CANTARE.
| | TIRANDO git La buffa.a i rispessi. Non havendo più rispetto, o riguardo al-
cuno. Sxffa intendiamo una berretta, la quale e fatta a

f » € mandata gil cuopre anche tutta la faccia, e i collo: Eda questo
la faccia, mandar gis /a buffa., vuol dire oprare senza riguardo, e scaza






495

foggia di morione, che

STANZA VIIL.

Mentre 8 alcun t' osserva, ella pon mente
Per canfarsi enon esser appostaca 5
Ecco in un tratto vedefi presente
Martinazza la sua confederata,

Che poco dianzi anch' ells fimiimeute

Di man di Calagrillo e feapolata,

E seco vanne in luoghi occulti, e fenré

A fare wncanti, es faliti (congiuri,
STANZA Ix,

eit a © un certo vento non le gusta, Nes quali aiuto ella chiede a Plutone,
Che fa le (pade,e ognor per l'aria sischia, Ed ¢i comparfo quixi in uno ispante

wiil —- E.grd vedendo che (a morte aggiufta Dice, c' ha fatto a lor riquifizione
yee] Chipievnol far det brano,e pin starrischia, Già [pedire un tacche per un gigante
at Bel bello fuigna, e vanne alla rifrufta Qual' è quel famofissime Brancone,
it | Dun luego da faluarsi da tal mischia, Che col bartaglio,ch' era di Morgante,
pt} —- Adtischiayche non gli par di porer credere, Verrd quini tra poco in lor foccorso
- Ee Percio sospira, e non si puo discredere, ef dar picchiate,e' hanno a pelar  orsa,
& votes: f STANZA X. xi

Ed eccolo ( foggiunfe) ovvé battaglio\ E 8 anuedra,c' al fin piscio nel vaglio,

© desi fo dir ych'il primo,ch' egit accoppa,
Tatra l'armata a irfene in sharaglio
Che la barba penso farci di froppa;

E che al pigliar un Reeno non è loppa;
Cot scaciata abbaffera la crefta
dn veder, che de' suci non campa testa,

Si rappicca la battaglia, e Bertinella essendosi perduta d' anima, per vederes
i ritornato suo nimico., quand' ella pensava d' haverlo tutto dalla sua, es
-temendo di non esser ammazzata in quella Foote » meditava di faluarsi in qual-
4 che ficuco, ed appunto-s' imbatcé in Martinazza scampata da Calagrillo,
J € con essa en' andò in iuogo appartato a fare incanteGimi, per costringer Plutone
F -ad aiutarle; ed Egli comparfo quivi dice, che si fara venire il Gigante Biancone,
il uals in questo dire arrivO quivi, e Piutone rincuora le donne con raccontare
la bravura di flo, dalla quale da loro per distrutta l'armata di Baldone.
LE casca il fate. Si perde d' animo. E soggiungendo: Il fegato, la milza ye,
, te budelia, intende Si perda d' animo affatto
— QFeANDO men fet è pensaro. Quando meno dubitava. Non expettato valvus
ab hoffe culit..

VSCIRE i pesci fuor della padella. Perder quel ches' era acquiftato, e sopra di
che s' era fatto aflegnamento certo, e sicuro.

VENIR alla peegio del facco, Venire al maggior segno di discordia, e di rottu-
ta, Nelle guerre il peggior grado, che sia, €, quando le Città,0l'Armate son
messe a facco; e però dicendosi /e peggio de! facco in peggior grado, e condizio-
ne, che è haver il facco. VL

,


|

























ee
496 — MALMANTILE © 7
VIGLIACCO,, Vile, codarda., EB voce spagnuola, vells
significa furbo,¢ furfante, poltrone. i
SEL bella, Con bella maniera, e senza dar © del
antichi differ; bedlamente,manoneinufo..
SVIGNA. Se ne va con preftezza, o fugge. Forfeda questo
viene e omprare if porco, che vuol dite anch' egli Andarsene
fuinam, 010 fuillans emere. Ed e usate questo verbo svignare
besco. Vedi sopra C. 4. stan. 51, Si potrebbe anche dire, come pei
erudito, che questo verbo fuignare ligniticaado scappar dalla Vigna, s°:
scappare di foro la Vigna, strumeato o macchina milicare, che serviva
tichi per andare (otto ie muraglic a combatier le Piazze, con le quali”
difeadevano gli atiecianti da i (aii, ed altre cose, che erano: buttace lor
dagli affediati, le quali necetiitavano quelil, che vi erano.coperti a
sotto alle medesime vigne; extra vineam exire, che (uona fuignare.
VANNE ala rifrufia, Vuol dice cerca mioutamente, e con diligenza
NON si pus discreaere, Non può non credere. Non può creder, che
a cffer così, e non habbia a eficre altrimenti. Non può capacitarli
SCAPULAT A, Fuggita; Scuppata. 3' intende scampato il pericolo
LACCHE', Ragazzi, cae corrogo appiedi per servizio de' loro
di sopra C. 2, fan. 29. 2 ae
BLANCONE. B' quel coloffo di marmo bianco., fattura dell' Ammannato, il
quale e posto in Firenze nella Piazza dei Gran Duca, dentro a una valea gran-
de, la quale riceve l'acqua da diverse fontane, che scacuriscono da detto: fo
¢ suoi annetii; e se bene rappreicnta Nettugno, e chiamaco da cutta M Biancones — ui,
ai ee ray Vaca; 1 hi
MORGANT E, 11 Pulci in un suo Poema intitolato il Morgante narra'; che












Ms
questo era un Gigante, 1 quale nog adoprava per coubattere alt': he un Ya
gran battaglio da campana, joe alo tf

PICCALATE ¢' hanno a pelar  orso. Picchiate gagliarde, perché il) pelo dell' Oh
orso efiendo difficile a suellere, e pelare » non si fa caicare con' ky
se leggieri, Pelare, wattandosi di muraglie, o pietre vuol dire-space; ol
si, o (crepolare, onde potrebbe dirli hanno a peiare  orso, cioè tare fore yi
rompere l' orso, che Gi dice quel pictronc, che adoprano gii fiyfaiuol FN
lire i piano delle stufe, onde nabbiamo poi menar 1' orso.a Atoaan Pre
re ripulir Modana, e Ggnitica mecterii a far una cosa umpolsibue uk

PENSO' farci la barba di Stoppa, S'intcnde; E poi dargh tuoco.
Penso ingaonarci,.¢ por farci ogni maggior danuo, ie
PISCIO' nel vagiio, Blo stesso che far la zuppa nel paniere desto sopra C.
stan.7. E.che cosa sia vaglio, Vedi sopra C, 2. stan, 79. Luciano in ab
co volendo spiegare, che il far bene a' crifti e come un tar la 2upp.
perché 1 benetizzi riceuti (cappano jaro prettissimo dalla memora; 4
buomo cattivo,e sconoicente a una bog forata, che uo quello, che va i met. tes,
te, si ver(a. Plauto nei Pfeudolo, o vogiiam dire Bugiardello; 2Vae piuris refert, ="
quam si imbrem in cribrum geras, Corcisponde questa maniera alia noltes (char
nel vagtio, Luciano nei Live dic; come da in cofano forato, ©

ey







VNDECIMO CANTARE:




497

)zuppa nel panicre. Playto pure nel Pleudolo, la pertu/um ingerimus. dicta do-

opera ludimus. La favola delle Danaidi ha fatto luog
nifica non e cosa facile. Loppa; che si dice

19 al prouerbio.




NON: ne Detto bafio, che
anche lolla,
anche

ce » gli c levata.

a STANZA XL
Qui tacqueil Diawol,perch' e fatto rece,
“ él aria al capo git è maligna,
anuerzo 4 flar sempre nei fuco,
Vatea alle donne il dietroacafa,e/uigna,
EB lafesaus il Gigante nel sua law,
Che douendo a Baldon grattar latigna,
 Sull ulcio det falon già perwenuto,
edge Hf batraglinje questo fu il faluto,
STANZA Ali,
Sei braccia era ti bascagtio aito, e ds paffo,
| Bm injragnena aimen arciotto,o vent,



4; Ma dando fu nei patcormando a baffo
cha  Van trang intatiata, e tre correnti,

; E fece tai frafiuano,e cal fracasso
14 Che shalord? « un tratto i combattenti,
oh OE per pawra, a chi non fu percoffo

we |, Nomrimafe sr quel punto/anguc addosso,

il gulcio, che si leva di sopr' al grano quando si bacte, che si chia-
inche pyle. Lat, apinde secondo Nonio Marcello gramatico. 5

Y SCACIAT A ~ Rimanere scaciato; vuol dir Rimaner buriato, ches' intendes
; nd' ugo credendosi confeguice una cosa, e facendolela sua, o non Ia confe-









 ABBASSERA la crefta. Gli feemera!* umore, o I alterigia, I Galli d' In-
dia, quand' entrano in frenefia, gonfiano,, e cresce loro la crefta, € patleggiano
on una certa intronizzatura, che par (uperbia; ed usciti di quella frenefia, sce~
ma, ed abbaifa loro la creita,¢ di qui vicne il presente dettaco, che significas
readersi umiie, contrario di Rizzar (4 crea,;

STANZA XIII,

Ed infra gli altri Piaccianteo, il quale
S' era schermito bene infizo aliora,
Vedendo un fantoccion si badiale,
Dopo il terror di tanre [pade fuora,
Di quel detto farebbe capuaie,

a9 C' un bel fuggir faina la vita ancora,
444 perché in quae in la v'é mal riscotro,
Vede hauer vifo di sentenra coniro..

STANZA XIV,

Poiché non fa tronar modo, ne via
Per nellun verse da [campar laguerra,
Ech' ovis e forza, che chi v'é vi stia,
Pond morto, gettasi gilt in terra,

E ritrouando la botrigleria

Apre t armadia, e dentro vi si ferra,
Con pensiero di fiarni sempre occulto,
Fin che si quiet così gran tumulto,

! Plutone si paite dalle Donne,e la(cia quivi il Gigante Biancone, il quale andò
, alla lanza,dove si faceva ia zuffa, ed arrivato in fu la porta alzd il battaglio,
: per comigciar con esso a perquotere, ma al primo colpo dette in una traye, la.
quale per esser fradicia, si fraca(so insieme con pill correati. Tal colpo spauri

» tutti coloro, che eran quivi, e particolarmente Piacciantco, il quale fino allora

8 era ben difefo, ma per lo spavento, che hebbe dei Gigante,

getto in terra,

s fingendosi morto, ed a poco a poco si condufie all' armadio della bortiglicria '

b nel quale entrato vi si erro.



si #ATIOr«0, Divenuto fioco. Vno, che per catarro, o per altro impedi-
} mento aell' aspera arteria ha perduta la chiarezza della voce, li dice rancus,don-
y de rancedine, e reco. Dan, Int. C. 14.:

A, Erendele a colui ch' era già reco, £.
(| Li aria glié maligna, 1? aria gli nuoce, gli cagiona danno,

1, dietra a casa,¢fuignua. Volta le reni, e se ny « Bil verbo /ujznare,detto
rr GR.

Poco sopra nell' ovtava fectuma,

AT.











MALMANTILE® ©
S' initende perquotere. 'Così I intende
. fo direi anche, maio temo, che'ella

ny Won s apparecchi a grattarmi la tigne.

Si dice anche cacciar la mo/ca da deffo, in questo C. stan. 20, !
dajfar la lana, sopra C, 7, stan,63. Adandare a Legnaia,sopra C.
ter Ta poluere, sorto C, 12. stan, 1, E tutti hanno lo stesso signi

INFRAGNERE. Ammaccare, o pigiare una cosa tani
forma., come farebbe Peftare un fico maturo, ec, e il Lat. ¢/
Vedi sopra C, 4. stan. 76, e sotto in questo C, stan. 17.) *

INT ARLAT A, Rofa dai tarli, che sono quei vermi, li
dentro al legname.,, e di ele si nutricono; da i Latim detti rer
'C.6. stan. 59.

FRASTVONO, Fracasso. Sinonimi, che significano Romore, stre

NON gli rimafe fangute in deffo. Acbbero così grande spavento, che

mate spirito, Dicono, che a uno, che habbia ha'vuro un granditimo sp
© paura, se in quel punto gli fule tagliata una vena, non gliu
per le ragioni accennate sopra in questo C, stan. 2, d

S' eva Jchermito bene. Cio',s' ra difefo.. Havea scampato il toccatne

BAD/LALE, Grande, Si dice anche machofo, imperialc, € simili,
'scherzo; e significa grande più del naturale. Kose ee
VN bel fuggir falna la vita ancora, Alla (entenza che dice Vn'bel morir tutta le
vita honora, rispondono coloro, che flin:ano più il vivere 5, oe















Sa ney

bony






Vo bei fuggir fainn In vita ancora, 7 ag
V" è mal viscontro, V' è male il modo. Non W'@ buona congiuntura, ~ io
VPEDE baker vifo di sentenza contro, Conosce di non'haver ragione, cioè, che il Mt

'ncgozio non è per seguire, com' ei vorrebbe. A tthe? Wy
CAl ve vi sia, Chi ha havuta la disgrazia, se la ianga: E si dice: Obi v'é ui

vi sia, e chs nen 0° è non v" entri, qui però intende; chi in quella stanza vistia, tt

perché non se ne può ustire. Reet eee
BOTTIGIERIA, Armadio,o stanza, ove si tengono V afi da Vino ' a)

¢ servizio della mensa. Voce, che vicn dal Francele Borteille, che e Cl

'fiasco, o altro vaso simile da vino. 4 4

STANZA XV. 7

'Col battaglio di nucuo agile, e presto © già ch' egli non puo tt

Tira il Gigante 5 e da nella lumiera', etrmeggiar col bat: 'et lento y f k,
Ls qual cadendo fece del suo refto, 'Pero che il toga non ha gran diffanza, fr
Perché i [pense, e rope ciò che v' era; “Cagion ch' ei trowa fempre' mento; a
Hor, 8 eglie in bestia, dicanelo questo, 'Lafeialo andar bawendo pin fidana x
Mentre ch' ei da ne' lumi intal manera, Nelle sue manych' in simile Srumenb hei
E dice che'! Demonio lo jtafila, E piglia quells ciurma abbietra,e sbricia a
Poiché eli fa faltir due colpi in fila. 'eA4 menate 5 com' anici im camicia yj
' STANZA XVIL. oe ee fg
'Così tutto arrabbiato, come un-cane Talche'l me/chin non mangera più par ad
Piglia un pel coho,¢ feactialo nel muro, Perciò gli amics [uci, a
Di sorta, che disfatto ei ne rimane “We voglion, che il ribaldo,
Som' wie ficaccia piattalo matures Gli andaron alta-vien tush quant

Stet a. ac? ae






VNDECIMOCANTARE. 492
STANZA XVIIL STANZA XIX.

"sion cofforo un brance di galletti, E come la mia Serva, quana' in fretta
Quando la state, a tempo di ricolta, Dee fare ilpesce a uovo,e che si caccia,
Antorne a qualche bica units, e spretti Trama due nova einfigmele picchiet:s,
non di loro a berricar s' afolta, Sicche in untempoturte due le(chiaccias

Pere il Gigante fa certi feambierti, Bs che dall' tra e [pinto alia yenderca
Che re ne [uifa quattro,o se per volta; Softien quei due,es' apreneliebraccia;
Infaffidico al fin da quel baccano, Poryciacche,pacte insieme quello,e quefie;

Si china,ed aggavignane un per mano, Stcche e diwentan prit che pollo pefto.

. Biancone con un coipo fracafia la lumiera, e spegne tutti i lumi. Nota che,
i se bene era di giorno, la lumiera era tuctavia accela, il che spesso aveiene in ta-
lioccafioni di veglie y che i segiivorl distratti dal gusto del ball,fanno mezzo
— senz' avvedersi, che sia pafiata la notte, Ll Gigante in collera lascia il
ttaglio, e comincia a pigliar quella gente, e bacteria per le mura, onde tut-
tian tratto gli corsero addosso, ma egl si difendeva, facendo di loro un gran
-maccilo.
LVMIER A, EB vno strumento, col quale si softengono in aria più lumi acce-
si, che i Latini dicono Lychauchus pensius, luceraiere in aria.
FECE del sue refto. Far dei retto s' intende fipire la roba, la vita, ec. qui dun-
que vuol dire si speafero atfatto 1 lumi. <
B in bestia, B in collera., Dar ne i lumi, vuol dire entrar grandemente'ty col-
Tera, dar nelle (candescenze; ed è lo steilo che dar nelle furie, ed il Poeta (cherza
 con questa metafora di dar ac' lumi, ed intende dare etfettivamente col batcta-
- glio ne i lumi della lumiera.

ail; 4L Dianol to feafiia, 11 Diavolo lo perfeguita; Gli e contrario.

IN fila, Vo doppo l'altro, senz' intramezzo.
ot CARMEGGIARE, Questo metaforicamente significa Aggirarsi, o affaticarsi in
ibs vano; e signitica anche ingaonarsi, per esempio: Tu armeggi, se tu (peri d' ot-

tenere, ec, ma qui e preso anche nel suo proprio signiticato di mineggiar lara;

gli Cnell' altro d' aggirarsi. —

wo) CWRMA. Genraccia vile. Vedi sopra C, 3, fan. 76 e C. g. stan. 16,

ABBIETT A, e sbricia, Sinonimi, che figaincano vilitfima, minurifsima gente,

A manate, Da i più si dice menare. Quanti a' entrano in uaa mano; e per la
grandezza della mano del Gigante fuppone il Poeta, che fica moltiimi per vol-
ta, perché dice: came anici sn camicia, che sono anici coperti di 2ucchero, de i
quali con una mano se ae pigliauo le centinaia.

FICO piattole, E' una specie di fico detta così.

NON voglion ch' ci se ne vanti. Lo voglion gattigare, perch' ci non s' habbia a.
gloriare d' hayer ammazzato quel loro amico.

». BlC-AQuafi da il Lat. Barbaro apica dal buono -dpex. Così chiamano i Conta-
dini quel monte di grano in paglia a mazzi, da loro così accomodato, affinché
si flagioni, pec poterlo cavar dalla spiga; deta da 1 Latini rrieict congeries. Das
questa voce bica habbiamo il verbo sdbicare per accamulare. Dante laf, C. 9.

Come le rane innanzi alla mmica,
Biscia per l'acqua si dileguan tucte
Per e alla terra ciascuna s abbua, Rrr2z~ BEZ-

SEE CERCA ES











500

MALMAN TYLER: 1 v

BEZZIC ARE, MW beccare'de i pollaftrelli si dice bezs
FA certi feambietti, Cioè contraccambia le percofie,



ra

Scambietto * termine di ballo, che significa mutanea'
INF AST IDITO da quel baccano, Klicndogii v«








si
sopra C, 4. stan. 9.









Allor Bieco non ha pite fofferenza,
E giura, che di questot: Bacchillone
Von andra al Prete per la penitenza,
Perch'ei vnol, chee' la faccia col bastone;
Ei fui, che di ral arme ban da teenza
Gite ne daran a una fanta ragione's
Così guida i fuvi ciechiyow' e il coloffe,
Accto gli caccin le mosche da defo.
STANZA XXL.
Eglino tutti quini fermi a tiro
Presso.a Biancone aun fiscbioco' baitoni,
Senza tramezzo alcun, senza respiro
We diedero un carpiccio di queé buoni,
Ed egli con un piede alzato in giro
Fa lor sentir, s' egli ha fodii talloni,
E mentre questo paffa ye quel rientra,
'Con quel pedino te li chiappa,e /uentra,



'Bieco veduto questo fa vehire-i suoi Ciechi,i quali tutti in giro ini

la importunita. La voce baccano, che significa combat esett
piglia nel senso, che si piglia musica, felta » bordello, '

Quand! ecco rt veccbio Paolino





Ve anti

AGG AVIGNA, Piglia,¢s' intende cinger con la 'mano 'tu

glia, in maniera, che si possa tenere stretto con factiita,
PESCE a' weno, Vova fritte »0 frittata, che dicemmo sopra C. 9
s' intende propriamente la frittata, che dopo eer cotta 5 0
ruotolo, pure nella padella; rifritca, e ridorta in figura “di p
ta pesce d'uono, La Compagnia della Lefina dice: La 'consner
antichi, i quali conrenti a' un pesce d' uouo di due woun al pile

ClACCHE. Questa parola non ha verun significato', ma folo
no, che fanno l'uova, ed altre cose similt, quando si rompono, edil
ne serve pr esprimer quel bateere, che fa il Gigante di'quei due hi
tr' all' altro, ed immita Dante, che nell' laf. C,32,dice:
LVon hauea pur dail' orlo fatto Crich
E seguita i Latini, che pure 'hanno a finta voce Tax,
come si vede in Plauto in Perla; dove per intender buie dice > Tax
meo. E noi pure diciamo'tach, e pach; anzi le percotie da molti in F
cono pacche, come dice anche il noltro Poeca sopraC, 5. st. 47. Da
ta la parola Fiorentina dcciaceare,, che e lo ttetio, che' Pefeare
dicefi 'Pepe acciaccary; modeftamente infranto,e Acciaceo sopi
do uno per così dire calpefta,¢ maktratta un'alero., ”
5 j

3¢ per



la hile elbetae,

a;
STANZA XXN

Aquat fa pits cagon,cblT efti,e'
E ( perchegti e bizzarre) bam
Condotti com' ei suole,un par a

OveSalito a Petigion di

Vavol matel,ch'egis' ha de ee

T aftando,owe il Gigs















£ darel ccc ieP bocca









SEafi Si =F eo &










oy
Wes,

ee
a

aie





VNDECIMO CANTARE. yor

affaltano co bastoni, e Paolino falito sopr' a i suoi trampoli metie i) suo
iuolo sopr' alla faccia-di eflo Biancone, il quale però s' adira, e beltemmia

i suot falfi Dei. Pah
| BACCHILLONE, o Bacchiglone, E nome d'un fiume, che paffa dalla Cita
| Vicenza, in Latino detto Azedoacus minor (econds Fra Leandro Alberti; ed ¢
ida Dante Inferno 15.-ove discorre d' uno, a cui fu permutato il Velco-
irenze in quello di Vicenza, che dal servo de' servi Fu trafmmutato d' Arno
one. Da questo fatto di Messer' Andrea Mozzi, che così si domanda-
Vescovo, o pure dal verso di Dante nacque in Firenze il proverbio; del
fanno teftimonianza il Varchi nell' Ercolano, e il Borghini. Sacare d''e4r-
in Baechilione, aitudendo al saito dal Vescovado di Firenze a quello di Vicen-
y che significa faltar d'un proposivo in un' altro s Saitar ai palo i frafea: Ma
-questa voce Bacchillona aggiunta a huomo significa huomo infipido, e buono 4»
. oe » ancorché di persona grande; e suona lo tteflo, che Gaicone, Palamidonc,
i: » e simili,.¢ credo, che sia il medesimo dire a un! huomo Lacchillone,
scheCaftrone, e che venga da Bacchio, che in alcuni juoghi di Toscana vuol dire
we — agnello,e cos: Bacchi/one voglia dire agnelio grade,cioè Caffrone. O pure viene dal
© | Lat. bacuius,quati Perticone, Scuriscione, O vero & deo quali Baleceone; che si
»¢€ non fa niente dibuono, ne di ferio.

WON andra al Prete per la penitenza. Questo modo di dire usiamo per fare in-
'tendere, che ci vogliamo vendicare del oprufo, o torto fattoci, o che yogliamo
galligare uno di qualche mancamento commeffo; quasi diciamo: lo medesimo
i dard la pena di questo suo fallo, (enza che egit vada per efla al Confefore sed

il Poeta l' e(prime dicendo: Perché vnol, ch' ei la facia col baffane.,
| AIANNO ficenza-di porter tale arme, Cioè hanno permiflione di portare il ha-
it scherza, peso ivciechi portano il bafione per necefita, per farsi lan

























QW VINA fanca ragione, Gli daranno le'bastonate,.come vanno date, e quella

pi |  WoCe Sama, se ben pare riempitura per emfali, nondimeno detta in questi termi-
sf ablignifica perfezione, quasi dica divera, e di tutta ragione, e d' intera giufti-
a Zia, che la voce Sanetus fiacopata da Suncitus vuol dire Nabilito, determinato.,

» Nov. 10. £ battnrala adungque d' una fanta regione, cioè.con una folenne ma-
niera; dateglicie delie'buone. Vedi l'Orava 25. seguente..

GLI caccino le''mofebe da defo, Lo battonino. Vedisopra in questo C. st, 11,

SENZA tramezzo, e senza respiro, Senz' intermiffione di tempo, e senza pi-
igliare riposo.

NE dettero un-carpiccio di quei buon, Ne detterouna buona,'¢ gran quantità.
Carpiccio viene dal verbo carpire,-¢ pero vuol dire. manata., o manciata, e cence
Aeruiamo per intender quantità., ma per lo più di bufie, comel'intese ilFiren-
-2uola nell' Afin d' oro + £ poscia, che per nua volta gle x' hebbe dati un-carpiccia de

i

TALLONI + Quella parte del piede, che e tra la noce., e il calcagno,:ma qui
'piglia la parte per cueto il piede. Vien dai Latino Tans. C. 8, st..69.
 PEDINO, Deito ironico., ed.intende gran picde, pedone,

SPER











goz MALMANTILE

SVENTRA. Rompe, spezza, o sfonda il ventre,
attivo, che fventrare neutro ha il figaitca
PAOLINO Cieco. Questo fu un Cieco compo!
zonette, le quali si fentono ancora cantar per Firenze da al
azzi, e per questo il nostro Poeta dice: Fs pil canzoni, ch
oeti celeberrimi del nostro secolo. Tali sue canzoni anda'
le piazze, dove per adunare il popolo faceva fare diversi
cani, ed egli medesimo, benché affatto cieco, e decrepito, 2
trampoli di legno a i piedi, Questitrampoli erano duc pertiche y in
ciascuna,delle quali era fitto un pivolo, e sopr'a questi dae pivoli falis
sopr' ad essi i piedi, e foftenendo la persona col rimanente di de
con adattarfele sotto le braccia,camminava con granditima franchezza
poli da' Latini si domandano Graiie, 'ccondo Nonio Marcelle; e quei,
minano su' trampoli, Gratlatores. Feito dice; Grattarores i
ni, qui, ut in faltatione tmitarentur agipanas, adiettis perticis furculas h
que in bis superftances aa similitudinem crurum eins generis gradiebantir
prer difscuteacem confiffendi, Plauto Vinceretis curfuceruas,© gallatorem.
D1 cento scampolt, Tutto rappezzato; che scampole Jiciamo quel pezzo d
no, o drappo, ec, che al mercante avanza d'uua tela quasi pezzo,così
pato, cioè avanzato a far' un' abito 1nccro; e qui intende toppes o pezei
anno. ere a
. (MB ACVCC ARE. S! intendé coprire il capo,¢ ilwifo.. Vedi
si. 73. Varchi Stor. Fior, ub. 1.4 Subso fu preso,¢ smbacnceato col eapp
dotto alle carceri,
Sl feandolezza, S' adira. Vedi sopra C.
di scandolezzare e quel, che dicemmo sopra C.
BREZZA, Vento freddo; Vedi sopra C. 7. st. 18. ue
PAbP AICO. E' un pezzo di drappo incre(pato da una parte, e ridotto quai Ht
in forma di facco, quale portano in capo le donne per difendersi freddo, ed 'afl
oggi lo chiamano anche cufia, Mattio Franzefi in lode delle Malehere dice e » all































£Lvvi un fegreto, che a noi dir si puore, vet ~ Yay
Che la mascheraé me' a! un pappafico, si
E pero si vente in van. cufola, e [quote ty
Ed il medesimo in lode della Potta uso il verbo impappaficarft di aay
Chi ale tempse si fascia gli vechiati 'ake
Chi sopr' a i berrettin impappafica, ine
PORCO, Aggiunto a huomo vuol dire Schifo. ps a
0740". Intend, Che schitezza e questa? Vedi sopraC, 8.67 yy
ALLEZZA, Vedi sopra C.-3. st, 64. & wota, che il verbo allezeare tantoat =”
tivo, quanto neutro ha lo stesso significato,; 3 sur (oy
SA di refe azzurro, Per tigncre in azzurro adoprano i Tintori ere
fetore orrendo, o sia galla, o sia guado', o uno, 1' altro infiemes 1M

rimane per qualche rempo in fu la roba tinta, e particolarmence in sul 1in0
pero dice quel cenciaccio fa ai refe azzurro, ed intende. Ha gran fetores'
verbo appeftare ha lo fictio significato, e natura, che ha il verbo 4
di al detto C, 3, st. 54.

bee








STANZA XXIV.
levare intanto hawea Perlone
| La srane dal Gigante roninata;
Abe ancor quini ciondolone,

he la lumiera già tenea legata,
“Ed 4 foggia d! eAriere, o eMontone

7 nla addietro, e dannole l' andata

- Verso quel torvion, che si distese,




> STANZA XxV.
Hor' quando ( perch' egls sbalordito,
~ Etutto intenebrato in terra giace )
 LCieehi più che mai fanno pulito,
 Edegli se le piglia in fanta pace,
OB fra le maxe innolto.a quel partite
Vn facco diventato par di brace,
© Eben quel panno al vifo gli è dovuto,
—— Dovendosi si-cappuecto aun batturo,












lo stesso significato.

= =.
—

orribili Giganti.










| Col si pile voite in bocca del Franzefe, Perché quivinon è troppo-buon' aria.



VNDECIMO CANTARE 503

TI vuo' dar l'incenso con le peta. In vece di farti honore, ed incensarti, voglio
sprezzarti, offerendoti cose puzzolenti, come suol'esser il peto, del quale Vedi sopra C.\ 6.\ st.\ 100, Orazio. Vin tu Curtis Iudaeis oppedere?

STANZA XXVI.

Mentre gli rompon Poa, € poi gli fanne

Così t incannucciata co' randelli
E talor, non wedendo ove si danno,
Si tamburan fra lor come vitellt',



Gli altri soldati a gambe se la danno,

Ed ognun dice: alla larga seabells;
Euege la parte amica,e la contrariay



STANZA XXVIII,

Ma reftin pure a rinfrescarle gli orbi,

Con quell? snfalatina di mazzocchi,
Ed et riposi all? ombra di quei forbi,
Che gli grattan la rognaco' lor nocchi
Mentre quivi per far dispetto aicorbi,
Sotto quel cencio tien-coperti gli ovcht 5
Che sugnun parte,ed io mi partoacoray
'Pen tornare a Baldone,¢ Celidora:



~ Con inucazione, e macchina di Perlone, il Gigante e atterrato, ed i Ciechi
'gli vanno tutti addoflo col bastone, ed in questo grado lo lascia il Poeta, e torna
'a dilcorrer di Baldone, e di Celidora.

CIONDOLONE. Vna cosa, che fla pendente da alto a baffo [enz' esser ferma
'in verun' altro luogo, che dove è appiccata, come farebbe il battaglio.nella cam-
. » si dice far ciondolone, o ciondojoni dal verbo ciondolare, come dal verbo pen-
Gee si dice pendotoni, o penzoloni; da dondolare., dondoloni, che tutti hanno quali

ARIET E, o montone, Macchine,'0 strumenti bellici antichi, de' quali si servi-
| -vanoiper rovinare le muraglie; Sono notidimi., parlandone tutti gli Storici La-
“tin; ma particolarmente Giulio Cefare ne' suoi comentarj.
quel sorrione. Così & chiamato dal nostro Poeta il Gigante, perché
avanza sopra gli altri huomini., come avanzano i torrioni sopra lemuraglic; ed
anche perché servendosi dell' Ariete, o Montone, lo deve adoperare, non in un'
huomo, ma inuna torre, come è solito adoprarsi simili arnesi. Da questa gi-
'gantesca flatura, per la quale.essi sono affomugliati alle torri; fece Dante il ver-
'ho Torreggiare assai galantemente. Inf. 31. Vorreggiavan dé mezzra ta persona Gli

* COL si del Franzefe in becca.'Gridando®: bud, but, che voce dimostrativa di
p| dolore, ed in lingua-Franzefe vuol dire si. 4

¥ - SBALORDITO. Siordito, fuori del sentimento'per le percoffe ricevute..
¥) © INTENEZRATO, Si pwd dir finonimo di sbalordito: e qui vale per intormen-
i * 'tito daile percotie. Vn fatio, muraglia, o altro simile materiale folido, e dura,





“Ai dice intenebrato, quando, per le peccolic., che se gli danno per romperlo., e 4i-

Bio,











504 MALMANTILBE. | ¥

dotto in termine, che dal suono si conosce, che si comincia.a
F ANNO puiite, Vuol dire Ripulire 5 ma detto in questi te
da vero, o perfettamente; E' lo stesso, che Fardi buono detto fo
SE le pigia in fdnca pace. Se le piglia con tutta, ed intera quiete. Ci
bastonare, e non si rivolta, ne's'adira. E la voce Santa ha la forza,
detto sopra in questo C, st. 20. ' olay
KINVOLTO fra le mazze. Coloro, che portano la brace a vende
ze, la mettono ne i facchi; e per ammagiiarii, e legargli sopra tie,
tatamente gli rinuojtano in alcunc imazze; ed il Poeta schereando dice
gante e simile a uno di questi facchi pieni di brace, perché egli € rinu
mazze,¢ intende di quelle mazze, con le quali i ciechi lo bastonano.
BATTVTO. Chiamiamo Barrasi coloro delice Contraternite fecolari
proceflionalmente vanuo con velti line in dowlo, le quali chiamiamo faccht
figueino vesti di penitenza ) cappe, o velti da bactu,, cioè, che f bane,
si disciplina, ed il capo, e faccia coperta con un cappuccio appiccato a dettas
vefle. Ed il Poeta scherzando con l'adicttivo barrute, cio bastonate, e col sa
stantivo barruto, cioè humo di Confraternita, dice, che ai Biancone flaya
il Cappuccio, perché era barrato; e per cappuccio piglia quel ferraiuolo, che
lino Cieco havea meflo in capo al Gigante. '
INC ANNVCCIAT A co' randelli. A coloro, che G sompono braccia,gambe,
© cosce, ec, Nel raflettare tal rottura, «fhuche ' off Rando fermo al luogo,ac-










comodato si rappicchi, fanno una fa(ciatura con pezzi d' afficeile, o ttecche, la is
gual fa(ciatura chiamano / incannucerata, e pero dice, che, hayendo rouse | offa Us

al Gigante, gli fanno hora l'incannucciata co' randelli, cioè con quei afloat, Dk
0' quali lo perquotono. 37h Une)

s1 tamburano come vitelli. Si bastonano ben bene. Quando i Macellari hanne Ni
ammazzato un Vitello, o Bue, ec. lo gontiano, ed acciocché il vento pall Da
da per tutto faccia spiccare la pelle daija carne, bastonauo la bestia con alcune>

f
mazze, e questo si dice tamburare » o tambu/sare, che vedemmo sopra C. 2. AL34. ie
ed a questo ramburare aflomiiglia le bastonace, che si danno fra loro i Ciechi 5 whe

wuol dire molte, fode, e spete « Sidice samburare, perché date in quelle pelli di si
Bue, ec. gonfie, fanno il suono simile a quello del tamburo strumento 3 i

E per altro ramburare uno vuol dire quereiario; e questo perché anti

Firenze 4 tenevano in alcuni Juoghi pubbiici de' Magiltrati certe;

hi da chiunque si voleva, erano meile le denunzie legrete, € guefte calle C
vano tambari, e da essi tamburare, era il medesimo, che acculare, o quetelare.
Vedi gli Staunti di Firenze al libro intitolato. Ordinamenta snspicia contra Magnares
(citau aicune -voite da Gio, Villani ) al capitolo, ove Gi tratta del mettere nel tam~
buro. ieee
ALLA larga /gabelli'. Allontaaiamoci. Quando dopo la cena si fa balla, 0al-
tro paflatempo simile nella medesima stanza, nelia quale s' e cenato sche 1 com
mensali si rizzano, e per dar Juogo si fanno levar via le tavole, le seggiole, e Blt
sgabelli; ed ogn' altro, che potetic dare impedimento, si suol dire: alla

belli, e s' intende; si levi di mezzo ogui impedimento; il che e in ¢
che significa; facciafi ala, o si taccia largo 5 ma per lo pil s' 1






HOEZBE SE 2E x PRZ





VNDECIMO CANTARE. 505

er ae: ' [
troppo buon aria, Li gon' y'é buono tare; Intendi: v'é pericolo di
- MAZZOCC H!, Così chiamiamo i Talli del radicchio, ne i quali nasce il se-
de iquali si fanno infalate, che sono rinfrescative, ed il Poeta, (cherzan-

'con I equivoco dimazzocchio., che vuol dire: bastone, dice che con questi

1 i taano al Gigante l'infalata per rinftescarlo, ed intende; /e ha/tonare,
SURAT. 1 bastoni de'Ciechi per to più sono di forbo,o a' altro legnaine simile
chiuto, fodo.,.¢ grave,¢ dicendo 1) Poeta; Si riposf alPombra di guei sorbi, che
i grartan ia rogna co' lor nocehi, intende + si riposi sotto quelle bastonate de i





Ai “@ t
© PER far disperco a i corbi tiem coperti gli ccchi, Per fare Mizza a i corui per la.
» che hanno di non poter beccare, e cavare glivocchi al Gigante, poiché gli
, © difefi col mantello di Paolino cicco,
PANZAXXVIIL 5 STANZA XXIX,
Che la-nel mezzo a's suoi nimics comba Su via figtixoli; orto buon piceini,





“Di moda, ch? essi foeman per bollire, Faccian di quepti furbiyun tracto,ciccioli,
Che dove i colpi ella indirizza,e pidha, Nim remete di questi spadaccins,
Te sli manda in un subito a dormire, C' alcimento non vaglion poi tre pictiali;




“Che ne meno col fhan della (un tromba

¢ Es in vista.vi paton Paiadini
NCamprian eli fardbbe rifentire

Han facce di Lionije cuor di fericciell;










|B quamo brava, similmente accorta', Efel-gridare,e ilbravar lor v' afforda,
ait Acombatrere i suci così conforta, Ut can chabbaiayraro avyien che morda,
ya ov Deferive laibravaca, e prudenza di Celidora, e riferisce #* orazidne da essa fat-

z Pe inanimire i foidati, la quale € veramente appropriata al personaggio, che
ae::

~ ZOMBA, Perquote ». Vedi sopra C, 6. st. 104.
| SCEALAN per bollire, Vuol dire sminuiscono, e quell' aggiunta per bollire, si
'poneiper un costume introdotto da un quoco goffo, e ghiotto, il quale havendo
mMeflo'a quocere Ieile alcune merle, se ne mangid pil della meta, e portate il re-
A 'in' #gli domando il padrone, che cosa havea fatto dell' altre merle? ed
“il'quoco gli rispose; Sig, sono scemate per bollive, E da questa goffa aftuzia quando
diciamhoe Laral cofaé (cemara per bodire, intendiamo, che una tal cosa e (cemata
assai, senza potersene ritrovare il conto, o sapersi la causa del mancamento.
~~ PIOMBA.. Precipita'; lascia calare, o calcare il colpo.
"LA tromba di Campriano, Questo Campriano fu un contadino aftuto, come sé
'accennatosfopra C. 4. st. 47.,¢ Come si vede dalla sua fayolofa foria ttampata,
€Ol titolo Storia di C ampriano, 11 quale per'far denati trovd diverse inucnatoni di
gabbare le persone semplici; e fra l'altre quella d* una pentola, che bolliva fen-
(2a fuloco yperché da efio levata, mentre gagliardamente bolliva, € portat®in,
~mezzo a) una stanza, la fece vedere al corrivo, a cui voleva venderla; coflui ve-
dmala veramente bollire, senz' haver fuoco avanti, subito se ne inuaghi, ed ac.
» Sordosi di compraria per il prezzo, che convewncro. Giunto poi guetlo tale a,
casa con la pentola, e volendo senza fuoco farla bollire, e non gli riuicendo, si
~ quereld con Campriano, dicendogli, che I ne ingannato; Campriano chiamé
ss la






SU OS eae st ey




















506 MALMANTILE

la moglie, e la sgridd, dicendo, che non potev esser 5

cambtata, La donna fingendo un gran timore, con gran la;

per haverla inavvertentemente rotta, glien' havyeva data un' a!

paura, che havea del marito. Di che Campriano mostrand

to, cavo fuori un colrello, e con esso feri la moglie nel petto

ascofa sotto i panni una gran vescica piena di sangue, il quale fg

che uscitie dalla terita factale da Campriano; per la quale fingendo |

fer morta, calcd in terva. LU gonzo si doleva, che Campriano per e C

gicra havetle commedo un delitto così grave; Ma Campriano con facia.

£'i die + Sc ben la donna € morta, 10 1apro rifulcitarla, quando vorro

basta, ch' io suoni questa trombetta; e stimolato dal fempiice a far

piacque » e fonata la tromba, la donna ff rizzo, mostrando di rifalc

il (emplice con grand' instanza chiefe la tromba a Campriano, il quale d

te preghicre a gran prezzo gliela vendé: Costui andato a cala prefe o

gridar con la moglic, ed in fine le diede una pugnalata, con la quale

€ poi si mefle a (onar la tromba, ava quella infelice elendo veramente morta,n0a

rifalcitd altrimenti, B per questa caula, e per altre sue fei. aggini fu Cam:
riano condaanato alla morte, che dicemmo sopra C, 4. st. 27, E di questa trom-

Es parla i) Poeta nelprefente luogo. sot then is ERE
SOTTO buon piccini. Esortazione, che si fa a' cani, quando s' incitano,o am- —

mettono contro qualche fiera, come vedemmo sopra C. 2. £.87,; ed il

si softiene sempre in fu le burle, fa che questa Capitanefla esorti, edit

suoi soldati con questi termini da cani.: - 1

cicciold. Frammenti di graffo di porco, che avanzano nel tegame,o altro

va(o, quando i fa lo strutto, o lardo, da alcuni detti ancora dardings, ficche> — jy


























vuol dire facciamo di costoro minutissimi pezzi. Ciccio/o diminutivo, che vieoe> (
da Ciccia; la quale nel linguaggio delle Balie 5 e de” fancimili vale'apprefodinot
Carne; siccome appresso i fanciuili Greci Tria. ei Whi
SPADACCINI, Così si dicono per derilione coloro, che portano laspadas =p,
folo per pompa. juin seemnigadgtit Une
PALADINI, Cioè Conti Palatini, Quegli huomini bravi, evalorolidifran- —j,
cia cantati dal Boiardo, dail' Ariofto, e da altri; e da questi dicen ty

Mena (e mani come un Paladino, intendiamo buome valorofo; poiche t O tay,



do. Cosisappresso gli Antichi,Ercole, e Achille si veniva a chia a
rofo,¢ dicevano: editer Hercules,¢ di Lucio Sicinio Denotato agg
mano braviflimo, riferisce Gellio lib. 2. cap. 11.5 che per la gt: eras Oy
appellato Achilles Romanus, Di guefti Conti Paladini, '0 del Palagzo intese il ei





Petrarca nel Trionfo della Fama Cap. 2, ro) Ne
Cingean costus + uci dodici robufti, + ' d } yy

FACCE di Lioni, ecuor di scriccioli, Mostrano d' esser bravi', ed animofit te

codardi. Lo scricciolo essendo il più piccolo uccello, che ti trovi, ha per conie- ».

guenza il cuore piccolissimo, ed huomo di piccol cuore s'i huomo timido,

e codardo. Vedi sopra C. 10. st. 30, Latino pari, © angu(ti anim Mi-

eropsychvs«

4















I ee





VNDECIMO CANTARE.

597



eet. di rado morde:,. Chi fa molte parole, suol far pochi faci. SE

ape

Suol far poche parole.
STANZA XXX.

bb wel ch? Ella da ritto,e da rove/cio,
— Condicende, va fenando a doppio,

Da sul vifaal Cornacchiann marove/cio

Cun mighio si senti lonan sale 3
et =
anc' egli eA cantocneall pio,
| Mail fapor non gusto già de' buon oo
Come chi prefe il suo de' cartoccini,
» STANZA XXXI,
Sperance per di id gran colpi tira
Con quell' “infornapan della sia pala,

We barte in terra,sempre ch' e la gira,

5 shafiti per la fala,
Tal che ciascuno indietro si ritira,
»O per franco schifandolo fa aia,

Bhi l' asperta, come bavete inteso,

' ek elon Ai) ie i pf



Perch Alsicardo, c' al pafsol? attende y
4 gozz0 gli trafora col pugnale
Ete lo manda a far le sue faccende;

ANZA
r ome il fuggir questa volta non gli vale,

proverbio con dire, Caneche murde y non abbaia s' esprimera la

Curzio:, Aleffima queque siumina minimo labuntur sono; ed. anche
Polidoro. Vergilio: Cave ribs
lontano il detto di Catone

se flefic sentenze,habbiamo in uso. anche\nel pariar nostro dicendosi:

a @ acque chete, Guardati dail? acque chere; Chi far di farsi vuole;

acane mnto., & ab aqua filenics A
1 Demiffos.animos, tacitos vitare me-




STANZA XXXIL

eAmostante, che vede tal fiagello

D? se! arme non usata più in battaglia,
erica la spadaye quando vede il bello,
Tira unfenditeein mezoglicla taglia;
Riman brusto Sperante, e per rowcllo
Li refto, che gli auanza all aria scaglia;
Vola il trovone, eil Dianol fack' eicaschi
Sula bottiglierra tra vetri, e fiaschi,
STANZA X&XLIL

Dalle diacciate bombole,e guaktade

i vino sprigionato bianco ye rosso
Fugge per b afse, e dann felso cade
Git dow' è Piaccianteo,e dagli addofso:
Ei che nel capo ha sempre stoccht,e spade,
04 quel fresco di (ubita riscofso,
Pensando sia qualche [pada, ocoltello,
Si lancia fuora,evia farpa fraselle.,
XXXIV.

Così dal gozz0 venne ogni (uo male,.
Per tui fadi, per Ini la vita spende;

E vanne al Diavolche di nuouo piccalo,
A ustolare a mensa appie di Tantaio,

-Celidora esortando i suoi a combattere non lascia di menare le mani; Si nac-

tano diversi avvenimenti, e la morte del Cornacchia, e di Piaccianteo.
SVONA a doppio, Intendi perquote inceflantemente. Suonare a doppio inten-
do tucte le campane, o la maggior parte dicfle, che sono in un
campanile » fyonano insieme.

Vedi sopra C, 6, st. 107. Sonare per percuotere,,

il Boccaccio Novella 67. E alzato il baflone i comincié a fonare. Latino

are.

 MANROVESCIO, E} quel colpo, che si da col braccio all' indietro,cioè con la
|p conuefia della mang, e da quella parte con baflone, o altro, che s' habbia

in mano,
ako feepi se h seid Meneoee un miglio, Il romore si senti molto da lontano, Ioke:

roposito.

een rere sopra C, 3. st. 21.
SICLIANDO un fempiterno aloppio « qu 20 alloppiarsi, o pigliar L oppia;

© cor-








. | ee
so8 MALMANTILE 10%
© corrottamente'? alloppie vuol dire addormentarsi da Opis
Sicch€ qui intende, che prefe un sonnoeterno, cioè mori.)
Oui dura quies oculos, & ferrens unger Somnus; in arernam cli
Dice; che per —— ¥ oppi rein perché It haveva dato.
tempo, per mostrare, che quis peccat,per hac torquetar,,
di Pincha y che per caula dab guacnisoeetipibans F c
zo muore, " sees

INFORNAP ANE, Cioè la pala da infornare il pane,
per arme, oh obi
SBASIT!, Morti, VedifopraC.2. f. 79.0 a a
FA ala, Fa largo; fa piazza'. 'Latino Viam prebere 3 win decedere y fun
HA finito il peso. sa fivito di fare quel, che gli era flare ordinaro; ha
compito; € s' intende ha fino ta vita: Metaforico di questa porgione di
che si da alli bactilani dali loro Capodieci di tance libbre@i lana, che
vorare, la qual porzione chiamano wn peso, e dicono bauer finito il peso
peafum, quando hanno finito di lavorar quel tanto', che era stato loro daro.
QUANDO vedde il bello, Quando vedde il destro; il tempo a proposito.. >
REST A brute. Kiman bettato, essendogli avvenuto quello y.che egh non s'al-
pettava 5 nel qual caso il vifo resta macchiato di tristezza'y € i.
confufione.. We
SOMBOLA. Vedi sopra C. 8. st. 44.,.
FESSO. Fetura apertura di legname, o d' altra'materia, ©
vasi di terra cotta, Latino Rima, ' ' 3
WEL capo frucchi, e spade. Dubita, che tutto quello, che egli sente, sieno ar-
mi per l'immaginazione depravata della paura; per la quale #%¢ rifeofo
tremore, che viene per qualche accidente inaspettato; 'che. ci cugioni

































per lo spavento, ches' abbia di qualche cosa improwvifa. Vedi fo he
C. st.2. se RitEe '
SARPA. Se neva. E verbo marinarclco. Latino foluir, anchoram vellit. Bog.
l' aggiuata della voce fratello è posta per emfafi, e quali per un giuro "g hl
LO manda a far le fne faccende. Lo spediice. Quis' intende 'ammazzay — te
PIANT ALO a ustolare. Latino ardere, inbiare. Lo mettevaliato a Tantalo 2 i
desiderar ancor' egli il cibo. Ed usuiare è latino; 'quafi dica + re dal ri
desiderio d*haver quella tal cosa, che egli vede. 'Ovidio negli Ai ¢
indomitis ignem exercentibus curis Fertilis, accenfis menfibus arder %
proposito ci feraiamo anche del verbo spirare. Vedi sopra C. 1, A. 31, diciamo ch
anche Vrolare; particolarmente de'cani, che fanno col mofo atte vie
vande, € per così dire le mangiano coal occhi, € col desiderio. ee, &
TANT ALO. E' nota la favola di Tantaio hglivolo di Gioves-e di Plotenin'2, 7
il quale per far prova del valore degli Dei 'gli convitd, € diede loroim tavola cot. i
to, e spezzaco un suo figliuolo detto Pelope; Ma gli Deis' aftenaero op
cibo, eccecto Cerere, che mangié le (chiene, le quali le furono'poi Fit 1
Dei, che lo fecero rifalcitare, e confinarono all' Inferna T: r be
cendolo patire di concinova fame, e fete, per thaggior suo te
'metcere sopra il flame Ereditaao, che moltra acque doiciifims,a!;.



VNDECIMO CANTARE:; 509
1 felabbra, ma non tanto, che ne possa bere, e sopra alla tefla ha un'
albero-carico di frutte bellissime le quali s' allontanano quand' egli s* allunga per
'pigiiarle' 41 nostro Poeta, che-ha de(critto Piaccianteo per un' huomo golofo

'y che morendo,egli fara confinato all Inferno, € per questo suo peccato di
ola fara mefio allato a Tantalo a #/flare anch' egli, come fa Tantalo,vedendo

ha da faziarsi, e che non possa haverla. Bologninus. i
Tantalus bic etram fitiens potare vetatur,
Ha 'a quod Pelopis Dijs epulanda dedit,
quali Omero nell' 11. dell' Viiflea descrive la pena di Tantalo, tradot-
Latini suonane così:
Stat mifer in medio; medijs exardet in undis

Tantalus,& fruftra circumfert pallidus ora,
Proximus illudit mento circumfluus humor
Et prope Yorantes contingunt corpora grtra y
Et crines,@ barba madent a/pergine crebra;
Dumque undam captar fitienti Tantalus ore




















STANZA XXXV.
Era un camer ata un tal Guglieimo,
Cha la labarda,e ifuci calzonia strisce
Virbigonicinolohaincapo in vece d'elmo,
E'tutto il reffo armaro a flocchefisce.
» “Alemnnno è costui Perneiter [celmo,
» Econ quel dir che brava,ed atterrisce,
Sbruffi ferenti (earicando; e rutti
Ln un tempo spaventa,e ammorba tutti,

STANZA XXKVL

Humoremque cavis, fentat tomprendere palmis.
Hen /upito, ben longe fugitura recurfitat unda,

STANZA XXXVIL
Perché voltando il ferro della cappa

Verso Alticardo a vendicar [ amica,
Quei ghetascafaye glittra sotto,e'l chinppa
Con la spada meixo del bekico,
Ond'sl vim pretto in maggior copiascappa,
Che no mesce in tre dil Inferno,e tl Fico,
Ala non va mal, perch'e: caduto allotta,
Hentre boecheegia tutto lo rimborta,

STANZA XXXVIIL



















1 Costud a quel ghiortone a tutte Uhore Gira Sperante pegeio a' un mulino

Fu buon compagno a ber la maluagia, Perch'arme alcuna in manpiiend.gl resta,
rer non cadere adeffo in qualch'errore, Par trova un tratte un pie d'un tavolino,
yi E far' un torto alla cavaleria, £ Ciro incontra,e gls vuol far la fefta,
a] Pur'anco gli vuoi far,mentre chrei muore Aa quei preso di quivi un sharagline,
, Con farsi dar due crocchie, compagnia, Voa casa con esso a ini fain refia,
E non duri molta farica in questo, Perché paffando ? offo oltr* alta pelle,
: ~ Chet trove chi [pedilo'e bene,e presto. Nel capo gli raddoppra le cirelle.,
. Seguitando il Poeta a narrare gli accident occorsi in questa zaffa, dice, che







Alticardo ammazz0 Guglielmo Lanzo, che volle seguicare in morte Piaccianteo,
come l'haveva seguitato sempre all oiterie; B Ciro Serbatondi ammazza Spe-
rante, con battergii un tavoliere da giocare a-sbaraglino in fu la testa,
GVGLIELMO Tedesco. Fu questo Ledelco Soldato della Guardia pedestre del
Serenitfimo Gran Duca, la quale e composta d' Alabardicri veltiti a livrea con,
brache larghe fatte a strisce paonazze, e role,¢ si chiamano Linzi. Vedi fo-
pra C. 4. stan. 4. E perché questi non portano ferraiuolo, o cappa, diciamo per
ascherzo ferraiuolo, o cappa quella labarda, che portano in spalla., come vedre-
me




















gre MALMA NETLLTW

mo appresso stan. 27. e s' e accennato sopra €. 9. fan. 48..€  r
date, o percofie colla jabarda. Costui era molto amico di-

aiuto a mandar male la roba, e però il Poeta dice, ch' ei lo vuol

in morte 2) La OORT

BIGONCIVOLO.. Diminutivo di bigoncia, detto sopra C, 10, stan. 7
costui con un bigonciuolo, arnese, che per lo pi s' adopra al vino,
che in tutte le (ue operaziont egli haveva l'animo al viao, e con
( che vuol dir pesce bastone, vivanda assai usata dai Tedescht ) per m
alla voglia del vino haveva unita ancora quella del mangiare. Si. ars
ancora, che il Poeta voglia mostrare, che costui era fudicio,.¢ c
in effetto egli era, e come per lo più (ono questi Lanzis a caula forse di
pelce, che veramente ha sempre malo odore.

BEKNEIDEK Scelm. Voci Todesche le quali in nostra ipa suonane
cone, scellerato,

ATT ERRISCE, Spaventa. La pronunzia Todesca ha un certo accento,
fa credere, che colui, che parla bravi sempre, € per questa rozzezza di 'al
gua dicono che ella sia propria, ed il caso a comandare eserciti, come la Fran-
ccle a aoe con dame, la Spagnuola al comando politico, ie cuaanaraerey
guefte cose Pr

SBRVEFL, BE? quel mandar fuori per bocca il vento jonato in carded
prabbondanza di ae E ratti si ie dire lo stesso, aso che per rasse inten
diamo il puro vento,¢ sbruffo si dice quando il vento vicn fuor del corpo «
no firepito, che non viene il rutto, ma accompagnato con un poco sae;
¢fiendo lo cheafare un mandar fuori di bocca con violenza vino, o altro i
AMMORBA, Fa putire. Vedi sopra in questo Cant. stan, 23. quie pe

significato attivo, cioè appefta; mette la pefte in tutti. '
GHIOTTONE,, Gran go.olo; Gran ghiorto. lntende di Pisesbame a

MALV AGIA, Specie di vino assai noto; ed a noi viene di Vi qui ie
pigliando la specie per il genere, intende che gli fu sempre compo be a tain
sorta di vino.















CROCCHIE; Percofle, Da ereechiare che in significato attiyo vuol dire Ps
motere

2 SPEDILLO bene ye preso. In poco tempo gli diede buona sp t 7"
ammazzo presto, ed affatto. Questo detto bene, e presso era il mol 3 7
cademia Fiorentina detta de' Rifritti, ed il Poeta se ne serve, p pil
fu già di detta Accademia, ed immita un' altro Poeta, che nelj' umprovvila, © \s
byona morte d' uno pure di detta Accademia difie; 'aban bar
E per mostrar, come Rifritto ville y pide eh e to

Mor:, come Kifriteo E PRESTO, E BENE, Ca

EEE e il Fico, Sono due Ofterie di 'Eirenze così nomina' die oo i;
Infega

Bosc HEGGIARE. Quel moto, che fanno con aprire, e (errage la bosaia
mandar fuora gli ultimi spiriti coloro, che muoiono 4

LO rimborra, Rimette nella bors 9 lO¢ in corpo 5 'ribeve -
che gli era ulcito di corpo.














.

VNDECIMO CANTARE) Sat

CLI vel far la feta, Cide lo vuole finire, lo vuole ammazzare '
GLI fauna casa in testa. Nel giuoco di sareg eee una casa,vaol dire rad-
iar le girelle, o tavole sopr' a uno de' 24. segni, che sono nel tavoliere, cd
i. scherza con questo addoppiar Ie gireile con dire che batrendogti il ta-
 yoliere in cefta gli raddoppia le girelie, che quiui haveva, e così gli fa una case,
- intesta, che haver girelle in testa s' intende tuomo col cerucllo che gira. Vedi

C. 9. stan. 10.
STANZA XXXX.

-) STANZA XXXIX.
Ritvaffe già Perlone un certo Marte, Tofelloych in fere.ra ad bxom non cede
Riesce adefio qus tutto garbato,

© Cthaucua il nafo da fiurar poponi,

| E perch ei nol pago mai del ritratto y Perch'ei rifana un zoppo da un piede,
Pere fa seco adeffo agli fgrugnoni; Cregnor fu quella parte andò seiancato,

| Ediegtien' un si forte. ch' in quell' atto Mentre di taglio un sopramanglidiede

Gli si fhianto la firinga de' calzoni, dn quel, che fano havea dali' alrra late,
| Che qual tenda calando alle calcagna Che pareggiolio, ond' ei fu poi di quci
 Scopri scena di bosco, e di campagna, Che dicon: qui¢ mioye qua vorres,

% STANZA XXXXIL
Grazian di sangue in terra ha fatt'un bagno Che vie da un trcbettier di Carla Atagne
Onde glié ya 4 chi va gin che nnoti; Quando le molfe dar fece ai tremors;
 Afetta un Salta,e xn Birrocolcopagns Toglie ad unl'asta,tl qual fail Paladine
1 E frroppia uneal, che fale erucce aiboti, Se ben con efsa fu [parzacammino,

“Seguita a narrare varj accidenti occorsi in quella zutla, e le racconca le bravu-
re di Tofello Gianni, e di Grazian Molletto.
SU ffianto la firinga de' caizoni, Si roppe la stringa, cioè quel legame, che ferra
calzoni in fulia pancia.
TENDLe4. Intende nel presente luogo quella tela, che si mette d' avanti a i
chi »sopra i quali si rappresentano Commedie, affinché cuopra le scene per
Doprine nel dar principio alia Commedia; Lat. /iparinm, e però dice, che i snoi
calzoni. essendogli cascati,, scoperfono scena di bosco, ec, cioè quel, che da loro
'eraycoperto. Caso veramente seguito a Perione, che,per voler ae pagato d'ua
Fitratto., che egli havea'fano a uno, gli conuenne fare alle pugna, ed ia quel
re gli cascarono i calzoni.
SCIANC ATO. Vno, che va zoppo per haver difecto nell' anche, offo princi-
pale delle cosce. Vedi fupra C. 6, stan. 82.
. CHE dicon; quie mio, e qua worrei, Così diciamo di quelli zoppi, che vanno
a gambe larghe per difecco, che habbiano nell' anche, o in ambedue le ginocchia,
€ non posano i piedi in dritto, secondo J' uso comune, ma pare, che vogliaao
can un piede andare in un iuogo ye con' altro in un' altro, e che accennino qui
# mio, €qua vorrei, Di questi tali diciamo ancora Andare a feiacquabarili, perch
fanno lo fieflo moto con ia persona, che fa uno, che (ciacqui un barile «
APPETT-A, Taglia da una parte all' altra, come si fa al pane, del quale pro-
-Priamente si dice affectare,o far fette.
VN Sata, Si chiamano Salti quei famigli, e donzelli dell' Arte dell' honefta
“(che in Firenze € il Magiftrato, al quale son fottoposte le Meretrici ) i quali fan-
'NO ogni sorta d' cfecuzione tanto Civile, quanto Criminaie contro le Meretricé,
t VN

me

i ee




le figure di carta petta:, le
di boto, e d' haver ricevuto ae
cono Bari. Vedi sopra C, 4. tts







sente 'uogo il nostro Poeta,

5 tz MALMANTILE | (¥
VN tal che fale erucce a boti « Intende' uno seultore dappoed 5 che.
qnaii @ mettono alle Immagini facre.
razia; e queste figure:co
c. £7. Gruccia è dal Lac, barb:
€ baflone fatto a croce; onde in alcuni Juoghi della '
Far le grucce a una figura, s intende fra.i pittori.
fan, 27, Intendi dunque, che costui era (cultore stroppiatore dit
fabbricava se non faacecci di carta pella, formati confornie di gi
no di quella bellezza, che può yedere?chi andra nelle! Chiele
miracolofi; e queste figure faceva così male, che le strop
da sapere che /eultor da bori suona fra gli scultori lo stesso, che fra i
Pittor da fgabelli, dewo sopra ©, 4. stan, 10, Questo tale ancorché fulie:
¢ nato d' intima plebe,  ttimava un Buonarruot; efi piccavaidi nobile
dice, che yen da wn tromberta di Carlo Adagno, quandevle moffe dar fac
Cioè ha origine da un trombettiere,dei
re.i bandi, che dar de mofe a' tremori, vuol dir comandarfo\

ticamente, se bene in deco scherzolo,¢ per derisione, come se ne serve nel pre.
> aa

arias
java affatto e Ino

quale:Carlo Magno i serviva per manda

Ae

SPALZZACAMMENO.. Vanno per Firenze aleuni © Marchigiani o Lom-
bardi con una pertica in spalia gridando: Spazcacammina y acciocthé pia
che efi ripuliscono le cappe., o gole de i cammini-dalle filiggine » Vino:

tait era cului y il quale con queli' alta, clod con ta pertica tr

ladino.
STANZA XXXXIL
Tutto tinte ne va Puccio Lamoni
Stoccheggianda nel merzo della Vuffa,
£ in Pippa un tratte da del Castitioni,
Che majcherato ancor tira di buffa;
Ea ci che nel sentir quei farfallont,
Venir più tofta sentefi la musa,
Paffandalo pel petto banda banda
Ai far rider le piattole lo manda,
STANZA XXAXKTL
Nanniruffa ha più la pien di ferite,
Pericolo, che fu [copa meftieri,
Fu pailaio, Senfale, etitor di lite;
Srette Bargelioy ed abbaco di xeri
Prefel appalte alfin dell! acquavire:
Ala pris fuaniro i fuvi peafieri,
Lon pite il wana fhillando, ma il cernello
Per mettervi poi il moffo,el'acquerelio,

Continoya a narrar quel, che segue neheombattimento y

mazzamenti,

TVTTO tinte, Vuol dire adirato, ma il Poeta si serve'di
ché detco Puccio è di faccia bruna, come s'€ detto sopra C.

6 OP
STANZA XXXREV) ©
Con Duriano ii Purba eccoalle wank ©
Di ferro da fradceri i Safe,
Ev altro una paletta tq
E con est a tui cerca,e sbracia
Ma percht quei le fqnete, tome canis
Gi fraricatt fuaf hs chibufo, (4

Chreghi ha a' Monnini,evane

Fatto d ognun polpette
S' 4 tanto mal non se:
Col dar sul grifo-a tui Salue Rofata y
Chef. oui 2
Vuol ch' e facia pere

Cb? essendo prefa
Lo spinge fuor


































«=. FREER

=a.




wii =

a a ae

Pre Sst = = -

Sate Sb tes eee



VNDECIMO CANTARE 513

 TIRA4di bufa, Fa i buffone. Le buffe, come accennammo sopra C. 2. staa.
2. alla voce bu/chette, sono pezzetti di mazza rifefla, e formano quaai un dado,
se non che hanno te parti piane, ed una conuefia, e si tirano come idadi, fa-
eendo Con esse quei giuochi, che si resta d' accordo con fei, orto, o pil di cali

 buffe; e per me stimo, che s' usino, come s' usavano dagli aatichi gli aliogi: ma

: è i e giuoco da fanciulli,percio habbiamo il detto sirar ds buffayche vuol
ire Far cose da fanciulli, ec. da persone di poco giudizio, che poi da questo in
una parola si dice buffone, e far il buffone; che i Latini dicendolo scarra lo delcri-
vono per uno, che rifum ab audientibus caprar, non habita ratione verecundie, aut di-
gnitatis, © così per uno, che non habbia l'intero giudizio da distinguere i tempi,
ee wetl ne le persone, come e per lo più il giudizio d' un fanciullo. UI P.
', Vincenzo Maria Carmelitano Scalzo nel suo viaggio all' [adic Oricntali lib. 4,
¢.26. descrivendo un' uccello detto Buffo [ che è forse.quello che i Launi Bubo,
€ noi chiamiamo Gufo } dice così,, I nottri antichi lo chiamaron Buffo, onde:
y» forse hebbe origine il nome di buffone, poiché è incredibile, quanto questo
a uscello sia inclinato agli scherzi, ed alle burle, con Ie quali bene (peffo atcer-
y rice di notte, ed inganna la gente.
|. BARFALLONI, Denti spropositati, e sciocchi.

SENT ES! venir la muffa. Si sente venir V' ira; Entra in collera.

LO manda a far rider te piattole, Lo manda a far il buffone nell' altro mondo,
dice /e piartole, perché questi son vermi, che stanao negli aucili, ed hanno oc-
¢afione di rallegrarsi per 11 nuovo cibo che a lor viene dall' andar egli nell' avello,

PERICOLO, che fa Scopameftieri, Si dice Scopameftieri colui, il quale seguita
poco tempo a far un' arte, ma lasciandola stare ne vada a fare un' altra, perché
la prima non gli piaccia,come appunto fece questo Aleilandro Violant detto
Pericolo, nominato sopra C. 3. stan. 58. il quale veramente fece tutti i mefticri
enunciati nella presente Ottava 43. ed in ultimo si diede.a trovare invenzioni di
Mettere appalti; cominciò dal Tabacco, e poi l'Acquavite, i quali senza suo
utile, o pochiffisno conchiufe per altri. Dice, che abbaco di zeri,perché veramen-
te ci fu un grandissimo abbachifta,e per questo havendo saputo trovar degli erro-
ri. contro a' ministri grandi, fu da essi perfeguitato si, che fu mandato in gaiera;
Ma havendo le notizie date da lui fatto al fine (coprir la verita, furono i delin-
Foe gaftigati, ed egli cavato di galera. Dice abbaco; ma percht questo verbo

gnifica ancora flar dietro a fare una cosa,¢ non trovare la via a terminarla,
per non haver tanto giudizio, o scienza che a ciò basti, il Poeta piglia tal detto
in questo luogo nell' uno, e nell' altro senso, cioè, che egli fulle veramente gran-
de abbachifta, e che egli abbacaffe, cioè armeggiafle col ceruello senz' utile es

conchinfione, e però v' aggiunge di zeri, perché, sia pur grande un' abba-
¢hifta quanto si vuole, che mai non rilevera somma alcuna, se non si servira d'
altra hgura che del zero, Cos} in effetto fu costus che con tutto il suo grand' ab-
baco non pes mai far conto, che gli tornafle bene, e con tutte le sue arti, ed
invenzioni si può dire che abbacaffe, perché in ultimo si mori quasi di fame,

PIGLIAR ? appatto. Quand' uno col pagare ai Principe una somma convenuta
Piglia ' assunto di provvedere uno Stato d' una mercanzia, e fa proibire che -al-
tri la possa vendere, o fabbricare senza sua licenzia, diciamo pigiare appaito, che
Sil Las, Adonopolinm. Tre MET-






514 MALMANTILE si

MET TERVI il mofto, eI acquerello « Consumarvi tanto le bu
tive fuftanze.. Oleam, & operam perdere,

FVSO da StradieriChi fiend gli Stradieri dicemmo sopra C. 3.
sto lor fufo e un ferro sottile lungo, ed acuta, col quale forano i
altro a fine di vedere, se vi sia occulrata roba 5 che paghi gabella.

PALETT A da Caldani, B' una meltoletta di ferro con manico: g0 » che
serve per iftuzzicare i fuoco 'nel caldano,0 focone,il quale, che cosa sia, Vi
C. 3. stanza 3.; ee

SBRACTARE.. Vuol dire iftuzzicar la brace, perché s' acceada, o P'accelas tie
spandere alquanto, e qui dicendo: gf sbracta il mufo, intende, 10 perquote con la






paletta nel vifo, e gli¢ lo sCortica. 1. 20.) aga Deke
4 LE squote come fanno icani, Non ttima, Non cura le buffe.. Vedi sopra C, 10, Suan
anza 36. 5 'sun Obey Mec
eARe HIBESO ch' egli ha a' Monnini, Doriano fa morire il Fucba con 'uno: Sino
quei suoi Monnini detti sopra C, 1. @. 44. i quali Monnini ij Poeta insieme cons Nan
ogai altro flimava tanco sciocchi,e odiosi, che credeva fuflono abil a far morire Celido
uno di naufea,; al fen

SQV ARCINA, Spada corta,e larga,altrimenti detta colrellao mezza/pada. Te
POLPETT A... Vivanda nota fatta di carne benissimo bactuta con coltello sed
impaftata con uova, cacio, pan grattaco, fale, spezierie,ecs > an Oh Difer
CERVELLAT A, & specie di falficcia fatta di carne, © di ceruelli di poreo La
triturati, ed imbudellati come la falficcia. E dicendo far poiperte 5 e ceruellaths Tere
4' huomini intende far macello, e strage d' huomini. OLS AES
CONT ADINA., Specie di danza usata nel Carnovale 5 la quale confifte tutta hag





in forze in questa maniera,, Octo-, o dieci huomini si fermano ritti col im Cel
fieme in giro con le braccia alla coliottola l'uno all' altro; opr' alle. di ha
quciti faigono quattro, o fei», sopra i fei altri tre,¢ soprai tre wao,¢ fatea que- atts
ita regolata mafia vanno girando a tempo di suono,, ed in ultimo quello, che € to
cima sopra a tutti, fa un capitombolo sopr' alle spalle di quei tre alla volta delter= e
reno, dove e ripigliato da due, che sono quivi a tale effetto;:nello feflo modo ty
fanno poi i tre,¢ poi i (ei, e dopo questi gli otto, o i dieci fanno iltcapitombolo ile
in terra; e questa dicon far /a tombeiata. EB percht Mato di Coccio.in: for- te
ta di bal'o era Maeftro, € però dice, che Salvo Rofara sapendo, bea la re
Contadina, lo fa fare la tombolata gil perla scala. aan Ui
STANZA XAXXXVL STANZA XX¥EKVIL- ti
Palamidone in tanto con la mano, Quasi di viver Bariftone uso, > '
In tasca a Belmaforto andana in volta, Egeno affronta con un prmerwoloy hes
Per tirarne la borsa in suw pran piano, E perché quei |" uccedia come nn gifoy i
Per carita che non gli fuffe tla; Salea ch' ei pare'un gailestovmapanele ip
Mail buon pensier ch' egii bayrie/ce vano E raito fa cht iL manbeaeenpe y ta
Perch' egli col pugnal se gli rinolta's Manda My
E fa per carizade anch' e che muoiar, E por to pi: the
'extecio fa vita non gli tolga il boa, 'Per dario per un 1






EB paffagli un vestir

ee

STANZA XXXXVIIL



Exquei gli duol che'l rinnono quell anno,
- Bfee' si muor vnol che gli paghi il danno,

ae VNDECIMOCANTARE, 515

STANZA XXXXIx,

Romolo infilza “to mezzo al bufto > L' armi Papirio ad un Prandron guadagna,
'tes iyoenunise un canto erafugviasco, Che. fae apiacuhine lo Swillerra;
Efe ne muor con molto suo difeuito, Ma # a parole gli è Spaccomontagna,
«Perché egli haveva a esser aun fiasco; AUP ergo poi riesce Spada fanta,
Tira inun tempo fifo aun bell imbufto, Perchheifactee it al Cel dar lecalcagna,
demmafeo, 'Won una voir dice, ma cinguanta:

Sta[uch'in terra i pari miei non danno
Ed ei risponde: S'io sto (uy mio danno,
L



STANZA
riga il Mula, ePoste degli allori, E nelle parti git posseriori
» Son mandati per sempre a far un sonno, Panfiloagginfta Meoyche vendeil tonno,
 Miccioge'l Baggina da Strazildo Nori Tal che s* allor putina, hor chi accofta

Sono inuiati done andò il lor Nonno, Sente che raddoppiata egli ha la potta,

\ Narra' morte d' alcuni disensori di Mal mantile, e le bravure de' Soldati di

a, Se'brami tanto d' intendere i nomi anagrammatici, quanto di sapere
chifieno gli altri. Vedi sopra al C. 1. ed ai C. 3,
STVEO. Sazio. Annoiato. 2

 PENT ERVOLO.. Piccolo file di ferro aeuto, del quale infra gli altri si servo-
no i farti per far buchi agli abiti.

DB aecelta > Lo baria; lo schernisce. Dice come un gufo, cioè come fanno gli
ucceiletth al: gato, che è uno uccello notturno, e simile alla Civetta, ma assai più
grande } chey Latini dicono babenem, donde bubbofone si dice a uno spropositato
chiacehierone; e bubbole i racconti spropositati, e non' veri ( forse da Bubbola uc-
cello, Lat. «pupa. ) In questo uccello detto gufo, o barbagianni, favoleggiano
git atichy Poeti, che fufle mutaco da Proferpina quell' Ascalafo, che fece la spia
a.Proferpina d' haver ella mangiato la melagrana, il che fu causa, che ella non

¢ ulcir daii' Inferno. Ovid, 5. Met. Questo uccello € forse lo stesso, che quel

Pgeedel quale habbiamo detto sopra in questo C. stan. 42.

~ GALLETTO marzxolo, | galli, che nascono del mese di Marzo, quando poi
fifega il grano son pil grandi,e fs gagliardi di quelli, che nascona d' Aprile,
eper queitofaicano piii alto alle spighe del grano, onde col dire: Salea come un
galletto marxvalo, s' intende falta gagliardamente.

 LL mal tarenfo, Vuol\ dire huomicciuolo di cattivo animo, che i Latini purer
dicono boma fungini generis.

4VEFETTO. lntendiamo una specie di tavolino; ma quis' intende un colpo,
che si da'col dito di mezzo accomodato a guila di molla a! dito pollice,o ( come
diciamo ) dito geoffo, e poi lasciato (appar con violenza al Juogo, dove si vuol
colpire «| Moiti pero per bufferto, o buffertune, intendono.colpo di tutta la mano;
¢ appresso gli muoli Boferada, o Boferon vuol dire moftaccione, guanciata.,

Macon questo huomicciuolo, che non era da pugna, o simili, si può credere,
che intenda veramente pufferro dato con un fol dito.

BAR querciuole, Cioè con le gambe alzate all' aria, € s' intende st ammazza,

-Lnoftri ragazzi dicono far querciuolo, quando no pola le mani, ea testa in,

terra, e manda le gambe all' aria; quaft mostrando qd essere una.pianta, la sc
od Tee "2 a




——














516 MALMANTILE | ©






















ha,della quale sia il capo, il corpo sia il futto,e i rami le zampe. ho
seguente dice dar /e ca/cagna al Cielo sche vuol dir caduto in terra b Bul
così si mostrano le calcagna al Cielo, e fi'dice anche mandare a gambe | no
FVGG/ASCO. Riurato, fuggitivo. Vao, che per paura de' birri sg
vedere, se non ne i luoghi immuni. we ky
HAVEVA a offer a un fiasco, Croe 8 haveva a trovare a bere i 5
Quando alcuni voglion bere insieme un fiasco di vino, € pagarne i
ii valore per mettere insieme la cricca, dicono Chi vaol essere 4 un fiasco? Mi,
tende chi vuol accordarsi a bere, € pagar cia(cuno la sua parte? BY termiae! Bad
fo, ed usato fra l'infima plebes ate a0
BELL imbuffe. Bella preteaza, Va di coloro, che Manno in fa la ky
quaii non hanno di buono che la prefenza, da 1 Launi soprannominati 4
per metatora, perché /folones si dicono quci bet rami, che noa ab
donde noi diciamo folly a uno che non € buoao se non a far comparla,o v
za,come si dice qui #7 bell' smbuffo, che diciamo ancora wa bel coram Vobis. A
Tulipano, diciamo a uno, che abbia buono aspetco; e poche altre quali Ti,
similitudine del fore così detto, venutoci di Turchia, che va imitando la! hare
¢ la vaghezza della Tulipa, o del turbante Turche(co,ondehailnome, =u
DOMMASCO, Deito così dalla Città di Damatco in Levante. Specie di v
drappo fottile di feta fatto a fior1, o ( come diciamo ) a opera. os baa
RINNOVO! quedl'anno, Se ' era fatto di nuovo quell' aano, Pare che sia foli-
to quando altri si fa un veltito nuovo per li primi giorai, che -adopra havers = nd
git qualche riguardo di più, come faceva costui, che per esser ii (uo vettito nuo- T
vo, l'apprezzava più della propria vita, poiché rinfaccia, e proreiladeldanno
del vestito, e di quello della vita non ne dilcorre, oem oie ¢
StanDROWE, Huomo di Fianiira, Ma perché huomo di Fiandea diciamo j
Fiammingo, la voce Fiandrone ci fertic per esprimere Vino spaccone, éhe si vanti P
di bravo,raccoatando le prodezzc tacte da im fuori di qua, ed uno di quelli, che b
i Latin dicono milires gloriofos, ed in questo senso lo piglia il Poeta nel presente i
luogo, se ben (cherza con l'equivoco; Ed egli stesso lo dichiara dicendoy Che» I
fan Taghiacantons,e lo Smillanta; all' ergo poi riesce Spada fanta, cioè fa da bravo 5 ha
ma dovendo venire a i fatti, e alia conclufione, riesce una (pada, che non fa mal ¢
veruno, e pero Santa; ed in fultanza un poitrone. Dicefi nell' uso, "i
buona pada; cioè € huomo, che fa bene adoprare la spada. Nel Pianto che't Pe
Carlo Magno nella morte di Rolando da' nostri Poeti detto Orlando, appresso vy
Tarpino Arcivescovo di Rems, e compagno in guerra del medesimo Carlo: 6 die fing
ce. O brachium dextrum corporis mei, barba optima, decus Gallarums, inf hg
Carlo chiama Oriando Spada della giuftizia alludendo alla formidabile spada da ie
Turpino detta durenda, da' duri colpt ch' egli dava con etia da' poeti Darindana, Hs
oh wrath rf 5 © fmill. dich un nottro pi bio in. di,
che dice La fradera del' kiba, che vuol div vantatore di gran cose 50;
re; Equesto perché la stadera dell' Elba; che serve per pefare barche piene ey
ferro, acile sue tacche comincia a contar da/ mille, e seguita s -a migiiaia
Tagliacantoni, cioè, che tira gill pezzi di muraglia corrifj | Pyrgo ii
wices di Riawto, Che vorrcbbe dire in noltra Lingua Atrerrasor ty




ee eS

4
Ai. si

a
VNDECIMO CANTARE. 537



Lo Smillanea, cioè Smillantatore si esprime dal Greco Thrafon, cioè Audace.,

BHES Ske

ee

Sesh © £F

SPTVSILRS Pee Thr ees CUR ET



Baldanzofo; e dal Latino Adiles gloriofus. E la parolaé fatta da Adidanea, (cher~
'zofamente usato dal Boce. in vece di mille; dandogli la desinenza di quaranta,
cinguanta, e simili; quasi uno non sia contento di dire la semplice parola di mil-
le, ma la voglia go > e far parere la cosa pil di quel ch' ell' e in esserto.
'S' io Ho fu, mio danno, Non mi rizzo al certo. Questo termine mio danno usa-
to in questa forma, e specie di giuramento, ed ha la forza del termine appon/o 4
noi, decto sopra C. 8. stan, 72. € 3° io non' ho,egli e fallo,detto sopra C. 6. (tan, 86,
MiCC4O, Così era nominato un garzone della pallaa Corda, che e uno di
coloro i quali stanno nel mezzo della stanza, mentre si gioca, a raccorve la pals
la, e rammentare il giuoco.
BAGGILANA, o Baggina, Eva un Battilano, che in occasione di felte serviva
ai Bawtilant per tamburino.
DOVE anao il lor Monno, Cioè nell' altro Mondo. Vedi sopra C. 4, tan. 2.
 NELLE parti posseriori. Cioè nel c....0 come baflamente si dice, nel prete-
rito, dove dice che e prima putiva, hora pute il doppio, che questo vuoi dire
ha raddoppiato la posta.
e4GGIVST A. B' preso ne) senso medesimo, che è preso sopra C, 2. stan. 41.
CHEO che vende il Tonno, Fu un venditore di peice falato,¢ tali huomini
hanno (empre addosso cattivo odore.

STANZA LI. STANZA LIL
In abito Scarnecchia da Coviello, Gustavo Faibi con un soprammane,

Tinta de brace l una,el altra guancia, Di nerty il capo fmoccola a Santella
EB per sua [pada sfodera un fuscelio, Scaramuccia si muar fotte Erauano,
C" al pome a' una bella melarancia, C' aimazza anche Gaba da Berzighella,
Rinolto con quest' armi a Sardonello, E fuentra quel birbon dell Ortolano,
Perma, gis dice, guardati la pancia, Che fa il minchion per non pagar gabella,
Ed enrisponde: ueftoé pensier mio, Ma colto poi vi reffa ad ogni modo,
z rant un colpo, ete lo manda a Scio.  Mentr' adeffo gli va la vita iv frodo,
Descrive } abito, ed armi di Scarnecchia, che refto morto da Sardonello;

Eravano ammazza Scaramuccia, Gaban da Berzighella,¢ l'Ortolano.
COVIELLO. Cioè lacoviello maschera, che finge un bravo sciocco Napole-
tano, 'a quale s' aggrotte(ca con fargli i bafi alla Spagauola col nero 41 b ace,es~
PerO dice Tinto di brace? una, el' akira guancia,¢ con armaria d' una spada faca
d' una mazza, che ha in vece di pome una mela, o melarancia, o altra frutras
simile per rendere il personaggio più cidicolo,e così vestiva questo Montambanco,
facendosi.chiamare Scarnecchia... Vedi sopra C, 3. tt. 62, Così Cosa, e Zanni,
personaggi ridicoli di Commedia sono nomi proprj de' loro paefi, donde si fingo~
.no»s accorciati dagl'interi nomi Niccola, e Giovanni; onde va in terra lorigine
di Zanni, che alcuni ingegnofamente hanno tirato dal Latino Sannio, mis.
LO manda-a Scio, Lo manda all' altra vita, ed e lo stelio, e si dice per ia me-
defima ragione, che mandar a Pasraffo,0 a Buda, detto sopra C, 5. st. 134
- SMOCCOLA il capo. Taglia il capo... Smoccolare si dice tagliare i} Lucignolo
di una candela, o altro lume per levar quegli escrementi, che fa la fiaccola, che
hiamali f i. » che queiti Spagi sear'













8 MALMANTILE










desfavilar quasi exfavillare; il Vives disse exfungare formando la all
Virg. 1. Georg, Scintillare oleum, & putres concrescere fungos, ol
SCARAMVCCIA, Vo' aitea maschera, come Scarnecehia - tit
Ourava 51., ma questo era Iftrione, e non Montambanco. i owe Roi
GABAN da Berrighella, Questo pure era Iftrione, ce rappresentava wo |
dt un Romagauolo ttoito. ' = Oe
L'ORTOLANO, Costui fa un yeechio aftuto, che: per ein
dovutali per aicuni delitti commeili, s' era finto ae 1 Del
chion per non pagar gabella, Menandro, Rufticum essete fimulas, tam Par
vi resta colto, cioè viene (caperta questa sua malizia da Bravano, che Per
vita in frodo, a colui, che non volea pagar la gabella,e¢ vuol wae Sin
in vece di frede solamente l'usiamo di dire dalla fraude, che si comm el
pagare la gabella. Ta
STANZA LIL is
e4rmato a priuileo} omai Rofaccio Che piove al
Marte sguaina, e Venere influente, Ond''ci in quel pumoandada Nan
Ma ae Sardonello sul moftaccio Vede le elle, e linac t altrasfera un
Gli fece con la spada un' ascendente, Nel vifo ectifia,e dice: Ty
Rofaccio ricoperto di privilegj cava fuora Marte; e Venere; che Pe
tivi influffi, ma Sardonello fece piombare sopr' a di luiun pefimo % Tee
tagliandogli con un soprammano parte del vilo, e del collo, ed un braccio Rani
il qual dolore egli vede le stelle, ed eclifiando l'una, € laltra sfera del Coats
ferrando gli occhi dice: Buona fera, cioè perme, fatto buio, «B Mi
sto Rolaccio si piccava d' Aftrclogo, come s' e detto sopra C, 31M. 63.5 11 Poeta tg
con la presente Otrava descrive la di lui morte con equivoci di termini affrolo- pred
ici. f Lapa
: ARMATO 4 priniteg|. Questo Rofaccio, come ancora gli altri Montamban- ro
chi per accreditare i rimedj, che da essi son dispensati, mostrano una infiuna di iw)
privilegj concefli loro da diversi Principi; e pero ii Poeta lo fa axmato di privi- the
legi. Uontanlald ken

SGVAINA, Virgilio vagina eripit, Sfodera Marte, © Venere; che predicono
rovine; B dice /gnaina, che vuol dir cavar la spada dal fodero, o guaiaa, perché
s*intende, che non haveva alcr' armi offenfive, che Venere, e Marte wnfluili
cattivi 'a duaaiead

ASCENDENTE, Termine astrologico, col quale qui intende colpo di taglio, Un

che viene da alto a baflo, piovendo, cioè calando in sul capo, ec,

OCCIDENTE, Intendiamo l'occalo del Sole', maqui intende ocealo y cio' &
morte di Rofaccio, ily oni aalaan ey baatee tm
VEDE le free. Quand' uno fence gran dolore; si dice + Eeli ha veduto le fielle y hi
perché le lagrime, che vengono in (ugli occhi per il dolore, G

la rescazione della luce, che yi batte, una cosa simile a una quantità di mi -

nute stelle in Ciclo, che pil volgarmente diciamo veder 'nce i

mo sopra C. 9, st. 60, 5 ma qui si serve di questo, perché.gii

re di farlo morire astrologicamente, i
ECLISS.A. Chiude, cuopre; ficome alla Luna 'restano i

»hajean > x

#3




VNDECIMO CANTARE. sip
 dail interposizione della Terra 1 raggi del Sole, quando seguono I ecliffi.
DICE buona fera, Cioè si fa buio per lui, ven donate 10. st. 5. Qui intende
 & finito il giorno del mio vivere. Virgilio in-aternum clauduntur lumina noflem, ©
i asadostndelcginnanalo » che, havendo:manco un' occhio, e Ji fa ca-
vato l'altro, disse: Buona worte per tutto lo tempo, '

i STANZA LIV. STANZA LV. i
Mein per fiancofentefi percoffo Già per la franca il sangue era a tal segue
Datlo stidion del cuciniere Melicche, C” andar vi si potea co' mauicelli —

 Parafiraccio porco grande, e grosso Istrion Vespi tutto furia,e sdegne
Perch' il ghiowso si fa di buone micche; Rinualto ha quivi tl povero Adaffelli,
| Sirivolta eAeino,¢ da al coleffo E col coltel da Pedrolin di legno
Nelda gola ch' egli ha pien di pasticche, Su pel capo eli squotola icapellig,
«Tal che morendo dolcemente il guitto: acleciopratcane poi la lifoa, el Lota...
» Addio cucina dice, ch'iobo frito, Pius bella faccian la conocehia a Cloto.,
ils STANZA LVL
NGatsi,, e Paol Corbi inveleniti A tal ch'i pacfani sbigottiti,
| Quali villan ch' i tronchsyed i rampolli E dal difagio feonquaffati, e froki
 Taglin di marzo ai fratti ed alle viti (Oktre che a' pachi il numero è ridotto)
| Potanda i basts braccia, gambe,e colli; Cominctaron le gambe a tremar sotto,
. Termina con te presenti Ortave i} racconto del combattimenco (egaito in Mal-
mantile,, e dice la morte:di Melicché, ¢del.Mailelli., e qui tinisce ' Vadecimo

re

MELICC HE, Vedi sopra C, 3. st. 59, lo chiama Parafiraccio,perché era huo-
Ȣdel continuo havrebbe mangiato: EB questa voce Parafito, che appresso
di noi ha dell'ingiurioso, non era così appresso gli antichi,come si può de-
durre da molti Autori tra'guali Luciano; ma particolarmente da Piutarco, dove
fitrova': Parafitos nontancumappellabant strici adalatores illos, qui apud Dinitums
tmensas wutriuntnr, fedietianm tos,qus ob rem egrecit geftam,publico /umptu in Prytaneo
atebautur Oc, Ondedelie Stinche di Firenze, nel capitolo in lode del Debito, il

Bernt; è
Voi fore quel famofo Priranco y è
Ab bower yas Doe renews in grafsoin fisoi baront
I popaly che discefe' due F efeo,

Exit Atheneo Parafiti olim appelabuntur foci, 7 fideles Pontificum, eAMagiftratiiz,
Ibmedefimo Plutarco.. ¥

PASTICCHE., Specie di confezione fatta col zucchero\muschiato y:ec; e però
dice more 'doicemente, perché ha gii per la gola 11 zucchero, Pa/feca voce Spa-
boas » siccome anche 'Pa(figiia, che vale lo stedo; e sono tutte due diminutivi di
pasta.

GVITTO, Huomo vile, abbietto, fudicio, sporco., e sciatto. Vedi sopras
C, 3.\f..9.è:voce Napoletana, ma usata oggi anche da noi, 'Nella raccolta de'
 Poeti antichi-dell' Aliacci, Pra Guittone-scrivendo un Sonetto, siccome da esso si
raccoglie.a Messere Oneito da Bologna 'Poeta, e amico suo; scherza sul nome di
turer € die, *

— SAS QF Cisasn sees

=

Pita

A.. SSeS

















g20 MALMANTILE | © ¥

Voktre nome, Mefsere,¢ caro,e onratoj
Lo meo afsai ontofo,e vil pensando, =
Ma al vostro non vorrei auercangiato, =
10 ho fritto, Scherza col verbo friggere, che vuol dir Quocere carne,0
padella con lardo, o olio; ed il detto ho fritto, che significa il
in malora. Latino Attum est de me; perij. Vedi sopra C. 8. st. 54, tor
nel presente luogo, perché par che dica; Addio cucina, ti lafio non
più bilogno di te, perché io ho già fritto, ed intende ho finito di vivere.
IST RION Vespi. Pietro Sufini. Questo fu cognato dell' Autore, e git
grandissimo (pirito, copiofissimo d' invenzioni, come si vede in una
commedie da lui composte, e da altre sue Opere poetiche, B pecige p
fentava in commedia ottimamente tutte le parti, ma in specie quella del se
zanni, ( cioè servo sciocco Lombardo ) che usa armare con un coltello di Tegao
simile a quello,col quale si batte, e si scotola il lino per purgarlo dalla lisca,
perciò chiamafi Scotola; però il Poeta lo fa azzuttare col Mafielli, e sc
con quel coltello la zazzera. Dice coltello da Pedrotino, perché con tal
ceva chiamare in commedia detto Sufini nella parte di servo feiocco. Questo mo-
ri giovane poco dopo l'Autore; e con esso si può dire, che in Birenze morifles 4
la moderna arte comica, o almeno la franchezaa, e leggiadria nel maneggiarlag =
SQVOTOLARE. Vuol dire battere il lino. Ma qui intende squotere i capelli
per facilitare a Cloto, una delle tre Parche, il farne la conocchia, aleism
INVELENIT/, Ancrudeliti, inviperiti, inaspriti, incancheriti, arrabbiati
son finonimi per intender' uno, che sopraffatto dalla collera operi of
te, e con ira, in maniera, che non sappia quasi distinguer ch'eififaccias, =
Similitudine prefa dal ferpente in collera; di cui Virgilio lib, 2, En, tcolentems
tras,& coerula colla tumentem. wm abean
POT-ANO. Latino amputant,demetunt, obtrancant, tutte similitudini trates
dal' agricoltura. Potare si dice de' traici delle viti, € de' rami degli alberi; ma il
Poeta si ferne di questo verbo per corrisponder' alia similicudine, havendo dewto
quasi viltan ch! e' tronchi, eds rampolli taglin ds Marzo, ec,. sd
SCONGVASSATI, Stanchi, € rovinati walla fatica del combattere.
FROLL], Qui vale per stanchi, ed indebolits 5 t¢ ben per altro Frode vuol di-
re stantio. Vedi sopra C. 3. st. 55. alla voce Leazo, iahesh
TREMAR le gambe sotto. Vuvi dir haver paura. Virg. Eo. ry.
ae folvuntur frigore membra, Se ben si puo anche intendere, che le §
mente tremafiero per la debolezza, e thancneaza.

FINE DELL' VNDECIMO CANTARE. +














Ze






Berbnanheansr ~ekk dé




BER TEER BEES









HifwMAMALAM

Fea dattacbute

A R GOMENTO,
e A, Montelupo. da Paride 1). nome,
Poi gapigar la Maga,e Biancon vede,
Rimef[a sn. Trano è. Colidera 3 3 e1come.
~ Aarito, al general dd ln fuafede.
a Baldon, che la fortuna ha per.le chieme
Con Calagrille aVgnan rivolgeil picde y
E al suo bel. Regno con Amor va, Psiche
A corre il frutto, delle sue fariche.







ae

speyrepeage

sae PP Regen

STANZA HL
Che sono fratt com! io diffi sopra,

were STANZA I.
Swanco già di vangar tutta mattina




“Abconcadino al fin a va-a rifelnere,
- Te forniar Vopresed-in chiamar la T ina

* Cokmerize guarto,eil petal dell'afoioluere;

( Nella Maga affidatifi) asperranda
Da' Diavoit im lor pro veder quale'eprs;

ea chi-vive a speranta muor acids;

eee tn Caffelle ancor non firifina Perch in Dite son tutti fottofopra,
Phu quei-marei di squotersi la poluere; 'Per non saper dove, come, ne quando
Onde: Badldon quei popoli-di/per de Laftiaffe il Cornocapolfo,c! ale (chiere
Tal che a' joldati Malmantileé al verde, Esser tromba dovea nelle carricre.
STANZA Il. TANZAILV

E vase Sta, perché porevan dianzi,

vedean col peggivandar sicuro,

~\ Cederil campo, @ non tirare innanzi

ra Star avoler cozzar col mura:

< E così va, che questi son gli avanei,

Che fafempre colsie'ha is capo duro,

» Che dentro.a feifi reputa un' Oracolo,

Ne crede al Santo,se non fa miracolo,

Di modo, che Plurone omai scornato,

Poiche quel corno pitenon si ricrova,
Pel Proconfolo dice haver pefearo +
Pero connien pensare a invenzion nuova;
Ha innanss ch' ¢i-risolua col Senato,
Eche'l:foccorso 4 Atalmantil si muova,
Ch'egli habbia.aeffer proprio pot savvisa
Di Meffinail foccorso, v quel di-Pifs,

wey introduce i Poeta in questo Duodecimo Cantare con la rifleilione, che i (ol-
7

vv dati

|







22 IRKTVUA MAL MAN TILE > 1009

dati-di Bertinella non, haurebbono ricevyto,così gran danno\ |
sono accordati,, e non fufione faut in tanta 'tingaiones la.;
in loro per la speranza, che havevano negl'incanti di,Mar
havevano havuto effetto alcuno, 1 Diavoli non feppe:
dove fufie ii Corno d' Aftvlfo., non si ricordando, che. an
quando Affolfo andò per il fenno.d' Orlando, comedice.|'/
| KANG AKE...Lavorar la. cerca conia vanga... Bipalio

FERALAR l'opre,.Cioè far defiltere dal lavorare eer an
raion Depera. fra. i\contadini.s' intendedlJayoro;, che fa.un'
no, e s' intende.ancora lo Relio huomo.s.che ya.alavorare a
io.ho; chamato due, opere, per iacender due huomint; In questo lavoro ci
dicci opere, per intender dicci giorai di iavoro, ec,

p44 Ting. La Caterina, intende ladonna del Contadino

MEZZO, quarto... Così chiamano i Contadini un gran valo —
foggia. da boccale 5.del.quale si leswo, fespartag da bere ai Javor.
po, e gli danno questo cee perché'e forse di ce een
staio. x ae
PER SL afeigluere 1 cont ania ebaainesior il desinare asciolvere, Seren csnidal
foluere il digiung, dali. sdigunarsi, ¢d.il desinare, lo chismano wien
terzo mangiare Aicono./a cen

eA non si rifina., 'Nanay refla,,
esprima una.op e feng'
Ciog perquoterii,, bastonarsi.. Vedi rae C7. tt. 63. t by

ESSER? al verde., Eji¢r' ajla.fine. 'Tratto dalle candele. di, se ce seieprion dig
son unte di verde nel piede.. Viano nel Magiftrato del Sale di.
le tafe dell'Oftcric.,,¢ darle.al più esserente., e agl.tempo, che aMtneenapi thd
colissima candela di cera tinta da picde di color verde ognuno puo.otferine, es ida
conlumata quella noo può. pil veyung offerire sopr' a quell' ofscria, ma shintende Ps
reflata.a, colui), che ha cfiertoyi maggior prezzo, ovver0 non arrivando.lotier. Cm
ta.aldovere, ' Ofteria ai suoyo si dubasta un' altro giorno con nuova candelerta', deat
EB digui habbiamo il dewato hs ha che dir,dicada candela e al.verde, che significa ted







on, si fa fine. Ma — che p00 iar



BINED a ET SERRE SSESE









sbrighiamoci y che il cmpo fugge.. E questo eficr' al verde e pafiato in thes
per tutte le cose, come.cficr' al verde ot danari, vuol dire esser' alia ka
pari,...Va mpderan Poeta teleth scritto nell'Osteria di Radicofaa Pre
trata » qenm ° Re he age

| Cohanr, p Spebater ridotto alverde.: ee un ie ' hy



. » Gineca, uper ricattarsi, e Sempre perdes

COzzAR col muro», Tentac l'impotlibile,. Contrattar con chiha pb: forea di re
poi, Clavam, ¢. manu Herculis extorquere. Diceli anche: saree a co4hi co! mre othe
ciuoli. Nell' Ecolcfiafico cap, 3. Ditiori re ne focins fuerss 3 Qu Ep
cabys ad ollam ? Quandoenim se colliferint, confringerur. La Feces Fest
tole nel. fume galleggianti 2 una di rame, l'altra dicerra fa a. e oka
quale viene, Anaae ad. Efopo,¢ troyafi refa in versi Latini gala,
CAPL dur ag te oftinati. Dure pom ebb —o Ne
'ST tacas lends » Amico della sua opinionc,¢. che Li, Gli
uw

rs




~~ oN eeenrvc

S88

SSBB ER CEES £5:



DVODECIMDJSIVLTIMO'CANTARE, 8
reat fate', e dit meglio w ogni altro. Huomo di quefd naritat dice de'
= Setpe thea Nim di lapete;e d-etere wngtan” buotao - Baxi.
; edi fe'micltefimo ',¢ pereid ine diviene contumace 3! &

PUR Hess ONT I OM 9G; ty
VENOM orede al Sarito';\se*hon fa miracoli, Non crede'; che una cosa pli poma'ii-
teruenire', fe'rion la vede fegitire.Generario prava quarit signum videre. B per lo
più s' usa in' occasione' dammonire, o rinfacciare j'come e nel/pretente hiozo} ll
tale è lato pir volte: avvertito dition contindvare @ fat'quella tale operazione-,
perche'gliene' potrebbe seguir male'; ma' egli oftinato wor erede at Bantoy se nor se
miracoh, cioè non da retca agli aveertimenti; ma! vudl-seguitare?,finvhe la die
ee succeda » 4' Proverbifti Greci mettond un Proverbio,-che dice: Primes
a rem. PURI Toe h) LL Mes bas! Pe TNE Dey 99@ 1951b

CHI vive con (peranzia mior cacando. Detto. sporco » ¢d usato per lo pil fa,

genterviles;\c vuol dire -'chiMfi palce di speranza-,'muore di faine'y"ed-in fulliinza

a €*vanitaril'fondarsi nelle speranze. “ai /pe'neratar 5 wi reer

mats ig 2. 0g



SON tutti fortofopra, Sono in grandissima confufiune. '
sI DOVEA fer tromba alle carriere.' Dovea' fare feappat tut? peome facev't il
Corno 4' Aftolfo: e'come fa' scappare dalle motfe i cavaili'barbati'y che edrreno
al palio quella tromba, che suona il banditore, pet dare if feghd della [otpperied
SCORN ATO' »» Vuol dif beffato; ma qui 410 (cherzu di /eorWard\, che Vadeiie
senza corna, come era rimafo Piutone fenza'¢orno, cine senza it Corti dA Nbk
fo. Var animale, che abbia perdute; © tronche le cortia, vient ad avere per
del decoro; onde scornato diciamo per beffato. Acheloo 'fiume; 'efleatlogli d2*Er-
colelevato un corno', rimafe feornato; e fuergonato. Onde Ovidio 9; Met Muh
tas Achelons agrefies 5 Et! Laverne cornu; meuijs capur abdidlit sndis,  Hivtc tanith abla
ti dommie idibure decoris, Gc} 229 10109 Lb 2h 4 HW) 199 > piri
> SPBSCAR per il Proconfalé. Ho Neff, che durar fatica per impoverite; sean,
CG operam perdere. \\'Proconfolo'é ia Firenze il Magiftrato', che soprdutenie 21
dottori,¢ Notai, ed ha la'saa refideriza otto Ie logge,dove sono git altri Viizzi,
acll'ultima abitazione versoril fiume d' Arno; il qual fiume pet quello spazio',
che e fra l'un ponte, e/' altro} ', 6 almeho efa già fortopotto alla 'giurifdizione
del medesimo Magiftrato del Proconfolo'; come pesca'ad elo rifetuata ne' vr ti
poteva pescare senza licenza del detto Magiftracs 3! non vi era-già ditra pena aIfi
contraffacienti, se non la perdita delle reti, e del pesce, che hanno preso 5 feads
acchiappati in sul fatto; E Pett utes! aie s

STAN ZAV E>) > STN ZA VEU.
ae Paride ritorno, OO Ada quegti 5 ¢° obligate si non Witende,
onGhte nellvoffe alla quarta sboccatura;\ Wor vnol phr quanto un capo di spilletto;
« Eperché dal pacfeegls ha in quelgiornd E subito ogni cosa indietro vende,
0 Foleo ogni nota', liberando ib Tura; Ringrariande cinscsin det buon' afetto,
8| La\gente quini corre @ intornd E'dwe', che da lor nulla precende;

ed rallegrarfidelia fuabravkrh >) 0! 2) EB Te aiteddisfarle bhnho concerto,
Ne lo ringraxiasewrallegrarsi intenta, °°! Perital niemoria gli fara più griro
Chi gli da chin lt dona z'chi gli-avvina, ©: Che it tuogo'Aonteliipa sia chizmaro;

ang, ~:

Vvv 2 STAN-



















524

Si si, ch' eli è dover da tutte quanté
Gli fu risposte, ed in un tempo stefo
Li editto pel Caspello fu pe i canti
Per notizia de' Popols fu meffory>
Che dinuleato pos di te avanti yo. «\
Fu osservato si, che finoadefo-..- \
Lucho nome confernan quelle mura,
E'l manterranno,fin che'l mondo.duna, ¢

STANZA AX ) f \“ A

E che fuor del Caspella il,popoh proves: «3000 (- ) SERB;

Che ognor ne feappa qualche sfucinata,

Per to piit gemse yeh? a peta 31 i loxofexes fer o

Cotantaé rifinita,¢ maltrattata, — Qui pinto innansi stwile sentiva) >

Tornaril Poeca a discorrer di Paride, il quale hayendo ridosto il Tura nely

fino flaco, haveva liberato quei popoli, i quali per riconofeimento del

ordinarono, che que! luogo si chiamafie daallora avanti Montelupo

torna al campo, e trova ogni cosa murata..
LA quarta sboccarura, Cie ha sbaccato, 'cioe.: manomeflo

vuol dire: ha bevuto tre fiaschi di vino, ec cominciato ibquarto2\Iperbole, che

significa: ha bevuto molto vino, sborcare propriamrnte Qgettare via

vino, che e nel collo del fialco., per purgarlo affaordalll'obia.yec, LiAQpesas!
CHI gli da, chi gli dona, e chi git avventa iB' detto giecofo nfato per burlares

uno, che figlorij d' essere sspesso:regalato; es) intende; chido ee 1

avventa, cice fafiate, ec. € lo scherzo dell'equivoco't:nelwerbomare, e '
NON enol, quant' un puntale d' agherto, Racufarurto.. Vedisfopra Capt, 10;
RINGR AZIO' del buono affetto. Termine di cirimonia julaci si

ringrazia uno'del regalo, e nello fefid tempo si Ficnfa di rice -dicia- f

mo;non voglio,© non stimo il regalo, servendo, per obligarmiy Pinclinazio- '

ne, che io veggio in voi di farmelo; e questa teftimonianza  chehio dal:voltro
affetto verso di me. ist
eH/ONTE Lupo. Finge, che Montelupo Castelio wicino a
anch' eg) quasi distrutto bavefle nome da quota azionedi:
biamo per tradizione vuigata, che eglilfufleanticamente
flare il Castelio di Capraia luogo allora forte' fituato rincontro
cendo coloro, che.' edificarono: Perdifiragger questa Capra 'Non sci guole altro, che
un Lupo, e perciò lo nominarono Castello Lupo, che. per esser Mopraiun monte si
detto Monte Lupo. Coca bg ED
GLI venne il grille. Gli venne voglia: E' 1o-stesso, che tocedsill

sopra G. 9. st. 56. con Sp bei
ST &VGGIMENTO. Vn continuo ardente pensiero 50% I

iftruggimento vuol guarire, cioè vuol' adempire questo.fne-desiderio

all armata. Ii Burchiello, fe'ben mi ricorda; Se/piri:d.amo rd

































\item[SPARITO ciò che v' era] Non v'era più persona alcuna, ip
Baldone era diloggiato, ed entrato'in Malmanule. 495






“DVODECIMO;ETVLTIMOCANTARE. 525
SEVCINAT A, mola neg Vana gran quantità, Fuciaayyicn dal

che wuol-dic
ani



(O facina @ i

» o luogo dove Gi ri

no mercangic; es
be capire una fucina prela. pec
operasoni

le ic
' et re Bocce, Nov. 2. ee ane eena di diaboliche

igion dire; 4

si erika, 'vuol anche dire il Barccaten de' fabbri o delle fonderie,.ec,
| RIELNIT. « Malconcia,@aaca, finica, sopunatai ¢s.intcnde di fanita,e roba,

or STANZA a
pala 5 e ne ri/contra un branco,
Preeti lemgean,
'bi dietro fr ascicar fivedeun fianco
; | gli gi





STANZA XL

Chi ha scatole, chi sacchi,¢ chi sieehiee

Di givie di mifoee 5 dibiancheriay \
Va" altro ha una ranaca di scrittwe 4,
Ch' agli ha @un Pinto della dtercariby

agli senza.adar albaco, £ piange,ch).¢i le vede mal sicure y
& nee Sete egli ha riscoffo; Pero che *l vento gliene porta via;
 Ciascuno hail/uofardel) diguelletre/ibe, Vat altro dopo bauer mille imbarazzi,
a a og si ha potuco beans ee Port' addofso nna gerla di ragazzi,
STANZA XIilL
reimbacuccato Arete feretto Le dine agliocchihantutteilfazzoletto,

a ria > 'eJpelse, Ipesso si szattiene 9

E sgombrane 2 Py rocche,e pergamene



tra ys' elle, le stanno.

Chi'lf il ye chi >

Chi porta nngatto,e La caninainbraccio,

Sono



lex
te yede una gran quantità di gente, che fugge da Maimantile, per (cam-
parila vita, e porta seco, le cose più grate; nel che il Poeta s' accomoda a' gen) di
quelle tali perionc, che fuggono, ed a quello che,per lo più,luol seguire ia simili
Seapets;

at ENC INGO. Se ben significa quantità di polli, o di pecore, o simili, tuttavia

ne serviamo per esprimere ancora quantità d' huomini, Lat. bomnum manus.
Vedi aC, 6, tan. 35;

T ASCIC A dittro on Fucwene Va zoppo., per esser Mroppiato da.un fianco.
HA »ifeoffo Senza aspettare al abate, Glioperarj ordinariamente ri(quotono le
ro mercedi, e prezzidelliloro lavori il giorno del fabato 5. ed il Poeta scherza

col. verbo rifquotere, che vuol dire ricever denari e ce.ne serviamo ancora per
intendere Ricever butle.

GVIDALESCO. Malcalcia; Scorticatura. Vedi sopra C, 10, fan. 11.

TRESCHE. Qui intende bagattelle, bazzecole, arnesi di poco,prezzo; Lar,
trica, Vedi sopra C, 10. stan, 12,

SCATOLA, Lat, cap(ule. Sono caflette con fondo,e coperchio, fatte con,
fottilissime afiicelle in varie figure, secondo che richiede la roba, che dentro ay
efie si ripone.

SLANCHER/E, S' intende ogni sorta di panno.lino., come tovaglic, lenzuo-
la, camice, ec.

PLATO, Lite civile » dal Lat, placitum, VedifopraC. 7, stan.27.

MERE ANZI A. Altrimenti Afercatanzia. Così chiamiamo, in Firenze quel
Foro, o Magittrato, al quale si ricorre, per far l'efecuzioni civili,¢ ai we son

fone-




ee






526
fortoposti tutti li Mercanti, ec. il quale ha particolari fat
'MB ARAZZI, Spagnuolo, Embarazes » Roba', th
6 feommodo; ed' aBBHaED il verbo imbarakzare y cht
nefi(, te tina Qanza ¥ ec * » ASSAM vaya
GERLA Da gero Latino','che vuol dire
Ma Voce il nostro Chimentelli nel' Azsr nie?
di bastoni a guisa di gabbia da uceelli
larga je fondato hella' parte pity tretca 5 det
per portare il pane eotto da un luogo all'altro'y adatrandoselo
alle reni; € et eeitind nim
firo Autore nella-eecera alla Serenissima Arciduchessa Claudia,
nelProemio'j dove Wie' Che i Prascica diecro'lina gerbi- tdi farfaallond COR
gran quantità dipropositi) Può bene anche essere Che il' Poeta intend'
mente ger/a, e che voglia dire, ché havessero due }o tre bambini'in u
talé gerle §'per:portari'pilr comodamente's coiné veggiamo tutto ”ll B
parire povere donne della Garfagnatiay e d* altrove, che portino due, 0
gaaai addotio imgerie y 6 altri trabicvolifimill /-)) >) 9
tM BACVEC ATO} Copertd5¢ Ito 'bene, ¢'s* ihtende pi
pert ibcaporn Vedi sopra C)1't, Man22: se bendal Cy 6, fan. "64.
ne serve per intendere Mettersi l'abito addosso, tuttavia e da norare,'
intende il lucco, che e l' abito Curiale "if quale'aiicledmente haveva il
per coprir la testa, e però mietrerfivtal"abito si diceva Pmbackecarff; Si
inbavagliare. Giovanbatifta Bufia? asBehedetto 'V archi lettera nona, | ¢
da AMona'coiet, “ed imbavagliatala la conddffero alle Palle se 3
OLE risconera » Cive riconta la moneta 7 per vedere} i '
traruino y vuol dire imbatcersi in-who Pma risconttare libri;' ferirrtite 5
danariy contipecyvuolidit Rivederé je rorha®l Ay
Z HO29! ib. 920074 2» MH299)  tegal cuba
» conteaffegnd Wi piabtd', 6'di dolore'
il fazzokeio agli ocehi', Veli topra G..9! stan.48! ahaa
SCOMBRANO « Portan via Seombrare [ quati dal Latind excumiiliré, ton?
trario d? Ingombrare, che ¢'come se fotle dal Lit! ficidmindare] deco
t6, ci serve per intendere. portar le? mafferivie"die ania casa a tn” altras
mo in-vece'del verbo diloegiare', sieggiare 'Biaiken archi
BeASPL rocthe je pergamene' 7 re fruthenti atcenenti
habbiamodenowoped nel ©. ye Rad.'>\\E -pertamend intertddid tes aS
Carta con'la quale'fermano ia 'condechia' ia (u'l# roced "per fadifitarell Blare
la'dicond paielibead #pershe per tovpit: ol esser facta di carta pecora,€ he ti














et anche, carta perdaminay i 9
199 9.99 roel vp SAB NON Z AOR 249 Pr cxortert
Entra: Paride al fimdentro alla portarys 9 * Ma quel che mardni¢ha p tt dppor
OOue\g lipar a! entrasdentroun matelles \ -<Si% st' veder tn' pi m Capen
Chr ad ogni palje troua gente morta. Di scope, e di fascine f
1Oiper lo-mer 5 che (Ps per far fardeliay > ani? i iy:
Oud i IMGselis e edusiiwi wh sil? - otary ate wage "A Ve Loi 4



aw



——— a

BOE Gg CUaio 'Set etree eee





DVODECIMO;EDVETIMOCANTARE: = 527

oo ye STAN ZAXWE oo cen! Singeatte:
arriuato in pragza, Egliftaben, pere una simil raza,
Perchi(domanda) ésigran fuoceaccefor.. C? ha fatto se @ ogni lana un peso,
 Egh érisposto:egheper Martinarra, E' si vorrebbe ( Dia me lo perdoni )
bid v'e dentro,escrine:latopre/o;. Gaftigar a milura di carboni. 7
'aride entra oe! Castello, e vede molta gente morta, o malameate ferita., e+
jartinazza mefia nel fuoco per gaftigo dellc sue stregonerie.
 MACELLO, Beccheria. Luogo dove s'ammazzano le bestie per vitto dell?
mo: E per macedo intendiamo Strage, o difipamento di che che, sia. Qui
Iptende, che a-Paride par d' entrare in una bottega di un macellaro in riguardo
=| molto sangue, che vede (parfo per il Castello. Così quel che dice Dante, che
V go Ciapecta tofle figliuolo d' un beccaio di Parigi., Sccfano Pafquier va interpe-
trando, che abbia voluco dire di un bravo soldato, quale era suo Padre, che per
la @trage che faceva, era riputato, come un maceliaro. '
CHE fea per far fardelio, Lat, vafa collig't, Che & vicino a morte;, fla-per an-
darlene da questo mondo. Vedi sopra C, 4, stan, 21.
CAPANNELLY di feope. Piccola capagna, mucchio, monte di (cope., ec, il
eee quando era per l'cffetto, che era fatto, questo, era dat Lacini:detto eon

Inc reca Pyra dal Greco Pyr, vuol dir fucco,e noi pure lo diciamo Pira, Dang
126.: ii
i Chi è in quel fuoco, che vien.si dinife,
eed z Ds sopra, che par furger dalla pira,
iets: Ove Exeocle cul fratel fu mifo. »
SCRIVE; lato preso, Antendi; ha cieito per sc quel luogo - /edem occupauit;ma
Per maggior chiarczza di questo detto, e da sapere, che in Firenze G fanno ogni
@nco tra gli altri quattro mercati, uno per Quartiere, che il primo nel Quar-
Gere, e in fu la piazza di S, Maria Novella il primo giorno di Quarefima, ach
quale Gi vendono Icgumi, feccumi, e frutte. Li secondo nel giorno di SS, Simone
nel Quartiere,, e in fu la piazza di S. Croce, Li terzo la.vigila dituitii Santenel
Quartiere e in fu la piazza di S, Giovanni, acl quale si vendevano oche.; mas
questo € andaco in defuciudine; perché e perduta l''ulanza di regalar l'oca lay
mattina di cucti i Santi. LI quarto nel giorno di San Martino nel Quartiere,.e»
in fu la piazza di S. Spirito. In questo, come nel secondo Gi veadono abiti, pans
Hine, ed ogni sorta d' arncfi, e maflerizic;.¢ come-che acile dette fire concorros
Ho molti mercanti di panni, ed altri artefici d' ogai sorta.,. così alle. volremanca
doro il luogo, dove polarsi, per farvi.ia quel giorno la lox boxtega; onde. piglia-
Ho il luogo qualche giorno avant, e segnano jo spazio dei juogo,, che piguano
con getio 50 altra unta, e vi (crivono in leere cubicali LATO PRESO,, e que-
flo servc per impedire, che altri entrino in quel luogo.: Edi. qui dicendosi; I
tale ha (critto /ato preso in quella cala, ec, intendiamo: quella cala, ec. e per iui,
wne gli può esser tolta. Così. dice, che Martinazza scriveva dace pre/o in quel mon=
te di scope, per iagendere y.chc havea tatto in modo, che.qucl fuoco.non le po-
teva esser tolto.... 4g Neds e402 an ic
|fatto a' ogni Jana un peso, Ha commefio ogni sorta di de'i\to-senza riguar-
do alcuno. “Si dice anche far d' ogms erba fa/cio, Che in (uitaaza s' jntende un' nue.
mo





|
















38 “1 ALMA NIP 1

mo scellerato, di coscienza larga fhe Hon' tetne
giuttizia; che'in Latino' pure si ditebbe, ex guoliber,
mea quella; Aivdum fie-pratum, quod non persranfeie lit

b10 me loperdent, Detto da Ipocriti, perch e in' un' certo
cenza a Dio di fare un peccato impune.1 Latini havevano'una i
che parte simili + Si Dijs«placet', " eee
. GASTIG ARE a mifura di-carboni. Dar maggior gattigo di
il detingtente. 11 carbone e fra le più vili/ mercanzie; chef
mifura, € per questo nom ff guards così: per la minuta in darne
bra, e pero habbiamo questo detrato, che significa: dar' pile |
nel Morgante. ef mifura di crufea, e dt carboni, + o RE Oa

STANZA XV.; STANZA HVE)!

Ia quespo., e ognum parla della Strega, i i a
Si sente dire; A voi; largo, Signori,
E un bnomaccion più lungo a' unalega,

Dal Palazzo si vede conaur- fuori, Per esser vogavanti di galere










Poi sopra il Carro, ove Birrenoil leva, Chetal fa d Amoktante
E cinto ( come già gl Lmperadort ) Eperch'egli@un ?
Dialorowmvece, a' uncarton le chioma, Sentengtaro I hanea' nfarey
Va trionfante al Remo, non a Roma, Che Atalmantil non ha legniyne Mare
STAN ZA XWPLY

Perciò, mentre che tutto ignudo nato, Lat confulte it decreto ha renocate 5 ~
Senonch' egli ha due frasche per brachetra, Sicche di luimndn' ordine 8 >
Sh) bel trofeo si muone, ed e tirato Ed ¢\ Stato spedito un Cancel res
Da quattro canallaccs dacarretta, €on più famigli « farlo-ratzenere o>

. I. Gigante Biancone legato ignudo sopra un carro e condorto fuori di: Palazzo
per esser menazo in Galera; ma quella esecuzione resta sospela, perehé Malman+
tile non haveva', ne Mare, ne galere-, Haba 3 sun-

LARGO Signori', Date luogo; Fate ala. I Latini far far largo dicevano Sum
monere, Orazio. Neque confularis Summoner liter. Vedi sopra C, 11. fam

PIV" lungo a wna lega, Iperbole usatifiima per esprimere Lunghitiimoy Di
atiche pis /ungo a una picea, 6 LO alae

BIKRENO, Intende birro, e'fi dice'cos) per la. similitadine
con Kirreno, che fu amante d' Olimpia,secondo |" Arioito', dal! snes.
capertamente birro diciamo: lo /poso a' Olimpia, th ial ene

CINTA di cartone (a chioma, A coloro,, che per delitti-son las
frufta, asino, o berlina, fogiiono per maggior vilipendio meceereinteta un bet
rettone di fogiio', che per-esser a foggia ai mitra-epiicopale lovehiamano milena,
quali' sono 'quelle, colle quali farono:dipinti nelle itira del PalagiodehPorelta
oggi derro del Bargello', 1 seguaci del caceato Duca @ Areael, le
per l'antichita appena si veggono'. VeditopraG, 6. Man, 56, €eque
per cartone, che pet altro vuol dire quella carta grotia, che (erie
incartar pauat, cc, r

HAVO MO abandiera, Haomo a caso, inconsiderato » volubil
riofo nelle sue operazioni. +k Saga, Url al









SS ae
DVODECIMO,ETVLTIMOCANTARE, 525
IGNVDO nate. Affaito'igdudo % Vedi sopra ©, 2. fan.-64. IL Coloffo ad*noi
; e"mto ignudo; faluo.che ha due frasche per braghertas cio' duc
fogliedi vite-fatte di ferro 5.0 d' altro metailo dorato, che gli cuoprono. le parti





& e SESLOU Re Ub =e a
« CAVALLAGCCE da carretta, Coloro., che in Firenze tengono carrette a vet
ra? per-portar mercanzie yed arnesi da un luogo.a un' altro hanno sempre caval-
lacci vecchi, rifiniti, ¢-ai poco valore, e pero dicendosi cavalio da carretta ys"
intende cavailaccio di tal sorta. Qui il Poeta finge, che il Gigance Biancone fal
smelo sopra.avun carro tirato da quattro di questi cavallacci » perché 1l Colosso
detto Biancone sta sopra ad un carro, che si. figura tirato da quattro, Cavalli
anarini,. > ' 2 a
LA vinocate il Procefso.. Intendi ha: mutata la sentenza, o decreto della galera
havendo considerato, che non se li poteva dare esecuzione, perché Malmantile
non ha gaiere,ne dominio di mare.; '

» » STANZA XVIIL STANZA XIX,
~Hragazzi infrattanto, che son triffi, E perch! ei.nonha in dosso alcuna vefay
© Aveder cio che fuffe, essendo corsi, Lo segnan colpo colpo in modo,tate

Epaich' eglié un prigion,/i fona avvisti, Ch' mmnanzi ch' e finiscan quella feta,
let Bich eglieben legara, e non puo sciarsi, Ne lo fuifaron, e conciaron male;
ly) Wnitamente in un balen provuifi E al miteron, che atorre haueainsefa,
Di bucce, di meluzze, rape, etorsi, ( Bench giammaispuntate auefel' aig,)
* Cominciarono a far achi pri tira, Conquei suot merli, che non ban lepeane,
Ed anche non tiranan fuor di mira. Pigtiar volo alt aria al fin conuenne,
Narra gli strapazzi, ed infulti, che yengon fatti al Biancone, e con questo
smostra il coflume de i ragazzi Fiorentini, i quali quando un malfattore e condot-
-to per la Città in full' afiao, o metio alia berlina, lo trattano nella forma, che
dice del Biancone, tirandogii torli, cioè gambi di cavoli 5 bucce di poponi, e si-
“mili immendizie. £ nota che havendo egli.detto, che Biancone haveva Jamice-
»ra, perché il Coloffo detto Biancone ada ha veramente la mitera » fa che i sa-
~gazzi la levino co i faifi di capo-al Gigante Biancone». i
-4N-nn baleno. Subito; In.un batter d' occhio, detto sopra C, 11, stan. 42. Di-
ciamo anche: in men che noo,balena; essendo il baleno, o il Jampo 4. siccome yil
vento,¢'l fulmine cosa velocifiima, Onde noi d' uno yche corra e sparisca.yia
fuggendo, diciamo = £' pare il vento, Ha fatto comenu baleno. Corre y come units
Yacsta, Pare che"! vento se loporti, Virg. En. |. 5- J,
Primus abit, longeque ante omnia corpora Nifus ont
Emicat,& ventis 5 & fulminis ocyor alis,
Dove quell' Emicat vaic: Scappa fuora,¢ innanzi agli altri, come um lampo, Si
Swede correr la piazza in un baleno,
«»LVON tiran fuer di mira. Colpivano nel luogo, dove segnavano.. Vedi sopra
~C. 1. stan..37. dove troverai co/po colpo, che significa ogai coipo, che ¢' tirana.
Che diciamo anche Zorto bette, Mira e lo stelio che Scopus, voce Greca usata.da'
« Latini,; facta da Scopein, mirare,;
le PkIa Ache finife ques foffa. Primachee' finisse quell? operazione; Si dice
anche + quel/a musica; quel baccano; aes Ȣsimili, Vedi sopra C. ae fh 53+
c xx.; +, tbe

rad

=eSh 2

See CUR SSS

=e

~  a ae

MBAS, %
eng






ss







530 MALMANTILE

MITERONE a torre. Quel foglio, che per derisione si mette i
fattori detto mitera, come habbiamo accennato poco |
doil capo al delinquente, apparisce a i circostanti una roronda t
la parte di sopra di detto foglio molte volte l'intagliano a guisa d
farsi sopr' alle muraglie delle Città; e così havevano fatto a quelle
e'perd il Poeta scherza con la voce merlo, che è un' uccello note
glia dicendo, che se bene i merli, che haveva in capo Biancone n
mar mefle le penve, e non havevano mai spuatate / ali; tuttavia
vouare,ed intende, che quel Afirerone fu fatto volare dalle bucciate,
che gii tirarono quei ragazai, con le quali glielo levarono di testa, s
STANZA AX. STANZA XXIL oy
Paolin Cieco, il qual non ha fuvi pari Ed ci lo donaa Bieco,e a Pasian
Nel fare in piazza giuocolar' i cani, Col carro,e tutcel' altre ap,
E vendea l' operetic, ed e lunari,
E proprio ha genioa spar coi Ciarlatani,
Pens[ato ch' ex farebbe eran denari,
Se quel bestion veniffe alle sue mani,
Pere' baurebbe,a moffrarsi,quel Gigante
Pix caica, che non hebbe l Eiefante.
STANZA XX1
Così prefa fra se risoluzione,; Subito qui Paolino feende,
| Vain Corte a Bieco,¢ lo conduce fuora; Per trouar qualche st buon.
Gili dice il suo pensiero, € lo dispone Havendolo ferrato fra due ee
etchieder il Gigante a Celidora; Accio non sia veduti da persona, Vey































E Bieco andato a ritronar Baldone Bieco a tenerlo con due altri atendey
Tanto l'infipilla, e allora allora E se lo vede muouer 510 ha;
Ei corre alla cugina, e gliene chiede; Ma egliha fortuna, perch écni grande,
Ed ella volentier elielo concede, Che non gli arrina mancod

fande,
Paolino Cieco ottiene da Celidora in dono il Gigante insieme co! carro, sul quale
era, e sul quale lo condufle a Firenze, e si fermo ia fu la Piazza della Signoria,
havendo chiufo dewto Gigante fra due tende; affinché non fatie venduto, e men-
tre.così stando, Paolino cerca d' una stanza, per metteruelo, e farlo poi vedere
a coloro, che havessero pagato un tanto per uno, come si faceva dell' Biefaate,
fuccetle quel, che sentiremo appretio, * ie
“ ELEF ANTE, ¥u condotto in Firenze più anni ono un' Elefant
il popolo per la curiosica correva in gran numero a vederio sotto ie logge
Signoria ( hoggi detta de' Lanzi, perché quivi € il quartiere de' Trabanti, o fan-
ti della guardia del Serenils, Gran Duca da noi chiamati Lanai') dove fava rin-
chiufo in un tavolato, e si pagavano alcune crazie per entrarvia vederlos ¢
fio animale fingulare ne i noltri paefi, mori in Firenze per lo gra freddy elas
sua pelle ripiena, e lo scheletro nettato, e meffo insieme si confervano nella Gal-
leria del Serenifs. Gran Duca. ucoensini aie
INZIPILLO'. Inttigo, stimold, pregd instantemente, e forse voce corrottas —
Sill:







da hbillare, Latino foilare, infufurrare, trovandolt nella' flor

traccaco feume: Di-ninwa miferedenca era stato antore 5 e nulla male:
date, ta





DVODECIMO,ED VLTIMOCANTARE, = 31

TRAINO. Diciamo quella quantità di roba, che possono strascinare duc buoi,
che i contadini dicono trainare, ed il veicolo chiamano traino, o treggia, La-
tino traba, o trahea, a trahendo, Virg. Georg. 1, Tribulaque, trabeaque, © ini-
que pondere rafiri. Si dice anche sraine una mafura di travi, che contiene quattro
Breccia quadre. Qui intende quel carro, sopra il quale era il Biancone con tutti
phate arnesi, e pigia la voce sraino nel significato della voce rreno usata per

rsi intendere carro, e bagaglio dell' artiglierie; !a qual voce s'accorda 'colla.s
Franzefe Train. Noi percio la diciamo ora Treno,rappresentando quella prooun-
zia; ora 77a:mo coll' accento fulia prima, non facendo conto della pronunzias
Oltramontana, ma della (crittura. Qui il Poeta dice Traine coll' accento fulla.
penultima; per accomodarsi alla neccitsta della rima. Franco Sacchetti nelle Ri-
me fimiimente pose questa voce nelia fine d' un verso,

Per tirar colti piedi un gran traino,

LA Piazza dela Synoria, La Piazza, che hoggi si dice Piazza del Gran Du-
a,¢ si diceva de' Signori, o della Signoria, perché è d' avanti al Palazzo de'
Priori, e Gonfalonicri di Firenze, che si dicevano la Signoria, nella qual Piazza
@ la fuddewta loggia, detta de' Lanzi

CHE non gli arrsva manco alle matande, Cioè non gli arriva ai bellico, perché
mutande chiamiamo propriamente certe piccole brache, le quali si potiany,quan-
do si va a bagnarsi in Arno, per coprire le parti vergognofe, le quali mutagde»
per ordinario cuoprono dai bellico fino al principio della colcia.

STANZA XAlV, STANZA XXV.
Piange Siancone, e chiede altrui mercede, Quei tre yc ognor came cuciti a i fianchi,

E mentre il Fato,e la Fortuna accufa,

Euor delle tende si guardo gira,e vede
« Perfeoy'ha in man la testa di Medufa,

E immoto resta li da capo a piede,

Ne piit si duol yma tien la bocca chiusa,

Perché col Carro, e tutta la sua muta

De cavallaccs in marmo si tramuta,

Gi favan quivi,accioch'ei nofeappaffe,
Privi di senso allora,e freddi,e bianchi
eAnch' eglino si fanna immobil faffo.
Ata perchs'l protungarmi non vi stachi,
Giie me',c' a Malmantile io mene palji,
Ove git amici Paride ritrova,

E sente,¢' ogni cosa si rinnova,



Ii Gigante Biancone era così grande, che avanzava il capo sopr' alle tende;
nel girare, che egii fece la testa verlo la loggia de' Lanzi, vedde i tcichio di Me-
dufa tenuta in mano da Perfeo; per la qual vilta rimafe immobile, e diveanes
faflo tanto lui, quanto il carro, i cavalli, e coloro, che gli erano d' attorno; B
così il Poeta da la sua fine, e si sbriga dal Gigante; di poi ritcorna a dilcorrer di

 quel che si faceva a Malmantile.

PERS EO, ¢' ha in man la testa di Medufa, Questa è una statua di bronzo, las
ae € fiuata sotto un' arco di detta loggia de' Lanzi; opera di Benucnuto Cel-

i; e rappresenta Perfeo con la testa di Medula in mano, verso ia quale flacua,
guarda il Coloffo detto Biancone, percht e di marmo bianco. E nota la fayola
di Perfeo figliuolo di Giove, edi Danac, 11 quale uccile Medufa figiiuola di For-
co strupata da Nettunao nel Tempio di Pallade, la quale percio sdegnata conver-
tii capelli di Medwla in ferpi, e fece che la sua facia faceili diventare di faffo
coloro, che la guardaflero: Ma il detto Perfeo havuti da Mercurio gli stivali, ¢
la scimitarra, mentre Medufa dormiva s le tagho ia celta, la quale pot ee

xX 2 mefle












sg IFATHAI OM ES +9 vi h
AS3HL ) pire Ofisip MALM he thy; 4 ie on si
miefie nel proprio 'feudo, Di questa favola si servé il! Poeta 4 } 7
gante;dicendo, che per haver' eghi mirato questa tettadé-]
marmo, € così da graziosamente una favoloia origine a questo f
rappresenta Nettenno Dio del Mare', ed! è»posto nella: Piazza' del G
sopr'ad un carro tirato da-quattro cavalli marini nel mezzo a una
quale riceve I acqua', 'che featurifee davaleuni niechi, © conchiglic
in mano da alcune statue di Tritoni-alte quanto le gamberdel d
or dette flawe stanno attorno:"E queste il Poera finge', che sieno

mipagni, che dice fargli cucits a i fianchi, e che non gli arrinano a le
dé'; E così viene a conformarsi col gruppo, che si vede di queste ttarue
fo tutto di marmo,

CVCIT 1 ai fanchi, Stretti attorno, come se fuffero euciti, Detto uk
per'esprimere uno, che mai si levi-d' attorno a un' altro;€ qui corna bene,|
Ché quelle flatue sono così strette attorno aj Coloffo, che paiono cavate:
fo marmo, del quale e cavato il Colosso.

GLle me', Glié meglioy. Vedi sopra C. 2. st, 10. <a? S

PSTANZA XXVEy STANZA XXVHE
Poicht Baldone eAalmantile ha preso, Cos} cercando le grandexe i |

E tutte quelle povere brigate Soe @altrihor feo Ve, ?
Saluopera chi non si fuffe arrefo ) Onde tornata Celidora, il Lage
mii se ne son ite a gambe akace 5 De i popoli padrona, e dello Stato
Sitché'da queite havendo al fin coprefo Temendo ancor de'








.


















'Pot Bertinella, ch ella l ha infilate; Nuovi Miniffri fa, nuove ve i
Perammazzarsi sfodera un pugrale, Se ben de i primi poco ha da temere
Ada quei,ch'é buono,non le vuol far male, Che tutes ban ripiegate le bandiert i
STANZA XXVIL. STANZA BALK
Clienon fo come gli esce fra le dita, E per eftinguer la memoria i
B/fulta in Strada, che le gabe ha destre, Di Bertinella in ogni gente ye-loco }
Ov" ella a ripigliarlo'é pos /pedita Si levan le sue armi, il suo ritratto,

Tagliato in croce si condanna al fusce's
aE perch'elt habbia a raccorciar, la gita, Vn bando va di poi, & averum patto
Le fa pigliar la via dalle finefire; Neffan ne parti pite punto ye poco
NEMa wa sh 5 ma poco poi le importa Sotto pena di fear in fu la fume
MT rovaricht amarza,se viginnge morta, Quattro mefi al palarzo del Com
Celidora tornata padrona di Malmantile fa buttar Bertinelia
ordina nuovi Magiftrati, e comanda, che non si parli più di Berti

villime'pene. jo faites

Dix'chi dopo di lei fa le mineftre;








ELLAL ha infilate. \ofilar le pentole, vuol dire Esser rovinato
ver finito-, o perduto la roba, e la vita, ec, che di tutto s*in cok
mente. “tale ? ha inflate. Latino decoxit. +20 sett

LE gambe ha dere, Non, che quel pugnale haveffe gan
dire } cheetlendo grave, gli fu facile andar' a baffo in strada';
perie 'finestre anche Bertinella da chi fa le minefre, cioè dachi
avichi comand; chee Celidora ritornata padrona di Malmanule.
gacge ae peccato, Ha la pena det suo fallire, e che ha m
whet; Fs










1 v¢ t E LAN 7
-.  DVODECIMO;EDVLTIMOCANTARE. | 533
' flaver voluto per strade indirette farsi Regina', usirpando queld' altri iio! is > /
be i. icsanlig voghamosintendere uno, che piocenea oe taper fare Ogni-cosa
meglio degii altri diciamo; M.raleeit Lagi, Che il Lagi fu anticamente un Sen-
icato wv Firenze, che faceva tutti i negozzj della piazza': Si dices
rO per scherzo, e per una certa ironia, e derisione. ho “ogee
* HANNO ripiegato le bandiere, Cioèhanno finito; Son morte, Il Petfiani,
parlando di se medesimo in questo proposito disse + ty
core edi primo tramontano a quest® ascintte ae)
si Be Ditems pure sl requie,¢ il Miferere,:
Perch' so fo vela, e piego le bandiere; '
 E buona notte; a rinederci tutti,
LE fue' arm, Intendi'}*infegne della sua cafata, o stirpe. ues
7 ~ STAR in [u la fune quattro mefi. Now & posibile' star in fa la coda quattro
y hore, non che quattro mefi., ond' io penso, che con questa iperbole voglia iaten-
sia condennato alla morte, alludendo agi' impiccati, che in un certo modo
quando pendono daile forche a vista del
popolo; st poslono dire stare in sulla corde,

be in fulla fune.
¢ STANZA XXX. “STANZA* XXXIIL
jee | Yr Orarore intanto de' più brani Spiegafi se desea 4 ttn tavolotto
" ACelidura Aaimamue inuia, Vol abito mavi di mezzalana,
a 'Che det Caffelo ad essa da le chiavi, Che infu fianchi appiccato ha per diforto
Evende omaggio con la diceria; Pn lindo rid aief alla Romana;



Ed ella in detti macffofi, e gravi
Pronta risp a tant' Ambasceria;
Inds le chiavi piglia ye nn' altro mazzo
= Wi quelle delle stanze det palazzo.
ae STANZA AXAIL,

E perché gli è un perro, ch' eli' ha voglia
Di riveder, come ad arnesi e pieno;
Del Mamoye d'altri addobbsfi di/poglia,
E comincia a girarlo dal terreno;
4Guardarobi aspetta, ead ogm fogiia,

Poi viene un verde nuouo camiciotto
Con bianche imbaftiture alla balana;
E poi due trincterate camicinole,
Che fanno piatza d' arme alle tignuole,
STANZA XXXIV,

Vua Rimarra pur difaianera, ~~
Per dove si fa a' faffi arcisquisita,
Perché gli aliorti, e it banero a spalliera
Pavan la teftaye in giu meza la vita,
Portandola alle

'i; te,o0anna fitra,
\C* ad aprer gli usci patono it baleno; Torre,e comprar si pio roba infinica,
è E subito poi lefto-uno safiere Cb elt" hadue manicon s) badiali,
mn Quand' elta palfa, le alza le portiere. Che è ine quattordici arfenali.
f STANZA XXAIL, STANZA -XXXV,
Ed ella se ne va sicura,e franca, Vina cappa tane bella, e pula
b Sapendo ogms traforo a munadito, Di cotone; se ben vesta indecifo,
ie Perché troppo.non è, ch'ella ne manca, S' ell'¢ di drappo, o pur ringiovanita,
EP abito, fin quando havea mario, Perché non se ie vede pelo in vifo,
'0 Scefe; )£i70 y fali ne mat fu hance s Evvi @ abiti pur copiainfinica,
2 t Sin che non hebbe di veder finite; Mia chi unto, chi roto, e chi ricifo;
2 All' ulssvia si fece in guardaroba Che il tempo guasta tutto; e per marura
% eAprir gli armadi,e cavar fuor la roba, Cosa bella quagzit pala, e non diira,
4 Malmantue manda un suo Ambalciadore, o Depataco a renaer' wbienes:
a Ce.
f.





E





44.534
a Celidora; ¢d ella attualmente, e corporalmen
tutte le stanze del Palazzo, ed in Guardaroba fa la
veramente adeguati a una Regina'di Malmantile.. 3
RENDE a la diceria, Cioè fece una Orazione d'
mone, o Discorso, col quale refe ubbidienza. is. 4
HA voglia di rinedere. Ii Poeta (prime benissimo il genio unit
fire donne, quale è di rivedere tutte le caffe 5 armadi, ec. subito, che
© maritaggio entrano in una casa a loro nuova, ho isch ete
TERRENO., S' intendono qui, secondo l'uso., le prime fanze d' una cal
che sono al piano della frada, Del reo Terrenoé la tetra stessa così,0 così ¢
dizionata. Latino terrenum; folum, ager.» - send 5 -
PALONO il baieno, Cioè tannopretto, Dante Pars 25. Subito
di baleno. Inf, 22. i2 men, che non bafena, vatiot ”
OGNI traforo. Antendi ogni porta, ognicriuscita,/ogni minima:
4A MENA dito, Sa benitimo. Latino caller, Le sono notissime st
L ABITO' fin quando banea marito, Celidora, comes' e detto sopra C,
Fu moglie del Re di Malmantile, e da lui haveva ereditato i Regno, i)
MAVE, Color wrchino chiaro. Azzurro sbiancato, i
GV ARDINF ANTE. Vedi sopra C. 5. tt. 8. *: geomet
MEZZ ALANA, Tela fatta di lino,è lana, che inuna fola parola si dices
ancora acce//ana, quali accia, e (ana; roba assai da i nostri Contadini.) |
C.AMICIOTTO.. Così chiamano le Contadine,quella'velteda donna, cheles
Fiorentine chiamano fortana, Et
CON bianche imbaftiture alla baixana, Costumano le nostre Contadine di fares
nelle loro vesti yicino a terra una cintura con punti di refe bianco in sul nero jun-
ghi, acciocché si veggano da lontano, e queiti punti foftengono una piegaturas
fatta nel giro di detta velte per accortarla, e serve a loro per ornamento,0 guat-
nizione, e si danno ad intendere di far creder nuova la medesima — causa
di quella punteggiatura, e che aliora sia uscita delle mani del Sarto; il ee
quando vuole imbaftire,.0 dar priueipio a cucire yo' abito per mettere int 9
eda segno i pezzi, che vuol cucire, e solito fare tal punteggiatura larga y das
queito imbaffire si dice imbaftitura altrimenti feffitura, © ritreppio, Latino /ubfutnr4.
E questo verbo smba/tire servc per intendere ogni cosa principiata,e non perfezio-




























nata; come éo ho imba(Pito L' orazione, che debbo recitare 5 ed in poche ere ”:
che diciamo abboyzare. we

BALZ ANA. Iniendono il giro da piedi della veste; altrove Pideos 'Latino
limbus « LF

TRINCIER AT E.camicinole.. Vuol dit camiciuole consumate dalle tignuoles »
per la similitudine, che e tra una campagna pieaa di trinciere, ed.un panno ple
no d' intignature, che percio apparisce bucato, € trinciato, Vedi sopraC. 8. st
51. E.che cosa sia camiciuola. Vedi sopra C, 6, st: 57, at otwe att

BANNO piagza a arme alle tignuole, Vedi opra Co. 51. -questo medesimo
concetto sopra il capo del Tura; B che sia tignuola al C, 6. st. 54. € Cs 10. (h 12+

ZIMARRA, Abito, che già ulavano portare le Donne Fiorenti all?
altro abito detto /orrana; il quaic da i Latini e detto amiculam, il qual'

' YY









tie
a

“= SSeeresiut







DVODECIMO,EDVLTIMOCANTARE. ~ 535

'veramente assai decorofo, e modeflo, e non come quello, che usano hoggi, del
quale si può dire:con Quinto Curzio lib. 5. Feminarum conniusa inenntinm in prin-
cipio modeftus eft habitus, dewde fumma quaque amicula exuunt, panlatimque pudoré
profanant, ad ultimum ima corporum velamenta proyjciunt, Ma tornando a proposi-
to: Questa specie d'abito detto Zimarra haveva intorno al collo un collare gean-
de (che chiamayano bavero ) fatto di tela incollata,e cartone,e ripieno di stecche
d' offo di balena; ed in fu le spalle, dove ha principio il braccio un giretto actor-
no al braccio farto della stessa roba, che il bavero ) qual giretto il nostro Autore
appella aliotti, perch così si chiama, ed alle volte si dice piffagne ) dal quaies
pendeva una manica larga.,¢ grande quanto una buona sporta, la qual manicas
non s' imbracciava, ma serviva così pendente per ornamento, e per una certas

“grave accompagnatura; ed oltre a questo dava commodita di riporvi fazzoletto,

Oaltro, che occorretie. Di queste maniche, tali se ne son vedute a' mici giorni,
che farebbono fiate capaci di cinquanta libbre di grano l'una, e più; © però il
Poeta dice, che sono il caso per andare alle nozze, ed ai mercati, perché vi si
può mettere molta roba dentro: E gli-aliorri, e banero difenderebbono da un col-
Po in riguardo della roba, di cui son compolli; E dice /a rea; perché questi ha-
veri, nascondevano dentro di loro tutto 11 capo di chi gli portava; e tali aliorti
si sono veduti, i quali coprivano più di inezzo il braccio.

DOVE si fa ai fafi, Dove si tirano le fafiate; il che segue in Firenze in Mer-
cato nuovo, dove 1 garzonetti delic butteghe de' Setaioli quindici, o venti giorni
avanti alla Solennica di S, Gio, Batilla fra il mezzodi, e il vespro fanno fra- di
loro alle fafiate, e necetiitano tutti li bottegai di quelle contrade intorno al Mer-
cato nuovo a star ferrace per quell' ore; e questo fanno per folennizzare la decta
fefta quel tempo innanai; e per questa ragione tutte le botteghe, che sono in quel-
la firada, dove tirano i fatfi, hanao la riuscita in aleca strada per di dietro, di
dove entrano i macitri, e lavoranti, senza aprire lo (portello principale, e quivi
attendendo a i lor lavori, laiciauo che i loro ragazzr Gi piglino per quell'ore tale
spaffo 5 anzi ci (ono taiuoica de i maeitri, che comandano a1 loro ragazzi, che
vadano a pigliarii, spaveatati da un profetico detto: Guai a Firenze, quando in.
Mercato non si fara ai faffi; V sano di fare a' fai anche in Roma i ragazzi Tra-
fleverini. E fare a' faji, Hgucacaiente s' iotende » Mandar male, rovinarsi, get-
tar via il suo, Latino di/apidare, fare alla peggio, e operare senza giudizio; si
faceva a' faffi ancora in Firenze per accafione d' allegreeze pubbliche, e una fine-
fica di rame traforata fu posta al Palazzo de' Medici,oggi de' Marchefi Riccardi
Per vedere questo spettacolo, come e sato da altri scritto, ed osservato.

ARCISLVISITO, Ui cafissimo, buoniitimo,, attissimo, e pil, se più si può E

dire. B' un termine, ches' usa per farsi intendere; più fu, che il superiativo, di-
cendosi buono, pil buono, buoniflimo, ed arcibuonitime. Ma dicendosi buono,
Migliore, in vece di più buono, e isquisito in vece di buonissimo, che fa.
V effetto del superlativo di buono, non pare che sia ben detto più isquisito, e»
isquifitissimo,facendosi cosi'un superlativo di superlativo; tuttavia per J'ulo inteo-
dotto non farebbe riprefo chi lo facetle; ed io crederei, che fufle meno biatime-
vole dire, arcisquisite, che isquifititfimo, perch non trovo troppo in uso il dire
pil isquisito 5 onde non può s' uso antrodurre isquifitisimo.s che toguirebbe al più
squi.

ae





536, uuie nohace dae aaa
isquisito..;L Latini dicono bonus, melior 9 che: Q d F
byono,, migliore 5 e i/quifite; ed io conde cic i che Bctedfinn piste ass s
miffiaus:, che faonerebbe più isquisito, isquifitissimo, se it I

trova eptimiffimus.. Appretio det nostri Autori Toscani si trova, 1
molto, aijai, e simili a i superlativi, come notammo Coat 17. >} ia r
buona grazia di efi, lo flimo.errore, perché molto, più 5" » Hiufilis
faculta di scemare, e non cre(cere il superlativo,. aa

er esempio if tale e Luonissimo, vuol dite il tale è perferramente
iamo molto, certo', che (cemiamo la perfezione di buono 5
molto buono, ma von perfettamente buono, eficado maolte una'
s2.5 € non indeterminata, come e i} superlativo: EB — » che
1iguifito, e isquifieissimo, © arcisquisite, hanno prefa la vace s
tivo da per se, e non come per superlative di buano; il che vi 7

tofna poi all' addigteivo aigiiore, che non riseve altrazione 5 nomdicendof » nondicendof i;
migliore, Be migliorissimo, le hen si dicewmalte migiione 5 e:alfai mia
marlod' essenza;. come ia bbiia thes detto s»perché solo 5.0 affat miglit
men buono, che non fa migeiore assolutamente detto:, se non comparando:
all: altra quale sia.di loro meglio, st Amr
i ZANE, Colore fra il paonazo,¢ i} lionato. 2p
OTONE, Vuoldire bambagia non filata, Manoi per cotone:
sorta dipanao col pelo annodato; come.è la saia rovelcia 50 il rovele [
Hon si dicono corone se non. hanno il pelo aanodato, che allora si dicano di coteney
© actoronati, Dice, che num e certo se sia rowescia 5 o drappa 5) pceaim
la feta. 2 ellendogli caduto il pelo, per esser logoro je perché:è senza pelo dice

che € riagiouanito; Sicch¢ in fuftanza vyol dire che: era usato, i lal.
R(CISO, Qui vale per intendere consumato nelle piegature d'un di !

















epanno 5, per essere stato così piegato lungo tempo; che per altro ri
“un Jegho, o altro materiale tagliato ne] mezzo yed e il contrario' div rife se






nel oy pela per illungo, Vedi sopra ©. 11, tan, 36, ricife,
ANZA XXXVI. STANZA X&K.
Basta es eve qualcofa un po cattind, Due altre 'armadj poi i fur

Che Celidora ha quint abiti, e panni, Che ? anoe tutto
Che al certo (tuttanolta ch' ella vina )
Puofrancamence andar in lq co gli anni y
Ma perché al [uo char magnonos'arriua E un' altro di pin tr
'Di certe roppe, scampoli, e foppanni + Bealze ye fearpe ye)
Top Wimpaccio vollese a quella gente, Chea vedersi p er
"Ch edt'ha a' intorno,farne un belpresete, Ve poi'la nidoigi
. STANZA xxxViil st
mui se si parte ed Apre uno Riperto A
2 intagli,e a' arabe/chi ornato,e ricco,
“E trois due cafferse di belletto
Cort! altre di pezrette, e @ orichitco
va il Poeta a narrare glia arnesi,e
hon si parte dallo feh
' faye | Ae a ee













n





ME






Si elcr ibs.

a

SERRE ES © =

SSE ST Pesrsr st i ta.



DVODECIMO,EDVLTIMOCANTARE: 537

contro alle donne, mostra; che se usano il belletto, ed il liscio, hanno anche
bisogno della medicina da rogna, e del rottorio.

VN po cattiua, Quel po vuol dir poco per la figtira Apocope; ed un poco cat-
tiva 5 trattandosi di abiti, e d' altri materiali, s' intende per lo pit', consumati,
2

vecchi.
TVTT AVOLT A, ch' ella viva, Pub francamente andar in da con eli anm, Pav

che voglia dire, che se Celidora vivera, ha tanti abiti, che le basteranno molti.

anni senza farsene di nuovo; Ma dall' essere gli abiti della detta qualita, si com-
prende, che scherzando vuol dire, che se Celidora vive, invecchiera, percht
andar in Id con gli anni vuol dire invecchiare, come s' accennd sopra C, 2, stan. 2.

(siginines Ritagli, pezzi di panno, o drappo. Scampoli, vedi sopra C. 11;

in. 22.

SOPP:ANNI, Fodere, cioè tele vecchie, che hanno servito per fodere d'abiti.
Scherzando burla la generosita di Celidora, la quale con queste galanti ciarpe,
che son fondacci d' una bottega di rigatticre, o ferravecchio, regala i (uoi pil:
cari per non apparir meno generosa di Bertincila, che regalo la patcona, come
vedemmo sopra:C. 1. stan. 81. a:

D* oronetro: Par che dica d' oro pulito, e puro, ma intende wetto d' oro, cioè
puro; senz' oro, Equivoco usatissimo in'quelto propotite,

LA miaferizia per la casa. Incendiamo 11 Cariello', o turacciolo del ceffo; es

flo 5 perché un tale detto Galeno, che andava per Firenze vendendo tali cariel-
li, gridava shi vnol la mafferizia per la casa, in vece di dire, chi vuol Carielli; od
era bene inteso.da cutti,

RABESCHI, o Arabe/chi, Specie di pittura fatta a fogliami, fiori,
mascheroni., © altro, tutto aggrottelcato, cioc sproporzionato dal naturale, detto cosi,
perché forse tal maniera sia venuta d' Arabia, secondo che si può dedurre
da. Cel. Rodig. Jib. 29. ¢. 5. dove trattando delle Lamie, e delic Sircae, dice;
LaAmmiam vero opera parerga ex Arabia maftichen vocant,

SELLETTO. Liscio. Mestura, con la quale si lisciano, ed imbellettano les
donne « Vedi sopra C. 9. stan, 38.

PEZZETTE. Sano pezzi di tela bambagina tinti col cremisi, e zucchero, ed
altre sono di carta fabbricate in Spagna, e se ne servono le femmine per colorirsi
di rosso la faccia.

ORICHICCO. Gomma di Ciriegio, di Pesco, o di Sufino, ec. della quale si
servono le femmine per lustrarsi la faccia, e per appiccarsi veli in fu la teita.

“PER Jambicco. Adagio adagio scaturendo da piccioli fori fatti nel coperchio
del fiaschetto., come s'ufa dei' acque odorifere. Lambicco e il nao della campa~
na,¢ d' ogni cappelio per uso di stillare, donde /ambiceare, e pafsar per lambicco,
# intende stillare; B /ambiccare, o lambiccarsi it ceruello, e lo stello che mulmare,
detto sopra C, 10, stan.7.

ALLERA, Pianta nota, le di cui foglie eruono per cauteri; e così i ceci bian.
chi, li quali per tal effetto erano ia quello (tipo.. Da queite cose vili comprenda
il Lettore, che il Poera si maaticne sempre in fu gli (cherai, deferivendo una Re-
gina, e Palazzo ricchi di quegli addobbi, che son conuenienti a una beac stant
cOntadina, e decenti alla grandezza d'una Regina di Maimantile,,

% Yyy. STAN,















Sh. MALMAN TILE 1980
STANZA XXXIX,.., ith NZ 3
dun caffon diferro vada REREO y c i i co#lor |
L Quiuitvoua il morto, nia dd vero, s -
Che i diamantiye le givie di gram pregro
Lon v'bano che far nidla,e sono un zero;
Lerche si tratta, che vi. Safe un wero
bi perle, che se ben pendeana in nero
Examsi grosse, che st [parfe vace,;
Ch' ell' eran poca manco d' una. noce; Sun i quartrini 5 i precioli ye i bateati,
STANZA XXXX STANZA -XXXXIL
D? anells ya! orecchini Vé1h marame} 'Poi ne venixan gli occhidiciueste;
“Tanti gioie!ls pot, ch' e un fracasso; Ma il proseguir più olere fa interrortes,
Perc' alla donwa:

Di medaghe dorate 50, vavindi-rame' a
dir, che" Duca levolea far

Vn. moggio ne mifurano,, @ di palo;
Ala quella e sparr ates, ed nn litame Ond? ella il tatto nelcafjon rimette
















Risperto alle monere 5 che più baffo E riferrato feende giwdi (orto,
Le piit belle comparfero del mondo; Oue Baldon ? aspercarn iftinali,
Ch! in faseri poses creffi stanvo al fonda: -» one partir di quini fha'im sul? ali >

: STANZA: EMBKI MO vinnd cow 2

Per e agginftare omas tutte le cose 5 In punto, @ questo fine aller
Che pin desiderar non si potea in: ier 'bined
Egli, ch' eva per far come le/pose La puliva.per metterie la fellay



LA ritornata s idef? alla Dacea, Licenrioffs costidullasorellay © >
Celidora trova il caffone de'.danari,, e coi tal-occatione i Poera'
monete Fiorentine eficttive, ed immaginarie.. kn tanto che Celidora va vedendo
guefte ricchezze; vien da lei Baldone-suo cugino per liceoziativ) 9
TROFA it morta, Cioè trova il buond. Diciamo rrewar it morta, o fare nits
morto, quand' uno trova ripod qualche gran vallente, © fa in gua-
dagno.. A. P
LON o ha che far nulla, Par che voglia dire non si stimano, vispette al? altres
Givie, che sono in.quet /uege; ma in eisai vuol dire; che quedo non e luoge per toro
cioè non ve ue fond, i b tone Ses
Sf trata. Si discorre; Termine assai usato per esprimere una che
s' habbia di qualche cola'; quasi-dica > Si difeorre comunemente, che'
così..
AL marame. Una quantità grandissima. Marame propriamente vuol dire ogni
rifiuto di mercanzia, come quella, che dal mare è gettata a' riva bi i”
tum, Ma quando diciamo marame nel modo; che! & detto: nel eel
intendiamo abbondanza così grandé.d' una cosa y che generi naulea, €
disprezzabile la medesima cosa. Fra i nottci Contadini Gedice
tendefi ? avanzo 5 e rifiuto delle frutte rimatte lord, dopo. la celta', o° vel
delle migliori » noa fo s¢ essi Rroppiano'la nostraparola y o-feonoi Cori
la loro 5 dico bene che mi pare più fighificante; Amaramejehe J
Fiorentino quello 5 che questo 5 che per così dire', ha del Nape
Vedi il Vocabolario della Cru(ca alla voce Cerna',

UN fracasso. È lo stesso che un flagello, un barbaglio detto sopra C. 7. stan. 5.

UN moggio. Il nostro moggio è di staia 24, lo staio è di libbre 50.\ di grano, e
la nostra libbra è once dodici, Ma qui è detto iperbolico, è significa quantità
grandissima.

RISPETTO a questo, A paragone di questo; cioè a paragone delle monete,
che son più basso.

I pesci grossi stanno al fondo, Detto, che significa: Il meglio sta nel fondo.

PIASTRA, È lo Scudo, o Ducato d'argento Fiorentino, che vale lire sette
ed è moneta effettiva. Il Fiorino è moneta immaginaria, e valeva quando più,
e quando meno, essendoci anche il fiorino d'oro, che forse è quello che habbiamo
ancora hoggi d'oro effettivo, e lo chiamiamo zecchino gigliato, ma il fiorino
ne immaginario, ne effettivo appresso di noi non è più in uso, Scudo d'oro
è moneta immaginaria usata da i Mercanti per facilita di scrittura, valutandolo
lire sette, e mezzo, se ben molti per scudo d'oro intendono la mezza doppia.
La Lira moneta d'argento effettiva, e si chiama Cosimo, e vale dodici crazie.
Il Giulio, che si chiama anche Pavolo è moneta d' argento, e vale otto crazie,
Il Carlino pur d'argento effettivo ne vale sci; ed il Testone val due lire; questa
moneta già in Firenze si chiamò Riccio, dall'impronta della testa del Duca
Alessandro de' Medici, che era ricciuta. La mezza piastra e d' argento effettiva,
e vale lire tre, e mezzo. La crazia è moneta d' argento basso, ed è l'ottava
parte del giulio. Il quattrino è moneta di bronzo effettiva, ed è la quinta parte
della crazia. Il soldo moneta immaginaria che vale tre quattrini; ed il battuto
ne vale due: hoggi l'habbiamo ambedue di bronzo effettive. Il quattrino si divide
in quattro denari di bronzo effettivi, ma hoggi non se ne vedono, se non in
occasione di tributi Ecclesiastici, che sono presentati, e son poi resi, perché gli
possano haver un'altr'anno.

OCCHI di Civetta, intende le monete d'oro, come il doblone, che vale lire
quaranta. La doppia, che vale lire venti. La mezza doppia, che vale lire dieci, Il
quarto di doppia, che vale lire cinque. L' ottavo di doppia, che vale lire due, e
mezzo, che tutte sono d'oro effettive. Habbiamo ancora il zecchino, il quale
chiamiamo gigliato, che vale lire dodici, ed è il più purgato, oro che si conij, e
si può dire il nostro unghero. Si trovano ancora de' dobloni di quattro, e cinque,
e di sei doppie l'uno, di conio Fiorentino.

SPAKTIMENT!, Divisioni, feparamenti. Chiamiamo spartimenti quelle,
divisioni di'tereeno, che Gi fanno ne 1 giardini per piantarvi le cipolle da tiori.

ali (partimenti:, se bene sono di diverle figure, si dicono anche quairi. Vedi
pe C,6,-ttan. 63..E per similitudine aiciamo spartimenti te divisioni » che si
trovano ineafiecte, o scatole, come erano queiti delle monere,

VENNERO pris hafere. Intendi Avvisi, o imbalciace 5 che Staferta appresso
di noi,¢:1o stesso, che Corriere. Sp. efafera.

\ BAR matte'. Elo stesso che abbaccarli con uno e parlargli, Vedi sopra C,
2, stan. 59.,in altro significato, — > ne

STA sm full aii. EP all' ordine per partirsi. SST

. FAR come le spose. Significa ritornare; lo dichiara il Poeta medesimo,dicendo:
Tdeft la ritornata; E questo perché già coflumavafi, e forse ancora in alcuni Iu




es:

aS

, ee ee ee

=e 2 SS Sw

PR 6S we

d-
Koyy 1s ghi




=






s4o
hi@eoRitma, che le si dopo'effere state dicti', o pre:
foie rotniao alla casa paceraa's Fer sephe qui git

Teniarns dell Achinea. Taupe lo fallone, 'che* cated
che Achinea, o Chinea, intendiamo il cavailo buon rer
éuina specie di cavaili particolare «Sp, bacanea. Franz, bacquenen'y
STANZA XXXXIV, ts “STANZA. Xx
O mai è tempo, cara Celidora,
Ch! inverso li miei [udditi m' apprefi >
Che 'l trattene*mi di vanvargio faora y






















Pregsndicar potrebbe a' miei interessi Dite, non ci oi fulle corda,

Pero qui refea tu co! tuoi,sn buon bee, Bifog a Lmeteae epee a
E farti anwe, e rispercar da essi y ( Rispsfe il General) 3 ella 8

Ed in ordine a questo i conviene ee ome t

Fare anche un' altra cosa per tuo bene,
STANZA XXXKXV.
Perché, s' io parte ei »cugina mia,
Non fo 2 se tn ci havraituttiitnsigufti,
Che qui non è neffun., che per te sia,
Mentre forsee poi nuowi difeusti,
Ma voglia il Ciel, ch' io dica la bugias

Ed ogni modo vo', che tut' aggiuiti, 'tipo presto sles of
Per sicurtà con an compatie » Ugquale Vuolotu? parla. a Her =e |

S accafi teco,¢ qucfto, e il Generale, D: mat più si, e daccela in fa

STANZA XXKXVL STANZA AKAM;

LT byei hati difender si da vanto, Ed ella nel sentir, cons eit affrin

Che tn vedi,egli ¢branoquarun Marte, A dar pronta: rispotta atal do

E se finor per noi ha fatto tanto, D' un modefto roffor tutta,

Pifa quel che eifara,s'egli entra aparte,

Orsit  daglt la mano; cana [it ilgnanto;

E voi non ve ne fiate più in disparte,

Casa Latoni, o Amostante nostro

Fareui innanzi, dite il fatto vostro

STANZA L

Degli dunque la mano in mia prejenzas 3 Ma per non recar tedio

E voi, o General, datela a les, Ideft a chi ascolta i versi mitiy

Ch io 'voglio prima della mia partenca

Veder folennizzar questi Flimenci. La[cidgliyadiame;

Baldone da per sposa Celidora al Generale Amoitaate Latoni “ai

dopo haver narrato il discorso fatto da Baldone a paliow per indurlaa

tarsi d' haver questo marito, ed i soliti lezzi donneschi farti da 2

dir di si; paffa a di(correr d' un' altra sposa, che e Psiche, cone ee i

Lagere ouave.

hai neJunsche per te sia « Non hai nefiyno, she aid















a


'

DVODECIMO,EDVLTIMOCANTARE. 548

OVVTA. Termine che significa spedizione, © incalzamenio a far presto. BE' il
Latino Hia,age. Vedi sopra C. 6, fan, go. alla voce, horse, aiegh 3
PASS ATE gud. Venite qua. Lat. ade/dum. B: modo di dire, che significas
comandar con imperio,.¢ con (everita, ed ha del bravatorio. R59,
SE vi piace la pannina, Se vi piace la mereanaia y cioè Celidora.
NON ¢i tenete piit in fulla corda. Non ci fate più Aentage,o desiderar la rispo-
fla. Nom cé renere piis coll' animp dubbio, e sospese,:
SON bell' e accordato. lo sono, affatto d' accordo; son contentissimo. Vedi fo-
pra C. 3. faa, 14, Questo termine bee, '
TERREL d bauerne di beato, Lo riputerei mia gran felicita, Stimerei d' haver
gran forte, WV' avrei di carti, Mi terrei d' etfer beato, ee
EGL1¢ dower sentir  altca campana. E' cosa giulta sentir I altra parte,
TRANA, Questa voce non havrebbe alcun significaco, se bene e assai usata 5
ma perché pace, che immiti il suono della tromba, quando si da la moffa a i ca~
vali, che corrono al palio; ci serve per esprimer mxovité  /pedi/citi, sbrigati a.
far la tal cosa, Q pure e detto Trana, cioè tra' pur/d tira avanti; dal verbo Tra-
nare, che vale trarre con fatica qualche cosa, e strascinarla.

, ALAL più. Questo termine usato nel modo, che è nella presente Ortava, ci è
familiarissimo, ¢d ha quati lo stesso significato che evvia detto poco sopra, e s'ula
Pua per F altro in occatione di stimolar qualcheduno a spedirsi; ed esprime unas
certa impazzienza di colui, che stimola. E' il Lat. ea tandem. Finiscila,. dille
ana volta,

DAG ELA in fanore. Rispondi secondo il nostro desiderio, Quando si vince
una lite, si dice haner la sentenza in fanore..

CUOKIE con (a ghirlanda. Significa morir vergine. A coloro che muoiono in
concetto di vergini, quando si portano al sepolcro, costumasi di porre in testa
una ghirlanda di fiori in segno della loro castità. Qui il Poeta scherza, come è
solito farsi, quando si discorre d' una donna impudica, che Gdice Elba giurate
di morir con (a ghirlanda, ¢d & detto ironicamence, e per intendere, e//a vual por=
i tare il vanto ye La corona delle donne impudiche, Ma non per queito il Poeta (che
molto ben si ricorda, che Celidora, per essere flaca moglie del Re di Malmanti-
le, non è pi da ghirlanda, intende, che Celidora fofle impudica, ma dice gosh
per ischerzo, e per segu tare il costume della plebe, la quale, quand'uno nomina
sorella, madre, o moglie, suol dire; purtana di me, e simili. Se si parla d'ammogliati
suol dire becco del diavolo, ee. Tal cohtume moitrd il Poeta ancor fo-
praC. 2. stan. 21. dove dicendo: 4 saper quante paia fan tre buoi, foggiugne sfybi-
to Se ben dat padre, ec. e vuole intender padre bue, secondo lo scherzo suddetto:
' Non è pero queito stimato offefa, percht avvien sempre detto per ischerzo; ma
4 ricice bene odiolo, e riaferescevole l'eder.u/aco spesso, ed in ogni congiuntura,
y come è usato fra i pil vili, che lo fanno per parer fagaci, e concettof.
¢ Sl riftringe nelle [paile. Cioè 8' accorda, ed accop/ente a quel, che altri dice,
ib
v

aS Sen

© propone. EB' un' arto solito farsi da quelli, che & rimettono, o aderiscono alla
yoloata d' uno, per non poter fare alttuncnti, o convinti dalle ragioni, o indo
ti dalla necessità, quasi dicano: Pazienza; Bisogna frarct. Bocc. Giorn, 2, nov, 8,
' Ada pure nelle [palte rifiretto casi quella dagiar a daee setae mole Mas /ihoorre AnGa,
yar 2. se















Sate
Eefubetiesaivolen nos si faceia essert “7
volta della testa 5 non dimend dictarho >. re;

0 garbate: O così sta'bene Lat, edge, perphtore belle Te
sue ii contento |, che's' ha», he una cofa  succeda secondo chefi desid
APREST Oye male, te cone dafser Meglio'¢ farimale'y¢ pre!
si mai col pensiero “dis voler far benew Chi fa o|,,emale pfiaalineare'
cha facenuy adagio 5 e bene, mainon conchiude, o-termina'quel'cheha
moidi fare, non si puddin che facciay e veramente nonfa'y e pend nell'c
dei fare e meglio far male, che non fare. '
DATE (a mano Dar ia mado (Latinoviuagere\ dexreras yO la.
nia, che fr faccia negii spontaiizai') e dice impalmare O:far Limpalmamentos,
STANZA Lie oS BAN ZAMLDL oub
Sogwitoical sxb Lvoe gud Phiche bakes,
(olLanSenegee |y. che sn: last frggiafh patra
mand eiskincdrfecon La ging iddaes, y\








ee © al dueilo non volle la gatta; Per eat sa i
quefamnalivnara Medesy) >) sion Lagwale
ieplaeaens ere ot mqe) asBe Pe ep

neater (grades ust & Biche trt/ud honor ae s
ones', aldan pian, Ces ie perdadayy 00% ~ nel e es aero ae we }
ia









STANZA LIL; oralga on nenaaet
Bit won potends bauer Cupide sposoy 08 90 Pereincomintance/m: ets a4 F
hori: Amardai martha tontana >: \\Bacendo com? il'can delParcolano,, ©)
Ou Revael,s elapher (can iia Ucanegdlah iG 9% (O'all! infatara now P
Che pur veduto sia da corpo humane: E non pao ines eae
Martinazza haveddo prdiilto, che dovea esser fatta imoriré, eiche per Gupi
do non dovea esser piirfucsspolo,, inttidiofa, che.questo'be ne havetioa epodie dd
alteiy: !-haveva incancatoun-udga igacto per impedire: yoche'altrinon havefe
\ EFOGIFA ratta., Boggiva velotemente, Ratto viene dal ae eee
verbio Fiorentino;.Cbrva-pianosa ratto, corri(pondente ai Latiaoy
GING ADEA, Intend laspada, come s'intende: conunenene wb al
deta dail'impugnarficé tutte cinque le dicasete bene itbastone pure simpugna coeur hur
te cinque de-dita, non si di¢e-cinquadea y pecché questo fipud im}
digch jal che\non Gi — fare delianspads ordinariayy 0fe pur ff
© con difficulta VS RSH As
wuollagstea, a vuolattendete pNomwuol'badarey
Rissmnneiir quel tal:negozio. Hl Berni nell\Orlando y=
« Chey come si suol dir, voglit la gutta, ~~
OVA Aeden B+ uora lacrudela, che eh Medea si
Oza Re de' Colchi:,»versaril fratello Absyreo- opr )
fo, Glauca sua rivale y¢' co yet th suo ne per 4



"ihe 'Vee Rte Sceey cr

jeicodeind Mamie) mateo; A Gatto ata

goD


Se peeve fats = &

-

ie
6

|



DVODECTMO,ED VLTIMOGANTARE '5432







ne fuan'o\pibes inet faitem a <, be aL a oe
do)0-da qualche donni at iftra ye wih won; sfiss alist ctloe

TIRA per dado. “Conia aplageresrnoraands pil Bettilenels
la milizia, soldati insieme habbiano commefio qualche delitto 'ca-

pitale, farmorice tn di loro'y,¢ falvar.la:vita a tutti gli altrt, facendo loro tiz
rariila-forte ne s€ però 5.1 orcas dettirdadi, e da-credere, che ace
compagnine tal funzione con, fo! i xe con pianti;.¢ stimo però sche il Poetas
digcndo ztiraper-dade y intenda, toipieay © plange pill di cuore che mai; /eguirae
Piangeress pisces gagardamene yes sie pare, she non heaas here aim > 6 sia -
da principio,

“hssan wage. Esser desiderosa d' una tahoe. « Saiwere vago, che vuol dir be
lo, adarne.yec. Sig igiia\ ancora in questo fendi bramefo, ec. Tiraleé — divbes cir
vuol dire: Zi tale genio', ha gusto di betle burle, e feberzi.

HA già fanoil-pianto. Liha già pianto per perduto. Termine assai usato rims:
Gimili congiunture.. Pianto & quellamento, che si fa-sopra il morto,decgo:così.dal
batterti, per.dolore il petto.» Latino planitus, rodalia = voce Lavina:hanne fat-
ta Gmiimente i Pranzefi la loro Peainte, seh ats eh

eA LZ AR capanne, eo, Cioè quei monti di scope 9 ec. chevsaveno fatei per &b-
biuciar Martinazza come fredetto sopra in questo G.t.-3» e queste sonove 2o/e
as Fusco, ie quali dices che sshanno a fare per-hongrdi Jet; \cheper altrovyquan-

do diciamo:: s! banno a fare.cose di ey on st afarcofe, bole; wine.

frofe, e fuori del consueto,

FAR come il cane dell'ortolano y Cioè non volere, o non potere havert una cosa,
ed impedire y che altri l'habbia, come fa il cane dell'orcolano'} che nin,
fuangia-! crbaggio y ¢.non vuole che altet lo. Piglt Canis in Praesepi + Provetbio
nlato da bucianoy

eT AN Ze: bItbbs

“iquid 6 of STANZA LV M
tio, \e Bsiche bebbtrogeuife Cos 'byes: affanni ¥¢ le fatiche 0!) Ob
< WDE extte quello eb' è fegiito'ie Corre; » Soffente per rant' anni, e lafri ee

~Gbda il teiogoappinte now si farprecifoy

Risrovatofi, Amore; ed egli, e Priche

AtRena si fainsaprer'tattele porte; » Rappattumato fu dai cavalieri 30
se Amanro crofeiar fenve/i wrgran rife, >: Onde foordats deli"ingiurie amicbe 5 a
atest \obie peg gio; poi fwonsr; ma forte Eriuniti più che volentieri:
& Aeafivmare.di-pefe sr aboccanti 5 + vad regp sposi fero i bactabaffi yoo 9
i obSemea sunosceriehi rece, Contantics' bq Restando # parte diver foe je (pai
STANZA LV. STANZA shVTR cor 6
“Gir per peniescate ognnn presto addirizza y (i Gluntis cialdeni pots e fare i bile,
Che dal timor gli # arricciane è peli. Ml Duca diede al fin 2 ultimo Addie' -
Ma C alagrillo aitiero, ¢-pien di fiicxa “E Jubiro conagni suo vassallo
ib oGem talus frrifeia fa colps cradels 5 =: dnnerfa Venano Spiele it-pendio
wi Wa per le stance fende,taglia.e infizza, E Catagrilio ix groppa al-fud <aualld
jlut 444 mon cliappa, se vende' logences cil' © \Preforoon Pficle it Raretrare Dia,)

~ien Rar tde inns e i¢ok fucertibroinranty y.
E il Diavol cacciaye manda vialincato,

o8e\Gupido per opra dij Pakide Aixicrova je per inekzo di-quei Cavalieri'

“3uVO

'¢Aashrei pars); eintefolildor 20
Gb ricomduffe 'ali' Amoroso. as






344
con Psiche, si fanno le fefte delio spolalizio di
lo di loa Bache con iain beioenta ~ dy,
lo accompagna Psiche se Regno d',
€ROSCLAR an ria. Rider gagliatdamenre- Vedi
GRAV, traboce: Gravi:pil del giufto pelo,
on delle nee ee on se ne feeuc per
¢ seguita 5 chi recé contanti ( che termine proprio
= intender, chi dava se heeds '
e4ADDIKIZZ 4Ciok va via. Fugge per la più hota temas
STKISCLA; Intendila spada, come intese (apra:C. 2: st, 60. ° ae
CHIAPPA, Coglig s ritrova, perquote s¢aipilce. Vedi sopra C7.
RAGNATELI, Ragni,piccolt vermis o inferti nati... Vedi sopra.
Le stanze piens di ragnateli significa vote dogui.altea fa. Siauimente |
yolendo dire il borficchio voto, dite; Plexys facculius est arancaram,
RAP PATTV MATL. \ocendiamo rappacificati. Da molti si dice
ge di pace donde: O vincere, o patrare, clo' pareggiare; far pace» ae
gredo venga questo verbo rappatramare, il quale e atlai usato, mala) 4
da pochi fuori della plebe.::

CIALDONI, Specie di pasta confetta, contorta sottile come l'ostie, ed attorta,
e ridotta come un grosso cannello di canna

STANZA LVIILED VETIMA.,
Finito è il nostro scherzo: hor facciam festa,
Perché la Storia mia non va più avanti,
Sicché da fare adesso altro non resta,
Se non ch'io riverisca gli ascoltanti.
Ond'io perciò cavandomi di testa
Mi v'inchino, e ringrazio tutti quanti;
Stretta la foglia sia, larga la via:
Dite la vostra, ch'i' ho detto la mia.

SCHERZO, Qui vale per trattenimento, Latino lusus, Sogliono i nostri
Contadini, quando fanno le loro veglie di ballo, dopo che hanno un pezzo ballato;
introdurre qualche intermedio, rappresentazione, o giocolamento di forze, o
altro, q questo chiamano lo scherzo, che per lo più finisce in burlar qualche
semplice, e dar'occasione di ridere, e questo tale è poi anche detto lo scherzo, e così
l'intendiamo comunemente, ed il nostro Poeta molto bea l'esprime servendosene
nella sua lettera alla Sereniss. Arciduchessa Claudia d'Austria, riportata sopra
nel Proemio, dicendo: Contentandomi io, che la mia Leggenda, come nata da scherzo,
mi faccia scherzo alle genti.

FATE festa, Cioè siate licenziati, Vedi sopra C. 10, st. 42.


Nota, amorevole Lettore, che il Poeta per terminare la presente sua Oera,
ringraziando con questa ultima Ottava gli uditori, si serve della chiusa inventata,
ed usata dalle donnicciuole, quand' hanno raccontata una novella; cioè
Stretta la foglia sia, larga la via;
Dite la Vostra, ch'io ho detto la mia,

E conchiude, che ha contata una Novella, come diede intenzione sul principio
di quest'Opera. Ed io pure me ne servo per incitare altri a dir qualcosa
meglio di quello, che habbia fatt'io, non so s'io mi dica nel dichiarare, o pure
confondere, ed intrigare quello che nella presente Opera ho stimato poco
intelligibile fuori della Città di Firenze, e prego il discreto Lettore a compatir
me, che per ubbidire ho pigliato a far' un volo superiore alle mie forze, ed a
contentarsi di biasimar me solo, e non quei, che mi comando, perché habbia
fatto errore nell'elezione. E fo punto.

FINE DEL XILEDVLTIMOCANTARE.
















è; MEE LY Big

I Molto Rev, Sig Gio; Domenico
cia di riconoleeté con ogni di
. Opera fouo il Titolo di Malmant
Zipolt, vi sia cov alecuna, che ¢
~ Cattolica, eda' buoni 'Coftumi 4
» Maggio 1686.; =e aoe '

Niccolo Castellam Vic. Cen. Fiorent, dam Ry ia

si
Mluftrifs. e Rev. Sig, g
Ho attentamente letto Oe cor Operetta al
le Racqusftato di Perlone Zipoli, insieme con le fae note
spiegazioni, e per non avervi trovato cosa, ne
alla Santa Fede Cattolica,ed a' buoni costumi,
mano mi soscrivo. Firenze 20, Settem. 1686,
Gro. Domenico del Bruno en Sac, Ti

Attesa la sopraddetta relazione si stampi > osservati gli ordini
soliti, Data z0.Settemb, 1686. > Z;
Niccold Casteliani Vic. Ge

I Molto Rev. Padre Lettore Dolci Minor Otfrvante Conf i
tore del Sant'Vfizio di Firenze legga attentamente la
sente Opera di, Perlone Zipoli,. intitolata Malmantile
quiftato, e ritrovandovi cosa repugnante alla Sat
Cattolica, e buoni costumi, riferisca, Dal 9, Vfizio.
renze 17, Ottobre 1686.::
Fr, Francesco Agostino Gambaroua Min,
Del S, Vyizso.

Reverendifs. Padre,:
Ho rivista, e ben considerata l'Opera intitolata A
le di Perlone Zipoli, e per non esslervi cosa repu
-aggiunte, stimo possa
D'Ogni Santi li 24. Febbr. 1686, Pe
Fr, Bragio Dolei Ain, Offer. Conf, del S. j

Attenta prefata relazione.
Imprimatur;

Fr. Ces. Pallarvicinus Ordimis Min, Convent, Vie, Ge
S. Off. Florentia.

Ruberto Pandolfini Senat. e Aud. di S, A. S.




Stel 3% oy rm i



BT 2Mh ood ww

LocalWords:  havrei Sinigaglia habbia habbiamo crazia crazie donnicciuole
% LocalWords:  Celidora Bertinella soggiugne Conchiude giuoco giuochi
% LocalWords:  Malmantile

Tesorero municipal
Licenciado José Gabriel Castillo
nos mudamos de donde estamos recibiendo el servicio,
ya no vamos a hacer uso del servicio
