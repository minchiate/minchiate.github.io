\documentclass[12pt,a5paper]{book}
\usepackage[utf8]{inputenc}
\usepackage[T1]{fontenc}
\usepackage[italian]{babel}
\usepackage{changepage}

\usepackage{comment} % serve per compilare un cantare per volta.
\usepackage{calc}% http://ctan.org/pkg/calc

\usepackage{etoolbox}
\apptocmd{\thebibliography}{\setlength{\itemsep}{-2pt}}{}{}

\usepackage{tikz}
\usetikzlibrary{decorations.shapes,shapes.geometric}

% avoid orphans and widows, allow for (a lot of) letter spacing.
\usepackage[defaultlines=2,all]{nowidow}
\usepackage[tracking]{microtype}
\sloppy

% how to format and space chapter titles
\usepackage{titlesec}
\titleformat{\chapter}[display]
            {\huge\bfseries\scshape}
            {\vspace{-1.5em}}
            {0pt}
            {}
\titleformat{\section}[display]
            {\large\bfseries}
            {\vspace{-1.5em}}
            {0pt}
            {}
            [\vspace{-12pt}]
\titleformat{\subsection}[display]
            {\normalfont\fontsize{12}{14}}
            {\vspace{-1em}}
            {0pt}
            {\centering}
% I like this font!
\usepackage{tgbonum}
\renewcommand{\rmdefault}{qbk}
\usepackage{lettrine}
\usepackage[left=13mm,top=11mm,right=13mm,bottom=14mm]{geometry}

\makeatletter
\renewcommand{\@makefntext}[1]{%
  \setlength{\parindent}{0pt}%
  \begin{list}{}{\setlength{\labelwidth}{18pt}%
    \setlength{\leftmargin}{\labelwidth}%
    \setlength{\labelsep}{2pt}%
    \setlength{\itemsep}{0pt}%
    \setlength{\parsep}{0pt}%
    \setlength{\topsep}{0pt}%
    \footnotesize}%
  \item[\@thefnmark\hfil]{#1}% @makefnmark
  \end{list}%
}
\makeatother

\renewcommand{\negthinspace}{\hspace{-3pt}}

\newcommand*\sepline{%
  \kern 3pt \hrule \kern 2pt
}

\title{%
  \kern -2em\fontshape{sc}\normalsize
\textls[180]{\Huge MALMANTILE}\\
\textls[360]{\normalsize RACQVISTATO.}\\\kern 8pt
{\LARGE POEMA}\\
\textls[240]{\Large DI PERLONE ZIPOLI}\\
{\small CON LE NOTE DI PVCCIO LAMONI.}
}

\author{%
{\normalsize DEDICATO}\\
\textls[220]{ALLA GLORIOSA MEMORIA}\\
\textls[-20]{\footnotesize Del Sereniss. e Reverendiss. sig.\ Principe Card.}\\
\textls[320]{\textsc{\Huge LEOPOLDO}}\\
\textsc{\LARGE de' medici}\\
\textsc{e}\\
\textsc{risegnato alla protezione}\\
\textsc{del}\\
\textls[-20]{\footnotesize Sereniss. e Reverendiss. Sig Principe Card.}\\
\textls[40]{\textsc{\Huge FRANC. MARIA}}\\
\textsc{\LARGE nipote di s.a.r.}
}

\date{%
\vfill\scriptsize
\textls[120]{\small\scshape In Firenze}\\
\sepline{}
\textls[-30]{\scriptsize Nella Stamperia di S.A.S.\ alla Condotta.\ 1688.\ \textit{Con lic.\ de Super.}}\\
\textls[120]{E PRIVILEGIO}\\
\textls[60]{Ad istanza di Niccolò Taglini.}
}

\newcommand{\flagverse}[1]{\vspace{-4pt}\hspace{-8pt}\makebox[26pt]{\fontshape{sc}\footnotesize \hfill{}#1\hspace{4pt}}}

\newenvironment{signature}{

\kern 1em

\hfill \begin{minipage}{4.5cm}\centering}
{\end{minipage}}

\newenvironment{poesia}{%
  \kern -2em
  \setlength{\parindent}{-1em}%
  \setlength{\parskip}{8pt}%
  \begin{adjustwidth}{5em}{}\-

  }
               {\end{adjustwidth}}

\usepackage{xifthen}
\newcommand{\backspace}{\-\kern -1.5em}
\newcommand{\verseprefix}[1]{\hspace{-5.5em}\makebox[5.25em]{\rm #1\hfill}\hspace{0.25em}}
\newcommand{\pfix}[1]{\hspace{-2.5em}\makebox[2.25em]{\rm #1\hfill}\hspace{0.25em}}
\newcommand{\items}[1]{ \- \\ \textit{\textbf{#1}} \makebox[2pt]{}}
\newcommand{\letter}[1]{\textls[-70]{\textsc{\lq#1\rq}}}
\newcommand{\cst}[2][]{\ifthenelse{\isempty{#1}}{questo C.~}{C.~#1.\ }st.~#2}
\newcommand{\canst}[2][]{\ifthenelse{\isempty{#1}}{questo Can.~}{Can.~#1.\ }st.~#2}
\newcommand{\cantst}[2][]{\ifthenelse{\isempty{#1}}{questo Cant.~}{Cant.~#1.\ }st.~#2}
\newcommand{\cstan}[2][]{\ifthenelse{\isempty{#1}}{questo C.~}{C.~#1.\ }stan.~#2}
\newcommand{\libcap}[2][]{lib.~#1.\ cap.~#2}

\renewenvironment{verse}{%
 \itshape\setlength{\parindent}{5.5em}
  \setlength{\parskip}{0pt}
  \obeylines}
               {}
\newenvironment{ottave}{%
  \noindent\hspace{36pt}\begin{minipage}{10cm}\vspace{4pt}
  \setlength{\parindent}{-18pt}
  \setlength{\parskip}{6pt}
}
               {\end{minipage}\\}

\newenvironment{argomento}{%
  \section*{\textsc{Argomento}}\hspace{36pt}\begin{minipage}{10cm}
  \setlength{\parindent}{0pt}
  \setlength{\parskip}{-2pt}
  \obeylines}
               {\end{minipage}\vspace{10pt}}

\newcommand{\ellipsis}[1]{\makebox[#1]{\rule{#1-2pt}{0.5pt}}}
\newcommand{\culo}{c\ellipsis{18pt}}

\newcommand{\stanzadash}{\rule[2pt]{54pt}{1pt}}
\newcommand{\markstanzablock}[1]{\item[\stanzadash] \textbf{#1} \stanzadash}
\newcommand{\makestanzalabel}[1]{\textit{\textbf{#1}} \-}
\renewenvironment{description}
                 {\begin{list}{}{%
                       \setlength{\labelsep}{1em}
                       \setlength{\labelwidth}{0pt}
                       \setlength{\topsep}{0pt}
                       \setlength{\parsep}{4pt}
                       \setlength{\parskip}{0pt}
                       \setlength{\itemsep}{-2pt}
                       \setlength{\leftmargin}{1em}
                       \setlength{\itemindent}{0pt}
                       \let\makelabel=\makestanzalabel}}
                 {\end{list}\setlength{\leftmargin}{0pt}}

\begin{document}
\raggedbottom

\pagenumbering{gobble}
\maketitle
\pagenumbering{roman}

\-

\vspace{6em}

{\centering\Large
{\footnotesize\textsc{al sereniss., e rev.\ sig.\ il sig.\ principe card.}}\\
\textls[44]{\huge FRANCESCO MARIA}\\
\textls[100]{\LARGE DE' MEDICI.}\\
\kern 2em}
Il
Sereniss. e Reverendiss. Principe Cardinale
Leopoldo de' Medici Zio di V.A.R.\ Principe
di quelle rare, ed ammirabili qualità,
che hanno fatto stupire tutto il Mondo,
fino da i più teneri anni dell'A.V.R.\
conobbe, che in lei dovea continuare quello
splendore, che hanno accresciuto alla
sua Sereniss. Casa le stimabili doti di
V.A.R; E per questo, siccome giudicò, che l'A.V.R.\ gli
dovesse succedere nelle virtù, e nella dignità, così volle, che
ella fusse anche erede della sua singolar Libreria. In questa,
havea l'A.S.Rev.\ destinato, che dovesse ottenere il luogo la
presente Opera di Perlone Zipoli, a cui S, A. R, m'onorò
comandarmi, ch'io facessi alcune note, grazia compartitami
(siami lecito il dirlo) forse con qualche scapito del
prudentissimo giudizio di S.A.R.; Ed havendo io ubbidito nella
miglior forma, che havevo saputo, già si pensava alla stampa,
quando i Fati invidiosi tentarono di privarla di così pregiato
onore: e sarebbe loro riuscito, se la somma prudenza di
quel gloriosissimo Principe non havesse a i medesimi impedito
il corso, con prepararle il rimedio nel rifugio alla protezione
di V.A.R.

Se ne vien però il povero Malmantile a' piedi di V.A.R.
umilmente supplicando la sua benignità a volersi degnare di
riceverlo nella sua grazia, e, come erede obbligato;
riverentemente convenendola al Tribunale della sua generosità,
perché gli faccia godere la giustizia, concedendogli il luogo
stabilitogli, acciò egli possa dirsi veramente rifatto dalle
rovine cagionategli da tante sue disgrazie, e da tanti suoi
sinistri avvenimenti: Ed io piglio l'ardire d'accompagnare
queste preci, che egli porge a V.A.R., come quello, che
conosco d'haverlo con la mia penna costituito in grado d'haver
maggiormente bisogno dell'autorevol patrocinio di V.A.Rev.\
alla quale intanto umilissimamente inchinato bacio
ossequiosissimamente la Sacra Porpora.

Di V.A.Rev.

\begin{signature}
Vmilissimo Servidore\\
Puccio Lamoni
\end{signature}

\clearpage
{\centering\Large
{\footnotesize\textit{Al Sereniss.\ Rev.\ Sig.\ il Sig.\ Principe Cardinale}}\\
\textls[44]{\LARGE LEOPOLDO DE' MEDICI}\\
       {\large PADRONE CLEMENTISSIMO.}\\
       {\normalsize PVCCIO LAMONI.}\\
       \kern 2em}

SERENISS. E REVERENDISS. SIG.

MENTRE stavo meditando d'ubbidire a i cenni stimatissimi
di V.A.Rev.\ col far le Note alla presente Leggenda di
Perlone Zipoli, mi cadde sotto l'occhio un sonetto del
Burchiello\footnote{Domenico di Giovanni, meglio noto come il Burchiello (Firenze, 1404 – Roma, 1449).}, nel quale havendo osservato, dove dice:
 Non sunt, non sunt pisces pro Lombardis,
mi saltò il ticchio d'esser' il Lupo nella favola, cioè che questo verso
m'avvertisse, che la faccenda da V.A.Rev.\ impostami non fusse
carne da' miei denti, ond'io havevo già quasi pensato di far conto, che
passasse l'Imperadore: Ma considerando poi, che farebbe stato errore in
gramatica, e da pigliar con le molle, il far'orecchie di mercante a i
riveritissimi comandamenti di V.A.R.\ ho risoluto di non metterla più in
musica, o in sul liuto, ne mandarla d'oggi in domani, dando erba
trastulla, e menando il can per l'aia, ma (venendo a dirittura a i ferri)
non tener più questo cocomero in corpo, e così cavarne cappa, o mantello
più per eseguire gli ordini di chi può comandare a bacchetta, che
perché io resti persuaso d'haver forze sufficienti a portar sí grave soma;
E quantunque io sappia, che havrei fatto molto meglio a lasciar la lingua
al beccaio, perché così havrei sfuggito il farmi dar la quadra, o la
madre d'Orlando, e sonar dietro le padelle da coloro, che si pigliano
gl'impacci del Russo, e ficcando il naso per tutto, fanno poi le Scalee
di S. Ambrogio, come quelli, che havendo mangiato noci, apporrebbono
al sale, senza considerare che ognun può fare della sua pasta
gnocchi, e che [come disse colui, che s'impiccò] ognuno ha i suoi
capricci; tuttavia ho voluto (legando l'asino dov'è piaciuto al padrone)
dare a conoscere che V.A.R.\ non farà, come il Podestà di Sinigaglia;
Se poi ad alcune di questi tali rincresce, mettasi a sedere, e, se non gli
piace, la sputi o mi rincari il fitto; e se dirà, che in fare alla presente
Opera le Note comandatemi, io non habbia preso il panno pel verso,
ma più tosto fatti de' marroni, e pigliato de' granchi a secco, lo lascerò
ragliare; perché son sicuro, che non mi farà baciare il chiavistello, ne
Pigliare il puleggio dalla casa mia; ne mi può accusare di delitto da
farmi mettere in Domo Petri fra i due Apostoli, o da farmi meritare d' esser'
ammazzato con una lancia da pazzo; E se l'indiscretezza di questi tali
mi condannerà per gli errori, che troveranno nelle Note fatte da me, la
mia ignoranza m'assolverà. Non ne ho saputa più: ho soddisfatto al
debito d'ubbidire, e mi quieto col detto di Donatello: Piglia un legno,
e fann'un tu. Mi fara forse detto: Tu porti frasconi a Vallombrosa,
cavoli a Legnaia, ed acqua in mare, e vai contrappelo alla buona
strada a comparire avanti a un Principe così erudito con questi tuoi
scritti; ed io a lettere d'appigionasi, e di scatola, senza saltare in sulla
bica, o entrar nel gabbione, rispondo a costoro, i quali fanno tanto il
Cecco suda, che portano ben loro le mosche in Puglia, e i Coccodrilli
in Egitto, e dandomi il mio resto, hanno trovato il modo d'intisichire,
senza però dirmi cosa, che io non sappia; perché conosco-ancor io il
pane da sassi, la Treggea dalla gragnuola, e le cornacchie dalle cicale; e
sapendo quanto il mio cavallo può correre, sarei venuto di male
gambe, e quasi come la serpe all'incanto, a metter questo cembolo in
colombaia; se non mi fusse noto, che colui, che è avvezzo a mangiar
sempre starne, desidera talora carne di Storno, e non fussi certo, che
la somma prudenza di V. A. R, (conoscendo, che il pruno non produce
limoni, e che dalla botte non esce mai, se non di quello che v'è
dentro, che parimente è impossibile, che il Gufo faccia il verso del
Rusignuolo) non è per isdegnare di ricevere le baie di Perlone Zipoli con
l'abito da villa messo loro in dosso dalla mia zucca, poco atta a
rappresentar l'impresa degli Accademici Intronanti, perché le manca il
Meliora Latent.

Supplico però l'impareggiabile umanità di V.A.R. a voler restar
servita di far conoscere a questi tali, che io ho legato il Cavallo a
buona caviglia, con fare degne queste mie insipidezze d'un benigno suo
sguardo; non perché lo meritino per se stesse, ma perché bensì conviene
alla continuazione di quel generoso aggradimento, col quale si compiacque
ricevere in vita dell'Autore il medesimo Malmantile. Il quale
se con le mie ciarle haverà fortuna di comparire in pubblico, godendo
sí pregiato favore, si potrà dire, nato vestito, ed io cascherò in piè
come i gatti, e mi pioverà il cacio in su i maccheroni: E così con
haver'immitato il cane di Butrione, non havrò timore di coloro, che passano
per la maggiore; perché sapendo essi, che l'Aquile non fanno guerra co'
Ranocchi, sdegneranno abbassarsi tanto con la loro critica, mettendo le
mani in si vil pasta; e quegli Aristarchi, i quali non contano, e non
hanno voce in capitolo, per haver poco di quel che il bue ha troppo, e
che sono come monete stronzate, o come i cavalli di regno; non saranno
causa, che io alzi i mazzi; ne mi faranno venire la muffa, o il moscherino
col loro gracchiare; perché oltre all'essere scritto pe' boccali, che il
Cieco non può giudicare de' colori, si sa ancora, che raglio d'asino
non entrò mai in Cielo, che però conoscend'io, che essi son per fare,
Come colui, che tosa il porco, non gli stimo il cavolo a merenda, e gli
ho dove si da al bossolo da spezzie, e dove si soffiano le noci; Sicché si
possono andar' a riporre a lor posta, e fare un mazzo de' loro salci.  E se
bene dice il proverbio, che la carne di Lodola va a Piacenza a ognuno;
io non mi curo, che me ne sia data, anzi per non mangiarne, son
contento far sempre di nero, purché non mi dieno di bianco questi Correttori
delle stampe, che tiranneggiando le lettere, perché si stimano il
Secento, cercano i fichi in vetta, e 'l nodo in sul giunco. Ma se poi mi
vorranno pure strazziare, io gli assicuro, che e' non hanno a mangiare il
cavolo co' ciechi, quantunque io non sia tanto addietro con l'usanza,
che io voglia mai far credere a haver cattivi vicini, o sia di natura
d'ungermi gli stivali a mia posta. Mi mandino, pure: all'Vccellatoio
quanto a lor piace, e mi facciano anche dietro lima lima, non faranno
però causa, che io faccia come Chele Masi, perché me la farebbono di
figura, e mi scotterrebbe troppo; se bene mi persuado, che ancor'essi
non fussero per uscirne netti; e che fusse per succeder loro il mangiar
noci col mallo, e far come i Pifferi di montagna, poiché, se essi si stimano
piccioni di Gorgona, ed io non son di Valdistrulla; perché sono uscito
di dentini ed ho rasciutto il bellico, e per questo so ancor'io quante
paia fanno tre buoi; onde a dirmi cattivo cattivo, la farà fra Baiante, e
Ferrante, perché io son d'una natura, che non posso ber grosso, e mi so
levar le mosche d'intorno al naso, ne mi morse mai cane, che io non
volessi del suo pelo, massimamente quando m'è saltato il capriccio di
voler la gatta, e badare a bottega, giuocando per la pentola; e s'io me
la son mai legate al dito, o l'ho presa co' denti, n'ho voluto vedere
quanto la canna; perché non mi suol morire la lingua in bocca, ed ho
tagliato lo scilinguagnolo, ne m'è piaciuto mai portar barbazzale, e so
lasciar la squola d'Arpocrate, quando è tempo, ed in particolare con
quei tali che, son più tondi dell'O di Giotto, e che stimando una stessa
cosa il chiacchierare, che il condennare, non sanno portare altre ragioni,
che quel maladetto \textit{non si può}.

Ma perché non paia ch'io saltando di palo in frasca voglia dar panzane
a V.A.R.\ e che questa mia lettera sia il vicolo di mona Sandra, conchiudo,
tornando a bomba, che stimerò d'haver toccato il Ciel col dito,
e tirato diciotto con tre dadi, se potrò conoscere, che l'A.V.R.\ resti
servita di credere, che in questa parte io l'habbia: ubbidita giusta mia
possa, come riverentemente la supplico a degnarsi di far apparire con l'onore
di nuovi suoi comandamenti. Mentre facendo la festa di S. Gimignano
umilissimamente inchinato bacio ossequiosissimamente a V.A.R.\
la Sacra Porpora.

\clearpage
\noindent\textsc{\centering
\textls[180]{\large al cvrioso e discreto lettore}\\
{\large pvccio lamoni.}\\
\kern 0.5em}

La presente Opera di Perlone Zipoli si manda alle stampe, per soddisfare
alla curiosità di molti, che bramosi di pigliarsi il passatempo di leggerla
ne hanno fatta instanza. E perché in alcuni detti, e proverbi usati
in Firenze, de' quali si serve il nostro Autore, possa esser' intesa anche da
color, che lontani dalla nostra Toscana, non hanno la vera cognizione del valore,
e senso di essi, vi ho aggiunto alcune note, con le quali se non ho appieno
soddisfatto, mi basta, che havrò forse data occasione col mio cicalare, che
venga ad altri voglia di meglio discorrere. Tu intanto ricordati, che questa è
una novella; e così ti accomoderai a compatire, se alle volte mi son fatto
lecito di dare qualche spiegazione favolosa. So, che havrai la bontà di sbandir la
censura, e ti tornerà commodo, perché facendo altrimenti havresti troppo da
fare, poche, o forse niuna essendo di quelle cose, che ho scritto, che non la
meritino con un nuovo foglio, e per questo non te ne prego: ti prego bene, se sei
Fiorentino, a legger' il Testo, e non le Note, perché queste non son fatte per te,
che, meglio di quel ch'io habbia scritto, intendi la forza de i detti, che ho
preteso dichiarare,

Dovrei notare gli Autori, a i quali son ricorso per tirare a fine la presente
fatica, ma perché gli bo nominati in tutti quei luoghi, dove è convenuto valermi
della loro autorità, tralascio di farlo; non voglio già tralasciare di confessar
l'obbligo, che queste mie Note, ed io habbiamo all'Eccell.\ e dottissimo Sig.\
Gio.\ Cosimo Villifranchi, ed agli Eruditiss.\ SS.\ Anton Casto, e Sig.\ Francesco
Maria Bellini, i quali m'hanno onorato di più erudite notizie; ed in ultima
attestar la fortuna che hanno havuto questi miei scritti di passar sotto l'occhio
dell'Ecc.\ Sig.\ Abate Anton Maria Salvini\footnote{Anton Maria Salvini, Firenze 1653 - ivi 1729. Grecista, con Antonio Maria Biscioni, 1674-1756 figura sulla copertina delle edizioni 1731 e 1750 del Malmantile.} il quale non solamente s'è contentato
d'emendar molti miei errori, ma d'ingagliardire ancora le mie debolezze con non
poche sue bellissime erudizioni, a segno, che ha fatto nascere in me una speranza,
che sia per esser ricevuta volentieri questa mia Opera, e d'haver guadagnato
non poco appresso al Mondo letterato, per haver dato occasione a questo dottissimo
huomo d'esercitare la sua stimabilissima penna, i tratti della quale, come
non ho dubbio che nobilmente risplenderanno dentro all'oscurità della mia, così
son certo, che saranno da tutti benissimo ravvisati: Ne confesso però al
medesimo il mio debito, e ne porto al pubblico questa attestazione, perché si sappia
che quello, che sarà riconosciuto per non mio, non è latrocinio, ma regalo
fattomi da questo, e da altri huomini dotti per loro generosità, e per sollevar
Perlone dal discredito, che haveriano fatto meritare a questa sua Opera i miei scritti.\\
Lettore, vivi felice.

\clearpage

{\centering\Large
\textls[244]{\LARGE PROEMIO.}\\
\kern 1em}

Lorenzo Lippi\footnote{Lorenzo Lippi, Firenze 1606 - ivi 1665, pittore. ``Perlone Zipoli'', poeta, scrittore.} (che in Anagramma nella presente Opera si chiama Perlone
Zipoli ) è stato ne i tempi nostri Pittore non poco celebre, come testificano
molte, e molte sue fatiche. Ciò lo fece meritare d' esser chiamato dalla
Sereniss. Arciduchessa Claudia d'Austria\footnote{Claudia de' Medici, Firenze 1604 - Innsbruck 1648. Reggente del Tirolo dalla morte del secondo marito Leopoldo d'Asburgo nel 1632 alla maggiore età del figlio Ferdinando Carlo nel 1646.} per valersi dell'opera sua a Inspruk,
dove dette principio a questa da lui chiamata Leggenda delle due Regine di
Malmantile, e la dedicò alla medesima Sereniss.\ Arciduchessa Claudia. Haveva però
l'Autore concepita nell'animo suo quest'Opera qualche anno prima, e nel
tempo, che essendo in Villa de' SS, Parigi a S. Romolo nell'andar per quelle campagne
a diporto, vedde le muraglie di Malmantile; ed haveva discorso questo
suo pensiero col sig.\ Filippo Baldinucci\footnote{Filippo Baldinucci, Firenze 1624 - ivi 1696. Storico dell'arte, politico e pittore, ``Baldino Filippucci''.}, dal quale poi nel tessimento del Poema
hebbe, come da persona erudita ( che tale lo dichiara la sua bell'Opera mandata
da esso alla luce intitolata Notizie de i Professori del disegno) non piccolo aiuto
in proposito della lingua, e d'altro, e particolarmente nei descrivere il Consiglio
de i Diavoli nel Canto sesto.

Tal composizione fece egli a solo fine di mettere in rima alcune novelle, le
quali dalle donnicciuole sono per divertimento raccontate a i bambini, e di sfogare
la sua bizzarra fantasia, inserendovi una gran quantità di nostri proverbi, ed
una mano di detti, e Fiorentinismi più usati ne i discorsi famigliari, sforzandosi di
parlare, se non al tutto Bocaccevole, almeno in quella maniera, che si costuma
oggi in Firenze dalle persone Civili, ed ha sfuggito per quanto ha potuto quelle
parole rancide, alle quali vanno incontro tal'uni, che per spacciarsi huomini
letterati, non sanno fare un discorso, se non vi mettono, guari, chente, e simili
parole, che per essere state usate dal Boccaccio\footnote{Giovanni Boccaccio, Certaldo 1313 - ivi 1375.}, essi credono, che dieno l'intero
condimento alli loro insipidi ragionamenti, e stimano, che quello sia il vero parlar
Fiorentino, che non è inteso, se non da i lor pari, e non s'accorgono, che
in tal guisa parlando, si rendono scherzo di chiunque gli sente, come bene attesta
questa verità il Lasca\footnote{Anton Francesco Grazzini detto il Lasca, Firenze 1505 - ivi 1584} in quel suo Sonetto sopra l'Opere del Berni\footnote{Francesco Berni, Lamporecchio 1497 - Firenze 1535. ``che dice le cose sue semplicemente, e non affetta il favellar toscano''.}, dicendo:
\begin{verse}
\backspace Non offende gli orecchi della gente
Con le lascivie del parlar Toscano,
Vaquanco, guari, mai sempre, e sovente
\end{verse}
Ed Antonio Abbati\footnote{Antonio Abati, Gubbio inizio secolo XVII - Senigallia 1667} dice
\begin{verse}
\backspace Peggio non ho, che quel sentir parlare
Con tanti quinci,e quindi, e, ec.
\end{verse}
Anzi in questa parte l'unica intenzione del nostro Poeta è stata di far conoscere
la facilità, e pienezza del parlar nostro, e \textit{Cogliendo della lingua materna il più
bel fiore}, mostrare, che ancora ad uno, che non ha (come appunto, era egli)
altra eloquenza, o poca più di quella, che gli dettò la natura, non è impossibile
il parlar bene. Questo, ed altri fini dell'Autore s'argumentano dalla seguente
Dedicatoria, che egli stesso scrisse alla Sereniss.\ Arciduchessa Claudia, la quale
lettera io pongo qui per confonder coloro, che pur vorrebbono fargli dire quel
che mai il nostro Poeta hebbe in pensiero.

\begin{adjustwidth}{1.5em}{}
  \itshape
Ati figliolo di Creso Re di Libia (se è vero, che io non ne so più la, e la vendo,
come io l'ho compra) vedendo il padre in pericolo, isso fatto cavò fuora
il limbello, e disse le sue sillabe, come un Tullio; Tutto il rovescio dovrebbe
fare il pesce pastinaca senza capo, e senza coda della mia Leggenda a mal tempo,
ch'io mando a V.A.S.\ perché vedendo ella quel dolce intingolo di quel
fantoccio di suo padre in procinto d'esser mandato all'Vccellatoio, e quasi ridotto
alla porta co' saffi, e che gli sien suonate dietro le padelle, anzi fra il
tocca, e non tocca di scior Pallino, potrebbe a sua posta far' un mizzo de' suoi
salci, e farsi ricucire la bocca per non haver più occasione di formar verbo.

Ma perché si compiace V.A.S.\ di volerne una secchiatina, benché questa mia
Leggenda non fusse degna di fiutare eziam i luoghi privati, verrà di gala col suo
ricadioso cicaleccio, che si strascica dietro una gerla di farfalloni, a farne una
stampita anche ne i Palazzi reali, perché ella è una prosontuosina da darle del
Voi; Ond'io conoscendo nella temerità di essa l'ubbidienza dovuta de iure a i
riveriti suoi cenni, gli è giuoco forza, voglia il mondo, o no, che ella si metta
giù a bottega a sfogare la fisima de' suoi fantastichi ghiribizzi, contentandomi
io, che ella, come nata da scherzo, mi faccia scherzo alle genti. Compatisca
dunque l'A.V.S.\ questa sconciatura partorita nel tempo, che io do
festa a i pennelli, mentr'ella non apprezzando un'ette gli applausi volgari, riceverà
per grazia sterminata, e per arcisbardellatissimo favore, se queste baie
riusciranno di qualche valezzo nel cospetto di V.A.S.\ alla quale profondamente
inchinandomi, con ogni debita rivereaza bacio la Veste.
\end{adjustwidth}

Da questa lettera adunque si viene in non piccola cognizione de i sentimenti
dell'Autore nel comporre la presente Opera; La quale fu da esso presso che
terminata in Inspruch, e dedicata come ho detto alla Sereniss.\ Arciduchessa
Claudia; Ma essendo S.A.S.\ in quei medesimi tempi passata all'altra vita,
convenne all'Autore tornare alla Patria, dove fu questa sua Novella veduta da diversi
amici suoi, fra i quali dal sig.\ Romolo Bertini Servidore del Sereniss Principe
Cardinale Leopoldo de' Medici\footnote{Leopoldo de' Medici, Firenze 1617 - ivi 1675, cardinale dal 1668.}, e molto accetto per l'ottime sue qualità, virtù,
e dottrina, e da esso hebbe S.A.R.\ la prima notizia della presente Opera, e fino
da allora mostrò l'A.S.R.\ non piccola inclinazione, che si pubblicasse, e se
tralasciò di comandarne la stampa, fu, perché sentì dal medesimo Bertini, che
l'Autore pensava d'accrescerla.

Fu veduta ancora dal sig.\ Francesco Rovai\footnote{Francesco Rovai, 1605-1647. ``Franco Vicerosa''}, e dal sig.\ Antonio Malatesti\footnote{Antonio Malatesti, Firenze 1610 - ivi 1672. ``Amostante Latoni''.};
ambi Poeti nel lor genere Eccellentitfimi, dal sig.\ Salvador Rosa\footnote{Salvator Rosa, Napoli 1615 - Roma 1673. ``Salvo Rosata''} non men celebre
nella Poesia, che nella pittura, e dal quale il Lippi hebbe notizia Dello Cunto
de li Cunti\footnote{Pubblicato da Adriana Basile fra gli anni 1634-1636.} di Gianalesio Abbattutis\footnote{Giovan Battista Basile, Giugliano di Napoli 1566 - ivi 1632.}, di dove l'Autore cavò poi alcune novelle,
che si trovano in quest'Opera: La quale in somma fu veduta da molt'altri eruditi
ingegni; e fu il Lippi da essi consigliato, e poco meno, che forzato a metterla
alla stampa, con persuaderlo, che meritava la pubblicazione: ma ricusò egli
sempre di far tal passo, conoscendo molto bene, che colui, che stampa l'Opere
sue, s'espone ad un certissimo pericolo, per una incerta gloria, e massime nel
presente secolo, che vi è maggiore abbondanza di spropositati, e mordaci Satirici,
quali con invidioso livore lacerano le fatiche altrui, che di Censori discreti, i
quali con dotti avvertimenti n'emendino gli errori.

Dalle grandi instanze fattegli dagli amici suddetti, che egli stampasse questa
sua Novella, insospettito il Lippi, che il libro di detta sua composizione non gli
fusse levato, e contro a sua voglia stampato, andava molto circospetto, non lo
lasciando in luogo, dove fusse sottoposto a tal caso; Ma essendo una volta andato
in villa de' SS. Susini suoi cognati, e di quivi alla villa del sig.\ Don Antonio de'
Medici\footnote{forse Anton Francesco de' Medici, 1618-1659, frate dell'ordine dei Cappuccini}; dove havendo portato il detto libro per passare, leggendolo, la veglia,
la notte, mentre egli durmiva, il sig.\ Piovano Gualfreducci, ed il sig.\ Tommaso
Fioretti con l'assistenza del medesimo sig.\ D. Antonio sciolsero il detto libro, e
fra tutte due lo copiarono e la mattina lo rilegarono, e lo raccomodarono in
maniera, che egli non s'accorse del virtuoso furto. Questa copia capitò poi in
mano a Paolo Minucci\footnote{Paolo Minucci, Firenze 1606 - Radda 1695. ``Puccio Lamoni''}, il quale facendo al Lippi la solita instanza di metterlo alla
stampa, ed egli ricusando, gli disse il Minucci, che l'havrebbe egli fatto stampare;
e replicando il Lippi, che se ne contentava, se vi era modo, il Minucci
col mostrargli la detta copia scoperse il furto, e fece conoscere la possibilità, che
havea di farlo stampare, S'alterò non poco il Lippi veduto questo, ma come
huommo virtuoso, ed onorato volle, che la vendetta di tal disgusto fusse il costituire
il Minucci, ed ogni altro in grado di non si curar più di stampar quell'Opera;
questo fu con aggiugner'ad essa alcuni episodj, ed altro, in maniera, che in
breve tempo la ridusse da fette piccoli canti, che ell' era, alli dodici, che è la
presente; e perché non gli avvenisse di questa, come gli era accaduto della prima
teneva l'originale di essa in modo riserrato, e ristretto, che non lasciava vederlo
ne meno all'aria, e poco altro poteva haversene, che sentirne recitar da lui
qualche Ortava alla spezzata, ed il Minucci più d'ogni altro haveva questo favore
da lui, perché col fargli sentire l'augumento, che dava a quest Opera, stimava
di fare scemare nel Minucci la volontà di stamparla, e conseguir l'intento,
che s'era prefisso, ma ne seguì tutto il contrario, perché havendo il Minucci
sparso fra gli amici, che il Lippi riduceva la sua Opera in stato ragguardevole,
pervenne questa notizia all'orecchie del Sereniss.\ sig.\ Principe Card.\ Carlo de' Medici\footnote{Carlo de' Medici, 1595-1666.}
Decano del Sa.\ Collegio, e S.A.R.\ curiosa di veder quest'Opera comandò
al Minucci, che operasse d'appagare tal sua curiosità. Il Minucci manifestati al
Lippi i sentimenti dell'A.S.R.\ esortò a non contraddire di ricever l'onore
che S.A.R\ gustava di fargli; ed egli conoscendo, che mal poteva negare d'ubbidire
a tanto Principe, per il quale (come fratello della Sereniss, Arciduchessa.
Claudia) riteneva congiunto al debito di suddito un genio non ordinario di servirlo,
e persuafo pure una volta; che il pubblicar detta Opera non gli poteva
apportar se non lode, condescese a lasciarne pigliar copia per S.A.R.\ la quale si
piacque di dar dimostrazione del suo benigno aggradimento con atti non piccoli
della sua solita generosità, e verso il Lippi, e verso il Minucci, che ne fece
la copia, perché così volle il Lippi, o per spaventar il Minucci con la gran macchina,
che appariva, e così levarlo dal pensiero di pigliarsi questa fatica, ed
addormentare intanto nel sig.\ Principe Card.\ la volontà d'haverlo (come disse il
medesimo Lippi) o pure, perché quella copia non capitasse in mano ad altri, che
del medesimo Minucci, del quale si fidava, e per sua bontà, e perché haveva
anche veduto, che di quella copia, che teneva detto Minucci della prima Opera,
non s'era mai saputo cosa alcuna, perché esso Minucci l'haveva sempre occulata,
e negata a ognuno d'haverla, Ma quel'ultima copia sendo in mano del
detto Sereniss.\ sig.\ Card.\ Decano, accrebbe nei SS.\ suoi Cortigiani la curiosità
d'haverla, e cosè per diverse vie ne trassero una copia. Da questa poi se ne sono
sparse infinite; ma perché l'Autore sopravvisse qualche poco di tempo, e sempre
accrebbe, o moderò qualcosa, ed oltre a questo, perché la poca avvertenza di
coloro, che hanno copiato, ha causato, che si trovino molte copie, e difettose,
o guaste, il Minucci riputandosi in un certo modo cagione di questo disordine risolvette
per rimediarvi, di supplicare il Sereniss.\ Principe Leopoldo (allora non
Cardinale, al quale dall'Autore stesso fu quest'Opera dedicata, dopo la morte
della Sereniss.\ Arciduchessa Claudia) di permettergli il mandare la detta Opera
alla stampa, per rinnovare la memoria de] già defunto Lippi\footnote{Siamo quindi fra il 1665 ed il 1668.}, e S.A.\ glielo
concedette, con obbligo però, che gli facesse alcune Note, ed esplicazioni; E così
contento l'universale, che desiderava tal pubblicazione, e diede al Minucci il
gastigo d'esscre stato causa del suddetto disordine, ed al Lippi la soddisfazione\footnote{postuma}
dovutagli dal Minucci per la violenza fattagli, con obbligare il medesimo Minucci
a sottoporre ancor'egli i suoi scritti a quei danni, che dalle stampe ne
risultano; Sentenza veramente giusta, come appoggiata al fondamento della pena del
Taglione, ma troppo severa nell'arbitrio per la gran disparità, che è fra la vaga
Opera del Lippi, e l'insipide chiacchiere del Minucci, sopr'alle quali, e non
sopra gli scritti del Lippi si fermeranno, e poseranno tutti gli Aristarchi; con
tutto questo non ha il Minucci voluto intentare appello, anzi, sendosi accinto
subito a dare esecuzione alla sentenza, ha aggiunto all'Opera le Note comandate,
con le quali ha egli preteso d'operare, che fuori di Firenze, e della nostra
Toscana, e per tutta Italia possano esser meglio intese molte parole, detti, frasi,
e proverbj, che si trovano nell'Opera, forse non intesi del tutto altrove, che in
Firenze; e prega il Lettore a compatire, se non sia da esso soddisfarto appieno, e
ricordargli, che non è stata mente del Minucci il portare l'etimoiogia delle parole,
frasi, e proverbj, ma d'esplicargli in maniera, che possano esser'intesi anche
fuori di Firenze, ed habbia il medesimo Lettore la discretezza di riflettere, che
molti Fiorentinismi sono in uso, nati dal puro caso, senza un minimo
fondamento, o ragione, perché si dicano, e che;
\begin{verse}
Non omnium, quae a maioribus nostris scripta, aut dicta sunt, ratio reddi potest.\footnote{Adattato da Tommaso d'Aquino, Summa Theologiae, Q.\ 95, Art.\ 2. ``Sed non omnium quae a maioribus lege statuta sunt, ratio reddi potest, ut iurisperitus dicit.''}
\end{verse}

\clearpage
\noindent\textsc{\centering\large
\textls[240]{\Huge MALMANTILE}\\
{\small DISFATTO}\\\kern 6pt
\textls[360]{\LARGE ENIGMA}\\\kern 4pt
{\normalsize DEL SIG.\ ANTONIO MALATESTI.}\\
\kern 1em}

\begin{poesia}
Ov'è l'Etruria indomita, e infeconda,\\
Già fui per molti figli e ricco, e bello,\\
Or c'una fascia a pena mi circonda,\\
Povero, brutto, e vil non son più quello.

M'hanno gli amici più che 'l vento, e l'onde\\
Levate l'ossa, e toltomi il cappello,\\
E fino il nome par che corrisponda;\\
Una mala tovaglia, o un mal mantello.

Così ridotto trovomi a mal porto,\\
Col corpo voto, e senz'un membro intero,\\
E pur con tuttociò non mi sconforto;

Anzi ora godo, e farmi eterno spero,\\
Mentre in Flora un' Augel per suo diporto,\\
Cantando in burla, mi rifà da vero.
\end{poesia}

\chapter{Primo Cantare}
\pagenumbering{arabic}

PRIMO Cantare. Ecco che il nostro Poeta mantiene l'intenzione
data di pubblicare una Leggenda,e non un Poema, mentre mette
sopra ogni Canto l'inscrizione, che si vede in diverse leggende
dove in vece di dire Canto 1., e Canto 2, ec. come usano nei
Poemi Italiani, egli dice Primo Cantare, e così seguita fino all'ultimo,
volendo per la sua modestia esser chiamato Compositore
di Leggende, non Autore di Poemi, ed in uno stesso tempo
con bell'arte difendersi dalle censure di chi lo tacciasse di non aver'osservate le
regole del comporre i Poemi, sapendosi, che a queste non sono sottoposti i
compositori di Leggende.

\begin{comment}

\begin{argomento}
Marte sdegnato perché il Mondo è in pace
Corre, e da letto fa levar la suora,
E in finto aspetto, e con parlar mendace
Mandala a svegliar l'ire in Celidora,
Fa la mostra de' suoi Baldone andare
Indi all imbarco non frappon dimora,
E per via narra con che modo indegno
a occupate avea il suo Regno.
\end{argomento}

Gli Argomenti a tutti li Canti di quest'Opera sono di Amostante Latoni, cioè
Antonio Malatesti, fatti di comandamento del Sereniss.\ Principe Cardin.\
Leopoldo de' Medici.

\section{Stanza I}

\begin{ottave}
\flagverse{1}Canto lo stocco, e 'l batticul di maglia,\\
Onde Baldon sotto guerriero arnese,\\
Movendo a Malmantil' aspra battaglia\\
Fece prove da scrivern' al paese,\\
Per chiarir Bertinella, e la canaglia\\
Che fu seco al delitto in crimen lesa\\
Del far' a Celidora sua cugina,\\
Per cansarla del Regno, una pedina,
\end{ottave}

Mostra l'Autore in questa sua introduzione, che egli vuol descriver da Guerra
fatta da Baldone in aiuto, e difesa di Celidora, e vuol persuadere, che se ben
dice \textit{aspra battaglia} fu una guerra di nulla, e però seguita: \textit{fece prove da
scrivern'al paese}, del qual detto ci serviamo per derisione, quando altri ha fatta
una azione da lui stimata grande, e bella, che in effetto non è poi tale, anzi è
tutta il contrario, e si dice:  \textit{Hai fatto assai, scrivi al paese}.

\begin{description}
\item[BATTICVLO di maglia] Intende il Giaco, arme difensiva di dosso, cioè una
camiciuola composta di maglie di ferro, ed è la lorica ansulata, che usavano gli
antichi. E se bene \textit{batticulo di maglia} non è veramente buon Fiorentino, nondimeno
è spesso usato, ma per giuoco, ed è comunemente inteso per il Giaco, e si dice
così, perché coprendo quest'arme le parti di dietro, nel moto che fa colui, che
l'ha in dosso, batte in quella parte; come si dice Picchiapetto quel Gioiello, che le
donne usano portare al collo pendente sul petto.

\item[MALMANTILE] E' un Castello antico vicino a Firenze circa dieci miglia,
  oggi del tutto rovinato, e distrutto, ne vi si vede altro che lé muraglie Castellane.

\item[CHIARIRE] Questo verbo, che oltre a gli altri significati, vuol dire Far conoscere
  l'errore, o Render capace; nel presente luogo vuol dice Scaponire, o
  Sgarire: \textit{Il tale mi faceva l'huomo addosso, gli ho dato una buona quantità di pugna, e l'ho
chiarito}; cioè con questo l'ho reso capace, e fattogli conoscere la stima, che io fo
  di lui, e quella che egli deve far di me. Questo verbo è traslato dal verbo Chiarire,
  che è Purificare ogni liquore torbido, e contaminato da materie crasse.

\item[CANAGLIA] Gente vile, ed abietta, che tali saranno, come vedremo, i soldati
  di Bertinella, i quali il Poeta mette Huomini d'infima plebe, che Cicerone
  chiama Imi subsellij homines. Il Sig\ Francesco Maria Bellini in alcune sue bellissime
  reflessioni, che si è contentato fare sopr'alla prsente Opera, ponderando la
  parola Canaglia dice, che l'allungamento delle parole in \textit{aglia} sta Oggi in
  Toscana un certo avvilimento, e disprezzo del subietto, e s'usi solo in cose vili, e
  plebee, e però si dica de' Birri sbirraglia; della Plebe. Plebaglia, e gentaglia; de i
Fanciulli, e popolo infimo Spruzaglia, (metaforico da spruzolo, acqua minuta)
e che questo sia antichissimo Latino, sia di neutro plurale, del quale si servirono i
Latini per comprender l'appartenenze della cosa, della quale parlavano, v.g.\
delle cose appartenenti alle navi dicevono Navalia; alla Cacina Popinalia, e molt'altri,
è corrotto da noi con l'aggiunta della lettera G.

\item[IN crimen lesa] È delitto di lesa Maestà cacciare una Regina del
suo Regno.

\item[FAR' una pedina] Si dice Fare una pedina a uno allora che procurando questo
  tale di conseguire cosa di suo gusto, ed essendo vicino a ottenerla, un'altro, a
  cui haveva confidato tal negozio; gliela leva su. Viene dal giuoco di Scacchi, dicendosi
  propriamente: Dare scacco di pedina.

In oltre, chi è pratico del giuoco di Scacchi sa, che quando s'è perduta la
Regina, si procura di racquistarla con far' arrivare una pedina al posto dove
stava la Regina dell'avversario al principio del giuoco, e così intendere, che Celidora
priva del Regno conveniva, che sotto nome di Pedina tornasse a ricuperarlo,
se voleva esser detta Regina.

Si potrebbe anche dire, che il nostro Poeta seguitando il costume che habbiamo
di chiamar Dame le Signore grandi, e Pedine le donne d'infima plebe, habbia inteso,
che Bertinella, togliendo il Regno a Celidora, l'habbia cavata del nome di
Dama, per haverla ridotta in grado miserabile, le habbia fatto meritare il nome
di Pedina; ma l'esser' il nome, di Celidora nel terzo caso, e non nel secondo, o
nel quarto; fa languire questa riflessione.
\end{description}

\section{Stanza II}
\begin{ottave}
\flagverse{2}O Musa, che ti metti al sol di state\\
Sopr' un palo a cantar con si gran lena, \\
Che d'ogn'intorno assordi le brigate,\\
E finalmenre scappi per a schiena;\\
S'anch'io sopr'alle picche dell'armate\\
Volto a Febo con te venga in iscena,\\
Acciò ch'io possa correr questa Lancia,\\
Dammi la voce, e grattami la pancia.
\end{ottave}

Quest'Ottava ha poco bisogno di spiegazione vedendosi chiaro, che il Poeta,
invoca per sua Musa la Cicala, e così dà a conoscere, che egli vuole scrivere affatto
mostrando, che per fare una composizione come egli ha in animo,
e per descrivere una guerra qual fu quella di Malmantile, gli basta haver
chiacchiere.

Si potrebbe anche dire, che il Poeta sapendo che non si trova, che le Muse habbiano
dato mai alcuno aiuto effettivo, ed evidente, come dette la Cicala a Eunomo
Locrense Suonatore nella disputa, che hebbe con Aristono, supplendo con
la voce al mancamento della corda strappata, come si legge in Strabone lib. 6.
voglia, come fece Eunomo, far più capitale della Cicala, che d'altre Muse:
E può anch'essere, che egli invochi la Cicala, perché stimi più nobili delle Muse le
Cicale per esser queste più riguardevoli, come nate avanti alle Muse (secondo la
favolosa credulità de' Gentili) d'Huomini, li quali per lo gran gusto, che hebbero
del cantare, furono in cicale convertiti, come si cava da Celio Rodigino
\libcap[17]{6}.\ le cui parole sono queste: \textit{Fertur enim hosce homines fuisse ante Musas;
natis deinde Musis, cantumque monstrato, illorum nomnullos voluptare cantus usque adeo
delinitos fuisse, ut canentes cibum, potumque negligerent, imprudenterque perirent; ex
quibus deinde cicadarum genuss sit propagatum, ec,}

Dice il Doni nella sua Zucca, che tutti li Poeti hanno la loro Cicala, e che
questa serva loro per Fama publicando le loro Poesie, onde il nostro Poeta seguitando
l'opinione del Doni invoca la Cicala destinata al suo servizio, perché gli
faccia questo di pubblicare le sue Poesie.

\begin{description}
\item[PALO] Pertica, Bastone di legno, che si mette per sostegno alle viti, ed altri
  arbuscelli simili.

\item[LENA] Significa quello, che i Latini dicono \textit{respiratio}, cioè quieto, e
  tranquillo
  anelito, il che mentre è nell'Huomo, egli si mantiene senza difficultà, nelle
  forze: ma la troppa fatica di corpo, o di mente spesso fa affannare tal Lena,
  però che uno, che s'eserciti assai senza posarsi, appunto come fa la Cicala col
  suo cantare senza riposo, si dice Haver gran Lena.

  Dante Inf.\ C.\ 1.\ \begin{verse}E come quel che con lena affannata, ec.\end{verse}

  Al Canto 24.\ \begin{verse}La Lena m'era dal polmon si sì smunta, ec.\end{verse}

  Vedi sotto C.\ 4.\ stanza 6.

  Varchi\footnote{Benedetto Varchi (Firenze, 19 marzo 1503 – Firenze, 18 dicembre 1565), umanista, scrittore e storico.}  stor.\ lib.\ 5. \begin{verse}Essendo egli di pochissimo spirito,
    e di gentilissima Lena\end{verse}

  Franco Sacc.\ Nov.\ 127.  \begin{verse}Alla fine perdendo questi ciechi
      la Lena per essersi molto bene mazzicati, ec.\end{verse}

  I Latini con la voce \textit{Vis}, e con la voce \textit{robur} esprimevano questa Lena.

\item[VENIRE in scena] Comparire in pubblico,  vedi sotto \cstan[4]{6}.

\item[CORRER questa lancia] Tirar' a fine quest'Opera.

\item[GRATTAMI la pancia] Col grattare il corpo alla Cicala, ti fa che ella canti,
  la Cicala a grattare il corpo a lui, acciò che egli canti. Quand'altri
  sa qualcosa, ed è duro a manifestarla, si dice; \textit{Grattagli la pancia, che egli
    canterà},
  cioè interrogalo, ed esaminalo bene, che egli dirà tutto quello, che tu
vuoi; sì che il senso di questo detto \textit{Grattare il corpo a uno}, è Incitarlo a discorrere.
Vedi sotto \cstan[2]{8}.

\end{description}

\section{Stanza III \& IV}

\begin{ottave}
\flagverse{3}Alcun forse dirà ch'io non so cica, \\
E ch'io farei 'l meglio a starmi zitto, \\
Suo danno; innanezi pur, chi vuol dir dica, \\
Fo io per questo qualche gran delitto? \\
S'io dirò male, il Ciel la benedica; \\
A chi non piace, mi rincari il fitto: \\
Non so, se se la sanno questi sciocchi,\\
Ch'ognun può far della sua pasta gnocchi.

\flagverse{4}Mi basta sol che Vostra Altezza accetta\\
D'onorarmi d'udir questa mia storia\\
Scritta così come la penna getta,\\
Per fuggir l'ozio, e non per cercar gloria;\\
Se non le gusta, quando l'avrà letta\\
Tornerà bene il farne una baldoria:\\
Che le daranno almen qualche diletto\\
Le Monachine, quando vanno a letto.
\end{ottave}

In queste due Ottave l'Autore piglia a difender se medesimo dalle male lingue,
e mostra, che poco gl'importa l'esser lodato, o biasimato in questa sua Opera, e
che, non essendo obbligato a veruno, vuol soddisfare a se medesimo, ed al suo capriccio;
e però dice: \textit{S'io dirò male il Ciel la benedica}, che significa Vadia il negozio,
come e' vuole, che non m'importa. E seguita \textit{A chi non piace mi rincari il
  fitto}, volendo mostrare, che per non essere obbligato a render conto ad alcuno delle
sue azioni, non teme d'esser ripreso, o di ricever danno; e soggiugne: \textit{Ognun
  può far della sua pasta gnocchi}, cioè ogni huomo libero puo fare del suo, a suo modo.
Conchiude in somma, che egli vuol dar gusto a se medesimo, e lasciar dire
chi vuol dire, bastandogli, che S.A., cioè il Sereniss.\ Principe Card.\ Leopoldo de'
Medici, a cui dedica l'Opera, si contenti di riceverla, e d'udirla, \textit{scritta come
  la penna getta}, cioè composta non ad altro fine, che di spassarsi; ne si cura
d'acquistar gloria per tal composizione, anzi supplica S.A.\ ad abbruciarla quando
l'haverà letta, che riceverà qualche gusto dal veder' \textit{andare a letto le Monachine}. E
per Monachine intende quello, che intendono i nostri Fanciullini, cioè quelle piccole
scintille, che, nell'incenerirsi la carta, a poco a poco si spengono, e facendo
un certo moto, pare che si dileguino, sembrando tante Monache, le quali col
loro lume in mano scorrano per il dormentorio, andando a letto.

\begin{description}
\item[CICA] Niente. Anzi vuoi dire (se si può) Manco di niente, dicendosi in
  diminuzione \textit{Poco, niente, Cica}. Viene dal latino \textit{Cicum}, che vuol dir Quel velo,
  che si trova nelle melagrane per divisione de' suoi granelli, che per esser così sottile,
  e di niun valore, serviva ai Latuini per dimostrare la poca stima, che facevano
  d'una cola, dicendo: \textit{Ne Cicum quidem dederim}, ec. e noi diciamo in questo
  proposito \textit{lappola, lisca, ec.}
\item[ZITTO] Quieto. \textit{Stare zitto} vuol dire Non parlare, Viene dal cenno. \textit{Zi},
che si suol fare, quando senza parlare si vuol fare intendere a uno, o più, che
quietino, come facevano ancora i Latini, che per accennare ad altri, che si
quietasse profferivano le due consonanti S.T.

\item[GNOCCO] È una specie di Pane gramolato, mescolato con anici; e questa
pasta fra le nobili è la più vile: Il proverbio \textit{Ognun può far della sua pasta gnocchi}
significa ognuno ha il libero arbitrio, ed esprime quello, che i Latini dissero:
\textit{Unusquisque in re sua moderator, \& arbiter, ec}.
\item[SUO danno] Non m'importa, Non stimo questa cosa. E diremmo; \textit{io so che la
tal cosa m'è nociva, suo danno io la voglio non ostante ec}, Esprime Io la voglio, se
bene mi può nuocere, ec. Vedi sotto \cstan[4]{26}.\ al termine \textit{In ogni modo}.
\item[RINCARARE] Accrescere il prezzo. E questo detto Rincarare il fitto usato in
  questi termini significa: Non fo stima, ne temo le male lingue, perché non mi
  possono far danno.
\item[FITTO] Pigione, Canone, cioè Quel danaro, che si paga annualmente per
una Casa, o Podere, o altri beni, che si posleggono d' altri con pagargit un tan-
'to lvanno. Locarionis canones,
\item[BALDORIA] Fiamma accesa in materia secca, e rara, come paglia, e simili,
  che presto s'accende, e presto finisce; detta forse \textit{Baldoria} da Baldore, O
  Baldanza, che vuol dire Allegrezza: quindi \textit{Lieta} significa poi Baldoria, come vedremo
  sotto \cstan[2]{56}. Diciamo anche \textit{Far baldoria}, quando altri spende
  allegramente, e si da bel tempo consumando tutto il suo havere; il qual detto vien
  forse da un religioso costume, che era fra gli Antichi, che delle vivande sagre
non si lasciassero avanzi, ma quello che avanzava s'abbruciasse; il qual rito
si cava dai Precetti di Moisè in proposito del'Agnello Pasquale. Questa specie
di Sacrifizio fu usata anche da i Gentili Romani, e la dicevano: \textit{Proterviam
  facere}, che vuol dire Far'una fiamma, o baldoria; E pigliavano ancor'essi \textit{proterviam
facere} nel senso detto sopra di consumare, e mandar male il suo, come si cava
da Macrob. lib.\ 6.\ Saturnal.\ 2., dove si legge, che Catone motteggiando un tal
Albidio, che haveva consumato tutto il suo havere, e solo gli era rimasta una Casa,
la quale gli abbruciò, disse: \textit{Proterviam fecit, propterea quod ea, quae comesse
non potuerit, quasi combussisset.}
\end{description}

\section{Stanza V}

\begin{ottave}
\flagverse{5}Offerta gliel'haveo già, lo confesso,\\
Ma sommen'anche poi morse le mani, \\
Perch'il filo non va ne ben, ne presso, \\
E versi v'è ch'il Ciel ne scampi i cani:\\
Ma poi ch'ella la vuole, e io l'ho promesso\\
Non vo mandarla più d'oggi in domani,\\
Che chi promette, e poi non ta mantiene,\\
Si sa, l'anima sua non va mai bene.
\end{ottave}

Mostra l'Autore, che la convenienza per haver'egli promessa a S.A.R.' quest'Opera,
l'obbliga a mantenere la parola, quantunque egli conosca, che non sia cosa
d'esser veduta da S.A.R., e per questo s'è morso le mani, cioè pentito
grandemente d'haverla promessa, perché vede che la tesstura dell'opera non sta
ne bene, ne presso a bene, e vi son versi \textit{che il Ciel ne scampi i cani}, cioè così
stroppiati, che tanto male non ne vorrebbe vedere, ne meno a un cane.
Ed il verbo \textit{scampare} attivo, come è in questo luogo, significa Liberare. Ma
conchiude poi, che già che S.A.R.\ la vuole, non sta bene che egli la mandi più in
lunga da hoggi in domani, ma è dovere osservar la promessa; al che fare s'accigne
adesso, non solo per questa convenienza, ma ancora per il timore della pena meritata
da colui \textit{che promette, e non mantiene} la quale è che \textit{L'anima sua non va
  mai bene}. Sentenza usatissima da i nostri Fanciulli; e viene dall'antico, poiché
l'usavano ancora i fanciulli greci secondo il Monosino\footnote{Agnolo Monosini (Pratovecchio, 29 ottobre 1568 --- Firenze, 5 luglio 1626) presbitero, linguista. Contribuì alla stesura del primo Vocabolario della lingua italiana dell'Accademia della Crusca, pubblicato nel 1623, in particolare compilando un indice delle parole greche.  } Fior. Ital. linguae lib. 3.9.109.
dove cava dal Greco le seguenti parole: \textit{Nos autem dicimus id, quod solent pueri:
quae recte data sunt non licere rursus eripi}: Che suona lo stesso che: \textit{Chi da, e ritoglie
il Diavol lo ricoglie}, che vale lo stesso che: \textit{Chi promette, e non  mantiene L'anima
sua non va mai bene}.

\section{Stanza VI}

\begin{ottave}
\flagverse{6}Ma che? si come ad un che sempre ingolla\\
Del ben di Dio, e trinca del migliore, \\
Il vin di Brozzi, un pane, e una cipolla\\
Talor per uno scherzo tocca il cuore; \\
Così la vostr'Idea di già satolla\\
Di quei libron, che van per la maggiore,\\
Fore potrà, sentendosi svogliata,\\
Far di quest'anche qualche corpacciata.
\end{ottave}

Ripiglia animo il Poeta; e spera che S.A.R.\ sia per contentarsi di leggere
questa sua Opera, se non per altro, almeno per distrarsi dagli studj più serij, e
considera, che si come colui, che è solito far vita lautissima, havea talvolta gusto di
mangiare un pane, e una cipolla; e ber vino da niente, così chi è solito legger
libri più sensati, talora averà non poco gusto a legger libri di baie, e facezie.

\begin{description}
\item[INGOLLARE] Vuol dir Mangiar presto, ed inghiottire senza masticare.
  S'usa più il verbo Ingoiare, essendo il verbo \textit{ingollare} usato nel Contado, se bene è
  forse meno barbaro che \textit{ingoiare}, perché è più prossimo alla sua latina origine,
  che è la proposizione \textit{In}, e \textit{gula}, ed in questa appunto inghiottita la lettera \letter{l}
  secondo la stretta pronunzia comune Toscana, e mutato in I serrato, o consonante
  si dice comunemente Ingoiare: Così dice il sig.\ Francesco Maria Bellini.

\item[DEL ben di Dio] Delle più buone vivande; che i Latini dicevano \textit{Jovis nectar},
e noi diciamo \textit{latte di gallina}, che vedremo in questo \cant{64}.

\item[TRINCARE] Bere assai; Voce che viene dal Tedesco; e diciamo \textit{Trinca}, o
  \textit{Trincone}, uno che beva sregolatamente; Vedi sotto \cant[7]{1}.

\item[DEL migliore] S'intende quel che vuol dire, ma il senso più astruso puro
  Fiorentino è, che gli Osti di Firenze vendono sempre due specie di vino rosso, uno
di poco prezzo, che lo dicono Vino di sotto, o di bassa, perché viene da' luoghi
di sotto a Firenze, dove fanno Vini deboli, e leggieri; e l'altro di maggior prezzo,
che lo dicono vino di sopra, o de migliore; e di questo intende il Poeta.

\item[TOCCARE il cuore] Dar soddisfazione intera: Quando altri mangia con gusto,
  e si conosce, che quella vivanda gli fa pro, diciamo: \textit{Le tal vivanda gli ha
  toccato il cuore}.

\item[SATOLLO] Sazio, Ripieno. Dal latino \textit{satur}. Qui vale per Stracco di leggere.

\item[BROZZI] È un di quei luoghi sotto Firenze, dove nasce il detto vino debole.
Vedi sotto in \cant{47}.

\item[PER scherzo] Intendi non per fame, o sete; ma per stravizio, o tornagusto.
E' voce Tedesca, e là pur suona lo stesso

\item[ANDAR per la maggiore] Esser della prima ' fle: Traslato da i Magitteati
dell Arti della Città di Firenze, delle quali 5: ena: 'che sono
Giudici, e Notai; Cambio; Mer 5 Lana 5 Seta; Speziati, i
se paflano a Cavalleria, Alere Minori, che art eenan *) Quota eee
non paflano, 0: ra non pafiavano aca 'quando 'in
ze si dice, // ale va per: 'delle:

maggiore ss Sete 'una

are Arti, ed' della cap sw classe, Come s' intende ie laogo's
\item[SVOGLIATO] Senz' appetito: senza puto di mungevo

eae opie..

\item[FAR una corpacciata] Saziarsi. Empier benissimo il corpo =
corpacciata, gu altri legge, ree ° fa altra cosa'
te fa una volta.
\end{description}

\section{Stanza VII \& VIII}

\begin{ottave}
\flagverse{7}Già dalle guerre le Provincie stanche,  \\
Non sol più non venivano a battaglia,   \\
Ma fur banditi gli archi, e l'armi bianche, \\
Ed etiam il portar un fil di paglia \\
Vedeansi i bravi acculattar le panche \\
E sol menar le man fu la tovaglia;      \\
Quando Marte dal Ciel fa capolino,    \\
Come il topo dall'orcio, al marzolino

\flagverse{8}Che d'haverlo non v'è ne via ne modo,  \\
Se dentr'ad un mar d'olio non si tuffa, \\
E reputa il padron degno d'un nodo,   \\
Che lo lascia indurire, e far la muffa. \\
Così Marte, che vede l'armi a un chiodo \\
Tutt'appiccate malamente sbuffa,      \\
Che metter non vi possa su le zampe    \\
E che la ruggin v'habbia a far le stampe.
\end{ottave}

Il Poeta dà principio all'Opera, descrivendo lo stato, in che erano le cose del
Mondo, e dice, che tutto era in pace, ne si usava più arme di sorta alcuna; ed i
bravi, ed huomini armigeri acculattavano le panche, cioè Stavano oziosi, e menavano
le mani solo in su la tovaglia, che viene a dire Attendevano solamente a mangiare.
E qui scherza con l'equivoco del menar le mani, che vuol dir Combattere, vedi
sotto \cstan[10]{2}, e trattandosi del mangiare vuol dir Mangiare assai, e presto,
vedi sotto \cstan[6]{46}. Marte però s'adira, che non s'adoprino più l'armi.
L'Autore assomiglia Marte quando s'affaccia al Cielo, ad un topo, che s'affacci
alla bocca d'un'orcio pieno di cacio, e d'olio, che s'adira per veder tal cacio
abbandonato dal padrone, e di non poterlo arrivare, se egli non entra in detto
olio.

\begin{description}
\item[ARMI bianche] Spada, e pugnale, ed eggi altra sorta d'Armi, a distinzion
  dell'Armi da fuoco.
\item[PANCA] Arnese noto fatto di legname per uso di sedere, e possono starvi
più in una volta; detto da i Latini \textit{subsellium}, e viene dalla voce Latina
\textit{Planca}, che significa Assamenti, e tavolati piani.

\item[ACCULATTARE le panche] Significa (siccome habbiam detto) Starsene
senza far cosa alcuna, e spensierato. Ter.\ in An.\ disse \textit{Oscitantes} di coloro, che
stanno in questa maniera, quasi dica. \textit{Stanno sbavigliando}, che noi diciamo:
\textit{Starsene con le mani in mano}, o \textit{Fare a tu me gli hai}, o \textit{Dondelarsela}, e simili, che
tutti ci servono per Per esprimere \textit{Perder' il tempo in vano}, ed è quello che i Latini
dissero; \textit{Manum habere sub pallio}.

\item[TOVAGLIA] Quel panno lino che si distende, sopr'alla mensa da i Latini
  detto Mantile, e noi l'habbiamo forse da Toralia, che erano i panni, che
  \textit{circumponebantur in toris discumbentium}, ec.
\item[MENAR le mani] Quando è posto assolutamente, vuol dire Far quistione,
E con aggiunta, vuol dire Affrettarsi al lavoro, che sara aggiunto; e si usa dire
Mena le mani a correre, d'uno che corra assai, Mena le mani a leggere d'uno
che legga presto, ed in somma d'ogni Operazione humana, ancorche non fatta
con le mani, e qui vuol dire Mangiar prsto, ed il simile sotto \cstan[6]{46}.
\item[FAR capolino] Guardar di soppiatto. Quand'altri procura di vedere, senza
esser veduto, suole asconder la persona dietro a un muro, o altro, e cavar fuori
tanta testa, che l'occhio scuopra quel ch'ei vuol vedere, e questo si dice \textit{Far
capolino}. Sotto \cstan[2]{78}. dice \textit{Fa pan da Montui}, che è lo stesso.
\item[ORCIO] Vaso grande di terra, per uso di conservar' olio, vino, ed altri
  liquori, sì come per conservarvi, ed ugnervi il cacio.
\item[MARZOLINO] Specie di cacio tondo fatto a piramide, e'col manico nel
fondo dalla parte più grossa; chiamato Marzolino,perché si comincia a farlo nel.
mese di Marzo, ed è il miglior cacio, che si faccia nei nostri paesi. E nel
presente luogo, se ben dice \textit{Marzolino}, intende ogni sorte di cacio.
\item[DEGNO di nodo] Cioè merita la forca per l'errore che fa a non mangiare
quel Marzolino, lasciandolo andar male.
\item[TUTTE l'armi appiccate a un chiodo] Dicendosi: tale ha appiccate l'armi
all'arpione, al chiodo, s'intende: Il tale ha abbandonate l'armi, cioè Lasciato
d'essere armigero. Ciò viene dagli antichi gladiatori, i quali quando dal popolo,
col porger loro una bacchetta erano assoluti, e liberati dal far più il gladiatore,
solevano dedicar l'armi ad Ercole, appiccandole nel di lui Tempio, come
ci mostra Orazio lib. 1. ep. 1.
\begin{verse}
\makebox[12em]{\dotfill} Veianius armis.
Herculis ad postem fixis, latet abditus agro.
\end{verse}
Et lib. 3, ode 26.
\begin{verse}
Vixi puellis nuper iduneus,
Et militavi, non sine gloria;
Nunc arma, desunttumqnue belle
Barbiton hic paries habebit.
\end{verse}

\item[SBVFFARE] Dar segni d'ira. Sbuffare è quel soffiare, che suol fare per lo
più uno, che sia in collera, Traslato forse da i cavalli: E si dice Sbuffare,
quando altri adirato si duole, e in uno stesso tempo minaccia con parole.

Dante Inferno C. 18,: Ud.,
\begin{verse}
\backspace Quindi sentiamo gente che si nicchia
Nell'altra bolgia, e che col muso sbuffi,
E se medesima con le palme picchia,
\end{verse}

Viene da Buffo specie di soffio, che vedremo sotto \cstan[3]{57}.

\item[CHE la ruggin v'habbia a far le Stampe] La ruggine, rodendo il ferro, vi fa
  sopra certe impressioni simili a quelle, le quali con acqua forte si fanno nel rame
  per Stampare, e pero le dice Stampe.
\end{description}

\section{Stanza IX}
\begin{ottave}
\flagverse{9}Sbircia di qua di là per le Cittadi, \\
Ne altre guerre, o gran Campion discerne, \\
Che battagie di giuoco a carte, e a dadi, \\
E Stomachi d'Orlandi alle taverne, \\
Si volta, e dà un'occhiata ne' contadi \\
Che già nutrivan nimicizie ererne     \\
E non vede i Villan far più quistione    \\
In fuor che con la roba del Padrone.
\end{ottave}

Marte, riguardando bene per le Città, vede solamente guerre di giuoco, e
gente valorosa, e brava nel mangiare. Voltatosi poi ne i Contadi, che eran già
pieni di nimicizie, e risse, vede, che dai Villani non si fa altra guerra, che
che fanno con la roba del Padrone.

\begin{description}
\item[SBIRCIA] Sbirciare vuol propriamente dire Socchindere gli occhi, acciò che
  l'angolo della vista, fatto più acuto, possa osservare con più facilità una
  minuzia, Se bene si piglia ancora per Guardar per banda, a fine di non essere
  osservato, come fanno spesso gli amanti; movendo la pupilla alla volta dell'angolo
  esterno dell'occhio, con quel muscolo, che per tal cagione da' Medici si chiama
  amatorio; E questo \textit{Sbirciare}, o \textit{Bircio}, e \textit{Sbircio} ha forse l'etimologia dal Latino
  \textit{hirquus}, che Vuol dir l'angolo dell'Occhio. Verg. Egl. 3. \textit{Transversa tuentibus
  hirquis}; la qual parola vuol Servio, che abbia origine da \textit{hircus}, essendo che
  questi animali infuriati per la libidine guardano obliquamente, e torto le capre,
  che amano.

  È pero vero, che il nome Bircio, o Sbircio si dice non solamente di chi ha gli
  occhi scompagnati, ma generalmente ancora di chi ha qualsivoglia sorta d'imperfezione
  agli occhi, essendo noi in questo non differenti da i Latini, appresso
  i quali se ben \textit{luscus} vuol propriamente dire Uno, che ha solo un'occhio, come
  si vede in Giovenale Sat.\ X.\ che parlando di Annibale dice: \textit{Cum Getula ducem
  gestaret bellua lufcum}; che il Petrar.\ disse: \textit{Sour' un grande elefante un Duce losco}.
  E Cic. de orat. \textit{Hic luscus familiaris mens Catus Sentius} :

  \textit{Lusciosus} vuol dire
  Quello, che ha la vista corta, come si può dedurre da Varrone lib. 8. disciplin.\

  \textit{Strabo} Quello che ha gli occhi torti, da noi chiamato Guercio. Cic.\ 1.\ de
  Nat.\ Deor.\ \textit{Et quos insigni nota Strabones, aut Paetos esse arbitramur};che Paetus significa
Uno che abbia gli occhi leggiermente abbassati, che noi lo diremmo Luschetto.
Porfirione annot.\ ad Horat.\ lib.\ 1.\ Sermonum Sat.\ 3. \textit{Paeti proprie dicuntur, quorum
  huc, atque illuc oculi velociter vertuntur}, ec,

Coclites Quelli, che son nati ciechi
da un'occhio. Plau.\ in Cur.\ \textit{Unocule salve; ex Coclitum prosapia te esse arbitror ec}.

Lucini; Quelli che hanno ambedue gli occhi piccoli Plin. \libcap[10]{37}. \textit{Ab
ijsdem qui alter lumine orbi nascerentur coclites vocant, \& quibus parvi utrisque ocelli,
lucini vocantur}, ec.

Nyctilopes Quelli di vista così debole, che non veggono se non
quando splende il Sole. Plin. \libcap[8]{50}. \textit{Si caprinum iecur vescantur, restitui
  vespertinam aciem his, quos Nyctilopas vocant}, ec.

Non ostante, appresso molti queste
differenze si confondono, pigliando spesso l'uno per l'altro; così appresso noi
si confondono i nomi Guercio, Bircio, Orbo, Lusco, e simili, ec, accomodandogli
spesso a qualsivoglia imperfezione degli occhi, come vedremo sotto in questo
Cant. stan.\ 37\ che Orbo, vuol dire Affatto cieco, cioè Oculis Orbatus, e stan. 66.
vuol dire Lusco.

\item[CHE a battaglia di giuoco, e a carte, e a dadi] Non vede nel Mondo altre risse
che di giuoco, nel quale egli non ha che fare. Perché torna non affatto fuor di
proposito una riflessione sopra la voce latina \textit{Alea}, e la voce \textit{Talus}: si contenti
il Lettore, che io faccia un poca di digressione. Sono molti de' moderni Latini,
che si servono della parola \textit{Alea} per intendere la carta da giuocare; ma forse
pigliano equivoco, se vogliamo credere a Polidoro Vergilio, al Meursio, al
Soutero\footnote{Daniel Souterius, Vlissingen 27.8.1571 --- Haarlem 1635. Hervormd Predikant te Haarlem, 1615-1634.}, a Raffaello Volterrano, ed altri, che hanno trattato de i giuochi antichi,
i quali la chiamano \textit{charta lusoria}; \& \textit{Alea} chiamano Ogni specie di giuoco
di Fortuna, se forse quei tali non volessero sostenere la loro opinione con dire,
che quando la voce alea è presa in genere generalissimo; allora significhi ogni specie
di giuoco di fortuna: ma presa in genere speciale, significhi la carta da giuocarel
nel che mi rimetto alla prudenza del Saggio Lettore. So bene che fino il
giuoco de' noccioli era detto Alea, come si cava da Marziale.
\begin{verse}
  Alea parva nuces, \& non damnosa videtur,
Saepe tamen pueris abstulit illa nates, ec.\end{verse}
Altra volta la presero per Fortuna, secondo Livio lib.\ 37.\ che parlando d'Antioco
il quale volle più tosto guerra, che pace co i Romani per le dure condizioni,
che gli offerivano, dices, \textit{Nihil ea moverunt regem, tutam fore belli aleam
ratum; quando perinde ac victo iam sibi leges dicerentur}, ec, E Colum.\footnote{Lucius Iunius Moderatus Columella; Lucio Giunio Moderato Columella (Cadice, 4 – Taranto, 70) scrittore. }\ in Praefat.\ lib.\
1.\ dice \textit{Maris, \& negotiationis alea}. Pare che errino ancora, coloro, che
pigliano la voce \textit{Talus} per intendere il Dado, perché veramente il Dado si dice tessera,
e \textit{talus} vuol dire il Tallone, cioè Quel'osso, che è sopra il calcagno del piede,
donde si dice veste talare, la veste lunga infino a i piedi; E questa voce \textit{Talus},
trattandosi di strumento per giuocare e l'astragalo Greco, che è quello che i nostri
ragazzi chiamano aliosso; ma questo è forse minore equivoco, poiché tal'osso
finalmente viene usato in cambio di dado, servendosi per numeri di quelle macchie,
o segni, che naturalmente sono in dett'osso, come più largamente diremo
sotto C.\ 8.\ stan.\ 69. Gioviano Pontano nel suo Dialogo di Caronte distingue questo
aliosso dal dado, dicendo; \textit{Atque ego numquam talis lusi, nec tesseris}. Lo stesso fa
il Gellio lib.\ 1.\ Cap.\ 20.\ che dice \textit{Talus cubus non est, cubus .n, est figura ex omni latere quadrata, tessera sex lateribus constat}. Marziale pure nel lib.14.ep, 15. mostra
tal differenza, dicendo: \textit{Non sum talorum numero par, tessera dum sit Maior quam
talis alea saepe mihi} ec. Tal differenza si deduce anche da Cicer.\ lib.\ 2.\ de Divinat.\
\textit{Quid .n. fors est? idem propemodum, quod micare, quod talos iacere, quod tesseras}.

E tanto basti per rispondere a quei che biasimarono l'haver noi messo per esplicare
le presenti due voci Carte, e dadi il latino Charta Luforia, \& Tessera, che per altro
non importava al caso nostro questa digressione, e torna più a proposito il sapere,
che tali giuochi tanto di dadi, quanto di carte, dice Platone in Pedro, che
fussero inventati da un tal Theut Dio de gli Egizzj. \textit{Daemoni autem ipsi nomen Theut,
hunc primum numerum, \& computationem numerorum, Geometriam, Astronomiam,
talorum denique, alearumque ludos audivi}, ec. Raffaello Volterrano, e Celio Calcag.
de Ludo Talario, e Tesserario, dicono, che questi giuochi fussero trovati da Palamede
nel campo Greco sotto Troia, e però gli domanda, \textit{Palamedis alea}; sì come
fa il Soutero; Ma Isidoro\footnote{Isidoro di Siviglia (Isidorus Hispalensis; Cartagena, 560 circa – Siviglia, 4 aprile 636) Dottore della Chiesa, scrittore, teologo, arcivescovo di Siviglia. } lib. 8. Originum\footnote{Etymologiarum libri viginti, comunemente citata come Etymologiae, o Originum, può essere considerata la prima enciclopedia del mondo occidentale, servì da testo di riferimento per tutto il medioevo.}, concorda bensì, che havessero origine
nel detto Campo Greco, ma da un Soldato, che havea nome Alea, e che
da lui il giuoco prese il nome d'alea, Herodoto lib. 1. riportato da Polid. Verg.
\libcap[2]{13}. dice, che l'inventassero i Lidi per le cause che si diranno sotto C. 6.
stan. 34.

\item[STOMACHI d'Orlando] Dicendosi: \textit{Il tale è buono stomaco}, o vero. \textit{È uno
stomaco d'Orlando}, ec. s'intende, il tale è coraggioso, e bravo; Qui pero valendosi
dell'equivoco di \textit{Buono stomaco}, che vuol dir \textit{Gran mangiatore}, intende Gente
brava nei mangiare,:

\item[DAR un'occhiata] Intendiamo: Guardar' alla sfuggita.
\item[FAR quistione] Far contesa, disputa, rissa; ma dicendosi assolutamente
  senz' aggiunta: Far quistione, s'intende: Combatter con le spade, ec.
\end{description}

\section{Stanza X}
\begin{ottave}
\flagverse{10}Ond'ei ch'in testa quell' umor s'è fitto,\\
Che l'huom si scrocchi pur giusta sua possa;\\
Senza picchiar, ne altro, giu sconfitto.\\
L'uscio a Bellona manda in una scossa;\\
Niun fiata perciò, non sent'un zitto,\\
Perch'ella dorme, e appunto è in su la grossa,\\
Poiché la sera havea la buona donna\\
Cenato fuora, e preso un po di nonna.
\end{ottave}

Marte risolve d'unirsi con la sorella Bellona a fine di mettere scompigli nel
mondo, e andato a trovarla, la vede in letto a dormire briaca ancora della sera
passata.

\begin{description}
\item[UMORE]
Questa voce, che per altro significa materia umida, e liquida, e
parlandosi d'animali significa Flemma, collera, malinconia, ec, viene spesso da
noi presa per Fantasia, o pensiero come nel presente luogo, che dicendo: S'è
fisso quel'umore in testa, vuol dire ha stabilito, ha fermato il pensiero, ha risoluto.
La pigliamo ancora per Desiderio. Bartolomeo Cerretani stor. nell'anno
1502. dice: \textit{Si senti che l'umore di Piero de' Medici, di tornare in Firenze non
era spento, ec, Ma Papa Alessandro, desiderando fare il Valentino suo figliuolo Signore
di Toscana, si volle anch'egli valere di questo umore de' Medici}, ec, Diciamo Bell'umore
Uno che ha fantasie graziose. Vedi sotto in questo C., stan. 58. Si dice Far' il
bell'umore Uano, che vuol far da bravo, e da ardito. \textit{Il tale volle fare il bell'umore
col salire sopra quell'albero, e cascò}, ec. Donde habbiamo Umorista, che significa
Uno di cervello instabile, ed inquieto. \textit{Haver grand'umore} vuol dir' esser superbo,
ed haver gran pretensioni di se medesimo.

\item[CHE l'huom si crocchi] Che l'huomo si perquota. Il verbo crocchiare del quale
  ci serviamo alle volte per il verbo cicalare; come si vedrà in questo Cant.\ stan.\
  4., o C.\ 3.\ stan.\ 3., e che vuol' anche dire Quel suono, che fa un vaso di terra
  cotta fesso, come Pentola, o altro vaso simile; ci serve anche nel significato di
  dar busse, e questo intende nel presente luogo: propriamente Quel cantare, che
  fa la gallina chioccia, quando ha i pulcini.
\item[GIUSTA sua possa] Per quanto egli può; Frase antica latina \textit{iuxta meum posse}, ec.
\item[FIATARE] Significa parlare. Vedi sotto \cstan[6]{12}.
\item[È in su la grossa] È in sul buono del dormire. Dorme profondamente. Traslato
  dal baco da seta, il quale quando dorme per la 3.\ volta, che è il suo dormire
  più gagliardo; si dice: \textit{È nella grossa}.
\item[NON sente un zitto] Non sente verun rumore, cioè ne pur' un di quei cenni,
  \textit{zi} che dicemmo sopra questo Cant.\ stan\ 3. Il Varchi stor.\ lib.\ 6.\ dice: \textit{Con avvertir che ne cenni, ne zitti, ne atti brutti si facessero}.
\item[CENAR fuora] Intendiamo Cenar in conversazione\footnote{``conversazione'' sembra indicare il moderno ``circolo'', o equivalente dell'inglese ``club''.} fuor di casa propria.
\item[PIGLIAR la monna] Imbriacarsi. Ci sono più specie di briachi, fra' quali son
  quelli, che si dicono cotti monne, che son coloro, che per lo troppo vino bevuto,
  danno nelle buffonerie, e saltano, e chiacchierano spropositatamente, facendo
  mille altre pazzie, e poi s'addormentano; e si dicono ancora \textit{Cotti nonne},
  o \textit{pigliar la monna}. E questo è il nome generico, il quale comprende tutte le
  specie di briachi, di che parleremo sotto \cstan[2]{69}. In \cstan{77}.\
  dice. \textit{S'imbriacaron come tante monne} dal che deduci, che si può dire:
  \textit{Prese la nonna}, e \textit{prese la monna}, che in ambedue maniere ha
  lo stesso significato,

\end{description}

\section{Stanza XI}
\begin{ottave}
  \flagverse{11}Le scale corre lesto com'un gatto,\\
  poi dal salotto in camera trapassa,\\
  E vede sopr'a un letto mal rifatto\\
  ch'ell'è rinvolta in una materassa;\\
  Sta cheto cheto, e con due man dipiatto\\
  Batte la spada sopr'ad una cassa,\\
  La qual s'aperse, ed ivi vistevi drento\\
  Robe manesche, a tutte fece vento.
  \end{ottave}

Bellona non ostante ogni romore,  che faccia Marte, non si sveglia, ed egi
ruba alcune cose, le quali trovò ivi in una cassa. Esprime il Poeta il genio furibondo
di Marte, e la natura del Soldato, che è sempre dedita al rubare.
Esprime ancora la briachezza di Bellona, dicendo, che ella dormiva \textit{rinvolta
nelle materasse sopra un letto mal rifatto}; il che mostra, che quando Bellona andò a
dormire era in grado, che non sapeva distinguere le coperte dalle materasse.

\begin{description}
\item[LESTO come un gatto] La voce lesto, che viene dal Latino \textit{sublestus}, che vuol
dir Leggieri, frivolo, e debole, appresso di noi significa Pronto, agile, e destro;
E questa comparazione \textit{Lesto, come un gatto}; da noi è usatissima per esprimere la
grande agilità d'uno. Vedi sotto C.\ 2.\ stan.\ 35.

\item[SALOTTO] Intendiamo Piccola sala, cioè un ricetto prima che s'entri nella
principal sala.

\item[MATERASSA] Arnese da letto, quello che si dice in Latino Greco Anaclinterium
  a distinzione di \textit{culcitra plumea}, che noi diciamo \textit{Coltrice}; essendo la
  materassa un sacco largo quanto è il letto, e ripieno di lana, ed impuntito nel
  mezzo.

\item[Chero cheto] Quietissimo. Nota che la replica d'una stessa voce, appresso di noi,
  ha la forza del superlativo.

\item[DI piatto] Cioè per lo largo della spada.

\item[MANESCO] Uno che sia, diciamo noi, delle mani, cioè pranto, ed inclinato
  a perguotere, ed no che sia inclinato a rubare. Qui però vuol dire Robe
  atte, e comode a esser portate via. Roba manesca intendiamo Roba, che ci sia
  prenta, e comoda a valersene.

\item[FECE vento a tutte] Portò via ogni cosa. Rubò ogni cosa. Che questo intendiamo
  quando diciamo; Far vento a una cosa.
\end{description}

\section{Stanza XII}

\begin{ottave}
  \flagverse{12}Ma non fa sì, che la sorella sbuchi,\\
Di modo ch'ei la chiama, e li fa fretta;\\
La solletica, e dice: Ovvia fuor bruchi:\\
Lo Spedalingo vuol rifar le letta,\\
S'allunga, e si rivolta, come i ciuchi:\\
Ella ch'ancor del vin ha la spranghetta,\\
E, fatto un chiocciolin su l'altro lato,\\
Le vien di nuovo l'asino legato.
\end{ottave}

Con tutto che Marte faccia ogni diligenza perché Bellona si svegli, solleticandola,
e gridando, che è hora di levarsi, non trova modo di farla destare; anzi,
essendosi ella alquanto sollevata per causa di que' romori, s'allunga, e si rivolta,
poi si rannicchia, e di nuovo si addormenta, perché il vino la tiene oppressa.
Ed è bella espressione d'uno, che dorma con gran gusto, e volentieri; perché
questo tale, sentendo strepito, si risveglia alquanto, e facendo, per lo più, le
operazioni, e moti descritti nella presente ottava, seguita a dormire.
\begin{description}
\item[SBUCARE] Intende svegliarsi, e levarsi; Uscir da quella buca, la quale si fa
  nelle materasse col peso della persona.
\item[FAR fretta a uno] S'intende Stimolar' uno a far presto.
\item[SOLLETICARE] Stuzzicare leggiermente uno in alcuna di quelle parti del
  corpo, le quali, toccate così, incitano a ridere, Viene dal verbo \textit{Sollicito},
  \textit{sollicitas}, quanto val per Tentare.
\item[FUOR bruchi] Dalla voce Bruco habbiamo il verbo \textit{Brucare}, che vuol dir Levar
  le foglie a gli alberi, e per metafora vuol dire \textit{Andar via}, onde quando diciamo:
  \textit{Il tale sbrucò}, intendiamo, Andò via, ed, il simile intendiamo nel dire
  \textit{Fuor bruchi}, cioè andate via. Luigi Pulci Bec.\ \textit{Ognun brucò, che,
    l'era la tregenda}, Onde qui s'intende \textit{Escì, dal letto}. Detto, usatissimo in
  questo proposito.
\item[LO Spedalingo vuol rifar le letta] Questo detto significa, È hora tarda, e da
  levarsi dal letto; ed ha origine da gli spedali, ne i quali si raccettano i Pellegrini;
  dove, quando è hora di levarsi, e che i poveri, e i Pellegrini seguitano a stare
  nel letto, lo Spedalingo, cioè il Guardiano, o Sopracciò dello Spedale suole
  per svegliargli gridare: \textit{S'hanno a rifar le letta}.
\item[CIUCO] Asino giovane, ò poledro. Forse dal latino \textit{Cicur}, che par che
  voglia dire Bestia addomesticata, ed agevole.
\item[HA la spranghetta] o \textit{stanghetta}. Quel duolo di testa, ed inquietudine, che
  si sente la mattina, quando, la sera avanti s'è troppo bevuto, e poco quella notte
  dormito, per lo qual duolo pare, che il capo sia sprangato, o legato con spranghetta,
  o stanghetta. Che così si chiama ogni verga di ferro, o regolo di legno,
  che unisca due materiali insieme; come si dice porta sprangata, una porta, in
  mezzo alle di cui imposte sia conficcato a traverso un regolo di legno, affinché
  dette imposte non si possano aprire, E stanghetta pure si dice quel ferro, che serra
  insieme l'imposte de gli usci, il quale s'apre, e serra con la chiave facendolo
  scorrere in certi anelli, come il chiavistello, dal quale è differente, perché il
  chiavistello non si può, o almeno non è in uso aprir con la chiave.
\item[FATTO un chiocciolino] Cioè Rannicchiatasi, o raggruppatasi quasi in figura
  di chiocciola, come sono quelle focattole, o stiacciate, che fanno le nostre donne
  per i Bambini, le quali chiamano chiocciolini, perché gli fanno a figura di chiocciola;
  e come vediamo, che nel dormire fa per lo più il cane.
\item[LEGAR l'asino] Addormentarsi, Detto, che viene da i Villani vetturali, che
  essendo per strada soprappresi dal sonno, legano l'asino, e s'addormentano nel
  luogo, dove gli piglia il sonno. E col dire: \textit{Il tale ha legato} senza l'aggiunta
  \textit{d'asino}, s'intende; \textit{Il tale s'è addormentato}. Francho Sacchetti\footnote{Franco Sacchetti (Ragusa di Dalmazia, 1332 – San Miniato, 1400), letterato. Visse principalmente nella Firenze del XIV secolo. È oggi ricordato soprattutto per la sua raccolta Trecentonovelle. } nov.\ 171.\ dice:
  \textit{Essendo Gulfo entrato nel letto, quando fu per legar l'asino, il compagno cominciò col
    mantaco a soffiare}. Bocc.\ gior.\ 4.\ nov.\ 9.\ \textit{Di che la donna spaventata,
    per svegliarlo cominciò a prenderlo per lo naso, e tirarlo per la barba, ma tutto
    era nulla, perché egli haveva a buona caviglia legato l'asino}. ec.
\end{description}
\section{Stanza XIII}
\begin{ottave}
  \flagverse{13}O corna disse il Re degli Smargiaffi,\\
E intanto le coperte havendo preso\\
Le ne tira lontan cinquanta passi,\\
Ma in terra anch' egli si trovò disteso;\\
O che per la gran furia egli inciampassi,\\
O ch' elle fusson di soverchio peso,\\
Basta ch' ei batte il ceffo, e che gli torna\\
In testa la bestemmia delle corna.
\end{ottave}

Incollerito Marte leva le coperte a Bellona, e le butta in terra, dove cascò
ancor' egli, e batté il capo, e si fece un bernoccolo, o tumore nella testa, quali
tumoretti da molti per scherzo son chiamati Corna per esser nel luogo, dove nascono
le corna a gli animali.

\begin{description}
\item[DICE bestemmia delle corna] e' piglia la voce bestemmia non nel suo proprio
significato di attribuire, o levare empiamente alla Divinità quello che se le conviene,
ma nel significato di maladizione, o imprecazione, come è preso
tal volta nella nostra Toscana, ed in altre parti d' Italia, e specialmente in
Napoli, dove \textit{iastemiare} è inteso comunemente  per Maledire. E qui dicendo:
\textit{Torna in testa a lui la bestemmia delle corna} intende: Quell'imprecazione che
haveva fatta, venne addosso a lui, e viene a dire Si fece un corno nella testa, cioè
uno di quei bernoccoli, o tumoretti, che per esser nella testa scherzosamente si
chiamano Corna.
\item[SMARGIASSO] Huomo bravo. Armigero. Ma però l'usiamo per derisione,
  e per intendere Un'huomo fuor dei limiti della ragione, e della prudenza,
  ed uno di quei petulanti, e minacciosi, che pretendono di spaventar ognuno con
  la lor pretesa bravura.
\item[CINQUANTA passi] Lontano assai, Detto iperbolico usato spesso anche in
  piccolissime distanze.
\item[INCIAMPARE] Dar co i piedi in qualcosa nel camminare: è il Latino \textit{offendere}.

\item[SOVERCHIO peso] Peso grande, peso fuor di misura, Petr.\ Canz.\ 17.
\begin{verse}
Altri ch'io stesso, e il desiar soverchio,
\end{verse}
E certo che le coperte eran di grandissimo peso, perché Bellona si serviva per
coperte delle materasse, come s'è detto sopra.
\item[BASTA] Termine conclusivo usatissimo da Noi, quasi diciamo: \textit{È a sufficienza},
  e si dice anche \textit{A bastanza}, dal verbo \textit{Bastare}, che è il latino
  \textit{sufficit}. I Latini dicevano \textit{Bat, Sat est}. Plau.\ nel Penuo si servì
  della voce \textit{Bat}, senza aggiunta di \textit{Sat est}, ed i Giosatori di esso
  dicono: \textit{Bat vox, qua utimur cum quempiam iubemus tacere}.
\item[CEFFO] Vuol dir propriamente il muso del cane, del porco, o simili, ma
  si dice anche del Viso, o faccia dell'huomo, ma per lo più in derisione, e per
  intendere una faccia brutta, e mal fatta. Vedi sotto C.\ 4.\ stan.\ 10.
\end{description}

\section{Stanza XIV.}

\begin{ottave}
  \flagverse{14}Ella svegliata allora escì del Nidio,\\
E dicendo ch'in ciò gli sta il dovere,\\
E ch'ei non ha ne garbo, ne mitidio,\\
Non si può dalle risa ritenete,\\
Cosa ch' a Marte diede gran fastidio,\\
Ma perch'ei non vuol darlo a divedere,\\
Si rizza, e froda il colpo che gli duole,\\
Poi dice che vuol dirle due parole.
\end{ottave}

Per l'insolenze di Marte, Bellona finalmente si sveglia, e dà la burla a Marte
perché egli è cascato, e Marte fingendo non sentire la percossa si rizza, e dice a
Bellona, che vuole alquanto discorrerle.
\begin{description}
\item[USCIR del nidio] Uscir del letto: quale chiama Nidio per la similitudine, che
  ha nelle materasse quel luogo, dove s'è dormito, col Nidio, entro al quale covano
  gli uccelli..
\item[GLI fra il dovere] Gli è intervenuto quel ch' ei meritava. \textit{Dovere},
  \textit{giusto}, e \textit{giustizia}, sono sinonimi.
\item[NON ha garbo] Non ha accuratezza. Per intelligenza di questa parola \textit{Garbo}
è da sapere che erano in Firenze due luoghi principali,  dove già si fabbricavano
panni lani d'ogni sorta, uno detto S.\ Martino da una Chiesa, che quivi è dedicata
a detto Santo, e l'altro si domandava il \textit{Garbo}, quali nomi di strade si
conservano fino al presente. Nel detto il Garbo si fabbricavano le pannine di
tutta perfezione; e quelle che si fabbricavano in S.\ Martino erano sempre
d'inferiore condizione, onde venne in uso il dire: La tal cosa è del Garbo,
volendo denotare la perfezione di quella tal cosa. E dalle robe venne alle persone, e si
cominciò a dire: Huomo di garbo, huomo, che ha garbo, ec. intendendo d'uno
che operi bene, e con accuratezza. Cosi dice il Monosino Flor.\ It.\ linguae
alla parola Garbo. E noi diciamo ancora in questo Senso: \textit{Non ha ne Garbo, ne
S.\ Martino},
\item[MITIDIO] Giudizio; ordine; Parola corrotta da metodo.
\item[NON si può dalle risa ritenere] Non può far di non ridere.
\item[DAR fastidio] Dar noia; dar disgusto.
\item[NON vuol darlo a divedere] Non vuol farlo conoscere. L'aggiunta della particella,
  di, al verbo vedere s'usa solo in questo caso per esprimere, far capace,
  o render bene informato.
\item[FRODARE] È noto il suo significato, venendo dal Latino \textit{fraudare}, che vuol
  dire Ingannare; Ma noi lo pigliamo ancora per Occultare, o non manifestare,
  come è preso nel presente luogo; ed è traslato da quel \textit{frodare}:, che vuol dire
  Nascondere qualche roba alla porta della Città, o alla Dogana per fraudare la
  Gabella con il non pagarla, che si dice \textit{Far frodo} Vedi sotto C.\ 6.\ stan.\ 28.
\end{description}

\section{Stanza XV}
\begin{ottave}
  \flagverse{15}Dì pur: la Dea risponde, ch'io ascolto;\\
Hai tu finito ancora? Ovvia, dì presto:\\
Ma prima di quei panni fa un rinvolto,\\
E gettalo in sul letto ch' io mi vesto.\\
Quello non sol; ma quanto haveva tolto\\
Di quella cassa, ei rende, e mette in sesto,\\
E postosi a seder su la predella,\\
Con gravità dipoi così favella.
\end{ottave}

Descrive assai bene il genio inquieto, e furibondo di Bellona, mentre mostra
l'ardenza, con la quale ella stimola Marte a dir quanto gli occorra, interrogandolo
se egli ha finito, quando sa che non ha ancora cominciato, ed in uno stesso
tempo gli comanda, che rimetta le coperte in sul letto: Ubbidisce Marte, e
s'accomoda a sedere per dar principio al discorso, che sentiremo.

\begin{description}
\item[FAR un rinvolto] È lo stesso che Affardellare, abballinare, o far balle,

\item[METTERE in sesto] Accomodare; aggiustare. E in Latino \textit{aptare}, e da
\textit{Metter in sesto} diciamo \textit{Rassettare}, o \textit{metter in assetto}. Varchi Storia libro 8.
\textit{Havendovi dì, e notte lavorato per mettere il Salone in assetto}. L'Autore della
storia de' Piacevoli, e Piattelli lib.\ 2.\ dice \textit{Non pareva possibile distender la
  fila,allogare i lasci, e dar sesto al tutto, e pure ben tosto si vedde mettere ogni
  cosa in assetto}.

\item[PREDELLA] Qui intende Quella seggiola fatta a cassetta, la quale si tiene
vicina al letto per l'occorrenze del corpo; che per altro questa voce \textit{predella} ha
molti significati, chiamandosi predella ancora quell'arnese sopra il quale si posano
le donne quando partoriscono; Predella si dice quello scaglione di legno, sopra
il quale sta il Sacerdote quando celebra Messa; e quella seggiola dove siede
il Sacerdote quando in Chiesa ascolta le Confessioni detta altrimenti Confessionale.
Predella pure è detta quella parte della briglia, che si tiene in mano, come si
cava dal Landino esposizione a Dante nel Purg.\ C.\ 6.
\begin{verse}
  Guarda com'essa fiera è fatta fella,
  Per non esser corretta dagli sproni,
  Poi che ponesti man alla predella.
\end{verse}
\item[FAVELLARE] S'intende Ragionare, discorrere; Strettamente vuol dire
  Parlar con ordine, e massime quando è contrapposto agli verbi Cicalare, gracchiare,
  chiacchierare, e simili. \textit{Il tale non chiacchierava ne cicalava, ma favellava
    e discorreva}. Cioè parlava con fondamento, regolatamente, e seriamente.
\end{description}

\section{Stanza XVI}
\begin{ottave}
  \flagverse{16}Sirocchia, male nuove; poi ch' in Terra\\
Veggiam ch'all'armi più nessuno attende,\\
Onde il nostro mestiero, idest la guerra,\\
Che sta in sul taglio, non fa più faccende;\\
Sai, che la Morte ne molesta, e serra,\\
Che la sua stregua anch'ella ne pretende,\\
E se non se li dà soddisfazione,\\
La ci farà marcir n' una prigione.
\end{ottave}

Marte in questo suo discorso mostra alla sorella la necessità, che ambedue hanno
che si faccia guerra, per il bisogno, che hanno di guadagnare almen tanto da
pagare il dazio alla morte, acciò che ella non gli faccia metter prigioni, e quivi
morire, se non le pagano detto tributo.

\begin{description}
\item[SIROCCHIA] Sorella. Parola Fiorentina; ma oggi poco in uso. Dante nel
  Purg. C.-4, e Canto 21.; 4
  \begin{verse}
    Che se Pigrizia fusse sua Sirocchia, ec.
    L'anima sua ch'è tua, e mia sirocchia, ec.
  \end{verse}
\item[STA in sul taglio] Due specie di Mercanti di drappi, o diciamo Setaiuoli sono
  in Firenze. I primi fabbricano drappi per mandargli fuor di Stato, o per vendergli
  a merciai di Firenze a pezze intere; i secondi fabbricano, e vendono in
  Firenze a braccia, o diciamo a minuto, e questi si chiamano \textit{Setaiuoli, che stanno
    in sul taglio}, Marte dice alla Sorella, che la loro arte, che sta in sul taglio non
  lavora più, ed il Poeta scherza con l'equivoco di Tagliar drappi, e tagliar huomini;
  e che di questa lor'Arte di taglio vuole la morte, che essi paghino il dazio, dando
  alla medesima tanti morti l'anno; onde se la guerra non lavora, non possono
  pagar questo tributo.
\item[SERRARE] O far serra a uno, Affrettare, stimolare, violentare uno. Vedi sotto
C. 9. stanza 13.
\item[STREGUA] Intendi quel dazio, che devono alla morte. La voce stregua, che
vuol dir Porzione dovuta, vien forse dal Latino strena, che significa mancia.
Varchi Stor. lib. 10, \textit{In alcune cose vanno quei tali rispettati, ma in molte più devono
andare alla medesima stregua, e ragguaglio degli altri}, ec.
\item[DAR soddisfazione] Soddisfare, Adempire ogni sorte di convenienza, o di
  debito che uno habbia con un'altro: Ma strettamente s'intende Pagar quel danaro,
  del quale uno è debitore.
\item[CI fara marcir n'una prigione] Ci fara star tanto in carcere, che noi vi moriremo
  di stento; V'infradiceremo.
\end{description}

\section{Stanza XVII}
\begin{ottave}
  \flagverse{17}Bisogna qui pigliar qualche partito,\\
Se noi non vogliam' ir nella malora\\
Ed un ce n'è ch' è buono arcisquisito,\\
Qual'è, che si risvegli Celidora\\
C'ha dato un tuffo nelle scimunito,\\
Mentre di Malmantil si trova fuora,\\
E passandola sempre in piagnistei,\\
Pigra si sta, come non tocchi a lei.
\end{ottave}

Seguitando Marte il suo discorso, propone che si ponga in animo a Celidora
già cacciata da Malmantile, di risolversi alla vendetta, e così far nascere la
guerra; per rimediare a' lor bisogni.
\begin{description}
\item[PIGLIAR partito] Risolversi a pigliar qualche modo di rimediare.
\item[ANDAR nella malora] Intendi Andare in prigione per questo debito. E il
  latino \textit{In malam Crucem abire}.
\item[ARCISQUISITO] A buono, diciamo in augumento; buono, più buono, buonissimo,
  ed in luogo di buonissimo diciamo anche squisito, facendolo superlativo
  di buono e cosi non, dovrebbe patire agumento; tuttavia si dice Squisito, più
  squisito, squisitissimo, o arcisquisito, imitando forse i Latini, che da \textit{optimus}
  superlativo di \textit{bonus}, hanno, \textit{optimissimus}, Si trova anche nelli Scrittori antichi della lingua nostra.
  L'accrescimento al superlativo, il Bocc.\ nov.\ 19.\ dice \textit{Così santissima donna}, E nov.\ 60. \textit{Così ottimo parlatore}, ec, Gio.\ Villani\footnote{Giovanni Villani (Firenze, 1280 – Firenze, 1348) mercante, storico e cronista. Scrisse la Nuova Cronica, un resoconto storico della città di Firenze e delle vicende a lui coeve. } \libcap[12]{104}, dice: \textit{Rimase in più pessimo stato}, ed al \libcap[7]{100}, \textit{La quale era della maggiore di S.\ Gio.\ ed era molto
    fortissima} e cap. 101. \textit{A pié delle Montagne dette Pirre molto altissime}, e questo
  Autore l'usò sempre, che gli venne occasione d'esprimer un gran superlativo; ma
  da i moderni non pare, che sia molto usato, e con ragione, perché con l'aggiunta
  di molto, così, più, o simili, il superlativo che ha la natura del suo nome, riceve
  moderazione, e più tosto scema, e torna indietro della sua essenza;; e così volendo
  dire, che una Montagna sia altissima con Aggiungervi il \textit{molto}, \textit{così}, o \textit{assai}, si
  viene a dire che la Montagna sia alquanto alta, e non in tutto alta, o altissima
  ricevendo in questa maniera il superlativo limitazione, e non agumento. Salustio
  disse \textit{multo pulcherrimam} quando riporta il discorso fatto da Catone Uticense
  a Cesare in proposito della congiura di Catilina.

  La particella arci, che vien dal Greco archos,    che significa Superiore, s'usa
  anche da i moderni pen esprimere (se si, può) di là o più su del superlativo, ed il
  nostro Poeta l'usa anche nel Cant, 12. stan. 34    ma appresso di me anche questa
  particella arci aggiunta al superiativo fa l'effetto    che l'altre dette sopra di
  moderare, e non accrescere, ec.
\item[RISVEGLIARE] Non dal sonno, ma dalla Pigrizia.
\item[HA dato un tuffo nello scimunito] Ha fatta una azione da sciocca, e da stolta,
  Metaforico da i vintori, i quali volendo, che la seta, o altro, pigli il colore,
  l'intingono nel bagno di quel tal colore tante volte, quante par loro che serva.
  E questo dicono \textit{Dare un tuffo}, o \textit{più tuffi}. E dicendoti \textit{Il tale ha datoun tuffo nello scimunito}
  S'intende che quel tale habbia fatta un'azione da scimunito, non però
  che egli sia del tutto scimunito. Questo termine \textit{dar' un tuffo} può forse anche
  venire da coloro, che affogano, i quali prima di morire tornano alla superficie
  dell'acqua due, o tre volte, il che diciamo: \textit{Dare i tuffi}; e che, s'intenda è prossimo
  essere del tutto scimunito, come è vicino a esser del tutto morto colui, che da i
  tuffi nell'acqua. La voce \textit{scimunito} credo che sia composta di due dizioni, cioè
  \textit{scemo}, (che vuol dir' uno che habbia manco giudizio di quel che si conviene) e
  unito, e venga a dire \textit{unitamente scemo}, cioè scemo ugualmente, o del pari, o in
  tutte le parti a un modo, che conchiude affatto sciocco, e insensato.
\item[Si trova fuor di Malmantile] È priva di Malmantile perché le è stato tolto da
  Bertinella, o se ne trova effettivamente fuora. Diciamo: \textit{Io son fuora di tal
    pensiero} per intendere: io non ho più questo pensiero.
\item[PAGNISTEI] Singulti, solpiri mescolati con pianti. Voce da donnicciuole,
Vedi sotto \cstan[]{23}.
\item[COME non tocchi a lei] Cioè come l'interesse in questo negozio non sia, o
  S'aspetti a lei, ma ad un'altro.

\end{description}
\section{Stanza XVIII}
\begin{ottave}
  \flagverse{18}Ma come quella, pare a me, che aspetta,\\
Che le piovano in bocca le lasagne,\\
Senza pensar un' Iota alla vendetta\\
La sua disgrazia maledice, e piagne; \\
Hor mentre ch'ella in arme non si metta\\
Per racquistar lo scettro, e sue campagne;\\
Molto male per noi andra il negozio,\\
Che muoiam di mattana,e crepiam d'ozio.
\end{ottave}

Marte pone in considerazione a Bellona, che se non trovano il modo di far risolver
Celidora ad armar gente per racquistar il suo stato di Malmantile, il negozio
andra mal per loro, che non hanno faccende.

\begin{description}
\item[CHE le piovano in bocca te lasagne] Vuol del bene, e non vuol durar fatica a
domandarlo: come per esempio uno che ha gran fame, si lascia più tosto finire
da quella, che chiedere il cibo dovutogli, ma aspetta che il cibo gli corra in
bocca da se. Costume di Cuccagna.
\item[LASAGNE] Specie di pasta tirata, ed assottigliata come un velo.
\item[UN Iota] Piccola lettera dell'Alfabeto Greco, e si piglia per esprimer il \textit{niente}.
\item[MORIR di mattana] Morir di malinconia; quasi dica: È così grande la malinconia,
  che mi nasce dall'ozio, che mi fa divenir matto, e morire. Viene da
  \textit{macto}, \textit{mactas}, e forse prima si diceva: Perire di morte mattana, ec. che era una
  occasione speciale, che si faceva da gli Aruspicj nell'immolar le Vittime, le quali
  sventravano vive, e così morivano a poco a poco crudelmente; La onde i Latini
  aggiungono sempre a questo verbo la parola morte o supplicio, come si vede
  in Cicerone, che dice \textit{Morte mactavit}, \& \textit{supplicio mactari}.
\item[CREPARE] Questo verbo Crepare, che significa Quando un legname si spacca,
  o fende da per se: significa ancora Morire a stento, ed in questo senso è preso
  nel presente luogo, o forse e preso nel senso d'Allentare, che vuol dire Quando
  a uno per la soverchia fatica cascano gl'intestini, e voglia Ironicamente
  parlando, che s'intenda; è così grande la fatica, che duriamo, che ci fa allentare.
\end{description}
\section{Stanza XIX \& XX}
\begin{ottave}
  \flagverse{19}Chi sa? forse costei se ne sta cheta\\
Perch' ella vede esser legata corta,\\
Che s'ell'havesse un dì gente, e moneta\\
Tu la vedresti uscir di gatta morta;\\
Ma qui Baldon farà dall'A alla zeta\\
(So quel chi dico, quando dico torta)\\
Ritrova tu costei, sta seco in tuono,\\
Che quant'al resto anch'io farò di buono.

\flagverse{20}Vattene dunque, e in abito di mago,\\
Dopo il formar gran circoli, e figure\\
Conchiadi, e dille che tu sei presago,\\
Che presto finiran le sue sciagure,\\
E quel tuo corazzon pelle di drago\\
Imbottito d'insulti e di bravure\\
Mettile in dosso, che vedrala poi\\
Far lo spavaido più, che tu non vuoi.
\end{ottave}

Marte facendo riflessione che se Celidora havesse chi la soccorresse, ed
aiutasse, ella si muoverebbe a procurare di racquistare lo stato, perciò ordina a
Bellona, che la vadia a trovare, e la rincuori con dirle, che presto riavera il suo
stato, e le metta addosso l'usbergo incantato.

\begin{description}
\item[CHI sa?] Questo termine significa che la tal cosa può essere, o non può essere,
  quasi dica: Chi è colui, che sa di sicuro, che la cosa sia, o non sia così?

\item[È legata corta] Cioè non ha forze bastanti a far quello, che ella  vorrebbe.
  Traslato dal cavallo, asino, mulo, o simili, i quali quando son fieri, e bizzarri si
  legano dovunque si sia con la cavezza corta, affinché non offendano chi va loro
  d'attorno.

\item[VSCIR di gatta morta] Farsi vivo, dimostrarsi fiero. \textit{Far la gatta morta} vuol
dir Simulare. Il Lalli En. Trav, Cant. 2. stan. 12. parlando dsl Cavallo Troiano
dice:
\begin{verse}
  e stanno i Greci ascosti in questo legno,
  e v'attendono a far la gatta morta.
\end{verse}
I Latini dissero \textit{lepus dormiens}, E noi diciamo anche \textit{far la gatta di
  Masino}. Vedi sotto \cstan[7]{69}.

\item[FARÀ dall'A alla zeta] Farà puntualmente quanto bisogna. Farà il tutto.
  L'A, e la Z. sono il principio, e il fine del nostro Abbicci, onde con questo
  termine intendiamo \textit{Sarà fatto il tutto}, come appunto appresso i Greci Alpha, \&
Omega; che è lo stesso che \textit{Capite ad calcem} de' Latini.
\item[SO quel ch'io, dico, quando dico torta] So benissimo come sta questo negozio,
  Esprime \textit{m'intend'io}, Il Pulci nel suo Morgante fa dire a quello scellerato di
  Margutte.
  \begin{verse}
    Io credo nella torta, e nel Tortello:
    Sò quel, ch'io dico, quand'io dico torta,
  \end{verse}
E vuol dire M'intend'io, quel ch'io voglio dire, e quello ch'io intenda per
torta.
\item[STA seco in tuono] Sta seco unita; Va d'accordo seco. Traslato dalla Musica.
\item[FARÒ di buono] Negozierò da vero. Farò quanto bisogna. Quando uno
giuoca di danari si dice \textit{Far di buono}, che vuol poi dire Operar con attenzione; il
chee non si fa quando non si giuoca di buono, non ponendosi attenzione quando
si giuoca da burla.
\item[ABITO da Mago] Non hanno i Maghi abito particolare, ma il Poeta se lo
  figura in quella guisa, che ha veduto in commedia, cioè veste lunga, gran barba,
  e la verga in mano. E \textit{Mago} è voce Persiana, che significa \textit{Sapiens}, e
  quello che i Greci dicono Filosofo. E di questa sorte Filofofi furono quelli Magi, che
  andarono ad adorare Giesù bambino. Ma perché Zoroaste fu anch'egli uno di
  tali Filosofi detti Magi, e secondo Plin. \libcap[30]{1}. fu inventore dell'Arte
  dell'incantare, però tal arte è detta Magia, e coloro, che l'esercitano son chiamati
  Magi. Tasso Gerusal. C.\ 10.\ stan.\ 29.
  \begin{verse}
    Son detto Ismeno, i Siri appellan Mago,
    Ma che dell'arti incognite son vago.
  \end{verse}
  E perché quest'arte, secondo Polid. Verg. \libcap[1]{33}. è di sei specie, cioè
  Negromanzia, Geomanzia, Chiromanzia, Piromanzia, Aeromanzia, Hydromanzia,
  però questi Magi son detti ancora Negromanti, ec, Vedi sotto Cant, 2. stan. 5.
\item[SCIAGURA] Questa voce  parrebbe che significasse Scelleraggine, o
  Sciagurataggine si piglia da noi per Disgrazia.  Boccaccio Novella 36. \textit{La
    storia del mio ardire, e della mia sciagura vi racconti} E N. 43. \textit{E della sua sciagura
  dolendosi}. I Latini pure dicevano \textit{Scelus}, e se ne servivano nello stesso modo, che
 facciamo noi per intendere Disgrazia. Plaut. in Capt. \textit{Maior potitus hostium est,
   quod hoc est scelus? Quasi in orbitatem liberos produxerim}. Ter. in Eun. \textit{Neque quemquam
 esse ego hominem arbitror, cui magis bonae Felicitates omnes adversae sint. P. Quid
hoc est sceleris?} Il medesimo significato ha la voce latina — che a noi ha la
voce Sciagurato.

\item[CORAZZONE] Corazza grande, Armatura di petto, e schiene; dal latino
\textit{Thorax}, si dice anche Petto a botta, perché è a figura d'una botta, o perché si
presume, che regga a una botta d'archibuso.

\item[IMBOTTITO] Ripieno, e trapuntato non di cotone, o altro simile, \textit{ma d'insulti
  e di bravure}, che vuol'intendere Incantato, come vedremo appresso nell'ottava 27.

\item[SPAVALDO] Huomo avventato; Huomo inconsiderato, Dal latino \textit{supervalidus}
  Soverchiamente ardito, e quasi temerario, e tutto impertinente.
\end{description}

\section{Stanza XXI \& XXII}
\begin{ottave}
  \flagverse{21}Bellona c'ha il medesimo capriccio\\
Di far braciuole, va col sarrocchino.\\
Con il bordone, e un bel barbon posticcio,\\
Sembrando un venerabil pellegrino;\\
E fatto di parole un gran pastriccio\\
Esser dicendo astrologo, e indovino,\\
Che vien di quel discosto più lontano\\
La ventura le fa sopr'alla mano;

\flagverse{22}Ove doppo mostrato ogni accidente\\
Di tutta la sue vita pel passato,\\
Seggiunge, che per via d'un suo parente\\
In breve tempo riavrà lo stato;\\
Però si metta in arme, ch'un presente\\
Le fa d'um panceron, che ancorché usato\\
Ripara i colpi ben per eccellenza,\\
E poi piglia da lei grata licenza
\end{ottave}

Bellona va a trovar Celidora, e fingendosi Astrologo, le dice molte cose occorsele
per il passato, per accreditarsi; poi le predice, che fra poco tempo ella
riavrà il suo Stato, però si metta in armi; e le dona la corazza incantata, e si
parte.
\begin{description}
  \item[CAPRICCIO] E Pensiero, fantasia, volontà., come intende anche sotto C. 6,
stan. 101. E per altro \textit{capriccio} significa quello, che i Latini dicono \textit{orrore}, che è
quando i peli s'arricciano; il che segue o per lo freddo, o per qualche subito spavento,
o ne i casi di febbre, come s'intende sotto \cstan[6]{14}. e \cstan[20]{2}.
Donde poi habbiamo il verbo \textit{accapricciare}, che vuol dire Havere spavento. Dante
Inf. C22.
\begin{verse}
  Lo viddi, ed anche il cor men' accapriccia
\end{verse}

\item[BRACIUOLE] Si dicono quelle fette, o strisce di carne di porco, o d'altro
  animale, che sono così tagliate per cuocerle sopr'alla bracie, e però dette
  \textit{braciuole}, Ma qui intende fette d'huomini, e vuol dire che Bellona havea la
  medesima volontà di far guerra, che haveva Marte.
\item[SARROCCHINO] È un collarone di cuoio, il quale adattato al collo cuopre
tutte le spalle, e buona parte delle braccia, e petto a foggia di Manteiio, ed è
usato da i Pellegrini, che vanno a piede a visitare i luoghi santi; E questi tali
sono da noi chiamati Pellegrini corrottamente da Peregrini; la qual voce è latina, e
ritiene appresso di noi gli stessi significati di singolare, e grazioso, ed anco di
forestiero, \textit{Peregrinus in domo patris mei}, Petrarca Can. 12.
\begin{verse}
  Mosse una Peliegrina il mio cor vano
\end{verse}
Et intende, che una graziosa, e bella donna mosse il suo cuore. E la detta voce
Sarrocchino credo, che venga da San Rocco il quale portava forse questa parte
d'abito, quando andò peregrinando il Mondo.
\item[BORDONO] È nome particolare, e proprio di quel bastone, che portano i
Pellegrini.
\item[PASTRICCIO] Massa confusa di diverse robe. Qui vuol dire quantità di
  parole mal' ordinate.
\item[DAL discosto più lontano] Più lontano della lontananza stessa, come diremmo:
Vero più del vero, o della stessa verità.
\item[FAR la ventura] Strolagare. Sono alcune donnicciuole originarie d'Egitto,
le quali in Toscana vengono il più delle volte di Sicilia, e si chiamano Zingane.
Queste, dando a creder d'esser perite di chiromanzia per buscar denari, vanno
considerando i lineamenti delle mani alle persone, e palesano (dicono esse) le cose
passate, e predicono le future: E perché discorrono artifiziosamente con certi
lor generali sempre di bene; esse chiamano, ed anche da tutti noi vien detta questa
operazione; \textit{Far la ventura}, o \textit{la buona ventura}.
\item[PARENTE] Intendiamo ogni sorte d'affini, o consanguinei in qualsisia grado;
  così è inteso nel presente luogo, che vuol dire Baldone cugino di Celidora.
Così l'intese Dante nel Parad. C.6., e il Petr. Son. 191. E se bene strettamente
vuol dire il genitore, venendo dal latino \textit{Parens}, e usato da noi in tal senso
assai di rado, e forse non mai fuor che nel numero del più, come l'uso Dante Inf.
Cant. 1.
\begin{verse}
  \makebox[8em]{\dotfill} Homo già fui
E li parenti miei furon Lombardi,
Mantovani per Patria ambi dui,
\end{verse}
Ed il Petr. Canz. 29.:
\begin{verse}
Madre benigna, e pia,
Che cuopri l'uno, e l'altro mio parente,
\end{verse}
\item[PANCERONE] Intende quella gran corazza detta sopra in \cstan{20}.
\item[ANCORCHÉ usato] Adoperato, Vecchio, Antico.
\item[PIGLIAR buona licenza] Pigliar commiato, Licenziarsi da uno per andarsene.
  E quell'epiteto di \textit{buona}, o \textit{grata} s'aggiugne per esprimere, che quel
  tale parte con buona grazia dell'altro, e con il di lui consenso, e non forzato,
  o scacciato.
\end{description}

\section{Stanza XXIII \& XXIV}
\begin{ottave}
  \flagverse{23}Già il termine d'un anno era trascorso,\\
  Che Celidora havea perduto il Regno;\\
  Quando non pur le spiacque il caso occorso,\\
  Ma volle un tratto ancor mostrarne segno,\\
  Perciò richiesto ai convicin soccorso,\\
  Che un piacer fatto non havrian col pegno,\\
  e tenevano il lor tanto in rispiarmo,\\
  ch'egli era giusto, come leccar marmo.

  \flagverse{24}Fece spallucce a Calcinaia, e a Signa,\\
  Ma la pania al suo solito non tenne,\\
  Perché terren non v'era da por vigna;\\
  Calò nel piano, e ad Arno se ne venne,\\
  Ove Baldon facea nella Sardigna\\
  Vele spiegare, e inalberar' antenne,\\
  Fermato havendo lì come buon sito\\
  D'armati legni un numero infinito.
\end{ottave}

L'Autore toccando la finta storia della perdita dello Stato di Celidora, dice,
che era già passato un'anno, quando la medesima cominciò ad haver pensiero di
ricuperarlo, e per ciò fare, richiese soccorso a diversi vicini, ma senza frutto; la
onde si risolvé di venirsene verso Firenze, e trovò in su la riva d'Arno in un
luogo detto Sardigna Baldone con una buona armata.
\begin{description}
\item[UN tratto] Una volta, La voce tratto ha molti significati dicendosi \textit{tratti di
  fune}, Quello scarrucolamento, che si da a i delinquenti nel martirio della corda.
  \textit{Tirar i tratti}, diciamo Quelli ultimi moti, che fanno i moribondi nell'esalar lo
  spirito. \textit{Tratto} si dice in vece di estratto, cavato, o dedotto, ec, \textit{Tratto} val per
  distanza, dicendoli tratto di tempo, tratto di via, e simili, \textit{Tratto} di cortesia per
  Atto di cortesia, \textit{Tratto} per maniera, Ed in questo luogo significa Finalmente, ed è
  il latino \textit{tandem aliquando}.

\item[VN piacer fatto non havrian col pegno] S'intende Uno, che non fa mai servizio
  a veruno, eziam se li fusse dato il pegno in mano.

\item[TENER il suo in rispiarmo] Tenere il suo a fe, e con riguado, molti dicono
  \textit{risparmio}, e \textit{risparmiare}.

\item[GIVSTO] Questo termine significa Per l'appunto.

\item[ERA come leccar marmo] Era vana ogni diligenza per appunto, come è vanità
  leccar' il marmo.

\item[FECE spallucce] Si raccomandò. Questo detto seas dai poverelli, che per
  muovere a compassione in domandando l'elemosina, fanno tutte le smorfie, e
  gesti, che fanno, e possono, e fra gli altri il più comune il \textit{Fare spallucce}, cioè
  Stringer le spalle alla, volta del collo.

\item[LA pania non tenne] Non fece cosa di buono, cioè non hebbe aiuto da coloro
  da' quali lo sperava; intendendosi con questo dettato, che quel tale, che fu richiesto,
  non adempì il volere di chi lo richiese; che diciamo ancora: \textit{Non ha trovato
    appicco}. I Latini pure in questo proposito dissero \textit{Evanuerunt insidia}. \textit{Pania}
  intendiamo il visco, col quale si pigliano gli uccelli. E diciamo \textit{Non tenere} quando,
  o per il molle, o per altro la pania non appicca, ne li prende.

\item[AL suo solito] Secondo il suo costume, Dice al suo solito per dimostrare, che
  in quei paesi era da sperar poco bene al solito, \textit{perché non v'è terreno da por vigne},
  che vuol dire: Non è da far fondamento, o da sperare da loro favore alcuno, e
  scherza con l'equivoco del \textit{porre vigne}, perché veramente quei paesi non hanno
  terreni buoni a porvi le viti.

\item[CALO' nel piano] Scese nel piano, perché Calcinaia, e Signa sono piccole
  collinette vicino ad Arno.

\item[OVE Baldon facea nella Sardigna] L'Autore, che vuol sempre stare in su le
  burle, e servirsi dello scherzo degli equivoci, fa che Celidora trovi Baldone nella
  Sardigna; e pare che voglia dire l'isola di Sardigna, ed intende di un luogo fuori
  delle mura di Firenze in fa la riva d'Arno, così detto per il fetore, che quivi
  sempre si sente a causa delle bestie del piè tondo, che morte si fanno in quel luogo
  scorticare: e tal nome viene dai Latini; che chiamavano; Sardinia. quei luoghi,
  li quali per li mali odori sono sottoposti all'infezione dell'aria, come è l'isola
  di Sardigna, la quale per havere da Settentrione monti altissimi, che le
  impediscono i venti, è sempre di cattiva aria, e sottoposta alla pestilenza. Di qui
  ancora li nostri Medici hanno dato il nome di Sardigna a quel luogo, nello Spedale
  di Santa Maria Nuova di dove si mettono gli infermi più fetenti
  per piaghe, o altro simile. In detta riva d'Arno chiamata \textit{Sardigna}, si fermano,
  e scaricano, e si ricaricano, i Navili, che da Livorno vengono a Firenze su
  per lo fiume d'Arno, e tali legni, che quivi son sempre in gran numero, finge
  che sieno l'armata di Baldone. Su questa riva, come s'è detto sono gli scorticamenti
  delle bestiacce morte, e però dice, \textit{che vi era buon sito}, e si serve di questa
  voce \textit{sito} per \textit{posto} ed in effetto vuol dire Puzzo, o Mal'odore, che scaturisce da
  quelle Carogne, e la parola \textit{sito}, che vuol dire l'uno e l'altro, fa nascere un bello
  scherzo.  Quello medesimo scherzo può farsi anche nel Latino, perché dicono
  \textit{Situm casprorum} secondo Ces. de bello Gallico, ed intendono ancora puzzo secondo
  Plin. lib. 21, \textit{Pessimum esse Crocum, quod situm redolet}.
\end{description}

\section{Stanza XV \& XXVI}
\begin{ottave}
  \flagverse{25}Costui quando Bellona fu inviata\\
A Celidora, come già s'intese,\\
Da Marte haveva havuta una fardata,\\
Che lo tenne balordo più d'un mese,\\
E gli messe una voglia sbardellata\\
Di far battaglia, e mille belle imprese;\\
Ond'egli entrato in fregola sì fatta\\
Fece toccar tamburo a spada tratta.

\flagverse{26}Poi che pedoni egli hebbe, e gente in sella\\
Tanta ch'al fin si chiama soddisfatto,\\
Render volendo il Regno alla Sorella,\\
E farle far bandiera di ricatto,\\
Destinò muover guerra a Bertinella,\\
Ch'a lei già dato havea la scacco matto;\\
Cosè con quell'armata, e quei disegni\\
In Arno messe i sopradderti legni.
\end{ottave}

Marte era stato a trovar Baldone, conforme haveva detto alla Sorella, e l'haveva
fatto rifolvere a mettersi in arme per aiutare Celidora, e rimetterla nello
Stato; e perciò con questa gente a tal fine s'era imbarcato.

\begin{description}
\item[FARDATA] Percossa data con un pannaccio intinto in sporcizia; perché
  farda vuol dire sornacchio, che è Un grande sputo catarroso. Vedi sotto in questo
  Cant. stanza 47. E s'intende ancora per Una quantità di sporcizia bituminosa,
  che tirata in qualche luogo s'appicchi, e s'interni in quel luogo dove è buttata,
  come farebbe una manata di fango, o altro simile buttato in un muro; Dal
  che per metafora intende in questo luogo per Un colpo, che s'appicchi, e s'interni,
  quella persuasione, che Marte haveva fatto a Baldone di far guerra.
\item[BALORDO] Questa voce che vuol dir Inavvertito, Smemorato, che è il
  latino \textit{mente captus}, ci serve per intendere D'uno, che per qualche accidente
  occorsogli, resti sopraffatto, e non sappia a qual partito appigliarsi, per rimediare
  al danno che da quello accidente gli resulta, e si dice anche \textit{Sbalordito},
  \textit{Stordito}. Vedi sotto C.~11, stan.~25.
\item[SBARDELLATO] Una cosa che eccede i termini del naturale, ed in un certo
  modo avanza il superlativo, perché si dice: Grande, più grande, grandissimo, e
  Sbardellato; è però parola bassa, e poco usata; È forse meglio Disorbicante, o Immoderato,
  che suonano lo stesso. L'Autore del Capitolo in lode de' peducci dice.
  \begin{verse}
    Io sto cinque hore del giorno in mercato
    A pascer gli occhi di sì bell'oggerto,
    E ne cavo un piacere sbardellato,
  \end{verse}
\item[FREGOLA] Voglia grande. Onde vuol dire \textit{Entrata in fregola sì fatta} intende
  Essendogli venuta così gran voglia. È traslato dai pesci, che si dice \textit{Andare in
    fregolo}, quando s'adunano molti insieme per la generazione; ed è il latino \textit{libido},
  o \textit{cupido}, E diciamo \textit{In Fregola} I gatti, quando sono in amore. Vedi sotto Cant.
  3. stan. 30.
\item[TOCCAR tamburo] Vuol dir Suonare il tamburo, ma s'intende Arruolare
  Soldati, il che si dice anche \textit{Batter la cassa} Vedi sotto \cstan[]{56}.
\item[A spada tratta] Incessantemente, senza riposo, Senza intermissione,  senza levar mano.
\item[FAR bandiera di ricatto] Ricattarsi, Vendicarsi. Questa voce Ricatto, che
  vien dal verbo Ricatcarsi, il quale vuol propriamente dire Liberarsi di schiavitudine,
  da noi è presa per Vendicarsi, e Far venddetta, ed è il Latino \textit{par pari
    referre}. Il dettato \textit{Far bandiera di ricatto} stimo che venga dal costume dei Corsari,
  li quali, quando pigliano qualche legno, che stimino d'essere in grado da esser
  ricattato, v'inalborano una bandiera bianca, con la quale, danno cenno alle
  Terre vicine se lo vogliono ricattare; il che se voglion fare, corrispondono con
  alzar bardiera dello stesso colore; e questo dicono Metter bandiera di ricatto.
\item[DATO havea lo scacco matto] Le havea fatto questo danno, o cagionata questa
  rovina. Il giuoco delli scacchi è antico, e fu usato prima da i Greci, che
  ora lo dicono \textit{Zatrici}, e poi seguitato da i Latini, che lo dissero \textit{Ludus
    latrunculorum}. A questo giuoco si da fine quando e fatto prigione il Re, e si dice allora
  scacco matto; onde qui vuol dice, che Celidora havea toccato Scaccomatto, havendo
  perduto il suo Regno: E s'allarga quello detto a tutto quello, che ad altri
  succeda di gran perdita, o di grave danno.
\section{Stanza XXVII}

\begin{ottave}
  \flagverse{27}Ov'anco in breve Celidora arriva\\
Con armi in dosso, ed altro da far fette,\\
Perché una volta al fin fattasi viva\\
Ha risoluto far le sue vendette;\\
Che l'usbergo incantato della diva\\
L'ha fatto diventar l'Ammazzasette,\\
Ed alle risse incitala talmente,\\
Ch'ella pizzica poi dell'insolente.
\end{ottave}

Celidora arriva all'armata di Baldone nella Sardigna, e quivi comincia a mostrare
gli effetti della Corazza incantata.
\begin{description}
\item[ARME da far fette] Intende la spada, e vuol dire che era larga, ed abile a
far fette.
\item[FATTASI viva] Rifentitasi, e fattasi ardita., E lo stesso che P7cir di-garra
morta detto sopra in questo Cant. stan. 19.:,
\item[USBERGO] Cioè quella Gran corazza di pelle di drago: detta sopra, la quale il
  Poeta qui dichiara, che ha inteso, \textit{incantata} quando ha detto sopra \textit{imbottita
  d'insulti, e di bravure} alla stan. 20.
\item[AMMAZZA fette] Contano le donne una novella per trattenimento de' Fanciulli;
  e per accomodarsi alla loro capacità, dicono::, Fu una volta un bel giovanetto
  in Garfagnana detto Nanni, il quale per la sua mendicità dormiva in una
  capanna da fieno; quivi essendo egli un giorno per riposarsi, e ripararsi dal caldo,
  si messe a pigliar le mosche, e ne haveva ammazzate sette, quando comparve
  quivi una bella Fata, e gli disse; che se le donava quelle sette mosche per cibare
  una sua passera, l'havrebbe fatto ricco. Gliele concesse egli più che volentieri;
  ond'ella innamorata di questa sua cortese prontezza lo prese per la mano,
  e lo condusse alla sua caverna, dove rivestitolo, e datogli danari, ed armi, gli
  pose in testa un'elmo, o berretta in cui era scritto a lettere d'oro: Ammazzasette;
  e lo mando al Campo de' Pisani, i quali in quel tempo. con l'aiuto de Franzesi
  guerreggiavano co i Fiorentini. Arrivato Nanni a detto Campo, chiese
  soldo a i Pisani, e domandatogli del nome rispose: Io mio chiamo Nanni, e per haver
  io solo in un giorno ammazzato sette, ho per soprannome: \textit{Ammazzasette}. Fu per
  questo, e per esser' anche ben formato, con buon soldo, e con non minore stima
  accettato. Essendo poi fra pochi giorni in una scaramuccia morta il Capo
  delle truppe Franzesi, e volendone essi fare un altro, erano fra di loro in gran
  differenza, perché essendone proposti diversi, coloro, a' quali non piacevano. i
  Soggetti proposti, gridavano Nani, Nani, onde i Soldati Italiani, che credettero,
  che dicessero Nanni, Nanni, e che havessero creato lui: cominciarono a
  gridar Nanni, Nanni; viva Nanni; e così a voce di popolo Nanni detto l'Ammazzasette
  restò eletto capo di dette truppe, e divenne ricco, si come gli haveva,
  promesso la Fata. E di questo intende il Poeta, volendo mostrare, che Celidora
  era divenuta brava, quanto questo Ammazzafette, il quale non fece maggior
  bravura, che ammazzar quelle sette mosche, si come ne anche Celidora
  non fece maggior bravura, che affettar quei Cavoli, che vedremo nell'ottava
  29. seguente.
\item[ALLE risse incitala talmente, ch'ella pizica d'insolente] Bellona le fa venir voglia
  così grande di far risse, che ella vien poi a noia, e si rende odiosa con i suoi
  modi impertinenti. Il verbo \textit{Pizicare} vuol dire Cominciare a essere, o Esseres
  alquanto. \textit{Il tale è stato tanto tempo in Firenze, ch'ei pizica di Fiorentino}, Lo trovo
  anche usato da i Bolognesi in questo senso, e l'usò Francesco Negri\footnote{Giovanni Francesco Negri, Bologna 1593 --- ivi 1659, pittore} nel suo Tasso
  in lingua Bolognese Cant. 1, stan. \makebox[1em]{} dove \begin{verse}El pizigava di sei ann' ch'i Tramuntan\end{verse},
ec. per intendere, Era già presso a sei anni, ec.

\item[INSOLENTE] Si dice colui che dà fastidio, e noia a ognuno, e che si rende
odioso a tutti con le sue azioni impertinenti.
\end{description}
\section{Stanza XXVIII \& XXIX.}

\begin{ottave}
  \flagverse{28}Non così tosto al campo si conduce,\\
Come la suora vuol del Dio Soldato,\\
La Marfisa di nuovo posta il luce,\\
Ch'ell'esce affatto fuor del serminato;\\
E col brando che taglia, com'ei cuce,\\
Da far proprio morire un disperato,\\
Vuol trucidar' ognuno, ognun vuol morto,\\
E guai a quello, che la guarda torto,

\flagverse{29}Se guarda, è dispettosa, e impertinente,\\
 E sempre vuol che sia la sua di sopra;\\
 Talor' affronta per la via la gente\\
 Cercando liti, quasi franchi  l'opra:\\
 Ne venga (dice) pur chi vuol niente,\\
 Però che, chi mi da che far mi sciopra;\\
 Giunta in quest' in un campo pien di cavoli\\
 N' affetto tanti, che Beati Pavoli.
\end{ottave}

Descrive il Poeta una brava spropositata, e impertinente, per mostrare in Celidora
gli effetti dell'incantata Corazza; e con queste azioni, che le fa fare, dipigne
al vivo uno di questi spacconi, e ammazzatori, che noi diciamo che Campano
di fegati d'huomini, e son poi il ritratto della poltroneria, e sfogano la
lor bravura come fa Celidora, in un campo di Cavoli.
\begin{description}
\item[COME la suora vuol del Dio soldato] Come vuol la sorella di Marte, Bellona,
per opra della quale Celidora e capitata a quel campo.
\item[MARFISA] Donna guerriera nota, favoleggiata dall'Ariosto, e però la dice:
  \textit{di nuovo posta in luce}, ed intende una Marfisa moderna fatta brava da Bellona,
  cioè Celidora.
\item[USCIR del seminato affatto] Perder' il senno del tutto, Impazire. Quando altri
  per un grandissimo contento si railegra più del dovuto, diciamo: \textit{Il tale impazisce
    per l'allegrezza}; e così intende di Celidora, non che veramente sia impazita.
  I Latini hanno il verbo \textit{delirare}, che vuol dire Impazire, ed è metaforico dal
  bifolco, sendo composto dalla preposizione \textit{De}, che suona \textit{extra}, \& \textit{lirare},
  che vuol dir Fare i solchi nel campo con l'aratro; e con questo sol verbo
  \textit{delirare} intendono \textit{extra liram incedere}, dove noi diciamo Uicir del seminato, che
  è lo stesso che \textit{extra liram incedere}, o \textit{delirare}, del qual verbo ci ferviamo ancor
  noi nel medesimo senso, come si vede in Dante. Inf.\ C.\ 11.
  \begin{verse}
    Ed egli a me; perché tanto delira
    Hoggi l'ingeguo suo da quel che suole.
  \end{verse}
  E si dice anche deliro uno, che sia fuori del senno, Dan. Par. C. 1.
  \begin{verse}
    Che madre fa sopr' al figliuol deliro,
  \end{verse}

Alcuni vogliono, che.questo verbo \textit{Delirare} venga dal Greco, \textit{Lirin}, che vuol
dir scioccheggiare. Diciamo nel medesimo significato Uscire del seminario, E questo
forse deriva dal Latino \textit{Seminarium}, che secondo Colum, lib, 1. de arboribus
c. 1. 3. vuol dir quel luogo, nel quale si seminano le piante per trapiantarle, il che
quando segue, la pianta cavata dal detto \textit{Seminario} resta come un pesce fuor dell'acqua,
e piantata poi ripiglia il vigore, quando ha cominciato ad attaccarsi nella
nuova terra; e da quello, dicendosi huomo fuori del Seminario, s'intende Huomo
sbalordito. Si dice ancora \textit{fuori del secolo}, e habbiamo \textit{strasecolato}, ed il verbo
\textit{Strasecolare}, Vedi sotto Cant, 6, stan. 36. pur tutto a questo proposito. Ma si questo,
come gli altri suddetti termini, con \textit{tutto che possano credersi l'accennate
derivazioni, io stimo che intanto s'usino in questo proposito, in quanto hanno il
principio della parola, che somiglia quello della parola senno}; e che si dica fuori
del \textit{Seminato}, \textit{Seminario}, o \textit{Secolo} in vece di dire Fuori del \textit{senno}. E questa specie
di parlare, che è specie di parlar furbetto, è molto usato in Firenze per scherzo,
e lo dicono parlare Ianadattico, il qual parlare riesce assai grazioso, quando è maneggiato
da persone spiritose, perché talvolta con parole, che non hanno che
fare con quella materia, della quale si discorre, vien descritta per allusioni, ò
per metafore, ò altrimenti quella tal cosa, della quale si parla. Per esempio: Ad
un Priore, il quale a tre mogli, che haveva havuto, non hebbe mai figliuoli, ed
havea nome Antonio, dicevano \textit{Priapo annebbiato}. Ad un Proposto. che havea
nome Girolamo, ed era lungo, secco, e di colore olivastro, dicevano; \textit{Prosciutto girato}.
Di questo parlar' Ianadattico si serve sotto \cstan[9]{1}.
\item[TAGLIA come ei cuce] Tanto è buono a tagliare, quanto buono a cucire, che
  vuol dir: non taglia. Detto usatissimo per intender Ogni sorte di coltello, o arme,
  o forbice, che per la ruggine, o altro non sieno atte a tagliare.
\item[FAR morire un disperato] Dicono che le ferite fatte con i ferri rugginosi, ò
  intaccati, sieno pericolose di cagionare spasimo, e perciò quando si vede un coltello,
  o arme di tal sorte, si suol dire \textit{Farebbe morire uno disperato}, cioè di dolori eccessivi,
  o di spasimo, E tale era la spada, o brando di Celidora.
\item[GUAI a quello] Male, o gran disgrazia avverrebbe a colui, che la guardasse
torto. E il Latino \textit{Vae illi}.
\item[GUARDA torto] Quand' uno non è molto nostro amico, diciamo: \textit{Il tale non
  mi vede con buon'occhio}; O vero \textit{mi guarda torto}, Che i Latini pure dicono \textit{Non
  rectis aspicere oculis}.
\item[DISPETTOSO] Huomo altero, e che disprezza, ognuno, e d'ogni piccola,
cosa s' adira.
\item[IMPERTINENTE] Uno che vuol più del suo dovere, o del giusto, o più di
quel che gli s'appartiene.
\item[VUOL che la sua, stia sempre di sopra] Vuol sempre haver ragione, che si dice
anche Soprastante. E questi tre modi cioè \textit{Dispettoso}, \textit{Impertinente}, \textit{Soprastante}
si posson dire Sinonimi, e significanti Huomo d'una certa imperiosa arroganza, o
superbia, compagna indivisibile di tutti gli Sgherri, e bravanzoni a credenza.
\item[AFFRONTARE] Vuol propriamente dire Assaltare il nemico, ma si piglia
  ancora per Andar' incontro, o affacciarsi a uno per parlargli, e così è preso nel
  presente luogo, per intendere che Celidora cercava spropositatamente l'occasione
  di far quistione, e tutto per descriverla simile a i detti bravi di parole.
\item[CHI mi da che far mi sciopra] Dovrebbe dire Mi sciopera, secondo che da
  alcuni troppo delicati, e punto considerati ne fu avvertito il Poeta, ma la figura
  Sincope (ammessa fra i Latini) Verg. 5. AEn. dice \textit{gubernaclo} in vece di \textit{gubernaculo}
  da noi è accettata anche nella prosa, ed adoprata comunemente in molte voci,
  particolarmente in questa, dicendosi più pesso \textit{Opra}, \textit{Adoprare}, \textit{Scioprare}, che
  \textit{Opera}, \textit{Adoperare}, e \textit{Scioperare}, lo libera da questa censura. E questo termine \textit{Chi
  mi da che far mi sciopra} è proprio di certi Taglia cantoni, che voglion con esso
  mostrare che chi dà loro occasione di far questione gli \textit{sciopera}, cioè li leva dal
  farne un'altra, che han in mano, e li leva da un lavoro per impiegargli in un'altro simile.
\item[N'AFFETTÒ tanti, che Beati Pavoli] Ne tagliò in fette grandissimo numero.
  Quando vogliamo beffare un bravazzone codardo, sogliamo dire: \textit{Gran
  danno che farebbe costui in un'orto di cavoli, o di raduchi}, E quel detto \textit{Beati Pavoli},
  ha origine da un Montanbanco, il quale vendeva il rimedio contro a' veleni con
  dichiarazione di voler donare (come effettivamente donava) la pietra di S.Paolo
  a tutti coloro, che havevano nome Paolo, onde infiniti plebei per buscar quella
  pietra dicevano di haver nome Paolo; sicché egli cominciò ad esclamare
  O quanti Paoli, o quanti Paoli. E perché quelli, che ottenevano quella pietra
  si tenevano fortunati per haver havuto il regalo, ne nacque il dettato. \textit{Son più
    che non furono i Paoli Beati}, che vuol dire, furon moltissimi; Che la voce \textit{Beati} in
  questo caso è sinonimo della voce \textit{felice}, o fortunato, \textit{Beato voi che siete ricco}, per
  Felice, o Fortunato voi, che siete ricco.
\end{description}
\section{Stanza XXX}
\begin{ottave}
  \flagverse{30}Così piena di fumi, ed umor bravi\\
 Che te l'hanno cavata di Calende,\\
 Rivolge l'occhio al popol delle navi,\\
 Là dove Brescia romoreggia, e splende,\\
 E va per infilzarne sette ottavi:\\
 Ma nel pensar di poi, che se gli offende\\
 Far non porrebbe lor, se non mal giuoco;\\
 Gli vuol lasciar campare un'altro poco,
\end{ottave}

 Celidora facendo queste sue bizzarrie, vede la gente di Baldone, ed essendosi
inferocita in quei cavoli, gli vien voglia di far io stesso in quelle genti, ma si
rattien di farlo per non dar loro disgusto, e per lasciargli campare un'altro poco.
\begin{description}
\item[PIENA di fumi, che te l' hanno cavata di Calende] Mostra il Poeta, che
  Celidora sia poco meno, che briaca in questa sua bravura, i fumi della quale le
  habbiano offuscato il cervello, come fanno i fumi del vino a chi troppo beve, che
  questo intende dicendo l'hanno \textit{cavata di calende}, ed è quelio che i Latini dicono
  \textit{extra callem esse}, ed io credo che da questo Latino \textit{callem} venga la corruttela di calende; e per parlare Ianadattico detto sopra in \cstan{28}. si voglia dir \textit{cavata
  del calle} per intendere (come facevano i latini) Cavata di Cervello.
\item[BRESCIA romoreggia, e splende] Si sente romor d'armi, e si vedono risplender
  le medesime. A Brescia si fabbricano buone, e belle armi, e però il Poeta
  pigliando La Città per L'armi, che in quella si fabbricano, seguita l'uso nostro, che
  è di dire \textit{Il tale ha tutto Brescia addosso}, per intendere \textit{Ha molt'armi addosso}.
\end{description}

\section{Stanza XXXI \& XXXII}
\begin{ottave}
\flagverse{31}Al fin, deposto un'animo sì fiero,\\
In genio cangia a poco a poco l'ira,\\
E' come un'orsacchin, c'a pié d'un pero\\
A bocca aperta i pomi suoi rimira;\\
Ferma impalata quivi com' un cero\\
Fissando in loro il sguardo, sviene, e spira,\\
Ne può viver al fin se non domanda\\
Ove l'armata vada, e chi comanda.

\flagverse{32}S'abbocca appunto con Baldone steffo,\\
E sentendo ch'  egli ha tal gente fatte\\
Per rimeiter in sesto, ed in possesso\\
Una Cugina sua ch'è per le fratte,\\
Ben ben lo squadra, e dice: Egli è pur desso!\\
Or su ch'io casco in piè, come le gatte,\\
Ed esclama di poi: quest'è un'azione,\\
Che veramente è degna di Baldone.
\end{ottave}

Celidora pero appiacevolitasi, si ferma a guardar con gusto grandissimo quei
Soldati, e domanda di chi è l'Armata, e chi la comanda; e s'abbatte a domandarlo
a Baldone, il quale gli dice, che ha fatto quella gente per aiutare una sua
cugina, ond'ella riconosciuto Baldone, si rallegra, e dice: veramente questa è
un'azione degna di Baldone.
\begin{description}
\item[CANGIA l'ira in genio] Cioè dove prima haveva l'animo d'infilarne sett'ottavi,
  adesso comincia ad haver genio con loro, ed a portargli affetto. Questa
  voce genio se ben non pare che Toscanamente significhi cosa alcuna, nondimeno
  è molto usata dicendosi \textit{Huomo di buon genio}, o \textit{di cattivo genio} per intendere Huomo
  di buona, o cattiva indole, o inclinazione. \textit{Haver genio con uno} È lo stesso
  che Haver simpatia con uno. Appresso i Latini pure se ben genio non si distingue,
  va dall'anima ragionevole, e molti lo pigliassero spesso per Lares; altri per gli
  Dei Penati, altri per il Dio del piacere, altri per li quattro elementi, altri per li
  dodici segni del Zodiaco, altri per lo Dio che faceva nascere,  ed altri per diverse
  altre cose; tuttavia essi pure se ne servivano per intendere inclinazione, come
  ci mostra Plauto in Truculento 1, 2. \begin{verse}cum genijs suis belligerare, ec. idem quod defraudare genium.\end{verse}
\item[COME un'orsacchino a piè d'un pero] Si dice L' orso sogna pere; Leva le peres
ecco l' orso, Dal-che si cava, che questo animale sia molto ghiotto delle pere; il
be anche attesta Vincenzo Martelli nel suo Capitolo in lode delle menzognes
jicendo: \begin{verse}
  Oggi a voi più ch' ad altri si conviene,
  Benché noi siam tant' orsi a queste pere, ec.
\end{verse}
E si dice che in rimirarle gioisca tutto per la sola speranza di conseguirle; e
perciò l'Autore assomiglia Celidora a un picciolo Orso a pie d'un pero, perché in
veder quella gente, la quale ella spera che sia per lei, si rallegra, gode, e brilla,
come fa l'orso stando a pié del pero, vagheggiando le pere.
\item[FERMA impalata quivi come un cero] Per esprimere la stpidità nella quale si
trova Celidora nel vedere quei Soldati, l'Autore dopo haver detto che \textit{stava a
bocca aperta come fra l'orso a pié del pero}, soggiunge \textit{che ella stava impalata, come un
cero}, cioè ritta ritta, e fermata nel posto, come stavano quelle torrette, fatte di
carta, o di panno, o di tavole, che la mattina di S. Gio.\ mettevano li nostri
antichi attorno alla piazza del Tempio di S. Gio. Batista, entro alle quali stava
un'huomo, che le moveva, e queste le domandavano \textit{ceri} secondo che dice Goro
Dati\footnote{Gregorio Dati, 1362-1435, mercante fiorentino.} nei suoi discorsi Storici lib. 6. in fine. Hoggi in vece di tali torrette portano
in due, dello Spedale del Bigallo; sopr' alle spalle processionalmente, uno sgabellone,
sopr' al quale è fermato un gran cero fatto di legno, per sfuggire il pericolo
di romperlo sendo di cera, e faranno 26. o vero trenta ceri, che manda
detto Spedale per tributo al detto Tempio di S. Gio. Batista. Si può anche dedurre
questa similitudine da quei poveri Cristiani, i quali da i Turchi sono impalati,
che verisimilmente stanno intirizzati, e come l'Autore vuol che s'intenda,
che stesse Celidora.
\item[SVIENE, e spira] Svenire vuol dir Perdere i sentimenti, e Spirare vuol dire
Esalar l'anima, sicché si possono dir quasi sinonimi, ma in questo luogo il verbo
\textit{spirare} significa \textit{Ustolare}, che vuol dir Guardare con desiderio di conseguire,
come fa uno che havendo grandissima fame, stia a vedere un che mangi, ed habbia
d'avanti molte vivande; Vedi sotto \cstan[14]{34}.
\item[ABBOCCARSI] Trovarsi, o abbattersi in uno per parlargli. \textit{Io non son ben'
informato di questo negozio, ma m'abboccherò col tale, che m'informerà}.
\item[E' per le fratte] È rovinato. È per la mala. Quello che i latini dissero \textit{De
eo actum est}. \textit{Fratta}. S'intende Borroncello, o Macchia, che suol render' aspro
un paese, e vien dal Greco Frattin che suona Far siepe.

\item[BEN ben lo squadra] Lo guarda benissimo, che la forza della replica è di far
  nascere il superlativo, come accennammo sopra in \cstan{11}. Ed il verbo
  squadrare, che vuol dir Misurar con la squadra, significa Considerare, e
  Guardare un' oggetto minutamente, e con diligenza.
\item[CASCARE in pie come i gatti] Ottener da un male, o da un cattivo accidente,
  un bene impensato; che i latini dissero \textit{excidere extra mala},
\end{description}

\section{Stanza XXXIII \& XXXIV}
\begin{ottave}
  \flagverse{33}Maravigliato allora il Sir d'Ugnano,\\
  E chi sei (disse) tu, che sai il mio nome?\\
  Io ti conosco già di lunga mano,\\
  (Ella rispose) e acciò tu sappia il come,\\
  Celidora son'io del Re Fioriano\\
  Fratello d'Amadigi di Belpome,\\
  E con tutto, che già sien' anni Domini\\
  Ch'io non ti viddi, so come ti nomini.

  \flagverse{34}S'ell'è (dic' ei) così noi siam cugini,\\
  E subito si fan cento accoglienze,\\
  Ed ella a lui ne vende mill' inchini,\\
  Egli altrettante a lei fa riverenze,\\
  Così fanno talor due fantoccini\\
  Al suon di cornamusa per Firenze,\\
  Che luna incontro all'altro andar si vede\\
  Mosso da un fil, che tien, chi suona, al piede.
\end{ottave}

Baldone, e Celidora si riconoscono per cugini, e si fanno molte accoglienze.
\begin{description}
\item[CONOSCER di lunga mano] Conoscer di gran tempo. \textit{Lunga mano d'anni}
tanto suona quanto Lunga serie d'anni, o gran quantità d'anni, che diciamo
anche \textit{È un gran pezzo ch'io ti conosco}.
\item[BALDONE, Celidora, e Amadigi] sono nomi a caso:, ma l'\textit{Infante Floriano} è
anagrammatico, da \textit{Raffaello Fantoni}.
\item[SON' anni Domini] Son' anni infiniti. Sono tanti anni, quanti sono dalla nascita
  di Nostro Signore che diciamo Anno Domini. iperbole usatisima in Firenze.
\item[ACCOGLIENZA] Ricevimento con amorevolezza, e cortesia, e con una
  certa dimostrazione d'affetto, che s'usa verso le persone grate. Vien dal Latino
  \textit{Colere}, che esprime Amar con riverenza, ed honore.
\item[INCHINO] È lo stesso che \textit{riverenza} facendosi con abbassar la testa, e piegare
  le ginocchia, ed è proprio delle Donne; \textit{Riverenza} si fa con abbassar la testa,
  e piegandosi un sol ginocchio si manda l'altra gamba addietro a foggia di
  genuflessione, ed è propria degli huomini, come si vede nel presente luogo, che
dice,\begin{verse}
Ella a lui ne rende mille inchini;
Egli altrettante a lei fa riverenze,
\end{verse}
\item[COSÌ fanno talor due fantoccini] Suol' andar per Firenze un contadind, suonando
  una cornamusa, e porta alcune figurine di legno, che hanno le congiunture
  delle membra mastiettate, e contrappesate con piombo in modo, che si
  muovono per ogni verso; queste infilza per lo petto in una sottilissima corda da
  chitarra, o diciamo minugia, la quale da una parte lega ad uno de' suoi ginocchi,
  e dall'altra ad una tavoletta posta in terra a tal fine, e col muovere quella gamba,
  alla quale è legata la corda; fa, che quelle due figurine infilzatevi ballano al tempo
  del suono della cornamusa. Intesa dunque questa operazione, che fanno i due figurini,
  s'intende ancora come facessero fra di loro questi due parenti.
\item[CORNAMUSA] Zampogna doppia, composta d'un basso perpetuo, e di un
  soprano, che canta le note come gli altri Zufoli, e si da il fiato ad ambedue con
  un sacco di quoio, da colui che suona, ripieno di vento: col soffiare in un piccolo
  cannello animellato; ed il suonatore premendo col braccio il detto sacco da il
  fiato a dette due Zampogne.
\end{description}

\section{Stanza XXXV}
\begin{ottave}
  \flagverse{35}Poi che le fratellanze, e i complimenti\\
Furon finiti, a lei fece Baldone\\
Quivi portar un po di sciacquadenti,\\
O volete chiamarla colezione,\\
Hor mentre, ch' ella scuffia a due palmenti\\
Pigliando un pan di sedici a boccone,\\
Si muove il campo, e sott'alla sua insegna\\
Ciascun passa per ordine a rassegna.
\end{ottave}

Dopo finite le cirimonie Baldone fa portar da bere, e da mangiare, e mentre
che Celidora mangia, si fa la mostra de' Soldati.
\begin{description}
\item[FAR le fratellanze] È tratto dall'uso che nelle nostre Compagnie, ò Confraternite
  di secolari, nelle quali a i tempi determinati si vanno tutti ad abbracciare
  l'uno con l'altro; e questa azione dicono \textit{Far le fratellanze}, E da questo
dunque intendi dopo finiti gli abbracciamenti e le cirimonie.
\item[SCIACQVADENTI] Quel che significhi lo dichiara il Poeta medesimo dicendo;
  \textit{O volete chiamarla colazione}. Che vuol dire parcamente cibarsi fuor del desinare,
  e della cena, e viene dal Latino \textit{collectio prandij vel coenae}. Ma siccome son
  diversi li pasti che si fanno in Firenze, così son diversi li nomi che loro danno. Il
  primo mangiare che si fa fra l'alba, e il mezzo giorno si chiama \textit{Asciolvere}, ed
  alle volte colazione. Quello, che si fa a mezzo giorno si chiama \textit{desinare}. Quello
  che si fa tra 'l mezzo giorno, e la sera si dice \textit{Merenda} quali \textit{meridie edenda}.
  Quello della sera si dice cena, ed allora che per il digiuno la sera si mangia poco
  si dice colazione; E la voce \textit{sciacquadenti} vuol veramente dire Quando si mangia
  qualche poco, per bere con gusto.
\item[SCUFFIARE] Mangiar con ingordigia, o divorare. È voce Fiorentina, ma
hoggi usata solo per scherzo, e vien forse da \textit{Scuffina} che è una raspa, o lima da
legno detta così, perché adoprandola leva molto legno per volta, e per questo è
chiamata anche \textit{ingordina}.
\item[A due palmenti] Da ambedue ganasce: Traslato dal Molino, che si dice
  \textit{Macinare a due palmenti} quando Due ruote lavorano; che \textit{palmento} vuol dire
  tutta la macchina, che fa macinare, dicendosi Molino \textit{d'un palmento}, o di \textit{due
    palmenti}, quando Un molino ha una, o due macini. E stimo che si dica \textit{Palmento},
  quasi Palamento, perché le ruote, che fanno andar le macine son composte di
  tavole a foggia di pale per prender l'acqua, che le fa girare.
\item[UN pan di sedici, ec] Con questa iperbole esprime l'ingordigia di Celidora;
  perché per altro un pane di sedici de' nostri quattrini malamente si può consumare
  anche con sedici bocconi, intendendo \textit{Boccone} quella quantità, che l'huomo
  può pigliar dentro alla bocca in una volta.
\item[PASSAR a rassegna] Quando i Soldati si portano avanti, al loro Capitano,
  e fanno scrivere il lor nome si dice \textit{Passar a rassegna}. E qui Baldone come supremo
  Capitano per fare honore alla cugina, Fa la rassegna, nominando, però solamente
  gli Ufiziali prinicipali; il che pare che più propriamente si dica \textit{Dare}, o
  \textit{far la mostra}, Vedi sotto \cstan[2]{36}.
\end{description}

\section{Stanza XXXVI}

\begin{ottave}
  \flagverse{36}E per il primo viensene in campagna\\
Pappolone il Marchese di Gubbiano,\\
Colui, che nel conflitto della Magna\\
Estinse il Gallo, e seppellì il Germano;\\
È la sua schiera numerosa, e magna,\\
E perch'egli è Soldato veterano,\\
Ha nell'insegna una tagliente spada,\\
Ch'è in pegno all'osteria di mezza strada
\end{ottave}

L'Autore in questa sua Opera mette una mano d'amici suoi sotto nomi anagrammatici,
la maggior parte de' quali è nominata in questa mostra, che Baldone
fa dell'esercito, descrivendone alcuni con qualche loro azione, ò con un'epilogo
della loro vita oltre all'Anagramma. Il primo che viene in mostra e Pappolone,
cioè \textit{Paolo Pepi} anagramma proprio, perché questo gentilhuomo era giovanotto
grande di persona, e grasso, e mangiava assai; e per questo il Poeta lo
dice \textit{Pappolone}, che vuol dir gran mangiatore. Vedi sotto \cstan[6]{70}.,  e lo fa
\textit{Marchese di Gubbiano}, che è un Castello; e Ingubbiare (detto però plebeo) significa
Empier il ventre. Dice \textit{nel conflitto della Magna}, cioè Nel mangiare, se ben
par che voglia dire in una sanguinosa battaglia seguita in Alemagna. Estinse il
Gallo, e seppellì il Germano; par che dica ammazò Francesi, e Tedeschi, ma vuol
dire ch'ei mangiò galli, e germani; e gli fa fare per insegna una spada impegnata
all'oste di mezza strada, che è un'osteria fuor di Firenze un miglio, e così
mostra, che ogni fine di questo tale era il mangiare.
\section{Stanza XXXVII}
\begin{ottave}
  \flagverse{37}Bieco de Crepi Duca d'Orbatello\\
Mena il suo terzo c'ha il veder nel tatto,\\
Cioè perch'ei da un occhio sta a sportello,\\
Soldati ha preso c'hanno chiuso afatto,\\
Son l'armi loro, il bossolo,e il randello,\\
Non tiran paga, reggonsi d'accatto,\\
Soffiano, son di calca, e borsaiuoli,\\
E nimici mortal de' muricciuoli.
\end{ottave}

Segue dopo Pappolone \textit{Bieco de Crepi}, cioò Piero de Becci huomo di faccia non
troppo bella, con occhi biechi, e lusco, e però il Poeta con l'equivoco \textit{d'orbo},
che vuol dir mezzo cieco, come vedemmo sopra in questo Cant. stanza 9., lo fa
\textit{Duca d'Orbatello}, e dice, che vedendo egli alquanto, ha preso per Soldati gente,
che è affatto cieca, avverando il detto. \textit{Beati Monoculi in terra caecorum}. Hanno
questi soldati il bossolo, e il bastone, non tirano paga, ma vivono di limosine,
son tutti spie, ladri, monelli, e nimici de' muricciuoli.
\item[UN terzo] Numero di soldati comandati da pil capitani, e dal Colonnello;che
i Latini dicevano /egionem, ed il Colonnello forse era Tribunus,
\item[MENARE] Condurre, Ma qui sta proprio il verbo Menare secondo il pro-
verbio che dice: Solo tciechi si menano, —~;

\item[HA il veder nel tatto] U ciechi non hanno altra vista, che il tatto, el odorato
nelle cose corporee, e materiali; e 1' udito nell' incorporee.

\item[STA a spertello] Intende mezzo cieco. Metafora tolta da quelle botteghe; le
gualt quando non è festa intera, e comandata stanno mezze aperte, che si dices
Star' a sported, perché aprono folo quella parte del legname, che si chiama se
tello; e seguita la metafora dicendo: Su/dati ha preso channo.chiufo affatto: cioè s0-
no affatto ciechi. Varchi Stor. Fior. lib.~11. dice: Won si tennero le botteghe Aperte,
ne a sportello, ma chiuse affatto.

\item[BOSSOLO] E' quel valoa foggia di calice, col quale si raccolgono i voti ne-
gli Squittini. Vedi sotto Cant.\ 6. stan.\ 109., e per la similivudine intendiamo quel
valo di latta, di rame, d' ottone, o d' altra materia, che è usato da i ciechi per
ricevervi l'clemofine, ay

\item[RANDELLO] Intende Quel bastone, che adoprano i ciechi per farfila strada.
  Se ben randello s'mmtende un Pezzo ci bastone grosso quanto quello de' ciechi,
  ma assai pil corto, che s' adopra per stringere le legature delle baile, che però
  tale operazione si dice \textit{Arrandellare}.

\item[REGGONSI d'accatto] 1 verbo Reggersi in questo laogo, ed in questi termini
vuol dir Cavar il guadagno per mantenersi: M tale si regge col far' il farto, Cive
vive col guadagno, che cava dal far' il farto, ec. 4

\item[SOFFIARE] In lingua furbesca vuol dir Far la spia, se bene è inteso comunemente.
  Ed il Poeta parlando di ciechi, i quali hanno per costume di parlar furbesco,
  e serve di questa, ed altre lor parole, come \textit{Esser di calca}, che vuol dir
  Huomo da far qualsivoglia furfanteria, e viene dalla voce \textit{Calcagno}, che in
  lingua furbesca vuol dir \textit{Monello}, cioè ladro di calca nella quale entrano per rubar
  le borse, e di qui si dicono Borsaioli, e Taglia borse. Vedi sotto \cstan[6]{64}.

\item[NIMICI de' muricciuoli] Chiamiamo muricciuoli quel pezzo di muro, che avanza
  sopr'a terra attorno alle case; d'altezza d'un braccio più, o meno, e di
  simile larghezza; fatto, o per uso di sedere, o per difesa de i fondamenti. Di
  questi sono nimici i ciechi, perché spesso vi Pp jotono dentro co' i piedi, ingannati
  dal sentir al viso, ed alle mani l'aria libera, il che fa lor credere, che non
  possa esservi impedimento veruno anche in terra.
\end{description}
\section{Stanza XXXVIII}
\begin{ottave}
  \flagverse{38}La strada i più si fanno col bastone,\\
Altri la guida segue d'un suo cane,\\
Chi canta a più d'un'uscio un'Orazione,\\
E fa scorci di bocca, e voci strane;\\
Chi suona il ribecchin, chi il colascione;\\
Così tutti si van buscando il pane.\\
Han per insegna il diavol de' Tarocchi,\\
Che vuol tentar un forno pien di gnocchi.
\end{ottave}

Descrive il modo del marciare di questi ciechi, e fa lor fare quei gesti, ed operazioni,
che son soliti fare andando a cercare elemosine, Dice che \textit{I più si fanno
la strada col bastone; altri si fanno guidare a un cane, ed altri vanno cantando Orazioni
a pié d'un'uscio}; E questi son ciechi stipendiati dalle persone pie, acciocché ogni
giorno, o ogni settimana vadano alle case delle medesime persone a cantare un'orazione
avanti al loro uscio, dove per esser sentiti fanno voci strane, cioè Gridano
forte, e fanno \textit{brutti scorci di bocca}; E questo avvien loro perché, per lo più,
li ciechi oltre alla loro cecità, sogliono havere altri stroppi nella faccia. Molti
suonano il ribechino, cioè il violino, altri il \textit{Colascione}: questo strumento (che da
i più è detto corrottamente \textit{Ganascione}) E' un corpo, come quello della tiorba,
con manico lungo, con due sole corde, il quale si suona con un pezzo di suolo
da scarpa, che volgarmente si dice taccone; E perciò tale strumento è detto anche
Tiorba a Taccone da Filippo Scrutendio da Scafato\footnote{Felippo Sgruttendio de Scafato. Ignoto, forse anagramma di persona reale, in vita nel 1646.}, il quale così intitola il
suo grazioso Canzoniero Napoletano. Alcuni furbi per \textit{colascione} intendono la
forca, perché ancora a questo s'adoprano due corde, la grossa, e la sottile, come
alla forca. Questi ciechi suonatori soglion sempre andar vendendo qualche
Orazione, o Rappresentazione, o altre Leggende, e così tutti si vanno buscando
il pane, cioè guadagnano da vivere. E volendo il Poeta mostrare quanto la gente
di questo terzo sia affamata, le da per insegna un diavolo, che tenta un forno
pieno di gnocchi; e mostra che sia sempre intenta a procacciarsi il vitto con ogni
sorta d'invenzione, che il verbo tentare significa Procurare, o Provarsi di far
una tal cosa, e si deduce, che questo diavolo \textit{tentasse}, cioè si provasse a rubar da
quel forno il pane, che vi era dentro. E per \textit{gnocco} intende Ogni sorte di pane;
Se bene \textit{gnocco} è quella specie di pane, che dicemmo sopra in \cstan{3}.
\begin{description}
\item[SCORCI di bocca, e voci strane] Voci strane, e bocche diverse dal naturale;
perché se bene la voce \textit{scorcio} è termine di prospettiva, che mostra la figura esser
resa capace della terza dimensione del corpo; s'intende anche per positura di corpo,
o parte d'esso diversa dal naturale.
\item[TAROCCHI] Carte, con le quali si giuoca alle Minchiate. Vedi sotto C.8. stan.
  61. in una delle quali carte al num. 14. è effigiato un Diavolo; e questo dice, che
  \textit{tenta il forno pien di gnocchi}. Il nostro Poeta haveva dato a questi Ciechi l'impresa
  del Buio, come si vede in alcuni suoi sbozi, che diceva.
  \begin{verse}
    Hanno un' impresa, dove Bieco mette
    Il buio che a svegliar va le Civette.
  \end{verse}
\end{description}

\section{Stanza XXXIX, XXXX, XXXXI}

\begin{ottave}
  \flagverse{39}Dietro al Duca, c'ognun guarda a traverso\\
Vanno cantando l'aria di Scappino,\\
Ma non giunsero al fin del terzo verso,\\
Che venuto alla donna il moscherino,\\
Fatto a Bieco un rabbuffo a modo, e verso,\\
Gli disse: S'io v'alloggio dimmi Nino,\\
Perch'io non veddi mai in vita mia\\
Pigliar i ciechi fuor c'all'osteria
\end{ottave}

\begin{ottave}
  \flagverse{40}Signora, rispos'egli, benché cieca,\\
Fu però sempre simil gente sgherra;\\
Con quel batocchio zomba a moscacieca\\
Senza riguardo, come dar' in terra;\\
Sort'ogni colpo intrepida s'arreca,\\
Che non vede i perigli della guerra:\\
E' cieca è ver, ma pur il pan pepato\\
E' più forte, se d'occhi egli è privato,

  \flagverse{41}Ovvia (diss'ella) tocca innanzi il cocchio,\\
E se costoro a guerreggiar son'atti\\
Tienteli pure, e non mi star' a crocchio,\\
Mentre gli è tempo qui di far di fatti.\\
Va dunque o forte, e invitto bercilocchio,\\
Che i nimici da te saran disfatti,\\
Perch' in veder la tua bella figura\\
Cascan morti, senz'altro, di paura.
\end{ottave}

Questi ciechi andavano dietro a Bieco cantando l'aria di Scappino, (che è una
canzonetta, la quale cantavano i ciechi in Piazza del G. Duca, quando l'Autore
principiò la presente opera) ma Celidora adirata di ciò, dice a Bieco, che
non vuol tal gente, ed egli rispose, che se bene eran ciechi eran però fieri, che
il non vedere i pericoli gli rendeva arditi, e forti, come appunto è il pan pepato,
che è più forte, quando non ha occhi; ond'ella gli dice, che se gli tenga, e vada
allegramente, che ella ha speranza di cavar frutto da lui solo senza loro, perché
stima, che il nimico sia per cascar, morto subito, che vedrà il suo brutto viso.
\begin{description}
\item[GVARDA a traverso] Uno che ha gli occhi scompagnati, come haveva Bieco
  diciamo Guardare a traverso. Vedi sopra in questo Cant. stan. 9. \textit{Transversa tuentibus
  hirquis}, Virg. Egl. 3.
\item[VENUTO alla donna il moscherino] La donna, cioè Celidora, s'adirò. Si dice
  \textit{Venire il moscherino al naso}, perché si trovano alcune piccole mosche, le quali
  volando, talvolta entrano nel naso altrui, e toccando quella parte così sensitiva,
  danno grande alterazione, e mettono l'huomo in una subita impazzienza, e stizza.
  Si dice ancora \textit{Venir la senapa}, o \textit{la Mostarda al naso}, perché nel mangiar la
  mostarda (che è un'intingolo fatto di senapa, e mosto cotto) quando è ben carica di senapa,
  viene al naso un certo pizicore, che forza a, lagrimare. Si dice
  anche \textit{Venir la muffa}, o altri puzi odiosi, e sporchi, come si dice sotto C. 4. stan.
  23. E tutti significano Venir collera.
\item[FATTO un rabbuffo] Bravato. Fare un rabbuffo, o Rabbuffare vuol dire Riprender
  uno con minacce, o Spaventarlo con asprezza di parole. Il Landino nell'esposizione
  a Dante C. 7. dell'Inferno alla parola Buffa, e Rabbuffare dice: \textit{Ma
  proprio Buffa è vento, onde diciamo Buffettare chi getta vento, per bocca,e Sbuffare, quando
  con suono di parole, o a dir meglio Con ventose, ed enfiate parole alcuno minaccia.
  Di qui diciamo Rabbuffare, Conturbare e muover le cose dell'ordine loro, e scompigliarle
  e chiamiamo Rabbuffo, quando Con parole conturbiamo, e Scompigliamo la mente d' uno}.
  Vedi sotto \cstan[3]{57}, la voce \textit{Buffi}.
\item[A modo, e a verso] Con tutta perfezione. B il latino \textit{modis, \& formis}.
\item[DIMMI Nino] Dimmi pazzo, e senza Cervello, come fu Nino, il quale per lo
  grande amore, che portava a Semiramide sua Meretrice o moglie, le concesse
  che per un giorno ella fusse assoluta Regina, ed ella in quel giorno lo fece ammazzare,
  e si confermò Regina per sempre, come si legge in Plutarco in Serm.
  Amator.
\item[PIGLIAR i ciechi fuor c'all'osteria] Quand' uno vince assai, sogliamo dirgli: \textit{Si
  torrà i ciechi}, e s'intende \textit{all'osteria}. E questo perché si suppone, che quel
  tale, che vince per l'abbondanza del denaro venutogli in mano fenza
  fatica, sia per spenderlo profusamente in pigliarsi tutti li suoi gusti fino con
  l'andare a cena all'osteria, e chiamare alla sua mensa a suonare alcuni ciechi, i
  quali in su l'hora del mangiare vanno girando, per l'osterie a tale effetto, e questi
  sono i Ciechi, li quali Celidora dice haver veduto pigliare all'osterie.
\item[SGHERRO] Bravo. Ammazzatore; Tagliacantoni. Vedi sotto, Cant. 3. stan. 42.
\item[BATOCCHIO] Quel bastone, col quale si fanno la strada i ciechi si chiama
\textit{Batocchio} dal batterlo in terra, che fanno i ciechi, per farsi riconoscere per quel
battere da gli altri ciechi. E però vuol dire anche il Battaglio delle Campane.
\item[ZOMBA] Perquote, bastona. Vedi sotto C.~6. stan.~104., e C.~11. stan.~28.
\item[MOSCA cieca] Il giuoco detto Mosca cieca è trattenimento da Fanciulli, che
  deriva dall'antico, e si diceva \textit{Musca aenea}, e si faceva nel modo, che usano
  hoggi, che è in questa maniera.

  Tirano le sorti fra più ragazzi a chi debba bendarsi gli occhi, (che in questo
  giuoco dicono Star sotto) ed a quello, a cui tocca, sono bendati gli occhi in modo,
  che non possa vedere, e poi con uno sciugatoio, o altro panno avvolto, che ciascuno
  tiene in mano, si danno da gli altri delle percosse a colui, che è sotto, ed egli
  così alla cieca va rivoltandosi, e quello che egli arriva con la percossa deve bendarsi
  in vece del, percuziente, il quale si leva la benda, e va fra gli altri a percuotere
  il nuovo bendato; Quello, al quale di mano in mano tocca a star sotto, mena
  senza riguardo, colpi spietati, sì perché commosso da tanti colpi vorrebbe
  vendicarsi, sì anche perché, cogliendo, il colpo sia in modo da non poter'
  esser negato, procurando ognuno di non toccarne, e d'occultarla, se può,
  quando l'ha toccata, per non haver' a stare in quel martirio, in che è colui, che
  sta sotto. E però dice \textit{Zomba a mosca cieca senza riguardo come dare in terra}.
  Si dice \textit{mazzate da ciechi} per intendere Percosse spietate.

\item[IL Pan pepato è più forte se d'occhi egli è privato] Si suole in Firenze per la sesta
  di tutti i Santi fare un certo pane che da noi si dice \textit{Pan pepato}, il quale è
  composto di sapa, aceto, farina, pepe, ed altri aromati, e mescolanvi pezzetti di
  bucce di poponi, zucche, cedri, e d'aranci conditi in zucchero, o miele, li quali
  pezzetti, quando il pane si taglia, restano nella tagliatura a similitudine d'occhi,
  e perciò da i nostri Fanciulli son chiamati Occhi; E cavandosi dal pane tali
  occhi, che sono dolci, il pane resta \textit{più forte}, cioè più acido; ed il Poeta si serve
  della parola Forte in significato di Gagliardo, dicendo che i ciechi sendo senz'occhi
  son più forti, ed intende gagliardi, scherzando con questo equivoco di forte.
\item[TIRA innanzi il cocchio] Seguita il tuo viaggio, e tanto s'intenderebbe a dir
solamente \textit{tira innanzi} senza porvi l'aggiuata \textit{Cocchio}, ma il Poeta ve lo pone
per seguitar l'uso Fiorentino.
\item[STAR a crocchio] Il verbo \textit{Crocchiare}, e la frase \textit{stare a crocchio}
  significano Cicalare, o Ciarlare di cosa di poco frutto, o importanza per finire il giorno.
  Onde questi tali si dicono \textit{Crocchioni}, \textit{Cicaloni}, \textit{Perdigiorni}, e simili.
  Vedi sotto Cant. 3. stan. 5. Questo verbo \textit{Crocchiare} serve anche per intendere Dar
  delle buffe. Vedi sopra in questo Cant. stan. 10.
\item[BERCILOCCHIO] Epiteto composto dal Poeta, che vuol dir Bircio di che
sopra in questo Cant. stan. 9.
\end{description}
\section{Stanza XXXXII \& XXXXIII}
\begin{ottave}
  \flagverse{42}Ne Segue intanto Romolo Carmari
Cavalier di valore, e di gran fama;
Ma sfortunato, perché coi danari
Giuocando egli ha perduta anco la dama.
Con le pillole date a suoi erarj
L'affetto evacuò l'Arpia ch'egli ama.
Tal che senz'un quattrino ammartellato
Alla guerra ne va per disperato.

  \flagverse{43}Dop'un'insegna nera che v'è drento,
Cupido morto con i suoi piagnoni
Marciar si vede un grosso Reggimento,
Ch'egli ha d'innumerabili tritoni,
Al cui arrivo ugnun per lo spavento
Si rincantuccia, ed empiesi i calzoni,
E da lontano infin dugento leghe
S'addoppiano i ferrami alle botteghe.
\end{ottave}

Segue \textit{Romolo Carmari}, Questo fu un Fiorentino, del quale non stimo bene scioglier
l'anagrammma, e dirne il nome. Questo Gentilhuomo havendo durato un
gran tempo a godere una sua Meretrice, e spesovi molto danaro, o gli fu tolta,
o ella non lo volle più perché egli abbandonò lo spendere; come è proprio di
simili donne; e ciò esprime il Poeta in quei due versi.
\begin{verse}
Con le pillole date a suoi erarj
L'affetto evacuò l'Arpia ch'egli ama.
\end{verse}
I quali versi suonano: L'havergli fatta votar la borsa fece disperdere l'amore,
che ella fingeva di portargli, Onde egli disperato, se ne va alla guerra; e
mostra questo suo spento amore nell'insegna, che egli porta, in cui è dipinto
Cupido morto, che ha d'attorno i suoi piagnoni. E perché questo Signore era
nel vestire positivo, e senza boria alcuna, anzi più tosto abbietto, il Poeta fa,
che egli conduca un reggimento di gente mal vestita, e questi huomini chiama
\textit{Tritoni}, perché Huomo trito, o Tritone tanto vale appresso di noi quanto dire
Huomo mal vestito; E questa gente per esser così mal vestita e stimata una schiera
di Monelli, e di Ladri, e perciò è causa, che s'accrescano i serrami alle botteghe,
e che ognuno fugga per la paura, che ha di loro.
\begin{description}
\item[DAMA] Vuol dir Donna nobile, venendo dal Greco \textit{Damar}, secondo alcuni;
e suona Signora dal Francese Dame, Madame, cioè Signora, mia Signora; ma
si piglia anche per l'amata, come è preso nel presente luogo.
\item[CON le pillole date a suoi erarj] Con l'evacuatorio dato alla sua borsa, cioè con
avergli fatti finire i danari mandò via dal suo corpo la bile amorosa, cioè lasciò
d'amarlo.
\item[L'Arpia] Intende Meretrice, ed esprime una donna rapace, come sono le
  Meretrici (che Arpia in Greco suona come Rapace) e quali sono figurate
  Arpie, che i Poeti fingono esser tre, Aello, Ocipete, e Celeno; e le
  fanno figlie di Nettunno, e della Terra; altri figlie di Thaumante, ed
  Elettra, altri d'altre Deità; basta che se ne servivano per esprimer l'avarizia.
  Vergil. 3. AEn.
  \begin{verse}
Tristius haud illis monstrum, nec sævior ulla
Pestis et ira deum Stygiis sese extulit undis,
Virginei volucrum vultus, foedissima ventris
Proluvies, uncaeque manus, et pallida semper
Ora fame.
  \end{verse}
  E Dante nell'Inf. Cant, 13. seguitando Vergilio dice
  \begin{verse}
    \backspace Quivi le brutte Arpie lor nido fanno,
    Che cacciar dalle Strofade i Troiani
    Con tristo annunzio di futuro danno.
    \backspace Spalle hanno alate, colli, e visi humani;
    Piè con artigli, e pennuto il gran ventre;
    Fanno lamenti su gli alberi, strani.
  \end{verse}
  Questo nome d'Arpia dette a una Meretrice anche il Coppetta nel suo Capitolo
  in biasimo della Signora Ortenzia Greca dicendo
  \begin{verse}
  Arpie crudeli, infide, inique, e ladre
  da venire a fastidio a mille Rome
  Voi, la vostra fantesca, e vostra madre.
  \end{verse}
\item[AMMARTELLATO] Haver martello, o esser' ammartellato vuol dire
Quand'uno innamorato ha gelosia della cosa amata, ovvero ha qualche sdegno
con la medesima. Il Firenzuola nel suo Capitolo in lode del legno santo, chiama
pazzia l'esser'ammartellato dicendo:
\begin{verse}
\backspace Hor nuovamente vi dico che cava
Di fastidio un, che crepi di martello,
Guarda se questa è un'opera brava.
\backspace E s'i pazzi volesson provar quello,
E conoscesson la lor malattia,
Tutti ritornerebbono in cervello;
C'altro non è il martel c'una Pazzia.
\end{verse}
\item[PER disperato] La disperazione è una soverchia inquietudine, cagionata da
  grave disgusto, la quale ci leva affatto il dominio di noi medesimi.
\item[PIAGNONI] Trova spesso nelle storie Fiorentine questo nome Piagnoni, che
vuol dir Coloro che seguitavano la parte di F. Girolamo Savonarola; ma qui vuol
dir Quegli huomini, che si mettono a i mortori de i gran personaggi attorno al
cadavero, tutti coperti di nero, e con lunghi veli, ed in mano hanno uno stendardo,
o pennoncello di taffettà nero: E si dicono Piagnoni, dal piagnere che
dovrebbon fare per la morte di quel tale.
\item[MARCIARE] È il muoversi degli eserciti. Voce restata a noi dal Francese;
  e da molti si dice Marchiare, perché questi tali, vedendola scritta con l'aspirazione,
  la pronunziano all'Italiana, non si curando di riflettere che il C-H suona
  sci, e non chi.
\item[REGGIMENTO] Quantità di Soldati comandata da più Capitani, e dal Colonnello;
  e forse lo stesso, che Terzo detto sopra in \cstan{37}.
\item[TRITONI] Sono Dei, o Mostri Marini, i quali si dipingono ignudi, o al
più coperti d'aliga, e di qui gli huomini mal vestiti si chiamano da noi Tritoni,
quasi huomini triti, che suona Huomini vili, ed abbietti. Vedi sotto in questo
Cant. stan. 86.
\item[INCANTUCCIARSI] Nascondersi, o mettersi per i canti per non esser
veduto.
\item[EMPIESTI i calzoni] Per la paura, se li move il corpo, e gli empie le brache.
Questo detto esprime, che Quei Tritoni facevano gran paura a chi gli vedeva, non
che veramente se gli empiessero i calzoni.
\item[S'ADDOPPIANO i serrami alle botteghe] Per afficurarsi da costoro, che sono stimati
  tanti ladri, in gran tratto di paese rinforzano le serrature alle botteghe. E qui
  l'Autore dice tutto quello, che egli può, per mostrar costoro affatto birboni, e
  vera canaglia.
\end{description}

\section{Stanza XXXXIV}
\begin{ottave}
  \flagverse{44}Hor comparisce Dorian da Grilli,\\
che nella guerra e così buon soggetto,\\
Che metterebbe gli Ettori, e gli Achilli,\\
E quanti son di loro in un calcetto:\\
Scrive sonetti, canta ognor di Filli,\\
E' buon compagno, piacegli il vin pretto,\\
Rubato, per insegna, ha nel Casino\\
Il quattro delle coppe c'ha il monnino.
\end{ottave}

Segue nella mostra Doriano da Grilli che è Lionardo Giraldi\footnote{}. Questo gentilhuomo
fu bellissimo humore, molto dedito alla poesia burlesca, buon discorritore,
ed huomo di conversazione; e perché egli haveva per costume il dar de Monnini,
il Poeta gli fa fare per impresa Una carta da giuocare, nella quale in mezzo a
un quattro di coppe è figurato un Monnino\footnote{La bertuccia, nel mazzo delle Minchiate, vedi sotto \cst[8]{61}.}.

\begin{description}
\item[METTERE uno in un calcetto] Confondere uno, Superar' uno nel sapere, o
  nel valore, e ridurlo tanto avvilito, che si vorrebbe nasconder dentro a un calcetto,
  vilissima, e piccola parte dell'abito dell'huomo, come quella che non
  cuopre se non il piede, Questo Doriano veramente non fu mai soldato, se ben
  l'Autore dice, che egli è \textit{buon soggetto nella guerra}; ma dice così di lui, perché
  essendo egli di sua conversazione, lo sentiva spesso discorrer delle guerre con gran
  fondamento mostrandosene assai pratico.
\item[VIN pretto] Vino puro, e senza commistione d'acqua, o d'altro; e sentendosi
  in più luoghi del nostro Contado chiamarlo \textit{vino puretto}, non son lontano da
  credere, che la voce \textit{pretto} sia o figurata, o corrotta da \textit{puretto}.
\item[CASINO] Intendi quella Casa nella quale la nobil gioventi Fiorentina s'aduna
  per giuocare,
\item[MONNINO] Le carte de' Ganellini, o Minchiate hanno in se effigiate quattro
  cose diverse, che una parte hanno spade, una parte bastoni, una parte danari,
  ed una parte coppe, e tutte quattro queste specie di carte comingiano da
  uno fino a 14. Nella carta del quattro di coppe in mezzo è figurata una bertuccia
  a sedere, la qual bertuccia da noi è detta \textit{Monnino}. E questa dice il Poeta, che
  è l'insegna di Doriano; perché egli solito di dare i \textit{Monnini}, che vuol dire,
  Quand'uno parlando con un'altro, questo lo forza a dir qualche parola, che rimi
  con un'altra, che a quel tale dispiaccia; per esempio Doriano disse ad un Cherico:
  \textit{Non fu mai gelatina senza \makebox[1.5em]{\dotfill}} E qui si fermò fingendo non si ricordare
  della parola che finiva il verso; ed il Cherico, il quale ben sapeva la sentenza
  gliela suggerì dicendo: \textit{senz'alloro}, e Dorian soggiunse: \textit{Voi siete il maggior bue
  che vada in coro}. E questo si dice dare i \textit{Monnini}.
\end{description}

\section{Stanza XXXXV \& XXXXVI}
\begin{ottave}
  \flagverse{45}Fra Ciro Serbatondi il Sir di Gello\\
Che in Pindo a Mona Clio sostiene il braccio,\\
Egeno de Brodetti, e Sardonello,\\
Vasari, ch'è padron di Butinaccio,\\
Conducon tanta gente ch'è un flagello\\
Da far che le pagnotte habbiano spaccio,\\
Di cui (perch'il mestar diletta a ognuno)\\
Si pigliano il comando a un dì per uno.

\flagverse{46}Di foglio per impresa un bel Cartone\\
Insieme con la pasta egli hanno messo,\\
Dei lor Fantocci, i quali da Perlone\\
Soglion copiare, o disegnar dal gesso,\\
Nel mezzo v'han dipinto d'invenzione\\
L'impresa lor, nella quale hanno espresso\\
Su le tre hore il venticel rovaio\\
C'ha spento il lanternone a un bruciataio.
\end{ottave}

Seguitano tre gentilhuomini scolari dell'Autore; uno è Fra \textit{Ciro Serbatondi},
che vuol dire \textit{Cristofano Berardi}, quale fa Sir di Gello perché ha forse una sua
villa così detta. Dice che \textit{sostiene il braccio, a Mona Clio}, perché egli è huomo
letterato. L'altro è \textit{Egeno de Brodetti}, che vuol dir \textit{Benedetto Gori}. Il terzo è
\textit{Sardonello Vasari}, che vuol dire \textit{Alessandro Valori}, il quale fa Sig. di Botinaccio,
perché ancor'egli ha una Villa così detta. Conducono questi molta gente, la
quale comandano vicendevolmente a un giorno per uno, e perché si conosca che
sono stati tutti tre scolari dell'Autore, fa lor fare una bandiera de i fogli di quei
disegni, che hanno fatto in squola sua; Ma perché questi attesero più alle lettere,
che alla pittura, però non fecero altro acquisto in essa, che quanto bastava per
una certa infarinatura, e per saperne discorrere; egli volendo mostrare questo
lor poco profitto, fa che di lor propria invenzione ritraggano nella detta lor
bandiera una cosa invisibile, come appunto è il Vento.

\begin{description}
  \item[È un flagello] Questo termine significa Infinità, ed Abbondanza grandissima,
ed esprime un numero indeterminato. Vien, forse dai Latino, che tal volta
significa Quantità immensa. Martial. lib. 2. 30. \textit{Et cuius laxas arca flagellat opes},
parlando d'uno che havea gran quantità di danari,
\item[CHE le pagnotte habbiano spaccio] Che s'esiti, che si consumi molto pane. E pagnotta
  se bene non è voce Fiorentina, è nondimeno spesso usata.
\item[MESTARE] Qui val Ministrare, Comandare.
\item[CARTONE] I pittori chiamano Cartone Quella carta grande fatta di più
  fogli, sopr'alla quale fanno il modello di qualche grand'opera, che devono dipignere
  nel muro a fresco, o a tempera, o vero per tessere arazzi.
\item[FANTOCCI] Figure mal fatte. \textit{Pittor da Fantocci} s'intende Pittore da poco,
  appunto come da questa loro impresa vuol l'Autore, che si argomenti che fussero
  questi Signori.
\item[DAL gesso] Cioè dalle figure fatte di gesso. I pittori hanno per costume di
  chiamare dette figure di rilevo, (delle quali si servono per disegnare) col solo
  nome di gesso, senza dir figure, o statue, come si vede nel presente luogo, che
  dice disegnar dal gesso.
\item[LANTERNONE] Arnese noto, che serve a portarvi dentro il lume, e difenderlo dal vento.
\item[BRUCIATAIO] Colui che vende marroni arrostiti alla fiamma, o nel forno,
  che noi chiamiamo Bruciate, donde Bruciataio,
\end{description}

\section{Stanza XXXXVII}

\begin{ottave}
  \flagverse{47}Nanni Russa del Braccio, ed Alticardo\\
Conduce quei di Brozzi, e di Quaracchi\\
Che, perché bevon quel lor vin gagliardo,\\
Le strade allagan tutte co i fornacchi,\\
Hanno a comune un lor vecchio stendardo\\
Da farne a corvi tanti spauracchi,\\
E dentro per impresa v'hanno posto\\
Gli spiragli del di di Ferragosto.
\end{ottave}

Seguitano due altri Gentilhuomini Nanni \textit{Russa del Braccio}, che vuol dire
\textit{Alessandro Brunaccini} ed \textit{Alticardo} che vuol dice \textit{Carlo Dati}; a quali
fa condurre le genti di Brozzi, e di Quaracchi, due luoghi vicini a Firenze, ne i quali nasce
vino debolissimo, e però dice che questi soldati son mal sani; e pieni di catarro,
perché bevono quei vini deboli,  (che egli ironicamente parlando, chiama gagliardi)
che per la loro debolezza danno prima alle gambe, che alla testa.
E perché tali infermi pare che si rihabbiano, e piglino qualche vigore, quando si
trovano all'allegrie; perché fa loro portare una insegna nella quale sono espressi
alcuni di quei bagordi, gozzoviglie, ed allegrie, che già si facevano \textit{il dì di
  Ferragosto}, che s'intende il dì primo d'Agosto, venendo questa voce da Feriare
agosto, e per intelligenza di questo è da sapere, che anticamente solevansi cele
brar le ferie Augustali con grandi allegrie; e ciò si faceva forse, perché essendo
gli huomini nel maggior fervore della state, erano necessitati dal gran caldo a stare
allegramente, perché l'allegria e il primo rimedio della squola Salernitana:
\textit{Haec tria: mens hilaris, requies, moderata diaeta}. Essendo dunque molto pericoloso in quei
tempi d'infermarsi, e perciò molti giorni infausti allora si notavano dagli Egizj,
essendo vicino al Sirio, o Canicula da tutti detta pestifera, come ci mostra Stazio
lib, 1. Silvar, \textit{Illum nec calido latravit Sirius astro}, E' necessario riposarsi, bere, e
mangiare, e stare allegramente; al che consiglia nelle sue Odi Orazio più volte;
Ed habbiamo una cantilena assai praticata, che dice.
\begin{verse}
Quando sol est in Leone,
Bonum vinum cum mellone,
Et agrestum cum pipione.
\end{verse}
E perché veramente il fervore del Sol Leone, o Sirio, e allora nel maggior colmo,
sono le stagioni molto calde; e peggiori, che in tutto l'anno; onde appresso
a' Greci ancora si facevano molte allegrie, e sacrifizzj a segno, che appresso
gli Attniesi secondo alcuni il mese d'Agosto acquistò il nome d'\textit{Hecatombaeon}. Tal feste,
ed allegrie si facevano già a Firenze non solo per la detta ragione, ma ancora per
causa di alcune vittorie ottenute da i Fiorentini in quei primi giorni d'Agosto, e se
ne conserva ancora il costume, ma non si fanno tante feste, quante già si facevano,
poiché solamente si fa correr al Palio alcuni Asini: Sì che s'argumenta, che
il nostro Poeta intenda, che in questa insegna, o stendardo fusse rappresentato il
palio de gli asini, mentre dice spiragli del dì di Ferragosto, che vuol dire un poca
di memoria delle gran feste, che già si facevano in quei giorni.
\begin{description}
\item[SORNACCHIO] Sputo grosso, e catarroso, detto anche farda, Vedi sopra in
\cstan{25}. Monsignor della Casa nel suo Galateo dice; \textit{Di soffiamenti di
naso sporcamente, di tirar sornacchi, e sputamenti}.
\item[SPAVRACCHIO] Così chiamiamo quei pannacci, che sopra ad un palo, pertica,
  o albero si mettono per li campi a fine di spaurire i colombi, ed altri uccelli,
  Vedi sotto \cstan[5]{49}.
\item[SPIRAGLIO] Vuol dir fessura in muro, o in tetto, o imposte di usci, o di
finestre, per la quale, trapela l'aria, o lo splendore, che i Latimi dissero \textit{rima}.
In questo luogo però è inteso metaforicamente per Piccola notizia, come è assai
in uso, e forse non lontano da i Latini, che dissero \textit{Spiraculum tantum ius rei ad
me venit} per intendere io ho havuta di ciò qualche notizia,
\end{description}

\section{Stanza XXXVIII}
\begin{ottave}
  \flagverse{48}Gustavo Falbi Cavalier di petto\\
Con Doge Paol Corbi hor n'incammina\\
Gl'Incurabili tutti, e il Lazzeretto;\\
Gente, che uscia di far la quarantina.\\
Van molti a grucce, in seggiola, e nel letto,\\
Perché non sono ancor netta farina;\\
Fan per impresa in un lenzuol che sventola\\
Un Pappino rampante a una pentola.
\end{ottave}

Seguono \textit{Gustavo Falbi}, cioè \textit{Ugo Stufa} Senatore Fiorentino, e lo chiama
\textit{Cavalier di petto}, perché ha la Croce in petto essendo Bali della Religione di
S.Stefano; E l'altro è \textit{Doge Paol Corbi}, che vuol dire \textit{Cavalier Iacopo del Borgo}.
A questi due gentilhuomini fa condurre una quantità di convalescenti, e di stroppiati,
per mostrare, che essi nel tempo; che l'Autore componeva la presente Opera
non erano d'intera sanità per qualche poca d'ipocondria, che gli molestava, e
fa però lor fare per impresa un Servo dello spedale di S.Maria Nuova con le
mani alzate a una pentola.
\begin{description}
\item[INCVRABILI] Così si chiama in Firenze uno Spedale, nel quale vanno a curarsi
  i Maifranzesati.
\item[LAZZERETTO] Luogo, o Spedale in cui si mettono gli huomini, e robe
  sospette di peste per far lor fare la quarantina, e renderle praticabili, che \textit{Far la
    quarantina} vuol dire Star riserrato in uno di questi luoghi quaranta, o più, o meno
  giorni per spurgar il sospetto d'infezione. E questo nome Lazzeretto viene
  da Lazzero risuscitato da N. Sig. Giesù Cristo, quando era di già fetente il di lui
  corpo.
\item[GRUCCIA] Specie di bastone per gli stroppiati, sopra una testata del quale
  essendo confitto un legnetto fatto a guisa di mezza luna, si sostiene il corpo mettendo
  detta mezza luna sotto il braccio, e l'altra testata del bastone in terra; e
  perché questo bastone è simile a una croce mi par di poter credere, che la voce
  Gruccia sia corrotta dal Latino \textit{scipio cruciatus},
\item[NON son netta farina] Non sono schietti, non sono affatto sani.
\item[LENZUOL, che sventola] Costoro in vece di bandiera, usano un lenzuolo, e
  ciò per mostrare, che tutte le loro cose sono da spedali; in esso lenzuolo è dipinto
  un'Astante, o Servo dello spedale di S. Maria Nuova, rampante a una pentola,
  cioè con le mani alzate a una pentola, che è in alto; a similitudine del Lione, il
  quale quando si trova dipinto ritto con le branche dinanzi alzate a qualche cosa,
  si dice Rampante. Franco Sacchetti Nov. 133, \textit{Ed hebbero ritrovato per cimiero un
  mezzo orso con le zampe rilevate, e rampanti}.
\end{description}

\section{Stanza IL \& L}

\begin{ottave}\flagverse{49}Bel Masotto Ammirato anch' egli passa\\
Lindo garzon d' ogni virtù dotato,\\
Che può, de' soldi havendo nella cassa\\
Pisciar a letto, e dire : io son sudato;\\
Ma per l'ipocondria, che lo tartassa,\\
Ei si dà a creder d'essere Ammalato;\\
Ma è mangia, beve, e dorme il suo bisogno,\\
Ch'è fino a vespro, e poi si leva in sogno,

\flagverse{50}Con lo scenario in mano, e il mondo fuora\\
Va innanzi a nobil suoi commilitoni,\\
Pancrazio, Pedrolino, e Leonora\\
Lo seguon con un nugol d'Istrioni,\\
C'hanno una insegna non finita ancora,\\
Perché Anton Dei con tutti i suoi garzoni,\\
Incambio di sbrigar quella faccenda,\\
È ito al Ponte a Greve a una merenda.
\end{ottave}

Passa Belmasotto Ammirato, che è Mattias Bartolommei Marchese giovane di bell''aspetto,
ricco, e letterato; il quale fu un tempo, che si persuadeva d'haver tutti
i mali. E perché questo Cavaliere si diletta di comporre commedie, e volentieri
recita in esse lui medesimo, ed appunto nel tempo, che l'Autore accrebbe la presente
Opera, havea detto Signore messa insieme una conversazione di giovani nobili,
che recitavano all'improvviso; però lo fa capo di nobili commedianti, e
gli da uno stendardo non ancor finito, perché \textit{Antonio Dei} ricamatore (e questo
è il vero suo nome, cognome, e professione) in cambio di finirglielo, era andato
a un'allegria al Ponte a Greve, luogo poco lontano da Firenze. Caso seguito
al detto Sig.\ Marchese Bartolommei, che aspettando alcuni abiti per una commedia
, che si dovea far la sera, il Dei in vece di finirgli sen'era andato con tutti
i garzoni della sua bottega fuori di Firenze.

\begin{description}
\item[HAVENDO de soldi nella cassa] Essendo ricco: Non gli mancando denari
\item[PISCIAR a letto, e dire: lo son sudato] E' proverbio assai vulgato, che significa.
Può fare a suo modo, che, o male, o bene che egli faccia, gli è sempre
ascritto a bene; E s'intende d'uno, che sia ricco, e fortunato.
\item[LEVARSI in sogno] Levarsi più presto dell' ere solita di levarsi, quasi dica
S'é levato di notte, sognado esser'hora di levarsi,e qui Autore intende, che a questo
Cavaliere il mezzo giorno, alla quale hora cominciava a destarsi, serviva per aurora,
\item[SCENARIO] È un foglio, sopr'al quale son descritti i recitanti, le scene della
commedia, la quale si dee recitare, ec. i luoghi, per i quali volta per volta devono
uscire in palco i recitanti, afinché quel tale, che assiste gli possa fare uscire
aggiustatamente, ed a i tempi debiti. Tal foglio si domanda anche \textit{Mandafuora}, se
bene il \textit{Mandafuora} è alquanto differente dallo \textit{Scenario}, perché questo s'appicca
al muro dietro alle scene affinché ciascuno recitante lo possa da se stesso vedere,
ed il Mandafuora è tenuto in mano da colui, il quale invigila, che l'opera sia,
recitata ordinatamente; ma tuttavia, come ho detto, s'intende, e si piglia spesso
l'uno, per l'altro.
\item[PANCRAZIO, Pedrolino, e Leonora] Nomi di recitanti nella suddetta conversazione.
\item[NUGOLO a' Istrioni] Gran quantità di commedianti. Questa voce \textit{nugolo}, che
nel presente luogo significa numero infinito, si usa più propriamente parlando di
volatili, perché questi volando gran numero insieme, come farebbono storni,
colombi, ec.\ occupano il sole, ed oscurano l'aria, appunto come fa il \textit{nugolo}. La voce
Istrioni è latina, tolta dall'antico Toscano, come dice Polid. Verg. \libcap[3]{14}.
le cui parole son queste. \textit{Et quia Hister Fusco verbo ludus vocabatur, ideo nomen histrionibus
est inditum}, ec. Ma hoggi ce ne serviamo per nome speciale, chiamando
Istrioni solamente i commedianti, che recitano per prezzo.
\item[GARZONI] Intende lavoranti; se ben Garzone vuol dir propriamente Giovane
  scapolo, e senza moglie, come si vede nell'ottava antecedente lindo garzone;
Tuttavia s'intende anche Servitore, o lavorante, che stia a salario in botteghe
di qualsivoglia mestiero.
\item[MERENDA] Specie di mangiare, che si fa tra mezzo giorno, e sera. Vedi
  sopra in \cstan{35},
\end{description}
\section{Stanza LI ... LVI}

\begin{ottave}
  \flagverse{51}Don Panfilo Pilori move il passo\\
Che, tra che per usanza mai sta cheto,\\
Hor ch'ei fa moto fa si gran fracasso,\\
Ch'io ne disgrado il Diavol n'un canneto,\\
Assorda il mondo più d'agn'altro il grasso\\
Papirio Gola, c'appunto gli è dreto,\\
Il qual vestì di lungo, e fu guerriero,\\
Perocché poco gli fruttava il Clero
\end{ottave}

\begin{ottave}
  \flagverse{52}E n'ha fatto con esso de rammanzi,\\
C'un po' di campanile non gli alloga,\\
E questa è la cagion, che là tra i lanzi\\
Da soldato n'andò in Oga Magoza;\\
Ne quivi essendo men tirato innanzi,\\
Posò la spada, e ripigliò la toga,\\
E per lo meglio si risolse al fine\\
Tornar' a casa a queste stiacciatine.
\end{ottave}

\begin{ottave}
  \flagverse{53}Al che tra molti commodi s'arroge;\\
Quel ber del vin; ch'è troppo cosa ghiotta,\\
Qua birre, qua salcraut, qua cervoge,\\
A casa mia dicea, del vin s'imbotta,\\
Però finianla; cedant arma togae:\\
Io non la voglio, in quanto a me, più cotta;\\
Guerreggi pur chi vuol, s'ammazzi ognuno,\\
Ch'io per me non ho stizza con nissuno.
\end{ottave}

\begin{ottave}
  \flagverse{54}Così rinunzia l'armi a Giove, e stima\\
D'esser il più lieto huom che calchi terra,\\
Pensa stato mutar, cangiando clima, \\
Ma trovata l'Italia tutta in guerra,\\
E forzato ferrarsi, più che prima;\\
Ecco il giudizio human come spesso erra\\
Crede tornar fra gente quiete, e gaie,\\
E fugge l'acqua sotto le grondaie.
\end{ottave}

\begin{ottave}
  \flagverse{55}Tra don Panfilo, e lui uno squadrone\\
Dal Pontadera aspettano, e da Vico,\\
Che parte per la via vanno a Vignone,\\
E parte fanne un sonno a piè d'un fico,\\
Costoro empion di rena un lor soffione,\\
E quando sono a fronte all'inimico,\\
Gliela schizzan nel viso, ed in quel mentre\\
Gli piglian gli altri la misura al ventre.
\end{ottave}

\begin{ottave}
  \flagverse{56}L'insegna di costoro è un Montambanco,\\
C'ha di già dato alli suoi vasi il prezzo,\\
E detto che son buoni al mal del fianco,\\
E strolagato, e chiacchierato un pezzo,\\
Ma trovandosi alfin sudato, e stanco,\\
E non havendo ancor toccato un bezzo,\\
Si scandolezza, ed entra in grande smania,\\
Poi dice, che si parte per Germania.
\end{ottave}

Segue Don Panfilo Piloti, che è Ipolito Pandolfini gran chiacchierone, e Papirio
Gola, che è Paolo Parigi, il quale ne i suoi primi anni vestì abito da Prete (che
questo intende col dire \textit{vestì di lungo}) ma poi lo posò, e sen'andò in Alemagna,
alla guerra vedendo, che quell'abito non gli era di frutto; Visto poi, che anche
quel mestiero non gli fruttava, tornò alla patria, e ripigliò l'abito. Ma trovato,
che ancora l'Italia era sottosopra per causa della guerra del Duca di Parma, fu
forzato dal debito di suddito, e dalla convenienza della provvisione, a tornare
alla guerra in servizio del Sereniss.\ Gran Duca, e a lasciar di nuovo l'abito da
Prete. Finita detta guerra il medesimo Paolo Parigi si rimette l'abito, e fattosi
Sacerdote, morì poi Rettore della Chiesa di S.\ Angelo a Vicchio. Questo Paolo
Parigi fu figliuolo di Giulio, e fratello d'Alfonso ambedue Architetti celebri,
come fu ancor'egli, ed Andrea altro suo fratello, che fu Maestro di campo, e
nominato dal nostro Poeta Paride Gurani sotto nel C. 3. stan.\ 10.

I suddetti due conducono genti dal Pontadera, e da Vico, (Terre vicine a Pisa)
le quali genti dice il Poeta, che \textit{l'aspettano}, perché venendo di lontano per la
stanchezza del viaggio s'erano fermate per la strada a riposarsi; E per mostrare,
che questo \textit{Papirio} era grand'ingegnere, fa che questa gente habbia per arme
un'ordigno per facilitare la distruzione del nimico, il quale e un mantrice pieno
di rena, e per alludere al genio vagabondo di Papirio, ed alle chiacchiere
di Don Panfilo, figura nella loro insegna un Montambanco, che sono genti
chiacchierone, (e però detti anche \textit{Ciarlatani}) e che non hanno patria ferma,
sendo oggi in Firenze, e domani altrove, secondo che gli porta la speranza del
guadagno.

\begin{description}
\item[FRACASSO] Strepito, romore; Vien dal latino Frangere, che vuol dir
  Rompere, e veramente il significato proprio di fracasso e quel romore, che procede
  da frattura, o spezzamento di materiali; se bene si piglia per ogni sorte di
  strepito. Dan. Inf. C. 9.
  \begin{verse}
    già venia fu per le torbide onde
    Un fracasso d'un suon pien di spavento.
  \end{verse}
  E nel Purg. Cant, 14,
  \begin{verse}
    ecco l'alzra con si gran fracasso
  \end{verse}
  Dove l'espositore Landini dice, che Fracaffo vien dal verbo frangere.
\item[NE disgrado il Diavol n'un canneto] Farebbe manco romore il Diavolo in un
  postime di canne. Si figura il diavolo, per lo più, un'huomo con le corna, con
  l'ali, e co i piedi di gallo; onde si dice un \textit{Diavol n'un canneto}, perché si suppone,
  che passando il detto diavolo dentro a un postime di canne, pigli con le corna,
  con l'ali, e con gli artigli le canne, le quali scappando dalle dette corna;
  ali, ed artigli a guisa di molla, perquotono nell'altre canne, che per esser vote
  fanno strepito, e rimbombo non piccolo. Quand'uno s'affatica per conseguir
  qualcosa diciamo: \textit{Il tale ha fatto il diavolo per haver la tal cosa}, e s'intende \textit{ha
  fatto il diavol n'un canneto}, cioè gran romore, Il termine; \textit{Ne disgrado} Vuol dire
  lo stimo manco: lo levo il luogo, o grado: per esempio \textit{Il tale compone versi Latini
  così bene, che io ne disgrado Vergilio}, cioè io stimo, che questo tale habbia tolto
  il luogo a Vergilio, e faccia meglio di lui. Vedi sotto Cant, 3. stan. 34. C. 6.
stan. 61.¢ \cstan[7]{25}.

\item[RAMMANZO] Far un rammanzo, o rammanzina vuol dire, Riprender' uno,
  con minacce; e suona lo stesso, che far' un rabbuffo, o Rabbuffare detto sopra in
  \cstan{39}.
\item[NON gli alloga un po' di campanile] Piglia la parte per il tutto, e vuol dire Non
gli fa conseguire una Chiesa.
\item[LANZI] Così chiamiamo i Soldati a piedi guardie del Sereniss. Gran Duca, i
quali son tutti Alabardieri Tedeschi: E pero dicendo: \textit{Andò fra i Lanzi} intende
Andò fra i Tedeschi, cioè in Alemagna; la voce Lanzi e Todesca lasciataci da
loro medesimi, che in salutarsi sogliono chiamarsi \textit{Lantzman}, che suona Paesano;
e \textit{Lanzchnect} vuol dir soldato a piede, e per questo gli Scrittori Fiorentini si
servono della voce \textit{Lanzichenecchi}, per intendere Soldati Alemanni a piede. Ed
il Varchi storie Fiorentine lib. 2, dice così: \textit{Quanto più s'avvicinavano i Lanzi, che
così per maggior brevità gli chiameremo da qui avanti, e non Lanzichenecchi, ec}.
\item[OGA magoga] Quand' uno va lontano dalla sua patria, dicono le nottre donne,
  \textit{Gli è andato in Oga magoga}, Ed intendono gli è andato a casa maladetta, nel
  qual senso è preso anche nella sacra scrittura; e S. Gio; nell'Apocalisse al 20,
  dice \textit{Og magog, \& congregabit eos in praelium}. Ed al cap. 7. dice \textit{In
    dispersionem gentium}, e si trova anche in altri libri della Sac. Bibbia. Vedi Angel. Mons.
  Fio. Ital. linguae alla parola oga magoga. Dicono ancora \textit{Gaga magoga}. E forse
  intendono dei Regno di Goaga in Affrica. Il Vocabolista Bolognese\footnote{Vocabolista Bolognese (1660), opera di Ovidio Montalbani (Bologna 1601 - ivi 1671), matematico, medico, astonomo.} dice, che Og fu
  gigante d'Astarotte Rede Baraniti, della creazione del Mondo 2492, contro al
  popolo d'Israel ne i campi d'Edrai, ove fu destrutto con tutto il suo esercito, e
  cinquanta Città; e che di qui venne il significato Andare in dispersione, e in fumo.
  o a casa del Diavolo, essendo interpetrato Og magog, per il Diavolo. Sin qui
  il Vocabolista. Gli antichi secondo Plinio chiamavano Magog la Città d'Edessa,
  (che Strabone dice, che è l'istessa, che Hierapoli) dove era il celebre Tempio
  della Dea Atergatide detta la Dea Siria, e dove gli Ebrei vissero in cattivita, onde
  da questo dicendosi Andare in Magog, per gli Ebrei era lo stesso che dire:
  Andar' in servitù. Gio: Villani Stor. Fior. \libcap[5]{29}. dice: \textit{Le genti, che si
  chiamano Tartari uscirono dalle Montagne di Gog Magog chiamate in latino monti di
  Belgen}. Conchiudo dunque, che non dire \textit{andò in Oga Magoga}. Significa Andò
  in paesi lontanissimi, e di pericolo: ed è quasi lo stesso, che dice \textit{Andò a Buda},
  che vedremo sotto Cant. 5. stan. 13.
\item[TIRATO innanzi] Avanzato a gradi, a dignità, a utili, ec.
\item[TOGA] Vuol dir propriamente abito da Dottori, ma si piglia bene spesso per
l'abito da Prete, come è presa in questo luogo.
\item[TORNAR a casa a queste stiacciatine] Tornare a goder'i comodi della propria
  casa, che si dice anche: Tornare al Pentolino, che i latini dissero: \textit{Redire ad
    pristina Praesepia}. Stiacciatina è diminutivo di Stiacciata, la quale è specie di pane, che
  dopo lievito si stiaccia con le mani per farlo più sottile, affin che si quoca più presto,
  e faccia minor midolla.
\item[S'arroge] ll verbo Arrogere vuol dire aggiugnere. Al che \textit{s'arroge}; al che
  s'aggiugne, e vuol dire; Ci è anche di più. Il Lasca Nov.~5.
\begin{verse}
  E così per non arroger peggio al male, si stava quieta, ec,
\end{verse}
Petr. Canz, 9.
\begin{verse}Eduolmi, c' ogni giorno arrage al danno.
\end{verse}

\item[COSA ghiotta] Cola desiderabile, cosa appetitosa; che \textit{ghiotto} si dice Uno avido
  di mangiar del buono; e viene da \textit{indulgere gutturi}.

\item[SAL craut] Cavolo salato. Voce, e vivanda Tedesca.

\item[BIRRA] o \textit{Cervogia}, Bevanda, che s'usa in Alemagna, ed in altri paesi,
dove è poco Vino; ed è composta di biade, acqua, e fiori di luppoli; ed è lo
stesso \textit{Birra}, che \textit{Cervogia}, e questa ultima è dal Latino.

\item[IMBOTTARE] Metter nella botte. Se bene qui si potrebbe intendere Bere,
costumandosi dire: \textit{Io non imbotto acqua}, in vece di dire: Io non bevo acqua, si
come è inteso sotto \cstan[7]{4}.

\item[NON la voglio più cotta] Per la mia parte mi basta così,ne mi curo di meglio.
Sum presenti Catone contentus, dilic Auguito.
\item[STIZZA] Ira, collera; e vale anche per Inimicizia.

\item[FERRARSI] Intende Armarsi. È detto scherzoso, perché Ferrare, senza dir
più s'intende mettere i ferri all'unghie de' piedi de' cavalli, muli, ed altre
bestie.

\item[GENTI gaie] Genti allegre, ricche, e abbondanti d'ogni comodo, e quiete;
che la voce Gaio è forse sincopata da Gandio.

\item[GRONDAIE] Quel cascare, che fa l'acqua da i tetti, quando piove; e si
dice Grondaia da Gronde, che sono quelle tegole più larghe, le quali son poste
nell'estremità de' tetti. Ed il Proverbio \textit{Fuggir l'acqua sotto le grondaie} vuol dire;
Procurar di fuggire un pericolo, e andarli incontro, che è quello forse, che i Latini
intesero col dire \textit{Incidit in Scyllam cupiens vitare Charybdim}.

\item[ANDARE a Vignone] Andar nelle vigne altrui a corre l'uva; e si dice così
per rendere il detto oscuro, mostrandosi d'intendere d'Avignone in Francia, o
del Bagno di Vignone, che è nello Stato di Siena.

\item[SOFFIONE] Quel piccolo Mantaco, o Mantice, del quale comunemente ci
  serviamo per soffiar nei fuoco, usandolo a mano.

\item[SCHIZARE] Qui è verbo attivo, e vuol dice: Gli gettano con violenza nel
  viso quella che è dentro al soffione.

\item[MONTANBANCO] Uno di coloro che vendono i rimedj nelle pubbliche piazze,
  detti \textit{Montambanchi} dal montare sopra i banchi quando vogliono vendere;
  e detti anche \textit{Ciarlatani} dalle gran ciarle, che sogliono fare.

\item[TOCCATO un bezzo] Preso, o buscato un quattrino. \textit{Bezzo} è moneta, e
  Parola Veneziana, ma usiamo, se non la moneta, almeno la voce \textit{bezo} ancor noi
  per intender Denari in generale.
\item[SI scandolezza] In questo luogo, ed in questi termini significa Adirarsi, e
  mostrar con le parole, e con gli atti la collera, che uno ha. Vedi sotto C.~11. stan.~23.
  Verbo che viene dal Greco \textit{scandalizesthai} che suona, a loro, come a noi
  Offendersi, o adirarsi d'una cosa.

\item[ENTRAR in smania] Entrar in grandissima collera; che Smania è una soverchia
  inquietudine, cagionata da febbre, o da eccessivo caldo, o da soverchio
  amore, la quale riduce l'huomo quasi insano, e furioso.
\end{description}

\section{Stanza LVII \& LVIII}
\begin{ottave}
  \flagverse{57}Huomini bravi quanto sia la morte\\
Scandicci n'ha mandati, e Marignolle,\\
Gente, che si può dir che habbia del forte,\\
Poi ch'ella ammazza gli agli e le cipolle,\\
Sue lance i pali son, targhe le sporte,\\
Airchiusi le man, le palle zolle,\\
Va ben di mira, e colpo colpo imbreccia,\\
Maffime quand'altrui vuol dar la freccia,
\end{ottave}

\begin{ottave}
\flagverse{58}Vien comandata da Strazildo Nori,\\
Ch'è Chimico, Poeta, e Cavaliere,\\
Ed è quel, ch' in un quadro co i colori\\
Fece quei fichi, che divenner pere.\\
E perché questo è il Re de bell'humori,\\
Per dimostrar quanto gli piaccia il bere; \\
Ha per impresa un Lanzo a due brachette;\\
Ch'il molle insegna trar dalle mezzette.
\end{ottave}

Seguita la gente di Scandicci, e di Marignolle, Ville vicine a Firenze, dove
nascono Cipolle, Agli, ed altri fortumi simili in grande abbondanza. Questa
gente dice che è \textit{brava quanto la morte, perché ella ammazza gli agli, e le cipolle, e
si può dire che habbia del forte}, E pare che intenda che ella superi in fortezza, e
bravura gli agli: E vuol poi dire, che ha molti fortumi, ed Ammazza, cioè Fa
mazzi delle cipolle, e degli agli. E perché questi contadini habitando intorno
a Firenze praticano molto la Città, dove è occasione di spendere più che nel
contado, dice l'Autore, che son genti che \textit{danno la freccia}, che vuol dir Chieder
denari in presto; e par ch' ei voglia intendere che son bravi tiratori di freccia,
e d' archibuso. Son comandati da \textit{Strazzildo Nori}, cioè Rinaldo Strozzi Cavaliere
di S. Stefano; ed è quello, che in squola dell'Autore volendo dipignere
alcuni fichi non trovò mai il modo di fare, che non paressero pere. Questo fu
un geatilhuomo di grandissimo garbo, faceto, allegro, e spiritoso, e buon bevitore;
e perciò gli fa fare per impresa un Lanzo, che vota una mezzetta di vino,
e gli fa comandare questa gente, perché fu poi P\ellipsis{2em} in vicinanza dei
lor paesi.

\begin{description}
\item[SPORTA] Specie di paniere fatto di giunchi, ed ha due manichi; serve per
portarvi dentro erbaggi, ed altro, che si provvede in piazza giornalmente per il
Vitto.
\item[ZOLLA] Gleba, pezzo di terra sollevata nel lavorare i campi, Vedi sotto
in questo Canto stan.\ 82.
\item[COLPO colpo] A ogni colpo. Intendi: sempre ch' ei tira; colpisce, che la forza
  della replica e di far nascer il superlativo.
\item[IMBRECCIA] Forse meglio \textit{imbercia}; E Significa Pigliar di mira; donde
  \textit{imberciatore} colui che fa professione di tirar d'archibuso; e par che venga da
  sbirciare, e bircio, che è guardar con occhi socchiusi, come dicemmo sopra in
  \cstan{9}. e come s'usa a tirar con l'archibuso. Ma puo anche essere che
  venga da breccia che vuol dir Quelle rotture che vengon fatte nelle muraglie
  dall'artiglierie, e si dica imbrecciare per colpire, si come intende nel presente
  luogo pigliando colpire in senso di conseguir l'intento.
\item[DAR la freccia] Come habbiamo accennato, vuol dire Chieder denari in presto;
  e s'intende Uno che habbia poco modo, e minor voglia di rendergli. Gli
  antichi Etiopi, e gli abitatori di Maiorca, ec. non solevano dar mangiare alli
  loro figliuoli, se questi con le frecce non facevano cascare dallo stile, o albero
  il cibo, che vi era posto, ond'io stimo, che questo frecciar per vivere habbia dato
  origine al presente detto. Vedi Alex. ab Alex.\footnote{Alessandro d'Alessandri, ``Alexander ab Alexandro'', Napoli 1461 - Roma 1523. Umanista e giurista.} dier. gen.\footnote{Genialium Dierum, Parigi, 1532.} lib. 2. c. 25. Il Monosino
  dice, che questo \textit{frecciare} habbia origine dal Latino \textit{ferire} che appresso
  loro haveva il medesimo significato, e lo cava da Teren. in princ. Phormionis:
  \textit{Porro autem Geta Ferietur alio munere ubi hera pepererit}. Diciamo; i denari sono il
  secondo sangue; dar ferita cava il sangue, come il dar frecciate, cava il sangue;
  e per questo dicendo \textit{dar freccia} intendiamo Dar freccia alla borsa, e cavare questo
  secondo sangue, che è il danaro.
\item[BELLUMORE], Huomo allegro, faceto, ec. vedi sopra in \cstan{10}.
Quando diciamo, Il tale è Re della tal cosa; intendiamo Vale in superlativo
grado in quella tal cosa; onde \textit{Re de belli humori} vuol dire Grandissimo bell'humore.
Significato che viene da i Greci, i quali chiamavano Re colui, che nei
giuochi fanciulleschi vinceva, e superava gli altri, ed Asino, o Mida era chiamato
colui che perdeva; il che più diffusamente vedremo nel 2. Canto.
\item[LANZO a due brachette] Lanzo dicemmo sopra, che vuol dir soldato Tedesco
  a piede; ma qui vuol che s'intenda uno proprio di quelli della guardia del Serenissimo
  Gran Duca; dicendo a due brachette, perché questi tali Lanzi vanno vetiti
  a livrea, con un paro di brache larghe, fatte a strisce, come son quelle delli
  Svizeri del Papa in Roma, e come quelle de' Trabanti dell'Imperatore.
\item[INSEGNA trarre il molle dalle mezzette] Insegna col suo bere, come si fa a votare
  i vasi pieni di vino, Che \textit{mezzetta} è un vaso fatto di terra invetriata, che
  serve per misurare il vino, ed è capace della quarta parte d'un fiasco Fiorentino\footnote{il Fiasco Fiorentino si divide in 4 mezzette, la mezzetta in 2 quartucci. Come misura per il vino, un quartuccio equivale a circa 0.285 litri}.
\end{description}

\section{Stanza LIX \& LX}
\begin{ottave}
  \flagverse{59}Morbido Gatti, Henrigo Vincifedi\\
A far venir innanzi ecco son pronti\\
I fanti, che ne dà il Ponte a Rifredi,\\
Che mille sono annoverati, e conti.\\
Han certi Santambarchi fino a piedi,\\
Che chiaman' il zimbel di là da monti,\\
E paion con la spada in su le polpe\\
Un che facia lo strascico alla volpe.
\end{ottave}

\begin{ottave}
\flagverse{60}Nell'insegna han ritratto u' huom canuto,\\
Che troppo havendo il crin (per osser vecchio)\\
Fioccoso, e lungo, un fanciullino astuto\\
Dietro gli grida: Gli abbrucia il pennecchio.\\
Da questa schiera qui s'è provveduto\\
Gran ceste piene d' huova, e di capecchio\\
Con fasce, pezze, e taste accomodate\\
Per farsi alle ferite le chiarate.
\end{ottave}

Passa l'ultima truppa di Soldati, la quale è composta d'huomini dal Ponte a
Rifredi, che è un luogo vicino a Firenze. Costoro son comandati da \textit{Morbido
  Gatti}, cioè \textit{Migiotto Bardi}, e da \textit{Henrigo Vincifedi}, che è \textit{Vincenzio
  Sederighi}, due gentilhuomini già scolari dell'Autore: E perché questi si pigliavano gusto di
ragionare spesso con un tal Dottor Cupers, glielo fa fare per impresa.

A Questo Dottor Cupers negli ultimi anni della sua vita, che durò sopra ottanta
anni, entrò in frenesia d'esser bello, e si persuadeva che ogni donna s'innamorasse
di lui, e lo volesse per marito, e però andava lindo, e con la chioma
folta, e lunga, e ben coltivata; ma canutissima: onde i ragazzi quando passava
per le strade gli gridavano dietro: Guarda il Pennecchio, gli abbrucia il Pennecchio,
intendendo di detta sua chioma, e lo facevano adirare, e maggiormente
impazire. E perché li contadini del Ponte a Rifredi si danno a credere d' haver
maggior Civiltà degli altri contadini per esser nati, ed allevati, si può dire, nei
Borghi di Firenze, ed intorno alla Petraia, e Castello, Ville spesso habitate
da Principi della Serenissima Casa, perciò per lo più vengono alla Città col
ferraniuolo, o santambarco, che sono le Toghe de i Barbassori, e Dottori
del Contado; e per questo il Poeta dice \textit{Han certi Santambarchi fino a piedi, Che
  chiamano il Zimbel di là da' monti}, cioè incitano i ragazzi a dar loro delle Zimbellate.
E per esser questa l'ultima schiera fa, che ella conduca seco il bagaglio
de i medicamenti per l'Esercito.
\begin{description}
\item[SANTAMBARCO] Specie d'abito, o sopravveste, o diciamo mantello
usato da i nostri contadini per difendersi dall'acqua, e dal freddo; ed è composto
di due larghe strisce di panno cucite in forma di croce con una buca in mezzo,
per la quale passano il capo, e vengono coperti da una parte di detto panno le
schiene, e il petto, e dall'altra le braccia, e i fianchi, Si dovrebbe dire \textit{Salta in
barco}, e così dice Mattio Franzesi nel Capitolo del suo viaggio da Roma a
Spoleto.
\begin{verse}
\backspace Gli osti, c'a profferir mai non son parchi
Volean ch'io scavalcassi a sì mal tempo,
E m'offerivan fuoco, e Saltambarchi.
\end{verse}

Ed è forse meglio detto \textit{Saltambarco}; perché questo abito è composto in tal
forma; che tiene tutta la persona difesa dal freddo, e non l'impedisce il saltare
i fossi, e passare i barchi. Ma si dice \textit{Santambarco} perché così lo chiamano i contadini
che se ne servono, ed è lor abito proprio.
\item[CHIAMAR una cosa di là da i monti] Questo termine significa Meritare una
  cosa grandemente, come per esempio \textit{Il tale è così insolente, ch'ei chiama le bastonate
  di là da i monti}.
\item[ZIMBELLO] In questo luogo intende un sacchetto pieno di crusca;
o di cenci, o di segatura, legato a una cordicella lunga circa due braccia,
col quale i fattorini delle botteghe de setaiuoli nel tempo del Carnevale, quando
passano i contadini per quei luoghi, dove sono le botteghe de i setaiuoli, uno di
loro perquote il contadino; e mentre questo si volta per veder chi ha percosso,
gli altri ragazzi lo perquotono dall'altra banda: E questo per lo più vien fatto a
certi contadini, che se ne vengono in Firenze intronizzati, e in sul grave, come
appunto fanno quei del Ponte a Rifredi. E per altro la voce Zimbello ha il significato,
che vedremo sotto \cstan[7]{76}.
\item[FAR Io strascico alla Volpe] E' una specie di caccia, che si fa alla Volpe, pigliando
  un pezzo di carnaccia fetida, che legata a una corda si va strascicando per
  terra; per far venir la Volpe al fetore di essa Carne; ed il Poeta assomiglia il portar
  della spada di questi Contadini a questa corda, dicendo che stava pendente
  \textit{in su le polpe} (cioè dietro alle gambe, che così chiamiamo cotesta parte) appunto
  come sta la fune di colui, che fa lo strascico alla Volpe.
\item[PENNECCHIO] Qui è preso per chioma, ò Zazzera, come habbiamo accennato
  sopra, metaforico da quell'involto di lino, stoppa, lana, o altra materia
  simile, che adattano le donne sopr'alla rocca per filare, il quale involto si dice
  Pennecchio.
\item[QUESTA schiera qui] La voce \textit{qui} è superflua, bastando per farsi intendere il
dir solamente \textit{da questa Regina} senza aggiungere la particella \textit{qui}: Ma non per
questo il nostro Poeta ha fatto errore, havendo seguitato il nostro Fiorentinismo
usatissimo. Dicendosi comunemente (forse a maggior' emfasi) \textit{Questo negozio qui},
\textit{questa cosa che è qui}, e simili; e la particella \textit{qui} esprime \textit{il negozio, del quale ragioniamo presentemente}, \textit{Questa cosa, la quale habbiamo fra le mani}: Anzi stimo, che
l'habbia fatto ad arte, e per mostrare questo nostro modo di dire, (forse riprensibile)
del quale non mi pare, che in tutta l'Opera si sia servito mai più; quantunque
non gli sieno mancate l'occasioni; E se bene nell'Ottava 65. seguente,
pare, che l'usi nel medesimo modo, osservisi, che quivi è termine dimostrativo
necessario, e non riempitivo, operando che s'intenda di quella Cugina, che è lì
presente, e non d'altra, come si potrebbe intendere, se non vi mettesse la particella
\textit{qui}.
\item[CESTA] Intendiamo un gran paniere, che fa mezza soma di bestia, ed è contesto
  d'assicelle di castagno, o d'altro legname a foggia di cassa, per uso di portare
  da un paese all'altro uova, vino in fiaschi, ed altre cose frangibili; e per lo
  più son fabbricati due attaccati l'uno all'altro con quattro legni gagliardi aggiustati
  in maniera da adattarsi sopra i basti a traverso alla bestia, in modo che tengono
  equilibrate, e ferme dette due ceste anche senza legarle. Se ne fabbricano
  ancora della stessa forma, e materia sciolte, cioè senza i detti quattro legni, e
  queste s'adattano, e fermano in su i basti con le funi, come si fa i Cestoni, che
  sono ancor'essi panieroni di mezza soma fatti di vinciglie di castagno, o altro albero
  intessute, de i quali si parla sotto C.\ 10.\ stan.\ 7.
\item[CAPECCHIO] La pettinatura, cioè quella stoppa più grossa, che si cava dal
  lino sodo la prima volta, che si pettina detta capecchio, perché si cava dai due
  capi del lino, cioè barbe, e cime, le quali sono più ripiene d'immondezze, e di
  filo morto, e inutile.
\item[FAR la chiarata] Il primo medicamento, che si faccia alle ferite è l'albume,
o chiara d'huovo, entro alla qual chiara s'intigne il capecchio, e si pone sopra
alle ferite; E questo si dice \textit{far la chiarata},
\end{description}

\section{Stanza LXI}

\begin{ottave}
\flagverse{61}E' general di tutta quella mandra\\
Amostante Laton Poeta insigne\\
Canta improvviso, come una calandra,\\
Stampa gli enigmi, strolaga, e dipigne.\\
Lasciò gran tempo fa le polpe in Fiandra,\\
Mentre si dava il sacco a certe vigne, \\
Fortuna, che l'havea matto provato\\
Volle, ch' ei diventasse anche spolpato.
\end{ottave}

Generale di tutto questo esercito e Amostante Latoniy, cioè \textit{Antonio Malatesti}
Poeta celebre per molte sue opere, ma specialmente per quella Sfinge, la quale,
come vedremo sotto \cstan[8]{26}. è una scelta d'enigmi in sonetti, de' quali se
ben la stampa ne fa goder pochi, se ne sperava numero maggiore, volendone
egli pubblicare 400. scelti da una infinità, che ne ha composti; ma la di lui morte
seguita poco tempo fa, ci priva per ora di questa consolazione. Ne gli anni suoi
giovenili cantò all'improvviso molto lodatamente, si dilettò d'Astrologia, e nel
disegno fu scolare dell'Autore, e suo amicissimo, come mostra, facendolo capo,
e saperiore di tutti gli amici suoi, che nomina in questo esercito. E perché questo
Amostante era di corpo adulto, ed havea le gambe sottili, dice, che \textit{lasciò le polpe
in Fiandra}, e che \textit{la Fortuna che l'havea provato matto}, volle che egli diventasse
anche \textit{spolpato}, cioè senza polpe; ma aggiunto alla voce \textit{matto} vuol dire
\textit{matto affatto}; non che Amostante fusse affatto privo di cervello; che la voce
\textit{matto} appresso di noi significa ancora Allegro, Faceto, e simili, nel qual senso è presa
nel presente luogo; e però vuol dire, che Amostante era huomo facetissimo.
\begin{description}
\item[MANDRIA] Vuol dire Una gran quantità di bestie; ma qui intende Grani
  quantità d'huomini. Mandra è voce Greca, che suona Spelonca, e luogo, entro
  al quale le pecore s'adunano all'ombra, ma la pigliavano anche per la greggia
  medesima, e da essa dissero Archimandrita il governatore della greggia.
  Dante pure prese \textit{Mandria} per quantità di huomini, nel Purg. C. 3.
  \begin{verse}
    Sì vidd' io muovere, e venir la testa
    Di quella Mandria fortunata allotta,
    Pudica in faccia, e nell'andare onesta,
  \end{verse}
\item[CANTA improvviso] È costume in Firenze al tempo de i gran caldi la notte
  cantare dell'ottave all'improvviso, mentre ne i luoghi più aperti della Città si
  va pigliando il fresco; e perché in tal'esercizio valeva molto il Malatesti; il Poeta
  l'assomiglia alla Calandra uccello di bellissimo cantare.
\item[ENIGMI] Indovinelli. Voce Latinogreca. Vedi sotto C.6, stan.34.c C.8 stan. 26.
\item[LASCIO' le polpe in Fiandra] Non è, che Amostante fusse mai stato in
  Fiandra; ma, perché lo fa generale di questo esercito, è dovere, che egli mostri,
  che Amostante ha vedute, e provate altre guerre, e che egli si sia trovato a
  dar de' sacchi, ne i quali ha lasciate le polpe delle gambe, il che serve per accreditarlo,
  poiché si come ad un soldato gli stroppj, e le cicatrici son di gloria, così
  ad Amostante era di gloria  haver perduto le polpe delle gambe nelle guerre di
  Fiandra; ma il vero è, che quand'uno hale gambe sottili, diciamondi lui: \textit{Egli
  ha lasciato le polpe in Fiandra}: ed il Poeta con questo equivoco, che accredita
  Amostante, vuol dire, che egli haveva le gambe sottili; e seguita con l'altro
  equivoco di \textit{matto spolpato}, che significa, come s'è detto,  matto del tutto, e
  vuol che s'intenda \textit{senza polpe affatto}. E la voce polpa, che significa ogni pezzo, o
  quantità di carne, che sia senz'osso, da noi si piglia per le polpe delle gambe,
  quando è detta assolutamente. (Vedi l'ottava 59. antecedente; E sotto al C.6.
  stan. 99. dice \textit{ossccia senza polpe}, che s'intende tutta la carne di quel'corpo) e
  significa pure \textit{Matto spacciato}.
\end{description}
\section{Stanza LXII}

\begin{ottave}
\flagverse{62}Passati tutti con baule, e spada\\
Serransi in barca, come le sardelle;\\
Gli affretta il Duca, e chi lo tiene a bada,\\
O ferma un passo; guai alla sua pelle,\\
Ch'ei lo bistratta, e come che ne vada\\
Giù la vinaccia, e il sangue a catinelle,\\
E ben che lesto ciaschedun rimiri,\\
Non gli dà tanto tempo ch'ei respiri.
\end{ottave}

Dopo fatta la mostra se n'entra la soldatesca nelle barche con ogni suo arnese,
e Baldone affretta all'imbarco i soldati.
\begin{description}
\item[BAVLE] Intendiamo ogni sorte di cassetta, valigia, o tamburo, che facilmente
  si possa adattare in su la groppa d'un cavallo, mentre si viaggia. Viene
  dal verbo \textit{baiulo}, e l'allarghiamo ad ogni sorta di cassa portatile in su le some, ec.
  Qui intende quell'involto, che portano i soldati sopr'alle reni per lor proprio
  bagaglio, detto altrimenti zaino.
\item[SERRANSI, come le sardelle] Si serrano strettissimi appunto, come stanno le
  sardelle ne i cestoni, quando da Livorno son portate a Firenze, o nei bariglioni,
  quando ci vengono salate. Comparazione assai usata per intendere stetti, e
  serrati insieme, che in voce marinaresca si dice stivati.
\item[TENERE a bada] Trattenere uno. Varchi stor, lib, 4. \textit{Conoscevano, che erano
tutte cose finte, e solo per tenere a bada trovate}, Viene dal Verbo \textit{Badare}, che ha
  molti significati. \textit{Badare} al negozio per \textit{Attendere al negozio}. Significa
  Indugiare, o perder il tempo, come è inteso nel presente luogo, che dice \textit{tiene a bada},
  ed intende, Chi gli è causa d'indugio, o gli fa perder tempo; il Petrarca Son.23.
  \begin{verse}
    Consolate lei dunque, che ancor bada.
  \end{verse}
 Cioè aspetta la venuta del Pontefice, e perde tempo. Significa ancora \textit{continuare}, o
 \textit{seguitare} a far una cosa, Vedi sotto C.1, stan. 20. Significa \textit{Osservare} C.9.
 stan. 28.  Significa \textit{Disprezzare}, \textit{non curare}, per esempio; \textit{Io non bado al tuo gridare}. Intende
\textit{io non stimo, o non curo il tuo gridare}, Da questo \textit{badare}, o \textit{bada} habbiamo \textit{badalone}
che vuol dire Un' huomo perdigiorno, e che non sa, e non vuol far nulla.
\item[GVAI alla sua pelle] Mal per lui. Vedi sopra in \cstan{28}.
\item[BISTRATTARE] Trattar male, Strapazzare, o Stranare.
\item[VA giù la vinaccia] È necessario far presto per sfuggire il danno, che si patisce
  e che si teme più grave dall'indugio. Quando il mosto, cioè il liquore cavato
  dall'uva, il quale è nel tino, ha bollito a bastanza; perde il vigore, e non
  può più sostenere a galla, cioè nella sua superficie, la vinaccia (che così si chiamano
  i raspi, e bucce dell'uve) onde la lascia cascare in fondo, ed incorporandosi
  con essa di nuovo, si guasta; E questo si dice \textit{andar giù la vinaccia}; che
  poi passato in proverbio significa Quel che habbiamo detto.
\item[NE va il sangue a catinelle] Ne va molto del mia. Per intender, che Un'indugio
  apporta grave dispendio, ci serviamo di questo detto; e si dice anche: \textit{a bigonce}.
  Vedi sotto C.\ 10.\ stan.\ 20.
\item[LESTO] Qui vuol dir Pronto, ed all'ordine.
\item[NON gli da tempo che respiri] Non gli lascia ripigliare il fiato. Questo detto
  esprime un grande affrettamento, o incalzamento.

\end{description}

\section{Stanza LXIII \& LXIV.}
\begin{ottave}
\flagverse{63}Perciò imbarcati tutti in un momento,\\
Poi che Baldon facea così gran serra,\\
Si spiegaron l'insegne, e vele al vento,\\
Quando le Navi si spiccar da terra;\\
Ed egli allora entrò in ragionamento\\
Di quel che lo spingeva a far tal guerra;\\
Ma per contarla più distesa, e piana,\\
Incominciò così dalla lontana.
\end{ottave}

\begin{ottave}
\flagverse{64}Risiede Malmantil sour' un poggetto,\\
E chiunque verso lui volta le ciglia\\
Dice, ch'i fondatori hebber concetto\\
Di fabricar l'ottava meraviglia,\\
L'ampio paese poi, ch'egli ha soggetto\\
Non si sa, vuo giuocare, a mille miglia;\\
V'è l'aria buona azzurre oltramarina,\\
E non vi manca latte di gallina.
\end{ottave}

Fatta la mostra, ed imbarcate in brevissimo tempo le soldatesche, si partirono
le Navi dal lido e fecero vela spiegando le loro insegne. Intanto Baldone dà
principio a narrare la causa, che lo muove a far la guerra di Malmantile, e comincia
dal descrivere la situazione, qualità, e dominio.

\begin{description}
  \item[FAR serra] Affrettare. In alzare. Vedi sotto \cstan[9]{13}.
\item[CONTARLA difesa, e plana] Intendi, Raccontarla puntualmente, e con
tutte le circostanze,
\item[NON si sa uno giuocare a mille miglia] Io giuoco, che non si trova chi sappia,
o possa giudicare a mille miglia, quanto paese gli è suggetto; perché è così gran
paese, che mille miglia non si considerano, essendo parvità di numero, e di materia
in riguardo del tutto, che gli è suggetto. E questa voce \textit{suggetto}, che vuol
dir \textit{sottoposto}, s'intende Situato sotto, e non sottoposto al dominio di Malmantile,
che per esser Posto nella sommità d'un poggetto, ha d'attorno molta pianura,
e colline sottoposte, cioè più basse di lui; se ben par, che voglia dire, che
Malmantile ha dominio immenso.

\item[ARIA azzurra oltramarina] I pittori dicono buon'aria quella, la quale e colorita
  con l'azzurro oltramarino, perché questo non perde mai il colore, come
  perde l'indaco, e lo smalto; ma è però anche vero, che quando l'aria si vede di
  colore azzurro, come è il buono oltramarino, è segno, che è purgata da ogni
  imperfezione di nebbia, o d'altri maligni vapori, e per conseguenza e aria buona;
  il Poeta però dice, che a Malmantile è aria azzurra oltramarina per intendere,
  che a Malmantile è aria, che dura sempre azzurra, come fa quella colorita
  con l'oltramarino, cioè sempre buonissima. E \textit{L'oltramarino} è quel colore, che si
  cava dalla pietra detta Lapislazzuli.
\item[NON vi manca latte di gallina] Vi sono tutte le cose squisite, è abondante d'ogni
  bene. Detto antico, si come si cava da Strabone lib, 14., dove discorrendo delle
  campagne di Samo dice, che erano così fertili, che si diceva comunemente,
  che producessero fino il latte di gallina, cioè quelle cose, che è impossibile, ch'altrove
  si trovino, come è il latte di gallina. \textit{Samus}, dice egli, \textit{feracissima, unde
laudantes non dubitant illud ei proverbium accommodare, quod ferat etiam Gallinae
lac}, ec.
\end{description}

\section{Stanza LXV \& LXVI}
\begin{ottave}
  \flagverse{65}Il Re di questo Regno giunto a morte\\
La mia Cugina qui, che fu sua Donna\\
(Non havendo figliuoli, o altri in Corte\\
Propinqui più) lasciò donna, e Madonna:\\
Ma come volle la sua trista sorte,\\
Un certo diavol d'una Mona Cionna\\
Figliuola d'un guidone ignudo, e scalzo\\
Ne venne presso a farie dar lo sbalzo.
\end{ottave}

\begin{ottave}
  \flagverse{66}Gobba, e zoppa è costei, e mancina,\\
Ha il gozzo, e da due sfregi il vifo guasto,\\
Scorse in Firenze ognor la cavallina\\
Ne i lupanari con gran pompa, e fasto,\\
E perché ossequij havea sera, e mattina,\\
E il titol di Signora a tutto pasto,\\
Fatta arrogante, al fine alzò il pensiero\\
A voler questi onori da dovero.
\end{ottave}

Narra Baldone, che il Re di Malmantile instituì Celidora erede del Regno, e
che questo le fu usurpato da Bertinella, la quale descrive per una donna tutta
contraffatta, e la mostra una vera sgualdrina: ed imita Dante nel Purg. C.19.
che dice.;
\begin{verse}
  Mi venne in sogno una femmina balba,
  Con gli occhi guerci, e sopra i piè distorta,
  Con le man monche, e di colore scialba.
\end{verse}

Qui è da considerare, che i tanti difetti da Baldone attribuiti a Bertinella,
realmente in lei non fussero, perché, ed egli non se ne farebbe innamorato, come
si dice sotto nel Cant. 9., ed ella non havrebbe havuto tanti altri amanti; Ma
Baldone non l'havendo mai veduta, e volendo concitar contro di lei odio di
quei soldati, che lo seguivano, per istigargli ad andar più volentieri alla ricuperazione
di Malmantile, la rappresenta loro una donna così nefanda.

\begin{description}
\item[SVA donna] Sua moglie, Se bene i Poeti dicendo La mia donna, o La sua
  donna, intendono l'amata.
\item[LASCIO' donna, e madonna] Termine notariesco, e curiale, che significa Padrona
  assoluta. Sincopato di Domina.
\item[VN certo Diavolo] Si dice così quando vogliamo esprimere uno, che è cagione
  di qualche nostra disgrazia: per esempio: \textit{Il negozio andava bene, ma un certo diavolo
    d'un Sensale con le sue chiacchiere lo rovinò} quasi dica \textit{Il diavolo, che guastò
    questo negozio, fu un Sensale}.
\item[MONA Cionna] È un detto di disprezzo, che significa Donna da poco in
ogni operazione: ed il senso della voce Mona, Vedrai sotto \cstan[5]{18}.
\item[GUIDONE] Intendiamo huomo vilissimo, abietto, senza roba, e senza creanza,
  o riputazione.
\item[DAR lo sbalzo] Mandar via; Scacciare.
\item[ORBO] In questo luogo vuol dir Uno, che vede poco, che noi chiamiamo
  lusco, se bene il suo vero senso è di cieco affatto. Vedi sopra in \cstan{9}.
  alla voce sbirciare.
\item[MANCINO] Uno che per assuefazione ha maggior forza, ed attitudine nella
  mano sinistra, che nella destra; E perché questo tale si può dire difettoso;
  perciò huomo mancino, vuol dire Huomo non buono; ed in questo senso è preso
  nel presente luogo. E però voce che ha del furbesco. Se ne servì il Lalli nella
  sua En. trav. nel C.2. stan. 40, dicendo,
  \begin{verse}
    Perch' io non fui mai orbo, ne mancino.
  \end{verse}
Ed al \cstan[4]{67}.
  \begin{verse}
    E riuscito in somma un buom mancino,
    Una delle più vili creature
    C' habbia sto mondo; e pazzo da catena;
  \end{verse}
\item[HA il gozzo] È parola nota, venendo dal latino guttur: Ma qui vuol dire
un gonfio, o scrofa, che vien nella gola, che i medici, che scrivono di simil
male pongono al trattato il titolo de \textit{Boccijs}.

\item[SFREGIO] Cicatrice di taglio nel viso. Ed una donna sfregiata è numerata
  fra le infami, e per la deformità del volto, e per la causa, per la quale si suppone,
  che le sia stato fatto. Vedi sotto \cstan[2]{3}. dove si mostra esser tali sfregi
  vituperosi anche negli huomini, ed al \cstan[6]{54}.

\item[SCORRER la cavallina] Pighiarsi tutti li suoi gusti liberamente, e senza riguardo
  alcuno. \textit{Havere scorsa la cavallina ne i lupanari}, vuoi dir, che era meretrice
  vecchia, ed avanzata ai bordelli, e lupanari. Gli antichi Egizj, quando volevano
  esprimere la sfacciataggine meretricia, figuravano una cavalla senza freno;
  il furore della quale nelle cose Veneree esprime Vergilio 3, Georg. dicendo.
  \begin{verse}
    Scilicet ante omnes furor est insignis equarum.
  \end{verse}
\item[IL titol di Signora a tutto pasto] Cioè continovamente era chiamata Signora.
  Termine usatissimo per intender voglia cosa, che si faccia molto, e continovatamente.
  Il Mauro\footnote{attribuito come ``Mauro'', autore di versi inclusi le opere burlesche di Francesco Berni et al.} nel Capitolo in lode della Torniella dice.
  \begin{verse}
    E ragionò di voi a tutto pasto
  \end{verse}
\item[DA dovero] Per debito, Per giustizia, Per merito. Intendi che volle proccurar
  d'havere stato, o signoria per meritare il titolo di signora, ec. ed osserva che quel
  \textit{da dovere} non è la voce \textit{vero} con l'aggiunta della sillaba do, ma è il nome
  \textit{dovere} messo in uso di dirlo così correttamente in casi simili a questo, e per
  esprimere una cosa di dovere o doverosa, e dovuta, e giusta.
\end{description}

\section{Stanza LXVII \& LXVIII}
\begin{ottave}
\flagverse{67}Così la mira ad alto havendo messa\\
A suoi Frustamattoni un dì ricorsa,\\
Bramar dice una grazia, e che in essa\\
Non si tratta di scorporo di borsa;\\
Ma, perché aspira a farsi Principessa,\\
Desidera da loro esser soccorsa\\
Col loro aiuto, volendo, e consiglio,\\
Provar, s'a Malmantil può dar di piglio,

\flagverse{68}Pronto è ciascuno, e vuol tra mille stocchi\\
Esporre il ventre, e come un Paladino,\\
Che per servire a Dame, tali allocchi\\
Cercan l'occasion col fuscellino;\\
Ma non si parli, o tratti di baiocchi,\\
Perché non hanno un becco d'un quattrino;\\
E credon, promettendo Roma, e Toma,\\
Di spacciar l'oro della bionda chioma.
\end{ottave}

Bertinella havendo fatta la suddetta risoluzione, richiese li suoi amanti, che
la volessero aiutare a farsi Principessa con impadronirsi di Malmantile, ed i suoi
Drudi s'esibiscono a servirla, perché sentono di non haver a spendere, il che è
cercato da tutti coloro, i quali con simil donne pretendono di passar per belli,
che è una delle tre specie di persone, che voglion queste femmine d'intorno, cioè
Il bello per sua propria sodisfazione. Il bravo per farsi rispettare. Ed il ricco
minchione, o corrivo, per cavar danari da lui, per campare se medelime, ed i
primi due, Il Persiani\footnote{} dice,
\begin{verse}
Il bravo, ed il corrivo, ed il valente.
Nella mia Mea fallisce
Questo antico dettato
Per c' al bravo, ed al bel non apparisce,
Ma sol vorrebbe il suo minchione allato.
\end{verse}

\begin{description}
\item[PORRE ad alto la mira] Aspirare a cose grandi. Mira si dice quel segno, che
  è nella canna dell'archibuso, o nelle balestre, nel quale s'affissa l'occhio per aggiustare
  il colpo al berzaglio. E di qui \textit{Porre la mira a una cosa} s'intende \textit{Volgere
    il pensiero}, o \textit{aspirare a una cosa}.
\item[FRVSTAMATTONI] Si dicono Quelli, che giornalmente vanno in una
  casa, o bottega, e non vi spendono mai un soldo, o vi portano utile alcuno,
  E si dicono Frustamattoni, perché non son d'altro giovamento, che frustare,
  cioè spazzare, e ripulire con le scarpe i mattoni; i quali son quelle lastre fatte di
  terra cotta, con le quali si lastricano i pavimenti delle stanze, da i Latini detti
  \textit{Lateres}.
\item[SCORPORO di borsa] Spendere. Scorporare vuol dit Estrarre da una massa, o
  da un corpo, o quantità di roba, o una porzione di essa.
\item[DAR di piglio] In questo luogo vuol dir Pigliare, impadronirsi; ed alle volte
  vuol dir Principiare come sotto C.6, stan 60.
\item[ESPORRE il ventre a mille stocchi] Vanti d'innamorati d'andare  soli contro
  a un'esercito intero, come i Poeti favoleggiano, che facessero i Paladini, che
  sono quei dodici Conti di Palazzo, ordinati da Carlo Magno per combattere
  contro a i nimici della S, Fede Cattolica, che furono detti \textit{Comites Palatini}, cioè
  Compagni nel Palazzo, che sono forse gli odierni Pari di Francia: the noi poi
  corrottamente chiamiamo Paladini, e con questa voce intendiattio. Haomé bravo.
\item[ALLOCCO] Specie d' uccello con il capo cornuto, come l'assiuolo, ma è
  più grande, e di colore lionato, con occhi grandi, e lucenti, È animal goffo,
  e se bene vive di rapina, tuttavia è tanto poltrone, che per cibarsi aspetta di pigliare
  gli uccelli, quando gli vanno scherzando attorno, tratti dalla di lui goffaggine;
  e quando se li avvicinano, non con rapacità, ma con flemma, e gravità
  non ordinaria gli prende col rostro, o con gli artigli; E da questa goffaggine
  nel far all'amore, ed aspettare gli uccelli, per Allocco intendiamo Uno, che
  se ne stia perdendo il giorno in vagheggiar Dame senza profitto, ed è lo stesso
  che \textit{Frustamattoni}, \textit{Colombi di gesso}, e simili.
  Con questo nome di \textit{Allocco} in molte parti d'Italia è chiamata ancora la Civetta,
  e credo, perché è di figura, se ben più piccola; simile a quella dell'Allocco, e
  vive con le medesime arti.
\item[CERCAR col fuscelino] Cercar minutamente, e con diligenza; \textit{Il tale cerca le
busse col fuscellino} vuol dire; Il tale fa tutto quel che egli può, per esser percosso,
  o per toccarne. Questo detto vien da quei ragazzi dell'infima plebe, i quali dopo
  che è venuta in Firenze una gran pioggia, che habbia fatta correr l'acqua
  per la Città, vanno cercando per le strade vicine alle gran fogne, che portano in
  Arno, se trovano fra le commettiture delle lastre delle strade spilli, chiodi, ed
  altre cose simili portate, e lasciate quivi dall'acque correnti; e per far ciò si
  servono d'uno stecco, o fuscelletto di scopa, o d'altro, col quale vanno rifrugando
  i fessi di dette commettiture, e perché così gran diligenze son troppe al
  poco utile, n'è nato il suddetto proverbio, che ha l'acceanato senso, ed è lo
  stesso che chiamar' una cosa di la da i monti, detto sopra in questo C, stan.\ 19.
\item[BAIOCCO] E parola, e moneta romana, la qual parola è talvolta usata da
  noi per intender Danari, come qui, che dicendo \textit{Non si parli di baiocchi} intende
  \textit{Non si parli di danari}, cioè di Spendere.
\item[NON hanno un becco d'un quattrino] Non hanno pure un denaro, e quella parola
  Becco si mette a maggiore espressione, quasi dica Non hanno ne pure un sol
  \textit{quattrino becco}; cioè cattivo, e non il caso a spendersi; Se non volessimo dire,
  che venisse questo detto dall'antica moneta Romana di rame; nella quale era impresso
  da una banda il volto di Giano con le corna, e dall'altra un rostro di nave,
  e che il dire; Un becco d'un quattrino sia lo stesso, che dire, ne anche la
  parte d'un quattrino, cioè  la faccia di Giano, che è cornuta.
\item[PROMETTE Roma e Toma] Promette cose grandissime, e che da persona
  alcuna non si possono mantenere, o osservare; i Latini dissero \textit{Maria, Montes polliceri},
  La voce toma non so che habbia nel nostro idioma significato alcuno, e stimo;
  che sia usata in questo detto per darle la rima con la parola Roma; Se forse
  non fusse il verbo spagnuolos tomar, che vuol dir torre, o pigliare, ed intendersi
  \textit{Ti prometto Roma}, (che è a dir tutto il mondo) \textit{e tu toma}, cioè piglia quel che
  ti piace. Lasca Nov. 8. \textit{Però non restava, di sollecitarla promettendole Roma, e toma,
    come se egli fusse il primo Principe del mondo}.
\end{description}

\section{Stanza LXIX, LXX \& LXXI}

\begin{ottave}
  \flagverse{69}Era tra molti suoi più fidi amanti\\
Un ciarlon, che però detto è il Cornacchia,\\
Ed è di quei pittor, ch' i viandanti\\
Con lo stioppo dipingono alla macchia;\\
E perché nella lingua ha il suo in contanti,\\
Molto si vanta, assai presume, e gracchia;\\
E finalmente colorisce, e tratta\\
Questo negozio, come cosa fatta.
\end{ottave}

\begin{ottave}
  \flagverse{70}Scrive un viglietto poi segretamente\\
Ad un compagno suo capobandito,\\
Dicendo, che veduta la presente,\\
Il suo bagaglio subito ammannito,\\
Di notte tempo meni la sua gente\\
A Rimaggio alla Svolta del Romito;\\
Ma vada alla spezzata, e pe i tragetti,\\
E senza pensar' altro ivi l'aspetti.
\end{ottave}

\begin{ottave}
  \flagverse{71}Andò la carta, e quei c'hebbe l'intesa,\\
Come quel ch' invitato era al suo giuoco\\
Andonne, e guidò seco a quell'impresa \\
Cent'huomin con le lor bocche di fuoco,\\
Quivi il Cornacchia, e quella buona spesa\\
Di Bertinella giunsero fra poco,\\
Anch'eglino con grossa, e folta schiera\\
D'una gente da bosco, e da riviera.
\end{ottave}

Fra questi suoi più fedeli amanti era un tale detto il Cornacchia. Costui era
uno con tal soprannome; perché havea la voce d'un suono simile al gracchiare
della cornacchia, ed era un solennissimo briccone, e ladro, e spia. Questo da a
Bertinella il negozio per fatto, e s'ammannisce a far la sorpresa di Maimantile;
con scrivere ad un capo di ladri da strada suo corrispondente, che si conduca a
Rimaggio con le sue genti con armi, e panni, e l'aspetti alla Svolta del Romito,
che è una contrada in vicinanza di Malmantile. Eseguì l'amico, giunse
con cento huomini ben' armati nel luogo ordinatogli: fra poco vi arrivò ancora
il Cornacchia con Bertinella, con grande schiera di bravi furbi, che questo intende
\textit{gente da bosco, e da riviera}; che i Latini dissero \textit{homines omnium horarum}.

\begin{description}
\item[CIARLONE] Uno, che chiacchiera assai, L'Autore intende, che chiacchierava
  assai alla giustizia, cioè faceva la spia, e perciò detto Cornacchia, che è uccello
  di cattivo augurio; perché il suo ciarlare era di danno al prossimo. Ed in
  vero costui, mentre visse, fu sempre chiamato il Cornacchia, o per questa causa,
  o per quella che habbiamo accennato sopra.
\item[DIPINGERE alla macchia] Dipinger un Ritratto senz'haver d'avanti l'originale,
  ma col solo haverlo veduto. E l'Autore però intende, che egli era ladro di strada,
  e pigliando la voce macchia nei suo vero senso di selva densa, dice,
  che alla macchia ritraeva i viandanti con lo stioppo, ed intende Assaltava la gente alla
  strada con l'archibuso per rubarla, Questa però è finzione, perché il Cornacchia,
  se hebbe la malizia, non hebbe già tanto cuore di far' il ladro di strada, e l'Autore
  lo finge tale per mostrare, che egli era un furbo da far qualsivoglia sciagurataggine.
\item[HA nella lingua il suo in contanti] Vuol dire eloquente, pronto di lingua.
\item[VANTARSI] Promettersi molto di se medesimo, Esaltar le proprie opere,
è il Latino \textit{Iactare}.
\item[GRACCHIARE] Cicalare con poco fondamento, Vedi sotto C. 4. stan 29.
  \cstan[7]{9}, e \cstan[8]{65}. Ma perché costui è chiamato Cornacchia, il Poeta si
  serve del verbo gracchiare per esprimer il cicalar di esso.
\item[COLORIRE] Metafora assai usata, e vuol dire discorrer d'una cosa con aggiustatezza,
  con termini proprj, e con colori rettorici per persuadere, e fare
  apparir vera quella tal cosa, della quale si discorre.

\item[VIGLIETTO] o \textit{biglietto}. Vuol dir lettera; Ma strettamente significa quella
lettera, che si manda in luoghi vicini, come da una casa all'altra, dentro alla
medesima Città, o Terra. Voce che forse viene dal Francese \textit{Poulet}, che vuol dir
lettera, amorosa, o da \textit{Billet}, Vedi sotto \cstan[6]{54}.

\item[BAGAGLAIO] Quelle some, che si conducono appresso gli eserciti per utile, e
  comodo dell'armata, o dietro qualsivoglia viaggiante per servizio della propria
  persona; si dicono \textit{Bagaglio}, forse dal Francese \textit{Bagage}; o dal verbo Bainlare,
  che val Portare, come habbiamo osservato sopra in \cstan{62}. alla voce
  Baule, ed è quel che i latini dicevano \textit{impedimenta}.

\item[AMMANNIRE] Metter'all'ordine, Allestire, approntare; quasi dica \textit{ad
  manus habere}. Dante Purg. C. 23.  \begin{verse}
    Di quel ch'il Ciel veloce loro ammannna,
\end{verse} ed al C. 29.\begin{verse}La virtù, c' a ragion discorso ammanna.\end{verse}

\item[ALLA spezzata] A pochi insieme per volta, non in squadre o truppe formate.
  Si dice anche \textit{Alla sfilata}, Vedi sotto \cstan[6]{85}. ed è il \textit{diminutim}
  dei latini.

\item[PE i tragetti] Per le balze, per luoghi, e strade non praticate; e il puro Latino
  \textit{Traiectus}.

\item[HAVER l'intesa] Rimaner d'accordo. Haver l'instruzione di come si debba contenere.
\item[INVITAR uno al suo giuoco] Chiamar' uno a fare una cosa, che sia di suo genio,
  e gusto. I Latini dissero \textit{Musas hortari ut canant}, ec.

\item[BOCCHE di fuoco] Intendiamo Ogni arme da fuoco, atta a portarsi addosso,
  come Moschetti, archibusi, pistole, e simili.
\item[BVONA spesa] Huomo astuto, e scaltrito, e suona lo stesso, che Tristo,
  e Volpe vecchia.
\end{description}

\section{Stanza LXXIL \& LXXIIT.}
\begin{ottave}
  \flagverse{69}Dopo ch' insieme tutti fur costoro\\
Si fece de' più degni una semblea,\\
Del come discorrendo fra di loro\\
Sorprender' il Castello si dovea,\\
Ond'il Cornacchia in mezzo al concistoro\\
Rizzato in pié con gran prosopopea,\\
Ed una toccatina di cappello,\\
In tal modo cavò fuora il limbello.
\end{ottave}

\begin{ottave}
  \flagverse{69}Io so c'a un'ignorante, a un'idiota\\
L'esser il primo a favellar non tocaa;\\
Ma perdonate a questa zucca vota,\\
Signori, s'io vi rompo l'huovo in bocca;\\
Scricchiola sempre la più trista ruota,\\
Così la lingua mia più rozza, e sciocca\\
V'infastidisce, è ver ma v'assicura,\\
Che Malmantile è nostro a dirittura.
\end{ottave}

Ragunati costoro insieme, quei più degni si ristrinsero a consiglio, per fermar
il modo, che si doveva tener per sorprender Malmantile, ed il Cornacchia, fatte
sue cirimonie, comincia a mostrare il modo certo di pigliare detto Malmantile.
\begin{description}
  \item[PRESOPOPEA] Questa voce, che vien dal Greco Prosopopea compostasdi
due dizioni \textit{Prosopon}, che suona \textit{personam} (ed a noi Personaggio) e poeeo, che
suona \textit{facto}, se bene è una figura con la quale fingesi un perlonaggio, come
farebbe introdurre una cosa inanimata, che parli con una animata, \& è contra, tuttavia
noi ce ne serviamo per intender una certa superbia, arroganza, fasto, o
presunzione di se medesimo, dimostrata con gli atti; di che vedi sorto C.6. stan. 85.
Ed in tal senso, secondo il Monosino era pigliata ancora da i Greci. Si dice
da noi anche sussiego, derivando la voce dallo Spagnuolo.
\item[VNA toccarina di cappello] Atto che esprime detta Prosopopea.
\item[CAVÒ fuora il limbello] Cominciò a parlare. Limbelli; Si dicono quei pezzi
  di pelle di bestia, che dalle dette pelli tagliano i Conciatori, donde poi
  \textit{limbellucci} i ritagli delle pelli più sottili, come di cartapecora, che servono per far
  colla da Pittori. E perché tali \textit{limbelli}, quando son freschi; ed umidi sono simili alle
  lingue, perciò per \textit{limbello} intendiamo lingua; e però detto scherzoso, come si
  vede, che l'usò il nostro Autore anche sopra in quella sua lettera alla Sereniss.
  Arciduchessa, riportata da me nel Proemio. \textit{Cavò fuora il limbello, e disse le sue
    Sillabe, come un Tullio}, ec.
\item[IGNORANTE, \& idiota] Sono Sinonimi, ne vi si fa alcuna differenza, se
  bene strettamente \textit{Ignorante} vuol dire uno, che non sa nulla, e \textit{Idiota} par che si
  convenga a coloro, che non hanno cognizione di lettere.
\item[ZVCCA] S'intende il capo dell'huomo per la similitudine, e Zucca vera vuol
  però dire testa senza cervello, che si dice \textit{vota di sale}, o non haver sale in zucca.
  E questo perché è solito nelle cucine tenere il sale in una Zucca secca appesa al
  muro del Cammino. Vedi sotto Can. 4. stan. 15. I Latini pure dicevano \textit{sale} per
  giudizio, e trovasi in Catullo.
  \begin{verse}
    Nulla in tam magno corpore mica salis
  \end{verse}
Vedi sotto \cstan[8]{26}., e Marziale C. 7.
  \begin{verse}
    Nullaque mica salis, nec amari fellis in illis
  \end{verse}
\item[ROMPER l'huovo in bocca] Torre la parola di bocca a uno, ciò è Dire che
  doveva, o voleva dire un'altro. Terenzio disse \textit{Bolus ereptus e faucibus est}.
\item[SCRICCHIOLARE] Stridere, strepitare. S'intende quel romore, che fa
  nel muoversi un legno fortemente stretto, o aggravato da altro legno, o materiale
  duro; come appunto segue nelle ruote da carro. Ed il proverbio: \textit{Sempre
    Scricchiola la peggio ruota del carro}, Significa \textit{Il più sciocco della conversazione, vuol
    sempre parlare}, Detto antico, e vien dal Latino, che dice \textit{semper deterior
    vehiculi rota perstrepit}, ec.
\item[A DIRITTVRA] Cioè assolutamente, sicuramente, e senza difficultà aleuna,

\end{description}

\section{Stanza LXXIV.}
\begin{ottave}
  \flagverse{74}Credete a me: Ciascun si stia nascosto\\
In queste macchie, in questi boschi intorno\\
Ed io da voi fra tanto mi discosto,\\
Ne questa notte farò più ritorno.\\
Rivedremci colà doman sul posto,\\
Perché vicino al tramontar del giorno\\
Vi farò cenno, hor voi ponete mente,\\
E poi venite via allegramente.
\end{ottave}

\begin{ottave}
  \flagverse{75}Parte il Cornacchia, e corre presto presto\\
Da certi suoi amici contadini,\\
Da' quali le lor bestie piglia in presto\\
E carica più some di buon vini,\\
E di soppiatto, come fante lesto\\
Cavò di tasca certi cartoccini\\
Pieni d'alloppio, e dentro al vin li pone\\
Quello impepando, senza discrezione.
\end{ottave}

\begin{ottave}
\flagverse{76}Così carreggia, e giunto a Malmantile\\
All'aprir della porta la mattina\\
Scarica in piazza il vino, ed un barile\\
A regalar ne manda alla Regina.\\
Poi vende il resto a prezzo tanto vile,\\
C'ognun ne compra, e in fin che n'ha in cantina\\
Per rivenderlo altrui, il fiasco attacca,\\
Si cala al buon mercato, a quella macca
\end{ottave}

\begin{ottave}
\flagverse{77}Due, o tre fiaschi davane a quattrino,\\
Ed a' poveri davalo a Isonne,\\
Tal che tutti tuffandosi a quel vino\\
S'imbriacaron come tante monne,\\
E subito dal grande al piccolino\\
Tanto de gli huomin, quanto delle donne\\
Cascaro in sonnolenza sì gagliarda,\\
Che desti non gli havrebbe una bombarda.
\end{ottave}

Cornacchia instruisce i compagni di quello devon fare, e si parte, e va da,
certi contadini suoi amici, da' quali piglia le lor bestie in presto, e lo carica di
vino alloppiato, quale porta in Malmantile, e lo vende così a buon mercato, che
Ognuno ne comprò, e bevvero tanto, che tutti s'imbriacarono, e si messero a dormire

\begin{description}
  \item[PRESTO presto] Prestissimo: per la replica d'una stessa parola, che ha forza di
superlativo, come habbiamo detto altrove.
\item[DI soppiatto] Di nascosto. Vien dal verbo impiattare, che vuol dir Nascondere
  una cosa corporea, come s'è detto altrove.
\item[FANTE lesto] Huom sagace, astuto, e che sa il conto suo.
\item[CARTOCCINO] Diminutivo di Cartoccio, che è una piegatura di foglio, fatta
a Piramide usata da gli speziali per mettervi dentro zucchero, pepe, ed altro simile.
\item[ALLOPPIO] Specie di sonnifero composto di sugo di papavero, coagulato,
secco, e polverizzato, e d'altri ingredienti; e si chiama \textit{oppio}.
\item[CARREGGIARE] Venendo da carro dovrebbe intendersi solamente per Camminar
  col carro, o traghettar robe col carro, ma ci serve per lo più per intender
  ogni sorte d'andare, o camminare, a piede, o a cavallo, conducendo o non
  conducendo roba.
\item[BARILE] Vaso di legno per uso di portarvi olio, vino, ed ogni altro liquore
  simile, ed è la misura comune del vino, capace di 20. fiaschi, e quello da olio
  di 16 fiaschi. Tali vasi son composti, ed aggiustati in maniera da adattarne due
  per volta addosso a una bestia da soma.
\item[ATTACCA il fiasco] Coloro, i quali in Firenze vendono il vino a fiaschi alla
  propria casa, attaccano per segno di ciò sopr'alla porta un fiasco, acciò che il
  popolo vegga il luogo, dove si vende il vino: e pero quando si dice \textit{Il tale ha oggi
  attaccato il fiasco}, s' intende, \textit{il Tale oggi ha cominciato a vendere il vino a fiaschi}.

\item[SI cala a buon mercato] Si lascia persuadere dal prezzo vile a comprare. È
  traslato da gli uccelli, che si calano alla vista della preda.
\item[MACCA] Abbondanza grande. Vien forse dal Latino Mactus, che s'intende
  abbondanza grande, quasi \textit{Magis auctus}. Plau, milit, 4.22. \textit{Macte amare}. E
  si trova \textit{Puer macte virtute}; giovanetto virtuosissimo. Dice il Vocabolista
  Bolognese, che macco vuol dir' abbondanza, che induce disprezo, e così è vero nel
  parlar nostro, che si dice \textit{smaccare} per intender Vituperare, o screditare.
\item[A Isonne] Per niente. Senza spesa, È detto plebeo, ed è usato per lo più tra
  i battilani, i quali hanno per tradizione, che Isonne fusse già un'huomo de' loro,
  il quale mangiava tanto volentieri a spese d'altri, che essendo morto, e seppellito
  già di qualche mese, scappasse dell'avello al discorso, che da alcuni si faceva
  di voler dar mangiare a tutti i Battilani per tre giorni, senza che spendessero,
  Costui havea due fratelli l'uno detto Salicone, e l'altro lo Scrocchina, e però
  \textit{scroccare} mangiare a \textit{Salicone}, a \textit{Scrocco}, e a \textit{Isonne} significano tutti Mangiar senza
  spendere, che Terenzio disse \textit{Asymbolum} composto dalla proposizione A, che
  suona Senza, e \textit{symbolum}, che vale quota, o scotto, e significa senza denari; E si
  come ne i Latini questo \textit{Asymbolum}, fu usato da i parassiti, e guatteri, così il nostro
  \textit{Isonne}, è usato dalla plebaglia, fra la quale è nato.

  Può anch' essere, che questo detto \textit{Isonne} venga da un Iiogo poco fuori di Firenze
  detto \textit{Isonne}, dove anticamente andavano a desinare aicune volte l'anno
  molti battilani, senza spendere, non perché veramente non spendessero, ma perché
  il denaro, che si spendeva in quel desinare, era di mance fatte per le Pasque,
  S. Giovanni, e Carnevale, che messo in una lor corbona, si serbava, e distribuiva
  per questi desinari; e può essere, che questi battilani dessero tal nome
  \textit{Isonne} a quel luogo dove andavano a far questi lor desinari, chiamati da loro
  \textit{desinari a Isonne}; ma sia come si voglia, basta che appresso noi il termine \textit{Isonne} è
inteso per Senza spesa.
\item[TVFFANDOSI] Tuffarsi a una cosa, significa Pigliare, o fare assai una tal cosa.
\item[S'imbriacaron come tante monne] Vedi quel che s'è detto sopra in \cstan{10}.
\end{description}

\section{Stanza LXXVIIL}
\begin{ottave}
  \flagverse{78}Quando il Cornacchia vedde il suo disegno\\
Già riuscito, andò sopr'alle mura,\\
Ed ai compagni fece il detto segno,\\
Che bene havendo al tutto posto cura,\\
Saliro al poggio senz'alcun ritegno,\\
Senza sospetto haver, senza paura\\
Dietro al Cornacchia lor guidone, e scorta\\
Dentro al Castello entraron per la porta
\end{ottave}

\begin{ottave}
\flagverse{79}E perc' ognun dormiva, come un Tasso,\\
La donna fece farne una funata,\\
E condursegli a piedi a baciar basso,\\
E renderle il tributo ognun pro rata,\\
A Celidora poi restata in Nasso,\\
Cioè da' suoi vassalli rinnegata,\\
Già che tutti voltato havean mantello,\\
Comandò che baciasse il chiavistello.
\end{ottave}

\begin{ottave}
\flagverse{80}Ell'ubbidì, temendo, ancor di peggio,\\
E ben che fusse un pezzo in la di notte,\\
Il pigliarsene subito il puleggio\\
Un zucchero le parve di tre cotte.\\
Così finito il solito corteggio\\
Con due strambelli, e un par di scarpe rotte\\
Triffa, e strascina poi per la boccolica\\
Un tozzo mendicava all'accattolica
\end{ottave}

I Compagni di Bertinella veduto il segno dato dal Cornacchia, andatono a
Malmantile, ed entrati dentro, e trovati tutti a dormire gli legarono, e gli condussero
a render ubbidienza a Bertinella, la quale comandò a Celidora, che uscisse
del Castello, ed ellam tutta mal' all'ordine se n'andò, benché fusse assai di notte,
e si condusse a mendicare il vitto.
\begin{description}
  \item[GVIDONE, e scorta] Guidone s'intende Colui che guida; e Scorta è quello che
mostra la strada; ma la voce \textit{Guidone} è forse per scherzo presa dall'Autore nel
senso, che sopra stan. 65. e sotto al Cant, 8. stan.~72.
\item[FAR una funata] Legar con una fune più persone: Quando molti insieme
  commettono un delitto, si suol dire: \textit{Se vengono i birri, voglion far la bella funata}.
  Non perché crediamo, che vogliano effettivamente legargli tutti a una fune, ma
  intendiamo, \textit{Vogliono farne molti prigioni}, e così intendi nel presente luogo.
\item[BACIAR basso] Cioè inchinarsi a baciar i piedi in segno di vassallaggio.
\item[RIMANERE in Nasso] Dai più si dice \textit{rimanere in Asso}, e ciò segue per
  corruzione nella pronunzia, che tanto suona \textit{rimanere in asso} che \textit{rimanere in Nasso}
  come si dovrebbe dire, e significa abbandonato, senza aiuto, e senza consiglio;
  Ed è derivato dalla favola d'Arianna abbandonata da Teseo nell'Isola di Nasso;
  E si dice anche rimanere in su le secche di Barberia, il che corrobora che si debba
  dire \textit{in Nasso}, e non in asso che non ha verun senso, o allegoria. Vedi sotto
  C.\ 10.\ stan.~2.
\item[VOLTAR mantello] Rinnegare. Ribellarsi; andar da un partito all'altro. Il
  Lalli En. trav. C. 2, stan.~39.
  \begin{verse}
    Hor che mi lice di voltar mantello
  \end{verse}

\item[BACIARE il chiavistello] Andarsene senza speranza di tornare. Usiamo questo
  detto per esprimere che non si vuole, che quel tale, che è stato per li suoi
  mali portamenti scacciato d'una tal casa, viva con la speranza di ritornarvi, e
  pero si potrebbe dir con Vergilio \textit{Supremum vale dixit}.
\item[CHIAVISTELLO] Serratura da porte, o finestre, che confiste in un ferro
  lungo, il quale fa la sua operazione, passando per diversi anelli pur di ferro
  adattati nel legname; ed è il Latino \textit{vectis}.
\item[PIGLIAR il puleggio] Andar via. Pigliar il cammino, E' frase marinaresca, ma
  però usata comunemente in questi termini d'andar via presto. Dante Par. C. 23.
\begin{verse}
  \backspace Non è puleggio da piccola barca
  Quel che fendendo va l'ardita prora
  Ne da nocchier, c' a se medesmo parca.
\end{verse}

Da questa voce Puleggio viene \textit{spulezzare}, che vedremo sotto \cstan[7]{18}, che
pure significa Andar via. Forse si potrebbe dir anche \textit{prueggiare} verbo pur
marinaresco, che significa Andar via bel bello.

Vincenzio Tanara nella sua Economia del Cittadino in villa Lib. 6. trattando
dell'erba \textit{Puleggio} dice, che sparsa in luogo dove sieno pulci ha virtù di
scacciarle; onde può essere che da questo effetto dell' erba \textit{Puleggio} venga il presente
dettato. Da \textit{puleggio} forse anche vengono \textit{Pulegge}, che sono quelle piccole
girelle, che si congegnano, ne i legni per facilitare i veicoli, come farebbe dentro a i
regoli da piede alle scene, o prospettive da commedie per renderle più facili a
strascicarsi dentro a i canali in occasione di mutazione delle medesime scene.
\item[UN suechera le parne di tre cotte] Le parve d' haverla a buon mercato: le parve
  d'haver fortuna grandissima, perché s'aspettava malto peggio. Lo Zucchero
  di tre cotte fatte bene si stima che sia il miglior grado di perfezione, della
  quale sono tre i gradi. secondo il detto \textit{omne trinum est perfectum}. Ed i Franzesi
  denominano il superlativo col tre, cioè buono, for buono, e tre buono\footnote{bien, \textit{fort} bien, \textit{très} bien}, per
  buono, molto buono, buonissimo,:

\item[STRAMBELLE] Vesti vecchie, e stracciate. Vedi sotto C, 3., stan.~65.
\item[UN tozzo] Detto così assolutamente senz' altra aggiunta vuol dire un pezzo di pane.
  E \textit{frustum panis}, che usò Dante nel Parad. C. 6. \textit{Mendicando sua vita a frusto a frusto}.
\item[TRISTA, e strascina] Huomo tristo vuol dire Huomo mal vestito, e Strascino
  suona quasi lo stesso, perché Strascini chiamiamo alcuni huomini, i quali vanno
  comprando carne fuori della Città, e l'introducono in Firenze occultamente per
  rubarne la gabella, e perché costoro son sempre unti, sudici, e stracciati, perciò
  dicendosi \textit{Strascino} intendiamo mal' all'ordine di vestito, ec.
\item[BOCCOLICA, e accattolica] Sono due parole dette per scherzo, e per la similitudine
  che hanno con Bocca, e con Accattare, e per parlare Ianadattico, non
  sono però fuori dell'uso della gente più Civile, la quale spesso si serve di parole
  latine a quel proposito, che le pare che facciano giuoco stroppiandole, e interpretandole
  a lor modo, come le presenti \textit{Boccolica}, e \textit{accattolica} che l'una vuol dir Bocca,
  e l'altra Accattare, e così intendesi che Celidora accattava per mangiare. Tal'uso
  d'allusione scherzosa era pur'anche appresso ai Latini trovandosi \textit{Ab Ilio nunquam
  recedis}, che par che voglia dire tu non ti parti mai dalla Città di Troia, e
  s'intende poi; tu non abbandoni mai l'Ilo intestino, cioè sempre mangi.
\item[MENDICARE] Vuol dire durar fatica a conseguire. \textit{Il tale mendica le parole},
  cioe Dura fatica a parlare; ma il suo significato più inteso è Chiedere elemosina,
  Dante Parad. C. 6.
  \begin{verse}
    \backspace Indi partissi povero, e vetusto,
    E s'il mondo sapesse il cor ch' egli hebbe,
    Mendicando sua vita a frusto a frusto, ec.
  \end{verse}
\end{description}

\section{Stanza LXXXXI, LXXXII \& LXXXIII}
\begin{ottave}
\flagverse{81}In tanto Bertinella del Reame\\
Garbatamente fecesi padrona,\\
E de' villaggi, e d'ogni suo bestiame\\
Prese il possesso in petto, ed in persona\\
Poi per letizia cavalieri, e dame\\
Regalò di confetti, e di pattona;\\
E segue ogn'anno di mandarne attorno,\\
Per la dolce memoria di quel giorno.
\end{ottave}

\begin{ottave}
\flagverse{82}Tosto che ci hebbe fitto il capo, volle\\
C'ognun serrasse il traffico, e il negozio,\\
Donando a ciascheduno entrate, e zolle,\\
Acciò se la passasse da buon sozio,\\
Ed allegro, a piè pari, ed in panciolle\\
Senza briga vivesse in pace, e in ozio,\\
Ognun vi s' arvecò di buona gana,\\
Che la poca fatica a tutti è sana,
\end{ottave}

\begin{ottave}
\flagverse{83}Così mai sempre in feste, ed in convito\\
Tirano innanzi questi spensierati;\\
Ne moverebbon per far nulla, un dito,\\
Ben ch' ei credesson d' esser' impiccati;\\
Non teme della Corte, chi e fallito,\\
Che tutti i giorni a lor son feriati;\\
Non v'e giustizia, ne il bargel va fuora,\\
Se non per gastigar chiunche lavora.
\end{ottave}

Sbandita Celidora dal regno, Bertinella prese l'attual possesso di tutto lo stato,
e per acquistarsi la benevolenza de' sudditi cominciò dal regalare le dame, e
cavalieri, con regali degni della vilissima condizione di se medesima, ed appropriati
alle qualità de' Cavalieri, e Dame di Malmantile; poi con feste, ed allegrie
per contentare il popolo, e con levare i Ministri della giustizia tanto odiosi alla
plebaglia, e con fare altri ordini che si leggono nelle presenti ottave.

\begin{description}
\item[IN petto, ed in persona] Attualmente, e corporalmente. \textit{Animo \& corpore}.
\item[PATTONA] Torta, o pane fatto di farina di castagne, con altro nome
  detto \textit{polenda}, dal Latino \textit{Polenta}, che era vivanda fatta di farina d'orzo con
  altre polveri odorifere secondo Varrone. È vivanda vilissima appresso di noi; e
  da questa sua viltà habbiamo un detto di disprezzo, che è; \textit{Mangiapattona},
  \textit{Mangiapolenda} a un huomo vile, e buono a poco. Qual detto usò Plauto chiamando
  questi tali \textit{Pultiphagj}; ma il disprezzo non nasceva dalla viltà della \textit{polenta},
  (che era finalmente il cibo comune anche per le persone di garbo, e generalmente
  mangiando questa sorte vivanda i Romani vissero lungo tempo, Vedi Plin.
  \libcap[18]{8}.) nasceva bene dall'intendersi con tal detto un huomo buon'a
  poc'altro, che a mangiare, e come noi diciamo \textit{Sparapani}, \textit{Votamadie},e simili

\item[V'hebbe fitto il capo] Se n'era impadronita: N'haveva preso l'attual possesso;
  perché essendo il capo la più nobile, e principal parte della persona, noi diciamo
  \textit{Ficcare il capo in un luogo} per intendere Entrare in un luogo, e pigliarne il
  possesso personalmente.
\item[TRAFFICO] e negozio. Sinonimi, se bene \textit{traffico} par, che si ristringa all'arti
  manuali; onde con dire \textit{Traffico}, e \textit{negozio} intende non lavorare, ne
  mercanteggiare, o negoziare.

\item[ZOLLA] È il Latino gleba, che vuol dire Pezzo, o massa di terra smossa,
  come s'è accennato sopra in \cstan{57}., ma qui pigliando la parte per il
  tutto, intende terreni fruttiferi: \textit{Il tale ha delle zolle}, comunemente s'intende
  Ha de' terreni.

\item[SOZIO] Dal latino \textit{Socius}. Compagno \textit{Viver da buon sozio} vuol dir Viver
  da buon compagno, alla reale, ed alla schietta. E questa voce Sozio non so che
  sia usata se non in questo caso, e con l'aggiunta di \textit{buono}, o \textit{malo}: dicendosi
  Il tale è buon sozioxe, o \textit{non è mal sozio}, per intendere E' galant'huomo.

\item[A piè pari, ed in panciolle] S'usa questo detto per esprimere Un huomo poltrone,
  che non voglia far'altro, che godere i suoi comodi, e la voce \textit{panciolle}
  è composta di due parole, cioè \textit{pancia}, ed \textit{olle}, e suona pancia di pentola, la quale
  col posar pari, e con quella sua gran pancia è il vero ritratto della: comodità, e
  poltroneria. Il Bronz. nel Cap. in lode della Galea dice.
  \begin{verse}
    \backspace Guarì, ma in capo al giuoco, come volle
    Il Ciela, ne fu tratto il poverino,
    E fu privato di stare in panciolle.
  \end{verse}

\item[BRIGA] Noia, fastidio, fatica. Qui è preso per faccenda, o pensiero d'operare.

\item[DI buona gana] Molto volentieri. È detto spagnuolo, e la voce gana è usata da
  noi per intender Voglia, o gusto grande. \textit{Il tale mangia di gana}; \textit{Lavora di gana}, ec,

\item[SCIOPERATO] Uno che non ha, e non vuole haver faccende. Vedi sopra,
  stan. 29. Scioperati s'intendono quei Cittadini, che senza arte, o impiego vivono
  con le loro entrate.
\item[CORTE] Intendi la Corte della giustizia da i Latini \textit{detta Curia} a differenza
  di \textit{Aula}; e vuol dire Ministri della giustizia.
\item[FALLITO] Uno che negoziando ha fatto così gran debito, che non ha
  possibilità di pagarlo. E il latino \textit{decoctus, qui fallit creditores, ipsumque fefellere negacia}.
\item[TUTTI i giorni son feriati] Sempre è festa per loro; Feriato s'intende quel giorno,
  nel quale ancor che lavorativo non si tien da i Magiftrati ragione, e non si
  possono fare esecuzioni civili contra a i debitari, e questo intende dicendo \textit{Non
  teme della corte, chi è fallito}, perché è feriato, e non può esser menato prigione.
\end{description}

\section{Stanza LXXXIV}

\begin{ottave}
\flagverse{84}Ma s'io non erro il tempo è già vicino,\\
Che n'ha a venir la piena de' disturbi, \\
Mentre doman per far un buon bottino\\
Andremo a dar'addosso a questi furbi.\\
Così panno sarà di Casentino,\\
Ne se lamenti alcuno, o si sconturbi;\\
Che che nuoce al compagno in fatti, o in detti\\
Deve saper che; Chi la fa l'aspetti,
\end{ottave}

Baldone, havendo fatto il detto raccanto della cacciata di Celidora, dice
sperare, che sia vicino il tempo, nel quale faranno gastigati coloro, che hanno sorpreso
Malmantile, perché il giorno futuro vuol andare a dar loro addosso.
\begin{description}
\item[HA da venir la piena de' disturbi] Ha da venir grandissima quantità di disgusti a
sturbare i loro commodi. E \textit{Piena} diciamo quando Arno, o altro Fiume cresce
per le pioggie.
\item[SARA' panno di Casentino] Casentino è una Regione in Toscana, dove si fabbrica
  una specie di panni, che bagnati scemano di lunghezza, e larghezza perché
  rientrano. E da questo detto \textit{sarà panno di Casentino}, intendiamo Rientrerà,
  cioè tu hai fatto a me questo, ed io farò a te il simile, cioè Mi vendicherò.
\item[CHI la fa, aspetti] Chi fa un torto al compagno, aspetti pure d'esser contraccambiato.
  Il Petr. disse;
  \begin{verse}
    Chi si prende diletto di far frode,
    Non si dee lamentar s'altri l'inganna,
  \end{verse}
E questi due versi posson servire per dichiarazione delli quattro ultimi della
presente ottava.
\end{description}
\section{Stanza: LXXXV.}
\begin{ottave}
\flagverse{85}Qui racque il Duca; e subito rattacca,\\
Col dire alla cugina in voce bassa\\
Che, perch'egli ha la bocca asciutta, e stracca\\
Il soggiunger a lei qualcosa lascia\\
Non ho che dir (gli rispond'ella) un hacca,\\
Oltre che la sarebbe carne grassa,\\
Dì più tosto, in che mo noi siam parenti,\\
Ch'io non paia a costor de gl'Innocenti;
\end{ottave}

\begin{ottave}
 \flagverse{86}Ed io che non ne ho gran cognizione,\\
E sempre me ne sono stata a detta \\
(Che tutta la mia gente andò al cassone,\\
Come tu sai ch'io ero fanciulletta:)\\
T'udirò volentieri. Allor Baldone\\
Soggiunse: Or or ti servo, e a tanta fretta,\\
Perché non gli moria la lingua in bocca,\\
Ricominciò quest'altra filastrocca.
\end{ottave}

Baldone termina il discorso, e volto a Celidora le dice, che ella soggiunga,
se ha di più; ed essa dicendo, che non ha che soggiugnere lo prega a narrare, in
che modo sieno parenti: E Baldone s'accinge a contentarla. E qui termina il
nostro Poeta il suo primo Cantare.

\item[NON ho che dire un hacca] L' H vogliono, che non sia lettera, ma semplice
  aspirazione, e però dicendosi \textit{Non ho che dire un hacca}, è lo stesso che dire: \textit{Non
  ho che dir nulla}.

\item[SAREBBE carne grassa] Stuccherei il popolo; Mi renderei odiosa. Il Lasca
  Nov. 4. dice: \textit{E poi io non vorrei anche tanto infastidirlo, che egli m'havesse a dire,
    che io fussi carne grassa}. La carne grafia suole a i più che la mangiano cagionare
  nausea; il che diciamo stuccare.

\item[CH' io non paia costor de gl'Innocenti] Che costoro non pensino, che io sia
  bastarda, o senza parenti. In Firenze lo spedale de gl'Innocenti si chiama quello,
  nel quale si mettono ad allevare i bambini, per lo più, nati di congiunzioni
  illecite, i quali corrottamente chiamiamo \textit{Nocentini}. Vedi sotto Cant.\ 10.\ stan.~7.

\item[ME ne sono stata a detta] Non ho cercato di saperne più là; ma ho creduto quel
  che m'è stato detto, o raccontato.

\item[LA mia gente andò al cassone] Mio padre, mia madre, e tutti gli altri miei parenti
  morirono; che per mia gente in questo luogo, ed in questi termini s'intende
  Miei parenti, e non altri.

\item[A tanta fretta] Subito, Prestissimo.
\item[NON gli moria la lingua in bocca] Era loquace, eloquente. Havea facilità a
  parlare. È lo stesso che \textit{Havere il suo in contanti nella lingua} come s'accennò
  sopra stan. 69.
\item[FILASTROCCA] Serie di parole, e per lo più s'intende d'un discorso male
  ordinato, e proprio del racconto, che talora fanno le balie a' Fanciulli in quelle
  lor novelle, come appunto è questa che narra Baldone, che l'Autore oltre all'haverla
  sentita forse raccontare alle sue donne, quando era fanciullino,
  ha tratta dallo Cunto degli Cunti di Gianalesio Abbattutis.
\end{description}
\section*{FINE DEL PRIMO CANTARE.}

\chapter{Secondo Cantare}

\begin{argomento}
De i due gran figli del Signor d'Ugnano
Prodigioso il natal narra Baldone;
Come s'acquista moglie Floriano,
E vien dall'Orco poi fatto prigione.
Come Amadigi libera il germano;
E il mostro spaventoso a terra pone,
E dice al fin, che l'un di questi dui
Fu padre a Celidora, e l'altro a lui.
\end{argomento}

\section{Stanza I.}
\begin{ottave}
 \flagverse{1}Era in Ugnano il Duca Perione, \\
Che sempr'all'Altarin fidecommisso \\
Faveva notte, e di tanta orazione, \\
E tante carità, ch'era un subbisso.\\
Ne per altro era tutto bacchettone,\\
Che per un suo pensiero eterno, e fisso\\
D'haver prole, perché della sua schiatta\\
Non v'era, morto lui, ne can, ne gatta.
\end{ottave}

Il Duca Baldone dà principio alla narrativa del parentado, che passa fra lui,
e Celidora, come havea promesso  nell'antecedente Cantare, e dice; Che fu già
in Ugnano il Duca Perione, il quale faceva molte opere pie per disporre il Cielo
a concedergli prole. La favola del nascimento di questi figliuoli trovasi nello
Cunto degli Cunti di Gianalesio Abbattutis Giorn. 1. Cunto 9. ll nostro Poeta
pero non la cavò di quivi; ma la narrò, come l'haveva sentita contare alle sue
donne, quando era fanciullo; e questo è certo, perché questa era nel suo primo
Poema fatto molto prima, che il Basile Autore dello Cunto de li Cunti la stampasse,

\begin{description}
\item[ALTARINO] Così chiamiamo un' inginocchiatoio a foggia d' altare, il quale
  per lo più si tiene allato al letto per inginocchiarsi, e fare orazione.
\item[STAR fidecommisso in un luogo] è detto iperbolico, che significa Star moltissimo
in un luogo; che qui vuol dire Stava sempre, o non si levava mai dall'Altarino;
che s'intende faceva orazioni infinite.

\item[TANTE carità ch era un subisso] Carità, ed elemosine infinite. Per denotare
  una quantità indicibile usiamo dire: \textit{Son tanti, che è un subisso}, \textit{un fracasso}, \textit{un flagello},
  e simili. Questa voce \textit{subbisso} vien forse dal Greco \textit{abyssos}, che significa voragine,
o smisurata profondità d'acque, come suona ancora nel nostro idioma,
donde \textit{subissare} Andar nel profondo, quasi dica \textit{sub abysso}.

\item[BACCHETTONI] Così chiamiamo noi certi colli torti, e graffiasanti, che
  stimano peccato il portare un fiore in mano, e credono poi di far'un'atto meritorio
  a dare a usura; con altro nome chiamati Ipocriti, cioè Pseudobeati; huomini
  da bene per interesse, e per gabbare il compagno; e sono insomma coloro,
  de' quali Giovenale disse: \textit{Qui Curios simulant, \& Bacchanalia vivunt}. E diciamo
  \textit{Bacchettone}, quali \textit{Va chetone}, perché questa Canaglia, che studia di simulare la
  bontà, per arrivare a suoi fini, è simile all'acque profonde, che vanno chete,
  delle quali parlandé Q. Curzio dice: \textit{Altissima quaeque flumina minimo labuntur sono}.
  E come queste acque son sempre di pericolo, così li \textit{bacchettoni} nella loro taciturnità
  occultano il malo animo, che hanno contro al prossimo. Il costume di costoro
  tocca Orazio lib. 1. Ep. 17. dicendo che son devoti di Laverna Dea de
  ladri.
  \begin{verse}
    Labra movens, metuens audiri; Pulchra Laverna,
    Da mihi fallere; da iustum, sanctumque videri.
  \end{verse}

Di questa voce \textit{Bacchettoni} si serve anche il Tassoni nella sua Secchia. \textit{Nimico
natural de' Bacchettoni}. Ed un dottissimo de' nostri tempi, il quale fa un
discorso poetico sopra a costoro, lo termina con dire \textit{Furfante, e bacchetton suona
il medesimo}, Vedi sotto \cstan[6]{97}. dove si dice esser lo stesso \textit{Bacchettoni}, che
\textit{Ipocriti}, i quali S. Matteo chiamò \textit{similes sepulchris dealbatis}; il Berni nel'Orlando
disse. \textit{O agghiacciati dentro, e di fuor caldi}, \textit{In sepolcri dipinti gente morta}.

Giovenale aggiunge al detto di sopra.
\begin{verse}
  Fronti nulla fides; quis enim non vicus abundat
  Tristibus obscoenis? castigas turpia, cum sis
  Inter Socraticos notissima fossa Cinaedos.
  \end{verse}

Di questi tali parla in diversi luoghi la Sacra Scrittura detestando tal vizio, come
abominevole, ma per brevità tralascio di riportarlo, contentandomi di chiudere
col detto dell'Evangelilta \textit{Atendite a falsis prophetis, qui veniunt in vestimentis
ovium, intrinsecus vero sunt lupi rapaces} e rimetter il Lettore a quello, che scrive
S. Matteo Evangelista al Cap. 6. 15.23.

Tale era appunto questo Perione, che faceva le dette Opere pie, non perché
veramente fusse buono, ma perché con esse pretendeva d'estorcer dal Cielo la
grazia d'haver figliuoli.

\item[SCHIATTA] Stirpe, Prosapia, famiglia.

\item[NON v'era, ne can ne gatta] Non vi rimaneva pur'uno. Plauto disse: \textit{Ne
  musca quidem domi est}, Del qual detto si servì quel servo dell'Imperator Domiziano
  che domandato, se Domiziano era solo in camera, rispose: \textit{Ne musca
  quidem est}, Perché Domiziano stava là dentro ammazzando le mosche. Ter.
  disse: \textit{Ne Sannione quidem relicto}.
\end{description}

\section{Stanza II.}
\begin{ottave}
 \flagverse{2}Così durò gran tempo, ma da zezzo,\\
Vedendo ch' ei non era esaudito\\
Essendo omai con gli anni in là un pezzo, \\
A mangiar cominciò del pan pentito;\\
E quant'ei far solea posto in disprezzo\\
Senza voler più dar del profferito,\\
Gettatosi all'avaro, ed al furfante\\
Cambiò la diadema in un turbante.
\end{ottave}

Continuò gran tempo Perione a far le narrate opere pie, ma veduto ch'ei non
era esaudito, e ch'ei non haveva figliuoli, e trovandosi già vecchio, perché veramente
egli era un di quei Bacchettoni furbi, che habbiamo detto sopra, e che
faceva bene solamente per interesse, si pentì d'haver fatto tante elemosine, ed
altro bene, e mutò costume.
\begin{description}
  \item[DA zezzo] Da ultimo. Forse meglio \textit{sezo}, venendo dal Latino \textit{secius} opposto
di \textit{ocius}. Vedi sotto \cstan[4]{72}.
\item[ESSENDO un pezzo in là con gli anni] Essendo grave d'età. Havendo molti
  anni. Vedi sotto \cstan[12]{36}.
\item[MANGIAR del pan pentito] Cioè si duole, si pente d' haver fatto del bene; ed
è quel \textit{facti poenitere} di Cicerone,
\item[POSTO in disprezzo quanto far solea] Cioè lasciando stare di fare elemosine, e
orazioni, ed altre opere pie come solea fare.
\item[SENZA voler dar del profferito] Senza voler dare più niente; e ne meno quello,
  che havea promesso, o proferto.
\item[GETTATOSI all'avaro] Divenuto avaro per elezione, o diremmo A posta.
\item[FVRFANTE] Vuol dir furbo scellerato, e ladro, e simili venendo dal latino
  barbaro \textit{foris faciens}, operante fuori del dovere, ma si piglia anche per Spilorcio,
  ed avaro, come è preso nel presente luogo.
\item[CAMBIO' la diadema in un turbante] Di Santo divenne Turco, che Diadema
appresso di noi vuol dire quell'ornamento, ò corona di splendori, che si vede
dipinto attorno alla testa de' Santi. Dice che cambio la diadema, che meritava
come Santo, in un turbante, cioè cappello da Turco, non che veramente si mettesse
il Turbante, ma intende, che d'huomo da bene diventò tutto il contrario.
\end{description}
\section{Stanza III}
\begin{ottave}
\flagverse{3}Di poi tutto diverso, e mal disposto\\
In modo degli Dei faceasi beffe,\\
Che s'egli udia trattarne, havria più tosto\\
Voluto sul mostaccio uno sberleffo;\\
La moglie un miglio si tenea discosto,\\
E dov'ei dava a' poveri a bizzeffe,\\
Quando picchiavan poi dalla finestra,\\
Facea lor dar il pan con la balestra.
\end{ottave}

Divenuto Perione tutto diverso da quel che era, come s'è detto, cominciò
anche a non stimar più gli Dei, anzi gli strapazava in modo, che havrebbe voluto
più tosto un sfregio sul viso, che sentirgli nominare; sbandì la moglie, ed in
vece di dar limosine a i poveri gli bastonava.
\begin{description}
\item[DIVERSO] Cioè differente da quel ch'era prima. Se ben questa voce diverso
  significa ancora stravagante. Vedi sotto \cstan[8]{17}. ed in questo senso la piglia
  Franco Sacchetti Nov. 29, E questa natura pare a me, che fusse delle strane, e diverse
  che trovar si potessero. E Nov. 78. \textit{Ed era un'huomo malizioso, reo, e di
    diversa natura}.
\item[FACEASI beffe] Si burlava. Non faceva stima. E il latino \textit{flocci facere}.

\item[SBERLEFFE] Taglio, o sfregio, che i Latini dissero stigma; \textit{Rigido signata
 stigmate fronte}. E perché gli sfregi in sul vifo sono cosa ignominiosa, come s'è
  detto sopra \cstan[1]{66}. da ciò si deduce che Perione havria più tosto sopportata
  ogni grande ingiuria, ed ignominia, che sentir nominare gli Dei. Il Coppetta
  nel Cap. in lode della sig. Ortenzia piglia la voce \textit{sberleffe} in significato di burlare
  uno, con oltraggi, e punture, che hoggi da molti si dice Fare uno scappeneo.
\begin{verse}
Allor l'amico in mezzo a i dolor miei
Mi fece uno sberleffe di velluto,
E mi fece arrossir dal capo a piei.
\end{verse}
E più sotto nel medesimo capitolo lo stesso mostra, che habbiamo anco il verbo
sberleffare dicendo.
\begin{verse}
E col rider di grazia andate piano,
Che non è per infermi util conforto,
E chi vuol sberleffar, sberleffi in vano.
\end{verse}

L'origine da questa voce \textit{sberleffe} vien forse da \textit{Berlina} in questo modo:

Si suole alle volte, dopo haver tenuto in Berlina i ladroncelli, segnargli in
qualche parte del corpo con un ferro infuocato, acciò che fieno dalla Giuitizia
riconosciuti, se altra volta per commessi delitti li tornassero nelle mani. E di
questi segni vedremo sotto \cstan[6]{54}. Ciò si costumava ancora appresso gli
antichi Romani ne i servi fuggitivi, e gli segnavano nella fronte come si cava da
Aulonio Epig. 15. che parlando di un servo nominato Pergamo dice.
\begin{verse}
  \backspace Iam segnis scriptor, quam lentus, Pergame, cursor
  Fugisti, \& primo captus es in stadio;
  \backspace Ergo notas scripto tolerasti Pergame vultu,
  Et quas neglexit dextera, frons patitur.
\end{verse}

Et aggiungesi alla voce \textit{berlina} quella finale \textit{effe}, da quella lettera maiuscola F,
che è il segno, o marchio, col quale si marchiano i detti delinquenti. Che cosa
sia berlina. Vedi sotto in questo C. stan 15.

\item[MOSTACCIO] Faccia, Volto, ec.

\item[TENEA la moglie discosto un miglio] Tenea la moglie lontana da se, intendi non
  volea più commerzio con la moglie. Lat; \textit{secubabat}.

\item[DARE a Bizzeffe] Dare, o donare largamente. Questa voce, che è composta
  dal latino \textit{bis, \& effe}, cioè due volte, f, vuol dir pienamente, largamente,
  abondantemente, e simili; Quando il sommo Magistrato Romano intendeva
  fare ad un supplicante la grazia senza limitazione, ma pienamente faceva il rescritto
  sotto al memoriale, che diceva \textit{Fiat Fiat}, che poi per brevità costumarono
  di dimostrare questa pieneza di grazia con segnare i memoriali con sole due
  effe, onde quello che conseguiva tal grazia diceva: Io ho havuta la grazia a \textit{bis
    effe}, cioè due volte ff che s'intende grazia intera, e piena, al costrario di quella
  limitata, che era con una sola effe aggiontavi la limitazione, o condizione
  con la quale il Magistrato havea conceduta la grazia. E' da questo \textit{bis effe} s'è poi
  corrottamente introdotto il dir Bizzeffe, che ha il signiticato, che habbiamo
  detto. Nella storia di Semifonte scritta sopra 300 anni sono, si legge al trattato
  terzo. \textit{La Terra di Semifonte era piena di torri merlate, e piombatoie, e di Torricelle
    a bizzeffe}.
\item[DARE il pan con la balestra] Vuol dice strapazare. Fare in maniera, che il
  benefizio sia di disgusto a chi lo riceve. Deriva forse dall'uso, che era in Firenze
  avanti che usasse andar a caccia con l'archibuso, di tenere al suo servizio huomini
  a posta i quali con qualche fsalvaticina mantenessero le mense de i grandi, e
  questo esercizio essendo d'utile, ma assai laborioso, può haver data origine a
  questo Proverbio \textit{dare il pan con la balestra}, cioè accompagnato da fatica, e disagio
  grandissimo. Ma nel presente luogo intende che effettivamente facesse tirare
  balestrate a i poveri.

Si dice ancora in questo proposito. \textit{Porger il pane con la spada}, e ciò forse deriva
da quello, che fece Dionisio Tiranno a un tal Democle Filosofo, il quale
(perché adulando eccedeva in lodare le grandezze di quello stato di Dionisio)
egli fece sedere ad una mensa ripiena delle più esquisite vivande, che per un banchetto
reale inventar si potessero; e fece attaccare per il manico ad una setola
pendente con la punta sopr'alla sua testa, una spada sfoderata, la quale veduta
dal Filosofo, gli cagionò così grande spavento, che egli non potè se non con molta
paura, e con poco gusto pigliare di quei cibi. Di costui parla Orazio Od.\
pr. lib.~3.
\begin{verse}
 Districtus ensis cui super impia
 Cervice pendet, non siculae dapes
 Dulcem elaborabunt saporem.
 \end{verse}

Si dice ancora, a questo proposito, \textit{dare il par col bastone} che ha origine da
quel che fece il Piovano Arlotto; il quale per gastigar l'indiscretezza d'alcuni
cacciatori, che gli havevano lasciato in casa un branco di cani, quando a questi
dava il pane, l'accompagnava con una mano di bastonate, onde i poveri cani
s'erano assuefatti quando vedevano il pane a fuggire; per lo che divennero cotanto
magri, che a pena si reggevano in piedi. Ritornati i cacciatori per li loro
cani, vedutigli così sfatti si dolevano del Piovano; ma egli preso in mano il solito
bastone, tirò loro in terra alcuni pezzi di pane, ed i cani ricordevoli di come
era solito passare il negozio, in vece d'accostarsi al pane fuggivano, onde il
Pidovano si scusò co i cacciatori dicendo: Come volete che ingrassino, se quando
io do loro il pane, fuggono come vedete? E da questa facezia venne questo
proverbio \textit{dar il pan col bastone}, che significa mostrar di voler far del bene a uno,
e fargli del male. Seneca ci fa veder questo modo di dire anche appresso a i Latini,
raccontando il detto di Fabio per soprannome Verrucoso, che il piacere
fatto da persona zotica, e con maniera salvatica chiamava \textit{Panem lapidosum}, che
è appropriato al nostro detto \textit{Dare il pane, e la sassata}.
\item[BALESTRA] Strumento, o arme da caccia, col quale si scagliano palle di
terra secca, nella guisa che si fa delle frecce; e serve per ammazzare uccelletti.
È composta d'un'arco d'acciaio accomodato in cima a un'asta, o legno torto,
dentro al quale sono adattati altri ordinghi di ferro per facilitare l'operazione.
Viene dall'antica ballista arme guerriera, che dicevano ballista forse dal Greco
\textit{ballein}, che significa scagliare.
\end{description}

\section{Stanza IV.}
\begin{ottave}
\flagverse{4}La plebe, i grandi, ed ogni lor ministro\\
Ch'il Duca così buono havean provato,\\
Mentre fu scudo ad ogni lor sinistro\\
Ed in lor pro sarebbesi sparato,\\
Vedutolo così mutar registro,\\
E diventar un turco rinnegato,\\
Eran talmente d'animo cattivo,\\
Che l'havrebbon voluto ingoiar vivo.
\end{ottave}

Per questa mutazione del Duca di buono in cattivo, li suoi sudditi, che prima
l'amavano, cominciarono a portargli odio, e bramargli ogni male.
\begin{description}
  \item[SI sarebbe sparato in lor pro] Havrebbe fatto loro ogni favore immaginabile.
    Havrebbe messa, e spesa la propria vita a benefizio loro, e la  voce \textit{pro} è un
    sustantivo che significa giovamento, utile, ec. dal latino \textit{prodest}.
\item[MUTAR registro] Mutar maniera di fare. \textit{Registro} diciamo quell'ordine di
  ferri, il quale è negli organi strumenti musicali, con ciascuno de' quali ferri alzandolo,
  o abbassandolo si dà, o leva il fiato a quelle canne, le quali si vuol,
  che suonino o no, ad effetto di far mutar voce all'organo, il che si dice \textit{mutar
    registro}, che passato poi in proverbio significa Mutar maniera, o modo di fare
  in qualsivoglia cosa. Vedi sotto C.\ 8.\ stan.\ 52.\ alla voce protocollo \textit{Registro} in
  altro significato.
\item[INGOIARE] Trangugiare. Mandar giù in corpo una cosa senza anche
  masticarla, che si dice anche ingollare. Vedi sotto C.\ 1.\ stan.~6.
\end{description}

\section{Stanza V.}
\begin{ottave}
\flagverse{5}Avvenne, che già inteso un Negromante \\
C'un'huom com'era quei sì giusto, e magno,\\
Faceva novita sì stravagante, \\
Un'atto volle far da buon compagno;\\
E per ridurlo all'opre buone, e sante\\
Non per speranza di verun guadagno\\
Fintosi un baro, a dargli ando l'assalto,\\
Un po di ben chiedendo per sant'alto.
\end{ottave}

Stando le cose ne i suddetti termini, un tal mago, inteso che un huomo da
bene come era Perione s'era cangiato in così cattivo, volle fare un'atto da huomo
da bene, cercando di rimettere Perione nella buona strada, e però fintosi
un'accattone, andò a chiedergli l'elemosina per amor di Dio.

\begin{description}
  \item[NEGROMANTE] È lo stesso che Mago: Se bene Negromante venendo da
    negromanzia s'intende colui, che \textit{per mortuos vaticinatur}, che è una delle sei specie
    di Magi detti sopra C. 1, stanza 20., tuttavia da noi si piglia per nome generico,
    e per intendere ogni specie di mago, e di magia.
\item[BARO] Biante. Accattone falso. Vien forse dal Greco \textit{Barijs Bareos}, che
  suona molestus, importuno, sfrontato, come appunto sono questi tali; e se bene
  questa parola ha del furbesco pure s'usa comunemente, e l'usò il Varchi St. Fior.
  lib.~11, \textit{Ed in segno, che lo rifiutava, e non gli creduea più, havendolo per baro, e
  giuntatore, arse i suoi libri}.
\item[PER Sant'alto] Cioè per Dio. È parlar furbesco, il quale forse è noto fuori
della nostra Toscana, come inventato da Vagabondi, Monelli e Pianti per non
esser intesi, se non da i lor pari, e poi fattosi familiare a molt' altri, a segno
che ne è fatto, stampato il vocabolario. Si dice anche parlare \textit{in gergo, ed in lingua
furfantina}, come ci mostra il Varchi St. Fior. lib. 15. \textit{Appariscono più lettere scritte
non in cifra, ma in gergo a uso di lingua furfantina molto strano}. Il nostro Poeta si
serve di tal parlare nella persona di questo Biante perché, come ho detto; simili
huomini son soliti parlar in questa forma.
\end{description}

\section{Stanza VI.}
\begin{ottave}
\flagverse{6}Rispose Perione : Fratel mio\\
se tu te lo credessi tu t'inganni,\\
Tu vuoi ch' io doni per l'amor di Dio, \\
Ne sai ch'io piglierei per San Giovanni,\\
Se t'hai bisogno, che posso far'io?,\\
Che son Fraffazio, che rifaccia i danni\\
E che pensi, che qua ci sia la cava?\\
non è più tempo che Berta filava.
\end{ottave}

Alla richiesta del Mago Perione non si muove a far limolina, anzi dice che
piglierebbe anch' egli qualcosa, e che è passato quel tempo che egli dava via
il suo.

\begin{description}
\item[PIGLIEREI per San Giovauni] S. Gio. Batista è il Santo protettore della nostra
  Città di Firenze, e perciò il giorno della sua festa e grandemente solennizzato, ed
  in quel giorno son sicuri nella Città fino i banditi capitali, sicché gli Sbirri non
  posson pigliar nessuno. Da questo è nato l'equivoco Proverbio; \textit{Pigiterebbe il dì
    di San Giovanni}, o \textit{per San Giovanni}, che vuol dice Piglierebbe anche quel dì,
  nel quale ne meno i birri pigliano, e s'intende piglierebbe, cioè accetterebbe tutto
  quel che gli fusse dato in ogni occasione, ed in ogni tempo. E lo scherzo è nel
  verbo pigliare che vuol dir Far cattura, o Catturare, e vuol dire anche Accettare,
  o ricevere, come s'intende in questo proverbio; che esprime; Lo piglierei, ed
  accetterei sempre, e non darei mai.

\item[CHE son Fraffazio], Raccontano una favola d' una donna non troppo honesta,
  la quale havendo commerzio con un tal' huomo detto Fraffazio, fu con esso
  una volta trovata dal marito; ed essendo ella altrettanto sagace, quanto il marito
  semplice, e di cervello grosso, gli diede facilmente a credere, che colui era
  un' huomo da bene, che andava rifacendo i danni a chiunque occorreva qualche
  disgrazia, e che l'haveva chiamato in casa affinché le ricomprasse una sua conca,
  la quale s'era rotta, e che appunto gli narrava questo suo danno; foggiungendo;
  E come, Marito mio! Non conoscete dunque Fraffazio? Il buoa marito
  se la bevve, e così la donna scampò la furia, E da questa favola, quando si
  dice: \textit{esser Fraffazio}, vuol dir: \textit{Esser colui che spende il suo per sollevar
    l'altrui miserie}, e che \textit{rifà i danni} come dice il nostro poeta.

\item[CHE pensi, che qua ci sia la cava] Pensi che io habbia la cava de' danari, cioè
  la Zecca.  Torna bene a questo detto quel che si trova in Salustio; \textit{Censes me
  vicem aerarij praestare}. Non è pero che cava voglia dire la Zecca, ma si piglia per
  questa nel presente detto (da noi usatissimo in questo proposito) perché si suppone,
  ed è verisimile che la Zecca, come luogo dove si batte la moneta, ne sia
  abondante, come sono abondanti le cave di quelle cose, che da esse estraggonsi.

\item[non è più ib cempo che berta stava] Non è più il tempo, che le cose andavano
  come si bramava. I tempi son mutati. Pipino Re di Francia per mezzo di suoi
  Ambasciadori sposò Berta dal Gran pié figliuola di Filippo Re d'Ungheria, la
  quale havendo saputo, che questo suo Sposo era brutto, e nano, malvolentieri
  s'accomodava a dare il consenso; ma pure, vinta dalla riverenza dovuta ai padre,
  condescese, Arrivata in Francia, lasciandosi governare dal giovenil sentimento;
  richiefe Elisetta di Maganza sua segretaria (la quale d'Ungheria, dove era
  nata del Conte Guglielmo di Maganza ribello di Francia, se ne veniva con Berta a
  Parigi) che volesse, fingendosi la sua persona, in sua vece sposarsi con Pipino
  il quale, e pera somiglianza, che era fra lor due, e per non haver Pipino mai
  veduta Berta, non l'havrebbe assolutamente riconosciuta, Elisetta da principio
  si mostro renitente; ma persuasa poi da Grifone, e Spinardo di Maganza suoi
  parenti, condescese a i voleri di Berta. E così arrivati a Parigi, Elisetta si sposò
  con Pipino in vece di Berta. La qual Berta in tanto di consiglio di detti due
  Maganzesi s'era ritirata in ludgo vicino a Parigi, con pensiero fermato con
  detti Maganzesi di quindi occultamente partir, e tornarsene alla patria con
  l'aiuto de' medesimi; ma questi la tradirono, perché in vece di servirla alla volta
  della patria sua, l'inviarono ad un bosco, con ordine a quelli, che la conducevano,
  che l'uccidessero: Mu costoro mossi a pietà, in vece d'ucciderla, la
  spogliarono, e legatala ad un'albero la lasciarono in preda alla Fortuna, e tornarono
  a i Maganzesi, dicendo che l'haveano uccisa.  I Maganzesi per occultare
  sì atroce delitto fecero morire tutti quei ficarj, havendo prima anche d'arrivare
  a Parigi fatte ritornare in Ungheria tutte le dame, ed altre persone non
  complici, ne consapevoli di sì grande scelleraggine.

  Berta intanto, che se ne stava così legata dolendosi, e lamentandosi fu sentita
  da un tal Lamberto Cacciatore del Re Pipino; Costui seguitando la voce si condusse
  dove stava Berta legata all'albero, e scioltala, alla propria casa la condusse,
  e la consegnò alla moglie vestendola d'abiti vili, e conformi alla posibilità
  di lui, ed alla povera condizione, della quale Berta disse d'essere. Quivi
  stette Berta circa cinque anni, nel qual tempo guadagnò molti denari di filare,
  ed altri lavori, che insieme con le figliuole di Lamberto faceva. Avvenne un
  giorno, che essendo Pipino a caccia si condusse solo alla Casa di Lamberto, ove
  veduta Berta s'invaghì di lei, e con essa si congiunse sopra ad un suo carro, nel
  qual congiungimento fu generato Carlo, così detto dal medesimo Carlo. In tale
  occasione Berta scoperse a Pipino il tradimento de i Maganzesi narrandoli
  tutto il seguito; perloché Pipino fece abbruciare Elisetta, ed una mano di Maganzesi,
  e rimesse nel trono Berta.
  Da questa favolosa storia nacque il proverbio; \textit{Non è più il tempo che Berta
    filava}, Cioè non è più il tempo che Berta stava nelle selve filandode., e ricamando,
  che significa; \textit{Le cose son mutate}.

  Di questo detto si servì Berta moglie d'Arrigo IV, Imperatore, come si vede
  nello Scardeonio Monumenta Patavina lib. 3. Classe 14. de Berta ex Montagnano,
  le di cui parole son queste. \textit{Memoratur in iisdem Patavinis Annalibus celebris
fama Bertae ex Vico Montagnani, quae quidem fuit ruslicano genere, sed moribus certe
perquam nobilis \& animo perquam generosa},

\textit{Haec enim tempore Henrici IV Imperatoris, cum eius uxor, Berta \& ipfa nuncupata,
  Patavij moraretur, vel eiusdem forte nominis similitudine, vel propria generositate
  animi allecta, obtulit ei dono filum tenuissimum, quod eleganter suamet neverat manu,
  \& in Vrbem venale detulerat. Quod munus Regina hilari vultu accepit; \&
  cum cognovisset nomen, \& animum mulieris, eam indignam censuit, ut vitam inopem
  famineo colo amplius sustineret suam. Dato itaque filo procuratori suo, iubet ad Pagum
  Montagnani statim proficisci, ubi mulier habitabat, \& pro referenda gratia tot terra
  iugera ei ex publico adscribi, quantum spacij filum dono datum extensum comprehendere,
  \& circumdare posset, Quod cum caeterae mulieres vidissent, illico Bertae exemplo
  attulerunt, \& ipsae filum, quod Regina dono darent. At ipsa renuens id ab alijs accipere
  percante respondit}, Pertransiit tempus, dum Berta filabat.

Gli antichi dicevano \textit{Non est amplius aetas Cyclopum}, ed in molte altre maniere, si come
Ancor noi diciamo: \textit{È finita la cuccagna}, o la vignuolaxe. \textit{non è più tempo di Bartolommeo},
ec. Con i quali, ed altri detti intendiamo Non si godono più quelle felicità che già
si godevano.
\end{description}

\section{Stanza VII.}
\begin{ottave}
\flagverse{7}Signor (soggiunse il Mago) mi sa male \\
Di veder, c' un sì gran limosiniere,\\
Ed huom tanto benigno, e liberale \\
Caduto sia nel mal del miserere.\\
Hor basta; Chi del mio fa capitale\\
(Diss'egli) fa la zuppa nel paniere.\\
Pero va in pace tu co' tuoi bisogni,\\
Perché per me tu mangerai de' sogni.
\end{ottave}

Il negromante vedendosi cacciar via con tal risposta; replicò, che gli dispiaceva,
ch' ei fusse diventato avaro. E Perione li soggiunse, ch'ei non sperasse da
lui fastidio alcuno.
\begin{description}
\item[CADUTO nel mal del miserere] Divenuto misero, cioè avaro, tenace, che se
bene il mal del Miserere è una infermità mortale; Noi ci serviamo della voce
Miserere nella forma che habbiamo detto sopra \cstan[1]{80}. della voce \textit{boccolica},
e per intender \textit{misero}, che nel presente luogo vuol dire avaro; e così è inteso
comunemente, se bene la voce \textit{Misero} propriamente vuol dire infelice.
\item[FAR capitale] Far' assegnamento; o sperare nell'aiuto d'alcuno. Vedi sotto
\cstan[7]{82}. Questa voce capitale è dedotta da \textit{capitatio capitationis}, che era una
tassa, o tributo, che determinavasi \textit{in capita populorum} per assegnamento; e propriamente
capitale del Principe, come è forse la Decima, che pagano hoggi i
nostri contadini, che pure si dice decima in su la testa.
\item[PANIERE] È un vaso intessuto, e composto di fili di vetrice, o d'altra specie
  d'albero,  o di sottilissime strisce di legno in figure, e forme varie, in tutte
  le quali che sieno, ha sempre il manico; che senza manico si chiama corbello, o
  paniera, e servono per portar frutte, o altro che sia; detto paniere, o paniera
  forse dal pane, perché gli antichi tenevano il pane in tal sorte di cesta in mezzo
  alle mense, e perciò da i Latini detto \textit{Panarium}.
\item[FAR la zuppa nel paniere] Questo proverbio dice:
  \begin{verse}
    Chi fa l'altrui mestiere
    Fa la zuppa nel paniere.
  \end{verse}

E così dichiara il suo significato, quale è: Che colui, il quale si mette a fare
una cosa, che non fa fare, non farà nulla di buono; ed in sustanza vuol dire; Affaticasi
in vano. Ovid. lib. 12.
\begin{verse}
  Vique liquor rari sub pondere cribri
  \end{verse}

 Ed è forse meglio dir suppa, che zuppa venendo dal verbo suppurare, che vuol
dire attrarre l'umido; o da Suppen Tedesco. Vedi sotto \cstan[4]{25}. Ma l'uso
ci obliga a dir zuppa.
\item[VA in pace] Così usiamo dire, quando mandiamo via i poveri, che accattano.
  E l'usò in un certo modo Plauto in milit. dicendo \textit{Pax, abi},
\item[MANGERAI de sogni] Mangerai cose immaginarie, cioè non mangerai.
  Mattio Franzesi\footnote{Mattio (Matteo) Franzesi, San Gimignano 15.. - 1555, poeta burlesco.} nel Capitolo della povertà dice.
  \begin{verse}
    Che sfacciata talor non si vergogni,
    E che spesso permetta, e faccia male,
    Si scusa, che non può viver di sogni.
  \end{verse}
  I Latini pure havevan simil modo di dire, come si vede in Giuvenale Sat. 6.
  \begin{verse}
    Qualiacumque voles Iudaei somnia vendunt.
  \end{verse}
E coloro, che hanno una vogllia ardentissima d'una cosa, sogliono sognarla; perché
altro non è il sogno, che
\begin{verse}
  Un'immagen del dì guasta, e corrotta
  \end{verse}
La onde Teocrito\footnote{} Eglog. 9. introduce un Pastore, che raccontando le sue felicità così ragiona:
\begin{verse}
Possideo quaecumque solent in nocte videri
In somnis, vim magnam ovium multasque capellas.
\end{verse}

Et anco notò Nonio, che appresso gli antichi Romani, il verbo vescor significava
vedere: \textit{Prius quam infans esses, tui oculi facinus vescuntur} cioè \textit{vident}; come noi
pure diciamo; \textit{Mangiar un con gli occhi}, quando altri guarda uno con grande attenzione;
e diciamo anche: \textit{Dar pasto agli occhi}. Dan. Par. Ci 27,08
\begin{verse}
\backspace E sa natura, ed arte le pasture
Da pigliar occhi \makebox[5em]{\dotfill}
\end{verse}

Sì che dicendo mangerai de sogni, si può anche intendere, \textit{Ti sazierai, o soddisfarai
con dar pasto a gli occhi}; o \textit{della vista}; che è lo stesso che non mangerai. Vedi
sotto \cstan[6]{55}. che dice \textit{pascer la vista}.
\end{description}
\section{Stanza VIII — X.}
\begin{ottave}
\flagverse{8}Come (replicò quei) se è si cicala,\\
Che tu daresti via fin la gonnella,\\
Vedendomi spedato, e per la mala\\
Potrai haver' il granchio alla scarsella?\\
Poi che tu gratti il corpo alla cicala\\
(Disse il Duca) io levsi questa cannella\\
Per quel ch'io ti dirò, perché se già\\
Donai, non era tutta carità.
\end{ottave}

\begin{ottave}
\flagverse{9}E non batteva la mia fine altrove,\\
C'ad haver prima ch'io serrassi gli occhi\\
In ricompensa un dì, piacendo a Giove,\\
Della mia donna quattr'o sei marmocchi,\\
Ma finalmene dopo mille prove\\
Di dar' il lustro a marmi coi ginocchi,\\
Tenendo gli occhi in molle, e il collo a vite,\\
E le nocca col petto sempre in lite;
\end{ottave}

\begin{ottave}
\flagverse{10}Io l'hebbi bianca a femmine, ed a maschi,\\
Ond'io sbraciar volendo a bel diletto,\\
Mi risolvei levar quel vin da fiaschi,\\
E non dar più quant'un puntal d'aghetto,\\
Perché po poi (diss'io) gli è me' ch'io caschi\\
Dalle finestre prima, che dal tetto;\\
E il cavarmi di mano adesso un pelo,\\
Sarebbe un voler dare un pugno in Cielo,
\end{ottave}

Il Mago mostra di non poter credere, che havendo Perione nome di liberalissimo,
non s'habbia a muover' a compassione di lui, e Perione vinto dall'importunità
di costui, gli dice, che fu già liberale per disporre il Cielo a concedergli
figliuoli; ma perché egli non era stato esaudito, lasciò di far più limosine, ed
hora era impossibile cavargli di mano un picciolo.

\begin{description}
\item[SÌ cicala] Cioè si dice; Si discorre. Il verbo cicalare usato in questi termini
esprime discorso di cosa incerta, che si dice anco \textit{bucinare}, o \textit{buzicare}, E si dice:
la tal cosa non fu poi vera; ma fu una cicalata, cioè se ne parlò; ma non è poi stata
vera.
\item[DARESTI via fin la gonnella]  Daresti via fino al proprio vestito; daresti via
tutto il tuo havere. E se bene \textit{gonnella} s'intende una specie d'abito da donna, in
questo proverbio diventa nome generico per ogni sorte d'abito.
\item[SPEDATO] Cioè co' piedi laceri dal viaggio.
\item[PER la mala] Cioè per la mala via, e s'intende mal condotto di sanità, e mal'all'ordine
  di vestito, e senza danari.
\item[HAVER il granchio alla scarsella] Chiamiamo \textit{Granchio}, o \textit{grancia} una specie
  di malattia di spasimo, la quale quando viene alle mani impedisce il maneggiare le
  dita; E da questa quando diciamo \textit{Il tale ha il granchio alla scarsella} intendiamo
  non può adoperare le mani intorno alla borsa, che vuol dire; è pigro a cavar denari
  della borsa, cioè, a dire: è tenace, o avaro, ed uno, de' quali parlando
  Marziale dice.
  \begin{verse}
    \backspace Litigat, \& podagra Diodorus, Flave, laborant;
    Sed nil Patrono porrigit; haec Chiragra est.
  \end{verse}

  E noi pure diciamo di questi tali; \textit{Haver la gotta alle mani}, \textit{Haver i pedignoni
    alle mani}; \textit{Haver le mani aggranchiate}; \textit{farebbe a pagar co' monchi},
\item[SCARSELLA] Intendiamo ogni sorte di tasca, o borsa di danari, come si
vede sotte \cstan[3]{5}., se bene scarsella è propriamente una borsetta di quoio
Con serrature di ferro fatta alla foggia delle Carniere da cacciatori; la qual sorte
di di borsa usava già in Firenze portarsi da tutti legata a cintola.
\item[GRATTAR il corpo alla cicala] Incitar' uno a discorrere. Vedi sopra Cant.
primo stan. 2. I Latini pure dissero in questo proposito \textit{Cicadam ala comprehendere}.
\item[LEVAR la cannella] Desistere di fare una tal cosa. Traslato dalla botte, alla
  quale si leva la cannella, quando è finito il vino, che era in essa. E cannella intendiamo
  quel legnetto tondo forato per lungo, che si adatta al fondo della botte
  per cavarne il vino, la quale da i Latini con voce Greca si dice \textit{epistomium}. Si dice
  anche in questo proposito.
\item[LEVAR il vino da fiaschi] come vedremo appresso.
\item[PRIMA ch'io serrassi gli occhi] Prima ché io morissi.
\item[MARMOCCHI] Ragazzi. Queita voce marmocchio in significato di fanciullo,
viene da marmo, alla pulitezza, e liscio del quale s'assomiglia il liscio, e pulitezza
del volto de i fanciulli, e delle fanciullette. Or. Od. 19. lib. 1.
\begin{verse}
 Urit me Glycerae nitor
 Splendentis Pario marmore purius.
\end{verse}
\item[DAR il lustro a' marmi co' i ginocchi] Cioè stava tanto tempo, e così spesso in
  ginocchioni, che il lungo fregare con le ginocchia faceva divenir lucenti i marmi,
  sopra i quali s'inginocchiava.
\item[TENENDO gli occhi in molle] Cioè lagrimando, e così tenendo gli occhi in
  molle nelle lagrime.
\item[COLLO a vite] Collo torto, come fanno i Bacchettoni. Si dice a vite per similitudine,
  essendo \textit{la vite} uno strumento; il quale serve per serrar un materiale
  con l'altro, che per essere attorcigliato come \textit{la vite} pianta, che produce l'uva,
  da essa piglia il nome, e si dice anche \textit{torchio}, e \textit{chiocciola}: quello dal torcere, col
  quale  fa la sua operazione; e questa per la similitudine, che ha la sua figura con
 il guscio della chiocciola.
\item[E LE nocca col petto sempre in lite] Cioè dandosi delle pugna nel petto; il
  che mostra che le \textit{nocca} sieno in lite col petto, mentre non cessano di perquoterlo.
  E nocca intendiamo nodelli delle dita. Vedi sotto \cstan[3]{8}., e \cstan[9]{54}.
  In somma il Poeta con queste quattro maniere di dire, cioè \textit{Dar' il lustro a' marmi
  co' ginocchi}; \textit{Tenere gli occhi in molle}, \textit{Haver il colle a vite}; e \textit{le nocca sempre in lite
  col petto}, Intende, che \textit{costui stava sempre orando}; e descrive assai bene
un' Hipocrito, o devoto in apparenza, e falso.
\item[IO l'hebbi bianca] Quando un premio s'ha da conseguire per via d'estrazione
  di polizze (come si fa al lotto) sono scritte solamente le polizze premiate, e l'altre
  son bianche; e chi ha una polizza bianca, non conseguisce premio alcuno. E
  di qui viene il detto \textit{Io l'ho havuta bianca}, che è fatto comune, e per intender di
  tutte quelle cose, che si tenta di conseguire, e non si conseguiscono.
\item[SBRACIARE] Vuol propriamente dire, allargare, e sollevare la brace a fine,
che meglio s'accenda, e renda più calore; ma per metafora intendiamo spender
prodigamente, e largamente, come s'intende nel presente luogo, e sotto Cant, 3. stan. 2.
\item[A bel diletto] A posta; o per gusto, ma senza buon fine, e utile, e si dice anche
  a \textit{bello studio}, a \textit{bella posta}, a \textit{bella prova}, che tutti si posson pigliare in questo
  senso. Se bene alcune volte significano quel che i latini dissero \textit{dedita opera} e
  massime quando non v'è l'aggiunta di \textit{bella}, che in questo caso e detto ironicamente,
  ed ha forza d'esprimere \textit{biasimevole}, come per esempio \textit{Veramente tu hai
  fatta una bella cosa}, cioè tu hai fatto una cosa biasimevole, e che sta male. Virg.
  \textit{Egregiam vero laudem, \& spolia ampla reportas}.

\item[NON darei quanto un puntal d'aghetto] L'aghetto è una cordicella fatta di seta,
  o d'altro, che serve per affibbiar le vesti, e adattarle alla persona, alla qual cordicella
  è solito fare una punta di sottil lamina d'ottone, o d'altro metallo, e
  queste punte si dicono \textit{puntali}, e di queste punte se n' hanno due, o tre per un
  quattrino; e da questa viltà serve il presente detto per esprimere; \textit{Non darei niente},
  ne meno una cosa, che non val nulla. Che i latini dissero fra l'altre molte,
  \textit{Vitiosam nucem non dederim}. E noi pure diciamo un fico secco, un lupino, e simili.
  Vedi sotto \cstan[3]{8}.
\item[LEVAR il vin da fiaschi] Il senso metaforico è lo stesso, che levar la cannella
  detto poco sopra stan. 8.
\item[PO poi] Alla fine. All'ultimo de gli ultimi. Opera anco in questo detto la
  forza della replica, che induce superlativo, Vedi sotto in \cstan{73}.
\item[GL'è me ch'io caschi dalle finestre prima che dal tetto] Nel male è il meglio, l'eleggere
  il meno. Intende; egli è meglio, che io lasci stare di dare il mio che seguitare,
  e darlo via tutto, cioè mi contenti di questo danno, e non lo faccia
  maggiore col seguitare a profondere il mio. E quel me per meglio è la figura
  Apocope da noi spesso usata; e l'uso Dante più volte, ma notabilmente nel C.
  32, dell'Inferno, che l'usò nel principio del periodo.
  \begin{verse}
    Me foste state qui pecore, o zebe.
  \end{verse}
  Ma di questa figura Apocope, e come l'usiamo, vedi sotto in \cstan{36}.
\item[CAVARMI di mano un pelo] Conseguir da me cosa alcuna, ancor che di niun valore.
\item[SAREBBE un voler dare un pugno in Cielo] Sarebbe un voler tentar, una cosa
  impossibile, \textit{Facilius Caelum digito attingeres}.
\end{description}

\section{Stanza XI — XIII}
\begin{ottave}
\flagverse{11}Che pagheresti (disse lo stregone)\\
Se la tua moglie havesse il ventre pregno?\\
Se cio fusse (rispese Perione)\\
Ancor ch'io non ne faccia alcun disegno,\\
E tal voglia appiccata habbia all'arpione\\
Io ti vorrei donar mezz il mio regno\\
Sogginnse quei: Non vo pur'una crazia,\\
Ma solamente la tua buona grazia.
\end{ottave}

\begin{ottave}
\flagverse{12}Altro da te non aspettar ch'io chieda,\\
Ne c'alcuno interesse mi predomini,\\
Perché, quantunque abietto altri mi veda,\\
Io ho in \culo{} la roba, e schiavo son de gli uomini\\
Hor basta se tu brami d'aver reda,\\
ch'il regno dopo te governi, e domini,\\
Commetti al Mosca, al Biondo, e a Romolino,\\
C'un cuor ti portin d'asino marino.
\end{ottave}

\begin{ottave}
\flagverse{13}Ed ordina di poi, che se ne quoca\\
La terza parte in circa arrosto, o lessa,\\
(Ch'in tutti modi è buona) e dann'un poca \\
In quel modo a mangiar alla Duchessa; \\
Presa che l'ha, gli è fatto il becca all'oca,\\
Che subito ch'in corpo se l'è messa,\\
Senza che tu più altro le apparecchi,\\
Dottela pregna infin sopr'algli orecchi.
\end{ottave}

Il mago s'esibisce a dare a Perione il modo, che la sua moglie impregni;
Perione gli dice che se ciò segue li vuol donar mezzo il suo regno; ed il mago
ricusando il tutto, da a Perione la ricetta dell'Asino marino per impregnar la
moglie.

\begin{description}
\item[CHE pagheresti?] Quando veggiamo uno, che sommamente brama di sapere, o
  d'ottenere una cosa, per mostrare, che è in nostra potestà l'adempire il suo desiderio
  sogliamo dire: \textit{Che pagheresti? Che spenderesti? Quanto daresti?} o simili,
  \textit{se io ti dessi, o dicessi la tal cosa?}
\item[STREGONE] Maliardo, Mago, Negromante, ec, Viene dal latino, secondo
  che osservò il Mureto\footnote{Marc-Antoine Muret (Muret, 12 aprile 1526 – Roma, 4 giugno 1585) filologo e umanista. Dovette fuggire la Francia per eresia e sodomia. Mureto è il nome con cui è conosciuto in Italia, dove poté stabilirsi protetto dalla Chiesa.  } nelle sue varie lezioni lib. 12, c.19. emendando un luogo
  di Plauto nelle Bacchidi. \textit{Longum est Strigonem maleficum exornarier. Strigas}
  (dice egli) \textit{vocabant mulieres, quas etiam noctu volare arbitrabantur, eodemque modo
    strigones homines maleficos, quorum vocabulorum vulgus in Italia utitur}, Vedi sotto C.
3. stan. 69.
\item[IO non ne fo più disegno.] Io non ho più la speranza d'ottenere questa cosa. N'ho
affatto levato l'animo, o il pensiero.
\item[APPICCARE la voglia all'arpione] Haver lasciata la voglia, o il desiderio
  d'una tal cosa. È lo stesso che \textit{Appiccar al chiodo} visto sopra \cstan[1]{8}. E questo
  modo di dire forse procede da i voti, che anticamente facevano i Gentili,
sospendendogli nel Tempio, i quali non si potevano levare, di dove eran posti,
ne convertirgli in uso comune, o profano.
\item[ARPIONE] È una specie di chiodo uncinato per uso di regger l'imposte delle
parte, e finestre, girando, quelle sopra di essi. Da i Latini detti \textit{Cardines}.
\item[NON vo pur' una crazia] Non voglio danari. \textit{Crazia} è delle più vili monete d'argento
che habbiamo,  essendo, l'ottava parte del giulio.
\item[HO in \culo{}] Detto usatissimo, e massime dalla gente vile per esprimere: non stimo,
  non apprezzo questa tal cosa.
\item[SCHIAVO SON de gli huomini] Son servitore a gli huomini virtuosi, e di garbo.
  Quando noi diciamo Il tale è un' huomo (Seguitando il detto di Diogene
  \textit{hominem quaero}) intendiamo huomo dotto, virtuoso, e di tutta perfezione.
\item[HOR basta] Questo termine (del quale l'Autore si serve anche nell'ottava, 7. antecedente)
  è usatissimo per denotare la terminazione d'un discorso, e passaggio ad
un'altro conclusivo del primo, quasi dica: \textit{E a bastanza quanto habbiamo detto per
conchiudere il come, o il quando, o il se si deva fare, o non fare la tal cosa}.
\item[REDA] Cioè successione, heredi, e s'intende figliuoli. \textit{Il tale ha havuto reda},
\textit{il tale ha havuto un figliuolo}. E buona parola Fiorentina, ma hoggi poco usata, e
solamente per i contadi; dove per \textit{reda} intendono anche i figliouli delle bestie.

\item[MOSCA, Biondo, e Ramolino] Tre venditori di pesce, che vivevano al tempo.
che l'Autore compose quest'Opera.
\item[GLI è fatto il becco all'Oca] Il negozio è conchiuso, che i Latini dissero: \textit{Iacta
  est alea}. Il Lalli nella sua En. Tr. \cstan[]{64}. disse:
\begin{verse}
Ne vanno tuiti: il marcio hora si giuoca,
Non v'è rimedio. È fatto il becco all'oca.
\end{verse}

Dice Francesco Cieco da Ferrara nel suo Poema intitolato il Mambriano (Opera
nota per esser l'origine, ed antefatto dell'Orlando innamorato, Poema del
Boiardo, ed in conseguenza dell'Orlando furioso di Lodovico Ariosto) al Canto
secondo, che
\begin{adjustwidth}{0.5em}{}
  Fu già nel Regno di Cipri un Re chiamato Licanoro il quale havea una sola
figuola nominata Alcenia, la quale amando egli al pari di se stesso, volle
sapere, se buona, o ria fortuna ella fusse per havere; fatti però chiamare alcuni
Astrologi fece fare la natività alla medefima sua figliuola, e tutti concordarono,
che ella farebbe prima stata madre, che moglie; Onde il Re per evitare
il presagito vitupero, fece fabbricare un giardino contiguo al suo palazo reale,
e dentro al detto giardino edificò una fortissima, ed altissima Torre con
molte stanze, e con tutte le comodità, ma senza finestra alcuna, che riuscisse
fuori della Torre: Dentro a questa messe la figlia con alcune Matrone, e
Damigelle, assicurandosi dell'ingresso della medesima non solamente col tenerne
egli proprio le chiavi della porta, ma con haver deputate accuratissime,
e raddoppiate guardie di soldati intorno, ed alla porta della torre, ed alle
mura del giardino; ne altri entrava nella torre, che una sola donna, della
quale il Re si fidava, e le dava la chiave ogni volta, che a lei occorreva andare
alla Torre con provvisioni di vitto, o d'altro.

In questo tempo morì un tal Co. Gio: di Famagusta huomo ricchissimo, ed
alquanto parente del Re, e lasciò erede delle sue immense facultà Cassandro
unico suo figliuolo; Questo giovane fece fabbricar un palazo sontuosissimo, in
cui teneva corte bandita con tanta splendidezza, che fino al medesimo Re venne
voglia d'andarvi, e lo messe ad effetto. Andatovi dunque fu dal giovane
invitato a cena, ed il Re accetto l'invito, credendo fargli conoscer, che non
era in grado di banchettare decentemente un Re all'improvviso. Ma tutto il
contrario avvenne, perché il Re fu così ben servito, e di vivande, e di musiche,
e d'ogni altra cosa conveniente ad un banchetto regio, che gli parve
che Cassandro havesse maggior possanza, che non haveva egli; onde cominciò
ad havergli invidia, ed a pensare come potesse mortificarlo; Havendo
però veduto sopra ad una maravigliosa fonte, che era nel giardino, un motto
Che diceva \textit{Omnia per pecuniam facta sunt}. Si voltò a Cassandro, e disse: Quel
motto è troppo presuntuoso, essendoci molte cose, che non si posson fare col
danaro. Al che rispose Cassandro: Sire, io ho posto quivi quel motto, perché
mi son sempre creduto, che il denaro apra la strada anche all'impossibile,
e fino a hora mi è riuscito, come appunto mi son figurato, Horsù (replicò il Re)
Già che ti da il cuore di poter fare ogni cosa col denaro, io ti do tempo un'anno
a procurare per le strade, che vorrai, di godere la mia figluola, che io
tengo nella torre guardata, come tu sai, e de dentro a questo tempo ti verrà fatto,
sarà tua moglie; quando no, la tua testa pagherà la pena. E questo fece
il Re, perché essendo entrato in sospetto della potenza di Cassandro, voleva
sotto qualche pretesto levarselo d'avanti.

Il povero Cassandro rimasto sbalordito da tal proposta, meditava di pigliarsi
bando dalla patria, quando Euripide sua Balia, saputa la cagione del suo disgusto
gli disse, che si consolasse, perché ella haveva un un suo nipote dotato di
così grande ingegno, che assolutamente gli havrebbe aperta la strada all'ingresso
nella Torre.

Questo nipote della Balia Euripide fabbricò un'Oca di legname, grande
tanto, che potesse agiatamente ascondersele in corpo un'huomo, che v'entrava,
e usciva per di sotto l'ali, e per via di certi ordinghi faceva fare a tal'Oca
tutte l'operazioni, e moti, come se fusse stata viva, ed era del tutto perfetta
se non che le mancava il becco. Cassandra fece sparger voce, che era
andato in lontani paesi; ed intanto havendo fatta portare occultamente la
detta Oca in un luogo remoto, entrò nella medesima, ed Euripide sua Balia in
abito moresco la guidava, fingendo di venir dal Cairo (dove era veramente
nata, ed allevata detta Euripide) e parlando in quella lingua ben' intesa da
Cassandro, toccava con una bacchetta l'Oca, ed era il concerto, che Cassandro
per via di certe Zampogne facesse cantar l'oca. L'astuta Balia, accennate
a pena l'operazioni dell'Oca, andava dicendo, che a volerla vedere
operar cose galanti, e maravigliose, bisognava spendere; e però il popolo,
messa insieme buona somma di monete, la diede alla Balia, la quale fece fare
all'Oca diverse belle operazioni.

Arrivò la fama di quest'Oca all'orecchio del Re, e della Regina, onde
fattala venire a se, dopo haverla veduta operare, regalata Euripide, la mandarono
ad Alcenia loro figliuola per farle pigliar qualche spasso, e divertimento
ne i giuochi dell'Oca; la quale condotta nella Torre, il negozio andò in
maniera, che per via de trattati della Balia, Cassandro nello stare in camera
d'Alcona ascoso in quell'Oca, si godé Alcenia, e si diedero la fede di sposi.
Fatto questo, Cassandro accomodò all'Oca il becco; e con la Balia ascosto nell'Oca
se n'usci della torre, e presentatasi la Balia con l'Oca d'avanti al Re, ed
alla Regina per domandar licenza; i Re disse: Quest'Oca ha il becco, e prima
non l'havea? E la Balia rispose: Non se le era messo, perché non era
ancor fatto: e Vostra Maestà tenga a memoria quel che ora ha detto.

Fra pochi giorni spirò il termine, dentro al quale Cassandro doveva haver
goduta Alcenia, onde il Re se lo fece condurre avanti, e Cassandro disse; Sire
V.M. faccia venire Euripide mia Balia. Il Re lo compiacque, e comparsa
Euripide con l'Oca, fu dal Re subito riconosciuta, ed ella gli disse: V.M. si
ricordi \textit{che è fatto il becco all'Oca}; e fatta quivi condurre l'Oca fece entrarvi
dentro Cassandro, e lo fece fare le solite operazioni, acciò che il Re
conoscesse che quella era la stessa Oca, che in quella stessa maniera era dimorata
più giormi con Alcenia nella Torre: onde il Re conosciuta l'astuzia di
Cassandro, e saputo più precisamente il fatto, e che Alcenia era gravida, ed
havea data la fede di sposa a Cassandro, confermò il matrimonio per osservar
la parola, contentandosi di cedere alla disposizione del fato;
\end{adjustwidth}

E da questa travestita
trasformazione di Giove in Cigno è nato il proverbio: \textit{È fatto il becco
  all'Oca}; che significa (come habbiamo detto) il negozio è fatto, o perfezionato.
Questa, o simile novella leggesi in quelle di Giovanni detto il Pecorone.

\end{description}

\section{Stanza XIV \& XV.}
\begin{ottave}
\flagverse{14}O questa (disse il Duca) è veramente\\
Da pigliar con le molle; Ch'un samaro\\
Possa col cuore ingravidar la gente;\\
Vedi non ti son finto, io non la paro.\\
Hor su il provar non ha a costar niente,\\
E quando mi costasse anco ben caro,\\
Vo farlo, per veder, se ciò riesce;\\
Però si mandi al mar per queste pesce.
\end{ottave}

\begin{ottave}
\flagverse{15}Benche fusse costui com' una pina\\
Tanto largo, ignorante, e discortese;\\
Per non balzar un tratto alla berlina,\\
I pescatori vennero in paese:\\
Così pescando lungo la marina,\\
Questo benedett' asino si prese,\\
E il cuor n'un bel bacino inargentato\\
A suon di pive al 'Duca fu portato.
\end{ottave}

Il Duca sentendo che il cuor d'un' Asino marino era atto a ingravidar la
moglie, si ride del mago; ma tuttavia era così grande il desiderio d'haver figliuoli,
che volle provare, e comandò che i pescatori vedessero di trovarlo, ed essi finalmente
lo presero, e portarono il cuore al Duca.

\begin{description}
\item[È DA pigliar con le molle] È una grossa minchioneria, è uno sproposito grandissimo.
  \textit{Molle} intendiamo quello strumento di ferro, che serve per pigliar carboni
  ardenti, ec.
\item[VEDI] Questo termine ha del giuratorio, quasi dica: \textit{in fede mia}, ec, \textit{io non
lo credo}, \textit{Credi a me che tu fai male}, ec, Vedi sotto: \cstan[8]{63}.
\item[NON la paro] Non la credo. Tratto dalla Riffa, o Massa giuoco di dadi, nel
quale quando uno tien la posta dice; \textit{Paroli},  e non la tenendo dice \textit{Non la paro}.
\item[LARGO come una pina] Si dice \textit{largo com' una pina verde}, la quale strettissima,
e ben serrata; Comparazione ironica, perché huomo \textit{largo} vuol dir liberale, ed
huomo \textit{stretto} vuol dire avaro, e tenace; Sì che sendo la pina verde strettissima,
comparandosi un huomo a questa; s'intende trettissimo, cioè tenacissimo, avarissimo,
che i Latini dissero \textit{Laro sacrificat}; che suona, Gli è divoto della folaga,
la quale perché è di natura vorace, serviva a i latini per esprimere un huomo
avido del denaro, e lo dicevano \textit{Larus hians}.
\item[IGNORANTE] Uno che non sa. Vedi sopra \cstan[1]{73}. Ma vale ancora
  per \textit{ingrato, zotico, villano, e poco amorevole}, ed in questo luogo è preso in tal senso
  nel quale è sempre, o per lo più preso nel contado.
\item[PER non balzare] Cioè per non andare. Si costuma dire balzare per andare,
  o cadere in cose di disgusto, come \textit{balzar infermo in un letto}, \textit{balzare in una
    prigione}, ec. Non si direbbe \textit{balzare a un banchetto} e simili. \textit{Per non balzare in una prigion, quanti noi siamo, sarà necessario che altri di noi balzino in campagna, ed altri si
salvino in Chiesa}, Disse l'Autore, che scrisse la vita di quei tre famosi ladri Fiorentini.
\item[BERLINA] È una specie di tormento, o gastigo, che si dà a i ladroncelli
mettendo loro al collo un' anello di ferro incatenato a una colonna, o a un muro
in luoghi pubblici, e più frequentati della città, e quivi si lasciano esposti
all'insolenza della plebe. Quel strumento si chiama ancora Gogna. Vedi sotto C. 3.
stan. 62. e \cstan[]{50}.
\item[VENNERO in paese] Cioè comparvero, si lasciaron trovare. Esprime un ritrovamento
  di cose ascoste; Ed è lo stesso che \textit{venire in scena} detto sopra nel Cant. 1. stan. 2.

\item[QUESTO benedetto Asino] L'epiteto \textit{benedetto} in tali occasioni vuol dire tanto
  bramato. Io cerco del tale, del quale ha grandissimo bisogno, e questo benedetto
  huomo non si trova.
\item[BACINO] o bacile. È  un piatto d'argento, o d'altro metallo grande più
della solita misura de i piatti da tavola, e serve propriamente per ricever l'acqua,
che si dà alle mani alle tavole de' grandi, se ben s'adopra anche in molte altre occasioni,
e per altri effetti.
\item[PIVA] Dicemmo, che cosa sia sopra C. 1, stan.\ 34. alla voce \textit{cornamusa}. I
  contadini sogliono per il maggio andare attorno cantando, e suonando la Cornamusa,
  ad effetto di ragunar denari per far con essi regalo a qualche luogo pio,
  e ricevono l'elemosine, che vengono lor fatte in un bacino, ed in un'altro portano
  quel tal regalo, che voglion fare, o vero l'appendono ad un ramo d'alloro,
  o altro albero, e dicono questa lor gita, \textit{andare a cantar maggio}. Tal costume
  tocca il nostro Autore con questo modo di portare \textit{il cuore dell'Asino marino} al
  Duca.
\end{description}

\section{Stanza XVI — XVIII}

\begin{ottave}
  \flagverse{16}Ed egli preso il prelibato Cuore,\\
Lo diede al Cuoco, al qual mentre lo cosse,\\
Si fece una trippaccia la maggiore,\\
C'a i dì de' nati mai veduta fosse,\\
Le robe, e masserizie a quell'odore\\
Anch'elle diventaron tutte grosse,\\
E in poco tempo a un'otta tutte quante\\
Fecer d'accordo il pargoletto infante.
\end{ottave}

\begin{ottave}
\flagverse{17}Allor vedesti partorire il letto\\
Un tenero, e vezzoso lettuccino,\\
Di qua l'armadio fece uno stipetto,\\
La seggiola di là un seggiolino,\\
La tavola figliò un bel buffetto,\\
La cassa un vago, e piccol cassettino,\\
E il destro canteretto mandò fuore,\\
C'una bocchina havea tutta sapore.
\end{ottave}

\begin{ottave}
\flagverse{18}Il Cuoco anch'egli poi non fu minchione,\\
Perché bucar sentitosi n'un fianco,\\
Si vedde prima uscirne uno stidione;\\
Dipoi un Guatterino in grembiul bianco,\\
Ch'in far vivande saporite, e buone,\\
Fu subito squisito, e molto franco,\\
E in quel ch'il padre stette sopr'a parto,\\
Cucinò in Corte, a lui, e al terzo, e al quarto.
\end{ottave}

Il Duca dette il Cuore al Cuoco, il quale nel cucinarlo ingravidò, sì come ancora
tutti gli arnesi, e masserizie, che ne sentirono l'odore, ed a una medesima
hora partorirono.

Qui vorrei, che il lettore si ricordasse che il Poeta, nel comporre quest'Opera
ha havuto per fine il mettere in verso quelle novelle, che dalle Donne son raccontate
ai Fanciulli (come habbiamo detto) e che però sta dentro a' termini di
quelle favole, le quali come per lo più inventate, e composte da quelle medesime
donnicciuole, non possono superare la capacità di queste, ne di quelli, e si
contentasse di non prender ammirazione nel sentir da lui una cosa tanto favolosa,
e fuori del naturale, come è il far partorire le masserizie, e d'osservare, che
ancora Gio. Batista Basile, che pur fu homo dotto, nel suo Cunto de li Cunti
ha descritto questa, ed altre novelle simili, a solo oggetto di trattenere li piccirilli,
come egli dice.

\begin{description}
\item[PRELIBATO] Vuol dire una cosa gustosa, o singolare, ma significa ancora
  leggiermente narrata, o detta avanti, come è nel presente luogo, che significa
  il suddetto, o accennato cuore; ed habbiamo anche il verbo \textit{prelibare} Dan.
 Purg. Cant, 10.
\begin{verse}
Hor ti rimanclettor soprail tua banco
Dietro pensando a cio, che si preliba.
\end{verse}
\item[A dì de nati] Non nacque mai veruno, che vedesse un ventre maggior di quello,
  che haveva il cuoco. E un termine, che amplifica la voce \textit{mai}; V.g. Nessuno
  di quelli, che sono stati al mondo, mai vedde, ec. \textit{Post bominum memoriam}.

\item[A un'otta] A uno stesso tempo; a una medesima hora. Usandosi da noi spesso
la voce otta in vece d'hora: \textit{allotta} in vece \textit{d'allora}, \textit{Che otta è egli?} in vece di che
hora e egli?
\item[FECER d'accordo il pargoletto infante] S'accordarono a partorire a un'hora
medesima.
\item[LETTUCCCINO] Intende piccolo lettuccio, Ma lettuccio intendiamo una gran
cassa, la quale per di dietro ha una spalliera, e dalle testate i bracciuoli, sopr'alla
quale è solito tenersi uno strapunto, e serve per riposo, e per dormirvi sopra
dopo desinare.
\item[ARMADIO ec] Arnese di legno per riporvi ogni sorte di roba, il quale per lo
  più si tiene affisso, o accosto al muro, e si apre come le porte, ed ha dentro diversi
  palchetti, o cassette; e per stipetto qui intende piccolo armadio.
\item[BUFFETTO] Intende piccola tavola.
\item[DESTRO] Quello che diciamo anco luogo Comune, ed è quello, dove si va
a scaricare il ventre.
\item[CANTERETTO] Piccolo Cantero, e questo è un vaso di terra, o di rame
o d'altra materia, il quale si mette dentro alle predelle per recipiente all'uso
suddetto, chiamato così per esser per lo più di figura: simile a quel bicchiere che
i Latini chiamavano Cantharas.
\item[UNA bocchina havea tutta sapore] Il Poeta scherza, sapendosi bene, che simil
  sorte d'arnesi suol' esser sempre fetida, e però dice \textit{che era tutto sapore}, cioè sapeva
  di qualcosa.

\item[MINCHIONE] Vuol dir semplice, corrivo: Ma qui vuol dire uno, che non
fa meno di quello, che fanno gli altri v.g. \textit{Se tu pigli della tal cosa, non voglio esser
Minchione, ne voglio pigliar' anch' io}.
\item[SCHIDIONE] o stidione, E questo ultimo è più comune, Vuol dire quello
strumento da cucina, nel quale s'infilza la Carne, o Uccelli, per quocerli arrosto,
\item[GVATTERINO] Diminutivo di Guattero, che è colui, che serve d'aiuto al
cuoco. Qui intende piccolo cuoco.
\item[GREMBIVLE] È un panno, col quale si cinge la persona sotto lo stomaco
per difendere il vestito da' gli untumi; detto così \textit{quia regit gremium}, ed in altri
luoghi d'Italia \textit{Senale} quia \textit{sinum regit}, e molti \textit{Zinale} da \textit{Zinne}.

\item[MOLTO franco] La voce franco, che vuol dir libero, ci serve ancora per
  esprimere un'huomo ardito, coraggioso, pratico, o disinvolto, come intende
nel presente luogo.
\item[SOPRA parto] Quel tempo, che le donne stanno nel letto dopo haver parto
  rito, per riaversi da gli sconcerti cagionati loro dal parto, diciamo: Star sopr'a parto.
\end{description}

\section{Stanza XIX. \& XX.}

\begin{ottave}
\flagverse{19}La Duchessa ch' il cuore havea inghiottito,\\
Cotto ch'ei fu con ogni circostanza,\\
Anch'ella con gran gusto del marito\\
Stampò due Bamboccioni d'importanza;\\
Grazie, e belleze haveano in infinito,\\
E così grande, e tanta somiglianza,\\
Tant' eran fatti uguali, ed a capello,\\
Che non si distinguea questo da quello.
\end{ottave}

\begin{ottave}
\flagverse{20}Crebbero insieme, ed all'adolescenza\\
Pervenuti mangiaro il pane affatto;\\
Nel far santà, nel far la riverenza,\\
Hebbero il corpo a meraviglia adatto:\\
Tra lor non fu mai lite, o differenza,\\
Ma d'accordo voleansi un ben matto;\\
L'Infante Floriano uno hebbe nome,\\
E quell' altro Amadigi di Belpome.
\end{ottave}

La Duchessa pure partorì due bellissimi figliuoli, tanto simili di fatteze, che
non si distinguevano l'uno dall'altro. Questi crebbero, e furono allevati con
buona creanza, e fra di loro cordialmente s'amarono. Uno di essi hebbe nome
l'Infante Floriano, che vuol dire Raffaello Fantoni, e l'altro Amadigi di Belpome;
E questo è nome a caso.
\begin{description}
\item[STAMPO' due bamboccioni d'importanza] Partorì due bellissimi figliuoli, e che
  havevano tutte le condizioni, e parti desiderabili; E nota che il termine \textit{d'importanza}
  usatissimo da noi in simili occasioni, vale in questo caso quanto il termine
  di garbo, e per esprimere una tal quale perfezione del subietto. Il Lalli En. Tr.
  \cstan[1]{54}. dice.
\begin{verse}
E produrrà, se ben non senza duolo,
Due garbati bambocci a un parto solo.
\end{verse}
\item[A capello] Per l'appunto. E il latino \textit{ad unguem}. Termine usato da coloro,
  che si regolano col filo nello squadrare, come sono i muratori, ec. E vuol dire
  non vi corre la grossieza d'un capello dall'uno all'altro; ma si usa in ogni congiuntura
  di paragonare, o misurare una cosa con l'altra, non solo in quantità,
  come \textit{Ho riscontrato i denari, e tornano a capello}; ma anche nella qualità come nel
  caso nostro, che s'intende: erano uguali di mole di corpo, e simili di fatteze.
\item[MANGIAR il pane affatto] Mangiar bene, e senza far rosumi, o tozi; ma
  significa huomo di buon pasto. Vedi sotto \cstan[8]{56}.
\item[FAR santà] È lo stesso, che far la riverenza; ma è un termine, che è proprio
  dei bambini, quando cominciano a imparare a andare, che quel lor muoversi
  timidamente e detto dalle balie \textit{far santà}, o pure è, quando fanno la riverenza
  baciando altrui la mano; ed è così detto fare sanità, cioè fare salute; salutare.
  \textit{Diciamo insegnare al Bue far santà} per intendere: \textit{Insegnar le scienze, o i
    termini civili a un'huomo zotico, villano, e di difficile apprensione}.
\item[SI volevano un ben matto] S'amavano grandemente, o svisceratamente. È quel
  termine \textit{Mactus}, del quale habbiamo detto sopra \cstan[1]{76}.
\end{description}
\section{Stanza XXI \& XXII}
\begin{ottave}
\flagverse{21}Arrivati che furono ambiduoi\\
A conoscer homai il pan da' sassi,\\
E saper quante paia fan tre buoi,\\
Se ben dal padre havean de gli spassi,\\
Vedendosi già grandi impiccatoi,\\
Ed a soldi tenuti bassi bassi, \\
Ostico gli pareva, e molto strano, \\
Ed in particolare a Floriano.
\end{ottave}

\begin{ottave}
\flagverse{22}Di modo che sdegnato, come ho detto,\\
Ch'il Duca per la sua spilorceria\\
Ogn'hor vie più tenevalo a stecchetto,\\
Un dì si risolvette d'andar via,\\
Ma tacquelo per fare il gioco netto,\\
Fuor ch'al fratello, al qual n'una osteria\\
Disse (veduto havendo a un fiasco il fondo)\\
Volersene ramingo andar pel mondo.
\end{ottave}

Cresciuti questi due Giovani, ed arrivati a conoscer il ben dal male, vedendosi
così grandi pareva lor malagevole il non haver denari, perché il padre per la sua
spilorceria non gliene dava, di che più d'Amadigi sentiva disgusto Floriano, onde
si risolvette d'andato via, e perché l'adempimento di tal sua risoluzione non gli
fusse impedito, non ne parlò ad alcuno, fuori che al fratello Amadigi.

\begin{description}
\item[CONOSCER il pan da sassi] e \textit{saper quante paia fan tre buoi}, Significano lo
  stesso, cioè conoscere il ben dal male. Hor. disse, \textit{Novit quid distent aera lupinis}. Si
  dice ancora in questo proposito \textit{Sapere a quanti dì è San Biagio}, E questo detto ha
  origine da un costume antico, il quale era in Firenze, che i ragazi\footnote{sic. Dalla Wikipedia: anche se l'ortografia italiana distingue in posizione intervocalica, per motivi storici, una -z- scempia e una -zz- doppia, a tale differenza grafica non corrisponde nessuna differenza di pronuncia: la zeta intervocalica, che sia scritta scempia o doppia, che sia sorda o sonora, si pronuncia sempre e comunque intensa, cioè come se fosse scritta doppia. } fattori delle
  botteghe d'arte di seta, che son situate nel Mercato Nuovo vicino alla Chiesa di
  S. Biagio, havevano licenza, passato il dì della festa di esso Santo (che sarebbe
  alli 2. di Febbraio, e se ne fa alli 3. per causa della Purificazione, il che
  ha dato occasione di usare questo dettato) di fare alle sassate, e pigliarsi ogni
  sorte di passatempo in alcune hore del giorno, ed abbaadonare la bottega per infino
  a tutto il giorno di Carnovale; e per questa causa era quel giorno tanto desiderato
  da i ragazi, che sapevano benissimo il dì, che si solennizzava la detta
  festa; onde colui, che non sapeva tal giorno, era fra i ragazzi riputato un baggeo,
  e che non havendo notizia delle cose del mondo (giudicata da loro questa
  una delle più importanti) non fusse persona abile, e di tanto giudizio da saper
  fare i fatti suoi. E questo proverbio s'è fatto poi comune a tutti gli huomini per
  intendere un'huomo scervellato, melenso, e buono a poco. Il Lasca Nov. 4. dice:
  \textit{Lo Scheggia, ed il Pilucca, che sapevano a due once, quanto colui pesava, ed a quanti
    dì è San Biagio}.
\item[SE ben dal padre havean de gli spassi] Se bene il padre dava loro de gli avvertimenti,
  e passatempi. Nota che per scherzare il nostro Poeta, subito che ha detto \textit{buoi}
  seguita \textit{dal padre}, e questo fa per toccare quel costume burlesco, il quale è
  in Firenze (ma pero fra gente bassa) che quando uno nomina \textit{bue}, \textit{becco}, o
  \textit{castrone}, l'altro dirà \textit{di tuo padre}, e dicendo \textit{vacca}, dirà di tua madre, e simili, Vedi
  sotto \cstan[12]{49}. annot. al termine \textit{morire con la grillanda}.
\item[GRANDI impiccatoi] Proibiscono le leggi l'impiccare chi non passa 18 anni;
e di qui noi diciamo \textit{grandi impiccatoi}, cioè abili a esser'impiccati, per intender
quelli, che passano la detta età di 18. anni.
\item[A SOLDI tenuti bassi bassi] Tenuti con pochi denari. Traslato dall'acque, delle
quali quando ne son poche nei laghi, pozzi, o fiumi, si dice basse. Vedi sotto in
\cstan{61}., e parlando d'uno che habbia pochi denari si dice: \textit{L'acgue
son basse} sì come intese colui con quel suo motto \textit{L'acque son basse, e l'oche hanno
gran sete}, cioè \textit{Alle gran voglie i danari son pochi}.
\item[SOLDO] Vale per intender danari, riccheza. E soldo è moneta immaginaria
(hoggi in Firenze effettiva di bronzo) che vale tre de nostri quattrini; Spesso usiamo
questo termine per una certa generalità: Il tale ha de' soldi, de' quattrini, dell'oro,
per intendere è ricco, non che habbia quantità di soldi, di quattrini, o d'oro effettivamente,
ma molti ne vale il suo stato; E qui intende Monete.
\item[OSTICO] Spiacevole, Malagevole, Insopportabile. È il Latino \textit{Hosticus}, che
vale per cosa da nimico.
\item[STRANO] Qui ha lo stesso significato d'ostico. Vedi sotto C.\ 3.\ stan.~1. E per
altro vuol dire stravagante da \textit{extraneus}. E molti dicono \textit{strano} a uno che habbia
cattiva cera, e per infermità sia mal condotto.
\item[SPILORCERIA] Sordidezza, Avarizia. Io credo che questa parola venga da
Pilorci, che i pellicciai chiamano quei ritagli di pelle, che non essendo buoni a
metter' in opera, gli riducono in spazzatura, la quale poi vendono per governare
i terreni, e si dica spilorcio quasi huomo vile, ed abietto quanto sono questi pilorci.
\item[TENER' uno a stecchetto] Fare star'a segno, o far patire uno di quello, che
egli ha bisogno; come non lo lasciar mangiare quanto ei vorrebbe; o haver de'
danari quanti bramerebbe. Quand'uno per la scarsezza di danari vive miseramente
si suol dire: \textit{Il tele si difende, si schermisce}, ec. ond'io non son lontano da;
credere, che questo termine sia corrotto, e che si dovesse dire a \textit{stocchetto} da
stoccheggiare, che è l'istesso che schermirsi, e può significare essere scarso, o haver
bisogno di denari.
\item[VEDUTO il fondo a un fiasco] Dopo haver bevuto un fiasco di vino; e così haver
  veduto il fondo di dentro del fiasco; ed in sustanza qui vuol dire; Dopo haver
  bevuto molto bene, o assai.
\item[ANDAR ramingo pel mondo] Andarsene errante. Ramingo vien da ramo, e
si dice \textit{Ramingo} de gli uccelli di Rapina, come esprime il Crescenzio nel Cap, 3.
della bontà degli Sparvieri lib. 18. con le seguenti parole: Si chiama \textit{nidiace, o
vero che di nidio uscito di ramo in ramo va seguitando la madre, e però si chiama Ramingo}.

Ed alli sparvieri si danno tre nomi, cioè \textit{Nidiace}, che è quello, che è cavato di
nidio, ed allevato. \textit{Ramingo} quello che uscito di Nidio non fa gran volate; e
\textit{Grifagno} quello, che già passato l'anno ha mutato alla Campagna. Ma questo
non fa a proposito nostro, bastandoci, che a similitudine di tali uccelli, dicesi
Andar ramingo colui; che hora va in un luogo, hora s'incammina in un'altro,
senza sapere positivamente, dove egli voglia andare.
\end{description}

\section{Stanza XXIII.}
\begin{ottave}
\flagverse{23}Anadigi a distorlo tutto un giorno \\
S'arrabbiò, s'aggirò com'un Paleo; \\
Ma perché quanto più gli stava intorno \\
Egli era più ostinato d'uno Ebreo, \\
Tu vuoi ir disse: e vero? o va in un forno:\\
E dopo un grande, e lungo piagnisteo;\\
Hor su vanne (diss'egli) io men'accordo,\\
Ma lasciami di te qualche ricordo.
\end{ottave}

Amadigi sentita questa risoluzione del fratello, molto s'affaticò per distornelo;
ma veduto che per la di lui ostinazione s'affaticava in vano, concorse con lui,
con questo però che gli lasciasse qualche ricordo di se,
\begin{description}
  \item[PALEO] Così chiamiamo una specie d'erba, che nasce intorno alle lagune.
    Ma diciamo anche Paleo uno strumento di legno, che serve per trastullo, e giuoco
    de' ragazzi, il quale è di figura piramidale all'ingiù; e nella testata, che viene
    di sopra ha un manichetto tondo, il quale avvoltato con uno spago, o cordicella
    s'infila in un'asticella, bucata, e tirandosi quello spago si svolta, ed il \textit{Paleo}
    scappa dal buco dell'asticella, e va per terra girando, portato dall'impulso di quello
    spago. Tale strumento da i Latini è detto \textit{Turbo} forse dalla figura piramidale.
    Verg. 7. Aneid. \textit{Ceu quondam torto volitans sub verbere turbo}, Tibull. \textit{Namque agor,
    ut per plana citus sola verbere turbo}, Dante nel Paradiso C. 18.
    \begin{verse}
      Ed al nome del alto Maccabeo
      Vidi moversi un'altro roteando
      E letizia era ferza del paleo.
    \end{verse}

E dice, così, perché a tale strumento si fa continovare il girare perquotendolo
con una sferza, dopo che egli ha havuto il primo moto, ed impulso dal suddetto
spago. Ed il proverbio \textit{aggirarsi come un paleo} vuol dire affaticarsi assai, e conchiuder
poco; che i Latini pure dissero \textit{Trochi in morem circumagi}, perché dicon \textit{Trochus}
tanto il paleo, che la trottola, portandolo dal Greco \textit{Trechos}, che vuol dir
ruota, o altro strumento che giri. Vedi sotto \cstan[6]{22}. E forse anche la voce latina
\textit{Turbo} significa tanto il paleo, che la trottola, perché \textit{Turbo} vuol dire
ogni cosa che habbia figura Piramidale, a rovescio, cioè il largo di sopra, e da
piede acuta, come appunto è il Paleo, e la Trottola; se bene non sono lo stesso
come ci testifica una certa cantilena assai praticata fra i ragazi, che dice,
\begin{verse}
E il Cristiano non è giudeo,
E la trottola, non è paleo,
E paleo non è trottola, ec.
\end{verse}
\item[PIÙ ostinato d'uno Ebreo] Ostinatissimo, che non si trova nazione più ostinata
nella sua legge, che quella de gli Ebrei, che però ha meritato il titolo, che le da
la santa Chiesa di perfidi. Cino da Pistoia, \textit{O voi, che sete ver me si giudei}: cioè
perfidi.
\item[VA in un forno] Va dove tu vuoi. E specie d'imprecazione, che suol far' uno
vinto dall'impazienza. E si suol dire anche in questo proposito: \textit{Va in malora}, \textit{va
al diavolo}, \textit{va in galea}, e simili, Abi\textit{ in malam crucem}, e Plaut. Epid. Atto 1. sc.2.
disse: \textit{Malim istius modi mihi amicos furno mersos, quam foro}.
\end{description}
\section{Stanza XXIV — XXVII}
\begin{ottave}
\flagverse{24}Allor per soddisfarlo Floriano,\\
Acciò che più tener non l'abbia in ponte,\\
Con un baston fatato, c'havea in mano\\
Toccò la Terra, e fece uscirne un fonte\\
E disse: Quindi poi ben che lontano\\
Vedrai s'io vivo, o s'io sono a Caronte;\\
Perché quest'acqua ogn'or di punto in punto\\
In che grado so sarò diratti appunto.
\end{ottave}

\begin{ottave}
\flagverse{25}S'al corso di quest'acqua porrà cura,\\
Tutto il corso vedrai di vita mia;\\
Mentr'ella è chiara, cristallina, e pura,\\
Di pur ch'io viva in festa, ed allegria; \\
Ed all'incontro, se torbida, e scura\\
Ch'ella mi va come dicea la Cia;\\
Ma quand'ella del tutto ferma il corso,\\
Di ch'io sia ito a veder ballar l'Orso.
\end{ottave}

\begin{ottave}
\flagverse{26}Ciò detto in capo il berrettin si serra,\\
Mette man, chiude gli occhi, e stringe i denti\\
E dà si forte una imbroccata in terra,\\
Ch'il ferro entrovvi fino ai fornimenti.\\
In quel che i grilli, e i bachi di sotterra\\
Sgombrano tutti i loro alloggiamenti\\
Pullula fuori un cesto di mortella,\\
E di nuovo Florian così favella
\end{ottave}

\begin{ottave}
\flagverse{27}Fratel mio caro, questa Pianta ancora\\
Com' io la passi ti darà ragguaglio,\\
Cioè mentr'ell'è verde, anch'io allora\\
Son vivo, fresco, e verde com'un'aglio;\\
E quand'ella appassisce, e si scolora,\\
Anch'io languisco, od ho qualche travaglio,\\
In somma s'ell'è secca, leva i moccoli,\\
Per farmi dire il canto in scarpe zoccoli.
\end{ottave}

Floriano per contentare il fratello, toccò la terra con un bastone incantato,
che haveva in mano, e ne fece nascere una fonte, e disse che dalla mutazione di
quell'acque haverebbe egli conosciuto lo stato, nel quale egli si trovasse. Dipoi
messe mano alla spada, e con essa bucò la terra, e scappò fuori un cesto di mortella;
E mostrò ad Amadigi, come egli si davea contenere in conoscere ancora
da questa mortella, in che grado egli si trovasse.
\begin{description}
\item[TENERE in ponte] Tener un sospeso, o irresoluto. I Latini pure dissero: \textit{In
  pontes detinere}; e però stimo, che questo nostro detto venga dall'uso antico de'
  Romani, che nell'elezione de i Magistrati chiamavano \textit{Pontes} quelle piccole tavole,
  sopr'alle quali eran posate le paniere dei voti; di che fa menzione Cic. 1.
  Rhet. \textit{Pontes disturbat, Cistas deijcit}; e tanto stavano incerti, e sospesi coloro, che
  pretendevano, quanto le ceste de i voti stavano sopra i detti Ponti; E pero dicendo:
  \textit{Ego sum super pontes}, vuol dire il mio Voto è ancora nelle Ceste, o coperto,
  e per conseguenza io sono sospeso, ed incerto di quel che habbia a esser di
  me. E ci serve poi questo detto \textit{Tener' uno in ponte} per esprimere; trattener' uno
  con le speranze, o con altro secondo il subietto.
\item[SONO a Caronte] Son morto. Son fra l'anime, le quali passano la Barca di
  Caronte, che secondo la falsa credulità de' Gentili era il Navalestro, il quale conduceva
  l'anime de i morti con la Barca alla Città di Dite. Vedi sotto \cstan[6]{19}. \& seqq.
\item[COME dicea la Cia] Mi va male, e peggio. Che questo voleva inferire una
  tal Cia, o Scia Fruttaiola con un detto sporco da lei molto usato.
\item[SON ito a veder ballar l'Orso] Anche questo detto significa son morto.
\item[IN capo il berrettin si serra, ec] Con questi due versi esprime uno, che s'accinga
  a fare un'operazione, nella quale sia necessario usar molta forza, perché in
  essi mostra quelle azioni, che per lo più son solite farsi in simili congiunture.
\item[METTE mano] Quando diciamo assolutamente metter mano; intendiamo metter
  mano all'armi. \textit{Distringere ensem}.
\item[SGOMBRANO] Vanno via; Si partono.

E qui non mi pare fuor di proposito il notare una generale portata dal
Varchi nel suo Hercolano, cioè che la lettera \letter{s} aggiunta nel principio di qualsivoglia
dizione nel nostro parlare ha la forza di privazione, come appresso a i
Latini la particela \textit{in} ha forza di negativa, come \textit{doctus}, \textit{indoctus}, ec. Ed
appresso di noi \textit{calzare}, \textit{scalzare}, ec, Ha però questa regola anch'essa le sue eccezioni,
come \textit{sbalordito} vuol dir \textit{balordo}, e non vuol dire \textit{senza balordaggine}; \textit{Turbare}, \textit{sturbare},
\textit{disturbare}, che suonano lo stesso con l'aggiunta, che senza. Talvolta
ancora s'aggiunge alla detta \letter{s} la particella \textit{di}, e particolarmente quando la
parola comincia per lettera vocale, come \textit{amare}, \textit{disamare}; \textit{interessato},
\textit{disinteresato}, ec.
\item[CESTO] Intendiamo pianta di virgulto, o d'erba, come Cesto di lattuga, di
  mortella, ec. Se bene de i virgulti si dice anche \textit{Pianta}, come si vede nella presente
  ottava 27. \textit{Fratel mio caro questa Pianta ancora}. Viene dal latino \textit{Cespes}, e noi
  pure diciamo Cespuglio. Io stimo, che pianta sia nome generico, poiché serve,
  per tutti li vegetabili, dicendosi Pianta di prezemolo, pianta di grano, e pianta
  di querce, ec. E non si direbbe di tutti cesto, ne cespuglio.
\item[VERDE come un'Aglio] Un bel verde si paragona ad un'Aglio, perché questo
  ha le sue frondi di bellissimo color verde, e che si mantengono verdi,
  è segno di sua perfezione. E però dicendosi \textit{Il tale è verde come un'aglio}, s'intende:
  è di sanità perfetta Virc. \textit{cruda Deo, viridisque senectus}. Horat. \textit{Dumque
    virent genua}, Questa similitudine si piglia da tutte le piante, la sanità delle quali
  s'argumenta dall'esser ben verdi, che dimostra non havere esse patito, ne essere
  in grado di seccarsi. Ed alle volte s'intende uno di mala sanità quando si dice
  \textit{verde come un'aglio}, ma s'intende non la frescheza, che denota il verde dell'aglio,
  ma il colore, che essendo verde nella faccia dell'huomo denota poca sanità.
\item[LEVA i moccoli per farmi dire il canto in scarpe, e zoccoli] Compra la cera per farmi
  il funerale: che moccolo vuol dire ogni piccola candela di cera, e qui è preso
  per ogni sorte di candele di cera. E quel \textit{farmi dire il canto scarpe zoccoli} è detto
  giocoso usato fra i nostri Contadini; il qual detto non è forse senza fondamento
  ne affatto improprio, che possa haver origine dalla diligenza, che si pone nel fare,
  che i morti quando son portati alla sepoltura habbiano, se sono huomini un paio
  di scarpe nuove, e se son donne un par di pianelle, o zoccoli nuovi; e \textit{zoccolo} è una
  scarpa col fondo di legno, che serve per difendere i piedi dall'acqua, che è per terra.
\end{description}

\section{Stanza XXVIII — XXX.}
\begin{ottave}
\flagverse{28}Poi che queste parole hebbe finite,\\
Dal suo caro Amadigi si licenza,\\
Il qual rimase tutto sbigottito, \\
Però che gli dolea la sua partenza, \\
Quand' in sella Florian di già salito\\
Senza gran doble, o letter di credenza \\
Andonne a benefizio di natura\\
Con due servi cercando sua ventura.\\
\end{ottave}

\begin{ottave}
\flagverse{29}E il primo giorno fece tanta via\\
Ch'i suoi lacché spedati, e conci male\\
Si rimasero, l'uno all'osteria,\\
E l'altro scarmanato allo spedale;\\
Ond'ei più non havendo compagnia,\\
Se bene accanto havea spada, e pugnale,\\
Per non haver paura in andar solo,\\
Cantava ch'ei pareva un rosignuolo.
\end{ottave}

\begin{ottave}
\flagverse{30}Così nuove canzoni ogn' hor cantando \\
Con una voce tremolante in quilio,\\
E qualche trillettin di quando in quando\\
Alle stelle n'andava, e in visibilio:\\
Onde ai timori al fin dato di bando\\
Tirava innanzi il volontario esilio;\\
E giunto a Campi, lì fermar si volle\\
A bere, e far la zolfa per bi molle.
\end{ottave}

Floriano si parte dal fratello Amadigi, il quale ne rimase afflitto. Lasciò per
la strada i Lacché stracchi, ed egli solo si condusse a Campi, dove si fermò a bere.
\begin{description}
  \item[SBIGOTTITO] Afflitto; perduto d'animo. I Latini dissero \textit{Animo deiectus}.
Quand' uno sta allegramente diciamo: Il tale sta in gote, o sta in barba di micio.
Vedi in \cstan{48}. Sì che uno che non stia allegramente si dice \textit{non sta in
gote}, \textit{non sta in barba di micio}; E però non farebbe gran fatto, che questa voce
sbigottito venisse dallo Spagnuolo bigottes, che vuol dir basette, e che per la lettera \letter{s}
che aggiunta al principio d'una parola ha forza di privazione (come,
habbiamo detto poco sopra) significasse senza \textit{bigottes}, che vuol dir senza basette,
cioè non in barba, non allegramente: o forse sbigottito, quasi sbattuto.
\item[A BENEFIZIO di natura] A caso; dove la Fortuna lo guidava.
\item[LACCHÉ] Servitori, che corrono a pié; e per lo più sono ragazzi o givanetti.
  Vedi sotto. C.~11. stan. 9.
\item[SPEDATI] In questo caso non vuol dir Senza piedi, ma con i piedi affaticati, e
  stanchi dal viaggio.
\item[SCARMANATO] Scarmana è una specie d'infermità, che viene a coloro,
  che dopo essersi soverchiamente riscaldati per violente fatica, o viaggio si raffreddano
  o col bere, o con lo stare al vento, o in luoghi freschi, e si dice: \textit{Pigliar
    una scarmana}, o \textit{scarmanare}. È forse specie di quel male che i medici chiamano
  Pleuritide, ed è comunemente chiamato mal di petto. Qui intende Affaticati
  dal viaggio, in maniera che l'anelito se li rendea difficile, e però non potevano
  camminar più.
\item[CANTAVA che pareva un Rosignuolo] Il Rosignuolo, Uccelletto noto, da i
  Latini detto philomela, ha il più bello, e gagliardo cantare di qualsivoglia Vccelletto,
  e per questo quand'uno canta bene, lo paragoniamo al Rusignuolo.
\item[VOCE tremolante] Voce, che tremava per cagione della paura; Si come i \textit{trilli}
  eran fatti per timore, e si potevano dire più tosto tremoli, o interrompimenti
  di canto cagionati dalla paura, che veramente \textit{Trilli}, che sono un riperquotimento
  di voce musicale nel medesimo tuono. Horazio disse: \textit{Cantu tremulo}.
\item[IN quilio] Secondo che mi disse il Signor Nigetti, fra i musici del nostro secolo
  il Maestro; la voce \textit{quilio} significa un cantare in voce non sua, come se uno
  havesse voce di basso, e cantasse di soprano; sì che s'intende, che Floriano cantava
  per la paura in voce falsa, e non sua naturale, che i Latini secondo Cic. lib.
  3. de Orat. la dicevano \textit{Vocula falsa}. E Titinio appresso Sesto disse \textit{Succrotilla vocula}.
\item[ANDAR alle stelle col canto] Cantar in tuono alto. Se ben qui par che voglia
dire, \textit{se n'andava in gloria}, cioè cantava con gran soddisfazione, e gusto; poi
che soggiugne \textit{in visibilio} che appresso di molti de' nostri vuol dire Andarsene in
estasi, e perdere i sentimenti per il gran gusto, Matteo Franzesi nel Cap. del suo
viaggio da Roma a Spoleti dice.
\begin{verse}
\backspace Vedea passar con torvo supercilio
Qualche Satrapo tronfio, ed appoggiato
Al tappeto, n'andava in visibilio.
\end{verse}
Vergilio Egl. 5. disse: \textit{Voces ad Sydera iactare}.
Ed ottavo Aen. \textit{Effundere voces ad athera}.
\item[TIRAVA innanzi il volontario esilio] Continovava il viaggio, che egli medesimo
  s'era eletto, esiliandosi dalla propria casa.
\item[FAR la zolfa] Detto scherzoso, che signisi a Cantare, far musica, ed è composto
  di tre note musicali, la, sol, fa. Il Signor Salvador Rosa in una sua bella
  Satira parlando della musica dice,
\begin{verse}
\backspace Quanto gira la terra a tondo a tondo,
Luogo alcuno non v'è che di schiamazzi
E di zolfe non sia pieno, e fecondo.
\end{verse}
\item[PER b molle] Il b molle è chiave musicale, o segnatura di semituono; Ma
  qui dicendo \textit{far la zolfa per b molle}, si serve della voce \textit{molle} per intendere:
  ammollare la bocca, cioè bere. E così scherzando sopra alla musica, ed havendo
  detto, che Floriano cantava; soggiugne, che volevaa seguitare a cantare anche
  nell'osteria, \textit{ma per b molle}, ed intende Vuol bere.
\end{description}

\section{Stanza XXXI \& XXXII.}
\begin{ottave}
\flagverse{31}A Campi, hora spiantato alla radice\\
Dominava in quei tempi Stordilano,\\
Se ben Turpino scrive, ed altri dice,\\
Ch'ei regnasse in un luogo più lontano,\\
Hebbe una figlia detta Doralice,\\
C'havea un'occhio c'uccidea il Cristiano,\\
Ma quel che più tirava la brigata\\
È l'esser sola, e ricca sfondolata.
\end{ottave}

\begin{ottave}
\flagverse{32}Com'io dissi, Florian nella Cittade\\
Entrò per rinfrescarsi, e toccar bomba,\\
Ma il gran frastuono, ch'in quelle contrade\\
D'armi, di bestie, e d'huomini rimbomba,\\
Il sentir su pe i canti delle strade\\
Tutt'a cavallo risuonar la tromba,\\
Ed il voler saperne la cagione,\\
Lo fecero mutar d'opinione.
\end{ottave}

Il Poeta finge Città Regia il Castello di Campi, luogo vicino a Firenze, che
hoggi ha poca forma di Castello, per esser distrutto, e dice che già vi regnava
Stordilano, che hebbe una bellissima Figliuola nominata Doralice, la quale per
esser sola, e ricchissima, era da molti bramata in moglie. E perché questa non
sia creduta la stessa, che quella che l'Ariosto fa Figliuola di Stordilano Re di
Granata dice: \textit{Se ben Turpino scrive, ed altri} (cioè  Ariosto) \textit{dice, ch'ei regnasse
in un luogo più lontano}, cioè in Granata.

Floriano dunque, il quale era entrato in Campi solamente per pigliare un poco
di riposo, e rinfrescarsi, e andarsene, sentendo tanti strepiti d'armi, e romori
di tamburi, si risolve di trattenersi alquanto per intenderne la cagione.

\begin{description}
\item[HAVEA un occhio c'uccidea il Cristiano] Havea così begli occhi, che facevano
  innamorare ognuno. Questo detto vien forse dalla comune opinione di quel serpente
  da i latini detto \textit{Regulus}, e da i Greci, e da noi chiamato \textit{Basilisco}, il quale
  col solo sguardo avvelena, ed ammazza coloro, che egli mira. E molti Poeti
  nostrali per lodare l'occhio di bella donna hanno detto: \textit{Occhio di Basilisco},
  intendendo, che han forza di metter nel cuore il veleno d'amore. Apul.
  \textit{morsicanstibus oculis}.

\item[TIRAVA la brigata] Lusingava, incitava, allettava il popolo a desiderarla.

\item[RICCA sfondolata] Ricca senza fondo: Ricchissima. Diciamo \textit{Ricco in fondo},
  \textit{senza fondo}, \textit{sfondato}, o \textit{sfondolato}, per denotare una ricchezza,
  senza numero, o misura.

\item[RINFRESCARSI] Cioè reficiarsi col riposo, e col cibo. I Latini pure dicevano
  tal volta rinfrescarsi per ristorarsi, trovandosi \textit{refrigeratus} in vece di \textit{refocillatus}.

\item[TOCCAR bomba] Arrivare in un luogo e dimorarvi poco. Questo detto è
  tolto da un giuoco fanciullesco detto \textit{birri e ladri}, il quale fanno in questa maniera.
  S'uniscono molti Fanciulli, e tirate le sorti a chi di loro debba esser birro,
  chi ladro, quelli che sono eletti birri si mettono in mezzo della stanza, o piazza
  dove s'ha da fare il giuoco, e ciascuno de i ladri piglia il suo posto, il quale è
  già stato consegnato per immune; e questo luogo da essi è chiamato \textit{bomba}, che i
  latini dicevano \textit{meta} in questo medesimo giuoco usato ancora da i loro ragazzi, e
  da quelli de i Greci, se bene in qualcosa differentemente. Questi ladri vanno
  scorrendo da un luogo all' altro, e i birri procurano di pigliargli, ed i ladri,
  quando si veggono stracchi, corrono a trovare un di quei luoghi immuni detto
  \textit{bomba}, dove stando, sono franchi, ed i birri non possono pigliargli, e si guadagna,
  o si perde il premio stabilito, secondo che son convenuti d' esser presi, o non
  presi in tante gite; ed il ladro preso ( continovandosi il giuoco ) diventa birro,
  ed il birro, che ha preso diventa ladro. E perché nel toccar bomba si trattengono
  pero diciamo toccar bomba per esprimere arrivare in un luogo, e partirsene presto.
  E questa voce \textit{bomba} vien dal Greco \textit{bombeo}, che vuol dire Strepitare,
  o far suono, (donde \textit{rimbombare}) è da quel romore, che fanno i ragazzi con
  la voce, e con le mani per far conoscere che toccano il luogo immune, questo
  luogo è chiamato bomba. Diciamo \textit{tornare a bomba} che significa \textit{tornare al primo
  discorso}. Vedi sotto \cstan[8]{15}.

\item[FRASTVONO] Fracasso, Strepito, romore confuso, quasi dica fuor di tuono.

\item[CANTO] Cioè l' angolo che fanno le case a capo a una strada che volti in
  un'altra; detto così secondo alcuni, dal Greco \textit{Canthos}, che vuol dire Angolo
  dell'occhio, o dal canto, che nello sboccar delle strade in su le cantonate soleva
  farsi dagli antichi, come si cava da Verg. Egl. 3.
  \begin{verse}
    Non tu in trivijs indocte solebas
    Stridenti miserum stipula disperdere carmen?
  \end{verse}

Ma è detto dai Greco \textit{camptin}, che vuol dire Piegare.
\item[TUTTI a cavallo] Così chiamano i Soldati quella suonata di tromba, che fa intendere
  a i medesimi il montar' a cavallo, la quale par che esprima; \textit{Tutti a cavallo}.
  Costume tolto da i Latini, che per significare il suono della tromba dicevano
  secondo Servio, ed Ennio \textit{Taratantara}.
  \begin{verse}
    At tuba terribili sonitu taratantara dixit.
  \end{verse}
\end{description}

\section{Stanza XXXIIL}
\begin{ottave}
\flagverse{33}Era già scavalcato ad una ostessa,\\
Per far, sì com'ei fece, un conticino,\\
Ne altro hebbe che pane, e capra lessa,\\
Che fitra anche gli fu per mannerino.\\
Bevve al pozo una nuova manomessa,\\
Perch'il vinaio havea finito il vino;\\
Fece conto, e pagò ben volentieri\\
Poi chiese il fin di tanti Strombettieri.
\end{ottave}

\begin{ottave}
\flagverse{34}Ella rispose: E come; E non lo fai?\\
Se per Campi non è altro discorso,\\
Che havendoil Re una figlia, c'hoggi mai\\
Abbraccerebbe un'huom prima c'un'orso;\\
E perché reda ell'è bell', e d'assai,\\
Di pretendenti havendo un gran concorso,\\
Bandire ha fatto, acciò nessun si lagni,\\
Ch'in giostra chi la vuol se la guadagni.
\end{ottave}

\begin{ottave}
\flagverse{35}Ma c'occorre ch'in ciò più mi distenda,\\
Mentre la cosa è tanto divulgata?\\
Però lasciami andar, ch'io ho faccenda\\
Havendo sopra un'altra tavolata.\\
Dice Florian, che ai suoi negozzi attenda,\\
Scusandosi d'haverla scioperata\\
E rimessa la briglia al suo giannetto,\\
Come un pardo saltovvi su di netto.
\end{ottave}

Floriano essendo scavalcato a un'osteria, dopo che hebbe mangiato, e pagato
intese dalla padrona dell'osteria, che quei romori di trombe si facevano
perché il Re voleva maritare la Figliuola a quel Cavaliere, che meglio si portasse
la giostra; onde Floriano montò subito a cavallo per andare a veder questa festa,
\begin{description}
  \item[FARE un conticino] Così usiamo dire per farsi intendere copertamente Andar a
mangiare all'osteria.
\item[FITTO gli fu] Gli fa fatto credere. Gli fu dato ad intendere che e' fusse
Mannerino. Il verbo ficcare usato in questi termini serve per esprimere, che
quella tal cosa fu data per maggior prezzo di quel che ella valeva, o per di miglior
qualita, che ella non era. Vien da ficcar carote, che vedremo sotto questo Cant.
stan. 70. e Cant. 6, stan. 68. Lat. \textit{imponere alicui}.
\item[MANNERINO] Specie d'agnelli castrati, che nella nostra Toscana è ottima
  nel Territorio, e contado di Pistoia, ed è carne squisita al contrario della capra,
  che è la peggiore, che si mangi, ed in particolare cotta a lesso.
\item[MANOMESSA] Quando all'Oste arriva portatogli dalla montagna il vino
  primo cavato dalla botte si dice: \textit{l'oste ha havuto la manomessa}, Ed i Fiorentini,
  che son di buon gusto, o più tosto ghiotti nel bere, lo pigliano più volentieri,
  quando è vino di manomessa, non tanto per la curiosita di gustare quel nuovo
  vino, quanto perché non piacendo loro le fondate, hanno caro di bere del primo,
  che esce della botte, onde pare che il Poeta voglia intendere, che Floriano se
  bene bevve acqua hebbe nondimeno gusto, perché era nuova manomessa, ma in
  effetto gli da la burla dicendosi che \textit{bevve una manomessa nuova} cioè insolita, non
  essendo solito, ne costume, che si manometta il pozzo, se non per le bestie.
\item[VINAIO] Cioè colui che nell'osterie dà il vino. Per maggior intelligenza di
  questo è necessario sapere, che nell'Osterie di Firenze stanno due maestri, e tengono
  garzoni differenziati; Uno di questi maestri è il padrone principale ed in
  lui dice l'Osteria, e questo si chiama il Vinaio; altro è maestro anch'egli, ma
  solamente della Cucina, della quale paga un tanto il mese di pigione al Vinaio,
  dal quale può esser mandato via. Ho voluto dir questo, perché so che a i Forestieri
  è di non poca confusione questa distinzione, perché si fanno far il conto da
  uno, e pensando d'haver finito; gli sopraggiugne poi il secondo Oste, che fa loro
  il conto della Cucina, e cresce la somma del primo conto fatto dal Vinaio.
\item[FECE conto] \  Domandò quanto dovea pagare. Trattandosi d'osterie \textit{Far conto}
  s'intende Haver finito di mangiare.
\item[STROMBETTIERI] \  Intende il romore, che fa il suono delle trombe.
\item[ABBRACCEREBBE un huom prima c'un'orso] \makebox[8pt]{} Così diciamo d'una Fanciulla,
che sia in età da maritarsi, e che sia bella, grande, e ben formata, intendendo
che sia in eta da bramar l'huomo, e da distinguerlo da un'orso, o da non fuggirlo,
come farebbe all'orso. Virg. \textit{Iam matura viro, plenis \& nubilis annis}.
\item[D'ASSAI] Valente, contrario di Dappoco: pare che suoni lo stesso che in latino
  \textit{praestans}.
\item[REDA] Vedi sopra in questo Canto stan. 12. Qui è preso nel suo proprio significato
  d'herede, o successore nelle facultà; e vuol dire che essendo ella Figliuola
  unica del Re, dovea hereditare tutto quello che egli possedeva.
\item[TAVOLATE] Così chiamano li nostri Osti tutti coloro, che vanno a mangiare
  alle tavole delle loro osterie, tanto se fusse un solo per tavola, quanto
  se fussero più, pur che seggano a mangiare a tavola.
\item[SCIOPERATA] Levata dal lavoro, o dall'opera. Vedi sopra C.~1. st.~29.
\item[GIANNETTO] Intende cavallo. Sendo i giannetti specie di cavalli, che vengono
  di Spagna del paese d'Asturia, e perciò dai Latini detti \textit{Asturcones}.
\item[PARDO] Il Gatto pardo è animal noto, come è anche nota la di tui feroce
  agilità, e destrezza; e però appresso di noi è in uso questa comparazione quando
  vogliamo intender l'agilità di vita d'alcuno. Vedi sopra C. 1. stan.~11, \textit{Le scale
  corre lesto come un gatto}.
\end{description}
\section{Stanza XXXVI — XXXVIII}

\begin{ottave}
\flagverse{36}Tocca di sproni, e vanne, e giunge in piazza\\
Dov'egli ha inteso che s'ha far la giostra,\\
Che per vedere il popol vi s'ammazza,\\
E appunto i Cavalier facean la mostra.\\
Sedeva il Re presente la Ragaza,\\
Che quanto adorna, e bella si dimostra,\\
Tanto è confusa havendo a haver consorte,\\
Non a suo mo, ma qual vorrà la sorte.
\end{ottave}

\begin{ottave}
\flagverse{37}Floriano in contemplar faccia sì bella,\\
Dove quel crudo balestrier d'amore\\
Tira frecciate, come la rovella,\\
Sentissi anch'esso traforare il core,\\
E com'huomo di marmo, in su la sella\\
Restò perplesso, e pieno di stupore,\\
Scorgendo Amor, le Grazie, e in un raccolto\\
Le Trombe, e il non plus ultra d'un bel volto.
\end{ottave}

\begin{ottave}
\flagverse{38}Po' far! (dicea) che bella creatura !\\
Quell' Ostessa da vero havea ragione, \\
Perch'ella è bella fuor d'ogni misura,\\
Per me non saprei darle eccezione.\\
Capperi può ben dir d'haver ventura\\
Quello a cui tocca così buon boccone;\\
Ma s'ella s'ha da vincer con la lancia,\\
Hoggi è quando ci arrischio anch'io la pancia
\end{ottave}

Floriano giunto in piazza veduta Doralice così bella se ne invaghisce, e risolve
però di tentare la fortuna, e cimentare la sua persona per avventurare il conseguirla
per moglie.

\begin{description}
\item[Il Popol vi s'ammazza] V'è tanto popolo per veder quella giostra, che s'ammazzano
  l'un l'altro per la strettezza. Hiperbole usatissima in questo proposito
  per esprimere la gran calca, o quantità di popolo.
\item[FANNO la mostra] Quando i Cavalieri, o soldati, o altre genti, che devono
  fare qualche operazione guerriera (ancor che finta) avanti di cominciare a operare
  compariscono in ordinanza questo si dice far la mostra.
\item[LA Ragazza] Intende Doralice figliuola del Re.
\item[A SVO mo] Secondo il suo gusto. Quel \textit{mo} vuol dir modo, usandosi da noi,
come da i Latini, e da i Greci la figura Apocope, che leva l'ultime sillabe alle
parole, e da noi alle seguenti particolarmente; \textit{Modo}, \textit{meglio}, \textit{fede}, \textit{voglio}, \textit{vedi},
\textit{frate}, \textit{santo}, \textit{piede}, ec. Che diciamo: \textit{mo}, \textit{me'}, \textit{fè}, \textit{vo'}, \textit{vè}, \textit{fra}, \textit{san}, \textit{pié}. Ho voluto
notar queste, perché spesso nel nostro parlare ci vagliamo di questa figura, e
si troverà ancora spesso usata nella presente Opera, come habbiamo accennato
ancora sopra \cstan[1]{10}.
\item[TIRA frecciate come la rovella] Tira dardi, e frecce in quantità. Di questo
  termine \textit{come la rovella}, \textit{come la rabbia}, \textit{come il canchero}, ci serviamo per esprimere
  quantità grande, o  vero operazione violenta in superlativo grado; come per
esempio \textit{Il tale corre fortissimo}, \textit{il tale perquote gagliardamente} diremmo \textit{Il tale corre
come la rovella}, \textit{rabbia} o \textit{canchero}, o \textit{perquote come}, ec, E si deduce la comparazione
dalla violenza, con la quale opera il male della rabbia, o del canchero. La
voce \textit{rovella}, o rovello, credo inventata dalle donnicciuole per non profferire la
parola rabbia, come si dice \textit{cappita} in vece di \textit{canchero}, E se bene hanno del furbesco, son tuttavia, molto usate, e l'usò il Malatesti in alcune sue ottave.
\begin{verse}
 Da poi ch'io ho servito per zimbello,
 E sono andato trenta mesi aioni
 Gridando per la rabbia, e pel rovello
 Come fa il Gatto quand'ha i pedignoni ec,
\end{verse}
Ed habbiamo il verbo \textit{arrovellare}, e l'addiettivo \textit{arrovellato}. In somma in
questo luogo dicendo \textit{Tira frecciate come, la rovella} intende, che Doralice con le sue
 gran bellezze faceva innamorare ognuno, che la vedeva.

\item[LE Grazie] I Poeti fingono, che le grazie sieno tre figlie di Giove nominate
  Aglaia, Eufrosine, e Thalìa. \textit{Aglaos} in Greco val per splendido, Eufrosine, ilarità,
  allegrezza, e Thalìa, verdeggiante. sì che dicendo \textit{si scorge in quel volto le
    grazia} vien' a dire: Si conosce in lei splendidezza, allegrezza, e freschezza, cioè
  gioventù sana.
\item[RACCOLTO in uno] Unito in un solo luogo, Termine latino, usato alle volte
  anche da noi in questo proposito.
\item[LE Trombe] Nella più stimata carta de' Ganellini, o Minchiate è effigiata la
  Fama con due trombe alla bocca, e da questa tal carta si chiama le Trombe; E
  per esser questa la superiore a tutte l'altre carte quando si dice: \textit{La tal cosa è le
    trombe} s'intende, che questa tal cosa sia la meglio, che si trovi nel suo genere.
  Ed è detto assai usato per esprimere l'eccellenza d'una cosa, ed ha la forza del
  superlativo.
\item[NON plus ultra] È noto il motto delle colonne d'Hercole, che vuol dire:
  \textit{Non si vadia più avanti}, E noi ce ne serviamo nelle congiunture simili alla presente,
  che s'intende; non si può andar più là, cioè non si può avanzare, o superare
  tal bellezza, o vero non si può far più bella. Esprime anche questo termine
  un superlativo,

\item[PUO' fare] E' termine d'ammirazione,o stupore quasi diciamo: Può mai fare
  il Cielo, o la natura una cosa tanto bella, e perfetta come questa?
\item[CAPPERI?] Ancor questo è termine d'ammirazione; e si dice ancora \textit{cappita},
\textit{canchita}, \textit{canchigna} forse per non dir canchero: Voci inventate dalle donne, come
habbiamo accennato poco sopra alla voce \textit{rovella}. Consuona col latino \textit{Papae}, che
noi diciamo \textit{Pà!} e col latino \textit{babae}, che noi diciamo, \textit{o babbo !} E la parola \textit{capperi},
che tanto in Greco, che in Latino vuol dire il \textit{cappero} frutto noto, serviva anche
a' medesimi per termine d'ammirazione, o giuratorio, come si vede in Laerzio
nella vita di Zenone. \textit{Sed, \& per capparim iurabat, sicut Socrates per canem}, ec. Lo
stesso riferisce Alex.\ ab Alex.\ dier.\ gen.\ \libcap[5]{10}. Il Lalli nella sua En.\ trau.\
\cstan[1]{85}.
\begin{verse}
Capperi disse Enea, come sì tosto
Fatt' ha sì gran Città questa Signora!
\end{verse}
\item[A CHI tocca così buon boccone] Chi havrà così buona sorte. Chi havrà per moglie
  così bella, e ricca Giovane.
\item[CI arrischio anch'io la pancia] Ci avventuro anch'io la vita.
\end{description}

\section{Stanza XXXIX}
\begin{ottave}
\flagverse{39}O per tutt' hoggi beccomi su moglie \\
Nobile, ricca, e bella; o veramente \\
Vi lascio l'ossa; s'ella coglie, coglie \\
Se nò a patire: O Cesare, o niente.\\
Ciò detto salta in campo, e un'asta toglie,\\
Intruppandosi là dov'ei già sente,\\
C'appunto il Re sollecita, e commette,\\
Che pe' i primi si tirin le bruschette.
\end{ottave}

Risoluto Floriano di provarsi in questa giostra si fa innanzi, e piglia una lancia.
Qui bisogna supporre, che Floriano, e gli altri Cavalieri fussero armati di
dosso, come è necessario, che sieno i Cavalieri, che giostrano a corpo a corpo.

\begin{description}
\item[BECCOMI su moglie] Questo verbo beccare ha signiticato di rubare, guadagnare, o acquistare,
  Gio. della Casa nel Capitolo in lode del martello d'amore dice
\begin{verse}
So che sapete del ladro sottile,
C'a Giove fe la barba già di stoppa,
Quando gli beccò fu l'esca, e il fucile.
\end{verse}

E però usato per lo più scherzando in occasione di maritaggi, come appunto
nel presente luogo, E si dice \textit{Il tale pigliò moglie, e becca su una buona dote}. E lo
scherzo nasce dal verbo \textit{beccare}, che è noto quel che significhi trattandoli d'ammogliati.

\item[S'ELLA coglie, coglie] S'io m'appongo, sarà bene. S'io vincerò l'havrò indovinata,
  e sarò felice, \textit{Se no a patire}, Se non m'appongo, sarà disgrazia, havrò
  pazienza. In somma con questi due detti vuol mostrare, che Floriano ha l'animo
  accomodato a tutto quel che sia per succedere, o male, o bene che sia.

\item[O Cesare, o niente] \textit{Aut Caesar, aut Nihil}, O morire, o esser qualcosa di garbo.
  Questa sentenza latina si profferisce da noi corrottamente, O Ceseri, o Niccolò, ed esprime
  \textit{Aut Rex, aut asinus} de i Greci, cioè uno de due estremi.

\item[SI tirin le buschette] Si tirino le sorti. Credo che si chiamino bruschette, e non
buschette, o forse in ambedue i modi; che è un giuoco da Fanciulli, e si fa con
pigliare tante fila di paglia, o altra materia simile, quanti sono coloro, che hanno
a concorrere al premio proposto, e quel filo, che tira il premio, si fa o più
lungo, o più corto de gli altri; detti fili s'accomodano fra due assi, o in mano
in modo, che non si veda se non una delle due testate di essi, per le quali testate
ciascuno de' Ragazzi cava fuori il suo, e quello che tira il più lungo, o il più
corto, secondo che è destinato, conseguisce il premio proposto; Questo giuoco
serve ancora ai Ragazzi per fare le divisioni ne i loro giuochi Fanciulleschi, come
farebbe ne i Birri, e Ladri detto sopra in \cstan{32}. alla voce Bomba, che
allora pigliano tanti fili, quanti sono i Ragazzi, la metà lunghi, e la meta corti,
e cavandoli da loro a uno per volta detti fili; quelli, che hanno i lunghi, vanno
da una banda, e quelli de' corti dall'altra; e così serve a loro, come serve nel
presente luogo, per un modo di tirar le sorti. E da questi bruscoli, o fili di paglia
mi do a credere, che si dica \textit{bruschette}; e che \textit{buschette} sia quel giuoco, che si
con certi pezzetti di mazza rifessa, e che si tirano, come i dadi, con altro
nome dette \textit{le buffe}. Vedi sotto C.~11. stan. 42.
\end{description}

\section{Stanza XXXX \& XXXXI}
\begin{ottave}
\flagverse{40}Come volontaroso Floriano,\\
Senza chieder licenza, o cosa alcuna,\\
Si fece innanzi, e postavi la mano\\
Di trarne la più lunga hebbe fortuna,\\
Poco dopo il Marchese di Soffiano\\
Simile a quella anch'egli ne trasse una\\
Ond'essi, come pria fu destinato,\\
Furono i primi a correr lo steccato.
\end{ottave}

\begin{ottave}
\flagverse{41}Piglian del campo, e al cenno del trombetta\\
Si vanno incontro con la lancia in resta;\\
Il Marchese a Florian l'havea diretta;\\
Per chiapparlo nel mezzo della testa;\\
Ma quei, ch'e furbo, a un tempo fa civetta,\\
E aggiusta lui, dicendo: Assaggia questa,\\
Perché gli diede sì spietata botta\\
Ch'egli andò giù come una pera cotta.
\end{ottave}

Floriano prese una di dette Bruschette, ed una ne prese il Marchese di Soffiano;
e questi due furono i primi a correre la lancia, nel qual' incontro il Marchese rimase
abbattuto. \textit{Marchese di Soffiano}, È nome a caso, e fa Marchesato una contrada,
o villa vicina a Firenze detta Soffiano.
\begin{description}
\item[CHIAPPARE] Val per colpire.

\item[FURBO] Se ben la voce furbo deriva dal latino \textit{Fur}, che vuol dir Ladro, tuttavia
  ce ne serviamo per esprimere un'huomo scellerato, e che habbia ogni sorta di vizio,
  come s'è detto sopra in \cstan{2}. Ed ancora per denotare un'huomo
  astuto, e che sappia il conto suo, come segue nel presente luogo.

\item[FA CIVETTA] Abbassa la testa. Viene dal giuoco di civetta, che da i giovanotti
  si fa in questa maniera. S'accordano tre, ed uno di loro, al quale è toccato
  in sorte, si pone in mezzo a gli altri due, i quali s'ingegnano di cavargli il
  berrettino di testa con le percosse della mano; e quando egli tocca terra con le
  mani, non puo esser percosso; e però hora alzandosi, hora abbassandosi; tira
  guando all'uno, e quand'all'altro di gran mostaccioni; dura il giuoco fintanto
  che da uno delli due gli sia fatta cascare con un colpo la berretta dalla testa, che
  allora perde il premio proposto, e lo vince colui, che gliel'ha fatta cascare, il
  quale (seguitandosi il giuoco) va nel mezzo in luogo del primo. Tal giuoco si
  fa a tempo di suono, e piglia il nome dalla Civetta uccello, che per buscare il
  vitto scherza con gli uccelletti alzando, ed abbassando la testa, come appunto fa
  colui, che sta nel mezzo. E da questo poi \textit{far civetta} s'intende Abbassare il capo.
  Da Scops, che è un'uccello notturno del genere delle Civette. Era appresso i Greci
  una sorta di giuoco, o passatempo detto \textit{Scopias}, nel quale veniva contraffatto a
  tempo di ballo il muoversi in giro, e l'alzare, e l'abbassare della testa di quell'uccello;
  onde ne fu formato il verbo \textit{Scoptein} irridere, che appresso i Greci vale,
  quel che appresso noi Toscani, Uccellare. V. Giulio Polluce l. 4. cap. 14.

\item[AGGIUSTA lui] Aggiustar uno, s'intende Fargli il suo dovere, e trattare uno
  come egli merita, Lat. \textit{concinare}. Vuol dire ancora conciar male uno, come s'intende
  nel presente luogo, e sotto C.~11. stan. 50. E per altro vuol dire Saldare, o pagare
  un debito. Lat. \textit{pariare}.

\item[BOTTA] Colpo, o percossa. E questa voce \textit{botta} per altro vuol dire una specie
  di Rospo. Lat. \textit{rubeta}.

\item[ANDÒ giù com'una pera cotta] \makebox[1em]{} Cascò giù facilmente, ed a piombo, come fanno
  le pere cotte dal Sole, che cascano facilmente dall'albero; o forse come le
  cotte al fuoco, che son facilissime a andar giù in corpo quando si mangiano.
  Plauto disse: \textit{Tam crebri ad terram decidunt ut pyra}; da che si deduce che s'intenda
  delle pere, le quali cascano dall'albero,
\end{description}

\section{Stanza XXXXII.}
\begin{ottave}
\flagverse{42}In quanto a Sposa, homai questo è ascolto; \\
S'ei toccò terra, ancor la voglia sputi:\\
Così Florian dicea; ne stette molto\\
Ch'il secondo ne viene a spron battuti,\\
Che mette lui per morto, anzi sepolto,\\
Ma il giovane, che dà di quei saluti,\\
Gli mostra in avviarlo per le poste\\
L'error di chi fa i conti senza l'Oste.
\end{ottave}

Comparve il secondo Cavaliere il quale si dava a credere d'haver già morto
Floriano; ma questo col buttarlo a terra, gli fece conoscere quanto s'era ingannato.

\begin{description}
\item[È ASCOLTO] È licenziato. I ragazi, che vanno alle squole, quando sono
stati sentiti leggere dal Maestro si dicono \textit{ascolti}, e s'intendono licenziati: e così
questo Cavaliere essendo passato per le mani del Maestro, che è Floriano, si può
dire \textit{ascolto}, e licenziato dalla Sposa.
\item[TOCCAR terra, e sputar la voglia] Dicono le donne, che quando son pregne,
  venendo loro voglia di qualche cosa, se in quello stante si toccano con le proprie
  mani in alcuna parte del corpo, quivi nasca alla creatura un segno simile a quella
  tal cosa desiderata; e i segni poi chiamano voglie; e che per sfuggire che
  la creatura non nasca con tali segni, o voglie, il rimedio sia, che la Donna pregna,
  quando le viene tal desiderio, tocchi subito terra con la mano, e sputi dicendo
  \textit{A terra vadia}. E però il Poeta, seguitando questa opinione, dice, che se
  il Marchese ha toccato terra per liberarsi dalla voglia della Dama, è necessario
  ancora che egli sputi, a voler che il rimedio sia fatto compitamente, Tal detto
  \textit{sputar la voglia}, è assai vulgato per intender uno, che habbia gran desiderio d'una
  tal cosa, che sia a lui impossibile a conseguire. Vedi Plin. lib. 28.c. 4.

\item[A SPRON battuti] A tutta carriera; Velocemente. Fran. Sacc. Novella \textit{mihi}
31. \textit{E così salito a cavallo n'ando a spron battuti al Palazzo de' Signori}.

\item[LO mette per morto, anzi sepolto] Intende; che questo secondo Cavaliero non
solo credeva di havere a uccidere Floriano; ma gli pareva già d'haverlo ucciso.
Esprime la gran presunzione, che havea di sé stesso questo Cavaliero, e la poca
stima, che faceva di Floriano.

\item[DI quei saluti] Intende di quelle percosse.

\item[FAR il conto senza l'Oste] Stabilire per fatta una cosa, alla quale deve intervenire,
  e concorrere anche la volontà d'un'altro. Dove è l'interesse del compagno,
  si può metter in sicura la propria volontà, ma non quella del compagno.
\end{description}

\section{Stanza XXXXIII.}
\begin{ottave}
\flagverse{43}Comparso il terzo, in testa della lizza\\
S'affronta seco, e passalo fuor fuora;\\
Soggiunge il quarto ed egli te l'infizza\\
Sbudella il quinto, e fredda il sesto ancora\\
All'altro manda il settimo indirizza;\\
L'ottavo, e il nono appresso investe, e fora;\\
E così a tutti con suo vanto, e fama\\
Cavò di testa il ruzzo della Dama.
\end{ottave}

In questa ottava l'Autore narra la vittoria, che hebbe Floriano di sette Cavalieri,
e descrive la lor perdita in sette modi di dire diversi; il primo \textit{lo passa fuor
fuora}, il secondo \textit{l'infizza} (si dovrebbe dire infilza ma non solo perché gli è permessa
questa licenza per causa della rima, quanto anche perché per i più si dice infizza,
e non infilza, s'è fatto lecito dirlo anch'egli) il terzo \textit{lo passa fuor fuori}, il quarto
\textit{lo fredda}, il quinto \textit{l'indirizza all'altro mondo}; il sesto \textit{l'investe}, ed il settimo
\textit{lo fora}. E questi sette modi di dire havendo quasi tutti lo stesso significato d'ammazzare
danno l'occasione d'ammirar l'artifizio del Poeta in mostrate la fecondità
della nostra lingua Fiorentina.

\begin{description}
\item[LIZZA] Che si dice anche Nizza. Vuol dir linea; ma da noi s'intende quel
tavolato, o muro, rasente al quale corrono i Cavalieri le lance al Saracino.
\item[CAVÒ di testa il ruzzo della Dama] Fece uscir di testa il desiderio della dama.
  La voce ruzzo, che dal verbo ruzzare vuol dir Baie, usata in questi termini significa
  prurito, umore, desiderio, ec, sì che dicendosi. \textit{Il tale ha questo ruzzo in
  testa}, vuol dire il tale ha questa voglio, questo humore, ec. Il Laica nov. mihi
  8. dice. \textit{Deliberarono di dargli così fatta gastigatura, che gli uscisse per sempre
  l'humore, e il ruzzo di testa}.
\end{description}

\section{Stanza XXXXIV.}
\begin{ottave}
\flagverse{44}Il Re si rallegrò con Floriano;\\
Sceso di sedia poi con la Figliuola\\
Le fece allor' allor toccar la mano,\\
Come nel Bando havea dato parola;\\
Ond'ogni altro ne fu mandato sano;\\
Ed ei nelle dolcezze infino a gola\\
Bem pasciuto, servito, e ringraziato\\
Rimase quivi a goder il Papato.
\end{ottave}

Il Re fece toccar da Floriano la mano alla Figliuolo, e gliela diede per moglie,
licenziando ogni altro pretendente, e Floriano rimase quivi a godere
queste sue felicità.
\begin{description}
\item[TOCCAR la mano] È lo stesso in questo caso, che che diciamo \textit{impalmare},
o far \textit{l'impalmamento} dal toccamento, che si fa della palma della mano dagli sposi;
che è il primo atto che si faccia per lo stabilimento del contratto del matrimonio,
Vedi sotto \cstan[12]{50}. '
\item[MANDATO sano] Cioè licenziato, ed escluso. Il verbo: \textit{valeo}, che significa
  Star sano, e usato da i latini anche per licenziarsi: \textit{parentibus vale dixit}, ed il
  simile facciamo noi, come si vede nel presente luogo., che diciamo \textit{Mandar sani}
  in vece di licenziargli. Anzi il medesimo verbo \textit{valeo} è tal volta usato da noi per
  intendere Addio, cioè licenziarsi. Il Vai in una sua frottola (se ben pedantesca)
  lo mostra dicendo.
\begin{verse}
  Hore liete,
  Iam vatlete. valete.
  Iam valete amati serculi;
  E tu vale,
  O sodale,
  Che maneggi i miei liberculi.
\end{verse}
Il nostro Poeta sotto \cstan[6]{18}.
\begin{verse}
  Restò la donna, ed ei le disse vale.
  \end{verse}

\item[NELLE dolcezze infino a gola] Immerso nei piaceri, e ne i gusti, sotto C. 4.
stan. 42. dice \textit{esser ne guai a gola}.

\item[GODERE il Papato] Goder le felicità concedutegli dal Cielo.
\end{description}

\section{Stanza XXXXV. — XXXXVIII.}
\begin{ottave}
\flagverse{45}Tre dì suonaro a festa le campane,\\
Ed altrettanti si bandì il lavoro,\\
E il Suocero, che meglio era del pane,\\
Vn' huom discreto, ed un coppa d'oro,\\
Faceva con gli Sposi a scaldamane,\\
Tal'hora a Mona luna, e Guancial d'oro,\\
E fece a' Paggi recitare a mente\\
Rosana, e la Regina d'Oriente.
\end{ottave}

\begin{ottave}
\flagverse{46}L'andar il giorno in piazza ai Burattini,\\
Ed agli Zanni furon le lor gite;\\
Ogni sera facevansi festini\\
Di giuoco, e di ballar veglie bandite;\\
E chi non era in gambe, ne in quattrini\\
Da trinciarle, e da fare ite, e venite,\\
Dicea novele, o stavale a ascoltare,\\
Faceva al Mazzolino, o alle Comare.
\end{ottave}

\begin{ottave}
\flagverse{47}Altri più là vedevansi confondere\\
A quel giuoco chiamato gli Spropositi,\\
Che quei ch'esce di tema nel rispondere\\
Convien ch'il subito depositi,\\
Ad altri piace più Capanniscondere,\\
Hann' altri varij humor, varij propositi,\\
Perché ognuno a un mo non è composto,\\
Però chi la vuol lessa, e chi arrosto.
\end{ottave}

\begin{ottave}
\flagverse{48}Chi fa le Merenducce in sul bavaglio;\\
Chi con amico fa a Stacciabburatta\\
Chi all'Altalena, e chi a Beccalaglio;\\
Va quello a Predellucce, un s'acculatta;\\
Per tutti in somma sempre vi fu taglio\\
Di star lieto così in barba di gatta,\\
E tra Floriano, il Re, e la Figliuola\\
Mai fu che dir n' un' anno una parola.
\end{ottave}

In queste quattro ottave il Poeta narra le feste, ed allegrie, che si fecero in
Campi per lo sposalizio di Doralice con Floriano; le quali feste fa che non trascendano
il genio puerile per continovare a scrivere una novella per i Fanciulli.
\begin{description}
\item[ERA meglio che il pane] Era un' huomo buonissimo, un' huomo che si
accordava a ogni cosa, appunto come è il pane, che s'accorda, ed unisce con
tutte le vivande, almeno appresso a i Fiorentini. In questo proposito i Greci
dissero, \textit{Columba mitior}.
\item[VNA coppa d'oro] Uno al quale non sia da apporre alcun difetto, \textit{omni exceptione
  maior}. Credo che si dica \textit{coppa d'oro}, per intendere oro coppellato, o di coppella,
  cioè raffinato, che Coppella si dice quello strumento, col quale si riduce
  l'oro alla sua vera purità, e perfezione; e \textit{Coppa} vuol dir bicchiere, o altro
  vaso simile, donde poi \textit{Sottocoppa} quella tazza, sopr'alla quale si portano i bicchieri,
  dando da bere, e \textit{Coppiere} quel che porta da bere al Signore.
\item[SCALDAMANE] Quattro, o più s'accordano, e mette ciascuno ordinatamente
  le mani sopra quelle del compagno, e poi vanno cavando per ordine quella
  mano, che è in fondo, e mettendola di sopra all'altre mani, e con quello
  modo; e confricazione pretendono scaldarsele; e però tale operazione è detta
  Scaldamane; ed è giuoco Fanciullesco, che ha la sua pena per chi erra cavando la
  mano, quando non tocca a lui.
\item[MONA luna] S' accordano molti Fanciulli, e tirano le sorti a chi di loro habbia
  a domandar consiglio a Mona luna, e quello a cui tocca vien segregato dalla
  conversazione, e serrato in una stanza, acciò che non possa intendere chi sia,
  quello di loro, che, resti eletto in Mona luna, della qual Mona luna si fa l'elezione
  fra gli altri, che restano dopo che colui è serrato. Eletta che è Mona luna,
  si mettono tutti a sedere in fila, e chiamano colui, che è serrato, acciò che
  venga a domandar il consiglio a Mona luna, Questo tale se ne viene, e domanda
  il consiglio a uno di quet ragazzi, quale egli crede, che sia stato eletto in Mona
  luna, e se s'abbatte a trovarlo, ha vinto; se no, quel tale, a cui ha domandato
  il consiglio gli risponde; io non son Mona luna, ma sta più giù, o più su, secondo
  che veramente è posto quel tale, che è Mona luna; ed il domandante perde
  il premio proposto, ed è di nuovo riserrato nella stanza per tanto, che dai
  Fanciulli sia creata un'altra Mona luna, alla quale egli torna a domandar consiglio,
  e così seguita fin a che una volta s'apponga, ed allora vince; e quello che
  è Mona luna perde il premio, e vien riserrato nella stanza, diventando colui, che
  deve domandare, e quello che s'appose, s'intruppa fra gli altri ragazzi. Il domandante
  richiede fino a quattro volte il consiglio, e può perder quattro premi,
  e poi fimescola fra gli altri ragazzi, esente però da dover più esser domandante,
  se non nel caso, che fatto Mona luna, egli perdesse, e sempre ritorna a
  creare nuova Mona luna, e si deputa nuovo domandante, quando il primo s'apponga,
  o habbia domandato, quattro volte il consiglio, la qual funzione, come
  detto, non può esser forzato a fare, se non quattro volte: ed i premj si adunano,
  e si distribuiscono poi fra di loro ripartitamente, e dal rendergli poi a di
  chi sono, cavano un'altro passatempo, come diremo. Da questo giuoco viene
  il proverbio \textit{Più su sta Mona luna}, che significa Nella tal cosa è misterio più
  importante di quel che altri si pensa.

Nota che tanto questo giuoco, quanto ogni altro, che troveremo nella presente
Opera s'altera, e diversifica secondo li gusti, e convenzioni puerili;
e non mi riprendere se tu ne havessi nella tua puerizia fatti, o veduti fare
alcuni, o tutti diversamente da quello, che io gli descrivo.

\item[GVANCIAL d'oro] Questo pure è giuoco Fanciullesco, quale è fatto così:
  S'adunano più Fanciulli, ed uno si mette a sedere sopra a una seggiola, ed un'altro
  se li pone inginocchioni avanti, e posa il suo capo in grembo a quel che
  siede, il quale gli chiude gli occhi con le mani, acciò che non possa vedere chi
  sia colui, che lo percosse in una mano, che egli si tiene dietro sopr' alle reni, dovendolo
  egli indovinare; e calui che gli serra gli occhi, dopo che questo tale è
  stato percosso gli dice : Chi t'ha percosso? ed egli risponde: \textit{Ficoseccho}; e l'altro
  replica: \textit{Menamelo qua per un'orecchio}. Ed allora quello si rizza, e va a pigliar
  colui, che egli crede il percussore, e se s'appone, ha vinto, e pone il percussore
  in luogo suo, e  li fa dare il premio in mano a quello che siede, e se non s'appone
  perde il premio, quale consegna, al detto sedente, e ritorna al luogo di prima
  per continuare; fin tanto che s'appone, ed alla quarta volta si fa nuova elezzione,
  come sopra a Mona luna. Questo mi par di poter credere, che sia quel
  gioco, che i Greci chiamavano Collabismo riferito dal Buleng.\footnote{Jules-César Boulenger, Loudun 1558- Cahors 1628, storico e gesuita. } de lud. vet.\footnote{De Ludis Privatis ac Domesticis Veterum, Lyon 1627.} cap. 37.
  qual giuoco da quel \textit{Propheriza: quis te percussit?} detto per disprezzo da i Giudei a
  Giesù Cristo sig.\ nostro, si può argumentare, che fusse anco appresso a i Latini.
\item[ROSANA, e la Regina d'Oriente] Sono due Leggende, o Rappresentazioni notissime,
  per esser cantate giornalmente da ogni donnicciuola.
\item[BVRATTINI] Intende quei Figurini di legno, che son fatti muover da uno, che
  a tal effetto s'asconde in un castelletto di legna coperto di panno; e gli fa operare
  mettendo egli sopra alle punte delle dita, e ad un certo suo fischio gli fa parlare.
\item[ZANNI] Per Zanni, che s'intehde servo sciocco Lombardo, qui intende ogni
  sorta di Bagattellieri, che fanno il buffone per le piazze.
\item[FESTINI di giuoco, ec] Quando s'adunano in una casa più Dame, e Cavalieri
  per giuocare insieme, o per ballare nella prima parte della notte, dice fare un
  \textit{Festino}, o \textit{Veglia}. E se bene veglia strettamente presa, pare che significhi più trattenimento
  di ballo, che di giuoco, tuttavia la pigliamo per intendere ogni sorta
  di trattenimento, o di Giuoco, o di Ballo, o di qualsivoglia altra cosa, nella
  quale si spendano le prime hore della notte, dicendosi: \textit{Noi facemmo la veglia a
    studiare, a ballare, a cantare, ec}. Ma volendo pigliare queste due voci nel suo proprio
  significato; \textit{Festino}, S'intende adunanza di persone nobili, sia per ballare,
  o per giuocare in quelle hore della notte; e \textit{Veglia} s'intende d'ogni sorta di persone
  ordinarie; E si come s'avvilirebbe dicendo: \textit{Io fui alla veglia nel Palazzo
  del Principe} così pare, che si burlerebbe dicendo: \textit{Fui al festino in casa un Battilano},
  Quando si dice \textit{Festino pubblico}, o \textit{Veglia bandita} s'intende \textit{Festino}, o \textit{Veglia} a porta
  aperta, dove può andare ognuno. Vedi sotto: \cstan[]{51}. e Cant. 10. stan. 28.
\item[NON era in gambe; ne in quattrini] Non si sentiva gagliardo da ballare, e non
haveva monete da poter giuocare.
\item[DA trinciarle] Intende da far capriole, cioè saltare. Vedi sotto \cstan[7]{23}.
\item[DA fare ite, e venite] Cioè giuocare. Quando si giuoca, e perdendo si paga
  la posta volta per volta, o si risquote quando ella si vince, diciamo \textit{fare ite, e venite},
  e s'intende pagare il denaro subito perduta la posta; e riceverlo nello stesso
  modo vincendo; ed è il contrario del detto \textit{Fare a tu me gli hai}; che significa giuocare
  in su la fede, o a credenza.
\item[MAZZOLINO] Ancor questo è trattenimento da Fanciulli, e si fa in tal guisa.
Più ragazzi si adunano insieme, e si piglino il nome d'un fiore per ciascuno,
e di questi fiori un di loro, che è il Giardiniere compone un mazzo, e poi dice:
Questo mazzo non sta bene per causa della Viola; e colui, che ha preso il nome
della Viola deve risponder subito: Dalla Viola non viene, ma sì ben dal Giglio, o
altro fiore, che a lui verrà nella mente; e se non risponde subito, o vero se nomina
un fiore che non sia in quel mazzo, perde un premio, il quale si dà al
Giardiniere. E così vanno seguitando fino a che il Giardinere habbia in mano
tanti premj da potere alla fine del giuoco distribuirne almeno uno per ciascuno di
quei ragazzi, che sono nel giuoco; ed il Giardiniere è sottoposto anch'egli alla
perdita del premio, perché se un fiore darà la colpa a lui, e che egli non risponda
subito, e nomini un Fiore, che non sia nel mazzo; perde come gli altri, e il
suo premio va dato in mano a colui, che l'ha fatto errare; ma come in deposito,
perché alla fine del Giuoco va poi con gli altri distribuito dal Giardiniero,
il quale non lo può però dare a se medesimo; E questi premj si domandano \textit{pegni},
e di questi intende il Poeta dove dice: \textit{Convien ch' il pegno subito depositi}.

Finito il Giuoco il Giardiniere distribuisce ripartitamente e pegni pigliandone
ancora per se. Tali pegni poi sono da coloro, che gli hanno dal Giardiniere havuti,
restituiti a i proprj padroni, i quali, se li rivogliono, devon fare una cosa secondo
il gusto di colui, al quale e toccato in sorte il detto pegno; E questo dicono
\textit{far la penitenza}, la quale se egli non fa, il pegno resta in mano a colui, al quale
è toccato, e però questi pegni devono esser di qualche valore, acciò che i padroni
habbian caro di riavergli. Alle volte fanno questo giuoco i Giovanetti di maggiore
età, e riducono questi pegni a moneta, quale depositano ogni volta, che
perdono in mano a un depositario, e se ne servono per far merende, ec, tal giuoco
è poco dissimile a quello, che facevano i Greci detto Basilinda riferito da Giulio
Polluce tab.\ 9.\ C.\ 7.\ e dove noi diciamo Giardiniere essi dicevano Re, come
facevano anche i Latini, e ciò si deduce da Hor. Ep. pr. lib. pr.
\begin{verse}
  \makebox[3em]{\dotfill} At pueri ludentes, Rex eris, aiunt,
  Si recte facies, hic murus aheneus esto, ec.
  Roscia, dic fodes, melior lex, an puerorum
  Naema? quae Regnum recte facientibus offert.
\end{verse}

Se bene potrebbe dirsi, che Orazio non intenda di questo giuoco particolarmente,
perché in tutti li giuochi Fanciulleschi tanto i Greci, che i Latini chiamavano
Re colui, che vinceva, ed asino quello che perdeva; ma perché nel giuoco presente
era fatto Giardiniere (o diciamolo Re) quello che in altri giuochi era rimasto
superiore a tutti, però non m'anlontano da interpretare Orazio, ed applicare
questo suo luogo al presente proposito, nel quale, se il Re errava diventava
l'asino, e Re si faceva colui, che havea fatto errare, o tenendosi il conto di
chi di loro haveva meno errato, quello alla fine era il Re, e quello che più volte
haveva errato era l'Afino, o Re Mida. Vedi il Meursio \textit{de Ludis veterum}. Gli
Spartani similmente per Legge di Licurgo, secondo che riferisce Plutarco nella
vita del medesimo, ai Ragazzi di più di sett'anni, proponevano come Principe
il più savio tra loro, che soprantendesse a' loro giuochi, e Fanciulleschi esercizzj.
\item[ALLE comare] Questo giuoco è trattenimenco di Fanciullette, e lo fanno così:
Mettono una di loro in un letto con un bamboccio fatto di cenci, e fingendo che
questa habbia partorito, le fanno ricever le visite da altre Fanciullette con far
quelle cirimonie, ed accompagnature, che si costumano in occasione di vere parturienti.

Tal giuoco era usato ancora dalle Fanciullette Greche secondo Giulio Pol.lib.9.c.7;
ma in vece d'una Parturiente fingevano una Sposa; e lo dicevano \textit{Phittamelia}.
Qual giuoco fanno pure ancora le nostre Fanciulline, e lo chiamano \textit{far' alle Zie}.
Non ha questo giuoco delle Comare, o Zie altro fine, che di passare il giorno in
quelle loro cirimonie, e ricevimenti, ne i quali alle volte si consuma quello, che
le Fanciullette hanno havuto per merendare.

\item[GLI spropositi] E lo stesso in sustanza, che quello del mazzolino, se non che
  dove in quello si finge un Giardiniere; in questo i Ragazzi s'adattano a qualsivoglia
  altra cosa, con pigliarsi quei nomi, che attengono a quella tal cosa; per
  esempio: Faranno il giuoco sopra il pane; il Maestro sarà il Fornaio, e questo farà
  quello che nel Mazzolino fa il Giardiniere; uno farà la farina, uno l'acqua, uno
  il forno, ed altre cose attenenti alla construttura, e perfezione del pane; Il
  Fornaio dirà: Questo pane non è buono per causa della Farina; quello che ha
  il nome della Farina, deve risponder subito: Dalla farina non viene, ma dall'acqua,
  o da altra cosa che gli venga in mente, attenente al pane, e che sia fra
  loro Ragazzi; e se non risponde presto, o non da la colpa a qualche cosa, il nome
  della quale non sia in quella adunanza, o non sia attenente al pane, perde, e
  deposita il pegno; e si fa nel resto per appunto come nel giuoco del Mazzolino:
  E questo giuoco universale è forse quello, che habbiamo detto sopra, che facevano
  i Greci detto Basilinda, E da noi si chiama \textit{il giuoco de gli Spropositi},  perché
  dovendo quei Ragazzi risponder presto, attribuiscono al pane cose spropositatissime,
  e che non hanno che far punto col pane, o sua bontà, oltre a non esser
  il nome di quella tal cosa in veruno di quei Ragazzi. E quello vuol dire \textit{Uscir di
  tema}.

  Habbiamo un'altro modo di far questo giuoco, ed è così: Mettonsi più persone
  a sedere in giro, e ciascuno dice al compagno in uno orecchio una parola, o
  due al più, e finito il giro, ciascuno ordinatamente dice forte quella parola, che
  gli e stata detta dal vicino, e volendone comporre il periodo si sentono gli Spropositi,
  che risultano da quelle parole; e si da la pena a colui, che ne è stata la
  cagione.
\item[CAPO a niscondere] Vno si mette col capo in grembo a un'altro, che gli tura
  gli occhi, ed un'altro, o più si nascondono, e nascosti danno cenno, e colui che
  haveva gli occhi serrati si rizza, e va cercando di coloro, che sono nascosti, e
  trovandone uno basta per liberarsi da tornare in grembo a colui, dove mette
  quello, che ha trovato, e questo perde il premio proposto, e il trovatore va a
  nascondersi; ma se non trova il nascosto in tante gite, o in tanto tempo, quanto
  sono convenuti, perde il premio, e ritorna a star con gli occhi chiusi come prima;
  e seguita così fino a quattro volte, perdendo quattro premj, come s'è detto
  sopra a \textit{Mona luna}, ed i premj poi si distribuiscono come si fa al giuoco del
  Mazzolino, E quello star con gli occhi serrati si dice star sotto, che i Greci in un simil
  giuoco dicevano \textit{catamyein}, Lat. \textit{connivere}. E colui che è stato sotto quattro
  volte, e non ha mai trovato il nascosto, e per consegucnza perduti i quattro premj,
  occupa il luogo di colui, che teneva sotto, e questo s'intruppa con gli altri Ragazzi,
  fra i quali si tira la sorte a chi dee star sotto, o nascondersi. E così seguitano
  tanto, che si riducano tutti liberi; perché quello che ha pagati li quattro
  premj nel modo suddetto, ed ha occupato il luogo di tenere gli altri sotto, come
  ne vien cavato nella maniera accennata, resta fuori del giuoco, del quale solo
  attende la fine per conseguire anch'egli la sua parte de i premj da distribuirsi. Era
  ancor questo giuoco appresso a i Greci, e lo chiamavano \textit{Apodidrascinda} secondo
  Giulio Polluce lib. 9. c. 7., ma diversificava alquanto; Ed in questo giuoco
  pure il vincente era detto il Re, ed il maggior perdente l'Asino. Vedi il Buleng.
  de lud. Graec. cap. 22. ed il Meursio in verbo \textit{Apodidrascinda}. Simile a questo
  era ancora il giuoco detto da' Greci \textit{Myinda}.

\item[OGNVNO a un mo non è composto] In questo proverbio sentenzioso habbiamo
  ancor noi come i Latini più modi di dire, come: \textit{Le nature son diverse}. \textit{Tanti
    huomini tante berrette}, o \textit{tanti cervelli}, \textit{Tutte non possono esser a un modo}, \textit{Chi la
    vuole a lesso, e chi a rosto}, e molti altri; e ne i Latini si trova. \textit{Quot homines tot
    sententiae}, \textit{Suus cuique mos}, \textit{Trahit sua quemque voluptas}. \textit{Non omnes eadem mirantur,
    amantque}, ed altri infiniti, e tutti con lo stesso significato.
\item[FAR le merenducce] I nostri Stovigliai in alcune Fiere, che si fanno in Firenze
  il giorno della festività di San Simone, ed in quello di S. Martino conducono
  gran quantità di stoviglie piccolissime, come piatti, tegami, pentole, ed ogni altra
  specie di arnesi,  vasellami da cucina, che da essi si fabbricano di terra. Di
  queste si provveggono li nostri Fanciulli per quanto vien loro permesso dalla loro
  borsa, e da queste vien poi loro l'occasione di far le \textit{Merenducce}, perché havendo
  altre masserizie adeguate, come tavole, sgabelli, bicchieri, salviette,  simili,
  imbandiscono una mensa, accordandosi più Fanciulletti, e Fanciulline a portare quello,
  che è dato loro per merenda, ed accomodando tutto in piccole particelle,
  le distribuiscono in quei piattellini, figurando di fare un Banchetto, e mettono
  a sedere a quella tavolina li loro Bambocci; E queste son da loro chiamate
  \textit{Merenducce}, delle quali parla il Poeta, e le quali erano usate ancora dalle Fanciulline
  antiche in occasione del suddetto appellato \textit{Phitrameliae}, come si cava
  dal Meursio, dal Soutero, e dal Bulengero.
\item[BAVAGLIO] Salvietta, o Tovagliolino da Bambini, che si lega al collo con
due cordelline, o nastri, detto così dalla bava, che sopra vi casca dalla bocca de
bambini; i Latini pure secondo l'Onomastico lo dicono \textit{pectorale salivarium}, e con
questi \textit{Bavagli} come lor proprj arnesi apparecchiano le loro piccole tavole quando
fanno le \textit{Merenducce}, e si mangiano quelle particelle distribuite in quei piattellini,
come s'è detto sopra. E di queste \textit{Merenducce} parla il Poeta.
\item[STACCIABBVRATTA] Due seggono incontro l'uno all'altro, e si pigliano
  per le mani, e tirandosi innanzi, e indietro; come si fa dello staccio abburattando
  la farina, vanno cantando una lor frottola, che dice.
  \begin{verse}
    Staccia abburatta
    Martin della gatta
    La gatta andò pel vino, ec.
  \end{verse}

E questo è trastullo usato dalle Balie per acquietare i Bambini di quella età, che
 appena si reggono in piedi.
\item[ALTALENA] Passatempo da Fanciulli; Legano due funi al palco, o vero a
due alberi, e le fanno calare a doppio fino presso a terra un braccio, e sopra di
esse funi accomodano un'asse, sopr'alla quale si pone uno, o più a sedere, e fatto dare
il moto a detta asse vanno cantando alcune canzoni con un'aria aggiustata
al tempo dell'ondeggiamento di quell'asse, e questa l' Æora de' Greci, dai Latini
detta \textit{Oscillatio}, ed altre, volte \textit{Petaurum pensile}, e noi la diciamo \textit{Altalena} dal
Latino \textit{Tollenon}, che vuol dir quella Macchina di legno, con,la quale si cava l'acqua
de i pozzi (come si vede in Plin. lib. 19, c. 4. \textit{Vel Tollenonum haustu rigandos})
da noi detta \textit{Mazzacavallo}. Vedi sotto \cstan[6]{86}. E questo perché facevano
l'Altalena, come la fanno talvolta anche li nostri Fanciulli con incrocicchiare
una trave sopr'all'altra, e ponendosi uno o più ragazzi per testata della trave,
che è di sopra, la fanno alzare, e abbassare a foggia di \textit{Mazzacavallo}. Di questa
parla il Bulenger, de lud. vet. c.~11. Questa \textit{Altalena}, in alcuni luoghi di Toscana
è detta \textit{biciancole}.
\item[BECCALAGLIO] E' un giuoco simile alla mosca cieca detta sopra. C. 1. stan.
40. ne vi è altra differenza, che dove in quello si da, con un panno avvolto, o altra
cosa simile, in questo si da con la mano piacevolmente una sola volta da colui, che
bendò gli occhi a quel, che sta sotto, ed il bendato in vece di dare,  affanna di
pigliare un di coloro, che in quella stanza sono del giuoco, e colui che resterà
preso, deve bendarsi in luogo del bendato, e perde il pegno, e premio, ed il primo
bendato resta libero, e s'intruppa fra quelli, che hanno a esser presi, e si fa
come sopra nel giuoco di Guancial d'oro. Si dice \textit{Beccalaglio} perché questo tale
bendato vien condotto in mezzo della stanza, o piazza, dove s'ha da fare il giuoco;
e colui che lo bendò, e che quivi l'ha condotto gli dice; \textit{Che sei tu venuto a
  fare in piazza?} Ed egli risponde; \textit{A beccar l'aglio}, E quello dandogli leggiermente
con le mani sur' una spalla soggiugne: \textit{O beccati codesto}. Dopo la qual funzione
il bendato s'affatica, di pigliar uno per metterlo in suo luogo. I Greci appellavano
questo giuoco \textit{Chytrinda} da pentola che in Greco, si dice \textit{Chytra}, e lo facevano
nella stessa maniera; ma in vece di bendare gli occhi, mettevano a colui, o
fingevasi, ch'egli tenesse colla sinistra una pentola in capo, e girandogli intorno
lo solleticavano, o percotevano; onde, se egli rivoltandosi, prendeva chi gli
tirava, il preso rimaneva in cambio suo a essere quel della pentola. I Latini lo
dicevono \textit{ludus ollarius}.

Simile a questo era un'altro giuoco usato dalle Ragazze Greche, detto \textit{Chelichelona},
nel quale, messa a sedere quella, a cui davano nome di Chelona, che vuol
dire Testuggine, le dicevano: \textit{Chelichelona quid facis in medio?} e quella rispondeva:
\textit{Lanam texo, \& filum milesium} con quel che segue riferito dal Buleng. de
lud. vet. cap. 41.

Nel giuoco poi della \textit{Chytrinda}, ovvero, \textit{ludus ollarius} dicevano: \textit{Quis ollam?}
e chi teneva la pentola rispondeva: \textit{Ego Midas}, e s'affannava non di pigliare un
di coloro, ma di toccarlo co i piedi, e quel tale così tocco perdeva, e si metteva
la pentola in capo; E perché (come s'è detto sopra) i Greci havevano per costume
di chiamare Re il vincitore, ed asino il perditore, però questo tale, che
havea la pentola in capo si appellava \textit{Mida}, cioè \textit{Re asino}, Vedi Giulio Polluce
lib. 9. c. 7. ed il Buleng. de Lud. Vet, c, 17.

\item[ANDAR a predellucce] Due si pigliano per i polsi d'ambedue le mani l'uno
con l'altro in croce, e formano come una seggiola, e un'altro vi siede sopra, e
questo si dice \textit{andar' a predellucce}. Da i Greci s'usava un giuoco detto \textit{In Cotyla},
ed era il portare uno in su le spalle, e reggerlo, tenendo le di lui ginocchia nelle
palme delle mani voltate dietro alla persona, e detto \textit{In Cotyla}, cioè \textit{nella
ciotola}, o cavo della mano. Ma questo credo che sia un'altro giuoco, che noi
diciamo \textit{a cavalluccio}, che vedremo sotto \cstan[3]{30}. tanto più che i Greci secondo
lo stesso Polluce chiamano questo giuoco detto \textit{In Cotyla}, per altro nome
\textit{Hippada} dal verbo \textit{Hippazin}, cavalcare. E questo se bene è giuoco, tuttavia è
specie di pena per quei, che portano per haver perduto ad altri de' suddetti giuochi.

\item[ACCVLATTARE] È passatempo da Ragazzi, ma è specie di pena, e di tormento
  dovuto a colui che è acculattato. Quattro ragazzi pigliano uno per le
  braccia, e per i piedi, e formandone un quadrato, lo sollevano, e gli fanno battere
  il culo in terra tante volte, quanto merita il suo delitto, o perdita, che ha
  fatto in altri giuochi, come sopra. E questo si dice \textit{acculattare}, che in altro
  significato vedemmo sopra \cstan[1]{7}. Gli Spagnuoli chiamano l'Acculattare
  \textit{mantear}, perché mettono colui che si ha da acculateare in una coperta, o mantello, e
  tenendola da quattro capi, lo sbalzano in alto, e lo fanno ricadere in essa, e noi
  lo diciamo \textit{dar la coperta}.

\item[Vi fu caglio per tutti] Vi fu da dar soddisfazione a tutti. Ognuno hebbe in che
  impiegarsi. Traslato da' Sarti, che dicono in questa roba ci è taglio per un'Abito,
  o per due, ec. per intendere, ci e tanta roba, che si può fare un'Abito, o due, ec.

\item[STAR in barba di Gatta] o \textit{di Micio}, come si disse sopra in \cstan{28}.
  annotazione alla voce \textit{sbigottito}, Pare che questo detto possa venire dall'antica
  superstizione degli Egizzj, i quali credendosi, che il Gatto fusse consegrato alla
  Dea Iside, che era la loro Deità maggiore, non solo nutrivano con grandissima
  cura, e splendidezza questo animale, ma secondo Pierio Valeriano reputavano
  degno di morte colui, che ne ammazzasse, o facesse loro oltraggio. E riferisce
  Alex.\ ab Alex.\ dier.\ Gen.\ \libcap[3]{7}.\ e \libcap[6]{14}.\ che quando moriva un Gatto,
  i medesimi Egizzj per contrassegno di dolore si radevano le ciglia,e poi mettendo
  addosso al morto gatto sale, ed aromati, e coprendolo con un panno bianco lo
  seppellivano, facendoli talvolta sepolcri notabili, tanta era la stima che ne facevano.
\end{description}

\section{Stanza XXXXIX \& L}

\begin{ottave}
\flagverse{49}Mai fu tra lor fin qui nulla di guasto,\\
Se non che Florian volto ale cacce,\\
Havendone più volte tocco un tasto,\\
E sentendosi dar sempre cartacce,\\
Dispose al fin di non voler più pasto,\\
Ne curando lor preghi, ne minacce\\
Fece invitar da i soliti Bidelli\\
Per l'altro dì i Piacevoli, e i Piattelli.\\
\end{ottave}

\begin{ottave}
\flagverse{50}Bench'il Suocero allora, e la Consorte\\
Maledicesser questo suo motivo,\\
Dicendogli che la fuor delle porte\\
Un' Orco v'è sì perfido, e cattivo,\\
Che perseguita l'huomo infino a morte,\\
E che l'ingoierebbe vivo vivo;\\
Con genti, ed armi uscì su l'aurora\\
Gridando: Andianne, andianne, eccola fuora.
\end{ottave}

Non hebbero (come s'è detto) questi Sposi mai occasione d'addirarsi, se non
che Floriano inclinato alla caccia si risolvette andarvi a dispetto della Moglie,
e del Suocero.

\begin{description}
\item[NON fu nulla di guasto] Non furono tra loro mai rotture; cioè non s'adirarono
  mai; e, come si dice, non s'ingrossarono i sangui.
\item[HAVENDONE toccato un tasto] Havendo di ciò domandato alla sfuggita, o
  discorsone con brevità. Tratto da i tasti del Cimbalo, o vero Organo strumenti
  musicali.
\item[DAR cartacce] Non rispondere secondo il gusto di chi richiede; Traslato dal
  giuoco di minchiate, nel quale si dicono cartacce quelle che non contano, e sono
  di niun valore. Vedi sotto \cstan[8]{61}.
\item[DAR pasto] Trattenere uno con scuse, o chiacchiere. E il latino \textit{verba dare};
  \textit{spelactare}. E si dice così, perché il polmone degli animali (che da noi si dice pasto) stracca
  colui, che lo mangia, ma non lo sazia. Si dice anche dar pasto, quando uno, che
  fs giuocar bene a un tal giuoco, finge di saper poco, e si lascia vincer da principio,
  a fine d'indurre il semplice a far grosse poste per vincergli assai.
\item[BIDELLO] Donzello, o Servitore d'Università, o d'Accademia, come sarebbe
  quel Donzello, che serve allo Studio di Pisa, o ad altri simili. E questo
  nome di Bidello secondo l'Autore delle Notizie Ecclesiastiche è corrotto da \textit{Pedullus},
  perché questo Uffiziale, (dice egli) che nell'Accademie, e negli Studj
  pubblici haveva cura d'eseguire le commissioni appartenenti allo studio, soleva
  portare in mano un bastone chiamato \textit{Pedo}; Quantunque altri (soggiunge il medesimo)
  tirino la sua etimologia dalla parola Sassonica \textit{Bydell}, che vuol dire il
  Banditore.

  Ma io credo che il nome \textit{Bidello} sia tolto da \textit{Betulla}, che è quell'albero, del
  quale si facevano le verghe per i fasci, che anticamente portavano i Littori
  d'avanti a i Magistrati del popolo Romano, e che da questo portare i fasci di
  verghe di Betulla, sia poi venuto il nome di Bidello a tali serventi di Università,
  i quali fanno figura di Littori, e nello studio di Pisa portano ancora una grossa
  mazza d'argento (significante gli antichi fasci) quando vanno in funzioni pubbliche
  avanti al Collegio de i Dottori. Alex, ab Alex, dier. Gen, lib. 1. c. 17. in
  fine, dice così.
  \textit{Quodque fascibus, quos praeferebant Lictores, betullas virgas maxime commodas duxere,
    itaque ex illorum virgis tum proper candorem tum propter tenuitarem publicos
    fasces, qui magisiratibus praeirent, effecere}. E Plinio lib. 6. c. 18. \textit{Gaudet frigidis
    sorbus, \& magis Betulla; Gallica haec arbor, mirabilis candore atque tenuitate, terribilis
    Magistratuum virgis}. Lo stesso attesta Polid. Verg. lib. 4. c. 3.
\item[PIACEVOLI, e Piattelli] Sono in Firenze due conversazioni di cacciatori, le
  quali andando alle cacce gareggiano fra loro a chi faccia maggior preda, e quella
  che rimane superiore, tornando, suole entrare nella Città trionfante con fuochi,
  carri, ed altro; e l'una si dice la Compagnia de' \textit{Piacevoli}, e l'altra de' \textit{Piattelli};
  ciascuna ha la sua stanza entro alla quale s'adunano. gli Ufiziali, e Serenti,
  ed Altri; e questi son quelli de' quali dice il Poeta, e chiama i loro serventi
  Bidelli.
\item[VN'Orco] Questa è una bestia immaginaria inventata dalle Balie per far paura
  ai bambini, figurandola uno animale specie di Fata, nimico dei bambini cattivi,
  ed il Poeta, che non s'allontana mai dal genio puerile, mostra che il suocero
  Stordilano voleva indurre nel genero Floriano il timore per farlo astenere da
  andare a caccia, con dirgli che fuori della porta v'era l'Orco, che ingoiava gli
  huomini: Questo nome però viene dall'antica superstizione de i Gentili, i quali
  chiamavano Orco l'Inferno Virg. AEn. lib. 6. \textit{Primisque in faucibus orci}. Ed intendevano
  per Orco anche Plutone, quasi \textit{urgus, sive Uragus ab urgendo} perché egli
  sforza, e spinge tutti alla morte\footnote{Pluto sic dictus, non ab urgendo, ut quidam volunt, sed a Graeco $O\upsilon\rho\alpha\gamma o\varsigma$ dicitur, hoc est, qui in acie extremam agminis partem ducit. Unde non invenuste ad Ditem trausfertur, qui postremum humanae fabulae actum excipit.

Hofmann J. Lexicon universale. 1698. }; e perciò dalle madri, e nutrici per far paura
  alli lor bambini si dice che l'Orco porta via: il che pure vien da i Gentili, che
  pigliando Orco per la morte, lo chiamavano Inesorabile, e rapace. Orazio
  Ode 18. lib. 2,\begin{verse}
    Nulla certior tamen
    Rapacis Orci fine destinata.
  \end{verse}
\item[GRIDANDO andianne andianne ec] Così vanno gridando i cacciatori suddetti
  la mattina avanti giorno per svegliare i compagni. Lo stesso, che \textit{Alò Alò}; ovvero
  \textit{Alon} dal Franzese \textit{Allons}.
\end{description}

\section{stanza LI — LV}

\begin{ottave}
\flagverse{51}Senza veder ne anche un'animale\\
Frugò, bussò, girò più di tre miglia;\\
Pur vedde un tratto correr un Cignale\\
Feroce, grande, e grosso a meraviglia,\\
Ond'ei, che il dì dovea capitar male\\
Si mosse a seguitarlo a tutta briglia,\\
Non essendo informato ch'in quel Porco\\
Si trasformava quel ghiotton dell'Orco.
\end{ottave}

\begin{ottave}
\flagverse{52}Che a posta presa havea quella sembianza,\\
E gli passò fuggendo allor d'avanti\\
Per traviarlo solo con speranza\\
D'haver a far di lui più boccon santi;\\
Così guidollo fino alla sua stanza\\
Dov'ei pensò di porgli addosso i guanti;\\
Poi non gli parve tempo, perché i cani\\
Havrian più tosto lui mandato a brani.
\end{ottave}

\begin{ottave}
\flagverse{53}Però volendo andare in sul sicuro\\
Non a perdita più che manifesta,\\
Perché a roder toglieva un'osso duro\\
Mentre non lo chiappasse testa testa;\\
Gli sparì d'occhio, e fece un tempo scuro\\
Per incanto levar, vento, e tempesta,\\
E gragnuola sì grossa comparire,\\
Che havrebbe infranto non so che mi dire.
\end{ottave}

\begin{ottave}
\flagverse{54}Il cacciator, che quivi era in farsetto,\\
E dal sudore omai tutto una broda,\\
Havendo un vestituccio di dobretto,\\
Ed un cappel di brucioli alla moda,\\
Per non pigliare al vento un mal di petto,\\
O altro, perché il Prete non ne goda,\\
Non trovando altra casa in quel salvatico,\\
Che quella grotta, insaccavi da pratico.
\end{ottave}

\begin{ottave}
\flagverse{55}A tal gragnuola, a venti così fieri\\
C'ogni cosa mandavano in rovina,\\
Tal freddo fu che tutti quei quartieri\\
Se n'andanano in diaccio,  in gelatina,\\
Ed ei ch'era vestito di leggieri,\\
E mai meglio facea la furfantina,\\
Non più cercava capriolo, o damma,\\
Ma da far, s'ei poteva, un po di fiamma.
\end{ottave}

Floriano scorse molta campagna, e cercò buon pezzo, e non trovò mai nulla,
se non che pur vedde un grosso Cignale, al quale si messe dietro co i suoi cani,
non sapendo, che era l'Orco trasformatosi in quel cignale per pigliar Floriano;
dalla vista del quale sparì, e per via de' suoi incanti fece venire una gran
pioggia, e tempesta, la quale obbligò Floriano a ricovrarsi in una grotta, che era
quivi fra quelle macchie, nella quale entrato, si messe a cercare se trovava modo
di fare un po di fuoco.

\begin{description}
\item[FRVGÒ] Cioè cerco minutamente frugando per le siepi con i cani, e bussando
  con le pertiche per tutto.
\item[DOVEA capitar male] Doveva haver disgrazie. Doveva rovinare, E il Lat.
  \textit{Perdo, perire},
\item[A TUTTA briglia] A tutto corso senza punto fermarsi, come fa il cavallo
  quando se gli lascia liberamente la briglia. \textit{Laxatis habenis}.

\item[GHIOTTONE] Epiteto solito darsi a un huomo maligno, e di genio cattivo,
e suona quasi lo stesso, che Briccone, furbo, vizioso, scellerato.

\item[PIV boccon santi] Più buon bocconi. La voce santi in casi simili significa perfezione
  in generale. Vedi sotto \cstan[3]{8}.

\item[PORRE i guanti a dosso] Piglia guanti per mani, e vuol dire Pigliarlo. Habbiamo
  il verbo \textit{agguantare}, cioè pigliare. Guanto dal Germ. Hendt, mano.

\item[ANDARE in sul sicuro] Andar senza paura. Mettersi a fare un negozio con
  sicurezza di non esser'impedito, e che riesca secondo l'intento.

\item[TORRE a rodere un'osso duro] Pigliare a fare una cosa difficile.

\item[CHIAPPARE] Qui val per ritrovare, e sopra in \cstan{41}. per perquotere;
  ed il suo proprio significato è Pigliare; dal Lat, \textit{capere}.

\item[TESTA testa] Cioè a solo a solo. \textit{Remotis arbitris}, Diciamo anche a
  quattr'occhi.

\item[GRAGNVOLA] Grandine, che è gocciola d'acqua congelata nell'aria per
  forza di freddo, e di vento, e si fa di vapore freddo, e umido stropicciato nelle
  parti interiori del nugolo. \textit{La pioggia} nasce da vapori freddi, e umidi adunati
  ne i nugoli. \textit{La neve} è impressione generata di freddo, e d'umido; e questo freddo
  è minore di quello, col quale dalla pioggia vien generata la grandine, ed ha in
  se qualche parte di caldo. \textit{La rugiada} è generata di freddo, e di umido non rappreso,
  e questa congelandosi nell'aria diventa la \textit{brinata}. Ho voluto, benché fuor
  di proposito, notare l'origine de i sopraddetti accidenti dell'aria, perché da
  questa s'intendano i loro nomi; in qualche parte d'Italia per avventura differenti.

\item[HAVREBBE infranta non fo che mi dire] Havrebbe schiacciata, o diciamo anche
  ammaccata qualsivoglia cosa per dura che fusse; Non so immaginarmi, ne
  dire cosa tanto dura, che ella non l'havesse infranta. Questo termine \textit{non so che
    mi dire} usato nella forma, che si vede nel caso presente, significa quel che s'è detto;
  ma per altro l'usiamo anche per denotare di non havere, o saper trovar
  modo di rimediare a qualche accidente, per esempio: \textit{Io non so che mi dire, se il
    tale vuol far male i fatti suoi}.

\item[IN farsetto] Vestito leggiermente. Farsetto hoggi intendiamo ogni sorta d'abito
  leggieri, e disinvolto, che sopr'alla camicia si porta sotto gli altri abiti, come
  sarebbe camiciuola, o giubbone, ec.

\item[TVTTO una broda di sudore] Tutto molle dal sudore; Sudatissimo per la fatica
  del viaggio violento.

\item[DOBRETTO] Intendiamo una specie di tela di Francia fatta di lino, e bambagia
  (che è il cotone filato). Si dice anche \textit{Dobletto} da \textit{duplex}, perché nel tesserlo, è fatto
  di doppia orditura, e riempitura. Così \textit{dobbla} e \textit{dobbra} dissero gli antichi.

\item[BRUCIOLI] Quelle sottili strisce, che il Legnaiolo cava da qualsivoglia legno
  lavorandolo con la pialla, si dicono \textit{brucioli}, forse dalla similitudine de' brucioli,
  bachi; e da questi si dicono \textit{cappeli di brucioli} quelli, che son composti, ed intessuti
  di strisce d'un'erba particolare, nello stesso modo, che si fa con la paglia, alla
  similitudine, e larghezza della quale sono ridotte le dette strisce.

\item[ALLA moda] Cioè alla foggia che usa; la quale era nel tempo, che l'Autore
  compose la presente Opera, che i cappelli havevano piccola falda. sì che non
  tanto per esser di brucioli, quanto per esser piccolo, era poco atto a difendere
  dal acqua. Si dice \textit{alla moda} quasi \textit{all'usanza}, che \textit{è modo}, cioè adesso,Fr, alla moda.

\item[MAL di petto] Così chiamiamo volgarmente quell'infermità, che i Medici
  dicono Pleuritide.

\item[PERCHÉ il Prete non ne goda] Cioè per non morire, e così far che il Prete
  non goda il guadagno della cera del funerale.

\item[QVEI quartieri] Intendi per quelle campagne, per quei contorni. Che per altro
  noi Fiorentini per \textit{quartiere} intendiamo una delle quattro parti, nelle quali è
  divisa la nostra Città. E \textit{quartiere} in lingua militare significa Habitazione e dar
  \textit{quartiere al nimico} significa salvargli la vita, e farlo prigione.

\item[INSACCAVI da pratico] V'entra dentro come se egli, per esservi entrato altre
  volte, sapesse la strada, e vi fusse pratico. Se bene \textit{huomo pratico} usato nella
  maniera, che è qui, vuol dire huomo savio, e da saper pigliar compenso in ogni occasione.

\item[GELATINA] Vivanda nota fatta per lo più col brodo di carne di porco cotta
  in aceto, e poi congelato; Ma qui per \textit{Gelatina} intende che l'acqua s'andava
  congelando sopra il terreno, e fa \textit{Gelatina} sinonimo di \textit{Diaccio}, come fa D. inf. 32.

\item[FAR la Furfantina] Si trova una specie di Bianti, i quali per muover le persone
  pie a far loro elemosina, dopo haver bevuta buona quantità di generoso vino,
  ne i tempi più freddi si distendono mezzi ignudi nelle strade più frequentate, e
  tremando fingono di morirsi dal freddo, e questo lor tremare si dice \textit{far la Furfantina},
  cioè fare il giuoco che fanno questi furfanti, ch'è poi passato in dettato,
  che significa, e comunemente s'intende Tremare.

\item[MA meglio] Benissimo. Già mai si trovò chi facesse meglio. Quel \textit{ma} vuol
  dir mai; la figura apocope.

\item[DAMMA] È lo stesso, che Daino specie di capron salvatico. Lat. \textit{dama} D. Inf. 4.
\begin{verse}
  Sì sì starebbe un'cane infra due dame, ec.
\end{verse}
\end{description}

\section{Stanza LVI.}

\begin{ottave}
\flagverse{56}Trovò fucile, ed esca, e legni vari,\\
Ond'un buon fuoco in un cantone accese,\\
E in su due sassi posti per alari,\\
Sopr'un'altro sedendo i pié distese,\\
Così con tutti commodi a \culo{} pari,\\
Dopo una lieta, il crogiolo si prese,\\
Essendosi a far quivi accomodato,\\
Mentre pioveva, come quei da Prato.
\end{ottave}

Floriano havendo trovato in quella grotta comodità d'accendere il Fuoco,
l'accese, e vi s'accomodò a scaldarsi, aspettando che intanto cessasse la pioggia.

\begin{description}
\item[FVCILE] Intendiamo quello strumento d'acciaio, del quale ci serviamo per
  battere nella pietra focaia ad effetto di cavarne il fuoco; detto \textit{Fucile} da fuoco,
  quasi focaio, o focile. Che però dissesi anche \textit{Focile}.

\item[ESCA] Quel fungo, o sia cuoio corto conciato col salnitro, che facilmente
  piglia fuoco, e serve per tener sopra alla pietra quando in essa si batte per trarne
  il fuoco, da i Latini detta \textit{fomes}. La qual voce, se ben per translato significa
  incitamento, o stimolo, che noi pure diciamo fomite, nondimeno era intesa per ogni
  cosa facile a pigliare quel fuoco, che Vergilio appella \textit{Semina flammae abstrusa
    in venis silicis}. Sì come noi, ancora diciamo \textit{Esca} ogni sorte di cibo d'animali,
  pure dal latino \textit{Esca}, che vuol dir cibo, ed incendiamo ancora questa materia,
  che è atta a pigliare subito il fuoco, quasi sia il cibo del fuoco; anzi a questa
  non diamo altro nome, che \textit{d'esca}, e dicendosi \textit{Esca} assolutamente, e senza
  aggiunta, s'intende solamente questo cuoio cotto, o fungo conciati con salnitro.

\item[ALARI] Sono due Ferri, o Sassi, che si tengono nei focolare, perché mantengano
  sospese le legne, acciò che più facilmente ardano. È voce rimastaci dal
  Latino \textit{lares}, la qual voce spesse volte era presa per fuoco, come si può dedurre
  da Ovid. 1. fast. 18.
\begin{verse}
  Omnis habet geminas hinc, atque hinc ianua frontes,
  E quibus haec Populum spectat, \& illa Larem.
\end{verse}

E da Colum. \libcap[11]{1}. de Villico. \textit{Consuescat rusticus circa larem Domini,
focumque familiarem semper epulari}. Il Sipontino dice così: \textit{Lares Dij erant apud
Gentiles, \& colebantur domi, focusque illis sacer erat, unde vulgus focum focolare appellat
quasi laris focum}. Molti in vece di dire \textit{alari} dicon \textit{arali}, o sia corrottamente,
o pure, perché gli piglino da \textit{Ara}, intendendo strumenti da mettere in
su l'altare per sostenere le legne per il fuoco de i sacrifizzj, e così fanno che sia
ben detto tanto \textit{arali}, che \textit{alari}.

\item[A C. pari] Agiatamente si dice anche \textit{A pié pari}. Vedi sopra Cant. pr. stan.
82. Lasca Novella 4. lib. 2, \textit{Serviti delle buone vivande, che voi sapere bene acconce,
e stagionate se ne stessero a pié pari}. Si dice anche \textit{a gambe larghe}. Vedi. sotto C. 9.
stan. 32. Ed in molti altri modi, che tutti mostrano la spensierata agiatezza d'uno.

\item[DOP' una lieta] Dopo una fiamma.  Diciamo \textit{lieta} una fiamma chiara, senza
  fumo, e che presto passa detta \textit{lieta} da \textit{laetitia}, come anche \textit{baldoria}, da \textit{baldore} (cioè
  baldanza) voce antica. Gli Spagnuoli similmente dicono \textit{alegron}, un fuoco d'allegria.
  Vedi sopra \cstan[1]{4}. O forse si dice \textit{lieta} dalla parola \textit{lietamente}, che
  appresso ai nostri Contadini vuol dire \textit{prestamense}, cioè cosa, che passa prestamente.

\item[PIGLIARE il Crogiolo] \textit{Stagionarsi}. Quando son formati i bichieri, ed altri
  vasi di vetro, gli mettono così caldi in un fornelletto, che a tal fine è sopr'alla
  Fornace, da i Vetrai chiamato Camera, dove è un caldo moderato, e quivi gli
  lasciano stagionare, e freddare a poco a poco, conducendoli con un ferro alla
  bocca del detto Fornello per da basso, dove non si sente più caldo, il che da essi si
  dice \textit{dar la tempra}, \textit{temperare}, o \textit{dar il Crogiolo}, o \textit{Crogiolare}. E di qui parlando
  dell'huomo intendiamo \textit{pigliare il Crogiolo}, quando dopo una fiamma egli continova
  a stare attorno al fuoco, fino che sia tutto incenerito. E da questo verbo
  \textit{Crogiolare} piglia, o ha l'origine, il \textit{Crogiuolo}, che è quel vasetto di terra
  cotta, il quale serve per mettervi dentro a liquefare, o fondere i metalli nella Fornace,
  detto corrottamente \textit{Coreggiuolo}.

\item[FAR come quei da Prato] Proverbio vulgatissimo, che significa Lasciar piovere.
I Popoli della Città di Prato, che è suddita, e vicina a dieci miglia a Firenze,
nel tempo, che i Fiorentini si reggevano a Repubblica, domandarono licenza di
poter fare una Fiera il dì 8 di Settembre, (la qual Fiera si continova fino al presente
in detto giorno) e per tal' effetto. mandarono Ambasciadori alli SS.\ Priori
di libertà, da i quali fa loro conceduta la domandata licenza, con questo che
pagassero una certa somma di denaro. Accordato il negozio gli Ambasciadori si
partirono; Ma essendo nell'uscir del Palazzo, sovvenne loro, che se in tal giorno
fusse piovuto, non havrebbono potuto far la Fiera, e nondimeno sarebbe loro
convenuto pagare il danaro accordato; onde per assicurar quello punto tornarono
indietro, ed entrati di nuovo da i SS.\ Priori, uno di essi ambasciadori senz'altre
parole disse: Signori, se e' piovesse? Al che uno de' Signori subito rispose:
Lasciate piovere. E di qui nacque questo proverbio \textit{Far come quei da Prato}, che
significa Lasciar piovere.
\end{description}

\section{Stanza LVII — LVIII.}

\begin{ottave}
\flagverse{57}L'Orco fra tanto con mille atti, e scorci,\\
Affacciatosi all'uscio, ch'era aperto,\\
Pregò Florian con quel grugnin da Porci\\
Tutto quanto di fango ricoperto,\\
Che (perch'ella veniva giù con gli orci)\\
Ricever o volesse un po al coperto,\\
Ritrovandosi fuora scalzo, e ignudo\\
A sì gran pioggia, e a tempo così crudo
\end{ottave}

\begin{ottave}
\flagverse{58}Hebbe il giovane allora un gran contento\\
D'haver di nuovo quel bestion veduto,\\
E facendogli addossa assegnamento,\\
Quasi in un pugno già l'havesse havuto,\\
Rispose: Volentieri; entrate drento,\\
Venite, che voi siare il ben vennto,\\
Che dopo il fuggir voi l'umido, e il gielo,\\
Fate a me, ch'ero sol, servizio a Cielo.
\end{ottave}

Mentre Fioriano stava a scaldarsi; l'Orco s'affacciò alla bocca della grotta
senz'haver mutata la figura di Cignale, e pregò Florian, che lo lasciasse entrare;
Ei gli risponde, che entri allegramente, e che ne riceve servizio, perché
essendo solo, ha cara un poca di Compagnia.

Non si maravigli il lettore, che un Cignale parli; e si ricordi, che è una Novella
per i Fanciullini, e che queste cose seguivano.
\begin{verse}
 Al tempo, che volavano i pennati,
 Tutte le cose sapevan parlare;
\end{verse}
Secondo, che dice quel che descrive la guerra di Carnovale con Madonna Quaresima.
Apul. As.1. 2. \textit{Parietes locuturos, boves, \& id genus pecora dictura praesagium.}

\begin{description}
\item[GRVGNO] S'intende la faccia del Porco, da \textit{grunnitus}, che è lo stridere del
Porco. \textit{Grugnino} è detto per vezzi, ma qui è ironico, e per derisione \textit{Guardate
bella faccettina}, o \textit{bel grugnino}, o \textit{bel grugno}, quando vogliamo intendere una
brutta faccia. E si dice \textit{haver il grugno}, dell'huomo quando è in collera, donde ingrugnare
per entrar in collera. Vedi sotto \cstan[8]{61}. e \textit{sgrugnoni} si dicono le
pugna date nen viso.

\item[ELLA vien giù con gli orci] Cioè piove gagliardamente, quasi dica: Ogni gocciola
  è di tanta acqua; quanta ne cade a dar la volta a un'Orcio, che ne sia pieno.
  Si dice anche \textit{Ella viene a bigonce}, a \textit{catinelle}, ec, tutte iperboli per denotare, che
  piova gagliardamente. Vedi sotto \cstan[10]{20}.

\item[FACENDOGLI addosso assegnamento] \makebox[1em]{} Disegnando quello, che voleva far di
  quasi fusse già in suo potere, e dominio, come esprime il Poeta medesimo dicendo:
  \textit{Quasi in un pugno già l'havesse havuto}.
\item[FAR servizio a Cielo] Far un servizio, o favore accettissimo, o grandissimo.
\end{description}
\section{Stanza LIX - LXIII.}

\begin{ottave}
\flagverse{59}Si eh! (soggiunse l'Orco) fate motto!\\
Voler ch'io entri dove son due cani!\\
Credi tu pur ch'io sia così merlotto!\\
Se non gli cansi ci verrò domani.\\
S'altro, dice il garzon, non c'è di rotto\\
Due picche te gli vo' legar lontani,\\
E preso allora il suo guinzaglio in mano\\
Legò in un canto Tebero, e Giordano.
\end{ottave}

\begin{ottave}
\flagverse{60}Poi disse: Hor via venite alla sicura.\\
Rispose l'Orco: Io non verrò ne anco,\\
Guarda la gamba! perch' io ho paura\\
Di quella striscia, ch'io ti veggo al fianco,\\
Allor Florian cavossi la cintura,\\
Ed impiattò la spada sott' un banco,\\
Disse l'Orco: (vedutala riporre)\\
Io ti ringrazierei; ma non occorre.
\end{ottave}

\begin{ottave}
\flagverse{61}E lasciata la forma di quel verro,\\
Presa l'antica,e mostruosa faccia,\\
Con due catene saltò là di ferro,\\
E lo legò pel colle, e per le braccia,\\
Dicendo: Cacciatar tu hai pres'erro,\\
Perché credendo di far preda in caccia,\\
All fin non hai fatt'altro ch'una vescia,\\
Ment'il tutto è seguito alla rovescia.
\end{ottave}

\begin{ottave}
\flagverse{62}Rimasto ci sei tu, come tu vedi\\
Senza bisogno haver di testimoni,\\
E perché con levrieri, e cani, e spiedi\\
Far me volevi in pezzi, ed in bocconi;\\
Così perch'ella vadia pe' suoi piedi\\
Farassi a te, ne leva più ne pani,\\
Acciò che, procurando l'altrui danno,\\
Per te ritrovi il male, ed il malanno.
\end{ottave}

\begin{ottave}
\flagverse{63}Ed io c'hebbi mai sempre un tale scopo\\
D'accarezzar ognun, benché nimico,\\
Come la Gatta, quando ha preso il topo,\\
Che, se ben' è tra lor quell' odio antico,\\
Scherzando con esso alquanto, e poco dopo\\
Te lo sgranocchia come un beccafico,\\
Così perché più a filo tu mi metta\\
Voglio far' io, e poi darti la stretta.
\end{ottave}

L'Orco alla cortese offerta risponde, che ha paura de' cani, e della spada; e
Floriano lega quelli in un canto, e ripon questa sotto un banco; Allora l'Orco
si scuopre, ed entrato nella caverna prese Floriano, ed incatenollo.
\begin{description}
\item[SÌ eh?] E un termine, del quale ci serviamo per dimostrare che habbiamo, conosciuto
  l'inganno, o cattivo trattamento, che alcuno ci habbia fatto, o habbia in animo
  di farci, quasi dica: \textit{Così eh vorresti ch'io facessi?} o vero \textit{Così mi
  tratti eh?}

\item[FATE motto] Proferito col primo \letter{o} stretto. Vuol dire ascoltate, sentite.
  Fate motto a me; ed usato nella forma che è nel presente luogo, ha forza d'esclamazione,
  e vale per un certo modo di domandar consiglio, quando ci detta una
  cosa, che sia impossibile a farsi, o a credersi, quasi chiamiamo altra gente, che ci
  consigli se questa tal cosa sia da farsi, o da credersi; e che senta lo sproposito che
  ci è stato detto. Dirò per esempio; \textit{Costui dice che ha trent'anni, e Sono più di cinquanta
  ch' ei nacque}; Fate motto! Cioè udite sproposito; O vero giudicate, se ciò può essere.

\item[SIA così merlotto] Cioè sia così semplice, così minchione, così privo di senno.

\item[CI verrò domani] Detto ironico, che significa Non ci verro mai. Questo \textit{Domani}
  è il Domani eterno di quell'Oste, che haveva scritto sopr'alla sua bottega
  \textit{Doman si dà a credenza, e hoggi no}. Che l'hoggi era sempre, e il Domani havea
  sempre a venire. Berni \textit{A rivederci alle Calende Greche}, preso da Svet. in Aug. c. 87.

\item[DUE picche] Detto indeterminato, se ben pare determinato, e significa molto
  lontani, e non per appunto la lunghezza di due picche ma forse assai più, e forse
  assai meno.

\item[GVINZAGLIO] È quella corda, o striscia di quoio, con che si tengono i levrieri
  a lassa; e da molti è preso per ogni sorte di legame, derivandolo dal verbo
  latino \textit{vincio}, come \textit{vincastro}, \textit{vinciglia}, ec. ma strettamente guinzaglio,
  s'intende solo la corda, o quoio, col qual si tiene il levriero alla lassa, sebene
  da qualcuno è inteso ancora per quel legame, col quale s'accoppiano insieme
  i bracchi, o altri cani da caccia, Lat. \textit{copula}.

\item[GVARDA la gamba!] Il Cielo me ne liberi, Il Cielo mi guardi, che io sia per
  far questo. In Firenze nella Corte della Mercanzia, che è il Tribunale dove si
  fanno l'esecuzioni Civili, sono alcuni Donzelli, i quali si chiamano Toccatori.
  Questi dopo che in una causa si son fatti tutti gli atti, e si vuol venire all'esecuzione
  personale, vanno ad avvisare il debitore, che se egli non pagherà in termine
  di ventiquattro hore sara condotto in carcere; e senza tale atto, che si dice
  Toccare, o fare il tocco, non si si può con Cittadini Fiorentini venire a detta esecuzione
  personale. Tali Toccatori anticamente per esser conosciuti portavano
  una calza d'un colore, ed una d'un'altro, onde nel passare che facevano fra le
  Botteghe, e per i luoghi più frequentati i ragazzi gridavano: \textit{Guarda la gamba};
  affin che chi era in grado d'esser toccato si potesse fuggire, e guardarsi, non potendo
  i Toccatori far tale azione ne i luoghi immuni; e si dice Toccare perché
  non serve, che costoro avvisino con la voce il detto debitore, ma devono formalmente
  toccarlo con la mano. E da questo è venuto il modo di dire.
  \textit{Guarda la gamba}; che significa mi guarderò, o fuggirò di far tal cosa. Il Lalli
  nell' En. trav. lib, pr. stan. 67. si serve di questo detto nel medesimo proposito.
  \begin{verse}
    Venere allor rispose; Honor Celeste
    Guarda la gamba! usurpare io non voglio.
  \end{verse}

\item[IMPIATTARE] Nascondere, e si dice di materiali; e non pare che
  suonerebbe bene il dire Impiattare la verità, la virtù, ec. Vedi sopra C, 1. stan.
  75. Il Poeta se ne serve sotto \cstan[19]{5}. parlando dell'Aurora; ma la considera
  come donna, e corporea, come si considera il Sole, la Luna, e le Stelle,
  delle quali si dice \textit{Impiattarsi}, o \textit{rimpiattarsi} dietro a i nugoli, o dietro le
  montagne. Petr. Canz. 9. \textit{E lei non stringi che s'appiatta, e fugge}.

\item[BANCO] Vuol dir la Tavola, sopra alla quale si posano le vivande per mangiare:
  se bene \textit{Banco} ha molti altri significati.

\item[IO ti ringrazierei, ma non occorre] Cirimonia che si usa con chi ci habbia fatto
  un favore a rovescio, o vero ce l'habbia fatto quando non occorreva, o quando
  havevamo gia fatto da per noi quel che speravamo da lui; o che di sua cortesia ci
  faccia un favore del quale non havevamo bisogno; ed è lo stesso che dire \textit{Io t'ho
  negli orecchi}, \textit{Io t'ho stoppato}, e simili.

\item[VERRO] Porco maschio senza castrare. Dal Latino \textit{verres}.

\item[TV hai preso erro] Tu hai fatto errore. È detto hoggi poco usato fuor che nel contado.

\item[FARE una veglia] Non conchiudere. Non adempire il suo intento, come
fanno coloro, che andando a tirare con l'archibuso mettono nella canna minor
quantità di polvere di quella richiesta, e scaricando poi non colgono, e fanno
uno scoppio così debole, che a pena si sente, e tale scoppio di dice \textit{vescia}. Si dice
ancora \textit{vescia} una specie di fungo; E vescia dicono le donne un racconto de fatti
d'altri donde \textit{vesciona}, e \textit{vesciaia} una donna, che ridice tutto quello che sente
discorrere.

\item[NE leva più, ne poni] Non aggiungere, e non levare. Cioè sarai trattato
  ugualmente, o per appunto come volevi trattar me \textit{Nec addas, ned adimas}. E
  Dante Parad. C. 30.
  \begin{verse}
    Presso, e lontano lì ne pon, ne leva.
  \end{verse}

\item[IL male, ed il malanno] Il male, e peggio ch' il male.

\item[SGRANOCCHIA] Mangia con l'ossa, e con ogni cosa; ed il Poeta medesimo
  lo dichiara, dicendo: come un beccafico, i quali uccelletti da i più si mangiano
  senza buttar via l'ossa. E \textit{sgranocchiare} se ben s'usa alle volte ne i casi come il
  presente, non lo trovo usato se non per esprimere il romore, che fa coi denti in
  romper quell'ossa colui che le mangia, il qual romore è simile a quello che fa il
  ranocchio quando canta.

\item[HEBBI un certo scopo] Hebbi un certo fine, un certo genio, un certo riguardo.
  La voce \textit{scopo} vien dal Greco \textit{scopos}, che tanto appresso a Greci quanto ai
  Latini, ed appresso a noi vuol dir Berzaglio, e per metafora significa quel fine,
  al quale tende, ed è diretta la nostra mente nelle nostre operazioni, per lo più
  in bene; che non stimerei si potesse dire senza riprensione. \textit{Scopo di rubare}. Si
  dice anche \textit{haver mira}, il qual termine è per avventura più generico, dicendosi
  \textit{haver mira di far bene}, ed \textit{haver mira di far male}.

\item[METTERE a filo] Far venir gran voglia, Traslato dal coltello, ed altri ferri
  taglienti, i quali quando sono ben' arruotati (che si dice \textit{messi in filo}, o \textit{affilati})
  tagliano meglio.

\item[DAR la Stretta] Vuol dire opprimere uno. Ma qui è preso nel suo vero significato
  di stringere, ed intende stringere co i denti, cioè mangiare.
\end{description}

\section{Stanza LXIV.}

\begin{ottave}
\flagverse{64}Così spogliollo tutto ignudo nato,\\
E veduto ch'egli era una segrenna,\\
Idest asciutto, e ben condizionato,\\
Snello, lesto, e leggier com' una penna,\\
Lo racchiuse, e lo tenne soggiornato,\\
Perch' ei facesse un po miglior cotenna,\\
Però che a guisa poi di mettiloro\\
Voleva dar di Zanna al suo lavoro.
\end{ottave}

L'Orco spogliò Floriano per mangiarselo, e vedutolo così magro risolvé di
non toccarlo, ma lasciarlo stare tanto che ingrassasse, e poi mangiarselo.

\begin{description}
\item[IGNVDO nato] Cioè ignudo, come quando ei nacque. Diciamo così per intender
  uno, che non habbia in dosso ne pure una minima parte di vestimento, ed
  ha la stessa forza che dire \textit{Ignudo ignudo}, che per la ragione della replica, vuol
  dire Ignudissimo, o Affatto ignudo.

\item[SEGRENNA] Quella voce, usata per lo più dalle donnicciuole, vale per
  esprimere una persona magra, sparuta, e di non buon colore, che i Latini, tolto
  dai Greco, dicono \textit{Monogrammus}; ed il Poeta medesimo la dichiara dicendo:
  \textit{Idest asciutto}, che \textit{huomo asciutto} intendiamo huomo magro; ond'io mi credo che
  \textit{segrenna} venga da \textit{segaligno} che vuol dire Animale magro e di temperamento non
  atto a ingrassare. Diciamo ancora \textit{mummia}, che sono quei Cadaveri secchi nel
  mare d'Etiopia, o ne i sepolcri dell'Egitto: come vedremo sotto \cstan[6]{52}.
  per intendere Huomo soverchiamente magro. Diciamo \textit{Segrenna} a una donna
  magra, dispettosa, maligna, incontentabile, e che non approva, ne loda: mai
  l'operazione d'altrui.

\item[BEN condizionato] Questo termine, se ben pare riempitura del verso, o (come
  diciamo) borra, non è così, ma è pure che quando si vuole intender un magro,
  habbiamo questo dettato vulgatissimo \textit{Asciutto, e ben condizionato}, tolto forse
  da quello che son soliti dire i mercanti, \textit{la tal mercanzia ci è comparsa asciutta,
    e ben condizionata}, per avvisare il Corrispondente della diligenza del Latore, o
  Condotttiero.

\item[SNELLO, lesto, leggier come una penna] Queste tre voci nel presente luego Sono sinonimi
  significando, ed esprimendo tutte la poca carne che haveva addosso Floriano,
  e che era al maggior segno magro. E la voce \textit{snello} forse origine dal
  Tedesco \textit{Sknel}, che vuol dir Veloce.

\item[LO tenne soggiornato] Lo trattava bene di mangiare. Gli faceva buone spese.
  Che \textit{soggiornare uno} vuol dire Spender il tempo in ben custodire, governare, e
  ristorare uno con quello che occorra, e s'usa questo termine per lo più, trattandosi
  di bestiami, e perciò appropriatamente detto in questo luogo, perché, se
  ben Floriano era huomo, era nondimeno trattato dall'Orco come beitia da ingrassare.

\item[FACESSE miglior cotenna] Ingrassasse. Per intendere uno assai grasso diciamo:
  \textit{Egli ha buona cotenna}; traslato da i porci, la pelle de i quali si dice propriamente
  \textit{cotenna}, che dell'huomo si dice \textit{cotenna} solamente la pelle del capo, o per disprezzo,
  e per intendere un' huomo Zotico, che si dice \textit{huomo di grossa cotenna},
  o \textit{Cotennone}, o \textit{Coticone},

\item[A GVISA di mettiloro, Volea dar di zanna al suo lavoro] Coloro che indorano i
  legnami si chiamano \textit{Metti l'oro}, ed in una parola sola \textit{Mettilori}. Questi per
  brunire, o dar il lustro a i loro lavori si servono de i denti più lunghi, o diciamo
  maestre di cane, di lupo, o d'altro animale simile, (i quali denti chiamiamo \textit{zanne},
  o \textit{sanne} come vedremo sotto \cstan[7]{54}.) e tal lavorare dicono \textit{zannare}, o
  \textit{dar di zanna}. Ma qui \textit{dar di zanna} s'intende il naturale adoperar de i denti, che è
  mangiare; e scherzando con l'equivoco dice che l'Orco voleva \textit{dar di zanna al
    suo lavoro}, cioè mangiarsi Floriano, che era il suo lavoro, che egli havea fatto
  pigliandolo, ed ingrassandolo.
\end{description}

\section{Stanza LXV. \& LXVI.}

\begin{ottave}
\flagverse{65}Amadigi c' andava per diporto\\
Due volte il giorno almeno a rivedere\\
La fonte, e la mortella, che nell'orto\\
Lasciò Florian per tante sue preghiere;\\
Trovato il cesto spelacchiato, e smorto,\\
E l'acque basse puzzolenti, e nere,\\
Qui (dice) Fratel mio noi siam sul curro\\
D'andar a far un ballo in campo azzurro.
\end{ottave}

\begin{ottave}
\flagverse{66}E piangendo diceva; O Tato mio,\\
Se tu muori, che ver sarà pur troppo,\\
S'ha a dire anche di me, telo dich'io,\\
Itibus, come disse P\ellipsis{18pt} Pioppo,\\
Così, senza dir pure al Padre addio,\\
Monta sour' un cavallo, e di galoppo\\
Vscì d' Ugnano molto ben' armato,\\
E seco un cane alano havea fatato.
\end{ottave}

In questo tempo Amadigi s'accorse dalla fonte, e dalla mortella, che Floriano
era in pericolo, e perciò montato a cavallo bene armato, e con un grosso
cane incantato, andò a cercar di lui.
\begin{description}
\item[SPELACCHIATO] Pelato in qua, e in la, cioè parte delle foglie cascate, e
  parte no. Spelacchiato s'intende un'huomo, che stia male a sanità, ed a roba, e
  sia mal vestito per la sua povertà.
\item[SMORTO] S'intende che non ha il suo natural colore buono.
\item[SIAM sul curro] Siamo in procinto; siamo all' ordine; siamo vicini, \textit{Curro}
  son pezzi di quali si metton sotto alle pietre, o ad altre cose gravi per
  facilitargli il moto quando si strascicano, dai Latini detti \textit{Palangae}.
\item[FAR un ballo in campo azzurro] Vuol dire Esser' impiceato; perché \textit{campo
  azzurro} s'intende il campo, che fa l'aria, il quale è azzurro, e colui, che è impiccato
  movendo le gambe, pare che balli in aria, Per maggiore intelligenza la
  voce \textit{campo} pittorescamente parlando, vuol dire quel luogo, che avanza in un
  quadro fuori delle figure, ed altro che vi sia dipinto, come si dice una insegna
  entrovi un lione in campo azzurro. Ed i medesimi Pittori ne cavano il verbo
  \textit{campire}, ché vuol dire Dare il colore, del quale ha da essere il campo.
\item[TATO] Vuol di Fratello. È parola usata dalle Balie per insegnar parlare a i
  Bambini, come Babbo in vece di Padre, Mamma, Bombo, e simili, che per esser
  parole labiali tornano più facili a proferirsi. Furono usate anche dai Latini
  come si vede in Marz. lib, 1. 95.
  \begin{verse}
    Mammas, atque tatas habet Aphra, sed ipsa tatarum
    Dici, \& mammarum maxima mamma potest.
  \end{verse}

Vedi sotto \cstan[3]{13}., e \cstan[4]{5}.

\item[TE lo dich'io] Vale per Te lo giuro; Ti assicuro. Vedi Oraz.\ lib.\ 2.\ Ode 17.
dove parlando con Mecenate infermo, dice:
\begin{verse}
  Ab te meae si partem animae rapit
  Maturior vis, quid moror altera?
\end{verse}

Con quel che segue simile al presente lamento, che fa Amadigi per il Fratello,
che Orazio fa per Mecenate.

\item[ITIBUS come disse P\ellipsis{18pt} Pioppo]\footnote{``Prete Pioppo''} Significa s'ha dire anche di me: gli è morto.
  Questo P\ellipsis{18pt} Pioppo era uno, che havea poca amicizia con Prisciano\footnote{Priscianus Caesariensis, Cesarea 512 - dopo il 527, Grammatico, linguista.}, e
  non ostante sempre slatinava, e fra l'altre quando voleva dire il tale è morto diceva
  Itibus, e intendeva Egli è ito. E da questo suo detto diciamo \textit{Come disse
    P\ellipsis{18pt} Pioppo}, E s'intende il tale è morto.

\item[DIR' addio] Intendiamo quel saluto, che si fa nel pigliar congedo, o licenziarsi
  da uno, ed è lo stesso, che il Latino \textit{Vale}, usato da noi ancora come dicemmo
  sopra, e vedremo sotto \cstan[]{18}.

\item[GALOPPO] Corso di cavallo, da i Latini detto \textit{cursus gradarius}, che è in
  mezzo tra il trottare, e il correre. Forse meglio \textit{gualoppo} secondo Dante Inf.
  Cant. 22.
  \begin{verse}
    \makebox[10em]{\dotfill} di rintoppo
    A gli altri disse a lui, se tu ti cali
    Io non ti verrò dietro di gualoppo.
  \end{verse}
\item[CANE Allano] Cane grosso per caccia da Cignali, e simili animali feroci, ed è
  maggiore, più fiero, e più gagliardo del Mastino.
\end{description}

\section{Stanza LXVII \& LXVIII.}

\begin{ottave}
\flagverse{67}E cavalcando con la guida, e scorta\\
Del suo fedele, ed incantato Alano,\\
Ch'innanzi gli facea per la più corta\\
La strada per lo monte, e per lo piano;\\
A Campi giunse, dove in su la porta\\
la morte si  leggea di Floriano,\\
Che perché fu creduta da ognuno,\\
Era la Corte, e tutto Campi a bruno.
\end{ottave}

\begin{ottave}
\flagverse{68}L'apparir d'Amadigi agli abitanti\\
Raddolcì l'agro de i lor mesti visi,\\
Che per la somiglianza a tutti quanti\\
Parve il lor Re creduto a' Campi Elisi,\\
Perciò per buscar mance, e paraguanti\\
Andaron molti a darne al Re gli avvis,\\
Altri alla figlia, ed ambi a questi tali\\
Perciò promesser mille bei regali.
\end{ottave}

Amadigi arrivò a Campi, dove dal bruno, che vedde addosso a gli abitatori
conobbe, che era morto il lor Principe; subito che costoro veddero Amadigi,
credettero ch'i fusse Floriano, e perciò molti corsero a darne avviso al Re, e
a Doralice.

\begin{description}
\item[ERA la Corte, e tutto Campi a bruno] Cioè i Cortigiani, e gli abitanti di Cam-
i erano velliti di nero in: segno di mestizia, per la morte del Re Floriano. Petr. Canz. 5.
\begin{verse}
  E vedrai nella morte de' Mariti
  Tutte vestite a brun le donne Perse
\end{verse}

Da alcuni si dice \textit{vestire a lutto}, o \textit{a scorruccio}. Ma credo che essi habbiano
accattate queste voci da i moderni Romani.

\item[AGRO dei lor mesti visi] Viso agro vuol dir Malinconico; e si dice \textit{agro} perché
  uno, che habbia havuto qualche disgusto; suol mostrarlo nella faccia con increspar
  la fronte, e fare altri gesti appunto come fa uno, che mangi cose aspre,
  acide, o agre. E però dice \textit{Raddolcì l'agro dei lor mesti visi}, che significa di
  melancolici, gli fece ritornare allegri.

\item[CREDUTO a i Campi Elisi] Creduto nell'altro mondo; creduto morto, che
i Campi Elisi dalla superstiziosa Gentilità erano creduti il Paradiso. Vedi sotto
\cstan[6]{32}. '

\item[PARAGYANTO] Mancia, o regalo. \textit{Paraguanto}, \textit{dono}, \textit{regale}, \textit{mancia}
  appresso di noi si possono dir sinonimi; E se bene molti vogliono che \textit{mancia},
  e \textit{paraguanto} si dica quello, che dal Superiore si da all'inferiore; e \textit{dono} e \textit{regalo} si
  dica quello, che dall'inferiore si da al superiore (che in questo caso non si direbbe mancia)
  o dall'uguale, all'uguale, nondimeno nel buon parlar familiare si piglia
  uno per l'altro, ne s'osserva tanta strettezza, ed il nostro Poeta pure si
  vede nel presente luogo, che non osserva questa distinzione come poco, o punto
  necessaria.
\end{description}
\section{Stanza LXIX.}

\begin{ottave}
\flagverse{69}Doralice brittande a tai novelle\\
A rinfronzirsi andossene allo specchio,\\
Si messe il grembinl bianco e le pianelle\\
Il vezzo al collo, e i ciondoli all'orecchio,\\
E non potendo più nella pelle\\
Saltò fuor di palazzo innanzi al vecchio,\\
Ed incontro correndo al suo cognato,\\
Ecco Florian (dicea) risuscitato.
\end{ottave}

Doralice sentita questa nuova si raffazzonò, e subito corse incontro al suo cognato
Amadigi, credendolo Floriano suo marito.

\begin{description}
\item[BRILLANDO] Giubbilando. \textit{Brillo} si dice uno che sia allegro per haver beuuto
  molto vino. Vedi sotto \cstan[6]{35}. ed è il primo grado di briaco dicendosi in
  augumento \textit{Brillo}, \textit{cotto}, \textit{briaco}, \textit{spolpato}, Molti vogliono, che questa voce \textit{brillare}
  venga da \textit{birillo} specie di gioia, e che brillare significhi scintillando tremolare,
  appunto come fa il \textit{birillo}, e come fanno coloro, che sono sommamente allegri,
  ©che habbiano soverchiamente bevuto.

\item[RINFRONZIRSI] Raffazzonarsi, abbellirsi, aggiustarsi la persona tolto dal
  Latino \textit{refrondescere}, che vuol dir quando gli alberi si vestono di nuove frondi,
  le quali nell'antico Fior. si dicevano fronze. Terenz. in Heaut.
  \begin{verse}
    \makebox[4em]{\dotfill} Et nosti mores mulierum;
    Dum moliuntur, \& comuntur, annus est.
  \end{verse}
  Cioè si rinfronziscono (dice l'espositore Landino) s'accomodano, ed acconciano
  la testa.

\item[CIONDOLI all'orecchio] Orecchini. Quelle gioie, che le donne portano pendenti
  all'orecchio, Latino \textit{Inaures}. Da noi chiamati pendenti, e per scherzo
ciondoli.

\item[VEZZO] Quell'ornamento di gioie, che le Donne portano al collo.

\item[PIANELLE] Specie di scarpa, che cuopre solamente la parte dinanzi del piede,
  da i Latini dette \textit{sandalia}, E con dette gioie adornandola, mostra il Poeta quale
  possa essere una Regina di Campi, che non eccede il lusso d'una pulita contadina
  de i Contorni di Firenze.

\item[NON può star nella pelle] Non può aspettare, perché l'allegrezza le ha cagionata
  una inquietudine tale, quale vogliono havere tutti coloro, che dovendo conseguir
  qualcosa di lor gusto, ogni hora d'indugio stimano mille. A questo
  si può applicare quell' \textit{In fermento totus est} de i Latini, che pare che esprima
  quella inquietudine, che suol cagionare l'ira; Lasca Novella 5. \textit{Sì che per la
    passione, e per la rabbia non poteva star nelle cuoia}.

\item[COGNATO] | Latini per cognazione intendevano ogni sorta di parentela. Ma
  noi per \textit{cognato} intendiamo un Fratello di nostra moglie, o un marito d'una
  sorella di nostra moglie, o un marito di nostra Sorella, e nello stesso modo respettivè
  il Fratello del marito, si dice cognato, come intende nel presente luogo.

\item[INNANZI al vecchio] Cioè prima che uscisse di casa il Re suo padre, intendendosi
  comunemente Padre quando in questi termini si dice il vecchio, ancor che
  talvolta il Padre sia giovane.
\end{description}

\section{Stanza LXX — LXXIV.}

\begin{ottave}
\flagverse{70}Noi vi facevam morto; o giudicate,\\
Se la carota c'era stata fitta!\\
Pur noi ci rallegriam, che voi tornate\\
A consolar la vostra gent'afflitta,\\
Domandar non occorre come state,\\
Perché v' havete buona soprascritta,\\
E siate grasso, e tondo com'un porco\\
Per le carezze fattevi dall'Orco.
\end{ottave}

\begin{ottave}
\flagverse{71}M'immagino così perch' io non v'ero:\\
Tu sai com' ell' andò, che fusti in caso,\\
So ben, che mi dirai, che non fu vero\\
Ma la bugia ti corre su pel naso,\\
Hor basta. Tu ritorni sano, e intero\\
(C'a pezzi tu dovevi esser rimaso)\\
Per la Dio grazia, e sua particolare,\\
Perché tel' ha voluta risparmiare.
\end{ottave}

\begin{ottave}
\flagverse{72}Dunque s'ei fa così gli è necessario,\\
Ch'ei non sia là quel furbo ch'un lo tiene,\\
Anzi tutto il revescio, ed il contrario\\
Mentr'egli tratta i forestier si bene.\\
Ed io, che già havea sul calendario,\\
Gli voglio in quanto a me tutto il mio bene,\\
Perch'ei non t'ingoiò; Se ben da un lato\\
Ti stava bene, havendolo cercato.
\end{ottave}

\begin{ottave}
\flagverse{73}Così nel mezzo a tutta la pancaccia,\\
Ch'è quivi corsa, e forma un giro tondo,\\
La sua caponeria gli butta in faccia,\\
E quel ch'ei ne cavò po poi in quel fondo\\
Già che (dicea) con l'andar' a caccia\\
Ai dispetto di tutto quanto il mondo\\
Cavasti, senza far alcun guadagno\\
Due occhi a te, per trarne uno al compagno.
\end{ottave}

\begin{ottave}
\flagverse{74}Mio padre te lo disse fuor de denti,\\
Ed io pur te lo dissi a buona cera\\
Non una volta, ma diciotto, o venti\\
Che l'Orco ti faria quatche billera;\\
Ma tu volesti fare a gli scredenti,\\
Perché te ne struggei come la cera,\\
E quasi un rischio tal fusse una lappola\\
Volesti andarvi, e desti nella trappola.
\end{ottave}

In queste cinque ottave mostra, che Doralice ingannata dalla somiglianza,
che haveva Amadigi con Floriano, gli fa un discorso di congratulazione mescolata
con rimproveri, col quale il Poeta esprime assai bene il costume delle nostre
Femmine in simili casi; tacendo che dal principio del discorso, che è la congratulazione,
lo tratti del Voi, e quando viene a' rimproveri lo tratti del Tu.
\begin{description}
\item[SE La carota c'era stata fitta] \makebox[1em]{} Ficcar carote vuol dire quand'uno inventando
  qualche novella, o trovato, lo racconta poi per non suo, acciò che più agevolmente
  gli sia creduto; sì che Doralice vuol dire; guardate s'ella c'era stata data
  a credere. Vedi sotto Can. 6. stan. 67. e 68. Mattio Franzesi nel Capitolo sopr'alla
  Corte dice:
  \begin{verse}
    Chiama piantar carote il popolaccio
    Quel che diciamo: Mostrar nero per bianco
    Per distrigarsi da qualunque impaccio
  \end{verse}

  E per tutto il medesimo Capitolo discorrendo sopra questo detto, mostra che
  habbiamo anche il verbo \textit{Carotare}, e \textit{Carotiere}, quello che ficca carote. Il Lalli
  En. Tr. lib. 2. stan. 2.
  \begin{verse}
    Egli che ben conobbe al primo tratto
    Ch'era in un campo da piantar carote.
  \end{verse}

  Si dice \textit{Piantar carote}, perché questa pianta fa grossa radice, e cresce assai nei
  terreni dolci, e teneri, ed uno facile a credere si dice \textit{Homo dolce, e tenero}.

\item[VOI havete buona soprascritta] La faccia suol esser dimostratrice delle passioni
  interne, e però dicendosi \textit{haver buona soprascritta} s'intende haver buona sanità, come
  dichiara il Poeta medesimo dicendo; \textit{Non occorre domandarvi come voi state, perché
  si conosce dalla buona soprascritta}, cioè la sembianza, la buona cera, ed aria
  del volto ci dice, che vai state bene. E così la voce \textit{soprascritta}, che vuol dire
  Inscrizione, che si fa alle lettere, ci serve per intender quanto sopra s'è detto.

\item[LA bugia vi corre su pel naso] Tu dai colore. Tu ti muti di colore in viso, perché
  tu hai detto una falsità, \textit{Tui oculi declarant}, Lo Scoliaste di Teocrito spiegando
  quei versi dell'Iditio 12. che in Latino furono così tradotti: \textit{Verum ego te
  laudans, formose, haud mentiar umquam, Nec tenui gravis innascetur pustula nari};
  dice così. Vuol dire, che nel lodarti, io non mentirò, non mi nascerà sopra
  al naso la bugia; poiché alcuni sogliono chiamare certe bollicine bianche, che
  vengono su pel naso, bugie: e colui che le aveva, era notato, come bugiardo.
  Fin qui lo Scoliaste.

\item[RISPARMIARE] O \textit{rispiarmare}. Vale per perdonare. Qui s'intende l'Orco
  che non ha voluto far male alcuno.
\item[HAVER uno sul calendario] Havere a noia, o vero odiar' uno.
\item[QUANTO a me gli vo tutto il mio bene] Per quanto s'aspetta a me gli porto
  tutto quell'affetto, che si può portare; l'amo di tutto cuore.
\item[TI stava bene] E' lo stesso che Ti stava il dovere. Tornava bene, che l'Orco
 t'havesse ingoiato, perché t'haverebbe fatto quello che tu meritavi.
\item[PANCACCIA] Così si chiama da noi quel luogo dove si ragunano i novellisti
  per darsi le nuove l'un l'altro, ed ha questo nome di Pancaccia, perché nel tempo
  di state questi tali si radunavano già per sentire il fresco vicino alla Chiesa
  Cattedrale, sedendo sopra a un muricciuolo coperto di tavoloni, o panconi, e
  da questi prese il nome di Pancaccia. E da questa \textit{pancaccia}, \textit{Pancaccieri}, e \textit{Pancacciai}
  intendiamo quei perdigiorni, che stanno oziofamente ragionando de i fatti
  d'altri, ed in questo senso è preso nel presente luogo, che dicendo \textit{quei della pancaccia},
  intende una quantità di questi Crocchioni. Vedi sotto \cstan[6]{69}. Canti
  Carnascialeschi, \textit{Chi vuol udir bugie, o novellacce Venga ascolar costoro; che
  si stan tutto il dì su le pancacce}.

\item[GLI butta in faccia la sua caponeria] Gli rimprovera la sua ostinazione.

\item[QVEL ch' ei ne cavò po poi in quel fondo] Quel ch'ei guadagnò, ed acquistò alla
  fine delle fini, o in ultimo degli ultimi. Tanto servirebbe dir \textit{po poi} senz'aggiugnervi
  \textit{in quel fondo}, ma così è il nostro costume in simili casi per dar maggior
  emfasi, quasi dica una fine più la delle fini, Vedi sotto \cstan[8]{51}.

\item[CAVAR due occhi a te per trarne uno al compagno] Detto vulgatissimo, che ci
  serve per esprimere \textit{Far a se molto male, per farne pochissimo al nimico}.

\item[FVOR de' denti] Apertamente; chiaramente è il Lat. \textit{Eloqui}, ed è il contrario
  di parlar fra denti, o a mezza bocca, che significa non si lasciare intendere, forse
  e il \textit{Mussitare} de i Latini.

\item[A BVONA cera] Con allegra faccia; cioè non sopraffatto da collera, o altra
  passione, ma con animo riposato; diciamo anche \textit{sul sodo}, \textit{sul serio} tolto lat
  Lat. \textit{Serio admonere}. Il Lalli Eo. Te. \cstan[4]{103}.
  \begin{verse}
    Prega, scongiura, e dille a buona cera.
  \end{verse}

\item[BILLERA] Burla nociva,o se non cattiva del tutto, almeno che non piace;
  voce corrotta da \textit{Villera} voce antica che vuol dir Villania.

\item[TE NE struggei come la cera] Il verbo struggersi, che vuol dine Liquefarsi, serve
  a noi per farsi intendere d'uno che ardentemente desideri qualcosa. Il Lalli
  En. Tr. \cstan[4]{109}. disse.
  \begin{verse}
    Che se ne strugge come le candele.
  \end{verse}

\item[LAPPOLA] Cosa da non stimarsi. L'erba da nostri contadini chiamata \textit{Lappola}
  fa un seme pieno d'acute spine, ma fragili; E però dicendosi: \textit{non lo stimo una
    Lappola}, s'intende non lo stimo punto, e s'usa per lo più trattandosi di bravura,
  e valore, alludendo a quell'armatura di spine, che ha la Lappola, le quali se
  ben son molte, ed acute, non hanno con tutto ciò forza d'offendere, per esser
  fragilissime.

\item[DESTI nella Trappola] V'incappasti, Vi rimanesti preso. \textit{In laqueum incidisti}.
  \textit{Trappola} intendiamo ogni sorte d'artifizio, che si trova per pigliare animali
  tanto di terra, quanto d'aria, e d'acqua, donde \textit{Trappolare} val per Ingannare.
  Ma \textit{Trappola} strettamente presa s'intende un'artifizio per pigliare i topi,
  ed una specie di rete da pescare ha il solo nome di \textit{Trappola}.

  Si dice \textit{Trappola da quattrini}, per intendere Invenzioni per fare spendere.
\end{description}

\section{Stanza LXXV — LXXIX}

\begin{ottave}
\flagverse{75}Amadigi alla donna mai rispose,\\
E fece il sordo ad ogni suo quesito,\\
Ma si ben' attingea da queste cose\\
Quanto a Florian potea esser seguito,\\
E venne immaginandosi e s'appose,\\
Ch'ella fusse sua Moglie, ei suo Marito,\\
E ch'egli essendo tutto lui maniato\\
Fusse pel suo Fratel da ognun cambiato.
\end{ottave}

\begin{ottave}
\flagverse{76}Ma perch' ei non credea veder mai l'hora\\
D' haver il suo Fratello a salvamento,\\
Dà un ganghero a tutti, e torna fuora\\
Dietro al suo can veloce come il vento;\\
Ne era un trar di mano andato ancora\\
A caccia all'Orco ch' ei vi dette drento\\
Come il Fratel vedendo un bel cignale,\\
Ma non fu quanto lui dolce di fale.
\end{ottave}

\begin{ottave}
\flagverse{77}Che seguitollo anch'ei per quelle strade\\
Dond'ei conduce l'huomo alla sua tana,\\
Ove mentre diluvia, e dal Ciel cade\\
E broda, e ceci, il Cristianello intana.\\
Ed egli tanto poi lo persuade\\
Ch'ei lega i cani, e posa durlindana,\\
Havendo havuto innanzi la lezione,\\
Si stette sempre mai sodo al macchione.
\end{ottave}

\begin{ottave}
\flagverse{78}E quando l'Orco poi venne anc'a lui\\
A dar parole con quei tempi strani,\\
Ed all'uscio facea Pin da Montui\\
Affin che l'arme e i cani egli allontani\\
Ei disse: Su piccin piglia colui,\\
E chiappata la spada con due mani\\
Si lanciò fuora, e quivi a più non posso\\
Gli cominciò a menar le man pel dosso.
\end{ottave}

\begin{ottave}
\flagverse{79}E mentre ch'or di punta, ed hor di taglio \\
Di gran finestre fa, di lunghe strisce\\
Più presto che non va strale a berzaglio\\
Il can s'avventa anch' egli, e ribadisce.\\
Tal che tutto forato come un vaglio\\
Il pover'Orco al fin cade, e basisce,\\
E lì tra quelle rupi, e quelle macchie\\
Rimase a far banchetto alle Cornacchie.
\end{ottave}

Amadigi argumentò dal discorso di Doralice, che ella fusse Moglie di Floriano,
e compreso quanto poteva esser' avvenuto al medesimo; e però senza dar altra
risposta dette addietro, ed uscito di Campi, fu dal Cane guidato alla tana
dell'Orco, il quale fu da lui con aiuto del suo cane, ammazzato.
\begin{description}
\item[MAI] Questo avverbio che significa In alcun tempo serve anche per negativa,
  come è nel presente luogo, e come l'usò più volte il Boccaccio ed in specie Nov.
  73. \textit{Mai frate il Diavol ti ci reca ec.} E Nov. 54. \textit{Che mai ad animo riposato si sarebbe
  potuto ritrovare}, e Nov. 77. \textit{Mai di ciò che hora mi parli dubitai}, Matteo Villani
  \libcap[8]{39}. \textit{I Perugini mai si vollero dichiarare}, ed in molti altri luoghi del
  Boccaccio, del Passavanti\footnote{Jacopo Passavanti, Firenze, 1302 circa – Firenze, 15 giugno 1357, scrittore, architetto e religioso.}, e d'altri Scrittori del buon secolo si trova usato per negativa.
  Ho voluto dir ciò in questo luogo per toccare la difesa dell'Autore dalla
  critica datagli d'haver usato questa voce \textit{Mai} per negativa senza l'aggiunta della
  particella \textit{ne}, o \textit{non}, e senza correlazione alla negativa anteposta nel medesimo
  periodo, e che tanto vale il dire \textit{Io non farò mai questo}, quanto il dire \textit{Io mai
    farò questo}, E mi rimetto all'uso, ed al \textit{TORTO, E DIRITTO} del P. Bartoli, per
  la difesa di questa opinione.
\item[FECE il sordo] Finse di non sentire.
\item[ATTINGEA da queste cose] IL verbo \textit{attingere} o \textit{attignere}, che è il Latino
  \textit{attingere} per arrivare a un luogo, o a un fine; \textit{Metam attingere}: da noi è preso
  ed usato come il verbo \textit{haurio}, che vuol dir Cavar l'acqua da i pozzi, che noi diciamo
  attignere, ed in significato di \textit{Comprendere}, \textit{vedere}, \textit{udire},
  \textit{oculis \& auribus haudire}. E nel significato di \textit{Comprendere} è preso nel presente luogo.
\item[S'APPOSE] Verbo neutro che val per indovinare: Ed attivo vuol dire Dar
  la colpa a uno. \textit{Io m'apposi di chi haveva fatto il male, e però l'apposi a lui}. Io
  m'indovinai chi fusse stato quello che havea fatto il male, e però ne diedi la
  colpa a lui.

\item[TVTTO lui maniato] Come lui per appunto: Similissimo a lui: \textit{Fatto a capello},
  che vedemmo sopra in \cstan{19}. Lasca Nov. 7. dice: \textit{Il qual fantoccio
  vestito de' panni del Pedagogo, tutto maniato parea lui}. Io credo che sia parola
  corrotta da \textit{miniato} cioè diligentemente dipinto, o forse corrottamente derivato
  dai Latino barbaro \textit{Emanatus}, tanto simile a lui, che pare \textit{emanatus ab illo}.

\item[NON credea di veder mai l'hora] Amadigi havea così gran desiderio di vedere
  il suo Fratello libero, che dubitava non fusse per arrivar mai quell'hora, ed ogni
  momento, gli pareva un'anno.
\item[DÀ un ganghero] Dà volta addietro. Ganghero diciamo uno strumento per
  uso d'affibbiar le vesti, fatto di filo di ferro, o d'altro metallo, il quale è fatto
  in forma d'uncino, e da quella rivolta, che egli fa, \textit{dare il ganghero} intendiamo
  tornar indietro. \textit{Retrorsum vela dare}. Dare il ganghero, diciamo quando la lepre
  fuggendo avanti al cane, torna indietro, e lascia correr il cane, che portato
  dalla velocità non si può ritenere, e voltarsi subito come fa essa, che in tanto piglia
  campo in maniera ch'ella scampa, dal che diciamo \textit{Far lepre vecchia} per intender
  \textit{tornar indietro}. Vedi sotto \cstan[10]{23}.

\item[NON fu si dolce di sale] Non fu sì credulo: Sì minchione: Sì sciocco. Una
  vivanda poco salata si dice \textit{dolce di sale}, cioè sciocca. Donde esser senza sale, o
  non haver sale in zucca vuol dire Huomo sciocco, senza giudizio, senza cervello.
  Sale chiamiamo l'arguzie, e detti ingegnosi. Vedi sotto \cstan[8]{26}. Diciamo
  \textit{il tale è dolce}, e senza l'aggiunta \textit{di sale} intendiamo è corrivo, credulo
  minchione, e senza giudizio; e per coprire più questo detto, usano molti dire
  \textit{Lupinaio} (che vuol dir colui che vendendo per Firenze Lupini va gridando \textit{dolci
    dolci}) per intendere \textit{Costui è dolce}. Qui dunque vuol dire, che Amadigi non fu
  corrivo quanto era stato il Fratello a credere all'Orco. Bocc. Gior. 4. n. 2, \textit{Madonna
  Zucca al vento, la quale era anzi che nò un poco dolce di sale}, Lasca Nov. 2.
  \textit{E perché egli era nato in Domenica, non sendo la gabella del sale aperta, tenne sempre
  molto bene del dolce}.

\item[TANA] Caverna, grotta, buca. Donde \textit{intanare}, entrar nella \textit{tana}.

\item[BRODA, e ceci] Intendi acqua, e gragnuola. Fu un ragazzo ghiotto delle
  civaie, per il quale suo padre (per mortificare questa sua gola) ordinò, che nella
  sua scodella non si mettesse altro, che il puro brodo de' ceci, o d'altre civaie
  respettivamente, onde il povero ragazzo vedendo gli altri con le scodelle piene
  di legumi si disperava. Ed essendosene andato un giorno in camera mentre pioveva
  se ne stava alla finestra gridando \textit{acqua, e gragnuola}, e questo per la rabbia,
  che haveva, che si stagionassero i legumi per gli altri, e non per lui. Sentì il
  padre questo suo gridare, che gli disse: perché preghi il Cielo a mandar la grandine,
  cosa tanto nociva? L'astuto ragazzo per scampar la furia subito rispose:
  Padre mio io non ho mai desiderato, o pregato male per nessuno, e se io pregavo
  che insieme con l'acqua venisse anche della grandine, ho voluto intendere, che
  il Cielo vi mettesse una volta in testa di farmi dare con tanta broda una volta
  anche de' ceci, che di questi intendevo quando dicevo gragnuola. Il Padre rise
  dell'astuzia, e dette ordine, che per l'avvenire fusse trattato, come gli altri.
  E da questo intendiamo acqua e gragnuola, quando diciamo broda, e ceci.

\item[CRISTIANELLO] E' detto d'avvilimento, e significa Huomo dappoco, o di
  poca fortuna, o di piccola figura; che i Latini dicono \textit{homuncio}, e noi talvolta
  in questo senso diciamo \textit{Homicciuolo}.

\item[DURLINDANA] Intende la spada,e piglia questa denominazione dalla famosa
  spada d'Orlando Paladino, la quale da i Poeti hebbe il nome di \textit{Durlindana},
  o \textit{Durindana}.

\item[HAVENDO havuto innanzi la lezione] Essendo stato prima informato; avvisato,
  instruito: Cioè havendo compreso dal discorso di Daralice, che questo era
  quell'Orco, che ingannava.

\item[STAR sodo al Macchione] Intendiamo non condescendere alle richieste, o non
  si lasciar lusingare dall'esortazioni di alcuno. Questo detto viene da quegli uccelletti,
  che stanno per le macchie, dove si tendono le ragne, i quali, per essere
  stati altre volte molestati, hanno imparato, che quello scacciargli col battere la
  macchia era di lor poco danno stando fermi, però non si muovono a ogni romore,
  e questi si dicono \textit{star sodi al Macchione}, Di tali uccelli si dice anche
  \textit{accivettati}, Vedi sotto \cstan[9]{22}.

\item[FACEA Pin da Montui] Cioè facea capolino, che vuol dir quel che accennammo
  sopra \cstan[1]{7}. Questo detto viene da una canzonetta, o villanella,
  che dice.
  \begin{verse}
    Pin da Montui, Fa capolino
    Dreto è Menghino, E Mon con lui, ec.
  \end{verse}

  Plauto disse \textit{Ex insidijs clanculum aucupari}.

\item[SU piccino] È modo di incitare il cane contro a uno, È l'\textit{irritare}, o \textit{immittere}
  de i Latini, che noi diciamo anche \textit{ammettere}. Vedi sotto C.~11. stan.~29. si dice
  anche \textit{aissare} verbo originato da quel suono, che fa la voce dicendosi: \textit{su su}; O
  dalla parola \textit{iza} voce antica, che vuol dire Ira, dalla quale habbiamo il verbo
  \textit{aizzare}, o \textit{adizzare}, o \textit{aissare}, Dan, Inf. C.27.
  \begin{verse}
    Dicendo, issa ten va: più non t'aizzo.
  \end{verse}

\item[A PIÙ non posso] Con ogni maggior potere; Quasi dica con animo di seguitare
  a far quella tal cosa fino ache non sara stanco, e non possa più.

\item[MENAR le man pel dosso] Adoperar le mani nella persona d'uno, cioè Perquoterlo.
  La voce \textit{dosso} dal Latino \textit{dorsum}, da noi s'intende per tutto il torso
  dell'huomo, parendo che s'eccettuino da molti il capo, le braccia, e le gambe.
  Lasca lib. 1. Nov. 7. \textit{Non contento di ricercargli col bastone le braccia, e le gambe,
  volle ancora con esso ritrovargli tutto il dosso}.

\item[GRAN finestre, e lunghe strisce] Gran ferite di punta e di taglio \textit{Punctim, \&
  caesim}, disse Vegezio. Dice strisce per la similitudine che ha una lunga ferita di
  taglio con la striscia, e lo fa per esprimere che eran ben lunghe, come dice \textit{finestre}
  quelle di punta perché s'intenda, che eran larghe.

\item[AVVENTARSI] Spingersi, gettarsi, o andar velocemente, o con impeto
  alla volta d'uno, che i Latini dicono \textit{irruere}.

\item[RIBADIRE] Ribattere. Quando si mette un chiodo dentro a una tavola, e che la
  punta di esso chiodo passa dall'altra parte, la detta punta si piega, e si riconficca
  perché il chiodo faccia l'effetto d'una legatura; e per far questo, uno batte in
  su la punta del chiodo, e l'altro tiene a riscontro in sul capo del chiodo un ferro;
  e questo si dice \textit{ribadire}; e però perquotendo Amadigi da una parte, e il cane
  mordendo dall'altra l'Autore per esprimer questo atto si serve del verbo \textit{ribadire}
  usato da molti ed in questi termini, ed anche per replicare.

\item[FORATO come un vaglio] Havevano fatto nella persona dell'Orco più buchi,
  e tagli che non ha un vaglio, strumento col quale si separa il grano dall' immondizie,
  detto dal Latino \textit{Vannus}, e talvolta Crivello dal Latino \textit{Cribrum}, e
  \textit{Cribellum}, voce usata dall'Agricoltore Palladio. Questa comparazione era usata anche
  da i Latini trovandosi in Plauto \textit{Carnificum cribrum}, parlando di un servo che
  era stato mal concio dalle bastonate.

\item[BASISCE] Muore. Questo verbo ha forse l'origine dalla Greca voce \textit{Basis},
  che vuol dire \textit{incessus}, e che intendiamo \textit{il tale se n'andò}, per il tale mori, che diciamo
  \textit{basì}: vedi l'Ottava 82. seguente, e da questo verbo deriva la voce \textit{basto}, che
  vuol dir huomo senza sentimento, e quasi morto. Messer Gio: della Casa nel
  Capitolo del Martello d'Amore dice.
  \begin{verse}
    Perché ti guardi torto la Signora;
    Parti haver le budella in un catino,
    E doventi bafito allora allora.
  \end{verse}

Vedi sotto \cstan[6]{97}.
\end{description}

\section{Stanza LXXX — LXXXII.}

\begin{ottave}
\flagverse{80}Amadigi dipoi fece pulito,\\
Perché trovato havendo il suo Fratello\\
Con una barba lunga da Romito,\\
E più lordo, e più unto d'un panello,\\
Lavatolo, e rimessogli il vestito,\\
Ch' era ancor quivi tutto in un fardello,\\
Lo ricondusse a Campi, ove la Moglie\\
Di lui già pregna, appunto avea le doglie.
\end{ottave}

\begin{ottave}
\flagverse{81}Corse la Levatrice, ed in effetto\\
Fra mille hoimè, se' soldi, e doglien hora,\\
Partorigli una bella piscialletto\\
Che fusti tu, poi detta Celidora,\\
E maritata al Re, come s'è detto,\\
Di Malmantil del qual tu sei Signora;\\
Ne sei, e ne sarai, io lo raffibbio,\\
Se ben non puoi per hor dir come il nibbio.
\end{ottave}

\begin{ottave}
\flagverse{82}Ma presto come lui potrai dir mio.\\
Hor senti pur: Basito Perlone\\
Anco Amadigi subito tuo Zio\\
Venne a tor donna, e n'hebbe un bel garzone,\\
Che Baldo fu chiamato, e quel son' io,\\
Che poi cresciuto detto son Baldone.\\
Hor eccoti dal primo al terzo grado\\
Narrato tutto il nostro parentado.
\end{ottave}

Amadigi trovato il Fratello Floriano lo rivestì e lo ricondusse a Campi dove
Doralice partorì Celidora; e d'Amadigi nacque Baldone. E con terminare il
racconto, termina il Poeta il secondo Cantare.
\begin{description}
\item[FECE pulito] Fece il negozio aggiustatamente, e come andava fatto.

\item[BARBA da romito] Barba lunga, e incolta, che tale per lo più suol esser la
barba de i Romiti.

\item[LORDO] Sudicio schifo. Dal latino \textit{Luridus}, che vuol dir Livido, quasi \textit{per
  lorum cussum, \& lividum factum}. E questo epiteto s'adatta non solamente all'huomo,
  ma ancora ad ogni materiale, o strumento, sopra il quale sia schifezza.

\item[PANELLO] Così chiamiamo un viluppo di cenci intinti nell'olio, sego, o
  altra materia oleacea, e bituminosa il quale serve per abbruciare in far luminarie
  in occasione di pubbliche feste, ed allegrezze in luoghi eminenti, e dominati
  da i venti, a' quali questi resistono.  Dal Greco Panos, che val lo stesso. Varchi
  stor. lib.~11. \textit{Si fecero per tutto feste, ma la sera non s'arsero panelli per difetto d'olio}.

\item[LEVATRICE] Raccoglitrice. Quella che raccoglie, e leva la Creatura dalla parturiente
  da i Latina detta \textit{obstetrix}, ed in alcuni luoghi detta Mammana.

\item[HOIMÉ] Voce, che esprime afflizione d'animo, e di corpo, che i Latini
  dicevano \textit{hei mihi}, e noi forse l'habbiamo dal Greco \textit{hoi moi}. E quell'aggiunta
  \textit{Sei Soldi} e \textit{doglien' hora} è posta per scherzo, e per burlare chi talvolta si duole, o si
  rammarica, o fa lezzj senza cagione, o per dolori leggieri, che noi diciamo,
  \textit{fare il monello}, e non è riempitura intentata dal Poeta, ma è pur così in uso, dicendosi
  a questo tale: O pover' huomo! \textit{Aimé! sei soldi, e dogliene hora}; e si nomina
  una somma di monete per haver occasione di dire \textit{dogliene}, che è il verbo
  \textit{dare}, ed in questa occasione si dice, perché ha similitudine con la voce \textit{doglia}.

\item[PISCIALLETTO] Una bambina. Quando una donne partorisce una Femmina,
  niuna di quelle donne che sono attorno alla parturiente le vuol dar la
  nuova, che ella sia femmina, ma perché pure al fine ella lo deve sapere, per non
  profferire la parola femmina dicono: Una \textit{Piscialletto}, \textit{Una come me}, e simili. E
  da questo noi habbiamo \textit{far' un bambina}, che vuol dir Fare un'errore.

\item[LO rafibbio] Lo replico.

\item[NON puoi dir come il nibbio] Cioè non puoi dir Mio. Il Nibbio uccello rapace
  non fa altro canto, ne si sente da lui altra voce, che un certo fischio, o strido,
  che par che suoni \textit{mio mio}, e da questo per avventura i Latini lo dicon \textit{Miluus},
  g1i Spagnuoli \textit{Milano}, e i Francesi \textit{Milan}; E noi da questa sua voce volendo
  esprimere, che una cosa sia veramente mia, dichiamo: \textit{Posso dire come il nibbio}, cioè Mio;
  l'autore lo dichiara nel primo verso dell'ottava seguente dicendo: \textit{Ma presto
  come lui potrai dir mio}.

\item[BASITO] Vedi l'ottava 79, antecedente.

\item[ZIO] Fratello del padre, o della madre, o marito d'una sorella del padre, o
  della madre: Qui è fratello del padre.

\item[VN bel garzone] Cioe un figliuol maschio. E qui il Poeta seguita a mostrare il
  costume delle nostre donne accennato nell'ottava antecedente, che quando il parto
  è di maschio, ognuna di loro vorrebbe esser la prima a darne la nuova, e
  danno alla creatura sempre qualche epiteto, come \textit{un bel garzone}, \textit{un bel giovane},
  \textit{un garbato fantoccione}, \textit{un bamboccione d'importanza}. Vedi sopra in questo C.
  stan. 19. ma quando è femmina, tutte le assitenti ammutoliscono, o quando pur' al
  fine lo dicano, danno alla creatura epiteti d'avvilimento, come \textit{Piscialletto},
  \textit{Pisciacchera}, \textit{una sguaiatuccia}, e simili, come habbiamo detto poco sopra.
\item[IL nostro parentado] La nostra Genealogia: In che modo noi siamo parenti.
\end{description}

\section*{FINE DEL SECONDO CANTARE.}
\chapter{Terzo Cantare}

\begin{argomento}
Vengon d'Arno a seconda i legni Sardi,
Sbarcan le genti, e vanno a Malmantile,
Ma per vari accidenti i più gagliardi
Non fan quel tanto, che di guerra è stile.
Arma i suoi Bertinella, alza stendardi,
E mostra in debol corpo alma virile.
Nascon grandi scompigli in quella piazza,
E ognun si fugge in veder Martinazza.
\end{argomento}

\section{Stanza I \& II.}

\begin{ottave}
\flagverse{1}UN che sia avvezzo a starsene a sedere\\
Senza far nulla con le mani in mano,\\
E lautamente può mangiare, e bere, \\
E in festa, e giuoco viver lieto, e sano, \\
Se gli son rotte l'uova nel paniere, \\
Considerate se gli pare strano, \\
Ed io lo credo; c' a un' affronto tale \\
Al certo ognun l'intenderebbe male.
\end{ottave}

\begin{ottave}
\flagverse{2}E pur chi vive, sta sempre soggetto\\
A ber qualche sciroppo che dispiace,\\
Perché al Mondo non è nulla di netto,\\
E non si può mangiar boccone in pace,\\
Hor ne vedremo in Malmantil l'effetto,\\
Che immerso nei piacer vivendo a brace,\\
Non pensa che patir ne dee la pena,\\
E che fra poco s' ha mutare scena.
\end{ottave}

Il Poeta volendo trattare dell'assalto dato a Malmantile, e del disturbo, che
è per apportare l'esercito di Baldone a quelli spensierati, che sono nella Terra,
introduce il presente Cantare con una reflessione, che sia un gran disturbo a coloro,
i quali standosene co i loro commodi, e senza un minimo pensiero, si veggano
sopraggiugnere chi gli privi di questi loro agi; mentre simili accidenti sarebbono
di gran disgusto, e noia anche a coloro, che non stessero con tutti i lor
commodi; perché niuno, o bene, o male, che gli stia, vuol mai ricordarsi, che
tutti siamo sottoposti alle disgrazie, e che nel mondo non si dà felicità perfetta.

\begin{description}

\item[STARSENE con le mani in mano] A cintola, o in seno. Si dice d'uno, che
sia tutto dato in preda all'ozio, ed alla poltroneria, e che non vuol lavorare.
Vn accidioso, nighittoso, o scioperato. I Greci, e Latini dissero: \textit{In choenice
sedere}: \textit{de homine ocioso, \& desidioso}.

\item[GVASTAR l'uova nel paniere] Guastare i disegni altrui, Traslato dal guastar l'uova
  nel nidio, dove son dalla chioccia covate. Vedi Esopo Favola dell'Aquila, e dello
  Scarafaggio. È il \textit{covatum frangere} de i Latini.

\item[SE gli pare strano] Se gli par duro, e difficile a soffrire. Vedi sopra Cant. 2.
stan. 21.,  ed il proprio significato è di \textit{strano}. Stravagante, o forestiero, o non
del nostro parentado; valendocene in tutti questi, ed altri significati, come segue
ne i Latini della voce \textit{extraneus}.

\item[AFFRONTO] Significa Aggressione, assalto, abboccamento. Vedi sopra
Cant. 1. stan. 29. ma si piglia ancora per Sopruso, come è preso nel presente
luogo.

\item[BERE una sciroppo, che dispiaccia] Sopportar per forza una cosa, che sia di
  disgusto, che in Latino: si disse: \textit{Calicem bibere}; perché \textit{Calix} era una specie
  di bicchiere, col quale gli antichi bevevano caldo, come appunto si bevono gli sciroppi;
  e lo facevano ancor' essi per medicamento; e per conseguenza era tal bevanda,
  come a noi, per lo più, di poco gusto.

\item[NEL Mondo non è nulla di netto] Il Mondo non ha felicità perfetta. \textit{Unicuique
  dedit vitium natura creato}.

\item[VIVER a brace] Viver' a caso, senza regola, o considerazione. Ha forse
  questo detto origine dalla misura, che si fa della brace, che per esser cosa vile, e
  di poco prezzo si misura inconsideratamente senza guardare a darne un poca più
  o un poca meno. Da questo poi habbiamo \textit{sbraciare} veduto sopra Cant. 2. stan.
  10, che significa Consumare il suo inconsideratamente.

\item[MVTARE scena] Mutar faccia, o stato, mutar maniera di vivere, Traslato
  dalle prospettive, dove si recitano le commedie, quali prospettive sono da noi
  vulgarmente chiamate Scene.
\end{description}

\section{Stanza III \& IV.}

\begin{ottave}
\flagverse{3}Era in quei tempi la, quando i Geloni \\
Tornano a chiuder l'osterie de' cani,\\
E talun, che si spaccia i millioni \\
Manda al presto il tabì pe' panni lani;\\
Ed era appunto l'ora, ch' i Crocchioni \\
Si calano all assedio de' caldani; \\
Ed escon con le canne, e co' i randelli \\
I ragazzi a pigliare i pipistrelli.
\end{ottave}

\begin{ottave}
\flagverse{4}Quand in terra  l'armata con la scorta\\
Del gran Baldone a Malmantil s'invia,\\
Ond' un famiglio nel serrar la porta\\
Sentì rumoreggiar tanta genia.\\
Un vecchio era quest'huom di vista corta,\\
Che l'erre ogni hor perdeva all'osteria,\\
Tal che tra il bere, e l'esser ben d'età\\
non ci vedeva più da terza in là.
\end{ottave}

Descrive la stagione, che correva; quando la soldatesca sbarcò in terra, e s'avviò
verso Malmantile sotto la condotta di Baldone; e dice che era sul finire dell'Autunno,
poiché cominciava a diacciare, ed i ricchi finti mandavano a impegnare i vestiti da
state per risquoter quelli da inverno; costume assai usato da coloro,
che sfoggiano in vestire quantunque sieno poverissimi, e questi intendi
\textit{ricchi finti, che si spacciano i millioni}, che si suol dire; \textit{Mezzettin non risquote
  Pantalone}, e s'intende, che gli abiti da state non vagliono tanto, che impegnandoli
possano risquotere quei da inverno, come appunto è l'abito povero di Mezzettino
servo sciocco in commedia, e l'abito ricco di Pantalone vecchio in Commedia.
Narra parimente l'hora appunto che era, quando costoro s'accostarono a Malmantile,
e dice, che fu su l'annottare, che è quell'ora, su la quale i Crocchioni
si mettono nelle botteghe intorno a un caldano per passar la veglia. In tale stagione,
e fu quest'ora adunque arrivarono i soldati, condotti da Baldone, sotto
Malmantile, ed un famiglio nel serrar la porta gli scoperse più al romore, che
perché gli vedesse, essendo egli poco men che cieco.

\begin{description}
\item[GELONI] Intende freddi grandi, che fanno gelare, o addiacciare. Detto
equivoco da Geloni Popoli di Scitia, quali popoli pare che voglia dire, che sieno
coloro, che tornano a chiudere l'osterie de' cani. Le quali diciamo alcune buche nel
terreno della nostra Città cagionate dal mancamento delle lastre, le quali buche
nel tempo dell'inverno stanno piene d'acqua, e vulgarmente s'appellano pozze;
ma son chiamate Osterie de' cani, perché a queste vanno i cani a bere, e quando
vengono i diacci (che sono questi Geloni) ancor'esse addiacciano, e così restano
sode, e chiuse in modo che i cni non vi possono bere, e però dice, che i Geloni
tornano a chiuder l'osterie de' cani.

\item[TALUN che si spaccia i millioni] Uno che dà a creder d'esser ricchissimo Diciamo
  \textit{millantare} o \textit{smillantare}, come si vedrà sotto C.~11. stan.~49. d'uno che si
  spacci, o si vanti di ricco, di nobile, di dotto, ec. che da i Latini si dice: \textit{Sese
    iactare}. E questi tali si dicono \textit{Homines gloriosi, thrasones} per smillantatori tanto
  di ricchezze, quanto d'ogni altra cosa.

\item[PRESTO] Luogo pubblico, dove si pigliano in presto denari, con dare in pegno, e pagare g'interessi del denaro.

\item[TABÍ] È una specie di drappo leggieri di seta; E Dicendo: \textit{Manda al presto
  il tabí pe i panni lani}, intende Manda a impegnare l'abito da state per risquoter
  quello da verno.

\item[CROCCHIONI] Chiacchieroni, Cicaloni. Intendi certi perdigiorni, che si
  confinano a sedere in una bottega senza far'altro, che cicalare, il che si dice
  \textit{crocchiare}, o \textit{star'a crocchio}, donde poi \textit{Crocchioni}. Vedi sopra \cstan[1]{41}.

\item[SI calano] Cioè se ne vanno. Detto da gli uccelli, che in su quell'ora si calano
  a i lor pollai per dormire.

\item[CALDANO] Intendiamo quel vaso di rame, o di ferro, o di terra, o di altro
  materiale, che è usato per tenervi dentro brace, o carboni accesi per scaldarsi,
  e questo intende nel presente luogo; che per altro, \textit{Caldano} appellano i
  fornai quella stanza, o volticciuola, che hanno sopra il forno.

\item[PIPISTRELLO] Che si dice anche Vispistrello, o Vipistrello dal Latino Vespertilio,
  è il topo alato, animale notturno notissimo, come ancora è nota la caccia,
  che fanno i ragazzi del medesimo con brandire una canna, al fischio e sibilo,
  della quale egli vola, e da essa vien percosso, e fatto cadere a terra sbalordito;
  e perché alla detta caccia tanto serve una canna, che un bastone, però dice:
  \textit{con le canne, e co' i randelli}, cioè bastoni.

\item[FAMIGLIO] Qui intendi Birro guardia della porta.

\item[GENIA] Dal Grec. \textit{Genea}. Generazione. E vuol dire Gente vile, abbietta,
  e sciagurata: Sinonimo di gentaglia, genticciuola, ec.

\item[PERDER l'erre] Imbriacarsi: perché i briachi stentano a profferire la lettera R
  per avere la lingua legata dal troppo bere.

\item[Non ci vedeva più da terza in là] Se gli faceva buio, o notte a Terza, che è
quasi il principio del giorno, sì che si può dire, che costui fusse sempre al buio,
o non vedesse punto in tutto il giorno. È detto assai vulgato per intender uno
debole di vista, come intende nel presente luogo. Vedi spra \cstan[1]{9}. E forse
vuol intendere Uno di coloro, che perdono la vista alla levata del sole, e la
riacquistano quando il sole va sotto.
\end{description}

\section{Stanza V. — VII.}

\begin{ottave}
\flagverse{5}Per questo mette mano alla scarsella,\\
Ov' ha più ciarpe assai d' un rigattiere,\\
Perché vi tiene infin la faverella, \\
Che la mattina mette sul brachiere; \\
Come suol far chi giuoca a cruscherella, \\
Due hore andò alla cerca intere intere, \\
E poi ne trasse in mezzo a due fagotti \\
Un par a occhiali affumicati, e rotti.
\end{ottave}

\begin{ottave}
\flagverse{6}I quali sopra il naso a Petronciano\\
Con la sua flemma pose a cavalcioni;\\
Tal che meglio scoperfe di lontano\\
Esser di gente armata più squadroni.\\
Spaurito di ciè, cala pian piano,\\
Per non dar nella scala i pedignoni;\\
E giunto a basso lagrima, e singozza,\\
Gridando quanto mai n'ha nella strozza.
\end{ottave}

\begin{ottave}
\flagverse{7}Dicendo forte, perché ognun l'intenda: \\
All'armi all'armi, suonisi a martello,\\
Si lasci il giuoco, il ballo, e la merenda,\\
E serrinsi le porte a chiavistello,\\
Perché quaggiù nel piano è la tregenda,\\
Che ne viene alla volta del Castello;\\
E se non ci serriamo, o facciam testa,\\
Mentre balliamo vuol suonare a festa.
\end{ottave}

Il detto famiglio scoperse col mettersi gli occhiali, che era gente armata, e
per questo si messe a gridare; all'armi.
\begin{description}
\item[SCARSELLA] Tasca, Vedi sopra \cstan[2]{8},

\item[CIARPE] Intendi robe vili, stracci, bazzecole, che i Latini dissero \textit{Scruta};
ed in altro senso \textit{Ciarpa} vedi sotto \cstan[5]{33}.

\item[RIGATTIERE] Rivenditore d'ogni sorta masserizie, ed arnesi da i Latini
detto \textit{Propola} dal Greco; ed a noi viene da rigaglie, che intendiamo robe diverse
di poco prezzo, ed avanzumi usati. L'Autore assomiglia la tasca di costui a una
bottega di Rigattiere, perché queste per lo più son ripiene di diversi arnesi, fra i
quali e talvolta difficile ritrovarvi una cosa, quand'altri la voglia.

\item[FAVERELLA] Fave macinate, ed impastate con acqua. E di questa si fanno
  torte cotte nel forno, che si dicono ancora Macco forse dal Grec, \textit{Matto}. Lat. \textit{pinso},
  Tale \textit{Faverella} dicono, che sia lenitivo a i dolori d'allentatura, ed habbia virtù
  d'assodar quelle parti; e però dice, che costui \textit{la mette in sul brachiere}, che è quella
  fasciatura, che s'applica all'estremità del ventre per sostenere gl'intestini.

\item[CRUSCHERELLA] È giuoco da Fanciulli. Fanno in sur' una tavola un monticello
  di Crusca, e vi mettono dentro quelle crazie, o quattrini, che vogliono
  giuocare, e mescolando poi bene, si fanno da uno del giuoco, a ciò deputato,
  tanti monticelli di detta crusca, quanti sono i giuocatori, i quali (lasciando da
  quello, che ha fatto i monti, perché deve esser l'ultimo a pigliare il monticello)
  tirano le sorti a chi debba esser il primo a pigliare uno di detti monti, e
  ciascuno nel monte, che gli è toccato va cercando de i denari, che la fortuana,
  v'habbia fatti restare. Stimo, che questo giuoco fusse usato ancora da i Fanciulli
  Latini, perché si trova \textit{Ludere furfure}, Ed a questa ricerca, che fanno i ragazzi
  del denaro assomiglia quello, che  il famiglio per trovare gli occhiali.

\item[FAGOTTI], Involti, o fardelli piccoli. Il Francese ancora dice \textit{Fagots}.

\item[PETRONCIANO] e Petonciano Specie di pomo simile alla mandragora, o
forse specie di Mandragora; e di color paonazzo lucente, nasce d'una pianta
simile alla Zucchetta, e sta appiccato al gambo con un poco di guscio come la
ghianda, alla quale s' affo a figura; ed in alcuni luoghi d'italia
si appella Marignano. A questo \textit{Petronciano} s'allomiglia comunemente, e da tutti
un naso di straordinaria grofiezza, e di colore rosso livido, come vuol che
s' intenda', che havesse questo famiglio.

\item[CAVALCIONI] Vuol dire una gamba da una parte, e l'altra dall'altra,
come si sta in sul cavallo, e come stanno gli occhiali sopra il naso, uno specchio
da una parte, l'altro dall'altra.

\item[PIAN piano] Cioè adagio adagio, bel bello: Adagissimo. La voce piano
aggiunta al verbo fare, ed al verbo andare significa quel, che hel presente luogo,
cioè Adagio, e con diligenza, che i Latini dicono placide incedere; ed aggiunta al
verbo parlare significa parlar con voce bassa, \textit{Submissa voce}.

\item[PEDIGNONI] Specie d'infermità, che viene ne i piedi, e nelle mani per lo
  troppo freddo dai Latini detti \textit{Perniones}.

\item[SIGNOZZARE] O singozzare, o singhiozzare. E' un moto del setto transverso,
  o mediastino, cagionato da soverchia votezza, o ripienezza; ma per similitudine
  significa anche sospirare vehementemente con pianto, come significa nel
  presente luogo. I Latini ancora se ne servivano nel primo significato, e nel secondo;
  \textit{Singultus}, \& \textit{singultire}, \& \textit{singultibus ingemere}.

\item[GRIDA quanto mai n'ha nella strozza] Grida quanto può più, e quanto può
resister la gola. Che \textit{strozza} vuol dire La canna della gola, altrimenti detta
\textit{Gorgozzule}. I Latini pure dicevano \textit{in gutture exclamare}, E da questa voce
\textit{strozza} viene strozzare, che vuol dire Strangolare.

Dante Inf. C. 7. \begin{verse}Quest' inno si gorgoglia nella strozza.\end{verse}
E. C. 28,  \begin{verse}Con la lingua tagliata nella strozza.\end{verse}

\item[SVONISI a martello] Si suonino le campane a rintocchi, che si dice anche: \textit{A
corr' homo}.

\item[TREGENDA] Moltitudine, e quantità di gente. Dalle persone semplici si
  crede, che vadano fuori la notte anime dannate, ed altri spiriti per spaurire la
  gente, e queste chiamano la \textit{Tregenda}. Tal' opinione se bene è di persone semplici,
  e idiote, nondimeno pare che venga seguitata da S. Agostino, poiché nel
  lib. 4. de Civit. Dei dice. \textit{Lamiae dicuntur animae hominum depravata, \& in malis
  vitae meritis maculosae, quae a corpore separatae terriculamenta sunt mortalibus}: nel presente
  luogo è intesa per moltitudine di gente.

\item[SVONARE] Il verbo suonare si piglia talvolta in vece del verbo percuotere,
  e però ne nasce l'equivoco del \textit{suonare mentre coloro ballano}, che vuol dire
  perquotergli, se ben pare, che voglia dire suonare alloro ballo: Ed in ciò imitiamo i
  Latini, che hanno il verbo \textit{pulsare}, che vuol dir perquotere, e vuol dire anche
  suonare ogni sorta di strumento musicale, e le campane; ed il suonatore si dice
  \textit{pulsator}.

\end{description}

\section{Stanza VIII. \& IX.}

\begin{ottave}
\flagverse{8}In quel che costui fa questa stampita,\\
E che ne i gusti ognun si balocca,\\
L'armata finalmente è comparita\\
Già presso a tiro all'alta Biccicocca.\\
Quivi si vede una progenie ardita,\\
Che si confida nelle sante nocca,\\
E se ne viene all'erta lemme lemme\\
Col Batthil Toffie tutto Biliemme.
\end{ottave}

\begin{ottave}
\flagverse{9}Tra questi guitti ancora sono assai,\\
Oltre a Marchesi, Principi, e Signori;\\
Huomin di conto, e grossi bottegai,\\
Banchieri, Setaiuoli, e Battilori,\\
Lanaiuoli, Orefici, e Merciai,\\
Notai, Legisti, Medici, e Dottori,\\
In somma quivi son gente, e brigate\\
D' ogni sorta; chiedete, e domandate.
\end{ottave}

Mentre il suddetto vecchio andava gridando, e che non ostante questo, coloro,
che erano in Malmantile seguitavano a darsi bel tempo, l'armata arrivò
presso le mura; Il Poeta narra la qualità di questi soldati.

\begin{description}
  \item[STAMPITA] Vuol dir suonata, o cantata, Bocc. Nov. 97. \textit{Con una sua viola
    suonò alcuna stampita}.  Varchi stor. lib. 10. Malatesta andò in persona sopra il bastion
    e di S. Miniato con tutti li suoi suanatori, e dopo più lunghe strombettate, e stampite,
    ec. Ma qui intende romore, e cicalamento odioso, che è il senso, nel quale oggi
    per lo più è presa da noi questa parola, ed ha lo stesso significato che \textit{bordello},
    \textit{chiasso}, \textit{musica}, e simili, presi pure metaforicamente, il che vedremo altrove.

\item[BALOCCARSI] Trastullarsi, Perder'il tempo, e trartenersi in cose di poco
  momento, o trastulli da ragazzi, de i quali è proprio il verbo \textit{baloccarsi}, o \textit{balocco};
  e forse è sincopato daf verbo \textit{Badaluccare}, e \textit{Badalucco}; Vedi sotto \cstan[6]{32}.

\item[BICCICOCCA] Diciamo anche \textit{Bicocca}. Varchi stor. lib. 15. \textit{gli furono portate
  le chiavi di non so che Bicocca}; Vuol dir fortezza piccola, e di poca considerazione
  posta in luogo eminente, come appunto è Malmantile, il quale con questa
  sola parola \textit{Biccicocca}, il Poeta benissimo descrive; perché per Biccicocca volgarmente
  intendiamo un Casolare, o castelluccio posto in luogo eminente, ma
  da farne poca stima. Lasca Nov. 3. \textit{Salita che hebbe con non poca difficultà quell'alpestre
  Montagna, credeva entrare in un bel castello, ma riguardando all' intorno, vedde
  che era una Biccicocca più per refugio di capre, che per ricetto di soldati}.

\item[SI confida nelle sante nocca] Ha la sua fidanza nelle pugna. E l'epiteto \textit{sante} è
  messo per esprimere il modo del parlare de i Battilani: Se bene e usato dalla gente
  anche più civile per interider perfezione come vedemmo sopra C. 2. stan.\ 52.
  E qui è benissimo posto, perché \textit{sanctus} vuol dir determinato, o stabilito, sendo
  sincopato da \textit{sancitus}, e le pugna sono s'armi stabilite, e proprie de' Battilani.
  Che per \textit{nocca}, che sono i nodelli delle dita, s'intende tutta la mano serrata, che
  in questo pugno, ed in questo più che in altra maniera si scorgono le nocca.

\item[LEMME lemme] È della medesima natura, ed ha lo stesso significato di pian
  piano detto sopra in \cstan{6}., ma è termine restato ne i Battilani, o se
  pure è usato da altri sarà detto \textit{lieme lieme}, che viene dal Latino \textit{leviter}, o \textit{leve}, e
  significa leggiermente, o dal Toscano Lieve, che vuol dir Leggieri.

\item[BATTI, e Tessi] Battilani, che son coloro, che conciano la lana, e Tessi
  quelli che la tessono.

\item[TUTTO Biliemme] Chiamiamo Biliemme quell'ultime contrade della Città
  di Firenze, dove abita questa sorta di gente, la quale veramente, benché
  nata, ed allevata in Firenze, è affatto differente da gli altri Fiorentini ne i costumi,
  e nel parlare; farebbe leggi a suo modo; mangia d' ogni sorta sporcizie,
  come gatti, cani, pesce, e carne fetida; beve ogni sorta di vino sregolatissimamente,
  come afferma il nostro Poeta sotto in \cstan{60}. dicendo: \textit{Gente
  che a bere è peggio delle spugne}. In somma è un Popolo da se, che noi chiamiamo
  gli \textit{Unti}, il \textit{Batti}, o \textit{Biliemme}, la qual voce serve ancora per esprimere la più vil
  plebe, come è nel presente luogo.

\item[GVITTI] Guidoni, plebei, sudici, sporchi, e sordidi. E' parola che ha del
  Napoletano, se bene il Varchi stor. lib. 10. se ne serve anch' egli per esprimere
  un' hvomo d'animo vile, dicendo: \textit{Egli era tanto d'animo guitto, e tanto meschino,
  che usava dire: Chi non va a bottega è ladro}.

\item[HVOMINI di conto] Huomini di stima; huomini riguardevoli. Translato
  forse dal giuoco delle Minchiate, nel qual giuoco si stimano, ed apprezzano
  solamente le carte, che contano, le quali son quelle, che vedremo sotto C. 8. stan.
  61. Si dice \textit{Il tale conta} per intendere; il tale è huomo adoperato, o e buono a
  qualcosa.

\item[BATTILORI] Mercanti d'oro filato. \textit{Banchieri} Mercanti di cambio, che
  si dicono Negozianti. \textit{Setaiuoli} Mercanti di drappi, e di seta, \textit{Lanaiuoli}
  Mercanti di pannine, e Lana. \textit{Orefici} Mercanti d'oro, e d'argeato sodo.
  \textit{Merciai} Coloro, che vendono nastri, seta, telerie, ed altre merci simili. E tutti
  questi suddetti in generale si chiamano Mercanti, o mercatanti.

\item[BRIGATE] Quantità di gente, Vedi sopra C, 1. stan.\ 2.

\item[D'ogni sorta, chiedete, e domandate] Cioè domandate, ed eleggete pure, che
  sorta di gente volete, che la troverete fra costoro; perché vi è d'ogni specie
  di persone.

\end{description}

\section{Stanza X. \& XI.}

\begin{ottave}
\flagverse{10}Sul Colle compartisce questa gente\\
Amostante con tutti gli Vfiziali; \\
Tra' quali un grasso v'è convalescente, \\
c' haveva preso il dì, tre serviziali; \\
E appunto al corpo far' allor si sente \\
L'operazione, e dar dolor bestiali, \\
Tal che gridando senz' alcun conforto \\
In terra si butta come per morto.\\
\end{ottave}

\begin{ottave}
\flagverse{11}Il nome di costui, dice Turpino,\\
Fu Paride Garani, e il legno prese,\\
Perch' ei voleva darne un rivellino\\
A un suo nimico traditor Francese,\\
Che per condurlo a seguitar Calvino\\
Lo tira pe' capelli al suo paese,\\
E per fuggirne a i passi la gabella,\\
Lo bolla, marchia, e tutta lo suggella.
\end{ottave}

Ii Generale Amostante distribuisce sul colle di Malmantile i Soldati, fra i
quali era Paride Garani, che havendo preso un gran vacuatorio sentiva dolori acerbissimi,
e però si rammaricava. Il nostro Poeta per accredirare questa opera,
come fece il Pulci nel suo Morgante, e Ariosto nel Furioso, le da anche
egli il fondamento della storia; allegando l'autorità di Turpino, come fece anche
sopra \cstan[2]{31}. e da quello che scrive Turpino, cava che costui havea
nome Paride Garani, il quale havea preso il legno per dare una quantità di legnate
a un suo nimico Francese, che per condurlo a seguitar Calvino, lo voleva
tirare pe i capelli in Francia, e per risparmiarne la gabella d'haveva già marchiato,
e bollato, e sigillato. E scherzando l'Autore con questi equivoci, vuol
dite che Paride prese il Legno santo per medicarsi del mal Franzese.

\begin{description}
\item[PRESE il legno] Cioè bevve il decotto di Legno Santo per medicare il Mal
Franzese; se ben par che voglia dire, prese un pezzo di legno per bastonare quel
suo nimico Francese.

\item[DARE un rivellino] Dare una quantità di legnate. Rivellino e una specie di
  fortificazione, che si suol fare d'avanti alle porte delle Città, o fra le cortine
  delle Fortezze, così detto forse perché \textit{revellitur a linea}, o perché \textit{revellat
    hostium vim}, e da questa rivolta nelle cortine, o dal quasi rivoltarsi egli al nimico habbiamo
  il presente translato, che ci serve per esprimere, Rivoltarsi a uno con
  gran quantità di bastonate, bravate, riprensioni, ec, E dicendosi assolutamente
  e senz'aggiunta: \textit{Gli fece un rivellino}, s'intende \textit{Gli fece una solenne bravata}, o
  \textit{buona passata}, o \textit{gran rabbuffo}; E dare un rivellino, s'intende dar quantità di percosse.

\item[RIDURLO a seguitar Calvino] Par che voglia dire ridurlo a seguitare la setta
  di Calvino Eretico, e vuol dire, che per farlo divenir calvo, questo suo mal
  Francese lo tira per i capelli, e glieli fa cascare.

\item[LO bolla, marchia, e tutto lo suggella] Fa bullette, marchia, e suggella. E vuol
  dire che questo suo mal Francese gli havea cagionato bolle, croste, e lividi; che
  il verbo suggellare vuol dire Far de i lividi nel viso a uno con le percosse, i quali
  noi chiamiamo Pesche: I Latini in questo senso dissero; \textit{suggillare}. Vedi sotto
  \cstan[6]{54}. metaforico da \textit{suggellare} che vuol dire imprimere in cera, ostia, e
  simili nelle lettere, ec. e si dice anche \textit{sigillare} Dant, Purg. C. 7.
  \begin{verse}
    La sua impronta quand'ella sigilla.
  \end{verse}
  E suggellare Dante Purg. C. 10. \textit{Come figura in cera si suggella}. E Canto 33.
  \textit{Ed io sì come cera da suggello}.
\end{description}

\section{Stanza XII. \& XIII.}
\begin{ottave}
\flagverse{12}Disse Amostante, visto il caso strano, \\
A Noferi di casa Scaccianoce: \\
Per Ser Lion Magin da Ravignano, \\
Ch' il venga a medicar, corri veloce; \\
Io dico lui, perché ce n' è una mano, \\
Ch' infilza le ricette a occhio, e croce,\\
O fa sopr' all' infermo una bottega, \\
E poi il più delle volte lo ripiega.
\end{ottave}

\begin{ottave}
\flagverse{13}Gloria cerca Lion, più che moneta,\\
Però ch' ei bada al giuoco, e fa progresso;\\
Per l' acqua in Pindo andò come Poeta,\\
Ond' agl' infermi dà le pappe a lesso.\\
Gli è quel che attende a predicar dieta\\
E farebbe a mangiar con l' interesso;\\
Ma perché già tu n'hai più d'uno indizio,\\
Va via, perché l'indugio piglia vizio.
\end{ottave}

Amostante veduto lo stravagante accidente, ordinò a Noferi Scaccianoce (che
vuol dir Francesco Cionacci\footnote{Francesco Cionacci, 1633-1714, ``Accademico Apatista''}) che andasse per Ser Lion Magin da Ravignano
(che vuol dire Giovann' Andrea Moniglia\footnote{Giovanni Andrea Moniglia, Firenze 1624 - Prato 1700, medico, autore di teatro, librettista, Accademico della Crusca.}) e facesse venire lui medesimo, che è
un valent'huomo, e non come qualcuno, che non sa dove s'habbia la testa, ed
in vece di medicare un'infermo il più delle volte l'ammazza con le sue spropositate
ricette, ed è di quelli, de i quali si può dire.
\begin{verse}
  His, \& si tenebras palpant, est facta potestas,
  Extenuandi agros, hominesque impune necandi.
\end{verse}

Il che non si può dire di Lione, che procura più d'acquistar gloria che oro.
Egli è Poeta, e però non è maraviglia, se andando egli per l'acqua al fonte di
Parnaso dia poi molte pappe con l'acqua a gli ammalati. L'Autore dice così,
perché in una sua leggieri infermità non volle questo medico, che gli pigliasse
medicamento alcuno, ma lo volle curare con la sola dieta, facendoli mangiar
sera, e mattina pappe; e però dice; \textit{Attende a predicar dieta, E farebbe a mangiar
con l'interesso}; perché veramente in quel tempo Lione essendo giovanotto
sano e robusto, mangiava assai. Questo Lione non era stato nominato dall'Autore
nel primo componimento della presente sua Opera, benché suo amicissimo,
havendo solamente nominato quel medicastro, che dice gli spropositi, che vedremo
poco appresso, ma dopo la suddetta infermità, per vendicarsi graziosamente
dell'haverlo tenuto tanto a dieta ce lo volle mettere. Hor tornando a cammino.
Il Generale dopo haver dato a Noferi molti contrassegni, affinché conoscesse
questo medico, manda a cercarne.

\begin{description}
\item[CE n' è una mano] Ce ne son molti. Termine che vien dal Latino. Verg. 4.
En, \textit{Iuvenum manus emicat ardens}.

\item[INFILZA le ricette a occhio, e croce] Si dice anche a occhio, e voce: Fa le
  ricette senza regola, considerazione, o fondamento. Opera senza scuola, o riprova,
  E' termine meccanico.

\item[FAR una bottega sopra uno infermo] Far allungare il male per cavarne maggior
  guadagno. E questo termine s'usa in qualsivoglia negozio, del quale uno procuri
  di prolungar la spedizione per buscar più denaro.

\item[RIPIEGARE uno] Intendiamo Far morir uno, Vedi sotto \cstan[10]{4}.

\item[BADAR al giuoco] Attender con applicazione a quella professione, che uno
fa, o a quel negozio, che ha fra mano, e si dice anche Badare a bottega. Vedi
sopra \cstan[1]{62}. questo verbo \textit{badare} in altri significati.

\item[PAPPA] Cioè pane bollito nell'acqua; o in altro liquore. E' di quelle parole
  inventate dalle Balie per facilitare il parlare a i bambini, come babbo, mamma,
  e simili. I Latini dissero, \textit{pappare}, e i Greci pure dicevano \textit{Pappa} se bene
  in altro senso, volendo esprimere il Padre, il Babbo, Vedi sopra \cstan[2]{66}.
  E sotto C, 4. stan.\ 5 e 12.

\item[ATTENDE a predicar dieta] Sempre dice che si mangi poco; che questo intende
  per far dieta. Se bene appresso a' Medici \textit{dieta} vuol dire regola di vita universale.
  Dieta si dice congresso di gran personaggi per trattare negozzi gravissimi,
  come si dice Dieta il Congresso de i Priacipi Elettori all' Elezione dell'Imperatore.

\item[FAREBBE a mangiar con l'interesso] Mangerebbe sempre di giorno, e di notte,
  come fanno i cambi, o usure, che mangiano dì, e notte, mentre che il tempo
  fa crescer la somma degl'interessi. L'usura in Ebreo dicesi morso.

\item[L'INDVGIO piglia vizio] L'indugiare, o trattenersi è pericoloso di cagionare
  qualche danno, o far perder la congiuntura di conseguir l'intento. \textit{Mora trahit
  damnum}.
\end{description}
\section{Stanza XIV.}
\begin{ottave}
\flagverse{14}Noferi vanne, e sente dir ch' egli era \\
Con un compagno, entrato in un fattoio, \\
Ov' egli ha per lanterna, essendo sera,\\
L'orinal fitto sopra a un schizzatoio,\\
E di fogli distesa una gran fiera,\\
Ha bell', e ritto quivi il suo scrittoio,\\
Si che presto lo trova, e in su l'entrata\\
Dell unto studio gli fa l'ambasciata.
\end{ottave}

Noferi trova il Medico nel Fattoio da olio, che quivi era il suo studio, e gli
fa l'ambasciata.

\begin{description}
\item[FATTOIO] Quella stanza, dove è la macine per infragnere l'olive, e lo
  strettoio, ed altri ordinghi per cavar l'olio dalle medesime olive. Vien dal Latino
  \textit{Olei factorium}.

\item[ORINALE] Vaso di vetro o d'altra materia, nel quale s'orina, da i Latini
  detto \textit{matula}, \textit{vas urinarium}, e \textit{scaphium}, donde i Sanesi chiamano scafarda,
  o scanfarda quella catinella, che a tale effetto usano le donne.

\item[SCHIZZATOIO] È una grossa canna di stagno, o d' altro metallo, con la
  quale si danno i serviziali agl' infermi. Vedi sotto \cstan[10]{4}.

\item[DISTESA una fiera di fogli]  Sparsa una quantità di fogli. Dice \textit{fiera} per la
  similitudine, che haveva quella distesa di fogli con le \textit{fiere}, o mercati, che alcune
  volte all'anno si fanno in Firenze, nelle quali per le piazze si veggono moltissime,
  e diverse mercanziuole, disegni, leggende, ed altri arnesi confusamente.
  Latino \textit{Nundinae}, abbiamo forse questa voce \textit{fiera} dal Latino \textit{forum}, che era inteso
  per la piazza dove si facevano le fiere o mercati, o pure dal Latino \textit{feriae}.

\item[HA bello, e ritto] Ha con facilità aggiustato il suo scrittoio; che la voce bello,
  in questi termini altro non vuol dire, che Ormai, o di già, e serve per emfasi,
  e per denotare la franchezza in terminare una operazione: Si dice \textit{rizzare una
  bottega}, \textit{rizzare un negozio} per dar principio a un negozio.

\item[VNTO studio] Si chiama studio quella stanza, nella quale uno sta a studiare;
  e perché questo Medico haveva deputata per suo studio la stanza del fattoio, lo
  chiama \textit{studio unto}, perché tali stanze sono, o verisimilmente devono essere unte.
\end{description}
\section{Stanza XV \& XVI.}
\begin{ottave}
\flagverse{15}Ei c'alla cura esser chiamato intende\\
Risponde haver' allora altro che fare, \\
Per c'una sua commedia ivi distende \\
Intitolata il Console di Mare,\\
E che se opra sua colà s'attende,\\
Un buon suggetto quivi suo scolare,\\
Di già sperimentato, ed in sua vece\\
Havria mandato lui; e così fece.
\end{ottave}

\begin{ottave}
\flagverse{16}Era quest'huomo un certo Medicastro,\\
C' al dottorato suo se piover fieno\\
E perch' ei vi patì spese, e disastro,\\
E stato sempre grosso con Galeno;\\
E giunto là: Vo far (disse) un' impiastro,\\
Onde s' il mal venisse da veleno\\
Presto vedremo; in tanto egli si spogli,\\
E siami dato calamaio, e fogli.
\end{ottave}

Sentendo Lione d'esser chiamato a medicare, risponde, che per allora non
può venire, ma che manderà un suo scolare valent'huomo. Costui era un gran
bue, e però giunto dove era l'infermo, cominciò subito con gli spropositi.

\begin{description}
\item[CONSOLE di mare] Questa fu una Commedia intitolata \textit{La Serva nobile}\footnote{La Serva Nobile, Musica: Domenico Anglesi (161? - 1674), Libretto: Giovanni Andrea Moniglia, Prima rappresentazione: Firenze Teatro della Pergola, 1660.}, nella
  quale è introdotto per l'Eroe un Console di Mare in Pisa, onde molti la chiamano
  il \textit{Console di mare}, ancor che il titolo stampato in fronte di essa sia, \textit{La
    Serva nobile}, e fu composta dal medesimo Lione, e recitata in musica con grandi
  Apparati d' ordine del Serenissimo Principe Cardinal Gio: Carlo nel suo bellissimo
  Teatro fabbricato allora di nuovo. Ed il nostro Poeta nella presente ottava
  vuol mostrare la poca applicazione, che Lione haveva in quei tempi alla medicina,
  come giovane, se ben per altro dotto; e che poi voltatosi a tale studio ha
  saputo acquistarsi la fama, che ha acquistato, e meritare una delle prime Cattedre
  dello studio di Pisa, e di servire attualmente al Serenissimo Gran Duca per Medico.

\item[MEDICASTRO] Medico di poca scienza, o (come diremo) salvatico,

\item[FE piover fieno nel suo dottorato] Quando si sente uno, che vaole spacciarsi per
  huomo dotto, e dal parlare si fa conoscer per uno ignorante, si suol dire quando
  ci parla \textit{Tirate giù del fieno} intendendosi: Per dargli a questo bue che parla. Sì
  che dicendo che \textit{nell'addottorarsi costui, piovve fieno}, intende che costui fu conosciuto
  per un solennissimo bue; e però venne gran quantità di fieno senz' esser chiesto,
  che diciamo: \textit{La roba ci piove} per intendere vien roba in abbondanza, senza
  chiederla.

\item[È STATO sempre grosso con Galeno] Esser grosso con uno vuol dire essere in
  collera, o esser adirato con uno; sì che dicendo, che costui \textit{è stato sempre grosso
    con Galeno}, perché l'haveva disastrato, e fatto penare, s'intende era adirato
  seco; e però non lo guardava mai, e conseguentemente non havea pratica con
  Galeno, e non sapeva quel che egli dicesse, sì che in sustanza vuol dire un grandissimo
  ignorante nella Medicina.

\item[VELENO] Questa parola ha due significati: uno proprio che è tossico, e l'altro
  improprio, che è fetore. Il primo è quello, che s' intende nel presente luogo,
  il secondo si vedrà nell'ottava seguente.
\end{description}

\section{STANZA XVII.}
\begin{ottave}
\flagverse{17}Mentre è spogliato, per la pestilenza,\\
Ch' egli esala, si vede ognun fuggire,\\
Pervenne una zaffata a Sua Eccellenza,\\
Che fu per farlo quasi che svenire; \\
Confermata però la sua credenza\\
Rivolto a i circostanti prese a dire:\\
Questo è veleno, e ben di quel profondo,\\
Sentite voi ch' egli avvelena il Mondo?
\end{ottave}

Mentre che Paride si spogliava ognuno per lo gran fetore cominciò a fuggire,
onde il sig.\ Medico, che sente ancor' egli l'orrendo fetore, si confermò nel credere,
che fusse veleno, perché avvelenava.

\begin{description}
\item[PESTILENZA] Intendi fetore grandissimo. E si serve della parola \textit{pestilenza},
  per la parola \textit{veleno} presa in significato di puzzo, o fetore, e per altro
  \textit{Pestilenza} vuol dire mal contagioso.

\item[ZAFFATA] Parte del vapore di quel puzzo, portato dal moto dell' aria.
  E si dice anche \textit{zaffata} d' ogni liquore per intendere \textit{spruzzaglia} d'ogni liquore.
  Franco. Sacc. num. 136. \textit{L'orina gli andò sul Cappuccio, e nel viso, ed alcune zaffate in
  bocca}.

\item[AS. Ecc.] Questo titolo benché non sia così conveniente a' Medici, nondimeno
  è usato dalla nostra plebe in vece dell' Eccellentissimo, e l'Autore lo dà a questo
  medico per derisione.

\item[PROFONDO] Per traslato significa Grandemente, smoderato, o perfettissimo,
  come usavano anche i Latini.

\item[AVVELENA] Rende puzzolente. Ecco la voce \textit{veleno}, ed \textit{avvelenare}
  presa nel secondo senso detto di sopra di \textit{puzzo}, o \textit{fetore}; E l'equivoco, che da
  ciò ne nasce, serve a questo Medico per farsi stimar dotto mostrando conoscere,
  che questo è veramente \textit{veleno}, perché egli avvelena, che vuol dire far putire, ed egli
  lo piglia in significato d'attossicare, e Veleno in significato di tossico, Vedi sotto in
  \cstan{54}. la voce lezzo.
\end{description}
\section{Stanza XVUL}
\begin{ottave}
\flagverse{18}Rispose il general, commosso a sdegno:\\
Come veleno? o corpo di mia vita!\\
E dove è il vostro naso, e il vostro ingegno? \\
Lo vedrebbe il mio bue, ch'egli ha l'uscita.\\
A ciò soggiunse il Medico: Buon segno,\\
Segno che la natura invigorita\\
A' morbi repugnante, adesso questo\\
A nostri nasi manda sì molesto.
\end{ottave}

Il Generale s'adira, e dice: Che non havete odorato da sentir questo puzzo,
ne ingegno da conoscere, che egli ha l'uscita! Al che replica il Medico: Questo
è buon segno, perché la natura havendo preso vigore, come quella, che repugna
a i morbi, espelle ora questo morbo, e lo manda ai nostri nasi. Per intender
bene lo sproposito, che fa dire a questo Medico, è necessario sapere, che
la parola \textit{morbo} ha due significati, il primo è infermità, e dicendo \textit{repugnante a i
morbi} intende all'infermità; ed il secondo è \textit{fetore} o \textit{puzzo}; e dicendo \textit{manda a'
nostri nasi questo morbo} intende Manda questo fetore. Ed il buon medico, che stima
che \textit{natura morbo repugnans} voglia dire repugni al puzzo, cava la conseguenza,
che il sentir questo puzzo sia buon segno, perché la natura scacciando il puzzo,
dal corpo dell'infermo, lo manda a i nasi de' circostanti, e così va scemando
il morbo al pazziente.

\begin{description}
\item[LO vedrebbe il mio bue] Lo vedrebbe uno, che non havesse punto di giudizio.

\item[USCITA] Stemperamento di Corpo, Soccorrenza; da' Latini con voce Greca
  detta \textit{Diarrhoea}.

\item[BVON segno] L'Autore mostra in questa Ottava il modo, col quale soglion
  parlare i Medici ignoranti per accreditarsi appresso agl' idioti, dando ragioni
  spropositate, e inducendo aforismi improprj, pur che lusinghino il pazziente con
  una certa apparenza di sperar bene, come fanno gli Zingani, e i Montambanchi.
\end{description}
\section{Stanza XIX.}
\begin{ottave}
\flagverse{19}Vedendo poi, ch'il flusso raccappella \\
(Come quelle c'ha in zucca poco sale) \\
Comincia a gridar: Guardia, la padella; \\
E (quasi fusse quivi uno spedale) \\
Chiamagli astanti, gl'infermieri appella,\\
Il cerusico chiede, e lo Speziale,\\
E venuto l'inchiostro, al fin si mette\\
A scriver una risma di ricette.
\end{ottave}

L'Eccellentissimo Medico vedendo, che il corpo faceva nuova operazione, cominciò
a chiamar la Guardia, che portasse la padella, pensando che quelle parole
havessero virtù di fermare il flusso, havendole sentite dire negli Spedali in
occasioni simili, e però credendo esser nello Spedale chiamava gli Astanti, ec. e
poi si messe a scriver una gran ricetta.

\begin{description}
\item[RACCAPPELLA] Opera di nuovo. Reitera, Replica. Raccappellare si dice
  quando coloro, che stringono l' olive per cavarne l'olio, o le vinacce per cavarne
  il vino, dopo haver dato qualche stretta, allentano lo strettoio, e nelle
  gabbie mettono nuove olive, o nuova vinaccia sopr'all'altra, che v' era prima.
  Alcuni dicono \textit{rincoppellare}, traendolo dalle coppelle de' purgatori d'oro, nelle
  quali rimettono più volte lo stesso metallo per raffinarlo, il che dicono \textit{rincoppellare}.

\item[HAVER poco sale in zucca] Haver poco cervello, poco giudizio. Bocc.n.2,
g. 4. \textit{Per porre la sua belezza innanzi ad ogn'altra, sì come quella che haveva poco sale
in zucca}. Vedi sopra \cstan[1]{73}. e sotto \cstan[4]{15}.

\item[GVARDIA, la padella] Questo e un detto, che s' usa, quando si sente, che
  altri faccia romore per di sotto per causa dell'uscita del vento, e si dice così, perché
  gl' infermi, che sono negli spedali, quand' hanno bisogno di votare il ventre,
  chiamano colui, che è di guardia, che porti la \textit{padella}, che è un vaso di rame,
  ec, il quale è adattato in maniera da potersi mettere, in caso di bisogno, nel
  letto sotto all' infermo, acciò che possa fare il fatto suo, senza muoversi dal
  letto.

\item[STANTi] o \textit{Astanti}, Son coloro, che assistono al servizio degl'infermi, come
  vedemmo sopra \cstan[1]{48}. Lat. \textit{adsantes}.

\item[INFERMIERE] Chiamano negli spedali \textit{Infermiere} colui, il quale invigila
  che gl' infermi sieno messi a letto, quando son condotti allo spedale, e gli piglia
  nota per fargli visitare dal Medico, e gli registra al libro degli entrati, e de gli
  usciti, ed al libro de' morti.

\item[CERVSICO] Quello che medica le ferite, piaghe, ed altri mali esterni, che
  richieggono opera manuale, e cava sangue, ec, detto ancora con voce Greca
  usata da' Latini \textit{Chirurgo}.

\item[LISMA] o \textit{risma}, Diciamo un fagotto, o balletta di carta, che sarà di circa
  500.\ fogli. Dal Gr. \textit{arithmos}. Qui però è detto iperbolico, e per mostrare, che
  questo Medico scrivesse assai, non che veramente consumasse una Lisma di carta.
\end{description}

\section{Stanza XX.}

\begin{ottave}
\flagverse{20}Dove diceva (dopo millioni \\
Di scropoli, di dramme, e libbre tante) \\
Che già, che questo mal par che cagioni\\
Stemperamento forte, umor piccante,\\
Per temperarlo; Recipe in bocconi \\
Colla, gomma, mel, chiara, e diagrante, \\
Quindici libbre in una volta sola \\
Di sangue se gli tragga dalla gola;
\end{ottave}

\begin{ottave}
\flagverse{21}Accio che tiri per canal diverso \\
L'umor che tende al centro, \textit{ut omne grave}\\
Che se durasse troppo a far tal verso \\
Dir potrebbe l'infermo: Addio fave. \\
Poi tengasi due dì capo riverso \\
Legato per i piedi a una trave;\\
Se questo non facesse giovamento, \\
Composto gli faremo un'argomento.
\end{ottave}

\begin{ottave}
\flagverse{22}Peré presto bollir farere a sodo\\
Un'agnello, o capretto in un pignatto;\\
N' un' altro vaso nelle stesso modo\\
Un lupo per infin che sia disfatto;\\
Poi fare un servizial col primo brodo,\\
E col secondo un' altro ne sia fatto;\\
Farà questa ricetta operazzione\\
Senz' alcun dubbio, ed eccola ragione
\end{ottave}

\begin{ottave}
\flagverse{23}Questi animali essendo per natura\\
Nimici, come i ladri del Bargello,\\
Ritrovandosi quivi per per ventura,\\
Il lupo correra dietro all'agnello;\\
L'agnello, che del lupo havrà paura\\
Ritirandosi andra per il budello;\\
Così va in su la roba, e si rassoda,\\
E i due contrarj fan, ch'il terzo goda.
\end{ottave}

In queste sue ricette mostra l'Eccellentissimo Medico la sua goffaggine con
proporre farmachi, e rimedj spropositati, come è quello de i due brodi di lupo,
e d'agnello, e quello del tenere il pazziente appiccato al palco per li piedi col
capo all'ingiù.

\begin{description}
\item[MILLIONE] È un numero determinato di dieci centinaia di migliaia, ma qui
è preso per indeterminato; come succede spesso, che per esprimer, grandissima
quantità di cose, si dice E' un millione delle tali cose, ancor che sieno molte meno,
ed alle volte molte più. Così i Latini in questo senso \textit{sexcenta}, \textit{centum milia},
e Greci \textit{myria}, cioè diecimila.

\item[STEMPERAMENTO forte] Stemperare vuol dir Ammollire, o liquefare, e
  nel ventre di costui era sollevamento d'umori, e stemperamento di materie forti,
  cioè acide, e di umori piccanti. Gli epiteti di forte, e piccante son epiteti
  convenienti al vino, dicendosi vino forte quello, che comincia a diventare aceto
  ed in molti luoghi d'Italia si dice Vin forte, il vino gagliardo, o grande; e vino
  piccante quello che in beverlo fa frizzare le labbra, e la lingua. Questo Eccellentissimo
  Medico però intende quel \textit{forte} per acido, e per grande, e gagliardo;
  E piccante dal verbo \textit{piccare}, che vuol dir Pugnere, Offendere, che si dice anche
  \textit{dar nel naso}. Vedi sotto \cstan[7]{59}. l'Eccellentissino cava l'argumento, che
  questi umori sieno piccanti, perché danno nel naso col loro fetore: Ora per rassodare,
  e coagulare tal stemperamento vuole il prelibato Medico, che si dia al
  pazziente a bere gran quantità di \textit{colla, miele, gommma, chiara d'uovo, e diagrante},
  le cose nella somma, e quantità, che egli pone se s'incorporassero, in
  grandissima quantità d'acqua e sarebbono atte a coagulare, e seccare un lago;
  e se vi havesse aggiunto gesso, e matton pesto havrebbe dato una ricerta da stoppare
  quante rotture si possano mai trovare ne i vivai.

\item[DIAGRANTE] Specie di gomma, o colla, che serve per incollare i drappi
  ne i rovesci de i ricami, o per altre cose simili.

\item[SE li tragga 15. libbre di sangue per la gola] E cavandosi 15. libbre di sangue
  dalla vena della gola del pazziente; e legandolo per i piedi al palco col capo
  all' ingiù (che questo vuol dir caporiverso) pretende il Medico, che la roba sia
  per mutar viaggio, se vorrà condursi al suo centro, che non è più nel luogo,
  dove era prima, ma stante la positura del corpo è diventato suo centro il capo.

\item[CONTINOVASSE a far tal verso] Continovasse a fare nella medesima forma,
  o maniera, Vedi sotto \cstan[7]{1}.

\item[ADDIO fave] Significa Noi siamo spacciati; Noi siam finiti; Siam morti, Fu
  un Villano nel contado d' Imola d' ingegno più tosto grosso che no, il quale haveva
  un bellissimo campo di fave, e nel mezzo di esso era un gran ciriegio carico
  di ciriege. A tal Ciriegio haveva il villano fatta una fortissima prunata, perché
  le ciriege gli fussero colte; e vantandosi di questa sua diligenza, fu sentito
  da un Cieco suo amico, il quale gli disse: Con tutti li tuoi pruni io vi salirò, e
  se non lo faccio, voglio perdere dodici lire, ch' io mi ritrovo, ed il villano replicò:
  Se tu non pigli la scala, o vero non porti il forcone, o altro per levare
  i pruni io voglio giuocare questo campo di fave, e che tu non vi sali. il Cieco
  si contentò, e così convennero. L'astuto Cieco si coperse tutta la vita con buone
  pelli di bue, e così armato passando per mezzo de i pruni senza sentir puntura,
  alcuna, salì sopra il ciriegio. Il villano, veduto questo, tardi accortosi della sua
  balordaggine, piangendo il suo danno gridava: \textit{Addio fave}, cioè io ho perduto le
  fave, Vedi il Cornazzano\footnote{Antonio Cornazzano (Piacenza, 1430 circa – Ferrara, tra il 1483 e il 1484) scrittore e poeta} Novella 10. dove troverai questa favola non travestita,
  e meglio espressa.

\item[TRAVE] Legno grosso, e lungo, che s'adatta a reggere i palchi.

\item[ARGOMENTO] E' lo stesso, che Serviziale, o Cristero detto sopra in questo
  C. stan. 10. e 12. E qui torna bene, perché vuol medicarlo per via d' argumenti
  logici, ma di conseguenze spropositate.

\item[BOLLIRNE a sodo] Cioè bollire molto tempo, e gagliardamente.

\item[BRODO] Decotto di carne. Acqua ingrassata con carne. Se ben la parola
  brodo è comune a ogni sorta di decotto, o minestra, ancorché non di carne.

\item[I DVE contrarj fan che il terzo goda] \textit{Inter duos litigantes tertius gaudet}. Con
  questo argumento, e con questa sen\-ten\-za, e con altre ragioni da squartati, pretende
  l'Eccel\-lentissimo d' haver trovato il modo di fermare il flusso.

\end{description}

\section{Stanza XXIV \& XXV.}

\begin{ottave}
\flagverse{24}Ciò detto rivoltessi al mormorio\\
Di quell'ambrette, ov' a mestar si pose; \\
E, perch' elle sapevan di stantio,\\
Teneva al naso un mazzolin di rose. \\
Soggiunse poi: Costui vuol dirci addio, \\
Che queste flemme putride, e viscose \\
Mostran, che ben' affetto a gli artolani \\
Ei vuol' ire a ingrassare i Petronciani.
\end{ottave}

\begin{ottave}
\flagverse{25}In quel che questo capo d'assivolo\\
Ne dice ogni or dell'altra una più bella,\\
Tosello Gianni, il quale è un buon figliuolo\\
Mosso a pietà, con una sua coltella\\
Tagliate havea le rame d'un querciuolo,\\
Sopr' alle quali a foggia di barella\\
Fu Paride da certi Contadini\\
Portato a' suoi poder quivi vicini.
\end{ottave}

L'Eccellentissimo Dottore, dopo haver fatte le suddette belle ordinazioni, si
mette a stuzzicare quella materia, e da quel puzzo fa pronostico, che il pazziente
sia per morire; e l'argumento, che egli fa di tal morte non è dissimile dalle
ricette. In canto Tosello Gianni accomodò una barella, sopr' alla quale Paride
fu posto, e portato da certi contadini ad una villetta de' Signori Parigi vicina a
Malmantile in luogo detto Santo Romolo; nella qual Villa trovandosi l'Autore
concepì nella mente il far la presente Opera, come dicemmo sopra nel Proemio.

\begin{description}
\item[AMBRETTA] Così chiamiamo guanti, ed altre pelli conciate con odore
d'ambra. Ma qui intende, ironicamente parlando, quella materia fetida.

\item[SAPEVA di stantio] Haveva cattivo odore. Quando una materia per la lunghezza
  del tempo ha cominciato a perdere la sua perfezione, si dice \textit{stantia}; che
  se sia carne, o pesce, non da troppo buono odore; e queste si dice \textit{puzzo di stantio},
  La qual voce viene da stanziare lungo tempo, ed è il Latino \textit{obsoletus}. Vedi
  sotto in \cstan{54}.

\item[VUOL dirci addio] Se ne vuol'andare. Ci vuol lasciare, cioè vuol morire.

\item[FLEMMA] Vmor freddo, e umido che i Medici chiamano in Pituita, e comunemente
  si dice hemma dal Greco.

\item[VUOL' andare a ingrassare i Petronciani] Vuol andare a ingrassare gli orti col
  suo corpo, facendoli sotterrare; e piglia \textit{Petronciani} (che vedemmo in
  \cstan{6}, quello che sieno) per tutto l'orto. E nota che per autenticare la
  castroneria di questo Medico, l'Autore gli fa dedurre il pronostico della morte
  di Paride dal credere, che il suo corpo sia già corrotto, e ridottosi tutto in quel
  la terza putrida sustanza, ed in conseguenza atto, ed il caso a ingrassare i terreni;
  E vuol dire, che Paride morrà: Dicendosi vulgarmente per intender questo
  \textit{Il tale andò a ingrassare i cavoli}, cioè il tale morì.

\item[CAPO d'assiuolo] A uno ignorante si dice Capo di Bue, Capo di Castrone,
  Capo d'assivolo, e simili, L' \textit{assiuolo} è un'uccello in tutto simile alla Civetta, se
  non che ha sopra il capo, alcune penne ritte, che sembrano corna.

\item[TOSELLO Gianni] Agostino Nelli Gentil' huomo Fiorentino buon letterato,
  e veramente huomo da bene, Che intendiamo \textit{buon figliuolo}.

\item[COLTELLA] Specie di scimitarra, Arme che s'usa portare, quando si va a  caccia.

\item[BARELLA] Arnese fatto di tavole, che ha quattro manichi, serve per portar
  sassi, e altri pesi in due persone; qui intende una barella da portare i corpi
  d'huomini infermi, o morti, che è simile alle bare, o cataletti co i quali si soglion
  portare detti corpi, e da Bara e chiamata barella. Vedi sotto in \cstan{44}.

\end{description}

\section{Stanza XXVI.}

\begin{ottave}
\flagverse{26}Fu del Garani ascritto successore \\
Puccio Lamoni anch'ei grand'ingegnere, \\
Bravissimo Guerrier saggio Dottore, \\
Cortigiano, Mercante, e Taverniere, \\
Dicon ch' ei nacque al tempo delle more,\\
Per ch'egli è di pel bruno, e membra nere,\\
Hor qua di Cartagena eletto Duce,\\
Il fior de' Mammaganuccoli conduce.
\end{ottave}

Al Garani fu dato per successore Puccio Lamoni, il quale è Paolo Minucci.
Il Poeta dice che costui era ingegnere, e Mercante; ma tali attributi gli sono finti,
perché io posso giurare, che egli non sa ne dell'una, ne dell'altra professione.
Lo chiama guerriero, e questo perché detto Puccio fece una campagna
nell'esercito Pollacco in Prussia, seguitando quella Real Corte, alla quale era
stato inviato dal Serenissimo Principe Mattias di Toscana alla Maestà del Re Gio:
Casimiro. E perché detto Puccio godé per molti anni, e fino che S.A. visse,
l'honore di servire all' A.S. in qualità di Segretario, però dice che era Cortigiano.
Dice che è Dottore perché veramente egli e addottorato in Legge, se bene
per l'applicazione alla corte, non esercitò tale professione. Lo chiama Taverniere,
perché spesso lo vedeva entrare nell'Osterie, e trattare con Osti, il che
seguiva perché egli vendeva loro del vino raccolto nei suoi beni, e gli conveniva
lasciarsi rivedere spesso per risquoterne il prezzo. Dice che si vocifera, che
gli nascesse al tempo delle more, Perch'egli è di pel bruno, e membra nere, essendo
egli così in effetto: E facendolo Duca di Cartagena dice, che egli conduce \textit{il
fiore de' Mammagnuccoli}, cioè i migliori, e più valorosi \textit{Mammagnuccoli}. Questi
Mammagnuccoli erano una conversazione di galant' huomini, i quali facevano
professione di sapere il conto loro in ogni cosa, e particolarmente nel giuocare,
e spendere bene il lor danaro, e d'essere il fiore della reale, ed onorata
scapigliatura. Havevano un loro capo, che si chiamava Abate, dal quale erano
galtigati, quando facevano qualche errore o nel giuocare, o nello spendere, ma
però tutto era in galanteria. Le loro adunanze si facevano in casa l'Abate, dove
si giuocava a giuochi più di spasso, che di vizio, e si facevano altre allegrie,
di cene, merende, ed altri passatempi. Costoro erano tutte persone serie, e
quiete, e della più riguardevole Civiltà, e perciò era la lor conversazione molto
bramata, onde era numerosissima; Se bene non era ammesso a quella veruno, che
non havesse provata prima la sua dabbenaggine, e non fusse stato riconosciuto
dal Abate, e da altri suoi Consiglieri meritevole d'essere ammesso. Fra costoro
era detto Puccio, e perché egli era forse de' più affezionati, i1 Poeta lo fa loro
Condottiero, e per la stima che faceva di lui nel giuoco delle Minchiate, era solito
chiamarlo il Re delle carte; perciò lo fa Duca di Cartagena, ed è ancora appropriato,
perché detto Puccio per esser di faccia bruna, ha qualche sembianza,
ed aria di Spagnuolo; oltre che nel tempo, che l'Autore lo aggiunse a questa sua
Opera, il detto Puccio, era stato destinato dalla Maestà del Re Gio: Casimiro
per suo Segretario dell' Ambasciata di Spagna.

\section{Stanza XXVII.}

\begin{ottave}
\flagverse{27}L' Armata havea tra gli altri un Cappellano\\
Dottor, ma il suo saper fu buccia buccia,\\
Pero ch' egli studiò col fiasco in mano,\\
Ed era più buffon d' una Bertuccia,\\
Faceva da Pittor, da Tiziano;\\
Ma quant'ei fece mai n'andava a gruccia,\\
Hebbe una Chiesa, e quivi a bisca aperta\\
Si giuocò fino i soldi dell'offerta.
\end{ottave}

\begin{ottave}
\flagverse{28}Franconia si domanda Ingannavini, \\
E fu pregato come il più valente, \\
Perch' egli sapea leggere i Latini, \\
A far quattro parole a quella gente, \\
Egli c' havea in casa it Coltellini,\\
Già fatta una lezione, e falla a mente, \\
Subito accetta, e siede in alto solio\\
Senza mettervi su ne sal, ne olio.
\end{ottave}

Fra gli altri Cappellani, che erano nell' Armata, era un Dottore, ma di poca
scienza; perché il suo studiare era stato il darsi bel tempo. Fu scolare dell'Autore
nella pittura, ma imparò poco, e se bene si presumeva di saper molto, non
fece mai cosa, che non fusse stroppiata. Fu Rettore della Chiesa di Petriolo;
Villaggio vicino a Firenze circa due miglia, e perché egli era huomo allegro, e
di conversazione, dice che egli si giuocò fino i soldi dell'offerta, ed intende che consomava
tutte le sue entrate in allegrie. Il suo nome era Franconio Ingannavini,
cioè Giovannantonio Francini. A questo dunque, come al più dotto fu fatta instanza,
che facesse un poco di discorso a quei Soldati, ed'egli che haveva un
tempo fa recitata una lezione nell'Accademia del Coltellini, e l'haveva ancora
a memoria, si contentò di fare quanto gli era stato imposto, e senza mettere più
tempo in mezzo montò in pulpito.
\begin{description}
\item[BUCCIA buccia] Leggiermente. Cioè sapeva poco; non haveva gran fondamento;
  che si dice anche \textit{in pelle in pelle}. Vedi sotto C, 8, stan.58. ed i Latini
  dissero \textit{superficie tenus}.

\item[PIÙ buffone d' una bertuccia] Huomo arguto, allegro, e faceto. \textit{Buffone} diciamo
  colui, che tiene il popolo allegramente con facezie, e moti, è il Latino
  \textit{Scurra}, Vedi sotto \cstan[11]{42}. E \textit{Bertuccia} diciamo la scimmia.

\item[TIZIANO] Pittore celeberrimo. E con dire \textit{facea da Tiziano}; intende per
  antonomasia, che egli si presumeva d'esser il più valente Pittore del Mondo.

\item[QVANT' ei facea, n'andava a gruccia] Tutto quel che egli faceva, era stroppiato,
  cioè mal fatto, mal dipinto, Vedi sotto C.~11. stan.~41.

\item[BISCA] Luogo pubblico, dove è permesso giuocare a ognuno; \textit{E giuocare a
  bisca aperta}, vuol dire Giuocar sempre, e senza riguardo alcuno.

\item[IL Coltellini] Questo è il Signor Agostino Coltellini\footnote{Agostino Coltellini (Firenze, 17 aprile 1613 – Firenze, 26 agosto 1693), accademico e letterato. } Avvocato Fiorentino huomo
  dotto, ed amatore de i Letterati, il quale in molte opere composte da lui si
  chiama col nome anagrammatico Ostilio Contalgeni. In casa di esso si raguna
  l'Accademia degli Apatisti da esso fondata, nella quale si fanno discorsi Accademici,
  ed altri esercizzj virtuosi. Mirabile per haver saputo far durare per lo spazio
  di cinquanta, e più anni la detta Accademia, sempre in florido, cosa insolita
  a' nostri secoli in questa Città. Interveniva spesso in detta Accademia questo
  Francini, ed alle volte vi faceva qualche lezione; nelle quali mostrò i suoi dotti
  ed eruditi talenti, e se bene l'Autore dice che il suo sapere fu \textit{buccia buccia}, e sotto
  lo chiama huomo senza fondamento, non è però, che egli fusse tale, anzi fra
  gli huomini de' nostri tempi non era dei secondi in dottrina non meno sagra,
  che profana; ed era veramente Dottore di legge.

\item[SENZA mettervi su ne sal, ne olio] Presto, subito, senza replicare, o metter
  difficultà, \textit{Nulla interposita mora}. Fu un tale, che tornato la sera a casa, disse al
  suo servitore: Fammi una insalata, e fa presto, ch' io sono aspettato, e non
  voglio mangiare altro che quella; fa presto. dico. Il servitore presa l'insalata
  senza condire la portò in tavola al padrone; il quale ciò visto lo scridò; Ma il
  servitore rispose; Signore per servirvi presto, non vi ho messo su ne sale, ne olio.
  E da questa goffaggine del servitore viene il presente detto, che significa Fare una
  cosa subito, e senza considerazione.
\end{description}

\section{Stanza XXIX.}
\begin{ottave}
\flagverse{29}Sale in Bigoncia con due torce a vento, \\
Acciò lo vegga ognun pro tribunali, \\
Ove, mostrar volendo il suo talento, \\
Fece un discorso, e fece cose tali, \\
Che ben si scorse in lui quel fondamento,\\
Che diede alla sua casa Giorgio Scali,\\
E piacque sì, che tutti di concordia\\
Si messero a gridar: misericordia,
\end{ottave}

Il Poeta continuando, a voler mostrare, che Franconio fusse di poco valore,
e che però il discorso da lui fatto fusse scimunito, e senza alcun fondamento, lo
burla, e dice che piacque tanto, che il popolo, si messe a gridar \textit{misericordia}; del
qual termine ci serviamo per mostrare, che qualche cosa ci sia venuta a fastidio,
come per esempio. \textit{Ei durò tanto a discorrer, che misericordia}, \textit{Disse tante
  scioccherie, che misericordia}, \textit{Oh misericordia, quanto volete voi durare?} Quali dica,
habbiate misericordia, e compassione di noi, e non ci tediate più,

\begin{description}
\item[BIGONCIA] È un vaso di legno, del quale si servono i Contadini in tempo
  di vendemmia per pigiarvi dentro l'uva, prima di metterla nel tino, e ce ne serviamo
  anche in altre occorrenze, come di portar' acque, e simili.

Il Bini nel Capitolo del Pilo\footnote{Capitolo 29 dal libro ``Le terze rime de messer Giovanni della Casa, di messer Bino, ed altri'', pubblicato per Curtio Navo, senza imprimatur, nel 1532.} dice:
\begin{verse}
  Vuo dir, che se ben' ella il pil mi desse,
  Ed oprassi (non ch'altro) una bigoncia,
  Ognun direbbe, che ben fatto havesse.
\end{verse}
 E perché questo vaso detto Bigoncia è molto simile a una cattedra tonda, però
da molti tal Cattedra si chiama \textit{bigoncia}, come anche tutte l'altre cattedre. Il
Davanzati\footnote{Bernardo Davanzati, 1529 - 1606.} nel suo Cornelio Tacito postille al 2. libro num. 18. dice: \textit{Arringavano
i nostri antichi al popolo in piazza, in ringhiera, e nei Consigli in bigoncia, che
era un pergamo in terra a foggia di bigoncia}.

\item[TORCE a vento] Torce grosse che si fanno di funi di cotone filato attorte per
servirsene a far lume la notte per le strade; e si dicono \textit{a vento}, perché resistono
al vento; e a distinzione di quelle, che si fanno a Venezia, che per esser gentili
si spengono a ogni poco di vento. E \textit{Torcia}, che da i Latini e detta \textit{funalia},
\textit{funalium}, viene a noi dal Francese \textit{Torche}.

\item[CHE diede alla sua casa Giorgo Scali] Giorgio Scali\footnote{Giorgio degli Scali, uomo politico associato alla Rivolta dei Ciompi, 1378, fu arrestato il 16 gennaio 1382 e giustiziato il giorno seguente.} fu in Firenze un riputatissimo
  Cittadino Popolano, il quale nelle dissenzioni, che seguirono a suo tempo
  fra i nobili, e Popolani di Firenze, si fece capo di questa parte, con promessa, e
  speranza d' esser sollevato a cose maggiori, cioè all' assoluto dominio di Firenze,
  e benché per altro accortissimo, e prudentissimo, lasciatosi portare dal dolce desiderio
  di dominare, si fidò nelle vane promesse della instabil plebe, con la quale
  parendogli haver forze bastanti per conseguire l'intento, s' accinse all' opera;
  ma nel più bello il popolo, o spaventato, o pentito l'abbandonò, ond' egli
  venuto in potere del Governo fu decapitato: E da lui e detto il Proverbio: \textit{Far
  come Giorgio Scali}, che vuol dir Pigliare a far' una cosa senza fondamento, che i
  Latini con similitudine della Scrittura, dissero \textit{Scipione arundineo inniti}, Di questo
  caso di Giorgio Scali parlano tutti gli Storici, che scriveno le cose di
  Firenze di quei tempi, ed il Nerli fra gli altri aggiunge, che allora cominciò questo
  proverbio.
\end{description}

\section{Stanza XXX \& XXXI.}

\begin{ottave}
\flagverse{30}Il tema fu di questa sua lezione, \\
Quand' Enea già fuor del suo pollaio \\
Faceva andar in fregola Didone, \\
Com' una gatta bigia di Gennaio;\\
E che se i Greci ascosi in quel ronzone \\
In Troia fuoco diedero al pagliaio, \\
E in man a Enea posero il lembuccio, \\
Ond' ei fuggi col padre a cavalluccio;
\end{ottave}

\begin{ottave}
\flagverse{31}Così, dicea, la vostra, e mia Regina\\
Qui viva, e sana, e della buona voglia,\\
Cacciata fu dal empia concubina\\
Tre dita anch' ella fuor di questa soglia;\\
Però s'un tanto ardire, e tal rapina\\
Parvi, e' adesso gastigar  si voglia,\\
V'havete il modo senza ch'io lo dica.\\
Io ha finito. Il Ciel vi benedica.
\end{ottave}

Il tema del discorso, che fece Franconio, fu quando Enea esseno fuggito da
Troia fece innamorar Didone, 'ed assomigliando Celidora cacciata di Malmantile
ad Enea scappato da Troia, esorta quei soldati a gastigar l'ardire di Bertinella,
e rimettere Celidora nel suo stato, già che hanno il modo.

\begin{description}
\item[POLLAIO] Si dice da noi quella stanza, nella quale stanno, e dormono i polli:
  E chiamiamo pollaio quelle selve, o macchie, dove la sera vanno gli uccelli
  a dormire; Ma qui intende per translato la nostra Casa, Patria, o luogo, dove
  siamo soliti abitare.

\item[ANDARE in fregola] Dicemmo quel che significhi sopra-\cstan[1]{25}, Ma
  che Didone fusse innamorata d' Enea, come favoleggia Vergilio, è falsità, perché
  oltre che Didone fu così casta, che vedendosi violentata da Iarba Re di Mauritania
  a rimaritarsi seco, volle più tosto da se stessa uccidersi, che offendere il
  suo morto marito Sicheo con nuovi sponsali'; È anche vero, che non potette
  seguire il detto innamoramento, perché Enea fu 360. anni prima di Didone; Tal
  verità si cava da diversi Autori, e si scorge in Darete Frigio', e Ditti Cretense,
  che scrissero la vera Storia dell'eccidio di Troia. Che il nostro Dante poi seguiti
  questa bugia di Vergilio, dicendo nell' Inf. C. 5.
  \begin{verse}
    L' altr' è colei, che s'ancise amorosa,
    E roppe fede al cener di Sicheo.
  \end{verse}
  Non è meraviglia, perché Dante s'era eletto per suo Maestro, e guida Vergilio.
  Che Enea fusse tanto tempo avanti a Didone, si deduce anche dal sapersi, che
  Didone fuggendo l'insidie di Pigmalione suo fratello, che per desiderio di tesoro
  le haveva ammazzato il marito Sicheo, come pure accenna Dante, Purg. C. 20.
  \begin{verse}
    Noi ripetiam Pigmalione allotta,
    Cui traditore, e ladro, e patricida
    Fece la voglia sua dell oro ghiotta.
  \end{verse}
  Portandosene il tesoro in Affrica, chiese a quegli abitatori tanto di terreno
  quanto poteva circondare una pelle di toro, e l' ottenne; Ed astutamente tagliò la
  detta pelle in strisce così sottili, che abbracciò con esse tanto terreno, che vi
  edificò Cartagine, il che fu dopo 70. anni della edificazione di Roma, \textit{la quale fu
    edificata circa 300, anni dopo la morte d' Enea}, Sant' Agostino disse in difesa di
  Didone, che quando Vergilio non fusse stato dannato per altro, meritava l'Inferno
  per questa falsità cotanto pregiudiciale alla ripucazione di Didone, la quale
  difende ancora Ausonio col seguente Epigramma tradotto dal Greco.

  {\centering Ad Didus Imaginem CXI.\\}
  \begin{verse}
\backspace Illa ego sum Dido, vultu quam consipicis hospes,
Assimilata modis pulcraque mirificis:
\backspace Talis eram, sed non Maro quam mihi finxit erat mens,
Vita nec incestis Laeta cupidinibus.
\backspace Namque nec AEneas vidit me Troius unquam,
 Nec Lybiam advenit Classibus Iliacis;
\backspace Sed furias fugiens, atque arma procacis Iarbae
Servavi, fateor, morte pudicitiam
\backspace Pectore transfixo, castos quod pertulit enses,
Non furor, aut laso crudus amore dolor.
\backspace Sic cecidisse iuvat; Vixi sine vulnere fama;
Vita virum, positis moenibus oppetii.
\backspace Invida cur in me stimulasti musa Maronem,
Fingeret ut nostrae damna pudicitia ?
\backspace Vos magis Historicis lettores credite de me,
 Quam qui furta Deum concubitusque canunt;
\backspace Falsidici Vates, temerant qui carmine verum,
 Humanisque Deos assimilant virijs,
\end{verse}

\item[GATTA bigia] È quella, che noi chiamiamo Soriana, che è un misto di color
  bigio, e lionato serpato di nero, qual colore soriano si dice solamente di Gatti,
  onde io argumento, che i primi gatti di questo colore venissero a noi di Soria,
  come vennero alcuni anni addietro quelli del colore del topo portati da Pietro
  della Valle dalla Persia, e però da molti chiamati Persianini. Vedi sotto C. 9.
  stan. 19.

\item[RONZONE] Con la \letter{z} cruda\footnote{sorda - \t{ts}} vuol dir Cavallo stallone, o per la monta, da
  i Latini detto \textit{equus admissarius}; e per ronzone, ronzine, o rozza intendiamo
  cavallo cattivo, Ronzone con la \letter{z} dolce\footnote{sonora - \t{dz}} vuol dire una specie di Moscone, o
  tafano. Qui l'Autore intende quel cavallo di legno fabbricato da i Greci per ingannare
  i Troiani come dice Vergilio. In alcuni Testi si trova scritto \textit{cassone} invece
  di \textit{ronzone}, ma nel mio, che è di mano dell'Autore, è scritto \textit{ronzone}.

\item[PAGLIAIO] È proprio quel cumulo, o massa di paglia, che si fa da i Contadini
  dopo haver battuto il grano, per lo più avanti alle case; ma dicendosi dar
  fuoco al pagliaio, s' intende Dar fuoco alla Casa.

\item[PORRE il lembo]; o il lembuccio \textit{in mano}, Significa Mandar via uno; E questo,
  perché quand' altri vuol mandar via uno di qualche luogo senza parlare, gli fa
  il ferraiuolo addosso, e gli mette un lembo di esso (che \textit{lembo} vuol dire
  Una parte dell'estremità del ferraiuolo, o d'altro abito, o veste simile) nelle
  mani; e da questo colui s'accorge d' esser licenziato, essendo notissimo, che
  questo detto \textit{Pigliare, o dare il lembo} significa Esser licenziato; Tratto dai maestri
  delle botteghe, i quali, volendo licenziare un garzone, gli dicono: piglia il
  lembo; piglia il cencio, ec. e intendono Vattene.

\item[A CAVALLUCCIO] Cioè in su le spalle. E noi diciamo portare \textit{a cavalluccio}
  da un giuoco, che fanno i nostri ragazzi in questa forma. Vno mette il capo fra
  le gambe all' altro per di dietro, e sollevatolo così da terra, lo porta fra le spalle,
  e il collo, e per questo si dice, \textit{a cavalluccio}. I ragazzi Greci, che pure lo
  facevano lo dicevano \textit{in cotyla}, perché facevano porre le ginocchia del portato
  sopr' alle palme delle mani del portatore rivoltate dietro alle reni, ed il portato
  non accavalciava le gambe al collo, come fanno i nostri, ma con le braccia s'atteneva
  al collo del portatore; e lo dicevano \textit{in cotyla} dalla palma, o cavo della
  mano di colui, che portava, come si cava dal Buleng.\ de lud.\ vet.\ cap.\ 20.\ e
  da Cel.\ Rodig.\ lect.\ antiq.\ \libcap[27]{27}. E questo era più tosto, che giuoco,
  una pena data a quei fanciulli, che haveano perso a qualche altro dei loro
  giochi, che habbiamo accennati sopra nel 2.\ Cantare. E si come erano varj i
  modi, con li quali portavano, così erano diversi i nomi, che davano a questo
  giuoco; perché si trova chiamato \textit{Cubesinda}, ed \textit{Hippas}, si come si vede in Giulio
  Polluce lib. 9. c. 7. Che questo giuoco fusse usato anche dai Latini, si può dedurre
  da Vergilio En.\ lib.\ 2.\ il quale dice che Enea portò il Vecchio Anchise suo
  padre in su le spalle in tal maniera.
  \begin{verse}
    Ergo age chare pater cervici imponere nostrae
    Ipse sibibo humeris, nec melabor iste gravabit.
\end{verse}

\item[DELLA buona voglia] Intendiamo sano, allegro, e con buona speranza. Il
  Lalli En. Trau. lib. 1, stan. 51. disse:
  \begin{verse}
    Stanne, diletta mia, di buona voglia.
  \end{verse}

Parafrasando Vergilio, dove dice: \textit{Parce metu}, E noi diremmo: Non dubitare.

\item[FUOR di questa soglia] Cioè fuori di Malmantile, Piglia la soglia, che è la
  parte di sotto della porta, per tutto Malmantile; o intende soglia per soglio
  reale.
\end{description}

\section{Stanza XXXII \& XXXIII.}

\begin{ottave}
\flagverse{32}Poiché da esso inanimite furo\\
Le schiere, si portaron a i lor posti,\\
E già sdraiato ognun lasso, e maturo\\
In grembo al sonno gli occhi haveva posti,\\
Quand'a un tratto le trombe, ed il tamburo \\
Roppe i riposi, e i sonni appena imposti;\\
Ma svanì presto così gran fracasso, \\
Ch'il fiato a i trombettier scappò da basso.
\end{ottave}

\begin{ottave}
\flagverse{33}E questo cagionò, che incollorito\\
Il Generale di cotanta fretta,\\
Con occhi torvi minacciò col dito,\\
Mostrando voler farne aspra vendetta\\
Seguì c'un' Ufizial suo favorito,\\
Che più d'ogn'altre meno se l'aspetta\\
Toccò la corda con i suoi intermedi\\
De' tamburini, e trombettieri a piedi
\end{ottave}

Dopo che Franconio hebbe dato animo a i soldati ognuno andò a quartiere,
e già tutti stracchi s'erano addormentati, quando in un subito fu dato nelle
trombe, e ne i tamburi, che fecero svegliare tutta la soldatesca; ma questo romore
presto cessò, perché i trombettieri, e tamburini lasciarono star di sonar per la
paura, che hebbero del Generale, il qualee entrato in collera di così gran fretta
giurò di voler gastigar colui, che era stato il capo di al sollevamento, e lo mandó
ad effetto, facendo dare la corda a uno Vfiziale suo favorito, che non se lo
sarebbe mai aspettato, e gli fece mettere i tamburini, e i trombettieri a piedi.
\begin{description}
\item[SDRAIATO] Disteso con comodità. Voce usata da noi per esprimere la
  consolazione, che sente uno, che sia stanco a distendersi con comodità e spensieratamente.
  Vedi sotto C. 6. stan.26. E non crederei d'errare, se dicessi \textit{sdraiato}
di Cerbero, parafrasando Vergilio; dove dice
\begin{verse}
  \makebox[4em]{\dotfill} Atque immania terga resoluit
  fusus humi, totoque ingens extenditur antro.
\end{verse}

\item[A VN tratto] In un subito. E questo termine \textit{a un tratto} significa anche tutti
  due, o più alla volta, e si può intender, che le trombe, ed i tamburi, cioè uno,
  e gli altri svegliassero.

\item[CASCÒ il fiato da basso a' trombettieri] Cascare il fiato vuol dire Haver paura,
  o timore; onde con questo dire intende, che i trombettieri hebbero paura del
  Generale, e perciò lasciarono di sonare; non perché veramente perdessero, o
  uscisse loro il fiato dalle parti da basso.

\item[INCOLLORITO] Adirato. Entrato in collora.

\item[OCCHIO torvo] Frase latina; usata da noi, e significa, e mostra l'ira che
  uno habbia; e dicendosi: il tale mi guarda con mal' occhio, o con occhi torti,
  s'intende il tale è adirato meco: \textit{Haec autem toruitas a taurorum ferocia dicitur}.

\item[MINACCIÒ col dito], Coloro che vogliono gastigare qualche delitto, o vendicarsi
  d' alcuna ingiuria, sogliono brandire il dito indice verso quel tale, che
  vogliono gastigare, e tal brandimento si dice \textit{minacciare} dal Latino Minari, o
  \textit{minitari}.

\item[CHE più d' ogni altro meno se l'aspetta] Per esser questo soldato amico, e molto
  in grazia al Generale; non havrebbe mai creduto, che egli l'hauesse a gastigare,

\item[TOCCÒ la corda] In Firenze danno la corda legando il paziente per le mani
  legate insieme dietro alle reni; e per quelle appiccate a un grosso canapo, che
  passa per una carrucola, tirano il paziente in su, lasciandolo leggiermente scorrer
  in giù, e poi ritirandolo in su tante volte, a quante è condennato, e questo diciamo:
  \textit{dare tratti di corda}. Qual tormento da i nostri antichi era detto \textit{dar la
    colla}, o \textit{collare}, e noi diciamo: \textit{dare la corda}. Soggiunge poi: \textit{Co' suoi intermedj di
    tamburini, e trombettieri a' piedi}; cioè con tutto quello che ci andava; il che era,
  che i tamburini, e i trombettieri, i quali erano stati complici a tal delitto, stessero
  quivi a pié di lui assistenti a vedere eseguire la giustizia, come si costuma,
  quando molti sono complici d' un delitto, per lo quale vien gastigato severamente
  il capo principale, e gli altri complici ricevono minor gastigo, ed assstono a
  vedere il gastigo del loro principale. Io però non sono lontano dal credere, che
  il Poeta per sostenere questa sua Opera sempre in su le burle, habbia voluto
  intendere, che i tamburini, e trombettieri fussero effettivamente legati a i piedi di
  colui, che era tirato su, e voglia mostrare con questo il costume, che si tiene in
  Firenze di legare a' piedi di tali pazienti qualche cosa, che significhi il delitto da
  lui commesso, acciò che il popolo comprenda la cagione di quel martirio, come
  per esempio: a un fornaio, che habbia fatto il pane cattivo, o di minor peso del
  dovuto, faranno legare a' piedi un filo di pane, e così gli daranno la corda: e mi
  lascio indurre a creder, che il Poeta habbia voluro intender questo, dal vedere,
  che egli nell'Ottava seguente dice: \textit{alla corda vuole che sia attaccato così}: i qual
  detto pare che esprima, che il paziente debba toccare la fune co'i trombetti, e
  tamburini legatigli a i piedi.
\end{description}
\section{Stanza XXXIV \& XXXV.}

\begin{ottave}
\flagverse{34}Alla corda così vuol che s' attacchi,\\
Perché d'arbitrio, e senza consigliarsi,\\
Facea venir all'armi, allor che stracchi\\
Bisogno havean più di riposarsi,\\
Ed eran mezzi morti; e come bracchi,\\
Givano ansando inordinati e sparsi,\\
E con un fuor di lingue, e orrenda vista\\
Soffiavan, ch'io ho stoppato un Alchimista.
\end{ottave}

\begin{ottave}
\flagverse{35}Amostante non solo era sdegnato,\\
Che di suo capo, e propria cortesia\\
Senza lasciar, che l'huom riabbia il fiato,\\
Ei volesse attaccar la batteria;\\
Ma perché seco havea concertato,\\
Ch'egli stesso, che sa d'astrologia,\\
Vuol prima, ch'il nimico si tambussi,\\
Veder ch'in Gielo sien benigni influssi.
\end{ottave}

ll Generale fece dar la corda a quell' Vifiziale non solo, perché egli s'era
preso l'arbitrio di far dar' all' armi senza il suo consenso ma ancora perché era
uscito fuori del concertato, il quale era di osservare prima di muovere il campo,
se le stelle presagivano buona, o trista sorte. E qui il lettore si ricordi, che si
sta in su le burle, e sappia, che l'Autore non stimava che l'astrologia arrivasse
a tanta precognizione, ma si bene, che Habeant sua sydera lites, come dicono i
legisti.
\begin{description}
\item[D'ARBITRIO, e propria cortesia] Suonano lo stesso; ed ambedue significano
Di suo capriccio, o volontà.

\item[ANSARE] È quell'impeto, o romore, che fa il respiro, quando si ripiglia
il fiato (che noi pure dal Latino diciamo \textit{anelare}) e viene a \textit{Ansima} Gr. \textit{Asthma}.

\item[BRACCO] Cane per uso di caccia, il quale quando è stracco respira con gran
veemenza, e tiene la lingua fuori; E se bene fanno così tutte le specie di cani, è
nostro solito far questa comparazione solamente ai bracchi, perché questi veramente
sono più sottoposti a straccarsi;  percio che stimolati dal naturale desiderio
di trovar preda, fanno maggiore, e più violento viaggio che gli altri cani.
Persio Sat, 1. \textit{Nec linguae quantum sitiat canis Appula tantum}.

\item[ORRENDA vista] Vista spaventevole; che tale è il veder un'huomo con la
bocca aperta, e con la lingua fuori, perché per lo più restano in questa forma
gl' impiccati.

\item[SOFFIAVAN ch'io ho stoppato un' Alchimista] Alchimisti son coloro, che soffiano
nel fuoco per trovar l'oro, e senza nominare Alchimista, col solo dire: \textit{il tale
soffia} s'intende, è Alchimista, Se bene s' intende anche Fa la spia, come
accennammo sopra C 1, stan. 37. anzi dicendosi \textit{Il tal fa l'Alchimista}, s'intende il
tale fa la spia, e tutto è fondato sul verbo soffiare, che significa \textit{Far la spia}.

\item[IO ho stoppato] Significa io stimo meno, o io non stimo punto il soffiare, che
fanno gli Alchimisti in paragone di quello, che soffiavano questi soldati. Ha lo
stesso significato, che il termine ne disgrado detto sopra \cstan[1]{51}. e che vedremo sotto \cstan[6]{61}.

\item[TAMBVSSARE] Perquotere, dar delle fusse. È parola oggi propria de i
macellari, che dicono Tambussare quando bastonano le bestie morte e gonfiate,
perché la pelle si spicchi bene dalla carne, e dicono anche Tamburare, come vedremo
sotto C.\ 11.\ stan\ 26. E tutto ha Origine dal tamburo, perché il romore,
che fa esso, s'assomiglia al romore, che fanno i macellari.
\end{description}

\section{Stanza XXXVI. — XXXIX.}

\begin{ottave}
\flagverse{36}Homai la Fama, che riporta a volo\\
D'ogn' intorno nuove, e le gazzette,\\
Sparge per Malmantil, che armato stuolo\\
Vien per tagliare a tutti le calzette,\\
Già molti impauriti, e in preda al duolo\\
Non più co i nastri legan le scarpette,\\
Ma con buone, e saldissime minuge,\\
Perché stien forti ad un \textit{rumores fuge}.
\end{ottave}

\begin{ottave}
\flagverse{37}In tal confusione, in quel vilume,\\
All' udir quei lamenti, e quegli affanni\\
A molti ch' eran già dentr' alle piume\\
Lo sbucar fuori parve allor mill anni:\\
Chi per vestirsi riaccende il lume,\\
Però ch' al buio non ritrova i panni,\\
Chi nudo scappa fuori, e non fa stima,\\
Che dietro gli sia fatto lima lima.
\end{ottave}

\begin{ottave}
\flagverse{38}Perché s'egli ha camicia, o brache, o vesta,\\
Non bada che gli facciano il baccano;\\
Ben si del triste avviso afflitto resta,\\
Onde più d' un poi giuoca di lontano,\\
Chi torna indietro a fasciarli la testa,\\
E chi si tinge con il zafferano,\\
Chi dice, c' una doglia sergli è presa,\\
Per non haver a ire a far difesa.
\end{ottave}

\begin{ottave}
\flagverse{39}Altri, che fugge anch' ei simil burrasca,\\
Finge l' infermo, e vanne allo spedale,\\
E benché sano ei sia come una lasca,\\
Col medico s' intende, e col speziale,\\
Perché all'uno, ed all'altro empie la tasca,\\
Acciò gli faccia fede ch' egli ha male;\\
Ed essi questo, e quel scrivon malato,\\
E chi più da, do fan di già spacciato.
\end{ottave}

Sparso per Malmantile l'avviso dell'arrivo di detta Soldatesca, gli abitatori
di quel luogo s' accinsero più al fuggire, che al difendersi. Narra il Poeta diversi
effetti di tale spavento, e le varie scuse, ed invenzioni, che trovano coloro
per non haver ad andare alla difesa della muraglia.

\begin{description}
\item[GAZZETTE] Novelle, Avvisi, Carte d'avvisi. E \textit{gazzetta} diciamo anche
  la crazia, Veneziana.

\item[TAGLIAR le calzette] Tagliar le gambe. E s' intende, dare delle ferite in
  qualsisia luogo del corpo, se ben le calzette non vestono se non le gambe: Come
  diciamo anche rompere la testa, ed intendiamo Ferire il nimico in quelle parti
  del corpo che ci verrà fatto. E diciamo \textit{fiaccar le braccia a uno con le bastonate}, se
  bene in ogni altra parte gli daremo che nelle braccia.

\item[NASTRO] È una specie di tela, o benda che non eccede la larghezza d' un
  sesto di braccio, e serve per legare, o fasciare;  da i Latini però detto, \textit{Vitta}, ed
  in alcuni luoghi d'Italia detto fettuccia.

\item[MINVGE] Corde da strumenti musicali come Tiorbe, Liuti, ec. fatte di budella
  di bestie; e pero Dante Inf. c. 28. per intender budella disse.
  \begin{verse}
    Tra le gambe pendevan le minugia.
  \end{verse}

  Dice che non si sono legate le scarpe coi nastri, ma con le minuge, perché sono
  più sode, e da resister più; ed è costume usatissimo il dire: \textit{Il tale s'era legato
  le scarpe bene, o con le minuge}, per intendere Correva forte, o volava: fuggendo
  i pericoli, che ciò intende con quella sentenza, \textit{Rumores fuge}.

\item[CONFUSIONE, e vilume] Sono in questo luogo quasi sinonimi havendo lo
  stesso significato di Viluppo, imbroglio, ec.

\item[DENTRO alle piume] Cioè nel letto.

\item[FAR lima lima] Beffare, dileggiare. è un modo proprio da Fasciulli, i quali
  quando vogliono dar la burla a uno, si fregano il dito indice sopra l'indice
  dell' altra mano a guisa di coloro che limano, e voltandosi verso colui, che voglion
  burlare dicono. \textit{Lima, lima}. Vedi sotto \cstan[9]{66}. annot.

\item[NON bada] Non cura; Non osserva, Non gl' importa'; Il verbo \textit{badare}, che
  vuol dire osservare, ha più significati, come Attendere, continovare, usare diligenza,
  curare, stimare, ec. Bada a tuoi negozzi. Bada a andare, Bada a chi
  viene. In somma ha la forza del Latino \textit{Curare}, \textit{Vacare}: si dice: \textit{Tener uno a
    bada}, per intender Trattenerlo. Star a bada d'uno: per intendere Stare aspettando
  l' opera, i favori ec. d' uno.

\item[BRACHE] Calzoni. Brache da noi propriamente si dicono quei calzoni larghi,
  che usano i Soldati a piede Tedeschi guardie del Serenissimo Gran Duca, ed
  i Paggi nobili. E si dicono talvolta Brache quei calzoni che si portano di sotto,
  chiamati ancora Mutande; Vedi sotto C.6, stan. 20.

\item[FAR il baccano] Qui vuol dir beffare, dileggiare con fischiate, o strida, o simili;
  ed il suo significato proprio è Fare strepito, far romore e viene da Bacchanalia.

\item[GIUOCA di lontano] Cioè non s'accosta: ed è lo stesso che \textit{starsene alla larga},
  che vedremo nell'ottava seguente.

\item[BVRRASCA] S'intende propriamente il travaglio del mare; ma lo pigliano
  per ogni sorta di sturbamento, o pericolo. Forse meglio borrasca da \textit{Boreas}.

\item[SPEZIALE] Colui che manipola, e vende medicamenti; e però da i Latini
  detto \textit{Pharmacopola}; ed altrimenti \textit{Aromatarius} da \textit{aromata}, e noi lo diciamo
  \textit{Speziale} da spezierie, come si trova anche in Latino Barbaro \textit{speciarius}.

\item[TASCA] Scarsella, che è un sacchetto appiccato a i calzoni, o altre vesti per
  uso di tenervi dentro quello, che occorra alla giornata, e particolarmente danari;
  è il Latino \textit{marsupium}. Ed \textit{empier le tasche a uno}, vuol dire Dargli molto
  danaro.

\item[LO fanno spacciato] Cioè dicono, che egli è in grado di morte. Intende il Poeta,
  che i Medici regolando le attestazioni delle infermità con le somme de i danari,
  che erano lor date, facevano fede esser in grado di morte quello, che più
  ne dava; e quel che ne dava pochi attestavano, che era leggiermente infermo.
\end{description}

\section{Stanza XXXX.}
\begin{ottave}
\flagverse{40}Sì che con queste finte, e con quest' arte \\
Costor, c'usan la tazza, e non la targa, \\
Servir volendo a Bacco, e non a Marte, \\
Che non fa sangue, ma vuol che si sparga, \\
D' uno stesso voler la maggior parte\\
Trovan la via di sparsene alla larga,\\
Ed il restante non sì astuto, e scaltro\\
Comparisce, perch' ei non può far altro.
\end{ottave}

Questi abitanti di Malmantile con tali scuse, ed invenzioni cercano di sottrarsi
dall' andare alla guerra, e solo vi va chi non ha danari, ne invenzioni da liberarsene.
\begin{description}
\item[TARGA] Brocchiero, Scudo, Rotella. Intende, che son più avvezai a bere
  che a guerreggiare, ed hanno più genio con Bacco \textit{Re} del vino, che non
  hanno con Marte \textit{Re} delle guerre; perché quello fa nascere nel corpo il sangue,
  e questo lo fa disperdere.

\item[STARSENE alla larga] Significa non s impacciare d' una cosa, ed è lo stesso
che giuocar di lontano, che vedermmo nell'ottava antecedente.

\item[ASTVTO, e scaltro] Sinonimi di sagace, ed accorto, Huomo, che fa il conto
  suo. Ma per maggior intelligenza di queste parole \textit{Astuto}, e \textit{scaltro}, \textit{sagace}, ed
  \textit{accorto} è da sapere che, se bene ce ne serviamo per sinonimi, tuttavia ci è
  qualche differenza; particolarmente fra \textit{sagace}, ed \textit{astuto}; perché l'arti, che dalla
  sagacità s' adoprano, non meritano biafimo, per non esser se non avvedimenti
  sottili, ma schietti, reali, e senza fraude, o inganni: E l' astuzia oltre alle
  suddette lodevoli arti si serve anche delle menzogne, fraudi, e falsità, e d' altre
  cose indegne d'animo nobile. E però \textit{Scaltro, ed accorto} pare che meglio s' adattino
  per sinonimi a \textit{sagace}, che ad \textit{astuto}, al quale più proprio sinonimo sarebbe
  Malizioso, o tristo, o furbo; quando però la voce furbo e presa in senso d'huomo,
  che sa il conto suo; Ma, come ho detto, nel comun parlar civile non
  usiamo così esatta diligenza, e puntualità; ma pigliamo l'uno per l'altro.
\end{description}\
\section{Stanza XXXXI \& XXXXII.}

\begin{ottave}
\flagverse{41}Mentr' in piazza si fa nobil comparsa, \\
Anch' in Palazzo armata la Regina\\
Con una treccia avvolta, e l'altra sparsa,\\
Corre alla Malmantilica rovina; \\
Benché ne i passi poi vada più scarsa,\\
Perché all' uscio da via mai s' avvicina; \\
Da sette volte in su già s'è condotta \\
Fino alla soglia; ma quel sasso scotta.
\end{ottave}

\begin{ottave}
\flagverse{42}Viltà l'arretra, honor di poi l'invita\\
A cimentar la sua bravura in guerra,\\
L'esorta l'una a conservar la vita,\\
L' altro a difender quanto può la Terra.\\
Pur fatto conto di morir vestita\\
Voltossi a bere, e divenuta sgherra\\
(Però che Bacco ogni timor dilegua)\\
Dice: O de'miei: chi mi vuol ben mi segua.
\end{ottave}

Mentre che la men codarda gente si raguna in piazza, anche la Regina Bertinella
al romore, nuova Semiramide con i capelli non ancora finiti d' aggiustare,
corre a difender Malmantile; ma non con tanto ardire, perché questa nostra
Semiramide non s' arrischiò così subito a passare la porta della Casa; ma si fermò
in quella, sospesa, e travagliata da due gran passioni Poltroneria, ed Honore,
che quella l'esorta a starsene; e questo, obbliga ad andare, Al fine lasciatasi
persuadere dall' Honore prese animo, ed esortò i suoi a seguirla.

\begin{description}
\item[TRECCIA] I capelli delle donne si chiamano \textit{trecce}, perché per lo più sogliono
  le donne far due parti de i lor capelli, e ciascuna di quelle suddividere in tre
  altre parti, ed intesserle in terzo, il che si dice \textit{treccia}; E Bertinella stava così
  Intrecciandole, quando sentì il romore, per lo che lasciato il lavoro corse con
  una parte intrecciata, e l'altra no, come dicono, che facesse Semiramide, quando
  senti il pericolo, che sovrastava a Babillonia.

\item[MA la soglia scotta] Quando uno o per debiti, o per delitti sta ritirato in casa,
  o in Chiesa, diciamo: \textit{Non esce, perché la soglia scotra}; cioè se egli uscisse di
  casa, o di Chiesa, sarebbe fatto prigione: ed a Bertinella \textit{scotta quella soglia},
  perché se uscisse di quella, pericolerebbe di toccarne.

\item[VILTA] Qui vale per poltroneria, o codardia.

\item[MORIR vestito] S'intende di coloro, che sono ammazzati, i quali muoiono
  con le vesti in dosso, e però dicendo che fa conto di morir vestita, s' intende che
  ella ha risoluto d' andar a farsi ammazzare.

\item[SGHERRA] Brava, Animosa; fatta così dal vino, che leva di testa ogni timore.
  Bacco da i Latini fu detto \textit{Liber}, perché libera l'huomo da i pensieri noiosi,
  e però dice ogni pensier dilegua, ed il Chiabrera\footnote{Gabriello Chiabrera (Savona, 18 giugno 1552 – Savona, 14 ottobre 1638) poeta e drammaturgo, quasi moderno Pindaro. } disse.
  \begin{verse}
    Beviamo, e diansi al vento
    I torbidi pensieri.
  \end{verse}

  Seneca de Tranquillit. disse; \textit{Nonnunquam ad ebrietatem veniendum, non ut mergat
    nos, sed ut deprimat curas \& elevat enim curas, \& ab imo animum movet, \& ut
    morbis quibusdam, ita trisitiae medetur}, Di questa regola si servsempre il Galasso
  Generale dell'Imperadore Ferdinando 2., il quale mai si portò ad alcuno consiglio
  di guerra, ne si messe ad impresa alcuna importante, se prima non aveva
  molto bevuto. E Bertinella imita questo gran guerriero.
\end{description}
\section{Stanza XXXXIII \& XXXXIV.}

\begin{ottave}
\flagverse{43}Dietro a suoi passi mettesi in cammino\\
Maria Ciliegia illustre damigella; \\
Tutto lieto la segue il Ballerino,\\
Che canta il titutrendo falalella. \\
Va Meo col paggio, Zoppica Masino,\\
Corre il Masselli, e il Capitan Santella.\\
Molti, e molt'altri amici la seguiro,\\
E più Mercanti c'hanno havuto il giro.
\end{ottave}

\begin{ottave}
\flagverse{44}La segue Piaccianteo suo servo, ed Aio,\\
C'in gola tutto quanto il suo si caccia,\\
Le cacchiatelle mangia col cucchiaio,\\
Ed è la distruzion della vernaccia.\\
Già misurò le doppie con lo staio,\\
Finito poi che fu quella bonanaccia,\\
Portò per il contagio la barella,\\
Ed hora in Corte serve a Bertinella.
\end{ottave}

Alle voci, ed ordini di Bertinella obbedirono diversi suoi seguaci Birboni, e
Matti.

\begin{description}
\item[MARIA Ciliegia] Fu una Donna creduta pazza, la quale andava per Firenze
ricevendo elemosina senza domandarla, Costei con una flemma, e gravità non
ordinaria discorrendo sempre da per se, diceva belle, e sensate sentenze; la onde
da molti non era stimata pazza, ma uguale a Diogene, che abitava nella botte;
e per tale azione farebbe stato riputato matto, se non havesse lasciato così
belle sentenze, e dogmi, come appunto fece questa madonna Maria, i detti della
quale, o parte di essi sono stati raccolti da un buon letterato, che forse una
volta gli darà alle stampe: Come Diogene, anch' essa non si curava di casa, ma
dormiva nelle strade sotto qualche portico o loggia, e perciò portava seco sempre
un granatino per spazzare quel luogo, dove si metteva a dormire, ed una
spazzola per spazzolarsi la veste, la quale benché poverissima, era nondimeno
molto pulita, e se bene piena di toppe, assai bella per esservi le medesime toppe
messe forse anche senza bisogno, con vago, ed aggiustato ordine. Nella suddetta
sua sporta haveva ancora qualche biancheria, e molte volte un laveggio, o caldanetto
pieno di fuoco, nel quale, passeggiando per le strade, andava quocendo
le sue vivande; sotto la gonnella haveva più sacchetti, entro dei quali riponeva la
pentola, e piatti per suo uso, e quello che le avanzava a' suoi mangiari. Haveva
sorelle, e nipoti i quali si trattavano comodamente, ed habitavano in una
buona casotta, che era di detta madonna Maria, dove ella alle volte andava per
mutarsi; ma non volle mai fermarvisi, ne dormirvi ancor che pregata, e forzata
anche da' detti suoi parenti a volere star con loro. Buscava molti denari con
li quali comprava quello, che parcamente le bisognava, ed ogni Sabato sera dava
per l' amor di Dio tutto quello, che le avanzava, e per lo più a povere Monache,
dove alle volte portò anche fino a dieci scudi, Domandata da alcuno di
qualche parere, non rispondeva; ma seguitando il suo solito chiacchierare, prima,
che quel tale is partisse da lei restava appagato con qualche sentenza, o motto,
che ella diceva a proposito del quesito. Per esempio: Una mattina, sendo
ella  sotto le logge d' avanti al Tempio della Santissima Annonziata, un giovane
netto le domandò, se ella credeva, che la sua moglie bella, da madonna Maria
molto ben conosciuta, fusse honesta; ma glielo disse con la più sporca maniera,
che dir si potesse. Madonna Maria senza alzar la testa, o dar segno d' attenzione
al quesito del giovane, seguitando il suo discorso, che faceva del poco rispetto,
si portava alle Chiese; dopo molte chiacchiere disse; Vedete voi questo
giovane sboccato, il poco rispetto, ch' ei porta alla Chiesa? La sua moglie è
bella, e la prese, che ella era onesta; ma che può ella havere imparato da lui, se
non il modo di diventare altrimenti? ed hora io ho, che ella sia diventata; perché
ogni geloso è becco. E seguitò il suo cicaleccio, entrando in diversi altri gineprai,
come era solita; e così chiacchierando tutto il giorno dalla mattina alla
sera, buscava molti denari. Costei morì, e si trovò nella sua sporta una borsetta,
nella quale era una ricevuta di cinquanta scudi, dati a certe Monache con
obbligo di far dire una messa il mese all' altare della Santis. Nunziata per l'anima
sua; dal che si cava argumento che ella non fusse pazza.

\item[FALALELLA] Così e chiamato un contadino tristo, il quale non havendo voglia
  di lavorare, s'è dato a chiedere elemosina; e per far venire le donnicciuole
  alle finestre, e cavar loro di mano robe, e danari, va per le strade cantando alcune
  sue ottave amorose, e ad ogni due versi fa intercalare con la voce dicendo
  \textit{Falarera tututrendo}, con che si persuade d' imitar il suono del Chitarrino; ed
  all'ultimo dell' Ottave, al medesimo suono della voce, si mette a ballare, e per questo
  il Poeta lo chiama il Ballerino; e poi va attorno chiedendo la limosina.

\item[MEO] Era uno scemo di cervello provvisionato dal Palazzo; e perché egli
  son si reggeva bene in piedi, pero andava sempre appoggiato a un ragazzo; e
  perciò dice: \textit{Va Meo col paggio}.

\item[MASINO] Era uno stroppiato nelle gambe, e nelle braccia; il quale era anch' egli
  provvisionato dal Palazzo per quella sua figura cotanto contraffatta da
  gli stroppi.

\item[MASSELLI] Era un matto, o creduto tale, provvisionato pure dal Palazzo.
  Costui haveva in mente tutte le feste del anno, e quali Ofizzj, e commemorazioni
  dovean farsi da i Preti giorno per giorno. Sapeva in oltre, quali erano quei
  Rettori, e Curati di Chiese, tanto in Firenze, che nel Contado, i quali nelle
  feste trattavano bene, o male ai loro desinari; e da essi si lasciava in tali giorni
  rivedere; e mangiava, e beveva tanto, che è impossibile a crederlo anche da chi
  l'ha più volte veduto. Era soprannaturale nel digerire, e s' è veduto smaltire
  gran quantità di roba, si può dire impossibile, come sarebbe un gran piatto di
  carta straccia bollita in brodo di bue, e condita a guisa di maccheroni; altre volte
  bisso, e tela d' olanda nella stessa forma, e questo in breve tempo, e senza
  difficultà, o dolori. Il Poeta dice; \textit{Corre il Masselli}, perché veramente costui,
  benché decrepito, era di gamba velocissima. Haveva il Sereniss. Gran Duca dato
  per servitore al Masselli un giovanotto gagliardo, perché lo seguitasse per tutto
  dove egli andava, e osservasse tutte le sue azioni, senza mai contradirgli, o
  impedirlo, ed ogni sera riportasse quanto il Masselli haveva fatto in quel giorno.
  Quando il Masselli riceveva alcun disgusto da costui, non s' alterava seco, mas
  si metteva la via fra gambe, e senza mai fermarsi, o voltarsi ne meno a dietro,
  non la guardava a camminare di buonissimo passo 25., o trenta miglia con grandissimo
  travaglio, e rabbia del servitore, che non poteva, ne doveva distorlo, e
  conveniva, che lo seguitasse; onde andava molto cauto in strapazzarlo (come
  sul principio del suo servire faceva fino a bastonarlo) non tanto per paura del
  gastigo da S. A. S. minacciatogli, quanto per il timore, che il Masselli per vendetta
  non viaggiasse.

\item[CAPITAN Santella] \makebox[3pt]{} Questo fu un soldato della Banda di Pistoia, il quale dette
  la volta al cervello (o così finse) perché gli fu rubata la moglie da chi ne poteva
  più di lui. Costui venne in Firenze, e vi dimorò qualche tempo, facendo diverse
  pazzie; ma perché fu conosciuto, che sotto questa sua finta pazzia si nascondeva
  una gran tristizia, fu mandato forzatamente in Candia al servizio de' SS.
  Veneziani, donde non è più tornato.

\item[MERCANTI, c'hanno havuto il giro] Cioè gente impazzata. Si serve della
  parola Giro per intendere il girare del cervello, che vuol dire Impazzare, non
  per il Giro de' Mercanti, che si dice, quando un Banchiere tiene in mano il denaro
  di tutta la Piazza; il che in Firenze tocca a fare una volta per uno a tutti li
  banchieri, o negozianti più grossi per tanti mesi; il che è fatto per comodità de'
  mercanti; e dicesi: avere il Banco giro.

\item[PIACCIANTEO] Fu un Fiorentino di così vili natali, che non si sa trovare
  la casata, ne il vero nome suo, essendo sempre stato inteso col solo soprannome
  di Piaccianteo. Costui dalli parenti suoi fu lasciato assai comodo, ma come quello,
  che era dedito alla crapula, consumò in breve tempo tutto lo stato suo, ed
  a pena haveva dato principio a provare  le miserie della poverta, e gli stenti, che
  la Fortuna di nuovo lo sollevò facendoli redare da un suo congiunto una somma
  considerabile di doppie; e però il Poeta dice: \textit{Già misurò le doppie con lo staio}. A
  queste ancora il buon Piaccianteo diede presto fine, pensando d' haver ad avverare
  il sentenzioso proverbio, che dice; \textit{A uno scialacquatore non mancaron mai
    denari}. Ma s' ingannò, perché ridotto in estrema poverta, e non sapendo far
  mestiero alcuno, si ridusse a portare quella barella, con la quale si portavano gli
  ammorbati al Lazzeretto nel tempo, che fu la Peste in Firenze, e fin che durò
  tal contagio campò di cotesta sua fatica; finita poi la peste viveva di quel che
  buscava con far servizj alle meretrici; e però il Poeta lo fa servitore di Bertinella,
  e suo Aio, e direttore. \textit{Piaccianteo} voce che ha dell' antico \textit{Piacentiero}.

\item[MANGIAR le cacchiatelle col cucchiaio] Iperbole usatissima per intendere un
  gran mangiatore, Cacchiatella, E' una specie di pane finissimo fatto alla foggia
  ed alla grandezza d' una pera bugiarda; onde con questa iperbole, intendiamo
  che pigli in bocca in una volta tante di queste cacchiatelle, quante piglierebbe
  delle fragole, o piselli, o altra cosa simile, e così viene a essere iperbole doppia,
  perché il cucchiaio comune è capace a fatica d' una sola cacchiatella, e la bocca
  dell' huomo difficilmente riceve una sola cacchiatella per volta: e però intendi,
  che mangiava le cacchiatelle in grandissima quantità, e senza numerarle, come
  non si numerano le fragole, ec, che si pigliano col cucchiaio.

\item[È LA distruzione della Vernaccia] \makebox[3pt]{} È gran bevitore. Vernaccia è una specie di vino
  bianco, ma l'Autore per Vernaccia intende ogni sorta di vino.

\item[MISURÒ le doppie con lo staio] Haveva gran denari. Iperbole usata per intender
  un gran ricco; e ci viene dal Latino \textit{Modio pecuniam metitur}.

\item[BONACCIA] Significa placidezza di mare; ma noi la pigliamo anche per
  sorta di bene stare, e di buona fortuna, come e intesa a presente luogo.

\item[BARELLA] Specie di veicolo simile alla bara, o feretro, col quale si portano
  a sotterrare; ma questa che serviva per pertare gli ammorbati era
  coperta sopra con cerchiate, e tela incerata a foggia di calsa tonda di sopra, come
  i tamburi da viaggio.
\end{description}
\section{Stanza XXXXV — L}

\begin{ottave}
\flagverse{45}Comanda la padrona ch' egli scenda,\\
E stia già fuori con gli orecchi attenti\\
Fra quelle schiere, fin ch'ei non intenda\\
A che fine son là cotante genti;\\
Ma quegli, al qual non piace tal faccenda,\\
Se la trimpella, e passa ai complimenti,\\
E, perché a' fichi il corpo serbar vuole,\\
Prorompe in queste, o simili parole.
\end{ottave}

\begin{ottave}
\flagverse{46}Alta Regina, perché d'Obbedire\\
Più d'ogni altro a' tuoi cenni mi dò vanto,\\
Colà n'andro, ma (come si suol dire)\\
Come la serpe, quando và all'incanto;\\
Non ch'io fugga il pericol di morire,\\
Perch'io fo buon per una volta tanto;\\
Ma perché, s'io mi parto, non ti resta\\
Un huom, che sappia, dov'egli ha la testa.
\end{ottave}

\begin{ottave}
\flagverse{47}Non ti sdegnar, s'io dico il mio pensiero,\\
Che possibil non è ch'io taccia o finga,\\
E, se n'andasse il collo, sempr'il vero\\
Son per dirti, e chi l'ha per mal, si cinga.\\
Ti servirò di cor vero, e sincero\\
Senz'interesse d'un puntal di stringa,\\
E non come in tua Corte sono alcuni\\
Adulator, che fanno Meo Raguni.
\end{ottave}

\begin{ottave}
\flagverse{48}Io dunque che non voglio esser de' loro,\\
Ma tengo l'adular pessimo vizio,\\
Soggiungo, e dico, per ridurla a oro,\\
Che mal distribuito è questo ufizio,\\
E che non può passar con tuo decoro;\\
Poiché mostrando non haver giudizio,\\
Un tuo Aio ne mandi a far la spia\\
Quasi d'huomin tu havessi carestia.
\end{ottave}

\begin{ottave}
\flagverse{49}Manda manda a spiar qualche Arfasatto,\\
O un di quei, che piscian nel Cortile,\\
Questo farà il mestier, come va fatto\\
Senza sospetto dar nel Campo ostile:\\
Ostile dico, mentre costa in fatto,\\
Che cinto ha d'armi tutto Malmantile,\\
Tal gente si puo dire a noi contraria,\\
Perché non vien quassù per pigliar' aria.
\end{ottave}

\begin{ottave}
\flagverse{50}E perch' ei non vorrebbe uscir del covo\\
Soggiunge dopo queste altre ragioni;\\
Ma quella, che conosce il pel nell'uovo,\\
S'accorge ben, che son tutte invenzioni;\\
Però senza più dirglielo di nuovo\\
Lo manda fuori a furia di spintoni,\\
E, mentr'ei pur volea imbrogliar la Spagna\\
Gli fa l'uscio serrar su le calcagna.
\end{ottave}

Bertinella vuol mandar Piaccianteo nel Campo di Baldone a spiare; ma egli,
che non vorrebbe andare, adduce mille scuse; quali non gli sono ammesse, ed è
cacciato fuori di Malmantile a furia di spinte.

\begin{description}
\item[TRIMPELLARE] Intendiamo quel suonare adagio, e tentoni la chitarra,
  liuto, o altro strumento simile, che fanno coloro, che imparano a suonare: e
  da questo per \textit{trimpellare}, o \textit{trimpellarsela}, intendiamo indugiare, o trattenersi
  senza profitto, \textit{tempellare} che diciamo anche \textit{metterla sul liuto}, o metterla in musica,
  e suona quasi lo stesso che.

\item[SE la passa in complimenti] Che significa Perder il tempo in vane cirimonie; e
  senza toccare la sustanza del negozio.

\item[VUOL serbare il corpo a i fichi] Vuol veder di viver, quanto ei può, e non
mettersi a rischio d' essere ammazzato.

\item[OBBEDIRE a tuoi cenni mi dò vanto] Professo d'esser' il più obbidiente servitore
  che tu habbia, e di sapere intenderti anche a i cenni.

\item[COME la serpe quando va all'incanto] Cioè mal volentieri, e forzatamente.
  \textit{Volens nolenti animo}, Omero. Il Lalli En. Tr. \cstan[2]{32}. dice
  \begin{verse}
    Come la biscia all' odioso incanto.
  \end{verse}

\item[FO buon per una volta tanto] Posso morire una sol volta, Quando si giuoca il
  danaro, che s' ha in tavola, allora che uno ha perduta quella porzione, che haveva,
  cava di tasca nuovo danaro, o vero dice: \textit{fo buono}, cioè prometto per uno
  scudo, o per due, secondo che gli pare; e s'intende, che non vuol passare quella
  somma, per la quale ha fatto buono, cioè promesso, Per esempio io fo buono
  per uno scudo, l'avversario invita di due, io tengo la posta, ma non posso
  vincere, ne perder più che uno scudo, perché non fo buono di più.

\item[SE n'andasse il colle] Se bene io sapessi, che ci fusse pena la vita. \textit{Neque si
  securim in manibus tenens aliquis cervici esset incursurus meae, conticerem}.

\item[CHI l'ha per mal, si cinga] Non m'importa, che altri l'habbia per male, e
  si cinga pur la spada, ch'io son pronto a rispondergli. Nel primo testo di mano
  dell'Autore dice \textit{si scinga}, e vuol dire si levi pur da lato la spada, perché a ogni
  modo io non voglio far quistion seco. L'Autore, che sapeva, che in tutti due
  i modi si dice, stimo forse meglio detto \textit{si cinga}, perché nel secondo, che pure è
  di sua mano, dice \textit{si cinga}.

\item[SENZ'interesse d'un puntal di stringa] Non voglio da te cosa alcuna, ancor
  che minima. Suona lo stesso che \textit{un puntal d'aghetto}, che vedemmo sopra C. 2.
  stan, 10. e che il Lat. \textit{Ne ligulam quidem}.

\item[FANNO Meo Raguni] Cioè ragunano danari. La forza sta nella voce \textit{raguni}
  che se ben pare, che sia il cognome di Meo, è il verbo ragunare, che significa
  mettere insieme, e \textit{Meo} e preso in vece di \textit{meus, mea, meum}, e vuol dire Meo
  raguni \textit{marsupio}, cioè raguni alla mia tasca.

\item[E TENGO l'adular pessimo vizio] Non è dubbio, che l'adulazione è vizio esecrando,
  e perciò Dante mette gli adulatori nell'Inferno gastigati con quella severa
  pena, che si legge al C. 18, dell'Inf. Cicerone nel suo lib. de Officiis parla
  de gli adulatori così: \textit{His denique temporibus cavendum est, ne assentatoribus patefaciamius
  aures, neve adulari nos sinamus, in quo falli facile est; tales enim nos putamus,
  ut iure laudemur, ex quo innumerabilia nascuntur peccata, cum homines inflati opinionibus
  turpiter irridentur, \& in maximis versantur erroribus}. Diogene Cinico domandato
  qual bestia mordesse più ferocemente rispose; Nelle salvatiche il detrattore,
  nelle domestiche l'adulatore, perché con le sue false lodi ti conduce alle
  rovine; Ed aggiungeva; che le parole composte non per aprire il vero, ma per
  compiacere, sono un capresto melato. Si potrebbono addurre infiniti detti di
  gravissimi Autori, ma si lascia di farlo, perché non torna affatto al proposito, e
  si rimette il lettore a Plutarco nel suo libro \textit{de dignoscendo amico ab adulatore}.

\item[PER ridurla a oro] Per ridurla alla perfezione del discorso, Per venire alla
  conchiusione. Vedi sotto \cstan[8]{1}.

\item[COME se tu havessi carestia a huomini] Come se ti mancassero huomini di spirito.
  Ancora appresso di noi quando si dice: \textit{Il tale è un huomo} s'intende huomo
  buono a qualcosa, seguitando il detto di Diogene \textit{Hominem quaero}. Nella scrittura
  \textit{Confortamini, \& viri estote}. Omero, \textit{Viri estote}.

\item[ARFASATTO] Huomo vile, mal fatto, scimunito, e da poco; che i Latini
  dicono \textit{Vappa}, \textit{Cerdo}, e simili, come si vede in Plauto da noi in questo proposito
  citato \cstan[6]{98}. E questo nome d'Arfasatto viene da \textit{Arfaxaed}
  della scrittura sagra, che nel barbaro secolo non essendo dal volgo inteso, fu
  reso per uno Babbaleo, o Babbano.

\item[DI quei che pisciano nel Cortile] Pisciar nel Cortile vuol dire Far la spia, e questo,
  perché coloro, che fanno la spia, essendo veduti entrare, e uscire del Palazzo
  della Giustizia, hanno qualche rossore, e però essendo veduti da alcuno
  lor conoscente, si fermano nel cortile di detto palazzo a pisciare per scusa. Si
  può anche dire, che il verbo \textit{pisciare} sia preso in significato di buttar fuori, ed intendere
  che \textit{piscino}, cioè buttino fuora quello che sanno nel Cortile della Giustizia,
  ove è la Cancelleria del Bargello, nella quale le spie portano le denunzie.
  Si può anche far reflessione, che detto Cortile sta sempre pieno di Sbirri, i quali
  son' anche per lo più spie, e vi sono due pisciatoi spessissimo adoprati da loro,
  ed intendere, che venga da questo il detto Pisciar nel Cortile. Ma sia come esser
  si voglia, l'effetto è, che \textit{pisciar nel Cortile} s'intende comunemente, Far la spia.

\item[CAMPO ostile] Campo nimico, Dice che è campo ostile, perché osta; e fa
  nascere il bisticcio dalla parola \textit{ostile}, e dalla parola \textit{costa}, la quale nel parlare
  pare che dica \textit{che osta}, che vuol dire s'oppone, e fa ostacolo, facendola di due
  dizioni, cioè \textit{che}, ed \textit{osta}, quando è d'una sola, cioè \textit{costa} dal verbo \textit{costare}, che
  vuol dire Esser manifesto. Modo usato da Franc. Barbarino ne' Mottetti\footnote{forse Barbarino (Barberini), Manfredo Lupo
Sec. XVI. Compositore italiano nato probabilmente a Correggio, Reggio Emilia, prima metà del XVI. Attivo fra la Svizzera e la Baviera.}.

\item[NON vengon quassù per pigliar' aria] Vengon per altro fine, che per andare a
  spasso, o pigliare aria. Detto usatissimo per intendere uno, che vada sotto altri
  pretesti in qualche luogo, e sia poi per negozio importante, e per cavar utile
  da quella gita; che i latini dissero: \textit{Non sine ratione lupus ad urbem}. E noi pure
  diciamo: \textit{Questa cosa non è fatta sine quare}. Vedi sotto C. 4. stan.~11.

\item[CONOSCE il pel nell'uovo] E' sagace, e astuto, e fa considerare ogni minuzia:
  forse è quello, che i Latini dissero: \textit{Ventura per dioptram prospicit}.

\item[A furia di spintoni] Con quantità grande, e spessa di spinte, che tale è la forza
  della parola \textit{furia} in questi termini forse dal Greco \textit{Phora}, che vuol dir' abbondanza,
  o moltitudine, Vedi sotto \cstan[9]{49}.

\item[IMBROGLIAR la Spagna] Quand'uno s'affatica con chiacchiere fuor di proposito
  per divertire uno dal principiato discorso, per non gli dire quel che egli
  vorrebbe sapere, o non fare quel che egli è imposto diciamo; \textit{Egli imbroglia la
  Spagna}.

\item[SERRAR l'uscio in su se calcagna] Vuol dir Serrar'uno fuori della porta. \textit{È il
  contrario di dare dell'imposta sul mostaccio}, che vedremo sotto \cstan[10]{27}., che
  vuol dir proibire l'ingresso a uno che venga per entrare; e quello vuol dire Obbligar uno a uscire.
\end{description}

\section{Stanza LI.}

\begin{ottave}
\flagverse{51}Sperante resta alla Regina intorno\\
Spianator di pan tondo riformato; \\
Gridan le spalle sue remo, e Livorno, \\
Ed ha un C\ellipsis{18pt} che pare un vicinato;\\
La pala nella destra tien del forno,\\
Nella sinistra un bel teglion marmato\\
In cambio di rotella, che gli guarda\\
Da i colpi il magazzin della mostarda.
\end{ottave}

\begin{ottave}
\flagverse{52}De i Rovinati anch' ei passò la barca,\\
Perché la gola, il giuoco, e il ben vestire\\
Gli haveano il pane, la farina, e l'arca\\
In fumo fatto andar come elisire,\\
Tal che, cantando poi, come il Petrarca,\\
Amore io fallo, e veggio il mio fallire,\\
Al giuoco del barone, e alla bassetta\\
Giocava, apparecchiando alla Crocetta.
\end{ottave}

\begin{ottave}
\flagverse{53}Fu dalle dame amato in generale,\\
(Io dico dalle prime della pezza)\\
Poi Bertinella stavane sì male, \\
Ch' ella fece per lui del ben bellezza, \\
Perché spesa la rola, e concia male, \\
Fatta più bolsa d'una pera mezza, \\
Potea di notte, quanto a mezzo giorno, \\
Andar sicura per la fava al forno.
\end{ottave}

\begin{ottave}
\flagverse{54}Ma poi venuta quasi per suo mezzo\\
A porsi sopr'al capo la Corona,\\
E lasciati di già gli stenti, e il lezzo\\
Profumata si sta nella pasciona,\\
N'impazza affatto, e non lo vede a mezzo,\\
E pospostane lei, c'è la padrona,\\
e Martinazza ch'è la Salamistra,\\
Sperante sempre va in capo di listra.
\end{ottave}

\begin{ottave}
\flagverse{55}Hor perch'egli è di nidio, e navicello,\\
E forte, e sodo come un torrione,\\
Gli dà l' ufizio, e titol di Bargello\\
Con la solita sua provvisione,\\
Perché s'in questo caso alcun ribello\\
Si scuopre, facil sia, farlo prigione,\\
Acciò sul letto poi di Balocchino\\
Se gli faccia serrare il nottolino
\end{ottave}

Partito Piacciantco resta appresso Bertinella Sperante; questo era Fornaio assai
comodo; ma tra il suo mandar male, e tra l'essergli stata fatta serrar la bottega,
si ridusse anch'egli malissimo, e nondimeno non usciva mai di casa le meretrici,
dalle quali veramente cavava il vitto, perché essendo bell'huomo era da
esse amato, e se ne servivano per bravo, e per ogni occorrenza loro: E per questo
il Poeta lo fa consigliero, e Bargello di Bertinella.
\begin{description}
\item[SPERANTE] Così veramente haveva nome costui, e faceva il mestiero del
  Fornaio, e però dice \textit{Spianator di pan tondo}: E lo dice riformato, perché fu proibito
  a quei tempi il fare il pan tondo (che così si chiama il più nobil pane, che si
  faccia in Firenze per il pubblico)\footnote{Pan tondo Ducale, prodotto e commercializzato esclusivamente dai ``Forni dell'Abbondanza'', e dagli Appaltatori del Pan Ducale.  Era di fior di farina, e destinato ad un mercato ristretto.  Era proibito produrre pane ordinario in forma che potesse confondersi con il Pan Ducale.} in riguardo dell'appalto, che fu preso di questa
  sorta pane; e però gli convenne serrare la bottega. Ci è però anche lo scherzo
  dell'equivoco, perché \textit{spianatore di pane} vuol dire Colui che fa il pane, ma
  significa ancora uno, che mangi molto pane. Vedi sotto \cstan[6]{47}. Sì che
  si può intendere gran mangiatore di pan tondo, ma riformato; cioè che non ne
  può più mangiar tanto, per non havere il modo da comprarlo. \textit{Riformato} è termine
  militare, e s'intende quel soldato, che è privato della carica, la quale havea;
  che si chiama poi \textit{Ufiziale riformato}.

\item[GRIDAN le spalle sue remo, e Livorno] Ha spalle così grandi, che son desiderate
  a Livorno per mettere a un remo di galera. Questo \textit{gridare, ec}, è un modo di
  dire, che ha lo stesso significato, che \textit{Chiamar di là da' monti}. Visto sopra \cstan[1]{59}.

\item[Un C\ellipsis{18pt} che pare un vicinato] Ha un \culo{}\footnote{``Culo''} grande quanto una contrada.
  Iperbole usatissima per denotare un \textit{sedere} estremamente grande, e per vicinato
  intendiamo una contrada.

\item[TEGLIA marmata] Coperchio fatto di marmo minutamente pesto, e terra,
  col quale, sendo infuocato, si cuoprono le teglie, o tegami per rotolare le vivande:
  ed è forse il Latino \textit{clibanus}; che per altro vuol dire armatura fatta di cuoio
  cotto, se crediamo a Pietro Ulloa Vita di Carlo V\footnote{Può trattarsi di una svista, e riferirsi ad Alfonso Ulloa, e del suo ``Vita dell'Invittissimo Imperatore Carlo V'', ``nuovamente mandata in luce'' nel 1560. Pagina 36.}.

\item[IL magazzino della mostarda] Cioè il ventre. \textit{Mostarda} è uno intingolo fatto
di mosto cotto, e senapa, ec. ma qui è presa (come da molti) per quella roba,
che sta nel ventre per qualche similitudine che ha quell'escremento col colore
della mostarda, e \textit{magazzino} diciamo una stanza destinata a riporvi, e
conservarvi, ec. Spagna. almazèn.

\item[PASSO' la barca de' rovinati] È nel numero de' poveri.

\item[ARCA] Voce latina, che vuol dir Cassa in generale, ma noi intendiamo specialmente
  quella gran madia, entro alla quale i  Fornai tengono il pane cotto, o la farina.

\item[FATTO andar' in fumo d'elisire] Fatto andar male senz'alcun frutto appunto
  come fa l'elixire, che lasciato in un vaso aperto svapora, e si disperde.

\item[AL Barone, e alla Bassetta] Sono due giuochi noti, i primo di dadi, e l'altro
  di carte; ma qui scherzando vuol dire, che era divenuto \textit{Barone}, cioè mal vestito,
  guidone, e ridotto al basso, che vuol dire Impoverito; traslato dalla botte,
  che si dice \textit{esser' al basso} quando il vino che v'è dentro è alla fine, e che la botte è
  quasi vota.

\item[APPARECCHIA alla crocetta] Vuol dir non haver da mangiare. \textit{Far degli
  sbavigli} significa non haver da mangiare. Vedi sotto \cstan[4]{}ultima. Ed essendo
  costume di molti nello sbavigliare\footnote{forma arcaica per ``sbadigliare''.} farsi la croce col dito pollice incontro
  alle fauci, pero \textit{far le crocette} intendiamo stare a bocca aperta, e vota, che in
  sustanza vuol dire non haver da mangiare, Qui il Poeta rende il detto più oscuro,
  più coperto dicendo \textit{apparecchia alla crocetta}, che è un Convento di Monache,
  nel qual luogo par che voglia dire, che costui desini, e ceni: che questo significa
  il verbo apparecchiare, quando è messo assolutamente, e senza aggiunta.

\item[PRIME dela pezza] E' lo stesso che di prima Classe, o passar per la maggiore
  detto sopra \cstan[1]{6}.

\item[STAVANE male] Tribolava per l'amore, che gli portava, Era grandemente
  innamorata di lui, Latino \textit{deperibat}.

\item[FECE del ben bellezza] Cioè spese, e consumò, quanto ella havea, Havendo
  consumato tutto il suo bene, le rimase solo la bellezza, o vero fece bellezza, ed
  allegria d'ogni suo havere. E' quel \textit{Proterviam facere}, che vedemmo sopra \cstan[1]{4}.

\item[BOLSA] Mal sana per troppa umidità, e ripienezza. E perché questi tali
  \textit{bolsi} soglion esser per lo più ripieni di carne liquida, e di colore fra il verde, e il
  giallo, gli paragoniamo a una pera troppo matura, o fracida, che questo vuol
  dire pera mezza. Virg. \textit{mitia poma}; cioè \textit{maturi}.

\item[POTEVA andar sicura, ec] Questo si dice d'una donna vecchia, e brutta, intendendo,
  che ella è sicura di non esser rapita.

\item[LEZZO] Puzzo, Fetore, Propriamente \textit{lezzo} e un' odore che dispiace, il
  quale non nasce da corpo corrotto, come è quel puzzo, che nasce da una carne
  troppo frolla, o altra cosa marcia, o fracida, che si dice stantia; ma è odore
  naturale, o procede da sudore, o da altra evaporazione, che getta un corpo,
  benché non sia corrotto, onde quello che si sente dal becco, e dalla capra vivi, si
  dice lezzo, e quella che si sente da i medesimi quando son morti, e corrotti si dice
  puzzo o fetore, o sito di stantio. Vedi sopra in \cstan{24}. Questo
  \textit{lezzo}, così d. da \textit{olezzo}, è proprio quello, che i L. dicono \textit{Virus}. Noi diciamo \textit{puzzo}, \textit{lezzo},
  \textit{veleno}, \textit{morbo}, \textit{fetore}, \textit{sito}, e simili pigliando l'uno per l'altro, anzi tanto l'uno,
  che l'altro è vocabolo di mezzo, perché tutti si possono intender per buono odore,
  come si cava da Caio Iurisconsulto: \textit{Qui igitur} ( dice egli ) \textit{venenum dicit debet
  adijcere utrum bonum, an malum}. E Statio lib. 2. Sylvarum: \textit{Atque omne benigni
    Virus, odoriferis Arabum; quod crescit in aruis}, Noi ancora diciamo: \textit{sento sito}, e
  \textit{puzzo di muschio}; \textit{sa di muschio ch'egli avvelena}. \textit{Gli ammorba d'ambra}, \textit{sa di
    zibetto ch' egli attossica}, ec.

\item[PASCIONA] Intende Comodità, e abbondanza d' ogni cosa necessaria al vitto,
  se ben \textit{pasciona} vuol propriamente dire Il pascolo delle bestie.

\item[N'IMPAZZA affatto] È di tal maniera innamorata di lui, che ha perduto
  il cervello. L, \textit{efflictim}, \textit{perdite amat}.

\item[NON lo vede a mezzo] Non gode la vista di lui alla metà di quello, che vorrebbe;
  termine, col quale s'esprime l'affetto grandissimo, che uno porta a
  un'altro, \textit{Non veder più avanti; ne più qua, ne più là}; usò il Bocc.

\item[SALAMISTRA] Maestra di sala. Ma noi intendiamo una donna saccente,
  dottoressa, affannona, e simili, ma per derisione, diciamo Madonna Salamistra.
  Qui intende direttrice del governo; e la chiama Salamistra pur per derisione.

\item[VA in capo di listra] Cioè toltone Bertinella, e Martinazza egli è il il padrone, o
  il primo huomo che sia in Malmantile.

\item[È DI nidio] E' tristo, E' astuto fino dalla culla. \textit{Ab incunabulis vaferrimus}.
Noi pigliamo questo detto da gli uccelli cavati dal nidio, ed allevati, che per
l'uccellatura son sempre migliori, che i presicci.

\item[NAVICELLO] Vuol dir huomo lesto, e che sa tutte le furberie, che diciamo:
  \textit{sa navigare a tutti i venti}. Ha lo stesso significato che esser di nidio.

\item[IL letto di balocchino] S'intende le forche. Da un tale detto Balocchino, che
  fu impiccato in Firenze al Canto alle rondini per ladro di bestie, delle quali fu
  Sensale, e si chiamò anche il Parola. Vedi sotto \cstan[6]{67}.

\item[SERRARE il nottolino] Vuol dire strozzare: intendendosi per Nottolino\footnote{Nottola, più spesso nottolino: elemento di serratura.} quella
  parte della canna della gola, che vulgarmente chiamiamo \textit{gorgozzule}, e questo per
  la similitudine, che ha nell'andare in giù, e in su, quando s'inghiottisce, all'andare
  in giù, e in su delle nottole da serrar porte, ec.
\end{description}

\section{Stanza LVI.}
\begin{ottave}
\flagverse{56}Fa in tanto nel Castel toccar la cassa, \\
E inalberar l'insegna del Carroccio, \\
E comandante elegge della massa \\
Il nobil Cavalier Maso di Coccio, \\
Ch' in fretta alla rassegna se ne passa\\
Con le schiere pero fatte a babboccio,\\
Che ad una ad una accomoda, e dispone\\
Sotto sua guida, e sotto suo campione.
\end{ottave}

Bertinella fa toccar tamburo, e inalberar l'insegna generale, e dichiara generale
della sua gente Maso di Coccio, il quale subito si mette a far la rassegna,
ed accomoda tutti i soldati sotto i suoi Capitani, e Comandanti.

\begin{description}
\item[CARROCCIO] Questo era anticamente un gran Carro di figura quadrata, sopra
  il quale s'inalberava appiccata a una grande antenna l'insegna Generale
  della Signoria di Firenze, e si metteva fuori in occasione di trionfi, o quando i
  Fiorentini uscivano in campagna alla guerra con esercito formato, ed è forse lo
  stesso Carro, e della stessa figura, e grandezza quello, sopra il quale si porta oggi
  il Palio di S, Gio; Bauita.

\item[MASO di Coccio] Tommaso di Coccio fu un Pescivendolo huomo fiero, e di
  gran seguito di suoi uguali, a i quali egli in tutte l'occasioni di feste, cacce, ed
  altre cose simili comandava come a' suoi servitori, ed era benissimo ubbidito da
  chi per genio, ed affetto, e da chi per timore, e però il Poeta lo fa Generale de'
  soldati di Bertinella, che son tutti di condizione simile a lui, come vedremo.
  Lo dice \textit{nobil Cavaliereo}, perché in Firenze egli era conosciuto, e nominato più che
  qualsivoglia gran Cavaliero.

\item[A BABBOCCIO] In confuso, a caso, e senza considerazione.
\end{description}
\section{Stanza LVII.}

\begin{ottave}
\flagverse{57}Si primo è il Furba nobile stradiere, \\
Che non giuoca alla buona, e meno a' goffi, \\
A noccioli bensì si fa valere, \\
Perch' ei da bene i buffi, e meglio i soffi. \\
Il secondo è il Vecchina il gran Barbiere,\\
Che vuol ch'ogni hor si trinchi, e si sbasoffi,\\
E dove a mensa metter può la mano,\\
Si fa la festa di San Gimignano.
\end{ottave}
Al Poeta mette in questa rassegna una mano di plebei noti per qualche loro
azione o buona, o cattiva, e gli nomina con i loro soprannomi. Il primo è il
Furba stradiere, cioè uno di coloro, che alle porte della Città cercano i passeggieri
se hanno roba da gabella, i quali pizzicano di spia; ma questo Furbo era
anche in effetto spia. Il secondo e il Vecchina Barbiere.

\begin{description}

\item[ALLA buona, ed a goffi] Sono due giuochi di carte assai noti: ma con dir così
intende, che costui non era ne buono, cioè semplice, ne goffo, cioè corrivo.

\item[A' NOCCIOLI ben sì] Già che il Poeta porge la congiuntura di narrare, qual
sia appresso a i nostri Ragazzi il giuoco de' noccioli, ed in quante maniere si
faccia, il Lettore si contenterà, che io spieghi con un poco di digressione i modi
co' quali si trastullano i nostri Ragazzi a questo giuoco de' noccioli, e non
si sdegnerà di volgere gli occhi a leggere il discorso di quei trattenimenti, a'quali,
non sdegnò di volger l'animo, ed impiegar l'opera un Cesare Augusto, secondo
che riferisce Svetonio Tranq. riportato, e considerato da Alex. ab Alex, dier.
Gen. \libcap[3]{24}. e ricordandosi che tutta quest'Opera è fatta per i Fanciulli
più che per quelle persone, che già \textit{reliquerunt nuces}, havra la bontà di concedere,
se non per necessaria, almeno per non affatto fuori di proposito tal digressione
Dico dunque che il giuoco, che fanno i nostri Ragazzi co' noccioli
di pesca (costumato anche da i ragazzi Greci, e Latini, che lo dicevano ludus
ocellatarum, secondo il Buleng, de Lud. veterum, \& Alex. ab Alex. dier. gen.
\libcap[3]{21}, le di cui parole poco appresso riporteremo) è usato in molte maniere;
ma specialmente giuocano, \textit{a Cavalca}, \textit{alle Caselle}, \textit{alla Serpe}, \textit{a Ripiglino}, \textit{a Sbrescia},
\textit{a Cavare}, \textit{a Sbricchi quanti}, \textit{a Truccino}, ed \textit{alle Buche}. Di tali giuochi, e
di ciascuno di essi narreremo il modo, che tengono a esercitargli, e diremo quali
sieno simili, o gli stessi, che erano usati da gli antichi.

\item[A cavalca] S' accordano due o più, e tirano sopra un piano i noccioli a un
per uno, e tanti ne seguitano a tirare, quanto stieno a far salire sopr' agli altri
tirati un nocciolo, che sopra vi resti, e si regga senza toccare altro che noccioli;
e colui che ha tirato il nocciolo rimasto sopra, vince, e leva via tutti i noccioli
tirati. Lo dicono a Cavalca da quel cavalcare, che fa il nocciolo sopr' a gli altri.

\item[ALLE Caselle] o \textit{Capannelle}. Mettono sopra ad un piano tre noccioli in triangolo,
  e sopra di essi un'altro nocciolo, e questa massa dicono \textit{casella}, o \textit{capannella}
  e fatto di esse il numero tra loro convenuto, ed allontanatisi nella distanza
concordata, tirano in dette Caselle un' altro nocciolo, e colui che tira, e coglie,
vince tutte quelle caselle, che fa cascare col colpo. Questo fu usato ancora da
gli antichi, e dicevano \textit{Ludere Castello nucum} secondo il Buleng. C. 8. Queste caselle
vengono descritte da Ovidio in Nuce in quei versi:\textit{Qutuor in nucibus non
amplius, alea tota est, Cum sibi suppositis additur una tribus},

\item[ALLA serpe] Fanno una di dette caselle, la quale figura il capo della serpe, e
da quella fanno partire un filare di noccioli, che figura il resto del corpo della
serpe, e poi vi tirano dentro con un' altro nocciolo, e chi fa col tiro scappare
uno, o più noccioli del tutto fuori del detto filare, vince tutti li noccioli, che
sono dalla rottura in giù verso la coda di detta serpe, e durano così, fino a che
sia rovinata da un di loro queila casella, che figura il capo della serpe. Questo
pure era usato da i Greci, e Latini, e forse facevano co' noccioli altre figure,
come si cava dal Buleng. Cap. 8, dove si vede, che in vece della serpe, facevano
co i noccioli un triangolo equilatere, o [come dice egli] il delta $\Delta$ de' Greci.

\item[A RIPIGLINO] Pigliano quella quantità di noccioli, che convengono, e tirandogli
all'aria gli ripigliano con la parte della mano opposta alla palma, e se
in tal' atto sopr' alla mano non resta alcun nocciolo,colui perde la gita, e tira
colui, che segue; e così si va seguitando fino che resti sopra detto luogo della
mano qualche nocciolo, e questo al quale e rimasto il nocciolo,dee di quivi tirarlo
all' aria, e ripigliarlo con la palma, e non lo ripigliando perde la gita: se ne
restasse più d'uno sopra alla mano, può colui farne scalare quanti gli piace pur
che ne resti uno; che se non restasse, perde la gita. Ripigliato il nocciolo la
seconda volta, deve costui tirarlo all'aria, ed in quel mentre pigliare uno, o più
de i noccioli cascati, e con essi in mano ripigliar per aria quello che tirò, e non
seguendo, posa i noccioli presi, e perde la gita; e se ne ha pigliati qualcheduno
senza fare errori, restano suoi, e si seguita il giuoco fino a che sieno levati tutti,
Giulio Polluce lib. 9.c. 7. mostra che facessero questo giuoco ancora li Greci, e lo
dissero \textit{Pentalitha}, perché usassero di farlo con un numero determinato di cinque
sassolini, o aliossi.

\item[SBRESCIA] È lo stesso, che ripiglino, se non che nella terza ripigliata devonsi
ripigliare quei noccioli, che cascarono in terra la seconda volta non a
uno, o due per volta, ma tutti a un tratto (il che si dice fare sbrescia) e
lasciandovene pur' uno, o cascandogliene, perde la gita, e così fiva seguitando, fin che
uno pulitamente gli raccolga tutti.

\item[CAVARE] Infilano un nocciolo con una setola di crine di cavallo, alla
qual setola ridotta in forma di campanella, o anelletto legano uno spago, di poi
segnato un circolo in terra, vi mettono i noccioli, che son d'accordo, e colui,
al quale è toccato in sorte, deve, girando in ruota con quello spago il nocciolo
infilato, a tal girare, buttar con esso nocciolo fuori del circolo uno, o più noccioli
di quelli, che son dentro al circolo, e vince quelli, che cava, e se col nocciolo
che gira, tocca terra, perde la gita; ma guadagna i noccioli cavati, e dà il nocciolo
da girare a un' altro. E così si va seguitando fino a che sien cavati tutti i
noccioli, Similmente nel giuoco detto da' Greci \textit{Eis amillan} descrivevano un
cerchio, dentro 'l quale però si doveva buttare l'aliosso in maniera, che vi rimanesse,
e non uscisse di detto cerchio. Appresso di noi anche negli Alioffi si fa a cavare.
Canti alcialeschi; \textit{Perch' al cavare un' eliosso brutto, ec.}

\item[SBRICCHI quanti] Occultano dentro al pugno, o dentro ad ambe le mani
  quella quantità di noccioli, che vogliono, poi domandando ad altri, che indovinino
  il numero de' noccioli occultati, ed indovinandolo vince tutto, se no; deve
  dare quel numero di noccioli, che ha detto di più, o di meno; E questo si fa
  una volta per uno, dovendo il primo, che domandò far' anch' egli domandare,
  e così si va continuando i giuoco. Questo \textit{sbricchi quanti} è lo stesso, che pari, o
  caffo, nel quale si domanda, se il numero è pari, o caffo, e chi s'appone vince
  tutti li noccioli occultati; se no, perde altrettanta somma. I Latini dissero: \textit{ludere par impar}.
  I Greci \textit{artiazein}, Di questo giuoco parla Giulio Polluce sopra
  citato, ed il Meursio \textit{de ludis veterum}, i quali mostrano, che si faceva, come
  pure oggi si fa con i danari, e con altra materia, come mandorle, e simili, atta
  a potersi accomodare dentro alle mani, Ovidio in Nuce. \textit{Est etiam par sit
    numerus qui dicat, an impar. Ut divinatas auferat augur opes}.

\item[A TRUCCINO] Uno tira un nocciolo in terra, e l' altro tira un nocciolo a
  quello, che è in terra, e cogliendolo vince, se no, quello, che tirò in terra il
  primo, raccoglie il suo nocciolo, e lo tira a quello, che tirò l'avversario, e così
  continovano, e chi coglie vince il nocciolo che coglie, o quello che sieno convenuti.
  È simile al giuoco detto da Greci \textit{Streptinda}.

\item[ALLE buche] Fanno diverse buche in terra in giro, formandone come una
  rosa, nelle quali tirano i noccioli, e colui vince, che entra in una di dette buche,
  quella somma, che è prezzata quella buca, nella quale entrò il suo nocciolo: per
  esempio le buche sono sette, la prima che è volta verso donde si tira, che è la più
  facile a entrarvi non fa vincere, non essendo tassata in cosa alcuna, e da i nostri
  ragazzi è detta la buca del Nisio (forse da nihil) E dell'altre una vince tre, una
  quattro, ec. E perciò ho detto, che vince chi v'entra quanto è prezzata la buca,
  e poi va con gli altri ad aiutar condurre il nocciolo nella buca a colui, che al primo
  tiro non v'entrò, e spingendolo di dove e alla volta delle buche col dito indice
  (che dicono limare). Ovidio \textit{Aut pronas digito bisve semelve petit} o col buffare,
  o col soffiare nel nocciolo, (e la differenza da buffare a soffiare vedremo
  poco appresso) nel che adoprano ogni arte per difficultare all'avversario il condurre
  il nocciolo dentro alle dette buche; E così facendo a una volta per uno a
  limare, buffare, o soffiare, colui vince, che ha fortuna di condurre il nocciolo
  dentro a una di dette buche, ancor che il nocciolo sia degli avversarj. Simile
  al fare alle buche è quel d'Ovidio. \textit{Vas quoque saepe cavum spatio distante locatur, In
    quod missa levinux cadat una manu}. Fanno questo giuoco ancora con una palla, e
  giuocano danari, come vedremo sotto C. 8. stan.\ 69. alla voce Aliosso. Ed è simile
  quello che i Greci, secondo Giulio Poll. lib. 9. c. 7. chiamana \textit{Aphetinda}: e secondo
  il Meursio de Lud. Graec. alla voce \textit{Aphetinda}, \& alla voce \textit{Amilla}, ed il
  Buleng. cap. 14. e 40. Se bene tanto nell'\textit{Aphetinda}, quanto in quello, che si chiamava
  \textit{Eis amillan}; tiravano in un circolo, e non nelle buche. Alla buca bensì
  tiravano in quell'altro detto \textit{Tropa}, che corrispondeva a questo nostro. Conchiudo
  dunque, che la maggior parte di detti giuochi erano usati anche da gli antichi;
  E se ben pare, che si servissero delle noci, io non son lontano dal credere,
  che la parola Nuces voglia dire ogni sorta di nocciolo, e mi fondo in Plinio
  \libcap[15]{21}., dove mette in dubbio, se le noci in quei primi tempi fussero
  ancora arrivate in Italia; ed oltre a questo trovo ne i Latini \textit{Iuglans}, per noce,
  ed ardirei però affermare, che ancor' essi adoperassero noccioli di pesca, o pure
  (come fanno anche i ragazzi de' nostri tempi) alle volte noci, ed alle volte noccioli
  di pesca, seguitando Alex. ab Alex. lib. 3. c. 21., che dice così: \textit{Memini doctos
  viros super nucibus ocellatis eiusmodi, quae essent, ancipitem diu cogicationem duxisse,
  variaque in opinione versari, \& alios nuces avellanas, alios amygdalas putare,
  neque satis ratam sententiam ferre super Tranquilli verbis, quibus Augustum laxandi
  animi causa cum pueris facie liberali ocellatis nucibus lusisse dicit. Quod vere nos sentimus,
  \& probabilius putamus id est: Eiusmodi nuces ocellatas nucleos, quos in persicis
  pomis sitos inspicimus dicamus esse, quibus persaepe ludere nostrates pueros hodie videmus
  dictasque ocellatas propter ocellos, \& foramina, quibus muniuntur undique, neque de
  ansyedala, aut avellana, sicut error habet, sed de persicorum ossibus, quibus  tunc ludebatur,
  \& nunc frequens puerorum ludus est, intelligi convenire credimus explorata,
  \& non ambiguae sententiae fore}. Dalle quali parole s' intende, che anticamente ancora
  si giuocava a questo giuoco de' Noccioli, Ovidio de Nuce, corrobora questa
  verità, e mostra che havessero molti de' suddetti giuochi, o poco dissimili. E
  Marziale attesta, che erano gli stessi genj ne i fanciulli de' suoi tempi, che in quelli
  d'oggidì, e che il portare in tasca noccioli causava a quelli delle mazzate, come
  segue ne i nostri, dicendo:
  \begin{verse}
    Alea parva nuces, \& non damnosa videtur;
    Saepe tamen pueris abstulit illa nates
    \verseprefix{Et altrove.}Iam tristis nucibus puer relictis
    \verseprefix{Ed Horatio.}Postquam te talos, Aule, nucesque
    Ferre sinu laxo vidi, ec.
  \end{verse}
  Sono dunque, e furono sempre puerili tutti li suddetti giuochi; e perciò noi habbiamo
  un detto di disprezzo; \textit{Va a giuoca a' noccioli}, che significa Tu non hai maggior
  giudizio di quel che habbia un fanciullo: Qual detto era usato da i Latini
  pure, come si cava da Persio Sat. 1.
  \begin{verse}
    Et nucibus facimus queacumque relictis
  \end{verse}
  E dicevano \textit{reliquit nuces} d'uno, che dalla puerizia passava a maneggiar cose serie;
  Dal che potrebbe argumentarsi, che 11 Poeta dicendo, che il Furba giuoca
  bene a i noccioli, intendesse, che egli fusse huomo di poco giudizio, e che
  \textit{nucibus imcumbat}; Ma si conosce, che non intende questo, perché prima disse,
  \textit{Non giuoca alla buona ne a i goffi}, significando che non era ne buono ne goffo, ed
  ora col dire, che egli \textit{giuoca bene a' noccioli, perché da bene i buffi, e meglio i soffi},
  vuol dir fa ben la spia, che \textit{buffare}, e \textit{soffiare} vuol dir Far la spia. Vedi sopra
  \cstan[1]{37}.
\item[BUFFI, e soffi] Buffo è un soffiare non continuato, ma fatto a un tratto, come
  si farebbe a sputare, o a profferire la parola \textit{buffi}, donde \textit{bufera}, o \textit{bufea} un gran
  nodo di vento, che passa presto. \textit{Soffio} è un soffiare con la bocca tanto quanto si
  può durare senza ripigliare il fiato, e ciò dico per mostrare la differenza che è
  fra \textit{buffo}, e \textit{soffio}; che per altro sò che \textit{soffio} è generico, e comprende ogni sorta di
  rompimento d'aria fatto col fiato di che che sia, dicendosi \textit{soffiare}, quel fiato, o
  vento, che manda fuori il mantice, \textit{soffiare} si dicono i Venti, ec. Vedi sopra
  \cstan[1]{39}, la voce \textit{rabbuffo}.

\item[IL Vecchina] Era un barbiere così chiamato, il quale ogni sera andava ricercando
  per l'osterie le conversazioni, che erano a cena, e trovandone di suoi amici,
  con varie chiacchiere poco a poco senz'essere invitato si metteva a sedere,
  e mangiava, e beveva quanto più poteva, ed al far de' conti sen' andava senza
  pagare, e quello gli era comportato, perché faceva il buffone; Procurava, che
  le conversazioni di cene si facessero in bottega sua, dove apparecchiava, e provvedeva
  assai pulitamente, e bene, e con spesa aggiustata faceva star bene, e avanzava
  tanta roba per se da viver più giorni, e però dice \textit{Vuol che ogn' hor si trinchi}
  (che dal Tedesco \textit{trinchen} vuol dir bere) \textit{e si sbasoffi}, cioè si mangi assai, donde:
  \textit{basoffione} un che mangia assai: Queste voci \textit{basoffia}, e \textit{basoffione} sono in uso
  appresso alla plebe più bassa, ed i più civili l'adoprano per scherzo, per intendere uno
  soverchiamente grasso, e che mangi molte minestre, le quali si dicono \textit{basoffie} dal
  Latino \textit{vas offae}, cioè Vaso pieno di minestra.

\item[SI fa la festa di San Gimignano] San Gimignano è una grossa Terra del Dominio
  Fiorentino nel Vescovado Volterrano; e la principale, e più solenne festa,
  che si faccia in questa Terra è di Santa Fine, la qual Santa fu di quel luogo: E
  dicendosi \textit{far la festa di S. Gimignano} s'intende si fa fine; e qui vuole esprimere,
  che questo Barbiere dava fine a ogni cosa, che veniva in su la mensa.

\end{description}
\section{Stanza LVIII}
\begin{ottave}
  \flagverse{58}Dalle fredde acque il Mula i fanti approda \\
A spiaggia militar fra fronde, e frasche, \\
Ha nobil bardatura tinta in broda \\
Di cedri, e di ciriege d' amarasche, \\
Co i pescatori al Mula hora s'accoda\\
Dommeo  Treccon de ghiozzi, e delle lasche;\\
Pericol pallerino anch' ei ne mette\\
Dugento suoi armati di racchette
\end{ottave}

\begin{description}
\item[IL mula dalle fredde acque] Fu uno che nel tempo di state vendeva l'acque diacciate
  così soprannominato. Pare che questo Mula sia un gran sig.\ di lontani paesi
  e vicino al Mar gelato, di dove approdi alla spiaggia del mare; ma \textit{approda}, cioè
  s' accosta al restante dell' armata di Bertinella. Dice \textit{fra frondi}, e \textit{frasche}, perché
  questi tali venditori d'acque diacciate sogliono per allettamento ornare le loro
  di verzure, fiori, e frasche.

\item[S' ACCODA] Seguita, o vien dietro immediatamente. Quasi \textit{ad caudam ire}.
  Noi usiamo questo verbo per le bestie da soma, che seguitando in viaggio l'una
  l'altra, viene alla prima legata la seconda, alla seconda la terza, ec, con
  la cavezza alla groppa dell'antecedente, e così chi seguita va con la testa vicina
  alla coda di essa, e questo si dice accodare, benissimo usato qui dal Poeta,
  per il Mula, sendo che a i muli più, che ad ogni altra bestia segue questo accodare.

\item[DOMMEO] È una parola sola, e dovrebbe dire \textit{Dommeone}, che così era
  chiamato un venditore di pesce, e salumi, il quale era amato da tutti i ghiotti
  di Firenze, perché vendeva sempre il miglior pesce, che venisse in mercato, ed i
  giorni di grasso haveva sempre qualche galanteria, o ghiottornia singolare. E
  però lo chiama \textit{treccone}, che vuol dire Rivendugliolo, cioè rivenditore di cose
  commestibili di poco prezzo (che si dice anche barullo) forse dal Latino \textit{tricae},
  bagattelle, cose di poca stima, e di vil pregio; Marziale, \textit{Sunt apinae, tricaeque,
    \& si quid vilius istis}. Dice di \textit{ghiozzi}, e di \textit{lasche} (due specie di pesce note) non
  per intendere, che vendesse solamente questi, ma per mostrare, che vendeva
  pesce in generale.

\item[PERICOLO] Questo fu un tale Alessandso Violani detto Pericolo antonominato
  per il suo gran valore nell'abbaco, come diremo sotto C.~11. stan.~41. E
  perché egli era anche bravissimo giuocatore di Palla a corda, e tenne gran tempo
  a fitto una di quelle stanze dove si giuoca a tal giuoco, lo fa venire con gente
  armate di \textit{racchette}, o \textit{lacchette}, che sono mestole, con le quali si giuoca alla palla
  a corda, e sono composte d'un cerchio di legno col manico, ed il vano è ripieno
  d'una rete fatta di grossa minugia: per \textit{lacchetta} intendiamo anche la coscia
  di dietro del porco, e del castrato; Non so già se la \textit{lacchetta} da giuocare pigli il
  nome da questa, o questa da quella, so ben che si chiamano così l'une, e l'altre
  per la similitudine, che è fra di loro della figura. Questa da giuocare era da i
  Latini detta \textit{reticulum} da quella rete, della quale è composta, come si cava da
  Ovidio: \textit{Reticuloque pilae leves fundantur aperto}. Vedi sotto \cstan[6]{34}. alla
  parola \textit{Pillotta}.
\end{description}

\section{STANZA LIX, STANZA L}
\begin{ottave}
\flagverse{59}Melicche quoco all'ordine s'appresta, \\
Per giannettina ha in mano uno stidione, \\
Ed un pasticcio per visiera in testa \\
Con pennacchio di penne di cappone \\
Un candido grembiul per sopravvesta \\
Gli adorna il \culo{} e l'uno, e l'altro arnione, \\
Vina zana è il suo scudo, e nell'armata \\
Conduce tutta Norcia, e la vallata.
\end{ottave}

\begin{ottave}
\flagverse{60}L'unto Sgaruglia con frittelle a iosa\\
Alla squadra de Quochi hora soggiugne\\
Quella de' Battilani assai famosa,\\
Gente che a bere e peggio delle spugne,\\
A cui battien (diceva) la calcosa,\\
Ch'affeddeddieci là dove si giugne\\
Noi non habbiamo a scardassar più lana,\\
Ma s'ha a far sempre la lalunediana.
\end{ottave}

Segue Melicche Zanaiuolo di Mercato vecchio, uno di coloro, de' quali ci serviamo
per mandare a casa le robe commestibili, che si comprano in Mercato
vecchio, e ci servono ancora per Quochi. Costoro son per lo più della Vallata
e Cantoni Svizzeri, e dimorando in Firenze soglion far camerata co i Norcini,
che vendono i tartufi, e per questo dice che egli conduce Norcia, e la Vallata. E
perché egli era hvomo pulitissimo, gli fa per sopravvesta un grembiule candido,
come veramente egli sempre portava.

\begin{description}
\item[GIANNETTA] onde \textit{Giannettina}; specie d'arme in asta, nella guerra usata
da gli alfieri. \textit{Gineta} in Spagn. è una piccola lancia; corsesca.

\item[PENNACCHIO] S'intende una quantità di penne di Struzzolo; ma costui
  l'havea di Cappone come trofeo di Quoco.

\item[ZANA] Specie di paniere senza manico composto di strisce di legno gentile,
  e da tale Zana costoro son detti \textit{Zanaioli}. Di questi tali il Poeta fa Capitano Melicche,
  perché in vero egli era riverito da essi, come quelli che nel loro paese
  l'havevano veduto esercitare Cariche riguardevoli, e sapevano, che era de i più
  reputati della sua patria, dalla quale era in quei tempi bandito.

\item[SGARUGLIA] Fu un Battilano assai celebre, e fra i suoi pari Capopolo, e
  da costui quando in commedia e stato introdotto il Battilano l'hanno nominato
  Sgaruglia. Questi condece la schiera de' Battilani, che dice \textit{famosa}, e scherzando
  con l'equivoco, vuol dire Affamata, da Fame, e non da Fama.

\item[FRITTELLE] Così chiamiamo una vivanda fatta di pasta quasi liquida fritta
  nell'olio da i Latini detta \textit{Artolaganus}; e sì come essi mescolavano con detta pasta
  latte, ed altro, così noi pure vi mettiamo delle mele affettate, uva fecca,
  latte, riso, erbe, ed altro secondo i gusti. I nostri contadini nel tempo, che fanno
  l'olio costumano di far molte di tali frittelle, indotti a ciò da havere olio in
  abbondanza, e ne danno anche a i vicini, e parenti; sono però soliti coloro, che
  vanno a veder lavorare, chiedere le frittelle, ed i lavoranti con poca grazia, e
  meno discrezione spruzzano l'olio addosso a quel tale dicendo: Eccoti le frittelle.
  E da questo forse per \textit{frittelle} intendiamo macchie, che vuol dire Ogni segno, o
  tintura, che sia nella superficie d'un corpo diversa dal proprio colore di quel tal
  corpo, come segue, quando l'olio casca sopra ad un panno. Ed il Poeta dicendo,
  che costui \textit{havea molte frittelle}, intende, che egli era assai unto, come sempre
  sono i Battilani per il continuo maneggiare olio, e lane unte.

\item[A IOSA] In quantità grande. Diciamo nel medesimo signifitato \textit{a cafisso}, \textit{in
  chiocca}, \textit{a biscia}, \textit{a fusone}, voce usata da Giovanni Villani, a similitudine della
  Franzese \textit{A foison}, cioè con effusione, senza risparmio, \textit{a furore}, \textit{a precipizio}, \textit{a
  bizzeffe}, \textit{a Isonne}, e simili. Che se bene son modi bassi, nondimeno sono tulvolta
  usati anche fra la gente civile. E questo a \textit{Iosa} credo sia parola corrotta, e che
  dovesse dire a \textit{chiosa}, che significa quelle cappelle, che hanno le bullette, e
  ogni piccola piastra di piombo, di rame, o d'ottone ridotta tonda, e simili
  alle nostre monete, delle quali chiose i nostri ragazzi si servono per giuocare alla
  trottola in vece di monete, e però \textit{chiosa} s'intende per moneta di niuf valore:
  Il Persiani disse:
  \begin{verse}
    Ma s'in tasca non ho pure una chiosa
    A mantenermi, in tanto quae pars est ?
  \end{verse}
  Si che dicendosi: Della tal mercanzia ve n'era a \textit{Iosa}, o a \textit{chiosa} s'intende,
  che di quella mercanzia ve n'era così grande abbondanza, e per questo era a così
  vil prezzo, che se n'haveva fino per una chiosa. Il Berni nel suo Capitolo in lode
  de' Ghiozzi disse:
  \begin{verse}
    Segue da questo un' altra disciplina,
    Che havend' ingegno, e del cervello a iosa,
    Bisogna che v' habbiate gran dottrina.
  \end{verse}
  Il Domenithi in lode della Zuppa.
  \begin{verse}
    E quinci vien, ch' ella si suol gradire
    Da chi ha cervello, ed intelletto a iosa.
  \end{verse}
Questa voce \textit{chiosa} per similitudine significa ancora le Croste delle bolle, E vuol
anche dire Esposizione, o comento, forse dal latino greco Glossa. Dante num.~2.
Purg. C.~11.
  \begin{verse}
    E serbolo a chiosar con altro resto,
  \end{verse}
  E nel'Inf C.25.disse \textit{Faranno sì che tu porrai chiosarlo},

  Il Varchi nel Capitolo dell'uova sode dice:
  \begin{verse}
    E s'io fussi Dottor, consiglierei
    Che sopra questo si dovese fare
    Leggi, e statuti, e poi gli chioserei.
  \end{verse}

\item[PEGGIO delle spugne] Succia il vino più che non farebbe uaa spugna; cioè
  beve assaissimo, come veramente fanno i Battilani, i quali chi sieno, dicemmo sopra
  in \cstan{8}.

\item[BATTER la Calcosa] Frafe Furbesca, che vuol dir batter la strada, camminare;
  e questo parlar furbesco è praticato assai da questa sorta di gente.

\item[AFFEDDEDDIECI] Giuro proprio de' Battilani profferito come è scritto in
  una sola parola con due ff, e quattro d. Quando i Battilani hanno gran lavori
  e sono molte persone a lavorare, hanno ogni dieci huomini un Sopracciò, che
  chiamano il Capo dieci, che è da loro ubbidito, e stimato, e però giurando a
  fe del Dieci, intendendo di costui, stimano di fare un giuramento solenne.
  Credo nondimeno che dicano a fe de Dieci per non dire a fe di Dio, come pure
  dicono per Dianora, Corpo di Dianora per la medesima ragione.

\item[SCARDASSAR la lana] Cioè pettinare la lana con quei pettini, che chiamano
  Cardi, perché hanno i denti torti, e simili a quelli spuntni, che hanno le
  foglie, il fusto, ed il fiore dell'erba detta cardo, del qual fiore quando è secco
  si servono per pettinare, ed unire il pelo de i panni, e però lo dicono cardare,
  ed è il latino \textit{carminare}. Vedi sotto \cstan[7]{37}.

\item[FAR la lunediana] Appresso a i battilani significa non lavorare; e questo, perché
  nel tempo, che l'arte della lana lavorava, costoro guadagnavano assai, ed
  erano pagati dalli loro maestri il lunedì, dove gli altri manifattori sono pagati il
  sabato, e però questo giorno del lunedì, essendo per loro giorne d'allegria stante
  la riscossione, era da essi solennizzato, e non volevano lavorare, (ma stando in festa)
  a consumare in bere, ed in mangiare quel denaro, che havevano riscosso,
  e questa loro solennità chiamavano \textit{Lunediana}, cd alle volte \textit{Lunigiana} ed era
  da essi tal festa così osservata, che tra loro era la seguente cantilena,
  \begin{verse}
    Chi non fa la lunediana,
    E' un gran figlio di puttana.
  \end{verse}
  Ed oltre a questa ce n' è un' altra che dice:
  \begin{verse}
    Il Venerdì de Beccai,
    Il Sabato de gli Ebrei,
    La Domenica de' Cristiani,
    E il lunedì de i Battilani.
  \end{verse}

  Sì che dicendo \textit{lunediana} s'intende festa, come si vede nel presente luogo
  che Sgaruglia dicendo \textit{s'ha a far sempre la Lunediana, ec}, intende hada esser sempre
  festa. Questo nome di Lunediana resta ancor' hoggi, ma come che i Battilani
  sono pochi, ed i lavori meno, convien loro per forza stare alle volte le Settimane
  intere senza lavorare, e così non è messa troppo in uso detta solennità,
  anzi hanno di grazia, lavorare anche il lunedi.
\end{description}
\section{Stanza LXI.}
\begin{ottave}
  \flagverse{61}Conchino di Melone ecco s' affaccia, \\
Che l'Offerta tenendo de gli allori\\
Col fine, e saldo d'un buon prò vi faccia \\
Ha dato un frego a tutti i debitori, \\
Che tutti allegri, e rubicondi in faccia\\
Cantando una canzone a quattro cori,\\
Di gran coltelli, e di taglieri armati,\\
Si son per amor suo fatti soldati.
\end{ottave}

Segue \textit{Conchino di Melone}, il quale si conduce dictro una mano de' suoi debitori,
che si son fatti soldati per la cortesia, che ha fatto loro di scancellare a tutti
il debito, che havevano seco. Fu costui già quoco d'Osterie, e per esser molto
grasso, e di statura piccolo fu chiamato Conchino, gli venne voglia di diventar
maestro, onde prese sopra di se un'Osteria detta \textit{gli allori}, dove subito hebbe
molti bottegai, ma tutti a credenza, per lo che presto fallì; e non trovando modo
di risquotere un soldo gli venne rabbia, ed abbruciò i libri per non haver di
più quella passione di vedere scritti i suoi denari, e non gli potere spendere. E
questo intende dicendo \textit{col fine, e saldo d' un buon pro vi facia ha dato frego a tutti i debitori}.

\begin{description}
\item[S'AFFACCIA] Si fa innanzi. L'Autore si serve di questo verbo afacciarsi,
  per denotare, che costui havea la faccia larga; scherzo assai praticato con uno,
  che habbia gran ceffo dicendosegli affacciatevi, facciami favore, facciami buon viso,
  e simili.

\item[TAGLIERE] Intendiamo un'arnese da cucina, fatto di legno, tondo a foggia
  di piatto per uso d' affettare sopra di esso carne, e per triturarla con quei \textit{gran
    coltelli}, e farne polpette, o altri battuti. I Tedeschi usano in molti luoghi i piatti
  da tavola fatti di legno, e gli chiamano \textit{Talier} con voce venuta d'Italia, come
  si può credere; già che i nostri antichi i piattelli, o tondini dal tagliarvi su le
  vivande, domandavano \textit{taglieri}, onde il proverbio. \textit{Due ghiotti a un tagliere}, cioè
  \textit{a uno stesso piatto}. Trovasi questa voce nella antica lingua Gallese, o Francesca;
  e dicevano \textit{tailleor}; come leggesi in un' antichissimo libro in quella lingua, dal Lat.
  volgarizzato, appellato del Conquista della terra Santa di Gerusalemme, il quale si è
  ritrovato essere di Guglielmo Arcivescovo di Tiro; e si conserva nella preziosissima
  libreria di Manoscritti del Sereniss. Gran Duca, appresso alla Chiesa, e Collegiata
  di S. Lorenzo. Il passo tutto voltato in Toscano dice così; La dentro (in
  Cesarea) fu trovato un vasello di pietra verde, e chiara assai di troppo gran
  beltà, fatto così, come un tagliere\footnote{Si riferisce al ``Sacro Catino'', ora ritenuto manufatto islamico in vetro di color verde smeraldo, del IX-X secolo.}. Li Genovesi pensarono, che ciò fusse uno
  smeraldo. Perciò lo prenderono a lor parte, del guadagno della Città per troppo
  gran somma d'avere. Portaronnelo in lor Città, e l'appesero nella Mastra
  Chiesa, ove egli è ancora. L'huomo vi mette la cenere, che si prende il primo
  giorno di Quarefima, e si mostra altresì come ricchissima cosa. Perché e' dicono
  veracemente, ch'egli è di smeraldo. Nel margine vi è questa postilla in nostra
  lingua. Quando, e dove e' Genovesi guadagnano el \textit{catino} di smeraldo, che tengono
  ancor'oggi nel monte di S. Giorgio, e credesi, che sia \textit{il piatto}, dove mangiò
  Cristo Giesù alla gran cena.
\end{description}
\section{Stanza LXII.}
\begin{ottave}
\flagverse{62}Scarnecchia che di guerra è un ver compendio,\\
L'Eroe degli arcibravi, e dico poco,\\
A cui dovrebbe dar piatto, e stipendio\\
Chiunque governa in qualsivoglia loco,\\
Perché quando seguisse qualche incendio\\
Ei fa il rimedio per guarir dal fuoco,\\
Mena gente avanzata a mitre, e gogne,\\
Da vender fiabe, chiacchiere, e menzogne.
\end{ottave}

\begin{ottave}
\flagverse{63}Rosaccio con altissime parole\\
Movendo il pié racconta, c'a pigione,\\
Fa per quel mese dar la casa al Sole,\\
E nel zodiaco alloga lo Scorpione;\\
Cosi sballando simil ciance, e fole\\
Si tira dietro un nugol di persone,\\
Fa per impresa in mezzo all intervallo\\
Di due sue corna un globo di cristallo.
\end{ottave}

Seguita \textit{Scarnecchia}. Questo fu un Montambanco o Ciarlatano, il quale vendeva
unguento per medicare scottature, e montava in palco sempre in abito da
Coviello col nome di Capitano Scarnecchia, e faceva una mano di braverie a
fine di ragunate il popolo, e però l'Autore lo dice \textit{compendio di guerra, ed Eroe
de li arcibravi}. E perché è Ciarlatano, lo fa capo di Monelli, e gente avanzata
alla berlina, e che è buona a vender bugie, come per lo più sono i Montanbanchi.
Dice che doverebbe esser provvisionato\footnote{``provvisionato'' è chi riceve sussidio pubblico nella Toscana granducale.}, perché ha il rimedio di liberare
dal fuoco le case, che abbruciassero, e scherza, burlando l'unguento, che vendeva
detto Scarnecchia buono a guarire le scottature in un corpo humano, facendolo
buono a rimediare a gl'incendj.

\begin{description}
\item[MITRA, o Mitera] Diciamo quel foglio, che a foggia di corona si mette in
  capo a coloro, che per delitti son frustati, o mandati in su l'asino. Vedi sotto
  \cstan[6]{50} e \cstan[12]{19}.

\item[GOGNA] \makebox[4pt]{} È lo stesso che Berlina detto sopra C.\ 2.\ stan.\ 15. I Latini la dicono
  \textit{Numellae}, se ben questa era più tosto una specie di ceppi da serrare i piedi, onde
  forse meglio con Plauto, e con Lucilio la chiameremo \textit{collare}.

\item[FIABE, e menzogne] Sinonimi, che significano Bugie. \textit{Fiaba} da \textit{fabula}, e
  \textit{menzogna} dal verbo \textit{mentior}.

  Dopo li suddetti vien \textit{Rosaccio}, il quale conduce seco una gran mano di persone
  tirate dalle sue chiacchiere. Costui fu uno de i più superbi ciarloni, che sia
  mai stato nella Ciarlataneria, e spacciavasi per Astrologo. Non montava in
  banco, ma stava a cavallo allato a una tavola elevata, sopr' alla quale posava
  una faragine di cartapecore di privilegi havuti (diceva egli) per il suo valore
  da i maggiori Potentati della Cristianità, qualche scheretro di gatto, o cane, una
  sfera d'ottone, tre corni neri lunghi, all'uno de' quali era appeso un pezzo di
  calamita, all'altro una palla di limpidissimo Cristallo di Monte, ed al terzo un
  corno, che egli diceva essere d'Unicorno. Vendeva una sua mestura da lui chiamata
  con vocabolo Greco \textit{Nepenthes}, che diceva esser buona a tutte l'infermità
  conforme al medicamento d'Elena chiamato con queste medesimo nome di \textit{Nepenthes}
  (cioè di contrario al dolore) dal Poeta nel 4. dell'Ulissea, ed a chi la comprava
  donava un'anelletto d'osso, che spacciava per ottimo al dolor di testa,
  per esser fatto di dente di Cavallo marino. Diceva havere, imparata l'astrologia
  da un gran Mattematico, ed Astrologo suo Zio nominate Gioseppe Rosaccio,
  che predisse (vantava egli) la rovina della palla della Cupola del Duomo
  di Firenze molto tempo avanti, che cella seguisse. In somma con le ciarle, e
  fandonie ragunava sempre, che montava a cavallo, infinite persone, e pigliava
  buone somme di danari; Il Poeta lo fa condottiere di questa gente adunata con
  le chiacchiere, e gli fa fare per impresa quei tre suoi corni suddetti con la palla
  di cristallo.

\item[ALTISSIME parole] Chiama parole altissime quelle di Rosaccio; perché egli
  sempre discorreva di pianeti, di stelle, e d'altre cose celesti come mostra l'Autore
  con dire, che egli ha affittata la casa al Sole, e messo lo Scorpione nel Zodiaco.
  Senza ironia Dante Inf. 4. chiamò Virgilio; l'altissimo poeta. E poco appresso:
  Così vidi adunar la bella scola Di quel Signor dell'altissimo canto, Ove il Landino:
  Altissimo canto chiama la Poesia, la quale in ottimo, e ornatissimo canto di versi abbraccia
  tutte le dottrine, e massime la Teologia, imperoché i primi Poeti furono Teologi.

\item[SBALLARE] Vuol Propriamente dire disfar le balle, ma ci serve anche per
  esprimere uno che racconti molte, e molte cose più vicine alla bugia, che alla
  verità, ed è il medesimo, che \textit{schiantare}, che vedremo sotto \cstan[10]{66}. Questa
  voce \textit{sballare} in altro significato vedremo sotto C.~11.\ stan.~4.

\item[CIANCE, e fole] Sinonimi;  e l'ultimo è Sincope di favole; ed intendiamo
  chiacchiere lontane dal vero. Petrarca \textit{Sogni d'infermi, e fole di Romanzi}. Il
  Mauro in biasimo dell'Onore\footnote{Capitolo in Dishonor del Honore, al Prior di Iesi.} disse:
  \begin{verse}
    Hor vi dich'io, che le son tutte fole,
    Tutti argumenti da ingannar gli sciocchi,
    Le cose che consistono in parole.
  \end{verse}
  Il Persiani in una sua canzone dice:
  \begin{verse}
    Se con tagliare o fole
    Vo pagar di bravara.
  \end{verse}
  Ottavio Ferrari\footnote{Ottavio Ferrari (Milano, 20 maggio 1607 --- Padova, 8 marzo 1682) accademico, archeologo, filologo, bibliotecario. } nelle sue Origini deduce le parole \textit{Ciance}, e \textit{Cianciare} da
  \textit{Cantiones}, \textit{Cantionare}. Il Bocc. Nov. 61. quando disse \textit{La landa di donna Matelda, e
  cotali altri ciancioni} volle dire senza dubbio \textit{canzoni}, le quali (perché erano molto
  in pregio le Provenziali, o le fatte su l'arie di Provenza, come si vede da alcune
  intitolazioni di Lande antiche) chiama come per istrazio, e contraffacendo
  in questo, sì come in molti altri luoghi, la pronunzia delle lingue straniere;
  \textit{ciancioni}; Scherzando anche nel medesimo tempo sull'altro significato, cioè di \textit{ciancia},

\item[VN nugolo di persone] Questa voce nugolo per Quantità grande, è assai usata
  dai noi, e l'usò il nostro Poeta sopra C.\ 1.\ stan.\ 50.
  Così Giuvenale Sat.\ 13.\ imitando
  in ciò Omero; chiamò la moltitudine delle combattenti grù, \textit{nubem sonoram}.

\end{description}

\section{Stanza LXIV.}
\begin{ottave}
\flagverse{64}Sopr' un letto ricchissimo fiorito \\
Portar: Pippo si fa del Castiglione, \\
Ove coperte sta tutto vestito,\\
Ch'in tal modo lo scalda al suo padrone; \\
E pur, s'in arme ei non fu gran perito,\\
Guerrier comodo almen nel padiglione.\\
Questo impera dal morbido piumaccio\\
A quelli del mestier di Michelaccio.
\end{ottave}
Seguita Pippo del Castiglioni portato in un ricco letto, di dove comanda a i soldati,
che son tutta gente senza voglia di lavorare. Costui era il più grazioso, e
faceto umore, che sia mai stato in Firenze, e si chiamò Pippo del Castiglioni, perché
servì lungo tempo a i SS.\ di Casa Castiglioni con fedeltà indicibile, e però da'
medesimi SS, amato a segno, che non ostante le burle, che in diversi tempi, ed
occasioni faceva a essi SS.\ non potettero mai mandarlo via, perché, se lo licenziavano,
egli trovava sempre vaghe invenzioni per non sen' andare, come fra le
molte fu questa: Il sig.\ Cavalier Vieri da Castiglione, al quale per ordinario
serviva, lo licenziò con queste parole: \textit{Sgombrami di Casa}. Pippo andato in Piazza
chiamò quattro Carrettai, e condottigli con le loro carrette d' avanti alla
porta dell'abitazione di essi SS.\ in su l'ora, che il sig.\ Cavalier Vieri soleva tornare
a desinare, ordinò loro, che, se il medesimo sig.\ Cavaliere gli domandasse
quello, che facevano quivi, gli rispondessero, che ve gli haveva mandati Pippo; si
come seguì ed il Sig, Cav.\ disse: che ha da far Pippo delle carrette? Ed egli a
queste parole scappato di dietro a una di esse carrette, rispose: Sgombrare, come
VS.\ Illustriss.\ m'ha comandato; Onde il Sig, Cav.\ ridendo della faceta interpretazione
del suo comandamento lo richiamò in casa, e pagati i carrettai gli licenziò.

\begin{description}
\item[IN un letto riechissimo fiorito] Il medesimo Sig.\ Cav, una sera comandò a Pippo
  che facesse, che il letto fusse caldo, quando egli tornava a dormire, che sarebbe
  stato assai di notte. Pippo si scordò di mettere il caldanetto nel letto, onde,
  tornato il Padrone, e volendo andare a dormire, Pippo si trovò imbrogliato,
  perché stante l'ora tardissima non vi era modo di trovar fuoco; ricorse però alle
  solite astuzie, e questa fu, che egli per la parte di dietro del letto v' entrò
  così vestito com' egli era, ed il padrone, credendo che egli andasse movendo
  lo scaldaletto, si spogliò da per se per non lo scioperare, e spogliato andò alla
  volta del letto, e disse: Cava il fuoco, ed alzata la cortina per entrare nel letto,
  vedde Pippo, che sollevata alquanto la testa disse: Signore il letto non è ancora
  caldo a bastanza. Il sig.\ Cavaliere vedutolo così, e conoscendo l'umore della
  bestia senz' alterarsi lo fece uscire, e toltasela in pace entrò nel letto così come
  era. E per alludere a questa, facezia il Poeta fa venir Pippo portato in un ricchissimo letto.

\item[PIVMACCIO] Guanciale lungo quanto la larghezza del letto; della grossezza
  d' un sacco ordinario da grano, ed è ripieno di piume, e però è detto \textit{Piumaccio}.
  Qui per piumaccio intende tutto il letto.

\item[QUELLI del mestiero di Michelaccio] Gente, che non ha voglia di lavorare,
  che il mestiero di Michelaccio dicono, che era mangiare, bere, e andar a spasso.
\end{description}
  Qui pure bisogna, che il Lettore si contenti ch' io faccia un poco di digressione
  per narrare alcune delle facezie del detto Pippo, meritando la graziosa sagacità
  di questo huomo, che si spenda qualche di tempo in sentire le di lui arguzie,
  il quale è vissuto fino a pochi mesi addietro d' età di 85.\ anni sempre con
  la medesima bizzarria, salvo che, dove prima frequentava molto l'osterie per
  trovar le conversazioni, che gli pagavano lo scotto, (perché mai haveva un
  quattrino, dando egli tutto quello che guadagnava alli suoi vecchi Padre e Madre,
  alli quali continovò d'ubbidire come un fanciullo fino all'età sua di sopra
  75.\ anni, che essi passando cento anni, morirono) dopo la morte del Padre frequentò
  più le Chiese pregando S.D.M.\ per la salute del Sereniss.\ G.\ Duca, dal
  quale godè fino, che visse, onorata provisione per il buon servizio reso alla Serenissima
  Casa.

  Essendo una volta il medesimo sig.\ Cav.\ Vieri al Poggio a Caiano (villa del
  Sereniss.\ G.\ Duca) a servire il Sereniss.\ Sig.\ Principe Card.\ Gio.\ Carlo, mandò
  Pippo a Firenze la vigilia del Santiss.\ Natale ordinandogli, che si facesse-dare dal
  sarto un suo vestito nuovo, e lo portasse al Poggio, e l'ordine, che gli diede fu
  con queste parole: \textit{Va a Firenze, e fatti dare dal sarto il mio vestito, e portalo}.
  Ubbidì Pippo, e la sera medesima tornò col detto vestito del padrone in dosso, ed
  entrato in Chiesa dove era tutta la Corte per udir la Messa (mancandovi
  sig.\ Cav.\ Vieri, che se ne stava in camera aspettando il vestito per metterselo) fu
  veduto da tutti i Cortigiani, e da tutti li Sereniss.\ Principi che quivi erano, ed
  il sig.\ Principe Card.\ Gio.\ Carlo gli disse: sig.\ Filippo che cosa è questa? Voi
  siate molto nobile ? Ed egli rispose: Sereniss.\ queste son grazie che mi fa il mio
  Padrone. E S.A.Rev.\ immaginandosi di come stava il fatto si rallegrò con
  Pippo, il quale fatte, più spasseggiate per la Chiesa sen'andò alle stanze del suo
  Padrone, che vedutolo con quell'abito in dosso lo sgridò dicendo, \textit{Briccone, che
    siam fratelli?} Rispole Pippo: \textit{Perché sig.?} Replicò il sig.\ Cav.\ \textit{Che furfanteria,
    è la tua mettersi il mio vestito?} Mi maraviglio di V.S.\ Illustriss.\ (soggiunse Pippo)
  non me l'ha ella donato? Come donato! (disse il Sig.Cav.) Ti par' egli abito da
  par tuo? Sig.\ sì che mi pare, e mi sta benissimo; E V.S.\ Illustriss.\ medesima
  m'ha detto, che io me lo faccia dare dal sarto, e lo porti, ed ecco ch'io l'ubbidisco,
  già tutta la Corte ha saputo questa generosità di V.S.Illustriss, e si sono
  rallegrati meco del regalo, che V.S.Illustriss.\ mi ha fatto in questa solennità.
  Il Sig.\ Cav.\ conoscendo, che non era suo decoro il mettersi quel vestito, che era
  stato veduto in dosso al suo servitore, stimò bene il quietarsi, e fargliene un regalo,
  per non poter far' altro; E cosi Pippo si godè quell'abito, che per la sua
  ricchezza era decente a un Principe.

Era grande amico di Pippo il Prete Fantacci oggi vivente Rettore della Chiesa
di Varlungo fuori di Firenze circa un miglio, il qual Prete è stato sempre huomo
assai faceto, e piacevole; e fra esso, e Pippo son seguite diverse graziose burle
e fra l'altre il Fantacci disegnò una volta di fare star Pippo senza cena, e necessitarlo
a dormire all'aria; e per questo l'invito ad andare alla sua Chiesa a Cena
quella sera appunto, che il Prete havea fermato d'essere a cena nella Villa de' SS.
Bonsi quivi vicina; e ad effetto, che gli riuscisse il disegno haveva ordinato alla
serva che andasse a dormire a casa una sua parente, e detto al Contadino, che
era presso alla Chiesa, che, se fusse accaduta cosa alcuna attenente alla cura, mandasse
al Prete di Rovezzano, Chiesa vicinissima a quella di Varlungo. Pippo chiesta,
ed ottenuta licenza dal suo padrone, la sera al serrare delle porte della Città, se
n'andò a Varlungo, e trovata serrata la porta della Casa del Prete, dopo haver
molto picchiato, conosciuto, che non era veruno in casa, disperato s'accostò alla
casa di quel Contadino, che haveva l'ordine di mandare la gente a Rovezzano.
e da esso intese, che il Prete era andato a cena fuor di cura, e gli ordini che havea
lasciato. Pippo accortosi molto bene, che il Prete l'haveva burlato, volle
rendergli la pariglia, e perciò fare trovata una scala a pivoli, con essa montò sopra
il tetto della chiesa, e quivi portata buona quantità di paglia, ed altro ciarpame
combustibile, e raro, gli dette fuoco, ed andato alle funi delle campane
si messe a suonare a rintocchi. Il Prete Fantacci, che era.poco lontano sentendo
suonare a martello, st affacciò a una finestra per sentire, che cosa fusse quella,
e veduto il fuoco sopr'alla sua Chiesa, tutto spaventato lascio la cena, e l'allegria,
e corse alla volta della sua casa, nella quale subito entrò per vedere dove
era il fuoco, e rimediarvi con l'aiuto d'una parte de' SS. Commensali, e con
una quantità di contadini, che già erano quivi concorsi con zappe, e pali per rovinare,
e tagliare dove bisognasse. Pippo intanto sceso dal tetto se n'andò
ad arno, e si fermò a cena da un tal Bonini mugnaio suo grande amico, bastandogli
d'havere sturbata l'allegria, nella quale era il Prete, il quale girato e sotto
e sopra per tutta la casa, e non havendo trovato ne meno segno di fuoco,
fece visitare il tetto della Chiesa, e trovò la paglia, che era finita d'ardere, e
vista la scala appoggiata alla muraglia, s'accorse che era stata una contraburla
di Pippo, tanto più che il contadino detto di sopra disse haverlo veduto poco
prima, e perciò sopportandosela in pazzienza, tornò a cenare, dove non mancarono
le minchionature, e barzellette, che furono da quei SS. della conversazione
dette al Prete.

Commesse una volta Pippo non fo che mancamento, per lo quale il Padrone
volle mortificarlo col mandarlo in carcere, onde gli fece dare (come è solito)
un biglietto, acciò lo portasse al Segretario del Magistrato degli Otto, qual viglietto
diceva, che fusse ritenuto il Latore in segrete fino a nuovo ordine. Pippo
prese il viglietto, e indovinatosi del contenuto, e parendogli duro havere a stare
in prigione in tempo di Carnevale, e sapendo, che il non portare il viglietto era
delitto da galera, andava mulinando come potesse salvare la capra, e i cavoli,
quando la fortuna, nell' andar' egli come la serpe all'incanto, gli fece capitare
innanzai un Tedesco giovanetto servitore di livrea del medesimo sig.\ Cav. Vieri suo
Padrone, alla volta del qual Tedesco andato Pippo, quali bravando disse: il
Padrone è in collera, che tu sei stato tanto a venire', perché voleva che tu portassi
questa lettera al Sig: Segretario de gli Otto, e perché è negozio di fretta,
mandava me; se bene, ho da fare assai fu in Palazzo; pigliala, e va via correndo.
Il buon Tedesco non pensando alla malizia porto la lettera, in esecuzione
degli ordini della quale il Tedesco latore fu ritenuto in carcere, e fu risposto
che S.A.S. era restata ubbidita. Pippo il dopo desinare del medesimo giorno, in
vesti da donna, e senza maschera con le sue proprie basette, e barba se ne passeggiava
il corso delle maschere, havendo d' attorno un popolo infinito. S'abbatté
a vedere questo tumulto il Sereniss. G, Duca, che passava in carrozza per
quella strada, onde spedì uno staffiere per intendere che cosa fusse. Lo staffiere tornò,
dicendo che era Pippo del Cattiglioni in maschera da donna. Ma S.A.S.
che già sapeva del viglietto, replicò: non può essere, onde il Caporale de gli staffieri
andò da per se, e tornò replicando esser veramente Pippo nel modo, che haveva
detto lo staffiere; in tanto S.A.S., s'accostò, e Pippo che gli andava incontro,
ed haveva osseruato, che S.A.S. haveva mandato due volte a veder chi egli era,
fattole una grandissima riverenza disse: \textit{Sereniss, io son io, io son'io, perché il Tedesco
m' ha fatto il servizio di portar la lettera lui; Finalmente conosco hora più che mai che
chi si fa ben volere può sperar sempre questi, e maggiori servizzj}. Il Sereniss. G. Duca
rise dell'astuzia, e ordinò che fusse scarcerato il Tedesco.

Il Sig. Cav. Bernardo fratello del sig.\ Cav. Vieri haveva presa la seconda moglie.
Questa dama volendo esser servita da Pippo per bracciere, perché egli era
huomo d'età, e vestiva di nero, e non con la livrea, come gli altri servitori di
quella Casa, pregò il suo sig.\ Consorte, che lo chiedesse al fratello, perché servisse
a lei, il sig.\ Cavaliere Vieri gli compiacque, se bene con poco suo gusto,
perché era avvezzo con lui, che fuori di quelle sui bizzarrie lo serviva raramente,
e con meno gusto di Pippo, che non avvezzo a servir dame gli pareva duro haversi
ad avvezzare in sua vecchiaia, e mal volentieri lasciava il suo padrone, la
discretezza del quale non sperava trovare in chi che sia; onde pregò la Signora,
che lo volesse lasciare al servizio, che era solito; ma la Signora non volle mai
mutarsi di proposito; per lo che Pippo si gettò alle invenzioni per liberarsene
con riputazione, e con operare, che la Signora lo licenziasse, senza che egli
commettesse mancamento. Chiamò dunque a se alcuni ragazzi, e distribuiti fra essi
alcuni pochi soldi, impose loro, che quando lo vedevano con la padrona, s'accordassero
tutti a gridare Pippo, Pippo, Ecco Pippo, e gli facessero il bordello dietro.
I ragazzi invitati al loro giuoco, e che havrebbono dato qualcosa a lui per havere
occasione di far quel chiasso, appena lo veddero uscir di casa, dando il braccio
alla Padrona, che cominciarono a strepitare, e ragunarono quivi quanta gente
era in quei contorni, e Pippo savio, senza mutarsi in faccia seguitava a dare il
braccio alla Signora, la quale vergognandosi, che il suo servitore fusse lo scherzo del
Popolo, e che egli fusse trattato come un pubblico buffone, s'affrettò di
giugnere in Chiesa, pensando, che quivi almeno dovesse fermarsi il baccano,
ma, se cessò il romore, non finì il tumulto, perché quei ragazzi standosi tutti
attorno, non gridavano per rispetto della Chiesa, ma erano cagione, che tutto il
popolo guardasse verso quella parte; per lo che la Signora per liberarsi ordinò a
Pippo, che andasse a casa, e mandasse un'altro servitore, e tornata poi a casa
le parve mill'anni render Pippo a chi gliel'havea conceduto; E così egli ritornò
al primo servizio, sicuro, che alla Signora non farebbe mai più venuta voglia di
farsi servire da lui.

Haveva il sig.\ Cav. Vieri una bella cagna da Fermo, la quale diede in cura a
Pippo dicendogli: Tien conto di questa cagna, ed avverti a non la smarrire,
perché se la smarrisci non ti aspettare altra licenza. Prese Pippo la cura della cagna,
e col trattarla bene l'avvezzò a fare mille giuochi, e se la rese così affezionata,
che era impossibile, che egli la smarrisse. Avvenne, che Pippo fu invitato
a una festa, che si dovea fare in un luogo poco lontano da Firenze, dove
era per trattenersi almeno tre giorni, onde chiese al padrone licenzia per a quel
tempo; ma non l'ottenne, Pippo senza mostrar di ciò disgusto, la mattina avanti
alla vigilia di detta festa comparve in casa senza la cagna, ed il sig.\ Cav. domandò
dov' ell'era. Pippo disse quasi piangendo: Sig.\ io non lo so,, quando io
fui vicino a case mia iersera ella cominciò a fuggire, e per molto, che io le corressi
dietro chiamandola, non fu possibile farla tornare, ne arrivarla. Replicò il
Sig.\ Cavaliere; Tu sai i patti; però va a fare i fatti tuoi, e non haver' ardire di
mettere il piede in casa nostra senza la cagna. Pippo fingendo un dirottissimo
pianto sen' usci di casa, e andò alla festa, alla quale era stato invitato, e passati
alcuni giorni in grandissima allegria se ne tornò a Firenze, e andato fuori della
porta alla Croce da uno Ortolano suo amico, al quale haveva lasciata la cagna,
se la prese, e l'infangò tutta, e le insanguinò l'ugna, acciò paresse spedata, e legatala
con una corda: la condusse al padrone, il quale veduto Pippo con la cagna
gli disse: Dove l'hai trovata? In Casentino, Illustriss. Sig., e non ci voleva altri
che me per trovare il luogo dov'ell'era fitta. Il sig.\ Cav. credette quanto disse
Pippo, il quale con tale invenzione godè la soddisfazione, che bramava. E tanto
basti per un saggio delle facezie di Pippo, il di cui intero nome, e cognome
era Filippo Bussi.

\section{Stanza LXV. \& LXVI.}
\begin{ottave}
\flagverse{65}A gire a Batistone adesse tocca\\
Gran gigante da Cigoli di quelli,\\
Che vanno a corre i ceci con la brocca\\
E batton con le pertiche i baccelli:\\
Per sue bellezze amore ha sempre in cocca\\
Per ferir Dame i dardi, ed e quadrelli,\\
Fa il Cavaliere nelle cavalcate,\\
E va spesso furiero alle nerbate.
\end{ottave}

\begin{ottave}
\flagverse{66}Cento suggetti egli ha della sua classe\\
Anch'eglino Pigmei distorti, e brutti,\\
Fanti che nacquer nelle magne basse,\\
Mi se ben son piccini, vi son tutti,\\
Mangian spinaci, arruffan le matasse,\\
Ed ha più vizzj ognun, di sei Margutti,\\
Cosa è questa che va per il suo dritto,\\
Che non è in corpo storto animo dritto
\end{ottave}

Segue Batistone Nano con una gran quantità di compagni uguali a lui; ma, se
bene son così piccoli, son tutti viziosissimi, e non possono essere altrimenti,
perché in un corpo mal fatto, di rado si trova anima ben composta.

\begin{description}
\item[BATISTONE] Questo fu un Nano levato da guardare le pecore, e condotto
a servire il Serenissimo Principe Mattias di Toscana, dove insuperbitosi, si messe
in sul posto di bello; e facendo lo spasimato di tutte le Dame, e però il Poeta dice:
\textit{Per sue bellezze Amore ha sempre in cocca Per ferir Dame i dardi, ed i quadrelli}, ed
arrivò a segno questa sua inclinazione alle dame, che per potere liberamente praticare
con esse, si contentò che il suo Serenissimo Padrone lo facesse castrare,
come seguì, ma però in burla, e stette nelle mani di Maestro Agnolo Santerelli
Castratore circa un mese, sempre credendo d'essere stato castrato: E perché
egli, non ostante che fusse di statura piccolissima imparò assai bene a cavalcare,
e maneggiare ogni cavallo aggiustatamente, supplendo con la mano a quello, in
che gli mancavano le gambe, era solito ancor egli andare nelle cavalcate dei
i Cavalieri, e però dice: \textit{Fa il Cavaliero nelle cavalcate}. Ma perché questa sorta
di Caramogi e assai sottoposta alle mazzate del padrone, ed egli ne haveva la
sua parte, però il Poeta dice; \textit{Va spesso Foriero alle mazzate}. Questo Nano dopo
la morte del Serenissimo Principe Mattias servi al Serenissimo Gran Duca in qualità
pure di Nano, ma esercitava anche la cucina segreta di S.A.S., nel qual mestiero
s'era fatto peritissimo, per lo che oltre alla buona provvisione e stipendio,
buscava gran mance; ma la Fortuna l'abbandonò in sul buono, perché essendosi
egli innamorato d'una bellissima giovane sua pari di natali, la prese per moglie,
ed in pochi giorni morì. Lo chiama \textit{Gigante da Cigoli}, e che era uno di
\textit{quelli che colgono i ceci con fa brocca}, come si fa de i fichi, e che \textit{battono i baccelli con
la pertica}, come si fa delle noci, non potendo arrivargli altrimenti. Di questo
Gigante da Cigoli, in una collinetta vicina a S.Minkato al Tedesco, si fra
le donnicciuole, una Iperbolica cantilena antica, la quale dice:
\begin{verse}
E d'una punta d' ago
Ne facea pugnale, e spada,
E di quello che gli avanzava
Ne faceva uno spuntoncin,
\end{verse}

E continova questa cantilena con altre iperboli retrograde simili per esprimere
la picciolezza di questo Gigante da Cigoli; e di qui e in uso comune il dire
Gigante da Cigoli a un Nano, che i Latini dissero \textit{Pumilio}, e noi diciamo anche
\textit{Pedina}, similitudine tratta dal giuoco della dama; \textit{Scricciolo} da un'uccello
piccolissimo di questo nome, \textit{Pimmeo} dalla voce Greca \textit{Pygmaios}, che significa
dell'altezza d'un pugno. I Greci dicevano \textit{Nanus, Pusillus quantus Molo}, ed altre
volte \textit{gutta}; ed un Pedante lo chiamo \textit{Titivillitium Scarabei umbra}. Famiano
Strada nelle sue Prolusioni, parlando d' un Nano dice: \textit{Fungino hic genere est,
  capite se totum tegit}, Ed altrove, pure nello stesso proposito dice: \textit{Hominus indicium,
Somninm hominis, salillum animae}.

\item[BROCCA] Voce, che viene dal Greco \textit{Brochos} secondo il Monosino, e secondo
  altri dal Greco \textit{Prochoos}; il che è più verisimile, essendo questo vaso da
  acqua, e quello vaso da vino; e vuol dire un vaso di terra per uso di portare
  acqua, e però detto \textit{Aydria}, e noi lo chiamiamo brocca;, Chiamasi brocca
  ancora uno strumento fatto di canna rifessa in più parti; se quali allargate, e rintessute
  con salci, formano come una piramide a rovescio, e di tale strumento
  fermato in cima a una pertica, ci serviamo per corre i fichi, quando non si possono
  arrivar con le mani; e di questa brocca dice nel presente luogo.

\item[FVRIERO] Si dice colui, che va innanzi a preparare gli alloggi nel viaggiare
  che fa un' Esercito, o altra gente in buon numero. Lat. \textit{metator mansionum}.
  In Latino barbaro dicesi \textit{fodrarius} da \textit{fodrum} voce che vien dal Germanico, la
  quale in buon Latino si direbbe \textit{alimentum}, \textit{pabulum}, \textit{annona}; Onde \textit{Foraggio}, e
  \textit{Foraggiare}, \textit{Provisione di guerra}, e \textit{provvedere l'esercito}. Tutto ciò si osservò dal
  Ferrari nelle Origini alle voci \textit{Foraggio}, e \textit{Foriere}, Ma erra quando piglia \textit{Friere
    dello spedale}, che si trova in Gio: Villani lib. 8. c. 95. per accorciato da \textit{Foriere},
  quasi sia \textit{Provisor hospitij} poiché quivi, si come appresso al Bocc. Nov, 92. significa
  \textit{frate} dal Franzese \textit{frere} cone si domandano anche oggi i Cavalieri di
  Malta. Qui si serve della voce \textit{Furiero} per intender \textit{furia} che suona quantità,
  come dicemmo sopra in questo Cant. stan. 50. e vuol intendere, che questo Nano
  spesso toccava qualche furia, cioè quantità di nerbate. Vedi sotto \cstan[9]{49}.

\item[PIMMEI] Erano popoli nani, che habitavano nell'ultime parti dell' Indie,
  i quali crescevano fino all' altezza al più d' un braccio, e le loro mogli di cinque
  anni partorivano, ed otto erano vecchie. Di questi fa menzione Plinio
  \libcap[4]{11}. ove dice i barbari chiamarli Cathizi, e \libcap[7]{2}. Costoro per esser
  così piccoli erano infestati, e rapiti dalle Gru, onde per difendersi andavano
  armati di frecce; e cavalcando sopra alle capre in grandissime schiere a guastare
  i loro nidi, e romper loro l'uova. Di questi parla Giuvenale sat. 13. dicendo.
  \begin{verse}
    Ad subitas Thracum volucres, nubemque sonoram
    Pygmaeus parvis currit bellator in armis.
    Mox impar hosti raptusque per aera curvis
    Unguibus a saeva fertur grue: Si videas hoc
    Gentibus in nostris, risu quatiare, sed illic,
    Quamquam eadem alssidue spectemur, praelia ridet
    Nemo, ubi tota cohors pede non est altior uno.
  \end{verse}

\item[NELLE magne basse] Intende che sono di statura bassa, se ben par che dica
  sieno nati nella bassa Alemagna. Lat. \textit{Germania inferior},

\item[SE bene e' son piccini, vi son tutti] Benché piccoli hanno malizia quanto un
  grande. \textit{Tydeus corpore, animo vero Hercules}; da Omero, il quale descrive Tideo
  il padre di Diomede piccolo si di statura, ma gagliardo.

\item[MARGVTTE] Che Nano fusse costui, e quanto sagace, e scellerato, vedilo
  nel Pulci nel suo Poema intitolato il Morgante. Questo nome di Margutte forse
  fu finto dal Pulci a similitudine di \textit{Margite}, Personaggio famoso per la sua scempiataggine,
  il quale fa il suggetto d'un intero Poema burlesco di Omero; e ciò
  poté avere imparato il Pulci dal suo dotto amico messer Agnolo da Montepulciano.

\item[NON è in corpo storto anima dritta] Non è in corpo mal fatto, animo ben
  composto, giusto, e che tiri al buono; che tanto significa la voce dritto in questo
  luogo. Si dice anche: \textit{Un segnato da Dio, non fu mai buono}: (alludendo per
  avventura a Caino, Gen. c. 4. vers. 15.: quali che quel tale sia in un certo modo
  contrassegnato, affine, che ognuno, che lo vede si guardi) qual sentenza è
  praticata comunemente, e si vede da i seguenti versi maccheronici.
  \begin{verse}
    Nulla fides gobbis, \& parum credite zoppis,
    Si guercius bonus est, inter miracula scribe.
  \end{verse}
  Un' altro Poeta in questo proposito disse: \textit{Chiude un' anima bigia un corpo nero}.
  Che huomo bigio intendiamo huomo cattivo, di poca coscienza, e manco religione.
  Marziale. \textit{Crine ruber, niger ore, brevis pede, lumine laesus Rem magnam
    praestas Zoile, si bonus es}. Quel Tersite, che quanto sconcio di viso, e scontraffatto
  nel corpo, altrettanto era brutto nell'animo, e di costumi orgogliosi, e insopportabili;
  vien descritto da Omero al 2. dell' Iliade ( secondo la traduzione di
  Pietro la Badessa Messinese, stampata in Padova l'anno 1564.)
  \begin{verse}
    Lusco a' un' occhio, e d' un pié zoppo, e stretto
    Negli omeri, che gobbi ha infin' al collo;
    Aguzzo il capo, e 'l capel crespo, e raro,
    Sucido, e ner, lentiginoso, e marcio,
  \end{verse}
\end{description}

\section{Stanza LXVII.}

\begin{ottave}
\flagverse{67}Piena di sudiciume, e di strambelli \\
Gran gente mena qua Palamidone, \\
Ch'il giorno vanne a Carpi, ed a Borselli, \\
E la notte al Bargel porta il Lancione, \\
Maestro de' Bianti, e de' Monelli,\\
E veste la corazza da bastone,\\
Perch'egli quant'ogni altro suo allievo\\
È tutto il dì figura di rilievo.
\end{ottave}

Palamidone conduce seco una quantità di birboni, stracciati, e sudici come
era lui. Questo fu un guidone mezzo matto, ma tutto tristo, ed al maggior segno
birbone, il quale faceva servizio a' carcerati, e perché continovamente
brontolava, dicendo di pazze scioccherie, haveva sempre dietro una gran quantità di
ragazzi che lo facevano stizzire. La notte per guadagnar qualcosa portava dietro
al Capitano, o Caporale de' Birri un'arme in asta solita portarsi dalla Famiglia
del bargello, quando la notte va facendo la guardia, la quale arme è da noi
detta \textit{lancione}. Ma che egli rubasse non posso crederlo, perché assolutamente non
havea tanto giudizio, e stimo che il Poeta dica questo nel presente luogo, e altrove
per descriverlo per uno di quei furfanti, de' quali si può credere ogni ribalderia.
Palamidone e accrescitivo di \textit{Palamides}, Eroe noto nella guerra Troiana,
secondo la pronunzia Greca più moderna dicesi \textit{Palamide}, e non \textit{Palamedes}; onde
è fatto il soprannome di \textit{Palamidone}; che significa un lungo e sottile, come un
palo, una persona grande di statura.

\begin{description}
\item[ANDARE A Carpi, ed a Borselli] Carpi è un Principato in Italia notissimo; e
  Borselli è un luogo sul Fiorentino, e scherzando con questi due nomi \textit{Carpi} intendiamo
  carpire, cioè rubare, ed a \textit{Borselli}, cioè alle borse per rubare. Aristofane
  Poeta Greco nella Commedia intitolata i Cavalieri, citato dal Monosini
  nel \textit{Flos Italicae linguae}, (ove egli tocca la maniera di parlare Fiorentina; \textit{E'
    piglierebbe per San Giovanni}, usata anche dal nostro Poeta;) dice così: \textit{manus in
    Actolis habet}.  Vuol dire: \textit{sempre chiede, ed è apparecchiato a pigliare}; scherzando sul
  nome di certi popoli chiamati \textit{Aetoli}, per  l'allusione che ha questa voce alla parola
  \textit{atein} che significa chiedere.

\item[PORTARE il Lancione al Bargello] Questo mestiero solito farli da birro novizio,
  lo faceva alle volte Palamidone. come s'è detto.

\item[BIANTI] Si trova una specie di Bricconi, e Vagabondi che vanno buscando
  danari con invenzioni, come si vede da un libretto intitolato \textit{Sferza de' Bianti, ec,}
  E si dicono anche Monelli; se ben veramente per monelli intendiamo quei poveri,
  che si fingono stroppiati, malati, impiagati, o morti dal freddo per muover
  le persone a far loro elemosine, donde poi diciamo \textit{far il monello} quel ragazzo,
  che havendo toccate leggiermente delle busse dal Maestro, o da altri, mette a
  sogquadro il vicinato con le strida per mostrare d' essere stato dalle busse stropiato,
  ed in vero non ha mal nessuno, che si dice anche \textit{far marina}: vedi sopra
  \cstan[1]{37}. alla voce \textit{soffiano}, e sotto \cstan[4]{8}. Di questi intende il Persiani
  nei seguenti versi.
  \begin{verse}
    \backspace Signor non so se voi sapere il bando
    Di chiuder tutti dentro a' Mendicanti
    Mascalzon, vagabondi, e malestanti.
    \backspace Che vanno per le strade mendicando,
    Io che sono in arnese tanto male
    Mi ritrovo in grandissimo viluppo;
    Temo esser preso in vece d'un Gaiuppo,
    E finir la mia vita allo Spedale.
  \end{verse}

\item[VESTE la corazza da bastone] E' armato a bastonate, veste un' armatura da
  difenderlo dalle bastonate; s'intende che è sottoposto a toccare spesso delle
  bastonate.

\item[RILEVARE] Intendiamo buscare, conseguire, ottenere. Petr. Canz. 22.
  \begin{verse}
    Il sempre sospirar nulla rilieva.
  \end{verse}

  Onde se bene \textit{figura di rilievo} vuol dire statua di marmo, o di altro materiale,
  noi incendiamo \textit{rilevare}, cioè \textit{buscare} e qui intende \textit{buscar mazzate}. Il verbo
  rilevare piglia questo significato da rilievo, che sono gli avanzi delle mense de' grandi,
  quali avanzi si buscano per lo più da coloro che servono a tavola, donde diciamo
  Viver di rilievi che vuol dir Campare d' avanzi. Vedi sotto C., 5. stan. 47.
  Franco Sacch. Nov. 154. \textit{Quando la crostata fu mangiata tutta, senza far rilievo ne
    meno de' topi}, \textit{Rilevare} vuol dir Quello esprimere che fanno delle parole i ragazzi,
  quando imparano a compitare.

\section{STANZA LXVIII.}

\begin{ottave}
\flagverse{68}Comparisce fra tanto un carro in piazza\\
Da Farfarel tirato, e Barbariccia \\
Ubbidiente al cenno della mazza\\
Soda, nocchinta, ruvida, e mafficcia. \\
Con che la formidabil Martinazza\\
A lor, ch'è ch'è, le costole stropiccia,\\
E quei Demonj in forma di Camozza\\
Van tirando a battuta la carrozza.
\end{ottave}

In tanto, che si fa la mostra de' soldati di Malmantile comparisce in piazza
un carro tirato da due Demonj in forma di capra salvatica, che questo vuol dir
\textit{camozza}, la quale per lo più si trova ne i monti del Tirolo. Plin, \libcap[12]{37}
la chiama \textit{Rupicapra}. I nostri antichi dissero \textit{Stambecco}, il Lat. \textit{ibex}.

\item[FARFARELLO, e Barbariccia], Nomi di due Demonj dal nostro Poeta cavati
  da Dante, del significato de' quali nomi vedi gli Spositori sopra il medesimo
  Dante.

\item[NOCCHIVTA] Piena di nocchi, che sono quei piccioli rilevati come bolle,
  i quali si veggono per lo più ne i bastoni di pruno, di sorbo, ec, che gli rendono
  ruvidi, e li chiamamo ancora \textit{nodi}, come fanno i Latini.

\item[MASSICCE] Intendiamo tutte quelle cose, che dal peso mostrano esser fatte
  di materia stabile, e solida, e non vote, o vane, o in altra maniera fragili, o
  deboli.

\item[CH'è ch'è] Ad ora ad ora, di quando in quando, spesso.

\item[STROPICCIARE] Fregar qualcola con panno, o altro, ed i Latini \textit{Perfricare}.
  Forse è corrotto da \textit{stoppicciare}, che pare si dovesse dire, da stoppa, o stoppaccio,
  con che per lo più si stropicciano gli arnesi per liberargli dalla polvere. Ma
  \textit{stropicciar le costole a uno} vuol dire \textit{Bastonare uno}.

\item[TIRANO la carrozza a battuta] Non a battuta di musica, ma a battuta della
  mazza, con la quale Martinazza la bastona.
\end{description}

\section{STANZA LXIX — LXXI.}
\begin{ottave}
\flagverse{69}Costei è quella strega maliarda,\\
Che manda i cavallucci a Tentennino,\\
Ed egli un punto a comparir non tarda\\
Quand'ella fa lo staccio, o il pentolino,\\
Come quand'ella si unge, e s'inzavarda\\
Tutt'ignuda nel canto del cammino,\\
Per andar col Barbuto sotto il mento\\
Con la granata accesa a Benevento.
\end{ottave}

\begin{ottave}
\flagverse{70}Ove la notte al noce eran concorse\\
Tutte le Streghe anch'esse sul caprone,\\
I Diavoli col Bau, le Biliorse\\
A ballare, e cantare, e far tempone;\\
Ma quando presso al dì l'ora trascorse\\
Fa di mestieri battere il taccone\\
Come a costei, ch'or viensene di punta,\\
E in su quel carro nel Castello è giunta.
\end{ottave}

\begin{ottave}
\flagverse{71}E la cagion si è, ch'ella ne vada\\
Adesso a casa tutta in caccia, e in furia,\\
L'haver veduto dentro alla guastada\\
Un segno, che le ha data cattiv'uria;\\
Perché vi scorse una sanguigna spada,\\
C'alla sua patria minacciava ingiuria;\\
Perciè, se nulla fusse di quel regno,\\
Ne viene anch'essa a dar il suo disegno.
\end{ottave}

Martinazza è una di quelle streghe, le quali costringono il Diavolo con fare
lo staccio\footnote{stàccio s.m., arnese da cucina, simile al colino. da setaccio, per sincope.}, e il pentolino, e con ungersi per farsi portare a Benevento al congresso
de' Diavoli sotto il noce: Questa Martinazza adesso, si fa riportare furiosamente
da quei Demonj a Malmantile, perché ha veduto nella caraffa una spada
sanguigna, che le presagisce la caduta di Malmantile, onde vi si vuol trovare
ancor'essa per dare il suo aiuto. Questo nome di Martinazza è nome a caso; E
quella strega, e stregherie son tutte dal Poeta dette per accennare l'opinione d'alcune
donnicciuole, le quali portate dall'illusioni diaboliche, si danno a credere
d'havere effettivo commerzio col Diavolo.
\begin{description}
\item[STREGA] Vedi sopra \cstan[2]{11}. Viene da \textit{strix} uccello notturno così
  detto a \textit{stridendo}, secondo Ovid, fast. 6.
  \begin{verse}
    Est illis strigibus nomen, sed nominis huius,
    Causa, quod horrenda stridere nocte solent.
  \end{verse}
E questo uccello (che forse era l'Arpia, ma Plinio, dice, che non si sa qual si
fosse) credevano gli antichi più superstiziosi, che rapisse i bambini dalle culle:
\textit{Et ab huius avis nocumento striges Latini appellabant mulieres puellos fascinantes suo contactu}.
E di qui ancor noi le chiamiamo streghe, che tanto vale quanto \textit{maliarde}
da far malie, fattucchierie, ed incantesimi, e però chiamate ancora \textit{Veneficae}.

\item[MANDARE un cavallucio] Mandare una citazione, cioè chimare uno in
  giudizio criminale con polizza. E queste polizze de' Giudizzj Criminali in Firenze
  si dicono cavallucci a differenza di quelle de' giudizzj Civili, che si chiamano
  Citazioni; e questo nelle polizze criminali è stampata l'impresa, o
  contrassegno del Magistrato criminale, che è un' Huomo a cavallo armato; qual
  contrassegno è chiamato comunemente Cavalluccio.

\item[TENTENNINO] Nome dato dalle nostre donne al Demonio per non lo chiamare
  Diavolo; quali tentatore; col qual nome è nominato presso San Matteo
  Cap, Vers. 3.

\item[FA lo staccio, e il pentolino] Favoleggiano, che quelle donne Maliarde, e Streghe,
  che habbiamo detto, sappiano fare diversi incantesimi per ritrovare cose
  perdute, e per ottenere altri loro intenti, e fra questi incantesimi \textit{fare lo staccio}
  o \textit{il Pentolino}, o \textit{la caraffa}; sì che dicendo \textit{Fa lo staccio, e il pentolino} intende
  fa incantesimi. Quei che indovinano per via di staccio sono detti dai Greci \textit{Coscinomanteis}.

\item[COME quand'ella s'unge, e s'inzavarda] Inzavardare, è uno impiaftrare con
  materia morbida, e viscosa, atta a distendere come il lardo. Il Poeta seguita,
  la vana, e superstiziosa opinione, che queste tali donne vadano ogni tanti
  giorni al congresso de' Diavoli sotto il Noce di Benevento: \textit{Ove la notte al noce
    eran concorse}; al qual luogo dicono esser portate dal Diavolo in forma di caprone,
  che questo intende \textit{il Barbuto sotto al mento}, e cavate dalle loro case per la gola
  del cammino (e però dice \textit{nel canto del cammino}) dal medesimo diavolo forzato a
  far tal funzione da quegli untumi, che dice essersi messa addosso la medesima
  donna; la quale poi a detto congresso \textit{fa tempone}, cioè si da buon tempo; si piglia
  tutti quei piaceri, che le vengono in fantasia quella notte; Ma sul far del giorno
  le convien partire, e il Diavolo in un baleno la riporta al suo paese. Tale opinione
  hanno simili scimunite; ed o sia per effetto di matrice, o pure per opra
  del Diavolo, che per illusione faccia loro apparir per vere tutte quelle scioccherie,
  che esse si fingono nella testa, l'effetto è, che esse si credono d'esser'andate
  veramente a Benevento, ed essere state riportate dal Demonio al loro paese,
  quando effettivamente non si sono mosse del letto.

\item[GRANATA] È un mazzetto di scope, o d' altra cosa simile, che s' adopra
  spazzare, e ripulire le stanze. E con queste granate accese in mano dicono,
  che tali streghe vadano cavalcando sopra un Caprone al detto Noce di Benevento.

\item[BAU, e Biliorse] Questi nomi bau, biliorse, orco, befana, versiera, e altri
  simili, sono tutti inventati dalle Balie per spaventare i bambini, e rendergli ubbidienti,
  persuadendo loro, che questi sieno spiriti infernali, e però il Poeta numera
  fra i Diavoli il Bau, e le Biliorse, per accomodarsi alla capacita de' Fanciulli,
  per li quali professa d'haver composta la presente opera. Vedi sopra \cstan[2]{50}.
  I Greci il cembalo per chetare i bambini dicono \textit{Catabau}.

\item[FAR tempone] Darsi bel tempo; Stare allegramente, pigliandosi tutti quei
  gusti che uno può, e sa pigliarsi, che diciamo anche \textit{sguazzare}, \textit{trionfare}, \textit{far
buona cera}, \textit{Genio indulgere}, \textit{litare Genio}, dissero i Latini. La Compagnia della
Lesina insegnando, in qual luogo si deva pigliare la casa per risparmiare, dice:
\textit{Vorriano le nostre case esser in una quasi dall' altre separata contrada, lontana da vie, e
piazze pubbliche, dove all' occasioni si festeggi, e si faccia trebbi, e tempone}.

\item[BATTER il taccone] È lo stesso, che \textit{batter la calcosa}, detto sopra questo
C. stan. 60, cioè camminar via; andarsene. Si dice anche \textit{battersela}; E \textit{taccone} si
dice il suolo della scarpa, cioè quella parte, che posa in terra. In questo senso trovasi
nei Latini \textit{solum vertere}.

\item[VENIR di punta] Venir con velocita, a dirittura; che diciamo anche \textit{venir di
vela}. Vedi sotto \cstan[6]{10}, Credo sia originato dalle barche, le quali si dice
\textit{venir di punta} quando vengono a dirittura senza volteggiare.

\item[IN caccia, e in furia] Cioè in fretta, frettolosamente, e con furia, come fanno
  coloro, che son cacciati; che però diciamo; \textit{Corre, che par che egli habbia i birri
dietro}, \textit{Incedit quasi in fugam versus}.

\item[GVASTADA] Specie di vaso di vetro per uso di conservarvi liquori, ed è lo
  stesso, che caraffa dai Latini detta \textit{Phiala}, L'Autore disse sopra nell'ottava
  antecedente, che Martinazza era solita fare lo \textit{Staccio}, e il \textit{Pentolino}, e qui dice
  la \textit{Guastada}; queste maliarde, e streghe empiono di superstiziosi liquori una caraffa,
  o guastada, e facendovi mirar dentro da un fanciullo innocente, gli fanno
  dire di vedervi dentro quel che hanno desiderio di sapere, e tutto per ingannare
  le persone semplici, e cavar loro denari di mano. Questo indovinare per via d'acqua,
  fu anticamente presso i Persiani, e da' Greci si chiama \textit{Hydromantia}. Da
  questo habbiamo un detto \textit{Gli ha il diavolo nell'ampolla} per intendere: Costui indovina
  ogni cosa.

\item[CATTIV' uria] Cattivo augurio. Questa voce Vria corrotta da augurio usata
  per lo più dalle donnicciuole, detta senza aggiunta di cattiva, o buona, s'intende
  cosa, che non piaccia. \textit{La tal cosa mi dà uria}: e s'intende mi dà fastidio,
  mi da impedimento, mi da noia; da che si può credere che sia usata in vece di
  uggia, che pure vuol dir noia, fastidio, impedimento, ec. o forse in vece d'\textit{ubbia},
  che suona lo stesso, che \textit{uggia}, o forse in vece d'\textit{ombra}, che è il medesimo,
  quando vale per impedimento, \textit{la tal cosa mi dà ombra}, per \textit{la tal cosa mi dà noia}, ec.
  Sì che \textit{uria}, \textit{uggia}, \textit{ubbia}, ed \textit{ombra} suonano tutte lo stesso; \textit{uría}, e \textit{ubbía} sono
  usate per lo più dalle donne, e l' altre son più comuni. Si potrebbe anche dire
  secondo il Monosino, che la voce \textit{uria} venisse dal greco \textit{vria}, che suona vento
  prospero, e che sì come habbiamo per costume di dire buona, o o cattiva \textit{sorte},
  quantunque \textit{sorte} significhi assolutamente bene, e felicità; così habbiamo per costume
  di dire buona, o cattiva \textit{Vria}, quantunque \textit{Vria} significhi sempre felicità, secondo il
  Greco \textit{Vria}. Nello stesso modo, benché presso i Francesi \textit{heur} significhi sorte, felicità;
  voce a loro derivata similmente dal Latino \textit{augurium}; dicono \textit{bonheur}, e
  \textit{malheur}, quali \textit{buona}, e \textit{cattiva uria}, cioè buona, e mala ventura; e però volendoci
  servir bene di questa parola Uria, come vocabolo di mezzo, dovremmo aggiungerci
  buona, o cattiva, e non dirla assolutamente, e senza detta aggiunta,
  come habbiamo accennato, che molti se ne servono; ma l'uso ci libera da tali
  astruse stiracchiature. '

\item[SE nulla fusse] Per tutto quel che potesse succedere, Se accadesse qualche disgrazia.
  I Latini in un simil modo per isfuggire il cattivo augurio, e non nominare
  cosa infausta, come è la morte, dicevano: \textit{Si quid patiar}. \textit{Si quid mihi
    humanitus acciderit}, Se Dio facesse altro di me, con tutto ciò, ec.h

\item[NE viene anch' essa a dare il suo disegno] Con queste parole mostra l'Autore
  quanta gelosia haveva Martinazza di non perdere l'autorita, che teneva sopr' a
  Malmantile, ed il sospetto di non esser levata dal grado di Salamistra, che
  godeva, come accennammo sopra in \cstan{54}.
\end{description}

\section{Stanza LXXII. — LXXIV.}

\begin{ottave}
\flagverse{72}Fuggì tutta la gente spaventata\\
All'apparir dell'orrido spettacolo,\\
La piazza fu in un' attimo spazzata,\\
Pur un non vi rimase per miracolo,\\
Così correndo ognuno all'impazzata\\
Si fé l'un l'altro alla carriera ostacolo;\\
Chi dà un'urton, quell'altro dà un tracollo,\\
Chi batte il capo, e chi si rompe il collo.
\end{ottave}

\begin{ottave}
\flagverse{73}Figuriamci vedere un sacco pieno\\
Di zucche, o di popon sopr' a un giumento,\\
Che rottasi la corda, in un baleno\\
Ruzzolan tutti fuor sul pavimento,\\
E nell'urtarsi batton sul terreno:\\
Chi si perquota, e chi s'infranga drento\\
Chi si sbucci in un sasso, e chi s'intrida,\\
Ed un altro in due parti si divida.
\end{ottave}

\begin{ottave}
\flagverse{74}Così fa quella razza di coniglio, \\
Che nel fuggir la vista di quel cocchio \\
Chi se rompe la bocca, o fende un ciglio\\
E chi si torce un piede, e chi un ginocchio; \\
A tal che in veder quello scompiglio,\\
Io ho ben preso (dice) qui lo scrocchio,\\
Mentre a costor così comparir volli:\\
Sapeva pur chi erano i miei polli,
\end{ottave}

Il Poeta descrive assai vagamente il timore, e lo spavento, che eatro addosso a
quei di Malmantile per la vista del Carro di Martinazza, la quale vedendo coloro
così spaventati, si pente d'esser quivi arrivata in quella guisa.

\begin{description}
\item[IN un attimo] In un momento. Corrotto da atomo. Si dice anche \textit{In un baleno} ,come
nell'ottava 73. seguente. \textit{In un batter d'occhio}. V. sotto \cstan[10]{42}. dal Lat.
\textit{Ictu oculi}. \textit{En atomo} dissero i Greci. Dante Inf. C. 22. \textit{Subito, e spesso a guisa di baleno}.

\item[NON ve ne rimase un per miracolo] Fuggiron tutti, che non ve ne restò pur'
  uno. Tanto esprimeva se havesse detto: \textit{Non ve ne restò pur' uno}, Ma col dire
  \textit{miracolo} da maggior' emfasi, e seguita l'uso; e vuol dire sarebbe stato creduto miracolo
  se un solo vi fusse restato.

\item[ALL'impazzata] A caso; Come fanno i pazzi, cio senza considerar quello
  che facevano, o dove essi andavano. È il latino \textit{perperam}.

\item[URTONE] Percossa che si dà con tutta la vita in un' altra persona, o in un
  muro, o altrove, ed è lo stesso, che Spinta, ne vi so fare altra differenza se
  non che \textit{Urtare} vuol dir percuotere a caso, ed è il Latino \textit{offendere}; e \textit{Spingere}
  vuol dir Mandar uno innanzi, o indietro con violenza, ed è il latino \textit{impellere};
  Ma nondimeno \textit{urtone}, \textit{spinta} si pigliano l' uno per l'altro, se bene non si direbbe
  Dare una spinta in un muro, o altra cosa immobile, che fatta mobile come
  farebbe un muro sciolto per farlo rovinare, si direbbe Dare una spinta. A
  un'albero quasi reciso da piede per atterrarlo, si direbbe Dar la spinta per farlo
  cadere, ec.

\item[TRACOLLO] Accennamento di cadere. \textit{Extra collum pedis ire}; o pure detto
  così quasi \textit{Tracrello}. Vocabolario della Crusca. Tracollato addiettivo da tracollare,
  che vale lasciar' andar giù il capo per sonno, o simile accidente.

\item[GIUMENTO] Si dice propriamente l'asino  benché s'intenda anche ogni bestiaccia
  da soma. Così presso i Latini: Quello che in S. Gio, cap. 12, è chiamato
  \textit{pullus afinae}, in S. Matteo cap, 21, è detto \textit{pullus filius subiugalis}, \textit{Puledro},
  \textit{figliuolo della giumenta}.

\item[RVZZOLARE] Girare per terra; che diciamo anche Rotolare.

\item[INFRANGERSI] Sflagellarsi, ammaccarsi, disfarsi. Vedi sotto C. 4. stan.
76. C.11, stan. 12.

\item[RAZZA di Coniglio] Gente timida, e codarda. Si dice \textit{poltrone come un coniglio},
  perché questo animale, che è specie di lepre; come quella è timidissimo.

\item[PIGLIAR lo scrocchio] Ingannarsi, Far' errore. Lo sono stato a cena con voi,
  credendo di star bene, ma ho preso lo scrocchio; cioè mi sono ingannato, perché
  sono stato male. Il proprio significato della parola, scrocchio è quando uno per
  trovar danari, piglia a credenza una mercanzia per venticinque scudi, la quale
  non ne vale venti, e poi la vende a quindici, e questo si dice pigliar lo scrocchio.
  Plauto disse: \textit{Emere coeca, vendere oculata die}. Vedi sotto \cstan[6]{60}. E da
  questo, quando noi facciamo una cosa, che non ci torna poi bene, ne in nostro
  utile, e gusto, ma più tosto ci è di danno, si dice \textit{pigliar lo scrocchio}.

\item[SAPEVO chi erano i miei polli] Sapevo di che qualità eran costoro, è il Latino
  \textit{Cognosco oves meas}.
\end{description}

\section{Stanza LXXV. \& LXXVI.}
\begin{ottave}
\flagverse{75}Scese dal carro poi per impedire \\
Così gran fuga, e rovinosa fola;\\
Ma quei viè più si studiano a fuggire, \\
E mostra ognun se rotte ha in pié le suola, \\
Chi finalmente, come si suol dire \\
Chi corre corre, ma chi fugge vola, \\
Ond'ella, ben che adopri ogni potere, \\
Vede che fara tordo a rimanere.
\end{ottave}

\begin{ottave}
\flagverse{76}Perciò si ferma strambasciata, e stracca,\\
Ritorna indietro, ed un de' suoi caproni\\
Dalla Carretta subito distacca,\\
E gli si lancia addossa a cavalcioni;\\
Così correndo tutta si rinsacca,\\
Perché quel Diavol vanne balzelloni;\\
Pur (dicendo: arri là, carne cattiva)\\
Lo fruga sì, ch'al fin la ciurma arriva.
\end{ottave}

Martinazza scese dal carro per fermar quella gente, che fuggiva, e si messe a
correr lor dietro, ma allora sì, che coloro fuggivano, onde ella montata sopr'a
uno di quei caproni al fine gli arrivò. E qui termina il terzo Cantare.

\begin{description}
\item[FOLA] Quantità di popolo, che furiosamente corre a qualche luogo; Traslato
  da i Cavalieri, che giostrano, che dopo, che si sono soddisfatti li concorrenti
  a uno per volta a giostrare, in ultimo corrono al Saracino (così chiamano una
  mezza figura, o busto, di Moro, o Saracino, fatta di legno, e fitta in un palo)
  corrono dico al Saracino tutti in truppa, uno però dopo l'altro, e questo dicono
  \textit{far la fola}, In Latino potrebbe dirsi: \textit{exerceri ad palum}. Vegezio de re militari
  \libcap[1]{14}. \textit{Tyro, qui cum clava exercetur ad palum, bastilia quoque ponderis
  gravioris, quam vera futura sunt iacula, adversus illum palum tamquam adversus hominem
  iactare compellitur}. E si dice \textit{fola}, o \textit{folata} d'uccelli, di popolo, ec, per
  intender di cose che velocemcate si muovono.in quantità, e presto finiscono. \textit{Folata}
  di vento, Studiare \textit{a folate}. Lavorar a folate, ec, Forse meglio \textit{folla}, che significa
  quel che i Latini dicono \textit{Magna hominum vis, vel turba, aut summa frequentia
    hominum}, Sì come noi dal calcare le strade, che fa il popolo e dallo esser calcati,
  e stretti, diciamo Una molticudine numerosa di gente, una gran \textit{calca}: così i
  Franzeei nella lor lingua la dicono \textit{foule}, cioè folla dal verbo \textit{fouler}, calpestare,
  calcare. Da \textit{folla} abbiamo fatto \textit{Affollarsi}, e \textit{Folto}, denso, calcato; Onde
  \textit{Afoltarsi}, \textit{far furia}, \textit{far pressa}: lo stesso quasi che \textit{Affollarsi} tutto derivando per
  avventura dal Latino \textit{follis}, nel quale sta l'aria serrata in modo, che più non ve ne può
  capire.

\item[STVDIANDOSI] Il verbo \textit{studiarsi} per affaticarsi a far presto, o spedire
  una cosa, che diciamo anche menar le mani. Per esempio: studiatevi, perché il
  tempo è breve, e non finirete, se non fate presto. Qui intende s'affaticavano a
  fuggire. \textit{Operi instare}: al che s'adatterebbe i verbo \textit{incumbo}, \textit{laboro}, ed anche
  \textit{studeo}, e questo dal Greco \textit{speudo}, \textit{affrettarsi}. Nel Salmo: \textit{Domine ad adivandum,
    me festina}. \textit{Signore Iddio, studiati d'aiutarmi}. Orazio.  \textit{Sic festinanti semper
    locupletior obstat}, \textit{a colui che si studia d'arricchire il più ricco dà impaccio}.

\item[MOSTRAR le suola delle scarpe] Corser velocemente; perché così s'alzano
  assai i piedi, e si mostrano le suola delle scarpe. I Greci pure dicevano in questo
  proposito \textit{Cavum pedis ostendere}, Si dice anche \textit{Battere il taccone}, che vedemmo
  sopra in \cstan{79}.

\item[CHI corre corre, ma chi fugge vola] Detto sentenzioso, che significa, che molto
  più forte corre quello, che è perseguitato, che non corre colui, che lo perseguita,
  perché la paura gli mette l'ali a' piedi, e per questo dice \textit{Chi fugge vola}.
  Vergilio disse: \textit{Pedibus timor addidit alas}, e Dante Inf.C. 22.
  \begin{verse}
    E poco valse, che l' ali al sospetto,
    Non potero avanzar. \makebox[3em]{\dotfill}
  \end{verse}
  Intendendo, che il gran timore, che hebbe del Demonio quel dannato, lo fece
  esser più veloce, che l' ali di quel Demonio, che gli correva dietro. Della parola
  \textit{Fugit} spiegantissima della velocità appresso Vergilio, vedi Seneca Epist, 108.

\item[FARE tordo a rimanere] Cioè rimarra a dietro, e non arrivera quella canaglia.
  Il giuoco de' tordi ha qualche similitudine con l'Amilla de' Greci, \textit{Quia de
    certo iactu inter ludentes certemen est}, come dice il Buleng. de Ludis Veterum cap.
  14., e la gara si dice in Greco \textit{amilla}. Nell'Amilla si tirava una palla dentro a
  un segno, o circolo, e colui perdeva, la di cui palla usciva, o non entrava nel
  circolo. Nel tordo non si fa ne segno, ne circolo, ma si tira una piccola palla,
  (da noi a distinzione dell'altre palle detta \textit{grillo}, come vedremo sotto \cstan[6]{22})
  e colui, che la tira dice: \textit{A passare}, cioè a passare con la palla il detto
  grillo, o a rimanere, cioè restar con la detta palla di qua dal detto grillo; così
  tirando ciascuno, s'ingegna di passare, o rimanere il più vicino a detto grillo, che
  egli può; perché chi meno lo passa, o meno addietro gli rimane vince la posta,
  ed a quelli, che non passano, o non rimangono, quando devon rimanere, o passare,
  vince il doppio, e questi perdenti si chiamano Tordi, e sono di tre sorte,
  perché tre sono i casi del tiro; cioè Tordo a passare è quello, che passa di là dal
  grillo quando deve rimanere. Tordo a rimanere quello che rimane di qua dal
  grillo, quando deve passare. E Tordo semplicemente si dice quello, la di cui palla
  resta in dirittura del grillo per banda, e questo da alcuni si fa che non vinca, ne perda,
  da alcuni, che perda solo la metà degli altri tordi, se è più lontano dal grillo
  di quello che vince, e se è più vicino non perde; da alcuni gli è permesso ritirare
  fino a tre volte, quando però sempre resti in dette tre volte nella medesima
  dirittura del grillo; e quando non passi, o non rimanga perde una sola posta: e
  sempre s'intenda passata, o rimasta la palla quando fra essa, e il grillo possa
  interporsi un filo in squadro, se però non 1o tocchi non per banda, ma per quella
  parte, dove ha da rimanere, o restare; e tutto si fa secondo le convenzioni, e
  patti. Questo giuoco per lo più è usato da' ragazzi, o dagl'infimi bottegai di
  Firenze; i quali nei giorni delle feste, uscendo dalla Città per andar' a pigliar'
  aria nel camminare giuocano a questo giuoco, e segnano i danari di mano in mano
  a chi perde, e quando n' hanno segnati tanti, che servan loro per comprar da
  bere, e da mangiare, si fermano alla prima Osteria, e quivi ognuno paga quella
  quantità di danaro, che ha perduto. Hor tornando a proposito dice, che Martinazza
  \textit{farà tordo a rimanere}, ed intende, che rimarra a dietro, e non arriverà
  quella ciurma.

\item[STRAMBASCIATA] Affannata; Oppressa dall'ambascia, che è una certa
  difficultà di respirare cagionata dalla violente fatica nel correre, che muove
  soprabbondanza d'alito. Dante Inf. C.24. \textit{E però leva sù; vinci l'ambascia}. Di
  qui per avventura \textit{Ambasciadore}, che piglia a fare \textit{ambascia}, cioè viaggio per andare
  a quel Personaggio, o Città, a cui egli è inviato.

\item[SI lancia] Si getta; cioè con un salto montò prestamente a cavalcioni al caprone.
\item[SI rinsacca] Assomiglia Martinazza (che cavalcata in sul suo Caprone corre)
  a quando s'empie un sacco di roba leggieri, la quale si mandi giù con fatica, e per
  stivarla, ed empier bene il sacco, questo s'alza, e s'abbassa squotendolo, e così
  faceva Martinazza a cavallo in sul Caprone, il quale faceva a lei questo effetto
  andando \textit{balzelloni}, cioè a salti, come è il proprio correr delle capre. Questa
  voce \textit{balzelloni} viene da \textit{balzellare}, che lo diciamo il saltellar delle lepri nel tempo
  di Maggio, e Giugno, che elle sono in amore, e la caccia che in tal tempo si fa
  si dice andare al \textit{balzello}. Del cavalcare la bestia nera, e cornuta V. Bocc. 8.9.

\item[ARRI là] Cammina là, Va là. Termine stimolatorio usato per asini, e muli,
  ec, dai vetturali. È ben vero, che vedendosi uno a Cavallo, che vi stia su sconciamente,
  si suol dire per derider colui \textit{Arri là} quasi diciamo va a cavalca un asino,
  e portato da questo uso l'Autore fa dire a Marcinazza \textit{Arri là}. Il Monosini
  lo fa venire dal Greco \textit{Errhe}, cioè, \textit{va via}.

\item[CARNE cattiva] Animale vituperoso. Diciamo \textit{carne cattiva}, o \textit{cattivo pezzo
  di carne} ancora a quegli huomini, che sono di genio sciagurato, e maligno. Onde
  si dice quasi in proverbio, e per ironia di chi sia magro, o piccolo di persona,
  ma sia maligno, e astuto, e come si dice ne' suoi panni e' vi sia tutto, \textit{Egli è come in
  Stornello, poca carne, e cattiva}, E qui si può anche dire, che l'Autore la chiami
  \textit{carne cattiva}, perché era capra, che fra le carni, che si mangiano, è la più cattiva.

\item[CIURMA] Dal Lat. \textit{turma}. Si dice propriamente degli Schiavi remiganti di galera:
  Ma si Piglia ancora per quantità di gentaglia, e qui intende di quella canaglia,
  che fuggiva. Vedi sotto \cstan[5]{16}., e C.~11, stan. 16.
\end{description}

\section*{FINE DEL TERZO CANTARE.}
\chapter{Quarto Cantare}

\begin{argomento}
I guerrier di Baldon son mal disposti
Perché la fame in campo gli travaglia;
Il fendesi, e Perlon lasciano i posti,
Non vedendo arrivar la vettovaglia.
Psiche non tiene i suoi pensieri ascosti
A Calagrillo Cavalier di vaglia,
Che promette aiutar la damigella,
E poscia ascolta una gentil novella.
\end{argomento}

\section{Stanza I. — IV.}
\begin{ottave}
\flagverse{1}\textit{Omnia vincit amor}: dice un Testo,\\
E un'altro disse, e dette più nel segno:\\
Fames Amorem superat. E questo\\
È certo, e approva ognun c'ha un po d'ingegno\\
Perché quantunque Amor sia sì molesto,\\
Che tutti i Martorelli del suo Regno\\
Dicano ogn'ora; Ahi lasso, io moro, io pero,\\
E non si trova mai, che ciò sia vero,
\end{ottave}

\begin{ottave}
\flagverse{2}Non ha che far niente con la fame,\\
Che fa da vero, pur ch' ella ci arrivi;\\
Posson gli amanti star senza le dame\\
I mesi, e gli anni, e mantenersi vivi;\\
Ma se due dì del consueto strame\\
I poveracci mai rimangon privi,\\
Ei basta, che de fatto andar gli vedi\\
A porre il capo dove il Nonno ha i piedi.
\end{ottave}

\begin{ottave}
\flagverse{3}Tal che si vien da questi effetti in chiaro,\\
Che d'Amore, la fame e più potente,\\
Ond'è c'ognun di lui più questa ha cara,\\
E quand'alle sue hore ei non la sente\\
Lamentasi, e gli pare ostico, e amaro;\\
Perciò riceve torto dalla gente,\\
Mentre ciascun la cerca, e la desia,\\
E s'ella viene, vuol mandarla via.
\end{ottave}

\begin{ottave}
\flagverse{4}Anzi la scaccia, come un'animale\\
Sul buon del desinare, e della cena,\\
Per questo ella talor, che l'ha per male,\\
Più non gli torna; ovver per maggior pena\\
In corpo gli entra in modo, e nel canale,\\
Che non l'empierebbe Arno con la piena,\\
Come vedremo, c'a Perlone ha fatto,\\
C'a questo conto grida come un matto.
\end{ottave}

Il nostro Poeta riflettendo, che nel presente Cantare gli convien descrivere la
fame, che era nel campo di Baldone, per non esservi ancora comparsa la munizione
di bocca, s'introduce col provare, che la fame è superiore ad Amore, quantunque
la maggior parte degli huomini, seguitando Vergilio Egl. 10. dove cantò:
\begin{verse}
  Omnia vincit amor; \& nos cedamus amori.
\end{verse}
dica che Amore sia più potente,e superi qualsivoglia passione. E dopo haver
provata questa sua intenzione, si maraviglia per qual causa la Fame, essendo
più potente, e più stimabile, e desiderabile, che non è Amore, habbia poi ad essere
scacciata nella maniera, che ognun procura di fare; considera però, che ella
habbia ragione di vendicarsi di tal disprezzo, e con l'andarsene in sul più bello
del mangiare, o col venir troppo, quando non si ha che mangiare, come vuol
mostrare ch'è seguito a Perlone.

\begin{description}
\item[MARTORELLI del regno d'Amore] Innamorati, travagliati, martirizzati da Amore.

\item[AHI lasso] Interposizione, che denota dolore. Quasi dica son lasso, e stanco
dal dolore, dal travaglio, ec. È il Lat. \textit{heu}, \textit{hei mihi}. Francese \textit{Helas}.

\item[NON ha che far niente] Non c'è luogo da far comparazione. Non è nulla,
rispetto alla fame.

\item[STRAME] Si dice il fieno, paglia, o altro simile che si dà per vitto alle bestie:
  Ma qui lo piglia per cibo degli huomini, come è scherzoso costume; e diciamo
  \textit{strameggiare}, quando uno va trattenendosi col mangiare alquanto, aspettando
  che venga in tavola la vivanda per desinare, o per la cena, che si dice
  \textit{sbocconcellare}. Vedi sotto \cstan[7]{10},

\item[POVERACCIO] Epiteto che esprime la compassione, che s'ha della disgrazia
  di colui, il quale si nomina. Vale per infelice, disgraziato, ec.

\item[PORRE il capo dove il Nonno hai piedi] Farsi sotterrare. Morire. Nella Scrittura
  si dice; \textit{Apponi ad patres suos}.

\item[RICEVE torto] Non se le fa il giusto: Non se le fa il dovere, \textit{Torto} è il contrario
  di \textit{diritto}. E significano questo Giusto; e torto Ingiusto, come vedemmo sopra
  \cstan[3]{66}. \textit{Non è in corpo storto animo dritto}.

\item[ANIMALE] E' nome generico, che significa ogni  specie di vivente; Ma è
  costume pigliarlo in specie, e per \textit{animale} intender solamente le bestie, donde segue
  poi che dicendosi animale a un huomo s'intende un huomo senza ragione, o
  giudizio, in somma un huomo bestia. Bocc.n.79, dice: \textit{Conoscendo questo medico
  esser un'animale}, Vedi sotto in \cstan{51}. Cic. Nonne vides, bellua?

\item[IL canale] Cioè il canal del cibo, che è la gola: il \textit{condotto de' bocconi}, che
  così vien descritto in lingua furbesca dalla plebe Fiorentina.

\item[NON empierebbe Arno con la piena] Non l'empierebbe Ano, quando per le
  pioggie vien grosso. Iperbole usata per intender'uno, che non si sazzi mai, ingordo
  tanto del cibo, quanto dei denari, che i latini dissero \textit{Dolium inexplebile}
  d'un huomo, \textit{quem eos non nutriet, illum nec AEgyptius}. Empiti Arnaccio: dicesi
  per dispetto a uno, che non si trova mai sazio; modo basso.
\end{description}

\section{Stanza V. \& VI.}

\begin{ottave}
\flagverse{5}Desta l'Aurora omai dal letto scappa,\\
E cava fuor le pezze di bucato,\\
Poi batte il fuoco, e quocer fa la pappa\\
Per il giorno bambin c'allora è nato;\\
E Febo ch'è il Compar già con la cappa,\\
E con un bel vestito di broccato,\\
C'a nolo egli ha pigliato dall'Ebreo,\\
Tutto splendente viensene al Corteo.
\end{ottave}

\begin{ottave}
\flagverse{6}Ne per ancora l'Ugnanesi genti\\
Hanno veduto comparire in scena\\
La materia che dà il portante ai denti,\\
E rende al corpo nutrimento, e lena;\\
Perciò molti ne stanno mal contenti,\\
Che son'usi a tener la pancia piena,\\
E ben si scorge a una mestizia tale,\\
Che la mastican tutti più che male.
\end{ottave}

Il nostro Poeta (come habbiamo detto altrove) hebbe notizia da Saluadore
Rosa d'un libro Napoletano intitolato \textsc{Lo Cvnto de li Cvnti}, ed in
comporre l'aggiunta alla presente opera se ne valse, cavandone qualche pensiero,
o concetto, come vedremo; e questo è quello della presente descrizione della levata
del Sole. Dice dunque che \textit{svegliata l'Aurora, esce del letto, e cava fuora le
pezze bianche di bucato}; il che allude alla chiarezza che apporta l'Alba. Di poi
\textit{accende il fuoco, e fa quocer la pappa per darla al Giorno bambino che allora è nato}.
E per questo fuoco intende quell'albore che si vede all'apparir dell'Aurora, il
quale va crescendo, e piglia un colore gialliccio per lo vicino apparir del Sole;
e però dice che \textit{Febo viene con  abito di broccato d'oro tutto splendente al Corteo del
giorno bambino}. E così intende che alla levata del Sole i Soldati di Baldone non
ancora havuta la provvisione per vivere, onde sono in collora, e particolarmente
molti di loro, che sono assuefatti a star sempre col ventre pieno.
\begin{description}

\item[PEZZE di bucate] Pezze bianche pulite perché sono di bacato, cioè non adoprate
  dopo che furono imbucatate; ed intende quei panni lini, che servono per fasciare, ed
  involtare i bambini.
\item[BATTE il fuoco] Accende il fuoco. Così diciamo, quando per accendere il
  fuoco si batte nella pietra focaia, se ben non si batte il fuoco, ma la pietra.
  Vergilio nel 6, dell' En, dice.
  \begin{verse}
    \makebox[3em]{\dotfill} quaerit pars femina flammae
    Abstrusa in venis silicis \makebox[4.5em]{\dotfill}
  \end{verse}
\item[PAPPA] Pane bollito in acqua; è la vivanda solita darsi a i bambini quando
  s'allattano, e cominciano a balbettare, e si dice \textit{pappa} perché essendo la lettera,
  \letter{p} puramente labiale, è facile a profferirsi come sono le lettere B, M. e
  però ne i bambini si trova maggiore attitudine a profferir queste, che l'altre
  consonanti, sì che più facilmente profferiscono \textit{babbo}, \textit{mamma}, \textit{pappa}, \textit{bombo}
  che \textit{padre}, \textit{madre}, \textit{minestra}, \textit{bere}, onde le balie si servano di queste parole per
  facilitare. la loquela a i bambini. Tal costume era forse anche negli antichi Romani,
  come si cava da Varrone, (nel libro intitolato Catone, Ovvero dell' allevare
  i figliuoli) che per \textit{Papas} intende quello, che intendiamo noi Toscani per
  \textit{Pappa} e da Persio, che nella Satira 3. disse
  \begin{verse}
    Et similis Regum pueris pappare minutum.
  \end{verse}

  I Greci pure  per i loro bambini si servivano come noi, e come i Latini, di
  voci di due sillabe con raddoppiarne la prima sillaba, per maggiore agevolezza
  del rilevare la parola. Di queste parole bambinesche ne troveremo molte nella
  presente Opera, usate dal Poeta per scherzo, o per accomodarsi alla qualità di
  colui che farà parlare, e non perché sieno in uso altrimenti. Vedi sotto in questo
  Cant, stan,12. dove dice d'un bambino che impara a parlare.

\item[BROCCATO] È una specie di drappo fatto a fiori, e s'intende Drappo tessuto con oro.

\item[A NOLO egli ha pigliato dall'Ebreo] Dice che il Sole ha pigliato a nolo il suo
  splendente abito, per significare che lo rende la sera, come lo restituiscono coloro,
  che pigliano gli abiti a nolo per un giorno; ed intendere che il Sole ascondendosi
  la sera alla nostra vista, lascia quell'abito risplendente, che s'era messo
  la mattina.

\item[CORTEO] Corteggio. Codazzo di donne, ec. che accompagnano una donna
  quando va a marito, o un bambino portato a Battesimo.

\item[UGNANESI genti] I soldati del Duca d'Ugnano; costume de i soldati d'appellar
  esercito dal nome del Generale, come Vaimaresi dal generale Vaimar\footnote{Si riferisce all'esercito del Duca di Weimar — forse per il suo ruolo nella fase iniziale della guerra Franco-Spagnola del 1635-1659.}, ec.

\item[COMPARIRE in scena] Venire in pubblico. Vedi sopra \cstan[1]{2}.

\item[LA materia che da il portante a' denti] La materia, che fa muovere i denti,
  cioè la roba da mangiare; si dice anche Da far ballare il mento. Vedi sotto in
  wuesto C, stan. 23. E \textit{portante} si dice una specie d'andare di cavalli. Il Lalli
  Tr. \cstan[3]{58}. dice.
  \begin{verse}
    Per dare il lor portante ai denti asciutti.
  \end{verse}

\item[LENA] Vedi sopra \cstan[1]{2}.

\item[LA masticavan male] L' intendevano male, la sopportavano mal volentieri.
  È solito quando si pensa a qualche cosa fissamente, e con applicazione il masticare,
  onde Persio delle composizioni ben pensate disse: \textit{Remorsum sapium unquem}:
  E tal \textit{masticare così} pensando si dice anche \textit{ruminare}, o \textit{digrumare}, che è quel
  masticare che fanno gli animali del pié fesso perciò detti \textit{ruminantia} da i Latini.
  Vedi sotto \cstan[6]{5}. Qui fa bell' effetto ' equivoco del verbo \textit{masticar male},
  che pare che voglia dire \textit{l'intendevano male}, e vuol poi dire che masticavano male,
  perché non mangiavano, non havendo che mangiare.
\end{description}

\section{Stanza VII. — IX.}

\begin{ottave}
\flagverse{7}E tra costoro un certo girellaio,\\
Che per l'asciutto va su i fuscellini,\\
Male in arnese, e indosso porta un saio\\
Che fu sin del Romito de Pulcini.\\
Ci è chi vuol dir ch'ei dorma n'un granaio\\
Per c'ha il mazzocchio pien di farfallini\\
È matto in somma, pur potrebbe ancora\\
Un dì guarirne, perché il mal dà in fuora.\\
\end{ottave}

\begin{ottave}
\flagverse{8}E, perch' ei non havea tutti i suoi mesi,\\
Fu il prima ad esclamare, e far marina\\
Forte gridando: Ohimè ch'io vado a Scesi\\
Pel mal che viene in bocca alla gallina,\\
Onde Eravano, e Don Andrea Fendesi\\
C'abbruciavano insieme una fascina;\\
E per cibare i lor ventri di struzzoli,\\
Cercavan per le tasche de' minuzzoli,
\end{ottave}

\begin{ottave}
\flagverse{9}Mentre di gagnolar già mai non resta\\
Colui ch'è senza numero ne rulli,\\
Anzi rinforza col gridare a testa, \\
Lasciano il fuoco, e i vani lor trastulli, \\
E per vedere il fin di questa festa\\
Se ne van discorrendo grulli grulli\\
Del bisogno ch'essi han ch'il vitto giunga\\
Perché sentono omai sonar la lunga.
\end{ottave}

Fra li suddetti soldati affamati l'Autore pone se medesimo descrivendo la sua
perfona, e genio; e dice che egli fu il primo a gridare per la fame, e per questo
Eravano, e Don Andrea Fendesi ancor essi affamati s'accostarono a lui per sentir
la cagione di quelle strida,

Nota che il Poeta divide il periodo nelle due ottave, ottava, e nona, di che è stato
da qualcheduno criticato d' errore, ma pero senza ragione, non adducendo
regola poetica, la a quale vieti il poterlo fare, come habbiamo detto altrove.

\begin{description}
\item[GIRELLAIO] Huomo stravagante. Huomo che gira, s'intende huomo inconsiderato, e
  che fa scioccaggini, e pazzie.

\item[ANDAR per l' ascivtto] Signi esser magro, e con poca carne addosso.
Vedi sopra Ca: stan. 68.

\item[VA in su i fuscellini] Ha gambe così sottili, che rassembrano due fuscelli; termine
  usatissimo da noi in questo proposito; che diciamo, Camminare su fuscelli.

\item[MAL in arnese] Mal vestito: Mal' all'ordine di sanità, d'abito, ec. Lalli
  En. tr, lib, 1. stan. 34.
  \begin{verse}
    Con sette navi Enea che gli avanzaro
    Qui si condusse assai male in arnese.
  \end{verse}
  Lodovico Dolce in lode dello sputo dice.
  \begin{verse}
    Eccomi qui per raccontarne cento,
    Ben ch' io non sia d' accordo col cervello,
    E malagiato in arnese mi sento.
  \end{verse}
  Il Persiani scrivendo al Serenissimo Principe D. Lorenzo dice.
  \begin{verse}
    Io, che sono in arnese tanto male,
    Mi ritrovo in grandissimo viluppo,
    Temo esser preso in vece d' un galuppo,
    E finir la mia vita allo Spedale.
  \end{verse}
  Franco Sacchetti Nov, 122. \textit{Il Saccardo era guarito, e stava bene in arnese}. Bocc.
  g.2.n.8. \textit{Partitosi assai povero, e mal' in arnese da colui, col quale lungamente era stato}.

\item[DEL Romito de' Pulcini] Questo fu uno che abitava poco lontano da
  Malmantile, e teneva vita eremitica, vestendo di lendinella a foggia di Francescano
  scalzo; Da costui prese il nome di Romito quel luogo vicino a Malmantile
  che dicemmo sopra \cstan[1]{70}. E perché egli oltre al procacciarsi il vitto con
  chiedere elemosina s'aiutava ancora col nutrire nella sua abitazione buon numero
  di Polli per vender l'uova, fu nominato il \textit{Romito de Pulcini}. Quando l'Autore
  compose la presente Opera, detto Romito era morto di gran tempo prima,
  e però dice che il saio che egli haveva addosso fu fino del detto Romito, volendo
  inferire che era gran tempo, che quell'abito era fatto, ed in conseguenza oltre
  all'esser vile per essere stato d'un povero Romito, era ancora lacero, e consumato
  dal tempo.

\item[SAIO] Gonnelletto, o casacca, o simile parte d' abito da huomo; dal Latino
  \textit{Sagum}. Il Varchi stor. fior. lib 9, E sotto il Lucco chi porta un saio, chi una gabbanella,
  o altra vesticciola di panno chiamata casacca.

\item[DICONO ch'ei dorma in un granaio] L'Autore medesimo lo dichiara, seguitando
  perché \textit{ha il mazzocchio pien di farfallini}, se uno dorme, o si trattiene in
  un granaio, si suol'empiere di quei farfallini che stanno fra il grano; e quando
  diciamo: Il tale ha de' farfallini, o delle farfalle, intendiamo E' mezzo matto; e
  di cervello volante, o instabile. E per \textit{mazzocchio} intendiamo il capo, perché
  mazzocchio era una parte del Cappuccio, che già portavano i Fiorentini, secondo che
  dice il Varchi nelle sue storie Fiorentine lib. 9. \textit{Il Cappuccio} (dice egli)
  \textit{ha tre parti, cioè il mazzocchio, il quale e un cerchio di borra, che gira, e fascia intorno
    intorno alla testa, e di sopra, soppannato di nero di rovescio, copre tutto il capo}. Si
  dice oggi corrottamente \textit{mazzucco}, e così havea detto l'Autore, ma havendo il
  medesimo a dipingere uno dell'antico Magistrato di Firenze, mi domandò come
  era veramente l'abito Civile antico, ed io gli feci vedere questo luogo del Varchi,
  onde egli poi mutò, e disse mazzocchio per quanto vedo dal suo secondo
  originale, che è appresso di me.

\item[IL male dà in fuora] Quando il male da in fuora, cioè manda alla cute
  l'interna malignità, suol' essere indizio di salute; costui essendo infermo di
  pazzia, il dare in fuora di tale infermità è il far pazzie; e però il Poeta dice, che
  potrebbe guarirne, perché il mal da in fuora, cioè spera ch'ei guarisca, perché
  fa molte pazzie, che è lo sfogo del suo male, ed il suo dare in fuora.

\item[NON ha tutti i suoi mesi] È spropositato. Non ha l'intera perfezione del cervello.
  Non è stato tutti a nove i meli nel ventre di sua madre a perfezionare il
  cervello. In somma vuol dire Non ha giudizio; è scemo.

\item[FAR marina] Diciamo far marina coloro, che fingendosi stroppiati, ed impiagati
  gridano, e si rammaricano per farsi creder tali; che tanto vale in questo
  proposito \textit{Marinare}, o \textit{far Marina}, quanto rammaricarsi, o dolersi di cosa, che
  dispiaccia, ma per lo più s'intende di coloro, che fingono; come per esempio
  lo scolare battuto dal maestro, si dice far marina, quando fingendo che il maestro
  gli faccia gran male, piange, e stride a più non posso; che di dice anche fare il
  monello, Vedi sopra C, 3. itaa. 67.

\item[VADO a Scesi] Quando diciamo; Il tale è andato a Scesi, intendiamo è morto,
  se ben pare che diciamo è andato alla Citta di Scesi, o Assisi, perché il verbo
  scendere ci serve per intendere morire, Virg. \textit{facilis descensus}.

\item[PEL mal, che viene in bocca alla gallina] Il male che viene in bocca gallina
  da noi è detto \textit{pipita} dal Lat. \textit{pituita}, E perché fra da gente bassa in vece di
  dire \textit{appetito} si dice \textit{appipito}, pero cavano questo detto: \textit{Il tale ha il mal che viene
    in bocca alla gallina}, cioè \textit{la pipita}, e intendeno \textit{appipito}, cioè fame. E questo intende
  il Poeta nel presente luogo con questo detto piebeo.

\item[ERAVANO] Cioè Averano Seminetti. \textit{Don Andrea Fendesi}. Ferdinando
Mendes.

\item[FASCINA] Fascetto di legne; \textit{Ed abbraciare insieme una fascina}, vale star al fuoco
  a scaldarsi, e spender ciascuno la sua porzione nelle legne; E vuol dir anco copertamente
  andare all'osteria, Oraz. \textit{Ligna super foco large reponens}.

\item[STRVZZOLO] Vccello noto, il quale mangia così voracemente, che inghiottisce
  fino il ferro, Dicendosi \textit{ventre di struzzolo} s'intende Ventre insaziabile.
  Plin. degli struzzoli. \textit{Concoquendi sine delectu devoratu mira natura}.

\item[MINVZZOLI] Quei minuti fragmenti, che cascano dal pane, quando si
  spezza. E quest'atto di cercare i minuzzoli nelle tasche, esprime uno che habbia
  grandissima fame.

\item[GAGNOLARE] Voce corrotea da cagnolare, che è il guaire, che fanno i
  cagnolini quando hanno bisogno della poppa. Se per avventura non lo derivassimo
  dal verbo Latino \textit{gannire}, che signitica Rammaricarsi con parole non
  affatto intese mescolate con sospiri, e singulti, che è quelio, che nel presente
  luogo vuol dir gagnolare.

\item[È SENZA numero ne i rulli] È matto. Nel giuoco de rulli si pigliano sedici,
  o più, o meno rocchetti di legno, ciascuno de i quali ha il suo numero, eccetto
  che uno, il quale si chiama il Matto: E però dicendosi: \textit{il tale è il senza numero
    fra i rulli}, s'intende è il rocchetto, che è senza numero, cioè il matto. Questi
  rocchetti si chiamano \textit{rulli}, perché rizzati in terra in ordinanza col detto Matto
  nel mezzo, vi si tira dentro con un Zoccolo di legno grave tondo di figura piramidale,
  il quale si chiama rullo, e il giuoco si domanda \textit{A' Rulli}, ed alle volte
  \textit{a' rocchetti}; E chi più ne fa cadere con quel tiro vince. Si costuma anche tirare
  con una palla di legno.

\item[RINFORZA] Cioè Cresce lo stridere, o il guaire. L. \textit{ingeminat}. Si raddoppia.

\item[GRIDARE a testa] Gridar quanto più si può. Si dice anche \textit{gridare a corr'huomo},
  o quant'uno n'ha nella \textit{strozza}; nella \textit{canna}; o \textit{nella gola}. Vedi sopra C. 3. stan 6.

\item[TRASTVLLI] Trattenimenti. È voce da Fanciulli, e qui vuol esprimere,
  che fussero veramente trastulli da bambini, perché aggiunge l'epiteto vani, come
  era veramente il cercare de i minuzzoli nelle tasche.

\item[PER vedere il fine di quella festa] Per vedere in che haveva a terminare, o a
  che fine fusse fatto quel romore. Quando un discorso, o un suono, o un cantare,
  o altro romore comincia a venirci a fastidio diciamo: \textit{Quando finirà questa
  festa}; questa \textit{musica}; questo \textit{chiasso}; questo \textit{bordello}; questo \textit{baccano}; questo \textit{moscaio?}
  e simili. Vedi sotto \cstan[9]{51}. e \cstan[10]{53}.

\item[GRVLLO] Intendiamo uno melancolico, sbattuto da cattivi effetti, e non affatto
  sano, che si dice anche Acquacchiato; E tal voce è presa forse dalla Grue uccello
  (Sp. grulla) che quando sta fermo posa un sol piede, e tiene l'ale basse in maniera,
  che pare un pollo ammalato; che però tal pollo, ed ogni altro uccello
  così ammalato si dice \textit{grullo}, o \textit{che porta i frasconi}, Vedi sotto C.10, stan. 20.

\item[SENTONO suonar la lunga] Quando il Prete per invitare i popoli alla Messa
  suona la campana, e dura lungo tempo, in contado dicono \textit{suonar la lunga}. E
  da questo durate lungo tempo dicendosi: il tale sente suonar la lunga, s'intende
  ha fame per esser lungo tempo, che non ha mangiato. E per significar più
  copertamente diciamo: Egli ha quella del Carmine, s'intende la lunga, perché nella
  Chiesa del Carmine di Firenze, avanti si dica la prima messa, suonano una campana
  per un grande spazio di tempo, e questo suonamento si dice da tutti \textit{la lunga
  del Carmine}.
\end{description}

\section{Stanza X. --- XII.}

\begin{ottave}
\flagverse{10}Così domandan chi sia quei ch'esclama,\\
E mette grida, ed urli sì bestiali !\\
Gli è detto; Questo è un tale, che si chiama\\
Perlone dipintor de' miei stivali,\\
Un huom c'al mondo s'acquista gran fama\\
Nel far de' ceffautti pe' boccali,\\
E con gl'industri, e dotti suoi pennelli\\
Suo nome eterno fa negli sgabelli.
\end{ottave}

\begin{ottave}
\flagverse{11}Si trova in basso stato, anzi meschino,\\
Ma ben che il furbo ne manceggi pochi,\\
Giuocherebbe in su pettini da lino,\\
Che un'ora non può viver ch'ei non giuochi.\\
Ma s'ei vincesse un dì pur'un quattrino\\
In vero si potrebbon fare 'e fuochi,\\
Perché giocando sempre giorno, e notte,\\
Farebbe a perder con le tasche rotte.
\end{ottave}

\begin{ottave}
\flagverse{12}Giuocossi un suo fratel già la sua parte;\\
Suo padre fu del giuoco anch'egli amico,\\
Però natura qui n'incaca l'arte\\
Havendo hereditato un genio antico.\\
Costui teneva in man prima le carte,\\
Che legato gli fusse anco il bellico,\\
E pria che mamma, babbo, pappa, e poppe\\
Chiamò spade, baston, danari, e coppe.
\end{ottave}

Costoro intesero, che colui, il quale così gridava era Perlone, cioè Perlone
Zipoli, che vuol dire Lorenzo Lippi Autore della presente Opera; e fa che venga
descritto per uno sfortunato, ed ostinato giocatore.

\begin{description}
\item[METTE strida, ed urli bestiali] Stride, ed urla gagliardamente. Dice \textit{bestiali},
perché lo stridere è proprio del porco ferito, ed \textit{urlare} è proprio della volpe,
cane, e lupo; se ben ce ne serviamo anche per l'huomo in questi casi.

\item[DIPINTORE de' miei stivali] Pittore dappoco. È termine comune per coloro,
  che sanno poco in qualsivoglia scienza, o arte. Vedi sotto \cstan[6]{106}.
  E \textit{stivale} diciamo un huomo goffo, e di poco giudizio. \textit{Stivali} diciamo quella
  scarpa, che cuopre tutta la gamba, e s'usa per cavalcare. Ma di i pittori dappoco
  si dice \textit{Pittor da sgabelli, da boccali, da colombaie, ec.} come si vede nella presente
  ottava, che dice: \textit{Fa de' ceffautti ne i boccali}, \textit{E con gl'industri suoi pennelli,
  eterna il suo nome negli sgabelli}. Ma perché questa sua modestia, ed humilità non
  sia di pregiudizio al merito di così gran valent' huomo, replico, che egli fu Pittore
  riputatissimo, come le belle opere sue chiaramente testificano, e come mostrerà
  il sig.\ Filippo Baldinucci, se mandera alle stampe la sua Genealogia de'
  Pittori, Opera degna d'ammirazione si per le belle notizie, che si hanno in essa,
  e si ancora per sapersi, che questo erudito huomo l'ha ritrovate, e messe insieme
  in brevissimo tempo rubato alli tanti riguardevoli affari, che per pubblico
  benefizio lo tengono continovamente occupato.

\item[CEFFAUTTI] Voce composta delle note Musicali \textit{Ce fa, ut}, e non ha significato
  veruno, se non che mostrandosi di dire la chiave del \textit{Ci sol fa ut}, s'esprime
  \textit{Ceffo}, che si piglia per viso, o faccia, se bene appresso di noi \textit{ceffo} vale per muso
  di cane, o grifo di porco, E quantunque venga forse dal Greco \textit{Cephali}, che vuol
  dir Capo, onde anche i Latini, chiamano \textit{Cephalea} un certo dolor di testa, e che
  in Franz.\ \textit{chef} sia \textit{capo}; nondimeno noi non ce ne serviamo se non per ischerzo, e
  per intendere \textit{una faccia brutta, e fatta male}; e però l'Autore, volendo che s'intenda,
  che Perlone dipigne male, chiama \textit{ceffi} quelle facce, che egli dipigne, che
  per altro parlando pittorescamente chiamerebbe Teste.

\item[BOCCALE] È una misura fatta di terra cotta invetriata capace della metà
  d'un fiasco Fiorentino, ma intendiamo ogni sorta di vaso sia più piccolo, o più
  grande, che sia però di questa materia, e figura. E perché questi boccali da Vasellai,
  che gli fabbricano in Montelupo sono dipinti malissimo, e senza un minimo
  disegno, però a uno, che dipinga male si dice \textit{Pittor da Boccali}, o \textit{Pittore da
    Montelupo}.

\item[BASSO stato, anzi meschino] Povero mendico; Poverissimo.

\item[FVRBO] Propriamente ladro dal latino. \textit{fur}, ed è parola ingiuriosissima
  tutavia si piglia per astuto, sagace, scaltrito, e che sa il conto suo: Qui vuol dir vizioso,
  perché ha il vizio del giuoco, \textit{Fur a furuo i[dest] nigro dictus}, Papias\footnote{Papìa il Lombardo, autore del primo dizionario moderno: \textit{Elementarium doctrinae rudimentum} 1040-1060 circa. Papias è greco bizantino, \textit{Precettore}.}.

\item[NE maneggi pochi] Intendi: maneggi pochi danari. Non gli venga alle mani
  gran quantità di danari.

\item[GIOCHEREBBE su i pettini da lino] Intendiamo uno, che giocherebbe con
  ogni maggiore scomodo, come farebbe, s'egli stesse a sedere in su i pettini da lino,
  che son composti d'acutissime punte di ferro.

\item[SI potrebbon fare i fuochi] Si potrebbono fare i fuochi in segno d'allegrezza,
  come d'una cosa insolita. Detto usatissimo, quando succede qualcosa di nostro
  gusto, che siamo stati buon pezzo aspettandola; Che si dice anche \textit{Suonare a
  doppio}, Vedi sotto \cstan[6]{107}.

\item[FAREBBE a perder con le tasche rotte] Perderebbe sempre: Farebbe a gara a
  chi perde più con le tasche rotte, quantunque queste perdano tutti li danari, che
  in esse si mettono.

\item[INCACARE] Disprezzare: La natura non sa grado, e non ha obbligo \textit{all'arte},
  non essendo stato opera dell'arte, che egli giuochi, ma effetto della natura,
  che l'ha prodotto con questo vizio di giuocare. Dan. Pur. C. 10. disse:
  \begin{verse}
    Ma la natura gli haverebbe a scorno.
  \end{verse}

\item[VN genio] Vedi sopra \cstan[1]{31}.

\item[PRIMA che gli fusse legato il bellico] Subito ch'egli usci del ventre della madre.
  \textit{Bellico}. Diciamo quella parte del corpo, d'onde è preso il nostro primo alimento
  nel ventre della madre; la qual parte nel venire al mondo è legata dalle nutrici.
  E ciò serva per dichiarazione del presente detto.

\item[BABBO, Mamma, Pappa, e Poppe] Sono delle prime parole, che si profferiscono
  dai bambini, come s'è detto sopra in \cstan{5}. Ma questo Perlone
  prima \textit{spade, baston, denari, e coppe}, che sono li quattro segni differenti
  figurati nelle carte da giuocare, che si appellano semi, come vedremo sotto C. 8.
  stan. 6. E qui gliele fa dire per mostrare, che prima d'ogni altra cosa questo
  Perlone chiamò il giuoco, e che venne fuora con cotesto genio naturale di giuocare.
\end{description}

\section{Stanza XIII. --- XV.}

\begin{ottave}
\flagverse{13}Ma perché voi sappiate il personaggio, \\
Che ciò racconta, è il Franco Vicerosa,\\
Cavaliero, del qual non è il più Saggio; \\
Scrittor subblime in verso, quanto in prosa; \\
Dipinge, ne può farsi da vantaggio \\
Generalmente in qualsivoglia cosa: \\
Vince nel Canto i musici più rari, \\
E nel portare occhiali non ha pari.
\end{ottave}

\begin{ottave}
\flagverse{14}È suo amico, ed è pur seco adesso\\
Salvo Rosata un huom della sua tacca,\\
Però che anch'ei s'abbevera in Permesso,\\
E Pittor passa chiunque tele imbiacca;\\
Tratta d'ogni scienza, ut ex professo,\\
E in palco fa sì ben Coviel Patacca,\\
Che sempre ch'ei si muove, o ch'ei favella\\
Fa proprio sgangherarti le mascella.
\end{ottave}

\begin{ottave}
\flagverse{15}Hor perché Franco, ed egli ogni maniera \\
Proccuran sempre di piacere altrui, \\
Di Pertone dan conto, e, dov'egli era, \\
Di conserva n'andar con gli altri dui, \\
Là dove minchionando un po la fiera\\
Il Franco disse lor: Questo è colui\\
Ch'in zucca non ha punto, anzi ragionasi\\
D'appiccargli alla testa un'appigionasi.
\end{ottave}

Acciò che si sappia chi è colui, che da tal notizia di Perlone, dice; che egli
haveva nome \textit{Franco Vicerosa}, cioè Francesco Rovai Cavaliere dotto, Poeta,
Musico, Pittore, e veramente dotato di quelle buone qualità, e virtù, che dice
il Poeta, e che stanno benissimo in suo pari, come testificano alcune poche sue
Poesie stampate dopo la di lui morte, che non sono anche le migliori, che egli facesse
Dice \textit{che nel portare occhiali non ha pari}, perché haveva naso aquilino assai grande.
Con esso è \textit{Salvo Rosata}, cioè Salvador Rosa huomo anch'egli dotto, e Pittore
eccellente, il cui valore e notissimo, mostrandolo a bastanza le di lui stimatissime
Opere; e quanto valesse nella Poesia si conoscerebbe da alcune Satire da lui fatte,
le quali si spera vedere una volta alla stampa. Questo era amicissimo dell'Autore,
e fu causa, che egli tirasse avanti la presente Opera, persuadendoli, che
era per godere l'aggradimento universale, e gli dette anche notizia de lo Cunto
degli Cunti pubblicato in quei tempi.  Saluator Rosa recitava da Napoletano
in commedia mirabilmente, e si faceva chiamare Coviello Patacca. Questo
Franco Vicerosa, e Salvo Rosata insegnarono dunque ad Eravano, ed al Fendesi
chi, e dove era Perlone.

\begin{description}
\item[AVOMO della sua tacca] Huomo simile a lui. Uniformi di genio. \textit{Questa Tuca}
  detta anche taglia è un pezzo di legnetto fesso in due parti per lo lungo, il
  quale serve per libro di conti a coloro, che non sanno leggere, in questa forma
  Uniscono dette due parti di legnetto, e nella parte più spianata fanno alcune
  tacche, o segni col coltello, i quali segni denotano il numero delle cose prese a
  credenza, o dei danari, che si devono, o de i lavori fatti, ec. Ed un pezzo
  esso legno rimane appresso al creditore, e l'altro appresso al debitore: e quando
  si voglion dar nuovi danari, o segnare nuovi lavori, s'uniscono detti legnetti, e
  vi si fanno i segni che occorrono; E volendo aggiustare i conti si numerano i segni,
  e si vede la quantità del debito, o credito: ne vi può nascere inganno, perché
  se in una delle dette parti di legnetto fara fatto un segno di più, non si può
  far nell'altra, perché non riscontrerà, se il debitore, e creditore non si concedono
  scambievolmente detti pezzetti. Era in uso questa maniera di tener conti
  anco appresso ai Latini, che tal legnetto, che noi appelliamo \textit{Taglia}, o \textit{tacca},
  la dicevano tessera: \textit{Suam uterque tesseram habet; ratio constat}. Havevano ancora
  un'altra \textit{taglia}, che chiamavano \textit{Tessera hospitalis}, la quale serviva per riconoscere
  gli amici, e corrispondenti di diversi paesi, serbando ciascuno il pezzo del
  legnetto; il quale si lasciava anche a gli Eredi; E quando andava uno nel paese
  dell'altro portava la parte del legnetto; e unendolo si dava a conoscere per ospite;
  e però detti legnetti erano custoditi diligentemente. Questo pure si cava da
  Plauto in Pen. \textit{Ego sum ipsus, quem tu quaeris. P, hem quid ego audio? Antidamae
    gnatum esse. P. Si ita est, Tesseram me conferre hospitalem Si vis eccam
    attuli}, Donde havevano poi, \textit{Tesseram frangere hospitalem}, che significa \textit{Violare Ius
  hospitii}. Dal che si cava, che \textit{homo eiusdem tesserae}, sia lo stesso, che huomo della
  medesima taglia, che significa delli stessi genj, e corrispondente. Di quo habbbiamo
  il verbo \textit{attaccare}, che vuol dire Unire due materiali insieme, Ed il verbo
  \textit{attagliare}, che vuol dire Esser uniti di genio. Ricord, Mal. Stor. Fior. cap. 87.
  dice: \textit{Lucca, Pistoia, e Volterra feciono taglia co' Fiorentini},  e s'intende, si collegarono,
  e fecero lega; E si trova ne gli antichi nostri Storici spesso Taglia per
  lega.

\item[PASSA chiunque tele imbiacca] Supera ogni Pittore.

\item[FA sgangherar le mascella] Fa ridere sregolatamente, che è, quel Risu quatere\footnote{Risu quatere aliquem}
  che dicemmo sopra \cstan[3]{66}. alla voce Pimmei. E veramente questo Rosa
  ne gli anni suoi più giovenili, che dimorò in Firenze recitava  (come habbiamo
  detto) questa parte di Napoletano così bene, che si può dire, che egli sia stato il
  Maestro in far questo Personaggio.

\item[ANDAR di conserva] Andare insieme. Detto Marinaresco, che ha questo significato.

\item[MINCHIONANDO la fiera] E' il latino \textit{derideo}, E tanto vale il verbo minchionare,
  che Co\ellipsis{24pt}\footnote{``Coglionare''?} Che non si dice per essere sporco, ed usato da genti vili.
  Quell'aggiunta di \textit{fiera} è solita mettervisi, ma non so già a qual fine, perché
  tanto suona il solo verbo \textit{minchionare}, se non che potrebbe dirsi \textit{Minchionare la fiera}
  esser detto da coloro, che non avendo voglia di comprare passeggiano per
  le fiere domandando del prezzo di questa, o di quella cosa, e non offerendo niente,
  o pochissimo; e stanno a vedere, e osservare chi compra. E venuto poi a
  significare il \textit{Minchionare} assolutamente, e si dice ancora \textit{Minchionare la Mattea}.
  Vedi sotto \cstan[7]{15}. E pur qui ancora senza l'aggiunta di \textit{Mattea} suona
  \textit{burlare}.

\item[IN zucca non ha punto] cioè punto di sale, e s'intende: Non ha cervello in testa,
  Vedi sopra C, 1, Man. 53. Il Mauro in lode della Caccia dice:
  \begin{verse}
    Ed io, che sono un buom materiale,
    Tencando ciò ben mostrerei ch'io fusse
    Da dovero una Zucca senza sale.
  \end{verse}
  Catullo di Quinzia disse:
  \begin{verse}
    Nulla in tam magno est corpore mica salis.
  \end{verse}

\item[ATTACCARGLI alla testa un'appigionasi] Essendo la sua testa vota; per mostrare,
  che ella si può affittare si discorre d'appiccargli l'appigionasi, che così chiamo
  quella cartella, in cui sta scritto a lettere grandi APPIGIONASI, e s'appicca
  sopr'alle porte delle case disabitate, affin che si conosca, che quella è casa
  da affittarsi, o appigionarsi, appunto come dice, che era la testa di Perlone, che
  per esser vota di cervello, era in grado da potersi affittare, o appigionare. In
  alcuni luoghi d'Italia conservano l'uso antico, scrivendo in L. \textit{Est locanda}.
\end{description}
\section{Stanza XVI \& XVII}

\begin{ottave}
\flagverse{16}Spiacque il suo male ad ambi tanto tanto, \\
E mentre ei piange, che si getta via, \\
Il pietoso Eravan pianse al suo pianto \\
Verbigrazia per fargli compagnia; \\
Poi tutto lieto postosegli accanto \\
Per cavarlo di quella frenesia, \\
Di quelle strida, e pianto sì dirotto, \\
Che fa per nulla il bietolon mal cotto.
\end{ottave}

\begin{ottave}
\flagverse{17}Se forse dice; tu sei stato offeso,\\
Che fai tu della spada il mio piloto?\\
A che tenere al fianco questo peso\\
Per startene a man giunte come un boto?\\
S'al corpo alcun dolor t'havesse poi\\
Gli è qua chi vende l'olio dello Scoto;\\
Se t'hai bisogno d'oro io ti fo fede,\\
Che qualsivoglia Banca te lo crede.
\end{ottave}

A costoro dispiacque molto il male di Perlone, ed Eravano dopo haver compianta
questa sua disgrazia, si messe a consolarlo, è ad esaminarlo strettamente
per sapere la cagione di sì gran suo pianto.

\begin{description}

\item[BIETOLONE mal cotto] Huomo sciocco insipido, svenevole, appunto come è
  la bietola: Marzial. 13. \textit{Ut sapiunt fatuae fabrorum prandia beta}, voce \textit{Bietola},
  che viene dal Latino \textit{beta}, che vuol dire una specie d'erbaggio, tanto nel
  nostro idioma, quanto nel Greco, e nel Latino serve ancora per esprimere un'huomo
  sciocco, ed insipido. Laerzio nelle vita di Diogene Cinico dice così:
  \textit{Circumstantibus se adolescentibus est dicentibus: Caveamus ne mordeat nos: Bono inquit
  estote animo filioli, carnis enim betis non vescitur}. Plin. \libcap[20]{22}. mostra, che
  i mariti volendo dire villania alle mogli dicevano loro \textit{bliteae}, raccogliendolo dalle
  commedie di Menandro; e si legge in quelle di Plauto, intendendo una cosa
  sciocca, e che non è buona a nulla; E come noi da \textit{bietola} caviamo il verbo \textit{sbietolare},
  che vuol dire Scioccamente piangere. (Vedi sotto \cstan[7]{93}.) e \textit{imbietolire},
  che vuol dire Commoversi, o effeminarsi. (Vedi sotto \cstan[9]{57}.) così
  gli antichi havevano \textit{betizare}, che ha lo stesso, o poco differente significato.

  \textit{Bietolone} dunque suona lo stesso, che Scimunito; ma con l'aggiunta di mal cotto
  vuol dire Scimunitissimo, perché la bietola cotta poco, dicono, sia più insipida
  della cruda.

\item[Piloto] Si chiama colui, che governa la Nave: dagli antichi Toscani detto
  \textit{Pedotto} forse dal L. \textit{pedes} preso per remi, come appresso Plauto \textit{navales pedes},
  o per funi da nave, come appresso altri. Ma questa voce Piloto ci serve per esprimere
  un'huomo da poco, poltrone, irressoluto, e flemmatico, ed in questo senso
  è preso nel presente luogo, Vien forse in tal caso dal Lat. \textit{plotus}, che vuol dir
  huomo, che per havere i piedi troppo piatti, e contraffatti cammina male. Vedi
  sotto \cstan[6]{90}.

\item[A CHE portare] A che fine portare? Che occorre che tu porti? \textit{Ad quid hoc
facis? Ad quid venisti?} nel Greco dice \textit{eph' hoo}, cioè per l'appunto \textit{A che?}

\item[STARSENE a man giuite come un boto] Boti chiamiamo quei Fantocci, o statue,
  che si mettono attorno all'Immagini miracolose per contrassegni di grazie
  ricevute, e però si dovrebbe dir \textit{Voti}, ma per iscambiamento di lettera si dice Boti.
  Berni in biafimo d' un' huomo brutto.
  \begin{verse}
    \makebox[3em]{\dotfill}Fugge da' ceraioli
    Acciò che non lo vendan per un boto.
  \end{verse}
  Che anticamente detti Fantocci si facevano di cera, e per lo più con le mani
  giunte in atto d'orare, e per questo dice \textit{starsene a man giunte come un boto}, che
  s'intende d'uno, che non sappia, o non voglia operare, e muover le mani per
  lavorare; e vuol'inferire, Che fai tu delle mani, e della spada, che tu non l'adoperi
  a vendicarti, se v'è stata fatta ingiuria? Mons. della Casa Galateo. Fo
  boto per modo di dirlo sempre.

\item[LO Scoto] intende di quel Ciarlatano, che vendeva Lattovarj, ed olj contro
  a veleni detto lo Scoto.

\item[TE lo crede] Scherza con l'equivoco, dicendo \textit{ogni banca te lo crede}, cioè ogni
  banca ti crede, che tu habbia bisogno dell'oro, e pare che voglia: Ogni
  banca ti fiderà, o presterà l'oro.
\end{description}

\section{Stanza XVIII. --- XXII.}

\begin{ottave}
\flagverse{18}Dopo Eravano poi nessun fu muto,\\
Ch'ognun gli volle fare il suo discorso\\
Offerendo di dargli ancora Aiuto,\\
Mentre dicesse quanto gli era occorso;\\
Ond'ei che havrebbe caro esser tenuto\\
D'haver più tosto col cervello scorso\\
Alzando il viso in loro gli occhi affisa,\\
E sospirando parla in questa guisa.
\end{ottave}

\begin{ottave}
\flagverse{19}Non v'è rimedio amici alla mia sorte; \\
Il tutto è vano, già che la sentenza \\
È stabilita in Ciel della mia morte, \\
Che vuol ch'io muoia, e muoia in mia presenza,\\
Già l'alma stivalata in su le porte\\
Omai dimostra s'esser di partenza.\\
Già con il corpo tutti i sentimenti\\
Le cirimanie fanno, e i complimenti
\end{ottave}

\begin{ottave}
\flagverse{20}Mutar devo mestier s'avvien ch'io muoia,\\
Il soldato cioè nel ciabattino,\\
Però che mi convien tirar le quoia\\
Per gir con esse a rincalzare il pino;\\
Un'altra cosa ancor mi dà gran noia,\\
Ed è che sotto son come un cammino,\\
E là dinnanzi a Minos, e agli altri Giudici\\
Rappresentar mi devo co pié sudici
\end{ottave}

\begin{ottave}
\flagverse{21}Ma ecco omai l'hora fatale è giunta, \\
Ch'io lasci il mio terrestre cordovano; \\
Già già la morte corre che par' unta \\
Verso di me con la gran falce in mano; \\
Spinge ella il ferro nel bel sen di punta, \\
Ond'io mancar mi sento a mano a mano: \\
Però lo spirto, e il corpo in un fardello \\
Tiro fuor della vita, e vo all'avello.
\end{ottave}

\begin{ottave}
\flagverse{22}Hormai di vita son uscito, e pure\\
Non trovo al mio penar quiete, o conforto,\\
O Cielo Mondo, o Giove, o creature\\
Dite, s'udiste mai così gran torto?\\
Se Morte è fin di tutte le sciagure,\\
Come allupar mi sento ancor che morto?\\
E come, dove ognuno esce di guai,\\
Mi s'aguzza il mulino più che mai?
\end{ottave}

Anche gli altri dopo Eravano gli offersero il loro aiuto, ed egli fingendosi pazzo
comincia a dire una mano di scioccherie, e mostrando di creder d'esser morto,
si maraviglia, che \textit{mors, qua omnia soluit} non gli habbia levato l'appetito
di cibarsi.

\begin{description}
\item[HAVERE scorso col cervello] Esser' impazzato. Haver dato la volta al cervello.
  Metafora tolta dall'orivolo a ruote, che si dice guasto, quando le ruote
  scorrendo escono del lor moto regolato.

\item[AFFISSAR gli occhi in uno] Guardare senza punto movere gli occhi; atto da
  pazzo di quella specie, che domandano Maniaci.

\item[ALLA mia sorte] Di quel che m'ha da succedere. Questa voce \textit{sorte} appresso
  di noi si piglia in diversi significati, come seguiva anche appresso a i Latini, da i
  quali si diceva \textit{fors} ogni avvenimento di Fortuna. Cic.lib.2. de Divinatione. \textit{Quid
    enim sors est? idem propemodum, quod micare, quod talos iacere, quod tesseras}, ed in
  questo senso è preso nel presente luogo. Si dice tirar le sorti, per intender quel
  \textit{super vestem meam miserunt fortes} dell'Evangelista.

  La pigliavano per carica, o incumbenza, secondo Livio: \textit{Si id gravaretur
    facere, quod non suae fortis id negotium esset}.

  La pigliavano per stirpe, secondo Ovid. 6. fast.
  \begin{verse}
    \backspace Si genus aspicitur, Saturnum prima parentem
    Feci, Saturni sors ego prima fui.
  \end{verse}

  La dicevano anche il capitale, e quello che noi pure diciamo sorte principale;
  Plaut. Most, \textit{Quatuor quadraginta illi debentur mina, Et sors, \& foenus DA, tantum est}.

  Altre volte pigliavano \textit{sors, pro iudicio} secondo Verg. 6. Aneid.
  \begin{verse}
    Nec vero hae sine sorte data, sine iudice sedes,
  \end{verse}

  Perché (secondo Servio) non s'udivano le cause \textit{nisi per sortem ordinate, nam,
  quo tempore causae agebantur, conveniebant omnes, \& ex sorte dierum ordinem accipiebant,
  quo post trigesimum diem causas suas exequerentur.}

  Dicevano sorte gli Oracoli, o risposte, o le polizze sopra alle quali si scrivevano
  le risposte. Val. lib. 1. \textit{Cuius rei exploranda gratia legati ad Delphicum oraculum,
  retulerunt: praecipi sortibus, ut aquam eius lacus emissam per agros diffunderent}. Virg.
  in questo senso disse: \textit{Lycie sortes}. Appresso noi ancora (come ho accennato)
  \textit{sorte} si piglia per fortuna, o destino, per condizione, stato, o essenza. E diciamo
  toccare in sorte, che significa ottenere la benefiziata, quando s'estraggono
  le polizze, che è quel \textit{mittere sortes}; e se bene in significato di fortuna vogliono
  alcuni, che si debba dire \textit{sorte}, ed in significato di qualità, o condizione \textit{sorta}
  hoggi (almeno nel parlar familiare, e Civile) non trovo, che s'usi tal distinzione,
  ma sento usare alcune volte l'una per l'altra indifferentemente.

\item[CIABATTINO] Uno che raccomoda scarpe rotte; da \textit{ciabatta}, che vuol dire
  Scarpa vecchia, e scarpa all'Appostolica, che sono quelle, che oggi usano i
  Cappuccini. In molti luoghi de' contorni Fiorentini chiamano Ciabattini ancora
  quelli, che fanno di nuovo; che noi chiamiamo Calzolai, in Ispagnuolo detti
  similmente \textit{zapateros}; e questo nome di \textit{Ciabatta} viene secondo alcuni da \textit{Clavata},
  cioè scarpa ferrata con chiodi; quali son quelle che usano i contadini, e i cacciatori.

\item[TIRAR le quoia] Havendo detto, che \textit{di soldato doveva diventar Ciabattino}, dà
  la ragione perché; ed è questa, che gli convien tirar le quoia, come fanno i Ciabattini,
  e i Calzolai, che tirano i quoi per condurgli a quella misura, che vogliono,
  delle quali quoia dice, che si dee servire per \textit{rincalzare il pino}, cioè far le
  scarpe al pino. Nota che lo scherzo dell'equivoco, nasce da \textit{tirar le quoia}, che
  vuol dir Morire, e \textit{rincalzar con esse il pino}, che vuol dire Farsi sotterrare a pié
  del pino, e così alzandogli la terra attorno rincalzarlo, che questo vuol dire rincalzare
  un'albero. Osserva ancora, che facendolo parlar da pazzo vuol, che
  coloro credano, che egli habbia concepito nel cervello questo sproposito d'haver
  a fare le scarpe a i pini; perché quando un Calzolaio dice; Io calzo il tale, s'intende
  \textit{Io gli fo le scarpe}. Plut. in Dem, \textit{E calzandosi dicea}, Il Gr, \textit{crepidas subligant}.

\item[SOTTO, son come un cammino] Sono schifo, ed ho le carni sudice, come è un
  cammino, dove si fa il fuoco, Comparazione usatissima particolarmente dalle
  donne.

\item[MINOS, e gli altri Giudici] I Giudici dell'Inferno secondo le favole degli antichi
  Poeti, e della Gentilità sono tre, cioè Minos figliuolo di Giove, e di Europa,
  che fu Re di Candia, Eaco, che fu figliuolo di Giove, e d'Egina, che fu
  Re d'un Isola già detta Enopia, la quale egli poi dalla madre chiamo Egina, e
  Radamanto, che fu figliuolo di Giove, e d'Europa, che fu Re di Licia. Questi
  Re, perché furono severi amatori della giustizia, dicono i detti Poeti, che Plutone
  gli eleggesse per Giudici dell'Inferno, affinché esaminassero l'anime, ed
  assegnassero loro le pene, che meritavano, e da quello, che di loro scrive Verg.
  Aen. 6. si può comprender il lor preciso, e particolar ofizio, che di Minos dice:
  \begin{verse}
    \backspace Quaesitor Miinos urnam movet, ille silentum
    Conciliumque vocat, vitas, \& crimina discit.
  \end{verse}
  E di Radamento dice;
  \begin{verse}
    \backspace Gnosius haec Rhadamanthus habet durissima Regna,
    Castigatque, auditque dolos, subigitque fateri.
  \end{verse}
  D' Eaco parla Ovidio così;
  \begin{verse}
    \makebox[10em]{\dotfill} Tuasque
    AEacus in penas ingeniosus erit.
  \end{verse}

  E conchiude il Poeta, che uno di questi Giudici esamini, l'altro giudichi, il terzo
  mandi ad esecuzione. Se ben Dante nel 5. dell'Inferno dice:
  \begin{verse}
    Stavvi Minosse orribilmente, e ringhia,
    Esamina le colpe nell'entrata,
    Giudica, e manda secondo ch'avvinghia.
  \end{verse}

\item[CORDOVANO] Specie di quoio da fare scarpe, la concia del quale fu forse
  inventata in Cordova, e perciò tali quoi chiamansi propriamente cordovani, e son
  pelli di Castroni, o d'altri animali, ma qui intende pelle humana, e dicendo
  \textit{lasci il mio terrestre cordovano} intende io muoia, come intendon quelli, che dicono
  \textit{Terrestre salma, Terrena spoglia}, e simili, Cunto de li Cun. \textit{Pesto, e concio per cordonano}.

\item[CORRE che pare unta] Corre velocemente; comparazione dalle carrucole, o
  pulegge, o altre simili, le quali quando sono unte con olio, sapone, o altro,
  scorrono velocemente.

\item[FALCE] Strumento, col quale si sega il fieno; e col quale spesso si vede dipinta
  la morte con essa in mano.

\item[GUAI] Travagli, sventure, sciagure, afflizioni. Vedi sopra \cstan[1]{28}.

\item[ALLUPARE] Haver gran fame, perché dicono, che il lupo sempre habbia
  gran fame; quindi il volgo chiama Male della Lupa quello di coloro, che sempre
  mangerebbono, perché da loro vien prestissimo smaltito il cibo con pochissimo
  nutrimento, ed è quella infermità, che i Medici chiamano Fame canina.
  Vedi sotto C. 5. stan.\ 61. E da questo male chiamato della Lupa diciamo \textit{allupare}
  uno che habbia gran fame.

\item[AGUZZARE il mulino] Far venire, o crescere l'appetito: perché aguzzare
  la macine del mulino vuol dire Metterla in taglio in maniera, che si renda più
  ingorda. Vedi sotto \cstan[7]{31}.
\end{description}
\section{Stanza XXIII. --- XXXVIII.}

\begin{ottave}
\flagverse{23}Va a dir che qua si trovi pane, o vino\\
O altro da insegnar ballare al mento;\\
Se non si fa la cena di Salvino,\\
Quanto a mangiar non c'è assegnamento,\\
O ser Isac, o Abramo, o Iacodino,\\
Quando v'havete a ire al monumento,\\
Voi l'intendete, che nel cataletto\\
Con voi portate il pane, ed il fiaschetto,
\end{ottave}

\begin{ottave}
\flagverse{24}Orbè compagni? olà dal cimitero,\\
S'il Ciel danari, e sanità vi dia\\
Empiete il buzzo a un morto forestiero,\\
O insegnateli almeno un'osteria;\\
Se ben voi fate qui sempre di nero,\\
Perché di carne havete carestia:\\
È tale l'appetito che mi scanna\\
Ch'un Diavol cotto ancor mi parrà manna.
\end{ottave}

\begin{ottave}
\flagverse{25}Se ben non c'è da far cantare un cieco,\\
Di questa spada all'oste fo un presente,\\
C'ad ogni mo, da poi ch'ella sta meco,\\
Mai batté colpo, o volle far niente;\\
Per una zuppa dolla ancor di greco.\\
Ma che gracchio io? Qui nessun mi sente.\\
Che fo? s'i morti son di pietà privi\\
Meglio sarà ch'io torni a star tra i vivi.
\end{ottave}

\begin{ottave}
\flagverse{26}Qui tacque, e per fuggir la via si prese\\
Facendo sempre il Nanni, ed il corrivo,\\
Perch'egli è un di quei matti alla Sanese,\\
C'han sempre mescolato del cattivo;\\
Per haver campo a scorrer il paese\\
Ne fece poi di quelle con l'ulivo\\
Mostrando ogn'hor più dar nelle girelle,\\
E tutto fece per salvar la pelle.
\end{ottave}

\begin{ottave}
\flagverse{27}Perché uno, ch'il soldato a far s'è messo,\\
Mentre dal campo fugge, e si travia,\\
Sendo trovato, vien senza processo\\
Caldo caldo mandato in piccardia;\\
Però s'ei parte non vuol far lo stesso,\\
Ma che lo scusi, e salvi la pazzia,\\
Onde minchion minchion facendo il matto,\\
Se ne scantona, che non par suo fatto.
\end{ottave}

\begin{ottave}
\flagverse{28}Il Fendesi a scappare anch'ei fu lesto\\
Con gli altri tre correndo a rompicollo,\\
Volendo risicar prima un capresto,\\
E morir con la stomaco satollo,\\
Che restar quivi a menarsi l'a\ellipsis{24pt},\\
Ed allungare a quella foggia il collo.\\
Il danno certe è sempre da fuggire,\\
S'egli avvien peggio poi, non c'è che dire.
\end{ottave}

Perlone seguitando a dire spropositi per esser tenuto matto si parte, e per salvar
la vita continovò a fare delle sciocchezze, sapendo, che un soldato che scappa
dal campo, e si parte senza licenza è reo di morte, ed il Fendesi, e gli altri
scamparono anch' essi.

\begin{description}
\item[VA a dir che qua si trovi] È vanità il credere, o dire che qua si trovi; s'inganna
  chi crede che qua si trovi.

\item[INSEGNAR ballare al mento] Mangiare. E' lo stesso che Dar il portante a'
  denti detto sopra in \cstan{6}.

\item[FAR la cena di Salvino] Andare a letto senza cena; che la cena di Salvino era
  Pisciare, e andare a letto.

\item[O SER Isac, o Abramo, o Iacodino] Intende tutti gli Ebrei, e seguitando l'opinione
  del volgo, il quale crede, che quando gli Ebrei seppelliscono i loro morti
  mettano lore appresso del pane, e del vino dice: \textit{Voi l'intendete che morendo
    portate con voi il pane,e il vino}, poiché nel mondo di qua non si trova ne da
  mangiare, ne da bere.

\item[CATALETTO] Quella barella, entro alla quale si portano i morti al sepolcro,
  che i Latini dicevano \textit{feretrum}. Voce composta di \textit{Letto}, e \textit{Cata} preposiz. Gr.\footnote{dal greco $\kappa\alpha\tau\alpha$ cioè "giù, in basso, sotto"}

\item[ORBÈ, olà, alò] E simili; sono voci, e termini usati per farsi sentire da chi è
  alquanto lontano; come fa il Latiao \textit{heus}, Orbè, e fatto da Ora bene; Or beat
  Latino \textit{age vero}; \textit{Alò} dal Fr. \textit{allons}; andianne.

\item[CIMITERO] Piazza, nella quale si fanno i sepolcri per li morti, Voce che
  viene dal Greco \textit{coemasthae}, che suona dormire, riposarsi. Onde \textit{coemeterion}, è lo
  stesso, che Dormentorio. Quindi i Cretensi chiamavano Cimeterio una casa
  pubblica, la quale serviva per alloggiare i pellegrini. Vedi sotto \cstan[7]{27}.

\item[S'IL Ciel danari, e sanità vi dia] Dice questo sproposito per accrescere in
  coloro la credenza, che egli sia matto, sapendo bene che i morti non hanno bisogno
  di sanità, ne si curano di denari.

\item[BUZZO] Intendi il ventre dell'huomo, da busto che s'intende tutta quella
  parte del corpo humano, che è dal collo al pettignone, senza le braccia.

\item[FAR di nero] Mangiar di magro. I venerdì, sabati, Quaresima, ed altre vigilie
  si chiamano giorni neri, quasi giorni di lutto destinati alla penitenza, ed il
  Poeta scherzando con l'equivoco del nero, col quale è solito farsi l'apparato a'
  morti, par che voglia dire non mangiate mai carne, perché soggiunge \textit{di carne
  havete carestia}, e par che intenda non havete carne da mangiare, e vuol dire
  non havere carne in su l'ossa, perché i morti in breve tempo restano puri scheletri
  senza carne.

\item[APPETITO che mi scanna] Fame così grande, che mi fa morire, che mi fa
  perder la canna della gola; che scannare uno, vuol dir Tagliarli la canna della
  gola. Cunto de li Cunti Giorn. 1.\textit{Se la necessità non la scannava}.

\item[MI parrà manna] Mi parrà buonissima; come parve, e fu a gli Ebrei la Manna,
  che mandò loro Dio nel deserto, che ricevendola esclamavano \textit{Manu}, cioè
  Che è questo? onde sortì il nome.

\item[NON ho da far cantare un cieco] Non ho ne meno un quattrino da darlo a un
  cieco, perché canti un' Orazione.

\item[IN ogni mò] Per: a ogni modo. È termine assai usato in Firenze in diversi sensi,
  perché, o significa disprezzo, come nel presente luogo, \textit{Voglio dar via la spada,
  perché ad ogni modo non battè mai colpo}, cioè perché io non la stimo per non haver
  ella mai lavorato. O significa necessità di fare, o non fare una cosa per esempio,
  \textit{si può far quanto si vuole, che ad ogni modo s'ha da morire}. Significa contentarsi di
  quello, che uno ha conseguito: \textit{Io ho guadagnato poco, ma ad ogni modo io mi
    contento}. Significa Ostinazione. \textit{So che la tal cosa mi può nuocere, ma la voglio
    fare ad ogni modo}. Vedi sopra Can. 1. stan. 3. il termine; \textit{suo danno}, che par che
  habbia correlazione al termine, A ogni modo, V.gr. \textit{Se io ho perduta la tal
    cosa, suo danno; ad ogni modo io non me ne servivo}, E quel \textit{mo} per modo è
  la figura apocope da noi molto usata come vedremo altrove.

\item[MAI batté colpo] Diciamo: \textit{il tale non batté mai colpo} per intendere, il tale
  non lavora mai, e qui intende, che la spada di Perlone nelle sue mani non lavorò mai.

\item[ZUPPA] Pane intinto nel vino, o in altro liquore. Forse meglio \textit{Suppa}, Franco
  Sacc, Nov. 86. \textit{E fatta la suppa con le spezie, subito porta in tavola il ventre, e la
    suppa}. Stimo che venga dal Tedesco \textit{Suppen}, che vuol dir Brodo di carne, o d'altro,
  che si quoca lesso. In questo senso una sorta di minestra chiamiamo \textit{zuppa
    Lombarda}, Vedi sopra \cstan[2]{7}. Ma l'uso ha introdotto il dir corrottamente
  zuppa, e da molti inzuppa; come zolfa, e zezzo, e zinfonia in vece di solfa,
  sezzo, sinfonia, e simili.\footnote{Vedi anche ``berzaglio'', o la parola ``verzicola'', che in altri testi appare come ``versicola''. }

\item[GRACCHIARE] Discorrer senza proposito, o profitto. Da Graccio Latino
  \textit{gracculus}. Il tale mi chiese dieci scudi in presto, ma io lo lasciai gracchiare. Vedi
  sotto \cstan[7]{59}. e \cstan[8]{65}.

\item[FAR il nanni, ed il il corrivo] Fingersi corrivo, goffo, semplice, basèo.

\item[MATTI alla Sanese] Si dice \textit{Sanesi Matti}, ma in effetto son più sagaci degli altri,
  e però dice \textit{Matti alla Sanese, c'han sempre mescolato del cattivo}; cioè dell'astuto,
  del sagace, ed ingegnoso.

\item[NE fece di quelle con l'ulivo] Fece delle scioccherie grandissime. In alcune solennità,
  suole la generosa pietà del Sereniss. G. Duca liberare dalle carceri alcuni
  debitori con pagare il loro debito, o parte di esso; e questi tali vanno processionalmente
  a render grazie a Dio al Tempio della Santiss. Annonziata, o di S. Gio: Batista;
  e quelli che hanno pagato tutto il debito, e sono affatto liberi portano
  in mano un ramo d' olivo a distinzione di quelli, che per non haver pagato
  tutto il debito, ma parte di esso devono tornare in carcere, i quali non hanno
  l'olivo in mano, ma son legati. Da questo ramo d'ulivo, che in tal congiuntura
  denota pagamento intero, credo che sia nato il dettato; La tal cosa e con l'ulivo,
  che significa cosa grande nello stesso modo, che i Latini dissero \textit{palmaris}, ed
  esprime un'azione ardita, che diciamo anche \textit{marchiana}; \textit{da pigliar con le molle},
  ec, come s'intende qui, che vuol dire, che questo fece cose grandi, ed ardite.

\item[DAR nelle girelle] Impazzire. Vedi sopra \cstan[3]{43}., e sotto C.9. stan.10.

\item[MANDATO in Piccardia caldo caldo] Impiccato subito preso senza far processo:
  \textit{Caldo caldo} subito, e prima che la cosa si raffreddi. \textit{Piccardia},
  \textit{in ipso ardore criminis}, Provincia della Francia, serve, scherzando con la
  similitudine della parola, per intendere \textit{impiccare}. I Latini pure havevano
  un termine coperto per fare intendere impiccare, che era \textit{literam longam facere},
  come si vede in Plauto; il che ha data occasione a molti letterati di discorrere per
  chiarire qual fusse questa lettera  e Celio Rod. leet, Ant: \libcap[10]{8}. conchiude,
  che fusse il T maiuscolo, che è simile alla forca, che facevano i Latini.
  Noi ancora diciamo: \textit{Andare a Lungone} che è un Porto in Toscana; \textit{Andar a Fuligno},
  cioè a \textit{fune}, e \textit{legno}; \textit{Dar de' calci al vento}: \textit{Ballar in campo azzurro} sopra
  C.2. stan. 65. \textit{Ballar nel Paretaio del Nemi}, sotto c. 6. stan. 50. E tutti significano
  Esser impiccato.

\item[MINCHIONE] Da minchia detto sopra in \cstan{15}.\footnote{Qui il Minucci sembra non voler insistere con le Minchiate, gioco delle carte descritto al \cstan[8]{61}., dove il \textit{Matto} offre proprio l'opportunità di scantonare \textit{che non par suo fatto}, proprio come di seguito.}

\item[SE ne scantona, che non par suo fatto] Se ne va via e non pare che faccia
  questo per andarsene, È forse quell'\textit{Agere se} di Ter. in Andr.

\item[CORRER a rompicollo] Correr velocemente; e a precipizio senza considerare
  la strada buona, o cattiva.

\item[ARRISCHIARE un capresto] Avventurare a essere impiccato. Corre più tosto
  il rischio d'andare in su le forche, che quello di morir di fame.

\item[MENARSI l'A\ellipsis{24pt}] Perder il tempo senza far nulla. Se vuoi intender bene
  questo detto, leggi discorso d'Anibal Caro in difesa di Ser' Agresto\footnote{Menarsi l'Agresto: \textit{Far cosa di poca riputazione, per non aver da far altro}. È possibile che il Minucci non volesse davvero indicare un ``discorso in difesa di Ser' Agresto'', pseudonimo di Annibal Caro, ma semplicemente cercasse l'occasione di menzionare la parola mancante, senza cadere nella volgarità.\\per ``agresto'', vedi \cstan[7]{7}}.
\end{description}

\section{STANZA XXIX. --- XXXI.}
\begin{ottave}
\flagverse{29}Lasciam costoro, e vadan pure avanti \\
Cercando il vitto lì per quel contorno,\\
Che se fame gli caccia, e' son poi fanti\\
Da battersi ben ben seco in un forno: \\
Perché d'un gran guerrier convien ch'io canti\\
Mezzo impaniato, perch'egli ha d'intorno \\
Vna donna straniera in veste bruna,\\
Che s'affligge, e si duol della fortuna.
\end{ottave}

\begin{ottave}
\flagverse{30}Calagrillo è il guerriero, e via pian piano\\
Cavalcando ne va con festa, e gioia,\\
Ognor tenendo il chirarrino in mano,\\
Perché il viaggio non gli venga a noia,\\
È bravo sì, ma poi buon pastricciano,\\
E' farebbe servizio infino al Boia,\\
Venga chi vuol, a tutti dà orecchio,\\
Se bene fusse il Bratti Ferravecchio.
\end{ottave}

\begin{ottave}
\flagverse{31}Poiché bella è colei che si dispera \\
Sempre piangendo senz' alcun ritegno, \\
E vanne, come io dissi, in cioppa nera \\
Per dimostrar di sua mestizia il segno,\\
Perciò con viso arcigno, e brutta cera\\
Par un Ebreo c'habbia perduto il pegno,\\
E di quanto l'affligge, e la travaglia,\\
Calagrillo il Campion quivi ragguaglia.
\end{ottave}

Il Poeta lascia il discorso di quegli affamati, e si mette a narrare la favola travestita
di Psiche; la quale chiede aiuto a Calagrillo, che è Carlo Galli Capitano
di Cavalli, gli racconta i suoi travagli.

\begin{description}
\item[SON fanti] S'intende son huomini c'hanno cuore, e spirito da fare quella
  tal cosa; e da pigliare ogni risoluzione.

\item[DA battersi ben ben seco in un forno] Da combatter con la fame anche dentro a
  un forno pien di pane, e mangiandoselo, vincerla, e farla fuggire.

\item[MEZZO impaniato] Imbrogliato; Intrigato: Traslato da gli uccelli, che havendo
  toccata la pania\footnote{Pà-nia, s.f., Materia appiccicosa ricavata dalle bacche del vischio, usata per catturare piccoli uccelli. Sinonimo di vischio. ``Impaniato'' quindi sinonimo di ``invischiato''.}, volano sì, ma con difficultà per l'impedimento, che dà
  loro la pania, che hanno sopra alle penne.

\item[BVON papricciano] Huomo dolce, grossolano, huomo alla buona. \textit{Pastricciano}
  è specie di Pastinaca. Il detto antico e Buon pasticcione, cioè di buona pasta.
  \textit{Placidus tanquam aqua silens}.

\item[BATTI feravecchio] Molti vogliono, che si dica il Bratti ferravecchio, il quale
  fu un huomo facultoso, ma di cattiva fama: Costui lasciò poi tutto il suo havere
  a una Confraternita di secolari intitolata in S. Gioseppe, perché delle rendite
  se ne dessero tante elemosine, come segue fino al dì d'hoggi; ma a me pare,
  Che meglio sia dire il \textit{Batti}; perché il \textit{Batti}, cioè i \textit{Battilani}, quando noo possono
  più lavorare non sapendo far altra arte, si mettono a fare il rivendirore di
  cenci, e ferri vecchi, e dall'andar gridando per la Città \textit{Chi ha ferri vecchi}, hanno
  acquistato il nome di Ferravecchio. E perché queste sono vilissime persone, ed
  alle quali si ha poco riguardo;  quando vogliamo esprimere, che uno sia di mansueta,
  ed umil natura, e indifferente con tutti, sogliamo qualificarlo con questo
  termine. \textit{Saluta, o farebbe servizio, anche al Batti ferravecchio}. Che se dicesse il
  \textit{Bratti} calzerebbe tanto bene; perché finalmente il \textit{Bratti}, fu persona di qualche
  riguardo, e Civiltà. \textit{Imbratta} soprannome trovasi nel Bocc.\footnote{Guccio Imbratta, anche detto Guccio Balena, o Guccio Porco, fante di Frate Cipolla nella novella 10 della giornata sesta. Compare in chiusura della novella 7 della giornata quarta.}

\item[PSICHE] È nota la favola di \textit{Psiche}, descritta maravigliosamente da Apuleio,
  la quale il Poeta incastra in questa sua Opera, e l'immaschera assai aggiustatamente.

\item[VISO arcigno] Viso aspro, che denota dolore, o altra passione travagliosa. Lat.\ \textit{Torva facies}.

\item[BRUTTA cera] Haver brutta, o cattiva cera vuol dire Faccia, che dal suo
  cattivo colore indichi poca sanità, o grave disgusto, che travagliando l'animo,
  il corpo, E \textit{brutta cera} vuol dir ancora Fisonomia cattiva.

\item[PARE un'Ebreo c'habbia perduto il pegno] Quand'uno per qualche disgusto mostra
  faccia malinconica ci serviamo di questo detto, perché o sia vero, o sia nostra
  opinione, rarissimi sono gli Ebrei, che habbiano faccia allegra; ma un' Ebreo
  che habbia perduto il pegno aggiunge melanconia a malenconia, e però
  mostra faccia deformatissima.
\end{description}

\section{Stanza XXXII. --- XXXIIII.}

\begin{ottave}
\flagverse{32}Signore (incominciò) devi sapere,\\
Ch'io hebbi un bel marito, ma perch'io \\
Dissi chi egli era contro al suo volere,\\
Già per sett' anni n' ho pagato il fio; \\
Perché egli allor per farmela vedere \\
Stizzato meco sen' andò con Dio \\
In luogo; che a volerlo ritrovare \\
La carca ci volea da navigare.
\end{ottave}

\begin{ottave}
\flagverse{33}E quando poi io l'ho bell', e trovato,\\
Martinazza, ch'è sempre lo Scompiglia,\\
Fa sì che pur di nuovo m'è scappato,\\
Ed in mia vece all'amor suo s'appiglia,\\
Tal ch'io rimango cacciator sgraziato;\\
Scuopro la lepre, e un'altro poi la piglia.\\
Ti dico questo; perché havrei voluto\\
Che tu mi dessi a raccatrarlo aiuto.
\end{ottave}

\begin{ottave}
\flagverse{34}El le promette, e giura, ch'il marito \\
Le renderà, però non si sgomenti,\\
E se non basterà quel che ha smarrito, \\
Quattro, e sei bisognando, e dieci,e venti.\\
Ed ella lo ringrazia, e del seguito\\
Di tante sue fatiche, e patimenti\\
(Fatta più lieta per le sue promesse)\\
Così da capo a raccontar si messe.
\end{ottave}

Psiche espone a Calagrillo il suo bisogno, e lo richiede d'aiuto; Ei glielo promette,
ed ella fatta allegra per tal promessa, incominciò a discorrere, narrando
tutte le fatiche, e disagi patiti da lei in ricercare del Marito.

\begin{description}
\item[N'HO pagato il fio] N'ho pagato la pena; è il Lat. \textit{poenas dare}. Fio è voce
  Fiorentina antica, che vuol dir \textit{feudo}. Gio. Villani \libcap[5]{1}. \textit{Scomunicò
  Federigo, ed assolvette tutti li suoi Baroni da fio, e saramento}, ec, ma da noi hoggi non
  usata se non nel senso suddetto; nel quale anche l'usò Dante Purg, C, 10.
  \begin{verse}
    Di tal superbia qui si paga il fio.
  \end{verse}

\item[CI voleva la carta da Navicare] Era impossibil ritrovar quel luogo senz'haver
  la carta da navicare, o la bussola.

\item[L'HO bell'e trovato] L'ho già trovato. Vedi sopra \cstan[3]{14}. la forza di
  questo addiettivo \textit{bello} in questi termini.

\item[M' HA scartato] M' ha rifiutato. Traslato dal giuoco delle carte, che quando
  una carta, che habbiamo in mano non fa per noi, la buttiamo sopr'al monte
  delle carte; il che si dice scartare, vedi sotto \cstan[8]{6}. alla voce Minchiate.

\item[A RACCATTARLO] Cioè ritrovarlo, riaverlo, ricuperarlo. Il proprio
  significato di raccattare è Ragunare, mettere insieme. Vedi sotto \cstan[10]{37}.

\item[NON si sgomenti] Non si perda d' animo, non si sbigottisca. Petr. 42. 4.
  \begin{verse}
    E fol della memoria mi sgomento.
  \end{verse}
  Dante nel Purg. C. 14. in significato attivo. '
  \begin{verse}
    Cacciator di quei lupi in su la riva
    Del fiero fiume, e tutti gli sgomenta.
  \end{verse}

\item[SMARRIRE] È un certo perdere con speranza di ritrovare. Dan. Inf, C, 1.
  \begin{verse}
    Che la diritta via era smarrita
  \end{verse}

\item[QUATTRO, sei, e dieci, e venti] Scherza facendo, che Calagrillo prometta
  più di quel ch'è richiesto, come fanno tutti i bravazaoni, e in tanto mostra, che
  a una bella donna non mancano mariti.
\end{description}

\section{Stanza XXXV --- XXXIX}

\begin{ottave}
\flagverse{35}Cupido è la mia cara compagnia,\\
Ricco garzon, se ben la carne ha ignuda,\\
Anzi non è, t' ho detto una bugia,\\
Perch'ei non mi vol più cotta, ne cruda,\\
Ma senti pure, e nota in cortesia:\\
Quando la madre sua ch'era la Druda\\
Del Fiero Marte, idest la Dea d'amore\\
Gravida fu di questo traditore;
\end{ottave}

\begin{ottave}
\flagverse{36}Perch' una trippa havea, che conveniva,\\
Che dale cigne homai le fusse retta,\\
Cagion ch'in Cipro mai di casa usciva,\\
Se non con i braccieri,  ed in Seggetta,\\
Pur sempre con gran gente, e comitiva,\\
Com' a Regina; com' ell'è, s'aspetta,\\
I paggi ha dietro, e gli staffier dinanzi,\\
E dagl'inlati due filar di Lanzi.
\end{ottave}

\begin{ottave}
\flagverse{37}Essendo così fuori una mattina \\
Per suoi negozzi, e pubbliche faccende,\\
Urtò per caso una Vacca Trentina,\\
E tocca a pena, in terra la distende;\\
Ond' ella dopo un'alta rammanzina,\\
Perch'una lingua ell'ha che taglia, e fende:\\
Va, che tu faccia, quando ne sia otta\\
Un figliuol (dice) in forma a una botta
\end{ottave}

\begin{ottave}
\flagverse{38}E così fu ch'in vece d'un bel figlio\\
Di suo gusto, e di tutti i Terrazzani\\
Un rospo fece come un pan di miglio,\\
C'havrebbe fatto stomacare i cani;\\
Che poi cresciuto, fecesi consiglio\\
Di dargli un po di moglie, ma i mezzani\\
Non trovaron mai donna, ne fanciulla,\\
Che saper ne volesse, o sentir nulla.
\end{ottave}

\begin{ottave}
\flagverse{39}Se non ch'i miei maggiori finalmente\\
Mio padre ch'il bisogno ne lo scanna,\\
Con un mio Zio ch' andava pezziente,\\
E un mio fratello anch'ei povero in canna,\\
Sperando tutti tre d' ungere il dente,\\
E dire: O corpo mio fatti capanna,\\
E riparare ad ogni lor disastro,\\
Me gli offeriro; e fecesi l'impiastro.
\end{ottave}

Racconta Psiche a Calagrillo la dolorosa storia, e facendosi dalla nascita di
Cupido dice, che nacque in forma di rospo per la maladizione d' una vecchia, e
che poi cresciuto fu a lei dato per marito.

\begin{description}
\item[NON mi vuol cotta, ne cruda] Ne a lesso, ne a rosto. Non mi vuol più in
  maniera nessuna. Il Lalli En. Tr. lib. 2. stan. 42. dice:
  \begin{verse}
    Non gli volle annasar crudi, ne cotti
  \end{verse}
\item[DRUDA] Innamorata, tanto in bene quanto in male; perché si dice amante,
  innamorato, damo, non sempre in significato disonesto.\\
  Dan. Par. C. 12. \textit{Dentro vi nacque l'amoroso Drudo}\\
  \makebox[30pt]{}\textit{Della fede Cristiana il S. Atleta.} Parla di S. Domenico.

  Se bene nel presente luogo s'intende Meretrice, concubina.

\item[CIGNE] Sono striscie di quoio, o d'altra materia adattate a sostenere, e
  tenere insieme qualsivoglia cosa, dette cigne, da cignere.

\item[BRACCIERI] Coloro, sopr' alle braccia de' quali con una mano s'appoggiano
  le Dame andando a piedi per la Città.

\item[DAGL'inlati] Dalle bande, da i lati. Idiotismo usato assai \textit{in lati} per lati.

\item[LANZI] Così chiamiamo i soldati Tedeschi della guardia pedestre del Seren.
  G. Duca. Vedi sopra \cstan[1]{52}.

\item[VACCA Trentina] Così chiamiamo certe donnicciuole poco honeste, sfacciate,
  ed ardite, che non portano rispetto a veruno; e credo che si dica così per
  la similitudine, che hanno con le Vacche di Trento, le quali per esser' avvezze a
  sempre per le campagne del Tirolo, sono falvatiche, e feroci.

\item[RAMMANZINA] È lo stesso, che rammanzo detto sopra C.1.st. 52., e che
  rabbuffo nel med. C. st 39. Da alcuno è definita così: Riprensione fatta con parole
  minaccevoli, e ingiuriose. Forse dalle dicerie de' romanzi.

\item[HA una lingua che taglia, e fende] Ha una cattiva lingua, che dice ogni sorta
  male senza rispetto, o riguardo alcuno, che \textit{lacera l'altrui riputazione}.

\item[HAVREBBE fatto stomacare i cani] Così sporco, e nefando, che havrebbe
  provocato il vomito fino a i cani per la sua schifezza. In questo senso i Latini
  pure si servivano del verbo \textit{stomachari}.

\item[DARGLI un po di moglie] La voce \textit{poco} è usata da noi in diverse maniere; o
  declinabile, che significa quantità, come \textit{dategli un poco di carne}; o indeclinabile
  per avverbio; come andare un poco a Roma; Dategli un po di moglie, e serve per
  emfasi al discorso, e non per quantità, potendosi dire \textit{andate a Roma}: \textit{Dategli
    moglie}, che tanto esprime senza la voce \textit{poco}, la quale però nel presente luogo non
  è ripienezza, o (come diciamo) borra; ma è così detto per mostrarne l'uso,
  che appresso di noi e frequentissimo, ma nel caso come il presente è tanto usato,
  che non pare si possa dire altrimenti. Quel po per poco è la figura apocope
  usatissima da noi in questa, ed in altre voci enunciate sopra \cstan[1]{36}.

\item[MEZZANO] Sensali. Coloro che sono mediatori a conchiudere ogni sorta
  d'affare.

\item[IL bisogno ne lo scanna] È poverissimo; muore di necessità; la voce \textit{scannare}
  s'usa da noi per esprimere un soverchio desiderio di qualsivoglia cosa, se bene il
  suo più proprio è della fame, come s'è veduto sopra in \cstan{24}.

\item[PEZZIENTE] Povero, che chiede limosina. Deriva dal Latino \textit{petere} onde,
  povero pezziente vuol dir \textit{pauper petens eleemosinam}; ed è lo stesso che \textit{povero
    in canna}, quasi ignudo come una canna; altri vogliono, che quello \textit{incanna} sia una sola
  parola, e voglia dire \textit{incannatore}: Che quando un' huomo si mette a incannare,
  è segno, che è miserabile, perché il guadagno dell'incannare è infelicissimo.
  Il Varchi Stor. Fior. lib. 12. \textit{Perderono tutto quello, che in molt'anni havevano raggruzzolato,
    e diventarono poveri in canna}. Franco Sacc. Nov. 181. \textit{Voi altri Astrologi,
    per guardar sempre il Cielo, perdete la Terra, e siete sempre poveri in canna}.

\item[UNGER il dente] Mangiar roba, che unga il dente come carne, ec. e non
  sempre pane, come son necessitati fare i mendichi; e vuol dire Far miglior vita,
  mangiar un po meglio.

\item[DIRE al corpo: fatti capanna] Haver tanto da mangiare, che gli convenga
  pregare il Cielo, in diventare il suo corpo capace quanto una stanza da
  riporre il fieno (che questo vuol dir Capanna) per haver luogo dove riporre
  tanta roba. Usiam questo termine quando veggiamo uno avvezzo a vivere miseramente,
  e che si trovi poi a un banchetto lautissimo.

\item[SI fece l'impiastro] Cioè s'accordo, si conchiuse il negozio.
\end{description}

\section{Stanza XXXX \& XXXXI}
\begin{ottave}
\flagverse{40}Fu volentier la scritta stabilita, \\
Io dico sol da lor, che fan pensiero \\
Di non havere a dimenar le dita,\\
Ma ben di diventar lupo cerviero;\\
E, perché e' son bugiardi per la vita,\\
Dimostrano a me poi il bianco pel nero\\
Dicendomi, che m'hanno fatta sposa\\
D'un giovanetto, ch'è sì bella cosa.
\end{ottave}

\begin{ottave}
\flagverse{41}Soggiunsero di lui mill' altre bozze,\\
Ma quando da me poi lo veddi in faccia\\
Con quella forma, e membra così sozze,\\
Pensate voi se mi cascò le braccia,\\
Anzi nel giorno proprio delle nozze,\\
C'a darmi ognun venia il buon prò vi faccia,\\
Ogni volta con mio maggior dolore\\
Sentivo darmi una stoccata al cuore.
\end{ottave}

Psiche continova il racconto, e dice, che finalmente fu conchiuso il parentado
fra lei, e il Rospo figliuolo di Venere.

\begin{description}
\item[STABILITA la scritta] Fermato, e conchiuso il contratto del Matrimonio,
  che appresso di noi si dice La scritta del parentado.

\item[NON havere a dimenar le dita]Cioè haver a viver senza liavorare, senza durar fatica.

\item[DIVENTAR lupo Cerviero] Divorare, mangiar voracemente, come fa il Lupo
  cerviero\footnote{Lupo Cerviero: lupo che dà la caccia ai cervi. Altro nome della Lince.}. Plin. 1.8.c.22, de \textit{Lupis} dice così: \textit{Sunt in eo genere qui Cervarii vocantur,
  qualem e Gallia Pompeij Magni arena spectatum diximus, huic quamvis in
  fame mandenti si respexit, oblivionem cibi surrepere aiunt, digressumque quaerere aliud}.
  E da tale agonia di mangiare s'assomiglia un huomo, che mangi voracemente,
  ad un lupo cerviero.

\item[BOZZE] Intendi bugie, fandonie, trovati non veri, finzioni, e simili.
  Quando non vogliamo credere qualche novità, che ci sia raccontata diciamo:
  \textit{Io l'ho per bozza}. Traslato da i Pittori, che dicono \textit{bozze}, e \textit{abbozzare} quelle
  prime pennellate, che danno in una tela, e gli Scultori quei primi colpi, che
  danno in un marmo, o altro; i quali additano un non so che del vero, che vi
  faranno col finirle. Vedi sotto \cstan[7]{5}.

\item[MI cascò le braccia] M'abbandonai; mi perdei d'animo; mi sgomentai.
\end{description}

\section{Stanza XXXXII --- XXXXIV}

\begin{ottave}
\flagverse{42}Non lo volevo; pur mi v' arrecai \\
Veduto havendo ogni partito vinto:\\
Ma perché non è il Diavol sempre mai\\
Cotanto brutto com' egli è dipinto,\\
Quand'io più credo a gola esser nei guai\\
Ecco al mio cuore ogni travaglio estinto,\\
Vedendo  ch'ei lasciò, send'a quattr'occhj;\\
La forma delle bozze, e de' ranocchi.
\end{ottave}

\begin{ottave}
\flagverse{43}E molto ben divenne un bel garzone,\\
Che m'accolse con molta cortesia,\\
Ma subito mi fa commissione,\\
Ch'io non ne parli mai a chi che sia;\\
Perch'io sarò, parlandone, cagione,\\
Ch'ei si lavi le mani de fatti mia,\\
E per ne men sentirmi nominare\\
Si vada vivo vivo a farsi sotterrare.
\end{ottave}

\begin{ottave}
\flagverse{44}E perché quivi ancora havrà paura\\
Ch'io non vada a sturbargli il suo riposo,\\
Havrà sopr' ad un monte sepoltura,\\
Che mai si vedde il più precipitoso,\\
Ed alto poi così fuor di misura,\\
Che non v'andrebbe il Bartoli ingegnoso;\\
Oltre che innanzi ch'io vi possa giugnere\\
Ci vuol del buono, e ci sarà da ugnere.
\end{ottave}

Cupido si mostra a Psiche in forma d'un bel giovane, lasciata la fozza figura del
rospo, ed a lei fa comandamento, che di ciò in maniera alcuna non parli, perché altrimenti
facendo; sarà cagione, che egli la lasci, e se ne vada in luogo da non poter
esser più trovato.

\begin{description}
\item[MI v'arrecai] Condescesi; acconsentj, mi v'accomodai; vedi in questo Can,
stan. 80. preso per accomodarsi col corpo; e qui è preso per accomodarsi con
l'animo.

\item[VISTO il partito vinto] Veduto che la cosa haveva a andare in quella guisa.
  La voce \textit{partito} ha diversi significati: perché vuol dire Scrutinio, che noi
  corrottamente diciamo \textit{squittino}. Vedi sotto Can. 6. stan. 109., e di qui \textit{Visto il partito
  vinto}, vuol dire Visto, che il negozio era stabilito così, perché quando il partito
  è vinto, il negozio s'intende stabilito. \textit{Metter il cervello a partito}, significa
  metter in dubbio uno se deva fare, o non fare una tal cosa. \textit{Donna di partito} vuol
  dir meretrice. Si piglia in vece \textit{d'accordo}, \textit{patto}, \textit{baratto}, o \textit{condizione}. Io vendo
  una cosa col tal partito, ec.  Significa \textit{risoluzione}, o \textit{determinazione}. Io ho preso
  partito d'andarmene. Significa \textit{termine}, \textit{pericolo}, Il tale si condusse a mal partito,
  cioè a \textit{cattivo termine}, o a \textit{pericolo di vita}, o \textit{povertà}. Ci serve per esprimer
  \textit{maniera}, \textit{modo}: lo non vi verrò a partito alcuno. Significa \textit{rimedio}, \textit{espediente}.
  Presero per partito di segargli la gamba, ec.

\item[IL Diavol non è brutto com' egli è dipinto] Il Male non è poi sempre tanto, quanto vien raccontato.

\item[NE' guai a gola] Immerso nelle disgrazie. Vedi sopra \cstan[2]{44}. il suo contrario.

\item[A QUATTR'occhi] A solo a solo, \textit{Remotis arbitris}.

\item[SI lavi le man de' fatti mia] Non voglia saper più nulla dime. Tratto dall'antico,
  come si vede in Pilato, che col lavarsi le mani pretese di non haver, che fare
  della Sentenza data contro al nostro sig.\ Giesù Cristo. Il Lalli Eneid. Trau.
  C.4. stan. 92.
  \begin{verse}
    E mi lavo le man de fatti tuoi
  \end{verse}

\item[IL Bartoli ingegnoso] Il Bartoli\footnote{potrebbe riferirsi a Cosimo Bartoli, (Firenze, 20 dicembre 1503 – Firenze, 25 ottobre 1572), che tradusse in lingua toscana il trattato De Architectura di Leon Battista Alberti.}, che ha stampato un trattato dell'architettura,
  perché dice ingegnoso cioè ingegniere, che appresso di noi vuol dire Architetto;
  e non Bartolo legista\footnote{Bartolo da Sassoferrato (Sassoferrato, 1314 – Perugia, 13 luglio 1357) giurista.} (come si trova in alcuni testi, dove dice Bartolo, e non il
  Bartoli) perché trattandosi di salire un luogo erto può giovar più i sapere
  d'un'Architetto, che quello d'un Legista.

\item[CI vuol del buono] Ci sarà molto da faticare, o da spendere, o da camminare,
  o simili, servendoci questo termine per intender tutto quello ci possa esser necessario
  in uno affare, secondo la subietta materia, come per esempio: A scriver
  la presente Opera ci vuol del buono, e s'intende ci vuol molto tempo, molta
  fatica, molti fogli, ec. ed è lo stesso che \textit{ci sarà da ugnere}. Il che viene dal
  medicare i feriti, e però per lo più s'usa in cose di poco gusto, e fastidiose, per esempio:
  Il tale ammazzò uno, vuol haver da ugnere, cioè vuol haver molti travagli,
  spese, difficultà, ec. ad aggiustare il negozio. Il Mureto \libcap[9]{13}. Var.
  lect. disse: \textit{Non parva \& pauca sed multa \& magna ad hoc efficiendum requiruntur}.

\end{description}
\section{STANZA XXXXV.}
\begin{ottave}
\flagverse{45}Poi ch' una strada troverò nel piano,\\
Che veder non si può già mai la peggio,\\
Poi giunto a pié del monte alpestre, e strane\\
Con due uncini arrampicar mi deggio\\
Menando all'erta hor l'una, hor l'altra mano,\\
Come colui, che nuota di spasseggio,\\
Ed anche andar con flemma, e con giudizio\\
S' io non me ne vogl'ire in precipizio.
\end{ottave}

\begin{ottave}
\flagverse{46}Scosceso è il monte in somma, e dirupato,\\
E il viaggio lunghissimo, e diferto,\\
Così disse Cupido smascherato,\\
Dopo cioè ch'ei mi si fu scoperto;\\
Ond'io promessi di non dir mai fiato,\\
E che prima la morte havria sofferto,\\
Che trasgredir d'un punto in fatti, o in detti\\
A suoi gusti, a suoi cenni, a suoi precetti.
\end{ottave}

Cupido accenna a Psiche parte delle fatiche, e travagli, che ella havrà nell'andare
a ricercarlo; e Psiche gli promette di non dir mai nulla a nessuno.

\begin{description}
\item[VNCINI] Strumenti di ferro adunchi, ed aguzzi, servono per appiccarsi a
  gualcosa, e si fanno anche di legno per uso di corre frutti, e per altre occorrenze
  rustiche.

\item[RAMPICARE] E proprio dei gatti, e d'altri animali simili, che salgono
  su per gli alberi, appiccandosi co' rampi, cioè con l'ugna delle zampe. Vedi
  sotto in \cstan{68}. E ci serviamo del verbo rampicare per esprimere un
  che salga in qualche luogo difficile, ancor che lo faccia senza rampicare. Vedi
  sotto \cstan[9]{25}.

\item[NVOTA di spasseggio] Nuotare di spasseggio diciamo quand' uno essendo tutto
  nell' acqua dalla testa in fuori, cava fuora di essa un braccio per volta ordinatamente,
  battendolo sopra all'acqua per romperla, e spingersi avanti.

\item[NON dir fiato, e non fiatare] È lo stesso che non parlare. Vedi sotto C.6. st. 12.
  Si dice anche \textit{non alitare}. \textit{Non far verbo}, Berni Orland.
  \begin{verse}
    E senza più fiatar mi stava chiotto.
  \end{verse} Vedi sopra \cstan[1]{10}.

\item[GVSTI, cenni, precetti] In questo luogo hanno tutti tre lo stesso significato
  di comandamento. Considerandosi \textit{gusto} per il meno stimato, \textit{cenno} nel secondo
  luogo, e \textit{precetto} per lo più stimato, denotando dominio.
\end{description}
\section{Stanza XXXXVII --- XXXXVIII.}

\begin{ottave}
\flagverse{47}Ne tal cosa a persona haurei scoperta,\\
Perché tuttavia la gente sciocca\\
Ridea del rospo, e davami la berta;\\
Ed io, che quand'ella mi venne in cocca,\\
Non so tener un cocomero all'erta,\\
Mi lasciai finalmente uscir di bocca,\\
Che quel non era un rospo, ma in effetto\\
Un grazioso, e vago giovanetto.
\end{ottave}

\begin{ottave}
\flagverse{48}E che, se lo vedesson poi la notte\\
Quand'in camera meco s'è serrato,\\
E getta via la scorza delle botte\\
Ch'un sole proprio par sputato,\\
Le male lingue forse starian chiotte\\
Che sì de' fatti altrui si danno piato,\\
Però che non si può tirar un peto\\
Ch'il comento non veglian fargli dreto.
\end{ottave}

Vinta Psiche dalla collera, che le venne per esser burlata dall'altre donne,
scoperse il segreto, E nota che l'Autore mostra il costume delle nostre femmine,
e quelle di tutto il mondo, le quali obligate a narrar qualche loro mancamento;
si fanno dalla lontana, e cercano di persuadere d' haverlo commesso, necessitate, e
forzate da' maggiori mancamenti d' altri.
\begin{description}
\item[DAVANMI la berta] Mi davano la burla, mi beffavano, mi minchionavano.
  Berta si dice quel ceppo, col quale, impernato sopra i pali, si fanno le palizzate
  ne i fiumi, battendo sopra i pali per via di corde, o manichetti, che sono in
  detto ceppo. E il Latino irridere, Raccontano le nostre donne, che quel sagace
  villano nominato Campriano, del quale diremo sotto \cstan[11]{48}. essendo venuto
  in mano della giustizia per le sue cattive opere fu condennato a esser messo
  in un sacco, e buttato in mare; In esecuzione di che fu messo dentro al sacco, e
  consegnato a i famigli, che lo buttassero in mare. Nell'andar costoro ad eseguire
  gli imposti furono per strada assaliti da alcuni masnadieri, i quali si crederono,
  che in quel sacco fusse roba di valore; onde i famigli per scampar la vita
  lasciato ivi il sacco con Campriano, si fuggirono. Campriano piangendo
  si doleva della sua disgrazia, il che sentito da uno di quei masnadieri gli domandò
  perché piangeva, ed a qual fine era stato messo in quel sacco. Il sagace Campriano
  gli rispose; Io piango di quel, che altri gioirebbe, ed è, che questi SS.
  voglion darmi per moglie Berta unica figliola del Re nostro, ed io non la voglio,
  conoscendomi inabile a tanto grado, per esser' un povero villano. E perché essi
  dicono, che se ella non si marita a me, l'oracolo ha detto, che questo Regno andrà
  sottosopra, m'hanno messo in questo sacco per condurmi a farmela pigliar
  per forza; e questa è la causa del mio pianto. Il masnadiero credendo alle parole
  di costui, si concertò con i Compagni d'andar'esso a pigliare questa buona
  fortuna, e ripartirla con essi: onde fattosi mettere dentro al sacco da Campriano
  che non restava di pregarlo a volergli far del bene quando fusse poi Re,
  fece allontanare i compagni, e serratolo entro al sacco, stette aspettando, che
  ritornassero coloro, i quali non stettero molto a comparire con nuova gente, e
  veduto quivi il sacco abbandonato, lo ripresero, ed essendo vicini alla riva del
  mare, ve lo precipitarono, e così sposarono a Berta il balordo mafnadiero. E
  di qui venne \textit{dar la berta}, o \textit{la figliuola del Re}, che vuol dir \textit{burlare}, \textit{minchionare},
  come habbiamo accennato. Si dice anche \textit{dar la madre d' Orlando}, perché da
  alcuni si crede, che la madre d'Orlando Paladino havesse nome \textit{Berta}.

\item[QUAND' ella mi viene in cocca] Quando mi viene in proposito di dire. E si dice
  anche \textit{ella mi viene in cocca} per intendere quand' io entro in collora, come
  s'intende nel presente luogo. E cocca diciamo quella tacca la quale e nella freccia
  per adattarla in su la corda dell'arco\footnote{Da cui evidentemente \textit{incoccare} e \textit{scoccare}.} da i Latini detta \textit{Crena}, donde poi diciamo
  \textit{cruna}, quella tacca, o fessura, che è nella parte opposta alla punta dell'ago da
  cucire, dal Gr. \textit{Acocche}; \textit{estremità acuta}, Dan, Inf. C. 12.
  \begin{verse}
    Chiron prese lo strale, e con la cocce
    Fece la barba indietro alle mascello
  \end{verse}

\item[NON so tenere un cocomero all'erta] non posso far di meno di non la dire. Si
  fa questa comparazione al cocomero, perché essendo questo di, figura sferica, e
  liscio, facilmente ruotolando può scorrer giù per un'erta, o monte, e facilmente
  può esser anche tenuto fermo; onde molto ben si dice Non fa tener un  cocomero
  all'erta d'uno che sia facile a palesare un segreto, che con ugual facilità
  potria tacerlo.

\item[PRETTO sputato] Similissimo a lui: per appunto come lui, e senza alterazione
  alcuna come è il vino pretto, cioè senz'alterazione d'acqua, o d'altro. E
  quella aggiunta di sputato si toglie da coloro, che pigliano le misure col filo, come
  muratori, e legniaioli, i quali in qualche occasione per andar giusti, e' per appunto
  sogliono tirare il filo, e sputandovi sopra lasciano cascar lo sputo nella
  parte, che gli è sotto, e da quello conoscono se il lavoro e per appunto.

\item[CHIOTTE] Chete. Voce Fiorentina, ma poco usata fuor di scherzo, se bene,
  come poco sopra s'è visto, l'usò il Berni nell'Orlando. \textit{E senza più fiatar ne stava chiotto}.

\item[SI danno piato de' fatti d'altri] Gli danno pensiero; Gli sono a cuore i fatti d'altri.
  Si metterebbero a litigare per i fatti d' altri; Che \textit{Piato} vuol' dir \textit{litigio}.
  Vedi sotto \cstan[7]{27}.

\item[NON si può tirar un peto ec.] Non si può far una cosa benché minima, che il
  popolo non vi voglia far sopra i suoi discorsi.
\end{description}

\section{STANZA IL --- LIII}

\begin{ottave}
\flagverse{49}Le ciglia inarca, e tien la bocca stretta\\
Chiunque da me tal maraviglia ascolta;\\
Ma quel ch'importa a sordo non fu detta,\\
Che Vener, ch'ogni cosa havea ricolta,\\
Per veder s'ell'è vera, o barzelletta,\\
Poiché a dormire ognun sel' era colta,\\
Entra in camera, e vien pian, Piano al letto,\\
E trova il tutto appunto come ho detto.
\end{ottave}

\begin{ottave}
\flagverse{50}E nel vedere in terra quella sposiglia,\\
Che per celarsi al mondo il giorno adopra,\\
Di levargliela via le venne voglia, \\
Acciò con essa più non si ricuopra: \\
Così la prende, e poi fuor della soglia \\
Fa un gran fuoco, e ve la getta sopra, \\
Ne mai di lì si volle partir Venere \\
Infin che non la vedde fatta cenere.
\end{ottave}

\begin{ottave}
\flagverse{51}Fu questa la cagion d'ogni mio male,\\
Perché quando Cupido poi si desta\\
Si stropiccia un po gli occhi, e dal guanciale\\
Per levarsi dal letto alza la testa,\\
E và per rivestirsi da animale,\\
Ne trovando la solita sua vesta\\
Si volta verso di me, si morde il dito,\\
E nello stesso tempo fu sparito.
\end{ottave}

\begin{ottave}
\flagverse{52}Non ti vo dir com'io restassi allora,\\
Che mi fovvenne subito di quando\\
Il primo dì mi si svelò, c'ancora\\
Mi fece l espertissimo comando,\\
Ch'in alcun tempo io non la dessi fuora,\\
Ed io son' ita sciocca, a farne un bando,\\
E poi mi pare strano, e mi scontroco,\\
S'egli è in valigia, ed ha comprato il porco.
\end{ottave}

\begin{ottave}
\flagverse{53}Sospesa per un pezzo io me ne stetti, \\
Chi io aspettavo pur ch' ei ritornasse; \\
A cercarne per casa poi mi detti \\
Per le stanze di sopra, e per le basse; \\
Guardo su pel cammin, giro in su i tetti, \\
Apro gli armarj, e a scostar le casse,\\
Ne trovandolo mai, al fin mi muovo\\
Per non fermarmi fin ch'io non lo trovo.
\end{ottave}

Il segreto palesato da Psiche, venne all'orecchie di Venere, la quale quando
Cupido dormiva gli abbruciò la veste da rospo; il che veduto Cupido la mattina
se ne fuggì, e Psiche si messe a cercar di lui.

\begin{description}
\item[NON fu detta a sordo] Fu detta a chi ne fece capitale, a chi importava saperlo.

\item[OGNI cosa havea raccolto] Haveva sentito, e inteso ogni cosa.

\item[BARZELLETTA] Cosa non vera, ma detta per scherzo. E si dice Barzellettare uno,
  che discorra burlando, e scherzando.

\item[PIAN piano] Questo termine, che vuol dire Adagio adagio, significa ancora
  (come nel presente luogo) Senza far punto strepito, o romore.

\item[GUANCIALE] Piccolo piumaccio, sopra il quale si posa la guancia, quando
  si sta nel letto, detto \textit{guanciale} da guancia, come in diversi luoghi, è detto
  \textit{origliera} da orecchio.

\item[RIVESTIRI] Rivestirsi da rospo. Ecco la voce generica animale, che noi
  usiamo le, come accennammo sopra in \cstan{4}.

\item[NON  ti vo dire] È lo stesso termine, \textit{che pensate voi}, visto sopra in questo C.
  stan 41. Ed esprime Non voglio dirlo, perché da per voi vel' immaginerete; Vedi sotto la stan. 76.

\item[NON la dessi fuora] Non la manifestassi, ed io n' ho fatto un bando; ed io, ho
  pubblicata per tutto. \textit{Non modo tubam, sed etiam praeconem adhibui}.

\item[MI scontorco] Scontorcersi e proprio delle serpi ferite; e parlandosi d'huomini
  s'intende un certo atto, che denota dolore per qualche disgusto, o travaglio insopportabile.

\item[È IN valigia] È in collora, in ira; Nel bugnolone, nel gabbione, e simili,
  che in moltissimi ne habbiamo in questo significato.

\item[COMPRAR il porco] Significa andarsene; ed è come l'interpetrazione di \textit{svignare},
  quasi voglia dire \textit{suinam}, cioè \textit{suillam emere}, o che più tosto sia detto
  \textit{svignare} quasi \textit{scappar via dalla vigna, e fuggirsene}, come quei che son colti a
  cogliere, o mangiare uva nell'altrui vigna. Diciamo \textit{battere il taccone}, \textit{battersela},
  \textit{corsela}, e che se ben son voci, che hanno del furbesco, sono però comunemente
  usate, e sempre intese in questo senso. Vedi sotto C.~11. stan.~11.

\end{description}
\section{Stanza LIV --- LVIII}

\begin{ottave}
\flagverse{54}Scappo di casa, e via vò sola sola,\\
Ne son lontana ancora una giornata,\\
C'io sento dire: Aspettami figliuola,\\
Mi volgo, e dietro veggomi una Fata,\\
E perch'ella mi diede una nocciuola,\\
Quest'è meglio, diss'io, d'una sassata,\\
Di ciò ridendo un'altra sua compagna\\
Mi pose in mano anch'ella una castagna.
\end{ottave}

\begin{ottave}
\flagverse{55}Ed io, c'allora harei mangiato i sassi\\
M'accomodai per darvi su di morso,\\
Ma fummi detto ch'io non la stiacciassi,\\
S'un gran bisogno non mi fusse occorso.\\
Vergognata di ciò con gli occhi bassi\\
Il termine aspettai del lor discorso,\\
Poi fatte le mie scuse, e rese ad ambe \\
Mille grazie, le lascio, e dolla a gambe.
\end{ottave}

\begin{ottave}
\flagverse{56}Ripongo la nocciuola, e la castagna,\\
E rimetto le gambe in sul lavoro\\
Per una lunga, e sterile campagna\\
Disabitata più che lo Smannoro;\\
Dopo cinque anni giunta a una montagna,\\
Mi si fe innanzi un grande, e orribil toro,\\
Che ha le corna, e i più tutti d'acciaio,\\
E tira che correbbe nel danaio.
\end{ottave}

\begin{ottave}
\flagverse{57}E come Cavalier ch'al saracino\\
Corre per carnovale, o altra festa,\\
Verso di me ne viene a capo chino\\
Con la sua lancia biforcata in testa,\\
Io già con le budella in un catino\\
Addio dicevo al Mondo, addio chi resta.\\
Addio Cupido dove tu ti sia,\\
A rivederci ormai in pellicceria.
\end{ottave}

\begin{ottave}
\flagverse{58}O Mamma mia, che pena, e che spavento \\
Hebbe allor questa mezza donnicciuola? \\
Tremavo giusto come giunco al vento, \\
Che quivi mi trovavo inerme, e sola; \\
Pur come volle il Ciele io mi rammento\\
Del dono delle Fate, e la nocciuola\\
Presa per caso presto sur' un sasso\\
La scaglio, ella si rompe, e n'esce un masso.
\end{ottave}

Messasi in viaggio Psiche s'imbatté in due Fate, dall'una delle quali hebbe
una nocciuola, e dall'altra una castagna, e le dissero, che non le stiacciasse, se
non a un gran bisogno. Dopo cinque anni di cammino per un deferto arrivò a
pié d'una montagna, dove le venne incontro un toro con le corna d'acciaio;
dal quale spaventata Psiche stiacciò la nocciuola, e ne nacque un masso.

\begin{description}
\item[FATA] Fate sono donne indovine dette secondo alcuni dal Greco \textit{Phatis}, che
  suona Donna indovina, e quelle forse che i Latini co' Greci chiamano \textit{sibille},
  ma dalle nostre Balie nel contare le novelle a i fanciulli son prese per donne di
  buon genio, e che fanno servizio al prossimo con le loro azioni, e son contrarie
  all'Orco, al Bau, e alle Befane, che sono nimici de' bambini, a i quali queste
  sempre fanno servizio, ed il Poeta, col regalo, che fa lor fare a Psiche, mostra
  questa verità. Da gli antichi furono anche chiamate Ninfe, e Dee, e l'Ariosto
  nel suo Furioso ciò afferma, dicendo:
  \begin{verse}
    Queste c'hor Fate, da gli antichi furo
    Chiamate Ninfe, e Dee con più bel nome.
  \end{verse}
  Di queste Fate discorre  l'Autore sotto, nel Canto settimo, ed è credibile, che
  questa voce Fate venga dal Latino \textit{Fata fatorum}, che Dan, Inf, C, 9, disse le fata.
  \textit{Che giova nelle fata dar di cozzo?}

\item[QUESTO è meglio a una sassata] Quando si riceve da uno qualche regalo di poco
  valore, si dice per scherzo: \textit{Questo è meglio d'una sassata}, o vero \textit{d'un calcio di
    mosca}: volendosi inferire, che da quello, al nocivo, o al nulla vi è poca differenza.
  Plau. in Tr. disse \textit{Melius est quam deterrimum}.

\item[ALLOTTA haurei mangiati i sassi] Allora havevo così gran fame, che haurei
  mangiata qualsivoglia cosa, ancor che dura quanto un sasso. Io crederei, che il vestitore
  di questa favola havesse seguitato i compositori de' Palmerini, degli Amadis,
  ed altri Cavalieri erranti, che mai in tanti viaggi, che fanno lor fare, pur'
  una volta si trova, che in campagna mangiassero; ma il sentir, che Psiche discorre
  di mangiare, e che fu levata dond'ell'era, perché non vi morisse di fame,
  mi fa credere diversamente, cioè che in questo suo iungo viaggio le Fate le empiessero
  il corpo, che ella non sen' avvedesse.

\item[SCHIACCIARE], Corrottamente diciamo anche \textit{stiacciare}, vuol dir Rompere,
  o infragnere, ed è proprio di quelle cose, che hanno guscio, come noci, mandorle,
  uova, e simili.

\item[DOLLA a gambe] Comincio a camminare; è lo stesso che \textit{rimetto le gambe in
  lavoro}, che è nell'ottava 56. seguente. Il Lall. En. Tr. \cstan[2]{33}.
  \begin{verse}
    Quand'io la diedi a gambe,e dentro a un fosso
  \end{verse}
  Lasca Nov. 6. \textit{Temendo, che colui non gli uscisse dietro, s'uscì di casa prestamente, e
    la dette a gambe, e per la fretta si scordò di serrar l'uscio}. I Lat. pure dissero \textit{conijcere
    se in pedes}.

\item[LO smannoro] Così è detta una gran pianura posta poco lontana per di sotto alla
  Città di Firenze, la quale dura più miglia per ogni verso, senza mai trovarsi
  una casa, se bene è tutta coltivata. Si dovrebbe dire \textit{Ormannoro} dalla famiglia
  antica degli Ormanni, la quale era già padrona di tutte quelle pianure, che si dicevano
  \textit{Campi Ormannorum}.

\item[TIRA che correbbe in un denaio] Tira così aggiustatamente, che egli correbbe
  in ogni piccolo berzaglio, come è un denaro, che è la quarta parte del quattrino
  Fiorentino, con altro nome detto picciolo, ed un giulio ne vale 160.

\item[SARACINO] Così chiamiamo quella statua, o fantoccio di legno, che figura
  un Cavaliero armato, al quale (come a berzaglio) corrono i Cavalieri le lance;
  E si dice anche \textit{Buratto}, che è un' altra sorta di berzaglio (il quale si mette
  in vece del Saracino) ed è una mezza figura secondo alcuni, che nella sinistra.
  tiene lo scudo, nella destra la spada, o bastone; la quale se non è colpita nel petto,
  girando si rivolta, e percuote colui, che fallì.\footnote{Vedi pure al cantare precedente, ottava 75, la voce ``FOLA''.}

\item[LANCIA biforcata] Intende le corna del Toro.

\item[CON le budella in un catino] Mi credeva già morta; Mi credeva già essere stata
  sbudellata dal Toro. Luigi Groto Cieco d'Adria, in una sua lettera al Petr.
  dice: Quei cani con il loro bau bau ci fecero parere d'havere le budella in un
  catino.  E \textit{Catino} Intendiamo un vaso di terra, o d'altra materia per servizio di
  Cucina, e per uso di lavar piatti, ec.

\item[A RIVEDERCI in pellicceria] A rivederci fra i morti. Questo è il comiato,
  che noi finghiamo, che si diano le volpi  l'una con l'altra, perché sapendo, che
  devon esser'ammazzate, e le lor pelli vendute, dicono alli lor figli, quando da
  esse si separano: \textit{A rivederci in pellicceria}, che così si chiama in Firenze quella
  strada, nella quale sono le botteghe di coloro, che comprano, e vendono pelli
  di animali per foderare abiti, ec. ed in mano di costoro, o tardi, o per tempo
  sanno che devon capitare.

\item[O MAMMA mia] O mia madre. Esclamazione di spavento, e di timore,
  usata propriamente da' fanciullini, quasi dica: O mia madre soccorretemi in
  questo pericolo.

\item[DONNICCIVOLA] Vuol dir Donna di spirito minore di quel che converrebbe
  al suo naturale, da i Latini detta \textit{Muliercula}. Sì che mezza donnicciuola vuol
  dir Donna quasi da nulla, e senza spirito.

\item[GIUNCO] Specie di virgulto, che nasce in luoghi padulosi\footnote{sic: ``padulosi''.}, del quale si servono
  i Villani per legare i stralci teneri delle viti, ec.

\item[MASSO] S'intende un sasso grande. Questi nostri scarpellini chiamano il
  masso La cava delle pietre.
\end{description}

\section{ STANZA LIX. STANZA LX.}

\begin{ottave}
\flagverse{59}Tal pietra per di fuora è calamita,\\
E ripiena di fuoco artifiziato, \\
Hormai arriva il Toro, ed alla vita \\
Con un lancio mi vien tutto infuriato, \\
Ma perché dietro al masso ero fuggita \\
Il ribaldo riman quivi scaciato,\\
Ch'in esso dando la ferrata testa \\
da qulla calamita affisso resta.
\end{ottave}

\begin{ottave}
\flagverse{60}Sfavilla il masso al batter dell'acciaro,\\
E dà fuoco al rigiro ch'è nascosto,\\
Ed egli a' razzi ch' allor ne scapparo\\
Un colpo fatto haver vede a suo costo,\\
Perché non vi fu scampo, ne riparo,\\
Ch'ei fra le fiamme non si muoia arrosto,\\
Ed io scansato il fuoco, e ogni altro affronto,\\
Lieta mi parto, o tire innanzi il conto.
\end{ottave}

Il detto sasso era per di fuori calamita, e dentro era fuoco lavorate, onde il
Toro perquotendovi con le corna ch'erano d'acciaio vi rimasero appiccate, e
da quella percossa nacque il fuoco, il quale s'appiceò all'ordigno, ed abbruciò
il Toro. Psiche libera da questo incontro seguitò il suo viaggio.
\begin{description}
\item[CALAMITA] È la pietra simpatica del ferro, o forse madre, dai Latini
detta \textit{Magnes}. Vedi sotto \cstan[8]{45}. e 66.

\item[FUOCO artifiziato] Vuol dire ogni forma di composizione fatta con polvere
(che diciamo Da archibuso) tanto per guerra, quanto per feste.

\item[RIMANE scaciato] Riman burlato. È lo stesso, che \textit{rimaner con un palmo di
  naso}, che vedremo sotto \cstan[6]{5}.

\item[RIGIRO] Intende l'ordigno di fuoco lavorato, che è composto dentro al
  masso.

\item[RAZZI] Raggi di fuoco o del Sole, o d'altro scintillante. Ma dicendo assolutamente
  razzi, intendiamo quei fuochi artifiziati, che si fanno in occasione
  di feste con polvere d'archibuso constipata, e benissimo legata entro alla carta,
  ridotta come pezzi di canna,

\item[TIRO innanzi il conto] Seguito il mio viaggio, Vedi sotto \cstan[6]{16}.
  Tanto serviva \textit{tiro innanzi}, e senza mettervi \textit{il conto} suonava il medesimo,
  ma l'uso nato da quei, che tengono libri di debitori, e creditori ci obliga a dir così.
\end{description}

\section{ STANZA LXI --- LXVI}

\begin{ottave}
\flagverse{61}Più là ritrovo un grand' uccel grifone,\\
E topi assai, che giran come pazzi,\\
Perch'egli entrato in lor conversazione\\
Gli becca, grafia, e ne fa mille strazzj,\\
Di lor mi venne gran compassione,\\
E vo per ovviar, ch'ei, non gli ammazzi,\\
Ma quei mi sente al moto, e in pié si rizza,\\
E per cavarsi, vien con me la stizza.\\
\end{ottave}

\begin{ottave}
\flagverse{62}Questo animate ha il busto di cavallo\\
Di bue la coda, e in su le spalle ha l'ale,\\
Il capo, e il collo giusto come il gallo,\\
E i pié di nibbio vero, e naturale,\\
Gli artigli di fortissimo metallo\\
Grandi grossi, e adunchi in modo tale\\
Che non vedesti quando leggi, o scrivi,\\
Mai de tuoi di più bei interrogativi.
\end{ottave}

\begin{ottave}
\flagverse{63}Son' appuntati poi c'a far più acuto\\
Un'ago altrui darebbe delle brighe,\\
Tal che, s'al viso fussimi venuto\\
Con essi, mi lasciava assai più righe\\
D'un libro di maestro di liuto,\\
Ed una stamperia di falsarighe,\\
Con farmi a liste come le gratelle\\
Da quocervi le triglie, e le sardelle.
\end{ottave}

\begin{ottave}
\flagverse{64}Hor per tornare. In quel ch'io ho timore\\
Ch' il mio grifo sia scherzo del grifone\\
La castagna ch'io in tasca caccio fuore\\
La rompo, e n'esce subito un Lione,\\
Che mi scemò non poco il batticuore\\
Perch'egli in mia difesa a lui s'oppone,\\
E mostrogli hor con l'ugna, ed hor co' denti\\
In che mo si gastigan gli insolenti.
\end{ottave}

\begin{ottave}
\flagverse{65}L'uccello anch'egli, che non ha paura\\
Gli rende molto ben tre per pan per coppia,\\
Ma quel che haver del suo nulla sicura\\
Il contraccambio subito raddoppia,\\
E ben ch'ei voglia star seco alla dura\\
L'afferra, e stringe tanto ch'egli scoppia\\
Di poi garbatamente gli riesca\\
Gli stinchi su i nodelli, e me gli reca.
\end{ottave}

\begin{ottave}
\flagverse{66}Metto uno strido, e mi ritiro in dreto\\
Io ch'ho paura allor ch'ei non m'ingoi,\\
Ma quegli ch'è un Lione il più discreteo,\\
Che mai vedesse il mundo prima, o poi,\\
Ciò conoscendo tutto mansueto\\
Gli lascia in terra, e va pe' fatti suoi,\\
Ed io gli prendo allora, essendo certa\\
D'averne a haver bisogno in sì grand'erta
\end{ottave}

\begin{ottave}
\flagverse{67}Lá dove non si può tenere i piedi,
Ma bisogna che l'huom vada carponi,\\\
Perciò con quegli uncini poi mi diedi\\
A costeggiar il monte brancoloni,\\
E convenne talor farsi da piedi\\
Battendo giù di grandi stramazzoni,\\
Perché non v'è dove fermar il passo:\\
Cagion che spesso mi trovai da basso.
\end{ottave}

Psiche superato il pericolo del Toro s'imbatte in un' uccello Grifone, che havea
l'ugna d'acciaio, onde roppe la castagna, e n'usci un Lione, che la difese
da quello uccello, e tagliandogli gli artigli, li portò a lei, la quale gli prese, e
con essi attaccandosi all'erto monte, cominciò a salirvi.

\begin{description}
\item[TOPI che girano come pazzi] Sorci, che vanno in'qua e in la correndo senza
  saper dove determinatamente, appunto come fanno i pazzi.

\item[CAVARSI la stizza. Sfogar la collora, la rabbia, l'ira.

\item[NIBBIO] Uccello di rapina noto. Qui descrive il Grifone, e lo fa mezzo cavallo, e
  mezzo uccello, e con la coda di bue, e se bene da i pi e descritto mezzo lione,
  e mezzo uccello, e nimico mortale de' cavalli, come si deduce da Verg.
  Eg.8. \textit{Iungantur iam Gryphes Equis}, tuttavia non fa errore a comporlo di che bestie
  gli è piaciuto,  perché questo mostruoso animale in ogni maniera che sia è
  del tutto favoloso, secondo Plinio lib, 10. c.44. \textit{Pegasos} (dice egli) \textit{equino capite
    volucres, \& Gryphes aurita aduncitate rostri fabulosos reor, illos in Scythia, hos in
    AEthiopia}.

\item[INTERROGATIVO] È un contrassegno d'ortografia, il quale si pone in fine
  de' periodi, che conchiudono interrogare, o richiedere, e perciò è detto Punto
  interrogativo. E perché tal contrassegno è di figura simile a un'uncino, però a
  questo assomigliamo gli artigli degli uccelli, come fa qui il Poeta, assomigliandogli
  a quelli del grifone.

\item[LIBRO di maestro di liuto] Intendi libro da musica,, che son pieni di righe,
  affine di icrivervi sopra le note musicali.

\item[FALSARIGHE] Carte rigate, e lineate di nero, le quali si mettono sotto al
foglio, sopr'al quale si scrive, affine di far i versi diritti, ed uguali camminando
sopra quel segno, che dalla falsariga per trasparenza si vede sopra il foglio, ove
si scrive.

\item[LISTE] Qui vale per striscette di ferro, con le quali son composte le gratelle
  strumenti da cucina, che servon per mettervi sopra il pesce, o altro a quocere
  arrosto. E con tutte queste similitudini intende, che se l'uccelio havesse messo
  gli artighi addosso a Psiche, l'haverebbe malamente graffiata, e segnata.

\item[GRIFO] Vuol dir Faccia di porco, o simili; e s'intende alle volte: la faccia
  dell'huomo, ma per scherzo, o per disprezzo; e qui il Poeta se ne serve per far
  bisticcio di Grivo, e Grifone.

\item[BATTICUORE] Paura, timore. Da quella frequenza di battere, che fa il
  polmone dalla parte del cuore, quando si ha qualche spavento: I Latini pure dicevano
  \textit{animi, vel cordis percussio}.

\item[INSOLENTE] Arrogante, fastidioso, petulante. Uno che tratta, e procede fuori del dovere.

\item[GLI rende tre pani per coppia] Gli rende più del suo dovere, perché a render
  tre pani per i due, che è la coppia, si rende la meta più del dovere: E con questo
  modo di dire s'intende, che uno si difenda da un' altro con parole, e con fatti
  sempre con vantaggio, che diciamo anche \textit{render pane per focaccia}.

\item[NON si cura haver niente di suo] Intendi Non vuol'esser da lui superato.

\item[AFFERRARE] Abbrancare, pigliare stretto; \textit{Vi apprehensum detinere}.
\item[NODELLI] Intendi la congiuntura delle gambe co' piedi.
\item[ANDAR carponi] Camminar co' piedi, e con le mani per terra, ed è lo
  stesso, che \textit{Andar brancolone}, che si vede nel verso seguente; se non che questo
  vuol dir Salire adoperando le mani, e i piedi; e \textit{carponi} è camminare alla piana
  con le mani, e co' piedi, Dante Inf. C, 26. descrivendo una simil salita dice:
  \begin{verse}
    \backspace E proseguende la folinga via
    Tra le schegge, e tra i rocchi dello scoglio
    Il pié senza la man non si spedia.
  \end{verse}
\item[STRAMAZZONI] Intendi Cascate; che per altro ramazzone intendono gli
  schermitori una specie di taglio.
\end{description}

\section{Stanza LXVIII --- LXXI.}

\begin{ottave}
\flagverse{68}Tutti quei topi via ne vengon ratti,\\
E furon per mangiarmi dalla festa,\\
Però che dalle granfie io gli ho sottratti\\
Di quella bestia a lor tanto molesta;\\
Così vò rampicando come i gatti\\
Sull'aspro monte dietro alla lor pesta,\\
Sopportando fatiche, stenti, e guai,\\
E fame, e fete quanto si può mai.
\end{ottave}

\begin{ottave}
\flagverse{69}Pur finalmente in capo a due altri anni\\
Giungemmo al luogo tanto desiato;\\
Ma non finiron qui mica gli affanni,\\
Perché di muro il tutto è circondato;\\
E qui s'aggiugne ancor male a malanni,\\
Ch'io trovo l'uscio, ma'l trovo diacciato;\\
Pensa s'allor mi venne la rapina,\\
E s'io dicevo della Violina.
\end{ottave}

\begin{ottave}
\flagverse{70}Hora tu sentirai ch'il dare aiuto\\
A tutti quanti sempre si conviene,\\
Perché già mai quel tempo s'è perduto,\\
Che s'è impiegato in far' altrui del bene,\\
Non dico sol all'huom, ma anco a un bruto,\\
Che forse immondo, e inutile si tiene,\\
E che tu non lo stimi anche una chiosa,\\
Però che ognuno è buono a qualche cosa.\\
\end{ottave}

\begin{ottave}
\flagverse{71}Se tu giovi al compagno, allor tu fai\\
(Quasi gli presti roba) un capitale,\\
Anzi talor per poco, che gli dai\\
Ti rende più sei volte che non vale.\\
Ma non si dee ciò pretender mai,\\
Perch'ell è cosa, che starebbe male;\\
Questo è un censo il quale a chi lo prende\\
Richieder non si può s'ei non lo rende.
\end{ottave}

I topi, che Psiche liberò dagli artigli del Grifone la seguitarono facendole
gran festa, e con quella compagnia in capo a due altri anni arrivé Psiche al luogo
dove era Cupido, che era un recinto di mura, dentro al quale non si poteva
passare se non per una sola porta, e questa era serrata.

\begin{description}
\item[VENGONO ratti] Vengono velocemente dal Latino \textit{rapidus}, D. Infer. C. 21.
  \begin{verse}
    Perch'io mi mossi, ed a lui venni ratto
  \end{verse}
  Ed habbiamo rattezza, per prestezza, o velocità. Varch, Stor. lib. 4. \textit{In quel mezzo il
  sig.\ Sciarra Colonna partissi con gran rattezza da Roma}.

\item[FAR festa a uno] Rallegrarsi con uno. Ricevere, o trovar uno con atti di
  amorevolezza, e cortesia; Che nelle bestie si conosce tal rallegramento da i gesti,
  come nel cane dal dimenar della coda, ne i gatti dal fregarsi addosso a uno,
  ed altri animali dal moto degli orecchi, come forse si conosceva in quei topi. Il
  Lat. adulari fanno venire alcuni da \textit{ad, \& ura}, che in Greco significa coda quasi sia
  \textit{cauda adblandiri}.

\item[RAMPICANDO] Intendi salire appiccandosi con gli artigli del Grifone, come fanno i gatti.
  Viene da \textit{rampi} che s'intende ugne di gatto, lione, tigre, e simili.
  Si dice anche \textit{inerpicare} da erpico strumento rustico da romper le terre.
  Mattio Franzesi sopra alle maschere dice:
  \begin{verse}
    \backspace Non vi crediate, che qualunque saglie
    Havesse da un posta tanto ardire,
    Ch'inerpicasse sopra alle muraglie
  \end{verse}
  Ma oggi corrottamente si dice \textit{innarpicare}, e \textit{annarpicare}. Vedi sotto Can. 9. stan. 25. e 28.

\item[DIETRO alla lor pesta] seguitando le lor pedate.

\item[MICA] È una particella riempitiva in compagnia della negazione per emfasi
  del discorso, appunto come i Latini dicono ne quidem » se bene \& diferente dal
  Latino, perché non s' usera per affermativa, io voglio mica, come essi dicono ero
  quitem volo, sì che se bene e per emfafi ha però qualche parte del negativo, quasi
  diciamo: Io now voglio ne pur' una mica, che vuol dir minuzolo di pane, o granello
  di sale. IL Petr. Son..91. We mica trove il mio ardente defio.

\item[AFFANNI] \-  Dolori di cuore, che fanno quasi venire in angoscia, Petrar.\ son.~11.
  \begin{verse}
    Se la mia vita dal aspro tormento
    Si puo tanto schermire, e dagli affanni.
  \end{verse}

\item[AGGIUNGER male a malanni] Al male accrescer male, e peggio.

\item[USCIO diacciato] \- Cioè porta serrata. Vedi sopra C.~3.\ stan.~3.

\item[MI venne la rapina] Mi venne rabbia, collora, o stizza. Rapina vuol dire rubamento
  violento: quindi uccello di rapina; ma dalle nostre donne è presa in
  cambio di \textit{rabbia}, per sfuggir di dire \textit{rabbia} creduta parola peccaminosa, e dicono
  \textit{arrapinare}, \textit{arrapinato}, per \textit{arrabbiare}, ed \textit{arrabbiato}.

\item[DICEVO della violina] Dicevo del male fra me medesimo, perché le cose non
  andavano a mio modo. Questo so che significa \textit{Dir della violina}, non so già da
  che habbia origine questo dettato, che è lo stesso che \textit{Dir l'orazione della bertuccia}.

\item[NON lo stimi una Chiosa] Non lo stimi punto. Vedi sopra \cstan[3]{60}, alla voce \textit{iosa}.

\item[FAR un capitale] Metter insieme una somma considerabile di denaro per haverlo
  pronto a ogni suo bisogno: Si dice anche \textit{far un'assegnamento}.

\item[CENSO] La natura del censo, è che colui, il quale presta danari a censo, non
  può richieder la somma principale, che egli dà, ma solo i frutti d'essa; può ben
  colui che gli piglia render la medesima somma principale a ogni suo piacimento,
  e colui, che la diede è forzato a riceverla, come dice il Poeta assomigliando colui,
  che fa il piacere a un'altro, a uno che dia a censo, e dice, che colui che
  fa il piacere non dee, ne può pretender la ricompensa, ma la può bene sperare,
  e ne vive creditore: Che perciò ben dice Seneca \textit{de Beneficijs} lib.3.c.14, \textit{Vide etiam
    atque etiem cui des, nulla actio erit, nulla repetitio}, B \libcap[4]{39}. \textit{Alia conditio
    est in credito, alia in beneficio}.

\end{description}

\section{STANZA LXXII --- LXXV}

\begin{ottave}
\flagverse{72}Guarda s'ell'è così; Io per la mia\\
Pietà di prender di quei topi cura,\\
Da lor vinta restai di cortesia,\\
E n'hebbi la pariglia con l'usura,\\
Però ch'in questa zezza ricadia,\\
Ch'io ho d'haver trovata clausura,\\
Eglino tutti sul cancel saliro,\\
E si fermaro, ove è la toppa, in giro.
\end{ottave}

\begin{ottave}
\flagverse{73}E gli denti appiccando a quel legname, \\
Come s'in bocca havessero un trapano, \\
Presto presto vi fecero un forame \\
Da porre il fiasco, e vender il trebbiano,\\
Tal ch'in terra cascando ogni serrame\\
Spalanco l'uscio di mia propria mano,\\
E passo dentro, e resto pur confusa,\\
Perch' ancor quivi è un'altra porta chiusa.
\end{ottave}

\begin{ottave}
\flagverse{74}Ma parve giusto come bere un'uovo\\
A i topi il farvi il consueto foro,\\
E dopo questa a un'altra, e poi di nuovo\\
Infino a sette fanno quel lavoro;\\
Quando fra i verdi mirti io mi ritrovo,\\
Che fan corona a una cassa d'oro,\\
Ch'è a pié d'un Tempio, c'è dipinto a graffio\\
E a prima faccia tien quest'epitaffio.
\end{ottave}

\begin{ottave}
\flagverse{75}Cupido Amor, che tanti ha sbolzonato\\
Berzaglio qui si giace della morte,\\
Ei ch'era fuoco, il naso hora ha gelato,\\
Se i cuor legò, prigione è in queste porte.\\
Hallo trafitto, morto, e sotterrato\\
Quella Cicala della sua consorte,\\
Ne sorgerà, se pria colma di pianto\\
Non sarà l'urna, che gli è qui da canto.
\end{ottave}

I Topi suddetti rimunerarono Psiche, perché rodendo fino a sette porte, che
erano in quel Serraglio, fecero cascare i serrami, e Psiche entrata dentro, trovò il
sepolcro d'Amore, e dall'Inscrizione, che in esso era, comprese quello, che le
restava da fare.
\begin{description}
\item[HEBBI la pariglia] Hebbi il contraccambio. È il Latino \textit{Par pars referre}.
Pariglia intendiamo due cose uguali nel giuoco di Carte, o dadi, come due sei,
due assi, due figure, ec, e di tal voce non ci serviamo se non nel giuoco, o nel
caso del presente luogo di render contraccambio sì in bene, come in male. Vedi
sotto \cstan[6]{69}. Io l'ho per voce Spagnuola, ed il Varchi nella stor. lib. 8.
l'usò in un certo modo come straniera dicendo: \textit{Dopo essersi vendicati, ed haver
renduto il contraccambio, o, come si suol dire, la pariglia}.

\item[CON l'usura] Col frutto. Cioè mi contraccambiarono, facendo maggior servizio
  a me, che non havevo io fatto a loro.

\item[ZEZZA] Ultima. E' voce antica hoggi poco usata fuor che nel contado.
Vedi sopra \cstan[2]{2}, Si trova anche \textit{sezza}, \textit{sezzaia}, o \textit{zezzaia}.

\item[RICADIA] Noia, travaglio, avversità, molestia, o simili che vengono dopo
  a un'altro disgusto; da \textit{ricadia}, che è quando uno infermo già quasi sanato, viene
  a riammalarsi, o per lo mal governo, o per altro. Nella storia di Semifonte
  Trattato terzo. \textit{Con li loro misfatti, dando alli Fiorentini non poca ricadia}. Franc.
  Sac. Nov. 98. \textit{Che ricadia è questa di questi porci?}

\item[CANCELLO] Intende il legname, che chiude una porta: ma propriamente
  \textit{cancello} diciamo una chiusura di porta fatta di stecconi, o strisce di legno, o di
  ferro separate l'una dall'altra a guisa di gabbia.

\item[TOPPA] Intendiamo quella piastra di ferro, sopr'alla quale son fabbricati gl'ingegni
  della serratura, detta assolutamente, o senza aggiunta, perché per altro
  Toppa si dice ogni pezzo di panno, legno, quoio, ferro, ec. che s' adatti a
  rotture di cose di sua qualità, ec.

\item[TRAPANO] È uno strumento specie di fucchiello, col quale si forano materiali
  duri come pietre, e metalli, ec. Dal Greco \textit{Trypanon}.

\item[DA porre il fiasco] Coloro che vendono il vino a fiaschi, appiccano un fiasco
  sopr'alla porta della loro casa, come dicemmo sopra \cstan[1]{76}, ed oltre a
  questo hanno per lo più nella porta, o nel muro una finestrella, per la quale danno
  fuora il fiasco, che vendono; a questa finestrella assomiglia il foro fatto da i
  topi; e se bene dice \textit{da vendere il trebbiano} pigliando questa specie di vino per tutte
  le specie di vino, intende esser questo tale sfondato simile a quello, che si fa
  nelle porte per vendere il vino.

\item[SPALANCARE] Aprire largamente, quanto si può.

\item[PARVE come bere un'uovo] Fu cosa facilissima, come è il bere un'uovo: i Greci
  pure dissero in questo proposito \textit{Quo pacto quis ovum sorberet}, e trovasi questa
  frase presso Ateneo.

\item[DIPINTO a graffio] Dipingere a graffio, sgraffio, o graffito, è un'imprimer
figure, ec. con un ferro acuto nell'intonacatura fresca de' muri con detto ferro,
che si chiama graffio, forse dall'antico \textit{graphium}, che era lo stilo di ferro, col
quale scrivevano.

\item[BOLZONARE] o \textit{sbolzonare}. Sacttare, frecciare, da bolzone specie di freccia
  Mattio Franzesi sopra alla boria dice:
  \begin{verse}
    \backspace Di qui Amore accorto balestriere
    Bolzona qualche giovane galante
  \end{verse}

\item[HA il naso gelato] Ha il naso freddo. Pigliando la parte per il tutto, vuol dire,
  che Cupido è freddo, cioè morto.

\item[CICALA] Animale noto; ma qui si dice una, che chiacchierando assai, non
  può ne sa tener segreta cosa alcuna; e degli huomini diciamo \textit{Cicaloni}. Appresso
  i Greci \textit{cicala} non suona male, poiché alle cicale sono da essi rassomigliati in più
  d'un luogo i Poeti per il continovo cantare, che fanno, e questi, e quelle. E
  questo nostro Poeta graziosamente chiamò Musa la cicala sopra \cstan[1]{2}.
\end{description}

\section{Stanza LXXVI --- LXXX}}

\begin{ottave}
\flagverse{76}Non ti vuo dire adesso sin quel caso\\
Mi divennero gli occhi due fontane,\\
E feci come chi s'è rotto il naso,\\
Che versa il sangue, e corre al lavamane;\\
Cors'io a pianger a quel vaso\\
Durando a lagrimar sei settimane,\\
E, per haver quel più voglia di piagnere,\\
Mi diedi pugna sì ch'io m'ebbi a infragnere.
\end{ottave}

\begin{ottave}
\flagverse{77}Quand io veddi ch' egli era poco meno\\
In su ch'all'orlo, ed esser a buon porto,\\
Volli innanzi ch'e fusse affatto pieno,\\
E ch'il marito mio fusse risorto.\\
Lavarmi il viso, e rassettarmi il seno,\\
Acciò sì lorda non m'havesse scorto;\\
Perciò mi parto, e corro, se in quel monte\\
Per avventura fusse qualche fonte.
\end{ottave}

\begin{ottave}
\flagverse{78}In quel ch'io m'allontano com'io dico,\\
Martinazza, che era in Stregheria,\\
Passò di là portata dal nimico,\\
Che non porette star per altra via;\\
E perché sempre fu suo modo antico\\
Di far pertutto a alcun qualche angheria;\\
Lesse il pitaffio, squadro l'urna, e tenne,\\
Che lì fusse da farne una solenne,
\end{ottave}

\begin{ottave}
\flagverse{79}Se qua, dice fra se, Cupido dorme,\\
Vuo risvegliarlo per veder un tratto\\
S'egli è come si dice, e se conforme\\
A quel che dai Pittori vien ritratto\\
Se ben chi lo fa bello, e chi deforme,\\
Basta mi chiarirò com'egli e fatto;\\
Per questo ad empier mettesi quel vaso,\\
A cui poco mancava ad esser raso.
\end{ottave}

\begin{ottave}
\flagverse{80}Con l'animo di pianger vi s'arreca, \\
Ma ponza ponza, lagrima non getta,\\
Si prova a far cipiglio, e bocca bieca, \\
Ne men queta è però buona ricetta; \\
Al fin si pone a un fumo, che l'accieca\\
Sì che per forza a pianger è costretta,\\
Onde la pila in mezzo quarto d'ora\\
Restò colma, e Cupido scappò fuora.
\end{ottave}

In ordine al Cartello havendo Psiche con le sue lagrime quasi piena l'urna,
andò a lavarsi il viso, e raccomodarsi la testa; Intanto Martinazza arrivò al
sepolcro, e con le lagrime sue finì d'empier l'urna, e Cupido usci dal Sepolcro.

\begin{description}
\item[NON ti vo dire] Questo termine serve per esprimere. \textit{Da te puoi ben sapere
  questa cosa meglio di quello che io sapessi dirti}; o vero \textit{so che tu hai da per te tanto spirito da
  giudicar come io rimanessi, senza che io te lo dica}, Suona lo stesso che \textit{pensa tu}, \textit{giudica tu},
  \textit{dica tu}, \textit{tu puoi sapere}, ec. Vedi sopra in \cstan{41}. stan. 52, e stan. 59.
  Simile è quello: Non domandar, se Durlindana taglia.

\item[LAVAMANE] È uno strumento di legno, o d' altro, che con tre piedi forma
  come una piramide in triangolo equilatere, e sopra esso si pola la catinella, o
  altro vaso per lavarsi le mani.

\item[ERA poco meno che all'orlo] Era quasi pieno. L' acqua arrivava quasi all'estremità del vaso:
  che questo vuol dire \textit{orlo}, che viene dal latino \textit{ora}, che significa
  l'estremità di qualsivoglia cosa.

\item[LORDO] Schifo, intriso. Dal latino \textit{Luridus}.

\item[VA in stregheria] Dicemmmo sopra C.2. stan. 11, donde derivi tal nome di Strega,
  ed al \cstan[3]{69}, dicemmo esser fama, che tali Streghe vadano la notte a
  cavallo in sul caprone a Benevento al congresso de' diavoli. E questo: intende dicendo
  \textit{Andare in Stregheria portata dal nimico}, che vuol dire il Demonio, in forma
  di Caprone. Che queste donnicciuolucce credute Streghe vadano in sul Caprone
  a Benevento è opinione vulgata, e molti di cervello debole l'hanno per
  indubitata, e le medesime Streghe se lo credono, perché il Diavolo con illusioni
  fa loro apparir per vera questa falsità; Ma la graziosa sagacita d'un Superiore
  ne fece chiarire tutti i dubbj in questa forma.

  Fu condotta alle carceri una di queste tali inquisita di maliarda, ed il Giudice
  dopo molte esame havendo trovato, che veramente costei era una donna, che si
  credeva far malie, stregar bambini, ed altre scioccherie, ma in effetto non v'era
  cosa di conclusione, o di proposito, risolvette di gastigarla per la mala intenzione,
  ed in tanto soddisfare alla propria curiosità. Fattala però venire a sé l'interrogò
  se andava ancor' ella a Benevento, rispose che sì, onde egli le disse: Io vi
  voglio perdonare se voi andrete questa notte a Benevento, e domattina mi racconterete
  quanto vi sarà successo. Bisogna che mi diate la libertà, replicò la donna,
  acciò io possa nella mia stanza fare i miei scongiuri, e le mie unzioni; il
  Giudice gliela concedette con questo che voleva dargli da cena insieme con un
  compagno: il che accettò la donna, bastandole esser fuori di quel luogo, dove il
  Diavolo non poteva capitare. Andata dunque a casa cenò con il detto compagno,
  che era un giovanotto ortolano, e con un'altro giovane, che la donna si
  contentò che egli conducesse, e bevuto abbondantemente come era il suo costume
  in tali sere di viaggio, lasciati i commensali a tavola sen' entrò nella solita
  camera, e quivi spogliatasi, senza serrar la porta, ne le finestre della medesima
  camera (che tale è l'ordine del Diavolo) s'unse con più forte di bitumi puzzolenti,
  e postati a diacere in sul letto, subito s'addormentò; I due compagni, così
  instruiti, entrarono in camera, e legarono la donna per le braccia, e gambe alle
  quattro cantonate del letto, e benissimo la strinsero con funi, e si messero a chiamarla
  con altissime voci, ma come fusse morta non faceva moto, ne dava segno
  alcuno di sentire, onde i detti cominciarono a martirizzarla bruciandole hora
  una poppa, hora una coscia, e finalmente così l'impiagarono in diverse parti del
  corpo, e le arsero fino alla cotenna la metà della chioma; Cominciando a venire
  il giorno la donna con sospiri, e lamenti diede segno di svegliarsi, onde i detti
  le sciolsero i legami, ed uno di loro andò per una seggetta, e l'altro la rivestì
  tutta sbalordita e dal sonno, e molto più da i martorj; giunta la seggetta, in essa
  la portarono al Giudice, il quale l'interrogò se era stata a Benevento, ed ella
  rispose che sì, ma che haveva patito gran travagli, ed era stata bastonata con
  verghe di ferro infuocate, e strascinata, e legata per le braccia, e per le gambe,
  era stata riportata dal suo Caprone, che nel lasciarla le haveva abbruciate con la
  granata mezze le trecce, e questo perché ella haveva ubbidito al Giudice, e che
  si sentiva morire dal gran dolore delle piaghe. Il Giudice ordinò, che subito fusse
  medicata, come seguì; ed intanto disse alla donna: Io v' ho fatto scottare, e
  battere per gastigo del tuo errore, e perché tu conosca, che non altrimenti a
  Benevento, ma in casa tua hai ricevuto questi travagli, e ti risolva a lasciar queste
  false credenze; che se lo farai, io ti perdonerò. Da questo bel modo di gastigare
  cavò l'arguto Giudice quella verità, che appresso lui era certissima.

\item[NON potette star per altra via] Non potette essere in altra maniera, perché
  Martinazza non havrebbe mai potuto salire su quel monte; se non ve l'havesse
  portata il Diavolo.

\item[ANGERIA] Violenza, dispiacere, sopruso. Viene dal Latino greco \textit{Angaria},
  che suona \textit{coactio}. Varchi Stor. Fior. lib. 2. \textit{E perché i Fiorentini nuovi tributi,
    ed angherie ritrovare havevano}.

\item[SQUADRÒ] Guardò diligentemente, ed accuratamente. Vedi sopra \cstan[1]{32}.

\item[FARNE una solenne] Fare un'angheria delle maggiori, che si possano fare.
  La voce \textit{solenne} è da noi spesso usata in vece di grandissimo, ed è tolta da i riti
  della Chiesa, che si dicono feste solenni, le maggiori feste, che seguono nell'Anno.
  Così \textit{hieros}, cioè sagro, presso i Greci, e \textit{sacer} presso i Latini vale talvolta
  grandissimo, \textit{Anchora sacra}, \textit{Morbus sacer}, è lo stesso, che \textit{Anchora maior},
  \textit{Morbus maior}. E Virgilio quando disse; \textit{Auri sacra fames}, per avventura intese
  grandissima.

\item[VIEN ritratto] Vien dipinto. Se il dipinto è come il vero. Dice: \textit{chi lo fa
  bello, e chi deforme}, per intendere, che i pittori da pochi soldi lo dipingono male.

\item[AD esser raso] Ad esser pieno affatto. Viene dal misurare il grano con lo
  staio, che per dare, e ricevere il dovere s'empie lo staio, e quando è pieno si
  striscia sopra con un bastone, e si fa cascare quel grano, che è sopr'alla bocca
  dello staio, e questo si dice \textit{radere}, e tal bastone si dice \textit{rasiera}, e lo staio così pieno
  si dice \textit{raso}, cioè pieno per appunto fino all'orlo della bocca.

\item[VI s'arreca] Vi s'accomoda con positura del corpo; sopra in questo C. stan.
  42, s'arrecò con l'animo.

\item[PONZA ponza] Ponzare è una forza che si fa in se medesimo, ritenendo il
  fiato, quasi riducendo tutto lo sforzo in un punto, come fanno le donne, quando
  mandano fuora il parto. Questo \textit{ponzare} è corrotto dal buon Toscano,
  \textit{pontare}, come si vede dal Petrarca, che dice:
  \begin{verse}
    Io riconobbi a guisa huom che ponta
  \end{verse}
  L'Espositore dice \textit{idest che spinga}. Vedi l'Alunno fabr, num, 609. la voce \textit{pontare}.
  Ed il termine \textit{ponza ponza} serve per esprimere uno, che assai lavorando, conchiuda
  poco; che si dice anche \textit{tresca tresca}. \textit{Ticche ticche}, \textit{Ienneinne}, che vedremo
  sotto \cstan[5]{51}, \textit{In vanum laborare}. Se bene qui si può intendere, che Martinazza
  moltissimo ponzasse.
\item[CIPIGLIO] È uno increspamento della fronte fatto in giù alla volta degli
  occhi, ed è una guardatura d'uno adirato, o d'uno estremamente superbo, quasi
  \textit{piglio del ciglio}. Gli antichi, come Dante dissero \textit{Piglio} la guardatura.

\item[BOCCA bieca] Bocca storta. La voce \textit{bieco} Latino \textit{obliquus}, è usata assai da i
  Legnaioli per intendere l'inegualità d'un legno, e dicono \textit{sbiecare} quando lo
  pareggiano, e fanno uguale.

\item[PILA] È proprio quel sodo, sopra il quale posano gli archi de i ponti. Ma
  si piglia anche per quel vaso grande di pietra, nel quale si mette l'acqua per abbeverare
  le bestie, o per altro uso simile; in somma per pila intendiamo ogni vaso
  di pietra che tenga, o riceva acqua.
\end{description}

\section{STANZA LXXXI, STANZA LXXXII}

\begin{ottave}
\flagverse{81}Quand'ella verso lui volte le ciglia,\\
E vedde quella sua bella, figura \\
Disposta, e graziosa a meraviglia, \\
Che più non si può far n' una pittura, \\
Gli s'avventa di subito, e lo piglia, \\
E, senza ricercar della cattura, \\
Da' suoi staffieri tenebrosi, e bui\\
Portar se ne fa via con esso lui.
\end{ottave}

\begin{ottave}
\flagverse{82}Fermossi a Malmantile, e per marito\\
Lo volle, e già le nozze han celebrate.\\
Come sai tu (dirai) tutto il seguito?\\
Lo so, che me lo dissero le Fate,\\
Quelle, che mi donar quel che hai sentito\\
Ch'in due Aquile essendo trasformate,\\
Perché lassù facea degli sbavigli,\\
M'han trasportata qua ne i loro artigli.
\end{ottave}

Martinazza porta via Cupido, ed in Malmantile lo piglia per marito; Così
havevano raccontato a Psiche le Fate, le quali trasformate in due Aquile l'havevano
portata via da quel monte co' loro artigli. E qui finisce il quarto Cantare.

\begin{description}
\item[CATTURA] Si dice quella somma di danaro, che si dà a i birri quand'hanno
  pigliato uno; e si dice anche cattura quella polizza, e ordine che si dà alli
  sbirri perché piglino uno. Di qui il Poeta cava lo scherzo dicendo, che Martinazza
  piglio Cupido senz'haver l'ordine della cattura, e lo portò via, e non aspettò,
  che le fusse dato il denaro della cattura, che havea fatta di lui.

  \item[FACEA degli sbavigli] Si dovrebbe dire \textit{sbadigi}. Dan. Inf. C. 45.
  \begin{verse}
    Anzi co' pié fermati sbadigliava
    Pur come sonno, o febbre l'assalisse
  \end{verse}
  Ma hoggi si dice \textit{sbavigli}, e \textit{sbavigliare}; che un'aprimento di bocca, ripigliando
  il fiato, e poi mandandolo fuora, il che per lo più è cagionato dal sonno, da
  pensieri, da tristizia o malinconia, o da altro rincrescimento, perché lo sbaviglio
  nasce da vapori grossi, e frigidi generati nello stomaco da ozio, e da pigrizia,
  i quali salgono alla bocca per la via del cibo, e spargonfi per le mascella, e
  la natura bramosa di mandargli fuora, alita con aperta bocca, il che da i Latini
  si dice \textit{oscitare}. \textit{Fare degli sbavigli}. Significa non haver roba da mangiare, ne
  altro da recrearsi al bisogno, ed habbiamo una rima, che dice:
  \begin{verse}
    \backspace Chi sbaviglia non può mentire
    O egli ha sete, o egli ha fame, o e' vuol dormire.
  \end{verse}
  Sicché la povera Psiche stando in quel luogo, dove non era da mangiare, ne da
  bere, haveva occasione di sbavigliare non potendo cavarsi la fame, ne la sete.

\item[ARTIGLI] Dal Latino \textit{articuli}. Zampe degli uccelli, o altri animali ditati.
  Qui intende le mani delle Fate, le quali convertite in Aquile, havevano artigli in
  vece di mani. Se bene diciamo talvolta artigli le mani dell'huomo. Bocc. Canz.
  alla Nov. 6.
  \begin{verse}
    \backspace Amor, s'io posse uscir de' tuoi artigli,
    A pena creder posso,
    Che alcun altro uncin mai più mi pigli.
  \end{verse}
\end{description}
\section*{FINE DEL QVARTO CANTARE.}

\chapter{Quinto Cantare}

\begin{argomento}
  Vuol con gl'incanti dar la Maga aita
  In Malmantile al popolo assediato,
  Ma dagli spirti è così mal servita,
  Che tra i nimici e il suo saper beffato;
  Vien Calagrillo, e a duellar l'invita,
  E l'invito è da lei tosto accettato.
  Il Fendesi, e altri due com'è usanza,
  Sparir di Piaccianteo fan la pietanza.
\end{argomento}

\section{STANZA I \& II}

\begin{ottave}
\flagverse{1}E si trova talun, ch'è sì capone,\\
Ch'ad una cosa, che si tocca, e vede,\\
E che di più gli afferman le persone,\\
Vuol' esser ostinato, e non la crede.\\
Un'altro è poi sì tondo, e sì minchione,\\
Che se le beve tutte, e a ognun dà fede;\\
E ci son' huomin tanto babbuassi,\\
Che crederebbon, c'un'asin volassi.
\end{ottave}

\begin{ottave}
\flagverse{2}Gli estremi non fur mai degni di lode:\\
Ci vuol la via di mezzo, e chi ha cervello,\\
Se vere, o false novitadi egli ode\\
A crederle al compagno va bel bello:\\
Le crede, s'elle son fondate, e sode,\\
Ma s'elle star non possone a martello,\\
Non le gabella mica di leggieri,\\
Come fa il Duca a certi messaggieri.
\end{ottave}

Volendo il Poeta nel presente Cantare narrar l'inavvertenza de' due Diavoli
mandati da Martinazza per far diloggiar Baldone, e lo scambiamento delle palle,
per lo quale apparvero a Baldone diversamente da quello, che dovevano, il
che fu causa, che egli non prestò fede alle loro parole, s'introduce col dire:
Che l'esser' huomo testardo, e capone non è bene, ma che non è però anche
bene l'esser così credulo, che si dia fede a tutto quello, che si sente dire, onde è
degno di lode colui, che sa pigliare la via del mezzo, dando credito a quelle cose,
le quali egli conosce haver fondamento di verità, come fece Baldone alli due
messaggieri di Martinazza.

\begin{description}
\item[CAPONE] Testardo. Huomo ostinato nella sua opinione. In Latino pure
  potrebbersi chiamare questi cali \textit{Capitones}; da noi altrimenti \textit{Caparbi}.
\item[TONDO] Huomo groffolano; semplice, facile, credulo, ec, Epiteto, che si
  dà a i panni lani, che si dice \textit{tondi}, quando sono grossi, contrario i \textit{fini}. E così
  diciamo \textit{huomo fine}, che è il contrario d'\textit{huomo tondo}, Lasca Novella 2. \textit{Mariotto
  fu huomo di sì grossa pasta, e così tondo di pelo, che in quattr'anni di squola non
  potette mai imparare l'Abbiccì}. Vedi sotto \cstan[6]{80}.
\item[MINCHIONE] Semplice. Vedi sopra \cstan[4]{15}.
\item[SE le beve tutte] Crede tutto quello, ch'ei sente dir.
\item[BABBVASSI] Ignoranti, huomini di cervello grosso. Vedi sotto C.6. st. 80.
\item[CREDEREBBON ch'un asin volasse] Per esprimer'uno, che crederebbe etiam
  le cose impossibili a credersi, ci serviamo di questo detto. In Empoli in un dì
  solenne dell'anno, fanno una antica festa, o rappresentazione di far volare l'Asino:
  Quindi è, che nel Capitolo in lode dell'Asino, che va colle rime del Berni, si
  dice:
  \begin{verse}
    \backspace Ben mostran gli Empolesi aver cervello,
    Quanto conviensi ad ogn' huomo da bene;
    Che l'Asin diventar fanno un uccello.
  \end{verse}
\item[NON può stare a martello] Non corrisponde al vero. Tratto dat Cimento
  dell'argento, che quando non stà, cioè non resiste al Martello, non è vero argento.
  I Latini pure direbbero in questo proposito \textit{non est aurum igni probatum}.
\item[NON le gabella] Non le passa per vere. Non le crede: dal \textit{Passaggio}, ovvero
  \textit{Gabella} delle porte, o de' passi; onde il verbo \textit{Gabellare}, per ammettere, e approvare
  una cosa per buona, e per vera. \textit{Mica} particella riempitiva a maggior
  enfafi della negativa, come già, e mai, ec, \textit{Io non vuò mai, che si dica}. \textit{Io non vuò già,
  che si dica}, \textit{Io non vuò mica, che si dica}. Vedi sopra \cstan[4]{69}.
\end{description}

\section{STANZA III --- VII.}

\begin{ottave}
\flagverse{3}Ma, perché chi m'ascolta intenda bene; \\
Tornar' a Martinazza mi bisogna, \\
La qual dianzi lasciai, se vi sovviene, \\
Ch'in sul Caprinfernal, pigra carogna, \\
Quel popolaccio ha aggiunto, e lo ritiene \\
Dal fuggir via con tanta sua vergogna, \\
Perché quando per lei la raffigura,\\
Rallenta il corso, e piscia la paura.
\end{ottave}

\begin{ottave}
\flagverse{4}E quivi con l'affanno in sulla pena\\
Tutto lamenti, con doglienze, e strida,\\
Tremando forte, come una vermena,\\
La prega, perch'in lei molto confida,\\
E perch'addosso giunta gli è la piena,\\
E lì tra lor non è capo, ne guida,\\
A far in mo, se si può far di manco,\\
Ch'ei non s'abbia a cacciar la spada al fianco.
\end{ottave}

\begin{ottave}
\flagverse{5}Ella risponde allor, ch'è di parere,\\
Ch'il pigliar l'arme faccia di mestiero,\\
Che per la Patria par che sia dovere\\
Il farsi bravo, e diventar guerriero,\\
Se ben fra tanto vuol un po vedere\\
S'ella con Gambastorta, e Baconero\\
Trovar potesse il modo, che costoro\\
Vadan a far il bravo a casa loro.
\end{ottave}

\begin{ottave}
\flagverse{6}Ciò dette balxain casa, € cold dentro
Per aenerfi dispogliafi in capelti,
E cacciatasi addosso quant' unguento
Haveva ne! [uci feridi alberelli,
Vin gran circolo fa nel pavimento,
E con un uafo in man,scritei, e Cartelli,
Borbortando parole tutravia,
Che ne men si direbbono in Turchia,
\end{ottave}

\begin{ottave}
\flagverse{7}Fa un salto a pié pari in mezzo al segno, \\
E quivi havendo all ordine ogni cosa,\\
Per mandar ad effetto il suo disegno \\
Grida così con voce strepitosa:\\
O colaggiù dal sotteraneo Regno\\
Cornuti mostri, e gente spaventosa,\\
Filigginosi abitator di Dite,\\
Badate a me; le mie parole udite.
\end{ottave}

Torna adesso a Martinazza, la quale sopra nel C.~3.\ stan.~76, lasciò, che montata
a cavalcioni in sul Caprone, haveva arrivato quel popolo, che fuggiva per
la paura, ma riconosciutala, la prega a dar' aiuto a Malmantile, e far, che essi
non habbiano a combatter, se si può. Ella dice, che stima necessario il combattere,
ma che intanto vuol vedere, se gli riesce cacciar via il nimico per altre,
strade, e vassene in casa a fare i suoi incantesimi a questo effetto.

\begin{description}
\item[CAPRINFERNALE] Due dizioni come ridotte in una significante Caprone d'Inferno:
Ed intende quel Diavolo in forma di Capra sopr' al quale era cavalcata
Martinazza, e sopra il quale si favoleggia, che vadano le Streghe a Benevento,
come s'è notato sopra C.~3. stan. 69.

\item[CAROGNA] Vuol dire Cadavero d'huomo, o di bestia. Cavalcanti stor.
fior. \libcap[3]{2}. dice: \textit{Se volete veder quanto la lor perfidia si distese contro al sangue
de' nostri maggiori, cercate i Conventi de' Frati, e troveretegli pieni di corpora, e di
carogne de i vostri antichi}. Da questo dire del Cavalcanti m'induco a credere, che
la voce \textit{Carogna} significhi cadavero d'huomo ammazzato con ferite, e straziato,
e che però ci serviamo di tal voce per intendere una bestia piena di mascalcie, e
guidaleschi, e stimo con Pier Vettori nelle Varie lezioni, che venga da \textit{Charonia},
che intendevano già le voragini del fuoco, che in diverse parti del mondo si trovano,
e le dicevano \textit{Charonia} da Caronte, perché la superstiziosa Gentilità stimava,
che tali vorapini fussero bocche d'Inferno, e che per quelle s'andasse da
Caronte; E perché hanno sempre puzzo orrendo, che procede da acque sulfuree,
da questo cominciarono a chiamare \textit{Charonia} tutte quelle cose, che grandemente
putivano; E noi seguitando gli antichi diciamo \textit{Carogna} a tutte le cose,
che putono, come fanno le bestiaccie guidalescose, e le morte. Dicigmo \textit{Carogna}
anche un'huomo, che habbia cattivi sentimenti, perché un'azione mal fatta si
suol dire \textit{Questa pute; o non ha buono odore}.

Gli Ateniesi chiamavano \textit{Charonia} quella porta del Pretorio, o Palagio del Potestà,
per la quale uscivano coloro, che erano condotti al supplizio, secondo
che riferisce Giulio Polluce nell'Onomastico, e Alex.\ ab Alex.\ lib.4.\ c.16.\ e Cel.\
Rod.\ lect.\ antiq.\ lib.4.\ c.8, e lib.17.\ c.9. Tolta la derivazione di tal voce pure
da Caronte, che conduce l'anime al supplizio, passandole in barca, e si dice \textit{mandar'
uno a Caronte} per intender mandar uno alla morte.

\item[PISCIA la paura] Ripiglia animo. Non ha più paura. Dopo che i cani si
sono azzuffati sogliono pisciare; e comunemente dalla plebe si dice che pisciano
la paura; e da questo diciamo \textit{pisciar la paura}, quand'uno spaventato, o impaurito,
perde quel timore.

\item[L'AFFANNO in sulla pena] Era aggiunto alla pena, che hebbe per la paura
  l'affanno cagionato dal correre. Vedi la voce Affanno sopra C.4. stan.\ 69.

\item[VERMENA] Un sottile, e giovane ramo d'una pianta si dice Vermena dal
  Latino \textit{Vimen}. Quel passo di Vegezio; de re militari \libcap[1]{11}. \textit{Quemadmodum
    ad scuta viminea, vel ad palos antiqui exercebant tyrones}: L'antico volgarizzatore
  traduce così. \textit{Come a scudi fatti di vermene, o pali, si provavano i Cavalieri}.

\item[GLI giunta addosso la piena] Sono accadute loro tutte le maggiori disgrazie, e
  piena è presa nel senso detto sopra \cstan[1]{84}.

\item[A FAR in mo che non s'habbia a metter la spada al fianco] Far in modo che il
  negozio s'aggiusti, senz'havere ad adoprare le armi, che si dice \textit{Aggiustarla
    la spada nel fodero}.

\item[SE si può far di manco] Se la necessità non forzi a fare in questa maniera.

\item[GAMBASTORTA, e Baconero] Nomi di Diavoli inventati qui dal  Poeta,
  nello stesso modo, che inventati furono i nomi di \textit{Barbariccia},
  e \textit{Farfarello}, e simili.

\item[BALZA in casa] Va velocemente in casa. \textit{Balzare} propriamente si dice quel
  saltare, che fa la palla, o pallone perquotendo in terra, Vedi sopra C.2. stan.15.

\item[SPOGLiASI in capelli] Si spoglia ignuda, e scioglie le trecce de i capelli, così
  vuol intender il Poeta, se bene si serve del detto \textit{spogliarsi in capelli}, che significa
  adoperare ogni suo sapere, e tutta l'applicazione per fare una tal cosa; per intendere
  ancora, che Martinazza s'era tutta applicata a far, che Baldone per
  via d'incanto diloggi da Malmantile,

\item[CACCIANDOSI addosso] Mettendosi addosso, E se bene il verbo \textit{cacciare} vuol
  dir intromettere con violenza, noi lo pigliamo in senso di mettere, come si vede
  nell'Ottava antecedente \textit{cacciar la spada} per metter la spada.

\item[ALBERELLI] Vasi di terra, o di vetro, entro a' quali si conservano
  Unguenti, e cose simili; e son forse quei vasi, che i Latini chiamano \textit{alveoli}, e
  pigliano il nome da questi.

\item[BORBOTTARE] È un certo parlar fra i denti poco inteso da chi l'ascolta,
  che diciamo anche \textit{brontolare}, E' il Latino \textit{submurmurare}. \textit{Borboryttein} appresso i
  Greci è il \textit{romoreggiare}, o \textit{mormorare che fanno le budella}: Verbi formati dal suono
  stesso naturale.

\item[A PIÉ pari] Cioè a piedi giunti insieme. Questa voce pari, che per altro
  vuol dire \textit{ugualità di numero}, ed il suo contrario è \textit{dispari} (che diciamo \textit{caffo}) che
  i Latini dicono \textit{par, \& impar}, serve ancora per denotare ugualità di misura di un
  corpo, come qui, che s'intende, che un piede non era ne più innanzi, ne più
  indietro dell'altro. Si dice \textit{esser pari} quando uno s'è vendicato con un'altro, o
  ha pagato tutto quello che doveva, E ancora : esser pari e pagati. \textit{Andar pari},
  quando non si pende per nessun verso. \textit{Strada pari} per strada spianata. In somma
  l'adopriamo in tutte quelle cose, dove entri ugualità.

\item[FILIGGINOSI] Affumicati. Tinti da fumo, come sono i cammini, che son
  neri per la filiggine, che è composta di fumo, e d'umido. Lat. \textit{fuliginosi}.

\item[BADATE a me] Attendete a me; Osservate le mie parole, e state attenti a quel c'io dico.
\end{description}
\section{Stanza VIII \& IX}
\begin{ottave}
\flagverse{8}Vi prego, vi scongiuro, e vi comando\\
Per la forza, e virtù ai questi incanti,\\
Per quest'acqua, che a gocce in terra spando\\
Dagli occhi distillata degli amanti,\\
per questa carta ov'è stampato il bando\\
Di questa porcheria de' guardanfanti,\\
Che di portar le donne han per costume,\\
Ricettacol di pulci, e sudiciume.
\end{ottave}

\begin{ottave}
\flagverse{9}Per gl'imbrogli vi chiamo, e l'invenzioni,\\
Che ritrova il Legista, ed il Notaio,\\
Quando per pelar meglio i buon pippioni\\
Gli aggira, che ne anco un'arcolaio;\\
Horsu, pezzi di sacchi di carboni,\\
Per quei ladri del Sarto, e del Mugnaio,\\
Che ti voglion rubar a tuo dispetto,\\
Uscite fuor, venite al mio cospetto.
\end{ottave}

Martinazza con diversi scongiuri chiama gli Spiriti infernali, per servirsene
a far diloggiar Baldone da Malmantile: E l'Autore mostra il disprezzo, che egli
fa degl'incantefimi, facendo che Martinazza costringa i demonj con le cose ridicole,
che egli mette in queste due Ottave.

\begin{description}
\item[SCONGIVRARE] Questo verbo t da noi usato per inteddere Esorcizzare, cioè
  costringere il Diavolo per via di giuramenti di formule sacre dette per questo
  Esorcismi, cioè scongiuri; e comunemente è preso in questo senso, ed anche più
  largamente si tira, come qui, alla maniera d'invocare gli spiriti, usata da' Maghi,
  se bene il suo proprio significato è domandare, o chiedere con grande ardenza,
  ed è in augumento del verbo pregare dicendosi. vi prego, vi supplico, vi scongiuro.
  Latino \textit{obsecro}, \textit{obtestor}.

\item[PORCHERIA] Si dice non solamente un'atto sporco, ed illecito, ma ancora
  una materia schifa, sporca, e brutta, o mal fatta. Come per esempio: \textit{Il tale
  fece un'orazione, che riuscì una bella porcheria}, \textit{La vostra mercanzia non hebbe esito,
    perché fu stimata una porcheria}: \textit{I Libri di quel Mercante furono abbruciati, perché
  eran pieni di partite false, e d'altre porcherie}. Varchi nelle storie Fiorentine dice:
  \textit{Era appunto sparsa in Firenze l'usanza d'andar in zazzera, e mantello, che era una
    bella porcheria}. « Questa voce \textit{Porcheria} significante disprezzo potrebbe venire dal
  Latino \textit{porcaria}, che vuol dir l'utero delle Vacche, o delle Troie, dopo che hanno
  partorito, o per dirla colle parole di Plinio \libcap[11]{37}. \textit{Vulvam partu edito},
  e tali vulve, particolarmente quando non avevano condotto il parto, ma si erano
  sconciate, dagli antichi Romani erano mangiate per una cosa singulare, dove
  la \textit{Porcaria} non la mangiavano tanto voientieri, forse per esser cosa più schifa. Era
  chiamata porcaria in un certo modo per disprezzo, e così ha portato a
  nol il ignificato, che ritiene di disprezzo, ed abbominazione. Ma la più semplice
  origine è da porco animale immondo; e così detta \textit{porcheria}, cosa da porci,
  come furfanteria, cosa da furfanti, e simili.

\item[GVARDANFANTE] È uno strumento composto di cerchi di filo di ferro in
  tondo; il quale portano le donne Spagnuole, e circonda loro la cintura sotto le
  vesti, le quali fa gonfiare: E lo dicono \textit{guardanfante}, perché egli difende dalle
  percosse l'infante, cioè la creatura, che hanno le donne pregne dentro all'utero.
  Perché questa foggia di vestire, che havevano cominciata ad usare le donne di
  Firenze, conosciuta presto per spropositatamente dispendiosa, e scomoda, s'andava
  a poco a poco disusando, il Poeta in questo Incantesimo di Martinazza pone
  il Bando, cioè l'esilio, e proibizione di tale usanza.

\item[PIPPIONI] Piccioni. S'intende gente semplice, e corriva, come appunto
  sono i pippioni, \textit{colambarum pulli}, colombi giovani. \textit{E pelare un pippione} vuol dire
  Cavar danari di mano al corrivo.

\item[ARCOLAIO] Strumento, sopr' al quale s'adattano le matasse, d'accia o
  d'altra materia per incannarle, o aggomitolarle col girare, il che è assai veloce,
  ed è un moto perpetuo, e però dice \textit{aggira che ne anche un'arcolaio} intendendo aggira
  bene, ed assai:  ed aggirare in questo luogo vuol dire ingannare; donde \textit{aggiratore},
  ingannatore, Così \textit{Bindolo}, si prende per huomo aggiratore; e \textit{Abbindolare}
  per girare, cioè non si rinvenire col cervello. L. \textit{delirare}, o pure per Aggirare,
  Ingannare; Latino \textit{circumvenire}.
\end{description}
\section{Stanza X. --- XII.}

\begin{ottave}
\flagverse{10}Tutto l'Inferno a così gran parole\\
Vien sibilando, e intorno le saltella,\\
Come dall'alba al tramontar del sole,\\
Fa quel, ch'è morfo dalla Tarantella,\\
Domandale Pluton quel ch'ella vuole,\\
Che stridendo ogni dì lo dicervella,\\
E lui, c'hor mai ha dato nelle vecchie,\\
Fa ire in giù, e in su come le secchie.
\end{ottave}

\begin{ottave}
\flagverse{11}Ed a far ch'ei si pigli quella stracca \\
Senza cagion, gli par ch'ell'abbia il torto, \\
Perché dalla profonda sua baracca \\
A Malmantil non è la via dell'orto. \\
Corpo! (dic'ella, ed al Celon l'attacca) \\
A venir infin qui tu sarai morto!\\
Ma senti il mio Pluton, non t'adirare, \\
Che venir non t' ho fatto sine quare, \\
\end{ottave}

\begin{ottave}
\flagverse{12}Ma perché tu mi voglia far piacere\\
Di darmi Baconero, e Gambastorta,\\
Perch'io mi vuò dell'opra lor valere\\
In cosa che mi preme, e che m'importa.\\
Plutone allor quei due fa rimanere,\\
E la strada si piglia della porta,\\
Seguito da i sugi sudditi, che tutti\\
Posson fondar la Compagnia de' brutti.
\end{ottave}

Agli scongiuri di Martinazza le comparisce avanti Plutone con molti Diavoli,
ed ella gli chiede Baconero, e Gambastorta. Ei le lascia quivi li detti due demoni,
e con gli altri se ne torna all'Inferno.

\begin{description}
\item[SIBILANDO] Soffiando, fischiando. E' voce Latina, che ritiene il suo.
10. Verg, En. 11, edrrettis horret fquamis, © sibilat ore. Intendiamo
mente il fischiare de i serpenti.

\item[SALTELLA] Fa spessi, e piccoli salti; è il saltar delle Rane, Vedi
6. stan. 37,

\item[MORSO dalla Tarantella] Per la Calavria, e Puglia dicono si trovi un piccolo
  ragno detto Tarantola, o Tarantella, il quale nato \textit{ex putri} scappa dalle fessure
  della terra in tempo di state. Questo mordendo un'huomo, gli mette addosso
  una infermità specie di rabbia, che lo forza a ballare continovamente dalla
  levata, al tramontar del sole, ne prova quiete, se non quando sente sonare con
  chitarra o con altro strumento simile, un'aria detta perciò la Tarantella, al qual
  suono questo tale attarantato si affatica a ballare tanto, che stracco casca come
  morto; e stato in questo svenimento qualche hora, si rizza, e cessa di ballare, restando
  sano per qualche giorno: E perché in quel paese si trovano molti infettati
  da tal veleno, vi sono anche molti, che fanno il mestiero del sonare, e son
  pagati dall'attarantato. Dicono, che tale infermità duri quanto dura la vita di
  quell'insetto che morsicò l'attarantato, la quale dicono, che non passi tre anni;
  e vi sono però huomini a posta pagati da quei Comuni, i quali vanno cercando
  questi animalucci per ammazzargli per universal benefizio, e ne hanno un tanto
  per tarantola, rassegnandola a un Rettore a ciò deputato. Dicono in oltre, che
  questo tale morsicato provi la detta infermità ogni anno per un mese poco più, o
  poco meno intorno a' quei giorni, ne' quali fu morsicato, che sarà intorno al Sol
  leone, e che se ne trovino di quelli che la provino ogni mese per qualche giorno.
  Si chiama \textit{Tarantola}, o \textit{Tarantella} dalla Città di Taranto, nel cui territorio forse
  più frequentemente si trova. Il Lalli nell'En.\ Tr.\ lib.\ 1, stan.\ 22. dice,
  \begin{verse}
    Enea quantunque bravo anch' ei tremante
    Morso dalla Tarantola parea,
  \end{verse}

\item[LO dicervella] Gl'introna la testa con le strida: lo sbalordisce; lo fa assordare
  con le stride.

\item[HA dato nelle vecchie] E' invecchiato, s' intende uno che si tratti da vecchio; ancor che non lo sia.

\item[SECCHIA] Vaso di rame, col quale si cava l'acqua dai pozzi. Vedi sotto
  C.7. stan. 3. Ed il detto \textit{far come le secchie} senz'altra aggiunta, significa andar in
  giù, e in su, appunto come fanno le secchie infunate nella Carrucola.

\item[BARACCA] Intende abitazione, Che \textit{baracca} vuol propriamente dire quel
  luogo, che s'eleggono i soldati in campagna per loro abitazione, nel quale fanno
  un ricinto, e capannello di frasche, o d' altro,col quale si difendono dal
  sole, e dalle acque. Viene dal verbo barrare, che vuol dir Circondare, o accerchiare.
  Si dice anche \textit{trabacca}, o corrottamente, o pure \textit{eo quod trabibus constructa sit}.

\item[NON è la via dell' orto] Questo dettato significa; la via è lunghissima, e disastrosa,
  perché perr ordinario dall'orto alla casa non è più lungo viaggio, che cavare
  un piede fuori della porta, la quale di casa esce nell' orto, essendo per lo
  più nella Città gli orti appiccati alle case.

\item[CORPO! ed al Celon s'attacca] Vuol dir Corpo del Cielo. Si dice Corpo del
  mondo, Corpo del diavolo, ec, Ma quando uno passa più là bestemmiando le
  Deità, diciamo: \textit{Ei l'attacca al Celone} per intendere, egli entra nel Cielo, cioè
  bastemmia i Numi Celesti; E per render più oscuro questo detto, ci serviamo della
  vace \textit{Celone}, che vuol dir quel panno, che si mette sopr' alla tavola da mensa
  avanti di distendervi sopra la tovaglia.

\item[TU sarai morto] Detto ironico per mostrar la poca stima, che si fa della fatica,
  che habbia durata uno a nostro pro, ed il poco grado, che gli sen' habbia, massime
  quando quel tale ne fa grande ostentazione.

\item[NON sine quare] Voci latine usate nel suo significato; e dicesi: \textit{Non sine quare
  lupus at urbem}, e significa non senza qualche fine, o cagione. Franco Sacch.
  Nov.2. \textit{Gli venne voglia d'andar a trovare il Re Adovardo, e non sine quare, perché
    egli havea sentito molto lodarlo}.

\item[POSSON fondar la Compagnia de' brutti] Sono tutti bruttissimi. Habbiamo in
  Firenze un'Accademia, o Compagnia detta de' Brutti, la quale si raguna ogni
  anno il giorno di Befana (che così si dice il giorno dell'Epifania) ed in un lautissimo,
  e stravagante simposio si crea il Console nuovo per un'anno, e si appella
  il Fondatore, e si fa sempre il più brutto. E di questa intende il nostro
  Poeta.

\end{description}
\section{STANZA XIII --- XVI.}
\begin{ottave}
\flagverse{13}Lascian Plutone, e corron dalla Druda\\
I due spirti, aspettando il suo decreto,\\
Ed ella allor che fa da Cecco suda\\
Per far sì che Baldon dia volta a dreto,\\
Ed anche se si può ch'ei vada a Buda,\\
Gli prega, che le dien qualche segreto\\
Da far Senz'altre guerre, ovver contese,\\
Che quelle genti sfrattino il paese.
\end{ottave}

\begin{ottave}
\flagverse{14}Io ho (diec un di lor) bell'e trovato\\
Un invenzion, che ci verrà ben fatto,\\
Perché il Duca Baldone è innamorato\\
Della Geva di Corte, e ne va matto,\\
Ma la furba lo tiene ammartellato,\\
E a due tavole dar vorrebbe a un tratto,\\
Tenendo il piè in due staffe, amando lui,\\
E parimente il Duca di Montui.
\end{ottave}

\begin{ottave}
\flagverse{15}Però se non finghiam ch' egli le scriva\\
Ch'il suo rivale (adesso ch'egli ha inteso\\
Ch'ei s'è partito) con la gente arriva,\\
Per volergliela su levar di peso,\\
E che se proprio è ver, che per lei viva\\
(Com'ei spesso giurò) d'amore acceso,\\
E se gli è cara lo dimostri, e prenda,\\
Ed armi, e bravi, e corra, e la difenda.
\end{ottave}

\begin{ottave}
\flagverse{16}Vedrai ch'il Duca torna allota allotta\\
Correndo a casa, come un saettone\\
Con quanta ciurma, ch'egli ha qua condotta,\\
Per voler ammazzar bestie, e persone.\\
Hor dunque tu che sei saputa, e dotta,\\
Che non la cedi manco a Cicerone,\\
Scrivi la carta, che tu sai che noi\\
Sian tutti un monte d'asini, e di buoi.
\end{ottave}

I Diavoli trovano l'invenzione di far diloggiar Baldone da Malmantile, e
questa è fargli intendere, che la Geva sua dama è in pericolo d'esser rapita, e
dicono a Martinazza, che scriva la lettera.

\begin{description}
\item[DRVDA] Innamorata, amante, ec, se bene non sempre si piglia in significato
  disonestoso; Qui intende dama di Plutone, che era Martinazza, che, come
  strega, haveva lui per innamorato.

\item[FA da Cecco suda] S' affanna,s' affatica. Scherza con questo nome \textit{Cecco suda},
  perché quand'uno s'affatica, e s'affanna senza proposito, mostrando di far gran
  cose diciamo: \textit{Il tale suda}. Di questa natura era quel Cortigiano descritto dal
  Berni nelle Rime. \textit{Ser Cecco non può star senza la Corte, Ne la Corte può star
    Ser Cecco}.

\item[VADA a Buda] Vada via per non tornar più. Proverbio nato dalla guerra,
  che già fece il Turco contro Lodovico Re d'Ungheria quando acquisto Buda circa
  l'anno 1626., che vi morirono quasi tutti li Cristiani, che vi andarono, ed
  il medesimo Re; E però da quel tempo in qua dicendosi:\textit{Il tale è andato a Buda},
  s'intende è andato via per non ritornar più, o vero è morto, ed ha il medesimo
  senso, e per la medesima cagione; \textit{Il tale e andato a Scio}. \textit{È andato a Patraffo}, scherzo
  sulla Città d'Acaia famosa per il martirio di S. Andrea, come se si dicesse in
  Latino: \textit{ivit Patras}; e sulla frase usata dalla scrittura, sopra quei che muoiono,
  e si seppelliscono, quasi dica; È andato \textit{ad patres suos}.

\item[SFRATTINO, o sbrattino il paese] Ripuliscano il paese, cioè se ne vadano.

\item[DAR' a due tavole a un tratto] Far due negozzi in uno stesso tempo. Tratto
  dal giuoco di sbaraglino, nel quale con un sol tiro, si dia due, e tre tavole, o
  girelle. Si dice anche: \textit{far' un viaggio, e due servizj}. Vedi sotto C.~6, stan.~7.

\item[TENERE il piede in due staffe] Attendere a due partiti. \textit{Unum eligere, \& alterum
  non dimittere}. Tacito \textit{Diversas spes spectare}.

\item[MONTUI] Villaggio vicino a Firenze. Dovrebbe dirsi Mont'Vghi dalla famiglia
  degli Ughi antichissima Fiorentina. Ricordano Malespini nella Stor. Fiorentina
  cap 32. Il sesto compagno ebbe nome Ugo, questi anche fue nobilissimo
  gentiluomo Romano, e di questo discesono gli Ughi, e per innanzi il poggio,
  che oggi si chiama Montughi, s'è chiamato per loro. Lo stesso conferma Gio,
  Villani \libcap[4]{11}.

\item[ALLOTTA allotta] Allora allora; Subito subito. \textit{Nulla interposita morula}.

\item[SAETTONE] Specie di Serpe, detto così, perché forse vada veloce come
  una saetta, e credo sia il \textit{coluber} dei Latini.

\item[CIURMA] Propriamente vuol dire Remiganti di galera: Ma qui è presa per
  soldatesca, come si trova anche presa in più Storie Fiorentine antiche, e sopra
  C.3. stan.76., e sotto \cstan[11]{76}. dal Latino \textit{turma}, se bene propriamente si
diceva di soldati a cavallo.

\item[VUOL ammazzar bestie, e persone] Vuol disertare il paese. Quando vogliamo
  esprimer uno, che vanti di voler far gran bravure, e non lo giudichiamo atto a
  farne veruna, diciamo \textit{Vuol ammazzar bestie, e persone}. Ed in tal senso di derisione
  è preso nel presente luogo. Il Berni nelle rime congiunse queste due voci curiosamente
  allor che disse: \textit{Con un mondo di bestie, e di persone}.

\item[SEI saputa] Sei dotta; sei scientifica. Donna \textit{saputa}, \textit{sacciuta}, \textit{saccente} vuol
  dir di Una donna, che in tutte le cose vuol far da maestra. Colla stessa figura di
  \textit{saputo} per saccente, dicesi \textit{Avvertito}, \textit{Accorto}, \textit{Avvisato}, e dagli antichi \textit{Sentito}
  per huomo  che avverta, e che s'accorga delle cose, e che stia sull'avviso, e
  simili. Il participio passivo in forza di attivo.

\item[SIAMO una mana d'asini, e di buoi] Siamo tanti ignoranti. Per lo più a queste
  due bestie ed al Castrone assomigliamo coloro, che non hanno scienza alcuna.
  Se bene l'Autore sapeva, che il Demonio possiede tutte le scienze, che così
  suona il suo Greco nome \textit{Daemon}, cioè sapiente; e noi d'uno che sappia eccellentemente
  qualche cosa dichiamo; Egli è un Demonio; nondimeno ha voluto,
  che questi due Diavoli si dichiarino ignoranti, acciò che si creda più facilmente
  l'errore, che fecero di scambiare le palle, come vedremo.
\end{description}
\section{Stanza XVII --- XXI.}
\begin{ottave}
\flagverse{17}Non ti dò contro, rispond'ella, a questo, \\
Ed h gusto chee voi vi conoschiate: \\
Hor sù, dice il Demonio, scrivi presto\\
Due parole in tal genere aggiustate: \\
sì, dic'ella; ma vedi, io mi protesto, \\
Ch'io non portai mai lettere, o imbasciate. \\
Scrivi, soggiunge quei, che quanto al porta \\
Eccomi lesto qui con Gambastorta.
\end{ottave}

\begin{ottave}
\flagverse{18}E per dar al negozio più colore,\\
In forma voglio ir' io d' una comare\\
Della sua Geva detta Monafiore\\
Confidente del Duca in ogni affare;\\
Gambastorta verrà da servitore,\\
Che mostri di venirmi a accompagnare,\\
E già per questo ho fatte far di cera\\
Due palle, una ch'è bianca, e l'altra nera.
\end{ottave}

\begin{ottave}
\flagverse{19}Quand'un tien questa nera in una branca,\\
Di subito d'un'huom prende figura; \\
E s'ei vi chiude quell'altra ch'è bianca,\\
In femmina si muta, e trasfigura. \\
Sì che riguarda ben s'altro ci manca, \\
E distendi mai più questa scrittura; \\
Ch'il mio compagno, ed io qua per viaggio \\
Ci muterem l'effigie, e il personaggio.
\end{ottave}

\begin{ottave}
\flagverse{20}La nera a lui darò ch'altrui lo faccia\\
Parere un huom di venerando aspetto;\\
La bianca terrò io, che membra, e braccia\\
Della donna mi dia, che già t'ho detto.\\
La Strega qui gli dice, ch'ei si taccia;\\
Perch ella scrive, e guasto le ha un concetto\\
Ma lo scancella, e mettelo in postilla.\\
Così piega la carta, e la sigilla.
\end{ottave}

\begin{ottave}
\flagverse{21}Le fa la soprascritta, e poi finisce\\
A piè d'un ghirigoro in propria mano,\\
E con essa quel diavolo spedisce\\
Alla volta del Principe di Ugnano;\\
Là dove l'uno e l'altro comparisce\\
Con una delle dette palle in mano,\\
Credendo l'un rappresentar la Fiore,\\
E l'altro il Servo, ma sono in errore.
\end{ottave}

Martinazza scrive la lettera a Baldone in nome della Geva, e i diavoli pigliano
la medesima lettera per portarla un di loro trasformato in Mona Fiore, e
l'altro in un Servo per via di due palle, e se ne vanno così da Baldone. Ma per
havere scambiate le palle, chi dovea apparire la Fiore, appare il Servo, e furono
scoperti.

\begin{description}
\item[NON portai mai lettere, o imbasciate] La maggiore offela, che si possa fare a
  certe donnicciuole, è il dir loro; porta lettere; porta imbasciate; fa servizzi, porta
  polli (detto credo io dal Franzese Poulet, che significa letterino d'amore, quasi
  portatrice di lettere amorose) perché vuol dire Ruffiana: E però madonna
  Martinazza, che non vuole quest'offesa addosso si dichiara, che non è donna da
  portar lettere, o ambasciate, cioè da far la ruffiana.

\item[ECCOMI lesto] Eccomi pronto: Eccomi all'ordine. \textit{Lesto} in questo
  luogo vuol dir disinvolto, e senza imbarazzi.

\item[DAR colore al negozio] Far' apparir per vero quel che è incerto; Dargli verisimilitudine.
  Questo fanno appresso i Rettorici quei, che da loro sono chiamati
  Colori. Givvenale dice: \textit{dic, Quinctiliane, colorem}.

\item[COMARE] Quella che tiene la creatura al Battesimo. E qui il Poeta osserva
  il costume, che in simili amori per lo più la Balia, la Comare sono mezzane, e
  portano le parole,

\item[MONA] È parola sincopata da Madonna, ed è il titolo che si da comunemente
  alle donne d'infima plebe dicendosi in diminuzione Signora, Madonna,
  Monna, come Signore, Messere, Sere. Ma perché Monna oltre al significato di
  Bertuccia, ha ancora altro significato osceno (almeno in lingua Veneziana) noi
  per sfuggir l'equivoco; hoggi costumiamo dire \textit{Mona} e non Monna:

\item[MAI più] Hormai, Cioè finiscila una volta: È termine dimostrative d'una
  certa impazienza, e si dice: \textit{Oh mai più}: ed è il latino \textit{tandem aliquando}: e si
  confà con l'imperativo, dicendosi: \textit{Oh mai più: finitela}.

\item[POSTILLA] Nel nostro idioma ha diversi significati; perché, o vuol dire
(figuratamente secondo Dante) immagine d'un'oggetto, che ritorni alla nostra
veduta da un vetro, o dall'acqua chiara, Dan, Par. C. 30,
\begin{verse}
  \backspace Quali per vetri trasparenti, e tersi,
  O ver per acque nitide, e tranquille,
  Non sì profonde, ch'i fondi sien persi;
  \backspace Tornan de nostri visi le postille
  Debili si, che perla in bianca fronte
  Non vien men tosto alle nostre pupille.
\end{verse}

O vuol dire annotazioni, o glossa, che i Latini dicono expositio. O si piglia per
breve scrittura aggiunta, ed è composta di due dizioni \textit{post}, \& \textit{illa}. Quasi dica,
\textit{Postilla verba}, cioè dopo quelle parole, scrivi, o aggiungi questo, e questo. E
da queste annotazioni, glose, o aggiunte hoggi per \textit{postilla} intendiamo anche
la margine del libro, cioè quel bianco che si lascia di sotto, e di sopra, e dalle
bande del foglio scrivendo, o stampando: sì che scrivere in possilla vuol dire scriver
in detta margine; e s'intende ogni aggiunta, che si faccia al testo scritto, o
stampato in qualsivoglia luogo della carta o sia di sotto, o di sopra, o dalle bande
fuori de i versi ordinati, e regolati; ed in questo modo, e luogo, dice che
scrisse Martinazza.

\item[GHIRIGORO] È un tratteggio di penna usato per lo più nelle soprascritte
elle lettere, come mostra il Poeta nel presente luogo, che faccia Martinazza.
\textit{Ghirigoro} da' nostri antichi era detto in volgare il nome Latino di Gregorio;
onde \textit{Papa Ghirigoro} trovasi sempre costantemente scritto nel Malespini, e nel
Villani; come era la lingua di quel tempo. Ma qui Ghirigoro apparisce per avventura
dal girare, e rigirare della penna così detto. E le parole \textit{In propria mano}
s'usano nelle soprascritte di quelle lettere, le quali si mandano a uno, che sia nel
medesimo luogo, o Città, o vero poco lontano da colui che scrive.
\end{description}

\section{STANZA XXII --- XXXVI.}
\begin{ottave}
\flagverse{22}Che Baconero il quale è un'avventato,\\
Nel dar la palla all'altro di nascosto,\\
Senza guardarla prima havea scambiato,\\
E preso un granchio, e fatto un grand'arrosto,\\
Perciò quand'a Baldone egli è arrivato\\
Dice cose dal ver troppo discosto,\\
Mentr'egli afferma d'esser donna, e sembra\\
Huomo alla barba, all'abito, e alle membra.
\end{ottave}

\begin{ottave}
\flagverse{23}E Gambastorta anch'ei balordo, e stolto,\\
Mentr'apparir si crede un'huom dabbene,\\
Alla favella, alla presenza, e al volto\\
Per Una fa servizzj ognun la tiene.\\
Il foglio intanto il Duca havea lor tolto,\\
E veduto lo scritto, e quel contiene;\\
Resta certo di quanto era indovino,\\
Ch'i furbi vorrian farlo Calandrino.
\end{ottave}

\begin{ottave}
\flagverse{24}E poiché gli hanno detto, che la Geva\\
A lui gli manda con quel foglio a posta\\
Ma prima che da loro lo riceva\\
Hann'ordine d'haverne la risposta;\\
E soggiunto, che mentr'ella scrivea,\\
Gettava gocciolon di questa posta,\\
Per il trambusto grande ch'ella ha havuto,\\
Come potrà sentir dal contenuto.
\end{ottave}

\begin{ottave}
\flagverse{25}Egli è (dic'egli) un gran parabolano,\\
Chi dice ch'ell'ha scritto la presente,\\
Quand'ella non pigliò mai penna in mano,\\
E so di certo ch'ella n'è innocente.\\
Che poi tu sia la Geva, ch'in Ugnano\\
A me fu molto nota, e confidente,\\
E tu sia huom, a dirla in coscienza,\\
A me non pare, e nego conseguenza.
\end{ottave}

\begin{ottave}
\flagverse{26}I buon non compagni a una risposta tale\\
Guardandosi in viso, in quel sendosi accorti,\\
Ch'egli hanno equivocato, e fatto male;\\
Restan quivi allibbiti, e merzi morti,\\
Ed alle gambe havendo messo l'ale\\
Fuggon ch'e par ch'il diavol se li porti\\
Con una solennissima fischiata\\
Di Baldone, e di tutta la brigata.
\end{ottave}

Giunti quei Diavoli da Baldone, credendosi rappresentare uno la Fiore e
l'altro il Servo, non essendo accorti d'havere scambiate le palle, fecero la loro
ambasciata: Ma Baldone, compreso, che questa era una furberia, non tanto da
ciò, quanto dall'essergli noto, che la Geva non sapeva scrivere, se gli levò dinanzi
con una gran quantità di fischiate.
\begin{description}
\item[AVVENTATO] Uno che opera senza considerazione, e furiosamente. Huomo
  inconsiderato, e precipitoso; dal ivo Latino \textit{adventare} in significato
  d'avvenirsi, cioè imbattersi in una cosa con velocità, e con furia.

\item[DI nascosto] E lo stesso, che Di soppiatto detto sopra C.~1. stan. 75.

\item[PIGLIAR un granchio] Pigliare errore; Intender una cosa per un'altra. Si dice
  \textit{pigliare un granchio a secco} quando uno nel picchiar qualche materiale, scambiandolo,
  si batte il martello sopr'alle dita, o si serra le dita fra due materiali: e da questo
  errore intendiamo poi far un'errore, quando diciamo \textit{pigliare un granchio}. Berni
  \textit{Che Virgilio ha preso Un granciporro}.

\item[FAR un arrosto] Far un'errore. E' lo stesso che \textit{pigliar un granchio}. Viene
  per avventura dal verbo \textit{arrostarsi}, che vuol dir affaticarsi spropositatamente, e
  furiosamente; e le cose fatte in furia non si fanno mai bene.

\item[BALORODO, e stolto] Sinonimi che significano Huomo senza giudizio. La voce
  stolto è pura latina, e balordo è lo stesso che in Lat. \textit{bardus}.

\item[UNA FA servizzj] Come s'è detto sopra s'intende una Ruffiana.

\item[VOGLION farlo Calandrino] Calandrino, secondo che dice il Boccaccio nelle
  sue Novelle, fu un'huomo tanto credulo, che gli fu dato ad intendere fino, che
  egli era pregno, e però da costui diciamo \textit{Tu mi vuoi far Calandrino} per intendere
  \textit{Tu mi vuoi far credere quel che io so, che non è vero}. Si dice anche \textit{far
    Cappellino}, da uno de' nostri tempi della natura di Calandrino.

\item[HANNO ordine d'haverne la risposta] Il Poeta per maggiormente esprimere la
  castronaggine di costoro, fa che chieggano la risposta prima di presentar la proposta.

\item[GETTAVA goccioloni di questa posta] Lagrimava gagliardamente. Il termine
  \textit{Di questa posta} significa grossezza; \textit{erano pere di questa posta}, cioè pere grossissime,
  e si suppone, che colui, il quale dice così, accompagni il parlare col gesto
  delle mani dimostrante la grossezza di quella tal cosa. Si dice anche \textit{tanto fatte};
  \textit{tanto grosse}, come vedremo sotto \cstan[10]{17}. 18. e 36.

\item[TRAMBVSTO] Travaglio, rimescolamento, sollevamento d'animo per causa di disgrazie.

\item[PARABOLANO] Bugiardo; chiacchierone; spropositato: Da Parabola, cioè
  similitudiae, o Racconto; ne' Capitoli di Carlo il Calvo si legge. \textit{Parabolaverunt,
    simul, \& consideraverunt}. Parlarono insieme, Du Fresne\footnote{Charles du Fresne, sieur du Cange, più noto come Du Cange (Amiens, 18 dicembre 1610 --- Parigi, 23 ottobre 1688), storico, linguista e filologo. Minucci utilizza il suo \textit{Glossarium ad scriptores mediae et infimae latinitatis}, che nel 1688 era una novità editoriale.} alla V. \textit{Parabola}.

\item[SO ch'ella n'è innocente] Intende; io so ch'ella non sa scrivere. Per esprimere
  uno che non habbia ne pure una minima notizia d'una tal cosa diciamo:
  \textit{Il tale non ha peccato alcuno nella tal cosa o è innocente della tal cosa}.

\item[NEGO conseguenza] Nego il tutto: perché negando la conseguenza, si viene
a negare implicitamente tutto l'argomento, e così tutto il discorso.

\item[ALLIBBITI] Confusi, sbalorditi per un subito timore, o vergogna, e perciò
  diventati di colore smorto, e gialliccio, come, seccandosi, diventano le potature
  degli olivi, che si chiamano \textit{libbie}, dalla qual voce viene \textit{allibbito}, e \textit{allibbire}.
  Vedi il Vocabolario della \textit{Crusca} alla Voce \textit{Allibbìre}. Il Varchi Stor. Fior. lib, 10.
  \textit{Niuno udì tal cattiva nuova, il quale incontinente (quasi le fusse venuta meno la terra
  sotto a' piedi) non allibbisse}.

\item[FISCHIATA] Romore di voci, fischi, urli, battimenti di mani, e d'altro,
  che si fa dietro a uno per dargli la burla. Far le fischiate a uno, quel che i Lat.
  dissero \textit{exsibilare}.
\end{description}

\section{Stanza XXVII. --- XXX.}
\begin{ottave}
\flagverse{27}Adesso a Calagrillo me ne torno
Con la dolente Psiche ognor d'attorno, \\
Ch' ad ggni quattro passi fa un lamento.\\
Ha camminato tutto quanto il giorno,\\
E domandato cento volte, e cento\\
La via di Malmantile, e similmente\\
Di Martinazza, e se v'è di presente.
\end{ottave}

\begin{ottave}
\flagverse{28}Dè in un, ch'al fin la mette per la via \\
Con dirle, che quest'orrida Befana, \\
Che già d'un tozzo haveva, carestia \\
E stava come l'erba porcellana, \\
In hoggi ha di gran soldi in sua balia, \\
Ed ha una casa come una Dogana, \\
E nella Corte è in grado, e giunta a segno, \\
Ch'ell'è il \textit{totum continens} del Regno.
\end{ottave}

\begin{ottave}
\flagverse{29}Che la padrona il tutto le comparte\\
Come s'in Malmantil sien due Regine,\\
Anzi il bando si manda da sua parte,\\
Perch'ella soffia il naso alle galline,\\
Così poi c'hebbe dato libro, e carte,\\
Entra nell'un viè un, che non ha fine,\\
Costui, che quivi s'è posso a bottega\\
A legger sopra il libro della Strega,
\end{ottave}

\begin{ottave}
\flagverse{30}Quest'altro, che non cerca da costui\\
Di questi cinque soldi, havendo fretta,\\
Poi ch'egli ha inteso quel che fa per lui,\\
Sprona il cavallo tutto a un tempo, e sbietta,\\
La donna che trovare il suo, colui\\
Di giorno in giorno per tal mezzo aspetta,\\
Per non lo perder d'occhio, e ch'ei le manchi,\\
Segue la starna, e gli va sempre a i fianchi
\end{ottave}



Torna il Poeta a parlar di Calagrillo; il quale camminando Psiche s'imbatte
in uno, che le dà avviso di dove sia Martinazza.
\begin{description}
\item[MARCHIARE] Si dice marciare, che vuol dir camminare. Voce Francese,
  ma già fatta Italiana. Vedi sopra \cstan[1]{43}. E più accosto alla pronunzia,
  Oltramontana, dicesi anche \textit{Marciare}, forse da \textit{Marcia}, contrada, pace, cammino
  \textit{danesmarce} disse il Villani la Danimarca; cioè \textit{Danese Contrada}.

\item[BEFANA] Intendiamo Donna brutta, malfatta. Vedi sotto \cstan[8]{30}.
e C.9, stan. 1.

\item[TOZZO] S intende pezzo di pane. \textit{Haver carestia d'un tozzo}, Vuol dire esser mendico, pezzente.


\item[STAVA come la porcellana] Cioè terra terra, come l'erba porcellana, che serpeggia
  per terra, e non alza mai virgulti; detta \textit{porcellana} dal Latino \textit{Portulaca}.
  E questo detto significa Uno che sia in povero stato, e non habbia modo di sollevarsi,
  che i Latini pure dicevano \textit{humi jacere}.

\item[IN sua balìa] In suo potere, e dominio. Balìa e voce fatta venire dal Monosini
  dalla Greca \textit{Buleia}, che suona lo stesso che \textit{Bulo}, cioè consiglio, Parlamento,
  Senato. A noi suona Potestà, giurisdizione, autorità, e quel che i Latini dicevano,
  \textit{potestas imperium}. Dan. Purg, C. 1.
  \begin{verse}
    \backspace Ed hora intende mostrar quegli spirti,
    Che purgan se, sotto la sua balìa.
    \verseprefix{Petr.C.36.}Mentre ch'il corpo è vivo,
    Hai tu il freno in balia de' pensier tuoi.
  \end{verse}

\item[HA una casa come una Dogana] Cioè piena di robe, come sono le Dogane piene di mercanzie.

\item[IL Bando va da parte sua] Cioè, ella comanda.
\item[SOFFIA il naso alle galline] Ella fa tutte le faccende. E questi tre modi di dire
  \textit{Totum continens del Regno}, \textit{Bando va da parte sua}, e \textit{soffia il naso alle galline} hanno tutti
  lo stesso significato; ma di questo ci serviamo per lo più per derisione, per intendere d'uno
  che habbia ambizione d'esser creduto gran ministro, ed habbia i maggiori maneggi
  d'un governo, e non sia vero; che per ischerzo direbbesi anche \textit{Arcifanfano}.
  En.Tr.l.4.st.15.\textit{Soprattutto a Giunon, che del far razza È detta l'arcifanfana, e 'l fac todo}.

\item[DAR libro, e carte] Dare esatta notizia d'alcuno. Viene da coloro, i quali
  havendo debito co' Magiftrati, son mandati in esazione a i Ministri forensi, alli
  quali Ministri i Magistrati mandano il contrassegno del libro, nel quale è scritto
  il debito di quel tale, il nome, e casato di esso, l'origine, e somma del debito,
  ed a quante carte è la sua partita: E questo si dice \textit{dar libro, e carte}, che passano
  in proverbio, significa Dar notizia chiara, ed esatta d'alcuno; o palesare chi
  habbia fatta un'azione per altro occulta.

\item[ENTRA nell'un viè uno] Fa un discorso da non uscirne mai, come avverrebbe
  se uno volesse seguitare \textit{Un vie uno fa uno}, \textit{due viè due fa quattra}, ec, che s'anderebbe
  nell'infinito. Dice il Varchi nel suo Ercolano, che in questo senso si dice
  \textit{Cantar la canzone dell'uccellino}. Con tal dettato s'esprime un chiacchierone,
  che cicalando, faccia molte digressioni spropositate per allungare il suo cicalamento
  con racconti assai sconvenevoli, che si dice; \textit{Entrare in un ginepraio}, \textit{saltare
  di palo in frasca}.

\item[S'È messo a bottega] S'è preso per arte, per suo mestiero, o negozio. Quando
  uno fa qualche operazione con tutta applicazione, ed attenzione, e con dimostrazione
  di voler durare assai, diciamo; \textit{Costui s'è messo a bottega}.

\item[LEGGER sul libro d'alcuno] Narrar le azioni, qualità, e stato d'alcuno.

\item[NON cerca questi cinque soldi] Non cerca, non gl'importa, non proccura
  sapere questa cosa. Quand'altri fa un discorso, e fa una digressione senza tornar
  più al primo proposito, se li dice: \textit{Voi pagherete la pena de' cinque soldi}. Vedi sotto
  C.~8. stan. 15. E però dicendo: \textit{Non cerco questi cinque soldi},s'intende; non mi
  curo di guadagnar questa pena de' cinque soldi, con obligarti a seguitare il principiato
  discorso.

\item[SBIETTA] Scappa via presto, Vedi sotto \cstan[7]{87}.

\item[IL suo colui] Il suo amante, cioè Cupido.

\item[PER non lo perder d'occhio] Perché non le esca di vista. Per non lo smarrire.

\item[SEGVITA la starna] Quand'uno seguita un'altro per haver da lui qualche
  favore, diciamo: \textit{Ei seguita la starna}. E si dice la starna, e non altro uccello,
  perché queste si vincono col seguitarle, osservandole dove si posano, e straccandole nei loro voli.
\end{description}
\section{Stanza XXXI ---- XXXV}


\begin{ottave}
\flagverse{31}Quando al Castello al fin sono arrivati,\\
Là dove altrui assordano l'orecchie\\
Gli strepiti dell'armi, e de' soldati,\\
Che d'ogn'intorno son più delle pecchie;\\
Domandan soldo, ed a Baldon guidati,\\
Che havendo del guerrier notizie vecchie,\\
Gli va incontro, l'accoglie, e riverisce,\\
Ed egli a lui con l'armi s'offerisce.
\end{ottave}

\begin{ottave}
\flagverse{32}Ma piacciati, soggiunse, ch'io ti preghi\\
Per questa donna rimaner servito,\\
Che questo ferro pria per lei s'impieghi\\
Per conto qua d'un certo suo marito.\\
A tanto Cavalier nulla si nieghi,\\
Risponde a ciò Baldon tutto complito,\\
Tu sei padrone; fa ciò che tu vuoi,\\
Non ci van cirimonie fra di noi.
\end{ottave}

\begin{ottave}
\flagverse{33}Ti servirò di scriverti alla banca,\\
E in tanto per adesso io ti consegno\\
Il gonfalon di questa ciarpa bianca,\\
Che tra le schiere è il nostro contrassegno;\\
Tal che libero il passo, e scala franca\\
Haurai per dar' effetto al tuo disegno;\\
Che non so qual si sia, ne lo domando;\\
Pero va pur ch'io resto al tuo comando.
\end{ottave}

\begin{ottave}
\flagverse{34}Ei lo ringrazia, E gito più da presso,\\
Ove sta chiuso di Psiche il bel Sole,\\
Ad essa dice: In quanto al tuo interesso,\\
Fin qui non t'ho servito, e me ne duole,\\
Che tu non pensi, havendoti promesso,\\
Ch'io faccia fango delle mie parole,
E ch'il mio indugio, e il non risolver nulla\\
Sia stato un voler darti erba trastulla.
\end{ottave}

\begin{ottave}
\flagverse{35}O ver ch'io me la metta in sul liuto,\\
O ti voglia tener l'oche in pastura,\\
Come quel che ci vada ritenuto\\
Per mancanza di cuore, e per paura,\\
Perché si come havrai date veduto,\\
Non ho fin qui trovata congiuntura\\
Di chi m'indirizzasse qua al Castello,\\
Per poterne cavar cappa, o mantello.
\end{ottave}


Calagrillo con Psiche arriva al Campo, e chiede soldo: Baldone l'accetta, e
gli da licenza d'andare a servir Psiche, con la quale avviandosi verso Malmantile,
Calagrillo si scusa di non' haver prima servita.

\begin{description}
\item[SCRIVER alla banca] Arrolare uno per soldato: Banca diciamo quel luogo
  dove sono scritti i soldati, e dove son loro pagatii denari degli stipendj.

\item[GONFALONE] Vuol propriamente dire vessillo; ma si piglia per ogni sorta
  d'insegna, Vedi il Vossio\footnote{Gerhard Johannes Voss (Heidelberg, 1577 --- Amsterdam, 1649), noto in Italia come Vossio, o in latino Vossius. Teologo, filologo, storico. } \textit{de vitijs sermonis} lib, 1. ove di questa voce.

\item[CIARPA] E' una legaccia di drappo, che dai soldati si cinge come la cintura
  della spada. E per altro \textit{ciarpa} vuol dire quel che accennammo sopra Cant.
  3. stan. 5. Franzese \textit{Escharpe}.

\item[SCALA franca] Franchigia; Libertà d'andare, o stare. Passo libero.

\item[FAR fango delle sue parole] Disprezzare la parola data e non osservar le promesse.

\item[DAR erba trastulla] \textit{Metterla sul liuto}; \textit{mandar l'oche in pastura} hanno tutti tre
  lo stesso significato, che è trattener' uno con chiacchiere. Lat. \textit{verba dare}.
\end{description}
\section{Stanza XXXVI. --- XXXVIII.}

\begin{ottave}
\flagverse{36}Risponde Psiche a questa diceria:\\
Io non entro Signore in questi meriti,\\
Non ho parlato mai, ne che tu sia, \\
Tardo, o spedito, o ver che tu ti periti,\\
Quel che tu fai, tutt'è tua cortesia,\\
Per tal l'accetto, e 'l Ciel te lo rimeriti,\\
Con darti in vita honor, fama, e ricchezza,\\
Sanità dopo morte, ed allegrezza.
\end{ottave}

\begin{ottave}
\flagverse{37}Sta quieta, le dic'egli, e ti conforta, \\
Ch'io voglio adesso dar fuoco al vespaio, \\
Così col Corno, il quale al colle porta,\\
Chiama la guardia, o vero il portinaio, \\
Non è sì presto il gatto in su la porta, \\
Quand'ei sente la voce del beccaio; \\
Quanto veloce a questo suon la Ronda \\
Sopr' alle mura accostasi alla sponda.
\end{ottave}

\begin{ottave}
\flagverse{38}Un par d'occhiacci orlati di savore\\
Così addosso a un tratto gli squaderna,
Che par quand'il Faina alle sei hore\\
In faccia mi spalanca la lanterna,\\
E mediante un certo pizzicore,\\
Ch'ei sente al collo, i pizzicotti alterna,\\
Ond'alle dita egli ha fatti i ditali\\
D'intorno a innumerabili mortali.
\end{ottave}


Psiche rende grazie a Calagrillo della carità, che le promette, e facendo le lor
cirimonie, s'accostano al Castello, dove Calagrillo, sonando il Corno, chiama
la sentinella, la quale subito s'affaccia alle sponde delle mura.

\begin{description}
\item[DICERIA] Vuol dire Ragionamento, Discorso, Orazione: ma hoggi questa
  voce è usata per lo più per intendere Ragionamento stucchevole, e odioso per la
  lunghezza.

\item[NON entro in questi meriti] Non parlo di queste cose. Ma questo detto ha
  una certa forza d'esprimere: io non ardisco d'entrar tanto in la col discorso; maniera,
  che viene dal solersi dire; il merito della lite, o della causa, cioè l'importanza
  del fatto.

\item[SANITA, ed allegrezza dopo morte] E detto giocoso, perché un corpo morto
  non può haver sanità, ne allegrezza, ne altre passioni. Ma si potrebbe anche
  dire, che questa donna, parlando iperbolico, voglia dire che egli viva sano, ed
  allegro sempre eziam dopo morte, il che è imposhibile, come è imposibile viver
  mill'anni, e pure si dice: vi prego mille anni di vita, \textit{Sanità} è un'augurio, che
  corrisponde at Greco \textit{hygiainein}, cioè \textit{star sano}, che metteva innanzi alle sue epistole
  Pittagora devotissimo della sanità; \textit{Allegrezza} corrisponde a quel saluto, che
  in principio esprimevano i Greci comunemente nelle lor lettere, perché dove i Latini
  pongono \textit{Salutem dicit}, essi scrivevano \textit{Chairein}, cioè come tradusse Orazio
  in una sua Epistola \textit{Gaudere}, volendo dire, Il tale al tale desidera \textit{allegrezza},
  siccome in quell'altro modo usato da Pittagora: il tale al tale, desidera \textit{sanità}.

\item[DAR fuoco al vespaio] Violentare a uscir fuora uno, che sia dentro; come segue,
  quando si da fuoco a un vespaio, che le vespe son forzate dal fuaco a scappar
  fuori. Vedi Omero lib, 16, dell' Iliade,

\item[LA voce del Beccaio] Vanno per Firenze alcuni Beccai, o Macellari vendendo
  carne per dare a' gatti, e fanno certe lor voci così ben conosciute da i medesimi
  gatti, soliti havere la carne, che appena costoro hanno aperta la bocca, che i
  gatti sono in sulla porta. A questi gatti assomiglia la guardia di Malmantile, che
  a pena sentito il suono del corno s'affaccia alla muraglia. Delle voci, e de' versi
  che fanno i venditori, che vanno attorno per invitare il compratore, Seneca ep.
  56. \textit{Iam libarij varias exclamationes, \& botularinm, \& crustularium, \& omnes popinarum
    institores, mercem sua quadam, \& insignita modulatione vendentes}.

\item[RONDA] Si dice quel Soldato di guardia, che rigira, e passeggi per la muraglia
  della fortezza, visitando la Sentinella, detta così dall'andare in volta, e
  come i Franzesi dicono, \textit{aller en rond}.

\item[SPONDA] Parapetto della muraglia; Quel pezzo di muro, che avanza alle
  muraglie sopra il terreno del terrapieno, e si dice \textit{sponda} quel muretto, o spalletta,
  che avanza sopra il terreno, a i pozzi, a' fiumt, ec.

\item[ORLATI di savore] Circondati di cispa per la similitudine, che ha con la cispa
  il savore secco. E \textit{savore} è uno intingolo fatto di noci, e pane pesto, e liquefatto
  con agresto. E \textit{cispa} diciamo quell'umor crasso, che si condensa intorno alle palpebre,
  e su i peli degli occhi.

\item[COSÌ a un tratto gli squaderna gli occhi addosso] Subito fissa sopra di lui gli occhi
  ben'aperti. E questo verbo \textit{squadernare} s'usa per divolgare, manifestare, ec.
  Dan. Par. C.33.
  \begin{verse}
    Ciò che per l'universo si sqaderna
  \end{verse}

\item[FAINA] Celebre Luogotenente di Birri così chiamato per soprannome.

\item[SPALANCARE] Aprir quanto si può una porta, un' armario, e simili: levare
  la palanca, cioè il palo, che tiene in alcune porte fermato tutta, o una
  banda della porta; aprire affatto. Vedi sotto \cstan[6]{43}.

\item[ZILOTTO] È uno stringimento, che si fa in qualche parte del corpo,
  pigliando la pelle col dito indice, e stringendola col dito pollice; e così faceva
  costui intorno al collo, \textit{alternando i pizzicotti}, cioè facendoli hor con l'una, hor
  con l'altra mano per pigliare i pidocchi, che sono quegli \textit{innumerabili mortali, che
    col loro gli hanno fatti i ditali}, cioè ricoperte le dita; Che \textit{ditale} intendiamo
  parte del guanto, che cuopre il dito.
\end{description}

\section{STANZA XXXIX. \& XXXX.}

\begin{ottave}
\flagverse{39}Non tanto s'abburatta per la rogna, \\
E pe' bruscol, che vanno alla goletta, \\
Quanto che dir non può quel che bisogna \\
Ch'ei tartaglia, e scilingua anche a bacchetta, \\
Qual il quartuccio le bruciate fogna, \\
Ne senza quattro scosse altrui le getta, \\
Tal si dibatte, e a vite fa la gola \\
Ogni volta ch'ei manda fuor parola.
\end{ottave}

\begin{ottave}
\flagverse{40}Bu bu, bu bu comincia, ch'il buon giorno\\
Vorrebbe dar al Cavalier, ch'ei tiene\\
Il Corrier, mediante il suon del Corno,\\
Del popol d'Israel ch'or va, hor viene;\\
Van le parole a balzi, e per istorno\\
Prima c'al segno voglian colpir bene;\\
Pur pinse tanto, che gli venne detto;\\
Buon dì Corrier, che nuova c'è di Ghetto.
\end{ottave}


Descrive il Poeta la guardia, la quale havendo creduto che Calagrillo fusse
un' Ebreo, lo saluta come tale.

\begin{description}
\item[S'ABBURATTA] Si dimena: Si dibatte. Abburattare propriamente vuol
dire Separare la farina dalla crusca con lo staccio.

\item[BRUSCOLI che vanno alla goletta] Intende i pidocchi, che vanno alla gola.
  \textit{Goletta} intendiamo l'estremità dell'abito da huomo intorno alla gola. Ed il
  Poeta copre questo detto con l'equivoco di \textit{Goletta}, fortezza in Barberia, e con la
  voce \textit{bruscoli}, che sono minutissime particelle di legno, o paglia, o simili, ed egli
  intende pidocchi.

\item[TARTAGLIARE] Intoppare nel profferir le parole; pronunziar con difficultà,
  e \textit{scilinguare} vuol dir Balbettare.

\item[A BACCHEITA] Comandare a bacchetta vuol dire Comandare assolutamente
  e dispoticamente in ogni congiuntura, come Re, o Capitano, che porti
  scettro, mazza, o bastone di comando, e di qui intendesi, che costui tartagliava,
  e scilinguava ogni lettera.

\item[QUARTUCCIO] Misura Fiorentina capace della sessantaquattresima parte
  dello staio\footnote{Staio, usato per misurare volumi aridi, come i cereali. Uno staio equivale a circa 24.3ℓ, la sua sessantaquattresima parte è circa 380ml.  Lo stesso nome Quartuccio si riferisce anche alla centosessantesima parte del barile da vino, circa 45.6ℓ. In questo caso un quartuccio equivale a circa 280ml. Il peso è approssimativamente uguale nei due casi. }, e per lo più è un vaso di legno.

\item[BRVCIATE] Marroni cotti arrosto in padella, o in forno, o sotto la brace.

\item[FOGNARE] Fogna vuol dire quel vacuo fatto ad arte sotto terra per dove
  passa l'acqua, e si conduce scolando al fiume dal Lat, \textit{fovea}: E di qui \textit{fognare la
    misura} vuol dir metter la roba nella misura in maniera, che apparisca piena,
  ma dentro vi sieno molti vacui, come facilmente segue nel quartuccio, entro al
  quale non si possono stivare i marroni, i quali per esser di figura rotonda non
  riempiono lo spazio, ma fanno naturalmente, che rimangano fra l'uno, e l'altro
  molti vacui nella misura; la quale poi, volendoli votare, è necessario squotere;
  perché s'affrontano nell'uscire, e soqquadrano alla bocca del quartuccio
  in maniera, che non potriano scappar fuori, se non si squotesse il vaso, ed uscendo,
  fanno un romore simile a uno che tartagli, le di cui parole pare, che non
  possano uscir di bocca, se egli non si squote, dibatte, o storce; e quell'intervallo
  che egli mette fra una parola, e l'altra lo figura il vacuo che sta fra un marrone,
  e l'altro. E questo intende col dire \textit{qual il quartuccio le bruciate fogna}, cioè
  fogna le parole con interuallo di tempo, e non di luogo.

\item[FAR la gola a vite] Storcer la gola. Vedi sopra \cstan[2]{9}.

\item[PER storno] Si dice quel ritornare indietro, che fa la palla che ha percosso
  nella parte opposta dove è stata tirata o sia muro, o sia altro, ed è termine proprio
  del giuoco delle pallottole, e s'intende quand'uno tira per accostarsi al segno
  per via di detto storno, e non direttamente: E così indirettamente uscivano
  di bocca a costui le parole. In somma vuol dire, che egli impuntava nel parlare,
  tartagliava, e parlava a salti.

\item[GHETTO] Così chiamiamo il Serraglio, nel quale stanno in Firenze, ed in
  altre Città gli Ebrei: E perché questi hanno nome di tener di mano a stregherie,
  però dice che il Corriere di quel luogo è solito spessoo andare a Malmantile a trovar
  la stregha Martinazza. \textit{Ghetto} e voce Caldea, che significa libello di repudia;
  onde noi diciamo \textit{Ghetto} per intender luogo di gente segregata, e repudiata
  dal commercio degli altri huomini. Gli Ebrei quando vogliono dire loro
  mogli, che le gastigheranno col repudiarle dicono; \textit{Ti manderó al Ghet}.
\end{description}

\section{STANZA XXXXL STANZA XXXKIL}

\begin{ottave}
\flagverse{41}Rispose l'altro, tal parola udita:\\
D'esser corriere già negar. non posso,\\
Perch'io l'ho corsa a far questa salita,\\
Ma quanto al Ghetto io non la voglio addosso; \\
Non ho che far con gente Ifraelita;\\
Ben ti farà il mio brando il cappel roffo,\\
E col darti sul viso un soprammano \\
D'Ebreo farà mutarti in Siciliano.\\
\end{ottave}

\begin{ottave}
\flagverse{42}Ma che vo il tempo qui buttando via\\
In disputar con matti, e con buffoni?\\
Il trattar teco credomi che sia\\
Come a' Birri contar le sue ragioni;\\
Ne dissi mal, perch' hai fisonomia\\
D'un di color, che ciuffan pe' calzoni,\\
E l' esser tu costì, par ch'ella quadri,\\
Ch' i Birri sempre van dove son ladri.
\end{ottave}

\begin{ottave}
\flagverse{43}Benché voi siate come cani, e gatti,\\
Ch'essi non han con voi gran simpatia,\\
Perché peggio de' diavol sete fatti,\\
Usando nel pigliar più tirannia;\\
Dell'alma sola quei son soddisfatti,\\
Ma voi col corpo la portate via.\\
Hor basta, se tra voi tant'odio corre.\\
Meglio a i lor danni ti potrò disporre.
\end{ottave}

\begin{ottave}
\flagverse{44}Hor dunque tu, che sei così pietoso,\\
Che pigli i ladri, acciò Mastro Bastiano\\
Sul letto a tre colonne almo riposo\\
Dia lor del tanto lavorar di mano,\\
Perch'a qualunque ladro il più famoso\\
Martinazza di rubar non cede un grano,\\
Che non uccella a pispole, ma toglie\\
Cupido a questa donna, ch'è sua moglie.
\end{ottave}

\begin{ottave}
\flagverse{45}Lo stesso devi oprar, c'a lei sia fatto;\\
Mentr'a costei non renda il suo Consorte.\\
A cui (perch'ei consente in tal baratto)\\
Questa potrebbe far le fusa torte;\\
Ed ei si cerca esstr mandato Hn tratto\\
Sull'asin con due rocche dalla Corte,\\
Sì che, se tu nol sai, ti rappresento,\\
Cè un disordine qui ne puo far cento,
\end{ottave}

\begin{ottave}
\flagverse{46}Però se voi adesso, a cui s'aspetta,\\
Costà non impiccate questa Troia,\\
Io stesso vuò pigliarmi questa detta,\\
E farle il Birro, e in sulle forche il Boia,\\
Mentre però Cupido non rimetta:\\
Ma se lo rende non vi do più noia,\\
Va dunque, e narra a lei quanto t'ho detto,\\
Ch'io qui t'attendo, e la risposta aspetto.
\end{ottave}


S'adira Calagrillo, che colui l'habbia preso in cambio del Corriere degli Ebrei,
e lo minaccia di rompergli la testa, e sfregiarlo; e dopo havergli detto molti improperj,
gli ordina, che da sua parte avvisi Martinazza, che renda Cupido; altrimenti
glielo farà render per forza.
\begin{description}
\item[L'HO corsa] Ho fatta questa cosa senza considerazione. Quand'altri fa qualche
  risoluzione, che non riesce poi buona, diciamo: \textit{Ei l'ha corsa} dall'armeggiare,
  e dal correre la giostra. Similmente diciamo; \textit{Fare una carriera}. Qui fa
  giuoco la voce corsa, che è cosa da Corrieri.

\item[NON la voglio addosso] Non la voglio sopportare. Si dice anche \textit{non la voglio in
  sul giubbone}.

\item[GENTE Israelita] Intende Ebrei: Popolo d'Israel.

\item[IL cappello rosso] Gli Ebrei in Firenze portano per contrassegno il Cappello
  rosso. Il Poeta dice, farò ben' io diventare Ebreo te col farti il cappello rosso col
  sangue. E poi d'Ebreo ti farò diventar Siciliano tagliandoti il viso, ed intende
  quel Siciliano Montambanco, che per accreditare il suo Olio da Ferite si faceva
  gran tagli nella persona, e con esso se le medicava.

\item[SOPRAMMANO] Quel colpo, che si dà con spada, o bastone, cominciando
  da alto, e calando a basso. Vedi sotto \cstan[10]{52}.

\item[BUFFONE] Uno che fa professione di trattener la brigata con facezie.

\item[DIR le sue ragioni a Birri] Raccomandarsi a chi non può, e non vuol far
  servizio; anzi ha caro il tuo male. Vuol anche dire discorrer con uno, che non
  bada a quel che tu dica; o vero buttar le parole al vento, Plauto disse nel Pseudolo:
  \textit{apud novercam queri}.

\item[CIUFFAN pe' calzoni] Cioè i Birri; i quali pigliano pe' calzoni. Il verbo
  \textit{ciuffare} ha del furbesco; e vuol dir Pigliar con presa stabile, e buona, come è
  quella che si fa, pigliando uno per il ciuffo, cioè pe' capelli. Petrarca. \textit{Le man
    l'avess'io avvolte entro a' capegli}.

\item[ESSER come cani, e gatti] Esser poco d'accordo, o poco uniti, anzi sempre
  nimici, come naturalmente sono i cani, e i gatti.

\item[NON ha gran simpatia\ La voce \textit{simpathia} Greca fatta Toscana significa inclinazione
  scambievole, o similitudine di genio, di voleri, e d'affetti.

\item[MAESTRO Bastiano] Intende il Boia, che allora così havea nome, e prima
  era stato Maestro Biagino. Vedi sotto \cstan[6]{56}.

\item[LETTO a tre colonne] Cioè le forche, le quali veramente son tre colonne con
  una stanga sopra a traverso, ed in molti luoghi sono in triangolo.

\item[LAVORAR di mano] Vuol dir rubare. scherza dicendo, che il Maestro,
  (cioè il Boia) perché essi ricevano qualche riposo da tanto lavorare (cioè rubare)
  gli mette in su 'l letto a tre colonne (cioè in sulle forche) ed in sustanza buol dire
  gl'impicca, perché son ladri. E Calagrillo, seguitando l'equivoco del riposo,
  dice alla guardia, che se ella ha punto di pietà, e discrezione, dovrebbe dar questo
  riposo in sul letto di tre colonne a Martinazza per il suo tanto lavorare, cioè
  impiccarla, perché è ladra. I Latini pure per dir copertamente rubare dissero:
  \textit{manu sinistra uti} secondo Catullo in Asinium.
  \begin{verse}
    Marrucine Asini, manu sinistra
    Non belle uteris in ioco, atque vino;
    Tollis lintea negligentiorum.
  \end{verse}
  E per dire copertamente Impiccar'uno, dicevano; \textit{literam longam facere}, come habbiamo
  notato altrove\footnote{Quarto Cantare, Ottava 27.}.

\item[NON cede un grano] Non cede punto. Che grano si può dire una particella
  inconsiderabile del peso\footnote{Misura per orefici, un grano toscano equivale a poco meno di 50mg}, poiché 24. grani fanno un danaro, 24. denari fanno
  l'oncia, e 12. once fanno la libbra\footnote{Una libbra di Firenze era circa 339.5g}.

\item[NON uccella a pispole] Non si cura di conseguir cose di poco momento, come
  è fra gli uccelii la pispola. I Latini dissero \textit{Non captat muscas}.

\item[FAR le fusa sorte] Far le corna. Vuol dir quand'una donna si mescola con
  altri huomini, che col suo marito. Il Burchiello Poeta capriccioco, il quale va
  sotto nome d'Accademico Fiorentino incerto, nella Raccolta delle Rime Piacevoli
  del Berni, Casa, ec.
  \begin{verse}
    Non ti fidar di femmina, ch'è usa
    A far le fusa torte al suo marito.
  \end{verse}
  Il Berni nel suo primo capitolo dell'orto dice:
  \begin{verse}
    E finalmente non fara mai fusa
    Donna alcuna per lui torte al marito.
  \end{verse}
  Si dice \textit{fusa torte} per intender copertamente Corna.

\item[MANDATO con due rocche in sull'asino] È costume in Firenze, al gastigo del
  delitto del pigliar più d'una moglie, aggiugnere una dimostrazione obbrobriosa,
  che è il far' andar' per la Città il delinquente legato sopra ad un'asino, con una
  mitra di foglio in capo, ed a cintola due, o più rocche inconocchiate, che significano
  le due, o più mogli.

\item[QUESTA troia] Questa porca. Epiteto vituperosissimo nelle donne, perché
  vuol dire Laida meretrice:: nell'huomo non è tanto ingiurioso il dirgli porco.

\item[MI vuò pigliar questa detta] Vuò pigliarmi l'assunto di far questa cosa. \textit{Star della
  detta} vuol dire prometter per un'altro, o star mallevadore, cioè di far una tal
  cosa, se non la farà quello, che è principalmente obbligato. \textit{Comprar una detta}
  vuol dir comprar un'avviamento, un credito, ec. \textit{Detta} è dal plurale Latino
  \textit{Debita}.

\end{description}
\section{STANZA XXXXVIIL}
\begin{ottave}
\flagverse{47}La Ronda, che far lite non si cura,\\
E vuol riguardar l'armi dalle tacche,\\
Quantunque ad alto sia sopr'alle mura\\
Molto lontana, e già in salvummeffacche,\\
Non vuol teners mai tanto sicura,\\
Che rilevar non possa delle pacche; \\
Però veduto havendo il Ciel turbato\\
Tace, ch'ei par un porcellin grattato.
\end{ottave}

\begin{ottave}
\flagverse{48}Lascia la sentinella, e caracolla\\
Giù pel castello, dando questa nuova,\\
E benché il Maggioringo della bolla\\
Gli habbia promesso, mentre ch'ei si mova\\
Di fargli porre a' piedi la cipolla,\\
Cercando della morte in bella prova\\
Vuol avvisar di ciò Mona Cosoffiola,\\
Ch'è per basire a questa battisoffiola.
\end{ottave}

La Guardia, che è un vero poltrone, sentendo le bravate ai Calagrillo, zitto
zitto si parte, e tremando va a dare questa nuova a Martinazza.

\begin{description}
\item[RIGUARDA l'armi dalle tacche] Non vuol cavar fuori la spada, per non la
  guastare. Intendi che costui era un codardo, perché per dir copertamente poltrone
  a un soldato, se gli dice: \textit{Rispiarma foderi}.

\item[IN salvummeffacche] Parole latine corrotte, e ridotte in una, usato assai dalla
  plebe ignorante per intendere Andare in salvo, ed è il Latino \textit{ad asylum confugere}.

\item[RILEVAR delle pacche] Buscare, o toccar delle ferite, che questo intendiamo
  \textit{pacche}, ma è detto plebeo. Il Vocabolista Bolognese dice che \textit{pacca} significa percossa
  gagliarda. La forza di questo verbo rilevare vedemmo sopra \cstan[3]{67}.
  Il Varchi stor. Fiorent. lib.6, dice \textit{Il figliuolo del quale nominato Lorenzo, rilevò una
    ferita}.

\item[HAVENDO veduto il Ciel turbato] Havendo conosciuto, che costui era in collora.
  Si dice anche \textit{la marina turba}.

\item[TACE che pare un porcellin grattato] Similitudine assai usata per intender uno,
  che non risponda alle grida d'un'altro o per paura, o per riverenza, o per la
  coscienza macchiata, o per altro; e si fa la comparazione al porco, perché il
  porco che stride, grattandolo si quieta, ed i porcai gli rendono maneggiabili col
  grattargli.

\item[CARACOLLA] Il verbo \textit{caracollare} vuol propriamente dire Volteggiare col
  cavallo, ma non ostante qui torna assai bene per esprimere, che costui per la
  paura andasse girando per il castello, non gli parendo trovare luogo sicuro. E
  però anche in uso \textit{caracollare} per camminare a piede, volteggiando d'una strada in
  un'altra, e diciamo \textit{far un caracollo} per intendere una girata. Viene dalla voce
  Spagnuola \textit{caracol}, che vuol dire \textit{chiocciola}.

\item[IL maggioringo della bolla] Termine della lingua furbesca, che in Firenze vuol
  il Fiscale;  ma s'intende per il Superiore in quegli affari di che si tratta. Vale,
  il Maggiore della Città, chiamata in quella lingua \textit{Bolla} dal Greco \textit{Polis} e
  barbaricamente \textit{Polla}.

\item[FARGLI mettere a' piedi la cipolla] Fargli troncar la testa, e mettergliela a
  piedi: come si costuma in Firenze quando, il cadavero del giustiziato dee stare
  esposto per qualche ora al pubblico; che gli mettono la testa a i piedi.

\item[È PER basire] È per transire, per svenirsi, per morirsi. Vedi sopra Cant. 2, stan. 79.

\item[MONA Cosoffiola] Nome usato per intender una donna faccendiera, affannona, o
  sudatora. Sebbene \textit{Cosoffiola} (secondo il Varchi nel suo Ercolano alla voce
  Battisoffiola) è lo stesso che battisoffiola, e significano affanno, paura, rimescolamento
  grande, ma breve, che cagioni battimento di cuore, o frequenza d'alito,
  il che si dice soffiare: Franco Sacc. Nov. 44. \textit{M'hai data così gran battisoffia, che
    io non sarò mai più lieto, e forse me ne morrò}. Non credo che sia lontano da questo,
  quello che diciamo \textit{soprassalto al cuore}; lo stesso che batticuore, affanno cagionato
  per paura, o dolore improvviso dagli Spagnuoli detto \textit{sobresalto}, nel Franz. \textit{sursaut},
  Corn, Tacito lib. 5. dice: \textit{Exterritae sunt acri magis quam diuturno timore}. Ed il nostro
  Davanzati parafrasando queste parole dice \textit{hebbero battisoffia}.
\end{description}
\section{Stanza IL. \& L.}

\begin{ottave}
\flagverse{49}Ella insieme le schiere ha già ridotte \\
Di genti, che non vagliono un pistacchio, \\
Cioè di quelle, a cui fece la notte \\
Col suo carro sì grande spauracchio, \\
Ed hor quivi parare, e dar le botte \\
Insegna lor, che non ne san biracchio \\
Ma quand'innanzi a lei costui, si ferma\\
Così tremante, la cavò di scherma.
\end{ottave}

\begin{ottave}
\flagverse{50}Mentre del fatto poi le dà contezza\\
Con quell'ambascia, e lingua di frullone\\
Fa (perché nulla mai si raccapezza)\\
Chi lo sente morir di passione;\\
Ma quella, c'a sentirlo è forse avvezza,\\
L'intende un po così per discrezione.\\
E qui finiscon le lezion di guerra,\\
Perch'ella non dà più ne in Ciel ne in terra.
\end{ottave}


Martinazza stava appunto instruendo quei soldati, che s'eran fuggiti per paura
de' suoi Caproni, quando arrivò un la sentinella con l'ambasciata di Calagrillo,
che la turbò tutta, ond'ella lasciò stare il dar lezione.
\begin{description}
\item[NON vagliono un pistacchio] Non son buoni a nulla. Si dice un pistacchio, un
lupino, una lisca; una sorba, una lappola, un pelo, un baiocco, un bagattino,
un picciolo, un zero, un'ette, un fico, cica, un iota, una chiarabaldana, un
puntal di stringha, o d'aghetto, una succiola, un soldo, un quattrino, un corno;
tutti per esprimer la poca stima, che si faccia d'uno, o d'alcuna cosa. E si
dice anche  non lo stimo il cavolo a merenda. Latino \textit{cicum}, \textit{titivillitium}.

\item[SPAVRACCHIO] Significa quel che accennammo sopra C. pr. stan. 40. E di
lì si dice fare spauracchio a uno per intendere spaventar uno, o mettergli paura
con fatti, o con parole.

\item[NON ne san biracchio] Non ne sanno nulla. Si dice anche straccio, brano, o
brandello, e simili.

\item[CAVARE un di scherma] Vuol dire far perder il filo del discorso a uno, ed è
  lo stesso che cavar di tema. Ma qui vuol dir' anche far lasciare star di schermire,
  e torna bene, perché Martinazza lasciò la scherma, ed uscì di tema, e di
  proposito per l'ira, che le cagionò l'ambasciata fattale in nome di Calagrillo.

\item[AMBASCIA] Affanno, o difficile respirazione d'alito, Fran. Sacc. N.139.
\textit{Tosto colui di chi erano stati, sen' andò con l'ambascia della morte a ripigliarli}.

\item[LINGUA di Frullone] Cioè che parla a salti, o a intoppi, come è il rumore,
  che fa il frullone, che è quell'ordingo, col quale, per via d'una ruota dentata, si
  separa la farina dalla crusca\footnote{il Frullone è un dispositivo meccanico per abburattare (da buratto ‘setaccio’) le farine. Attestato per la prima volta nel 1550 come “pulcherrimum instrumentum”, in un’opera di Girolamo Cardano, innovò il processo di macinazione dei grani e la panificazione.}.

\item[NON raccapezza nulla] Non intende nulla. Vedi sotto \cstan[6]{101}.

\item[L'INTENDE per discrezione] Quando per altro ci è noto un negozio, e che
  taluno ce lo racconti confusamente, o lo scriva con cattivi, e non intelligibili caratteri,
  sentito, o letto da noi, sogliamo dire; \textit{l'habbiamo inteso per discrezione},
  cioè habbiamo havuto la discrezione di non gli far ripetere il discorso, o di farlo
  di nuovo scrivere, già che per qualche informazione, che havevamo di quel fatto,
  intendevamo quel discorso, o scritto.

\item[NON dà ne in Ciel, ne in terra] È fuori di se. Non sa quel che ella si faccia.
  \textit{Non tocca ne ciel, ne terra}; dissero anche i Greci in questo proposito; e l'usa Luciano
  nel \textit{Pseudamante}, o vogliam dire \textit{Falso indovino}.
\end{description}
\section{Stanza LI --- LIII.}
\begin{ottave}
\flagverse{51}Tutta in un tempo vedesi cambiare\\
L'amante ingelosita Martinazza,\\
Hor ora è bianca, come il mio collare\\
Hor bigia, or gialla, or rossa, or paonazza.\\
Hor più rossa del \culo{} d'uno scolare\\
Dopo ch'egli ha toccata una spogliazza;\\
In somma ella ha in sul viso più colori\\
Ch'in una bottega non han cento Pittori.
\end{ottave}

\begin{ottave}
\flagverse{52}Rabbiosa, il capo versa il ciel tentenna,\\
Quasi col piede il pavimento sfonda,\\
Hor si gratta le chiappe, hor la cotenna,\\
Hor dice al messaggiero che risponda,\\
Hor lo richiama mentr'egli è in Chiarenna,\\
Grida, e minaccia, e par che si confonda,\\
Mille disegni entro al pensier racchiude\\
Ienne inne, e nulla mai conchiude.
\end{ottave}

\begin{ottave}
\flagverse{53}Il guardo al fine in terra havendo fisso\\
N'un vasto mare ondegga di pensieri,\\
E lagrime diluvia sopra il viso\\
Grosse come sonagli da sparvieri,\\
Che lavandole il collo lordo, e intrifo\\
Laghi formano in sen di pozzi neri;\\
Al fin tornata in se, con la gonnella\\
S'asciuga, e al messaggier così favella.
\end{ottave}


Narra gli accidenti, ed i moti diversi cagionati in Martinazza dall'ambasciata
di Calagrillo, ed in fine Martinazza s'accinge a dar la risposta. L'Autore
descrive Martinazza per una solenne sgualdrina poiché dice, che è così grande il
sudiciume che ella ha addosso, che le lagrime che le cascano dagli occhi fanno
parerle nel collo tanti laghi di pozzi neri, cioè di cessi, i quali ella s'asciuga
con la veste.

\begin{description}
\item[BIANCA come il mio collare] Diventa bianca comie un panno curato, E queste
  mutazioni di colore son proprie d'uno che habbia l'animo alterato sì in male,
  come in bene, perché la palidezza, e sbiancamento denota sollevamento d'animo
  non essendo altro, che un mancamento di sangue, il quale per la paura se
  ne fugge al cuore, e lascia le vene del volto; ed il rosso denota ira perché questa
  cagiona ribollimento di sangue intorno al cuore, che scorre per tutte le vene,
  ma apparisce più nella faccia, perché quivi sono molte vene intercutanee, o vogliamo
  dire in pelle, che facilmente lo scuoprono; ed lo stesso effetto viene parimente
  dalla vergogna, la quale però si dice anche erubescenza.

\item[DOPO ch' egli ha toccata una spogliazza] Dopo che egli è stato frustato in sul
\culo{} dal maestro. Spogliazza quali \textit{expoliatio}, spogliagione si dice quando il
Maestro fa cavare i calzoni a uno scolare, e mettendolo sopr'alle spalle d'un'altro,
gli dà con la sferza in sul \culo{}. E quando gli dà nella stessa forma, ma
senza i mandar giù i calzoni si dice dare una mula, o un cavallo. A questo
\culo{} frustato assomiglia l'Autore il viso di Martinazza quando le diventa rosso.
Una simile spogliazza, quasi come a ragazzo insolente, o minacciata là nel secondo
dell'Iliade a quel brutto mostaccio di Tersite, a cui Omero (secondo la
traduzione Latina ad verbum del Gifanio) fa dire da Ulisse: \textit{Ne posthac Ulyssi
  caput humeris adsit, \&c. Si non ego te comprehensum, \& charis vestibus exutum Pallioque,
  \& tunica, quae pudenda contegunt, Flentem veloces ad naves dimisero, Caedens e
  concione duris verberibus}\footnote{esistono differenti traduzioni dello stesso passo.}.

\item[TENTENNA il capo inverso il Cielo] Dimena la testa verlo il Cielo. Atto che
  si fa da molti quando accade loro cosa di poco gusto, quasi vogliano minacciare
  il Cielo perché cagiona loro quella tal disgrazia: i Latini dissero; \textit{caput quatere}.

\item[SFONDA il pavimento col piede] Per la collora batte i piedi in terra così fortemente,
  che fa quasi rovinare il Palco. Properzio. \textit{Et crepitum dubio suscitet ira pede}.

\item[SI gratta le chiappe, e la cotenna] Si gratta le natiche, il capo, che è un atto
  solito farsi per lo più dalle donne quando succede loro qualche disgrazia. Per \textit{cotenna}
  s'intende il capo, perché la pelle del capo dell'huomo si dice cotenna; se ben
  vuol dire la pelle del porco, ed impropriamente si dice la pelle d'ogni animale
  vedi sopra \cstan[2]{64}. ed in ciò noi ci conformiamo co' Latini, che dicono \textit{cutis}
  la pelle del capo dell'huomo, e dicono anche \textit{cutem detrahere} per scorticare qualsivoglia
  pelle, il proprio vocabolo della quale è \textit{pellis}.

\item[QLVAND' egli e in chiarenna] Quand' egli è molto lontano. \textit{In oras longinquas},
  e da questo noi diciamo: \textit{Quand'egli e in orinci}. Usato dal Davanzati nel Tacito.

\item[IENNE inne] Di questo termine ci serviamo per esprimere uno che s'affanni
  di operare, e non conchiuda. Viene da quello stento che fanno i ragazzi quando
  imparano a compitare; quasi dica compita compita, e mai rileva, ed ha lo stesso
  significato, e forza che \textit{ponza ponza} detto sopra \cstan[4]{80}.

\item[SONAGLI da sparvieri] Intende lagrime grosse come sono i sonagli, che s'appiccano
  a i piedi degli sparvieri; comparazione iperbolica, ma assai usata per intender
  grosse lagrime. AEn.11. \textit{It lacrymans guttis. humectat grandibus ora}.
  \textit{Sonagli}, e \textit{campanelli} chiamiamo quelle gallozzole, che fa l'acqua quando piove, cadendo sopra
  i rigagnogli; o altrimenti nello scorrere.

\item[POZZI NERI] Bottini. Quei luoghi sorterranei, entro a' quali si getta ogni
  sorta d'immondizia; ma propriamente \textit{pozzo nero} è bottino, o fogna smaltitoia
  del cesso, a differenza di quella degli acquai.
\end{description}
\section{STANZA LIV --- LVI. }
\begin{ottave}
\flagverse{54}Torna, e rispondi a questo Scalzagatto, \\
Che si crede ingoiar con le parole: \\
Ch'io non so quel ch'ei dica, e s'egli è matto \\
Non ci posso far' altro, e me ne duole, \\
Poi circa alla domanda, ch' egli ha fatto; \\
Che gli darò Cupido, e ciò ch'e' vuole, \\
Se con la spada in mano, o ver con l'asta\\
Prima di guadagnarlo, il cor gli basta.
\end{ottave}

\begin{ottave}
\flagverse{55}Però s'in questo mentre umor non varia,\\
Domani al far del dì facciami motto,\\
E s'io gli farò dar le gambe all'aria,\\
Quella sua landra ha da pagar lo scotto,\\
Mia se la sorte fosse a me contraria\\
Vuol c'a me tocchi andar col capo rotto,\\
Prenda Cupido allor, ch'io le prometto\\
Lasciarglielo segnato, e benedetto.
\end{ottave}

\begin{ottave}
\flagverse{56}Ciò detto parte, e quei ch'era huomo esperto \\
(Essendo stato Cavallaro, e Messo) \\
Al Cavaliere ad unguem fa il referto \\
Di quel che Martinazza gli ha commesso;\\
Ed in viso vedendolo scoperto,\\
Quest'ha bisogno dice d'un buon letto,\\
Perch'egli è duro, e non punto pupillo,\\
Lo conosco bensì, gli è Calagrillo.
\end{ottave}


Martinazza manda a dire a Calagrillo, che gli dara Cupido, s'ei lo guadagnerà
con l'armi; ma se ella vince, vuol Psiche: la ronda porta l'ambasciata, e
riconosce Calagrillo.

\begin{description}
\item[SCALZAGATTO] Huomo vile, Guidone.

\item[CREDE ingoiar con le parole] Crede farci paura con le chiacchiere. E si dice:
  \textit{Mangiar vivo uno con le parole}.

\item[S'IN questo mentre umor non varia] Se fra tanto non si muta d'opinione.

\item[LANDRA] Sgualdrina, donna di bordello, ed intende Psiche; Landra è epiteto
  conveniente alle più infami, e laide meretrici, quasi \textit{latrina}, che la fogna,
  e ricettacolo di tutte le schiferie.

\item[HA da pagar lo scotto] Ha da pagare la pena. Pagar lo scotto vuol dire pagar
  all'oste quello, che s'è mangiato, pagar la sua porzione, la sua quota; Terenzio
  disse \textit{symbolam dare}. Ma qui intende il Latino \textit{poenas luere}, Dan. Purg. C.~30.
  \begin{verse}
    \backspace L'alto fato di Dio sarebbe rotto
    Se Lete si passasse, e tal vivanda
    Fusse gustata senz'alcuno scotto
    \backspace Di pentimento, che lagrime spanda.
  \end{verse}
\item[ANDAR col capo rotto] Andar con la peggio; cioè ch'io perdessi il duello.

\item[SEGNATO, e benedetto] Liberamente, e senz'eccezione alcuna. Fran.Sacc. Nov.
  104. \textit{Vattene ogni hora pur Segnato, e benedetto}. Esprime un dar via qualcosa, o
  uno volentieri, e con anime di non rivolerlo; Un licenziare affatto.
Virc. Egl. \textit{longumque vale, vale, inquit Iola}.

\item[CAVALLARO] È un famiglio, che porta le citazioni criminali mandate da
  i Ministri forensi, chiamato \textit{Cavallaro}, perché stante il largo dominio, e giurisdizione,
  che ha il suo tribunale, e necessario che vada a cavallo; \textit{Il Messo} è quello
  che porta le citazioni civili pure de i Ministri forensi, e fa i gravamenti, ec. e
  non va a cavallo, perché non gli occorrono lunghe gite, come al Cavallaro; a
  Roma si domanda \textit{Cursore}; nome simile al \textit{Viator}, col quale era disegnato dagli antichi
  Romani il donzello, o fante pubblico.

\item[AD unguem] Per appunto\footnote{Esattamente, precisamente, magari ``spaccando il capello''.}. Frase latina usata assai da noi.

\item[FA il referto] Riferisce. Frase curiale, che vuol dire quando il Cavallaro, o
  Messo havendo data la citazione, riferisce in atti d'haverla data, che dicono anche
  \textit{fare il rapporto}. E l'Autore si serve di questa frase (per altro non usata in
  questi termini) perché ha detto, che questa Guardia era stato Cavallaro, e Messo.

\item[EGLI ha bisogno d'un buon lesso] E' carne dura, e però ha bisogno di bollire
  assai nell'acqua. È detto vulgato per esprimere un'huomo, che sa il conto suo,
  forte, gagliardo, e difficile a superarsi, che diciamo: \textit{Osso duro} per esempio; Il
  tale ha tolto a rodere un'osso duro.

\item[NON è pupillo] Non ha bisogno di Tutori, suona lo stesso che \textit{ha bisogno a un
  buon lesso}, se bene \textit{non è pupillo} si ristringe a saper fare i fatti suoi, ed \textit{ha bisogno
  d'un buon lesso} esprime saper fare i fatti suoi, ed esser bravo, e valente in ogni
  cosa.
\end{description}

\section{STANZA LVII. --- LXII.}
\begin{ottave}
\flagverse{57}Ma qui la dama, e Calagrille resti,\\
Quest'altro giorno rivedremogli poi. \\
Il passo meco hora ciascuno appresti \\
Per giunger il Fendesi, e gli altri duoi, \\
Che seguitaron come voi intendesti\\
Perlon, che sen'andò pe' fatti suoi,\\
Che troveremgli, se venir volete\\
Più presto assai di quel che vi credete.
\end{ottave}

\begin{ottave}
\flagverse{58}Che giò giò se ne vanno giù pel piano \\
Sbattuti com'io dissi dalla fame; \\
Ma non son iti ancora un trar di mano \\
Che senton razzolar fra certo strame; \\
Perciò con l'armi subito alla mano \\
Corron dicendo: Qui c'è del bestiame, \\
Sì che quando crediamo di tirar minze, \\
Il corpo forse caverem di grinze.
\end{ottave}

\begin{ottave}
\flagverse{59}Cursosi quei che fusse di vedere \\
Dentr'a una stalla inabitata entraro, \\
E vedder, ch'era un'huom posto a giacere \\
Sopr'alla paglia a guisa di somaro; \\
Accanto havea da mangiare, e bere, \\
E gli occhi distiliava in pianto amaro, \\
E tra i disgusti, e il vin ch'era squisito \\
Pareva in viso un gambero arrostico.
\end{ottave}

\begin{ottave}
\flagverse{60}Questo è il Piaccianteo già subblimato\\
Al grado honoratissimo di spia,\\
Quel che per soddisfar tanto al palato\\
Ha fatto in quattro dì Fillide mia,\\
E lì con la sua spada s'è impiattato,\\
Dell'honor della quale ha gelosia,\\
Che havendola fanciulla mantenuta\\
Non gli par ben ch'ignuda sia tenuta.
\end{ottave}

\begin{ottave}
\flagverse{61}Ma perché un huom più vil mai fe natura,\\
Si pente esser'entrato in tal capanna,\\
Però che a starvi solo egli ha paura,\\
Che non lo porti via la Trentancanna,\\
E perché tutto il giorno quant'ei dura,\\
Egli ha il mal della lupa, che lo scanna,\\
Non va mai fuor s'a cintola non porta\\
L'asciolver col suo fiasco nella sporta.
\end{ottave}

\begin{ottave}
\flagverse{62}Ovungne egli è, d'untumi fa un bagordo, \\
Ch'ognor la gola gli fa lappe lappe; \\
Strega le botti di lor sangue ingordo, \\
E le sustanze usurpa delle pappe; \\
Aggira il beccafico, e pela il tordo,\\
E a poveri cappon ruba le cappe,\\
E prega il Ciel, che faccia che gli agnelli\\
Quanti le melagrane, habbian granelli.
\end{ottave}


L'Autore torna a parlare di Perlone, e degli altri, che lasciò sopra C.4. stan.28.,
i quali per la fame s'andavano ailontanando dal Campo, e narra, che costoro
trovarono in una Capanna quel Piaccianteo, che fu da Bertinella mandato
fuori a spiare, come vedemmo sopra \cstan[3]{45}. il quale haveva seco da mangiare,
e da bere. Nella presente Ottava 62. descrive assai vagamente la ghiottornia
di Piaccianteo.

\begin{description}
\item[GIÒ giò] Adagio adagio. È la figura \textit{aphaeresis}.
\item[RAZZOLARE] Fregare, raspare, fragare; ec. Qui vuol dir quel romore
  che fa la paglia, o cosa simile, quando è maneggiata in massa.

\item[STRAME] Paglia, fieno, o altra materia simile per cibo delle bestie. Vedi
  sopra \cstan[4]{2}.

\item[TIRAR minze] Vuol dite stentare. Ma s'intende morire: Si dice milza, ma
  il Poeta si serve della licenza, e seguita intanto i più che dicono; \textit{minza} e non
  \textit{milza}.

\item[CAVARE il corpo di grinze] Mangiare assai, che in questa maniera gonfiando
  il ventre, si levano le grize al corpo. Plauto disse \textit{ventrem distendere}. Virg.
  Georg. \textit{distendunt nectare cellas}, cioè empiono.

\item[PAREVA un gambero arrostito] Era rosso in viso come sono i gamberi fritti:
  similitudine assai usata per esprimere un rosso in viso, per il soverchio vino
  bevuto.

\item[HA fatto Fillide mia] Ha finito, ha consumato, o mandato male tutto il suo
  havere. E' detto ianadattico \textit{Filide} per fine, Ma per avventura ha la sua origine
  da Fillide figliuola di Licurgo Re de i Traci, la quale s'innamorò di Demofonte
  figliuolo di Teseo, e di Fedra, quando nel tornare dalla guerra di Persia
  essendo stato spinto da i venti contrarj nel Regno di Tracia, fu da Fillide ricevuto
  con segni di grande amorevolezza; ma egli senza riguardo a i benefizzi da
  essa ricevuti, sen'andò; per lo che Fillide disperata s'impiccò. Da questa disperata
  morte di Fillide, quando diciamo \textit{far Fillide}, intendiamo finir la vita, e
  finire la roba.

\item[IMPIATTATO] Nascosto, Vedi sopra \cstan[2]{60}.

\item[DELL'honor della quale ha gelosia] Ha gelosia dell'honor della sua spada, perché
  havendola tenuta sempre fanciulla, cioè vergine (che s'intende non mai
  adperata) stima poco honesto il lasciarla vedere ignuda, come è veramente poco
  onesto a una vergine lasciarsi vedere ignuda. E con tali scherzi vuol dire, che
  costui era codardo, e vile, e di poco animo, ed uno di coloro che \textit{umbram
    suam metuunt}.

\item[TRENTANCANNA] Una bestia ch'ingoia o tracanna trenta per volta;
  ed è una di quelle larve immaginarie inventate dalle Balie per far paura a i bambini,
  come bau, befana, e simili dette altrove.

\item[IL male della Lupa] È inteso da noi per una infermità, che fa stare il paziente
  in continua fame, ed i Medici la chiamano \textit{fame canina}.

\item[CHE lo scanna] È un termine che significa grandezza di passione, ed ha forza
  d'avanzare ll superlativo, perché dicendosi, \textit{Ha una fame, una sete, un desiderio,
    ec. che lo scanna}, s'intende fame, sete, o desiderio grandissimo, e più, Vedi sopra C.4. stan.24.

\item[ASCIOLVERE] Solvere il digiuno; sdigiunarsi, fare colazione. Vedi sopra
  C.1. stan. 35. ma qui è preso per mangiamento in generale, cioè per la materia
  da mangiare.

\item[UNTUMI] Intende roba da mangiare, che sia unta, come polli, carne,
  pesce, ec.

\item[BAGORDO] Bagordare, o far bagordo vuol dir Giostrare, giuocar d'armi,
  far conviti, ed ogni altra sorta d'adunanza festiva, ancorché non d'armi. E
  potrebbe dirsi scherzando bagordo, quasi \textit{vagus ordo}, confusione ordinata; onde
  da quel numero di gente in confuso, la quale interuiene a tali bagordi, pigliamo
  poi \textit{bagordo} per commistione di varie cose, come nel presente luogo, che intende
  mescolanza d'untumi. Vedi sotto \cstan[6]{2}. Del resto \textit{Bagordo} viene da \textit{Bigordo},
  che vuol dire \textit{Asta}. E Bigordare trovasi presso gli antichi; per correr la
  lancia, Fazio degli Uberti nel Dittamondo al Canto 32,
  \begin{verse}
    \backspace Giovani bigordare alli chintani,
    E gran tornei, e una, e altra Giostra
    Farsi veder con giuochi nuovi, e strani.
  \end{verse}
  Poi si disse \textit{Bagordo}, e \textit{Bagordare}; e si trassero queste voci a significare ogni sorta
  di stravizio, e di ricreazione. Che Bigordo voglia dire \textit{Asta}, ci è l'esempio di
  Giovanni Villani lib. 7. rubric. 132. \textit{E recossi palio di drappo ad oro sopra capo
    Messer Amerigo di Nerbona portato sopra bigordi per più Cavalieri}. Folgore
  da San Gimignano\footnote{Folgóre da San Gimignano, pseudonimo di Giacomo di Michele o Jacopo di Michele secondo fonti diverse (San Gimignano, 1270 – San Gimignano, 1332), poeta comico-realistica, uno stile basso con tematiche laiche è mondane. } Rimatore antico citato dai Conte Ubaldini nelle Annotazioni a
  Messer Francesco da Barberino: \textit{E rompere, e ficcar bigardi, e lance}.

\item[LA gola gli fa lappe lappe] Signitica desiderar ardentemente di mangiare. Voci
  nate dal suono che fa il palato con la lingua, e con le labbra quand'uno biascia senza
  havere nulla in bocca, che è segno di fame, qual suono pare che dica lappe
  lappe; donde poi il verbo allampare, che vuol dire haver gran fame. Così \textit{Lapto}
  in Greco, che è lo stesso, che \textit{Lambo} in Latino, è fatto dal medesimo suono.

\item[STREGA le botti] Stregare vuol dir succiare il sangue, perché dicono, che le
  Streghe succiano il sangue a i bambini; e però dicendo \textit{strega de botti} intende succia
  il sangue delle botti, che è il vino, del quale è \textit{ingordo}, cioè avidissimo.

\item[VSVRPA le sustanze dele pappe] Divora la carne, che è la sostanza del brodo,
  del quale si fanno le pappe.

\item[AGGIRA il beccafico, e pela il tordo] Aggirare, e pelare, metaforicamente parlando,
  significa ingannar'uno, e cavargli da dosso danari, come habbiamo accennato
  sopra in \cstan{9}. Il Poeta scherzando piglia detti due verbi nel lor
  vero senso, ed intende girar nello spiede i beccafichi, e pelare i tordi per quocergli,
  e mangiarsegli.

\item[LEVA le cappe ai capponi] Cioè divora la pelle de' capponi.

\item[E PREGA il Ciel che faccia, che gli agnelli, ec.] \- Dove giù agnelli hanno solamente
  due granelli, (cioè testicoli) vorrebbe, che ne havessero quanti n'hanno le
  melagrane. E così descrive un solenne ghiotto; e crapulone. Similmente un certo
  Filofleno solenne mangiatore, siccome riferisce Aristotile lib.\ 3.\ delle Morali
  indirizzate a Nicomaco, cap.\ 10.\ desiderava d'avere il collo più lungo d'una
  grue supponendo, che così fusse per essere il gusto maggiore.
\end{description}
\section{STANZA LXIII. --- LXVI.}
\begin{ottave}
\flagverse{63}Vedenda quivi comparir repente \\
L'infolite armi, sbigottisce il ghiotto,\\
E dal timor ch'egli ha di tanta gente \\
Trema da capo a pié, si piscia sotto: \\
Con tutto ciò digruma allegramente, \\
E spesso spesso bacia il suo barlotto, \\
E acciò stremata non gli sia la vita \\
Non dice men: degnate, o a ber gli invita.
\end{ottave}

\begin{ottave}
\flagverse{64}Ma i Cavalier famosi a quel plebeo,\\
Che non profferì lor della rovella,\\
Furon per insegnare il Galateo\\
Con battergli già in terra una mascella, \\
Chi sei? (diss' un di loro) e Piaccianteo, \\
Ch'è un pover huom, risponde, e in quella Cella \\
Molt' anni in astinenza ha consumati \\
Per penitenza de' suoi gran peccatei.
\end{ottave}

\begin{ottave}
\flagverse{65}E quei soggiunge: Mi rallegro, e godo\\
Che voi facciate bene, e vi son schiavo;\\
Ma s'il patire è fatto a questo modo,\\
Penitente di voi non è più bravo,\\
Tal ch'io per me vi mando a corpo sodo\\
Non nel settimo Ciel, ma nell'ottavo,\\
Donde ai mondani, e a me che sono il capo,\\
Pisciar potrete a vostra posta in capo.
\end{ottave}

\begin{ottave}
\flagverse{66}Ma perch al certo Vostra Reverenza,\\
Ch'è stenuata, come un Carnovale,\\
Havrà fatta fin'hor tant' astinenza,\\
Che basti a soddisfar a ogni gran male,\\
Hor puo lasciar a noi tal penitenza,\\
Acciò baciam la terra del boccale,\\
Per più mondi accostarsi a quest avanzi\\
Delle reliquie, ch'ell'ha qui dinanzi.
\end{ottave}


Piaccianteo vedendo comparir coloro armati, hebb'un grande spavento, ma
non per questo abbandono ii mangiare, anzi si studiava più per il timore, che
haveva, che coloro non gli stremassero la provvisione. Domandato poi, chi egli
era, rispose esser uno, che faceva penitenza de' suoi peccati in quella cella con digiuni,
e astinenze: Dalla qual risposta accortisi, che egli era un birbone, uno di loro
scherzando sopr'al digiunare, gli dice, che lasci un po fare il medesimo digiuno,
ed astinenza ancora a loro.
\begin{description}
\item[SBIGOTTISCE] Spaurisce. Si perde d'animo. Vedi sopra C.2. stan.28. Dan. Inf.C.22.
  \begin{verse}
    Così mi fece sbigottir lo Mastro,
    Quand'i gli vidi sì turbar la fronte.
  \end{verse}

\item[GHIOTTO] Goloso; Avido di mangiar del buono. Lat. \textit{gluto}.
\item[SI piscia sotto] Vuol dire haver gran paura. Vedi sopra in \cstan{3}.
\item[DIGRUMARE] Intendi mangiare; se bene digrumare è il masticare, che fanno
  le bestie dal pié fesso, che si dice anche ruminare dal Latino, che però chiama
  \textit{ruminantia} le dette bestie, come habbiamo accennato sopra \cstan[4]{6}.5 e vedremo
  sotto, \cstan[6]{5}.
\item[BACIA il barlotto] Beve. Barlotto è un vaso di legno di figura simile al barile,
  ma è assai minore, perché sarà di tenuta o più, o meno fino a dieci fiaschi, che tenendo
  dieci fiaschi si chiama mezzo barile. Qui pero non intende strettamente questa specie
  di barlotto, ma un vaso da vino portatile addosso, comunque si sia o di vetro, o di
  terra, o una Zucca, anzi stimo che intenda più tosto di terra, perché più giù
  dice \textit{baciamo la terra del boccale}.
\item[STREMARE] Vale scemare, sminuire, quasi ridurre allo stremo.
\item[DEGNATE] È un modo di dire usato da coloro che mangiano all'osteria,
  quando passa intorno alla loro tavola alcun loro conoscente, e dicono: \textit{degnate},
  cioè degnatevi di bere. E perché è termine usatissimo dalla plebe, il Poeta fa
  che costoro si maraviglino, che Piaccianteo non l'usi, e fa prendere argumento,
  che egli non l'usi per paura, che non sia accettato l'invito, e scematagli la
  provvisione.
\item[CAVALIERi famosi] Cavalieri illustri, e di fama. Ma qui \textit{famoso} non deriva
  da fama, ma allude a fame, e vuol dir Cavalieri affamati.
\item[PLEBEO] Vuol dire huomo di Plebe; ma ce ne serviamo anche per intendere
  huomo infame, senza honore, e senza creanza. Qui se ne serve per contrapposto
  di Cavalieri famosi, e vuol dire, che si come quelli erano famosi, cioè affamati,
  costui era infame, cioè senza fame, perché havea ben mangiato.
\item[NON profferì della rovella] Non offeri nulla; usandosi spesso il verbo \textit{profferire},
  In vece del verbo \textit{offerire}; e la parola \textit{della rovella} è posta a maggior' emfasi per
  esprimere non offerì nulla, ne meno una cosa nociva.

\item[INSEGNARE il Gatateo] Insegnare le creanze, i buoni termini. Galateo è intitolata
  un' Operetta di Monsignor Gio. della Casa, la quale insegna le buone
  creanze.

\item[BATTERGLI giù una mascella] Dargli un taglio nel viso, e fargli cadere una
  ganascia.

\item[IO vi son schiavo] Vi son servitore. E' un detto usato, quando alcuno faccia,
  bella azione, che meriti lode, per esempio Il tale fece una bellissima Orazione;
  io gli son schiavo. I Caporali nella vita di Mecenate dice,
  \begin{verse}
    E si legge ch'Augusto un dì gli disse:
    Capitan Mecenate io vi son schiavo.
  \end{verse}

\item[NELL'ottave Ciclo] L'Autore tenendo l'opinione, che i Cieli sieno otto dice,
  che costui merita d'andare nell'ottavo, cioè nel supremo; perché ha fatta tanta
  penitenza, che merita il sovrano posto nel Cielo.

\item[MONDANI] Intende peccatori. Coloro che sono dediti a i piaceri mondani.
\item[STENVATO come un Carnovale] Magro, come un Carnovale: comparazione
  ironica, che vuol dire Grassissimo, come si figura il Carnevale.
\item[BACIAMO la terra dei boccale] Baciar la terra è un'atto, che si fa dalle persone
  divote per umiltà, Ma costui sostenendo l'equivoco del far penitenza, dopo
  haver detto, che gli piace il modo del digiunare, che fa Piaccianteo, dice che
  vuol ancor'egli far'un'atto d'uiilta con baciar la terra, ma però quella del
  boccale, cioè bere. \textit{Boccale} è un vaso di terra capace della meta d'un fiasco, ma
  si piglia per tutti li vasi di terra a quella foggia, ancorché maggiori, e di tenuta
  di un fiasco anche più.

\item[PER accostarsi più mondi] Per accostarsi più puri, havenod fatto l'atto di penitenza,
  e d'umiltà con baciar la terra.

\item[RELIQVIE] Avanzi, fragmenti; e scherzando sempre con la bontà, e perfezione
  del penitente, par che pigli \textit{reliquie} nel senfo speciale, che l'intendiamo
  noi, cioè ossa, ed altri fragmenti di Santi, ed ei vuol poi dire gli avanzi del di
  lui mangiamento. Latino \textit{mensae relique}. Ed in quest'ottava l'equivoco è sostenuto
  da costui in mostrare a Piaccianteo di credere, che egli fusse un penitente,
  che stesse quivi per fare astinenza, come haveva detto; e per indurlo a contentarsi,
  che essi ancora s'accomodino con lui a far la penitenza nella stessa maniera,
  che faceva egli.
\end{description}

\section{STANZA LXVII. --- LXVIII.}
\begin{ottave}
\flagverse{67}Qual madre, che ripara il suo figliuolo, \\
Ch'è sopraggiunta da mordaci cani, \\
Ei cuopre tutto con il ferraiuolo,\\
Ed eglino gli danno in su le mani; \\
E col lazo del Piccaro Spagnuolo, \\
Che dalla mensa vuol tutti lontani, \\
Acciò poi a tal cose non arrivi, \\
Con due calci lo fan levar di quivi.
\end{ottave}

\begin{ottave}
\flagverse{68}Così fan carità di più rigaglie\\
Oltr' ad un'Oca grossa arciraggiunta;\\
Ma vedendo più là fra quelle paglie\\
D'un pezzo d'arme luccicar la punta,\\
E del giaco scappare alcune maglie\\
Da quella sua casacca unta, e bisunta,\\
Insospettiron, com'un'altra volta\\
Potrà sentir chi volentier m'ascolta.
\end{ottave}


Piaccianteo vedendo, che costoro s'accostavano per torgli la roba, cerca di
salvarla, coprendola col ferraiolo, ma essi con una mano di calci l'allontanarono,
e d'accordo si messero a mangiare: Ma intanto, osseruato, che egli era armato,
presero sospetto, e fecero quello, che sentiremo sotto nel \cstan[8]{60}.

\begin{description}
\item[RIPARARE] Rimediare. Val per difendere. Ed in questa comparazione
  imita Dante Infer. C. 23. che dice:
  \begin{verse}
    Come la madre, ch' al romore è desta,
    E vedo preso a se le fiamme accese,
    \backspace Che prende il figlio, e fugge, e non s'arresta,
    Havendo più di lui, che di se cura;
    Tanto che solo una camicia vesta.
  \end{verse}

\item[FERRAIVOLO] Mantello. Un panno ridotto tondo, e adattato a coprire
  tutta la persona sopra agli altri abiti, mettendolo in su le spalle.

\item[LAZO del Piccaro Spagnuolo] Gli zingari, quando s'abbattono nel corrivo;
  per truffarlo, e rubargli qualcosa, che gli habbiano veduta, trovano diverse invenzioni,
  come di farlo ballare, o cantar con loro, o fargli mettere in capo
  qualche ordingo, che gli occupi la vista, o con fargli metter il capo in un'armario
  a vedere il Mondo nuovo, e molt'altre invenzioni per distrarlo, ed haver
  comodità di rubargli quel che hanno disegnato, mentr'egli astratto da tali operazioni
  non bada a quel che gli facciano d'attorno; come spesso veggiamo seguire
  in commedia, che il servo astuto, per truffare il servo stolto si vale di simili
  astuzie. E questo si dice \textit{il lazo del Piccaro Spagnuolo}, cioè invenzione dello Spagnuolo
  furbo.  Donde poi \textit{lazo}, \textit{lazeggiare} significa qualunque azione, che facciano
  i Comici per esprimere il lor pensiero. E \textit{lazo}, che in Spagnuolo significa
  \textit{laccio}, si prende da noi per quel che i Latini direbbero \textit{captio}, \textit{sophisma}, \textit{commentum},
  \textit{technae}, \textit{versuria}, \textit{fallacia}, \textit{artes}, \textit{doli}, Ed in questo significato va profferito con
  la \letter{z} dolce, e non cruda, ed aspra, perché con la cruda significa sapore aspro,
  ed astringente, come quel della prugna, della sorba mal matura, e simili, che i
  medici dicono \textit{acido}; Dante Inf, C. 15,
  \begin{verse}
    Ed è ragion, che là tra i lazzi sorbi
    Si disconvien fruttare il dolce fico
  \end{verse}
  La lazzeruola\footnote{\textit{Malpighia emarginata} DC., pianta arbustiva tropicale, originaria del Nuovo Mondo. Non ha molta diffusione in Italia, si chiama da noi con i suoi nomi spagnoli di \textit{acerola}, \textit{manzanita}, \textit{semeruco}, o anche \textit{ciliegia delle Barbados}.
  }, perché è frutta di sapore, \textit{lazzo}, cioè \textit{acido} dicesi da gli Spagnuoli
  \textit{azerola} quasi dal Lat. diminutivo \textit{acidula}.

\item[FAR carità] Fra i Bacchettoni s'intende mangiare insieme. E tra gli antichi
  Cristiani, i conviti, che si facevano a' Poveri; di limofine, si domandavano \textit{Agapae},
  cioè \textit{Caritadi}. E \textit{Pietanza}, voce conservatasi tra' Frati, e tra le Monache,
  significa piatto, o mangiare offerto dalla pietè, e carità de' benefattori; non
  significando altro \textit{Pietanza}, che \textit{Pietà}. Il Beato Fra Iacopone: \textit{Vorria trovar
    alcuno, Che avesse pietanza De lo mio cor afflitto}.
\item[ARCI raggiunta] \- Grassissima. Uccello soprammodo grasso si dice raggiunto.
\item[LUCCICARE] Risplendere; Rilucere. Viene da Lucciola.
\item[CASACCA] Parte d'abito da huomo, che copre la persona da mezza la pancia
  in su fino al collo. Così \textit{Casula} in Latino; se bene altra sorta di veste, diversa
  dalla Casacca, fu detta così, perché copre tutta la persona a guisa, che fa la
  casa, se crediamo a Isidoro nel lib. 19. delli Origini, al cap. 24.
\end{description}
\section*{FINE DEL QVINTO CANTARE.}

\chapter{Sesto Cantare}
\begin{argomento}
Nel tenebroso centro della Terra,
Ove regna Plutone entra la Strega,
E vuol che seco per finir la guerra
Di Malmantile entri l'Inferno in lega.
Fanno concilio i mostri di sotterra,
Ove ciascun buone ragioni allega;
Certa al fin le promette l'assistenza,
Rend' ella grazie, e fa di lì partenza.
\end{argomento}

\section{Stanza I --- III}

\begin{ottave}
\flagverse{1}Miser chi mal' oprando si confida:\\
Far' alla peggio, e ch'ella ben gli vada,\\
Perché chi piglia il vizio per sua guida,\\
Va contrappelo alla diritta strada.\\
E benché qualche tempo ei sguazzi, e rida\\
Col vento in poppa in quel che più gli aggrada,\\
E' vien poi l'ora, ch'ei n'ha a render conto,\\
E far del tutto, dondola, ch' io sconto.
\end{ottave}

\begin{ottave}
\flagverse{2}Di chi credi Lettor tu qui ch'io tratti?\\
Tratto di Martinazza iniqua Strega,\\
C'ha più peccati, che non è de' fatti,\\
E pel Demonio ogni ben far rinnega;\\
Di darsi a lui già seco ha fatto i patti,\\
Acciò ne' suoi bagordi la protega,\\
Ma state pur; perché tard, e per tempo\\
Lo sconterà; da ultim' è buon tempos.
\end{ottave}

\begin{ottave}
\flagverse{3}Non si pensi d'haverne a uscir netta;\\
S'inrighi pur col Diavol, ch'io le dico,\\
Se forse haver da lui gran cose aspetta,\\
Che nulla dar le può, ch'egli è mendico,\\
E quand' ei possa, non se lo prometta,\\
Perch'ei, che sempre fu nostro nimico,\\
We può di ben verun vederci ricchi,\\
Vna fune daralle, che l'impicchi.
\end{ottave}

Il Poeta havendo pensiero di narrar la gita, che fece Martinazza al Regno di
Plutone per muoverio ad aiutarlo a diloggiar Baldone da Malmantile, ed a
gastigare Gambattorta, e Baconero, fa l'introduzione al presente Cantare con
una riflessione morale ponderando, che quei, che opera male, non può sperare
d'haver mai bene, e principiando come l'Ariosto C. 6.
\begin{verse}
  Miser chi mal' oprando si confida
\end{verse}
Conchiude, che Martinazza, la quale non fa se non sciagurataggini, e s'è data
al Diavolo, non può sperar d'haver a haver bene, perché il Diavolo è nimico
del genere umano, e non può vedergli ben veruno.

\begin{description}
\item[FAR alla peggio] Far' ogni male senza riguardo alcuno.
\item[VA contrappelo] Non va per il verso buono. Va al contrario di quello, che
  deve fare per andar per la diritta via. Sen. epist. 122. \textit{Omnia vitia contra naturam
    pugnant, omnia debitum ordinem deferunt; hoc est luxuriae propositum gaudere perversis:
    nec tantum discedere a recto, sed quam longissime abire; deinde etiam e contrario stare}.
  Si dice anche andare a ritroso dal Latino \textit{retrorsum}. Dan. Purg. C. 10, in simil
  proposito dice.
  \begin{verse}
    O superbi Cristian miseri, e lassi,
    Che della vista della mente infermi
    Fidanza havete ne i ritrosi passi.
  \end{verse}
 E la metafora d'andar contrappelo è tolta da i pezzi di panno, o di pelle pelosa,
 che in cucirle insieme s'osserva, che il pelo vada tutto per un verlo, acciocché
 non si confacciano. A tastar un panno, o pelle pelosa per il verso, che va il
 pelo, torna più facile, e non si trova resistenza alcuna, come a andar contro a
 pelo.

\item[SGUAZZI] Goda allegramente.

\item[COL vento in poppa] Secondo che ei desidera: Come succede quando si ha il
  vento in poppa della nave: e significa \textit{i negozzj vanno bene}. I Greci pure dissero
  \textit{Secundo vento navigare}.

\item[DONDOLA ch'io sconto] Vuol dire sconterà il buon tempo, che ella si è data,
  provando altrettanti disgusti, E' detto usato dalla Plebe, nella quale e nato; essendo
  lato detto da un macellaro, a cui era stata rubata in più volte gran quantità
  di carne, ed essendo fatto ritrovato il ladro, fu impiccato, ed il macellaro
  vedutolo appeso alle forche disse: \textit{Dondola, ch'io sconto}; intendendo a vederti
  dondolare Sconto il debito, che hai meco per la carne rubatami. Dondolare, è
  lo stesso che ciondolare, come appunto fa l'impiccato; e tal Verbo \textit{dondolare}
  piglia il nome da quel don don, che fa il suono delle Campane. E da quello medesimo
  suono, che faceva quel tanto rinomato vaso dell'Oracolo di Giove, che
  era in Dodona Città dell'Epiro, stima, e con molta ragione, derivarsi il nome
  di Dodona Abramo Berkelio Olandese\footnote{Abraham van Berkel, 1639-1686, figura poco nota dell'illuminismo radicale olandese, contemporaneo del Lippi e Minucci.} nelle Osservazioni al Frammento dell'Opera
  originale di Stefano de Urbibus\footnote{Stefano di Bisanzio, di lui si ignorano i dati biografici, comunque con tutta probabilità era un grammatico costantinopolitano vissuto nel VI secolo. Fra le sue opere, pervenuteci in forma piuttosto frammentaria, un dizionario geografico in 50 o 60 volumi.}. \textit{Dondolare}, o \textit{dondolarsela} vuol dire Starsene
  a sedere senza far nulla, di dove \textit{Dondolone} vuol dire un perdigiorno. Quindi
  un moderno Poeta intendendo di questi tali disse:
  \begin{verse}
    Voi dal notturno al mattutin crepuscolo
    Vi dondolate, e fate a tu me gli hai,
    Ne conchindete o proponete mai,
    Se non rovine al popolo minuscolo.
  \end{verse}

\item[C'HA più peccati, che non è de' fatti] Ha più peccati ella sola, che non sono
  quelli, che sono stati fatti, o commessi da tutto il mondo insieme infino a ora.

\item[BAGORDI] Festeggiamenti. Vedi sopra \cstan[5]{62}.

\item[TARDI, o per tempo] Diciamo anche \textit{Tardi, o accio} (cioè avaccio, parola,
  antica, rimasa in contado, che vale tosto) o vero \textit{tardi, o avale}; che dissero
  ancora gli antichi \textit{aguale}; cioè ora, in questo punto; vuol dire; questo seguirà una
  volta o presto, o tardi. Lat. \textit{serius, ocyus}.

\item[DA ultimo è buon tempo] Da ultimo verra il sereno. \textit{Post nubila Phoebus}. Qui è
  detto ironico, perché significa, che da ultimo per Martinazza verrà il tempo cattivo,
  cioè sarà gastigata del suo mal fare.

\item[INTRIGARSI] Vuol dire impacciarsi, o interessarsi: e vuol dir' anche imbrogliare,
  o mescolar una cosa con un' altra in maniera di confonderle, donde \textit{intrigo}
  per imbroglio.

\item[UNA fune daralle, che l'impicchi] Quand'altri ci ha mal serviti, per mostrargli,
  che non merita rimunerazione, si suol dire; Gli vuò dare un par di corna;
  Un par di funi, o una fune, che l'impicchi.
\end{description}
\section{STANZA IV. STANZA}

\begin{ottave}
\flagverse{4}Horsù tiriamo innanzi, ch'io ho finito,\\
Perch' a questi discorsi le persone\\
Non mi dicesser: Questo scimunito\\
Vuol farci qualche predica o sermone,\\
Attenti dunque. Già v'havete udito\\
L'incanto, ch'ella fece a petizione\\
Di quei del luogo, c'hebbero concetto\\
Scacciarne il Duca; ma svanì l'effeto.
\end{ottave}

\begin{ottave}
\flagverse{5}Ella ch'in tanto havuto havea sentore,\\
Che quei due spirti sciocci ed inesperti\\
Havean dinanzi a lui fatto l'errore,\\
Sì che da esso furono scoperti;\\
Se la digruma, che ne va il suo honore,\\
Mentre gli accordi fatti, ed i concerti\\
Riusciti alla fin tutte panzane,\\
Con un palmo di naso ne rimane.
\end{ottave}



Il Poeta lasciando da parte la moralità, viene al racconto, e torna alla memoria
del Lettore l'incanto fatto da Martinazza per cacciare il Duca, che non
hebbe effetto, per lo che ella è in collera, perché le pare di perdere di quella
stima, nella quale era tenuta dai popoli, e soldati di Malmantile.

\begin{description}
\item[SCIMVNITO] Sciocco, scempiato. Vedi sopra \cstan[1]{17}.
\item[SVANÌ l'effetto] Non riusci l'effetto: il negozio andò in fumo. I Lat. pure
  dissero \textit{Evanuit}, \& \textit{evanescere}.
\item[SE la digruma] Seco stessa la pensa, e masticandola non la può inghiottire,
  cioè non la può sofferire. E si dice digrumare, e ruminare, e dagli antichi fu detto
  \textit{rugumare}, onde forse è fatto digrumare; (che è il rodere che fanno le bestie dal
  pié fesso, come vedemmo sopra C.4. stan. 6. e \cstan[5]{63}.) perché uno, a cui
  succeda cosa di poco suo gusto, suole per lo più stando pensoso masticare, o biasciare
  appunto come fanno dette bestie quando digrumano, al che per avventura
  ebbe riguardo Omero in quel verso tradotto da Cicerone.
  \begin{verse}
    Ipse suum cor edens, hominum vestigia vitans.
  \end{verse}
  quasi che chi maninconico rumina, e biascia masticandola male; mostri di
  beccarsi il cuore.
\item[RIVSCITI tutti panzane] Son riusciti tutte vanità, tutte chiacchiere. \textit{Che dar
  panzaze, bubbole, chiacchiere, ec}, vuol dir promettere, e non mantenere, che si
  dice \textit{inzampognare}, \textit{infinocchiare}, ed è il Lat, \textit{Verba dare}.
\item[RIMANE con un palmo di naso] Riman burlata, beffata, Il Lalli En, tr. lib. 1.
stan. 11. dice.
\begin{verse}
  Ed io son per restar in queste caso
  Con sei palmi lunghiffini di naso.
\end{verse}
\end{description}

\section{Stanza VI. \& VII.}

\begin{ottave}
\flagverse{6}Ma non si sbigottisce già per questo,\\
Che vuol cansar quell'armi dalle mura,\\
Ai Diavoli, da' quali hebbe il suo resto,\\
E che gliel'hanno fatta di figura,\\
Vuol, dopo il far, che rompano un capresto\\
Squartare, e poi ridurre in limatura,\\
Perché non fu mai can, che la mordesse,\\
Che del suo pelo un tratto non volesse.
\end{ottave}

\begin{ottave}
\flagverse{7}Basta, ch'ella sel'è legata al dito,\\
E l'ha presa co' denti, e ser'affanni;\\
Tal c'andarsene in Dite ha stabilito,\\
Perché ne vuol veder quanta la canna,\\
Ed oprar, che Baldon resti chiarito\\
C'ambisce in Malmiantil sedere a scranna;\\
Hor mentre a questa volta s'indirizzi,\\
Potrà far un viaggio a due servizj.
\end{ottave}


Martinazza non si perde d'animo, e vuole in ogni maniera scacciar l'esercito
di Baldone da Malmantile. Risolve però d'andare all'inferno in persona a trovar
Plutone, per ottener da lui il gastigo di quei due diavoli, che fecero l'errore,
ed un nuovo modo di far diloggiar Baldone da Malmantile.
\begin{description}
\item[NON si sbigottisce]Non si perde d'animo; Non si sgomenta. Vedi sopra C. 2.
  stan. 28. e \cstan[5]{63}.
\item[HEBBE il suo resto] Hebbe finito di conoscergli. Hebbe viflo quanto essi valevano.
  Si dice \textit{Tu m'hai dato il mio resto}: \textit{Tu m'hai pieno}: \textit{Son sazio}, \textit{son stufo di te},
  per intendere Non mi varrò mai più dell'opera tua.
\item[GLIEL'hanno fatta di figura] Le hanno fatto una ingiuria grandissima, una
  solennissima burla. Tratto dal giuoco di primiera, quando uno havendo buon,
  punto, ed essendo per vincer la posta, un'altro con figura fa una primiera, e gli
  leva la posta.
\item[ROMPANO un capresto] Restino impiccati. Chiamano capresto quella cordicella
  sottile, che il Boia lega al collo a coloro, che egli impicca, la quale dicono,
  che morto il paziente si rompa; e però dice rompano un capresto; detto
  usatissimo per intendere farsi impiccare.
\item[RIDURRE in limatura] Ridurre in minutissimi pezzi. Limatura si dicono quei
  fragmenti che cascano dal ferro, o altro metallo, quand'altri lo lima.
\item[NON mi morse mai cane, ch'io non volessi del suo pelo] Nessuno mi fece mai ingiuria,
  che non mi volessi vendicare. Nessuno mi morse, che io non lo rimordessi.
  Dicono che il pelo del cane sia medicamento alle morsicature fatte dal medesimo
  cane. Vedi sotto \cstan[9]{58}. E da questo rimedio ha origine il presente
  dettato; che i latini dissero \textit{Nemo impune abiit, qui me ausus sit laedere},
\item[SEL' è legata aj dito] Ne ha presa memoria per vendicarsi. Sogliono molti per
  haver memoria di qualche negozio, che devano fare, legarsi un filo intorno a dito;
  il che ha dato origine al presente dettato. Il Lalli En. Tr. Can. 2, stan. 25, dice:
  \begin{verse}
    Sel' attaccò, come suol dirsi, al dito.
  \end{verse}
  Nel Deuteronomio al sesto, \textit{Eruntque verba haec, quae ego praecipio tibi hodie in corde
  tuo: \& narrabis ea filijs tuis., \& meditaberis sedens in domo tua, \& ambulans in itinere,
  dormiens atque consurgens: \& ligabis quasi signum in manu tua}. E sono al cap. 11.
  \textit{Ponite haec verba mea in cordibus, \& animis vestris, \& suspendite ea pro signo in manibus}.
  Fra Giordano Predicatore antico Domenicano; nel Vocabolario della
  Crusca alla Voce \textit{Filateria}. Le filaterie si erano una carta, ove erano scritti i
  comandamenti della Legge, e portavanla intorno al braccio apertamente. E quivi
  va spiegando, cred'io, il passo di San Matteo cap. 23. \textit{Dilatant enim phylacteria
    sua}. È voce Greca; da \textit{phylattein}, guardare, custodire, significante certe strisce
  di quoio, o di cartapecora, che gli Ebrei si legano al braccio per tenere maggiormente
  a memoria i passi della Scrittura, che quivi sono notati, le quali da loro si
  domandano: \textit{Tephilim}.
\item[L'HA presa co' denti] S'è adirata grandemente, e s'è messa in animo di vendicarsi.
  Vuol impiegare ogni suo studio per vendicarsi. Sogliono i calzolai per far
  venire il quoio a quel segno che loro bisogna; tirarlo co' denti, e di qui nasce il
  presente termine, che esprime uno, che si sia preso a cuore di fare un negozio,
  e che voglia impiegare ogni suo talento per conchiuderlo.
\item[SE n'affanna] Se l'è presa a cuore. N'ha premura. Se ne dà pena, e pensiero.
\item[IN Dite] Dite, secondo il favoloso creder de i Gentili è lo stesso, che Plutone,
  l'uno, e l'altro nome significando ricchezze delle quali, perché si cavano di sotterra,
  facevano Custode, e Padrone quel loro Dio sorterraneo; ma qui si piglia Dite
  per la Città, e per il Regno di Dite.
\item[NE vuol veder quanto la canna] Cioè quanto tira, o è lunga la canna da misurare;
  e s'intende vederla per la minuta, e quanto si può, e fare ogni sforzo per
  arrivare al suo intento.
\item[RESTI chiarito] Resti sgarito: Scaponito. Vedi sopra \cstan[1]{1}.
\item[SEDERE a scranna] Vuol dire comandare; esser padrone. \textit{Scranna}, o
  (come diciamo noi) ciscranna, è una specie di seggiola da i Latini detta \textit{sella plicatilis}.
  Dante Purg. C, 19. dice:
\begin{verse}
  \backspace Hor chi sei tu che vuoi sedere a scranna
  Per vindicar da lungi venti miglia
  Con la veduta corta d'una spanna ?
\end{verse}
Buratto nell'Apologia contro al Castelvetro dice: \textit{Non habbiate tanto cervelle, che
  basti, se ben volete sedere a scranna per giudicare gli altri}.
\item[FAR un viaggio a due servizzj] Che dichiamo anche: \textit{Fare un viaggio, e due
  servizzj}). Con un medesimo viaggio far due negozzj, che è impetrar da Plutone
  il gastigo di quei due diavoli, e lo sfratto di Baldone. Ne i Latini si trova in
  questo senso \textit{Duos parietes de eadem fidelia dealbare}. E si dice anche \textit{Dare a due
    tavole a un tratto}, Vedi sopra \cstan[3]{14}.
\end{description}
\section{Stanza VIII. --- X.}
\begin{ottave}
\flagverse{8}Giù da Mammone andar vuole in persona, \\
Che più non è dover, ch'ella pretenda, \\
Che sua bravicornissima corona \\
Salga a suo conto a ogni poco, e scenda, \\
Chieder grazie, e dar brighe non consuona, \\
E chi ha bisogno, si suol dir, s'arrenda, \\
Per questo a lei tocca a pigliar la strada, \\
Per c'alla fin convien, che chi vuol vada.
\end{ottave}

\begin{ottave}
\flagverse{9}Perciò s'accontia, e va tutta pulita\\
Col drappo in capo, e col ventaglio in mano\\
A cercar chi l'informi della gita;\\
Ne meglio sa, che Giulio Padovano,\\
Che l'ha su per le punta delle ditta,\\
E più di Dante, e più del Mantovano,\\
Perch'eglino vi furon di passaggio,\\
E questo ogni tre di vi fa un viaggio.
\end{ottave}

\begin{ottave}
\flagverse{10}Onde a trovarle andata via di vela\\
Domanda (perchè in Dite andar presume)\\
Che luoghi v'è, che gente, e che loquela\\
Ed ei di tutto le dà conto, e lume;\\
E poi per abbondare in cautela,\\
Volendola servire infino al fiume,\\
Le porge un fardellin piccolo, e poco\\
Di robe, che laggiù le faran giuoco.
\end{ottave}

Martinazza risolve d'andare in persona a trovar Plutone, considerando, che
non è dovere, che questo Re per lei a ogni poco si scomodi; e però sapendo, che
Giulio Padovano è più informato d'ogni altro della strada dell'Inferno, se ne va
a pigliar da lui informazione, e della gita, e dei costumi di quei paesi; ed egli
l'istruisce, e per servirla meglio la vuol accompagnare fino al fiume Acheronte,
ed intanto le dà un fardellino di robe, che laggiù verranno a bisogno.
\begin{description}
\item[BRAVICORNISSIMA corona] Epiteto, e titolo composto dall'Autore a Plutone.
  Il Lalli En. Tr. lib. 1. stan. 16. parlando d'Eolo Re de' Venti dice:
  \begin{verse}
    Dunque poi che Giunone alla presenza
    Di sua Real ventosità fu giunta.
  \end{verse}
\item[MAMMONE] Da Mammona; parola usata nell'Evangelio. Alcuni espositori
  della Sacra Scrittura vogliono, che Mammona sia voce Caldea, e significhi
  \textit{opes}, ed altri che sia voce Siriaca, e significhi quello, che in Greco significa
  \textit{Plutos}, che è \textit{divitiae}, sì che concordano, e tanto è a dir Mammone, che
  Demonio, ovvero Plutone, che qui s'intende per il Re dell'Inferno. Viene
  dalla radice Ebrea \textit{Taman}, che propriamente significa \textit{nascondere}, \textit{riporre}, e
  per così dire \textit{intanare}; onde si fece \textit{Matmon}, e alla Siriaca \textit{Matmona}, cioè \textit{ricchezze
  nascoste}, o vogliam dire \textit{tesoro}. Mammona poi venne a dirsi per più agevolezza di pronunzia.
\item[DAR brighe] Dare scommodi, dar molestie. La voce briga significa operazioni
  scommode, faticose, e noiose.
\item[CHI ha bisogno s'arrenda] Chi ha bisogno non sia superbo, ma si pieghi a raccomandarsi,
  e pregare; Che il verbo \textit{arrendersi} val per cedere piegarsi, o condescendere.
\item[CHI vuol vada] Chi vuol ottenere una cosa vada a chiederla da per se, ed il
  proverbio dice \textit{Chi non vuol mandi, e chi vuol vada da se}. Che diciamo anche \textit{Non
    è più bel messo, che se stesso}, o vero, \textit{Chi va lecca, E chi sta si secca}.
\item[ACCONCIARSI] Rinfronzirsi, raffazzonarsi. Vedi sopra C.\ 2. stan.\ 69.
\item[DRAPPO] Dicendosi drappo assolutamente s'intende drappo da donna, che
  è una striscia di taffettà, o d'ermisino larga fino a due braccia, e lunga fino a quattro,
  la quale dalle donne Fiorentine di condizione ordinaria è portata in capo,
  o alle spalle quando vanno fuori di Casa. In Venezia \textit{drappo} significa ogni sorta
  di vestimento, sì come presso i Toscani antichi scrittori. Vedi sotto C.7.stan.22.
\item[VENTAGLIO] Strumento noto usato dalle donne la state per farsi vento.
\item[L'INFORMI della gita] Le insegni la strada, che conduce all'inferno.
\item[GIVLIO Padovano] Io veramente non ho saputo ritrovare chi sia questo Giulio
  Padovano\footnote{Potrebbe anche trattarsi di Giulio Cesare Casseri (1552-1616). Sebbene originario di Piacenza, si stabilì a Padova, dove studiò ed insegnò anatomia. Autore delle \textit{Tabulae anatomicae}, pubblicato nel 1627 a Venezia, principale trattato di anatomia del secolo diciassettesimo, magari noto al Lippi per le 97 tavole anatomiche. }, se forse non ha inteso di Giulio Hygino scrittore d' Astronomia.
  Ma costui fu liberto, o vogliam dire schiavo affrancato d'Augusto; condotto da
  lui ragazzo d'Alessandria, secondo che alcuni vogliono; i quali perciò lo stimano
  Alessandrino; o pure di nazione Spagnuolo, secondo la testimonianza di Svetonio
  nel Libro \textit{de illustribus Grammaticis}.
\item[L'HA su per le punte delle dita] La sa benissimo; Latino \textit{in numerato habet}.
  Aldo Manuzio nella dedicatoria di Giuvenale disse: \textit{Quando eas tenebas memoria, quam
    digitos unguesque tuos}, Cicerone nella Orazione contra Cecilio intitolata \textit{Divinatio}:
  \textit{Quid cum accusationis tuae membra dividere ceeperit, \& in digitis suis singulas partes
  causae constituere ? Quid, cum unumquodque transigere, expedire, absolvere?}
\item[DANTE, e il Mantovano] Dante Poeta Fiorentino; e Vergilio, il quale Dante
  finge, che fusse sua guida all'Inferno, e pero dice: \textit{Eglino vi furon di passaggio}.
\item[OGNI tre dì] Questo modo di dire, se bene è determinate, significa spesso spesso,
  o a ogni poco indeterminatamente.
\item[ANDAR via di vela] Andar via velocemente, e a dirittura, come fa la nave
  quando va a vela.
\item[PER abbondare in cautela] Cioè per servirla bene. Diciamo \textit{abbondar in cautela}
  quando uno fa più di quel che sia richiesto, o più di quel che sia necessario; per
  esempio. Io darò dieci scudi a uno, perché mi compri una mercanzia, la quale so
  che non vale così gran somma; ma per assicurarmi del caso, che valesse non
  più, li do due altri scudi \textit{per abbondare in cautela}, cioè per andar cautelato, e in
  sul sicuro, che non gli manchi denaro, se ella valesse più. Qui però vuol dire
  Abbondare, ed eccedere in cortesia nel servirla.
\item[LE faranno giuoco] \makebox[8pt]{} Le torneranno a proposito. Le verranno a bisogno. Le
  saranno d'utile.
\end{description}
\section{STANZA XI. --- XV.}

\begin{ottave}
\flagverse{11}Così la Maga se ne va con esso,\\
Che l'introduce in una bella via \\
Tutta fiorita, sì che al primo ingresso \\
Par proprio un Paradiso, un'allegria; \\
Ma non più presto l'huom il pié v'ha messo \\
Ch'ella diventa un'altra mercanzia \\
Per i gran morsi, e le punture acerbe, \\
Che fanno i serpi ascosi fra quell'erbe.
\end{ottave}

\begin{ottave}
\flagverse{12}Entravi Martinazza, e sente un tratto\\
Due, o tre morsi a pié dove calpesta,\\
Perciè bestemmia, che non par suo fatto,\\
E dice: O Giulio mio, che cosa è questa?\\
Ed ei ridendo allora come un matto;\\
Non è nulla (rispose) vien pur lesta;\\
Che pensi tu ch'io sia privilegiato?\\
Anch'io mi sento mordere, e non fiato.
\end{ottave}

\begin{ottave}
\flagverse{13}Questa è la via, che mena a casa calda,\\
Perch'ella è allegra, o almeno ella ci pare,\\
Perché a martello poi non istà salda;\\
La scorre ognor gente di mal'affare,\\
Le serpi sono ogni opera ribalda\\
Ch'ella ci fa, le quali a lungo andare\\
Di quanto ha fatto, scavallato, e scorso\\
Ci fa sentir al cuor qualche rimorso.\\
\end{ottave}

\begin{ottave}
\flagverse{14}Ma se ravvista un tratto del suo fallo\\
Bada a tirar innanzi alla balorda,\\
Perch'il vizio rifiglia, e mette il tallo\\
Vien sempre più a aggravarsi in su la corda,\\
Ul male invecchia al fine, e vi fa il callo\\
Sì che venga un Serpente pure, e morda,\\
Ch'ei non sente ne meno anc'un ribrezzo,\\
Così peggio che mai la dà pel mezzo.
\end{ottave}

\begin{ottave}
\flagverse{15}Nella neve si fa lo stesso giuoco,\\
che l'huom sul primo diacciasi le dita,\\
Poi quel gran gelo par che manchi un poco,\\
E sempre più nell'agitar la vita;\\
Al fine ei si riscalda come un fuoco\\
Sì che non la farebbe mai finita,\\
Ne gli darebbe punto di spavento\\
Quand'ei v'havesse ancora a dormir drento.
\end{ottave}

Martinazza se ne va con Giulio, il quale la conduce per una strada, che al
primo ingresso pare una bella cosa, ma presto si conoice, ch'ell'è altrimenti per li
morsi che danno i serpi ascosi infra quell'erbe; Giulio mostra a Martinazza,
che questa strada, che guida all'Inferno è facile, e gustosa, e se bene e ripiena
di malanni, non son sentiti ne conosciuti da quelli, che la camminano, perché
vi si sono assuefatti; appunto come fanno coloro, che mettono le mani nella neve,
che a principio la toccano fredda, e col seguitare a maneggiarla, par loro che ella sia calda.
\begin{description}
\item[PARE un Paradiso] Pare una cosa tanto allegra, e vaga, che più non si può
  fare. Telemaco figliuol d'Ulisse nel quarto dell'Ulissea, arrivato in Sparta; nel
  considerare attentamente la ricchezza, e l'ampiezza del Regio Palazzo di Menelao,
  prorompe in quella esclamazione: \textit{Tal dentro è del gran Giove il gran Palazzo}.
\item[DIVENTA un altra mercanzia] Diventa un' altra cosa. Usiamo dir \textit{mercanzia}
  per esprimere ogni sorta di cosa ancor che incorporea, come \textit{lo studiare è una certa mercanzia},
  ec.
\item[NON par suo fatto] Non par che faccia quella tal cosa. Vedi sopra Can. 4.
  stan. 16.
\item[CASA Calda] Intende l'Inferno. Il Lalli En. Tr. parafrasando \textit{facilis
  descensus Averni} ec. dice:
  \begin{verse}
    \makebox[8em]{\dotfill} Enea mio bello,
    A casa Calda si va presto presto;
    Ma ritornar in su, questo è il bordello.
  \end{verse}
\item[NON è nulla] Queste due negative secondo la buona regola doverebbono affermare,
  ma è nostro idiotismo tanto inveterato, che l'uso ci libera dall'errore, se
  ce ne serviamo in questo modo per negativa. Appresso i Greci due negative, o
  affermano, ma negano maggiormente, ed è maniera, siccome appresso
  noi, così appresso loro usatissima.
\item[NON sta a martello] Non regge alla prova. Non è com'ella pare. Metafora
  tolta dal Cimento dell' oro. Vedi sopra \cstan[5]{2}.
\item[A LUNGO andare] Col tempo. In processo di tempo; Se continoverai lungo tempo.
\item[SCAVALLATO] Cioè datasi ogni sorta di bel tempo. Si dice anche \textit{scorrer
  la cavallina}. Virg. 3. Georg. \textit{Scilicet ante omnes furor est insignis equarum, Et
  mentem Venus ipsa dedit}. E poi: \textit{illas ducit amor trans Gangara, transeque sonantem}, \&c.
  Vedi fopra \cstan[1]{66}.
\item[QUALCHE rimorro] Senton rimorder la coscienza per gli errori commessi.
\item[ALLA balorda] Senza considerazione.
\item[METTE il tallo] Tallisce, fa nuove messe. Vuol dire: un vizio ne genera
  molti. Tallo è parola venuta a noi dalla lingua Greca, che significa germoglio,
  usata ancora dagli agricoltori Latini.
\item[VIENE a aggravarsi in su la corda] Vien più che mai a crescere il male; perché
  quando uno tocca il martirio della corda, e s'aggrava in su la medesima corda,
  fa crescere il dolore; Ed altrimenti \textit{aggravarsi in su la corda} vuol dire quando uno
  esaminato in su la corda dice cose, che fanno crescere l'indizio, che egli habbia
  commesso un delitto.
\item[FA il callo] Vi s'assuefà. \textit{Et ab assuetis non fit passio}, dice, che non sente
  ne meno un ribrezzo.
\item[RIBREZZO] Che vuol dire capriccio di febbre; cioè quel tremore, o brivido,
  che si sente prima, che entri la febbre. Latino \textit{rigor}. Il Cavalcanti Stor. Fior.
  \libcap[2]{21}, dice: \textit{Antipatro di Sidonia in quel giorno, che egli nacque, ogn'anno
    gli arrivava qualche ribrezzo di febbre, e tanto continua, che un'anno gli rinvestì in
    mortale accidente}. Ma Dante nell'Inf. C. 17 mostra, che si dicesse riprezzo.
  \begin{verse}
    \backspace Qual è colui c'ha sì presso il riprezzo
    Della quartana, c'ha già l'ugna smorte,
    E trema tutto pur guardando il rezzo.
  \end{verse}
  E al C.32.dice:
  \begin{verse}
    \backspace Poscia vedd'io mille visi cagnazzi
    Fatti per freddo, onde mi vien riprezzo,
    E verrà sempre de i gelati guazzi.
  \end{verse}
  Ma noi lo pigliamo anche (come e preso nel presente luogo) per ogni leggiero
  sollevamento d'animo, o spavento, o per un semplicissimo dolore. Ed alle volte
  per fastidio, o travaglio per esempio \textit{Il tale commesse quel mancamento; ne vuole
    haver de' ribrezzi}. Vedi sotto \cstan[11]{2}.
\item[LA dà pel mezzo] Fa tutto quello, che gli vien volontà senza riguardo alcuno.
  È dedotto da quelli; che in tempo di pioggia camminando per la Città vanno
  per il mezzo della strada, e non si guardano dall'ammollarsi per l'acqua
  caduta, che scorre pel mezzo, e per quella che vien dal Cielo.
\end{description}
\section{Stnaza XVI \& XVII.}
\begin{ottave}
\flagverse{16}Hor tu m'hai inteso: rasserena il volto, \\
Che tu vedrai tirando innanzi il conto \\
(Perché di qui a poco non c'è molto) \\
Che delle serpi non farai più conto,\\
Ma dimmi, c'ha' tu fatto del rinvolto?\\
L'ho qui, dic'ella, sempre lesto, e pronto: \\
Sta ben, soggiunge Giulio, adunque corri,\\
Perché qui non è tempo da por porri.
\end{ottave}

\begin{ottave}
\flagverse{17}Resta, dic'ella, omai ch'io ti ringrazio\\
Dell'instruzion, ch'appunto andrò seguendo:\\
Promissio boni viri est obligatio,\\
Dic'egli; T'ho promesso, e però intendo\\
Ancor seguirti questo po di spazio,\\
E quivi con un tibi me commendo,\\
All' in qua ripigliando il mio cammino\\
Ti lascio, come io dissi al colonnino.
\end{ottave}


Giulio esorta Martinazza a non haver paura, ed a camminare; ed ella lo ringrazia
dell'instruzione datale, e lo prega a partire, ed egli ricusa di farlo, perché
le ha promesso di accompagnaria infino al fiume Acheronte.
\begin{description}
\item[DI qui a poco non c'è molto] Questo termine giocoso e usato per esprimere
  \textit{fra pochissimo tempo}.
\item[TIRANDO innanzi il conto] Seguitando il suo viaggio, È termine mercantile
  che vuol dir portare un conto avanti da un libro a un'altro, o da una carta a
  un'altra nel medesimo libro. Donde poi \textit{tirar innanzi il conto} vuol dir
  Camminare avanti. Vedi sopra \cstan[4]{60}.
\item[NON è tempo a por porri] Non è tempo da perdere. Non è da indugiare.
  Quando si pongono i porri, sono così fottili, che richiedono molto tempo a
  porgli, e da questo habbiamo il presente proverbio, che si dice anche \textit{Non è
    tempo da dar fieno a oche}.
\item[PROMISSIO boni viri est obligatio] Sentenza latina, che vuol dire un'huomo
  da bene è obligato a mantener la parola, ed osservare quel che ha promesso.
\item[CON un tibi me commendo] Detto latino, che suona con un mi raccomando a te;
  cioè con salutarti. Quando diciamo: \textit{Addio}, ci s'intende; vi raccomando.
  Saluto di congedo, Catullo: \textit{Commendo tibi me}.

\item[TI lascio al colonnino] Ti abbandono. \textit{Lasciar al colonnino} vuol dire lasciar uno
  nel pericolo, perché \textit{colonnino} intendiamo quella colonnetta di legno traforata,
  la quale è davanti alle forche, e vi legano i malfattori quando gli strozzano.

\end{description}

\section{Stanza XVIII --- XXI.}
\begin{ottave}
\flagverse{18}Ed essa allora abbassa il capo, e tocca,\\
Se ben de' serpi ell'ha qualche paura;\\
Pur via zampetta, e fatto del cuor rocca,\\
Va calcando la strada alla sicura,\\
Sè ch'ella non si sente aprir la bocca,\\
Perché non è più morsa, o non lo cura:\\
Giunti alla fine al gran fiume infernale,\\
Restò la donna, ed ei le disse: Vale.
\end{ottave}

\begin{ottave}
\flagverse{19}Questo è il famoso fiume d'Acherorte,\\
Ove s'imbarca ognun, che quivi arriva,\\
S'affaccia anch'essa, ma il nocchiere Caronte,\\
Da poi che tratto ognuno hebbe da riva,\\
Sta in dietro, grida a lei con torva fronte,\\
Che quà non passa mai anima viva;\\
Ond'ella, messi fuor certi baiocchi,\\
Gli getta un po di polvere negli occhi.
\end{ottave}

\begin{ottave}
\flagverse{20}Ed egli, che da essa hebbe il sapone,\\
E che si trovò lì come il ranacchio,\\
Preso dalla medesima al boccone,\\
Mentr'ella saltò in barca, chiusdfe l'occhio.\\
La Strega fra quell'anime si pone,\\
Quai con le brache son fino al ginocchio,\\
Dovendo a' Soprassindaci di Dite\\
Presentar de' lor libri le partite.
\end{ottave}

\begin{ottave}
\flagverse{21}Piangendo, come quando uno ha partito\\
Le cipolle fortissime malige:\\
Passan quel fiume, e poi quel di Cocito;\\
Ultimamente la palude stige,\\
Che a Dite inonda tutto il circuito,\\
E in se racchiude furbi, e anime bige,\\
Ove Caronte al fin sendo arrivato\\
Sbarcò tutti; ed ognun fu licenziato.
\end{ottave}



Martinazza seguita il suo viaggio, e non fa più stima delle morsicature de i
serpi, ed arrivati al fiume d' Acheronte', Giulio si licenzia dalla donna, la quale
s'accostò per entrar nella barca; ma Caronte lo sgridò dicendo, che non poteva
entrarvi, ond' ella gli diede un poco di mancia, ed ei finse di non la vedere entrar
in barca, dove ella si mescolò con gli altri, e fu condotta all' altra riva, e
quivi con essi sbarcata.
\begin{description}
\item[TOCCA] Si dice \textit{tocca il cocchio}, e significa: cammina innanzi. Vedi sopra
  \cstan[1]{41}.
\item[ZAMPETTA] Muove le gambe: Cammina. Zampettare si dice propriamente
  de' bambini quando cominciano a imparare a andare.
\item[NON si sente aprir bocca] Non si sente parlare. Sono infiniti i modi, che habbiamo
  per esprimer il silenzio d'uno, come \textit{star zitto; non fiatare; non far verbo;
    ammutolire; star chiotto, lasciar la lingua al beccaio, haver visto il lupo; diventare
    Arpocrate} ec.
\item[GLI disse] Vale. Gli disse Addio.
\item[ACHERONTE] I fiumi dell'Inferno da i Gentili si dicevano quattro, e che
  nascessero dalle lagrime de' mortali, per lo stato de' quali figura Dante la statua,
  che vedde in sogno Nabucdonosor, che havea la testa d'oro, le braccia, e
  petto d'argento, il corpo fino alle cosce di rame, le gambe di ferro, ed il destro
  piede di terra cotta; da questa dice che scaturiscono le dette lagrime, le quali
  formano li detti quattro fiumi Infernali, e così la descrive nell'Inf. C. 14.
  \begin{verse}
    \backspace    Dentro dal monte sta dritto un gran veglio,
    Che tien volte le Spalle in ver Damiata,
    E Roma guarda si come suo speglio,
    \backspace    La sua testa è di fin' oro formata,
    E puro argento son le braccia, e il petto,
    Poi è di rame fino alla forcata.
    \backspace    Da indi in giuso è tutto ferro eletto,
    Salvo che 'l destro piede è terra cotta,
    E sta in su quel più, ch'in su l'altro, eretto.
  \end{verse}

Il primo dunque di detti fiumi e Acheronte, che in un certo modo significa
privazione d' allegrezza;da Acheronte nasce Stige, che significa cosa dispiacevole,
odiosa, quale e il Dolore; perché questo ne viene dopo la privazione dell'allegrezza,
Il terzo è Flegetunte, che signfica pensiero ardente travaglioso. E da
questi tre fiumi si genera il quarto, che è Cocito stagno,o fiume del lamento, e del
pianto. Questa favolosa opinione de' Gentili tocca Dante nell' Inf, C, 14. seguitando
i sopraddetti versi.
\begin{verse}
  \backspace Ciascuna parte, fuor che l'oro è rotta
  D' una fessura, che lagrime goccia,
  Le quali accolte foran questa grotta,
  \backspace Lor corso in questa valle si diroccia:
  Fanno Acheronte, Stige, e Flegetonta;
  Poi sen va già per questa stretta doccia.
  \backspace Infin là dove più non si dismonta,
  Fanno Cocito, e qual sia quello stagno
  Tu'l vederai, però qui non si conta.
\end{verse}

\item[CARONTE] Notissimo barcarolo dell' Inferno. Vedi sopra C. 2, stan.24.
\item[HEBBE tratto ognun da riva] Hebbe levate d'in su la riva tutte l'anime, imbarcandole.
\item[TORVA fronte] È latino usato da noi; E vuol dire Viso burbero, aspro, agro,
  arcigno.
\item[ANIMA viva] Intendi huomo, che non sia morto. Virg, 6. En. \textit{Corpora
  vina nefas Stygia vectare carina}. Sa bene il nostro Poeta, che anime sono immortali,
  ma seguita il costume d' intendere huomo vivente, quando diciamo anima
  viva (Genesi cap.2, Et factus est homo im animam viventem) ed imita Dante Inf.
  C. 3. che dice:
  \begin{verse}
    E tu che sei costì anima viva,
    Partiti da codesti, che son morti.
  \end{verse}
  Il Lalli En, Tr. \cstan[3]{16}.
  \begin{verse}
    E non v' è mai entrata anima viva.
  \end{verse}
\item[GLI gettd un po di poluere negli occhi] Gli decte un po di mancia, I Latini pure
  dissero: \textit{Pulverem oculis offundere}. E s'intende dar mance per corrompere il giusto,
  quasi diciamo: \textit{Abbagliare gli occhi del giudice con l'oro, acciocché non vegga
    la giustizia}.
\item[HEBBE il sapone] Era stato subornato, e corrotto con la mancia; Gli erano state
  insaponate le carrucole (che vuol dire Tirar' uno al nostro volere, e renderlo
  facile a quel che noi bramiamo, e fare che non strida contro di noi) con
  dargli la mancia; come con l'insaponare una carrucola, o una ruota si facilita
  il veicolo, e si fa, che non strida. Ed è lo stesso che \textit{gettar la poluere negli occhi}
  detto poco sopra. Dicesi anche: \textit{Ugner le mani}. Bocc, Nov.6. \textit{Il buono huomo per
 certi mezzani gli fece ugner le mani}.
\item[PRESO al boccone, come il ranocchio] Obbligato a tacere, per havere havuta la
  mancia. È lo stesso che li suddetti due modi di dire, cioè \textit{Havere il sapone}, e
  \textit{Havere la polvere negli occhi}. Qui non vorrei che il Lettore credesse, che il Poeta
  avesse opinione, che i regali potessero corrompere i demonj, se ben la sentenza
  portata da Ovidio dice \textit{munera (crede mibi) placant hominesque deosque}, ma sapesse haver'
  egli detto così, per mostrare che l'oro arriva a corromper quelli, che ne meno si
  crederebbe, e che meno dovriano lasciarsi arrivar dall'oro, e finalmente ha voluto esprimere
  la possanza, che hanno i regali di far conseguire ciò che si vuole.
  \textit{Ombia enim per pecuniam facta sunt}. Si racconta di Filippo Macedone, che havendo
  fatto riconoscere una fortezza, ed essendogli riferito, che era impossibile il
  pigliarla, domandasse agli sploratori, se vi era modo di farvi andare un' asino carico d'oro,
  volendo inferire, che dove non potevano l'armi, sarebbe arrivato l'oro;
  \textit{Auri Sacra fames, quid non mortalia pectora cogis?} E Orazio. \textit{Aurum per
    medios ire satellites, Et perrumpere amat saxa potentius Ictu fulmineo}.
\item[CHIUSE l'occhio] Finse di non vedere. È il latino connivere, Vedi sotto \cstan[10]{5}.
\item[CON le brache fino al ginocchio] Il proverbio \textit{Caschar le brache} è il medesimo,
 che \textit{cascar le braccia}; che vuol dir perdersi d'animo, Omero: \textit{Animus in pedes decidit}.
 Cascò il cuore; cascò l'animo a' piedi, Onde dicendo, che costoro \textit{havevano le brache
 fino al ginocchio}, intende che eran loro cascate affatto, cioè erano
 del tutto perduti d'animo, perché doveano render conto delle loro azioni. Vedi sotto C.~9. stan.~24.
\item[SOPRASSINDACI] Così chiamiamo noi quel Magistrato, che ha l'autorità di riveder
  conti a tutti i Magistrati, Ofiziali, e Ministri del dominio Fiorentino.
\item[CIPOLLA maligia] Specie di cipolla da mangiare, che è fortissima, e fa venir le
  lagrime a tagliarla, e maneggiaria; Bocc. gior. 8.n.2, \textit{E talora un mazzetto di
    cipolle malige, o di Scalogni}. Il Lalli En, Tr. C. 3.
  \begin{verse}
    Così dicea, e tutto il volto molle
    Havea di pianto, come se schiacciato
    Vi fusse sopra il sugo di cipolle.
  \end{verse}
\item[COCITO] Vedi sopra alla stan. 19, alla parola \textit{Acheronte}, e quivi troverai
  ancora quel che sia la Palude Stige, della quale vedi anche sotto in questo Cant. stan. 76.
\item[GENTI bige] Genti scellerate, e da non se ne fidare. Per comporre il color
  bigio i Pittori mescolano tutti i colori, e lo chiamano il color dell'asino; e però
  dicendosi huomo bigio s'intende uno, che ha tutti i vizzj. Un moderno Poeta\footnote{Il Pulci},
  come notammo sopra \cstan[3]{66}. disse parlando d'uno di questi tali, che era
  moro.
  \begin{verse}
    Chiude un' anima bigia un corpo nero,
  \end{verse}
  L'origine di questa parola bigio in questo significato stimo, che nasca da questo.
  Erano in in Firenze ne' secoli passati tre fazioni, l'una de' Fautori di Fr. Girolamo
  Savonarola, la quale era detta de' \textit{Piagnoni}, l'altra de' contrarj a detto Fr.
Girolamo chiamata gli \textit{Arrabbiati}, o \textit{Compagnacci}; e fra di loro erano in tutto
nimici, e discordi, salvo che univano nell'esser contrarj alla terza fazione, che era
de' fautori de' Medici, la quale era detta de' \textit{Palleschi}, i quali non convenivano
ne con l'una, ne con l'altra fazione. Di questi che inclinavano alla fazione de'
\textit{Palleschi} talvolta alcuno per suoi fini particolari s'univa o con l'una, o con l'altra
delle prime due, ma era ricevuto con sospetto, che non fusse per spiare le loro
deliberazioni, e però dicevano: \textit{Non è da fidarsi di loro, perché son Bigi}\footnote{grigi,
quindi figuratamente indecisi.}. E da
questo forse ha havuto origine questa voce \textit{bigio} in significato d'huomo da non
se ne fidare. Vedi la Relazione di Firenze del Foscari, e il Nardi nelle Storie
Fiorentine lib. 2.
\end{description}
\section{STANZA XXII. --- XXVI.}
\begin{ottave}
\flagverse{22}Ch'entrar dovendo in Dite, e salta, e gira,\\
Che par quando mi barbera la trottola,\\
Andar non vi vorrebbe, e si ritira\\
Grattandosi belando la collottola;\\
Pur finalmente forza ve la tira,\\
Come fa il peso al grillo, una pallottola;\\
Così ne van quell'anime nefande\\
Chi dal piccin tirata, e chi dal grande.
\end{ottave}

\begin{ottave}
\flagverse{23}Per la gran calca nel passar le porte\\
Convenne a ognuno andarne con la piena, \\
Ma la Strega non hebbe tanta sorte, \\
Che tienla il can, che quivi sta in catena; \\
E perché per tre bocche abbaia forte, \\
Ella dice: Ti dia la Maddalena; \\
E in tanto trova il pane, e in pezzi il taglia,\\
E in tre gole ch'egli apre, gliene scaglia.
\end{ottave}

\begin{ottave}
\flagverse{24}Il Mostro, che mangiato havria Salerno,\\
Che quanto al masticar quei ser saccenti,\\
Voglion (perch'egli è guardia dell'inferno)\\
Tenerlo sobrio, acciò non s'addormenti,\\
Ond'è ridotto per il mal governo\\
Sì strutto, che e' tien l'anima co' denti,\\
Perch'egli è ossa, e pelle, e così spento,\\
Ch'ei par proprio il ritratto dello stento.
\end{ottave}

\begin{ottave}
\flagverse{25}Sì che quand'ei si sente il tozzo in bocca,\\
Perché la fame quivi ne lo scanna\\
L'ingozza, che ne manco non gli tocca\\
Ne di qua, ne di la giù per la canna,\\
Ma subito gli venne il sonno in cocca,\\
Ond'ei s'allunga in terra a far la nanna;\\
Ch'il papavero,e il loglio ch'è in quel pane\\
Faria dormir un' orso, non ch'un cane.
\end{ottave}

\begin{ottave}
\flagverse{26}Hor mentre fa il sonnifero il suo corso,\\
La donna che più là facea la scorta \\
(Perocché havea timor di qualche morso) \\
Vedendo che la bestia, come morta \\
Sdraiata dorme; e russa com'un'orso,\\
Legno da botte fa versa la porta,\\
E poi (bench'ella fusse alquanto stracca)\\
Dà una corsa, e in Dite anch'ella insacca.
\end{ottave}

L'anime rimaste attorno alla Città di Dite mostrano co' gesti, quanto malvolentieri
vadano dentro alla Città; ma i loro peccati a forza ve le tirano. Queste
anime nell'entrar della porta fecero così gran calca, che la Strega non potette
star con esse, e tanto più, che ell'hebbe paura di Cerbero; onde per liberarsene
gli gettò del pane fatto col sonnifero; per lo che il cane si addormentò, ed
ella entrò nella porta. E qui il nostro Poeta imita Verg. nel 6. dell'Eneide, dove fa
dare a Cerbero dalla Sibilla una stiacciata col sonnifero, e nelle presenti Ottave
23. 24. 25. parafrasa, si può dire, i seguenti versi del medesimo Vergilio.
\begin{verse}
Cerberus haec ingens latratu regna trifauci
Personat, adverso recubans immanis in antro
Cui vates horrere videns iam colla colubris
\backspace Melle soporatam, \& medicatam frugibus offam
Obijcit; ille fame rabida tria guttura pandens
Corripit obiectam, atque immania terga resolvit
Fusus humi, totoque ingens extenditur antro.
\end{verse}
\begin{description}
\item[BARBERA] Il verbo barberare è usato da' nostri fanciulli per intender quando
  la trottola gira a salti, e non va unita per cagione dell'esser mal contrappesata.
\item[TROTTOLA] Strumento, del quale si servono i ragazzi per giuocare, ed è un
  legno fatto a foggia di piramide, che finisce in una punta di ferro. Vedi sopra
  C.2. stan.23. e si fa girare avvoltandola con uno spago, e poi scagliando la in terra,
  tirando con velocità a se la mano, alla quale è legato detto spago.
\item[GRATTANDOSI la collottola] Grattandosi il capo nella parte di dietro da i
  Latini detta \textit{cervix}. E questo è un'atto solito farsi per lo più dalle donne, e da'
  fanciulli quando hanno qualche disgrazia, o gran disgusto. Vedi sopra Cant. 3.
\item[BELANDO] Vale piangendo: perché se bene il belare e proprio delle pecore,
  e simili, e viene dalla voce, che fanno tali bestie, che suona be be, ce ne
  serviamo anche per esprimere il pianto dell'huomo, ma per derisione; donde si
  dice \textit{belone}, \textit{pecorone} a uno che pianga assai. Un moderno Poeta disse:
  \begin{verse}
\backspace Hor ch'è per te finita la pasciona,
 Che fai che tu non beli, o pecorona?
\end{verse}
\item[GRILLO] È un verme piccolo volatile noto: Ma trattandosi di pallottole,
  \textit{grillo} s'intende quella piccola palla, che si tira per segno nel giocare alle pallottole,
  o alle piastrelle, o murelle. Vedi sotto in \cstan{34}, e \cstan[9]{17}.
\item[PALLOTTOLA] Intende quelle palle di legno, che servono per giuocare,
  nelle quali sono tre contrappesi di piombo, per via de' quali si fanno fare alle
  pallottole, l'operazioni, e voltamenti, che si vuole, l'uno di questi si chiama
  \textit{la catena}, l'altro \textit{il grande}, ed il terzo \textit{il piccino}, ed il Poeta, assomigliando
  quell'anime a queste pallottole, dice, che ancor'esse son forzate a entrar nell'inferno
  chi dal \textit{piccino}, e chi dal \textit{grande}, cioè chi da i peccati piccoli, e chi da i grandi.
\item[CALCA] Quantità grande di popolo; folla.
\item[ANDAR con la piena] Andar co' più; andar in truppa con tutte quelle anime,
  che \textit{piena} per similitudine significa inondazione, o furia di popolo. Virg. Georg.
  \textit{Mane salutantum totis vomit aedibus undam}. Andar con la piena significa ancora seguitar
  l'opinione comune; andar co' più.
\item[IL Cane che quivi sta in Catena] Cerbero cane con tre teste, due delle quali stanno
  sempre svegliate. Hercole lo legò, ed il nostro Poeta imitando Vergilio come
  s'è detto, lo fa addormentare col pane alloppiato.
\item[TI dia la Maddalena] Possa tu esser impiccato. Dicevasi porta di Caronte dagli
  Ateniesi quella porta del Palagio del Podestà, dond'uscivano coloro, che andavano
  alle forche, come accennammo sopra \cstan[5]{3}. e noi diciamo \textit{Ti dia
    la Maddalena}, da quella Campana, che è nella torre del Bargello, la quale suona,
  quando alcuno va alle forche, e si chiama la \textit{Maddalena}, perché con tal nome
  è battezzata, per esser la Cappella di quel Palazzo sotto i titolo di S. Maria Maddalena.
\item[GLIENE scaglia] Gliene tira da lontano; Glien' avventa, perché per la paura
  non segli volle accostare.
\item[HAVERIA mangiato Salerno] Havrebbe mangiato i sassi, Vergilio come sopra
  disse; \textit{fame rabida}. E si trova \textit{Baetylum voraret}, che \textit{baetylum} chiamarono quella
  pietra, che si divorò Saturno.
\item[SER saccenti] Si dice: Ser faccenti, o Barbassori quasi Valvassori, parola feudale,
  a coloro, che tutte le cose fanno, e dicono magistralmente, e da superiori
  degli altri; È però detto scherzoso, e per burlare uno. Qui intende i Governatori
  dell'Inferno. E' parola derivata dall'antico verbo \textit{saccio}, per \textit{so}: Latino
  \textit{sapio}.
\item[PER il mal gaverno] Per il poco mangiare, che gli danno. Nell'uso diciamo
  Governare le galline; cioè dar loro da mangiare. Similmente i Latini quando i
  soldati pigliavano un poco di rinfresco, dicevano; \textit{corpora curare}. Dall'istesso uso
  \textit{Governare gli ulivi} disse Pier Vettori, cioè concimargli; quasi questo sia un cibargli.
\item[Si strutto che tien l'anima co' denti] Sì macilente, e magro, che pare che esalerebbe
  l'anima, se non la ritenesse con lo stringer i denti. Giobbe per esprirere
  se medesimo emaciato, e consunto. \textit{Pelli meae, consumptis carnibus, adhaesit os
    meum}.
\item[EGLI è ossa, e pelle] Non ha carne addosso: È magrissimo, Plauto disse in
  questo proposito \textit{Ossa, atque pellis}. E Dant. Purg. C.23. dice;
  \begin{verse}
    \backspace Negli occhi era ciascuna oscura, e cava,
    Pallida nella faccia, e tanto scema,
    Che dall'ossa la pelle s'informava.
  \end{verse}
\item[SPENTO] S'intende al maggior segno magro.
\item[LA fame lo scanna] Muore di fame. Vedi sopra \cstan[4]{24}.
\item[CANNA] Intendi la canna della gola, la quale si dice canna per la similitudine,
  che ha il gargarozzo con la canna, Dan. Inf. C. 28.
  \begin{verse}
    Restato a riguardar per meraviglia
    Con gli altri, innanzi agli altri aprì la canna:
  \end{verse}
Onde Scannare, sgozzare: tracannare., ingollare, pha
\item[GLI viene il sonno in cocca] Cioè nell'estremità delle palpebre, che vengono a
  chiudersi, Gli vien voglia grandissima di dormire.
\item[S'ALLVNGA in terra] Si distende in terra, \textit{Immania terga resolvit Fusus humi,
  totoque ingens extenditur antro}; dice Verg. com'habbiamo accennato sopra.
\item[A FAR la nanna] A dormire. Termine insegnato dalle Balie a i bambini, che
  imparano a parlare, per esser più facile a dir nanna, che dormire, Lasca N. 2.
  \textit{Non lasciò mai certi detti che haveva imparato da bambino, chiamando pappo il pane, il
  vino bombo, i quattrini dindi, e quando voleva andare a dormire, diceva andar' a far
  la nanna}. I Latini similmente l'addormentarsi de' bambini alla \textit{Ninna Nanna},
  cantilena delle Balie, da lor detta \textit{Lallus}, e da' Greci \textit{Nynnius}; dicevano \textit{Lallare}.
\item[MENTRE il sonnifero fa il suo corso] Il sonnifero fa la sua operazione.
\item[PAPAVERO, e Loglio] Il papavero è quell'erba, il seme, ed estratto della
  quale compone l'oppio, o sonnifero; ed il loglio è un'erba, che nasce fra i
  grani, il seme della quale mangiandolo, dicono, che faccia sbalordire, e venir
  sonno. E da questi mali effetti del Loglio habbiamo un proverbio, che dice \textit{Io non
    dormo nel loglio}, che significa Io non son balordo.
\item[SDRAIATA] Vedi sopra C.3.stan. 32. \textit{sdraiarsi} è il verbo recumbere; E Vergilio
  dicendo: \textit{Tityre tu patulae recubans sub tegmine fagi}; stimo che intenda sdraiato
  senza pensiero alcuno te ne stai all'ombra d'uno spazioso faggio. E nota,
  che da questa voce \textit{patula}, che vuol dir largo, o spazioso, è stato cavato il verbo
  \textit{patullarsi}, che vuol dire trastullarsi, e passare il tempo senza pensieri, il che
  chiamano \textit{patullo}. Idiotismo assai usato.
\item[RUSSARE] Ronfare: Quel romore, che si fa da molti nel respirare dormendo,
  è il Latino \textit{stertere}.
\item[FAR legname da botte] Vuol dire accostarsi. Perché le doghe, e l'altre parti
  del legname da botte son lavorate in modo, che si compaginano, ed uniscono,
  quanto ognuno sa.
\end{description}
\section{Stanza XXVII. \& XXVIII.}
\begin{ottave}
\flagverse{27}Perché d'alloro ha sotto alcune rame\\
Vien fatta a' gabellier la marachella,\\
Tal ch'un di lor, ch'arrabbia dalla fame\\
Fermate (dice) olà, che roba e quella?\\
Ti gratterai (dic' ella) nel forame,\\
Perch'io non ho qui roba da gabella, \\
Se non un po d'allor c'a Proserpina \\
Porto, perch'ella fa la gelatina.
\end{ottave}

\begin{ottave}
\flagverse{28}S'ell'è come voi dite a questo modo:\\
(Ei le risponde) andate pur madonna,\\
Perch' altrimenti c'entrerebbe il frodo,\\
E voi staresti in gogna alla colonna,\\
Horsù correte pria che freddi il brodo,\\
Che la Regina poi sarebbe donna\\
Da farci per la stizza, e pel rovello\\
Buttar' a' pié la forma del cappello.
\end{ottave}

Martinazza havea sotto alcune rame d'alloro; e da i gabellieri le fu domandata
la gabella, ma essa con dire, che era per servizio di Proferpina, si libera,
dall'insolenza del gabelliere. 11 Poeta imita Vergilio, il quale fa che Enea d'ordine
della Sibilla porti a Proferpina il ramo di quell' albero con le foglie d' oro,
Come si vede al lib. 6, dell'Eneide.
\begin{verse}
  \makebox[8em]{\dotfill} Latet arbore opaca
  Aureus, \& folijs,\& lento vimine ramus
  Iunoni Infernae dictus sacer.
\end{verse}
\begin{description}
\item[MARACHELLA] Quella cosa mala; cioè la spia.
\item[ARRABBIA della fame] Ha grandissima fame, perché non guadagna denari
  da comprar roba per mangiare. Quando i mestieri non lavorano si dice: \textit{i legnaiuoli
    i sarti, i calzolai, ec, arrabbian della fame}, cioè non hanno da lavorare.
\item[TI gratterai il forame] Per beffar'uno, che dandosi a creder d'haver fatto
  qualche guadagno a spese, e dispetto nostro, e non t'ha fatto, diciamo: \textit{Tu ti
    gratterai il forame}. Qui vuol dire: tu credevi di haver guadagnato il quarto, che
  tocca alle spie, ma non è stato vero.
\item[Proserpina] Fu figliuola di Giove, e di Cerere, la quale fingono gli antichi
  Poeti, che essendo un giorno a corre i fiori, fusse rapita da Plutone Re dell'Inferno,
  e fatta sua moglie: Ma Cerere non potendo comportare, che la figliuola
  rimanesse appresso al rattore; supplicò Giove, che volesse levarla dall'Inferno,
  ed egli glielo concede, pur che ella non havesse preso cibo alcuno; Ma havendo
  Proserpina mangiato alcuni granelli di Melagrana non potette uscire; Cerere di
  nuovo supplicò, e stimolò tanto Giove, che ottenne, che Prosperina stesse sei
  mesi dell'anno nell'Inferno con Plutone, e sei mesi con la madre in Cielo. E
  così Proserpina restò sei mesi in Cielo, dove è chiamata Luna, e sei mesi
  nell'Inferno, dove è chiamata Proserpina, ed in Terra è chiamata Diana. E per questa
  triplicata essenza Verg. disse:
  \begin{verse}
    \backspace Tergeminamque Hecaten, tria Virginis ora Dianae.
  \end{verse}
  E perché la Luna sei mesi dell'anno cresce, e sei mei scema, però i Poeti
  Gentili fingono, che ella stesse sei mesi in Cielo, e sei mesi nell'Inferno, e tutto
  l'anno splenda in Terra, ed è detta Diana. A questa finzione allude Dan. inf. c. 10.
  \begin{verse}
    \backspace Ma non cinquanta volte sia raccesa
    La facia della Donna che qui regge.
  \end{verse}
\item[GELATINA] Brodo fatto con la carne di porco, s rappreso; e si fa anche di
  brodo di pesce. Vedi sopra \cstan[2]{55}
\item[C'ENTREREBBE il frodo] Ci farebbe la pena d'haver frodata, cioè non
  manifestata la roba, per non pagare il dazio, o gabella.
\item[IN gogna] Alla berlina, che è quel gastigo vituperoso, che dicemmo sopra C.
2. stan. 15.
\item[STIZZA] Ira, Vedi sopra \cstan[2]{78}, al termine \textit{su piccino}. E \textit{rabbia,
  rovello, collora, e simili}, i possono dir sinonimi di stizza, quando è presa in quello
  senso; che per altro diciamo stizza Una specie di lebbra, che viene a' cani, e
  ad altre bestie.
\item[SAREBBE donna] Questo termine significa Havrebbe animo: Si farebbe lecito,
  ardirebbe, non la guarderebbe, ed ha lo stesso significato, che \textit{Son poi fanti}
  detto sopra \cstan[4]{29}.
\item[BVTTAR a i pie la forma del cappello] Cioè buttar la testa a i piedi; Troncare
  il capo, che è la forma del Cappello.
\end{description}
\section{STANZA XXIK}
\begin{ottave}
\flagverse{29}La Maga senza dir più da vantaggio,\\
Mentr'egli aspetta un po di mancia, e intuona;\\
Ripiglia prontamente il suo viaggio,\\
E incontra Nepo già da Galatrona,\\
C'havendo dato là di se buon saggio,\\
In oggi è favorito, e per la buona,\\
Perché Breusse in oltre a' premi, e lode\\
L'ha di più fatto Diavolo a due code.
\end{ottave}

\begin{ottave}
\flagverse{30}Hor che gli arriva all' improviso addosso\\
Il venir della maga ch'è il suo cuore,\\
Lui Mago pur tagliatole a suo dosso\\
Le spedisce per suo trattenitore,\\
Mentr'il petardo col cannon più grosso\\
Sentesi fargli strepitoso onore,\\
Cavalier Nepo, com'io dissi dianzi,\\
Col riverirla se gli affaccia innanzi.\\
\end{ottave}

\begin{ottave}
\flagverse{31}E perché a Benevento essa con lui,\\
Com'ei di lei, havuto havea notizia,\\
Non prima si riveggon ch'ambedui\\
Rifanno il parentado, e l'amicizia.\\
Tra i diavoli poi van ne i regni bui,\\
E perché Martinazza v'è novizia,\\
E non intende il gracidar che e' fanno,\\
L'interprete fa egli, e il torcimanno.
\end{ottave}

\begin{ottave}
\flagverse{32}Per via informa, e le dà molti avvisi\\
D'usanza, e luoghi, e intanto di buon trotto\\
Lo guida a i fortunati Campi Elisi,\\
Dove si mangia, e beve a bertolotto.\\
E tra quei rosolacci, e fiordalisi\\
Si passa il tempo in far di quattro e d'otto,\\
Chi un balocco, e chi un'altro elegge,\\
Che lì non è un negozio per la legge.
\end{ottave}

\begin{ottave}
\flagverse{33}Quivi si vede un prato ch'è un'occhiata\\
Pien di mucchietti d'un'allegra gente,\\
Che vada pure il Mondo in carbonata,\\
Non si piglia un fastidio di niente;\\
Ma (com'io dico) tutta spensierata\\
Ballonza, canta, e beve allegramente,\\
Come suol far la plebe a gli Strozzini,\\
O sul prato del Pucci, o del Gerini.
\end{ottave}

\begin{ottave}
\flagverse{34} Quivi si fa al pallone, e alla pillotta,\\
Parte ne giuoca al Suffi, e alle Murelle,\\
Con le carte a Primiera un'altra frotta\\
I confortini giuoca, e le ciambelle,\\
Altri fanno a Civetta, altri alla lotta;\\
Chi dice indovinelli, e chi novelle;\\
Chi coglie fiori, e un'altro un ramo a un faggio\\
Ha tagliato, e con esso canta maggio.
\end{ottave}

\begin{ottave}
\flagverse{35}Più là un branco ha messo l'Oste a sacco,\\
Sì che tutti dal vin già mezzi brilli,\\
Mentre la gira fan brindis a Bacco,\\
Altri giuoca a te tè con paglie o spilli;\\
Altri piglia, o dispensa del tabacco\\
Altre piglia le mosche, un' altro grilli,\\
E tutti quanti in quei trastulli immersi\\
Si tengon il tenor, si vanno a i versi.
\end{ottave}


Martinazza seguitò il suo viaggio, e s'incontrò in Nepo da Galatrona molto
favorito da Plutone, il quale per fare onore a Martinazza da lui tanto amata,
glielo havea spedito per trattenitore, sapendo che erano amici. Così dunque
accompagnata da Nepo, che de faceva l'interprete, perché ella non intendeva il
parlar di quei Diavoli, se ne passò ne i Regni bui; ed il primo luogo, che veddero,
furono i Campi Elisi; li quali il Poeta descrive ripieni di quei trattenimenti
geniali, e fanciulleschi, che son soliti farsi dai bottegai più vili per le festività
ne i luoghi suburbani, come sono le Ville degli Strozzi, Pucci, e Gerini, dove
questa gente si posa per godere allegramente, e senz'un pensiero al mondo quella
libertà, che concede la campagna, e sospendere alquanto i pensieri noiosi del
lavorare.
\begin{description}
\item[MANCIA] Vedi sopra \cstan[2]{68}.
\item[INTVONARE] Vuol dire dar principio al canto; Ma qui significa chiedere
  con motti, o cenni la mancia; e ci serve per intendere domandare con cenni, o
  con motti qualsivoglia cosa: per esempio: Il tale \textit{insuona, vorrebbe andar' a cena},
  \textit{vorrebbe serrar la bottega}, ec.
\item[NEPO da Galatrona] Fu uno nel contado di Galatrona luogo nel Valdarno di
  sopra, il quele o con polveri simpatiche, o con altro medicava tutte le ferite, e
stroppi sì d'huomini, come di bestie, senza vedere il paziente ma solo in su le
 pezze bagnate nel sangue di esso, o sopra un panno, che havesse toccato lo stroppio;
 e per le bestie in qualsivoglia lor malore, pigliava la loro cavezza, o briglia,
 o capestro, e sopra quelli diceva alcune parole, e le medicava; e per questa
 sua diabolica superstizione da molti fu stimato stregone, come lo stima il Poeta
 dicendo, che s'era conosciuto con Martinazza a Benevento, e che era mago
 tagliatole a suo dosso.
\item[DAR buon saggio di se] Farsi conoscere con le sue azioni per huomo di garbo,
  e prudente, o virtuoso.
\item[È per la buona] S'intende, è per la buona strada; e vuol dire. È in buono
  stato; si tira innanzi bene.
\item[BREUSSE] Intende Plutone; ed è lo stesso, che la \textit{Biliorsa}, colla qual voce
  fanno paura le Balie a' bambini, forse dal Lat. \textit{Erebus}, originato così: \textit{Erebusse},
  \textit{Breusse}.
\item[L'HA fatto diavolo a due code] L'ha privilegiato. Il poeta s'è ricordato qui
  del proverbio \textit{Haver la lucertola a due code}, che vuol dir Esser affortunato, perché
  fra la gente del cervello debole corre una superstiziosa voce, che uno che tenga
  addosso una lucertola con due code sia fortunatissimo in ogni cosa, ma
  particolarmente nel giuoco, e perciò vuol dire, che questo Nepo era fortunatissimo, e
  grandemente privilegiato da Plutone, perché haveva le due code.
\item[GLI arriva addosso] Cioè sopraggiunge inaspettatamente a Plutone la Maga
  Martinazza tanto amata da lui.
\item[TAGLIATOLE suo dosso] Fatto per appunto come lei, che ha i medesimi genj,
  ed inclinazioni, che ha lei. Traslato da' gli abiti, che si dicono tagliati a suo
  dosso quando tornano bene in dosso.
\item[TRATTENITORE] Si dice quel Cortigiano, che vien deputato a servire uno
  Ambasciatore, o altro forestiero, che sia ricevuto, e spesato dalla Corte.
\item[PETARDO] Specie d' artiglieria nota, che serve per buttare a terra, le porte
  delle Città. In Latino fa detta da \textit{Famiano Strada} con' voce Greca
  \textit{Pyloclastrum}, Quasi Spezzaporta.
\item[RIFANNO il parentado, e l'amicizia] Quando due amici stati lungo tempo
  lontani l'uno dall'altro senza vedersi, si ritrovano insieme, e fanno le cirimonie
  diciamo: \textit{Rifare il parentado, e l'amicizia}.
\item[V'È novizia] Non v'è pratica, perché non v'è mai stata in quel luogo. Lat.
  \textit{hospes}, e noi lo traslatiamo ad uno, che è nuovo, e non praticato in qualche
  affare. Lat. \textit{novus}, \textit{rudis}.
\item[GRACIDARE] E' proprio delle ranocchie, ma qui intende il parlar de'
  Diavoli, che forse se lo figura come quello delle ranocchie. Dan. Inf. C.32 dice:
  \begin{verse}
    E come a gracidar si sta la rana.
  \end{verse}
\item[INTERPRETE, e Turcimanno] Si possono dir sinonimi, se non che \textit{Interprete} è
  propriamente quello, che esplica i sensi delle parole, e \textit{Turcimanno} è quello che
  parla in vece di colui, che non intende il linguaggio, riportando le parole; che
  sente dire nella lingua dell'uno, e dell'altro respettivamente. Da alcuni dicesi
  \textit{Dragomanno} dalla voce Greca \textit{Dragomenos}, che significa \textit{Interprete} usata da' Greci
  Orientali de' tempi bassi; da \textit{Thargum}, che in Levante significa \textit{interpretazione}.
  \textit{Thirghem} in Caldeo vale \textit{esporre}, \textit{esplicare}, e da questa radice è detta specialmente
  \textit{Thargum} la Parafrafi Caldea della Scrittura. Ma hoggi \textit{Turcimanno} da i più s'intende
  ruffiano da quel portare le parole.
\item[DI buon trotto] Camminando di buon passo. Trotto diciamo una specie d'andare
  del Cavallo, che è fra il passo ordinario, ed il correre, ed è il latino
  \textit{succussare}.
\item[CAMPI  Elisi] È il creduto Paradiso de i Gentili. Vedi sopra C,2. stan. 68:
\item[BERTOLOTTO] Senza pensare al pagamento, che si dice anche 4 Vf
a Youne; a Scrocco; a Salicone, Vedi sopra C, 1stan.77, e sotto \cstan[7]{5}.
\item[ROSOLACCI, e fiordalisi] Specie di vilissimi fiori silvestri.
\item[FAR di quattro, e d'otto] Se ben par che voglia dire giuocare inuitando di
  quattro, e d'otto; tutta via s'intende starsene senza far nulla, che si dice anche
  \textit{far a teco meco}, \textit{dondolarsela}, \textit{fare a tu me gli hai}, onde un nostro Poeta moderno
  disse:
  \begin{verse}
    Voi dal notturno al mattutin crepuscolo
    Vi dondolate, e fate a tu me gli hai,
    Ne proponete, o concludete mai, ec,
  \end{verse}
\item[BALOCCO] Passatempo. Trattenimento. Da \textit{Badalucco}, che vol dire propriamente
  scaramuccia, o leggiere combattimento. Latino \textit{velitatio}, e figuratamente
  trastullo, o trattenimento piacevole. Ma la parola \textit{balocco}, o \textit{baloccarsi} è
  usata per lo più co' bambini; e nel contado è preso per indugiare.
\item[È UN'OCCHIATA] È grandissimo, quasi dica spazioso tanto quanto un'occhio è
  bastante di vedere.
\item[MUCCHIETTI] Diminutivo di \textit{mucchio}, che vuol dir quantità di cose ristrette insieme,
  quasi monticelletti, Latino \textit{cumuli}, \textit{acervi}; e così \textit{mucchietti di gente}
  vuol dire truppe d'otto, o dieci persone ristrette insieme. Dan. Inf, C, 27.
  \begin{verse}
    E di Franceschi sanguinoso mucchio
    Sotto le branche verdi si ritrova.
  \end{verse}
\item[VADA pure il mondo in carbonata] Diventi carbone, e abbruci pure il Mondo,
o vero rovini, e vadia sottosopra il Mondo.
\item[NON si piglia un fastidio di niente] Non vuol sentir noia, o pigliarsi pensiero
  alcuno, succeda quel che si vuole, o dibene, o di male.
\item[BALLONZOLARE] Ballare senz' ordine, o regola. Vien forse da \textit{Ballonchiare}
  e \textit{Ballonchio}, che se bene è parola non usata, pur l'usò il Boccaccio Nov. 72. per
  intender ballo di contadini.
\item[ALLI Strozzini] Gli Strozzini, come habbiamo d., è una villa de' SS.Strozzi poco
  lontana da Firenze, così detta. Sì come il Prato del Pucci, e del Gerini sono due
  ville suburbane de' SS. Marchesi Pucci, e Gerini; a' quali luoghi, suole l'estate
  andar la plebe Fiorentina a spassarsi, con far merende, balli, ed altro, che le torni
  gusto, come dice il Poeta nelle presenti Ottave.
\item[AL pallone, e alla pillotta] Il pallone è una grossa palla da giuocare fatta di
  quoio, e ripiena di vento, alla quale si dà con il braccio armato d'un bracciale
  di legno: e la pillotta e una palla piccola pure ripiena di vento, e se le dà con
  una mestola di legno. Questi giuochi di palla, sono antichi, perché secondo Plinio
  lib.7. c.59. furono trovati da un certo Pytho. Herodoto lib. 1, riportato da
  slid. Verg. \libcap[2]{13}. dice, che l'inventafiero i Lidi.
  \begin{adjustwidth}{8pt}{}
    Alea vero tesserarumque
    ludos, \& pilae, caeteraque lusoria recreandi animi gratia inventa,
    praeter quam talaria, Lydi populi Asiae omnium primi, excogitavere \&c. Atqui
    Lydos eiusmodi aleatorias artes non tam voluptatis, quam compendij gratia
    excogitasse idem Herodotus tradit, nam cum gravitate annonae patria tempore
    Atydis Manis Regis filij premeretur, sic famem consolari solebant, altero
    quidem die cibum sumentes, altero ludis operam dantes; atque hoc modo
    inediam solantes vixere annis duodeviginti.
  \end{adjustwidth}
  E da' popoli \textit{Lydi} alcuni vogliono,
  siccome è Isidoro nelle Origini, che venga la parola \textit{Ludus}, o \textit{Ludius}, che
  è lo stesso, che Istrione. E ognun sa, che i Lidi dall'Asia passarono in Italia, e
  vi popolarono l'Etruria, ovvero Toscana; E da loro i Latini le cirimonie sacre,
  e i \textit{ludi}, che si domandavano \textit{scenici} particolarmente appresero; E \textit{Hister} in lingua
  Etrusca, onde è detto \textit{Istrioni} significava in Latino \textit{Ludio} siccome dice Tito
  Livio. Poi questo nome \textit{ludus} significante a principio spettacolo attenente, o fatto
  per causa di religione, si stesse a significare: in generale ogni giuoco. Ateneo
  lib. 1, e Suida dicono, che Anagallide Gramatica di Corfù attribuisce il ritrovamento
  della saltazione a palla, cioè del giuocare alla palla a tempo di ballo
  a Nausica figliuola d'Alcinoo Re di Corfu; volendo fare questa grazia di dare
  il vanto d'una tale invenzione a una sua paesana. E veramente Nausicaa, è introdotta
  sola tra l'Eroine da Omero a giuocare alla palla. Del reflo Dicearco
  attribuisce quest'invenzione a' Sicionj, e Hippaso altro Autore citato da Ateneo
  a' Lacedemoni, come anche tutti gli altri corporali esercizzj. E che fusse molto usato
  dagli Spartani, o Lacedemoni lo mottra Properzio in quel verso. \textit{Quum pila
    veloce fallit per brachia iactu}, dell'Elegia che cominicia, \textit{Multa tuae, Sparte,
    miramur iura palaestrae}. Dal che si viene in chiaro, che il giuoco della palla sia antichissimo,
  e si può credere col Soutero de lud. Veterum lib. 3. C. 14. e con Polid.
  Verg. \libcap[2]{13}. che questa variazione d'origini proceda dall'havere havuto
  gli antichi diverse specie di palla, sì come habbiamo noi, e che gli accennati inventori
  habbiano ciascuno inventata la sua specie: perché se noi habbiamo il pallone,
  i Latini havevano, \textit{ipse follis, pila, \& ipsis genus; constatque aluta vento
  inflata}. Habbiamo la pillotta, \& essi il \textit{folliculus, pila, \& ipsa parva, \&
    similiter constat aluta vento inflata}. Simile a questa è la palla bonciana, ma
  in vece d'esser ripiena di vento, è ripiena di borra\footnote{tecnicamente, la \textit{borra} sarebbero i resti della tosatura, non adatti ad esser filati, ed usati per imbottitura; per estensione: riempitivo in genere. }, la qual palla hoggi per lo
  più è usata da i contadini, o questa havevano anche gli antichi e la dicevano \textit{Pila paganica}.\\
  Marz.lib.14.
  \begin{verse}
    \backspace Haec, quae difficilis turget paganica pluma,
    Folle minus laxa est, \& minus arcta pila.
  \end{verse}
  Habbiamo la palla simile alla bonciana, ma assai minore, che chiamiamo \textit{palla
    lesina}, che pure l'havevano anche secondo alcuni i Latini, e la dicevano \textit{Pila
    fluentina}, perché forse nel paese Fiorentino si lavorassero le migliori. Habbiamo
  la palla fatta di cenci impuntita, che i Latini pure havevano, e la chiamavano
  co' Greci \textit{Phaennida}, o vero \textit{Harpastum}, perché se ne servivano per far il giuoco
  da noi detto il Calcio secondo il Sipontino, che dice:
  \begin{adjustwidth}{8pt}{}
    Harpastum pilae genus
    est; grossior, quam pila paganica, tenuior, quam follis; E panno fere fit,
    aliquando ex pelle, lana, comentove impletur. Non repercutitur, sed cum
    multi sint ludentes in duas partes divisi, ita ut utrique e regione sibi invicem
    oppositi sint, ad suos quisque transmittere pilam conatur, quam adverlarij
    conantur arripere; \textit{Harpastum} dictum a Graeco \textit{Harpazin}, quod est rapere, quia
    proiectam pilam multi simul conantur arripere, sed ob eam causam invicem
    prosternuntur.
  \end{adjustwidth}
  Marz. lib. 7. ep. 31. \begin{verse}Non harpasta vagus pulverulenta rapis
  \end{verse}.
  Habbiamo la palla a corda, che serve per giuocare con la racchetta nelle stanze
  fabricate per tale effetto; ed essi havevano \textit{pilam trigonalem}, così detta non perché
  futie di figura triangolare, ma perché era triangolare la stanza, dove con essa giuocavano,
  e per dare a questa palla si servivano del \textit{reticulo}, che è lo stesso che la
  racchetta, o lacchetta, come accennammo sopra C.3. tan, 58. Di questa lacchetta
  parla Ovid. lib. 3.
  \begin{verse}
    \backspace Reticuloque pilae laeves fundantur aperto,
    Nec, nisi quam tollas, ulla movenda pila est.
  \end{verse}
  e Marziale lib. 12.
  \begin{verse}
    \backspace Captabit tepidum dextra, laevaque trigonem.
  \end{verse}
  Che poi a' tempi antichi usasse la palla ripiena di borra o d'altro pelo, si cava
  da quel che dice il Sipontino riportato qui sopra, e dal nome di esfa, perché
  molti vogliono, che sia detta \textit{Pila} dal pelo, col quale è ripiena; se bene altri voglino,
  che venga dal Greco \textit{Peleo}, idest \textit{aequo}, perché  di figura sferica, che è
  uguale in ogni parte, o pure (il che è più: verilimile) dal \textit{pallesthai}, cioè
  \textit{dall'essere librata, e sbalzata}, e perciò anche in Greco, si come in Toscana è detta
  Palla, poiché Dionisodoro antico gramatico, dove nel testo dell'Ulissea comunemente
  leggevasi \textit{Spheran}, col qual nome chiamano i Greci \textit{da palla}; si dice,
  che egli scrivesse \textit{Pallan}, come per chiosa, e interpetrazione della voce d'Omero;
  e tutto questo vien riferito da Eustazio, che sopra quel Poeta \textit{il gran comento fa}.
  Che i Greci ancora havessero molte specie di palle, si può dedurre non solo
  dall'essere stati inventati i giuochi di palla nel tempo, che fiorivano i Greci, e
  dal trovarsi appresso di loro la Spheromachia, l'Amilla, ed altre specie di giuochi
  di palla riferiti da Giulio Polluce, e dal Bulengero; ma da quello, che scrive
  Celio Rodigino lib. 20, C. 14. dove dice, che fra i Greci giuocavano alla palla
  tanto gli huomini, che le donne; e ciò cava da Homero. Si trova in oltre, che
  Dionisio Siracusano giuocava alla palla, ed alla pillotta per ricuperar le forze.
  Alex. ab Alex. dier. gen. lib. 3. c. 21. E si può credere, che si come noi habbiamo
  diverse palle, e diversi modi di giuocar con esse, così non mancassero a loro ancora
  l'invenzioni per soddisfarsi.
\item[AL sussi] È un giuoco solito farsi per lo più da ragazz' in questa maniera. S'uniscono
  due, o più ragazzi, e pigliano una pietra, e posatala per ritto in terra vi
  metton sopra quel danaro, che son convenuti di giuocare, ed allontanatisi in
  distanza, che sono d'accordo, tirano una lastra per uno ordinatamente
  in quella pietra ritta; sopr'alla quale sono i denari, e che si chiama il Sussi, e se
  questo sussi vien colpito, e fatto cadere, i danari, che cascano, sono di colui, la lastra
  del quale ha fatto cascare il sussi, se però sono più vicini alla sua lastra, che al
  sussi, e quella moneta, che è più vicina al sussi, le gli rimette sopra, e quello a
  cui tocca ira, e seguitano, come sopra, tanto che la moneta messa sopra al sussi
  resti finita di levare nel modo, che s'è detto. Da questo giuoco: habbiamo un
  proverbio che dice \textit{Esser il sussi}, il che significa esser quel berzaglio, dove ognuno
  tira, cioè sopra il quale devon cadere tutte le burle, e tutte le minchionature.
  Questo giuoco è forse lo stesso, che da' Greci era detto \textit{Ephedrismo}, secondo
  Giulio Polluce, il Buleng. c. 48., ed il Meurs, de lud. Graecor.,ste bene non giuocavano
  denari, ma colui, che non buttava in terra il sussi, portava a cavalluccio
  quello, che lo buttava, il quale gli turava gli occhi colle mani, finché senza errare
  lo portale alla lastra, o pietra, che si chiamava \textit{dioros}, cioè \textit{Meta} o \textit{Confine}, e
  faceva quello, che comandava il vincitore, il quale in questi loro giuochi era,
  chiamato Re, ed il perditore era detto Mida, o vero Asino, come habbiamo visto altrove.

\item[MURELLE] È giuoco simile alle pallottole, se non che in vece di palle adoprano
  lastrucce, ed un piccolo sasso per grillo, e tal giuoco si dice anche \textit{piastrelle}.

\item[PRIMIERA] Giuoco noto, che si fa con le carte.
\item[FROTTA] Flotta, o fiotta. Vuol dire quantità di gente unite insieme, che si
  muove; dal Latino \textit{fluctus}, Virg. Georg. \textit{Mane salutantum totis vomit aedibus undam}.
  Varchi Stor.lib. 15. \textit{E vedendo sopra a un monticello non molto quindi lontano una gran
    frotta di contadini}.
\item[CIAMBELLE, e confortini] Sono specie di pafie fatte col zucchero farina, e
  uova, e queste son portate a vender da alcuni più pel contado, dove si fanno feste,
  e raddotti, che in Città; e questi portan seco anche le carte per giocare, oltre alle
  quali hanno diverse invenzioni di giuochi, come la mora, il tocco, ec. E questi
  venditori quando giuocano, danno in vece di danari quei \textit{confortini}, e \textit{ciambelle},
  se perdono; e se vincono, ricevono danari. L. \textit{circuli}, \textit{crustula}.
\item[CIVETTA] Quel giuoco fanciullescho, che dicemmo sopra \cstan[2]{41}.
\item[INDOVINELLI] Latino \textit{griphi}, \textit{aenigmata}; Quello che in latino dal greco si dice
  \textit{enigma}, noi circoscrivendolo diremmo \textit{detto oscuro, e difficile a interpretarsi}.
  E la voce enigma s'è fatta Toscana, e l'usiamo come l'usò il Malatesti nella
  sua Sfinge, Vedi sotto \cstan[8]{26}.
\item[CANTA Maggio] Nel principio di Maggio sogliono le Ragazze della plebe
di Firenze, o del Contado suburbano accordarsi tre, o quattro, e portando
una di loro in mano un ramo d'albero adornato di fiori andar cantando per la Città
diverse canzonette per l'allegria del nuovo maggio, e per buscar mance da coloro,
che si pigliano il passatempo di farle cantare al suono d' uno strumento detto
cembolo, che è un'assicella ridotta in cerchio, e fondata di cartapecora da una
parte sola a guisa di tamburo. Questo costume di rallegrarsi il Maggio viene dall'antico,
e si trova, che appresso i Romani \textit{Kalendis, Nonis, \& Idibus maij Lari
Deo sacra fiebant asello panibus coronato}, Quindi forse ancora Maggio si chiama il
mese de gli Asini, che per altro fu detto \textit{mensis hilaritatis}. Che nel mese di
Maggio si facessero allegrie forse più di quello, che comportasse  l'onestà, e la
verecondia, ne fanno fede gl'Imperatori Arcadio, e Onorio nella loro Costituzione
inserita da Giustiniano nel Codice lib. 11. 45. \textit{de maiuma}, la quale era una
allegria, che si faceva per il Maggio secondo che spiega Suida. Da questo mese quel
ramo d'albero, che i contadini piantano la notte di Calen di Maggio avanti all'uscio
delle loro innamorate, si chiama \textit{Maio}; Questo costume d'appiccare il
maio alla casa della Dama è riferito come proprio anche della Francia da Marziale
d'Alvergna ne' suoi Arresti d'Amore, all'Arresto quinto, il quale scrittore
fiorì nel 1400. Qual luogo Benedetto Curzio comentando dice: \textit{Prima die
  Maij mensis iuvenes pluribus ludis, ac iocis sese exercere consueverunt, arborem seepenumero
  deportantes, ac in loco publico, aut etiam ante alicuius egregii viri januam, vel
  frequensius amicae fores plantantes, vestitam nonnumquam promiscuis adamantibus, intersignijs,
  atque emblematibus}.
\item[BRANCO] Quantità di popolo indeterminata; ma si dice più di bestie; come
branchi di polli di pecore, di buoi, di asini, ec, Vedi in questo C. l'Ottava 37.
seguente.
\item[HA messo l'Oste a sacco] Cioè mangiato, e bevuto quanto l'Oste vi haveva
  nel modo, e con quella furia, che segue nel dare il sacco a una Città.
\item[MEZZI brilli. Mezzi briachi] Brillo vuol dir briaco allegro. Vedi sopra \cstan[2]{69}.
\item[MENTRE la gira, fan brindis a bacco] Una villantella che si canta per incitare
  a bere, principia:
  \begin{verse}
    Facciam brindis a bacco,
  \end{verse}
  E cantandosi questa, va il bicchiere attorno, ed ognuno  beve, intuonando prima
  la detta Villanella, e però dice \textit{mentre la gira}, cioè mentre il bicchiere va a torno:
  E perché tal costume è usatissimo in simili allegrie, però il Poeta, che s' inpegna
  di mostrar, che quivi si sta in feste, e in giuoco, dice che facevano \textit{brindis a bacco},
  cioè cantavano bevendo. I Latini dicevano \textit{Propinare}, cioè \textit{praebibere} dal
  Greco \textit{propinein}, che suona lo stesso che il far brindis, ed usavano anch'essi questo
  modo di bere in giro, che dicevano \textit{in orbem bibere, \& circumferebant scyphum
    plenum}, ed essi pure cantavano in tale occasione di bere; come scrive Dione, che
  facesse il Senato Romano a Commodo Imperatore, quando al banchetto, che fece
  nel Teatro, bevve a un bicchiere, che li fu porto da una bella femmina.
\item[LA voce brindisi] Se ben pare che venga dal Tedesco \textit{pringen}, perché volendo
  alcuno di quella nazione bere, ed invitare il compagno, suol dire: \textit{Ick Vellan
    pringen} che vuol dire \textit{Io ve lo presento}; e questo già facevano, perché quel vino,
  che havevano a bere restasse benedetto dal Compagno, il quale soleva rispondere
  \textit{Got zenges}, che vuol dire Dio lo benedica. Tuttavia il Lalli nella sua Moscheide
  C.1. stan.61. graziosamente gli da origine dalla Città di Brindis, dove chi va ad
  abitare è sicuro da ogni vessazione curiale tanto Criminale, che Civile, onde a
  far \textit{brindisi} par che s'inviti uno ad andar ad abitar quella Città, cioè a lasciar i
  pensieri; le parole del Lalli son queste:
\begin{verse}
\backspace  Brindisi bella s'io m'appongo al vero,
  Da te son messi i brindisi in usanza,
  Quasi l'huom dica; Lascia ogni pensiero;
  Beviamo allegri, e rinfreschiam la panza,
  E se poi il creditor duro, e severo
  Ci fa da' birri apparecchiar la stanza,
  Brindisi habbiamo, Brindisi diletta,
  Che quanto più si bee, viè più n'alletta.
\end{verse}
\item[TE TÈ con paglie, o spilli] E' un giuoco da' fanciulli, che si fa così: Pigliano
  due spilli, o due corte fila di paglia, e posandole sopra un piano liscio vanno
  spingendole con le dita tanto, che uno di detti spilli, o fili cavalchi l'altro, e quello,
  che resta di sopra vince, giuoco così detto dal \textit{Tetè}, cioè \textit{togli, togli}. In
  Latino \textit{ludere aciculis}. E perché questo giuoco è di niuna, o poca conchiusione, habbiamo
  il proverbio: \textit{Fare a tetè con gli spilletti}; che significa affaticarsi, e perdere
  il tempo senz'utile, o profitto; ed esprime ancora \textit{Far una cosa con sordido
risparmio}.

\item[SI tengone il tenor, si vanno a' versi] S'aiutano l'un l'altro, e s'accordano.

\end{description}
\section{Stanza XXXVI. --- XXXX.}

\begin{ottave}
\flagverse{36}La donna resta lì trasecolata,\\
Vedendo quanto bene ognun si spassa; \\
E perché Nepo l'ha di già informata, \\
Non ragiona di lor, ma guarda, e passa: \\
Per tutta la Città vien salutata,\\
E infin le stanghe, e ogni forcon s'abbassa,\\
Ed ella hor qua, hor là voltando inchini,\\
Pare una banderuola da cammini.
\end{ottave}

\begin{ottave}
\flagverse{37}Però che tutti quanti quei Demoni, \\
Per vederla, n'uscian di quelle grotte, \\
Ronzando com' un branco di moscioni,\\
Che s'aggirin d'attorno a una botte, \\
Saltellan per de strade, e sui balconi, \\
Com'al piover d'agosto fan le botte, \\
E fan, vedendo sue sembianze belle,\\
Voci alte, e fioche, e suon di man con elle.
\end{ottave}

\begin{ottave}
\flagverse{38}Così fra quel diabolico rombazzo\\
La strega se ne va con lo stregone,\\
Si c'alla fine arrivano a Palazzo \\
La dove s'abboccaron con Plutone.\\
Ma perché tra di loro entrò nel mazzo\\
Scioccamente il Mandragora buffone, \\
Ch'in quel colloquio fe sì gran frastuono, \\
Che finalmente ognuno uscì di tuono.
\end{ottave}

\begin{ottave}
\flagverse{39}Perciò passano in casa, e colà drento\\
Tirato con la Strega il Re da banda,\\
Le dà la ben venuta, e poi, che vento\\
L'ha spinta in quelle parti, le domanda,\\
Ella per conseguir ogni suo intento\\
Gli dice il tutto, e se gli raccomanda,\\
Ch'ei voglia a Malmantil, c'omai traballa\\
Far grazia anch'ei di dar un po di spalla.
\end{ottave}

\begin{ottave}
\flagverse{40}Sta pur, dic'ei, con l'animo posato,\\
C'a servirti mo mo vuò dar di piglio,\\
Io già, come tu sai, haveo imprunato;\\
Ma il tutto è andato poi in iscompiglio\\
Horsu: fra poco adunerò il senato,\\
E sopra questo si farà consiglio,\\
Acciò Baldon batta la ritirata.\\
E tu resti contenta, e consolata.
\end{ottave}

Martinazza resta maravigliata, che costoro stieno così allegramente, e passando
pel mezzo a una infinità di Demonj, che tutti la riveriscono, giunse con
Nepo a Palazzo, dove se le fece incontro Plutone, che la condusse dentro, e
quivi havendole essa detto il suo bisogno, Plutone le promette di consolarla.
\begin{description}
\item[RESTA trasecolata] Resta maravigliata: Strabilisce. Vedi sopra \cst[1]{28}.
\item[STANGA] Pezzo di travicello, cioè un legno grosso più d'un bastone.
\item[FORCONE] E'un'asta di legno, sopra alla quale è adattato un tridente di
  ferro, e serve per uso delle stalle.
\item[INCHINO] Vedi sopra C.1. stan. 34:
\item[BANDERVOLA da Cammini] Banderuola vuol dir piccola bandiera, o
  pennoncello, che è quel pezzetto di drappo, che già portavano i Cavalleggieri
  appiccato vicino alla punta della lancia a guisa di bandiera; ed a guisa di questa in
  Firenze se ne vedono fatte di lama di ferro poste in più eminenti luoghi delle
  case, come sono le pergamene, dond'esce il fumo dei cammini, e queste servono
  per far conoscere i venti col lor girare, e voltare in sul ferro nel quale sono
  infilate, e bilicate; ed a queste allomiglia Martinazza.
\item[RONZANDO] Ronzare si dice propriamente delle mosche; e però \textit{dice
  come fanno i moscioni}, che sono quei piccoli vermi alati, che nascono dal vino.
\item[COME fanno le botte al piover d'Agosto] S'è veduto dalla sperienza, che la pioggia
  di state, cascando nella polvere scaldata dal Sole invigorisce le rane, o botte
  nate di poco, se bene molti hanno creduto, che le faccia nascere quell'acqua, con
  quel Sole; il che è falso, perché prese subito scappate dalla polvere si son trovate
  col ventricolo pieno d'erba. Ma sia come si voglia, basta che a tal'acqua si veggono
  saltar, ma d'un salto debole, e fiacco, appunto come il Poeta vuole esprimere,
  che saltassero quei Diavoli. Un Poeta faceto Fiorentino\footnote{il Burchiello, Il sonetto citato è ``Per la Gente del Re''} descrivendo alcuni
  cavalli stanchi in un suo sonetto dice:
  \begin{verse}
    \backspace Sì si vergognan che passan di notte,
    E tutti s'inginocchian per la fame
    Trottando, e saltellando come botte.
  \end{verse}
\item[VOCI alte, e fioche, e suon di man con elle] Così canto Dante Inf. C. 3.
\item[VOCI alte] Intendi strida, e colui, che continova a gridare, affioca per l'affaticamento
  dell'aspera arteria, sì che il secondo nasce dal primo. \textit{E suon di man
    con elle}; cioè con quelle voci accompagnano il romore: che fanno col batter le
  mani insieme.
\item[ROMBAZZO] Rombazzo vien dal verbo rombare, che vuol dir ronzare, o frullare,
  che è quel romore, che fa per l'aria una cosa lanciata con violenza, e si
  piglia per ogni sorta di strepito, o fracasso. Il Varchi Stor. lib. 10. in questo medesimo
  significato dice \textit{bombazzo} voce formata dal suono, nella stessa maniera, che
  Persio formò \textit{bombus}: \textit{Torva Mimalloneos implerunt cornua bombos}, perché dice
  egli: \textit{Dopo lunge strombettate, e stampite fatte con incredibile Bombazzo, quasi in tal
    modo salutassero i nimici}. Ma l'Autore della Storia di Semifonte dice al trattato 4.
  \textit{I nimici assaltarono la Terra, allotta sentitosi per quelli della Città il rombazzo}, E l'uso
  pare che ci obblighi a dire \textit{rombazzo}.
\item[SI mette nel mazzo] S'accompagnò con loro, Che diciamo; \textit{incruscarsi, ficcarsi}:
  Vien dal giuoco del mazzolino detto sopra \cstan[2]{46}.
\item[IL Mandragora], Costui era un buffone, o più tosto un matto di Corte, che
  chiacchierava sempre, e senza proposito, o conchiusione.
\item[COLLOQUIO] Voce latina usata di rado in Firenze, e vuol dire Ragionamento,
  che fanno insieme due, o più persone. Corrisponde alla Greca \textit{Dialogos}, che
  significa secondo la parola \textit{interlocutio}, discorso che si tiene fra due, o più persone;
  dai Franzesi detto \textit{Entretien} quasi \textit{Trattenimento}.
\item[USCÌ di tuono] Perde il filo del ragionamento, si dice anche: \textit{uscir di tema},
  smarrire l'argumento, il proposito. Vedi sopra \cstan[2]{47}. E' presa la similitudine
  della Musica, scherzando sul doppio significato della parola scordarsi la quale tanto
  si dice d'un'huomo, che non si ricordi più di quel che ha proposto di dire;
  quanto d'uno strumento, che non sia in corde, e non sia temperato al giusto
  tuono, o d'uno, che non canti giusto, e fuor del legitimo tuono, il che si dice
  anche \textit{stonare}.
\item[TIRATISI de banda] Condottisi in un'altra parte della stanza, separatisi, o
  allontanatisi da quel congresso.
\item[CHE vento l'ha spinta in quelle bande] Qual cagione l'ha mossa a andar in quel luogo.
\item[TRABALLARE] E' quell'ondeggiamento, che fa uno, quando non può sostenersi
  in piedi, Mattio Franzesi in lode della Posta dice.
  \begin{verse}
    \backspace Chi domanda per nome la cavalla,
    Ch'egli ha sentito dir, ch'è favorita,
    Poi partendo chi trotta, e chi traballa.
  \end{verse}
  Qui vuol dire, che Malmantile era in pericolo di cadere, cioè esser preso da Baldone.
  Diciamo in questo senso anche \textit{balenare}, \textit{barcollare}. In certe rime
  manoscritte nella Libreria di S. Lorenzo, si dice d'un cotto, che barcollava: \textit{E s'e'
    balena, e' non balena a secco}\footnote{attribuita a Lorenzo de' Medici, il Magnifico.}. Qui si nea sul doppio significato di balenare.
\item[MO' mo'] Adesso adesso. È il latino \textit{modo}. Usato in Lombardia, e poco in
  Firenze. L'usò più volte Dante nel suo Poema, sì come non è stato schifo d'usare
  altre parole Lombarde; E il Bocc. Nov. 32. \textit{Mo vidi vu?} per imitare la parlata della donna, ch'era Veneziana\footnote{Curioso che il Minucci non menzioni che \textit{mo' mo'} è l'espressione napoletana per esprimere il concetto di ora, subito, immantinente.}.
\item[DARO' di piglio] Darò di mano, cioè comincerò. Appresso gli antichi
  significava quasi quel, che i Latini dissero \textit{Expilare}; i Franzesi \textit{piller}. Dante Inf. 12. \textit{Che
dier nel sangue, e nell'aver di piglio}, E 'l suo contemporaneo \textit{Fazio degli Uberti} nel
Poema che fece in terza Rima, ove è introdotto Solino a dettare a Fazio le cose
di geografia, e del mondo; che perciò lo intitolò \textit{Dicta mundi}, ovvero \textit{Ditta, mondo},
dice così al canto 142. (Parla del Saladino)
\begin{verse}
\backspace Costui per sua franchezza, e gran consiglio,
Tolse la terra santa a' Cristiani,
\backspace Vincendo quegli, e dando lor di piglio.
\end{verse}
\item[HLAVEO imprunato] Havevo ordinato il rimedio, Vien da quell' imprunare,
  che dicemmo sopra \cstan[3]{21}, Addio fave.
\item[HOR sù] Termine esortativo, e conclusivo, e diciamo nello stesso senso. \textit{Ovvia},
  quasi \textit{Or via}, Latino \textit{Eia age}, Vedi sotto \cstan[12]{47}. Diciamo: \textit{hor su} quasi
  diciamo \textit{hac ipsa hora surge, \& hoc facias}.
\item[BATTA la ritirata] Se ne vada da Malmantile, Batter la ritirata è quando
  col tamburo si fa quella sonata, per la quale i soldati intendono doversi
  e lasciar l'impresa. Gio. Villani ciò disse: \textit{sonare la ritirara}; quasi accennando
  il Franzese \textit{retraite}.
\end{description}
\section{STANZA XXXXI. --- XXXXIII.}

\begin{ottave}
\flagverse{41}Io ti ringrazio sì, ma non mi placo\\
Perciò (gli rispond'ella) di maniera,\\
Ch'io non voglia pigliar la spada e 'l giaco,\\
Ch'in bugnela son più di quel ch'io era;\\
Così con quei due spirti havendo il baco\\
Sogginnge (per c'a lor vuol far la pera)\\
Io l'ho con quei briccon furfanti indegni, \\
C'hanno sturbato tutti i miei disegni.
\end{ottave}

\begin{ottave}
\flagverse{42}Dico di Gambastorta il tuo vassallo,\\
E di quel pallerin di Baconero,\\
Che fa nel giuoco con due palle fallo\\
Scambiando il color bianco per nero,\\
Error, che nol farebbe anc' un cavallo,\\
Ma e' vien che gli strapazzano il mestiere,\\
Che s'egli andasse un po la frusta in volta,\\
Imparerebbon per un'altra volta.
\end{ottave}

\begin{ottave}
\flagverse{43}Risponde il Re; Facciam quanto ti piace, \\
Ma ti verranno a chieder perdonanza,\\
Sì che tu puoi con essi far la pace,\\
Pero t'acquieta, e vanne alla tua fianza. \\
Non penso di restar già contumace,\\
S'io non ti servo, perch'io so a fidanza \\
Dunque ti lascio, e sono al tuo piacere;\\
Fatti servir da questo Cavaliere.
\end{ottave}


Martinazza ringrazia Plutone, e dolendosi del danno cagionatoli da Gambastorta,
e Baconero lo prega a gastigargli: Plutone l'esorta a placarsi, e le dice,
che andranno a chiederle perdono dell'errore; e fatte con essa sue cirimonie, la
rimanda alle stanze.
\begin{description}
\item[NON voglia pigliar la spada, e il giaco] Non voglia armare contro di loro per
vendicarmi.
\item[SONO in bugnola] Sono in collera. \textit{Bugnola} si chiama un' arnese fatto di
  cordoni di paglia, entro al quale si conferua grano, biade, ec, da i Latini detta
  \textit{cumera}. E si dice esser' in \textit{bugnola}, nel \textit{bugnolone}, \textit{in valigia}, \textit{nel gabbione} ec. per intender
  esser in collera. E tutte queste maniere vogliono esprimere il gonfiare,
  un fa per l'infiammazione della bile commossa. Orazio \textit{Bile tumet iecur}; dove
  altrove aveva detto: \textit{meum iecur urere bilis}. Ovidio ne' Fasti. \textit{Intumuit Iuno}, cioè
  \textit{Intronfiò, entrò in valigia}. Gli Spagnuoli similmente dicono, \textit{embotijarse}.
\item[HAVENDO il baco] Havendo ira. Traslato da i cani, i quali quando hanno
  un certo baco nella lingua per di sotto, par che sieno sempre adirati, ed il simile,
  dicono, segue ne i Montoni, quand'hanno il baco, o tarlo dentro alle corna.
\item[FAR la pera] Anticamente s'abbruciavano i corpi morti sopr' ad un monte
  di legne, qual monte quando era acceso, chiamavano \textit{Pyra}. Lall. En. Tr, lib. 5.
  st. 1.
  \begin{verse}
    \backspace Già l'alta pira di Didone ardea,
    E vibrava lontan fiamme, e faville.
  \end{verse}
  E da questo credo, che venga il nostro \textit{far la pera}, e che s'intenda anche ammmazar'uno,
  quasi si dica: \textit{Io voglio far la pira al tale}. S'intende anche \textit{far la spia a uno}.
\item[FAR fallo] Far' errore. E' termine del giuoco di palla: e però il Poeta se ne
  serve; perché l'errore fu fatto con le palle. Properzio lib. 3. \textit{Aut pila veloces fallit
    per brachia iactus}.
\item[NOL farebbe anco un cavallo] Error grossissimo, e che non lo farebbe anche
  una bestia; e si dice un cavailo, perché questo animale pare, che habbia discorso, e
  giudizio più che ogni altro animale. I Greci di \textit{hippos}, che vuol dire \textit{cavallo}, se
  ne servono per una particella, che aggiunta a' nomi, importa grandezza. \textit{Hippomarathram}
  perciò è il finocchio salvatico, e \textit{Hippomyrmeces}, certe formiche,
  che passano di grandezza l'ordinarie, e comuni. Onde errore, o sproposito da
  cavalli è un'error grande. O pure si dice così, perché sia degno di cavallo, cioè
  di gastigo, qual si suol dare nelle scuole a' fanciulli.
\item[STRAPAZZANO il mestiero] Cioè nell'operare, non considerano quel che fanno.
\item[ANDASSE la frusta in volta] Se la frusta andasse attorno. Se fussero di quando
  in quando bastonati, frustati.
\item[NON penso di restar contumace] Termine di cirimonia, che significa: non penso
  di commenter mancamento. La voce \textit{contumace} è Latina; però il Lettore si può
  soddisfare circa i suoi significati.
\item[FO a fidanza] Confido, che per tua cortesia non l'haurai per male, e mi scuserai;
  termine usato fra gli amici intrinsechi, e si dice anche; \textit{Fo a sicurtà}.
\item[SONO al tuo piacere] Termine usato da' superiori con gl'inferiori, in vece di
  \textit{suo servitore}.
\item[DA questo Cavaliero] Intende Nepo.
\end{description}
\section{STANZA XXXKXIV.}

\begin{ottave}
\flagverse{44}Nepo la mena allora alle sue stanze,\\
Ch'i paramenti havean di quoi humani\\
Ricamati di fignoli, e di stianze, \\
E sapean di via de Pelacani,\\
Ove gli orsi facendo alcune danze\\
Dan la vivanda, e da lavar le mani,\\
Volati al cibo al fin, come gli astori,\\
Sembrano a solo a sol due toccatori.
\end{ottave}

\begin{ottave}
\flagverse{45}Fiorita è la tovaglia, e le salvette\\
Di verdi pugnitopi, e di stoppioni,\\
Saldate con la pece, e in piega strette,\\
Infra le chiappe state de' Demonj.\\
Nepo fra tanto a macinar si mette,\\
E cheto cheto fa di gran bocconi,\\
Osservando Caton, ch'intese il gioco,\\
Quando disse: in convito parla poco.
\end{ottave}

\begin{ottave}
\flagverse{46}Fa Martinazza un bel menar di mani;\\
Ma più che il ventre gli occhi al fin si pasce,\\
E quel pro falle, che fa l'erba a' cani,\\
Che il pan le buca, e sloga le ganasce,\\
Perché reste vi son come trapani,\\
Ne manco se ne pro levar con l'asce.\\
Crudo è il carnaggio, e sè tirante, e duro,\\
Che non viene a puntare i piedi al muro.
\end{ottave}

\begin{ottave}
\flagverse{47}Tal che s'a casa altrui suol far lo spiano,\\
E caseo barca, e pan Bartollomeo,\\
Freme che lì non può staccarne brano,\\
Pur si rallegra al giunger d'un cibreo\\
Fatto d'interiora di Magnano,\\
E di ventrigli, e strigoli d'Ebreo,\\
E quivi s'empie infino al gorgozzule,\\
E poi si volta e dice: Acqua alle mule.
\end{ottave}

\begin{ottave}
\flagverse{48}Preziosi liquori ecco ne sono\\
Portati ciascheduno in sua guastada\\
Essendovi acqua forte, e inchiostro buono\\
Di quel proprio, c'adopera lo Spada.\\
Ella che quivi star voleva in tuono,\\
E non cambiar, partendosi, la strada,\\
Perché i gran vini al cerebro le danno,\\
Ben ben l'annacqua con agresto, e ranno.
\end{ottave}

\begin{ottave}
\flagverse{49}E fatte due tirate da Tedesco\\
La tazza butta via subito in terra;\\
Però ch'ell'è di morto un teschio fresco,\\
Che suona, e tre dì fà n'ando sotterra,\\
Nepo, che mai alzò viso da desco,\\
E intorno a i buon boccon tirato ha a terra;\\
Anch'egli al fine dato a tutto il guasto,\\
La bocca sollevò dal fiero pasto.
\end{ottave}

Nepo conduce Martinazza alle sue stanze, dove era imbazidita la mensa, e subito
si mettono a mangiare. L'Autore descrive la qualità de i paramenti,
dell'imbandimento, de i trattamenti, e de i cibi, tutto appropriato a uno appartamento,
e banchetto da Diavoli.

\begin{description}
\item[QUOI humani] Pelli d'huomini. Se ben quoio vuol dir pelle di bestia conciati,
  si piglia ancora per pelle d'huomo, come s'è veduto sopra \cstan[4]{20}, e come
  lo prese il Ruspoli\footnote{Francesco Ruspoli, Fiorentino, 20 agosto 1579 --- 3 dicembre 1625.} dicendo:
\begin{verse}
  Un certo ch'in sull'ossa ha secco il quoio;
\end{verse}
\item[FIGNOLI] Specie d'apostema nella cute; da i Medici detti \textit{Furunculi}.
\item[STIANZE] Quelle croste, che fa nella pelle la rogna; o altre bolle; da i
  Latini dette \textit{cruste}. Varchi Stor. Fior.lib.1.4, \textit{Gli trovarono roso dello stomaco quanto
  un giulio con una stianza nera sopr' a quel roso}.
\item[SAPEAN di via de' Pecalani] Puzzavano di bestia morta di più giorni. La
  via de' Pelacani si dice in Firenze quella; dove son le conce delle pelli, nella
  quale è sempre un puzzo orrendo cagionato, e dalle conce, e dalla corruzione di
  quelle carni.
\item[VOLATI al cibo come gli astori]\footnote{Astore: Accipiter gentilis.} Entrati a tavola velocemente. Avventatisi al
  cibo, come fa l'astore, il quale, benché habbia il cibo a suo dominio, vi s'avventa,
  e lo divora con rapacità grandissima.
\item[SEMBRANO a solo a fol due Toccatori] Dicemnmo sopra \cstan[2]{66}. quel che
  sieno i Toccatori\footnote{incaricati, nelle esecuzioni personali, del ``toccare'' il debitore.}. Questi sono solamente due, e volendo andare a cena all'osteria
  son forzati andar da lor due soli, che le conversazioni de' galanthuomini non
  gli vogliono, perché son riputati infami, e con i BIrri non vogliono andar'essi,
  perché si stimano più onorati di loro, sì che, quando si veggono due soli a una
  tavola nell'osteria, si dice: \textit{paiono due toccatori}.
\item[PUGNITOPI, e stoppioni] Virgulti, o piante che hanno le foglie spinose, e
  pungenti.
\item[SALDATE con la pece] Data loro la salda con la pece, in cambio d'amido, e però nere.
\item[STRETTE in piega] Le salvette, e tovaglie si piegano in diverse maniere, e
  si fa loro pigliare la figura, che si vuole, col tenerle così piegate strette in un torcolo,
  o strettoio fatto a posta per tal' effetto, in vece del quale strettoio, queste
  sono state strette fra le natiche de i Demonj; e ciò dice per esprimere, che son
  nere.
\item[INTESE il giuoco] Sapeva, come era conveniente fare, quando disse: \textit{Pauca in
  convivio loquere}.
\item[FA un bel menar di mani] Si studia; s'affatica a mangiare. Vedi sopra \cstan[1]{7}.
\itemLE fa il pro, che fa l'erba a cani] Non le fa pro. Quando i cani mangiano l'erba, vomitano.
\item[RESTE] Quei fili sottilissimi, che stanno appiccati alla spiga del grano, dell'orzo,
  e della segale; dal Lat. \textit{aristae}.
\item[TRAPANO] Specie di succhiello, o foratoio atto a bucar pietre, ferro, ed
  ogni altra materia per dura che sia, e s'adopra facendolo girare con una corda.
  Noi l'habbiamo dai Greco \textit{Trypanon}, Vedi sopra \cstan[4]{73}.
\item[NON se ne può levar con l'asce] È così duro, che ci vuol l'asce a levarne un pezzo.
\item[NON viene, a puntar i piedi al muro] Non se ne può strappare a fare ogni maggiore sforzo.
\item[FAR lo spiano a casa a altri] Mangiare a casa d' altri senza spendere. Vedi
  sopra \cstan[3]{51}. Questo detto viene dallo spiano del grano, che vien dato
  dal Magistrato dell'Abbondanza a i Fornai per smaltire il vecchio, che si ritrova
  ne i magazzini pubblici, e da questo rifinimento \textit{spianare}, o \textit{far lo spiano a casa
    d'altri} intendiamo rifinire, o consumare quello, che colui ha di commestibile in casa.
\item[E CASEO barca, e pan Bartolommeo] Precetto della squola de' ghiotti, che vuol
  dire Mangiar la midolla del cacio, e la corteccia del pane.
\item[FREMERE] È voce latina, che conserva appresso noi lo stesso significato:
  Verg, 1. AEn, \textit{Cuncti simul ore fremebant}. E altrove descrivendo il Furore; \textit{fremit
    horridus ore cruento}.
\item[BRANO] Pezzo di carne (forse dal Latino \textit{membrana}) o d'altro strappato
  con violenza, e si dice sbranare; e sbranato. Vedi sopra \cstan[2]{52}. mandato
  a brani.
\item[CIBREO] Guazzetto fatto di colli, e ventrigli di polli, \textit{Minutal}. Può essere
  originata questa parola dalla Latina \textit{Gigeria}. Festo Gramatico: \textit{Gigeria ex multis
    obsoniis decerpta}.
\item[MAGNANO] Quasi \textit{machinarius} fabbricatore di ferri minuti, e di piccoli ingegni,
  come chiavi, toppe; a distinzione di Fabbro, che fabbrica ferri grossi, come
  Zappe, vanghe, ec, e del Manescalco, che fabbrica ferri per le bestie. E perciò
  i Magnani son sempre tinti di nero, il Poeta dice che il cibreo era fatto di
  loro interiori, per esprimere, che era nero.
\item[VENTRIGLIO] Ventricolo degli uccelli; in altri luoghi detto \textit{groscile}\footnote{È lo stomaco trituratore, o stomaco meccanico, secondo la Treccani: ``gesiere'', o ``gisiere''.}.
\item[STRIGOLI] Diciamo quella membrana, o rete gratia; che sta appiccata alle
  budella degli animali.
\item[ACQVA alle mule] È un detto di gente bassa, che significa \textit{date da bere}.
\item[GVASTADA] Vasetto di vetro corpacciuto, e col collo lungo, e stretto, che
  serve per lo più tenervi l'acqua per annacquare il vino, quando si beve. Gli
  antichi dissero \textit{Inguistara}. Il Canini la fa venire dal Siriaco \textit{Gastar}, che vale lo
  stesso. Potrebbe anche comodamente dedursi dal Greco \textit{Gaster}, che vale, \textit{ventre},
  \textit{corpo}; e così \textit{Guastada} esser detta dalla figura corpacciuta, nello stesso modo appunto
  che \textit{Grasta} voce Siciliana usata dal Boccaccio neile Novelle indubitatamente
  viene, sì come molte della Sicilia, dalla Greca \textit{Gastra}, un poco trasposte le
  lettere; la quale significa \textit{un vaso che habbia pancia}.
\item[LO Spada] Valerio Spada\footnote{Valerio Spada, Colle Val d'Elsa 1613 --- 1688. Pittore, incisore, calligrafo.} celeberrimo Maestro di scrivere, huomo singolare
  e che non resta addietro a veruno nella galanteria del tratteggiare con velocità di
  mano, e frappeggiare, e far paesi con la penna; come d' intagliare in rame con
  bulino, e acqua forte, fu amicissimo dell'Autore, e suo scolare nel disegno,
  vive ancora; e ben che d'età sopra settant'anni, indefessamente lavora per eternare
  il suo nome.
\item[VOLENDO star in tuono] Volendo star' in cervello, e non s'imbriacare.
\item[CAMBIAR la strada] Quando vogliamo dir copertamente a uno, Tu sei briaco, diciamo.
  \textit{Tu hai smarrita la Strada}, e però intende, non si vuole imbriacare.
\item[RANNO] Acqua passata per cenere, detta anche \textit{liscia}, dal Lat, \textit{lixivium}. Il
  dottissimo Ferrari nelle Origini della lingua Italiana, dice così; \textit{Ranno; lixsvia
  Unde vox ortum trahat, omnibus vestigijs indagata bactenus fefellit}, Chi sa, che non
  si origini dalla voce Greca \textit{Rhanis}, che significa, \textit{stilla, gocciola}, perché il ranno
  stilla a gocciola a gocciola da quel vaso, che perciò diceli textit{\Colatoio}?
\item[FATTE due tirate da Tedesco] Fatte due gran bevute. Manda giù del vino; i
  Latini dicono: \textit{pocula obducere}; i Franz. \textit{avaller}.
\item[SVONA] Di questo verbo sonare ci serviamo per intender copertamente \textit{putire}.
\item[MAI alzò viso dal desco] Stette empre attento alla roba, che era in tavola.
  Termine usato per intendere uno, che a tavola mangi con avidita, e non pigli divertimento
  di sorta alcuna; E desco se ben vuol propriamente dire la tavola, dove
  si sta a mangiare, onde il dettato: Chi non mangia al desco Ha mangiato di fresco,
  oggi è poco inteso per altro, che per quel legno, sopr' al quale i macellari tagliano
  la carne, e per quel banco, al quale nelle Confraternite, o compagnie de'
  secolari siede il Giovernatore.
\item[TIRATO ha a terra ai buon bocconi] Ha mangiato assai de' buon bocconi, È
  lo stesso, che menar le mani detto sopra.
\item[LA bocca sollevò dal fiero pasto] Verso di Dante Inf. C. 33. Lasciò star di mangiar
  quell'orride vivande.
\end{description}
\section{Stanza L. --- LII.}
\begin{ottave}
\flagverse{50}Lasciatii voti, e i piatti scemi \\
Vanno al giardino pieno di semente \\
Di berline, di mitere, e di remi, \\
E di strumenti da castrar da gente; \\
Risiede in mezzo il paretaio del Nemi \\
D'un pergolato, il quale a ogni corrente \\
Sostien, con quattro braccia di cavezza, \\
Penzoloni, che sono una bellezza.
\end{ottave}

\begin{ottave}
\flagverse{51}Spargon le rame in varia architettura\\
Scheretri bianchi, e rosse anotomie,\\
Gli aborti, i mostri e i gobbi in su le mura\\
Forman spalliere in luogo di lumie;\\
D'ugna, di denti, e simil'ossatura\\
Inseliciate son tutte le vie;\\
N'un bel sepolcro a nicchia il fonte butta\\
Del continuo morchia, e colla strutta.
\end{ottave}

\begin{ottave}
\flagverse{52}Le statue sono abbrostolite, e scure \\
Mummie del mar venute della rena, \\
Ch'intorno intorno in varie positure \\
In quei tramezzi fan leggiadra scena.\\
Su i dadi i torsi nobili sculture\\
(Perch'in rovina il tutto il tempo mena)\\
Ristaurati sono, e risarciti\\
Da vere, e fresche teste di banditi.
\end{ottave}

Finito che hebbero di mangiare, Nepo condufie Martinazza nel giardino. Qui
principia a descrivere un giardinu da Diavoli mostrandolo ripieno di tutti quei
Malanni, e disgrazic, che-alla giornata accadono a i mortali.
\begin{description}
\item[LASCIATI i bicchier voti, e piatti scemi] Havendo bevuto, e mangiato quanto
  loro era piaciuto.
\item[GIARDINO] Luogo dove si piantano fiori, ed altre delizie simili da i Latini
  detto \textit{Florarium, seu pomarium}. Viene questa voce dal Tedesco \textit{Garten}, e questo
  dal Latino \textit{hortus}, secondo il Ferrari, il quale biasima il Perionio\footnote{Joachim Périon. (Cormery, 1498 o 1499 --- ivi, 1559), umanista, filologo, traduttore. }, che la fa venire
  dal Greco \textit{ardevein}, \textit{innafiare}, seguitato in ciò dal Monosini. Ma tanto quegli
  nella sua lingua Francese, quanto questi nella nostra Toscana, sono troppo appassionati
  nel far venire le voci dal Greco, il che non è sempre vero, ch'elle vengano.
\item[BERLINA] Gogna. Vedi sopra C. 2. stan 15., e \cstan[3]{62}.
\item[MITERA] È quel berrettone, o cartoccio di foglio; che dalla Giustizia si
  fa mettere in testa a coloro, che sono frustati in sull'asino. Vedi sotto Can, 12. stan. 19.
\item[IL Paretaio del Nemi] Intendiamo le forche, perché queste son situate in un
  campo, che era, e forse è ancora della famiglia de' Nemi, e lo diciamo Paretaio
  per coprire il detto. Il \textit{Pareraio} è un boschetto fatto per uccellare a fringuelli,
  ed altri uccelletti simili nominato \textit{Paretaio} dalle reti, che s'adoprano a tal caccia,
  le quali si chiamano \textit{parete}. Vedi sopra C. 4. stan,27. al termine \textit{mandato in Piccardia}.
\item[PERGOLATO] Le viti che sostenute in aria da pali, e pertiche, formano come
  una coperta, o tetto si dicono pergole, o pergolati, come dicono anche i Latini.
\item[CORRENTE] È lo stesso che travicello, cioè un legno lungo, grosso più d'un
  bastone, e s'adatta a formare, e sostenere i palchi, e tetti delle case.
\item[CAVEZZA] S'intende quella fune, con la quale si legano per il capo le bestie,
  e però è detta \textit{cavezza} quasi \textit{capo}, e il Poeta la chiama così, perché è legata
  per il collo, e capo degl' impiccati a quei correnti, e gli chiama \textit{Penzoli}, perché
  gli figura grappoli d'uva pendenti a questa pergola.
\item[SPARGON le rame] Gli alberi che sono in questo giardino distendono i loro
  rami in diverse maniere; ma in vece d'alberi sono scheletri bianchi, e rosse anotomie.
  Scheletro, o scheretro diciamo tutta l'ossatura d'un corpo d'huomo, e di
  ogni altro animale, ripulita dalle carni, e rimessa insieme con legature. Gr. \textit{Scheletos}:
  \textit{Anotomia} chiamiamo il corpo d'un'huomo, e d'altro animale scorticato, che
  mostra tutti li nervi, muscoli, e vene, che sono sotto la pelle.
\item[SPALLIERE] Quelle piante, ed alberi, che si fanno distendere su per le mura
  con i rami, come limoni, e susini, ec, si dicono spalliere. e qui pigliando \textit{lumie}
  per ogni specie di pomi d'agrumi, dice, che in vece di tali pomi erano in
  questi alberi a spalliera \textit{gli aborti, i mostri, e i gobbi}.
\item[INSELICIATE] Seliciato dal Latino \textit{silices} diciamo un lastrico fatto in terra,
  ma strettamente, intendiamo quei lastrichi fatti di pietre piccolissime, che si soglion
  fare ne i viali de i giardini a foggia di Mosaico con pietre, però maggiori di
  quelle del mosaico, e minori assai di quelle degli acciottolati, e sono di varj colori
  in maniera che se ne formano figure, ec. come col Mosaico. E in vece di queste
  pietruzze, dice che son fatte d'ugna, di denti, e d'altre ossature minute.
\item[MORCHIA] Intendiamo la fondata dell'olio dal Latino \textit{amurca}, e questo dal Gr. \textit{amorge}.
\item[ABBROSTOLITE] Abbronzate. \textit{Abbrostolire} propriamente vuol dire quell'abbruciamento
  che si fa agli uccelli pelati, acciò si abbrucino quei peli vani che
  non si son potuti levare con le mani; ma qui vuol dire tinte dal fuoco con un
  leggieri abbronzamento, che diciamo: \textit{abbruciacchiare}.
\item[MVMMIE] Sono cadaveri d'huomini che hanno la carne appiccata in
  sull'ossa seccatavi sopra da balfami, bitumi, ed aromati, come son quei corpi,
  che si trovano sotto le Piramndi di Egitto, i quali sono di persone principali, che
  gli Egizzj havevano per costume di riempiere di balsami, ed aromati, fasciandogli
  con strette strisce di tela, o di drappo con mirabilissima maestria, e ponendoli
  insieme con qualche idoletto fatto di metallo dentro a una cassa, che haveva
  la faccia d'huomo; così gli riponevano sotto quelle piramidi, dove non si putrefacevano;
  ma si seccava la carne, e si riduceva tanto quella, che l'osso come impietrito;
  per lo che si sono conservati quei corpi fino a1 tempi nostri, ed ancora se
  ne trovano. Polid. Verg. de Rer, imuen, lib. 3. 6, 10. riferisce con le seguenti parole
  il modo di questo sotrerrare i cadaveri degli Egizzj:
  \begin{adjustwidth}{8pt}{}
    AEgypij statim mortuo
    homine ferro incurvo cerebrum per nares educebant, iocum illius
    medicamentis explentes, deinde acutissimo lapide AEthiopico circa ilia conscindebant,
    atque illac omnem alueum protrahebant, \& ubi repurgaverant, rursum
    odoribus contusis resarciebant, inde iterum contuebant. Vbi haec fecissent, saliebant
    nitro adulto septuaginta dies, nam diutius salire non licebat; quibus
    exactis Cadaver sindone involvebant gummi iilinentes; Eo deinde recepto propinqui
    ligneam hominis effigiem faciebant, in qua inserebant mortuum, inclusumque
    ita reponebant; Et id, ut arbitror, ita factitabant, ut eo pacto condita
    cadavera diuturnius incorrupta servarent.
    \end{adjustwidth}
  Altri cadaveri secchi ci vengono pure dagli Egizzj i quali corpi hanno gl'interiori,
  e tutto, secco, e come impietrito; e sono senza fasciature; e questi sono
  corpi d'huomini, che dal vento sono stati sotterrati vivi nella rena, e quivi conservatisi
  forse per causa de' venti meridionali, e però il nostro Poeta dice: \textit{Venuti
    dal mar della rena}. Di queste Mummie si servono i Medici per diversi farmachi,
  ma particolarmente per la Triaca. La voce \textit{Mummia} è Araba; e il Vossio
  la tira da \textit{Mum}, che in Arabesco vuol dire, \textit{cera} (de vitijs Sermonis \libcap[2]{12})
  avendo la cera e 'l miele facultà conservatrice; e della cera si servivano gli
  antichi per mantenere i cadaveri secondo Erodoto, lib, 1. Ma la pece mescolata
  con altro bitume, era forse quella materia, per quel che apparisce, con la quale
  per lo più gli Egizzj condivano tali corpi, la quale in Latino greco dicono \textit{Pyssasphltum}.
\item[DADI] Intende quelle basi, sopr' alle quali son posate le statue.
\item[TORSI] Intende torsi d'huomini, che pittorescamente parlando vuol dire il
  solo corpo senza testa, e braccia, e cosce Latino \textit{truncus}; e questi dice, che sono
  risarciti; cioè raccomodati, rappezzati, ristaurati con havervi messe in vece
  delle lor teste già consumate dal tempo, altre teste nuove, e fresche di banditi; e
  vuol quelle teste, che alle volte si veggono al Palazzo della Giustizia, e
  sopr'alle forche esposte alla vista del popolo, essendo state tagliate di poco tempo
  a i malfattori banditi, e però fresche.
\end{description}
\section{STANZA LIII. --- LIV.}

\begin{ottave}
\flagverse{53}In terra sono i quadri di cipolle,\\
Ove spuntano i fior fra foglie, e natiche;\\
Sonvi i ciccioni, i signoli, e le bolle,\\
Le posteme, la tigna, e le volatiche.\\
V'è il mal Francese entrante alle midolle,\\
Ch'è seminato dalle male pratiche,\\
I cancheri, le rabbie, e gli altri mali,\\
Che vi mandano gli Osti, e i Vetturali.
\end{ottave}

\begin{ottave}
\flagverse{54}Pesche in su gli occhi sonvi azzurre, e gialle,\\
Gli sfregi fior per chi gli porta pari,\\
I marchi, che fiorir debbon le spalle\\
A i tagliaborse, e ladri ancor scolari;\\
Le piaghe a masse, i peterecci a balle,\\
Spine ventose, e gonghe in più filari,\\
V'è il fior di rosolia, e più rosoni\\
D'ortefica, vaiuolo, e pedignoni.
\end{ottave}

Seguita a descrivere il giardino dell'Inferno, ed in queste due ottave narra
quel che contengono gli spartimenti.
\begin{description}
\item[QVADRI di cipolle] Intende quelli spartimenti, che si fanno in terra ne i giardini,
  ne' quali si pongono le cipolle de' fiori. Latino \textit{areolae}, \textit{pulvini}.
\item[FRA foglie, e natiche] Dice così per mostrare, che questi mali vengono nella
  carne esteriormente, e pigliando natiche per tutta la pelle dell' huomo, dice che
  fra quelle foglie nascono questi mali in su le natiche, intendendo la pelle, e perché
  anche la maggior parte de' medesimi mali per lo più viene in su le natiche,
  come luogo più carnoso.
\item[CHE vi mandano gli Osti, e i Vetturali] Questa sorta di gente ha per costume
  d'imprecar sempre male, come venga la rabbia, il canchero, la peste, e simili.
\item[PESCHE in su gli occhi] Quei lividi, che vengono attorno agli occhi, quando
  sono stati percossi da pugna, o da altri, e sono di colore azzurriccio, e intorno
  giallo, onde:  \textit{Dar le pesche}: i Latini dicono \textit{suggillare aliquem}, vedi sopra C, 3.
  st. 11. che noi pure diciamo anche sigilli tali lividi, e diciamo anche: sigillare un'occhio a
  uno.
\item[GLI sfregi fior per chi gli porta pari] Gli sfregi son fiori, che stanno bene in sul
  viso di coloro\footnote{mi permetto una interpretazione: \textit{che li portano pari}, cioè nella simmetria lo sfregio diventa adorno.} che \textit{portan pari i polli}, cioè fanno bene il raffiano, che \textit{portar i polli}
  vuol dir fare il ruffiano dalla voce \textit{pouler} Francese che vuol dir, \textit{viglietto amoroso,}
  quasi diciamo \textit{porta poulets}.
\item[MARCHI] Intende quei segni, che dalla giustizia si fanno nelle schiene a i ladroncelli,
  quando per esser giovanetti non sono capaci della pena ordinaria, Lat. stigmata.
  Vedi opra C. 2. st, 3. alla voce \textit{sberlefe}.
\item[PIAGHE a masse, peterecci a balle] Piaghe, e peterecci in quantità grandissima.
  Nell' uso diciamo anche \textit{Patereccio}; e \textit{Panareccio} dal Greco, usato anche
  da' Latini \textit{Paronychia}, postema che si forma alla radice dell'ugna, che i Latini
  chiamano \textit{Redivias}, o \textit{Reduvias}.
\item[GONGHE] Intendiamo \textit{gavine} infermità che viene nel collo, e quei tumoretti,
  che son talvolta \textit{spine ventose}, perché diciamo \textit{haver le gonghe} ognimalore, che
  venga apparentemente nella pelle della gola sotto le ganasce, Latino \textit{tonsilla},
  \textit{glandulae faucium}.
\end{description}
Ma perché non paia che io voglia fare un trattato di chirurgia, tralascio
l'esplicazione di questi mali; tanto più che io stimo, che faranno intesi per tutta
Italia, nella quale son chiamati nell'istessa, o poco differente maniera, e per
intelligenza dell'opera serve sapere, che in questo giardino sono tutte l'infermità,
che vengono agli huomini esteriormente, le quali il Poeta vuol mostrare,
che si generano nell'Inferno, come sentina di tutti i mali.
\section{Stanza LV. --- LVII.}
\begin{ottave}
\flagverse{55}Si maraviglia, si stupisce, e spanta\\
Martinazza in veder sì vaghi fiori,\\
E rimirando hor questa, hor quella pianta\\
Non sol pasce la vista in quei colori,\\
Ma confortar si sente tutta quanta\\
Alla fragranza di sì grati odori,\\
E di non corne non può far di meno\\
Un bel mazzetto, che le adorni il seno.
\end{ottave}

\begin{ottave}
\flagverse{56}Alla ragnaia al fin si son condotti\\
Di stili da toccar la margherita,\\
Ove de' tordi cala, e de' merlotti\\
Alla ritrosa quantità infinita,\\
Che son poi da Biagin pelati, e cotti\\
Sgozzando de' più frolli una partita,\\
Altra ne squarta, e quella ch'è più fresca\\
Nello stidione infilza alla Turchesca.
\end{ottave}

\begin{ottave}
\flagverse{57}Veduto il tutto, Nepo la conduce\\
Al bagno, ov'ogni schiavo, e galeotto\\
Opra qualcosa: Un fa le calze, un cuce, \\
Altri vende acquavite, altri il biscotto, \\
Chi per la pizzicata, che produce\\
Il luogo, fa tragedie in sul cappotto,\\
Un mangia, un soffia nella vetriuola,\\
Un trema in sentir dir: fuor camiciuola.
\end{ottave}

Martinazza resta maravigliata, e si stupisce, e rimirando tutte quelle piante,
pasce la vista, e soddisfa all'odorato con quella suave fragranza, ne può non fare
un mazzo di quei fiori galanti per adornarsene il seno. Visto il giardino,
Nepo la conduce alla ragnaia, dipoi al Bagno, dove stanno i galeotti, descritto
come è appunto quello di Livorno circa l'operazioni, che fanno i galeotti.

\begin{description}
\item[SPANTARSI] Dallo Spagnuolo \textit{espantarse}. Vuol dire estremamente
  maravigliarsi, e si dice in augumento \textit{maravigliarsi}, \textit{strabilirsi}, \textit{spantarsi}, che è il verbo
  spaventarsi sincopato. Habbiamo l'addiettivo spanto che significa estremamente
  maraviglioso. Ma forse è da Spandere, quali voglia dire largo, magnifico, grande,
  ampio, e in conseguenza maraviglioso. E di \textit{Spanto} addiettivo del verbo
  \textit{Spandere} ce n'è l'esempio in Messer Cino, \textit{Quando ha per gli occhi sua
    potenza spanta}.
\item[UN bel mazzetto, che le adorni il seno] Bello ornamento del seno d'una femmina
  havervi croste, rogna, e simili galanterie, delle quali poteva esser composto
  quel mazzo: ma il Poeta scherza per esprimere la laidezza di Martinazza.
\item[RAGNIAIA] È una selva, o macchia folta posta per lo più lungo i rivi, per
  mezzo la quale si tende una rete sospesa a due stili, e questa rete si chiama \textit{ragna}
  perché si tende a imitazione di quei veli, che fanno i ragni per pigliare le mosche,
  i quali si chiamano \textit{ragne}. Pietro Angelo da Barga nel suo Poema della caccia degli
  uccelli: \textit{Hos casses, has ipsa plagas, haec retia quondam Ante alias omnes telam
  contexere docta Invenit dixitque suo de nomine Arachne}. E da questa rete ragna si
  dice poi \textit{ragnaia} quella macchia, ove si tende per pigliar tordi, beccafichi, ec.
\item[STILI da toccar la margherita] Cioè quelle stanche, sopr'alle quali si da il martirio
  della Corda, che questo vuol dir \textit{toccar la margherita}.
\item[TORDI, merlotti] Vuol dir merli giovani, ma dicendosi merlotto, o Tordo
  a un'huomo s'intende Huomo semplice, corrivo, che cala; che si lascia pigliare.
  Vedi sopra C.2. st. 59.
\item[RITROSA] Gabbia fatta a foggia d'una trappola da topi, con la quale per
  via di certo ordigno si pigliano vivi gli uccelli, detta così per esser la parte, da
  aprire, e serrare rivolta in dietro. Vedi sopra in questo C. st, 1. alla voce contrappelo.
  Qui per \textit{ritrosa} intende Carcere.
\item[BIAGINO] Maestro Biagio, o Biagino vuol dire il Boia, che così havea nome,
  quando l'Autore compose le presenti Qttave; ed a questo successe Maestro
  Baltiano detto sopra C. s. tt. 44.
\item[FROLLO] Poco gli manca a essere stantio; s' intende animale morto di più
  giorni, Vedi sopra C.3. stan. 24. la voce stantio.
\item[INFILARE alla Turchesca] Cioè impalare.
\item[BAGNO] Così chiamiamo quel ferraglio, entro al quale si tengono gli schiavi,
  e coloro, che per delitti son condennati alla galera, detti pero Galeotti, i
  quali dimorando quivi, fanno i mestieri enunciati dal Poeta, che si serve della voce
  \textit{bagno} per l'equivoco, il quale fa credere, che in questo giardino sia ancora il
  \textit{bagno} da bagnarsi per mostrarlo ripieno d'ogni delizia; come il Paretaio, e la ragnaia.
  E questo serraglio di galeotti credo, che si dica \textit{bagno}, perché in esso quei
  delinquenti purgano i loro misfatti, come con l'acqua del bagno si purgano le
  lordure delle membra. \textit{Gagno} si disse ancora un luogo simile. Il Pulci nel Morgante:
  \textit{Disse Morgante allora: io son nel gagno De' diavoli}.
\item[PIZZICATA] Specie di confezione minutissima, ma per la similitudine della
  figura di essa confezione, e per il senso del verbo pizzicare intendiamo (come
  qui s'intende) pidocchi.
\item[FA tragedie in sul cappotto] Ammazza pidocchi in sul cappotto, che è quella,
  sopravveste, che portano gli schiavi, o galeotti, remiganti, ed ogni altro marinaro;
  detto, siccome \textit{Cappa}, \textit{a capiendo}, perché piglia, e cuopre tutta la vita.
\item[SOFFIAR nella vetriola] Cioè bere, perché bevendo si soffia, o respira col naso
  nella vetriola,  cioè nel vetro. Detto che ha del parlar furbesco. \textit{Vetriola} erba
  nota. Latino \textit{herba parietaria} detta da alcuni. Il Monosini lib. 9. Indicare volentes
  aliquem multo vino se ingurgitasse, dicimus. \textit{Egli ha toccato ben la vetriola.
    Vetrivola  est herba infectoribus notissima, de qua Petrus Crescentius lib, 6. c. ult. pocula
    vero vitrea vulgo fiunt}.
\item[Fuor camiciuola] Quando l'Auzzino vuol bastonare un galeotto per qualche
  suo mancamento suol dire \textit{fuor camiciuola}, intendendo, che si spogli quel tale, che
  ha da esser bastonato; e però dice: \textit{Chi trema in sentir dir: fuor camiciuola}, cioè
  trema per il timore delle bastonate.
\item[CAMICIVOLA] Bun piccolo farsetto di panno lino, bambagino, e lano,
  che secondo la stagione si porta sotto gli altri abitisopra alla Camicia per difendersi
  dal freddo, come habbiamo detto sopra alla voce Farsetto: gli schiavi la
  chiamano \textit{giulecca}.
\end{description}
\section{Stanza LVIII. --- LX.}
\begin{ottave}
\flagverse{58}Vanno più innanzi a' gridi, ed a' romori,\\
Che fanno i rei legati alla catena,\\
Ove a ciascun secondo i suoi errori\\
Dato è il gastigo, e la dovuta pena.\\
A i primi che son due Proccuratori\\
Cavar si vede il sangue d'ogni vena,\\
E questo lor avvien, perché ambidui\\
Furon mignatte delle borse altrui.
\end{ottave}

\begin{ottave}
\flagverse{59}Si vede un nudo, che si vaglia, e duole\\
Però ch emolta gente egli ha alle spalle,\\
Come sarebbe a dir tronchj, e tignuole,\\
Punteruoli, moscion, tarli, e farfalle,\\
Talche pei morsi egli è tutto cocciuole,\\
E addosso ha sbrani e buche come valle,\\
Ed è poi flagellato per ristoro\\
Con un zimbello pien di scudi d'oro.
\end{ottave}

\begin{ottave}
\flagverse{60}Quei, dice Nepo, è il Re degli usurai, \\
Che pel guadagno scorticò il pidocchio, \\
Un servizio ad alcun non fece mai,\\
Se non col pegno, e dandoli lo scrocchio;\\
Il gran se gli marcì dentro a' granai,\\
Che nol vendea se non valea un'occhio,\\
Così fece del vino, ed hor per questo\\
Gli intarla il dosso, e da' suoi soldi è pesto.
\end{ottave}


Passano avanti a vedere i delinquenti legati alla catena, e gastigati per loro
falli. I primi sono due Causidici, ed il secondo è un'Usuraio, i quali son puniti
secondo il merito.
\begin{description}
\item[PROCCURATORI] Agitatori di liti. Causidici tanto Civili, che criminali.
\item[MIGNATTE] Sanguisughe. Quei vermi acquatici, de i quali si servono i Cerusici
  per cavar sangue; e perché si dice, che i danari sono il secondo sangue,
  però \textit{esser mignatta delle borse altrui} vuol dir Succhiare, cioè cavar il denaro dall'altrui
  borse, come fa la mignatta succhiando, e cavando il sangue dalle vene,
  diciamo \textit{mignatta}, o \textit{mignella} a uno, che è stretto del suo, e volentieri piglia di
  quello d'altri: A questi tali può quadrare ciò, che disse Orazio. \textit{Non missura cutem
    nisi plena cruoris hirudo}.
\item[VAGLIARSI] Intendi dimenarsi come fa uno, che habbia rogna, o altro per
  la vita, che si dimena, e scontorce per grattarsi il prudore; o pizzicore con l'abito,
  che ha in dosso, e fa con la vita un moto simile a quello, che fa uno, che
  vagli il grano.
\item[TONCHI] Forse dal Latino \textit{tondere} preso per \textit{mietere}, e \textit{divorare}. Sono
  vermi piccoli, o insetti, che si generano nelle fave, piselli, ed in altri legumi, ec.
  e votano i granelli rodendoli; da i Latini detti \textit{Curculiones}. Virg, 1. Georg.
  \textit{Populatque ingentem farris acervum Curculio}.
\item[TIGNVOLE] Bachi simili, ma si generano ne i pani, e fogli impastati; da i
  latini detti Tineae. Di queste ne nascono ancora dal grano, e si chiamano \textit{punteruoli}.
\item[MOSCIONI] Quei moscherini, che nascono dal vino, che dicemmo sopra in
  \cstan{37}.
\item[TARLI] Vermi piccoli, che si generano nel legno, e lo rodono; da i latini detti \textit{Teredines}.
\item[FARFALLE] Intende quei farfallini, che si generano nel grano. \textit{Pyrausta}, con
  voce Greca sono appellate quelle farfalle più grandi, le quali volano attorno al
  lume, e vi s'abbruciano. Di queste disse il Petrarca. \textit{Semplicetta farfalla al lume
    avvezza}.
\item[COCCIUOLE] Piccoli tumoretti, o enfiature cagionate da' morsi d'animaletti
  come zanzare, bruchi, e simili.
\item[SBRANI] Rotture; Scorticature. Vedi sopra in \cstan{47}.
\item[PER ristoro] Per ricompensa. Dan. Par. C. 5.
  \begin{verse}
    Dunque che render puossi per ristoro?
  \end{verse}
  E qui se ben pare, che il nostro Poeta voglia dire, per ristoramento, o alleggerimento
  de i travagli, e pene, nondimeno è tutto il contrario, perché è parlare
  ironico, e vuol dire; oltre a gli altri travagli ha di più, che lo flagellano, e pestano
  con un sacchetto pieno di scudi d'oro. Questa voce \textit{ristoro} vien dal verbo \textit{ristorare}
  derivante dal verbo \textit{restaurare}, ed ha quasi lo stesso significato,
  se non che questo vuol dire Acconciare, o rassettar case, ed altri materiali; e
  quello vuol dir Ricompensare, o rifar danni.
\item[ZIMBELLO] Intende quel sacchetto appiccato a una cordicella; intendendosi per
  zimbello quel sacchetto pieno di segatura, o di cenci, che adoprano i ragazzi
  per perquotere i contadini, come dicemmo sopra C, 1. st. 59. \textit{Zimbello} detto,
  cred'io, quasi \textit{cennello}, cioè piccol segno, argumentandolo dallo Spagnuolo, che il
  chiama \textit{señuelo}.
\item[IL Re degli usurai] Il maggiore usuraio del mondo. Detto che viene da i Greci,
  i quali chiamavano Re, quello che avanzava, superava, e vinceva gli altri ne i
  lor giuochi fanciulleschi; ed Asino quel che perdeva, come habbiamo detto altrove.
\item[SCORTICÒ il pidocchio] Significa esser avido del denaro, e far' ogni maggior
  sordidezza per guadagnare; si dice \textit{scorticar il pidocchio, per vender la pelle}, e con
  Plauto si può dire, \textit{Vel unguium praesegmina colligere}.
\item[DAR lo scrocchio] Prestar danari a usura, ed in vece di dar denari effettivi, dar
  roba che vaglia dieci, per venti. Vedi sopra C. 3. st. 74. ed è la più esecranda
  usura, che si trovi, e forse la più praticata.
\item[MARCIRE] Intendiamo infradiciare, corrompersi, Dal Latino \textit{marceres}; \textit{marcescere}.
\item[SE non valeva un'occhio] Se non si vendeva caro, e a prezzo rigorosissimo: Non
  vi è cosa più cara dell'occhio. Onde Catullo. \textit{Ni te plus oculis meis amarem}.
\item[INTARLARE] Esser mangiato da i tarli, o tignuole, che i Latini dicevano:
  \textit{Cariem sentire}.
\item[È PESTO da i suoi soldi] Infranto dalle percosse di un sacchetto pieno delle
  sue monete. Vuol mostrar in somma il nostro Poeta, che
  \begin{verse}
    Per qua quis peccat, per haec torquetur.
  \end{verse}
\end{description}


\section{STANZA LXI. --- LXIII.}

\begin{ottave}
\flagverse{61}Un altro ad un balcon balla, e corvetta,\\
Ch'un diavol con la sferza a cento corde\\
Ch'un grand'occhio di bue ciascun ha in vetta,\\
Prima gli dà certe picchiate sorde,\\
Con una spinta a basso poi lo getta\\
In cert'acque bitumose, e lorde,\\
Che e' n'esce poi, ch'io ne disgrado gli orci,\\
O peggio d'un Norcin mula de' porci.
\end{ottave}

\begin{ottave}
\flagverse{62}Dice la maga questo è un po ariosa\\
Quand'ella vedde simil precipizio,\\
Costui ha fatto qualche mala cosa,\\
Pur non so nulla, e non vuò far giudizio:\\
Domanda a Nepo (fattane curiosa)\\
Tal pena a chi si debba, ed a qual vizio,\\
Ed ei che per servirla è quivi apposta\\
Prontamente così le da risposta.
\end{ottave}

\begin{ottave}
\flagverse{63}Quei fu Zerbino, e d'amoroso dardo \\
Mostrando, il cuor ferito, e manomesso, \\
Credeva il mio fantoccio con un sguardo \\
Di sbriciolar tutto il femmineo sesso;\\
Ma dell'occhiate sue ben più gagliardo,\\
Hor sentene il riverbero, e il riflesso,\\
E com'ei già pensò far alle dame,\\
Dalla finestra è tratto in quel litame.
\end{ottave}

Quel che segue è uno che peccò d'ambizione di bello, e lindo, e credeva con
la sua bellezza di far' innamorar tutte le dame, ed hora riceve la pena dovuta al
suo peccato.

\begin{description}
\item[CORVETTA] Salta. Corvettare è un certo saltellar de' cavalli, dal Lat. \textit{curvari},
  Spagnuolo \textit{corvar}; piegare, innarcare, torcere. E questo verbo è assai
  appropriato in questo luogo per esprimer il moto, che faceva costui, il quale per
  evitare le sferzate, era necessario che saltellasse a tempo, ed in quella guisa appunto,
  che fa il cavallo, quando corvetta.
\item[VN grand'cchio di bue ciascuna ha in vetta] Pone in vetta, cioè nella cima di
  queste corde, l'occhio del bue, e non d'altro animale; perché \textit{bovis oculo oculorum
    pulchritudo, \& nitor significatur}, e trovatene l'esempio in Omero, dal quale
  Giunone è chiamata \textit{boopis}, cioè \textit{bovinos oculos habens} o vero \textit{Dea dagli occhi grandi},
  e perciò maestosa. E costui doveva esser gastigato con la bellezza degli occhi;
  perché con la pretesa bellezza de' suoi occhi, haveva egli peccato.
\item[PICCHIATE sorde] Picchiate, e percosse gagliarde. Percosse, che facciano
  molto male, e non paia che lo facciano; servendoci in questo caso la voce \textit{sordo}
  per la voce \textit{occulto}, come si dice \textit{ricco sordo}, per ricco non palese, o non conosciuto.
\item[NE disgrado] Quel che vaglia questo termine vedi sopra \cstan[3]{37}. al
  termine \textit{ho stoppato}.
\item[ORCIO] Che cosa sieno orcj. Vedi sopra C. 1. st. 7. Qui intende orci da olio,
  che son sempre schifi.
\item[NORCINO mula de' porci] Coloro che in Firenze ammazzano i porci, e così
  morti gli portano sopr' alle spalle alle botteghe de' Macellari, sono per lo più del
  paese di Norcia, e però gli chiama \textit{mule Norcine}, cioè \textit{portatori da Norcia} e
  costoro son sempre tutti unti di grasso di porco, lordissimi, e schifi di sangue.
\item[QVESTA è ariosa] Questa è cosa grande, ardua, e che arreca stupore; o straordinaria,
  e stravagante, e che non si può credere.
\item[NON vuò far giudizio] Cioè giudizio temerario, e falso; Maniera da Ipocriti,
  e falsi bacchettoni scrupolosi.
\item[ZERBINI] Così chiamiamo quei giovani, che persuadendosi d'esser belli, fanno
  tutte l'usanze, e vanno lindi credendosi di far innamorare ognuno con la lor
  bellezza; Da quel Zerbino, che l'Ariosto nel Furioso descrive per il più bello, e
  grazioso giovane di quel tempo. E si dice anche Mirtillo; nome cavato dal Guarino\footnote{Giovanni Battista Guarini, (Ferrara, 10 dicembre 1538 – Venezia, 7 ottobre 1612), poeta, drammaturgo, diplomatico per la famiglia Este. Il Pastor Fido è il suo dramma più noto, base per madrigali, opere, sonate.}
  nel Pastor fido.  Vedi sotto \cstan[10]{30},
\item[MOSTRANDO  il cuor ferito, e manomesso]\items{d'amoroso dardo} Facendo da innamorato di tutte le Dame.
\item[FANTOCCIO] Nibbiaccio, Uccellaccio, ec. tutti servono per intendere un
  huomo sciocco, e scimunito.
\item[SCRICIOLARE] Romperere in minutissimi pezzi, o ridurre in bricioli, ed
  intende Far morir di spasimo, e disfarsi per amor di lui tutte le dame.
\item[REVERBERO, e riflesso] Sinonimi che significano li riperquotimenti, che fanno
  i raggi del Sole, o il fuoco nella parte opposta a quella, dove direttamente battono,
  donde i Chimici dicono; Fuoco di riverbero, o di riflesso. Qui intende,
  che costui con quelle frustate piene d'occhi, ha il gastigo dell'occhiate amorose,
  che egli nel mondo dava alle donne.
\item[E COME egli pensò far alle dame] Cioè si come egli pensò che le dame cascassero
  dalle finestre per la sua bellezza, (il che appresso di noi vuol dir farle morire
  per suo amore) così egli è buttato da quei balconi entro al litame, per maggior
  sua pena; perché questi tali sono schizzinosi ne possono vedersi addosso un
  bruscolo, che guasti la loro attillatura, e lindura.
\end{description}
\section{Stanza LXIV. ---  LXVI.}

\begin{ottave}
\flagverse{64}Si vede un ch'è legato, e che gli è posto\\
In capo un berrettin basso a tagliere,\\
E il diavol colpo colpo da discosto\\
Con la balestra gliene fa cadere.\\
Il misero sta quivi immoto, e tosto\\
Battendo gli occhi a i colpi dell'arciere,\\
Che s'ei si muove punto, o china o rizza,\\
Per tutto è un cultello che l'infizza.
\end{ottave}

\begin{ottave}
\flagverse{65}Qui Nepo scuopre la di lui magagna,\\
Mostrando ch'ei fu nobile, e ben nato,\\
E sempr'hebbe il Pedante alle caleagna;\\
Contuttociò voll'esser mal creato;\\
Perché se e' fusse stato il Re di Spagna,\\
Mi cappello a nessun mai s'è cavato;\\
Però s'ei fu villano, hora il maestro\\
Gl'insegna le creanze col balestro.
\end{ottave}

\begin{ottave}
\flagverse{66}In hoggi questa par comune usanza,\\
Martinazza risponde al Galatrona;\\
Stanno i Fanciulli un po con osservanza,\\
Mentr'il Maestro, o il padre gli bastona.\\
Se e' saltan la granata, addio creanza,\\
Par ch'e' sien nati nella Falterona,\\
Ma per la loro affinità superba,\\
Son poi fuggiti più che la mal' erba.
\end{ottave}

L'altro che segue è uno, che nel Mondo non volle mai imparare i buoni costumi,
e non si volle mai cavar il cappello di testa per riverir nessuno, per grande
che egli fusse, onde gli avviene il gastigo, che si dice nelle presenti ottave; E
Martinazza dice a Nepo, che hoggi di questa sorta mal creati è pieno il Mondo.

\begin{description}
\item[BERRETTINO a tagliere] Berretta bassa e piatta, nella quale non si vede la
  forma del capo, come sono \textit{le coppole Napoletane}.
\item[COLPO colpo] Ogni volta ch'ei tira. Vedi sopra \cstan[2]{57}.
\item[STA tosto] Sta duro; Sta saldo; Sta fermo; Non si muove.
\item[ARCIERE] Colui che tira con la balestra. \textit{Arciere} in molti luoghi del nostro
  contado s'intende il Caprone, o Becco. Lat, \textit{aries}.
\item[MAGAGNA] Mancamento, difetto. E parlandosi d'huomini s'intende tanto
  d'animo, che di corpo, Dante Inf. C. 33. dice.
  \begin{verse}
    O Genovesi huomini diversi
    D'ogni costume, e pien d'ogni magagna.
  \end{verse}
  Lalli En. Trau, C.3. stan. 114 disse:
  \begin{verse}
    Ogni trattato contr' ogni magagna,
  \end{verse}
  \textit{Magagna} in Lat barb. è detta \textit{Mahamium}, ant. Franz. \textit{Mahain}, e \textit{Mehain},
  e vuol dire propriamente mutilazione di membra, e si stende a significare ogni danno,
  e detrimento. Vedi Du Fresne nel Glossario alla Parola \textit{Mahamium}\footnote{Nostratibus Mahain et Mehain, membri mutilatio, vel enormis laesio, qua quis ad serviendum principi in bello redditur imbecillior.}.
\item[BEN nato] Nato di nobili, ed honesti parenti.
\item[HEBBE sempre il Pedante alle calcagna] Hebbe sempre il Maestro attorno che
  gl'insegnava i buoni costumi, e termini.
\item[MAL creato] Senza creanza, Uno che non sa i buoni termini o costumi.
\item[VILLANO] Contadino. S'intende uno scortese, e mal creato. Plauto \textit{rus
  merum}, intende un' huomo rustico, senza civiltá, senza galanteria, un pretto
  villano, Catullo, \textit{Pleni ruris, \& inficetiarum}. Il contrario di \textit{villano} è \textit{gentile}.
\item[SE saltan la granata] Se essi escono di sotto la cura del padre, e del maestro.
  Si dice \textit{saltar la granata}; quand'uno esce de' pupilli che i Latini dissero, \textit{Excedere
    ex Ephoebis}. Dicono che quando uno è arruolato per birro, debba stare qualche
  mese a fare il noviziato, e finito questo tempo gli faccian fare una cirimonia di
  saltare sopr'a una granata, che gli mettono d'avanti in terra, e che fatta questa azione
  resti libero dal noviziato, ed in un certo modo esca de' pupilli; e da questa cirimonia
  (che se non è vera, e assai vulgata) credo io, che habbia origine il presente
  detto.
\item[PAIONO nati nella Falterona] Paiono nati in luoghi incolti, e disabitati, come
  sono le montagne della Falterona in Casentino, dove poche creanze possono impararsi,
  non essendo in quei luoghi con chi praticare, se non con pecore, e porci.
  Ci serviamo però di questo termine per esprimere un'huomo incivile, e rozzo,
  e che tratti da villano; come è \textit{quercubus, aut saxis natus}.
\item[SON fuggiti più che la malerba] Nessuno gli vuol praticare. Sono sfuggiti da
  tutti. Malerba intendiamo l'ortica erba nora, la quale è sfuggita da tutti, perché
  pugne.
\end{description}
\section{Stanza LXVII. \& LXVIII.}

\begin{ottave}
\flagverse{67}Ma chi è quel, c'ha i denti di cignale \\
E lingua così lunga, e mostruosa?  \\
Si vede, che son fuor del naturale \\
A me paion radici, o simil cosa. \\
Nepo rispose; Quello e un Sensale \\
Che si chiamò il Parola, ma la glossa \\
Huom di fandonie, dice, e di bugie, \\
Perché in esse fondò le senserie.
\end{ottave}

\begin{ottave}
\flagverse{68}Ora per queste sue finzioni eterne,\\
Chi egli hebbe sempre nella mercatura,\\
Lucciole dando a creder per lanterne,\\
Sbarbata gli han la lingua, e dentatur,\\
Ma in bocca havendo poi di gran caverne,\\
Perché non datur vacuum in natura,\\
Gli hanno a misterio in quelle stanze vote\\
Composto denti, e lingua di carote.
\end{ottave}

Segue un Sensale, il quale e gastigato delle bugie, che disse, havendogli cavato
la lingua, e i denti, ed in quella vece messovi delle carote. Il Poeta si serve
dell'assioma Peripatetico: \textit{Non datur vacuum in natura}, col quale intende che fusse
necessario riempier quei voti, cagionati dall'estrazione della lingua, e denti, ma
scherza, sapendo bene anch'egli, che quei medesimi voti erano già ripieni d'aria.\footnote{Intorno al 1644 Evangelista Torricelli, professore di matematica presso l'Accademia Fiorentina, e Matematico del Granducato di Toscana, dimostrò che l'aria pesa, e che ``riempie i voti''. Lorenzo Lippi, coetaneo del Torricelli, lo ritrasse poco prima della sua morte nel 1647. }

\begin{description}
\item[SENSALI] Coloro, che son mediatori a far vender una mercanzia:
\item[IL parola] Così fu soprannominato in Firenze un Sensale di bestie, huomo
  scellerato, e ladro, che per le sue furberie fu impiccato a forche erette a posta per
  lui, dentro alla Città, al Canto delle Rondini, ed è lo stesso, che quegli che fu
  detto \textit{Ballonchino} detto sopra C. 3. st. 55.\footnote{dove lo chiama Balocchino.}
\item[FANDONIE, e bugie] Cose lontane dal vero, e sono si può dir sinonimi, se
  ben \textit{fandonia} vuol dir chiacchierata vana, e bugia propri vuol dire attestazione falsa.
\item[FAR una senseria] S'intende, quando uno di questi Sensali fa vender qualcosa, e
  guadagna la sua retta.
\item[DAR a creder lucciole per lanterne] Dar a creder una cosa per un'altra. Il Lalli
  En. Tr. C. 2. st. 82.
  \begin{verse}
    Lucciole qui rimiro per lanterne.
  \end{verse}
\item[LUCCIOLA] È quel vermicello alato, che di notte riluce da i latini detto \textit{Cicindela},
  \textit{Noctiluca}; da Tedeschi \textit{animaletto di S. Giovanni}, e da' Greci \textit{Lampyris} dal
  luccicare, e lampeggiare nelle tenebre, come egli fa; e \textit{lanterna} è quello arnese,
  dentro al quale si porta il lume la notte serrato da talco, osso, o vetro per difenderlo
  dal vento; ed è voce pura latina.
\item[CAROTA] Specie di radica, Latino \textit{siser}. Ma il proverbio \textit{Piantar}, o \textit{ficcar
  carote} significa dare a creder bugie. Latino \textit{imponere alicui}. Onde \textit{Impostura}, e
  \textit{Impostore}. se bene si dice in più grave significato. Vedi sopra C. 2, st. 70. Dice
  che il mistero, perché vi son messe tali carote, è non solamente per riempiere i
  vacui, ma per dar il gastigo a costui delle tante carote, che esso haveva piantate,
  mentre era in vita, facendogli haver sempre dentro alla bocca effettive, e naturali
  carote.
\end{description}

\section{Stanza LXIX. \& LXX.}
\begin{ottave}
\flagverse{69}Quell'altro, c'all'ingiù volta ha la faccia, \\
E un diavol legnaiuolo in sul groppone \\
Gli ascia il legname, sega, ed impialliaccia, \\
Sì che lo fa servir per suo pancone; \\
Un di coloro fu; c'alla pancaccia \\
Taglian le legne addosso alle persone, \\
Sì che de non tener la lingua in briglia \\
Così si sente render la pariglia.
\end{ottave}

\begin{ottave}
\flagverse{70}Vedi colui, c'al colle ha un'orinale;\\
Cieco, rattratto lacero, e piagato? \\
Ei fu Governator d' uno spedale,\\
Ov'ei non volle mai pur un malato,\\
Ora per pena ogni dolore, e male,\\
Che gl'infermi v'havrebbono portato\\
(Mentr'alla barba lor pappò sì bene)\\
Sopr' al suo corpo tutto quanto viene.
\end{ottave}

Segue il gastigo dato a' Mormoratori, ed a quelli, che, essendo stati Soprantendenti
a Spedali, non hanno havuto carità; ma solo hanno atteso a crapulare per
loro con quello, che dovevan somministrare a' poveri, ed infermi.

\begin{description}
\item[GROPPONE] Codrione. Le parti di dietro dell'huomo fra le reni, e le natiche
  Vedi sotto C. 10; st. 50. Il Persiani disse.
  \begin{verse}
    \backspace Ciascun teme, e si caca nelle brache
    In vedervi appiccato sul groppone
    Lo stocco da scannar le pastinache.
  \end{verse}
  Donde si cava che è usato, ma per lo più in scherzo. Viene secondo il Ferrari dal
  Latino greco \textit{Orrhopygium}, che significa lo stesso.
\item[ASCIARE] Tagliar con l'asce, che è uno strumento da legnaiuoli noto,
  chiamandolo così anche i Latini, che lo dicono \textit{Ascia}. Isidoro nelle Origini lib. 19.
  c.19. \textit{Ascia ab astulis dicta quas a ligno eximit, cuius diminutivum nomen est asciola}
  (forse accetta) \textit{Est autem manubrio brevi, ex adversa parte referens vel, simplicem
    malleum, vel cavatum, vel bicorne rastrum}. Vitruvio disse \textit{Asciare} Lib. VI. c. 2. \textit{Sumatur
  Ascia, \& quemadmodum materia} (Qui intende il \textit{legno}; che gli Spagnuoli dal
  Latino chiamano, \textit{madera}) \textit{dolatur, sic calx lacu macerata ascietur}.
\item[IMPIALLACCIA] Qui la rima forse ha necessitato  l'Autore a servirsi di
  questo verbo \textit{impiallacciare} in vece del verbo \textit{piallare}, che vuol dire ripulire i
  legnami con la pialla come intende qui, ed il verbo \textit{impiallacciare} vuol dire ricoprire
  un legname con \textit{piallacci} (\textit{sectiles laminae}, \textit{laminae praetenues} le disse Plinio) che
  sono sottilissime asticelle di noce, con le quali si cuopre altro legname più vile in
  far casse, tavole, ed altro, nella forma che si fa con  l'ebano, granatiglia, ed altri
  legnami nobili. Plinio discorrendo di legnami, de' quali gli antichi si servivano
  per impiallacciare lib. 17. 43. \textit{Quae in laminas secantur, quorumque operimento
    vestiatur alia materies, praecipua sunt cedrus, terebinthus, etc}, E poco appresso:
  \textit{Haec prima origo luxuriae, arborem alia integi, \& viliores ligno pretiosiores cortice fieri}; E
  poi. \textit{Excogitate sunt, \& ligni bracteae, nec satis, Coepere tingi animalium cornua,
    dentes secari, lignumgue ebore distingui, mox operiri}.
\item[PIALLA] Chiamano i Legnaiuoli quello strumento di legno, che ha un ferro
  incassato, col quale assottigliano, appianano, puliscono, ed addirizzano i
  legnami, da i Latini, secondo molti, detto \textit{Dolabra}, ma forse con qualche equivoco.
  Un antico Grammatico par che la confonda coll'ascia. \textit{Dolare fabri, lignum est
    ascia laedere}. Si legge in Colum. lib. 3. \textit{Quae falce amputari non possunt, acuta dolbra
    abradito}, il che pare che voglia dire più tosto accetta, o pennato, o vanga,
  che pialla: E corrobora questa opinione il medesimo Colum. lib. 4. ¢-24, servendosene
  in diminutivo; \textit{Semper circa crus dolabella dimovenda est terra}, cioè \textit{Intorno
    al gambo della vite è da levare la terra con una accettina}. Il Calepino\footnote{Ambrogio Calepio, il Calepino (Castelli Calepio, 1435 circa --- Bergamo, 1511) umanista, latinista, noto per il Dictionarium latinum, una monumentale opera di natura lessicografica ed enciclopedica sulla lingua latina.} tiene, che
  la \textit{pialla} si dica \textit{runcina}, e porta  l'autorità di Plinio \libcap[16]{42}. \textit{Ad incitaros
    runcinarum raptus}, ove pare, che descriva appunto l'operazione della pialla, e
  per infino l'arricciolinamento de' trucioli: Tutto il testo dice così: \textit{Et ad
    quacumque libeat intestina opera aptissima} (parla dell'abeto) \textit{sive Graeco, sive Campano,
    sive suculo fabricae artis genere spectabilis, ramentorum crinibus pampinato semper orbe
    se voluens ad incitatos runcinarum raptus}. Ma io ardisco contraddirgli con
  l'autorita d'Hermolao che dice: \textit{Runcinae sunt maiores serrae, quibus fabri materiarij
    secant arborum moles subiectis canterijs}, Sì che non la pialla, ma la sega grande,
  che adoperano i Marangoni per ricidere i legnami, adattandoli sopra quel cavalletti,
  che noi chiamiamo \textit{canteo} (dal Latino \textit{cantherius}, cioè \textit{cabalus} e più
  volgarmente \textit{pietiche}, i quali sono composti di due correnti inchiavardati insieme
  a guisa di cesoie (che propriamente si dicono \textit{pietiche}) e d'un'altro pezzo di corrente,
  che si mette a traverso alle pietiche (e questo si dice \textit{Canteo}) e formando
  così un triangolo vi adattano per via di piuoli il legno da segarsi. \textit{Runcare} è termine
  d'agricoltura, che vuol dir propriamente tor via, onde se ne formò per avventura
  la parola antica Latina \textit{averruncare}, cioè \textit{avertere}; e se ne creò l'Iddio
  \textit{Averruncus} detto così, perché \textit{ab eo precari solent, ut pericula avertat}; sì come dice
  Varrone. E in proposito d'agricultura se ne fabbricarono le parole \textit{Roncola}, e
  \textit{Roncone}, le quali significano strumenti da nettare i campi, da rimondare frutti, e
  governare le siepi. Plinio lib. 18, c. 21. \textit{Siliginem, far, triticum, semen, hordeum
    occato,  sarrito, runcato}. E appresso. \textit{Runcatio, cum seges in articulo est, evulsis inutilibus
    herbis, frugum radicem vindicat, segetemque discernit a cespite}. E Catone cap.
  2.3. dice: \textit{spinas runcari cremarique}; sì che più tosto \textit{Runcina} parrebbe, che avesse
  ad essere la roncola, o cosa simile, che la sega, o la pialla. Ma forse non
  tanto il Calepino, quanto anche il Vocabolario della Crusca dal levar via, e
  svellere, e ripulire (che questo significa, come s'è vitto il verbo \textit{Runcare}) hanno
  dato il nome di rancina alla pialla, perché ella pulisce, appiana, e leva il soverchio
  da' legnami. Tuttavia anche per questa ragione la direi \textit{dolabra}, perché finalmente
  questa ancora pulilce, e rade, come dice Colum, nel luogo sopra citato.
  Ma sia come esser si voglia, poco fa \textit{ad rem nostram}, bastandoci intendere, che
  la \textit{pialla} è quello strumento da legnaiuoli; che habbiamo accennato.
\item[PANCONE] Chiamano i legnaiuoli quella loro panca grossa, sopra la quale
  appoggiano i legnami per lavorargli, detta \textit{pancone}, perché è fatta d'un \textit{pancone}
  che vuol dire un'asse grossa circa un quarto di braccio, che sono asse da rifendere.
\item[ALLA pancaccia] Così si chiama quel luogo dove in Firenze si tiene il crocchio,
  e si discorre de' fatti d'altri, e delle nuove. Vedi sopra C. 2. st. 73, E perché
  il dir male del prossimo si dice \textit{Tagliar le legne addosso a uno}. Latino \textit{famam
    alicuius lacerare, proscindere}, però a costoro vien dato il gastigo adeguato, con
  tagliar loro addosso il legname effettivamente.
\item[TENER la lingua in briglia] Parlar consideratamente, e con riguardo, e si dice
  anche: \textit{Tener la lingua a freno}.
\item[RENDER la pariglia] Render il contraccambio. Pariglia vuol dire una cosa,
  che può dividersi in due parti uguali; come nel numero due si può far' uno, e uno.
  E di qui \textit{render pariglia} vuol dir render ugual contraccambio. Vedi sopra C, 4. st. 71.
  È il \textit{par pari referre} de' Lat. Dan. nel Parad. C. 26. dice:
  \begin{verse}
    \backspace  Perch' io lo veggio nel verace speglio,
    Che fa di se pareglie l'altre cose,
    E nulla fece lui di se pareglio.
  \end{verse}
  Hoggi però in questo senso, e maniera, che si serve Dante di questa voce \textit{pareglia}
  non mi pare, che si usi, se non da' Franzesi, che dicono \textit{pareil}.
\item[ALLA barba loro] A spese loro, Questo termine esprime Pigliare, o consumare
  una cosa d'altri contro al gusto, e volontà del padrone di essa; o a dispetto, e
  onta del medesimo.
\item[PAPPÒ] Cioè mangiò, Donde Pappolone uno che mangia assai che vedemmo sopra C.1. st. 36.
\end{description}
\section{Stanza LXXI.}
\begin{ottave}
\flagverse{71}Chi è costui, c'habbiamo a dirimpetto\\
(Dice la donna) a cui guegli animali\\
Sbarban con le tanaglie il cuor del petto?\\
Nepo risponde: Questo e un di quei tali, \\
Che non ne pagò mai un maladetto;\\
Tenne gran posto, fe spese bestiali;\\
Ma poi per soddisfare ei non havria\\
Voluto men trovargli per la via.
\end{ottave}

\begin{ottave}
\flagverse{72}Colui, c'ha il viso pesto, e il rotto\\
Da quei due spirti in femminili spoglie,\\
Huom vile fu, ma biscaiuolo, e ghiotto,\\
Che si volle cavar tutte le voglie;\\
Ogni sera tornava a casa cotto,\\
E dava col baston cena alla moglie; \\
Hor finti quella stessa quei demoni,\\
Sopra di lui fan trionfar bastoni.
\end{ottave}

\begin{ottave}
\flagverse{73}Riserra il muro, che c'è qui davanti,\\
Donne, che feron già per ambizione\\
D'apparir gioiellate, e luccicanti\\
Dar il \culo{} al marito in sul lastrone,\\
Hor le superbe pietre, e i diamanti\\
Alla lor libertà fanno il mattone,\\
Però che tanto grandi, e tanti furo\\
C'han fatto per lor carcere quel muro.
\end{ottave}

Termina la mostra delle pene date a i delinquenti con tre sorte di martirj, che
il primo è dato a coloro, che non vollero mai pagare i loro debiti. Il 2. è quello
dato a i crapuloni strapazzatori della moglie.  Il terzo è quello dato alle donne
ambiziose, e vane.

\begin{description}
\item[TANAGLIE] Strumento di ferro fatto a foggia di cesoia, e serve per cavar
  chiodi da i legni, ec. Da i latini detto \textit{forcipes}.
\item[NON ne pagò un maladetto] Non volle mai pagare un debito, Non pagò mai
  un quattrino di debito. L'epiteto \textit{maladetto} ha la forza d'un becco d'un
  quattrino detto sopra C. 1. st. 68.
\item[TENNE gran posto] Si trattò alla grande, e fece spese bestiali, cioè grandi, ed
  inconsiderate. Lat. \textit{immanes}.
\item[NON hauerebbe volute crovargli per la via] Quand'anche egli havesse trovato per
  la strada il denaro, del quale era debitore; non havrebbe ad ogni modo pagato il
  suo debito. Questo termine ci serve per esprimere, che nessuna cosa havrebbe
  potuto muoverlo dal suo proposito, e fargli venir voglia di pagare.
\item[PESTO] Infranto, ed ammaccato dalle bastonate, che gli danno quel Demoni
  finti la sua moglie. E questo vuol dire \textit{trionfar bastoni}\footnote{peraltro nei giochi di carte il \textit{trionfare} di un palo indica l'essere briscola, o \textit{trionfo}.}.
\item[HUOM vile] Qui vuol dire huomo di bassa condizione.
\item[BISCAIUOLO] Huomo che pratica le bische. \textit{Bische} diciama quei raddotti
  pubblici, dove si giuoca a carte, e a dadi; nome forse venuto dal verbo \textit{biscazzare},
  che vuol dir Mandar male spropositatamente il suo havere: e corrisponde
  al Latino prodigere. L'usò Dante nell' Inferno C, 10.\
  \begin{verse}
    Biscazza, e fonde le sue facultadi.
  \end{verse}
\item[GHIOTTO] Huomo, a cui piace mangiar del buono. Vedi sopra C. 5. st. 63.
\item[DAVA col baston cena alla moglie] In vece di portar da-cena alla moglie; la bastonava,
  Costume assai usato dalla gente d'infima plebe, imbriacarsi all'osterie,
  e non pensar' a mandare da cena a casa alla moglie, e così-briachi tornare a
  casa, e perché la povera moglie si duole d' eer digidna', bastonarla
\item[DAR del \culo{} in sul lastrone] Quand'un mercante fallisce; diciamo: \textit{Il tale ha
  dato il \culo{} in sul lastrone}. Brunetto Latini nel Patasso disse \textit{Dar del \culo{} sul petrone}.
  Questo proverbio è nato da un proverbio antico, che era in Firenze; che color,
  i quali fallivano, o rifiutavano l'eredità del padre, andavano nel mezzo di
  Mercato nuovo (luogo dove si ragunano i mercanti per negoziare) e quivi era,
  ed è ancora una gran lastra di marmo tonda, che si chiama il \textit{carroccio} (perché vi
  è posta per segno, dove si fermava il carroccio, sopra il quale s'inalberava l'insegna
  generale de' Fiorentini, quando andavano alla guerra) e sopra detta lastra
  posavano tre volte il \culo{} a vista del popolo, che nell'hora, che si doveva fare tal
  funzione era quivi radunato. E questo atto assicurava la loro persona dalle molestie
  per causa di debito, ne potevano i creditori molestare se non la roba, la
  quale s'intendeva ceduta tutta a favore de i Creditori, non essendo per questo
  atto tenuto il debitore a pagare \textit{ultra vires}, essendo questo come un \textit{cedo bonis} del
  Capitolo Odoardus\footnote{la \textit{Cessio Bonorum} è una opzione disponibile a tutti i creditori non fraudolenti, probabilmente introdotta da Giulio Cesare, per evitare la messa in schiavitù del debitore insolvente, è confermata nel Codice di Giustiniano, e sussiste in varie forme nelle attuali legislazioni.}. Così questra lastra alle persone de' falliti, che a quella rifuggivano,
  era come una Ara, o vogliam dire altare, o luogo sacro, o asilo, o
  franchigia, che dall'esser presi gli assicurava, e questo, perché essendo dedicata
  a servigio pubblico di sostenere il solenne carro, e la tanto famosa insegna della
  Signoria, rendeva per questo riguardo franchi, ed immuni coloro, che col sedervi
  sopra prendevanne solennemente, e con cirimonia il possesso. Di qui \textit{dar
    il \culo{} in sul lastrone} vuol dir fallire. E di qui pure, quand'uno casca, e batte il \culo{}
  in sulle lastre diciamo \textit{Il tale ha rifiutato il padre}. \textit{Fallire} ancora dichiamo
  \textit{infilare le pentole}: E \textit{Il tale l'ha infilate}; che corrisponde al Latino \textit{decoxit}.
\item[MATTONI] Sono il latino \textit{lateres} detto sopra C.1. st. 67. E \textit{fare, o dare il
  mattone}. Vuol dir fare a uno qualche danno grave; e qui vuol dire; sono il lor
  gastigo, e pena.
\end{description}
\section{Stanza LXXIV --- LXXV.}
\begin{ottave}
\flagverse{74}Ma sta in orecchi, che mi par che e' suoni\\
Il nostro tabellaccio del Senato,\\
Sì che e' mi fa mestier ch'io t'abbandoni,\\
Però ch'io non voglio esser' appuntato;\\
A veder ci restavano i lioni,\\
Ma non posso venir ch'io son chiamato,\\
Ed ecco appunto Diavoli co' lucchi;\\
Però lascia ch'io corra, e m'imbacucchi.
\end{ottave}

\begin{ottave}
\flagverse{75}Dice la Maga; Vo' venir anch'io,\\
Perch'il veder più altro non m'importa,\\
Ed in questa Città così a bacìo\\
A dirla mi par d'esser mezza morta;\\
Voglia trattar col Re d'un fatto mio,\\
Ed andarmene poi per la più corta,\\
Ed ei le dice in burla; Se tu parti,\\
Va via in un'ora, e torna poi in tre quarti.
\end{ottave}

Veduti li suddetti gastighi dati a i delinquenti, Nepo sentendo la Campana
del Senato si licenzia dalla Strega, ma dovendo esser'anch'ella nel Senato per
parlare al Re, dice volerlo seguir fin quivi, di dove spedita se ne vuol andare per
la più corta.
\begin{description}
\item[STAR in orecchie] \- Ascoltare con attenzione. \textit{Auribus arrectis auscultare}.
\item[TABELLACCIO] Così è chiamata da molti la Campana del palazzo del Podestà
  (hoggi del Bargello, la quale è detta la Maddalena, come vedemmo sopra
  in \cstan{23}.) forse dal latino \textit{Tabelliones}, che vuol dir Notai, i quali
  dimoravano, e tenevano i lor banchi dentro, ed attorno al detto palazzo, ragunandovisi
  al suono di detta campana, la quale hoggi e detta anche \textit{la furba}, perché
  fuori d'alcune feste, non suona, se non per esecuzioni criminali di teste, e forche,
  e la notte per mostrar l'hora, che non si può più portare armi; o pure è così
  detta dal suono oscuro, e malinconico, o che almanco\footnote{Or che si può, caviamo d'errore \textit{almanco} il sig.\ Simplicio\\Romano lo volemo lo Papa, o \textit{almanco almanco} Italiano} rappresenta cosa mesta,
  come il suono delle tabelle ne' giorni Santi.
\item[NON voglio essere appuntato] Coloro che son del Consiglio del Dugento, e d'altri
  Magistrati di Firenze se non vanno al detto Consiglio, quando si raguna a suono
  della Campana, son condannati in certa somma di danaro; e questo diciamo \textit{esser appuntati}.
\item[LUCCO] È la sopravvesta, o mantello Curiale di Firenze, ed era anticamente
  l'abito civile ordinario; e perch questo haveva già un cappuccio, quando uno si
  metteva in dosso detto lucco, si doveva dire \textit{imbacuccarsi}. Varchi Stor. Fior. lib.
  141. \textit{Subito fu preso; e imbacuccato col cappuccio, fu condotto alle carceri}. Vedi sotto
  C. 11, st.22.
\item[A BACÌO] Campagna, dove batte poco il Sole, che diciamo Al rezzo, all'uggia.
  Vedi sopra C, 3. st. 71. alla voce Vria, e sotto C, 9. st. 44, e C. 10, st 51.
  I contadini in vece di dire: luogo o piaggia volta a mezzo giorno, dicono: a solatìo,
  ed in vece di dire: volta a tramontana, o a settentrione dicono: a bacìo,
  o a paggino che è il contrario di solatio, Credo venga dal Latino \textit{opacus}, \textit{opacivus},
  sì come \textit{natio} da \textit{nativus}. Da molti si dice meriggio quel luogo, dove non penetrano
  i raggi del Sole per interposizione di che che sia, e (pare a primavista) non
  troppo lodevolmente, perché \textit{meriggio} da \textit{meridies} vuol dir mezzo giorno, quando
  appunto i raggi del Sole sono più quocenti, e però andare al meriggio parrebbe
  che volesse dir più tosto andare a scaldarsi a' raggi del Sole di mezzo giorno,
  che andar all'ombra per difendersi da i raggi del Sole. Per corroborazione di
  questo idiotismo, si trova in Autore approvato per buon Scrittor Toscano: \textit{Non
  vollero fare il viaggio di notte per lo gran freddo, ma sì bene in sull'ora meriggiana
  allora che il Sole con i suoi raggi havesse addolcito i rigori hiemali.} Ma questi tali si
  difendono con l'uso, e potrebbe dirsi anche colla ragione, perché meriggio nel significato
  di luogo ombroso, e difeso dal Sole, è lo stesso, che luogo da \textit{passare d'ore noiose
  del mezzo dì}, la quale cosa i Latini dicevano \textit{meridiari}, Catullo. \textit{Iube ad te veniam
    meridiatum}. Ora dal meriggiare, cioè stare all'ombra nell'ore calde è detto
  meriggio, e, da meriggio, rezzo. Va in un'ora, e torna poi in tre quarti. Questo
  è uno scherzo usato assai fra gente bassa, ed intende \textit{Va hora in uno}, cioè va intero,
  e poi torna poi diviso in tre quarti; \textit{sij impiccato}; se ben pare che voglia dire: Va
  in un quarto d'ora, ritorna in tre quarti. Cirimonia da Diavoli,
\end{description}
\section{Stanza LXXVI. \& LXXVII.}

\begin{ottave}
\flagverse{76}Tu vuoi, gli rispos'ella, sempre il chiasso;\\
Nel Consiglio così ne va con esso,\\
Ove ciascun l'honora, e dalle il passo,\\
Sbirciandola un po meglio, e più da presso,\\
Ella baciando il manto a Satanasso\\
Lo prega d'osservar quanto ha promesso,\\
Ei gliel conferma, e perché stia sicura,\\
Per la Palude Stige glielo giura.
\end{ottave}

\begin{ottave}
\flagverse{77}Ed ella per oferta così magna\\
Ringraziamenti fattigli a barella,\\
Dice, c'hor mai sbrattar vuol la campagna,\\
E tornar a dar nuove a Bertinella.\\
Pluton le dà licenza, e l'accompagna\\
Fino alla porta, e lì se ne sgabella,\\
Ond'ella in Dite a un Vetturin s'accosta,\\
Che la rimeni a casa per la posta.
\end{ottave}

La Maga così scherzando, e burlando con Nepo se ne va con esso in Consiglio,
dove ognuno l'honora. Fa riverenza a Plutone, e lo prega a mantenerle
quanto le ha promesso; Ei glielo  giura solennemente, ed accompagnatala fino
alla porta del Consiglio la licenzia, ed ella va a cercar d'un Vetturino, che la
riconduca per la posta a Casa.

\begin{description}
\item[TU vuoi il chiasso] Tu vuoi la burla. Tu scherzi. Chiasso nel proprio è via
  stretta, vicolo Lat. \textit{vicus}, quali erano le strade di Roma antica, e del primo
  cerchio in Firenze. Gio, Vill. 10. 29. \textit{S'apprese fuoco in Firense in Borgo Santo
    Appostolo nel Chiasso tra' Bonciani, e gli Acciaiuoli}. E perché in queste straducole abitavano
  talvolta donne di mal affare, Chiasso (detto forse da \textit{Vicus Vicario}, Borgata;
  in buon Latino \textit{Vicinia}) venne a significare \textit{Postribolo}, e perché in tali disonesti
  luoghi si fa gran baccano, e si scherza, e si burla senza rispetto; perciò
  \textit{chiasso} si piglia per burla, per ischerzo. Se bene e molto verisimile, che in questo
  ultimo significato di strepito, e di baccano, quale fanno quelli, che licenziosamente
  trattano, e burlano, venga dal Latino de' tempi bassi; che il suono di
  tutte le campane, o degli organi, e degli altri strumenti domandavano \textit{Classicum},
  il che i buoni Latini dicevano della \textit{tromba}, a cui son succedute le campane. Il
  Franz. lo dice Glas,

\item[SBIRCIANDOLA] Guardandola bene. Vedi sopra \cstan[1]{9}.

\item[GLIELO giura per la Palude Stige] Giuramento solenne, ed inviolabile degli
  Dei secondo la falsa credenza de i Gentili, come si cava da Omero in più luoghi
  del Iliade, e da Verg. AEn. lib. 6.
  \begin{verse}
    \makebox[8em]{\dotfill} Stygiamque paludem,
    Dij cuius iurare timent, \& fallere numen.
  \end{verse}
  La ragione, per la quale questo sia giuramento solenne, secondo Servio, è questa
  \begin{adjustwidth}{8pt}{}
    Styx moerorem significat, Dij autem laeti sunt semper; ergo qui moerorem non
    sentiunt iurant per tristitiam, quae res est suae naturae contraria; ideo
    iusiurandum per execrationem habent.
  \end{adjustwidth}
  L'altra ragione è, perché havendo Vittoria
  Figliuola di Stige aiutati gli Dei nella guerra contro ai Giganti Titani, Giove
  per rimunerarla, volle che coloro, che giuravano per Stige di lei madre,
  fussero privi del nettare delli Dei, se non osservavano il giuramento. E queste
  cose furono finte, e credute di Stige, perché secondo Teofrasto questo Stige era
  un fonte in Arcadia, le cui acque, e pesci erano velenosi per la di lui estrema
  frigidità; e di questa acqua dice Plin. \libcap[30]{16}. che Antipatro volesse dare
  ad Alessandro Magno, quando volle avvelenarlo per consiglio d'Aristotile.
  \begin{adjustwidth}{8pt}{}
    Ungulas tantum mularum repertas, neque ullam aliam materiam, quae non
    perroderetur a veneno Stygis aquae, cum id dandum Alexandro Magno Antipater
    mitteret, memoria dignum est, magna Aristotelis infamia excogitatum.
  \end{adjustwidth}
\item[A barella] In quantità grande, Si dice a balle a masse, a sacca, ec. sono
  pero modi bassi, e più tosto scherzosi, e s'usano parlando tanto di cose corporee,
  quanto incorporee.
\item[SBRATTAR la campagna] Andarsene: Sbrattare propriamente significa nettare,
  o ripulire, contrario d'Imbrattare; sì che \textit{sbrattare il paese} vuol dire
  ripulire il paese, e per confeguenza andarsene da quel luogo.
\item[SE NE sgabella] La lascia; Si sbriga; si libera, e si licenzia da lei. Dedotto
  dalla Gabella, che si paga, perché, come è pagato il dazio, o gabella d'una
  mercanzia, si dice sgabellata, e così si spedisce, e manda via.
\item[DITE] Qui la Città di Plutone, detta così da \textit{divitiae}, le  ci vengono tutte
  di sotto terra. I Latini chiamarono \textit{Dite}, quel che con Greco vocabolo dicevano
  altrimenti Plutone, che vuol dire il medesimo, e significa il ricco Iddio,
  Iddio delle ricchezze, come s'è veduto sopra.
\item[VETTURINO] Colui che presta cavalli a nolo, o a vettura.
\end{description}

\section{Stanza LXXVIII.}
\begin{ottave}
\flagverse{78}Il Re fatta con lei la dipartenza\\
Al salon del Consiglio se ne torna,\\
Onde ciascuno alla Real presenza\\
Alza il Civile, e abbassa giù le corna.\\
Salito alla sua sbieca residenza\\
Di stracci e ragni a drappelloni adorna,\\
Voltando in qua, e in là l'occhio porcino\\
Si spurga, e sputa fuora un Ciabattino.
\end{ottave}

Plutone licenziata la Maga se ne torna in consiglio, e postosi a sedere in sulla
sua refidenza si prepara a discorrere.
\begin{description}
\item[FATTE le dipartenze] Licenziatisi scambievolmente.
\item[ALZA il Civile] Alza le natiche. \textit{Civile} è una prospettiva di scena rappresentante
  abitazione di Città; contraria a quella, che si dice \textit{Bosco} rappresentante
  campagna. I Latini similmente aavevano due entrate principali in iscena, Una
  di quelli che venivano dalla piazza, o dal mercato; l'altra di coloro, che si fingeva
  che venissero di lontani paesi, o di fuori dalla Città; La prima entrata si
  diceva \textit{a foro}, l'altra \textit{a peregre}, siccome riferisce Vitruvio. Noi per questo chiamiamo
  \textit{Foro} la parte in Faccia della scena.
\item[RAGNI] Quei veli che fanno i ragni. Narrano le favole degli antichi Gentili,
  che in Lidia fu una femmina detta Arachne nata in contado di bassa gente, la
  quale fu così valorosa nel ricamare, ed in ogni sorta d'artifizio di tela, e di ago
  che non solo superava tutte l'altre femmine, ma hebbe ardire di contrastare con
  la Dea Pallade; onde Pallade superata, e vinta da lei, per dispetto le guastò il
  lavoro, e la convertì in Aragno verme, che è quell'insetto che fabbrica quei
  veli per pigliar le mosche da noi chiamato, \textit{ragno}, o \textit{ragnatelo}. Ovid,
  metam. Dante nel Purg. C. 12. tocca questa favola.
  \begin{verse}
    \backspace O folle Aracne, si vedeva io te
    Già mezz'aragna trista in su gli stracci
    Dell'opera, che mal per te si fe.
  \end{verse}
\item[DRAPPELLONI] Così chiamiamo quei pezzi di drappo, i quali si appiccano
  pendenti al cielo de i baldacchini, e delle residenze de i Prinejpi; e se ne parano
  le Chiese, ec. Varchi St. Fio. lib. 14. \textit{Ed al vano dela Cupola era tirato in su
    le funi un bellissimo ottangolo di drappelloni}. Matt. Villani \libcap[9]{43} descrivendo
  le nobili esequie fatte nella sepoltura dei Cavaliere Messer Biordo degli Ubertini.
  \textit{E sopra la bara un drappo a oro con drappelloni pendenti coll'arme del popolo e del
    comune, e di parte Guelfa, e degli Ubertini}. Tali drappelloni coll'arme si veggono
  appiccati in gran numero nella insigne Chiesa Collegiata di S. Lorenzo un tal
  giorno dell'anno, per memoria di antichi benefattori.
\item[SPUTA un ciabattino] Quando uno per soprabbondanza di catarro ha difficultà
  in spurgarsi, sogliamo dire: \textit{Egli ha un ciabattino giù per la gola}, e però dicendo,
  \textit{Sputa un ciabattino}, intende sputa molto catarro. Il Bocc. disse nel Laberinto,
  Sputar farfalloni. \textit{Coll'occhiaia livida tossire, e sputar farfalloni}.
\end{description}
\section{Stanza LXXIX. --- LXXXIII.}

\begin{ottave}
\flagverse{79}Spiegar volendo poi quanto gli occorre, \\
Comincia il suo proemio in tal maniera:\\
Voi che di sopra al Sole in queste forre \\
Cadesti meco all'aria oscara, e nera, \\
Onde noi siam quaggiù in fondo di torre\\
Gente, a cui si fa notte avanti sera,\\
Voi ch'in malizia, in ogni frode, e inganno\\
Siate i Maestri di color che sanno.
\end{ottave}

\begin{ottave}
\flagverse{80}Se ben fusse una man di babbuassi\\
Minchioni, e tondi più che l'O di Giotto,\\
Ma poi nel bazzicar taverne, e chiassi\\
S'è fatto agnun di voi sì bravo, e dotto,\\
Ch'in oggi è più cattivo di tre assi,\\
E viè più tristo d'un famiglio d'Otto;\\
Voi dunque, benché pazzi Cittadini\\
Nel viuupero ingegni peregrini,\\
\end{ottave}

\begin{ottave}
\flagverse{81}Siete pregati tutti in cortesia \\
Da Martinazza nostra confidente, \\
Poiché Baldone ancor cerca ogni via \\
D'entrar in Malmantil con tanta gente.\\
Ad oprar ch'egli sbandi, e trucchi via \\
Però ciascun di voi liberamente\\
Potrà dir sopra questo il suo parere \\
Del modo che e' ci fusse da tenere.
\end{ottave}

\begin{ottave}
\flagverse{82}Cominci il primo: Dite, Malebranche,\\
Quel che e' vi par che qui v'andasse fatto,\\
Levato i Tocco, e sollevate l'anche\\
Allor quel Diavol n'un medesmo tratto\\
Un capitombol fa sopr' alle panche,\\
E salta in pié nel mezzo com' un gatto:\\
Ma perch'il Lucco s'appiccò a un chiodo,\\
Si ricompone, e parla a questo modo:
\end{ottave}

\begin{ottave}
\flagverse{83}O Re, cui splende in mano il gran forcone\\
S'il Cappello speziale ha quel segreto,\\
Col qual si fa stornare un pedignone,\\
Io l'ho da far tornar un'huomo a dreto.\\
So già, che qualche debito ha Baldone,\\
E che e' lo vuol pagare in sul tappeto,\\
Perciò manda Pedino là in campagna,\\
Ch'ei giuocherà di posta di Calcagna.
\end{ottave}

Questo consiglio de i Diavoli fu composto dall'Autore, dopo che egli ottenne
un Magistrato, nell'esercitare il quale conobbe l'autorità, che si usurpano i Cancellieri
in essi Magistrati, mette per Cancelliere di questo Consiglio un Ciappelletto,
che fu un notaio scellerato, secondo che riferisce il Boccaccio nelle sue Novelle,
e fa che egli contraddica a tutto quello, che vien proposto. I nomi di questi
Diavoli i più son cavati da Dante nel suo Inferno; e sappia il Lettore, che li spropositi
che dicono, son poco lontani da quelli, che l'Autore sentiva dire nel medesimo
Magistrato, ed i personaggi che finge in questi Diavoli sono simili alli suoi Colleghi,
ed egli medesimo in leggermi questo Canto mi diceva; il tal Diavolo è simile al
tal mio Collega, e il tale, al tale; e mi parvero appropriati benissimo; non stimo
già bene nominargli. Ma tornando a proposito dico, che Plutone volendo
sentire il parere de' suoi Senatori, fatta una breve orazione nella quale inserisce
un verso del Petrarca \textit{Gente, a cui si fa notte avanti sera}, ed uno da Dante \textit{Siete i
Maestri di color che sanno}, ordina a Malebranche il dire, quel che egli farebbe
per mandar via Baldone da Malmantile, ed egli, fatte prime sue diaboliche cirimonie
dice che il suo pensiero sarebbe di farlo citare alla Mercanzia\footnote{Corte della Mercanzia, che è il Tribunale dove si fanno l'esecuzioni Civili, di cui già abbiamo visto sopra \cstan[]{60}.} da qualche
suo creditore.
\begin{description}
\item[FORRA] Valle lunga, e stretta posta fra poggi alti, onde poco dominata dal
  sole, e però ben detto forra il paese infernale dove non batte mai sole.
\item[GENTE a cui si fa notte avanti sera] Con questo verso del Petrarca, l'Autore
  intende che costoro son sempre di notte, cioè al buio.
\item[BABBUASSO] Huomo senza giudizio, scimunito. L'origine sua è scura;
  forse da \textit{Valvassor} parola feudale, dalla quale è fatto anche \textit{Barbassoro}, lo stesso
  che Satrapo, o dottoraccio; saccente; e che si dà scioccamente ad intendere di
  sapere o pure da \textit{Buaccio} peggiorativo di bue. Vedi sopra \cstan[5]{1}. Il Bini
  in lode del Malfrancese dice.
  \begin{verse}
    E rispondendo a certi Babbuassi,
    Che voglion dir, che questa malattia
    Tatto il corpo ci storpi, e ci fracassi.
  \end{verse}
  Ed il Molza in lode de' fichi:
  \begin{verse}
    Hor fa tu l'argumento, babbuasso.
  \end{verse}
\item[TONDO più che l'O di Giotto] Huomo tondo vuol dire huomo grosso d'ingegno,
  ed ignorante, come s'è accennato sopra \cstan[5]{1}, sì che \textit{più tondo dell'O
    di Giotto} vuol dire ignoranuthmo, e più, perché lO, che fece Giotto Pittore fu
  tondissimo, secondo che riferisce Giorgio Vasari nella vita di esso Giotto.
\item[BALZICARE] Praticare; Converlare: Bocc. Giorn. 9. Nov. 5. £ vatrene nel
  la cosa dela paglia, ch' e sh miglior nego che ci sia, perciocché non vi bazzica mai per
“ote. 1 sona.
\item[CHIASSI] Bordelli, lupanari, luoghi, e contrade, nelle quali habitano le
  meretrici, come era in Firenze il \textit{Chiasso de' Buoi}, e il luogo, dove ora è il \textit{Ghetto},
  detto anticamente Chiasso. E perché in tali luoghi usa di fare fracasso, e rumore
  disonesto; di qui forse e che \textit{chiasso}, e \textit{bordello} si prende ancora per tumulto
  disordinato, insolente, e lascivo.

\item[PIU' cattivo di tre assi] Asso si dice il numero uno de i dadi, che è il minor
  numero, e per conseguenza nel più è il peggiore che vi sia tirando tre dadi, e da
  questo il presente termine significa cattivissimo, che vale per astutissimo, ed è lo
  stesso che \textit{Più tristo d'un famiglio d'Otto}, che pur vuol dire sagacissimo, e che sa il
  conto suo. \textit{Famiglio d'Otto}. E' uno de' Birri del Magistrato degli Otto di Balia
  di Firenze, che è il Magistrato Criminale; e perché si suppone che costoro sappiano
  tutte le furberie, però si dice: \textit{Il tale è più tristo d'un famiglio d'Otto}, per
  esprimere; è huomo sagacissimo. I Greci dissero \textit{Cantharo astutior} , che questo
  Cantharo fu un'oste d'Atene astutissimo. \textit{Assum} in antico Latino voleva dire,
  \textit{solo, senza accompagnatura}; onde chi cantava  senza strumento che s'accompagnasse,
  si diceva costui: \textit{canere assa voce}. Di qui può esser venuta la voce \textit{Asso} e
  \textit{Restare in asso}, cioè esser lasciato solo, se bene altri gli assegnano altra origine: o
  pure da \textit{Asino} che così chiamavano ne' dadi \textit{l'unità} i Greci, dicendola \textit{Onos}. Il
  nostro Proverbio: \textit{O asso, o sei} i Greci dicevano, \textit{o diciotto, o tre}. \textit{O tre sei, o tre
    assi}. Giulio Polluce lib. 9, al cap. di giuochi fanciulleschi, e de' trattenimenti deg1i antichi.
\item[PAZZO Cittadino] Questo epiteto si suol dare \textit{a colore che fanno tutte le lor cose
  a caso, e senza considerazione}; ed è lo stesso che dire \textit{un cervellaccio}.\footnote{Pazzo, possibilmente dal latino \textit{Patior}, soffrire, sopportare.}
\item[SBANDARE] Disfare le bande, cioè licenziare i Soldati.
\item[TRUCCHI via] Se ne vada. E' modo basso, cavato forse dalla parola \textit{Zeruck}
  Tedesca profferita da i Lanzi, quando con le loro alabarde fanno allontanare
  il popolo; O forse dal giuoco del Trucco, che si dice \textit{truccare}, o \textit{trucciare}
  la palla, quando cogliendola con un'altra palla si manda via dal luogo, dove
  era; dal frequentativo Latino \textit{trusare} usato da Catullo.
\item[TOCCO] Con il primo o largo; Specie di berrettone, che anticamente usava
  in Firenze in vece di cappello. Varch. Stor. lib. 11. \textit{Con le calze soppannate di
    teletta bianca, e le berrette, o vero tocchi di colore rosso}.
\item[SOLLEVATE l'anche] Alzati i fianchi, cioè rizzatosi da sedere, che \textit{anca}
  diciamo quella parte del corpo, che è fra il fianco, e la coscia, da \textit{Ancon} greco,
  che vuol dire gomito; e si piglia per ogni sorta di piegatura, come lo mostra il
  nome della Città d'Ancona così detta dal gomito, che fa quivi la spiaggia;
  Plinio \libcap[3]{13}. \textit{In iisdem colonia Ancona apposica promontorio Cumero in ipso
    flectentis se orae cubito}.

  Dan. Inf. canto 34.
  \begin{verse}
    Quando noi fummo là dove la coscia
    Si volge appunto sul grosso dell'anche.
  \end{verse}

  E di qui sciancato è un zoppo, che habbia mancamento in tal luogo. Vedi
  sotto \cstan[11]{40}. E il Latino \textit{Coxendices}.
\item[CAPITOMBOLO] È quando uno, posando il capo in terra, volta sopr'a
  quello tutta la vita. Vedi sotto C.~7. st.~20.
\item[O RE, cui splende in mano sì gran forcone] Fingono che Nettunno Re del mare
  fratello di Plutone usi in vece di scettro una forca con tre punte, e però detta
  Tridente, la quale in realtà è una fiocina da pescatori, Latino \textit{fuscina}, e Plutone
  un Bidente, cioè \textit{forca con due punte}; E questo è il \textit{gran forcone}.
\item[IL Cappello Speziale] È uno Speziale in Firenze, che fa per insegna un cappello.
\item[PEDIGNONE] Enfiagione che viene ne i piedi, e nelle mani per causa del
  freddo.  Latino \textit{Pernio}. Vedi sopra C. 3. st. 6.
\item[LO vuol pagare in sul tappeto] La vuol pagar per via di Corte, con tutte le
  solennità, non vuol pagar se non gli mandano j birri a gravarlo, o catturarlo,
  e però dice che Baldone \textit{giuocherò di calcagna}, cioè fuggirà per la paura
  d'esser preso per debito, quando vedrà \textit{Pedino}, che così si chiamava uno già birro
  della Mercanzia, che è il Magistrato, per via del quale si mandano l'esecuzioni civili.
\item[DI posta] Subito. Latino è \textit{vestigio}. Traslato dal giuoco di palla, che si dice
  dar \textit{di posta} quando si dà alla palla prima, che tocchi terra. Vedi sotto C. 7, st.92.
\end{description}
\section{Stanza LXXXIV --- LXXXIX}

\begin{ottave}
\flagverse{84}Pluton diede con tutti una risata,\\
Che feceli stiantar fino il brachiere,\\
E dissegli: Va via bestia incantata,\\
Com'entra con l'assedio il dare, e havere?\\
Segue l'altro che vien della pancata.\\
Rizzato Barbariccia da sedere\\
Si china, e mentre abbassa giù la chioma\\
Alza le groppe, e mostra il bel di Roma.
\end{ottave}

\begin{ottave}
\flagverse{85}Poi s'intirizza, e dice in rauco suono:\\
Se non si leva dalle squadre ll capo,\\
Quale è Baldone, e non si dà nel buono,\\
Mai si verrà di tal negozio a capo,\\
Dove, se manca lui quanti vi sono,\\
Restati come mosche senza capo,\\
A poco a poco, a truppe, e alla sfilata\\
Partendo, in breve disfaran l'armata.
\end{ottave}

\begin{ottave}
\flagverse{86}Circa il pigliarlo, s'io non l'ho, egli è fallo:\\
Facciam conto ch'in branco alla pastura\\
Un toro sia costui, o un cavallo;\\
Tiriamgli addosso qualche accappiatura\\
Legata innanzi a un bel mazzacavallo\\
Collocato in castel presso alle mura,\\
Ond'ei si levi un tratto all'aria, e poi\\
Si tiri dentro, e dove piace a noi.
\end{ottave}

\begin{ottave}
\flagverse{87}Buono, rispose il Re, non mi dispiace;\\
Ma il Cancellier di subito riprese:\\
Sia detto, o Senator, con vostra pace,\\
Tant'oltre il poter nostro non s'estese,\\
Il tutto saria nullo, e si soggiace\\
Ad esser condennato nelle spese,\\
Ed io sarei stimato anc'un Marforio;\\
Acconsentir a un'atto perentorio.
\end{ottave}

\begin{ottave}
\flagverse{88}Perché sempre \textit{de iure} pria si cita\\
L'altra parte a dedur la sua ragione,\\
Poi s'ella è in mora, viensi a un'inibita,\\
E non giovando, alla comminazione,\\
Ch'in pena caschi delle forche a vita,\\
E se la parte innova lesione,\\
Allor può condennarsi, havendo osato\\
Di far causa pendente un'attentato.
\end{ottave}

\begin{ottave}
\flagverse{89}Sommelo anch'io, che in altro tribunale\\
Si tien, dice Pluton, cotesto stile,\\
Ma qui, dove s'attende al criminale,\\
S'esclude ogni atto, e ogni ragion Civile.\\
Ma sia com'ella vuole, o bene, o male\\
Io vuò levar quest'huom da Malmantile;\\
Però chetiamci, e dica Calcabrina;\\
E quei si rizza, e verso il Re s'inchina.
\end{ottave}

Plutone, ridendo con gli altri della proposizione di Malebranche, ordina al
secondo, che viene nella pancata, nominato Barbariccia, che dica il suo pensiero,
e questo propone che si tiri un laccio a Baldone, e per via d'un mazzacavallo
s'alzi, e si porti dove più piacerà; ma ciò non è approvato dal Cancelliere, onde
Plutone ordina al terzo nominato Calcabrina, che dica il suo parere, e costui
si rizza, e fa riverenza al Re per far il discorso, che sentiremo nelle seguenti
Ottave.
\begin{description}
\item[SCHIANTARE] Rompere, spezzare detto da \textit{Spiantare}. E Brachiere è
  quello, che si disse sopra C, 3, st. 5.
\item[BESTIA incantata] Così diciamo per esprimere un'huomo faceto, e buffone,
  traslato da quelle bestie, che alle volte conducono con loro i Montanbanchi, alle
  quali essi fanno far molti giuochi, e dicono che tali: bestie sieno incantate, ed
  operino per vie diaboliche. Si dice \textit{bestia incantata} a uno di poca considerazione,
  ed avvedimento, come il Lalli En, Trau. C. 20. 56.
  \begin{verse}
    \backspace Così gridammo, e con la propria zappa
    Ci dessimo in sul pié bestie incantate
  \end{verse}
\item[COM'entra con l'assedio] Significa come s'accorda, o che ha che fare con
  l'assedio.
\item[IL bel di Roma] Così diciamo per intender apertamente \culo, perché il \textit{bel di
  Roma} intende il Colosseo, da noi corrottamente detto \textit{Culiseo}.
\item[S'INTIRIZZA] Si rizza, si distende in su la persona. È un'atto, che denota
  una certa superbia, e presunzione di se stesso, ed è quella presopopea,  che
  dicemmo sopra C, 1, st. 72.
\item[NON si verrà a capo di tal negozio, ec.] Non si conchiuderà, o terminerà il negozio.
\item[RESTATI come mosche senza capo] Cioè senza consiglio, direzione, o guida.
  Senza sapere che cosa havere a fare, o risolvere: Poiché questi insetti scemi del
  capo, s'aggirano inutilmente, strascicando il restante del corpo, senza saper
  dove.
\item[ALLA sfilata] Senza ordine; confusamente, e senza andare in fila, o in
  ordinanza: Sbandati.
\item[S'IO non l'ho, egli è fallo] Io son sicuro di pigliarlo. Se io nol lo piglio, sarà
  per errore. È specie di giuramento vantatorio, come \textit{apponlo a noi} che vedremo
  sotto \cstan[8]{72}. \textit{È mio danno} che vedremo \cstan[10]{49}.
\item[ACCAPPIATURA] Una fune accomodata, e fattovi un cappio con un
  nodo, che scorra, il qual nodo si dice \textit{cappio scorsoio}.
\item[MAZZACAVALLO] È un corrente, o pertica grossa congegnata per traverso,
  e come posta a cavallo sopra un legno ritto; la quale s'alza da una parte
  con tirare a basso la parte opposta. E questo ordingo è usato assai ne i piani di
  Firenze per cavar l'acqua dai i pozzi. I Latini lo dissero \textit{tollenonem a tollendo},
  che è forse simile a quella macchina, della quale si servivano i nostri antichi a
  scagliar pietre chiamata \textit{Mangano}. Livio dice: \textit{In ariete Tollenonibus libramenta
    plumbi, aut saxorum, stripitesue robustos incutiebant}. Questa macchina militare
  vien descritta da Vegezio così; \textit{Tolleno dicitur, quoties una trabs in terram praealte
    defigitur, cui in summo vertice alia transversa trabs longior, dimensa medietate,
    connectitur, ed libramento, ut si unum caput depresseris, aliua erigatur}. L'antico Volgarizzamento
  \textit{Altaleno è detto, quando una trave alta si ficca in terra, alla quale nel
    capo di sopra una altra trave più lunga per lo traverso, e nel mezzo misurata, si commette
    in tal modo che se l'uno capo si china, l'altro in alto si leva}. Da questa voce
  \textit{altaleno} (Lat. \textit{tolleno}) si dice l'\textit{Altalena} giuoco, che i ragazzi fanno con due travi
  incrociate, e bilicate l'una sopr'all'altra a foggia di Mazzacavallo. Vedi sopra
  C.2. stan. 48, Mattio Franzesi contro alle sberrettate dice.
  \begin{verse}
    Ma chi trovasse il modo a bilicallo,
    Sarebbe un schifanoia, e faria bene
    Un contrappeso d'un mazzacavallo.
  \end{verse}
\item[SIA detto con vostra pace] Perdonatemi; s'io v'offendo in dirlo, Non vi adirate,
  non vi offendete, s'io lo dico. Frase de' Latini \textit{Pace tua hoc dicam}.
  Nell'epigramma di Quinto Catulo, \textit{Pace mibi liceat, Coelestes, dicere vestra. Mortalis
    visus pulcrior esse Deo}. Che Annibal Caro nel primo Sonetto delle sue Rime voltò.
  \textit{Volsimi, e 'ncontra a lei mi parve oscuro, Santi Numi del Ciel, con vostra pace
    L'Oriente, che dianzi era sì bello}.
\item[ESSER condennati nelle spese] Cioè buttar via la fatica, e il denaro, \textit{oleum, \&
  opera perdere}. Ma propriamente esser condannato nelle spese vuol dire, quando
  uno per aver litigato una cosa ingiusta, e dal giudice condannato a rifar tutte
  le spese all'avveriario; e però questo Cancelliere dice, che non vuole acconsentire
  a tale atto per essere ingiusto, e \textit{da esser condannato nelle spese}.
\item[SAREI stimato un Adarforio] Sarei stimato un'huomo senza sentimento, o giudizio;
  come è la statua di Marforio in Roma.
\item[ATTO frustratorio]  Atto vano, e fatto senza proposito, E questo termine,
  come tutti gli altri delle seguenti stanze 88. e 89. son termini Curiali, che venendo
  dal latino, ed essendo praticati in tutti li Tribunali d'Italia non dubito, che
  saranno intesi da ognuno; però ne tralalcio la spiegazione.
\end{description}

\section{Stanza LXXXX. STANZA LXXXXIL}

\begin{ottave}
\flagverse{90}E poi c'ha fatte riverenze in chiocca \\
Co' suoi pié lindi a pianta di pattona, \\
Si soffia il naso, e spazzasi la bocca, \\
E posta in equilibrio la persona,\\
Come quel che si pensa dar' in brocca,\\
Tutto sfrontato dice: Alta Corona, \\
Circa l'ordingo, pur si metta in opra; \\
Perch'io concorro, e affermo quanto sopra.
\end{ottave}

\begin{ottave}
\flagverse{91}Ma in vece di quel cappio da beltresca,\\
Ch'è il tossico de' ladri, si provvegga\\
Una bilancia, o rete per la pesca,\\
Con una lunga fune, che la regga;\\
E perch' il fatto meglio ci riesca\\
Si tinga tutta, acciò che non si vegga,\\
E in terra quanto ell'apre, ivi si spanda,\\
Fino ch'il porco vengane alla ghianda.
\end{ottave}

\begin{ottave}
\flagverse{92}Perché s'e' muovuon l'armi, di ragione\\
(Se dal capo l'esercito è condotto)\\
Innanzi a tutti marcerà Baldone,\\
E quand'ei giunga, ed ha la rete sotto,\\
Fate che leste allor sien più persone\\
A farla tirar fu con l'avannotto,\\
Operando in maniera, ch'egli insacchi\\
In luogo, ove si vede il sole a scacchi.
\end{ottave}

\begin{ottave}
\flagverse{93}Questo dice Plutone, ha più disegno,\\
Ma il Cancellier di nuovo s'attraversa,\\
Con dire: o laccio, o rete habbia quel legno\\
È tutta fava, \textit{\& idem per diversa},\\
Perché manca il Cipolla a questo segno,\\
Concede il molestar la parte avversa,\\
Se poi comandi, anch'io non me ne parto,\\
Lodando il \textit{suspendatur} con lo squarto.
\end{ottave}

\begin{ottave}
\flagverse{94}Qui, dice il Re, si dà sempre in budella, \\
Sì che mi cascan le braccia, e l'ovaia,\\
Mentre costui a ogni cosa appella,\\
E co' suoi punti mena il can per l'aia;\\
Gli ha sempre più ritorte che fastella,\\
Ma e' non lo crede s'ei non va a Legnaia.\\
Horsù dite costà voi Cappelluccio.\\
Ed ei si rizza, e cavasi il cappuccio.
\end{ottave}

Il terzo Diavolo, che è Calcabrina, dopo haver fatta riverenza al Re, ed una
mano di smorfie, come fanno certi Oratori affettati, dice, che approva il
mazzacavallo, ma che in vece del cappio scorfoio piglierebbe una rete da pescare. Ma
il Cancelliere s'oppone; onde Plutone sgridando il medesimo Cancelliere, ordina
al quarto Diavolo, che è Cappelluccio, che dica il suo parere.

\begin{description}
\item[IN chiocca] In quantità grande, in abbondanza, un diluvio di riverenze.
\item[PATTONA] Specie di pane fatto di farina di castagne, che per l'esser per lo
  più di figura lunga, s'assomiglia a un piede mal fatto di un'huomo. Fam. Strada,
  Prolusione Plautina prima dice: \textit{Qui enim pedibus sunt planis ploti vocantur}; sì
  che piede di pattona si può dir \textit{plotus} dalla voce Latina \textit{Plautus}, che significa lo stesso;
  e questa dal Greco \textit{Platys}, lato, largo; donde noi a tali huomini, che hanno
  i piedi malfatti diciamo \textit{Piloti}. Vedi sopra C. 4. st. 17. Il Franzese dice \textit{Patte}; lo
  Spagnuolo \textit{Pata} il suolo del più di bue, gatto, oca, e simili; dal Gr. \textit{Patein}, che
  vuol dire battere col pié; calpestare; calcare; E Patán similmente in Ispagnuolo
  è il contadino, che porta le scarpe grandi, e grosse, e rozzameate fatte.
  Potrebbe anche esser detta \textit{Pattona}, in un certo modo quasi \textit{Pastona}, cioè \textit{pastaccia},
  \textit{pasta grossa}, perché è quella a similitudine d'un pastume grossolano, e mal fatto.
  \textit{Pattume} disse Ser Brunetto nel Patassio quello, che oggi dichiamo \textit{Pacciume}; cioè
  \textit{spazzatura, e mescuglio di cose fracide}; e ciò pure cred'io, dal Greco \textit{Patein} calpestare.
  \textit{Ed si pattume vien rammuricando}. Il che ha qualche similitudine con
  \textit{Pattona}, cosa sordida, e vile, e di brutto colore. I Greci (per dire anche questo)
  lo sterco, perché si scarica il ventre lungi dalla strada comunale, che dall'essere
  strada battuta si dice \textit{Patos}; dissero \textit{Apopatema}, il che può aver dato origine alle
  parole \textit{Pattume}, e \textit{Pattona}. Gli dice \textit{lindi}, ma per ironia, che in vece d'intendere
  piede ben fatto, \& attillato, vuol dir piede sconcio, e mal fatto. Lindo è parola
  venuta a noi modernamente di Spagna; e ì come \textit{senda} in quella lingua viene
  dal Latino \textit{semita}, e \textit{linde} dal Latino \textit{limite}; così \textit{lindo} credo che sia detto quasi
  \textit{limito}, cioè limitato, aggiustato, ben assetto, composto. Da \textit{Lindo} diciamo anche
  \textit{Allindarsi}, e \textit{Allindirsi} Sp. \textit{alindarse}.
\item[SI soffia il naso, e spazasi la bocca] Ei purga il naso, e sputa, e con la lingua si
  netta i denti, che sono quei lezzj, che fanno molti Oratori, come \textit{porre in equilibrio
    la persona}; cioè dopo haver dimenato in qua, e in là il corpo, fermarsi in positura
  \textit{intirizzato}, come ha detto nell'Ottava antecedente, che sono tutte smorfie,
  che denotano nell'Oratore una sciocca superbia, e presunzione di se stesso;
  ed il Poeta lo tocca col verso che segue, dicendo: \textit{Come quello che se pensa dare in brocca},
  che vuol dire, stima di haver trovata l'invenzione buona, e d'haver imbroccato,
  cioè dato nel segno.
\item[TUTTO sfrontato] Arditamente, sfacciatamente. I) Franzese similmente ¢f-
\item[BERTESCA] o Bertresca, o beltresca; E' una specie di cateratta, che s' alza,
  e s'abbassa, e serve per riparo di guerra in su le torri, e in su le mura fra un
  merlo, e l'altro; o così si dice ogni luogo, sopr'al quale si salga con pericolo
  di precipizio. Di qui viene il verbo \textit{bertescare}, o \textit{bertrescare} usato da molti per
  intendere Armeggiare, o affaticarsi intorno a un lavoro, e non trovar la via a
  farlo. Qui  per \textit{bertesca} intende la forea; per similitudine delle \textit{bertesche}, le quali
  erano edifizzi di legname, che si ponevano in alto. Gio. Villani lib. 9. 114. \textit{Perché
    il porto era tutto impalizzato, e incatenato e di sopra di grosso legname
    imbertescato}. Queste bertesche, o torri di legname alzate su le mura doveano servire
  cose a gettar pietre, onde forse è la parola \textit{pertrechos}, che significa
  presso i Spagnuoli munizioni, e ripari da guerra, cioè le nostre \textit{bertesche},
  detta forse così da \textit{echar las pedras}.
\item[BILANCTA] Specie di rete da pescare, detta così per esser a foggia di bilancia,
  strumento, col quale si pesa la roba.
\item[QUANT' ella apre] Cioè quant'ella allarga per ogni verso.
\item[FINO a ch'il porco vengane alla ghianda] Fino a che venga a dare nella trappola;
  si cali al zimbello. E s'intende fino a che Baldone andando alla volta di Malantile
  dia nella rete suddetta.
\item[SIENO lesle] Se bene lesto vuol dir Agile. Vedi sopra C. 1, st, 11. Tuttavia
  \textit{star lesto} vuol dire star pronto, all'ordine, o preparato.
\item[AVANNOTTO] Pesce piccolissimo. Voce corrotta da Uguannotto, o Unguannotto,
  che significa pesce nato quell'anno: perché \textit{uganno}, o \textit{unguanno} vuol
  dir quest'anno, se bene usato solo nel contado, e l'Autore se ne serve in bocca
  a un contadino sotto C. 10. st. 35. I Latini dicevano \textit{Hornus}, ed \textit{hornotinus} una
  cosa d'un'anno. Il Poeta dà nome d'avannotto a Baldone, perché dovea esser
  preso con la bilancia, che è la rete, con la quale si pigliano gli avannotti.
\item[IN luogo, ove si vegga il Sole a scacchi] Cioè in prigione; perché le finestre ferrate
  della prigione, battendovi i raggi del Sole, fanno a figura dello scacchiere,
  nel luogo dove termina il loro sbattimento, o ombra dei ferri. Da queste finestre
  ferrate, o \textit{grate} di ferro delle prigioni, si formò il verbo \textit{Aggratigliare} usato
  dal Bocc. Nov. 85. \textit{Tu m'hai aggratigliato il cuore colla tua ribeba}, cioè \textit{imprigionato}
  col suono della tua \textit{ribeca}, come oggi diremmo: e da Brunetto nel Patassio.
\item[TVTTA è fava] Tutta è una stessa cosa. \textit{Sol est Apollo, \& ipse Apollo Sol}, Dice
  il Cornazzano Nov. 11. che fu una Signora, la quale volendo riprender copertamente
  il marito, perché lasciando lei andava dalle Meretrici, gli fece un
  lautissimo desinare, ogni vivanda era condita, e ripiena di fave con diversi stravaganti
  ma delicati sapori. Il marito le domandava; Che cosa e questa? ed ella
  rispondeva; \textit{Fava}, E quest'altra? \textit{Fava}. In somma gli disse in ultimo:
  Signor marito scegliete quanto volete, perché \textit{tutta è fava}; Onde egli intesa l'arguta,
  e faceta riprensione della moglie, mutò vita, conoscendo che da una donna
  all'altra non può esser'altra differenza, che quella che nasce da un soverchio
  sfrenato appetito. E di qui poi venne il dettato \textit{Tutta è fava} che significa è tutt'una,
  e come \textit{idem per diversa}.
\item[IL Cipolla] Autore noto\footnote{Bartolomeo Cipolla (Verona, 1420 circa – Padova, 10 o 11 maggio 1475) giurista, diplomatico italiano, le sue opere, di poco anteriori alla stampa, continuarono ad essere riedite per secoli, fino all'Ottocento. }, che ha scritto in Criminale.
  a Plutone, che se bene quivi, \textit{esclusa ogni ragione Civile s'attende al Criminale}.
  Tuttavia gli Autori criminali non approvano quell'operazione. Ma in ultimo si
  rimette dicendo; Se tu lo comandi, io non ho che replicare, e concorrerò quand'anche
  tu lo volessi far'impiccare, e squartare; che questo intende \textit{suspendatur con
    lo squarto}.
\item[SI dà in budella] Non si conchiude cosa di buono, Questo proverbio, si dice
  copertamente: \textit{Far come il cane del peducciaio}, e s'intende dare in budella, che
  esprime discorrer'assai, e conchiuder poco, ed è lo stesso che dar in cenci.
\item[MI cascano le braccia, e l'ovaia] Mi perdo d'animo affatto. Si dice \textit{cascar il
  cuore, le braccia, le brache, il fegato, il fiato}, e da molti \textit{l'ovaia} per intender
  copertamente \textit{i testicoli}, e tutti hanno lo stesso significato, di perdersi d'animo. E
  qui accoppiandone due, cioè le braccia, e l'ovaia, esprime perdersi affatto d'animo.
  Latino \textit{ovaria}, che si sono scoperte ultimamente nelle donne,\footnote{Erano di poco precedenti gli studi di Reinier De Graaf (Schoonhoven, data di battesimo 30 luglio 1641 --- Delft, 17 agosto 1673) medico, anatomista), che riconobbe la funzione delle ``trombe di Falloppio'', identificate da Gabriele Falloppio (1523–1562)} dagli antichi
  erano creduti, e detti i loro testicoli.
\item[A OGNI cosa appella] Non c'è cosa che stia a suo modo, dà difficultà a ogni
  cosa, a ogni cosa ha che dire; e non se ne sta, e non se n'acquieta, detto
  dell'\textit{appellarsi} termine legale.
\item[CO' suoi punti mena il can per l'aia] Co' suoi punti legali, e con le difficultà,
  che oppone, manda in lungo le cose senza venire a conclusione alcuna. \textit{Aia}
  vien dal latino \textit{area}, e vuol dir quel pezzo di terra spianata, ed accomodata per
  battervi, e mandarvi sopra il grano, e biade.
\item[HA più ritorte, che fastella] Ha più ripieghi, e compensi, che non gli accidenti,
  che succedono, ovvero egli trova subito riparo a ogni accusa. \textit{Ritorte}
  si dicono quei legami fatti di vinciglie d'alberi, con i quali si legano i fasci di
  legne, e di fieno, o d'altro, detti \textit{ritorte}, perché quella vinciglia si attorce per
  renderla maneggiabile, e flessibile a fine d'adattarla a legare. Dan. Inf, C. 19.
  \begin{verse}
    Che spezzate haverian ritorte, e strambe.
  \end{verse}
\item[EI non lo crede] Questo termine significa Tu non ti vuoi emendare; e si dice
  \textit{Non crede al Santo, se non fa miracoli}; cioè non crede d'haver a esser gastigato, fin
  che ei non prova il gastigo. Qui dice \textit{se ei non va a legnaia}, cioè se egli non è legnato,
  e bastonato: Legnaia e un borghetto vicino a Firenze, ed il nome di \textit{Legnaia}
  ci serve per esprimere legnate, o bastonate. Vedi sotto C. 11, st. 11. \textit{grattar
    la tigna}. Dove si mettono diversi modi di dire per intendere Bastonar uno.
\item[CAPPUCCIO] Il Varchi Stor. Fiorentina lib. 9. dice:
  \begin{adjustwidth}{8pt}{}Il Cappuccio ha tre
    parti: il Mazzocchio, che è un cerchio di borra coperto di panno, che gira, e
    fascia d'attorno alla testa, e di sopra, soppannato dentro di rovescio cuopre
    tutto il capo. La Foggia è quella, che pendendo in su le spalle, difende la
    guancia sinistra. Il Becchetto è una striscia doppia del medesimo panno, che
    va fino in terra si ripiega in su la spalla, e bene spesso s'avvolge al collo, e
    da coloro che voglion' esser più destri, e più spediti intorno alla testa, ec.
  \end{adjustwidth}
  E questo è il cappuccio, che già portavano le persone civili, e del quale parla il
  Poeta. Vedi sopra C. 4. st. 7. alla voce Mazzocchio.\footnote{DICONO ch'ei dorma in un granaio}
\end{description}
\section{STANZA LAXAXV.}
\begin{ottave}
\flagverse{95}E disse: Io dico che direi, o Sire, \\
Poi che da te ch'io dica mi vien detto, \\
Ma dir non oso, ch'io non ho che dire, \\
Se non quanto qui quell'altro ha detto; \\
Perch'ei l'ha detto con sì terso dire,\\
Ch'io sto per dir che mai s'udì tal detto;\\
Però dico ch'a dir non mi dà il cuore,\\
E lascio dire a un'altro dicitore.
\end{ottave}

Cappelluccio, che è il quarto diavolo, fatte sue cirimonie, fa un discorso senza
conchiusione, come si vede nella presente Ottava tutta di scherzo sopra il verbo
\textit{dire}, la quale non richiede spiegazione, ma solo riflessione al grazioso, ed
ingegnoso artifizio del Poeta.

\section{STANZA XCVIIL}
\begin{ottave}
\flagverse{96}Anch'io l'ho detto, che tu sei un buffone,\\
Risponde il Re, e in tanto Libicocco\\
Tagliar ad Arno l'argine propone,\\
Acciò nel campo l'acqua habbia lo sbocco:\\
E come vuoi, risponde allor Plutone\\
Mandar Arno all'insù, viso di sciocco?\\
E poi dal fiume d'Arno a Malmantile\\
V'è un ghiandellino. Dica Baciapile.
\end{ottave}

\begin{ottave}
\flagverse{97}Questo, che fa il baseo, ma è tristo, e accorto,\\
E perch'egli è auditor d'ipocrisia,\\
Veste cilizio, e con viso smorto\\
Canta sempre laldotti per la via,\\
Risponde a occhi bassi, e viso smorto:\\
Fate motto di là in Cancelleria:\\
E qui va in mezzo, bacia terra, e in fine \\
Tornando al luogo piovon discipline.
\end{ottave}

\begin{ottave}
\flagverse{98}Voltati, dice il Re, spropositato;\\
S'alcuna cosa qui non hai proposta,\\
Come vuoi tu buaccio che 'l Senato\\
Vada in Cancelleria per la risposta?\\
Pur sento, rispond'ei, ch'in Magistrato\\
Così dir s'usa, ed io l'ho detto apposta;\\
Ma s'io vi scandolezzo, e alcun m'incolpa\\
D'errore in questo, io me ne rendo in colpa.
\end{ottave}

\begin{ottave}
\flagverse{99}Non occorre brunir co i labbri i sassi,\\
Dice Plutone, ossaccia senza polpe,\\
E fare il torcicollo, e ovunque passi\\
Seminar discipline, e dir tue colpe,\\
Ch'io so, che chi per lepre ti comprassi,\\
Havrebbe almen tre quarti della volpe;\\
Pera va a siedi, e segua il Tiritera;\\
E quei s'assetta, e parla in tal maniera:
\end{ottave}


Plutone riprende Cappelluccio, ed in tanto il quinto Diavolo, che è Libicocco
propone di fare sboccar' Arno in Malmantile, qual consiglio è riprovato come
impossibile; Onde Plutone ordina al sesto Diavolo, che è Baciapile, il proporre,
e questi dice, che vadano in Cancelleria per la risposta, che è lo stesso che
proporre nulla, però Plutone lo sgrida, ed ordina al Tirirera che è il settimo
Diavolo, che dica, ed egli s'accinge a parlare.

\begin{description}
\item[BUFFONE] Quel che significhi dicemmo sopra C, 3, st.27. E il Latino \textit{scurra}.
\item[UN ghiandellino] Un poco poco. E qui, essendo detto ironico significa; e un
  spazio da Arno a Malmantile.
\item[BASEO] Balordo, melenso, stupido, basoso, A questa voce allude la Franzese
  \textit{esbahi}, smarrito, {confuso}, {quasi sbasito}. E \textit{far il baseo} vuol dire finger di non intendere,
  o fingersi huomo senza giudizio, dal verbo \textit{basire} visto sopra \cstan[2]{79}.
  È lo stesso che \textit{far la gatta di masino}, o \textit{la gatta morta}, visto sopra C. 1, st. 19.
\item[AUDITOR d'ipocrisia] E' un grandissimo ipocrito. La voce Ipocrito vien dal
  Greco \textit{Hypocrinesthai}, che suona \textit{contraffare}; e l'Ipocrisia si difinisce Una
  callida\footnote{``Il callido Ulisse''}, ed astuta palliazione del vizio occulto; perché Ipocrito si chiama colui, che
  essendo uno scellerato, nondimeno nell'abito; negli atti, e nelle parole mostra
  d'esser buono, e s'affatica di parere quel che egli non è, e propriamente \textit{hypocrita}
  significa \textit{commediante, istrione}. S. Agostino nel Sermone del Venerdi dopo la
  Domenica della Quinquagesima.
  \begin{adjustwidth}{8pt}{}
    Hypocrita Greco sermone simulator interpretatur,
    qui, dum intus malus sit, bonum se palam ostendit. Hypo enim
    falsum, crisim vero iudicium sonat. Nomen autem hypocritae translatum est a
    specie eorum, qui spectaculis tecta facie incedunt, distinguentes vuleum coeruleo,
    niveoque colore, \& coeteris pigmentis, habentes simulacra oris lintea gypsata,
    \& vario colore distincta, nonnumquam colla, \& manus creta perungentes,
    ut ad personae colorem pervenirent, \& populum, dum in ludis agerent, fallerent,
    modo in specie viri, modo in forma feminae, \& reliquis praestigijs.
  \end{adjustwidth}
  Il  Berni nell'Orlando contra gl'Ipocriti, \textit{Non han da fare le maschere a i Cristiani}.
  Questi sciagurati sono di tre sorte. La prima è di coloro, che fingono nel
  cospetto degli huomini d'esser pieni di religione, ed internamente sono ateisti.
  La seconda è di coloro, che fanno del bene non mossi dalla virtù, o dall'amore
  del bene, ma per esser creduti buoni. La terza è di coloro, che dimostrano di
  non esser buoni, perché altri credano, che eglino sien buoni da vero, e non
  ipocriti. In questo Diavolo si scorgono tutte tre queste specie d'Ipocriti, che
  appresso di noi sono lo stesso, che Bacchettoni; detto sopra \cstan[2]{1}. Dante
  nell'Inf. C. 23. parlando di loro dice:
  \begin{verse}
    \backspace Laggiù trovammo una gente dipinta,
    Che gira attorno assai con lenti passi.
    Piangendo, e nel sembiante stanca, e vinta.
  \end{verse}
  E qui dice \textit{viso smorto}, cioè faccia pallida, e scolorita; e dice che \textit{piovono
    discipline} per intender uno di tali Bacchettoni falsi o diciamo Ipocrito. E sotto
  nell'ottava 99. seguente dice, \textit{Seminar discipline}, che ha lo stesso senso. E s'usa
  assai il servirsi di questi due termini per esprimere: È passato per questa strada
  un bacchettone. Veramente questi tali infami non lasciano di valersi di tutte le
  sorte d'apparenze, ed io ne conosco uno della prima specie d'Ipocriti, che trovandosi
  in una pubblica adunanza, in cavarsi ii fazzoletto di tasca lasciò cadere
  una disciplina a vista d' ognuno; ed essendogli detto, che avvertissi, che gli era
  cascato non so che dalla tasca, egli raccogliendola disse: Non è mia roba, Non
  son così buono, che io adopri tali arnesi. \textit{Disciplina} chiamiamo quella sferza,
  che le persone veramente buone adoprano a battersi per far penitenza; così detta
  dall'ammunire, ovvero gastigare il corpo, per renderlo servo ubbidiente al suo
  Signore, e ben disciplinato; cioè instrutto del suo dovere, che è la summissione
  alla ragione. L'uso frequente della disciplina cominciò in Toscana, e si diffuse
  per tutta Italia, e si eressero Compagnie de' Disciplinanti, o Battuti l'anno 1260.
  \textit{Sigonius de Regno Italiae}.
\item[SPROPOSITATO] Uno che non sa, ne dice cosa a proposito.
\item[BVACCIO] Ignorantaccio. Che si dice anche \textit{Asinaccio}, \textit{Castronaccio},
  \textit{babbuasso}, \textit{bue di panno}. Vedi sopra \cstan[3]{49}. la voce \textit{arfasatto}. I Latini pure
  havevano diverse voci, che esprimevano questo stesso, come si vede in Plauto Bacchid.
  Act. 5. Sc. 1, dove dice.
  \begin{adjustwidth}{8pt}{}
    Quicumque ubique sunt, qui fuere, quique futuri
    sunt posthac Stulti, stolidi, fatui, fungi, bardi, blenni, buccones, Solus
    ego omnes longe ante eo stultitia; \& moribus indoctis, E Terent. in Heaut. 5.
    In me guiduis harum rerum convenit quae sunt dicta in stultum, caudex,
    stipes, asinus plumbeus.
  \end{adjustwidth}
\item[L'ho detto a posta] È quello, che i Latini dicono \textit{ultro}, \textit{consulto}, ovvero \textit{dedita
  opera}. Cioè non per errore, o inconsideratamente.
\item[VI scandolezzo] Il verbo scandolezzo portato dal Greco al Latino, e dal Latino
  a noi, ha significato d'inciampare, e d'adirarsi come vedemmo sopra \cstan[1]{56}.
  e se gli da anche il significato di quelle parole \textit{Si oculus tuus scandalizat te
    \&c}.\footnote{Matthaeus 5:29 Vulgata Clementina. Avviata da Sisto V (1585-90), istituita sotto Gregorio XIV (1590-1591), terminata sotto Clemente VIII (1592-1605), è ancora oggi la traduzione latina più utilizzata negli studi storici e filologici.} come è nel presente luogo, che preso in significato attivo vuol dire: Se io vi
  dò occasione di far errore: se io vi sono cagione d'inciampo; \textit{si tibi offensioni sum};
  \textit{offensionem afero}, per esempio, \textit{Io credeva, che il tale fusse huomo da bene, ma il sentir
    poi, che egli dà a usura, m'ha scandolezzato}, cioè fatto mutare il concetto, che
  haveva di lui.
\item[BRUNIR co' labbri i sassi] Brunire, parlandosi di materiali sodi come ferro,
  osso, oro, ec, vuol dire Dar il lustro, e però intende qui dar il lustro ai sassi co
  i labbri, baciandoli spesso, atto, che si fa da i Cristiani devoti per segno d'umiliazione
  sopra \cstan[2]{9}. disse \textit{dar il lustro a' marmi co i ginocchi}.
\item[OSSACCIA senza polpe] Carne cattiva, perché quando si compra la carne, che
  sia con molto osso, si dice: Vi è poco del buono; e da questo dicendosi a un
  huomo \textit{ossa senza carne} s'intende tristo, ribaldo, o scellerato.
\item[CHI ti comprasse per lepre]\items{havrebbe almeno tre quarti di volpe} Chi ti credesse
  semplice, troverebbe poi in te tre quarti almeno di malizioso, o furbo. In Latino
  si direbbe: \textit{Pro simplici columba, astuta vulpes}. In tutta questa Ottava narra
  molte di quelle azioni che fanno gl'Ipocriti, e Bacchettoni falsi.
\end{description}
\section{Stanza C.}

\begin{ottave}
\flagverse{100}Io che sono un'insano, e ignaro ogni hora,\\
Perché saper supir non voglio, o vaglio,\\
Dico c'al Duca, perché a' muri ei mora\\
Tosto in testa si dia pel meglio un maglio,\\
Finché lo spirto sporti al foro fora,\\
Dond'ei fa i peti, e pute d'oglio, e d'aglio,\\
Acciò l'accia su l'aspo doppo addoppi\\
La Parca, e il porco con la stoppa stoppi.
\end{ottave}


Il Tiritera, che è il settimo Diavolo propone che si dia in sul capo a Baldone,
e s'ammazzi. Il Poeta lo fa parlare in bisticcio a imitazione del Pulci nel suo
Morgante lib. 23. che dice.
\begin{verse}
  \backspace  La casa cosa parea bretta, e brutta
  Vinta dal vento la natta, e la notte,
  Stilla di stelle, c'a tetto era tutta,
  E suina, e suena di botto una botte.
  Pere havea pure, e qualche fratta frutta,
  Del pane a pena ne dette a tal dotte,
  Poscia che pesci, e lasche prese all'esca,
  Il letto allotta alla frasca fu fresca.
\end{verse}
\begin{description}
\item[MAGLIO] Dal Lat. malleus. Martello grande di legno per uso di battere i
  cerchi alle botti, o per ammazzare i buoi, o per altri lavori di legname, nei
  quali si richieggano percussioni gagliarde, e gravi.
\item[SPORTARE] Avanzare in fuora, come avanzano le gronde de i tetti fuori
  delle muraglie delle case; donde \textit{Sporti} quelle aggiunte che son fatte alle case
  fuori del muro maestro, e rette da' beccatelli, sorgozzoni, o colonne, (in Latino
  \textit{Meniana}, che il Filandro sopra Vitruvio definisce \textit{protectae proiectaeque pergulae,
    dictae a Menio, \&c.}) e qui vuol dire \textit{scappi, o esca fuori lo spirito}.
\item[PETO] Quel romore che fa il vento scappando all'huomo dalle parti di basso.
  Lat. peditus.
\item[ASPO] E' un bastoncello con due traverse in croce contrapposte, e distanti
  alquanto l'una dall'altra, sopra vi quale si raguna il filo per ridurlo in matasse,
  detto dall'annaspare, \textit{Naspo}, e poi, \textit{Aspo} atrimenti \textit{Guindolo} onde \textit{Agguindolare}.
\item[PARCHE] Le tre donne appellate \textit{Cloto}, \textit{Atropo}, e \textit{Lachesi}, e dette Parche,
  \textit{quia nemini parcunt, sive quod parce, \& pene avare vitam tribuant}. La Gentilità
  stimava, che queste fussero Figliuole dell'Erebo, e della Notte, secondo Cic. \textit{de
  Natura Deor}, e secondo altri, che fussero Figlie di Demogorgone; e credevano, che
  figurassero le tre cose necessarie all'huomo, cioè il nascere, il vivere, e il morire;
  dicendo che una di loro detta \textit{Cloto} fila, che è il nascere, la seconda detta
  \textit{Atropo} annaspa, che è il vivere, la terza detta \textit{Lachesi} taglia il filo, che è il
  morire. Le chiamarono anche Nona, Decima, e Morte.
\end{description}
\section{STANZA Cl. STANZA CHL —}
\begin{ottave}
\flagverse{101}Ben tu puzzi di pazzo ch'è un pezzo, \\
Disse Pluton, bestiaccia, per bisticcio, \\
Perch'io per me non so, ne raccapezzo\\
Quel che tu voglia dir nel tuo capriccio, \\
Ma non son Re, s'io non te ne divezzo,\\
E perché tu non temi grattaticcio,\\
Mentre stima non fai delle bravate,\\
Quest'altra volta le saran pecciate.
\end{ottave}

\begin{ottave}
\flagverse{102}Hor via seguite: Qui lo Scamonca \\
Si rizza, in viso tutto insanguinato, \\
Perch'ei, ch'è un fastidioso, appunto havea \\
Fatto a graffi con un, che gli era allato.\\
Però con la bisunta sua giornea,\\
La qual traluce come Ciel stellato, \\
Sì ch'ella un'Argo par fatto alla macchia, \\
Si netta, al Re s'inchina, e così gracchia:
\end{ottave}

\begin{ottave}
\flagverse{103}Io non so se Baldon sogna, o frenetica,\\
Perché s'ei vuol sturbar la nostra pratica\\
Fa male i conti, e con la sua aritmetica\\
Nel zero l'ho fra l'una, e l'altra natica;\\
Poiché s'un bacchio il capo a lui solletica,\\
Sbrattar l'armata non sarà in grammatica,\\
Che tutta a brache piene, ancor che stitica\\
Tremando andranne come paralitica.
\end{ottave}

\begin{ottave}
\flagverse{104}Olà, dove siam noi (dice Plutone)\\
E che sì, scorrettaccio, ch'io ti zombo;\\
Darò ben'io sul capo a te il forcone,\\
Sì c'alle stelle n'anderà il rimbombo.\\
Guarda quel'che tu dì porco barone,\\
E va più lesto, e col calzar del piombo;\\
Sta ne i termini, e parla con giudizio\\
Che per mia fe ti privo dell'ufizio.
\end{ottave}


Plutone dopo haver riprefo il Tiritera, comanda, che 'dica Scamonea ottavo
Diavolo; il quale dà anch'egli un consiglio spropositato, e con parole sporche,
onde Plutone lo sgrida, minacciandolo di levargli la degnità Senatoria, se egli
non s'avvezza a parlare con termini onesti, e rispettosi.
\begin{description}
\item[BISTICCIO] È la figura che i Greci dicono \textit{Parechesi}, ed è quando si dicono
  due parole che hanno lo stesso, o poco differente suono, e diverso significato,
  come si vede nell'antecedente ottava 100. e ne i due primi versi della presente
  101. Detto \textit{Bisticcio} quali \textit{Disticcio} dal Latino greco \textit{Distichum}; nella stessa forma
  che \textit{Bistorto} è fatto dal Lat. \textit{distortus}; \textit{Bistento}, dal Lat. \textit{Distentus}, \textit{Bistrattare} quasi
  \textit{Distrattare}, cioè  maltrattare, e simili. Imperciocché i primi bisticci, de' quali ci
  sono rimasti gli Esempi, consistevano in Distici, o vogliam dire coppie di versi
  rimati colla stessa voce, la quale significava due cose diverse, secondo che o più larga,
  o più stretta, o intera, o dimezzata si, profferiva. Fra Guittone d'Arezzo,
  nella Raccolta di Poeti Antichi di Mons. Allacci, tutta una Canzone va tessendo
  diq euste allusioni di parole ed è quella che si trova a carte 385. nella Licenza,
  o conclusione della qual Canzone dice cosi:
  \begin{verse}
    Movi canzone adessa,
    E vanne a Rezzo ad essa,
    Da cui contegno, ed o,
    Se n'ancun ben mi do,
    E dì che presto so,
    Se vuol di tornar so.
  \end{verse}
\textit{Adessa} in primo luogo vale \textit{ad hanc ipsam horam}, siccome \textit{adesso} vale \textit{ad hoc ipsum tempui}.
Nel secondo luogo \textit{ad essa} savuol dire \textit{ad essa mia donna, a lei}. Il primo \textit{ed o},
vale \textit{\& habeo}. Il secondo, \textit{mi do}, L. \textit{me dedo}. Il primo \textit{so} vuol dir \textit{sono} verbo. Il secondo
\textit{suo} nome. Ne sono esempi in Bindo Bonichi\footnote{Bindo Bonichi (Siena, 1260 – Siena, 3 gennaio 1337) religioso, poeta. }, ed in Francesco da Barberino\footnote{Francesco da Barberino (Barberino Val d'Elsa, 1264 – Firenze, 1348) notaio, poeta. }.
\item[NON raccapezzo] Non so ridurre a capo: Non rinvergo: Non rinvengo:
  Non ritrovo: Non intendo.
\item[CAPRICCIO] Qui vuol dire opinione, o pensiero. Vedi sopra C.~1, st.~21.
\item[NON son Re] Laicio d'esser Re. È termine giuratorio che esprime Tanto
  è vero che io ho fatta, o farò la tal cosa, quanto è vero che io sono quale io sono.
  Non son Padre di Telemaco, cioè non sono Ulisse se io non ti frusto; Disse
  Ulisse a Tersite presso a Omero.
\item[S'io non te ne divezzo] S'io non ti fo lasciar questo vizio, o questo tuo modo
  di trattare. È il contrario d'avvezzare. Vengono da \textit{Vizio} quasi \textit{avviziare} per
  assuefare a un vizio: e \textit{disviziare} per liberare da un vizio. E questi due verbi
  tanto attivi, che neutri hanno sempre lo stesso significato. Diciamo per esempio
  \textit{avere il vizio del tabacco}, cioè \textit{essersi assuefatto a pigliarne}.
\item[TU non temi grattaticcio] Tu non fai stima de i piccoli gastighi; Tu non temi
  le bravate, e non curi le riprensioni. Nelle Raccolte de' Greci trovasi un certo
  verso iambico, che voltato in Latino suona così:
  \begin{verse}
    Incus maxima non timet strepitus.
  \end{verse}
  E \textit{grattaticcio} intendiamo grattatura, che leggiermente offende la cute.
\item[PECCIATE] Percosse nella peccia, Calci nel ventre. Termine basso, e più
  tosto scherzoso. Peccia lo istesso, che \textit{pancia}, se bene della parte, che è dallo stomaco
  al pettignone \textit{Peccia} pare più verso lo stomaco, \textit{Pancia} più verso il
  pettignone. Questa è dal Latino \textit{pantices}; intestini; quella forse dallo Spagnuolo \textit{pecho}.
  Latino \textit{pectus}, onde \textit{Rimpecciare}.
\item[BISUNTA giornea] Veste assai unta. E per giornea s'intende la sopravveste
 de i soldati, che i Latini dicono \textit{Chlamydem}, e Lispiglia per veste d' aurorita,
 donde habbiamo un proverbio che dice.
\item[AFFIBBIARSI la giornea] Che significa presumersi molto di se medesimo. Il Lalli
  En. tr. lib. 1. st. 102, parlando di Didone dice:
  \begin{verse}
    Come Diana allor che uscirne a caccia,
    Lungo l'Eurota, o pure in Cinto suole,
    Fra tutte l'altre la giornea s'allaccia
    E suol parer fra le sue Ninfe un sole.
  \end{verse}
  Il Forti, parlando della Prammatica delle donne al cap. \textit{mihi} 242. e cava le
  parole da i libri pubblici di questa Città, dice: \textit{Non potevano portare alcuna giornea
    o mantello o altro vestito sparato, ne maniche sparate, o tagliate per il lungo delle braccia}.
  Donde si deduce, che questa era una sopravveste, o zimarra aperta tutta
  dinanzi, usata anche dagli huomini di conto nelle case. Ma da noi hoggi si piglia
  per toga, o veste curiale, che chiamiamo lucco, e nel. presente 1uogo vuol
  dir questo.
\item[TRALVCE] Traspare; E s' intende, che era piena di buchi, perché soggiunge
  pare un'Argo fatto alla macchia, cioè s'assomiglia a un'Argo malfatto,
  Argo fu quei pastore, che havea cento occhi, e fu lasciato da Giunone in guardia
  d'Io figliuola d'Inaco convertita da Giove in vacca; ed a questi occhi assomiglia
  i buchi, che erano nella veste di Scamonea. Plauto, se ben mi sovviene
  chiamò casa illustre quella, per la quale per essere il tetto rotto, si vedeva il Cielo.
  Quel che voglia dire \textit{dipingere alla macchia}, vedilo sopra C. 1. st. 69, dove
  vedrai anche il significato di \textit{gracchiare}.
\item[PRATICA] Intendiamo Consulta, o Congreffo di Consultori dallo Spagnuolo
  \textit{Platica} ragionamento, discorso, donde \textit{Praticare um negozio} vuol dir trattare,
  o maneggiare un negozio. Varchi St. Fior. lib. 14. \textit{Ragunasi la Pratica, e deliberò,
    che per esser la Città ferma, non faceva bisogno fare altra spesa}. Ma questo diavolo
  credo, che intenda \textit{sturbar la nostra pratica}, cioè dar disturbo a Martinazza
  nostra amica, perché \textit{haver una pratica} si dice quand'uno ha, o si tiene qualche
  donna, o innamorata: e corrobora questa opinione il sapere, che Baldone non
  sturbava il Consiglio de' Diavoli, ne li loro congressi, o pratiche, ma sturbava
  Martinazza con assediar Malmantile.
\item[L'HO nel zero] L'ho nel forame: Non lo stimo. Zero è la figura tonda dell'Abbaco
  detta forse da Giro, la quale forma le decine, e per similitudine s'intende
  il forame, e ci serviamo di questa parola per coprire il detto sporco: t'h in
  \culo{}, usatissimo fra la gente bassa in questo significato di disprezzo; e qui torna
  bene, perché dice \textit{con tutta la sua aritmetica}, cioè abbaco, \textit{io l'ho nel zero}, che è
  figura di aritmetica.
\item[BACCHIO] Bastone, o pertica dal Latino baculus. E \textit{solleticare} qui intende
  perquotere; e parla ironico, perché le bastonate sono contrarie del solletico.
\item[NON sarà in gramatica] Non sarà difficile, e che ci voglia grande studio.
  \textit{Gramatica} presso gli antichi volea dire \textit{lingua Latina}, come quella, per intender
  la quale ci bisognava lo studio della gramatica. E perciò la Greca antica, ovvero
  Ellinica, e litterale, che si conferma solamente nelle scritture; a differenza della
  volgare, e moderna, la quale oggi si parla, corrotta da quell'antica, e si chiama
  \textit{Romeca}, cioè \textit{Greca de' sempi bassi}, ne' quali i Greci non più tennero il lor antico
  nome di \textit{Hellines}, ma per gl'Imperatori Romani, che in Oriente avevan trasferito
  l'imperio \textit{Romei} cominciaronsi a nominare; quella Greca antica, dico, trovasi
  chiamata \textit{gramatica greca}; perché gli odierni Greci per apprenderla hanno bisogno
  di gramatica, si come noi per imparare la Latina. Nel principio dell'antico
  Volgarizzamento manoscritto delle vite di Plutarco si legge. \textit{Qui comincia la
    Cronica di Plutarco, la quale fue traslatata di gramatica greca in volgare greco in Rodi, ec.
  } E perché la Grammatica è cosa spinosa, e difficile; per questo il dichiarare, e
  agevolare l'intelligenza di qualche fatto, o questione oscura, e imbrogliata dichiamo
  \textit{sgramaticare}.
\item[A BRACHE piene] Per la paura si moverà loro il ventre, e s'empieranno le
  brache. Vedi sopra C. 1. st. 43.
\item[STITICO] Uno che difficilmente ha il benefizio del corpo.
\item[COME paralitica] Cioè tutta tremante come sono i paralitici.
\item[DOVE sia noi?] Dove credi tu d'essere ? Termine che significa \textit{Porta rispetto
  alle persone, ed al luogo dove tu sei}. Alessandro sentendosi recitare da uno, che
  aveva distesa la Storia de' suoi fatti, una narrazione lontana dal vero; disse allo
  Storico, \textit{E dove eramo noi allora?} quali dicesse: Che non ti ricordi; che io v'era
  presente? Altre volte significa: \textit{Che non hai giudizio?} per esempio \textit{Tu dai cento
    scudi al tale, che non ha haver 50, dove siam noi?} cioè dove siam noi col cervello?
\item[E CHE sì?] Termine usato per indurre timore, ed ha del giuratorio; \textit{E che sì,
  ch'io ti zombo?} quali dica: \textit{Giuro che sì; ch'io ti zomberò, se tu non parli meglio}. Si
  usa assai per fare star a segno i fanciulli, \textit{E che sì, che io vengo costà, e vi sferzo}.
  Si dice anche, \textit{Vale, o giuochiamo, o stiamo a vedere, che io vi sferzo?} Un Poeta
  moderno se ne servì per \textit{giochiamo}\footnote{sic, ``giochiamo'', di seguito ad un ``giuochiamo''.}, dicendo:
  \begin{verse}
    \backspace E che si, padron mio, ch'io m'indovino
    Del vostro andar girando la cagione?
  \end{verse}
\item[SCORRETTACCIO] Huomo scorretto diciamo colui, che senza rispetto alcuno
  dice parole sporche ed oscene, ed indecenti in ogni luogo.
\item[ZOMBARE] Perquotere. È il Latino \textit{Verberare}, Dal suono. Così \textit{Typto} de'
  Greci, che vuol dire \textit{verbero}, è verbo fatto dal suono; onde ne nacque \textit{Typanon},
  e \textit{Tympanon}, il Tamburo; dal quale abbiam fatto noi \textit{Tamburare}, e \textit{Tambussare};
  e da \textit{Tympanum}, \textit{Zombare}. Appresso i Greci \textit{bombos} e il rombo, o romore delle
  Pecchie: appresso i Latini \textit{bombus} è il suono che fa il corno. Appresso di noi \textit{Bombarda}
  è detta dal gran rimbombo nello spararsi; e così tutte queste lingue si sono
  accordate, contraffacendo il suono medesimo, che da cose concave uscendo, e
  rigirando, e ampliandosi pervene all'orecchio.
\item[RIMBOMBO] Risuonamento, l'Eco, cioè quel suono che resta alquanto
  dopo un romore, e massime ne i luoghi cavernosi. Dante Inf. C, 16,
  \begin{verse}
    \backspace Già era il loco, ove s'udia il rimbombo
    Dell'acqua che cadea nell'altro giro
    Simil a quel che l'arnie fanno rombo.
  \end{verse}
\item[VA col calzar del piombo] Cammina adagio, e considerato nelle tue operazioni.
  Governati con prudenza, Lat. \textit{Matura lente}. Dante Par. C. 13,
  \begin{verse}
    \backspace E questo ti sia sempre piombo a' piedi
    Per farti muover lento come buon lasso,
    Ed al sì, ed al no, che tu non vedi.
  \end{verse}
\end{description}

\section{STANZA CV. --- CVII.}
\begin{ottave}
\flagverse{105}S'alza Scorpione allora, e vien da esso\\
D'Astolfo il Corno orribile, proposto,\\
Che gli eserciti dice in fuga ha messo\\
Conforme scrive, e accerta, l'Ariosto.\\
Si rallegra Pluton, e dice: Adesso\\
Non ci sarà dal Cancelliere opposto\\
Perché ci calza bene, e certo questa\\
Cosa del corno a me va per la testa.
\item
\end{ottave}

\begin{ottave}
\flagverse{106}Risponde sogghignando Ciappelletto:\\
(Ch'in tal modo si chiama il Cancelliere)\\
Voi già m'havete per dottore eletto,\\
E non ch'io serva qua per candelliere:\\
Per mio debito dunque io son costretto\\
A dire all'occorrenze il mio parere:\\
Su, dice il Re, Dottor de' miei stivali\\
Metti anche il Corno in termini legali.
\end{ottave}

\begin{ottave}
\flagverse{107}Vuoi forse darci qualche eccezione?\\
Stiamo in decretis; dì, peto vestito;\\
Va ben, risponde il Sere, ch'ei propone\\
Cosa, che non deprava ordine, o rito.\\
Sognate un doppio, disse allora Mammone,\\
Ch'ei la passò, facciam dunque il partito,\\
Perch'ella segua di comun consenso,\\
E ognun favorirà siccome io penso.
\end{ottave}


Fatta che hebbe Plutone la bravata a Scamonea, si rizzo Scorpione nono Diavolo,
e propose, che si pigliasse il Corno d'Astolfo, il che piacque a Plutone, e
per questo si voltò al Cancelliere domandandoli, se ci haveva difficultà, ed egli
l'approvò; Onde Plutone ordinò, che si facesse il partito.
\begin{description}
\item[SOGGHIGNARE] Mostrare, o far segno di ridere quali da \textit{subchachinnari}. Se bene
  in sua forza è il latino \textit{subridere} ed è un certo ridere per segno di disprezzo,
  o di poca stima, che altri faccia di qualcosa, e si, chiama riso annacquato,
  cioè non puro, non vero; ma finto.
\item[IO non son qui per candelliere] Io non son qui solamente per far nuemro, ma
  devo dire ancor'io il mio parere, quando occorra.
\item[DOTTOR de' mei stivali] Termine di disprezzo, e vuol dire Dottore da nulla.
  Vedi sopra C. 4, st. 10.
\item[PETO vestito] Che cosa sia peto, vedemmo nell'ottava 100.\ antecedente. E
  quando il vento esce dalle parti da basso accompagnato con qualcosa altro, si dice
  peto vestito. E da questo il Lettore può comprendere quel che significhi.
\item[SONATE un doppio] Quand'altri dopo molte cose mal fatte ne fa una bene
  dal medesimo solita farsi di rado, o vero dopo, che uno habbia terminata una
  faccenda con grande stento, ed in molto tempo, diciamo: \textit{Sonate un doppio}, cioè
  tutte le campane per l'allegrezza di questa cosa insolita, o della terminazione
  di questa faccenda, che si pensava non havesse a esser terminata mai.
\item[FAR il partito] Far lo scrutinio, che noi volgarmente diciamo far lo \textit{squittino},
  e \textit{squittinare}.
\end{description}

\section{Stanza CVIII. \& CIX.}

\begin{ottave}
\flagverse{108}Vanno le fave attorno, e di lupini,\\
E sentesi stuonato, e fuor di chiave \\
Alle panche gridar: Tavolaccini,\\
Raccogliete pel numero, e le fave \\
Pigliate in man; che questi cittadini, \\
Che in simil Luogo far dovrian sul grave \\
Rendano (il capo havendo pien di baie) \\
Male i partiti, e mangian le civaie.
\end{ottave}

\begin{ottave}
\flagverse{109}Vanno i donzelli ognun dalla sua banda,\\
Ma perché ne ricevan mille scherzi,\\
Che più nessuno ardisca il Re comanda,\\
Se non vuol, che a pien popolo si sferzi,\\
Di nuovo attorno i bossoli si manda\\
Da vincersi il partito pe' due terzi,\\
E cercate alla fin tutte le panche,\\
Fu vinto non ostante cento bianche.
\end{ottave}


Si fa lo scrutinio, ed i donzelli vanno raccogliendo i voti co' bossoli, e finalmente
non ostante cento voti in contrario fu vinto, che si pigliasse il Corno d'Astolfo
per far diloggiar Baldone da Malmantile. E qui termina il sesto Cant.
\begin{description}
\item[IL corno d'Astolfo] Vedi l'Ariosto nel suo Orlando furioso, che lo finge un
  Corno, il cui suono fugava la gente.
\item[VANNO le fave attorno, ed i lupini] È costume in Firenze, come era anche
  in Atene, di fare i partiti, o squittini con fave, e lupini; e però havendo il Poeta
  veduto, che nel Consiglio grande di Firenze chiamato il Consiglio dei Dugento,
  nel quale intervengono centinaia, e centinaia di persone (come in questo
  Consiglio de' Diavoli è necessario, che intervenissero sopra 300. Demonj, mentre
  cento voti non impedivano il vincere il partito) i Tavolaccini, e Donzelli vanno
  distribuendo le fave, ed i lupini a coloro, che devon rendere il partito, fa
  usare il medesimo costume nel presente consiglio de' Diavoli, dove dice che si sentì
  gridare \textit{stuonato}, e \textit{fuor di chiave}, cioè in voce, che non intuona, e non accorda,
  e questo procede, perché essendo più d'uno, ed in diverse parti della stanza a
  gridare, è impossibile che s'accordino nel tuono, come anche perché dette voci
  son profferite fra tanta gente, che bisbiglia, il che le rene ottuse, ed offuscate.
\item[TAVOLACCINO] Servo, o Donzello di Magistrato; così detto secondo alcuni
  da \textit{Tabellio} detto sopra in questo C. st. 74., ma io credo, che i Tavolaccini,
  che sono un numero determinato, e differenti dagli altri Donzelli, sieno quelli
  che al tempo della Repubblicha stavano sempre in palazzo, e servivano alla tavola
  de' SS. ciascuno il suo, e due n'haveva il Gonfaloniere, e si dicevano Tavolaccini
  dal servire alle Tavole; e che habbiano conservato il nome, sì come si
  conserva ancora l'ufizio, essendo costoro obbligati a andare a servire alle tavole
  in palazzo del Sereniss. G. Duca in occasione di Forestieri, o di Sposalizzj, ec.
  ma per altro aprono ogni mattina, e serrano ogni sera le Porte della Città.
\item[RACCOGLIE le fave per il numero] A fine di saper con facilità, quanti sieno
  coloro, che rendono il voto, il Tavolaccino piglia in mano da ciascuno una fava,
  e queste poi si contano, ed indicano il numero de i votani, e questo si dice
  \textit{raccorre per il numero}. E pigliano le fave in mano, e non nel bossolo, per assicurarsi
  che non vi sia chi ne metta più d'una, ed alteri il numero.
\item[STAR sul grave] Tener il decoro, la gravità. Star savio.
\item[HA il capo pien di baie] Sempre vuole scherzare.
\item[RENDER jl partito] È quel dare, o metter la fava, o lupino nel bossolo, che
  si dice: \textit{dare il voto}.
\item[A PIEN popolo] In presenza, ed a vista di tutto il popolo.
\item[BOSSOLO] Quel vaso, nel quale si mettono i voti dagli Ateniesi detto \textit{Camus}.
  Vedi sopra C. 1. st. 37.

\end{description}
\section*{FINE DEL SESTO CANTARE.}

\chapter{Settimo Cantare}

\begin{argomento}
Paride dop' haver molto bevuto
Entra, d'andar' al campo, in frenesia,
E come il sonno havea pel ber perduto,
Perde nel gir di notte anche la via:
Cade in un fosso, onde a donargli aiuto
Corron le Fate, e gli usan cortesia;
Vien condotto in un'antro, e per diporto
La storia gli è narrata di Magorto.
\end{argomento}

\section{Stanza I. --- IV.}

\begin{ottave}
\flagverse{1}\textit{Vino tempera te} disse Catone,\\
Perché si dee berne a modo, e a verso,\\
E non come colà qualche trincone,\\
Che, giorno, e notte sempre fa un verso;\\
Ond' ei si quoce, e perché ei va a Girone,\\
La favola divien dell'universo,\\
E vede poi morendo in tempo breve\\
Ch'è ver, che chi più beve manco beve.
\end{ottave}

\begin{ottave}
\flagverse{2}S'il troppo vino fa, che l'huom soggiace\\
A tal' error di tanto pregiudizio;\\
Chi non ne beve, e quello, a cui non piace,\\
A questo conto dunque ha un gran giudizio;\\
Anzi che no, sia detto con Sua pace,\\
Per c'ogni estremo finalmente è vizio,\\
E se di biasmo è degno l'uno, e l'altro\\
Questo ha il vantaggio al mio parer senz'altro.
\end{ottave}

\begin{ottave}
\flagverse{3}Perché se quel s'ammazza, e non c'invecchia,\\
Ed è burlato il tempo di sua vita,\\
Almen sent'il sapor di quel ch'ei pecchia,\\
E tien la faccia rossa, e colorita.\\
Burlar anche si fa chi va alla secchia,\\
E insacca senza gusto acqua scipita,\\
Che lo tien sempre bolso, e in man del Fisico,\\
Il qual l'aiuta a far morir di tisico.
\end{ottave}

\begin{ottave}
\flagverse{4}Però sia chi si vuole egli, è un dappoco\\
Chi imbotta al pozzo come gli animali,\\
S'avvezzi a ber del vino a poco a poco,\\
Ch'ei fa che l'acqua fa marcire i pali;\\
Ma com'io dico si vuol berne poco\\
Basta ogni volta cinque, o sei boccali,\\
Perch'egli è poi nocivo il trincar tanto,\\
Com'udirete adesso in questo Canto.
\end{ottave}

Volendo il Poeta narrare in questo Canto l'accidente occorso a Paride Garani,
per haver troppo bevuto, s'introduce col riflettere, che siccome è male il ber
molto vino, così che sia anche male il bere solamente acqua; e conchiude, che
dovendosi eleggere uno dei due mali, sia meglio eleggere quello del ber vino
ma pero regolatamente.

\begin{description}
\item[A MODO, e a verso] Regolatamente, E' il latino vulgato: \textit{modis, \& formis},
  cioè \textit{rite, decenter}.
\item[TRINCONE] Uno che beva assai. Da \textit{Trinchen} Tedesco bere, tirar giù.
  Vedi sopra C. 1, st. 6. Si dice anche \textit{pecchiare} nella presente Ottava terza, quasi
  succiare il vino come fanno le pecchie, (cioè l'api che fanno il miele, così dette
  dal Latino \textit{apiculae}) le quali succiano il dolce da i fiori, ed i vini bianchi generosi:
  e dal detto verbo \textit{pecchiare} si dice \textit{pecchione} a uno, che beve assai; e \textit{pecchione}
  si chiama un'ape salvatica, e maggiore dell'altre, che succia il miele prodotto dall'api
  da' Latini chiamato \textit{fucus}. Virg. \textit{Ignavum fucos pecus a praesepibus arcent}.
  Si dice \textit{cioncare} nella presente Ottava quarta. Vedi il Landino esposizione a Dante
  Inf. C, 9, alla parola \textit{cionca} nel verso \textit{Che sol per pena ha la speranza cionca}, dove
  dice, \textit{che cionco è parola Lombarda, e significa mozzo, ma cioncare in Fiorentino
    significa disordinatamente bere}; Sì che questi tre verbi \textit{trincare}, \textit{pecchiare}, e \textit{cioncare}
  hanno lo stesso significato, e se bene hanno del forestiero, tuttavia sono usati in
  Firenze
\item[SEMPRE fa un verso] Sempre fa la medesima cosa. Diciamo \textit{Verso} il canto
  dell'uccello, Verso del rusignuolo, Verso del fringuello. E da tal verso viene
  il presente dettato.
\item[VA A Girone] Huomo, che gira; intendiamo pazzo. E però servendoci della
  voce Girone, che è un Villaggio vicino a Firenze, copertamente intendiamo
  uno che fa delle pazzie, come s'intende nel presente luogo.
\item[DIVIEN la favola dell'universo] È burlato da tutti. \textit{In ore est omni populo}, Il
  Lalli En. Tr. C. 4. st. 78.
  \begin{verse}
    Son fatta oimé la favola del mondo
  \end{verse}
  Il Petr. \textit{Ma ben veggio or, sì come al popol tutto Favola fui gran tempo}. Tibullo lib.\ 1.
  \textit{Ne turpis fabula fiam}. Nella scrittura. \textit{Et factus sum illis in parabolam}.

\item[CHI più beve manco beve] Cioè, chi troppo beve s'ammala, e muore, e così vive
  poco, e per conseguenza beve manco, cioè dura a bere manco tempo di colui,
  che beve poco. Marz. lib. 6. \textit{Immodicis brevis est aetas, \& rara senectus}, che da
  noi poi si dice in proverbio \textit{Poco ci vive chi troppo sparecchia}. A similitudine di
  questo si dice: \textit{Chi più studia, manco studia}.
\item[OGNI estremo è vizio] Ogni estremo e male. Ogni troppo è troppo. Questa
  sentenza usiamo dirla \textit{Il troppo, e il poco Guasta il giuoco}, Al che pare, che facciano
  molto a proposito i seguenti versi di Orazio.
  \begin{verse}
    \backspace Est modus in rebus, sunt certi denique fines,
    Quos ultra, citraque nequit consistere rectum.
  \end{verse}
  E Terenzio mettendo in Latino una sentenza d'un savio della Grecia disse: \textit{Ne
    quid nimis},
\item[SENZ'altro] Assolutamente; senz'alcun dubbio. Latino \textit{sane}, \textit{procul dubio}.
\item[VA alla secchia] Beve acqua. Secchia diciamo quel vaso, col quale si cava
  l'acqua da i pozzi dal Latino \textit{situla}. Vedi sopra C. 5. st. 10.
\item[INSACCA] Per similitudine diciamo sacco al ventre dell'huomo; quindi \textit{Insaccare}
  vuol dir Mandar giù nel ventre. Pulci Morg. C. 19. st. 137.
  \begin{verse}
    E mangia, e beve, e insacca per due verri
  \end{verse}
  Per il contrasto \textit{sacar} in Ispagnuolo è trarre, cavar fuori.
\item[SCIPITA] Che non ha sapore alcuno. Dal Latino \textit{Insipidus}.
\item[BOLSO] Vedi sopra C. 3. st. 53. Grasso non naturale, con difficultà di respiro.
  Cavallo bolso i Franzesi dicono \textit{pousif} dal pulsare, cioè battere i fianchi per
  la lena affannata. Lucano lib. 4. \textit{Pectora rauca gerunt, quae creber anhelitus urget,
    Et defecta gravis longe trahit ilia pulsus}.
\item[IN man del Fisico] Col medico sempre attorno; cioè sempre infermo,
\item[CHI imbotta al pozzo] Chi beve sempre acqua. E' lo stesso che \textit{insaccare}
  detto sopra.
\item[ANIMALE] Intende animale irrazionale. Se bene la voce animale è generica,
  e comprende sotto di se anche l'huomo, noi ce ne serviamo per speciale, intendendo
  solamente le bestie, sì che dicendosi a un'huomo \textit{Tu sei un'animale}, intendiamo
  \textit{Tu sei una bestia}; \textit{Un'irragionevole}.
\item[S'AVVEZZI] S'assuefaccia. Vedi sopra C. 6, st. 101.
\item[FA marcire i pali] Vuol dire: il vino si guasta annacquandolo, quasi dica; fa
  infradiciare i pali, che reggono le viti, che producono il vino; o pensa se farà
  infradiciare il vino, che nasce dalle viti, che sono più deboli de i pali, mentre
  son da essi sostenute. Dichiamo anche per biasimare l'uso dell'acqua: \textit{l'acqua
    rovina i ponti}: quasi s'abbia a intendere. O pensate, se non rovinerà gli stomachi
  degli huomini, che sono più deboli!
\item[BOCCALE] È una misura capace della metà d'un fiasco Fiorentino\footnote{Come menzionato sopra \cstan[1]{58}, la metà di un Fiasco Fiorentino è poco più di un litro, circa 1.14 litri.}. Dice
  cinque o sei boccali per scherzo, sapendo bene, che ogni maggiore bevitore non
  beverà mai sì gran quantità in una volta.
\end{description}
\section{Stanza V. --- VII.}
\begin{ottave}
\flagverse{5}Omai serra gli ordinghi, e le ciabatte \\
Chiunque lavora, e vive in sul travaglio, \\
E difilato a cena se la batte\\
A casa, o dove più gli viene il taglio.\\
Chi dal compagno a ufo il dente sbatte,\\
Tanti ne va a taverna ch'è un barbaglio,\\
Parte alla busca, e infin, pur che si roda,\\
Per tutto è buona stanza, ov'altri goda.
\end{ottave}

\begin{ottave}
\flagverse{6}E Paride, c'anch'egli si ritrova\\
A corpo voto in quelle e catapecchie,\\
D'Amor chiarito figlio d'una Lova,\\
Che svaligiar gli ha fatto le busecchie,\\
Dice al villan: Va a comprarmi dell'uova\\
Ecco sei giuli, tonne ben parecchie,\\
Piglia del pane, e sopra tutto arreca\\
Buon vino sai! non qualche cerboneca.
\end{ottave}

\begin{ottave}
\flagverse{7}E se t'avanza poi qualche quattrino,\\
Spendilo in cacio, non mi portar resto:\\
Messer sine, rispose il Contadino,\\
Io torrò, s'io ne trovo, ancor cotesto.\\
E partendo gli ride l'occhiolino, \\
Sperando haver a far un po d'agresto;\\
Ma, facendo i suoi conti per la via,\\
S'accorge che e' non v'è da far calìa.
\end{ottave}


Descrive assai vagamente il venir della notte; su la quale ora Paride assalito
dalla fame comanda a Meo suo contadino, che vada a comprar roba da
mangiare, e da bere, e per tale effetto gli dà sei giuli\footnote{Giulio, è nome di moneta papalina, che si applica pure al Paolo toscano, poco più di mezzo franco.}, con ordine che gli
spenda tutti.
\begin{description}
\item[ORDINGHI] Intende ogni sorta d'arnesi, ingegni, macchine, e strumenti per
  lavorare. Diciamo anche \textit{Ordigni}; anzi gli antichi non dissero altrimenti.
\item[CIABATTE] Vuol dir propriamente scarpe vecchie, e quelle scarpe all'Appostolica,
  che usano i frati scalzi, ma s'intende anche ogni frammento di materiali
  di coloro che lavorano, e per ogni sorta di masseriziuole vecchie, e consumate,
  che i Latini dicono \textit{scruta}.
\item[VIVE in sul travaglio] Latino \textit{manibus victum quaeritat}, \textit{Campa delle sue braccia}.
  Travagliare in lingua Francese vuol dir lavorare, ed in Firenze pure è usato in
  questo senso dicendosi: \textit{cosa ben travagliata} in vece di ben lavorata; e di qui si dice
  \textit{travaglio} in vece di viver col lavoro, o con le sue fatiche, cioè di quel che si
  guadagna a lavorare. Petr. C. 3.
  \begin{verse}
    \backspace A qualunque animale alberga in Terra,
    Se non se alquanti c'hanno in odio il Sole,
    Tempo da travagliare è, quanto è il giorno:
    Ma poi che 'l Ciel accende le sue stelle,
    Qual torna a casa e qual s'annida in selva,
    Per aver posa almen infin all'alba.
  \end{verse}
  Se ben per altro \textit{travagliare} vuol dire esser' angustiato da infermità, o da altro.
\item[DIFILATO] A dirittura: Latino \textit{recta}. Con prestezza, e senza fermarsi.
  L'Autore se ne serve anche sotto in questo C. st, 63. Varchi Stor. Fior, lib. 9.
  \textit{Raffaello non prima giunto a Firenze che andandosene difilato, senza pur cavarsi gli
    stivali a Palazzo}.
\item[SE la batte] Se ne va via. E' termine assai usato fra la gente bassa per esprimere
  fuggir via, o partirsi in fretta, ed ha del furbesco \textit{battere la calcosa}, cioè
  \textit{batter la strada}, andar via, camminare, donde \textit{strada battuta} vuol dire strada,
  che è spesso camminata, o strada di passo. Latino \textit{via trita}, Lucrezio \textit{Avia
    Pieridum peragro loca, nullius ante Trita solo}, 1\ Petcarcha disse: \textit{Ogni segnato calle,
    Provo contrario alla tranquilla vita}.
\item[DOVE gli viene il taglio] Dove gli torna più comodo. Vedi sopra C. 2. st. 48.
\item[A UFO] Senza spendere. E' detto plebeo. Si scrivono da i Magistrati di Firenze
  lettere di commissioni ai Ministri forensi, le quali da coloro, che le chieggono,
  e le presentano, si pagano a i Magistrati, che le fanno, ed a i Ministri,
  che le ricevono; e quando non sono chieste, ma son fatte, e mandate per proprio
  interesse di quel Magistrato, che le fa, non vi è spesa alcuna, e però affinché
  tali lettere, le quali non si pagano, si possano distinguer da quelle, che si pagano,
  scrivono nella soprascritta \textit{ex offitio}, ma l'abbreviano scrivendo \textit{ex Vffo}, ed i tavolaccini,
  o donzelli, che le consegnano non leggono se non \textit{ex Ufo}, e distinguono
  queste due specie di lettere, dando a quelle, che si pagano il nome di lettere
  \textit{col diritto}, cioè con la dovuta spesa, ed all'altre il nome dell'\textit{Ufo}, cioè senza
  spesa. E di qui è nato questo detto \textit{a Ufo}, che vuol dir senza spesa, e serve in
  ogni occasione.
\item[SBATTE il dente] Cioè mangia.
\item[È un barbaglio] Son tanti, che fanno abbagliare; Non se ne può raccorre il
  conto senza sbagliare; o abbarbagliarsi, cioè errare; dal Parpaglione, che dissero
  gli antichi alla Provenzale; cioè dal Latino \textit{papilio}; farfalla; di cui è noto l'errare
intorno al lume.
\item[ALLA busca] Cercando sua ventura. \textit{Buscare}. Vuol dir Acquistare, ottenere,
  guadagnare. E dalla Spagnuola \textit{buscar} venuta a noi questa voce insieme
  con molte altre negli ultimi tempi.
\item[SI roda] Si mangi. Se bene \textit{rodere} si dice de' topi, de' tarli, e simili. \textit{Per tutto
  è buona stanza, ov'altri gode}, \textit{Ubi bonum, ibi patria}. Dove si sta bene, quello è
  buon paese; \textit{E per ogni paese, è buona stanza}, Disse come in proverbio il Petrarca.
\item[CATAPECCHIE] Intendiamo luoghi orridi, inculti, e disabitati. Mattio
  Franzesi in lode delle gotte: \textit{Hor per uscir di queste carapecchie}. Nello stesso
  modo che \textit{pecchia} è fatto da \textit{apes}, \textit{apecala}, o \textit{apicula} così verisimilmente \textit{catapecchia}
  può dedursi da \textit{apex apiculus}, che vuol dire piccola sommità e \textit{cata} preposizione
  Greca, la quale dice un certo ordine, o è aggiunta per maggior forza, come si
  vede nelle parole, \textit{Catafalco}, \textit{Cataletto}, \textit{Catuno}, che dissero gli antichi per
  ciascheduno, e simili.
\item[CHIARITO] Aggiustato Vedi sopra \cstan[1]{1}. Vuol dir che Amore l'haveva
  accomodato, perché s'era pieno di mal di chiasso, come si disse sopra C. 3.
  stan. 11.
\item[LOVA] Lorda; Poltrona. È parola d'ingiuria a una donna. È voce straniera,
  e vuol dir Lupa; che similmente gli Spagnuoli dicono \textit{loba}\footnote{Nello spagnolo moderno le due lettere ``B'' e ``V'' suonano uguale, per esempio in \textit{Bello} (bello) e \textit{Vello} (peluria).}; e s'intende
  meretrice. Gio. Vill. \libcap[1]{25}. parlando di Romulo, e Remo allevati da
  una Lupa dice: \textit{Questa Laurenza era bella, e di suo corpo guadagnava come meretrice,
    e però dai vicini era chiamata Lupa; onde si dice furono nutricati da lupa};  il che
  cavò egli da Livio lib. 1. \textit{sunt qui Laurentiam vulgato corpore lupam vocatam inter
    pastores putent: inde locum fabulae, \& miraculo datum}.
\item[SVALIGIARE] Cavar della valigia. Qui intende; gli ha fatto consumare i
  denari, perché \textit{busecchie} se bene si dicono i ventricini del porco Bocc. gior. 6.
  Nov. 10. \textit{Dove le femmine vanno in zoccoli su pe i monti rivestendo i porci delle lor
    busecchie medesime} noi le pigliamo per tasche, o borse, nelle quali si tengono i
  danari. E svaligiare propriamente intendiamo, quando i ladri di strada rubano a
  uno tutto quello, che egli ha addosso; e lo pigliamo per sinonimo di \textit{saccheggiare}.
\item[PARECCHIE] Numero indeterminato che esprime, Molti; dal Lat. \textit{plurique},
  secondo alcuni: Volgarizzamento di Palladio manoscritto; Nel mese di
  Marzo al cap. de ficu. \textit{Si metta sotto alle barbe parecchie pietre}.
\item[CERBONECA] Vino fradicio. L'Accademico Fiorentino incerto, così nominato
  in una Raccolta di Rime piacevoli, che dicemmo altrove essere il Burchiello,
  descrivendo un cattivo vino dice.
\begin{verse}
\backspace Staccio non passerebbe ne stamigna
Tant'è morchiato, e con la feccia misto;
Sciroppo mi par ber, ma non di vigna;
\makebox[5em]{} Chi ne beve non ghigna,
Ch'egli: è ciprigno, e cerboneca fina;
Chiudendo gli occhi, mi par medicina.
\end{verse}

Brunetto Latini nel suo Pataffio disse \textit{Cerbonea}.
\begin{verse}
  Nel ver quest'è pur nuova Cerbonea
\end{verse}
Forse si dovrebbe dir \textit{cerconeca}, derivando questa voce da \textit{cercone} che vuol dir
vino fradicio e si dice \textit{cercone} dal circolare, che fa il vino quando da la volta e
si guasta.

\item[MESSER sine] Vuol dir Messer sì, Ma dice Messer sine, perché fa parlare a
  un contadino: \textit{nostri sic rure loquuntur}.
\item[GLI ride l'occhiolino] Vuol dir si rallegra. Il rider dell'occhio forse accennò
  Ovidio in quel verso;
  \begin{verse}
    Risit, \& argutis, quiddam promisit ocellis.
  \end{verse}
\item[FAR agresto] Avanzare; ma intende d'avanzo illecito, come sarebbe, quando
  uno mandato a comprare roba, dice havere speso più di quello, che ha speso,
  per rubar quell'avanzo. Vien da i cantadini, che per rubare al padrone pigliano
  l'uva non matura, (che si chiama agresto) e ne fanno sugo, e lo vendono.
  Questo  termine ha lo stesso significato anche in Napoli, come si cava da lo Cunto
  de li Cunti di Gianalesio Abbattutis gior. 1. Cunto 8. dove dice; \textit{Mostrannole
    le frisole, co' li quale maritattero tutte l'autre figlie, restannole pure agresta pe' gliottere
    co gusto le travaglie de la vita}.
\item[NON v'è da far calìa] Non v'è da far avanzi. Calìa si dicono quei rimasugli
  d'oro e d'argento, che nel lavorarlo cadono, e si dicono \textit{calìa} quasi \textit{calo}
  dell'oro o dell'argento, che ridotto poi in proverbio esprime ogni sorta di piccolo
  avanzo.
\end{description}
\section{Stanza VIII. --- X.}
\begin{ottave}
\flagverse{8}All'oste se ne va per la più corta,\\
E l'uova, il pane, e il cacio, e il vin procaccia\\
E fatto un guazzabuglio nella sporta, \\
Le quattro lire slazzera, e si spaccia. \\
L'altro l'aspetta a gloria, e insù la porta \\
Per veder s'egli arriva ognor s'affaccia, \\
E per anticipare, il fuoco accende,\\
Lava i bicchieri, e fa l'altre faccende.
\end{ottave}

\begin{ottave}
\flagverse{9}Perch'egli è tardi, ed ha voglia di cena,\\
Poi c'ogni cosa ha bell' e preparato,\\
Si strugge, e si consuma per la pena,\\
Che lì non torna il messo, ne il mandato;\\
Ma quand'ei vedde con la sporta piena\\
Giunger al fine il suo gatto frugato:\\
O ringraziato, dice, sia Minosse,\\
Ch'una volta le furon buone mosse.
\end{ottave}

\begin{ottave}
\flagverse{10}Chiappa le robe, e mentre ch'ei balocca \\
In quocer l'uova, e il cacio ch'è stupendo; \\
Sente venirsi l'acquolina in bocca, \\
E far la gola come un sali scendo.\\
Sbocconcellando intanto, il fiasco sbocca,\\
E con due man alzatolo bevendo,\\
Dice al villan, che nominato è Meo:\\
Horsù ti fo briccone, addio, io beo.
\end{ottave}


Il Contadino mandato da Paride a provveder la roba, andò all'Oste per sbrigarsi,
e comprò il tutto. Paride in tanto stava aspettandolo con grande ansietà,
e subito giunto, egli messe a quocer l'uova, e il cacio, e in tanto vinto dall'impazienza,
e dalla fame cominciò a mangiar del pane, ed a bere.
\begin{description}

\item[PER la più corta] Vuol dir per la strada più corta; ma qui intendi per
  sbrigarsi più presto.
\item[PROCACCIA] Provvede. Vuol propriamente dire cercar di trovare una cosa,
  e trovarla; Lat. \textit{persequi \& affequi}, esprimendosi con questo solo verbo
  procacciare la diligenza, che s'usa in cercare, e andare a caccia d'una cosa, e la
  fortuna, che s' ha di trovare quel che si cerca; onde poi molti dicono: buon procaccino
  uno che s'ingegna per ogni maniera di guadagnare.
\item[GUAZZABVGLIO] Mescolanza, mescuglio. Il Casa nel suo Capitolo del
  Martello di amore dice;
  \begin{verse}
    \backspace Non era donna ricca,o poverina,
    Si facea d'ogni cosa un guazzabuglio
    Ogni stanza era camera, e cucina.
  \end{verse}
  Mattio Franzesi nel suo viaggio di Venezia dice:
  \begin{verse}
    \backspace Far a una tavolata allegra cera,
    E di varj discorsi un guazzabuglio.
  \end{verse}
  Il Lasca Nov. 10, \textit{Versarono aceto, vino', olio, sale, e farina, e fecero un
    guazzabuglio il maggior del mondo}. Dal che si cava, che questa voce esprime mescolanza.
  di cose materiali, ed anche di non materiali; Voce composta di \textit{Guazzare}, che
  è dibattere cosa liquida, e di \textit{Bollire}: quali da una Ricetta che dica, \textit{Guazza, e
    Bolli}; fattone \textit{Guazzabuglio}.
\item[LIRA] E' una moneta Fiorentina, che vale un giulio e mezzo\footnote{Mentre la Lira Italiana discende direttamente dalla Lira Sabauda,  equivalente al Franco Francese, la Lira Toscana non fu creata in relazione a valute straniere. Una Lira Toscana conteneva 3.66g d'argento fino, mentre il Franco Francese ne conteneva 4.5g, per cui una lira toscana valeva all'incirca 81 centesimi di Franco. La Lira Toscana fu sostituita dal Fiorino in argento di 100 quattrini nel 1826, al cambio di 60 quattrini per lira.}, detto anche
  \textit{Cosimo}, perché il nostro G, Duca Cosimo l'inventò, e fu il primo, che battesse
  in Firenze questa moneta.
\item[SLAZZERA] Cava, conta, mette fuora, fa venir fuora a forza. È parola
  furbesca, se bene assai usata.
\item[SI spaccia] Si sbriga: Si spedisce.
\item[L'ASPETTA a gloria] L'aspetta con gran desiderio, con pazienza estrema.
  Si dice anche \textit{aspettare a bocca aperta}. \textit{Larus hians}.
\item[HA bell'e preparato] Ha di già mess'all'ordine. Vedi sopra C. 3. st. 14.
\item[NON torna ne il Messo, ne il Mandato] Non torna lui, e non manda alcuno a
  dir quel che sia di lui. Diciamo anche \textit{Il ho mandato il corvo}, dal corvo, che mandò
  Noe fuori dell' arca, il quale non tornò mai.
\item[GATTO frugato] Così son chiamati per ischerzo da i ragazzi i contadini.
  \textit{Catus} in Latino è cauto, astuto; e con queste nome chiamafi anche il \textit{Gatto} animale
  noto, il quale quando è stato frugato con pertiche, o con bastoni, non fa
  altro, che volgersi spaurito, e che \textit{guatare}; onde vogliono alcuni, che abbia il
  nome. Così il contadino, quando scende alla Città, Dante Purg. 26.
  \begin{verse}
    \backspace Non altrimenti stupido si turba
    Lo montanaro, e rimirando ammuta,
    Quando rozzo, e salvatico s'inurba.
  \end{verse}
\item[VNA volta furon buone mosse] Vna volta ei tornò, Questo detto usatissimo in
  questo significato, vien da coloro, che stando a veder correre al palio per lo gran
  desiderio, che hanno di vedere arrivare i cavalli, spesso gridano; \textit{eccogli} se ben
  veramente non sono; ma pure al fine vengono, ed allora dicono, \textit{Queste son
    buone mosse}. Il che passato in proverbio; significa a terminazione di qualsivoglia
  evento, o negozio.
\item[SI balocca] Si trattiene. Si dice anche: star' a bada, o badaluccare; È voce
  usata per i bambini. Vedi sopra C. 6. st. 32.
\item[STVPENDO] Buonissimo. Vedi sopra C. 6. st. 35. Cosa maravigliosa, e sì
  perfetta, che induce stupore.
\item[VENIR l'acquolina in bocca] Si sente consumar dall'appetito, e per questo gli
  soprabbonda la saliva in bocca, la qual saliva è causa che \textit{la gola gli fa come un
    saliscendo}, perché il gorgozzule gli va ingiù, e insù per inghiottir quell' umido.
  E saliscendo è una striscia di ferro, che s'adatta a serrar le porte, facendoli fare
  l'operazione con'alzarla, ed abbassarla. In questo significato diciamo ancora:
  La gola gli fa lappe lappe, vedi sopra \cstan[5]{62}.
\item[SBOCCONCELLANDO] Diciamo sbocconceilare, quand'uno, mentre aspetta
  che vengano i compagni a mensa, o che sia portata la roba in tavola, piglia de'
  pezzetti di pane, e mangia.
\item[SBOCCA il fiasco] Stura il fiasco, e fquotendolo butta fuora il vino, che è nel-
  per arlo dall' immondizie, o fiore, che vi pos' essete.
  hot Bartolomeo » Ela figura Apherefis spesso usata da noi ne i nomi
  me Cecco per Francesco fatto da Cesco ( che trovafi nel Decamerone
  cioè Francesca ) Menico per Domenico; cos! Lippo, Stagio, Coppo,
  , Noferi, accorciarono i nostri antichi da Filippo, Anastagio, Iacopo,
  01, Onofrio; ed altri infiniti.
\item[TI so briccone] Ti fo brindisi. Questo e quel modo di parlare, che dicono Za.
  come accennammo sopra C.~1. st.~28, al termine uscir del seminato;
  qui dice Briccone per brindisi.
\end{description}
\section{ STANZA XI.}

\begin{ottave}
\flagverse{11}Così per celia cominciando a bere,\\
Dagliene un sorso, e dagliene il secondo;\\
Fe sì, che dal vedere, e non vedere,\\
Ei diede al vino totalmente fondo;\\
A tavola di poi messo a sedere,\\
Lasciato il fiasco voto sopra il tondo,\\
Voltossi a dieci pan da Meo provvisti,\\
E in un momento fece repulisti.
\end{ottave}

\begin{ottave}
\flagverse{12}Dieci pan d'otto, e un giulio di formaggio\\
Non gli toccaron l'ugola, e s'inghiotte\\
Due par di serque d'uova, e da vantaggio,\\
Poi dice; Meo spilla quella botte,\\
Che t'hai per l'opre, e dammi il vino a saggio,\\
Io vuò stasera anch'io far le mie lotte,\\
Ben ch'io stia bene, sia ripieno, e sventri,\\
Perché mi par, ch'una lattata c'entri.
\end{ottave}

\begin{ottave}
\flagverse{13}Il Rustico che dar del suo non usa:\\
Non saper, dice, dove sia il fucchiella,\\
Che per casa non v'è stoppa ne fusa,\\
E che quel non è vin, ma acquerello.\\
Ci vuol, risponde Paride, altra scusa,\\
E, rittosi, di canna fa un cannello,\\
E, in su la botte posto a capo chino,\\
Con esso, pel cocchiume succia il vino.
\end{ottave}

\begin{ottave}
\flagverse{14}E perché è buono, e non di quello, il quale\\
È nato in su la schiena de' ranocchi,\\
A Meo, che più tosto a Carnovale,\\
Che per l'opre lo serba, esce degli occhi,\\
E bada a dire; Ovvia, vi farà male,\\
Ma quegli che non vuol ch'ei l'infinocchi,\\
Ed è la parte sua furbo, e cattivo,\\
Gli risponde: Oh tu sei caritativo.
\end{ottave}

\begin{ottave}
\flagverse{15}Non so, se tu minchioni la Mattea;\\
Lasciami ber ch'io ho la bocca asciutta,\\
Che diavol pensi tu poi, ch'io ne bea?\\
Io poppo poppo, ma il cannel non butta,\\
Risponde Meo: Po far la nostra Dea,\\
Che sei buttasse, la beresti tutta,\\
O! discrezione s'e' cen'è minuzzolo;\\
Paride beve, e poi gli dà lo spruzolo.
\end{ottave}

\begin{ottave}
\flagverse{16}Non vi so dir se Meo allor tarocca:\\
Ma l'altro, che del vin fu sempre ghiotto,\\
Di nuovo appicca al suo cannel la bocca,\\
E lascia brontolare, e tira sotto.\\
Ma tanto esclama, prega, dagli, e tocca,\
Ch'ei lascia al fin di ber già mezzo cotto,\\
Dicendo ch'ei non vuol ch'il vin lo quoca,\\
Ma che chi lo trovò non era un'oca.
\end{ottave}

Paride in burla in burla bevendo, votò il fiasco, e poi si mangiò dieci pani,
l'uova, il cacio provveduto da Meo, il quale egli pregò, che gli desse a saggio
il vino della sua botte, e Meo adduce diverse scuse per non glielo dare; onde
Paride fatto un bocciuolo di canna si messe a fucciare il vino per il buco del cocchiume
Meo, a cui duole il vedersi consumare il suo, cerca di levar Paride da
bere; ma egli seguita, e per farlo più arrabbiare gli sbruffa il vino nel viso, e
torna a bere. Al fine già sazio, lasciò star di bere, dicendo che il vino era una
buona cosa, e che l'Inventore fu un gran valent'huomo; ma che non voleva
ber più, per non s'imbriacare.
\begin{description}

\item[PER celia] Voce usatissima in Firenze, per denotare \textit{buria}, \textit{scherzo}. Viene da
  una giovane Commediante, la quale era di genio scherzoso, e burlesco, e faceva
  la parte della serva; e si domandava \textit{Celia}.
  \begin{verse}
    \verseprefix{Il Persiani}Il tuo canto è più dolce d'una avelia;
    Ma scusami, se teco io fo la Celia.
  \end{verse}
\item[DAGLIENE un sorso \&c.] Cioè bevi un poco, e poi un'altro poco. Sorso è quella
  quantità di vino, o d'altro liquore, che si può bere senza ripigliar fiato, dal
  Latino \textit{sorbere}.
\item[FA sì che dal vedere, e non vedere] La cosa andò in maniera, che in un
  momento; in un batter d'occhio. \textit{In ictu oculi}.
\item[DIEDE fondo al vino] Cioè votò il fiasco. Finì il vino. Dar fondo a una
  cosa vuol dir consumare affatto. Termine marinaresco; e si dice \textit{dar fondo}, quando
  la nave si ferma in porto, finito il viaggio.
\item[TONDO] Così chiamiamo quel piatto spianato di stagno, o d'altra materia,
  sopra il quale in tavola si posano i bicchieri.
\item[FECE repulisti] Finì; ripulì, consumò ogni cosa, ne volle veder la fine.
  Termine basso, e usato dalla plebe.
\item[NON gli toccaron l'ugola] Non gli scemarono l'appetito. Quando a un grande
  affamato si dà poco cibo, diciamo: \textit{Non gli ha toccato l'ugola}, e ancora: \textit{Non
    gli ha toccato un dente}, e proverbialmente: \textit{È stata una fava in bocca all'orso}. \textit{Labra,
    non palatum rigat}. \textit{Ugola} si dice quella particella carnosa, che pende fra le fauci
  per uso di formar convenientemente la voce. Latino \textit{vua}, Columella.
\item[SERQUA] Numero di dodici, ma si dice d'uova, di pere, e simili, che per
  altro si dice dozzina.
\item[SPILLA la botte] Buca la botte. \textit{Spillare} si dice da spillo, che è quel ferro acuto,
  col quale si bucano le botti, e questo forse dal Latino \textit{spiculum}, o pure da
  \textit{spinula}, Crescenzio lib.~4.\ c.~41.\ chiama \textit{spina foecaria}, e 'l suo antico Volgarizzatore,
  \textit{spina fecciaia}, la cannella posta nel fondo de' vasi da vino, per farne
  uscire la feccia.
\item[OPERE] Coloro che aiutano lavorare a i contadini, ricevendo il prezzo delle
  loro fatiche giorno per giorno si dicono \textit{opere}, o \textit{opre}. In Latino similmente
  \textit{operae} si dicono i lavoranti.
\item[VUÒ far le mie lotte] Voglio far le mie forze. Voglio pigliarmi tutte lo
  soddisfazioni possibili. Diciamo; \textit{il tale vuol troppe lotte, troppe invenie, troppi sfoggi,
    troppe cirimonie}: quand'uno in far' un'operazione la vuol far con ogni requisito,
  ancor che superfluo, e non necessario.
\item[SVENTRI] Scoppi per lo troppo mangiare, e bere.
\item[UNA lattata c'entri] Ci stia bene una lattata. Diciamo: \textit{fare una lattata} quando
  dopo che s'è mangiato, e bevuto bene, si fa venir in tavola nuovo vino, e
  nuovi bicchieri puliti. Che per altro \textit{lattata} è una bevanda fatta con zucchero,
  orzo, e semi di popone, che benissimo pesti, e liquefatti con acqua gli fanno
  passare per stamigna\footnote{La stamigna, o stamina, è un tessuto ad armatura tela, con riduzione larga cioè con fili radi, di mano molle e medio peso.}, la quale si da per lo più a' febbricitanti per rinfrescare: ed
  io credo, che i gran bevitori habbiano dato il nome di \textit{lattata} al suddetto nuovo
  bere superfluo, come che vogliano intendere, che questo secondo bere non sia
  spropositato, ne per gola, ma per rinfrescare l'ardore del vino bevuto, come fa
  alla febbre la \textit{lattata}, la quale diciamo più comunemente \textit{orzata}.
\item[SUCCHIELLO] Diminutivo di \textit{succhio}, che vale lo stesso. Strumento d'acciaio
  per uso di bucar legnami: e il Latino \textit{Terebra}.
\item[NON ha stoppa, ne fusa] Il villano per non dar bere, trova scusa di non poter
  metter la cannella alla botte, perché non ha stoppa da avvoltare in sulla cannella
  per adattarla al buco della botte, ne meno può bucarla, perché non ha fusa
  da turare il buco dello spillo, delli quali fusi (che per altro servono alle donne
  per adunarvi sopra il filo, quando filano a rocca) ci serviamo per turare simili
  buchi, perché per esser ben tondi, e di figura piramidale, serran bene ogni buco.
  Aggiugne di più per scusa, \textit{che quello non è vino, ma acquerello}, che è la lavatura
  delle vinacce, e serve per bevanda de i contadini, da molti detto vinello, e da
  altri \textit{mezzingo}, e da i Latini \textit{Lorea}, o \textit{Lora}. Ma Paride, che molto ben conosce,
  che queste sono tutte invenzioni, gli dice: \textit{Ci vuol altra scusa}, ed intende; Non
  m'asterrò per questo di far quel che io ho in animo, cioè di bere.
\item[COCCHIUME] Quel turacciolo di legno, col quale si tura la buca di sopra
  della botte; e si chiama così anche la stessa buca. I Latini lo dicono \textit{dolij operculum}.
\item[SUCCIARE] Attrarre a se l'umido, o sugo. Dal Latino \textit{sugere}.
\item[NATO in su le schiene de' ranocchj] \- Nato ne i pantani, dove stanno i ranocchi,
  che non è vin buono.
\item[ESCE degli occhi] Non può vederlo consumare: lo da mal volentieri, Gli
  duole il veder consumar quel vino, quanto gli dorrebbe il perdere il lume degli
  Occhi. Detto assai usato in simile proposito.
\item[NON vuol che l'infinocchi] Non vuol che con le chiacchiere lo ritenga dal bere
  \textit{Infinocchiare} è lo stesso, che \textit{dar panzane}, \textit{bubbole}, o \textit{chiacchiere} ed è il Latino
  \textit{Verba dare}. Il Lalli En.\ Tr.\ C.~4.\ st.~107.\ dice.
  \begin{verse}
    Per ch'il parlar di lei non l'infinocchi.
  \end{verse}
\item[OH TU sei caritativo] Tu hai la gran pietà di me. E' detto scherzoso, usato in
  simili congiunture, e si dice: \textit{Tu hai carità pelosa}, o \textit{la carità di mona Candida}, che
  biascicava\footnote{Biascicare, oltre a pronunciar male, vuol dire rigirarsi il cibo in bocca con molta saliva senza masticarlo.} i confetti agli ammalati per levar loro la fatica.
\item[NON so se tu minchioni la mattea] Non so se tu burli. Vedi sopra C. 4. st. 15.
\item[Che pensi tu mai ch'io ne bea?] Quanto pensi tu, ch'io al fine ne beva. Altrove
  habbimo detto di questa particella \textit{mai}, che altre volte afferma, altre volte
  nega, ed altre volte significa tempo, come qui, che vuol dire, \textit{quanto pensi tu,
    che io in ultimo ne beva}. In Latino direbbesi, \textit{Quid demum censes?}
\item[IO poppo poppo] Cioè io attendo a succiare, ma io tiro fu poco vino, perché il
  cannello ne dà poco.
\item[PUÒ far la nostra Dea] Esclamazione, o giuramento di contadini; quasi volendo
  significare la \textit{Dea Pales}. Virg. 3. Georg. \textit{Te quoque magna Pales \&c.}
\item[SE e' cen'è minuzzolo] Se cen'è punto. Se ei cen'è pur un poco, Ser Brunetto
  Latini nel Patassio. \textit{Io non ho fior, ne punto ne calìa, Minuzzol, ne scamuzzol}.
\item[GLI dò lo spruzzolo] Gli sputa il vino nel vilo a minute stille. \textit{Spruzzolare}
  diciamo quando comincia a piovere minutamente, onde \textit{Spruzzaglia} osservò il
  Vettori dirsi da' contadini una piccola quantità di pomi per similitudine.
\item[TAROCCA] Entra in collora; arrabbia. Voce usata in Firenze, e anche in
  Lombardia. Francesco Negri nel suo Tasso in lingua Bolognese, portando in
  quello il verso d'un'argumento, che dice \textit{Il Re si turba alla novella rea}, parafrasa
  \textit{Re al sente, e c'minza a taruccar}.
\item[BRONTOLARE] È un rammaricarsi, o dolersi di qualche sopruso, o sinistro
  avvenimento con parole non affatto espresse, ma confuse, e male articolate, e
  fra i denti, che si dice anche \textit{bofonchiare}; (Nella Valdinievole bofonchio è
  detto il calabrone) Viene per avventura dal Greco \textit{Brontan}, che vuol dir tonare.
  Virg. in quel verso, ove nomina i Ciclopi affaccendati a lavorare il ferro, e fulmini
  nella fucina di Vulcano. \textit{Brontesque, Steropesque \& nudus membra Pyracmon}.
  Il primo nome lo cava dal tuono, il secondo dal folgore, il terzo dall'ancudine,
  e dal fuoco.
\item[TIRA sotto] Attende, continova, seguita a fare quella tal cosa.
\item[DAGLI, e tocca] Questo termine significa, fa, e rifa la tal cosa, ovvero prega,
  e riprega; e si dice \textit{Dagli, picchia, e tocca}. Ovvero \textit{Dagli, tocca, picchia, e
    martella}.
\item[MEZZO cotto] Quasi briaco. Vedi sopra C. 6. st. 35.
\item[CHE lo trovò non era un'oca] Chi lo trovò non era huomo senza cervello, ma
  un valent'huomo. Cervel d'oca, o capo d'oca vuol dir huomo di poco giudizio.
\end{description}
\section{Stanza XVII. --- XXI.}
\begin{ottave}
\flagverse{17}Poiché dal cibo, e da quel vin che smaglia,\\
Si sente tutto quanto ingazzullito, \\
Risolve ritornar alla battaglia, \\
Donde innocentemente s'è partito, \\
Che scusa non gli pare haver, che vaglia,\\
Che non gli sia a viltade attribuito;\\
Così ribeve un colpettino, e in cambio\\
D'andar a letto s'arma, e piglia l'ambio.
\end{ottave}

\begin{ottave}
\flagverse{18}Senza lume, ne luce via spulezza,\\
E corre al buio, che ne anche il vento,\\
Non ha paura mica della brezza,\\
Perch'egli ha in corpo chi lavora drento;\\
Per la mora sì ben si scandoleza,\\
Che dando il \culo{} in terra a ogni momento,\\
Quanto più casca, e nella memma pesca,\\
Tanto più sente ch'ell'è molle, e fresca.
\end{ottave}

\begin{ottave}
\flagverse{19}Dopo ch'ei fu cascato, e ricascato,\\
Per non sentir quel molle e fresco ancora,\\
Che'l vino, e quanto dianzi avea ingubbiato,\\
Opra di dentro sì, ma non di fuora,\\
Giunto al mulin dal mezzin giù sbracciate\\
Si sciaguatta i calzoni in quella gora,\\
Per dopo nella casa di quel loco\\
Farsegli tutti rasciugar al foco.
\end{ottave}

\begin{ottave}
\flagverse{20}Mentre si china dando il \culo{} a leva;\\
Ei fece un capitombolo nell'acqua\\
Ond'avvien ch'una volta ei l'acqua beva\\
Sopra del vin, che mai per altro annacqua;\\
Quanto di buon si è; che s'ei voleva\\
Lavar i panni, il corpo anche risciacqua,\\
E divien l'acqua si fetente, e gialla,\\
Ch'i pesci vengon tutti quanti a galla.
\end{ottave}

\begin{ottave}
\flagverse{21}Le regole ben tutte a lui son note, \\
Ch'insegnò per nuotar bene il Romano;\\
Distende il corpo, gonfie fa le gote, \\
Molt'annaspa col piede, e con la mano.\\
Intanto si conduce fra le ruote,\\
Che fan girando macinare il grano,\\
Ben sen' avvede, e già mette a entrata\\
Di macinarsi, e fare una stiacciata.
\end{ottave}


Paride sentendosi invigorito risolvette di ritornare al campo; e così senz'altro
si messe in viaggio, ma sendosi infangato, volle lavare i calzoni in una gora,
e vi cascò dentro, e se bene egli sapeva nuotare, e s'affaticava per uscir dell'acqua,
tuttavia conobbe, che portava pericolo d'entrar sotto le ruote del mulino,
e restarvi infranto, se non gli accadeva quello, che sentiremo appresso.
\begin{description}

\item[VINO che smaglia] Vino potente, e generolo. Si dice \textit{smagliare}, perché il vino
  nel mescersi nel bicchiere lascia nella superficie una stummia, che fa certe cose
  come maglie, le quali il vino generoso rode, e consuma subito; e questo disfar
  quelle maglie si dice smagliare, e quando non le disfà, è segno, che ha poco spirito.
  E di qui i ciechi hanno un detto: \textit{Baloccom'io, o vommene?} ed intendono
  così di domandar al compagno alluminato, il quale ha mesciuto nel bicchiere, se
  quella stummia se ne va, o si trattiene, ed in conseguenza s'il vino e buono, o
  cattivo. Lasca Nov. 4. \textit{fecero uno scotto regio con quel vino, che smagliava}.

\item[INGAZZULLITO] Forse meglio ingazzurlito. Vuol dir rinvigorito, ringagliardito,
  o rallegrato di quella allegrezza, che mette addosso il buon vino.
  Si dice \textit{entrar in zurlo}, o \textit{in zurro}, corrottamente da \textit{ruzzo}, e questo dal Latino
  \textit{ruere}.

\item[INNOCENTEMENTE s'è partito] Dice innocentemente, perché in vero Paride
  non haveva errato a partirsi dal campo, poiché n'era stato cavato da coloro,
  che lo portavano via infermo, come s'è detto sopra C. 3. st. 25.

\item[UN colpettino] Un'altra volta, Un'altro poco. I Franzesi similmente dicono
  per esempio; \textit{boire encore un coup}. Bere un'altra volta. Provarsi a bere un'altro
  poco. Ed è traslato dal provarsi in giostra.

\item[PIGLIAR l'ambio] Andarsene. Voce corrotta da \textit{ambulo} latino, che vuol dir
  andare, o pur vien da \textit{ambio} specie d'andatura di cavallo, con altro nome detto
  \textit{portante}, perché per esprimere andarsene diciamo \textit{Pigliare il portante}.

\item[SENZA lume, ne luce] Affatto al buio. Senza lume terreno, e senza splendor celeste.

\item[SPULEZZA] Va via furiosamente. Parmi che possa venire da spulare il grano,
  che il vento furiosamente porta via la pula, cioè i gusci del grano; o da
  \textit{pigliare il puleggio} detto sopra C. 1. st. 80.

\item[MOTA] Terra inzuppata nell'acqua, e ridotta quasi liquida. Così appresso
  i Franzesi \textit{moite} è il Latino \textit{udus}, \textit{madidus}, e quel che noi diremmo molle.

\item[MEMMA] o \textit{melma}. Quella terra, che nel fondo de' fiumi, fossi, laghi, e
  paludi, ridotta liquida, che la diciamo anche \textit{belletta} per \textit{melmetta} Latino Limus
  Verisimilmente dal Greco \textit{Migma}, che vuol dire \textit{mistura}.

\item[INGUBBIATO] Messo in corpo. Detto plebeo. Vedi sopra la voce Gubbiano C. 1, st. 36.

\item[DA mezzo in giù sbracciato] Così dice per scherzo, sapendo bene che sbracciato
  significa, quand'uno tirando la manica in su fino al gomito, lascia ignuda quella
  parte del braccio, e non quand'uno si cava i calzoni', come dice, che havea fatto
  Paride, il che si dice \textit{sbracato}; ma-l'Autore si serve della voce \textit{sbracciato} per
  intendere spogliato; e non è vero che habbia a dire \textit{sbracato}, come alcuni hanno
  corretto, non solo perché l'originale di mano dell'Autore, che è appresso di me,
  ed in un suo primo sbozzo dice \textit{sbracciato}, ma anche perché sed dicesse \textit{sbracato da
    mezzo in giù}  s'intenderebbe che ei si fusse tirato su i calzoni fino a mezza coscia,
  e non che se gli fusse affatto cavati, come era necessario, che egli facesse, se e'
  voleva lavargli.

\item[SCIAGUATTARE] Dimenare un panno, o altro simile nell'acqua.

\item[GORA] Vuol dire un canale d'acqua, che corre, e propriamente s'intende
  quella fossa, per la quale si conduce l'acqua a i mulini per macinare, e queste
  tali fosse, o gore si fanno a quei mulini, che sono in su' rivi, o piccoli fiumi, ne'
  quali è scarsità d'acqua, non essendo necessarie a i fiumi reali, ne i quali per esservi
  abbondanza d'acqua, basta un sostegno, o steccaia (che noi diciamo pescaia)
  che volti l'acqua al mulino, e serva per \textit{Colta}, che è una larga fossa, entro
  alla quale si raguna tutta l'acqua, che porta la gora. Gli antichi finivano molte
  voci in \textit{Ora} non solamente quelle, che aveano similitudine col Lat. come le \textit{latora}
  le quattro \textit{tempora}, come ancor oggi diciamo; mia anche le \textit{Bergora}, l'\textit{Ancora},
  le \textit{Campora}; E simili. Onde il Sannazzaro nelle Ecloghe della sua Arcadia prese
  licenza di dire \textit{Pratora} per \textit{Prati \&c}. Si poté dunque dare benissimo il caso, che
  quest'acque così ragunate essi chiamassero \textit{Lacora} dal Lat. \textit{lacus}, e poi si venisse
  a staccare la voce, e dirsi La \textit{gora}. Da i latini si trova esser tali, o simili ridotti
  d'acqua chiamati \textit{Euripi}, e \textit{Nili}, ma credo che fussero iperboliche adulazioni,
  come si può dedurre da Cic. 2. de legibus, dove dice; \textit{Ductus aquarum, quos isti
  Nilos, Euriposque vocant quis non irriserit?} E veramente è cosa da ridere, perché
  Euripus è uno stretto di Mare, ove è il flusso, e reflusso; Ed il Nilo è de' maggiori
  fiumi del Mondo; E queste son fosse semplici, e laghetti, che gli antichi Romani
  fecero correre infino di vino in occasione di feste; e da ciò piglio argumento,
  che gli adulatori per piacere a' Signori, le chiamassero \textit{Nili}, ed \textit{Euripi}.

\item[DANDO il \culo{} a leva] Cioè alzando il \culo{} ed abbassando il capo.

\item[FECE un capitombolo] Rivoltò il corpo sul capo sottosopra; fece un tomo col
  capo, rivoltandosi sottosopra. Vedi \cstan[6]{84}.

\item[A GALLA] Nella superficie dell'acqua. Dai verbo \textit{galleggiare}, che piglia
  origine da galle, che sono quelle leggierissime palle, che nascono dalle
  donde \textit{leggieri, com'una galla}.

\item[IL Romano] Fu uno Stufaiolo, che insegnava nuotare alla gioventh Fiorentina
\item[MOLTO annaspa] Annaspare vuol dir mettere il filato sopr'all'aspo per ridurre
  il filo in matasse, e dipanare, Lat, \textit{glomerare}, affine d'adattarlo a tessere,
  dal Greco \textit{anaspan}, che vale \textit{retrahere}, \textit{revellere}. E da questo quando uno
  perde molto tempo a far qualche operazione, e non conchiude cosa di buono diciamo
  annaspare. Qui vuol dire, che egli muoveva i piedi, e le mani, come muove
  le mani colui che annaspa; e si può anche intendere che armeggiava, ed annaspava
  molto, e conchiudeva poco.
\item[GIA mette a entrata di far una stiacciata] Già tien per certo d'havere a restare
  infranto dalle ruote del mulino. I cassieri, ed ogn'altro che tenga libri d'entrata
  e uscita, mette a entrata, quando ha ricevuto il denaro; e da questo noi
  intendiamo Tien per certo, o ha già per ricevuta quella tal cosa.
\end{description}
\section{Stanza XXII --- XXV.}
\begin{ottave}
\flagverse{22} In questo, che il meschin già si presume \\
D'andar a far la cena alle ranocchie,\\
Aprire vede una porta, e in chiaro lume \\
Sventolar drappi, e campeggiar conocchie, \\
Che le Naiadi Ninfe di quel fiume, \\
Coronate di giunchi, e di pannocchie\\
Corrono ad aiutarlo infin c'a riva \\
(Là dove il dì riluce) in salvo arriva.
\end{ottave}

\begin{ottave}
\flagverse{23}E vede all'ombra di salcigne frasche \\
Fra le più brave musiche acquaiole \\
Parte di loro al suon di bergamasche, \\
Quinte, e seste tagliar le capriole, \\
Chi tien che queste Ninfe fien le lasche\\
Chi le sirene, ed altri le cazzuole;\\
Il non so chi di lor dia più nel buono,\\
E le lascio nel grado, ch'elle sono.
\end{ottave}

\begin{ottave}
\flagverse{24}Ognun si tenga pure il suo parere;\\
O quelle, o altre, a me non fa farina,\\
Bastini per adesso di sapere,\\
Che queste non son bestie da dozzina;\\
E, s'ella non m'è stata data a bere,\\
Elle son Fate c'han virtù divina,\\
E che sia il vero, fede ve ne faccia\\
Li Garani scampato dalla stiaccia.
\end{ottave}

\begin{ottave}
\flagverse{25}Il quale così molle, e sbraculato\\
Il cadavero par di Mona Checca,\\
Ch'essendo stato allor disotterrato,\\
Habbia fatto alla morte una cilecca;\\
Si squote, e trema sì, ch'io ho stoppato\\
Per San Giovanni il carro della Zecca,\\
E mentr'ei si debatte, e il capo scrolla,\\
Il pavimento, e i circostanti ammolla.
\end{ottave}

\begin{ottave}
\flagverse{26}Ma le Fate, che specie son di pesce,\\
Ed hanno il corpo a star nell'acqua avvezzo\\
Più che l'esser bagnate a lor rincresce,\\
Il vederlo così fradicio mezzo;\\
Perciò lo spoglian; ma perché riesce,\\
Quando un vuol far più presto, star un pezzo,\\
Per trattenerlo (mentr'hor questa, hor quella\\
L'asciuga) una contò questa novella.
\end{ottave}

Paride stava con timor d'affogare, fu foccorso da alcune Ninfe, le
quali lo cavarono dell'acqua, e lo condussero alle loro stanze, dove dette Ninfe
si messero a spogliarlo, ed intanto una di loro contò la novella, che vedremo
appresso.
\begin{description}
\item[MESCHINO] \- Infelice; Povero.\ È voce, che denota commiserazione.
\item[ANDAR a far la cena de' ranocchi] Cioè affogare, annegare, e così diventar cibo
  de' ranocchi.
\item[CONOCCHIE] Pennecchi in sulla rocca, che sono quei rinvolti di lino, o
  lana, o altra materia simile, che le donne per filarla accomodano in sulla rocca
  strumento da esse usato per filare; Voce corrotta da cannocchie, secondo il Ferrari,
  perché le rocche per lo più sono di canna; Il Vossio la fa venire dal Lat.
  \textit{colus}; quasi storpiata da \textit{colucula}.
\item[DRAPPI] Cioè quei drappi da donna, che dicemmo sopra C.~6. stan.~9.
\item[CAMPEGGIAR conocchie] Supposto che le mura di quelle stanze fussono bianche,
  ogni cosa di qualsivoglia colore vi si discerne ben sopra, e però (servendosi
  del verbo pittoresco campeggiare) intende; si distinguevano sopr'a quel bianco i
  drappi, che sventolavano, e le rocche appiccate alle muraglie.
\item[GIUNCO] Pianta, o virgulto noto, che nasce vicino all'acque, ed in luoghi
  umidi, e padulosi, e non fa foglie, ne tronchi; ma fusti, come paglia, lisci, (senza
  nodi, se non uno in vetta, dove nasce il seme. E per questo habbiamo un
  proverbio, che dice: \textit{Cercar il nodo in sul giunco}; Lat. \textit{nodum in scirpo quaerere}, che
  significa cercar le difficultà, dove elle non sono.
\item[PANNOCCHIE] Spighe, che si producono dalle canne, dalla saggina, e dal
  panico, \&c, dal Latino \textit{Panicula}, voce usata da Plinio, ove tratta delle canne
  \textit{Coeterum gracilitas nodis distincta levi fastigio tenuatur in cacumina, crassiore
    paniculae coma}.

\item[SALCIGNE frasche] Frondi di salcio albero noto, che nasce; e vien più vigoroso
  in luoghi padulosi, Lat. \textit{frondes salignae}.

\item[AL suon di bergamasche] Chiamiamo Bergamasca un ballo composto tutto
  di salti, e capriole, e però dice \textit{quinte, e seste tagliar le capriole}.

\item[CAZZUOLE] Sono certi animaletti neri, che vivono nell'acqua, e sono tutti
  pancia, e coda, e col tempo diventano ranocchie, e mettendo le gambe, e
  cascando loro la coda, mutano colore di nero in verde macchiato, e \textit{cazzuola} diciamo
  la mestola da muratori; Lac, \textit{trulla}, e che l'abate Baldo da Urbino nel dizionario
  sopra Vitruvio dice al suo paese chiamarsi Cucchiara.

\item[LE lascio nel grado ch'elle sono] Sieno chi elle si vogliono, io non do loro più un
  nome, che un' altro; perché ciò \textit{non fa farina}, cioè non m'importa; e non fa al
  proposito mio. E qui l'Autore mostra d'haver notizia delle diverse opinioni de'
  Gentili circa alle Ninfe, le quali tutti concordano esser Figliuole dell'Oceano,
  conchiudono che le più fussero Deità aquatiche; le quali Deità noi poi interpretiamo,
  che sieno diversi effetti, che produce l'umidita. E che parte di queste
  Ninfe sieno de i prati, parte de' boschi, parte de i monti, e con diversi nomi di
  Nereidi, Napee, Oreadi, ec.

\item[NON son bestie da dozzina] Non son bestie ordinarie, e da farne poca stima.
  Diciamo cosa da dozzina, o dozzinale, quella, che è lontana dalla perfezione, e
  che è lavorata con poca diligenza.

\item[S'ELLA non m'è stata data a bere] S'ella non m'è stata data a credere.

\item[FATE] Vedi sopra \cstan[4]{54}.

\item[STIACCIA] Si dice quella trappola, che si tende con le lastre a i topi, ed agli
  uccelli, così detta, perché nel cadere addosso all'animale, lo stiaccia.
\item[SBRACULATO] Senza brache, e senza calzoni.
\item[CADAVERO di Mona Checca] Si suole in Firenze nel giorno della commemorazione
  di tutti i morti, ne i sotterranei della Basilica di S. Lorenzo, che sono il
  sepoltuario, esporre uno scheletro di morto con veli in testa, ed altri abbigliamenti,
  e questo da i ragazzi è detto \textit{Mona Checca}; cioè \textit{Madonna Francesca}, e
  questo nome poi comunemente s'usa per esprimere uno sbattuto, ed afflitto dalla
  fame, dal freddo, e da altro stento. Aristofane portato in Latino dice: \textit{Nihil
    a Chaerephonte differt}.
\item[FARE una cilecca] o \textit{scilecca} \- Far una burla; cioè finger di voler una
  cosa, e poi non la fare; Sicché vuol dire: habbia finto d' esser morto, e poi non
  sia stato vero. Habbia gabbato la morte. Diciamo anche \textit{pare un morto
    dissotterrato}. Il Bini nel secondo Capitolo dell'orto dice:
  \begin{verse}
    \backspace Ho una vasca, ma ell'ha una pecca
    D'un certo suo turacciol benedetto,
    C'ogni volta mi fa qualche cilecca.
  \end{verse}

\item[IO ho stoppato] Qui ha lo stesso significato, che \textit{ne disgrado} detto sopra C. 1. st.
  51. \cstan[3]{34}. e \cstan[6]{61}. che per altro \textit{havere stoppato uno}, vuol dire \textit{Haver uno
    negli orecchi, ec}. per esempio. Tu mi hai fatto il servizio tanto tardi, che
  io non ho havuto più bisogno, e però \textit{io t'ho stoppato}.
\item[IL carro della Zecca] Il giorno di S. Giovanbatista è la maggior solennità, che
  si in Firenze per esser del Santo Avvocato, e Protettore della Città, ed
  in tal giorno tutti i Magistrati di Firenze, e tutte le Terre, e Castella subordinate
  al Dominio fanno la cirimonia dell'offerta al Tempio dedicate al detto Santo,
  e fra gli altri il Magistrato della Zecca offerisce un gran Carro trionfale in figura
  piramidale alto circa 20. braccia\footnote{20 braccia: quasi 12 metri, essendo il braccio fiorentino 584mm;}, e nella sommità di esso Carro è un'huomo
  tutto coperto di peli, legato con fune a un palo di ferro alto circa un braccio
  e mezzo, che formando in cima un mezzo circolo gli fascia lo stomaco, dove
  è fermato detto huomo, acciò non caschi, il quale rappresenta San Giovanni nel
  Deserto. E perché tal Carro nell'essere strascicato brandisce, e squote, però
  costui, che è nella cima del Carro s'agita grandemente ancor'egli; Ed il Poeta
  di questuo huomo intende dicendo, che Paride si squote più del Carro della Zecca,
  cioè di colui, che è sopra detto Carro.
\item[RINCRESCE] o \textit{incresce} Vuol dir venire a noia, o a fastidio, ed è il Latino
  \textit{Taedet}. Bocc. gior. 5. Nov. 6. \textit{Io farò sì, che la vedrai tanto, che ella ti increscerà}.
  Significa haver dispiacere, c'una cosa sia fatta, o non fatta. Bocc. Nov. detta
  \textit{Ma di ciò, che fatto havea, gl'increbbe}. Significa compassionare uno, come nel presente
  luogo, e sotto in \cstan{50}. Significa ancora haver dispiacere intendendosi
  esser nelle Fate maggiore la compassione, che havevano di Paride per vederlo
  così mal condotto, che non era il disgusto d'esser bagnate: E sono questi
  due significati tanto prossimi, che spesso col solo verbo rincrescere s'esprime
  l'uno e l'altro, come segue qui, e nel Petr. Son. 44.
  \begin{verse}
    Onde il lasciare, e l'aspettar m'incresce,
  \end{verse}
  Che si può intendere mi pesa, mi dispiace il lasciare, e mi viene a noia l'aspettare.
  Il Persiani nella lettera al sig.\ Principe D.Lor. disse:
  \begin{verse}
    Ml mio bisogno ho già detto a parecchi
    E ciascun se ne duole, e gli rincresce.
  \end{verse}
\item[FRADICIO mezo] Con l'\letter{e} stretta, e con una sola \letter{z} che fa aspro (perché
  con l'\letter{e} larga, e con due zete, che fanno dolce, secondo l'opinione del dottissimo
  Carlo Dati, vuol dire metà) significa bagnato assai; e la voce \textit{fradicio}
  che vuol dire corrotto, qui significa inzuppato d'acqua. La voce \textit{mezo} vuol
  dire una cosa tenera per esser troppo matura, come farebbe una mela o pera, ec.
  vedi sopra \cstan[3]{53}. o una cosa intenerita per haver inzuppato molto umido
  come sarebbe una spugna intinta nell'acqua, e questo è il senso del presente
  luogo. \textit{Mezo} è dal Lat. \textit{mitis} per \textit{maturo}; ed è il contrario di acerbo, che così
  chiamiamo la frutta non per anco matura. Volgarizzamento antico di Palladio nel
  mese di Gennaio, tit. 15. Serbansi le sorbe, se si colgano dure, ec. e ivi comincian
  a immezzare. Il Lat. dice: \textit{ubi mitescere coeperint}.
\end{description}

\section{Stanza XXVII. --- XXX.}
\begin{ottave}
\flagverse{27}Furo un tratto una dama, e un Cavaliero\\
Moglie, e Marito in buono, e ricco stato,\\
Che fatti vecchi, contr'ogni pensiero,\\
Dopo a'haver qualche anno litigato\\
La grinza pelle con il cimitero,\\
Convenne loro al fin perdere il piato,\\
E senz'appello haver a far proposito\\
Di dar per sicurtà l'ossa in deposito.
\end{ottave}

\begin{ottave}
\flagverse{28}Lasciaron due Figliuoli i più compliti\\
Che'l mondo havesse mai su le sue scene,\\
Perch'essi havevan tutti i requisiti\\
Dovuti a un galant'huomo, e un hom dabbene;\\
Aggiunto che di soldi eran gremiti,\\
(Che questo in somma è quel che vale, e tiene)\\
Stavan d'accordo, in pace, ed in amore,\\
Et eran pane, e cacio, anima, e core.
\end{ottave}

\begin{ottave}
\flagverse{29}Cosa che fare in hoggi non si suole,\\
Perch'i Fratelli s'han più tosto a noia,\\
E se lor hanno due cenci, o terre al Sole,\\
All'un mill'anni par che l'altro muoia.\\
E questo è il ben c'a i prossimi si vuole,\\
E siam di così perfida cottoia,\\
Che se ben fuser anche al lumicino,\\
E non si sovverrebbon d'un lupino.
\end{ottave}

\begin{ottave}
\flagverse{30}Perché e' son un una man di mozzorecchi,\\
Al contrario costor di chi io favello\\
I quai di cortesia furon due specchi,\\
E trattavan ciascun da buon Fratello.\\
S'havrebbon portat'acqua per orecchi,\\
E si servian di coppa, e di coltello,\\
E per cercar dell'uno il bene stare,\\
L'altro voluto havrebbe indovinare.
\end{ottave}


La Fata principiò a contare la novella (la quale è tolta da lo Cunto de li Cunti
gior. 4. Cunto 9, e gior. 5. Cunto 9.) e dice: Furon già una dama, e un Cavaliero
marito, e moglie, i quali venendo a morte lasciarono due Figliuoli ben
costumati, e ricchi, i quali s'amavano grandemente l'un l'altro. Qui il Poeta
fa una digressione, e considera, che questo modo di trattarsi fra i Fratelli
hoggidì non usa più.
\begin{description}
\item[PIATO] e \textit{piatire} Lite, o litigare d'avanti a' tribunali, detto dal Lat. barbaro
  \textit{placitum} per lite, e \textit{placitare}; la qual voce ritengono bella e intera i
  Veneziani. \textit{Placitum} è il decreto, sentenza del giudice, o Magistrato, e quel che i Franzesi
  dicono \textit{Arresto} secondo il Budeo da \textit{arescein}, che in Greco vuol dire \textit{placere}.
  Ne' Senatusconsulti, ovvero Decreti, e Sentenze del Senato di Roma usavano
  questa formula: \textit{Senatui placere \&c.} come si ricava da Cicerone Filippica 3. e 5.
  Nell'Ordinanze Regie in Francia si legge sempre in fine: \textit{Car tel est nostre plaisir}.
  Percioccé il nostro \textit{piacere} è tale. E nella legge si dice; che \textit{Principum placita
    legis habent vigorem}. Venne poi da' Latini bassi a tirarsi questa parola a significare
  il processo della lite medesima, sì come anche \textit{iudicium} significa \textit{la sentenza, e
    la lite medesima}, che fa nascere la sentenza. \textit{Piatire} lo Spagnuolo dice \textit{pleytear} il
  Franzese \textit{plaider}; tutti dall'istessa fonte Latina. Il Doni nel suo Cancelliere dice:
  \textit{Sempre ne i piati la rovina va innanzi, e chi piatisce ha quant'ei vuole il tempo lungo}.
  Ed il Varchi St. Fior. lib. 14. \textit{Erano assegnate le cause delle povere persone, che non
    potevano piatire per la lor povertà}. E poco appresso, dice: \textit{Perché bisognava notificare
    quel piato al terzo possessore}. Ed in questi ultimi versi della presente Ottava
  27.\ dice metaforicamente, che a costoro già fatti vecchi dopo haver fatta desiderar
  lungo tempo la loro carne a i sepolcri, conuenne morire, e farsi sotterrare.
  Il proverbio \textit{piatire i cimiteri} vuol dire Esser d'età cadente, che Luciano portato
  in Latino dice: \textit{Alterum pedem sepulcro}, o vero \textit{in cymba Charontis habere}; che
  noi pure diciamo; \textit{Havere il pié su la bara} o vero \textit{il pié nella fossa}.
\item[GALANT'huomo, ed huomo dabbene] Si posson dir sinonimi; ma strettamente
  \textit{galant'huomo} vuol dire huomo di garbo, e come dicono i Franzesi, \textit{onest uomo},
  e oltre a ciò amorevole, ed alla mano, ed \textit{huomo dabbene} vuol dire huomo di coscienza,
  huomo d'anima, e che fa opere buone. Spagn. \textit{hombre de bien}. L'uno
  e l'altro comprendono i Greci colla sola parola \textit{Caloscagathos}. \textit{Calos} significa
Onesto, digarbo. \textit{Agathos} buono, da bene.
\item[GREMITI] Ripieni. È il latino \textit{Spissus}. \textit{Densus}. E qui vuol dire havevano
  gran quantità di danari; se bene è detto improprio,perché gremito s'intende un'albero
  pieno di frutti, un luogo pieno di mosche, o simili; perché tal voce si dovrebbe
  usare in quelle occasioni, nelle quali cade la similitudine del proprio di
  essa foce. \textit{Greto} vuol dire terreno ghiaioso, e pieno di sassi, come sogliono rimanere
  le rive de i nostri tinmi, scolata che è l'acqua piovana, quali rive però
  si chiamano Greto, come greto d'arno, greto di mugnone, ec, Ora Grero \textit{addiettivo},
  dice il Vocabolario della Crusca, /\textit{l diciamo in significato di spesso; forse dalla
    multitudine spessa de' sassi de' greti; e diciamo anche in questo significato Gremito}. Quanto
  a me, inclinerei a credere, che \textit{Gremito} dal dirli propriamente degli alberi, quando
  sono pieni di fiori, o  carichi di frutta, venisse da \textit{Gremium} perciocché il
  grembo è quella parte, che suole empiersi di tali cose. Gli antichi Volgarizzatori
  quel che i Latini dissero \textit{littus} essi tradussero \textit{greto}; laonde potrebbe ad alcuno
  parere, questa parola fatta da quella. Seneca epist, 15; \textit{Illos reperti in littore calculi
    leves, \& aliquid habentes varietatis delectant}. I Fanciulli si dilettano  in cose di
  piccol pregio, sì come sono pietre, che l'huomo truova nel viaggio, e nel \textit{greto}
  del mare, e ne' fiumi. Palladio nel Gennaio tit. 14. favellando della lattuga.
  \textit{Candida fieri putantur, si fluminis arena, vel litoris frequenter spargatur in medias}.
  E possono diventare bianche se entra loro, e intra le loro foglie spesse volte si
  sparga rena del fiume, o del greto. Qnde a dire \textit{gremito di soldi} s'intenderebbe
  che havesse sopra il vestito, o sopr'alla persona sparso gran numero di soldi,
  come \textit{gremito di mosche} s'intende haver molte mosche addosso, e non nella tasca,
  o in cassa. Tuttavia, se bene improprio, è alle volte usato, come qui.
\item[ESSER pane, e cacio, anima, e cuore] Andar' uniti, e d'accordo in ogni operazione.
  \textit{Bene conveniunt, \& in una sede morantur}.
\item[S'hano cenci, o terra al Sole] Se hanno masserizie, o poderi; per esprimere,
  uno che habbia poca roba diciamo: \textit{Il tale ha quattro cenci}, o se ha beni stabili
  in terreni: \textit{Egli ha della terra al Sole}.
\item[SIAMO di sì perfida cottoia] Siamo così iniqui, e di mal' animo, Quei legumi,
  che per molto che si tengano al fuoco non si quocono, ne inteneriscono
  mai, si dicono \textit{di cattiva cottoia},  e però con dire huomo di \textit{cattiva cottoia}, s'intende
  di genio maligno, e difficile a persuadersi al bene. Gr. \textit{ateramon}.
\item[ESSER al lumicino] Vuol dire esser in estremo di vita; e viene dall'uso, che è
  nello Spedale di S. Maria Nuova di mettere un piccolo lume a un Crocifisso al
  letto di coloro, che sono agonizzanti. Si dice ancora; esser alla candela.
\item[NON gli sovverrebbon d'un lupino] Non gli darebbono un minimo aiuto. \textit{Sovvenire}
  neutro vuol dir ricordarsi. Non mi sovviene, quando fu questo. Non mi
  ricordo, quando fu questo. Lat. \textit{mentem subire}, \textit{in mentem venire}, \textit{succurrere}, Fr.
  \textit{se souvenir}.
\item[MOZZORECCHI] Huomo scellerato, ed infame: E questo, perché quei
  malfattori, che per la tenera età sono esenti dalla ordinaria, vengono dalla
  giustizia contrassegnati, come dicemmo sopra C.2. stan. 3. e \cstan[6]{54}. e fra
  gli altri contrassegni uno è il mozzar loro una parte degli orecchi.
\item[PORTAR acqua per orecchi] Fare a uno tutti i servizzi possibili.
\item[HAVBEBBON volute indovinare] Questo termine esprime la grand'attenzione
  che uno ha in servir l'altro, e compiacerli in tutto quel che possa accadere.
\end{description}
\section{Stanza XXXI. --- XXXIV.}

\begin{ottave}
\flagverse{31}Essendo un giorno insiemo a um convito, \\
Quand'appunto aguzzato hanno il mulino, \\
E mangian con buonissima appetito, \\
Non so come il maggiore dette Nardino\\
Nell'affettar il pan tagliossi um dito,\\
Sì ch'egli insanguinò il tovagliolino,\\
E parvegli sì bello a quel mo intriso,\\
Ch'ei si pose a guardarlo fiso fiso.
\end{ottave}

\begin{ottave}
\flagverse{32}E resta a seder lì tutto insensato, \\
Ch'ei par di legno anch'ei come la sedia,\\
Può far (tanto nel viso è dilavato)\\
Con la tovaglia i simili in commedia,\\
E mirando quel panno insanguinato\\
Hor mai tant'allegria muta in tragedia, \\
Meatre nel più bel suon delle scodelle \\
Si vede ognun riposar le mascelle.
\end{ottave}

\begin{ottave}
\flagverse{33}E tutti quei che seggon quivi a mensa\\
I servi, i circostanti, ed ogni gente\\
Corrongli addosso, che ciascun si pensa\\
Che venuto gli sia qualch'accidente;\\
Ne sanno che il suo male è in quella rensa\\
Com'appunto fra l'erba sta il serpente,\\
Rensa non già, ma lensa ond'il suo cuore\\
Preso al lamo col sangue haveali Amore.
\end{ottave}

\begin{ottave}
\flagverse{34}Che gli par di veder, mentre in quel telo\\
Contempla in campo bianco i fior vermigli,\\
Un carnato di qualche Dea di Cielo\\
Composta colassù di rose, e gigli,\\
E sì gli piace, e tanto gli va a pelo,\\
Che finalmente mentre ch'ei non pigli\\
Una moglie d'un tal componimento\\
Non sarà de i suoi dì mai più contento.
\end{ottave}

Essendo gli suddetti giovani a un convito, Nardino, che era il maggiore,
affettando il pane, si tagliò un dito, ed insanguinò il tovagliolino, e nel mirar quel
bel rosso in sul bianco, s'innamorò in maniera, che si propose di non haver mai
a restar consolato, s'ei non pigliava una moglie composta di quel colore del
tovagliolino insanguinato.
\begin{description}
\item[CONVITO] Desinare, o cena splendida, Dal latino \textit{Convivium}, o più tosto
  da \textit{Convitare} nel senso che gli Spagnuoli pigliano il loro. \textit{Combidar}, per \textit{invitare},
  e nel quale il prese il Boccaccio, che disse, \textit{Convitò a mangiare}. E, \textit{Convitati alle
    tavole}.
\item[AGUZATO il mulino] All'ordine con la fame per mangiare. Così tratta la
  similitudine dal mulino; dicesi \textit{Macinare a due palmenti}, cioè mulini; di chi per
  prestezza, o voracità mastica da amendue i lati a un tratto. Vedi sopra C. 4.
  stanza 22.
\item[APPETITO] Vuol dir appetenza, e desiderio im generale, ma quando è
  detto assolutamente, e senz'aggiunta, vuol dir Fame o voglia, o gusto di
  mangiare. Vedi sopra C. 4. st. 8, \textit{il mal che viene in bocca alla gallina}.
\item[TOVAGLIOLINO] Quasi piccola tovaglia. Quel pezzo di panne lino, che si
  tiene avanti, quando si mangia essendo a mensa. Il Boccaccio disse \textit{tovagliolo}. Noi
  lo dichiamo anche salvietta dalla voce Spagnuola \textit{Servilleta}, perché serve molto
  al ministro, e al servizio della tavola.
\item[INTRISO] La polvere, o altra materia simile stemperata con liquore, come
  sarebbe farina, e acqua si dice intriso, e intridere. Ma significa ancora imbrattato,
  sporcato, ec. come significa in questo luogo.
\item[FISO fiso] Senza batter' occhio, con grandissima attenzione: \textit{intentis}, \textit{inconniventibus
  oculis}. I Greci dicono in una parola \textit{Ascardamycti}, che è lo stesso che senza batter' occhio.
  Petrarca.
  \begin{verse}
    \backspace Così vedess'io filo,
    Come Amor dolcemente gli governa
    Sol un giorno da presso,
    Senza volger giamai rota superna,
    Ne pensassi a' altrui, ne di me stesso,
    El batter gli occhi miei non fusse spesso.
  \end{verse}
\item[DILAVATO] Impallidito.\ Smorto.\ Si dice dilavato ogni colore, che non
  arriva alla perfezione della sua essenza: come rosso \textit{dilavato} si dice un color rosso,
  che sia più sbiancato, e più chiaro del vero rosso. Latino \textit{dilutus}.
\item[PUÒ far con la tovaglia i simili in commedia] Intende ch'egli è bianco appunto
  come è la tovaglia. Latino \textit{non ovum sic ovo simile}. \textit{I due simili} è un suggetto di
  Commedia, come quello de Menechmi di Plauto, a molti vi hanno scherzato, perché
  è argumento fecondo d'intrecci.
\item[RENSA] Specie di tela lina fatta a un'opera, che si chiama rensa, detta così
  dalla Citta di \textit{Rens} Francia. Così \textit{Perpignano} sorta di panno dalla Città della
  Navarra di questo nome. \textit{Arazzi} dalla Città d'Arras in Fiandra: e \textit{Duagio} al
  tempo del Boccaccio si diceva un panno, che veniva di \textit{Dovay} Città di Fiandra,
  che Gio: Villani secondo l'uso de' suoi tempi, chiama Doagio. Latino \textit{Duacum}.
  \textit{Baldacchino}, drappo di Levante; da Babbillonia, che i Levantini chiamano \textit{Bagdad},
  i nostri antichi \textit{Baldacco}, Gio: Villani l. 7. \textit{E messo fuori della Città, sopra
    la sua persona un ricco palio di Baldacchini di seta e d'oro}.
\item[LENZA] o \textit{lensa}, Lat, \textit{linea}, \textit{filum piscatorium}, detta così quasi dal Latino
  \textit{lintea}. Quella cordicella fatta di crini di cavallo, o di seta cruda, con la quale
  si lega il lamo da pescare. Franco Sacchetti Nov. 163. \textit{Egli haves preso l'alluminato
    alla lenza pescandolo con 200. Fiorini d'oro}. Lasca Nov. 166. \textit{Fu un pescatore
    di piccole pescagioni pescando con lami, e con lenze}.
\item[TELO] Con l'\letter{e} stretta. Pezo di tela in larghezza del suo essere, e lungheza
  \textit{ad libitum}; come un telo di lenzuolo, o di paramento sdrucito in tutta la lunghezza
  di esso lenzuolo, o paramento. Diciamo \textit{telo} da pane quella tovaglietta,
  o striscia di panno lino, con la quale si cuopre il pane in su l'asse. Qui intende
  il tovagliuolo. \textit{Telo} con l'\letter{e}  largo usato da alcuni in Poesia, vuol dire il dardo.
  Lat. \textit{telum}.
\item[GLI va a pelo] Gli va a genio. Se gli confà: è secondo il suo gusto; è l'opposto
  d'Andar contrappelo detto sopra \cstan[6]{1}.
\end{description}
\section{Stanza XXXV. --- XXXVI.}
\begin{ottave}
\flagverse{35}E già se la figura nel pensiero, \\
E bianca, e fresca, e rubiconda, e bella,\\
Co' suoi capelli d' oro, e l'occhio nero, \\
Che più ne men la matturina stella. \\
E come ch'ei la vegga daddovero\\
Divoto se le inchina, e le favella,\\
E le promette, s'egli havrà moneta,\\
Di pagarie la fiera all'Improneta.
\end{ottave}

\begin{ottave}
\flagverse{36}E vuol mandarle il cuore in un pasticcio,\\
Perch'ella se ne serva a colazione;\\
E gli s'interna sì cotal capriccio\\
E tanto se ne va in contemplarione,\\
Che il matto s'innamora come un miccio,\\
D'un'amor che non ha conclusione,\\
Ma ch'`e fondato, come udite, in aria,\\
D'una bellezza finta, e immaginaria.
\end{ottave}


Nardino s' immagina, e si compone nel pensigro una'
parendogli d' haverla veramente avanti a gliocchi., le parla, e se
le dona il cuore; ed in questa guisa s' inaamora ardentemente d' una b
maginaria. sie eye Han HOR ee IPE

PRES C.A. Trattandosi d' huomo s' intende Vino dipoca eta; ed h
donna freschi s' intende fani, gagliardj 5. di buona cera,quantunque
grave. Virg. cruda deo, viridi/que fenetius. Frefeo Secondo il Ferrari

ney

Origine dal Latino vire/cens. La marturina Hella. Virg. Qualisie
fer undis,; to

PAG ARLE la fiera all' Improneta, Pagarle un regalo all
giorno di S, Luca 18, d' Oucbre all' Impruncta', la quale'è una Chiesa
tana da Firenze, celebre, e frequentata per un' immagine miracolosa
stima vergine, che è quivi; la quale in tempo di calamita, e di
portata solennemente a Firenze; e nella venuta di questa Immagine f
una Lauda in una Raccolta antica di Laude spiriwali,

E SE gli imerna si cotal capriccio, Gli si ficca nel cervello, o-gli
mente questo capriccio, fantasia, opinione. Vedi sopra C, 15M. a4
S' INMAMORA come un miccio., S' innamora come un' asino
mente, perché l'asino e ostinatissimo, e capone. 5 itt

\section{Stanza XXXVII --- XXXIX.}
\begin{ottave}
\flagverse{37}Così a credenza infacca nel frugnuolo,\\
Ma da un canto egli ha ragion da vendere,\\
Che s'egli è ver, c'Amor vuol esser solo,\\
Rivale non è qui con chi contendere.\\
Ma Brunetto il fratel, che n'ha gran duolo,\\
Poich'il suo male alcun non può comprendere,\\
Tien per la prima un'ottima ricetta,\\
Di rimandarlo a casa in una seggetta.
\end{ottave}

\begin{ottave}
\flagverse{38}Ove condotto, e messolo in sul letto,\\
Il medico ne venne, e lo speziale,\\
Chiamati a visitarlo, ma in effetto\\
Anch'essi non conobbero il suo male.\\
Disperato alla fin di ciò Brunetto,\\
Col gomito appoggiato in sul guanciale,\\
A cald'occhi piangendo più che mai;\\
Io vo saper dicea, que che tu hai.
\end{ottave}

\begin{ottave}
\flagverse{39}Ei che vagheggia sott' alle lenzuola \\
Il gentil volto, e le dorate chiome, \\
Ne anche gli risponde una parola, \\
Non che gli voglia dir ne che, ne come,\\
Replica quello, e seccasi la gola:\\
Lo fruga, tira, e chiamalo per nome\\
Ed ei pianta una vigna, e nulla sente\\
Pur tanto l'altro fa, ch'ei si rifente.
\end{ottave}


Così Nardino sianamora ardentemente senza saper di chi. Brunetto fu)
tello lo fece portare a casa, dove lo meflero in sul letto, e vennero,
Speziali a vifitarlo, ma non conoscevano ne meno essi il di lui male; onde
netto si messe a pregarlo, che gli dicesse quel che egli havea; e Nardino!
la sua contemplazione non rispondeva; pure alla fine vinto da tanti
fratello parlo nella maniera, che vedremo nell' Octave seguenti.
eA CREDENZ A, Vuol dire, quando si compra qualche mercanzia,



See aS SELB RTE SE \&

ASE

=~

SETTIMO CANTARE. 343
si sborsa il danaro allora, as garlo in altro tempo. Ma qui vuol
dire feniza proposito, o senza son *

mento. fH Varchi nel Cap. dell' vova fede.
o) Chiba fquadrato ben la quintesenza,
“. 9) Dite ch' ella non ha color neffuno-,
o 5 Bebe quel giallo v e posto a credenza.
pTr@ng) Rir67." > ° '
Contro di noi bravavano a credenza.
Questa maniera è corrispondente al graris de' Latini. Perfecuti sunt me gratis, La
version Greca dice dorean; in dono, cioè di lor cortesia, senza che io il meritassi.
INS ACCA nel frugnuolo: S' innamora; Se bene entrar nel frugnolo vuol dire
anche entrar' in collera. Frugnuolo è  Janterna; con la quale si va di notte
a caccinagli vecelli,ed'a pescare; ed è parola corrotta da fornuolo, perché tal
trnaessendo simile alla bocca d'un forno, così e chiamata.
EGLI ha ragion da vendere Gli avanza della ragione. Ha grandissima ragione.
\ SEGGETT-.A; Seggiola portatile con due stanghe. Vedi sopra C. 1. st. 48.
 GOMITO, La congiuntura del braccio dalla parte di fuori, dove si piega a.
mezzo il braccio,, dal Latino cubirs.

VAGHEGGIA, Fa all'amore, amoreggia, con desiderio d' avere la cosa amata,
Yagguarda., come difie il Buti cittadino, e Lettore Pisano nella sua lettura sopra
Dante, Vedi sotto C. ro. st. 44. Dan, Purg. C. 16.
aT A Esce di mano a [ui, che la vaghegvia,

\ S.* Prima che sia a guisa di fanciula.
Enel Parad) 10. | Eli comincia a vagheggiar nell' arte Di quel Macfire.
Fazio degli Vberti nel Dictamundi; canto 143.
ale Efe @' udirlo proprio ti vaghegsi.
(cioè feivagho; ardentemente desideri ) E canto 144.
we Bios va pur, che quanto priego', e chieggio
Al fommo bene, e fol, che tosto sia

2 Vaeay Wel paefe, ch' i bramo, e ch* i vagheggio.
cioè desidero, ne son vago; col quale io fo all' amore; ea cui mi pare un' ora,
mille anni di ritornare » Vagheggiare il Ferrari deduce dal Latino vistare, frequen-
ter ae, citaa ptoposico i versi di Lucrezio lib, 1. che descrivono Marte, che

Venere. uae
—— in gremium qui fepe tuum se
'yO Reycit aterno devinitus vulnere 'amoris,
Arque ita suspiciens tereti cervice reposta,
Pascit amore avides inbians in te Dea vifus,
O:pure view da Vago, avido; perché chi e avido di godere la cosa amata, va at-
torno percercarla, e si rigira come farfalla intorno al lume della bellezza di
quella, Dante in un suo Sonetto.
. To son se vago dela bella luce
Degli occhi traditor, che m' anno occifo,
Che la dov' io son morto, e son derifo,
La gran vagherza pur mi riconduce.
NE che, ne come. Intendi, che non solo non gli volle dire ne il male, ne la,
Caul@ di efig, ma ne meno volle parlare, SEC-

'a

344

MA LMDANMTILE o

SECC-ASI la gola., Se glia icequanie fauci per. isemnan eb Li strc d
PIANT A una vigna.,. Non bada, 0-non attende.a, ice «Che noi

diciamo anche far orecchte di mercante » che è:'

titi, che lif

propongono, attento solo al, suo vantaggio., irc 57 'Ear conte che

L Imperatore o far conto, che uno cants.. Per il conteario schi parla
non bada, o non vuol badare, dicesi Predicare al defer

C. 10, st,

bere.
Studio iaktabat inani,

46. Jn Latino pyre.trovanGi molt: detti in questo
Vento loqui., Surdo canere, Frufira 3 velin wannm cantare, cum pisce,
Aliam rem agere » Oc, Virg. Ech 2, tbi bec sanouiteeli

@ gente,
10, Predicare:
significato, come:
Lire

z ve è

SZrifene. Cioè si sloeglia da quella applicazicne o filamin unis sili

\section{Stanza XXXX --- XXXXIII.}
\begin{ottave}
\flagverse{40}Dicendo; Fratel mio, se nae mi vuoi
Quel benyche tu dicei volermi a faced,
Non mi dar noia,va pe' fatti.tuor,
Lerche ii mio mal non è male da biacca,
Al quale ad ogni mo trovar non puoi
Un rimedia, che vaglia una patacca,
Perch'eglie e stravagante, ed alla moday,
Che non se ne rinuien Caposne coda
\end{ottave}

\begin{ottave}
\flagverse{41}Brunetto udito il caso, e quanto e' sia
Il suo cordogtio,anch' eidolente refea,
Se ben per fargli cnor mostra allegria
Ma(com'io dico)dentro e chi la pefia
Perch' in veder si gran malinconia,
Ed un umor si fifo nella testa,
Jn quanto a lui gli par che la fucchielli,
Per terminare il giuoco a! pagrerelli,
\end{ottave}

\begin{ottave}
\flagverse{42}Kedi jSoggsnnfest' altro, och' io m' adirey'
O par. Rpiuernred etme
Hai tus quistione? hai tu qualche rigire
Tx me 0s 4 dire in tutte le manicre,
Lardin rispose, dopo on ne
Tu sei importuno oi pbanmal
Ma da chiio devo, iro eccomi prow; 4
Così guivi-di tutto fa unragcontes.)
\end{ottave}

\begin{ottave}
\flagverse{43}E conoscenda, c' a ridurlo in sesto
Ci vuol'alero che il medico,oilbarbiert,
Vifi spenda la visa, e vadailrefios. |
Vuol rimediarvi in tutte le ae
E sav Ff risolue pr
D andar girando il i meal kh
Di trovargli una mogtic di suo gfe,
Com' ei gliel' ha dipinta ginsto gino,
\end{ottave}


Fratel mio, se veramente tu mi porti quell'affetto, che tu dici, lasciami stare,
e non mi dir più altro, perché ad ogni modo tu non puoi rimediare al mio male,
che è grandithmo, Brunetto di nuovo lo prega, onde Nardino, vinto dalla sua
importunità gli racconta tutto il caso, e Brunetto, se bene dentro haveva gran
travaglio facea buon viso, e datogli animo si risolvé d'andar girando il Mondo
per veder di trovare una donna secondo il gusto di Nardino, e cavarlo di quella
frenesia.

Una esortazione, e richiesta simile a quella, che fa Brunetto a Nardino, fa il
Maccherone allo Gnocco per saper la di lui afflizione come si vede ne i seguenti
versi dello Stefonio\footnote{Bernardino Stefonio (Poggio Mirteto, 8 dicembre 1560 – Modena, 8 dicembre 1620) gesuita, umanista, drammaturgo. } nella sua Commedia Intitolata \textit{Maccaroides, sive Gnoccheides}
Atto pr. Sc. pr., quali riporto qui, perché il Lettore veda, che a un'huomo Letterato
(come era lo Stetonio) non si disdice alle volte lasciare gli studj più serj
per le bizzarrie fanciullesche, e spero, che non sarà discara questa poca digressione

\begin{verse}
  \setlength{\parindent}{2.5em}\upshape
\textsc{GNOCCHVS, ET MACCHERO}
\pfix{Gn.}O me tapinum! Mundo traviare venivi,
Cur non tum morui, cum primim lucis in aura
Sborsavit genitrix ? Cur me disgratia semper
Perseguitat manigolda senem? Cur ladra placerum
Abstulis, \& cunctis caricas me saeva malannis?
Quando finalmentum dabitur misura travai?
Quando refinabis streghissima filia streghae?
Dum me pensabam biancam reposare vechiezam,
Mille diabolicis straziorque, creporque ruinis.
Vh me meschinum! Poterit quis ferre socorsum?
\pfix{Ma.}Appuntum Gnoccum video. Quid brontolas? olà!
Fronte malinconica, quid tecum, Gnocche, favellas?
Deh poverome, pares viridas magnasse lucertas;
Tam demagratus, tam disvenutus apares.
Testa dolet forsan? Sciatica? Fistula? peius?
An potius placidam sturbant penseria mentem?
Dic mihi quæso tuam scannat quid, Gnocche, coradam?
\pfix{Gn.}Vade viam, Macherone, tuam. Fradele, fogare
Me volo, nec quidquam poteris succurrere Gnocco.
\pfix{Ma}Ohimé! cur sprezas fradelli verba pregantis ?
Quis scit ? parlando passabit forte dolorus,
Praesertim caro dum palesatur amico.
\pfix{Gn.}Deh nolis, quaeso, nolis mihi rumpere testam:
Deh lassa me star; sum plenus; vade bonhoram,
Nec des impaccium, quoniam mihi crescis afannum.
\pfix{Ma.}Deh poffar mundus! Tortum mihi facis adessum.
Cur mihi, Gnocche, tuum non vis sfogare lamentum?
Sum pro te chi lo: praestum dic, quaeso, travaium.
\pfix{Gn.}Pur ibi: Vade tuum, cancar! tu vade viaggium.
Me miserum! ad mundum veni trascinare cordam;
Mancum nonne malum fuerat non nascere, vel si
Nascere debebam, plus praestum nascere fungus,
Quam male stentando scontentus vivere semper,
Omnibus \& giornis centum morire fiatis ?
\pfix{Ma.}Maide! Cordoglio sciappas, \& spernis aitam?
Vadis \& ad guisam matti, Lanzique briachi?
Insuper, \& sdegnas, si quis tua vulnera curat?
\pfix{Gn.}O bellum tempus, Machero, pocasque facendas!
Omnes consilium semper dare novimus altris,
Sed fibi medesimis nolunt procurare parerum.
Bene dicit vulgi proverbium: Ducere danzam,
Atque nuces omnes, qui sedent, bactere norunt,
Cum sunt ad terram. Me lafits dico, malhoram.
\pfix{Ma.}Ah Zucarine meus, meus, ah Gnocchine, galantus,
Quid facies hosti, si desdegnaris amico?
Cur mihi nascondis, quæ mazant vulnera cordem?
Non ego partibo, nisi contes ante marezam.
Su, fradelle, tuum crepacorum quaeso raconta.
Non parlas? Deh butta fora, meschine, venenum,
Dic mihi, quae carpunt fastidia tristia mentem,
Quae lacerant curae, quae te suspiria rumpunt?.
Nonne recordaris strictos nos esse parentes?
Est tua mamma meae carnalis, Gnocche, sorella,
Atque ego natura si non carnalis, amore
Sum tibi fradellus plus quam carnalis: aitam,
Quam potero tibi, Gnocche, dabo, fac denique provam,
Nam tibi porto benum, nec me fradelle licenties.
\textls[-90]{Namque} amo te plus quam me \textls[-70]{stessum, Gnocche}, si \textls[-70]{certum}.
Dicito cuncta mihi, nec te meschine sasines,
Consilium forsan potero tibi dare galantum.
Quid turbulentus guardas? su butta deh foras;
Eia, valent'homus; non singhiottire bisognat;
Vulneris ascosti nunquam medicina trovatur;
At sborsando foras sanatur saepe dolorus;
Fistulae, quae tumuit, totos corrumperet artus,
Ni lancetta viam barbieri lesta taiaret,
Susum, Gnocche valens, cordolia dire comenza.
\pfix{Gn.}O fortuna mihi nimium traversa tapino,
Quae mihi per forzam non strappas ventre magonem;
Est ne possibilum, quod non sborsare fiatum,
Unam nec potero gambam distendere voltam?
Sum desperatus: volo me impiccare da verum;
Cerne, mei, Machero, cavezam porto somari.
\pfix{Ma.}Impiccare? mai. Non impiccare te,non non;
Mattescis; costat troppum impiccare: nientum
Tu facies. Guardes gambam! impiccare? Diavol!
Et te, meque simul piccares, Gnocche
\pfix{Gn.}\makebox[19em]{}sodannum.
\pfix{Ma.}Maidé, quis tantum milzam tibi rodit afannus?
Dic, saporite meus, quae te sventura chiapavit?
\pfix{Gn.}Sime impiccabo, cunctos scappabo, travaios.
\pfix{Ma.}Pur illuc: istam mattezam manda malhoram.
\pfix{Gn.}Sola meum stentum poterit sbandire, caveza.
\pfix{Ma.}Ah nimium certe te stessum, Gnocche, sasinas:
Mancum donna timet, mancum se donna sgomentat.
Ne facias cosam talem pazescis adessum,
Incidis in brasam cupiens evitare padellam,
Qui fugiens damnum, soccorsum a Morte rechiedis,
Qua nullum maius damnum reperitur in orbe.
Dicas, quid peius furca maginare potestur?
Nonne vides furcas ipsos odiare sasinos,
Millantas furcas meritant qui mille fiatis?
Forse putas bellam cosam piccare sestessum?
Nullos audisti, nullos nec, Gnocche, latrones
Esse volenterum piccatos. Canchere! robbam
Perdere, poderos, filios, atque moieram
Possumus; at contum non mittit perdere vitam.
Parlemus d'altro, bona notte; porge cavezam,
Fac sennum matti, caveas non talopram.
\pfix{Gn.}Si sennum matti facerem, mattissimus essem; '
Sum deliberatus cannam truncare una volta;
Nec parles; quoniam mandas tua verba Patrassum,
Et liquidas tentas accogliere retibus auras;
Dextra orecchia bibit, sed versat laeva parolas;
Surdo verba canis; oleum simul opera perdis.
\pfix{Ma.}Qui pro te robbam propriam, vitamque gitarem,
Pocum stimo malum pro te gittare parolas.
\pfix{Gn.}Indarnum gracchias, indarnum dico, va viam.
\pfix{Ma.}Litera vis tandem fieri longissima?
\pfix{Ga.}\makebox[\widthof{Litera vis tandem fieri longissima?}]{}Certum.
\pfix{Ma}Et godis cortum laqueo difrumpere collum ?
\pfix{Ga.}Audis.
\pfix{Ma.}\makebox[\widthof{Audis.}]{}Et tandem cornacchis essere pastum?
\pfix{Ga.}Sentis.
\pfix{Ma.}\makebox[\widthof{Sentis.}]{}Bavosam buccam torquere?
\pfix{Gn.}\makebox[\widthof{Sentis.Bavosam buccam torquere?}]{}Cosinum.
\pfix{Ma.}Et tralunatos oculos mostrare?
\pfix{Gn.}\makebox[\widthof{Et tralunatos oculos mostrare?}]{}Davanzum.
\pfix{Ma.}Lucentem faciem, lucentia bracchia, fusa
Viscera, contradam totam pestare fetore,
Et vitiare diem vitiato viscere laetum?
\pfix{Gn.}Sinum; si dico, sinum; volo rumpere cannam.
\pfix{Ma.}Heu ipsis fugiende lupis, buttande fosatis,
Terribilis stratiande modis, privande sacrato.
Denique penserus nullus te, Gnocche, tuorum
Tangit? Cui lassas pupillos, paze, chiatinos?
Cui robbam? cui consortem? miserosque parentes?
Teque finalmentum ? Casae qui scribitur haeres?
Vis proprias carnes tecum mandare Patrasium?
Vis proprios natos panem cattare per uscios,
Dispersos pueros pitocorum more per urbes?
Et post de fora veniet quae fama da verum?
Gloria que Casae lassatur? Respice tandem
Teque, tuosque simul, miserae miserere fameiae
Et miserere tui, qui proijciere fosato,
Indignum sacro corpus recoprire tereno.
Forsan ad Stygias ibis ? seu forsan Achaeum
Ibis ad Infernum ? pensa, pover'home, to factos;
Pensala, dico, benum: facile est calare decorsum,
Sed montare super cancar; stentare bisognat;
Sed nec stentando brutto scapulabis ab Orco.
Horsus tornemus casas; su, Gnocche: cavezam
Casae mitte tuae. Pensas piccare? bel opram;
Essere non vellem, Veneto pro boia tesoro,
At tu, te stessum si piccas, boia sarabis.
Ah tibi, ne quaeso, tibi sis ne boia medemo,
Et qui pro centum mundis non essere velles;
Essere pro nihilo nolis. Cavezam porge, da quaium,
Spettemus pocum, spettemus dico pochettum,
Forsitan ipsa dies saldabit, Gnocche, feritam.
Dura remollescunt paleis, \& tempore sorba,
Nespula dura die mitescunt, nespula dura,
Guarda mo, si Gnocchi poterit mitescere noia.
\pfix{Gn.}Tu bene cicalas; dottorus, \& esse videris:
Sed cicala purem; giettas nam carmina saxis.
\pfix{Ma.}Almancum facias moriturus, Gnocche, placerum,
Extremumque mihi praestes, care Gnocche; favorem.
\pfix{Gn.}Quem nam ? dil.
\pfix{Ma.}\makebox[\widthof{Quem nam ? dil.}]{}Iura; facies, quod certe domando
\pfix{Gn.}Dummodo fare queam, fabo, sta supra parolam,
\pfix{Ma.}Et potes, \& legrus facies.
\pfix{Gn.}\makebox[\widthof{Et potes, \& legrus facies.}]{}Dic ergo, quid opta
\pfix{Ma.}Est mihi botazus vinetti, Gnocche, rubentis,
Quod disamoratis posset rubare coradam,
Illius humore taze cum plena planura est,
Saltitat, \& brillat, brillando lumina frezat,
Et rubor in vitro liquefatti more rubini,
Ac dicto citius spumat; hunc inde dileguat
Puri sbottigliata meri vis fervida, qualis
Cum soffiat Boreas, nubes sfrattare per auras
Cernitur, \& Calum late purgare serenum.
Sat scio, si nasum praestabis ad ante bicherum,
Optabis fieri totum te, Gnocche, nasonem;
Piccantum retinet pulcrum, garbumque galantum,
Quod refucitaret mortos. De hoc, quaeso, pochettum
Gustes, ante tuum claudas quam toste fiatum,
Atque mei hoc portes extremi pignus amoris.
Vis rechem chi lo?
\pfix{Gn.}\makebox[\widthof{Vis rechem chi lo?}]{}Reches, sed frettola passum.
Nigotta proderit, cum sim piccandus adessum.
\pfix{Ma.}Attamen hanc lasses, dum torno, Gnocche, cavezam,
Ne te gire viam tua tantum spafima cogant,
Et fine gustando vinum, morire, galantum.
\pfix{Gn.}Sum contentus; abi, gratum sed porta fiascum,
Nam sitio certum, \& vampat brusore fegatum.
\end{verse}
\begin{description}
\item[VOLER bene a sacca] Portar grandissimo affetto. È frase usata da' fanciullini.
\item[VA pe' fatti tuoi] Cioè vattene, e bada a te, \textit{Res tuas tibi habeto}; dicevano i
  mariti anticamente alle mogli, quando secondo le leggi Romane le rimandavano.
  Vedi sopra C. 5. st. 57.
\item[NON è mal da biacca] Non è male ordinario, e che si risani con poco rimedio,
  perché la \textit{Biacca}; che è un bianco cavato dal piombo, ed è adoprato da i Pittori,
  serve anche per fare un'unguento buono a poco altro, che ad alleggerire il
  dolore alle semplici contusioni; E però dicendosi: \textit{Non è mal da biacca}, s'intende,
  è un gran male.
\item[CHE vaglia una patacca] Che vaglia nulla. Che \textit{patacca} è moneta, che in
  Firenze non vale, \textit{Patacon} una moneta di rame usata in Portogallo, che vale
  tre guattrini, Così noi d'una cosa da noi tenuta in poco pregio, diciamo. \textit{Non
    vale un soldo, Non ne darei un soldo}.
\item[ALLA moda] Vuol dire all'usanza, come vedemmo sopra C. 2. st. 54. ma
  in questo luogo vuol dire stravagante, o nuovo, e non più sentito, o visto, e del
  tutto insolito; Diciamo: \textit{cervello alla moda} per significare cervello stravagante, o
  fantastico; dal mutar che si fa tutto giorno della moda nel vestire.
\item[NON si rinviene ne capo, ne coda] Non si ritrova, ne il principio, ne la fine di
  questa cosa. Non si sa, non s'intende, o non si ritrova come la cosa si stia. \textit{Nec
    caput, nec pedes}, disse Cic. È traslato dalle matasse del filo, e si dice anche \textit{Non si
    ritrova il bandolo}, che è il principio della matassa.
\item[HAI tu quistione?] Intendiamo havere inimicizie.
\item[HAI tu qualche rigiro?] Hai tu qualche innamorata? Che la voce \textit{rigiro} usata
  come nel presente luogo, vuol dir Pratica di donne per vizio; che per altro \textit{rigiro}
  significa Ripiego, dicendosi: Il tale fa molte faccende, perché egli ha molti \textit{rigiri},
  cioè ripieghi, ed occasioni di vendere la sua roba. Alle volte si piglia per
  Ordigno. Vedi sopra C. 4. st. 60.
\item[DENTRO è chi la pesta] Quand'uno si sforza di mostrarsi nel viso allegro, ed
  ha travagli da star malinconico, diciamo; \textit{Ei fa buon viso, ma dentro è chi la pesta},
  cioè dentro sta in altra guisa. \textit{Risus in ore, fletus in corde}. Virg. \textit{Spem vultu
    simulat, premit altum corde dolorem}.
\item[Humore fisso in testa] Pensiero, o fantasia ostinata. Vedi sopra C. 1. st. 10.
\item[PAR ch'ei la succhielli] Egli sta fra il sì, e il no di fare una tal cosa, che diremmo
  Irrefoluto. Dante Inf. 8.
  \begin{verse}
    Che'l sì, e 'l no nel capo mi tenzona.
  \end{verse}
  Traslato dal giuoco delle carte, che si dice \textit{succhiellare} quando si tira su la carta,
  adagio adagio; il che pure è traslato dal bucar col succhiello, che è una azione
  simile al tirar su la carta. Qui vuol dire. Pare che questa sua fissazione lo voglia
  adagio adagio fare impazzire, e ridurlo a i Pazzerelli\footnote{L'Ospedale di Santa Dorotea, fondato nel 1597, gestito dal 1643 come istituzione psichiatrica dalla Congregazione di Santa Dorotea dei Pazzerelli, passò nel 1754 sotto la diretta gestione granducale.}, che è lo spedale, dove si
  mettono i pazzi.
\item[RIDURLO in sesto] Ridurlo alla giusta misura; Raggiustarlo, rimetterlo in
  buon' essere: fargli ritornare il giudizio. Vedi sopra C. 1. st. 15.
\item[SI spenda la vita, e vada il resto] Si spenda la vita, e la roba. Tratto dal giuoco,
  nel quale si suole scommettere, e dire. \textit{Vada il resto}; \textit{fo del resto}. E qui è detto
  per figura; perché quando è andata la vita, che è la più cara cosa, che noi
  habbiamo, par che non ci resti quasi altro da buttar via.
\item[GIUSTO giusto] Per appunto. E la replica ha la solita forza di superlativo.
  Catullo. \textit{Magis magis increbrescunt}. Nell'Ebraico \textit{Meod}, che vuole dice \textit{assai},
  \textit{molto}, raddoppiato vuol dire \textit{assaissimo}, \textit{moltissimo}.
\end{description}
\section{Stanza XLIV. --- XLVI.}
\begin{ottave}
\flagverse{44}Perciò d'abiti, e soldi si provvede,\\
E dà buone speranze al suo Nardino,\\
E preso un buon cavallo, e un huomo a piede,\\
Esce di casa, e mettesi in cammino,\\
Sbirciando sempre in qua, e in la se vede\\
Donna di viso bianco, e chermisino;\\
E se ei ne incontra mai di quella tinta,\\
Vuol poi chiarirsi, s'ella è vera, o finta.
\end{ottave}

\begin{ottave}
\flagverse{45}Perc'hoggidì non ne va una in fallo,\\
Che non si minij, o si lustri le quoia,\\
E dov'ell'ha un mostaccio infrigno, e giallo\\
Ch'ella pare il ritratto dell'Ancroia,\\
Ogni mattina innanzi a un suo cristallo\\
Quattro dita vi lascia su di loia,\\
E tanto s'invernicia, impiastra, e stucca,\\
Ch'ella par proprio un' Angiolin di Lucca.
\end{ottave}

\begin{ottave}
\flagverse{46}Di modo ch'ei non vuol restarvi colto,\\
Ma starvi lesto, e rivederla bene,\\
E per questo una spugna seco ha tolto,\\
E sempre in molle accanto se la tiene,\\
Con che passando ad esse sopra il volto,\\
Vedrà: s'il color regge, o se rinviene;\\
Ma gira gira, in fatti ei non ritrova\\
Suggetto, che gli occorra farne prova.
\end{ottave}

Brunetto date buone speranze al suo fratello, montò a cavallo; ed havendo seco
un'huomo a piedi, sen'andò cercando d'una donna bianca, e rossa di carne
naturalmente, e sapendo che tutte le donne hoggi si lisciano, haveva preso una
spugna bagnata, per far con quella la prova, se il colore era finto, o naturale. Ma
per molto, che egli cercasse, non trovò mai donna, nella quale occorresse far tal
prova, perché si conosceva senza farla, che tutte eran tinte, e lisciate. Quello
colore finto, che chiamiamo liscio, o belletto, si dice anche \textit{fuco}, che è un'erba
buona a tignere i pani; da i Latini detta \textit{fucus}, e l'intendevano ancora essi
per questo liscio, o belletto. Plaut. Most. 4. 118. \textit{Vetulae edentulae, quae vitia
corporis fuco occultant}. E di qui i Latini per \textit{fuco} intendono una sorta d'inganno,
che ricopre con artifizio un mancamento in una mercanzia, ec, onde: \textit{fucum facere}.

\begin{description}
\item[SBIRCIANDO] Guardando attentamente. Vedi sopraC, 1.f9,
\item[CHERMISINO] Rosso di Chermisi, o Cremesi, E' il rosso porporino, che si
  fa col sangue di certi vermi chiamati con voce Spagnuola \textit{Cocciniglia} dal Latino
  \textit{coccineus color}, \textit{colore di grana}, \textit{colore vermiglio}; ed è il più nobile, ed acceso
  colore, che si trovi, ne mai perde il suo colore: e da questo nel presente luoge intende
  rosso naturale a perfezione, e che non perde, come farebbe il finto: \textit{Kermes},
  o \textit{Karmes} in Arabico vuol dire \textit{grana}. Latino \textit{coccum}, secondo lo Scaligero
  esercitazione 325.
\item[DI quella tinta] Di quel colore. E' termine pittoresco, cotumandosi da essi il
  dire: \textit{La tale ha una carnagione, nella quale sono belle tinte}, per intendere belli colori
  di carne.
\item[VUOL chiarirsi] Vuole accertarsi.
\item[NON si minij] Non si tinga, Minio è specie di color rosso cavato dallo stagno,
  e miniare è una specie di dipignere con finissimi colori sopra cose sottili, come
  cartapecora, ec.
\item[SI lustri le quoia] Si lisci la pelle.
\item[MOSTACCIO infrigno] Vilo grinzoso, o cresposo, o rinfrignato. In Franzese
  \textit{refroigné}.
\item[ANCROIA] L' Ancroia è finta una donna brava in un Poema intitolato la
  Regina Ancroia; e perché questo Poema è degli antichi, che si trovino nella
  lingua nostra, mi do a credere, che quando si dice l'Ancroia, s'intenda una vecchia.
  Il Berni, descrivendo la sua serva in un Sonetto dice.
  \begin{verse}
    \backspace Io ho per cameriera mia l'Ancroia,
    Madre di Ferraù, Zia di Morgante,
    Arcavola maggior dell'Amostante,
    Balia del Turco, e suocera del boia.
  \end{verse}
  Ma può esser ancora, che questa voce Ancroia sia un'addiettivo, che venga da
  \textit{croio}, che vuol dire Zotico, e duro dai Lat. \textit{corium} quasi \textit{inquoito}, fatto duro, come
  il quoio.
  \begin{verse}
    Col pugno gli percosse l'epa croia.
  \end{verse}
  Da questa voce \textit{croio} habbiamo il verbo \textit{incroiare}, che vuol dire aggrinzare, ed
  indurire, e \textit{incroiato} per intender pelle grinza, e fecca, e indurita, come è quella
  delle vecchie, alle quali però si dice per scherzo \textit{Mona incroia}, che nel parlare
  l'ultima lettera di \textit{Mona} confonde, e mangia la prima d'\textit{incroia}, viene a
  suonare \textit{ancroia}, che vuol dir vecchia grinzosa. \textit{Incroiato} si dice un quoio, che per
  stato preflo al fuoco, sia divenuto duro, e grinzoso, ed il simile una cartapecora
  abbruciacchiata. Si dice \textit{incroiato} anche un panno divenuto sodo per gli
  untumi, e lordure; ma di questo è più proprio \textit{incorezzato}, dal Lat. \textit{corrigia}. Il
  Vocabolista Bolognese dice, che Ancroia signitica vecchia, che va crollando il capo,
  e che viene dal Greco \textit{Craein} che vuol dir crollare. Ma venga donde si voglia,
  basta che appresso di noi vuol dir Donna vecchia, e brutta, ed in questo
  senso è presa nel presente luogo.
\item[LOLA] Sudiciume. Terra stemperata con acqua, e ridotta liquida, che con
  altro nome chiamiamo mota. Qui vuol dir quelle materie, che si mettono in sul
  viso le donne, le quali s'imbellettano, Voce fatta per avventura dal L. \textit{illuvies}.
\item[IMPIASTRA] S'unge con materie bituminose, e viscose come è l'unguento.
\item[STVCCA] Stucco è quella composizione di gesso, e colla, e d'altre materie
  tenaci, che serve per riturar fessure, o magagne ne i legnami. E \textit{stucco} è una
  specie di gesso, o terra, o altra composizione, con che si tanno le figure di rilievo.
  Qui per stucco intende quelle materie, che le donne si mettono sopra il viso per
  imbellettarsi la faccia, e turarsi le margini del vaiolo, o altre cicatrici; che il
  verbo stuccare vuol dire intasare, cioè riempiere i buchi, e ragguagliare una
  superficie; donde gli Orefici dicono stuccare, quando con una certa loro lima
  detta lima stucca, spianano i lavori d' argento. \textit{Stuccare} vuol dire ancora quando
  un cibo ci apporta nausea, o i discorsi d'alcuno ci vengono a fastidio.
\item[UN'Angiolino di Lucca] A Lucca fabbricano certi figurini di cera, o di gesso,
  o d'altra materia, a'quali dopo formati danno il colore di carne con un rosso
  lustrante; per questo d'una donna lisciata diciamo; \textit{Pare un'Angiolino di Lucca}.
  Così i Greci, che le belle persone assomigliano alle statue ben fatte, le chiamano
  \textit{Agalmata}, e Properzio, disse che il colorito del viso della sua donna era giusto
  come quello, che si scorgeva nelle pitture del famoso Pittore Apelle. \textit{Qualis Apellaeis
    est color in tabulis}. In un'Epigramma Greco una faccia imbellettata, e lisciata,
  con elegante bisticcio vien detta \textit{Prosopeion}, non \textit{Prosopon}, cioè maschera, e non
  \textit{faccia}. Vedi Cel. Rod. Lect, antig. lib. 29. C. 7.
\item[NON vuol restarvi colto] Non vuol rimanare ingannato.
\item[STARVI lesto] Stare Accorto, o avvertito.
\item[GIRA gira] Cammina in diversi luoghi; cammina moltissimo paese cercando.
\item[IN fatti] E' lo stesso, che in somma, o in effetto. L. \textit{reapse, in summa, profecto}.
\end{description}
\section{STANZA XLVIL..}

\begin{ottave}
\flagverse{47}Dopo che tanto a ricercare è ito,\\
Che i calli al \culo{} ha fatto in su la sella,\\
Giunse una sera al luogo d'un Romito,\\
C'a restar l'invitò nella sua Cella, \\
A lui parne toccar il Ciel col dito \\
(Per non haver a star fuori alla stella) \\
Il passar dentro, ed egli, e il servitore,\\
Ringraziando il buon huom di tal favore,
\end{ottave}

\begin{ottave}
\flagverse{48}Vestia di bigio il Vecchio Macilente,\\
Facendo penitenza per Macone,\\
E perch'ei fu nell'accattar frequente, \\
Per nome si chiamo fra Pigolone. \\
Costui, (com'io diceva) allegramente \\
In Cella raccettò le lor persone, \\
Spogliò il cavallo, e gli tritò la paglia; \\
Sul desco poi distese la tovaglia.
\end{ottave}

\begin{ottave}
\flagverse{49}E lì trovò buon pane, e buon formaggio\\
Tutto accattato, ed erbe crude, e cotte,\\
E del vino fiorito quanto un Maggio,\\
Ch'egli è di quel delle centuna botte.\\
Di che spesso ciascun pigliando a saggio,\\
Stettero a crocchio insieme tutta notte,\\
E perché per proverbio dir si suole:\\
La lingua batte dove il dente duole.
\end{ottave}

\begin{ottave}
\flagverse{50}Brunetto, che teneva il campanello,\\
Dice chi sia, e che di casa egli esce\\
Non per suo conto, ma d'un suo fratello,\\
Del quale insino all'anima gl'incresce,\\
Perché gli pare uscito di cervello,\\
Non si sa s'ei si sia più carne, o pesce.\\
Così piangendo in far di ciò memoria\\
Per la minuta contagli la storia.
\end{ottave}

Capitò Brunetto una sera alla Cella d'un Romito, dove essendo stato raccettato,
stando a tavola raccontò al Romito il caso del Fratello, dicendo, che era
fuora per far servizio al medesimo suo Fratello.
\begin{description}
\item[TOCCAR il Ciel col dito] Conseguir l'impossibile.
\item[STAR alla stella] Dormire all'aria; a ciclo scoperto; alla stella diana, Lat.
  \textit{sub dio}.
\item[MACILENTE] Mal sano; Cioè magro per lo stento, e giallo di carnagione.
\item[FU frequente nell'accattare] Due testi di mano dell'Autore dicono uno \textit{frequente},
  ed è l'ultimo; e l'altro \textit{fervente}, e questo è la prima bozza, e se bene l'uno
  e l'altro può stare, io piglierei l'ultimo, perché in sustanza vuol dire che costui
  era attento, e diligente nell'accattare, e sempre chiedeva, che da questa sua
  importunità, s'acquistò il nome di \textit{fra Pigolone} che così chiamiamo coloro, che
  sempre chieggono, e che mostrando una certa ingordigia di roba, si dolgon sempre
  dello stato loro. Pigolare è il verso de' puicini, che beccano. Lat. \textit{pipilare}.
  Spagn. \textit{piar} dal fare \textit{pio pio}, che così è il lor verso.
\item[DESCO] Tavola, sopra la quale si pongono le vivande, quando si mangia,
  dal Lat. \textit{discus}, che è \textit{pietra rotenda, o lastra da scagliarsi}, Vedi sopra C. 9. st. 49.
\item[TVTTO accattato] Ogni cosa havuta per limosina.
\item[FIORITO quanto un Maggio] Fioritissimo; perché il mese di Maggio è la stagione
  de i fiori; O pure perché queili, che vanno a cantar maggio, portano un ramo
  d'albero tutto pieno di diversi fiori, il qual camo d'albero chiamano un maggio,
  o maio. Diciamo: \textit{vino fiorito}, quando o per esser al fondo della botte, o
  per altro mancamentoj il vino dosi nel bicchiere, ha nella superficie minutissimi
  frammenti d' una cerca specie di muffa bidrica; che è il panno, che si fa
  dal vino, e questi si chiamano \textit{fiori}; sì che qui s'intende, che il vino era vicino al
  fondo della botte, o havea altro mancamento, che produce la detta muffa; se bene
  par che voglia dire Vino isquisito; perché \textit{fiorito} è attributo di perfezione in tutte
  le cose, eccetto che nel vino, che l'esser fiorito è segno d'imperfezione.
\item[DI quello delle centuna botte] Questo numero centuna, benché sia determinato,
  si dee intendere per indeterminato; e vuol dire Cavato da infinite botti di coloro,
  che l'havevan dato per limosina. E questo pure è imperfezione del vino, che
  perde lo spirito, e la bontà in tanti travasamenti, e mescolamenti.
\item[STETTERO a crocchio] Stettero chiacchierando. Vedi sopra C, 1, st. 41., e
  C. 3. st. 3. \textit{Crocchio} così detto dallo strepito, che si fa ridendo, e chiacchierando
  nelle conversazioni di trattenimento, perciò dette Crocchi, Dal romore similmente
  e dal suono che rendono, sono dette da' Francesi \textit{Cloches} le Campane. Così
  diverse lingue s'accordano nel rappresentare con l'arte i semplici suoni inarticolati
  che sono un'inalterabil linguaggio della natura.
\item[LA lingua batte dove il dente duole] Si discorre sempre volentieri di quelle cose,
  dove si ha la passione, o sia di gusto, o di disgusto.
\item[TENEVA il campanello] Parlava sempre lui, Questo detto viene da i Magistrati
  di Firenze, ne i quali uno dei Colleghi si chiama il Proposto, e questo sempre
  parla, e risponde a i litiganti, e chiama, e licenzia dall'udienze, ed i compagni
  stanno sempre cheti; e questo Proposto tiene allato alla sua seggiola un campanello.
  E da questo, quand'uno in una conversazione sempre parla lui, diciamo:
  \textit{Ei tiene il campanello}.
\item[M'INCRESCE fino all'anima] Gli ho grandissima compassione; Vedi sopra
  in questo C. st. 26. Mi dispiace, mi pesa. Dante Inf. 6.
  \begin{verse}
    Mi pesa sì, ch' a lagrimar m'invita.
  \end{verse}
  Il Greco dice \textit{Achthomai}, mi dolgo; e lo Spagnuolo similmente \textit{pesame}. Onde quel
  che in Toscano si dice \textit{dare il mi dispiace}, esso dice, \textit{dar el pesame}: La stessa forza
  ha il dire. \textit{M'incresce} quali \textit{mihi ingravescit}, secondo il Ferrari; \textit{mi grava, e
    pesa}. E perché Amore e peso, cominciò Dante una Canzone. \textit{E' m'incresce di
    me. ec.}
\item[NON sa s'ei si sia carne, o pesce] Non fa quel ch'ei si sia. Non è in cervelio.
  Non ha l'intero conoscimento. \textit{Nuovo pesce} dicevano gli antichi \textit{un'huomo strano,
  o semplice}.
\end{description}
\section{STANZA LI --- LVII.}
\begin{ottave}
\flagverse{51}Sta Pigolone attento a collo torto
Ad ascoltarlo, e poi ch'egli ha finito;
Figliuol, risponde a lui, datti conforto,
E sappi, che tu sei nato vestito,
Che qui è l'huom salvatico Magorto,
Ch'è un bestione, un diavol travestito,
Che se te lo vedessi, uh egli è pur brutto!
Basta a suo tempo conterotti it tutto,
\end{ottave}

\begin{ottave}
\flagverse{52}Egli ha un giardino posso in un bel piano,
Ch' e ognor frorito, e verde tutto quato;
Giardiniero non v' t, ne Ortolano,
Che a entrarvi nefjin pus darsi vanto,
Da per se lo lavora di sua mano,
E da se (0 fondo per via a! incanto,
Con una casa bella di frupore,
Che vi potrebbe ar  Imperadare.
\end{ottave}

\begin{ottave}
\flagverse{53}Ma io ti uno dar' adesso un? abbozzata
Lui presto presso della sua figura.
Ei nacque a' un Folletto, ed' nna Fata
A Fiefol n' una buca delle mura,
ed è si brutto, poiche (a brigata
Solo al suo nome crepa di paura;
O questo e il caso a por fra i nocentini
ed far manciar la pappa a quesbabini,
\end{ottave}

\begin{ottave}
\flagverse{54}Oltre ch' ei pute come una carogna
Ede pin nero della mezza norte,
Ha il ceffo d'Orfo, e it collo diC arogna,
Ed una pancia » come una gran borte
Va in sui balefirs, ed ha bocea di fogna
Da dar ripiego a un tin di mele cotte
Zanne ha di porco, e naso di csvetta,
Che piscia in bocca, e del continuo getta,
\end{ottave}

\begin{ottave}
\flagverse{55}Gli cuopron gli occhi i peli delle ciglia,\\
Ed ha cert'ugna lunghe mezzo braccio;\\
Gli huomini mangia, e quando uno ne piglia,\\
Per lui si fa quel giorno berlingaccio\\
Con ogni pappalecco, e gozzoviglia,\\
Ch'ei fa prima con sangueil suo migliaccio,\\
La carne affetta in vari buon bocconi,\\
E della pelle ne fa maccheroni.
\end{ottave}

\begin{ottave}
\flagverse{56}Dell'ossa poi ne fa stuzzicadenti;\\
Niente in somma v'è, che vada male.\\
Sicché Brunetto figliuol mio, tu senti,\\
Ch'egli è un cattivo, ed orrido animale.\\
Hora torniamo a' suoi scompartimenti,\\
Ove son frutte buone quanto il sale,\\
Vaghe piante, bei fiori, ed altre cose,\\
Com'io ti potrei dir maravigliose.
\end{ottave}

\begin{ottave}
\flagverse{57}Ma lasciando per hor  altre da parte,
Cocomeri vi son di certa raza,
Che chi ne puo haver uno, e poi la parte,
Vi trova una bellissima ragazza,
Che per esser astuta la sua parte,
Diratti che tu gli empia una sua tazza,
A un di quei fonti lì sì chiari, e freddi,
Ma se la servi, a Lucca ti riveddi.
\end{ottave}


Pigolone inteso il bisogno di Brunetto, gli da animo con dirgli, che Magorto
huomo salvatico ha quivi un'orto, dove son cocomeri, che tagliandoli n'esce
fuora una bella fanciuila, la quale chiede da bere, ma se e' segli dà, ella sparisce.
Deicrive ancora in queste quattro Orta ve la qualita di questo Magorto.

\begin{description}
\item[SEI nato vestito] Hai havuto buona fortuna, o quello che bramavi. Usiamo
questo termine per esprimere, quand' uno desiderando qualcosa difficile a trovarsi,
s'abbatte accidentalmente a trovarla per appunto, come ei la desiderava, ed a
proposito del suo bisogno. Dicono te Levatrici, che talvolta nascono bambini
con una certa spoglia sopr'alla pelle, la quale spoglia non si leva loro subito
nati, ma si lascia, e casca poi da per se in processo di giorni; e tal creatura da esse
si dice \textit{nata vestita} ed è preso per augurio di felicità di quella tal creatura; il che
ha dato origine al presente dettato.
\item[VN diavol travestito] Vin diavolo immascherato da huomo; intende un'huomo
brutto, quanto il Diavolo.
\item[BELLA di stupore] Bellissima mirabilis vifu. Tanto bella, che fa stupire chi la
vede; ma per venire la voce stupore dal latino, può ognuno intendere il suo valore.
\item[VOGLIO darti un'abbozzata] Cioè ti voglio descrivere alquanto, o in parte.
  I Pittori dicono AbbozzareY > quelle prime pennellate, che danno in.una tela, o
  altrove, dove voglion fare una pittura. Vedi sopra C. 4. st. 41.
\item[FOLLETTO] Uno di quelli spiriti infernali, che dicono che stieno per l'aria.
  Il Ferrari nell'Origini alla Voce \textit{Folle}, citando Dante Inf. 30, \textit{Mi disse, quel
  folletto è Gianni Schicchi}, dice che i Folletti sono \textit{lascivi genij ac Lemures rifu ac strepitu
  domos implentes}.
\item[FATA] Vedi sopra C, 4. st. 45.
\item[A FIESOL n'una buca delle mura] A Fiesole si veggono ancora alcune reliquie
  delle mura di quella antica Città, ed in essi frammenti di muraglie fra l'altre si
  vede una gran buca di fogna, o d'altra cosa simile, la quale dalle donnicciuole è
  creduta, ed è data a credere a i fanciulli per abitazione delle Fate, e però
  volgarmente è detta \textit{la buca delle Fate}. E questa è quella buca, nella quale dice
  l'Autore, che Magorto era nato d'un \textit{Folletto, e d'una Fata}. Angelo Poliziano
  lib. 3. al titolo Lamia dice: \textit{Vicinus quoque adbuc Faesulano rusculo meo lucens fonticulus
    est, secreta in umbra delitescens, ubi sedem esse nunc quoque Lamiarum narrant
    mulierculae}. Questa credo sia quella caverna, che oggi si chiama \textit{la fonte sotterra}
  luogo orrido  e spaventevole, ma sempre pieno di limpidissima, e freschissima acqua.
\item[NOCENTINI] Cioè quei ragazzi, che s'allevano nello Spedale degl'Innocenti
  detto sopra C. 1. st. 85.
\item[A FAR mangiar la pappa a quei bambini] Così diciamo d'un'huomo, o donna
  estremamente brutti, quasi che sieno come il Bau, la Befana, e simili larve
  inventate dalle Balie per render i bambini ubbidienti, e fare che per il timore mangino
  la pappa.
\item[CAROGNA] Vedi sopra C. 5. st. 3. E questo putire da i Latini era espresso
  col medesimo paragone, perché dicevano \textit{vivum cadaver}. Il Monosini.
\item[PIÙ nero della mezza notte] Negrissimo, più nero del buio.
\item[VA in sui balestri] Ha le gambe sottili, e torte come sono i balestri,
  comparazione vulgata, sendoci una cantilena di Balie, che dice.
  \begin{verse}
    \backspace Ben ne venga Mignamau,
    Ch'a le gambe a balestrucci.
  \end{verse}
  Così Bilenco, e Sbilenco, dicesi chi ha le gambe torte; e ancora \textit{Aver le bilie};
  tratta la similitudine da certi legni torti, o randelli, co' quali i vetturali legano
  stretto, e arrandellano le some; da loro dette \textit{bilie}.
\item[BOCCA di fogna] Alla bocca delle fogne maestre, o principali, che ricevono
  acqua delle strade, quando piove, e la conducono nel fiume d'Arno, è figurato
  un gran mascherone di pietra, il quale ingoia l'acqua ed ogn'altra sporcizia,
  e di queste intende il Poeta; e da questo diciamo: \textit{Bocca di fogna} a uno, che
  mangia, ed ingoia ogni sorta di cibo, se bene sporco, senza distinzione, o riguardo
  alcuno. Latino \textit{helluo, gurges}. Queste fogne in altri luoghi d'Italia sono
  dette \textit{Chiaviche} dal Latino \textit{Cloaca}.
\item[DA dar ripiego] Cioè dove entrerebbono tante mele cotte, quante n'entrerebbe
  in un \textit{tino}, che è quel gran vaso di legno, entro al quale si mette l'uva pigiata
  a bollire per farne vino.
\item[ZANNE] Denti: Propriamente s'intende di quei denti lunghi, che hanno i
  cignali, i lupi, i cani, ec. che noi li chiamiamo anche \textit{denti Maestri}, o \textit{Maestre}.
  Vedi sopra C.2. st. 64. Forse è meglio dir \textit{sanne}, ed è più conforme all'origine,
  Onde \textit{subsannare} buriarsi d'uno ridendo, in maniera, che tutti identi, come disse
  il Bocc. si potessero trarre; mostrando le sanne. Dan. Inf, C, 6,
  \begin{verse}
    \backspace Quando ci scorse Cerbero il gran vermo,
    Le bocche aperse, e mostrocci le sanne.
    \verseprefix{e C.22.}\backspace{}E Ciriatto, a cui di bocca uscia
    D'ogni parte una sanna come a porco,
    Gli fa sentir come l'una sdrucia.
  \end{verse}
\item[NASO, che piscia in bocca] Cioè naso aquilino che ha la punta torta in verso
  la bocca, e pare che vi coli dentro.
\item[BERLINGACCIO] i Giovedi grasso, che è l'ultimo giovedì del Carnovale,
  detto \textit{Berlingaccio} da \textit{Berlingare}, che vuol dire bere, e mangiare, e stare
  allegramente, come si fa in quel giorno: e così Magorto, quando pigliava un'huomo,
  faceva conto, che quel giorno fusse il Berlingaccio, solennizzandolo con mangiammenti,
  pappalecchi, e \textit{Gozzoviglie}, dal godere, Latino \textit{gavisare}, come si trova in
  antico Glossario, onde lo Spagnualo \textit{gozar}, godere, e 'l nostro \textit{gavazzare}.
  Tutti sinonimi, che voglion dir ghiottornie Bocc. g. 8.n. 2. \textit{Si rappattumò con lui, e
    più volte insieme fecero gozzoviglie}, ec.
\item[MIGLLACCIO] Sangue di porco, o d' altro animale mescolato con uova, e
  farina, e poi fritto nella padella a uso di frittata da alcuni Latini detto \textit{Tyrotarichus};
  se bene questa era una composizione di cacio, e salame dal \textit{tyros}
  che vuol dir cacio, e \textit{tarichos}, che vuol dir salame.
\item[STVZZICADENTI] Nettadenti: Sottilissimi, ed acuti stecchi di legno silio,
  d'osso, o d'altra materia per uso di nettare i denti. Latino \textit{dentiscalpia}.
\item[BUONI quanto il sale] Saporitissimi. Una vivanda con molto sale si dice saporita,
  che vuol dire il contrario di sciocca, o insipida, e senza sale, e perché il
  saporito è meglio al gusto, che l'insipido, e però per \textit{saporito} intendiamo gustoso,
  e dicendosi; \textit{buoni quanto il sale}, s'intende saporitissimi, cioè gustosissimi, e tutti
  sapore.
\item[COCOMERO] Specie di mellone acquoso di sapore dolce, che si mangia nella
  stagione calda per rinfrescarsi. In moiti luoghi d'Italia si chiama \textit{anguria}, e
  così la chiama il Mattiolo, e dice che era incognita a i Latini, se bene si trova
  \textit{cucumis}, ma intendono il cetriuolo, che pure in alcuni luoghi si chiama cocomero.
  \textit{Anguria}, dice il Ferrari, e detta quasi \textit{cucumus anguineus}, e così questo nome,
  che era proprio del cetriuolo, per mancanza di vocabolo fu tratto a significare quel
  frutto, che noi Toscani chiamiamo \textit{cocomero}.
\item[A LVCCA ti riveddi] Questo detto significa Non la vedrai più. Tommaso
  Buoni da Lucca nel suo tesoro de' Proverbi dice, che havendo un Gentilhuomo
  Lucchese veduto un Gentilhuomo Pisano a Lucca, usò seco cortesia invitandolo a
  desinare a casa sua, dove condotto, fu trattato con ogni sorta d'humanità. Partitosi
  il Pisano, e ritornato alla patria, avvenne che fra poco tempo il Luccese
  andò a Pisa, dove parvegli convenevole visitare il Pisano suddetto: Trasferitosi
  però alla casa di esso, dopo haver molte volte bussato, al fine s'affacciò il Pisano,
  e gli disse che non lo conosceva; onde il Lucchese disse: \textit{A Lucca ti veddi, e a
    Pisa ti conobbi}, o con questo si licenziò. Così scrive un Lucchese, ma i Pisani
  rivoltano il proverbio dicendo: \textit{A Pisa ti vedi, e a Lucca ti conobbi}. Facendo ingrato,
  e scortese quello da Lucca, e non quello da Pisa. Se bene il Lalli, che non
  era ne Lucchese ne Pisano nella sua En. Te. C. 3. st. 4. dice:
  \begin{verse}
    E dicon spesso altrui: Ti veddi a Lucca.
  \end{verse}
\end{description}
\section{Stanza LIX.}
\begin{ottave}
\flagverse{58}Tu puoi far conto allor d'averla vista,\\
Perché mentr'ella beve un'acqua tale,\\
Ti fuggirà in un subito di vista,\\
E tu resterai quivi uno stivale:\\
Se tu non l'ubbidisci, ella ch'è trista,\\
Vedendo che il pregar, e il dir non vale,\\
Intorno ti farà per questo fine\\
Un million di forche, e di moine.
\end{ottave}

\begin{ottave}
\flagverse{59}E se di compiacerla poi ricusi,\\
Dirà, che tu buon Cavalier non sia,\\
Mentre conforme all'obbligo non usi\\
Servitù con le Dame, e cortesia.\\
Ma lascia dire, e tien gli orecchi chiusi,\\
Non ti piccar di ciò, sta pure al quia,\\
Gracchi a sua posta, tu non le dar bere,\\
Acciò non fugga; e poi ti sia il dovere.
\end{ottave}

\begin{ottave}
\flagverse{60}Con questa, che sarà fatta a pennello, \\
Come te cerchi, lenerai dal cuore \\
Ogni doglia, ogni affanno al tuo fratello, \\
Ed io ten' entro già mallevadore. \\
Vientene dunque meco, e sia in cervello.\\
Cammina piano, e fa poco romore,\\
Che se e' ci sente a sorte, scuopre il cane,\\
Non occor' altro; noi habbiam fatto il pane.
\end{ottave}

Pigolone seguita a narrar la favola del Cocomero, ed instruito Brunetto di come
si debba contenere, perché la fanciulla non gli scappi, s'avvia con esso alla
volta del giardino di Magorto.

\begin{description}
\item[TU puoi far conto d'haverla vista] Ti puoi dare a credere d' haverla veduta quanto
  tu l'hai a vedere, perché non la rivedrai più.
\item[RESTERAI uno stivale] Resterai beffato, Retterai uno scimunito. Vedi sopra
  C. 4. st. 10. I Greci dissero \textit{Bagas constitisti}, da un tale detto \textit{Baga}, o pure \textit{Bagoas}
  nome da Eunuco; che fu un' huomo insipidissimo; Donde poi noi diciamo Baggeo,
  o Baggiano, a un'huomo scimunito se non forse da Baseo, e da \textit{Babbano}; o
  da Baggiano sorta di fave maggjore dell'altre.
\item[UN millione di forche, e di moine] Vna quantità grandissima di finte carezze, e
  lezzi, i Latini dissero \textit{blanditiae}, Ed in questo proposito tanto è dire \textit{far le forche},
  quanto \textit{lezzi}, quanto \textit{moine}, significando tutte tre una sorta di lufinghe fatte con
  gesti, o con parole, e sono quasi lo stesso che adulazione; perché ancor le
  \textit{moine}, ec, son atti, gesti, e discorsi, i quali contengono, se non false lodi, come
  contiene l'adulazione, almeno false dimostrazioni d'affetto affine di compiacere,
  e di acquistar la grazia di colui, a cui si parla, e queste son proprie di
  fanciulli, e di femmine, e l'adulazione e conveniente ad ogni sorta di persone,
  ma è sempre indizio d'animo vile, ed effeminato. Il Landino nell'esposizione
  a Dante Inf, C. 18. dice, che gli adulatori in lingua Fiorentina si dicono \textit{moinieri};
  Ma questa voce non si [segue] dicendo in oggi, ne avendo autorità di Scrittore nell'antico,
  mi fa credere, che il Landino la derivasse a capriccio dalla voce Fiorentina
  \textit{Moine} non trovando parola corrispondente alla Latina \textit{Adulatores}. Il Casa
  nel Galateo volendo mettere in volgare il Latino \textit{adulari}, lo espresse colla parola
  \textit{Piaggiare}. Il Bini in lode del mal Francese dice:
  \begin{verse}
    \backspace Io non roppi già mai; ne corsi lancia
    Ma chi mi va con sì fatte moine,
    Vorrei potergli sfondolar la pancia.
  \end{verse}
  La Stor. di Semifonte Trattato 4. \textit{Quand'altri ha offeso un supremo, non è da
    fidarsi di lui, ne delle sue astute moine, e lusinghe}.
\item[NON ti piccare] Non t'offendere, non t'adirare; Non entrare in gara; Non
  ti stimare ingiuriato. Vedi sopra \cstan[3]{20}. Tanto il Franzese \textit{Piquer} quanto
  lo Spagnuolo \textit{Picar} voglion dire \textit{Pugnere}; forse da \textit{Picca}, \textit{Asta}, il ferir della
  quale Omero appella \textit{nyttein}, cioè \textit{pungere}. Vino \textit{piccante} è quel vino, che par che
  morda, e che punga, quele è il \textit{brusco}, e l'\textit{amaro}, di cui si dice in un proverbio;
  \textit{Tienlo caro}. Il Persiani:
  \begin{verse}
    \makebox[4em]{}Va menati l'agresto,
    Cervellaccio pestato per Lambiceo,
    Che 'l tuo mordente ha trove poco appicco.
    \makebox[4em]{}Di questo io non mi picco,
    Che s'io non ho la nobiltà a bigonce,
    Mi basta di non esser d'undici once.
    \makebox[12em]{}\textup{(cioè bastardo)}
  \end{verse}
\item[PICCARSI] Vuol dir anche persuadersi, o darsi a creder d'esser eccellente
  in una cosa, come \textit{piccarsi di bravo, di bello, di dotto}, ec, e vale quanto esser
  ambizioso, o haver ambizione.
\item[STA al quia] Sta sodo: Non badare a quel che ella dice; e non ti lasciare
  svolgere, o persuadere a darle da bere. Dante. \textit{State contenti, umana gente,
    al quia}.
\item[GRACCHI a sua posta] Gridi, cicali, esclami pure quant'ella vuole; lasciala
  dire, lasciala cantare. Quand'uno vuol qualcosa da un'altro, ed attende a
  domandargliela, e colui non gliela vuol dare, suol replicare a i detti di quello:
  \textit{Gracchia, gracchia}; quasi dica: Tanto mi muove il tuo dire, quanto il gracchiare
  d'una cornacchia. Vedi sotto \cstan[8]{64}.
\item[TI stia il dovere] Ti succeda quel che tu meriti.
\item[SARÀ fatta a pennello] Cioè sarà similissima, ed appunto come quella.
\item[T'entro Mallevadore] Te ne assicuro. Ti fo sicurtà, che leverai di testa al tuo
  Fratello questa frenesia. \textit{Mallevadore} è il Latino \textit{Fideiussor}, quasi \textit{affidatore},
  \textit{assicuratore}; detto \textit{Mallevadore} secondo il Menagio\footnote{Egidio Menagio, o Gilles Ménage, (Angers, 15 agosto 1613 --- Parigi, 23 luglio 1692) poeta, saggista, grammatico, critico letterario, autore del primo dizionario etimologico della lingua italiana (1669 e 1685). }, \textit{dal levare in alto la mano}; per
  segno d'assicurazione. Lo Spagnuolo lo chiama \textit{Fiador}, la qual voce in un'antico
  Volgarizzamento Toscano manoscritto delle Vite di Plutarco tradotte dalla
  lingua Aragonese, restò senza interpretazione insieme con alcune altre, il che
  seguiva in queste tali traduzioni, o per vezzo del traduttore, o per infingardaggine,
  o perché non ne sapesse più là. \textit{Cato non volle il diposito, ma stette fiador per
    tutti}.
\item[NOI habbiam fatto il pane] Noi habbiam dato nel laccio. Noi habbiamo havuto
  la disgrazia senza rimedio. Diciamo ancora; \textit{Voi habbiam fritto}, Vedi
  sotto \cstan[8]{54}.
\end{description}
\section{STANZA LXI. --- LXIII.}
\begin{ottave}
\flagverse{61}Zitti dunque, nessun parli, o risponda:\\
Andiamo che e's'ha a ir poco lontano.\\
Così va innanzi, e l'altro lo seconda,\\
O il servitor lo segue anch'ei piano piano,\\
Ma quel Demonio, che va, sempr'in ronda,\\
Gli sente, e gli vuol vincer della mano,\\
Perché gli aspetta, e il vecchio c'alla siepe\\
Vien primo chiappa su, come dir: pepe.
\end{ottave}

\begin{ottave}
\flagverse{62}A casa lo strascina e te lo ficca\\
N'un sacco, e con la corda ve lo serra,\\
E fatto questo a un canapo l'appicca,\\
Che vien dal palco giù vicino a terra;\\
E per pigliar il resto della cricca,\\
Esce poi fuora, ma nel fatt'egli erra,\\
Che quand'ei prese quello, gli altri due\\
Ad aspettarlo havuto havrian del bue.
\end{ottave}

\begin{ottave}
\flagverse{63}Ed oggimai si trovano in franchigia, \\
Sicché Magorto quivi ne rimane\\
Un bel minchione, e n'è tanto in valigia ,\\
Che ne manco daria la pace a un cane; \\
Sfogarsi intende, e a quella veste bigia\\
Vuole un po meglio scardassar le lane,\\
Perciò su verso il bosco col pennato\\
A tagliar un Querciuol va difilato.
\end{ottave}


Pigolone esortando i compagni a far romore,s'avvia con essi verso il giardino,
ma appena giunsero alla siepe, che Magorto gli sentì, e prese il Vecchio,
che era più vicino alla detta siepe, e condottolo a casa lo serrò in un sacco, e
legatolo al palco, tornò per pigliare il resto, ma non gli trovando, sen'andò
al bosco per fare un buon bastone, col quale haveva in animo di bastonare Pigolone.

\begin{description}
\item[ZITTI] Cheti, Vedi sopra \cstan[1]{10}.
\item[LO seconda] Gli va dietco: Lo seguita, Petr. Canz. 8.
  \begin{verse}
    Ed un gran vecchio il secondava appresso.
  \end{verse}
\item[E spesso in ronda] Gira per l'orto facendo la guardia. \textit{Ronda} dal Lat, \textit{rotundus};
  dal quale è fatto il Franzese \textit{Rond} ritondo.
\item[GLI vuol vincer della mano] Vuole esser più diligente, e più lesto di loro; gli
  vuol prevenire. E traslato da quei giuochi di dadi, ec, ne i quali il punto uguale
  non è pace, ma vince quello, che è il primo a tirare; per esempio, io sono il
  primo a tirare, e scuopro sei; tira il secondo, e parimente scuopre sei, e se bene
  il punto è uguale, vinco io, che sono stato il primo a tirare; e questo si dice
  \textit{Vincer della mano}, perché colui, che è il primo a tirare, si dice \textit{haver la mano}. E
  tanto basta ai nostro proposito, se bene moiti altri giuochi di carte danno questo
  privilegio alla mano.
\item[SIEPE] Chiudenda, o riparo fatto di pruni, e d'altri sterpi agli orti, ed a
  i campi. E' voce latina. Franco Sacc. Nov. 83. \textit{E giungendo dove era la vigna,
    questa era molto affossara, e con una buona siepe}.
\item[CHIAPPA su, come di pepe] Piglia subito, e senza contrasto, o fatica alcuna.
  Credo, che questo dettato sia corrotto, e che si debba dire: \textit{Come dir: pepe},
  che è facilissimo a profferirsi, come tutto labiale, e di sillaba raddoppiata; e che
  da questa facilità si cavi il significato di facilità in dire, o fare una tal cosa, perché
  a dire; \textit{Come di pepe} non ci so trovar significato, o sale alcuno. \textit{Chiappare}
  dal Lat. \textit{capere}. Da \textit{Arripere} fece il Bocce. \textit{Arrapare}, Nella Lettera del
  medesimo scrittay a Messer Francesco Priore di Santo Appostolo, \textit{E finalmente con
    più largo parlare scrivi, che io non doveva così subito il partire, anzi la fuga dal tuo
    Mecenate arrapare}. Volle esprimere il Lat. \textit{fugam arripere} con dare a quel verbo
  una terminazione Toscana. Così \textit{strappare} abbiamo fermato da \textit{extra}, e \textit{rapere}.
\item[STRASCINARE] Strascicare un materiale per terra senza sollevarlo,o porlo
  sopra veicoli. Lat, \textit{Trabere}.
\item[FICCARE] Vuol dir mettere una cosa in un recipiente con violenza dal Latino \textit{figere}.
\item[CRICCA] S'intende conversazione, o compagnia di più persone: metaforico
  da quei giuochi di carte, ne i quali tre figure uguail insieme si chiamano \textit{cricca},
  come tre Re, tre Dame, o tre Fanti.
\item[HAVRIANO havuto del bue] Havrebbono havuto poco giudizio, poco avvedimento.
\item[SI trovano in franchigia] Si trovano in sicuro, in luogo, dove non temono esser
  presi; che franchigia intendesi un luogo immune per privilegio di Chiese o di
  Principi, Lat. \textit{asylum}, che pure alcuni Toscani dicono \textit{asilo}, ed altri più
  bramosi di voci nuove, dallo Spagnuolo dicono \textit{amparo}.
\item[RIMANE un bel minchone] Riman burlato, riman beffato. Vedi sopra C. 4.
  stan. 15. si dice ancora \textit{restare uno stivale} sopra in \cstan{58}.
\item[È in valigia] È in collera. Si dice anche \textit{in bigoncia, in bugnolo, nel bugnolone,
  nel gabbione}, ec, come habbiamo notato sopra \cstan[6]{41}. E \textit{valigia} si chiama
  un'arnese di quoio, entro al quale si mettono cose necessarie per la propria persona,
  quando si viaggia, e s'adatta in sulla groppa dei cavallo, e quelli che
  vanno a piedi la portano in su le reni, ma questa propriamente si dice \textit{Zaino}.
\item[NON darebbe la pace a un cane] Non darebbe la pace a Veruno; cioè tale è la
  stizza, o collera, che egli ha, che se gli venisse avanti un'amico, lo tratterebbe
  come nimico, perché la rabbia gli ha fatto perdere il conoscimento, Si dice
  \textit{un cane}, e non un'altro animale, perché l'uso nostro è di dire: \textit{Non ha cane, che
    lo guardi in viso}; \textit{Non ha cane che gli voglia bene}; \textit{non ha cane che lo soccorra, o
    l'aiuti}, e questo perché il cane è simbolo della fedeltà, ne si trova animale più familiare,
  ed amico dell'huomo, che il cane; e pero dovendosi pigliare un'animale
  vicino all'humanità, e prossimo al ragionevole; nel presente luogo, come ne
  i sopraddetti proverbi, pigliamo il cane.
\item[SFOGARSI intende] Si vuol cavar la rabbia. Vuole \textit{sfogar} l'ira; dare esito
  all'ira, come si fa del fuoco, del fummo, che gli si da apertura, perché esali.
\item[VUOLE un po meglio scardassar la lana]\items{A quella veste bigia} Scardassar la lana
  vuol dir battere, e pettinar la lana; con denti di fil di ferro auncinati detti
  ancche cardi (dalla similitudine del cardo erba spinosa) raffinare la lana, acciocché
  si possa filare. Vedi sopra \cstan[3]{60}. e per metafora significa bastonare uno;
  e però qui dicendo, \textit{vuole scardassare, ec}. intende Vuol battonare Pigolone, e
  torna bene l'equivoco, perché par che voglia dire rilavorare, e di nuovo cardare
  la lana, con la quale è fatta la veste di Pigolone. Il Pulci nel Morgante:
  \begin{verse}
    Adatterà il battaglio ancor dal Cielo
    In qualche modo a scardassargli il pelo,
  \end{verse}
\item[PENNATO] Coltellone adunco, il quale serve per potar le viti, appellato
  forse così da quella cresta, o penna tagliente, che ha nella parte di sopra.
  Nonio Marcello alla Voce \textit{Bipennis} dice così: \textit{Bipennis manifestum est id dici, quod ex
    utraque parte sit acutum. Nam nonnulli gubernaculorum partes tenuiores ad  hanc
    simulitudinem pinnas vocant eleganter}. \textit{Pennato} ancora è epiteto, che è stato dato in
  Latino a volatili. Onde scherzando sull'equivoco, disse il Bocc. Gior. 6. Nov.
  18. \textit{I vidi volare i pennati, cosa incredibile a chi non gli avesse veduti}. E noi avendo
  a raccontare gualche novella, per renderla più credibile, facciamo il caso esser
  seguito nell'antico assai, quando gli huomini eran più semplici, e \textit{Nel tempo
    che volavano i pennati}. Palladio de Re rustica tit. 43. discorrendo de' ferramenti
  de Contadini vi nomina \textit{i pennati}, e gli chiama \textit{falces a tergo acutas, atque lunatas}.
\item[DIFILATO] È lo stesso che Andar di vela, di filo, addirittura. Detto sopra
  \cstan[6]{10}, Vedi sopra in \cstan{5}.
\end{description}
\section{Stanza LXIV. ---  LXVI.}
\begin{ottave}
\flagverse{64}Brunetto, che l'osserva di nascosto,\\
Vedutolo partire, entra nell'orto,\\
E corre a casa di veder disposto,\\
Quel ch'è del vecchio s'egli è vivo, o morto\\
Così chiuso in quel sacco il trova posto,\\
Che 'l poverin trovandosi a mal porto\\
E' trema, e stride, e par che giù pel gozzo\\
Egli habbia una carrucola da pozzo.
\end{ottave}

\begin{ottave}
\flagverse{65}Ed ei le corde al sacco a un tratto sciolte,\\
E fatto quel meschino uscirne fuore,\\
Che lo ringraria, e bacia mille volte,\\
E fa un falto poi per quell'amore.\\
Vi mette il can che guarda le ricolte,\\
Dandogli aiuto, ed egli, e il servitore,\\
Poi con i piatti, e più vasi di terra\\
Due fiaschi di vin rosso, e lo riserra,
\end{ottave}

\begin{ottave}
\flagverse{66}E l'attacca alla fune in quella guisa,\\
Ch'egli era prima, e poi di quivi sfratta,\\
E del fatto crepando delle risa\\
Di nuovo con quegli altri si rimpiatta; \\
Quando Magorto in già viene a ricisa\\
Con una stanga in man cotanto fatta,\\
Perche è gli par mill' anni con quel tronco\\
Di far vedere altrui ch'ei non è monco,
\end{ottave}

Brunetto, che stava nascosto a osservare, veduto partirsi Magorto, corse alla
casa di esso, e trovato il vecchio nel sacco lo cavò, e vi messe dentro il cane con
alcuni vasi di terra, e due fiaschi di vino, e rattaccatolo come stava prima si nascose
con gli altri, perché vedde venir Magorto con una grande stanga in mano.
\begin{description}
\item[POVERINO]infelice, ' paroia di commiserazione,come meschino, e simili.
\item[YANDOS! 4 mal porto] Trovandoli a cattivi termini.

arrucolada pozzo, Carrucola e una catiecta di legno, e tal volta di fer-

alla quale e impernata una gircila scanalaca, e (ope'a tal girella s'a-

, o catena per tirar fu pesi con facilita, e questa carrucola si tiene co-

ente appiccata al pozzo per tirar fu acqua, ed il moto, che fa cal girella

ta cagiona per lo più strepito, al quale il Poeta atiomiglia i sospiri,
; 'igolone.
ae SFA fais, per quell amore. E' un detto faceto, col quale s' esprime la gran-
a, e contento d'alcuno: E tal detto viene da quzi Ciechi, che per
i Popolo fanno nelle piazze giocolare i cani, e fra gli altri giuochi gli

s e al bastone con dire: fa un falto per amor d' un pane, ed il cane tutto

» o per il contrario dicendogli; /alta per una mano di bastonate, il ca-

ein atto di mordere, e non saita; ed il termine per quell'amore figati-

lazione, O in riguardo; come Lo fo la tal cosa per amor tuo, s! in-

bh tende Io la fo in riguardo, o a contemplazione tua per l'amore ch' 10 ti porwo,

 SERATT-A, Vedi sopra \cstan[5]{13}.

flere f delle rifa. Rider gagliardamente. Rider come fece Margutte, che

baenpp:s secondo che favoleggia il Pulci nel suo Morgante; Ll' verbo

a altro vuol dire allentarsi gi' inceltini, vale anche qunato /eoppiare,

parities pur si dice: Scoppiare, e morire dalle rifa, Ed \& quel re quati che

“habbiamo decto sopra C, 3. ttan. 65. Li Pulci nella Beca dice:

Petty wit ' La fet nel letto, e crepi dake rifa.

st enone Sitorna a nascondere. Vedi sopra \cstan[20]{60}. e sotto C.9.

bis he fa cht ek s* appiartd miffer gli denti.

in era i emi a Trattaco —— dice: \textit{Quejte cose ho cavate da un libro
bro del Comune, che fu impiattato da uno de' Buonhuomini, e poi portato via}.
\item[A ricisa] Senz'intermissione; senza fermarsi, a precipizio; è lo stesso che
  \textit{difilato} detto poco sopra Octava 63. antecedente. Il Pulci nella Beca dice:
  \begin{verse}
    E s'io mi metto a cantar a ricisa.
  \end{verse}
\item[COTANTO fatta] Grossa in questa guisa. Vedi sopra \cstan[5]{24}. e C. 10.
  stan. 36.
\item[FAR veder, ch' ei non è monco] Far conoscere ch'egli ha le mani; o che egli
  non ha mancamento alle braccia. \textit{Monco} vuol dir uno che ha manco una, o
  tutte due le mani. Lat. \textit{Mancus},
\end{description}
\section{STANZA LXVIIL.}
\begin{ottave}
\flagverse{67}Arriva in casa, e sbracciasi, e si mette \\
(Serrato l'uscio) con il sue randello \\
Sopr'a quel sacco a far le sue vendette, \\
Suonando quant'ei può sodo a martello.\\
Il Romito che stava alle velette, \\
Perché l'uscio ha di fuora il chiavistello \\
Andò (benché tremando, e con spavento \\
Che havea di lui) e ve lo serrò drento.
\end{ottave}

\begin{ottave}
\flagverse{68}Ed ei, ch'è in sulle furie non vi bada,\\
Che infin ch' ei non si sfoga non ha posa.\\
Sta intanto il vecchio all'uscio, fermo in strada\\
Ad origliare per udir qualcosa,\\
E sente dire: O lecca peverada,\\
Carne stantia, barba piattolosa,\\
Ribaldo, Santinfizza, e gabba Dei,\\
C'a quel d'altri pon cinque, e levi sei.
\end{ottave}

\begin{ottave}
\flagverse{69}Guardate qui la gatta di Masino, \\
Che riprendeva il vizio, ed il peccato, \\
Se il monello ha le man fatte a uncino \\
Per gire a sgraffignar pel vicinato? \\
Ma quel c'hai tolto a me, ladro assassino,\\
Non dubitar ti costerà salato,\\
Che tante volte al pozzo va la secchia,\\
Ch'ella vi lascia il manico, o l'orecchia.
\end{ottave}


Magorto, arrivato a casa, si messe a bastonar quel sacco, credendo che vi
fusse dentro Pigolone; Ma questo essendo uscito di casa messe il chiavistello per
di fuori alla porta, e fermatosi alquanto quivi, senti che Magorto bastonando il
sacco gli diceva una mano d'improperj.

\begin{description}
\item[SBRACCIARSI] Vuol dire Denudarsi il braccio da mezzo in giù verso la
  mano come accennammo sopra in \cstan{19}. E \textit{sbracciarsi};
  metaforicamente parlando vuol dire Impiegare ogni sua forza, diligenza, ed attenzione
  in un'affare. Lat, \textit{manibus, pedibusque eniti}.
\item[SVONANDO a martello] Cioè bastonando. Suonar' a martello si dice quando
  la campana suona a rintocchi, come fa il martello sull'ancudine, si che si fa
  quando si vuol ragunare il popolo per li bisogni della Città. Il verbo \textit{suonare} è il
  Latino \textit{puslo}, e vale appretio di noi, come appresso i Latini per suonare, e per
  perquotere. Vedi sopra \cstan[3]{7}.
\item[STAVA alle velette] Stava osservando. Veletta, o vedetta diciamo quel soldato,
  che sta in sulle mura d'una Città, o Fortezza a far la guardia detto più
  comunemente \textit{sentinella}, ed il luogo dove sta detto soldato si dice \textit{veletta}, o \textit{vedetta}.
  Stimo che sia traslato da i Marinari, che tengono la detta guardia in cima all'albero
  della nave, e dicono metter l'huomo alla vela, o veletta forse da qualche
  piccola vela, che sia in quel luogo. Tarcagnotta\footnote{Giovanni Tarcagnota (Gaeta, 1490, o 1506 --- Ancona, o Gaeta, o Napoli, 1566), storico.} Stor. lib, 5. p. 3. Tomo 1. dice:
  \textit{Partitosi però il Priore Strozzi da Marsilia con 23. Galere, ed una galeotta, poste le
    elette in mare lo venne ad incontrare}. Dal che ficava che si chiamino velette alcune
  barche, le quali camminino avanti a una armata con huomini per sentinelle,
  opure da vedere \textit{vedetta} e poi corrottamente \textit{veletta}. Si come da \textit{specio} antico
  Latino significante Io veggio, si fece \textit{specula} luogo eminente che signoreggi
  molto paese. Ma sia come si sia basta il sapere, che stare alle velette vuol
  dire Stare a osservare.
\item[È in su le furie] E'colmo d' ira.
\item[ORIGLIARE] Star in orecchi, Star a sentire, e vedere con attenzione, e di
  nascosto. Franzese \textit{oreillier}. Spagn. \textit{otear} forse dal Gr, \textit{Ota}, orecchie, che il
  Franciolini spiega: \textit{spiare, e guardare da luogo alto, come fanno le sentinelle}.
\item[PEVERADA] Brodo di carne, o d'altro, E \textit{lecca peverada} vuol dir Brodaio,
  il che significa porco, perché il porco mangia volentieri ogni sorta di broda.
  Var. St. Fior. lib. 14. dice: \textit{Gli diede una minestrina bollita, cotta in peverada di
    pollo}. Detta \textit{Peverada} dal \textit{Pevere}, cioè dal pepe, che per dar sapore si metteva
  su le minestre, come fu da altri dottamente osservato.
\item[CARNE stantia] Carnaccia vecchia, e frolla. Vedi sopra \cstan[3]{24}. e 54.
\item[BARBA piattolosa] Termine ingiurioso per un vecchio; e vuol dire barba schifa,
  e piena di pidocchi, e d'altre lordure.
\item[SANTINFIZZA] Ipocrito; de i quali a bastanza s'è detto altrove; E per
  satinfizza s'intendono certi Torcicolli, che stanno tutto il giorno d'avanti a
  una immagine d'un Santo, perché si creda che essi facciano orazione.
\item[GABBADEI] Rinnegato. Uno che gabba, cioè inganna le Deità, adorandone
  oggi una, e domani un'altra, rinnegando la prima. Se bene \textit{Deus non
    irridetur}. Si dice ancora \textit{Gabbasanti}.
\item[PON cinque, e leva sei] Vuol dire Tu sei ladro; perché ponendo cinque dita
  della mano, fai il numero di sei con aggiugnere alle cinque dita la roba, che
  porti via. Plauto disse: \textit{Trium literarum Homo}, cioè \textit{fur}. Habbiamo diversi
  modi di dire copertamente Ladro, come \textit{Sgraffignare}. \textit{Havere le mani a oncini},
  che si vedono nella presente Ottava 69. \textit{Bestemmiar con le mani}, \textit{Andar a Carpi,
    e a Borselli}. \textit{Far il Lanzo} (che in lingua Ianadattica vuol dire Ladro) \textit{giuocar, o
    lavorar di mano}, e simili.
\item[LA gatta di Masino] Questa fingeva d'esser morta, e non era, e però vuol
  dire huomo finto. Huomo che fa il semplice, e non è. Lat, \textit{Lepus dormiens}.
  \textit{Tenere gli occhi aperti}, \textit{haver l'occhio}, ed \textit{aprir l'occhio} vuol dire andar cauto nell'operare:
  e perché tanto la lepre, che il gatto tengono gli occhi aperti anche dormendo,
  servono a i Latini, ed a noi per esprimer un'huomo vigilante, ed avveduto,
  e che mostri di non essere. Vedi sopra \cstan[1]{19}.
\item[MONELLO] Così chiamiamo quei guidoni, che per Firenze battono marina,
  come s'è detto sopra \cstan[4]{8}. Siccome Guidone di nome proprio si è
  fatto appellativo, così forse anche Monello, in principio diminutivo di \textit{Mone},
  accorciato dal nome proprio di \textit{Simone} è venuto a significare una tal razza di
  persone.
\item[ASSASSINO] Vuol dir ladro di strada, ma qui è detto in vece di furbo, o
  briccone, e può anche intendersi ladro di strada.
\item[NON dubitar ti costerà salato] Sta sicuro, che ti ha da costare assai, o che ne
  pagherai un gran fio.
\item[TANTO va la secchia al pozzo, ec.] Tante volte si torna a fare un male, che
  una volta vi si riman colto. Una volta fa per molte; e diciamo ancora; \textit{Tante
  volte va la gatta al lardo, che una volta vi lascia la zampa}. Lat. \textit{Exitus legem saepe
    violantium malus est}, ed orecchie della secchia diciamo quelle due parti di essa
  forate, nelle quali è infilato il manico di essa secchia.
\end{description}
\section{Stanza LXX. \& LXXI.}
\begin{ottave}
\flagverse{70}Poi sente, ch'egli dopo una gran bibbia \\
D'ingiurie dà nel facco una percsssa, \\
Che tutte le stoviglie spezza, e tribbia, \\
E ch'ei diceva; Horsu gli ho rotto l'ossa; \\
E che di nuovo un'altro ne raffibbia,\\
E che (facendo il vin la terra rossa) \\
Soggiunge: O quanto sangue ha nelle vene! \\
Questo ghiottone, a me, beeva bene.
\end{ottave}

\begin{ottave}
\flagverse{71}Ben ch'ei creda finita haver la festa\\
Tira di nuovo, e dà vicino al fondo,\\
Ed il suo cane acchiappa in su la testa\\
Che fa urli che van nell'altro mondo\\
Ond'egli stupefatto assai ne resta\\
Dicendo: Qui è quand'io mi confondo\\
Se tutt'il sangue egli ha di già versato\\
Come a gridar può egli haver più fiato?
\end{ottave}

Seguitando Magorto a dire ingiurie, da una bastonata in sul sacco, e rompe
i piatti, e fa versare il vino, e credendolo il sangue di Pigolone resta maravigliato,
che ne possa haver tanto; e replicando un'altra bastonata, coglie in sul capo
il cane; il quale cominciò a urlare, ed ei credendo, che fussero strida di
Pigolone, strabilisce e non resta capace, che egli possa haver più forza di fare quelle
strida, mentre ha versato tutto il sangue.
\begin{description}
\item[DOPPO una gran bibbia] Dopo una lunga diceria, o filastrocca; quasi dica:
  Dopo haver dette tante ingiurie, che farebbono un gran libro, da \textit{Biblia} Greco
  Latino, che vuol dir \textit{libri}; E se bene la voce Bibbia oggi comunemente e intesa
  per il libro della Sagra Scrittura, cuctavia noi la pigliamo ancora ne i casi, come
  il presente nel detto senso di libro, o di lettera, o di discorso lungo, come pare
  che la pigliassero gli antichi secondo Herodoto lib, 1. dove dice: \textit{Harpagum in
    clusisse, leporis ventri biblion ad Cyrum}. Se bene qui è \textit{Viglietto}, \textit{lettera}. Dal
  poema d' Omero intitolato l'Iliade, il quale è d' una prodigiosa quantità di versi,
  come quelli, che ascendono al numero di quindicimila settecento ottantatre; una
  gran moltitudine di cose, o di parole, dissero i Latini \textit{Ilias}, o \textit{Iliades}. Propezio
  lib. 2. elegia 1.
  \begin{verse}
    \backspace Tunc vero longas condimus Iliadas.
    Seu quicquid fecit, sine est quodcumque locura
    Maxima de nihilo nascitur historia.
  \end{verse}
\item[RAFFIBBIA] Replica. Trasiato dal congingner con fibbia bottoni, e simili
  il che si dice \textit{Affibbiare}, Vedi sopra C, 2. st. 81.
\item[STOVIGLIA] Intendiamo ogni sorta di piatti, e vasellami di terra per uso di
  cucina. Il Ferrari, \textit{Stoviglie, Fictilia, vascula, \& frivola. Vandenam, nondum
    comperi}. Io stimo che sia parola storpiata dalla Latina. Utensilia, Cresc, 12:12.
  E molti altei arnesi, e \textit{stovigli} di bisogno. Pallad. volgarizzato lib. 1. tit. 6 Fabbri
  da far ferramenti, e di legname, e di sovigli da vino, da lavorare, \textit{e da usare}.
  Questo ultimo non è nel Launo, ed è aggiunto nella traduzione per impiegare la
  voce \textit{stovigli}.
\item[TRIBBIARE] Propriamente vuol dire Batter il grano in sull'aia dal Latino
  \textit{Tribula tribulae}, o \textit{tribulum tribuli}, che vuol dire una specie di carro, col quale si
  squoteva il grano in su l'aia, come si cava da Colum. '\libcap[2]{21}. \textit{Si pauca iuga
    sunt adijcere Tribulum, \& traham possis}, e Varr. lib. 1. C, 25. E \textit{spicis in area
    excuti grana iuvencis iunctis, \& tribula}. E questo dal Greco \textit{tribein} pestare, tritare.
  Latino \textit{terere}, o da \textit{thlibein} schiacciare, dal qual verbo viene il Latino \textit{tribulatio},
  travaglio detto anche da' Santi Padri \textit{pressura}.
\item[A ME] Questo termine significa A mio giudizio; Secondo me. Secondo il
  mio parere, o intendimento; e per lo più si dice replicatamente \textit{a mè a mè}, \textit{Quanto a me},
  cioè per quanto io giudico i Franzesi \textit{Quant' a moi}, I Greci similmente
\textit{cat'eme}, cioè secondo me, secondo il mio giudizio.
\item[CREDE haver finita la festa] Crede haver terminato il negozio, cioè d'haver'
  ammazzato Pigolone. Similitudine tratta dalla solennità, colla quale son fatti
  morire quei, che si giustiziano.
\item[ACCHIAPPA] Coglie: perché se bene acchiappare vuol dir pigliare uno con
  fraude, e violenza, ci serve per esprimere \textit{colpir bene}. Latino \textit{Certo ictu
    assequi}. Spagnuolo, \textit{acertar}, Vedi C. 2. st. 41.
\item[STUPEFATTO] Rimasto stupido per la meraviglia grande. Latino \textit{obstupefactus}.
\end{description}
\section{STANZA LXXII. STANZA LXXIIL}
\begin{ottave}
\flagverse{72}Brunetto in questo mentre col suo fante \\
Havea di già scorrendo pel giardino \\
Il luogo ritrovato, e quelle piante \\
Ov'è colei, che chiede il suo Nardino, \\
E già l'ha trata fuor bell'e galante, \\
Che non si vedde mai il più bel sennino, \\
E con un suo bocchin da sciorre aghetti\\
Chiede da ber, ma non già sel' aspetti.
\end{ottave}

\begin{ottave}
\flagverse{73}Perch'ei del certo in quanto a contentarla\\
Non ci ha ne meno un minimo pensiero,\\
E però quante volte ella ne parla,\\
Muta discorso, e la riduce al zero;\\
Ma perch'ella è mozzina, e con la ciarla\\
Le Monache trarria del Monastero,\\
Vede, che s'ella bada troppo a dire\\
Si lascerebbe forse convertire.
\end{ottave}

\begin{ottave}
\flagverse{74}Però per non cadere in questo errore\\
La piglia a un tratto, e se la porta in strada, \\
Ed al vecchio fa dir pel servitore, \\
Che più tempo non è di star' a bada, \\
E ch'ei ne venga ch'ei l'aspetta fuore,\\
Acciò con essi anch'egli se ne vada,\\
Che lì non vuol lasciarlo nelle peste,\\
Ma condurlo al paese alle lor feste.
\end{ottave}


Mentre che Magorto si studia a bastonare, il savio Brunetto col servitore era
andato nell'orto, ed havea trovato il Cocomero, e tagliatolo n'era uscita la
fanciulla, che egli cercava, la quale si messe a pregarlo, che egli l'empiesse la
tazza, ma ei non volle contentarla, anzi la prese, e la porto in strada, e mandò
il servidore a chiamar Pigolone per condurlo seco alle nozze di Nardino.
\begin{description}
\item[FANTE] Si dice il servitore; dall'intero infanre,si come in Latino Puer significa
  servo, da noi detto anche garzone, se ben Fante però comunemente vuol dire
  soldato a piede, perché ne' tempi dell' Imperio baflo, che la milizia cominciò a
  riputarsi più per la cavalleria, che per la soldatesca a piede; il pedone si venne a
  stimare come ministro., e servitore del Cavaliere; e perciò fu detto fante.
\item[SENNINO] E' una parola, che si dice per vezzi a una femmina bella, savia,
  e pulita, e che operi cen giudizio con senno, e con puntualità. Latino \textit{scita puella,
    scitula}.
\item[BOCCA dat feiorre agherti] Così diciamo di quelle femmine, le quali per parer
  belle tengono la bocca ferrata, e ridotta forzatamente pi Mretta del. suo naturale;
  ne muovono i labbri di come se gli sono accomodati allo specchio, onde
  par proprio, che habbiano la bocca accomodata a sciorre un ngdo co' denti.
  Aghetto è quello, che vedemmo sopra C. 2. st. 10.
\item[NON se l'aspetti] Non lo speri. Cioè non aspetti, che le dia bere. In
  Ispagnuolo \textit{esperar} è lo stesso, che aspettare.
\item[LA riduce al zero] La riduce al nulla; Zero quella figura d'abbaco, che per
  se stessa non rileva numero alcuno\footnote{Qui il Minucci fa una perdonabile confusione fra numero, e quantità.}, ed accompagnata, forma le decine, e ci serve
  per esprimer \textit{il nulla}.
\item[MOZZINA] Huomo astuto, tristo, e che fa il conto suo, ma s'intende nel
  genio maligno. Latino \textit{Vulpis reliquiae}. Questa voce vien forse da orecehi mozzi
  che così son segnati quei furbi, che meriterebbono le forche, ma per la tenera
  età non ne son capaci, sopra C. 6. st. 54., ed in questo C, st. 30. e credo questo,
  perché diciamo \textit{Mozzorecchi} in vece di \textit{mozzina} nello stesso significato.
\item[TRARRIA le Monache dal Monastero] Conseguirebbe l'impossibile con la
  sua industria, persuasiva, ed eloquenza. Diogene disse: \textit{Oratio non ex animo
    proficiscens, sed ad gratiam composita meleus est laquens, quod scilicet blande
    complectens hominem iugulet}.
\item[NON è tempo di star' a bada] Non è tempo di trattenersi. Non v'è tempo da
  perdere.
\item[LASCIAR' uno nelle peste] Abbandonar' uno nel pericolo. Uno fa' qualche
  insolenza, o mala creanza, e per non esser percosso fugge via, e lascia i compagni,
  e questo si dice \textit{lasciar nelle peste}, cioè nelle pedate, o nella strada, che è
  mancamenti ha fabbricato ai pericolo, colui che è fuggito; si pronunzia con la prima
  \letter{e} stretta a differenza di peste infermità, che si pronunzia con l'e larga, e
  però questa rima ha un poco di falsità, ma tollerabile, ed è ammessa.
\end{description}
\section{STANZA L\&XV. STANZA LXXVL}

\begin{ottave}
\flagverse{75}Così di là poi ti fer partita; \\
Ma più d'ogni altro allegra la fanciulla,\\
Perché non prima fu dell' orto uscita \\
C'ogni incanto, ogni voglia in lei s'annulla, \\
Anzi a i lor preghi in sul caval, salita,\\
Senza più ragionar di ber, ne nulla,\\
Va sempre innanzi agli altri un trar di mano \\
Fiera, e bizzarra come un Capitano.
\end{ottave}

\begin{ottave}
\flagverse{76}Brunetto si ridea di Pigolone\\
Perch'ei parea nel viso un fico vieto,\\
E menava a due gambe di spadone\\
Com'egli havesse havuto i Birri dreto.\\
E la donna diceva: Giambracone,\\
Che la duri; ed il vecchio mansueto,\\
Che si vedeva fatto il lor zimbello:\\
Dagli pur (rispondea) ch'egli è sassello..
\end{ottave}

Uscita che fu la fanciulla dell'orto cessò l'incantesimo, e la voglia del bere,
anzi con la maggior'allegria del mondo montò a cavallo scherzando, e motteggiando
il vecchio, il quale era ancor pallido per lo spavento havuto.
\begin{description}
\item[BIZZARRO] Vuol dir Iracondo, Stizzoso, o cosa simile, secondo che l'usarono
  gli antichi. Ma si piglia anche per spiritoso, e vivace, come è preso nel
  presente luogo. In Spagnuolo \textit{Bizarro} significa uno che vada bello, e superbo nel
  vestire. E similmente \textit{roba bizarra}, che i Franzesi direbbero \textit{bigearree}, vuol dire
  roba, cioè veste bellissima, varia, e pomposa, donde poi da noi si prende \textit{Bizarro}
  per capriccioso, strano, stravagante.
\item[FICO vieto] Fico annebbiato, o afato. Un fico, il quale al colore, e tenerezza
  par maturo, non è, ma dalla nebbia è ridotto giallo, come se fusse maturo:
  comparazione, che esprime assai bene la faccia gialla, e grinza di Pigolone.
  E l'epiteto \textit{Vieto} e proprio della carne fsalata, lardo, burro, e olio, quando
  per essere stantij, e corrotti mutano il colore, l'odore, ed il sapore.
\item[MENAR di spadone a due gambe] Fuggire; Correre. Spadone a due mani si
  chiama quella spada più grande delle spade comuni ordinarie, la quale s'adopra
  con ambe le mani, e per derisione di coloro, che, vantandosi di bravi, all'occasione
  poi fuggono, col solo dire; \textit{menò di Spadone}, o \textit{giocò di spadone}, s'intende a
  \textit{due gambe}, che vuol dir Fuggi. Vedi sotto C. 10, st. 3.
\item[COM'egli havesse havuto i Birri dreto] Detto usato per esprimere, che uno
  corra velocemente.
\item[GIAMBRACONE, che a duri] Dubito, che voi non siate per durare a camminare.
  Giambracone fu un matto, che sempre andava gridando: \textit{Che la duri}, e
  però quando noi veggiamo, che uno faccia un'operazione con grande attenzione,
  e che noi dubitiamo, che egli non sia per durare sogliamo dire \textit{Giambracone},
  e senza dire, \textit{che la duri} intendiamo: \textit{piaccia al Cielo, che egli continovi}, e così è
  comunemente inteso.
\item[FATTO il loro Zimbello] Divenuto lo scherzo. Zimbello, oltre al significato,
  che accennammo sopra C. 1. st. 59., vuol dire ancora quell'uccello, che si lega per
  un piede allato al boschetto de' paretai, o altri luoghi, dove si tende per pigliare
  uccelli, che tirandosi quella cordicella, che ha legata al piede si fa svolazzare
  per incitare gli altri uccelli a calarsi. Latino \textit{avis illex}, e dallo strapazzo, che
  tale uccello riceve diciamo \textit{zimbello} uno quando è burlato, beffato, e strapazato
  da tutti; nel qual senso e preso nel presente luogo; e sotto C. 9. st. 66.
\item[DAGLI ch'egli è sassello] Dagli ch'ei lo merita. Osservisi che il verbo Dare
  ne i casi come il presente, vale per continovare, seguitare, durare, ec. e con dire
  solamente \textit{dagli} senz'altra aggiunta s'intende \textit{seguita}; ma s'aggiunge \textit{ch'egli è
    sassello} per una certa vaghezza, e per un genio, e naturale inclinazione, che hanno
  i Fiorentini di parlar per proverbio, metafore, comparazioni, o similitudini;
  e forse è aggiunto per confondere, ed oscurare il detto, perché \textit{dare al sassello} vuol
  dir perquoterlo, e non vuol dir seguitare. Habbiamo due specie di tordi, cioè
  bottacci, e sasselli\footnote{Tordo bottaccio: \textit{Turdus philomelos} C.L.Brehm, 1831;\\Tordo sassello: \textit{Turdus iliacus} Linnaeus, 1758}; i primi son meno astuti, e più facili a lasciarsi pigliare, i secondi
  sono più astuti, e ad ogni poco di romore scappano, pero quando la notte col
  frugnuolo si scuoprono, si dice \textit{dagli con la ramata}, che \textit{questo e sassello}, che aspetta
  poco. In sustanza nel presente luogo vuol dire \textit{continuate, o seguitare, a burlarmi,
    beffarmi, e strapazarmi, ch'io lo merito}. Da questa astutezza del sassello si dice
  sassello a un'huomo, che sa il conto suo, ed esercita il suo sapere a vantaggio,
  pretendendo sempre più del giusto, e del dovere, avido di guadagnare, e tenace
  del suo più del conveniente.
\end{description}
\section{STANZA LXXVII. \& LXXVIII.}
\begin{ottave}
\flagverse{77}Così scherzando, com'io dico, in briglia \\
Ne vanno senza mai sentirsi stanchi, \\
E sempr'ognun più calda se la piglia, \\
Perché il timor gli spinge, e sprona i fianchi; \\
Perciò dopo haver fatte molte miglia,\\
E che lor parve un tratto d'esser franchi,\\
Tutti affannati per sì lunga via,\\
D'accordo si fermaro a un'Osteria.
\end{ottave}

\begin{ottave}
\flagverse{78}Dove il padron che intende fare a pasto\\
Trova gran roba per parer garbato,\\
Ch'ei tien che a far non habia troppo guasto,\\
Ma e' non sa che e' non hanno desinato;\\
Ben sen'accorge al fin ch'ei v'è rimasto,\\
Quando in sul desco poi non restò fiato,\\
E che quella per lui è una ricetta,\\
Che il guadagno va dietro alla cassetta.
\end{ottave}


Brunetto con la sua compagnia seguita allegramente il suo viaggio, camminando
per il timore, che hanno di Magorto, ma stimandosi già sicuri, si fermarono in
un'Osteria, dove mangiaron più di quello, che il padrone non s'aspettava.
\begin{description}
\item[SCHERZARE in briglia] Questo detto, che significa uno, che stando benissimo di
  facultà, e d'ogni commodo, non ostante si duole dello stato suo, è da noi usato
  anche per intender'uno, che stia allegramente, e scherzando senza considerare,
  che egli è in grandissimo pericolo; e così s'intende nel presente luogo, che coloro
  scherzano senza pensare al pericolo, nel quale sono, che Magorto arrivi loro
  addosso.
\item[OGNVNO se la piglia più calda] Ognuno se ne piglia maggior pensiero. Questo
  pigliarsela calda i Franzesi esprimono col verbo \textit{chaloir}, e noi \textit{calere}
  dal Lat. \textit{calere}; Boccaccio nel Poema in ottava rima intitolate il Teseida, cioè
  de' fatti di Teseo l. 2.
  \begin{verse}
    Onde lì fe nuova vision vedere;
    Perché di ritornar li fu in calere.
    \verseprefix{\textls[-60]{E appresso.}}Uscì d'Atene, ne li fu in calere,
    D' Ipolita amor dolce, e pudico.
  \end{verse}
  Spiegò la forza di questo verbo il Petrarca quando disse:
  \begin{verse}
    Ne dentro sento, ne di fuor gran caldo;
  \end{verse}
  Che fa come una spiegazione de' due versi immediate precedenti:
  \begin{verse}
    Ne del volgo mi cal', ne di fortuna;
    Ne di me molto, ne di cosa vile.
  \end{verse}
\item[GLI parve d'esser franchi] Parve loro d'esser in sicuro, ed esser liber da
  Magorto.
\item[FARE a pasto] Si dice quando l'Oste senza prezzare cosa per cosa di quello
  che mette in tavola vuole un tanto per persona, e mette in tavola quello che
  pare a lui.
\item[NON habbiano a far troppo guasto] Non habbiano a mangiar molto,
  L'Etrusco incognito dice.
  \begin{verse}
    Io ero sazio, e non sei troppo guasto.
  \end{verse}
  Il Berni in lode delle pesche:
  \begin{verse}
    \backspace Dioscoride, Plinio, e Teofrasto
    Non hanno scritto delle pesche bene
    Perché non ne facevan troppo guasto.
  \end{verse}
  Cioè non ne mangiavano molte, perché non gli piacevano.
\item[V'è rimasto] L'ha sgarrata. E' rimasto ingannato, come chi rimane alla
  trappola.
\item[NON vi resta fiato] Non vi resta nulla. Vedi sopra in questo C. stan.~77.
  Mattio Franzesi contr'alle sberrettate dice:
  \begin{verse}
    A cavarsela, e metter più di cento
    Volte per hora, il che non serve a fiato.
  \end{verse}
\item[IL guadagno va dietro alla cassetta] Cioè non si guadagna, ma più tosto si perde.
\end{description}

\section{Stanza LXXIX. ---  LXXXI.}
\begin{ottave}
\flagverse{79}Magorto intanto finalmente stracco \\
Di menar il randello a quel partito, \\
Sciolto, ed aperto havendo omai quel sacco \\
Per cucinar la carne del Romito, \\
Ed in quel cambio vistovi il suo bracco \\
Tra cocci, e vetri macolo, e basito, \\
Resta maravigliato in una forma \\
Ch'ei non sa s'ei sia desto, o s'ei si dorma.\\
\end{ottave}

\begin{ottave}
\flagverse{80}S'io percossi quel vecchio marivolo\\
Com'ho io fatto, disse, un canicidio?\\
So ch'io lo presi, e lo serrai qui solo,\\
Che gnun potea vedermi, o dar fastidio,\\
Non so s'io sono il Grasso Legnaiuolo\\
A queste metamorfofi d'Ovidio,\\
Che sono in ver meravigliose, e strane\\
Poi c'un Romito mi diventa un cane.
\end{ottave}

\begin{ottave}
\flagverse{81}Cane infelice! povero Melampo \\
Che  netto qua tenei quanto si scerne,\\
Chi più farà la guardia al mio bel campo \\
Adesso, che t'hai chiuse le lanterne? \\
Io ho una rabbia addosso ch'io avvampo\\
Con quel vecchiaccio barba d'Oloferne\\
Ch'al certo fatto m'ha così bel giuoco;\\
Che dubbio! metterei le man nel fuoco.
\end{ottave}


Stracco Magorto dal bastonar quel sacco lo spiccò dal palco, ed apertolo vi
trovò dentro il suo cane; e restando maravigliato, suppone che sia stato Pigolone,
che gli habbia fatta questa burla.

\begin{description}
\item[A QUEL partito] In quella guisa; in quella forma, in quella maniera.
\item[COCCI] Intendi frammenti di piatti, pentole, ed altri vasi di terra,
\item[MARIOLO] Ladro, giuntatore. E' voce Napoletana, ma già fatta Fiorentina.
\item[CHE gnun potea darmi fastidio] Che niuno poteva impedirmi, La voce gnuno
  per niuno, hoggi è usata solo da i nostri contadini.
\item[NON fo s'io sono il Grasso Legnaiuolo] Non so s'io mi sia diventato un'altro, il
  Grasso Legnaiuolo fu un Fiorentino, il quale fu tanto semplice, che gli fu dato
  a credere, che non era più lui ma diventato un'altro  e per questo tale fu messo
  prigione, dove alloppiato, e fatto dormire, quando si risentì, s'accordò a pagare
  le spese, e le cancellature per il preteso delitto, del quale fu assoluto, benché
  havesse confessato d'haverlo commesso come nuovo personaggio, e pagò il denaro
  un fratello di quello, che il Grasso si credeva d'essere, e durò in questa credenza
  qualche tempo; e fin che li suoi veri parenti lo fecero riconoscersi, e ritornare
  che egli era. La Novella pare a me, è stampata dietro alle cento Novelle
  antiche dell'edizione de' Giunti\footnote{\textit{Libro di Novelle et di bel Parlar Gentile}, a cura di Carlo Gualteruzzi, Firenze, nella Stamperia de i Giunti, 1582. Contiene 100 ``antiche novelle'', e quattro aggiunte di seguito, di questa del Grasso Legnaiuolo è la terza.\\Carlo Gualteruzzi (Fano, 5 marzo 1500 – Roma, 26 maggio 1577) letterato, filologo. }. Da costui diciamo il \textit{Grasso Legnaiuolo} per
  intendere un'huomo semplicissimo, e facile a creder ogni cosa, bench'ei sappia
  non esser vera, ed esser'imposiibile, che ella sia. Si dice ancora Calandrino, e
  Cappellaino, come accennammo sopra C. 5, st. 23.
\item[UN Romito mi diventa un cane] Se bene intende, che il Romito era diventato un
  cane, perché nel sacco trovò il cane, e vi haveva messo il Romito, si potrebbe
  anche dire, che intendesse parergli gran metamorfosi, che un Romito, cioè un'huomo
  da bene, diventi un cane, cioè uno scellerato.
\item[MAL chiuse le lanterne] Hai chiusi gli occhi; ed intende sei morto. Chiamansi
  anche gli occhi \textit{luccicanti} in lingua furbesca; e così li chiamò in un verso del suo
  Pataffio Brunetto Latini Maestro di Dante.
\item[IO ho una rabbia addosso ch'io avvampo] Latino \textit{Jn fermento totus sum}. Io ho una
  collora, un'ira grandissima. \textit{Avvampare} significa abbruciare leggiermente, per
  esempio; Un panno bianco accostato a una fiamma s'infuocola, e piglia il nero, e
  si dice arso, o abbronzato, o avvampato.
\item[BARBA d'Oloferne] Barbaccia. E' nota la Storia sacra di Iuditta, che tagliò
  la testa ad Oloferne. Nel rappresentar detta storia, li Pittori per far conoscere
  Oloferne per un'huomo crudele, dipingono la di lui testa tagliata brutta, e con
  barba lunga, folta, e rabbuffata; e da questo il dire a uno \textit{barba d'Oloferne} è
  ingiurioso, perché suona anche lo stesso, che \textit{testa d'impiccato}.
\item[METTEREI la mano nel fuoco] Mi par d'esser così certo di questa cosa, che io
  la giurerei con metter la mano nel fuoco. Uno de' giudizzi, che chiamavano Divini,
  appresso i Sassoni era la prova, che faceva il reo, per via del fuoco, tenendo
  in mano ferro infocato. E le solennità, colle quali si veniva a questa prova,
  sono descritte puntualmente dietro all' Istoria Angelica di Polidoro Vergilio.
\end{description}
\section{Stanza LXXXII. --- LXXXVI.}
\begin{ottave}
\flagverse{82}Oimé le mie stoviglie, e il vin di Chianti \\
Ch'io tolsi in dar la caccia a un vetturale \\
A cagion di quel tristo Graffiasanti \\
In un tempo e versato, e ito male, \\
Giuro al Ciel ch'io non vuò ch'ei se ne vanti, \\
E, s'ei non vola, puo far capitale \\
Ch'io voglia ritrovarlo, e s'ei c'incappa \\
Che mi venga la rabbia s'ei mi scappa.
\end{ottave}

\begin{ottave}
\flagverse{83}Lo troverò bensì, perch' io vuò ire \\
Qua intorno per veder s'io lo rintraccio; \\
Così corre alla porta per uscire, \\
Ma ei non può farlo, perché e' v'è il chiavaccio \\
Lo squote, e sbatte per volerlo aprire,\\
Ed hor v'attacca l'uno, hor l'altro braccio; \\
Noiato al fine vanne, e corre ad alto, \\
E dai balconi in strada fa un salto.
\end{ottave}

\begin{ottave}
\flagverse{84}Ma perch'ei vede quivi le pedate\\
Volte al giardino, e poi verso la via,\\
Che Brunetto, e quegli altri havean lasciate,\\
Quando v'entraro, e quando andaro via,\\
Insospettito, lascia andar il Frate,\\
Ed entra nel giardino, e a quella via\\
Scorge quel suo cocomero diviso\\
Ch'è stato il fargli un fregio sopr'al viso.
\end{ottave}

\begin{ottave}
\flagverse{85}Poiché levata gli han quella figliuola,\\
Ch'in esso (com'io ho detto) si trovava,\\
Per la stizza non puo formar parola,\\
Si sgaffia, batte i denti, e fa la bava,\\
E spalancando poi tanto di gola\\
Urla, bestemmia il Ciel, minaccia, e brava\\
Dicendo; O Macometto, e tu comporti,\\
Che si facciano al mondo questi torti?
\end{ottave}

\begin{ottave}
\flagverse{86}In quanto a te chi ti pisciasse addosso \\
So ben che tu non ne faresti caso;\\
Ma io che da miei dì mai bevvi grosso, \\
E le mosche levar mi so dal naso \\
Sapro ben'io a costor far il \culo{} rosso,\\
Credilo pur, perché, se si dà il caso\\
(Che si darà senz'altro) ch'io, gli arrivi\\
Io me gli vuò di posta ingoiar vivi.
\end{ottave}

Seguita Magorto a dolersi della sua disgrazia; poi fatta risoluzione d'andar'a
cercar del Romito, salta dalla finestra in strada, dove vedute alcune pedate verso
il giardino, insospettito lasciò il pensiero d'andar cercando di Pigolone, e se
ne va alla volta del giardino  e quivi accortosi del ratto della fanciulla, giura
di voler trovare coloro, che gli hanno fatto questo torto, e di, volergli tutti
ingoiar vivi. Nota che il nostro Poeta in qn ottava 84. e stato criticato, perché
s'è servito della voce \textit{via} in tutte tre le rime, ma tal sottigliezza si può più
tosto chiamare ignoranza, perché se bene è sempre la stessa voce, \textit{Via}, ha però
sempre diverso significato, perché la prima significa strada; la seconda significa,
altrove, o moto da un luogo a un'altro, e la terza significa modo, guisa, maniera,
ec, E di simili rime troverai altrove in questa Opera, e sempre le vedrai
lodevoli per l'artifizio, più tosto, che biasimevoli per la poca avvertenza.

\begin{description}
\item[HOIMÉ] Esclamazione, che esprime disgusto, o dolore. Latino \textit{Hei mihi};
\item[CHIANTI] E' una regione in Toscana dove nasce vino buonissimo. E Vetturale
  intendiamo colui, che sopra alle bestie conduce vino, ed altre robe da un
  luogo all'altro; a differenza di Vetturino, che presta, ed accompagna cavalli,
  lettighe, ec, a i Viaggianti. Vedi sopra C. 6, st. 37.
\item[DAR fa caccia] Correr dietro a uno. E propriamente si dice \textit{Dar la caccia},
  quando i birri corron dietro a uno per pigliarlo.
\item[GRAFFIASANTI] Bacchettone, Ipocrito. E' lo stesso, che Santinfizza
  detto sopra in questo C. st. 68.
\item[PVO' far capitale] Può esser certo. Qusta voce Capitale significa lo stato, o
  sustanze d'uno: \textit{Il tale ha 10. m. scudi di capitale}. Significa assegnamento. \textit{Chi
    del mio fa capitale} detto sopra C. 2. st. 7. Significa sorte principale. Latino \textit{Sors},
  dai Greci detta \textit{cephalaion}, cioè \textit{caput}; dagli Spagnuoli \textit{caudal}, che corrisponde
  al nostro \textit{Capitale}, e \textit{Caudaloso} dicono colui, che ha gran capitale\footnote{Nello spagnolo moderno, secondo la RAE, \textit{Caudal} mantiene il significato qui menzionato, ma più comunemente si usano con riferimento alla portata dei fiumi, ed alla turbolenza del flusso.}, cioè grandi
  sustanze. \textit{Il tale ha havuto la sentenza contro, ed è Stato condennato nelle spefe, ed a
    pagare cento scudi di frutti, e mille di capitale}. Significa quello vedremo sotto C. 8.
  st. 65. Qui significa può credere, può esser sicuro.
\item[SEI c'inciappa] S'ei mi da nelle mani. Se, c'incoglie. S'egli casca ne' miei
  agguati.
\item[MI venga la rabbia] Giuramento imprecativo contro se stesso. Giuro di voler
  far la tal cosa, e se non la fo, mi sottopongo a ogni maggior tormento.
\item[S'IO lo rintraccio] Traccia significa orma, o vestigio; onde tracciare vuol dir
  seguitare le pedate, e per conseguenza qui intende: Se io lo ritrovo; \textit{Traccia} si
  dice quella strada, che fa il cane per la passata della lepre, o d'altro animale
  fiutando; viene questo verbo \textit{rintracciare}, che vuol dir Ritrovare, e
  \textit{tracciare} cercare, Latino \textit{vestigare}. 
\item[CHIAVACCIO] È lo stesso, che chiavistello detto sopra C. 1. st. 69. che i Sanesi
  dicono \textit{pestio} dal Latino \textit{pessulus}. Il Conte Ugolino presso Dante Inf. 33. \textit{Ed
    io sentì chiavar l'uscio di sotto all'orribile torre}; cioè mettere il chiavaccio.
\item[A QVELLA via] A quella foggia. In quella guisa.
\item[FARGLI uno sfregio in sul viso] Fargli una ingiuria ignominiosa, si come sono
  gli sfregi. Vedi sopra C. 2. st. 3. e c. 6. st. 54.
\item[FA la bava] Intendi ha gran rabbia. Latino \textit{stomachatur}, Che \textit{bava} è
  quell'umore viscoso, che da per se stesso casca dalla bocca come schiuma, come si vede
  ne i cani arrabbiati, donde è presa la presente metafora. Si dice ancora: \textit{Mi
    fa venir la bava}; di chi mi fa entrare in collora, e di chi noia forte.
\item[IL Ciel minaccia, e brava] Sgrida, e minaccia il Cielo. Vedi sopra C. 5. st. 62,
  che dice \textit{Rabbiosa, il capo verso il Ciel tentenna}, che è quel minacciare il Cielo.
  Di questo verbo \textit{bravare}, che vien dal Provenzale il Varchi ne fa un lungo discorso
  nel suo Hercolano, e lo giudica molto esprimente il latino \textit{obiurgare}. Catullo.
  \begin{verse}
    \backspace Gellius audierat, patruum obiurgare solere,
    Si quis delitias diceret, aut faceret.
  \end{verse}
\item[TANTO di gola] Gola assai larga. Vedi sotto C. 16. st. 18, la forza della
  voce \textit{tanto} usata in questi termini.
\item[NON ne faresti caso, quand'uno ti pisciasse addosso] Non ti chiameresti offeso, o
  non t'importerebbe quand'uno ti pisciasse addosso; ed intendi: Sei tanto briccone,
  e codardo, che sopporteresti qualsivoglia grandissima ingiuria senza risentirtene.
  Un'antico Poeta per voler esprimere uno scellerato, e ingiurioso fino alla memoria
  di suo padre, dice: \textit{patrios minxerit in cineres}\footnote{Orazio, Ars Poetica, v.471.}. E Pittagora in uno de' suoi Simboli
  per dinotare il rispetto, che si dee portare alla Divinità, comanda, che
  non si pisci in faccia al Sole.
\item[NON bevvi grosso] Non sopportai mai ingiuria alcuna. Ber grosso vuol dire
  Non la guarda così per la minuta, ma sopportare ogni ingiuria senza risentirsene,
  fingendo non sen'avvedere. Tratto dal bere le medicine, le quali non s'assaporano,
  ma si mandano giù a occhi chiusi.
\item[MI so levar le mosche d'intorno al naso]  Mi so vendicare dell'ingiurie, con
  facilità, Omero nell'Iliade La prestezza, colla quale un Dio fa tornare indietro i
  colpi avvelenati contro a un'Eroe compara al cacciare d'una mosca, che fa la
  Madre dal corpo del suo figliuolo.
\item[FAR il \culo{} rosso a uno] Gastigar'uno. Tratto da i Pedanti, i quali gastigano
  i ragazzi perquotendola in sul \culo, e glielo fanno rosso con le percosse. Vedi
  sopra C. 4. st. 51.
\item[DI posta] Subito, Viene dal giuoco di palla, che si dice Dar di posta, quando
  si dà di primo tempo, cioè avanti, che la palla tocchi terra. Lavino è \textit{vestigio}.
\item[INGOIARE] È lo stesso, che ingollare detto sopra C. 1. st. 6, e vuol dir
  mandar la roba giù nello stomaco.
\end{description}
\section{Stanza LXXXVII. --- LXXXXIII.}
\begin{ottave}
\flagverse{87}Ma dove col cervel son' io trascorso?\\
Più bue di me non è sotto le stelle, \\
Perch'innanzi ch'io habbia preso l'orso \\
Vuò (come si suol dir) vender la pelle, \\
Fatti ci voglion qui, perch'il discorso,\\
Fuor che ai Sensali non fruttò covelle, \\
E mal per chi ha tempo, e tempo aspetta,\\
Che mentre piscia il can, la lepre sbietta.
\end{ottave}

\begin{ottave}
\flagverse{88}E però prima, che a viola a gamba\\
Una fuga mi suonin di concerto.\\
A casa Pigolon vuogl'ir di gamba, \\
Che vi sarà co i complici del certo.\\
Così conchiuso, corre ch'ei si sgamba,\\
E come un bracco va per quel deserto \\
Tutti quanti quei luoghi a uno a uno \\
Cercando s'ei vi scuopre, o sente alcuno.
\end{ottave}

\begin{ottave}
\flagverse{89}Quel della Cella del Romito è il primo,\\
Ove trovando il passo, e porto franco\\
Intana dentro, e non vi scorge niuno,\\
Fruga, e rifruga in qua, e il là, ne anco;\\
Sgomina ciò che v'è da sommo, e imo,\\
Ma tutto in vano, ond'egli al fine stanco\\
Sen'esce con le man piene di vento,\\
Ma dieci volte più di mal talento.
\end{ottave}

\begin{ottave}
\flagverse{90}Entrò nel bosco, e ogni contrada scorse,\\
E in somma ne cercò per mari, e monti,\\
E vedde, senza metterla più in forse,\\
Il pigiato esser lui al far de' conti.\\
Onde nel fine all'arti sue ricorse,\\
Che pur vuol vendicar sì grandi affronti\\
Così v'arriverà po' poi in quel fondo,\\
Se voi fusse (dicea) di là del mondo.
\end{ottave}

\begin{ottave}
\flagverse{91}E poiché fatti ha certi suoi incanti,\\
Che gli riescon bene, e vanno a vanga:\\
Andate, dice, o stummia di furfanti,\\
Poi c'a pianger volete ch'io rimanga,\\
Che sieno in casa vostra eterni pianti,\\
Tal che ciascuno, e fino il gatto pianga,\\
E così poi, di quanto haveva detto,\\
Ne più, ne manco ne segui l'effetto.
\end{ottave}

\begin{ottave}
\flagverse{92}Poiché Brunetto, e le sue camerate\\
Pagaron l'oste, (il quale assai contese,\\
Perché le gole lor disabitate\\
Gli eran parute care per le spese)\\
Partiron, e poi dopo altre fermate,\\
Ei le condusse salve al suo paese,\\
E giunto a casa, ringraziando il Cielo\\
Entra in sala, e di posta fa un belo.
\end{ottave}

\begin{ottave}
\flagverse{93}Entra la donna col Romito appresso,\\
E cominciaro a pianger ambedui,\\
Entra il famiglio, e anch'egli fa lo stesso\\
Senza saper perché, ne men per cui.\\
Trovan Nardino ancor di male oppresso,\\
E sbietolar lo veggono ancor lui,\\
L'Astante, che porgevali l'orzata\\
Purine faceva lafua quattrinata
\end{ottave}


Magorto lascia lamenti, e si mette a cercar di coloro, che gli havevano rubata
la Figliuola, e non gli trovando nella Cella del Romito, ne in alcun altro
luogo, ricorse a gl'incanti; co i quali costrinse tutti della casa di Brunetto a piangere
sempre, onde Brunetto con i compagni arrivato a casa subito cominciò, ed
egli, ed i compagni a piangere.
\begin{description}
\item[DOVE son' io scorso col cervello!] Che armegg'io? Che giro io? Che frenetich'io?
\item[NON è sotto le stelle il più bue di me] Io sono il maggiore ignorante che sia nel
  Mondo. Vedi sopra \cstan[6]{98}. \textit{Sotto la Luna}; il Petrarca. \textit{Arda, o mora, o
    languisca, un più gentile Stato del mio non è sotto la Luna}.
\item[VENDER la pelle dell'orso prima di pigliarlo] Far assegnamento sopra una cosa,
  che ancora non s'è conseguita, ed è anche molto dubbioso il conseguirla. Essendo
  andati tre Giovani per ammazzare un'orso, il quale faceva molto danno,
  prima che arrivassero al luogo dove soleva trovarsi l'orso, si fermarono a un'Osteria
  ed havendo assai ben mangiato, dissero all'Oste, che lo pagherebbono
  con li denari del donativo, che havrebbono dato loro le Comunità per l'orso,
  che volevano ammazzare, ed inviatisi verso dove stava la fiera, subito, che la
  veddero si diedero a fuggire, e uno di loro salì sopra ad un'albero, l'altro scappò
  via, ed il terzo fu sopraggiunto dall'Orso, il quale havendoselo cacciato sotto
  l'infranse bene bene, di poi gli accostò il grifo all'orecchio, ed intanto quel
  meschino se ne stava come morto, senza muoversi punto; e perché l'orso naturalmente
  (secondo dicono alcuni) quando crede, che l'animale da lui assaltato
  sia morto, non gli da più fastidio, credendo che costui fosse morto, sen'andò, e
  colui si levò su, ed avviossi verso la Città tutto mal concio. Quello, che era salito
  in sull'albero scese, ed accompagnatosi con esso, gli domandò quel che gli
  havesse detto l'orso nell'orecchia, ed egli rispose: Mi ha detto, che io non mi
  fidi più di simili compagni come sei tu, e che io non venda la pelle dell'orso, se
  prima non l'ho preso. E da questa novella habbiamo il presente proverbio, che
  si dice anche: \textit{Vender l'uccello in su la frasca}. I Geesi dissero: \textit{Antequam pisces
    ceperis, muriam misces}.
\item[MAI fruttò covelle] Non fu d'utile alcuno, \textit{Covelle} è voce romagnuola e vuol
  dire. \textit{Qualcosa}. E' poco usata nel Fiorentino fuor che da qualche contadino. Il
  valore di questa voce è assai copiosamente espresso dal Copetta\footnote{Francesco Beccuti, detto il Coppetta. Perugia 1509 --- 1545. \textit{Capitolo nel quale si lodano le Non covelle}, 1548.} in un suo Capitolo
  sopra il non covelle, Nel Decameron trovati \textit{Cavelle} per lo stesso, quasi da un
  Lat. \textit{quod velles}.
\item[E' mal per chi ha tempo, e tempo aspetta,]\items{che mentre, ec.} Mal fa colui, che havendo
  l'occasione pronta perde il tempo, e non la piglia, perché mentre s'indugia,
  l'occasione fugge: È noto il verso: \textit{Fronte capillata post haec occasio calva}, ed il
  verbo \textit{sbiettare} l'habbiamo anche sopra \cstan[5]{30}. \textit{Mentre il can piscia la lepre
    se ne va}. I Latini dissero \textit{Semper nocuit differre paratis}; secondo Lucano, di dove
  forse Dante nell'Inf. C. 28. disse:\
  \begin{verse}
    \backspace Questi scacciato il dubitar sommerse
    In Cesare affermando, che il fornito
    Sempre col danno l'attender sofferse.
  \end{verse}
\item[PRIMA che a viola a gamba, ec.] Intende prima che d'accordo se ne fuggano.
  \textit{Viola a gamba} e il basso di viola. \textit{Fuga} è specie di sonata a capriccio. Di \textit{concerto}
  vuol dir Suonata concertata con diversi strumenti, ec. E con questi equivoci intende
  quel che s'è accennato.
\item[INTANA] Entra dentro. Si serve di questo verbo anche sotto \cstan[10]{25}.
  se bene è improprio, perché vuol dire Entrare in una tana, o buca, e si direbbe
  intanare una volpe, un tasso, un granchio, ec, tuttavia è pur talvolta usato
  come nel presente luogo. 
\item[NIMO] Niuno. Dal Lat \textit{nemo}. Voce oggi usata solo dai contadini, ed il
  nostro Poeta se ne serve anche sotto \cstan[10]{37}. in bocca d'un contadino.
\item[SGOMINA] Si dice anche \textit{sgombinare}, (contrario di \textit{combinare}, che è
  accoppiare, unire) e vuol dir mettre in confusione, o sottosopra tutto quel che si
  maneggia. Lat. \textit{perturbare}.
\item[DA sommo, a imo] Frase latina, che significa Da capo a piedi: Dalla sommità
  della casa, fino a i fondamenti di essa: Petrarca Trionfo della Fama, cap. 2.
  \textit{Onde da imo Perdusse al sommo l'edificio santo}.
\item[LE man piene di vento] Cioè senz'haver trovato, o conchiuso nulla. Nella
  Scrittura. \textit{Et nihil invenerunt in manibus suis}; che diciamo ancora \textit{Con le trombe nel
    sacco} Ter. disse \textit{Infectare}. 
\item[DI mal talento] 1 collera, e con volontà di far del male, e di vendicatl:
Varchi Stor. lib. 4. Erano verso i nobis di malissimo talento, ne altro per manvweit:
gli alpettavano, che quel che avvenne. BY frase usaca dai Boccaccio. =)
\item[NE cerco per mars e monti] Questo detto iperboiico è uiacutimo per esprimett
Ne cercd da per tutto; Viene dal Latino. ee
\item[SENZA metteria in forse] Senza dubicar più. Senza metterla in dubbio. Dd
mettere in forse fece Dante il verbo inforfare, seguivato in ciò dal Petrarca +
\item[IL pigiato esser lui al far de' conti] A considerarla bene l'offeso, e beffato era
  solamente lui. Quattro giuocano insieme, tre vincono, ed un di loro solamente
  perde; questo tale si dice \textit{il pigiato}, cioè quello, che ha gli altri addosso, e da
  cui si spreme il denaro. E s'intende in ogni caso, che la disgrazia tocchi a un
  solo della conversazione, e tutti gli altri habbiano soddisfazione, o utile dal
  danno di lui.
\item[PO POI in quel fondo] Vedi sopra \cstan[2]{13}.
\item[VANNO a vanga] Vanno secondo il desiderio. \textit{Ex animi eius sententia illae res
  fluunt}. Noi l'habbiamo da i Contadini, che quando si rende loro facile il lavorar
  la terra con la vanga dicono; \textit{il lavoro va a vanga}, cioè bene, e come si desidera.
  E \textit{vanga} è quello strumento rustico fatto a foggia di pala, ma di ferro più
  massiccio, e più acuta, del quale i contadini si servono per rivoltolar la terra.
  Vedi sopra \cstan[6]{69}, al verbo \textit{impiallacciare}. Columella lib, 3, la chiama
  \textit{dolabra}, e perché questo nome vuol dire più tosto la pialla, forse Columella intende
  qualche strumento usato a suoi tempi, che faceva sopra alla terra l'effetto che
  fa la pialla sopra il legno, (come è hoggi la marra scopaiola, della quale si servono
  i contadini per ripulire, e radere i boschi di scope\footnote{Scopa è il nome comune di varie piante arbustive estremamente ramificate, come per esempio la \textit{Erica scoparia} L.} per disporgli alla sementa
  della segale) perché, se volesse dire la vanga, havrebbe detto \textit{acuta dolabra sodito},
  e non abradito: E la vanga si trova \textit{bipalium}, in Varrone: \textit{Id prius bipalio vortito}.
\item[STUMMIA di furfanti] Scelleratissimi, \textit{ex omni vitiorum colluvione concreti}.
  \textit{Stummia}, \textit{schiuma}, o \textit{spuma}, è quello escremento, che nel bollire una pentola
  piena di carne, e di acqua manda alla superficie, il quale si butta via, perché è
  immondizia; onde \textit{stummia di furfanti}, il peggio, che sia nella furfanteria. 
\item[GOLA disabitata] Lat. \textit{gurges}. Così diciamo di colore, che sempre
  mangiano, ne mai si veggono sazzi.
\item[GLI eran paruti cari per le spese] Era parso all'Oste, che costoro havessero
  mangiato troppo. D'uno che sia buono a poco, e mangi assai, e che vada a servire
  diciamo: \textit{egli è caro per le spese}; e intendesi, se gli dà più del dovere, e di quel che
  merita la sua abilità a dargli solamente mangiare, senza dargli danari per provvisione.
  Il Lalli nella sua En. Ir. \cstan[2]{130}.
  \begin{verse}
    Non vaglio un pel; son caro per le spese.
  \end{verse}
\item[DI posta fa un belo] Subito comincia a piangere a belare. Vedi sotto C.9, st.21.
\item[SBIETOLARE] Cioè piangere. Vedi sopra \cstan[4]{16}.
\item[ASTANTE] Intende colui, che assiste al servizio di Nardino infermo, \textit{Astanti}
  si dicono quei Serventi, che assistono a servire gl'infermi negli Spedali, e questi
  soglion esser chiamati dalle persone comode ad assistere alli loro infermi, e però
  qui lo chiama col nome d'\textit{Astante}, supponendolo uno di questi tali.
\item[ORZATA] Bevanda rinfrescativa fatta di seme di popone, orzo, e zucchero,
  benissimo pesti e liquefatti con acqua, e passati per stamigna, si dà per lo più
  a febbricitanti; detta anche \textit{lattata} come habbiamo veduto sopra in questo C. st. 12.
\item[NE faceva la sua quattrinata] Cioè faceva la sua parte del pianto.
\end{description}
\section{Stanza XCIV. --- XCVIII.}
\begin{ottave}
\flagverse{94}Nardin vede colei bell', e vezzosa \\
Com'appunto l'haueva nel pensiero, \\
E dice: Benuenuta la mia sposa, \\
Voi mi piacere a fe da Cavaliero. \\
Ma voi piangere ? ditemi una cosa \\
Voi ci venite a malincorpo, è e' vero? \\
Non vogliate risponder che e' non sia, \\
Perché voi mi diresti una bugia.
\end{ottave}

\begin{ottave}
\flagverse{95}Mettere pur così le mani innanzi\\
(Rispond' ella) Signor per non cadere,\\
Mentre, temendo ch'io non mi ci stanzi,\\
Specorate si ben, ch'è un piacere:\\
Ch'io mi vi levi, ditemi, dinanzi,\\
Che voi non mi potete più vedere,\\
Senza darmi la burla, ch'io m'acquieto,\\
E Senza replicar do volta a dreto.
\end{ottave}

\begin{ottave}
\flagverse{96}Ne sossopra la man non volterei,\\
Che l'andare, e lo star mi son tutt'una,\\
E ben c'al mondo io sia come gli Ebrei,\\
Che non han terra ferma, o patria alcuna\\
Andrò pensando intanto a' fatti miei\\
Per veder di trovar miglior fortuna,\\
Perché, come diceva Mona Berta:\\
Chi non mi vuol segn'è che non mi merta.
\end{ottave}

\begin{ottave}
\flagverse{97}Ed ei risponde: Oimè Signora mia\\
Non vi levate in barca così presto;\\
S'io non v'ho detto, o fatto villania\\
Perché venite voi a dirmi questo?\\
Habbiate un po più di flemma in cortesia\\
C'ogni cosa andrà bene in quanto al resto,\\
Voi siate bella, ed anco di più sposa\\
Però non vogliat'esser dispettosa.
\end{ottave}

\begin{ottave}
\flagverse{98}Ella soggiunge, ed Egli ribadisce;\\
Ella non cede, ed ei risponde a tuono;\\
Pur gli acquieta Brunetto, e al fin gli unisce,\\
Sicché l'un l'altro chiedesi perdono.\\
Ma non per questo il lagrimar finisce\\
c'ogmora in casa, fuora, e ovunque sono\\
(Perché sempre si smoccica, e si cola)\\
Hanno a tener agli occhi la pezzuola.
\end{ottave}

Nardino vede la Fanciulla; e la trova per appunto comie fel" era'
ma vilto che ella pi angeva le dice, che dubita, che ella  sia venuta mi
ed ella gli risponde, che dubita, che più tosto egh non la riceva volentieri, e
sopra questo seguitavano a contrastare, ma Brunetto al finé gli rappacificò, e con
tutto questo ognuno seguitava a piangere.
\begin{description}
\item[VOI ci venite a malincorpo] Voi ci venite malvolentieri, e con poco gusto, e
  soddisfazione; contra stomaco, cotra voglia, fattone una sola parola, come avverbio.
\item[METTETE le mani innanzi] Questo termine ci serve per esprimere uno che
  accusa un'altro di qualche mancamento, del quale merita di esser accusato lui,
  per esempio: I ragazzi dello Spedale degl'Innocenti, i quali si suppone, che
  sieno tutti bastardi, in occasione di contrastare con altri ragazzi, la prima
  ingiuria che dicano a quelli è, \textit{Tu sei bastardo}, pecche non sia detto a loro. E questo
  si dice: \textit{Metter le mani innanzi}: e vi si aggiugne anche; \textit{per non cascare}. Lat
  \textit{praevertere}, \textit{occupare}.
\item[NON mi ci stanzi] Non mi fermi in questa Casa per sempre.
\item[SPECORATE] Piangete. Diciamo belare per piangere per la similitudine
  che ha col belar degli agnelli, e delle pecore certo pianto lungo, che soglion fare
  i bambini, come accennammo sopra \cstan[6]{22}. e da questo si dice anche
  \textit{specorare} in vece di \textit{belare}, e s'intende piangere.
\item[SI ben ch'è un piacere] Tanto bene, che è un gusto a sentirvi, e vedervi.
\item[NON ne volterei la mano sottosopra] In questa cosa io sono indifferente, cioè
  poco m'importa il farla, o non farla. Viene da i Latini che dicevano anch'hessi:
  \textit{Ne manum quidem verterem}.
\item[ESSER come gis Ebrei] Cioè non haver luogo che sia suo proprio, e lo dichiara
  il Poeta medesimo dicendo: \textit{Non ho terra ferma}, che intende terra, luogo, o
  abitazione fermata, e stabilita per lei, che per altro Terra ferma si dice quel
  paese, che non è Isola di mare, Lat. \textit{continens}.
\item[VOI vi levate in barca] Voi entrate in collera, Vedi sopra \cstan[6]{41}. Si
  dice anche \textit{Imbarcare}, e \textit{l'iracondo}, o vero \textit{facile all'ira}, che i Greci chiamavano
  \textit{Acrocholos} è detto da noi \textit{Huomo di poca levatura}, cioè che ci vuol poco a farlo levare
  in collera.
\item[FLEMMA] Qui vuol dire sofferenza, o pazzienza, che per altro Flemma
  significa quel che accennammo \cstan[3]{24}.
\item[DISPETTOSA] Iraconda, Vedi sopra \cstan[1]{29}. Alcuni critici hanno
  fiutato ancora questa \textit{rosa}; giudicandola rima falsa in riguardo dell'\letter{s}, dolce di
  \textit{sposa}, e della cruda di \textit{dispettosa}, e dell'\letter{o}, largo di quelle, e stretto di queste,
  ma io non gli voglio quietare, e difendere il nostro Poeta col Ruscelli, o con
  altri, perché non mi son voluto pigliar la briga di vedergli come non necessaria,
  porto ben loro un'esempio d'Autore classico\footnote{Ludovico Ariosto, \textit{Orlando Furioso}, Canto 1. Ottava 42.} il quale dice.
  \begin{verse}
    \backspace La verginella è simile alla rosa
    Nel bel giardin si la nativa spina,
    Mentre sola, e sicura si riposa
    Ne gregge, ne pastore se le avvicina
    L'aura suave, e l'alba rugiadosa, \makebox[6pt]{} ec.
  \end{verse}
  E mi pare con questo esempio (il quale sia per regola, o per licenza) di salvare
  il nostro Poeta, e quietargli, ancor per l'altre, che hanno osservate, e sopra
  \cstan[4]{13}. Rosa,  prosa, e cosa; e sotto in \cstan{103}. Sposa,
  cosa, e generosa.
\item[RIBADIRE] Ribattere, conficcare dall'altra parte un chiodo. Vale per replicare.
  Vedi sopra \cstan[2]{79}.
\item[RISPONDE a tuono] Risponde aggiustatamente, ed a proposito di quel, che si
  dice. \textit{verbum audit, tale dicit}. Si dice anche \textit{Rispondere per le Rime}. La
  prima similitudine è tratta dalla Musica; la seconda dalla Poesia; E allude al costume
  de' Poeti che indirizzando l'uno all'altro Sonetti, e proponendosi questioni,
  rispondevano, e le scioglievano in altra eguale composizione tessuta delle
  medesime rime, il qual costume venuto dall'antico, si mantiene anche in oggi.
\item[SI smoccica, e si cola] Si manda escrementi dal naso, e lagrime dagli occhi
  per causa del pianto, che smoccicare vuol dire mandar fuori mocci, che è quello
  escremento del cervello, che esce dal naso detto da i Latini \textit{mocus}.
\item[PEZZVOLA] Fazzoletto, o Moccichino; ed è quel pezzo di panno lino, che
  si porta appresso di se per uso di nettarsi il naso.
\end{description}

\section{SEANZA XCIX. --- CII.}
\begin{ottave}
\flagverse{99}Vivono in somma in un continuo pianto, \\
Piangono i servi, e piangon gli animali, \\
Onde il guazzo per terra è tale, e tanto, \\
Che e' portan tutti quanti gli stivali. \\
Ma torniamo a Magorto, che fra tanto \\
Per saper quel che sia di questi tali, \\
E dove la sua figlia si ritrovi, \\
Ha fatto al consueto incanti nuovi.
\end{ottave}

\begin{ottave}
\flagverse{100}E veduto ch'ell'è tra buona gente,\\
Moglie d'un ricco, e nobil Baccalare,\\
E che giammai le può mancar niente,\\
Per ch'ella è in una casa come un mare,\\
Non vi so dir s'ei gongola, e ne sente\\
Contento grande, e gusto singolare,\\
Di modo ch'ei si pente, affligge, e duole\\
Di quanto ha fatto, e rifarcir lo vuole.
\end{ottave}

\begin{ottave}
\flagverse{101}Perciò per un suo cogno, se ne corre,\\
E nell'orto lo porta, dove è un frutto,\\
C'ha i pomi d'oro, e ne comincia a corre, \\
Durando fin che l'hebbe pieno tutto;\\
E poi che dentro più non ne può porre,\\
Sapendo, che 'l suo aspetto è molto brutto\\
Si lava, ripilisce, e raffazzona,\\
E rimbellisce tutta la persona.
\end{ottave}

\begin{ottave}
\flagverse{102}E presa addosso poi quella sua cassa,\\
Ch'è tanto grave, ch'ei vi crepa sotto, \\
Si mette in via, e presto se ne passa\\
Ov'è la figlia, e il flebile raddotto ,\\
Che al suo venire ogni mestizia lassa\\
Mutando in riso il pianto sì dirotto,\\
E versa i pomi in mezzo della stanza,\\
Poi si sberretta in termin di creanza.
\end{ottave}

Mentre che costoro piangono, Magorto per via de' suoi incanti scuopre dove
è la Figliuola, e conoscendo che ella è bene allogata, si muta di proposito, e
risolve di regalare gli sposi d'una quantità grande di pomi d'oro colti nel suo
orto, e così fece, ed all'arrivo suo in casa degli sposi tutti cessarono di piangere.
\begin{description}
\item[GUAZZO] Luogo pieno d'acqua, dove si possa guazzare, cioè passare a
  piede senza navilio, che noi dal latino diciamo, \textit{vado}, o \textit{guado}; onde il Porto di
  \textit{Vada} così detto perché quel luogo dicevasi \textit{Vada Volaterrana}, e \textit{guadare} per passo,
  e passare: Ma si piglia ancora per ogni grande ammollamento, che si faccia
  nelle case, o altrove in sul suolo, come è preso nel presente, luogo, ed in questo
  caso viene da \textit{guazza}, la quale cade dal Cielo, altrimenti detta \textit{brinata} dal Lat.
  \textit{pruina} come \textit{gelata} disse Dante dal Lat. \textit{gelu}; e non da \textit{guazzare} il fiume; Se
  forse non volessimo pigliarlo per parlare iperbolico, come è l'adoperare gli stivali
  per passar tal molle, che è in quella stanza.
\item[BACCALARE] Huomo di stima. Uno dei principali del paese, che si dice
  anche \textit{Barbassoro}. \textit{Baccalare} da Baccalaureus si dice colui, che nelle scienze ha
  acquistato un grado prossimo al Dottorato, o Maestrato detto altrimenti Licenziato;
  il che usa nelle Fraterie, e corrottamente lo dicono \textit{Baccelliere}, il qual
  grado si ritrovava anche nell'ordine della Cavalleria.
\item[È una casa come un mare] Cioè sempre piena di roba ed abbondante d'ogni 
  bene, si come il mare, che è immenso, detto perciò da Omero \textit{atrygeton}, cioè \textit{che
    non ha fin, ne fondo}, Si dice anche \textit{Una casa come una Dogana}\footnote{Vedi sopra \cstan[5]{28}.}.
\item[GONGOLA] Greco cancharei » Giubbila:-Si rallegra 2-5i
certa allegrezza interna. B' voce usata assai dalla piebe.: Re
\item[RISARCIRE] Ristorare; Rifare il danno, o ricompensargli d'havergli tenuti
  tanto in pianto. E per altro questo verbo \textit{risarcire} vuol dir \textit{rassettare} come s'è
  visto sopra \cstan[6]{52}.
\item[COGNO] E' una misura immaginaria\footnote{\textit{Misura immaginaria}, nel senso che è una unità di misura, a cui non corrisponde un oggetto fisico che realizzi questa quantità. Il concetto opposto è misura \textit{effettiva}, come il barile, o il quartuccio.} di vino, che contiene dieci barili; la
  quale corrottamente si dice \textit{Conio}; Deriva dal Lat. \textit{congius}. Onde Bigonce\footnote{Vedi sotto C.\ 10.\ stan.\ 20.} quasi
  da un Lat. \textit{bicongius}; a Pistoia perciò dette più prossimamente all'origine \textit{Biconge}
  Gio. Villani lib. 8, rubr. 116. \textit{Valse lo staio del grano in Firenze soldo 8, e 'l cogno del
    mosto in certe parti meno di soldi 40}, Ma qui è preso, come è costume, per una
  certa sorte di cassa, o più tosto cesta fatta, e contessa di strisce d'albero come i
  corbelli, ma è di foggia lunga, ed ha il coperchio, come hanno le casse.
\item[SI raffazzona] Si ripulisce; Si rinfronzisce. Vedi sopra \cstan[]{69}. Quasi
  si rifà; si rimette in fazione, in abito; su la galanteria; su la bella foggia e
  maniera. Gli antichi dal Provenzale dissero \textit{Ragenzare}, cioè \textit{Raggentilire}, dalla
  voce Gente usata dagli antichi Toscani ancora per \textit{Gentile}, Fra Guittone, Se di 
  voi, donna \textit{gente} M'ha preso amor, non è già maraviglia. Dante. Ma pregia 
  il senno, e li \textit{genti} coraggi. Il Beato Iacopone disse che la Penitenza l'anima
  ragenza, cioè non risciacqua, come spiegò alcuno, ma \textit{rafazzona}, \textit{ringentilisce}.
\item[VI crepa sotto] Vi muor sotto per lo soverchio peso; ed il verbo \textit{crepare}, che
  vale per morire a stento, come vedemmo sopra \cstan[1]{18}. qui è nel suo vero
  significato d'allentare, perché quella gran fatica può cagionare l'allentamento.
\item[SI sberretta] Cioè si cava di capo; dalla Berretta, che è propriamente il \textit{pileus}
  de' Latini essendo il nostro cappello più tosto il \textit{petasus}.
\section{Stanza CIII. --- CV.}
\begin{ottave}
\flagverse{103}E dice ch'egli è il padre della sposa,\\
E che di lui non habbiano spavento,\\
Per ch'egli hormai scordato d'ogni cosa,\\
L'antico sdegno totalmente ha spento;\\
Anzi come persona generosa\\
Vuol dare agli sponsali il compimento,\\
Ch'è quello, che la sposa habbia la dote,\\
E che non vadia a marito a man vote.
\end{ottave}

\begin{ottave}
\flagverse{104}E perché qualsivoglia donnicciuola,\\
Porta la dote, ed il corredo appresso,\\
Acciocch'in quella casa la Figliuola\\
Possa mostrar d'haver qualche regresso,\\
Ne che gli abbin a aver quel calcio in gola\\
C'un piccolo ne anche v'habbia messo,\\
La vuol dotar conforme al grado loro\\
Con quel gran monte di bei pomi d'oro.
\end{ottave}

\begin{ottave}
\flagverse{105}Gli sposi allor brillando con Brunetto\\
Gli rendon grazie, e fan grata accoglienza;\\
Ed ordinato un grande, e bel banchetto\\
Reiterar le nozze in sua presenza,\\
Ed egli poi al fin com ogni affetto\\
Riverì tutti, e volle far partenza,\\
Lodandosi del furto del Romito,\\
Che si grand'allegrezza ha partorito.
\end{ottave}


Magorto si fa conoscere per il padre della Sposa, ed assicurando Pigolone, e
tutti d'havergli perdonato, e d'haver gusto, che segua quel parentando, costituisce
per dote quella cassa piena di pomi d'oro. Si fanno però di nuovo gli sponsali,
ed il banchetto: e Magorto se ne torna al suo paese, dando molte lodi a
Pigolone per esser'egli stato autore di così gran contento. E qui con la fine della
novella raccontata dalle Fate a Paride termina il settimo cantare.
\begin{description}
\item[A MAN vote] Senza nulla in mano: cioè si mariti senza dare dote alcuna.
\item[CORREDO] Quegli arnesi, abiti, ed altre robe, che si danno alle Femmine,
  oltre alla dote, quando si maritano, che i Giureconsulti digono \textit{Parapherna} dal
  Greco \textit{Para}, che vuol dire oltre, e \textit{pherna}, che vuol dir dote.
\item[HAVER regresso] Termine legale, che vuol dire haver azione di domandare,
  contro a uno, per rifarsi del pagato ad un'altro; Vedi sotto C. 8. st. 42. E
  comunemente significa un certo ardire, ed autorità sopra ad una persona, o sopra
  i suoi beni ed effetti: \textit{Il tale gli ha preso regresso addosso}, per intendere ha preso
  ardire sopra di lui.
\item[NON gli abbin a haver quel calcio in gola] Non habbiano a poter rinfacciarle,
  o rimproverarle, che ella non v'habbia portato nulla: Non habbiano a haver
  quella causa di conculcarla\footnote{L'incoerenza delle \letter{h} iniziali nel testo, fra ottave, citazione, e spiegazione, come della forma \textit{habbiano} / \textit{abbino} è fedele al testo stampato, e il Minucci ci passa sopra senza prenderne nota.}.
\item[BRILLANDO] Giubbilando, Vedi sopra C. 2. st. 69.
\item[ACCOGLIENZE] Vedi sopra C. 1. st. 34.
\item[SI reiteraron le nozze] Cioè di nuovo si fecero gli sponsali, e solennemente si
  diedero la fede di sposi.

\end{description}
\section*{FINE DEL SETTIMO CANTARE}
\end{comment}
\chapter{Ottavo Cantare}

\begin{argomento}
Dalle fu Fate Paride vestito
Vede la galleria di quell'albergo;
D'un avventura grande è poi avvertito,
E appresso ha un libro, che non parla in gergo,
Con una spada d'un'acciar fortito,
Ond'ei piglia licenza , e volta il tergo.
Vien Piaccianteo condotto al Generale,
Che non gli volle far ne ben, ne male.
\end{argomento}

\section{Stanza I. --- V.}
\begin{ottave}
\flagverse{1}Vorrei, che mi dicesse un di costoro, \\
Che giostran tutta notte per le vie, \\
Che gusto v'è, perch' a ridurla a oro \\
Non v'è guadagno, e son tutte pazzie;\\
Poiché (lasciando, che e' non è decoro)\\ 
L'aria cagiona cento malattie, \\
Mille disgrazie possono accadere, \\
Mille malanni, Diauoli, e Versiere.
\end{ottave}

\begin{ottave}
\flagverse{2}Sapete, che e' s'inciampa, e che e' si casca, \\
Si può in cambio a' un' altro esser'offeso, \\
O dar in un, se t' hai monete in tasca, \\
C' alleggerir ti voglia di quel peso,\\
Manca in qual mo si può correr burrasca,\\
Però vi giuro, ch'io non ho mai inteso\\
La fin di questi tali, e tengo a mente \\
Quel c'un tratto mi disse un huom valente.
\end{ottave}

\begin{ottave}
\flagverse{3}La notte, disse, è un vaso di Pandora,\\
Che versa affronti, risichi, e tracolli,\\
Però, che nel suo tempo sbucan fuora\\
Tutti i ribali, ladri, e rompicolli;\\
Onde sia ben riporsi di buon' hora,\\
E deve esempio l'huom pigliar da i polli,\\
Che l'un di loro al più vale un testone,\\
E pria ch' il sol tramonti, si ripone.
\end{ottave}

\begin{ottave}
\flagverse{4}Ed egli, che d'un mondo assai più vale \\
Sta fuori tutta notte, o diacci, o piova,\\
E gira al buio come un' animale\\
Cercando di Frignuccio in bella prova,\\
Ne sia gran fatto poi se gli avvien male,\\
Che ben sapesti, che chi cerca trova,\\
Ed eccovene in Paride il riscontro,\\
In modo, che non v'è da dargli contro.
\end{ottave}

\begin{ottave}
\flagverse{5}Perché le son tutte cose provate, \\
E vere, che non v'è spina ne osso,\\
E non si trovan poi sempre le Fate,\\
Che vengano a levarti il mal da dosso, \\
Come al Garani quand' a gambe alzate\\
Andato era la notte giù nel fosso,\\
Che, mentre conteggiava con la morte,\\
Da esse ebbe un favor di quella sorte.
\end{ottave}

Volendo il Poeta seguitare a narrare quanto avvenne a Paride s' introduce col
mostrare di che nocumento sia l'andar fuori di notte, e che però sia cosa da
huomo poco prudente il non considerare quanti pericoli si possono correre; Ed
assomigliando la notte al Vaso di Pandora conchiude, che si dovrebbe imparar
da i polli, che vanno a dormir subito, che e' s'è riposto il sole, e così sfuggire
tutte le disgrazie, perché non si trova sempre chi liberi dal male, come avvenne
a Paride, che dalle Fate fu liberato dal pericolo di morte.

\begin{description}
\item[GIOSTRARE] O armeggiare. Metaforicamente s' intende andar girando, o
  passeggiando senza saper dove, o senza fine determinato, che si dice anche
  andare aioni, o a gironi.
\item[A RIDVRLA a oro] Per ridurla alla conchiusione. Vedi sopra C. 3. st. 48.
\item[MILLE malanni Diavoli, e Versiere] E' un modo di dire assai usato in simili
  congiunture per esprimere possono avvenire tutte le sorte di disgrazie.
\item[VERSIERA] Furia infernale, che dalle nostre donnicciuole è intesa per una
  diavolessa moglie del Diavolo. Forse viene dal Latino \textit{Versutia}, che vuol dir
  malizia; e si dice \textit{Versiera} un ragazzo malizioso, fastidioso, e insolente, ma
  è più verisimile, che venga dal Latino \textit{adversarius}, col quale nome è disegnato il
  Diavolo nella scrittura. \textit{Adversarius noster diabolus}, Petrarca.
  \begin{verse}
    \backspace Sì che avendo le reti indarno tese,
    Il mio duro avversario se ne scorni.
  \end{verse}
  Da \textit{adversarius} nello stesso modo, che i Francesi fecero \textit{adversaire}, così i nostri
  antichi, \textit{Avversiere}, \textit{l'avversiere}, e poi finaimente \textit{la Versiera}. Il Beato Iacopone da
  Todi canto~62.
  \begin{verse}
    Lo nemico ingannatore
    Aversier de lo Signore.
    \verseprefix{E cant.21.}Fata gli aversere venire,
    Che 'l degian accompagnare.
  \end{verse}
 Nell'uso dicesi \textit{Far la Versiera}, \textit{fare il Diavolo}, \textit{e peggio}.
\item[INCIAMPARE] È il latino \textit{offendere}. Vedi sopra C. 1. st. 13.
\item[TASCA] Quella sacchetta, che si porta comunemente appiccata agli abiti per
  uso di portar roba necessaria alla giornata, come denari, e simili da' Latini detta
  \textit{Pera}, o \textit{Zona}\footnote{Papìa riporta: ›Zona:cingulus cintorium‹}.
\item[ALLEGGERIRE di quel peso] Cioè portar via i denari, e così alleggerirlo del
  peso, ed ella noia, che per quello gli veniva.
\item[MANCA in che mo] Cioè sono infiniti i modi. Il termine \textit{manca} in questo caso
  è usato ironicamente, perché s'intende: \textit{Non mancano i modi}.
\item[CORRER burrasca] È termine Marinaresco, che significa Correr pericolo, ed
  in questo significato è preso comunemente, se bene \textit{burrasca} vuol propriamente
  dire sollevamento di mare per il cattivo temporale di venti, ec.
\item[VASO di Pandora] È nota la favola di Pandora, la quale fu una Femmina,
  che Giove fece fabbricare da Vulcano, e darle in dono di ciascuno degli Dei le
  più belle parti, affine di farne innamorar Prometeo, edd indurlo ad aprire un
  vaso pieno di tutti i mali, che Giove haveva dato alla medesima, che lo donasse a
  Prometeo, che vuol dire Prevvidente; che antivede, per vendicarsi dell'ingiuria
  da esso fattogli quando rubò il fuoco celeste, ma non l'havendo Prometeo
  voluto accettare, lo prese Epimeteo suo fratello, che significa prudente dopo il 
  fatto, il quale l'aperse, e vennero fuori tutti i mali, che sono nel mondo: E
  questo è il vaso, che il Poeta intende nel presente luogo, e del quale parla in Berni
  nel secondo capitolo della peste dicendo:
  \begin{verse}
    \backspace Io lessi già d'un vaso di Pandora,
    Che n'era drento il canchero, e la febbre,
    E mille morti, che n'usciron fuora.
  \end{verse}
  Orazio lib, 1, Ode 3.
  \begin{verse}
    Post ignem aetheria domo
    Subductum, macies, \& nova febrium
    Terris incubuit cohors,
  \end{verse}
  La favola, è raccontata da Esiodo.
\item[RISICO] Ristio, o risico dal verbo \textit{arrisicarsi}, \textit{arrischiarsi}, o \textit{arristiarsi}, che
  vuol dire Esporsi al cimento, o avventurarsi a qualche pericolo. In Spagnolo
  \textit{Risco} significa, \textit{rupae pricipizio}, luogo pericoloso. Cic. se bene mi sovviene; \textit{Scio
    quam in difficile, \& scopuloso loco verser}, risicoso.
\item[TRACOLLI] Da tracollare: altrimenti barcollare, che è accennar di cadere;
  è il Latino \textit{netare}, o \textit{titubare}; e qui vuol dir Disgrazia, o pericolo.
\item[ROMPICOLLI] Huomini; che consigliano, o inducono altri a far male.
  Latino \textit{in omnem audaciam proiecti}.
\item[TESTONE] Moneta Fiorentina, che vale tre giuli, o paoli\footnote{tre paoli, due lire fiorentine, 7.32g di argento fino.}.
\item[VAL più d'un mondo] Questa iperbole significa non vi e prezzo, che lo paghi.
  \textit{Star discosto un mondo}, disse il Bronzino nelle rime burlelche; cioè \textit{grandissimo
    spazio}.
\item[CERCAR di Frignuccio] Cercar le disgrazie. Andar incontro a' pericoli, che
  Frignuccio dalle nostre donnicciuole è preso per il Diavolo, e diciamo anche
  \textit{cercar il male come i Medici}. I Latini in questo proposito dissero; \textit{Camarinam
    movere} da una pianta, la quale ha le foglie così fetenti, che movendole, o toccandole
  lasciano un puzzo terribile: o forse da una palude detta \textit{Camarina} posta
  vicina al castello detto Camarina in Sicilia, la qual palude, perché cagionava in
  detto Castello la peste, i paesani domandarono ad Apollo, se era bene far seccare
  detta palude, e l'Oracolo rispose: \textit{Camarinam non esse movendam}; ma eglino
  fatto poco conto di detta risposta, vollero seccarla, e n'hebbero il gastigo, perché
  i nimici passando per quella palude già secca, entrarono nel Castello, e sen'impadronirono.
\item[In bella prova] A posta; e l'addiettivo \textit{bella} s'usa in questi casi per emfasi, e
  per esprimere un superlativo, quasi dica \textit{in provissima}. Vedi sopra C. 3. st. 14. Così
  nell'uso: L'ho bell'e fatta questa, o quella cosa; cioè l'ho fatta fattissima; l'ho 
  terminata, fornita. 
\item[CHI cerca trova] Detto sentenzioso, che significa, che colui, che va intorno 
  al male, merita che gli succeda.
\item[NON v'è spina, ne osso] E' negozio spianato. E cosa liscia, Non vi è da dubitare,
  non ci è da incontrare difficultà alcuna.
\item[A GAMBE alzate] Cioè col capo all'ingiù. Si dice anche \textit{Andare a gambe levate}.
  Usò questa frase \textit{A gambe alzate} Ser Brunetto Latini maestro di Dante nel Pataffio,
  ovvero Capitoli pieni di gerghi, e di vocaboli Fiorentini; e volle spiegare
  l'atto di chi si accomoda in terra per iscaricare il ventre. \textit{I vidi a gambe
    alzate un che tortiva} (cioè, con riverenza, cacava) che questo vuol dire \textit{torrire} in
  lingua furbesca.
\item[CONTEGGIAVA con la morte] Faceva conto di morire. Temeva di morire
  infranto nel mulino.
\end{description}

\section{Stanza VI. --- XII.}
\begin{ottave}
\flagverse{6}Hor questo vuol pur ch'io di lui discorra, \\
Onde di nuono a i fatti suoi ritorno. \\
Le Ninfe, ch'il vedean batter la borra \\
Tutte gli son co' panni caldi attorno, \\
E già tra loro par che si discorra \\
Di fargli dare una scaldata in forno, \\
Ma perché questo in danno suo rifsulta \\
Dir volle il suo parere anch'ei in Consulta.
\end{ottave}

\begin{ottave}
\flagverse{7}Che terminò di non farn' altro; ond'esse \\
Lo feron rivestire a spefe loro; \\
Una camicia nuova una gli messe, \\
C'ha dal collo, da man trina è lavoro, \\
L'altra il giubbone, un'altra le brachesse \\
Tutto d'un ricco, e nobil quoio d'oro, \\
Un'altra gli ravvia la capelliera, \\
E gli mette il benduccio, e la montiera. 
\end{ottave}

\begin{ottave}
\flagverse{8}A spasso poi lo menan per la mano \\
A veder la lor bella abitazione, \\
Ma poi più buona, benché sia in pantano, \\
Perché a pagar non hanno la pigione, \\
La quale è un negozio odioso, e strano\\
Quando quell' insolente del padrone \\
Ti picchia a casa, e con sì poca grazia \\
Chiede il semestre ch'ei non v'è una crazia.
\end{ottave}

\begin{ottave}
\flagverse{9}Circa questo, pensiero elle non hanno,\\
Ne di fare altre spese, come accade\\
Ad ogni galant'huomo a capo d'anno\\
D'acconci, tasse, lastrichi di strade:\\
IL vento, e il freddo non può far lor danno,\\
Perch'il tetto, che scorre, e mai non cade\\
L'Inverno su i pilastri di corallo\\
Si ferma, e forma un palco di cristallo.
\end{ottave}

\begin{ottave}
\flagverse{10}Di Stare il Sole giù ne' suoi quartieri\\
Non può col frugnolone haver l'ingresso,\\
Tal ch'elle stanno bene, e volentieri,\\
E godono un pacifico possesso.\\
Paride intanto infra tazze, e bicchieri,\\
E di più forte vini, e frutte appresso,\\
Con esse ritrovandosi in cantina, \\
Volle provarne almeno una trentina.
\end{ottave}

\begin{ottave}
\flagverse{11}Ne per questo alterato egli ne resta,\\
O venga ch'egli è avverzo in Alemagna,\\
O c'a salvar quel vin faccia la testa,\\
Ed in quel cambio dia nelle calcagna;\\
Ragion, che quadra bene, e quella, e questa,\\
Perch'ei non urta mai chi l'accompagna,\\
Ma sempre in tuono, e dritto com' un fuso\\
Con esse per le scale torna suso.
\end{ottave}

\begin{ottave}
\flagverse{12}Ov'egli entrato in una bella sala,\\
Ch'ella sia l'Accademia si figura,\\
Perché vi son l'aratolo, e la pala \\
Strumenti da studiar l'agricoltura.\\
Di lì poi salgon sopr' a un' altra scala\\
Di baston congegnati infra due mura,\\
Donde, arpicando come fan le gatte\\
Vanno a passar per certe cateratte.
\end{ottave}

Di Paride dunque vuol seguitare a discorrer il Poeta, e dice, che conoscendo
le Ninfe, che egli sentiva un gran freddo, volevano metterlo a rasciugare, e
riscaldarsi in un forno, ma.egli non volle, onde esse gli fecero un vestito nuovo
a loro spese nella maniera, che viene espresso in questa Stanza settima; Di poi
lo menarono a vedere la loro abitazione, ed in cantina dove bevve assai, e non
gli fece danno per le ragioni, che adduce il Poeta; e di cantina salirono alle
stanze si sopra.

\begin{description}
\item[BATTER la borra] Intendiamo Tremare, e battere i denti per causa del
  freddo: E si dice così per la similitudine, che ha tal battimento di denti col batter,
  che si fa della borra\footnote{Vedi sopra, \cst[6]{34}.\ dove è menzionata come possibile imbottitura dei palloni; o \cst[6]{94}.\ come elemento del cappuccio.}: la quale è specie di lana triturata col coltello, e serve per
  empiere i basti delle bestie da soma, ec.\ e per liberar detta borra dalla polvere
  si mette sopra a un'asse forata con piccoli, e spessi fori, o si batte con un mazzo
  di corde adattate a questo effetto; e questo battere fa uno strepito, che ha qualche
  similitudine col batter de i denti, che faccia uno tremante per causa del freddo,
  ec. Si dice anche batter la Diana; tremare tutto, stando all'aria, a Cielo 
  scoperto; Latino \textit{sub dio}. Vedi sotto C. 9. st. 6.
\item[BRACHESSE] Brache, calzoni, Voce Veneziana talvolta usata anche da noi.
\item[QUOI d'oro] Pelli di bestie conciate, e dorate', servono per adornare le stanze
  in vece di drappi.
\item[GLI ravvia la capelliera] Gli pettina la zazzera, O chioma. \textit{Benduccio}. Da benda.
  Striscia di panno lino bianca, che s'appicca pendente alla spalla, o alla 
  cintola de i bambini, perché si possano con essa nettare il naso.
\item[MONTIERA] Specie di berretta usata dai bambini. Dallo Spagnolo
  \textit{montera}, berrettino.
\item[PANTANO] Palude, che diciamo anche padule, luogo pieno d'acqua ferma,
  che renda il terreno inzuppato, riducendolo come fango, da i Latini pure
  detto \textit{Palus, paludis}.
\item[PIGIONE] Cioè quel denaro, che si paga per fitto d'una cosa; E parlando
  con termini proprj \textit{fitto} si dice quel danaro, che si paga per poderi, e terreni, e
  \textit{pigione} si dice quel denaro, che si paga per Case, o botteghe, dicendo \textit{affittare
    botteghe, o casamenti}: Ed \textit{appigionare case, e botteghe}. Di queste si dice \textit{affittare}
  ma dei terreni mai si direbbe \textit{appigionare}. Pigione dal Latino \textit{pensionis}. Fitto 
  forse da \textit{feudum}, \textit{fio}, e questo dal Latino \textit{fides}.
\item[STRANO] Stravagante. Qui intende noioso, odioso, fastidioso. La voce
  \textit{strano} dal Latino \textit{extraneus} ritiene anche appresso di noi il significato di Straniero,
  o lontano dal parentado nostro. \textit{Viso strano}, vuol dir viso arcigno, e brutto,
  o cruccioso; \textit{viso strano} vuol dir anche faccia macilente, e pallida.
\item[SEMESTRE] Numero di sei mesi; ma intendi il denaro, che si dee per la
  pigione di sei mesi.
\item[TASSE, e lastrichi di strade] Spese, che occorrono farsi alla giornata da coloro,
  che posseggono case in Firenze; che \textit{lastrichi}, intende quella spesa, che si
  ripartisce fra i padroni delle case per rassettamento, e lastricamento delle strade
  della Città.
\item[TETTO, che sempre scorre, e mai non cade] \- Abitano sotto acqua, la quale è
  il loro tetto, che sempre scorre, e mai non cade.
\item[PILASTRI di Corallo] Pilastri si dicono quelle colonne fatte di mattoni, o d'altri
  sassi, per sostener volte. Latino \textit{pilae}. E perché il corallo nasce nell'acqua,
  finge, che questo tetto si regga sopra i pilastri di corallo, e vuol dire quando l'Inverno
  s'agghiaccia l'acqua, e si ferma. 
\item[NON può col frugnolone haver l'ingresso] Non può il Sole tramandare, o far penetrare
  i suoi raggi sotto l'acqua, Frugnolone da Frugnuolo detto sopra C. 7. st. 37.
\item[ALTERATO] Commosso, o perturbato da qualsisia accidente. Ed alterato
  dal vino vuol dir \textit{Briaco}. Onde gli Alterati Accademici già famosi in Firenze
  facevano per Impresa un Tino; in cui si pigiava l'uva, e ogni Accademico usava
  per impresa particolare cose attenenti a vino; sì come quella della Crusca,
  le succedé, usa per impresa tutte cose attenenti a grano.
\item[FACCIA a salvar la testa] Non offenda co' suoi fumi la testa, perché è vino
  debole. Detto scherzoso tratto da quelli, che giuocando di scherma non fanno
  a tutto gioco, ma pattuiscono di salvare la testa, cioè non si colpire nella testa.
\item[RAGION che quadra bene, e quella, e questa] \- Tanto può esser per questa ragione,
  che per quella, che egli non sia rimasto alterato dal tanto bere.
\item[NON urta chi o accompagna, ma è sempre in tuono] Non barcolla come fanno i
  briachi, e non dà spinte a chi è seco, ma sta in cervello, e va dritto.
\item[ARATOLO] Si dice anche \textit{aratro} dal Latino. E \textit{Arato} si trova nell'antico
  Volgarizzamento di Palladio:; donde è fatto il diminutivo \textit{Aratolo}. Strumento
  noto, col quale i villani rompono la terra, facendolo tirar da i buoi.
\item[ARPICANDO] È il verbo arrampicare sincopato, e vuol dire il salire, che
  fanno i gatti sopr'a un'albero, o simili, e viene da \textit{rampicone}, che è un ferro
  grande uncinato, che usano i marinari per pigliare, e fermar le navi. Latino
  \textit{Harpago, harpagonis}; da che noi pure lo diciamo anche \textit{arpagone}, e \textit{arpagonare}.
\item[CATARATTE] È voce latina, che vien dalla Greca \textit{catarrhactes}, con la
  quale intendiamo ancora quelle buche fatte ne i palchi, per le quali si passa di sotto
  per entrare in luoghi superiori con scala a pioli, come sarebbe salire per di
  casa in sul tetto. E per lo più tali cateratte s'usano per entrar nelle colambaie;
  e di questa sorta era la cateratta, che dice in questo luogo.
\end{description}
\section{Stanza XIII. ---  XVI.}
\begin{ottave}
\flagverse{13}Ma qui la Mula vuol ch'io mi dischiari 
Circa il descriver queste loro Stanze, \\
Che s'io vi pongo addobbi un po ordinari, \\
Non sia per dir bugie, ne stravaganze; \\
Perché le Ninfe han solo i necessari, \\
Ne voglion pompe, ne moderne usanze, 
Per insegnare a noi c'habbian le borie \\
Di quadri, e letti d'oro, e tante storie.
\end{ottave}

\begin{ottave}
\flagverse{14}Ch'ognun vuol far il Principe al dì d'oggi, \\
Se ben chi la volesse rivedere, \\
Molti si veggon far grandezze, e sfoggi, \\
Che sono a specchio poi col rigattiere: \\
Il lusso è grande, e già regna in su i poggi, \\
E son nelle capanne le portiere,\\
E tra i cannelli infin qualsivoglia unto \\
Ha i suoi stipetti, e seggiole di punto. 
\end{ottave}

\begin{ottave}
\flagverse{15}Horsù per ch'io non caschi nella pena\\
De cinque soldi; ecco ritorno a bomba\\
A Brache d'or, che nel salire arrena\\
Per quella scala, che va su per tromba,\\
Perché se bene ei fa il Mangia da Siena\\
Gli è disadatto, e pesa ch'egli spiomba,\\
E con le Ninfe a correr non può porsi,\\
Massime lì, che v'è un salir da Orsi.
\end{ottave}

\begin{ottave}
\flagverse{16}Elle di già, com'io diceva adesso\\
Uscite son di sopra a stanze nuove,\\
Aspettando, che faccia anch'ei l'istesso,\\
C'appunto com'il gambero si muove;\\
Onde convien poi loro andar per esso,\\
Ed aiutarlo fin, che piacque a Giove,\\
Che quasi manganato, e per strettoio\\
Passasse ad alto il Cavalier di quoio.
\end{ottave}


Protestandosi l'Autore di voler dire la verità, prega il Lettore a non pigliare
ammirazione, se in descrivere le masserizie delle Ninfe metterà addobbi, ed arnesi
un poco ordinarj, perché in effetto eran così; e da questo piglia occasione di
biasimare il lusso, che è oggi in Firenze. Di poi tornando a proposito dice,
che le Ninfe salirono alle stanze di sopra, dove con gran fatica fecero salir
Paride, il quale chiama il Cavalier di quoio, perché era vestito di quoio, come s'è
detto.

\begin{description}
\item[ADDOBBI] Masserizie, ed arnesi per uso, ed ornamento delle stanze: dal
  verbo \textit{addobbare}, che vuol dire Adornare. Du Fresne nel Glossario infimae \&
  mediae Latinitatis. \textit{Addobbare}\footnote{In realtà, Du Fresne riporta 'Adobare', senza doppie.}\textit{, armis instruere, militare cingulum alicui conferre, vox
    confecta ex adoptare, quod qui aliquem armis instruit, ac militem facit, eum quodammodo
    adoptet in filium}, sì che Addobbare secondo questo autore viene dall'antica 
  solennità del vestire i Cavalieri.
\item[BORIA] Albagia. Vanagloria.
\item[SFOGGI] Usanze sontuose tanto di vestire, quanto d'addobbamenti di casa
  fatti con splendideza, e più del consueto; Donde si dice \textit{fare sfoggio}, o \textit{sfoggiare}
  quando i frutti fanno quantità grandissima di frutte, o quando chi che sia lavora
  più del solito; ed in somma s'intende d'ogni operazione, che esca del consueto,
  o del naturale; come si dice frutta \textit{sfoggiata} quella, che eccede in grossezza, ed in
  bellezza, e supera l'altre frutte della sua specie. E la forza della lettera \letter{s} e 
  venendo da foggia, cioè usanza, al solito, antepostavi l'\letter{s} vuol dir fuori della
  foggia, cioè fuor del solito, e del consueto. Gio. Villani quel che noi diremmo
  foggi, chiama \textit{disordinati ornamenti} lib. 9. C. 245., e \libcap[10]{10}. Il medesimo
  autore \libcap[12]{4}. \textit{E non è da lasciare di fare memoria d'una sformata
  muntazion d'abito}, che ci recaro di nuovo i Franceschi. E poco sotto, \textit{Come per
    natura siamo disposti noi vani cittadini alle mutazioni de' nuomi abiti, e istrani
    contraffare}. Sfoggio dunque vale fuori di \textit{foggia}, cioè della fazione, o vogliam dire
  maniera di fare ordinaria, e usitata; che il Villani come s'è visto chiama
  \textit{sformata mutazione d'abito; e disordinati, e sconvenenoli, e disonesti, e soperchi
    ornamenti, e nuovi, e istrani abiti}.
\item[CHI la volesse rivedere] Cioè chi la volesse bene esaminare, o ricerchare in che
  maniera questi tali possano fare simili sfoggi.
\item[SONO a specchio] Hanno debito. Traslato da coloro, che hanno debito alle
  decime, che si pagano al Principe, i quali si dice esser' a specchio, perché sono
  notati a un libro, che si chiama lo specchio. Qui dicendo: \textit{sono a specchio col righattiere},
  dà due colpi, uno che costoro, che fanno tante borie non l'hanno pagate,
  l'altro, che questi loro sfoggi sono di robe usate, e vedute, altrove,
  poiché l'ha prese dal \textit{rigattiere}, che vuol dire Uno, che vende masserizie vecchie,
  ed abiti usati. Vedi sopra C. 3. st 5.
\item[PORTIERA] Paramento di drappo, o d'altro, che serve per mettere alle 
  porte delle stanze nelle case Civili. Da alcuni detta in Latino \textit{velum admissionale}.
\item[TRA i cannelli] Vuol dire fra la gente più vile; perché \textit{fra i cannelli} intendiamo
  fra i tessitori di lana, che son gente d'infima plebe, ed è lo stesso, che dire:
  \textit{qualsivoglia unto}; perché questi tali maneggiando sempre lane unte, son'ancor'essi
  sempre unti; e qui aggiungendo al detto \textit{fra i cannelli}, il detto \textit{qualsivoglia unto},
  intende, che fino i Battilani, che fra gli unti sono i più vili fanno le foggie.
\item[SEGGIOLE di punto] Cioè seggiole ricamate, o trapuntate di seta, che diciamo:
  \textit{Punto Unghero}, o \textit{punto Franzese}.
\item[CASCAR nella pena de' cinque soldi] Quand'altri nel discorso fa una digressione,
  e non torna mai al primo proposito, gli diciamo: \textit{Voi cascherere nella pena
    de' cinque soldi}. Il Varchi nel suo Hercolano parlando di questa pena dice: \textit{E chi
    haveva cominciato alcun ragionamento, e poi entrato in un'altro, non si ricordava più di
    tornare a bomba, e fornire il primo, pagava già, secondo testimonio dal Burchiello, un
    grosso, il qual grosso non valeva per avventura in quel tempo più di quei cinque soldi, che
    si pagano oggidì} Nelle quali parole vegghiamo, che Il Varchi si serve del detto
  \textit{Tornare a Bomba} per tornare a segno, o al proposito del primo discorso, come fa
  il nostro Autore nel presente luogo. L' Ariosto Satira prima dice;
  \begin{verse}
    \backspace Ma perché i cinque soldi da pagarte,
    Tu che leggi, non ho, ritornar voglio
    La mia favola, donde ella si parte.
  \end{verse}
\item[ARRENA] Intoppa; Si ferma; Non seguita il viaggio. Traslato dalle navi
  quando si fermano, perché toccano il letto dell'acqua, che si dice \textit{arrenare}, o
  \textit{incagliare}. De i quali verbi ci serviamo per esprimere non tanto il fermarsi in un
  viaggio, quanto il fermarsi in un discorso, o nel proseguimento di qualsivoglia
  azione, negozio. Latino \textit{haerere}.
\item[FA il mangia da Siena] Fa il bravo. Fa il valoroso. Il Mangia da Siena è
  una statua di metailo assai grande, la quale è posta sopra la Torre dell'oriuolo
  del Comune di quella Città, la qual figura dicono, che sia il simulacro d'uno antico
  huomo bravo detto il Mangia; Ma io son d'opinione, che ella sia il simulacro
  di qualche antico Podestà di Siena, e che habbia acquistato il nome di \textit{Mangia}
  da qualche inscrizione, che havesse appresso, la qual dicesse Il \textit{Magna di Siena},
  cioè il \textit{Magnifico di Siena}, che s'intendeva già il Podestà: Ma sia com esser
  si voglia, a noi basta sapere, che questo detto serve per in tender con derisione
  un bravo, o valente; quasi voglia mangiare le persone, e ingoiarle,
\item[DISADATTO] Contrario d'Atto, destro, agile, ec, Uno che duri gran
  fatica a maneggiarsi, o muoversi per la gravezza, o per altro accidente. \textit{Sciatto}
  ancora è contrario di \textit{atto}, e significa uno, che fa male, o negligentemente
  quel ch'e' fa; poco pulito nelle sue faccende, e nella persona.
\item[CON le Ninfe a correr non può porsi] Non può gareggiare con le Ninfe a chi più
  corre. Intende, che le Ninfe al sicuro lo supererebbono nel corso.
\item[V'È un salir da Orsi] V'è cattivo, o difficil salire. L'Orso è un'animale, che
  se ben, par goffo, e disadatto, nondimeno è assai destro, e facilmente sale anche
  in luoghi inaccessibili; donde noi habbiamo: \textit{Esser come L'Orso}, cioè \textit{goffo, e destro}.
  Il Berni nel Cap, al Fracastoro dice:
  \begin{verse}
    \backspace    Conviene ivi lasciar l'usato corso,
    E salir su per una certa scala,
    Dove havria rotto il collo ogni destr'Orso.
  \end{verse}
  Omero nell'Iliade al nono chiama una rupe, o balza \textit{AEgilips}, cioè \textit{dalle capre
    abbandonata}; o questo medesimo nome di \textit{AEgilips} danno gli antichi a una Città
  dell'Isola di Cefalonia, e un'altra dell'Epiro. Noi diciamo di luoghi simili erti ripidi,
  o scoscesi: \textit{Non vi salirebbero le capre}, le quali Virgilio nell'Egloghe disse:
  \textit{pendentes rupe}. Quella montagna altissima nell'India; su la quale fu il primo
  Alessandro Magno a salire, fu detta da' Greci \textit{Aornos}, cioè senza uccelli, quasi
  montagna da non potersi ne anche da chi avesse l'ale sormontare.
\item[SI muove come il gambero] Cioè va all'indietro. \textit{Nepam imitatur} disse Plauto.
\item[MANGANATO] Infranto; Mangano (dal Greco magganon) è una macchina,
  con la quale si distendono, e si da il lustro a i panni, e drappi facendogli passare
  a forza di rulli sotto un gravissimo peso, e tal panno, o drappo così passato
  si dice poi manganato. E Mangano come s'accennò sopra C. 6. st. 86. è una macchina
  militare della quale i nostri antichi si servivano per scagliar pietre nelle Città
  assediate, e con essa scagliavano anche huomini, che dicevano poi cadaveri \textit{manganati},
  cioè sflagellati, e pesti dalla percossa; e così si potrebbe intendere di Paride;
  ma perché soggiunge \textit{passato per strettoio}, che è un'altra macchina, che serve
  per stringer ulive, ec., o per mettere in piega a panni, si vede, che intende di
  quel mangano da panni.
\end{description}
\section{Stanza XVII. --- XXI.}
\begin{ottave}
\flagverse{17}N' un Dormentorio grande, ma diverso,\\
Ove ciascuna in proprio ha la sua cella,\\
Che sta com' io dir per questo verso,\\
(Se non erra Turpin, che ne favella)\\
Una stanga a mezz'aria evvi a traverso, \\
Dov' alla tien le calze, e la gonnella, \\
Il penzol delle sorbe, e del trebbiano,\\
E quel che più le par di mano in mano.
\end{ottave}

\begin{ottave}
\flagverse{18}Più giù da banda un tavolin si vede,\\
Che su i trespoli fa la ninna nanna, \\
E fa spalliera al muro, ove si vede \\
Una stuoia di giunchi, e sottil canna, \\
Evvi una madia zoppa da un piede, \\
E il filatoio con la sua ciscranna, \\
Non v'è letti, se non un per migliaio, \\
Che tutte quante dormono al pagliaio.
\end{ottave}

\begin{ottave}
\flagverse{19}Paride guarda; e par che gliene goda,\\
Che la gente alla buona, e positiva\\
Sempre gli piacque, e la commenda, e loda.\\
In questo mentre a un'altra porta arriva,\\
E nel sentire un certo odor di broda,\\
Che tutto lo conforta, e lo ravviva,\\
Entra di punta, perché s'indovina,\\
Che quella sia senz'altro la cucina.
\end{ottave}

\begin{ottave}
\flagverse{20}Dal che sentitosi allegare i denti,\\
Si pensa, che vi sien grand'apparecchi,\\
Ma trova in ozio tutti gli strumenti,\\
E i piatti ripuliti come specchi,\\
Teglie, e padelle, inutili ornamenti,\\
Star'appiccate al muro per gli orecchi,\\
Ed anche son per starvi più d'un poco,\\
Perché il gatto a dormir vede in sul foco.
\end{ottave}

\begin{ottave}
\flagverse{21}Ond'egli offeso molto se ne tiene,\\
Ch' una mentita per la gola tocca;\\
Ma quelle, che s' avveggon molto bene,\\
Ch'egli ha l'arme di Siena impressa in bocca\\
Gli accannan ch'ei vedrà se 'l corpo tiene,\\
Ed ei ghignando allor più non balocca,\\
E con esse ne va di compagnia,\\
Per ultimo a veder la Galleria.
\end{ottave}


Descrive nelle presenti Ottave il dormentorio delle Ninfe, e le loro masserizie.
Arrivano alla cucina, dove Paride resta scandolezzato, perché non vi vede
preparata cosa alcuna per mangiare; Ma le Ninfe lo quietano con dirgli, che non
mancherà da mangiare, ed intanto lo conducono a veder la Galleria.
\begin{description}
\item[DIVERSO] Differente, o dissimile agli altri Dormentorj, perché in questo le
  Celle non son fatte di muraglia, ma son tutte in una grande stanza, distinte e
  divise con stanghe appiccate al palco ciondoloni attraverso a mezz'aria, sopr'alle
  quali ponendo ciascuna le sue robe, e panni le fa servire per muro divisorio, e 
  così vengono formate le Celle. Si può anche dire, che la voce \textit{diverso} havendo 
  due  significati il primo, che vuol dire differente (e questo segue allora, che è 
  messo per contrapposto, come la tal cosa è diversa dalla tale) il secondo quando
  è posto assolutamente, che vuol dire strano, o stravagante, il Poeta lo piglia in
  questo secondo significato. come lo pigliò Dante Inf. C. 7.
  \begin{verse}
    Entrammo g pen via diversa, \&c.
  \end{verse}
  Il Cavalcanti nelle sue storie lib. 12. parlando di Cammillo quando distese il
  Campidoglio dice:
  \begin{adjustwidth}{8pt}{}
    Non guardò all'ingiusto cacciamento, ma con grandissimo
    esercito corse alla difesa della patria, e liberolla da così diversa fortuna.
  \end{adjustwidth}
  Ricordano Malesp. Stor, Fior. cap. 80. dice:
  \begin{adjustwidth}{8pt}{}  
    E ciò fu per l'invidia della Signoria, che non era al loro volere, e fu diversa, ed aspra guerra.
  \end{adjustwidth}
  Vedi sopra \cstan[2]{3}.
\item[PENZOL del trebbiano] Che cosa intendiamo per penzolo vedemmo sopra \cstan[6]{50}.
  e \textit{Trebbiano} è specie d'uva bianca, ma qui è preso in generale  per
  ogni sorta d'uva, che s'appicca nelle stanze per serbare all'Inverno.
\item[DI mano in mano] Di tempo in tempo. Lat. \textit{Deinceps}, che s'intende \textit{successivo
  ordine}.  Cic. 7. Ep. Fam. disse \textit{De manu in manum}. Dan. Par. 6. dice:
  \begin{verse}
    E sotto l'ombra delle Sacre penne
    Governò il mondo lì di mano in mano.
  \end{verse}
  Ed è  detto figuratamente dal far passaggio una cosa dalla mano d'uno nella
  mano dell'altro. Dal giuoco detto \textit{Lampade dromiæ}\footnote{Nel testo è riportato come due vocaboli, ma il Minucci intende il greco $\lambda\alpha\mu\pi\alpha\delta\eta\delta\rho{}o\mu\iota\alpha$, a cui tutta la descrizione è relativa. }, nel quale colui aveva il
  vanto che va una fiaccola accesa correndo, e così bella, e accesa la consegnava
  a chi aveva a correre dopo di lui; disse Lucr.~lib.~2. \textit{Augescunt altae gentes,
aliae minuuntur, Inque brevi spatio mutantur sæcla animantum, Et quasi cursores vitæ
lampada tradunt}, cioè \textit{succede l'uno uomo all'altro, l'uno vivente all'altro di mano in mano}.
\item[TRESPOLO] Dal Lat. \textit{tripus, -odes}. È un pezzo di legno, o ceppo, in cui
  son fitte tre mazze, sopr' alle quali posando, serve per sostener tavole, e deschi,
  da i Latini detto \textit{Trapezophorus}, quasi \textit{mensam ferens}.
\item[FA la ninna nanna] Non sta forte in terra, ma dimena o per l'inegualità delle
  tre mazze, o del suolo, o per altro mancamento; e diciamo \textit{far la ninna nanna}
  da quel dimenare, che si fa della culla de i bambini, quando dalle balie si procura,
  che dormano, che si dice \textit{Ninnare}, perché per lo più sogliono accompagnare
  tal moto con una lor cantilena, che dice \textit{Ninna nanna il mio bambino}. Vedi sopra
  \cstan[6]{25}. Questo dimenare si dice anche \textit{cullare} pur dalla culla de' bambini.
\item[SPALLIERA] Quella parte della seggiola, alla quale s'appoggiano le spalle
  sedendo: E per spalliere intendiamo quelle muraglie,  alle quali sono appoggiate
  piante d'agrumi, ec.\ come s'è detto sopra \cstan[6]{51}. Questo artifizio di
  parare le mura colle piante dicesi da alcuni in Lat. \textit{opus topiarium}. E qui intende
  quel muro parato di stuoie fatte di giunchi, o canne palustri, che sovrasta alla
  panca, sopr'alla quale dice, che sedevano le Ninfe, e serve, per spalliera alla
  medesima pancha.
\item[STUOLA] È il Latino \textit{Storea} che conserva appresso noi il suo significato.
\item[MADIA] Dal Latino \textit{mactra}, il qual pure è Greco; ed una cassa adattata
  sopra quattro piedi, dentro alla quale si lavora la patta per far il pane; La dice
  \textit{Zoppa da un piede} perché le mancava, o era rotto uno di questi piedi. \textit{Zoppa}
  similmente da un piede era la tavola della vecchierella Baucide la presso Ovidio
  lib. 8. delle Trasformazioni; ma ella la fece stare pari con metterci sotto un ciotto;
  \textit{mensam succincta, tremensque Ponit anus; mensae sed erat pes tertius impar.
    Testa parem fecit}.
\item[FILATOIO] Strumento col quale per via d'una gran ruota si fila lana,
  canapa, ec, e si fanno le funi.
\item[CISCRANNA] Specie di seggiola come accennammo sopra \cstan[6]{7}.
\item[DORMONO al pagliaio] Cioè dormono in su la paglia.
\item[HVOMO alla buona], Huomo schietto, sincero, e senza malizia; Huomo senza
  cirimonie, e nimico del lusso, e delle borie \textit{sine fuco, \& fallacijs}, \textit{more
    maiorum}, ed \textit{Huomo positivo} intendiamo uno, che non fa sfoggi nel veltire, e che in
  ogni cosa si tratta senza lusso.
\item[SENTITOSI allegare i denti] Vuol dite sentitosi stimolare dalla gola, e dal
  desiderio di mangiare; se bene allegare i denti vuol dire quando i denti per haver
  masticata qualcosa acida, o agra come il limone, ec.\ s'intormentiscono, e si
  sente una certa difficultà nel masticare. Ma usandosi come nel presente luogo, vuol
  dir venir voglia di mangiare.
\item[TEGLIA] Specie di tegame fatto di rame stagnato per di dentro, serve per
  quocervi torte, e migliacci\footnote{Migliaccio, Focaccia salata a base di farine varie e miste, ed uva passa. Il castagnaccio ne è un esempio. Da miglio, cereale.}, ec. Il Monosini lo fa venire dal Greco \textit{Telia}, la
  qual voce tra l'altre cose significa l'\textit{asse da pane}, e 'l turacciolo, o coperchio del
  fummaiuolo, o vogliam dire di quel canale, che gli antichi, in vece di cammino, avevano
  per servizio di cucina, buono solo a ricevere, e portar via il fummo. Ma
  dicendolo molti \textit{Tegghia}, e gli antichi in particolare, mi muovo a credere, che
  venga più costo dal verbo Latino \textit{Tegere}. Queste teglie hanno nell'orlo appiccata
  una campanella di ferro per comodità d'appiccarla, e le padelle hanno un'anello
  in cima al manico per il medesimo effetto; e questi sono gli \textit{orecchi} de' quali
  parla il Poeta dicendo: \textit{Stanno appiccate al muro per gli orecchi}. Ovidio lib.8.
  Metam. \textit{erat alveus illic Tagineus, dura clavo suspensus ab ansa}.
\item[TORNIRE] Parlando di gatti s'intende quel ronfare che fanno; perché è simile
  a quel romore, che fa il tornio quando gira.
\item[TOCCA una mentita per la gola] Dar una mentita per la gola a uno è quando
  se gli dice, che egli afferma il falso, ed è grandissima ingiuria, e che muove ad ira;
  e però il Poeta scherzando dice, che Paride si adira per l'offesa, che riceve di
  quella mentita per la gola, cioè di quel supposto che vi fusse roba per la gola,
  che fu falso.
\item[L'arme di Siena impressa in bocca] L'arme di Siena è una Lupa, ed il mal della
  lupa è inteso comunemente per una infermità, che fa stare il pazziente in continova
  fame; onde quando vogliamo intendere; il tale ha gran fame diciamo:
  \textit{Egli ha il mal della lupa}, e più copertamente \textit{Egli ha l'arme di Siena}, e s'intende
  la lupa, cioè la fame. Vedi sopra \cstan[3]{22}.
\item[VEDRÀ s'il corpo tiene] Cioè mangierà, e berà. Detto assai usato dalla
  gente di vil condizione.
\item[GHIGNANDO] Ridendo leggiermente. Lat. \textit{subridere}.
\item[GALLERIA] Così in voce straniera chiamiamo alcune stanze piene, ed adornate
  di galanterie, e di cose singolari, e maravigliose,  quali stanze da i Latini
  son dette \textit{Pinacotheca} dal Greco \textit{Pinax}, che suona \textit{tabula picta}, e \textit{theca}, luogo per 
  riporre alcuna cosa. E per altro Galleria voce militare è specie di fortificazione.
\end{description}

\section{Stanza XXII. --- XXV.}
        
\begin{ottave}
\flagverse{22}Di Maiolica nobil di Faenza\\
Ivi le foglie sono, e i frontespizzi,\\
Quivi son quadri di gran conseguenza\\
Di Principi ritratti, e di Patrizzi,\\
Originali farti già in Fiorenza\\
Da quel, che ggli vendea sotto gl'uffizi,\\
Ed evvi dello stesso una Sibilla,\\
Ed una bella Cittadina in villa.
\end{ottave}

\begin{ottave}
\flagverse{23}Di cartapesta mensole, e sgabelli\\
Intorno intorno inalzan sopra al piano\\
Statue eccellenti di quei Prasitelli\\
C'a i sassi danno il moto in Settignano,\\
Cedano i Buonarroti, e i Donatelli\\
A quel basso rilievo di lor mano\\
C'a i padri scalzi pur si vede ancora\\
Su l'arco della porta per di fuora.
\end{ottave}

\begin{ottave}
\flagverse{24}Sì che quest'opre, che non hanno pari,\\
Quanto i suddetti quadri, c'han del vago\\
Non si posson pagar mai con danari,\\
Perché son gioie, che non hanno pago;\\
Uno scaffale v'è di libri vari,\\
Ch'eran la libreria di Simon Mago,\\
C'abbellita di storie, e di romanzi\\
Fu poi venduta lor dal Pocavanzi.
\end{ottave}

\begin{ottave}
\flagverse{25}Evvi un tomo fra gli altri scritto a penna,\\
C'a me par bello, e piace fine fine,\\
Ove si legge in carta di cotenna\\
Tradotte le librettine in sestine,\\
E che Galeno, e il medico Avicenna\\
In musica mettean le medicine;\\
Però, s'il corpo sempre a chi le piglia\\
Gorgheggia, e canta, non è meraviglia.
\end{ottave}


L'Autore dà principio a descrivere la Galleria delle Fate, e narra la bellezza
d'alcune pitture, e statue non dissimili dal resto delle masserizie, per esser'opra
de più scimuniti Artefici, se bene scherzando gli esalta sopra i più eccellenti
maestri. Oltre alle pitture v'è anche \textit{uno scaffale pieno di libri} dei medesimo valore,
e stima, che sono le pitture, e scolture.

\begin{description}
\item[FRONTESPIZZI] Vedi sotto \cstan[9]{15}.
\item[MAIOLICA] Specie di piatti, ed altri vasellami di terra, la quale meglio;
  che in altri luoghi si lavora oggi in Faenza; e questa terra è detta maiolica dall'Isola
  di \textit{Maiorica}, o \textit{Maiorca} dove già si fabbricava; e l'Isola che diciamo oggi
  \textit{Maiorca} già si diceva \textit{Maiolica}, come si vede in Gio: Villani \libcap[4]{30}. \textit{Negli
  anni di Cristo 1117. Gli Pisani feciono una grande armata di Galee, e Navi, ed
  andarono sopr' all'Isola di Maiolica}. E che in questa Isola si fabbricassero tali vasellami
  si deduce, non solo dal nome, che ritengono di Maiolica, ma anche dal
  vedersi nelle fabbriche antiche di Pisa, e particolarmente nelle facciate delle
  Chiese murati di tali piatti come per trofeo, e memorie delle vittorie havute da
  i Pisani contro a i Maiorchini.\footnote{All'epoca della spedizione pisana, e fino alla conquista aragonese del 1230, il \textit{Regnum Maioricæ} delle Isole Baleari era un regno islamico indipendente. Non è infrequente trovare piatti con iscrizioni islamiche sulle pareti esterne delle chiese pisane. }
\item[UNA bella Cittadina in villa] Era già in Firenze un Pittore da pochi soldi, il
  quale faceva ritratti di Principi, di donne Fiorentine in abito da Villa, e da Città,
  de Sibille, le Muse, ec, e tutto così malfatto, che non eran comprate tali
  pitture se non da genti di contado, e per vilissimo prezzo. Dette pitture si vendevano
  sotto le Logge, che sono d'avanti a quelle stanze, dove si radunano i Magistrati
  di Firenze, e questo luogo si dice \textit{sotto gli Ufizzi, e per una bella Cittadina in
    Villa, e una Sibilla} intende di queste belle pitture.
\item[DI quei Prasitelli] Di quelli Scultori valorosi, e celebri, come fu Prasitele; 
  parla però ironicamente, e per derisione. \textit{Prassitelle} detto poeticamente come
  \textit{Anniballe}, \textit{Ettorre}, e simili per in rima,in vece di \textit{Prassitele}, \textit{Annibale}, \textit{Ettore}.
  Così i Latini raddoppiarono la Lat. in \textit{Relligio}, \textit{Relliquias} a ciò concedendogli
  la legge del verso.
\item[CHE a i sassi danno il moto in Settignano] Dar il moto a i sassi, ed animare i sassi
  vuol dire Formar figure di pietra: Virg. \textit{vivos ducent de marmore vultus}. In
  Settignano Borgo vicino a Firenze abitano quasi tutti scarpellini, i quali se bene
  fabbricano poco altro che stipiti, scaglioni, ed altre pietre per uso di fabbriche
  di case, ec, talvolta lavorano anche delle figure, ma per lo più belle come 
  le suddette pitture; e però il Posta scherzando dice, \textit{danno il moto a i sassi}, e par
  che voglia dire animano i sassi, fabbricando statue, che paiono vive, ed intende,
  che danno il moto ai sassi, cioè gli muovono, ed estraggono dalle cave, le
  quali sono in quei monti di Settignano, luogo detto così quasi \textit{Septimianum}, podere,
  o possessione della casa \textit{Septimia}, antica Romana, siccome \textit{Petrognano},
  della \textit{Petronia}, e altri molti luoghi dello Stato, che ritengono ancora il nome de'
  padroni, nobili Cittadini dell'antica Roma.
\item[QUEL basso rilievo di lor mano, \&c.] Perché si possa conoscere di che qualità
  erano queste statue, porta l'esempio d'una figura, che è nell'architrave della
  porta della Chiesa di S. Paolo de i Carmelitani Scalzi, che è una figura fatta di
  basso rilievo, la quale rappresenta, o almeno dovrebbe rappresentare un S. Paolo,
  ma è lavorata così maravigliosamente male, che s'è resa celeberrima per la 
  sua storpiataggine; ed è compagna delle stupende pitture del Famoso Lombardo
  Zannino da Campugnano. Intendendo dunque il nostro Poeta di questa, e d'altre
  figure, che le sono attorno fatte della medesima maniera vuol dire, che le statue
  che si vedevano in quella Galleria eran malissimo fatte.
\item[NON hanno pago] Non hanno prezzo. È parlare ironico, e vuol dire non
  hanno prezzo, cioè non s'apprezzano, non si stimano, non vaglion nulla.
\item[SCAFFALE] Armadio aperto fatto a palchetti per uso di tener libri. Col
  nome di \textit{Schapha}, e di \textit{Scaphos} si dicono in Greco molti arnesi, e strumenti, ma
  tutti o concavi, o scavati per uso di tener roba, dal verbo \textit{scaprein}, che vuol dire
  \textit{cavare}, \textit{scavare}, Onde \textit{scaffale}, arnese, che ha varie capacità, e spartimenti,
  ne' quali si ordinano, e si pongono i libri Lat. \textit{pluteus armarium}.
\item[SIMON Mago] Fu l'Autore, e capo de' Simoniaci; essendo stato il primo,
  che tentafie di comprar da S. Piero i beni Sacri, e Spirituali, come si legge negli
  atti degli Apostoli. E che cosa sia Mago, Vedi sopra \cstan[1]{20}.
\item[POCAVANZI] Fu un Libraio Fiorentino così detto, ii quale nel tempo, che
  l'Autore compose la presente Opera era ridotto in poverta, e vendeva poc'altro,
  che leggende.
\item[CARTA di cotenna] Intende Cartapecora.
\item[LIBRETTINE] Quel libretto, che insegna conoscere le figure dell'Abbaco,
  e le prime regole del medesimo. Il Burchiello. \textit{Vedilo andar, ch'e' par delle
    librettine}, cioè è tanto magro, secco, ch'e' pare \textit{una figura d'abbaco}. I 
  Latini un macilente, estenuato, e deforme nello stesso modo chiamavano \textit{Monogrammo},
  cioè delineato solamente, e fattovi il solo, e puro dintorno, senza carne
  o colorito.
\item[MEDICINA] Quando si dice semplicemente \textit{medicina} da noi s'intende quella
  bevanda solutiva, che si beve con la preparazione, o disposizione del corpo fatta
  prima con alcuni sciloppi, ec.
\item[GORGHEGGIARE] È termine musico da i Lat. detto \textit{Vibrissare}, ed è un trillo
  di voce fatto con la gola, al quale in un certo modo è simile quel romore, che
  fa nel corpo il vento, o altra sollevazione d'umori cagionata dalla medicina,
  ed il Poeta intendendo di questo romore, che fa il corpo dice, che il pazziente
  può far di meno di non cantar così, poiché Galeno, ed Avicenna havevano
  messo in musica tali medicine.
\end{description}
\section{Stanza XXVI.}
\begin{ottave}
\flagverse{26}Un ve n'è in rima, che la Sfinge è detto \\
Scelta d'Enigmi, che non hanno uguali, \\
Perch' ognuno è distinto in un sonetto, \\
Che il Poeta ha ripien tutto di sali;\\
Perch'ei, che sa, che è Sale hebbe concetto,\\
Acciò che i versi suoi sieno immortali,\\
E i vermi dell'obblio non dien lor noia\\
Porgli fra sale, e inchiostro in salamoia.
\end{ottave}

Fra questi libri delle Fate si trova anche la Sfinge, che è una scelta d'Indovinelli
distinti ciascuno in un sonetto, opera del sig.\ Antonio Malatesti\footnote{Antonio Malatesti, \textit{La Sfinge - enimmi}, in Venezia, presso il Sarzina, 1640.}; la quale
il nostra Poeta (facendo di essa quella stima che merita) non haverebbe messa
fra queste leggende, se il medesimo Malatesti non l'havesse forzato a farlo, componendo
egli medesimo la presente Ottava non alterata punto dal nostro Poeta.
E perché tale Opera contiene (come habbiamo detto) Indovinelli, il Malatesti
il nome di Sfinge, che fu un Mostro appresso a Tebe, Figliuolo (secondo Igino)
del Gigante Tifone, e di Echidna, che significa Vipera; e Fratel carnale,
secondo il medesimo, della spaventosa Gorgone, del Can Cerbero, del Serpente
di più teste chiamato Idra, e di più altri mostri e animalacci, il qual mostro dimorava
in un monte contiguo a Tebe sopr'ad uno scoglio vicino alla strada, ed
a chiunque passava proponeva \textit{un dubbio} (che i Greci dicono enigma, i Latini
\textit{gruphus} pure dal Greco; e noi \textit{indovinello} come s'è detto sopra \cstan[6]{34}.) e
se quel tale non lo scioglieva, il mostro improvvisamente lo pigliava, e l'uccideva.
Accadde, che Edipo figlio di Laio Re di Tebe fu quivi mandato, ed il Mostro
gli propose: Qual'era quell'Animale, che da principio andava con quattro
piedi, poi con due, ed in ultimo con tre. Edipo rispose, questo esser l'huomo,
che da bambino va carponi con le mani, e co i piedi, e così con quattro piedi,
se poi ritto in su due piedi, ed in vecchiaia con tre, perché va col bastone; E con
tal soluzione vinse il mostro, che percio si morì.
\item[RIPIENO di sali] Ripieno di belli, ed arguti pensieri. I Latini ancora chiamavano
  sali l'arguzie, trovandosi in Orazio. \textit{Nostri proavi Plautinos landavere
    sales}. Giusto Lipsio Antig. lect. \textit{dicit se amare elegantes Plauti sales}. Lucano: \textit{Non
    soliti lusere sales}. Ter. in Eun. \textit{Qui haber salem, qui in te est}, intende scienza,
  sapere. Ma qui l'Autore scherzando con l'equivoco del sale dice: Che il Malatesti,
  il quale sa che cosa è il sale, e che effetti partorisca (perché egli era guardiano
  de i Magazzini del Sale di Firenze) sia messo de i sali ne i suoi sonetti, per
 far loro una salamoia con l'inchiostro, affinché i suoi versi si conservino, e si
 difendano da i tarli della dimenticanza, sapendo, che il sale conserva, o difende
 dalle putedrini; e le composizioni si conservano da i vermi dell'obblio con
 scriverle, e questo si fa con l'inchiostro, e pero lo chiama \textit{salamoia}. I Latini
 dicono la salamoia \textit{Muria}, del che noi componghiamo la voce \textit{salamoia}, quasi
 \textit{salis muria}. L'inchiostro da Monsignor Ciampoli\footnote{Giovanni Battista Ciampoli (Firenze, 1589 --- Jesi, 8 settembre 1643) presbitero, poeta, umanista. } fu chiamato dal conservare le
 memorie e i nomi degli huomini \textit{Balsamo della fama}.
\section{Stanza XXVII.}
\begin{ottave}
\flagverse{27}Altri Poemi poi vi sono ancora, \\
Ed hanno caparrato alla Condotta \\
Grillo, il Giambarda, Ipolito, e Dianora \\
I sette Dormienti, e Donna Isotta;\\
E un certo Malmantil, che se e' va fuora\\
Ecco subito bell'e messe in rotta\\
Le Dee col Bambi, che l'ha chesto, e vuole\\
Fare all'acciughe tante camicioule.
\end{ottave}

Narra che molt'altri Poemi sono in detto scaffale, e mette tutte leggende, e
frottole composte da' Ciechi per le donnicciuole, e per i fanciulli. Fra queste
leggende dice, che sarà ancora la presente sua opera.
\begin{description}
\item[INCAPARRATO] Data la caparra cioè dato danari innanzi per fermar una
  mercanzia per conto proprio. (Voce formata, dice il Ferrari, da \textit{cape arrham}.
  Qui vuol dire che hanno chiesto il MALMANTILE, Gli antichi dissero \textit{Innarrare}
  da \textit{Arra}, caparra\footnote{Il Papía riporta \textit{Arra} senza \letter{h}, ma nelle moderne espressioni legali si scrive più comunemente \textit{Arrha} con \letter{h}. \textit{Inarrare} è presente nei dizionari italiani moderni per essere stato utilizzato dagli ``antichi'' Petrarca ed Ariosto.  }.
\item[ALLA Condotta] Così è chiamata a Firenze una strada, nella quale hanno le
  botteghe i Librai, e alcuni Stampatori, ed è così appellata, perché nella medesima
  strada haono i magazzini coloro, che tengono i muli per la condotta delle
  mercanzie a Roma, a Bologna ed altrove.\footnote{
Provo a sciogliere i riferimenti\\\textit{Grillo}: il Padre Reverendo Angelo
Grillo, medico, ``il quale fiorì negli An.\ 1100.'', ``che sana, e non si sa
come'' (La Biblioteca Aprosiana, in Bologna, per li Manolesi,
1677.);\\ \textit{il Giambarda}: Frottola di tre vecchi, uno de' quali è
chiamato Giambarda, l'altro Nardo, il terzo Nofrone, secolo XVI, in altre
edizioni s'intitola la Comedia del Giambarda;\\ \textit{Ipolito e Dianora}:
Historia di Ipolito Buondelmonti e Dianora de Bardi cittadini fiorentini,
Stampata in Firenze, alle Scalee di Badia, 1618. La versione fiorentina di
Romeo e Giulietta, con lieto fine;\\ \textit{I sette Dormienti}: dalla
Legenda Aurea;\\ \textit{Donna Isotta}: Isotta Nogarola, (Verona 1418 --- ivi 1466), scrittrice, umanista.
``Nobilissima cittadina, non men dotta ed
eloquente, che pudica e bella […] scrisse un Dialogo, chi più peccasse,
Adamo, o Eva.'' (Girolamo della Corte, Istorie della città
di Verona, in Venezia, presso Savioli e Camporese, 1744). La sua fama fu
per secoli vittima di pregiudizi di genere e pessime scelte editoriali.
 Più recentemente è stata molto rivalutata.
``Isottae Nogarolae Veronensis opera quae supersunt omnia'', pubblicate a Budapest nel 1886.
}
\item[MESSE in rotta le Dee col Bambi] Il Bambi era uno, che vendeva salami,
  formaggio, ec.\ che noi chiamiamo \textit{Pizzicagnoli}. Dice che le Ninfe sono per appiccar
  lite con detto Bambi, perché esso impedirà, che elle non habbiano il Poema
  di MALMANTILE, volendolo egli per \textit{farne alle acciughe tante camiciuole}, cioè
  per involtar salumi. Ed in sustanza vuol dire, che la presente sua Opera sarà buona
  per vendere a peso per carta al pizzicagnolo; che così diciamo per esprimere 
  che un libro non habbia in se di buono altro che la carta. E qui se bene il Poeta
  dice questo per sua umiltà, e modestia (non essendo la sua Opera da vendersi a 
  peso per carta) tuttavia, non sapendo che la mia penna dovea farle meritare
  tal fine, fece buon pronostico. E non dubito, che haverà dato nel segno. Il
  Lalli nella sua Franceide \cstan[4]{21}. Si servì di questa medesima frase.
  \begin{verse}
    E le cartacce lor servono al fine
    Per avvolger l'acciughe e le Tonine.
  \end{verse}
\end{description}
\section{STANZA XXVIII.}
\begin{ottave}
\flagverse{28}Evvi anch'un libro di segreti, il quale\\
Giova a chi legge, e insegna di bei tratti\\
Ed infra l'altre a far che le cicale\\
Cantin senza che 'l corpo se le gratti,\\
E a far ch'i tordi magri con l'occhiale\\
Guardandogli divengan tanto fatti,\\
Descrive poi moltissimi rimedi\\
Per chi patisce de i calli de' piedi.
\end{ottave}

\begin{ottave}
\flagverse{29}S'io vi narrassi tutto il continente\\
Costui, diresti, ha i lucidi intervalli;\\
Pur vuo' contarven' una solamente\\
Ch'è vera, ne crediate ch'io sfarfalli,\\
Racconta d'una tal parturiente\\
Ch'una carrozza fece a sei cavalli,\\
E ch'una voglia fu, che havea havuta,\\
Ed io lo crederò senza disputa.
\end{ottave}

\begin{ottave}
\flagverse{30}Perché la donna come altera, e vana\\
Sopr'agli sfoggi ognor pensa, e vaneggia,\\
E ben ch'ell'habbia un ceffo di befana\\
Pomposa, e ricca, vuol che ognun la veggia;\\
Perciò colei hebbe la voglia strana\\
Della grandezza dell'haver la treggia;\\
Ancor che tutte, perché il cervel gira\\
Le girelle vorrian, che 'l sangue tira.
\end{ottave}

\begin{ottave}
\flagverse{31}Ma basti circa i libri quanto ho detto,\\
Perch'io che negli studi non m'imbroglio, \\
E questi mai ne altri non ho letto \\
Che forse i fatti lor saper non voglio, \\
A qualch' error non voglio star suggetto,\\
Che pur troppi n'ho fatti sopr' al foglio,\\
E poi perché son tanti, e tanti i tomi,\\
Che ne anche so dir d'un terzo i nomi.
\end{ottave}


Termina il racconto de i libri, che sono nello scaffale, e narraado un favoloso
iperbolico parto, fa una leggieri satira contro al lusso delle donne.
\begin{description}
\item[IO sfarfalli] Io aggiunga al vero: Io m'avvantaggi nel racconto. Dalla
  \textit{farfalla}, che gira e s'avvolge or qua, or la, è detto \textit{sfarfallare}.
\item[UNA voglia fu] Che cosa sia voglia in questo proposito. Vedi sopra C. 2. st. 42.
\item[ALTIERA, e vana] Altiero, si può dir sinonimo di superbo, pigliandosi
  spesso l'uno per l'altro; se bene \textit{altiero} si dice colui, che per grandezza d'animo
  non riguarda, e non applica a cose vili, anzi dimostra verso di quelle una certa
  schifezza generosa, e senza vizio, e \textit{superbo} si dice colui, che per vizio, e per
  capriccio spropositato disprezza tutti, e tutte le cose indifferentemente, e senza
  distinzione alcuna. Qui, dicendo \textit{altera} intende piena di prefunzione di se stessa,
  che è lo stesso che \textit{superbo}; e \textit{Vana} dedita alle vanità, o vanagloriosa, boriosa.
  Il Petrarca distingue queste due voci, dicendo nella Canz. 22.
  \begin{verse}
    Ch'in vista vada altiera, e disdegnosa,
    Non superba, e ritrosa.
  \end{verse}
\item[BEFANA] Significa Donna malfatta: perché \textit{befana} diciamo un fantoccio fatto
  di cenci, che si suole da alcuni mettere alle finestre il giorno dell'Epifania, il
  quale da Epifania e detto corrottamente il giorno di Befana. Vedi sotto \cstan[9]{1}.
\item[TREGGIA] Intende carrozza. Se ben \textit{treggia} è un veicolo rustico senza ruote
  per uso di portar paglia, e legne, ec.\ facendolo tirar strasciconi da i buoi.
  Servio sopra quel verso di Virg. 1. Georg. \textit{Tribulaque, traheaeque, \& iniquo pondere
    rastri} dice così. \textit{Traha genus vehiculi dictum a trahendo; nam non habet rotas},
  ed è la nostra Treggia.
\item[IL sangue tira] L'inclinazione, o genio le spinge, le forza. Intende che le
  girelle, che le donne hanno in testa, havendo simpatia con l'altre girelle, fanno
  desiderare alle donne quelle della carrozza.
\item[NON m'imbroglio negli studi] Cioè; \textit{non attendo agli studi}; \textit{non ho che fare con
  loro}; \textit{non m'intrometto di studiare}; \textit{non me ne impaccio}.
\item[PUR troppi n'ho fatti sul foglio] Per modestia intende; Pur troppi sono gli errori
  che ho fatti nel comporre la presente Storia.
\end{description}

\section{STANZA XxXU,}
\begin{ottave}
\flagverse{32}Però seguiam con Paride le Dee\\
A veder cose belle, e Stravaganti;\\
E prima troverem di gran miscee,\\
Corpi di Mummie, ed ossa di Giganti;\\
Esser in corpo a pesce due galee,\\
Impietrite con tutti i naviganti,\\
Legni, li quali esse han per tradizione\\
Che fur fatti del giuggiol di Nerone.
\end{ottave}

\begin{ottave}
\flagverse{33}Chiuse nel vaso poi vedrem le gotte\\
C'hebbe quel Vecchio Chioccia di Sileno,\\
E l'asta che fu, dicon, di Nembrotte\\
Con che volle infilzar l'Arcobaleno;\\
Benché si creda più di Don Chisciotte,\\
E veramente non può far di meno,\\
Perché in vetta nel mezzo della lama\\
V'è scritto Dulcinea, ch'era sua dama.
\end{ottave}

\begin{ottave}
\flagverse{34}Pende dal palco un secco gran Serpente,\\
Che quasi al Cocodrilo s'assomiglia,\\
E dicon che la coda solamente\\
Per la lunghezza arriva a cinque miglia;\\
Ma quel che più curioso di niente\\
È certo, è una grandissima conchiglia,\\
Ove fra minuta alga, e poca rena\\
Sta congelato un'uouo di Balena.
\end{ottave}


Chi vi dipana fa quant'
C” al fin @ ogni gomitol si

Lasciato il racconto de' libri, torna l'Autore a narrar le cose maravigliose, e
singolari, che sono in questa Galleria. E perché in tali Gallerie si proccura da chi
le fa di riporvi cose stravaganti, ed anticaglie ragguardevoli, e molte da essi se
ne fingono per accreditare il luogo, e però il nostro Poeta mette anche egli una
mano di cose iperboliche, come sono due galee impietrite in corpo a un pesce,
e favolose, come un vaso pieno di gotte, ec, Vedi Liaciano nell'Istoria vera,
ove descrive terre, ed huomini in corpo a una balena; E Esiodo, ove descrive
il vaso di Pandora, ove erano tutti i malori, e tutti i malanni.
\begin{description}
\item[MISCEE] Intendiamo bazzecole, masieriziuole, ed arnesi vecchi di poco
  prezzo, che habbiano del curioso; mescuglio di bagattelle, di curiosità varie.
\item[MUMMIE] Vedi sopra \cstan[6]{52}.
\item[GIUGGIOLO di Nerone] Habbiamo un nostro detto, che è: \textit{Neron tu sei in sul
  giuggiolo}, che serve per esprimere; \textit{la fortuna mi s'attraversa}; \textit{Il Diavol m'impedisce
  l'esecuzione del mio pensiero}, E viene non da Nerone imperadore, ma da un
  contadino chiamato Neri, il quale stava sopra un giuggiolo, osservando alcuni,
  che entravano in casa sua per rubare, e costoro accortisi d'esser veduti, per
  mostrare che gli volevano fare una burla, e non rubare: gli dissero; \textit{Ah Nerone
    tu sei in sul giuggiolo}, intendendo: Noi t'havevamo ben veduto. E del legname
  di questo giuggiolo dice, che eran fatte le due galee impietrite in corpo al pesce.
\item[VECCHIO chioccia] Vecchio malandato. D'uno, che sia alquanto infermo diciamo
  chiocciare; dalla chioccia, gallina vecchia, e spelata, che cova i pulcini,
  come il malato cova il letto; e l'Autore chiama Sileno \textit{vecchio chioccia} perché Sileno
  Pedante, ed Aio di Bacco si faceva portare sopra a un'asino, quasi che fusse
  mezzo infermo; ed i Gentili dicevano, che egli si trattava in questa forma, perché
  essendo egli il maestro di Bacco, il quale è numerato fra gli Dei poltroni, ed
  amici delle comodità, e del piacere, era giusto, che fusse un'huomo di tutti i
  suoi comodi.
  
\item[VOLLE infilzar  Arcobaleno] Volle infilzar l'Arco Celeste; che i Latini chiamavano
  Iride, e la dicevano insieme co' Greci. Ambasciatrice degli Dei. Verg. AEn. 5.
  \begin{verse}
    Irin de  Coelo misit Saturnia Iuno.
  \end{verse}
  Ed il nostro Poeta dice che Nembrotte vole \textit{injfilgar \& Arcobitteno}, perché egli
  fu quello; che pazzamente si pensò di voler guerreggiar col Cielo, ed a tal'effetto
  fabbricò la famosa Torre di Babel, cioè della confusione.

\item[DON Chisciotte] Che in nostra lingua vorrebbe dire: Don Stivale, o cosa simile.
  Fu un Cittadino della Mancia, il quale havendo letti molti di Cavalleria,
  cioè Amadis di Gaula, Palmerino d'Oliva, ec.\ s'imbriacò, ed invaghì
  del mestiero di Cavaliere Errante di tal maniera, che si messe ad immitare le azioni
  di detti Cavalieri, facendosi armare con quelle cirimonie, che eran soliti fare
  quei Cavalieri, andò anch'egli a cercare l'avventure, come graziosamente racconta
  Don Michel Cervantes nel suo Don Chisciotte, il quale fu molto bene tradotto
  in nostro volgare da Lorenzo Franciosini da Castel Fiorentino, assai benemerito
  della lingua Spagnuola; (l'aggiunta, o secondo libro del qual racconto vogliono,
  che sia stato composto da Carlo V. Imperatore) E perché i Cavalieri Erranti
  non erano stimati veri Cavalieri, se non havevano l'innamorata, però questo
  Don Chisciotte si finse ancor egli la sua, che fu \textit{Dulcinea del Toboso}; E da questa
  \textit{Dulcinea} il nostro Poeta prova scherzosamente, che questa Asta fusse più tosto di
   Don Chisciotte, perché nella lama, che era in cima alla detta asta v'era scritto
   \textit{Dulcinea}, ed intende, che questo ferro era dolce, cioè di cattiva tempera.
\item[UN gran Serpente] Questa iperbole del Serpente è posta qui ad immitazione, o
  per dir meglio, in derisione di coloro, che scrivono le Storie d'Etiopia, che
  dicono esservi tali Serpenti, che ingoiano un Cervio, o un Bue intero per volta,
  e sono di lunghezza di più di trenta piedi\footnote{Sorprende l'uso del \textit{piede} come unità di misura, posto che in Firenze la base è il \textit{braccio}. Potrebbe dire \textit{piede} in riferimento all'antica misura romana, e approssimarla a mezzo braccio, nel qual caso 30 piedi sarebbero 8.76m. Il \textit{Python sebae} ssp \textit{sebae} (Gmelin, 1788), presente in Etiopia, raggiunge tipicamente i 3.5m, raramente i 4.8m, eccezionalmente raggiunge i 6m, per cui 30 piedi sarebbero una esagerazione non così iperbolica, solo il 50\% in più della misura massima nota. Un Pitone africano può mangiare antilopi e facoceri, di nuovo meno che il Cervo o il Bue menzionati, ma di nuovo non iperbolicamente meno. }; E che M. Attilio Regulo nella prima
  guerra contro a i Cartaginesi ne uccidesse uno in Affrica presso al fiume Bagrada,
  che era lungo 120. piedi.\footnote{Questo sì che è esagerato ed assolutamente inverosimile.}

\item[MANTICE] o \textit{mantaco}. Vedi sopra \cstan[1]{55}.
\item[ARCOLAIO] Strumento fatto di canne rifesse, o stecche di legno, sopra il
  quale s'adatta la matassa per comodità di dipanarla, o incannarla come s'è detto
  sopra \cstan[5]{9}. E \textit{dipanare} è raccorre il filo, formandone una palla per comodità
  di metterlo in opera, e tal palla si dice \textit{gomitolo} dal Latino \textit{glomerare}, e
  \textit{glomus}, che il gomitolo, che a Roma ancora si dice \textit{glomero}.
\end{description}

\section{Stanza XXXVI. --- XXXXII.}
\begin{ottave}
\flagverse{36}Una Sfera bellissima si vede, \\
Ch'è sopr'a un ben tornito piedistallo, \\
Che per giustezza tutte l'altre eccede, \\
O sien fatte di legno, o di metallo:\\
Vada pure, e sotterrisi Archimede \\
Con quella sua, ch'ei fece di Cristallo, \\
Che bisogna guardarla, e starsi addietro \\
Per timor di non romper qualche vetro.
\end{ottave}

\begin{ottave}
\flagverse{37}Che questa, che con ogni diligenza \\
Di purgate vesciche fu commista,\\
Se per disgrazia, o per inavvertenza \\
Perquote, o cade, ell'è sempre la stessa; \\
E se 'l cristallo ha in se la trasparenza, \\
La vescica al Diafano s'appresta, \\
Ed è un corpo, che giammai non varia, \\
E quel si cangia ognor secondo l'aria.
\end{ottave}

\begin{ottave}
\flagverse{38}S'in Grecia fatta fu la cristallina,\\
E questa di vesciche vien da Troia,\\
Che a Fiesol fu portata a Catilina\\
La notte ch'ei fuggì verso Pistoia,\\
Ch'ei non giunse ne anc' alla mattina,\\
Che 'l poveraccio vi tirò le quoia,\\
Sicché due Capitan sue camerate\\
La presero, e la diedero alle Fate.
\end{ottave}

\begin{ottave}
\flagverse{39}Mentre s'ammira così bel lavoro,\\
E vi si fanno su cento argumenti,\\
Paride guarda, e vede una di loro\\
Cavarsi un'occhio la parrucca, e i denti,\\
E dargli a un' altra, perch'in tutto il coro\\
Delle Naiadi ch'ivi fon presenti,\\
O fuora (che pur anche son parecchi)\\
Han sol quei denti, un'occhio, e due cernecchi.
\end{ottave}

\begin{ottave}
\flagverse{40}Pero ch' elle son cieche, e vecchie tutte,\\
E loro i denti son di bocca usciti,\\
Ma non per questo ell'appariscon brutte,\\
Ch'ell'hanno volti belli, e coloriti,\\
E se mangiar non posson carne, e frutte\\
Elle s'aiutan con de' pambolliti,\\
Perché quei denti, come gli occhi, e i ricci\\
Non hanno più virtù, che son posticci.
\end{ottave}

\begin{ottave}
\flagverse{41}Gli portan per bellezza solamente\\
Una per volta, acciò che per la via,\\
S'ell'ha ir fuora a vista della gente,\\
Asconda ogni difetto, e mascalcia;\\
Ma il tenergli, la legge non consente,\\
Se non un'hora, e poi a quella via\\
A riportargli a casa vien costretta,\\
Acciocch'un'altra dopo se gli metta.
\end{ottave}

\begin{ottave}
\flagverse{42}Così per osservar le lor vicende\\
Questa ch'io dico se gli cava adesso, \\
Già ritornata dalle sue faccende, \\
Perch'il portagli più non le è permesso,\\
Ond'a quell'altra gli consegna, e rende,\\
Cedendo ogni ragione, e ogni regresso,\\
Perch'in quest'ora a ornarsi ad essa tocca\\
La fronte e il capo, e riferrar la bocca.
\end{ottave}


Descrive una Sfera fatta di vesciche di Porco, e mostra, che sia molto migliore
di quella di Cristallo, che fece Archimede Siracusano, perché e più stabile, e
più sicura. Mentre che Paride stava mirando, e discorrendo sopra il bel lavoro
della Sfera di vesciche, una delle Ninfe si cavo la Parrucca, un'Occhio, e i denti
e dette il tutto a un'altra, perché così e l'ordine fra loro. Qui pare, che alluda
alle Lamie, Donne, o Larve per dir meglio, che con carezze allettatrici erano
stimate da' superstiziosi Gentili mangiarsi i bambini; le quali fra tutte tre havevano
un'occhio solo, e quello usavano a viceada hor questa hor quella, secondo che
descrive Angelo Poliziano lib. 3. tit, Lamia, che dice:
\begin{adjustwidth}{8pt}{}
  Lamiae habent oculos
  exemptiles, hoc est quos sibi eximunt, detrahuntque cum libuit, rursumque
  cum libuit resumunt, atque affigunt; aliae vero etiam dentibus utuntur atque
  exemptilibus, quos nocte non aliter reponunt, quam togam, sicut uxorculae comam
  suam illam dependulam, \& cincinnos, \&c. Sed lamia haec quoties domo
  egreditur oculos suos sibi affigit, vagatur per fora per plateas, \&c, domum vero
  cum revenit, in ipso statim limine demit illos sibi oculos, abijcitque in
  loculos; ita semper domi caeca, foris oculata.
\end{adjustwidth}
\begin{description}
\item[PIEDISTALLO] È quella pietra, che è sotto al dado, sopra il quale posa la
  colonna: e qui è preso per tutta la base, che regge questa sua Sfera, come è preso
  comunemente. 
\item[VADA, e sotterrisi Archimede] È oscurata la gloria d' Archimede; Quand'uno
  fa un'operazione meglio d'un'altro diciamo al superato; \textit{Tu ti puoi ire a riporre,
  o a sotterrare}. Intendendo; Tu hai perduto tutto il credito, o la stima,
  che è quella senza la quale uno è tra gli huomini come morto; sì che vuol dire,
  che non si dee più far tanta stima della Sfera d'Archimede fatta di cristallo, perché
  questa fatta di vesciche l'ha superata.
\item[DA Troia] Non dalla Città di Troia, come pare, che voglia dire, ma dalla
  Troia femmina del porco, delle cui vesciche era formata questa sfera.
\item[VI tirò le quoia] Vi morì. Vedi sopra C. 4. st. 20. Qui tocca la comune opinione,
  che Catilina famoso capo di congiura descritto da Salustio morisse a Pistoia.
\item[VI fanno cento argumenti] Cioè discorrono assai sopra questa Sfera. 
\item[PARRUCCA] Voce straniera fatta nostrale, e vuol dire Zazzera, o chioma
  finta, che diciamo: \textit{Zazzera posticcia} dal Francese. \textit{Perrouque}, chioma. Potrebbe
  forse dirsi in Latino \textit{capillamentum}.
\item[CERNECCHI] Capelli pendenti alla testa; qui intende quella parrucca, o capelli posticci;
  se ben \textit{cernecchi} si dicono quei soli capelli, che pendono dalle tempie
  agli orecchi con altro nome dette \textit{fiaccagote}, che i Latini, secondo il Poliziano
  nel luogo sopra citato dicevano \textit{cincinnos}. E noi diciamo \textit{cincinni} quei ciondoli
  di pelo, che sogliono haver i capretti, ed i Becchi sotto la gola, i quali hanno
  qualche similitudine con questi capelli, che noi chiamiamo cernecchi.
\item[PAN bollito] Pappa fatta di pane, bollito in acqua.
\item[MASCALCIA] Magagna; Difetto; mancamento. E' lo stesso, che guidalesco,
  ma questo si dice solo nelle bestie, e mascalcia, che farebbe veramente solo
  delle bestie, l'usiamo anche per gli huomini, e talvolta per i materiali. Vi è
  un'antico libro Toscano intitolato Libro \textit{di Mascalcia}, che è dell'arte del manescalco,
  \textit{de re veterinaria},
\item[DA quella via] o \textit{a quella via}. Subito. Senza metter tempo in mezzo. Latino
  \textit{extemplo}, e \textit{vestigio}. Se bene si potrebbe intendere ancora per In quella maniera;
  in in quella guisa, come è inteso sopra C. 7. st.84.
\item[CEDE ogni regresso] Cede ogni azione; ogni autorità. Vedi sopra C. 7. st. 104.
\item[RIFERRAR la bocca] Intende rimettere i denti. Bocca sferrata si dice a uno,
  che habbia meno i denti dinanzi dal ferrare le bestie, e rimeteer loro i chiodi a'
  piedi, quando si sono sferrate.
\end{description}

\section{Stanza XXXXIII --- XXXXV.}
\begin{ottave}
\flagverse{43}Piena di cibi intanto una credenza \\
Vien pari pari aperta spalancata, \\
E fatta da vicin la riverenza \\
Parole pronunziò di questa data: \\
Cavalier, se tu vuoi far penitenza, \\
E in parte a noi piacere, e cosa grata \\
Ho munizion da caricar la canna, \\
E poi da bere un vin ch'è una manna. 
\end{ottave}

\begin{ottave}
\flagverse{44}Credilo a me ch'egli è del glorioso\\
Però qua dentro, via, distendi il braccio,\\
Che troverai del buono, e del gustoso,\\
Se tu volessi ben del Castagnaccio.\\
Paride fece un po del vergognoso,\\
Ma nel veder le bombole nel ghiaccio,\\
Mandò presto da banda la vergogna,\\
E fece come i Ciechi da Bologna.
\end{ottave}

\begin{ottave}
\flagverse{45}Levatoli poi via la calamita \\
Di quel buon vino, e massime del bianco \\
Gli fataron le Dee tutta la vita \\
Dalla basetta in fuor del lato manco, \\
Sicché in quanto ad haver taglio o ferita\\
In altra parte era sicuro, e franco,\\
Poi dangli un brando con la sua cintura,\\
E del trattarlo  l'intavolatura.
\end{ottave}


Mentre stavano guardando le suddette galanterie, comparve una credenza aperta
piena di roba da mangiare, e da bere, ed inuiid Paride a soddisfarsi; egli dopo
haver fatto alquanto lo Riiakian'> mangid, e bevve; Terminato il mangiare se
46  'Ninfe lo fatarono, rendendogli impenetrabile tutta la persona, eccetto che la basetta
mancina + Qui il Poeta immuta l'Autore, che favoleggia Orlando impenetrabile
in tutta la persona, eccetto che nelle piante de' piedi.

\begin{description}
\item[CREDENZA] Così chiamiamo un' armadio, entro al quale si ripongono, ¢
conservano gli arnesi, ed avanzi della mensa; il quale armario si dice ancora,
4 tredenziera  perché quei bicchieri vaii, e baciji d' argento, ec.\ che si mettono alle
tavole de' Grandi per servizio; 0. per apparato della
diti tutti insieme, si dicono credenza, e i si rij
vriano riporre in detto armadio, che però lo china
tino eroacns re

\item[SPALANCATA] Affatto aperta. Vedi sopra'C. gf. 38, Pal
cato diciamoa la chindenda\, o riparo fatto con è pali, a un >
vuol dir Senza palanca, e per conleguenza totalmente aperto, o
tegno, o inipedimento. ei è

\item[PAROLE di questa data] Parole simili a queste, o di
ta, la quale si attende moltissimo nel gioco delle carte, per esser
minchiate; Onde si dice: Ha fatta una buona, o una cattiva data,

\item[SE tu vuoi far penitenza] Se tw vuoi mangiare. Termine usato per:
invitar' uno a desinare, o cenar connoi, guafi ditiamo:venite a digi;
ché la nofira mensa e povera, e (eatlardi cibi. Sidicc.amcoralfar carisa 5,00

s'è visto sopra C. 5. st. 68, otis verb 9!
\item[HO munirione per caricar la canna] Ho roba da mangiare\, eda
care la canna della gola., e non quella dell'archibuso <2 5, Gate

\item[VN vin, ch' una manna] Vino isquisitissimo, che tale si legge fusse!
che mando Dio nel deferto al Popolo cletta., Vedi ferro Cip.st58..Ma
stranicra, ma fatta nofirale, che significa una brina:condenfata: tenera'y'
detta cos} dall' Ebraico, Azanbm; ioe Quid of bee: come: si dice nell?
16. Poiché maravigliati gli Ebrei di questo nuovo, e faporolo cibo 5
uno all' altro; che è ciò, che no? mangiamo ? Dalquesta dolcezza viene '
nostro detto. I Latini dicevano in questo proposito dowis Velar..
\item[EGLI¢ det glorioso] I Battilani chiamano vino glorioso il vino pga 4
roso, e buonissimo, e dicono groliefe in vece di gloriose; cioè valor lo
vaalle stelle. In certe Prose Toscane antiche, delle quali alcune si ritrovanom
nuicritte nella Libreria di S, Lorenzo date fuora dal Doni, vi e una leteera amt:
rosa, nella quale e accennato Amore con dire: Quel gloriose; titolo dato
da' nostri Battilani al vino; e veramente Amore non imbriaca meno di
si faccia il vino il pith glorioso, a
VIA. Questo termine serve per follecitare, o incitare uno. Latino Eia ae,
\item[CASTAGNIACCIO] Pane fatto di farina di Castagne: qui vuol
per opera d' incanti quella credenza dava tutto quello, che uno fapeva
itis g
tern gab
rare.: a
\item[FECE il vergonoso] Finse di non si ardire a mangiare. Mostrava vergognitt
d'accettar l'invito, che gli faceva quella credenza.
\item[BOMBOLE] Vasi di vetro, i quali servono per mettere il vino in
ghiaccio, o neve, detti così (secondo alcuni) dai suono, che fanno nel |
fuori il vino, che par che suoni bombof. Al Rotenano  vuole, che i Latini
da tal suono le dicessero amphore bilbina; ma può anch' essere, che agi ie
così da bombo voce.puerile, che vuol dir bevanda, decta così dal fueno, —
\item[COME i Ciechi da Bologna] Si dà loro un soldo, perché comincino a cantare,
e bisogna poi dargliene due, perché si chetino. Ci iciue per esprimere uno, che
si faccia molto pregare a far'una tal cosa mostrando non voler farla, e bisogna
poi pregarlo, che resti di farla, Orazio.
 \begin{verse}
   \backspace Omnibus hoc vitium est cantoribus, inter amicos
   Ut nunquam inducant animum cantare rogati,
   Iniussi numquam desistant.
 \end{verse}
 Si dice \textit{Ciechi da Bologna}, \textit{da Ferrara}, o \textit{da Milano}. I Latini in questo proposito
 dissero \textit{Arabicus Tibicen}. Qui intende, che Paride si fece pregare a mangiare, e
 bere,  e poi non si trovava il modo, che egli restasse.

\item[CALAMITA] È la pietra \textit{Magnes}, la quale ha proprieta d'attrarre il ferro,
  come appunto ha il vino di tirare a se Paride, ed è fra esso, ed il vino la stessa
  simpatia, che è fra la calamita, e il ferro. Vedi sopra \cst[5]{59}. E sotto in
  \cst{66}.

\item[DI trattario l'intavolatura] L'instruzione di come si debba adoprar quella spada.
  Intavolatura e scrittura, che per via di note, e di numeri regola la mano del sonatore.
\end{description}
\section{Stanza XXXXVI. --- LI.}
\begin{ottave}
\flagverse{46}E perché il tempo ormai era trascorso,\\
Che inviarlo dovean di quivi altrove\\
Prima in sua lode fatto un bel discorso,\\
Che l'agguagliava a Marte, al Sole, e a Giove,\\
Figliuol  (dissero)  quanto t'è occorso\\
In qui stanotte, e il come, e il quando, e il dove,\\
A noi palese è tutto per appunto,\\
Anzi sei qui per opra nostra giunto.
\end{ottave}

\begin{ottave}
\flagverse{47}Acciò tu vada incontro a un'avventura\\
A pro d' un pover huomo questa notte;\\
Questo è un tal cognominato il Tura,\\
Ch' in Parion gonfiava le pillotte.\\
Era in bellezze un mostro di natura,\\
Sicché tutte le donne n'eran cotte,\\
E lasciando i rocchetti, ed i cannelli\\
Per lui ch'è ch'è facevano a capelli.
\end{ottave}

\begin{ottave}
\flagverse{48}Non ch'ei ne desse loro accasione,\\
Come qualche narciso inzibettato,\\
C'una cuffia, che e' vegga a un verone\\
Di posta corre a far lo spasimato;\\
Anzi, è un di quei, c'al Mondo sta a pigione,\\
A bioscio nel vestire, e sciammannato,\\
C' addosso i panni ognor tutti minestra\\
Tirati gli parean dalla finestra.
\end{ottave}

\begin{ottave}
\flagverse{49}Ed esse eran capone; ma chiarite;\\
Al fin lasciando quel suo cuor di smalto,\\
Fecer come la Volpe a quella vite\\
C'havea sì bell'uva, e tanto ad alto,\\
Che dopo mille prone, anzi infinite\\
Arrivar non potendovi col salto,\\
Gli è mé, disse, ch'io cerchi altra pastura,\\
Che questa da ogni mo non è matura.
\end{ottave}

\begin{ottave}
\flagverse{50}Così non la saldò già Martinazza,\\
La qual non vi trovando anch'ella attacco,\\
Poiché gran tempo andata ne fu pazza\\
Havendo il terzo, e il quarto, e ognuno stracco\\
Condurre un giorno fecelo alla mazza,\\
E per via d' un che le teneua il sacco\\
Avvezzo a tosar pecore, ed agnelli,\\
Mentr'ei dormiva gli tagliò i capelli.
\end{ottave}

\begin{ottave}
\flagverse{51}Quei capelli c'un tempo havea chiamati\\
Del suo fascio mortal funi, e risorte,\\
Le bionde chiome, oh Dio,quei crini aurati\\
Che ricoprivan tante piazze morte,\\
Onde scoperti furo i trincerati\\
Ove il nimico si facea sì forte;\\
Perché (per quanto un'Autore accenna)\\
Lo rimondavon fino alla cotenna.
\end{ottave}

Le fate dopo haver lodato Paride per bravo, per bello, e per valoroso gli dissero,
che  l'havevan fatto capitar quivi, perché egli andasse a liberar il Tura,
quale loda ironicamente, e dice, che tutte le donne erano innamorate di iui; ma
accortesi, che non corrispondeva a nissuna, lo lasciarono, e Martinazza, perché
egli non volle mai corrisponderle, haveva fattagli la malia, che sentiremo nelle
ottave seguenti.

\begin{description}
\item[AVVENTURA] I Romanzatori Spagnuoli in quei loro Amadis di Gaula, e
  Palmerini d'Oliva chiamavano avventure (\textit{aventuras}) quegli incantesimi
  ne i quali s'imbattevano i Cavalieri Erranti, e però il nostro Poeta, haevndo
  creato il Cavalier di Quoio, vuol, che ancor'egli sia stimato Cavaliere Errante, e
  che vada a provare l'avventura di liberare il Tura dall'incantesimo. I Francesi
  similmente dissero \textit{adventures}. E i nostri Toscani ancora, sentendosi in questo del
  termine cavalleresco, chiamarono gli accidenti, che accadevano a' Cavalieri, e
  davan loro materia di fare prodezze \textit{Avventure}. L'Alamanni\footnote{Luigi Alamanni (Firenze, 6 marzo 1495 --- Amboise, 18 aprile 1556) poeta. Autore fra l'altro di \textit{Girone il Cortese}, Venezia, per Comin da Trino di Monferrato, 1549.}  nel Girone in
  principio.
  \begin{verse}
    \backspace Narrerò di Girone l'alte avventure.
  \end{verse}
  E da ciò il Bocc, Tess. lib. 5. disse:
  \begin{verse}
    \backspace Mettersi in avventura:
    Ma non li parve via ben ben sicura.
    Però non se ne mise in avventura.
  \end{verse}
\item[IL Tura] Costui era un pover'huomo, che gonfiava le pillotte in Parione
  che in Firenze è la strada, dove si giuoca alla pillotta detta così da marmo Pario,
  perché in essa anticamente haveano le botteghe coloro, che lavoravano di marmi,
  o pure (il che forse è più verisimile) quasi \textit{Ripat Regio} Ripe Roine; poiché
  tale strada sbocca sul Passeggio di Lung'Arno; In Roma ancora vi è la contrada di
  Parione detta similmente così detta quasi \textit{Rione a Ripa}. Regio Ripensis. O
  pure è così chiamata, quasi Parte di Rione; \textit{Pars regionis}, come mi vien riferito
  leggersi in alcune Carte, o Contratti. E perché veramente costui era bruttissimo
  di faccia, ed haveva la zazzera avviluppata, e lorda, lo chiama \textit{mostro di natura
  in bellezza}, ed intende Deforme, se ben par, che voglia dire, di bellezze
  sopranaturali.
\item[PILLOTTA] Specie di palla da giuocare, Vedi sopra \cst[6]{34}.
\item[N' ERAN cotte] Erano abbruciate dal fuoco d'amore per lui Virg. \textit{Uritur
  infelix Dido}: dice briache del suo amore, c s'intende innamoratissime di lui. Lat.
  \textit{ebriae amore}. Plauto nel \textit{Milit gloriso}, o \textit{Soldato}, al quale da nome di \textit{Pyrgopolinices},
  cioè di \textit{Abbattitore di Torri, e di Città}; o, came noi diremmo \textit{Tagliacantoni}, e
  \textit{Spacca Montagne}; fa dirgli da \textit{Artotrogo}, cioè in nostra lingue \textit{Sparapane} Parassito,
  suo adulatore; che tutte le donne sono di lui fieramente innamorate. \textit{Quid
  tibi ego dicam; quod omnes mortales sciunt; Pyrgopolinicem te unum in terra vivere
  Virtute, \& forma, \& factis invictissimus? Amant te omnes mulieres, neque hercle
  iniuria, Qui sis tam pulcher}. Ed egli sprezzatore altero di tali amori compiange
  solamente la sua disgrazia, beccandosi su queste lodi; dell'esser troppo bell'huomo,
  da fare innamorare di lui tutto il Mondo, \textit{Nimia est miseria pulchrum esse hominem
    nimis}. 
\item[LASCIANDO i rocchetti, ed i cannelli] Lasciando star di lavorare. Le aveva
  prese tanto forte l'amore; e tanto le teneva fisse nell'amoroso pensamento, che
  non potevano più attendere a' loro usati lavori. Quando Didone si fu innamorata
  d'Enea, non tirava innanzi gli edifizj, e le fabbriche della sua Città (onde 
  Virgilio ebbe a dire: \textit{pendent opera interrupta, minaeque Murorum ingentes}) come
  quella, che era occupata da più possente pensiero. Col presente detto di lasciare
  \textit{i rocchetti, e i cannelli}, s'intende questo, perché le donne d'infima plebe (che tali
  vuol, che s'intenda, che erano l'innamorate di costui) per lo più non hanno
  altro lavoro che l'incannare, e tessere, a' quali lavori s'adoprano i \textit{Rocchetti} (che
  son legnetti tondi forati per lungo, e servono per ragunarvi sopra la seta, ed
  ogni altro filo: ed i \textit{Cannelli}, che sono pezzuoli di canna tagliata fra un nodo, e
  l'altro, dai Latini però detti \textit{internodia}, e servono per lo medesimo effetto d'adunarvi
  sopra la seta, ec.\ per adattarla a tessere; I; che si dice \textit{incannare}.
\item[CH'è ch'è] Ad ora ad ora; Di momento in momento. Vedi sopra Can. 3. st.~68.
\item[FACEVANO a i capelli] Si perquotevano. S'azzuffavano. Quando due donne
  combattono tra di loro diciamo \textit{fare a i capelli}; perché il lor perquotersi, è per lo
  più il pigliarsi l'una l'altra per i capelli.

\item[CUFFIA] Berretta a foggia di sacchetto, entro alla quale le donne si serrano
  i capelli in testa, e quando noi diciamo nel modo, che è detto ne! presente luogo
  una \textit{cuffia}, un ciapperone, e simili arnesi usati dalle donne, intendiamo una Donna.
  Così dal portare lancia, o barbuta; i soldati medesimi si chiamavano \textit{Lance},
  e \textit{Barbute}, come si cava da Matteo Villani, 11. 81. e Erodoto volendo dire, che
  que' di Nasso si ritrovavano avere in piedi ottomila soldati, che portavano rotella,
  o brocchiere; disse \textit{octacischilian aspida}, cioè scudi militari, o rotelle ottomila.
\item[VERONE] Latino \textit{moenianum}, \textit{podium}, \textit{pergula}, e in Greco secondo alcuni
  \textit{Peribolos} da \textit{periballein} abbracciare, circondare, che i Francesi dicono \textit{environner}.
  Propriamente vuol dire \textit{andito}, o \textit{terrazzo scoperto}: Qui credo, che habbia a dir
  \textit{Balcone}, e non \textit{Verone}. \textit{Verone} è detto quasi \textit{girone}, cioè \textit{giro}, dall'andarvi sopra
  e rigirare, \textit{Andito}, che è lo stesso par fatto da \textit{Andare}. Latino \textit{ambulatio}.
\item[EGLI è a pigione al mondo] Così diciamo d'un'huomo spensierato, sciatto, senza
  considerazione, e che vive a caso, che si dice anche \textit{Huomo a bioscio}, \textit{sciamannato}
  (cioè male ammannato, male all'ordine) e che i \textit{panni gli paiono tirati addosso
    dalla finestra}. E con questi quattro modi di dire l'Autore descrive l'attilatezza del Tura;
  del resto, parlando secondo moralità, ognuno dovrebbe stare in questo mondo,
  come a pigione; perché la nostra propria casa è nel Cielo. E nel Salmo 118.
  \textit{Incola ego sum in terra}, il Greco dice \textit{Parcecos}, e alcuni Salteri dicevano, come riferisce
  S. Agostino sopra i Salmi, \textit{inquilinus}, cioè \textit{pigionale}. 
\item[CAPONE] Ostinato Latino. \textit{Pertinax}. \textit{Perticax}.
\item[FAR come la volpe alla Vite] La Volpe dopo haver molto saltato, e dopo essersi
  molto affaticata per arrivare un grappolo d'uva, e non l'havendo potuto arrivare
  disse: La voglio lasciare stare, perché ad ogni modo ella non è matura.
  Può aver data occasione a questa novelletta quella d'Esopo, della Volpe, e del
  Pruno; in cui la Volpe, che voleva salire una siepe, mi suppongo, per mangiar
  l'uva, della quale è ghiottissima, pensando di trovare il Pruno buon'amico, resto
  ingannata del suo pensiero; poiché attaccandovisi restò intaccata, e l'appoggio
  le fu ferita, e volendola poi disputar con lui, ebbe il torto: E questo detto
  ci serve per esprimere uno, che habbia usata ogni possibil diligenza per conseguire
  una tal cosa, e non l'havendo potuta ottenere, o habbia abbandonata l'impresa
  come impossibile, o sia quella tal cosa stata data a un'altro, ed egli poi si
  vanti di non l'haver voluta, perché non era buona, o non era il caso suo, che
  diciamo: \textit{farsi honore d'una cosa}.
\item[COSÌ non la saldò Martinazza] Cosè non finì, o terminò l'amore di Martinazza
  la quale \textit{non trovando attacco}, cioè non trovando luogo di sperare in questo
  suo amore verso il Tura, del quale \textit{andò pazza}, cioè stette innamoratissima di lui.
\item[CONDURRE uno alla mazza] Tradir'uno: Condurre uno con inganni, e lusinghe
  in mano de' suoi nimici, o della giustizia, o in qualche altro pericolo, o
  come si suol dire; \textit{al macello}. Latino \textit{In insidias ducere}.
\item[TENER il sacco] Tener di mano. Aiutare a cometter un delitto. Habbiamo
  un proverbio sentenzioso, che dice: \textit{Tanto ne va a chi ruba, quanto a chi tiene il
    sacco}, che esprime \textit{Agentes, \& consentientes pari poena puniuntur}. E diciamo anche
  \textit{Tenersi il sacco l'un l'altro}; che esprime il detto di Teren. \textit{Tradere operas mutuas}.
\item[FUNI, e ritorte del suo fascio mortale] Metafora amorosa: Sì come le funi, e
  ritorte tengono unite più legne in un fascio, o fastello, così i capelli del Tura,
  quasi funi, e ritorte tengono unita col corpo l'anima, cioè tengono in vita le
  Amanti del medesimo Tura. E \textit{ritorte} dicemmo, che cosa sieno sopra \cst[6]{94}.
\item[PIAZZE morte] Si dicono i luoghi vacanti de i soldati; per esempio un Capitano
  è pagato per cento soldati, e non ne ha se non novanta; quei dieci infino a 
  cento, che mancano si dicono piazze morte. Ma qui intende quelle \textit{piazze}, che
  lasciano le margini, o cicatrici de i mali, che vengono nel capo; sopr'alle quali
  non nascono capelli.
\item[I TRINCIERATI] I luoghi, dove erano le trinciere. Intende, che col tagliargli
  i capelli si sono scoperti quei luoghi, i quali con quelle margini parevano
  una campagna piena di trinciere. \textit{dove il nimico si faceva forte} cioè dove
  s'ascondevano i pidocchi. 
\item[TRINCIERA] o \textit{Trincea}, È un'alzamento di terreno: condotto a foggia di
  bastione, nel ricinto del quale dimorano i soldati per difenderti dall'artiglierie, ec.\
  de i nimici. Franzese \textit{trenchée}, cioè \textit{tagliata}.
\item[LO rimondaron fino alla cotenna] Gili tagliarono i capelli fino rasente la pelle.
  \textit{Rimondare} vuol dir Tagliare a un'albero i rami: E \textit{cotenna} s'intende solo la pelle
  del porco, ma quando si tratta del capo s'intende anche quella dell huomo.
  Vedi sopra \cst[5]{52}.
\end{description}
\section{Stanza LII. ---  LVI.}
\begin{ottave}
\flagverse{52}E così Martinazza hebbe il suo fine \\
Volendo vendicarsi per tal via, \\
Però, che buona parte di quel crine,\\
Ch'alcun non sen' avvedde, leppò via,\\
E fabbriconne al Tura le rovine\\
Con una potentissima malia,\\
Che registrata in Dite al protocollo\\
In un Lupo rapace trasformollo.
\end{ottave}

\begin{ottave}
\flagverse{53}E questo Lupo raggirar si vede \\
Intorno a un montuoso casamento \\
D'una Gente, che,mentre muove il piede\\
Sopra alla terra, v'è rinvolta drento.\\
Di questa cosa il tempo non richiede\\
Così per hora fartene un comento,\\
Perch'egli è tardi, e pria che tu l'intenda\\
Spedir devi lassù questa faccenda.
\end{ottave}

\begin{ottave}
\flagverse{54}Hor dunque vanne, e perché tu non faccia \\
Qualche marron, ma venga a arar dritto, \\
Acciò tal magistero si disfaccia,\\
Perché scattando un pel tu havresti fritto,\\
In questo libro qui faccia per faccia\\
L'ordine, e il modo si ritrova scritto,\\
Portalo teco, e acciò che tu discerna, \\
Perch' egli è buio co questa laterna.
\end{ottave}

\begin{ottave}
\flagverse{55}Egli la prende con il libro insieme,\\
Dicendo, che varrassi dell'avviso,\\
E che d'incanto, e diavoli non teme,\\
Perché egli è huom, che fa mostrar il viso.\\
Si parte, e per c'al campo andar gli preme,\\
In due parti vorrebbe esser diviso;\\
Pur vuol servirle, perché si figura,\\
Che non ci vada gran manifattura.
\end{ottave}

\begin{ottave}
\flagverse{56}Considerando poi nel suo cervello,\\
Che s'a quel luogo a bambera s'invia\\
Potrebbe andar a Roma per Mugello,\\
Perch'ei non si rinvien dov' ei si sia, \\
Ricerca nel suo mastro scartabello\\
Di quei paesi la Geografia,\\
Ma quel (per quanto noi potrem comprendere)\\
Non si vorria da lui lasciar' intendere.
\end{ottave}

Martinazza hebbe il suo intento, perché presa buona parte de i capelli del Tura
con essi gli fece una malia, che lo trasformò in lupo, e lo confinò in un monte
vicino a Malmantile. Finito questo racconto le Fate licenziaron Paride, e gli
diedero un libro, dove era scritto il modo da tenersi per disfar quell'incanto, ed
una lanterna per farsi lume; e Paride si partì con risoluzione di sbrigar questa
faccenda prima d'andare al Campo.

\begin{description}
\item[LEPPO' via] Portò via di nascosto. Il verbo \textit{leppare} ci serve per esprimere
  velocità nell'andar via o nel levar via qualcosa.

\item[MALIA] Incantesimo, fattucchieria, stregoneria. 

\item[PROTOCOLLO] Libro pubblico tenuto da i Notai per scrivervi sopra i contratti,
  e testamenti, e così è inteso da noi; se ben \textit{protocollo} vuol dire libro da registrarvi
  sopra, che che sia. Il Berni sonetto in biasimo d'una mula dice:
  \begin{verse}
    \makebox[7em]{\dotfill} E troppo sta digiuna
    Ch' il protocollo memoria non fanne.
  \end{verse}
  Perché veramente \textit{Protocollo} è un libretto, sopra il quale si segnano, e registrano
  brevemente le cose, per distenderne poi scrittura più largamente, ed autenticamente
  detto così quasi, \textit{primo libro incollato, e legato}. \textit{Liber ex glutine compactus,
    in quem acta referuntur}. Ma il nostro Poeta lo piglia nel senso, che oggi usiamo
  di libro da Notai, e intende che Martinazza haveva fatto contratto col Diavolo
  di questa malia; il qual contratto era già messo al libro del Notaio del Diavolo
  e per questo detta malia era autenticata, e non si poteva alterare, perché
  era passata per mano di Notaio, e registrata al suo protocollo.
\item[CASAMENTO montuoso] Intende il Castello di Montelupo, che oggi è quasi
  distrutto, però più tosto \textit{Casolare}, che \textit{Castello}, e lo dice montuoso, perché è
  sopra un monte come lo mostra il nome medesimo. E nota, che ancor qui il nostro
  Poeta va imitando i Romazatori Spagnuoli, che fanno parlare oscuramente,
  e come gli Oracoli quei loro Alchisi, Zirfee, Urgande, ec, incantatori.
\item[MENTRE move il pié sopra alla terra v'è rinvolta drento] Le reliquie di questo
  Castelio sono abitate da persone, che fabbricano vasellami di terra, come pentole,
  boccali, ec, quali si fabbricano per via d'una ruota, la quale va mossa co'
  piedi, e fa l'effetto del tornio, e perché in muover detta ruota, e fabbricare il
  vaso, la terra schizza addosso a chi lavora però dice \textit{Mentre move il pié sopra alla
    terra v'è rinvolta drento}.
\item[FAR' un marrone] Far' un error grandissimo; \textit{un'errorone}.
\item[ARAR dritto] Operar giustamente. Non fare errori. Tolto dal Bifolco. Diciamo
  ancora, rigar diritto.
\item[SCATTANDO un pelo] Se tu uscissi punto dell'instruzione, che tu hai. \textit{Scattare},
  o \textit{scoccare}, si dice della freccia quando scappa dalla cocca\footnote{Vedi sopra \cst[4]{47}.}, e dall'arco, e
  di qui è tolta la metafora, o forse dall'orivolo a ruote.
\item[TU haveresti fritto] Il Proverbio dice: \textit{Come disse la Tinca ai Tincolini}, senz'altra
  aggiunta s'intende: \textit{noi habbiam fritto}. Qui intende tu havresti finito, cioè
  tu havresti rovinato questo negozio. È lo stesso che: Noi habbiam fatto il pane
  detto sopra \cst[7]{60}.
\item[HUOM che fa mostrar il viso] Huomo ardito, e che non fugge i cimenti.
\item[A BAMBERA] A caso. Latino \textit{Iureconsulto}. Vien forse da Bamberottolo, che
  vuol dir ragazzuolo, senza giudizio. È il ragazzo in alcuni luoghi chiamato \textit{Bamberottolo}.
  Dicesi anche \textit{A fanfera}.
\item[ANDAR a Roma per Mugello] Far' una strada al tutto contraria, come sarebbe
  andar da Firenze a Roma, e pigliar la strada per il Mugello, che è direttamente
  contraria. 
\item[NON si rinviene] Cioè non riconosce in che parte ei si sia, e non fa quel ch'ei
  si debba fare. 
\item[MASTRO scartabello] Intende quel libro, che gli haveano dato le Fate, che
  è il suo maestro, e direttore; Questa voce \textit{scartabello}, e corrotta da \textit{Cartabello}, che
  anticamente era intesa per un libro di stima; come mostra il Dottissimo, ed
  Eruditissimo sig.\ Francesco Redi\footnote{Francesco Redi (Arezzo, 18 febbraio 1626 --- Pisa, 1º marzo 1697) medico, naturalista, letterato. Redi è considerato uno dei più grandi biologi di tutti i tempi. Il suo ``Ditirambo'': \textit{Bacco in Toscana}, Firenze, per Piero Matini, 1685} nelle annotazioni al suo bellissimo Ditirambo a
  c.~18. Gli Spagnuoli chiamano \textit{Cartapel} una scrittura continuata nel foglio senza
  voltarlo, come s'usa negli editti; dall'essere cred'io, non ripiegata, come i fogli,
  ma stesa, come una pelle; o perché si distendessero tali sorte di scritture non
  in carte ordinarie, ma in pelli, ovvero in cartapecore.
\end{description}
\section{Stanza LVII. ---  LX.}
\begin{ottave}
\flagverse{57}Fu Paride persona letterata, \\
Che già studiato havea più d'un saltero,\\
Ma poi, non ne volendo più sonata, \\
Alla squola studiò di Prete Pero, \\
Però s'ei non ne intende boccicata \\
È da scusarlo; e poi, per dire il vero, \\
Lettere, ed armi van di rado unite \\
Per c' han di precedenza eterna lite. 
\end{ottave}

\begin{ottave}
\flagverse{58}Ma benche la lettura sia fantastica\\
A un che, si può dir non sa niente,\\
E c'altro di virtù non ha scolastica,\\
Che pelle pelle l'Alfabeto a mente,\\
Tanto la biascia, strologa, e rimastica\\
C'a compito leggendo finalmente \\
Il sunto apprende, e fra l'altre sue ciarpe\\
Ripone il libro, e sprona poi le scarpe.
\end{ottave}

\begin{ottave}
\flagverse{59}Così commina, e a quel Castello arriva, \\
Passa dentro, lo gira, e si stupisce, \\
Che quivi non si vede anima viva \\
Per c'a quell'ora in casa ognun poltrisce. \\
Ma perché non è tempo ch'io descriva\\
Quanto col Tura a Paride sortisce,\\
Con buona gratia vostra farem pausa,\\
Per diffinir di Piaccanteo la causa.
\end{ottave}

\begin{ottave}
\flagverse{60}Che da quei tristi, com' io dissi avanti \\
(Fatto mentre pappava assegnamento \\
D'insaccarsi per lor quei boccon santi) \\
Tocco de i pié nell' Arsenal del vento; \\
Di poi gli stessi sel cacciaro innanzi\\
Giusto come un Villano in su il giumento,\\
Pungolandolo, come un'animale\\
Fin, che lo spinser dove è il Generale.
\end{ottave}


Descrive le qualità di Paride, e dice, che egli era letterato, perché havea letto
più un saltero, che è quel libricciuolo, contenente alcuni Salmi; che si dà a
leggere a' ragazzi quand' hanno imparato a conoscer le lettere dell'Abbicci; E
con questo dire, intende che egli non sapeva troppo leggere; e dice, che non è
da far meraviglia di questo, perché l'armi, e le lettere mai furon d'accordo, e
però egli, che era armigero, era scusabile, se non era letterato; con tutto ciò
compitando lesse in quel libro, ed intese quel ch'ei doveva fare; ed arrivato al
Casamento montuoso trovò che ognuno dormiva. E qui l'Autore lascia il parlar
di lui, e torna a parlar di Piaccianteo, che lasciò sopra nel fine del Canto 5. e
dice, che a furia di calci e pungolate fu da coloro condotto dov'era il Generale.
\begin{description}
\item[NON ne volendo sapere più suonata] Non volendo più sentirne discorrere di fare
  una tal cosa, e qui intende non volendo più studiare.
\item[LA squola di Prete pero] Insegnava dimenticare.
\item[NON intende boccicata] Non ne intende punto. Non conosce a pena le lettere,
  perché \textit{boccicata} stimo che venga da \textit{abbiccì}, quasi dica non sa l'Abbiccì, che
  è quello, che con i Greci ancor noi diciamo \textit{Alphabeto}, e l'usa il nostro Poeta
  nella presente Ottava 58. Procopio nella Storia segreta narrando l'ignoranza di
  Giustino Imperadore, che poi si adottò Giustiniano; dice che egli era \textit{Analfabeto},
  cioè, che non sapeva l'abbiccì; ne scrivere il suo nome.
\item[PELLE pelle] Superficialmente. È lo stesso che \textit{buccia buccia} detto sopra \cstan[3]{27}.
\item[BIASCIARE] Masticare senza denti; cioè con la lingua, e col palato. Qui
  intende quello studiare, che fanno i fanciulli, quando imparano a leggere, che
  prima di rilevare, o profferir la parola, che leggono, la compitano sotto voce
  con la bocca il medesimo gesto, che fa uno, che biascia; e lo stesso vuol
  dire quel \textit{rimastica, ec. e strologa} intendi: cerca d'indovinare quel che dica quella
  scrittura.
\item[A compito] Leggere a compito, è quello accoppiar le lettere, e sillabe, che
  fanno i fanciuili, quando cominciano a imparare a leggere, il che si dice \textit{compitare},
  cioè contare a una a una le lettere, per poi sommarle, per così dire, in
  una parola; il che si dice, \textit{rilevare}.
\item[CIARPE] Bazzecole, Vedi sopra \cstan[3]{5}.
\item[SPRONAR le scarpe] Detto usato per burlar' uno, che viaggi a piedi.
\item[ANIMA viva] Ancor sopra \cstan[6]{19}, si serve di questo detvo assai usate
  da noi, se ben si sa che l'anima sempre vive; e qui vuol dire, che tutti dormivano.
\item[POLTRIRE] Dormire. Vien da \textit{Poltro}, che vuol dir letto; circa che vedi
  sotto \cstan[9]{19}.
\item[FACIAM pausa] Riposiamoci: o fermiamoci. Frase Latina venuta dal
  Greco, usata anco da noi, i quali da \textit{Pausa} abbiamo fatto \textit{Posa}, e da \textit{Pausare}
  usato pure da' Latini de' tempi bassi, \textit{Posare}.
\item[ARSENAL del vento] Ripostiglio del vento; cioè il ventre. Arsenale vuol
  dire una stanza, entro alla quale si fabbricano i navilj. Dante Inf. C. 21.
  \begin{verse}
    Quale nell' Arzana de' Veneziani.
    \end{verse}
  Ma hoggi si dice \textit{arsenale}, e credo che sia parola corrotta, e venga dal Latino
  \textit{arx navalis}, la quale origine viene approvata-dal Ferrari.\footnote{Arsenale viene più probabilmente dall'arabo \textit{(dar) as-sina'a}, da cui \textit{darsena} ed \textit{arsenale}. Il significato della locuzione araba è \textit{(casa) della fabbrica}.}
\item[PUNGOLARE] Stimolare. \textit{Pungolo} è quel bastone con una punta acutissima
  d'acciaio in cima; del quale si servono i contadini per pungere i buoi, acciocché
  camminino; Lat. \textit{stimulus}. E questo si dice \textit{pungolare}.
\end{description}

\section{Stanza LXI. \& LXII.}
\begin{ottave}
\flagverse{61}Appunto il Generale a far s'è posto\\
Alle minchiate, ed è cosa ridicola\\
Il vederlo ingrugnato, e mal disposto\\
Perché gli e stata morta una verzicola,\\
Le carte ha dato mal, non ha risposto,\\
E poi di non contare anco pericola\\
Sendo scoperto haver di più una carta,\\
Perché di trado, quando ruba, scarta.
\end{ottave}

\begin{ottave}
\flagverse{62}Costoro al fine se gli fanno avanti\\
Per dirgli del prigion c'hanno condotto,\\
Ma e¢ possom predicar ben tutti quanti\\
Perch'egli ch'è nel giuoco un'huom rotto\\
E perde una gran mano di sessanti\\
E gliene duole, e non ci può star sotto\\
Lor non dà retta, e a gagnolar' intento,\\
Pietosamente fa questo lamento.
\end{ottave}


Costoro, che conducevano Piaccianteo, arrivarono al Generale, il quale stava
giuocando alle Minchiate, ma perché egli haveva fatto una mano di errori, e
perdeva, e però era in collera, in vece d'ascoltare quel che essi dicevano, si
messe a dolersi della Fortuna, come sentiremo appresso.

\begin{description}
\item[MINCHIATE] È un giuoco\footnote{Qui il Minucci si cimenta nella descrizione di un gioco abbastanza elaborato, nello spazio di una nota a pié di pagina. La nota si capisce a fondo solo se già si conosce il gioco, altrimenti l'effetto è di destare la curiosità. D'altro canto questa nota è utilissima per riconoscere nel testo del Poema i molti riferimenti al gioco.} assai noto detto anche \textit{Tarocchi}, \textit{Ganellini}, o
  \textit{Germini}. Ma perché è poco usato fuori della nostra Toscana, o almeno diversamente
  da quel che usiamo noi, per intelligenza delle presente Ottava stimo necessario
  sapersi, che il giuco delle Minchiate si fa nella maniera che appresso.
  
  È composto questo giuoco di novantasette carte, delle quali 56. dicono \textit{cartacce}\footnote{Cartacce sono definite nel seguito con miglior precisione.},
  e 40. si dicono \textit{Tarocchi}, ed una, che si dice \textit{il matto}: le carte 56. son divise
  in quattro specie, che si dicono semi, che in quattordici sono effigiati i Denari (che
  da Galeonto Marzio diconsi essere pani antichi contadineschi) in 14. Coppe, in
  14. Spade, ed in 14. Bastoni, ed in ciascuna carta di questi semi comincia da uno
  (che si dice \textit{asso}) fino a dieci, e nell'undecima è figurato un Fante, nella 12. un
  Cavallo, nella 13. una Regina, e nella 14. un Re, e tutte queste carte di semi,
  fuor che i Re si dicon cartacce. Le 40. si dicono \textit{Germini} o \textit{Tarocchi}, e questa 
  voce Tarocchi vuole il Monosino che venga dal Greco: \textit{Etarochi}; qual voce, dice 
  egli con l'Alciato, \textit{denotantur sodales illi, qui cibi causa ad lusum conveniunt}. Ma
  quella voce non so, che sia\footnote{Il Minucci molto correttamente non prende posizione sulla origine del nome \textit{Tarocco}. L'interpretazione più moderna sulla sua origine è che non si sa.}; so bene, che \textit{Hetaeroi}, e \textit{Hetaroi} vuol dire \textit{sodales}, e
  da questa voce diminuita all'usanza latina si può essere fatto \textit{Hetaroculi}, cioè
  \textit{Compagnoni}. Germini forse da \textit{Gemini} segno celeste, che e' fra i Tarocchi col num.
  è il maggiore. In queste carte di Tarocchi sono effigiati diversi Geroglifici, e
  Segni celesti, e ciascuna ha il suo numero da uno fino a 35. e l'ultime cinque fino
  a 40. non hanno numero, ma si distingue dalla figura impressavi, la loro
  maggioranza, che è in questo ordine \textit{Stella}, \textit{Luna}, \textit{Sole}, \textit{Mondo} e \textit{Trombe}, che
  è la maggiore, e sarebbe il numero 40. L'allegoria è, che siccome le stelle sono
  vinte di luce dalla Luna, e la Luna dal Sole, così il Mondo è maggiore del Sole, 
  e la Fama figurata colle Trombe, vale più che ii Mondo, talmente che, anche 
  quando huomo n'è uscito, vive in esso per fama, quando ha fatte azioni gloriose.
  Il Petrarca similmente ne Trionfi fa come un giuoco, perché Amore è superato
  dalla Castità, la Castità dalla Morte, la Morte dalla Fama, e la Fama
  dalla Divinità, la quale eternamente regna. Non è numerata ne anche la carta
  41. ma vi è impressa la figura d'un \textit{Matto}, e questa si confà con ogni carta, e
  con ogni numero, ed è superata, da ogni carta, ma non muor mai, cioè non
  passa mai nel monte dell'avversario, il quale riceve in cambio dei detto Matto
  un'altra cartaccia da quello, che dette il Matto, e, se alla fine del giuoco questo
  che dette il Matto, non ha mai preso carte all'avversario, conviene che gli dia
  il Matto, non havendo altra carta da dare in sua vece, e questo è il caso, nel
  quale si perde il matto; Di tali \textit{Tarocchi} altri si chiamano \textit{nobili} perché contano
  (cioè chi gli ha in mano vince quei punti, che essi vagliono) altri \textit{ignobili}, perché
  non contano. Nobili sono \textit{Uno}, \textit{due}, \textit{tre}, \textit{quattro}, e \textit{cinque}, che la Carta
  dell'\textit{Uno} conta cinque, e l'altre quattro contano tre per ciascuna. Il numero 10.
  13. 20. e 28. fino al 35. inclusive contano cinque per ciascuna, e l'ultime cinque
  contano dieci per ciascuna, e si chiamano \textit{Arie}. Il \textit{Matto} conta cinque, ed ogni
  \textit{Re} conta cinque, e sono ancor' essi fra le carte nobili. Il numero 29 non conta
  se non quando è in \textit{verzicola}, che allora conta cinque, ed una volta meno delle
  compagne respettivamente: Delle dette carte nobili si formano le \textit{Verzicole}, che
  sono ordini, e seguenze almeno di tre carte uguali, come tre Re, o quattro Re;
  o di tre carte andanti, come \textit{uno}, \textit{due}, e \textit{tre}, \textit{quattro}, e \textit{cinque}, o composte, come
  \textit{uno}, 13. e 28. \textit{Uno}, \textit{matto}, e \textit{quaranta}, che sono le \textit{Trombe}; \textit{Dieci} 20. e 30.
  o vero 20. 30. e 40. E queste verzicole vanno mostrate prima, che si cominci il
  giuoco, e messe in tavola, il che si dice \textit{accusare la Verzicola}, Con tutte le verzicole
  si confà il matto, e conta doppiamente, o triplicatamente come fanno l'altre,
  che sono in verzicola, la quale esiste senza matto, e non fa mai verzicola se
  non nell'\textit{uno}, \textit{matto}, e \textit{trombe}. Di queste carte di verzicola si conta il numero
  che vagliono, tre volte, quando però l'avversario non ve la guasti ammazandovene
  una carta, o più, con carte superiori, che in questo caso quelle, che restano,
  contano due voite, se però non restano in seguenza di tre, per esempio: Io
  mostro a principio del giuoco 32. 33. 34. e 35. se mi mi muore il 33. o il 34. che
  rompono la seguenza di tre, la verzicola è guastata, e quelle, che vi restano contano
  solamente due volte per una, ma se mi muore il 32. o il 35. vi resta la seguenza
  di tre, e per conseguenza è verzicola, e contano il lor valore tre volte
  per ciascheduna. \textit{Il Matto}, come s'è detto, non fa seguenza, ma conta sempre
  il suo valore due volte, o tre secondo, che conta la verzicola o guasta, o salvata,
  e quando s'ha più d'una verzicola, con tutte va il \textit{Matto}, ma una sol volta
  conta tre ed il resto conta due; e questo s'intende delle verzicole accusate, e
  mostrate, prima, che si cominci il giuoco, perché quelle fatte con le carte ammazzate
  agli avversarj, come sarebbe; se havendo io il 32. ed il 33. ammazzassi
  all'avversario il 31. o il 34. ho fatta la verzicola, e questa conta due volte.\footnote{Quante volte conta una carta, si riferisce a quante volte si somma al proprio punteggio il valore della carta in questione. La prima volta durante le accuse, la seconda volta se difesa con successo (cioè non ``ammazzata'' dall'avversario), e la terza se contribuisce a formar verzicola. }
  Quando è ammazzata alcuna delle carte nobili, ciascuno avversario segna a colui,
  a cui è stata morta tanti segni, o punti, quanti ne valeva quella tal carta;
  eccetto però di quelle, che sono state mostrate in verzicola, delle quali, sendo
  ammazzate, non si segna cosa alcuna (se non da quello, che per privilegio non
  giuoca) perché tali segni vengono dagli avversarj guadagnati nello scemamento
  del valore di essa verzicola, che dovria contar tre volte, e morendo conta due:
  ed il 29, morendo la verzicola, dove esso entrava, conta solo cinque. L'altre
  carte poi, le quali si dicono carte ignobili, e \textit{cartacce} non contano (se bene ammazzano
  tal volta le nobili, che contano come i tarocchi dal numero 6. in su
  ammazzan tutti i \textit{piccini}, cioè l'1. 2. 3. 4. e 5. dal 14. in su ammazzano anche
  il tredici, e dal 21. in su ammazzano anche il 20. ed ogni tarocco ammazza i 
  Re) ma servono per \textit{rigirare i giuoco}; il qual giuoco appresso di noi non usa se 
  non in quattro persone al più, ed allora si danno 21. carta per ciascuno, e quando
  si giuoca in due, o in tre, se ne danno 25. E giocandosi in quattro persone\footnote{Le regole sul rinunciare al giocare sono per il caso di quattro giocatori, ciascuno per sé.} il
  primo che seguita dopo quello, che ha mescolate le carte in su la mano dritta (che si
  dice haver la mano) ha la facultà di non giuocare, e paga segni trenta a quello
  che nel giuoco piglia l'ultima carta, e questo che piglia l'ultima (che si
  dice far l'ultima) guadagna a ciascuno ai quelli, che hanno giuocato dieci segni.
  Colui, che non giuoca guadagna ancor' egli de i morti, cioè segna ancor lui il valore
  della carta a colui, al quale e ammazzata detta carta. Se questo primo 
  giuoca, il secondo ha la facultà di non giuocare pagando 40. segni, se il 2. giuoca
  il 3. ha detta facultà pagando 50. segni, se il 3. giuoca passa la facultà nel 4.
  che paga 60. segni come sopra. Ma se il giuoco e solamente in tre persone, non
  ci è questa facultà di non giocare.

  Mescolate che sono le carte, quello de i giocatori, che è a mano sinistra di
  quello, che ha mescolato, n'alza una parte, e se v'è volta nel fondo di quella parte
  del mazzo, che gli resta in mano una delle carte aobili, o un tarocco dal 21. al
  27. inclusive, la piglia, e seguita a pigliarle fino a che non vi trova una carta
  ignobile: Quello, che ha mescolate le carte dopo haverne date a ciascuno, ed a
  se stesso dodici la prima girata, e tredici la seconda, e scoperta a tutti l'ultima
  carta la scuopre anche a se medesimo, e poi guarda quella, che segue, e la piglia
  se sarà carta nobile, o tarocco dal 21. al 27. e seguita a pigliarne come sopra, e
  questo si dice rubare, e queste carte, che si rubano, e si scuoprono, sendo nobili,
  guadagnano a colui, a chi si scoprono, o che le ruba, tanti segni, quanti ne vagliono;
  e coloro, che le rubano è necessario, che scartino; cioè si levino di mano
  altrettante carte a loro elezione, quante ne hanno rubate per ridurre le lor carte
  al numero adeguato a quello de i compagni; e chi non scarta, o per altro accidente
  di carte mal contate, si trova da ultimo con più carte, o con meno degli 
  avversarj per pena del suo errore non conta i punti, che vagliono le sue carte,
  ma se ne va a monte; Colui, che da le carte, se ne da più, o meno del numero
  stabilito, paga 20. punti a ciascuno degli avversarj, e chi se ne trova in mano
  più, e deve scartare quelle, che ha di più; ma non può far vacanza\footnote{vacanza, altrove detto \textit{faglio}, magari noto come \textit{piombo} dal tressette.}, cioè gli
  deve rimanere di quel seme, che egli scarta; Se ne ha meno, la deve cavar dal
  monte a sua elezione, ma senza vederla per di dentro, cioè chieder la quinta, o 
  o la sesta, ec. di quelle, che sono nel monte, e quello, che mescolò le carte (che
  si dice \textit{far le carte}) fattele alzare gli dà quella, che ha chiesto.
  
  Cominciasi il giuoco dal mostrar le verzicole, che uno ha in mano, poi il primo
  dopo quello, che ha mescolate le carte in su la mano destra, mette in tavola
  una carta, (il che si dice \textit{dare}) quegli altri, che seguono devon dare del medesimo
  seme, se ne hanno; e non ne havendo devono dar tarocco, e quello si dice
  \textit{non rispondere}. E dando del medesimo seme si dice \textit{rispondere}. Chi non risponde,
  ed ha in mano di quel seme, che è stato messo in tavola, paga un sessanta punti
  a ciascuno, e rende quella carta nobile, che havesse ammazzato; per esempio il
  primo dà il Re di danari, ed il secondo benché habbia denari in mano, da un
  Tarocco sopra il Re, e l'ammazza; scoperto di haver in mano denari, rende
  il Re a colui di chi era, e paga agli avversarj sessanta punti per ciascuno, come
  s'è detto. Ogni tarocco piglia tutti i semi, e fra lor tarocchi il maggior numero
  piglia il minore, ed il matto non piglia mai, e non è preso, se non nel caso detto
  di sopra. Così si seguita dando le carte, ed il primo a dare e quello che piglia
  le carte date; ed ognuno si studia di pigliare all'avversario le carte, che contano,
  e quando s'è finito di dare tutte le carte, che s'hanno in mano ciascuno conta
  le carte, che ha prese, ed havendone di più delle sue 25.\footnote{\textit{Le sue 25.}, intende \textit{quante ne aveva ricevute}.  25 è per il caso del giocare in due o in tre.  Giocando in quattro, la base è 21.} segna a chi l'ha meno
  tanti punti, quante sono le carte che ha di più, dipoi conta i suoi onori, cioè
  il valore delle carte nobili, e verzicole, che si trova in esse sue carte, e segna
  all' avversario tanti punti, quanti con li suoi onori conta più di esso, ed ogni
  sessanta punti si mette da banda un segno, il quale si chiama \textit{un sessanta}, e questi
  \textit{sessanti} valutavano secondo il concordato. E tanto mi pare che basti per facilitare
  l'intelligenza delle presenti ottave a chi non fusse pratico del giuoco delle
  Minchiate, che usiamo noi Toscani, che è assai differente da quello, che con le
  medesime carte usano quelli dalla Liguria\footnote{Del gioco Ligure con le 97 carte delle Minchiate non è restata traccia. In altre parti d'Italia si usano ancora mazzi formati da 56 cartacce e 22 tarocchi (Tarocco Piemontese, da 78 carte), o 22 tarocchi e una selezione delle 56 cartacce (Tarocchino Bolognese, 62 carte; Tarocco Siciliano, 64 carte), per giochi simili fra loro, ma abbastanza differenti dalle Minchiate Fiorentine.}, che lo dicono \textit{Ganellini}; perché
  \textit{Minchiate} in quei paesi è parola oscena. Da questo giuoco vengono molte maniere
  di dire; come \textit{essere il matto fra tarocchi}; \textit{entrare in tutte le verzicole}; \textit{Essere le
  Trombe}, \textit{carracce}; \textit{Contare}; \textit{non contare}; e simili.
\item[INGRUGNATO] In collera. Chi s'adira, o entra in collera suol mostrarlo
  con la mutazione di volto, torcendo la bocca, o increspando la fronte, con
  atti simili, che si dice anche \textit{far muso}, e \textit{far grugno}, o \textit{ingrugnare}. Vedi sopra \cstan[2]{57}.
  Lasca Nov.~10. \textit{Ma Beco non la potendo sgozzare se ne stava ingrugnato
    anzi che no}. Dicesi anche \textit{portare, tener broncio}; \textit{imbronciare}. Nonio Marcello
  antico Gramatico. \textit{Bronci sunt producto ore, \& dentibus prominentibus}.
\item[AMMAZZATA una Verzicola] \textit{Ammazzare}, \textit{rubare}, \textit{scartare}, \textit{dar mal le
  carte}, \textit{non contare}, \textit{verzicola}, \textit{non rispondere}, \textit{sessanti}, ec. leggi quel che habbiamo
  detto qui sopra alla voce Minchiate.
\item[HVOMO rotto] Huomo collerico. Lat. \textit{praeceps in iram}, che si dice ancora in
  questo senso huomo precipitoso.
\item[NON ci può star sotto] Non la può soffrire. Lat. \textit{sustinere}, \textit{pati}.
\item[LOR non dà retta] Non bada, o non attende a quel che essi dicono. Non dà
  orecchie. Lat. \textit{non facilem accomodat aurem}. \textit{Dar retta} in altro senso dissero gli
  gli antichi nelle cose di guerra, per quello che i Latini dissero, \textit{imperum sustinere}.
\item[GAGNOLARE] Rammaricarsi. Vedi sopra \cstan[4]{9}.
\end{description}
\section{Stanza LXIII. --- LXVI.}
\begin{ottave}
\flagverse{63}Che t'ho io fatto mai fortuna ria \\
Che t'hai con me sì grand' inimicizia; \\
Mentre tu mi fai perder tuttavia \\
Che e' non mi tocca pure a dir: Galizia? \\
Questo non si farebbe anch' in Turchia,\\
L'è proprio un'impietade, un'ingiustizia;\\
Vedi, non lo negar che tu l'hai meco;\\
E poi sen' avvedrebbe Nanni cieco.
\end{ottave}

\begin{ottave}
\flagverse{64}Ma, se volubil sei quanto sdegnosa,\\
Facciam la pace, manda via lo sdegno;\\
E se tu sei de' miseri pietosa, \\
Danne, col farmi vincer, qualche segno, \\
Fu il vincer sempre mai lodevol cosa,\\
Vincasi per fortuna, o per ingegno,\\
Perciò de' danni miei restando sazia,\\
La Fortuna mi sia non la Disgrazia.
\end{ottave}

\begin{ottave}
\flagverse{65}Ma che gracch'io? FOrse che tai preghiere\\
Mi faran dopo così gran disdetta\\
Vincer la posta, o porre a Cavaliere!\\
Sì sì, ma basta poi non haver fretta.\\
O Baccellaccio! l'orso sogna pere\\
L'è bell'e vinta, ovvia tientela stretta.\\
Capitale! Sai tu quel che tu hai a fare!\\
Se tu non vuoi più perder, non giuocare.
\end{ottave}

\begin{ottave}
\flagverse{66}E così finiran tanti schiamazzi\\
Di chiamar la Fortuna, e i giuochi ingiusti, \\
Che mentre vi ti ficchi, e vi t'ammazzi \\
Tu spendi, e paghi il Boia che ti frusti.\\
Gli è ver, ma il libriccin del Paonazzi,\\
Ov'io ritrovo ognor tutti i miei gusti,\\
Per forza al giuoco mi richiama, e invita\\
Appunto, come il ferro a calamita.
\end{ottave}

Il Generale si duole della Fortuna perché gli è contraria, e lo fa perder sempre:
la prega a volersi mutare, ed essergli una volta favorevole: e con l'Ariosto
\cstan[15]{1}.\ dice \textit{Fu il vincere, ec}. Ma poi accorgendosi, che il suo pregare
è inutile, riprende se medesimo, del vizio, che ha di giocare, ma conosce, che
l'ammonizioni non sono abili a farlo desistere dal giuocare.
\begin{description}
\item[NON mi tocca a dir: Galizia] Non ho punto il conto mio. Il Bronzino in lode
  della Galea disse:
  \begin{verse}
    \backspace E se non ne facean tanto romore
    Non saria lor toccato a dir: Galizia;
    Tanta gente n'andava per amore.
  \end{verse}
  Ed il Persiani dolendosi, che un suo fratello era più lesto, e più astuto di lui disse:
  \begin{verse}
    \backspace E prima: Il mio fratello è una giustizia,
    Che mi rivede molto bene il pelo,
    I credev' esser furbo, e giuro al Cielo
    Che seco non mi tocca a dir: Galizia.
  \end{verse}
  Da questo che dice il Persiani può, chi legge, comprendere il vero senso di questo detto.
\item[NON si farebb' anch' in Turchia] Non si farebbe in luogo veruno, ne a persona
  del mondo, se ben fusse il maggior nostro nimico, come è il Turco. Vedi
  sopra \cstan[5]{6}.
\item[SEN' avvedrebbe Nanni cieco] Lo conoscerebbe uno, che non havesse giudizio;
  Lo vedrebbe un Cieco, come era Nanni. Il Proverbio dice: \textit{come disse Nanni
    cieco}, e senz'altra aggiunta s'intende, \textit{vedere}, perché questo Nanni cieco diceva
  sempre, \textit{vedere}, Si dice anche semplicemente \textit{Nanni cieco}, e s'intende il
  medesimo. Si dice anche: \textit{Lo vedrebbe Cimabue, che nacque cieco}, o che havea gli
  occhi di panno, detto antichissimo, vendendo da Cimabue, ritrovatore della Pittura
  in Firenze, non perché egli fusse cieco, ma si voleva denotare, che egli 
  fusse nato al mondo cieco, vive affatto al buio del disegno. I Greci \textit{Vel caeco
    clarum}.

\item[MA che gracchio io?] Ma che sto io a ciarlare in vano. \textit{Gracchiare} è il gracidare 
  della Cornacchia, o del graccio, quasi Lat. \textit{gracculare}. Ma ci serve per esprimere
  un cicalare senza fondamento, senza frutto, o al vento, Vedi sopra \cstan[1]{69}
  \cstan[4]{25} e \cstan[7]{59}. Ser Brunetto Latini nel Pataffio; in quel
  verso: \textit{Mi disse, s'io non fo,  ch' avrem cornacchie?} volle dire in gergo; alludendo
  al suono della cornacchia; Che avremo per il giorno di domani, Lat. \textit{cras}.
\item[DISDETTA] Disgrazia. Mala fortuna. È il contrario di \textit{Detta}, che vuol
 dir buona fortuna nei giuoco, o in altro. Sp. \textit{desdicha} L. \textit{malum fatum}, \textit{mala sors}.
\item[VINCER la posta] Guadagnare quello, che va in giuoco. Vedi sotto in \cstan{75}.\
  e vuol dire vincere una volta sola.
\item[PORRE a Cavaliere] Rimaner superiore, Cavaliere si chiama quella Torretta,
  che nelle Fortezze avanza sopra a tutte le muraglie della medesima fortezza; e di
  qui \textit{Essere}, o \textit{stare a Cavaliere}, vuol dire Esser superiore, o avanzare il compagno.
  Varchi Stor. lib. 9. \textit{Tutta questa parte delle mura di qua d'Arno non havendo
    ne monti, ne colli sopraccapi, non può dal di sopra, o (come si dice) a cavaliere essere offesa}.
\item[BACCELLACCIO] Scimunito, Sciocco; Insensato. Augusto Imperadore
  diceva \textit{bacelus}.
\item[L'orso sogna pere] Ognuno si figura di goder quel ch'ei vorrebbe, ognuno sogna
  quel ch'ei brama. Virg. ed. 8. \textit{an qui amant ipsi sibi somma fingunt}. Vedi sopra
  \cstan[2]{7}. E per qual causa ti dica \textit{l'orso}, e non altri animali, Vedi \cstan[1]{31}.
  Teocrito disse: \textit{Omnis canis panem somniat, ec.}
\item[CAPITALE] Questo termine oltr'a i significati, che dicemmo sopra \cstan[7]{82}
  profferito nel modo, che è nel presente luogo, ha la forza del Latino
  \textit{Utinam} e vuol dire piaccia a Dio, che non sia per essere, e che non segua, in
  contrario.
\item[SCHIAMAZZO] Romore, Strepito. Traslato dalle galline, il gridar delle
  quali si dice schiamazzare. Il Vocabolista Bolognese dice, che il verbo schiamazzare
  significa Esclamare in darno, dal Verbo Greco \textit{Sciamocheo}, che vale
  \textit{pugnare cum umbra}, Ma è vanità; perché schiamazzo vien dal Latino \textit{exclamatio}.
\item[VI ti ficchi e vi t'ammazzi] In questo caso son quasi Sinonimi, e significano
  immergersi, o applicarsi tutto a una cosa.
\item[PAGHI il boia che ti frusti] Spendi per haver danno. Teognide\footnote{} disse: \textit{Sibi
  ipsi vincula cudit}.
\item[LIBRICCINO del Paonazzi] Intende carte da giocare, perché già un tale de'
  Paonazzi fabbricava dette carte.
\item[APPUNTO come il ferro a calamita] Per simpatia, come fa la calamita al ferro;
  per questo detta da Franzesi \textit{aimant}, cioè Pietra amante.
\end{description}
\section{Stanza LXVII. --- LXX.}
\begin{ottave}
\flagverse{67}E sarà ver, ch'io habbia a star soggetto \\
Ad una cosa, che mi dà tormento? \\
Come tormento ? oibo ! s'io ci ho diletto.\\
Sì ma intanto per lui vivo scontento.\\
O perfido giuocaccio ! e maledetto\\
Chi t'ha trovato, e me, che ti frequento,\\
Tu non ci hai colpa tu, a me il gastigo\\
Si dee dar, poiché con te m'intrigo.
\end{ottave}

\begin{ottave}
\flagverse{68}Datemi dunque un mazzo in su la testa\\
Vedete; eccomi qui ch'io non mi muovo,\\
Ne voi farete cosa men che honesta,\\
Se dal giocar, morendo, io mi rimouovo,\\
So c'ogni dì farebbe questa festa,\\
C'altro diletto, che giocar non provo,\\
Ed a giuocare omai son tanto avvezzo\\
Che 'l pentirmi non giovami da zezzo.
\end{ottave}

\begin{ottave}
\flagverse{69}L'usare ogni sapere, ogni mia possa\\
Non vale a farmi contro al gioco schermo,\\
Imperocch'io l'ho fitto sì nell'ossa\\
C'amo il mio mal qual assetato infermo,\\
E forse giocherò dentr' alla fossa,\\
Che forse? diciam pur: tengo per fermo;\\
E se trovar le carte ivi non posso,\\
Farò, (pur che e' si giochi) all' aliosso.
\end{ottave}

\begin{ottave}
\flagverse{70}Van co i libri alla fossa i gran Dottori,\\
I bravi con la spada, e col pugnale\\
Con libro, ed armi anch'io da giuocatori\\
Sarò portato morto al funerale,\\
Grillandato di fiori, e a picche, e cuori\\
Trapunta havrò la veste, e per guanciale\\
Quattro mattoni, e poi che pien di vermini\\
I quarti avrò, vo' far un quarto a Germini.
\end{ottave}


Seguita il Generale a lamentarsi, e combattendo in lui la voglia del giuocare,
con la ragione, e con la convenienza, prega gli amici, che l'ammazzino,
perché vede, che non c'è altro modo; che egli si rimanga di giocare; anzi chi par
d'esser certo d'havere a giocare anche dopo morte, e che alla sepoltura vuol
andare con le carte da giocare nel feretro nella maniera, che esprime con
l'Ottava 70.

\begin{description}
\item[OIBÒ] Questa voce ha diversi significati, perché ce ne serviamo per negativa
come nel presente luogo: per dimostrazione di naufea, come oibò, che schifezza,
e questa ? (orto C, 10, tt. 23. per riprenfione, o difapprovazione: Oibò non fare tal
cosa,ed esprime il latino Kab, \& espace, E gue) che i Greci distero Aibos. Di
ciamo anche: aibo, eibo, e ibò.

\item[SCONTENTO] Scontclato, disgustaro » La \letter{s}, aggiunta nel princi
pio di nomi, verbi, ec, ha nel pariar nofire la forza, che appresso a i Latini ha
particella i» privativa di Circa di che vedi il Varchi nell' Hercola. E corrispon
de alla particella ex.

\item[MAZZO] Quei martellone di legno, che adoprano i Macellari a dare in su
la tela a' buoi, donde mazzuola quella, che a Roma adoprano per ammazzare
i malfattori. Si dice anche mageio, nia questo € propriamente quello, che ado
prano i bottai a cerchiar le botti, Dal Latino malleus.

\item[FARE schermo contro al gioco] Difendersi, o riposarti dat non giocare. Vien
  dal verbo schermire, che vuoi dire Eiercitarsi per imparare a difendersi da i colpi
  il qual viene dal Germano be/chirmen, siccome vuole il Voto. Dan. Inf. C. 13.
  \begin{verse}
    \backspace O Giacopo dicea da Sant'Andrea,
    Che t'è giovato di me fare schermo?
    \verseprefix{\textls[-110]{Il Petr.Son.17.}}Ch'i non son forte ad aspettar la luce
    Di questa donna, e non so fare schermo
    Di luoghi tenebrosi, e a' hore tarde ?
  \end{verse}
\item[L'HO fitto nell'ossa] Ho un desiderio di giocare internatissimo; Virgilio del
  giovane innamorato disse, Georg. lib. 3. Quid iuvenis magnum cui versat in ossibus
  ignem Durus amor? E il Petrarca.
  \begin{verse}
    E ricercami le midolle, e l'ossa,
  \end{verse}
\item[AMO il mio mal qual assetato infermo] Come brama: il febbricitante di bee
  che gli e nocivo, così bramo io di giocare, che mi e dannoso, she?
\item[ALIOSSO] Come habbiamo detto sopra \cst[1]{9}. tutti li giuochi di fortuna
  da i Latini si dicono alea: da che io deduco, che questa voce Aliosso venga dal
  Latino alea, \& osso, e significhi, come in efictto significa osso da giocare, qual è il
  Talus da i Latini, e l'aftragalo de i Greci. Dicesi ancora Catriosso; quasi quadro,
  cioè dado di osso. Quest' osso si trova nelle gambe di dietro di tutti gli animali d'ugna
  fesse, come nell' agnello, bue, ec, che negli animali d'ugna sode, come il
  Cavallo, ec, o ditate come il Lione, ec. non si trova, eccetto, che nell'Alicorno
  secondo Pol. Virg. \libcap[2]{13}. e Dianel Soutero de Aleatoribus lib. primo
  Cap. 25., Buleng. de lud, Veter. c. 58. ed è un' ossetto di figura quadrilunga da
  una parte concavo, e dall'altra convesso: Nel mezzo del concavo apparisce un
  picciol buco, ed il convesso, che è la parte opposta al concavo, forma: in ciascuna
  delle sue fiancate due piccoli buchi; nelle testate del fianco al concavo, e
  convesso sono  due superficie quasi piana, se non che in una si vede un segno come
  un S, e nell' altra un segno come un 8., e queste duc parti quando l'Alioffo si
  butta in tavola sono le più difficili a rimanere scoperte, perché ono di più dificil
  posare, del concavo, e del convesso, e l'altre due fiancate non restano mai scoperte,
  perché niuna per la sua rotondità può posare. I nostri ragazzi dell'infima
  plebe, nel giuocare con quest'osso s'adattano a quei segni, servendosene per numero
  con fare il concavo il numero uno, il convesso farina, cioè nulla, per esser
  questo il più facile a rimanere scoperto, la parte dove è il segno 8. vince otto
  perché tiene la figura di quel numero; e da' Greci quello numero, di otto negli
  aliossi era chiamato \textit{Stesichoro}, e la parte dove è il segno S, vinca dodici, perché
  ha figura quasi di libra, che si divide in 12. parti; o secondo, che convengono,
  diversificando, o variando questo giuoco, secondo i patti: E l'usano detti ragazzi
  dalla Pasqua di Resurrezione (nel qual tempo s'ammazzano gli agnelli,
  nelle zampe de' quali si trovano questi ossi) fino a che vengono le pesche, ed allora
  lasciato l' Aliosso, e giuocano ai noccioli ne i modi detti sopra \cst[3]{37}.
  al qual giuoco durano a giuocare, fino a che stiacciati i noccioli vendono l'anime
  di essi agli spezziali, che sarà per tutto Ottobre in circa  e da questo tempo
  fino a Quaresima giuocano alla rulla, o alle buche com la palla di legno nel
  modo, che si disse sopra \cst[3]{57}.; e per tutta la Quarefima giocano alla trottola.
  E così distribuiscono i loro trattenimenti per tutto l'anno. Ma tornando
  all'Aliosso: appresso agli antichi Romani era usato dagli huomini più sensati, ed
  in diverse maniere; e fra l'altre il concavo era chiamato Cane, o canicula forse
  da quella stella lucida, che si vede nella bocca del Cane Celeste; stella cattiva, e
  malefica; e colui, che tirando faceva apparire detto lato, posava in tavola due
  denari, o quello che erano convenuti fra loro i giocatori, ed era cattivo, onde
  Persio disse. Damnosa Canicula quantum Raderet, la parte opposta a detta era
  chiamata Venus stella benigna, e benefica; e significava il num, ses Latino Senio,
  da noi detto Sino, nel giuoco dello Sbaraglino, quasi Seino da' Greci chiamato
  Hexites, e chi tirando scopriva questa Venere guadagnava sei, e tutto quello,
  che havevano posato in tavola coloro, che havevano scoperto Cane, o Canicola.
  Giulio Polluce lib. 9. dice, che da i più, il Sei era chiamato Coo, e il Cane,
  ovvero l'asso; Chio: e che in questo lor talo non havevano, ne il due, ne il cinque.
  Con quello osso giocavano tanto i Greci, quanto i Latini in altre maniere,
  e fino con sei, e otto ossi per volta, ma a me basta haver accennata la suddetta
  per testimonio, che anticamente ancora era in uso questo giuoco; e tralascio di
  narrare l'altre maniere che son molte, perché non fa a proposito nostro, ma
  se il Lettore ne fosse curioso legga Polid. Verg. \libcap[2]{13}, Daniel Soutero de
  Aleatoribus \libcap[pr]{29}. Buleng. de lud. Vet. Cap. 58. \& Alex. ab Alex. Dierum
  gen. \libcap[3]{21}. Ho detto, che questo Aliosso oggi è giuoco da' ragazzi,
  ed il nostro Autore ci addita questa verità, facendo dire dal Generale: farò,
  perché si giochi, all'aliosso, Se trovar le carte ivi non posso; e intende: voglio giocar
  sempre, e se non troverò carte, giuocherò all'aliosso, quantunque sia giuoco da
  ragazzi, pur ch'io soddisfaccia al vizioso genio, che ho di giocare.

\item[VAN co libri, ec] A' Dottori, quando portati alla sepoltura è costume di
  mettere nel feretro, o bara i libri, ed a i Cavalieri la spada al fianco sinistro, e così
  dice, che fara fatto a lui, che per far conoscere, che mentre visse era giuocatore,
  gli faranno una ghirlanda di quei fiori, che sono impressi nelle carte, la sua
  veste fara ricamata di picche, e di cuori, e sotto la testa gli metteranno quattro
  mattoni; ed in questa maniera havrà anch'egli attorno tutti quattro i semi, che
  sono impressi nelle carte da giocare a primiera.

\item[FAR un quarto a' germini] Giocare in quattro alle minchiate, Vedi sopra in
\cst{61}.

\end{description}
\section{Stanza LXXI. STANZA LXX}
\begin{ottave}
\flagverse{71}Volea seguir, ma tutti della stanza \\
Gli dieron su la voce con il dire, \\
Che il perdere è comune, e star' usanza, \\
E perde una miseria di tre lire, \\
Però si quieti pure, e habbia speranza,\\
C' un giorno la disdetta ha da finire, \\
Però che i tempi variabili sono, \\
E dopo il tristo n' ha a venire il buono.
\end{ottave}

\begin{ottave}
\flagverse{72}Intanto gli mostraron il Prigione,\\
Che sott' il manto dell'Ipocrisia \\
In carità, dicendo, in divozione \\
Faceva lo scultore, idest la spia; \\
Però, perch' in effetto egli è un guidone \\
L' impicchi s' ei vuoi far opera pia: \\
Serragli pur, dicean, la gola, e poi, \\
S' ei dice più nulla, apponlo a noi.
\end{ottave}

\begin{ottave}
\flagverse{73}Amostante ch'è buon di buona pasta,\\
E por da bene, ancor ch'egli habbia il vizio\\
Di questo suo giocar, dov'ei si guasta,\\
Fa liberarlo senz'alcun supplizio,\\
Dicendo c'a impiccarle non gli basta\\
L' haver femp.icemienteunpo dm o
Ma quand snch ezti havesse ciò commesso\\
Del far la spia non se ne fa processo.
\end{ottave}

\begin{ottave}
\flagverse{74}Ed al prigion preterito imperfetto\\
Rivolto con le carte in man l'invita,\\
Già fattoselo porre a dirimpetto \\
A giocar d'una crazia la partita\\
Ovver si metta fuor in sul buffetto\\
Un testoncino, e sia guerra finita:\\
Così lo prega, lo sconginra, e in parte\\
Bada pur sempre a mescolar le carte.
\end{ottave}


Voleva il Generale contiouare il suo lamento, ma i circostanti lo fecero quietare
consolandolo, e mostrandogli, ch' ei si faceva scorgere a far tanto scalpore
per una perdita di sì pochi soldi: Intanto gli presentarono Piaccianteo dicendogli,
che lo facesse impiccare, perché egli era Spia; Ma il Generale buon huomo
lo fece liberare, dicendo, che un poco d'indizio non era bastante a farlo
impiccare, ed oltre a questo del far la spia non se ne fa ne meno precetto, ed intende
che se s'havessero a fare impiccare tutte le spie ci sarebbe faccenda. Di poi il
medesimo Generale invita Piaccianteo a giocar seco di poco, e solo per trattenersi.
Nel che il Poeta esprime il vizio internato di giuocare, che era nel Generale, poiché
nello stesso tempo, che determina di non voler mai più giocare, torna a mettersi
a giocare fino con un vil prigione, con l'ansietà, che mostra in quell'atto di
attender sempre a mescolar le carte; come fanno coloro, che punti dal giuoco,
per haver perduto, vorrebbono pur trovare con chi giocare per ricattarsi.

\begin{description}
\item[GLI dieron fu la voce] Lo fecero chetare. Latino: \textit{Vocem alicui comprimere}.
\item[PERDE una miseria di tre lire] Perde poco. La voce \textit{miseria}, che per altro
  significa infelicità, o avarizia, usata in questi termini serve per avvilire; e però
  esprime qui una somma di niuna considerazione.\footnote{Che 3 lire, equivalenti a 180 quattrini, siano di \textit{niuna considerazione} è abbastanza discutibile, o forse indicativo dell'ambito in cui si muove il Minucci.}
\item[SOTTO a manto d'Ipocrisia] Sotto scusa, sotto pretesto, sotto coperta di far
  del bene.
\item[FACEVA lo scultore] Cioè faceva l'ascoltatore, e non lo statuario, ed intende,
  Stava alla scolta, cioè stava ascoltando i discorsi d'altri per ridirgli; e con
  questo termine equivoco viene a dir copertamente Far \textit{la spia}, come dichiara il
  Poeta medesimo.
\item[GUIDONE] Furfante, Huomo d'infima plebe senza riputazione, Vedi sopra \cst[1]{65}.
\item[APPONLO a noi] \textit{Illius crimen affinge nobis}. Se e' fa più la spia, gastiga noi.
  Ti assicuriamo, o t'entriamo mallevadori, che e' non farà più la spia. È lo
  stesso, che \textit{mio danno}, che vedremo sotto \cst[11]{49}. cioè \textit{mio sia il danno, se non
  segue così, come io dico}.
\item[HVOMO di buona pasta] Huomo di buona natura. Latino \textit{Oleo tranquillior}.
 Plautus in Poenulo. \textit{Ita hunc canem faciam tibi oleo tranquilliorem}, farò stare zitto,
 com' olio.
\item[DOV' ei si guasta] Dove egli pecca. Con che egli varia la sua buona natura.
\item[DEL far la spia non se ne fa processo] Gastigar uno senza far processo vuol dir
  Gasticarlo sommariamente. Latino \textit{indicta causa}, o più tosto, \textit{de plano}, cioè
  senza solennità di giudizio, senza sedere a banco di ragione, o come si dice anche
  volgarmente \textit{pro tribunals}; ma qui par che voglia dire, che le spie noa solo
  non si gastigano, ma ne anche se ne fa processo.
\item[PRIGION preterito imperfetto] La voce preterito, che suona passato, qui vuol
  dir, che il prigione era dietro al Generale; e la voce imperfetto denota l'imperfezione,
  e vigliaccheria di Piaccianteo.
\item[UN testoncino] Testone è una moneta, che vale tre paoli, e da molti in occasione
  di giuoco si dice \textit{Un testoncino}, per intendere giochiamo solo un testone, e \textit{sia
  guerra finita}, cioè non si giuochi più.

\item[BADA a mescolare ve carte] Con questa azione di badare (cioè continovare) a
  mescolar le carte invitando colui a giocare esprime, come habbiamo detto, la
  gran voglia, che il Generale ha di giocare.
\end{description}
\section{Stanza LXXV. \& LXXVI.}

\begin{ottave}
\flagverse{75}Quegli, che compiacerlo non gli costa,\\
E vede haverl'havuta a buon mercato; \\
L'invito tiene, e regge a ogni posta, \\
Ben ch'ei non habbia un bagattino allato,\\
E dice, al più faremo una batosta \\
Quand' ei mi vinca, e voglia esser pagato, \\
Di rapa sangue non si può cavare,\\
Ne far due cose, perdere, e pagare.
\end{ottave}

\begin{ottave}
\flagverse{76}Duraro a battagliar forse tre hore,\\
Poi la levaron quasi che del pari;\\
Se non ch' il General fu vincitore\\
Di certa po di somma di danari,\\
E perché gli domanda, e fa scalpore,\\
Quei, che gli spese in cene, e in desinari,\\
Non haver (dice) manco assegnamento\\
Tal c' Amostance resta al fallimento.
\end{ottave}


Piaccianteo accetta l'invito, e messisi a giocare il Generale rimase in vincita
d'alquanti denari; ma perché Piancianteo non ne haveva il Generale non fu pagato.
Così fa la Fortuna, quando perseguita un giuocatore facendolo vincere
solamente quando non vi è modo d'esser pagato.

\begin{description}

\item[L'HA havuta a buon mercato] Ha scampato un gran pericolo con facilità, cioè
  non ha havuta quella pena, o gastigo, che egli conosceva di meritare.

\item[TIENE l'invito] Accetta l'invito, e s'accorda a giocare.

\item[REGGE a ogni posta] Posta (trattandosi di giuoco) vuol dir quella somma di 
  danaro, che i giuocatori concordano, che corra volta per volta nel giuoco, che
  si dice \textit{invitare}, e \textit{reggere a ogni posta}, s'intende tenere tutti gl'inviti.

\item[BAGATTINO] La quarta parte del quattrino Fiorentino, con altro none
detto Picciolo, Latino \textit{We obolum quidem, Voce}. È moneta Veneziana.\footnote{\textit{Bagattino} è il nome assunto in molte parti d'Italia dal \textit{Denaro} introdotto da Carlo Magno. Alla ripresa dell'economia europea dalla fine del secolo XII, le varie nazioni europee ---dove il denaro s'era svalutato in maniera indipendente--- iniziarono ad essere coniati multipli del denaro.  A Firenze quattro denari formarono il \textit{quattrino}.  Quasi ovunque dodici denari formavano un \textit{soldo}. Denaro, Picciolo, Bagattino, e tutti nomi equivalenti, ma il cui valore dipende dalla nazione emittente. }

\item[FARE una batosta] Combattere, e questionare con parole, ec. Latino \textit{Altercari},
  ed habbiamo ancora il verbo \textit{batostare}, per combattere, battagliare. Storia
  di Semifonte trattato quarto, \textit{Non havendo tanta gente, che bastasse per la
    Terra batostare}. E più sotto. \textit{Hor di qua, hor di là si batostasse}.

\item[NON si può cavar di rapa sangue] Non i può cavare una cosa di dove ella non
  è. Latino. \textit{Aquam e pumice postulare}. Plauto. \textit{Nam tu aquam e pumice non
    postulas, qui ipsus sitiat}.

\item[LA levaron quasi del pari] Cioè s'intende \textit{la scrittura}: Non vi corse quasi niente,
  cioè si vinse, e si perdé poco.

\item[FA scalpore] Fa romore; Contende alzando la voce. 

\item[NON haver manco assegnamento] Non haver danari, ne modo da trovarne.
  E la voce \textit{manco} in questi termini ha la forza del Latino, \textit{nec etiam}, ovvero \textit{Ne
    quidem}, che noi pure diciamo, \textit{ne pure}, \textit{ne meno}, \textit{ne anco}. Io credo, che sia
  voe corrotta da \textit{ne anco}.

\item[RESTA al fallimento] Resta con quel credito da non risguoter mai, perché
  fallito s'intende colui, che non ha denari, ne assegnamenti.
\end{description}
\section*{FINE DELL' OITTAVO CANTARE}
\end{document}
ae



: ARGOMENTO
" Ginnti i rinfrescbi, e inusgorito il Campo

ie

Qe

a
Corre all' affalto, e segue aspra baruffa;

~ Malmantil quaft e preso, ond' al un scampo %,
f Chiama all' accordo, e termina la rufa, [ae
i Chi tratta più di guerra hor trova inceampo, >

a = Perché nell' allegrezze ognun si tnffa,
5 Faffi in Corte il conuito, e poi, dal vino
. Riscaldati quei Principi, il feltine. Ol
«ERPS EARS Pb Pape Pape ce ere? 7
: Nie
as 48
STANZAI. STANZAILIL
{1}LA guerra, ch' in Latino e detta bello
Par brutta a me in volgar per sei Befane,
Non c'altro se e' comincia quel bordello
Di quell' artiglierie, che fom mal fane,
E che e' non v'è da metter' in castello;
E slenti poi per altro com' un cane
Sere' un quattrino, e pien di vitupero
( Ditelo vei, se questo e un bel mestiero,

{2}E pur la gente corre, e vi s' accampa
Ognit per farfiua huomo, e acquistar gradi,
Quasi degli buomms cosa sia la frampa
Mentr! il canarne l'ossa avvien aradi,
LA gli buomin si disfanno, e chi ne scapa
Ha tirato diciatto con tre dadi,
E pria ch' ei ginnga a esser Caporale '
CHangierd certo, più d'un fraiodi fale,

{3}Sì che e' mi par ben tondo,ed un corriva,
Chi pus fear bene in casa allezro, e sano,
E lascia il proprio per ? appellativo
Cercando miglior pan, che que! di grano,
Cen' un' altra ancorch'io non arrixe 5
Ch' e quell'assalir un con Parmi in mano,
Che non fol non m' ha fatto viliania,
Ma, che mai viddi in viso in vita mia,

{4}Florsit cerchi chi vuot bartagliae risse,
E si chiarifea, e prow: un po le chiare,
Che s' io. credessi farmi un altro Viifse
L'armi,percioné m'hano ainzapognare,
Ognuno ha il suo capriccio, come disse
Quel Lanzoghe volea farft impiccare,
Pero mi quiero, ma perc' bora brama
Atoftrarus il vero;attenti, e cominciamo,

Per introduzione de! presente Cantare, nel quale il Poeta vuol deferiver | af.
Ito dato. a Malmantie, si serve della dimostrazione, che la guerra sia una brutta
cosa, e che pero habbiano poco giudizio coloro, che vt vauno; perché se bene
i Latini la chi o Bello ( il che secondo alcunt facevano per aatifrali, cig'

Gee 2 pec



420 MALMANTILE

per una figura di parlare contraria a » che s'intende, c
bosco, che € senza luce; Parce le, che memine proctnt
guerra, che non ha in se cosa alcuna di bello, egli nondimeno
tissima, e ripiena di pericoli, come farebbe a dire i colpi 3
abbondante di patimenti, e stenti come farebbe il non haver, che
non haver mai denari; onde un Poeta\footnote{} per ispiegar la bruttezza di
Lelia horrida bella, Oltre a questo @ contro alle ragioni della
gnar l'armi a danno di chi mai ci fece ingiuria alcuna, disse un G
lum a beluis dicirur, perché \& cosa da bestie, Si maraviglia pero
vada volentieri ingannata dalla speranza, che in quella si face
¢ non s' accorgono, che più tosto vi si disfanno, e quand' anche g
ci vuol degli anni prima, che uno conseguisca i minori gradi della o
la guerra Vx fol ne premia, € un million ne ammiazra. Conchivde p
vo di giudizio colui, che potendo stare a casa sua con ogni commodo,
trigarsi con la guerra, e che quanto a se, quand' anche fusse certo d
ventare il maggior' huomo del mondo, non si lascera mai lusingare da
ranze: Ma perché egli fa, che ognuno può far di se a suo modo, sosp
scorrer più de i mali, che nascono dalla guerra, e s' accinge a;
con deferivere l'affalto dato a Malmantile dall' esercito di Baldone.
IN volgare, Cioè a parlare chiaro, fuor di gramatica. '
BRVTT A per jei befane, Befane come dicemmo si C..8, st. 30. vu
Panioccio fatto di cenci, e di qui per Befana intendiamo non solamente
na brutta, e mal fatta; Ma le Balie si servono della voce befana per i
una di quelle Larue, che nuocono a i bambini, come il Baw, er.;'¢ gli p
no, che ci sia la Befana cattiva, e la buona, e che venga nellecale perk
del cammino del focolare, e però la notte avanti al giorno dell' Bpifania,
Gio, Villani lib. 7, e I nostro popolo anc' oggi chiama Befania, onde;
mente vien questo nome di Befana, come s'\& detto sopra, fanno che i
appicchino le calze a i cammini, perché le dette Befane gliel' empiano di
buona, o cattiva, secondo, che essi sono stati 6 buoni, o cattivi ze tali
buone, o eattive si figurano sempre brutte; onde bratro per sei befane vuol dit
estremamente brutto. J Filofofi scolastici per esprimer più la, che i
dicono M2 «fo, dando alle qualita gradi fino in otto, e volgarmente per elprimt
lo Reflo si dice Sei, come di sei corre, ec, se bene e un termine, che ha
furbelco, Cscala per sei putte, e simili. Ll Ferrari cavando la definizione
na dal Politi Autor Sanese la descrive così: Larwale simulacrum, ——
nia puerss terriculamentum Suspenditur; unde nomen invenit, B foggiunfe w
mulieres deformes Befane dicuntur larua illa turpiores. Dice finalmente, che i Frar
cel dicono Tphanie dal Greco Tbeophama, cioè Apparizione.d' iddio.
nocte danno ad intendere le superftiziot(e, e ignoranti femmine a' semplici
li, che seguano molte cose fuor dell' ordine della natura., miracoloic,
per esser la vigilia della festa de' Magi, né sanno, che con questo nt ¢
Persiani, ond' ebbe origine, eran chiamati i Savi, e intendenti
Natura, delle Stelle, e de! Ciclo. ia 3 NM

at

EPS \&

eee FF FaTRRSE

oe: esx EB >i

WEL bordello, La voce bordello, che propriamente vuol ie i igo



NONO CANTARE. qat

blico dove abitano:le meretrici., e presa da noi in più fenfi, come per frepito, 0

per una cosa stucchevole:, e noiosa, come è presa nel presente luogo, e altri la

iglian inteoder Difficulta, o fatica »comela prefe il Lalli nella sua Ea.tr.

le paroled: Verg: Hoc opus bic labor,:

enn Ene aio bello 5

8 et Cafacalda si va presto presto;

}) gameboeene| 2-1 | Ma vitornar in su, quefioe il bordello,

aa "0 è da mettere in Castelo. Specie di pariar Janadattico, del quale par-

2 a. sopra C, 1. st, 29. alla voce /eminato, es' intende Non v' è da mettere in
~,

» che significa poi; non v' e reba da mertere sm corpo, cioè non y'è da man-

'. In furbesco; Non v'è da smorfire; Non-v' e da empiere il fuflo, che così

, dicesiil corpo nello stesso modo, che in Greco volgare si dice Cormi da literale
a Corner, che vuol dire Fusso,o Ceppo, Latino /ipes, candex.

| ~ STENT A come uncane, Patilci, ed hai carestia delle cose necessarie.al vivere.

: eo della Caccia lib. 5. Ergo age duro dffuescant vittu catuli, Si dice frentar

bracco, quando uno per la sua poverta ha male il modo di provvedersi il

we

ie mee

-PIENO ai vitupero. Pieno di pidocchi, rogna, ed altre tattere, e porcherie
4 i indivitbil della soldzvefea yi chet dice anche: Pieno “4s Bobbio, dal

— Latinovepprobrium, ebbrobrio ) e Peno di fastidio; del resto Vitupere significa infamia

bye vergegna, Bocc, Nov. 63.
in ET.. Abi vitupero del guasto mondo
ptt T] medesimo Boccaccio nella Teleide lib. 1.

BOs Abi vitupero della gente Achiva,

ee Omero, e Epimenide citato da', Paolo diticro in questo senso mala probra, cioè
id vituperosi.
o Per farff haomo. Per diventar un' huomo valoroso: Che essere un huomo, 6
sit an'huome, serve apprelso di noi per intender quello, che intendeva Diogene,
of ES diceva 1 Aominem quero, diccfi Esser un' huomo Givven. f wis efe aliquis,
ie scrittara Confortamini, \& essere robuffi, Omero, Viri effore, \& forte cor fumite,
af VCHivescampa. Scampare vuol dire fuggire, scappare, o liberarsi da un peri-
PI colo + € qui intende chi eicè vivo, o avanza alla guerra, Scampare, quali u/cire
J dal.campo; dalla battaglia. !
| o MA twraterdiciotto con tre dadi. Ha havuto la maggior fortuna, che si possa,
haere,/perché il cum, 18, ¢)il maggiore, che si potia fare con tre adi. 1.Gre~
J cl pure ond eae dicevano: Ter /ex iattare, come si ricava da Giulio
| Pollucesnell? Onomattico. Sy aah it
Bi CAPORALE. Capo di fquadra, che fra gli Vfiziali e il minor grado, che si
j dia nella milizia, Caporati disser gli antichi per Principale,Latino Capitalis. Gio,
Villani 1. 28. parlando di Roma dice:

ee Fu caporale regno di se medesima
— Biib. 12.89. eA tutte le caporali Citia a' tralia.
La voce è formata dall' antico plurale 'Capora', come Campora, Borgora', e

simili. °°:
- MANGERA pri: duno fraio di fale, Significa-consumesa molto tempo, perch
x molto

—

4uz MALMANTILE™
molto tempo ci vuole aun' huomo solo a consumare 'uno f

chi, quando volevano significare-un 'tempo lungo; dicevano com
che sled da mangiare a d' un -moggio di sale, Cicerone de Ami
que illud est, quod vulgo dicitur wultos modior 'falis simul edendos esse
nus expletiim fit, Questamaniera proverbiale pure in. pro
usata da Piurarco nel libro della multiplicita degli amici » Si può
che inghiottira più d' un boccone amaro 5: e di poco suo:gusto.
con troppo fale si dice amara; e pero mangiando molto faleman
amaro. ' ' +.
TONDO, e-corrivo.. Si poslon dir-sinonimi; e il primo signific
fo, ed inGpido, ed il secondo, che:si dice\anche: Corribo., huomo leg
cile a creder' ogni cosa. Latino credu/xs:.. 1\Napolerani dicono ¢
minchionare, burlare.,.¢ dar. pasto'a uno; sopra-C..6..f, 80, disse.«
tondi più dell' O di Giotto, chesuona loyftessa, Tonsa fimiimente: pre
}i vale balordo, dappoco., semplice, goffoxCunto degli cunti?
Bue.:
LASCIAR il proprio per  appellative. Maniera di dire tratta dalla
in cui si danno nomi di due forte, alcuni chiamati propri, altri appell
dire; Lasciar il certo per It incerto, Bar come il Cand' Biopo ci
che haveva in bocca,per pigliar quella,della quale vedeva,lo shattimento 4
qua, che gli pareva magguore, e lo stesso significato ha; Cercarmmighor
grano, Eliodo Poeta Greco: Folle e colui, che lascia andar le cose facili
¢ con certa speme segue le pin difficili, e lomrane..\4\ pene
10 non arrino, Cioè lo noa comprendo + lo non arrivo col mio giudizio a it
tendere. In lingua furbesca.. foo» ammasco s non redo  cive non piglio, nonae
zanno, non comprendo. Lat. non affeguor. iru

ESPs SGP SSE

ee ee rset

VILLANIA, [ngiuria,Sopruso, mal termine s LG

S? io credessi farmi un nuouo Viiffe ec, s'io credefsi di diventare il maggior hut
mo del mondo. Diciamo Va nuoxe Orlando. L Greci Alter Hercules, 'gh

SI chiarisca col pronar le chiare, S' accerti di que(ta cosa con provare le feri
perché chiara intendiamo quell' albume deli' uova, i quale s' adopra a medicit
le ferite, vedi sopra \cstan[1]{60}. ed il Poeta servendosi del verbo ehrarive che
vuol dire (caponire, o (gannare,€ della voce chiare fa nascere lo (cherad.
 INZAMPOGNARE., Ingannar con infinghe. Lat, Verba dare:ed è 10 hielo
che iatinocchiare detto sopra \cstan[7]{14}. Dalla natura del suono, e della Me
fica.; incancatrice delle meoti degli nomin.. Fra tutti gli trumenti però. que d
fiato, levano più di (esto, e pare, che percuotano l'anima più gagit ¢
onde furono, ad esclufione degli altri,usati nelle battaglie, nelle quali facevad
meitieri tor via da cuori l'appreafione del pericolo., e infonderni la a
speranza. Noi habbiamo un Proverbio. Far come i Pifferi di montagna (|
nator di piffero strumento di fiato contadine(co.) che andarono per piferare
rono pifferats. Volcano minchionare gli altri coi, darne, c. furonc.
col toccaine. Fare uno cornamu/a appresso il Pulci, ¢'] Burchiclioe
inzampognare verbo facto da stampogna strumento di fiato rusticale, così a
Symphonia, della qual voce servcndosi Daniello al cap. 3. nell' Litoria

cn x aie



 ciulli 5

= a

ae

S

S=etis \&. BEx es a i EO

, NONO CANTARE.

43

¢ narrando che essi non attesero-punto il cenno., che per comando Regio
si dava, @ adorare la Statua, col suono di tromba, di cevera, di finfonia:, e di

ees ae suoni; sg si può dire [ siami lecito qui dt servicmi. di questa baga ma-
inzampognare,come giù altri. Tromper in Vranz,

=» e pur dal Latino carmina,

we

be Dis LEAN ZAWV..
2 aurora, e come diligente
azza le stelle in Cielo, ¢fapulito,
eae ffi alla finestra d! Oriente
Evora l orinal del suo Marito,
. Ma perché il Carretton ricco, e lucente

. Acciocch'ei non la vegga/cociase/ciarta,
eee: amegerneved ei iirimpiatta.
: STANZA:
Quande il vutto easone ' (rinfresco '
St che,chi hauea col masticar dinieto,
pe oma iecamente il corpo al desco,
E come si /uot dir ) riebbe il pero;
ae Hi General, che tutta notte al fresco
nda con? Afirolabio innanzi, e indreto,
Batrendo la Diana in sul lunario
Hanea fatto di Stelle un calendario,

~ Edi nostro Autore dice =

. Già muone il Sole,ed ella U' ha sentiza,.

as = forse a corno, o tromba de' ciurmatori: E Charmer Ancantayes >

UGNFNO ha il suo capriccio. Virg. Quifque fuas patimur manes, Ognuno ha je
: fantalic, Vo Lanzo, essendo riprelo, perché faceva cose da esser impiccato,
ve Che solerce tire » lasciatte far a ie, perché ho ancor ie mie pelle capricce, BE chi
ha Lanzw, Vedi sopra \cstan[1]{52}. e c. 4. stan, 36.

STANZA VII.
seaienaat era anch' egli riuedere
Tutto quanto aggrez.rato al pappalecco,
Done per hauer meglio it suo doxere
Fece in principioun bel murare afecca:
Quando fu pieno,al fin chiese da bere,
E poi ch' egli hebbe in molle polto ilbecco:
Fighnoli, 3 4iffe, omai venutat l'hora,
Ch' e' si tratta d' hanerla acauar fuora.
STANZA VIIiL.

S' a mensa ognun di voi tanto s' affolta,
Atangia per quattro, e bewerpoi per fecte 5
Che par proprio che sia giuntoa ricolta,
Anrich'egls bablia afar le fuevendette,
Tat ch' io pensai vedern' anc' una volta
La tonaglia ingoiare, e le faluette,
Ed bebbi un tratto anche di me paura,
Per una spalla dauola sicura,

“I nostro Poeca de(crivendo la levata del Sole imita Daate nel Purg, C.2.dove,
descrivendo anch' egli il parcir dell” Aurora dice:

65 bid Sicche le bianche, e le vermiglie.guance
La doue io era, dela bella Aurora,
Per troppa etade dieninan rance.

Accio ch' ei non la vegea sconcia, e [ciatta,
Manda giù impannata, e si rimpiatta.

Ed intendono Vaoy et Alero, che quei colore, 1: quale appariva nell' Oriz-
lente per causa dell' Aurora, era quai sparito; ed in iu queit' hora comparve la
munizione da bocca, edi soldati i rinfrescarono. Dopo di che il Generale det-
'We principio a far l'orazione per inanimire i Soldau j quaic Orazione militace si
Soutiene nelle presenti stanze fettima, e ottava, e nelle quaccro segueati.

ere de fielle in Cielo, e fa pulico,. L' Aurora coi ivy ipicndore, offusca

quelio



4z4 MALMANTILE ~

quello delle Stelle, € così le leva dai Cielo e lo fgombra;
VOT AP orinas del uo Mdarito, Cioè del vecchio Titone favol
la Aurora, Virg, Tithont croveum tinquens Aurora cubile. D
cubina di Titon antico già s* imbiancaua al balzo a' Oriente Puor delle
dolce amico, Qui pero descrive Aurora nei suo primo app
la parola  imbiancana. Li noitro Poeta poi-, per 'Vorsmale e J
tende quella rugiada, la quale caica sopr' alla terra cicca'l apparic
la qual' hora l'Alba, o Aurora si perde, pero dice Adanda gin o impan
rimpiatta, cioè ferra le tinciire, es' asconde J “
SCONCIA, e feiatta. Si potion dir Sinonimi.. Se bene /oon cia
mente dire una Donna, che non si sia ancora accomodata icapelli
quale accomodaiento di capelit dice Accunciatu ase feiatea vaold
scompotta, e che habbta gu abiti male adattati, e agguitati
sconcio e più generica, che nome la voce /sarto, core:
tine. Znconcinnus, inbonestus, wdecens, incompositus',
1M? ANNAT A, Così chiawiiamo questeiat di legno sportellatt
tono alle fineilre per chiuderle con carta, tela 50 vetsr, che vi si
fendersi dai treddo,o dal Soe, \& mandar giù  émpannaca vuoldir se
tclio di gueito telaio, e chiuder la fineitra; perché per lo più deceit:
aggiu(tati in maniera, che per aprire, e Chiudere s'\ alzano, ed. abbul
diciamo tar fu, e manaar gin. 6
SJ rimpiatta, S? a(conde. Vedi sopra C, 7. stam. 66.
HAVE A col mafwar dimcto, A chi era vietato i mangiare t
havevano ) traslato da 1 Magittract di Firenze, Re' quaii ti dice baxer
non poter conseguirgli, e aver proibizione per qualche tempo di et
jut, che v' habbra parenu, oche gi habbia efereman di corto, Oo) per  @
givni ttabilite dalle leggi. Dan. Purg, C. 14. one
Lav' e meffier ds conforto Diniero, asthe
Negli Statuti Fiorentini diceti barbaramecate Dewerum ou itl
LIET AMENT E, Vuol dire-Allegramente da lito; se bene i noltti Contile
pi dicono /eramenre in vece di prettamente; e forse qui i Autore lo Cee
fio tenlo; perché si può credere, che 1 soldatis' accostassero \& mangiare:
gramente, e prestamente. Li Lat edacer donde e venutu il Poscano Allegri s*
1 Frangele Alaigre ( che più mostra la iua origine ) vale pronto, H
E /efo per avventura può eiler fatto da servs ae
AP POGGLARE il corpo al desco. Si dice anche di chi rifeuote danari o prove
fione da banco, o luogo pubbico. Cie accoltarti alla menia per mangiare.
RIEBBE wf pero, Svritociiia + Ripreie forza sok pero quello tia, vedi
6. ttan, 107. Del riavere i) peto vedi una curiosa noveilettain Giovannt:
te,detto Gioviano Poatano,ne! Diaiogo iniacolato earenio p
cipio. Del maic che:fa al vento caccaiuly, o del beue, che neit:
cice, se ne legge un'epig) Greco di Nv » melita 3 1
dire Fiorita Kaccolta de' medetunt bpigramun 51 quaic cradon ave
suona così. Peditus occadst muitos incinjus in aluo; Lipiojts batoo,
Seriat,@ occidie rurfum si peditus; ergo Regibus augustis qui

Fz ER THELIST.

Be

eee 2 baw ere



cue NONOCANTARE ~- — 45

BATTENDO 14 Diana in sul lunario, Tremando dal freddo per essere thao
all' aria a considerar le stelle. Batrer la Diana, Vuol dir battere il tamburo all'
pparir del giorno, quando si vede la Stella mattutina, ovvero Stella Diana, cioè

del di. Ma per metafora intendiamo battere i denti per il freddo, che di- ae

mo anche barter la bora, Vedi sopra \cstan[8]{6}, >, a

 TVTTO aggrezzato, Intirizzato per il freddo; Affiderato; Agghiacciato;;

sghiadato; morto di freddo. sggrinzato truovafi nell' antico per secco, e
liato di carne, quali sogliono restare i morti ( appellati perciò da Greci /i-

res, ci0è privi d' umidore, secondo che vuole Pjutarco nel libro intitolaro 'J
inal sia de' due più profitrenole; ! acqua, o pure i fuoco, e quali si veggono essere

is mie structe, smunte, e secche. da Aggrinzaro forte e nato Aggrizzato, p
| PAPPALECCO, Antende al mangiamento in generale: che per altro Pappa-

 decco se - leccornia, ghiottornia (Franzese; friandife ) come habbiamo veduto

1C.7. stan. 55.
i hes Os niece il suo donere, ec, Mostra che il Generale, essendo affamato,
yi aifolratle anch' egli a mangiare, acciocché gli toccasse la sua parte; intenden-
j ' do che mangio assai prima di bere Tee murare a secco, vuol dir murare senza
eaicina o alcro bicume, ma con i foii safi, e trateandosi di mangtare vuol cir
jot Mangiare senza bere. Nell' antico facevano la parte a mangiare, e a cia~
feheduno toccava la sua; il Juffo poi levd questa usanza; dice Plucarco nelle Que-
 stioni Conviviali lib, 2. g. 10.
; MESSE il becco in molie, Vuol dit bere, pigliandosi la voce becco, che vuol dir
re il rosteo degli uccellr, per la bocea deli huomo, Questo detto merrer il becco in
molle Gguinca anche parlare, aprir la bocca. Gli Spagnuoli la faccia dell' humo

dicot roffro da quella degli uccelli.
i 'S' afolta'. S? atfacica con furia, e con vehemenza.
im STA sitmo 4 ricotra, Cioè ch' e' G sia nell' abbondanza maggiore, come si sup-
pone che e' si sia nel tempo, che si faono le raccolte: Se forse nua voletim» dire,
che costoro mangiando facevano uno sparecchiare simile a quello, che fanno
coloro che fegano 1 grano, ec.

PAR cbt egli habbia a far le sue vendette. Quand' altri mangia, e beve assai,o
fa quaififia operazione sen' iatermiiione, riposo, o rispiarmo, ci serviano di
questo detto, assomigliando quel tale a uno, che per vendicarsi portato dall' ira
Opert veementemente.

PER una spalla davola sicura.M'era entrato così gran timore, che non mangiassero
anche me, che d'accordo havrei daca una delle mie spalle per confecuarim: 1 ceito,

STANZA IX. STANZA xX.
Redeamus ad rem; Se ( come ho detto') Che quasi fui per dar nelle girelle,
Qua fuste al ber infer mie al magiar fani, Perché dopo ch' i punti della Luna

Eco+ coltelli sn man, (Pandoui a petto, Hebbs deferitti, e che extse'le Helle
| Runfeiste si brani (parapant, fic Haneuo rassegnate ad una ad una
bli battaglia vedervi ancora aspetto Trouo smarrite haver le Gallinelle:
| Con la spada così menar le mant, Ma dopo è, ch' io mi dauo alla fortund,
Ona ib aimico vino, ed abbartuto Che fra le elle fiffe, efral' erranti,
NNe sia, come franotre ho preveduto « Won vedenone anche i Mercatanti,
VR Hhh <2 Ska

CRRREBALERE EMASE.

=



26 MALMANTILE

STANZA XI.

M€a dissi poi da me, che poco importa
Se quel branco di Polli non si troua
Ani che questo a noi risparrio apporta,
Peroche magian molto, e non fann' nova;

E [e ne anche alcuna Stella ho scorta
De! Mercatanti, qui creder mi gioua,
Che e'fieno in fierayo vero al lor viaggio,
Per laViaLatrea a mercatar formaggio, Essi cerchin la roba, e mo
Seguita il Generale la sua orazione militare, con la quale dopo hai
suoi Soldati di bravi nella maniera, che si vede, termina suo
che si vada ad affaitare il nimico, perché spera, che sieno per h;
tuna per le ragioni, che dice, con le quali da un poco di bur! ara
FVSTE al bere infermi, al mangiar fani, Beveste, e mang te assai,
gi' Intermi per lo più vorrcbbono sempre bere, ed i fani mangiano
cassai.:
ST-ANDOVI a petio co' coltelli in mano. Par che voglia dire,
fronte per far alle coltellate » ed intende, che stavano a mensa uno
altro co' coltelli in mano per tagliar pane, e c,, ec.
SPAR AP ANT, Così diciamo per derisione a un bravazzone, e qui ton
ne, perché questi soldati mangiavano gran quantità di pane, 4 '
PIÙ per dar nelle gireile. Fui per dare la volta al cerucllo. Vedi sopraC.t.
GALLINELLE, Quelle sette Stelle, che si veggono fra il Tauro, ef
dette Pleradi; in Lat. Vergilie, Il comento d' Arato Latino. Pleiades 4 plartit
te Graci vocant, Latini eo guod Vere exoriantur Vergilias dunt. Aicum dil
Pleiades sieno nominati, quasi Plefiades cio che si Ranno accoflo,per.
ci le chiamaton anche B try, cioè Grappol d' uva, e noi Galinelley p
piccole, e in un mucchio. Lt Vberti nel Dittamondo.
Poi disse: guarda nella frome a quelle,
Le qua' da' fani 'Pliadi [on dette,
E che i volear le chiaman Gallinelle, 4
AU! dauo alla fortuna, Mi tribolayo: Mi disperavo: Si dice an
alle freghe, al diauolo, alla versiera, alle bertucce, a' cani e simili,
fortuna: tratto per avventura, da' Marinari, quando disperati, ab
in braccio alla borra(ca; la quale da' nostri Toscani fortuna di mare 5¢
folutamente vien detta. Il Petrarca s' era dato in un certo, modo alla
quando,descrivendo il suo stato infelice diceva. a wi
Fra si contrari venti in frale barca.
Ui trouo in alto mar senza gouerno,.
E poi. Ch'io medesmo non so quel ch'io
MERE AT ANTI. Le tre stelle del cingold @
Tauro, così dette perché sono infeme, e paion compagne,
ragione. Adercatante dicevano gli antichi quel che noi. oggi p
-reante. L' arte de' Mercatanti nella nostra Città ancora al,
servato l'antico nome.,: % '

SREERGERE

2RERSE

ee ae



NONO CANTARE. 27

 BRANCO 4i polli.Latende le Gallinelle dette di sopra.ll Ferrarialla voce Branca
dice in fondo: Branco eream pro grege.Vin branco-di pecore.Vaa mano di pecore.
Mon n pro mulritudine, ec, Manus autem est branca, ut alibi anumaduerfurm,
REDER mi vious che fien per la Via Latcea, ec. Scherzando con queiti aoint di
clot Gallinelle, e Mercatanti discorce di esse, come se quelle fussero galline,
che son difatili,perché mangiano, e non fanno uova, e che questi Mer-
i non eran nel Cicio, percèé erano andati a provvederd di formaggio
Via Lactea, la quale egli suppone di latte, e che pero vi sia il formaggio a
Mercato; e conchiade, che ancor questi sono difutili, perché fond intenti
ente a' guadagni, e non si curano di gloria di guerre; e pero che è bene, che
. questi non si trovino in Cielo, perché torna a ior favore, e pero si poila
8 “ entrar' in guerra con buono augurio. Ridicole confeguenze altrologiche, con le
'quali mottra la poca stima, che egli fa dell' Astrologia come di cosa frivola, e vana,
— Fra laren, 8 quel circolo bianco, che divide da una parte all' altra l'Oriz-
"-zonté, edi nose i vede 1m Ciclo la meta, il quale dicono tia formato di minucil
fime fielle; Da molti è cniamato /a va Romana, Dan. nel Parad, C. 14. la chia-
| m0 Galafia, dalla voce Greca, colla quale questo yalibul cercnio del Ciclo si caia-
Ma Galaxsas, cive laccco,
| Come distinta da minori in maggi
ee Lum biancheggia tras pols del mondo,
a Galafiass, che fa dubbiar ben Sages,
SON boti; Son huonuni di gesso, e di Aucco; che s'intende huomini buo-
ot at ia yilolidi; Lat. frpites, caudices. Vedi sopra C. 4. tan. 17. e sotto C.
ws Tt, fap, 41. Similitudine tratta da quell' immagini, che appicca nelic Chicle chi
ge 8 botato. In ispagayoio Sore e (puatato, che ha il caglio morto, Lat, hebes,
age tt Oftde boro de ingenso vale huomo d' ingegno poco vivace; ouylo.
se | DANNO te ferste con (a penna, Cioè terilcono sella borla, quando scrivono
Te partite in debico a uno. EB verameute le partite in debita sono ferite, perché
sidiceL denars sono it secondo sangue, i) quale con tali ferite si cava d' addosso al
Proilimo, Così i dice volgarmcnte Tarare ana frecesa, calui, che chiede a uo' al-
tro in danari,vedi topra ( 2.¢ insdguinarti chi comincia a toccar guattrini,
sh) Dl dar foro, Deve dare, coe divicae lor dzbitore, e per l'equivoco inten-
de deve Perquocergli; e da cio cava la coalegueuza, che noa fiea buont per la
Suerra, poiche se cia piantaav una partica ( snteadi dispongono una parte, una
quaama di Soldati Jognuno gli dee dare (taccadi perquocere tali Soldati ) e
j gueilt che da tutti ac coccane, boo son buoui per la guerra, Psancare una par-
Ma Cinferire, o descrivere nel Giorudle, o uubro di uegozio una parte, o arcico-
lo, capo di (crittura, che da dcbuo, € credito a chi s' alpetia; 1 che si dice
anche decendere una parsita, decendere uno debore, e creanwe, toric dal Latino
recerfere, deiccivere, regiltrare.
STANZA XIIL

| Non prima fabili l'andare in GMErTA, Com un bratcod uccelliil quale in terra
Che vede/ts pie presto ch' 10 nol dico Sts calato a beccar grano, o paniva;
Vitleuaiena, «ur trattoyun ferra ferray Va che si muons basta, che quct solo
Ed ir correnas contr' Alil inimne. £4 fuoice pyuare a tats nw volo,
è: Hhh z STAN:

zat,



428 MALMANTILE™
STANZA XIV.
J coraggioft al primo, che si mosse,
Gli altri (già fendo meglio [ui piccenali )
Non poterono star più alle mosse,
Ma corsero ancor lor come Terzuoli 5
Giunti di Malmantile in su le fose,
Drizrate al muro assai feale a pinuoli
i falirvi renewano una baia
Com' andar pe' piccions in colombaia.
STANZA
Gli fiipits, le foglie, e gli architraui
A quclto efecto essendo già (murati
Per via di curri, dargani, e ditrani
Gli hanevan su le mura strascinati, Faceano un venga addossoat
Stabuito d' entrare in guerra, e dar U affalto a Malmantile i più ¢
rono i primi a muoverdi, e gli altri meno coraggiosi (eguicarono. \&
Dante, che nei Purg. C, 2, dice:
Come quando cogliendo o biada, o loglio
J colombi adunati alla paffura
Quieti senza mostrar U usato orgoglio 5
Se cosa appar ond' essi habbian paura
Subitamente lasciano fRar ? esca
Perché affaliti son da maggior cura,
Arrivati dunque alle mura di Malmantile, credendosi di trovar fac
s' ingannarono, perché quei di sopra gagliardamente si difendevano
altro. Qui e da considerare, che se bene Capitelli, e Srontespizzs son me
shitettura, il Poeta (cherzando con l'equivoco.di capi, e fronti, e serve
verbo Pampare nel senso, che lo pigliano i Legnaiuoli, ec, che dicen
C. 1, tt. 8., vuol die, che tali merii pictre, ed altro devano sopra 1 2
alic fronti dei soldati, e gli stampavane, cive gli facevano di quei-
chiamano stampe, ed in sustanza vuol dire, che rompevano tefle, e
suono, che rendono i corpi battuti fecero i Greci il lor verbo typrein,
re; da questo verbo ne venne Typus voce pur Greca accettata da'|
una forma imprefia, o cavata fuori col battere: Se ne fece ancora 7}
tamburo, che Omero più conforme all' origine disse Tympanon seguito
Catullo nel Poema Gailiambico. Noi abbiamo voci da riferire a queste' \
come farebbe Stampa, Stampita, Stampare, Stampanare, Ma in pro
fiampe fatte sul mostaccic d' un' antico Giucatore di pugna, evvi un
gramma del Greco Lucilio, che in nostra lingua voltato dice Così;
2 un vaglio, Appollofane, il tua capo,
O qual fu mai pin traforato arnese,
Son tane di formiche 90r dritte, or torte,
E par, che con bizzarre, e varie nore
Un Lirico eccellente il Lidio v' abbia
Inravolate sopra, ol Frigio canto.

escestie?

jie te te en ee i a.


NONO CANTARE,
6 Or franco vibra il minacevol pugno
 Ecombarci pur liero in duro arringo 5
 Che se colpo novelio a te discende,:
Quel ch' ai riscoffo, aurai, ma non già nuond
et Capir nel capo tuo potra ferita,
PIP preso chrio nol dico,, Prestitiino consumaron manco tempo a far tal cosa,di
silo, che io consumo a dirlo. Latino dicto citsus.
“N lena leua, un ferra ferra, Quando vogliamo intendere, che una gran quan:
: di popolo adunata in qualche luogo si sia partita in un subito, e velocemente
ia one di questo.detto 5B signiticano quasi lo stesso, se non che l'ultimo ef-

» quando uno è da altri incaizato a correr, ec, vedi sopra C. 1, st. 63. e»

ke
- f hail

pero nel p luogo si potrebbe anche dere, che i primi volon-
 tarj, ed 1 secondi forzati dalla riputazione. 11 Varchi Stor, lib. 2. dice: Pa /ubir
| Wegridato: armi armi, lena lena, ferra serra, ec, Dal che si cava, che questo detto
tog significhi Leva la roba di sopr' alle;moftce delle botteghe, e ferrale come (eguiva
at | Firenze nelle sollevaziont di popolo, e che ii medesimo detto sia poi facto co-

Mune a oga: sorta di tumulto, e per ¢sprimer un moto turioso di qunatita di po-

4
Ll

| AR correnda. Andar correndo. Il verbo ire venendo dal Latino, vale appresso
di aot qunato il verbo anaare, ma ci serviamo solo dell' Intinito ire, del partici-
Pi9 ito, o solo, o accompagnato col verbo essere, e dell' Lmperfetto ina, ixano,
che si dice poi, giva, e giwano, Nella vita di Cosa di Rienzo (critta in lingua Ro-

Mana antica trovali jio, e seffero, e simili, che i Toscani cangiando l'[ coafonan-
foi *ealpra nella doice lettera si dicono gio, cioè andò, € gifero, cioè andassero.
wi fimiimente prende alcuni tempi, come farebbe i presenti di tutti i modi,
'i dai verbo Vado, io vo; ancorche Dante viatle foresticramente, edadi per Vada;
gg o 0i0 cofretto dalia rima.

» ST ANDÒ mestio in sui piccinoli, Essendo pi gagliardi nelle gambe; e questo
gi AVVeniva, perché havevano mangiato. £ piccinols, che è il gambo delle fruttes.
g Latino pedicutus, e pref comunemente in questo caso per le gambe dell' huomo,
ia NON porersero Rar falc alie mosse. Non potettero contenersi, che non corref-
a fero. Toho da j Wavalli Bacbari, i quali corrono a i palj, che essendo tenuti per

lo freno dai loro Stallon: al luogo donde a) suono della tromba deeono partirsi,
7 che si dice le mosse ( Latino carceres ) molte volte scappano, prima che sia dato i)
' detto segno, e questo si dice non far ferme aie mosse, che poi passato in proverbio
! non haver pazzicnza, o lofferenza, ma per il gran desiderio d' arriva-
i Tea Uo luogo, partirsi prima del dovere; ed esprime quella inquietudine, che uno
, hanelitaspercar, che segua una tal cosa da iui ansiosamente bramata. Del Ca-

vallo generoso Virg. Georg. 3.
Stare loco nescit, micat auribus,si tremit artus,
Colettumgue premens volvit sub naribus ignem,

CORSERO come terzxoli, Corsero con la stessa velocita,con la quale vola alla.
preda il terzuoio (pecie di falcone. Perché così sia detto rende ta ragione il Tua-
No de re accipitraria lib. 1, edtrque ad co. cum tres foetu enitatur eodem Predones gene~
rosa parens mas kitinus imo despectus letto incet appellatur, \& inde Tertius,

SCA-



4jo MALMANTYPLE \&
SCALE 4 pivoli. Scale fabricate di due corredti «
glioni sono pivoli ficcati fra 'uho ¥'¢ l'altrore C
fine in distanza uguale a riscontro, ovvero'i detti f
o stecche, o regoli di legno conficcati in deeti correntt Mampati
riscontro. B pinole, ( Latino clanicx/a, civxt cavicchio; ovvero
de ogni pezzo di-bastone adattato a porerd mettere in un buco,
TENEV ANO una baia, Stimavand cof: facile;*Stima
burla, ec. Latino mage, Ii Ferrari dice poter venire questa voce da
iflar' a bada, in ozio, Latino wataré, o O01 i
COLOA18 ALE, Quelle stanze fabricate per lo più nelle form
per uso de i colombi, € nelle quali wascono i piceionit) «>
FEC ERO parergli altro suono, Fecero lor conotcere, che |
ment.. voy
ewERLI, Qvei picco}i murelli'; in distanza uguale'y ned quali per!
mioano te muraghie delle Città, ¢servond per: parapetti'. ad soldati,
per difesa della muraglia; così dette quali. murnlesdice il Berrari; fume
primes parus murs,Dichiamo @una-cosa;che ancora abbia delle dific
rarsi, e che non si fiano per anco spuntate: £ ci è de merio, cioè non è elpy
to il cutto, Ci rea ancora qualche parte da abbattere 2 Vedi sotto o 12)
ISSO fatro. Subito. Due voci Latine corrotte, e ridotte Toscane,
loro lo stesso signincato.
DISEATTO (e restuggini, Infrante le Testuggini animali Terrestri,
che hanno la coccia, o guscio durissimo da alcuni detti Tartaruche
he, da altri bezzache ( dal bezzicare,.ch' elle fanno raspando in terra
atinl Tefudmes, E § potrinanche dire, che ? Autore intendetle di qu
razions da guerra, che usavano gli antichi dette Te/udines; nelle gi;
no foo alle mura, reggendosi fulie spalie gli uni gli altri, e aiutandofia m
tarui sopra, coperti turu di scudi, e terran iteme per ripararsi da' colpi, che
si (cagliavano per di sopra; E questa operazione s' addimandava refixggine spe
ché stavano col capo, € 'colla vita dentro agli icudi, come stanno le
(in Lp. torragas in Beanz. ortaes ) dentro alle loro scodelle 3 le quali )
dette da' quei dello stato di Muang, come racconta il Ferrari bi/se fo
bijce (codellaie, perché anno 1i capo di bilcia, e stanno rinchiufe cone i
della; Onde potrebbenfi dire, dom:porte, come un' antico Poeta chiamé le chien
de. Autione famoso ceteratore e fatto parlare da Pacuvio così, delcrive
tetluggine con que' versi portati da Cicerone de divin, ub. 2. Q@madrapes 1am
da, agrestis, bumilis, aspera, Capite breui, cernice anguina, adjpettu trad
ruche, e BR2uhe, sovo voci usate dai Caro ne' Mauiaccint; e il Veneziniol
chiama Gv/ane dal Gr. Chelonei, da noi si dicono anche butte seodellaic,
BAST le NO Seré, Celebre, e nouttime scrittore d' archucuura.
EbDIFIZ/0, Preto largamente s' inteuce Ogni sorta di faborica, €
ma preso ttrettamente vuol dir faia, ec, Case, ed altre niuraghe, |
ades, @ facio; ed in questo andiamo uniti co' Latini, che per earfien
no ogni sorta di scrittura. Gio, Villani t, 128. Pauose/f ad ascdin, 00, o
difici, e per cane per forza ebbe, Li lib, del conquiito, Per joraa a

gE Es PSs SEES. =

'> ie P

= Fo



NONO CANTARE.

1. Capiteli, e frontejpizi,, Columnarum capitula, o fronts bespitii, >
(ATT H Srglie.s c. aui, Stipi sono le pietre de i tianchi, e foglie quel-
a parneey quelle dilopra, che tutte insieme formano una por-
a» Suipice dal Latino :pes.. Architrave; quasi trave principa-

: « Quei ruotoli di legno, che servono per facilitare lo strascico de i pei;
atini li dissero Palange, Vedi sopra C. 2, st. 65. Dichiamo: mertere une fal exr-
Spiguerlo a poco.a poco, e condurlo doicemente a fare alcuna cosa, La
Voce viene probabilmente dal Latino baiudare; questo aggiuttar' un corpo
}a un' altro in maniera, che quello lo porti con sicurezza. E la seconda
| Latino xmbdicus, cioè punto ne) mezzo, E ilicare quali ponere in umbitica,
ARGANO. Strumento, che servc pct tirar fu pesi in alto, che da huomini è
" moflo in giro per via di leve. Alcuni Latini lo dicono Sucu/e, i Greci oniffi, cioè
 Afineli:, e questo \& V argano,secondo il Filandro, cum axe iacente, quello pui cum
axe ereite, dice che in Latino e Ergeta, cioè macchina da lavoro; donde, o da
voce(lecondo il Baido sopra Vitruvio)è fatta la noltra Argano,
MSADATT 1. Scommodi; Non atti a esser portati, o Arascicati.
MC ATI, Meili in bilico-,-0. equilibrio., Latino Jibratis.. Diciamo.bilico
ofitura d' un corpo sopra ad un' altro in maniera, che posando quasi in un
non penda, o aggravi più da un lato, che dall' alo. L nostri Scarpellini
 dicono baggiclare per biluare. i
it. BOTT O porto, Si dice. Ch' è cb' € 5 colpo colpe sec. e s'intende Spefiime volte

PAR* un venga, Tirar roba da alto a batio sopra auno, che sia foo.

“a ay STANZALXVI. STANZA XVIIL
a Le Donne anch? esse corron co' figlinoli, Chi, perché già non piglin l imbeccata
f i 2 dy che troxan, gettan dalle muray Cuopre i capi con tegoli, e mattoni,
o con la conca, o vaso da vinolt Chi verssa giù bollente la rannata,

a 9» Pighia a qualcun del capo la. misura; Che pela i vifie porta via i bordoni,
a8 Profuma il piscio i panni, ei ferraixoli Nei? olto un'altra intigne la granata,
yet Ne guardan vc v'é penail far bruttura, E fal asperges sopra i morioni,
ps Chi tira gi: unjastrone alic cerned, Altre buttan le casse,accio i soldaté
ie Che se ewe orili serva per murella, Partir si debban, poiché son cafjati,

ie ooNarraiil Poeta la difesa, che facevano queidi Malmantile, e descrive diverle
we" Operazioni militari adeguate alla composizione burie(ca di cutta. opera.
CONCA, Valo grande fatto di terra cotta, entro al quale si fanno i bucati
Ke ASO da vivoli, Sono vatetti di terra cotta simili alle conche, ma piccoli, en.
| 80a! quali Gpongono vivoli, cd altre pianterelle d' erbe, o fiori. Dice che.con
v — gucfi pigliano la misura a.ijcapi, perché hanno il vacuo capace della tetta d? unt
Td huomo; al quale quando i Cappellat voglion pigliare la misura della testa, metto-
u ~—-'NO in capo un tappello; € cestaco di Malmanzile per pigliar tal milura, in vece
sso un cappello., mettevano-un valoda vivoli: e cosìscherzando intende, che ti-
@ — ravano (ule teste a i soldati di Baldone i deni vali.;
@ \SEvi dipenail far brurture', Se\vi e pena il fare sporcizie; Dice che tirano fino
Dorina, e non guardano,-se. ciò sia proibito,: e con questo dire, accenna il co-
ef flume, che è in Firenze a” affiggere alle muraglic dove non si vuole, che fien fat.
r te



432 MALMANTILE

te sporcizie, certe tavolette di pietra, nelle quali \& scritto il
flrato degli Otto, che proibisce, e mette la pena a'chi fa
niuno si posia pretendere ignoranza; Ed intende anche di
¢ grave pena, che è in Firenze a buttare dalle finestre nel
torno a' quali dispone anche la ragion comune, come si vede
De his, qui deiecerine, vel effuderint, ' '
SE v' ¢grilli, Sopra nel C, 6, st. 22. dicemmo, che grille si cl
cosa palla, che si tira per segno, giucando alle palloctole; ed all
strelle, qual giuoco dicemmo come ti facia sopra ia detto C.6.t,
rché tirandosi, or qua, or la alla ventura, o alla volontà
a il falto del grillo, che dopo un breve falteilare si ferma, e-poi
-dicesi ancora Lecco, quali i/ex eMurelle chiamansi anco
nelle sue Rime. orate
Ch' io do sempre nel lecco alle murelle OP R
dal Toscano antico eora, che è lo stesso, che il Latino Moles }ép
si dice di pictre. A'awer la resta piena di griili s' intende uno, che ha capric
vaganti; ¢d il Poeta scherzando con questo equivoco di' grille dice
quelle lastre a' grilli, che sono neile tette di coloro, come se piocatietd
strelle, o murelle. Dal pazzo similmente, e curioso faito del grillo son detti
icapricci, e fantasie stravaganu, che faltano in capo, e per così dire
PIGLIAR' un' imbeccata, Infreddare: B diciamo ancora: Pigliare df mitt
caffrone, perché il beccd, ed il castrone hanno una tal raucedine, che
pre, che cofiano, appwato come fanno gl infreddati.
Té£GOL/, Pezzi di terra cotta adattati a coprire i tetti delle case.

ap

HlAe.
: RANNAT 4, Liscia forte; che è quell acqua bollita con cenere; ¢
dalla conca, quando si fanno i bucati. Lacino /ixinium,
BORLON/, Inteudiamo quelle penne, che non del cutto spun
scorgono dentro alla pelie degli uccelli, e per similitudine intendiamo il)
spunta nella faccia degli huomini « way
FAI alperges con la granata, Diciam far ? asperges quando con spugha
tra cosa si (pruzza acqua, o altro liquore,.a minute stilie; la qual cosa il
chiama e4/pergere, qui dice, che spruzzavan' Olio con le pranate;
aiciato un mazzo di scope, o d' altro simile adattato per (pazzare,)

stanze.

SOLDAT! caffati. S' intendono quelli, che sono stati pri
la milizia, perché cafare vuol dire cancedare: Ed il Poeta
guivoco di <afaté, cioè percotli dalle cafie', dice, che se son
nou dal Campo, perché non son più nel numero de” float,

SLANZA XIX,

Vi? altro con un gatto vwol la berta, Ed il primo ch' et trova
Legato il cala,ond' es fra quei.d'Vgnano - Che dou'ticbiappar
Sguawnialugna, econ la bocca aperta

Griaa ina/prio in sue parlar Soriano

oF

ee ee ee ee eT oe eS Ol ee

=~ aw



NONO CANTARE: 433

arnt) re Bie XX, e
Miagola, e soffia it gatto, es' arronciglia,
Ed Gite endian heeree
janes quel che oa " trattopigla
Beli è miracol poi se pite gli scappa;
thie oat peter tee cos riglia,
jie Lo tira fu con qualche bella cappa,
a «Ci qustcheciarpayo qualche pinacchiera,
ye  Ecosi gli riesce di far fiera,
ame cool (STANZA XXL
due Quand una volta lasciale calare
ib oi iaers al buffo di Grazian Molletto,
Che fu;di posta per ispiritare,
«Quel pelliccion vedendo intorno al petto,
we Le bestia intanto falta, e dal coliare
'hoe "

=

Tutto prima gli straccia un bel gigisetto,
fet  Dipos si lancia, e al capo se gli ferra,
ebst  sì che il cappello gli mando per terra,

STANZA XXIL
Non.sa Grarian, che Diauol si sia quello:
Pur tanto fac' al fine ei se ne sbriga,
Ea aiza il vif per farne un maceilo,
¢Ha vedendo il rigiro, e ch'ei s' intriga
Con dame, vuol canarsi di cappello;
Ma perch' il micio gis ha tolto La briga,
La Dama accsuetrata, anzi civetta
Lo burla, che gli è corsa la berretta,
STANZA XXIILL
Ed ei, che da colei punger si sente,
Onde al naso lo stronzolo gli fale,
Perde il rispetto, e quiui si rifente
Con dirgli, Atona merda, e ogni male,
Vain questo al aria ungraromar digeete,
Che 4 terra scende a masse dalle scale
Fiaccate,erotte ach'elfe dagli /prazrolt
'Di pierre, c' ancor grattana § cocuzzoli,

oa Continova il Poeta a narrare gli accidenti, che (eguono nell' aflalto di Mal-
ie mantile, e dopo haver detcritto una Donna, la quale con un gatto legato a un
" i miazzacavallo andava levando rcba da dosso a questo, e a quello, come segue a
ol Graziano Molletto ( che è il sig.\ Conte Lorenzo Magalotti ceicbre per aobilta,
HF 'e dottrina ) dice che le scale degli AGalitori furon rotte dagli Allediati: e con i
r sassi, e con altro, che tiraco di sopra alle mura, dava ancora addosso a i soldati.
at IL (a berta, Vuol la burla ( vedi sopra C, 4. st. 47..) onde shertare, lo stef-
4 fo, che beffare. [i Davanzati ped dite Swerrare nella (un traduzione di Tacito.
mY Corte poesie senza antore, che fuertavano le sue crudeid. Se bene in questo luogo si
; poirebbe intender per berta quello strumento, che serve per ficcare i pali ne i
ea pfiumi nel far le fleccaie, che è un gran ceppo di legno ferrato, il quale infilato in
“ln pernio, o ago di ferro confitto sopr' alia testa d' un palo, s'alza per via di fu-

ni, e si lascia ca(care sopr' alla testa del detto palo ( già fitto in terra) per fario

sf andar pita drento. E perché in questa medesima guisa faceva Colci coi gatto,in-

yo teade, che desse così /a berra, eruendosi del mazzacavaljo, che appretio gli an-
ti"  tichi era usato per arnese militare, come s'è toccato sopra C.6. st. 86. In propo-
i)" fito di Berta per Bxrla, il Ferrari dice così: ognuno poi la creda, come gli pare
4 f verisimile, Dopo aver detto, che que' delio stato di Milano chiamano Berta
8 ta Gazzera, e ciò dal balbettare,ch' ella fa; foggiugne; (aoniam autem fanne,
gil! At que irrifionis [pecies est aliena verba imitando reperere,inde Berta pro Inda,ae derisione
gi accipitur, o fare una berta illudere, \& decipere. O pure finalmente e forte più
credibile, che venga questa maniera di dire dalla novella raccontata sopra nelle
Annotazioni alla St, 47. del quarto Cantare,
d\& —. SGVAINA I agna, Cava fuori  ugna, che tiene alcoste dentro alla pelle, la
we bed gli serve per guaina, ed il Poeta scherza, dicendo /guaina U' ugna. lock que
gnano

Wo + Appropriando benissimo uns, a Vgnano.
yd 4NASPRITO. Incollorito, meflo in ira, in stizza, in rabbla. Latino exa/~
lii IN

peratis,

i



54

IN parlar Soriano, Cioè oer gatti in ling
si dice quello, che ha la pelle di color lionato serpato d
ché si dia in altri animali, o in panai, non si dice foriana; se
perché i gatti di tal colore fien venuti di Soria, come ai
di Persia quelli di color di topo portati da Pietro della Valle,
chiamati Persiani, o per Persianini. 'one

DISERT A, Cioè ttroppia; concia male. Guasta.

VVOL ievarne il brano, Brano dal Latino barbaro mn.
il pezzo, Vedi sopra C. 6. st. 47.

MIAGVLARE, o ignaulare. Bi ii

I gridar de i gatti; + il soffiare dic
quello strepito, che fanno aprendo la gola, quando fond in rabb

S' ARRONCIGLIA., si torce in sse stesso, come fa la serpe quan
viene da ronca, roncola, ronciglia; specie d' arme; o più” 2
agricoltori, ed è fata come una spada, ma è torta in cima a guisa d
serve per eflirpare i pruni: o pure da Ronciglio, usato  da' Dante per graf
fauto a uso d' uncino. i "

E MIRACOL 8 egli scappa, E cosa soprannaturale, o imposiibile,
degli artigli, £1 Petrarca. soe eee
E cio, ch' in me non era

Mi pareua un miracolo in altrui
cioè una cosa, che non potesse stare. '

LO tiene in brigla, Cioè 10 maneggia bene, facendolo operat

CIARPA, Dal Franzese e/charpe, banda, bandiera.. Quel draj
tano i soldati cinto:de' soldati era proprio il cintolo, onde cinguoie fol
dalla milizia, Vedi sopra C. 5. st. 33. 5) ie 7

FAR fiera, Buscar, o acquistar roba = per esempio ends pirando per
torni, e chi gli dette pane, cht voua, chi una cosa, e chi un' altra tanto,
Satta un poco di fiera, se ne tornd, mn.

DI posta, Subito: Di primo tempo. Vedi sopra C. 7. st. 92. BY
giuoco di palla, che si dice dar ai posta quando si da alla palla, prima
terra, ed è il Latino ilico, e vefigio, Gli antichi dillero: Di colpo,
fo, che di Borto. 7

FV per spiritare. Hebbe un grandissimo spavento, o paura.

GIGLIETTO. Specie di trina con punte; così detta, perch ha
col giglio.

Avr. Cioè gnell' ordigno, col quale la donna alza, ed ab
Vedi sopra C. 4. st. 69. Se bene \& può ree la voce rigire nel
mo sopra C. 7, st. 41,, ed intender, che Graziano, alzando il ca
giro, cioè la donna, e dedurre questa opinione da quel, che soggiung
Vedendo, che s' intriga con Dame,,

ACCWETT AT A, Afiuta; Sagace. Tolto dagli uccelletti,
civertats, quando havendo altre volte veduta la civetta sono dit
non si la(ciano lusingare a volarle attorno, come fanno quelli
mai più veduta. ae

eANZL cinetta. Più toto troppo ardita, e sfacciata. Si dice'

, eel ee8 TR

e— lL eBeuwe ete

— 7 ate

= ~ - — =



x A juomo da poco, però con tale equivoco

NONO CANTARE: 435

vane troppo ardita nel trattar con gli huomini, quasi faccia con essi, come la
corerasss gi uccelletti, che cerca con gli suoi gesti di tirargli a se. Vedi for-
to in questo C, st. 60, E Plin, \libcap[10]{17}.;
CHE gli e corsa la berresta; Che il gatto.ha fatto preda, e gli ha portaro yia il
ppello.. Ma perché, La/ciarficorrer, pee via la berretta, vuol dice Elicre
mentando G h diveherch iandnban tone
raziano womo da poco dal veder, che si lascia rubare, € portar
via il cappello, gli an burla; di che egli s' adira, perché si sente fete
if r¢ dall' etiere burlato da que(ta donna,
-, GLI fale lo frronzuso ai naso. Derro sporco, che significa entrare in collera, ma
= poco usato, dicendosi più tolto fair la muffa, o la fenapa, o la mostarda, o it
herimo, ec. Vedi sopra C. x, st. 39. E il Lalli En, Trau, C, 2. st, 65,
Waapn 6 Airs Corebo un tale strazio, e tanto,
i Con la mostarda al naso, e nol comporta,
AGli Ebrgi.colla fiessa voce significano, ¢'/ na/o, e ira, perciocché par, che qui-
¥iclla particolarmente rifegga, siccome disse Teocrito + acris bits ad nafum fedet,
Onde noi dichiamo Arric¢iare il naso per isdegnarsi; simile in parte quel che dicevano
gli antichi Leware il miffo. La voce Ebrei fie Aph, in Siriaco Apha; onde
. itorcec: e venuta la nostra 4fa, colla quale a ete una cosa fomi-
gliaptitima alle vampe dell' ira; cioè un vapore, e yn caldo fallidioso, e affan-

HO!»
t 'sop SLrifenre. S'adira: Entra in collera, perché e burlato,
pjat 'A merda, Detio ingiurioso usato fra le donne di vil condizione, e del-
Ta voce mona vedi sopra C, 5. tt, 18.1 Lagini similmente (asum, conum, frerquili-

me,

. FLACCATE. Spezate, Fiaccare \& verbo proprio per esprimer, quando un le-
£00, o altro. materiale si rompe in mezzo per fouerchio pelo, Latino fari/cere.,
 springs. Donde poi bxeme fiacco vuol dir huomo affaticato, e stracco; se bene \&

ver) imile, che venga dal Latino faces, faccidus, dichiamo, fiaccare |e braccia

A uno, clive infragnerglicle, e romperglicle colle bastonate.

SPRVZZOLARE. Vedi sopra C, 7. st. 15. E qui è detto ironico, ed intende

f Bingge pict 7 '
V2ZZOLO. Latino vertex, cacumen. La parte di sopra del capo dissesi an-
she. Zwecolo, siccome da Cocuzza de' Napoletani ( Latino cacarbita ) e si dice an-
Gora. comiznole, se bene questo e proprio delle fommita de' tetti, e de' camumini;
dal Latino cudmen quali culminnlum «
ares ST NZA XXIV, STANZA XXV.
Chi con, chi per banda,.¢ chi supino Quantunque il.campo annaffi tal rugiada,
i se ne viene, e fa certe cascate, Come le zucche, annarpican le scale,
Che manco ie farehbe un' Arlecchino, Onde più a' xno in già versala strada
, Quand in commedia fa le sue scalare; Fa pur di nnono un bel [alto mortaic;
sì che, stinnanzi fecero il fantino, M44, piché ammonti ne traboechije cada,
Le brache in fasti glieran pui cascate, Sardonello [2a forte, e in alto fale,
> B infranes, e pesti andando gis nel foffo E trai mimici al fine a lor mal grado

Mette [u il piede, e agli altri, e Uguado
2 PAN.

, Hanatolere a quchlo nuove scate adddfe,,
geet ii



436 MALMANTILE™
STANZA XXVI.
Chi vidde in un pollaio, ove fisrona | *
Un numero di polli senza fine
Tra lor cascar qualche pokafira nuona,'
Che roft addoss' elt ha gullie galline
Ciascun per far di lei l'ultima prona 5
Eye aa solelapariea athee, 1
Che la difende, e da beccar (e porta Ma Eravan, che
Stroppiata rimarrebbe, e forse morta, Aiuto a un cempo,ed
Rotte le scale coloro, che erano sopra di esse cascarono nel fofla
r0 corpi furon polate nuove scale, in fa le quali intrepidamente
neilo falto sul muro, e scel nella Terra, dove fu da mojti di quei
falito: Ma Eravano, che lo vedde in pericolo d' esser ammazzato
¢gli dentro a dargli aiuto.

BOCCONIL. Dittefo in terra, o altrove con la pancia, e faccia ve
no, Lat. pronus contrario di Sapino, fusse reni; Lat. supinus, e Per,
la doppia posicura che resta, diversa dall' una, e dal' altra, la diciamo 4 x
Per franco, e Per latv, Lat. in latus. Bocconi  detto colla stessa forma, che!
nocchioni, Brancoloni, Saltelloni, e simile, che si -dicono anche Boccone
vhione, ec, anzi questa ultima maniera è l' usata dagli Autori antichi Ti

eARLECCHINO. Va secondo Zanni, cioè un servo.semplice in
Così nominato, il quale faceva assai bene le scalate, che son quei giuoc
Ai suol fare detto Zanni in commedia con una scala a pivoli, sopra alla
affaticandosi di voler falire, casca in diverse manicre. f

FECERXO il fantine, Pecero il bravo, l' ardito, il coraggioso, Si
gura. Egli e fantino cioè persona, da fare questo, e altro, Fantino di
faate. Lat, infans, cioè Ragazzino usato dagli antichi in generale, @
oggi a un significato particolare. Chiamando noi fantini quei R i, ¢
pr' a cavalli spogliati corrono al palio, Si dice anche fare if Baiardina, da
lardo celebre Cavallo di Rinaldo Paladino, così detto dal suo mantello,
yea essere Baio accefa.:

GLI eran cascate le brache. Gli era entrata la paura addosso
animo. Vedi sopra \cstan[6]{20}, Lat. aninsum desponderant,

ANNAF SI tal rugiada, Annaffiare vuol dire Ammollare, o af
giada vuol dire quel che accennammo sopra \cstan[2]{55}. alla voce gr
Ma qui da nome di ragiada a quelle pietre ec, che buttavan già gli

-dnnafiare detto da Adacqware, che si dice anche /anacquare, e Annacquare,
Ui duc ultimi verbi diconfi propriamente del remperare coll acqua il vino;
equare propriamente e dare [ acqua alle piante. + la
INARPICARE, Aggrapparsi, forse dal Gr. herpein chet in
Pere, reptare, Salire in alco, appiccandosi con le mani, € co' piedi,
no i gatti. Si dice anche rampicare sopra C. 4, lan. 68. ed-«
vedremo nella seguente ottava 28,:
SALTO mortale. Chiamano i Giocolatori falto mortale,quando
tecra Con le Mani', o con alcro faltano, voltandy la persona fo}

> eran

WEITAF.

wee Se peg RRFESZLTE=



NONO CANTARE, 437

verisimilmente facevano coloro, che ca(cavano, o erono gittati da alto 'a batfo.
) TRABOCCARE, Intende precipitare, o cascare da alto a baflo, rompersi
la bocca; andar colla bocca per terra. E se bene il proprio significato di trabuc-
“care è quando mettendosi in un vaso maggior quantica di liquore, o d' altro, di
PS yche possa capire', casca dalla bocca del vaso quel, che vi e di più; onde per
figura si dice un Trabocco di sangue, ec, tuttavia si piglia ancora in senso di calca-
te. Traboceo ne i vizzi, ec).
hie = ROMPE il guado. Apre la strada, o il passo. Ovid. de arte amandi,comandando
'ex che si rompa il guado per via di viglietto, dice: Cera vadum tenter, Guado vuol
s dir quel luogo ne i fiumi,per dove si può passare senza navilio, che si dice guada-
ve; Eda questo guadare, o rompere il guado s' intende aprirsi il passo in qual.
“Voglia occasione, o congiuntura.. Parrebbe che fletie meglio vado dal Latino
mis » siccome si dice ancora volgarmente il porto di Yada, dal Lat. Wada Vuo-
 taterrana; perch così si fuggi V equivoco di guado (pecie di tintara, ma
ivell quelli stitichi, i quali si vergognano, che la nostra lingua sia aiutata dalla sua+
frit madre Latina,non ci concorrerebbono, e darebbono una turbativa a chil' usaiic.
hist = MANDAK 4 Purafso. Par morire; E perché significa il medesimo che man-
aoe » o 4 Scio credo che derivi da i foccorsi maadati in diverse occasioni,
| “tempi ai detti tre Juoghi, da i quali non essendo tornato veruno di quelli, che
al —andarono, quando si vedeva mancare uno in paefe, si cominciasse a dire. Eel
stl e andato a Buda, a Scio,0 4 Patrasso; per intendere egli € andato in luogo, don-
de non tornera mai più, duc, unde negat redire quemquam; e s' intende egli è
i = Morto. Vedi sopra C. 5. itan. 13.
j TIRAR l'ainxolo. Vuol dir morire, dalle cunvulfioni della persona, che pa-
§\&  tilcono quei, che si muoiono, Aixslo è (pecie di rete da pigliare uccelli. E la for-
2a, che fa ' uccellatore nel tirare l' aiuoio, o simil sorta di rete, e deferitta da
id Petro de Angelis da Barga in que' versi +
0! Tum vero innitens pedibus confurgit,\& omne
Intendens neruos magno trabit impete funem.
4 ZO scorge debito. 1.0 vede in pericolo di morte. '
STANZA XXVIII. STANZA XXIX,
1 ' Chinmgue è 'n Castelle allor pien di paura - Auitiene a lor ne pri, ne meno un' iota
% Corré per far o auanti et prit non vada, Com' ai fancinlli, quando per la via,
Fan la tura at rigagnol con la mota,

«RB memrtil vuol rispinger dalle mura,
' “\ Ch altri più la 2 arrampica non bada; El! acqua ne comincia a portar via,
| | itr db ouniare anco di gua proccura Che,mentr' affodan quixi ov'ellaé vota,
| Main fete lui ghit ged farts la frrada, Essa distende altrone la corsia,
, E se riparan la, prt qua fracassa,

| E a cogs intorno tanto il popol cresce,
C" ogni riparo innalido riesce. Tal ch' ella rompe, e a lor dispetto pa/sa,

«\ [Soldati di Baldoné superate tutte le difficuita, finalmente entrarono in Mal-
o mantile, ed il Poeta paragonando questa cacrata ad un' acqua corrente, che rom-
 pe, € passa ogni ostacolo, che le 4 pari avanti, esprime I" inutil difela, che fan-
“no i Terrazzani.
ARRAMPLARE. E' lo stesso che inarpicare detto poco sopra, ed è il Latino
Perreptare.
VN

™"

a


438  MiALLIMIAN TLE Bot

VN ita, Un niente, detto sopra \cstan[1]{18}.)

RIG AG NOLO. Diminutivé di ome, Piccolo riva,
è proprio per intendere da parte più: bassa, che è nel 0
di Firenze per dove scotre l'acqua's che piove, efic 7
intende nel presente luogo 4 € ¢' aacenide comuacmente s che un
rigo, o rio diremmo rixolo o ra/celloy dewro così da Riuiceday la
presso alcuno antico. Se bene Dante nell' Inf. C, 1g.dices Ed
Sente rigagno, ec. ed intende quel fiatnieell@, o rivos il, 0
nali. Li Varchi Stor. Fior, libro 13. Commiciarono ad nscar fuara
e che i rigagnoli correuano, ele ve erat piene di motayedifarge rt
Nov. 16. 4 rigagnolo della qual via corre, chepare un fiumicclian |

MOT A, 'Lerra ben inzuppata acl? acqua. Ai Percariz, Lupum
 immora, Per intelligenza della \iuddetta comparazione e ince
i ragazai dell' 1afima piebe di ae 'sogliono per loro pa
dopo la pioggia (corre l'acqua per detti rigagnoh pigliate del
ond ae come un Danian opposte ai corso dell' non
passaggio al fume, e questa chiamano la twra.; ma fiocome d'
quel iuogo sempre va crescendo,)così 0. per 10 pelo, rompe
bondanza traboccando la superay e pada via noa oltaace dri
v' appiichine, come dice il Poeta. Qunero nell' Aliads ib, a 5s,

De! Troiani fereci allagranturbay...
dt folgorante eApollo andanainnanze
Tenendo in mano il preziuso fondo:

Ei degls Achini il muro aterra Sefe;
Ne coffogli fatica, appunto.come
Lungo il mare il fanciulfacoll arena,
Che poich¢ fabbricato ha per. suo gsoco
Va gentil fanciullecsco alto lanoro;
Colle mani, e co' pie scherzando il guasta,

A lor dijpetto,, Contro alor voglia. Lat. ijs innitis, Il Boce, disse
Per di(peto. A Dante prima, e poi al Petrarca in uecedlica della rl
il servirli della parola De/picto accordandosi in ciò, siccome ima
col dialetto. Provenzale, o Francelco. Virg, ecl. 2. Despectus tibi Jum ne
queris, Tu m' hai in dispetto,ne ti cale il sapere,chi io mi sia, Confiache
la strada, che è.per il mezzo della galera; onde que) groilo Canaone.
diceli Cannone di corsia, S' intende ancora per la correate dell' acqua..

FeSlFaer- ae

eet

it Se OR Peas aw

STANZA Xxx, opqas ae a
Già tutti son di sopr' alla muraglia, Celidora a due man 4
Che la circonda un lunge terrapieno; Che ne-anche un vi
Già si fiorisce in si crudel barcaglia Tanti fil d'erba gol

Di sanguinacci la gran madre il seno: Lane' buomini così


4
NONO CANTARE:  439.
o- - STAN ZAXXEL oo. STANZA XXX.
ee, jth Amiffame —. — Adafa di Coccio a questo, e quel comand,
Da toccatori fan col brandispocco, Ed all'un dane, e aun'altronepromerte,
d h Lacompagnia del Furbainnanci mada,

“Pere che della morte almen Ceffane,

'Se non prigion si fa chi è da lor tocce, Che resti ai fianchia Batiston commette
AIP incontro ritrovafi Sperante; Com Pippoyil quale (Pa dal' altrabanda,
WA) + Che fa menando (a sua pata, il fiocco, Ma egli imretreguardia poi si mette,
Wh E se già le fustanze ha difipace, E mentr'ognun favanza agloriasmtente
a Ei fiede a gambe larghe, e si fa vento.

Hor mand'a male gli buomini a palate,
+ Essendo già wtci i Soldati di Baldone faliti sopr' alla muraglia, e padati oclla
PS di dentro si mettono alla difesa, Sinarra la bravura di Celidora, di
y edi Amostante, s' accenna 1l valor di-Sperante, !a diligenza di Mafo
S eraccc pane wragtoe ut Coot cies A
La gran madre si se i sanguinacci 11 seno » Ci terra s'asperge di sangue:
88 Ounero nell” Lliade (petisind « =:
pm 8 di sangue la terra intrifa corre.
® La Gran madre per la Terra intese if Petrarca nel Trionfo della Morte.
elf SEG ORY O ciechs, if tanto affaticar che giova?
jeu 08 Tutti tornate alla gran madre antica 5
pone E'L nome vostro appena si ritrova.
  TOCCATOR?, Vedi sopra C. 2. tan. 60.¢ \cstan[6]{44}. 3
 “ BRANDISTOCCO.. Specie a' armein asta; simile alla picca, ma l'asta più
corta, ed i ferro più' largo, ¢'pily lungo, che non è quel della picca; e credo
venga dal Tedesco froch, che vuol dir battone, € brando che da' Pocti Eroici mo-
derat si prende per Iipada, e significhi Spada in sul bastone.. Stocco e dal Greco
Felechot Lat. Pipes, candex, da cui è facta anche la voce feecco,  perciocché pri-
ma per battersi si adoprarono le-mazze, e poi si venne a ferri;( Orazio Serm.
1.1. Sat, 3. Vaguibus o pugnis dem fustibus, atque ita porro Pugnabant armis » que
'pelt fabricaverat nfus i nomi potleduti già dall'arme di legno, furono ereditati
'dalle arme di ferro, che a quelle succederono. Onde Stocco, che in Germanico è
baitone, a nOi significa /pada corta, e stoccata la ferita, che si da con quella. Brand
* jn Saflonico e riz one, o fuoco; onde Brandispoccbi poterouo essere cio che Virgi-
“tio lib. 7.¢ 11. chiaina /fipires, o /udes pranffas, ovvero obustas cioè bastoni, 0

mazze appuntate col fuoco. 3
' CESSANTE. Si dice quel debitore, che essendo stato toccato da i toccatori

“può esser fatto prigione dopo le 24. hore da che è lato toccato, ( del quale ato
me rt e. (a 60. e \cstan[6]{44}.) ed il Poeta scherzando coll'

'Paclammo sopra o.
egnivoco toccare, cide esser percosso; dice che quello, che da costoro è tocco di-
viene almeno Cefante della moree, se non prigione, ed intende che quello, che da
costoro è ferito o muore; o resta vicino al morire, com” è proto ad andar in

Prigione colui che è tocco « ' <

FAR il focco, Fioccare vuol dir quando nevica gagliardamente, € da questo
diciamo fare il fioceo per esprimere un' abbondanza di che che Ga, per esempio si
fa ii fioece delli uccelli, o de' pesci, o de' denari, ec. si direbbe a uno, che pigliaf.

se molti uccellt, molei pesci, o molti danari, ¢c. \& così nel preteate luogo inten-
de

%

a



440 MALMANTILBE

de che Sperante ammazzasse molti huomini con
il vello della lana Lat. foccus., Si trae anche come's' ¢detto
ve, che Marziale appella tacitarum vellera aquarum, La
in abbondanza, si dice Fioceare; e stendesi anche r
aver dewo di Mcnelao: Poco dicea, ma bene, viene a dire d'
Atandaua fuor diluvi di parole, '
Come allor che di verno ilnembo fiocca,.
E fu pe' monti nena a! ogn' intarnos dohlgioee
MANDAR male a palate. Vuol dire mandar male il fay
gamente ed inconsideratamente. E qui ii Rocta ia Spe
vendo havuto per costume di mandar male ii tuo a 0
l'antica ulanza di mandar male a palate ancora gli huomini, e d
con quella sua pala, concia male moltihuomini, '
A chine dd, ¢achi ne promette. Diciamo così d'uno insolente
che tutto il giorno facia risse, perquotendo quand' uno .< quand"!
con questo dettato il Poeta deferive la,natura. di Malo di Coccio, il
s'è detto sopra al suo luogo ) era huomo di conversazione, e nelle tel
ordi, ne 1 quali si trovava, foieva vOler (empre soprastare gli aluri
¢ ca Cf farsi ubbidire con le grida, e tainolta con ie butie,
 gambe larghe. S' esprime con questo termine la commeaita, e (pe
ginc,con la quale uno ficce a pigliarh riposo; (e si dimotira un pimuo ¢
sare, ed amico dell' ozio, e delja pigrizia) che si dice: Stare iw Rane
\cstan[3]{72}, € \cstan[3]{1}, 60m s¢ mani in mano; Con ie mani in cintola, —
STANZA -XXXiLL STANZA AARIV,
Amostante alt incontro un nuoko eAarte.  Vedendo i Terrazzanigbe stannoin fa
Senbra fra tutti anants alia testata, Che il nimico ad S[pade, e gioca
Lo segue Pao C orbi da una parte s Ler non far Mole 1H fab MALCON KG i
E aa quest altra Egeno alta franceta, Ritsransi, e non sengon più
Vengonsi in tanto a mescolar le carte Ma speron ben ( moftcanaoa
E vien /pade,ebaston per ogni armata, Denari, e coppe)indurghs a far p
Ectidam puche, e 4 gsmocar none leflo a) si
Vs perde la figkra, e fa acl r¢sto. Speaiscon, che pario in
eile preicnu due otrave il Poeta dopo haver lodato per vaiorolo
seguicato dai Corbi, e da Egeno, icherza in sull' equivaco del giuoco, \&
sucne rai as/corso dai proverbio « Vengonsi a mescolar le carte,( che
€ \¢ ne Locca, O se ne 1iceve, Come vedremo sotto C. 10, Ble
auibedue 1 campi vanno ( cioè s' adoprano ) /pade, e ha/tom, e che chi
che ( ive urta nelle picche ) perde /a figura (che € una di quelle carte, nell
Ji sono efhgiaui ques fantocci, che ne 1 giuochi di daia tono te carte,
cive perde la propria perlona, e fa del resto ( cioè muore ). £ Terr
in fors, CIU¢ hanno i lor punto in fiori, ( ed intende tanao ip
Bria ) vedende che 41 nimico ad /pade ( cioè adopra ic ipade). Per non,
+ maitom: ( cloe per non fare un monte di mori in iu 4 mattoni, e ¥
fui terreno.) ff rtir ano da chore ( cide lasciano J' ardire,) me tengon
Vuoi dike HU VoOguon più giuocare, ed intends non voglion più

gs gp emer Ee ae PP ee EsP soo eee eee FEE



« NONO CANTARE;

ano di ridurgli a far partite, cioè accordarli, mostrandogli

44t
i ddwari', e coppe, cine

 offerendo loro dell oro: E pee questo mandano al Campo un' Ambasciadore,

che parld nella maniera che se
 STANZA XxXxy.

 Spida Signori ? armi ognun sospenda,

Ache far questa guerra aspra, e mortalel
Fermi per grazia; più non si contenda,
Per c! alsrimenti vi farete male.
Fate che la cagion aimen s' intenda,

| Ca cherichedi a questo mo.non vale

F

ni

it

ee

| Bchi pretende venga con le buone,
Che dara glifard soddisfarione,

'ntiremo nelle seguenti ottave.

“STANZA XXXVI.

Con queiyche dona per amor non s' nf4
4n tal modo ta forza, e la rapina,
Chiedere,imperciocch? giammai ricufa
Ui ginfto, ed st douer la mia Regina,
No entraron mai moschein bocca chinfa,
E con chi tace qua non s indonina ?
Poss' egli accomodarla con danayi?
Dunque parlace, e vengafi ai ripari,

 L Ambaiciadore de 1 Terrazani espone la sua ambasciata, e chiedendo tregua,
-elolpenfione d' armi conchiyde che la Regina di Malmanule e pronta a dar loro
fodistazione, pero domandino, che faranno esauditi.
|. SPiDA, Questa è una parola usata da j ragazzi ne i loro giuochi fanciul-
ye non hay ( ch'io sappia ) significato nessuno universalmente, ma nel
modo, che se ne servono i ragazz: signitica sospenfione di giuoco, o permuffione
@ eleacarsi per alquanto da efio senza pregiudizio, appunto come si fa con la fo-
spenfione d' armi in occasione di distide o particolari, o generali, ond' io crede-
rei che si potesse dire, che questa voce /pida fusse corrotta da ssida, o disfida, I
- Fagazai si servono di queita voce così, per esempio. Wel giuoco de' birri, e ladri
detto sopra C, 2. staa. 32, quand' uno occa bomba o per qualche sua faccenda
on attenente al giuoco, vuol partire,per assicurarsi dal' esser catturato dice;
Spida, E con queita parola s'intende per lui fatta sospenfione di giuoco: E quan-
do il ragazzo, che è Ggnore del giuoco dice Spida s' intende sospenfione generale.
Ed il Poeta » che si ricorda che egli scrive una Novella per i fanciulli s' accomo-
daa i termini da loro praticati,ed intesi, facendo servirsi a questo Ambasciadore
della voce Spida per farsi intendere che vorrebbe sospenfion d? armi.

Cae hericheli + Chetamente; occultamente, senza parlare. Varchi St. Fior.
lib, 15. Per Ze case si facenano delle ragunate a chetichelli.

WON vale. Questo pure e termine fanciullesco, se ben talvolta usato anche
dagli huomini d' eta, e significa Non è dovere, Non conuiene, Non sta bene,
ec. Preso per avvenitura dal giuoco, in cui chi scommette dice per esempio; Va-
dedi tanto ? E quegli che non accetta dice: Non vale, cioè non fo buona questa
erate 90 pure quando si fa contra le leggi del giuoco, si dice similmente

NON entraron mai mosche in bocea chiufa, Chi non chiede, non conseguisce;
chi non parla non è inteso. Lo Stefonio nella sua Gnoccheide atto primo sce-
a prima dice,

hee pak Vulneris alcosti nunquam medicina paratur,

£ viene a fonar lo stesso che con chi tace, qua non ' indoxina, Plauto nel Pseu-
dolo Att. 1, se. x. ove introduce lo [chiavo, che così parla al suo giovane Padro-
Ae Innamorato,

Kkk Si



PrViot ovoy

E poi conchiiile:
ina fuggire i litigi.

dice Così

“STANZA AdXxVIL
A quel sl General,c' ha un pod' ingegno
Kusene il colpo, e in dietro si discosta
Che si fer mina i suoi, aipoi fa segno,
' Pala parola, e manda gente 4 posta,
Ne bado molto a fargli har a segno,
Chela materia si trove disposta;
Crascun a! amie le parci ferte faldo,
pC? ognun cerca fuggire il ranno caido.
STANZA XxxXvVill,
ch della pelle ha punto panto cara y'
ch che von vorrebbe esser nccifo
'empre de feiarre di fuggir procexra
~ BYe mai c entra, ha caro esser ditsfo,
' Ben ch} ei. mostré non, baner pasra
S? in quel Cimento lo guardate in vif
Lisciato lo vedrete d' un belletto
* Composto di giuncate, e di brodetto,

* Ordiaa i) Generale, che si fermi il combatteré, e trova i'Sol
dieatidivai » perché a ognuno piace il vivere; e sia uno coraggiol
mai essere, al cimento poi non haura carestia di timiore. Fermato que'
battere, Chi era ferito s' andò a far medicare e ah

PASS AR parola, \& termine militare, che significa far fapete |
Capitano  per tucco l'esercico con dirlo a uno, che'lo dica a un?
vada seguitands fiaché lo sappia ognuno senza che si faccia
Qi. Gli aatichi Capicani fa

fiziali fubordinati wa
si conteneva l'ordine di cio, ch

'di yoo! ior cl icvar qiaao da ij i

aes ie Hic Sen ae

eee UM AN TY Le Mi

v hominum parsi vem
6 °)\Nees te rogandi, o eileen, e x
Nunc quoniam id fieri non potest
| Me fubiget, ut ve rogitem'; Fjord
Eloquere ut quod ego ne/cio, id tet '
“PVOS S? egis actomviarla con danari. Ci è egli modod
trovVar rant denaro, che aggiuitj questa ca

*''Dunque parlare, Quest' ultimo verlo pat tolto: di oda g
1, ove Teti patia al'iuo Figliuolo addolarato;

Parla; sis Wb habs sit digi by beds, ue

Tener la cofd Wath tua beentt ascofa'y mins aA

eAiciocché tu ae thee a
TA

oe

sa Pee

Dewo uli

1h te ada,

0G ease
Bao i a
ade

SE FLSFFSRPSR oS Staete

A
Sien: Projo brau,
Se mai vengono a
Crediare che elo fan
Perec' a rutei viene il bi
Ech ela palferebban
Se lo potesser far con tor
snithewsate i a quella opiniiie
Di veder Cuanro viner fa
STANZA XXWK
E questi che badauane ax
in Malmanty, 8 accorfe
Che que; none meflier
Pero si contetaron dell”
Gai tagle alcuno impi
Hitri rimette braccia, e ¢
Altri da capo a
Echi si fa uae ed

SEF

pest

a
. =.

ih
nye
aaa

- o = \& wo Z.

f es haat 1803 ae Udsarntare Tessera. Amminiand'



eee -

NONOPQANTA RE

} 88
'Siliodtalico,, eee etn ee ff
pases temmalumeaee €.con ording,.0 de Da by eat re
ea - oa Leib se sittin

ST ROV Oar niaseria,disposta. Veove. prontezza d! ubbidire » perché cialcur

- inclinava a lasciare il combattere. Sante eT ae
\ \ EVGGARE it ranto valde  Buggire i pericoli,o le fatiche, ~
| HA care eferdinifoHa caro che-qualcuno entri di mezzo,.¢ impedi(ea i
tocombatreresiche questo vuoldire diwidere una quistione. Lac, pugnam dir e.
elLilcio Lateadiamo tutte quelle mesture,, con le quali aicune»
sper parce-bellesi lisciang. ta faccia, che diciaino imbelietrarfe: decto [:con-
do aleuaisda wRerlerra. cioè. melmay fango. In Franaefe il (cio dicesi Farò, onde
ciog unbratwace, e 44re ana farda, e una fardaca, il che figuratameate>
- Sluergognare uno.con mato pungente in pubblico, che alccimenti dice si; dur /a
 Ceretata, E. dare una cenciata fudices, ccacta dal costume de' Ragazzi Fiorentini,
che il'di di yuezza Quarefima, quando ( per usare un loro idvotismo ) si (ega la
eal cioè viene ad.ctlere partita per mezzo quella Stagione di penitenga;
Peete ior abuso,ednsolenza batcono el viso alla gente grossolana, o fenipiice
pd al COntado.cenci intinti.nell'tnchiottro, o in altro fudiciuine.. Branco Saccheci
disse Dane ca Fare, e dare nna zapare, per offeadere coa marto. Vedi sopra, a,

ae Pilla 45.0: base wid Ca Te ya
jit «OM Ne ATA, Latte rapprefo,, e (errato in fogli¢ di farfara con giynchi,, e»
Gdecta ginncata, la.quale mescolaca con broderro, che è mincitra, fata d'
Wlovauidette liquide con brodo, o acqua, e agrelto, o fugo.di limone,,. farebbe
un color¢ fra ij, giallo 5, e il bianco, appunto come diventa la faccia di coloro, che
\& i da subito timore, sink, 1
ASN ADIERL, Huomini sanguinarij: Da Mafnada, che vuol dire truppa.
l'di Soldatic: what, militum manus Ma per lo piii intendiamo compaguia di ajiaii-
at poeaid Aieada.
TIRARLA

fuori, Cio.cavar fuori la spada per combattere. Virg. vagina.

VOISN Gx

aetkbEiks =

TREE

RESEE

er aay;.
= SATEICVORE. Eccessiva paura, e spavento. Dicesi solo dal frequente bat-
'eres che si fence dalla parte del cuore in uno, che habbia timore. Se bene af but-
ter del cuore e indizio ancora d' altre pastioni, che futte anno quivi lor, seggio;
“eae gran defio, congiunto colla speranza di vicino conseguimeato del desi.
rato bene, la quale pero dai timore, non è mai io tuto disgiunca,:
sualptelten'arctben 4 Jeegiert, Paciimente lascerebbono (tire dt far quella quittio.
'Re. ln un frammcnto di Storia, Fioreaciaa manoscritca, che dame oa tisfa di -
i ncarvi il principio si legge: 5, Gli difero un monte di villagia, ¢
ond 'ingiurie, ma.il Cattellano, che era di uci Soldat, che avg iano canioin
dt ight doula Cavalleria, se la palso di leggiers, e la Ciaaiogli gracchiare,
sgnattendeva.a star. deatro;.ed a i suoi Suldaci, che lo pregavag» a ulcire, e dare
vs, addosso.al nimico,, rispondeva; Lo noa vogity ultirs, percaé nog voglio cae
Se CEDER guurs [a vivere un poltrone. Con questo termine descriviamo 490, che
yuo brighe, ac faciche, o.pensiert, a¢ meno f yuule esporce.2 rifthi, o.pe~
SS Ree:, e oe

MSS ERE E

ae
ar eet

444 MALMANTILE

ricoli di sorta alcuna.. Il Ferrario seguitando il Salmafio nel |
le che la voce poltrone venga da Police trunco, dicendo che:
andare alla guerra si trova che si troncassero a posta da lor
dito grosso; B dovea essere usata tanto questa furfanteria, ¢
tali il soprannome, e furono appellati Azurci secondo che
Cellino lib, 15. il che.volea dire poltreni; poiché Murcia pret
mava la Dea dell' oziofita, e della poltroneria, Origine et
non la credo vera, stimando che la voce polerone venga pill: sto da
poledro, ( come alcuni spiegano quel be/fie poltre di Dante Purg. )
Poltrone a.uno, che non vuole, o non può durar fatica, appu 0
dro, il quale non è ancora atto alla fatica. Ovvero da poltro, che
secondo 1 Landino sopra quel patio di Dante Inf. 24. che dice

Hor mai conuien che tu così ti spoirre,

Disse it maestro; che feggendo in pinma

in fama non si vien, ne sotto coltre.

Donde poltroni gli huomini pigri  e dormiglioti, dice il

zione di questo patfo.

PREG Sk FS oye = oe

— mestiero da abborracciare, E' cosa da farsi consideratan t
caso,
LMPLAST R ARSI con le chiare, Medicarsi con le chiare d' uovo le ae
di sopra in \cstan{4} A a Re
PARSI aar de' punti in sul cefs, Ricucired tagli, che ha nel viso,: quale cae 9 pe
ma cefo, perché guatto da i tagli, non merita nome di faccia. Cefe o Fran a
se € parola nobile, che significa Capo, come alcuai vogliono, dal Gr. gi grps mH
nol e parola di dispregio, e significa vifaccio brutto. ae 'a
STANZA XXxxl. STANZA XXKX ui
Baldane in questo per la più sicura * Et essi andaron con la lor patente tp
Due gran Dottori atrattamentiinuia, Di poter dire, e fare, € alto ¢
Lun Fitfolan Branducci che proccura Lor camerata fa tra? a
D' haver se non po in Pifa,oin Paxia, Che gli seguia curioso per. =
<ilmeno in refettorio una lettura, Baldino Filippucei lor yy
ZL! altro è Meinforcon da Scarperia, Huom, che più tosto canta py
ChefeVbuom vine per mangiar vi ginro, Crescer volea come gli altri appa e
Ch' ei vuol campar mill anni del sicuro, 3 44a si pent),quand'a e
. STANZA XXXXIL. STANZA XXXXIV, 9 \&
Calfandro Cala Cheleri fra tanto Son alti gli altri due fuor di mifar «
Del Duca allora il primo Segretaria Ond! ei nel me? o camm
“ 7° loro un discorso di quel tanto Refha aduggiato sv hed)
evan dire al lo aunerfario Ne men pro crescer pit
Cacciatof, Giosieiae: ar
Escorso turto if suo vocabolario

Scriffe in manierayefeceun tale Spoglio,
Che mese un mar diCruscain mexico feglio,



NONO CANTARE: 445

PRES os | HROMOVE TH ANZ AX X "BV. lov
ella pure alor quiui's'inchina, Purche il nome confervi di Regina,
Dando a ciafennoi fut debiti riroli., Luando per t annenire altras' intitoli,
Econ essi ferme IL altra mattina Che questons le nieghin, chiede al mato.
| Mdiscorrere, e far patti, e capitoli, Wel resto por da loro il foglio bianco,

manda suoi Amba(ciadori a Bertinella, i quali con essa fermarono di
stabilire i capitoli della pace per la matuna seguente, promettendo la medesima
| Bertinella d' acconfentire a tutto,pur che le retti il titolo di Regina.
DE gran Dettors, Dice due grandi, perché veramente erono ambedue di. sta~
a ce alta, ed un solo di essi era veramente Dottore, cioè Ficlolano Branducci,
ai che è Frdncesco Baldovini giovane dotto, e spiritoso; ma perché nel tempo, che
i fu composta la pretente Opera era assai difapplicato, pero lo motteggia, dicendo,
che egli proccura d' havere una lettura in un refettorio, se egli non la può otte-
 Berein Pifa; o in Pavia. Ma non voglio già io lasciar nelle menti di chilegge-
 fala presente Opera l'imprefiione', che questo Baidovini fulie lettore da' Retet-
fod t0rj, € pero dico, che le (ue beile, ed erudite composizioni lo fecero conolcere»
infin in Parigi, dove essendu fate fenuite in diverse Accademie dall' Em. Sig.
ym Card, Chigi tino di la lo fece chiamare a Roma, e lo diede per Segr. all' Em. Sig.
» Cardinal Nini, la qual carica eghi esercito pi anni molto Jodevoimente; ma
kit Beceilitato dalla poca buona sanità, che godeva in quel clima, se ne tornd alla
| patria, dove essendo stato prowvilto d' una Pieve, quivi se ne vive godendo mag-
b,@ Blor quiere, e miglior faluce, che non godeva a Roma. i
él MELN forcon da Scarperia, Pierfrance(co Mainardi grandissimo di statara, ma
G8 ware dottore. Questo per esser,si può dire,un colotio, ed in sul fiore della gio~
veotl thangiava ati, e però il Poeta dice, che se 1 mangiare fa campare, ¢gli
(Ill Per viver molto tempo. L'iperbole di mile anni (e bene \& di numero determi-
'ge ato; si piglia per indeterminaco, e signitica lunghissimo tempo.
I * CASS ANDRO Cheieri, Cive il sig.\ Alessandro de' Cerchi Cavaliere, e Sena-
we tore Fiorentino Segretairo della Sereni(s. Granduchessa, e però ii Poeta lo fa pri-
mo Segretario del Duca. E perché veramente egli € un Gentilhuomo di gutto
"i isquisito, e d' una cloquenza aggiustacissima, dice, che con la direzione del Boc-
sil caccio (le cuj opere regolano la lingua Fiorentina per esser' egli il nostro Cicero-
Ne ) ¢scorrends il suo Vocabulario ( cive il Vocabolario della Crusca ) messe um mare
di crufea in mezzo fostio, e (cherzando l'Autore con l'equivoco di Crusca buccia.
uv del grad, ee CRVSCA Accademia Fiorentina, intende, che questo Caflandro se-
id 'ce un diflefo compotto di parole approvate dalla medesima Accademia della,
», 'Crufea, nella quale si fa proteifione di pariare, € scriver pulitamente la vera
“| lingua Fiorentina.
7 PER far un diffefo di quello, che doveano dire, Cioè per metter loro in scritto
I Iattruzione di come doveano contenerai in trattar 'accordo,si come si faa tutti
gli Ambasciadori, e plenipotenziari, che si mandano da' Principi, Repubbliche ec,
 FAR to spoglio a! un libro, Mercantilmente's' intende copiare le partitede' i de-
 bitori; e per altro s'intende quando si cavano da un libro quei concetti, tentenze,
'parole, delle quali ci voguamo servire in far qualche composizione.
POTER dire, e fare, e alto, e bao, Potcr negoziare, e conciudere a lor gusto, e volontà, che

patentee Bi

libero.

LALDINO Filippucci, Filippo Baldintcci d
e questo intende il Poeta dicendo Huomo', che canta ben
¢reicera più, perché egli e duggiato da quei due huomini lunghi
e Meio, de' quali egli lo dice parevte, non perché vera

eg ee ee

e accomodarsi alla rima. Questo¢
jamo detto sopra nel Proemro. ~
* LVUOGO

5 STANZA XXXXVIL
Eperché ore già finian del giorno
Siconfuled, che fulfe fatrafera
, Percio tutti alle spanze fer ritorno
Com! un fatto digatti, fuor di Schiera,
I Cittadini Pavan @ ogn! intorno

* Welle radesfu i cantize alla fronciera, Che non si
Bicivcgh' ognun secondo il suo porere Gis teiehnzs Gene Bl
o 5 forestieri in ala dia quartiere, Sti Mab spefa dicey men Wid

a ST AN-ZIA*REXRWVNA ome DAVIN
o"Del Principe a' Vgnan pot si domanaa, Poeperre ner

“\  perché la labarda anch' egls appoog
* 'Staffer attorno a rivercar si manaa;

un facco, a quait
LA quarticre »
fied a ME i

ee

uae

BG

anggiaco, Vuol dir luogd, dove nonatt
Pinterposizione di muraglic } o d” altro, EY Gail doghile pian 00
tate, € con poco vigore, e i dicona auggiare; da Yggia » ombra,;
TENNE un mexro miglio di pace. 'Per mbitrar', che queni t
haveano le gambe lunghe, si servc di queste"iperbole'd? un imezzo mi
DA loro il fogito bianco, Apptova tutto quello'; che essi conchi
loro Jil foglio, bianco firmato di tua mano,acctocche vi ferivano lee
capitoli della pace, come più piacera loro, 'che è lo stesso, chedit
in voi in tuto, e pertutto, In questo senso dific il Petrarca ». my

"Chi Lhabbia racceteato, e chil' alloggi; x
Etiendosi già fatta (era ciascuno sbandd, €d i Terrazami tt
sex dar' alloggio a | soldati di Baldone. Bertinella iawn Pala x

¢d il Generale, 1 quali accctcarono Pinvito. Si Cered deiDuca per co

'ch' eli in Palazzo, dove-bnalmente egli venne dopo qualche di

o che non voleva parursi dalla iocanda, nella quale s' era accomodato..

COME un facco ds Gatts, Cr0e lenz' Ordine, o'regola, ma con!

~ tende, che ifoldau sbandarono, chi io qua, chivin Jay come

Gi dja! andare.
rova aliogyio, Dar

aan ta
a i

os MBA

aa

: wa
STANZA XXEK
Grants a palarro Bi
In Amospame eC
E-wuol che (gli odj mai:
Stien seco, ma ciafe
» Puer' finalmence ne i preg

Se, es 8. SERS PES EL EETE RBPRERPS SR

S” era decniarovoue
Priaichiei n'wferf
Nand per:

dort ~ 1
quarticre significara
ae Swan grote hk ey



em, Sista 30a sobre ipaaie
dA 12. epill. 33. quidem,
r Fer ime —m sed egoa egh, ur eee - Croe noo
wesmercnen gliteci croppe cirimonie. E appresso. Pall pot C. Ca~
» Hlorum ego vix attigi penulam; ramen remanferunt Dichia=;
e ferraiyo'o jinuitare uo. aitaseawate » o pregario a voler 'rima-
co noi. £ ta/ciarsi tirare pel ferrainole, e non accettarc l'invito » € ari
Koa > '
CH! vs disagio, Quand' altri e invitato a un conuito aed
teatro. datalcuno.y.per licenziarsi da chi lo tratticne ta full' ora del ¢o.

s te la-causa speria quale ei i parte, suol servirsi di qu:sto ate
al eons (a, non dia aifaeio: cioè se 10 son causa, che egli (peade, aun e dovere 5
'disagio-col tarmi aspettare.
“ ee ~Andar a mangiar a casa d' altri senza spendere...:
operat ferraiuolo, o ¢appa.s perché in vece di quello la porcano sul-
i:Alabardieri + i quali in occasione 4' avere aire a tavola  s¢ ne, spa-
ae appoggiuala-aila parece, e perdo.con quest) decto intendiamo. Posare ra
ior (ad! aters5c.quivi mangiare, se bene Pe/are tl ferrainolo.s' ay
“4 ancora un giovane, che non ha provisione, ma serve in uo banco,, o 'in who ff.
2ibegravissy bastandogl d' edereimpiegato, e d” abuuart per poter goder€ col
oe
MWAMBRA locanda. Incendiamo reli Alberghi, o vero Osterie, che danno, da
 dOrmice a vforetticri.
SERA nce wiare. a era nome eed Havea eletto quel lyogo per' Abto
Fipotor, exis Wiens t
VOLLE mille Porei. Vole iacpdofiaith di citimonie, e lusinghe: ed. e io. neiio hc
'chevwererdetto: itopra'< Com fran Bche Janene, così dewto dal Latido vente c1oe
di corpo, e gi fl
“WCODAZZO\, Intende seguito di gente “dictto.« Warchi Stor, Fior, lib. I2.Faé
al Primt Cittadini eli fecero codazrodietro, accompagnandolo, eraccompagnandolo gaila
we ius Cufanl Palarrxo; comes' ei fusseril padrone di Firenze,

hat

Ltd fate

oh “WHSPANZA thy STANZA L...,
A cena (perché il giorne in questo loco dn cambio di guarir dell' appetica ~~

a: * Lblebbertvairra faccenda le brigate, Facenano un collo come nna. Giz est

; 8 arta cucinave intorno al foco ) Se vien frictate, og un Sana accinits,
wt Senses furia ds friteate', Che per aria chi puofe.la scaraffa;
od e nem ipresba si, ma duran poco, Si riduffero in brene a tal partita.,
3 \Che-uppena farte ellveran già ingoiate, C' ogms volta faceanoa rufa raffac,
a Presse gente a rauolaera molta', tn ultimo seguendo Bertinella. >»
gi sR, We" miangiawan dueye tre per wolta, L! andanano @ cauar.dela padella.
gf oWDelerivetarcena fatta da' Bertinella a i Forestieriy la. aleconfiflettga, in,
pt fritcate » mangiate con fa fiiria, che egli dice: passo Reale, e cirimonie conue-
if se a una Regina di Malmantile.

iin fueria di fritrare, Beitvate in quantità; 3 Waa gran quantica di Fricta.
sopra C, 3. st. 50. EXIT:

eet



.

448 aan F EDR
PRITT APA SEE viv eda! factard WOVa bE
felid'pddella  asfoge ia aveortah,ielde mene a
125 appresso 'atirort baslerebe dine, petcheirgioy
sce Sen eal: as tra ng “
GIRAFF-A, 'Avimale quadeupede § ikqualess se bene
fidema,€ s citaaaiea Dencaaneg eine toy -havil €onouid
a'quello del' Cammello ylegambe dinanai abo i quelled
coda j ed è del colore meuctia®,- che q
i Latini lo dicono  Camelopardalis, cioè bela Yeheticne! I
'Pantera, Pannoil-coo comenine ewafnd inwndealiangas Lio eel
interpretare, che non' fifazialleroy" perchemmeareare | dial
cibo con gran'deiiderio';Latino-¥ehiare } 0) chesaliuagassero ene
beta

a

per vedere donde, e quandowenivanolle Feiecace ena
refize'a tempo tuo fa menzione ibPolizignò-nelle® pellance, » Gitiog

Scaligero  simil dit questo:
ail Esercitazionie 209. nutn 3s OVedice "hei Persiani Girmafa P. f
E Abts il BOM Gina Parse Omit » Ha" eH) bho mee
o STAPA accinite, Sravarateetito'y Teo} oprepataco sidal Laci
-didiatho stavalattento', <u'all' ordiné cones, tnleMtaro.chigmaw,

ho tifato i ahtivo's particolarmenté dation Villanty*s sempre in

spele sei ptovvedere danatir: "Ora /peritintratctared! Origine softy
nendosi il danaro a fructo, la Corte  prititipale 9 siccome da"Greciy dalla
detta Capo scost-da nO1'fi ehiamo Capitalc; e Fondo! ancora, dai tei idere.y 6
la petunia data a intereties a:pensa  di fondo, e»pedere!, orpotieth
ta'; Che Perd:' nftra, come geactata dai danaro \y-the! ayprincipi
Greti Chiamarono Torr stioeParre, I Latiniyemes siqua nig
fu Ud Varrdne', e da Norio Marctlio Oticrpabome apiraies p
posito'; ff disse la forte  pquafipecinia capitales principal ndan
che da questa pechaiirpolta 12/a%phincipio s hevenivd poirdngu

da' Holtri anticht Crvaren, voce che finulmentestrovatiun Gio «
la) che i Franzesi didero chewanee,'cioe rendita envratayda Chef, capo.Ora
cinire, che anche dillero, Cixamgare, e lo stesso, che Provvedere ti
<cidé \& chiesact, aflegnar fondi's*¢ ludghi da rischotere; foraire, e:

See. > ea ae PERFEPE RSET ERE RRR E

rnito, nogeiLesto » sircensp
. oP OPP Dee oes ai
y uP Via Con firia, come si-fardellescara

atrornd Peiitrelehh Voce alle vdice usacar; enn Jaycredo

i rofto 'fied' per bi 'iaS* a la asad
Pi Tn ape. Si dice  ido sono più gente d' act
Gialcuno  affatina con prestezza's € (eti2"Ordine, O-regola dip
'egli pud'dic Shae Sad repair med, toa? inciutlese

i e da notare 'Poeta | ' i
Pin pane sopraveiene fiipro
fritearemifttvie? dalle macenier Unica feu

ve



|

!
!

Stanchi di mangiar, non sazz}

Finito

BPA ficsass-

STANZAL
'at anna
Tal musica fini po poi in quel fondo;
Ma perché dopo cena sl vin lauora
Facean parzie le 'ior del mondo,

| Fra' akre Bertinella, e Celidora
inganancieree per burla un bale tando,
 Eapooa

4 0. entrouni altra brigata
Tal che si fece poi veglia formata.

sien STANZA LIL

'Fano poi com' è  usanka
Moite candele intorno alla muraglia,

 Lesplendor delle quali in quella franca
E sale, e tanto, chelagente abbaglia,

+ he distinte si vedeva in danza

bt meglio capriole intreccia, e taglia
Wannaccio in tanto [opr' alla spinetta 2
S' era mefioa xappar la Spagnoletta.

NONO CANTARE.

Z rel taano gestive insane discadess lnnetira nazioneda
orit quali dicono, che i Fiorentini fanno je frittate d'un' uova !'una per rilparmiare;
 \& però dices che durano poco, e per questo ce ne vogliono molte pi + sì che per
sta ragione non è vero, che si facciano sottili per risparmiare, essendo certo,
he tanto. 3¢ tanto unto si con/uma a far' una frittata d'un' uovo [olo,quan-
wm to a farne una-di sci; onde si viene a consumare cinque volte più, perché una
- fristata di sei uova faziera tre persone, e fet frittate d' un' uovo l'una.non sazic-
un' huomo solo. sì che non di sordidi, ma di ghiotti in questo partico-
potion esser tatiati i Fiorentini, che fanno ie frittate di poche uova l'una,
inché sieno più cotte, e più gustofe. Di questa verita si può chiarire, chi non
erede, con fare a quattro persone due frittate di sei uova l'una, e vedrà, che
eranno fatica a finirle » come le finiranno ben presto quattr' altri, a'quait fa
dieno dicci anche di due uova l'una, purché ben cotte, e questi si ridurrando
a rufa raffa, ed a rubarle anche dalla padella, come facevano coioro di
tile, Raffa raffa \& lo stesso, che il Latino rape, rape, dal Latino rapere,
 fifece rabare, e si poté ancora formare, rappare, come il Boccaccio in una sua
'manolcritta da fugam arripere, formd Arrapare, o dillero la fuga.
r « Leppare, voce della lingua furbesca può venire di qui, o più toflo da
vare, significando portar via con prestezza, La figura è la medesima, come
Tose dice Prometter Roma, e toma, per avvcatura dallo Spag. tomar; quali;
E piglia, ch' 10 la fo già un, e tela dd. Tre agiole, e barugule. L. naga, varie,

mgé. Daa rufa è facto gure; scompigliare.

449
ei detrat-

STANZA LIID

Un gobbo [no compagno un tal delfino
C' alle borse. più rosto, che nel mare
Tempesta induce; prefe un violino,
Che fonando parea pien di zanzare,
Intanto un ben dipinto mefolina
Si porge in mano a quei ch'ha dainitare,
Et Ygnanefe, al quale il balle tocca
Sciorina a Kertinella in susse nocca,

STANZA LIV.

2' grave il colpo, e gingne in modo tale,
Che quanto piglia tanta pelle sbuccia:
La Danna, bench fentasi far male
Senx' alterarsi in burla se la fugcia,
No vol parer ma infel'ha poi per male,
E dice l' orazion della bertuccia
Sorride, ma nel fin par che riesca
tn un rider più tosto alla Tedesca.

» che ebbero di cenare i Conuitati cominciarono a ballare così in burla,

Ma crescendo il popolo riusci poi veglia formata. Così per lo più segue fra lay

dalla quale nel tempo di Carnevale, dopo le cene solite farsi

x i, si da ne i suoni, e cominciano a ballare fra di loro pa-

Ren, e fenvesi da chi patia per le Se e da i viciui vi concorre altro Boge
it: 1 e



ayo MALMANTILE

e si fa vera veglia di ballo, come segui fra questi connitati
quali essendo toccato a fare da mai 'del batto alla messola
egli inuité Bertinella, perquotendola co! messolino
che le sbuccid le nocca, di ché la donni's'adirò,se bea non ta
ballo alla mefola si costuma in queste veglie per introdu
lo, che è eletto Maestro rocca con que! mestolino le mania
vita al ballo, e poi tocca le mani ad alcrertanti huomini, ¢q
vitate vanno a ballare, e nel ballare il Maeltro da il me!
ella va con esso a toccare tanti huomini, e tante donne, € così
tri usano questo ballo con fare, che il Maestro tocchi ante:
lato che hanno alquanto fra di loro, vanno senza mestola a
mini come e solito, e si seguita senza adoprar più la mefola',” Q
si dice batlo alla messola, si ta anche colla pezzuola, o Oy
lando si getta a quello, che si vuole inuitare, e così di mano in
chiamato Ballo alla pexruola, 6
ST ANCH di mangiare, non faxxj. Stanchi dal? affaticarsi a maflicar pi
ma non già fatolli, perché havevano mangiato poca roba. Ll Petrarca nel T
fo d' Amore, nel principio:; ne
Sranco già di mirar, non fagio ancora,
Giuvenale Sat. 4. ragionando di Meffalina moglie di Claudio
Et laffata viris, nondum fatiata recessit.
TAL misura fini po poi in quel fondo, Alla fine delle fini tal' opet
nd: Pur una volta fini. Latino ad extremum, tandem, aliquando,
C, 4. st. 9. in questo C, st. 1, alla voce Bordello, € sotto C. 10,
ne po pot, ec, Vedi sopra C. 2, st. 73.

sR SERS TSE RPESEE

=
=

Ha SPR a=

a W

iL vin laxora, 1\ vino opera,fa la sua operazione con dar” alla testa, e '
briacare. Del suo lavoro, € della sua operazione si può dire quel che difie} ka
delle pecchie. Ferner opus. i ty

B ALLO tondo, Specie di ballo, che si fa, pigliando più persone per! »
¢ formando così di tutti loro un circolo, ch' è forse Latino Choreas m
nostri Toscani detto Carolare. ee Ye

VEGLIA formata. Veglia vera, e solenne con tutte 'le formalita, i 4
Vedi sopra C. 2, st, 46. dove teoverai Jutrecciare, e tagliar capriole, \& ie 4
st A

23. q
Nunn acco. Questo fu un tale nominato Giovanni, € si diceva
cio per la sua (ciattezza, e spensicrataggine [ poicht fo nome \&
del vero nome Giovanni; sopra il qual nome è da vedi tole
della  Casa ]; Questo insegnava fonare la chitarra 4/ed if
pochissimo come quello, che non haveva cognizione cna della
rd dice epee 4a spagnoletta ( specie di danza ) assomigliando il
cato delle dita in su lo Arumento, a uno, che zappi: e Spinerra
balo, o Bonaccordo,,
VN gobbo. Intende il gobbo Trafedi, il quale faceva p

violino, ma fonava assai male, e per questo iI Poeta dice: ch
@i xanxare, assomigliando il fonar di lui al ronzare delle



d NONO CANTARE,
'It! mipiccoli alati, co acutidimo pungiglione, Questo Gobbo servl alla Sere-
oleemmt aioe. quaita di Nano, e per le sue facete manicre piacque
" salia Serentis, Arciduchessa Anna d'Austria, chg o condusse con se, quando an-
do dove entro tanto in grazia al Serenils, Arciduca Ferdinando Car-
Jodi lei marito, che  arricchi non solo con li suoi gro fipendj, € molto più
con I regaii', ma ancora con 4 denari, che questo generoso Principe si lasciava.
da efio nel Bins delle mane » nel quale il Trafedi era astutidimo, e face-
 'Ya grosse,potte, perché fapeva, che perdendo S, A, S.non voleva eller pagata,
lige se vinceva era pagato puuwwalmente. E per questo il Poeta dice, che ip un di
Wh quei Delfini, che predicono rempesta alle borse, come vogliono; che il pesce Delfino
ica la tempesta nel Mare, e perché questo pesce pare, che sia gobbo, però
i ) per coltyine chiamar Lojfini, + gobbi, Mori poi questo Trafedi, e la-
jit scid mece.ie sue faculta a una donna di camera della Sereniss. Arciduchessa, della
Co qual donna haveva tatco scmpre¢ da innamorato, con patto, che si maritasse con
un Fiorentino suo amuco, che era in Insprug, come segui.
1 MESTOLINO, Cucchiaio di jegno per uso di cucina: Diminutivo di 4zefo-
4, la quale in Lombardia chiamano 44¢/cosa, dal mescolare,
Ada inuitare. k4a da chiamare ai bailo, '
—— SCWWRINA, Chog batte gagliardamente, Il proprio di sciorinare \& quando si
get ort > abit: di paano fuori delle casse ne i tempi di State, e si disten-
 dono per targlt pigliac aria, batcendogli con (curisci,( che dichiamo camari dal
pot Greev camaces) donde scamarare si dice questo battere, per cavargli la poluere,
st o Per liberacgis dalle cigauole - E da questo scamatare, o perquotere j panni, ec.
igel Pigliamo il verbo sciorinaré per perquotere, E sciorinarf? intendiamo uno, che per
 A gran caldo si leyi gli abiti daddotia; Dal Latino ara detta poi ora coll' o lar-
f £9, quale si fence, quando.la plebe de' ragazzi con sua antica canzone grida al-
sath le matchere u carnovale efiora Ter, in Adelph, Accipiunds, o muffitanda in iyria
adalescentium est. L' huomo se ladeve fucciare. Quivi Donato, Adafitare enim,
pe
4

4

Proprit'ef? dissimulandi canfatacere. E Sopra. eHufficanda; Patienda, consideranda
cum filentio, Gc, e dal sao diminutivo non usato orina, cioè auretra, ne riufei il
verbo Sciorinarsi, che è lo stelio, che se dicetle,con Latino barbaro, e ridico-
fo exawrinare. Netia Valdiaicvole dicono; scfobacare quando exopacare, cavare
i day'. opaco,;
IN buria se la fuccia, La comporta come fatta in ischerzo; dal fucciare-, che
"| si fa, quando si feate grave dolore; tirando a se il fiato,
| NeMivuel parere, mat' ha poi per male. Non vorrebbe, ch' e' si conoscesse;
mane ha veramente havuato diigulto. Virg. premit alcum corde dolorem,
DICE Porazione della bertuccia, Dice de] male borbottando, o brottolando
sotto voce, e così facendo con la bocca quei getti, che fa la bertnceia, o scimmia,
“quando@in rabbia, che pare, che elJa borbouti, e discorra dentxo a i denti; che
-diciamo comunemente, che ella dica orazioni. —;
| RISO alla Tedesca, Rifus fardonicus. Kifo finto, e che par più tosto pianto.
In lingua Tedesca ridere si dice Jache; ond' io credo, che il noflro Autore, che
“haveva qualche cognizione di quella lingua per essere stato alquanto tempo in la-
sprug » habbia detto ri/o alla Tedesca ° non perché Bertinella ridetie, come fanno
12 i Te-



st¢
pero pla
argla
Fase bine ) Che fiand' similt 1
meazione.
STANZA  ae ai
Al Det veramebie pare Bratig? 2G Ya beffii ?
vse "babii bya A onde or Bhreded 00 ee arvlnesy z

Perch gli par a' haverle dato prano, Ci morde in qualche part
ernei d haverla tocca a malo envoy “" Ech) se
Ma quando sanguinar vedde la mano,” “ts ee
Io mi difdico, disse, e me ne penta,?\ "Faia
Finalmente to ho tl diauol nelle braccia, uel mespolino
E [ono, e faro sempre una bestiacci@ ha vette!

STANZA LVR?
Wer carargliene pena, è 'Biri
nonfacome,al paren h
Dror

Sl e WSRoERRER

ae
Rin arap in Canberit ih fablerro 2 « o 1009 Syaadermapuoraee:
2” aaanai più TPinig ane Se raceme ie Casaliadonma,

: STANZA LVIL: STANZA Lk
He Principe'a quel oriad ) Wigule? emairep 'LLG ridsa\ Dortma ator come'
3 'a foggiradrd ¥2 Wictrtdto here', 192% «Bldipolaize mance,

= call tro 'du hhh VeAbite dhe JOU 2 « 28 IRE scowe/l ah aldorne gal
Co amore in tui vuol far le sue vendette, —- OUR GNMEGC cated fareRiifwe

Ui quel vive fhiattin combean picebio', "0 « Ds iene yx

erkriwet:

es

“CG abiriih aiiplerofkdW maiekirE\2) 6 OG RLU enareeinura pei

IL mefolina, o quei, che glienc dette Di non mostrar in ranto 8
«NB per BP laa bdr qa Ol 19 LY pPERRE OS Gece vel wsusorat medic
SO Po igeres in terrain Cente rnild pores |. 129 UD anguenre che teyfan

“0: Bj doe 'G mara vighias che ta Donna*faccit st gran/laniento pparendyy
Osporer haverle® ad maida (anpucuccortifi,, che ib male
2 Se G7 QUEP EY pli HOU eredcvaYy-ripreh de se Reffo |) 2 si metre ivo!
WY CON Medica Me HE biedtlediAratite si feuop! namiorato 0
OCP ARS, enande torino.' Rifentirts'¢ dolerf tanta) 1c! ol amie le
>. Ona ABD Read, “ANfatiday AppehalNon glipard' haverla quai
Sreata, e da Stevicate; e Srenravee dal Liarino settentarey come owii ¢
z ati Cie, Si adPAtcic, ALE wir msiferdque fustento..10 MeHor} cioè paul
yea thiala pera mi Condued's @ ati reggo'.° Non folametite dicta
Sich ea oial caeueanae ” | @ mala faricay
C'ferto'y Batinovint' ech ytenid} catkanten\ B fitcometi dies 464
Bebe » cioè grandissima. Ho auuta una buona malartiey
1
Ou

ee ae on ane:

imal¥orza'; pochissimo,.»vsn wed wy


'

ere a
icp. divenfamnenne 4 ne

races ne aa Ow at

sor we! iy 51
nym ORG Shots ovo

vale CMOS en no Mle ghini bizzef.
a RS nme S'S on ioeaeemaneenenie gee

wh ib sen oh
+ CANom artes adenine
a “od medesimo in lode dellt,Vimor.malancolico,..,. A drow Seqeeay Bw '9 sConUL AE

611 » Bvan fuggendo ogns altra compagnia, ) 2 ASWA bE
aaa SOL op Ae Cd ghizibizein 98 concerti, e 4 CARTE, 96 sui grktis WY
yo viens o, Lheecompagman pur sempre.vada.. a8 iA sce s\c08 DAN
j Story Bior-liba.t5e.dige < acca, LAA AER, Seonpes ghicibiczand
Plow dy Dihoeae'D sua | ee
@bArvcca il-ticchio,.. Gliwien, questa volonts ».pen eG @ gape o afoul dal
« Branzcle,%ix.,. mosca caninay;,Sumili, ma Aatnabate Penie, bvalilla.s.¢ Al
adaliailillo.»che ¢,una-molca pungentitima » che infesta 4 i da noi ia
i coenaadad pacerba faransy quo tora aermorndeyert peta NB
mda S 4 APS A
ae Relacanantciocansti, e! dolore. che; prova uo pazeiente,, quap-
See una fericafirmettelale, accto, o,altra.cosa, simile.4.che, moktihica, e>
Corrodede. partielle de iquali carpi acri, e mordags sembra gg..al, ate a
Buila difrecoje feriicanms oPRAZRHO. —.sisshwo9 90. 9\ml)o 5
RAR un tira 4 an Suniendettar uo mal Ceri: c0 a che a iactia a
UNOswiaw A oie ar re we-sfhow vow wh srsivsusily stls,, in
h\STAACCIA comme 4 pleebiawsE grand a.collera Bg i
' sofehiacciare Ggnificabatiene identi per la collera, —- per a ae; ed ha
piquetto: significato fenz! aggivagerus come vom picrhio ma,tal Gmilitnding s ay
questo. uccelioiha propriesa: naturale dij batter, sceau cere
rosted.in su sramiideg|t aibert per; fueginsdefarmu e sliggual ce
concbellittima giz »che;¢ queiasiMope haysr, molgo., eet 2
'¢ ville uscir le formiche si diflende come morte sopra, quel amo,» €, Ca'
ladingua g, che è lunga, e carnosa, e quella distends opera il, medesimo a an 2c
ose formighe, vi vanno sopra per palcerti, e quando.al Picchio, pare di haveruene
——— abastanea), ura ate taolinguayeddngoia, aDa.quest '0 uccelio deco in
» Gea Oryscalaptes » 0198: Pictinatere di quencery € InnLaty pics li.c.formaso,probabil-
ovamente il, verbo Picchiare,cioè. batierese. chi batted demtlperila stizea,paresche face
lorfiedle romore,ca.tdenuyche fa ak prcchio cal becco », Plasto spel, pro-
Seeremersaniice srond Sai S108 OH,. ai )
= MANDA giù Trinigante,eAacomisto,. Bestepymia >maledice tua tal Be,

WAKA

=

adil SeEEELEe dpe



a

454 MALAEAN TILE Oe
¢ suoi falsi Profeti ¥ pn eee
colle maladizioni, coprecesteats e bestemmie oe
GV AIRE, RawmaricarS, eoeee aie: i'
gagnolare. Vedi sopra-\cstan[4]{}vventura da wagire ee
guaina; perché i cani quando ne ae tocche,fanno um mug
gito de' bambitii'. 'Si può anche dire, che venga da 1 ase i
rammaricarff dell' huomo. 1 wales Now, 2 bn R
comincia  ffridere, e Puaire., a he wl
METTE 4 foqquadro, Solleva, e mette axofgr tutti i vi
re, Soqquadro \& voce usata dat muratori'y eee Ȣ simili, e v1
squadro, che è quando per accidente d*
mancamento un pelo tirato, o strafeiaatonon può fare” ib suo corlo,
rd cagiona, che giù steomenti del veicolo, o treno facciano si ito at
per lo sforzo, ed affaticamento yche riceyono, eda Yale
drare, e mettere a fogquadro iv vecedi Rordirecobromorey) \&
/MBLETOLIRE, Commuoyerti } Intensrire « Vedi sopra C.
tini pure in vece di /anguere, dicevano volzarmente ne! sane
eficr cenero., e moscio, pigliando la similitudine das real ¢
signitica erbageio 20 ortaggio; Auguito Imperadore formé una 5
rola, e dilie Serizare pigliando la similudine dalle bietule, ~per vi
languids'; non istar bene. Vedi Suetonio nella Vita d Augulto » Ove:
voci, e maniere particolari, che questo Principe usava, nel. par
Celio Rodigino lib. 15. c10. Now similmente, diciamo! fauna
si, illanguidirsi per il aah d'amore, B Bretolone pre a hu
mii fatta;
BESTIA scimunita, Spend spropositato senza jmenlitnaiasiiya -
zio affatto. Lisca Nov. 2, dts perché. ellaera ponera, a queste se
torre senza dote, ec, Scimunito; sciacco, Scimunito'é lo stesso che wren
Lat. incastigarns, Gr. acolafes, che not riceve lammoniziani;) €
fictti, monitoribus aff ah E perché questi, o simili a loro fogliono essere
ale il giovane deloritto da Orazio, Sublimis cupidusque,o amara reli
nix; E qual'é quei, che difvaol cio, che volle: come disse Dante nf
ro nell' Fliade al terzo libto; Delle giowani genti rigogliofe Sempre per:
tere menti; cioè per dirla volgarmente hanno il cervello sopra Jab:
@ che Scimwunito', che di sua natura yale Non ammonito, non riprefo 5
stigato, o che non vuol essere amimonito, ne riprefo, ne galtigato; ¢
rio, € mentecatti fanno; venga' a 'signiticare /eiocco, e haomo dt
to, L' esempio del Bocce. nel Filocolo lib, 4. dove: parlando come
Il tno diletto e dimorar ne! vani occhi delle foimunte femmine, pwd elle
voglia dire ancora licenziofe, immodette, intemperanti, e non
ze solamente,
RAGNATELO. Ragno, infetto noto, Dicono che perm
dej cane si piglia del suo pelo, e fiipone sopr' alla parte offela,
HO sopra C.6 stan. 6.¢ che il ragno, e 'o scorpione aumpa
foper a la piaga che hahao faita coi loro morfo,suaino il pazziene

Page

eFEEEE

 REEE

ea

Se Ss See Se


'*
NONO CANTARE: 455
necredendo chest pezzi delmestolino, habbiano la stessa virtù; lega sopralia se-
rita, che ha fatta col mestolino a-Bertinella, idetti pezzi Maforle Baldone:, co-
me Soldato bravo » haveva notizia della jancia,con la quale Achille feci Telefo,
ee nea sehen havea detto J' Oracolo, i, Qua
. iabit medebirur, Donde Dante afer. C, 31, disse:, '
lo) loi Cosnod! toche foiena la lancia,

he 14, 0D! Achille, e del fu padre esser cagione

tHe Prima di trista, e poi di buona mancia,
| -\Bierede; che il mestolino habbia la medesima virtù della detta lancia.

>

Buk

qt ALAN del Cielo » Quali che Adanna def Cielo, ¢s' intende orto rimedio per
at fanar male,»come fu ottimo rimedia per liberar.daila fame 11 popojo eleno
wiytt inane che. Dio giù mando nel deferto.. diFirenzuola in lode del iegno fante
io, 3 > <oSy
se) sbiaib shoizwe2 S\& uno non mangia, s' un non si riposa,
lags i Osha il fegato guasto, ole budella,
Rab > Bgli è a man del Crelo a ogni cosa.

** Nota!che in:questo detto la parola:4¢an:non vuol dir mano, non, essendo pa-
Ola figurata per apocope, ma nell*intera sua efieaza Adem, che così si trovan
scritta nelSacro Testo quella, che Dio mando al suo.Popolo (che noi poi chia~
jamO manna )¢tal man si dice nella Sapienza al capo 16. che havetle ogni buon
x vien chiamata quivi Paze approncato, e apprestato dal Cielo fenya fatica
o pero iniqucito detto credo che fr debba intender e Zanna, e non mano per si-
se uba cosa ottima in ogni gencre; e-che ciò sia vero, quando sopravvie-
he a*yao® qualcosa di suo gulto,suoi dire: ' wa manna, e non mano: e se uno
ricercaté | se per un su6 conuito una tal vivanda gli piacera prisponde farò Atan-
adScome si Vede 'fopra G, 8. stan. 43. Se bene potrebbe anche dirsi», che colla
feta parola si aljndetle a due signincati, e a quello che ora di sopra si è detto,
WMtan; cioè manna, e dian, cioè mano, E ALano de! cielo potrebbe parer det-
ta Colla medcfima forma', con cui diciamo di qualche rimedio., o medicamenio
cfitace Kyi e (Paro la man di Dio, il che coceisponde a ciò»; che dice Piutarco
fOnumM Conuiuialiam lib. 4. quacit.1.)cheun certo Filone medico,aicuni me-
'Witdinenti Reali, così decti perché erano da Re ) enon da Poveri, o per essere
i*! fepreti di Ré jo per la loro eccellenza; e che dal (occor(o potente, che se ne ri-
; ceveva, erano-chiainati /exipbarmaca, appclld com-particolare.appellazione
mani degl Idaij.
jd) WPREGLAT-A, e neva, Intrifa, sporcata, tinta, Da i venti, che portanan via le
i mmelecine Bal gran vento, che per le parti da baflo gli usciva dal corpo accom-
ip 'pagnato da qualche altra cosa; la-quale ricoprendo le mele che sono quella par-
ce più eafnola delle:cosce, che forma il sedere } ” alconde alia vita o costin un,
w? Cert modo fe' porta via; sì che il Poeta Meoppiando quel verlo, che dice. «Dai
md 'venti, che Portanan via le vele, intende, che la Camicia di Baldone era tinta dallo
z)
6
è

RELILE wtuateale

'sterco ':
SQVADERNA fuori, Cava fuori de i calzoni, e la distende. Morg. Le chiap-
/quaderno con rinerenza, Dante Par. 33. Cio che per o uninerfo si /quaderna.

! tele, ciò che è sciolto, e s(parfo per l'universo, prendendo la simulicudine da'

J libri sciolti, e squadernati. DR



'436 MALMANTILE |

DIRGLI manco che messere, ec. Dirgli iurie
tia dissero i Lat, ed il Lalli Bitar kon by eHloee Lich
è Teitt m' ha detco peggio che messere. 6)
Molti dicono + Asessere él' asino: ond' io stimo:che dic
che meficre s' intenda, l'ingiurid più che se gli havesse
Comico Fiorentino nella Moglie Ato 4. sc, 10. in-derisione del t
dice: Si; Adefsere e  safine, che va nel mezzo. Quali dica:
quando passa per le strade gli fa largo, eva nel mezzo,
BEL vedere, 1 bel di Roma ¥' intende it Colofico ycheinoi
Ciamo Culilco; eda questo per belmadere, Obel di Kama iptei
che Bertinella pericolava di mostrare alzando le gambe.
Bellofguardo, son nomi di juoghi, e ville nobilidime nel Fic
vato, e donde si scorge molto, e bel paefe.
eHEDICO da fucciole, Medico spropositato, e dipoca:scienza,
mo i marroni cotti col guscio nell' acqua, e preadono tal nome dal /ucciare 5§
fanno i ragazzi per trarne senza aprir wutto 1 gu(cio, la pasta, che vi \& dent
E perché questo cibo e vilissimo; pero foros iamo da si i
nulla. I Latini dissero bomo manct cioè di niua pregio,
fico; per Naucum intendendo il Gufer, o buccia di quaifivoglia cof
la, che si butta via, e non buona a nulla,
LE fa veder le lucciole, Le fa pianger per il dolore, Quando uno}
tale, che gli muova. le lagrime, pare al pazziente di veder per ari
14 di minutissime stelle, simili alle lucciole, il che è cagionato dall'
lagrime, e che pafiando sopra alle pupille offende, ed altera la virtù v va

Oe STANZA LXL STANZA LXUL,

Non dimostra la taccia così mesta S' impiccherebbe, ma dall' altro:
Quel ragagxo scolar,quel cauczzmola, Ei va pos retinente, e¢

Allor che motti giorni e (ato festa
E che finita poi quella vignuala,
Ji matadetto tempo ecco s' appresta,
Ch' e's ha di nuouo atornar alla/quola,
We si gualta belando si la bocca
uuand il matftro col bajion to chiecca, Gli vada in (u le forche
STANZA LXiL STANZA Lxl
Qrante cambiate in viso, e mal contento, Poiche '1 cundotto delle pat
Adefo pare il pouero Baldone, S' ha da ferrar(dic' egli,
\ o ha nna stizna,ch'ei si rode drento, Perché si la leva alle sue.
Per non haver cervel, ne discrizione,
Che benc' altrni la morte dia [pauento,
Se e' non fusse che e'c'è condenvariong
Achis ammarza pena della vita,
Con una fune baurebbela finita,

Con quella mane  alei dif

un mutmameess ofa. sealers a2 ESEG~0FE

oe Se


pas sro LIDUE

8 ois! me nyo 34 fan ne iipauni,
Ps em wre - dif oe vs onngoy eRereheymensre th' I ami, ella v anuifea
'chap yh Ob bu) Chomsany, u lite
a, e fooppia dali th si Sent habbia on acgquauite.
intovaci) Poetara Sabeshens cepaeha cheba baleen
ps9 sorim re eran moped Da questo a srcorgentof Bering
di.lei 3-4 h

a iste ine goticiod si sbaseni tmodh 1b i od LE": sy sais: Nas. aoe}
ianbopkematendonstetniens faney Olaltra sorta di)logame, con
eaeioees ed<altre:bestiefimili + Evcaves.t; filidice ancora,

fa merce) collooa: malfastori pquandog)' i =
Oy: 6. Saatgo! aah Eda emadl noiidiciamora un rapazzo-malig

 lioiywenn\ LiVai facendorparlare tn Pedantesdice 2a (09 11309 isos to att?
doped d jwoda, Hee, Seana \& osjuy 1k1gGs esm2) 901813 Iq issagayi Onnst
ab omtetharion Seieababtanitioher ants omiiiiv s odio olaup dioieg ¥
of Mey aba so, O folerro-vrifar ciferory PO Over owsd Orstid imaed bellow
OR iiteade 4a sesete st O, eyed |i ccaaiate marae) 19g e Colt
' A quella id nines ehocien en quel-pdatemps,
itp aeaast iid? eavpolonannavOaiee:
ivcen se sar bebe vignnola', l'ela dnrage pes inten:
credo che sia pata He Tulnadaeeibe deseo opera

mad C.9, nf eke an la) Petite Tn: Eictirhnadutemne 'a a' Buoh

cL aa

uh fa già'an val Sse da 'Panzano, ib quale havendo din' fola pic
ue co) Poe facevz a ex faye barilittivino; ede ae,
on rei hi achiorbailt, ed haVeva 2*OB Hi Torte frattesyichie tro -
7 AO' al non eglnop it si meow rubsiidoltuva,

mee ore Ha" mvae e sempre dieeva";' 'Chie'ratdoplievaer bani bofa
nefla; rem POscoPLe®, thie per fuor bifog ni" tai vende iadetenvigng ye
fre do pia Fitoperta della derta vight', HoH potevarabare 'come
ee: or Salis fo S*aFri(chiava a imbdttare ance witio's per fo” che
dom abdate alii fo? amici 'da CHe*procedeva  phe eel" err 'vino 5
edali Prisporideva, che era fina la vignuota ae a teiccoe dice: il
può eifer che” venga it i decacos ee ae a) “wip rehea
ov mang an =q oon

rig Pgh Syeepoda balie Roca Pane o.°6. stan.
nares @¢ fo stessd', titi duc verb} stei dal Tao's IP BalerNov, 7,
ane A ractomandana ' a phi poteres; e coloré antendtnand a vbinctirts 'chi di

pinseebed ls sha olehnon aol,
te una fine baurebbela fin ita: Havrebbe fiaito ee 'fub" CRN aziO' ton-im-

TANTO, o quanto, Termine, che significa piccola quantità, ed 2 lo feffo.
che par un poco; alquanto, Petrarca. E tu, se taxto, o quanto.d' Amor senti,

4 un sopractieni', Parca waa folpenfione, un preety di foptatrenere;
' 'Profiiagato il termine. Mm coNn-

bie —



453 MALMANTILE >

CONDOTTO delle pappardelle\, Cioè la cannaidella gola,
del cibo detto da' Greci Ocfophages, e da noi scherzosamente él condotto de! ho
che risponde alla parola Greea significante il porta cibo, o il Port i
piglia pappardelle, che sono lasagne corte nel brodo di carne per ogni cibo,
ti chiamano pappardelle la ricotta stempcrata con acqua rola), eu ova, a
¢€ poi fritta a toggia di frittelle.

TLR AR le quoia, Signitica morire, come dicemmo LC. 4. 20,
scherza, moitrando, che per la legge del Taglione si gattigar le gu
( civé la pelle ) dei Duca per haver egli commesso un delitto nel
nella, rompendogli quella della mano, € seguita lo (cherzo dicendo, «
morire in /« tre degus  [ che vuol dire in sule forche ] perché con un
col mestolino J fece la decta ferita nella mano di Bertinella; e di più
Ballerino a vento (che vuol dire ballerino da qulla ) per mostrare:
egli commefio ' errore bailando, farebbe gastigaco con esser fatto mori
do, come pare che muoia colui che è impiceato, Vedi sopra\C, 2, st
re un ballo in campo aczurro; che è lo stesso, che Tirar de' calci a Ronaio,
vento Borea, o Tramontano, Quel che sopra dice: in /u tre legni per i

forche; è simile a quel di Plauto, che volemdo intender Far, cioè ladro, disse
trium literarum homo, vel
FACENDO il Nanni. Facendo il goffo. Fingendo di non badare, oofferm
re,Vedi sopra \cstan[4]{26}. Mostrando di non s'accorger di quel che faceva Bal-
done, facendo le viste di non vedere. *
SCOPPLA dalle rifa, Ride secgolatamente. Vedi \cstan[3]{66}, alla yo
Pimmei, e \cstan[7]{66}. 0

\&
Been SSB SRS Ewe ow o=

PER l'allegrezza non può fear nei panni. Si rallegra geandemente.
capir nella pelle. Per il gran gusto si rallegra tanto, che non trova qui
di sopra C, 2. @aa, 69, Piatone acl Carmide, poco dopo al principio, volend
esprimere una gran paiione di piacere, e di gioia fa dire a Socrate, \&
più in me flefo. i o ae

cANDARE in fumo ad acquauite, Risolucre in nvila. Suanire. Lat, 4
re. Sidice anche in tu:no d' elisire, Od' eferuite, sopra C, 3. han. 52.

STANZA LXVL STANZA LXVIL-
Atentre Baldon qual fempluerto uccello, 4a ridan pure, e faccian ci
Coast d! tntorno alla cinetta armeegia Per ch' ci vuol far orecehieds Merci,

Lo burtino te genti, Amor ta,
C” ad ogni mo farò fido, e
Come talor 3! abbrucia ico
“i garto al fuoco, e frau

ed tuti guint serve per zimbello,
Senzache mai vi badi, o sen' annecgia
Ogun lo burla, e dice; Pelle vello;
Crafexn dice la sua,ciascun motteggia,, «

Beato chi pu bella te la feianta Baldon già fenve tl fuocose
E pot leuansi crosci dell ottanta, Aa com un pan di ”
; STANZA LXVIH. 6

ne ot See wa

E cos} wa,per ca principio Amore, Ma nel getrarla allor
Par bella cosa, e sembra giusto giusto Perché riftringe, e ride
Vira pera cotogna, il cui colore, Ecosi Amor, al primae ani
Odor » saper aslesia, e piace al gxfboy C! allerta, e piace 54



lal

aa,t
ie
ib,

NONO CANTARE. 459

STANZA LXIX,

Ed agli cht impaniato, € 4 qualche segno
Credeil suo amor da lei esser gradito,
Altero vanne, e fhima a' esser degno,
D'invidia più che d'esser mostro a dito,
ta lasciamla per hor cbt io fo disegno,
Che questo canto resti qui finito,
Perché dife un Dottor da Palestrina
Breuis oratio penetra in cantina.

. era così fit di la, che faceva mille me-
lenfaggini, per le quali era da ogauno burlato, ed egli Fingeva di non se n' ac-

c » o continovava a fare (cioccherie ostiaato in quell' Amore, come tal

volta @ un gatto ostinato a stare intorno al fuoco, ancorché si feata abbruciare.

4 Poeta adsauglia Amore alle pere cotogne, le quali dilettano con l'odore, col

colore, ¢daano gusto nel mangiarle, ma si dura poi fatica a digerirle, € diven-

do che Baldone si reputava più degno d' esser invidiato, che compatito, termina

il nono Cantare.
| CWETT A, Vedi sopra in \cst{22}.

SERVE per zimbello. Servc per scherzo di tutti. O pure per allettatore degli
altriamanti a venire ad amar la sia Dama. Ii Malatesti parlando in persona d
un villano mandato d' oggi in domani, e burlato dalla sua Dama, disse;
' Da poi ch io ho fernito per zimbello,

E son andato trenta mesi aiont

Gridando per la rabbia', e pel ronello
vibr od Come fa il gatte quando ha i pedignoni
ud id « Alla mia Betta ho pur dato? anello, ec, 7
DICE: vello vello, Termine, che fenifica Derisione', quasi dica; guarda, >
guarda lo feiocco, il pazzo,o simili, ed è lo stesso che Esser mostrato a dito per de-
rifione, che vedremo appresso nell'ottava 69. e che far lima lima dittro a uno vi-
sto sopra \cstan[3]{37}.

MOTT EGGIARE. Burlare, o beffare copertamente uno con detti acuti, e+
mordaci. I Greci di C diare uno; noi p biarlo, egiarlo, Da
motto, parla; che si piglia anche dagli antichi per sentenza, o concetto, o det-
to intero; B Azorsetto, cine breve detto, e sentenziofo, come son quelli intitolati
Motterti ne' documenti d' amore di mefler Francesco da Barberino. Asutire, loqui
disse Sesto, foggiugnendo l'autorita d' Ennio nel Drama intitolato Telefo. 2a.
am missive piebero piaculum est, EB ttimato un delitto a ud plebeo il far motto,cioè
aprir bocea, e parlare: onde Azertegesare non è altro, che parlare con qualche.
bel dettoy @acuto. Dal Greco Azythos viene il Latino murire, €'| noltro Adorze,
Ui Casa)però nel Galateo col definire i Motti /pectal pronrezza, e leggiadria, ed
ostano movimento a' animo; pare che in un certo modo lo faccia venire, O pure
scherzaquafi, che venga da A4oto, movimento.

BEAT O chi più belo te ta frranta, 'BE' lodaco colui, che la dice più bella in bef-
famento di Baldone; ci serviamo dell' cpiteto bearo per felice, avventurato,
fortunato.,'efimili (come se ne serve il Poeta anche sopra C. 1. st. 29. come nel
presente:luogo-, cheesprime, Fanno a gara a chi più bene lo burla: Latino Cer-

sare conuitijs ) Petr. i
NED Beato venir men che 'n lor (Rhos:
Me più caro il morir, che viner fenxa;
= Mmm 2 Le
\& ca

q



460 MALMANT ELE:

LEV AN crosci dell! ottanta., Si ride fmoderatamente. La vt
quel bollore gagliardo, che fa la pentola, Fee era 9 Op
€ si dice croscrare dal suono +. ik gal verbo
Dan, Inf. C. 24,

O giustizia di Dio quant at
Che carai colpi per venderta crefeia oi

Tl termine dedil otrawta significa squifitezza., o p
ne logico a to: o forse dalle, ralce specie dipannine; le quali
tanta paiole sono a buonidimo grado di perfezione 50 finezza..

Ck ALECC!, o cicalices. Dilcorsi faytida più persone insieme
priamente dire Discorsi dell! azioni, ed snteredi altrui con.
di bene: éd intended per lo più, Cigalamenti fatti dadonn
digiorni, novellieri; per questo quando si sente Ree nuova
dice € un cicaleccio, o una cicalata. >

FARK orecchie di mercante. Finger di non ascoltare 2 o nan.
che altri ti discorra. E propriamente s'intende far oregchie di mercante coll,
che essendo richiesto di qualcosa, o ripreso d' 4leun vizio non
richieste, o non si emenda agli avvertimenti, o riprensioni... Si dice piantare me
vyna lopra C. 7. It. 39. Far conto, chee passi l. Leperadore. ne to. si

COSTERECCT, intendi le Costole: Li costato..

EVN certo imbroglio, E' un certo negozio imbrogliato, is difficile, cele
mo anche ana cosa così fatta, intendendo una cosa. che pon ha eo del banat
del giusto, dell' onesto o del fattibile. ons,

WEL gettarla, Dicono, che la pera cotogna viloinga il venton-a coed stil
mangia, e lo rifecchi rendendolo stiticho, e però dive;Vel.gerranla da dolore se
più lotto dice; Nel fine ti vogtio, nello smaitirla si man. in fuori
mu dica le ti riesce così di gusto come pel principio s:cioèiquando lama

41d impaniato, E' rimatto preso alla pania, come rimane-il pettiroflo
do la Civetta, intende s' è innamorato 4moris yorte dmplicitus, aK or
parazione, che ha fatta sopra dicendo,
etientre Baldon qual, semplicerta angela 2

". Così d intorna alla Civetta armecoia..
Quando uno ha male grave, da non ne potere ( non iisimene err
dichiamo; £g/i ha impaniato, eq o¢ eam

ALTERO vanne, Vedi sopra C. 8. st. 30, Qui-vuol dire gout,
mando, che questo amore lo renda degno d' eGere invidiato per haver
bene, come stima l'amore.di Bertinella, che d' eles ¢ompatito del
d' cllersi innamorato di costei. B così si da.a,credere digodere ogni
sapendo, che come disse Erodoto nel libro intjtolaca, Talia 5-2 meglio
diato, che compatito; la quale sentenza colle essi parole appunta, a
fa l'usò Erodoto, dichiamo noi comunemence tutto giorno; E.chee ji: ue
ce Pindaro nella Raccolta morale dello Stobea eHMiglian Minvidiak F,
le quali sentenze dalla nostra plebe ridotte in una Cantilena Fiore
Così e sa sincoomate

Meglio e invidia fop| tare h

Che di se compajfion dare,



NONO CANTARE. 451

 DOTTOR! di Palestrina, Se ioffapeti, che Catone havesse detto. Brevis ora

Caios crederei, 'che volesse dir di lui, perché fu originario di Tusculo,

di Prafeai »eche havette pigliato Palefrina, cioè l'antico Prenelte per Fra-

7 € S'i0" fapeti » che un montambanco, il quale si faceva chiamare il Dotto-
redi Palestrina, e faceva da Attrologo fusse solito dire tal sentenza, stimerci, che
ee questo, Ma intenda di chi egli vuole, basta che con questa fencenza
dai opps ha voluto significare, che i difeort brevi piacciono inating ai

2 icantinieri, ( perché ne' suoi Originali trovo una volta im excints,

'ra volta i in cantina ) ed in sustanza intende, che ancora gi' idioti amano, e>

ei eee idiscorsi brevi.
fo
ime i

nt FINE DEL NONO CANTARE.

DECIMO CANTARE,
Peeabasdlasibabastiasdbarls 8

ARGOMENTO,
Per far la Adaga col Rival quistione
Va, ma in vederlo pot le spalie volta,
E, con lui dietro,
Ove e la gente per balare accolta,
Del Lupo in traccia Paride si pone,
Ui trova,e'l prende con induftria molta, we

ugge nel falone,

i E uccifo quel, da fine alf avventura,

STANZA I.
wanti ci stan. che vestono armatura
0° Dartor di feberme, e ingoiator di fquole
4 ditminedaces » che fanno altrui paura,
o Premar la Terrase [paventare tl Sole;
' o BE ratcontande ognor qualche branura
f

o

Sempre ogn'un cone parole;
St fda sl caso di venire all' ergo,
Labial om! olia, poi voltano «| tergo.

STANZA

tpien mostra in zucca haver del Sale,
hb ee jon [fanio sempre fugge ta guistione,
Anxi veder facendo quanto ei vale
odMebpicare al bisogna di spadone,

| Ed wu tal guisa è liberate il Tura,

| pene Reps ep pe geste eer a

we

STANZA II,

Mae son da compatir fee fanno errore,
benché non sembri mancamento questo,
Se chi 4 menar le man nonglidailcuore
In quel cambio a menare è piedi è leo,
Ob mi direte: Vanne del tuo bonore

Si, ma un po di vergogna pala presto,
Helio è dir: Un Poltron qui si fugvi,
ee: qui fermofi un bravo, e si mori.

L,

E che ( chi a nessun vorria far male )
Sa ritirarsi dall'occasione,

E Senza pagar tafteso chi lo medichi
La campo, che ai ni re se Fee

«dh theme

i eee}


462 MALMANTILE.

STANZA TV.
Ma voi, che di question fate bottega
Credendo immortalarvi; e che vi giova
Far la spada ogni di com! una fega, imparate
E porni a rischise far ogni gran prova, eg
Il nostro Poeta volendo deferivere nel presente Cantare la di
lagrillo a Martinazza, per la paura, e poltroneria della
segui, s' introduce con dire, che quei Bravazzoni,ed Amm
pre discorrono di far rissle, e quistioni, quando si vien poi ai
ratamente, e loda il lor pensiero, contiderando, che 1
la vita, che far fermo, ed esser' ammazzato per il vano pretesto di rij
eche non può esser biafimato colui, che non havendo cuore a menar |
mena in quel. cambio i piedi, e fa intanto un' azione degna di lode, fug;
male. Conchiude al fine, che tali bravi, che cercano d*immortalarai
ro bravure, e smargiafferie s' ingannino, perché dopo la lor morte:
ur minima menzione di loro: Giù esorta pero ad imparare da i
DOTTORI di scherme, e Ingoiatori di (quole.. Cioè che fanno da mae!
ma, e che si prefumono di saper tenere in mano la spada meglio di chi
da nelle squole di scherma. Ma qui scherzando.con l'equivoco di (quola'
che cofioro son bravi mangiatori, poiché ingetano /e /axole, che fo
ne fatto di farina mescolata con anici, ed € chiamato squola,
figura d' uno strumento, col quale si tese, detto corrottamente /guola
dixs, come vuole il Ferrari; ed è quella cassetta fatta a foggia di na
ro chiamata anche navicella)entro alla quale s' adatta il cannello pieno dil
passarlo a riempier l'ordito: Si dovrebbe dire (paola, ma l'uso ha
la notizia di tal voce. Dan. Inf. C. 20,
Vedi le triffe, che lasciaron U ago
La (puola, e il fuso, e fecersi indovine,
E nel Purgatorio Can. 31.
E, tirandosi me dietro, fen giva.
Sour! esso ? acgua liene come pola. ?
FANTONSACC!/, Huomaccioni; Huomini di statura grande; ma dicendol
Fantonacei §' intende in un certo modo grardi, e poleroni,o difutili. B dict:
Galeonaces, @Uanizoldacei, ec, Omero nell' Liiade lib, 3. introduce Extore,
del male a Paride suo fratello. £ tra gli altri mali, che gli dice, unoedi
marlo, Eidos ariffe, cioè un bel fantone, d'ottime fattezze; o come meer
significando la bellezza del corpo,disgiunta daila virtù dell' animo;un
un Dongelione, o come dice qui il noitro Poeta; un Fantonaccio, cio? che!
mostra, ma e poco buono a auila, *
AMMAZLAR con le parole, Legiones difflare spiritu,come disse Pl
dato millantatore. Pretender di farsi stimare, e temere col dilcorrer
ritie, quistioni, ammazzamenti, e con esercitar sempre con chi fil
arrogante superiorita. Di-questi parla Famiano Strada Jib, 2, Pro
Gloriofi isti duces. Det homsnumque contemprores, \& gut se atijs faci
Calo minitabundi gre 'p ATLis, Guam profil d 08

DR Pweg er ge ep roeae: =. wa

Sener



DJECIMO CANTARE: 493.
tini chiamano milites gloriofos, questi vantatori poltroni, de i quali intende il Poe.
ta nel presente luogo, e se ne dichiara col dire: Se view mas il ca/o di venire all'ergo,
~ ifica, se vien mai il caso d' haver ad adoprar l'armi, non parlano più, ¢
fuggono, che € quell' abijcere Clypexm de i Latini.

VN poco di vergogna passa presto. Quel poco di roflore, che si ha per una cosa
mal fatta fuanisce, essi disperde: Seatenza usata, e praticata da coloro,
che fanno poca stima della riputazione.

(i MEG LIO e dire: Vin Poltron qui si fuggi, ec. Buona sentenza, e vera, e prati+
jig cata da coloro, che bramano pe tosto vivere con poca riputazione, che glorio-

gi famente morite; il che bene esprime il detto Latino Vir fugiens denuo pugnabit.

m Der, che s'era srmato, ed havea fatto (Crivere nel suo scudo a caratteri
iamt d' oro BON FORT VN\& vantandosi di voler-far gran bravure, se egli entra
è,g Va in guerra; quando si venne al combattere, buttd via lo (cudo, e si fuggi, ed
misit a. coloro, che lo taflavano poi di codardo disse: Vir qui fugie, ruxfus redinregra-
nme bit pralium, indicans ueilins Patria fugere, quam pralio mori, mortuus enim non pi.
sen grat (che noi diciamo: / morti non fan pin guerra; ) at qui falurem quefiuit in fuga,
poet pote/? sm multis pralijs patria u/ui efe. Tuttavia anche appresso gli Antichi era vitu-
dda Peroso questo tuggire; e si trova, che 1 Lacedemoni bandirono Archiloco sola-
digi mente, perché havea scritto, che era meglio abijcere clypeum, quam interire,
jue a del fale im <ueca, Kaver giudizio. Vedi sopra C. 4. st. 15. e C. 8. st,
wi, o CHOCAR di spadone, Par che voglia dire, che questo tale si difenda con gio-
jgad care di spadone a due mani, ma intende, che gioca di spadone a due gambe.,
yal Slot fugge: motteggiamento usatissimo verlo coloro, che fuggono per paura il
ie dite sinora ben di /padone, e \enza dite a due gambe s' intende fuggi. Vedi sopra

| C, 7.0.76. Giuocar di spadone si usa ancora di dire in proposito d' una casa, che

sia igauda, e (pogliata di maflerizie; in questa maniera. Vi si può giuocare di [pa-
done, ciaé Non vi e cosa alcuna, che possa arrestare, o impedire questo esercizio,
che ha bisogno di iuogo largo, e difimbarazzato.
TaSTE, Vedi sopra C. 1. st 60. Talte fila, che si mettono nelle ferite, dette
così dal taflare, che fanno la lunghezza, e larghezza di quelle. Latini panicidi
ai Vulnerary, lineamenta, i
g DAR campo, che si predichi di ivi, Dac' occasione, che si discorra di lui con
wm) lode. £1 ver! predicare usato in questi termini figaifica Far' cn:omj, o lodare,
| Quand' uno fa qualche azione bella, e di cia si pavoneggia, (ogliamo dire in de-
Be 2 Chese ne predich,
PAR botreca di quistioni, Viuer di risse. Haver care le risse per guadagnares.
E tanto questo detto quanto far da spada come una fega, cioè intaccaria nel far qui-
fione, come è intaccata, o denotata una fega ) sono detti deriforj a tali Bravaz-
zoni, e Tagliacantoni.
LA morte vi si piega, Voi morite, e dopo la vostra morte non si discorre più
de! vostri gran fatti, e si perde la unemoria delle voitre azioni » e vanne del pari

la bravura, e la codardia » Quell' importuno, che per la via facra s'avvid dictro

a Orazio, enon lo voleva lasciare; domandatorda lui, se avava netiuno de'fuoi,

che  aspettassero a caia; pee maggior suo dolore gli rilpose: Omues compo/ui,(a-

no accomodati, la morte gli ha ripicgaui tucci, Sa) th ee SUN



44

Colei c ha fatto buio
Paga di sogni i debiti a ciafiuno, .  (Benche si
Quella, che dianzi tolfe al di la vitay Per fuggir
Cagion, che tutto il mondo porta ae Comincea a
Descrive con vaga maniera in quest' Ottava V apparir
con equivoci; uae far buio vuol dit Consumar tutto il suo Sed '
tendedo della notte)vuol dire ha oscurato: e se ha confamato | !
¢ fallita » e non prod pagare i suoi debitife non con i
ricca se non di sogni;e pagar di fogns vuol dir pagar di moneta
non pagare, Vedi sopra C. 2. st. 7. fugge dunque la notte per
giona non solamente, perché è fallita, ma ancora i ella te
sia fatta la spia, che ella poco diana. uccife il giorno perché la
oscurita uccide il giorno ) per la qual morte tutto i) mondopi ee
dir, che per tutto il mondo la notte e buio, enter: bruno, e €0
te di gualche nostro conginen i se bene ella non dovrebbe temere di tal!
zione, perché Si chinde gli ocehi a, che fgets on off.
re, finger di non sapere; e il eos connivere., Vedi sopra C, 6.8 vit
vuol dire che si chiudano effettivameate gli occhi, perché og ne
fuggir I iba c' ha le calze gialle, per fuggix V Alba, A e spia del gi
che ha le calze gialle, perché il primo albore del giorno è i colore frail
€ giallo, e così s' accomoda all' equivoco delle calze gialle shee
ze il contraflegno delle spie, o de i toccatori come accenn sopra C
stan. 60. 03 99g
COMINCTA a ragionar dt far le balle + Comincia a ragionare, o r
partenza, che questo intendjamo quando diciamo: 4 rale fa te balle

fa colligar:
TANZA VI, STANZA VII
E denna » che di quei balletti Jf astidita poi da ranto fran) +
Sarebbe in corte tutto il condimento, Suvi mulinelli, forge cL

Ler ch' in un tempo fol con i calcerti £ data nna Seofferra come i fae
Ballandosuona al par d' ogni strumento, La laciachiede, britdospi 7
Lupo cena per degni suoi risperti Perché il mmico all? alba de' Ta
Prefe dag altri un canto in pagamento, Vuol trucidare in singolar
E sopra un pagliericcio angusto, e fod Ed a fargli servixio, più
Fino ad hora s'è cotta nel | suo brodo. Vuol ee
STANZA VIL. ANZ

Pero che wel pensar che la mattina i vi intrepid
Entrar in campo dee alla tenzone, Espaccia il Baiardino, eit
Fa ginfto, come quella Nocentina, Chi la fringesse,
C's giorno andar douendo a processione, Pagherebbe qualcosa ay

Occhio non chinde, e tuttania mulina, Ma tutto questo

(ZRF, BORED ER RESEP RL Bae. ELLER ew eee

Tanto che ud capoell' bacome uncestone; La faccia tosta 7
Così la Strega in cella solitaria Sperando
eAtrende afar mille caspelli in aria, Chie! non fen,

101 Sig



DECIMOCANTARE. 465

-'Martinazza, che farebbe stata la perfezione di quella veglia, se ne ritiro in
camera, e possafi in sul letto stava pensando alla battaglia, che doveva fare con
jagrillo, ed alla fine, se ben veramente non farebbe voluta andare a combat-
ere, finge coraggio per non esser cae codarda, ed in sul far del giorno chie-
le sue armi, (perando pure, che habbia a succeder qualcosa; che impedilca, ¢
® sia causa che non segua il detto duello.

SAREBBE fata ii condimento, Cioè Carebbe stata la perfezione di quei bali,
 di quell' allegria. Così quando sopraggiugne qualche persona gradita in una con-
" jone, si dice per ilcherzo, Venir ella, come il cacto fu maccheroni, come lo
- xuechero in fusse fragote, o fusse vinande; valendo con queste batie similitudini si-
gaificare ciò che più nobilmente si direbbe. Essere ella il condimento della con-
tm ucriazione, e non vi mancare altro per renderla gustofa, faporita, e perfera.
hued SVON-A al par d' ogni strumento, Ghediio vogliamo dir copertamente, che una
wet cosa pute diciamo: La talcofa suona, Vedi sopra \cstan[]{49}, ed il Poeta cava
da ciò lo scherzo dell' equivoco, mostrando di dire che Martinazza suoni d' ogni
mit strumento, ed intende che le putano assai i piedi, poiche dice, che ella /uona co'
'mj ¢alcetti, che sono scarpini di panno lino, che si portano in piedi in su la carne fot-
shay to te calze; e si dicono cascerti ancora quelle scarpe di quoio forcile, senza suolo,
gum ma con la fola piantella, che usano i ballerini, e che usavano già l¢ nostre donne
ga di portare sopr' alla calza quando portavano le pantofole.
ott  PIGLIAR un canto in Pagamento. Significa Andarsene. I debitori, che volen-
rag ticri (cantonano i suoi creditori,si dicono dare un canto in pagamento,cioè fug-
gigi gite il creditore per non pagarlo, e per non avere occasione di trattare con lui:
|. PAGLIERICCIO. E quel gran sacco pieno di paglia, che usiamo tenere in su
gig Fletti sotto le materasse, detto anche saccone.
wt ~~ 8° 6 cotta nel suo brodo, Non ha havuto veruno d' attorno. Quando alcuno f2
: qualche risoluzione, che non è approvata, o non piace agli altri, e non è da ve-
yi tuno in quella seguitato diciamo; E /* quocerd nel /no brodo, cice senza che altri
vi a \& nulla del suo; o vero Farò come gli (pinaci, e s' intende che si quo-
cono ir brodo
già FA come quella Nocentina, Nello Spedale deg!*Innocenti di Firenze (che è quel
“4 nel guale s' allevano i nati per lo più di copula iliecita, si come accennam-
i Te sopra \cstan[1]{85}. ) stanno riferrate molte Fanciulle, che noi chiamiamo
a Mocentine le quali non escon fuori se non una voita ! anno, che è la mattina,
, della vigilia di San Gio: Batista, che vanno per la Città procethionalmente; e
Pe ciascuna di loro ha gran desiderio di far tal gita, non vi € aubbio, che
f speranza d' haver a godere si bramata foduistazione, fa, che pare a' ciascuna
 mill' anni, che venga il giorno, e che per tal pensiero poco derma la notte avan.
 £1, rivoltando per la mente wweti li modi di comparire atullata, e bene all' ordi-
ne; il che è causa, che la mattina ella ha poi un capo c me un ceffone, cice grof-
o €pieno di confusicni per haver poco dormito, ed affaticaia la mente in quei
Pensieri; € queste son quelle, alle quali il Poeta assomiglia Martinazza.
MVLINARE + Pensare; Disegnare, andar vagando con la immaginazione;
che diciamo anche: Ghiribizzare. Vedi sopra \cstan[]{56}. Viene dal Latino
molior » che vuol dir wacchinare,O ne dal volgare Aduino, quali girare coi pen-
aa ficro

te


466 MALMAN TILLEY

ficro come un mulino. Virg, disse spedissimo +| Corde:
che fanno le persone innamorate peulando fidamen
giamente ne diede la descrizione in Didone,
Multa viri virtus animo, multufque
Gentis honos  barent infixs pe'tore vultu
Verbaque, nec placidam membris dat
Tutta la notte va mulinando « E lo stesso, chevaculer. Ho
Quid brexi 'fortes iaculamur auo
multa ?
E' detto ballo scagliarsi col pensiero ora in una cosa ora, inua)
Mattio Franzesi acl Capitolo delle Nuove,
Lasciamo aftroiegare a chi indovina
Per wie di conetiure, e di difeorsi,
E col vernel fantaitica, e mulinay.

HLA il capo come un cefeone. Gli si confonde ik cerucilo, Pai p
do diciamo fa ii capo grosso, 9 se gli ingrossa il capo, intendiamo
de il giudizio: EB Cefone \& un gran paniere fatto di vinciglic dt
te, ed \& capace di mezza (oma, e perché ha la figura a:
queta comparazione. vil

CAST ELLO in aria, Pensieri senza fondamento, ed affegnamenti
nt, e che non poslono riuscire. Laili Ha, Tr. C. 2. st, 2470 ADA

Fra me facea mille Caffelli im aria ode
Aristofane intitola una sua Commedia, in cui. si burla di
Nuuole; e lo fa falire, e passeggiare in aria.y per mostrate, pr !
vana, e senza fondamento la (un filofofia. Noi quando vogliamo dire!
badare a' discorsi terij, e avere il capo altrove, e a bagatelle; Dichiamo i
fare a' nuuoli, (e non vuol dire più toflo in lingua Lanadattica: Pensare a milla.

MVLINELLO. E uno strumenco di ferro, che serve per sollevar peli
derivandojo dal verbo malinare detto sopra significa inucnzioni,
ne, disegni, ec,

DATA una scofetta come i cani, S intende, che Martinazza I
veilita, e levandosi dal paglicriccio, fece come fanno 1 Cani, quando,
no, che per lo più si fquotono. A

ALBA de' Tafani, Si dice quell' ora del giorno, che il Bolee
re vigore, nella qual' ora i Tafani sono più vivaci, Tafano. Lati
un verme volatile simile alla vespa nel colore, e nella figura; ma
assai maggiore, ed ha ancor' egli un' acuto pungiglione 5, ficche
de' Tafani s' intende leyarsi di la da mezzo giorno wi) |) say Pr
PAR vegnia uno, Far cortefie, o carcazeasuno, an iL
no affetcate, si dicoao /ezzi, quali iddicia'y o intedtus » come k i
sca Novella 10. Serallegro con Nencio [poso della Ragaread, 6 \&
bene, e le facesse verxi. Col dire.  farls servizia, e pin chee
orecclu sieno i maggiar pezzé,intende, che Martinazza gli fara g
tarlo in pezzi così minuti, che un' orecchio intero sia: 1 mag)
trovi del suo corpo »detto acim per suena un



a
om

DECIMO CANTARE 467

'SP ACCTA il Baiardino, e il Rodomonre, Si fa Mimar bravo, come favoleggia,
' Ariosto, che fusse il Cavallo di Rinaldo Paladino appellato Baiardo, € quel
¢ Saracino detto Rodomonte. Può anche essere, che far il Baiardino, signifi-

chi far il bravo da un tal Pietro Terraglio soprannomniato Baiardo, che fr un
soldato di-valore, e d'inufitate forze, il quale mori forro Milano militando al

-servizio del Re Francesco di Francia, come narra il Varchi Stor. Fior. lib, 2,
| CHI la fringesse fra uscio, ¢' mare, Chi l efaminatie bene; chi glielo do-

mandasse da solo a solo,

segua, o non vada la posta, o l'invito
tutte le cose, che intenzionate, non s' ¢!

 PAGHEREBBE quatcofa a farne monte, Spenderebbe qualcosa a non far questo
“duello. in ructi i giuochi si dice far monte, quando si reita d' accordo, che non
roposto; e questo e fatto poi comune a.

i(cono: per esempio / tal matrimonio,

he era già conchinfo', ando poi 4 monte, cioè non si stabili. lo voleva andare a Ro.

with
joie

ma, ma poi ne feci monte, cio non andai.

IN se tien duro, Lo tien segreto in se. Non si confida con veruno,

FA factia tosta,, La faccia fol' esser dimostratice delle interne paffioni; e pe-
ydiciamo; / rale fa faccia toa, intendiamo il tale si sforza di non sco-

iia prit co mutamenti del voito 1 suoi segreti, essendone richieflo, ¢-di non confet-

waco

a

giù
:
3
eo
è
:

i

:

“STANZA X.
Spada, e lancia fra taro un Servo apprefia

i Col perto.a borta in man Laltro galoppa,

'a altro o elmo da coprir la testa
Da distder unalcro, e braccia, e groppa,
Di che coperta in ricca sopranuesta
Par un pulcin rinvolto nella stoppa,
Ed allestica in sul cantar del gallo
eitro quivi non resta, che il Canalo

fare itdelino »essendone claminato. Latino frontem perfricit,

STANZA XL

Percio fa comandare a i Barbereschi,
Che lo menin n' un campo di gramigna
Accioech'ei pasca un poco, e si rinfreschi
Perché per altro il poverm digriena.
La marca hebbe del Reeno,es enidalescbi
Gis hanno rifatta quella di Sardigna,
Maglie, e reti ha negli occhi,ode per cena
Vanne a pescar nel lago di Bolfena

B servi di Martinazza le portano l'armi, delle quali armatasi, ordina, che le
sia condotto:il Cavallo, quale il Poeta de(crive per una solennissima Carogna.
“GALOPP A, Cioè Corre', Verbo usato in questo significato,ma però impro-

prio, perché galoppare, o gualappare \& specie di correr di Cavallo; la qual voce
concorrono gli eruditi a farla venire dal Greco calpareia,

GROPPA, Si dice la parte di dietro del cavallo, o simile animale, ma qui in-

tende la schiena di Martinazza.

PARE un pajein rinnolto nella stoppa, Quando si vede uno, che non fa portare

l'abito in dotfo, e che pare impastoiato nel camminare per causa deg!i abbiglia-
menti, che had' attorno, l'assomigliamo a un pulcino, o pollastrello rinvolto
nella stoppa; e non siamo is ciò dissimili dai Latini, che in questo proposito
didero. Herer ranguam mus in pice.

SVL cantar del gato, All' apparir del giorno, che a talora fogliano per Io più

cantare i Galli Vedi sotto C. rr. st. 5. Orazio.

etd galli cantum con fultor ubi ostia pulfar,
BARBERESC HI, Intende gli Stalioni; se bene Sarbere/chi chiamiamo coloro,
N ai

on 2 i quali

Bikes e, 4


468; MALMANTILE —

t quali cvflodi(cono, e gevernano i Cavalli Barbari, ¢
Poeta gli chianya così per derisione del Cavallo di Martina
Firenze 1 Cavaili, che corrono a i palj della Città, ton
frica, che noi chiamiamo Barberia,
CRAMIGNA, Erba nota buona per pascolo degli Asini più
li, ma a quelio di Martinazza non par poco haver di questa,
zerin digrigna, clue s¢ nou havesse di questa, non havrebbe.
ci serviamo del verbo digrignare per intendere flentar per la fa
nare, e acrocare i denti per non hauer altro, in che ado
canl, ec. che si dice digrigware, quando per la rabbia
Tat Cas.
x Non vedi tu, che digrignano i denti
Econ le ciglia ne minacctan anoli?
Ed egliame: Non vno, che cu paventi,
Lascsagli digrignar pure 4 lor senno,,
MARCA, Contraflegno. Es' intende quel fegao, che hannoi

li, o di razza in una coscia, o nel collo, perché da essi si possa
razza sono. Virg. 3. Georg. Continuoque notas, nomina gentis inurunt,
che questo Destriero di Martinazza havea già la Marca del Regno di
sono oggi i migliori) ma che i guidade/chi gue n' haveano mutata in
digna, € non intende dell' Liola di Sardigna, ma di quel luogo fuori
Firenze, dove si scorticano le bestie morte detto la Sardigna, came
pra C, 1. st. 2g., ed intende, che questo Cavallo per li guidaleichi, ed
fetti, che haveva, era buono a mandare in Sardigna allo Scorticatoio
te/co diciamo ogni scorticatura fatta alle Bestie dalle selie, balti, o altro. Mau

Franzesi descrivendo un cavallo fintile a questo disse; wig
Dinanzi ei non è 21d troppo gagliardo; “iy
Ma in sa la scbiena ha qualche curdalescho,
E le spronate mostran, ch' e infingardo, ™

MAGLIE, e veri, Così chiamiamo alcuni mancamenti, che vengono si
occhi alle bestie; ed il Poeta servendosi dell' equivoco dice, che con 'quelle ra
può andar a pescare nel Lage dé Bolfena; ed intende, che il cavallo-era bof
dicemmo sopra C, 3. st. 53. » che cosa sia. E così sotto questi equivoci iroa

mente loda il-Cavailo di Martinazza. sagt
STANZA XIl. STANZA XIIL
Hor mentre pajce 1 misero animale, E ti faluta, e tt si raccomanda, —
Eche si fala cerca aclla fella, E per cha inteso, che rm fai duclly
Giunge un Diavol più ner aet caviale Un rotelion di fughero ti manda,
Con un marteile in mano, e una rorella, Spada non già,ma ben gnejto:
Ed un liquor botiente ix un pitale Con una potentissima benanday —
'Ed inchinato a lei cos favella: Ch' 10 ti prefemo emr'a
I Re dell' Infernal Diavoleria Bell! e caiduceta come la.

Con queste trescherelle a te m' innia, edilo [pedal si ad ta medicinal
;: aie



DECIMO CANTARE. 469

. STANZA XIV. STANZA XVI.

Hor fenrj; che qui batte sl fondamento Ma se per non haver buon corridore'
Quand' ih nimico ti verra a ferire Quivi a canfares tu non fulfe leffA,
Va pure innanzi, e non haver spavento, O per altra disgrazia; o per errore
el ferro questa targa a offerire, Ei r'appoggiassi qualche calpo in tej 5

 E tuffo ch' ei la passa per di drento, Vorlio, che tu per sicurtà maggiore

» Sia presto col martello a ribadire, Hor per allor4 ti tracanni quests,

Ma lasciagnene subito alla spada Quale e una bevanda sh squifita,
Peich'egli a se tirando, tu non cada, Che chi Lha in corpo no pua uscir di vita.
Ni STANZA. XV. STANZA xVIL
Face  egli poi con essa quanto vuole Così le fa rngoiar tanty dt micca
| Che pix di punta non può farts offesa, D! una colla renace di tal forte,
» Di taglio manco, essendo c' una male Che dove per fortuna ella si scca
Si fata a maneggiar pur troppo pela; wl mondo non è presa la più forte;
Portila dunque per ombrello al Sole, Luesta ( die' egli) Uanima t appicca
» Pere alia resta non gis muona scefa Ben ben col corpo, e s'aitre non è morte
 Edigli( già che queila non è il case) o? una fepararion di quests Aussi,
Che  egli ti vuol dar, ti dia di naso, Oxgi timor non hai de' fasti [uci.

» che Martinazza aspetta il suo Cavallo riceve un regalo da Plutone. 5
confistente in armi jd in. una bevanda per difendersi dalle ferite, e dalla morte,
Nota che in questo bel regalo il Poeta immita coloro, che hanno scritto le pro-

 dezze d? Amadis di Gaula, ed altri Romanzatori, i quait, quando il loro Erce
dee esporsi a qualche battaglia pericolofa, fanno sempre, che qualche Mago
“amico di esso Eroe io mandi a regaiare d' armi incantate, o altri difenfivi, ed
inttruziom, '

St fata cerca della fella. Si ta cercando della fella, Dice così per mostrar, che
ae era tanto iniolito ad adoprar la (ella, che non si lapeva più dov'
ella fusse, >

PIÙ ALE. Alberello, o vaso di terra, come dichiara il medesimo Autore nell'
Ottava seguente dicendo; ch io ti presento entr' a questo aiberelio, Se ben Pitale \&
Piopriamente quel va, che si mette centro alle predelle con altro nome detto
tantero..L' uno, e il altro nome dai Greco, quello da Pitharion, piccol valo di
terra, doioiwm; quetio da Cantharos voce usaca anche da' Latini. o significa un
vao lungo, e stretto in fondo. E con manichi, quale e queilo, che si vede cal-
volta figurato in mano a Bacco.

« TRESCHERELLE, Lato trice, Bagatelle; Coferelle di poco prezzo, Ve-
di sotto in questo C, f, 28.

SVGHERO. Pianta aota simile alla Quercia, e fa le ghiande ferotine, e la
faa leggierissima (corza serve per far lavort da refiitere all' acqua, come farebbe
caiietce per mettervi bomboie di vetro piene di vino, o d' altro per diacciare.

 BELL e calduccia,, Temperatamente calda; e come si da la medicina, che intea-
diamo bevanda folutiva. Vedi sopra C, 8. tt. 25.

CHUVOVERE fiefa, Fer venire l infreddatura. Scefa diciamo una distillazione,
o catarro, che dalla testa casca nell'altre membra per causa del freddo.

Tl dia di naso. Detto iporco usatissimo nella Picbaglia in segno di disprezzo, e

sin-

————————————< =

i

se

472 MALAANTILES @
s intende di nafoine,... che per ricoprire si dice 6
serve = esprimere la poca stima, che si fa della ——

NON fussi lefiaa canfarci. Noa fai presta a-fuggirli,.
Effugere, delinearesy nes lisdab Greco compre ara
detto così quali CG x F

TRACANNI, they bevay logolli i

TANT Adi mica, Vina gran quantità di ininefied = "
tore del Capitolo in lode de' Peducci, parlando:della min Secccea i
E gli ho tutes per cari, non che buoni

Von ostante, che sia chi dica espreffa,

Che tanta micca e cosa da bricconiy
Ser Brunetto Latini servendosi di questa voce nel suo libro vco
tutto di gerghi, e vocaboli, e proverbi Hinsanwsats 7 intitolaco
che sia antica Cittadina-di Firenze, 1

Non ti darei una mica di beata; ¢
Se bene qui par, che voglia dire un bricivlo, dal tele
tanta si pronunzia col gelto, che accennammo sopra OC. soft.
Luefia pea, e vedremo (orto nell'Ottava 18.¢ 36. seguenti

FICC-A. Ficcare vuol dir Mexeré » 0)Cacciar per forza'.

NON è presa (a pix forte, Diciamo fan presax, quando la collay cal
o simili s' appiccano gagliardamente in quet noghi »ne\bquali-sono

L'ANIMA \& appieca, Si ricordi il Lettore, che quella 6
fu le burle, e particolarmente dove si trata diyincanti,ne iquali, q
trava luogo di fare apparir qualche azione spropositata,non lafera
segue in questa bevanda, la quale dice, che appicca ' Anima al
che egli creda, o voglia periuadere, che ciò possa per incanto farsi
firare la goflaggine di Martinazza, e di coloro j che hanno tanta a
caatelimi, e ne i Demouj,

STANZA XVIIL
Quando la Maga vede un tal presente,~
C' ha in se tanta virtù, tanto valore
Da. morte 4 vita riauer si sente,

Si ringalluzza, e fa tanto di cuore, 'Cusiabe 'hontai i se
E dove fares ita un po.arilente Percio fatracal ronzin ha fell
Nel far con Calagrillo il bellumore, Vi monta sopra, € poi te xomb

Hor e ha la barca assicuraca in porto

Pere! adesso ch' eg ha ratte
Per sette volte almanco lo vuol morte, rddy

Camminerebbe più in
STANZA \&X

Perché ei bada a spudiar declinazioni Pur.grazia del mated
Pin non si pua farlo levare « panca; Tentenna tanto,
Le polizze non Pwo, parca i i frasconi, Chiesvien. done n'
E con lo spalle s*¢ givcaio un' anca; M14 « carinetie il fang

Martinazza inansmita dai regalo mandatole da Plutone, etlendo-
Sole, monta a cavallo, e taaro io fruga con gli sproni,, e col im.
zoppicando pur alia fiac  conduile ai luogo dove havea ote \

si
at

reese erProczsleezeTEt.2f:r2=...



DECIMO CANTARE. 471
ST sente viauer da morteavies.Cioè le passa quel timore, c' havea dvessere
- ammazzata da Calagrillo | x,.
mi. SI ringaliarza. Si caliegea. Lat. Gefire, Si dice ringalluzzarsi, quasi mo-
~ strarli ficro.,.¢4 animoso come fanno 1 Galletci,quando si preparaao per -
“ ter fra oro, @ dopo che hanne combattuco, e vinto. Lucilio 4ib, 8, satyr. dice:

eee Galli nacens cum victor se Gallus boneste
ee ite o Sufulie in digitos, primoresque erigit ungues.
@ [Lalli En, Te. C.5. dan. a6. ditle 3. Jn quetta nacas anor si ringalluzza. Stor:
di Seumifonte TLratt, 321 Semifoutesi, credendo d'hawer ogni dsfficuied fopita, rinesl-
— bnzxaronfi, @ fidandosi di (un valentia, ec, B pi Lowe dice: Veds quanto noi fama
om 4 iti, e 8 mimici ringalluRrati, eC.
gibi FAltanto-de cuore, Piglia animo, le cresce V' ardire. E il termine Tanto nel fix
infos gail » che diceauno nell' Occava 17, antecedeate ed altrove,.¢ si suppone
i sho deteo.aicrove:), che colui, che per la faczia la dimottrazioac con la
| Mano accennando la grotiezza, e pants di quella cal cosa, Quei che i.La-
ect ual daimus, vooltci quali sempre dicono coraggia, e cxore.
ia | SAREBSE a a rilente., Sarcbbe andata adagio. Circospetta ) O rattenuta a4
¥ risoluersi,, )L? havrebbe pensata, o contidcrata. Significa infomma operar coa
tuwore. Leace per lento, siccome Violente per Violento dicesi da alcuat; come
Questo filo, queita corda e fenre,.cloe non tela, non urata. Da Lente si fece Ri-
ing (fn, che noo ti usa se von in questa Maniera: eFadare a rilente, e significa lo
cai stesso, che Lente cioc ientameate., Nello tteflo modo chel'antica voce Diricapo
ail usatardal?-anuco volgatizzatore di Virgilio;c lo iteilo che Dacapo,
PAR ih bed: umore. de dea huomo dell' umore, vuoi dire huomo faceto, e
SFAZ1010,5-come vedenyno fapra \cstan[1]{1}o..c.58. s*intende anche wao., che
si Voglia: sOpcattarc 1 Compagna-di parole, e di fatti, e c, comes' intende nel. pre-
feute Moga, f
aia AOR eC ha la barca aficarats in porto, Cioè le par d' haver assicurata la vita col
regalo mandatoie da Piuwone
ih nae Vheche racing a | bucats wi fu i terrazzi', Cioè il Sole, che asciuga i panui
“moi deabucati, Dereazzo! »( quali Terrazzo) diciamo quella parce superiore >
jul dele case la quale per loipiu edasciata da ana banda aperta', e feoza muro, in
Se vece dei quaie lita, solteacre al tetta- da colonac, e fom fabbricati in questa forma
ww per comodità d' havere idole e percio das Latinidetti Solarimm, eda i Greci
Wil hewocaminus', Cio fornace del Sole,
iM CAM AMNGREBBE più in tre di che in uno, None dubbio., che qualsivoglia
m! — Animaie.camuninerebbe pith in tee giorni, che.in uao, ma uGamo questo modo
wih di dure per moitrar la fiacchezza d'uao Aaiuale, quasi diciamo: Quel viaggio
che egli na da farein un giorno, 10 farcbb¢e pwd voleauieri in tre giorni, che in
yun foiow ue ¥
ul BADA a fhudiare-declinagioni, Attende,:o-continovaad accennare di -cadere
w” ~— perladebolezza. Declinare's' intende uno, che*etfendo in buono stato:, o  dt fa~
ie hita,o di roba, cominci amancare nell' uuo, O.nell' altra; equi (cherza cont
4) equivoco delle declinaziom de 1 noim 5'ed-insende, che-il cavalo per la deboica
za era fempre per. cascare..
0 Wow


47% MALMANTILE o

NON si pui far lenare 4 panca, Non si può farlo riavei
star ritto: quand' uno è stato lungo tempo afflitto da i difaftri
to per terra, o vero terra terra) € che a poco a poco si va
Comincia a rizzarsi a panca; BE' traslato da 1 Bambini, quando:
dar ritti appoggiandosi alle panche; onde habbiamo un detto per
uno sia più astuto d' un' altro, che dice: Quando it Diauolo del tale nat
del? altro andaua alle panche, Franc, Sac: Nov. 158. dice: ach 60)
nostra mercanzia, che non ce ne rizzerems più a per questo anno, o
NON pwd le polizze. Non ha tanta forza ch' ei possa portare una po
Latini pure dissero: We folium quidem fuspinet. t

PORT Ai frasconi, ec. Diciamo portare i frasconi uno, che sia alg
mo, traslato dagli uccelli, ne i quali e contraflegno d' infermita,
abbastate, che paiono bestie cariche di fastella di frasconi. Vedi.,
g. alla voce grado. Qui vuol dir che il Cavallo era infermo, e malandato per lt
vecchiaia. | Lb
CON 10 [palo s* e giuscato un anca, Scherza con l'equivoco del giuoco di
nel 8 quand' uno piglia tante carte, che col lor contare
31. si dice spallace, o ha baxuto lo spallo, e perde, sì che intende che il

\&

Martinazza è spallato. von lil
GRAZIA del martello, e degli sproni, Con ' aiuto del martello, che le mand)
Plucone, e degli sproni, cio perquotendolo col martello, epi 1
gli sproni: Diciamo anche mercé del martello, ec, er
S* arranca, Diciamo arrancarsi, quand” uno per qualche difetto non pot
muover le gambe s' affatica per camminare, e forse e il verbo p
pato. Vi chi lo fa venire da Anca, che è l'offo tra "I fianco, ela coleiay
questa dalla Greca Ancon,colla quale si significa il gomito, e si stende ad
gature, che somigliano quella del gomito, Onde Sciancaro, quasi ex:
pun ha intere, enon senza mancamento l'anche. B Arrancarsi quasi tirarh, 2
straicinarsi distro l'anche. 15h) aga
NE ha da ire il sangue a catinelle, e ha bigonce, Ha da verlari moltissimo far
ue. Vedi sopra \cstan[2]{57}. (perbole usaca quando due Poitroni
ducllo, Vedi sopra \cstan[1]{62}. in altro signiticato. BC, 3. Ran, 29, che ol
sia bigoncia, Quando l'indugio piglia vizio, e-che fa di bisogno la prestezza jl
altro proposito dichiamno. ee ne va il sangue a catinelle, aah
STANZA XXI. STANZA XXIL
Quand! ti Nimico, ch' ius faa disagio Se tu sapessi, come tu non faiy
A tal pigrizia,grida ad alta voce, C' armi son queste pie
Vieni Afinaccia, moniti Santagw Fareste forse il brauo mance,
Cb' so son qui pronto acaricarti anoce, O parleresti almen-a' altro ling
Ella risponde: A noce? Biagio; Ma già che tn venisti a tno
Fate un popian Barbier ohe'lranoquoce; i
S' altro viso non haivallo a procura,

SeweR.> es > sre

cr

a repo RB re =

atten

o

alee

Lerche codesto non mi fa para. rrotté
Arrivata Martinazza al luogo dove s' haveva a fare il duello.wi tr
¢o Calagrillo, il quale vedendola venire così adagio la fgrida, ela



SSk EL

8 EE Sei ESei a oak

=
\&

DECIMO CANTARE. 4B
ella gli risponde;che non ha tanta furia, dicendogli ch' ei non' farebbe tante»

bravure, se egli fapesse di che armi ell' e armata), e che ella veniva per ammaz-

zarlo.

“STA 4 disagio. Patisce aspettando: Sente incommodo in aspettarla,
 eASINACCIA. Parola ingiuriosa, e benissimo iesire in questo caso a

Martinazza, perché veniva pigramente, come fal' Asino.

. SANT AGIO, Si dice veramente Ser egio; che fu un Medico così nominato,
perché taceva tutte le sue faccende con ogni maggior suo agio, e commodiia fino
a tirighare, e ripulire la sua mula, senza muoversi dal letto; ed è passato poi in
verbio, e yuo! dir Huomo di turti i suoi comodi, e tardo nell' operare, che
ju una parola diciamo - Agiato. O forse-dalla voce Toscana, che vuol dire Len-
fecha, Comodird,

A caricarts a noce, Quando il noce è carico di noce, si scarica con le baflonate,
e pero dice, che wuol caricarla alla foggia, che si carica il noce, pec scaricarla
poi-con le percosse

» @LAGIO Biagso. Modo di dire usatissimo, e particolarmente de i Fanciulli,
€ credo che si dica per causa della rima, e del bisticcio, perché per altro il nome
Biagio e superfluo all'espressione, valendo tanto il dir solamente adagio, quanto
adagio Biagio, S¢ bene ci e una favola notissima d' un certo Contadino nominato
Biagio, i quale perché non gli fussero rubati i suoi fichi, se ne stava cutea la not-
te a far loro la guardia; onde alcuni Gioyanotti per levarlo da tal guardia, e
poter a lor gusto corre 1 fichi, fintifi Demonj una notte s' accostarono al capan-
nettoid) Biagio mentr' era dentro, e discorrendo fra loro di portar via la gente,
ciascuno narrava le sue bravure; ed uno di coltoro disse ad alta voce; Se voglia-
mo fare un' opera buona catriamo nella Capanna, e portiamo via Biagio; Bia-
B10 ciò -udito,scappd dai capannetto tutto pieno di paura gridando Adagio adagio.
o di qui può forse havere origine il presente dettato Adagio Biagio, o adagio disse

ago,

FAT £ pian Barbiere che'l ranno quoce. Di questo dettato ci serviamo, quando
Ron vogliamo acconscutre che si faccia qualcola in nostro danno.

» COT ESTO vifa non mi fs paura, Quando vogliamo mostrare di non temere
diciamo: Ha tu altro viso?e qui Martinazza dice: Va 4 cerca d! un' altro viso
perché corefho non mi fa paura.

SEVER AGGIO. Invende quella colla fehe le ha fatta bere il Diavolo, 1] Fran-
zele dice bexarage corcispondentemente alla nostra voce.

A tao ma' guar, Cioè a tuoi mali guai; Mal per te, che ci venisti, Ci sci ve-
puto per wrovare il tuo danno, Cusi 44a' passi diceli alcuna volta per cattivi pal-
si; ome 'Piano a ma' passi,

MANDA 1 faggio. Quando si da una piccola porzione di quella mercanzia,
che si vaol vendere:, acciocché il compratore possa riconoscere la qualità di etla
mercanzia si dice; dare, o mandare il faggio. B Martinazza dice a Calagrillo,
che intanto mandi il faggio della sua carne ai vermini, perché fra poco vuol
mandargli nell' avello tutto il corpo.

NON volti portar basto. Non son solita fopportare ingiurie.

Ooo STAN:
414 MALMANTILRS 2
STANZA XXIIL t ]
Horsh, dic' egli, all armiv apparecchia,
E vedrem se farai tante corenne,
o questo suono allor mona Pennecchia

y

Dice fra se: No,no:Non taro Ammenne, \-  E ch* io t° insegni far
Sard meglio qui far da lepre vecchia, Così tn ch' item
E fenva star a dir pur al o... vienne, Milafis a}
Fa proua ( già dilcefa dal destriero ) Ma fa pur quitof
Se le gambe (cldicon meglio il vero. Bt ual, se eu fu z.
STANZAXXV20 1)

S? al cimento, dic' ella, del duello C
A furta corsi, hor fuggolo qual peste, Però che dop' al muro f
Pero va ben, che chi non ha cervello Grid egli quanto vuol,
Habbia gambe, e così mena le [este, Che per le grida it Lupose
Mortinazza, vedendo, che Calagrillo non cede alle sue bravate,

che fara meglio per lei non indugiar più a fuggirsene, pero (non si:

cavallo) fmonto, e fuggi così a piede verlo il Castello!
rimproverandole il mancamento, ma essa stimando più il peri

la perdita della riputazione sen' entra in Malinantile, e lo lascia
SE farat tante corenne, Se farai tante bravure. Detto di derisione a wu

vantatore. wR
MONA Pennecchia, Detto derifivo alle donne. Da Pennecchio', ig

priamente si è quella quantità di lino, o lana, o cosa simile, che si

rocca per filarla, detta così quasi pensiculum.. Dal Lat. pensum.
NON tanto ammenne. Non fara così. Ogai parola non vuol risposta Per

io non voglio poi anche fidarmi in tutto di Platone'. Amen \& parola Bl

vale In verita, Per verita. er
FAR da lepre vecchia, Cioè tornare in dietro, La lepre vecchia per'

gnar terreno, quando e seguitaca dal levriero da in dictro, (il cS atto

La un ganchero, Vedi sopra \cstan[2]{76}. ) ed il cane furiofo se

scappa innanzi, e perde l'occasione di pigliarla. L' astuta maniera

della Lepre è descritta mirabilmente da Eliano nella Storia degli animali'
cap. 14.. are

SENZA dire alc,,., vienne, Andarsene subito, o senza 'merter tempo it

mezzo. II Pulci nel Morgante, £ non è tempo da dire ale.... vienne.
SE le gambe le dicon meglio il vero. Se cilia fara più presto a fuggire

a cavallo, Quando le gambe, braccia, o altre membra fanno bene la

razione diciamo: Le gambe, ec, mi dicono sl vero, cioè non mii fallifeone

mancano sotto., Wee

Cl hauessi detto ulmen Salamelech, Almeno ci' havefii ta detto.

Turchesca usata da noi per (cherzo; e significa, Pace, o Salurea voi. ~

FARM le feilecche. Betfarmi.. Vedi sopra \cstan[7]{25}, 11 Vor

goefe dice, che Cilecca wien dal Greco Cileo, che wwol dir mulceo far

feilecca far tl contrario di carezne, civ far burle. Ma può essere, che |

licta si fece Lezei forca di delicatczzc così Scileccke il contrario, che A

aliettare, e poi burlare,

E intana di riterno

= pe eet S*e2 ce oc We ieee se*8. BS screes.es.es


SSe

DECIMO CANTIARE. 475
WMI lasci a prima giunta in sulle secche. Subito,m') abbandoni + Milasci:senz'

- alcoltarmi..B' lo stesso.che lasciar in Naffo,visto sopra C, 1, an. 79, Si dice an-

che lasciare in Seco; lasciar sulle secche di Barberia. Lat. Syrtos.

AO teco il sarlo. Ho.rabbia teco., perché ilxoder. della rabbia s' assomiglia al
roder del tarlo nel legname:,Per il contrario si dice: auer baco.con una persona,
cioè averci paiione. Petrarca: Afentre che il cuar daeli amorofi vermi fu consumato

TI vegito se tu fussi in gremboa Carlo, Tiarriverd per.tuto, Diciamo; J.
grembo 4 Carlo, cioè Carlo Magno Imperadore, per mostrare che si vuole arri-
vare uno, e vendicarsi in ogni maniera, quand' egli anche si fuggisse fotco la pro-
tezione del più porente, valorofo Principe del. mondo, come fu Carlo Magno;
econ i Latini diciamoanche,in grembo a Gione.. at

| CORRER a furia,, Eo stessos che far una cosa senza considerazione.. Vedi
sopra \cstan[5]{41}. E qui (cherzaintendendo se corse nel venice corre anche
nel tornare in dietro. '
 CHI nan hs cervello habbia gambe. Significa chi non ha havuto giudizio,o me-
moria di pigliare, o fare tutto quello, che egli doveva in uo viaggio, habbia gam-
be, cloe lo faccia in due, o più viaggi, ma qui il Poeta scherza,.¢ motteggian-
do Martinazza si serve del proverbio, per intender, che se ella non hebbe cer-
ucllo ad accettare, e venire al cimento del, duello, habbia hora le gambe per

ire
MENA le feste, Pa speti, e lunghi padi, Le (ele, cioè il compasso, s' assomiglia
alle gambe dell' huomo; e pero mexar le feffe s! intende adoprar prelto le
gambe, cioc camminar velocemente, correre.
ANT ANA. Intendi se n' entea nel Castello di Malmantile. /ntanare da Tana;
cava foterranea.
DIET RO ai muro faluns efte, Chi ha un parapetto di muraglia non è dubbio,
che € sicuro dalle stoccate.. B/se dal Lat, \&è, formato all' usanza nostra, de'
li niuna parola intera finisce in confonante. Ii Burchiello nella fine del primo
Sonetto. on funt non funt pisces pro Lombardi. il primo fant va seritto, e letto
funte come qui Efe, acciocché il vero torni. E in quel verso, per dire anche
spe 2s'aliude a un vero Racconto, che si trova (critta nelle Craniche de'
Predicatori, alla vita di Giovanni da Vercelli Generale.
DALLE grida scampa il Lupo, Detto ulaciflino per mostrar la poca Rima,
che si fa di coloro, che gridano,

STANZA XXVL STANZA XXVIII.
Poich* egli vede in somma che costei, CHartinarza, che teme del suo male,
Alsrimenti non torna, fa i suot conti, Vedendo che 'l nimico se le accaita,
Che fara ben ch' ei vada a trouar Lei, Tre (caglionc'ba la porta,a un tepo fale,
Come faceua Macometto a i montis E gli da nel mostaccio dell! imposta «
E perch' ell ha due gambe, ed egli sei Ds poi dandola a gambe per ie soale,
( Mentre pero di fella ei non i/monts ) Senta dar tempo altempo,apigliar fosta
L arvriuerd:ne primaildestrier punge, Infacca nel falon, la done e il ballo,
GC? all entrar dé Palazzocs te lagsunge. Ed ei la segue foefo da cauallo.
O00 2 STAN-
.


476 MALMAN TIVE) 0

STANZA XXVIIL:
Appunto era seguito in sul festino, \
(Come interuienein ee
Che due di quei che fannoda xerbine
S' cron per Donne disfidari «morte
L' un forestiero, e /mentico pel vino he
L' a mi lafera,anch'eicenddoincorte } -
Ha Spada accato il Cortigian,ch'é l'altro,
Ma più per ornamento, che per altro, Alle spalle

ca STANZA XKX.
In quel ch'ei morde i guati,efaquei sees o Che im

Che van de plano all arte del Adirrilto, o

Ech'eglihafempriall'ufeioguoctes:

Dietro alla Serega giunge Calagrillo, Più des pie e

Calagrillo seguitando Martinazza entra con Lei nel is oO
che già fatto giorno ) continovavano a ballare,'¢ mette paura a
larmente a un-zerbiaello, che ¢flendosi sfidato con un suo tivale
fusse quelio, e pero si fuggi codardamente. 3
COME faceva Macometto ai monti, cioè se NON VengZORd Pr aendi si
noi da loro, che così e fama, che dicesse Macometto, per mofti
miracolo, comando a i monti, che scenueilero gilda iui, “e veer
venivano 'dicesse; Horst: andremo noi da loro.

HA sei gambe, Cioè due sua, e- quattro del Cavallo. 0

GLI da l'imposta nel mostaccio, Gli ferra la porta in faecia Che T
mo quel legname, che chiude le porte,'¢ fincitre da: Launo poiter) B
Serrar la porta in faccia:, per intendere operare - fare in modo, the =
vicino alla porta non entri,'¢ ferrar (4 porta tn fa le calcagna, 'intendere
uno fuori di casa., come vedemmo sopra C, 3. st. 50. 'Nenehe serial
T imposta nel viso., o ne i piedi. ae
DANDOLA a gambe, Cominciando a correre. Vedi sopra 0.4
SOST-A. Riposo.. Vien dai verbo sostare, chee: il. Laune/ ey
re,o fifere,

FESTINO, Trattenimento di giuoco;o di ballo. Vedi ropa Ca
celi Fefino., quasi festa piccola,, come quella, che fi fa\felle ca!
delle grandi, che si fanno-nel pubbiico..

TRESC.A, Così-anticamente dicevafiuna speeie di allo dal qual
hoggi Trescone specie di bailo, come vedremo sotto C; 11g. U
Purg. c. '10, la piglia per specie diballo, dicendot
Trescando alzatol' umile §. rosacea
E nel presente logo e presa per adunanza wi. gence', che'
che la piglia il medesimo nell' daf.C.14.

-  Senza Viposo mai eralatrefea
Da trefea; trefeare, ches' intende operaré; e Tre)
telle, che vuol dir cose di poco prezzo, o stima. Vedi a
£ANNO da Zerbino, Fanno dei bello, e del galante,

ao Pe oes SP THE SS E

-

a aie aa Ae



¥

DECIMO CANTARE 477

\ TVTT AY architettura's ec. Vuol dite, che quel tale usava nel veltire ogni ar~
te, € s' aggiustava con ogni maggior lindura, diligenza,edifegno.
GONFIO, Alticro, e superbo per la sua bellezza, come fa 11 Pavone,. che al

i detto delle persone più semplici,' gonfia perché si stima bello; donde poi pavonez-

giarsi, che vuol dir considerarsi, e vagheggiarsi per bello; E questo verbo <(pri-
'Me quel che vuol dir il Poeta nel presente luogo.

CREDE turar le Dame in Veffunio, Crede far perder 'tutte le Dame per il suo
amore. Crede, che la sua bellezza sia per far' ardere del suo amore 3 e Vefusio \&
il monte del Regno di Napoli, dove sono le voragini di fuoco.. rf
 HA paura del dilnnio, Cioè del diluuto delle percosse, le-quali spengono amor
nel cuore, e ' accendono nelle spalle ma differentissimo.

» VAN deplano all' arte del Mirtilio. Son-douute,-¢ si richiedono all' arte dell'innamorato,
da que! Mirtillo introdotto per innamorato dal Guarino avila fus
Tragicommedia incitolata Pafforyfide.

HA gli occhi a' mobi, Bada, oflerua, sta vigilante. E diciamo «' mochi, e non
allfaltre biade di maggior valore, perché essendo i Mochi cibo proprio de i
Colombi, sono da' essi prt, che l'alcre danneggiati quando sono di poco seminati,
€ peroé necessario haver J'.occhio, e badare con più attenzione a i mochi, che

Ll alte biade.

[pochi, Detto ironico, 'che significa moltissimi.
ha più cnor dun grille, È codardo, non ha animo, Sotto C.11.z9.dice,
Han facte di Leoni, e cnor di scriccioli, Appresso i Greci per il contrario trovafi
Thymaleon, cioè Cuar dt leone, per vomo valoro(o, forte, cortaggiofo.

FA più capicale de' piedi, che del ferro, Si confida più ne i piedi, che nella sp2-
da; cioè Mimd più ficuca dife(a quella del fuggire, che quella dell' armi: e circa
queita voce capit aie. Vedi sopra C. 7. It. 82. e C. 8. st. 6.

STANZA XX\&KI. STANZA XXXIL
Tosto tornando l' amicizia in parte, Prima, che tra costoro altro ci nasca,

Si viene allarmi, che ciascuna armata
Ciò tien del altra un segno fatto adarte
Per darle atradimento la-pierrara:
'Di qui si viene a mescolar le-carre,
Tal ch' in vederlatante scompigiara,
Rittrandosi a dir badan le Dame:
Baha basta; non più; dentro le lame.

£ che la rabbia affatto entri frat cani,
E ms conuien fattar di palo in frasca,
E ripigliar la Storia. del Garani,

Chre dietro a far che'l Turacirinafia,
eAccio,tornato pot come i Criffiani,
Ad onca della Strega-ogni mattina
Ritorni a vifitar ta' Kegolina

Di questo sollevamento ciascuna-del'e-parti prefe sosperto di tradimento,.e per=
ciò si venne all' armi dentro al medesimo salone.. 'Qui l'Autore.lascia costoro, ¢
torna a Paride Garani, il quale egli latciodopra C. 8. st. 59..

TORNO' ? amicizia inparce. L amicizia si divile; cive ritornd inimicizi

“mMeera*prima. Parre t quella; che i Latini dicevano parter, 'cioè fetta, fazione;
'onde Parziale, cioè affezionato,difenditore. Quel che sia parte per womo di spada
ch' egli era, e non di lettere, lo defini assai bene Farinata degli Vberti ti
vecchio, 'pretio.a Gio. Villani |. 12. Volere, e disuolere; € per oltraggi, e grazie riceuute,

DAR ta pietrata, Dar colpo mortale; o conclusivo, dare a tradimento la pic-

trata


478 -MALMANTILE \&

trata è ver in quel verso di Plauto; leera manu fere lapidem; panem oftemit
altera, Che risponde anche per appunto al nostro proverbio ane, e (a
Sajata.; o% SW
ST viene a mescolar le carte, Si me(cold la zuffa. Vedi sopra C. 9. st, 35.
SCOMPIGLLAT A. Confula. Qui intendi, rottalapace.
LA rabbia, e fra i cani, Così diciamo quando vogliamo esprimere »
s' azzuffano indistintamente: Ii Latino Xabies inter canes, Ee
SALT AR di palo im frasca, Passar da un discorso ad un' altro assai
dal primo. Far digressione. 11 Monosini dice, che con questa nostra
s' accorda quella de' Latini usata da Tertulliano, De calcaria in carbonariam..
Ma questa s' accorda più con quell' altra,. Dalla padelia nella brace. I |
Tertulliano nel lib, de Carne Christi dice così. dgitwr de calcaria, quod:
carbonariam; a Adarcione ad Apellen, i ose
LA regolina, Così chiamano i Ragazzi dell'infima Plebe Pornia ube
ga, la quale sta aperca in tempo di Quarefima, ed ivi si vendono frittelle,
Ii, baccala fritto, ed altre forte d' untumi simili, praticata, e frequentata da' ra
gazzi, ed altre genti vilissime, come era il Tura, che spesso v' andava.
STANZA XXXIIL STANZA XXXV.

Paride giunto in mezzo ai casolari,
Ove messer Morfeo aun tempo solo
Fa dir di sia molts in Pian Giullari
Strepitando fugeir lo fece a volo,
Sicognun deffo vanne a'fusi affari,
Ed ci, che Star non vuol quivi a piuolo
eAnzi dare al negurio spedizione,
Domanda di quel luogo informazione.

STANZA XXXIV.

Un gran Villano, un bnom acta matura

De' Quarantotti li di quel Contado,

Che perché ei non ha troppa [effitura,

ed è profontuofo al quinto grado

Junanzi se gi fece a dirittura,

E concerts suoi inchin da Fraccurrado,

Benevenga disse, Vostra signoria,

E Le buone Calende sl Ciel vs dia,

Jn quanto al Lupo egli e un! animale
Aa che aninial dich? io bue,
Via fiftol ds quei veri, un faci
C' ha fatto per sngenito gran dant,
E già con i forconi, e con le pale
J popoli affilliti rurto mguanno
Quin' oltre gli enno feati tutti riete
Per levar questo marbo da nn

STANZA XXXVL

Ma gli e un fetanalfo foatenato,

Che non teme legami, ne ea.
S? e carpito più voilti, ed ammagliatty
Ed ha ricifo funi tantogrofe,
Le bastonare non gli fanno fate
Chie' navha abriga rocehechethaledt
D' ammayyario co' ferri non c'è viy
Cb' egli e come frncar n' uaa matity

STANZA XXXVIL

La entro a quella felua ei si rappiarra, Che tutti gl' animaliycht ei raccat

* Perch' elia égrande,dirupata, e fitta, Cudfando gli trascina lvirittay
wacciocche nimu.un tratto lo cumbatta,, E chi guatar potesse; io.fopenfierd
Quand egli ha dato a'Socci la sconfitra, Chie' v' habbia fatto a' ofa uns "

Paride entraso ne i Calolari di Montelupo trovo, che tutti dormiyand,|
con strgpitare fece (uegliargli, ed havendo caro di sbrigarsi, proccurd
intormazione da qualcuno delle qualita ed abitazione del Lupo, ¢s' ak W
un Villano Sateapo del paefe, che gliene diede puntual ragguaglio. Ecol dif
fo., che.fa fare a questo Villano, mostra il modo di parlare del cont
KODZCp

tel on oe Ge Cee:

i el i


ee.

tak

Zo
7
a
a
q
i
5
j
:

%

DECIMO CANTARE:? 479

CASOL ARI, Intendiamo più case insieme in campagna scoperte, € spalcate;
qui intende di Montelupo, il quale se bene e Castello, ha più figura di Casolare
per esser le Case tutte quasi rovinate, e distrutte.

MOREFEO, Favoloso Ministro del Sonno, il quale i Gentili tenevano, che a»
i comandamenti del Sonno suo padrone si trasformafie nella facia, nel passare, €
ne i costumi in qualsivoglia vivente, e però fu scritto: Hominum fittor Morpheus',
bestiarum imirator. Ed altri, Atorpheus, o varijs fingit nova vultibus ora, detto
Morfeo da Morphe, che in Latino vuol dire forma, faccia; onde noi Smorfie,
per brutto arto, o gesto fvenevole, che si facial particolarmente col vio. E
roan in furbesco; mangiare. Qui dal nostro Poeta Morfeo, e preso per'lo Nef

fo sonno. \

FA dir di si a molti in Pian Ginllari, Fa dormir molti; perché colui, che dorme
senza posar la testa, l' inchina, e fa con essa il medesimo atto, che fa colui',
il quale con efia accenna di dir di si. In Piaw Gintlaré intende nel letto, che anticamente
si costumava il dire. / vo in' Pian Ginllari per intendere, io voa letio,
o mi pongo gil a dormire: Ma questo detto come oggi poco usaco è ancora poco
inteso. Per altro Pian Giullari è chiamato un Borghetto di Case nel concorno de'
Vilage di Firenze non troppo distante dalla Città, che anticamente era de' Giullari
cafata Fiorentina. Giullari, e Giulleria, dal Latino iaculares, vuol dir butio-
ne, e buffoneria, o allegria. Vedi il Varchi nei suo Hercolano; ed il medesimo
nelle Stor. Fior. lib. 15. Won gridavan con quella festa, e ginlleria ch' eran soliti.

STREPIT ANDÒ fuegir lo fece 4 volo. Facendo romore, fece fuggir Morfeo,
cioè sveglid i popoli.

NON vuol far a pivolo, Non vuole star' a disagio aspettando; diciamo: Tener
uno a pivolo, quando lo facciamo aspettar più del dovere, o più di quel che egli
vorrebbe, quasi che egli flia legato alla nostra volontà contro a sua voglia, come
si fanno star legate le bettie a i pinoli, che sono pezzi di bastone, che fitti per le»
mura servono a i Contadini per legarvi le bestie.

DE' Uuarantotte del contado, De i più riputati, e Aimati del paefe; perché il
Quarantotto in Firenze è la dignita Senatoria, la quale e il maggior grado, che
godano i Cittadini Fiorentini.

NON ha feffitura. E' huomo ardito, e libero nel parlare', non ha vergogna, o
-riguardo o timore, che lo ritenga; e s' intende anche Un' huomo, che operi, c
viva inconsideratamente, Sefirwra chiamano le Donne quella filza di puoti radi,
che son solite fare da piedi, o nel mezzo delle loro vesti per farle divenir più
corte, o per aliungarlo con sdrucire detti punti secondo, che torna loro in acconcio
dal Latino lectura, come vuole il Ferrari, Le Romane moderne la dicono ritrep-
pio, quasi piccol ritiramento della veste, ed è lo stesso, che imbastitura, che
vedremo sotto C, 12. st. 33.

PRESONTVOSO, Più che ardito, e poco men, che impertinente: Uno che
prefume assai di se medesimo, e s' arroga piii di quel ch' ei merita. Un'arrogante.
Daa. Purg. C. 11, dice.

Ed e qui perch fu prefontnofo

DA Fraccurrado, Da Fantoccino; da burattino; che intendiamo quei bam-

bocci, che dicemmo sopra o. 2. st. 46, 11 Bini nel Capitolo del Bicchicre <
Kuch



MALMANTILE

Questi perché son grandi, ancor son belli
Sends poca betta senza grandeRra,
\ wei paion Fraccurradi, e Spivitellig
Tra' canti Carna(cialeschi vi e un canto intitolato. Canta, ni

Fraccurradi, e Bagattelle, ove sono descritti, i giuachi, che
o giucatori di mano con tali legnetti, e burattini, detti-Frac

LE buone Calende il Ciel vi dia. Virconceda il Cielo, tutti i.
dia ij buon' anno.

SVE di panno, Sciocchissimo ch' io fone, Io ho manco giu
dicenci. Vedi sopra C, 6. st. 98. Lyf

VN fistolo. Le nostre Donnicciuole intendono Demonio, Diavolo. Vi
male maladetto, Bocce, gior. 7, Nou, 6. dufino a tanto, che il fiftolo u
Juo marito. Così detto dal filchiare de' serpenti, a' quali egli e affo

F ACIMALE, Huomo maligno, e da fare cout it
lefactor, Cavalcanti Storia \libcap[9]{11}. Cerri huomini besti
i quali mai alcun bene fecero, e now hanrebbono saputo farne, huomini faci
futili, n't

PER ingenito. Per naturale instinto, che questo vuol' intender quel Ci

eASSILLIT I, \oucleniti, adirati. L' Affillo è un vermicello volati
alla zanzara, ma più grande, ed ha un forte, e lungo pungiglione
quando il Bue e punto, entra in grandiilima smania, e tem eda qu
tadini quando vogliono intendere, che uno è in collera dicono; Eel:
o¢ afiduo, Sula in Firenze ancora questo termine, ma per ischerzo,,
con ammogliati con i quali farebbe termine ingiuriofo, quando non fusse usa
in burla, perché e un dirgli Bxe, ve hfe

¥GVANNO. Quest' anno, Vedi sopra C. 6. st. 92. alla voce auannotte,

SMINOLT RE glienno feati tutti rieto, Qui intorno gli sono stati cust dietro ct
cando di pigliarlo, Enno, e la terza persona del numero plurale dell' indi
del verbo essere, hoggi poco usato in questa forma fuor, che da i contadini;¢!
uso Dante Parad. C, 13, me

Non per saper lo numero, che enno oot

PER levar questo morbo da tappeto Per levar queita peste, e questa tribolazion
dal mondo; J sappete serviva già in Firenze per strato ai Supremi
quindi levare uno da tappero figuibca levario, o privario di quella dignita
quale e posto, che por pafiato in proverbio yuoi dire privare, o levar uno
qualsivoglia luogo, come qui che s' intende levar dal mondo,

SET AN ASSO, Satana; Demonio, dai Latino Saranas,come
nuovo testamento. Appelliamo Saranafo uno, che sia fiero, ¢
scrua di tal jua forza per far del male: e usato però dalle donne contro,
ciulli fieri, e vivaci, 1 quali chiamano anche WVabifi. In Ebraico:
onde il nostro Dante. i? acai

48a

Pape Satan pape fatan aleppe. ) arie

Evuol dire Aduersarins, Aduersarins noster dsabolus, ate
CakPITO. Cioè pigliato con violenza, dal Latino carpere.
i Contadini. 'sili

zeseseEer: |

ae pee ee ee

— a


ee

=

fet

he

ete

RERt E

DECIMOCANTARE: 481

2. Vedi sopra in \cst{18}. il termine santo di cuore,
NN giù fanno fata, Non gli fanno male, o danno

'TANTO

 NON? ha 4 briga tocche, che Ube feoe. Subito, che ¢gli ? ha toccate gli pal-
fa il-doiore, non stima 'e percoise. Quando i Cant hanno toccato delle bastona-
te si squotano, e restano di guarite, che è indizio, che non sentono, O non cura~
no più il doiore, e di qui viene questo significato di squotere |e bufic, e ne hab-
biamo il dettato Tw fai come i Cams, es' intende cu (quoti le bufie, che significa
Non le cur:, non le senti, non ne fai thma, ec. Vedi sotto C. 11, tt 44,

MACT A. Con Vi longa. Monte di fatii dal Latino Adaceria,

Sl rimpiatra, Sinaconde, Vedi sopra C. 9, fh. 5,

 dVia40.. iano « Latino nemo. Won sopra C, 7. st. 89.
. £0 combatra. Gli dia noia;! impedisca,,

QVAN DL egis ha dato a' Soccs la sconfitta, Quand' egli ha messo sottosopra, o in
contufione le mandrie', cioè fatti fuggire i bettiami afialtandogli: Che Socciv.s'in-
teade quel beltiame, il quale si da a un Contadino per far' a mezzo del guadagno,
quasi dica a Sccio, cloe a compagnia. L'azione, che nasce dal contratto di So-
Gita, si domanda da' Legifti Azione Pro focio; Ma noi per Seccio intendiamo
una focieta, o compagnia particolare, ovvero una Accomandita di beltiame, che
si.da altrui., perché lo custodi(ca, e governi » a mezzo guadagno, e perdita. So-
Zi /poj pure dal Latino Sectas intendiamo quel, che i Latini dissero sedatis iure
Sodalitijs iunetus, o Buon forse dichiamo a colui, che non guasta mai, e che acco-,
da le conversazioni,

CA' ei raccatta, Ch' ei raduna, Ch' ci trova, e piglia,

CIVEF ANDÒ. Cioè¢ pigliando con voracita; rubando.

LU ritza, Cioè in quel luogo li, Termine ruttico, Dal Latino i rea, Qui-
via diritto; in quella dirittura, 0,, come 1 Franceli dicono, en cer endroit,

10. fo pensiere ch' e v? habia fatto a' ofa un cimitero, lo credo ch' ci v' habbia ra-
gunato una gran quantità-d' offa. Che Cimitero diciamo 1] luogo, dove si forter-
Tano imorty. Vedi sopra C. 4. st.2g.¢ C. 7. tt, 27,

STANZA 'AxXVill
Sta Paride afenttrio molto attenta,

Ada pai vedendo quant' ei si prolunga

. Frafe dice; Costui cs ha dato drento
Come quel che vuol far mela ben lunga,
Gli e me troncargli qui il ragionamento

 checio prima, che il ds mi sopraggiunga
40 polfa lasciar l'opera compira,
Peri gis. dice: O via falia finita.

STANZA XXXIX,

Poi ch' egls ha inteso dow' ei possa bartere

e4.un diprefja 4 rinuergare il Tura,
Lell'efer soleo il boscose a' altre tartere,
Che gli narri costui, saper non cura:
La laterna apre, e il libro,od'alcarattere
Poa, vedendo., dar' una lettura,
Così leggendo senti darsi norma

Di quanto debba fare, in questa forma,

STANZA XXxX,

Vicino al boschereccio Scannatoia
Mentr' il froco di stipa vi riluca,
Palton grosse, Bracctale, e Schizzatoia
Co' Gucators a palleggiar conduca;

Ai rumbombar del suo diletto quoia
Toffe vedrà, che 'l Gocciolone sbuca
Keuei ricchi arnesi vago di mirare,
Che già in Firenze lo facean gunfiare >

Sta Paride attento al discorso dei Villano; ma conoscendo ch'egli era entrato
ip on discorio da non finir mai, lo fece chetare, e preso il libro, da esso compic-
BEDS tate;

se quel ch' ci doveva fare.

co.


482 MALMANTILE

COSTVI ci ha dato drento. Costui è entrato in un discorso da non'
fine; e me la vuol far (unga, Cioè vuol far' una lunga diceria,
OVVTA, E' lo steBo, che ors. Latino  Eia age. Termine, che
spedizione. ms " i ast
DOP' ci può battere, Cioè da qual parte egli habbia andare per ir:.
Tura. Tey et
APN dipresso, Alquanto vicino a dove egli sia. Si dice o a ited “tid
vel circa, Dal dirsi per esempio: Furono tanti, quanti io v' ho detto vel cireay
Cit, o in quel torno, haa alle se
RINVERG ARE. Rinuenire; Ri; Ri iare; Raccapezzare. o ist
ALTRE tattere, Altre zacchere, minuzie, o circostanze di poca considera- a
vione. Se ben Tattere per (cherzo s' intende una specie di malore, che viene in 4/4
torno al stesso per crescenza di carne. i ?
CARATTERE, La forma, o figura delle letcere dell' Abbiccl. Voce latina
tolta dal Greco Character, ¢d i) Monosino vuol che itia lio dir carartolo, ma si
non fo per qual cagione, se non fusse per allontanarsi dal Latino, che per altro te
non ho letto tai, ne sentito dir carartoso, se non a qualche Villano del tutto ru- ne
fico. eee
SCANNATO10, S) intende il luogo dove s' ammazzano i buoi, edaltrebe>
flie, ma qui intende quella felua, entro alla quale si nascondeva il Tura, elas Ki
chiama scannatoio, perché quivi il Lupo scannava le bestie,
BRACCIALE, Manica di legno dentata, della quale s' arma il braccio per laf
giocare al pallon grosso. Vedi sopra C. 6. st. 34. any Ta
SCHIZZ AT OIO ( qui intende il piccolo). Strumento d* ottone, o d' altro Ne
metallo fatto a foggia di canna da crilteri, ma assai minore, e serve per metter '
vento in qualunque luogo con violenza, come si faa gonfiar palloni,0 pillotte, wt
o per (chizzar liquori; e 'i maggiore per far serviziali. Latino e/yfer detto così,,

quasi frumento inondante, e lavativo. Vedi sopra C, 3. st. 14. Che

PALLEGGIARE, Dare alla palla, o Pallone, mandindoio, e rimandandolo Che
per tra(tullarsi, e per avviare il giuoco; ma non giocare regolatamente, Onde» Ry
quando uno tira in luogo un neguzio, coll' avviare chi glielo raccomanda, 2 un? Cad
altro, e che quello lo rimanda, al primo, e tutti due si accordono a burlare il Tes,

pover' huomo; si dice + Tra loro se ¢a palieggiane; che in Latino forse i direbbe Giaj
Coludunt, ast SOE?
GOCCIOLONE. Si dice a uno, che ta guardando una cosa con grande atten- Tu

zione, e con desiderio a' ortencria, e propriamente si dice di quelli innamoratt » Beri
che stanno i giorni interi appit d' una casa a guardar la dama, che € alla finellsa, Ng
¢ si coniumano, € si struggono a poco a poco, e per così dire a filia a flilla, o vu
però dice Gocetolone al Tura, e vuol' esprimere, che egli era innamorato di que L
guarnesi. Lucrezio lib, 4. Pariando degl'innamorati. ee fog
Wamque voluptatem prafagit multa cupido, Ri
Hac Venus eft vobis, bine autem est nomen amiorit; - Ca
Hine ila primum Veneris dulcedinis in cor B g

Stilauit gutta, o fucceffit frigida cura, ?
CHE già lo facean gorfiare, La voce gontiare vuol dire Andar superbo, comes oy

4
ra
53,



DECIMO CANTARE.
dicemmo sopra in questo.C. st, 29. sed il Poeta (cherzando con l'equivoco di gon-

fiar

3 ma in effetto vuol

483

Ic pillotte, e palloni, che era il mestiero del Tura, come acccnnammo sopra

Gf st, 47. pare, che voglia dire, che quegli arnesi eran causa, che il Tura (eo

andava sup poi dire, che quegli aracfi eran causa ch'ei

J Sontava le milous » ei palloni, e che egli gonfiava la pancia, buscando per mez-
zo

imi arnesi da comprar roba per empictia.

— STANZA XXXXL
Paride in soofe fatice ubbidisce 5
Accender fa le scope, e intorno al si
 Già questoje quel st spaglia ed allestisce
 M faa bracciale, e si comincia il gimoco;
Al suan del qual! Amico comparisce,
MA ritenuto, perch' e vede il fuoce,
 Elemento, che vien dall' animale
Fugritaper instinto naturale.
STANZA XXX AIL
NGarani che fava alle-velette,
 Fedendo che'd Compar viene alla cesta,
Che te scope si spengano commerte,
 Edin un tempo a i Giscator da festa:
WD un batrer docchio il giuoco si difmeste
La fipa si sparpaglia, e si calpesta;
Tal che sicuro t' animal ridotto,
Va Paride pian piano, e fa fagotto.
5

STANZA XXXXIII.
Ciò ch'é in giuoco in nn fascio egh ravvia,
E tra gambe la ferada poi si caccia
A tutto strascicando per la via
Con una fune a otto, o dieci braccia.
Spinto dal genio a quella ghiortornia
Da lunge il Tura scguita la traccia,
Come fa il Gatto dietro alle vivande,
E il Porco a' beveroni,ed alle ghiande.
STANZA XXXXIV.
Vaghecgialo, s'aliunga, xappa, e mugola,
Talor 8 appressa,econlexampe iltoces,
Hor mostra shavigliando aperta l'ugola
Hor per leccarlo appoggiavi la bocca,
Tutto lo fina, lo ronistia, e frugola;
Così mentre il suo cnor givia trabocca
Ej, che non rocea per letizia terra,
Entra nel Borgo, e in gabbia si riferra:

TANZA XXxxv.

Perché Paride fa ferrar le porte,
E poi comanda a un branco di Famigli,
Che quiui farti bauea venir di Corte,
Che di loy mano l' Animal si pieli;

Ma i Birri, che buscar temean la morte
Non voglion accercar simil consigli,
E fan conto ( se ben' ei fa lor cuore )
Che @ passi cutrania 2 Imperadore,

Paride in ordine a quel che trovd scritto nel libro datogli dalle Fate, fece acc
cender il fuoco d' avanti a) bosco, ed attorno vi messe gente a giocare a} pallo-
Ne: a quel romore il Tura ulci dal bosco, ed allora Paride fece un falcio de' brac-
Ciali, pallone, ed altri arnei, e legatolo a una fune lo fece strascicare per la
scada, la qual conduce al Castello di Monte Lupo, dentro al quale i condusse il
Tura, seguitando quegli arnesi, e Paride fece ferrar le porte, ed ordinò ad alcuni
Bi tri, che quivi haveva per questo fatti venire, che lo pigliassero, ma essi
impautiti non vollero accostarsi.;

CAALLESTIRE. Metter' all' ordine: Approntare

L) AMICO comparisce. Cioè il Tura esce dal bosco, e vien fuora spinto dal gu-
sto di vedere il pallone.

RITENVTO., Renitente; cicé non alla libera, ma con qualche timore per
¢aufa det fuoco:, del quale il Lupo naturalmente ha timore.

STAVA alle velette, Stava osserv'ndo. Vedi sopra C. 7. st. 67. It Burchiello
nella Novella del Medico Bolognef=. e dello Scolar semplice dice: Andando ¢ri-
dando cerci tutta in casa, e tronarlo non gli fu ordine, onde tratte dalla disperarione si

Ppp 2 parti,

484 MALMANTELE:
parth, e lo Scolare, che staua alle velette Vitorhato in'cafay ec,
. IL Compar viene alia cesta, Cioè Animale vien fuor
zimbello de i braceiali 5'¢ palloni, ec, iy HgOW
DA sia ai Giuocatori, Ha vestardi giocare; Licenzia iG)
agli Scolari vuol dir Licenziar la Squola, e di qui dicendosi dar,
cenziare ogni sorta di lavoro, Daag
IN un barter ad? occhio, Inun momento. I Latini pure ditono-Jr%è
SPARPAGLIATE, Spandere contufamente, e ae ee
come si fa della paglia, quanido si batte, e si spoglia il'grano'. 1 Pulei dite:
Sopr' alle spallela treccinsperpagtia., 0
FA fagotto, Fa un fa(ciode rbracciali, paltomt ec. Par fagotto; e 10!
quasi, che far le balle per bactersela, per andarsene. Latino v4/a colligere. ~
Sl caccia la via fra gambe, Comincia a camminare. Latino + viem,
SEGVIT A la traceia, Seguita, o va dictro allapesta, oalla sed étol-
to dai Bracchi, i quali si dice seguitar la traccsa, quando mel cercar della %.
ec. fiutando seguitano quella strada, e quel tratto', per dove ella ha tirato;
per dove e passata: di qui habbiamo il verbo inzracetare-detto sopra C, 7. st,
BEVERON!, Così chiamano i -nostri Comtadini quella bevanda grossa fatta di
crusca, € d'acqua, ec, la-quale danno a i Porci. Vega eh aie
LO vagheggia, 'Lo guarda aftewuofamente.. Sivaledi-questo verbo vaghectiog
per esprimer il gulto, col quale il Tura guardava quegli arnesi;:essendo tal ¥
proprio degl' innamorat', Vedi sopra C. 7. st79. (4 aay
MVGOLARE, Buna voce indistinta, e che non finitamuore fra i denti.
ROVISTIARE; Ravoltolare, netter soflopray Forte aeglio-romifia dal verbo
rovistare, che vuol dir Muovere da un luogo all'altro. Ji Pulci., Morgante va
rouistando ognt cosa.. Hh Wx
PER letizia non tocca terra, Sopra C. 9..st.63, Per V allegrezza non può*star
n¢ i panni, 'che € lo stesso; e figaisca haver'allegrezza', o gusto grandissimo; Si
dice ancora; ma in modo basso, La camicia non gli tocca il sedere, Ml Boccaccio
Novella 32. iam
FAMIGLI, Qui sintende Famighi di Giustizia,cioè Birri;la famiglia debPode-
sta,dal Boccaccio detti fergents, quali ferxientes, siccome da noisfamigli,cic' fa
FA conto, che pafii 'dmperadere,, Finge di-non intendere, o-di non lentire qu
che si dica\, Detto forse questo dal tempo', quando era l'dmper: rec
vanni Paleologo-in Firenzeal Conciiio »che per cfersi già tata familiate la un
vista, e forse, mancandogli i danari., non comparendo:così pompola, ne Co
bella compagnia; e appagata anche dalla prima volta in su, lacuriosita; quan-
do passava per le strade, non doveva far muovere la 'gente come prima, e come
andò egli arrivd; Onde si venne a dire, quasdo uno non si cura di qualche co-
f: Facciam conto, che passi lo Laperadore, t a7%8 one
: ST, AN 2, AoREXKEV End OV
Poiché gran perso ha i porri bapredicate, Senza pin (har a burtear via il fiato,

E che fan conto tuttania cb'eb cantiv, Totti di mano abc. iiguanti,
Pero che.da i Ribaldi gli vien dato: Bisogna, dice, con questa canaglia
Li udienz.a, che da il Papa i furfanti, Far come il Podestdidi

'AN:



DECIMO CANTARE: 485

ZAXXXXVIL o» STANZA XXXXVIIL.
ds caps Si resta il Lupo, e'l Tura buomo diviene;
Ma non pero, che libero ne sia,

ad una delle [ue legacce

a addosso al' Animale C' ambi fone appiccati per le rene
eee a uso di bifacce: Formandoun Leong ha e la Bugia,
r di tal.concia dé cauiale Dice Turpino, e par ch' et dica bene,

Ch' essendo questa si crudel malia,
ina di iupo, ed una d'huomo sembra, Lon erano.a disfaria mai bastani
di sua [pecie oguunna ha le/ue mzbra, » Gli odor birreschi semplici dei guanti,
4)! STANZA IL
opri tal masserizia E Paride, che gra ' chbe notizia
molto pin fatto le mani, Da quel suo libro,si da quint ai cani,
 Percheglincants in man delaGinffizia Perché pin oltre il libro non ispiega,
i fichi- alla nebbia vengon van, 'Ona' et fa conto al fin di tor la lega,
ide veduto che i Birri non ubbidivano, ed havendo per avvertimento dal
bro datogli dalle Fate, che gl' incanti rimangon vani iv mano della Giustizia,
sdiede a credere che havessero tal virtù ancora i guanti dei birri, e per questo
f eae al Caporale, e gli messe addosso alla bestia, la quale si converti
Induce corpi appiccati insieme, che uno d' huomo, e I altrodi lupo. A tal me-
tamorfofi resta Paride stupefatto, e non sapendo che cosa farsi, perché il libro
bon inlegna da vantaggio » risolué di chiamar due fegatori per(eparar It Animal
bruto*dal razionale. In questo mostro il nostro Poeta imita Dante nell' Inf. C.
 25. nella commistione di que! Serpe con ' anime di quei cinque Cittadini Fioren-
i € la delcrizion di tal mostro comincia al verso: Se tu sei hor Lettore acreder
dente,
 PREDICARE a' porri. Predicare al deferto,, Affaticarli in-vano a esortare
uno.a far bene, che i Latini dissero vento logui; Surdocanere.
 PANNO conto ch' ei cantiB lo stesso, che dar Pandienza che da il Papa ai furfanti
che ia fuiteza vuol dire n6 fare stima delle parole d'un0,0n6 badare a quel ch'es dice.
CAPOXRALE. 'Capo di squadra di birri, Grado che si di anche sia i Soldati.
Vedi sopra'C.'9. stan. 2.
 BAR come il Podestà di Sinigaglia, Cioè comandare, e farda se.. I) Duca di
Calauria Sigifmondo havea aflediato Sinigaglia,nella qual Terra era per Gover-
shatore foltituto da Gio: de Castro, Petruccio Piccolomini; Costui tentd di ab-
' la Terra, dicendo esser. meglio uccello di-campagna, che di gabbia,
¢d a lutaderiva il Podesta, ma i Cictadini featendo questo dissero di volergli get-
“tare dalle fiaeftre se più parlavano d' abbandonare la Città, e vennero tanco in
odio ved in disprezzo de i Cittadini, che. quando comandavano noa erono ubbi-
Giti, edi qui venne il Proverbio: Far.come it Podestà di Sinigaglia, cioè Coman-
dare, e'far da se, Cavalc. Scor.:
 Deeaa + S'intende quei legami, con i quali ff legano le calze, cingendo
ambe.
MSACCE, Così chiamiamo due sacchetti appiccati "uno contro all' altto a
'due cigne, i quali si mettono a traver(o ai cavallo, ec. sopra il quale si cavalca.,
'€ servono per porcar robe, come si fa con una valigia, sono appellate —

me

vit

486 MALMANTILE

bis sacche, due volte sacche, o sacche a doppio. Lat, AMdantica Bocce,
nov. 10.5, Haveva Frace Cipolla comandato che bea guardasse, che let
»» (ona ada toccaile le cose sue, e (pectalmeace le sue bilacce nelle q
»» cose rare. B più otto nella medeGa novella. La prima cof che venac
» presa fu la bifaccia, agila quale era la peana. weet
CONZI 4A. Quando si dice coacia di guaati s' iateade profumameato,

si dice guanti di coacia di Rona, di Venezia » di Spagna 7 ec. e 8 intend
mati alla foggia di Roma, ec. Qui dice concia di Cauiale, cioè feteatt »
fragore, o feagraaza e Detto ironico., Were ta
LA Sugia. La Bugia si figura uaa Femmina con due facce differenti, come
@' orso 0d' huo ny, o di lupo, e d' huomo, come è aci prefeace luogO,
DICE Tarpino, Scherza cone fa sopra \cstan[2]{31}, autorizzando en

te (un Novella com i detti di Turpino, come fa ? Aciosto. lve
MALIa, Iacantefimo. Suregoneria. Vedi sopra \cstan[8]{52}. Donde 44-
liarda una strega. iy
T AL maferizia, Iatende i guanti del birro. cust cee
Sd aicani, S'adica, Quando uno per la stizza grida, e fa altre dimostra+
zioni d' impazzienza, o di rabbia diciamo; Si daa' can, Vedi sopra C. the 10,
STANZA L STANZA LiL...

Per ciò fatti venir due Marangoni, E morta re la dd per cofacertay =
Con tutto quell' ordingo, che s' adopra MA quel Demonio insieme strappicch,
eA fegare i legnami, edi panconi, E qual porco ferito agolaaperta

ef dinider il Mostro metre in opra;
Mitre la fegaim mero ai dusigropponi
Scorre cosiva il mondo sottosopra
Mediante il rumor de i due parrienti,
Che un fa d' urli el altro dilamenti,
STANZA LL
Pur senza ch' inraccato elit habbia un osso
La [ez infino ail' uitsmo aileefe
Lasciando il Tura libero, ma rosso,
Dietro ds sangue com' un Genonese;
La Be(tia gli volea tornare addosso,
Ma Paride, che (ubito ? intese
Presa la (pada la cagio pel mexrd
Pensando di madarla un trattoalrezro,

aa
Per dinorarlo forte se gli ficeay - -
Ed eslic! alt' incontro stan. all! eta,
la [a La testa un sopramman gli appicedy
Ch in due parti diuifela di netto
Com' una testiccinola di capretto.
STANZA Luh
MA ritornato a penna, e4calamaio
Pur quello Heffo a Paride si volta,
Che per veder il fin di quel mofeaio
See' fulfe mai possibile una volta,
Mena le man chee pare un Berrettait,
Ed a chius' occhi pur suonas r4ccilta
E dagli, e picchia,risuona se mA 7
4a forbice, t e sempre bella.

Paride fatti venir due Segatori d' asse, fece fegare il Mostro in su It artasatu-
ra deli' huomo con la beltia, e così gli fepard; Ma la Bestia tentava di
carsi onde Paride caglid la Bestia pel mezzo, ma eifa presto strappiccd, B qui
il nostro Autore immita l'Ariosto nella favola d' Orillo; levata da Vergilioacil
Eneide, che finge un tal' Erillo Re di Paleftrina che haveva tre anime, onde era
necessario tre volte ammazzarlo per finirlo a.

tHARANGONT, Si dicono i Garzoni de i Legnaiuoli che lavorano peropra,
quando in una bottega, e quando in un' altra a tanto il giorno, e non jn
una boticga a falario di tanto il mese; ma qui l'Autore intende Segatori di le-

goami,

a ese

—————

eee

me

=

- +e

DECIMO CANTARE. 487

| guaind 3 €gli ordinghi, che ? adopra, sono la fega a due mani, lima per mettere
} Gags denti, e il cavalletto per adattarvi sopra quel materiale, che i dec (e-
ox. cavalletco si chiama pietiche. Vedi sopra \cstan[6]{6}g. alla voce im-

 PANCONT. Sono afi grosse circa un quinto di braccio, le quali si rifendouo
per farne o assi più sottili, che si dicono panconcelli, o per farne correnti.

GROPPONE. S' intende la parte di poms di tutti gli animali, o bipedi,o

yadrupedi, e lo diciamo ancora codione, ed è propriamente quella parte che re=

fra le natiche, e le reni. Vedi sopra \cstan[6]{69}.

VA sottosopra il mondo, Lo strepito confonde l'universo. I Latini pure dicono
Mundi fumma readit ima, o ima fumma;¢ vuol dire, che jo ttrepo era gran.
'dithmo per le strida del Tura, e per gli urli del Lupo.

 ROSSO come un Genoxeje, \&* in Firenze una Compagnia, o Confraternita di
Secoiari detta de' Genovesi, perché e formata di gente di quella Naziwne s Co-
storo hanno per coitume d' andar proceflionalmente la sera dei Giovedi Santo a
vifitare le Chiese, si battono le reni ignude con mazzi di corde entrovi alcune
ficiie di metailo acute come quelle degli sproni, e queste forando la pelle ne trag-
gono il sangue, il quale bagna loro le reni, ele tigne di roflo; E di questi in-
tende il nostro Poeta nel presente luogo.

. tH ANDARE uno al rezzo. Mandare uno nell' altro mondo, df fresco, cioè
il corpo suo sotto terra. Ammazzar' uno. Rezo, vuol dire un luogo dove non
arrivano i raggi del Sole per interposizione di che che sia, e si dice anche, me-
riggiv, bacìo, ombra, e uggia. Vedi sopra \cstan[6]{75} e c. 9. stan. 44.

ST.AV-A aif erta, Stava ocnlato; stava avvertito. Erta si dice la talita d'un

BRIO; e are all' erta e termine di caccia, perché la Lepre ha per propria di
for sempre alla volta della fommita de' monti, per non esser così facilmente
arrivata, e pigliando i suoi riposi, scoprir paefe, e minchionarc icani; e pera

in caccia State al? erta s? intende Habbiate l'occhio, osseruate; il che ¢

poi pafiato in dettato comune a ogni cosa.

PN sopramman gli appicca, Gli da un soprammano, che è quel colpo, che si da
¢ spada, bastone, ec. cominciando da alto, e calando a balio. Vedi sopra

5- stan. 41.

DI netto. S' intende lo taglid pulitamente in un fol colpo,

TESTICCIWOLA, Le telte degli Agnelli, e de i Capretti da noi si chiamano
Testucinote, e per friggerie si tagliano nel mezzo per lo lyago in duc parti ugua~
li; eda questo taglio afiomiglia quello, che fa Paride alia tetta det Lupo.

4 penna, e 4 calamaio, Per ! appunto. Vedi sopra \cstan[2]{19}.

VEDER il fin di quel mofeaio, Veder il tine di quetia cosa noioia. Vedi sopra
\cstan[4]{9}.e c. 9. stan. 51.

MEN A le man ch' ei par ux Berrettaio, Menar le mani dicemmo sopra C, 1, st,
7. quel che significhi, e qui intende che. mcnava le mani con ceierua,come fauna
1 Berrettai, e Cappellai, che nel felcrare i cappelli, o berrette menano le mani
Prelto in riguardo dell' acqua bollente, con la quale si fa tal lavoro,

SVONA a raccolta, Continova a perquoter a jungo, che così suona la campana;
quando suona a raccolta di popolo per le prediche, ec. ed 1 verbo sonare si

° gailca
+

ee

- — “gk ae
2 488
a
488 MALMANTILE |
gnifica anche perquotere, ¢d e della medesima natura, che il
habbiamo detto altrove. 6 ay eee
DAGLI, picchia, risuona, e marvella. Questo di dire
re uno, che adopri ogni fea induftria, per fare una cosa perf
do più vole le diligeaze. Vedi sopra \cstan[7]{16}. Similitudine,
tratta da' fabbri, quando Javorano il ferro sopra l'incudias; Qui:
d' Orazio incudi reddere versus, mettergli alP incudine, sotto
critica. Cioè efaminargli, rivedergli di nuovo.con somma, rigorofa
diligenza. La nottra maniera; Barrere il ferro quando è caida, ebbe
meate da questa prontezza, e macitria talieme, che si adopra per lavorat
nalmente l'dcudir degli Spagauoli, che vale aixeare, voce ormai si
è fatta dal Latino ddcudere, ciod battere insieme il medesimo ferro.
dichiamo per esempio. La prego a volere accudive « quefke megorio; o si
FORSICE.. Questo termine significa ostinazione,, per esempio. fo 2,
che tu non faccia la tal cosa; e tu forbice, cioè Tu ostinato l'hai voluta |
modo. Dicono che venga da uaa Donna offinata, e capona,, la quale
chieito al Marito un par di Forbice, e non havendoglicie il marito mai:
te,ella ad ogni cosa, che il marito le domandava rispondeva: Forbice;
impazzientato da queita sciocca ostinazione,le proibi il dirlo
più lo diceva, per lo che il marito la baflond, ma. non per: ella se
maneva, ficche egli un giorno sopraffauto dalla collera la gewo in ump
cila fino che potette parlare sempre dite; Forbice, ed in ultimo goa p
valersi della voce, si valfe delle mani cavandolg fuori del' aequa con le
giori alzate ed allargate in figura di forbice,per mottrare che moriva |
oitinazione, e caponeria. Questa novella e vulgatitiima fra le nostre
io ho trovata tra una raccolra di efempi facta da un Buontempr
mano del medesimo tengo fra i miei nianoscritci. 2 e
Lit sempre quella bella; L' e sempre quella medesima. Questo yien da un
co, 1] quale andava accartando, e cantava una cerca orazione al suono di un
tarrino, fermandosi alle porte de' suoi benefators i giorni destinati;
venuto a fastidio, do sempre la defima cosa, inci:

quelli, che gli facevano l'elemosina a dirgli, che se non cantava q 'ae
orazione non gli havrebbero dato più nulla, ed egli rispandeva; Pa
se', cht domani ve ne vaglio cantare una bella, Ma pecche il Povererto ai 4
se.non quella, tornava l'altra mattina, e cantava la steila, laonde i f ”
fattori.accortifi, che il Meschino non ne. fapeva altre compathonaadolo, giù te
cevono. L' è sempre quella bella, ed intendewano l'e tempre quella 1 ig
che è poi venuro in dettato, e significa noi siam sempre-alie medelim a
quanto racconto ancora fra gli scriti del medcfimo Bugnrempi top z
pucato ali' origine del presente dettato. ren “a i
S. TAN ZiAisbl Moni tap 'y¢:
Tal ch' ei si scosta none, e dieci passi, Pervia gli anuenta m
E piglia fato, perch! es pronar vuvle, i
Selavirtude a forte gli giouassi, i;

C* hanno! erbe, le pictre, e le parole;



489
gout STANZA  i
recasse a scorno, Resta in parata', molto gira il cnaré
alle gioftre, e alle quitaney | 'Pimceis pikes anc ielibbienicfie,
we b gli vada incorna, > Merce ch ei fache'l Diauvloe bugiardo,
EB latrartigo' sassi, come un cane; “E quanto en sia furtile, e filigrosso;
i, ver ch' e' fusse ! apparir del giorno, oPercia si merte un pezro a bellofguardo,
; L! Ombre,il Bau, ele Befane oCredendoognor che gli faltasse addosso,
Sparyce affatco, e più non si rinede, Aa poi ch' ei vedde omas d' esser sicuro
Ma Paride per questo non gli crede | = Andò all Oste, e cauollo di pan duro,
Vedendo Paride, che quel Mostro si rappiccava sempre » e che ci non trovava
'modo di liberarsene per ferite, che glisdette, gli venne in pensiero, che se era la
Werita, che in herbis, verbir, \& lapidibus stesse la virtù, poteiic essere che alcune
di queste cose havetie virtù di fare sparire, e svaniresl Moltro; e pero preso il
 [xa dove, il quale era pieno di parole, e dliverle erbe, € de i faili ogni cosa tirò
   addosso a quel Mostro, e l'indovinò, perché subito egli sparì, ed il Tura rimase
   libero”, 'Con tutto questo,Paride non si fidando, stette buon pezzo a osservare;
ma veduro, che il Lupo non compariva più si parti, e andò all' olteria a man-

Patt. i
Ors ' fiato, Cioè si riposa,
. MLAEST RO Grillo Contadino, 8 nota la favola di Grillo Contadino, il quale
per fardispetta @un sue fratello Medico, che non gli volle dar parte d' un tefo-
FO, che infizme! havevano trovato, si fece Medico anch' egli, e con i sui forcuna-
a fiti's' acquisto la grazia del suo Re, non solo per havergli: rifanata la
cavandoie una tilca di pesce della gola con ungerle ilc,..., ma ancora
per haver saputo indovinare i segreti del medesimo Re, e chi erano coloro, che
a lui rubato havevano, in somma fece diverse scioccheric, le quali tutte per gli
} spares fidondarono in stima del suo valore, e l'accreditarono per un valoro(o
Medico, e grandissimo Indovino, come si legge nella di lui favolosa vita, o di-
Ciamo spiritola Satira.
WINT ANA} Bruna campanella, che si tien sospefa in aria (oftenuta da una
molla dentro a un canacilo, alla quale per infilarla corrono 4 Cavaiiert con la
Jancia', come fanno anche al Saracino, che dicemmo sopra C. 4, than, 57. € si di.
Ce ancora Chintana, Varchi Stor, Fior, lib, 15. Fecera metrer della rena a! avanti
al palazze, ed appiccare la chintana, Dai noltri Ragazzi e detta corrottamente
Timana 9 ed è iatelo quel lor passatempo, che fanno, infilando una zucca fresca
in una corda, e pottala in aria attraverso a una Arada corrono con alle la mano
@ dare in detta zucca, unmitando i Cavalieri, i quali corrono alla quintana, 0
al Saracino, Dice che Paride era avvezzo alle gaintane, e alle gioffre [che nel
Prelence inogo son fitioninu; s¢ ben gioftra's' intende quando i Cavaiieri corrono
a corpo a om 70 al Saracino, e quintana significa quello, che diciaino qui fo-
Pra) perché Paride haveva più aout militato im Spagna, dove haveva cfercitaco
1 jor! gradi della mulizia, e tornato alla Patria tu dal Serenityaio Gran Duca
fatto Governatore deija forcezza veochia di Livorno, ed hunorato del titoio di
Macttro di Capo, I nome tuo era Andrea Parigi, fu fratello d Aifon(o, e di
Paoio detto sopra Papirio Gola, è Figliuolo di Giulio, e fu come custi questi va-
sa = Qqq jen-

-

Se

:

o

i

=i

aa

—

' Ye
*

490 MALMANTILE™
lentissimo Tngegnere, € periti archi Qui
Ferrari cusi. Ludus equeltris,cum diretta in encun simulachrnn:
gehtat, bala incurritur, Alcunt han detto come Vguecione Pisano.
zionario, che Già così detta dalla quinta parte della piazza yin
tri, come Balfamone sopra Fozio da un certo Quinto inventore 2ed
la vera origine mottra il Pertari essere da Comrus.cioè ee i
punta di ferro; e si raccoglie dabtitolo nel Godice:, de i,
radore chiama questo giuoco con voce Greca Kynranos., In ordi
Chintano, e non Chincana pare, che lo chiamaile, se sha a
ma, Fazio degli Vberti nel Dittamondo.:
Gionani bigordare alli Chintani,
E gran tornei, ed. una, ed altrag
Far si vedea con giuochi nuoui se ferant. -\

jofira '
>

CALAPPOLERIE. Cosa di poca stima: oda farne poco conto i “Apine; '

triceque, e buttubata, V. Feito, e ivi sopra lo Scaligere.,
BAV, e Befane, S' intendono quelle Larue inveatate dalle Balie per far paura
ai Bambini, come habbiamo decto sopra \cstan[2]{50}. et
REST A sn parata, Si ferma in guardia, cioè con la spada pronta, ed in posi-
tura comoda a ferire, E' termine da schermitori. yori
MERCE', Con la prima, €;, firetta, ela seconda longa, vuol dir mercede
che profferito al contrario vuol dir mercanzia: Nel modo che:è detta nel pre-
sente luogo, ed in molt' altre occasioni mere vuol dire per causa di ciò: qual di
ca io riconosco tal mercede, tal benefizio da questa cosa, o da i,
ec, ficome Paride riconosce questa mercede, o benefizio di non si fidare del Dia.
volo dal sapere, che quello e bugiardo, ed ingannatore. Questoidetto e lo ftelio,
che Grazia del marcello, e degli foroni, che vedemmo sopra in que(to C, fran, 20,
1L Diauoloé futtile, e fiia grofo. 11 Diavolo è fagace, ed inganna l'huomo,
facendo il goffo, ed il balordo. * inet
REST Aa bellu (guards, Reled guardando attentamente. Bello fguarde® una
villa poco lontana da Firenze: e per la similitudine che ha questo nome bella/enar-
do con il verbo guardare si piglia in detto significato. pn amaetir
 CAPOLLA di pan duro, Mangid adai. Gli mangid tutto il pane, che haveva
in casa, gliclo rifint. Detto usatissimo per esprimere Aeangiare assai ee,
spy paler

ays Nae

FINE DEL DECIMO CANTARE. eae



|

eee
eh ARGOMENTO, '
St

Cangia le dance in rissa un? accidente, a
iSe

Fuggonfi Bertinella, e Martinacza,

-VNDECIMO CANTARE,

Vien fuor Biancone, e fa morir gran gente;

5 Ma gli Orbi a tui fan poi sentir la mazza, 6%
es Da Celidora, e da Baldon possente 33
ee Mezza defirntta e quella trista razza; th

Tagliansi a pezzi in quelle squadre, e in queste '“*
E così in ata? fansi le feste. e a ge

2 —
RAPALA AAAS A

om STANZAL STANZA Ik
Chi mi.dard la voce, ele parole ui ci vorria chi scortica L' agnello,
'antia dir la guerra indiavolata; Es al mondoé persona più inumana,
Ond' oggimas dara le barbe al Sole o descriver la frrage,ed il flagello
Bertinella con tutta la sua armata; Che seguir si vedrd di carne humana;
C'alCiel Gagliarde alzando, e Capriole, Ch' io già oni sento, mentre ne favello,
\Farò.verso Volterra la Calata, A tremito venir della quartana.,
. Efe d' amor canto con cetra in mano, E n' ho si gran terror, ch'io vi confefi0,
<Derd col ferro il vespro Sicilsano ? Che mai più de' miei di farò quel desso,

Tinoftro Poeta volendo.nel presente Cantare narrar la battaglia seguita ia Mal-
mantile, e le crudcita grandi, che (uccessero nel Palazzo della Regina, dice, che
a fac tale descrizione vorrebbe esser un' huomo sanguinario, quanto è colui, che
scortica giù agnelli; che non si spavencerebbe, come fa egli acl rammentarsi i]
grande firazio, che fu fatto di carne humana in tal batcaglia. Qui immuta Dan-
te-nel principio del C..8. dell' Inf. che dice;

i Chi porrsa mas pur con parole scialte
Dicer del sangue, e delle piaghe a pieno
Ch! 10 hora vidi, per narrar più volte ?
4 mi lingua per certo uerria meno, L
— avventura seguita Vergilio nei 6. deli' Kncid., che dice, imitando pure

°

Qqq 2 Non


492 MALMANTILE
Non mibi'y si Sassen or ag
Omnia penaru ee omina polfem.—

E così rende l'uditore attento,  curioso, col promettere di vol
venimenti così maravigliofi, che non è per trovar parole adegu
ne esprimere. A! > bet: F e

'DARA le barbe al sole, Morira,. E' traslato dalle piante, le qu
cioè si feccano, quando si fuelgono, e si voltano loro le barbe al So

GAGLIARDA, e Calata, Sono-duc specie di danza, ob
scherza con la voce ealata., che vuol dir caduta, oftela, d
ver fatte qui Gagliarde, e capriole fara la calata,, cioè calera verso
comunemente s'intende andar sorterra, cioè morire. Jay +e

DIRA il Vespro Siciliano, Dopo haver cantato versi amosgh ante fj
Siciliano, che s' intende; vedrà, € provera stragi. B' nora la follev ne de
ciliani (orto Gianni di Procida contro a i Francesi nel cempo, che questi ti g
giavano la Sicilia nella qual sollevazione fu il egno, che un determina gi
al suona del Velpro ciascuno si movesse contro a i Francesi, come se
successe grandissima strage di essi Francesi; E da questo è nato il '
Vespro Siciliano; che vuol dir fare steagi,ammazzare. Vedi Gio. Villanil
61.¢ Giachecto Male(pini nella Continuazione della Storia di Ricor
cap. 209, >

Hil festive l'agetib + Sona tial yarenaisinmeeltaie
i quali nel tempo, che sono gli agnelli, vanno per Firenze gridando. Ch
scorticar l'dgnello; per bulcar denari in ammazzare, e scorticare Metti ani
il nostro:Poeta da quello (canaare,\¢:scorticar un' intinica di effranil,
puta huomini crudeli, e senza pieta, e questa per'accomodarsi abgenioy"e cap.
cita de i fanciulli, che stimano quell' atto una granditlima inumanita,
nando quelle bestiuole innoceati. ny sieht

FLAGELLO. Qui è preso in significato di eopine, farts ee CA

di. Vedi sopra C.1. tt. 45. invaltro signiticato. In Gio. Villani trovafi nel fen ayy
usato qui dal Poeta; Flagello, e Fragelio; come costuma di dire anche aug
piebe Fiorentina, e come dissero i Greci, e si legge ne} testo Greco dell®; pac
fey

uy

dy

hio, Phragellion per quello, che i Latini dicono Fracetium Omcto
sgrazia,sferza, o 2agello ds Givve vinci Node tibro 12) verlo 397%
831. Attila Re degli. Vani tu soprannominato per quelo, Frage! t
TREMITO dela quartana, Quei brividi, che si-sentono' dal pazavente nell'en-
trare della febbre quartana, i quali sono assai maggiori diquegli, che soglions fie
venire, quand' uno ha qualche spavento}-eperd 'con dives VA tvemiep dela pias edy
sana, intende, che lo (pavento era grandissima,€ fuori dell' ordinario: E «ali tend
brividi, o tremiti vengon' allt huomo:, perthela patira fringe il cuore; per lo

che il sangue corre tuctovin aiuto di esso;¢percio--membri esteriori, e ie parti te
superficiali, ed estreme rimangon\fredde; edi steddo facendo riftrit i pori, be
cagiona quel che i Latini dicono rigor » che farizeare i capelli, o pels "€ Cagio~ ni
na il cremito, il quale si domanda capriccio, e rsbrezzo, Vedi C. 6 Gig

MAL più, de' miei di fare quel defo, Spaurisco tanto, che esco



cro prima.

ets DAN ZALHI8.'

be il galio apportator del giorno
La notte nera più d! un Calabrone,

E il sua buio, e quant'abre eli'ba dintorne
Diognise qualunque grado, e condizione,
| Acid sicuri omai faccian ritorno
\ Gli nccei, cantando il lor falso bordone,

AIncitr'al Sol,ch'in quespa parte, e in quella
| Fa pel lor gorxo nascer le granella
lead ety

Perché-crafeun » che quini si ritrova,

VNDECIMOCANTARE. 493

“fino a che viverd » non farò mai più allegro, come era mio solito, perché questo
- spavento m' ha fatto mutar compicifione, e temperamento: Non saro piii, quel

STANZA IV.

Quand' infra Dame, e Cavalieri erranti,
C' al trescone in Palazzo eran intentiy
Comprefeun dietro all'alero i duellanti,
Armati tutti due, come fergenti,

-Si shallo il ballo, andar da cantoicanti,
Ele chitarre, ei musics Srumenti
Ai proprj suonatori, e balierini
Divenner rante cnfie, e berrestini,

STANZA V.

Si fa pero bifbiglio, e si rinnuou

 Kedendo entrar quell' armi coid dentro, L? odio fra te farion già quasi sperto,
 Subirovdisse: Qui garta cicacca: Che tirando ai rispere: gu la bufa,

: =, £ trama di qualche tradimento, Ruppe la tregua, e rappicce la xufa,

4 iver la levata del Soley e dice, che in su quell' hora entrarono nella stan-
22, ove si faceva il ballo, Martinazza, e Calagrillo, che la seguitava con l'armi
F inmano:; per lo che si lascid star il baliare, e si venne all' armi., rompendo la

tregua, perché ciascuna delle parti sospetto d' esser tradita, e che questo fusse uno
— militare, come i disse sopra C, 10, stan.31. dove laicid questi duel-
EL gaily apportaror del giarno sbandina la notte. il gallo e solito cantare in sull'ap-
pariridel giorno, c. però dice ch' egli è apporrasore del giorno, e che da 11 ban-
do alia notte col suo cantare. Somniaque excuffit nuncia lucis aus, disse un Poeta\footnote{};

Excubitorque diem cantu predixerat ales, canto un' altco, \& erista spettabilis alta,

Auroram gallus vecat applandentibus als, Disse il Poliziano nel suo Villano.

CALABRONE. E! uva specie d' infetto, o verme alato di figura simile alla
mosta »maatlai più grande, e di colore ncgrissimo, ed ha un jungo, forte, e»
acutissime pungiglione. Con questo nome chiamiamo,ancora il tafano detto fo-
Corot. 8. I Greci Prouerbilti dissero scarabao mgrior, Più nero dello scara-
B10, che è un' altra specie:di mosconaccio. i
4N comro.al Sole, Glivucceili vanno incontro al Sole cantando in ringraziamento
del benefizio, ch' ci fa loro, maturando le biade per loro alimento.

* GOZZO. E! il primo ventre degli uccelli, cloe quella vescica, che hanno ap.

Ppit-del colio, dove si ferma il cibo, che beccano, edi guivia poco a poco si di-

Mtribuilce al ventricolo; e da noi si piglia ancora per la gola dell' huomo, perché

vien da gutrur. re

° CAVALIERI erranti, Così son chiamati quei Cavalieri avventurieri, che son

descritti ne i Romanzi Spagnvoli da loro detti Cauaheros andanes; wa qui inten-

de, che erravano perché stavano ballando aliora, che bilognava combateere.

“| TRESCONE, Specie di ballo, cos detto da Tresca balio anuco. Vedi iopra

G. 10. st, 28, Dante Purg. 10.

: ee S=TePiie etre

= ee

a

i



494 MALMAINTILE® (0
Li precedewa al benedetto Vaso

Trescando alzato, o umile S. ane cond
cioè faltando, ballando. M As +
SBALLO'. \\ verbo shallare vuol dire disfare le balle; ma qui

re il balio, In buon Toscano non si direbbe shallare il dar fine al
pis la forza della lettera 5s, aggiunta al principio di verbo, 0
ignificato contrario si come la particella, i», appresso i latini, |
tare, spiantare; grariofo, fgrariaso, ec, ma il Poeta se ne
scherzo di ballare, e sballare, e seguita il bitticcio  dar da canto s canti
figuratamente sbaf/are, per eccedere la verita ne' racconti; e o ¢
numeri di cose con vantaggio, e con caricatura. " *
DIVENT AR casse, e berrettini, ec, Cuffia, come s'e detto sopra C, 8, fh. 48:
una berretta fatta di velo, o di tela.a foggia di sacchetio usata dalle |
ferrar dentro i capelli in capo; dice, che gli Prumenti vennero casse, e
perché le chitarre, ed altri strumenti simill corpacciuti, essendo bateuti in
capi di coloro, e per la loro fottigliezza sfondandosi, fecero I effetto
be in sul capo la cuftia, o berrettino, cioè lo ricoperfero, e ferrarono in
E' detto usatitfimo. Ti faro wm berretrino della chitarra, per intendere i
chitarra in su la cesta. Vina timil frafe venne in capo a Omero nell' Iliade, quan-
do disse, Lapidea indui tunica, per voler dire, Essere tapidato', quasi il ricoprire
uno di faffate, sia uo fargli un vestito di pietre, che gli stia bene alla vita.
GATT A cicova, Ciۏ misterio foro. Ci e inganno, eum Tras tiled
i Latini. tet ain
TRAMA, Si dice quella feta, ec., che serve per riempiere le a
renza dell' altra, che serve per ordire, che si dice orsoio; che per la più n
si dicono ordito, e ripieno. Dante Parad. C. 17. t soi aged Rl hn,

Poiché racendo si mostro spedita Tat

L! anima fanta di metter la trama Che

Jn quella tela, ch' io le porsi ordita, (SRE LS Sir

'Ma trama si piglia per concerto, ene habbiamo il verbo tramare, cheiwuol dir bag
negoziare copertamente, e forco mano, dilegnare,, concertare, Mraletrami ge ha,

fio affare,ec, Bdicendo: Queffaé trama ds qualche tradimento, intendes/Queho Oe)

@ tradimento concertato. Latino /ute/a doi. Varchi Stor, Fior, lib Cm
d' una conuenzione facta senza saputa d' un terzo dice + Orazio se ne? ada
rugia, senza che il Sig, Gentile fuspicasse non che sapesse cosa alcuna di questa' i
trama di gocciola per intedere specie d apopicsia,quasi una coperta apoplethia, e da 4
questo si potrebbe intendere per rrama, uaa (pecic; e dire questa è specie di qual i
Che tradimento. Storia di Scmifonte Trattat, 3. dice. 4 popolo fa fallewe, e grida tha

na, [uspwcando, che trama ui false, contro di lus, speotepecnh aot ¥

BIS BIGLIARE, Dilcorrer in segretor, che si dice anche Far Pith pifft; ij
Pispigliare, che usò Dante Parg. C. 5. Skit ise Bap w
Che si fa cio, che quini si pispielia, “ ¥ they

E si dice pispigio., e pispiglio, sorta di cicalamento; e viene da quel fafurrio, che: hi

scatiamo da coloro, che parlano in fegecto.. toggi pia comunemente si diceb® =
Soighiare, bifviglio, e bifriglio, 5 te te
Th Ry

ae ae


na, O rispetto
 STANZAVL
metre man da buon Soldato,
imico ritorna a Bertinella,
f quale in quel punto casco il fiato,
UM fegato, la milza, e le budella,
Vedendo, quando men' hauria pensato,
 Vicire i pefei fuor deta padelia,
 Mtentre la fa venir Adarte vigliacco

Col suo Baldone alle peggio del sacco.

STANZA Vil,

} VNDECIMO CANTARE.
| | TIRANDO giù La buffa.a i rispessi. Non havendo più rispetto, o riguardo al-
cuno. Sxffa intendiamo una berretta, la quale e fatta a

f » € mandata gil cuopre anche tutta la faccia, e i collo: Eda questo
la faccia, mandar gis /a buffa., vuol dire oprare senza riguardo, e scaza

495

foggia di morione, che

STANZA VIIL.

Mentre 8 alcun t' osserva, ella pon mente
Per canfarsi enon esser appostaca 5
Ecco in un tratto vedesi presente
Martinazza la sua confederata,

Che poco dianzi anch' ells fimiimeute

Di man di Calagrillo e scapolata,

E seco vanne in luoghi occulti, e fenré

A fare uncanti, es faliti (congiuri,
STANZA Ix,

eit a o un certo vento non le gusta, Nes quali aiuto ella chiede a Plutone,
Che fa le (pade, e ognor per l'aria sischia, Ed ei comparfo quixi in uno ispante

wiil —- E.grd vedendo che la morte aggiusta Dice, c' ha fatto a lor riquifizione
yee] Chipi e vuol far det brano, e pin starrischia, Già spedire un tacche per un gigante
at Bel bello fuigna, e vanne alla rifrusta Qual' è quel famosissime Brancone,
it | Dun luego da salvarsi da tal mischia, Che col bartaglio,ch' era di Morgante,
pt} —- Adtischiayche non gli par di porer credere, Verrd quini tra poco in lor foccorso
- Ee Percio sospira, e non si può discredere, ef dar picchiate,e' hanno a pelar  orsa,
\& votes: f STANZA X. xi

Ed eccolo ( foggiunfe) ovvé battaglio E 8 anuedra,c' al fin piscio nel vaglio,

o desi fo dir ych'il primo,ch' egli accoppa,
Tatra l'armata a irsene in sharaglio
Che la barba penso farci di froppa;

E che al pigliar un Reeno non è loppa;
Cot scaciata abbaffera la cresta
dn veder, che de' suoi non campa testa,

Si rappicca la battaglia, e Bertinella essendosi perduta d' anima, per vedere
i ritornato suo nimico., quand' ella pensava d' haverlo tutto dalla sua, e
-temendo di non esser ammazzata in quella Foote » meditava di salvarsi in qual-
4 che ficuco, ed appunto-s' imbatcé in Martinazza scampata da Calagrillo,
J € con essa en' andò in iuogo appartato a fare incantesimi, per costringer Plutone
F -ad aiutarle; ed Egli comparfo quivi dice, che si fara venire il Gigante Biancone,
il uals in questo dire arrivO quivi, e Plutone rincuora le donne con raccontare
la bravura di sto, dalla quale da loro per distrutta l'armata di Baldone.
LE casca il fate. Si perde d' animo. E soggiungendo: Il fegato, la milza, e,
, te budelia, intende Si perda d' animo affatto
— QFeANDO men fet è pensaro. Quando meno dubitava. Non expettato valvu
ab hoffe culit..

VSCIRE i pesci fuor della padella. Perder quel ches' era acquistato, e sopra di
che s' era fatto aflegnamento certo, e sicuro.

VENIR alla peegio del facco, Venire al maggior segno di discordia, e di rottu-
ta, Nelle guerre il peggior grado, che sia, €, quando le Città,0l'Armate son
messe a facco; e però dicendosi /e peggio de! facco in peggior grado, e condizio-
ne, che è haver il facco. VL

,

|

ee
496 — MALMANTILE o 7
VIGLIACCO,, Vile, codarda., EB voce spagnuola, vell
significa furbo, e furfante, poltrone. i
SEL bella, Con bella maniera, e senza dar o del
antichi disser; bedlamente,manoneinufo..
SVIGNA. Se ne va con preftezza, o fugge. Forfeda questo
viene e omprare if porco, che vuol dite anch' egli Andarsene
fuinam, 010 fuillans emere. ed è usate questo verbo svignare
besco. Vedi sopra \cstan[4]{51}, Si potrebbe anche dire, come pei
erudito, che questo verbo fuignare ligniticaado scappar dalla Vigna, s°:
scappare di foro la Vigna, strumeato o macchina milicare, che serviva
tichi per andare sotto ie muraglic a combatier le Piazze, con le quali”
difeadevano gli atiecianti da i (aii, ed altre cose, che erano: buttace lor
dagli affediati, le quali necetiitavano quelil, che vi erano.coperti a
sotto alle medesime vigne; extra vineam exire, che suona fuignare.
VANNE ala rifrufia, Vuol dice cerca mioutamente, e con diligenza
NON si pus discreaere, Non può non credere. Non può creder, che
a cffer così, e non habbia a eficre altrimenti. Non può capacitarli
SCAPULAT A, Fuggita; Scuppata. 3' intende scampato il pericolo
LACCHE', Ragazzi, cae corrogo appiedi per servizio de' loro
di sopra \cstan[2]{29}. 2 ae
BLANCONE. B' quel coloffo di marmo bianco., fattura dell' Ammannato, il
quale e posto in Firenze nella Piazza dei Gran Duca, dentro a una valea gran-
de, la quale riceve l'acqua da diverse fontane, che scacuriscono da detto: fo
¢ suoi annetii; e se bene rappreicnta Nettugno, e chiamaco da cutta M Biancones — ui,
ai ee ray Vaca; 1 hi
MORGANT E, 11 Pulci in un suo Poema intitolato il Morgante narra'; che

M
questo era un Gigante, 1 quale nog adoprava per coubattere alt': he un Ya
gran battaglio da campana, joe alo tf

PICCALATE ¢' hanno a pelar  orso. Picchiate gagliarde, perché il) pelo dell' Oh
orso essendo difficile a suellere, e pelare » non si fa caicare con' ky
se leggieri, Pelare, wattandosi di muraglie, o pietre vuol dire-space; ol
si, o (crepolare, onde potrebbe dirli hanno a peiare  orso, cioè tare fore yi
rompere l' orso, che si dice quel pictronc, che adoprano gli fiyfaiuol FN
lire i piano delle stufe, onde nabbiamo poi menar l'orso.a Atoaan Pre
re ripulir Modana, e Ggnitica mecterii a far una cosa umpolsibue uk

PENSO' farci la barba di Stoppa, S'intcnde; E poi dargh tuoco.
Penso ingaonarci,.¢ por farci ogni maggior danuo, ie
PISCIO' nel vaglio, Blo stesso che far la zuppa nel paniere desto sopra C.
stan.7. E.che cosa sia vaglio, Vedi sopra \cstan[2]{79}. Luciano in ab
co volendo spiegare, che il far bene a' cristi e come un tar la 2upp.
perché 1 benetizzi riceuti (cappano jaro prettissimo dalla memora; 4
buomo cattivo, e sconoicente a una bog forata, che uo quello, che va i met. tes,
te, si ver(a. Plauto nei Pfeudolo, o vogliam dire Bugiardello; 2Vae piuris refert, ="
quam si imbrem in cribrum geras, Corcisponde questa maniera alia noltes (char
nel vagtio, Luciano nei Live dic; come da in cofano forato, o

ey



VNDECIMO CANTARE:

497

)zuppa nel panicre. Playto pure nel Pleudolo, la pertu/um ingerimus. dicta do-

opera ludimus. La favola delle Danaidi ha fatto luog
nifica non è cosa facile. Loppa; che si dice

19 al prouerbio.

NON: ne Detto bafio, che
anche lolla,
anche

ce » gli c levata.

a STANZA XL
Qui tacqueil Diawol,perch' e fatto rece,
“ él aria al capo giù è maligna,
anuerzo 4 star sempre nei fuco,
Vatea alle donne il dietroacafa, e/uigna,
EB lafesaus il Gigante nel sua law,
Che douendo a Baldon grattar latigna,
 Sull ulcio det falon già perwenuto,
edge Hf batraglinje questo fu il faluto,
STANZA Ali,
Sei braccia era ti bascagtio aito, e ds passo,
| Bm injragnena aimen arciotto,o vent,

4; Ma dando fu nei patcormando a baffo
cha  Van trang intatiata, e tre correnti,

; E fece tai frafiuano, e cal fracasso
14 Che shalord? « un tratto i combattenti,
oh OE per pawra, a chi non fu percosso

we |, Non rimase sr quel punto/anguc addosso,

il gulcio, che si leva di sopr' al grano quando si bacte, che si chia-
inche pyle. Lat, apinde secondo Nonio Marcello gramatico. 5

Y SCACIAT A ~ Rimanere scaciato; vuol dir Rimaner buriato, ches' intende
; nd' ugo credendosi conseguite una cosa, e facendolela sua, o non la confe-

 ABBASSERA la cresta. Gli scemera!* umore, o I alterigia, I Galli d' In-
dia, quand' entrano in frenesia, gonfiano,, e cresce loro la cresta, € patleggiano
on una certa intronizzatura, che par (uperbia; ed usciti di quella frenesia, sce~
ma, ed abbaifa loro la creita, e di qui vicne il presente dettaco, che significa
readersi umiie, contrario di Rizzar (4 crea,;

STANZA XIII,

Ed infra gli altri Piaccianteo, il quale
S' era schermito bene infizo aliora,
Vedendo un fantoccion si badiale,
Dopo il terror di tanre spade fuora,
Di quel detto farebbe capuaie,

a9 C' un bel fuggir faina la vita ancora,
444 perché in quae in la v'é mal riscotro,
Vede hauer viso di sentenra coniro..

STANZA XIV,

Poiché non fa tronar modo, ne via
Per nellun verse da scampar laguerra,
Ech' ovis e forza, che chi v'é vi stia,
Pond morto, gettasi gilt in terra,

E ritrouando la botrigleria

Apre t armadia, e dentro vi si ferra,
Con pensiero di fiarni sempre occulto,
Fin che si quiet così gran tumulto,

! Plutone si paite dalle Donne, e la(cia quivi il Gigante Biancone, il quale andò
, alla lanza,dove si faceva la zuffa, ed arrivato in su la porta alzd il battaglio,
: per comigciar con esso a perquotere, ma al primo colpo dette in una traye, la.
quale per esser fradicia, si fraca(so insieme con pill correati. Tal colpo spauri

» tutti coloro, che eran quivi, e particolarmente Piacciantco, il quale fino allora

8 era ben difeso, ma per lo spavento, che hebbe dei Gigante,

getto in terra,

s fingendosi morto, ed a poco a poco si condufie all' armadio della bortiglicria '

b nel quale entrato vi si erro.

si ATIOr«0, Divenuto fioco. Uno, che per catarro, o per altro impedi-
} mento aell' aspera arteria ha perduta la chiarezza della voce, li dice rancus,don-
y de rancedine, e reco. Dan, Int. C. 14.:

A, Erendele a colui ch' era già reco, £.
(| Li aria gli è maligna, 1? aria gli nuoce, gli cagiona danno,

1, dietra a casa, efuignua. Volta le reni, e se ny « E il verbo /ujznare,detto
rr GR.

Poco sopra nell' ovtava fectuma,

AT.



MALMANTILE® o
S' initende perquotere. 'Così I intende
. fo direi anche, maio temo, che ella

ny Won s apparecchi a grattarmi la tigne.

Si dice anche cacciar la mo/ca da deffo, in \cstan{20}, !
dajfar la lana, sopra C, 7, stan,63. Adandare a Legnaia,sopra C.
ter Ta poluere, sotto \cstan[12]{1}, E tutti hanno lo stesso signi

INFRAGNERE. Ammaccare, o pigiare una cosa tani
forma., come farebbe Pestare un fico maturo, ec, e il Lat. ¢/
Vedi sopra \cstan[4]{76}, e sotto in \cstan{17}.) *

INT ARLAT A, Rofa dai tarli, che sono quei vermi, li
dentro al legname.,, e di ele si nutricono; da i Latim detti rer
'C.6. stan. 59.

FRASTVONO, Fracasso. Sinonimi, che significano Romore, stre

NON gli rimase fangute in deffo. Acbbero così grande spavento, che

mate spirito, Dicono, che a uno, che habbia ha'vuro un granditimo sp
o paura, se in quel punto gli fule tagliata una vena, non gliu
per le ragioni accennate sopra in \cstan{2}, d

S' eva Jchermito bene. Cioè,s' ra difeso.. Havea scampato il toccatne

BADILALE, Grande, Si dice anche machofo, imperialc, € simili,
'scherzo; e significa grande più del naturale. Kose ee
VN bel fuggir falna la vita ancora, Alla (entenza che dice Un'bel morir tutta le
vita honora, rispondono coloro, che flin:ano più il vivere 5, oe

Sa ney

bony

Vo bei fuggir fainn In vita ancora, 7 ag
V" è mal viscontro, V' è male il modo. Non W'@ buona congiuntura, ~ io
VPEDE baker viso di sentenza contro, Conosce di non'haver ragione, cioè, che il Mt

'ncgozio non è per seguire, com' ei vorrebbe. A tthe? Wy
CAl ve vi sia, Chi ha havuta la disgrazia, se la ianga: E si dice: Obi v'é ui

vi sia, e chs non 0° è non v" entri, qui però intende; chi in quella stanza vistia, tt

perché non se ne può ustire. Reet eee
BOTTIGIERIA, Armadio,o stanza, ove si tengono V afi da Vino ' a)

¢ servizio della mensa. Voce, che vicn dal Francele Borteille, che è Cl

'fiasco, o altro vaso simile da vino. 4 4

STANZA XV. 7

'Col battaglio di nucuo agile, e presto o già ch' egli non può tt

Tira il Gigante, e da nella lumiera', etrmeggiar col bat: 'et lento, f k,
Ls qual cadendo fece del suo resto, 'Pero che il toga non ha gran diffanza, fr
Perché i [pense, e rope ciò che v' era; “Cagion ch' ei trowa fempre' mento; a
Hor, 8 eglie in bestia, dicanelo questo, 'Lafeialo andar bawendo pin fidana x
Mentre ch' ei da ne' lumi intal manera, Nelle sue manych' in simile Srumenb hei
E dice che'! Demonio lo jtafila, E piglia quells ciurma abbietra, e sbricia a
Poiché eli fa saltir due colpi in fila. 'eAA menate, com' anici im camicia yj
' STANZA XVIL. oe ee fg
'Così tutto arrabbiato, come un-cane Talche'l meschin non mangera più par ad
Piglia un pel coho, e feactialo nel muro, Perciò gli amics [uci, a
Di sorta, che disfatto ei ne rimane “We voglion, che il ribaldo,
Som' wie ficaccia piattalo matures Gli andaron alta-vien tush quant

Stet a. ac? ae


VNDECIMOCANTARE. 492
STANZA XVIIL STANZA XIX.

"sion costoro un brance di galletti, E come la mia Serva, quana' in fretta
Quando la state, a tempo di ricolta, Dee fare ilpesce a uovo, e che si caccia,
Antorne a qualche bica units, e spretti Trama due nova einfigmele picchiet:s,
non di loro a berricar s' afolta, Sicche in untempoturte due le(chiaccia

Pere il Gigante fa certi feambierti, Bs che dall' tra e spinto alia vendetta
Che re ne [uifa quattro,o se per volta; Sostien quei due,es' apre nelle braccia;
Insassidico al fin da quel baccano, Poryciacche,pacte insieme quello, e queste;

Si china,ed aggavignane un per mano, Stcche è diwentan prit che pollo pesto.

. Biancone con un coipo fracafia la lumiera, e spegne tutti i lumi. Nota che,
i se bene era di giorno, la lumiera era tuctavia accela, il che spesso aveiene in ta-
lioccafioni di veglie, che i segiivorl distratti dal gusto del ball,fanno mezzo
— senz' avvedersi, che sia pafiata la notte, Ll Gigante in collera lascia il
ttaglio, e comincia a pigliar quella gente, e bacteria per le mura, onde tut-
tian tratto gli corsero addosso, ma egl si difendeva, facendo di loro un gran
-maccilo.
LVMIER A, EB vno strumento, col quale si sostengono in aria più lumi acce-
si, che i Latini dicono Lychauchus pensius, luceraiere in aria.
FECE del sue resto. Far dei retto s' intende fipire la roba, la vita, ec. qui dun-
que vuol dire si speafero atfatto 1 lumi. <
B in bestia, B in collera., Dar ne i lumi, vuol dire entrar grandemente ty col-
Tera, dar nelle (candescenze; ed è lo steilo che dar nelle furie, ed il Poeta (cherza
 con questa metafora di dar ac' lumi, ed intende dare etfettivamente col batcta-
- glio ne i lumi della lumiera.

ail; 4L Dianol to feafiia, 11 Diavolo lo perfeguita; Gli e contrario.

IN fila, Vo doppo l'altro, senz' intramezzo.
ot CARMEGGIARE, Questo metaforicamente significa Aggirarsi, o affaticarsi in
ibs vano; e signitica anche ingaonarsi, per esempio: Tu armeggi, se tu (peri d' ot-

tenere, ec, ma qui e preso anche nel suo proprio signiticato di mineggiar lara;

gli Cnell' altro d' aggirarsi. —

wo) CWRMA. Genraccia vile. Vedi sopra \cstan[3]{76} e C. g. stan. 16,

ABBIETT A, e sbricia, Sinonimi, che figaincano vilitfima, minurissima gente,

A manate, Da i più si dice menare. Quanti a' entrano in uaa mano; e per la
grandezza della mano del Gigante suppone il Poeta, che fica moltiimi per vol-
ta, perché dice: came anici sn camicia, che sono anici coperti di 2ucchero, de i
quali con una mano se ae pigliauo le centinaia.

FICO piattole, E' una specie di fico detta così.

NON voglion ch' ci se ne vanti. Lo voglion gattigare, perch' ci non s' habbia a.
gloriare d' haver ammazzato quel loro amico.

». BlC-AQuafi da il Lat. Barbaro apica dal buono -dpex. Così chiamano i Conta-
dini quel monte di grano in paglia a mazzi, da loro così accomodato, affinché
si stagioni, pec poterlo cavar dalla spiga; deta da I Latini rrieict congeries. Da
questa voce bica habbiamo il verbo sdbicare per accamulare. Dante laf, C. 9.

Come le rane innanzi alla mmica,
Biscia per l'acqua si dileguan tutte
Per e alla terra ciascuna s abbua, Rrr2z~ BEZ-

SEE CERCA E



500

MALMAN TYLER: 1 v

BEZZIC ARE, MW beccare de i pollastrelli si dice bez
FA certi scambietti, Cioè contraccambia le percofie,

ra

Scambietto * termine di ballo, che significa mutanea'
INF AST IDITO da quel baccano, Klicndogli v«

si
sopra \cstan[4]{9}.

Allor Bieco non ha pite fofferenza,
E giura, che di questot: Bacchillone
Von andra al Prete per la penitenza,
Perch'ei vuol, chee' la faccia col bastone;
Ei fui, che di ral arme ban da teenza
Gite ne daran a una fanta ragione'
Così guida i fuvi ciechiyow' e il coloffe,
Accto gli caccin le mosche da defo.
STANZA XXL.
Eglino tutti quini fermi a tiro
Presso.a Biancone aun fiscbioco' bastoni,
Senza tramezzo alcun, senza respiro
We diedero un carpiccio di queé buoni,
Ed egli con un piede alzato in giro
Fa lor sentir, s' egli ha fodii talloni,
E mentre questo passa, e quel rientra,
'Con quel pedino te li chiappa, e /uentra,

'Bieco veduto questo fa vehire-i suoi Ciechi,i quali tutti in giro ini

la importunita. La voce baccano, che significa combat esett
piglia nel senso, che si piglia musica, festa » bordello, '

Quand! ecco rt veccbio Paolino

Ve anti

AGG AVIGNA, Piglia, es' intende cinger con la 'mano 'tu

glia, in maniera, che si possa tenere stretto con factiita,
PESCE a' weno, Vova fritte »0 frittata, che dicemmo sopra C. 9
s' intende propriamente la frittata, che dopo eer cotta, 0
ruotolo, pure nella padella; rifritca, e ridorta in figura “di p
ta pesce d'uono, La Compagnia della Lefina dice: La 'consner
antichi, i quali conrenti a' un pesce d' uouo di due woun al più

ClACCHE. Questa parola non ha verun significato', ma solo
no, che fanno l'uova, ed altre cose similt, quando si rompono, edil
ne serve pr esprimer quel bateere, che fa il Gigante di quei due hi
tr' all' altro, ed immita Dante, che nell' laf. C,32,dice:
LVon hauea pur dall' orlo fatto Crich
E seguita i Latini, che pure 'hanno a finta voce Tax,
come si vede in Plauto in Perla; dove per intender buie dice > Tax
meo. E noi pure diciamo tach, e pach; anzi le percotie da molti in F
cono pacche, come dice anche il noltro Poeca sopraC, 5. st. 47. Da
ta la parola Fiorentina dcciaceare,, che è lo ttetio, che' Pefeare
dicesi 'Pepe acciaccary; modestamente infranto, e Acciaceo sopi
do uno per così dire calpesta, e maktratta un'alero., ”
5 j

3¢ per

la hile elbetae,

a;
STANZA XXN

Aquat fa più cagon,cblT efti,e'
E ( perchegli e bizzarre) bam
Condotti com' ei suole,un par a

OveSalito a Petigion di

Vavol matel,ch'egis' ha de ee

T astando,owe il Gig

£ darel ccc ieP bocca

SEafi Si =F eo \&

oy
Wes,

ee
a

aie



VNDECIMO CANTARE. yor

affaltano co bastoni, e Paolino falito sopr' a i suoi trampoli metie i) suo
iuolo sopr' alla faccia-di esso Biancone, il quale però s' adira, e beltemmia

i suot falsi Dei. Pah
| BACCHILLONE, o Bacchiglone, E nome d'un fiume, che passa dalla Cita
| Vicenza, in Latino detto Azedoacus minor (econds Fra Leandro Alberti; ed ¢
ida Dante Inferno 15.-ove discorre d' uno, a cui fu permutato il Velco-
irenze in quello di Vicenza, che dal servo de' servi Fu trafmmutato d' Arno
one. Da questo fatto di Messer' Andrea Mozzi, che così si domanda-
Vescovo, o pure dal verso di Dante nacque in Firenze il proverbio; del
fanno testimonianza il Varchi nell' Ercolano, e il Borghini. Sacare d''e4r-
in Baechilione, aitudendo al saito dal Vescovado di Firenze a quello di Vicen-
y che significa saltar d'un proposivo in un' altro s Saitar ai palo i frafea: Ma
-questa voce Bacchillona aggiunta a huomo significa huomo insipido, e buono 4»
. oe » ancorché di persona grande; e suona lo tteflo, che Gaicone, Palamidonc,
i: » e simili,.¢ credo, che sia il medesimo dire a un! huomo Lacchillone,
scheCaftrone, e che venga da Bacchio, che in alcuni juoghi di Toscana vuol dire
we — agnello, e cos: Bacchi/one voglia dire agnello grade,cioè Caffrone. O pure viene dal
o | Lat. bacuius,quati Perticone, Scuriscione, O vero \& deo quali Baleceone; che si
»¢€ non fa niente dibuono, ne di ferio.

WON andra al Prete per la penitenza. Questo modo di dire usiamo per fare in-
'tendere, che ci vogliamo vendicare del oprufo, o torto fattoci, o che vogliamo
galligare uno di qualche mancamento commesso; quasi diciamo: lo medesimo
i dard la pena di questo suo fallo, (enza che egli vada per essa al Confefore sed

il Poeta l' e(prime dicendo: Perché vuol, ch' ei la facia col baffane.,
| AIANNO ficenza-di porter tale arme, Cioè hanno permiflione di portare il ha-
it scherza, peso ivciechi portano il bafione per necefita, per farsi lan

QW VINA fanca ragione, Gli daranno le bastonate,.come vanno date, e quella

pi |  WoCe Sama, se ben pare riempitura per emfali, nondimeno detta in questi termi-
sf ablignifica perfezione, quasi dica divera, e di tutta ragione, e d' intera giusti-
a Zia, che la voce Sanetus fiacopata da Suncitus vuol dire Nabilito, determinato.,

» Nov. 10. £ battnrala adungque d' una fanta regione, cioè.con una solenne ma-
niera; dateglicie delle buone. Vedi l'Orava 25. seguente..

GLI caccino le''mofebe da defo, Lo battonino. Vedisopra in \cst{11}.

SENZA tramezzo, e senza respiro, Senz' intermiffione di tempo, e senza pi-
igliare riposo.

NE dettero un-carpiccio di quei buon, Ne detterouna buona,'¢ gran quantità.
Carpiccio viene dal verbo carpire,-¢ pero vuol dire. manata., o manciata, e cence
Aeruiamo per intender quantità., ma per lo più di bufie, comel'intese ilFiren-
-2uola nell' Afin d' oro + £ poscia, che per nua volta gle x' hebbe dati un-carpiccia de

i

TALLONI + Quella parte del piede, che è tra la noce., e il calcagno,:ma qui
'piglia la parte per cueto il piede. Vien dai Latino Tans. C. 8, st..69.
 PEDINO, Deito ironico., ed.intende gran picde, pedone,

SPER



goz MALMANTILE

SVENTRA. Rompe, spezza, o sfonda il ventre,
attivo, che fventrare neutro ha il figaitca
PAOLINO Cieco. Questo fu un Cieco compo!
zonette, le quali si sentono ancora cantar per Firenze da al
azzi, e per questo il nostro Poeta dice: Fs più canzoni, ch
oeti celeberrimi del nostro secolo. Tali sue canzoni anda'
le piazze, dove per adunare il popolo faceva fare diversi
cani, ed egli medesimo, benché affatto cieco, e decrepito, 2
trampoli di legno a i piedi, Questitrampoli erano duc pertiche, in
ciascuna,delle quali era fitto un pivolo, e sopr'a questi dae pivoli fali
sopr' ad essi i piedi, e fostenendo la persona col rimanente di de
con adattarfele sotto le braccia,camminava con granditima franchezza
poli da' Latini si domandano Graiie, 'ccondo Nonio Marcelle; e quei,
minano su' trampoli, Gratlatores. Feito dice; Grattarores i
ni, qui, ut in faltatione tmitarentur agipanas, adiettis perticis furculas h
que in bis superstances aa similitudinem crurum eins generis gradiebantir
prer difscuteacem confiffendi, Plauto Vinceretis curfuceruas,o gallatorem.
DI cento scampolt, Tutto rappezzato; che scampole Jiciamo quel pezzo d
no, o drappo, ec, che al mercante avanza d'uua tela quasi pezzo,così
pato, cioè avanzato a far' un' abito 1nccro; e qui intende toppes o pezei
anno. ere a
. (MB ACVCC ARE. S! intendé coprire il capo, e ilwifo.. Vedi
si. 73. Varchi Stor. Fior, ub. 1.4 Subso fu preso, e smbacnceato col eapp
dotto alle carceri,
Sl scandolezza, S' adira. Vedi sopra C.
di scandolezzare e quel, che dicemmo sopra C.
BREZZA, Vento freddo; Vedi sopra C. 7. st. 18. ue
PAbP AICO. E' un pezzo di drappo incre(pato da una parte, e ridotto quai Ht
in forma di facco, quale portano in capo le donne per difendersi freddo, ed 'afl
oggi lo chiamano anche cufia, Mattio Franzesi in lode delle Malehere dice e » all

£Lvvi un segreto, che a noi dir si puore, vet ~ Yay
Che la mascheraé me' a! un pappafico, si
E pero si vente in van. cufola, e squote ty
Ed il medesimo in lode della Potta uso il verbo impappaficarft di aay
Chi ale tempse si fascia gli vechiati 'ake
Chi sopr' a i berrettin impappafica, ine
PORCO, Aggiunto a huomo vuol dire Schifo. ps a
0740". Intend, Che schitezza e questa? Vedi sopraC, 8.67 yy
ALLEZZA, Vedi sopra C.-3. st, 64. \& nota, che il verbo allezeare tantoat =”
tivo, quanto neutro ha lo stesso significato,; 3 sur (oy
SA di refe azzurro, Per tigncre in azzurro adoprano i Tintori ere
fetore orrendo, o sia galla, o sia guado', o uno, l'altro infiemes 1M

rimane per qualche rempo in su la roba tinta, e particolarmence in sul 1in0
pero dice quel cenciaccio fa ai refe azzurro, ed intende. Ha gran fetores'
verbo appestare ha lo fictio significato, e natura, che ha il verbo 4
di al detto C, 3, st. 54.

bee



STANZA XXIV.
levare intanto hawea Perlone
| La srane dal Gigante roninata;
Abe ancor quini ciondolone,

he la lumiera già tenea legata,
“Ed 4 foggia d! eAriere, o eMontone

7 nla addietro, e dannole l' andata

- Verso quel torvion, che si distese,

> STANZA XxV.
Hor' quando ( perch' egls sbalordito,
~ Etutto intenebrato in terra giace )
 LCieehi più che mai fanno pulito,
 Edegli se le piglia in fanta pace,
OB fra le maxe innolto.a quel partite
Un facco diventato par di brace,
o Eben quel panno al viso gli è dovuto,
—— Dovendosi si-cappuecto aun batturo,

lo stesso significato.

= =.
—

orribili Giganti.

| Col si più voite in bocca del Franzese, Perché quivinon è troppo-buon' aria.

VNDECIMO CANTARE 503

TI vuo' dar l'incenso con le peta. In vece di farti honore, ed incensarti, voglio
sprezzarti, offerendoti cose puzzolenti, come suol'esser il peto, del quale Vedi sopra C.\ 6.\ st.\ 100, Orazio. Vin tu Curtis Iudaeis oppedere?

STANZA XXVI.

Mentre gli rompon Poa, € poi gli fanne

Così t incannucciata co' randelli
E talor, non wedendo ove si danno,
Si tamburan fra lor come vitellt',

Gli altri soldati a gambe se la danno,

Ed ognun dice: alla larga seabells;
Euege la parte amica, e la contrariay

STANZA XXVIII,

Ma restin pure a rinfrescarle gli orbi,

Con quell? snfalatina di mazzocchi,
Ed et riposi all? ombra di quei forbi,
Che gli grattan la rognaco' lor nocchi
Mentre quivi per far dispetto aicorbi,
Sotto quel cencio tien-coperti gli ovcht 5
Che sugnun parte,ed io mi partoacoray
'Pen tornare a Baldone, e Celidora:

~ Con inucazione, e macchina di Perlone, il Gigante e atterrato, ed i Ciechi
'gli vanno tutti addosso col bastone, ed in questo grado lo lascia il Poeta, e torna
'a dilcorrer di Baldone, e di Celidora.

CIONDOLONE. Una cosa, che sta pendente da alto a baffo senz' esser ferma
'in verun' altro luogo, che dove è appiccata, come farebbe il battaglio.nella cam-
. » si dice far ciondolone, o ciondojoni dal verbo ciondolare, come dal verbo pen-
Gee si dice pendotoni, o penzoloni; da dondolare., dondoloni, che tutti hanno quali

ARIET E, o montone, Macchine,'0 strumenti bellici antichi, de' quali si servi-
| -vanoiper rovinare le muraglie; Sono notidimi., parlandone tutti gli Storici La-
“tin; ma particolarmente Giulio Cefare ne' suoi comentarj.
quel sorrione. Così è chiamato dal nostro Poeta il Gigante, perché
avanza sopra gli altri huomini., come avanzano i torrioni sopra lemuraglic; ed
anche perché servendosi dell' Ariete, o Montone, lo deve adoperare, non in un'
huomo, ma inuna torre, come è solito adoprarsi simili arnesi. Da questa gi-
'gantesca flatura, per la quale.essi sono affomugliati alle torri; fece Dante il ver-
'ho Torreggiare assai galantemente. Inf. 31. Vorreggiavan dé mezzra ta persona Gli

* COL si del Franzese in becca.'Gridando®: bud, but, che voce dimostrativa di
p| dolore, ed in lingua-Franzese vuol dire si. 4

¥ - SBALORDITO. Siordito, fuori del sentimento per le percosse ricevute..
¥) o INTENEZRATO, Si pwd dir sinonimo di sbalordito: e qui vale per intormen-
i * 'tito dalle percotie. Un fatio, muraglia, o altro simile materiale folido, e dura,

“Ai dice intenebrato, quando, per le peccolic., che se gli danno per romperlo., e 4i-

Bio,



504 MALMANTILBE. | ¥

dotto in termine, che dal suono si conosce, che si comincia.a
F ANNO puiite, Vuol dire Ripulire, ma detto in questi te
da vero, o perfettamente; E' lo stesso, che Fardi buono detto fo
SE le pigia in fdnca pace. Se le piglia con tutta, ed intera quiete. Ci
bastonare, e non si rivolta, ne's'adira. E la voce Santa ha la forza,
detto sopra in questo C, st. 20. ' olay
KINVOLTO fra le mazze. Coloro, che portano la brace a vende
ze, la mettono ne i sacchi; e per ammagliarii, e legargli sopra tie,
tatamente gli rinuojtano in alcunc imazze; ed il Poeta schereando dice
gante e simile a uno di questi sacchi pieni di brace, perché egli € rinu
mazze, e intende di quelle mazze, con le quali i ciechi lo bastonano.
BATTVTO. Chiamiamo Barrasi coloro delice Contraternite fecolari
proceflionalmente vanuo con velti line in dowlo, le quali chiamiamo saccht
fi queino vesti di penitenza ) cappe, o velti da bactu,, cioè, che f bane,
si disciplina, ed il capo, e faccia coperta con un cappuccio appiccato a detta
vefle. Ed il Poeta scherzando con l'adicttivo barrute, cio bastonate, e col sa
stantivo barruto, cioè humo di Confraternita, dice, che ai Biancone stava
il Cappuccio, perché era barrato; e per cappuccio piglia quel ferraiuolo, che
lino Cieco havea meflo in capo al Gigante. '
INC ANNVCCIAT A co' randelli. A coloro, che si sompono braccia,gambe,
o cosce, ec, Nel raflettare tal rottura, «fhuche ' off Rando fermo al luogo,ac-

comodato si rappicchi, fanno una fasciatura con pezzi d' assicelle, o stecche, la i
gual fasciatura chiamano / incannucerata, e pero dice, che, havendo rouse | offa U

al Gigante, gli fanno hora l'incannucciata co' randelli, cioè con quei afloat, Dk
0' quali lo perquotono. 37h Une)

s1 tamburano come vitelli. Si bastonano ben bene. Quando i Macellari hanne Ni
ammazzato un Vitello, o Bue, ec. lo gontiano, ed acciocché il vento pall Da
da per tutto faccia spiccare la pelle daija carne, bastonauo la bestia con alcune>

f
mazze, e questo si dice tamburare » o tambu/sare, che vedemmo sopra C. 2. AL34. ie
ed a questo ramburare assomiiglia le bastonace, che si danno fra loro i Ciechi, whe

wuol dire molte, fode, e spete « Sidice samburare, perché date in quelle pelli di si
Bue, ec. gonfie, fanno il suono simile a quello del tamburo strumento 3 i

E per altro ramburare uno vuol dire quereiario; e questo perché anti

Firenze 4 tenevano in alcuni Juoghi pubbiici de' Magiltrati certe;

hi da chiunque si voleva, erano meile le denunzie legrete, € queste calle C
vano tambari, e da essi tamburare, era il medesimo, che acculare, o quetelare.
Vedi gli Staunti di Firenze al libro intitolato. Ordinamenta snspicia contra Magnare
(citau aicune -voite da Gio, Villani ) al capitolo, ove si tratta del mettere nel tam~
buro. ieee
ALLA larga sgabelli'. Allontaaiamoci. Quando dopo la cena si fa balla, 0al-
tro passatempo simile nella medesima stanza, nella quale s'è cenato, che 1 com
mensali si rizzano, e per dar luogo si fanno levar via le tavole, le seggiole, e Blt
sgabelli; ed ogn' altro, che potetic dare impedimento, si suol dire: alla

belli, e s' intende; si levi di mezzo ogui impedimento; il che è in ¢
che significa; facciafi ala, o si taccia largo, ma per lo più s' 1

HOEZBE SE 2E x PRZ



VNDECIMO CANTARE. 505

er ae: ' [
troppo buon aria, Li gon' y'é buono tare; Intendi: v'é pericolo di
- MAZZOCC H!, Così chiamiamo i Talli del radicchio, ne i quali nasce il se-
de iquali si fanno infalate, che sono rinfrescative, ed il Poeta, (cherzan-

'con I equivoco dimazzocchio., che vuol dire: bastone, dice che con questi

1 i taano al Gigante l'infalata per rinftescarlo, ed intende; le hastonare,
SURAT. 1 bastoni de'Ciechi per to più sono di sorbo,o a' altro legnaine simile
chiuto, sodo.,.¢ grave, e dicendo il Poeta; Si riposf alPombra di quei sorbi, che
i grartan la rogna co' lor nocehi, intende + si riposi sotto quelle bastonate de i

Ai “@ t
o PER far disperco a i corbi tiem coperti gli ccchi, Per fare Mizza a i corui per la.
» che hanno di non poter beccare, e cavare glivocchi al Gigante, poiché gli
, o difesi col mantello di Paolino cicco,
PANZAXXVIIL, STANZA XXIX,
Che la-nel mezzo a's suoi nimics comba Su via figlixoli; orto buon piceini,

“Di moda, ch? essi foeman per bollire, Faccian di quepti furbiyun tracto,ciccioli,
Che dove i colpi ella indirizza, e pidha, Nim remete di questi spadaccins,
Te sli manda in un subito a dormire, C' alcimento non vaglion poi tre pictiali;

“Che ne meno col fhan della (un tromba

¢ Es in vista.vi paton Paiadini
NCamprian eli fardbbe rifentire

Han facce di Lionije cuor di fericciell;

|B quamo brava, similmente accorta', Efel-gridare, e ilbravar lor v' assorda,
ait Acombatrere i suoi così conforta, Ut can chabbaiayraro avyien che morda,
ya ov Deferive laibravaca, e prudenza di Celidora, e riferisce  orazidne da essa fat-

z Pe inanimire i foidati, la quale € veramente appropriata al personaggio, che
ae::

~ ZOMBA, Perquote ». Vedi sopra C, 6. st. 104.
| SCEALAN per bollire, Vuol dire sminuiscono, e quell' aggiunta per bollire, si
'poneiper un costume introdotto da un quoco goffo, e ghiotto, il quale havendo
mMeflo'a quocere Ieile alcune merle, se ne mangid più della meta, e portate il re-
A 'in' gli domando il padrone, che cosa havea fatto dell' altre merle? ed
“il'quoco gli rispose; Sig, sono scemate per bollive, E da questa goffa astuzia quando
diciamhoe Laral cofaé (cemara per bodire, intendiamo, che una tal cosa e (cemata
assai, senza potersene ritrovare il conto, o sapersi la causa del mancamento.
~~ PIOMBA.. Precipita'; lascia calare, o calcare il colpo.
"LA tromba di Campriano, Questo Campriano fu un contadino astuto, come s'è
'accennatosfopra C. 4. st. 47., e Come si vede dalla sua favolofa foria ttampata,
€Ol titolo Storia di C ampriano, 11 quale per'far denati trovd diverse inucnatoni di
gabbare le persone semplici; e fra l'altre quella d* una pentola, che bolliva fen-
(2a fuloco yperché da efio levata, mentre gagliardamente bolliva, € portat®in,
~mezzo a) una stanza, la fece vedere al corrivo, a cui voleva venderla; costui ve-
dmala veramente bollire, senz' haver fuoco avanti, subito se ne inuaghi, ed ac.
» Sordosi di compraria per il prezzo, che convewncro. Giunto poi questo tale a,
casa con la pentola, e volendo senza fuoco farla bollire, e non gli riuicendo, si
~ quereld con Campriano, dicendogli, che I ne ingannato; Campriano chiamé
ss la

SU OS eae st ey



506 MALMANTILE

la moglie, e la sgridd, dicendo, che non potev esser 5

cambtata, La donna fingendo un gran timore, con gran la;

per haverla inavvertentemente rotta, glien' havyeva data un' a!

paura, che havea del marito. Di che Campriano mostrand

to, cavo fuori un colrello, e con esso feri la moglie nel petto

ascofa sotto i panni una gran vescica piena di sangue, il quale fg

che uscitie dalla terita factale da Campriano; per la quale fingendo |

fer morta, calcd in terva. LU gonzo si doleva, che Campriano per e C

gicra havetle commedo un delitto così grave; Ma Campriano con facia.

£'i die + Sc ben la donna € morta, 10 1apro rifulcitarla, quando vorro

basta, ch' io suoni questa trombetta; e stimolato dal fempiice a far

piacque » e fonata la tromba, la donna ff rizzo, mostrando di rifalc

il (emplice con grand' instanza chiese la tromba a Campriano, il quale d

te preghicre a gran prezzo gliela vendé: Costui andato a casa prefe o

gridar con la moglic, ed in fine le diede una pugnalata, con la quale

€ poi si messe a (onar la tromba, ava quella infelice elendo veramente morta,n0a

rifalcitd altrimenti, B per questa causa, e per altre sue sei. aggini fu Cam:
riano condaanato alla morte, che dicemmo sopra C, 4. st. 27, E di questa trom-

Es parla i) Poeta nelprefente luogo. sot then is ERE
SOTTO buon piccini. Esortazione, che si fa a' cani, quando s' incitano,o am- —

mettono contro qualche fiera, come vedemmo sopra C. 2. £.87,; ed il

si sostiene sempre in su le burle, fa che questa Capitanessa esorti, edit

suoi soldati con questi termini da cani.: - 1

cicciold. Frammenti di grasso di porco, che avanzano nel tegame,o altro

va(o, quando i fa lo strutto, o lardo, da alcuni detti ancora dardings, ficche> — jy

vuol dire facciamo di costoro minutissimi pezzi. Cicciolo diminutivo, che vieoe> (
da Ciccia; la quale nel linguaggio delle Balie, e de” fancimili vale apprefodinot
Carne; siccome appresso i fanciuili Greci Tria. ei Whi
SPADACCINI, Così si dicono per derilione coloro, che portano laspadas =p,
solo per pompa. juin seemnigadgtit Une
PALADINI, Cioè Conti Palatini, Quegli huomini bravi, evalorolidifran- —j,
cia cantati dal Boiardo, dall' Ariosto, e da altri; e da questi dicen ty

Mena (e mani come un Paladino, intendiamo buome valorofo; poiche t O tay,

do. Cosisappresso gli Antichi,Ercole, e Achille si veniva a chia a
rofo, e dicevano: editer Hercules, e di Lucio Sicinio Denotato agg
mano bravissimo, riferisce Gellio \libcap[2]{11}., che per la gt: eras Oy
appellato Achilles Romanus, Di questi Conti Paladini, '0 del Palazzo intese il ei

Petrarca nel Trionfo della Fama Cap. 2, ro) Ne
Cingean costus + uci dodici robusti, + ' d } yy

FACCE di Lioni, ecuor di scriccioli, Mostrano d' esser bravi', ed animosit te

codardi. Lo scricciolo essendo il più piccolo uccello, che ti trovi, ha per conseguenza
il cuore piccolissimo, ed huomo di piccol cuore s'i huomo timido,

e codardo. Vedi sopra C. 10. st. 30, Latino pari, o angu(ti anim Mi-

eropsychvs«

4



I ee

VNDECIMO CANTARE.

597

eet. di rado morde:,. Chi fa molte parole, suol far pochi faci. SE

ape

Suol far poche parole.
STANZA XXX.

bb wel ch? Ella da ritto, e da rovescio,
— Condicende, va fenando a doppio,

Da sul vifaal Cornacchiann marove/cio

Cun mighio si senti lonan sale 3
et =
anc' egli eA cantocneall pio,
| Mail fapor non gusto già de' buon oo
Come chi prefe il suo de' cartoccini,
» STANZA XXXI,
Sperance per di id gran colpi tira
Con quell' “infornapan della sia pala,

We barte in terra,sempre ch' e la gira,

5 shafiti per la fala,
Tal che ciascuno indietro si ritira,
»O per franco schifandolo fa aia,

Bhi l' asperta, come bavete inteso,

' ek elon Ai) ie i pf

Perch Alsicardo, c' al pafsol? attende,
4 gozz0 gli trafora col pugnale
Ete lo manda a far le sue faccende;

ANZA
r ome il fuggir questa volta non gli vale,

proverbio con dire, Caneche murde, non abbaia s' esprimera la

Curzio:, Aleffima queque siumina minimo labuntur sono; ed. anche
Polidoro. Vergilio: Cave rib
lontano il detto di Catone

se flefic sentenze,habbiamo in uso. anche nel parlar nostro dicendosi:

a @ acque chete, Guardati dail? acque chere; Chi far di farsi vuole;

acane mnto., \& ab aqua filenics A
1 Demiffos.animos, tacitos vitare me-

STANZA XXXIL

eAmostante, che vede tal fiagello

D? se! arme non usata più in battaglia,
erica la spada, e quando vede il bello,
Tira unfenditeein mezoglicla taglia;
Riman brusto Sperante, e per rowcllo
Li resto, che gli auanza all aria scaglia;
Vola il trovone, eil Dianol fack' eicaschi
Sula bottiglierra tra vetri, e fiaschi,
STANZA X\&XLIL

Dalle diacciate bombole, e guaktade

i vino sprigionato bianco, e rosso
Fugge per b asse, e dann felso cade
Giù dow' è Piaccianteo, e dagli addosso:
Ei che nel capo ha sempre stoccht, e spade,
04 quel fresco di (ubita riscosso,
Pensando sia qualche spada, ocoltello,
Si lancia fuora,evia farpa fraselle.,
XXXIV.

Così dal gozz0 venne ogni suo male,.
Per tui fadi, per lui la vita spende;

E vanne al Diavolche di nuouo piccalo,
A ustolare a mensa appie di Tantaio,

-Celidora esortando i suoi a combattere non lascia di menare le mani; Si nac-

tano diversi avvenimenti, e la morte del Cornacchia, e di Piaccianteo.
SVONA a doppio, Intendi perquote incessantemente. Suonare a doppio inten-
do tucte le campane, o la maggior parte di esse, che sono in un
campanile » fyonano insieme.

Vedi sopra C, 6, st. 107. Sonare per percuotere,,

il Boccaccio Novella 67. E alzato il bastone i cominciò a sonare. Latino

are.

 MANROVESCIO, E} quel colpo, che si da col braccio all' indietro,cioè con la
|p conuefia della mang, e da quella parte con bastone, o altro, che s' habbia

in mano,
ako scepi se h seid Meneoee un miglio, Il romore si senti molto da lontano, Ioke:

roposito.

een rere sopra C, 3. st. 21.
SICIIANDO un fempiterno aloppio « qu 20 alloppiarsi, o pigliar L oppia;

o cor-



. | ee
so8 MALMANTILE 10%
o corrottamente'? alloppie vuol dire addormentarsi da Opi
Sicch€ qui intende, che prefe un sonnoeterno, cioè mori.)
Oui dura quies oculos, \& ferrens unger Somnus; in arernam gli
Dice; che per —— ¥ oppi rein perché It haveva dato.
tempo, per mostrare, che quis peccat,per hac torquetar,,
di Pincha, che per causa dab guacnisoeetipibans F c
zo muore, " see

INFORNAP ANE, Cioè la pala da infornare il pane,
per arme, oh obi
SBASIT!, Morti, VedifopraC.2. f. 79.0 a a
FA ala, Fa largo; fa piazza'. 'Latino Viam prebere 3 win decedere, fun
HA finito il peso. sa fivito di fare quel, che gli era stare ordinaro; ha
compito; € s' intende ha fino ta vita: Metaforico di questa porgione di
che si da alli bactilani dali loro Capodieci di tance libbre@i lana, che
vorare, la qual porzione chiamano un peso, e dicono haver finito il peso
peafum, quando hanno finito di lavorar quel tanto', che era stato loro daro.
QUANDO vedde il bello, Quando vedde il destro; il tempo a proposito.. >
REST A brute. Kiman bettato, essendogli avvenuto quello,.che egh non s'al-
pettava, nel qual caso il viso resta macchiato di tristezza'y € i.
confusione.. We
SOMBOLA. Vedi sopra C. 8. st. 44.,.
FESSO. Fetura apertura di legname, o d' altra materia, o
vasi di terra cotta, Latino Rima, ' ' 3
WEL capo frucchi, e spade. Dubita, che tutto quello, che egli sente, sieno ar-
mi per l'immaginazione depravata della paura; per la quale  rifeofo
tremore, che viene per qualche accidente inaspettato; 'che. ci cugioni

per lo spavento, ches' abbia di qualche cosa improwvifa. Vedi fo he
C. st.2. se RitEe '
SARPA. Se neva. E verbo marinarclco. Latino foluir, anchoram vellit. Bog.
l' aggiuata della voce fratello è posta per emfafi, e quali per un giuro "g hl
LO manda a far le fne faccende. Lo spediice. Quis' intende 'ammazzay — te
PIANT ALO a ustolare. Latino ardere, inbiare. Lo mettevaliato a Tantalo 2 i
desiderar ancor' egli il cibo. Ed usuiare è latino; 'quafi dica + re dal ri
desiderio d*haver quella tal cosa, che egli vede. 'Ovidio negli Ai ¢
indomitis ignem exercentibus curis Fertilis, accenfis menfibus arder %
proposito ci feraiamo anche del verbo spirare. Vedi sopra C. 1, A. 31, diciamo ch
anche Vrolare; particolarmente de'cani, che fanno col mofo atte vie
vande, € per così dire le mangiano coal occhi, € col desiderio. 
TANT ALO. E' nota la favola di Tantaio hglivolo di Gioves-e di Plotenin'2, 7
il quale per far prova del valore degli Dei 'gli convitd, € diede loroim tavola cot. i
to, e spezzaco un suo figliuolo detto Pelope; Ma gli Deis' astenaero op
cibo, eccecto Cerere, che mangié le (chiene, le quali le furono poi Fit 1
Dei, che lo fecero rifalcitare, e confinarono all' Inferna T: r be
cendolo patire di concinova fame, e fete, per thaggior suo te
'metcere sopra il flame Ereditaao, che moltra acque doiciifims,a!;.


VNDECIMO CANTARE:; 509
1 felabbra, ma non tanto, che ne possa bere, e sopra alla testa ha un'
albero-carico di frutte bellissime le quali s' allontanano quand' egli s* allunga per
'pigliarle' 41 nostro Poeta, che-ha de(critto Piaccianteo per un' huomo golofo

'y che morendo,egli fara confinato all Inferno, € per questo suo peccato di
ola fara mefio allato a Tantalo a /stare anch' egli, come fa Tantalo,vedendo

ha da faziarsi, e che non possa haverla. Bologninus. i
Tantalus bic etram fitiens potare vetatur,
Ha 'a quod Pelopis Dijs epulanda dedit,
quali Omero nell' 11. dell' Viiflea descrive la pena di Tantalo, tradot-
Latini suonane così:
Stat miser in medio; medijs exardet in undi

Tantalus,\& fruftra circumfert pallidus ora,
Proximus illudit mento circumfluus humor
Et prope Yorantes contingunt corpora grtra,
Et crines,@ barba madent a/pergine crebra;
Dumque undam captar fitienti Tantalus ore

STANZA XXXV.
Era un camer ata un tal Guglieimo,
Cha la labarda, e ifuci calzonia strisce
Virbigonicinolohaincapo in vece d'elmo,
E'tutto il reffo armaro a stocchefisce.
» “Alemnnno è costui Perneiter scelmo,
» Econ quel dir che brava,ed atterrisce,
Sbruffi ferenti (earicando; e rutti
Ln un tempo spaventa, e ammorba tutti,

STANZA XXKVL

Humoremque cavis, fentat tomprendere palmis.
Hen /upito, ben longe fugitura recurfitat unda,

STANZA XXXVIL
Perché voltando il ferro della cappa

Verso Alticardo a vendicar [ amica,
Quei ghetascafa, e glittra sotto,e'l chinppa
Con la spada meixo del bekico,
Ond'sl vim pretto in maggior copiascappa,
Che no mesce in tre dil Inferno, e tl Fico,
Ala non va mal, perch'e: caduto allotta,
Hentre boecheegia tutto lo rimborta,

STANZA XXXVIIL

1 Costud a quel ghiortone a tutte Uhore Gira Sperante pegeio a' un mulino

Fu buon compagno a ber la maluagia, Perch'arme alcuna in manpiiend.gl resta,
rer non cadere adesso in qualch'errore, Par trova un tratte un pie d'un tavolino,
yi E far' un torto alla cavaleria, £ Ciro incontra, e gls vuol far la festa,
a] Pur'anco gli vuoi far,mentre chrei muore Aa quei preso di quivi un sharagline,
, Con farsi dar due crocchie, compagnia, Voa casa con esso a ini fain refia,
E non duri molta farica in questo, Perché passando ? osso oltr* alta pelle,
: ~ Chet trove chi spedito'e bene, e presto. Nel capo gli raddoppra le cirelle.,
. Seguitando il Poeta a narrare gli accident occorsi in questa zaffa, dice, che

Alticardo ammazz0 Guglielmo Lanzo, che volle seguicare in morte Piaccianteo,
come l'haveva seguitato sempre all oiterie; B Ciro Serbatondi ammazza Spe-
rante, con battergli un tavoliere da giocare a-sbaraglino in su la testa,
GVGLIELMO Tedesco. Fu questo Ledelco Soldato della Guardia pedestre del
Serenitfimo Gran Duca, la quale e composta d' Alabardicri veltiti a livrea con,
brache larghe fatte a strisce paonazze, e role, e si chiamano Linzi. Vedi fo-
pra \cstan[4]{4}. E perché questi non portano ferraiuolo, o cappa, diciamo per
ascherzo ferraiuolo, o cappa quella labarda, che portano in spalla., come vedre-
me



gre MALMA NETLLTW

mo appresso stan. 27. e s'è accennato sopra €. 9. stan. 48..€  r
date, o percofie colla jabarda. Costui era molto amico di-

aiuto a mandar male la roba, e però il Poeta dice, ch' ei lo vuol

in morte 2) La OORT

BIGONCIVOLO.. Diminutivo di bigoncia, detto sopra \cstan[10]{7}
costui con un bigonciuolo, arnese, che per lo pi s' adopra al vino,
che in tutte le (ue operaziont egli haveva l'animo al viao, e con
( che vuol dir pesce bastone, vivanda assai usata dai Tedescht ) per m
alla voglia del vino haveva unita ancora quella del mangiare. Si. ar
ancora, che il Poeta voglia mostrare, che costui era fudicio,.¢ c
in effetto egli era, e come per lo più sono questi Lanzis a causa forse di
pesce, che veramente ha sempre malo odore.

BERNEIDEK Scelm. Voci Todesche le quali in nostra ipa suonane
cone, scellerato,

ATT ERRISCE, Spaventa. La pronunzia Todesca ha un certo accento,
fa credere, che colui, che parla bravi sempre, € per questa rozzezza di 'al
gua dicono che ella sia propria, ed il caso a comandare eserciti, come la Fran-
ccle a aoe con dame, la Spagnuola al comando politico, ie cuaanaraerey
queste cose Pr

SBRVEFL, BE? quel mandar fuori per bocca il vento jonato in carded
prabbondanza di ae E ratti si ie dire lo stesso, aso che per rasse inten
diamo il puro vento, e sbruffo si dice quando il vento vicn fuor del corpo «
no firepito, che non viene il rutto, ma accompagnato con un poco sae;
¢fiendo lo cheafare un mandar fuori di bocca con violenza vino, o altro i
AMMORBA, Fa putire. Vedi sopra in questo Cant. stan, 23. quie pe

significato attivo, cioè appesta; mette la peste in tutti. '
GHIOTTONE,, Gran go.olo; Gran ghiorto. Intende di Pisesbame a

MALV AGIA, Specie di vino assai noto; ed a noi viene di Vi qui ie
pigliando la specie per il genere, intende che gli fu sempre compo be a tain
sorta di vino.

CROCCHIE; Percosse, Da ereechiare che in significato attiyo vuol dire P
motere

2 SPEDILLO bene, e preso. In poco tempo gli diede buona sp t 7"
ammazzo presto, ed affatto. Questo detto bene, e presso era il mol 3 7
cademia Fiorentina detta de' Rifritti, ed il Poeta se ne serve, p più
fu già di detta Accademia, ed immita un' altro Poeta, che nelj' umprovvila, o 
byona morte d' uno pure di detta Accademia difie; 'aban bar
E per mostrar, come Rifritto ville, pide eh e to

Mor:, come Kifriteo E PRESTO, E BENE, Ca

EEE e il Fico, Sono due Osterie di 'Eirenze così nomina' die oo i;
Infega

Bosc HEGGIARE. Quel moto, che fanno con aprire, e (errage la bosaia
mandar fuora gli ultimi spiriti coloro, che muoiono 4

LO rimborra, Rimette nella bors 9 lO¢ in corpo, 'ribeve -
che gli era ulcito di corpo.



.

VNDECIMO CANTARE) Sat

GLI vel far la feta, Cide lo vuole finire, lo vuole ammazzare '
GLI fauna casa in testa. Nel giuoco di sareg eee una casa,vaol dire rad-
iar le girelle, o tavole sopr' a uno de' 24. segni, che sono nel tavoliere, cd
i. scherza con questo addoppiar le gireile con dire che batrendogti il ta-
 voliere in cesta gli raddoppia le girelie, che quiui haveva, e così gli fa una case,
- intesta, che haver girelle in testa s' intende tuomo col cerucllo che gira. Vedi

\cstan[9]{10}.
STANZA XXXX.

-) STANZA XXXIX.
Ritvasse già Perlone un certo Marte, Tofelloych in fere.ra ad bxom non cede
Riesce adefio qus tutto garbato,

o Cthaucua il naso da fiurar poponi,

| E perch ei nol pago mai del ritratto, Perch'ei rifana un zoppo da un piede,
Pere fa seco adesso agli fgrugnoni; Cregnor fu quella parte andò seiancato,

| Edieglien' un si forte. ch' in quell' atto Mentre di taglio un sopramanglidiede

Gli si fhianto la firinga de' calzoni, dn quel, che sano havea dall'alrra late,
| Che qual tenda calando alle calcagna Che pareggiolio, ond' ei fu poi di quei
 Scopri scena di bosco, e di campagna, Che dicon: qui¢ mioye qua vorres,

% STANZA XXXXIL
Grazian di sangue in terra ha fatt'un bagno Che vie da un trcbettier di Carla Atagne
Onde gli è ya 4 chi va gin che nnoti; Quando le molfe dar fece ai tremors;
 Afetta un Salta, e xn Birrocolcopagns Toglie ad unl'asta,tl qual fail Paladine
1 E frroppia uneal, che fale erucce aiboti, Se ben con essa fu spazzacammino,

“Seguita a narrare varj accidenti occorsi in quella zutla, e le racconca le bravu-
re di Tofello Gianni, e di Grazian Molletto.
SU ffianto la firinga de' caizoni, Si roppe la stringa, cioè quel legame, che ferra
calzoni in sulia pancia.
TENDLe4. Intende nel presente luogo quella tela, che si mette d' avanti a i
chi »sopra i quali si rappresentano Commedie, affinché cuopra le scene per
Doprine nel dar principio alia Commedia; Lat. siparium, e però dice, che i snoi
calzoni. essendogli cascati,, scoperfono scena di bosco, ec, cioè quel, che da loro
'eraycoperto. Caso veramente seguito a Perione, che,per voler ae pagato d'ua
Fitratto., che egli havea fano a uno, gli conuenne fare alle pugna, ed ia quel
re gli cascarono i calzoni.
SCIANCATO. Uno, che va zoppo per haver difecto nell' anche, osso principale
delle cosce. Vedi sopra \cstan[6]{82}.
. CHE dicon; quie mio, e qua worrei, Così diciamo di quelli zoppi, che vanno
a gambe larghe per disecco, che habbiano nell' anche, o in ambedue le ginocchia,
€ non posano i piedi in dritto, secondo J' uso comune, ma pare, che vogliaao
can un piede andare in un iuogo, e con' altro in un' altro, e che accennino qui
 mio, €qua vorrei, Di questi tali diciamo ancora Andare a feiacquabarili, perch
fanno lo stesso moto con la persona, che fa uno, che (ciacqui un barile «

AFFETTA, Taglia da una parte all'altra, come si fa al pane, del quale
propriamente si dice affettare, o far fette.

VN Sata, Si chiamano Salti quei famigli, e donzelli dell' Arte dell' honesta
“(che in Firenze € il Magistrato, al quale son fottoposte le Meretrici ) i quali fanno
ogni sorta d' cfecuzione tanto Civile, quanto Criminaie contro le Meretricé,
t VN

me

i ee


le figure di carta petta:, le
di boto, e d' haver ricevuto ae
cono Bari. Vedi sopra C, 4. tt

sente 'uogo il nostro Poeta,

5 tz MALMANTILE | (¥
VN tal che fale erucce a boti « Intende' uno seultore dappoed, che.
qnaii @ mettono alle Immagini sacre.
razia; e queste figure:co
c. £7. Gruccia è dal Lac, barb:
€ bastone fatto a croce; onde in alcuni Juoghi della '
Far le grucce a una figura, s intende fra i pittori.
stan. 27, Intendi dunque, che costui era scultore stroppiatore dit
fabbricava se non faacecci di carta pella, formati confornie di gi
no di quella bellezza, che può vedere?chi andra nelle! Chiele
miracolofi; e queste figure faceva così male, che le strop
da sapere che seultor da bori suona fra gli scultori lo stesso, che fra i
Pittor da fgabelli, dewo sopra o, 4. stan, 10, Questo tale ancorché fulie:
¢ nato d' intima plebe,  ttimava un Buonarruot; essi piccavaidi nobile
dice, che ven da un tromberta di Carlo Adagno, quandevle mosse dar fac
Cioè ha origine da un trombettiere,dei
re.i bandi, che dar de mofe a' tremori, vuol dir comandarfo

ticamente, se bene in deco scherzolo, e per derisione, come se ne serve nel pre.
> aa

aria
java affatto e Ino

quale:Carlo Magno i serviva per manda

Ae

SPALZZACAMMENO. Vanno per Firenze alcuni o Marchigiani o Lombardi
con una pertica in spalia gridando: Spazzacammina, acciocthé pia
che essi ripuliscono le cappe, o gole de i cammini dalle filiggine » Vino:
tait era cului, il quale con queli' alta, clod con ta pertica tr
ladino.

STANZA XXXXIL
Tutto tinte ne va Puccio Lamoni
Stoccheggianda nel merzo della Vuffa,
£ in Pippa un tratte da del Castitioni,
Che majcherato ancor tira di buffa;
Ea ci che nel sentir quei farfallont,
Venir più tosta sentesi la musa,
Passandalo pel petto banda banda
Ai far rider le piattole lo manda,
STANZA XXAXKTL
Nanniruffa ha più la pien di ferite,
Pericolo, che fu scopa mestieri,
Fu pailaio, Senfale, etitor di lite;
Srette Bargelioy ed abbaco di xeri
Prefel appalte alfin dell! acquavire:
Ala pris fuaniro i fuvi peafieri,
Lon pite il wana fhillando, ma il cernello
Per mettervi poi il mosso,el'acquerelio,

Continoya a narrar quel, che segue neheombattimento,

mazzamenti,

TVTTO tinte, Vuol dire adirato, ma il Poeta si serve di
ché detco Puccio è di faccia bruna, come s'€ detto sopra C.

6 OP
STANZA XXXREV) o
Con Duriano ii Purba eccoalle wank o
Di ferro da fradceri i Safe,
Ev altro una paletta tq
E con est a tui cerca, e sbracia
Ma perché quei le fqnete, tome cani
si fraricatt fuaf hs chibufo, (4

Chreghi ha a' Monnini,evane

Fatto d ognun polpette
S' 4 tanto mal non se:
Col dar sul grifo-a tui Salue Rofata,
Chef. oui 2
Vuol ch' e facia pere

Cb? essendo presa
Lo spinge fuor

«=. FREER

=a.



wii =

a a ae

Pre Sst = = -

Sate Sb tes eee

VNDECIMO CANTARE 513

 TIRA4di bufa, Fa i buffone. Le buffe, come accennammo sopra C. 2. staa.
2. alla voce bu/chette, sono pezzetti di mazza rifessa, e formano quaai un dado,
se non che hanno te parti piane, ed una conuefia, e si tirano come idadi, fa-
eendo Con esse quei giuochi, che si resta d' accordo con sei, orto, o più di casi

 buffe; e per me stimo, che s' usino, come s' usavano dagli aatichi gli aliogi: ma

: è i e giuoco da fanciulli,percio habbiamo il detto sirar ds buffayche vuol
ire Far cose da fanciulli, ec. da persone di poco giudizio, che poi da questo in
una parola si dice buffone, e far il buffone; che i Latini dicendolo scarra lo delcri-
vono per uno, che rifum ab audientibus caprar, non habita ratione verecundie, aut di-
gnitatis, o così per uno, che non habbia l'intero giudizio da distinguere i tempi,
ee wetl ne le persone, come e per lo più il giudizio d' un fanciullo. UI P.
', Vincenzo Maria Carmelitano Scalzo nel suo viaggio all' [adic Oricntali lib. 4,
c.26. descrivendo un' uccello detto Buffo [ che è forse.quello che i Latini Bubo,
€ noi chiamiamo Gufo } dice così,, I nottri antichi lo chiamaron Buffo, onde:
y» forse hebbe origine il nome di buffone, poiché è incredibile, quanto questo
a uscello sia inclinato agli scherzi, ed alle burle, con le quali bene (peffo atcer-
y rice di notte, ed inganna la gente.
|. BARFALLONI, Denti spropositati, e sciocchi.

SENT ES! venir la muffa. Si sente venir V' ira; Entra in collera.

LO manda a far rider te piattole, Lo manda a far il buffone nell' altro mondo,
dice le piartole, perché questi son vermi, che stanno negli aucili, ed hanno oc-
¢afione di rallegrarsi per 11 nuovo cibo che a lor viene dall' andar egli nell' avello,

PERICOLO, che fa Scopamestieri, Si dice Scopamestieri colui, il quale seguita
poco tempo a far un' arte, ma lasciandola stare ne vada a fare un' altra, perché
la prima non gli piaccia,come appunto fece questo Aleilandro Violant detto
Pericolo, nominato sopra \cstan[3]{58}. il quale veramente fece tutti i mestieri
enunciati nella presente Ottava 43. ed in ultimo si diede.a trovare invenzioni di
Mettere appalti; cominciò dal Tabacco, e poi l'Acquavite, i quali senza suo
utile, o pochiffisno conchiufe per altri. Dice, che abbaco di zeri,perché veramen-
te ci fu un grandissimo abbachista, e per questo havendo saputo trovar degli erro-
ri. contro a' ministri grandi, fu da essi perfeguitato si, che fu mandato in gaiera;
Ma havendo le notizie date da lui fatto al fine (coprir la verita, furono i delin-
Foe gastigati, ed egli cavato di galera. Dice abbaco; ma perché questo verbo

gnifica ancora star dietro a fare una cosa, e non trovare la via a terminarla,
per non haver tanto giudizio, o scienza che a ciò basti, il Poeta piglia tal detto
in questo luogo nell' uno, e nell' altro senso, cioè, che egli fusse veramente gran-
de abbachista, e che egli abbacasse, cioè armeggiasse col cervello senz' utile e

conchinfione, e però v' aggiunge di zeri, perché, sia pur grande un' abba-
¢hista quanto si vuole, che mai non rilevera somma alcuna, se non si servira d'
altra hgura che del zero, Cos} in effetto fu costus che con tutto il suo grand' ab-
baco non pes mai far conto, che gli tornasse bene, e con tutte le sue arti, ed
invenzioni si può dire che abbacasse, perché in ultimo si mori quasi di fame,

PIGLIAR ? appatto. Quand' uno col pagare ai Principe una somma convenuta
Piglia ' assunto di provvedere uno Stato d' una mercanzia, e fa proibire che -al-
tri la possa vendere, o fabbricare senza sua licenzia, diciamo pigiare appaito, che
Sil Las, Adonopolium. Tre MET-


514 MALMANTILE si

MET TERVI il mosto, eI acquerello « Consumarvi tanto le bu
tive fustanze.. Oleam, \& operam perdere,

FVSO da StradieriChi fiend gli Stradieri dicemmo sopra C. 3.
sto lor fufo e un ferro sottile lungo, ed acuta, col quale forano i
altro a fine di vedere, se vi sia occulrata roba, che paghi gabella.

PALETT A da Caldani, B' una meltoletta di ferro con manico: g0 » che
serve per istuzzicare i fuoco 'nel caldano,0 focone,il quale, che cosa sia, Vi
C. 3. stanza 3.; ee

SBRACTARE.. Vuol dire istuzzicar la brace, perché s' acceada, o P'accelas tie
spandere alquanto, e qui dicendo: gf sbracta il mufo, intende, 10 perquote con la

paletta nel viso, e gli¢ lo sCortica. 1. 20.) aga Deke
4 LE squote come fanno icani, Non ttima, Non cura le buffe.. Vedi sopra C, 10, Suan
anza 36. 5 'sun Obey Mec
eARe HIBESO ch' egli ha a' Monnini, Doriano fa morire il Fucba con 'uno: Sino
quei suoi Monnini detti sopra C, 1. @. 44. i quali Monnini ij Poeta insieme cons Nan
ogai altro stimava tanco sciocchi, e odiosi, che credeva fuflono abil a far morire Celido
uno di naufea,; al fen

SQV ARCINA, Spada corta, e larga,altrimenti detta colrellao mezza spada. Te
POLPETT A... Vivanda nota fatta di carne benissimo bactuta con coltello sed
impastata con uova, cacio, pan grattaco, fale, spezierie,ecs > an Oh Difer
CERVELLATA, è specie di salsiccia fatta di carne, o di cervelli di porco La
triturati, ed imbudellati come la salsiccia. E dicendo far poiperte, e cervellaths Tere
4' huomini intende far macello, e strage d' huomini.
CONTADINA., Specie di danza usata nel Carnovale, la quale confiste tutta hag

in forze in questa maniera,, Octo-, o dieci huomini si fermano ritti col im Cel
fieme in giro con le braccia alla coliottola l'uno all' altro; opr' alle. di ha
questi faigono quattro, o sei», sopra i sei altri tre, e soprai tre wao, e fatta que- att
ita regolata mafia vanno girando a tempo di suono,, ed in ultimo quello, che € to
cima sopra a tutti, fa un capitombolo sopr' alle spalle di quei tre alla volta delter= e
reno, dove e ripigliato da due, che sono quivi a tale effetto;:nello stesso modo ty
fanno poi i tre, e poi i sei, e dopo questi gli otto, o i dieci fanno iltcapitombolo ile
in terra; e questa dicon far la tombeiata. EB perché Mato di Coccio.in: for- te
ta di bal'o era Maestro, € però dice, che Salvo Rofara sapendo, bea la re
Contadina, lo fa fare la tombolata gil perla scala. aan Ui
STANZA XAXXXVL STANZA XX¥EKVIL- ti
Palamidone in tanto con la mano, Quasi di viver Bariftone uso, > '
In tasca a Belmaforto andana in volta, Egeno affronta con un prmerwoloy he
Per tirarne la borsa in suw pran piano, E perché quei |" uccedia come nn gifoy i
Per carita che non gli fusse tla; Salea ch' ei pare un gailestovmapanele ip
Mail buon pensier ch' egli bayrie/ce vano E raito fa cht iL manbeaeenpe, ta
Perch' egli col pugnal se gli rinolta's Manda My
E fa per carizade anch' e che muoiar, E por to pi: the
'extecio fa vita non gli tolga il boa, 'Per dario per un 1



EB passagli un vestir

ee

STANZA XXXXVIIL

Exquei gli duol che'l rinnono quell anno,
- Bfee' si muor vuol che gli paghi il danno,

ae VNDECIMOCANTARE, 515

STANZA XXXXIx,

Romolo infilza “to mezzo al busto > L' armi Papirio ad un Prandron guadagna,
'tes iyoenunise un canto erafugviasco, Che. fae apiacuhine lo Swillerra;
Efe ne muor con molto suo difeuito, Ma  a parole gli è Spaccomontagna,
«Perché egli haveva a esser aun fiasco; AUP ergo poi riesce Spada fanta,
Tira inun tempo fifo aun bell imbusto, Perchheifactee it al Cel dar lecalcagna,
demmafeo, 'Won una voir dice, ma cinguanta:

Sta[uch'in terra i pari miei non danno
Ed ei risponde: S'io sto (uy mio danno,
L

STANZA
riga il Mula, ePoste degli allori,
Son mandati per sempre a far un sonno,
Miccioge'l Baggina da Strazildo Nori
Sono inviati done andò il lor Nonno,
E nelle parti giù posteriori
Panfiloagginsta Meoyche vendeil tonno,
Tal che s* allor putina, hor chi accosta
Sente che raddoppiata egli ha la posta,

Narra' morte d' alcuni disensori di Mal mantile, e le bravure de' Soldati di

a, Se Brami tanto d' intendere i nomi anagrammatici, quanto di sapere
chifieno gli altri. Vedi sopra al C. 1. ed ai C. 3,
STVEO. Sazio. Annoiato. 2

 PENT ERVOLO.. Piccolo file di ferro aeuto, del quale infra gli altri si servo-
no i farti per far buchi agli abiti.

DB aecelta > Lo baria; lo schernisce. Dice come un gufo, cioè come fanno gli
ucceiletth al: gato, che è uno uccello notturno, e simile alla Civetta, ma assai più
grande } chey Latini dicono babenem, donde bubbofone si dice a uno spropositato
chiacehierone; e bubbole i racconti spropositati, e non' veri ( forse da Bubbola uc-
cello, Lat. «pupa. ) In questo uccello detto gufo, o barbagianni, favoleggiano
giù atichy Poeti, che fusse mutaco da Proferpina quell' Ascalafo, che fece la spia
a.Proferpina d' haver ella mangiato la melagrana, il che fu causa, che ella non

¢ ulcir daii' Inferno. Ovid, 5. Met. Questo uccello € forse lo stesso, che quel

Pgeedel quale habbiamo detto sopra in \cst{42}.

~ GALLETTO marzxolo, | galli, che nascono del mese di Marzo, quando poi
fifega il grano son più grandi, e fs gagliardi di quelli, che nascona d' Aprile,
eper questofaicano piii alto alle spighe del grano, onde col dire: Salea come un
galletto marxvalo, s' intende falta gagliardamente.

 LL mal tarenfo, Vuol dire huomicciuolo di cattivo animo, che i Latini purer
dicono boma fungini generis.

4VEFETTO. Intendiamo una specie di tavolino; ma quis' intende un colpo,
che si da col dito di mezzo accomodato a guisa di molla a! dito pollice,o ( come
diciamo ) dito geoffo, e poi lasciato (appar con violenza al luogo, dove si vuol
colpire «| Moiti pero per bufferto, o buffertune, intendono.colpo di tutta la mano;
¢ appresso gli muoli Boferada, o Boferon vuol dire mostaccione, guanciata.,

Macon questo huomicciuolo, che non era da pugna, o simili, si può credere,
che intenda veramente pufferro dato con un fol dito.

BAR querciuole, Cioè con le gambe alzate all' aria, € s' intende st ammazza,

-Lnoftri ragazzi dicono far querciuolo, quando no pola le mani, ea testa in,

terra, e manda le gambe all' aria; quaft mostrando qd essere una.pianta, la sc
od Tee "2 a



——

516 MALMANTILE | o

ha,della quale sia il capo, il corpo sia il futto, e i rami le zampe. ho
seguente dice dar le calcagna al Cielo, che vuol dir caduto in terra b Bul
così si mostrano le calcagna al Cielo, e fi dice anche mandare a gambe | no

FVGGIASCO. Riurato, fuggitivo. Vao, che per paura de' birri sg
vedere, se non ne i luoghi immuni. we ky

HAVEVA a offer a un fiasco, Croe 8 haveva a trovare a bere i 5
Quando alcuni voglion bere insieme un fiasco di vino, € pagarne i
ii valore per mettere insieme la cricca, dicono Chi vaol essere 4 un fiasco? Mi,
tende chi vuol accordarsi a bere, € pagar cia(cuno la sua parte? BY termiae! Bad
fo, ed usato fra l'infima plebes ate a0

BELL imbuffe. Bella preteaza, Va di coloro, che Manno in fa la ky
quaii non hanno di buono che la prefenza, da i Latini soprannominati 4
per metatora, perché /folones si dicono quei bet rami, che noa ab
donde noi diciamo folly a uno che non € buoao se non a far comparla,o v
za,come si dice qui 7 bell' smbuffo, che diciamo ancora wa bel coram Vobis. A
Tulipano, diciamo a uno, che abbia buono aspetco; e poche altre quali Ti,
similitudine del fore così detto, venutoci di Turchia, che va imitando la! hare
¢ la vaghezza della Tulipa, o del turbante Turche(co,ondehailnome, =u

DOMMASCO, Deito così dalla Città di Damatco in Levante. Specie di v
drappo fottile di feta fatto a fior1, o ( come diciamo ) a opera. os baa

RINNOVO! quedl'anno, Se ' era fatto di nuovo quell' aano, Pare che sia foli-
to quando altri si fa un vestito nuovo per li primi giorai, che -adopra havers = nd
giù qualche riguardo di più, come faceva costui, che per esser ii suo vettito nuo- T
vo, l'apprezzava più della propria vita, poiché rinfaccia, e proreiladeldanno
del vestito, e di quello della vita non ne dilcorre, oem oie ¢

StanDROWE, Huomo di Fianiira, Ma perché huomo di Fiandea diciamo j
Fiammingo, la voce Fiandrone ci fertic per esprimere Vino spaccone, éhe si vanti P
di bravo,raccoatando le prodezzc tacte da im fuori di qua, ed uno di quelli, che b
i Latin dicono milires gloriofos, ed in questo senso lo piglia il Poeta nel presente i
luogo, se ben (cherza con l'equivoco; Ed egli stesso lo dichiara dicendoy Che» I
fan Taghiacantons, e lo Smillanta; all' ergo poi riesce Spada fanta, cioè fa da bravo, ha
ma dovendo venire a i fatti, e alia conclufione, riesce una (pada, che non fa mal ¢
veruno, e pero Santa; ed in sustanza un poitrone. Dicesi nell' uso, "i
buona pada; cioè € huomo, che fa bene adoprare la spada. Nel Pianto che't Pe
Carlo Magno nella morte di Rolando da' nostri Poeti detto Orlando, appresso vy
Tarpino Arcivescovo di Rems, e compagno in guerra del medesimo Carlo: 6 die fing
ce. O brachium dextrum corporis mei, barba optima, decus Gallarums, inf hg
Carlo chiama Oriando Spada della giustizia alludendo alla formidabile spada da ie
Turpino detta durenda, da' duri colpt ch' egli dava con etia da' poeti Darindana, Il
oh wrath rf, o fmill. dich un nostro pi bio in. di,
che dice La fradera del' kiba, che vuol div vantatore di gran cose 50;
re; Equesto perché la stadera dell' Elba; che serve per pefare barche piene ey
ferro, acile sue tacche comincia a contar da/ mille, e seguita s -a migliaia
Tagliacantoni, cioè, che tira gill pezzi di muraglia corrifj | Pyrgo ii
wices di Riawto, Che vorrcbbe dire in noltra Lingua Atrerrasor ty



ee e

4
Ai. si

a
VNDECIMO CANTARE. 537

Lo Smillanea, cioè Smillantatore si esprime dal Greco Thrafon, cioè Audace.,

BHES Ske

ee

Sesh o £F

SPTVSILRS Pee Thr ees CUR ET

Baldanzofo; e dal Latino Adiles gloriofus. E la parolaé fatta da Adidanea, (cher~
'zofamente usato dal Boce. in vece di mille; dandogli la desinenza di quaranta,
cinguanta, e simili; quasi uno non sia contento di dire la semplice parola di mil-
le, ma la voglia go > e far parere la cosa più di quel ch' ell' e in esserto.
'S' io Ho fu, mio danno, Non mi rizzo al certo. Questo termine mio danno usa-
to in questa forma, e specie di giuramento, ed ha la forza del termine appon/o 4
noi, decto sopra \cstan[8]{72}. € 3° io non' ho,egli e fallo,detto sopra C. 6. (tan, 86,
MiCCAO, Così era nominato un garzone della pallaa Corda, che è uno di
coloro i quali stanno nel mezzo della stanza, mentre si gioca, a raccorve la pal
la, e rammentare il giuoco.
BAGGILANA, o Baggina, Eva un Battilano, che in occasione di felte serviva
ai Bawtilant per tamburino.
DOVE anao il lor Monno, Cioè nell' altro Mondo. Vedi sopra C. 4, tan. 2.
NELLE parti posteriori. Cioè nel c....0 come bassamente si dice, nel preterito,
dove dice che è prima putiva, hora pute il doppio, che questo vuoi dire
ha raddoppiato la posta.
e4GGIVST A. B' preso ne) senso medesimo, che è preso sopra \cstan[2]{41}.
CHEO che vende il Tonno, Fu un venditore di peice falato, e tali huomini
hanno (empre addosso cattivo odore.

STANZA LI. STANZA LIL
In abito Scarnecchia da Coviello,
Tinta de brace l una,el altra guancia,
EB per sua spada sfodera un fuscelio,
C" al pome a' una bella melarancia,
Rinolto con quest' armi a Sardonello,
Perma, gis dice, guardati la pancia,
Ed enrisponde: uestoé pensier mio,
z rant un colpo, ete lo manda a Scio.
Gustavo Faibi con un soprammane,
Di nerty il capo fmoccola a Santella
Scaramuccia si muar fotte Erauano,
C' aimazza anche Gaba da Berzighella,
E fuentra quel birbon dell Ortolano,
Che fa il minchion per non pagar gabella,
Ma colto poi vi reffa ad ogni modo,
Mentr' adesso gli va la vita iv frodo,

Descrive l'abito, ed armi di Scarnecchia,
che resto morto da Sardonello;

Eravano ammazza Scaramuccia, Gaban da Berzighella, e l'Ortolano.
COVIELLO. Cioè lacoviello maschera, che finge un bravo sciocco Napole-
tano, 'a quale s' aggrotte(ca con fargli i bafi alla Spagauola col nero 41 b ace,es~
PerO dice Tinto di brace? una, el' akira guancia, e con armaria d' una spada faca
d' una mazza, che ha in vece di pome una mela, o melarancia, o altra frutra
simile per rendere il personaggio più cidicolo, e così vestiva questo Montambanco,
facendosi.chiamare Scarnecchia... Vedi sopra C, 3. tt. 62, Così Cosa, e Zanni,
personaggi ridicoli di Commedia sono nomi proprj de' loro paesi, donde si fingo~
.no»s accorciati dagl'interi nomi Niccola, e Giovanni; onde va in terra lorigine
di Zanni, che alcuni ingegnofamente hanno tirato dal Latino Sannio, mis.
LO manda-a Scio, Lo manda all' altra vita, ed è lo stelio, e si dice per la me-
defima ragione, che mandar a Pasraffo,0 a Buda, detto sopra C, 5. st. 134
- SMOCCOLA il capo. Taglia il capo... Smoccolare si dice tagliare il Lucignolo
di una candela, o altro lume per levar quegli escrementi, che fa la fiaccola, che
hiamali f i. » che queiti Spagi sear'



8 MALMANTILE

desfavilar quasi exfavillare; il Vives disse exfungare formando la all
Virg. 1. Georg, Scintillare oleum, \& putres concrescere fungos, ol
SCARAMVCCIA, Vo' aitea maschera, come Scarnecehia - tit
Ourava 51., ma questo era Iftrione, e non Montambanco. i owe Roi
GABAN da Berrighella, Questo pure era Iftrione, ce rappresentava wo |
dt un Romagauolo ttoito. ' = Oe
L'ORTOLANO, Costui fa un yeechio astuto, che: per ein
dovutali per aicuni delitti commeili, s' era finto ae 1 Del
chion per non pagar gabella, Menandro, Rusticum essete simulas, tam Par
vi resta colto, cioè viene (caperta questa sua malizia da Bravano, che Per
vita in frodo, a colui, che non volea pagar la gabella, e¢ vuol wae Sin
in vece di frede solamente l'usiamo di dire dalla fraude, che si comm el
pagare la gabella. Ta
STANZA LIL i
e4rmato a priuileo} omai Rofaccio Che piove al
Marte sguaina, e Venere influente, Ond''ci in quel pumoandada Nan
Ma ae Sardonello sul mostaccio Vede le elle, e linac t altrasfera un
Gli fece con la spada un' ascendente, Nel viso ectifia, e dice: Ty
Rofaccio ricoperto di privilegj cava fuora Marte; e Venere; che Pe
tivi influssi, ma Sardonello fece piombare sopr' a di luiun pefimo % Tee
tagliandogli con un soprammano parte del vilo, e del collo, ed un braccio Rani
il qual dolore egli vede le stelle, ed eclifiando l'una, € laltra sfera del Coat
ferrando gli occhi dice: Buona sera, cioè perme, fatto buio, «B Mi
sto Rolaccio si piccava d' Aftrclogo, come s'è detto sopra C, 31M. 63.5 11 Poeta tg
con la presente Otrava descrive la di lui morte con equivoci di termini affrolo- pred
ici. f Lapa
: ARMATO 4 priniteg|. Questo Rofaccio, come ancora gli altri Montamban- ro
chi per accreditare i rimedj, che da essi son dispensati, mostrano una infiuna di iw)
privilegj concefli loro da diversi Principi; e pero ii Poeta lo fa axmato di privi- the
legi. Uontanlald ken

SGVAINA, Virgilio vagina eripit, Sfodera Marte, o Venere; che predicono
rovine; B dice sguaina, che vuol dir cavar la spada dal fodero, o guaiaa, perché
s*intende, che non haveva alcr' armi offenfive, che Venere, e Marte unflussi
cattivi 'a duaaiead

ASCENDENTE, Termine astrologico, col quale qui intende colpo di taglio, Un

che viene da alto a baflo, piovendo, cioè calando in sul capo, ec,

OCCIDENTE, Intendiamo l'occalo del Sole', maqui intende ocealo, cioè è
morte di Rofaccio, ily oni aalaan ey baatee tm
VEDE le free. Quand' uno fence gran dolore; si dice + Eeli ha veduto le fielle, hi
perché le lagrime, che vengono in (ugli occhi per il dolore, G

la rescazione della luce, che yi batte, una cosa simile a una quantità di mi -

nute stelle in Ciclo, che più volgarmente diciamo veder 'nce i

mo sopra C. 9, st. 60, 5 ma qui si serve di questo, perché.gli

re di farlo morire astrologicamente, i
ECLISS.A. Chiude, cuopre; ficome alla Luna 'restano i

»hajean > x

3



VNDECIMO CANTARE. sip
 dail interposizione della Terra 1 raggi del Sole, quando seguono I ecliffi.
DICE buona sera, Cioè si fa buio per lui, ven donate 10. st. 5. Qui intende
è finito il giorno del mio vivere. Virgilio in-aternum clauduntur lumina noflem, o
i asadostndelcginnanalo » che, havendo:manco un' occhio, e Ji fa ca-
vato l'altro, disse: Buona worte per tutto lo tempo, '

i STANZA LIV. STANZA LV. i
Mein per fiancofentesi percosso Già per la franca il sangue era a tal segue
Dallo stidion del cuciniere Melicche, C” andar vi si potea co' mauicelli —

 Parafiraccio porco grande, e grosso Istrion Vespi tutto furia, e sdegne
Perch' il ghiowso si fa di buone micche; Rinualto ha quivi tl povero Adaffelli,
| Sirivolta eAeino, e da al coleffo E col coltel da Pedrolin di legno
Nelda gola ch' egli ha pien di pasticche, Su pel capo eli squotola icapellig,
«Tal che morendo dolcemente il guitto: acleciopratcane poi la lifoa, el Lota...
» Addio cucina dice, ch'iobo frito, Pius bella faccian la conocehia a Cloto.,
ils STANZA LVL
NGatsi,, e Paol Corbi inveleniti A tal ch'i pacfani sbigottiti,
| Quali villan ch' i tronchsyed i rampolli E dal disagio sconquassati, e frolli
 Taglin di marzo ai fratti ed alle viti (Oktre che a' pachi il numero è ridotto)
| Potanda i basts braccia, gambe, e colli; Cominctaron le gambe a tremar sotto,
. Termina con te presenti Ortave il racconto del combattimenco (egaito in Mal-
mantile,, e dice la morte:di Melicché, ¢del.Mailelli., e qui tinisce ' Vadecimo

re

MELICC HE, Vedi sopra C, 3. st. 59, lo chiama Parafiraccio,perché era huo-
Ȣdel continuo havrebbe mangiato: EB questa voce Parafito, che appresso
di noi ha dell'ingiurioso, non era così appresso gli antichi,come si può de-
durre da molti Autori tra'guali Luciano; ma particolarmente da Piutarco, dove
fitrova': Parafitos nontancumappellabant strici adalatores illos, qui apud Dinitum
tmensas wutriuntnr, fedietiaum tos,qus ob rem egrecit gestam,publico /umptu in Prytaneo
atebautur Oc, Onde delle Stinche di Firenze, nel capitolo in lode del Debito, il

Bernt; è
Voi fore quel famoso Priranco, è
Ab bower yas Doe renews in grassoin fisoi baront
I popaly che discese' due F efeo,

Exit Atheneo Parafiti olim appelabuntur foci, 7 fideles Pontificum, eAMagiftratiiz,
Ibmedefimo Plutarco.. ¥

PASTICCHE Specie di confezione fatta col zucchero muschiato,:ec; e però
dice more doicemente, perché ha gli per la gola 11 zucchero, \textit{Pasteca} voce
Spagnola, siccome anche \textit{Pastiglia}, che vale lo stesso; e sono tutte due diminutivi di
pasta.

GVITTO, Huomo vile, abbietto, fudicio, sporco., e sciatto. Vedi sopra
C. 3. st. 9.è:voce Napoletana, ma usata oggi anche da noi, 'Nella raccolta de'
 Poeti antichi-dell' Aliacci, Pra Guittone-scrivendo un Sonetto, siccome da esso si
raccoglie.a Messere Oneito da Bologna 'Poeta, e amico suo; scherza sul nome di
turer € die, *

— SAS QF Cisasn see

=

Pita

A.. SSe



g20 MALMANTILE | o ¥

Voktre nome, Messere, e caro, e onratoj
Lo meo assai ontofo, e vil pensando, =
Ma al vostro non vorrei auercangiato, =
10 ho fritto, Scherza col verbo friggere, che vuol dir Quocere carne,0
padella con lardo, o olio; ed il detto ho fritto, che significa il
in malora. Latino Attum est de me; perij. Vedi sopra C. 8. st. 54, tor
nel presente luogo, perché par che dica; Addio cucina, ti lafio non
più bilogno di te, perché io ho già fritto, ed intende ho finito di vivere.
IST RION Vespi. Pietro Sufini. Questo fu cognato dell'Autore, e giù
grandissimo (pirito, copiofissimo d' invenzioni, come si vede in una
commedie da lui composte, e da altre sue Opere poetiche, B pecige p
fentava in commedia ottimamente tutte le parti, ma in specie quella del se
zanni, ( cioè servo sciocco Lombardo ) che usa armare con un coltello di Tegao
simile a quello,col quale si batte, e si scotola il lino per purgarlo dalla lisca,
perciò chiamafi Scotola; però il Poeta lo fa azzuttare col Masselli, e sc
con quel coltello la zazzera. Dice coltello da Pedrotino, perché con tal
ceva chiamare in commedia detto Sufini nella parte di servo feiocco. Questo mo-
ri giovane poco dopo l'Autore; e con esso si può dire, che in Firenze morifles 4
la moderna arte comica, o almeno la franchezaa, e leggiadria nel maneggiarlag =
SQVOTOLARE. Vuol dire battere il lino. Ma qui intende squotere i capelli
per facilitare a Cloto, una delle tre Parche, il farne la conocchia, aleism
INVELENITI, Ancrudeliti, inviperiti, inaspriti, incancheriti, arrabbiati
son sinonimi per intender' uno, che sopraffatto dalla collera operi of
te, e con ira, in maniera, che non sappia quasi distinguer ch'eififaccias, =
Similitudine presa dal serpente in collera; di cui Virgilio lib, 2, En, tcolentem
tras,\& coerula colla tumentem. wm abean
POT-ANO. Latino amputant,demetunt, obtrancant, tutte similitudini trate
dal' agricoltura. Potare si dice de' traici delle viti, € de' rami degli alberi; ma il
Poeta si ferne di questo verbo per corrisponder' alia similitudine, havendo dewto
quasi viltan ch! e' tronchi, eds rampolli taglin ds Marzo, ec,. sd
SCONGVASSATI, Stanchi, € rovinati walla fatica del combattere.
FROLL], Qui vale per stanchi, ed indebolits, t¢ ben per altro Frode vuol di-
re stantio. Vedi sopra C. 3. st. 55. alla voce Leazo, iahesh
TREMAR le gambe sotto. Vuvi dir haver paura. Virg. Eo. ry.
ae folvuntur frigore membra, Se ben si può anche intendere, che le §
mente tremassero per la debolezza, e thancneaza.

FINE DELL' VNDECIMO CANTARE. +

Ze

Berbnanheansr ~ekk dé



BER TEER BEE

HifwMAMALAM

Fea dattacbute

A R GOMENTO,
e A, Montelupo. da Paride il nome,
Poi gapigar la Maga, e Biancon vede,
Rimessa sn. Trano è. Colidera 3 3 e1come.
~ Aarito, al general dd ln fuafede.
a Baldon, che la fortuna ha per.le chieme
Con Calagrille aVgnan rivolgeil picde,
E al suo bel. Regno con Amor va, Psiche
A corre il frutto, delle sue fariche.

ae

speyrepeage

sae PP Regen

STANZA HL
Che sono fratt com! io dissi sopra,

were STANZA I.
Swanco già di vangar tutta mattina

“Abconcadino al fin a va-a rifelnere,
- Te forniar Vopresed-in chiamar la T ina

* Cokmerize guarto,eil petal dell'afoioluere;

( Nella Maga affidatifi) asperranda
Da' Diavoit im lor pro veder quale eprs;

ea chi-vive a speranta muor acids;

eee tn Caffelle ancor non firifina Perch in Dite son tutti sottosopra,
Phu quei-marei di squotersi la poluere; 'Per non saper dove, come, ne quando
Onde: Badldon quei popoli-di/per de Laftiasse il Cornocapolfo,c! ale (chiere
Tal che a' joldati Malmantileé al verde, Esser tromba dovea nelle carricre.
STANZA Il. TANZAILV

E vase Sta, perché porevan dianzi,

vedean col peggivandar sicuro,

~\ Cederil campo, @ non tirare innanzi

ra Star avoler cozzar col mura:

< E così va, che questi son gli avanei,

Che fafempre colsie'ha is capo duro,

» Che dentro.a feifi reputa un' Oracolo,

Ne crede al Santo,se non fa miracolo,

Di modo, che Plurone omai scornato,

Poiche quel corno pitenon si ricrova,
Pel Proconfolo dice haver pefearo +
Pero connien pensare a invenzion nuova;
Ha innanss ch' ei-risolua col Senato,
Eche'l:foccorso 4 Atalmantil si muova,
Ch'egli habbia a esser proprio pot s'avvisa
Di Meffinail soccorso, v quel di-Pisa,

wey introduce i Poeta in questo Duodecimo Cantare con la rifleilione, che i (ol-
7

vv dati

|



22 IRKTVUA MAL MAN TILE > 1009

dati-di Bertinella non, haurebbono ricevyto,così gran danno\ |
sono accordati,, e non fufione faut in tanta 'tingaiones la.;
in loro per la speranza, che havevano negl'incanti di,Mar
havevano havuto effetto alcuno, 1 Diavoli non feppe:
dove fufie ii Corno d' Aftvlfo., non si ricordando, che. an
quando Affolfo andò per il senno d'Orlando, comedice.|'/
| KANG ARE...Lavorar la. cerca conia vanga... Bipalio

FERALAR l'opre,.Cioè far defiltere dal lavorare eer an
raion Depera. fra. i contadini.s' intendedlJayoro;, che fa.un'
no, e s' intende.ancora lo Relio huomo.s.che ya.alavorare a
io.ho; chamato due, opere, per iacender due huomint; In questo lavoro ci
dicci opere, per intender dicci giorai di iavoro, ec,

p44 Ting. La Caterina, intende ladonna del Contadino

MEZZO, quarto... Così chiamano i Contadini un gran valo —
foggia. da boccale 5.del.quale si leswo, fespartag da bere ai Javor.
po, e gli danno questo cee perché'e forse di ce een
staio. x ae
PER SL afeigluere 1 cont ania ebaainesior il desinare asciolvere, Seren csnidal
foluere il digiung, dali. sdigunarsi, ¢d.il desinare, lo chismano wien
terzo mangiare Aicono./a cen

eA non si rifina., 'Nanay resta,,
esprima una.op e feng'
Ciog perquoterii,, bastonarsi.. Vedi rae C7. tt. 63. t by

ESSER? al verde., Eji¢r' ajla.fine. 'Tratto dalle candele. di, se ce seieprion dig
son unte di verde nel piede.. Viano nel Magistrato del Sale di.
le tafe dell'Osterie.,, e darle.al più esserente., e agl.tempo, che aMtneenapi thd
colissima candela di cera tinta da piede di color verde ognuno può.otferine, es ida
consumata quella noo può. più veyung offerire sopr' a quell'osteria, ma s'intende P
restata a colui), che ha cfiertoyi maggior prezzo, ovvero non arrivando.lotier. Cm
ta.aldovere, ' Osteria ai suoyo si dubasta un' altro giorno con nuova candelerta', deat
EB digui habbiamo il dewato hs ha che dir,dicada candela e al.verde, che significa ted

on, si fa fine. Ma — che p00 iar

BINED a ET SERRE SSESE

sbrighiamoci, che il cmpo fugge.. E questo eficr' al verde e pafiato in the
per tutte le cose, come.cficr' al verde ot danari, vuol dire esser' alia ka
pari,...Va mpderan Poeta teleth scritto nell'Osteria di Radicofaa Pre
trata » qeum ° Re he age

| Cohanr, p Spebater ridotto alverde.: ee un ie ' hy

. » Gineca, uper ricattarsi, e Sempre perde

COzzAR col muro», Tentac l'impossibile,. Contrattar con chiha pb: forea di re
poi, Clavam, c. manu Herculis extorquere. Diceli anche: saree a co4hi co! mre othe
ciuoli. Nell' Ecolcfiafico cap, 3. Ditiori re ne focins fuerss 3 Qu Ep
cabys ad ollam ? Quandoenim se colliferint, confringerur. La Feces Fest
tole nel. fume galleggianti 2 una di rame, l'altra dicerra fa a. e oka
quale viene, Anaae ad. Efopo, e troyafi refa in versi Latini gala,
CAPL dur ag te ostinati. Dure pom ebb —o Ne
'ST tacas lends » Amico della sua opinionc,c. che Li, Gli
uw

r



~~ oN eeenrvc

S88

SSBB ER CEES £5:

DVODECIMDJSIVLTIMO CANTARE, 8
reat fate', e dit meglio w ogni altro. Huomo di quefd naritat dice de'
= Setpe thea Nim di lapete;e d-etere ungtan” buotao - Baxi.
; edi fe micltefimo ', e pereid ine diviene contumace 3!

PUR Hess ONT I OM 9G; ty
VENOM orede al Sarito'; s'e' non fa miracoli, Non crede'; che una cosa pli poma'ii-
teruenire', fe'rion la vede fegitire.Generario prava quarit signum videre. B per lo
più s' usa in' occasione' dammonire, o rinfacciare j'come e nel/pretente hiozo} ll
tale è lato pir volte: avvertito dition contindvare @ fat'quella tale operazione-,
perche gliene' potrebbe seguir male'; ma' egli ostinato wor erede at Bantoy se nor se
miracoh, cioè non da retca agli aveertimenti; ma! vudl-seguitare?,finvhe la die
ee succeda » 4' Proverbifti Greci mettond un Proverbio,-che dice: Prime
a rem. PURI Toe h) LL Mes bas! Pe TNE Dey 99@ 1951b

CHI vive con (peranzia mior cacando. Detto. sporco » ¢d usato per lo più fa,

genterviles; e vuol dire -'chiMfi palce di speranza-,'muore di faine'y"ed-in sulliinza

a €*vanitaril'fondarsi nelle speranze. “ai /pe'neratar, wi reer

mats ig 2. 0g

SON tutti sottosopra, Sono in grandissima confusione. '
sI DOVEA fer tromba alle carriere.' Dovea' fare scappat tut? peome facev't il
Corno 4' Astolfo: e'come fa' scappare dalle motfe i cavaili barbati'y che edrreno
al palio quella tromba, che suona il banditore, per dare if feghd della [otpperied
SCORN ATO' »» Vuol dif beffato; ma qui 410 acherzu di /eorWard\, che Vadeiie
senza corna, come era rimafo Plutone fenza'¢orno, cine senza it Corti dA Nbk
fo. Var animale, che abbia perdute; o tronche le cortia, vient ad avere per
del decoro; onde scornato diciamo per beffato. Acheloo 'fiume; 'e fleatlogli d2*Er-
colelevato un corno', rimase scornato; e svergonato. Onde Ovidio 9; Met Muh
tas Achelons agrefies, Et! Laverne cornu; meuijs capur abdidlit sndis,  Hivtc tanith abla
ti dommie idibure decoris, Gc} 229 10109 Lb 2h 4 HW) 199 > piri
> SPBSCAR per il Proconfalé. Ho Neff, che durar fatica per impoverite; sean,
CG operam perdere. Proconsolo è in Firenze il Magistrato, che soprintende a i
dottori, e Notai, ed ha la'saa refideriza otto le logge,dove sono giù altri Viizzi,
acll'ultima abitazione versoril fiume d' Arno; il qual fiume per quello spazio',
che è fra l'un ponte, e/' altro} ', 6 almeho efa già fortopotto alla 'giurifdizione
del medesimo Magistrato del Proconfolo'; come pesca ad elo rifetuata ne' vr ti
poteva pescare senza licenza del detto Magiftracs 3! non vi era-già ditra pena aIfi
contraffacienti, se non la perdita delle reti, e del pesce, che hanno preso, fead
acchiappati in sul fatto; E Pett utes! aie

STAN ZAV E>) > STN ZA VEU.
ae Paride ritorno, OO Ada quegli, e ° obligate si non Witende,
onGhte nellvoffe alla quarta sboccatura; Wor vuol phr quanto un capo di spilletto;
« Eperché dal pacscegls ha in quelgiornd E subito ogni cosa indietro vende,
0 Foleo ogni nota', liberando ib Tura; Ringrariande cinscsin det buon' afetto,
8| La gente quini corre @ intornd E'dwe', che da lor nulla precende;

ed rallegrarfidelia fuabravkrh >) 0! 2) EB Te aiteddisfarle bhnho concerto,
Ne lo ringraxiasewrallegrarsi intenta, °°! Perital niemoria gli fara più griro
Chi gli da chin lt dona z'chi gli-avvina, o: Che it tuogo Aonteliipa sia chizmaro;

ang, ~:

Vvv 2 STAN-



524

Si si, ch' eli è dover da tutte quanté
Gli fu risposte, ed in un tempo stefo
Li editto pel Caspello fu pe i canti
Per notizia de' Popols fu messory>
Che dinuleato pos di te avanti yo. «
Fu osservato si, che finoadefo-..- 
Lucho nome confernan quelle mura,
E'l manterranno,fin che'l mondo.duna, ¢

STANZA AX ) f  A

E che fuor del Caspella il,popoh proves: «3000 (- ) SERB;

Che ognor ne scappa qualche sfucinata,

Per to più gemse yeh? a peta 31 i loxofexes fer o

Cotantaé rifinita, e maltrattata, — Qui pinto innansi stwile sentiva) >

Tornaril Poeca a discorrer di Paride, il quale havendo ridosto il Tura nely

fino staco, haveva liberato quei popoli, i quali per riconofeimento del

ordinarono, che que! luogo si chiamafie daallora avanti Montelupo

torna al campo, e trova ogni cosa murata..
LA quarta sboccarura, Cie ha sbaccato, 'cioe.: manomeflo

vuol dire: ha bevuto tre fiaschi di vino, ec cominciato ibquarto2 Iperbole, che

significa: ha bevuto molto vino, sborcare propriamrnte Qgettare via

vino, che è nel collo del fialco., per purgarlo affaordalll'obia.yec, LiAQpesas!
CHI gli da, chi gli dona, e chi giù avventa iB' detto giecofo nfato per burlare

uno, che figlorij d' essere sspesso:regalato; es) intende; chido ee 1

avventa, cice fafiate, ec. € lo scherzo dell'equivoco't:nelwerbomare, e '
NON enol, quant' un puntale d' agherto, Racufarurto.. Vedisfopra Capt, 10;
RINGR AZIO' del buono affetto. Termine di cirimonia julaci si

ringrazia uno del regalo, e nello fefid tempo si Ficnfa di rice -dicia- f

mo;non voglio,o non stimo il regalo, servendo, per obligarmiy Pinclinazio- '

ne, che io veggio in voi di farmelo; e questa testimonianza  chehio dal:voltro
affetto verso di me. ist
eHIONTE Lupo. Finge, che Montelupo Castelio wicino a
anch' eg) quasi distrutto bavefle nome da quota azionedi:
biamo per tradizione vuigata, che eglilfusseanticamente
stare il Castelio di Capraia luogo allora forte' fituato rincontro
cendo coloro, che.' edificarono: Perdifiragger questa Capra 'Non sci guole altro, che
un Lupo, e perciò lo nominarono Castello Lupo, che. per esser Mopraiun monte si
detto Monte Lupo. Coca bg ED
GLI venne il grille. Gli venne voglia: E' 1o-stesso, che tocedsill

sopra G. 9. st. 56. con Sp bei
STRVGGIMENTO. Un continuo ardente pensiero 50% I

iftruggimento vuol guarire, cioè vuol' adempire questo.fne-desiderio

all armata. Ii Burchiello, fe ben mi ricorda; Se/piri:d.amo rd

\item[SPARITO ciò che v'era] Non v'era più persona alcuna, ip
Baldone era diloggiato, ed entrato in Malmanule. 495



“DVODECIMO;ETVLTIMOCANTARE. 525
SEVCINAT A, mola neg Vana gran quantità, Fuciaayyicn dal

che wuol-dic
ani

(O facina @ i

» o luogo dove si ri

no mercangic; e
be capire una fucina prela. pec
operasoni

le ic
' et re Bocce, Nov. 2. ee ane eena di diaboliche

igion dire; 4

si erika, 'vuol anche dire il Barccaten de' fabbri o delle fonderie,.ec,
| RIELNIT. « Malconcia,@aaca, finica, sopunatai ¢s.intcnde di sanità, e roba,

or STANZA a
pala, e ne riscontra un branco,
Preeti lemgean,
'bi dietro fr ascicar fivedeun fianco
; | gli gi

STANZA XL

Chi ha scatole, chi sacchi, e chi sieehiee

Di givie di mifoee, dibiancheriay 
Va" altro ha una ranaca di scrittwe 4,
Ch' agli ha @un Pinto della dtercariby

agli senza.adar albaco, £ piange,ch).ei le vede mal sicure,
\& nee Sete egli ha riscoffo; Pero che *l vento gliene porta via;
 Ciascuno ha il suo fardel) di quelle tre/ibe, Vat altro dopo haver mille imbarazzi,
a a og si ha potuco beans ee Port' addosso nna gerla di ragazzi,
STANZA XIilL
reimbacuccato Arete feretto Le dine agliocchihantutteilfazzoletto,

a ria > 'eJpelse, Ipesso si szattiene 9

E sgombrane 2 Py rocche, e pergamene

tra ys' elle, le stanno.

Chi'lf il, e chi >

Chi porta nngatto, e La caninainbraccio,

Sono

lex
te vede una gran quantità di gente, che fugge da Maimantile, per (cam-
parila vita, e porta seco, le cose più grate; nel che il Poeta s' accomoda a' gen) di
quelle tali perionc, che fuggono, ed a quello che,per lo più,luol seguire ia simili
Seapets;

at ENC INGO. Se ben significa quantità di polli, o di pecore, o simili, tuttavia

ne serviamo per esprimere ancora quantità d' huomini, Lat. bomnum manus.
Vedi aC, 6, tan. 35;

T ASCIC A dittro on Fucwene Va zoppo., per esser Mroppiato da.un fianco.
HA »ifeoffo Senza aspettare al abate, Glioperarj ordinariamente ri(quotono le
ro mercedi, e prezzidelliloro lavori il giorno del fabato 5. ed il Poeta scherza

col. verbo rifquotere, che vuol dire ricever denari e ce.ne serviamo ancora per
intendere Ricever butle.

GVIDALESCO. Malcalcia; Scorticatura. Vedi sopra \cstan[10]{11}.

TRESCHE. Qui intende bagattelle, bazzecole, arnesi di poco,prezzo; Lar,
trica, Vedi sopra \cstan[10]{12},

SCATOLA, Lat, cap(ule. Sono cassette con fondo, e coperchio, fatte con,
sottilissime asticelle in varie figure, secondo che richiede la roba, che dentro ay
esse si ripone.

SLANCHERIE, S' intende ogni sorta di panno.lino., come tovaglic, lenzuo-
la, camice, ec.

PLATO, Lite civile » dal Lat, placitum, VedifopraC. 7, stan.27.

MERE ANZI A. Altrimenti Afercatanzia. Così chiamiamo, in Firenze quel
Foro, o Magittrato, al quale si ricorre, per far l'efecuzioni civili, e ai we son

fone-



ee

526
fortoposti tutti li Mercanti, ec. il quale ha particolari fat
'MB ARAZZI, Spagnuolo, Embarazes » Roba', th
6 feommodo; ed' aBBHaED il verbo imbarakzare, cht
nesi(, te tina Qanza ¥ ec * » ASSAM vaya
GERLA Da gero Latino','che vuol dire
Ma Voce il nostro Chimentelli nel' Azsr nie?
di bastoni a guisa di gabbia da uceelli
larga je fondato hella' parte più tretca, det
per portare il pane eotto da un luogo all'altro'y adatrandoselo
alle reni; € et eeitind nim
firo Autore nella-eecera alla Serenissima Arciduchessa Claudia,
nelProemio'j dove Wie' Che i Prascica diecro lina gerbi- tdi farfaallond COR
gran quantità dipropositi) Può bene anche essere Che il' Poeta intend'
mente gerla, e che voglia dire, ché havessero due }o tre bambini in u
talé gerle §'per:portari pilr comodamente's coiné veggiamo tutto ”ll B
parire povere donne della Garfagnatiay e d* altrove, che portino due, 0
gaaai addosso imgerie, 6 altri trabicvolifimill /-)) >) 9
tM BACVEC ATO} Copertd5¢ Ito 'bene, ¢'s* ihtende pi
pert ibcaporn Vedi sopra C)1't, Man22: se bendal \cstan[6]{64}.
ne serve per intendere Mettersi l'abito addosso, tuttavia e da norare,'
intende il lucco, che è l' abito Curiale "if quale aiicledmente haveva il
per coprir la testa, e però mietrerfivtal"abito si diceva Pmbackecarff; Si
inbavagliare. Giovanbatista Bufia? asBehedetto 'V archi lettera nona, | ¢
da AMona Coiet, “ed imbavagliatala la conddffero alle Palle se 3
OLE risconera » Cive riconta la moneta 7 per vedere} i '
traruino, vuol dire imbatcersi in-who Pma risconttare libri;' ferirrtite 5
danariy contipecyvuolidit Rivederé je rorha®l Ay
Z HO29! ib. 920074 2» MH299)  tegal cuba
» con te assegnò Wi piabtd', 6'di dolore'
il fazzokeio agli ocehi', Veli topra G..9! stan.48! ahaa
SCOMBRANO « Portan via Seombrare [ quati dal Latind excumiiliré, ton?
trario d? Ingombrare, che ¢'come se fotle dal Lit! ficidmindare] deco
t6, ci serve per intendere. portar le? mafferivie"die ania casa a tn” altra
mo in-vece del verbo diloegiare', sieggiare 'Biaiken archi
BeASPL rocthe je pergamene' 7 re fruthenti atcenenti
habbiamodenowoped nel o., e Rad.'>\\E -pertamend intertddid tes a
Carta con'la quale fermano la 'condechia' ia (u'l roced "per fadifitarell Blare
la dicond paielibead pershe per tovpit: ol esser facta di carta pecora,€ he ti

et anche, carta perdaminay i 9
199 9.99 roel vp SAB NON Z AOR 249 Pr cxortert
Entra: Paride al fimdentro alla portarys 9 * Ma quel che mardni¢ha p tt dppor
OOuelg lipar a! entrasdentroun matelles  -<Si st' veder tn' pi m Capen
Chr ad ogni palje troua gente morta. Di scope, e di fascine f
1Oiper lo-mer, che (Ps per far fardeliay > ani? i iy:
Oud i IMGselis e edusiiwi wh sil? - otary ate wage "A Ve Loi 4

aw


——— a

BOE Gg CUaio 'Set etree eee

DVODECIMO;EDVETIMOCANTARE: = 527

oo, e STAN ZAXWE oo cen! Singeatte:
arriuato in pragza, Eglistaben, pere una simil raza,
Perchi(domanda) ésigran fuoceaccefor.. C? ha fatto se @ ogni lana un peso,
 Egh érisposto:egheper Martinarra, E' si vorrebbe ( Dia me lo perdoni )
bid v'e dentro,escrine: lato preso;. Gastigar a milura di carboni. 7
'aride entra oe! Castello, e vede molta gente morta, o malameate ferita., e+
jartinazza mefia nel fuoco per gastigo dellc sue stregonerie.
 MACELLO, Beccheria. Luogo dove s'ammazzano le bestie per vitto dell?
mo: E per macedo intendiamo Strage, o difipamento di che che, sia. Qui
Iptende, che a-Paride par d' entrare in una bottega di un macellaro in riguardo
=| molto sangue, che vede (parfo per il Castello. Così quel che dice Dante, che
V go Ciapecta tofle figliuolo d' un beccaio di Parigi., Sccfano Pafquier va interpe-
trando, che abbia voluco dire di un bravo soldato, quale era suo Padre, che per
la @trage che faceva, era riputato, come un maceliaro. '
CHE fea per far fardelio, Lat, vafa collig't, che è vicino a morte;, sta-per an-
darlene da questo mondo. Vedi sopra \cstan[4]{21}.
CAPANNELLY di feope. Piccola capagna, mucchio, monte di (cope., ec, il
eee quando era per l'cffetto, che era fatto, questo, era dat Lacini:detto eon

Inc reca Pyra dal Greco Pyr, vuol dir fucco, e noi pure lo diciamo Pira, Dang
126.: ii
i Chi è in quel fuoco, che vien.si dinife,
eed z Ds sopra, che par furger dalla pira,
iets: Ove Exeocle cul fratel fu mifo. »
SCRIVE; lato preso, Antendi; ha cieito per sc quel luogo - /edem occupauit;ma
Per maggior chiarczza di questo detto, e da sapere, che in Firenze si fanno ogni
@nco tra gli altri quattro mercati, uno per Quartiere, che il primo nel Quar-
Gere, e in su la piazza di S, Maria Novella il primo giorno di Quarefima, ach
quale si vendono Icgumi, feccumi, e frutte. Li secondo nel giorno di SS, Simone
nel Quartiere,, e in su la piazza di S. Croce, Li terzo la.vigila dituitii Santenel
Quartiere e in su la piazza di S, Giovanni, acl quale si vendevano oche.; ma
questo € andaco in defuciudine; perché e perduta l''ulanza di regalar l'oca lay
mattina di cucti i Santi. LI quarto nel giorno di San Martino nel Quartiere,.e»
in su la piazza di S. Spirito. In questo, come nel secondo si veadono abiti, pan
Hine, ed ogni sorta d' arncfi, e maflerizic;.¢ come-che acile dette fire concorro
Ho molti mercanti di panni, ed altri artefici d' ogai sorta.,. così alle. volremanca
doro il luogo, dove polarsi, per farvi.ia quel giorno la lox boxtega; onde. piglia-
Ho il luogo qualche giorno avant, e segnano jo spazio dei luogo,, che piguano
con getio 50 altra unta, e vi (crivono in leere cubicali LATO PRESO,, e que-
sto servc per impedire, che altri entrino in quel luogo.: Edi. qui dicendosi; I
tale ha scritto lato preso in quella casa, ec, intendiamo: quella casa, ec. e per iui,
une gli può esser tolta. Così. dice, che Martinazza scriveva dace preso in quel mon=
te di scope, per iagendere,.chc havea tatto in modo, che.qucl fuoco.non le po-
teva esser tolto.... 4g Neds e402 an ic
|fatto a' ogni Jana un peso, Ha commefio ogni sorta di de'ito-senza riguar-
do alcuno. “Si dice anche far d' ogms erba fascio, Che in (uitaaza s' Intende un' nue.
mo

|



38 “1 ALMA NIP 1

mo scellerato, di coscienza larga fhe Hon' tetne
giuttizia; che in Latino' pure si ditebbe, ex guoliber,
mea quella; Aivdum fie-pratum, quod non persranfeie lit

b10 me loperdent, Detto da Ipocriti, perch e in' un' certo
cenza a Dio di fare un peccato impune.I Latini havevano una i
che parte simili + Si Dijs«placet', " eee
. GASTIG ARE a misura di-carboni. Dar maggior gattigo di
il detingtente. 11 carbone e fra le più vili/ mercanzie; chef
misura, € per questo non ff guards così: per la minuta in darne
bra, e pero habbiamo questo dettato, che significa: dar' più |
nel Morgante. ef misura di crufea, e dt carboni, + o RE Oa

STANZA XV.; STANZA HVE)!

la quespo., e ognum parla della Strega, i i a
Si sente dire; A voi; largo, Signori,
E un bnomaccion più lungo a' unalega,

Dal Palazzo si vede conaur- fuori, Per esser vogavanti di galere

Poi sopra il Carro, ove Birrenoil leva, Chetal fa d Amoktante
E cinto ( come già gl Lmperadort ) Eperch'egli@un ?
Dialorowmvece, a' uncarton le chioma, Sentengtaro I hanea' nfarey
Va trionfante al Remo, non a Roma, Che Atalmantil non ha legniyne Mare
STAN ZA XWPLY

Perciò, mentre che tutto ignudo nato, Lat consulte it decreto ha renocate, ~
Senonch' egli ha due frasche per brachetta, Sicche di luimndn' ordine 8 >
Sh) bel trofeo si muone, ed è tirato Ed è stato spedito un Cancel re
Da quattro canallaccs dacarretta, €on più famigli « farlo-ratzenere o>

. I. Gigante Biancone legato ignudo sopra un carro e condorto fuori di: Palazzo
per esser menazo in Galera; ma quella esecuzione resta sospela, perché Malman+
tile non haveva', ne Mare, ne galere-, Haba 3 sun-

LARGO Signori', Date luogo; Fate ala. I Latini far far largo dicevano Sum
monere, Orazio. Neque confularis Summoner liter. Vedi sopra C, 11. fam

PIV" lungo a una lega, Iperbole usatissima per esprimere Lunghitiimoy Di
atiche pis lungo a una picea, 6 LO alae

BIRRENO, Intende birro, e'fi dice cos) per la. similitadine
con Kirreno, che fu amante d' Olimpia,secondo |" Arioito', dal! snes.
capertamente birro diciamo: lo sposo a' Olimpia, th ial ene

CINTA di cartone (a chioma, A coloro,, che per delitti-son la
frusta, asino, o berlina, fogliono per maggior vilipendio meceereinteta un bet
rettone di foglio', che per-esser a foggia ai mitra-epiicopale lovehiamano milena,
quali' sono 'quelle, colle quali farono:dipinti nelle itira del Palagio del Podestà
oggi detto del Bargello, 1 seguaci del caceato Duca @ Areael, le
per l'antichita appena si veggono'. VeditopraG, 6. Man, 56, €eque
per cartone, che per altro vuol dire quella carta grotia, che (erie
incartar pauat, cc, r

HAVO MO abandiera, Haomo a caso, inconsiderato » volubil
riofo nelle sue operazioni. +k Saga, Url al

SS ae
DVODECIMO,ETVLTIMOCANTARE, 525
IGNVDO nate. Affaito igdudo. Vedi sopra o, 2. stan. 64. IL Coloffo ad*noi
; e"mto ignudo; faluo.che ha due frasche per braghertas cioè duc
fogliedi vite-fatte di ferro, o  d' altro metailo dorato, che gli cuoprono. le parti

\& e SESLOU Re Ub =e a
« CAVALLAGCCE da carretta, Coloro., che in Firenze tengono carrette a vet
ra? per-portar mercanzie, ed arnesi da un luogo.a un' altro hanno sempre caval-
lacci vecchi, rifiniti, ¢-ai poco valore, e pero dicendosi cavalio da carretta ys"
intende cavailaccio di tal sorta. Qui il Poeta finge, che il Gigance Biancone fal
smelo sopra.avun carro tirato da quattro di questi cavallacci » perché 1l Colosso
detto Biancone sta sopra ad un carro, che si. figura tirato da quattro, Cavalli
anarini,. > ' 2 a
LA vinocate il Processo.. Intendi ha: mutata la sentenza, o decreto della galera
havendo considerato, che non se li poteva dare esecuzione, perché Malmantile
non ha gaiere,ne dominio di mare.; '

» » STANZA XVIIL STANZA XIX,
~Hragazzi infrattanto, che son triffi, E perch! ei.nonha in dosso alcuna vefay
o Aveder cio che fusse, essendo corsi, Lo segnan colpo colpo in modo,tate

Epaich' egli è un prigion,/i fona avvisti, Ch' mmnanzi ch' e finiscan quella feta,
let Bich eglieben legara, e non può sciarsi, Ne lo fuifaron, e conciaron male;
ly) Unitamente in un balen provuifi E al miteron, che atorre haueainsefa,
Di bucce, di meluzze, rape, etorsi, ( Bench giammaispuntate auefel' aig,)
* Cominciarono a far achi pri tira, Conquei suot merli, che non ban lepeane,
Ed anche non tiranan fuor di mira. Pigliar volo alt aria al fin conuenne,
Narra gli strapazzi, ed infulti, che vengon fatti al Biancone, e con questo
smostra il costume de i ragazzi Fiorentini, i quali quando un malfattore e condot-
-to per la Città in sull' afiao, o metio alia berlina, lo trattano nella forma, che
dice del Biancone, tirandogli torli, cioè gambi di cavoli, bucce di poponi, e si-
“mili immendizie. £ nota che havendo egli.detto, che Biancone haveva Jamice-
»ra, perché il Coloffo detto Biancone ada ha veramente la mitera » fa che i sa-
~gazzi la levino co i faifi di capo-al Gigante Biancone». i
-4N-nn baleno. Subito; In.un batter d' occhio, detto sopra \cstan[11]{42}. Di-
ciamo anche: in men che noo,balena; essendo il baleno, o il Jampo 4. siccome yil
vento, e'l fulmine cosa velocissima, Onde noi d' uno yche corra e sparisca.yia
fuggendo, diciamo = £' pare il vento, Ha fatto comenu baleno. Corre, come unit
Yacsta, Pare che"! vento se loporti, Virg. En. |. 5- J,
Primus abit, longeque ante omnia corpora Nifus ont
Emicat,\& ventis, \& fulminis ocyor alis,
Dove quell' Emicat vaic: Scappa fuora, e innanzi agli altri, come um lampo, Si
Swede correr la piazza in un baleno,
«»LVON tiran fuer di mira. Colpivano nel luogo, dove segnavano.. Vedi sopra
~C. 1. stan..37. dove troverai colpo colpo, che significa ogai coipo, che ¢' tirana.
Che diciamo anche Zorto bette, Mira e lo stelio che Scopus, voce Greca usata.da'
« Latini,; facta da Scopein, mirare,;
le PkIa Ache finife ques foffa. Primachee' finisse quell? operazione; Si dice
anche + quella musica; quel baccano; aes Ȣsimili, Vedi sopra C. ae fh 53+
c xx.; +, tbe

rad

=eSh 2

See CUR SS

=e

~  a ae

MBAS, %
eng

s

530 MALMANTILE

MITERONE a torre. Quel foglio, che per derisione si mette i
fattori detto mitera, come habbiamo accennato poco |
doil capo al delinquente, apparisce a i circostanti una roronda t
la parte di sopra di detto foglio molte volte l'intagliano a guisa d
farsi sopr' alle muraglie delle Città; e così havevano fatto a quelle
e'perd il Poeta scherza con la voce merlo, che è un' uccello note
glia dicendo, che se bene i merli, che haveva in capo Biancone n
mar messe le penve, e non havevano mai spuatate / ali; tuttavia
vouare,ed intende, che quel Afirerone fu fatto volare dalle bucciate,
che gli tirarono quei ragazai, con le quali glielo levarono di testa,
STANZA AX. STANZA XXIL oy
Paolin Cieco, il qual non ha fuvi pari Ed ci lo donaa Bieco, e a Pasian
Nel fare in piazza giuocolar' i cani, Col carro, e tutcel' altre ap,
E vendea l' operetic, ed è lunari,
E proprio ha genioa spar coi Ciarlatani,
Penssato ch' ex farebbe eran denari,
Se quel bestion venisse alle sue mani,
Pere' baurebbe,a mostrarsi, quel Gigante
Pix caica, che non hebbe l Eiefante.
STANZA XXI
Così presa fra se risoluzione,; Subito qui Paolino scende,
| Vain Corte a Bieco, e lo conduce fuora; Per trouar qualche st buon.
Gili dice il suo pensiero, € lo dispone Havendolo ferrato fra due ee
etchieder il Gigante a Celidora; Accio non sia veduti da persona, Vey

E Bieco andato a ritronar Baldone Bieco a tenerlo con due altri atendey
Tanto l'infipilla, e allora allora E se lo vede muouer 510 ha;
Ei corre alla cugina, e gliene chiede; Ma egliha fortuna, perch écni grande,
Ed ella volentier elielo concede, Che non gli arrina mancod

fande,
Paolino Cieco ottiene da Celidora in dono il Gigante insieme co! carro, sul quale
era, e sul quale lo condusse a Firenze, e si fermo ia fu la Piazza della Signoria,
havendo chiufo dewto Gigante fra due tende; affinché non fatie venduto, e men-
tre.così stando, Paolino cerca d' una stanza, per metteruelo, e farlo poi vedere
a coloro, che havessero pagato un tanto per uno, come si faceva dell' Biefaate,
fuccetle quel, che sentiremo appretio, * ie
“ ELEF ANTE, ¥u condotto in Firenze più anni ono un' Elefant
il popolo per la curiosica correva in gran numero a vederio sotto ie logge
Signoria ( hoggi detta de' Lanzi, perché quivi € il quartiere de' Trabanti, o fan-
ti della guardia del Serenils, Gran Duca da noi chiamati Lanai') dove fava rin-
chiufo in un tavolato, e si pagavano alcune crazie per entrarvia vederlos ¢
fio animale fingulare ne i noltri paesi, mori in Firenze per lo gra freddy ela
sua pelle ripiena, e lo scheletro nettato, e messo insieme si confervano nella Gal-
leria del Sereniss. Gran Duca. ucoensini aie
INZIPILLO'. Inttigo, stimold, pregd instantemente, e forse voce corrottas —
Sill:

da hbillare, Latino foilare, infufurrare, trovandolt nella' flor

traccaco feume: Di-ninwa miseredenca era stato antore, e nulla male:
date, ta



DVODECIMO,ED VLTIMOCANTARE, = 31

TRAINO. Diciamo quella quantità di roba, che possono strascinare duc buoi,
che i contadini dicono trainare, ed il veicolo chiamano traino, o treggia, La-
tino traba, o trahea, a trahendo, Virg. Georg. 1, Tribulaque, trabeaque, o ini-
que pondere rafiri. Si dice anche sraine una mafura di travi, che contiene quattro
Breccia quadre. Qui intende quel carro, sopra il quale era il Biancone con tutti
phate arnesi, e pigia la voce sraino nel significato della voce rreno usata per

rsi intendere carro, e bagaglio dell' artiglierie; !a qual voce s'accorda 'colla.
Franzese Train. Noi percio la diciamo ora Treno,rappresentando quella prooun-
zia; ora 77a:mo coll' accento fulia prima, non facendo conto della pronunzia
Oltramontana, ma della (crittura. Qui il Poeta dice Traine coll' accento fulla.
penultima; per accomodarsi alla neccitsta della rima. Franco Sacchetti nelle Ri-
me fimiimente pose questa voce nella fine d' un verso,

Per tirar colti piedi un gran traino,

LA Piazza dela Synoria, La Piazza, che hoggi si dice Piazza del Gran Du-
a, e si diceva de' Signori, o della Signoria, perché è d' avanti al Palazzo de'
Priori, e Gonfalonicri di Firenze, che si dicevano la Signoria, nella qual Piazza
@ la fuddewta loggia, detta de' Lanzi

CHE non gli arrsva manco alle matande, Cioè non gli arriva ai bellico, perché
mutande chiamiamo propriamente certe piccole brache, le quali si potiany,quan-
do si va a bagnarsi in Arno, per coprire le parti vergognofe, le quali mutagde»
per ordinario cuoprono dai bellico fino al principio della colcia.

STANZA XAlV, STANZA XXV.
Piange Siancone, e chiede altrui mercede, Quei tre yc ognor came cuciti a i fianchi,

E mentre il Fato, e la Fortuna accufa,

Euor delle tende si guardo gira, e vede
« Perseoy'ha in man la testa di Medusa,

E immoto resta li da capo a piede,

Ne più si duol yma tien la bocca chiusa,

Perché col Carro, e tutta la sua muta

De cavallaccs in marmo si tramuta,

si favan quivi,accioch'ei nofeappasse,
Privi di senso allora, e freddi, e bianchi
eAnch' eglino si fanna immobil sasso.
Ata perchs'l protungarmi non vi stachi,
Glie me',c' a Malmantile io mene palji,
Ove giù amici Paride ritrova,

E sente, e' ogni cosa si rinnova,

Ii Gigante Biancone era così grande, che avanzava il capo sopr' alle tende;
nel girare, che egli fece la testa verlo la loggia de' Lanzi, vedde i teschio di Medusa
tenuta in mano da Perseo; per la qual vista rimase immobile, e divenne
sasso tanto lui, quanto il carro, i cavalli, e coloro, che gli erano d' attorno; E
così il Poeta da la sua fine, e si sbriga dal Gigante; di poi ritcorna a discorrer di
quel che si faceva a Malmantile.

PERSEO, ¢' ha in man la testa di Medusa, Questa è una statua di bronzo, la
ae € fiuata sotto un' arco di detta loggia de' Lanzi; opera di Benucnuto Cel-

i; e rappresenta Perseo con la testa di Medusa in mano, verso la quale statua,
guarda il Coloffo detto Biancone, perché e di marmo bianco. E nota la favola
di Perseo figliuolo di Giove, edi Danac, il quale uccile Medusa figliuola di Forco
strupata da Nettunao nel Tempio di Pallade, la quale percio sdegnata convertì
i capelli di Medusa in serpi, e fece che la sua facia faceili diventare di sasso
coloro, che la guardassero: Ma il detto Perseo havuti da Mercurio gli stivali, ¢
la scimitarra, mentre Medusa dormiva s le taglio la testa, la quale pot ee

xX 2 messe



sg IFATHAI OM ES +9 vi h
AS3HL ) più Ofisip MALM he thy; 4 ie on si
miesse nel proprio 'feudo, Di questa favola si servé il! Poeta 4 } 7
gante;dicendo, che per haver' eghi mirato questa tettadé-]
marmo, € così da graziosamente una favoloia origine a questo f
rappresenta Nettenno Dio del Mare', ed! è»posto nella: Piazza' del G
sopr'ad un carro tirato da-quattro cavalli marini nel mezzo a una
quale riceve I acqua', 'che scaturisce davaleuni niechi, o conchiglic
in mano da alcune statue di Tritoni-alte quanto le gamberdel d
or dette stawe stanno attorno:"E queste il Poeta finge', che sieno

mipagni, che dice fargli cucits a i fianchi, e che non gli arrinano a le
dé'; E così viene a conformarsi col gruppo, che si vede di queste ttarue
fo tutto di marmo,

CVCIT 1 ai fanchi, Stretti attorno, come se fussero euciti, Detto uk
per'esprimere uno, che mai si levi-d' attorno a un' altro;€ qui corna bene,|
Ché quelle statue sono così strette attorno aj Coloffo, che paiono cavate:
fo marmo, del quale e cavato il Colosso.

GLle me', Gli è meglioy. Vedi sopra C. 2. st, 10. <a?

PSTANZA XXVEy STANZA XXVHE
Poicht Baldone eAalmantile ha preso, Cos} cercando le grandexe i |

E tutte quelle povere brigate Soe @altrihor feo Ve, ?
Saluopera chi non si fusse arreso ) Onde tornata Celidora, il Lage
mii se ne son ite a gambe akace, De i popoli padrona, e dello Stato
Sitché'da queste havendo al fin coprefo Temendo ancor de'

.

'Pot Bertinella, ch ella l ha infilate; Nuovi Miniffri fa, nuove ve i
Perammazzarsi sfodera un pugrale, Se ben de i primi poco ha da temere
Ada quei,ch'é buono,non le vuol far male, Che tutes ban ripiegate le bandiert i
STANZA XXVIL. STANZA BALK
Clienon fo come gli esce fra le dita, E per estinguer la memoria i
B/fulta in Strada, che le gabe ha destre, Di Bertinella in ogni gente, e-loco }
Ov" ella a ripigliarlo'é pos spedita Si levan le sue armi, il suo ritratto,

Tagliato in croce si condanna al fusce'
aE perch'elt habbia a raccorciar, la gita, Un bando va di poi, \& averum patto
Le fa pigliar la via dalle finefire; Neffan ne parti pite punto, e poco
NEMa wa sh, ma poco poi le importa Sotto pena di fear in su la fume
MT rovaricht amarza,se viginnge morta, Quattro mesi al palarzo del Com
Celidora tornata padrona di Malmantile fa buttar Bertinella
ordina nuovi Magiftrati, e comanda, che non si parli più di Berti

villime pene. jo faite

Dix'chi dopo di lei fa le mineftre;

ELLAL ha infilate. Infilar le pentole, vuol dire Esser rovinato
ver finito-, o perduto la roba, e la vita, ec, che di tutto s*in cok
mente. “tale ? ha inflate. Latino decoxit. +20 sett

LE gambe ha dere, Non, che quel pugnale havesse gan
dire } cheetlendo grave, gli fu facile andar' a baffo in strada';
perie 'finestre anche Bertinella da chi fa le minefre, cioè dachi
avichi comand; chee Celidora ritornata padrona di Malmanule.
gacge ae peccato, Ha la pena det suo fallire, e che ha m
whet; F



1 v¢ t E LAN 7
-.  DVODECIMO;EDVLTIMOCANTARE. | 533
' flaver voluto per strade indirette farsi Regina', usirpando queld' altri iio! is > /
be i. icsanlig vogliamosintendere uno, che piocenea oe taper fare Ogni-cosa
meglio degli altri diciamo; M.raleeit Lagi, Che il Lagi fu anticamente un Sen-
icato wv Firenze, che faceva tutti i negozzj della piazza': Si dice
rO per scherzo, e per una certa ironia, e derisione. ho “ogee
* HANNO ripiegato le bandiere, Cioèhanno finito; Son morte, Il Petfiani,
parlando di se medesimo in questo proposito disse + ty
core edi primo tramontano a quest® ascintte ae)
si Be Ditems pure sl requie, e il Miserere,:
Perch' so fo vela, e piego le bandiere; '
 E buona notte; a rinederci tutti,
LE fue' arm, Intendi'}*infegne della sua cafata, o stirpe. ue
7 ~ STAR in su la fune quattro mesi. Now è posibile' star in fa la coda quattro
y hore, non che quattro mesi., ond' io penso, che con questa iperbole voglia iaten-
sia condennato alla morte, alludendo agi' impiccati, che in un certo modo
quando pendono dalle forche a vista del
popolo; st poslono dire stare in sulla corde,

be in sulla fune.
¢ STANZA XXX. “STANZA* XXXIIL
jee | Yr Orarore intanto de' più brani Spiegafi se desea 4 ttn tavolotto
" ACelidura Aaimamue inuia, Vol abito mavi di mezzalana,
a 'Che det Caffelo ad essa da le chiavi, Che infu fianchi appiccato ha per diforto
Evende omaggio con la diceria; Pn lindo rid aief alla Romana;

Ed ella in detti macffofi, e gravi
Pronta risp a tant' Ambasceria;
Inds le chiavi piglia, e nn' altro mazzo
= Wi quelle delle stanze det palazzo.
ae STANZA AXAIL,

E perché gli è un perro, ch' eli' ha voglia
Di riveder, come ad arnesi e pieno;
Del Mamoye d'altri addobbsfi dispoglia,
E comincia a girarlo dal terreno;
4Guardarobi aspetta, ead ogm foglia,

Poi viene un verde nuouo camiciotto
Con bianche imbastiture alla balana;
E poi due trincterate camicinole,
Che fanno piatza d' arme alle tignuole,
STANZA XXXIV,

Vua Rimarra pur difaianera, ~~
Per dove si fa a' sassi arcisquisita,
Perché gli aliorti, e it banero a spalliera
Pavan la testa, e in giu meza la vita,
Portandola alle

'i; te,o0anna fitra,
C' ad aprer gli usci patono it baleno; Torre, e comprar si pio roba infinica,
è E subito poi lesto-uno safiere Cb elt" hadue manicon s) badiali,
mn Quand' elta passa, le alza le portiere. Che è ine quattordici arfenali.
f STANZA XXAIL, STANZA -XXXV,
Ed ella se ne va sicura, e franca, Vina cappa tane bella, e pula
b Sapendo ogms traforo a munadito, Di cotone; se ben vesta indecifo,
ie Perché troppo.non è, ch'ella ne manca, S' ell'¢ di drappo, o pur ringiovanita,
EP abito, fin quando havea mario, Perché non se ie vede pelo in viso,
'0 Scese; )£i70, fali ne mat fu hance s Evvi @ abiti pur copiainfinica,
2 t Sin che non hebbe di veder finite; Mia chi unto, chi roto, e chi ricifo;
2 All' ulssvia si fece in guardaroba Che il tempo guasta tutto; e per marura
% eAprir gli armadi, e cavar fuor la roba, Cosa bella quagzit pala, e non diira,
4 Malmantue manda un suo Ambalciadore, o Depataco a renaer' wbienes:
a Ce.
f.

E


44.534
a Celidora; ¢d ella attualmente, e corporalmen
tutte le stanze del Palazzo, ed in Guardaroba fa la
veramente adeguati a una Regina di Malmantile.. 3
RENDE a la diceria, Cioè fece una Orazione d'
mone, o Discorso, col quale refe ubbidienza. is. 4
HA voglia di rinedere. Ii Poeta (prime benissimo il genio unit
fire donne, quale è di rivedere tutte le casse, armadi, ec. subito, che
o maritaggio entrano in una casa a loro nuova, ho isch ete
TERRENO., S' intendono qui, secondo l'uso., le prime fanze d' una cal
che sono al piano della frada, Del reo Terrenoé la tetra stessa così,0 così ¢
dizionata. Latino terrenum; folum, ager.» - send, -
PALONO il baieno, Cioè tannopretto, Dante Pars 25. Subito
di baleno. Inf, 22. i2 men, che non bafena, vatiot ”
OGNI traforo. Antendi ogni porta, ognicriuscita,/ogni minima:
4A MENA dito, Sa benitimo. Latino caller, Le sono notissime st
L ABITO' fin quando banea marito, Celidora, come s'è detto sopra C,
Fu moglie del Re di Malmantile, e da lui haveva ereditato i Regno, i)
MAVE, Color wrchino chiaro. Azzurro sbiancato, i
GV ARDINF ANTE. Vedi sopra C. 5. tt. 8. *: geomet
MEZZ ALANA, Tela fatta di lino,è lana, che inuna fola parola si dice
ancora accellana, quali accia, e lana; roba assai da i nostri Contadini.) |
C.AMICIOTTO.. Così chiamano le Contadine, quella veste da donna, che le
Fiorentine chiamano fortana, Et
CON bianche imbastiture alla baixana, Costumano le nostre Contadine di fare
nelle loro vesti yicino a terra una cintura con punti di refe bianco in sul nero jun-
ghi, acciocché si veggano da lontano, e queiti punti sostengono una piegatura
fatta nel giro di detta veste per accortarla, e serve a loro per ornamento,0 guat-
nizione, e si danno ad intendere di far creder nuova la medesima — causa
di quella punteggiatura, e che aliora sia uscita delle mani del Sarto; il ee
quando vuole imbastire,.0 dar priueipio a cucire yo' abito per mettere int 9
eda segno i pezzi, che vuol cucire, e solito fare tal punteggiatura larga, da
questo imbaffire si dice imbastitura altrimenti feffitura, o ritreppio, Latino subfutnr4.
E questo verbo smbaltire servc per intendere ogni cosa principiata, e non perfezio-

nata; come éo ho imba(Pito L' orazione, che debbo recitare, ed in poche ere ”:
che diciamo abbozzare. we

BALZ ANA. Iniendono il giro da piedi della veste; altrove Pideos 'Latino
limbus « LF

TRINCIER AT E.camicinole.. Vuol dit camiciuole consumate dalle tignuoles »
per la similitudine, che è tra una campagna pieaa di trinciere, ed.un panno ple
no d' intignature, che percio apparisce bucato, € trinciato, Vedi sopraC. 8. st
51. E.che cosa sia camiciuola. Vedi sopra C, 6, st: 57, at otwe att

BANNO piagza a arme alle tignuole, Vedi opra Co. 51. -questo medesimo
concetto sopra il capo del Tura; B che sia tignuola al C, 6. st. 54. € Cs 10. (h 12+

ZIMARRA, Abito, che già usavano portare le Donne Fiorenti all?
altro abito detto sottana; il quaic da i Latini e detto amiculam, il qual'

' YY



tie
a

“= SSeeresiut

DVODECIMO,EDVLTIMOCANTARE. ~ 535

'veramente assai decorofo, e modeflo, e non come quello, che usano hoggi, del
quale si può dire:con Quinto Curzio lib. 5. Feminarum conniusa inenntium in principio
modestus eft habitus, dewde fumma quaque amicula exuunt, panlatimque pudore
profanant, ad ultimum ima corporum velamenta proyjciunt. Ma tornando a proposito:
Questa specie d'abito detto Zimarra haveva intorno al collo un collare grande
(che chiamavano bavero) fatto di tela incollata, e cartone, e ripieno di stecche
d'osso di balena; ed in su le spalle, dove ha principio il braccio un giretto attorno
al braccio fatto della stessa roba, che il bavero) qual giretto il nostro Autore
appella aliotti, perch così si chiama, ed alle volte si dice piffagne ) dal quale
pendeva una manica larga., e grande quanto una buona sporta, la qual manica
non s'imbracciava, ma serviva così pendente per ornamento, e per una certa
grave accompagnatura; ed oltre a questo dava commodita di riporvi fazzoletto,

Oaltro, che occorretie. Di queste maniche, tali se ne son vedute a' mici giorni,
che farebbono fiate capaci di cinquanta libbre di grano l'una, e più; o però il
Poeta dice, che sono il caso per andare alle nozze, ed ai mercati, perché vi si
può mettere molta roba dentro: E gli-aliorri, e banero difenderebbono da un col-
Po in riguardo della roba, di cui son compolli; E dice /a rea; perché questi ha-
veri, nascondevano dentro di loro tutto 11 capo di chi gli portava; e tali aliorti
si sono veduti, i quali coprivano più di inezzo il braccio.

DOVE si fa ai fafi, Dove si tirano le fafiate; il che segue in Firenze in Mercato
nuovo, dove 1 garzonetti delic butteghe de' Setaioli quindici, o venti giorni
avanti alla Solennica di S, Gio, Batilla fra il mezzodi, e il vespro fanno fra- di
loro alle fafiate, e necetiitano tutti li bottegai di quelle contrade intorno al
Mercato nuovo a star ferrace per quell' ore; e questo fanno per solennizzare la detta
festa quel tempo innanai; e per questa ragione tutte le botteghe, che sono in quel-
la firada, dove tirano i fatfi, hanno la riuscita in aleca strada per di dietro, di
dove entrano i macitri, e lavoranti, senza aprire lo (portello principale, e quivi
attendendo a i lor lavori, laiciauo che i loro ragazzr si piglino per quell'ore tale
spasso, anzi ci sono taiuoica de i maeitri, che comandano a1 loro ragazzi, che
vadano a pigliarii, spaveatati da un profetico detto: Guai a Firenze, quando in.
Mercato non si fara ai sassi; V sano di fare a' fai anche in Roma i ragazzi Tra-
fleverini. E fare a' faji, Hgucacaiente s' iotende » Mandar male, rovinarsi, get-
tar via il suo, Latino dilapidare, fare alla peggio, e operare senza giudizio; si
faceva a' sassi ancora in Firenze per accafione d'allegrezze pubbliche, e una fine-
fica di rame traforata fu posta al Palazzo de' Medici,oggi de' Marchesi Riccardi
Per vedere questo spettacolo, come e sato da altri scritto, ed osservato.

ARCISLVISITO, Ui cafissimo, buoniitimo,, attissimo, e più, se più si può E

dire. B' un termine, ches' usa per farsi intendere; più fu, che il superiativo, di-
cendosi buono, più buono, buonissimo, ed arcibuonitime. Ma dicendosi buono,
Migliore, in vece di più buono, e isquisito in vece di buonissimo, che fa.
V effetto del superlativo di buono, non pare che sia ben detto più isquisito, e»
isquifitissimo,facendosi cosi un superlativo di superlativo; tuttavia per J'ulo inteo-
dotto non farebbe riprefo chi lo facetle; ed io crederei, che fusse meno biatime-
vole dire, arcisquisite, che isquifititfimo, perch non trovo troppo in uso il dire
più isquisito, onde non può s' uso antrodurre isquifitisimo.s che toguirebbe al più
squi.

ae



536, uuie nohace dae aaa
isquisito..;L Latini dicono bonus, melior 9 che: Q d F
byono,, migliore, e isquisite; ed io conde cic i che Bctedfinn piste ass
miffiaus:, che faonerebbe più isquisito, isquifitissimo, se it I

trova eptimiffimus.. Appretio det nostri Autori Toscani si trova, 1
molto, aijai, e simili a i superlativi, come notammo Coat 17. >} ia r
buona grazia di essi, lo stimo.errore, perché molto, più 5" » Hiufili
faculta di scemare, e non cre(cere il superlativo,. aa

er esempio if tale e Luonissimo, vuol dite il tale è perferramente
iamo molto, certo', che (cemiamo la perfezione di buono 5
molto buono, ma von perfettamente buono, eficado maolte una'
s2.5 € non indeterminata, come e il superlativo: EB — » che
1iguifito, e isquifieissimo, o arcisquisite, hanno presa la vace
tivo da per se, e non come per superlative di buano; il che vi 7

tofna poi all' addigteivo aigliore, che non riseve altrazione, nomdicendof » nondicendof i;
migliore, Be migliorissimo, le hen si dicewmalte miglione, e assai mia
marlod' essenza;. come ia bbiia thes detto s»perché solo, o  affat miglit
men buono, che non fa migeiore assolutamente detto:, se non comparando:
all: altra quale sia.di loro meglio, st Amr
i ZANE, Colore fra il paonazo, e il lionato. 2p
OTONE, Vuoldire bambagia non filata, Manoi per cotone:
sorta di panno col pelo annodato; come.è la saia rovelcia 50 il rovele [
Hon si dicono corone se non. hanno il pelo aanodato, che allora si dicano di coteney
o actoronati, Dice, che num e certo se sia rowescia, o drappa 5) pceaim
la feta. 2 ellendogli caduto il pelo, per esser logoro je perché:è senza pelo dice

che € riagiouanito; Sicch¢ in sustanza vuol dire che: era usato, i lal.
R(CISO, Qui vale per intendere consumato nelle piegature d'un di !

epanno 5, per essere stato così piegato lungo tempo; che per altro ri
un legno, o altro materiale tagliato ne] mezzo, ed e il contrario' div rife se

nel oy pela per illungo, Vedi sopra o. 11, tan, 36, ricife,
ANZA XXXVI. STANZA X\&K.
Basta es eve qualcosa un po cattind, Due altre 'armadj poi i fur

Che Celidora ha quint abiti, e panni, Che ? anoe tutto
Che al certo (tuttanolta ch' ella vina )
Puofrancamence andar in lq co gli anni,
Ma perché al suo char magnonos'arriua E un' altro di pin tr
'Di certe roppe, scampoli, e foppanni + Bealze, e scarpe, e)
Top Wimpaccio vollese a quella gente, Chea vedersi p er
"Ch edt'ha a' intorno,farne un belpresete, Ve poi la nidoigi
. STANZA xxxViil st
mui se si parte ed Apre uno Riperto A
2 intagli, e a' arabeschi ornato, e ricco,
“E trois due cafferse di belletto
Cort! altre di pezrette, e @ orichitco
va il Poeta a narrare glia arnesi, e
hon si parte dallo feh
' faye | Ae a ee

n

ME



Si elcr ibs.

a

SERRE ES o =

SSE ST Pesrsr st i ta.

DVODECIMO,EDVLTIMOCANTARE: 537

contro alle donne, mostra; che se usano il belletto, ed il liscio, hanno anche
bisogno della medicina da rogna, e del rottorio.

VN po cattiua, Quel po vuol dir poco per la figlira Apocope; ed un poco cat-
tiva, trattandosi di abiti, e d' altri materiali, s' intende per lo pit', consumati,
2

vecchi.
TVTT AVOLT A, ch' ella viva, Pub francamente andar in da con eli anm, Pav

che voglia dire, che se Celidora vivera, ha tanti abiti, che le basteranno molti.

anni senza farsene di nuovo; Ma dall' essere gli abiti della detta qualita, si com-
prende, che scherzando vuol dire, che se Celidora vive, invecchiera, perché
andar in Id con gli anni vuol dire invecchiare, come s' accennd sopra \cstan[2]{2}.

(siginines Ritagli, pezzi di panno, o drappo. Scampoli, vedi sopra C. 11;

in. 22.

SOPP:ANNI, Fodere, cioè tele vecchie, che hanno servito per fodere d'abiti.
Scherzando burla la generosita di Celidora, la quale con queste galanti ciarpe,
che son fondacci d' una bottega di rigatticre, o ferravecchio, regala i suoi più:
cari per non apparir meno generosa di Bertincila, che regalo la patcona, come
vedemmo sopra: \cstan[1]{81}. a:

D* oronetro: Par che dica d' oro pulito, e puro, ma intende wetto d' oro, cioè
puro; senz' oro, Equivoco usatissimo in'quelto propotite,

LA miaferizia per la casa. Incendiamo 11 Cariello', o turacciolo del ceffo; e
sto, perché un tale detto Galeno, che andava per Firenze vendendo tali carielli,
gridava shi vuol la masserizia per la casa, in vece di dire, chi vuol Carielli; od
era bene inteso da tutti,

RABESCHI, o Arabeschi, Specie di pittura fatta a fogliami, fiori,
mascheroni., o altro, tutto aggrottelcato, cioc sproporzionato dal naturale, detto cosi,
perché forse tal maniera sia venuta d' Arabia, secondo che si può dedurre
da. Cel. Rodig. Jib. 29. c. 5. dove trattando delle Lamie, e delic Sircae, dice;
LaAmmiam vero opera parerga ex Arabia mastichen vocant,

SELLETTO. Liscio. Mestura, con la quale si lisciano, ed imbellettano le
donne « Vedi sopra \cstan[9]{38}.

PEZZETTE. Sano pezzi di tela bambagina tinti col cremisi, e zucchero, ed
altre sono di carta fabbricate in Spagna, e se ne servono le femmine per colorirsi
di rosso la faccia.

ORICHICCO. Gomma di Ciriegio, di Pesco, o di Sufino, ec. della quale si
servono le femmine per lustrarsi la faccia, e per appiccarsi veli in su la teita.

“PER Jambicco. Adagio adagio scaturendo da piccioli fori fatti nel coperchio
del fiaschetto., come s'ufa dei' acque odorifere. Lambicco e il nao della campa~
na, e d' ogni cappello per uso di stillare, donde lambiccare, e passar per lambicco,
intende stillare; E lambiccare, o lambiccarsi il cervello, è lo stesso che mulmare,
detto sopra C, 10, stan.7.

ALLERA, Pianta nota, le di cui foglie eruono per cauteri; e così i ceci bian.
chi, li quali per tal effetto erano ia quello (tipo.. Da queste cose vili comprenda
il Lettore, che il Poeta si maaticne sempre in su gli (cherai, deferivendo una Re-
gina, e Palazzo ricchi di quegli addobbi, che son conuenienti a una beac stant
cOntadina, e decenti alla grandezza d'una Regina di Maimantile,,

% Yyy. STAN,


Sh. MALMAN TILE 1980
STANZA XXXIX,.., ith NZ 3
dun caffon diferro vada REREO, c i i color |
L Quiuitvoua il morto, nia dd vero, s -
Che i diamanti, e le givie di gram pregro
Lon v'bano che far nidla, e sono un zero;
Lerche si tratta, che vi. Safe un wero
bi perle, che se ben pendeana in nero
Examsi grosse, che st sparfe vace,;
Ch' ell' eran poca manco d' una. noce; Sun i quartrini, i precioli, e i bateati,
STANZA XXXX STANZA -XXXXIL
D? anells ya! orecchini Vé1h marame} 'Poi ne venixan gli occhidiciueste;
“Tanti gioie!ls pot, ch' e un fracasso; Ma il proseguir più olere fa interrortes,
Perc' alla donwa:

Di medaghe dorate 50, vavindi-rame' a
dir, che" Duca levolea far

Un moggio ne misurano,, @ di palo;
Ala quella e sparr ates, ed nn litame Ond? ella il tatto nelcafjon rimette

Risperto alle monere, che più baffo E riferrato scende giwdi (orto,
Le più belle comparsero del mondo; Oue Baldon ? aspercarn iftinali,
Ch! in faseri poses creffi stanvo al fonda: -» one partir di quini fha im sul? ali >

: STANZA: EMBKI MO vinnd cow 2

Per e agginstare omas tutte le cose, In punto, @ questo fine aller
Che pin desiderar non si potea in: ier 'bined
Egli, ch' eva per far come le/pose La puliva.per metterie la fellay

LA ritornata s idef? alla Dacea, Licenrioffs costidullasorellay o >
Celidora trova il caffone de'.danari,, e coi tal-occatione i Poeta'
monete Fiorentine eficttive, ed immaginarie.. kn tanto che Celidora va vedendo
queste ricchezze; vien da lei Baldone-suo cugino per liceoziativ) 9
TROFA it morta, Cioè trova il buond. Diciamo rrewar it morta, o fare nit
morto, quand' uno trova ripod qualche gran vallente, o fa in gua-
dagno.. A. P
LON o ha che far nulla, Par che voglia dire non si stimano, vispette al? altre
Givie, che sono in.quet luogo; ma in eisai vuol dire; che quedo non è luoge per toro
cioè non ve ue fond, i b tone Se
Sf trata. Si discorre; Termine assai usato per esprimere una che
s' habbia di qualche cola'; quasi-dica > Si difeorre comunemente, che'
così..
AL marame. Una quantità grandissima. Marame propriamente vuol dire ogni
rifiuto di mercanzia, come quella, che dal mare è gettata a' riva bi i”
tum, Ma quando diciamo marame nel modo; che! è detto: nel eel
intendiamo abbondanza così grandé.d' una cosa, che generi naulea, €
disprezzabile la medesima cosa. Fra i nottci Contadini Gedice
tendesi? avanzo, e rifiuto delle frutte rimatte lord, dopo. la celta', o° vel
delle migliori » noa fo s¢ essi Rroppiano la nostraparola, o-feonoi Cori
la loro, dico bene che mi pare più fighificante; Amaramejehe J
Fiorentino quello, che questo, che per così dire', ha del Nape
Vedi il Vocabolario della Cru(ca alla voce Cerna',

UN fracasso. È lo stesso che un flagello, un barbaglio detto sopra \cstan[7]{5}.

UN moggio. Il nostro moggio è di staia 24, lo staio è di libbre 50.\ di grano, e
la nostra libbra è once dodici, Ma qui è detto iperbolico, è significa quantità
grandissima.

RISPETTO a questo, A paragone di questo; cioè a paragone delle monete,
che son più basso.

I pesci grossi stanno al fondo, Detto, che significa: Il meglio sta nel fondo.

PIASTRA, È lo Scudo, o Ducato d'argento Fiorentino, che vale lire sette
ed è moneta effettiva\footnote{Qui il Minucci descrive il sistema monetario toscano a lui contemporaneo, di molto anteriore la decimalizzazione del 1826. In quell'anno si pose come base il Fiorino moneta effettiva in argento, divisa in 100 quattrini moneta effettiva in rame. Nel nuovo sistema, la lira fu posta a 60 quattrini, il Paolo, o Giulio che dir si voglia, conservò il valore di 8 crazie o 40 quattrini, lo Scudo secentesco di 4.20 fiorini fu rimpiazzato dal ``Francescone'' di 4.00 fiorini.}. Il Fiorino è moneta immaginaria, e valeva quando più,
e quando meno, essendoci anche il fiorino d'oro, che forse è quello che habbiamo
ancora hoggi d'oro effettivo, e lo chiamiamo zecchino gigliato, ma il fiorino
ne immaginario, ne effettivo appresso di noi non è più in uso, Scudo d'oro
è moneta immaginaria usata da i Mercanti per facilita di scrittura, valutandolo
lire sette, e mezzo, se ben molti per scudo d'oro intendono la mezza doppia.
La Lira moneta d'argento effettiva, e si chiama Cosimo, e vale dodici crazie.
Il Giulio, che si chiama anche Pavolo è moneta d'argento, e vale otto crazie,
Il Carlino pur d'argento effettivo ne vale sei; ed il Testone val due lire; questa
moneta già in Firenze si chiamò Riccio, dall'impronta della testa del Duca
Alessandro de' Medici, che era ricciuta. La mezza piastra e d' argento effettiva,
e vale lire tre, e mezzo. La crazia è moneta d'argento basso, ed è l'ottava
parte del giulio. Il quattrino è moneta di bronzo effettiva, ed è la quinta parte
della crazia. Il soldo moneta immaginaria che vale tre quattrini; ed il battuto
ne vale due: hoggi l'habbiamo ambedue di bronzo effettive. Il quattrino si divide
in quattro denari di bronzo effettivi, ma hoggi non se ne vedono, se non in
occasione di tributi Ecclesiastici, che sono presentati, e son poi resi, perché gli
possano haver un'altr'anno.

OCCHI di Civetta, intende le monete d'oro, come il doblone, che vale lire
quaranta. La doppia, che vale lire venti. La mezza doppia, che vale lire dieci, Il
quarto di doppia, che vale lire cinque. L'ottavo di doppia, che vale lire due, e
mezzo, che tutte sono d'oro effettive. Habbiamo ancora il zecchino, il quale
chiamiamo gigliato, che vale lire dodici, ed è il più purgato oro che si conij, e
si può dire il nostro unghero. Si trovano ancora de' dobloni di quattro, e cinque,
e di sei doppie l'uno, di conio Fiorentino.

SPARTIMENT!, Divisioni, feparamenti. Chiamiamo spartimenti quelle,
divisioni di tereeno, che si fanno ne 1 giardini per piantarvi le cipolle da tiori.

ali (partimenti:, se bene sono di diverle figure, si dicono anche quairi. Vedi
pe C,6,-ttan. 63..E per similitudine aiciamo spartimenti te divisioni » che si
trovano ineafiecte, o scatole, come erano queiti delle monere,

VENNERO pris hafere. Intendi Avvisi, o imbalciace, che Staferta appresso
di noi, e:1o stesso, che Corriere. Sp. efafera.

\ BAR matte'. Elo stesso che abbaccarli con uno e parlargli, Vedi sopra C,
2, stan. 59.,in altro significato, — > ne

STA sm full aii. EP all' ordine per partirsi. SST

. FAR come le spose. Significa ritornare; lo dichiara il Poeta medesimo,dicendo:
Tdest la ritornata; E questo perché già costumavasi, e forse ancora in alcuni Iu

es:

a

, ee ee ee

=e 2 SS Sw

PR 6S we

d-
Koyy 1s ghi

=

s4o
hi@eoRitma, che le si dopo essere state dicti', o pre:
foie rotniao alla casa paceraa's Fer sephe qui giù

Teniarns dell Achinea. Taupe lo fallone, 'che* cated
che Achinea, o Chinea, intendiamo il cavailo buon rer
éuina specie di cavaili particolare «Sp, bacanea. Franz, bacquenen'y

STANZA XXXXIV, ts “STANZA. Xx
O mai è tempo, cara Celidora,
Ch! inverso li miei sudditi m' appresi >
Che 'l trattene*mi di vanvargio faora,

Pregsndicar potrebbe a' miei interessi Dite, non ci oi fusse corda,
Pero qui refea tu co! tuoi,sn buon bee, Bifog a Lmeteae epee a
E farti anwe, e rispercar da essi, ( Rispsfe il General) 3 ella 8
Ed in ordine a questo i conviene ee ome t
Fare anche un' altra cosa per tuo bene,

STANZA XXXKXV.
Perché, s' io parte ei »cugina mia,
Non fo 2 se tn ci havrai tutti i tnsi gusti,
Che qui non è neffun., che per te sia,
Mentre forsee poi nuowi difeusti,
Ma voglia il Ciel, ch' io dica la bugia

Ed ogni modo vo', che tut' aggiuiti, 'tipo presto sles of
Per sicurtà con an compatie » Ugquale Vuolotu? parla. a Her =e |
S accafi teco, e qucfto, e il Generale, D: mat più si, e daccela in fa

STANZA XXKXVL STANZA AKAM;

LT byei hati difender si da vanto, Ed ella nel sentir, cons eit affrin
Che tn vedi,egli ¢branoquarun Marte, A dar pronta: rispotta atal do
E se finor per noi ha fatto tanto, D' un modesto rossor tutta,
Pifa quel che eifara,s'egli entra aparte,
Orsit  daglt la mano; cana [it ilgnanto;
E voi non ve ne fiate più in disparte,
Casa Latoni, o Amostante nostro
Fareui innanzi, dite il fatto vostro

STANZA L

Degli dunque la mano in mia prejenzas 3 Ma per non recar tedio
E voi, o General, datela a les, Idest a chi ascolta i versi mitiy
Ch io 'voglio prima della mia partenca
Veder solennizzar questi Flimenci. Lascidgliyadiame;

Baldone dà per sposa Celidora al Generale Amostante Latoni “ai
dopo haver narrato il discorso fatto da Baldone a paliow per indurla a
tarsi d' haver questo marito, ed i soliti lezzi donneschi farti da 2
dir di si; passa a di(correr d' un' altra sposa, che è Psiche, cone ee i

Lagere ouave.

hai neJunsche per te sia « Non hai nefiyno, she aid

a

'

DVODECIMO,EDVLTIMOCANTARE. 548

OVVTA. Termine che significa spedizione, o incalzamenio a far presto. BE' il
Latino Hia,age. Vedi sopra \cstan[6]{90}. alla voce, horse, aiegh 3
PASS ATE gud. Venite qua. Lat. adesdum. B: modo di dire, che significa
comandar con imperio,.¢ con (everita, ed ha del bravatorio. R59,
SE vi piace la pannina, Se vi piace la mereanaia, cioè Celidora.
NON ci tenete più in sulla corda. Non ci fate più Aentage,o desiderar la rispo-
sta. Non cé renere piis coll' animp dubbio, e sospese,:
SON bell' e accordato. lo sono, affatto d' accordo; son contentissimo. Vedi fo-
pra C. 3. faa, 14, Questo termine bee, '
TERREL d haverne di beato, Lo riputerei mia gran felicita, Stimerei d' haver
gran forte, WV' avrei di carti, Mi terrei d' etfer beato, ee
EGLI¢ dower sentir  altca campana. E' cosa giulta sentir I altra parte,
TRANA, Questa voce non havrebbe alcun significaco, se bene e assai usata 5
ma perché pace, che immiti il suono della tromba, quando si da la mossa a i ca~
vali, che corrono al palio; ci serve per esprimer mxovité  spedisciti, sbrigati a.
far la tal cosa, Q pure e detto Trana, cioè tra' pur/d tira avanti; dal verbo Tra-
nare, che vale trarre con fatica qualche cosa, e strascinarla.

, ALAL più. Questo termine usato nel modo, che è nella presente Ortava, ci è
familiarissimo, ¢d ha quati lo stesso significato che evvia detto poco sopra, e s'ula
Pua per F altro in occatione di stimolar qualcheduno a spedirsi; ed esprime una
certa impazzienza di colui, che stimola. E' il Lat. ea tandem. Finiscila,. dille
ana volta,

DAG ELA in sanore. Rispondi secondo il nostro desiderio, Quando si vince
una lite, si dice haner la sentenza in sanore..

CUORIE con (a ghirlanda. Significa morir vergine. A coloro che muoiono in
concetto di vergini, quando si portano al sepolcro, costumasi di porre in testa
una ghirlanda di fiori in segno della loro castità. Qui il Poeta scherza, come è
solito farsi, quando si discorre d' una donna impudica, che Gdice Elba giurate
di morir con (a ghirlanda, ed è detto ironicamence, e per intendere, essa vual por=
i tare il vanto, e La corona delle donne impudiche, Ma non per questo il Poeta (che
molto ben si ricorda, che Celidora, per essere stata moglie del Re di Malmantile,
non è più da ghirlanda, intende, che Celidora fosse impudica, ma dice così
per ischerzo, e per seguitare il costume della plebe, la quale, quand'uno nomina
sorella, madre, o moglie, suol dire; puttana di me, e simili. Se si parla d'ammogliati
suol dire becco del diavolo, ee. Tal cohtume moitrd il Poeta ancor fo-
pra \cstan[2]{21}. dove dicendo: 4 saper quante paia fan tre buoi, foggiugne sfybi-
to Se ben dat padre, ec. e vuole intender padre bue, secondo lo scherzo suddetto:
' Non è pero questo stimato offefa, perché avvien sempre detto per ischerzo; ma
4 ricice bene odiolo, e riaferescevole l'eder.u/aco spesso, ed in ogni congiuntura,
y come è usato fra i più vili, che lo fanno per parer fagaci, e concettof.
¢ Sl riftringe nelle spalle. Cioè s'accorda, ed acconsente a quel, che altri dice,
ib
v

aS Sen

o propone. È un' arto solito farsi da quelli, che è rimettono, o aderiscono alla
voloata d' uno, per non poter fare alttuncnti, o convinti dalle ragioni, o indo
ti dalla necessità, quasi dicano: Pazienza; Bisogna frarct. Bocc. Giorn, 2, nov, 8,
' Ada pure nelle spalle rifiretto casi quella dagiar a daee setae mole Mas /ihoorre AnGa,
yar 2. se



Sate
Eefubetiesaivolen nos si faceia essert “7
volta della testa, non dimend dictarho >. re;

0 garbate: O così sta bene Lat, edge, perphtore belle Te
sue ii contento |, che's' ha», he una cofa  succeda secondo che si desid
APREST Oye male, te cone dafser Meglio'¢ farimale'y¢ pre!
si mai col pensiero “dis voler far benew Chi fa o|,,emale pfiaalineare'
cha facenuy adagio, e bene, mainon conchiude, o-termina quel'cheha
moidi fare, non si puddin che facciay e veramente nonfa'y e pend nell'c
dei fare e meglio far male, che non fare. '
DATE (a mano Dar ia mado (Latinoviuagere\ dexreras yO la.
nia, che fr faccia negli spontaiizai') e dice impalmare O:far Limpalmamentos,
STANZA Lie oS BAN ZAMLDL oub
Sogwitoical sxb Lvoe gud Phiche bakes,
(olLanSenegee |y. che sn: last frggiafh patra
mand eiskincdrfecon La ging iddaes,,\

ee o al dueilo non volle la gatta; Per eat sa i
quefamnalivnara Medesy) >) sion Lagwale
ieplaeaens ere ot mqe) asBe Pe ep

neater (grades ust \& Biche trt/ud honor ae
ones', aldan pian, Ces ie perdadayy 00% ~ nel e es aero ae we }
ia

STANZA LIL; oralga on nenaaet
Bit won potends haver Cupide sposoy 08 90 Pereincomintance/m: ets a4 F
hori: Amardai martha tontana >: \\Bacendo com? il'can delParcolano,, o)
Ou Revael,s elapher (can iia Ucanegdlah iG 9% (O'all! infatara now P
Che pur veduto sia da corpo humane: E non pao ines eae

Martinazza haveddo prdiilto, che dovea esser fatta imoriré, eiche per Gupi
do non dovea esser piirfucsspolo,, inttidiosa, che.questo be ne havetioa epodie dd
alteiy: !-haveva incancatoun-udga igacto per impedire: yoche altrinon havefe
\begin{description}
\item[FUGGIVA ratta] Fuggiva velotemente, Ratto viene dal Latino \textit{rapidus}. Il
  proverbio Fiorentino: \textit{Chi va piano, va ratto}, corrispondente al Latino
  \textit{Festina lente}.

GING ADEA, Intend laspada, come s'intende: conunenene wb al
deta dall'impugnarficé tutte cinque le dicasete bene itbastone pure simpugna coeur hur
te cinque de-dita, non si di¢e-cinquadea, perché questo fipud im}
digch jal che\non si — fare delianspads ordinariayy 0fe pur ff
o con difficultà VS RSH A
wuollagstea, a vuolattendete pNomwuol'badarey
Rissmnneiir quel tal:negozio. Hl Berni nell\Orlando,=
« Chey come si suol dir, voglit la gutta, ~~
OVA Aeden B+ uora lacrudela, che eh Medea si
Oza Re de' Colchi:,»versaril fratello Absyreo- opr )
fo, Glauca sua rivale, e' co yet th suo ne per 4

"ihe 'Vee Rte Sceey cr

jeicodeind Mamie) matto; A Gatto ata

goD

Se peeve fats = \&

-

ie
6

|

DVODECTMO,ED VLTIMOGANTARE '5432

ne fuan'o\pibes inet faitem a <, be aL a oe
do)0-da qualche donni at iftra, e wih won; sfiss alist ctloe

TIRA per dado. “Conia aplageresrnoraands più Bettilenel
la milizia, soldati insieme habbiano commefio qualche delitto 'ca-

pitale, farmorice tn di loro'y, e falvar.la:vita a tutti gli altrt, facendo loro tiz
rariila-forte ne s€ però 5.1 orcas dettirdadi, e da-credere, che ace
compagnine tal funzione con, fo! i xe con pianti;.¢ stimo però, che il Poeta
digcndo ztiraper-dade, intenda, toipieay o plange pill di cuore che mai; seguirae
Piangeress pisces gagardamene, es sie pare, she non heaas here aim > 6 sia -
da principio,

“hssan wage. Esser desiderosa d' una tahoe. « Saiwere vago, che vuol dir be
lo, adarne.yec. Sig iglia\ ancora in questo fendi bramefo, ec. Tiraleé — divbes cir
vuol dire: Zi tale genio', ha gusto di betle burle, e feberzi.

\item[HA già fatto il pianto] L'ha già pianto per perduto. Termine assai usato in
  simili congiunture. Pianto è quel lamento, che si fa sopra il morto, detto così dal
  batterti, per dolore il petto. Latino \textit{planctus}, dalla qual voce Latina hanno
  fatta similmente i Franzesi la loro \textit{Plainte}.

\item[ALZAR capanne, ec.] Cioè quei monti di scope, ec. che furono fatti per abbruciar
  Martinazza, come si è detto sopra in \cst{3}. e queste sono \textit{le cose
    di fuoco}, le quali dice, che s'hanno a fare per honor di lei; che per altro, quando
  diciamo: \textit{s'hanno a fare cose di fuoco} intendiamo, \textit{s'hanno a far cose belle,
  maestose, e fuori del consueto}.

\item[FAR come il cane dell'ortolano] Cioè non volere, o non potere haver' una cosa,
  ed impedire, che altri l'habbia, come fa il cane dell'ortolano, che non
  mangia l'erbaggio, e non vuole che altri lo pigli. \textit{Canis in Praesepi}\footnote{Brevissima favola di Esopo: \textit{Canis quidam in praesepio degens, nec ipse hordeum attingebat, nec equum esurientem attingere sinebat}.}.  Proverbio
  usato da Luciano.
\end{description}

“iquid 6 of STANZA LV M
tio, \e Bsiche bebbtrogeuife Cos 'byes: affanni ¥¢ le fatiche 0!) Ob
< WDE extte quello eb' è feglito ie Corre; » Soffente per rant' anni, e lafri ee

~Gbda il teiogoappinte now si farprecifoy

Risrovatosi, Amore; ed egli, e Priche

AtRena si fainsaprer'tattele porte; » Rappattumato fu dai cavalieri 30
se Amanro crofeiar fenvesi wrgran rife, >: Onde foordats deli"ingiurie amicbe, a
atest obie peg gio; poi fwonsr; ma forte Eriuniti più che volentieri:
\& Aeafivmare.di-pefe sr aboccanti, + vad regp sposi fero i baciabassi yoo 9
i obSemea sunosceriehi rece, Contantics' bq Restando \# parte diver foe je (pai
STANZA LV. STANZA shVTR cor 6
“Gir per peniescate ognnn presto addirizza, (i Gluntis cialdeni pots e fare i bile,
Che dal timor gli si arricciane è peli. Ml Duca diede al fin 2 ultimo Addie' -
Ma C alagrillo aitiero, ¢-pien di fiicxa “E Jubiro conagni suo vassallo
ib oGem talus frrifeia fa colps cradels, =: dnnerfa Venano Spiele it-pendio
wi Wa per le stance fende,taglia.e infizza, E Catagrilio ix groppa al-fud <aualld
jlut 444 mon cliappa, se vende' logences cil' o \Preforoon Pficle it Raretrare Dia,)

~ien Rar tde inns e i¢ok fucertibroinranty,.
E il Diavol caccia, e manda vialincato,

o8e\Gupido per opra dij Pakide Aixicrova je per inekzo di-quei Cavalieri'

“3uVO

'¢Aashrei pars); eintefolildor 20
Gb ricomdusse 'ali' Amoroso. a


344
con Psiche, si fanno le feste delio spolalizio di
lo di loa Bache con iain beioenta ~ dy,
lo accompagna Psiche se Regno d',
€ROSCLAR an ria. Rider gagliatdamenre- Vedi
GRAV, traboce: Gravi:più del giusto pelo,
on delle nee ee on se ne feeuc per
¢ seguita, chi recé contanti ( che termine proprio
= intender, chi dava se heeds '
e4ADDIRIZZ 4Ciok va via. Fugge per la più hota tema

STRISCIA; Intendila spada, come intese (apra:C. 2: st, 60. ° ae

CHIAPPA, Coglig s ritrova, perquote s¢aipilce. Vedi sopra C7.

RAGNATELI, Ragni,piccolt vermis o inferti nati... Vedi sopra.
Le stanze piens di ragnateli significa vote dogui.altra fa. Siauimente |
volendo dire il borficchio voto, dite; Plexys facculius est arancaram,

RAP PATTV MATL. Intendiamo rappacificati. Da molti si dice
ge di pace donde: O vincere, o patrare, clo' pareggiare; far pace» ae
gredo venga questo verbo rappatramare, il quale e atlai usato, mala) 4
da pochi fuori della plebe.::

CIALDONI, Specie di pasta confetta, contorta sottile come l'ostie, ed attorta,
e ridotta come un grosso cannello di canna

STANZA LVIILED VETIMA.,
Finito è il nostro scherzo: hor facciam festa,
Perché la Storia mia non va più avanti,
Sicché da fare adesso altro non resta,
Se non ch'io riverisca gli ascoltanti.
Ond'io perciò cavandomi di testa
Mi v'inchino, e ringrazio tutti quanti;
Stretta la foglia sia, larga la via:
Dite la vostra, ch'i' ho detto la mia.

SCHERZO, Qui vale per trattenimento, Latino lusus, Sogliono i nostri
Contadini, quando fanno le loro veglie di ballo, dopo che hanno un pezzo ballato;
introdurre qualche intermedio, rappresentazione, o giocolamento di forze, o
altro, q questo chiamano lo scherzo, che per lo più finisce in burlar qualche
semplice, e dar'occasione di ridere, e questo tale è poi anche detto lo scherzo, e così
l'intendiamo comunemente, ed il nostro Poeta molto bea l'esprime servendosene
nella sua lettera alla Sereniss. Arciduchessa Claudia d'Austria, riportata sopra
nel Proemio, dicendo: Contentandomi io, che la mia Leggenda, come nata da scherzo,
mi faccia scherzo alle genti.

FATE festa, Cioè siate licenziati, Vedi sopra C. 10, st. 42.

Nota, amorevole Lettore, che il Poeta per terminare la presente sua Oera,
ringraziando con questa ultima Ottava gli uditori, si serve della chiusa inventata,
ed usata dalle donnicciuole, quand' hanno raccontata una novella; cioè
Stretta la foglia sia, larga la via;
Dite la Vostra, ch'io ho detto la mia,

E conchiude, che ha contata una Novella, come diede intenzione sul principio
di quest'Opera. Ed io pure me ne servo per incitare altri a dir qualcosa
meglio di quello, che habbia fatt'io, non so s'io mi dica nel dichiarare, o pure
confondere, ed intrigare quello che nella presente Opera ho stimato poco
intelligibile fuori della Città di Firenze, e prego il discreto Lettore a compatir
me, che per ubbidire ho pigliato a far' un volo superiore alle mie forze, ed a
contentarsi di biasimar me solo, e non quei, che mi comando, perché habbia
fatto errore nell'elezione. E fo punto.

FINE DEL XILEDVLTIMOCANTARE.

è; MEE LY Big

I Molto Rev, Sig Gio; Domenico
cia di riconoleeté con ogni di
. Opera fouo il Titolo di Malmant
Zipolt, vi sia cov alecuna, che ¢
~ Cattolica, eda' buoni 'Costumi 4
» Maggio 1686.; =e aoe '

Niccolo Castellam Vic. Cen. Fiorent, dam Ry ia

si
Mluftriss. e Rev. Sig, g
Ho attentamente letto Oe cor Operetta al
le Racquistato di Perlone Zipoli, insieme con le fae note
spiegazioni, e per non avervi trovato cosa, ne
alla Santa Fede Cattolica,ed a' buoni costumi,
mano mi soscrivo. Firenze 20, Settem. 1686,
Gro. Domenico del Bruno en Sac, Ti

Attesa la sopraddetta relazione si stampi > osservati gli ordini
soliti, Data z0.Settemb, 1686. > Z;
Niccold Casteliani Vic. Ge

I Molto Rev. Padre Lettore Dolci Minor Otfrvante Conf i
tore del Sant'Vfizio di Firenze legga attentamente la
presente Opera di, Perlone Zipoli,. intitolata Malmantile
Raquistato, e ritrovandovi cosa repugnante alla Sat
Cattolica, e buoni costumi, riferisca, Dal 9, Vfizio.
Firenze 17, Ottobre 1686.::
Fr, Francesco Agostino Gambaroua Min,
Del S, Vyizso.

Reverendiss. Padre,:
Ho rivista, e ben considerata l'Opera intitolata A
le di Perlone Zipoli, e per non esslervi cosa repu
-aggiunte, stimo possa
D'Ogni Santi li 24. Febbr. 1686, Pe
Fr, Bragio Dolei Ain, Offer. Conf, del S. j

Attenta presata relazione.
Imprimatur;

Fr. Ces. Pallarvicinus Ordimis Min, Convent, Vie, Ge
S. Off. Florentia.

Ruberto Pandolfini Senat. e Aud. di S, A. S.

Stel 3% oy rm i

BT 2Mh ood ww

LocalWords:  havrei Sinigaglia habbia habbiamo crazia crazie donnicciuole
% LocalWords:  Celidora Bertinella soggiugne Conchiude giuoco giuochi Acciò
% LocalWords:  Malmantile bracciuoli Amadigi haver havevano alli pilorci ei
% LocalWords:  huomo dicesi nidio sustanza Franzesi perquote rovella havrà
% LocalWords:  Malatesti fussero sieno Doralice misterio donnicciuola fusse
% LocalWords:  giuocare perditore minchiate maraviglia cerviero Martinazza

\clearpage

{%
  \kern -2em\fontshape{sc}\normalsize \centering
\textls[180]{\Large Di M. Francesco Berni}\\ \kern 2pt
\textls[360]{\normalsize alla corte}\\\kern 7pt
\textls[240]{\Large del duca Alessandro a Pisa} \\ \kern 2em
}

\begin{verse} \large
  \backspace Non mandate Sonetti, ma prugnuoli,
  Cacasangue vi venga a tutti quanti,
  Qualche buon pesce per questi dì Santi,
  E poi capi di latte negli orciuoli.\\

  \backspace Se non altro de' talli di Vivuoli,
  Sappiam, che siate spasimati amanti,
  E per amor vivete in doglia, e 'n pianti,
  E fate versi come Lusignuoli.\\

  \backspace Ma noi del sospirare, e del lamento
  Non ci pasciam, né ne pigliam diletto:
  Perocché l'uno è acqua, e l'altro è vento.\\

  \backspace Poi quando vogliam leggere un Sonetto,
  Il Petrarca, e'l Burchiel n'han più di cento,
  Che ragionan d'amori, e di dispetto.\\

  \makebox[6em]{} Concludendo in effetto
  Che noi farem la vita alla divisa,
  \makebox[1em]{} Se noi stiamo a Firenze, e voi a Pisa.
\end{verse}

\clearpage
% LocalWords:  Gambastorta Calagrillo ianadattico primiera capresto hebbe
% LocalWords:  stiacciata forame veddero monticelletti conchiusione romore
% LocalWords:  maraviglioso maravigliarsi addiettivo Falterona legnaiuoli
% LocalWords:  cerboneca cirimonia salvatico cetriuolo oggimai Havrebbono
% LocalWords:  havuto nimico maravigliato fuor mascalcia haveva saltero
% LocalWords:  Amostante Latoni Analfabeto Piaccianteo giuocando verzicola
% LocalWords:  seguenza facultà Vocabolista aliosso gastigo havuta
