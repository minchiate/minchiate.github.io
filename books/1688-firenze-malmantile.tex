    
    
   
 
   
   
  
 
 

oe

"MALMANTILE —

RA Gu@unvodk SoTHAT .-O-

POEMA
DI PERLONE ZIPOLI

CON LE*NOTE DI PVCCIO LAMONI.
DEDICATO

ALLA GLORIOSA MEMORIA
’ Del Serenifs. ¢ Rewerendi/s, Sig. Principe Card.

LEOPOLDO

DE MEDICI

RISEGNATO ALLA PROTEZIONE
DEL
Serenifs, e Reeverendifs. Sig. Principe Card.

_ FRANC. MARIA >

_ NIPOTEDIS.AR.
io): CREE

mee WOR FR EN ZE. i
eee S. alla Condotta. 1688. Con lic. de’ Super.

  
  
  
 

 

: E PRIVILEGIO
i Ad ‘ens di Niccold nee
 
Ore...
13041 y avotsa ia
MAL GIOOVG IG AIC $8 AOD 3 s
OTADI-ad q
{IM A201IROID ALIA

att ge Tyhevawra hs ond We

 
 

 
 
 

 
 
    
 
  
  

 
  
   

‘4

 

.
e SS aN

AL SERENISS,, E REV, SIG, IL SIG, PRINCIPE CARD;

FRANCESCO MARIA
DE 'MEDICI.

 

SERENISS.'E REV. SIG,

  
  

ile

i, WSerenifs:c Reverendils. Principe Cardina-
}i le Leopoldo de’Medici Zio di VsA R.Prin-
) cipe diquelle rare,-ed ammirabili qualita ,
BY. che hanno fatto ftupire tutto il/Mondo ,
@ finoda i pid tenerianni dell’ A. V.R. co-
é , nobbe , she in lei dovea continuare quel~
(ee ox lo {plendore , che hanno accrelciuto alla
: 5a fua Serenifs, Cafa le ftimabili doti di
V.A.R ; E per quefto,ficcome giudicd, che IA. V.R. gli do-
vefle fuccederenelle virth , ¢ a i
ella

“
o$ Ye: »

la dignita , cost volle , che »

   
   
 
  
 

ella faffe anche erede della fua fingolar Libreria. In quefta,
havea I’ A, S: Rey.deftinato , che dovetle ottenere il luogo Ja
ptefente Opera di Perlone Zipoli , a cui S, A. R, m’onord
comandarmi,. ch’ io facefli Mesife note , grazia compartitami
(fiami lecito il dirlo ) forfe con qualche {capito del pruden-
nflimo giudizio di S, A. R. ; Ed havendo io ubbidito nellas
miglior forma , che havevo faputo, gia fi penfava alla ftam-
pa; quando i Fati invidiofi tentarono di privarla di cost pre-
giato onore :'¢ {arébbe loro tiufcito , fe la fomma, prudenza di
quel gloriofiffimo, Principe non havefl a i medefimi impedi-
toil corfo , con preparatle il rimedio nel rifugio alla prote~
zione di V. A. R.

Se ne vien pero il poveto Malmantile a’ piedi di V. A. R.
umilmente fupplicando la {ua benignita a volerfi degnare di
riceverlo nella fua grazia, ¢ , come erede obbligato ; rive-
rentemente convenendola al Tribunale della fua generofitd ,
perché gli faccia godere la giuttizia, concedendogli il luogo
ftabilitogli , accid egli poffa dirfi veramente rifatro dalle
rovine cagionategli day tante fue difgrazie, ¢ da tanti fuoi
finiftri avvenimenti : Ed io piglio I’ ardire d’ ee
quefte preci , che egli poo: * V. A. R., come quello , che
conoico d’ haverlo con la mia penna coftituito in grado d’ha-
ver maggiormiente bifogno dell’ autorevol patrocinio di V; A.
Rev. alla quale.intanto umiliffimamente inchinato bacio offe-
quiofiffimamente la Sacra Porpora .

Di V. A. Rev.

£3

Vmiliffimo Servidore
Puccio Lamoni

re

  
  
 
   
   
  
 
     
    
    
      
  
  
  
 

“Al Serenify. Rev, Sig. il Sig. ‘Principe Cardinale

ZOPOLDO DE MEDICI
PADRONE CLEMENTISSIMO.

SERENISS.EREVERENDISS. SIG.

S71 ENTRE ftavo meditando d’ ubbidire a i cenni ftimatiffi-
h zi di V. A. Rev. col far le Note alla prefente Leggenda di
‘etlone Zipoli , mi cadde fotto l’occhio un fonetto del
4 Burchiello , nel quale havendo offervato , dove dice:
pies Non funt , non funt pifces pro Lombardif,
mi faltd il ticchio d’ effer’ il Lupo nella favola , cioé che quefto verfo m’
—avvertifse, che la:faccenda da V. A. Rey. impoftami non  fufle car-
ne da’ miei denti, ond’ io hayevo gia quafi peniato di far conto , che
paffaffe P Imperadore : Ma confiderando poi , che farebbe ftato errore in!
‘ tica, ¢ da pigliar con le molle , il far’ orecchi¢'di mercante a i ri-
5 Lact eemeanieenadi V.A. R. ho rifoluto di non metterla pid in
mufica , 0 in ful liuto, ne mandarla d’ oggi in domani, dando erba
_ traftulla., aescen il can per |’ aia , ma(venendo a ditittura a i ferri)
mon tener pik quefto cocomero in corpo , ¢ cosi cavarne cappa , 0: man-
ae oa i per eleguire gli ordini di chi pud comandare a tercbetts, che
_ perche io refti perfuafo d’ haver forze fuificienti a portar si grave foma ;
E quantunque io fappia , che havrei fatto molto meglio a lafciar Ja lin-
gua al beccaio , perché cosi havrei sfuggito il farmi dar la quadra , o Ja
madre d’ Orlando, e fonar dietro le padelle da coloro, che fi pigliano
4 impacci del Roffo, e ficcando il nafo per tutto , fanno poi le Scalee
di S. Ambrogio , come quelli , che hayendo mangiato noci , oa
By out 0

 
 

ie at gc nee 5 eng

 
   
   
  
 
 
 
 
 
 
  
—s. em -_ — =< ~~ *

   

bono al fale 5 ‘fenza éoiifidéraré)). the ognun pud: fare della fua pata
gnocchi , ¢ che [, come diffe colui , che s’ mee | ognuno ha i fuoi ca-
Pricci’; tuttavia ho voluto ( legando I afino dov’ é piaciuto al padrone )
dare xconofcere the V. A. K. non fari, come il Podeita di Sinigaglia;
Se poi ad alcune di quefti tali rincrefce , mettafi a-federe , ¢ , fe non gli
piace , la {pti} o mi tincari il fitto ; ¢ fe dira , che in fare alla prefente
Opera le Note comandatemi , io non. habbia prefo il panno pel verfo ,
ma pitt tofto fatti de’ marroni , ¢ pigliato de’ granchi a fecco , lo lafcerd
/ ragliare ; perché fon ficuro , che non mi fara baciare il chiayiftello, ne
Pigliare il puleggio dalla cafa mia; ne mi pud accufare di delitto da far-
mi mettere in Domo Petri fra i due Apoftoli,o da farmi meritare d’ effer’
ammazzato con una lancia da pazzo; E fe I’ indifcretezza di quelti tali
mi condannera per gli errori , che troveranno nelle Note fatte da me, la
mia ignoranza m’ affolvera .. Non ne ho faputa pid. : ho foddisfatto al
debito d’ ubbidire , € mi quieto col detto di Donatello : Piglia un le-
gno,e fann’ untu. Mi fara forfe detto: Tu porti frafconi a Vallom-
brofa , cavoli a Legnaia , ed acqua in mare , ¢ vai contrappelo alla buo-
na ftrada a comparire avanti a un Principe cosi crudito con quelti tuoi
{criti ; ed io a lettere d’ appigionafi,e di‘fcatola, fenza faltare in fulla
bica , o enttar nel gabbione , rifpondo a-coftoro , i quali fanno tanto il
Cecco fuda , che portano ben loro le mofche in Puglia , e i Coccodrilli
in Egitto ,¢ dandomi il mio refto , hanno trovato il modo d’ intifichi-
re , fenza petd ditmi cofa , che io non fappia ; perché conofco-ancor io il
pane da faffi, la Treggea dalla gragnuola, ¢ le comacchie dalle cicale ;'¢
fapendo:quanto il mio cavallo pud corere , fatei venuto: di male gam-
be , ¢ quafi come Ja all’ incanto , a metter queito cembolo itr ‘co-
lombaia ; fe non mi fufle noto, che.colui', che ¢ avvezzo'a mangiar
fempre -ftame , defidera talora came di Stomo , ¢ non full certo 5 che
la fomma prudenzandi V. A. R, ( conoféendo ', cheil pruno« non produ-
ce limoni, ¢ che dalla botte non efce mai, fe non di quello’; che-v’ &
dentro., che parimente @ impoffibile , che il Gufo facia’ il verfo deb
Rufignuolo) non é per isdegnare diticevere le baie di Perlone Zipoli con
P abito da villa meffo loro in doffo dalla mia zucca, poco atta-a rapptes
fentar I imprefa degli Accademici'Inttonanti , perché le manca cil Aelio-
re Latent é) 2 ota SSlorm 0718) i . epee Ary

Supplico perd I’ impareggiabile umanita di V. A. R,<a volerireftar
feieien gitar edeae oa tali’, che io ho legato il Cavalloabno-
na caviglia , com fare degne quefte mie ‘infipidezze d’ un benigno uo
sguardo ; non perché lo meritinio per fe-fteile ,*ma perehe bensi convie-

ne

5

 

 
   
    
  
  
   
   
  
 
   
   
  
 
    
       
  

alla continuazione di qucl generofo aggradimento, col quale ficom-
ue ricevere in vita’ dell!’ Autore il medefimo Malmantile . | Il qua-
fe con le mie ciarle haveri fortuna di compatire in pubblico, goden-
_ do si pregiato favore fi potri dire,nato veltito , ed io cafcherd in pit
come i gatti, ¢ mi piovera il cacio in fu i maccheroni: E cosi con ha-
-_ ver immitato il cane di Butrione,non hayrd timore di coloro,che paifano
la maggiore ; perché fapendo efli, che ’ Aquile non fann» guerra co”
anocchi , sdegneranno abba(farfi tanto con la loro critica , mettendo le
mani in si vil pafta; e quegli Aniftarchi,i quali non contano,e non han-
no voce in capitolo , per haver poco di quel. che il bue ha troppo , ¢
che fono come monete ftronzate , o come i cavalli di regno; non {aranno
caufa , che io alzi i mazzi; ne mi faranno venire la muti, o il mofche-
tino col loro gracchiare ; perché oltre all’ e“fere {eritto pe’ boccali , che il

ieco non pud giudicare de’ coloti , fifa ancora , che raglio'd’ afino
fon entto mai in Cielo, che pero’ conolcend’ io , che effi fon ‘per. fare’ 5
Come colui , che tola il porco , non gli ftimo il cavolo a merenda,e gli
ho dove fi da al boffolo da spezzie , e dove fi {o.fiino Ie noci; Sicch? fi
fono andar’ a riporre a lor pofta , ¢ fare un mazzo de’ loro falci _ E fe

dice il proverbio , che la carne di Lodola va a Piacenza a ognuno ;
fo non mi curo , che me ne fia data , anzi per non mangiame , fon con-
tento far fempre dinero , purché non mi dieno di bianco quefti Corret-
tori delle ftampe"y-che\tiranneggiando Je \cttere ;)péfche fi ftimano il

Secento , cercano i fichi- yetta , ¢ 'l nodo inl giunco . Ma fe poi mi
Vorranno pure ftrazziare , io gli afficuro ,.che ¢’ non hanno a mangiare il

   
   
 
 
 
  
  
  
   
   
  

cavolo co tunque io non fia tanto addietro’con |’ ulanza ,
credere a haver cattivi vicini , o'fia di natura d°
@ mia polta.. Mi mandino, pure: all’ Vccellatoio

anche dietro lima lima, non faranno
me Chele Mafi ,perché me la farebbono di

‘troppo ; fe bene mi perfuado), che ancor’ effi
ni per retti ; ¢ che fuflle per fucceder Joto il mangiar no-
icol mallo ,

clircme e i Pifferi di montagna , poiché , fe effi fi {ti

“ 1 Pitieri, di montagna , poiche , fe effi fi {timano
iccic Gorgona 5 ¢ ‘io non-lon_ ai. Valdis ; perché fono ulcito
di d 3 ed ho raiciutto il bellico , € per queito fo ancor’ io quante
paia ¢ buoi ; onde a dirmi cattivo cattivo, la fara fra Baiante,e

Ievar le mofche d’ intorno al nafo > ne mi morfe mai cane , che io non
voleffi del {uo pelo, maffimamente quando m'é faltato il capriccio di

 

b:. Ferrante , io fon d’ una natura , che non poffo ber gtoflo,e mi fo

voler la gatta ,¢ badare a we perla pentola ; ¢s’ io me

a

 

  
  
 
 

 

Ja fon mai legate al dito , .o !' ho prefa co’ dentin’ ho voluto vedere
quanto la canna ; perché. non'mi fuol morire la lingua in bocca , ed ho
tagliato lo: {cilingnagnolo , ne m’ € piaciuto mai pottar barbazzale, ¢ fo
laiciar la fquola d@’i Arpocrate , quando é tempo , ed in. particolare com
quei tali che, fon pitt tondi dell’ O di Giotto, e che ftimandouna ftefla
cofa il chiacchierase , che il condennare , non fanno portare altre ragioni,
che quel maladetto nom fi pud.

Ma perché non paia ch’ to falsando. di palo: in ftafca voglia dar panza-
nea V_A.R. ¢ che quefta mia lettera fia il vicolo di, mona Sandra ,con-
chiudo, tomando @ bomba, che ftimerdd'haver toccato ik Ciel col. dito,
€ tirato diciotto con tre dadi, {e potrd conofcere, che I A, VaR. refti fer
vita di credere , che in quefta parte io I’ habbia: ubbidita giufta: mia. pof-
fa; come riverentemente la: fupplico a degnarfi di far’ apparire con I’ ono~
re di nuoyi {uoi comandamenti).. Mentre facendo Ia felta dirS,Gimi+
gnano ,,umiliffimamente inchingto bacio;offequiofiffimamente a, Vs AsRa
la Sacra Porpora » : ; )

oy Oxs r es 4

 

 

 

 
   
  
  
    
  
  
   
     
 
 
   
   
 
 
   
   
    
   
  
 
  
  

AL CVRIOSOSE DISCRETO LETTORE
Pvcclo LAMONT,

A Fi Opera ai Perlone Zipoli fi manda alle fampe, per ‘eahed
_, alla curiofita-ds molti , che bramofi gi Piglarfi il palfatempo di legger-
AH lane bsnno fatta inflanza . E perche in\alcuni detti, e proverb; ufa-
ti in Firenze, de’ quali fi ferve il noftro Autore , poffa effer' intefa anche da
color , che lontani dalla nopira Tofcana , non ee Ja vera cognizione del va-
lore ,e Jenfo di aN 5 Ui bo aggiunto Aleune Note , con le quali Se non bo appieno
5 -foddisfatto » mi bafta, che havro forfe data cal fione col mio cicalare , che ven-
ga ad altr: vogha di meglio difcorrere . Tu intanto ricovdati, che questa é
una novella ; ¢ cosi ti accomoderai a compatire , fe alle volte mi fon fatto leci-
to di dare qialche fpiegazione favolofa . So, che bavrai la bonta di sbandir la
eenlura., ¢ ti tornera commodo , perche facendo altrimenti havrefi troppo da
fare i poche , 30 fore niuna effendo di quelle cofe , ‘ee bo feritto, che non la me-
ritino con un nuovo foglio ,¢ per quefto non te ne prego Ti prego bene , fe feb
Fiorentino, a legger’ il I ‘efto, ¢ non le Note , perché quefle non fon fatte per te,
che , meglio di quel ch'io babbia foritto , intendi la forza dei detts , che be
pretefo dichiarare «
_ Dowrei ‘notare gli Autori, a i quali fon ricorfo'per tirare a fine la prefente

Satica , 5 ma perche glitbo nominati in tutti quei Inoghi , dove’? convenuto valer-
mi della loro arn wes lafcio di farlo ; non woglio gia tralaftiare di confé/-
fae . ’ ebbliga y che quefe mie Note ,ed io habbiamo all’ Becell. e dottifimo Sig. >
Gio. Cofina Villifrasiedi y ed agli Bruditifs SS, Anton Caflo, ¢ Sig. Prance/co eS
Maria Bellini, i-quali-s'banno onorato di pie erudite notizie ; ed in ultima
atteftar la fortuna che hanno bavuto quefti miei feritti di ‘pater Sotto T oechia
__ dell’ Eee. Sig. Abate Anton “Maria Salvini i? quale non folamente sé contentato sy
_ d’emendar molti mick errorijma a ingagliar ive ‘ancora le mie debolezze con mon —
poche fue belliffime erudizioni.s 3.4 fegna,, che ba fatto nafcere in me una speran- Sa
za, press wae ar cfr i? ricevuta volentieri quefta mig Qpera, ¢ d’ baver guadagnato *
. al Mondo letterato , per baver dato oceafione a quefto dot

ei @ sfercitare da enced Ehe ser iffima pinna,itratei della quale, come

“bo dubbio che nobilmente risplenderanno dentro all’ oe della mia;cosi

faranno da tutti beniffimo ravvifati : Ne confefo perd al me- _
ita , © ne porto.al pubblico quefia attefazione ,perche fi i Sittiay :
5 the fara riconofciuto per non mio, non é latrocinio , ma regalo
- fattomi da queflo,e da altri buomini dotti per loro generofita , ¢ per follevar
 Perlone dal difere a ee faa Operai _ “eet
| Lettore , vivi felice . 2

   
     
  
 

PROEMIO,

Orenzo Lippi (che in Anagramma nella prefente Opera fi chiama Perloné

Zipoli ) é ftato ne i tempi noftri Pittore non poco celebre , come teftifica-

no molte , ¢ molte fue fatiche . Cid lo fece meritare d’ effer chiamato dal-
Ja Serenifs. Arciduchefla Claudia d’ Auftria per valerfi dell’ opera {ua a Infpruk ,
dove dette principio a quefta da lui chiamata Leggenda delle due Regine di Mal-
mantile , ¢ la dedicd alla medefima Serenifs. Arciduchefla Claudia . Haveva perd
P Autore concepita nell’ animo {uo queft’ Opera qualche anno prima , ¢ nel tem-
po , che effendo in Villa de’ SS, Parigi a S. Romolo nell’ andar per quelle campa-
gne a diporto , vedde le muraglie di Malmantile ; ed haveva difcorfo quefto
fuo penfiero col Sig. Filippo Baldinucci , dal quale poi nel teffimento del Poemas
hebbe , come da perfona erudita ( che tale lo dichiara Ja fua bell’ Opera mandata
da effo alla luce intitolata Notizie de i Profefiori del difegno ) non piccolo aiuto
in propofito delja lingua , ¢ d’ altro, ¢ particolarmente nei defcriverc il Configlio
de 1 Diavoli nel Canto {efto .

Tal compofizione fece egli a folo fiae di mettere in rima alcune novelle, les
quali dalle donniccivole fono per divertimento raccontate a i bambini,e di sfoga-
re Ja fua bizzarra fantafia , inferendovi una gran quantita di noftri proverbi , ed
una mano di detti,e Fiorentinifmi pit ufati ne i discorfi famigliari, sforzandofi di
parlare , fe non al tutto Bocaccevole , almeno in quella maniera , che fi coftuma
oggi in Firenze dalle perfone Civili , ed ha sfuggito per quanto ha potuto quelle
parole rancide , alle quali vanno inconcro tal’ uni , che per fpacciarG huomini
Jetterati , non {anno fare un difcorfo , fe non vi mettono , guari , chente, ¢ fimili
parole , che per effere ftate ufate dal Boccaccio , effi credono , che dieno I’ intero
condimento alli loro infipidi ragionamenti , ¢ ftimano , che quello fia il vero par-
lar Fiorentino , che non é intefo , fe non da i Jor pari, € non s* accorgono , che
in tal guifa parlando , fi rendono {cherzo di chiunque gli fente, come bene atteita
guefla verita il La(ca in quel {uo Sonetto fopra I’ Opere del Berni , dicendo :

Non offende gli orecchi della gente
Con le lafcivie del parlar T ofcano ,
Vaquanco , guari , mas fempre , ¢ fovente
Ed Antonio Abbati did oe ‘
Peggio non ho , che quel fentir parlare
Con tanti quinci ,e quindi ,e@ , ec,
Anzi in quefa parte I unica intenzione del noftro Poeta é fata di far conofceres
la facilita , ¢ pienezza del parlar noftro , ¢ Cagliendo dela lingua materna il pis
bel fiore, moftrare , che ancora ad uno , che non ha ( come’appunto , cra sale
altra cleus » © poca pill di queila , che gli detcd la natura , non ¢ impothbile
il parlar bene . Quefto , ed altri fini dell’ Aucore s’ argumentano dalla feguentes
Dedicatoria , che egli fleflo {crifle alla SerenifS. Arciduchetia Ciaudia , 1a quaies
Jewera io pongo qui per confonder coloro , che pur vorrebbuno fargii dire quel
che mai il noftro Poeta hebbe in penficro. A
x» Ati figliolo di Crefo Re di Libia ( fe ¢ vero, che io non ne fo pit la, ¢ nae:
Sep

 

 

*
 

—

3 do, come io I’ ho compra ) vedendo il padre iin pericolo , iffo fatto cavd fugra

3» il limbello , ¢ difie le fue fillabe , come un Tullio ; Tucto il royelcio dovrebb=

x fare il pefce paftinaca fenza capo, ¢ fenza coda della mia Leggenda a mal tem-
49 po, ch’ io mandoa V. A. S, perché vedendo ella quel dolce intingolo di quel
“a » fantoccio di fuo padre in procinto d’ efier mandato all’ Vccellatoio, ¢ quafi ri-
39 dotto alla porta co’ faffi , ¢ che gli fien fuonate dietro Je padelle , anzi fra il
»» tocca , ¢ non tocca di {cior Pallino , potrebbe a fua pofta far’ un mizzo de'fuot
32 falci, ¢ farfi ricucire la bocca per non haver pili occafione di formar yerbo .
>» Ma perché fi compiace V. A. S. di voleroc una fecchiatina , benche quefta mia
>» Leggenda non futle degna di fiutare eziam i luoghi privati,verra di gala col (yo
3 ticadiofo cicaleccio , che fi ftrafcica dictro una gerla di farfalioni , a farne una
»» flampita anche ne 1 Palazzi reali , perché ella ¢ una profoutuolina da darle del
92 Voi; Ond’ io conolcendo nella temerita di efla l'ubbidienza dovuta de iure a i
y» tiveriti toi cenni , gli ¢ giuoco forza , voglia il mondo, o no, che ella fi met~
x ta git a bottega a sfogare la fifima de’ fuoi fantaftichi ghiribizzi , contentan-
32 domi io , che cila , come nata da {cherzo , mi faccia (cherzo alle genu. Com-
» patifea dunque |’ A, V. S. quefta fconciatura partorita nel tempo , che io do
3 feftaa i pennelli, mentr’ ella non apprezzando un’ ette gli applauGi valgari, ri- ;
9» cevera per grazia fterminata , ¢ per arcisbardellatifimo favore , fe queite bai¢
y» riufciranao di qualche valezzo nel cofpetto di V, A, S. alla quale profonda-
3» mente inchinandomi , con ogni debita rivereaza bacio la Velte.

Da quefta lettera adunque fi viene in nn piccola cognizione de i fentimenti

deli’ Autore nel comporre la prefente Opera; La quale fy da eflo preffo che»
terminata in [nfpruch, ¢ dedicata come ho detto alla Sereni(s, Arciduchefla Clau.
dia ; Ma effendo S. A, S, in quei medefimi tempi pafiaca all’ aitra yita , conven-
ne all’ Autore tornare alla Patria , dove fu quefta fua Novella veduta da diverfi uy
amici fuoi , fra i quali dal Sig. Romolo Bertini Servidore del Serenifs Principes
espaad Leopoldo de’ Medici , ¢ molto accetto per I’ ottime fye qualixa , vircii ,
e orig

 

   
  
 
  
 
 
 
 
    
 
 
   

3 ¢ daeffo hebbe S, A, R. la prima notizia della prefence Opera , ¢ fino
da allora moftré I’ A. $, R. non piccola inclinazione , che fi pubblicafle; ¢, fe»
tralafcid di comandarne la ftampa , fu., perché i¢ati dal medefimo Bertini , ches
} Autore penfava d’ acerefcerla . ;

~ . Fu veduta ancora dal Sig. Francefco Rovai , e dal Sig. Antonio Malatefti ;
ambi Poeti nel lor genere Eccellentitfimi, da! Sig. Salvador Rofa non men celebre
nella Poefia , che nella pittura , ¢ dal quale il Lippi hebbe notizia Dello Cunto
de li Cunti di Gianalefio Abbattutis , di dove !’ Aurore cay poi alcune aovelle ,
_ che fi trovano in quef? Opera: La quale in fomma fu veduta da mole’ altri ery-
diti ingegni; ¢ fu il Lippi da effi configliato, ¢ poco meno, che forzato a metter-
Ja alla fampa , con perfuaderlo , che meritaya la pubblicazione : ma ricusd egli
fempre di far tal pede » conofcendo molto bene , che colui , che fampa I’ Opere
fue , s' efpone ad un certiffimo per per una incerta gloria, ¢ matfime nel
_ prefente fecolo,che vi é maggiore ndanza di (propofitati,e mordaci Satirici,
quali con invidiofo livore lacerano le fatiche altrui, che di Cenfori difcreti, i
quali con dosti avvertimenti n’ emendino gli errori . 1 Seats
Dalic grands inftanze fattegii dagli amici fuddetti , che egli ampaffe queftas
: R ery

 

  
   
 

  
B
;

 

fua Novella , infofpettito il Lippi , che il libro di detta fua compofizione non eli
fuffe levato , ¢ contro a fua vogiia ftampato , andava molto circofpetto , non lo
lafciando in !uogo, dove fufle (orcopotto a tal cao ; Ma effendo una volta andato
in villa de’ SS. Sufini fuoi cogaati , € di quivi alla villa del Sig. Don Antonio de’
Medici ; dove havendo portato il detto libro’ per paflare , leggendolo , la veglia,
Ja notte , mentre egli durmiva , il Sig. Piovano Gualfreducci, ed il Sig. Tommato
Fioretti con I affiftenza del medefimo Sig. D. Antonio (ciolfero il detto libro , e»
fra tutte due lo copiarono ¢¢ la mattina lo rilegarono , € 10 raccomodarono ina
maniera , che egli non s’ accorfe del virtuofo furto. Quefta copia capité poi ins
mano a Paolo Minucci, il quale facendo al Lippi la folita inftanza di metterlo al-
la flampa , ed egli ricufando , gli diffe i] Minucci,, che I’ haurebbe egli fatto ftam-
pare ; ¢ replicando i! Lippi, che fe ne contentava , fe vi era modo, il Minucci
col moftrargli la detta copia {edperfe il furto , ¢ fece conofcere la potfibilita , che
havea di farlo fampare , S’ alteré non poco i! Lippi veduto quefto , ma comes
huommo virtuofo , ed onorato volle, che la vendetta di tal difgufto fufle il colticui-
re il Minucci , ed ogni altro in grado di non fi curar pid di ftampar queil'Opera;
 quefto fu.con aggiugner’ ad effa alcuni epifodj , ed altro, in maniera , che ins
breve tempo 1a ridufle da fette piccoli canti , che ell’ era , alli dodici, che é lan
pPrefente ; ¢ perché non gli avvenifle di quefta, come gli era accaduto della primay
teneva. |’ originale di efla in modo riferrato , ¢ ciftretto , che non laiciava veder-
lo ne meno all’ aria , e poco altro poteva haverfene , che fentirne recitar da lui
qualche Ortava alla {pezzata , ed il Minucci pili d’ ogni altro haveva quefto fa-
vore da Ini, perché col fargli fentire I’ augumenco, che dava a queft Opera , fti.
mava di fare {cemare nel Minucci la voionta di Aamparla , ¢ confeguir I intento,
che s' era prefifio , ma ne fegui tutto il contrario , perché havendo i] Minucci
{parfo fra sli amici,che i! Lippi riduceva la fua Opera in ftato ragguardevole,per-
venpe quefia notizia all’ orecchie del Screnifs. Sig. Principe Card. Carlo de’ Me-
dici Decano del Sa.. Collegio , ¢ S, A. R. curiofa di veder queft' Opera comandd
al Minucci , che operaffe d’ appagare tai fua curiofita. 11 Minucci manifeftati al
Lippr i fentiments dell’ A. S. R. efortd a non contraddire di ricever ! onore_ y
che S.A. R, guftava di fargli ; ed egli conofeendo , che mal poteva negare d’ ub.
bidire a tanto Principe , per il quale [ come fratello della Serenifs, Arciduchefla.
Ciaudia } riteneva congiunto al debico di fuddito un genio non ordinario di fer-
virlo , e perfuafo pure una volta; che il pubblicar detta Opera non gi ~: ap-
portar fe non lode , condefcefe a laftiarne pigliar copia per S. A. R. la quale fi
piacque di dar dimoftrazione del {uo benigno aggradi con acti non pic-
coli della fua folita generofita , ¢ verfo il Lippi, ¢ verfo il Minucci , che ne te
la copia, perche cosi volle il Lippi 5 o per {paventar il Minucci con la gran mac-
china y che appariva , ¢ cosi levarlo dal penficro di pigliarfi quefta fatica , ed ad-
dormentare intanto nel Sig. Principe Card. la volonta d’ haverlo (: come diffe il
medefimo Lippi ) 0 pure, perché quella copia non capitaffe ia mano ad’ altri, che
del médefimo Minucci , del quail Adare » ¢ per fua bonta , e pereiié havevas
anche veduto,che di quella copia , che teneva detto Minucei della prima‘Opera,
non s’ ¢ra mai faputo cofa aleuna , perché effo Minucci I" haveva fempre oecula~
ta, ¢negata a ognuno d’ haveria, Ma quel’ ultima copia fendo in Poe del
letto

 

  

 

 
ye a

  

L detto Serenifs, Sig. Card. Decano, accrebbe nei SS. fuoi Cortigiani la curioficd d

7 haverla , € cos} per.diverfe vie ne trafiero una gopia, Da gocfta poi fe ne fono

gi {parle infinite ; ma perche It Autore opravvifle qualche poco di tempo, e fempre

“4 accrebbe , 0 modcro qualcofa , ed oltrea quefto ,-perche la poca avvertenza di

i coloro , che hanno copiato , ha caufato, che fi trovino molte copie , ¢ difettofe ,

© guafte , il Minucci ripatandofi in wn certo mado cagione di queflo difordine ri-

foluette per rimediarvi , di fupplicare il Serenifs, Principe Leopoldo (allora non,

Cardinale , a] quale dal!’ Autore fteflo fu queft’ Opera dedicata , dopo la morte»

della Serenifj. Arciduchefla Claudia ) di permettergli il mandare Ja detta Opera.

alla fampa , per rinnovare la memoria de] gia detunto Lippi, ¢ 5S. A. glicio con-

cederte , con obbligo pero , che gli facefle alcune Note, ed efplicazioni; E cosi

contento |’ chivertates che defiderava tal pubblicazione , € diede al Minucci il

Balligo d’ eficre ftato caufa del fadderto difordine » €dal Lippi la foddistaziones

wutagli dal Minucci per Ja violenza fattagli, con obbligare il medefimo Minuc-

ia fottoporre ancor’ egli j uoi {critti a quei danni , che dalle ftampe ne rifulta-

no ; Senienza veramente giufla , come appoggiata al fondamento della pena del
‘Taglione , ma troppo feyera nell’ arbitrio per Ja gran difparita, che ¢ fra la vaga ;

Opera del Lippi »¢l' infipide chiacchiere del Minucci » fopr’alle quali , ¢ non,
fopra gli {criti del Lippi fi fermerantio , ¢ poferanno tutti gli Ariftarchi ; coms
tutto quefto non ha il Minucci voluto intentare appello., anzi, fendog accinto
fubito a dare efecuzione alla {entenza , ha aggiunto all’ Opera Je Note comanda-
te , con le quali ha egli pretefo d’ operare , che fuori di Firenze , € della noftra_,
Tofcana , ¢ per tutta Iralia poflano ctler meglio intefe:molte parole, demi, frafi >
€ proverbj , che fi trovano neil’ Opera , forle non intefi del tutto altrove , che in
Firenze ; © prega i) Lettore a sompatire, {e non fia da effo foddisfarto appicno ,¢

oe » che air é flata pet del Minucci il portare etimoiogia delle para. 4
frafl , ¢ proverb), ma d’¢fplicargli in mdniera, che poflano etier’ intefi anche
fuori di Firenze sed babbia il medefimo Lectore la diferetezza a leper y che ;

molti Fiorentinifmmi (ono in ufo , nati dal purd cafo, fenza un minimo fondamen-

fo ,0 ragione , perche fi dicano ; € che, ;
Non omnium} qua a maioribus’ noftris’ feripta , aut.

~ di&a-tunt’, ‘ratio teddipoteft ,

        
=
e}

 

MALMANTILE

DISFATTO

ENIGMA

DEL SIG. ANTONIO MALATESTI.

V’ é P Erruria indomita , ¢ infeconda ,

Gia fui per molti figli € ricco , ¢ bello ,
Or c’ una fafcia a pena mi citconda ,
Povero, brutto, e vil non fon pitt quello .

M’ hanno gli amici pid che ‘l vento ,¢ I’ onde
’ Levate I offa , ¢ toltom: il cappello ,

E fino il nome par che corrif{ponda ;

Vna mala tovaglia , 0 un mal mantello .

Cos} tidotto trovomi a mal porto, is
Col corpo voto , ¢ fenz’ un membro intero 5
E pur con tuttocio non, mi {conforto ;

Anzi ora godo ; ¢ farmi eterno {pero ;
Mentre in Flora un’ Augel per {uo diporto :
Cantando in burla, mi tifa da vero... ,

 

 

 

 
 

SBQUCHEHS

PRIMO Dae a

GP) RIAMO Cantare. Eoco che il noftro Pocta mantiene I’ intenzione
data di pubblicare una Leggenda,e non un Pocma,mentre mette
fopra ogni Canto l infcrizione , che fi vede in diverfe leggen-
des dove in vece di dire Canto 1.) ¢Canto 2, ec. come ufano nei
Z. Poemi Italiani, egli dice Primo Cantare , ¢ cosi feguita fino all’
¢) ultimu, volendo per la fua modeftia efler chiamato Compofito-
ya wedi Leggende, non Autore di Poemi , ed in uno fteflo tempo
con bell arte Bese dalle rena chi lo ae di non aver’ ors le»
_ gegole del comporre i Pocmi , fapen che a guefte aon fono fotcopolti i Com-
—— pofitori di ——
si Rese ses
3a ARGOMENTO.
ar Marte. oa perché il Adondo ¢ im pace
3

SES ae

  

Bua

Corre ,e me letto fa levar la fuora , 4

Ein finto afperto ,¢ con parlar mendace

Mandala a fuegliar 0 ive in Celidoray

Fala moftra de’ {uci Baldone andace

Andi all imbarco non frappon dimora 5

E per via narra con che modo indegno
a occupate avea il fuo Regno.

Blessrseus SEREISASAS

ti a totti li Canti di queft? Opera fono di Amoftante Latoni , cio’
alatefti', fatti di comandamento del Serenils. Principe Cardin, Leo-

STANZA &

bo feceop ‘L batticn! di maglia, . Per chiarir Bertinella, ¢ le canaglias
ns rmerritra arntfe, Che fit feco al delitto in crimen lafa
Del far’ a Celidora fua cugina ,

Per canfarla del Regno , una pedina,

in quefta “faa introdazione > che egli vuol de(criver da Guer-

n aiuto ,¢ difefa di Celidora , ¢ vuol perfuadete., che fe bea

una | ‘guerra di nulla, ¢ perd feguita : fece prove de

ache ci feruiamo per derifione , quando altri ha fatca

cosa » © beilay che in effetto non € poi tale, anzié

+ Hai fatto tes a Serivi al pacfe..

 

  
   
  
    
   
  
   

Sac

 

  
 

z MALMANTILE

BATTICVLO di maglia’, Intende il Giacoarme difenfiva didoffo’, ciot unas
camiciuola compotta di maglie di ferro, ed €'la lorica anfulata,, che ulavano gli,
ancichi. B (e bene bacriculo dé maglia non ¢ veramente buon Fiorcatino, nondime-,
no é (peflo vfato, ma per giuoco, ed ¢ comunemente intclv peril Giaco, ¢ fi dice)
cosi , perché coprendo quelt’ arme le parti di dietro , nel moro che fa colui , che
Vha in doflo, batte in quella parte; come fi'dice Pigchiapetto’quél Gidielig,iche le
donne ufano portare al collo pendente ful petso , J A 4h 4 j

MALMANTILE, E' un Caftello antico vicino a Firenze circa dieci miglias , |
oggi del'tutto rovinato , € diftrutto: , n¢ vi Gi vede altro che lé muraglie Caftel- ;

 

jane , : :

CHEARIRE . Quefto verbo, che oltre a gli altri fignificati,vuol dire Far cono-
{cere I’ errore , o Render capace ; nel prefente nogo vuol dice Scaponire yo Sga-
rice: Zi tale mi faceva l buomo addoffo , gli bo dato.nna buena quantita di pugna, ¢ Ube
chiarito ; cioe con quefto ’ ho refo capaces ¢ fattogit-conolcere la fima , che io
di lui , e quella che egli deve far dimes’ Quefto verbo ¢-ttaslato dal verbo-Chia-"
rire , che ¢ Purificare ogni liquore torbido ,» ¢ contaminato danmatericscrati¢ 2°:

CeANAGLIA, Gente vile, ed abietta, che cali {aranno, come vedremo, i fol:
dati di Bertinella , j quali il Poeta mette Huomini d’ infima pitbe, che! Ci i
chiama /mi /xb/eiltj bomines . li Sig, Francesco. Maria Bellini in alcune fue bellitli-
me refleffioni » che fit contentato fare fopr’ alla préfeate Opera, ponderando la
parola Canaglia dice , che I’ allungamento delic parole in ag/éa fla Oggi in Tofca-
na un certo avvilimento y ¢difprezzo del-fubietto , e's’ ufifolo in-eole vili , € pie-
bee, ¢ perd fidica de’ Birri sbirraglia ; della Piebe. Plebaglia , ¢ gentaglia; de i
Fanciulli , ¢ popolo infimo Spruzaglia, ( metatorico da {pruzolo , acqua minuta )
eche quefto fia antichilimo Latino, fia di neutro plurale, del quale i {eruirono 4
Latini per comprender ’ appartenenze della cola 5 deija: quale 5 aps 9 Ve Be
delle cofe appartenenti alle navi dicevono Navalia; alla Cacina Popinalia,e molt’
altri , € corrotto da noi con’ aggiunta della lettera Gi

ee in crimen. Jefe, B} delitto di lela. Maefta cacciare una Regina del
fuo Regno. © “ not

FAR’ una pedina ,-Si dice Fare una pedina a uno allora che procurando que-
fio tale di confeguire cofa di fuo guito , ed eficado vicino a oftenerla, un’ altro, a
cui haveva confidato tai negozio; glicla leva fa, Viene dal giuoco di Scacchi, di-
ceudofi propriamenze: Dare fcacco di pedina. > (SBA ‘sb oblong

In oltre , chi ¢ pratico del giuoco di Scacchi'sa , che quando sé perdata las
Regina, fi procura di racquiftarla con fat’ arrivare una pedina al poito dow
ftava la Regina dell’ avverfario al principio,del giuoco ;,¢.cost intendere , che.
lidora priva del Regno conneniva,che fotto nome di Pedina tornatic a ricuperar-
lo, fe voleva effer detta Regina... i igdle Pane alan e0a\_u oe

Si potrebbe anche dire,che iJ nofico Poeta feguitando jl coftumeyche

  
 
  
   
    
   
   
  
   
 
   
  

ainae 2
jabbiamo
oan

di chamar Dame le Signore grandi , ¢ Pedine le donne d'intima picbe; habbia,
tcfo , che Bertineila, toglicado il Regno a Celidora,d’ habbia cavatadel nome d
Daina, per haverla .ridotta in grado miferabile, le habbia facto meeitaredd

di Pedina ; ma I’ effer’ il nome, di-Celidora nel terzo.c. ‘

nel quarto ; fa languire quefta cifleifione . 237 &

 
  

PRIMO CANTARE. 3
pileds .c!! STANZA IL.

i O Miufa, che ti metti al fol di fate S?anch’ ia fopr' alle picche dell’ armate
Sopr’ un palo a cantar con fi gran lena, Volto a Febo con te venga ix ifcena ,

I Che d! ogn’ intorno affordi le birae: eAccio ch'io poffa correr quefta Lancia,

VE finalmenre feenpt per a fehiena ; Dammi ta voce ye grattami la pancia,

‘QueftOrtava ha poco bifogao di spiegazione vedendofi chiaro , che il Poeta,
inuoca per fua Mufa la Cicala, ¢ cosi da a. conofcere, cheegli vuole {crivere af-
fatto » mofrando, che ‘per fare una compofizione come egli ha in ani-
mo, ¢ per delerivere'wna guerra qual fu quella di Malmantile,gli bafta hayer

| ‘Sipotrebbe anche dire,che il Poeta fapenda che non fi trova, che le Mufe hab-
biano dato mai alcuno aiuto effettivo , ed evidente , come dette la Cicala a Eu-
| _, tomo Locrenfe Suonatore nella difputa, che hebbe con Ariftono , fupplendo con
fa voce ‘al mancamento della corda’ ftrappata , come fi legge in Strabone lib. 6. -
___-voglia , come fece Eunomo , far pit capitale della Cicala , che d’altre Mule :
Eipud anch’ effere, che egli inuochi la Cicala,perché ftimi pit nobili detle Mule le
Seieteenen pil riguardevoli , come nate avanti alle Mufe( fecondo la.
favolo! de"Gentili ) d’ Huomini ,1i quali per lo.gran gufto, che hebbe-
‘0° del ‘cantare’, farono in cicale conuertiti , come fi cava da Celio Rodigino lib.
‘7. cap. 6, te: cui parole fono quelte : Fertur enim hofce homines fuiffe ante Adnfas ;
Pinole; eatthanyhe torr neo » iMorum nomnullos volupcare cantus nfque adeo

de Saiffe , xt. cibum , p que negligerent', imprudenterque perirent ; ¢x
‘quils deinde clesdarum gens Sit propagatum , ec,
~DiceiitDoni aelia (ua Zucca , che cutti li Poeti hanno la loro Cicala , ¢ ches
‘quefta ferua toro per Fama publicando Ie loro Poefie , onde il noftro Poeta fegui-
taado opinions del Doni inuoca ja Cicala deftinata al (uo feruizio , perche gli
faccia quetto di pubblicare le fue Poefic .
PALO Y ica, © Battone di legno, che fi mette per foltegno alle vitised altri
LENA. Sigoifica quello’, che i Latini dicono re/piratia , cio’ quieto , ¢ tran-
‘quillo anclito ; iche mentre ¢ nell’ Huomo ; egli fi maatiene fenza difficulta, nel-
de forze'smaiia facica di wit » 0 di mente speffo fa affaanare tal Lena,
enza pofarfi , appuato come fa la Cicala col

    
  
 
 
   

 

   

 
  
 

_ La Lena m' era dal polmon fi fmuntayec,

4: ftanza 6.°Varchi'ttor. lib, 5. (endo exli di pochifime fpirito ,¢ di

‘Franco Sace. Nowi127, Alls fine perdendo quefti ciechi la Lena per
vzicatizec. I Latini-con la voce His 5 ¢ con la-voce robur efpri-

‘Comparire in pu vedi fotto C. 4. ftan. 6,
. Tirar*a fine queft’ Opera.
. Col grattare il corpo alla Cicala, ti fa che ella canti,
la Cicala a grattare il corpo a Iui,accid che'egli canti. Quand’
€ duro'a — fi dice ; "Grartagli la pancia, che eg

5 2 . ‘ Cae

  
     

 

  
 
  

 

  
 

4 MALMANTILE

canterd , cioé interrogalo , ed efaminalo bene , che egli dira tutto quello , che tm
vuoi ; fi che il fenfo di quefto detta Grartare 11 corpo a uno,é Incitarlo a difcorrere.
Vedi fotto C. 2. flan. 8.

STANZA IL STANZA IV.
Alcun forfe dira ch’ ia non fo cica y Mi bafta fal che Voftra Altexza accetta

E ch’ io farei'! meglio a farms Zito, Dionurarmi d' udir quefta mia feoria

Suo danno;innanes pur chi vual dir dica, Seritra cosk come la peanagetta y

Fo io per quelte qualche gran delitto 2 Per fuggir L0xs0,e non per cercar gloria;

S?* io dirt male , ilCiel /a benedica ; Se non le gufea, quando l! aura letta

A chi non piace , mi rincari il fitto: Tornerd bene il farne una baldoria: .
pion fo , fe fela. Sanne quefti feivcchi, Che le daranno almen qualche diletto
 Chognun pre far della fuapalta gnocchi, Le Adonachine, quando yanno a letto,

In quefte due Ottave I’ Autore piglia a difender fe medefimo dalle male lingus,
¢ moftra , che poco gl’ importa | elier lodato, o biafimato:in quefta {ua Opera 5¢
che, non eflendo obbligato a veruno,vuol foddisfare a fe medefimo , ed.al {uo ca-
priccio ; e perd dice ; S'ia dird male il Cicl /a benedica, che fignifica V adia il nego-
zio , come é vuole, che non m’ importa, E feguita 4 chi nom piace ms rincart il
fitto,voiendo moftrare, che per non efiere obbiigato a render conto ad alcuno del-
Je fue azioni ; non teme d’ effer riprefo, o di ricever danno ; ¢ foggiugne : Ognun
puo far deda /ua pasta gnocchi , cio€ ogni huomo likero puo fare del fuo,a fuo.mo-
do. Conchiude in fomma ,che egli yuol dar gulto a fe medefimo , ¢ lafciar dire
chi vuol dire, baftandogli,che S. A. , cio¢ il Serenils, Principe Card, Leopoldo de’
Medici , a cui dedica ? Opera , fi contenti di riceverla , ed’ udirla , foruta comes
Ja penna gotta , cioé compofta non ad altro fine, che dj {paflarfi; ne fi cura d’ ac-
quiftar gloria per ral compofizione , anzi fupplica S, A. ad abbeuci pegeve
V havera letta,che ricevera qualche gufto dal veder’ andare a letto le Adon 4 8
per Monackine intende quello ,che intendono i noftri Fanciullini , ¢ioé quelle pic:
cole (cintille, che, nell’ incenerirfi la carta , a poco a poco fi [pengono , ¢ facen-
do un certo moto , pare che fi dileguino, (embrando tante Monache, le quali col
loro Jume in mano {corrano per il dormentorio, andando a ietto , 5A

CICA, Niente. Anzi vuoi dire ({¢ fi pyo ) Manco di niente, dicendofi in di-
minuzione Poco, niente, Cica, Viene dal ato Cicum , che vuol dir Quel yelo ,
che fi trova nelle melagrane per divitione de’ {uo} granelli , che per efter cost (ot~
ule , ¢ di piun valore , feruiva ai Latum per dimolirare la poca ftima, che faceva-
no d’ una cola , dicendo : . Ve Cicum quidem agderim , ec. ¢ noi diciamo in quelto
propofito /appola, lifea, er, y z sftp ue
> ZITTO.. Quicto. Stare zitto yuol dire Non parlare , Viene dal cenno.. 2
che fr{uol fare, quando fenga parlare fi vuoi fare incendere a uno, o pil, che
quictino , come iacevano ancora i Latini, che per accennare.ad.aitri , che fiiquie
tatle protterivano le due confonantt o *

 

nd

 
  
 

 

; . Fe BOE
GNUCCO.« Enna fpecic di Pane gramolato , mefcolato con anici; ¢ queftas

patta fra le nobili € Ja pit vile: Li proverbio Ogun pus far dedia a
fignifica ognuno ha ai libero arburio, ea ciprime quello, che 1 Li
Vind quifque in re Jua moderator, arbiter, ec. ii aun S
* $¥O danno, Non mimporta, Non timo quefta cola, E dicemmo; 4/¢ f

. t

   

at

 
' PRIMO CANTARE: 5

tal cofa mt nociva , /uo danno io la voglio non offante ec, Efprime Io la vo
Bene mi pud nuocere, ec. Vedi {orto C. 4. flan. 26. al ermine Jn ogni modo .
-. RINC-ARARE, Accrefcere il prezzo. E quefto detto Rincarare i fisto ufato in
Sao fignifica : Non fo ftima,ne temo le male lingue, perch¢ non mi pof-
fono far danno.
_ BITTOQ.. Pigione , Canone , ciot Quel danaro , che fi paga annualmente per
una Cafa , o Podere , o altri beni , che fi pofleggono d’ altri con pagargit ua tan-
‘to lvanno. Locarionis canones ,
. BALDOR/IA, Fiamma accefa in materia fecca , e rara, come paglia, ¢ fimili,
che prefto s'accende ,¢ prefto finilce ; detta forie Baldoria da Baldore , 0 Baldan-
-za,che vuol dire Allegrezza : quindi Liera fignifica poi Baldoria , come yedremo
foro C. 2. tan. 56. Diciamo anche Far baidoria , quando altri {pende allegra-
mente , ¢ fida bel tempo confumando tutto il fuo havere:; il qual detco vien tor-
fe da un religiofo coftume , che cra fra gli Antichi , che delle vivande fagre
non @i Ja(ciatjcro avanzi , ma quello che avanzava s' abbruciaffe ; il qual rito
fi cava dai Precetti di Moist in propofico deli’ Agnello Pafquale. Quetta {pecie
di Sacrifizio fu ufata anche da i Geatili Romani , ¢ la dicevuno: Proteruiam fa-
cere,che vuol dire Par’ una fiamma,o baldoria ; B pigliavano ancor'eifi proteruiam
facere nel fenfo detto fopra di confumare , ¢ mandar male il (uo, come fi cava,
da Macrob. lib..6. Saturnal.2., dove fi legge, che Catoue motceggiando ua talt
Albidio,che haveva confymato gurto il uo havere,¢ folo gli era rimatta una Ca-
fa; laquale gli abbrucid , diffe : Proteruam fecit 5 proprerea quod ca , qua comfes
mon paruerit, quafi combufifer s ec,
racigh ites cia eet ee Vv, =
Oferta glie gid, loc lay Ma poi ch’ ella la vusleseiolbo promeffo
u ee o poi morfe le mant , Nae mandarla pit doges ae,
» Perch? il filo nen va ne ben, ne preffo, Che chi promette,e poi non ta mantiene,
E verfiv'é chtil Ciel ne feampi i canis Si fa,? anima {ua non va mai bene,
+ Mofira:l: Autore; che layconyenienza per haver’ ¢gli prometia a $. A. R. queft’
Opera, I’ obbliga a mantenere la parola,quantunque égli conofca ,che non fia co-
fa d’ efler.veduta da S,A. R., ¢ per quefto s’¢ morfo le mani ,-cioe peatito
‘d’ haverla promefia, perché vede che la tefligurg dell’ qpera noa ta
1a bene, ¢ vi fon verfi che #/ Ciel we /eampi i cani , civg cos: ftrop-
‘ i, che tanto mae nop. vorrebbe ess . jpeno a un cane:
) feampare attivo,come ¢ in quefto luogo,, fignifica Liberare, Ma con-
l ia che $. A. R. la Table » hon fta bene che egli ja mandi pia in
domapi,ma ¢ dovere offeruar la prometia; ai che fare s’accigne
-quelta conuchicnza , ma ancora per il umore deila pena me-
promette ,¢ nom mantiene la quale ¢ che L’ amma fua non va mai
a da i noftri Fanciulli ; ¢ viene dali’ aatico-, poich¢ Itula-
reci fecondo il Monofino Fior, Ical, lingue lib. 3.9.109.

  
 
 
 
 
   
 
 
 
 
   

   

|

 

     
 
  
 
 

    
  
 
 

<

    
  
    
  

  

rurfus eri

pi: Che fuona lo fleflo che ; Chi da, ¢ riteglie
ale Lo iteflo che

Chi promette 5 @ nor mantieng L) anima

STAN: »

  

2 Nos autem dicimus id , quod (olent pneri: —

   
  
   
a
:
*
Fi
’
.
<
r.

   
   
    
  
 
 
   

6 MALMANTTLE.
STANZA’ VI
etiache? fi comead un che fempre ingolla Cos) la voftr Idea di gia [aroha :
Del ben di Dio, ¢ trinca del migliore, Di quei libron,che van per la maggiore,
Ml vin di Broxzi , un pane ,e una cipolla Fore potra, femcendofi fuocliata ,
Talor per uno fi berzo tocca il cuore ; Ear di queft anche qualebevc cm

Ripiglia animo il Pocta; ¢ {pera che S, A. R. fia per’contentarti di le;
uctta {ua Opera, fe non per altro,almeno per diftrarfi dagli Mud) pit ferij;¢ con-

fidera , che fi come colui , che ¢ folico far vita lautitfima’, havea talvolea di
mangiare un pane, ¢ una Cipolla; € ber vino da niente , cost chi-é folito legger li-
bri piu fenlati, talora avera non poco gufto a legger libri di baie, facezie ¥

INGOLLARE . Veo! dit Mangiat prefto, ed inghiottire fenza mafticare. :
Stuf pitt il verbo Ingoiaresefendo il verb» ingodare ufato nel Contado , fe bene &
forfe meno barbaro che #ygoiare , perché ¢ pid proffimo alla fua latina origines ,
che é la propofizione Zr , ¢ guia, ed in quefta appunto ‘inghiortita la leteera ‘L. fe-
condo la ftretta’ pronanzia comane Tofcana , € mutato in I ferrato 50 confonante
fi dice comunemente Ingoiare: Cost dice il Sig. Francefeo Maria Bellini .

DEL ben di Dio. Delle pi buone vivande ; che i Latini dicevano Jovis neftar,
e noi diciamo (arte di gallina , che vedremo in ‘quetto Cant, ftanza'64.

TRINCARE . Bere aflai ; Voce che viene dal Tedelco ; € diciamo Trine , 0
Trincone , uno che beva {regolatamente ; Vedi forto Cant. 7, Manza 1,
' DEL migliore , S) intende quel che vuol dire’, ‘ma il nfo pid altrafo puro Fio-
rentino é¢, che gli Olli ‘di Firenze’ vendono fempre due fpecie di vino roffo 'y und
di poco prezzo , che lo dicono Vino di forco, o di baffla , percht viene da’ luoghi
dj {otro a Firenze , dove fanno Vini deboli, ¢ leggieri; ¢ I’ altro di maggior prez-
20, ché lo dicono vino di fopra , 0 de migiiore; ¢ di quetto intende il at

TOCC ARE il cuore, Dar foddisfazione intera ¢ ‘altri mangia con gu:
flo , ¢ fi conofce , che quella vivanda gli fa pro's diciamo : Le tal vivanda gli ha
toccato il cuore .

SATOLLO. Sazio, Ripieno’) Dal tatino /arar . era ye ae

oe BROZZL. B un'di quei luoghi orto Firenze, dove nafee' i deto vino debote:
Vedi orto in quefto Cant. ftanza 47. meen 7

‘PER fcberzo , Yarendi non per taint , 0 fete; sma per Revo so tomagufo.
E' voce Tedefea, ¢ la pur fuona lo ftelso

ANDAR per ta maggiore , Efler della prima ‘ fle: Traslato da i Magitteati
dell Arti della Citta di Firenze , delle quali 5 : ena: ‘che fono

 

‘Giudici, e Notai ; Cambio ; Mer 5 Lana 5 Seta; Speziati, i
fe paflano a Cavalleria, Alere Minori , che art eenan *) Quota eee
non paflano , 0: ra non pafiavano aca ‘quando ‘in
ze fi dice , // ale va per: ‘delle:

    
   

maggiore ss Sete ‘una

are Arti , ed’ della cap sw claffe , Come s’ intende ie laogo’s
LIATO . Senz’ appetito : fenza puto di mungeyo

eae opie . .

: FAR una corpacciata, Saziarli. Empiee benitiimo il corpo =
corpacciata , gu altri legge , ree ° fa altra cofa’
te'fa una volta.

 
' PRIMO. CAN TAR E ?
seS TAN ZA VIL STANZA. VIL
Gia dalle guerre le Provincie flanche , Che d' baverle non n'¢ nevia.ne modo,
Non fol pits von venivana a batragha , Se dentr'ad un mar d'olio nun fi tufa,
Ma fur banditi gh archize/ tarmibiiche, E repute il padron degno d' un nodo ,
Edsexiamiliportar un fil di pagiia Che lolafeia indurive , e far la muffa .
Vedeanfi i bravi accniattar /e panche > Coun Marte,che wede Parmiaun chiodo
fol menar le man fu la tovaglia; Tete! appiccace malamente sbiffa,
» Kuando Marte dal Ciel fa capolino, Che metter non vi poffa fu le zampe
pres iL topo dall-orcio , al marzolino 5 Eche laruggin v’babbia a far le flampe.
j ‘a principio all’ Opera ,. de(crivendo lo ftato, in che crano le cofe del
| tee Mondo 5 eidicey che turso era in pace , ne fi vfava pili arme di forta alcuna ; ed i
bravi,ed huomini armigeri aceu/artavano le panche, cioe Stavano Ozioli, ¢ menava~

no le mani folo in {u la tovagiia,che viene a dire Aucndevano folamente a mangiare .
E qui {cherza con I’ equivoco del menar le mant 5 che yuo] dir Combattere , vedi
fotto C. 10, flan. 2, , ¢ trattandofi de] mangiare vuol dir Mangiare aflai, ¢ prefto

j vedi fotto C. 6, flan. 46. Marie perd s' adira , che non s’ adoprino. pit l armi
f L’ Autore aflomiglia Marte quando s'afiaccia al Cielo , ad un topo, che s’affacci
4 alla bocca d’ un’ orcio picno di cacio , ¢ d’. Olio, che s’ adira per yeder tal cacio
eheroee: dal padrone , ¢ di non poterlo arsivare , fe egli non entra in detto

“hae ee Ate berle. Spada 7 pugaale sed egai altra forta d’ Armi, a diftiozion

oe "Arnele noto fatto di pane per ufo di federe, ¢ poffono ftarui
pid in una volta ; detto dai ini fabfeltinms Ȣ viene dalla voce Latina
Planca , che fignifica Aflamcati , ¢ tayolati piani.,

ACCHI eee fe panche . Signitica ( ficcome habbiam detto ) Starfenes
fenza far cola », ¢ fpenficrato.... Ter. in An. difle Ofeitanres di coloro , che
fianno in, maniera., quali dica . Stanno shavizliando , che noi.diciamo :
See ment 30 Fare aru megli hai ,o Dondelarfela , ¢ fimili , ches
tutti ci {eruono per Peries s Perder’ i tempo in-vano , ed & quello che i ‘Latini
diflero; Afanum shes {ub pailio ; f

TOK AG Ll» ‘panno lino che fi diftende, pralla ménfa dai Latini
“ac patemacot) thabbiame fo forle da Toralsa, che crano i pansi, che cirewm-

4

   
   
  
 
   
 
 
  

Ur si keris difeumbentinm,
EVAR fe mani, lee fo affolutamente 5 ire Far ione ;
’ ak te, Quando sMto affol vuol dire Far. quifti
r ara Affretiarlial lavoro , che fara aggiunto; ¢ Gi ula dires
» duno che corra afiai , Mena Je mania leggere d' uno
in fomma d’ ogni Operazione humana, ancorche non facca
qui yuol dire Mangiar pr ue cas forto C, 6. tan. 46.
Guardar di (oppiatto. Quand’ altri procura di vedere , fen2a
lc ‘Japan diciso aun muro , Oaltro,, ¢ cavar fuo-
fcuopra quel ch‘.¢i yuol vedere ,.¢ gucte fidice Far
Dee teas é lo fleflo
le di terra y per ulo di conferuar’ Olio , vino , ed a li-

uvi, ed uggerv il cacio .

 

ie oak

nae

  
   
  
%s

-
'
-

 

8 MALMANTILE

MARZOLINO . Specie di cacio tondo fatto a piramide, e’col manico nel
fondo dalla parte pit grofla ; chiamato Marzolino,perché fi comincia a farlo nel.
mefe di Marzo , ed € il miglior cacio , che fi faccia nei noftri pacfi. E nel/pre-
fence luogo,fe ben dice Asarzoline, intende ogni forte di cacio.

DEGNO di nodo , Cioe merita la forca per |’ errore che fa a-non-mangiare»
que! Marzolino , lafciandolo andar male .

TVTITE U armi appiccarea un obiodo, Dicendoli: tale ba appiccate l armi
all’ arpione , ai chiodo ,$’ intende : Ll tale ha abbandonate ! armi, cioé Lafciato
@' eflere armigero . Cid viene dagli antichi gladiatori , i quali quando dal or-
Jo , col porger Joro una bacchetta-crano afloluti , ¢ liberati dal-far pu il gladia-
tore, folevano dedicar !’ armi ad Ercole, appiccandole nel dilui Tempio , ¢o-
me ci moftra Orazio lib, 1. ep...

— —— Keianins armis .
Herculig ad poftem fixis , latee abdirus agro,
Et lib. 3, ode 26.
Vixi puellis nuper idumens ,
Et militavi , non fine gloria ;
iVunc arma , defunttumqnue belle
Barbiton hic paries habebit , :

SBVFF ARE. Dar legnid’ ira . Sbuffare & quel foffiare , che fuol fare per fo
pia uno, che fia in colloras Traslato forfe da i cavalli: E fi dice Shufare,
quando altri adirato fi duole , ¢ in uno fteffo tempo minaccia‘con parole’.

Dante Inferno C. 18, : Ud. ,

Quindi fenriamo gente che fi nicchia
Nell altra bolgia , e che col mufo shuffic ,
E [e medefima con le palme picchia ,

Viene da Bufo {pecie di {offic 5 che vedremo fotto C. 3. flan. 57).

CHE larnggin v' habia a far le Stampe, La raggine, rodendo il ferto , vi fas
fopra certe impreffioni fimili a quelle , le quali con acqua forte fi fanno nel ra~
me per Stampare , ¢ pero le dice Stampe . E

STANZA IX. ae
Shircia di quis di ld per te Cittadi , / Si-voltaseda un’ occhiata ne? contadi 5

We altreguerre,ograh Campion difcerne, Che gia nutrivan nimicizie ererne y

Che battagie di ginocoacarte,e adadi, Enon vede iVillan far pik quiftions

E Stomachi d Orlandi alle taverne , ‘In fuor che'con la roba'del Padrone .

Marte , riguardando bene per le Citta, vede folamente ‘di giuoco , ¢s
gente valorola , ¢ brava nel coins + Voltatoli poi ne i Coneadi > che eran gia
Pieni di nimicizic , e riffe; vede , che dai Villani non fi fa alcra guerra , ches

che fanno con la roba del Padrone. ha

S41R. Ld , Sbirciare yuol propriamente dire Socchindere gli occhi , acci
I angolo della vitta , fatto pit acuto , pofla offeruare con pil facilita naa m
zia; Se bene fi piglia ancora per Guardar per banda , a fine di non”

 
   

vato s ‘come fanno (peflo gli amanti ; movendo Ja pupilla alla volta dell’ angolo

efternio dell’ occhio , con quel mufcolo , che per tal cagione da’ Medici fi chiausa~

amatorio; E quelto Sbirciare , 0 Bircio, e Sbircéo ha forte!’ etimologia dal Laci-
* : oy eae ao

 

 

:
{
    
   
    
    
 
 
  
   

PRIMO CANTARE: 9

mus , che Vuol dir’? angolo dell’ Occhio. Verg. Egl. 3. Tran/uer/a tuenti-
mis ; 14 qual parola vuol Seruio , che abbia origine da hirens ,eflendo che
Quefti animali infuriati per la libidine guardano obliquamente , ¢ torto le capre ,
che amano.
__E pero vero, che il nome Bircio, o Sbircio fi dice non folamente di chi ha gli *
chi fcompagnati , ma generalmente ancora di chi ha qualfiuoglia forta d’ im-
petfezione agli occhi , etféndo noi in quefto non differenti da i Latini , apprefio
i quali fe ben /x/ens vuol propriamente dire Vno, che ha folo un’ occhio , come
fi vede in Giovenale Sar. X. che parlando di Annibale dice: Cam Getula ducem
‘aret bellua lufeum ; che il Petrar. diffe: Sour! un grande elefante un Duce lofco.
ic. dé orat. Hic lufeus familiaris mens Cains Sentius = Lufciofus yuol dire Quel-
To, che ha Ia vifta corta , come fi pud dedurre da Varrone lib. 8. difviplin, Stra-
be Quello che ha gli occhi torti , da noi chiamato Guercio . Cie, 1, de Nan
Deor, Et quos infigni nota Strabones, aut Patos effe arbitramur ;che Petns fignificas
Vno che abbia gli occhi leggiermente abbaffati , che noi lo diremmo Lufchetto.
*Porfirione annot, ad Horat. lib. 1. Sermonum Sat. 3. ‘Peti proprie dicuntur , quo-
rum bue , arqhe illuc oculi veloctter vertuntur ,ec, Coclites Quelli , che fon nati ciechi
da un’ occhio. Plau. in Cur, Vrocule falue ; ex Coclitum profapia te effe arbitror ec.
Lucini ; Quelli che hanno ambedue gli occhi piccoli Plin. lib. 10, cap. 37. 4
ijsdem qui alter lumine orbi nafcerentur coclites vocant ,& quibas parui utringne ocelli,
‘bucini vocantur , ec. one Quelli di vifta cosi debole , che non veggono fe non
“Quando fplende il Sole . Plin. lib. 8. cap. 50. Ss caprinum iecur vefcantur , reffitni
‘Velpertinam aciem lis , quos Nyilopas vocant ,ec. Non oftante , appreffo molti que-
fe differenze fi confondono , pigliando fpeffo I’ uno per I’altro ; cosi appreffo noi
fi confondono i nomi Guercio , Bircio , Orbs , Lufto , e fimili , ec, accomodandogli
ite a qualfivoglia imperfezione degli occhi , come vedremo fotto in quefto
Cant.
vuol

   

-

 
    
 
     
  
    
  
   

. flan. 37. che Orbe, vuol dire Affatto cieco, ciot Ocxlis Orbarus , ¢ ftan. 66. . *
I dir Lulco, 2
~ CHE abattagiia di ginoco , ¢ 4 carte , ¢ a dadi , Non vede nel Mondo altre rifse
che di giuoco , nel quale egli non ha che fare. Perché torna non affatto fuor di
polito una riflefsione fopra la voce latina diea ,¢ la voce Talus : fi contenti
al Beetore , Che io faccia un poca di digre(sione, Sono molti de’ moderni Lati-
“ni, che fi fervono della parola zea per intendere la carta da giuocare; ma forfe
ooo ica » fe vogliamo credere a Polidoto Vergilio , al Meurfio , al
: ro, a Raffaello Volterrano , ed altri, che hanno trattato de i giuochi anti-
la chiamano tharta luforia; & Alea chiamano Ogni {pecie di giuoco
fe forle quei tali non voleffero foftenere la loro opinione con dite ,
voce dlea ¢ prefa in genere generalitimo;allora fignifichi ogni fpe-
fortuna : ma prefa in genere fpeciale, fignifichi la carta da giuo- _
imetto alla prudenza del Saggio Lettore. So bene che fino if
fa detto lea, come'fi cava da Marzialc. = x
9 & non damnofa viderkr ; ‘
tris abpulit ila mares ec. ; , "
per Fortuna , fecondo Livio lib. 37. che parlando d
iu tofto — che pace co i Romani per le dure coa
x :

   
 

  

Soe

dials
dizioni, che gli offerivano, dices, Nihil ea moverunt regem , tutam fore belli aleams
ratum ; quando perinde ac vitteians fibi lezes dicerentur,ec, E Colum, in Praefat. lib,
1, dice Afaris , @ negotiationis alea, Pare che errino ancora , coloro , che piglia-
no la yuce Ta/es per intendére il Dado, perché veramente i] Dado fi dice refera,
¢ rales veo). dire il Tallone , cio¢ Quel’ offo , che é fopra il caicagno del piede-,
donde fi dice vefte talare , la vefte lunga infino a i piedi; E quefta voce Tals ,
trattandofi di Mrumento per giuocare ¢ |’ af/ragalo Greco. , che ¢ quello che i, no-
ftri ragazzi chiamano aéiofo ; ma quefto ¢ forle minore equivoco., poiché tal’ofio
finalmente viene ufato in cambio di dado , fervendofi per numeri di quelle mac-
chie , o fegni , che naturalmente fono in dew’ offo, , come pil largamente diremo
forto C, 8. ftan. 69. Gioviano Pontano nel fuo Dialogo di Caronte diltingue que-
fio aliotio dal dado,dicendo; 4eque ego nangquam talis juft , nec referis. Lo fteflo fa
il Gellio lib. 1, Cap. 20. che dice Ta/us cabus non eff, cubus .m, eft fizura ex omni la-
tere quadrata, te/sera fex lateribus constat, Marziale pure nel lib.14.ep, 15. moftea
tal differenza , dicendo.: Non /um talorum numero par , teffera dam fit Adaior quam
talis alea fepe mthi ec, Tal ditterenza fi deduce anche da Cicer. lib. 2. de Divinat.
Quid n, fors eft ? idem propemodum , quod micare , quod talos iacere , quod tefferas .

E tanto balti per rifponderea quei che biaGimarono i'haver noi meffo per efpli-
care le prefenti duc voci Carte,e dadi il latino Charta Luforia,© Te/sera, che per altro
non importava al cafo noltro quefta digreilione , ¢ torna pil a propofito il fape-
rc, che tali giuochi tanto di dadi , quanto di carte, dice Platone in Pedro, ches
futiero inuentati da un tal Theut Dio de gli Egizz). Demoni ance ipff nomen Thent,
hunc primum numerum ,@ computationem numerorum , Gi jam » Ap iam y
talorum denique , alearumgque ludos audivi, e¢. Raffaello Volterrano, ¢ Celio Calcag.
de Ludo Talario , ¢ Tefleraria,, dicono , che quefti giuochi fuflero trovatida Pa-
Jamede nel campo Greco fotto Troia, ¢ perd gli domanda, Palamedis, alea ; fi co-

, me fa il Soutero ; Ma Lfdoro lib. 8. Originum, concorda bensi, che haveflero ori-
gine nel detto Campo Greco , ma da un Soldato , che havea nome Alea, ¢ che
da lui il giuocorprefe ilnome d’alea, Herodoto lib,1. riportato da Polid.Verg. lib,
2. cap. 13. dice, che I’ inuentaflero 1 Lidi per le caute che fi diranao forto C..6.
tian, ‘

6 STOMACH d' Orlando. Dicendofi: 71 tale ¢ buono flomaco, 0 vero. Euno
fomaco d' Orlando, ec. sintende,il tale ¢ coraggiofo , ¢ bravo; Qui pero valendofi
dell’ equivoco di Buono flomaco , che yuol dir Gran mangiatore intends Gente bra-

4 va nei mangiare , :

DAR ur occhiara . \otendiamo : Guardar’ alla sfuggita . ‘its
_ FAR quiftione, Far contefa, difputa , rifla; ma dicendofi affolutamente
“fenz’ aggiunta: Par quitione , s' inseade : Combatter con le fpade , ec.
SFANZA X

Ond'ei ch’ in tefta quell’ umor 8 ¢ fittay Niun fiata percis J ren feta 3
Pet CheVhuom ficrocchi pur ginftafuapofsa; __ Perch'ella dorme,e appitzod in[u lagrof

10 MALMANTILE fl

      
 

 

    
  

 

Ay
Senza picchiar,ne altro, ci feonfitto . _ Poiche la fera havea la bu me
Lifcis a Bellona manda itz una foofsa; _ Cenato fuora, ¢ prefo un na.
¢ Marte cifoiue d’unirfi con la forgila Bellona a fine di meitere (compigli nel
moa lo, ¢ andato a trovarla , la vedo ia letco a dormire beia ca ancora delia fera
padaca. (5) ean. PAO.

   

£

fe:
if

   
    
PRIMO CANTARE. it

stent + Quefta voce , che per altro fignifica:materia umida., ¢ mii 5
parlandofi d’ animali fignifica Flemma , collera , malinconia, ec, viene [peffo da,
noi prefa per Fantafia , o penficro come nel prefente luogo , che dicendo: S’¢
Bevo quel? umore in refta,vuol dire ha ftabilito , ha fermato il penfiero , ha rifolu-
to . La pigliamo ancora per Defiderio. Bartolomeo Cerretani for. nell’ anno
“T5002. dice; Sifenti che [ umore di Piero-de' Medici , di tornare sn Firenze non,
era fpento, ec, ea Papa Ale[sandro, defiderando fare il Valentino {us figlinolo Signore
di Tofcana, fi volle anch'egli valere di questo umore de’ Medici,ec, Diciamo Bell'umo-
re Voo che ha fantafie graziofe . Vedi fortorin quefto ©, ftan, 58, Si dice#ar’.il
bell’ umore Vano, che vuol far da bravo, ¢ da ardito . “/tale-volle fare il bed umo-
re col falire fopra quell albero,e cafco, ec. Donde habbiamo Ymoriffa , -che figailica
Vno di-cerucllo inftabile, ed inquieto . ‘Haver yrand’ xmore vuol dit’ cer fuperbo,
ed thaver'gran pretenfiont di fe medefimo . Z

‘CHE Phnom fi crocchi , ‘Che I. huomo fi perquota . 1 verbo crocchiare del qua-
le ci feruiamo alle volte per il verbo -cica/are ;come fi vedra in quefto Cant. ftan.
4. , oC. 3. flan. 3. , e che vuol’ anche dire Quel fuono , che fa un vafo di terra
cortd feflo.y come-Pentola , o altro vafo fimile ; ci ferue anche .nel ‘fignificato di
dar buffe ,.¢ quefto intende nel prefeate luogo ; propriamente? Quel cautare, che
fa la gallina chioccia , quando ha i pulciai..
GIST A [ua pofsa, Pee quanto egli pud ; Prafe antica latina inxta meum pofse,ec.
FIAT ARE, Significa parlare . Vedi forto C, 6. ftan, 12.
- Ein (u la grofsa. & in al buono del dormire. Dorme profondamente. Trasla-
“to dal’ baco da fera , il quale quando dorme per la 3.-volta,, che é il {uo dormire
pitegagliardo; fi dice: & nella grofia .

WON fente u# xtto, Non'(ente verun rumore , cioé ne.put’.un di queivcenni.,
xi che dicemmo’ fopi quetto»Gant, ftan, 3. 1) Varchisfior, lib. 6, dice. Com,
<abdertir sche ne cenat yne ritti, we atei brutti fi facefrero , ;

      
    
   
 
   
       
  
      
    
 
  

 

 
    
  

   

“eae Tatendiamo Cenar in:conuerfizione fuor di cafa propri
- “PIGLIAR — + Ambriacarfi. Ci fono pit fpecie di briachi , fra’ quali fon
“quelli , che fi dicono corti monne , che fon-coloro , che per lo troppo vino be-

vutojdanno nelle buffonerie , ¢ faltano , € chiacchierano fpropofitatamente , fa~
cendo mille altre pazzie , ¢ pois’ addormeatano ; € fi dicono ancora Corti nonne ,
sti liar la roma | Equeo.€ nome a: il goae comprende tutte le {pecies
chi, di che pa forto'C.2, ftaa. 69, la quefto C.ftan.77. dice . 5” im-
briacaron come tante monne dal che deduci., che fi pad dire : Prefe'la nonna ye preje
_ da monna , che in ambedue maniere halo fteflo figaificato ,
s bg, 2A sty cS TAN ZA XL
= \ Sta cheto thetose con due man dipiatto
_ Barre ba {pada fopr’ ad una cafsa,
105 Ghia qual s aperfe., edei viftevi drento
4 Robe wane/che, a tutte feve vento.
‘ogni-romore 4" che faccia Marte. non ne. ed
rove quivi in‘una caffa.,“Efprime il Poera:il genio ta-
bondo di Marte , ela natura del Soldato, che & lempre dedita al Swe:
(prime ancora la briachezza di. Bellona , dicendo , che lla dosiuniva riaasiet.
4 . Ba «tee

 
     
    
 

 
   

    

  

 

 

   
 
   
   
   

bs,

12 MALMANTILE

nelle materaffe fopra un letto mal rifarto ; il che moftra , che quando Bellona andd a
dormire era ip grado , che non fapeva diftinguere le coperte dalle materaffe .

LESTO come un gatto, La voce lefto , che viene dal Latino /ubiefus , che yuol
dir Leggieri , frivolo, ¢ debole, appreflo di noi fignifica Pronto, agile, ¢ deftro ;
E quefta comparazione Lefo , come un garto ; da noi ¢ ulatillima per e{psimere la
grande agilita d’ uno. Vedi fotto C, 2, ftan. 35.

SALOTTO , Intendiamo Piccola fala , cio¢ yn ricetto prima che s’ entri nella
principal fala . i"

MATERASSA, Arnefe da letto , quello che fi dice in Latino Greco Ana-
clinterinm a diltinzione di culcitra plnmea , che noi diciamo Coltrice ; eflendo las
materafia un facco largo quanto ¢ il leo , ¢ ripisno di-lana., ed.impuntito nel
mezzo,

Chero cheto. Quietiffimo . Nota che Ja replica d’yna fteffa voce, appreflo di noi,
ha la forza del fuperlativo . y

DI piatro , Cioe per lo largo della {pada .

MANESCO , Vno che fia , diciamo noi , delle mani, cio¢ pranto , ed-inclina-
to a perguotere , ed.no che fia inclinato a rubare .. Qui perd vuol dire Robes
atte , ¢ comode a effer portate via. Roba manefca intendiamo Roba , che ci fia
prenta , ¢ comoda a valerfene. é ‘i t

FECE vento 4 tutte. Porto via ogni cofa. Rubd ogni cofa. Che quefto inten-
diamo quando diciamo; Far vento a una cofa . 4

STANZA XII,

Ma non fa fi, che la forclla sbuchi , S'allunga , ¢ fi rivolta, come i ciuchi:
Di modo ch'ei lachiama,e lifafretta; Ella ch'ancor del vin hala spranghetta,
Lafalletica,e dice:Quvia fuor bruchi: E, fatto un chiocciolin fu P altro late
Lo Spedatingo vuol rifar ie letra, Le vien di nuovo t afine legato.

Con tutto che Marte faccia ogni diligenza perché Bellona fi fuegli , folletican-
dola ye sridande » che € hora di levarfi , non trova modo di farla deftare ; anzi,
¢flendofi elia alquanto folleyata per caufa di que’ romori, s’ allunga , ¢ fi rivolta ,
poi fi rannicchia , ¢ di nuovo fi addormenta , perché il yino la tiene opprefla. 5
Ed ¢ bella is pirtinee duno, che dorma con gran gufto, ¢ volenticri; perché
guelto rale , fentendo ftrepito , fi risveglia alquanto , ¢ facendo,, per lo pill, le»
operazioni, ¢ moti delcritti nella prefente ottava , feguita a dormire. >

SAPCARE . Intende {vegliarhi, ¢ levarfi ; Vicie da.guella buca, Ja quale fi fq
nelle materafle col pefo della perfona , A

FAR fretta 4 uno , 8 intende Stimolar’ uno a far prefto , alta as

SOLLETIC ARE, Stuzzicare leggicrmente ugo.in alcuna di quelle parti del
corpo , le quali , toccate cosi , incitano a ridere , Vienedal. verbo Sedlicito. , folli-
{TPE SSO) TL PO AP | Liaw shgeucs ay oce't | ae

FVOR bruchi., Dalla voce Brucp habbiamo.jl verbo Bracare, che yual dir Le-
var Je foglic a gli alberi , ¢ per metafora ire Andar via , onde quando .di-
ciamo: 4 tale sbraco y inteadiamo, Andd via sed, il fimile incendia

For brichi y cioe andate via... Luigi Pulci Bec. Ognun bruco ,.che, Hera Ia tre-
senda, Onde qui s’ intende Efe, dat Jette. Detto, ulatifimo in, quefte propofi-

10, 3 says leant *¢

 

 

 
’ PRIMO CANTARE: By

© LO Spedalingo vuol rifar le letta. Quefto detto fignifica , E' hora tarda ,¢ cas
levarfi dal letto ; ed ha origine da gu {pedali , ne i quali fi raccegtano i Pellegri-
‘ni; dove , quando ¢ hora di levarfi,, ¢ che i poveri, ¢ i Pellegrini feguitano a.
flare nel letto, Jo Spedalingo , cioé il Guardiano, 6 Sopraccid dello Spedale fuole
per (vegliargli gridare: S’ hanno a rifar le letra, ,
CIVCO .. Afino giovanc , Oypoledro . Forfe dal latino Cicnr , che par che vo:
. glia dire Beftia addomefticata , ed agevole ,
HA la fprangherta ; 0 fanghetta . Quel duolo di tefta , ed inquietudine , che fi
fente la mattina , quando , Ja {era avanti s’é wore bevuto , ¢ poco quella not-
te dormito , per lo qual duolo pare , che il capo fia fprangato ,0 legato con /pran-
Gghetta , o frangherra . Che cosi fi chiama ogni verga di ferro , 0 regolo di legno ,
che unifca due materiali infieme ; come fi dice porta /prangata , una porta , in
- mezzo alle di cui impofte fia conficcato a traverfo un regolo di legno , affinché
dette impofte non fi poflano aprire , E frangherta pure fi dice quel ferro, che ferra
infieme ’ impofte de gli ufci, il quale s’ apre , ¢ ferra con la chiave facendolo
{Correre in certi anelli , come il chiaviftello , dal quale é differente , perché il 7
chiayiftello non fi pud, o almeno non é in ufo aprir con la chiave .
E.ATTO un chiocciline . Ciot Rannicchiatafi, 0 raggruppatafi quafi in figura
di chioccioja , come fono quelle focattole , o ftiacciate , che fanno le noftre don-
ne per i Bambini,le quali chiamano chiocciolini,perché gli fanno a figura di chioc-
siola ; ¢ come vediamo , che nel dormire fa per lo pili il cane .
LEG AR t afino,, Addommentarfi, Detto, che vienc da i Villani yetturali, che
effendo es rac foprapprefi dal fonno, legano I’ afino , e s’ addormentano nel
Juogo 5 dove gli piglia il fonno,, E col dire: Mrale ba legato (enza !' aggiunta.
@ Afino,s' intende ; Il tale sé addormentato . Francho Sacchetti nov. 171. dice:
Effendo Gulfo entrato nel letto,guando fu per legar I’ afino,il compagno comincio col man-
taco 4 fofiare . Bocc. gior. 4. nov. 9. Diche la donna spaventata , per fuegliarlo co-
qnineio a prenderlo per lo nafo , ¢ tirarlo per la barba, ma turto Gra nulla , perche c7li j
raviglia legato L afino . ec.
STANZA XIIL
O corna diffe il Re degli Smargiaffi, » Oche per lagran furiacgliinciampali ,
_, E intanta le coperte bavenda prefo Och’ elle fulfon di fovere!
Le ne tira lontan cinquanta paffi , Bafha cl’ ei barte ilceffo , eche gi

ss eee egli fi trove diftefo ; In testa la beffemmia delle corna..

Incollérito Marte leva le coperte a Bellona , ¢ le butta in terra, dove cascd
ancor’ egli ,¢ capo 5 ¢ fi fece un bernoccolo., o tumore nella telta , quali
tumoretti da |

ore. fon chiamati Corna per elicr nel Juogo , doves i
seers piglia la voce ptennia non nel fi proprio G- :
cil y © leyare empiamente alla Divinita quello che fe Ic convic-
fignificato di maladizione , 0 precazione > come & pres
ftra Tofcana , ed in altre parti d’ Italia , ¢ ipectalmenes ins
€ intefo comunem :

inte per Maledire . E qui dicendo :
fence Quell’ imprecazione che ha-

        
    

 
 

         
 
  
 

 

 
   
   
 

     
      
   
 
 

 

  
    
 

14 : MALMANTILE.

uno di quei bernoccoli , o tumoretti, che per-éffer nella tefta (cherzofamente fi
chiamano Corna.

SMARG/ASSO . Huomo bravo. Armigero . Ma perd l*ufiamo:per derifio-
ne , ¢ per intendere Vn’ huomo fuor dei limiti della ragione , e deila prudenza ,
ed uno di quei petulanti ,'¢ minacciofi , che pretendono di fp ognuno ‘con
Ja lor pretefa bravura. sao nd + Arey

CINQUANT A pafi , Lontano affai, Detto iperbolico:ufato fpeflo anche in
piccoliflime diftanze.

INCIAMP.ARE, Dar co i piedi in qualcofa ‘nel camminare : @ il’ Latino

‘endere..

Foran pefo, Pefo grande, pefo fuor dimifura , Petr. Canz.'17.
Atri ch’ io fieffo.,'¢ ildefiar foverchio , _

E certo che le coperte eran di grandidimo pelo , perché Bellona fi feruiva per
coperte delle materatie , come's’ ¢ detto fopra. 5

BASTA . Termine conclufivo ulatifimo da‘Noi,quafi diciamo’: 'E a fufficien-
za, e¢ fi dice anche i4 baffanza , dal verbo Baftare , che é il latino fufieie . 1 La~
tini dicevano Aat,Sat ef . Piau.‘nel Penuo fi ferut della voce Bat, fenza aggiunta
di Sat ef , edi Giofatori di eflo-dicono : Bar vox , qua htimur Chm quempiam ine
bemus tacere, 4 4

CEFFO , Vuol dir propriamente I] mufo del cane , del porco , ofimili , mas
fi dice anche del Vifo , o faccia dell’ huomo , ma per'lo'pia ‘in derifione , ¢ per
intendere una faccia bructa , e mal fatta . Vedi forto C. 4. 'ftanv'ro.

« STANZA XIV.

Ella fuegliata allora efc) del Nidio , Cofa cht a Marte diede gran faftidio ,
E dicendo ch’ in cio gli fa il dovere, “Ma perch’ ei nonvnel darlo a divedere ,
E ch’ ei non ha ne garbo , ne mitidio, Si rixza, e froda il colpo che gli duole,
Non fi puo dalle rifa ritenete , Poi dite chewuol diiedue parole .

Per |’ infolenze ‘di Marte.,:Bellona'finalmente ‘fi fueglia,\¢ Ja la’burla a Marte
perch egli é cafcato-, ¢ Marte fingendo non fentire la percofla‘fi rizza ,¢ dice a

cllona ,che vuole alquanto difcorrerle.. . oi

VSCIR del nidio , V {cir del létto : quale chiama Nidio per la fimilitudine , che
ha nelle-materafle quel luogo-, dove s' ¢ dormito., col -Nidio , entro al'quale co-
vano gli uccelli.. 2 eee
GLI fra il dovere . “Gli & interuenuto quel ch’ ei meritava’. Dovere  \ginffo ;

 

ginftreia-, (ono finonimi . i .

NON ba garbo , Non ha accuratezza | Per i di quefta parola Garbo
é da fapere che-erano in Firenze due luoghi pri » dove gia fifabbricavano
panni Jani d’ ogni'forta ,-uno detto S. Martino da'una’Chiela's chequivi«é dedi-
“cata a-detto Santo:,‘e I’ altro'fi domandava il Garbo} 9 idi Mtradefi con-
fervano’fino al prefente . ‘Nel detto il Garbo'fi icavano le pannine di

tutta perfezione ; ¢quelle che fi ‘icavano in S. Maitino‘erano ‘fempre
“feriore.condizione nae venne iniufo il dire’: ‘La tal cofa.é del Garbo,

denotare Ja petfezione di quella tal cofa., *E dalle robe venne alle
comincid a dire :‘Huomo di garbo., huomo , che ‘ha garbo yee, ndo d’
V.a0 »che operi bene , econ accuratezza.,'Cosi dice il aoe ete

  

 

 

 
  
  
  
 
   
  
  
   
 
   

PRIMO CANT.ARE ay

alla parola Garbo. E noi diciamo ancora in quefto Senfo: Non ha ne Garba , ne
S. H#artino ,
_ AUT LDI0, Gindizio; ordine ; Parola corrotta da metodo.

VON fi pu dalle rifa ritenere . Non pud far di non ridere .

DAR faftidio . Dac noia ; dar difguito .

NON vuol dario a divedere »Non vuol farlo conofcere . L’aggiunta della par-
ticeila , di, al verbo vedere s’ ula folo in quefto cafo per efprimere , far capace ,
© render bene informato . u ss

FRODARE , E noto il {uo fignificato,, vengndo dal Latino fraudere,, che vuol
dire Ingannare ; Ma noi lo pigliamo ancora per Occultare , 0 noa manife flare,
come ¢ prefo nel prefente Iluogo ; ed ¢ traslato da quel frodare:, che vuol dires
Na(condere quaiche roba alla porta della Citta , o.alla Dogana per fraudare la
Gabella con il non pagarla , che fi dice Far fredo Vedi foto C, 6. fan. 28.

STANZA Xv,
Di pura Dea rifponde, chr io afcolto Quello non fol ; ma quanto baveva tolte
» Hai tu finite ancora? Ovvia, di prefto: Di quella calfa, ci rendese mette in feffo,
Ma prima di queipanni faunrinuolto, E poftofi a feder fu la predelia ,
E gettaio in ful letto yh’ io mi vefto. Con gravitd atpoi cost favella .

Deicrive aflai bene il genio inquieto ,.¢ furibondo di Bellona , meatre moftra
l'ardenza , con Ja quale cilia ftimola Marte a dir quanto gli occorra , interrogan-
dolo fe egli ha finito , quando fa che non ha ancora cominciato , ed in uno fleffo
tempo gli comanda , che rimetta le coperte in ful letto: Vbbidifce Marte, ¢ s’
accomoda a federe per dar principio al dilcorfo , che fentiremo ,

EAR’ un rinuolto . Blo ftetlo che Affardellare, abballinare , 0 far balle ,

_ AUTTERE in fefto... Accomodare ; aggiuftare .. E ik Latino aprare, ¢ das
AMtetter in feflo diciamo Rafercare , 0 mercer in afsetto. Varchi Storia libro 8.
Hau 4 di , € notte favorato per mettese il Salone in afsetto ,.L’ Autore delias
ftoria de’ Piacevoli , ¢ Piattelli lib..2. dice Wo» pareva poffibile difender. ta fila, al-
degare i lafci, ¢ dar fefto al tutto , ¢ pure ben tofto fi vedde mettere ogni cofa in afsetto .
_PREDELL2A . Qui inteode Quella feggiola facta a cafletta , la quale fi tien,
vicina al letto per I’ occorrenze del corpo ; che per alrro queita voce predelia ha
mol ificati., chiainandofi prede//a ancora-guell’ arnefe lopra il quale fi pula.
0 | ido partorifcono ; Predella fi dice quelio fcaglioae di legno , fo-
pra il quale ffa il Sacerdote quando celebra Mefla ; ¢ quella feggiola dove ficde
il Sacerdote quando in Chiefa afcolta le Confe‘fioni detta alcrimenti Confeiliona-
Je. Pre pure ¢ deta sapete parts della briglia, che fi tiene in mano , coinc fi
dino efpofizione a Dante nel Purg. C, 6,

da com’ e[ea fiera ¢ farta fella

corretta dagls fproni ,
HA 5 OC. é “ha
fon SY ae re , mentees 4 seeanrate vuol dire
¢ madia contra, i i Cicalare, grac-
ur Gal: A tale non Chee heres » Me cicalava, ma Boek.
¢ parlava con fondamenco , regolatamente , ¢ feriamen-

rm 1 STAN:

   
 

 
     
   
   
 
 
 
  
   
 
   

     

 
 

 
    
7"

16 MALMANTILE:
STANZA XVL
Sirocchia , male nuove ; poi ch' in Terra Sai, che la Morte ne molefta , e ferra’,
Veggiam ch'all'armi pin ne[suno attende, Che la/ua firegua anth’ella ne pretende,
Onde il noftro meffiero,ideft la guerra , E fe non [e li d@ foddisfazione ,
Che [tain ful taglionon fa putfaccende; La ci fara marcir n' una prigione .

Marte in quefto fuo difcorfo moftra alla forellyla neceflita , che ambedue han-
no che fi faccia guerra , per il bifogno, che hanno di guadagnare almen tanto da
pagare il dazio alla morte , accid che ella non gli faccia metter prigioni , ¢ qui-
vi morire , fe non le pagano detto tributo .

a SIROCCHIA Sorelia . Parola Fiorentina ; ma oggi poco in ufo. ‘Dante nel
Purg. C.-4, ¢ Canto 21. ; 4
Che fe Pigrizia fufse ua Sirocchia , ev.
L? anima fua ch’ ¢ tua , e mia firocchia , ec.
5 ST A in ful tagtio, Due {pecic di Mercanti di drappi , 0 diciamo Setaiuoli fono
in Firenze .1 primi fabbricano drappi per mandargli fuor diStato , 0 per ven-
derglia merciai di Firenze a pézze intere ; i fecondi fabbricano , ¢ vendono ins
Firenze a braccia , 0 diciamio a minuto , ¢ quetti fi chiamano Setainoli , che fpanno
in ful eaglio, Marte dice alla Sorella ,che la loro arte, che /ta in ful raglio non lavo-
ra pit , ed il Poeta (cherza con I’ equivoco di Tagliar drappi,¢ tagliar huomini ;
€ che di quefta lor’ Arte di taglio vuole la morte , che effi paghino il dazio,, dan-
do alla medefima tanti morti !’ anno ; onde ¢ la guerra non lavora , non poffono ‘
pagar ee tributo . 3

SERRARE , 0 far ferra a uno, Affrettarc, ftimolare, violentare uno. Vedi forto
Cy 9. Manga 13. é

ST REGV-A , Intendi quel dazio, che devono alla morte . La voce fregua y che
vuol dir Porzione dovuta , vien forfe dal Latino frena , che fignifica mancia .
Varchi Stor. lib. 10, 47 alcune cofe vanno quei tali rifpettati, ma in molte pis devone

andare alla medefima firegua, e ragcuazlio degli altri, ev. r % 7
DAR foddisfaxione . Soddisfare , Adempire ogni forte di convenienza , 0 di
- debito che uno habbia con un’ altro: Ma mente s’ intende Pagar quel da-
4 naro , del quale uno é debitore . ; ey

CL fara marcir n° una prigione . Ci fara ftar tanto in carcere , che noi vi morf-
remo di ftento ; V’ infradiceremo . ;
STANZA XVIL

‘ . Bifegna qui pigliar qualche partite,  - C’ ha dato un toffo nelle feimunito ,
ie fi Se noi non vogliam’ ir nella malora 5 Mentre di Malmantil fitrova fuora,
IS Ed un cen’ ¢ ch’ é buono arcifquifito , E paffandola fempre in piagniftei » —
Qual! é , che fi rifuegli Celidora Pigra fi fla , come non rocchi a leis
: Seguitando Marte il fuo di(corfo , propone y che fi ponga in animo a Celidora

gid cacciata da Malmantile , di rifolverti alla vendetta 5 ¢ cosi far nafcere las
guerra ; per rimediare a’ lor bifogni . o2)- aaeras oe
PIG LIAR partite, Rifolverfi a pigliar qualche modo di rimediaré» ¢

 

 

; ANDAR nella malora , \ntendi Andare in prigione per quefto debito. E il

latino Zz malam Crucem abire . e
e4RoIS QVISITO . A buono , diciamo in augumento ; buono , pil ee ’
¥ age di -buo-.

   
1. wate 7, 1 ees Se
Sa yt he

PRIMO CANTARE, 17

- buonifiimo , ed in luogo di buoniffimo diciame anche {quifito , facendolo fuperla-
tivo di buono =e cosinon,dourebbe patire agumento; tuttavia fi dice Squifito , pit
{quifito, (quiGcidimo,o arci(quifizosimitado forle i-Latini,che da optimus {uperlativo
di bonus, hanno, opsimifimus, Si trova anche nelli Scrittori antichi della lingua nb-
ftra. Vaccrefcimento al {uperiativo, I] Bocc.nov,19.dice Cosi santiffima donna, E, nov.
Go. Gosh ortimo pariatore, ec, Gio; Villani lib. 12, cap, 104, dice : Rima/e in pitt pef=
Simo haze, ed: al lib. 7, cap. 100, La guile era della maggiore diS. Gio: , ed era moito
Sortifima ¢ cap. 101, A pie delle Atontag ne deste Pirre molto altifime, & queto Au-
tore I’ usdfempre, che gli venne occafione d’ e(primer un gran fuperlativo ; mas
da i moderni.non pare , che fia molto wlatoy e con ragione, perché con Paggiun-
tadi molto, cost, piu, © fmili 5 i) fuperlativo che ha la natura del fuo nome, riceve
moderazione , ¢ pili tofto fcema , e torna indietro della fua eflenza;; ¢ cosi volen-
do dire ,che-una Montagna fia altiffima con Aggiungerui il molto , cost ,v affai, fi
viene:a dire che la Montagna fia alquanto alta , ¢ non in tutto alta, o altiffima >
rigevendo in quefta maniera il fuperiativo limitazione,e non agumento. Saluftio
difse muito puleherrimam » quando riporta il difcorfo fatto da Catone Viicen(e as
Cefareim propofito della congiura di Catilina,
» ha particella arci, che vien dal Greco archos,
‘che da i moderni pen efprimere ({¢ fi, pud.) di la
noftro. Poeta I ula anche nel Cant, 12. ftan. 34
particella arci aggiunca.al fuperiativo ta |’ ecco
-derare: ,.e-non accreicere, ec, Ah ug
' » RISKEGLLARE, ,, Non dal fonno , ma dalla Pigrizia, then
Rada data un tuffaneHofeimunito, Ha factauna azione da {ciocea, ¢ da ftoltas*
~  Metaforico dari vintori-y,i quali volendoy che la feta.,'0 altro, pigli il colore, I'in-
tingone nel’ bagao-di quel’tal ‘colore tante volte 5 quante par joro che ferua. £
quefto dicono Dare uniruffo 50 pie ruff. E dicendoti W/ tale ha daroun tafe nello cim
unite St intende che quel tale habbia fatca un’ azione da {cimunito, non perd
i fiaideltutto fcimunito.. Quefto termine dar’ un tdfopud forle anche ve~
; da coloro,, che aflogano, iquali prima di morire tornano alla faperficie del?
i Sra due ,"0.tre volte 5 il chediciamo : Darei tuff; eche,'s! intenda ¢ profimo
Ee del tutto.{cimunito.,, come é-vicino a effer del tutto. morto ‘coluiy, che da i
a ‘nell acqua . La, voce /cimunito credo che fia compofta di due dizioni , cioé
__feemo , (che vuol-dir? uno che habbia manco giudizio di quel , che fi conviene) €
‘Hilton © venga a dire wnitamenre fceme , cig Iccmo ugualmente » Odel pari, oi
Je partia‘un modo, , che'conchiade affarto (ciocco y:¢ infenlato.. :
Sf trova fuor di Adalmantile., E priva di-Maimanti perché le.¢ ftato tolto da
&B ‘fence trova cflettivamente fuora. Diciamo : Jo fonsuora di tal pen=
Sero per intendere :'io non ho pit queflopenfiero. 2k
4, PAGNISTEL. Singulti , (olpiri mefcolati‘con pianti.. Voce-da donnicciuole,
Y cee 1s 23 én ; 8
= COME non ‘tocchi a lei, ‘Ciok ‘come I intereffe in quefto negozio non fia,
‘Stafpetti a lei, mavad un! altro-, : “

 

   
  

 
 
      
  
          
    
 
 

 
 

1

    
 
    
   
 

che fignifica Superiore, stula an-
» © piu fu del fuperlativo , ed it
ma appreflo dime anche quefta
» che 1’ altre dette fopra di mo-

 

 

\ donating irate.

 
   
 
     
     
  
  
  

     
      
 
 
   

    

 

  

 

 

   
 

 

ee

aoe)! ek

 

18 MALMANTILE

P STANZAKVIIE > ° 5
Afa come quella, pare a me, che afpetta, \Flor mentre ch'ella inarme'non fimetra
© Cheile piovano in bocca le tafugne ,- ‘Per racquiftar lo fcettro,e fue capagne;
Senza penfar un’ Lora allavendetta €4iolto male per noi andra il negozto,
La fua difgraxia maledice , @ piagne; \ Che muoiam di mattana,ecrepia d’oxio,
Marte pone in confiderazione a-Bellona yche fe non trovano il modo di far ti-
foluer Celidora ad armar gente per racquiftar’ i] (uo flaro’ di Malmantile 5 11 snes
gozio andra‘inal per loro, che non hanno faccende . 3D Bemnierse L
CHE le piovano in bocca te lafagne?, Vol del’betie’, ¢ non vuol’ durar fatica aus
domandarlo: come per efempio uno che ha'gran fame } fitaicia pit tofo'finired
da 5 aioe » che chiedere il cibo ddvutogli,’ma afpetta che il ‘cabo gli corra:in Boos
ca da fe, Coftume di Cuccagna.' ’ srobover
LAS AGNE , Specie di paita tirata ; ed affottigliata come un velo’, ‘
VN Iota, Piccola lettera dell” Alfabeto Greco, ¢ fi piglia per efprimer ib mente,
MORIR di mattana , Morit di malinconia'; quafi dica : E ‘cost'grande laymax
linconia , che mi nalce dallozio¥, che mi fa‘divenir marco; emorire, *Vieneda
mitto, mattas ,¢ forle prima fi diceva’: Perite di'morte mattanay ec. che era una
occifione fpeciale, che fi facevarda gli Arufpicj nell" immolar le Vittime;'le i
faentravano vive , ¢-cost morivano & poco a’ pdco'crudelmente ;La onde i Lati=
ni aggiungono fempre a-quetto verbo 1a parola’ morte ‘0 fupplicio , come filtres
dein Cicerone , che dice Aforre. mattavit .& fupplicia maktari, 8 oN SMI
CREP-ARE, Quefto verbo Crepare , che fignifica Quando un legname fi fpac-
ca, 0 fende da p c (e: fignifica ancora Morire'a ftento ,ed in quefto fenfo & pre~
fo nel prefente luogo’; © forle ¢ prefo ‘nel (enfo d’ Alléntare,'che vuol dire Quan-
do a uno per la foyerchia fatica'cal gl’ inteftini , ¢ voglia'Lroni pars
Jando , che s’ intenda ; ¢ cosi grande Ja fatica , che duriamo 4, che ¢i fa*allentares
ST. XIX. SoRAN' Z°A OMKE 0 op
| | Fartene’dungue, e sn abito'di mage, +.

 

Chia? forfo cofherife ne Ha cheta \

 

> Perch? ella vede\effer legata corta’, 1) Dopo il formar gran civcoli ye figures >
Che # el havefje un di gentese monera ~) Conchiadi, ¢ dille-che.twifer

age
Tu la vedrefti nfeir digatta morta; ' Che ee 3 4.4

Aa qui Baldon fard dal” A alla Reta

 

(So quel chi dico, quando dico torta’)

Ritrova tu coffer, fra feco in thono, ©

Che quit’al refta anchtio fard dé buono ,

tafie ella fi a'progurare di ra
Jona y che lavadia.atrovare,'¢ la tincuayi ton ditle'y cheypretto 'riayera
flato , ¢ le metta addoffo I’ usbergoincantatos 9) Hi -osoi akg

CHE fa? Quefto termine

i ‘(ignifica; la tal cofa pad cffere,o hon pad el vai |
dica : Chié colui , che fa di ficuro , che la cofa fia , 0 non fia cbsi?-? peng
forze baftantia farqucilo 5 che ella 3

E' legata corra,, Cio’ non ha f

E-quel tuo core lie'di drago
Ambottire Surah ta bradure 5
Mairile in deffo , che vedrala poi)

Bar lo fpavaido pin , che ru non vuoi;
Marte facendo tifleffiante’y che fe Gelidora havelle chi la foccofretie ; ed:aias

  

ifare Jo ttato,percio’ ordin aie

Traslato dal cavallo,, afino, mulo , o fimili, i quali quando fon fieri ye bizza fi
Jegano dovungue fi fia con lacavezza corta, affinché non deta va loro”

d’ attorno.

 

VSUIR

 

ae

wore ee Te whet ee

a

oh ted

 
    

    
 
 

PRIMO GANTARE: 19 ie

© VSCIR. di-gattt mortay.y Farfi vivo dimoftrarfi, fiero . Far la gatta morte vuol
‘aicSimulare oll LaliBn. Tran, Cant, 2. flan. 12. parlando dsl Cavallo Troiaso

»
ami'h ms (2 Beanie ixgoarinleolt eqsafin legney isiqi? Py se
. ib cE w'sttendono a far lagatta morta’,

Bi eatiai differo lepus dormiens, E noi diciamo anche far la gatta di: Atafino.
Vedi foro C, 7. ftan. 69.

FARA dal’ A alla seta, Fara puntualmente quanto bifognas Para il tutto.
LA, la Z. fono il principio, e il fine del noftro Abbicei, onde. con quefto ter-
mine-intendiamo Sardsfatro il tutro'y come appunte appreflo i Grech Alpha, C.
Omegza.;che ¢ lo fetlo. che a, Capire ad calcem.de’ Latint. » >:

SSO quel ch’ io, dicos quando dicoturta , So benifimo.come fla-quefto negozio ,
Elprime a intend’ io, U Pulci nel {uo Morgante fa dire a-quello fcellerato i

_—a

Opa» ca credo nella torts ,¢ nel Tortello:
2 9 Soquel, ch’ io dico , quand’ to dica torte,
Ez voo! dre M pao io aque! ch’ io voglio dire y¢ queHo ch’ io intenda per
terta ys
DSTA feco be tweet "iia feco unita ; Va @accordo feco.;Traslato dalla MuGcas
a FARO! di. buono. Negoziero. da vero. Paro quanto bifogaa.,. Quando uno
iuoca di danari fi dice Far di buono , che vuol poi dire Operar con attenzione; il
e non fi fa: epenieeneht fi giuoca di buono, » non ponendofi attenzione quando
ginacarda burla,. (o. 0
a Magy ‘Nonchanno a i Maghiabito particolare' mail Poeta fe lo
digara inpquella guilar; ehe ha veduto in commedia , cioe veltelungay granbar-
ba}c.lavergalinmano +.B Adagae voce Perfiana,chefigaifica Sapiens 5 ¢ quello
che i Grecirdicono Filofofo. E di quefta forte Filofofi furono quellisMagi , che
andarono ad adorare Gicsl bambino.) Ma perché Zoroafte fu anch’ egli uno di
cali Pilofofi detti Magiy:e{econdo Plin, lib. 30. cap. 1. fu inuentore dell Arté>
dell’ incantare,, perd:talrarte¢é detta Magia, ¢coluro, che l’efercitano fon chia-
_ mati: ‘Magi. .Tatio Gerulal. C, ro. flan, 29.
tee Som detto Ismeno ,i Siriappellan Magoy .
3 Weis [oo ls Machewdedl arti incognite fon vago.
E) Sper ene ares i feptodacP lids Verg..l lib. a:cap, 33. & dist fpecie ,ciot Ne«
oC A » ¢Hydro-
fi = a ‘dettitancora ablegromantes ec, Vedi forto Cant, 2.

   
 
 
    
 
   
    
     

 
   
 

b » parrebbe ihe Benificaile Scelleraggine , © Scia-
‘ acne fi pigtia < 20k per Difgrazia..» Boccaccio Novella 36. La»
. della mia ena BN. 43. E-della [ua feiagura do-
; toe adel paredeer ne fe ne fervivano nello fteffo modo , che>
_ facciamo noi per intendere Big ‘Plaut..in Cape, Atsior poritus hoftinns wa
hoc eft fcelus ? Quaji'in orbuatem liberos produxerim, Ter. in Kun. Neque Que-
um effe ego hominem arbitrary cui magis bons Felicitates omnes aduer{e fiat. P

“of feeleris ? Il medefimo fignificato ba Ja voce latina — che a not
voce Sei. iagurate, .
  

 

 

20,  IMALMANTILE

CORAZZONE , Corazza grande, Armatura di petto ,¢ fchiene ; dal latino
Thorax , fi dice anche Petto a botta, perché ¢ a figura d’ una botta , o perché fi
prefume , che regga a una botta d’ archibufo .

‘MBOTTITO . Ripieno , ¢ trapuntato non di cotone,'9 altro fimile, ma d'in-
{ulti ¢ di branure , che vuol’ ingendere Incantato , come vedremo appreffo nell’
ottava 27,

‘SP.AeALDO . Huomo avventato ; Huomo inconfiderato , Dal latino /uperua~
fidus Soverchiamente ardito , ¢ quafi temerario , ¢ tutto impertinent .

See STANZA Xa4l, STANZA XXH.
Bellona cha il medefimo capriccio Ove doppo mofPrate ogni accidente

Di far bracinole , va col farracchino. Di tucca la fue vita pel paffato:y
Con il bordoneje'un bel barbon pofticcio , Seggiungehe per viad'un fuaparente
Sembrando un venerabil peliegrino ; Jn breve tempo riaura ta fhato ;
E fatto di parole un gran paftriccio Peri fi metea in arme,ch’ un prefente:
Effer dicendo aftrologe , ¢ indovino, Le fad'um panceron, che ancorche ufato
Che vien di quel difcofto pik lantana Ripara i colpi ben per eccellenza y
La ventura le fa fopr’ alla mano ; E poi piglia da lei grata licenza

Bellona va a trovar Celidora , ¢ fingendofi Aftrologo , le dice molte cofe-ocs
corfele per il paflato , per accreditarfi; poi le predice’, che fra poco tempo cella
riaura il fuo Stato , pero i metta in armi; ¢ ke dona la corazza incanrata.,"¢ fi

arte.

CeAPRICCIO , E Penfiero, fantafia, volonta., come intende anche forto ©, 6,
flan, 101. E per altro capriccio fignifica quello , che i Latini dicono orrere , che
quando i peli s’ arricciano;il che fegue @ per lo freddo , 0 per qualche {ubito fpa-
vento , © ne i cafi di febbre, come s’ intende fotto C, 6. flan. 14.¢ C, 20. flan. 2,
ners 101 habbiamo il yerbo accapricciare, che vuol dire Havere {pavento. Dan+
te Inf. C22. y aK

Lo viddi, ed anche il cor men’ accapriccia ,”

BRACIVOLE , Si dicono guelle ferte cobain di carne di porco , o d’ altro
animale , che fono ¢osi tagliate per cuocerle fopr’ alla bracie , ¢ perd dette bray
cixole , Ma qui intende fette d’ huomini , ¢ vuol dire che Bellona hayea la mede+
fiusa volonra di far guerra , che hayeva Marte. =

SARROCC HINO. Eun collarone di cuoio , il quale adattato al collo cuopre
tutte le {palle , ¢ buona parte delle braccia , ¢ petto a jadi Manteiio , ed &
u(ato da i Pellegrini, che vanno a piede.a i Iuoght fanti ; B quefti tali fo-
no da noi chiamati Pe/egrini corrottamente da Peregrint ; la qual yoce ¢ latina y¢
ritiene appreflo di noi gli ftetfi fignificati di fingolare , ¢ graziofo , ed anco di fo>
refticro , Peregrinus in domo patris mei, Petrarca Canc, 12. t ete
3 Moffe una Peliegrina il mio cor vano., geen
Et intende , che una graziofa:, ¢ bella donna moffe il fuo cuore, E Ja derta voce,
Sarrocchiny ceedo y che venga da San Rocco il quale portava forfe quefta parte
abito., quando ando peregrinando.il Mondo. LS
; SOKO . E nome particolare , ¢ proprio di quel baftone:, elie’ portano

cllegrini. we owdey

PAST RICC/O. Maffa confula di diverfe rcbe .. Qui vnoldire quantita di pa-
role mai’ ordinaic, : £ ee Dal:

  

  

3

Bait
Sa
bt

 
 

—=_-.1. ee) Ce nt TS meee, Pee Pee ee ees ns
eer 1
PRIMO CANTARE, 24 qe

© DAL difcofto piis lonrano , Pitt lontano della jontananza ‘tefla, come diremmo :
Vero pit del vero , o della fteffa verita . .
| FAR la ventura, Stcolagare . Sono alcune donnicciuole originarie d’ Egitto ,
de quali in Tofeana vengono il pit delle volte di Sicilia , ¢ fi chiamano Zingane ;
Quefte , dando a creder d’ efler perite di chiromanzia per bufcar denari , vanno
confiderando i lincamenti delle mani alle perfone, ¢ palefano ( dicono effe ) le cos
fe paffate , ¢ predicono le future : EB perché dilcorrono artifiziofamente con certi
Jor generali empre di bene ; effe chiamano ,ed anche da tutti noi vien detta que-
fla operazione ; Far /a ventura, ola buona ventura . r
PAKENTE . Intendiamo ogni forte d’ afini , o confanguinei in qualfifia gra-
do ; cosi ¢ intefo nel prefente iuogo , che vuol dire Baldone cugino diGelidora .
Cost I’ inavefe Dante nel Parad, C,6, , ¢ i] Petr, Sons 191. E fe bene ftretcamente
vuol dire il genitore , venendo dal latino Parens , ¢ ufato da noi in tal fenfo aflai
dirado., ¢forfe non mai fuor che nel numero del pill, come I’ uso Dante Inf.
be Le : ‘
———— Homo gid fui
E ti parenti miei furon Lombardi ,
Atantovani per Patria ambi dui ,
Ed il Petr, Canz. 29. :
é © ¢ Madre benigna ye pia
Che cuopri t uno , € 1’ altre mio parente,
* P-ANCERONE ,Antende quella gran corazza detta fopra in quefto C. flan 20.
ANCORCHE" Hae, Adoperato , Vecchio , Antico, * °°
PIGLIAR buona licenza, Pighiar commiato, Licenziarfi da uno per andarfene .
E-qaell' epiteca di duona , ograra s' aggiugne pér efprimere , che quel tale parte»
con buona grazia dell’ altro , ¢ con il di lui confenlo, ¢ hon forzato, o {eacciato .
» STANZA XXIII STANZA’ XXIV."
Gia il termine d’ ns anno era trafior[a Fece [palucce a Calcinaia, ¢ a Signa,

  
 

 

 
    
 
 

 
   

 
     
     
 

 

» Che Celidora haves perduto il Regno ; Ma la pania al [uo folio non tenne ,
+ Quando né pur le/piacqueilcafooccorfo, Perche terren nun v'era da por vigna ;
» Adavolle un tratto acor moftrarne fegno, Calo nel piano ,e ad Arno fe ne venne,
> Percio rithiefio ai connicin foccorfo , Ove Baldon facta nella Sardiena
fatto ns haurian col pegno, Vele fpiegare , ¢inalberar’ antenne

   
 
 
 
 
  
 

Fermato havendo ts come buon fito
D! armati legni un numero infinito ,
erdita dello Stato di Celidora , dice ,
I ja patiato quando Ja medefima comincid ad haver penfiero di
ricuperarlo, ¢ per cid fare , richiele foccorfo a diverfi vicini , ma fenza frutto; la
_ onde fi rifolué di venirfene verfo Firenze , ¢ trové in fu la rivad’ Arno in uns
con una buona armata, ‘ ,

“voce tratto ha molti fignificati dicendofi rracri di
» che fi daa i delinquenti’ nel martirio delia corda .
ira ‘mo Queili uicimi moti, che fanno i moribondi neil efaiar lo
irito . Trarto fi dice in vece di eftratto , cavato , o dedotto , ec, Tratro val per
tanza,dicendoli tratto di tempos twatto di via, ¢ fimili, Trateo. dicortefia per”
i Atco

 

 
  
   

   
  

   

    
 

 

 

 

2, MALMAN TILE 9
Ato di cortefia,7‘ratto per manicra, Edin mate tuoge: ioe Peas &

il latino tandem aliquando

VN piacer fatto non haurian col pegno . Ss lees Vacs chemo
a yeruno:, eziam fe li fufle,daco-il pegno ia'mano.

TENER il.fuo in rifpiarme, Venere il fao ate ,econ riguada s > taal dicono
r ifparmio 2 ri/parmiare,,

GIVSTO , Quefto termine fignifica Perl? appunto.

ERA come leccar marmo,, Bravana ogni ist per. appanto;come vaniti
Tecear’ il marmo .

»FECE fpallucce . Si raccomandd., Quefto detto seas dai poverelli’, che per.
muovere a compaflione in domandando-!' elemofina , fanno tutte le fmorfic,» e»
gclti, che fanno, ¢ podiono , ¢ fra gli altriil pia comune i Fare /pallucce 5 Siok
StringerJe fpalle alla, volea del collo..

LA pania non tenne, Non fece cofa di buono, ciot non hebbe ainto da, colora,
dy quali lo {perava ; inteadendofi con quefto dettato,che quel tale, che fu richie-
flo , von adempi il volere di chi lo richiefe ; cite diciamo ancora: Vax.ba trovata
appicco. 1 Latiai pure ia quefto propofico ditiero Evannerunt infidia, Rania inten-
diamo il vifco , col quale fi pigliano gli uccelliy, B diciamo dom tenere quando 5
© per il molle; o per altro la pania non appicca , ne li prendgne son) ae LA

AL {uo felito , Secondo il {ao coftame , Dice al fao (olito-per-dimoftrare » che
‘in quei paefi era da fperar poco bene al folito , perch? mon v' é terreno da por viene,
che.vuol dire: Non ¢ da far fondamento so da [perare da loro favore alcuno, ¢
scherza con I’ equivoco del parre wigne,, perché verameaze quei paefi non ‘hanno
terreni buoni a poryite-viti..

C.ALO’ nel piano. Scefe;ne) plano , perch’, Calvinaia., ¢ Signaifono ncaa
cOllinette vicin¢ ad Arno...

OVE Baldon faces nella Sardigna . LD Autore y che vnol fears flare i in fa i;

 

 

* durle., ¢ (eruirfi dello {cherzo degli equivoci , fa che Celidora crovi Baldone nella

Sardigna; ¢ pare che voglia dire I’ lola di Sardigna, eddatende dium logo fuo~
ridelle mura di Firenze in fa la riva d’ Arno , cost detto per il fetorey: che.quivi.
fem, re fi fente acaufa delle-beftic del pit condo y che morte fi fanno in quel luo~
B0 fcorticare : ¢ tal nome viene dayi Latini ; che chiamavano; Sardinia. queisiuo-
ghi, li quali per li mali odori fono fotopot. »all’infezione dell’.ariayy. come é I
Fan di Sardigna , la quale per Aayere da Settentrione monti altifimi , che ie im-
‘pedifcona i venti, ¢ fe riraacaaee aria »,¢forropolta alla-peftilenza. Di qui
ancora [i noftri Medici ban ‘hanno peers nome di Sardigna a quel:luogo, acila ‘Spe-
es é Santa, Maria Nuova di dove fi mettono gi infermi te ferenti-
laghe , o-altro fe ip ear riva.d’ Arno chiamataSardigna,, fi — -
i s etait: 2) fi rigaricano, iNavili, che da) Livorno vengono a, Fir
sper lo flume d? Arno, ¢ tali es » che qnivi.fon fempre.in gran

che fieno. Jrarmata di Te detto. oun i
Sere ceae ae pool Renths Bretho: eG ferve,

‘yore /ito per pafo', ed in eff . Mat’ oiore sche
ie cueceee yela eet
+ Qiello medefimo: 04

 

  

atone Sos ee elo
fachi_ angheanel

ad

?

 

 
Pee Se Te PE eee) PE MTGE gage Ti, oF

PRIMO CANTARE: 23° .
Situm cafprorum {econdo Ces, de bello Gallico , ed intendono’ ‘aheord puzzo fecon-’ |
do Plin. lib. 21, Peffimam a Crocum , quod. feng redoler, © | . 43
STANZA X “STANZA XXVI. ;
Coftui quando Bellona fu ii pane ' Roi che'pedoni egli hebbe , e gente'in fella te

A Celidora , come ge s*intefe Tanta ch’ al fin ff i chiame foddisfarro ,
~ Da Marte ibaveon wuta una fardata, Render volendo il Regno alla Sorelia ,

« (Che lo tenne balordo pitt d’ un mefe , E farle far bandiera di ricatto iS
E vli meffe una voglia sbardeliata Deftind muouer guerra a Bertinella,
Di far bactaglia, ¢ mille belle imprefe; Ch aleigiadatohavea la feaceo mateo ;
Ona’ egli entrato in fregola si fatta Cos} con quell’ armata , ¢ qnei difegni
Feée toccar tamburo a /pada tratta , | In Arno meffe i fopradderti legni .
Marte erd ftaco a trovar Baldone , conforme haveva detto alla Sorella ye I’ ha-
veva fatto rifoiuere a metterfi in arme per aiutare Célidora , ¢ rimettecla nello
Stato; € percid con quefta gente a tal fine s’ cra imbarcato .

FARDAT-A, Percotla data con un pannaccio intinto in fporcizia ; perché
farda yuo) dire fornacchio , che ¢ Vin grande {puto catarrofo. Vedi forto in que-
fto Cant. flanza 47. E's’ intende ancora per Vna quantita di fporcizia bicumino-
fa yche tirata in qualche luogo s" appicchi,¢ s’ interai in quel Juogo dove é butta-
ta , come farebbe una manata di fango , 0 altro fimile buctato in un muro ; Dal
che per metafora intende in quello: luogo per Vin ay  ohe s’ appicchi, ¢ s*in-
terni, quella perfuafione , che Marte haveva fatto a Baldone di far guerra .

* BALORDO., Quolta voce che vuol dir fonavvertito , Smemorato , che é il
latino mente ¢ peas 5 ci ferue per intendere D’ tino’, che j per qualche accidentes
ito , enon fappia a qual partito appigliarfi, per rime-
eare al danno'che: da-quello accidénte gli rene 3 ¢ fi dice anche Shalordito , Stor~
Vedi fotto C. 11, flan. 25.
cebubbann LLATO a coia che leceede i termini del naturale , ed in-un certo
modo : che fi dice) Grande, pili grande j grandiffimo , ¢
ja’, ¢!poco‘ufata ; Bi forle meglio Diforbicante; o Ya=
moderato s ss juonaad lo ftetfo:. L#Autore del Capitalo i in ate de” peducei eer
7 (cde to cingueihore del giorno in mercato. °
<- Apafeer gli occhi di si belP oggertal, 2 96 bo BEM i
Von) \\9 Ewe cavowun pitcere shardeltato,
ifBBOO Voglia grande . Onde viol dire Entrata wnifreeit # fatta jatende
Effendogli venuta cosi gran ee  traslato dai pefci ¥ che fidice Andare ix ‘
quando’s' adunano molti inficme per la gentrazione ; ed é il latino Jibi
eae eae gatti ,quando | (0 in’amore / Veli forto Cant.,
mh G unGevt 0, RAL
\ Vuol dir siiadice aacblehttn 'sintende ‘Aruolare.s
dati 5: caffe Vedi Corto CG, 3 Ran. 56. ©
A fpada: ipofo%, Senza intrmulione 5 i Senate
' t ie
EAR bandera pane Ricactat Vendicart # Quetta voce Ricatto 3 hed
enidal verbo Ricatcar/ uO! propriamente dire Liberarfi di {chiavitu-
cf dg een watery ved © il Latino par par

referee

      

   
     
 

    

 
  
 
 

     
 
 
 
 
 

 

 
    
      
   
     
   

  

ae

 

   
: MALMANTILE :

; 24
y roferre. Ul dettato Far bandiera‘ di ricatto timo che venga dal coftime dei Corfa=
ri, li quali , quando pigliano qualche legno , che ftiminu d’ flere in. gradoda ef
fer ricattato ,, v’ inalborano una bandiera bianca , con Ja quale,danao cepno alle
Terre vicine {¢ lo vogliono ricattace ; i] che fe voglion fare , corrifpondono.¢on
alzar bardiera dello fteflo colore.; ¢ quefto dicono Metter bandiera di ricarto .
DATO havea lo fcacco matte, Le havea fatto quefto dango.5.0 cagionata-que-
fla rovina . Li giuoco delli (cacchi é antico , ¢ fu ufato prima da i Greoj., ches
oralo dicono Zarrici , ¢ poi feguitato da i Latini, che lo diflero Ludug latrun- ;
culorum,, A quetto giuoco fi da fine quando ¢ faito prigione i Re » ¢ fi dice allo- f
ra feacco matto ; onde qui vuol dice , che Celidora havea toccaro Scsccomarto, ha-
vendo perduto il fuo Regn : Es’ allarga quello detto a tutto quello, che adal-
tri fucceda di gran perdita , 0 di grave danno . ey
STANZA XXVIL,

Ou anco in breve Celidora.arriva Chel’ usbérgo incantato della diva
Con armi.in doffo,ed altro da far fete, Liha fasto diventar | Ammarxafette,
Perche una volta al fin fatrafi viva Ed alle riffe incitalatalmente ,
Ha rifaluto far le fue vendette ; Ch' ella pixzica poi dell infolente . 4

Celidora arriva all’ armata di Baldone nella Sardigaa , ¢.quivi comincia a. mo~
firare gli effetti della Corazza incantara . YO
é eh da far ferte. Intende Ja {pada , ¢ vuol dire che ¢ra larga, edrabile a»
far fetce . i 5
FATT AS! viva, Rifentitafi, ¢ fattafi ardita., E lo fteflo che P7cir di-garra
morta detto fopra in quefto Cant. fan, 19. : ,
YSBERGO, Ciok quella Gran,corazza di pelle.di drago: detta fopra,lacquale ib
Poeta qui dichiara , che ha intelo y ivcansata quando ha derto:fopra smbetrita a
infulti , e di bravure alla ftan. 20.
3 AMMAZZA fete, Contano le donne una novella per trattenimento.de’Fan-
ciulli ; ¢ per accomodarfi alla loro capacita 5 dicono::, Fuyna volta un beligiova»:
netto in Garfagnana detto Nanni, il quale per la fua:mendicita dormiva in una.
capanna da fieno ; quivi eflendo egli un giorno per, ripolarfi, ¢ ripararfidal cals
do, fi meffe a pigliar le mofche ,'¢ ne -haveva ammazzate fette, quando com-
arue quivi una bella Fata,e gli diffe; che fe le donaya quelle {ette mofche per ci-
ire una fua paflera, ? haurebbe fatto ricco, Gliele.concefle egli pili che volen-
tieri ; ond’ ella innamorata diquefta, fua cortele prontezza loprefe pera. mano, !
¢ lo'conduffe alla fua caverna, dove riveftitolo , ¢ datogli danari), ed apmi., gli
ofe in tefta un’ elmo, o bs cra lcritta@lettered! Ora, 2 lommanedn
‘tte; ¢ lo mando al Campo de’ Pifaniyi quali in quel:tempo. con lt-aiuto de Frans
zefi guerreggiavano co i Fiorentini'. Arrivato Nanni a “detto Campos:chieles
foldo i Pilani,edon i err te ena eal
: io folo in un giorno am es « foprannome: dmmazxa/fette, Bu per
Pa Se< quefto;¢ eee anche ben orcas buon {olde .¢: ‘con: noniminore fi-
2 ma accettato. Effendo poi fra pochi giorni in una {caramuccia ‘mortal Capo
‘delle truppe Franzefi , ¢ volendone effi fare un alteo,, crano fra didorain gran,
‘diflerenza , perché eflendone propolti diverfi , coloro , a’ quali.‘nom/piacevano. i 4
: fey edenee 2
ro,
.

 

 

   

    
 
 
 

exti propo gridavano Waniy Veni, onde i

 

 

 

i ae Fa a ii i

 
PRIMO CANTARR: as

0, che diceffero Nanni , Nanni, e che haveflero creato lui: cOminciarono as
gridar Nanni, Nanni; viva Nanni; ¢ cosi a voce di popolo Nanni detto ! Am.
mazzafette reftd eletto capo di dette trappe, ¢ divenne ricco, fi come gli haveva,

romefio la Fata. E di quefto intende ij Poeta , volendo moftrare , che Celido-
ra era divenuta brava , quanto queflo Ammazzafette , il quale non fece mag-
gior bravura’ , che ammazzar quelle fette mofche , fi come ne anche Celido-
f Ta non fece maggior bravura , che affettar quei Cavoli, che vedremo nell’ ottava
29. feguente . r
ALEE riffe incitald talmente , cl? ella pizita dt infolente , Bellona le fa venir vo-
glia cosi grande di far rifle, che ella vien poi a noia, ¢ fi rende odiofa con i fuoi
modi impertinenti. 11 verbo Pizicare vuol dire Cominciare a eflere , 0 Efleres
alquanto . // tale ¢ ftato tanto tempo in Firenze, ch’ atta di Fiorentino, Lo trovo
anche ufato da i Bolognefi in quefto fenfo , ¢ I’ usd Francefco Negri nel {uo Taffo
in lingua Bolognefé Cant. r, ftan. dove El pizigava di fei ann’ ch'i Tramuntan ,
ec. per intendere , Bra gia prefio a fei anni , ec.

INSOLENTE , Si dice colui che da faftidio , ¢ noia a ognuno , € che firende
ediofo a tutti con Ie fue azioni impertinenti .
STANZA XXVIII, STANZA XXIX.

Non cosi tofto al campo fi conduce , Se guarda , ¢ difpertofa , ¢ impertinente ,
‘Come la fuora viol del Dio Soldato , Efempre vual che fia la fu di fopra ;
La Marfifa di nuovo pofta th lice , Talor’ affronta per la via In gente
hell’ efce afarto fuor del ferminaro ; Cercando litt , quafi franchi  opra :
col brando che taglia ,com*ei cuce We venga ( dice ) pur chi vuol niente ,
ja far proprio morire un difperato, Pero che,chi mi da che far mi {eiopra;
‘nol trucidar’ ognunoognun vuol mort, Giunta in queft in un capo pien di cavoli
_ Eguas a quello , che (a guarda torto, WV afferto tanti, che Beati Pavoli .

Defective il Poeta una brava (propoficata , ¢ impertinente, per moftrare in Ce-
~ Jidora gli effetti del? incantata Corazza ; ¢ con quefte azioni , che le fa fave, dipi-
gne al vivo'wno di quefti fpacconi , ¢ ammazzatori , che noi diciamo che Cam-
“pano di fegati d’ huomtni , ¢ fon poi il ritratto della poltroneria , ¢ sfogano la,
“ for bravura comé fa Celidora , in un campo di Cavoli. or
COME 1a fwora vuol del Dio foldato , Come vuol la forella di Marte , Bellona. ,
per opra della quale Celidora ¢ capirata a quel campo
_ MARFISe4. Donna guerriera nota , favoleggiata dall’ Ariofto, e perd la di-
_ C0: di mnovo pofta im /uce, ed intende una Marfifa moderna fatta brava da Bellona,
~'cioé Celidora . , ; ov
=) VSCFR del feminato affarto, Perder’ il (enno del tutto , Fmpazire . Quando al-
i per un grandifimo contento fi railegra pit del dovuto , diciamo: II tale #m-
¢ per V allegretiza ; € cosi intende di Celidora , non che veramente fia im-
ita. { Latini hanno il verbo devirare , che vuol dire Impazire , ed é¢ metato-
ico ‘dal bifo! i com dalla prepofizione De , clie fuona extra, © li-
rare , che vuol dir Fare i tolchi nel campo con I’ aratro ; econ quefto fol verbo

lelirare intendono extra liram incedere , dove noi diciamo Vicir del {eminato, che
blo fteflo che e-xrra liram incedere , 0 delirare , del qual verbo ci feryiamo ancor
‘nei medefimo {calo,come fi vede in Dante . Nt ba

: eet

 
 
 
 
 
   
   
     
  
  

ee

a

   
   
   
 
   
    
  
 
   
 
     
 
 

   

   
      
    
      

Pat.

  
26 MALMAN TILE

Ed egli a me ; perché tanto delira
Hoegi l' ingeguo fuoda quel che fuole , ;
E fi dice anche dedivo uno ,. che fia fuori del fenno, Dan. Par. C, 1.
Che madre fa fopr' al figlinol deliro ,

Alcuni vogliono , che.quefto vesborDesirare venga dal Greco, Lirin , che vuol
dir (cioecheggiare . Diciamo nel medefimo fignificato Vjeire de/ feminario, E que-
fio forfe deriva dal Latino Seminarixm , che {econdo Colum, lib, 1. de arboribus
c. 1. 3. vuol dit quel luogo , nel quale fi feminano le piante per trapiantarle,il che
quando fegue,la pianta cavata.dal.detto Seminario refta come un pefce: fuor dell’
acqua , € piantata poi ripigiia il vigore,quando ha cominciato ad attaccarfi nella
nuova terra ; € da quello, dicendofi huomo fuori del Seminario s: intende' Huomo
sbalordito . Si dice ancora fuoridel fecolo , ¢ habbiamo rrafecolara, ed ii verbo
Strafecolare, Vedi forto Cant, 6, ftan.36. pur tutto.a quefto propofito .. Ma si que-
fto , come gli altri fuddetti termini , con cutco che poflano crederfi I’ accennate»
derivaziont, io ftimo che intanto s’ufino inquefto propofito , in quanto hanno il
principio della parola , che fomiglia quello della parola /enma ; e-che fidica fuori
del Seminato , Seminario , 0 Secolo in vece di dire Fuori del fenno ., E quefta {pecie
di parlare , che é {pecie diyparlar furbe(to, é molto ufarorin Firenze per {cherzo ,
¢ lo.dicono parlare /anadaitwo , il qual parlare rielce aflai graziofo,quando é ma-
neggiato da perfone (piritofe ,. perché taluolta con parole , chenon hanno ches
fare con quelia materia , della quale fi difcorre , vien defcritta per allufioni , 6
per metafore , 6 altrimenti quella tal cofa,della quale fi parla. Per efempio; Ad
un Priore , il quale a tre mogli ». che haveva havuto , non-hebbe mai figiinoli, ed
havea nome Antonio,di Priapo annebbiato, Adan Propofto ,.che haycas
nome Girolamo , ed era lungo, fecco, ¢ di colore olivaftro,dicevano; Pro/eintto gi-
rato. Di quefto parlar' lanadattico fi ferue fotto C, 9. flan. 1, "

T AG LLA come ci cuce, Tanto ¢ buono a tagliare , to buono a cucire, che
vuol dir:non taglia. Detto ufatitiimo per intender Ogni forte di coltello , o.ar-
me , O forbice , che per la ruggine , o altro non fieno atte a tagliare .

FAR morire ux difperato. Dicono che le ferite fatte con i ferri rugginofi, 6 in-
taccati , fieno pericolofe di cagionare {pafimo, ¢ percid do fi vede un coltel-
lo , o arme di tal forte , fi fuol dire Farebbe morire. uno ai} roe » cioe.didolori ec~
ceflivi , 0 di fpafimo , E tale era la {pada , 0 brando di lidora x :

GVA a quello, Male , 0 gran difgrazia avverrebbe a colui 5.chelaguardaffe
torto. Bil Latino Ye id, )

GVARDA torto, Quand’ uno non é molto noftro amico,diciamo ;-Zi,tale nen,
smi vede con buon'occhio ; 0-yero mi guards torte, Che iLatinipure diconoVen re-
is afpicere oculis , sgetier

ee ETTSSA » Huomo altero , ¢ chedifprezza,ognuno , ed‘ ogni piccala,
cofa.s’ adira . i a, q

IMPERTINENTE . Vno che vuol pid del fao-dovere , 0 del giufto; »\o;pit di_
quel che gli s*appartiene ; scot a:

VVOL che ls [ua y ftia,fempre di fopra , Waol fempre haver ragione.» che-fi dice
anche Sopraftante , E queftt tre modi cioé een » Sapertinente 5 eSopraftante
fi pofion dire Sinonimi, ¢ fignificanti Huomo: Pane. Gectaeampea OTRO ES ES °

i iper-

 

 

 

 
PRIMO CANTARE: 1

fuperbia , compagna indivifibile di tutti: gli Sgherri’, € bravazoni a creden-
za

i

f AFFRONT ARE , Vuol propriamente dire Affaltare il némico , ma fi piglia
‘s ancora per Andar’ incontro , o Affacciarfi a uno per’parlargli , ¢ cosi ¢ prefo nel
prefente luogo , per intendere che Celidora cercava {propofitatamente I" occafio-
ne di far quiftione , ¢ tutto per defcriverla fimile a i detti bravi di parole.

CHI mi da che far mi feiopra , Dourebbe dire Mi {ciopera’, fecondo che'da al-
cuni troppo delicati',.¢ punto confiderati ne fu'avvertito il Poeta, ma Ja figura
Sincopé-( ammefla fra i Latini) Verg. 5.nedice gubernacio in vece di gubernacule
da noi € accettata anche:nella profa, cd adopratacomanemente in molre voci ,
particolarmente in quetta;dicendofi pili [peflo Opra’, Adoprare , Scioprare , cheo
Opera, Adoperare, ¢ Scioperare, lo libera da-quefta ceafura., EB quefto termine Chi
mida che far mi {ciopra proprio di certi Taglia cantoni', che yoglion con effo
moftrare , chechi'da loro vccafione di far quiltione gli /ciopera , cio li leva dal
farne up’ altea',, che han tra mano , ¢ Ji leva da un lavoro per impiegargli in un’

altro fimile .
WN APPETT OP tami , che Beari Pavoli’, / Ne“taglid in fette grandiffimo nume-
ro. Quando. vogli beftare. amb: dardo, fog) dire; Gran,

 

danno che farebbe:cuftui imun’ orto di cavoli,odiraduchi , E quel detto Zeati Pavoli,
ha origine da un Mvntanbanco , il quale vendeva il rimedio contro a’ veleni con
dichiarazione-di- voler-donare ( come effettivamente donava ) la pietra di $.Pao-
lo-atutticoloro,, che havevano nome Paolo, onde infiaiti plebei per bufcar quel-
Ja pictra dicevanordi haver nome Paolo; ficch® egli comincid ad efclamares .
© quanti Paol, orquanti Paoli. E perché quelli , che ottenevano quella pietra
i tenevano fortunaciper haver’ havuto i) regalo’, ne nacque il dettato . Sow pik
che'non furonoi Paoli Beati, che vuol dire , furom moltiffimi ; Che la voce Bears in
| eget @finonimo della voce felice’, o forcuaato , Beso voi che fiete ricco, per

clice , o Fortunato voi , che fire ricco . <

STANZA XXX.

 

Cost pienadi fumi , ed umor bravi Eva per infilzarne [ette otravi :
- Chetet! hanno cavata dt Calende, » Ma nel penfar di poi,che fe gti ofendé
} Rivolze I occhio at popal delle navi , Far no porrebbe lor , fe non mal ginoco;
$ La dove Brefcia romoreggia, ¢{piende, Gli-vnol lafciar cainpare unt altro poco,

_ Celidora facendo queitefue bizzarrie y vede la gente di Baldone , ed efiendofi
inferocita/in'quei cavoli', gli vien voglia di far’ io fefio in quelle genti , ma fi
ractien di fario pet non dar loro difgufto, ¢ per la(ciargli campare un’altro poco:
PIEN Adi fumiche'tel’ hanno cavata di Calende . Moftra il Poeta, che Cell-
dora fia meno , che briaca in quefla (ua bravura-, i fumi della quale le hab-
Bias odlafeapoil ii\ceruélio , come fanno i fumi del vino a chi troppo beve , ches
quefto incendedi¢endo'l hanno:cavata di calende , ed @ quelio che i Latini dicono
extravcailem efeyedio credo:cheda’ Jatino callem venga la corructela di ca-
lende ;.¢ per»parlaellanadattico detto (opra in quefto G. flan, 28. fi voglia di'ca-
vara del calle per invendére ( come facevano i latini )’Cavata di Ceruelio .
_ . BRESLTA vomsrexgia, efplende. Si (ence romor d atmi , ¢ & vedono rifplen-
der i¢ medctiin:. A Breicia fefabbricand buone , ebelic armi, ¢ perd 1] Poca
x Di pigiian

ponte

   
   
  
 
     

 
shies

 

28 MALMANTILE,
aro
5 che:

pigliando Ia Cited per J'armi , che in quella fi fabbricano, feguita-T' nfo-noftro; che:
€ di dire tale ha tutto Brefcia addoffo , per intendere Hs moit'armi addoff.

STANZA XXXL STANZAXKXA&IL.

Al fin , depoffo un’ animo si fiero , ‘S' abbocca appunto con Baidone fheffo's .
Un genio cangia a poco a poco l' ira, E fentendo ch  egliha tai gente fatte
E’ come un’ orfacchin.,c’ apie d' un pero Per rimeiter in fefto , ed in poffeffo
wd bocca aperta i pomi fuoi rimira; Via Cugina {ua ch’ e-perie fratte ,
Ferma impalata quevi'com' un cero Ben belo (quadra,edice:Egli¢ pur defsol
Fif[ado in loro il (guardo, fuiene,e/pira, Or fu ch’ 10 cafco in pic, come le gatees
LVe puo viver al fin fe non domanda Ed efclama dipoi : queft! € un’ aRione 5,
Ove ? armata vada, e chi comanda . Che veramente ¢ degna:di Baldone,

Celidora pero appiacevotitadi , fi ferma a guardar con gufto grandiimo quei
Soldau , ¢ domanda di chi é P-Armata , ¢ chi la comanda ; e s'abbatte a doman-
darlo a Baldone , il quale gli dice , che ha facto quella gente per aiutare una fua
cugina , ond’ ella riconofciuco Baldone , fi rallegra , ¢ dice: veramente quefia ¢
un’ azione degna di Baldone .

Ce4NGLA I ira in genio. Cioe dove prima haveva l animo @’ infilarne fert’
ottavi , adeflo comincia ad haver genio con loro , ed a portargli aftetco.. Quefta
voce genio fe-ben non pare che Tofcanamente fignifichi cofa alcuna , nondimeno
€ moito ufata dicendofi Auomo di buon gemo , o di cattive genio per intendere Hue-
mo di buona , 0 cattiva indole, o inclinazione. Haver genio con uno E: jo ftetio
che Haver fimpatia con uno. Appreflo i Latini pure fe-ben genio non fi-diftingue-.
va dall’ anima ragionevole , ¢ molti lo pigliaficro {peflo per Lares; altri per ght
Dei Penati-, altri per il Dio-del piacere, altri per li quattro elementi, altri per lt
dodici fegni del Zodiaco , altri per lo Dio che faceva nalcere y ¢d altri per diver-
fe altre cofe ; tuttavia efli pure fe ne feruiuano per intendere inclinazione , — ;

' ¢imoftra Plauto in Truculento.1, 2. cam genijs fuis belligerare 4 ec. idem 4

defraudare geninm . : 1
COME un’ orfacchino a pit d’ unipero, Si dice L’ orfo fogna pere ; Leva le peres
ecco l’ orfo, Dal-che fi cava , che quefto animale fia molto ghiotto delle pere ; il
be anche atiefta Vincenzo Martelli nel fuo Capitolo in lode delle menzognes
jicendo : ‘ A “
Oggi i voi pitt ch’ advaltri fi connieney ;
si eati Benché noi fiam tant’ orf a qiefte yidepite i. < t gin
E fi dice che in rimirarle gioifca tutco cre ola tperene di confeguifle ;
percio |’ Autore afomiglia Celidora aun picciolo Orfo a piéd’un pero,perché in
yeder quella gente , la quale ella (pera che fia-per-lei, fi rallegra, gode, ¢ brilla,
‘come fa |’ orlo ftando a pié del pero,vagheggiando le pere. a
PERMA impalata quivi come un cero, Per efprimere la fupidita nella —
‘trova Celidora nel vedere quei Soldati , 1’ Autore dopo haver detto che , ae
bocca aperta come frat orfo a pic del pero, foggiunge che ella fava impalata, come
‘cero , cioe ritta rita , ¢ nel polto , come favano quelle torrette, fate di
‘carta, 0 di-panno, 0 di tavole, che la mattina di S. Gio; mettevano | ri
antichi attorno alla piaza del Tempio di S. Gio: Batifta , entro alle quali ftava

- mn’ huomo., che le-moycva, ¢ quefte le domandayano corr lengegmnaprenie Bore

tt
oe

 

 

Be is
PRIMO CANTARE:

22

Dati ‘nei fuoi difcorfi Storici lib. 6. ia fine . Hoggi in vece di tali torretee»portaao
in due , dello Spedale de) Bigallo ; fopr’ alle {palle proceffionalmente,uno sgabel-
i Jone , fopr’ al quale ¢ fermato un gran cero fatto di legno , per sfuggire il peri-
‘ celo di romperlo fendo di cera , ¢ faranno 26. 0 vero trenta ceri, che mandas
detto Spedale per tributo al detto Tempio di S. Gio; Batifla. Si pud anche de-
durre quefta fimilitudine da quei peveri Criftiani , i quali da i Turchi fono impa-
Tati , che verifimilmente ftanno intivizzatiy ¢-come |’.Autore vuol.che s*intenda ,

{ che Aefle Gelidora:, me hf
} SVIENE,, ¢ fpira , Svenire vuol dit Perdere ifentimenti , ¢ Spirare'vuol dire»
Efalar I’ anima, ficché fi poflono dir quai finonimi , ma in quelto luogo:il verbo
ii fignifica Volare, che vuol dir Guardare con defiderio di confeguire ,come
uno che havendo grandiffima fame., ftia a vedere un che mangi , ed -habbia.

d’-avanti mole vivande ; Vedi forto C. 14, ftan. 34,

~A8BO°C ARST, Trovarfi, o abbatterfi in uno per parlargli. Zo non fon Lew’
snformaro di quefto negoxio , ma m? abbocchero col tale , che m' informera.

E’ per le fravce . EB’ rovinato . -E’ per la mala . Quello che i latini differo D:
eocattumeft. Pratta. 8’ intendeBorroncello., 0 Macchia, che fuol render’ afpro
un pacfe , ¢ vien dal Greco Brattin che fuona Far fiepe-.

BEN ben lo-fquadra , Lo guarda benitlimo.,che la forza della-replica & di far
na(cere il fuperlauvo , come accennammo fopra in quefto C. ftan. 11. Ed il ver-
| bo /quadrare , che vuol dir Mifurar con la {quadra , fignifica Confiderare , >

Guardare un’ oggetto minytamente , ¢.con diligenza .

_ CASC ARE ia pie come'i gatei, Orcener da un male ,o da un-cattive acciden-
~ te ,un-bene impeniato; che i latini dullero excidere extra mala,

j STANZA XXXII STANZA XXXIV.
} Maravigliato aliorail Sir a’ Kgnano-, S? ell! ¢ ( dic’ ei.) cost noi fiam cuginiy
< E chi fer (difse) rusche fai il mio nome? E fubito fi fan centro accoglienze y
Mo ti conofco gia ds lunga mano , Ed ella a hii ne vende mili’ inchini-,
(Ellari{pofe) ¢ acto tu fappia il come, Egli altrectante a ei fa riverenze ,
Celidora fon’ 10 del Re Fioriano Cost fanno talor due fantoccini
. Fratello d’ Amadigi di Beipome , Al fuon di cornamufa per Firenze ,
E con tutto, che gid fien’ anni Domini Che luna incontroaltaltroandar fivede
Ch! io non ti viddi, fo come ti nomini , Mofo da un filche tié,chi fuonayal piede

Baldone ,'¢ Celidora fi riconofcono per cugini , ¢ G fanno molte accoglienze.,

CONOSCER ds-lunga mano. Conoicer di gran tempo. Langa mano: ad anni
tanto fuona quanto Lunga ferie d’ anni, o gran quantita d’anai., che diciamo
anche 2’ um gran pexxo ch’ io ti conofco., :

BALDONE , Celidora , eaAmadigi {ono nomi a-cafo:, ma I Lnfame Floriano &
anagrammatico , da Raffacllo Fantoni .

SON’ anni Bomini . Son’ anni infiniti. Sono tanti anni , quanti feno dalla na-
__ {cita di Noftro Signore-che diciamo Aano Domini.’ iperbole ufatisima in Fix
i enz

 
 
  
  
  
    

 

ce
_ ACCOGLIENZeA1. Ricevimento con amorevolezza,,¢ cortefia., ¢ con unas
‘eerta dimoftrazione d’ affetto , che s' ufa ver(o le perfone grate. View dal Lati-
Colere.y che efprime Amar-con riverenza., ed honore~ Bis

 

AM

  

se

 
 

30 MALMANTILE

INC HINO . © lo fteflo:che'riverenza facendofi con abbaffar la tefta’, ¢ piega-
re le ginocchia , ed ¢ propriodelle Donne ; | Riverenza fi fa: con abbaflar la'te+
fta , ¢ piegandofi un (ol.ginocchio fi manda {"altra gamba addietro a foggia di
genufictiione , ed & propria-degli huomini , come fi vede nel prefente luogo , che
dice,

Ella aluine rende mille inchini;

‘ li altrettante alei fa riverenze,

COSI fanno talor due fancoccini, Suol’ andar per Firenze un -contadind , fuo+
nando una cornamufa , ¢ porta alcune figurine di legno’, che-hanno le congiun-
ture delle membra malticttate , ¢ cuntrappefate con piombo in modo, chef
muovono per ogni verfo ; quefte infilza per lo petto in-una fortilitima corda da’
chitarra, 0 diciamo miaugia,la quale da una parte lega ad uno de" uoi ginocchi 5.
¢ dal’ altra ad una tavoletta potta in terra a tal fine,e col muovere quella gamba,»
alla quale € legata la corda;ta,che quelle due figurine infilzutevi bailano ab tempo
del fuono della cornamufa . Lovefa dunque quefta operazione,che: fanno*i* due: fis:
gurini , s’ intende ancora come facetlero fra di loro quefti due'pareati .

CORNAMVS.A , Zampogaa doppia , compolta d’ un badopecpetuojedi uns>
foprano , che canta le note come gh altri Zufoli , ¢ fi'da it fiato ad ambeduercoa!
un facco di quoio., da colui che-luona’, ripieno di vento: coh ofhare:imun picco-
Jo cannello animeliato ; ed il {uonatore premendo col braccio il detto-facco:da il"
fiato.a dette due Zampogne .

STANZA XXXV,

Poi che le fratellanze , ei complimenti Har mitre,ch ella fenfia aduepalmenti
Furon finiti , a lei fece Baldone Pigtiando un pan di fedici a'boccone ,-
Quivi portar un po di feiacquadenti , St muove il campo,e [ore'alla uta infe
O volete chiamarla colezione, Ciafcun paffa per ordine araffegna,
Dopo finite le cirimonie Baldone fa portar da‘bere’, € amuglars > © mentre

che Celidora mangia , fi fa la moftra de’ Soldati, . .
EAR le frateilanze, EB tratto dal’ ufo chet nelle: noftre Compagnie , 6 Con-
fraternite di fecolari , neile quali a i tempi determinati fi vanno*tucti-ad abbrac-
ciace Juno con I altro ;.¢ quefta azione dicono Far Ie: frarelianze, E-da>quetto
dunque intendi dopo finiti gli abbracciamenti ¢ le cirimonie . Sis
SCLAC QV ADENT S, Quel che fignifichi lo dichiarail Poeta medefimo-dicen-
do; O volete chiamaria colazione ,. Che vuol dire parcamente cibarlifuor del defi-
pare’, e-della cena 5 ¢ viene-dal Latino collettio prandij vel cons. Ma titcome fon
diverfi li pafti che fi fanno in Firenze, cost fon diverfi li nomi che lorodanno. Lb

* primo mangiare che fi fa fra I’ alba , c il mezzo giorno fi chiama’ Afciolvere, ed

aile volte colanione . Quelio, che tifa amezzo giorno fixchiama de/fnare... Quel-
lo che fi fa tra ’l mezzo giorno , ¢€ la {era fidice Aferends quali meridieredemdes >
Quello deiia ferarfi dice cena , ed-altora’che per il digiuao Ja (cra: fi: mangia: poco
fr dice colazione; Blawoce:/ciacquademévual dire Quandolti :
qualche poco , per bere con guito .

ee

SCVFELARE, Mangiar comingordigia , o'divorare. E voce Fioreaeiaa yma

hog 3i ufata folo. per fcherzo.5¢ vien foric daSemjina che ¢ unairafpaovlima da
Jezn9 desta cosi , perch? adoprdadoja leva malo wgag pee obs mmgpeo pense
eddut

Cablaaid anche? i ordiae ,

 

S

*
ae

fics

 

 
PRIMO CANTARE,

 

ee

 

“sche vuol dir mezzo cieco , come vedemmo fopra in quefto Cant, ftanza 9, ; Io fa

 

   

3t

A due palmenti, Da ambeduele ganafce: Traslato dal. Molino , che fi dice»
Muavinare a due palmenti quando Due uote lavorano ; che palmento yuol dire tut-
ta la macchina , che fa macinare , dicendofi Molino @' «n paimento , 0 di due pal-

ementi ,quando Va molino ha una ,.0.due macini «| E fimo che fi dica Palmento ,
quafi Palamento , perché le ruoce , che fanno.andar, la maciac fon compote di
tavole a foggia di ;pale per prender |’,acqua , che le fa girare. :

VN pan di fedici , ec, Con quefta iperbole efprime I ingordigia di Celidora ;
perché peraltro un pane di fedici de’ noftri quatteini malamente.fi pud confum.~
re anche con fedici bocconi , intendendo Soccone quella quantita , che |’ huomo
pud pigliar dentro alla bocca in-una volta ,

P-ASScAR a raffegna , Quando i Soldati fi portano avanti, al loro Gapitano ,

Ȣ fanno (crivere il lor nome fidice Paffar a raffegna . Equi Baldone come fupre-
mo Capitau.o per fare honore alla cugina , Ba la raflegna , nominando, pero {0-
Jamente gli Vfiziali prificipali ; il che pare che pil propriamente fidica Dare , 0
far la mofira , Vedi {oro C, 2. ftan. 36.

STANZA XXXVL

E per il primo vienfene in campagna E 1a {ua [chiera aumerofa, e-magua,

+ PPappolone il Marchefe di Gubbiano , E perch’ egti ¢ Soldato vererano
Goluiebe nel conjlitto della Adagna Ha nell’ infegna una tagliente [pada ,
Eftinfe il Gallo, ¢ sepeel ti Germano; Ch’ein pegno all ofteria di mewza ftrads
L’ Aucore in qucita fua Opera mette una mano d’amici fuoi fotto nomi ana-

j Btammatici , 1a waggior parte de’ quali ¢ nominata in quefta moftra , che Baldo-
ehofa dell efercito 5 defcrivendone alcuni con qualche loro azione , &.con un’ epi-
logo delia loro vita oltre all’ Anagramma . i primo che viene inanoftra ¢ Pappo-
clone, ci0ePaoto- Pepi anagramma proprio , perch quefto gentilhuomo: era gio-
-wanouo.grande.di perfona , ¢ gratio , ¢ mangiava afiai; e fo il Poeta lo
sidice. me» che yuohdir gran mangiatore Vedi forto C. 6. ftan. 70. » ¢ lo fa
eddarchefe di Gubbiano , che ¢ un Cattelio ; ¢ /ngubbiare ( detto perd piebeo.) figoi-
fica Bupier il veotre Dice nel conjlirto della Afagna , cioe Nel mangiare , fe ben
par che wogiia direin.una fanguinofa battaglia (eguita in. Alemagna. Eftinfe il
Galo, efeppeili il Germano; pac che dica. ammazo Francefi , ¢ Tedelchi , ma vuol
pdire ch’ ci mangid-gaili ,¢ germani;¢ gli, fa fare per infegna una-{pada impe-
\gnata all ofte dijmezza ftrada , che ¢ un’ ofteria fuor di Firenze-un miglio,e cost
~mottra , che ogai fine di quelto tale era ij mangiare ,
k STANZA XXXVIL
Bieco de Crepi Duca d' Orbatello Son L-armi loro, it-boffolo ye il randelio,
Menailfuoterzocthaitveder nel ratto, Non tiran paga, reegonfi d'accatro,
Cut perch ei da an acchio fa afportello, Sofiano , fon di calca , erborfainolr,
Soldati ha picfoe’ hanno chinfo fatto, E nimici-mortal.de’ muricciuoli ,
Segue dopo Pappolone Bieco de Crepi , cio Piero de Becci-huomo di faccianon
bella , con occhi biechi,, ¢ lu(co , ¢ pero ii Poeca con iequivoco d’ orbo ,

«Duca a’ Orbatello, ¢ dice, che vedendo egli alquanto y ha prefo per Soldati gente,
“che é affatto cieca , avverando il detto.: Bears Advnocali in terra cecorum . Hanno
| queiti foldati Ul bofiolo , ¢ il baftone , non tirano.paga 5 ma vivono di: verte .

o fon

ie

   

      
  
 

2 MALMANTILE™

fen tutti fpie , ladri, monelii, e nimici de’ muricciuoli .

Vn terzo . Numero di foldati comandati da pil capitani , e dal Colonnello;che
i Latini dicevano /egionem , ed il Colonnello forfe era Tribunus,

MEN ARE , Condurre , Ma qui fla proprio il verbo Menare fecondo il pro-
verbio che dice: Solo tciechi fi menano, —~ ;

Ha il veder nel tateo, U ciechi non hanno altra vifta , che il tatto , el odorato

nelle cofe corporee , ¢ materiali ; ¢ 1’ udito nell’ incorporee .
. ST Aa fpertello. Intende mezzo cieco . Metafora tolta da quelle botteghe ; le
gualt quando non é fefta intera , ¢ comandata ftanno mezze aperte , che fi dices
Star’ a (ported , perché aprono folo quella parte del legname, che fi chiama fe
tello ; ¢ feguita la metafora dicendo : Su/dati ha prefo channo.chiufo affarto : ciot {0-
no affatto ciechi. Varchi ftor. Hior. lib. 11. dice : Won fi tennero le borteghe Aperte,
ne a fportello , ma chinfe affatto, 4 j :

ZOSSOLO , E' quel valoa foggia di calice , col quale fi raccolgono i voti ne-
gli Squittini. Vedi fotto Cant. 6.ftan, 109. , ¢ per la fimilivudine intendiamo quel
valo di latta , di rame , d’ ottone , 0 d’ aitra materia , che ¢ ufato da i ciechi per
ricevervi I’ clemofine , ay

RANDELLO . Inrende Quel baflone , che adoprano i ciechi per farfila ftra-
da. Se ben randello s'mmtende un Pezzo ci baftone grofio quanto quello de’ciechi,
ma aflai pil corto , che s' adopra per firingere le legature delie baile , che perd
tale operazione fi dice edrrandeliare .

REGGONS! d’ accatto, 1 verbo Reggerfi in quefto laogo , ed in quefti termini
vuol dir Cavar il guadagno per mantenerfi: M tale fi regge cal far’ il farto, Cive
vive col guadagno , che cava dal far’ il farto , ec. 4

SOFFLARE , }n lingua furbefca vuol dir Far la fpia, fe bene & intefo comune-
mente. Ed il Poeta parlando di cicchi , i quali hanno per coftume di parlar fur-
beko , & ferue di quefla , ed altre lor parole , come E//er di calea’, che vuol dir
Huomo da far qualfivoglia furfanteria , ¢ viene dalla voce Calcagno , che in lin-
gua furbe(ca vuol dir Moneillo , cioe /adro di calca nella quale entrano per rubar

‘ic borfe , ¢ di qui fi dicono Borfatolt , ¢ Faglia borfe. Vedi forto C. 6. flan: 64.

NIMICT de’ muriceinoli, Chiamiamo muricciuoli quel pezzo di muro,che avan-
za fopr’a terra attorno alle cafe; d’ altezza d'un braccio pil’, o meno,e di
fimile largheaza ; fatto , o per wlo di federe , © per difefa de i fondameati. Di
guefti fono nimici i ciechi , perché {peflo vi Pp jotono dentro co"! piedi, ingan-
nati dal fentir al vifo , ed alle mani I’ aria libera, il che fa lor credere , che non
potia eflerui impedimento veruno anche in terra. nth

s N ZA xXXAVII.

      

 

    

La frradai pitt fi fanno col baftone 5 Chi fuonatt ribecchin,chi il colafetone;
Altri la guida fegue d'un fuo cane , Cosh tntti fi'van bufcando il pane. —
Chi canta a pit d'un’ ufcio wt Oraxione, Han per infegna il djaval de'T arocchi,
E fa [corci di bocca , ¢ voci frrane ; Che vuol tentar un fornopien dé.
Deferive il modo del marciare di quefti ciechi , ¢ fa lor fare que! .

razioni, che fon (oliti fare andando a cercare elemofine, Dice che ‘anno

la Prada col baftone ; altri fi fanno.guidare a un cane, ed altri vanno cantando Orarioni
a pic d' un’ xfcio; EB quetti fon ciechi ftipendiaci daile pexlone pic, aseiocche ogni
F 5s giorno,

iz

 

 
 

S es — Se eee Ls =
F ws ae

SS

-_Pighiar® & tiechi fuor ¢ all offeria

 

 

PRIMO CANTARE: 33

Le , 0 ogni fettimana vadano alle cafe delle medefime perfone a cantare un’
razione avanti al loro ulcio , dove per efier fentiti fanno voci frane , ciot Gri-
dano forte , ¢ fanno bratri feorci di bocca; E quefto avvien loro perché, per lo pill,
li ciechi oltre alla loro cecita , fogliono havere altri ftroppi nella faccia. Molti
fuonano il ribechino , cio’ il violino , altri il Co/a/cione : quefto ftrumento (che da
i pi ¢ detto corrottamente Gavafcione ) E’ un corpo , come quello della tiorba_ ,
con manico lungo , con due fole corde, il quale fi fuona con un pezzo di fuolo
da fcarpa , che volgarmente fi dice taccone ; & percio tale ftrumento é detto an-
che Tiorba a Taccone da Filippo Scrutendio da Scafato , il quale cosi incitola 11
fuo graziofo Canzoniero Napoletano . Alcuni furbi per co/a/cione intendono las
forca , perché ancora a quefto s'adoprano due corde , la grofla , ¢ la [ottile , co-
me alla forca. Quefti ciechi fuonatori foglion fempre andar vendendo q@alche>
Orazione , o Rapprefenitazione , o altre Leggende , e cosi tutti fi vanno bufcando
il pane, cioé guadagnano da vivere. E volendo il Poeta moftrare quanto la gen-
te di quefto terzo fia affamata , le da per infegna wx diauolo , che tenta un forno
pieno di gnocchi ; ¢ moftra che fia fempre intenta a procacciarfi ib vitto con ogni
forta d’ inuenzione , che il verbo rentare fignifica Procurare , © Provarfi di far
una tal cofa.,e fideduce, che quefto diavolo senrafe , cioe fi provafle a rubar da
a4 forno il pane , che vi era dentro . E per gnoceointende Ogni forte di pane ;
bene gnocco & quella {pecie di pane , che dicemmo fopra in quefto C. ftan. 3.
SCORCL di bocta , ¢ voci frrane . Voci ftrane’, ¢ bocche diverfe dal naturale ;
perch fe bene la voce//corcio & termine di profpettiva , che moftra la figura effer
re(a capace della terza dimenfione del cor po; s‘intende anche per pofitura di cor-
© parte d*etfo diverfa dal naturale , aI:
T AROCC AF, Carte,con le quali fi giuoca alle Minchiate.Vedi orto C.8. ftan.
« 61. in.una delle quali carte al num. 14. ¢ efigiato un Diavolo ; e guefto dice, che
senta il forno pien di gnocchi Il noftro Poeta haveva dato a quefti Ciechi It imprefa
del Buio , come fi vede in alcuni fuoi sbozi , che diceva . ‘
Hanno un’ imprefa , dove Bieco metre
Li buio che a fuegiiar vale Cinette ,

STANZA XXXIX, STANZA XXX¥X.

» Dietro al Duca,c'ognunguarda atraver[o Signora’, rifpos' egli, benché ciecay >
Vanno cantando 0 aria di Scappino, Fu pero jonpre fimil gente sgherra ;

Ma non ginnfero al fin delterzoverfo, Con quel batocchio zomba a mofeacieca

. Che venuto alla donna il mofcherino , Senza riguardo,come dar’ in terra;
Fattoa Bieco un rabbuffo a modo,e verfo, Sort'ogns colpo intrepida s' arreca y
Gli diffe: 5? ia v' alloggio dimmi Nino, Che non vede i perigl della guerra:
«Perch io non veddi mai in vita mia E' cieca ¢ ver ,ma pur il pan pepato

< E! pitt forte yfe d’ occhi egii é prsvato,
STANZA XXX41.. eoike

 

 

_ Ovvia (difs ella) tocca innanzi il cocehio, Va dunque oforteye inuitto bercilocchio,

 £fe cofforoa guerreggsar fon’ atti Che i nimici da te [aran disfatti y
Tienteli puree’ non mi par! a crocehias Perch' in veder la tua bella figura

 Mdentre gli tempo qui di far di fart, Cafcan morti,fenraitro , at para |
Queiti cicchi and dictro a Bieco doiaria di Scappino,( che ¢ uaa
alse E can

 

oi

 

 
 

 

34 MAL MAN TELE €
canzonetta , la quale cantavano i ciechi in Piazza del G, Duca , quando I Auto.
re principid la prefente opera ) ma Celidora adirata di cid , dice a Bicco yches
non vuol tal gerite , ed egli rifpole , che-fe bene eran cicchi cran perd-fieri, che
il non vedere 1 pericoli gli rendeva arditi,¢ forti , come appunto ¢ il pan pepato,
che é pil forte , quando non ha occhi ; ond’ ella gli dice, che fe gli tenga,¢ vada
allegramente , che ella ha {peranza dicavar fructo da lui folo fenza loro, perché
ftima , che il nimico fia per cafcar,morto fubito, che vedra il fuo brutto vifo.

GV-ARDA a traverfo. Vino che ha gli occhi {compagnati-, come haveva Bieco
diciamo Guardare 4 tranerfo. Vedi fopra in quefto Cant, flan. 9. Tran/uerfa tuen-
tibus hirquis, Virg. Egl. 3.

VENVTO alla donna il mofcherino, La donna , cioe Celidora,s'adird, Si dices
Venire il mofcherino al nafo , perché fi troyano alcune piccole mofche 4 le quali vo-
Jando , taluolta entrano nel nafo altrui , ¢ toccando quella parte cosi fenfitivas 5
danno grande alterazione , e mettono }' huomo in una fubita impazzienzaye ftiz-
za. Si dice ancora Venir la fenapa, 0 la Moftarda al nafo, perché nel mangiar la
moftarda (che ¢ un’ intingolo facto di fenapa , ¢ mofto corto) quando ¢ ben.cari-
ca di fenapa,viene al nafo un certo pizicorey che forza a, lageimare .. Si dices
anche Venir la muffa , 0 altri puzi odiofi , ¢ {porchi , come fidice fatto GC. 4; flan.
23, E tutti fignificano Venir collera.

E-ATTO ua rabboffo,, Beavato. Fare un rabbuffo, 0 Rabbuffare yuol dire Ri-
prender uno con minacce, o Spaventarlo-con afj di parole .. Ii Landino nell’
efpofizione a Dante C. 7. dell’ Inferno alla parola Bufa.,.¢ Rabbuffare dice: Ada
proprio Buffa ¢ vento, onde diciamo Buffettare chi getta vento, per bocca,e Shiffare,quan-
do con {nono di parole , 0 a dir meglio Con ventofe , edienfiace parole alcuno minaccia.
Di qui diciamso Rabbuffare,Conturbarese muover le cofe dell’ ordine loro, e fcompigtiarley
e chiamiamo Rabbuffo,quando Con parole conturbiamo , e Scompigliamo la mented’ uno y
Vedi forto C, 3. ftan. 57, la voce Buff. _ i ‘

A modo ,e 4 verfo. Con tutta perfezione . B il latino modis , & formis ,

DIMMI Nino, Dimmi pazzo,e fenza Ceraello, come fu Nino , il quale per lo
grande amore , che portava a Semiramide fua Meretrice 5.0 moglie » le concefic ,
che per un giornovella fife affoluta Regina y ed ella in quel giorno:lo fece am-
mannaie » ¢ficonfermd Regina per fempre » come fi legge in Plutarco ia Serm,
Amator, j ‘ oe

‘ PIGLIAR’ i ciechi fuor all’ ofteria, Quand’ uno vince affai , fogliamo dirgli: Si
torrd + ciechi, © s’ intende ail’ oferia. E quefto perché fi fuppone, che quel
tale , che vince per I’ abbondanza del aro. venutogli in. mano. fenzas
fatica , fia per {penderlo profulamente in pigliar& tutti.li fuoi guftt fino cons
P andare a cena all'ofteria , ¢chiamare alla faa menfa a fuonare alcuni ciechi 5 i

in fa !-hora del inno girando, perl’ ofteric.a tale eficttogexques

“ eee
: hi fono i Ciechi , iiquall Oelidaracdive haver veduto pigliare all’ ofterie, —

SGHERRO, Bravo... Ammazzatore ; Tagliacantoni.. Vediforo, Canty 3.
Rp. ers auey 2h ee at
BATOCCHIO, Quel baftone ; col quale fi fanno la ftrada.iciechi fi chiama
Batocchio dal et 7 oe fanno.i_ciechi, per aeneeneer!

dattere da gli altri cicchis E perd yuol dire anche 11 Battaglio delle Campane .

 

| ZOGH-

 

 

 

 
 

 

vo LS ell ee

PRIMO CANTARE: 35

ZOMBA. Perquote , baftona. Vedi forto C. 6. flan. 104. ,¢C. 11. flan. 28,
. MOSCA cieca. \\ giaoco detto Mof¢a cieca ¢ trattenimento da Fanciulli , che
deriva dall’antico, ¢ fidiceva Atu/ca aenea , ¢ fi faceva nel modo , che ufano
hope , che é in quefta maniera . i

irano le forts fra pid ragazzi a chi debba bendarfi gli occhi , ( che in quefto
giuoco dicono Star fotto) ed a quello,a cui tocea,fono bendati gli occhi in modo,
che non poffa vedere, ¢ poi con uno {ciugatoio, o altro panno avvolto,che ciafcu-
no tiene in mano, fi danno da gli altri delle percofse a colui,che é {otto , ed egli
¢ost alla cieca va rivoltandofi , ¢.quello-che egli arriva con'la percofla deve ben-
darfi in vece del, percuziente , il quale fi leva Ja benda,e va fra gli altri a percuo-
tere il nuovo bendato ; Quello,al quale di-mano:in mano tocca a ftar forto, me-
na y fenza riguardo , colpi fpictati, si-perche commoffo da tanti colpi vorreb-
be: vendicarfi , si anche perché , cogliendo , il colpo fia in modo da non poter’
efler negato , procurando ognuno di non toccarne ,¢ d’ occultarla, fe pud,
uando l' ha: toceata , per non haver’ a ftare in quel martirio ,in-che € colui,che
fotta . E perd dice Zomba a mofca cieca fenza riguardo come dare in terra, Si di-

Ce maxzate da ciechi per inteadere Percofie fpierate .

41 Pan pepato é pis forte fe a! occhs egli e private , Si fuole in Firenze perla fetta
di,tucti i Santi fare un certo. pane che da noi fidice Pam pepato, il quale ¢ com-
io di fapa, aceto., farina, pepe , ed altri aromati-, ¢ mefcolanui pezzetti di
ib di poponi., zucche’, cedri ,ed‘aranci:conditi in gugchero , o miele, li qua-
di pezzetti,quando il pane fi tagliasreftano nella tagliatura a fimilitudine d’occhi,
¢€ percio. da i noftri Fanciulli (on chiamati Occhi ; E cavandofi dal pane tali
echt sche fono dolci;il pane refta cnet »cioe pil acido ; ed il Poeta fi ferves
della parola Forre-in fignificato diGagliardo , dicendo che i ciechi fendo feng’
occhi fon pitt forti', ed intende gagliardi,{cherzando con quefto equivoco di forte,

T [8c innauzi il cocchio. Seguita il tuo viaggio, ¢ tanto's’ intenderebbe a dir

teturainnanei fenza-porui |' aggiuata Cocchio, ma il Poeta ve lo pone»
per feguitar I’ ufo Fiorentino .

ST AR a crocchio . Il verbo Crecchiare., ¢ la frafe fare a crocchio fignificano Ci-
ealare, o Ciarlare di cofa di poco frutto, o importanza per finire il giorno .
Onde quefti tali fi dicono Crecchions., Cicaloni 5 Perdigiorni , ¢fimili. Vedi for-
to Cant. 3. flan. 5. Quefto verbo Crecchiare ferue anche per intendere Dar delle
buffe . Vedi fopra in quefto Cant. ftan. 19. Se

AZERCILOCC HIO . Epiteto compolto-dal Poeta’, che yuol dir Bircio di‘che

fopra in quefto Cant, fan. 9.
STANZA XXXXII “ “STANZA XXXXUL

We Segue intanto Romolo Carmari ns * sen! infegua nera che v’ ¢ drenta y
Cavalier di valore , ¢ di gran fama; » Cupido morto con & {nor piagnoni
M14 sfortunato , perch coi danari - - Aarciar fi-vede un groffo Reggimenta,
Cee ere nee 0 Chi egl bad innumerabili tritoni 5
‘on Je prllole berarj All cui arrive ugnun per lo fpavento
L affetto evacns orc nike Si rincantuccia , od Smpieftealteny

Tal che fenz’ un quattrino dmartellate ° Eda lontana infin
Alla guerra ne va per difperato, 9 iano s
a v \vo. we E

 

eee

 

  
 

36 MAEMANTILE

Segue Romolo Carmari, Quéfto fa un Fiorentino,del quale non flimd bere {cio-
glicr l’ anagrammma , edirne il nome. Queflo Gentilhuomo havendo durato un
gran tempo a godere una fua Meretrice , ¢ (pefovi molto danaro , 0 gli fu tolta ,
o ella non lo volle pik perché egli abbandond lo f{pendere; come ¢ proprio di
fimili donne ; ¢ cid ¢fprime il Poeta ia quei due veri.

Con le pillole date a fusi erar} ,
L affeteo evacuo l' Arpia ch’ egli ama,

I quali verfi {uonano : L’ havergli fatta votar la bor{a fece difperdere I amo-
re , che ella fingeva di poreargli , Onde egli difperato , fe ne va alla guerra;¢
moftra quefto fuo fpento amore nell’ infegna , che egli porta , in cui ¢ dipinto
Cupido morto , che ha d’ attorno i fuoi piagnoni. E perché quefto Signore era
nel veltire pofitivo , ¢ {enza boria alcuna , anzi pil tofto abbietto , il Poeta fa ,
che egli conduca un reggimento di gente mal veftita , ¢ quefti huomini chiama
Tritoni , pexcht Huomo trito, 0 Tritone tanto vale appretlo di noi quanto dire
Huomo mal veftito ; E quefta gente per effer cost mal veltita ¢ ftimata una {chie-
ra di Monelii , ¢ di Ladri, e percid ¢ caufa , che s’ acerefcano i {errami alle bot
teghe , ¢ che ognuno fugga per la paura , che ha di loro . ;

DAMA, Vuol dic Donna nobile,venendo dal Greco Damar,fecondo alcuni;
¢ fuona Signora dal Francefe Dame, Madame , cioé Signora , mia Signora; ma
i piglia anche per |" amata , come ¢ prefo nel prefente luogo y i

CON le pillole date a fuoi erar) , Con I’ evacuatorio dato alla fua borfa,cioé con
avergli fatti finire i danari mand via dal fo corpo la bile amorofa , cio’ lafcid
d-amarlo. ?

L' Arpia « Intende Meretrice y ed efprime una donna rapacé , come fono le»
‘Meretrici ( che Arpia in Greco fuona come Rapace ) © quali fono figu-
rate Arpie , che i Pocti fingono effer tre , Acllo , Ocipete , ¢ Celeno; ele
fanno figlie di Nettunno , ¢ della Terra ; altri figlie di Thaumante , ed
Elettra, altri d’-altre Deita ; bafta che fe ne fervivano per e(primer I avari~
zia. Vergil. 3. 4En.

Tristins haud illis monffrim , nec fevior ula
Peftis& ira Deum fiygijs fefe extulit-undis,
. Virginei volucrum vultus , feediffima ventris
Proluvies, unceque. manus, & pallida femper
Ora fame. ele
+E Dante nell’ Inf. Cant, 13. feguitando Vergilio dice
Quivi le bruste Arpie lor nido fanno ,
Che caccinr dalle Strofade i Troiani | i
“Con trifto annunxio di futuro danno-. ‘ i
Spalle hanno alate’, colli ,¢ vsfi humaniz. ,
¢ Pit-con artigli ,¢ pennuto il gran ventres ww? ‘
‘Fanno lamenti fu glivalberi , flraniy 0 patect, aed.
nome d’Arpia dette a una Meretricé ancheil Coppetta nel fio Capito-

Quefto nome
lovin biafimo della Signora Ortenzia Greca dicendo
crudeli , infide, inique , ¢ ladte “
s venire a faffidio a mille Rome

Fes teva efease ofr madee; Sane

 

 
 

   

PRIMOPCANTAR 2.

AMMARTELLATO, Haver:martello , o:¢Ser' ammnarczilav var!
Qaand”uno innamorato ha gelofia della cofa amata , ovvero ha qualens 12
con la medefima . 1] Firenzuola nel fa Capitolouin lode del legna fantoy chia
paazial’ efser' ammartellaco.dicendo :

tor nnovamente vi dice che cava»
Di faftidioun’, che crepi di-martello,
Guarda fe'quefia é un! opera brava.

E si parri voleffon provar quello
E conofceffon la lor matattia,
Tutti rirornerebbono in cernella’;
C! altro non ¢ il martele una Paget.

PER difperata, La difperazione tuna foverchia inquietudine , cagionata das
” grave dilgufto , la quale-ci leva affatto‘il dominio di noi medefimi .

PLAGNONI. Trova (peffo nelle ftorie Fiorentine quefto nome Piagnoni , che
vuol dir Coloro che feguitavano la parte di F, Girolamo Savonarolajma qui vuo!
dir Quegli huomini , che fi mettono ai mortori de i gran perfonaggi atcorno al
cadavero , tutti coperti di nero-, ¢ con lunghi-veli , ed in mano hanno uno ften-
dardo , o penaoncello di taffetta nero: E fi dicono Piagnoni, dal piagnere ches
dourebbon fare per la: morte di quel tale.

MARCTARE , Bib muoverG degli eferciti . Voce reftata a noi-dal Prancefe ;
‘eda molti fi dice Marchiare ,perché quefii ali , vedzadoia fcritta con I’ afpira-
‘zione y la pronuaziano all’ Italiana ,non fi curando di riflectere che il C-H fuona
fer, enon chi.

*REGGIMENT O), Quantita di Soldati comaniata da pitt Capitani, ¢ dal Co-
donncilo ; ¢'forfe lo ttelo,che Terzo detto fopra in quetto C. Man. 37.

TRITON » Sono Dei , 0 Moftri Marini, iquali fi dipingono ignudi , o al
pil coperti d’ aliga , ¢ di qui gli huomini mal-veltiti fi chiamano-da noi Tritoni.,

G huomini triti , che fuona Huomini vili , ed abbiecti. Vedi fotto in quefto
‘ant. ftan. 86."
tee spinach + Nafconderfi:, o-metterfi per'i canti per non effer
veduto.

EAM PIESTi calzoni’, Per la’ pauta , (c li move iil corpo , e gli empie le brache.-
Quetto detto efprime,che Quei Tritoni facevano gran paura a-chi gli vedeva ,non
‘che veramente ‘fe gli empicilero i calzoni . ;

_ SAD DOP PIANO i ferrami alle borteghe Per afficurarfi da coftoro,che fono tima-
‘ti tanti ladri, im gran tratto di pacfe rinforzano le ferrature alie botteghe. B qui
I’ Autore dice tutto quello’, ‘che’egli pud , per moftrar coftoro affatco birbom ,-¢

‘vera canaglia-, 5
ee STANZA XXXXIV.-
etek Manmade © «\” 'Serive fonetri , canta ognor di Fillt.,
‘nella guerrae fogeetto., © °E’ bnon compagno, pi. i il vin pretto,
Che miciterebbe gl li Achilii; ~ ‘Rabaro, - Seu nel eae
* :E quanti fon di loro in un caléetto: * Uiquattrodelle coppe c’ bail monnino .
nella moftra Doriane da Grilli che t Lionardo Giraldi. Quefto gentilhuo-

 
     

bait: B33
“mo fu bellifimo humore 5 molto dedito alla poefia burlefca , buom difeorricors,

 

 
38 MALMANTILE |

ed huomo di conuerfazione ; ¢ perché egli haveva per coftume il dar de Monnini,
il Poeta gli fa fare per imprefa Vana carta da giuocare s nella quale in mezzo ay
un guattro di coppe é figurato.an. Monnino,

CHETTERE uno itz un calcetto . Confondere uno , Superar’ uno nel fapere 5.0
nel valore , ¢ ridurlo tanto avvilito , che fi vorrebbe nafconder dentro.a un cal-
cetto , viliflima , ¢ piccola parte dell’ abito dell’ huomo , come —_ che non,
cuopre fe non il piede , Quefto Doriano veramente non fu mai foldato , fe ben
lV Autore dice,che egli & buon foggetto nella guerra ; ma dice cosi di lui, perché ef.
fendo egli di {ua conuerfazione,lo fentiva {peflo.dilcorrer delle guerre con gran 4
fondamento moftrandofene aflai pratico,

VIN pretto. Vino puro , ¢ fenza commiftione d’.acqua , 0 d’ altro ; ¢ fenten-
dofi in pili luoghi-del noftro Contado chiamarlo vino puretto, non fon lontano: da
credere,che la voce presto fia o figurata , 0 corrotta da puretto .

CASINO. lotendi quella Cala nella quale la-nobil gioventi Fiorentina s*adu-
fa per giuocare ,

MONNLNO . Le carte de’ Ganellini , 0 Minchiate:hanno in (¢ efigiate quate
tro cole diverle , che una parte hanno {pade , unaparte baftoni, una parte das
nari., ed una parte coppe , ¢ tutte quattro quefte noma di carve comingiano da
uno fino a 14. Nella carta del quattro di.coppe in mezzoée una bertuc-
cia a federeyla qual bertuccia da noié detta Adonming , .E.quefta dice il Poetayche
é l infegna di tei i perché egli & (olito didare i Adonnini , che yuol dire»
Quand’ uno parlando con un’ altro,quefo lo forza a) dir qualche parola,che rimi
con un' altra , che a quel tale difpiaccia ; per efempio Doriano diffe ad ua.Che-
rico: Won fu mai gelatina fenza,-, «. B quisfiférmo fingendo non firricordare
della parola che finiva il verfo ; ed il Cherico , il quale ben fapeva. la fenrenza
glicla fuggeri dicendo : fenz’ allore , ¢ Dorian foggiunle: Voi free il maggior bwe
che vada in core, E, quelto fi dice dare i Monnini .

 

STANZA XXXXV, ‘ STANZA XXXXVI
Fra Ciro Serbatondi il Sir di Gello Di foglio per impre/a un bel Cartone
Che in, Pindo a Adana Clio foftiene ikbraccia,  Iafiame con lapafia egtibanno meffo 5
Egeno de Brodetti , e Sardonello , Dei lor Fantacci , i quali da Perlone
Vafari ,.ch' ¢ padron di Butinaccio, |. Saglion copiare, 0 difegnar dal gefso 5
Conducon tanta gente ch’ ¢.up flagella Ne 0 v'han dipinta d'inneneione
Da far che le pagnoste habbiano{paccia, L loi inaaieqeas hanno e/prego

Di cus (perch il mefbar dilettaaognuno) Su le tre hore il venicel revaio
PENN me ane oRWreen jem neh nies
ano tre ati \ puno ¢ Fra Ciro Serbarondi

“the vuol dirt Crifefane.Berandl quale fa Sir di Galle , perchs ha fonte woa fone ]
villa cosi deta. Dice che fofieneril braccio,a Adana Clio , perche egli é huomo
SecaelaFlrss che ol dt defend Pat henate Sit at dosoaoaes

‘ardone ‘ari -vuol dire aloré , il, fa. x 3
perché ancor’ egit ha una Villa cosi detta , Conducono. i molsa. 4

wale comandano vicendevolmente a un ne (per-uno ; ¢ perché fix

ong ftati tutti tee fcolari dell’ Autore ,fa lor fare una bandiera de)
difegni ,ch¢ hanno fatto in fquola {ua ; Ma perche gueiti

‘Si pighano il comanda a un di per uno

   

 

    
> agoftos e per it

 

fi alcuni di.
“Perr

gli huomini nel

 

 

> 7 >; + -

PRIMO CANTARE. 39

cheialla-pittura , perd non fecéro altro acquitto in effa, che quanto baftava per
una certa infarinaturar, ¢ per faperne difcorrere ; egli volendo moftrare queito
lor poco profitto , fa che di Jor propria inuenzione ritraggano nella decta lor
bandiera una cofa invifibile , come appunto é il Vento . q

E 4n flagello. Quefto termine fignifica Infinita , ed Abbondanza grandiffima ,
‘ed efprime un numero indeterminato . Vien, forfe dai Latino , che tal volta
fignifica Quantita immenfa . Martial, lib. 2. 30. Eromius laxas arca flagellat opes 5
parlando d’ uno che havea gran quantica di danari,

CHE le pagnotre babbiano fpaccia, Che s efitiy che fi confumi molto pane. E pa-

<gvorta {e bene non ¢ voce Fiorentina ;¢ nondimeno {pelio ulata .

MEST ARE , Qui val Miniftrare , Comandare ,

CARTONE , 1 pittorichiamano Cartone Quella carta grande fatta di pid
fogli., fopr’ alla quale fanno il modello di quaiche grand’ opera, che devono di-
pignerencl muro a frefco , o a tempera 5.0 vero per teflere arazzi .

FANTOCCH, Figure mal fate. Pitter da:Fantocci s intende Pittore da pocds
appunto.come da quelta loro imprefa vuol l' Autore., che fi argomenti che futle-
ro.quefti Signori .

DAL gefs, Ciok dalle figure fatte di geflo . I pittori hanno per coftume di
chiamare dette figure di rilevo ,( delle quali fi feruono per difegnare ) col folo
nome dige/> (enza dir figure , 0 itatue , come fi vede nel prefente luogo , che»
dice difegnar dal geffo,

LANTERNONE . Arnele noto, che ferue a portarui dentroil lume , ¢ di-
feaderjo dalivento.) .))
jy, BRVCIAT 410... Colui che vende marroni arroftiti alla fiamma , 0 nel for-
no, che noi chiamiamo Sruciate, donde Bruciataio ,

n STAN ZA XXXXVIL

tho : >

Nanni Ruffa del Braccio , ed Alricardo Hanno acomune un lor vecchio fredarde
Conduce quer di Broxziy edi Quaracchi Da farne a corui tanti fpauracchi ,
e ‘he bevon.guel lor vin gagliardo, E dentro per imprefa v' hanno pofto
Le ftrade allagan tutte co i fornacchi, Gli [piragli deldi di Ferragofto.

Scguitano due altri Gentilhuomini Wanni Ruffa del Braccio, che vuol dire Ale/=

ro Brunaccini od .Alticarde che vuol dice Carlo Dati ; a quali fa condurre le»

gentidi Brozzi., ¢ di Quaracchi.y,due loghi vicini a Firenze , ne i quali nafces

_ vino deboliifimo 5 ¢ pero dice che quefti foldati fon mal fani ;¢ pieni di catarroy

perche bevono.quei vini deboli , ( che egli ironicamente parlando , chiama ga-

en cree danao prima alle gambe, che alla tefla.

_E pecché tali infermi pare che fi rihabbiano , ¢ piglino qualche vigore, quando

_ .trovano all’ allegrie ; percié fa loro portare una infegna nella quale fono efpret-
b i, gozzoviglicy ed allegrie , che gia fi facevano il di di
‘8 intende il-di primo d’ Agofto , venendo quefta voce da Feriaré
igenza di quefto¢ da fapere , che anticamente folevanfi cele-
1.con grandi allegrie; ¢ cid fi faceva forle, perché eflende
feruore della ftate,erano neceffitati dal gran caldo a fta-
-ge allegramente , perché I allegria ¢ il ptiino rimedio delia (quola Salernitana:
Hac tria mens bilarisyequics,moderara diera:Eisédo duague molto pericolofo in quet

. tempi

   
   
  

agofto,

_-brar le ferie Ai

 

  
~~

 

GQ MALMANTILE

tempi d” infermarfi, e percid molti giorni infaufti allora fi notavano dagli Egizj,
cfiendo vicino al Sirio , 0 Canicula da tutti detta peftifera y come ci moftra Sta-
2io lib, 1. Siluar, Zam nec calido latravit Sirius «fro, E' necefiario ripofarfi,bere,e
maogiare , ¢ ftare allegramente ; al che configlia nelle fue Odi Orazio pid voice;
Ed habbiamo una cantilena affai praticata, che dice,

Quando fol oft in Leone ,

Bonum vinum cum mellone ,

Et agreftum cum pipione . ¥
E perché veramente il feruore del So! Leone , o Sirio , ¢ allora nel maggior col-
mo , fono le flagioni molto calde ; ¢ peggiori , che in tatto I anno ; onde appref-
fo a’ Greci ancora fi facevano molte allegric’, ¢ facrifizzj.a {egno , che appreffo
gli AttnieGi (ecGdo alcuni il mefe d'Agofto acquifto il nome a’ Hecatombeon. Tai felte,
ed allegrie fi facevano gia a Firenze non folo per la detta ragione, ma ancora per
caufa di alcyne vittarie ortenute da i Fiorentini in quei primi giorni d’Agotto,e fe
ane conferua ancora il coftume, ma‘non fi fanno tance felte , quante gia fi faceva-
no , poiché folamente fi fa correr al Palio alcuni Afini : Siche s’ argumenta, che
il noftro Poeta intenda , che in quefta infegna , o flendardo futile rapprefentato il
palio de gli afini, mentre dice /piragli del didi Ferragofto , che yuoldire un poca
di memoria delle gran felte , che gia fi facevano in quei giorni. © '

SORNACCHIO., Sputo groflo,e catarrolo, detto anche farda , Vedi fopra in
quefto C, flan, 25. Monfignor della Cafa nel {uo Galatco dice; Di /ofiamenti di
nafo {porcamente , di tirar fornacchi , e /putamenti ,

SP.AVRACCHIO , Cosi chiamiamo quei paunacci , che fopra ad un palo,per-
tica , 0 albero ft mettono per li campi a hne di {paurire i colombi , ed aleri uecel-
li, Vedi fotto C. 5. flan, 49, poe

SPIRAGLIO , Vucl dit feffara in muro , © in tetto , © impdfte di ufci, o di
fineftre , per la quale , trapela Itaria , 0 lo (plendore , che i Latihi dificro rima-,
In guclto luogo perd é intefo metaforicamente per Piccola notiziaycome é atiai
in.ulo , ¢ forte non lontano da i Latini, che dillero Spiraculmm tantu! ius rei ad
me venit per intendere lo ho havuta di cid qualche notizia ,

STANZA XAXXVIIL

Guftavo Palbi Cavalier di petto Van moltiagrucce,in feggiola,e nel letto,
Con Doge Paol Corbi hor ntincammina Perch non fono ancor neta farina ;
Gt’ Incurabili tutti, e 11 Laxzeretto ; Fat per imprefasn un lenzuol chefuetola
Gente, che ufcia di far la quarantina . Fn Pappino rampante a una pentola,

Seguono Guftavo Falbi , ciot Vgo Stufa Senatore Fiorentino , ¢ lo chiama Cava-
fier dé pesto , perché ha la Croce in ray efiendo Bali della Religione di $. Stefa-
no}; BI altro é Doge Paol Corbi , che vuol dire Canatier Lacopo del Borgo. A que-
fti due gentilhuomini fa condurre una q 1 { > ¢ di ftroppiath,
per moftrare , che effi nel tempo ; che |’ Autore componeva la prefenté Opera

non crano d’ intera fanita per quaiche poca d’ ipocondria , che gli miole(lava, ¢

fa perd lor fare per imprefa un Seruo dello fpedale di S.Maria Nuova con les
mani aizate a una pentola , Ue
INCVR ABILI, Cosi fi chiama in Firenze uno Spedale., ne} quale vannoa cu-

racfi1 Maitranzefati.
Laz.

 

+e

 
0

PRIMO‘CANTARE: “gt
LAAZZERETTO. Luogo ; o Spedale in cui fimettono gli huomiai ,¢ robe»

- folpette di pefte per far lor fare la quarantina , ¢ renderle praticabili , che Far la

‘qnarantina vuol dire Star riferrato in uno di quefti Juoghi quaranta, o pil, o me-
no giorni per/purgar il fofpetto-d’ infezione. E quefto nome Lazzerctto viene»
da Lazzero rifu(citato da N, Sig, Giesi Crifto, quando era di gia fetente il di lui
‘corpo. :
GRYCCIA, Specie di baftone per gli ftroppiati , fopra una teftata del quales
‘efiendo confitto un legnetto fatto a guifa di mezza luna , fi foftiene'il corpo met-
tendo detta mezza lina forto il braccio, ¢ I’ altra teftata del baftoné in cerra; ¢
-perché quefto’ baftone é fimile a una croce mi par di porer crederé , che 1a voce
‘Gruccia fia corrotta dal Latino /cipio cruciarus ,

ANON fon netta farina, Non fono {chietti, non fono affattd fani .

LENZVOL , che fuentola., Coftoyo in vece di bandiera , ufano un lenzuolo ,¢
cid per moftrare , che tutte le loro cofe fono da {pedali’; in effo leazuolo é dipin-
to-un’ Aftante ; 0 Seruo dello [pedale di S. Maria Nuova, rampante a una pentole,
cioé con le mani alzate a una pentola , che @ in alto ; a fimilitudine del Lione , i!

aié quando fi trova dipinco ritto con le bratiche dinanzi alzate a qualche cola,

‘dice Rampant. -Pranco Sacchetti Nov. 133, £d hebbers ritrovato per cimicro wx

MeXXo orfo con le campe rilevate ,¢ rampanti,
L

STANZA

Bel Mafatto Ammirato anch' egli pas

Lindo garzon d' ogni virtis dorato ,
Che pus de’ foldi bavendo nella caffa
Pifeiar a lerto , ¢ dire 3 lo fon fudato ;
Ma per I ipocondria, che lo rartaffa,
Ei fi da acreder d’ effere Ammalato;
Maé mangia,beve,e dorme il/uobifagno,
Chit fine 4 vefpio,e poi fi leva in fogno,

ST AN ZrASE.

Con lo fcenario in mano, e il mondo fusra

Va innanzi 4 nobil fuoi commilitoni,
Pancrazio , Pedrolino, ¢ Leonora

Lo feguon con un nugol d’ [frien

C hanno una infegna non finita ancora,
Perche Anton Dei co tuttii voi garzoni,
Incambio di sbrigar quella faccenda’,
E ito al Ponte a Greve a una merenda,

 

Patla Belmijatto Ammiraro,che € AZattias Bartolommei Marchele giovane di bell’
‘afpetto, ricco, € letterato ; il quale fu un tempo , che fi perfuadeva d? haver tucti
imali. E perché’quefto Cavaliere fi diletta di comporre commedie , ¢ volenticri
récita in effe lui medefimo’, ed appunto nel tempo , che I'Autore accrebbe la pre-
fente Opera, havea detro Signore mefla infieme una conuerfazione di giovani no-
“bili, che recitavano all’ improvvifo ; perd lo fa capo di nobili commedianti , ¢
gi da uno ftendardo ‘non ancor finito , perch¢ Antonio Dei ricamatore ( ¢ quefto

il vero {uo nome , cognome , ¢ profeifione ) in cambio di finitgliclo, era anda-
to a un’ allegria al Ponte a Greve , luogo poco lontano da Firenze. Cafo feguito
al detto Sig, Marchefe Bartolommei, che afpettando alcuni abiti per una com.
ti 1 garzoni della fua bottega fuori di Firenze. i

4LAVENDO de foldi nella ceffa. Eflendo ricco: Non gli mancando denari

PISCLAR’ a letto @ dire: lofon fudato, E’ proverbio affai vulgato , che igni-
fica . Pud fare a fuo modo, che, o male , o bene che egli faccia , git € fempre
afcritto a bene ; E s' intende d’Vno, che fia ricco, e fortuaato .

“media , che fi dovea far la {era , il Dei in vece di finirgli fen’ era andato con tut-

i LEVARSS iv fogno, Levarfi pit prefto dell’ ere folita di levarfi, quafi dica... :
ea ee E we

 

“aes

 

 

 
 

+

4

 
 

Qe MALMANTILE

S'é levato di notte,fognado effer’hora di levarfi,e qui ’Autore intende,chea quefto.
Cavaliere il mezzo giorno, alla quale hora cominciava a deftarfi,ferviva per aurora,
SCENARIO, Eun foglio , fopr’ al quale fon defcritti i reciranti, le fcene della
commedia , la quale fi dee recitare,ec. i luoghi,per i quali volta per volta devono
ufcire in palco i recitanti , afinché quel tale , che affifte gli pefia fare ufcire ag-
iuftatamente , ed a i tempi debiti . ‘Lal foglio fi domanda anche Atandafuora , fe
il Aandafuora & alquanto differente dallo Scenario , perché queftos appicca
al muro dictro alle fcene affinché ciafcuno recitante lo poffa da fe fteflo vedere»,
ed il Afandafuora & tenuto in mano da colui , il quale inuigila,, che l’ opera fia,
recitata ordinatamente ; ma tuttavia , come ho detto , s’ intende , ¢ fi piglia {pef-
fo l'uno , per I’ altro. =

PANCRAZIO , Pedrolino ,¢ Leonora. Nomi di recitanti nella faddetta con-
uerfazione . w=

NVGOLO a’ Ifriani , Gran quantita di commedianti . Quefta voce mugolo, che
nel prefente luogo fignifica numero infinito , fi ufa pid propriamente parlando di
volatili , perché quefti volando gran numero infieme , come farebbono ftorni ,
colombi,ec.occupano il fole,ed ofcurano l’aria,appunto come fa il magolo.La voce
Ufrioni & latina , tolea dall’ antico Tofcano , come dice Polid. Verg. lik.3-cap.14.
Ie cui parole fon quefte. Et quia Hifter Fnfeo verbo Indus vocabatur , ideo nomen hi-
Srrionibus eff indizum , ec. Ma hoggi ce ne feruiamo per nome fpeciale , chiamando
Iftrion: folamente i commedianti , che recitano per prezzo.

GARZONI, Intende javoranti ; fe ben Garzone vuol dir propriamente Giova-
ne {capolo , ¢ feaza moglie , come fi vede nell’ ottava antecedente /indo garzone ;
‘Tuttavia s’ intende anche Seruitore , o layorante, che ftia a falario in botteghe
di qualfivoglia mefliero .

MERENDA., Specie di mangiare , che fi fa tra mezzo giorno , ¢ fera. Vedi

fopra in quefto C, ftan. 35,
STANZA LI STANZA LIIL

Don Panfilo Pilori move il paffo
Che,tra che per ufanza mai fra cheto,
or ch' ei fa moto fa fi gran fracaffo ,
Ch? io ne diferadoil Diavol n' uncancto,
Aforda il mondo piis d'agn' altroilgraffa
Papirio Gola , c' appunto gli ¢ dreto ,
4i qual vefti di lungo, ¢ fu guerriero,
Perocche poco gli fruttava il Clere

STA i ZA LIL

E n' ha fatto con effo de rammanzi,

C* un pp di campanile pee ae alloga y
E quefta ¢ la cagion, che la trai lanzi
Da foldato n' andd in Oga Magoza ;
We PY, al men tirato innanzs ,
Posi la [pada , ¢ ripiglio la toga,

E per lo regloff rae fee
Tornar’ 4 cafa a quefie ftiacciatine,

 

 

Al che tra molti commodi s' arroge ;
Quel ber del vinych't troppo cofaghiorta,
Lua birre, qua Jalcraut , qua cervoge y
41 cafa mia dicea,del vin s'imbotta,
Pero finianla ; cedant arma toga :
Lonon laveglio,in quanto a me,pit cotta;
Guerreggi pur chi vuol,s'amazs ognuno,
Ch'io per me non ho fpixza con niffino.

STANZA LIV.

Cost rinunzia lt arm a Gisve, ¢ ftima_
Defer il pik lieto huom che calchi terra,
Penfa frato mutar, cangiando clima, -
Ma trovara ? ltalia tutta in guerra,
E farzato ferrarfi, piit che prima 5

99 Eeco il gindizio human comefpe/soerra
so tormar fra gente eer: ye gaie ,
E fugge t' atqua jatto le grondate

BS qua satete §' STA

 
 

 

 

PRIMO CANTARE.

é STAN ZA_ LV.

Tra don Panfilo , ¢ lui uno fqnadrone
Dat Pontadera afpettano,e da V ico,
Che parte per la via vanno a Vignone y
E parte fanne un fonno a pié d'un fico,
Cofhoro empion di rena un lo» foffone ,

. E quando fono a fronte all’ inimico ,
Gliela {chizxan nel vifo, ed in quel mitre
Gli piglian gli alers la mifuraalventre,

STANZA LVI

L infegna di cofforoé un Atontambanco ,
Cha di gia dato alls {uoi vafitl prexze,
E detto che fon buoni al mal del fianco,
E firolagato,e chiacchierato un pera 5
Ma trovandofi aifin fudato ,¢ /caico
E non havendo ancor toceato xn bexzo,
Sifcadolezza,ed entrain grade/mania,
Pot dice , che fi parte per Germania,

Segue Don Panfilo Pilots , che & dpoliro Pandoifint gran chiacchierone , ¢ ‘Papirio
Gola , che ¢ Paolo Parigi , il quale ne i {uoi primi anni vefhi abito da Prete ( ches
queflo intende col dire Ye/ti ds dango ) ma poi lo posd , ¢ fen’.andd in Alemagna_,
alla guerra vedendo , che quell’ abito non gli era di frutto ; Vifto poi,che anche
que! mefticro non gli fruttava,tornd alla patria , ¢ ripiglid ’ abito. Ma trovato,
che ancora !’ Italia era fottofopra per caufa della guerra del Duca di Parma , fu
forzato dal deb)to di fuddito , ¢ dalla conuenienza della provvifione ,a tornares
alla guerra in feruizio del Serenifs, Gran Duca , ¢ a lafciar di nuovo P abito da.
Prete , Finita deta guerra il medefimo Paolo Parigi fi rimefie I’ abito , ¢ fattofi
Sacerdote , mori pot Rettore delia Chiefa di S. Angelo a Vicchio: Quefto Pao-
do Parigi fu figliuolo di Giulio , ¢ fratello d’ Alfonfo ambedue Architetti celebri ,
come fu ancor’ egli , ed Andrea altro fuo fratello , che fu Maeftro di campo, ¢
nominato dal noftro Poeta Paride Gurani fotto nel C. 3. ftan, 10.

I fuddetti due conducono genti dai Pontadera,e da Vico, (Terre vicine a Pifa)
Ac quali genti dice il Poeta, che / a/pertano , perché venendo di lontano per la
Manchezza del viaggio s’ erano fermate per la Mrada a ripofarfi; E per moftrare,

ache quefto Papirio cra. grand’ ingegnere , fa che quelta gente habbia per arme un’

ordigno per faciiitare Ja diftruzione del nimico , il quale ¢ un mantrice pieno
di rena 5 ¢ per alludere al genio vagabondo di Papirio , ed alle chiacchicres
di Don Panfilo , figura nella loro infegna un Montambanco , che fono genti
chiacchierone , ( ¢ perd detti anche Ciar/atani ) ¢ che non hanno patria ferma ,
fendo oggi in Firenze, ¢ domani altrove , fecondo che gli porta la fperanza del

guadagno,

FReACeASSO . Strepito , romore ; Vien dal latino Frangere , che vuol dit
Rompere , ¢ veramente il fignificato proprio di fracaff ¢ quel romore , che pro-
cede da frattura , o peeameno di materiali ; {¢ bene fi piglia per ogni forte di

9.

ftrepito. Dan. Inf.

_ gia venia fu per le torbide onde

Va fracafso a! un fuon pien di fpavento .

E ncl Purg, Cant,
oc! Pulgghanes 14,

ecco P alzra con fi gran fracafso
Dove !'cfpofitore Landini dice , che Pracaffo vien dal verbo frangere
WE difgrado il Diavol »’ wn canneto, Farebbe manco romore il Diavolo in uhs
ftime di canne . Si figura il diavolo , per Jo pit , un’ huomo con le corna, con

! ali, ¢ co i piedi di gallo ; onde fi di

ice un Digvol n’ um canneto , perché fi fappo-

ne , che paflando il detto diavolo dentro a un poftime di canne , pigit con le cor-
“a aN ae Bo fa»

 

 

 

 
 

 

*

 

44 MALMANTILE

na, conl'ali , ¢ con gli artigli le canne , le quali (cappando dalle dette cor na;
ali , ed artigli a guifa di molla,perquotono nell’ altre canne’, che per efler yores
sinno frepito , ¢ rimbombo non piccolo. Quand’ uno s'affatica per confeguir
gualcofa diciamo: 7 tale ha fatto it diavolo per baver La tal cofa, es’ intende ha fat-
to il diavol n’ un canneto , ciot gran romore , Il termine ; Ve disgrado Vuol dire
Jo fimo manco : Io levo il luogo,, o grado : per efempio // tale compone verfi La-
tini cost bene , che io ne disgrado Vergilio , civé io ftimo , che quefto tale habbia tol-
to il luogo a Vergilio , ¢ faccia meglio di lui. Vedi forto Cant, 3. ftan. 34. C. 6.
ftan. 61.¢ C. 7, ttan. 25.

RAMMANZO , Far iin rammano , 0 rammanzina vuol dire , Riprendet? uno,
con minacce ; ¢ fuona lo ftefio , che far’ un rabbuffo, o Rabbuffare detto fopra in,
guefto C. flan. 39.

NON gli aloza un po di campanile . Pigtia la parte per il tutto,¢ vuol dire Non
gii fa confeguire una Chiela .

LANZI, Cosi chiamiamo i Soldati a piedi guardie del Serenifs. Gran Duca;i
guali fon tutti Alabardicri Tedefchi: E pero dicendo : dade fra i Lanzi intende
Andé fra i Tedefchi , cioé in Alemagna; la yoce Lanzi ¢ Tode(ca lafciatacida
loro medefimi, che in falutarfi fogliono chiaimarfi Lanrzman’, che fuona Paefano;
¢ Lanzchne% yuol dit foldato a piede , € per quelto git Scrittori Fiorentini fi
feruono della yoce Lanzichenecchi , per intenderé Soldati Alemanni a piede’. Et
it Varchi Rorie Piorentine lib, 2, dice cosi : Quanto pitt s* avvicinavano t Lanzi,che
cost pee mazgior brevita gl chiameremo da qui avanti,e non Lanzichenecchi , ec,

OG-A magoga , Quand’ uno va lontano dalla fua patria , dicono le nottre don-
ne, Gi é andato in Oga magoga , Ed intendono gli é andato a cafa maladetta, nel
qual fenfo é prefo anche nella facra {crittura ; ¢ 5, Gio; nell Apocaliffe’ al 20,
dice Og magog , & congregabit cos in pralinm . Ed al cap. 7. dice In difperfionem i
tiem , ¢ fi trova anche in altri‘ libri della Sac, Bibbia. Vedi Angel. Mons. Flo,
Ital. lingua alla parola oga magoga. Dicono ancora Gaga magoga. E forfe inten-
dono dei Regno di Goaga in Attrica . Li Vocaboiifta Bolognele dice , che Og fu
pigante d’ Aftarotte Rede Baraniti,della creazione del Mondo 2492. contro al
popolo d'ifrael ne i campi d’ Edrai , ove fu deftrutto con tutto il fuo efercito , es
cinguanta Citta; ¢ che di qui venne ii fignificato Andare in difperfione,e in fumo,

0a cafa del Diavolo , eflendo interpetrato Og magog , peril Diavolo. Sin qui
i) Vocabolifta . Gli antichi fecondo Plinio chiamavano Magog !a Citta d’ Edetia,

” (che Strabone dice , che é I’ iftefia , che Hierapoli ) dove era il celebre ['empio

della Dea Atergatide detta la Dea Siria , ¢ dove gli Bbrei viflero in cattivita,on-
de da quefto dicendofi Andare in Magog , per git Ebrei era lo ftetlo che dires:
Aadar’ in feruith . Gio: Villani Stor. ior, lib. 5. Cap. 29. dice; Le genti , che fi
chismano Tartari nfcirono dalle eMontagne di Gog Magog chiannate in latino monti-di
&eleen, Conchiudo dunque , che oon dire andowm Ova Ativoza . Significa An-
do in pacG lontaniffimi , ¢ di pericolo : ed & quali lo ftcllo , che dice Andé 4 Buda,
che vedremo fotto Cant. 5. flan. 13. ave
TIR ATO snnanzi, Avanzato a gradi » a dignita , a utili , ec. wer
TOG. Vuol dir propriamente abito da Dottori, ma fi piglia bene tpeffo per
Vabito da Prete , come ¢ prefa in quelto luogo. Mi oes * se oR

 

 

 
 

 

 

PRIMO CANTARE,. 45

~ TORNAR a cafa a quefte ftiacciatine , Tornare a goder”i comodi della propria
cafa,che fi dice anche: Tornare al Pentolino, che i launi diflero : Redire ad prifti-
na Prafepia . Stiacciatina ¢ dimioutivo di Stiacciara,la quale é {pecie di pane, che
dopo lievito fi ftiaccia con le mani per farlo pitt fottile, affin che fi quoca pili pre-
fto , ¢ faccia minor midoila .

S' arroge . ll verbo Arregere vuol dire aggiugnere. Al che »° arrege; al che»
s’aggiugne ,¢ vuol dire; Cié anche di pit . Il Lafca Nov. 5.

E cosh per non arrozer pezevo al male, fi flava quiera , ec, Petr. Canz, 9.
Eduolmi, c’ ogni giorno arrage al danno, .

COSA ghiotta . Cola defiderabile , cofa appetitofa ; che ghiorto fi dice Vno avi-
do di mangiar del buono ; ¢ viene da /ndulgere gutturi .

SAL crave, Cavolo falato. Voce , ¢ vivanda Tedefca..

BIRRA , 0 Cernogia, Bevanda, che s'ufa in Alemagna. , ed in altri pacii ,
dove ¢ poco Vino ; ed ¢ compofta di biade , acqua , ¢ fiori di luppoli ; ed ¢ lo
fieflo Birra , che Cervogia , e quelta ultima é dal Latino.

{MBOTT ARE. Metter neila botte . Se bene qui fi potrebbe intendere Bere ,
coftumandoli dire : Jo.mon imbotto acqua,yin vece didire : Lo non bevo acqua , fi
come é intefo forte C, 7. tan. 4. :

NON (a voglio pit cotta , Per la mia parte mi bafta cosi,ne mi curo di meglio .
Sum prefenti Catone contentus , dilic Auguito .
~S$TIZZA, Ira , collera ; ¢ vale anche per Inimicizia .

FERRARST Antende Armarfi.. B decto fcherzolo , perché Ferrare , fenza dir
‘pits s’ intende mettere i terri allt unghie de’ piedi de’ cavalli, muli , cd altres
beitie. $

GENT gaie , Genti allegre , ricche , c abbondanti d’ ogni comodo , e quiete ;
che la voce Guio é forfe fincopata da Gandio.

GRON DAME . Quel calcare , che fa |’ acqua da i tetti , quando piove; ¢ fi
dice Grondaia da Gronde , che fono quelle tegole pill Jarghe, le quali fon polte-
nicli”eAremita de’ tetti, Ed il Proverbio Fuggir 2 acqua (otto le grondare vuol dire;
‘Procurar di fuggire un pericolo , ¢ andarli incontro , che ¢ quello forfe,che i La-
tini intefero col dire Lucidit im Scyllam cupiens vitare Charybdim, 3

ANDARE 4 Vignone > Andar nelle vigne altrui a’ corre Puva ; ¢ fi dice cost
‘pet rendere il detto ofcuro , moftrandofi d’ intendere d’ Avignone in Francia , 0
del Bagno di Vignone,che ¢ ucllo Stato di Siena .

SOPFIONE . Quel piccolo Mantaco , o Mantice,, del quale comunemente ci
feruiamo per foffiar nei fuoco , ufandolo a mano. |

_SCHIZ ARE, Qui é verbo attivo, ¢ vuol dice: Gli gettano con violenza nel
vifo quella » che é dentro al foffione.. f
MONT AaB aNCO « Vno di coloro che vendono i rimedj nelle pubbliches

_ ~~ piaze y detti Adontambanchi dal moncare fopra i banchi quando vogtiono vende-
© re’; € detti anche Ciar/arani dalle gean ciarle , che fogliono fare .
* TOCC-ATO wa bezz0, Preto , 0 bulcato un guattrino . Bezzo é moncta , ©»
Parola Veneziana , ma ufiamo, {¢ non la moncta, almeno Ja voce bezo ancor not
- per intender Denariin generale.
Sl feandolezza . In quetto Inogo , éd'in quefti termini fignifica Adicarfi , if mo-
— 4 car

 

CT See

 
E
.
;

 

46 MALMANTILE?

firar con fe parole, € con gli atti la collera, che uno ha. Vedi foto C. 11. flan.
23. Verbo che viene dal Greco /eaudalizefthai che {uona, a loro,come.a noi Offen=
dorfi , o adirarfi d’ una cola.

ENT RAR in {mania , Entrar in grandifima collera ; che Smania é una fover-
chia inquietudine , cagionata da febbre , 0 da ecceilivo caldo , o da foyerchio

amore , la quale riduce |’ huomo quafi infano , ¢ furiofo ,
$

TANZA LVII
Hluomini bravi quanto fia la morte
Scandicci n' ba mandati ,e Marignolle,
Gente,che fi puo dir che habia del forte,
Poi ch'ellaammazzagli azlise lecipolles
Sue lance i pali fon , targhe le {porte ,
sirchibigi le man , te palle xolle 5
Vaben di mira, ecolpocolpoimbreccia ,

STANZA LVIIIL
Vien comandara da Strarildo Nori ,
Ch'é Chimico , Poeta, e Cavaliere y
Ede quel , ch’ in un quadro coi coloré
Fece quei fichi , che divenner pere .
E perche quefto ¢ il Re de bell’ bumori,
Per dimoftrar quanto gli piacciail bere ; i
Ha per iprefaun Lanzoa due bracherte; q

 

Maffime quad attrui vuol dar lafreccia, CBilmolleinfegna trar dalle mexrette, a

Seguita la gente di Scandicci , e di Marignolle, Ville vicine a Firenze , doves
nafcono Cipolle , Agi, ed altri fortum: fimili in grande abbondanza. Quefta
gente dice che ¢ brava quanto la morte , perché ella ammazza gli agli , e lecipolie, ¢
Ji puo dive che babbia del forte, E pare che intenda che ella fae in fortezza , es
bravura gli agli: E vuol poi dire , che ha molti fortumi, ed Ammazza, cio¢ Fa
mazzi delle cipolle , ¢ degli agli. E perché quefti contadini habitando intorno
a Firenze praticano molto la Citta , dove ¢ occafione di {pendere pit che nel
coatado , dice I' Autore , che fon genti che danzo Ja freccia , che vuol dir Chie-
der denari in prefto ; ¢ par ch’ ei voglia intendere che fon bravi tiratori di free-
cia, ed’ archibufo. Son comandati da Straczildo Nori, cio Rinaldo Strozzi Ca-
valiete di S, Stefano ; ed ¢ quello, che in fquola dell’ Autore volendo dipignere
alcuni fichi non trové mai il modo di fare , che non pareffero pere. Quefto fu
un geatilhuomo di grandidimo garbo , faceto , allegro , ¢ {piritofo, e buon be-
vitore ; ¢ percio gli fa fare per imprefa un Lanzo , che vota una mezzetta di vi-
no ,¢ gli fa comandare quelta gente , perché fu poi P...... in vicinanza dei
lor pacfi.

‘SPORT A « Specie di paniere fatto di giunchi , ed ha due manichi; ferve per
pereret dentro erbaggi , ed altro, che fi provvede in piazza giornalmente per il

itto.

ZOLLA, Gleba , pezzo di terra follevata nel lavorare i campi, Vedi forto
in quefto Canto ftan, 82. .

COLPO colpo. A ogni colpo . Intendi ; fempre ch’ ei tira ; colpifce,che 1a for-
za della replica ¢ di far nafcer il fuperlativo , provi t

IMBRECCIeA , Forfe meglio imbercia; E Significa Pigliar di mira; donde»
imberciatore colui che fa pi ione di tirar d' archibufo ; ¢ par che venga da»
sbirciare , ¢ bircio , che € guardar con occhi focchiufi , come dicemmo fopra in
quefto C, ftan. 9. ¢ come s' ufa a tirar con I’ archibufo. Ma puo anche effere che
venga da breccia che vuol dir Quelle rotture che vengon fe Y ie>
dail’ artiglicric , e fi dica imbrecciare per colpire , fi come intende nel prelente
Juogo pigliando colpire in (ealo di confeguir I" intento , .

 

sari

alas 5° < co pilmiila Rae aineae o
 

PRIMO CANTARE: 47

} ‘DAR Ia freccia , Come habbiamo accennato , yuol dire Chieder denari in pre-
. flo ; ¢ s’intende Vno che habbia poco modo, e minor voglia di rendergli. Gli
antichi Etiopi , ¢ gli abitatori di Maiorca , ec. non folevano dar mangiare alli
loro figliuoli , fe quefti con le frecce non facevano cafcare dallo ftile , 0 albero
il cibo , che vi era pofto , ond’ io ftimo , che quefto frecciar per vivere habbia da-
i to origine al prefente detto. Vedi Alex. ab Alex. dier. gen. lib. 2. c. 25. [1 Mo-
nofino dice , che quefto frecciare habbia origine dal Latino ferire che apprefio
loro haveya il medefimo fignificato , ¢ lo cava da Teren. in princ. Phormionis :
Porro autem Geta Ferietur alio munere ubi bera pepererit . Diciamo; idenari {ono il
fecondo fangue ; dar ferita cava il fangue , come il dar frecciate, cava il fangue ;
¢ per quefto dicendo dar freccia intendiama Dar freccia alla borfa , ¢ cayare que-
flo fecondo fangue , che é il danaro .
BELLY MORE , Huomo allegro , faceto , ec. vedi fopra in quefto C. ftan. 10.
Quando diciamo , Ii tale é Re della tal cofa ; intendiamo Vale in fuperlativo
grado in quella tal cofa ; onde Re de belli humeri yuol dire Grandiffimo bell’ hu-
more . Significato che viene da iGreci, i quali chiamavano Re colui, che nei
giuochi fanciuilefchi vinceva , ¢ fuperava gli altri, ed Afino ,o Mida era chia-
mato colui che perdeva ; il che pit ditfufamente vedremo nel 2. Canto .
: LANZO a due brachette , Lanzo dicemmo fopra , che vuol dir foldato Tedefco
a piede ; ma qui vuol che s’ intenda uno proprio di quelli della guardia del Sere-
} niffimo Gran Duca ; dicendo a due bracherte , perché quefti tali Lanzi vanno ve-
titi a Jiurea , con un paro di brache larghe , fatte a ftrifce , come fon quelle delli
*Svizeri del Papa in Roma , ¢ come quelle de’ Trabanti dell’ Imperatore .
INSEGNA trarre il molle dalle mezzette.. Infegna col {uo bere,come fi fa a vo-
tare i vafi pieni di vino, Che wezzerra ¢-un valo fatto di terra inuctriata , ches
ferue pee — il vino ,ed ¢ capace della quarta parte d’ un fiafto Fiorentino.

: ZA LIX STANZA LX.

. Morbido Gatti , Henrigo V incifedi Nell infegnahan ritratta uw huom canito,
A far venir innanzs ecco fon pronti Che troppo bavédoil crin(per offer vecchio)
I fanti , che ne da il Ponte a Rifredi, Fioccofo , ¢ lungo , un fanciullina aftuto a
Che mille fono annoverati ,¢ conti. Dietrogligrida:Gli abbrucia il penecchio.
Han certs Santambarchi fino a piedi , Da queffa febiera qui st ¢ pravveduto i
Che chiaman' il Rimbel di la da monti, Gran ceffe piene a’ huova,e di capecchio
E paton con la {pada in fu le polpe Con fafee , perze , ¢ tafte accomodate
Vn che facia lo ferafcicg alla volpe . Per farfi alle ferite le chiarate .

Pafla I ultima truppa di Soldati , la quale ¢ compofta d’ huomini dal Ponte a,
Rifredi , che € un luogo vicino a Firenze. Coftoro fon comandati da ALorbido
Gatti , C1Oe Migiotto Bardi ,e da Henrigo Vincifedi , che ¢ Vincenzio Federighi , due
gentilhuomini gia [colari dell’ Autore: E perché quefti fi pigliavano guito di ra-
gionare fpeflo con un tal Dottor Cupers , glicio fa fare per imprefa .
} _ A Quefto Dottor Cupers negli ultimi anni della fua vita, che durd fopra ot-
tanta anni , entrd in frepefia d’ effer bello , ¢ fi perfyadeya che ogni donna s’ in- :
namoraiie di lui , ¢ lo volefle per marito , ¢ pero andava lindo , e com la chioma
folta , ¢ lunga , ¢ ben coltivata ; ma canutiffima : onde i ragazzi quando pafiava.
per le ftrade gli gridavano dietro: Guarda il Pennecchio , gli abbrucia il Pennec-

thio s

    
Vas lle

48 MALMANTILE

chio , intendendo di detta fua chioma , ¢ lo facevano adirare ,-¢ maggiormentes
impazire . E perché li contadini del Ponte'a Rifredi fi danno a credere a’ haver
maggior Civilta degli altri contadini per efice nati , ed allevati, fi pud dire , nei
Borghi di Firenze , ed intorno alla Petraia , ¢ Caftello , Ville {petlo habitare>
da Principi della Serenifima Cafa , percid per Jo pili vengono ‘alla Citta col
ferrdiuolo , 0 fantambarco , che fono le Toghe de i Barbaflori , ¢ Dottori
del Contado ; e per quefto il Poeta dice Han certi Santambarchi fino a piedi, Che.
chiamano il Zimbel di ld da? monti , cio’ incitano i ragazzia dar loro delle Zimbel-
late. E per efler quefla P ultima {chicra fa , che ella conduca {eco il bagaglio
de i medicamenti per |’ Efercito .

SANT AMB ARCO, Specie d' abito , 0 fopravvefte , 0 diciamo mantello
ufato da i noftri contadini per difenderfi dall'acqua,e dal freddo ; ed € compofto
di due larghe ftrilce di panno cucite in forma di croce con una buca in mezzo ,
per Ja quale paflano il capo , e vengono coperti da una parte di detto panno les
{chienc , ¢ il petto , ¢ dall' altra le braccia , ei fianchi, Si dourebbe dire Saita in
barco , © cosi dice Mattio Franzefi nel Capitolo del {uo viaggio da Roma a,
Spoleto,

Gli offi, ¢ a profferir mai non fon parchs
Volean ch’ io {cavaleaffi a si mat re
E m offerivan fuoco , e Saltambarchi.

Ed é forfe meglio detto Sairambarco; perché quefto abito  compofto in tal
forma ; che tiene tutta Ja perfona difefa dalfreddo , ¢ non P impeditce il faltare
i foi , ¢ paflare i barchi. Ma fi dice Santambarco perché cosi lo chiamano i con-
tadini che fe ine feruono , ed ¢ lor abito proprio.

CHIAMAR: una cofa di la da i monti , Queflo termine fignifica Meritare una.
cola grandemente , come per clempio 4 tale € cosh injosente, ch' ei thinma le baffuma-
te di la dai monti, ff
. ZIMBELLO . In quefto logo intende un facchetto pieno di crufca ;
© di cenci , o di fegatura , legato a una cordicella lunga circa due braccia_ ,
col quale i fattorini delle botteghe de fetaiuoli nel tempo del Carnevale , quando

aflano i contadini per quei Iuoghi 5 dove fono le botteghe de i fetaiuoli , uno di
loro perquote i] contadino ; ¢ mentre quefto fi volta per veder chi ha percofio ,
gli altri ragazzi lo perquotono dall’ altra banda: B quefto per !o pili vien fatto a
certi ‘contadini , che fe ne vengono in Firenze intronizzati , ¢ in {ul grave, come
appunto fanno quei del Ponte a Rifredi. E per altro la voce Zimbetio ha il figni-
ficato , che vedremo fotto C, 7. flan.76.0 e

FAR Io ferafcico alla Volpe, EB’ una tpecie di caccia,che fi fa alla Volpe, piglian-
do un pezzo di carnaccia fetida , che legata a una corda fi va fira(eieaniso pes
terra; per far venir la Volpe al fetore di cfla Carne ; ed il Poeta aflomiglia i! por-
tar della fpada di quefti Contadini a quefta corda , dicendo che ttava pendentes
in [w le polpe ( cio’ dietro alle gambe , che cosi chiamiamo cotelta parte ) appun-
to come fta a fune di colui , che fa lo ftrafcico alla Volpe . sai”

PENNECCHIO, Quit prefo per chioma’, 6 Zazzera, come habbiamo accen-
nato fopra , metaforico da quell’ inuolto di lino , floppa; lana, © altra materia
fimile , che adattano le donne fopr'alla rocéa per filarey il quale inyolto fi dicc
Vennscchio. Ore.

i) 9 ihS

 
 

PRIMO CANTIARE: 42

¢ 5, cc api qui. La voce qui é fuperflua , baftando per farfi intendere il
dir folamente da guefta Regina fenza aggiungere 1a particella gui: Manon per

il noftro Poeta ha fatto errore, havendo feguitato il noftro Fiorentini{mo
wfatifimo ; Dicendofi comunemente ( forfe a maggior’ emfali) Quefto negurio qui,
quefta cofa che é qui, ¢ fimili; ¢ la particella qué efprime st megegioydel guale ragio-
niamo prefentemente 5 Quefia cofa , la quale habbiamo fra le mani: Anzi {timo , che
Tyhabbia fatto ad arte , ¢ per moftrare quefto noftro modo di dire, ( forfe ripren-
fibile ) del quale non mi pare, che in tutta l’Opera fi fia feruito mai pid ; quan-
tungue non gli fieno mancate I’ occafioni ; E fe bene nell’ Oicava 65. feguente— ,
pare , che I’ ufi nel medefimo modo , offeruifi, che quivi ¢ termine dimoftrative
neceflario , ¢ non riempitivo , operando che s’ intenda di quella Cugina , che é li
prefente , ¢ non d’ altra , come fi potrebbe intendere , fe non vi metteffe la parti-
cella qui.

CEST A, Intendiamo un gran paniere , che fa mezza foma di beftiajed ¢ con-
tefto d’ afficelle di caftagno , o d’ altro I¢gname a foggia di cafla , per ufo di por-
tare da un paefe all’ altro uova , vino in fiafchi , ed altre cole frangibili ; ¢ per Jo

it fon fabbricati due attaccati |’ uno all’alcro con quattro legni gagliardi aggiu
ati in maniera da adattarfi opra i bafti a traver{o alla beftia , in modo che ten-
gono equilibrate , ¢ ferme dette due cefte anche fenza legarle, Se ne fabbricano
ancora della ficla forma , ¢ materia {ciolte , cioé fenza i detti quattro legni, ©
s’ adattano , ¢ fermano in fu i bafti con le funi , come fi fa 1 Ceftoni , che»
ancor’ effi panieroni di mezza foma fatti di vinciglie di caftagno, o altro al-

bero inseflute de i quali fi parla fotto C, 10. ftan. 7.

‘C-APECC IO .. La pettinatura , ciot quella ftoppa pit groffa , che fi cava dal
lino {odo Ja prima volta;, che fi pettina :detta capecchio , perché fi cava dai due
capi del lino , cioé barbe , ¢ cime , le quali {ono pit ripiene d' immondezze , ¢ di
filo morto, e inutile.

FAR Ia chiaraca. M primo medicamento , che fi faccia alle ferite & P albume ,
“o chiara d@ huovo , entro alia qual chiara s’ iatigne il capecchio , ¢ fi pone fopra
alle ferite ; E quefto fi dice far la chiarara ,

STANZA LXh

“Ei general ai tutta quella mandra Lafcis gram rempo fale polpein Piddra,
Amofpante Laton Poeta infigne Mentre fi dava il are vigne 5
Canta improvvifo, come una calandra, Fortuna , che P havea matto provato

Stampa gli enigmi,ferolaca, ¢ dipigne Ville , ch’ ei diventaffe anche {polpate .

«> Generale di tutto quefto efercito ¢ Amoftante Latoniy cioe wAnromo Adalatefti
Pocta celebre per ieee fue opere, ma {pecialmente per quella Sfinge , la quale ,
come vedremo forto C. 8. flan, 26. ¢ una fcela d’ enigmi in (onewi y de’ quali (es
ben Ja ftampa ne fa goder pochi , fe ne fperava numero maggiore , volendone>
gli pubblicare goo. fcelti da una infinita , che ne ha compotti ; ma Ja di hui mor-
‘te {eguita poco tempo fayci priva per ‘ora di quefta con{olazione. Ne gli anni {uoi
Piovenili canté ail improvvifo molto lodatamente , fi dilewo d’ Aftrologia, e nel
-difegno fu fcoiare dell’ Autore , ¢ fuo amiciffimo , come mottra, facendolo capo,
‘e faperiore di cacti gli amici faoi,che aomina in quefto clercito . EB perché queito”
wimoftanre cra di corpo aduito , ed havea le o fori , dice 5 che /a/cio se poipe

an

 

 
«

50 MALMANTILE? %
in Fiandra , ¢ ché la Fortuna che’ t havea provato marto , volle:che egli diventaffes
anche /polparo , cioé fenza polpe; ma aggiunto alla voce marto vuol dire marro af=
fatto ; non che Amoftante fufle affatto privo di cervello ; che la voce marro a
pretio di noi fignifica ancora Allegro , Faceto , ¢ fimili, nel qual fenfo & ptefas
nel prefente Iuogo ; ¢ perd vuol dire , che Amoftante era huomo facetiffimo .
MANDRIA, Vuol dire Vna gran cuantica di bePies ma quirintende Granis
quantita d’ huomini , Mandra é voce Greca’, che fuona 'Spelonca + ¢lnogo, eri
tro al hs le pecore s’ adunano all’ ombra ,ma la pigliavano anche per la gregs
ga medefima , ¢ da efla diflerd Archimandrita eee meeps grepgias
ante pure prefe Adandria per quantita di huomint , nel Purg. C. 3.
Si vidd' io muovere , ¢ venir la tefia
Di quella Mandria fortunata allorta ,
Paudica in faccia , ¢ nell’ andare onefta ,
CANT A improvvifo . B coftume in Firenze al tempo de i granealdi la notes
cantare dell’ ottave all’ improvvifo , mentre ne i luoghi ey aperti-della Cictafi
va pigliando il frefco ; e perché in tal’ efercizio wave molto il Malatefti; il —
ta l' afomiglia alla Calandra uccello di bellidimo cantare.
ENIG AU. Indovinelli.Voce Latinogreca.V edi forto0.6, Nan.34.c C.8Man. 26.
LaSCIO' le polpe in Fiandra, Non é, che Amoftante futile mai ftato ins
Fiandra ; ma , perché lo fa generale di quelto efergito , ¢ dovere,' che egli mo-
firi , che Amoftante ha vedute,, ¢ provate dltre guerre ,¢ che egli fi fa trovatoa
dar de’ {acchi ne i quali ha Jafciate le polpe delle gambe;, il che Pea
ditarlo 5 2 peenghe fi come ad un foldato gli ftroppj , ¢ le cicatrici fon di gloria, cosh
ad Amoftante era di gloria  haver perduto Je polpe delle gambo adie guerre di
Fiandra ; ma il vero¢ , che quand’ uno hale gambe fortili , diciamondi lui: Reb x
ha la{ciato te polpe in Fiandra’: ed il Poeta con quefto equivoco, che accreditas i
Amoftante , vuol dire , che egli haveva le gambe fortili ; ¢ feguita con I mages E r
equivoco di e#ateo /polparo ; che fignifica ; come s'¢ detto » matto del tutto ,
-vuol che s' intenda /exxa polpe affasto . E la voce polpa , che fignifica oe ez
quantita di carne , che fia fenz’ offo , da noi fi piglia/perle pole
quando ¢ detta aflolutamente .( Vedi T ottava's9. antecedente ; E forto al oe &
Sian. 99..dice ofaccia fenza polpe che s' intcnde tuttala carne di quel’ corpo )¢
fignitica pure Afatro fpacciaro ¢
STANZA LXII,

 

Paffati tutti con baule , ¢ {pada ch chlo biftraca ye come che ne vada
. Serranfi in barca , come ia fardelle ; Gilt la vinaccia,e il fangue a catinelle 5
Gimil pan aaindamnrie « Eben eee eevanatens ae wud
O ferma un palo ; guai alla fua pelle Aion gli da tanto tempoch’ ei refpiris
Dopo fatta la mottra fe Renta Ja pide pulebarche con ogni {uo arncle,
¢ Baldone affretta all"imbarco i fold; »255 a wees i
S4VLE, Vorendiamo ogni Gdeate i Gates waligia., o:-tambato, che : (
mente fi pofla adattare in fu yale ae d'un cayallo , mentre wife
’ dal verbo bainlo, ¢ I allarghiamo ad ogni fortardi cafla portatile-inda le fome, €¢

Qui intende quell inuolto., che oe i foldate fopr' alle reniiper for preere
bagagho ,\detro alert! zaino ie oli 72, tb?

a SER-

 

 

 
 

PRIMO,CANTARE, “st

SERRANSL,, come te fardele.. Si {errano ftrettiffimi appunto , come ftanno le
Tardelle ne i cefloni, , quando da Livorno (on portate a Firenze , 0 nei bariglio-
ni, quando ¢ivengono falate. Comparazione aifai ufata per intendere Aretti , ¢
ferrati infieme , che in voce marinarefca fi dice stiuati ,

_ TENERE a bada., Tratteacre uno. Varchi ttor, lib, 4. Conofcevano y che erano
tutte cofe finte ye folo per tenere a bada trovate, Viene dal Verbo Badare, che has
molt fignificati, Zadare al negozio per Artendere al negozio. Signitica Indugia-
re ,0 perder il tempo , come ¢ intefo nel prefente luogo , che dice tiene 4 badas,
ed intende, Chi gli é caula d’ indugio,o gli fa perder tempo ; il Petrarca Son.23.

Confolate lei dunque , che ancor bada ,
| Cio’ afpetta 1a venuta del Ponteficese perde tempo . Significa ancora covtizuare,o
| Leguitare a far una cola, Vedi forco.C.1, ftan. 20. Sigaitica Oferware C.9. fan. 28.
| Significa Di/prezzare, non curare , per elemplo ; /o non bade al tuo gridare, latende
jo non ftima , onon curo il tuo gridare , Da quetto badare 50 bada habbiamo badaloncy
che.vuol-dire Vin’ huomo perdigiorno , ¢ che non (a, ¢ non vuol far nulla.

GV AL alla fua pelle, Mai per iui. Vedi fopra in quefto C, tan, 28,

BIST RATT -ARE , Tratcar male’, Strapazzare , o Stranare ,

VeA giit 1a vinaccia . E! neceflario tar pretto per sfuggire il danno , che fi pati-
Ace ¢ che fi teme pid grave dali’ indugio . Quando il mofo , cioe ib liquore ca-
vato.dall’uva , il quaic ¢ nei tino, ha boilito a battanza ; perde il vigore, ¢ non
puo pil foftenere a galla , cioe nella {ua fuperficie , la vinaccia ( che cost Gi. chiar
mano iralpi y¢ bucce-deil’ uve ) onde Ja Jafcia calcare in fondo , ed incorporan-
dofi_ con i nuovo, fi guata; B quefto fi dice uxdar git la vinaccia ; ches
poi paffato in proverbio figaitica Quel che habbiamo detto .

NE vail nea catineke, Ne " imnolto Ack mia Per intender, che Vn’ in-
-dugio apporta grave difpcndio , ct feruiamo di guefto detto ; ¢ fi dice anche : #
Soon Vedi Se c pain. 20. ;

4ESTO . Qui vuol.dir Pronto , ed all’ ordine,

WON gli-da tempo che refpiri... Non gli la(cia ripigliare il fiato .. Quefto detto
e(prime ua grande athrewamento , 0 incalzamento.

h STANZA LAL,

Percid imbarcati tutti in-un momenta y

STANZA, LXIV.
Rifiede Malmantil four’ un pazgerto 5

Poi che Baldon faces cost gran ferra,  chiungue verfo lui voltale cizlia
i Stfpiegaron Vinfeznd, ¢ veie al ventas Dice , ch'i fondatori hebber concerto “
e Qu ~~ Di fabricar L ortava meraviglia ,

te Navi fi fpiccar da rerra;
Ed egh: allora entro in ragi: 0
Di quel che lofpingevaa far talguerra;
Ma per contarla pik difte/a,e pranay
Incomincid cost dalla lontana.

L' ampro pacfe poi,ch' egli ha Seggetto
Non Fite, ginocare,a milie miglia ;
Ve laria buona azrurre oltramarinay
E.non vi manca latte dt gallina,

Fatta la moitra , ed imbarcate in brevitiimo tempo le foldatelche, fi partirono
Je Navi dal Jido ¢ fecero vela fpiegando Je loro infegae.. Intaato Baldone da

_ oprincipio a nareare la caufa, che lo muove a far la guerra di Malmantile 5 ¢ co-

-amincia dal delcrivere Ja fituazione , quaiita , ¢ dominio.
. FAR ferra. Affcettare. In,alzare. Vedi foro C. 9, flan, 13.

CONT ARLA dife/a,e plana, Intendi , Raccontarla punwualmente , ¢ cORe
G2 VON

~
: 35
* Bet

tutte le circoftanze ,

 

 
 

gu MALMANTILE

NON fi [a wvo gixocare a mille migia . Io giuoco , che non fi trova chi fappia
© pofla giudicare a mille miglia , quanto paele gli ¢ fuggetto ; perché é cosi gran
paefe, che mille miglia non fi confiderano , eflendo paruita di numero , ¢ di ma~
teria in riguardo del tutto , che gli ¢ fuggetto . E quefta voce /uggetto > che vuol
dir fortopoffe , s' intende Situato fotto , ¢ non fottopofto al dominio di Maimanti-
le, che per effer Pofto nella fommita d'un poggetto , ha d’attorno molta pia-
nura , € colline fortopofte , cioé pili baile di lui; fe ben par, che voglia dire, che
Malmantile ha dominio immenfo .

ARIA axzurra oltramarina . 1 pittori dicono buon’ aria quella , la quale ¢ co-
lorita con l’ azzurro oltramarino , perché quefto non perde mai il colore , come
perde I’ indaco , ¢ lo {maito ; ma € pero anche vero , che quando I'aria fi vede di
colore azzurro , come é il buono oltramariao , @ fegno , che é purgata da ogni
imperfezione di nebbia , o d’ altri maligni vapori, ¢ per confeguenza ¢ aria buo-
na ; il Poeta perd dice , che a Malmantile ¢ aria aczurra oltramarina per intende-
re, che a Malmantile ¢ aria , che dura fempre azzurra , come fa quella colorita
con I’ oli ino , cioé fempre buoni! ~ BP oft i € quel colore »che fi
cava dalla pietra detta Lapisiazzuli .

NON vi manca latte di gallina, Vi {ono tutte le cofe {quifite,t abondante dogni
bene . Detto antico , fi come fi cava da Strabone lib, 14. , dove difcorrendo del-
lc campagne di Samo dice, che crano cosi fertili , che fi diceva comunemente ,
che produceffero fino il latte di gallina , ciot quelle cofe , che ¢ impoflibile , ch’
altrove fi trovino , come é i] Jatte di gallina. Samus , dice egli , feracifima, unde
laudantes non dubitaut illud ci proverbium accommodare » quod fer i i
lac , ec,

 

 
 
 
  

STANZA LXV, STANZA L

Ut Re di quefto ‘Regno giunto a morte Gobba, e xoppaé coftei, € mancina,
La mia Cugina qui , che fu fua Donna Ha ilgox20,e da due sfregi il vifoguafte,
(Non havendo figlinoliza altri in Corte Storfe in Firenze ognor lacavallina
P. opingui pik)lafeio donna,e Madonna: Ae i lupanari con grate pompaye fafto s
Ata come volie la {ua prifta forte , E perch ofequij havea fera,e matting,
Vn certo diavol a’ una Mona Cionna E il ritol di Signora a tutto pafto , )
Figlinola d'un guidone ignudo, ¢ (calzo Patra arrogante, al fine alzs il penfiero
Ne venne preffo a farie dar (0 shaizo. eA voler quefis onuri da dovero.

Narra Baldone , che ii Re di Malmantile inflitui Celidora erede del Regno, €
che quetto Je fu ufurparo da Bertinella , la quale defcrive per una donna tuttas
contraffatta,, ¢ Ja moftra una vera fgualdrina : ed imita Dante nel Purg. C.19.
che dice . ;
Hi venne in fogno wna femmina batba ,

Con gli occhi guerci ,e fopra i pie diftorta ,
Con le man monche , e di colore (cialba , i

Qui & da confiderare, che i tanti diferti da Baldone attribuiti a Bertinella y
realmense im lei non fullero, perehé, ed egli non fe ne farebbe innamorato,come
fi dice {ouwo nel Cant, 9., ed ella non haurebbehavuto tanti altri amanti; Ma.
Baidone non I havendo mai yeduta , e voleado concitar contro di lei odio di
quet folaau , che lo feguivane , per uftigargli ad andar pil volensicri alia ricupe-

razion:

Bae oe ee

 

  

 

 
ee: ~s on =. ‘Sa ot

PRIMO CANTARE: 53 ’

sazione di Malmantile , la rapprefenta loro una donna cosi nefanda -
SV-A donna, Sua moglie , Se bene i Poeti dicendo La mia donna, o La fua.
donna , intendono I’ amata . "
LASCIO’ donna , ¢ madonna, Termine notariefco , e curiale , che fignifica Pa-
drona affoluta. Sincopato di Domina . é is
VN certo Diavolo , Si dice cost quando vogliamo efprimere uno , che é cagione
di qualche noftra difgeazia : per elempio : / negoxio andava bene , ma un certo dia-
volo d' un Senfale con le fue chiacchiere lo rovine quali dica » di diavolo,che guafto que-
+3 fo negoxio , fu un Senfale .
MONA Cionna, Bun devto di-difprezzo , che fignifica Donna da poco in
ogni operazione : ed il fenfo della voce Mona , Vedrai fotto C. s. ftan. 18.
GVIDONE . Intendiamo huome vilifimo, abietto, fenza roba , ¢ fenza crean-
t za, 0 riputazione . ‘i %
| DAR (0 sbaizo, Mandar via ; Scacciare ,
} ORO . In queito luogo vuol dir Vno , che vede poco , che noi chiamiamo
lufco , fe bene il {uo vero fenfo ¢ dicieco affatto. Vedi fopra in quefto C. flan.
g. alla voce sbirciare .
MAN. INO. Vno che per affuefazione ha maggior forza , ed attitudine nel-
la mano finiftra , che nella deftra ; E perché quelto tale fi pud dire djfettalo ;
—_ huomo mancino., vuol dire Huomo non buono ; ed in quefto feafo ¢ pre-
fo nel prefente luogo . E perd voce che ha del furbelco . Se ne fervi il Lalli nella
fua En. trav. nel C.2. flan. 4a,dicendo ,
. Perch’ io gan fui mai orbo , ne mancina,
Edal C, 4. flan. 67.
E rinfcito in famma un buom mancine ,
Vina delle pitt vili creature
: 0" habbia fio mondo ; ¢ paxxo da catena ;
HA il goxxo , EB’ parola nota , venendo dal latino gattar: Ma qui vuol dire
wn gonfio, o ferafa, che vien nella gola, che i medici , che {crivono di fiuil 4
{ male pongono al trattato il titolo de Boccis .

SFREGIO, Cicatrice di taglio nel vifo . Ed una donna sfregiata ¢ numerata,
fra le infami , e per la deformita del volto , e per la caufa, per 1a quale fi {uppo-
ne , che Je fia ftato fatto. Vedi forto C, 2. ftan. 3. dove fi moftra eller tali stregi J
vituperofi anche negli huomini , ed al’C, 6. ftan. 54.

SCORRER /a cavallina. PighiarG wutti li (voi guifii liberamente , ¢ fenza riguar-
do alcuno . Havere /corfa la cavailina ne i lupanari , vuoi dit , che era merctrice
vecchia , ed avanzata ai bordelli , ¢ lupanari. Gli antichi Egizj , quando vo-
levano efprimere la sfacciataggine meretricia , figuravano una cavalla fenza fre-
no ; il furore della quale nelle cofe Veneree efprime Vergilio 3, Georg. dicendo .

_ , Seilicer ante omnes furor eit infignis equarum . :
| » TL titol ds Signora a tutto pafto , Cioé continovamente era chiamata Signora . &
F _ Termine ufatiffimo per intender voglia cofa , che fi faccia molto,¢ contino- ‘
\yatamente’. Il Mauro nel Capitoio in lode della Torniella dice , :
2 » . Eragionidi voi a tutta 2
Daddovero. Per debito : Per giuitiziay Per merito. Intendi che volle proccu-
ror

 

 
34

STANZA LxVIL
Cost la mira ad alto havendo meffa s

A fuoi Fruftamationi ua da vicorfa,
Bramar dive una grarsa ye che in effa
Non fi tratta di feorporo di borfa ;
Ma, perche afpira a farfi Principeffa
Defidera da loro effer fovgarfa
Col loro aiuto ,volendo , € configtia.,
Provar,va Malmanil puo dar di piglo,

'MALMANTILE®

rap d havere ftato , 0 Ggnoria per meritare il titolo di Signora y-¢0, ed ,
che quel.da dovero non ¢ la yoce vero con!’ aggiunta della fillaba do ; ma ¢ il no-
me dovere mefio in ufo di dirlo cosi corrottamcnte in cali fimili.a quefe » © per
eiprimere una ¢ofa di doverey0 doverola , ¢dovuta ,¢ gina .

 

SLANZAY LEVIN,
Pronto ciafcunose vuoltra miile frocchi
Efporre si ventre, <ameun Paladino: y
Che per fervire a Dame , .raii allocchs
Cercan l occafion col fufcetlino;
Ma eordi pas » otratti ds baiocchi ,
Lerche no hanounbeccod'nn quactring;
Ecredon , promettendo Roma,e.Toma,
Di {pacciar P oro dela bionda chioma:,

 

 

Bertinella havendo facta la fuddetta rifoluzione y richie(e li {uci amantiy:che
Ja voletiero aiucace a farfi Pringipeda con mpadronirfi di;Maimantile , ¢di.taoi
Deudi 8 ¢fibifcono a (eruirla, perché fentono di nou haver a (penderes il che &
cercato da tutti coloro , i quali con fimil donne pretendono di. patiar per belli5
che ¢ una delle tre (pecie di perfone , che voglion quette femmine dntorao ,\cioé
1/ belio per fua propria (odisfazione . Ll bravo per farfi riipettare. Bd il ricco
minchione , 0 corrivo , per cavar danari da lui, per campare fe medelime 5 .edé
primi due, I) Perfiani dice, vd

Ul bravo , ed il corrivo , ed il valente «

Nella mia Mea fallifee 9 ctoae

Quefto antico detratoy

Per c' al bravo , ed al bel nom apparifee 5

Ma fol vorredbe il fuo minchione allAtone A)»

PORRE ad alto la mira, Alpivare'a cofe granti. Aira fi dice. quel fegno ,» che
é nella canna dell’ archibufo y o nelle baleftre, nel quale's' affitla |" occhio per ag-
giuftare il colpo al berzaglio, E di qui Porre La mira a unacofas’ intende Volgere
al penfiero , 0 afpirare a una cofa, i hema Hm

£RVST AMATTONL, Si dicono Quelli , che giornalmente vanno in una
cafa , 0 bottega , enon vi fpendono mai ud foldo y o vi portano. utile aleuno,
E Gi dicono Fraftamatroni, perché non fon d’ altro giovamento y che frustares,
cioe {pazzare , ¢ ripulire con le i marroni ; iquali fon quelle Jaftre fatte di
terra cotta , con Je quali fi laftricano i pavimenu ‘delle ftanze ».da1Jatini detti
Lareres:” s eg

SCORPORO di borfa, Spendere.. Scorporare vuol dit Eftrarre dauna mafia,
da un corpo , o quantita > O.una porzione di efla.. Ain hs cogent

DAR di piglio .. ln quelto luogo vuol dic Pigliare', impadronirGi ; ed.alle volte
vuol dir Principiare come fotco C.6, fans. weno S

ESPORRE il ventre a mille flocchi, Vanti a’ innamorati d’ andare-foli contro
aun’ efercito:intero , come i Poeti fa iano, che faceflero i Paladini , che
fono quei dodici Conti di Palazzo , ordinati da Carlo Magno per combatrere
coatro ai nimici della S, Fede Cattolica , che furono detti Comites Palatini , cioe
‘Compagni nel Palazzo , che fono forle gli odierni Pari di Brancia’: the noi poi

‘cor-

 

 

 
Te

PRIMO'CANTIARE: 35

corrottamentechiamiamo Paladini, ¢ con quefta voce intendiattio. Haomé bravo.

ALLOCCO . Specie d’ uccello con il capo cornuto, come |’ affivolo, ma ¢

| pit grande’, ¢ di colore lionato , con occhi grandi , ¢ lucenti, E* animal goffo ,

¢ (e bene vive di rapina , tuttavia é tanto poltrone , che per cibarfi afpetta di pi-

gliare gli uccelli »quando gli vanno {cherzando atrorno , tratti dalla di lui gof-

| faggine ; ¢ quando {e:liavvicinano,, non con rapacita , ma'con flemma , ¢ gra-

} vita non ordinaria gli prende col roftro , o con gli artigli ; EB da quefta goftaggi-

ne nel far all*amore , ed afpettare gli uccelli , per-Allocco intendiamoV no ,. che

fe ne ia perdendo il giorno in vagheggiar Damerfenza proficto. 5 edȢ Lo fteflo

che Fruffamattoni , Colombi di geffo , ¢ fimili .

Con nome -di/occo in molte parti d’ Italia é chiamata ancora la Civetta ,

| e credoy perché ¢ di figura, fe ben pili piccola ; fimile a quella dell’ Allocco , e>
vive con Ie medefime arti , ;

-CERC-AR cof fufceline, Cercar minutamente ye con diligenza ; \ft tale cerca le
buffe col fufcelline yyuol dite ; Il tale fa tutto quel che egli puo, per effer percoflo ,
© per toccarne + Quefto detto vien da quei ragazzi dell’.infima plebe , i:quali do-
po.che &venuta in Firenze una gran pioggia’, che habbia fatta- corer I! acquas
per la Citta’, vanno.cercando per le ftrade vicine alle gran fogne,che portano in
Arno , fe trovano fra le commenticure delle laftre delle ftrade fpilli , chiodi 5 ed
{ altre cofe fimili portate , ¢ la(ciate quivi dall' acque correnti ; ¢ per far cid fi
{ervono d’ uno: ftecco , o fulcelletto di opa’,o d? altro’, col quale: vanno rifru-
gando bfelli di dette commettiture 5 perché cos gran diligenze fon troppe al
poco utile ,n°¢.natoiil fuddetto proverbic , checha I’ acceanato fenfo', ed ¢ lo
itefio:che chiamar'una cofa di 1a da i monti , detto fopra in quefto C, ftan, 19.

- BALOC CO... E parola , ¢ moncta.romana’, la qual parolaé talvolta ufata da
noi per intender Danari y come qui , che dicendo Won fi park di baiocchi intende
Won fi parli di danari , cioe di Spendere .
<0 NOM hanno un becco d'4n quattrino, Non hanno pure un denaro , ¢ quella pa-
rola Beceo fi metce a maggiore elpreifione , quafi dica Non hanno ne pure un fol
quattriuabecco ; cioe cattiyo , ¢:non il cao’ a {penderfi ; Se non voleitino dire,
che-veniffe queito detto dail’ antica moncta Romana di rame ; pella:quale era im-

da uoabanda il yolto di Giano.con le corna , ¢ dal’ altra un toftro di na-
we eche al dire ; Wa-becco d’ un quattrino fia lo fteffo 5 che dire , ne-anche la
ete di pio’ la faccia di Giano , che é cornuta.

. BROMETTE Roma ,e Toma , Promette cofe grandiffime ,\e che da perfonas
alcuna non fi pofiono mantenere,o offervare; i Latini ditiero Adaria,> Afomtes polli=
eer, La voce toma non fo che habbia nel noftro idioma fignificato alcuno, ¢ fti-
mo; cheyfiaulaca in ydetto per darle Ja rima con la parola Roma; Se for.
denon bo fpagnuolo somar,che yuol dir torre: ,.0 pigliare , ed intender-
fi Ti prometto Roma , ( che ¢ a dir tutto il mondo ) e ru toma , ciok piglia quel che

refhava, di follecitarla promettendole Koma ye toma y
det mondo, ;

   

piace. Lafea Nov,
 igome fe gli fulfe il primo

a <8, ‘Perd-non

BAYS t ¢ 2 cS

STAN-

 

 

 

 
 

so MALMANTI LEB)

2 STANZA LXIX,

Eva tra molt {noi piie fidi amanti
Vn ciarla,che peré detto 8ilCornacchias
Ed ¢ di quei pittor , ch’ i viandants
Con lo ftioppo dipingono alla macchia ;
E perche nella linguaha ilfuoincotanti,
Molto fi vitayal]ai prefumesegracchia ;
E finalmente colorifce , ¢ tratta

STANZA‘ LXX.
Scrive un viglietto poi fecreramenté
Ad un compagro {uo capobandito y
Dicendo , che weduta la prefente's
Li {uo bagaglio fubito. ammannito 5
Di norte tempo meni la fua gente
A Rimaggio alla Svoita del Romito ;
Ata vada allafpeRrata, ¢ pei tragetriy

 

 

« Quefto negorso , come cofa fatta , E fenza penfar' altro svi L afperti ,
65 = STANZA LxXxX fe a
wind la carta , e queic' bebbe L inte/a,

Come quel ch’ inuitatoera alfuogiuoco
Andonne , ¢ guido feco'a quel? imprefa Anch eglino con groffia ye folta {chiera.
Cent’ huomin con le lor bocche di fuoco , D’ una gente da bofco ye dariviera .
. Fra quefti fuoi pil: fedeli amanti era un tale detto il Cornacchia . Coltui eras
uno con tal foprannome ; perché havea la voce d’ un fuono fimile al gracchiare>
della cornacchia , ed cra un folenniffimo briccone , ¢ ladro , € fpia.. Quefto daa
Bertinella il negozio per fatto, e s’ ammanni(ce a far ja ite di Maimantile;
con fcrivere ad un capo di ladri da ftrada fuo corrifpondente , che fi conduca a.
Rimaggio con l¢ fue genti.con armi , ¢ panni, e 1 afpetti alla Suolta del Romi-
to, che @ una contrada in vicinanza di Malmantile. Bfcgui )amico , giunfes
‘con cento huomini ben’ armati nel luogo ordinatogli: fra poco vi arrivd ancora
al Cornacchia con Bertinella , con grande fchiera di bravi furbi, che quefto in-
tende gente da bofco , e da riviera; che i Latin diflero bomines omninm horarem
CLARLONE , Vno , che chiacchiera aflai, L'Autore intende, che chiacchie~
rava afiai alla giuftizia , cioé faceva Ja (pia, e percid detto Cornacchia , che ¢ uc
cello di cattivo augurio ; perché il fuo ciarlare era didanno al profimo. Ed in
vero cofivi , mentre vifie , fu fempre chiamato i} Cormacehia 5 o per quefta.cauia,
© per quella che habbiamo accennato fopra, >
DIPINGERE alla macchia, Dipinger uo Ritratto fenz’ haver d'avanti-l ori
ginaie yma col folo-haverlo veduro, £ 1’ Autore pero intende 5 che egii era la-
dro di flrada., ¢ pigliando Ja voce macchta nei {uo vero fenfo di {elua denfa s dice,
che alla macchia ritraeva i viandanti cen lo frioppo, ed intende Afialtava la gente alia
ftrada con l’archibufo per rubarla , Quefta perd é finzione,perché ii Cornacchia,
fe hebbe la malizia,non hebbe gid tanto cuore di far’ il ladro di ftrada 5. ¢ P Auto-
relo finge tale per moftrare 5 che egli era un furbo da far qualfivoglia fciagura~
taggine . 4 ; ; :
tA nella iingua il fuo in contanti.. Vuol dire eloquénte , pronto dilinguay . :
VANTeARST, Prometterfi molto di fe: medefimo, Elakar-le propric-opere ,
éil Latino Zatare. wij uy" 5584), , a
. GRACCHIARE , Cicalare con poco fondamento , Vedi forto C. 4. flan 2g.
C. 7. ftan. 9, ¢ C. 8. flan. 65. Ma eoftui'é chiamato Cornacchia, il Poeta fi
ferue del verbo gracchiare per efprimer il cicalar di eflo. +
COLORIRE , Metafora affai ufata , ¢ viol dire difcorrer d’una cofa con ag-
 Biwfatezeay con termini proprj , ¢ con colori rettorici per perluadere, ¢ fare»
apparir

Kuiviil Cornacchin,e quedta bnonafpefa
Li Bertinella giunfero fra me ‘

J) ee a ee

   

 
PRIMO'CANTARE: 57

lapparir vera quella’ tal cofa , della quale fi difcorre . ?

VIGLIETTO., obigiietto. Vuol dir lettera ; Ma ftrettamente fignifica quella.
lettera ;,chefirmanda in luoghi vicini , come da una cafa all’ altra , dentro alla
amedefima Citta, o Terra , Voce che forfe viene dal Francefe Poulet , che vuol dir
ettera,amorola, o1da Bidet, Vedi {otto C. 6. flan. 54.

BAG AG-LAIO. . Quelle fome, che fi conducono appreffo gli eferciti per utile , ¢
«comodo dell’armata ,.0 dictro qualfivoglia viaggiante per (eruizio della propria
periona.; fi dicono Sagaglio , forfe dal Francefe Bagage ; 0 dal verbo Bainlares ,
che val Portare , come habbiamo offeruato fopra in quefto C. flan. 62. alla voce
Baule , ed é quel che i latini dicevano smpedimenta .

} AMMANNIRE , Meter’ all'ordine, Allefire, approntare ; quai dica 4d
manus babere . Dante Purg.'C, 23.
Di-quelich’ il Ciel veloce loro ammanwa,
edalC.r9, Lavirtit yc! a ragion difcorfo ammanna ,

| ALLA (pexzata., A pochi infieme per volta, non in (quadre o'teuppe Forma-
te. Sidice anche. dda sfilata, Vedi fotto C. 6.{tan, 85. ¢d ¢ il diminutins dei latini,

PEi tragetts , (Per le balze , per luoghi., ¢ ttrade.non praticate ; ¢ il puro La-
tino Traiectus.. ?

HAVER Vintefa Rit d’ do.Haver l'inftruzione dicome fi debba cé .
wT AR uno al fuoginoco, Chiamar’ uno a fare una cofa , che fia di fuo ge-
nio ,.¢.guito.d Latinidiflero Adu/as hortari ur canant , ec,

. °BOCC HE: ds fusco.,. Intendiamo Ogni-arme da fuoco , :atta a portarfi addoffo ,
«come Mot iy archibuli,, piltole , ¢ fimili,
BVONASpe/a.-Huomo aituto,¢ (caltrito,e{uana lo fteflo, che Trifto , e Vol-

 

ype vecchia.. ‘
STANZA LXXIL STANZA LXXIIT.
Dopo ch’ infieme tutti fur coflore to foc’ aun ignorante , aun idiora
. Si fece de’ pi degni una femblea , L’ effer il primo a favellar non toca;
Del. come difeorrendo fra di loro 41a perdonate aqueftarucca vora,
Sorprender’ il Capello fidovea-, Signori,s'io-vi-rompol'hvovo in bocca;
Ond'it Cornacobia in mexzo al coviffora Scricchiolafempre la pitt trifha ruota ,
| « iRixzaro'in pit con gran prefopopea , Cost la lingua mia pit-rorza, ¢ feiocca
} Ed'una toccatina di cappello y V infaftidifce ,¢-ver ma v afficura
| » dn.eal modo cave fuora il limbello.. Che Adalmantilee nofiro a dirittura.,
i coftoro infieme , quei pit degni fi riftrinfero a contiglio , per fermar
al y he fi doveva:tener per forprender Malmantile , ed il Cornacchia , fac-

te {uecirimonie, comincia a moftrare il.modo certo di pigliare detto Malmantile.
PRESOPOPE A. ‘Quefta voce , che vien dal Greco Profopopea compoftasdi
due dizioniPrefopon , che fuona perforam (ed a noi Perfonaggio ) ¢ jpoceo., ches
fuona facio , feibene ¢ una figura con la quale fingefi un perlonaggio, come fareb-
be introdurrewna cola inanimata., che-parlicon una animata., & contra, tut-
-tavia noice:ne feruiamo per intender’una‘certa {uperbia., -arroganza ,, ‘fasto., o
refunzione di fe medefimo., dimottrata con gli atti ; diche vedi {orto C.6. ftan.
5. Ed in‘tal fenfo., fecondo il. Monofino era pigliata ancora da.i Greci... Si dice
da noi anche fufliego-, derivando la voce dallo Spagnuolo.
: VNAtoccarina ds cappello.Atto che e(prime detta Profopopea. =H Cate

   
58, MALMANTILE

CAPO! fuora il limbello. Comincid a parlare . Limbelli ; Si dicono quei pezzi
di pelle di beftia , che dalle dette pelli tagliano i Conciatori', donde poi démbel-
/ucci i ritapli delle pelli pit: fortili , come di cartapecora , che feruono'per far col-
Ja da Pittori . E perche tali /imbel## , quando fon frefchi ; ed umidi fono fimili-alle
lingue , percid per /imbelfo'intendiamo lingua ; ¢ perd detto fcherzofo, come fi
vede , che !'usd il noftro Autore anche fopra’in quella fua lettera alla Serenifs,
Arciduchefla , riportata da me nel Procmio . Cave fuora il limbello, ¢ diffe le fue»
Siliabe , come un Tullio , ec,

IGNORANTE , & fdiota , Sono Sinonimi , ne vi fi fa alcuna differenza’, fe»
bene ftrettamente /gvorante yuol dire uno , che non {a nulla, ¢ /diora par che fi
conuenga a coloro , ch¢ non hanno cognizione di lettere. - .

ZV CCa, S intende il capo del’ huomo per la fimilitudine |, ¢ Zxcca vera yuol
pe dire tefta fenza cerucilo , ‘che fi dice vera di fale, 0 non baver fale in zucca .

quefto perché ¢ folito nelle cucine tenere i! fale in’una Zucca fecca-appefa’al
muro del Cammino . Vedi fotto Can. 4. flan. 15. 1 Latini pure dicevano fale per
giudizio , ¢ trovafi in Catullo , : bake

Walla in tam magno corpore mica falis
Vedi fotto C. 8. ftan. 26. , ¢ Marziale C. 7.
Nullaque mica falis , nec amari fellis in ilfis

ROMPER t hvovo in bocca. Torre la parola di bocca a uno, cid ¢ Dire ‘che
doveva , o voleva dire un’ altro. Terenzio'difle Bulus ereprus ¢ fancibus eff .

SCRICCHIOLARE . Stridere , ftrepitare « 8" intende quel romore , che fas
nel muoverfi un legno fortemente ftretto , o aggravato da altro legno , o mate-
riale duro ; come appunto fegue nelle ruote da carro. Ed il proverbio: Sempre
Sericchiola la peegio rugts del carro, Significa 7/ pitt fetocco della conuerfaxione , vnol
Jempre parlare, Detto antico , ¢ vien dal Latino , che dice femper deterior vebicu-
4 rota per ferepit , ec. R

 

A DIRITTVRA, Civ’ affolutamente , ficuramente, e fenza difficulta aleuna,
s

STANZA LXXIV.
Credete a me: Ciafcun fi lia nafcofto
Fn quefke macchie,in quofti bofchi intorne
Ed io da vor fra tanto mi difeofto ,
We quefta notte faro pit ritorno .
Rinedremci cola doman ful pofto,
Perché vicino aj tramontar del giorna
Vi faro cenna , bor voi ponete mente,
E poi venite via allegramente
STANZA LXXV.
Parte il Cornacchia , ¢ corre prefto prefia
Da corti {uci ami epithet 5
Da! qualt le lor beftie piglia ix prefto
2 covtch pla foes bona oh ,
E di [opptatto , come fante lefta
Cave di tafca certi cartoccini
Preni a’ alloppio, e dentro al vin li pone
Quelle impepando , fenza difcrerione .

   

TANZA LXXKVL
Cosh carreggia , ¢ gitmto a Malmantile
All’ aprir della porta la mattina
Scarica in piazza il vino, ed ut barile
A regalar ng manda alla Regina ,
Poi vende il resto a prezzo tanto wile y
Cognit necopra,e in he che n'haincatina
Per rivenderlo altrui, il fiafco attacca,
Si cala al buon mercatoya quella macche
STANZA LXXVIL
Due , 0 tre fiafchi davane a quattrinoy
Ed a’ poveri davalo a Lonne 5
Tal che tutti suffandofia quel vino
S' imbriacaron come tante monne 5
E fubito dal grande al piccaling
delle dine

Tanto de gli huamin, "
Cafcaro in fannolenga si gagliarda
Che defti non Guacilleane bebarde:

 

1

 
 

 

PRIMO CANTARE, 59

= Cornacchia inftrui(ce i compagni di quello devon fare , ¢ fi parte , ¢ va da,
certi contadini fuoi amici , da’ quali piglia le lor beftie in prefto , ¢ Ie carica di
vino alloppiato , quale porta in Malmantile, ¢ Jo vende cosi a buon mercato, che
Ognuno ne comprd,e bevvero tanco,che tucci s’ imbriacarono,¢ fi meffero a dormire
~ PRESTO prefto. Prettitfimo : per la replica d’ una ftefla parola, che ha forza di
fuperlativo, come habbiamo detco altrove , .

Di foppiatto. Di nafcofta, Vien dal verbo impiattare , che vuol dir Nafcon-
dere unacofa corporea , coine s’ ¢ detto altrove ,

FANTE lefo.. Huom (agace , aftato , ¢ che fail conto fuo.

CARTOCCINO . Diminutivo di Cartoccio,che ¢ una piegatura di foglio,fatca

a Piramide ufata da gli {peziali per metterui détro zucchero, pepe,ed altro fimile,
ALLOP PIO . Specie di fonnifero compofto di fago di papavero , coagulato ,
fecco ye poluerizzato , ¢ d’ altri ingredienti ; ¢ fi chiama oppie .
<CARREGGIARE, Venendo da carro dourebbe intenderfi folamente per Cam-
minar col carro , o traghectar robe col carros»ma ci (erue per lo pill per inten-
der ogni forte d’andare , 0 camminare , a picde, 0 a cavalle, conduceado o non

BARILE , Vato di legno per ulo di portarui olio , vino , ed ogai altro liquo-
re fimile , ed é la mifura comune del vino , capace di 20. fia(chi, ¢ quello da olio
di 16, fiafchi. Tali vafi fon compofti, ed aggi(tati ia maniera da adattarne duc
per volea addoffo a una beltia da (oma . ‘
~ ATT ACC Ail fiafco, Coloro, i quali in Firenze vendono il vino a fiafchi alla

“propria cafa y-attaccano per (egno di cid fopr’ alla porta un fiafco , accid che il
ib luogo’, dove fi veade i! vino : ¢ pero quando fi dice H/ tale ha oggi
attaccato il fiafco ,s' intende , il Tale oegi ha cominciato a vendere il vino a fiafchi.

S{cala.abuon mercato, Si la(cia perfuadere dal prezzo vile a comprarac. B’
traslato da gli uccelli , che fi calano alla vifta della preda.

MACC.A, Abbondanza grande . Vien. forfe dal Latino Adadus , che Sinten-
de abbondanza grande, quafieAsagis an‘tus . Plau, milit, 4.22. Adete amare. B
fi trova Pxer matte nirewe ; giovanetto virtuoGiimo . Dice il Vocabolifta Bolo-
gnefe , che macco vuol dir’ abbondanza y che induce difprezo, e cosi ¢ vero acl

~parlar noftro, che fi dice smaccare per ittender Vituperare , o fereditare.

A lfonne’, Pee niente . Senza fpefa, EB detto plebeo , ed é ufato per lo pik tra
i battilani, i quali hanno per tradizione , che [fonne falle gia un’ huomo de’loro,
il quale mangiava tanto volenticri a {pele d’ altri , che effendo morto , ¢ feppelli-
to gia di qualche mefe , fcappafic dell’ avello al difcorfo , che da alcuni fi faceva
di-voler dar mangiare a tutti i Battilani per tre giorni, fenza che {pendeflero ,
Coftui havea due fratelli ’ uno detto Salicone , ¢ I' altro lo Scrocchina y € perd
feroccare mangiare a Salicone , a Scrocco , ¢ a donne fignificano tutti Mangiar fea-
za {pendere, che Tx io ditle edsymb poflo dalla propofizione A , che
fuoaa Senza ,¢ symbolum, che vale ae » 0 {corto, ¢ fignifica (enza denari ; & fi
come ne i Latini quelto Arymbo/um,tu ufato da i paratliti,c guatteri, cosi i] aoitco
donne , ¢ ufato dalla plebaglia , fra la quale é nato .

Pud anch’ effere, che quefto detto donne venga da un Iiogo poco fuori di Fi-
renze detto //onne , dove anticamente andavano a definare aicune volte pate
2 Ha

 
60 MALMANTILE

,
molti battilani , fenza fpendere , non perché veramente non fpendefleros:ma fier-
ché il denaro , che fi (pendeva in quel definare, era di mance este perile Pa

5. Giovanni’, ¢ Carnevale , che mefioin una lor corbona , fi ferbava\, ¢ diftri-
baiva per — definari ; ¢ pud eflere , che quefti battilani.deffero tal nome»

Lfonne a que

luogo dove andavano a far quefti lor definari., chiamati da loro.de~

Jfinari a Monne ; ma fia come fi voglia , bafta che appreflo noi il termine Sonne é

intefo per Senza fpefa .

TVFF.ANDOS/, Tuffarfi a una cofa,fignifica Pigliare,o fare affai una tal cova.

S'imbriacaron come tante monne.V edi quel ches} é detto fopra in

STANZA LXXVIIL
Quando il Cornacchia vedde il{uo aifegne
Gia riufcito , ando fopr’ alle mura ,

efto C-ftan.i0.
TANZA LxxIx.
E perc’ ognun dormiva , come un Talos
La donna fece farne.una funata ,

Ed ai compagni fece il detto fegno , E condurfegli apiedi a baciar baffo.y
Che bene havendo al tutto pofte cura y E renderleil tribato ognun prorata y
Saliro al poggio fenz’ alcun ritegno 04 Celidora:pa: reftarain Naffos

Senza Sofpetto haver ,fenza paura
Dietro alc ornacchia lor guidonese fcorta

Cive da’ {uci vaffalli rinnegata:y
Gia che tutti voltarohavean mantelloy

 

Dentro al Caftelloentraron per la miners Comando che baciaffelil chiavifeelle .
STANZA LKXX :
Ell ubbidi , cemendo, ancor di.peegio., Coss finito il folito corteggio Shi
E ben che fuffe un pexxo in la di nottey Con due ftrabellijeun par di fearpe rotte
MI pigliarfene fubito il pulezgio ——  Triffa ,¢ firafcina poi perila boccolica

Vn xucchera le parue di tre cotte . Vin 10°20 mendicava all accattolicn s)
1:Compagni di Bertinella yeduto il fegno dato dal Cornacchia , andatono a»
Malmantile , ed entrati dentro , ¢ trovati tutti a dormire gli legarono ,.¢-gli con-
duficro a render ubbidienza a Bertinella ,la quale comandd a Celidora , che nicif>
fe del Caftello , ed ella mutta mai’ all’ ordine fe n’ ando , benché futile afflai di not-
te, ¢ fi condufe a.mendicare il vitto
GVIDONE , ¢ feorta, Guidone s’ intende Colui che guida; ¢ Scortaé quello che
moftra laftrada; ma la voce Guidove & forfe per (cherzo prefa dall? Autore:nel
fenfo,chefopra flan. 65. ¢ fotto.al Cant, 8. flan. 72. :
FAR una funata. Legat con una fute pit perfone : Quando molti infiemes
commettono un delitto., fi {uol dire : Se vengono ibirri 5 vogiion far la bella funata,
Non perché crediamo , che vogliano effettivamente Iegargli tucti a una funes ma
intendiamo , Vegliono farne molti prigiont , ¢ cosi intend) nel prefente luogo ~
BACAR baffe . Cioé inchinarfia baciar i piedi in fegno di vaflaliaggio «
RIMANERE in Nall. Dai pid fi dice rimanere im Ajso,e cid fegue per corru-
zione nella pronunzia., che tanto fuona rimanere in afso che rimsancre in Wafso
come fi dourebbe dire,¢ fignifica abbandonato., fenza-aiuto , ¢ fenza, configlio ;
Ed é derivato dalla fayola d’ Arianna abbandonata da‘Lefco nell’ J/oladi Natio ;
E fi dice anche rimanere in [u le fecche di Barberia, il che corrobora che fi debbas
dire i WNafo , ¢-non in allo che non ha verua fenfo, o allegoria . Vedi fosto C.
10. ftan, 2. i"
VOLT-AR mantello. Rinnegare. Ribellarfi ; andar da.un partito all altro . Il
Lalli En, trav. C. 2, flan, 39. . 3 3 sun
r

 
or Clee 7 wae

PRIMO CANTARE: or
+ i Flor che mi lice divoltar mantelo, ’

. BACTARE il chiavifeello. Andarfene fenza {peranza di tornare.. Vfiamo que-
flo detto per efprimere che non fi vuole , che quel tale , che é ftato per li {uof
mali portamenti fcacciato d’ una tal cafa , viva con la fperanza di ritornarui , ¢
pero fi potrebbe dir con Vergilio Supremum vale.dixit , :

CHIAVISTELLO . Scrratuta da porte , © fineltre, che confifte in un ferro
lungo , il quale fa la fua operazione , paflando per diverfi anelli pur di ferro
adatcati nel legname ; ed ¢ il Latino veitis.

PIGLLAR i paleggio:,. Andar via. Pigliar il cammino, E’ frafe marinarefca , ma
pero ufata comunemente in quefii termini d’ andar via prefto. Dante Par. C. 23.

Non ¢ puleggio da prccola barca
Quel che fendendava t ardita prora
We da nocchier , ¢ a fe medefmo parca, «

Da quefta voce Puleggio viene /pulezzare, che vedeemo fotto C, 7. flan. 18, che
pure fignifica Andar yia.. Forfe fi potrebbe dir anche prueggiare verbo pur mari-
; nare(co , che fignifica Andar via bel bello .

i Vincenzio Tanara nella fua Economia del Cittadino in villa Lib..6. tratando
y dell’ erba ‘Px/eggio dice »che {parfa in luogo dove fieno pulct ha virth di fcac-
; ciarle; onde puo effere che da quefto effetto.dell' erba Pu/eggio venga il prefente>
i dettata,, Da pulegeia forte anche vengono Pulegge., che fono quelle piccole girel-
' le.,.che fi congegnano,ne.ilegni per facilitare i veicoli , come farebbe dentro a i
regal da piede alie (cene , 0 propane da commedie.per'renderle pili facili a+

Atralcicarti dentro.a,i\canali in occafione di mutazione delle medefime fcene .
AN suechera le parne di tre cotte . Le parve d’ haverla a buon mercato : le par-
ve d’ haver fortuna grandilfima., perche s'a{pettava malto peggio.. Lo Zucche-
ro di tre cotte fatte bene fi ftima che fia aj miglior grado di perfezione , della.
ae fono trei gradi. fecondo il detto omne trinum eff perfetum . Bd i Franzzfi
lenominano il fuperlativo col tre , clot buono, for buono, ¢ tre buono, per

buona, molto buono, .¢ buonifimo , :

STRAMBELLE, Vehi yecchic ,¢ ftracciate .. Vedi forto C, 3.,ftan. 65.»
¥N tox20,Detco cosi affolutamente fenz’ altra aggiuntavuol dire un pezzo di pa-
Ne. E frufiumpanis,che usd Dante nel Parad, C, 6.AMdendicanda/navicaafrufaafrufeo,
TRIST A, ¢ Strafcina « Huomo.trifto vuol dire Huomo mal yeltito,¢ Stra/cino
fuona quafi lo fieflo , perche ,Stra/cini.chiamiamoaicun: huomini , i quali vanno
comprando carne fuori delia Citta, el’ introducono in.Firenze occulcamente pr
deans ¢ perché coftoro fon fempre uati , fudici , ¢ ftracciati , per-

cid dic Strafcine intendiamo. mal’ all’ ordine di veltico , ec, ¥
_ BOCCOLK a5 ¢.accattolica . Sono due parole dette per icherzo , ¢ per 1a fimi-
_litudine che hanno con Boccay¢ con,Accattare, ¢ per pariare lanadattico ;, non
fono perd fuori dell’ ufo della gente pitt Civile y la quale {peflo fi ferve di parole
ating a quel propofito., che Je pare: che facciano. giuoco {troppiandole , ¢ inter-
pretandole a lor modo,come le prefenti Boecolica,e accattolica che l'una vuol dir Boc-
a, ¢ l'altra Accattare,e cosi intendefi che Celidora accattava per mangiare . Tal’
ufo d' allufione (cherzofa ra pur'anche appreflo.ai Latini trovandofi 42 ilie nan-
_ Gham recedis , che par ch¢ voglia dire w non ti parti mai dalla Cua di Troia , ¢
LOX s'in-

es

 

ae
e MALMANTYILE®! ©

s' intende poi ; tu hon abbandoni mai I" Ilo inteftino’, ciot (empré mangi .
MENDIC ARE , Vuol dire durat fatica'a confeguire. tale mendica le parole,
‘cioe Dura fatica a parlare; ma il fuo fignificato pili intelo ¢ Chiedere elemofina ,
Dante Parad. C. 6,
Indi partiffi povéro , e'vetufto ,
E 8 il mondo fapeffe il cor ch’ eghi hebbe 5
Mendicando [ua vita a frafto a frusto , ec.
STANZA LXXXE

Ih tanto Bertinella del Reame
‘Garbatamente fecefi padi ona 5
E de’ villaggi , ¢ d' ogni [uo beftiame
Prefe tl poffeffo in petto , ed in perfona
. STANZA LXKXAIL
Toff che ci hebbe fitto il capo , volle
C' ognun ferraffe ul rraffico,e il negorie,
‘Donando a ciafcheduno entrate,e rolley
. Aecio fe ta palfaffe da'buon forio ,
Ed allegro , 4 pit pari , ed in panciolle

Poi per letizia cavalieri , ¢ dame
Regalo di confetti ,'¢ di pattona ;
E fegueogn' anno di mandarne attorno ,
Per la dolee memoria di quel giorno,
STANZA LXXAlill,
Cos} mai fempre in fefte , ed in tonuaito
Tirano innangyi quefti [penfierati ;
Neinsobelbes per ao axe lun dite’,
Ben ch’ eh credefson a’ effer impiceari;
Won treme della Corte , chi ¢ fallitoy

 

 

Senzabriga vivele m pace , ¢ in oxio y Che tutti i giorns a lor fon feriati ;
Ognun vi s' arvecd ds buona gana , Non v'e ginffizia ,neilbargel vafuora ,
Che la poca fatica a tutti? fana , Se non aftigur chiunche lavora ,

&
Sbandita Celidora dal regno , Bertinella prefe ! attwal poffeffo di tutto lo fla
to 5 ¢ per acquiftarfi la b x de? fudditi comincid dal regalare le dame ;'€
cavalieri , con regali degni della vilidima condizione di fe medefima , ¢d appro-
priati alle qualita de’ Cavalieri 5 e Dame di Malmantile; poi con fefte, ed alle-
rie per contentare il popolose con levare i Mini(tri della giuftizia tanto odiofi al-
da plebaglia , e con fare altri ordini che fi leggono nelle prefenti ottave .
AN petto , ed in perfona, Attualineate , € corporalmente. -Arimo & z
PATTONA, ees > 0 pane fatto di farina di caftagne , con altro nomes
detto polenda y dal Latino Po/enta , che era vivanda fatta di farina d’ orzo cons
altre polveri odorifere fecondo Varrone . E' vivanda viliffima appreffo di noi ; €
‘da quefta fua vilra habbiamo un detto di difprezzo y che; Afangeapatrona ;
Mangiapolenda a un huomo vile 5'¢ buono, a poco. Qual detto usd Plauto chia-
mando quefti tali P/tsphagj ; ma il difprezzo non nafceva dalla vilta della polenta,
(che era finalmente il cibo comune anche per Je perfone di garbo ,¢generalmen-
te mangiando quefla forte vivanda i Romani viflero lingo tempo, Vedi Plin.
lib. 18. cap. 8. ) nafceva bene dail’ intenderfi con tal detto un huomo buon’a,
poc’ altro , che a mangiare 5 ¢ come noi diciamo Sparapani ; Voramadi¢ ,  timili .
V bebbe fitto il capo, Sen’ era’impadronita+ N’ haveva prefo l’ actual pofletio;
perché eflendo il capo la pil nobile-;¢ eae parte della perfona ', noi dicia-
Mo Ficcare il capo in kn logo per intendere Entrare in un luogoy ¢ pigliarne il
poflefio perfonalmente. nh
TRAFFICO , ¢ negoxio . Sinonimi , fe bene trafic par , che fi riftringa all’ ar-
ti manuali ; onde con dire 77 » ¢ negoxio invende non layorare y me mercan-
teggiare , 0 negosiare, é ” Bok

 

 
 

 

PRIM O/C AIN T/ARE:? ‘e

ZOLLA. E il Latino gleba y che) vuol dire Pezzo, o maffa di terta smoffa ,
come s’ é accennato fopra in quefto C, ftan. 57.,ma qui pigliando la parte per il
tutto, intende terrent fructiferi: 11 e4/eha'delle zoie', comuncmente s’ intende»
Ha de’ terreni . ‘ . x

SOZl0.. Dal latino Socius . Compagno.. iver da buon foro yuol dir Viver
da buon compagno , alla reale , ed alla (chietta . E quefta voce Sozio non fa che
fia ufara fe non in quefto cafo, ¢ con I’ aggiunta di bueno, O male: dicendofi 7
tale ¢ buon foxia , 0 non é mal foxio , per intendere. EB’ galant: huomo .

A pit pari , ed in panciolle .. 3’ ula quelto detto per efprimere Vn» huomo pol-
trone , che non voglia far’ altco , che godere i fuoi camodi, ¢ la voce pancio/les
€ compofta di due parole , ciaé pancia , ed ol/e, ¢ (uona pancia di pentola , la quale
col pofar pari » ¢ con quella fua gran pancia ¢ il vero ritratto della: comodita , ¢
poltroneria . 11 Bronz. nel Cap, in lode della Galea dice .

Guari , ma in capo al gingco , come valle
$ ACiela , ne fu tracto il poverino y
E fui privato di fare in panciolle ,

BRIGA , Noia,fattidio,fatica . Qui ¢ oe per faccenda,o penfiero d'operare.

D1 buona gana. Molto volenticri . E detto {pagnuolo,e la voce gana é'ulata da
noi per intender Voglia,o gutto grande . Mi tale mangia di gana ; Lavora di gana;ec,

‘ SCIOPERATO . Vino che non ha y¢ non vuole haver faccende. Vedi fopra,
F ftan. 29. Scioperati s' intcndono quei Cittadini y che fenza arte , o impiega vivono
on le loro entrate .
» CORTE} Antendi la Corte della giuftizia da i Latini derra Curia a differenzas
dt Anta ;e wuol dire Miniftri della ginftizia , ;
FALLITO . Vino che negoziando ha fatto cosi gran debito , che nan ha pof-
fibilita di pagarlo. E il latino decu‘tus,qui fallit creditores,ip/umaue fefellere negacia
TVTTA i giorni fon feriati, Sempre é felta per loro ; Fertaro s'iintende-quel gior-
no, nel Ze ancor che lavorativo non Gi tien da i Magiftrati ragione , e non &
poffono fare efecuzioni civili contra a i debitari , e quefto intende dicendo Now

Nile aed

SS,

teme della corte , chi é falco , perché é feriato , ¢ non pud effer menato prigione , .
STANZA Laxxlv, ;
Ua 2 ia non erro il tempo € gid vicino, Cost panna [ard di Cafentina,
Che n' ba a venir ia piena de distyrhi, Ne fe lamenti alcuno, 0 fi fconturbi ;
Mentre daman per far un buon borting Che chs nuoce al copagnoin fatti,oin detti
<Andremo a dar’ fo a quefti furbi., Deve faper che y Chila fa I afperti ,

Baldone , havendo fatto il detto raccanto della cacciata di Celidora , dice (pe-
rare , che fia vicino iltempo , nel quale faranno gaftigati coloro, che hanno {or-
» prefo Malmantile , perché il giorno futuro yuol! andare a dar ioro addofio.
‘ HA da venir la piena de’ diffurbs. Ha da venir grandiffima quantita di difgufti a
t flurbare i loro commadi, E Piena diciamo quando Arno , 0 altro Fiume crefces
per le pioggie . ‘<
\_ SARA’ panno di Cafentino . Cafentino & una Regione in Tofcana , doye fi fab- ee
-hrica una {pecie di Pa » che bagnati {cemano di Iungheaza , ¢ larghezza per- ;
ché rientrano . E da quefto detto /ard panno diCafentino , intendiamo Rientrera,
cide tu hai fatto a me quefto , ed io tard a te il fimile , coe Mi vendicherd.
2 cHl

   
 

64 MALMYNTILE *
CHI la fa? afpetti , Chi favun torto al compagno, alperti pure'd’ eine contrac.
cambiato. Il Petr. diffe;
Chi fi prende diletto di far froile ,
Won fi dee lamentar 8 altri? inganna ,
E quefti due verli poffou feruire per dichiatazione delli quattro uti della,

 

pre(ente otrava.
STANZA: LXXXV.
Qui racque il Duca ; € fubito rarcacca,
Col dire alla cuginain voce bafsa, ca)
Che. sperch'egli hit laboccaafeintra,e frac
Hi fuggiunger a tei qualcofa lafja
Non ho che din( gli rifpond'ella)un bacca,
Ottre che la Sarebbe carne grafsa ,
Di pits tofto, in che mo noi fiam parenti,

STANZA LXXX¥I1.

Ed jo che non nebo gran cognizione 5
E fempre me ne fono fhata avderra. \
(Che tutta la mia gente ando al caffone,
Come tu {ai cb io ero fancinlletta:)
T' udira volentieri .» Allor Baldone
Soggiunfe : Or or ti ferno,e ararafretta,
Perche non gli morialatingua in bocca y

ta dailo Cunto degli'Cunti di Gianalefio Abbattutis

Ch’ io nor: paia a coftor de gl Innocenti; Ricomincio quef? altra filaftrocca .

Baldone termina il difcorfo , ¢ volto.a'Celidora le dicey che ella foggiunga. ,
fe ha di pid; edefia dicendo , ‘che non ha che foggiugnere lo prega a narrare, in
che modo fieno parenti: E Baidone’s' accinge a contentarla.. Equi termina il
noftro Poeta il fuo ee .

NON bo che dire un hacca, 1! H vogliono, che non fia lettera\, ma femplices
afpirazione , ¢ pero.dicendofi Avon bo che dire un haccas 10 ficfio cche'dire: Vox
ho che dir nulla,

SAREBBE carne grafsa.. Stuccherei il popoloy; Mi fendeiel odiofa, I) Lafca
Nov. q.dice 2 £ poi io non vorrei anche tanto inf apbidirto , che exlim’. bave/ie wdire,
che io fufi carne grafsa . La carne grafia fuole ai. pil che la mangiano cagionare
naule a; il che diciamo ftuccare .

CH io mene 4 coftor de gl’ Innocenti., Che coflaro‘non ‘penfino , the io fa.
batarda ,-o fenza parenti. In Firenze lo {pedale-de gl’ Innocenti fi ‘chiama quel-
Jo , nel quale fi mettono.ad allevarei bambini\y per Jo pi, nati di-congiunzioni
illecite, i quali corrottamente chiamiamo Woceurini.. Vedi Totto Cant,vooftan. 7.

ME ne fono feata a detta (Non ho cercato di fapernepib 4a; macho creduto ge
che m’‘é ftato détto , o raccontato..

LA mia gente ando al-caffone . -Mio padre, mia\madre;'¢ tuttingli aaleti miei pa-
renti morirono ;che per mia gente in quefio luogo’,) i in at termini’s'’ inten-
de Mici parenti , ¢ ‘non altri.

et tanta fretta. Subito , Preftifimo. ,

NON gli moria la lingua in bocca, Era loquace ; cloquente . Havea facies ss
a ‘Elo ftefloiche Havere il /uoin minioenaiana Sige ‘come s-accennd fo-

ra bE
2 VLA RODE Serie di parole ; e per lo pit Hinténiestumciistartocrtalls
ordinato , e proprio del racconto ,’che talora fanno le ae a’ Fangiulliyis
le lor’novelle , ‘come appunito & quella che narra Baldone, che ae
haverla fentita ‘forfe ueseneneenie fue donne ,’ A a cra fat

 

 

FINE DEL PRIMOCAN rae

 

om

 
 

i

q

 

 

ARGOMENTO. i
De i due gran figli del Signor d' Vgnano
Prodigiofo il natal narra Baldone ;

Gomes acquifta moglie Flortano , :
E vien dak’ Orco poi fatto Pprigione.

Seth at

Come Amaaigi livera il germano ;

E il moftra [paventofo a terra pone y
E dice al fin , che’ un di quefti dui
Fu padre 4Celidora, el altro a lui’,

2
Sn py ngt

STANZA I.

E Ra in Venano il Duca Peridne, We per altro era tutta bacchettone ,

Bu Che fempr’ all’ Altarin fidecommiffo Che per un fuo penfiero ecerno,e fifa
Faveva notte, edi tanta orurione, D'haver prole , perché deliafuafchiatta
E tanze carita , ch era un fubbiffo + Won v' era, morto lui, ne can, ne gatta,

Ui. Duca Baidone da:principio-alla narrativa del parentado., che pafia fra lui ,
¢ Celidora , come havea promedfo neil’ antecedente Cantare ye dice; Che fa gia
in Vgnano il Duca Perione , il quale faceva mole opere pic per difporre il Ciclo
@,-<oncedergl) prole, La favola del nafcimento di queti figtiuoli trovafi-nello
Cunto degli Cunti-diGianalefio Abbattutis Giorn, 1, Cunto 9. ll noftro Poetas
pero non la cavo di quivi ; ma la narro, come I’ haveva fenuica contare alle fue
donne , quando era tanciullo; ¢ guelto é certo , perché queita era nel {uo primo
Trem fatto molto prima., che ii Bafile Autore dello Cunto-de li Cunti la ftam-
pafie, fi Tht
ALT ARINO , Cosi chiamiamo un’ inginocchiatoio a foggia d' altare, iliqua-
Ieper Jo pit G tiene allato.al Jetto.p inginocchiarfije fare orazione .

. STAR fidecommifso in un luogo , & detto iperbolico, che fignifica Star molziffimo.
in un Juogo ; che qui vuol dire Stava fempre 0 non fi Jevava mai dail’ Altarino ;
che s'intende faceva orazioni infinite . ‘ rs

TANTE carita ch era-un fubsfvo,, Carita » ed clemofine infinite. Per denotare
Una quantita indicibile ufiamo dire: Son rantivche e un fubsfso, un fracafso , un fla-
Selo, ¢ Gmili. Quelta.voce Subbifso vien pre dal Greco aby/sos,, che fignifica vo~

ragine

 

 

 
 

*

ee

66 MALMANTILE

ragine , o fmifurata profondita’d’ acquie s'come fuond ancora nel foftfo idioma ;
donde /ubifsare Andar nel profondo , quafi dica /ub aby/so . ;

BACCHETTONT, Cosi chiamiamo.noi certi colli torti , ¢ grafhafanti , ches
flimano peccato il portare un fiore in mano , ¢ credono poi di far’ un’ atto me.
ritorio a dare a ufura ; con aitro nome chiamati Ipocriti , cio¢ Pfeudobeati ; huo-
mini‘da bene pér interefle, e*per gabbare il cofipagno ; ¢fono infomma cploro,
de’ quali Giovenale diffe : Qui Curios fintulant, & Bacchawalia dinunt, E diciamo
Bacchertone , quali Va chetone , perché quetta Canaglia , che fudia di fimuiare la.
bonta , per arrivare a fuoi fini,é fimile all’ acque profonde , che vanho chete. ,
delle quali patlandé Q. Curzio dice : Altiffima queque jlumina minime labuntur fono,
E come quefte acque fon fempre di pericolo , cost li bacchertoni nella loro taciwur-
nita occulcano il malo animo , che hanno contro al proffimo . 11 coftume di co-
ftoro tocea Orazio Jib. 1. Ep. 17. dicendo che fon devoti di Laverna Dea de»
ladri. :

Labra movens ; metuens andirs ; Pulchra Laverna ,
. Da mihi fallere ; da influm , fanttumque videri . ;

Di quelta voce Baccherroné fi ferue anche il Tationi nella fua Secchia. Mimico
natural de’ Bacchettoni , Ed un dottiffimo de’ noftri tempi, il quale fa un
difcorfo poetico fopra a coftoro , lo termina con dire Furfanre , © bacthetton /uuna
il medefimo , Vedi {orto C. 6. flan, 97. dove fi dice eller lo Melo Bacchertoni , che
Jpocriti , i quali S. Matteo chiamd fimiles fepulchris dealbatis ; i) Berni nel’Orlando
difle. O agghiacciati dentro , é di fuor caldi; In fepoleri dipinti gente morta ,.

Giovenale aggiunge al detto di fopra .

Fronti nulla fides ; quis enim rion vicus abundat
Triftibus obfcoms ? caftigas turpia , cum fis
Amer Socraticos notiffima fofsa Cinnedos .

Di quefii tali parla in diverfi tuoghi la Sacra Scrittura deteftando tal vizio, eo-
ine abominevole , ma per brevita trala{cio di riportarlo, contentandomi di chiu-
dere col detto dell’ Evangelilta Atendite a falfis propheris , qui veniunt in veftimentis
ovinm , intrinfecus vero funt lpi rapaces = ¢ rimetter il Lettore a quello , che (crive
S, Matteo Euangelifta al Cap. 6. 15.23. -

Tale era appunto quefto Perione , che faceva le dette Opere pic , non perché
veramente futic buono , ma perché con efle pretendeva d! eftorcer dal Cielo las
grazia d’ haver figliuoli . 4

SCALATT A, Stirpe y Profapia , famiglia .

NON v' era , we can ne carta, Non vi rimaneva pur’ uno. Plauto diffe: Wve
weifca quidem domi eff , Del qual detto fi ferui quel feruo dell’ Imperator Domi*
ziano che domandato ; fe Domiziano cra folo in camera, rifpole: We mufeas
quidem eff, Percht Domiziano ftava la dentro ammazzando le mofche. Ter.
dit: Ve Sannione-quidem relitto.

 oaeteegel at ‘STANZA-I. susie
Cost divs gran tempo, ma dacerto y . E quanv ti far yn in difpr
» Vedendo cli ei ton era efandito Senza voler pik dar del pr 5
Effendo omai con gli anniin la. un pereo, Gertarofi al! avaro , ed al furfante
A mangiar comincid del pan pentito 5 ‘Cambio la diadema in wren :
3 OR-

 

 
> at eee hh

 

ey. ee a ree

 

SECONDO CANTARE, 7

Continud gran tempo Perione a far-le narrate opere pic, ma yeduto ch’ci non
¢ra efaudito , ¢ ch’ ci non haveva figlivoli , ¢ trovandofi gia vecchio , percht ve-
ramente egli era un di quei Bacchettoni furbi , che habbiamo detto fopra , ¢ che
faceva bene folamente per interefie , fi penti d’ haver fatto tante elemofinc , ed
altro bene , ¢ muto coftume.

DA 7exx9. Da ultimo. Forfe meglio /exo , venendo dal Latino /ecius oppofto
dj ocius . Vedi foto C. 4. ftan. 72.
| ESSENDO un peo in id con gli anni, Effendo grave d’ eta. Havendo molti
| anni. Vedi fotto C, 12, flan, 36.
be MANGLAR del pan pentuo , Cioe fi duole , fi pente d’ haver fatto del bene ; ed
.

.

& quel fatti penitere di Cicerone,

POST O in difprexzo quanto far folea . Cie lafciando fare di fare clemofine , ¢
Qrazioni , ed altre opere pie come folea fare .

SENZA voler dar del profferizo. Senza yoler dare pili niente ; ¢ ne meno quel-
lo , che havea. prometio , o proferta ,

GETT ATOS/ all’ avaro... Divenuto.ayaro per clezione , o diremmo A poftas.

FVRF ANTE, Vuol dir furbo {cellerato , ¢ ladro , ¢ fimili venendo dal latino
barbaro foris factens , operante fuori del dovere , ma fi piglia anche per Spilorcio ,
ed avaro , come ¢ pre(o nel prefente luogo .
_ CAUBIO' Ja diadema in un turbayte . Di Santo divenne Turco , che Diadema
apprefio di noi vuol dire quell’ ornamento , 9 corana di {plendori , che fi vedes
dipinto attorno alla tefta de’ Sa nti. Dice che cambio la diadema , che meritavas

\ come Santo, in un turbance , cioé cappello da Turco, non che veramente fi met-

tefse il Turbante ;.maintende, che d’ huomo da bene-diventd qutto il contrario .

hg STANZA
Di poi tutto diverfo ye ma! difpofto La moglie un miglio fi tenea difcofto,
_ da modo degli Dei faceafi beffe , E dow’ ci dava a’ poveri a bizreffe y
| Che segli udia trattarne,bauria pik toa Quando picchiavan poi dalla finefira y
_ Valuco ful moftaccio.uno sherlefe ; Facea lor dar il pan con la balefra,

gees Perione tutto diverfo da quel che era, come sé detto ,,comincid
anche a non ftimar pil gli Dei, anzi g i Arapazava in modo , che haurebbe vor
Juco pid tofto un, sfregio (ul vifo , che fentirgli nominare ; sbandi la moglic, ed in
x t limofine a i poveri gli baftonava . ay

2 RECERSO 9 -differente da quel ch’ cra prima . Se benquefta yoce diver/o
fignifica ancora ftrayagante.. Vedi fotto C, 8. ftan, 17. ed in quefto fen(o 1a pigla
Franco Sacchetti Nou. 29, £ quc/ta natura pare a me, che fufse delle firane , ¢ dix
verfe che trovar fi pore/sero.,. B Noy. 78; Ed era wn! buome maliziofo y reo, ¢ di di-

aii iy ey sama | tar iogs 5 . se td ¢
FACE AST befe . Si burlava . Non faceva ftima. E il latino flocei facere .

SBERLEFFE ...Taglio ,. 0 sfregio’, che i Latini diflero frigma ; Rigido fignara
_ Htigmate frome. E. perché gli iin ful yifo fono cofa ignominiofa , come s\¢

ieee fears ch ftan, fs cid fi deduce fhe Perione hanria pie toll foppor-
4414 ogni grande ingiyriayed ignominia,che fentir nominare gli Dei. joppet-
tA nel Cap,in lode della sig.Qusnaia, iglia Ja voce sherlefe asta: di buriare a 3
Nao » con oltraggi s.¢ puaiure , che Beast - molsi fi dice Fare uno {cappenco a
org z=

. aller rae

  
 

68 MALMANTILE,

‘Allor l amico in mezzo a i'dolor mici
Ati fece uno sberleffe di velluro
; E mi fece arroffir dal capo a piei:. othe!
E pid fotto nel medefimo capitolo fo fteflo mofira , che habbiamo ancoilverbo
sberleffare dicendo
E col rider di grazia andate piano,
Che non é per infermi ucil conforto ,
E chi viol sberleffar , sberleff in vano, \

Lorigine da quefta voce sherlefe vien forle da Berlina it quelto modo :

Si faole alle volte , dopo*haver tenuto in Berlina i ladroncelli, fegnargli ing
qualche parte del corpo con un ferro infuocato , acid: che fiend dalla Giuitizia,
riconofciuti , fe altra volca per commeffi delitti li tornaffero ‘nelle mani’. E di
quefti fegni vedcemo foto C. 6. ftan. 54. Cid fi coftumava ancora appreflo. gli
aatichi Romani ne i (erui -fuggiciviy, € gli fegnavano-nella fronte comic fi cava‘da
Aulonio Epig. 15. che parlando di un feruo.nominato: Pergamo dice’,

: Lam fegnis {criptor , quam lentus’, Pergdme , curfor
Fugifti , & primo captus es in fhadio; t
Ergo notas fcripto tolerafti Pergame vultu ,
Et quas neglexit dextera , frons patitur.

Et aggiungefi alla voce verdina quella finale efe , da quella lettera maiufcola F ,
che é il {egno , o marchio , col quale fi marchiano i detti delinquenti. Che co
fia berlina . Vedi (otto in quefto C. flan ig.

MOST -ACCIO . Faccia, Volto , eo, we

TENEA la moglie difcoffo-un miglio, Tenea la moglic lontana da-fe,intendi non
volea pik commerzio con la moglie.. Lat; fecubabat. ap ty

DARE 4 Bizxeffe, Dares o'donare largamente, QuéNe voce; che’ & ¢ompo-
fta dal latino bis , & efe., cioé due volte, f, vuol dir pienamente-, largamente_ ,
abondantemente , ¢ fimili; Quando il fommo Magiltrato Romano intendeva.
fare ad‘aa fupplicante la grazia fenza limitazione , ma pienamente*faceva il re-
{cricto fotto’al memoriale , che diceva Fiat Fide y che poi perbrevitd tro.
no di dimoftcare quefta pieacza di grazia con fegnare 1 memoriali con Tole’ du»
effe s onde quello che confeguiva tal graziavdiceva: Io hovhavuta la grazia a bis
efe , ciot due volte ff che s' intende grazia intera , e/pitna , a} coftrario di quél-
Ja limicata , che’era con‘una {ola effe aggiontavi la limitazione , 0° condiz
con Ja quale il Magiftrato havea conc la graziay E'da quelto bis efe's' &
corrottamente introdotto il dir Bizzette, che ha il figniticatu,“che ‘habbidm

 
 

lea be gout TTh By of pret Ms gait CPavee
DARE il pan con la balefray, Vol dice Gvapazare «. Fare in maaitra’, che il
‘denefizio fa di i a chi lo-riceve . Deriva forfe dallufo-, obtera ire

ze avanti che: andar a caccia tonl’archibufo’, di tenere-al {uo

midi a pofta i quali‘con quaiche falvaticina manteneffero le | Pers ede e

quefto efercizio Eifendo d’ utile , ma affai laboriofo, pud hai ¢ a

quelto Proverbio dare il pan'con la balefira , ciot/accompagnato da’fatica, ¢ difa-
gio

 
wan Peeters.

SECONDO‘CANTARE: 69
| io grandifimo. Ma nel prefente!luogo intende che effettivamente faceffe ticare
eftratea i poveri.

Si dice ancora in quefto propofito .. Porger i! pane con la fpada,e cid forle de-

riva da quello , che fece Dionifio Tiranno.a un tal Democle Filofofo , il quale

( perché adulando cccedeva in lodare le grandezze di quello ftato di Dionifio )

egli féce (edere ad una menfa ripiena delle pia e(quifite vivande , che per un ban-

cherto realeinuentar fipotetfero ; ¢ fece attaccare per il manico ad una ferolas
pendente con la punta fopr’ alla fua tefta , una {pada sfoderata , la quale vedura

dal Fitofofo,gli cagiond cosi grande {pavento , che egli aon pott fe non con mol-

if ta paura, ¢ con poco-guito pigliare di quei cibi, Di coftui parla Orazio Od,

pre lib. 3. : ;
Ky 2 7 Diftrittus enfis cui fuper impia
Ceruice pendet , non sicule dapes
} ws (08 \Duleem elaborabunt Saporem
| - Sidice-ancora , aqaelto propofito , dare if par col baffone che ha-origine das

‘quel che fece il Piovano Arlotto ; il quale lp Baltigar I" indiferetezza d’ alcani
caccidtori ; che gti havevano lafciato 1n cafa ua branco di-cani , quando a quetti
dava il pane 5 ’ accompagnava con una mano di’baftonate , onde-i poveri cani
s' crano affuefatti quando vedevano il pane a fuggire ; per lo che divennero co-
tanto fimagti 5 che pena firreggevano in Piedi. Ritornati i cacciatori per li loro
Caini'y vedutigiivcost sfacti fi dolevano del Piovaho; ma egli prefo in mano il fo-
‘oli yticd a ate alcuni pezzi di pe edi sie dico-
me era folito patiare il negozioy, in vece @ actoftarfi al pane fugeivano 3 Onde il
Pidvane fi feusd c6-i cacciavoti dicendo: ‘Come volete sche ingraffino-,\(e quan-
pre ner faggono come vedere ? £ da’ quefta facezia venne
e Oak it pe ed/ ba/tone,che tignifica moftrat di yoler far del bene a uno,
-fargli del male \»Seneca'ci fa'veder guelto modo di dire anche appreffo.a i La-
fini, raccontando il detto di Fabio per foprannome Verracofo-, che il piaceres
Fate daper(ona ‘zotica’, € con indmiera [aivatica chiamava Panes lapidafum 9 che
€ Appropriatd at ndftroldetta-Dare i/ paneye ta fafsara , ;
_ BALEST RA‘ Sttumento's o arine da caccia , col quale fi (eagiiano palle di
fecea jhnelia guild che fi fa delle frecce ; © ferue cer alton ‘uccelletti,,
“d'un arcod’ ‘acclaio accomodaco in cima a un’atta 5 ‘© degno torto,,
e (Ono adactati altri ordinghi di ‘ferro’ pet ‘facilitare'operazione.,
ietie- dal? antica baliifta-arme Suerricra , che dicevano ballifta ‘forfe dal Greco
‘ballein » che fignifica (cagliare “= y
ee HG 9 eal Iv.
La plebe's i grandis mi) Vedutolo cos? ‘mutar'regiffro:
Chit DisvaiMuwbosacyraen E diventar pserintiy S
ene aun Pe lor finifro vt au talmente dt ‘tninvo catrivo:,
_ Edin lor pro r rato Chel biurebbon voluro ingoiar vive;
> Per giueita murazione del Duca di 'biiono in cattivo , li faoi fudditi , che pei-
_ “ma'l’ amavano., cominciarono:a portargli'odio , ¢ bramareli ogai niale,
81 farebbe fparato in lor pro, ‘Haurebbe fatto loro ogni eee immaginabile..
| -Hiatirebbe meffa ,'c (pela Ja propria vita.a-benefizio lero 3 ¢!> voce pro-é-ua fu-
flantivo

  
    

 

 

 

 
 

70 MALMANTILE 05%

fantivo che fignifica giovamento , utile, ec. dal:latino prodef

MUTAR regifiro, Mutar maniera di fare ., Registro diciamo quell'.ordine di

ferri , il quale & negli Organi ftrumenti muficali , con ciafeuno de’ quali ferri al-
zandolo , o abbaflandolo fi da , o leva il fiato a quelle canne , le quali fi vuol ,
che fuonino o nd , ad efferto di far mutar voce all’ organo , il che fi dice smuear
regifire , che pafiato poi in proverbio fignifica Mutar maniera , o- modo di
in qualfivoguia cofa. Vedi forto C8. ftan, 52. alla voce protocollo Regi/tro in
altro fignificato .

INGOLARE . Trangugiare . Mandar gil in corpo una cofa fenza anche ma-
flicarla , che fi dice anche sngullare. Vedi forto C. 1. flan: 6.

s

TANZA V.
duvenne , che gid intefo un Negromante E per ridurlo all’ opre buone, efante
C'un'huom com'era quei si ginfloye magne, Non per [peranza di verun guadagno
Faceva novita si firavagante , Fintofi un baro, adargliandol' afsalte,
Vn’ atto volle far da buon conipagno ; Fon po di ben chiedendo per fant’ alto .

Stando le cole ne i fuddetti termini , Va tal mago , intefo che un -huomo
bene come era Perione s’ cra cangiato in cosi cattivo , volle fare un’ acto da hus
mo da bene , cercando di rimettere Periane nelia*buona firada , ¢ perd fintoft
un’ accactone , andd.a chiedergli I’ elemofina per amor di Dio.

WEGROMANTE . Flo fleffo che Mago: Se bene Negromante venendo da
negromanzia s’ intende colui , che per mortuos vaticinarur , che é una delle {ei {pe>
cie di Magi detti fopra C, 1, ftanza 20.5 tuttavia da noi fi piglia per nome geng-
rico , ¢ per intendere ogni fpecie di ae ye di magia. i :

BARO . Biante. Accattone fallo .» ien forfe dal Greco Barijs, Bareos.,ches
fuona molefus , importuno , sfrontato , come appunto fono quefti tali; ¢ fe beng
quefta parola ha del furbefco pure s’ ula comunemente, ¢' usd il Varchi St. Fior,
lib. 11, Ed in fegno , che lo rifiurava ,¢ non gli oreduea pil, havendolo per baro af
giumatore ,arfe i {uoi libri, i

PER Sant’ alto . Cioe per Dio. E,parlar furbelco y il quale forfe &noto fuori
della noftra Tufcana , come inuentato da Vagabondi , Monelli ¢ tianti per non
effer intefi , fe non da i lor pati , ¢ poi fattofi familiare a mole’ altri, a.fegn
che ne-é facto, fampato il vocabolario.. Si dice anche parlare im gexgo,ed in lingwe

furfantina , come Cl moftra il Vaichi Sr, Fior, lib. 15. Apparifeono piu lertere feritte
non ins cifra , main gerga a ufo ds lingua furfantina moairo mse « I noftro Poeta. |
ferye di tal parlare nella perfona di quelto Biante perché , come ho detto ; fi
huomini fon foliti pariar in Pa ao i; eoaipouis ponte’

Rifpofe Perione -. Pratel mia, « See bai bifognos che poffo fare iedns,
se ru te lo credeffi tat? inganm 5 Che fon Frafaxia sche rifaccia’ :
Tu anoi ch’ io doni per ? amar di Dio, B che penft sche qus ci fia la cava?

Ne [ai ch’ s0 pigliorei per San Giovanni, Non ¢ piu sompo che Lerta filava

‘Aila richiefta del Mago Perione non fi muove a far limolina., anaid = che»

pigtersDhe anch’ egli qualcofa , ¢ che ¢ palato quel tempo che egli dava via
1b tua., ¥ e

PIGLIEREL per San Giovauni . §. Gio; Batifta ¢1l Santo provenore seen

atta

 

oh

;

 

 
 

 

 

 

SECONDO CANTARE, vhs

Citta di Firenze,e percid il giorno delia fua fefta ¢ grandememe folennizzato 5 ed
in quel giorno fon ficuri nella Citta finovi banditi capitali , ficché gli Sbirri non,
Hidnipislanatins + Daquefto é nato’ equivoco Proverbio ; Pigiterebbe il dé
di San,Giovannir, 0. per San Giovanni, che: vuol dice Piglierebbe anche quel di ,
nel quale ne meno i birri pigliano,; ¢s’ intende pigdieredbe , cioe accetcerebbe tutto

che gli futle-dato ih-ogai occaiione y ed in ogni tempo. B lo {cherzo & nel
verbo pigliare che yuol dir Far cattura, o Catturare,e vuol dire anche Accettare ,
© ricevere , come s’intende in quefto proverbio ; che efprime ; Lo piglicrei , ed
accetterei fempre , ¢ non darei mai.

CHE fon Fraffazio, Raccontano una favola d’ una donna non troppo hone-
fta , la quale havendo commerzio con un tai’ huomo detto Fraffazio , fu con effo
una volta trovata dal marico ; ed effendo ella altrettauto fagace , quanto il ma-
rito femplice:, ¢ di cervello grofio, gli dicde facilmente a credere , che colui cra
un’ huomo da bene 5 che andava rifacendo i danni a chiunque occorreva qualche
difgrazia, e che I’ haveva chiamato in cafa affinché le ricompraffe una faa con-
ca , la quales’ era rotta ,¢ che appunto gli narrava quefto fuo danao; foggiun~
gendo; B come, Marito mio! Non conofcete dunque Fraffazio? Il buoa ma-
rito fe la bevve , e cosi 1a donna fcampé Ja furia , E da quefta favola , quando &
dice: fer Fraffazid , vuol dit: » Effer colui che [pende il fuo per folevar t aitrui mi-
Seriesye che rifa. i danni come dice il noftro poeta .

CHE penfi , che qua ci fia la cava, Penfi che io habbia la cava de’ danari , ciot

+ Torna bene a quefto detto quel che fi trova in Saluftio ; Cen/es me vi-
em ararij praftare . Non ¢ pero che cava voglia dire la Zecca, ma fi piglia per
quefta nel prefente detto ( da noi ufatiimo tn quefto propofito ) perche fi (up;
ne 5 ed é verifimile che la Zecca , come luogo dove fi batte la moneta, ne fais
abondante , come fono abondanti le cave di quelle cofe , che da efle eftraggonfi.

NON ¢ pik ib cempo che berta flava. Non € pid il tempo , che le cofe andavano
come fi bramava. 1 tempi fon mutati. Pipino Re di Francia per mezzo di {uoi
Amba(ciadori fposd Berta dal Gran pi figiwuoia di Filippo Re d’ Vagheria, las
quaic havendo faputo\, che quefto fuo Spofo era brutto , © nano , malvolenticri
s? accomodava a dare H conienfo; ma pure , vinta dalla riverenza dovuta ai pa-
dre, condefcefe, Arrivata in Prancia , la(ciandofi governare dal giovenil fenti-
mento;richiefe Elifetta di Maganza ua fegretaria ( la quale 4’Vagheria,dove era
naca del Conte Guglielmo di Maganza ribello di Prancia,(e ne vemiva-con Berta a
Pacigi ) che voleile , fingendofi la fua perfona , in fua vece {pofarfi con Pipino
il quale ,¢ pera fomighianza , che era fra lor due , '¢ per non haver Pipino mai 4
veduta Berta , non’ haurebbe aflolutamente riconofciuta , Bliferta da ere
fi moftro renitente ; ma perfuala poi da Grifone ,¢ Spinardo di Maganza {ui
parenti , condefcele a i voleri di 3, B.cosi arrivatia Parigi , Elifecta fi {po-

80 con Pipino in vece di Berta. La quai Berta in tanto di configlio di detti due

Maganzefi s' era ritirata in ludgo vicino a Parigi , con penfiero fermato cons

-decti Maganzefi di quindi occultamente partir, ¢ tornarfene alla patria com

Aaiuto de’ medefimi ; ma guelti Ja cradirono , perché in vece di fervirla alla vol-

ta della patria fua , ' inniarono ad-un bofco , con urdine a quelli, che la con-

‘ducevano, che I’ uccidetlero: Mu coitoro mou a picea , in veced’ sagen Ihe
bi {po-

 
 

 

p MALMANTILE

{pogliarono , ¢ legatala ad un’ albero la lafciarono in preda alla Fortuna , ¢ tor-
narono a i Maganzefi , dicendo che I' haveano uccifa 1 Maganzefi per occulta~
re fi atroce delitto fecera morire tutti quei ficarj, havendo prima anche-d’ arri~
vare a Parigi fatte ritorpare in Vagheria tutte le dame , ed\altre: perfonenons
complici , ne confapeyoli di si geande {celleraggine 6} cs 2th aawmiss
Berta intanto , che fe ne ftava cost legata:dolendoGi 5 ¢:lamentandofi fu fentita
da un tal Lamberto Cacciatore dei Re Pipino ; Coftui feguitando la voce ficcon-
duffle dove ftava Berta legata all’ albero., ¢ (cioltala., alla propria.cafa Ja con=
dufle , ¢ la confegno alla moglic yeftendola d’ abiti vili, ¢ conformijalla ‘pofibili-
ta di lui, ed alla poyera condiziane , della: quale Berta diffe deflere 2» Quivi
ftette Berta circa cingue anni 5 nci qual tempo guadagad molti: denari di) filare}
ed altri lavori,, che infieme con:le figliuole di Lamberto facevacs: Avvenne una
giorno , che eflendo Pipino a caccia fi condutic folo alla Cala di Lamberto, ove
veduta Berta s’ inuaght di Jei 5.¢ con efla fi:congiunfe fopra.ad-un {uo carro 5 nel
qual congiungimento fu gencrato Carlo , coskdetto dal amedefimo Carlo. dn ta»
le occafione Berta {coperfe a Pipino ii tradimento. de i Maganzeft narrandoli
tutto il feguito ; perloché Pipino fece abbruciare Elifecta , cd una mano di Mas
ganzefi , ¢ rimefie nel trono Berta , 3h bh fob eomearr« plidhoss
Da quefta favolofa foria nacque ilproverbio ; Wom é pik. ikvempo che Berta fie
Java, Cioé non ¢ pil i! tempo che Berta flava neile feive filando., ¢ ricamandoy,
che-fignifica; Le cofe fon mutate , ‘ i Pas X\

Di guefto dettorfi (ervi Berta moglie d’ Arrigo 1V,Imperatore , come fi vede
nello Scardeonio Monameata Patavina lib. 3. Cialie 1g. de Berta ex Montagna~
no , le di cui parole fon queite . Ademonatur ia iifdem Pacavinis eAnnalibns celebris
fama Berteex Vico Montagnani , qus quidem fait ruflicano genere , fed moribus certe
perquam nobilis CO animo perguam generofa , ach
Hee enim tempore Henrici IV, Imperatoris , cum eius uxor, Berta & ipfa muncupa-

ta , Pacavij moraretur , vel cinfdem force nominis fimilitudine , vel propria generofitas
te animi allecta , obrulit ei dono filum tenuiffimum 5 quod-eleganter [amer neverat mar
nin 5, in Vrbem venale detulerat Quod munus Regina bilari vulew accepit 3:
cum, cognoviffet nomen , OF animum mulicris , cam indignam cenfuit , xt vitam inopem
Samineo colo amplius fufpineret fuam ,, Dato icague filo procuratori fuo, inber ad Pagnm
Monragnani frarim proficifci , ubi mufier habicabat ,& pro referends gratia tor terra
ingeraei ex publica adferihi , quantum [pacij filum dono datum extenfum. comprehen-
dere ,@ cwrcumdare poffet , Quod.cum catere mulieres vidifsent , ilico Berta exemple
attulerunt »& ip[e filum , quod Regina dono darent . At ipfa renuens id ab alijs acci
pere percanté re/pondir , Pertranfye tempus , dum Berta filabat . 3 sree
Gliantichi digevano Aon ef amplins atas Cyclopum,ed in. moluc.a)tremaniersficame
Ancor noi diciamo: E finita dacuccagna,o la vignuoladVen ¢ piri cempo ai Bartolommery
ec. Cont quali , ed alpri detti intendiamo Non:fi godono pil quelic felicitache gia:

   

fi godevano. STANZA. VIL 2
» Signor ( foggiunfe il Adago) mi [a male Hor bapa ; Chi del miofac.
Di veder , ¢' un si gran limofiniere, ( Difs egli) fa la xappa nel ‘
E4 huom tanto benigno , ¢ liberale Pero va in pace tu ca’ tuoi bifeget 5
Caduto fia nel mal del miferere .. Pevche per me tx mangerai de’ her's

  

"igi

 

 
SECONDO CANTARE: 3

ll fegromante vedendofi cacciar via con tal rifpofta ; replicd, che gli difpia-
eeya,;ch’ ei fuile diventato avaro . E Perione li foggiuafe , ch’ ci non {perafle da
Jui faffidio alcuno..

CADVTO net wial del miferere. Divenuto mifero , ciot avaro , tenace , che fe
bene il mal del Miferere é una infermita mortale; Noi ci ferviamo della voces
AMiferere nella forma che habbiamo detto fopra:C. 1. flan, 80. della voce boceo/i-
ea; per intender mifere, che nel prefente luogo vuol dire avaro ; ¢ cost é intefo
comunemente , fe bene la voce 44/ere propriamente vuol dire infélice 3

FAR capitate. Par’ aflegnamento ; o fperare nell’ aiuto d’ alcuno. Vedi forto
C. 7. flan, 82. Quefla voce capitals € dedotta da capirario capitationis , che era una
tafla , otributo , che determinavati ix capita Leaner per aflegnamento ; ¢ pro-

tk \priamente capitale del Principe, come ¢€ forle la Decima , che pagano hogei i
nofter contadini , chepure fidice decima in fu latefta .
| RANIERE B.un valo intefluto , ¢ compofto:di fili.di vetrice , o-d’\altra {pe-
| cie d’ albero ,.0 di fottiliffime ftrifce di legno in-figure:,¢ forme varie, in tuttes
i Je\quali che tieno., ha fempresil manico ; che fenza manico fi chiama corbello,o
’ panicra 5 € feruono Perporcar frutte , 0 altra.che fia ; detto paniere , 0 panicra
torfe es » perehé gli antichi tenevano il pane in tal forte di cefta in mezzo
alic menfe , ¢ percid dat Lacini detto Patarium .
\ Fed Kila ruppanelipanere . Quefo proverbio dice;
Chifal alsrni meftiere
ring ta es eeude. en
JE ¢osidichiara iluo fignificato., quale ¢ : Che colui, il quale fi mette a fare
tuna.cofa,, che non fa) fare, non fara nulla di buono; ed in fultanza vuol dire ; Af-
anne wane. Ovid. libs )2.
» Kique liquor rari fub pondere cribri
© Ede forle:meglio dir Juppa , che 2xppa yenendo dal verbo /uppurare , che vuol
dire attrarre ? umido ; 0 da Suppen r Tedeleo.- Vedi fotto C. 4. ftan. 25. Ma lufo
isiobligaa dir zup
ok tn pace, aan pfamo. dire y quando:mandiamo via i poveri , che accatta>”
oj |) asd.in un certo, modo Pjauto in milit. dicendo Pax , abi,
MANGERAT de fogni. Mangerai cofe immaginarie , clog non mangerai.
ee ele Capitolo della paverta dice . :
of can ls vn Chevsfacciara ralor non fi vergogni
| lo permetta ye faccia male ,
Latin i on » che non cake viver di fegni ,
if _ A Latin Latini pure bayevan Gmil modo di-dire , come fi vede in Givyenale Sat. 6.
ia Qualiacumaue voles ude fomnia vendunt .

E.coloro,che map veaeie accents dana cofa,fogliono fognarla; perché

Saag non ¢ il fogno , che
Vn’

ba corvette
. 9. introduce un, Paftore , che raccontando le fue felicica

Poff ideo quecumyue Tolent in notte videri
Ln [omnis , vim magnam ovium — capellas .

 

‘La onde, Teocrito,
(Gosi ragiona :
 

4 MALMANTIDE

Er anco noto Nonio , che appreffo gli antichi Roniani , il verbo ve/cer fignifi-
caya vedere: Prius quam infans efferytni oculi facinus vefeuntir ciok videnr;come noi
Pure diciamo ; Afangiar un con gli occhi , quando altri guarda uno:con grande ac-
tenzione; ¢diciamo anche: . Dar paffo acti octhi. Dan. Par. Ci 27,08

Efa natura, ed arte le pafture
Da pigliar occhi : 2 ¥

Si che dicendo mangerai de fagni , fi pud anche intendere, Ti faxieruijeifeddisfa.
rai con dar pafto a gli occhi; 0 della vifea ; che ¢ bo tictioche non mangerai . Vedi
fotto C6. fan, 55. che er la vifta.

 

 

 

STANZA. V4ill, STANZA IX)

Comer replico quei.) fe.e' fi.cicalay E non barteva la mia fine alerove 0%
Che tu darefti via fin ta gouneliay C’adhaver prima ch’ io ferraffi li ocehi
Vedendomi [pedato , ¢ per.la mala dn ricompenfa un d:, piacendo a Giove',
Petra haver’ jt eranchio.alla fearfella? Della miadonnaquattr’sfeimarmocchi,
foi che tu grarti ij corpe alla cicala Ma finalmene dopo mille prove 3
(Life it Duca) ia levi qucfta cannella Didar’ il luftro a marmi coi ginocchi',
ser quel ch'io ti diro , percht fe gid Tenendo giochi in molte,e il colloavite,
Donai-, non era tuttn carita. E Le nocca cot perro fempre in lite;

STAN ZA) Kina ere ‘ ars

Lot bebbi bianca a femmine,ed a mafebi , Perché. po poi(difs' io)2li¢ me'chtiocafchi
Ond’ io sbraciar volendo a bel diletto , Dalle fineftre prima, che dal tetto;
Mi rifoluei levar quel vin da fiafchi, - Bil cavarmii di mano aiteffo un pelo,

vE aomdar pile quant'un pinrabd'agherco, Sarebbe un voler dare un pug hioin Cielo,

1, Mago mofira di-non ipoter eredere , che havendo Perione nome'di liberalif-
firio, non s' habbia a muover’ a compatlione di Iwi 5 ¢ Perione vinto dal” impor-
tunira di coftur, gli dice , che fu gia liberale'per difporre’il Cielo ‘a concedergli
fig\iuoli; ma perché eg? non era ftato efaudito , lafcid di far pili limofine, ed
hora era impotlibile cavargli di mano'un picciolo.,

Sf cicula, Ciok fi dice ; Si dilcorre. Il verbo cicalare-ufato in'quéfi termini
«{prime difcorfo di-cofa incerta , che frdice anco bucinare y oburicare’, E fi dice:
Ja tal cofa non fu poi vera;ma fu una cicalatayciok {ene parlo; ma non'é poi ftata

vera. “ terigacih ;
‘DAREST 1 via fin'la ¢onnella « DaréRi via fino al’proprio veRivo ; datetti via
tutto il'tuo havere.. E fe bene gonnellas' intende una {pecie’d’ abito da donna, in
qucito proverbio diventa nome generico per ogni forte d’ abito .
SPEDATO,, Cie co’ piedi laceri dal viaggio . ’ vere
PER Ja wala. Cioe per la mala via’; e's’ intende thal condotto di fawita,e mal’
all’ ordine di veftito , ¢ feaza danari '. tiowk wie? 3 mS as
HAVER il granchio alla fearfella’. ChiatniathorGrianebio , © ee ndfpetie di
malattiadi (pafimo , la quale quando viene alle mani impedifee'il maneggiare le
dita; E da quefta quando diciatho W rule bait eranchio alla forfella intendiai
non pud adoperare lé tani incorno alla borfa, che'vuol dire; & pigro acavar de.
nari della borfa , cioe , adire : ¢ tenace,, o avaro., ed uno, de’ quali parlando
Marziale dice. : aie

 

ys

 

 
 

 

SECONDO CANTARE. #8
tpi. a Bitigat’y& podagra Diodorus , Flave', taborant ;
ne nh ‘ Sed mil Patrono porrigit ; bac Chiragra eff. : j
tx. | Enoi pure diciamo di quefi tali; Haver Ja gorta alle mani, Haver i pedignoni

alle mani; Haver le mani acgranchiate ; farebbe a pagar co’ monchi ,
SCARSELLA, Intendiamo ogni forte di tafca , 0 borfa di danari , come fi
vede forte C, 3. flan. 5. , fe bene {carfella ¢ propriamente una borfetta di quoio
if Con ferrature di ferro fatta alla foggia delle Carniere da’cacciatori ; fa qual forte
ei di borfa ufava gia in Firenze portarfi da tutti legata a cintola . .
GRATT AR’ il corpo alla cicala , Incitar’ uno a difcorrere . Vedi fopra Cant.
primo fan. 2.1 Latini pure difflero in quefto propolito Crcadam ala comprehendere.
LEV AR 1a cannella , Defiftere di fare una tal cofa . Traslato ‘dalla botte , alla

hi fileva la cannella,quando é finito il vino, che ¢ra in efla, K cannella inten-
t) amo quel legnetto tondo forato per lungo , che fi adatta al fondo della bottes
bi, per cavarne il vino,la quale da i Latini con voce Greca fi dice epiffominm, Si dice

anche in quefto propofito .
i LEVAK A! vino da fiafchi , come vedremo apprefio .
; PRIMA ch io ferraffi eli ochi . Prima ché io mor:ffi .
MARMOCC HE, Ragazzi . Queita voce marmocchio in fignificato di fanciullo,
viene da marmo, alla pulitezza,e lifcio del quale s' 9flomiglia il lifcio, ¢ pulitezza
del volto de i fanciulli , ¢ delle fanciullette. Or. Od.-19. lib. 1.
h Vrit me Glycere nitor
Splendentis "Paria marmore purius .
DAR il luftro @ warnii'eo' i ginocchi . Cioe Naya tanto tempo , € cost (peffo in,
ginocchioni , che il lungo fregare con le ginocchia faceva divenir lucentt i mar-
, ini’, fopra i quali s'inginocchiava .
1) FENENDO gli occhi in molle . Cio’ lagrimando , ¢ cos) tenendo gli occhi ins
molle nelic lagrime . 2
COLLO a'vite. Collo torto, come fanno i Bacchettoni. Si dice a vite per fi-
militudine , effendo /a ve uno ftrumento ; il quale ferue per ferrar’ un materiale
con I’ altro, che peg effere attorcigliato come /a vite pianta , che produce I’ uva ,
sn <r? nome’, ¢ fi dice anche marti ¢ chiocciola : quello dal torcere , col
‘ > fa ‘operazior ione ; ¢ quefta per Ja fimilitudine , cHe ha la {ua figura con
_ il'gufeio della chiocciola a .
BLE nocea cal petto fempre in lire, Ciok dandofi delle pugna net petto ; il
che moftra che le nvcca fiego in lite col petto, mentre non ceflano di pei quoterlo.
E nocca'intendiamo nodelii delle dita. Vedi foro C, 3. flan. 8. , ¢C. 9. Itan. 54.
In fomma i} Poeta con quefte quattro maniere di dire, ioe Dar’ il laitro a’ marini
0", ginocchi § ‘Tenere chi in male , Haver il colle a vite ; ¢ le nocea fempre in lite»
fe eee Pie meer” orande ; ¢ defcrive affai bene un’ Hipo-
a ; ao € Al» "
ALO F bebbi Slee pens ha da confeguire per via d'eftrazioae “
di-polizze ( come fi fa al lotto ) {ono fcritte Wolaniedele polizze premiate,¢ I'al-
tre fon bianche ; ¢ chi ha una polizza bianca,non confeguilee premio alcuno . E =
di qui viene:il detto 40 / bo bavuta bianca , che é fatto comune , ¢ per intender di ;.
futte quelle cofe , che fi tenta di confeguire , e non fi confeguifcono.
ene c K2 SBRA-

      

 

tt
Mar

 
96 /MAEMANTILE

S8RACIARE , Vuol propriamente dire, allargare , ¢ follevare la brace a fine,
che meglio s’ accenda , € renda pit calore ; ma per metafora intendiamo fpender
prodigamente, ¢ largamente, come.s' intende nel prefente luogo , ¢ fotto Cant, 3.

an. 2. is

A bel diletto., A pofta ; 0 per guito , ma fenza buon fine , ¢ utile 5 ¢ fidice an-
che a bello fiudio., a bella pofta , a bella prova., che wuiti fi pofion Pigliare in guefto
fenfo., Se bene alcune volte fignificano quel, che i latini diflero dedita opera es»
maffime quando non v'é l’ aggiunta di be//a, che in quefto.calo ¢ detto ironica~
mente , ed ha forza d’ cfprimere Liafimevole y come per elempio Veramente tu bat
atta wna bella cofa, cide w hai fatto una cofa biafimevole ,.¢ che fla male « 4irg,
Egreciam vero landem , & {polia ampla reportas .

NON darei quanto un puntal d' agherto, Liaghetto una cordicella fatta di.feta,
© d'altro , che ferue per afhbbiar le velti, ¢ adattarle alla perfona, alla qual. cor-
dicella é folito fare una punta di {ottil lamina d ottone, o d’.altro metallo,,¢
gucfte punte fi dicono panrali , ¢ di quefte punte fen’ hanno due, .0.tre per un»
quattrino ;¢ da quefla vilta ferue,il_prefente detto per efprimere ; Non dares wiente,
ne meno una cola , che non val nulla. Che 4 latini difiero fra’ altre molto, /7-
tiofam nucem non dederim . E noi pure diciamo we fico fecce , um dnpino-, ¢ Gmili.
Vedi fotto C, 3. ftan. 8.

LEV AR il vin da fafchi, 1 fenlo metaforico é lo fteflo, che levar la cannella
detto poco fopra ftan. 8,

- PO poi, Alla fine. All ultimo de gli ultimi.. Opera anco.in quefto detto la,
forza della replica , che induce fuperlativo , Vedi {orto in quefto C, flan, 73,

GL’ é me ch’ io cafchi dalle fineftre prima che da} tetra, Nel male & il meglio,’ ¢-
leggere il meno. Intende ; egli ¢ meglio, che io lafci flare di dare il mio. che fe-
guitare , ¢ darlo via tutto , cioé mi conteati di queflo danno,, ¢ non lo faccia
maggiore col feguitare a profondere il mio. E que! me per meglio ¢ la figuras
Apocope da noi fpeflo ufata ; ¢ I'uso Dante pili volte, ma notabilmente nel C.
32, dell’ Inferno , che |’ usd nel principio del periodo..._, aceatijaat

Me fofte fhare qui pecore , 0 zebe. sda ene t nos
Ma di quefta figura Apocope , ¢ come | ufiamo , vedi forto in quefto C.,ftan, 36.

CAVARMI di mano un pelo, Conleguir da. me cofa alcuna , ancor,che di nian
valore . - ; slob ned

SAREBBE un voler dare un pugno in Cielo, Sarebbe un yoler tentar,unacofas
impotbile , Facilius Calum digito actingeres, {>

  
  

 

       
 
  

eng ANZA Xt. RIAN Zh AM, Se

Che pagherefti ( diffe f editro da te nov alpettar ch‘ io. chiedds;\

dm “here
Se cio fulle ( rifpe/e Perion « Rerehisamanciique abiesta altri mivedas
Mocor eis nod yh fabtia tins dfegoa, \ deboine, vale jd li von
Eraluoglia appiccara habbiaaltarpiones rf fra fe ru brami a day
to ti vorrei donar mez" il f pil regno dopo regoverniy m!
Sogginn[e quei: Non vo pur'uns sane 2a ling y
4a folamente La tia 4 Cua cuor th portin a. i

13 Siglo th s D389. 3G

 

* STAN

 

 

 

 
 

ae per = en

SECON DO CANTARE, 77
S T.A.N)Z.A XH.
Ed ordina di poi , che fe ne quoca.» « Prefa che thaglie fateoit becca allvca,
> Ga terza parte in circa arrofto,e leffa, Che {ubito ch’in corpo fe eacfa,
» (Chim tutti modi é buona)edan'un poca Senza che tu pits altro le apparecehi,
nds nquel modo 4 mangiar alla Ducheffa; Dotrela pregna infin [opr'a sii th
1 mago s' efibifce a dare a Perione il modo , che la fua moglieimpregni ;

Perione glidice che fe cid fegue li vuol donar t mezzo il fuo regno ; ed il ak
ricufando il tutto y.da.a Perione laricetta dellAfino marino per impregnar la.

maoglic..
CHE : pagherefti ? Quando veggiamo uno , che fommamente brama di fapere,o
a otenete una cofa ; per moftrare , che € in noftra potefta l'adempire il fuo defi-
ae »fogliamo dire; Che pagherafti? Che /pendere/ti ? Quanto darefti? © fimili ,
ares fied » o.diceffi la tal cofa ?
| EGONVE »\ Maliardo , Mago, Negromante , ec, Viene dal.Jatino, feeon,
| doh oer Murcto nelle fue varie lezioni lib, 12, c.19. emendando ua jug-
di, Plauto nelle B « Longurs eft Stris ion... Strigas
( dice egli ) vocabant mulieres y quas etiam noitn velare arbicrabantnn » codemgue.sode
rer: bamines.maleficos, 5 quorum vocabuiorum vulgus in Italiansitur , Vedi (oto
3: 69.
, L0.non ve pite.difegne . Yo non ho pila (peranza d’ ottenere quefla cola. N'ho
affatto levato I’ animo , ° il penficro e
= ARRIGUCARE Ja: ong te arpione .. Haver lafciata la voglia y 0 i} defiderio
Eee lo. che e-dppicear’ al chido vio fopra C, 1. fan. 8, Eque-
fe procede, da i voti-, che anticamente facevanoii Gentili ,

 

   

jn Tempio ; i.quali non & potevano levare di dove eran polti ,

DS coupe ticels in ufo comune , © profano .
0 RINE, E una fpecie di chiodo uncinato per ulo di regger I’ impofle delle
parte netinetire, , girando, quelle fopra di edi. Da i Latiai decti Cardizes .

Fe 40 pear Eas Non, yoglio danari. Crazia & delle pid vili monete

rESOtO pas eflendo, ? ortava parte del giulio .

sn: 2 ¢ maflime dalla gente vile per cA ceaense Nona

sects ta cola.

 
 
 
 
  
  
   
   

a feruitore agli huomini xirmofi, edi gar- i
> I tale» & un’ hanna egpiesode il.detto di Diogene

diamo huomo dotto , virwolo , ¢ di tutta perfezione .

serine de ale PAutore fi {erue anche acliottava, 7, ante-

_sedente.), seaoecs is aserstaone dun difcorfo, ¢ paflaggio ad
‘oro sonia d ee 3 Ba baftanza quanto. babbiamodetta,per
‘gonchiudere iLcome., ndo; 0 pe 90 non farelaral cof... —~ ;
« REDA. Ciok lyecettion ide ° danende fighuoli..: Seer: reday
iL tale ba hanuto.nn figh E egies Vinee iorentina »: sprepufata x€
folamente per i contadi ; d intendono anche i Peet Lic beitic.

MOSCA, Bind hat be venditori di pelce, che vivevuno al tempo,

che P Autore compole quef?

) GLI é farto il becco aro bs Deneeid &conchiulo, che i Latini diflera: JZ
’ ales. {) Lailt nella fia Bh. Hee . 2. than. 64. ae “xe

    
78 MAEMANTILE:
We vanno tuiti: il warcie hora-figineca,
Non v2 rimedio ;. E fatto il becco all’ oca "

Dice Francefco Cieco da Ferrara nel fuo Poema intitolato il Mambriano(Ope-
ra nota per efler l’ origine , ed- antefatto dell’ Orlando innamorato, Poema del
Boiardo , ed in confeguenza dell’ Orlando fariofo di Lodovico Ariofto) al'Can-
to fecondo 5 che ibe & SORE
» Pugia nel Regno di Cipri un Re chiamato Li¢anoro il quale’ havea una fola
y» figuola nominata ‘Alcenia , la quale amande egli-al pari di fe fteflo 5 vole fat
»» pere , fe buona, o ria fortuna ella fule per havere 3 fatti perd chiamare*alcuni
»» Aftrologi fece fare Ja nativita alla. medcfima (ua figliuola ; © tutti concordaro-
x» BO ,che ella farebbe prima flaca madre, che moglic ; ‘Onde il Re per evitare
»» il prefagito vitupero , fece fabbricare un giardino contiguo al feo palazo rea>
»» le, ¢ dentro al detto giardino edificd una fortifima’, ed altifima Torte cons
x» Molte ftanze , ¢ con tucte le’ comodita , ma fenza fineftra aleuna, che riufcif-
» fe fuori della Torre: Dentro a quefta mefié Ja figlia'con alcune Matfone,e
xy Damigelle , aficurandofi dell*ingrefio della medefima » non folamente col te:
»» Herne cgli proprio le chiavi della porta, ma con haver deputate accuratiffime,
»» ¢raddoppiate guardie di foldati intorno , ed alla: porta della*torre , ed alles
2» mura del giardino ; ne altri cotrava nella torre , che una fola‘donna , della
» quale il Re fi fidava , © le dava Ja chiave ogni volta , che a lei occorreva an-

»9 dare alla Torre con provvifioni di vitto , od’ altro. bias * O33eik
>» _ In quefto tempo mori un tal Co, Gio: di Famagufta huomo ricchiffimo’;- ed
29 alquanto parente dg] Re, © Jalcid érede delle fue'immente faculta Caffandro
23 unico {uo figliuolo ; Queflo giovane fece fabbricar wn palazo fonwofifimo, ia

x» cui teneva corte bandita con tanta fplendidezza, che fino al medefimo Re ven-

»» ne voglia d’ andarui , ¢ lo mefie ad effetto / Andatovi dunque fu dal giovaned
» inuitato a cena , ed il Re accetto I’ inuito , credendo fargli conofer , che non
»3 era in grado di banchettare decentemente un Re all improvvifo . Ma tutto il
»» Contrario avyveane , perch i] Re fu cosi ben feruito’ , ¢ di vivande ,¢ di mufi-
sy che , ed’ ogni altra cofa coaveniente ad un banchetto regio , che gli parve++
by che Cattinio havefle maggior poflanza ,' che non haveva égli ; onde “eS
») cid ad havergli inuidia , ed a penfare come’ porefle ‘morti carlo’; Haven
pera veduto fopra ad una maravigliofa fonte , che era nel giardino , un- motto
ay Che diceva Omnia per. pecuniam fatta funt . Si volto a Caflandro ,'¢ difie+

3» Motto € troppo prefontuolo , efiendoci molte eofe , che non fi pofion fare col
» danaro, AJ che rifpofe Caflandro: Sire , lo ho pofto quivi quel motto | per-
3x ché mi fon fempre creduto 5 che i] denaro apra la frada anche al? les
31 ¢ fino a hora mi € riv(cito, come appunto m: (Gn figurato,Horfu (replied il! RE)
yy Gia che tida il cuore-di porer fare gni'cofa col'dcnard , io’ ti’ do ‘tempo

»» anno a procurare per le trade , che vorrai y'di godere 1a inia'fighuola,ehe io
sy tengo pella torre guardata,come tu (aise (@ dentro a quetto me quae
eleva

  

x t0, fara toa moglie ; quando no , Ja tua tefta paghera la pena. EB

» il Re, perché eflendo entrato in fofpetto della potenea di Catlandro 4
y fotto qualche pretefto levarfelo d’ avanti-. 7 A
y»» I povero Caflandro ritnatto sbalordico da tal-propofta,meditaya

 

<4

 

 
 

 
 

SECON DOC AINT AIR E, 99

y» bando dalla patria , “quando Euripide fua Balia.; fapata lacagion® de! fao di-
» {guito gli diffe, che fi confolafse., perché ella haveva un {ao nipots dotazo di
‘97 Cosi grande ingegno y che afsoltitamente gli haurebbe aperta Ja ftrada all’ in.
9) \grefso' nella Torrey =.) ee
x» Quefto nipote della: Balia,Ewripide fabbricdunOca di legname 5 grande,
y tanto, che potelse agiatamenté afconderfelesin corpo \un':huomo y chev’ en-
»» trava,"eutciva perdi forte Pali, e\per via discerti ordinghi facevafare a tal’
35 Oca tutte loperazioni ;e'moti-, come fe fulse fata'viva , edera del-tuttoper-
99 fettary {enon che lemancava i] becco . Cafsandra fece fparger voce’, che era
syeandato! in lontani-paefi»; ed: intanto ‘havendo: fatta portare: occultamente 1a
+9) detta Ocatin-un luogo-remoto:;¢entro nella medefimay ed/Buripide fia Batia io
“¥; abito: morefco la guidava | sfingendo di venirdal Cairo dove‘era verimente >
y nata , ed allevata detta Euripide ) e"parlando \in-guelladingua ben? intela a1
“§) ‘Gafsandro,'toccava con:una ‘bacchetta |!’ Oca 5 cd'eva ih concerto’, che Cal-
~gy fandro iper via di certe Zampogne facefse cantar Ivoca. L’ aftutasBalia , ac-
y» cennate a a I’ operazioni dell’ Oca , andava dicendo , che‘a-volerla vedose
$y operar colegalanti ,:¢ maravigliofe , bifognava fpendere; ¢'perd il popolo,
»» mefsa infieme buona’ fomma'di monete,la diede alla Balia plaquale fece tare
1p alrOca diver belle’operazioni. i
+4) Arrive la fama di quel? Oca al? orecchic del Re , \¢ della‘ Reginay ond»
«39 ‘fattala‘venire a fe , dopo haveria veduta operare’, regalata Euripide,la man-
3 darono! ad: Alcenia toro: figliuola per farle-pigliar qualche {palso, € diverti-
«Jy mento hei ginochi dell’ Ova ; 1a quate condotca nella Porre;il negozio ando ia
“59 maniera ,-che-per via deteattati della Batia , Cafsandro'nelto farein camera
«59)'d" Alcona afedtoinquell’Oca:; fi godé Alcenin-, efi diedero.ja fede di Spot.
‘fs! Batro-quefts,Cafiandro accomods'all’ Oca i! beceo; econ la Bulia‘afcofto nell’
y» Oca fen’ ufci delia torre, ¢ prefentacafi laBalia‘con I Oca dv avanti al Re, ed
§5)° alla Regina’ per domiandar’ licenea ; i Re'difwe : Queft’ Oca-hail becco, evpri-
» ma non I havea? £ Ja Balia rifpofe : “Non fe le‘era meso » perché non eras
x» ancor fatto: e Voftra Machatenga aimemoria-quel che ora ha detto,
«3° Fra pochi giorni(pird il termine’, dentro-abquale\Caffandro doveva baver
“Sy” goduta Alcenia,, onde il'Re fe lo fece condurre-avanti., ¢ Caffandro diffe ; Si-
“> te'V. Mvfacciarvenize Euripide mia Batia , I) Re lo-compiacquey'ecomparla
© Sys “Buripide'com I Oca, fu'dalRe fubito riconofciura’yed ella girdifse :°V. Me Gi
“yj ‘Tivordi che @ facto if becoo all’ ea; ¢ fatta quivi condurre Oca fece en-
a» trarui dentro Cafsandro , ¢ lo fece fare le folite operazioni , accid che il Re
vj, ‘Conolcefse »che quella era la fteSa Oca ,-che ‘in quella ftefsa_maniera era di-
“3, morata pili giormi con Alcenia nella' Torre: onde 1! Re conofciuta l aftuzia di
Sy, ‘Cafsan ,¢ fay recifameateril fatto , ¢ che Alcenia era gravida , ed
“49 havea data la fede di (pofaia Cafsandroyiconfermé i) matrimonio per ofseruar
Ja parola,contentandofi di cedere alla difpofizione delcfato'; Eda quefta trave-
~* Nita trasformazione diGiove in)Gigno & nato il'proverbio: E furto il becco
-alt' Oca ; che fignifica ( come habbiamo detto ) il negozio é fatto ,-0 perfezic-
snato . Quefta ,-o fimile novella leggefi in quelle di Giovanni detto il-Peeoroas.

 

STAN-

 

 
2 ou

 

80 2M AIM ANTIAUE BD 3 2
STANZA XIV.: " STANZA sKVeb-
O ques ( diffe i Duca) é veramente. Benche fuffe coftuicom! una pinay).
Da pighar con le malle ; Ch unfamaras << Tanto targo, y errs af
Poffa col cuore ingravidar la gente ; Per non balzar.un trarto alla 2
Vedi non ti fan finta:, io non la paro «; 1 pefextori wennera in pacfes \)
For sis il prowar stom ha acoftarmente,| Cost pefeando Lingo ta marina;
E quando mi coftaffe ancoben cara Leche benedett’ aofino fi-prefe m
Vo farlay per veder, fe cid riefce ; E il cnor nun bel bacina snargentato,
Pero fi mandi al mar per quefte pefce . 4 fuon di pive al ‘Duca fu portate ,

I) Duca fentendo che il cuor d’,un’ Afino- marino ¢raatto\asingravidarila:mo-
glie , fi ride-del mago ; ma tuttavia era cos ernie deere shaver figliuoli,
che volle p ze do|, che i p gediedi fi-
nalmente lo prefero , e portarono il cuore alsDuca gi 139) nupvaile bo sara

E DA pighar con le male. Buna grofia minchioneria 5 ésna0,(peapotito. gran-
diffimo . elle intendiamo quello fleumentordi ferro eg pypnitor anamrsan a
boni ardenti:, ec.

VEDI, Quefto termine ha del giuratorioy quali dica: oe fede mia: onescin ne
to credo, Credi a me che tu fai male, ec, Vedi foto: CyB, Nano 63.

NON Ia paro. Non la credo. Tratto dalla Riffa ‘so-Maila ginoco dicdadiael
quale side tino tien Ja polta dice’; Parodi y ¢ nomilastenendo dice Won la pare.
LARGO come-una pina . Si dice largo com’ una. Ss verde ,la quale ftrettiflima ,

¢ ben ferrata ; Comparazione ironica , perché ) fargo vuol dir liberale , ed
huomo frerta "viol dire avaro »etenace; Si che fendo agi pina, verde, ftrettidiima ,
comparandofi un huomo a quelta; sintende trettifimo ;.cioé tenacifimo , ava-
rifimo , che i Latini diflero Laro facrificar; che fuona;,Glié divoto della folaga,
Ja quale perché @ di natura vorace y fervivaa iJatini per siprimeresieileome
avido del denaro , ¢-lo dicevano Larus hians,,

JGNORANTE . Voo che non fa.) Vedi fopra C. 1. han. 73., Ma wale ancora
per ingrato y Zotico y villano , €'poco amorevole ed in quetto luogaé oo vinta fen-
fo , nekqualeé fempre » oper lo: pill, prefo-nel contado ..)

PER non balvare . Cio’ per non andare... Si coftumaidire belesre pee. andare ’
o cadere injcole di difguito , come balsar infermo in un letta,, baizare in una prigion
He y€ee Non fi direbbe balzare ainn banchetros ¢ fimili ; -Per non balzarein wna pri
kiove j quanti noi fiamo , fara necefjario che altri di noi balzino in ae yedaieri fi
Latuino in Chiefa, Difie ! Aurore , che {eride la vitadi quei tre fi ladei Fio~
rentini.

BERLIN.A, Euna (pecie di Moment 40 gaftigo pcheifi daca i Jadroncelii
re Joro eS 1h a ae alate
an iu pape i frequentati ye quivifila ae ’ in
folome tals = Quel firamento fi ebiama anvora Gogna.. itMeah fo110.C. 3.

 

 

 

sear eae tisuuist 1

VENNERO inpacfe . Ciok coat earn 5 fidafeiaron, trovare..«) ean ri-
trovamento di cole afcofe 3; Ed © lo ftetlo.che venire in foena deco nt.
1, ftan, 2,

QUESTO benedetto Afine, L? epiteto benederto in tali occaloal a ‘dir’ nag

   

 
 

Fae

SECONDO CANTARE, Sa

to bramatu . To cerco del tale , del quale ha grandiffio bifogno }¢ quefto bene-
detto huomo non fi trova ‘ a & chimneys t

BACINO , 0 bacile» EB’ un piatto d’ nto 5/0' d'altro metallo'grande'pit:
della folita mifura de i piatti da cavola.,¢ ferve propriamente per ricever l’ acqua,
che fda alle manv alle tavole de’grandi,fe ben s'adopta anche in molte altre occa-
fioni , ¢ per altri effetti . 4 i mE

PIVA, Dicemmo , che cofa-fia fopra C. 1, ftan, 34. alla voce cornamufa , T
contadini fogliono per il maggio andare attorno cantando , ¢ fuonando la Cor-
namufa , ad effetto di ragunar denari per far'con effi regalo.a qualche !uogo pio,
¢ ricevono I’ elemofine , che vengono lor fatte in un bacino , ed in un’ altro'por-
tano quel tal regalo,che voglion fare; 0 vero I appendono ad.un ramo d’alloro,
aaltro albero:, ¢.dicono.quefta:lor gita., andare acantar maggio, ‘Tal coftume»
tocca il noftro. Autore:con quefto modo? di. portdre if cuore dei!’ Afino mavino al
Duca.

STANZA XVI. STANZA XVIL
Ed.egli prefoil prelibaro Cuore’, Allor vedefti partorire il letto

Lodiede a! Cuatvoyalqual métre boicofse, Vn tenero: ye verzofo lertuccin';

Si fece una trippaccia la maggiore Di qual) armadio fece uzo Ripetto ,
OCT sdilde’ wachmai veduta fofse 3 La feggiola di ldun feggioline 3

Le robe , ¢ mafferizic a quell? odore Latavola figiio un bet buffetro.,
Anely elle divencaron tutte grofe:y xi. > . 9 Hacalfawn vago , @ piccol cafretrino ,
« Ein pocasompo.4 un’ otta tutte quate. Bildefirenncanteretto mando fuore ,)
» Becer A! sccorda,il pargeletto.infante', . C ana bocchina havea tutta fapore .

STANZA XVILL

4, Cuoco anchegli poi non fu minchione ,: Ch! in far vivande faporite , e buone 5

¢ bxce witofi n’un fiance , Fu fubso fqnifite ye molto franco ,
i vedde prima ufcirne uno fidione ; Ein quel ch’ il padre ftettefopr’ aparto.,
; oe Guatterinoingrembiulbianco, -  CavindinCorte,s lhi,e al terzoye al quarto
ol dette-il Cuore, al Cuoco, il. quale nel cucinarlo ingravidd, fi come an-
cora tutti gli arnefi , ¢ mafferizi¢ , che ne fentirono.!,odore »,¢d ana medefima-
hora Partarirondaisios sig 4 oom of 3 A

 Quivorreiry, leetore fi ricordaffe che il Poeta, nel comporre ued? Ope-t
ra ha, Pe pA pce cei qaeldeinonslieseied alte iat n't:

contate ai Fanciulli (come habbiamo decto ) ¢ che pera (ta dentroa’ termini di»

quelle favole y le qualt com: per lo pili inuenrate 5 ¢compofte da quelle meded-

me doanice  fuperare Ja capacita di queitey ne: di-quelli, ¢ fi
contentaffe di non, inasione nel (eosin dan tuna, cola tanto favolo.
fa , ¢ faori del, natur. ¢ ih far parcorire le:mallerizi¢ ; ed! oftervare j che

Oe. » doreo , nel (uo Cuntorde :li- Guati

ancora, Gio, E pur fashuomo

ha pais queffa , ed altre novelle fimuli ; a folo oggetto di cratteneee.li. picci-
illi 2 COME €Bhi dict.) 9! fon on asl sane) + Goons , oveMe KAO
_ LRELIBATO » Viol dire una cofa gutofa 5.0 fingolare ,, ma fignifica ancora
eo eperermenta narrata , o deta avanti, come ¢ nel prefente luogo , che Signi

 

fica il {uddetto , 0 accennato cuore ; ed habbiamo anche il verbo preibare Dan.
_ Purg. Cant, 10, ‘ a
“WEL L toy
-

 

 
82 MALMANTILE

Hor ti rimanclettor foprail tua banco
Dietro penfando a cio , che fi preliba.

4 di de nati, Non nacque mai veruno , che vedeffe un ventre maggior di quel-
lo., che haveva il cuoco «E un-termine: , che aroplifica la voce mai; V.giNeflu-
no di quelli » che fono ftati al mondo , mai vedde , cc. Py? bominum memoriam .

A un’ ota, A uno fteffo tempo ; a una medefima hora .. Vsandofi da noifpefio
la voce ora in vece d’ hora:: adefta in vece a’ allora, Che orta éegli? ia vece di che
hora ¢ egli? ne

FECER a! accordo il pargolettoinfante, S'accordarono’a partorire a un’ hora.
medefima .

LETTVCECINO., Intende piccolo lettuccio, Ma lettucciointendiamo una gran
ca(sa, la quale per di dietro ha wna fpallicra,e dalle teftate i bracciuoli , fopr’ alla
quale ¢ folito tenerfi uno Mraptnto , ¢ {erve per ripofo , ¢ per dormirvi fopras
dopo definare .

ARMADIO ec, Atnefe di legno per riporvi ogni forte di roba,il quale per lo
pid fi tiene affido , o aveosto al muro, ¢ fi apre come le porte , ed ha dentro ‘di-
verfi palchetti,, o catlette ; ¢ per fiperro qui intende piccolo armadio .

SVFFETTO , Intende piccola tavola .

DEST RO. Quello che diciamo anco luogo Comune , ed é quello , dove fi va
a {caricare il ventre . .

CANTERETTO.. Piccolo Cantero , e queflo &un vafo di terra , 0 di rames
o @ aitra materia , il quale fi mette dentro: alle predele ‘per recipiente all’ ufo
fuddetto , chiamato cosi per efler per lo pil) di figura: fimile a quel bicchicre che
i Latini chiamavano Cantharas .

VINA bocchina havea tutta fapore . Il Pocta {cherza , fapendofi bene , che fimil
forte d’ —- fuol’ efler fempre fetida , ¢ perd dice che eraeurto fapore , ciot fape-
va di qualcofa .

AUINCHIONE , Vuol dir femplice , corrivo : Ma qui vuol dire uno, che non,
fa meno di quello, che fanno gli altri v. g. Se tx pigli della tal cofa , non weglio effer
Minchione, ne vag tia pigliar' anch’ io, iva ae

SC HIDIONE , 0 fridione, BE -quefto ultimo & pil comune; ‘Vuol dire quello
firumento da cucina, nel quale s infiiza la‘ Carne, o Vecelli , per care arrofto,

GV ATT ERINO , Ditinutivo di’Guatcero, chet colui , che ferue d’ aiuto al
cuoce . Qui intende piccolo cuoco .

GREMBIV LE, Bun panno , col quale fi cinge la perfona forto lo omaco
per difendere il veftiro da’ g)i untumi ; decto cost 94a ‘regicoreminm ; ed in altri
luoghi d’ Italia Senale quia fimum tegic ;-¢ moltt Zomule da Bimie,

MOLTO franco, La voce franco, che vudl dir sees eer
— un’huomo ardito , coraggiofo , pratico 5.0 to 5) ne
nel fente luogo. au30 & (tiem Oug ROL odie: Ne ola
SOPRA parte - Quel tempo , che le donne flanno nel letto:dopo'haver parto-

i cagionati loro-dal'parto pei ef ca
: uae

 

ti. AYIb50

   
 

rito , per riaverfi da gli feoncerti
parto. i

 

Ou sty
STAN-

.- lt i

 

 

|
 

SECONDO CANTARE.

STANZA XIX. STANZA XX.

La Ducheffa ch’ il cuore havea inghiorrito,  Crebbera infieme , ed all’ adolefeenza
Cotto ch’ ci fu com ogni circoftanza, Peruenuti mangiaro il pane affatto ;

. Anch’ ella con gran guffo del marito Nel far fanta,nel far la riverenza,
Stampo due Bamboccioni d'importanta; Hebbero il corpa 4 meraviglia adatto:
Grazie,e bellexe baveano in infinito, Tra lor yon fu mai Inte ,0 differenza,
E coss grande, e tanta fomiglianra , Ma di accordo voleanfi un ben matto;
T ant eran fatti uguali,ed a capelle, L! Infante Floriane uno hebbe nome ,
Che non fi diffinguea quefto da quello. E quell’ altraeAmadigi di Belpome .
La Dachefla pure partori due belliffimi figliuoli , tanto fimili di fatteze , che

non fi diftinguevano |’ uno dail’ altro, Quefticrebbero, ¢ furono allevati con,

buona creanza , ¢ fra di loro cordialmente s’amarono. Vno di efi hebbe nome
P:-Jnfante Floriano , che yuol dire Raffaello Fantoni , ¢ I' altro Amadigi di Bel-
pome ; E queito é nome a cafo.

ST AMPO’ due bamboccioni a importanza . Pastori due belliffimi figliuoli , ¢ che
havevano tutte Ie condizioni, ¢ parti defiderabili ; B nota che il termine d'impor-
tanza ulatifiimo da noi ia fimili occafioni , vale in quefto cafo quanto il termine
di garbo , ¢ per efprimere una tal quale perfezione del fubietto . Li Lalli Za. Tr,
C. 1, flan. 54. dice.

E produrra , fe ben non fenza duale,
Due garbars bambucee a xn party fal .

ef capelo. Perl appunto. E il latino.ad wxguem . Termine ufato da coloro ,

che fi regolano col filo nello fquadrare , come fono i muratori , ec. E vuol dire

non vicorre Ja grofieza d’ un capello dall’ uno all’ altro ; ma fi ula in ogni con-

giuntura di paragonare , 0 milurare una cofa con!’ altra , non folo in quantita ,

coine Ho rifcontrate i denari , ¢tornano a capello ; ma auche nella qualita come nel

cafo noftro , che s’ intende : erano uguali di mole di corpo , ¢ fimili di fatteze .

MANGLAR il pane affatto. Mangiar bene , ¢ fenza far rofumi , 0 tozi ; mas
fignifica huomo di buon pafto. Vedi foro C. 8. fan. 56.

FAR {and , E10 fefio , che far la riverenza ; ma éun termine, che ¢ proprio
dei bambini,, ido comingiano a imparare.a andare , che quel lor muoverfi
timidamente ¢ detto dalle balie far /auta , 0 pure é , quando fanno la riverenza

-baciando altrui da. mano ; ed¢ cosi detto eae fanita , cioé fare faluce ; falu-
tare« ‘Diciamo infegnare al Bue far fanta per intendere : Zufegnar le feienze, oi ter

saini ciyili aun’ bueme xatico, villano, ¢ di difficile apprenfiu

83

 

 

SJ wolenano ne ben marto, S 8! 20 fuile - Equel
termine A4attus , del quale habbiamo dexto fopra C. 1. ftan. 76.
STANZA Xx STANZA XXIL
drrivati che furono ambiduai Di modo che fdegnato , come bo detto ,
A conofcer bomai il pan da’ faff , Ch'il Duca per La {ua {pilorceria
_ Efaper quante paia fan sre busi , Og hor vie piie tenevalo a freccherts,
Se ben dal padre havean de gli {paffi, Vn di fri ed andar via,
Vedendofi gid grandi impwcatoi Ma tdcquelo per fare il gioco netto ,
Ed a foldi tenusi baffi baffi , Fuor ch'al frarelioyal qual n'unaofferia
Oftico gli pareva , e molto ftrano, Difse(veduto havendo 4 un fiafeo ilfide)

Bd ia particolare a Fioriano .

Volerjene ramingo andar pel mondo,

 

 
 

84 MALMANTILE>

Crefciuti quefii due Giovani, ed arrivati a condfcer il:ben.dal nialé, vedendofi
cost grandi pareva lor malagevole il hon haver denari 5 perché il padre perlaifua
{puorceria non gliene davaydi che'pid d’Amadigi fentiva difgufto Piorianosonde
fi rifoiuette dsandatovia 5 ¢ perché: ’ adempimento di tal faa rifoluzione non gli
folle unpedito , non ne parld.ad alcuno , fuori che al fratello Amadigi,
CONOSCER il pan da faffi ;\efaper quante paia fan tre buoi , Significano Jo ftel
fo, cive conofcere il ben dal male. Hor, difle, Novit quid dient era tupinis Si
dice ancora in quetto propofito Sapere 4 quanti di ¢ San Biagw , E quetto'denoha
origine da.un coftume antico , il quale era in Pirenze , che i ragazi fattori délle _
bosteghe d’ arte di feta:, che (on fituate nel Mercato Nuovo vicino alla’ Chiefa di
5. Biagio, havevano dicenza , paflato il di della fella di elo Santo ( che fax
sebbe alli 2, di Febbraio , ¢ fe ne fa alli 3. per caufa della Purificazione , il che
ha daso occafione di ulare quefto dettato ) di fare alle falace, € pigliarfi ogai
forte di paflatempo in alcune hore del giorno, ed abbaadonare Ja bottega per in<
fico a. tucto i] giorno di Carnovale ; e per quefta caufa era quel giorno tanto defi-
derato da i ragazi , che fapevano benitfimo il di , che fi tolennizzava la deta
f€fia ; onde colui , che non fapeva tal giorno , era fra i ragazai ripucato ua bag=
&<0 , ¢ che non havendo notizia delle cofe del mondo ( giudicata da Joroquelta
una delle pid importanti ) non fufle perfona abile , ¢ di tanto°giudizio da faper
fare i fact fuoi . E quefto proverbio s’ é fatto poi comune a tutti gli huomini per
intendere un’ huomo (ceruellato, melenfo,e buono a poco. Il Lafca Nov, 4. dices
La Stheggia yed t-Pilneca\, che. Sapevano a due once , quanto colut pefava, ed a quanti
dit San Biagio.
SE ben dal padre havean-de gli fpafi, Se bene il padre dava loro de gli avverti-
menti., ¢ paflatempi. Nota che per {cherzare il noftro Poeta , fubito che ha det-
to duoi (eguita dal padre , ¢ quefto fa per coccare quel coftume burlefco ; il'qualeé
iu Firenze (ma pero fra gente:bafla ) che quando uno nomina bae , beccos 0 ca-
firone,Valtro dira di tuo padre , edicendo vacca,dira di twa madre, e fimili, Vedi
forto C, 12. flan, 49. annot.al termine wmorirecon la grillanda
GRANDI impiccatai . Proibi{cono le leggi Y-impiccare chi non paffa 18/anni ;
¢ di quinel diciamo.grandi impiceatoi , cioe abili a elicr*impiccat , per antender
guelin,y che pafiano la decca eta dir8. anni. $ ”
ft SOLDIL tenmti baffi bai, Tenuti con pochi denari , Traslato dail’ acque,delle
guali quando ne fon poche nei laghi , pozzi, o fiumi, fi dice bafe . Vedi forto in
quetlo C, flan. 61. € parlando d’ uno che habbia pochi denari G dice : iL acgue
Jon bafe si come intefe colui con quel fuo motto ZL’ acque Jon baffle, et! ache banno
gran fete, cioe Alle gran veglie i danari fon ne « 2 :
SOLDO. Vale per intender danari,riccheza ..E foldo moneta immaginaria
(hoggiin Firenze eftettiva di bronzo)che vale tre de noftri quattrini;Spetio ufiamo
quelto termine per una certa.generalita : Il tale.ha de’ foldi,de’ quatcrini,del!’ oro,
Rer intendere € ricco» nonche habbia quantita di foldi , di quattrini ,.0 dro ef-
fectivamente yma molti-ne vale il uo ftato ; Equi intende-Monete «-—
. OSTICO., Spiacevole , Malagevole , lnfopportabile. E il Latino J he
vale per cofa da nimico .
STLANO . Qui ha lo fteflo fignificato a’ ofico. Vedi (orto C, 3, ftan. ae

. oz ‘ Te tro

 
 

 

 
 

SECONDO' CANITARE. 85)

altro vuol dire ftravaginte da eatranens . E molti dicono' rate ajuno che habbia
cattiva cera ye perinfermita fia mal condorto. + } Hal
-SPILORCERTAS Sordidezza, Avarizia. lo credo che quefta parola venga da
Pilorci , che i pellicciai chiamatio ao ritagli di pelle ,' che non eflerido-buoni as
metter’ in opera ygli-riducono'in fpazzatura , la quale poi veadono per governa-
rei terrenijse li dica /pilorcio quali huomo vile,ed abietto-quato fono quefi pilorei .
“ TENER’ uno aftecchetto, Bare flar’a fegno , ‘0 far patire uno di quello; che»
églicha bifognio ; come’non'lo taftiar mangiare ‘quanto ei vorrebbe ; 0 haver de’
danari quanti bramerebbe ¢ Quand’ uno per la(carfezza di danari vive mifera+
mente fi fuol dite: Atele ( difende?, fi febermifce , ec) ond’ io ton fon lontano dat
credere , che queflo termine fia corrotto , ¢.che*fildovelie dire'a focbherro da ftac-
cheggiare , che é I’ iftetlo che {chermirG , ¢ pud fignificare Eifere (carfo , 0 haver .
bifogao di denari.
‘ VEDVTO il foudo a un fiaféo, Dopo haver bevuto un fiafco di'vino ;'e cosi ha-
ver veduto il fondo'di dencro/del fiaico ; ed in fuftanza qui-vuol dire; Dopo ha
ver bevuato molto'bene ; ovaflai . a

ANDAR ramingo pel mondo, Andarfene errante. Ramingo vien da ramo,¢
fi dice Ramingo de gli vecelliidi Rapina , come elprime dl Crefcenzio nel Cap, 3.
della *bonca degli Sparuicri lib. 18. con le feguenti parole: Si chiama nidiace , v
wero che di nidio ufcito di ramo in ramo va feguitandola madre,e pero fichiama Ramingo,

Ed alli fparuicri & danno tre nomi , cioe Widiace, che & quello ,che ¢ cavato di
nidio., ¢dallevato., amingo quello che u(cito di Nidio non fa gran volate ; e>
Grifagno quello’, che gia patiato I" anoo ha mutato alla Campagna . Ma quelte
aoateat ‘noitro:, baftandoci , che a’ fimilitudine*ditali uccelli,, dicefi
Andar ramingo coli; che hora va in un luogo’, bora’s* incammina ip un’ altro ,
fenza fapere politivamente , dove egli vogha andare , ’
& . STANZA XXII

Anadigi 4 diftoris tutto un giorno Tn vnoiir difse se verote vain un forno:
Sr arrabbio , s aggird com'un Paleo ; E dopo un grande, é lungo piagnifteo ;
Ma perche quanto peu eli ava interno HHorsn-vanne(difs' egli)io men’ accordo ,
Egii ord piu'oftinatod uno Ebreo, Ma lafciami dite qualche ricordo .

» Amadigi (entita quefta rifoluzione del fratcllo, molto s'affaticd per diftornelo;
ma veduto'che per Ja di lui oftinazione s’ affaticava in vano, concorfe con lui ,
con quefto perd che gli la(ciafle qualche ricordo‘di fe,
 P-ALEO Cosivchiamiamo una {pecie dt erba’, che nafceintorno alle lagune .
Ma diciamo anche Palcouno ftrumento di legno, che (erue per traftullo, ¢ giuo-
co de’ ragazzi, il quale ¢ di figura piramidale al’ ingiti; e nella teftata, che viene
+di fopra ha‘ua manichecto condo: , il quale.avvoltato con uno fpago , 0 cordiccl-
las’ infila in uo’ atlicella,bucata,e tirandofi quello (pago fifaolta, ed i Paleo feap-
»pa dal buco dell’ aificella', © va per terra girando,portato dail’ampul(o di quelio
_ 'fpago . Tale dtrumento da i-Latinié detto Tu*bo forfe dala figura piramidales .
+ WVerg. 7. Aneid. Cex quondam torto volitans fub verbere turbo, T ibull, Nam

ue aSOry
“at per plana citus fola verbere turbo, Dance nel Paradifo C, 18. a
Ed al nome del alto Uaccabeo
Vidi moverfi un’ altro roteanda, =
E letizsa era ferza del paleo « i EG

   
-
86 MALMANTILE

E dice,cosi, perché a tale ftrumento fi fa continovare il girare perquotendolo
con una sferza , dopo che egli ha hayuto il primo moto, ed.impulfo dal fuddetto
fpago . Ed il proverbio aggirarfi come ux paleo vuol dire affaticarfi aflai, ¢ conchiu-
der poco ; che i Latini pure difleru Trochi in morem circumagi, perché dicon Tro-'
chus tanto il paleo , che Ja trottola , portandolo dal Greco Treches , che vuol dir
ruota , 0 altro ftrumento che giri. Vedi forto C, 6, ftan,22. E forfe ancheJa yo-
ce latina T-«rbe fignifica tanto il paleo. che la trottola , perche Turbo yuol dire
ogni cola che habbia figura Piramidale , a rovefcio , cioé il largo di fopra ,.¢ da
piede acuta , come appunto ¢ il Paleo, ¢ la Trottola ; fe bene non (ono Jo fefio
come ci teftifica una certa cantilena aflai praticata fra i ragazi , che dice ,

E il Criffiano non ¢ gindeo.,
E la trottola , non ¢ paleo ,
E paleo non ¢ trottola , ec, q

PIV? oftinato d’ uno Ebreo, Oltinatitfimo , che non fi trova nazione pit oftinata
nella fua legge , che quella. de gli Ebrei , che pero ha meritato)il titolo 5 che le da
la fanta Chicfa di pertidi , Cino da Piltoia , O vei, che fere wer me sigindes: ciok

erfidi ,
. VA in nn forno, Va dove tu vuoi. E {pecie d’ imprecazione, che fuol far’ uno
vinto dall'impazienza, E fi fuo] dire anche in quefto propofita: Vain malorayva
al diavolo , va in galea,¢ (mili, Abi in malam erucemse Plaut, Epid, Ato ts {e.2-

ditle ; Atala iff usmodi mihi amicos furna merfos , quam fa
XIV. Ss

STANZA
Allor per fadisfarjo Flariano ,
Accio che pit tener non Labbiain ponte,
Con un baften fatato, c' hayea in mano
Tocco la Terra,e fece ufcirne un fate »
E diffe: Quindi poi ben che lontano
Vedrai sto vivo,o s'ia fono a Caroute ;
Perché quef'acquagzn’ or di pina inpito
In che grado so faro diratti appunto,
STANZA XXV.
Sal corfo di que? acgua porra cura,
Tutto il carfo vedrai di vita mia;
AMentr’ ella ¢ chiara, criftallinase pura,

foro,
TAN ZA XXVI.
Cio dette in capo il berrettin fiferra ,
Aerte man,chiude gl occhi,e frrige i déti
E da fi forte una imbroccata in terray
Ch’ il ferro entrovvi fino ai fornimiti.
La quel che i grills ,e + bachi di forterra
Sgombrano tutti i loro i .
Pullula fuori un cefto di mertella ,
E di nuovo Florian cosi favella
STANZA XXVIL
Fratel mio caro, quefta Piauta ancora
Com’ io la paffi ti dara epee ao
Ciot mentr’ell'e verde,anch' io allara

Di pur ch'io vwva. in feftaged allegria ; Son vivefrefco,e verde com’ un’ aglio ;
Ed all incantro, fe torbida, e feura 5 E quand’ ella appaffifce , efi coloray

Ch ella mi vacome dices la Cia; Anch'io lagui/ivod ho qualche travaglio,

Ma k gusapcdadlenste Sern il corfo, In fomma sell’ fecea , leva i moceoli.,

Di ch’io fia itoa veder baliar LOrfo. Per farmidire ilcantoinf{carpezoccoli.

Ficri wo per contentare il fratelio , toccd la terra con un baftone incantato ,

; che haveva in mano , ¢ ne fece.na(cere una fonte , ¢ difle che dalla mutazione di
guell' acque haverebbe egli conofciuto lo ftato, nel egli f tovafle . Dip

mefle mano alla {pada , ¢ con ¢ffa buco Ja terra, .¢ {cappo tuori. anor-

tella ; E moftrd ad Amadigi, come egli fi davea contencec in conalcere ancora.
; da quefla mortella , in che grado egli fi trovatic. i

 

<= ¢ pts act a |

 
SECONDO CANTARE: |

87

TENERE in ponte. Tener un fofpefo , o irrefoluto . I Latini pure differo: 2

pdetinere ; ¢ perd ttimo , che quefto noftro detto venga dall’ ufo antico de’
omani , che nell’ clezione de i Magiftrati chiamavano Pontes quelle piccole ta~

yole , fopr’ alle quali eran pofate Je paniere dei voti ; di che fa menzione Cic. 1.

Rhet. Pontes diffurbar , Ciftas deijcit ; ¢ canto Ravano incerti, ¢ fofpefi coloro, che

devano ; quanto le cefte de i voti ftavano fopra i detti Ponti; E' pero di-
cendo: Ego /um fuper pontes, vaol dire il mio Voto é ancora nelle Cefte, o coper-

to, ¢ per confeguenza io fono folpefo , ed incerto di que} che habbia a efler di

me, Eci ferue poi quefto detto Tener’ uno in ponte per efprimere ; trattener’ uno

con le-fperanze , 0 con altro fecondo il fubictto . ’

SONO a Caronte, Son morto . Son fra I anime’, le'quali paffano la Barca di
Caronte , che fecondo la faifa credulita de’Gentili era il Navaleftro ,il quale con-
duceva | anime de i morti con la Barca alla Citta di Dite. Vedi forto C. 6. ftan.
19: & feqq.

COME dicea la Cia, Miva male, ¢ peggio. Che quefto voleva inferire una
tal Ciay © Seia Fruttaiola con un detto fporco da lei molto ufato .

SON itoa veder ballar tOrfo. Anche quefto ‘detto fignifica fon morto .

IN cape itberrettin fi ferra yec, Con guefti due verfi efprime uno , ches’ accin-
ga a fare un’ operazione’; nella quale fia neceflario ular molta forza’, perche-ia
efi: moftra quelle azioni , che per lo pid fon (olite farfi in fimili congiunture .

METTE mano, Quando ditiamo aflolutamente meteer mano ; intendiamo met-
ter mand allrarmi. Diffringere enfem .

iene a via ; mae it ‘

‘qui comm pare i propofito il norarewuna ja generale portata dal
varehi ‘nel fo Hercolano ; civt che la lease tain hel frei qual-
ia dizione ne} noftro parlare ha la forza di privazione , come ai

Latin la particela m ha forza di negativa , come doftus , indottus, ec. Ed appref-

fo di noi eaitare y fealuare ec, Ha perd quefta regola anch’ effa le fue eccezioni,

come sbilordite vitol dirbatordo, € non vuol dire fenza balordaggine ; T urbare,ftur-
bare, diffeobare , che faonano'lo fieflo con I aggiunta , che fenza. Taluoltas
anitor’ s* aggiunge alla ‘deta yS), la particelia a , e particolarmente quando la,

eee ens vocale , come amare , difamare’; intereffato , difintere/-
Salto 5 0 ORES .

* CESTO', Intendiamo pianta di virgulto, o & erba , come Cefto di lattuga , di
mortella-,-ee.’ Se bene de ¢ virgulti fi dice anche Pianta , come fi vede nella pre-
fente ottava 27.Fratel mio caro queffa Pianta ancora, Viene dal latino Ce/pes, ¢ noi
pure diciamo' Cefpugtio. lo fimo, che pianta fia nome generico , poiché ferue,
per tutti li vegetabill , dicendofi Pianta di prezemolo , pianta di grano , e pianta
di oe , €¢. € noni direbbe di tutti cefto’, ne cefpuglio .

‘RDE come un’ Aglio, Vn bel verde fi paragona ad un’ Aglio , perch? quefto

ha le fue frondi di bellitiimo color verde ,€ che f) mantengono ver=_ .

di , ¢ fegno di fua pérfezione » E perd dic Ui tale ¢ verde come un’ aglio , s'ia+ 4

tende ; ¢ di fanita perfecta’ cruda Deo , viridifque fenettus, Horat, Dumques

virent genna , Quefta bmi fi piglia da tutte Je piante , la fanita delle quali

8’ argumenta dail’ efler ben verdi , che dimoftra non havere effe patito, ne eflere
12

 
 

 

MALMANTILE?

in grado di feccarfi'. Edialle volte s' intende uno di-mala’fanita quando fi'dices
verde come. un’ aglio, mas’ intende non Ja frefcheza,che denota il verde delitaglioy «
mail colore, che efiendo verde neila faccia dell’ huomo denota pocayfanitas 10
LEV.A i moceoli per farmi dire il canto in fearpe,e xoceoli,Compra la cerayper far-y
mi i] funerale:: che moccoleyuol:dire ogni piccola candeladi-céra ye quit prefol
per ogni forte dicandele di cera’. B quel farmé dire il canto fearpe Zoceoli & detto.,

gioco(o ufato fra 1 noftriContadini ; 1] cual.detto non é forfe fenza fondamento
ne_affatro: improprio, che pofia haver origine dalla diligenza, che fi pone nel fae,
che i morti quando {on portatialla {epoliura habbiano y fe fono huemini un parr
di {carpe nuove,e fe fon donne un par di pianelie,o zaccoli puovi ; eRveco/e\e-unay
{carpa col fondo di legno,che ferue pen difendere i piedi dall'acqua,che¢perterra.
Ss 0 s hovel

TANZA: XXVIII: TAIN ZiA>-X SAX. >
Poi che quefte parole hebbe finite , é Eth prima giorno fece tama via y\ 9.09 0h
Dal {uo caro eAmadigi fi licenza y Chi fuoi Lacché fpedati , e conci male
A qual rimafe tutto,sbigortito , Sirimafero 5 Punoall’ offeria ,
Pero che gli dolea la {ua partenza, c Ent altro fearmanato allo fpedale ;
Quand’ in feha Florian di gid falito 9», Ona’ ¢i pit non havende compagnias’.
Senza gran doble, o lester di credenza Se bene accanto havea fpadase pugnale;
Andonne abenefizio di natura i ‘Per non baver paurain andarfoloy
Con dug ferni cercande faa ventura, , » Cantava ch’ ci pareva unrofignnelo ,
exgh onssiber wid AD CAL MMe obyteD oir a \ ‘
Cost muove canzoni ogn’ bor cantando Onde ai timori al fin dato di bando .
Con una voce tremolante in quilio, on Tirava innanzi il voiontarwefilio 5,
E quaiche trillettin.di quando in quado E ginntova Campi, li fermar fi volle
Alle fielie n' andava ye in vifibilio : A bere , ¢ far la Rolfe per bi malle, -»

Floriano fi parte dal fratello Amadigi , il,quale ne rimafe aiflitto . Lalcio per,
la flrada i Lacché ftracchi,ed egli folo fi condufle a Campi, dove fi fermé a bere.
SSIGOTTITO , Afflino ; perduro d’animo . I Latini diflero «daimo deietius,
Quand’ uno fia allegramente diciamo : Il tale fa in.gote, 5.0 fha in barba di micio.,
Vedi in quefto C, ftan. 48. Si che uno che non ftia allegramente fi dice vom /ta im,
gore, non fia in barba di micio ;.E perd non farebbe gran fatto , che quefta voces
shigettito venifie dallo Spagnuolo bjeorses 5 che yuol dir bafette y,¢ che per-ia lette-,
ra, $,che aggiunta al principio d’ uha parola ha forza di privazione (come,
habbiamo detto poco fopra ) fignificatie fenza bigorres, che vuol dir (enza balette ,
cio¢ non in barba,, non allegrameate: 0.forfe sbigottico ,quafi sbattuto...
(od BENEFIZ10 di natura. A cafo ; dove la Fortuna lo.guidava... »
LACCHE’, Servitori , che corrono.a pié:; ¢ per lo) pil fonowwagazzi
yanetti. Vedi forto.C. 11. ftan. 9. 51k HObHat
SPEDATT., In quefto cafo non vuol dir Senza,

" eftanchi dal viaggio. Be it by AV Svan: sets ah
Scakddanart tna pecie d'infermitd, che viene. c:
‘caldau per violente fatica , 0 viaggio

   
  

 

 
 

che dope efferfi foverchiamente Hi orate
freddano_ 0 col bere .0,conJo (tare al vento , 0 in luogii frc(cht;¢ fi dice +) és
gliar una fearmana , 0 fearmanare.«.E forfe {pecie di quel maics che i medici chia~
mano Pleusitide,edé comunemente chiamato wal di petto.Qui sotcndh, ABBEY

 

 
 

SECONDO CANTARE: 89

dal viaggio , in maniera che I’ anelito fe li rendea difficile , ¢ perd ‘non poteva-
no camminar pil .

CANT AVA che pareva un Rofignuolo. 1) Rofignuolo , Vecelletto noto , da i
Latini detto phifomela , ha il pit bello , ¢ gagliardo cantare di qualfivoglia Vccel-
letto , e per quefto quand’ uno canta bene, lo paragoniamo al Rufignuolo .

VOCE tremolante. Voce , che tremava per cagione della paura ; Si come’i tril-
Ki eran fatti per timore , e fi potevano dire pil tofto tremoli , o interrompimen-
ti di canto cagionati dalla paura’, che veramente 77id che fono un riperquoti-
mento di voce muficale nel medefimo tuono. Horazio diffe : Cantu tremulo.

LN quilio,.. Secondo che mi diffe il Signor Nigetti , fra i mufici del noftro feco-
Jo il Maeftro ; la voce quilio fignifica un cantare in voce non {ua , come fe uno
havefle voce di bafio , e cantafle di foprano ; Si che s’ intende, che Floriano can-
tava per Ja paura in voce falfa, enon fua naturale, che i Latini fecondo Cic. lib.
3.de Orat. la dicevano Vocula fal/a.E Titinio appreflo Fefto ditle Succrotilia vocula,

ANDAR alle feelle col canto, Cantar in tuono alto . Se ben qui par che voglia
dire , fen’.andaya in gloria , cioé cantava con gran foddisfazione , ¢ gufto ; poi
che foggiugne én vifibiie che appreflo di molti de’ noftri vuol dire Andarfene in,
eftafi, ¢ perderei fentimenti per il gran guflo, Matteo Franzefi nel Cap, del {uo
viaggio da Roma a Spoleti dice .

Vedea pafsar con toruo fupercilio
Qualche Sarrapo tronfio, ed appoggiato
il tappeto , n° andava in vifibilio.

Vergilio Egl..5. ditle: Voces ad Sydera iattare ,

Ed ottavo Ma. Effundere voces ad athera ,

TLR AVA innanzi il volontario efilio , Continovava il viaggio , che egli medefi-
mo s’ era eletto,cfiliandofi dalla propria cafa .

BAR la zolfa : Detto fcherzolo , che fignifi a Cantare , far mufica, ed ¢ com-
pofto di tre note muficali, la, fol , fa. Ll Signor Salvador Rofa in una fua bella
Satira parlando della mufica dice ,

Quanta gira la terra a tondo a tondo ,
Lago alcuno non v' ¢ che di fchiamarzi
.) «Edi xolfe non fia pieno , ¢ fecondo,

PERS mole. Ib molle ¢ chiave muficale , o fegnatura di femituono ; Mas
qui dicendo far /a xolfa per b molle, fi ferue della voce mulle per incendere: am-
mollare la bocca , cioé bere , E cosi {cherzando fopra alla mufica , ed havendo
detto , che Floriano cantava ; foggiugne,, che voicva feguitare a cantare anche
nell’ ofteria, ma per b molle , ed intende Vuol bere . ;

STANZA XXXL STANZA XXKIL
A Campi , hora fpiantato alla radice Com’ io diffi , Florian nella Cittade
Dominava in ques i Storditano, Entré per rinfrefcarfi,e tocear bomb,
Se ben Turpine ferive’ , ed altri dice, — Mail gra fraftuono,cb in quelle cotrade
Ch’ ei regnaffe in we luogo piss lontanos —— —- D'arini,di beftie,e d'haomini rumbomba;
Hebbe una figlia detta Doralice, Al fentir fu pei canti delie rade
C'bavea un'occhioc'uccides il Criftiano, Tute a cavalio rifuonar la tromba 5

Ma quel che pin tirava la brigasa «Ed il voler faperne la cagione ,
El cffer fola ye ricca sfondolata, M_ = Lo fecero mutar a’ opinione .

  

 
p

as

he

i) MALMANTILE

1) Poeta finge Citta Regia il Caftello di Campi , luogo vicino'a Firenze y ches
hoggi ha poca forma di Caftello , per efler diftrutto , ¢ dice che gia vi regnava
Stordilano , che hebbe una bellidima Figliuola nominata Doralice , 1a quale per
etier fola , ¢ ricchiffima , era da moiti bramata in moglie . E perch¢ quefta non
fia creduta la ftefla , che quella che l' Ariofto fa Figliuola di Stordilano Re di
Granata dice : Se ben Turpino ferive , ed altri ( cioé  Ariofto ) dive , ch’ ei regna/-
Se in un Inogo pitt lontano , coe in Granata . \

Floriano dunqgue , il quale era entrato in Campi folamente per pigliare un po-
co di ripofo , e rinfrefcarfi , e andarfene , fentendo tanui ftrepiti d’ armi, € ro-
mori di tamburi , fi rifolue di trattenerfi alquanto per intenderne la.cagione .

HiVEA un octbio c' ucvidea il Criftiano, Havea cosi begli occhi ,che facevano
innamorare ognuno . Quefto detto vien forte dalla comune opinione di quel fer-
pente da i latini detto Regulus , ¢ da i Greci, ¢ da noi chiamato Bafilifco , 11 qua-
Je col folo fguardo avvelena , ed ammazza coloro, che egli mira. E moiti Poeti
noftrali per ledare I’ occhio di bella donna hanno detto : Occhio di Bafilifco , in-
tendendo , che han forza di metcer nel cuore il veleno d'amore. Apul. morficans
tibus oculis ,

TIRAVA la brizgata , Lufingava , incitava , allettava il popolo a defiderarla -

RICCEA sfondolata , Ricca fenza fondo: Ricchiflima . Diciamo Ricco in fon-
do, fenza fondo , sfondato , 0 sfondolato , per denotare una ricchezza. fenza nume-
ro,omifura.

RINFRESC ARS , Ciok reficiarfi col ripofo , e col cibo . I Latini pure dice-
vano tal volta-rinfre/car/i per ristorarfi,trovandofi refrigeratus in vece di refociliarus,

TOCC AR bomba, Arrivare in un luogo e dimorarvi poco. Quefto detto &
tolto da un giuoco fanciulle(co detto birri e ladri , il quale fanno in quefta manic-
ra. S'unifcono molti Fanciulli, etirate le forti a chi di loro debba efier birro,
chi ladro , quelli che ono eletti birri fi mettono in mezzo della ftanza , o piazza
dove s' ha da fare il giuoco , ¢ ciafcuno de i ladri piglia il fuo pofto , il quale &
gia ftato confegnato per immune ; ¢ quefto luogo da effi é chiamato bomba , che i
latini dicevano mera in quefto medefimo giuoco ufato ancora da i loro ragazzi , ¢
da quelli de i Greci, fe beac in qualcofa differentemente . Quefti ladri vanno
{correndo da ua luogo all’ altro , e i birri procurano di pigliargli , ed i ladri ,
quando fi veggono ftracchi , corrono a trovare un di quei Juoghi immuni detto
bomba , dove ttando , fono franchi , ed i birri non poflono pigliargli , e fi guada-

gna , 0 fi perde il premio ftabilito,fecondo che fon convenuti d’ efier prefi,onon

prefi in tante gite ; ed il ladro prefo ( continovandofi il givoco ) diventa birro ,
ed Ml birro , che ha prefo diventa ladro . E perché nel toccar bomba fi trattengo-
no 3 pero diciamo toccar bomba per ¢{primere arrivare in un luogo , ¢ par-
tirfene prefto. E quefta voce bomba vien dal Greco bombeo,che vuol dire Strepita-
ore , 0 far fuono , ( donde rimbombare ) é da quel romore,che fanno i ragazzi con
Ja voce , ¢ con Je mani per far conolcere che toceano i} luogo immune , quefto
Juogo é¢ chiamato bomba . Diciamo tornare a bomba che fignifica ternare al primo
difcorfo-, Vedi forto C. 8, ftan. 15. ws i

FRASTVONO . Fracaflo,Strepito,romore confufo , quafi dica fuor di tuono.

CANTO . Ciot l’ angolo che fanno le cafe.a capo a una ftrada che eae

Q — van"al-

 

 

 
SECONDO CANTARE: gt

ian’ altra ; detto cosi fecondo alcuni , dal Greco Canthos , che vuol dire Angolo
dell occhio , o dal canto , che nello sboccar delle ftrade in {u le cantonate folcva

farfi dagli antichi , come fi cava da V

. Egl. 3.

Won tu in trevijs indotte folebas
Stridenti miferum ftipula difperdere carmen ?

Ma é detto dai Greco camptin , che vuol dire Piegare .

TVTT 14 cavalo, Cosi chiamano i Soldati quelia {uonata di tromba,che fa in-
tendere a i medefimi il montar’.a cavallo, la quale par che efprima ; Tati a ca-
valle. Coftume tolto da i Latini , che per fignificare il fuono della tromba dice-
vano fecondo Servio , ed Ennio Taratantara .

At tuba terribili fonitu taratamara dixit .

STANZA XXXIIL

Era gta feavaicato ad una Oftefa ,
Per far , fi com’ ei fece , un conticino,
Ne altro bebe che pane ,e capra leffa,
Che fitra anche gii fu per mannerino
Bevve al pore una nuova manomelja ,
Perch’ il vinaiohavea finito il vino ;
Fece conto , ¢ pag ben volentieri
Poi chiefe il fin ditanti Strombettieri,

s

TANZA

Ma c’ occorre ch’ in cio pitt mi diftenda ,
Mentre 1a cofa é tanto dinulgata ?
‘Pero lafciami andar ,ch' ioho faccenda
Havendo fopra un’ altra tavolata

STANZA XXXIV.

Ella rifpofe: E come ; E non lo fai?

Se per Campi non é altro difcorfo ,

Che havendoil Re una figliaye’ hoggi mai

eAbbraccerebbe un' hué prima Cun or fo;

E percht reda ell’ ¢ bell’ , ¢ d' affai ,

Di pretendenti bavendoungrancocorfo y

Bandire ha fatto, acid nefun fi lagni,

Chin gioftra chi la vuol fela guadagni,

XXXV.

Dicé Florian che ai fuoi negoxzi atteda,

Scufandofi d’ haverla feieperara

E rimeffa la brigha al fuo giannetto,

Come un pardo faltovvi /u di netto ,

Floriano eflendo {cavaleato a un’ ofteria , dopo che hebbe mangiato , e pa-
gato intefe dalla padrona dell’ ofteria , che quei romori di trombe fi facevano
perché il Re voleva maritare la Figliuola a quel Cavaliere, che meglio fi portafle
1p gioftra ; onde Floriano monté fubito a cayallo per andare a veder quetta fefta,

F ARE un conticino, Cosi ufiamo dire per farfi intendere copertamente Andar a
mangiare all’ ofteria .

FITTO gli fu, Gli fa fatto credere . Gli fu dato ad intendere che ¢’ fufles $
Mannerino., Il verbo ficeare ufato in quefti termini ferve per efprimere , che :
quellatalcofa fudata per maggior prezzo di quel che ella valeva,o per di miglior eS

+ ualita, che ella non era . Vien da ficcar carote,che vedremo forto quefto Cant,
{ jan. 70, € Cant. 6, flan. 68. Lat. imponere alicui .
i MAN. a ie d’ agnelli caftrati , che nel!a noftra Tofcana é ottima
nel Territorio., econtado di Piftoia , ed ¢-carne (quifica al contrario della ca-
pra, chee ja pepgione ; che fi mangi , ed.in particolare cotta a leffo.

MANOMESS A, Quando all' Ofte arriva portatogli dalla montagna il vino
primo cavato dalla ,botte fi dice: 2 offe ba bausto la manomefa, Ed i Fiorentini, ‘ :

othe fon di buon gufto,o pil tofto ghiotti nel bere , lo pighano pitt volenticri ,

quando é vino di » non tanto per la curiofita di guftare quel nuovo

vino , quanto perché non piacendo loro le fondate,hanno caro di bere del primo,

che efce della botte , onde pare che il = voglia intendere , che Fpeene fes
14 2 ene

“a4

 
 

or MALMANTILE

bene bevve acqua hebbe nondimeno gufto,, perché era nuova manomeffa,, maia
effecto gli da la burla dicendofi che bevve una manomefa nuova ciot infolita, nons
efiendo folito , ne coftume, che fi manometta il pozzo,fe non per le:beltie .
VIN AIO , Ciok colui che nell’ ofterie da il viao . Per maggior intelligenza di
quefto é neceflario fapere’, che nell’Ofteric di Firenze ftanno due maeltri , ¢ ten-
gono garzoni differenziati ; Vno di quelti macftri ¢ il padrone principale ed in
Jui dice P Ofteria , ¢ quefto fi chiama il Vinaio ; altro ¢ macltro anch’ egli, ma
folamente della Cucina , della quale paga un tanto il mefc di pigione al Vinaio ,
dal quale pud etier mandato via . Ho voluto dir quefto , perch¢ {0 che a i Fore-
flieri é di non poca confufione quefta diftinzione , perch¢ fi fanno’far il conto da
uno , ¢ peafanao d’ haver finito ; gli fopraggiugne poi il fecondo Olte, che fa lo~
ro il conto della Cucina, e crefce la fomma del primo conto fatto dal Vinaio .
FECE conto . Domandd quanto dovea pagare.,. Trattandofi d’ ofterie Far con
to stintende Haver finito di mangiare .
ST ROMBETT /ER/ , Incende il romore , che fa il fuono delle trombe .
ABBRACCEREL BE un huom prima c' un' orfo, Cosi diciamo d’ una Fanciulla ,
che fia in eta da.maritarfi, ¢ che fia bella , grande » ¢ ben formata , intendendo
che fia in eta da bramar ? huomo., ¢ da diftinguerlo.da un’ orfo 5 0 da ‘non fug-
girlo , come farebbe all’ orfo . Virg, am maruraviro , plenis © nubilisvannis ,
*" D' -dSS-AL, Valente,contrario di Dappoco : pare che fuonio ftetio chein la-
tino preffans.. f
REDA, Vedi fopra in quefto Canto ftan, 12. Quié prefo nel fuo Poe fi-
gnificato d’ herede,o fuccetiore nelle faculta ; ¢ vuol dire che effendo ella Figliuo-
Ja unica del Re , dovea hereditare tutto quello:che ¢gli pofledeva., vise
TeMVOLAT £ , Cosi chiamano li noftri Ofti tutti coloro , ehe 'vanno a ‘man-
giare alle tavole delle loro ofterie , canto fe fuffe un folo per tavola , quanto
{e fuflero pit , pur che feggano a mangiare a tavola.. :.
SCIOPERAT A. Levata dal lavoro , o dall’opera.. Vedi fopra C, 1. ft. 29,
GIANNETTO., ‘lntende cavallo, Sendoi giannetti {pecie di cavalli»,che ven-
gono di Spagna del paefe d’ Afturia., e perciO dai Lavin detti4fPurcones ) -
‘P-ARDO , I Gatto pardo¢ animal noto, come ¢ anche nota la di tui feroces
agilita, e deftrezza ; e pero-appreflo di noié in ufo quefta ‘cemparazione quando
vogliamo intender I’ agilita di vita d*alcuno3 Vedi fopra C. 1, ftan..11, Le /oale
corre lefto come un gatto.. vita ol

STANZA XXXVL STANZA XXXVIL
Tocca di/proni,e vanne,e giunge in pi Floriano in comemplar facciass) bells
‘Dovietlibiatatese che riafer lagiofira, > Dave quel evade, bale de

Che per vedere il ws AY Rib y
B apeen | Cova factanie asia.”
Sedevail Re prefentela Ragaca ,
Che quanto adorna,e bella fi dimoftra,
Tanto é confufabavido ahaver coforte,
Won afuo mo ma qual verra ta forte.,

 
Le ee eee

SECONDO CANTARE.

i é STANZA XXXVIIL ,
Po far'\(dicea) che bella creatura | Capperipud ben dir d baver ventura
nell’ Offeffa da vero havea ragione , Quelloia cui tocca cos) buon boccone ;
Perch’ ella ¢ bella fuor d'ognimifura Ma's ellas' ha da vincer con la lancia,
Per me non faprei darle eccerione . Hoggie quadoci arrifchio ach'o la picia
Floriano giunto in Jenne veduta Doralice cosi bella fe ne inuaghilce , ¢ rifol-
ue pero di tentare la fortuna, ¢ cimentare la {ua perfona per avventurare i) con-
feguirla per moglie .
LL Popol vis ammazza, V'é tanto popolo per veder quella gioftra , che s’ ani-
mazzano l'un J’altro per la ftrettezza . Hiperbole ufatitlim’ in queilo propofito
per efprimere la gran calca , 0 quantita di popolo . i
F.ANNO la moftra,, Quando i Cavalieri , 0 foldati , o altre genti 5 che devono
fare qualche operazione guerricra (ancor che finta ) avantidi.cominciare a ops-
rare compari(cono in ordinanza quefto fi dice far /a.mofrra,
LA Ragazza . Intende Doralice figliuola del Re.
2A. SVO mo. Secondo il {uo guito. Quel me vuol dir modo, ufandofi da noi,
come da ii Latini, ¢ da iGreci la figura Apocope, che leva I’ nltime fillabe alle»
role , ¢ da noi alle feguentiparticolarmente; Afodo., meglio fede y vagliv 5 vedi,
Saendaate » piede ,ec. Che diciamo + mo, me, fe, vo’. ve ,fra , fan, pit. Howo-
Juto-notar quefte, perché fpeflo nel noftro parlare ci vagliamo di quelta figura ,’¢
fitrovera ancora {peflo ufata nella prefente Opera , come habbiamo.accennato
ancora fopra C. 1. ftan. 10,
TIRA frecciate.come la rovella, Tira dardi,e frecce in quantita. Di quefto
termine come la rovella , come la rabbia ,.come il canchero , ci {eruiamo per.c(prime-
re quantita grande 5.0 vero operazione wiolenta infuperlativo grado ; come per
efempio Mtale.corre fortiffimo , il tale perquote gagliar ate diremmo.// tale.corre
la rovella , rabbia 0 canchero , 0 perquore come., ec, E fi.deduce la. comparazio-
Rian violenza., con la:quale opera il male della rabbia, o del.canchero.. Las
evoce fovela:, O.rovello s \credoiinuentata dalle donnicciuole per.non profferire la
arola ‘rabbia ,.come'fi dicecappira in vece.dicanchero, EB fe bene hanno del fur-
ten » fon tuttavia, molto :wlate:, ¢.]’ usd i] Malateftiin,alcune. fue ottave,
: Da poi ch’.io,a feruito per rimbelloy
i) \Befonovandato.trenta mefi aioni
| Gridando per larabbia., ¢ pel rovello x 4
8 Come fail Gatto quand’ hai pediononi ec, ‘eis
Ed habbiamo il verbo. Mare, e}'.addietti Jato. ‘In formma’in.que~
__» flo luogo dicendo Tira frecciate come, /a,rovedaiintende ,.che |Doralice.con le {ues
___ gcan bellezze faceva ianamorare ognuno , che la vedeva ,
LE Grazie, | Poeti-fingono, che le grazie-fieno tre figlie di Giove nominates
Ve Aglaia, Eufrofine,, ¢ Thalia. Ag/aos\in Greco val per {plendido, Eufrofine, ila-

93

ne

   

 

 

rita , allegrezza, ¢ Thalia, verdeggiante . Si che dicendo 7 feorge in quel.volto les wn
_ grave vien’.a dire: Si {ce:in lei fplendidezza., alleg: » ¢ fre(chezza, cioe
gioventi fana . <

RACGOLTO in uno. Vnito in un folo'luogo, Termine latino , ulato.alle.vol-
ste anche da noi in quefto propofito , 3

    
 

94 MALMANTILE

LE trombe. Nella pitt ftimata carta de’ Ganellini , o Minchiate é effigiatala3
Fama con due trombe alla bocca , e da quefta tal carta fi chiama le Trombe; &
per efler quefta la fuperiore a tutte I’ altre carte quando fi dice :\ La tal cofart les
rrombe s' intende , che quefta tal cofa fia la meglio, che fi trovi nel fuo genere
Ed é detto afiai ufato per efprimere I’ eccellenza d’ una cofa , ed ha Ja forza del
fuperlativo. ?

NON plus ultra, E, noto il motto delle colonne d’ Hercole , che vuol dire;
Won fi vadia pit avant: ; E noi ce ne feruiamo nelle congiuncure fimili alla pre-
fente , che s’ intende; non fi pud andar pil la, cio¢ non fi pud avanzare,o fupe-
rare tal bellezza , o vero non fi pud far pid bella. Efprime anche quefto termine
un fuperlativo ,

PVO! fare, E’ termine d’ ammirazione,o flupore quafi diciamo: Pud mai fare
il Cielo , o Ja natura una cofa tanto bella , ¢ perfetta come quetta ? L :

CAPPER/? Ancor guefto ¢ termine d’ ammirazione ; ¢ fi dice ancora cappita,
canchita , canchigna for(e per.non dir canchero : Voci inuentate dalle donne;come
habbiamo accennato poco fopra alla voce reve/la , Confuona col latino Pape, che
noi diciamo 4 ! ¢ col latino babe , che noi diciamo , 0 babbo | E la parola capperi,
che tanto in Greco , che in Latino vuol dire il capper frutto noto, ferviva anche
a’ medefimi per termine d’ ammirazione, o giuratorio, come fi vede in Laerzio
nella vita di Zenone. Sed, @ per capparim iurabat , ficut Socrates per canem,ec. LO
fteflo riferifce Alex, ab Alex. dier. gen. lib, 5. cap. 10. I Lalli nella faa En, wrau.
C, 1, flan, 85. 2

Capper diffe Enea, come si tofto
Fatt’ ha si gran Citta quefta Signora \ :
ef CHI toca cosi buon boccone . Chi haura cosi buona forte. Chi haura per mo-
glie cosi bella , e ricca Giovane.
Cl arri{chio anch’ io la pancia, Ci avventuro anch’ io la vita.
TANZA XXXUX

O per tute’ hoggi beccomi fue moglie Cid detto falta incampo,e un’ aftatoglie,
Nobile , ricca ,¢ bella ; 0 veramente Intruppandofi ld dov’ ei gid fente ,
Vi lafeto Loffa ; s* ella cogkie , coglie C’ appunto il ReYolleciea , ¢ commette,
Se x0 a patires O Cefare , 0 niente. Che pe’ é prims fi tirin le brufchette.

_ Rifoluto Fioriano di provarfi in quefta gioftra fi fa innanzi’, ¢ piglia una lan.
cia . Qui bifogna fupporre , che Fioriano , ¢ gli altri Cavalieri futicro armati di
doffo , come € necetiario , che fieno i Cavalieri , che gioftrano a corpo a corpo.

BECCOMI fu moglic, Quefto-verbo beccare ha figniticato di rubare , guada-
“— © acquiftare , Gio, della Cafa nel Capitolo in lode del martello d’ amore
ice : : : dase,

So che fapete del ladro fottile , 4 oa
C’ 4 Giove fe la harba gid ai ftoppay ie
: Quando glibecco fut efca , eilfucile . ae Si

E perd afato per Jo pi {cherzando in occafione di maritaggi , come appunto

nel prefente luogo , EB fi dice M tale piclio moglie , ¢ becca fu una buona dore., Elo

aan nace dal verbo beccare’, che € novo quel che fignitichi trattandoli d’ am-
mogliati. a : t
+ SE

 

SMES

 
 

 

SECONDO CANTARE. 95

©) S* ELLA coglie , coglie .S' io m’ appongo,fara bene. S' io vincerd I’ haurd indo.
winata , ¢ {ard felice, Se no 4 patire, Se non m’ ayponee » fara difgrazia, haurd
pazienza . In fomma con que i due detti yuo! moftrare,che Floriano ha I'animo
accomodato a tutto quel che fia per fuccedere, o male , o bene che fia.

O Cefare ,oniente, Aut Cafar , aut Nihil, O morire y 0 effer qualcofa di gar-
bo . Quetta fentenza latina fi profferi(ce da noi corrottamente , O Ceferi,o Nic-
colo , ed efprime Aut Rex , ant afinus de i Greci , cioé uno de due eftremi .

STL tirin le bufcherte . Si tirino le forti. Credo che fi chiamino brufcherte ,¢ non
bufchette , 0 forfe in ambedue i modi ; che ¢ un giuoco da Fanciulli , e fi fa con,
pigliare tante fila di paglia , o altra materia fimile , quanti {ono coloto, che han-
no. a concorrere al premio propofto , ¢ quel filo , che tira il premio , fi fa o pilt
lungo , o pit corto de gli altri; detti fili s’ accomodano fra due afi , 0 in mano
in modo , che non Gi veda fe non una delle due teftate di effi, per le quali teflace

ciafcuno de’ Ragazzi cava fuori il fuo, ¢ quello che tira il pid lungo , o il pil
corto , fc do che é defti » confegui(ce il premio propofto ; Quefto giuoco
ferue ancora ai Ragazzi per fare le divifioni ne i loro giuochi Panciulle(chi,come
ofarebbe ne i Birri,e Ladri detto fopra in queflo C. flan. 32. aila voce Bomba, che
allora pigliano tant fili , quanti fono i Ragazzi , la meta Junghi , ¢ la meta cor-
ti, e:cavandoli da loro a uno per volta detti fili ; quelli, che hanno i lunghi,van-
no da una banda , ¢ quelli de’ corti dall’ altra ; ¢ cosi ferue a loro, come ferue nel
prefente lnogo , per un modo di tirar Je forti . E da quefti brufcoli , o fili di pa-
ia mi do a credere , che fi dica brafcherte ; e che bu/cherte fia quel giuoco , che fi
con certi pezzetti di mazza rifefla, ¢ che fitirano , come 1 dadi , con altro
nome dette /e buffe. Vedi forto C, 11. ftan, 42.
STANZA XXXXI, STANZA XXXKIL
Come volontarofo Floriano , Piglian del campo,e al cenno del trombetta
Senza cbieder licenza, 0 cofa alcuna, Sivannoincontro con la lancia in refta; £
Si fece innanzi, ¢ poftavi la mano Ii Marchefe a Florian t' havea diretta ;
j Di trarne (a pin langa hebbe fortuna, Per chiapparto nel merxo della tefta ;
Poo dopo il Adarchefe. di Soffiano Ma quei,ch’t furbo, aun tepofacivetta,
a ... Simile a quella anch’ egli ne traffe una E aggiufta lui,disendo: Afjaggiaquefia,
Ond’ effi , come priaifn deftinato ,

 

male abbattuto. AZarche/e di Soffano, E nome a cafo , ¢ fa Marchefato una con- , :
trada,o villa vicina a Firenze detta Soffiano . ‘f,

CHIAPP ARE, Val per colpire.

FVRBO , Se ben ja voce furbo deriva dal latino Fur , che vuol dir Ladro , tut-
tavia ce ne feruiamo per efprimere un’ huomo {cellerato , ¢ che habbia ogni for-
_ tadivizio , come s’ é.detto fopra in-quefto C. ftan. 2. Ed ancora per denotare un’

huomo aftuto , ¢ che fappia il conto fuo , come fegue nel prefente luogo »
ee FACIVETT A. Abbaiia la tella .. Viene dal giuoco di eivetta , che da i gid-
vanotti fi fa in quefta maniera . S’ accordano tre ; ¢d uno di loro, al quale @ toc-
cato in forte , fi pone in mezzo a gli altri duc , i quali s' ingegnauo di a i}
‘ erEsh

Perché gli diede fi {pierata borta ‘
. Furono i primi a correr lo freccato , Ch’ egli ando exit come una pera cotta ,
2 ¢ quefti due furono i primi a correre la lancia , n¢! qual’ inconcro il Marchefe ri-
ida
york

Floriano pre(e una di dette Brufchette,ed una ne prefe i! Marchele di Soffiano;
j
}

by.

 

  
 

 

96 MALMANTILE

berrettino di tefta con le percoffe della mano ; € ——_ egli tocca terra con le
mani , aon puo effer percoflo ; e perd hora alzandofi , hora abbaflandofi ; tiras
guando all’ uno , e quand’ all’ altro di gran moftaccioni ; dura il giuoco:
che da uno delli due gli fia fatta cafcare con un colpo Ia berretta dalla tefta , che
allora perde il premio propofto , ¢ lo vince colui , che gliel’ ha fatta cafeare , il
wale ( feguitandofi il giuoco ) va nel mezzo in luogo del primo. Tal-giuoco fi

rs a tempo di fuono , ¢ piglia il nome dalla Civetta yccello , che per bulcare if
vitto {cherza con gli uccelletti alzando , ed abbatiando la tefta, come appunto fa
colui , che fla nel mezzo.” E da quefto poi far civerra s'intende Abbaflareil capo.
Da Scops,che € un'uccello notturno de! genere delle Civette . Era appreflo i Greci
una forta di giuoco , 0 paflatempo detto Scopias , nel quale veniva contratfatto a
tempo di balio il muoverfi in giro , ¢ I alzare , e¢ I’ abbaflare della tefta di quell’
uccello ; onde ne fu formato il verbo Scoprein irridere , che appreflo i Greci vale’,
quel che appreflo noi Tofcani, Vecellare . V. Giulio Polluce |. 4. cap. 14.

AGGWST A iwi. Aggiultar uno, s’ intende Bargli il fao dovere, ¢ trattare uno
come eglimerita, Lat.coxcinare, Vuol dire ancora conciar male uno,come s’intende
nel prefente luogo, ¢ fotto C. 11.ftan. 50. E per altro vuol dire Saldareso pagare
un debito. Lat. pariare.

BOTT A. Colpo, o percofla. E quefta voce bortayper altro vuol dire una {pe-
cie di Rofpo. Lat. rabera.

ANDO gis com’ una pera corta, Calcd gid facilmente,ed a piombo,come fanno
Ie pere cotte dal Sole , che cafcano facilmente dall’ albero ; o forfe come le
cotte.al fuoco , che fon faciliffime a andar gilt in corpo quando fi mangiano
Plauto diffe : T'am crebri ad terram decidunt ut. pyra’; da che fi deduce che s'intea-
da delle pere , le quali cafcano dail’ albero ,

STANZA XXXKXIL J

tn quanto a Spofa, homai queftoe afcoito; Che mette lui per morto, anzi fepolto ,
S?ei rocco terra, ancor la voglia {pati : Ma il giovane , che da di quei faluti
Cosi Florian dicea ; ne Sette molto Gli moftra in avviarlo per le

» CH il fecondo ne viene alpren battnti , LL error di chi fai conti fenzal Ofte
Comparue il fecondo Cavaliere i! quale fi dava a credere d’ haver gia morto

Floriano ; ma quefto col burtarlo.a terra, gli fece conofcere quanto s’ era ingan-
nato.
£ ASCOLTO. EB licenziato. 1 ragazi, che vanno alle fquole , quando fono
ftati fentiti leggere dal Maeltro fi dicono a/eolrs , ¢ s' intendono licenziati : ¢ cosi
iefto: Cavaliere efiendo paflato per Je mani del Maeftro , che ¢ Floriano , fi pud
ire a/colto ,¢ licenziato dalla Spofa . . e
TOCCAR terra , ¢ [putar la veglia , Dicono le donne ,che quando fon pregne,

» venendo loro voglia di qualche cofa , fe in quello ante G toccano con le proprie

mani in alcuna parte di » quivi nafca alla creatura un fegno fimile a quel-
Ja tal cofa defiderata ; ¢ 1 fegni poi chiamano vogiic ; ¢ che per sfuggire che
Ja creatura non nafca con tali fegni 5 0 voglies il rimedio fia, che la Donna pre-—
goa , quando le viene tal defiderio , tocchi fubito terra con la mano, e4puti di-
cendo :. 4 terra vadia . B perd il Poeta,feguitando quefta opinione, dice, che fe>
il Marchele ha toccato terra per liberarfi dalla vogiia della Dama, ¢ neceflario
anco-

  

 
 

SECONDO CIANTARE. ‘97
ancora ‘chelegli fputi , a voler che il rimedio fia fatto’ compitamente, |'Tal detto
| fputar la voglia , & afiai vulgato per intender’ unosche habbia gran defiderio d'una

-bilcofa ; jche-fia-a Iii impofiibile.a-confeguire » Vedi Plin, lib. 28.c. a

\) (A SPROW baceati,, Acvatta carriera; Velocemente! Fran. Sacc, Novella ‘mibi
3%. E cost falito a cavatlen', ando'd {prom battnti al Palace de! Signori .

) LO. meite per marto , anzi fepolto . Intende ; che queftd {econdo \Cavaliero non
folo credeva di havere auccidere Floriano ; ma git pareva'gia d’ haverlo uccifo..
Efprime la gran prefunzione , che havea di {¢ fteflo muri Cavaliero. , ¢ la poca
ftima , che faceva di Fioriano .

D1 quei faluti , Invende di quelle percoffe ,

PAR il conto fenza Ofte . Stabilire per fattayna cofa, alla quale deve inter-
uenire , ¢ concorrere anche Ja volonta d’ un’ altro . Doveé T intereffe del com-
paguo, fi pud metter ig ficura la propria yolonta , ma non: quella del compagno.

STAN ZA XXXKUL

—

Comparfo il terzo , in te/a della lixza All’ altto manta il fettimo indirizza ;
S? affronta feco , ¢ paffalo fuor fupria Liortavose il none appre/so inueffe,e fora;

. Soggiunge ilguarteed coli tel infirza . E cotna tutti con fuo vant, e fama
Shudella il quinto, efreddailfeftoacora Cave di teftail ruzxo-della Dama.

In quefta otta va It Autorenactala vittoria’, che hebbe Floriano di fette’Ca-
valient ye defcrive la Jor, perdita’in fette modi di diré diverfi; il primo'lo pafsa fuor
fuora,il fecondo, <a(fi dourebbe dire infilzazma non folo perché gli f rmel-
fa oot tea per. ¢aulaidella.rima,quanto anche petché per i pit fi dice infizza,

. enon #85¢\facto lecito dirlo anch’ egli) ib terz0 do pafsa fuor fuori , i) quar-
toua fredda's il quinto Jindirizzaall' altro mondo; il {eto Pinnefte , edvil. {ettimo
dofora, EB 2 quell fette _ i dire havendo quafi eutti lo -ftetio fignificato d’ am-

TT

 

 

‘ dana I artifizio de) Poeta-in moftrate la fecon-
. ascenien. Jingua jenn
4. Che fi dice anche Nizza . Vuol dir linea ; ma da noi s'intende quel

  
   
  
  
   
  

tayolato ,"o/muro , rafente al quale corrono i Cavalieri le lance al Siracino .
CAVO! ditelea ilrnx20 della Dama, Fete ulcir'di tefta il defiderio della dama .
Eaormenes » che dal.verbo razcare vuol dir Baie, wfata in quefti. termini fi-
Prucio, — sidefiderio , ec, fi che dicendofi. Wi fale ha queffa ruzzo in

esmeld ale ha ono +e » gucfto liumore 5c, I] Laica nov, mihi
8. dices... cosh fates gafhigatura , ee rhe re fener £08:
M07 ey 8 wren ta

wn T: A NZA XXXXIV.
AM Re firal 3 con Plaine Ond’ ogni-altro ne fu mandato fano;
ny, Seefe di icon{aPiglnola | o>) (Bde nelle ze infino a gola

 

  

a fe fae ieinteaiaagee 4 moi | bo. > Bem paftinc, feruite , ¢ ringrasiato
. Gome nel Bando havea date parola; . ~ Rimsfe quivi a goder il Papaty .
he Re fesse da Foca iano alla ee as er mo-
2 nr -Bioriano rimaie quivi a godere que
ta fisop ‘eh Siitins 10

TOCC AR la mano. Elo fteffo in were ‘calo che che diciamo impalmare,
AL depeisoee tat toccamento , chefi iad Sins nama gry
N

3;

 

rc

 
ee
98 MALMANTILE? ,

fi;che il primo atto che fi faccia per lo ftabilimento del contratto del matrimo-
nio, Vedi forto C, 12. ftan. 50. ’

CH ANDAT O fana , Cioé licenziato , ed efclufo . Il verbo: valeo, (che fignifi-
ca Star fano , ¢ ufato da i latini anche per licenziarfi: parentibus vale\dixit , ed il
fimile facciamo noi , come fi vede nel prefente luogo., che diciamo Mandar fani
in vece di licenziargli. Anzi 11 medefimo verbo vaieo é tal volta ufato danoi per
intendere Addio, cioé licenziarfi . 11 Vai in una faa fiortola ( fe ben points)
lo moftra dicendo .

Hore liete ,

Tam vatlete. t
dam valete amati ferculi ; y q
Etnnale, gyoesse,
O fedale, 7
Che maneggi i miei liberculi ; i

U1 noftro Poeta fotto\C. 6, ftan. 18.
Refti la donna , ed er le diffe vale

 

oar:

WELLE dolcexze infinoa gola , Immerlo nei piaceri ,e ne igufti , foto C. 4.

fan. 42. dice effer ne guai agola.

GODERE il Papato . Goder le felicita concedutegli dal Cielo e

STANZA XXKKXV.
Tre di fuonaro a felta le campane',
Ed altrertanti fi band? il lavoro,
Eil Suocero., che meglioera del pane,
Viu' huom difcreto ed ua coppa d'oro,
Faceva con cli Spofi a fealdamane ,
Talhord'a Mona luna,e Guancial d’oro,
E fece a’ Paggi recitare a mente
Rofana ,e la Regina d’ Oriente ,
STANZA XXXXVI,_

LD andar il giorno in piazza ai Buratriniy
Ed agli Zanni furon'le tor gite ;\
Ogni fera facevanfi fepini
Di oe » ¢ di bakar veg lic bandite ;
E chi non eraingambe , nein quattrini
Da trinciarle ,e da fare ite, ¢ venite ,
Dicea novele , o ftavale a afcoltare ,
Faceva al Maxxolino , 0 alle Comare ,

In quefte quattro ottave il Poeta narra le fele , ed allegric , che fi fecero ing
Campi per lo fpofalizio di Doralice con Floriano ; le quali fefte:fa'che non tra~
fcendano eo nds pucrile per continovare a {crivere una novella per i Fanciulli .

meglio che il pane . Era un’ huomo buonitfiaio , un’ huomo che fi

accordava a ogni cola , appunto come é i] pane, che s'accorda , ed unilce con,
tutte le vivande , almeno appreffo a i Fiorentini. In quefto propofito i Greci

. diflero , Columba mitior . f

VAN-A coppa a ora, Vn0 »l quale non fia da apporre alcunrdiferto , omni exce-

ERA un

 

STANZA XXXXVIL
Altri pik la vedevanfi confondere
A quel ginoco chiamato gli Spropofiti,
Che quei ch’ efce di tema nel rifpondere
Connien ch’ il Subio depofiti 5
Aa altri piace pik Capannifcondere ,
Hani? altri vary) humor yvarij propofiti y
Perch? ognuno aun mo none compofto
Pero chi la vuol leffa , ¢ chi arrofto ,
STANZA XXXXVIIL.°
Chi fale Merenducce in ful bavaglo ;
Chi con amico fa a Stacciabburatta
Chival! Altalena , ¢ chi a Beccalaglio ;
Va quello a Predellucce,un s' acculatta;
Per tutti in fomma fempre vi fu taglia
Di fhar lieto cos} in barba di gatta ,
LE tra Floriano , il Re , ela Figlinola
Mai fu che dir n' unt anno una parola ,

 

pe

 
 

 

 

_ $ adunano pil Fanciulli , ed uno,

SECONDO CANTARE: 99

eptiont maior , Credorche fi-dica coppa d’ are , periintendere oro coppellato ; 0 di

ella , cioé raffinato , che Coppelia fi dice quello ftrumento , col quale fi ri-
duce I’ oro alla {ua yera purita, ¢ perfezione ;¢ Coppa vuol dir bicchiere, o altro
valo fimile , donde poi Sortocoppa quella tazza , fopr’ alla quale fi portano i bic-
chieri, dando da bere ye Coppiere quel che porta da bere al Signore.

SC ALDAMANE. Quattro, 0 pili s! accordano , ¢ sates Sac ordinata-
mente le. mani fopra del.com »€\poi vanno cavando per ordine quel~
la mano , che sehen ¢ meets fopra all’ altre sane © con qrelio
modo ; €.confricazionepretendono {caldarfele ; ¢ perd tale operazione ¢ detta,
Scaldamane ; ed ¢.giuoco/Fanciulle(co,che ha la {ua pena’ per chierra cavando la
mano, quando non tocca a lui. at

MONA luna... S' accordano molti Fanciulli,e tirano le forti.a chi di loro hab-
bia a domandar configlio a Mona luna , ¢.quello.a cui tocca vien fegregato dalla
conuerfazione , ¢ ferrato in una flanza ,.accioO.che non poffa intendere chi fia,
quello di Joro , che, refti elerco in Mona luna , deila qual Mona Juna fi fa I ele~
zione fra gli altri , che reftano dopo che coluié ferrato . Bletta:che ¢ Mona lu-
na ; fi mettono, tutti.a federe.in fila , ¢ chiamano colui , che é ferraco , \accid. che
venga a domandar il.configlio a.Mona,luna.,, Quefto tale fe ne viene ,\¢.doman-
da il configlio a uno di quet ragazzi , quale egli crede, che fia {lato eletto in Mo-
na luna, ¢ fe s’ abbatte a trovarlo,ha vinto ;fe nd; quel tale, a cui ha domanda-
to il configlio gli rifponde; lo non fon Mona luna, ma fla pid gil, o pili fu, fe-
condo che.veramente.¢. polto quel tale , che @ Mona luna ; ed il domandante per-
de il premio propofto , ed ¢ di nuovo riferrato nella ftanza per tanto, che dai
Fancuulli fia creata un’ altra Mona luna , alla, quale egli torna a domandar confi.
glio , ¢ cosi {eguita fin a che una volta s'apponga , ed allora vince; e quello.che
= Mona luna perde i) premio,, ¢ vien riferrato nella ftanza,diventando colui, che
deve domandare , ¢ quello che s' appofe ,s’ intruppa fra gli altri ragazai.. 1 do-
mandante richiede fino,a quattro volte i] configlio , ¢ pud perder quattro prevj ,
¢ poi fimefcola fra gli altri ragazzi, efente perd da dover pil efler domandan-
te, fe non nel cafo,,che fatto Mona luna , egli perdeffe , ¢ {empre fitorna as
ercare nuova Mona luna, ¢ fi deputa nuovo domandante, quando il primo s’ap-
ponga » 0 habbia domandato, quattro volte il configlio , 1a qual fuazione , come

detto » non pud effer forzato a fare , fe non quattro volte : edi premj fi adu-
nano, ¢ feditribuifcone poi fra di Joro riparticamente , ¢ dal rendergli poi a di
chi fono , cavano un’ altro paflatempo , come diremo, Da quefto ginoco viene
il proverbio Pit fu fa Afona luna,che fignifica Nella tal cofa é mifterio pil im-
portante di quel che altri fi penfa . 7

Nota che. taato quefto giuoco , quanto ogni.altro , che troveremo nella _pre-

fente Operas’ altera ,.7 ¢ diverfifica {econdo li gufti , ¢ conucazioni pue-
— nili;¢.noa mi Tipe ne haveffi nella tua puerizia Eatti » 0 veduii fare

alcuat , o tutti diverfamente da quello , che io gli defcrivo *
GV ANCLAL @ oro, Quetto eure € giuoco Panciullefco , quale € fatto cosi :
mette.a federe fopra.a una feggiola , ed un’
altro fe li pone inginocchioni avanti, ¢ pola il fuo capo in grembo a quel che>
ficde , il quale gli chinde gli occhi con le mel » accid che non poffa vedere chi
. 2 . ha

Oe ees 6
be

 

 

 

100 MALMANTILE 52 if

fia colui , che lo percoffe in unaimano »che egti fi tiene dietro \fopr’ alle ren; doz
vendolo egli indovinare ;.¢'calaiche gli fertagli o¢chi: , dopo \che queftotale ¢
flato percofio glidice 2 Chi sha percofa? edegli rifponde : Ficefeccho y ef altro
replica: Aderamelo qua per un’ orecchio, Ed allora quello fi'tizza’,¢ va @ pigtiar
colui, che egiterede it perdutiore y¢ (e $” appone , ha vinto y ¢ ponetil percutfo-
re in lnogo fuo;¢ li fa.dare il premio in mano a quello chic .fiede's @ fe hom s' ap-
pone perde ih premio. ; quale. coniegna, ai derto fedente, ¢ ritértia al udgo di 'pri-
ma per continuare ; fin tanto.che s*appone,ed alla quarta vol ‘fi fa huova'clez-
zione ,come fopra a>Monayiona ; ‘Guetto mi par di potcrcredere j\che fia quel
gioco; che i Greci.chigmavano Coldubi/mo siferivo dal Baleng. de lad.veteap.37:
gual giuoco da quel Propheriza : quis te percufit 7 detto per difprezzo da Giada a
Giesti Crifto Sig. noft'o , fi pudvarguinentare , che fafle anco appreflo a-i Latini .
ROSANA ,¢ la Regina @Oriente . Sono duc Leggende,o Rapprefentaziont n0-
Uffime , per effer cantate giornalmente da ogni donniccinola ngeBh ONO?
BVRATT IN1, litehde quei Figurini di: legho,che for fatti: muover da une,che
a cal effetto s’ afcondé in un caftelletto di legna coperto di‘ pannd; € gli fa operas
re.mettendofegli fopra alle punte delle dita,e cd un certo uo fifehio git fa parlare,
ZANNI, Per Zanni, che's’intehde feruo feioceo Lombardo, qui intendé ogni
forta di Bagateellieri , che fanno ilibuffone per le piazze ys) 8 .
FEST INI di ginocojec, Quando's’ adanano‘in wna cafa ae Dame’, e Cavalieri
per giuocare infieme , 0 per ballare nella ‘prima parte della norte, dice fare un
Feftine , 0 Veglia. E te bene veglia ftrettamente prefa, pare che fignifichi pid trat-
teaimento di-ballo , ch¢ di giuoco , tuttavia la pigliamo per intendere ogni forta
di teattenimento , 0 di Giuaco’,' 0! di Ballo yo di qualfivoglia altra cofa’, nellas
wale fi fpendano Ie prime hore della notte , dicendofi: Aoi facemmo la deglia &
dudiare » 4 ballare, a cantare, ec, Ma voleado pigliare quefte due voci nel fuo pro-
prio fignificato ; Feitino , S' intende adunanza di perfone nobili, fia per ballare,
© per giuocare in quelle hore deila notee ; ¢ Feglia ¥ intende d” ogni forta di per-
fone ordinarie ; E fi come s’ avvilirebbe ve : fo fui alla veglia nel Palazzo
del Principe cosi pare, che fi burlerebbe dic*ndO: Fué al fe/Pino im cafd'uh Battilano,
Quando fi dice Feffino'pubblico , 0 Vee liw bardica s intende Feffino , 0 Peglia & por-
ta aperta ,‘dove pud’ andare ognuno ¢ Vedi fotto: G@ 9 flan. 51. ¢ Cant. ro,

ftan, 28. ‘ i &
NON era in gémbe ; ne it quaterin’ | Now fi fentiva'gagliardo da ballare,.¢ non
flani23:

haveva monete da poter giuocare . © ‘

DA trinciarle , Intende da far capriole , cide fattare’. Vedi forto C, 7, flan.

DA fare ite, ¢ venite, Cio givocare. Quando fi'giudea, € rdendo fi paga
la pofta volta per rota 50 rlguot quia la vine dca oO fare ire’,
nite 5 ¢s' intende pagare il fubito pérdata la pofta’; ¢ riceverio'nello’
niodo vincendo; ed ¢ il contrario del detto Parenti ae-gli bai; che figaifica
care in fu la fede , 0 a-credenza , RF 2 OF Die gTo AND Vib AE er

| MAZZOLINO’. Ancor ‘queito'® trateehitienco da Panchull’, ¢ fifa in tal gui-

fa. Pia i adwnano inlieme;e fi piglino'i nome & tn fiore per ciafcuno,
¢ di quefti fori un di loro, che @ i!’ Giatdiniere compone un mazzo’, € poi dice :
Quefto mazzo non fa bene per caufa délla Viola ; € colui, che ha’prefo a

 

 

 

    

 

  

ice

ies

  

 

 
 

 
  
   
 
 
  
  
 
 
  

\

SECONDO CANTARE: ror

delta: Viola deve rifponder fabito: Dalla Viola non viene,ma fi ben dal Giglio, o
altro fiore , che’ a’ lui verra nella mente’; ¢ fe‘non rifponde fubito’, o vero fe no-
mina un'fiore') che non fia in quel mazzo , perde un premio , i) quale fi da at
Giardifiiere ; ‘E cosi vannio feguitando fino a che il Giardinere habbia in mano
tanti. premj da potere alla fine del giuoco diftribuirne almeno uno per ciafcuno di
quei ragazzi , che fono nel giuoco ; ed il Giardiaiere ¢ fottopotto anch’ egii alla
perdita del premio’, perché f€ ua fiore dara !a colpa a Ini , ¢ che egli non rifpon-
da'fubito , ¢ nomini un Fiore , ‘che non fia nel mazzo’; perde come gli altri, ¢ it
fio premio va dato in mano a colui , che I" ha fatto errare ; ma core in depofi-
to’, perché ‘alla’ fine’ del Giuoco va poi con gii altri diftribuito ‘dal Giardiniero
il quale’non Jo pud perd dare a fe medefimo ; E quefti premj ff domandano pegvi,
edi quefti intende i] Pocta dove dice : Convien ch’ if pegno fubiro depofit! .

Finito i] Giuoco i} Giardinicre diftribuifce ripartitamente ¢ pegui pigliandone
anebra per fe.Tali pegni poi fono da coloro,che gli hanno dal Giardiniere havuti,
reftituiti a i proprj padroni, i quali, fe li rivogliono’, devon fare una cofa fecon-
do il gufto'di colui), al quale ¢ toccato in forte il detto pegno ; E quefto dicono
far ta penitenea , Ya quale fe egli non fa , i! pegno refla in mano a colui’, al quale
€ roceato'; ¢ perd quefti pegni devono effer di qualche valore, accié che i padro-
ni habbian caro di riavergii . Alle volte fanno quefto giuoco iGiovanetti di mag-
giore eta ) € riducorio quelti pegni a moneta , quale depofitano ogni volta , ches

‘in’mano a un'dépofirario , ¢ fe ne fernono per far merende, ec, fal giuo-
eo peo diffimile a quello, che facevano i Greci dctto Bafilinda riferito da Giu-
lio 'Polltice tab:'9. '¢. 7..¢ dove noi dicianio Giardini¢re effi dicevano Re, comes
facevano anche i Latini , ¢ cid fi deduce da Hor. Ep. pr. ub, pr.

3 ——— At pweri ludentes, Rex evis', ainnt,
Si rette facies’, bic murns abenens effa , ce.
‘Rofeia , die fodes 5 melior lex , an puerorum

: WNama ? que Regnum rette facientibus ofert.

Se bene potrebbe dirfi,che Orazio tion imtenida di quefto giuoco particolarmen-
te,perché in tutti li givochi Fanciullefchi tanto i Greci , che i Latini chiamavano
Re colui’, che vinceva , ed afino quello che perdeva ; ma perché nel giuoco pre-
fente cra farto Giardiniere (0 diciamiolo Re Wghetto » che in altri giuochi era ri-
mafto fuperiore a tutci , perd sion m’ ajlontano da interpretare Orazio , ed appli-
‘care 3 fuo 1ndgo al prefente propofito , nel quale, fe it Re errava diventava
YP afino , ¢ Re fi faceva colui , che havea fatto errare , 0 tenendofi il conto di
‘chi di loro haveva meno errato , quello alla fine era il Re ,¢ quello che pit volte
haveva errato era I’ Afino’, 0 Re Mida. Vedi il Meurfio de Ludis vererum, Gli
Spartani fimilmente per Legge di Licutgo , econo che riferifte Plutarco’nellas
vita del medefimo’, ai Ragazzi di pil di fer’ anni , ei come Principe
i] pit favio tra loro , che fopranrendetie a’ loro giuochi , ¢ Fanciullefchi efercizzj.
~ “ALLE comare . 6 ginoco é trattenimenco di Fanciullette,e 10 fanno cost:

Mettono una di loro in un letto con un barrboccio fatto di celici , ¢ flngendo;che

“guefta habbia parrorito , le fanno ricever le vifite da-altré Fanciullette con far

quelle cirimonie , ed accompagnature, che f coflumano in occafione di vere par-

tucienti . Nk, : 3
t i tal

 
102 - “MALMANTILE

Tal giuoco era ufatoancora dalle Fanciullette Greche fecondo Giulio Pol.lib.9,¢i7;
ma in-vece d’ una Parturiente fingevano una Spofa ; ¢ lo dicevano Phitramelia .
Qual giuoco fanno pure ancora le nofire Fanciulline, ¢ lo chiamano far’ alle Zie
Non ha quefto giuoco delle Comare , 0 Zie altro fine , che di paflare il giorno in
quelle loro tirimonie , ¢ ricevimenti , ne i quali alle volte fi confuma quello, che
Ie Fanciullette hanno havuto per merendare .

GLI fpropofiti, E Jo fteflo in fultanza , che quello del mazzolino , fe non che
dove in quello fi finge un Giardiniere ; in quefto 1 Ragazzi s'adattano a qualfivo-
glia altra cofa , con pigliarfi quei nomi , che attengono a quella tal cola ;
¢lempio: Faranno il giuoco fopra i! pane;i] Macfiro fara il Fornaio, e gquefto fara
quello che nel Mazzolino fa il Giardiniere ; uno fara la farina, uno I’ acqua, uno
il forno , ed altre cofe attenenti alla conftructura , ¢ perfezione del pane ; IL
Fornaio dira: Quefto pane non é buono per caula della Farina ; quello che has
il nome della Farina , deve rifponder {ubito: Dalla farina non viene , ma dail’
acqua , o da altra cofa che gli venga in mente, atteneote al pane, ¢ che fia frais
loro Ragazzi ; ¢ fe non rifponde prefto,, o non da Ja colpa a qualche cofa , il no=
me della quale non fia in quella adunanza ,o non fia attenente al pane, perde 5 ¢
depofica il pegno; ¢ fifa nel refto per appunto come nel giuoco del Mazzolino :
E quefto giuoco univerfale é forfe quello , che habbiamo detto fopra , che face-
vano i Greci detto Bafiinda , E da noi fi chiama il ginoco de gi S; » perché
dovendo quei Ragazzi rifponder prefto , attribuifcono al pane cofe (propofitanf-
fime ,¢ che non hanno che far punto col pane , 0 fua bonta , oltre a non effer
il nome di quella tal cofa in veruno di quei Ragazzi. £ quello vuol dire VJew di
tema,

Habbiamo un’ altro modo di far queflo giuoco , ed é cosi: Mettonfi pit per-
fone a federe in giro , e ciafcuno dice al compagno in uno orecchio una parola,o
due al pil , ¢ finito il giro, ciafcuno ordinatamente dice force quella parola, che
gli ¢ fata detta dal vicino , ¢ volendone comporre il periodo fi {entono gli Spro-
politi , che rifultano-da quelle parole ; ¢ fida la pena a colui, che ne @ {tata las
cagione .

re 4 nifcondere . Vino fi mette col capo in grembo a un’ altro, che gli tura
gli occhi, ed un’ altro , o pid fi nafcondono , € nafcofti danno cenno, € colui sche
haveva gli occhi ferrati fi rizza , e va cercando di coloro , che fono nalcofti, e»
trovandone uno bafta per liberarfi da tornare in grembo a colui , dove mettes
quello , che ha trovato , ¢ quefto perde il premio propotto , ¢ il trovatore va as
nafconderfi ; ma fe non trova i] nafcofto in tante gite , o in tanto tempo , quan-
to fono convenuti,perde il premio, ¢ ritorna a flar con gli occhi chiuli come pri-
ma ; ¢ feguita cosi fino a quattro volte , perdendo quattro premj , come sé detto
fopra a Aduna luna, ed i premj poi Gi diftribuifcono come fi fa al giuoco del A¢az~
zalina, E quello far con gli occhi ferrati fi dice far forto , che i Greci in un Gmil
ginoco dicevano catamyerm , Lat. connivere. E coiui che é ftato fowto quattro vol-
te ,¢ non ha mai trovato il nalcofto , ¢ per confegucnza perduti i quattro premj,
occupa il Iuogo di colui , che teneva fowo , € quelto s-iniruppa con gii altri Ra-
gazzi,fra i quali fi tira Ja forte a chi dee Mar foro , 0 nafconderfi ; E cost fegui-

 

tano canto y che fi riducano tutti liberi ; perché quello che ha pagati li quatcro~

prem)

 

3
5
&

 

 

 
 

 

SECONDO CANTARE. 103

© premj nel modo fuddetto , ed ha occupato il luogo di tenere gli altri fotto, come
ne vien cavato nella maniera accennata , refta fuori del giuoco , del quale folo
attende la fine per confeguire anch’ egli la {ua parte de 1 premj da diftribuirfi, Era
“ancor quefto giuoco appreflo a 1 Greci , ¢lo chiamavano -Apodidra/cinda {econ-
do Giulio Polluce lib. 9. c. 7. 5 ma diverfificava alquanto ; Ed in quefto giuoco
pure il vincente era detto il Re, ed il maggior perdente ? Afino. Vedi il Buleng.
de lud. Grace, cap. 22. éd il Meurfo in verbo e4podidrafcinda . Simile a quetio
era ancora il giuoco detto da’ Greci Myinda ,

OGNVNO 4 un mo non é compofto . In quefto proverbio fentenziofo habbiamo
f ancor noi come i Latini pi modi di dire, come: Le nature fon diverfe. Tanti
)— buomini tante berrette , 0 tanti cervelli, Tutte non poffono effer aun modo, Chi la
te 4 leffe ye chia rofto ,¢ molti altri ; ene i Latini fi trova . Quot homines tor fen.
tentie,, Suns cuique mos, Trabit fua quemque ia « Won omnes cadem mirantar,

amantque , ed altri infiniti , ¢ tutti con Jo fteffo fignificato .
PAR le merenducce . \ noftri Stovigliai in alcune Fiere,che fi fanno in Firenze
il giorno della festivita di San Simone , ed in quello di $, Martino conducono

  

 

; gran quantita di stoviglic piccolitfime,come piatti , tegami , pentole , ed ogni al- ,

tra fpecie di arnefi ,¢ vafellami da cucina , che da effi fi fabbricano di terra. Di
; este fi provveggono li nostri Fanciulli per quanto vien loro permefio dalla loro
~ borfa , ¢ da queste vien poi loro I’ occafione di far le AZerenducce , percht haven.
-sdo altre'mafierizie adeguate , come tavole , sgabelli’ , bicchieri , faluiette ,¢ fimi-
‘li); imbandifcono una menfa , accordandofi pit Fanciulletti , ¢ Fanciuiline a por-
tare quello’, che ¢ dato loro per merenda , ed accomodando tutto in piccole par-
ticelle , le distribuifcono in quei piattellini,figarando di fare un Banchetto, e met-
*tono a federe a quella tavolina li loro Bambocci ; E queste fonda loro chiamate
© Merenanece,delie quali parla ij Poeta, ¢ le quali erano ufate ancora dalle Panciul-
line antiche in occafione del fuddetto appellato Phitramelie » come fi ca~
va da! Meurfio , dal Soutero , ¢ dal Buleagero .
BAV-AGLIO , Saiuietta » 0 Tovagliolino da Bambini , che fi lega al collo con
due cordelline , o nastri , detto cosi dalla bava, che fopra vi cafca dalla bocca de
~bambini ; i Latini pure fecondo 1!’ Onomastico lo dicono pe‘torale falivarium,e con
A mee Bavazlé come lor proprj arnefi apparecchiano Ie loro piccole tavole quan-
~ do fanno le A4erenducce, ¢ Gi mangiano quelle particelle distribuite in quei piattel -
“Aini3-come s*é detto fopra . EB di queste Aderenducee parla il Poeta .
ST ACCIABBVRATT A, Due {eggono incontro |’ uno all’ altro , ¢ fi pigliano
per le mani , ¢ tirandofi innanzi , ¢ indietro ; come fi fa dello staccio abburattan-
do Ja farina , vanno cantando una lor frottola , che dice .
} Staccia abburacta
» Martin della gatta :
La gatta ando pel vino , ec,
E questo é trastullo ufato dalle Balie per acquietare i Bambini di quella eta,che
_ appena fi reggono in piedi 8 |
\\ ALT ALENA, Paflacempo da Fanciulli ; Legano due funi al palco,o vero a
_* due al beri, ¢ le fanno calare a doppio fino pretio a terraun braccio , ¢ fopra di
» eff funi accomodano un‘afle, fopr’ alia quale fi pone uno , 0 pits a federe, ve
; are

 
   

 

ea | ee
 

104 MALMANTILE

dare il moto a detta affe vanno cantando alcune canzoni con, tin! aria aggiustata
al tempo dell’ ondeggiamento di quell’ affe, ¢ questa ¢l' £ora de’, Greciy,dai La-
tini detta Ofcillatio , ed alore, yolte. Petaurnm penjile, ¢noila diciamo Alealena dal
Latino Todewen, che yuol dir quella Macchina di legno., ,con,la quale fi cava.Jtac-
qua de i pozzi ( come fi vede in Plin, lib, 19, c, 4. Vel Tollenonum haufkn rigandos)
da noi detta Mazacavallo . Vedi forto C. 6. staan. 86.,E questo perché faceyano
I Altalena , come Ja fanno talvolta anche Ji nostri, Fanciuili con iacrocicchiare
una trave fopr’ all’ altra , ¢ ponendofi ugo o pil ragazzi per teftata della trave ,
che ¢ di fopra,la fanno alzare, ¢ abbailare a foggia di Atayzacavallo. Diguestas
parla i! Bulenger, de Ind. vet, c, 11. Questa ditaiera, in aicuni luoghi di, Tofeana
€ detta bictancole, 5 ,
BECCALAGLIO . E’ un givoco fimile alla mofca cieca detta fopra. Cox, flan,
40. ne vi é altra differenza,che dove in quello fi da,con yn panno.avwolto, 9 altra
cofa fimile,in gquelto fi da con Ja mano piacevolmeate una fola, volta da.colui,che
bendé gli occhi a qucl,che fta forto , ed il bendato in wece di dare, 5 affanna di
pigliare un di coloro, che in quella flanza fono del giuoco , ¢ colui che reftera
prefo,deve bendarfi in Iuogo del bendato , ¢ perde,il pegno,.0 premio.,ed.il pri-
mo bendato refta libero , ¢ s’ intruppa fra quelli, che hanno.a eden prefi sie fifa
come fopra nel giuoco di Guancia] d'oro, Sidice Becealaglia, . quefto tale
bendato vien condotro in mezzo della ftanza, 0, piazza, doye s! hada fare il giuo-
co ; ¢ colui che lo bendo, ¢ che quivi I’ ha condotto gli dice ; Che fei tu-venuto\a.n

   

 

fare in piagza? Ed egli rifponde; 4 beccar /’aelio, E quello,dandogli leggiermen-
‘te See fur’ una fpalla foggingne :, O beccati codefto, Dopo. la qual fungio-
ne il bendato s’ affatica,di pigliar uno per metterlo in fuo,luogo . 1 Greci appel-

lavano quefto pernen Claeiede da pencola che in Greco, fi dice Chysray edo fage-
vano nella ftefla manicra ; ma in vece di, bendare gli ,occhi,mettevano.a colui, 0
fingevaG , ch’ egli tenefle colla finiftra una _pentola in capo, ¢ girandogli intorno
Jo [olleticavano , o percotevano ; onde , fe egli rivoltandofi , prendeva chi gli
tirava; il prefo rimaneva in cambio {uo a.eflere quel della pentola, 4 Latini lo
dicevond tidus ollarins s 4 & , aillapre>- Su
Simile.a quefto era un‘altro giuoco ulato dalle Ragazze Greche, detto, Cheliche-
Jona, vel quale , mefla,a {edere quella , a cui dayano nome di Chelona x chevuol
dire Teftuggine ; Ie dicevano : Chelichelona quid facis ix medio ?. quella rifponde-
va: Lanam sexo ,@ filum miltfium con quel che fegue riferito dal-Buleng. de>
Jud. vet. cap. 41. F ae Tere
Nel giuoco poi della Chyrrinda , ovvero, ludus ollarins dicevano : Quis ollam ?
¢ chi teneva la pentola rifpondeva: Ego Adidas , cs! affannava non di pigliare un
di coloro , ma di toccarlo co i piedi , ¢ quel tale casi, tocco perdeya , ¢ fi metteva
la pentola in capo; E perché ( comes’ ¢ detto fopra ) i Greci havevano per co-
flume di chiamare Re il vincitore , ¢d afino il perditore., pecd quelto tale , ches

havea la pentola in capo fi. a Adida , clot Re efino, Vedi Giulio Polluy —

ce lib. 9. c. 7. ed il Buleng. de Lud. Vet, c, 17.

ANDAR a predellucce , Duc fi pigliaag peri pol a’ ambedue le: st tno!

con I altro in croce , ¢ formano come una {eggiola , ¢ un’ altro vi fiede: 3¢

quelto fi dice andar’ 4 predellucce . Das feis=is-ninne un guinea ae aekan{*> ,
ae ae

 

  
  

+

&

4
 

 

SECONDO CANTARE: 105

ed era il portare uno in fu le fpalle , ¢ reggerlo , tenendo Ie di lui ginocchia nelle
paime delle mani voltate dietro alla perfona , ¢ detto Zz Coryla , cic nella,
siotola , © cavo della mano. Ma quefto credo che fia un’ altro giuoco , che noi
diciamo 4 cavalluccio , che vedremo fotto C. 3. flan. 30. tanto pit che i Greci {e-
condo Io ftefio Polluce chiamano quefto giuoco detto 4x Cory/a , per altro nomes
Hippada dal verbo Hippazin , cavalcare . E quefto fe bene € giuoco , tuttayia &
fpecie di pena per quei,che portano per haver perduto ad altri de’ fuddetti giuochi.

ACCVLATT ARE, BE’ pafiaiempo da Ragazzi, ma é {pecic di pena , ¢ di tor-
mento dovuto a colui che € acculattato . Quattro ragazzi pigliano uno per les
braccia, ¢ per i piedi, ¢ formandone un quadrato , Jo follevano , ¢ gli fanno bat-
tere 11 culo in terra tante volte , quanto merita il fuo delitto, o perdita, che ha.
fatto in altri giuochi, come fopra . E quefto fi dice acculattare , che in altro fi-
guificato vedemmo fopra C. 1. ftan. 7. Gli Spagnuoli chiamano I’ Acculattares

“mantear,perché mettono colui che fi ha da acculateare in una coperta,o mantello,¢
tenendola da quattro capi , lo sbaizano in alto , € lo fanno ricadere in efla , ¢ noi
lo diciamo dar la coperta,

V1 fu caglio per tutti, Vi fa da dar foddisfazione a tutti. Ognuno hebbe in che
impiegarfi . Traslato da’ Sarti,che dicono :1n ete roba ci é raglio per un’ Abi-
to,0 per due,ec, per intendere,ci ¢ tanta rcba,che fi pud fare un’ Abito, o due, ec.

ST AR in barbadi Gatta, 0 di Micio , come fi diffe fopra in quefto C. ftan. 28.
annotazione alla voce sbigortito, Pare che oo detto poffa venire dall’anticas
fuperitizione degli Egizz} , i quali credendofi , che il Gatto fufle confegrato alla

fide: , che era ia loro Deita maggiore , non folo nutrivano con granditiima
cura , ¢ fplendidezza quefto animale , ma fecondo Pierio Valeriano reputavano
degno di morte colui , che ne ammazzafic , o faceffe loro oltraggio. E riferi(ce
Alcx,ab Alex, dier, Gen; lib. 3, cap.7. ¢ lib. 6. ¢. 14. che quando moriva un Gat.
to,i medefimi Egiz2j per contraficgno di dolore firadevanole ciglia,e poi metren-
do addofio al morto gatto fale, ed aromati, e coprendolo con un panno bianco lo
feppellivano, facendoli taluolta fepolcri notabil;tanta era Ja flima che ne facevano.

XXXXIX. STANZA L.

Mai fu tra lor fin qui nulta di guafto , Bench" if Suocero altura , ¢ la Conforte
Se non che Florian volto ale cacce , Maledicefrer quefto [uo motivo

. -Hatvendane pit volte tocco un taffo , Dicendogli che la fuor delle porte

 E fentendofi dar fempre cartacce , Va' Orco v' é st perfido , ¢ cattive y

Difpofe al fin di nan'veler pi pafto , Che perfegutta ? huomo infino a marte,
We curando lor preghi , ne minacce E che lt ingoierebbe vivo vivo ;
Fece innitar da i faliti Bidelti Con gentt , ed armi ufct ful axrora
Per Paltra di i Piacevoli ,¢ i Piattelli, Gridado: Andiane,adiane,eccolafuora .

Non hebbero ( come s’ ¢ detto ) quefti Spofi mai occafione d” addirarfi, fe non
che Floriano inclinato alla caccia fi rifoluette andarvi a difpetto della Moglie ,

_¢del Suocero: .

NON fu nulla di guafto « Non furono tra loro mai rotture ; ciot_non s’ adira-
Tono mai ; ¢ , come fi dice 5 non s’ ingroflarono i fangui .
 HAVENLONE toveato un vasta, Havendo di cid domandato alla sfuggita , 0

_ difcorfone con brevita. Tratto da i tafti del Cimbalo , o'vero Organo ftrumenti

ae

muficali . oO DAR

*

3%,
106 ‘MALMANTILE

DeAR cartacce, Non rifpondere fecondo il gufto di chi richiede; Traslato dal
giuoco di minchiate , nel quale fi dicono cartacce quelle che non contano.s:¢ fo-
no di niun valore. Vedi forto C, 8, ftan. 61.

DAR pafto, Trattenere uno con fcufe, o chiacchicre. E il latino verba dare ; /pe
laitare. E fi dice cosi,perché il polmone degli animali(che da noi fidice pafo)ftracca
colui,che lo mangia , ma non Jo fazia. Si dice anche dar paffo, quando uno, che
fa giuocar bene a un tal giuoco,finge di faper poco , ¢ fi lafcia vincer da princi-
pio , a fine d’ indurre il femplice a far grofle potte per vincergli aflai .

SIDELLO , Donzello , o Seruitore d’ Vniverfita , 0 d' Accademia , come fa-
rebbe quel Donzello , che ferue allo Studio di Pifa , o ad altri fimili . E quefo
nome di Bidello fecondo }’ Autore deile Notizie Ecciefiaftiche ¢ corrotto da Pe-
dullus, perché quefto Viiziale , ( dice egli ) che nell’ Accademie , ¢ negli Studj

pubbiici haveva cura d’ efeguire le commiffioni appartenenti allo ftudio , folevas
portare in mano un baftone chiamato Pedo; Quantungue aleri ( foggiunge il me-
see) tirino Ja fua etimologia dalla parola Saffonica Bydell , che vuol dire il
anditore,

Ma io.credo che il nome Bidello fia tolto da Berul/a , che & quell’ albero , del
quale fi facevano le verghe per i fafci , che anticamente portavano 4 -Littori
d’ avaati.a i Magifirati del popolo Romano , ¢ che da quefto portare i fafci di
verghe di Betulla , fia poi venuto il nome di Bidello a tali feruenti di Vaiverfita ,
i quali faono figura di Littori , € nello ftudio di Pifa portano ancora una grofia
mazza d’ argento ( fignificante gli antichi fafci ) quando vanno in funzioni pub-
bliche avanti al Collegio de i Dottori.. Alex, ab Alcx, dier. Gen, lib. 12.17. in
fine , dice cosi.

Quodque fafcibus , quos praferebant Lictores , betullas virgas maximt commodas dis-
sere , itague ex illorum virgis tum proper candorem tum propter tennitarem publices
Lafces , qui magifiratibus prairent, efecere . E Plinio lib.6. c. 18. Gander frigidis for-
bus ,& magis Berulla ; Gallica bac arbor , mirabilis candore arque tenuitace , terribilis
Maziftratuum virgis . Lo fteffo attefta Polid. Verg. lib. 4. c. 3. 1

oa » ¢ Piattelli , Sono in Firenze due conaerfazioni di cacciatori ~

wali andando alle cacce gareggiano fra loro a chi faccia. maggior predayequeila,
ane rimane fuperiore, nanos fuole entrare nella Citta teionfante cebaoth 5
carri , ed altro ; ¢ P una fi dice la Compagnia de’ Piacevolé ,¢ Valtra de’ Piatrelli;
¢ ciafcuna ha Ja fua fanza catro alla quale s' adunano. gli Vfiziali , ¢ Serucati ,
¢ Altci; ¢ quefti fon quelli de’ quali dice i] Poeta , ¢ chiama i Joro feruenti
idelli.

VN’ Orco. Quefta ¢ una beltia immaginaria inventata dalle Balie per far paura
ai bambini , figuraadola uno animale {pecie di-Pata , nimico dei bambini catti-
vi, ¢d il Pocta , che noms” allontana mai dal genio'pucrile, moftra che ibfuoce-
ro Stordilano voleva indurre nel. Floriana ihtimore per farlo aftenere dais
andare a caccia , con dirgli che fuori della porta v’ era l’'Orco , ers,
huomini : Quefto nome perd viene dall’ antica faperftizione de i yi quali
chiamavano Orce l'Inferno Virg. a. lib. 6, Priemifque in faucibus orci, Bd inten-
devano per Orco anche Plutone , quali wrgus , five Kragus ab urgende egli
sforza , ¢ Spinge rutti alla morte; ¢ percid dalle madri , ¢ nutrici per’ Peas
ial

> Skee? pp seeth seh ce etre, .. x

 

 
SECONDO CANTARE: 107

alli lor bambiai fi dice che ' Orco porta via : il che pure vien da i Gentili , ches
igliando Orco per la morte , lo chiamavano Ineforabile ,-¢ rapace. Orazio
Bae 18. lib. 2, Nulla certior tamen
Rapacis Orci fine deftinata,
~ GRID ANDO andianne andianne., ec, Cosi vanno gridando i cacciatori faddetti
Ja mattina avanti giorno per fuegliare i compagni. Lo fteflo , che Alo Alo ; ovs
vero lon dal Pranaele eAilons .
STANZA LL

 

 

Senza veder ne anche.un’ animale
Frugo , bufso y gird pi di tre miglia ;
Pur vedde un tratto correr un Cignale
» Ferace , grande, e groffo a meraviglia,
STANZA LIL
Che a posta prefabavea quella fembianza,
E glk pafso fuggendo atlor a! avanti
Per traviarlo folo con {peranza
D: haver a far di lus piit boccon fanti ;
Cosh guidollo fino alla [ua fpanra
» Dov'ei pense di porgli addoffoi guanti;
~ Poi nan gli parue tempo , perche i cani
» blauersan piit tafto lui mandate abrani ,
STANZA if,
Pero.volends andave in ful ficura
Won a perdira pitt che manifefia ,
Perché a reder roglieva un’ offo duro
~ M€entre non to chiappaffe tefta tefta ;
Glifparid' vcchio,e fece un tempo [euro
Per incanto levar, vento, ¢ tempefta,
£. uolash gros comparire ,
Cc Dearehte ph ans may che mi dire,

uFioriang {corfe moita campagna , ¢ ce

Ond'ei, cht il di dovea capitar male
Si moffe a feguitarloa tutta briglia ,
Won effendo infor mato ch'in quel Porce
Si trasformava quel ghiotton dell Orco,
STANZA LIV.

A cacciator , che quivi erain farfetto ,
E dal (udore omai tutto una broda ,
Havendo un veftituccio di dobretto ,
Ed nn cappel di brucioli alla moda ,
“Per non pigliare al ventoun mal di petto,
O altro’, perché il Prete non ne goda,
Won trovado attra cafain quelfainatico,
Che quellagrotta , infaccavi da pratico.

STANZA. LV.

Atal gragnizola , a venti cosi fieri
C* ogni cofa mandavano in rovina ,
Tal freddo fu che tutti quei quartieri
Sen! andananoin diaccio,¢ in gelatina ,
Ed ci ch’ era veftito di lercieri ,
E mai meglio facea la furfantina,
Won pit cercava capriole, 0 damma ,
414 dafar,s' ei poreva,nn po di famma,

  

rcd buon pezzo , ¢ non trove mai nulla,

fe non che pur vedde un grotio Cignale , a] quale fi mefle dietro co i {uoi cani ,
* non fapendo, che ¢ra P Orco trasformatofi in quel cignale per pigliar Fio.

riano dalla vitta lc fpari, ¢

pioggia, ctempefta, 1a-quale obbligo Fi

t via de’ uoi incanti fece venire una gran
loriano a ricovrarfi in una grotta , che cra

vi fra quelle macchie , nella quale entrato , fi meile a cercare {e trovava modo
deme un po tiesto 4
£.

AVGO .. Cioe cerco:minutamente
do con le pertiche per tutto... ~

‘frugando per le fiepi con i cani, ¢ buffan-

DOVEA capitar mate. Doveva haver difgeazic. Doveva rovinare, E il Lat.

-— Perdsyperire,
. cd TVTT Abrigiia, A quto corfo

.. GHIOTTONE . Epiteto folito da

«

 

f  fenza punto fermarfi , come fa il cavallo
quando fe gii lafcia ee « Laxatis babenis
( ) ‘aun’ huomo maligno , e di genio cattivo,
€luona quafi lo itetio, che Briccone , furbo , viziolo, {cellerato .
+ O2

py

 
108 MALMANTILE

‘PIV boccon fanti., Pid buon bocconi . La voce fanti in cafi fimili fignifica per-
fezione in generale. Vedi forto C, 3, flan. 8.

PORRE iguanti a deffo. Piglia guanti per mani, ¢ vuol dire Pigliarlo .; Hab-
biamo il verbo agguantare , cioe pigliare », Guanto dal Germ, Hend: , mano.

ANDARE in ful ficuro, Andar enza paura. Metterfi a fare un negozio cons
ficurezza di non efler’ impedito , e che rie{ca fecondo I’ intento .

TORRE a rodere ut? offo duro , Pigliare a fare una cofa difficile .

CHIAbP ARE, Qui val per ritrovare,e fopra in quefto C. flan. 41.per perquo-
tere ; ed il fuo propeie fignificato ¢ Pigliare ; dal Lat, capere .

TEST A tefta, Cioe a folo a folo. Remoris arbitris, Diciamo anche a quat-
tr’ occhi. :

GRAGNVOLA, Grandine , che é gocciola d’ acqua congelata nell’ aria per
forza di freddo , ¢ di vento , e fi fa di vapore freddo , ¢ umido ftropicciato nelle
parti interiori.del _nugolo, La pioggia nalce da vapori freddi , ¢ umidi adunati
ne i nugoli, La xeve ¢ impreffione generata di freddo , ¢ d’ umido; ¢ quefto fred-
do @ minore di qucllo,col quale dalla pioggia vien generata la grandine , ed ha in
fe qualche parte di caldo. La rugiada ¢ gencrata di freddo , ¢ di umido non rap-
prefo , e quefta congelandofi nell’ aria diveata la brinata « Ho voiuto,benché fuor
di propofito , notare I’ origine de i fopraddetti accidenti dell’ aria , perche da.
quefta s‘intendano i loro nomi; in qualche parte d'Italia per ayvencura differenti .

HAVREBBE infranta nun fo che mi dire, Haurebbe {chiacciata , o diciamo an-
che ammaccata qualfivoglia cofa per dura che fufle; Non fo immaginarmi , ne
dire cofa tanto dura , che ella non I’ havefie infranta . Quefto termine 2on fo che
mi dire ufato nella forma , che fi vede nel cafo prefente, fignifica quel che s’ ¢ det-
to; ma per altro.’ ufiamo anche per denotare di non havere , o faper trovar
modo di rimediare a qualche accidente » per clempio: Lo non fa che mi dire, fe it
tale vuol far male i fatti fuoi, ©

IN farfetto, Veltito leggiermente . Farfetto hoggi intendiamo ogni forta d’a-
bito leggieri , ¢ difinuolto , che fopr’ alla camicia fi porta fotto gli altri abiti, co-
me farebbe camiciuola , 0 giubbone , ec. “

TVTTO una broda di fudore . Tutto molie dal fudore ; Sudatiffimo per la fati-
‘ca del viaggio violento . i

DOBRETT O.Inrendiamo una fpecie di tela di Francia fatta dilino,e bambagia
(che é il cotone filato ) , Sidice anche Dob/erro da duplex ,perché nel tefferio,e fatto
di doppia orditura, ¢ riempitura . Cosi. dobb/a © dobbra diflero gli antichi .

BRYCIOLI. Quelle fortili ftrifce, che il Legnaiolo cava da qualfivoglia legno
lavorandolo con Ia pialla, fi dicono bracioli , forfe dalla fimilitudine de’ brucioli y
bachi;e da quefti fi diconocappelié di brucioli quelli,che {on compofti;ed intefiuti di
ftcifce d’ un’ erba particolare , nello ftcflo modo , che fi fa con la paglia y alla
fimilicudine, ¢ larghessa della quale fono ridotte le dette ftrifce .

e4LLA moda . Ciot alla foggia che ula ; la quale cra nel tempo , che I' Autore
compote 1a prefente Opera 5 che i cappelli havevano piccola falda. Si che non
tanto per efier di brucioli., quanto per effer piccolo , cra poco atro a difendere»
dai acqua.. Si dice alla mods quali all’ ufanza,che ¢ modo,cioe adeflo,Pr, alla moda,

ei AL di petto, Cosi.chiamiamo volgarmente quell’ infermita., che 1 Medici
Ficena Pletritide. PER.

 

 
 

 

LR Sree Te ga ere ne MRS OA MMeare | ice nared

SECONDO CANTARE: 10g

» PERCHE il Prete non ne goda , Cioe per non'morire , € cosi far che il Pretes
non goda il guadagno della cera del funerale .

QVEI quartieri . lotendi per quelle campagne , per quei contorni. Che per al-
tro noi Fiorentini per guartsere intendiamo una delle quattro'parti , nelle quali ¢
divisa la noftra Citta. E guartiere in lingua militare fignifica Habitazione ¢ dar
quartiere al nimico fignifica faluargli la vita , ¢ farlo prigione .

1NSACCAVI da pratico. V’ entra dentro come fe egli,per eflerui entrato altre
volte, fapefle la ftrada , e vi fufle pratico.. Se bene buomo prarico ufato nella ma-
ty che ¢ qui, vuol dire huomo favio, ¢ da faper pigliar compenfo in ogni oc-
ea

GELATINA, Vivanda nota fatta per lo pili col brodo di carne di porco cot-
ta in aceto’, © poi congelato ; Ma qui per Ge/atina intende che I’ acqua s’ andava
congelando fopra il terreno, ¢ fa Gelarina finonimo di Diaccio,come fa D, inf. 32.

PAR la Furfantina, Si ova una fpecie di Bianti , i quali per muover Je per-
fone pie'a far loro elemofina , dopo haver bevuca buona quantita di gencrofo vi-
no,ne i tempi pid freddi fi diflendono mezzi ignudi nelle ftrade-pid frequentate, €
tremando fingono di morirfi dal fieddo , e quefto lor tremare fi dice far /a Pur-
fant ina , cio’ fare it giuoco-che fanno quefti furfanti,ch’ ¢ poi paflato in dettato,
che fignifica,, ¢ comunemente s’ intende Tremare .
~ MA meglio, Benitimo., Gia mai fi trove chi facefle meglio. Quel ma vuol
dir mai; la figura apocope.

‘DAA A1 A.E' \o fteflo,che Daino {pecie di capron faluatico.Lat, dama D. Inf. 4.

Sh si farebbe un'cane infra due dame , ec.
STANZA LVI.

Trove fucile,ed efca,¢ legni var}, Cosi con tutti commodi ae... pari,
Ondun buon fuoco in uncantone accefe, Dopo una lieta , ilcrogiolo fi prefe 5 :
E in fu due faffi poftt per alari , Effendofi a far quivi Siu > :
Sopr’un’ altro fedendo i pit difhefe. Mentre pioveva , come quei da Prato i

Bloriano: havendo trovato’ ia ‘quella grotta comodita d’ accendere il Fuoco ,
P-accefe:, ¢ vis*accomodd a fcaldarfi, alpectando che intanto ceflafie la pioggia.
FVCILE . Intendiamo quello ftrumento d' acciaio , del quale ci feruiamo per
battere nella pietra focaia ad efferto di cavarae il fuoco; detto Fucileda fuoco,
quafi fecaio , 0 facile. Che per difiefi anche Focile.
£SC.A, Quel fango , 0 fia cuoio corto conciato'col falnitro , che facilmente»

 

 

iglia fuoco., ¢ ferue per tener fopra alla pictra quando in efla fi batte per trarne
i oa 3 dai Latini detta fomes. La qual rete,fot ben per translato fighifica inci-
tamento., © flimolo, che noi pure diciamo fomite , nondimeno era intefa per
ogni cofa facile a pigliare quel fuoco, che Vergilio-appella seers eth-
Sirufa in venis filicis < Si come noi, ancora diciamo:E/caogni forte di cibo d’ ani-
mali , pure dab latino £/ca.,. che yuol dir'cibo ,’ed incendiamo ancora quefta ma-
teria , che ¢ atta a pigliare fubito i! fuoco, quafi fia il cibo del fuoco ; anai a que-

fia non diamo altro nome, che a’ ¢/ca , e dicendofi £/ea aflolutamente , ¢ (enzas 3

Aggiunta , s’ intende folamente-quefto cuoio cotto, © fungo conciati con falnitro. = ih

“ALAR! , Sono due Ferri 5 o Safi, che fi tengono nei focolare’, perché man-

i ‘tengano folpele le legne , acid che pil facilmente ardano. £’ voce ease i
ue

 
“110 _ *MALMANTILE

Latino /ares , la qual voce fpefle volte era prefa per fuoce ».come fi pud dedutre
da Ovid. 1. faft. 18. ‘i % ve

Omnis haber gemings hincsarque bine ianua frontes

E quibus hec Populum /pettar 5 @ ila Larem .

Eda Colum. lib, 11, cap, 1..de Villico., .Con/uefcat rufticus. circa larem Damini,
focumque familiarem femper epularé. 1| Sipontina dice cost :) Lares Di erant apud
Gentiles, & colebantur domi , focu/que illis [acer erat , unde vulgus focum focolare ap-
pellat quafi laris focum. Molti in vece di dire 4lare dicon arali , o fia corrotta~
mente , 0 pure , perché gli piglino da era, intendendo ftrumenti da mettere in
fu I’ altare per foftenere le Jegne per il fuoco de i facrifizzj , ¢ cosi fanno che fias
ben detto tanto arali, che alari. ; ‘oe )

AC, pari. Agiatamente fi dice anche 4 pie pari .. Vedi(opra;Cant. pr, flan.
82. Lafca Novella 4. lib. 2, Serusti delle buone vivande:y che voi fapere bene acconces
e fragionatg fe ne frettero a pic pari, Si dice anche agambe larghe. Vedi.forto.C. 9.
flan, 32. Ed in miolti altri movi, che tucti maftrano la fpenfierata agiateaza duno,

DOP’ una liera. Dopo una famma.  Diciamo Aer una fiamma chiara , fenza
fumo , ¢ che prefto paflia detta diera da (etitsa , come anche baldoria,da baldore(cio®
baldanza ) voce antica .. Gli Spagnuoli fimilmente dicono alegro, un fugco dal
legria. Vedi fopra C. 1, flan. 4.0 fore fi dice seta selngeae Gieramente,che ap>
preGo ai noftri Contadini vuol dire prefamense, ciot cofa,che pafla preftamente.

PIGLIARE il Cregioio. Stagionarfi, Quando fon format i bi chieri , ed altri
vali di vetro , gli mettono cosi caldi in un fornelletto, che a tal fine ¢ fopr' alla
Fornace , dai ord chiamato Camera , dove ¢ un.caldo. moderato , ¢ quivi gli
lafciano flagionare , ¢ freddare a poco.a poce, conducendoli con un ferro alla
bocca del detto Fornello per da batio,dove non fi fente pid.caldo , il che da edi G
dice dar asempra , temperare 5 0 dar il Crogiolo., 0 Cragiolare. E. di qui parlando
dell’ hyomo intendiamo piglare i/ Cregiolo , quando dopo una fiamma egli conti-
nova a fare attorno al fuoco, fino che fia tutto incenerito. E da quelto verbo
Crogiolare piglia , 0 ha I’ origine, il Gregivele sche ¢ quel valetto. di terra cot-
ta, il aale ferue per mextervi de ’ @ liguefare , 0.fondere i metaili nella Fors
nacc,detto corrottamente Corergimalo, is gabe BAN

FAR come queida Pax Proyerbio vulgatiffimo, che. fignifica La(ciar piovere;
1 Fopoli della Citta di Prato., che ¢duddita , ¢ vicina a. dicei miglia a Fircnzes ’
nel- tempo ,\che i Fiorentini fisreggevano.a Repubblica , domandarono Jicenza di
poter fare una Fiera jl di 8, di Settembre, ( ja qual Fiera Gi continova fino al pre-
{ente in detto giorno ) ¢ per tal’ efietto. mandaropo Ambalciadori alli SS. Priort
di libeeta , da 1 quali fa Joro.conceduca la domandata Jicenza ».con Neincliag:
pagaticro una certa { di denaro . Accordato ib aegozio gli Amb; a
partirono ; Ma ¢flendo nell ulcir del Palazzo , fowvenoe loro, che {ein talgior-
no fufie piovuco , non haurebbono potuto far la Fiera, ¢ nondimeno farebbe loro
conyenuco pagare il danaro accordato ; onde per aflicurar quello punto tornaros
no indictro , cd entrati di nuoyoida i SS. Priori, uno di ch ambafeiadon feng
altre parole difle: Signori, {fe ¢'pioveile? Alicheuno.de’Signori iybita eifpole. =
Lafciate piovere, E di qui nacque quelto proyerbio Far come quel da.Prato’, che
fignitica Lalciar pioveres 1 ‘ ie 6 Asal Eas

3

 

 
Le a et eT et ee ee

 

=

SECOND'IO CANTARE. =

STANZA LVIL

LZ’ Orco fratantocon mike atri,e feorci,

etffacciatofi all nufcio , ch’ era aperto ,
Prego Florian con quelgrugninda Porci
Tutro quanto di fango ricoperto,

Che ( perch'ella veniva gin con gli orci)
Ricever o voleffe un po ul coperto ,
‘Ritrovande/i fuora fealzo , ¢ ignudo
A sigran pioggia,e a tempo tosh trudo

TIL
STANZA LVII1,
Hebbel giovane allora un eran contento

Dhaver di nuovo quel beftion veduto ,

E favendogli addoffa affegnamento ,
Luafi in wn pugno gid Phaveffe hanuto,
Rifpofe: Volentieri ; entrate drenjo ,
Venite, che voi frare il ben vennto,
Che dopo ilfugeir voi Uamite, eit Zielo
Fate a me; ch’ ero fol, fernizioaCiels.

 

Mentre Fioriano flava a fealdarfi; 1’ Orco s' affatcio alla bocca della grotta
fenz’ haver mutata la figura‘di Cignale , ¢ pregd Florian , che !o ldfciatle entra-
re; Eiglirifponde , che entriallegramente , ¢ che ne riceve fervizio , perché
effendo folo,ha cara un poca di Compagnia .

. Non fi maravigli il lettore , che un Cinaate parli ; ¢ fi ricordi , che ¢ una No-
vella per i Fanciullini , e che quefte cofe feguivano .
i Al tempo , che volavano + pennati ,
: Tutie'le cofe fapevan parlare ;
Secondo , che dice quel che de(crive la guerra di Carnovale con Madonna Que-
fefima « Apill. As.) i2. Parietes locuturos,boues,o id genus pecora dittura prefagint .

GRVGNO . S' intende ia faccia del Porco 5 da grannitus , che ¢ lo ftridere del
Porco . Grugnino ¢ detto per vezzi , ma qui ¢ ironico , ¢ per derifione Guardate
bela faccettina, 0 bel grugnino’, 0 bel. grugno , quando yogliamo jitendere una
brutta faccia 5 EB fi dice baver i/.gruero,dell* huomo quando ¢ incollera , donde ix-
gragnare por entrar in coliera’. Vedi forto C. 8. flan. 61. ¢ /erugoni fi dicono le
pugha dace net vil. '

ELLA vith gi ton gli orci. Cio’ piove Ne di¢a: Ogai goccib-
lad di tanta acqua ; quanta rie cade a dar Ja volta a un’ Orcio , che ne fia piend.
Sidice anche Zila viene a bigonce, a carinelle , ec, tutte iperboli per denotare , che
piova gagliardamente . Vedi forto C. 10. ftan, 20.

FALENDOGLI addoffo ufiegnamento , Difegnando quello , che yoleva far di
aa i# in fito potere Ȣ dominio , come efprime il Potta medefimo di-

: Quafiin ne gid P hawe/sé haunto, ' vt
BAR i naan fun feruizio , 6 favote accettiffime , ‘6 gratidifimo,
STANZA LIX.) STA EX. ;
Poi diffe > Hor vin venité Alta ficnra ..
Rifpofe ? Oreo: Jo non Verrd ne arco,

Credi tu pur ch? io fia cosi merlotto! Guarda la gamba | perch’ ia he panra
Se non glicanfi ci verrd domani, | ~ Diquellalrifeiagh iwrlveggo ut fiico,
5° altro,dice il garzon ,non ¢°é di rotto Allor Florian la cintura’,
Die pieche te gli vo" legar lontani', Ed impiarth la [pada ott’ un banco ,
© Eprefo allora il [uo guinziiglio in mano’ Diffe'l' Oreo : ( dedutald riporre ¥

© Lagi in un canto T ehero y¢Giordane, “Jo ti ringracierei; ma non accor.

STAN-
i ens

112
STANZA LXL
E lafciata la forma di quel verro,

Prefal' antica ,e moftruofa facia ,
Con due catene salto la di ferro ,
E lo lego pel colle, e per le braccia,
Dicendo: C acciatar tu bai pres’ erro,
Perché credendo di far predaincaccia,
All fin non bai fart’ altroch unavefcia,
Ment’ il tutto ¢ feguito alla rovefcia.

‘MALMANTILE

 

STANZA LXIL

Rimafto ci fei tu, come tu vedi
Senza bifogno haver di teftimoni ,
E perché con leurieri , ¢.cami se fpiedi
Far me volevi in peri, ed in bocconi;
Coss perch’ ella vadia pe’ fusi piedé
Faraffi ate , ne leva piit ne pani,
Accio che, procurando I! altrui danno ,
Ler te ritrovi il male , ed s1:malanno,

STANZA. LXIIL

Ed io c* hebbi mai fempre un tale fcopo
D’ accarezzar ognun , benché nimico ,
Come la Gatta,quando ha prefo il topo,
Che , fe ben’ ¢ tra lor quell’ odio antico,

Scherzandocon effo alquanto,e poco dopo
Te lo (granocchia come un beccafico y
Cosi perché piit a fila tu mi metta
Veglio far’ io, ¢ poi darti la firetta .

L’ Orco alla cortefe ofierta rifponde ,che ha paura de’ cani, ¢ della {pada ; e»
Floriano lega quelli in un canto , € ripon quefta fetto un banco ; Allora l’Orco
fi {cuopre , ed entrato nella caverna prefe Floriano , ed incatenollo .

S/ch? E un termine , de} quale ci feruiamo per dimoftrare che habbiamo,co-
nofciuto I inganno , 0 cattivo trattamento., che alcuno ci habbia fatto , o hab-
bia in animo di farci, quafi dica: Cos} eb vorrefti.ch’ iofaceffi? 0 vero Cost mi
tratti eh ?

FATE motto, Proferito col primo ,o , ftretto.,. Vuol dire afcoltate , fentite..
Fate motto a me; ed ufato nella forma che é nel prefente luogo,ha forza d’efcla-
mazione, e vale per un certo modo di domandar configlio, quando ci detta una
cofa, che fia impoffibile a farfi, o a crederfi, quafi chiamiamo altra gente, che ci
configli fe quefta tal cofa fia da farfi, o da.crederfi ; ¢ che fenta lo fpropofito che
cié ftato deito. Dird per efempio ; Cofui dice che ha trent’ anni ye Sono pin di cin-
quanta ch’ ¢i nacque ; Fate motto! Cio’ udite {propofito ; O vero giudicate , fe»
cid pud effere . : 3 é :

SLA cost merlotto . Cink fia cosi femplice , cost minchione., cosi privo di fenno.

Ci verré domani, Detto ironico , che fignifica Non ci verro mai. Quefto De-
mani ¢ if Domani eterno di quell’ Oite, che hayeya (eritto fopr’ alla {ua bottega
Doman fi daa credenza ,¢ heggi no, Ghe !' hoggi era fempre ;,¢ i] Domani havea
fempre a venire.Berni 4 rivederci alie Calenae Greche,prcio da Such. in Aug. c. 87.

DYE picche .. Detto indetecminaro , fe ben pare determinato,, ¢ fignifica molto
lontani , ¢ non per appuaro Ja lunghezza di due picche ma forle aGiai pili, ¢ for-
fe aflai meno . ) 4

GVINZ AGLIO , Fi quella corda, o ftrifcia di gor » con che fi tengono. i le-
vrieri a lafla;e da molti é prefo per ogni force dj legame , derivandolo.dal verbo
latino wincio , come vincafire , ynciglia , ec. ma ftrettamente guinzagiio »\ 0. vinza-
gis 3 intende folo la corda , 0 quoio ,col qe fi tiene al Jevriero alla lalla ; fe»

ene da qualcuno ¢ intefo ancora per quel Jegame, col quale $\accoppiano in-
fieme i bracchi , o altri cani da caccia , Lat. copula . ‘ :

GVARDA la gamba! 1 Cielo me ne liberi , Ll Cielo mi guardi , che io fia per ,
far queflo. In Firenze nella Corte della Mercanzia , che ¢il Lee dovefi

anno

¢:

 

 
 

pet
ti

iF

 

SECONDO CANTARE. 313

fanno ¥ efecuzioni Civili,fono alcuni Donzelli , i quali fi chiamano Toccatori .
‘Quefti dopo che in una caufa fi fon fatti tuces gliatti , ¢ fi vuol venire all’ efecu-
zione perfonale, vanno ad avvifare il debitore , che fe-egli non paghera in te:mi-
ne:di ventiquattro hore (ara condotto in carcere ; ¢ fenzatale atto , che fi dice»
Toccare ,o fare il tocco , non fi fi pud con Cittadini Fiorentini:venire a detta cfe-
cuzione per(onale . Tali Togcatori anticamente pet effer conofciuti portavano
una calza d'un colore ,ed nad’ un’ altro , onde nel paflare che facevano fra le
Borteghe ,¢ peri i noghi pil gepeeass ixagazzi gridavano: Guarda la gamba ;
affin che chi era in grado d’effer toccato fi porefic fuggire , ¢ guardarfi , non po-
tendo i Toccatori far tale azione ne i luoghi iumuni ; ¢ fi dice Toccare perch
non ferve , che coftoro avvifino con Ja voce il detto debitore , ma devono for-
malmente toccarlo con la mano, E da guefto é venuto il modo di dire .
Guarda la gamba ; che fignifica mi guarderd , 0 fuggira di far tal cofa. [1 Lalli
neil’ En. trav. lib, pr. flan. 67. fi ferve di quefto detco ne] medefimo propofito .
Venere allor rifpofe; Honor Celefie
Guarda 1a garuba | ufurpare io non vegtio,

IMPIATT ARE, Naicondere , ¢ fi dice di materiali ; € non pare ches
fuonerebbe bene il dire Impiattare la verita , 4a virth , ec. Vedi fopra C, 1. ftan,
75 +41 Poeta fene fervetotto C. 19. ftan. 5. parlando dell'Aurora ; ma la confide-
ra. come donna 5 ¢: corporea , come fi confidera il Sole , la Luna »tle Stelle,
delle-quali fi dice Lmpiatcarfe 9 0 rimpiattarf: dictro a i nugoli , o dietro le monta-
ce ‘ f oo lei-non fering: che s appiatta ,¢ fugge .

dir la Tayola , fopra alla quale fi pofano le vivande per man-

giare: ae bene:Banco ha: molti altri fignificati .

0. dO thringraxierei , vanon occorre ; Cirimonia che fi ufa con.chi ci habbia fatto
sun favore a rovefcin,.o vero ce I habbia fatto quando noy occorreva,o quando
hawevamo’gia fattoda per'noi quel-che {peravams da lui; 0 che difua cortefia ci

faccia un Tavares del quale non havevamo bifogno ; ed ¢ lo fieflo che dire 4 "ho

smegli orecchi , ee ¢ fimnili 5

40F2 Porco maichio fenza caftrare. Dal Latino verres.

TV has Pree ero, elaine stete E — hogei poco ufato fy che il

t orstm’c
BARE wpa vefeia “og conchiudere Non ‘adempire il Yao intento’, come.
aoe diquella i Sten fm r Sees mettono nella canna minor
gu richieda , ¢ fearicando poi non ono , ¢ fantio
uno feoppio. ychea pena 6 fente, ¢ tale sopeenen rd Si dite
ancora ve/ia una:fpecie di Sr B ve/cie dicono le donne un racconto de fatti
‘ ¢ velciai donna > che ridice tutto quello che fate

cor 2
“* 5:0 Devine pi ov gibi ion “dpgtungere ;'¢ non levare . Cio’ farai trattato
‘cepa eae oe cae E
e 2iath ie Mian ia woo

Tinie, edi walense pais, e io ch’ il male.
(os hggigamels diag cot, ‘econ ogni cofa ; ed i Posta mete

y

   

 
114 MALMANTILE;

mo lo dichiara ,dicendo : come um beccafico , i quali uccelletti da i pil fi mangiano
fenza buttar via  ofla .. E /eranocchiare fe ben s' ula alle volte ne i cafi come il
prefente , non lo trovo ufato fe non per efprimere il romore:, che fa coi denti in
romper quell’ offa colui che le mangia , il qual romore é fimile a quello.che fa il
ranocchio quando canta .

HEBBI un certo feopo, Hebbi un certo fine , un certo genio , un certo riguar-
do» La voce /copo vien dal Greco /copos , che tanto appreflo a Greci quanto ai
Latini , ed apprefio a noi vuol dir Berzaglio , ¢ ggr metafora fignifica quel fine ,
al quale tende , ed ¢ diretta la noftra mente nelle noftre operazioni , per lo pil
in bene ; che non flimerei fi potefle dire fenza riprenfione . Scopo di rubare . Si
dice anche baver mira , il qual termine ¢ per avventura pil generico , dicendofi
haver mira di far bene ,¢d baver mira di far male.

METTERE a filo, Bar venir gran voglia , Traslato dal coltello, ed altri ferri
tagiienti , i quali quando fono ben’ arruotati ( che fidice meffi in filo , 0 affilati)
tagliano meglio .

DAR (a Sretta , Vuol dire opprimere uno. Ma qui aia nel fuo vero fi-
gnificato di ftringere , ed intende ftringere co i denti , ci i

mangiare .
STANZA LXIV.
Cost fpogkollo tutto ignudo nato , Lo racchiufe , ¢ lo tenne foggiornato 5
E veduto ch’ egli era una fegrenna , Perch’ ei facefe un po miglior corenna,
Adeit afciutto ,¢ ben condixionato, Perd che a guifa pos di mettiloro
Snello, lefto, ¢ leggier com’ una penna, Voleva dar di Zanna al {uo lavoro,

L’ Orco {poglid Floriano per mangiarfelo , ¢ vedutolo cosi magro rifolvé di
‘non toctarlo , ma lafciarlo flare tanto che ingraflafle ye poi mangiarfelo. 4

JGNVDO rato , Cioe ignudo , come quando ei nacque. Diciamo cost per in-
tender uno, che non habbia in doffo ne pure una minima parte di veftimento , ed
ha la fiefla forza che dire Zenudoignude,, che per la ragione della replica’, vuol
dire Ignudidimo , ‘0 Affatto igaudo .

SEGRENNA . Quella voce , ufata per lo pilt dalle donnicciuole , vale per

‘efprimere una perfona magra,{paruta ,¢ di non buon colore, che i Latini, tol- _

to dai Greco, dicono Afonogrammus ; ed il Pocta medefimo la dichiara dicendo ;
Tdeft afciutto , che bxomo afciutto intendiamo huomo magro ; ond’ io mi credo che
JSegrenna veaga da fegaligno che yuol dire Animale magro edi tempéramento non
atto a ingratlare. Diciamo ancora mummia , che fono quei Cadaveri fecchi nel
mare.d’ Etiopia , 0 ne i fepolcri dell’ Egitto: come vedremo forto C. 6. flan. 52,
per intendere Huomo foverchiamente magro. Diciamo Segrenna a una donnas
‘magra , difpettola , maligna , incontentabile , ¢ che non approva , ne loda: mai
P operazione @’ alerui inv

fs delkg Sid RO NGS THINY'D a : ‘ 2
BEN condizsonato, Quefto termine , fe ben pare riempitura del verfo.y-0( ¢o-
me diciamo )borra , non € cosi ma ¢ pure che quando fi vuole intendér un ma-
gre » habbiamo quefto dettato vulgatifiimo /cintto ye ben condizionato; xolto for-
fe da quello che Ernie denne a >
eben condizionata , per-avvifare il Corrifpondente della diligenza de) Latore , o
‘Condottticro . . vba Sate

SNELLO leitosleggier come wna penna Quefte tre yori nel prefente luego Fono f-
ri nonimi

 

ie

 
SECONDO CANTARE: 4s

nonime fignificando , ed efprimendo tutte 1a poca carne che haveva addoffo Plo-
riano , ¢ che era al maggior fegno magro. & la voce /ve//aha forfe origine dal
Tedefco Skye! , che vuol dir Veloce .

ZO tenne foggiornaco. Lo trattava bene di mangiare . Gli faceva buone pele .
Che /oggiornare uno yuol dire Spender il tempovin ben cultodire , governare , eo
riftorare uno con quello che occorra’, ¢ s' ula quefto termine per lo pil , trattan-
dofi di beftiami , ¢ percid appropriatamente detto in quelto Juogo , perché , fe
ben Floriano era huomo , era gpndimeno trattato dali’ Orco come beitia da in-
grafface.

F ACESSE miglior coteyna. \ngraflafic. Per intendere uno affai graffo diciamo :
Egli ha buona cotenna ; trasiato da i porci , la pelle de i quali fi dice propriamenic
cotenna , che dell huomo fi dice corenna folamente la pelle del capo 5:0 per di-
{prezzo , ¢ per intendere un’ huomo Zotico , che fi dice buomo:digrofsa corenna ,
o Cotennone , 0 Coticone,

AAGVIS A di mettiloro , Volea dar di zanna al fuo lavoro, Coloro che indorano i
legnami fi chiamano Azeri 1 ore, ed in una parola {ola Azettilori,, Quelti per
brunire , o dar il Iuftro a i loro lavori fi fervono de identi pil lunghi ,0 diciamo
maettre di cane, di lupo,o d’ altro animale fimile , (i quali denti chiamiamo <az-
ne , 0 fanne come vedremo fotto C, 7. flan, 54. ):¢ tal lavorare dicono xannare ,o
dar di zanna, Ma qui dar di cannas' intende il naturale adoperar de i denti, che &
mangiare ; ¢ (cherzando con I equivoco dice che l' Orca voleva dar di anna al
{uo lavoro , cioe mangiarfi Floriano,che era il {uo lavoroy che egli havea fatto pi:
Bliandolo , ed ingraflandolo .

STANZA. LXV.

STANZA LXV}1.
— Amadigi c andava per diporto

Due volte il giorno almeno a rivedere

E piangendo diceva; O.T ato mio,
Se tu muori , che ver [ata par troppo 5

La fonte , ¢ la mortella., che nell orto S' hava dire anche dimes tele dich iog

Lafcio Florian per tante [we preghiere; Ttibus , come difse BP... . Pioppo,
Trovato il cefto {pelacchiato,e¢ {morta y Cosh, fenza.dir pure al Padre addio ,
. El acque baffe purzolenti ,e nere Adonta four’ un cavalo , ¢ di galoppo

Qui(dice)Fratel mio noi fiam ful curra

» Diandar a far un balloincapo azzurra,

Vici & Vanano molto ben' armato ,
E feca.un cane alany havea fatata .

In quefto tempo Amadigi s’ accorfe dalla fonte , ¢ dalla mortella , che Floria-

cane incantato., and6.a gercar di lui,

he

ny ipa ,
a no era in pericolo , € percié montato a cavallo. bene armato , ¢ con un groflo
os

si partend. Spelacchis

SPELACCHIATQ , Pelato in

BP qua ,¢ in la, cioz parte delle faglic cafcate , ¢
s’ intende un’ huomo ’ che ftia male a fanita, ed a roba oe

fia mal veftito per la {ua poyerta . 4
(oe SMORTO, S' intende che nonha i! fuo natural colore buono .
‘ah _ SLA ful curro, Siamo in procinto ; fiamo all’ ordine ; fiamo vicini , Cxrro
yo = fon pezzi di quali G metton fotto alle pietre,o ad altre cofe gravi per
rm facilitargli il moto quando fi frafcicano , dai Latin detti Palange .
rey EAR un ballo ix ¢

‘ ’ azzurro. Vuol dire Effer’ impiceato ; perché campo az.

i Rurros’ intende il campo , che fa I’ aria , il quale ¢ azzurro , ¢ colui, che ¢ im-

0 f j ‘Piscato movendo le gambe , pare she palit in aria, Per maggiore inteliigen2a Ja
dais 2

voce
1

 
|

 

116 MALMANTILE

voce campo pieeeluiemencients 5 wuol dire quel luogo , che avanza in. uns
qvadro fuori delle figure , ed-altra che. vi fia'dipinto » come fi dice una infegna»:
entrovi un lione in campo azzurro. Ed i medefimi Pittori ne cavano’ il verbo
campire , ché yuol dire Dare il colore , de) quale:ha da efiere il campo.

7 ATO. Vuol di Fratello, B’ parola ulata dalle Balie per infegnar parlare a i
Bambini , come Babbo.in vece di Padre , Mamma , Bombo , ¢ fimili, che per ef
fer parole labiali tornano pit facili a proferirfi. Furono ufate anche dai Latini
come fi vedein Marz.lib, £. 95.

Aammas , atque tatas habet Aphra , fea ipa tatarnm
Diti, & mammarnm maxima mamma poteft .

Vedi forto C. 3. flan. rz. , ¢ C. 4. flan, 5.

TK lodich'io. Vale per Te logiuro; Ti afficuro., Vedi Oraz. lib: 2, Ode 17.
dove parlando con Mecenate infermo , dice :

Ab te mex fi partem anima rapit
AMatiirior vis , quid moror alters?

Con quel.che fegue fimile al prefente lamento , che fa Amadigi per if Fratello,
che: Orazio fayper Mecenate:.

1T [BV S-come diffe P ..«. Pioppo, Significa sha dire anche dime : gli mor-
to. Quefto P..... Pioppo.era ‘uno , che havea poca amicizia con Prifciano., e
non oftaote fempre slatinava’, ¢ fra I’ altre quando voleva dire i) tale ¢ morto di-
ceva fibas , © intendeva Egli ¢ito. EB da quefto fuorderto diciamo Come diffe
P., 2. %Pioppo, B-s*intende il tale & morto ,. ost th

Dik’ addio, Intendiamo quel faluto,, che fi fa nel pigliar congedo,o licenziarfi
da uno., ed ¢ lo fteffo , cite i Latino Yale, ufato da noi ancora come dicemmo
fopra , e vedrémo forto-C, 6ttan. 18. ¥ ’

GALOPPO:, Corfo divcavalio ,ida i Latinhdetto'earfus gradarixs , che & ins
mezzo tra il trottare, ¢ il correre. Forfe meglio gualoppe fecondo Dante Inf.

Cant, 22, Bs
di rintoppo
A gli altri diffe a lui , fe tu ts cale e
Jo non tiverro dietro di guatoppo ,
CANE eAllano , Cane groffo per caccia da Cignali,e fimili animali feroci, ed &
maggiore , pil fiero, ¢ pir gagliardo del Mattino . re
STANZA LKV

 

It STANZA oe
E cavalcando con la guida , ¢ foorta L'apparir a! Amadigi agti abit
Dil fue fadeie,eul hicescars lame j Raddoler? agro dei lor mefti- vif
~ Chinnanzi gli facea per la pit corta . Che per la fomigtianza atucti quanti
La fhrada per lo monte, e per lo piano; iapahio dapenaion @ Campi Elifi,
A Campi giunfe:, dove in su la porta 0 2 mance, © paraguantt
rapa Leggenidi, Stan's ‘ davon moltva darne al Re gli avvify 4

- Che perchi fucreduta dwognuno, —§ ——-—' Alrrs alia figlia, ed ambi a queft
KraleCeveyovthiie Cohpoabionss oPRercia Cae é
Amadigi ee » dove dal bruno , che vedde addoffo’ a gli abitatori

conobbe, che era mortoiitlor Principe ; fubito che coftoro veddero Amadigi ,

credettero ch’ i fulle Florianos e'peccid molti corfero a darne avyviloal Re, ¢

a Doralice. oie ERA

 

 
 

 

SECON DOCANTARE. fry

ERA laCorte ; ¢ ratroC ampica bruno. Cioé i Cortigiani, e gli abitanti di Cam-

i crano velliti di nero in: fegno di meflizia , per la morte del Re Floriana. Pecr.
4a ; E.vedrai nella morte de eAMariti
Tutte vefiite a brun le donne Perfe

© Da aleuni'fi dice wefire #turto y 0 a feorrucciv .\ Ma credo che effi habbiano ac-
¢actate quefte voci da i moderni Romani . t

AGRO dei lor mefti vifi . Viforagro vuol dir Malinconico; @ fi dice agro perch
ung, che habbia hauato qualche difgutto; fuob moftrarlo nelia faccia con incre:
fpav la fronte , ¢ fare:altri gefti appunto come fa uno, che mangi cole afpre- 5
acide ,oagre. E perd dice Raddoler ? agro dei lor mefti vift , che fignificadi me-
lancolici, gli fece ricornare-allegri:. Ad

CREDIT O «i Campi Elifi , Creduto nellvaltro mondo: ;¢reduto morto., che
eens Eiifi dalla fuperftiziofa Geatilita erano creduti-ii Paradifo.. Vedi foro

. 6. tam. 32. ‘

PARAGYANTO, Mancia ,o regalo. Paraguanto , dono, iregale , mancia ap-
pretio dinoi fi poffono dir finonimi ; E fe bene molti vogliono-che: manvia ye pa-
raguanto fi dica quello, che dal Superiore fi da all”inferiore; © donoieregalo G
dica quello , che dal’ inferiore fi da al fuperiore (che-in-quefto cafo now fi dircb-
be mancia ) 0 dali'uguale, all’ uguale, nondimeno nel buon parlar familiare fi pi-
glia uno per I altro, nes offerua tanta ftrettezza , ed il noftro Poeta pure fi
vede nel prefente luego , che non oficrua quefta diftinzione come poco , 0 punto

c Orin ta STANZA LHIX
Doralice brittande a tai-sovelle Enon fear pil nella pelle
<A rinfronzirfi ardoffene allo fpecchio , Salts fuor dipatarre innanxi al vecchio,

Sb mefse il grembinl bianco le pianclie Ed invontro correnda sil: fia cognate ;

UI veRxo ab collo,e i ciondoli all'erecchio, Ecco Florian ( dicea ) rifucirato ,

Dordlice fentieaquefta nuova fi raffazzond , ¢ {ubito-corfe incontro al {uo co-
gnato: Amadigi , credendolo Floriano fuo marito’.

BRILL ANDO. Giubbilando. ; Brille fi dice uno che fia allegro per haver beuuto
molto vino. Vedi foro C. 6. flan. 35. ed é il primo grado di briaco dicendofi in
agugumento Brivo scorto , briaco , fpolpato , Molti voghiono,, che quefta voce brilla~
re venga da‘biril/é tpecie di gioia’ , e che brillare fignifichi (cintillando-tremolare ,
appunto come fa il biril/e , ¢ come fanno coloro , che {ono fonmmamente allegri ,
©che habbiano foverchiamente beuuto . :

RINF RONZIRSI, Ratfazzonarfi , abbellirfi, aggiufarfi la perfona tolto dal
Latino refrondefeere , che vuol dir quando gli alberi fi veftono disnuove frondi,
Je quali nell’ antico Fior, fi dicevano fronze. Terenz. in Heaur.

: —— Et noffi mores mulierum ; s

¥

> % Dammolinntur ,@ comunrur , annus off,
+ Cioé fi rinfronzi(cono f dice P efpofitore Landino js" accomodano , ed accon-
iano la oom ee accu ae og Hh

~ CIONDOLI al? io, ini. le gioi portano
“denti all’ orecchic , Latino Zaawres Bis aor cama pendent per cere

ciondoli .

 

 

 
118 MALMANTIEE

VEZZO. Quel!’ ornamento di gioie , che le Donne portano alicollo

PIANELLE , Specie di {carpa , che cuopre folamente la parte dinanai del pie+
ce,da i Latini dette fandalia, E,con dette gioic adornandola,moftra il Poeta qua-
le pofla eflere una Regina di Campi’, che non eccede il Inflo d’ una pulita con-
tadina de i Contorni di Firenze .

NON pus frar nella pele. Non ped afpettare , perché l’allegrezza le ha:cagio-
nata una inquictudine tale , quale /ogliono havere tutti coloro, che dovendo con-
feguir qualcofa di lor guflo , ogni kbra d’ indugio. ftimano mille. A quefto
fi pud applicare quell’ 4 Sermento torus eff de i Latini, che pare che efpri-
ma quella inquictudine, che fuol cagionare l''ira ; Lafca Novella 5. Si che per la
paffione, e per la rabbia non poteva [rar nelle cuoia ,

COGN-ATO . | Latini per cognazione intendevano ogni forta di parentela.Ma
noi per cognaro intendiamo un Fratello di noflra moglie youn marito d! una fo-
rella di noftra mogli¢ , 0 un marito di notira Sorella , ¢ nello fteflo modo refpet=
tive i] Fratello del marito , fi dice cognato , come intende nel prefente luogo s

INNANZI al vecchio, Cioe prima che ulcifie di cafa il Re {uo padre , inten-
dendofi comunemente Padre quando in quefti termini fi dice il sage

 

taluolta il Padre fia giovane,
STANZA LXX.

Noi vi facevam morto; o gindicate
Selacarotac era fata fital
Pur noi ci rallegriam, che voi tornate
A confolar la voftra gent’ afflitta,
Domandar non vcorre come ftate 5
Perthé v’ havete buona foprafcritta,
E fiate graffo , ¢ tondo com xn parco
Per le carezxe fattevi dail’ Orca,

STANZA LXXL

AM immagine cost perch’ io non vera:
Tu fat com’ ell' ando, che fufti in cafo,
So ben, che mi dirai,che non fu vero
Ma la bugia ti corre fu pel nafo,
Hor bafta, Tx ritorni fano, ¢ intero,
(C’ a pexzi tu dovevi effer rimafo )
Per 1a Dio grazia ,¢ {ua particolare ,
Perche tel’ ha voiuta rifparmiare ,

STANZA

Mio padre te lo diffe fuor de denti,
Ed io pur te lo aiffi a buona cera
Lon una volta, ma diciotto , o-venti
Che l' Orce ti faria quatche billera ;

io,ancor che

STANZA LXXII,
Dunque s ei fa cosh gli é neceffaria,
Gh'ei non fia la quelfurboch'unlotiene,
Anzi tutto iLrevefcie , ed il contrario
Mentr’ egli tratta i foreftier fi bene g
(Ed io, che gid havea ful calendario,
Gli voglioinquato.ametuttoil miobene,
Perch'ei non t ingoio; Se ben da wn lato
Ti flava bene , havendola cereato ,
STANZA LXXIIL
Cast nel mezzo a tutta la pancaccia y
Ch'é quivi corfa,e forma un giro tonde,
La {ua caponeria gli batta in facia,
E quel ch'ei ne cavo po poi ingquelfonde
Gia che (dicea.) con I andar’ a caccia
Ai difperto.di tutto quanto il mondo
Cavafi,fenra far alcun guadagna
Die occhi ate,per trarne una alcopagno,
LXXIy.
Ma tu volefti fare a gli feredenti,
Perche te ye firuggei come la cera y
E quafi un rifchio tal fuffe una lappola
Voleffi andaxxi, ¢ defti nella trappola .

» In quefte cingue ottave moftra,, che Doralice ingannata dalla fomiglianza,

che haveva Amadigi con Floriano,gli fa un dicorfo di congratulazione mefeola-
ta con rimproveri , col quale il Pocta efprime aflai bene il coftume delle noftre»
Eemmine in fimili caf; tacendo che.dal principio del difcorfo , che ¢ 1a congra-

tula-

 

 
SECONDO CANTARE., 119

tulazione , lo tratti del Voi , ¢ quando viene a’ rimproveri lo tratti del Tu .
SE La carotac’ era fpata fitta . Ficcar carote yuol dire quand’ uno inueatando
qualche novella , 0 trovato ,lo racconta poi per non fuo,, accid che pil agevol-
mente gli fia creduto ; fiche Doralice vuol dire ; guardates' ella c’ era ftata data
a crédere. Vedi forto Can. 6. ftan, 67. ¢ 68. Mattio Franzefi nel Capitolo fopr’
alla Corte dice:
: ‘Chiama piantar carote il popolacciv
Quel che diciamo : Adofirar nero per bianco
Per diftrigarfi da quaiunque impaccio
E per tutto il medefimo Capitolo difcorrendo fopra quefto detto , moftra che
thabbiamo anche iliverbo Carerare 5 ¢ Carotiere , quello che ficca carote . LU Lalli
En. Tr. lib. 2. flan, 2.
Egli che ben conobbe al primo tratto
Ch! era in un campo da piantar carote
Si dice Piantar carote, perché quefta pianta fa grofla radice , ecrefce affai nei
terreni dolci , ¢ teneri , ed uno facile a credere fi dice Homo dolce ye tenero . :
VOL havete buona fopraferitra, La faccia fuol' efler dimoftratrice delle paffiont
interne , ¢ perd dicendofi haver buona fopra/critta's’ intende haxer biota fanitd ,co-
me dichiara il Poeta medefimo dicendo; Von occorre domandarni come voi feate , per
whi ficondfce dalla buona foprafcritra , cioe la fembianza , la buona cera , ¢d aria.
del. volto ci dice, che vai ftate bene . E cosila voce fopra/critra , che vuol dires
Infcrizione , che fi fa alle lettere , ci ferue per intender quanto fopra s’é dot-

to.
LA bugis vi corre fu pel nafo . Tu daicolore . Tu timuti-dicoloré in vifo, per-
ché tu hai -detto una falfita, Twi oculi declarant, Lo Scoliafte di Teocrito spic-
gando-quei verfi dell’ Iditio.12. che in Latino furono cosi tradotei : Verim ego te
« laudans yformofe ; baud mentiar umquam , Nec tenni gravis innafcetur puftulanari ;
sdice cosi,.Vuol dire ; che tiekdodarti , io non mentird ,. non mi nafcera fopra.,
al nafo Ja bugia ; poiché alcuni fogliono chiamare certe bollicine bianche, che»

vengono fu pel nafo', bugie: c\colui che leaveva , era natato , come bugiardo .
'Fin qui lo Scoliafte .

RISPARMIARE 4 0'ri[pinrmare, Vale:per petdonare.. Quis intende 1’ Orco
~ehe non ha voluto far male alcuno.. ‘
HAVER uno ful.calendario , Havere a noia , o'vero odiar’ uno .
QUANT O w me gli vo turto il miobene.. Pee quanto s’ afpetta ame gli porto
tutto quell’ affetto , che fi pud portare ; |’ amo di tutto cuore .
TI fava bene . E’ \o ftetio che Ti flava il dovere.. Tornava bene ,-che 1’ Orco
» t havefle ingoiato , perché t’ haverebbe fatto quello che tu meritavi .
PANC.ACCIA , Cost fi chiama da noi quel luogo dove fi ragunanoi novelli-
fi per darfirle nuove PunsVaitroyed ha quefto nome di Pancaecia, perche nel tem-
di Rate quefti rali fi radunavano gia per fentire il fre(cowicino alla Chielas
» Cattedrale , fedendo fopra aun muricciualo coperto di tavoloni , 0 panconi’, e>
«da quelti prefeil nome di-Puncaccia . Eda quefta pancaccia, Pancaccieriye Pancac-
vai intendiamo quei perdigiorni., che ftanno oziofamente ragionando de i -fatti
daltci , ed in quefto fenfo ¢ prefo nel prefente Iuogo , che dicendo quei della pan-

ALLIS y

 
120 MALMANTILE

caccia, intende una quantita di quefti Crocchioni.. Vedi fotto C.6: ftan. 69. Can-
ti Carna(cialefchi ,.@hi.vxol udir bugie , 0 movellacce Venga afcolar coffuro ; sare
Sf fian wntia it dd fs te pancacce ,

GOLA butrarin faccia La fua caponeria., Gli rimprovera la fua oftinazione »

VEL ch' ¢ ne cavd po poi in quel fonda, Quel ch’ ei guadagnd , pert ona
fine delle fini , 0 in ultimo degli ultimi . Tanto fervirebbe dir po: fenz’ aggiu-
gnerui ix quel fonda , ma cosi & il noftro coftume in fimili cafi per dar maggior
emfafi , quafi dica una fine pitt la delle fini , Vedi forgo C. 8. fan. 51.

CAV-AR due occhi a te per trarne uno al compagno,, Detto vulgatitiimo , che ci
ferve per e(primere Far 4 fe molto male , per farne  packiffime al nimica ,

FVOR de’ denti ,, Apertamente ; chiaramente¢ i! Lat. Eoqui , ed &il contrario
di parlar fra denti , o a mezza bocca , che fignifica non fi laiciaré intendere , for-
fe ¢ il Atuffitare de i Latini .

et BVONA cera. Con allegea faccia ; cioe non fopraffatto da collera ,o altra
paflione , ma con apimo ripolato ; diciamo anche ful fodo , ful ferie tolto par
Serio, admonere . li Lalli Eo. Te. C. 4. flan. 103.

Prega y Seonginra , ¢ dille a buona cera,

AILLERA, Burla-nociva ,o {non cattiva del tuto, aimeno che non piace;
voce corrotta da Wil/era voce antica che yuol dic Villania,, «2%

TE WE frnegei come ta cera. liverbo fruggerfi, che yuol dine Ligutart fer-
ve a noi per farfi dered’ uno che ard ae » Ue Lali
En, Tr. C,.4, fan, 109.-difle . 5 i d¢ Motsiss

Che fe ne firugge come le candele .

LAPPOLA, Cofada non timarf. L’erba da noftricontadini'chiamata Lap.
pola fa un feme picno d’acute fpine, ma fragili; EB perd dicendofi : nom do ftime tna
Lappola ,s’ intende non lo fimo punto ,¢ s’ ula per.lo pid trattandofi divbravura,
¢valore , alludendo a quell’ armatura di {pine y chehala La » le qualifes
pon fon saraee 89 aeety Sananne. se belesemme te ag

agilifime..

DEST I nella Trappola. V' incappatti , Vi rimaneRti gee Te dnquewtn inci
fi. Trappoja intendiamo ogni forte d’ antifizio , che fi trova per pigiiare ani.
mali tanto di terra , quanto d’ arias ed? acqua, donde 7'rappalare valyper Ingan-
nare. Ma 7 rappo/a ftrettamente prefa s’ intende un’ aria per per eerste
ed una esicdia rete da pefcare ha ae di 7

 

Su egsd Tr. da.quaterini y Pe Invenzioni:per e fare i
ANZA TXKV a STANZA L XVL_

ai eee 97 Ma perch’ ei non credea veder mai P hera
E “as il fordo ad: ogni fuo quefice » _ DY baver il.fuo Fratello a faluamento y
da fiben' attingea.da queftecofe >... Daun ganghere a tuctiy.e rorua fuora
price 4 Florian porea eer feguita ys -Dietra al {ua can veloce'come it vento;

immaginandofi es appofey  Neera un trar di mano andacaancera

vicemne Atoglic, ci [uo Mariea, ed caccia al! Orcoch' ei vi dertedrento
Bch’ egli effendo tutto lui maniate. .. .. _ «Come il Fratel vedendo un bebsignale,
Fulje pel (uo Fratel dacgnuncambiato, si fe wale quante lui dolce di fale.

ab b>

STAN-

“RMS i ie aa ae

q

 
yh
ity

 

—

SECONDO CANTARE: — 120
~~ ASTANZA LXXVIL STANZA LXXVIII.
Che fegnitollo anch' ei per quelle ftrade EiquandotOrco poi venue anc!ia lui
Dond’ ei conducel huomo allafuatana, et dar parole com quei tempi firani ,
Ove rmentre diluvia,edal Ciel cade > ‘Bd \allufeio fatea Pin da Montus
OB broda ye cect, Criftianello intana., Affin chet’ arme  ¢4 caniegli allonrani
OcEd-egli tanto pai lo perfuade Eidiffe: Suipiccin pighacald ,
oCw ef lege ivcani:, ¢ pif durlindana, E chiapparata [pada con dueimani
Ftavende havuto innanzi la lezione, Si lancto fara, equivi a pit non peffe
Si fvetcefempre mai fodo al macchione. Gli comiacio aimenarile man pel defo.
; a S THAW ZA LXXIX.
E mentre chor di punta, ed bor di tagtio \\ ~ Tal-che tatto forato come nn vaglio
Di gran finefire fa , di Linghe Witee. ; Ut power: Orco\al fin cade ,  bafifee ,
Pike preftc che non va firale aberzactio Edheraquelte raps , e quelle: macchie
~\Mbcan savicuta anch' eg lize ribadifce ; Rimafe a far banchetro alle Cornacchie,

Amadigi/argumentd dal difcorfo di Doralice’, che-ella fulle Mogli¢di Floria-
nO.,,.¢ compre/o quanto poteva efler' avyenvto al medefimo ; ¢ perd fenza dar al-
tra’ rifpofta dette addietro , ed ufcico di Campi ; fu'dal Cane guidato alla tana
dell’ Orco |, il.quaic.fa da Juiicon ajuto' del fuo cane’, ammazzato .

44.4f, Quefto avvcrbio che fignifica In alcun tempo ferve anche per negativa ,
come & nel prefenicJuogo 5 ¢ come I’ usd pity volte i] Boccaccio ed‘in fpecie Nov,

. Mai frate il Dinvel tivcirecay ccs B Nov. 54. Che mai ad animo-ripofato fi fareb~
gee witrevare yo Nov, 77. Adai di cid che hora mi parti dubitai, Matteo Villani
lib, 8. cap. 39. / Perugini mas fi vollero dichiarare , ed in molti altri luoghi de) Boc-

caccio , del Paflavanci 5 ¢ d’ aleri Scrittori del buon fecolo fi trova ulato per ne-
gativa . Ho voluto dir cid in quefto luogo per toccare‘la difefa dell’ Autore dalla
critica datagli d’ haver ufato quefta voce 44si per negativa fenza l”aggiunta della
particella we , © om, ¢ {enza correlazione alla negativa antepofta nel medefimo
periodo:, ¢ che tanto vale il dire /o now fard mai quefto , quanto il dire . Jo nai fas
xo.quefo, E.mi rimetto all’ ufo yed al TORTO,, E D/R/TTO del P. Bartoli,per
la difefa:di 88 a ee ai
i sFECE it fordo,..Finfe di-non fentire. {
WAT TINGEA da quefie cofe. MW verbo attingere 0 attignere , che & il Latin’
me eens un Juogo 5 0.4 un fine; Azeram attingere : da noi € preloy
ed ulato come il verbo aurio , che vuol dir Cavar V'aequa da i pozzi, che noi di-
ciamo attignere y ed in fignificato di Comprendere, vedere’, udire , ocults © auribus
haarire « EB nel figni di cen € prefo nel prefente luogo .
8’ APPOSE . Verborncutro che-val per indovinare: Ed attivo vuol dire Dar
la i Bund « Jom appofi di chi baveva fatto il male , e perd lappoft a lai. \o
a ist oe havea fatto il male , € pero ne diedi las
adei.> onc aa! o5 Yb chsh ee Y
 TVYTTO lui maniato, Come lui per appunto: Similifimo a ui: Facto a cay
pelle, che vedemmo (opra in quefto C, flan, 19. Lacy Nov. 7. dice: M1 qual fan-
taccio veftsto de’ panns del Pedacoga,tutto maniato parea Ini, lo credo che fia parola

Sorrotta da mizvaro cioé diligentemente dipinto , 0 forfe corrottamente derivato
dai Latino barbaro Emanaeus , tanto fimile a lui , che pare emanatus ab illo,
3 2 NON

 

ae ak ee
fe MALMANTILE

NON credea di veder mai ? hora. Amadigi havea cosi gran defiderio'di vedere
il {vo Fratello libero , che dubitava non fade per arrivar mai quell’ hora , ed ogni’
momento , gli pareva un’ anno, 2
' un ganghero, Da volta addietro . Ganghero diciamo uno ft per
ufo d’ affibbiar le velti , fatto di filo di ferro , 0 d' altro metallo , il quale € fatto
in forma d’ uncino, eda quella rivolta, che egli fa , dare il ganghero intendiamo
tornar indietro . ‘Retrorfum vela dare. Dare il Banghero » diciamo quando la lepre
fuggendo avanti al cane, torna indietro , ¢ lafcia correr il cane , che portato
dalla velocita non fi pud ritenere , ¢ yoltarfi fubito come fa efla , che in tanto pi-
glia campo in manicra ch’.ella (Campa , dal che diciamo Far lepre veechia per in-
tender tornat indietro. Vedi {orto C. 10. ftan. 23. '
NON fu si doice di fale. Non fusi credulo : Si minchione: Si fciocco. Viaas
vivanda poco falata fi dice dulce di fale , cioe {ciocca. Donde effer {enza fale , 0
non haver fale in zucea vuol dire Huomo {ciocco, fenza giudizio , fenza ceruel-
lo. Safechiamiamo I! arguzie , e detti ingegnofi. Vedi otto C, 8. flan. 26. Di-
ciamo if tate ¢ dolce ,¢ fenza I’ aggiunta di fale intendiamo é corrivo, creduloy
minchione , ¢ fenza giudizio ; ¢ per coprire pit quefto detto , ufano molti dire»
Lupinaio (che vuol dir colui che vendendo per Firenze Lupini va gridando dolcé
dolci ) per intendere Cofui ¢ dolce, Qui dunque vuol dire,che Amadigi non fu
corrivo quantojera ftato il Fratello a credere all’ Orco. Boce.Gior. 4. ns 2, A444.
donna Zucca al vento , la quale era anzi che nd a. dolce di fale, Latca Nov. 2,
Experch? egli era nato in Domenica , non fendo la gabelia det fale aperta§ senne fempres
molto bene del dolce.
TANA, Caverna, grotta, buca. Donde intanare , entrar nella tana,
BRODA ,ececi. lntendi acqua ye gragouola. Fu un ragazzo ghiotto*delle
civaic , per il quale fo padre ( per mortificare geen fua gola ) ordind, che nel-
la fua feodella non fi metteffe altro , che il puro brodo de'ceci , o d’altre civaie
refpettivamente , onde il peer ragazzo vedendo gli altri con le f{codelle piene»
di legumi fi difperava,, Ed eflendofene andato un giorno in camera mentre pio-
veva fe ne fava alla finettra gridando acqua , e gragnuola , € quetto per la ia,
che haveva , che fi ftagionatlero i legumi per gli altri, enon per lui. Senti il
padre quefto uo gridare,che gli diffe: perche preghi il Cielo a mandar Ia grandi-
ne , cola tanto nociva? L'aftuto ragazzo per {campar la furia fubito rifpofe:
Padre mio io non ho mai defiderato , 0 pregato male per nefluno , ¢ fe io prega+
vo che infieme con I acqua veniffe anche della grandine, ho voluto intendere,che
il Ciclo vi metteffe una volta in tefta di farmi dare con tanta broda una voltas
anche de’\ceci , che dixquefti intendevo quando dicevo gragnuola. Ii Padre rife
dell’ aftuzia 5. dette ordine , che per 1’ avvenire fuffe trattato. , come gli altri.
E da quefto,intendiamo ai » ¢ gragnuola , quando diciamo broda , e cect.
CRISTIANELLO . E’ detto d’ avvilimento , ¢ fignifica Huomo dappoco,0 di
‘a fortuna 5 © di piccola figura ; che i Latini dicono bomuncio , ¢ nob talvolta
in queflo fenlo diciamo Homicciuolo. ae
DIRIND-ANA , Intende la {pada,e piglia quefta denominazione dalla famo-
{a fpada d’ Orlando Pajadino , la quale da i Poeti hebbe il nome di Durlindana ,
0 Duyindana y ‘
W #iA-

 

 

 
  

 

 

SECONDO CANTARE, 123
4 HAVENDO havuto innanzi la lexione , Effendo ftato prima informato ; ayvila-
t0,, inftruito.: Cioé havendo comprefo dal di(corfo di Daralice-, che quefto cra
quell’ Orco , che ingannava . 3

STAR fodo al Macchione . Intendiamo non condefcendcre alle richieRe , o'non
Gi lafciar lufingare dall’ efortazioni di alcuno . Quefto detto viene da quegli-uc-
celletti , che ftanno per le ma¢chie , dove fi tendono le ragne , i quali , per eflere
Mati altre volte molefati , hanno imparato , che quello (cacciargii col battere Ja
macchia era di lor poco danno flando fermi, perd ncn fi muovono a ogni romo-
re,¢ quefti fi dicono Par fodi al Afactbione, Di tali uccelli fi dice anche accivereati,
Veditotto C, 9, ftan. 22,

FACEA Pin da Montui, Cioe facea capolino , che yuol dir quel che accen-
sd fopra C, 1, flan, 7. Quefto detto viene da una canzonetta , 0 villanelia ,
che dice. A

Pin da Montui , Fa capolino
Dreto é Menyhino , E Mon con lus , ec,
Plauto difle Ex infidijs clanculum ancupari .

SP-piccino . B* modo di incitare il cane contro a uno, El irritare, oimmittere
de i Latini, che noi diciamo anche ammetcere. Vedi (otto C, 11. flan. 29. fi di-
ee anche ai/sare verbo originato da quel fyono , che fa la voce dicendofi: /u /a ; 0
dalla parola Ra voce antica » che vuol dire Ira , dalla quale habbiamo il verbo
aizzare 50 adizzare, 0 aiffare, Dan, Inf. C.27,

Dicendo , fa ten va : pitt non? aizxo,

A PI non pofa . Con ogni maggior potere ; Quafi dica con animo di feguita-
rea far quella tal cofa fino ache non {ara ftanco , € non poffa pil.

MEN AR Je man pel deffo, Adoperar le mani nella perfona d’ uno , cioe-Per-
quoterlo. La voce dof dal Latino dor/um , da aoi s' intende per tuto il torfo
dell’ huomo , ches’ eccettuino da molti il capo , le braccia , ¢ le gambe,
Lafca lib. 1. Nov..7. Non contento di ricercargli col baftene le braccia y ¢ le gambes',
volle ancora con effo ritrovargli tutto il doffo ,

GRAN fineltre , ¢ lunghe frrifee . Gran ferite di punta edi taglio Punitim , C-
tefim , difle Vegezio . Dice frri/ce per la fimilitudine che ha una hunga ferita di
taglio con la ftri(cia , ¢ lo fa per efprimere che eran ben lunghe » come dice fize-

requelle di punta ppctche s'intenda, che eran larghe . 3
CAVVENT ARS! , Spingerti , gettarfi ,o andar velocemente » 0 con impeto
alla volta d* uno , che i Latini dicono irruere .

R16 ADIRE, RibattereQuando fi mette un chiodo dentro a una tavola,¢ che la
punta di cflo chiodo pafla dall’ altra parte,la detta punta fi piega, ¢ fi riconficcas
perce il chiodo facia V effetto d’ una Jegawura ; ¢ per far Qacfio , uno baie ins

u la punta del chiodo 5 eV’ altro tiene a rifcontro in ful capo'delchiodo un fer~
£0 5 € quelto fi dice ribadire ; e perd Amadigi da una parte , ¢ il ca~
ne mordendo dallt altra !’ Autore per efprimer quefto atto fi ferve del verbo yba-
dive ufato da molti ed in quefti termini , ed anche per licare ,

FORATO come un vaglio. Havevano fatto nella perfona dell’ Orco pili buchi,
€ tagli che non ha un vaglio, firu le fi fepara il grano dall’ immon-

: in vaglio, col
dizi¢ , decto dal Latino Yannns + etal Cavell dal Latin Crsiram ye rie
2 lum

 

 
¥G

  

124 MALMANTILE

Jum , voce ufata dall' Agricoltore Palladio:, Quefta-comparazione’ era: ufitaan.
che da i Latini trovandofi in Plauto Carnificum cribrumspariando diua fetvo, che
era Nato mal concio dalle baftonate .

BASISCE . Muore. Quefto verbovha forfel’ origine dalla Greca voce Bafis =
che vuol dire incefxs , ¢ che intendiamoiil rale fe n’ andé , peril tale mori, che di-
ciamo basi; vedi.l'Orcava 82, feguente ,¢ da quefto verbo deriva la voce basto,che
vuol dir huomo fenza fentimento.,.¢ quafi morto. Meller Gio; della CaQnel

Capitolo del Martello:d' Amore ‘dice.

Perché ti guardi torto‘la Signora’; t
Parti haver le budelia in un catino ,
E doventi bafiro alloraallora .

Vedi fotto C. 6. fan. 97.
STANZA LXXX

Amadigi dipoi fece pulito ,

Perché trovato havendoil {uo Fratello

Con una barba lunge da Romito ,

E pitt lordo, epiit unto d'un panello ,

Lavatolo , e rimeffugli il veftito ,

Ch’ era ancor quivituttoin unfardello,

Lo ricondufse a Campi, ove la Adoglie

Di lui gia pregna,appunto waren lee Je doglie.

Ata prefto come luiporrai dir mio.
Hor fenti pur: Bafite Periine
eAnco eAmadizi fubito tuo Ziv

Vine ator donaye n' hebbe ut bel garzone,

NZA

STANZA LXXXI
Corfe la Levatrice , ed in effetro
\Framille boime,fe' foldi,e doglien hora,
Partorigli una bella pi/ciallerto
Che fufti tu , poi'derta Celidora ,
E maritaca al Re , come s' ¢ detto',
Di Maimantil del qual tufei Signora;
Ne fei , e ne farai, io lo raffibbio,
‘Sebennen puoi per bor dircomeitmibbio,
LXXX rs :
Che Baldo fu chiamato je quel fon’ io ,
Che poi crefeiuto detto fon Baldone '
Hor eccotisdal primoval terzo grade
Narrato tutto il noftro parentado .

Amadigi trovato: il Fratello Floriano:lo: rivefth; ¢ lo-riconduffe a Canglt dove
Doralice partori Celidora; ¢ d’ Amadigi nacque Baldone.. E. con. terminare il
racconto,termina il Poeta il fecondo Cantare's

FECE pulizo, Feceil negozio aggiuftatamente, come: andavSoihadls

BARBA da romita, Barba lunga,¢ incalta:, che-tale perio pu eer

barba:de i Romiti.,

LORDO . Sudicio Ihifo. Dal tno Lari sche wiol dir Livido 2 quali’ ‘per

lorum cuffum , © livid E

Be P
mo , ma ancora ad ogni pit 7o ftcumento,fopra il q
chiamiamo un:

PANELLO,. Cosi

nati da iwenti,a! squetti refiftano,
flor. lib. eee

parturiente da i Latini'd
HOLME . Voce’ che efprime:
diceyano bei mibi ‘ye noi forte

‘habbiamardal-Greco hoi moi
SG [aldi 5 ¢ dogtion’ bara pot. aS 2 burlare chi‘taiuolew : oe

E... Raccogliteice;. sams ooh, ries! la. Creatura dalla

 

‘alPhuo-
quale fia {chifezza.

viluppo'di cenci intinti-nell' olio'; fego i °
altra materia oleacea , ¢: bituminofa. gail.
narie in occafione webises iene

tron Tome ergy
eminent? , ¢ domi-
hin i val lo fteflo. Vatchi

-difétto'd? clio’.

 

‘animo. ype di a che i
E quell’

  

 

 

 
SECONDO CANTARE. 125

firathihiirica , © farlezzj fenza cagione yo per dolori leggieri , che noivdiciam> ,

Fareil monelio’, enone riempituca intentata dal/Poeta , ma é pur cos! in ufo, di-
cendofi a quefto tale: O pover’ huomo! dime | fei foldi , ¢ dogliene bora ; ¢ fi no~
shimano Toth di monete.per haver occatione di dire dagtiene, che & il verbo
dare , ed in quefta occafione fi dice , perché ha fimilitudine con la voce doglia .

PISCIALLEFTO } Wnabathbina’, Quando una donne partorifce una Fem-
mina, nitda diguelle donne che fono attorno alla. parturlenterlé yuo) dar las
nuova , che ella fia femmina, ma perché pure al fine ella lo deve fapere, per non
profferire la parola femmina dicono : Vaaipifcialletro ;-Vnacome'me ye finili, E
da quefto’ noi habbiamo far” ua bambina , che vuol dir Fare un’ errore.

LO rafibbio . Lo replico .

NON puoi dir come it nibbio, Ciot non puoi dir Mio. Il Nibbio uccello rapace
non fa altro canto , ne fi fente da lui altra voce, che ua certo filchio gio ftrido 5
sche par che {uoni mio mio, ¢ da quefto per avventura i Latini lo diconAdiluus,o1i

gauolieA4ilano , ¢ i Francefi Azilan ; E noi da quelta fua voce volendo efpri-
mere,che una cofa fia’ veramente mia,dichiamo: Poffo dire come il nibbia,cioe Mio;
P autoré Jo dichiara nel primo verfo dell’ ottava feguente dicendo > eta prefte
come lui potrai dir mio,

BASITO , Vedi! ottava 79, antecedente .

Z/O . Fratello del padre , o della madre , o-matito'd’ una forella del padre , 0
della madre: ‘Quié fratello del. padre’.

VN betgarzone . Cioeun figiiuol mafchio , Equi il Poeta feguita.a moftrare if
coftume delle noftre donne accennato nell’ ottava antecedente, che quando il par-
to édi ma(chio joghunia di loro vorrebbe effer la prima a: darne lal nfova , es
dannosalia:creatura fempre qualche epiteto,, come ma bel garzone.s ninbel giovane,
un.garbate fantoccione , um bamboccione a’ importanza. Vedi fopra in quefto C. ftan.
491 ma quandoé fematina-, tutte: le afiiftenti ammutolifcono , 0 quando pur’ al
t ine lovdicano®, dannovalla creatura epiteti d' avvilimento , come 2i/ciallerto , Pi

{ciatcbera', zm ¢uaintuccia,, © fimili , come habbiamo detto poco.fopra .
« 4B noftroparemeado pla poltca Genealogia : In che modo noi fiamo parenti «

FINE DEL SECONDOGANTARE.

sod opita 4

~ -RS

aa.

i

  
 

INARA TA
Ce Paes aetealaet
TERZO CANTARE,

al 70.
‘22 ARGOMENTO. §
5 Vengon d Arno 2 [econda i legni Sardi, ?
Sharcan le genti, e vanno a Malmantile ,
& she
ote Nafcon grandi fcompigli in quella piazza,
e335 E ognun fi fugge in veder Adartinazza,
cuve SPL Cte ws cure
PRE IAT

Ma per vari accidenti i pit gagliards
STANZA TL STANZA-IL

  

   

Non fan quel tanto, che di guerra é file .
Arma i fuoi Bertinelia , alza flendardi ,
E moftra in debol corpo alma virile, R

Nhe fia avoerxo a frarfeneafedere E pur chi vive y fta fempre to

| V Senzafar nullacon le mani in mano, va ber a Lincs ’
E lantamente puo mangiare , ¢ bere, Perché al Mondo non é nulla di netto 5
E in fofha ,e giuoco viver litte, ¢ fano , E non fi puo mangiar boccone in pace 5
Se gli fon rotte nova nel paniere , Hor ne vedremo in Malmantil Veffetto,
Confiderate fe gli pare firano 5 Che immer{o nei piacer vivendo abrace,
Ed to lo credo; ¢’ a un’ affronto tale Non penfa che patir ne dee la pena,
e4 certo ognun P intenderebbe male. E che fra poco s' ba mutare feena ,

Ii Poeta volendo trattare dell’ aflalto dato a Malmantile , ¢ del difturbo , che
@ per apportare I’ elercito di Baldone a quelli fpenfierati , che fono nelia Terra,
introduce il prefente Cantare con nna reflefione 5 che fia un gran difturbo a co-

loro , i quali ftandofene coi loro

commodi , ¢ fenza un minimo penficro, fi v

gano (opraggingnere chi gli privi di.
rebbono di gran difguito , pelt an

ucfti loro agi ; mentre fimili accidenti fa-
a coloro , che\non fteficro con tutti i lor

commodi ; perché niuno , 0 bene ,
tutti fiamo fottopofti alle dilgrazic

omale , che gli ftia, vuol mai ricordarGi, che
,¢che nel mondo non fi aa felicica perfecra .

ST ARSENE con le mani in mano, A cintola , 0 in feno . Si dice d’ uno, ches
fia tutto dato in preda all’ ozio , ed alla poltroneria , ¢ che non vuol Jayorare .
Viv accidiofo , nighittofo , 0 {cioperato , 1 Greci , ¢ Latini ditlero: Jn choenices
fearre : de bomine ociofo, & defidiofo. ‘3

GVAST AR ? wove nel paniere , Guaftare i difegni altrui, Traslato . guaftar

a are y nuova

in we

 

 
TERZO\CANTARE.- 127

Ituova nel nidio.; dove fon dalla chioccia covate. Vedi Efopo Favola dell'Aqui-
la; edello Scarafaggio . E il conatum frangere de i Latini .

SE gli pare frrano. Se gli par.duro’, ¢ difficile a foffrire . Vedi fopra Cant, 2.
ftan,'21. 5 ed il proprio ¢diffrano. Stravagante , o foreftiero 5 o non
del noftro parentado ; valendocene in tutti quefti , ed altri fignificati, come fegue

| ne i Latini della voce extraneus ,

; AFFRONT O. Significa Aggreffione , affalto,, abbaccamento .. Vedi fopras
Cant,1. ftan.29. ma fi piglia: ancora per Soprufe , come ¢ prefo nel prefeate»
luogo. ‘

BERE unafciroppo , che difpiaccia, Sopportar per forza una cofa , che fia di
difgutto 5 che in Latino: fi difle: Calicem bibere ; perché Calix era una {pecie di

) bicchiere , col quale gli antichi beveyano caldo , come appunto fi bevono gli {ci-

{ roppi; ¢ lofacevano ancor’ effi per: medicamento ; ¢ per confeguenza era tal be-
+yanda’, come.a noi , per lo pit , di poco gufto ,
° WEL eAiindo non é nulla ds netto . 11 Mondo non ha felicita perfetta. Vricuigue
dedit vitium natura creato ,

VIVER a brace. Viver’acafo , fenza regola , o confiderazione . Ha forfe-

efto. detto origine dalla mifura , che fi fa della brace , che per effer cofa vile, ¢

i poco prezzo fi mifura inconfideratamente fenza guardarea darne un poca pill,

©um\poca meno. Da quefto poi habbiamo /braciare veduto fopra Cant. 2, ftan.

10, che fignifica Confumare il fuo inconfideratamente .

MVT ARE foena, Mutar faccia , 0 ftato , mutar maniera di vivere , Traslato

 

 

dalle profp 3 dove fi le die » quali prolp fono da noi
vu Igarmente chiamate Scene .
‘ STANZA IIL, STANZAUIV.
Era in quei tempi la , quando i Geloni Quand in rerra  armata con la foorta
Tornano a chinder I’ offerie de’ cani , Del gran Baldone a Adalmantilsinuist, :
‘E talun , che fi [paccia i mitiioni Ond’ un famiglio nel ferrar la porta
~ | Manda al prefta il tabi pe panni lani; Senti rumoreggiar tanta genia .
Ed era appunto l oraych’ i Crocchioni Vn vecchiocraqueft hnom di viffa corta,
Si calano ull affedio de’ caldani ; Che L erre ogni bor perdeva all offeria,
» Ed efcon con le canne,e co! i randelli _. Tal che trail bere , el effer ben a! eta
* Tragaxzi a pigliare 5 pipiftrelli . Won ci vedeva pik da terza in la. :
_Delcrive la flagione , che correva ; quando la foldatefca sbarcd in terra, ¢ s'av q

 

vid verfo Malmantile foto la condotta di Baldone ; e dice che era ful finire dell’
Autunno , poiché cominciava a diacciare , ed i ricchi finti mandavano a impe-
gnare i veltiti da ftate per rifquoter quelli da inuerno ; coftume affai ufato da co-
loro , che sfoggiano in veftire quantunque fieno poveriffimi , ¢ quefti intendi
vicchi finti , che fi pacciano i milliont »che fi fuol dire ; Adexzettin non rifquate Pan-
talone , cs’ intende , che gli abiti da ftate non vagliono tanto, che impegnandoli
poflano rifquotere quei da inucrno’, come appunto é }abito povero di Mezzetti-
no feruo fciocco in commedia,¢ I’abito ricco di Pantalone vecchio in Commedia.
Narra parimente I’ hora appunto che era , quando coftoro s'accoftarono'a Mal-
mantile , ¢ dice , che fu fa l’ annortare,che € quell’ ora , fa la quale i Crocchioni
fi mettono nelle botteghe intorno a un caldano per paflar la veglia. In tale fla-

gione

  
 

we
o

se BE

 

TERZO CANTARE: “ag
Fiiova nel nidio, dove fori dalla chioctia covate. Vedi Bfopo Favola dell'Aqui-
: dello faggio . E i} conatum frangere de i Latini .

 

- SE gli pare firano. Se gli par duro, ¢ difficile a {offtire. Vedi fopra Cant. 2.
flan. 1 eli propdo i ‘ato ¢ difrano, Stravagante , 0 foreficra ,Onon

ado ; valendocene in cutti quefti , ed altri fignificati, come fegue
ne i Latini della voce extraneus .

AFFRONTO , Significa Aggreffione , aflalto , abboccamento , Vedi fopra.

Cant. 1. flan, 29. ma fi piglia ancora per Sopru/o , come é prefo nel prefente>
Iuogo.
NRERE uno » che difpiaccia . Sopportar per forza una cola , che fia di
difgutto , che in Latino fi difle: Calicem bibere ; perché Calix era una fpecic di
bicchiere , col quale gli antichi bevevano caldo , come appunto fi bevono gli fci-
roppi ; ¢ lo facevano ancor’ effi per medicamento ; ¢ per confeguenza era tal be-
vanda , come a noi , per lo pil , di poco gulto,

WEL eMondo non é nulla ds netto, 11 Mondo non ha felicita perfetta. Yricuize
dedit vitium natura creato , e

VIVER a brace. Viver’acalo, fenzaregola, 0 confiderazione . Ha forfe-
we detto a dalla mifura , che fi fa della brace , che per eter cofa vile, ¢

i poco prezzo fi mifura inconfideratamente fenza guardare a darne un poca pilt,
© un poca meno. Da quefto poi habbiamo /lraciare yeduto fopra Cant, 2. ftan.
10 , che fignifica Confumare il fuo inconfideratamente . :

MVT ARE feena . Mutar facia , o ftato , mutar maniera di vivere . Traslato

dalle profpettive , dove fi recitano le commedie , quali profpettive fono da noi
vu Igarmente chiamate Scene .

TANZA IIL STANZA IV.

Era in quei tempi la quando i Gelont Quand in terra armata con la feorta

Tornano a chinder I offerie de’ cani, Del gran Baldone a Malmantilsinuit,
Etalun , che fi (paccia i millions Ond’ un famiglio nel ferrar la porta
Mazda al prefta il rabi pe panni lani; Senti rumoreggiar tanta genia ,
Bd era appunto l orach' i Crocchioni Vn vecchiocra queft huors di vifta corta,
Si calano all affedio de caldani ; Che I’ erre ogni bor perdeva all offeria,
Ed tfcon con le canne,e co’ i randelli Tal che trail bere, el’ effer ben d’ eta
Tragazzia Piglares pipifPrells A Von ci vedeva pile da terza in la,

De(crive 1a flagione , che correva ; quando la foldatefca sbarcd in terrae s'av-
vid verfo Malmantile fotto la condotta di Baldone ; € dice.che era ful finire dell’
Autunno , poiché cominciava a diacciare , ed i ricchi finti mandavano a impe-
gnare i veltiti da (tare per rifquoter quelli da inuerno ; coftume aflai ufato da co-
loro, che sfoggiano in veflire quantungue fieno poveriffimi , ¢ quefli intendi
ricchi finti , che [i [pacciano i millioni , che fi fucl dire: Adexzettin non rifquote Pan-
talone , ¢ s' intende , che gli abiti da fate non vagliono tanto , che impegnandoli
poflano rifquotere quei da inuerno , come appunto é |’ abito povero di Mezzetti-
no feruo f{elocco in commedia,e l’abito ricco di Pantalone vecchio in Commedia.
Narra parimente !’ hora appunto che cra , quando coftoro s*accofarono a Mal-
mantile , ¢ dice , che fu fu I annottare,che ¢ quell’ ora , fa 1a quale i Crocchiogi
fi m:ttono nelle botteghe miorno'2 un caldano per paffar Ja veglia, Ia raie ta.

prone

 

 
 

ye

TERZO CANTARBE; 129
gab pane: ete . 7 ' = 39 re notte a Terza, che é
uafi il principio « = ie fi ire , che i fuffe fempre al buio ,
© non vedefle punto entro il giorno. E*detto aflai vulgato per intender uno

_ debole di yifla , come intende nel prefente luogo. Vedi {pra C. 1. flan. 9. E for-

fe vuol intendere Vno di coloro , che perdono la vifta alla levata del fole , ¢ las
eee fole ya fotto .
“ SSPTANZA’ V. STANZA VI.
Per quefto mette mano alla fcarfella 1 quali fopra il nafo a Petronciano
Ow ha pit ciarpe affai a un rigattiere, Con la [ua flemma pofe a cavalcioni ;
Perche vi tiene infin la faverella , Tal che meglio [coperfe di lontano
Che la mattina mette ful brachiere ; Effer di gente armata pitt [quadroni .

Come [uol far chi ginoca a crufcherella,

Spaurito di cio , cala pian piano ,

Due Ando alla cerca intere inrere, Per non dar nella {cala i pedignont ;
E poi ne traffe in mezzo a due fagorti E giunto a bajo lagrima, ¢ lingrra y
Vi par a occhiali affumicati,e rotti. Gridando quanto mai n'a nella ftrowza,
STANZA VIL
Dicendo forte , percht ognun  intenda : Perch quaggik nel piano é la tregenda,

Al} armi all armi, uonifi a marcello
Ss lafci il ginoco,il ball, ¢ la merenda ,
E ferrinfi le porte a chiaviftello ,

Che ne viene alla volta del Caffelio;
E fe non ci ferriamo,o facciam tefla,
Mentre balliamo vuol fuonare a fefta :

Il detto famiglio { fe col metterfi gli occhiali, che era gente armata, es
quefto fi mefle a gridare ; allt armi ,

SCARSELLA, Tafca, Vedi fopra C. 2. flan. 8,

CIARPE , Intendi robe vili , firacci , bazzecole , che i Latini differo Scruta. ;
ed in altro fenfo Ciarpa vedi fotto C. 5. flan. 33.

RIGATTIERE , Rivenditore d’ ogni forta mafferizie , ed arnefi da i Latini
detto Propola dal Greco ; ed a noi viene da rigaglic , che intendiamo robe diverfe
di poco prezzo, ed avanzumi pfati. L’ Autore affomiglia la ta(ca di coftui a una
bottega di Rigattjere , perché quefte per lo pid fon ripiene di diverfi arnefi, fra i
quali ¢ caluoita difficile ritroyarvi una cofa , quand’ altri la voglia ,

PAVERELLA, Fave macinate, ed impaftate con acqua . E di quefta fi fanno
torte cotte nel forno,che fi dicong ancora AZacco forfe dal Gree, AMatto.Lat, pinfo ,
‘Tale Favereda dicono , che fia lenitivo a i dolori d’ ajlentatura, ed habbia yirti
d aflodar quelle parti ; ¢ perd dice, che coftui /a mette in ful brachiere, che & quel-
la fafciatura , che s* applica afl’ eftremica del ventre per foftenere gl’ inceftini,

CRYSCHERELLA. E? giuoco da Fanciulli. Fanno in fur’ una tayola yn mon-
ticello di Crufca , ¢ yi mettono dentro quelle crazie , 0 quattrini » che vogliono
giuocare , ¢ mefcolando poi bene , fi fanno da uno del giuoco , a cid depucato,
tanti monticelli di detta cryfca , quanti fono i giuocatori , i quali ( lafciando da

quello , che ha fatto i monti , perche deve efler ’ ultimo a pigliare i] monti-
cello’) tirano le forti a chi debba effer il primo a pigliare uno di detti monti , ¢
ciafcuno nel monte , che gli é toccato va cercando de i denari, che la fortuaa,
v habbia fatti reflare . Stimo, che quefto giuoco faffe nfato ancora da i Fanciul-
li Latini , perché fi trova Ludere furfure, Ed a quefta ricerca , che fanno i ra-
gazzi del denaro aflomiglia quello , che _ il famiglio per trovare gli oc-
chiali PA.

 
 

 
  

   

  
 

 

    
 
 

 
 
  
  
 
   

 

 
  
 

 

le m:
4"

 
 

 
     
  
      
    

  

      
   
    
    
    
   
 
   

130 MALMANTILE; >
FAGOTT/, Inuolti, o fardelli piccoli» I ncorg
PET RONCIANO , ¢, pi ae ee

forle fpecie di Mandragora ; ¢

fimile alla Zucchetta , ¢ fla appicca }

ghianda , alla quale s' affo a figuea ;

fi appella Atarignano, A quelto Perroncians s' allomiglia ¢

ti un nafo di firaordinaria grofiezza , ¢ di colore roffo livido.,

s' intenda’, che havefle quelto famiglio . wend
e4C AVALCIONT, Vuol dire una gamba da una p:

come fi fta in ful cavallo , e come,ftanno gli occhiali &

da una parte , ¢' altro dall’ altra .. ee a
PLAN piano. Cioe adagio adagio , bel bello: Adaaie. »VOCe p

aggiunta al verbo fare , ed al verbo andare fignifica quel, che hel prefen te Ju

cioe Adagio, e con diligenza,, che i Latini dicono placide incedere ; ed aggiu
verbo parlare fignitica parlar con voce bafla , Submifva voce.
PEDIGNONT, Specie d’ infermita , che viene,ne i piedi , ¢ nell

troppo freddo dai Latinidetti Perwiones, A pecan svelte
S/GNOZZ ARE ,0 fingozzare, o finghiozzare . E' un moto del

fuer (0, 0 mediattino , cagionato da foverchia -yotezza,o ripienezza ;

litudine fignifica anche {ofpirare vehementemente con pianto , come

prelente luogo , | Latini ancora { ne fervivano nel primo fignificato , ¢

condo ; Singultus-, & fingultire , © fingultibus ingemere . : etee
GRIDA quanto mai n' ha neda Hrozzas Grida quanto pnd pilseg

refifter la gola. Che froxza vuol dire La canna della gola, altrin

Gorgoxzule. 1 Latini pure di¢evano in gusture exclamare, £ da quefta

vicne ftrozzare , che yyol dire Strangolare sh

Dinte Inf. C. 7. Quef? inno fi gorgoglia nella frozza, |

gE, C28, Con Ia lingua tazliaca nella ferozza
SVONISI a martello. Si fuonino Je campane a rintocchi, che fi

corr’ homo . :
TREGENDA, Moltitudine , ¢ quantita di gente, Dalle perfo

crede , che vadano fuori la notte anime dannate , ed altri (piriti per

gente , ¢ quelte chiamano la Tregenda . Tal’ opinione fe bene ei Dp

plici , ¢ idiote , nondimeno pare che venga feguitata da S. Agoftino

lib. 4. de Civit, Dei dice. Lamia dicuntur anima hominum depravata,

te meritis macufofa , qua a corpore feparate terriculamenta funt mortalibs

fente lu go é intefa per moltitudine di gente . : )

SVONARE.. li verbo fuonare fi piglia taluolta in vece del verbo |

e¢ perd ne nate I' equivoco del fuonare mentre colora ballano y che vuol,; K
tergli , fe ben pare , che voglia dire fuonare alloro ballo :, Ed in cid’

Latini , che hanno il verbo py//are , che yuol dir perquotere y & 4
fuonare ogni forta di frumento muficale , ¢ le campano ; ed il
pelaror. .

 
 

   

 

  
TERZO CANT ARE: 131

f ava <o STANZA! IX.

Tn qm A Tra quefPi paittl acer a fore afsai ,

© EB che ne cup ogh ‘baloce “Ole d Marthefi, Principi, e Siznori ;

Cad omparita 2S Fyamin' di Conto,e eP offi botregai,

a 1s eee mooyct "Wana Siralbe, ¢ Battilori 5

, Quivi le una progenie ardita Lanaiiiol, Ovefici', ¢ Merciai

‘ a ye erstiey phe ee ti Norai , Legisti, Medici, e Deter 5

“OES ne'viene all’erta lemme lemme “In fomma quivi fon gente , ¢ brigate
Col Batthil Toffie tutto Biliemme . D’ ogni forva ; chiedere , ¢ domandare,
be bd fiuddetto vecchio andava ptidandé , ¢ che non oftante quelto , colo-

fo, erano in 1 if€{eguicavano: a darfi bel cempo , P armata arrivd

Panera: ¥

 

a
z
=
3

   
    

soe

=

Bese

thufa ; 1} Poeea'natra la’ qualica di qaefti foldati .

‘A Viol dir adnate, o’cantata, Bocce. Nov. 97. Cox una fua viola
fronr ia Pampire  Vatchi itor. lib. 10, ALdarefPa ando in perfona fopi a il baftio-
ne di S, Miniato con tutti li [uoi fuanatori , e dopo piit lunghe Rrombertate , ¢ flampite,
ec. Ma gui iatende romore:, € cicalamento odiofo , che ¢ il fenfo , nel quale oggi
per lo pire tela da‘noi gueft# parola , ed ha Jo fteffo fignificato che bordeiio,
d Paar Le Binill red'p ¢iictaforicamente , il’ che vedremo altrove .
ip ALOCCARST:"Prettullar® , Perder”il tempo’, € trartenerl in cole di poco
momento , 0 traftalli da ragazzi , dé i quali é proprio il verbo baleccarft, 0 balocco;
¢ forfe & fincopato daf verbo Baddlaceare , ¢ Badahicta ; Vedi foto C. 6, Nan. 32,
Spr ewncOEMEee.» Dictatho Wich Bicacea”™” Vareli MOF. 1B: 15. 214 fureno pore
tare’ le'tbiavi di non fo! che Biticea’; Vuol dir forcezza piccola’, ¢ di poca confidera-
zione pofta in luogo eminente , come appunto ¢ Malmantile , il quale con glie-
fla fola patold Biccieocea’, il Poeta benitliio delcrive; percht per Bitcicocea vol.
garmente intendiamo un Cafolaré’, 0 caftelluccio poito in luogo eminente, ma
da:farne poca flima’. Lafca Nov. 3. Salita che hebbe ton non poca diffiultd quell al-
pehre Montagna , credeka entrare in un bel cafpello , ma riguardando all’ inturno, ved-
de che era ina Biecicoece piit per vefugio di capre, che per ricetto di foldati ,

‘ST confide nelle fante'nocca . Ha ja fua fidanza nelle pugha . EV’ epiteto unre &
meffo per efprimere if odo'de} parlare de i Battilani : Se bene ¢ ufato dalla gen-
te anche'piil civile per interider perfezione come vedemmo fopra C. 2. ftan, 52.
E quié beniffimo pofto’;’ perché /anttxs yuol dir determinato , © fabilito , fendo
fincopato da fancitas , ¢ le pugna fono {' armi ftabilite , ¢ proprie de’ Battilani ,
Che’ per mocea'; che foo 1 nodelli delle dita , s intende tutta la mano feriata, che

in quefto pili , che in altra maniera fi {corgono le nocca’,

* della medefima natura , ed ha lo fiefio fignificato di pian
piario dette in quefto C: flan. 6. ; ma ¢ termine reftato née i Battilani , o feo
we @/ufate da altri (ara detto lieme lieme , che viene dal Latino ieviter , 0 leve,e
' ‘leggierttiente; 0 dal'Téfcano Lieve , che vuol dit Leggieri .

°BATTL, e Teffi . Bauilani, che fon coloro , che cdnciano la lana _, ¢ Tefi
‘quelli che lateffono, 8) 8 es cerns ' $
0 TETTOBiliemme. Chiamiamo Biliemme quell’ ultime contrade della Citta
di Firenze , dove abita quefta forta di gente , la quale veramente, benché
‘nata , ed allevata in Pirenze,é affarto onary da gli altri Fiorentini - i'co-

: 2 lumi,

2.
=

as

  

  

SARA KLEEN.
dh
e
2
5

 

Digitiesenipuar

 
a

 

132 MALMANTILE

flumi , ¢ nel parlare ; farebbe leggi a fuo modo ; mangia d’ ogni forta fpo
come gatti , cani , pefce , ¢ carne fetida ; beve ogni (orta di vino |
mente , come afferma il noftro Poeta forto in quefto C. ftan. 60. dicendc
che a bere ¢ peggio delle fpugne. In fomma é un Popolo da fe, che noi chia
gli Vasi, il Batti, 0 Biliemme , \a qual voce ferue ancora per efprimere la p
plebe , come é nel prefente luogo . ak
GVITT/, Guidoni , ace » fudici , fporchi , ¢ fordidi . E’ parola che ;
Napoletano , {e bene il Varchi ftor. lib, 10. € ne ferue anch’ egli per efprimeres
un’ hvomo d’ animo vile , dicendo: Eli era tanto d’ animo guitto, e tanto mefebings
che ufava dire: Chi non va a bortega ¢ ladro. (rape
HVOMINT di como. Huomini di ftima; huomini riguardevoli .. Trans
forfe dal giuoco delle Minchiate , nel “ giuoco fi ftimano , ed
Jamente le carte , che contano , le quali fon quelle , che vedremo {otto C. 8.f
61, Si dice 2 tale conta per intendere ; il tale ¢ huomo adoperato , 0 ¢ buonoas —
walcofa . ah
a BATTILORI, Mercanti d’ oro filato. Banchieri . Mercanti di cambio , che
fi dicono Negozianti . Serainoli Mercanti di drappi, ¢ di eta, Lanaiuols . |
canti di pannine ,¢ Lana. Orefici . Mercanti d’ oro, e d' argeato fodo. Alem
ciai, Coloro , che vendono naftri , feta , telerie, ed altre merci fimili,
quefti fuddetti in generale fi chiamano Mercanti , o mercatanti .
BRIGATE . Quantita di gente , Vedi fopra C, 1. ftan, 2, 3
D ogni forta , chiedete , ¢ domandate. Cioé domandate , ed eleggete ‘
forta di gente volete , che Ja troverete fra coftoro ; perché vi é d’ pmete

      
   

     
     

mur te

 

 

   

mete

perfone. Hs
STANZA X. STANZA XL

Sul Colle compartifce quefta gente 1 nome di coftui , dice Turpin
Amoftante con tutti gli Vfriali ; Fu Paride Garani , ¢ il legno prefty
Tra’ quali un graffo v’ ¢ conualefcente, Perch’ ei voleva darne un rivelling —
©’ haveva prefo il di , tre feruiziali; A un [uo nimico traditor France,
E appunto al corpo far’ allor fi fente Che per condurlo a feguitar Galina
L operazione , dar dolor beftiali 5 Lo tira pe’ capelli al fuopacley — *
Tal che gridando fenz: alcun conforto E per fuggirne ai paffi lagabellay
In terra fi butte come per morto « Lo bolla , marchia , ¢ tutta lo fuggella,

Ii Generale Amoftaate diftribuifce ful colle di Malmantile i Soldati , fra iqua-
li era Paride Garani , che havendo prefo un gran vacuatorio fentiva duloci acer
biifimi , ¢ perd G rammaricava . Il noftro Poeta per accredirare quefta ope
ra , come fece il Pulci nel fuo Morgante , ¢ ’ Ariofto nel Furiofo , le da anche
egli il fondamento della ftoria ; allegando |’ autorita di Turpino., come fece an
che fopra C, 2, flan. 31. ¢ da quello che {crive Turpino , cava che coftui havea
nomie Paride Gatani , il quale havea pre(o il legno per dare una quantita di le
gnate a un (uo nimico Francefe , cheyper condurlo a fegnitar Calvino,s lo yoleva
tiraré pe i capelli in Francia, ¢ per rifparmiarne la an  hayeva gia mat
chiato , ¢ bollato , ¢ figillato. E {cherzando I' Autore con. quefti. equivori, wl
dite che Paride prefe il Legno fanto per medicarfi del mal Franzefe. |,

PRESE id legno, Cio bevye il decorto di Legno Santo pet medicare il Mal

t r : ‘ran-

 

 
 

 

TERZO CANTARE? 133
Franzefe ; fe ben par che voglia dire, prefe un pezzo di legno per baftonare quel
Sa :

DARE un rivellino, Dare una quantita di legnate . Rivellino ¢ una {pecie di
fortificazione , che fi fuol fare d’ avanti alle porte delle Citta , 0 fra le cortines
delle Fortezze , cosi detto forfe perché revellitur a linea , 0 perché revellat hoftium
vim ; ¢ da quefta rivolta nelle cortine , 0 dal quafi fivolarh cal al nimico hab-
biamo il prefente translato , che ci ferve per efprimere , Rivoltarfi a uno cons
gran quantita di baft » bravate , riprenfioni , ec, E dicendofi aflol
¢ (enz’ aggiunta : Gui fece un rivelline , s' intende Gli fece uma folenne bravata ,o buo-
na pafsata,o gran rabbuffo ; E dare un rivellino,s' intende dar quantita di percofle.

RIDVELO a feguitar Calvino . Par che voglia dire ridurlo a feguitare la fetta
di Calvino Eretico , ¢ yuol dire , che per farlo divenir calvo , quefto {uo mal
Francefe lo tira per i capelli , ¢ glieli fa cafcare .

£0 balla , marchia , ¢ tutto lo fuggela , Fa bullette , marchia , ¢ fuggella. E vuol

dire che ee fuo mal Francefe gli havea cagionato bolle , crofte , ¢ lividi; che

il verbo fuggellare vuol dire Far de i lividi nel vifo a uno con le percofie , i qua=

li noi chiamiamo Pefche : 1 Latini in quefto fenfo differo ; /uzgiliare. Vedi ford

C 6, flan. 54. metaforico da /uggellare che vuol dire imprimere in cera , oftia , ¢
fimili nelle Jettere, ec. ¢ fi dice anche /igi//are Dant, Purg. C. 7.

La {ua impronta quand’ ella figila .
E fuggellare Dante Purg. C. 10, Come figura in cera fi fuggella, E Canto 3}.

Ed io fi come cera da fuggello.
STANZA XIL STANZA XIIL

Dife Amoftante , viffo il cafo forano , Gloria cerca Lion , piit che moneta ,
© Noferi di cafa Scaccianoce : Pero ch’ ei bada al giuoco,efa progreffo;
Per Ser Lion Magin da Ravignano , Per l’ acqua in Pindo andocome Poera,
Ch’ sk venga a medicar , corri veloce ; Ondt agl infermi da le pappe a lefso .
do dico lus , perché ce n’ ¢ una mano, Gis é quel che attende a predicar dieta
Ch infilza le ricette a occhio ,e croce y E farebbe a mangiar con L' intere/so ;
O fa fopr’ al! inferme una bottega , Ma perché gid tu n'hai pits d'uno indizio,
E pos il pitt delie volte lo ripiega. Va via , perché P indugio piglia vivio ,

Amoftante veduto lo firavagante accidente, ordind a Noferi Scaccianoce (che
vuol dir Francefco Cionacci ) che andafle per Ser Lion Magin da Ravignano
(che vuol dire Giovann’ Andrea Moniglia ) ¢ facefle venire lui medefimo , che &
un valent’ huomo , € non come quaicuno , che non fa dove s’ habbia la tefta , ed
in vece di medicare un’ infermo 1! pil delle volte  ammazza con le fue {propof.
tate ricette , ed ¢ di quelli , de i quali fi pud dire.

His , & fi tenebras pep ant , off facta poreftas y

Extenuandi agros , bomine/que impuné necandi ,
. [che non fi pud dire di Lione , che procura pil d’ acquiftar gloria che oro.
Egli ¢ Poeta , ¢ pero non ¢ maraviglia , fe andando egli per J’ acqua al fonte di
Parnalo dia poi molte pappe con  acqua’a gli ammalati . L’ Autore dice cosi ,
perché in una fua leggieri infermica non voile quefto medico , che e¢gli pigliaties
amedicamento alcuno , ma lo volle curare con ia fola dicta , facendoli mangiar
fera , ¢ mattina pappe; ¢ perd dice ; sstende a preduwar dicta, E farebbe a man-

gia

 

 

 
ee

 

 

*

134 MALMAN TILE T
iar con U interefo ; perché veramente itil quel tempo Lione
{ano e robufto, mangiava aflai. Quefto Lione non cra fta
tore nel primo componimento de! te fia Oper: 28 fuo
havendo folamente eat quel 1 ream ad aie lf
dremo pocoapprefio,ia dopo Ja fudderta infermita,per ven
dell nalce whet to a Tieta ce lo volle ae ‘Hor tornandd
no. 1] Generale dopo haver dato a Noferi molti contrafiegni’
{cefle queflo medico, manda’a'cercarne. © © ssi

CE n' é una mano, Ce ne fon molti . Termine’che vien dal Latino ©
En, /unenum manus emicat ardens . aes -

INFILZ A le ricette a occhio, ecroce, Si’dice anche'a Occhio V0
ricette fenza regola , confiderazione , o fondamento . Opera fenza f
prova, E’ termine meccanico , - « eee ee ee
FAR una bortega fopra uno infermo , Far allungare i! male per cavarne'®@
guadagno , E quefto termine s' ufa in qualfiyoglia negozio, del quale uno pra
ri di prolungar la {pedizione per bufcar pity denaro. nse ae
RIPIEG ARE uno . \ncendiamo Far morir uno , Vedi fotto C, 10
BADAR al ginoco , Attender con applicazione a quella profeifioné
fa , 0 a quel negozio-, che ha fra mano, e fi dice anche Badare-a dott
fopra C, 1, ftan, 62. quelto verbo badare in altri fignificati , z

PAPPA, Cioé panc bollito nell’ acqua ; o'in altro liquore . E’ dig
le inuentate dalle Balie per facilitare il parlare a i bambini, come B
ma ,¢ fimili . I Latini difflero , pappare , ¢ i Greci pure dicevano ?. eb
in altro fenfo , volendo-efprimere il Padre , i) Babbo , Vedi fopra C, 2, flan!
E fotto C, 4. ftan, $e 12. : : bad ovale

ATTENDE 2 predicar dicta, Sempre dice che fi mangi poco; che q
tende per far diera. Se bene appreffo a’ Medici diera vuol dire regola
verfal¢. Dieta fi dice congreflo di gran perfonaggi per trattare n
mi , come fi dice Dieta il Congreflo de i Priacipi Elettori all’ Ele.

eratore . :

F AREBBE a mangiar con P interefso, Mangerebbe fempre di giorno, ¢di
te, come fanao i-cambi , 0 ufure’, che mangiano di , ‘notte, mentre che: a
po fa crefcer la fomma deg!’ intereti . L’ ufura in Ebreo dicefi mor/>', ie ‘one

L LN DVGIO piglia vizio. L’ indugiare, 0 trattenerff ¢ pericolofo'di cagio
quaiche danno , o far perder la congiuntura di confeguir I intento. Adoré

   

   

   

   
  
  
 
    
 
   
 

   
  
  
  

damnum , 7%
STANZA XIV. 0 ¢ Tai
Noferi vanne , ¢ fente dir ch? egli era Bedi foglirdiste/a una gran eras
"Con un compagno , entratoin nn fattoio, Ha bell", ¢ ritto quivi il [uo fe ’
Ow egli ba per lanterna , efsende feras\ ~ Siche prejfo lo trovi, eid
Li orinal fitto fopra a-un [chirxatoio , ©. Dell uned Sndio gh facta ]
Noferi trova il Medico nel Fartoio da olio , che quivi era il fuo ftudig? 3

 
 

fa !' ambaftiata . :
FATTO/O, Quella lanza , dove & la macine per infragnere I’ olive 5

ma
firettoio , ed altri ordinghi per cavar I’ olio dalle medefime olive « en it

tino Oles fattorinm . Ree

 

   
  
 

 
     
     
  
  
   
 
  
 
 
    

 
 

gi?

ae
SS

 

TERZO CANTARE,. 135
elfednehtetse 0.0 altea gaapaveris » ocl quale s!orina , da i Lati-.

ma cfc ane » donde i Sanefi chiamano {cafarda ‘
piers effetto ufanole donne .

Beis canna di ftagno , o d’ altro metallo , con la

sf agi’ infermi. Vedi forro C10. fan. 4. >
ames ae Sparfa una quantita di fogli . ee era per la fi-
quella diltefa di fogli con le ere , 0 mercati , che alcune

y tan ei eae nelle quail per le piazze fi veggono moltiii-
ee v dilegnt » leggende, ed. altri arneft confufamente..,
ra ;

  
 
    

 abbiamo forfe quefta voce fiera dal Latino forwm,che era intclo
a dove fi facevano le fire 0 mercati , 0 pure dal Latino ferie,.

l ¢ritto si Hacon facilita aggiuftaco il (uo ferittoio ; che la voce bello,
terthini ale

in 0 non vuol dire , che Ormai ,o di gia,,¢ ferve per cmfali,
¢ per denotare la franchezza in terminare una opcrazione: Si dice riccare unde
bortega., rizzare wo negorio per dar principio a un negozio..

VNTO fi: Oe chiama fiudio quella ftanza., nella quale uno faa Gutters? :

epee Medico haveva depurata per {uo (tudio Ja ftanza del fattoio , lo

 

 
  

IN? en au ftanze fono , 0 verifimilmente.devono effere uate.
‘ NZA X STANZA XVI,
writ chiamato Brae Era queft’ huomo.un certo Medicaftro y
y (ponde ide haver’ allora altro che fare , C? al dottorato. Luo fe piover fienos
nen RR 6 upa.fua commedia ing diftende Eperch’ ei. vi pati (pefe., e difaftro
* datizalaca. U,Confole di.Adare y E frato fempre grofjo-con Galena ;
“Eche/e? opra Jua cold s* attende Egiuntola: Vofar(aife)ua’ impisftro >
Pee feast qari fue fcolare Onde s' it mat venifJe ds velen
> nd (persmentata ed in Sua vece Prefto vedrema ; in tanto egli fe [pogli,
ia mandato lis; ¢ casi fece. E fiami dato aduenties efogli.

‘Scntendo Lione d’ efler chiamato a medicare » tifponde , che per allora nons
;wenire , mache. mandcra un (uo (colare valent’ haomo, Coftui cra un gran
_— O.giuata doye eral infermo., comincid fubito con. gli (propofiti .
COM OLE di mare, Quefta tu una Commedia intitolata La Serie nobile , nel-
Jaqualc ¢ introdotto per |’ Broe.un Confoic di Mare in Pifa,onde molti la chia-
mano il Con/ole di mare , ancor che il titolo ftampato in fronte,di cla fia,La Ser-
na nobiles ¢ tt compofta dal medefimo Lione,, ¢ recitata. in) mulica: con grandi
Apparati d’ ordine del Serenifsimo Principe Cardinal, Gio; Carlo nel. fuo belliti~
mo Teatro fabbricato allora.di nuovo, Ed il noftro Poeta nella prefente ottava
yuol moftrare Ja poca applicazione , che Lione haveva in quei tempi alla medi-
cina,, come giovane , {¢ bea per altro dotto ; ¢ che poi voltacofia tale ftudio ba
faputo acquiflarfi la fama , che ha acquiftato , € meritare una delle prime Catee-
dre deilo itudio di Bila , ¢ di feryire attualmente'al Serenifsimo Gran Duca per

CUEDICASTRO. Medico di poca {cienza ,.0.{ come diremo:) faluatico ,

FE piover fiera nel fuo dottoraro ,, Quando fi fente uno » che vaole fpacciach per
huomo dotto , ¢ dal parlare fi fa conofcer per. uno igaorante , fi fuol dire quaa-
do ci parla Tirate gia del fieno intendendoG : Per darg.a queito bus che pacia.. Si

whe

 

 

 
 

 

~ Galeno , ¢ non fapeva quel che egli dicelfe , fiche in fuftanza vuol dir un

   
   

136 MALMANTILE

che dicendo che nel addottorarfi coftui, piovve fiena,intende che coftui fi
to per un folenniffimo bue ; e perd venne gran quantita di fieno fenz’
fto , che diciamo ; La roba ci piove per intendere vien roba in abbon

chiederla . ; .
E’ STATO fempre groffe con Galeno ; Effer groffo con uno vuol dire

collera , o efler adirato con uno ; Si che dicendo , che coftui ¢ Pato fempr
can Galene , perché I’ haveva difaftrato ,¢ fatto penare , s" intendeera:
feco ; ¢ perd non lo guardava mai , ¢ confeguentemente non havea p

  
      
       
   
     

diffimo ignorante nella Medicina . 2

VELENO . Quefta parola ha due fignificati : uno proprio che & toffico,
tro improprio , che ¢ fetore . I! primo ¢ quello , che s' intende nel prefente
go , il fecondo fi vedra nell’ Ottava feguente . ae
STANZA XVIL eae
Confermata pero la fua credenza '
Rivolto at eeapagl oe a dire bed |

  
 
  
  

  

Mentre ¢ {pogliato , per ta peftilenza ,
Ch’ egli efala , fi vede ognun fuggire ,

Pernenne una zaffata a Sua Eccellenza, uefto ¢ veleno,e ben di quel profonde,
Che fu per farlo quafi che fuenire ; Ses voi ch’ egli avvelena it

Mentre che Paride fi {pogliava ognuno per lo gran fetore comincid a i
onde i! Sig. Medico , che fente ancor’ egli I’ orrendo fetore , fi conferméd nel cre
dere , che fufle veleno , percht avvelenava . ie

PESTILENZA_, Intendi fetore grandiftimo. E fi ferue della parola pe
zy per Ja parola veleno prefa in fignificato di pyzzo , o fetore , ¢ per altro Peft:
lenza vuol dire mal contagiofo, : ? -

Z AFF-AT A. Parte del vapore di quel puzzo , portato dal moto dell! arias:
E fidice anche 7afaea d! ogni liquore per intendere /prxzzagiia d’ ogni liquore «
Franco. Sace, num, 136, L'orina gli ando ful Cappuccia,e nel vifo,ed alcune rafate io
bocca, coe
4S, Ecc. Quefto titolo benché non fia cosi conueniente a’ Medici ,nondimeno
 ulato dalla noftra plebe in vece dell’ Eccellentiffimo , el’ Autore lo daa
medico per derifione . f

PROFONDO., Per traslato fignifica Grandemente , fmoderato , 0 perfettifi-
mo , come ufavano anche i Latini. F

AVVELENA. Rende puzzolente. Ecco la voce veleno , ed err
fa nel fecondo fenfo detto di fopra di paxzo , oferore ; El" equivoco , che
cid ne nafce , ferue a quefto Medico per farfi ftimar dotto moftrando conolcere
che quefto ¢ veramente veleno,perché egli avvelena,che yuol dire far uutire,ed!

lo piglia in fignificato d’attofficare,c Veleno in fignificato di toflico, Vedi forto in

quefto C, flan, 54. la voce lezzo . t
“STANZA XVUL 5
Rifpofe il general , commoffo 4 Sdegno + A cio foggiunfe il Medico: Buon fegney
Come veleno ? 0 corpo di mia vita | Segno che la natura inuigorita
Edoveeil Mefroaahauedtre ingegno? A’ morbi repugnante , adeffo quefte
Lovedrebbeil miabuesch'egl bal'nfcica, ef noftri nafi manda si moleflo.

Ui Generale s’ adira , ¢ dice; Che non hayete odorato da fentir quefto puzZ0»

 

   
 

TERZO CANTARE, 37

—- conolcere , che egli ha I’ ulcita! Alche.replica il Medico: Que-
foe fegno , perché la natura havendo prefo vigore , come quella , che re-

pugna ai morbi , espelle ora quefto morbo , ¢ lo manda ai noftri nafi. Per in-
- Render fito , fons direa quefto Medico , é:neceflario fapere , che
“lay fignificati , il primo ¢.iofermita , ¢-dicendo repygnante aii

morbi intende all! infermita ; ed il fecondo é fetore 0. puzz0;-¢ dicendo manda a’
noftri nafi queffo morbo intende Manda quefto fetore , Ea il buon medico , che fti-
mas che natura morbo repugnans. voglia dire repugni al puzzo, cava la confeguen-
za, che il fentir quefto puzzo fia buon fegno, perché la. natura fcacciando il puz-
20, dal corpo dell’ infermo, lo manda a i nafi de’ circoftanti , ¢ cosi va {cemando
il morbo al iente ..

_ £0. vedr mio bye. Lo vedrebbe uno , che non haveffe punto di giudizio .

YSCIT A, Stemperamento di Corpo , Soccorrenza ; da’ Latiai con voce Greca
detta Diarrhoea.

SVON fegna. L’ Autore.moftra in quefta Ottava i] modo , col quale foglion,
parla i Medici ignoranti per accreditarfi apprefio agl’ idioti , dando ragioni
(propofitate, ¢ inducendo ott improprj; pur che lufinghino il pazziente con
una eerta apparenza-di {perar bene , come fanno gli Zingani , ¢ i Montamban-

chi.
wot iaDaceamy § eeoS DANZA XIX.

Vedendo poi y chtil flufso raccappella Chiamagli afpati, cbinfermieri appella,
(Come quelle ctha in zucca poco fale ) Ui Cerafico chiede, ¢ lo Speriale,
Comincia a gridar:Guardia,lapadella; E veuuto Linchioftra, al fin fi mette

BE ( quafi fufse quivi.nno fpedale ) A feriver una rifma di ricette .
L leatiflimo Medico vedendo , che il corpo faceva nuova operazione,co-

mincida chiamarla Guardia, che portaffe la padella , penfando che quelle pa-
role havefiero virtii di fermare il flutfo , havendole fentite dire negli Spedali in,
occafioni fimili ,¢ perd credendo effer ne/lo Spedale chiamava gli Aftanti , ec. ¢
poi fi meffe a feriver una gran ricetta .

RACCAPPELLA, Opera di nuovo. Reitera , Replica. Raccappellare fi di-
ce quando coloro , che ftringonol’ olive per cavarne I lio, o le vinacce per ca-
varng il vino , dopo haver dato qualche {iretta , allentano lo ftrettoio , ¢ nelle>
gabbie mettono nuove olive , o nuova vinaccia fopr’ all’ altra , che v' era prima.
Alcuni dicono rincoppedlare , tracndolo dalle coppelle de’ purgatori d’ oro, nelies
quali rimetcono pits volte lo fteffo metallo per rathaarlo,il che dicono rincoppeliare .

HAVER poco fale in xucca , Haver poco cervello , poco giudizio. Boce.n.2,
Be 4: Per porre la fun belezza innanzi ad ogn' altra , fi come quella che haveva poco fa
4¢ in xucea,, Vedi fopra C, 1, ftan. 73. ¢ forto C. 4. tan. 15.

GV-ARDLIA , la padella, Quefto ¢ un detto , che s' ula, quando fi fente , ches
altri facciaromore per di {orto per caufa dell’ u/cita del vento, e fi dice cosi, per-

gl infermi, che fono negli (pedali , quand’ hanno bifogno di vorare il.ven-
tre, chiamano colui > che é di guardia , che porti la padel/a., che € un valo di ra~
me, ¢¢, il quale ¢ adattato in maniera-da poterfi mettere,in cao di bifogno,nel
0 fotto all’ infermo , accid che pofla fare il facto, fuo.; tenza muoverfi dal
etto . 3 Bibi 2% oo ui wha

7 ee

s “sTan-

 

 

 
 
   
        

138 MALMANTILE

ST ANT? , 0 Afanti, Son coloro , che affifiono al {eruizio deg!’
2 me vedemmo fopra C. 1, flan. 48. Lat. «d/antes . i628

INFERMIERE , Chiamano negii {pedali Znfermiere colui. 5
che gl? infermi fieno mefhi a ietto, quando fon condotti allo {pe p
nota per fargli vifitare dal Medico , ¢ gli regiftra al libro degli entrati, ¢
ufciti , ed al libro de’ morti . (andi 04

CERVSICO . Quello che medica le ferite , piaghe , ed altri ma!
richieggonc opera manuale, ¢ cava langue , ec, detto ancora con voce!
ufata da’ Latini Chirurgo.

LISMA , ori/ma , Diciamo un fagotto , o balletta di carta , che
a 500, fogli DalGr.arithmos, Qui perd é detto iperbolico , ¢ per mofir
quello Medico (crivetie afiai,non che veramente confumaffe una Liima di

TANZA Xx STANZA XX
Dove diceva ( dopo millioni Peré prefto boliir farere a fodo
Ds feropoli, ai drammeye libbre tante) Voit agnello,o caprette in um pi

  
    
   
    
   
     
     
      
 
 

 
 
   

   
 

Che gia, che quefto mal par che cagiont QV un’ altro vafo nelle fte/so
Stemperamento forte , umor piccante y Vn lupo per infin che fia disfe
Per temperarlo ; Recipe in bocconi Poi fare un feruizial col prii
Colla, gomma , mel,chiara,e diagrante, E col fecondo un’ altro ne fia fai

Quindici libbre in una volta fola Fard quefea ricerca operazsone

 
 
 
  

  

  
 
 
 
   

 

Di fangue fe gti tragga dalla gola ; Senx' alcun dubia, ed eccola rag
STANZA XXL
Accio che tiri per canal diverfo Quefti animali efsendo per nat
L’umor che tende al cétro, ut One grave Limici , come i tadré
Che fe duraffe troppo.a far ral verfo Ritrovandofi quivi per

Dir potreble dinfermo: Addio fave. 1 lupo correra dietro all’

Pot tengafi due di capo riverfo Lagnelio, che del lupo baurd

Legato per i piedi a unatrave 5 Ritirandofi andra per ibd

Se quefto non facefse giovamento, Cosi va in fu la roba y &.

‘Compote gli faremout argomento , E i due contrarj fan , cht. zodd,

To quefte fue ricette moftra |’ Eccellentifimo Medico Ja {ua g ine COs
proporre farmachi , ¢ rimedj {propofitati , come € quello de i due brodidi lupo, —
ed’ agnello, ¢ quello del tenere il pazzicate appiccato al palco per i a

. ca

    

   
  
     
       
   

    

‘capo all’ ingid , b Ph Sayy sky
eu/LLIONE, E! un numero determinato di dieci centinaia di migl
é prefo per indeterminato ; come fuccede {peflo , che per efprimer, u
quantita di cofe., fi dice E’ un millione delle tali cofe , ancor che fieno mol
no , ed aile volte molte pil . Cosi i Latini in queflo fenfo fexcenra;
4 Greci myria., cio’ diecimila . ou) , isbn aN
STEMPERAMENT O forte. Stemperare yuol:dic Ammollire ;10
nel ventre di coflui cra follevamento d’ umori , ¢ ftemperamentodi
ti, clot acide , ¢ diumori piccanti.. Gli epireti di forte, ¢) piccante fon’
conyenienti al yino,dicendofi vino forte quello , che comincia a diventare ace
ved in molti lugg hi d’ Italia fi dice Vin forte,il vino:gagliardo , o grande
Bectl: mee

 

   
  
  

    
 

 

  
 

piccante quello che in beverlo fa friazare le Jabra, ¢ la lingua. Quefto Bé

  

  
 

TERZO CANTARE: 539
ti?  - lentiffimo Medico /perd intende quel forte per acido , ¢ per grande , ¢ gagliardo ;
v4 E piccante dal ieee % che vual EE Duguere % Offendere che fi dice anche
ph dar nel nafo » Vedi forto C; 7, ftan, 59. I Eccellentiflino cava P argumento , che
ee i umori fieno’piccanti , perché danno nel nafo col loro fetore ; Ora per ra(-
Oe] fodare , ¢ coagulare'tal flemperamento vuole il prelibato Medico , che fi dia al d
pazziente a bere gran quantita di col/a , miele , gomnia , chiara d wow, ¢ diagran-
tit te, le cole nella ens antita , che egli pone {es incorporaffero , in

ase

Ge grandifiima® quantita: d’.a ¢ fsrésbous atte a coagulate , ¢ feccare un Iago ;

¢ (e vi havefle aggiunto gefio , ¢ matton pefto-haurebbe dato una ricerta da ilop-
ait pare quante’rorture fi poflano mai troyare ne i vivai .
rat \ DIAGRANTE , Specie di gomma , 0 colla , che ferue per incollare i drappi
aa ne i'rovefci de i-ricami’, 0 per altee cofe fimili.
1. SE li-tragga 15. libre di fangue'per la gola , BE cavandofi 15. libbre di fangues
dalla vena della gola del pazziente ; ¢ legandolo per i piedi al palco.col capo

Po all" ingid (che quefto vuol dir caporiverfo ) preteude il Medico y che ia roba fia,
ya per mutar viaggio , fe vorra condurfi al {uo centro , che non & pid nel Inogo,
fen dove era prima , ma ftante la pofitura del corpo é diventato fuo centro il capo,
a CONTINOVASSE afar tal verfo, Continovafic a fare nella medefima forma,

‘fe o maniera, Vedi fotto C. 7. ftan..1.

AD DIO fave, Significa Noi fiamo (pacciati ; Noi fiam finiti; Siam morti, Fa
et un Villano ne! contado d’ Imola d’ ingegao pi) tofto groffo che no , il quale ha-
i veva un belliffimo campo di fave , ¢ nel mezzo di effo era un gran ciriegio carico
o di ciriege . A tal Ciriegio haveva il villano fatta una fortidima prunata , perché
is pops soe gli fuflero colte; ¢ vantandofi di quefla fua diligenza , fu fentito
. da un Cieco {uo amico , il quale glidiffe: Con tutti li tuoi pruni io vi falird , e:
sf fe non lo faccio , voglio perdere dodici lire , ch’ io mi ritrovo + ed il villano re-
de
eh

 

plicd: Setu non pigli la fcala , o vero non porti il forcone , 0 altro per levare

1 pruni io voglio giuocare quefto campo di faye , e che tu non vi (ali. I] Cieco

fi contentd; ¢ cost.conuennero . L’ aftuto Cieco fi coperfe tutta Ja vita con buone

ie pellidi bue , ¢ cosi armato paffando per mezzo de i pruni fenza fentir puntura_,
y alcuna , fali fopra il ciriegio . Li villano, veduto quetto , tardi accortofi della fua
g* ee » piangendo il fuo danno gridava: Addio fave, cioé io ho perduto le

uoh fave , Vedi il Cornazzano Novella 10. dove troverai quefta fayola non travetti.
i ta, e meglio efpreffa.

TRAVE . Legno groflo ,¢ lungo , che s’ adatta a reggere i palchi. z
inet ARGOMENTO., E'lo fteflo , che Seruiziale , 0 Crificro detto fopra in quefto

lt C. flan, 10. ¢ 12. E.quitorna bene , perch¢ vuol medicarjo per via d’ argumenti
si ——_Jogici yma di canfeguenze {propofirate .
ao BOLLIRKE a fodo , Ciok bollire molto tempo , ¢ gagliardamente ,

BRODO.. Decorto di carne. Acqua ingraflata con carne .. Se ben la parola
PA brodo é comune a ogni forta di decotto , o mineftra , aucorché non di carne.
ie 1 DVE contrar} fan che it terzo goda, Inter duos stigantes tertius gaudet , Con que-
oe flo argumento , ¢ con queta fentenza , ¢ con altre ragioni da fquartati, pretende
cA F Eccellentifiimo d’ haver trovato il:modo di fermare i Hluilo . ;

S2 STAN-
 

 

140 MALMANTILE

    
 

STANZA XXIV; » STANZA XKV,

Cio detto rivolteffi al mormorio In quel che quefo t

Di quell’ ambrette, ov’ a meftar fi pofe; We dice ogni or delt’

E, perch' elle fapevan di ftantw , TofelloGrani, ilquale tu

Teneva al nafo un mazzolin di rofe. Mofo a pierd, con una fun

Soggiunfe poi : Coftui vuol dirci addio, Tagliace havea aun

Che quefie flemme putride, ¢ vifcofe Sopr’ alte quali a foggia di

Moftran , che ben’ affetto agli artolani Fu Paride da certi Conradini

     
  
  
   

Ei vnol' ire a ingraffare i Petronciani, Portato a’ {uci poder quivt vicini.

L’ Eccellentiffimo Dottore , dopo haver fatte le fuddette belle ordinazioni
mette'a Muzzicare quella materia ,¢ da quel puzzo fa pronoftico, che il
te fia per morire ; ¢ ’ argumento , che egli fa-di cal morte non ¢€ didimile.
ricette. In canto Tofeilo Gianni accomodé una barella , fopr’ alia qual
fu pofto , € portato da certi contadini ad una villetca de’ Signori —
Malmantile in Inogo detto Santo Romolo ; nella qual Villa trov:
concepi nella mente il far Ja prefente Opera , come dicemmo fopra ne)

wIMBRETT- A. Cosi chiamiamo guanti , ed altre pelli conciate con
d' ambra.. Ma qui intende , ironicamente parlando , quella materia fetida ,

SAPEVA di ftantio, Haveva cattivo odore. Quando una materia per
ghezza del tempo ha cominciato a perdere la fua perfezione,fi dice /antia; ches
fe fia carne , 0 pelce , non da troppo buono edore ; e quelte fi dice payee
tio, La qual voce viene da ftanziare lungo tempo, ed ¢ il Latino

 

{otto in queflo C, flan. 54. sith,
VVOL dirci addio , Sc ne yuol’ andare . Ci yuo) lafciare , cioé prire.
FLEMMe4. Vmor freddo , ¢ umido che i Medici chiamano in

munemente fi dice hemma dal Greco, reid

VVOL! andare 4 ingrafare i Petronciani, Vuol andare a ingraffare gli orti col
fuo corpo , facendoli forterrare ; ¢ piglia Perronciani (che vedemmo it
tio.C, fan, 6, quello che fieno ) per tutto lorto. E nota che per care la
caftroneria di queflo Medico , ' Autore gli fa dedurre il. pronoftico della morte
di Paride dal credere , che il fuo corpo fia gia corrotto, ¢ ridortofi tutto in quel
Ja terza putrida fuftanza , ed in confeguenza.atto , ed il'calo.a it i
ni; E vuol dire , che Paride morra : Digendofi vulgarmente per intender que
fto U tale ando a ingraffare i cavoli , cio’ il tale mori. ¥ oiioen at

CAPO a affiuele, A-uno ignorante fi dice Capo di Bue , Capo di Cafteones
Capo d’ ativolo ,¢ fimili, ZL’ afixolo é un’ uccello in tutto fimile alla’ Civetta , fe
non che ha fopra il.capo , alcune penne ritte , che fembrano cornas ©* atobigt

TOSELLO Gianni, Agoftino Nelli Gentil’ huomo Fiorentino buon:
¢ veramente huomo da bene, Che intendiamo baor figlinale : » SANA GE

COLTELLA . Specie di {cimitarra , Arme ches! ufaiportare 4 va

    

a caccia.. 3 0
BARELLA, Aruefe fatto di tavole » che ha quattro manichi 5 ferve pet por
tar faili, ¢ aleri pefi in due perfone ; qui intende una barelia da porearesane
d huomiai inferm) , 0 morti,, che é Gimile alle bares o-catalettico i quali fi 10
glion postare detti corpi , ¢ da Bara ¢ chiamata baredla .. Vedi oan”
dane 54. SEAN

 

 

li ee a
 

TERZO CANTARE. 14t
to STANZA XXVI.
Fu del Garani afcritto fucceffare Dicon ch' ei nacque al tempo delle more,
_ Puccio Lamoni anch'ei grad’ ingegnere, Per ch'eglié di pel brunoye membra neve;
Braviffime Guerrier faggio Datrore , Hor qua di Cartagena eletto Duce

Cortigiano , ante, ¢-Tanerniere, i fior de’ Adammaganuccoli conduce .
Al Garani fu dato\ per fucceffore Puccio Lamoni , il quale &, Paolo. Minucci .
Il Poeta dice che coftui era ingeenere , ¢ Adercante ;.ma tali attributi gli (no fin-
ti, perché io, poflo giurare., che egli non fa ne dell’ una , ne dell’ altra profettio-
ne. Loichiama guerriero , ¢ quefto perche detto Puccio fece una campagna.
neil’ efercito Pollacco in Pruifia ,feguitando quella Real Corte , alla quale era
fiato inuiato dal Sereniflimo Principe Mattias di Tofcana alla Maefta del Re Gio:
Cafimiro. B perché detto Puccio godé per melti anni , ¢ fino che S, A, vifle ,
P honore di fervire. all’ A, S. in qualita di Segretario , perd dice che era Corti-
giano. Dice che ¢ Dostere perché veramente egli ¢ addottorato in Legge , fc be-
ne per I’ applicazione alla corte.5.non clercito tale profeifione. Lo chiama Ta;
verniere., perché {peflo lo vedeva entrare nell’ Olteric, ¢ trattare con Ofti, il che
feguiva perché egli vendeva loro del vino raccolto nei fui beni, € gli conucni-
va lafciarfi rivedere {peflo per ri{quoterne il prezzo. Dice che fi vocifera , che
gli nafcefse al tempo delle more, Perch’ egli é di pel bruno, e membra nere, eficndo
li cosi in effetto : E facendolo Duca di Cartagena dice , che egli conduce if
ore de’ Mammagnuccoli ,ciot i migliori , ¢ pid valorofi Mammagnuccoli , Quefti
M gnuceoli ¢rano una fazione di galant’ huomiai , i quali f
profeifione di fapere il conto loro in ogni cofa , ¢ particolarmente nel giuocare ,
© pendere bene ildor-danaro,.¢ d’ cflere il fiore della reale , ed onorata
apialanite + Havevano ua loro capo , che fi chiamava Abate , dal quale erano
galtigati 5 quando facevano qualche crrore o nel giuocare,, 0 nello {pendere , ma
perd tutto ¢ra in galanteria. Le loro adénanze fi facevano in cala I’ Abate , do-
ve fi giuocava a giuochi pid di (paflo., che di vizio , ¢ fi facevano altre allegrie ,
dicene , merende:s ed altrispaflatempi. Coftoro crano tutte perfone {erie , es
quiste se:della pidriguardevole Civilta , ¢ percid era la lor conuerfazione molto
bramata,, onde cra-pumerolifima; Se bene non era ammefio a quella veruno,che
non haveile provata prima la fua dabbenaggine , ¢ non fuffe ftato riconofciuto
dal Abate, ¢ da altri (uoi Configlicri -meritevole d’ eflere ammeflo. Fra coftoro
era detto Puccio, ¢ perché egli era forfe de’ pili affezionati , i1 Pocta.lo fa loro
Condotticro., ¢ per Ja ftima che faceva di lui nel giuoco delle Minchiate , era fo-
lito chiamarlo il Re:delle carte ; percid lo fa Duca di Cartagena, ed ¢ ancora ap-
propriato , perché detto Puccio per effer di faccia bruna , ha qualche fembianza,
ed ariadi Spagnuolo ; oltre che nel tempo, che l’Autore Jo aggiunfe a quefta fua
Opera , il detto Puccio , era flato deftinato dalla Maefta del Re Gio; Cafimiro
per (uo Segretario dell’ Amba(ciata di Spagna.
STANZA XXVIL

 

L' Armatahaveatragli altri unCappellano Faceva da Pittor , da Tiziano ;
Dottormailfnofaper fu buccia a Maquat'ei fece main'adava agrnccia y
‘Pero ch’ egli Pudio col fafeo in mano , Hebbeuna Chicfa, e quiviabifcaaperta
Ed era pit bafon.a' una Bertuccia » Si ginoce fnoi foldi dell’ hee ©

Pee

 
 

 

142 MALMANTILE: &
STANZA XXVITL

 
   
 
    
  

Franconia fi domanda Ingannavini , Lelie havea in cafa it
E fu a come il pie valente , Gid fatta una lerione,e falla a
Perch’ eghs fapea leggere i Latini , Subito accetta y € fiede in alto
A far quattro parole a quella gente’, Senta metterui fu ne fal’,

Fra gli altri Cappellani , che erano nell’ Armata, era un Dottore y ma dig
fcienza ; perché il {uo ftudiare era ftato i] darfi bel tempo. Fu feolare d
re nella pittura , ma impard poco , ¢ fe bene fipprefumeva diaper’
fece mai cofa , che non fufle ftroppiata . Fu Rettore della Chiefa di Petriolo;
Villaggio vicino a Firenze circa due miglia ,¢ perché egli era huomo allegto ,
di conuerfazione , dice che egli ff gidscd fino i foldi dell? Sa} ed intende che co
fomava tutte le fue entrate in allegrie . I fuo nome cra Franconio th
cioe Giovannantonio Francini, A quefto dunque » come al pid dotto fu fatta’
ftanza , che facefle un poco di difcorfo a quei Soldati , ed’egli che “haveva’ ums
tempo fa recitata una lezione nell’ Accademia del Coltellini 5 ¢ 1 a ace

¥ j Wil

a memoria , fi content® di fare quanto gli era ftato impofto , ¢ fenza
tempo in mezzo montd in pulpito .

BYCCTA buccia, Leggicrmente . Cicé {apeya poco ; non haveva gran fonda
mento ; che fi dice anche s# pelle in pelle. Vedi foro C, 8, flan.58. edi
differo fuperficie renus . we ri

PUL buffone ad una bertuecia . Huomo arguto yallegro , © facetoyBaffune die
ciamo colui , che tiene il popolo allegramente con facezie , e moti, &
Scurra, Vedi forto C, 11. ftan. 42. E Sertuccia diciamo la fcimmia,

TIZIANO . Pittore celeberrimo. Econ dire facea da Tiziano; intende pet
antonomafia , che egli fi prefumeva d’ effer il pi valente Pittore del Mondo.

QVANT ' ¢i facea,n' andava a gruccia, Tutto quel che egli faceva rap
piato , cioé mal fatto , mal dipinto, Vedi fotto C. 11. flan. 41. mete

BISC-A. Luogo pubblico , dove & permefio giuocare a ognuno; Egiwecsre#
bifca aperta , vuol dire Giuocar fempre, ¢ fenza riguardo alcuno. re

JL Coltellini, Quetto & il Signor Agofino Coltellini Avvocato Fiorentino huo-
mo dotto , ed amatore de i Letterati , il quale in molte opere com, da tui fi
chiama col nome anagrammatico Oftilio Contalgeni. In cafa di eflo @raguilas
P Accademia degli Apatifti da effo fondata , nella quale fi fannoidifeorfi Acad
mici,ed altri efercizzj virtuofi: Mirabile per haver faputo far durare per lo (pa
zio di cinquanta , € pid anni la detta Accademia , (empre in florido, cola inl
lita a’ noftri {ecoli in quelta Citta. Lntcrueniva {pefio in detta Accademia quell
Francini , ed alle volte vi faceva qualche lezione ; nelle qualimofro i {uol
ed eruditi talenti 5 ¢ f¢ bene I’ Autore dice che il (uo fapere fu buecia it, OM
to lo-chiama huomo (caza forndamento , non é perd , che egli fulle tale 5
gli huomini de’ noftri tempi non era dei fecondi in dottrina non meno ’
che profana ; ed era veramente Dottore di legge . aim

SENZA metterui fu ne fal,ne olio, Prefto , iubito , fenza replicare , «o' mettet
dificulta , Nulla interpofira mora. Fu un tale , che tornato la feraa cafa'y difeal

fuo feruitore: Fammt una infalata , ¢ fa prefto , ch’ io forv afpertato, ¢ noms
yoglio mangiare altro che quella; fa preito. dico . Ll feruitore eee

 

 
 

i
ai

 

TERZO CANTARE. 143
fenza condire la portd in tavola al padrone’; il quale cid vifto lo fgridd ; Ma il
feruitore rifpofe; Signore per feruirui prefto , non vi ho meffo fu ne fale , ne olio,
E da quefta goffaggine del {eruitore viene il prefeate detto , che fignifica Fare una

 

 

cofa {ubito 5 ¢ fenza confiderazione .
+ neue toh Seca Se TAAN ZA .XEIX,
Sale in Bigoncia com due torce a vento , Che ben fi (corfe in Ini quel fondamito,
eicio lo wegga ognun pro tribunali , Che diede alla [ua cafa Giorgio Seali ,
Ove, moftrar volendo il [uo talento, E piacque si, che tutti di concordia
Fece un difcorfo , ¢ fece cofe tali, Si meffero a gridar : mifericordia,

Il Poeta continuando , a voler moftrare , che Franconio fufle di poco valore ,
¢ che perd il difcorfo da Jui fatto futle {cimunito., ¢ fenza alcun fondamento , lo
burla , ¢ dice che piacque tanto, che il popolo , fi meffe a gridar mifericurdia; del
qual termine ci feruiamo per moftrare , che qualche cofa ci fia venura a faftidio ,
come per efempio . Ei duré tanto a difcorrer , che mifericordia , Diffe tante feiocche-
rie , che mifericordia, Ob mifericordia,, quanto volete voi durare? Quali dica , hab-
biate mifericordia , ¢ compaffione di noi , ¢ non ci tediare pili,

BIGONCLA . Eun valo di legno , del quale fi {eruono 1 Contadini in tempo
di vendemmia per pigiarui dentro ’ uva, prima di metterla nel tino, ¢ ce ne fer-
viamo anche in alere occorrenze , come di portar’ acque , ¢ fimili,

» Hi Bini nel Capitolo del Pilo dice ;
Viua dir , che fe ben' ella il pit mi deffe ,
ind Ed opraffi,( non ch’ altro) una bigoncia ,
srobe ; Ognun direbbe , che ben fatto haveffe.
eo Epperche fo. vafo demo Bigoncia ¢ molto fimile a una cattedra tonda,perd
da moiti tai Cactedra: fi chiama bigoncia , come anche tutte I’ alere cattedre. Il
Davanzati ne} fuo Cornelio. Tacito poftilie al 2. libro num. 18. dice: Arringa-
vans i noftri antichi al popoloin piacza , in ringhtera, e nei Configls in bigoncia , ches
era un pergamo in terra a fogcia di bigomia .

TORCE a vento. Torce grofie che fi fanno di funi di cotone filato attorte per
feruirfene a far lume Ja notte per le Arade ; ¢ fi dicono 4 vento , perché refiftono
alivento ;¢ a-diftinzione di quelle , che fi fanno a Venezia, che per effer gentili
fi {pengono a ogni poco di vento. E Torcia , che da i Latini ¢ detta fusalia , fu-
natinm , viene'a noidal Francefe Terche

CHE diede alla [ua cafa Giorgw Scali., Giorgio Scali fu in Firenze an riputatif-
fimo Cittadino Popolano , it quale nelle diflenzioni , che feguirono a fuo tempo
fra i nobili , e Popoiani di Firenze, fi fece capo di quefta parte , con promefia, e
{peranza d! effer follevato a cofe maggiori, cio¢ all’ affoluto dominio di Firenze,
ebenché per altro accortiffino , ¢ prudentiflimo , lafciatofi portare dal dolce de-
fiderio di domiaare , fi fido nelle vane promefie della inftabil plebe , con la qua-

lep id haver forze baftanti per confeguire 1’ intento , s’ accinfe all’ ope-
ra; ma nel pil bello il popolo ,.o fpaventato., 0 pentito.!’ abbandond , ond’ egli
venuto in potere-del Governo fu decapitato: Eda Ini ¢ detto il Proverbio : Far
come Giorgio Seali , che vuol dir Pigliare a far’ una cofa fenza fondamento , che i
Latini con fimilitudine della Scrittura., diflero Scipione arundineo inniti, Di que-
flo calo di Giorgio Scali parlano tutti gli Storici , che {criveno Je cofe di Firen-
2

 

var”
144 MADMANTOELSST 9
ze digquei tempi, ed il Nerli fra gli aleriaggiunge , che allora‘¢omincid
proverbio. ; fancy Sutahogwies

STANZA XXX, i
Li tema fu di quefia fua lexione , Cost, dicea, la vofirase mia,

Quand’ Enea gia fuor del fuo pollaio SV Quis viva, e fanaye della 2
Paceta andar in'fregola Didone, » » Cacciara fu dal empia Mm

   
 

     

  

Com! una gatta bigia di Gennaro ;~ » Tredira ancl ella fuor ¢

E che fe i Greci afcoft in quel ronzone Pere sun rantoardire’, etal 1

In Troia fuoce diedero al pagliaio 5 Parui, @ adefo gaftigar fi

Ein man a Enea pofero il lembuccio , ¥ bavete il modo fenza cht a

Ond’ ¢i fuggi col padre a cavalinccio; to ha finito , 1) Ciel vi benedica
Il tema del difcorfo,, che fece Franconioy fu quando Enea eflend:Soggiand
Troia fece innamorar Didone , ‘ed aflomigliando:Celidora cacciata di Malman-
tile ad Enea fcappato da Troia , eforta quei foldati a gafligar Pardire di Bente
nella , € rimettere Celidora nel {uo ttato, gia:che hanno il modow
POLLAIO , Si dice da noi quella:ftanza , nella quale anno, edo
li: E chiamiamo pollaio quelle felue , o macchie , dove la {era vanno gli uece
a dormire ; Ma qui intende per translato la nofira Cafa , Patria yorluogo, dove
fiamo foliti abitare . > dog Gat
ANDARE in fregela, Dicemmo quel che fignifichi fopra-C. 1. flan, 25, Mas
che Didone fulle innamorata d’ Enea, come favoleggia Vergilio , &
ché oltre che Didone fu cosi cafta,che vedendofi violentata da Iarba f
ritania a rimaritarfi {eco , volle pi tofto da fe-ftefla ucciderfi , che il
fuo morto marito Sicheo con nuovi {ponfali'; EB’ anche vero, chene
feguire i] detto innamoramento , perche Enea fu 360. anni prima di

 

verita fi cava da diverfi Autori , e fi (corge in Darete Frigio’, ¢ Ditti a
che (crifflero la vera Storia dell’ eccidio di Troia. Che il noftro ic
ti gusita bugia-di Vergilio , dicendo nell’ Inf, C, 5. “aie a
Li altr’ ¢ colei , che s'ancife amorofa , y me
E roppe fede al cener di Sicheo , se gantiel 2

Non é meraviglia , perch Dantes’ era eletto per {uo Maeftro,  guida Vergilid.
Che Enea fude tanto tempo avanti a Didone, fi deduce anche songs
Didone fuggendo I’ infidie di Pigmalione fuo fratello , che per defiderio i
le haveva ammazzato il marito Sicheo , come pure accenna’ Dantes Parg, C.20.
Noi ripetiam Pigmatione allotta , onda
Cui traditore , ¢ ladro ye patricida toe illid
Fece la voglia fua dell oro ghiorta:, ». Seip 5
Portandofene il teforo in Affrica , chiefe a quegli abicatori tantovdi tert d
to poteva circondare una pelle di toro, cl’ ottenne ; Bd altutamente
detta pelle in firifce cosi fottili, che abbraccid con efle tanto terreno,
fico Cartagine , il che fu dopo 70. anni della edificazione di Roma » 7
edificata cirta 300, anni dopo la morte a’ Enea , Sant’ Agoftinodifein di It
done , che quando Vergilio non fufle ftato dannato per altro , i i
no per quelta falfita cotanto pregiudiciale alla ripucazione di Didone sl gl?
difende ancora Aufonio col feguente Epigramma tradotto:dal Grecow 0 ©

  

Ad

 

 

 
 

TERZO CAINT/AA RE:

145

  
 
  
     

2 S.\Ad Didus Tmaginem CXI.
Dida 5 nis guiase con/pici ‘
Seances (ase modis ipleagen aise sy

9») Talis cram, fed non Adaro quam mibi finsit erat men: ,

|. Vita nec inceftis Leta cupidinibus . ©

Namque nec dneas vidit me Troius unquam ,

dig Neo Lybiam aduenit Clafibus Miacts ;

L fens 5 atghe arma procacis Larbe

94 > vmorte pudicitiam

nificco 5 caStos quod pertulie enfes
5 wut lasa crudus amore dolor 2 *

 

nA ot Vita virnm, pofiris meenibus oppetiy..
Jnnida cur in me ftimulafti mufa eMaronem ,
Fingeret ut noftra damna pudiciria ?
Vos magis Hiftoricis lettores credice de me ,
SL Quan qui fart Denn eonoubitufqne canunt ;
bates Vates , temerant qui caPmine vernm,
: Humani[qne Deos affiemilanc vibijs,

GATT A bigia Er quella , che noi chiamiamio’Soriana , che é un mifto di co-
lor bigio , € lionato ferpato di neroy-qual colore foriano ff dice folamente di Gat-
ti, onde io argumento , che'i primi’ gatti di quefto colore veniffero a noi di So-
ria, come vennero alcuni anni addietro quelli del colore del topo portati da Pie-
tro della Vaile dalla Perfiaye petd da molti chiamati Perfianini ,. Vedi {orto C. 9.
flan. 19. 9) oe ; a :

RONZONE ; Conia jz ;-¢ruda vuol dir Cavallo ftallone’, o per Ja monta, da
i Latinidetto eguus admiffarins’:¢ per ronzone ,-ronzine , 0 rozza jntendiamo
eavallo cattivo , Ronzone'con la , z , dolce vuol dire una fpecie di Mofcone 50
tafano . Qui} Autore intende quel cavallo di legno fabbricato da j Greci per in-
gannare i Troiani come dice Vergilio . In. alcum Tefti fi trova {critto caffone in»
vecedi rontone ,. ma nel mio, che ¢ di mano dell’Autore, é {eritto ronzone ,
PAGLIAIO , BE” proprio guel cumulo , o maffa di paglia , che fi fa dat Conta
dini dopo haver battuto i! grano , per lo pid avanti alie cafe ; ma dicendofi dar
fuoco al pagliaio , s' intende Dat fuoco alla Cala .
PORRE il'lémbo'; 0 él lembuccio in mano , Significa Mandar via uno; E quefto,
perché quand’ altri yuol mandar via’ uno di qualche luogo fenza parlare , gli fas
il ferraiuolo addofio , e gli mette un lembo di eflo ( che /embo yuol dires
‘na parte dell’eftcemita del ferraiuolo , 0 d’ aitro abito , d yefte fimile ) nelles

« Sic eecidiffe imvat ; Viet ine vninere fama;

mani; ¢ da quefto ‘colui's aeoees d? effer licenziato , efiendo notiffimo , che»
uclto detto Pigtiare', o-dare il lembo fignifica Eller licenziato ; Tratto dai mac-
ri delle bore #'i quali , volendo licenziare un garzone , gli dicono: piglia il
lembo ; piglia il'cencio , ec. ¢ intendono Vattene,
ef CAFALIVECIO.. Cioe in fw le fpalle. B ndidiciamo portare a cavalluccio
da un giuoco, che fanno i noftri ragazzi in'quefta forma . Vino mette il capo fri
le gambe all’ altro per di dictro , ¢ Sue cosi da terra, lo porta fra le {pal-

le,

 

=. ay
    

146 1 MALMANTLELBE
‘ce, cil collo , e per quefto fii dice,a cavalinecios, Tir
cevano lo diceyano é coryla. fac Y
fopr’ alle palme delle mani del portatore rivoltate dietro.

non accavalciava le gambe al collo , come fanno.i noftri, «
teneva al collo del portatore ; € lo dicevano ix cotyla io
pt

 
 
     
   
 

mano di colui , che portava , come fi cava dal Buleng..
da Cel, Rodig, le, rig lib. 27..cap.27. E quefto.era.
una pena data a quei fanciuili , che baveano perfo
giochi, che habbiamo accennati fopra nel 2. Cantare »..B.
modi , con li quali portavano 5 cosi crano diverfi i,nomi_,.che dava
giuoco ; perche fi trova chiamato Cabefinda ,ed Hippas, fi come fi ved
Polluce lib. 9, . 7. Che guefto giuoco fufle,ufato anche dai L
re da Vergilio En. lib. 2. il quale-dice » che Enea. portd il Vi
padre in fu le fpalle in tal muaniers Lawn seta se N

Ergo age chare pater ceruici imponere

Ipfe [ibibo bumeris , nec anidodhearaen A ikea
allegro , .¢.con buona

 
   

 

 
     
   
  
 
 
  
  
 
 
  
 

 

DELLA buona vozlia, Intendiamo fano.y,

Lalli En, Trau, lib. 1, flan, s1..diffle. >

_ Stanne , diletta.mia, dibuana .

Parafrafando Vergilio, dove dice: Parce merx, E noidiremmo: 4

EVOR di quefta foglia, Cioe fuori di Malmantile , Pigliala {

parte di fotto della porta , per tutto Malmantile; 0. intende foglia

reale.. aria. dgis : B4ine 5 Oa:
STANZA XXX Mica civ STANZA X

Poiche da effo inanimite furo caviond ince
Lefchiere , fi portaron a itor pofti ,

E gid fdraiaco ognun laffo ,e maturo

Ingremboalfonnogliocchihavevapeffiy Mojtrando wwoler farne /
Quanda un trattoletrombe,ed il taburo Segui c-un' Vfirial [uo

  
   
 
 
    

Reppe i ripoff, eifonni appena impofti; ‘Che pik d'agn’altre men

Safi, presto cos) aoe fracafo, Tocco la redone 4 fuoi intern

Ch'il fiatonitrombettier feappo da baffe, De’ tamburini ye tro bert
Dopo che Franconio hebbe dato animo a i foldati ognuno andé,

¢ gia tutti ftracchi s'erano addormentati, quando in un fubito fu dato:

be, ¢ ne i tamburi , che fecero fuegliare tutta la foldatefea ; ma.

prefto celsd , perché i trombettieri , ¢ tamburini.lafciarono ftar di fonar

paura , che hebbero del Generale, il we entrato in collera di cosi gran

giurd di voler gaftigar colui , che era flato il capo di al follevamento, € 10:

dd ad effetto , facendo dare la corda a uno Vfiziale {uo favorito , ch

farebbe mai afpettato , ¢ gli fece mettere i tamburini, ¢i i

' SDRALATO , Diltelo con comodita. Voce da,

confolazione , che'fente uno , che fia ftanco a diftenderfi

eratamente . Vedi forto C. 6. ftan.26. E non crederei

di Cerbero , paraftalando Vergilio ; dove dice

  

  

Loo his 1D. Oe a co fi bs > te os ee Cs

 
 
  
}
i
i
a
i
A
@
ai
y
-

 

TERZO CANTARE: 147
6  teqce immania tered refoluit
ue a Fufies bumi , toroque ingens extenditur antro . xq

‘A Vivato, In un fubito . E quelto termine @ an rrarto fighified anche tutti
due ,0 pili alla volta , ¢ fi pud intender , che Je trombe , ed i tamburi, ciot uno,

eee fiato da baffo at trombettieri, Cafeare il fiato yuol dire Haver paura,
o timore ; onde con quefto dire intende , che i trombetticri hebbero paura’ det
een. di fonare ; non perch veramente perdeficio , o
ulciffe loro il fiato dalle parti da baffo .

YNCOLLORITO... Adirato-. Entrato in collora .

OCCHIO true, Prafe latina ; ulata da‘noi , ¢ fignificay ¢ moftra lira che»
uno habbia ;'¢dicendofi: 11 tale mi guarda ‘con mal’ occhio , 0 con occhi torti,
s’intende il tale @ adirato meco - Hec autem toruitas a taurorum ferocia dicitur ,

MIN ACCIO! col dito’, Coloro che vogliono gaftigare qualche delitto , o ven-
dicarfi d’ alcuna ingiuria , fogliono brandire il dito indice verfo quel tale , ches
vogliono gaftigare , ¢ tal brandimento fi dice minacciare dal Latino Afinari, o wi-
nitari , sur :

CHE pit d' ogni altro meno fe P afpetta . Per effer quelto foldato amico , e molto
in grazia al Generale ; non havrebbe mai ¢reduto, che egli !’ haueffe a gaftigare,
TOCCO! locorda , In’ Birenze danno Ja corda legando il paziente per Je mani
legate infieme dietro alle reni ; ¢ per quelle ran @ un groffo canapo , ches
pafla_per-una carrucola , tirano il p in fu , lafciandolo leggier: feor-
ter in git, ¢ poi ritirandolo in fu'tante volte, a quante ¢ condennato , ¢ quetto di-
ciamo; dare rratti di corda . Qual tormento da“1 noftri antichi era detto dar /a,
volla , 0 collare, ©'noi diciamo : dare la corda . Soggiunge poi : Co’ /uoi intermed) di
tamburini , ¢ trombettieri a’ piedi; cio con tutto quello che ci andava ; il che era ,
che i tamburini , ¢ i trombettieri , i quali erano ftati complici a tal delitto , ftel-
Bp ys! Hr lui affiflenti a vedere efeguire la piuftizia , come fi cofluma_ ,
quando molti (ono complici d’ un delitto , per lo quale vien gaftigato feveramen.
- te il capoy ‘ipale , € gli altri complici ricevono minor galtigo,, ed adiftono a
\\wedere iligaftigo del ‘loro principale, Io perd non fono Jontano dal credere, che
il Poeta per foftenere ‘a faa Opera fempre in fi Je burle., habbia -yoluto in-
5 che i camburini; ¢ trombettieri fuflero effectivamente legati a i piedi di
coini , che era tirato {a , ¢ voglia moftrare con quefto il coftume , che fi tiene ia
Firenze di legare.a’ piedi di tali pazienti qualche cofa , che fignifichi il delitto da
Jui commefio , acid che il popolo comprenda Ja cagione di quel martirio, come
per efempia :;a un fornaio , che habbia facto il pane cattivo, 0 di minor pelo del
dovuto , faranno legare a’ piedi un filo di panc, ¢ cosi gli daranno 1a corda:e mi
la(cio indurre a cia 5 che il Poeta habbia voluro intender quefto , dal vedere ,
che: egli: nel? Orrava feguente dice ; alla corda onole che fia attaccato cost: i qual
detto pare che efprima , che il paziente debba toccare la fune co’i trombetti, ¢
tamburini legatigli a i piedi , :

 

 

T2 STAN-

 
 

 
    
 
 
   
   
   
  
    
   
  
    
   

  

148 : methathethadhalet ities

STANZA Pepe ake:
Alla corda cosi vol ches’ attacchi 5 vstene
Perche a arbitrio ye fenza configtiarfiy aed di
Facea venir all! armtiallor che
Bifogno havean pit di ripofarfi ,
Ed eran mexxs morti; e come bracchi 5
Givano anfando inordinati  efparfi ,)
E con un fuor di lingue,e orrendavifta -

Sofiavan,ch'ioho spp un Alchimifia .

ll Generale fece dar la corda a quell’ Vifiziale non | 20 flo, perch
fol’ arbitrio di far dar’ all’ armi (enga il {uo confenfo ma \ancora
u(cito fuori del concertato , il quale era di offeruare prima dim ;
fe le flelle prefagivano buona ,o trifta forte, B qui il lettore fir
fta in fale burle , ¢ fappia , che I’ Autore non flimayva che I aft
a tanta precognizione , ma fi bene, che idabeaue faa Sepnarig sites

ifti , injilisa 6

D ARBIT RIO xe propria eerrefia Saonahe lo fteflo ; ed  ambesie fi
Di fuo capriccio , 0 volonta...) “th
cANS ARE , E} quell’ impeto 5 o.romore , che fa il relpiro 24
il - ( che noi pure dal Latino diciamoanelare ) e vient a

aCCO, Cane per ufo di caccia, il quale quando é ftracco re

ae , ¢ tiene Ja Jingua fuori; B fe bene fanno cosi tutte Je fp
noftro folito far quefta comparazione /olamente ai bracchi
meate fono pil fortopoftia fracear hij i percio che Rimolati a
di trovar preda, fanno maggiore; ¢ piu violento viaggio che
fio Sat, 1. Nec lingue quantum fitiat canis Appula tantum .
ORRENDA vifta. Vista {paventevole ;, che tale € il veder vat
bocca aperta, ¢ con la lingua fuori , perch Pee: lo pil reftano in
gl’ impiccatiy. 9 a

SOFELAV-AN ch’ io-ho fpoppare we Alchimifta . Alchimiti.fon’
fiano nel fuoco per trovar!’ oro, ¢ fenza nominare Alchimiftay
sale fofia s' intende,, ¢ Alchimifta , Se.bene s' intende , ¢ Bada
cennammo fopra c 1, Nan. 37. anzidicendoli Ji cal. fag, Me bimiffed
tale fa la {pia , ¢ tutto é fondato ful verbo fofliare » che fignitica Far

10 ho foppato Significa io repre meno, 0.io non ‘timo punto il
fanno gli Alchimifti in paragone di-quello , che foffavano quefi
fteflo tignificato,che il termine ne ferede dss pra Ga, 1068 ns
mo fotro C, 6. ftan. 61, ate

TAMBVSS-ARE., Rerquotere y aie dle bl Biparola gei pr
macellari , che dicond Tiamibaffare qu: se
perche Ja pelle fi fpiochi bene salislivaats © ae anc
dremo forto C. 11, flan, 26, E tutto ha Origine dal.ta

 
  
 

che fa effo,s’ aflomiglia al romore, che fanno i macellari 5 ‘
 

ase

SAAR RERERAEEORE ERGEE CE Set

-

TERZO CANTARE- 149

STANZA XXXVI.
Hiomai la Bama, che riporta a volo
| Diagn’ ii nnove , elegazzette ,
_ Spargeper Ma ilche ar mato fiuolo
View per tagliare a tutti te calzetre 5
Gig molti impauriti , ein preda al duolo
Non piie co i naftri legan le foarperte ,
_Ma_con buone,, ¢ faldiffime minuge ,
Perché ftien forti ad wn cumores tage.
STANZA XXXVIL
In tal confufione , in quel vilume y
| All udir queilamenti, ¢ quegli affanni
4 molti cht eran gid dentr’ alle piume
Lo shucar fuori parue allor mill anni:
Chi per veftirfi riaccende il lume ,
Pera ch’ al buio non ritrova i panni 5
«Chi nudo foappa fuori ,¢ non fa fima,
Che dietro gli fia facto lima lima .
Sparfo per

ntile 1" avvilo dell’

STANZA XXXVIIL

_ Rerché s'egli ba camicia, 0 brache,o ve/P4,

Non bada che gli facciano il baccano ;,
Ben si del sriffe avuifo afjlitto refha
Onde pitt a’ un poi ginoca di lonrano 5
Chi torna indierro a fafciarl 1a tefta y
E chi fi tinge con il raferano ,
Chi dice, ¢ una dogisa fegii ¢ pref »
Per non haver aire a far dife/a.
STANZA XXXIX,
Altri, che fugge anch’ ci fimil burrafca ,
Finge l' wnferamo , ¢ vanne allo {pedate,
E benchi [ano ei fia come una lafca
Col medico s' intende , € col /pexiale ,
Perché alt'uno, edall'altrocpielatafca ,
Accio gli faccia fede ch’ egli ha male ;
Ed effi quefho , e quel fervvon malato ,
E chi piis da, do fan dt gid /paccsato .

-Maimai arrivo di detta Soldatefca , gli abitatori
s' acciafero pil al fuggire , che al difenderfi. Narra il Poeta diver-

-diquel lnogo,

Sen tale fpavento , ¢ le varie {cule , ed inuenzioni , che rrovano coloro
sper nion hayer adandare alla difefa della muraglia,.

“ GAZZETTE., Novelle , Avid , Carte d’-avvili. Egazcetra diciamo anche

-lacrazia,Veneziana .

TAGLIAR ie calzetre. Tagliar le gambe . Es’ intende , dare delle ferite in.
pore del-corpo , fe ben le.ca/zere non veftono fe non le gambe : Come
jamo anche romperc ia tetla, ed intendiamo Ferire il nimico in quelle parti
idel-corpojeherci. verra farto.. E diciamo fiaccar le braccia a uno con le baftonate , fe

bene:in ogni altra parce

giidaremo che nelle braccia...

NaST RO . Et una {pecie di tela , 0 benda che non eccede la larghezza d’ un
feltordi-braccio., eferve'per legare , o falciare sda i Latini perd detto, Vitra , ed

in alcuai uogbi d' dralia derco fersuccia .

MINVGE,. Corde da ftrumenti muficalicome Tiorbe , Liuti, ec. fatte di bu-
<dellaidi beftie ;'¢ pero Dante Jaf. c. 28. per intender budella diffe,
wich ili \Tralegambe pendevan le minugia, i
-\) Diceche non 6 fonodegare de {carpe coi naftrt,.ma con le minuge, perche fo-
no pil fode , ¢ da refifter pill ; Ed ¢ coftume nfatifimo il dire: // tale sera dega-
16 le fearpe bene’, o.conile minuge y per intendece Correva forte, 0 volava: fuggen-

«do i pericoli , cheicidiintende con

quella fentenza;, Rumores fuge »

- CONEYSIONE,,¢ vilume., Sono in quetto uogo quaG finonimt havendo Jo
fteflo fignificato di Viluppo , imbroglio , ec.

DENT RO alle piume

Ji quando -vogliono

FAR lima lim Define, dicggiate
AR lima lima , Beffare , dileggiare . |
dar Ja-burla AUN 5

un modo proprio da Fasciulli , i.qua-
fi fregano al dixo indice fapra Iindice

« dell’ alta mano.a guifa’dicoloro cheJimano.s ¢iNoltanioG verlocolui , che ve-

va

 
   
 
 
 

150 MALMANTILE a

glion burlare dicono . Zima , lima . Vedi forto C. 9. flan. 66. annot; ©
WON bada. Non cura ; Non offerva , Non gl’ importa’; Il verb
vuol dire oflervare , ha pid fignificati , come Attendere ,
ligenza , curare , ftimare , ec. Bada a tuoi negozzi. Badaa
viene. In fomma hala forza del Latino Cwrare , Vacare: fi dice: 7
bade, per intender Trattenerlo. Star a bada d’uno: per inteadere
do I opera , i favori ec. d’ uno. ae FF
BRACHE , Calzoni. Brache da noi propriamente fi dicono qu
ghi, che ufano i Soldati a piede Tede(chi guardie del Serenifimo Gri
1 Paggi nobili. B fi dicono taluoita Brache quei calzoni che fi
chiamati ancora Mutande ; Vedi foro C.6, flan.z0, ©
FAR il baccano , Qui vuol dir beffare , dileggiare con fifchiate’yo ft
mili; ed il {uo fignificato proprio é Fare ftrepito , far romore ¢ viene:
nalia, ve
GIVOC A di tontano , Cioe non s' accofta : ed é lo fteflo'che Parfene
che vedremo nell’ ottava feguente cook ta
BVRRASC A. S intende propriamente i! travaglio del mare; ma 10:
per ogni forta di fturbamento , 0 pericolo . Forfe meglio borrafea:

  

 
 
     
 
     
 

     
    
 
 

 
  
 
 
  
    
  
  
   
 

SPEZIALE . Colui che manipola , e vende medicament ; ¢ perd
detto Pharmacopota ; ed altrimenti -dromatarius da aromata ye noi lo
giale da fpezicric , come fi trova anche in Latino aeenaae
TASCA, Scarlella , che ¢ un facchetto appiccato ai 3
ufo di tenerui dentro quello, che occorra alla giornata, e particolarm
ri; ¢il Latino mar/upinm , Ed empier le tafche a uno , yuol tie Dargli
naro. BN
LO fanno fpacciato, Ciot dicono , che egli é in grado di morte ,
ta, che i Medici regolando le atteftazioni delle infermita con le fomm
nari , che crano lor date , facevano fede effer in grado di morte q
ne dava; ¢ quel che ne dava ri atteftavano , che era leggi¢rment
STAN xX. 7
Si che con quefte finte , ¢ con quef arte D' uno fheffo voler La maggior parte
Coffor,c! ufan la taxxa,e non latarga, Trovan la'via di fparfene'all vd n
Seruir volendoa Bacco ,enona Adarte, Ed il reftante non fi afbute 7
Che non fa fangueyma vnol che fifparga, Compari/ce,perch? ei nom puo far alt
Quetti abitanti di Malmantile con tali feufe , ed inuenzioni cercano di for
fi dail’ andare alla guerra , ¢-folo vi'va chi non ha danari, ne da!
berarfene. ‘ 5 ae (eae
TARGA, Brocchiero ,Scudo , Rotella’. Intende 5 che fon pid avvezai a
re che a guerreggiare , ed-hanno piu genio con Bacco ‘Re del vi ~
hanno con Marte Xe delle guerre ; percné quello fa nafcere nel carpo il
equetto lo fa difperdere. ‘ 1b coreakant
ST ARSENE aa larga , Significa non s impacciare d' una cola, ed é.
che gixecar di lontano , che vedermmo nell’ Ottava antecedente. © 5
eASTVTO ,¢ fealtro, Sinonimi di fagace, ed accorto, Huomo, che fa il con-
to fuo. Ma per maggior intelligenza di quelte parole tute 5 ¢ /eakro, ee

 
 
  
   
 
  

 
     
 
  

 
 
   
   
ds)
eile

“

er ihe

 

‘TERZO CANTARE} 352

td dccorto & da fapere che , fe bene ce ne feruiamo per finonimi, tuttavia ci ¢
qualche differenza ; particolarmente fra,/agace , ed a/Puto ; perché I’ arti, che dal-
Ja fagacita s' adoprano , non meritano biafimo , per non effer fe non avvedimen-
tivortili , ma {chietti, reali, ¢ fenza fraude , o ingaani: El afuzia oltre alles
fuddette lodevoli arti fi ferve anche delle menzogne , fraudi ,¢ falfira , ¢ d’ altre
cofe poy scarslepgs nobile . E perd Scaitro, ed accorto pare che meglio s' adat-
tino per finonimi a fagace , che ad a/furo , al quale pid proprio finonimo farcbbe
Maliziofo , 0 trifto , 0 furbo ; quando perd la yoce furbo ¢ prefa in fealo.d’ huo-
mo , che fail conto {uo ; Ma, come ho detto , ncl comun parlar civile nots
ufiamo cosi efacta com »¢ puntualita ; ma pigliamo I’ uno per I’ altro.

eee men Xl, ree’ XXXXIL

lentr? in piarza fi fa nobii ar fa 5 Vilta  arretra , honor ds poi! innita

Auch in Palagzo Meda tans A cimentar la fua ee in guerra,
Con unatreccia avvolta,e Paltra/par/4, L cfortal' una a conferuar la vita ,
Corre alla Mdaimantilica rovina ; L' altroa difender quanto puolaT err.
Benche ne i paffi ape vada pik fearfa , Pur fatto conto di morir veftica
Perchi all’ ufcioda via mais’ avvicina; Voltoffi a bere , ¢ divenuta {gherra

+» Da ferte volte in fu gid s’ ¢ condotta ( Pero che Bacco ogni timer dilegua )
Fino alla fogha; ma quel faffo {cotta . Dice:O de'mici:cht mi vualben mi fegus.

« Mente che la men codarda gente fi raguna in piazza , anche la Regina Berti-
nella. al romore , nuova Semiramide con i capelli non ancora finiti d’ aggiufta-
xe, correa difender Mabmantile ; ma non con tanto ardire , perché quefta noftra
Semiramide non s' arrifchid cosi {ubito a paffare la porta della Cafa; ma fi fermd
in quella , fofpefa ,¢ travagliata da due gran paffioni Poltroneria , ed Honore ,
che quella J’ eforta a ftarfene ; ¢ gquefto Y obbliga ad andare, Al fine lafciataG
perfuadere dall’ Honore prefe animo , ed eforto i fuoi a feguirla .
go MT RECCLA . 1 capelli delle donne fi chiamano #recce, perché per lo pil foglio-
no le donne far due parti de i lor capelli , ¢ ciafcuna di quelle fuddividere in tres
altre parti , ed inteflerle in terz0 , il che fi dice treccia; B Bertinella ftava cosi
Antrecciandole , quando fenti il romore , per lo che lafciato il lavoro corfe cons
una. parte intrecciata,¢ I’ alcra nd, comic dicono ,che faceffe Semiramide , quan-
do fenti il pericolo , che fouraftava a Babillonia .

MA 1a foglia {cotta , Quando uno o per debiti , o per delitti fla ritirato_in ca-
fa , 0 in Chiela , diciamo : Nor efce , perché la faglia feotra ; ciok fe egli ulcitie di
cala., 0 di Chiefa , farebbe fatto prigione: ed a Hertinclla /corra quella foglia ,
perché fe ulciffe di quella , pericolerebbe di toccarac.

VILT A’, Qui vale per poltroneria , o codardia .

MORIR vefiite., §' mtende di coloro , che fono ammazzati , i quali muoiono
con le vefti in doflo , e perd dicendo che fa conto.di morir veftita , s’ intende ches
ella ha rifoluto d’ andar a farfi ammazzare.

SGHERRA, Braya, Animofa; fatta cosi dal vino,che leva di tefla ogni timo-
re. Bacco dai a fu detto ibe, rch liber I’ huomo da i penfieri noiofi ,
€ pero dice ogni penfier dilegua , ed il Chiabreca difle.

ia ssi oe 9 ¢ dianfi al vento
4 vorbidi penficré,

 
 
    

192 MALMANTICE! 27

Seneca de Tranquillit. diffe; 2vonnunguam ad ebrieracémv
gat nos fed ut deprimat curas = Elevat enim curas ,@ ab imo:
norbis quibufdam , ita trifitie medetur , Di tr
Generale del? Imperadore Ferdinando 2. , il quale mai fi
glio di — » ne fi mefse ad imprefa alcuna importante ,
molto bevuto. E Bertinclla imita quefto gran guerriero }

STANZA XXXXIIL ‘ STANZA XX
Dietro a {uoi paffi metteft in’ cammino Piaccianteo uo fe
“Maria Ciliegia illuftre damigella ; t
Tutto lieto la fegue il Ballerino , ci a
Che canta il titutrendo falalella. Ed ¢ la diffrurion

 
  
    
 
 
    
   
 

Va Meo col paggio , Zoppica Atafino , Gid miftro le doppie con la

Corre il Maffelli,e il Capitan Santella, Finiro poi che fu quella bona

Molti,e molt' altri amici la feguiro, 0 omtagit

E pie Mercanti channo havuto il giro, Ed hora in Corte ferne a

Alle voci , ed ordini di Bertinella obbedirono diverfi fuoi feguaéi’
Matti. ses Vie

MARIA Ciliegia , Fu una Donna creduta pazza, la quale andava’
ricevendo elemofina fenza domandarla , Coftei con una flemma ,
ordinaria difeorrendo fempre da per f€ , diceva belle , e fenfate

de da molti non cra flimata pazza , ma uguale a Diogene, che abitava
te ; ¢ per tale azione farebbe ftato riputato matto , fe non have
belle Eiietac Ȣ dogmi , come appunto fece quefta madonna Mai
Ja quale , o parte di effi fono ftaui raccolti da un buon fetterato 5
volta gli dara alle Rampe : Come Diogene,anch’ effa non fi curava
dormiva nelle frad¢ fotto qualche portico o loggia , ¢ percid porta’
pre un granatino per spazzare quel luogo , dove fi metteva a do
{pazzola per spazzolarfi la vefte , 1a quale benché poverifiima , era
molto pulita , ¢ fe bene piena di toppe , atiai bella per efserui le n
mefie forte anche fenza bifogno , con vago , ed aggiuftato ordine .
ta fua {porta haveva ancora qualche biancheria , ¢ molte volte un
danetto pieno di fuoco , nel quale , pa/seggiando per le rade, ai
le fue vivande ; forto Ja gonnella haveva pil facchetti , entro ©
pentola , ¢ piatti per fuo ufo, ¢ quello che le avanzava a’ fuoi man
va forelle , ¢ nipoti i quali fi trattavano comodamente , ed -habitas
buona caforta , che era di detta madonna M aria, dove ella alle volte
mutarfi ; ma non volle mai fermaruifi , ne dormirui ancar che prega
ta anche da’ detti {uoi parenti-a volere flar con loro, Bu(cava mo!
li quali comprava quello , che parcamente le bifognava , ed ogni
va per ? amor di Dio tutto quello , che le avanzava , ¢ per Jo pi
nache , dove alle volte port® anche fino a dieci fendi , _Domandata’
qualche parere , non rifpondeva ; ma feguitando il fuo folito chiaceh
ma, che quel tale i partifse-da lei reflava appagato con qualche fencenaa,
to , che ella diceva a propofito del quefito. Per efempio : Vna mattina ,
clia (otto le logge d’ avanti al Tempio della Santifima Annonziata, un

  
    
    

   

    
 
 

  
   

   
  
  
  
 
   
        
   
    

Bmnwceocacow =~
 

 

TERZO CANTARE.: 153
wie _netto le domandd,fe ella credeva , che la fua moglie bella , da madonna Maria
we, molto ben conofciuta , fufle honefta ; ma gliclo diffe con Ja pit {porca maniera ,
: che dir fi le. Madonna Maria fenza alzar Ja tefta , o dar fegno d’ attenzio-

ne al del giovane , feguitando il fuo difcorfo , che faceva del poco rilpet-

to, fi portava alle Chiele ; dopo molte chiacchiere difle ; Vedete voi quefto
oe sboccato , il poco rifpetto , ch’ ei porta alla Chicfa? La fua moglic &

la , ¢ la prefe, che ella era onefta ; ma che puo ella havere imparato da lui, fe
non il modo di diventare altrimenti? ed hora io ho,che ella fia diventata; perche
ogni gelofo ¢ becco . E feguitd il fuo cicaleccio , entrando in diverfi altri gine-
prai , come era folita ; ¢ cosi chiacchierando tutto il giorno dalla matting alla
fera , bufcava molti denari. Coftci mori , ¢ fi trové nella fua {porta una borfet-
ta , nella quale era una ricevuta di cinquanta {Cudi, dati a certe Monache con,
obbligo di far dire una meffa il mefe all’ alrare della Santis, Nunziata per I’ ani-
ma fua 3 dil che fl-cava-argumcato.. che.clla non fufle pazza. |!) __|

FAL. LA. Cosi ¢ chiamato un contadino trifto,il quale non hayendo vo-
glia di lavorare , s'é dato a chiedere clemofina ; ¢ per far venire le donnicciuole
alle fineftre , ¢ cavar loro di mano robe , ¢ danari , va per le ftrade cantando al-
cune fue ottave amorofe , ¢ ad ogni due verfi fa intercalare con la voce dicendo
Falarera tututrendo , con che fi perfuade d’ imitar il {uono del Chitarriao ; ed all’
ultimo dell’ Ottave,al medefimo {uono della voce , fi mette a ballare , ¢ per que-
fio i) Poeta lo chiama il Ballerino ; ¢ poi va attorno chiedendo la limofina .

4420 . Era uno {cemo di ceruello pro vnSAe dal Palazzo ; e perché egli
fon fi reggeva benc in piedi, pero andava fempre appoggiato a un ragazzo ; e+
Patent Palos ct peers .

SRELEEER ELSSEREPES Shs

AL ASINO . Exa uno ftroppiato nelle gambe , ¢ nelle braccia ; il quale era an-
ia’ i q

ch’ egli provvifionato dal Palazzo per quelia {ua figura cotanto contraffatta das
ale gli te og s
% MASSELLI, Era un mato 20 creduto tale, provvifionato pure dal Palazzo.

Coffui haveva in mente tutte le fefte del anno , ¢ quali Ofizzj , ¢ commemora-
zioni dovean farfi da i Preti giorno per giorno . Sapeya in oltre,quali erano quei
Rettori , ¢ Curati di Chiefe , tanto in Firenze , che nel Contado , i quali nelleo
fefte trattavano bene , © male ai loro definari ; ¢ da effi fi Jalciaya in tali giorni
rivedere ; ¢ mangiava , ¢ beveva tanto, che é impofiibile a crederlo anche da chi
I ha pil volte veduto. Era foprannaturale nel digerire , ¢ s’ é veduto fmaltires

SN

ae

we eerie vt : ¥ .
gran quantita di roba , fi puo dire impotlibile , come farcbbe un gran piatto di
= catia teacsiA bollita in brodo di bue, e condita a guifa di maccheroni ; fee vol-
é te biffo , ¢ tela d’ olanda nella fiefla forma , ¢ quefto in breve tempo, ¢ fenzas
i” __ difficulta , 0 dolori, Li Poeta dice ; Corre il eiaffelis, perché veramente coftui ,
“ benché decrepito , era di gamba velocifima. Haveva ui Serenifs. Gran Duca da-
i to per fervitore al Maflelli un giovanotto gagliardo , perché Jo feguitafle per tut-
i to dove egli andava , ¢ offeruatic tutte le {uc azioni , fenza mai contradirgli , 0
ee impedirlo , ed ogni fera riportafle quanto i] Matielli haveva fatto in quel giorno,
4 Quando i] Mafielli riceveva alcun diigulto da coftui , non s’ alterava {eco , mas»
wy fi metteva la via fra gambe , ¢ fenza mai fermarfi , 0 voltarfi ae meno a dietro »
i

non la guardava a camminare di buonuilimo palio 25., 0 trenta miglia con gran-
digimo

aS!

 

 
    

154 MALMANTILE | 4

diffimo travaglio , ¢ rabbia del fervitore , che non poteva , ne do
conveniva , che lo feguitaffe ; onde andava molto cauto in ftrapazzarlo
ful principio del fuo feruire faceva fino a baflonarlo ) non tanto per
gaftigo da S, A, S. minacciatogli , quanto per il timore , che il Mafie
detta non viaggiaffe .
CAPIT AN Santella, Quefto fu un foldato della Banda di Piftoia,il g
la volta al cervello ( 0 cosi finfe ) perché gli fu rubata la moglie da chi ne}
pit di lui. Coftui venne in Firenze , ¢ vi dimord qualche tempo,
fe pazzie ; ma perché fu conofciuto , che forto quefta fua finta pazzia fi
deva una gran triftizia , fu mandato forzatamente in Candia al feruizio
Veneziani , donde non é pil tornato. {
© MERCANTI, ¢ hanno havato il giro, Cioé gente impazzata. Si ferue
parola Giro per intendere i] girare del cervello , che vuol dire lmpazzare
per il Giro de’ Mercanti , che fi dice , quando un Banchiere tiene in mai
naro di tutta la Piazza; il che in Firenze tocca a fare una volta per uno
banchieri , o negozianti pit groffi per tanti mefi; il che ¢ fatto per co
mercanti ; ¢ dicefi : vere il Banco giro.
P/ACCIANT EO , Fu un Fiorentino di cos) vili natali , che non fi fa
la cafata, ne il vero nome fuo , eflendo fempre ftato intefo col folo fop
di Piaccianteo . Coftui dalli parenti fuoi fu lafciato aflai comodo, ma
lo , che era dedito alla crapula , confumd in breve tempo tutto lo flato
a pena haveva dato principio Serer le milerie della poverta, ¢ gli
la Fortuna di nuovo lo follevé facendoli redare da un fuo congiunto
confiderabile di doppie ; ¢ perd il Poeta dice : Gid mi/uro le doppie con Ia fh
quefte ancora il buon Piaccianteo diede prefto fine , penfando d’ haver
rare il fentenziofo proverbio , che dice ; e4 uno fcialacquatore non ma
denari , Mas’ ingannd , perché ridotto in eftrema poverta, ¢ non
meftiero alcuno , fi riduffe a portare quella barella , con la quale fip
ammorbati al Lazzeretto nel tempo , che fu la Pefte in Firenze , €
tal contagio campé di cotefta fua fatica ; finita poi la pelle viveva
bufcava con far feruizj alle meretrici ; e perd il Poeta lo fa feruitore
la , ¢ fuo Aio, e direttore . ‘Piaccianteo voce che ha dell’ antico Péacent
MANGTAR le cacchiatelle col cucchiaio. \perbole ufatiffima per i
gran mangiatore , Cacchiate/la , E’ una fpecie di pane finiffimo fatto
ed alla grandezga d’ una pera bugiarda ; onde con quefta iperbole , int
che pigli in bocca in una volta cante di quelte cacchiatelle , quante pigli
delle fragole , 0 pifelli , o altra cofa fimile , ¢ cosi viene a ¢flere ipafbae
perché il cucchiaio comune é capace a fatica d’ una fola cacchiatella
ca dell’ huomo difficilmente riceve una fola cacchiatella per volta:
di, che mangiava le cacchiatelle in grandiffima quantita , e fenza m
me non fi numerano le fragole , ec, che fi pigliano col cucchiaio.
ELA diftrugione della Vernaccia , B’ gran bevitore. Vernaccia & una
no bianco , mal’ Autore per Vernaccia intende ogni forta di vino.
AUSVRO’ Ie doppie con lo fPaio, Haveva gran denari. Ipetbole fata |
der un gran ricco; ¢ ci viene dal Latina Agdio pecuniams metitur

  
     
  

 
 
     
   
 
   

e

   
    
   
      
    
      
   
  
   
    
  
 
    
    
   
  
 

  

 

 
 

 

 

Bh

EEN

Tae

eae

as

TERZO CANTARE. 155
4 SON. V.ACCIA. Significa placidezza di mare ; ma noi Ja pigliamo anche per
ogat forta di bene (lare , ¢ di buona fortuna , come ¢ intela a prefente oe ‘
“ BARELLA , Specie di veicolo Gmile alla bara , 0 ferétro , col quale fi porta-

 

ert:

a fotterrare ; ma quelta che ferviva per pertare gli ammorbati cra

no
Coperta fopra con cerchiate , ¢ tela incerata a foggia di calsa tonda di fopra, co-

me i tamburi da via,

Comanda la padrona ch’ egli fcenda ,
_E fia gik fuori com gli ovecchi attenti
Fra (nth fheesfe ch'ei non intends
| Ache fine fon Id cutante genti;
Ma queglial qual no piace tal faccéda,
» Sela trimpelia , af pillein complimenti ,
E, perche at fichi il corpo ferbar vnoley
Ekin crf in quefte , 0 fimili parole,
STANZA XXKXV1,
Alea Regina , perché a Obbedire
Pitt d ogni altro a’ tuoi cenni mi do vito,
Cold n' andro, ma (come fi fuel dire )
Come ta ferpe, quando va all incanto ;
Won cbt 10 fusga il pericol di morire ,
_ Perch? so fo buon per una volta tanto ;
Ma perché ,s*io mi parto , non ti refea
Vet buom,che fappia,dov'eglibalarefta,
STANZA XXXXVIL
Non ti fdegnar, s'io dico ul mio penfiero,
at jibil non é ch’ io tacciaofinga,
E,(e w andaffe it collo , fempr’ il vere
Son per dirti,e chi Pha per mal,ficinga,
Ti fernird di cor vero , ¢ fincero
Senz? intereffe d' un puntal di ftringa y
E non. come in tua Corte ono aleuni
Adulator, che fanno Adeo Raguni ,

 

STANZA XXXXVIIL
Jo dunque che non voglio effer de’ loro ,
ea tengo P adular peffimo vizio,
Soggiungo,e dico, per ridurla a oro,
Che mal diftribuita ¢ quefto ufixioy
E che non pus palar con tuo decora ;
Peicht moftranda non haver gindizio,
Vn tuo dio ne mandi a far la {pia
Quali d' huomin tu havelfi careftia.
STANZA IL,
Manda manda a fpiar qualche Arfafatto,
O un di quei, che pifcian nel Curtile 5
uefio fard il meftier, come va fatto
Senza fofpetto dar nel Campo oftile :
Oftile dico , mentre cofta in fatto
Che cinto ha d’armi tutto Adaimantile,
Tal gente fi puo dire a noi contraria,
Berche nevien quafsh per pigliar’ aria,
TAN f

E perch’ ¢i non vorrebbe ufcir del cova
Soggiunge dopo quefte altre ragioni ;
Ma quella, che conofce sl pel nell'vove,
S' accorge ben,che fon tutte inuenzioni ;
Pero [enza pin dirglielo di nuovo
Lo mands fers afuria di [pintoni ,
Eynitr'ei pur volea vabrogliar laSpagna
Gli fal ufcioferrar fu le calcagna .

Bertinella yuol mandar Piaccianteo nel Campo di Baldone a [piare ; ma egli,
che non vorrebbe andare , adduce mille {cufe ; quali non gli fono ammelse , ed &
cacciato fuori di Malmantile a furia di {pinte ,

_ TRIMPELLARE , Intendiamo quel {uonare adagio , ¢ tentoni 1a chitarra ,
liuto , © altro ftrumento fimile , che fanno coloro , che imparano a fuonare :

da quefto per tri

0 fi fen.

er trimpellare , 0 BIA S
Za profitto,rempeliare che diciamo anche metteria fuk into, o metterla in mufica ,

€ fuona quafi lo {teffo che ,

SE (a paffa in complimenti, Che fignifica Perder il tempo in yane cirimonie ; ¢

fenza toceare la fuftanza dei negozio .

VVOL [erbare il corpo ai fichi.. Vuol veder di viver , quanto ci pud , ¢ nony

metterfi a rifchio d’ efscre ammazzato,

 

Vi VR.

 

a
q
%

-
. = ee

—

  
  
   
 
      
    
  
  
   
   
    
  
  
   
  
   
     

156 MALMANTILE

OBBEDIRE a tuoi cenni mi do vanto, Profelso defer’ il pit o
re che tu habbia, e di fapere intenderti anche aicenni. —
COME Ia ferpe quando va all’ incanto , Cicé mal volenticri ,
Folens nolenti animo, Omero, Ii Lalli En. Tr. C. 2. ftan. 32. dice
Come la bifcia all’ odiofo incanto :
FO buon rr una volta tanto, Poflo morire una fol volta, Quando |
danaro , ches’ ha in tavola , allora che uno ha perduta quella porai
veva , cava di tafca nuovo danaro, o vero dice : fo Laon, cioé pi
fcudo , o per due , fecondo che gli pare ; ¢ s’ intende , che non yuol
la (omma , per la quale ha fatto buono , cioé promefio , Per efempio
no per uno {cudo , ’ avvertario inuita di due , io tengo la pofta, ma
vincere , ne perder pili che uno feudo , perché non fo Fiat di pil.
SE w' andaffe il colle, Se bene io fapeth , che ci fuffe pena la vita, WV
curim in manibus tenens aliquis ceruici effet incurfurus mes , conticerem. —
CHIP ha per mal , fi cinga, Non m’ importa , che altri !’ habbia
fi cinga pur la {pada , ch’ io fon pronto a rifpondergli. Nel primo
dell’ Autore dice fi cinga , ¢ vuol dire fi levi pur da lato la fpada ,
modo io non voglio far quiftion feco. L’ Autore , che fapeva, che
i modi fi dice , ftimo fore meglio detto /i cinga , perché nel fecondo
di fua mano , dice f cinga .
SENZ! intereffe d’ un puntal di fPringa , Non voglio da te cofa alt
che minima: Suona lo fteflo che um puntal d’ agherto, che vedemmo
ftan, 10. ¢ che il Lat. We ligulam quidem .
FANNO Meo Raguni, Cio’ ragunano danari. La forza fa nella ¥
che fe ben pare , che fia il cognome di Meo , é il verbo ragunare , ¢l
meitere infieme ,¢ 44eo € prelo in vece di meus, mea, meum, © vuol ‘
raguni marfupio , cioe raguni alla mia tafca ,
ETENGO I adular pefimo vizio, Non é dubbio , che P adulazione’}
trando , ¢ percid Dante mette gli adulatori nell’ Inferno gaftigati cont
vera pena , che fi legge al C, 18, dell’ Inf. Cicerone nel fuo lib. de
de gli adulatori cosi: lis denique temporthus cavendum eff , ne afs
ciamius anres , neve adulari mos finamus , in quo falli facile eff ; tales enim
wt inre laudemur , ex quo innumeratilia nafcuntur peccata , cum homines
nibus turpiter irridentur , © i maximis verfantur erroribus . Di :
mandato qual beftia mordeffe pid ferocemente rifpofe ; Nelle faluatiche
tore, nelle domeftiche l’ aduiatore , perché con le fue falfe jodi ti
rovine ; Bd aggiungeva ; che'le parole compofte non per aprire il ¥
compiacere 5 fono un caprefto melato . Si potrcbbono addurre
gravitfimi Autori , ma fi lafcia di farlo , perché non torna affatto al pi
fi rimette il lettore a Plutarco nel {uo libro de digno/cendo amico ab
PER ridurla a oro, Per ridurla alla perfezione del difeorfo; Per
conchinfione. Vedi forto C. 8. ftan, ys
COME fe tu havefi carefia a huomini. Come fe ti mantaflero hu
‘to. Ancora apprefio di noi quando fidice: W/ tale oa
buono a quaicofa , feguitando il detto di Diogene Alomintem quare
x2, Corforramini » & virr ofoe, Omero, Viri gfe.

    

  
    
   
   
  

  

 

 
 

 

 

 

TERZO CANTARE: 157

@RPASATTO . Huomo vile , mal fatto , fcimunito , ¢ da poco ; che i Lati-
ni dicono Vappa , Cerdo, ¢ fimili , come fi vede in Plauto da noi in quelto pro-
> citato C. 6. flan. 98. E quefto nome d’ Arfafatto viene da Arfaxad
della fcrittura fagra , che nel barbaro fecolo non effendo dal volgo inicio, fu
refo per und Babbaleo,o Babbano. —-

Dl quei che pifciano nel Cortile . Pifciar nel Cortile vuol dire Far la fpia , ¢ que-
flo , perch coloro , che fanno Ja {pia , cfiendo veduti entrare , ¢ ufcire del Pa-
Tazzo della Giuflizia , hawno qualche roflore , e perd eflendo veduti da alcuno
Jor conofcente , fi fermano nel cortile di detto palazzo a pifciare per fcufa. Si
pud anche dire , che il verbo pi/ciare fia prefo in fignificato di buttar fuori, ed in-
tendere che pifeino , cioé buttino fuora quello che {anno nel Cortile della Giutti-
zia , ove é la Cancelleria del Bargello , nella quale le {pie portano le denunzie .
Si pud anche far refleffione , che detto Cortile fla fempre pieno di Sbirri , i qua-
Ji fon’ anche per lo pit {pic , e vi fono due pifciatoi {pefifimo adoprati da loro ,
ed intendere , che venga da quefto il detto Pifciar nel Cortile . Ma fiacome efler
Gi voglia , I’ effetto ¢ , che pifciar nel Corriles’ intende comunemente, Far la {pia .

CAMPO oftile. Campo nimico, Dice che ¢ campo oftile , perché ofta ; ¢ fa.
na(cere il bifticcio dalla parola offile , ¢ dalla parola cofa , la quale nel parlare>
pare che dica che v/a, che yuol dire s’ oppone , ¢ fa oftacolo , facendola di duc
dizioni , ciot che , ed offa , quando é Puna {ola , cioé coffe dal verbo cofare 5 cho
vuol dire Effer manifefto . Modo ufato da Franc, Harbarino ac’ Mottetti ,

NON vengon quafsh per pigliar’ aria. Vengon per altro fine , che per andare a
{patio , o pigliare aria . Detto ufatiffimo per intendere uno , che vada forto al-
ti pera in qualche luogo , ¢ fia poi per negozio importante , ¢ per cavar uti-
Je da quella gita ; che i latini diflero: Won fine ratione lupus ad urbem . E noi pu-
Fediciamo : OQusfta cofa non ¢ fatta fine quare. Vedi {orto C. g. flan. 11,

CONOSCE ii pel nell vove . E' fagace, ¢ aftuto, ¢ fa confiderare ogni minuzia:
forle é quello , che i Latinidiflero: Ventura per dioptram pro/picit ,

A furia di {pintoni , Con quantita grande , ¢ {pelsa di fpiate , che tale é Ja for-
za della parola faria in quefti termini forfe dal Greco Phura, che yuol dir’ abbon-
danza , © moltitudine , Vedi orto C. 9, ftan. 49.

LMBROGLIAR la Spagna, Quand’uno s’ aftatica con chiacchiere fuor di pro-
pofico per divertire uno dal priacipiato difcorfo , per non gli dire quel che egli
vorr fapere , 0 non fare quel che cgli ¢ impofto diciamo; Egdé tmbreglia la,
Spagna,

PERRAR 2 ufcio in Ju te calcagna, Vuol dir Scrrar’uno fuori della porta. E? il
contrario di dare dell impo/ta ful moftaccia,che vedremo foto C. 10, flan. 27., che
vuol dir proibire I’ ingretio a uno che venga per entrare ; € quello yuol dire Ob-

biigar uno a ufcire .
STANZA LL
Sperance refta alla Regina intorne La pala nella defira tien del forne
Spianator di pan tondo riformato ; Wella finiftra un bel teglion marmato
Gridan elle remo,e Livorno, dn cambio di rotella, chagli guarda
Ed ha un Co. che pare up vicinate 5 Da j colpe il magaxrin della mofrarda,

STAN-

 

 

 
 

 

  
   
 
 
 
 
 
 
 
 
 
 
 
 
 
 
 
 

158 MALMANTILE ~
STANZA LIL ‘4 STAN
De i Rovinati anch’ ei pafso la barca , ;
Perche la gola,il ginoco,e il ben veftire
Gii baveano il pane, la farina,e Parca
Jn fumo fatto andar come elifire 5
Tal che,cantando poi,come il Petrarcay
a2 CAmore io fallo, € vecgio il mio fallire,
Ail ginoco del barone , ¢ alla baffetta
Giocava,apparecchiando alla Crocetta,
STANZA LIII,
Fu dalle dame amato in generale ,
(4 dico dalle prime della perza)

Poi Bertinella ffavane si male , Gi dal ufizio,e
Ch’ ella fece per ini del ben bellezza, Crn la folita faa p
Perche [pefa la rola, e concra male, Perch sin quéfto cafo\aleun
Fatta pit bolfa a! una pera TUERZA y Sifcnopre , facil fia, farie pri

Potea dt notte, quanto a mezzo giorno, dccid ful lerto poi di B. «ch
eAndar ficura per la fava al forno . Se gli facia ferrare il m
Partito Piacciantco refta appreflo Bertinelia Sperante ; quefto era B
fai comodo ; ma tra il {uo mandar male >» ctra l'effergli ftata fara
tega , fi ridufle anch’ egli maliffimo , ¢ nondimeno non ufciva tai di
retrici , dalle quali yeramente cavaya il yitto , perché effendo bell” hi
efle amato , ¢ fe ne fervivano per bravo, e per ogni occorrenza loro:
fto il Poeta lo fa configlicro , ¢ Bargello di Bertinella,

SPERANTE . Cost veramente haveva nome coftui , ¢ faceya il
Fornaio , ¢ perd dice Spianator di pan tondo: E lo dice riformato, p
bito a quei rempi il fare i tondo (che cosi fi chiama ij pil n
faccia in Firenze per il pubblico ) in rgnardo dell’ appalto ,che fu |
fta forta pane; ¢ perd gli conuenhe ferrare la bottega , Ci é perd.anel
zo dell’ equivoco , perch /pianarore di pane yuol dire Colui che fa il
fignifica ancora uno , che mangi molto pane. Vedi forto C. 6, fta
fi pud intendere gran mangiatore di pan tondo , ma riformato ;
pud pil mangiar tanto , per non havere il modo da comprarlo
mine militare , ¢ s’ intende quel foldato , che é privato della
vea ; che fi chiama poi Vfziale riformato , ‘ 2 14g

GRID AN le {palle fue remo, e Livorno, Ha {palle cosh grandi ,
‘rate a Livorno per mettere a un remo di galera. Queftogridare, ec, &
dire,che ha lo fteflo fignificato,che Chiamar di id da’monti, Visto fopra (

Van C..,. che pare un vicinats. Haun C..:. grande quanto uaa
Tperbole ufatiffima per denotare un federe eflremamente grande j-¢
intendiamo una contrada, - AS : ula

TEGLIA marmata , Coperchio fatto di marmo minutamente pefto,, ¢ ter
col quale , {endo infuocato, fi cuoprono le teglie, © tey er rololare le
de: edé forfe il Latino clibanus ; che per altro yuo) d a
cotto , fe crediamo a Pietro Viloa Vita di Carlo V.

 
    
 

  

 
 
 

 
 

 
  
   

 
 
    
    
 
  
      
        
    
   

 

   
  

   
    

   

 

  
 

TERZO CANTARE: 159
IL midgaezino della moffarda . Cioé il ventre . AsoParda & uno intingolo fatto
y ofto cotto , ¢ fenapa 2 ¢¢. ma qui ¢ prefa ( come da molti ) per quella roba.,
a che fla nel ventre per qualche fimilitudine che ha quell’ efcremento col colores
della ¢ magazzimo diciamo una ftanza deftinata a riporui,¢ conferuar.
vi Spagna. almazén .

_— PASSO’ ta barca de’ rovinati . EF nel numero de’ poveri.
is ARCA, Voce latina , che yuol dir Caffa in generale, ma noi intendiamo {pe-
at

  

 

cialmente quella gran madia , entro alla quale  Fornai tengono il pane cotto , 0

FATTO andar’ in fumo d’ elifire. Fatto andar male fenz’ alcun frutto appun-
to come fa I’ elixire , che lafciato in un vafo aperto fuapora , ¢ fi difperde .
nem AL Barone , ¢ alla Bafferra , Sono due giuochi noti, i primo di dadi, e l'altro
16% — di carte; ma qui fcherzando yuol dire , che cra divenuto Barone , cio’ mal velti-
i to, guidone , € ridotto al baffo , che vuol dire Impoverito ; traslato dalla botte,
van che fi dice efer’ al bao quando il vino che v’ ¢ dentro é alla fine , ¢ che la botte ¢

| quafi vota .
. APP ARECCHIA alla crocetta, Vuol dir non haver da mangiare. Far degli
(ee — sbavigli fignifica non haver da mangiare . Vedi fotto C. 4. ftan. ultima. Ed ef-

 

wil’ {endo i¢ di molti nello sbavigliare farfi 1a croce col dito pollice incontro }
uk! alle fauci, pero far le crocette intendiamo ftare a bocca aperta , e vota , che in fu-
ii@ — ftanza vuol dire non haver da mangiare , Qui il Poeta rende il detto pid ofcuro,

1? = € pill coperto dicendo apparecchia alla crocetta , che & un Conuento di Monache ,
iB nel qual uogo par che voglia dire , che coftui defini , ¢ ceni : che quefto fignifica
il verbo apparecchiare , quando ¢ meffo affolutamente , ¢ fenza aggiunta .
ie PRIME dela pexxa. E’ Jo feifo che di prima Claffe , o paffar per la maggiore
detto fopra C, 1. flan. 6. p
ST AVANE male, Tribolava per |! amore , che gli portava, Era grandemen-
4 te innamorata di lui, Latino deperibar .
a FECE del ben bellezza. Cioé {pefe , ¢ confumd , quanto ella havea , Havendo
4 confumato tutto il fuo bene , Je rimate folo Ja bellezza , o vero fece bellezza , ed
nf? ene ogni {uo havere . E’ quel Procerusam facere, che vedemmo fopra C, 1,
iat D,
2 BOLS A . Mal fana per troppa umidita, ¢ ripienezza . E perché quefti tali
1i@ elf foglion effer per lo pid ripieni di carne liquida, ¢ di colore fra il verde, ¢ il
giallo , gli paragoniamo a una pera troppo matura , 0 fracida , che quefto vuol
ol dire pera mezza. Virg. mitia poma ; ciok maturi ,

we POT EVA andar ficura , ec. Quelto fi dice d’ una donna vecchia , € brutta , in-
jc _tendendo, che ella ¢ ficura di non effer rapita .
a LEZZO, Puzzo., Fetore, Propriamente /ezzo ¢ un’ odore che difpiace , il

pi quale non nafce da corpo corrotto , come ¢ quel puzzo,, che nafce da una carne
troppo frolla , 0 altra cofa marcia , o fracida , che fi dice ftantia ; ma ¢ odores
yt Raturale , © procede da fudore  o da altra evaporazione , che getra un corpo,
nt beaché non fia corrotio , onde quello che fi fente dal becco, ¢ dalla capra vivi, {i
ya! _ dice lezzo , e quella che fi fente ca i medefimi quando fon morti , € corrotil fi di
| ce puzzo 0 fetore , 0 fito di ftantio. Vedi fopra in gucfto C, fan. 24. Queite
a.

 
——

|
4

 

 
 

16° MALMANTILE,,

Jezzo,cosi dda olezxo,¢proprio quello,chei L.dicono Virus,Not
veleno, morbo, ferore,, he 1 ¢ fimili pigliando I’ uno per I’ al

che I' altro ¢ vocabolo di mezzo, perché tutti fi poflono
re, come fi cava da Caio Lurifconfiulto : Qui igitur ( dice egli )
ber adijcere utrum bonum, an malum . E Statio lib, 2, Syluarum:Atque
Virus , odoriferis eArabum ; quod crefcit in aruis , Noi ancora diciamo
purzo di mufchio ; [a dé mnfchia ch’ egli avvelena . Gls ammorba d'ambra
to ch’ egli attoffica , ec. sue

PASCIONA . Intende Comoditi,c abbondanza d’ ogni cofa neceffari
to , fe ben pa/ciona yuol propriamente dire Il pafcolo delle beflie .

N’IMP.AZZA affatto . E! di tal maniera innamorata di lui , che ha
ilcervello. L, efflittim , perdite amat ,

NON (a vede a mezzo. Non gode la vifta di lui alla meta di quello,
rebbe ; termine, col quale s’ clprime l affetto grandiffimo , che uno
un’ altro, Won veder pit avanti; ne pik qua , ne pin la ; usd il Boce,

SALAMISTRA, Maeltra di fala. Ma iol intendiamo una donna
dottorefla , affannona , ¢ fimili, ma per derifione , diciamo ALadonna Sa
Qui intende direttrice del governo ; ¢ la chiama Sa/amiffra pur per di

V-A in capo as liftra , Cioe toltone Bertinella, ¢ Martinazza eglié il
il primo huomo che fia in Malmantile ,

£ DI nidio, E' trifto , E’ aftuto fino dalla culla., e4 incunabulis

Noi pigliamo quefto detto da gli uccelli cavati dal nidio, ed allevati
V uccellatura fon fempre migliori , che i preficel ,

NAVICELLO , Vuol dir huomo lefto, ¢ che fa tutte le furberie’, che
sa navigare a tuttii venti. Ha lo fteflo fignificato che effer di nidio .

JL letto di balocchino , 5’ intende le forche . Da un tale detto Balo
fu impiccato in Firenze al Canto alle rondini per ladro di beftie , delle
Senfale , ¢ fi chiamd anche il Parola. Vedi fotto C, 6, ftan. 67.

" SERRARE il nottolino . Vuol dire ftrozzare: intendendofi per Nottolit
parte della canna della gola, che vulgarmente chiamiamo gorgorzule, €Qe
Ja fimilirudine , che ha nell’ andare in gii,e in fu,quando ’ ing hiotti ce

   
  
  
  
  
   
  
   
   
 
    
   
    
    
   
 
   
   
 
   
  
 

     

in git , ¢ in fn delle nowole da ferrar porte , ec. 2
STANZA LVL ee
Fa in tanto nel Caftel tocear la cafsa, Ch’ in fretta alla raffegna fe
” Einalberar U infegna del Carreccio, Con le {chiere pero fatte a bal
E comandante elegge della malfa Che ad una ad una
U nobil Cavalier Mafo di Caccio, Sotto /ua guida,e [orto {ua t

Bertinella fa toccar tamburo , ¢ inalberar I’ iniegna generale , e d
nerale della fua gente Malo di Coccio , il quale {ubico fi metee a far Ja
ed accomoda tutti i foldati fotto i fuoi Capitani , e Comandanti .

CARROCCIO . Quefto era anticamente un gran Carro di figura qu:

ra il quale s' inalberava appiccata a una grande antenna ! infegna
della Signoria di Firenze , ¢ fi metteva fuori in occafione di trionfi, 0 |
Fiorenuini ufcivano in campagna alla guerra con cfercito formato , ed é
flefio Carro, ¢ della ftetla Sgura,e grandezza quello, fopra il quale fi ,
ii Palio di S, Gio; Bauita , MA

 

 

 

OS eee a a a

Ss peer
ty

E PELLESE

WREARE BEALEE

TERZOQCANTARE, 161

. &€ASO di Coccio, Tommalo di Coccio fu un Pefcivendolo huomo ficro , ¢ di
gran feguito di {noi uguali , a i quali egli in tutte I occafioni di fefte , cacce , ed
altre cofe fimili comandaya come a’ fuoi {eruitori , ed era beniffimo ubbidito da
chi.per genio , ed affetto , ¢ da chi per timore, ¢ perd il Poeta lo fa Generale de’
foldati di Bertinella , che fon tutti di condizione fimile a !ui , come vedremo.
Lo dice mubil Cavaliere , perché in Firenze egli era conolciuto,e nominato pitt che
qualfivoglia gran Cavaliero .

4A BABBOCCIO . In confulo ,.a cafo , e fenza confiderazione .
STANZA LVI

Fi primoé il Purba nobile Bradiere , 1 \fateude il Vecchina il gran Barbiere,
Che non ginoca alla buona,e meno a gofi, Che vnol chrogni hor fitrinchi,e fi sbafof,
A noccioli bensi fifa valere E dove 4 menfa metter puo la mano,

,

Perch ci da benet buff ,¢ meglio i Sofi. Si fa la fefta di San Gimignano,

Al Poeta mette in quefta raficgna una mano di piebei noti per qualche loro
azione © buona , © cattiva , ¢ gli nomina con i loro foprannomi . Ii primo é il
Furba firadiere , cioé uno di coloro , che alle porte della Citta cercano i patseg-
gicri fe hanno reba da gabella , i quali pizzicano di {pia ; ma quefto Furbo era
anche in effetto fpia . li fecondo ¢ il Vecchina Barbiere .

ALLA buona , ed a gofi . Sono due giuochi di carte afsai noti: ma con dir cost
intende , che coftui non era ne buono , cioe femplice , ne goffo , cioe corrivo.

A NOCCIOLL hen sit, Gia che il Pocta porge Ja congiuntura di narrare , qual
fia appeelso a inoftri Ragazzi il giuoco de’ noccioli , ed in quante maniere fi
faccia,, il Leteore fi contentera , che io {pieghi con un poco di digreffione i mo-

i 5.€¢ i fi traftullano i noftri Ragazzi a quefto giuoco de’ noccioli , ¢ non
fi idegnera di volgere gli occhi a leggere il difcorfo di quei trattenimenti , a’qua-
Ji,non idegnd.di volger !' animo , ed impiegar l’opera un Cefare Augufto, fecon-
do che riferifce Suctonio Trang. riportato , ¢ confiderato da Alex. ab Alex, dicr.
Gen. lib. 3, cap. 24. ¢ ricordandofi che tutta quefl Opera é fatta per i Fanciul-
Hale » che.per quelle perfone , che gia reliquerunt nuces , haura la bonta di con-

»fenon per nece(saria , almeno per non affatto fuori di propofito tal di-
greflione « Dicodunque-che il ginoco , che fanno i noftri Ragazzi co’ noccioli
di ( Cofiumato anche da 1 ragazzi Greci , e¢ Latini, che lo dicevano ladus
acellatarum, fecondo i| Buleng, de Lud. vererum , & Alex. ab Alex. dier. gen, lib. 3.
cap. 21, ade di cui parole poco apprefso riporteremo ) ¢ ufato in molte maniere;
ma fpecialmente giuocano , 4 Cavalea , alle Cafelle , alla Serpe, a Ripiglino, a Shree
Seid , 4 Cavare ,.aShricchi quanti, aTruccino, ed alle Buche. Di tali giuochi,e
ai ciafcuno di edi narreremo ii modo , che tengono a efercitargli, ¢ diremo qua-
li Geno fimili , 0,gli Mei, che erano ufati da gli antichi, |

A cavaica. S’ accordano due o pit , ¢ tirano fopra un piano i noccioli a un,
per uno, ¢ tanti ne feguitano a tirare , ane ftieno a far falire fopr’ agli altri
trati un nocciolo che fopra vi refti , ¢ fi regga fenza toccare altro che noccioii;
€ cojui che ha tirato il nocciolo rimafto fopra , vince , ¢ leva via tutti i noccioli
tirati. Lo dicono a Cavalea da quel cavalcare, che fa il nocciolo fopr’ a gli altri,

ALLE Cafelle, 0 Capannelle .. Mettono fopra ad un piano tre noccioli in trian-
golo , ¢ fopra dieii.un’ alteo nocciolo , ¢ rs maiia dicono Ca/ells., 0 Capan.

nella ;

 

 

 
| 2 9

* Giulio Polluce lib. 9.c. 7. moftra che faceflero quefto giuoco ancora

_ al quale & toccato in forte,deve,girando in rnota con quello’

    
  
  
  
   
   
    
  
   
   
    
  
   
   
   
   
 
   
   

162 MALMANTILE™

nella ;¢ fatto di éffe il numero tra loro conuenuto, ed
concordata , tirano in dette Cafelle un’ altro nocciolo 5 ¢ colui cl
vince cutte quelle cafelle , che fa cafcare col colpo . Quefto fu wi
gli antichi , ¢ dicevano Ludere Caffello nucum fecondo il Buleng. C.
{elle vengono defcritte da Ovidio in Nuce in quei veri M
amplins, alea tora off 5 Cum fibi fuppofitis additur una tribus ,
ALLA ferpe . Fanno una di dette cafelle , la quale figura po d
da quella fanno partire un filare di noccioli , che figura il refto del corp
ferpe , e poi vi trrano dentro con un’ altro nocciolo , € chi fa col tiro
uno , © pili noccioli del tutto fuori del detto filare , vince tutti lino
fono dalla rotwura in git verfo la coda di decta ferpe , ¢ durano cosi , fino ache
fia rovinata da un di loro queila cafella , che figura il capo della ferpe.
pure era ulato da i Greci, ¢ Latini , ¢ forfe facevano co’ noccioli altre fig
come fi cava dal Buleng. Cap. 8, , dove fi vede , che in vece della fei
co i noccidli un triangolo equilatere , o [ come dice egli } il delta &
ARIPIGLINO, Pigliano quella quantita di noccioli , che conuel
randogli all’ aria gli ripigliano con la parte della mano oppofta alla
in tal’ atto fopr’ alla mano non refta alcun nocciolo,colui perde la gita, '
colut, che fegue ; ¢ cosi fi va feguitando fino che refti fopra detto luo
mano qualche nocciolo, ¢ quelto al quale ¢ rimafto il nocciolo,dee di qui
Jo all’ aria , ¢ ripigliarlo con la palma, e¢ non lo ripigliando perde la git
reflafle pi d’ uno fopra alla mano , pud colui farne fcalare quanti:
che ne refti uno ; che fe non reftafle , perde la gita. Ripigliato il
conda volta , deve coftui tirarlo all’ aria , ed in quel mentre pigliare
de i noccioli cafcati , ¢ con effi in mano ripigliar per aria quello che
seguendo , pofa i noccioli prefi , e perde la gita ; ¢ fe ne ha pigliati
fenza fare errori , reftano fuoi , ¢ fi (eguita il giuoco fino a che fiend

diflero Pentalitha , perché ulafiero di farlo con un numero det
faflolini , 0 aliofi .
SBRESCIA, E lo fteflo , che ripiglino , fe non che nella

vonfi ripigliare quei noccioli , che cafcarono in terra la feconda volta

uno, o due per volta , ma tutti a wn tratto { il che fi dice fare sb

dovene pur’ uno, 0 cafcandogliene, perde Ja gita, e cosi fiva feguitando,fia
uno pulitamente gli raccolga tutti. Sd

CAVARE, Infilano un nocciolo con una fetola di crine di ‘eavall
ual fetola ridotta in forma di campaneila’, o anelletto legano uno’!
fegnato un circolo in terra , vi mettono i noccioli , che fon d' accord

  
  
 

filato,a tal girare,buttar con effo nocciolo fuori del circolo uno y © pil
di quelli y che fon dentro al circoloy ¢ vince quelli , che cava je
the gira , tocca terra , perde Ja gita; ma guadagna i noccioli eavati, eda
ciolo da girare a un’ altro. E cosi fi va {eguitando fino a ¢he fien
noccioli, Similmente nel giuoco detto da’ Greci Eis amillan delerit
¢ehio , dentro ’l quale perd fi doveva buttarel’ aliofio.in mani

  

 

 

ian ite deen cna:

2 ante

ASL Be 2 62g & 2 eo pop
i
i
a
y
it
=
:
i‘
re
:
‘
of
i

a

as

nll

ad

vi

&

“4h!

va
a
yi

v
é
a

%
9

wiceeimidi _—

TERZO CANTARE. ¥63
fe,¢ non ulcifle di detto cerchio. Appreffo di noi anche negli Alioffi fi fa aca.
vare , Canti alcialefchi; Perch’ al cavare un’ elioffe bruito , ec,

_ SBRICCHI quanti, Occultano dentro al pugno , o dentro ad-ambe le mani
ita ioli , che vogliono , poi domandando ad altri, che indo-
vinino ni e’noccicli occultati , ed indovinandolo vince tutto , fe no; de-
ve dare quel numero di moccioli , che ha detto di pil , 0 di meno; E queflo fi fa
una uno, dovendo il primo , che domando.far’ anch’ egli domandare,
¢ cosifi va continuando i giuoco.. Quelto sbricchi quanti & lo fteflo , che pari, o
cafto, nel fi domanda , fe il numero é pari , 0 caffo , ¢ chis’ appone vince
tutti li noceioli occultati ; fe no , perde altretcanta fomma. I Latini differo : /u-
dere par impar . LGreci artiazcin, Di quefto giuoco parla Giulio Polluce fopra
citato , ed il Meurfio de /adts vererum , i quali moftrano , che fi faceva , comes
pure oggi fi facon i danari, econ altra materia , come mandorie , ¢ fimili , at-
ta a:poterfi accomodare dentro alle mani, Ovidio in Nuce. Ef etiam par fit nu-
merus qui dicat.,.animpar Vt divinatas axferat augur opes .

A TKYCCINO . Vno tira un nocciolo in terra , ¢l’ altro tira un nocciolo a,
quello , che @ in terra , ¢ cogliendolo vince, fe no; quello, che tird in terra il
primo , raccoglie il {uo nocciolo , ¢ Jo tira a quello , che tird ! avverfario , ¢ cost
continovano., ¢ chi coglie vince il nocciolo che coglie , 0 quello che fieno conue-
nuti . Ex fimile al ginoco detto da’Greci Streprinda.

ALLE buche . Fanno diverfe buche in terra in giro , formandone come unas
rofa , nelle quali tirano i noccioli, ¢ colui vince, che entra in una di dette buche,
quella fomma , che ¢ prezzata quella buca,nella quale entrd il fuo nocciolo : per
efempio le buche fono fette , la prima che ¢ volta verfo donde fi tira,che é la pil
facile a entrarui non fa vincere,non efiendo ta(sata in cofa alcuna , ¢ da i aoftri
fagazzié detta la buca del Niffo ( forfe da wibil ) E dell’ alere una vince tre , una
quattroyec. EB percid ho detto, che vince chi v' entra quanto é preazata la buca,
€ poi va.con gli altri ad aiutar condurre il nocciolo nella buca a colui,che al pri-
mo tiro non v’ entro , € {pingendolo di dove ¢ alla volta delle buche col dito in-
dice ( che dicono limare ) . Ovidio ut pronas di

igizo bifue femelue ee > 0 col buf-
fare , 0 col foffiare nel nocciolo,¢ ¢ la differenza da buffare a fo

fiare vedremo
poranepeeliy ).nel che adoprano ogni arte per difficultare all’ avverfario il con-
jurre il nogciolo dentro alle dette buche ; E cosi facendo a una volta per uno a
limare , buffare , 0 foffiare , colui vince , che ha fortuna di condurre i! nocciolo
dentro a una di dette buche , ancor che il nacciolo fia degli avver(arj . Similes
al fare alle buche & quel d’ Qvidiq. Vas quoque fape canum {patio diffance locatur , In
quod. milla levinnx cadat una manx , Banno quelto giuoco ancora con una palla, ¢
giuocano danari, come vedremo forto C, 8. itan, 69. alia voce Alinfo. Edé fimi-
le quello che i Greci , fecondo Giulio Poll, lib. 9. c. 7. chiamana pherinda : © {e~
condo il Meurfio de Lud, Grae, alla voce Apherinda , & alla voce milla, ed il
Buleng. cap. 14. € go, Se bene tanto nell’ dpberinda quanto in quello, che fi chia-
mava Eis amillan ; tiravano ia un circolo, ¢ non nelle buche. Alla buca bens}
tiravano in quelltaitro detto Tropa, che corrifpondeva a quefto nottro . Conchiu-
do we » che la maggior parte di detti giuochi crano ufati anche da gli an-
tichi ; & fe ben pare , che fi {eruitiero delic ae > 10 non {oa lontano dal crede-

2 te,

 

 
we

   
  
 
 

ey MALMANTILE ©

re, che la parola Nwces voglia dire ogni forta di nocciolo ,
lib. 15. cap. 21. , dove mette in dubbio , fe ie noci in :
ancora arrivate in Italia ; ed oltre a quefto trovone i gla
ed ardirei pero affermare, che ancor’ effi adoperafsero noccioli di p
(come fanno anche i ragazzi de’ noftri teaspi ) alle volte noci, ¢
cioli di pefca , feguitando Alex. ab Alex. lib, 3. c. 21.5 che dice
Gos viros fuper nucibus ocellatis einfmodi , qus effent, ancipitem ditt cogicationt
Se , variaque in opinione verfari y © alias nuces avellanas , alios amygdalas pa
neque fatis ratam fententiam ferre fuper Tranquilli verbis , quibus Ang ds
animi canfa cum pueris facie liberali ocellatis nucibus lnfiffe dict.
mus , & probabilins putamns id ef : Einfmodi nuces ocellatas nucleos 5
pomis fitos infpicimus dicamus effe, quibus perfape Iudere noftrares pueros d
dittafque ocellatas propter ocellos , & foramina , qnibus muniuntur undiqne
ansyedalas , aut avellana , ficut error haber 5 fed de perficorum offibus , quibus |
debatur 4°& nunc frequens puerorum Indus eff , intelligi conuenire credumus
© non umbigua fenrentie fore. Dalle quali parole s’ intende , che
cora fi ginocava a quefto giuoco de’ Noccioli , Ovidio de Nuce,ct
verita, € moftra che havefsero molti de’ fuddetti giuochi , 0 poco d
Marziale attefta,che crano gli fteti genj ne i fanciulli de’ {uoi tempi, ct
d’ oggidi , ¢ che il portare in tafca noccioli caufava a quelli delle maz;
fegue ne i noltri , dicendo ; y 4
Alea parna nuces ,& non damnofa videtur;
Sape ramen pueris abjpulit illa nares
Ec altrove . Zam triffis nuctbus puer relictis
Ed Horatio, ?offquam te talos , Aule , nucefque
Ferre finu laxo vidi y ec,
Sono dunque , ¢ furono fempre puerili tutti li fuddetti giuochi; e
biamo un detto di difprezzo ; Va 4 giuoca a’noccioli,che fignifica Tuo
gior giudizio di quel che habbia ua fanciullo : Qual detto cra ufato
pure , come fi cava da Perfio Sat. s.
Et nucibus facimus quscumque relittis i
E dicevano reliquit nuces d'uno , che dalla puerizia paflava a mani
rie; Dal che potrebbe argumentarfi , che 11 Poeta dicendo , che
ca bene a i noccioli , intendefle , che egli fufle huomo di poco giudizio, e cher q
nucibus imcumbat ; Ma fi conofce , che non intende.queflo , perché prima
Non ginsca alla buona ne 4 goff , fignificando che non era ne buono ne
ora col dire , che egli ginoca bene a’ noccioli , perchéda bene i buffi, ¢
vuol — ben la {pia , che baffare , ¢ fofiare vuol dir Bar la {pia
C. 1, flan. 37. 5 «
BVEPI fh. Buffo ¢ un fofiare non continuato, ma fatto:a un tratto - |
fi farebbe a {putare , 0 a profferire la parola buff , donde buferd., 0 bufea un grat
nodo dj vento , che paffa prefto , Sofiodun foffiare con la bocca:tanto quanto
ud durare fenza ripigliare il fiato , ¢ cid dico per moftrare Ja differenza'
Ee buffo, ¢ foffa ; che per altro sd che fof & generico , ¢ comprende og
sompimento d' aria fatto col fiato di che che fia, dicendofi /ofiare y

 
 
   
  
 
   
 

   
   
     
 

   
 
  
 
   
   
   
   
   

    
   
     
     
   
   
 

a

Sreiteetes

Sat

3 3%

Se

cARE

aee

- TERZO CANTARE;: 165

vento , Che manda fuori il mantice , /offare fidicono i Venti, ec. Vedi fopras
C. 1. flan. 39, la voce rabbuffo,

1L Vecchina , Era tin barbiere cosi chiamato , i] quale ogni fera andava ricer.
candoiper Poftérie le conuerfazioni , che erano a cena, ¢ trovandone di fuoi ami-
ci, con varie'chiacchiere poco a poco fenz’ effere inuitato fi metteva a federe ,
© mangiava’, € beveva quanto pit poteva , ed al far de’ conti fen’ andava fenza
feeds era comportato , é faceva il buffone ; Procurava, che

conuerfazioni di cene fi faceffero in a fia , dove apparecchiava, e prov-
vedeva affai pulicamente , ¢ bene, ¢ con {pela aggiuftata faceva ftar bene,e avan-
zava tanca roba per’ fe da viver pili giorni , ¢ perd dice Viol che ogn’ hor fi trinchi

che dal Tedelco rinchen vuol dir bere ) ¢ /7 sbafoff , cioe fi mangt affai , donde_:

ve un che mangia aflai : Quelte voci ba/ofia , e ba/ofione fono in ulb appref-

fo alla plebe pili bata , edi pili civili ! adoprano per (cherzo, per intendere uno

foverchiamente graflo , ¢ che mangi molre mineftre , le quali fi dicono ba/offe dal
Latino vas oft, cioé Valo pieno di mineftra .

St fala fefta di San Gimignano. San Gimignano é una grofsa Terra del Domi-
nio Fiorentino nel Vefcovado Volterrano ; ¢ la principale , ¢ pid folenne fella ,
che fi faccia in quefta Terca & di Santa Fine , la qual Santa fu di quel luogo: E
dicendofi far la feftn-di S, Gimignano’ intende fi fa fine ; € qui ale efprimeres,
che quefto Barbitre dava fine a ogni cofa , che veniva in fu la menfa .

‘ 3 TA

NZA LVIIL
Dalle freddeacqie il Mulaifanti approda Co i pefcatoré al Mula hora # accoda
A. id snilitar fra fronde,e frafche, DimeoT receon de ghioxzi,e delle lafche;
Ata nobil bardarura tina in broda Pericol pallerino ancl? ei ne mette
Divcedri edi ciriege d’ amarafche, Dugento fuoi armati di raccherte

4L mula dalle fredde acque, Pu uno che nel tempo di ftate vendeva l’acque diac-
kanes a Pare che quefto Mula fia un gran Sig. di lontani paeGi
evicino al Mar gelato , di dove approdi alla {piaggia del mare ; ma approda,ciot
s' accoffa alvreftante dell’ armata di Bertinella . Dice fra frondi , ¢ fra/che,perché
queftitali veaditori d' acque diacciate fogliono per all ornare le loro

di verzure , fiori , ¢ frafche.

8’ ACCOD A. Seguita , o vien dietro immediatamente . Quafi ad caudam ire,
Noi wliamo quefto verbo per 1e beftic da foma , che feguitando in viaggio Yuna.
I altra viene alla prima legata la feconda , alla feconda !a terza, ¢c, cons
la cavezza alla groppa dell’ antecedente , ¢ cosi chi feguita va con la tefta vici-
na alla coda di efla , ¢ quelto fi dice accodare , benitiimo ufato qui dal Poeta,
per il Mula , fendo che a i muli pil , che ad ogni altra beftia fegue quefto acco-
dare.

DOMMEO . EF’ una parola fola , ¢ dourcbbe dire Dommeone , che cosi cra_s
chiamato un venditore di pefce , ¢ falumi , il quale era amato da rutti i ghiotti
di Firenze , perche vendeva fempre il miglior pefce,, che veaiffe in mercato, ed i
giorni di geaflo haveva fempre qualche ee » © ghiortornia fingolare. £
pero lo chiama treccone , che vuol dire Rivendugliolo , ciot rivendicore di cole»
commettibili di poco prezzo [ che fi dice anche barnllo } forfe dal Latino ¢rice ,
bagattelle , cofe di poca ftima , ¢ di vil pregio ; Marziale , Sunt aping , triceque y

of

 

 
  
      
    
   

166 MALMANTIEB ¢

& fi quid vilius iis, Dice di ghioxzi, e di lafebe ( duc fpecie di ‘
per intendere , che yendefie {okamente quefti , ma per moftrare
pefce in generale. phi thad t
PERICOLO. Quefio fu un tale Alcflandso Violani detto.
nato per i] {uo gran valore nell’ abbaco, come _diremo foro.C, 4
perché egli cra anche braviffimo giuocatore di Palla a corda ,e 2
po a fitto una di quelle Rtanze dove fi giuoca a tal giuoco , if 1
armate di racchette , 0 daccheste , che {ono meftolescon Je quali fi giuo
a corda , ¢ fono compofte d! un cerchio di legno col manico , ¢d il ya
no d’ una rete fatta di grofia minugia : per /accherea intendiamo anche.
di dietro del porco , ¢ del caftrato ; Non (0 gia fe la /acchetea da ginocare,
nome da guefta , 0 quefta da quella, so ben che fi chiamano cogil’ une 5 ¢
er la fimilitudine , che é fra di loro della figura. Quefla da gi r

 

tini detta reticu/um da quella rete , della quale € compofta , come fi ¢a .
Ovidio : Reticwlogue pile leves fundantur aperto . Vedi foro C. et
at

   
  
  

viamo per mapdare a cafa le robe commeftibili, che fi comprano in
vecchio , ¢ ci fervono ancera per Quochi . Cofloro fon per lp pit
e Cantoni Suizzeri, ¢ dimorando in Firenze foglion far camerata co i
che vendono i tartufi , ¢ per quefto dice che egli conduce Norcia, ¢ la Vallatas ®
perché egli era hvomo pulitifimo,gli fa per (oprayyefta un grembiule candido »
come veramente cgil fempre portava . Z SIWOET
GIANNETT A, onde Giannettina ; {pecie d’ arme in afta, nella guerra wat
da gii alfieri, Ginera in Spago. ¢ una piccola lancia ; corfefca . jo amet
PENNACCHIO , S' intende una quantita di penne di Struzzolo; ma coftul
I havea di Cappone come trofeo di Googe . 5 BOE
Z#N4. Specie di panicre fenza manico compofto di ftrifce di Jegno gentile
eda tale Zana coftoro fon detti Zanaro/s . Di quefti tali il Poeta fa Capitano!
licche , perché in vero egli era riverito da efi , coe. quelli che nel loro. ?
} havevano veduto efercitare Cariche riguardevoli , ¢ fapevano,, che era d
reputati delja fua patria , dalla quale era in quei see «dhe
SGARVGLIA . Fu un Battilano affai celebre , ¢ fra i (uoi pari Capopolo , ©
da coflui quando in commedia ¢ ftato introdetro il Battilano }' hanno :

rola Pijlotta. ol
; STANZA LIX, STANZA L

Melicche quoco all’ ordine s° apprefia , L’ unto Sgaruglia con frittelle a tof
Per giannettina bain mano unoftidione, Alla [quadra de Qaochi-hora fogging
Ed un pafticcio per vifiera in tefta Lucha de’ Battilani affai
Con pennacchio di penne di cappone Genre che a bere ¢ peggio
Vn candido grembinl per fopravvcfta A cut battiers (diceva, )
Gii adornailc..,¢l'nno,el'altroarnione, Ch’ affeddeddieci 1a dove fi git
Vina zana é il [uo fcudo, e nel? armata Noi non habbiamoa fe i
Conduce tutta Norcia, ¢ la vallata, 4Ma-s ha a far fempre la ¥
Segue Melicche Zanaiuolo di Mercato vecchio , uno di coloro, de’ quali (¢

 

Sgaruglia . Quefti condnce la fchiera de’ Battilani , che dice famo/«, ¢! ee
wi

gf

 

do con I’ equivoco , vuol dire Attamata , da Fame, ¢ non $4 Sensis
_ =

 

 
 

BeSakeswFe gs

=a

TERZO CANTARE. 164

| PRITTELLE . Cosichiamiamo una vivanda facta di pafta quafi liquida feieta
nell’ olio da i Latini detta 4rro/aganus ; ¢ fi come effi mefcolavano con detta pa-
fia latte , ed altro, cosi noi pure vi mettiamo delle mele affettate , uva feccas,
latte , rifo , erbe , ed altro fecondo i gufti. 1 noftri contadini nel tempo, che fan-
no ¥ olio coftumano di far molre di tali frittelle , indotti a cid da havere olio ia
-abbondanza , ¢ ne danno anche a i vicini , e parenti ; fono perd foliti coloro,che
‘vanno a veder lavorare , chiedere le frictelle , ed i lavoranti con poca grazia, ¢
“meno diferezione fpruzzano Polio addoffo a quel tale dicendo : Eccoti le frittelle.
-E da quetto forfe per frirte/le intendiamo macchie , che vuol dire Ogni fegao, o
“tintura , che fia nella fuperticie d’ un corpo diverfa dal proprio colore di quel tal

corpo, come ieee > quando I’ olio cafca fopra ad un punno. Ed il Poeta dicen-
“do , che coftui molte frittelle , intende, che egli era alfai unto , come fempre
*fono i Bactilani per il continuo maneggiare olio , ¢ lane unte ,

A IOSA, In quantita grande. Diciamo nel medefimo fignifitato a cafifo,in.
chiocca,a biftia , a fufone , voce ufata da Giovanni Villani, a fimilitudine della.
Franzele 4 foifon , ciot con effutione , fenza rifparmio , 4 furore, 4 precipizio, a»

“bi > 4 Wome , ¢ fimili , Che fe bene fon modi baifi , nondimeno fono tuluolta
ufati anche fra la gente civile. E quefto a 4o/a credo fia parola corrotta , e che
doveffe dire a chiofa , che fignifica quelle cappelle , che hanno le bullette , ¢J
‘ogni piccola piaftra di piombo , di rame, o d? ottone ridotta tonda, ¢ fimil »
‘alle noftre monete , delle yuali chiofe i noftri ragazzi fi (eruono per giuocare alla
«trotrola ta vece di monete , ¢ perd chio/a s’ intende per moneta di niua valore ;
Hi Perfiani diffe:

ail © * Ma vin tafea non ho pure wia chiofa

A mantenermi , in tanto qua pars eff ?

Siche dicendofi: Delia tal mercanzia ue n’ era a Fof4, 0 a chio/a s intender y
the di quella mercanzia ve n’era cosi grande abbondanza, ¢ per quefto era a cost
vil prezzo, che fe n’ haveva fino per una chiofa, Ii Berni nel fuo Capitolo ia le-
de de’ Ghiozzi dilic - 3 i

Segue da’que/to un' altra difciplina ,
Che havend’ ingerno , ¢ del ceruello 4 iofa ,
A ' -  Bifogna che v' habbiate gran dottrina,
H Domenithi in lode della Zuppa .
an E’quincs vien , ch’ ella fi fuol gradire
Da chiha ceruello , ed intellerto a iofa, :
‘vote vhio/# per fimilitudine fignifica ancora le Crofte delle bolle, E vuol

anche dire E(pofizione, o comento, forfe dal latino greto Glofa.. Dante num,2,
Purg. C, 11, :

E ferbolo # chiofar con altro refto,
E nelInhC.2y.difle Paranno s) the tu porrai chiofarlo,
Hi Varchi nel Capitolo dell’ uova {ode dice : 3
Es io fuffi Dottor , configlieret 5
‘Che fopra quefto fi dovelfe fare
“| Leagi , e ftatuti , e pos oli chioferet .
© PEGGIO delle fpugne, Succia id vino pid che aon farébbe uaa fpugaa ; cide

deve

 

 
168 MALMANTILE®S o@

beve affaifimo, come veramente fanno i Battilani , i quali chi f
pra in quefto C, ftan, 8. ‘ subi
BATTER la Calcofa . Frafe Furbefca, che yuol dir batter Ja firad:
€ queflo parlar furbelco é praticato affai da ie forta di gente .
AFFEDDEDDIECT, Giuro proprio de’ Battilani profferit
una fola parola con due ff, ¢ quattro d . i Bargilani
¢ fono molte perfone a lavorare , hanno ogni dieci huomini un.
chiamano i] Capo dieci , che ¢ da loro ubbidito , ¢ Mimato , ¢ per
fe del Dicci, intendendo di coftui, flimano di fare yn gidramento. fo)
Credo mondimeno che dicano a fe de Dieci per non dire a fe di Dio ,
dicono per Dianora, Corpo di Dianora per Ja medefima ragione .
SCARD ASS AR /a lana, Cioe pettinare la lana con ini
no Cardi , perché hanno i denti torti , ¢ fimili a i
foglie , il fulo , ed il fiore deli erba detta cardo , del qual fiore
fi feruono per pettinare , ed unire i] peio.de i pauni, ¢ pero lo
ed ¢ i) latino car minare . Vedi {otto C. 7. flan. 37. : 3
FAR la lunediana, Apprefio a i batiilani fignifica non lavorare ;.¢ que
ché nel tempo , che I’ arte della Jana Javorava , coftoro guadagnayano. lai, ed
erano pagati dalli loro maeftri il lunedi, dove gli altri oo i
fabato , ¢ perd quefto giorno del lunedi,cfiendo per loro giorne d’.
Ja rifcoffione , era da effi Jolennizzato , ¢ non voleyano lavorare, (ma fl
fefta) a confumare in bere, ed in mangiare quel denaro , che hayevano’
e guefta loro folennita chiamavano Lunediana , cd alle volte Lunigian 15 ed
da effi tal fefta cosi offeruaza , che tra loro era la feguente cantilena,

 
  
   
 
 
    
   
 
 
        
  
 
 
   

  

  
 
 

 
   
 
 

Chi non fa la lunediana , on entngaite

E’ un gran figlio di puttana , z ae
Ed oltre a quefta ce n’ é un’ altra che dice me

UVenerd: de Beccai , 4 if

Mt fabato de gli Ebrei , r

La Domenica de’ Criftiani , TE

E il lunedt de i Battilani, ‘

Si che dicendo /unediana s' intende fella , come fi yede nel prefente
che Sgaruglia dicendo »° ba afar Sempre la Lunediana , ¢c, intende hada 7
pre fefla. Quefto nome di Lunediana refta ancor’ hogsi » ma come che i

‘orza flare alle volte Je See

Jani {ono pochi , ed i lavori meno , conuien loro per
timane intere fenza lavorare , ¢ Cosi non € mefla troppo in plo detta fo
anzi hanno di grazia , lavorare anche il lunedi ,

, ‘ Stren ZA oe

   

Conchino di Melone ecco s' affaccia , Che tutti allegris¢ rubicondi in
Che ? Offersa tenendo de gli alloré Cantando nua canzone A’
Col fineye aldo d'un buon pro vi faccta Di gran coltellise di ragliers arpa
Ha dato un frego a tutti s debitori, Si fon per amor [uo fatti foldati,
ue Conchino di Melone , il quale fi.conduce dictro una mano de* (oi dh

ae che fi fon fatti foldati per 1a cortefia , che ha fatto loro di fc
ti il debito , che havevano {eco , fu eoflui gia quoco d’ Ofteric , ¢ per eli }

 

 

ee . Se ee

ae
 

“ee Tt . 4
TERIZO CANTARE: 169

BR to gtaffo , edi flatura piccolo fu chiamato Conchino} gli venne voglia Ui diventar

_. _ macftro , onde prefe fopra di fe un’ Ofteria detta ii allori , dove fubito hebbe»

am = molti i , ma tutti a credenza , per Jo che prefto falli; e non trovando mo-

 

do di rifquotere un foldo gli venne rabbia , ed abbrucid i libri per-non haver di
pitt paflione di vedere {critti i fuoi denari , ¢ non gli potere fpendere . E
intende dicendo ¢ol fine , ¢/aldo d! un buow provi facia ha dato frego a tutti é

» S*e4 FP ACCIA, Si fa innanzi .. L’ Autore fi {erue di quefto verbo afacciarf ,
per denotare , che coftui havea Ja faccia larga ; (cherzo affai praticato con uno ,
cre habbia gran ceffo dicendolegli afacciarevi , facciami favore, facciami buon vifo,
efimili.

TAGLIERE » Intendiamo un’ arnefe da cucina,fatto di legno,tondo a foggia
di piatto per ufo d’ affettare fopra di efio carne , ¢ per triturarla con quei gran,
coltelli', e farne polpette, 0 altri batcuti . I Tedefchi ufano in molti luoghi 1 piat-
ti da tavola fatti di legno , ¢ gli chiamano T-alier con voce venuta d'Italia, come
fi pud eredere ; gid che i noftri antichi i piatcelli_, © tondini dal tagliarui fu le»
vivande , domandavano taglieri , onde il proverbio-. Due ghiorti a un ragliere,cioe
4 uno fheffa piatto’. Trovali quefta voce nella antica lingua Gallefe , o Francefca ;
¢ dicevano railfeor ; come leggefi in un' antichiffimo libro in quella lingua,dal Lat.
volgarizzato, appellato de] Conquijfo della terra Santa di Gerufalemme , i) quale fié
ritrovato effere di Guglielmo Arcive(covo di Tiro ; ¢ fi conferua nella preziofiti-
ma libreriadi Manoferitti del Serenifs. Gran Duca , appreffo alla Chiela , ¢ Col-
legiata di S, Lorenzo. 1) paffo tutto: volrato in Tofeano dice cosi ; La dentrofin

irea’) fu'trovato un vafello di pietra verde , e chiara aflai di troppo gran.
belta , fatto cosi , come un tagliere. Li Genovedi penfarono , che cid fuffe uno
fmeraldo , Percid lo prenderono a lor parte , de) guadagno della Citta per trop-
po gran fomma d'avere. Portaronncio in lor Citta , ¢ ? appefero nella Maftra
Chica , ove egli'¢ ancora. L’ huomo vi mette Ja cenere , che fi prende il primo
giorno di Quarefima., ¢ fi moftra altres! come ricchiffima cofa , Perché ¢’ dicono
veracemente, ch’egli ¢di {meraldo . Nel margine vi ¢ quefta poftilla in noftras
lingua , ido\, ¢ dove ¢’ Genavefi guadagnano el catino di fmeraldo , che ten.
gono ancor’

gio Criflo Giesi alla gran cena.
STANZA LXII.

Scarnecchia che di guerraé un ver copidioy
L! Eroe degli arcibravt , ¢ dico poco ,
A cui dourebbe dar piatto, e ftipendio
Chiungue governa in qualfivoglia loco ,
Percht quando feguiffe qualche incendio
Ei fa il rimedio per guarir dal fuoco,
Mena gente avanzata a mitre ,e gogne,
Da vender Siabeschiacchiere,e menzogne,

i nel monte di S. Giorgio , ¢ credefi , che fia i piatte, dove man-

STANZA LXIII.

Rofaccio con aleifime parole

Movendo il pie yacconta,c 4 pigione 5

Fa per quel mefe dar la cafa al Sole,

E nel zodiaco alloga lo Scorpione ;

Cusi shallando fimil ciance , ¢ fole

Si tira dictro-un nugol di perfone ,

Fa per imprefa in mezzo all internallo

Di due fue corna un giobe di crifpalio .

Seguita Searnecchia . Quefto fu un Montambanco 0 Ciarlatano , il quale ven-
deva unguento per medicare fcottacure , ¢ montava in palco {empre in abito da
Coviello col nome di Capitano Scarnecchia , . faceva una mano di braverie a

‘ fine

 
 

     
   
 
    
 
    

170 MALMANTILE( ¢

fine di ragunate il popolo , ¢ perd It Autore lo dice
de li arcibravi, B perché ¢ Ciarlatano, lo, faycapo di Monelli; © ON
alla berlina » ¢ che & buona.a vender bugie , come perlo pi fono
chi. Dice che doverebbe effer provvifionaro,; pecehe hs iene
dal fuoco le cafe, che abbrucialsero, e {(cherza, burl: Q
deva detto Scarnecchia buono.a-guarire le {costature in godin
dolo buono a rimediare a gl’ incendj.

MIT RA, 0 Mitera », Diciamo)quel:foglio , chera foggiadi coront fi
capo a coloro , che per delitti fon Tota © mandati in ute ‘afing
C. 6, ftan, soeC, 12. flan. 19

GOGNA., E' lo fteflo che Berlina detto fopra C, 2. ftan. ay. I aaa
no Wumelle, {e ben-quefla era pili toNo una {pecie di ceppi da {errare ip
de forfe meglio con Piauto,-e.con Lucilio Ja chiameremo colfare. 9
FLABE , ¢ menzogne* Sinonimi , che fignificano Bugie . Fiaba Pe fab
menorna dal verbo mention , 1 ate
Dopo li faddetti vien Rofaecio y il quale conduce feco una. gran mano. i
ne tirate dalle fue chiacchiere . Coftui fa ung de i pid fuperbi ciarloni
mai ftato nella Ciariataneria , e fpacciavafi per Attrologo. Noma
banco , ma ftava a cavallo allato,a una tavola elevata, fopr’ alla qual
una faragine di cartapecore di privilegi havati { diceva egl ) pee il
da i maggiori Potentati della [eon » qualche {eheretro di gatro;0
sfera d’ ottone , tre corni neri lunghi , ail uno de’ quali era appefe unip
calamita , all’ altro una palla di lumpidifimo Criftailo dil Monte, ed
corno , che cgli diceva eliere d'Vnicorno.. Vendeva una fuacmeftura:dailut chia~
mata con vocabolo Greco Wepensbes , che diceva-efler buona a rutte Pi
conforme al medicamento d’ Elena chiamato con quefte medefimo
penthes ( cio’ di contrartoal dolore ) da\ Pocta nel4.dell’ Viiflea, ed-a chi lacom-
prava donava un’ anelletto d’ oflo , che (pacciava’ per ottimo aldol fla;
per effer fatto di dente di Cavallo marino, Diceva havere, impa ;
gia da un gran Mattematico, ed Alirolego fuo Zio nominate Gio
cio., che predifig { vantava egli }.la rovina della palla della Cupol
di Firenze molto tempo avancd, che cella feguifie. In fomma:con le
fandonie ragunava fempre , che 1 montava a cavallosinfinite perfone;@
buone fomme di danari; 11 Poeta lo fa condotticre di quefta ge
le chiacchiere , ¢ gli fa fare per imprefa quei wre {uoi corni persiaier
di criftallo . pocikrutagt

eALT/SSIME parole... Chiama parole altifoime quelled Rofaccio; pete
fempre dilcorseva di plancti, di ftelle ,  d? alevevcofe:celefti.comeme
tore con dire , che egit ba affttaralacafa al Solose meffole Se
Senza ironia Dante nf..4, chiamo Virgilio ; A aleiffiens Poera -By
Cost vidi adunar la bela fcola Di-guet Signor det atei/jima canto, O
rifime canto chiamala on gale In Otbimo;e ornarifime
cia turte le dottring,e maffimela Teologia,imperochéi primi P
- SBALLARE, Vuol Propriamente dire cava
efprimere uno che racconti moire 5 ¢ molte cofecpiut

 
   
 
  
  
    
    
  
   

 
     
  
   
 

¢} Domo

  
 

     
 

 

oe ee
 

AS

SAN

ut

SERRE

Bak

TER ZO'CANTARE: i 71
verita,ed dil medefimo, che//chianrare, che vedremo fotto C. 10. ‘ftan. 66. Quefta
voce sballare in al ficato vedremo forto'C.'11. ftan. 4.

~ CLANCE ye fole, nimi}; eP ultimo @ Sincope di favole ; ed intendiamo-

chiacehiere lontane dal vero. Petrarca Sogni d'infermi , ¢ fole di Romanzs, li
Mauro jin biafimo'dell' Onore diffe: a

DS Har-abdieh® ia y che le fon butte-fole ,

SS uttiargumenti da ingannar gli feiocchi

BO OL 9 a Le cafe che confiftone in parole ,
Ti Perfiani’in'una fua canzone dice oon ,

> i Se con ragliare o fole
ity stew ‘Ve pagar di-bravara

“Ottavio Pertari nelle fue Origini-dedudele parole Ciance , © Cianciare da Can-
tiones j Cantionate’s It Boce. Now: 61. quando diffe tla landa di donria Dfatelda’, e
corali altri ciancioni volle dire fenza dubbio canzoni , le quali ( perehé erano molto
in pregio le Provenziali , o:le fatte fa VY ariedi'Provenza , come fi vede da alcu-
neinttolazioni'di Lande antiche*) chiama come’ per iftrazio , € contraffacendo
in quefto , ficome in molti altri luoghi,la pronunaia delle lingue Mraniere ; cian-
ciont'; Scherzando anche nel medefimo tempo {ull' altro fignificato, cioé di ciancta,

VN nugolo di perfone . Quetta voce nugolo per Quantita grande ,é afiai ufatas
dainoi y el*usdiil noftro Poeta fopra-C, 1. fan. 50. Cosi Giuvenale Sat. 13. imi-
tando inci Omero'; chiamd la molcitudine delle combattenti griy , nubem fo.

 

 

nora ON1 {2
: an iz §$ TAN ZA LXIV. :
Sopr' un lettoriechiffime fiorito E pur, vin arme ei non fu gran perito,
“Rartar: Pippa fifa del Caffigtione , Guerrier comodo almen nel padiglione ,
Ove coperte fea tutto vefito, - Queffo impera dal morbido piumaccio
| Ch'in tal mado (0 foalda al fuo padrone; et quelli del meftier di Michelaccio ,

Seguita Pippo det Ca/tiglioni portato in un ricco letto , di dove comanda a i fol.
dati 5 the on ata get ees vo di lavorare . Coftui era il pit graziofo ,'e
faceto umore , che'fia mai ftato in Firenze , ¢ i chiamd Pippo del Caftiglioni, per-
ché ferui lungo tempo a i SS. di Cafa Caftiglioni con fedelta indicibile, ¢ pero da’
medefimi $$, aniato a/fegno’, che non oftante le burle , che in diverfi tempi , ed
occafioni: faceva.a efi SS, non potettero mai mandarlo via, perch, {¢ lo licenzia-
vano’, egli trovava fempre vaghe inuenzioni per non fen’ andare ,' come fra Je»
molte fu quefta..: Il Sig. Cavalier Vieri da Caftiglione, al quale per ordinario fer-

-viva,lo-licenzid con quefte parole: Sgombrami di Cafa. Pippo andato in Piaz-

za chiamd prot Carrettai , e condottigli con le loro carrette d’ avanti alla,
porta delltabicazione di effi SS. in {y Y ora , che i) Sig. Cavalier Vieri foleva tor-
nare a definare , ordino loro, che, fe il medefimo Sig. Cavaliere gli domandaffe

quello, che facevano quivisgli rifpondeflero,che ve gli haveva mandati Pippo ; fi
come

fegui ed il Sig, Cay. diffe: che hada far'Pippo delle carrette? Ed-egli a
quefte parole {cappato di dietro a una di effe carrette’, rifpofe : Sgombrare , co-
me VS. Llluftrils. m’ ha'comandato ; Onde il Sig, Cav. ridendo della faceta in-

a amen => del- {ao comandamento Jo richiamo in cala , ¢ pagati i carrettai gli

‘ = Ys IN

 
. be flato affai di notte . Pippo fi fcordo di mettere il caldanetto nel let

ip MALMANTILE

LN un letto riechiffimo fiorite, HW medefimo Sig, Cay, una fera con
che facefle , che il letto fuffe caldo , quando egii tornava a dormire 5

 
 
   
 
  
   
    
     
   
  
   

tornato il Padrone , ¢ volendo andare a dormire , Pippo fi trovo:ii
perché ftante |’ ora tardifima non vi era modo di trovar fuoco ; ricorle:
folite afluzie , ¢ quefta fu , che egli per Ja parte di dietro del letto v’ ent
tro cosi veftito com’ egli era , ed il padrone , credendo che sli andafle mo
lo {caldaletto , fi poglid da per fe per non lo {cioperare’, e {pogliato a
volta del letto , ¢ difie: Cava il fuoco , ed alzata-Ja cortina y
vedde Pippo , che follevata alquanto la tefta difle; Signore il letto non
caido a baftanza . Il Sig. Cavaliere vedutolo cosi , ¢ conofcendo I’ umore:
beftia fenz’ alterarfi Jo fece ufcire , ¢ toltafela in pace entrd nel letto cost com
era. E per alludere a quefta, facezia il Poeta fa venir Pippo portato in un
chiifimo letto. . ors
PiVatACCIO., Guanciale lingo quanto la larghezza del letto; della grok
za d’ un facco ordinario da grano , ed ¢ ripieno di piumeye perd + Pinns
cio. Qui per piumaccio intende tutto il letto . 1 ah
QELLI del mepiero di Michelaccio, Gente , che non ha voglia di
che il meftiero di Michelaccio dicono , che era mangiare, bere,e ,
Qui pure bifogna , che il Lettore fi contenti ch’ io faccia un poco di.
ne per narrare alcune delle facezie del detto Pippo , meritando: la
cita di quefto huomo , che fi fpenda qualche di tempo in fentire:
guzie , il quale é viffuto fino:a pochivmefi addietro d’ cta di 8s. anni fe
Ja medefima bizzarria , fauo che, dove prima frequentava molt
trovar le conuerfazioni , che gli pagavano Jo {cotto, ( perché
quattrino,dando egli tutto quello » che guadagnava alli {uoi vecchi
dre , alli quali continovo d’ ubbidire come ua fanciullo fino al” eta fua di fopras
75. anui , che effi paflando cento anni , morirono ) dopo Ja morte del Pad
quento pil le Chiefe pregando S. D, M. per la falute del. Serenifs.: G. ‘Daca » dal
quale gode fino , che wile, onorata proyifione per il buon feruizio
niffima Cala. hbase
Effendo una volta il medefimo Sig. Cav. Vieri al Poggio a Caianot
‘Serenifs. G. Duca }.a fcruire il Serenifs, Sig, Principe Card, Gioy a
Pippo a Firenze la vigilia del Santifs, Natale ordinandogli, che fi facefle-dare dil
farto un fuo veftito nuovo.,'¢ lo portaffe al Poggio., <T ordine , cheigli diedef
con.quelte parole: Va a Firenze , ¢ fasti dare dal farto il anio veftito ye portale: We.
bidi Pippo > ¢ la fera-medefima tornd col -detto veftiro del padroneun f
entrato in Chicfa do ve era tutta la Corte per-udir Ja Mefla-(mancandovi
Sig. Cav. Vieri , che fe ne flava in camera afpettandoiil veftitu per
veduto da wttii Cort igiani , ¢ da-tuui li SereniGs. Principi che quivi .
il Sig. Principe Card . Gio, Carlo gli diffe: Sig. Filippo che:colaé quefta? Val
fiate molto nobile ? Ed egli tifpole: Screnil..quefte fon graziesche mi
Padrone. & S. A. Rev. immaginandofi di come ftava:il fatto fi,
Pippo sil quale fatte, pity, (paiicggiate per la Chiefa. fen.andd alle ftami
Padrone;, che vedutolo con quell’ abito in doffo lo {gridd dicendo.; Briccone!

 
 
 
 
 

 

See Se ——— ee

 

 

eR gS ce a
Se Eatetetarkies

ae.

SShERs

54
ee

RERLRASEE | Et

&
=

SERA!

TERZO CANTARE: 473

Siam fratelli? Rifpole Pippo: Perche Sig. ? Replicd il Sig. Cav. Che furfanteria.,
¢ la tua metterfi il mio veftito? Mi maraviglio di V. S, Lluftrils. ( foggiunfe Pippo)
non me I’ha ella donato ? Come donato ! (diffe il Sig.Cav. ) Ti par’ egli abito da

co eae ,¢ mi fla benifiimo ; E V.S. Iiluftri, medefima a
ha detto., che io me lo

cia dare dal {arto , ¢ Jo porti , ed ecco ch’ io 1" ubbidi-
{co , gia tutta la Corte ha faputo quefta generofita di V. S, llluftrifs. , ¢ fi fono
_rallegrati meco del:regalo , che V. S, Uluftrifs. mi ha fatco in quefta folennita .
Il Sig, Cay. conofeendo , che non era {vo decoro il metterfi quel veftito , che era
flato yeduto in doflo al {uo feruitore , ftimd bene il quietarfi , ¢ fargliene un re-
galo, per-non — far’ altro ; Ecosi Pippo fi godé quell’ abito , che per la fua
ricchezzacra ite a. un Principe .
Era grande amico di Pippo il Poon Fantacci oggi vivente Rettore delia Chiela
di Varlungo fuori di Firenze circa un miglio ,il qual Prete é flato fempre huomo
aflai faceto , ¢ piacevole ; ¢ fra eflo , ¢ Pippo fon feguite diverfe graziofe buries
¢ fra I altre il Fantacci difegno una volta di fare ftar Pippo fenza.cena , ¢ necel-
fitarlo a.dormire all’ aria; € per quefto l’ inuito ad andare alla fua Chiefa a Cena
ella fera appunto, che il Prete havea fermato d’eficre acena nella Villa de’ SS,
Pont quivi vicina ; ¢ ad effetto , che gli riu(cifle il difegnoshaveva ordinato alla.
ferua che andafle a dormire a cafa una {ua parente, ¢ detto al Contadino , che

on alla Chiefa, che, fe.fufle accaduta cofa alcuna attenente alla curayman-
7 ‘a oe

Prete di Rovezzano,Chiefa viciniffima a quella di Varlungo. Pippo chie-
fla,ed ottenuta licenza dal {uo padrone,la (era al ferrare delle porte della Citta,fe
anvandd.a Varlungo , ¢ trovata ferrata la porta della Cafa del Prete ,.dopo haver
molto picchiato,conofciuto » che:non era veruno in cafa , difperato s’accofd alla
cafa'diquel Contadino., che haveva l' erdine di mandare la gente a Rovezzano.,
eda eflo intefe, che i] Prete.era andato.a-cena fuor di cura , ¢ gli ordini che ha-
ea lafciato.s, Pippo accortofi molto bene ,che il Prete ’ haveva burlato , volles
renderglilapariglia , « percid fare trovata una {cala a pivoli, con effa-montd {a-
pra il tetto della chiefa ye.quivi portata buona quantita dijpaglia , ed altro ciar-
= combultibile,¢ raro., gli dette fueco , ed andato_alle funi delle campane

meffe a fuonare a rintacchi . 11 Prete Fantacci., che era.poco lontano fentendo
fuonare a martello , st affaccid\a una fineltra -per fentire, che cofa fufle quella. ,
evedutoil fudco fopr’ alla fua Chiefa , tutto {paventato la{cio la cena, ¢ I’ alle-
gria, e-corfealla volta della fua cafas ncljla quale (ubito.entrd per-vedere doves
era il fuoco., ¢ rimediarui-can.}’aiuto d’ una parte de’ SS. Commenfali , ¢ con,
uina-quantita di contadini., che.gia erano quivi.concorfi con zappe , ¢ pali per ro-
winare),¢ tagliare dove bifognafle. Pippo intanto {cefo.dal tetto fe.’ andd
-ad.arno ,¢-fi fermo a:cena da.un tal Boni mugnaio fuo.grande amico,, baltan-
doglid-havere furbata I allegria,nella.quale era.il-Prete , il quale girato ¢ oy
to.,\¢fopra..per tutta la cafa., .c non-havendo trovato ne meno-fegno di fuaco.,
fece viltase il tetto della\Chiefa ,:¢ trovo:la.paglia , che era.finita d’ardere , es
vitta Ja-feala appoggiata alla .muraglia., s’.accorfe che era Mata una contraburla.,
di-Pippo , tanto ne sche silcontadino, detto di fopia-difle haverlo.yeduto poco
prima , ¢percid (opportandofela in pazzienza ,tornda ceaare , dove non man-
carono Je minchionature 5x¢ barzellette y che fucono da quei SS..della conuee(a-
zione dette.al Prete. Cow-
174 MALM ANTIDE 7

Commeffe una volta Pippo non fo che mancamento, per’
volle mortificarlo col mandarlo in carcere , onde gli fece dare
un biglietto , accid lo portafse al Segretario del Magiftrato
glietto diceva , che fulse ritenuro il Latore in fegrete fino a nuovo
prefe il viglietto , e indovinatofi del contenuto, ¢ parendogli duro
in prigione in tempo di Carnevale , ¢ fapendo , che il’non portare il ¥i
delitto da galera , andava mulinando come potefse faluare la’capra jt
quando la fortuna,nell’ andar’ egli come Ja ferpe all" incantosgli fece
nanzai un Tedefco giovanetto (eruitore di liurea del medefimo Sig. C
Padrone , alla volta del qual Tedefeo andato Pippo » quali brava:
Padrone ¢ in collera , che tu fei flato tanto a venire’, perché vole
taffi quefta lettera al Sig: Segretario de gli Omo, ¢ perché én
mandava me; febene , ho da fare afsai fu in Palazzo; pigliala ye
do. Il buon Tedefco non penfando alla malizia porto Ja Jettera\, in
degli ordini- della quale i} Tedelco latore fu ritenuto in carcere’, ¢
che S. A,S. era reftata ubbidica .. Pippoil dopo definare'del medefi
vefti da donna , ¢ fenza mafchera con Je fue propric bafette ye barba fe
feggiava il corfo delle mafchere,havendo d’ attorno un popolo infinito *
téa vedere quefto tumulto i} Sereni(s. G, Duca, che pa(sava:in carroza
Ja firada , onde (pedi uno flafiere per intendere che cofafulse. Lo
no , dicendo che cra Pippo del Cattiglioni in‘matchera da: donna 5
che gia fapeva del viglictto,replico: non pud efsere, ondesil Caporal
fieri andd da per (¢ , etornd replicando efser veramente Pippo nel
veva detto lo flaffiere’; in tanto S, A. S.s' accofto,¢ Pippo che gli
tro,ed hayeva ofseruato,che S.A.S,hayeva mandato due volte a veder chieglies
fattole una grandiffima riverenza dilse : Seréni/s, io fon io,io fon'io,percht
wm’ ha fatto il feruizio di portar la lettera lui; Finalmente conofco bora
chi fifa ben volere po [perar fempre quefti, e maggiori feruizz) .
rife dell’ afluzia , ¢ ordind che fufse {carcerato 11 Tedefco.
ig. Cav. Bernardo fratello del Sig. Cav. Vieri haveva prefa I
efta dama volendo elser feruita da Pippo per bracciere ,

uomo ¢’ eta , ¢ veltiva'di nero , ¢ non con Ja liurea’,come gli altri’!
quella Cafa , prego i] {uo Sig. Conforte , che lo chiedelse al; ‘atello , perché fer-
vifsea lei, 11 Sig. Cavaliere Vieri gli compiacque , fe bene cop poco 3
perché era avvezzo a anes fuori di quelle i bizzarrie lo fe: rar
e con meno gufto di Pippo,che non avvezzo a feruir dame gli a
verfi ad aveensate in fua vecchiaia 'e mal! volenticri lalchea il fuo padroney it
diferecezza del quale non fperava trovare in chi che fia; onde prego la 4
che'lo yolefse lafciare al feruizio , che era folito ; ma la’Signoranom
_mutatfi di propofito ; per lo che Pippo fi_gettd alle inuenzioni il
con riputazione, ¢ con operare, che Ja Signora Jo licénzialse,
mettelse mancamento , Chiamd dunque a {fe alouni ragazziy ¢ dil
alcuni pochi foldi, impofe loro , che quando lo vedevanovcon)!
dafsero tutti a gridare Pippo , Pippo , Ecco Pippo ,'¢ glitacelsero
I ragazzi invitati al loro giuoco , ¢ che haurebbono-daco quaicola’

   
 
   
   
  
   
    
     
  
 
      
   

  

  

   
    
  
   
      
 
   

  

  
  
 

—_. —we oe Row eee

 

 

2e 2 ow an =o SS fF ea eB H.-S ese oe

x
 

 

5

TER:Z:O0 CANITARE.
nit reloceafione di far quel chiafso , appena Jo veddero ulcir di cafaydanto il braccio
alla.Padrona,s che cominciarono a ftrepitare , ¢ tagunarono quivi quanta gente |
i ¢ra in quei contoral’.¢ Pippo favio , feaza mutarfi in facia feguitaya a dare i!
sai) bracciovalia Signora,,.1a quale vergognandofi , che il fuo feruicore fae lo scicr
volun 20'del Popdlo., che egit fulse trattaco come un pubblico buifone ys) aftretto, di

giugnere in Chicla , pen(indo,, che quivi almeno dovelse fermarfi il baccano 5),
niayfe.celsd il pelacajion fini il inicia perché quei ragazai ftandoli cuttiat.,
» non geidavano per rifpetto della Chiela , ma erano cagione y che wiro i
pepo guardalse verfo quella parte ; per lo che Ja Signora per liberarfi ordind a»
ippo , che andafse a cafa ,¢ mandaGe un’ altro feruitore , ¢ tornata poi 4 ca/a
Je parue mill'anni render Pippo a chi glicl’ havea cunceduco ; E cosi egti ricorna
al primo ferwizio, ficuro , che alla Signora non farebbe mai pit venuta yogiia ai ,

eum 208 (ernireda lui. yeaah
jade, Havewadl Sig. Cav. Vieri una bellacagna da Fermo , la quale diede in cura.a
f dicendogli : Tien conto di quelta cagna 5 ed avvert a non Ja fimasrire y

om perché fe la finarrifei non ti afpeteare altra licenza. Prefe Pippo la cura della ca-
ga $24» €col trattarla bene avvezed a fare mille giuochi , ¢ fe la refe cost alic~

| 2onata', che era impoffibile , che egli la {marrife. Avvenne , che Pippo fu in-
vitat® a una fella’, che & dovea fare in un iuogo poco lontano da Firenze , dove
era per tratcenceli almeno tre giorni , onde chicle al padrone ticengia per a quel

  
 
 

 
 

te

   

oe tenipo';-ma non ltottenne , Pippo fenza moftcar di cid difguftoy la mattina avan-
| tivalla-wigilia di dewa fefta:comparue in-cala (enza la cagna, ed il Sig. Cav. do-
J mandi dov’ ell’era.. Pippo dilse quafi piangendo :. Sig. io non1o.s03, quando io
i fubvicino a: cafe mix ierfera ella comincid a fuggice, ¢ per hholto., che 40 le core

mf relsi diettd chiamandola, non fa poftibile farla cornare y ne arrivarla .. Replicd il
| Sig. Cavaliere; Tw fai i parti ; pero va a fare i faci moi, ¢ non haver’ ardire di
od mettere il piede ineala poftea (enza la cagna. Pippo fingendo un dirottiffimo
4 pianto fen’ ulci di cafa , ¢ ando alla fefta , alla quale era ftato inuitato , e pafsati
®  alcuni giorni in grandidima allegria fe ne torad a Firenze 4 e andato fuori della
porta alla Croce da uno Ortolano fuo amico , al quale haveva lafciata la cagna,
; fe la prele , ¢ I" infango tutta , ¢ le infanguino I’ ugaasaccid parelge (pedata , ¢ Ie~
i gatala con una corda: lx condu/se al padrone , il quale veduto Pippo con la cagna
giidibe: Dovel hai trovata? In Cafentino,Liuttrits: Sig, , ¢ nomci voleva altri
| cheme:per trovare il-luogo dov’ ell’ era fitta . I Sig. Gav. credette quanto dife
Pippo, il quale con tale inuenzione gode la foddisfazione, che bramava, E tan.
torbatti ptrun faggio delle facezic di Pippo , il di cni.intero nome , © cognome

oe
SPAN ZA LXV. STANZA LXVI.
Cenro fuggerti egli ha della fuselaffe

inch egtine Pigmei difterti., e brurti
Fanti che nacquer nelle magne baffe,
Mita fe ben fon piccini , vi fon tutti,
Mangian /pindci ,arrufian le mataffe.,
Ea ba pit viz2j egnunydi fei. dargusti,
Cofa é quota che va per il fue dritto,

Che non é in corpo ftorto animo drittas

tf girea Bariffone adele rocca
Gran gigante da Cigoli di quelli,
‘Che vanno a corre i ceci con la brocca
| Ebattoncon le perriche i baccelli:
Ler fue bellexze amore hasipreincocca
Per ferir Dame i dardijed ¢ guadrelli,
. Fa il Cavaliere nelle cavalcare ,
2 va [pel furiero alle nerbare .

 

|
|

 
 
   
    
   
 

196 MALMANTELBhAT

Segue Batiftone Nano con una gran quantita di compag al
bene fon cosi piccoli, fon tutti viziofifimi , ¢ non Segoe 3
ché in un corpo mal fatto, di rado fi trova anima ben outa

BATISTONE , Quefto fu un Nano levato da guardare le pecore  ¢
a fervire il Sereniffimo Principe Mattias di Tofcana , dove in! 6
in ful pofto di bello ; ¢ facendo lo fpafimato di tutte le Dame, ©
ce: Per fue bellexze Amore ha fempre in cocca Per ferir Damei
arrivO a fegno quefta fua inclinazione aile dame, che per porere liberas
ticare con efle , fi contentd che i] fuo Serenifimo Padrone lo facefle
come fegui , ma perd in burla , ¢ ftette nelle mani di Maeftro Agnolo”
' Caftratore circa un mefe , {empre credendo d' effere flato caftrato: EB
H egli, non oftante che fufle di ftatura piccolidima impard afai bene a
| € maneggiare ogni cavallo aggiuftacamente , fupplendo con la mano’

che gli mancavano le gambe , era (olito ancor egli andare nelle ca
i Cavalieri , ¢ pero dice : Fa i/ Cavatiero nelle cavaicate. Ma percht quel
( di Caramogi ¢ aflai fottopofta alle mazzate del padrone , ed egli ne hai
fua parte , perd il Poeta dice ; Va fpeffo Furiero alle mazzate . Quefto Ni
! la morte del Sereni(simo Principe Mattias {ervi al Serenifimo Gran Duca if
lita pure di Nano ,ma efercitava anche la cucina fegreta diS. A, S., nel
ftiero s’ era fatto peritifiimo , per lo che oltre alla buona provvifione¢
bufcava gran mance ; ma la Fortuna |’ abbandond ia ful buono,
fi egli innamorato d’ una bellifima giovane {ua pari disnatali 5 Ja

glic , ed in pochi giorni mori . Lo chiama Gigame da Cigoli Ȣ
quelli che colgono i ceci con fa brocca , come fi fa de i fichi , ¢ che base:
(a pertica , come fi fa delle noci , non _potendo arrivargli altrimenti.
Gigante da Cigoli,in una collinetta vicina a $.Minjato al Tedefco,fi
Je donnicciuole, una Iperbolica cantilena antica , la quale dice, ) joe

Ed! una punta d’ ago rogebiols

Ne facea pugnale , ¢ {pada , itis it
E di quello che gli avanzava de
Ne faceva uno {puntoncin ,

ae

 
  
  
 

 

   
   
    
 
      

 
 

 
  

E is quefta ilena con altre iperboli retrograde fimili ae
re la picciolezza di quefto Gigante da Cigoli ; e di qui ¢ in ufo comune il dires
Gigante da Cigoli a un Nano , che i Latini differo Pumitio , ¢ noi dici ne
Ledina , fimilitudine tratta dal giuoco della dama ; Sericciole da un’
coliflimo di quefto nome , Pimmeo dalla voce Greca Pygmaios 5 che’ t
dell’ altezza d’un pene « 1Greci dicevano Manus , Pujilius quantus Molo y ed ae
tre volte gutta ; ed un Pedante lo chiamo Titsviditinm Scarabei umbra. a
Strada nelle fue Prolufioni , parlando d’ un Nano dice: Fangino be seem
capite fe torum tegit , Ed altrove , pure nello fteflo propofito dice; iis #
cin 5 Somninm hominis , falsllum anima. xa

BROCC.A, Voce , che viene dal Greco Brochos fecondo il Monofino, ¢ &
condo altri dal Greco Prochoos ; il che ¢ pid verifimile , eflendo quetto valo d
acqua , ¢ quello vafo da vino ; ¢ vnol dire un vafo di terra per ufo di |
acqua , € pero detto Aydria , ¢ noi lo chiamiamo brocea ;, Chiamali broc

 

 

   

a <n
 

AA

SERL

We

zt

SR

Le ‘ om |

q

TERZO CANTARE,; 177
ancora uno flrumento fatto di canna rifeffa in pit parti; fe quali allargate ,¢ ria-
‘teflute con falci , formano comé una piramide'a rovefcio , ¢ di tale ftrumento
fermato in cima a una pertica , ci ferviamo per corre i fichi,quando non fi potlo-
no arrivar con le mani ; € di quefta brocca dice nel prefente Iuogo

~ FVRIERO , Si dice colui , che va innanzi a preparare gli alloggi nel viaggia-
re che fa un’ Efercito , o altra gente in buon numero. Lat, metacor mwanfionum ,
Tn Latino barbaro dicefi fodrarins da fodrum voce che vien dal Germanico , la.
in buon Latino fi direbbe'alinentum , pabulum’, annona ; Onde Foraggio , e
Foraggiare’, Provifione di guerra,e provvedere P efercito . Tuto cid fi offervd dal
Ferrari nelle Origini alle voci Foraggio , ¢ Foriere , Ma erra quando piglia Frie.
re dello /pedale , che fi trova in Gio: Villani lib. 8. c. 95. per accorciato da Foric-
re, fia Provifor befpirij poiché quivi , fi come appreffo al Bocc. Nov, 92. fi-
i. Srate dal Pranzefe pees cone fi domandano anche oggi i Cavalieri di

alta. Qui fi ferve della voce Furiero per intender fur:a che fuona quantita ,
come dicemmo fopra in quefto Cant. ftan. 50. ¢ vuol intendere, che queito Nano
{peflo toccava qualche furia , cioe quantita di nerbate. Vedi forto C, 9. fan. 49.

PIMMEL » Beano popoli nani , che habitavano nell’ ultime parti dell’ Indic ,
i quali. cre{cevano fino all’ altezza al pid d’ un braccio , ¢ le loro mogli di cinque
anni partorivano , ed otto erano vecchie. Di eu fa menzione Plinio lib, 4.
cap. 11. ove dice i barbari chiamarli Cathizi , ¢ lib. 7. cap. 2. Cofforo per effcr
cost piccoli erano infeftati , € rapiti dalle Gru, onde per difenderfi andayano
armati di'frecce ; ¢ cavalcando fopra alle capre in granditlime {chicre ,a guattare
iloro nidi, ¢ romper loro ! uova . Di quefti parla Giuvenale fat. 13. dicendo .

Aad fubicas Thracum volucres , nubemque fonoram
Pygmans parnis currit bellator in armis,

Mox impar hofti raprnfque per aera curuis
Vugiibusa fava fertur grue: Si videas hoc
Gentibus in noftris , rifu quatiere , fed illic,
Quamquam eadem alfidue /pettemur , pralia ridet
Nemo , xbirora cobors pede non eff altior uno

NELLE magne baffe . Intende che fono di flacura baffa , fe ben par che dicas
fieno nati nella bafla Alemagna . Lat. Germania inferior ,

SE bene e’ fon picein? vi fon tutti, Benché piccoli hanno malizia quanto un,
grande. Tydeus corpore , animo vero Hercules ; da Omero, il quale defcrive Tideo
il padre'di Diomede piccolo si di ftatura , ma gagliardo .

MAKGVT TE, Che Nano fuffe coftui , ¢ quanto fagace , ¢ feellerato , vedilo
nel Pulci nel {uo Poema intitolato il Morgante ? Quefto nome di Azargusre forfe
fu finto’ dal Puici a fimilitudine di 4¢ergite , Perfonaggio famofo per Ja fua fccm-
piataggine , il quale fa il faggetto d* un intero Poema burlefco di Omero ; e cid
pore avere imparato il Puici da} fuo dowo amico meffer Agnolo da Montepul-
ciano. ©
NON é in corpo forts anima dritta, Non & in corpo mal fatto , animo ben,
compolto , giufto , ¢ che tirial buono ; che tanto fignifica Ja voce dritto in que-
fto luogo . Sidice anche: Vn fegnato da Dio , non 4 mai buono: (alludendo per
avventura a Caino, Gen, c. 4. verf. 15. : quali che quel tale fia in un certo mo-

do

 

 

 
 

178 MALMANTILE ~ %

do contrafflegnato , affine , che ognuno , che Jo vede fi guardi ) qu:
praticata comunemente , ¢ fi vede da i {eguenti verfi maccheronici

Nulla fides gabbis ,& parum credite ruse, s

Si guercius bonus eft , inter miracula feribe , i

Vn’ altro Poeta in quefto propofito difle : Chiude un’ anima bigia
Che huomo bigio intendiamo huomo cattivo , di poca cofcienza , ¢
gione, Marziale . Crine ruber , niger ore , brevis pede , inmine lafus Rt
préftas Zoile , fi bonus es. Quel Terfite , che quanto fconcio di vilo, ¢ fc
to nel corpo , altrettanto era bructo nell’ animo , ¢di coftumi
fopportabili ; vien defcritto da Omero al 2. dell’ Iliade ( fecondo la tra
Pictro la Badefla Meilinele , ftampata in Padova I’ anno 1564.)

Lufco a’ un’ occhio , ¢ a’ um pic oppo, e firetto

Wegli omeri , che gobbi ba sfin' al colle;

Aguxro il capo, e'l capel cre/po ye raroy

Sucido ye ner , lentiginofo , € marcio,

ZA LXVII.

   

Piena di fudiciume ,¢ di frambelli eUacfiro de? Bianti, e de’ Monell,
Gran gente mena qua Palamidone , E vefte la corazza da baftone y
Chril giorno vanne a Carpi,ed a Borfelli, Perch egli quant’ egnialtra (uo alte
E la notte al Bargel porta il Lancione , E’ tutto il di figura di riliewe,

   

Palamidone conduce feco una quantita di birboni , ftracciati , ¢
cra Jui. Quefto fu un guidone mezzo matto , ma tutto trifto , ed al
gno birbone , il quale faceva fervizio a’ carcerati,¢ perché continovamente br
tolava , dicendo di pazze (cioccheric , haveva fempre dietto una gran quantita di
ragazzi che lo facevano ftizzire. La notte per guadagnar qualcofa portaya'
tro al Capitano , o Caporale de’Birri un’ arme in afta folita portarfi —
glia del bargello, quando la notte va facendo la guardia , la quale arme noi
detta /ancione , Ma che egli rubafle non poffo crederlo, perche afflolutamente nom
havea tanto giudizio , ¢ ftimo che il Pocta dica quefto nel prefente luogo , ¢ a
trove per defcriverlo per uno di quei furfaati, de' quali fi pud credere ogai ribal-
deria. Palamidone ¢ accrefcitivo di Pa/amudes, Eroe noto nella guerra Hy
fecondo la pronunzia Greca pil moderna dicefi Palamide,e non Palamedes onde
é fatto il (oprannome di Palamidove ; che fignifica un lungo ¢ fortile , come wie
palo, una perfona grande di ftacura . =o

eANDARE 4 Carpi, ed a Borfelli , Carpi ¢ un Principato in Italia notitfimo j ¢
Borfeili é un luogo ful Fiorentino , ¢ {cherzando con quefti due nomi Carpi itt
tendiamo carpire , cio¢ rubare., ed a 4or/edé , ciot alle borfe per rubare. Ati
ftofane Poeta Greco nella Commedia inticolata i Cavalieri , citato dal
nel Flos Ztalica lingwe , ( ove egli tocca la maniera di parlare Fiorentina; eng

ebbe per San Giovanni , ufata anche dal noftro Poeta ; ) dice cosi: manus in Attt-
lis hades « Vuol dire: fempre chiede , ed ¢ apparecchiato a pigliare ; {cherzando fal
nome di certi pork chiamati 4ro/i , per  allufGione che ha quefta voce alla pa
rola atein che fignifica chiedere _ ate

PORT ARE il Lancione al Bargello. Quelto meftiero (olito farli da birro novi2idr
lo faceva alle yolte Palamidone., comes’ é deo. 5 he ee

: ; BiANTI

   
 

 

      
   
  
 
 
 
  

 

a

reo.
RARER LARP LEE

weERAS

RNLSESE

mA

Be

 

TERZO CANTARE: 179

BIANTT, Si trova una {pecie di Bricconi ,e Vagabondi che v anno bufcando
danari con inuenzioni , come fi vede da un libretto intitolato Sferza de’ Bianti, ec,
E fi dicono anche Monelli ; (¢ ben veramente per monelli intendiamo quei pove-
ri, che fi ftroppiati , malati , impiagati , o morti dal freddo per muover
Ie perfone a far loro elemofine , donde rab po far it monello quel ragazzo,
che havendo toccate leggiermente delle dal Maeftro , o da altri , metre as
fogquadro il vicinato con le ftrida per moftrare d’ effere ftato dalle bufle ftrop-

iato., ed in vero non ha mal nefiuno , che fi dice anche far marina: vedi (opra
. 1. flan, 37. alla voce fofiano , ¢ forto C. 4. flan. 8. Di quefti intende il Perfia-
ni nei feguenti verfi .
Signor non fo fe voi fapere il bando

Di chinder tutti dentro a! Mendicanti

Mafcalzon, vagabondi , ¢ maleftanci ,

Che vanno per le frrade mendicando ,
fo che fono in arnefe tanta male

Mi ritrovo in grandiffimo viluppo ;

Temo effer prefo in vece d un Gaiuppo ,

E finir la mia vita allo Spedale .

VEST E la coracxa da baffone . E’ armato a baftonate , vefte un’ armatura da,
difenderlo dalle baftonate ; $' intende che ¢ fortopofto a toccare {pello delle ba-
ftonate .

. RILEV-ARE . Intendiamo bufcare , confeguire , ottenere . Petr. Canz. 22.
M fempre fofpirar nulla rilieva ,

Onde (¢ bene figura ai rilievo yuo! dire Ratua di marmo , o di altro materiale ,
noi incendiamo rilevare , cioe bafeare ¢ qui intende bufcar mazzate . I) verbo ri-
devare piglia quefto fignificato da rilievo , che fono gli avanzi delle menfe de’ gran-
di, quali avanzi fi bufcano per Jo pit da coloro che feruono a tavola , donde di-
ciamo Viver di rilievi che yuol dir Campare d’ avanzi. Vedi (otto C., 5. flan. 47.
Franco Sacch. Nov. 154. Quando la croitata fu mangiata tutta , fenza far rilievo ne
meno de’ topi, Rilevare yuo) dir Quello efprimere che fanno delle parole i ragaz-

zi,g imparano a compitare .

STANZA LXVIII.

Comparifce fra tanto umcarro in piaga Con che la formidabil Martinazza
Da Farfarel tirato , earbariccia A lor , ch’ ¢ ch’ é, le coftole fRropiccia ,
Vobidiente al cenno della\eayz4 E quei Demon} in forma di Camoxza
Soda, nocchinta ,ruvida,e mafficcia . Vin tirando a battuta la carrozza .

In tanto , che fi fa la moftra de’ foldati di Malmantile comparifce in piazzas
un carro tirato da due Demonj in forma di capra faluatica , che quefto vuol dir
camoxza , la quale per lo pil fi trova ne i monti del Tirolo . Plin, lib. 12.cap.37
la chiama Rapicapra . | noftri antichi diflero Stambecco , il Lat. ibex . .

_ PARFARELLO , ¢ Barbariceia , Nomi di due Demonj dal noftro Pocta cava-
- da Dante, del fignificato de’ quali nomi vedi gli Spofitori fopra il medefimo

ante .

, NOCCAIVT A, Piena di nocchi , che fono quei piccioli rilevati come bolle ,
iquali fi veggono per Jo pid ne i . di pruno , di forbo , ec, che gli rendono
2

rnvidi

x .
ita MALMANTILE (5

ruvidi , eli chiamamo ancora wedi, come fanao i Latini.,. .
MASSICCE , Intendiamo tutte quelle cofe,, che dal.
te di materia ftabile , ¢ folida , ¢ non vote, 0 vane ,0 in
deboli, siqech ¢ rralnepeiegig
CHé ch'é, Ad ora ad ora,.di quando in, quando 4 | argh
ST ROPICCIARE , Fregar qualcola con)p altro ,ed i Lat
Forfe € corrotto da froppicciare , che pare fi dove ie an ee
cio, con che per lo pil fi fropicciano gli arnefi per liberargli )
Sropicciar le coftole a uno yuol dire Baffonare uno... 07 RA
TIRANO (a carrozza a battuta, Nona battuta di mufica »maab
mazza , con Ja quale Martinazza la baftona, » ‘eerie
STANZA LXIX, STANZA LX
Coftei ¢ queila firega maliarda , Ove la notte al nace eran concorfe
Che manda i cavallacci a Tentennino,. < Tutte le Streghe anch'effe ful ca
Ed egliun punto a comparir non tarda Z Liavosi col Bau y le Biliorfe
Quand ella fa lo feaccio,o ul pentolino, A ballare ye.cantare ,e far te
Come quand'ella fi unge,e s'inzavarda Ma quando preffo al di
Tate ignuda nel canto del cammino , Fa di moffi
Ler andar col Barbuto foto il mento Come a coftei , chor vienfene:
Con la granata accefa a Benevento, E in [yu quel carro nelCaftell
STANZA LXKL
E la cagion si ¢ , ch’ ella ne vada Perche vi,
Adeffoacafa tutta in caccia,e in furia, _ C alla [ua patria
L’ haver veduto dentro alla guaftada Percio, fe nulla fuffe di
Vn fegno , che le ha data cattiv’ uria ; We viene anch'effa a dar il
Martinazza é una di quelle fireghe , le quali. coftringono il Dia
lo ftaccio , e il pentolino, ¢ con ungerfi per farfi portare a B
greflo de’ Diavoli (otto il noce : Quefta Martinazza adeflo, fi fa
famente daquei Demonj a Malmantile , perché ha veduto-nella cara
da fanguigna , che le prefagilce la caduta di Malmantile , onde vi fis
ancor’ efla per dare il fuo aiuto . (Quefto nome di Martinazza ¢ nome 3
quella ftrega , ¢ fregherie fon tutte dal Poeta dette pep accennare I’ opinioneé
alcune donnicciuole, le quali portate dall’ illufioni diaboliche y fi danno a ¢rede
re d’ havere effettivo commerzio col Diavolo . > tit oo
STREGA. Vedi fopra C, 2. ftan, 11. Viene da frix uccello
detto a fridendo , {econdo Ovid, fatt..6. aati
Eft illis frigibus nomen, fed naminis bnius, —
Canfa,quod horrenda ftridere noite folents
E quefto uccello ( che forfe era I’ Arpia., ma Plinig,dice 5 che non
fofle ) credevano gli antichi pid fuperftiziofi., che rapifié i bambini-
Et ab huius avis nocumento feriges Latini appellabant mulieres puellos'
raflu, E diqui ancor noi le chiamiamo ftreghe , che le:
da far malic , fattucchierie , ed incantefimi , ¢ perd chiamate.ancora J
MANDARE un cavallucio, Magdaceuna citaai _sioé chiamal
gindizio criminale con polizza. E quette polizze de Giudizz) Criminal

    

  
 

 

 

   
  
   
 
 
    
     
   
 
  
        
 
 
 
  
       
 
 
   
  
  
    
  
 
   
     
 
 
 
   
 

R SRSCER Esa ase EE

—
=

wERERRE

TERZO CANTARE. 181

renze fi dicono ¢avallucci a differenza di quelle de’ gindizzj Civili, che fi chiana-
no Citazioni; ¢ quefto nelle polizze criminali ¢ ftampata l imprefa, o
contraffegno del Magiftrato criminale ,che ¢ un’ Huomo a cavallo armato ; qual
con ¢ chiamato comunemente Cavalluccio .

-TENTENNLNO .Nome dato dalle noftre donne 2] Demonio per non !o chia-
mare Diavolo; quali rexrarore; col qual nome ¢ nominato preflo San Matteo
Cap, Vers. 3.00 t i :

BA lo paccio ye il pentolino, Favoleggiano , che quelle donne Maliarde, ¢ Stre-
ghe , che habbiamo detto, aes fare diverfi incantefimi per ritrovare cofes
perdute , © per ottenere altri loro intenti , ¢ fra quefti incantefimi fare lo Paccio ,
0 it Pentolino, o la caraffa ; Si che dicendo Fa /o ftaccia , e il pentolino intende fa in-
cantefimi. Quei che indoyinano per via di flaccio fono detti dat Greci Co/cino-
mantels .

COME quand’ ella s unge , es inzavarda . Inzavardare ,¢ uno impiaftrare con
materia morbida , e vifcofa , atta a diftendere come il lardo, I) Poeta feguita,
la vana., ¢ fuperftiziofa opinione , che quefte tali donne vadano ogni tanti
giorni al congreffo de’ Diavoli fotto i] Noce di Benevento : Ove da notre a/ noces
eran concorfe; al qual luogo dicono efler portate dal Diavolo in forma di caprone,
che quefto intende if Barbuto forto al mento , ¢ cavate dalle loro cafe per la gola.
del cainmino (¢ perd dice nel canto del cammino ) dal medefimo diavolo forzato a
far tal funzione da quegli uatumi , che dice efferfi meffa addoflo la medefinas
donna ; a quale poi a detto congreflo fa rempone, cioé fi da buon tempo; fi piglia
tutti quei piaceri , che Je vengono in fantafia quella notte ; Ma ful far del giorno
Ie conuien partire , ¢ i] Diavolo in:un baleno la riporta al fuo paefe . Tale opi-
nione hanno fimili fcimunite ; ed o fia per effetto di matrice , 0 pure per opras
del Diavolo , che per illufione faccia-loro apparir per vere tutte quelle (ciocche-
tie , che effe fi fingono nella tefta , I’ effetto ¢ , che effe fi credono d’ efler’ anda-
te veramente a Benevento , ed cffere ftate riportate dal Demonio al loro paele ,
quando effettivamente non fi fono moffe del letto..

. GRANAT A, Bran mazzetto di fcope, od’ altra cofa fimile , che s' adopras

{pazzare ,¢ ripulire le ftanze . E con quefte granace accefe in mano dicono,
che tali ftreghe vadano cavalcando fopra un Caproneal detto Noce di Benevento.

BAV, ¢ Biliorfe , Quelti nomi bau , biliorfe , orco., befana , verfiera , ¢ altri
fimili , (ono tutti inuentati dalle Balie per fpaventare i bambini , ¢ rendergli ub-
bidienti,perfuadendo loro , che.quefti fieno {piriti infernali , ¢ perd il Poeta nu-
mera fra i Diavoli il Bau, ¢le Biliorfe,, per accomodarfi alla capacita de’ Fan-
ciulli , per li quali profefla-d' haver compofta la prefente opera. Vedi fopra Cs
2, flan.'50. 1 Greci il cemibalo per chetare 1 bambini dicono Carabax

FAR tempone., Darfibeltempo ; Stare allegramente , pigliandofi tutti quei:
poliskhoee pud, ¢ si pigliarfi., che diciamo anche /euazxare ; trivnfare ; far
juonacera; Genioindulgere, litare Genio , diflero i Latini. La Compagnia della
Letina infegnando ., in-qual luogo fi deva pigliare laccafa per rifparmiare , dice:
Vorriano le noftre.cafe effer in una quafi dall altre feparata comrada , lontana‘da vie, ©

piagze pubbliche y dove all? occafiani fi fefteogi ,¢ fi facciatrebbi , e tempone.

BATTER il taccone.. Elo fteflo , che barter la caleofa, dewto fopra _ queftar

2 lan

 
YY Bate

* trovafi nei Latini /olwm vertere . &

    
   
  
 
  
  
 
    
   
   
    
   
 
     
     
   
  
   
 
 

182

C, flan, 60,,cioé ¢amminar via; andarfene . Si dice an
dice il fuolo della fcarpa , cioé quella parte , che pofa in
- th
VENIR di punta. Venir con velocita , a dirittura ; che diciamo :
vela . Vedi {otto C. 6, ftan. 10, Credo fia originato dalle barche,
venir di punta quando vengono a dirittura fenza volteggiare .
IN caccia , ein furia, Cioe in fretta , frettolofamente , e con furia,
no coloro , che fon cacciati ; che perd diciamo ; Corre,che par ch'egit| ba
dietro , Incedit quafi in fugam verfus . A
GV AST ADA. Specie di vafo di vetro per ufo di conferuarui
fteflo , che caraffa dai Latini detta Phiala, L’ Autore diffe fopra nell’
antecedente , che Martinazza era (olica fare lo Staccio , ¢ il Penroline, €
la Guaffada ; quefte maliarde , ¢ fireghe empiono di fuperftiziofi li
raffa , 0 guaftada , e facendovi mirar dentro da un fanciullo innoc
dire di vederui dentro quel che hanno defiderio di fapere , ¢ tutto
le perfone femplici , ¢ cavar loro denari di mano. Quefto indovinare
acqua , fu anticamente prefio i Perfiani , ¢ da’Greci fi chiama Aydro)
queflo habbiamo un detto Géi ha +i diavoio nell' ampolla per intendere
dovina ogni cola. Z
CATT IV’ uria, Cattivo augurio. Quefta voce Vria corrotta da
ta per lo pid dalle donnicciuole , deta fenza aggiunta di cattiva , o bt
tende cofa , che non piaccia. La tal cofa mi dé uria : © s' intende mi
mi da impedimento , mi da noia ; da che fi pud credere che fia
xegia , che pure yuol dir noia , faftidio , impedimento, ec. 0 forfe ia
che fuona Jo fteflo , che xgeia , o forfe in vece 4’ ombra, che é il
do vale per impedimento , /a tal cofa mi dd ombra , per la tal cofa
Siche Vria , xegia, ubbia , ed ombra faonano tutte lo fteflo ; Vria , e
ufate per lo pit dalle donne , ¢ ’ altre fon pi: comuni . Si potrebbe
fecondo i] Monofino , che la voce ria veniffe dal gteco Vria , che
profpero , ¢ che fi come habbiamo per coftume di dire buona , 0 ¢é
quantungue /orre fignifichi affolutamente bene,c felicita;cosi habbiamo.
di dire buona, o cattiva Vria,quantunque Vria fignifichi (empre feli
Greco Vria , Nello fteflo modo, benché prefso i Francefi bear figni
licita ; voce a loro derivata fimilmente da) Latino augyrinm ; dicono
malheur , quali buona , ¢ cattiva wria , cioe buona , e mala ventura ; €
doci feruir bene di quefta parola Via , come vocabolo di mezzo, dour¢!
giungerci buona , 0 cattiva ,¢ non dirla afsolutamente , ¢ fenza detta
come habbiamo accennato , che molti fe ne {eruono ; ma I’ ufo ci Lil
aftrufe ftiracchiature . ‘
SE nulla fue, Per tutto quel che potefse fuccedere, Se accadefse q
grazia . 1 Latini in un fimil modo per isfuggire il cattivo augurio , ¢
nare cofa infanita , come ¢ la morte , dicevano : Si quid patiar. Si gi
manitus acciderit , Se Dio facelse altro di me , con tutto cid , ¢c.
NE viene anch' effa 4 dare il {uo difegno. Con quefte parole moftra
quanta gelofia haveva Martinazza di non perdere J’ autorita , che «

  

     

   
    
    
    

 
 
 

ale

=
=

Se

 

wo
off
3
o
é
34
po!
4
eo

Y

 

 

TERZO CANTARE; 183
Malmantile , ed il fofpetto di non efser levata dal grado di Salamiftra , che go-

deva , come accennammo fopra in quelto C, ftan, it,

STANZA LXXIL STANZA LXXIIL

Fuggh tutta la gente [paventata Figuriamci vedere un [acco pieno
ell’ apparir dell’ orrido {pertacolo , Di zucche,o di popon fopr’ aun giuméto,
La praca fu in unt attimo fpayzara , Che rottafi la corda , in un baleno
Pur un non vi rimafe per miracolo, Ruzziolan tutti fuor ful pavimento
Cosi corrende ognuno all imparzata E nell’ urtarfi batton ful terreno:

Si fe Cun P altro alla carriera offacolo; Chi fi perquota,e chi s'infranga drento

Chi da un'urton,quell'altro da un tracollo, Chiff sbucci in un fafso,e chi sintrida,

Chi batte il capo ye re colle, Ed un altro in due parti fi dsvida .
STANZA LXXIV.

Cosi fa quella raxza di coniglio , A tal che in veder quello fcompiglio ,
Che nel fuggir 1a vifta di quel cocchio obo ben prefo( dice) qui lo ferocchia,
Chi fe rompe ta bocca yo fende un ciglio Mentre a coftor cosi comparir voli:
E chi fi torce un piede ,echiunginocchio; Sapeva pur chi erano i mici polli ,

1) Poeta defcrive aflai vagamente il timore , ¢ lo fpavento , che eatro addoffo a

ei di Malmantile per Ja vifta del Carro di Martinazza , la quale vedendo colo-
ro cosi fpaventati , fi pente d’ eer quivi arrivata in quella guila .

IN xn atrimo.\n un momento.Corrotto da atomo.Si dice anche nn baleno,come
nell’ ottava 73. feguente . Jn un batter d! occhio. V. {otto C, 10. ftan, 42. dal Lat.
Etu oculi:En atomo difiero i Greci. Dante Inf, C. 22. Subito,e (peffo 4 guifa di baleno,

NON ve ne rimafe sy miracolo , Fuggiron tutti, che non ve ne reftd pur’
uno . Tanto efprimeva fe havefie detto: Von ve ne refi pur’ uno, Ma col dires
miracolo da maggior’ emfafi , ¢ feguita I'afo; ¢ vuol dire farebbe ato creduto mi-
racolo fe un folo vi fufle reftato .

CALL impazzata., Acalo; Come fanno i pazzi , cio fenza confiderar gquel-
Jo che facevano , o dove effi andavano . #’ il latino perperam ,

VRTONE . Percofia che fi da con tutta Ja vita in un’ altra perfona, 0 in uns
muro, oaltrove , cd lo fleflo , che Spinta, ne vi (0 fare altra differenza (es
non che Vrtare vuol dir percuotere.a cafo , ed € i] Latino ofendere; © Spingere
yuol dir Mandar uno innanzi , o-indietrocon violenza , ed ¢ il latino émpellere.;
Ma nondimeno-#rtone , ¢/pmta fi pighiano |)’ ano per I’ altro 5 fe bene non fi di-
rebbe Dare una fpinta in un muro ,'0 altra cofa immobile , che fatta mobile co-
mie farebbe un muro {ciolto per farlo rovinare , fi direbbe Dare una fpinta. A
= quafi recifo da piede per atterrarlo 5 fi direbbe Dar Ja {pinta per farlo
cadere , ec,

TRACOLLO . Accennamento di cadere . Extra collum pedis ire; 0 pure detto
cosi quafi Tracrello. Vocabolario delia Crufca . Tracollato addiettivo da rracol-
dare 4 che vale lafciar’ andar gid il capo per fonno , o fimile accidente.

GIVMENT®O . Si dice propriamente I’ afino  benché s'intenda anche ogni be-
fiiaccia da foma.. Cosi prefio i Latini: Quello che in $, Gio, cap. 12, chiama-
to pullus afine yin S, Matteo cap, 21, & detto pulls filins fubiugalis, Puledro , figline-
do della giumenta, :

RVZZOLARE. Gisare per terra ; che diciamo anche Rotolare.

I3-
 

184 “MALMANOTILE?

INFRANGERSI, Sflagellarfi , ammaccarfi » disfarfi ,
76. C.11, flan, 12. i I ti
RAZZ A di Coniglio, Gente timida , ¢ codarda . Si dice poltrone come
giio, perché quefto animale , che é {pecie di lepre ; come quella ,
PIGLIAR (0 ferecchio., lngannarfi , Far’ errore . Lo fono ft
credendo di (tar bene , ma ho pre(o lo (crocchio; cioé mi fono it
fono ftato male. Il proprio fignificato della parola , ferecchio &
trovar danari,piglia a credenza una mercanzia per ventici

   

 
 

 
 

quefto,quando noi facciamo una cola , che non ci torna poi bene, ne
utile, ¢ gufto, ma pil tofto ci ¢ di danno , fi dice pigliar fo ferocchio ,
S-APEVO chi erano i miei polli
Cognofco oves meas.

   

STANZA LXXV. STANZA LXXV
Scefe dal carro poi per impedire Percio fi ferma ftrambafciata ye.

Cosigran fuga ,¢ rovinofa fola;
Ma quei vit pis fi frudiano a fuggire y Dalla Carretta fubito di
E moftraognun fe rotte hain pie le (uola, Eqgli fi lancia addoffa ac
Chi finalmente , come fi {ual dire Cosi correndo tutra fi
Chi corre corre , ma chi fugee vola, Perche quel Diaval vanne
Ond' ella y ben che adopri ogni putere , Pur ( dicendo: arrila 5
Vede che fara tordo 4 rimanere . Lo fruga si,ch' al fin la
Martinazza {cefe dal carro per fermar quella gente , che fuggiva ,@!
correr lor dietro , ma allora si , che coloro fuggivano , onde ella
a uno di quei caproni al fine gli arrivo. EB qui termina il terzo Cantat
FOLA, Quantita di popolo,che furiofamente corre a qualche luo
to da i Cavalieri , che gioftrano, che dopo , che fi (ono foddisfacti li
a uno per volta a giofirare, in ultimo corrono al Saracino ( cost
mezza figura , 0 bufto 3 di Moro , o Saracino , fatta di legno, efi
corrono dico al Saracino tutti in truppa, uno perd dopo I’ altro,
far la fola, In Latino potrebbe dirfi : exerceri ad palum. Vegezio
lib. 1. cap. 14. Tiyro, qué cum clava exercetur ad palum , baftilia quogue
gravioris y quam vera futura {unt iacula , adver{us illum palum tamquam:
minem sattare compellitur , E fi dice fola, 0 folata d’ uccelli , di popolo
tender di cofe che velocemcate fi muovono.in quantita , ¢ prefto finile
ta di vento, Studiare a folate. Lavorar a folate,ec, Forfe meglio fola y
fica quel che i Latini dicono Adagna hominum vis , vel turba , aut fummafi
bomsnum, Si come noi dal calcare le ftrade,che fa il popolo.e daliu ee
¢ ftretti , diciamo Vna molticudine numerofa di gente , una gran calc
Franzeéi nella lor lingua la dicono fowle , cioé fol/a dal verbo fouler ,
calcare , Da folla abbiamo fatto Afollarfi , ¢ Folto , denlo, calcato
tarfi, far furia ,far preffa : lo ftelso quali che -Afollarfi tutto deriva
tura dal Latino follss, nel quale fta I’ aria ferrata in modo, che pi
capire . i ’ vi

 
 

  
 

Sie
Vitae

 

 
     
  
   
    
   

non ne vale venti , ¢ poi Ja vende quindici , ¢ quefto fi dice pigliar lo fero
Plauto difle : Emere caca , vendere oculata die. Vedi forto C. 6, flan, 60. |

  
  
   
  
  
   
    
  
  
    
   
  
      
  
   
 
  
 
  
   

- Sapevo di che qualita eran coftoro , @ il

Kitorna indietro, ed un de’ fi i

 

a la

TE Ee a SE
 

SARRERLL ELLA TERRES DEN

~

 

TERZO GANTARE. 185

      

 STVDI. SZ, I) verbo fudiarfi er affaticarfia far prefto, o {pedire
tuna cofa , che diciamo anche menar le ioe . Per efempio : pie 5 nee il
tempo é breve , ¢.non finirete , fe non fate prefto . Qui intende s’ affaticavano a
fuggire infeare : al che s' adatterebbe i yerbo szeumbo , labore , ed anche+

Studeo, ¢ quelto dal Greco fpexda, afrertarfi .. .Nel,Salmo : Domine ad adivandium,
me feftina . conn Tddio , Tease ad! aiutarmi . eae Sic feftinanti femper lucn-
pletion obfear y a colni che fi fiudia @ arricchire il pi riceo da impaccio .

MOST RAR le fuola deile fearpe . Corscr velocemente ; perché cosi s’ alzano
aGai i piedi, ¢ fi moftrano le (uola delle (carpe. I Greci pure dicevano in queflo
propolito Canum pedis offendere , Si dice tan Battere iltaccone , che vedemmo
fopra in quefto C, flan. 79.

CAL corre corre y machi fuece vole. Detto fentenziolo , che fignifica , che mol-
to pili forte corre quello , che ¢ perfeguitato , che non corre colui , che lu perfe-
guita , perché la. paura gli mete I’ alia’ piedi ,¢ per quefto dice Chi fugge vole. .
Vergilio dilses Pedibus simor addidit alas ,e Dante Inf.C. 22.

E poco walle yches' ali al fofpetto , Non potcro avanzar .

Intendendo ,.che il gran timore , che hebbe del Demonio quel dannato,|o fece

efser pitt veloce , chet! ali di quel Demonio , che gli correva dictro. Della pa-

, , rola agit Ipiegantidima della yelocita appreflo Vergilio,vedi Seneca Epift, 108.

PARE tordo a rimanere . Cioe rimarra a dietro , ¢ non arrivera quella cana-
glia’. sfl.giuoco de’ tordi ha qualche {imilitudine con ’ Amilla de’ Greci , guia de
certo iafhu inter dudentes certemen eff , come dice il Buleng. de Ludis Veterum cap.
a4 clagara fi dice in Greco.amide . Nell’ Amilla fi tirava una palla dentro as
wa fegno , o-circolo ,.¢ colui perdeva , la di cui palla ufciva , o non entrava nel
ciccolo., Nel cordo non fi fa ne {egno , ne circolo , ma fi tira una piccola palla_,
( da noi a diftinzione dell’ altre palle detta grille , come vedremo forto C. 6. ftan.
pe colui., chelatiradice : 4 pe/sare , cioe a pafsare con la palla il detto
gsillo, © a rimanere , cioé rear con Ja detta palla di qua dal detto grillo ; cost
sirando ciafeuno,s’ ingegna di pafsare, o rimanere il pill vicino a detto grillo,che
egli pud ; perch chi meno lo palsa , o meno addietro gli rimane vince la pofla ,
¢d.a quelli, che-non pafsano , o non rimangono, quando devon rimanere, o paf-
fare , vince il,doppio ,¢ quefti perdenti fi chiamano Tordi , € fono di tre forte ,
perché tre fono i cafi del tiro ; cioé Tordo a pafsare é quello , che pafsa di Ja dal
grillo quando deve rimanere . Tordo.a rimanere quello che rimane di qua dal
grillo,quando deve paGare . E Tordo femplicemenie fi dice quelio,la di cui palia
refta in dirittura del.grillo per banda,e quefto da alcuni fifa che non vinca,ne per-
da, daaleuni , che perda folo la meta degli altri tordi, fe ¢ pili lontano dal gril-
Jo di oo che vince 5 efe é pid vicino non perde ; da alcuni gli € perme/so riti-
rare fino a tre volte , quando perd fempre refti in dette tre volte nella medefima
dirittura del grillo ; e quando non paffi , o non rimanga perde una fola pofta: ¢
f{empres’ intenda pafsata , o rimafta la palla quando fra ¢fsa , ¢ il grillo pola,
interporfi un filo in fquadro,{e perd non 1o tocchi non per banda , ma per quella
parte,dove hada rimanere , o reftare ; ¢ tutto fi fa {econdo le conuenzjoni , e+
atti. Quefto giuoco per lo pil ¢ ufato da’ ragazzi , o dagl’ infimi botiegai di

Firenze ; i quali nei giorni felte , ufcendo dalla Citta per andar’ a pigliar’
3 Aa aria
%
ar, tie 5:

 

 
 

 

    

- Aty I
(a 1 Al
‘ in z ‘hie sa
186 MALMANTIER()
aria nel camminare giuocano a quelto giuoco, \
‘no a chi perde, ¢ quando n’ hanno fegaatitanti, ¢
bere , e da mangiare , fi férmano alla prima Ofteria’; €
quantita di danaro , che ha pérduto . Hor tornando a pro
Unazza fard tordo arimanere, ed intende, che rimarra (ro >
quella ciurina . ' pt ge aag
STRAMBASCIAT A , Affannata ; Oppréa dal’ ambafcia 5:
dificulta di re(pirare cagionata dalla violence fatica nel correre ,
prabbondanza d’ alito. Dante Inf. C.24. & perd leva si; vinci I
qui per avventura e4mba/ciadore , che piglia a fare amba/cia, cio’
dare a quel Perfonaggio , o Citta , a cui eglié inuiatoys ©
S/lancia . Si getta ; cioé con un falto monto preftamente a.
rone . o eal
: S/rinfacca, AGomiglia Martinazza (che cavalcata in fal fuo Capi
a quando s’ empi¢ un facco di roba leggieri,la quale fi mandi gil co
stiuarla , ed empier bene il facco , quefto s’ alza , es’ abba(sa
faceva Martinazza a cavallo ia ful Caprone , il quale faceva a lei
andando baielloni , cio’ a falti,come ¢ il proprio correr delle capre
ce balzeloni viene da balzellare , che lo diciamo il faltellar delle le
di Maggio , ¢ Giugno , che elle (ono in amore, e la caccia che in talt
fi dice andare al ba/zedo. Del cavalcare la beftia nera , ¢ cornuta V.
ARR 1d, Cammina li, Va la. Termine ftimolatorio ufato ju
ec, dai vetturali. B’ ben vero , che vedendofi uno a Cavallo, i ;
ciamente , fi fuol dire per derider colui 4rri /@ quafi diciamo yaa cavalea un’ af.
no , ¢ portato da queflo ufo 1’ Autore fa dire a Marcinazza Arré jd. |
Jo fa venire dal Greco Errbe , cio , va via, a
CARNE cattiva, Animale vituperofo. Diciamo carne cattiva’, 0 cat
‘di carne ancora a quegli huomini , che fono di genio {ciagurato ,€ + Oe
de fi dice quafi in proverbio , ¢ per ironia di chi fia magro, opi di perl
ma fia maligno,e aftuto,e come fi dice ne’ faoi panni @ vi fia tutto 5” 0
Stornello , poca'carne ,e cattiva, Equi fi pud anche dire , che I Aurore la’

carne cattiva , perché era capra , che fra le carni, che fi mangiano’, ae
. VMs oyet

 
   
 

 
   
  
  
    
    
    

  
   
   
    
   
   
  
  
 
  

ya.
CIVRALA.Dal Lat. turmaSi dice propriamente degli Schiavi
Jera: Ma Gi Piglia ancora per quantita di gentaglia,e qui intende di
glia , che fuggiva. Vedi (otto C. 5. flan. 16.5 € C. 11, ftan. 16;

ae

FINE DEL TERZO CANTARE,
‘ ve

ee
atk

 

 
 

 
   
 

cee
OCANTARE,

ARGOMENTO.

Tguerrier di Baidon fon mal difpofti
Perche la fame in campo gli travaglia;
Lifendefi ,¢ Perlon lafciamo i pofti,
Won vedendo arrivar la vetrovagiia .
Pfiche non tiene i fusi penfiers afcofti
et Calagrillo Cavalier di vaglia ,
Che promette aiutar la damigella ,
E pofcia afcolta una gentil novella.

 

At

 

STANZA I.
Maia vincit amor : dice un Teftoy
Exun'altro diffese dette piit nel fegno:
Fames Amorem fuperat . E quefto
E’ certoye approwacgniic’ha un po d'igegno
Perch? quantunque mor fia si molefto,
Che tutti i Martorelli del fue Regno
Dicano ogn' ora; Ahi laffo,io moro,to pero,
Enon fi trova mai , che cio fia vero,
STANZA IL
Non ha che far niente con la fame y
Che fa da vere , pur ch’ ella ci arrivi;
Poffon gli amanti ftar fenza le dame
1 mefi, ¢ gli anni se mantenerfi vivi ;
Ma fe due di del confueto frame
1 poveracci mai rimangon privi.,
Ei bafta , che de fatto andar gli vedi
© porre il capo dove il Nonnoba i piedi.

SAAS SSS

STANZA IIL
Tal che fi vien da quefti effetti in chiaro 5
Che d' Amore, la fame ¢ piit potente ,
Ond'écognun di lui pitt quefta ha cara,
E quand’ alle {ue hore ei non la fente
Lamentafiye gli pare oftico, e amaro;
Percto riceve torto dalla gente ,
Mentre ciafcun la cerca, ¢ la defia y
Es ella viene , vue} mandarla via.
STANZA lV.
eAnzi la {caccia , come un’ animale
Sul buon del definare , e della cena ,
Per quefto ella talor, che i'ha per male 5
Psu non glitorna;ovver per macgior pena
In corpo gli entra in modo,e nei canale
Che non Lempiercbbe Arno con la piena,
Come vedremo,c' a Perlone ha fatto ,
C” a quefto conto grida come xn matto,

Il noftro Roeta riflettendo., che nel prefente Cantare gli conuien de(crivere las
fame , che era.neb campo di Baldone , per non efferui ancora comparfa la muni+
zione di bocca, s' introduce col provare, che la fame é fuperiore ad Amore, quan.
tunque la maggior parte degli dupatinl, leapitando Vergilio Eg), 10. dove canto:

é ; a2

Omnia
———————————7~E

 

 

   
      
 
 

188 MALMANTILE
Omnia vincit ‘amor ; @ ‘nos cedithnys ante
éica che Amore fia pid 56 fup
ver provata quefta fua in fi maravi;
pid potente , e pit ftimabiley ¢ defiderabile ych
fere fcacciata neJla maniera , che ognun procura di
habbia ragione di vendicarfi di tal difprezao , 6 con I" and
de\ mangiate ,' col venir troppo 5 Quand6 nof fi ha chen
moftrare ch’ é feguito a Perlone . oat
MATTORICLS agen nea es ma
4H! laffo . Inverpolizione , che deniota dolore . dica fon
da} dolore , dal travaglio , ec. B il Lat. bem , bei mihi, Francele Helas,
NON ha che far niente . Non ¢ ¢ luogo da far comparazione . Non
ifperto alla fame . ;
TRAME Si dice il fieno , paglia , © altro fimile che fi dap
fic : Maqui lo piglia per cibo degli huomini, come ¢ fcherzofo
ciamo /rameggiare,quando uno va trattenendofi col mangiare alqui
do che venga in tavola la vivanda per definare , o per Ja cena, che
concellare, Vedi fotto C, 7. flan. 10,
‘POVER ACC/IO , Epiteto che efprime la compaffione, che s*ha
di colui, il quale finomina. Vale per infelice , difgraziato , ec.
PORRE il capo dove il Nonno hai piedi. Farfi fotcertare. Motire .
tura fi dite ; Appowiad parres fuor. kr eae
RICEVE torto, Non (e le fa il giufto: Non fe le fa il dovere, 7%
rio di diritto . E fignificano quefto Giufto ; ¢ torto Ingiafto ,“cotne
ra C. 3. flan. 66. None sn corpo florto anime dritto. enh
ANIMALE . E' nome generico, che fignifica ogni’ {pecie di
coftume pigliarlo in fpecie , e per azmale intender folamente le!
gue poi che dicendofi animale a un huomo's”intende un hnuomo
giudizio,in fomma un huomo beftia. Bocc.n.79, dice: Conofeendo
efer un’ animale, Vedi forto in quefto C. ftan. 5 1.°Cic, Wonne vides, J
ZL canale , cine il canal del-cibo , che & la, Zola’ + il comiderto adesbarconiy OY
cosi vien defcritto in lifgua*furbefca dalla plebe Fiorentina . ea
NON ! empierebbe Arno con la piena, Non V-empicrebbe.Atnojqiia
pioggie vien groflo. Iperbole ufata per intender"nno ,vchenon fi
gordo tanto del cibo , ‘quanto dei denari , che ilatini differoD
dun huomo , quem eos non nutriet, illum nec-Bgypras .“Empiti
per difpetto a uno , che non fi trova mai fazio; modo'baffo .- oe
STANZA V. STAAwNoZ A OVd,
Defta ? Anrora omai dal letto feappa,
E cava fuor'le peyue di bucato ,
Poi barre il fuoce ,e quocerfaila pappa
Per il giorno bambin c? allora ¢ navo ;
E Feboch’é il Compar gid con la cappa,
Econ wr bel veftito di broccato ,
C a nolo egli ha pigliato dal’ Ebyeo ,
Tuste {plendente vienfene al Corteo,

      

r

   

 

  
 
 

  

  
 
 
   
 
 

Z

 
 
     
   
    
   
   
      

  

 
 

 

fest!
5
“le
¢
ys
yi
i
wo
5
a

 

wee

QVARTO CANTARE. 189.

»Inoftro Poeta ( come habbiamo detto altrove ) hebbe notizia da Saluadore
Refa d'un libro Napoletano intitolato LO CVNTO DE Li CVNTI, ed in.
comporre l'aggiunta alla prefente opera fe ne val (e,cavandone qualche peafiero,
© concetto’, come vedremo ; ¢ quefto é quello della prefente de(crizione delia lc-
vata del Sole. Dice dunque che /uegliata ’ turora , efce del letto ,e cava fuora le
perze bianche di tucato ; il che allude alla chiarezza che apHOrA P Alba. Di poi
accende il fuoco!, e fa quocer 1a pappa per darla al Giorno bambino che allora ¢ nato.
E per quefto fuoco intende quell’ albore che fi vede all’ apparir dell Aurora , il
va crefcendo , ¢ piglia un colore gialliccio per lo vicino apparir del Sole ;
e perd dice che Febo viene con  abito di broccato d' oro tutto [plendente al Corteo del
warno bambino. E cosi intende che alla levata del Sole i Soldati di Baldone non.
ino ancora hayuta la provvifione per vivere , onde fono in collora, epartico-
larmentemolti diloro,che fono afluefatti a far fempre col ventre pieno.
PELZE di bucatePezee bianche pulite perché {ono di bacato,cioe non adoprate
dopo che furono imbucatate ; ed intende quei panni lini, che feruono per falcia-
se, ed inuoltare i bambini .
BATTE il fuoco , Accende il fuoco , Cosi diciamo , quando per accendere. il
fuoco fi batte nella pietra focaia, fe ben non fi batte il fuoco, ma la pictra, Ver-
gilio nel 6, dell’ En, dice.

 

quarit pars femina flamme

Abftrufa in venis filicis ——————

PAPPA, Pane boilito in acqua ; é la vivanda folita darfia i bambini quan-
do s' allattano , ¢ cominciano taliittaro » ¢ fi dice pappa perché eflendo la let-
tera , P..puramente labiale , ¢ facile a profferirfi come fono le lettere B, M. ¢
pero ne ibambini-fi-trova maggiore attitudine a proffcrir quelte, che I’ altre
confonanti , fi che pitr facilmente profferifcono babbo, mamma, pappa , bombo s
che padre , madre , mineftra , bere , onde le balie fi (ervano di quefte parole per
facilitare. la loquela.a i baibini, Tal coftume cra forfe anche negli antichi te
mani , come fi cava da Varrone, (nel libro ane ee » Ovvero dell’ alle-
vare'® figliuoli ) che per Papas intende guello , che intendiamo noi Tofcani
Pappa yoda Pein »che oe Satira 3. dite cs sia

Et fimilis Regum pueris pres minutum.,

I Grecivpute per «i loro bambini 4i {eraivano come noi, ¢ come i Latini_, di
voci di due fillabe.con raddoppiarae la prima fillaba,, per maggiore agevolezza
del rilevare layparola... Di.quefte parole bambinefche ne troveremo molte nella
»prefente Opera., ulate dal Pocta per {cherzo., o per accomodarfi alla qualita di
colui che fara parlare,¢ non perché fieno in afo altrimenti. Vedi focto in quefto
Cant, ftan,12.dove dice d’ un bambino.che impara.a parlare.

BROCC.ATO . Buna fpecie di drappo fatto.a fiori, es’ intends Deappo tel-
futo'con.oro..

- A NOLO eli ha pigliato dal’ Ebreo , Dice che il Sole ha pigliato a noloil {uo
{plendente.abito., per fignificare che lo.rende la fera,, come lo reftitui cone caio-
ro’, che:pigliano gli. abiti.a nolo per.un giorno ; ed intendere che il Sole alcon-
dendofi la fera alla noftravifta , Ja(cia guell’ abito rifplendente , che.s’ cra mello
a mattina,, Y
3 COR.

 

 

 
 

 

 

 

196 MALMANTILE |

ao
CORT EO , Corteggio | Codazzo di donne,ec, che gn
quando va a marito , o un bambino portato a
VONANESI genti , | foldati del Duca d’ V;
pellar  efercito dal nome del Generale, come Vaimarefi
COMP ARIRE in (cena, Venire in pubblico. Vedi fopra C.
LA materia che da il portante a’ denti, La materia, che fam
cioé 1a roba da mangiare ; fi dice anche Da far ballare il mento.
uefto C, ftan. 23. 2 portance fi dicg una {pecie d’ andare di cavalli. Il Lali
fr. C, 3. fan. 58. dice . 1 aE
Per dare il lor pertante ai denti afciutti ,
LENA. Vedi fopra C. 1. fan. 2.
EA maiticavan male, L' intendevano male , la fop n
E folito quando fi penfa a qualche cofa fifamente , ¢ con applicazio
care , onde Perfio delle compofizioni ben penfate difle: Remorfum |
Suem : E tal mafticare cos: penfando fi dice auche raminare,o dig
mafticare che fanno gli animali del pié feflo percid detti ruminantia
Vedi forto C. 6. ftan. 5. Qui fa bell’ effetto ’ equivoco del verbo
che pare che voglia dire / iutendevano male, ¢ vuol poi dire che n
Ic , perché non mangiavano , non havendo che mangiare . i
STANZA VII. STANZA V
E tra coftoro un certo girellaia , E, perch’ ei non bavea tutti

  
   
 
   
    
   
 
 
    
    
  
      
     

   

    
 
 
  
 

Che per U' afciutto va fui fufechini , Fu il primaad efclamare, r¢
Male in arnefe ye indoffo porta un faio Forte gridando:Obime
Che fu fin del Romito de Pulcini , Pel mal che vienein

Cit chi viel dir ch'ei dorman'ungranaio Onde Eravano,e Dow
Per c'bail maxzocchio pien di farfallini
E' matto in fomma,pur potrebbe ancora
Wan di guarirne,percht il mal da in fuora,
STANZA

Mentre di gagnotar gid mai non vefta E per vedere il fin dé
Colni ch’ é fenza numero ne rulli y Se ne van difcorrende g
Anxi rinforza col gridare a tefta , Del bifegnoch' effi han cb'il
Lafciano il fuoco ye ¢ vani lor traftulli, Perche fentono omai fons
Fra li fuddetti foldati affamati |’ Autore pone fe medefimo defcri
erfona , ¢ genio ; ¢ dice che egli fu il primo a gridare per Ja fame,
ravano ,¢ Don Andrea Fendefi ancor effi affamati s’ accoftarono a
tir la cagione di quelle ftrida , 3
Nota che il Poeta divide il periodo nelle due ortave,ottava,e nona,di ¢
to da qualcheduno criticato d’ errore , ma pero fenza ragione , non a
regola poetica , ia a pale vieti il poterio fare , come habbiamo detto
_ , G/RELLAIO . Huomo firavagante. Huomo che gira,s’ intende

hs pre »¢ che fa fcioccaggini , ¢ pazzie.

ANDAR ? afcivtto, Signi efier ro, ¢ con poca

Vedi fopra Ca: ftan. 68. % aad celia
VA infu fefeeliim. Ha gambe cosi fortui , che rafiembrano

  
   
 
 
 
 
 

     
      
  

  
    
  

   
  

 

 
 

 

eee

QVARTO CANTARE, 19%

Mine wfatifiimo da noi in quefto propofito; che diciamo , Camminare fu fulcelii .
 ALAL? in arnefe . Mal veltito: Mal’ ail’ ordine di {anita, d’ abito , ec. Lalli
tr, lib, 1. flan. 34.
eben Pcs navi ao che gli avanzaro
Qui fi conduffe afai mate in arnefe.

ekildiise:Dotee ta inde dello {puto dice.
iAutee _ Eccomi qui per raccontarne centey

Ben ch’ io non fia d’ accordo col ceruello,

E malagiato in arnefe ms fento .
Il Perfiani {Crivendo al Sereniffimo Principe D, Lorenzo dice.

do, che fono in arnefe tanto male,

    

site Mi ritravo in grandifsimo viluppo ,

ics Teme efer pref in vece d’ un galuppo y

via E finir la mia vita allo Spedale .

4 Franco Sacchetti Nov, 122. // Saccardo era guarito, e ftava bene in arnefe. Bocce.

walt) 2+0. 8. Partitofi aljai povero ,¢ mal’ in arnc/eda colui , col quale lungamente crds

ato.
it DEL Romito de’ Pulcini . Quefto fu uno che abitava poco lontano da Mal-
_- mantile , ¢ teneva vita eremitica , veftendo di lendinella a foggia di Francefca- |
yy nefealzo; Da coftu prefe il nome di Romito quel luogo vicino a Malmantile
ae che dicemmo fopra C. 1. flan. 70. E perché egli oltre al procacciarfi il vitto con
. chiedere gwinae aiutava ancora col autrire nella fua abitazione buon nume-
ai ro di Polli per vender.’ uova , fu nominato il Romito de Lulcini , Quando I Aa-
he © torecompole la prefente Opera , detto Romito era morto di gran tempo prima ,
f © pero dice che il /aio che eg\i haveva addofio fu fino del detto Romito , volendo
inferire che era gran tempo , che qucli’ abito cra fatto , ed in confeguenza oltre
, | all'effer vile per eficre-ttaco d’ un povero Romito , era ancora lacero, ¢ confa-
oA mato dal tempo .
yi! S AIO, Gonnelletto , o cafacca, o fimile parte d’ abito da huomo; dal Latina
Sagums . WVarchi flor. fior. lib 9, E forte il Lucco chi porta un faio , chi nna gabba-
gl nella , 0 altra.vefticciola di panno chiamata cafacca.
wis DICONO ch’ ei dornsa inun granaio, L' Autore medefimo lo dichiara , fegui-
a tando + perch? ba il mazzocchio pien di farfallini , fe uno dorme , o fi trattiene iny
ifet ua granaio , fi fuol’ empiere di quei farfallini che ftanno fra il grano; e quando
id diciamo: I}taleha de’ farfalliai,o delle farfalle,intendiamo E’ mezzo matto ; ¢
wif — dicetuello volante , 0 inftabile. E per mazzocchio intendiamo il capo , perch?
wi! mazzocchio era una parte del Cappuccio , che gia portavano i Fiorentini , fe-
,  "ondo chediceil Varchi nelle fue ftorie Kiorentine lib. 9, ll Cappuccio { dice egli)-
wt ba tre parti , cioe il maxzocchio,il quale ¢ un cerchio di borra , che gira , ¢ fafcia intor-
wt! = no intorno alla refia , ¢ di fopra , foppannato di nero di ravefcio , copre tutto ilcapo, Si
got dice a er mazzucco , © Cosi havea detto |’ Autore , ma havendo il
yl - medefimoa dipingere uno dell’ antico.Magiflrato di Firenze , mi domandd come
i era veramente I’ abito Civile antico , ed 10 gli feci vedere quefto Juogo del Var-
a chi , onde egli poi mutd , ¢ diffe mazzocchio per quanto vedo dal fuo feconds
.  Originale , che ¢ appretio di me~ 4
we moi IL,

 

 
}

  
 
  
   
  
  
 
  
  
    
    
  
   
 
  
 

192 MA LMA NTILED

IL mate da in fuora , Quando il male da in fuora , cioé ‘man
te I’ interna malignita , (uol’ effere indizio di falute ; cofui eflendo infer
pazzia, il dare in fuora di tale infermita ¢ il far pazzie; ¢ il Poet
potrebbe guarirne , perché il mal da in fuora , c1oe fpera ch’ et
olte pazzie , che ¢ lo sfogo del {uo male, ed il fuodare in

  

ha tutti i fuoi mefi. BY {propotiraco. Non ha l'iatera pe:
uello . Non é ftato tutti a nove i meli nel ventre di fua madre a p
ceruello . In fomima vuol dire Non ha giudizio; ¢{cemo. tf
£.4R ma ina. Diciamo far marina coloro , che fingendofi ftroppiati , e
piagati gridano , ¢ fi rammaricano per farfi creder tali ; che tanto vale inigi
propolito Marinare ,0 Jar Adarina , quanto rammaricarfi, o dolerfi di cola,
difpiaccia , ma per lo pid s’ intende di coloro , che fingono ; come per
lo (colare battuto dal macitro,fi dice far marina , quando fingendo che il
gli faccia gran male , piange , ¢ firide a pid non pofio ; che di dice anche
monello, Vedi fopra C, 3. itaa. 67. ak
VADO a Scefi, Quando diciamo; I tale ¢.andato.a Scefi, intendiamo
to , fe ben pare che diciamo é andazo alla Citia di Scefi, o Affi, p
bo /cendere ci feruc pec intendere morire, Virg. fucilis defcenfus .
PEL mal , che viene in bocca alla gallina, M male che viene inbocca t
na da noi € detto pipint dai Lat, peruira , E perche fra da gence baila in}
dire apperito fi dive appipio, pero cavano quefto detto':. / tale ha stimal
in bocca alla gallina , c10t la pipita , © intendeno appipite, ciot fame. E

tende il Poeta nel prefente luago con quelto derto piebeo . peo! t
ERAVANO . Ciot Averano Seminctti. Den Andrea Fendefi. Besdinando
Mendes . oat

PASCINA, Fafcetto dilegne ;Ed abbraciare infieme una fafcinayy
afcaldarfi,, € {pender ciafcuno ja {ua porzione nelle legne ; E vuol dit 0+
pertamente andare all’ ofteria , Oraz. Ligna /uper foco /arge reponens . 6
STRVZZOLO . Vecello noto , il quale mangia cosi voracemente , che it
ghiotcifce fino il ferro, Dicendofi veutre di fruzzolo stintende Ventre i
Vlin. degli ftruzzoli. Concoguendi fine deleitu devoratu miranatura , agat
AUNVZZ OLLI, Quci minuti fragmenti , che cafcano dal pane , quando
fpezza . E queft’ atio di cercare i minuzzoli nelle tafche ,e(prime uno che:habbia
grandiffima fame . odes
GAGNOLAKE . Voce corrotea da cagnolare , che & il guaire , chefanno!
cagnolini quando hanno bifogno della poppa. Se per avventura non lo
vatfimo dal verbo Latino gannire , che fignitica Rammaricarf con. parole no
affatto intefe mefcolate con fofpiri., ¢ fingulti, che é quelio , che nel prefeate?
uogo vuol dir gagnolare . =
E SENZA namero ne i rulli, E' matto . Nel giuoco de rulli fipighi v
© pil, o sr eis SORT > a. de ses hail {uo
che uno , il quale jama 11 Matto ; E perd dicendogi : 4 zale ¢ ih fengammme?
frairulli, i uathnide @ il rocchetto , che ¢ fenza numero, cio? il mateo «Quel
rocchetti fi chiamano radi , perché rizzati in terra in socal oa
a

nel mezzo , vi fi tira dentro con un Zoccolo di leguo grave tondo di

 

 
 

QVARTO CANTARE: 193

midale , il quale fi chiama rullo, € il giuoco fi domanda «'Rutli, ed alle volte
Samer chi pil ne fa erlest i kee tiro vince . Si coftuma anche tirare

dil no. wu
SRINPOREA | Ciotctelee lo Grider,

GRIDARE a

 

‘tefl, Gridar quanto pitt

0 il guaire. L. ingeminat . Si raddoppia,

fi pad. Si dice anche gridare 4 corr'huo-

mo ,0 quant’ uno n ha nella frrotea ; nélle canna; o' nella gla. Vedi fopra C. 3,
Ranbgjeiq ish si. pasalty

TRASTVLLI . Trattenimenti*. E’ voce da Fanciulli , ¢ qui vuol efprimere.,
che futiero veramente traftulli da bambini , perché aggiunge l’epiteto vani, come
era veramente il cercare de i minuzzoli nelle tafche . :

PER vedere il fine di lla feta. Per vedere in che haveva a terminare 7» Oa,
che fine’ fuffe (aeoqelbnomice + Quando un difcorfo , o un fuono » 0 un Can-
tare, © altro romore cémincia a venirci a faftidio diciamo : Quando finird quefta
Sefta ; quefta mufica 5 quefto chiaffo’; quefto bordello ; quefto baccano ; queffo mirfcaiore
fmili, Vedi forto'C. 9: ftan. §1.€€, ro. ftan. 53.

GRVLLO., Int

‘eadiamo'uho melancolico,sbattuto da cattivi effetti,e non affat-

to fano , che fi dice anche Acquacthiato ; E tal voce € prefa forfe dalla Grue uc-
cello (Spyruila)che quando fta fermo pofa un fol piede, ¢ tiene Pale baffe in ma-
niera’, che pare un pollo ammalato ; che pero tal pollo , ed ogni altro uccello
Cost'ammalato fi'dice Zruilo , 0 che porta i frafeoni, Vedi fotta C10, flan. 20.

SENTONO fuonar la lunga, Quando il Prete per

inuitare j popoli alla Meffa,

fuona Ja campana , e' dura Hs tempo , in contado dicono /uanar la lunga. B

da-quefto durate lungo tempo

STANZA X.

Cosi domandan chi fia quei ch’ efclama,
E metre grida jd urli st befPiali |
Glié dette; Quefioeun tale,che fichiama
Perlone dipintor de’ miei ffivali,
Vahuom @al mondos'acquifta gran fama
Nel far de’ ceffantts pe’ boccali,

E con gt indufpri ,¢ dotti fuoi pennelli
Suo nome eerno fa negli sgabelli ,

icendofi: I] tale fente fuonar ja lunga, s’ intende
me per effer lungo tempo , che non ha mangiato. E
pertamente diciamo : Eeli ha quella de! Carmine, s' intende
Chiefa del Carmine di Firenze,avanti fi dica la prima meffa
na-per un grande {pazio di tempo , ¢ quefto fuonamento fi dice da tu
del Carmine .

Per fignificar pit co.
la lunga, perché nella
,fuonano una campa-
ttl fa lange

STANZA XI,

Si trova in bale frato , ani mefchino,
Ma ben che il furbo ne mancect pochi ,
Ginocherebbe in fw pettini da lino -
Che ur'ora non puo viver ch'ei né iginochi,
Ma £ti vincelfe un di pur'un quatiring
dn vero fi potrebbon fare é fuachi 7;
Perch giocando fempre Siorno,e notte >
Farebbe a perder con le tafche rotte ,

STANZA XII,

Ginocoffi un {uo fratel gid la fua parte;
Suo padre fu deleinoco anch'egli amico,
Pero natura qui n incaca P arte

Havendo itato un genio antico ,

Coftoro ‘intefero , che’colui , il quale cos} gridava cra Perione

Coftni teneva in man prima le carte >
Che legato gli fuffe anco il belico :
Epriache mamma, babbo,pappa,e Poppe
Chiamp [pade,bafton, danari,e coppe,
> cioé Periones

Zipoli , che vuol dive Lorenzo Lippi Autore della prefente Opera ; ¢ fa che ven.

§a deicritto per uno sfortunato , ed oftinato giocatore.
i Bb

MET.

 
 

 

 

 
    

194 MALMANTILE,
METTE frida , ed urli beffiali , Stride, ed urla gagliards
perché lo fridere € proprio del porco ferito , ed wrlare &
cane , ¢ lupo ; {¢ ben ce ne ferniamo anche per I’ huomo i
DIPINTORE de’ miei ftivali . Pittore dappoco . EB’
ro , che {anno poco in qualfivoglia {cienza , 0 arte. V
E frvale diciamo un huomo gotfo , ¢ di poco giudi:
{carpa , che cuopre tutta la gamba , es’ ula per ca)
poco fi dice Pittor da sgabelli , da boccali , da colombaie , ec. come fi
fente ottava , che dice: Fa de’ ceffaurti ne + boccali , Econ gl indufirss
eterna il fuo nome negli sgabelli. Ma perché quelta fua modeftia , ed h
fia di pregiudizio al merito di cosi gran valent’ huomo , repli
tore riputatiflimo , come le belle opere fue chiaramente teltifi
firera il Sig. Filippo Baldinucci, fe mandera alle ftampe Ja faa,
Pictori , Opera degna d’ ammirazione si per le belle notizie , che fi
fa , esi ancora per {aperfi , che quefto erudito huomo |’ ha ritrovate a
fieme in breviffimo tempo rubato alli tanti riguardevoli affari, che p
benefizio Jo tengono continovamente occupato , ;
CEFF AVYTT/, Voce compofta delle note Muficali Ce fa , wt, ¢ 00)
ficato veruno , fe non che moftrandofi di dire la chiave del Cé fol fa at
Ceffo , che fi piglia per vifo , 0 faccia , fe bene appreflo di noi cefo vi
di cane, 0 grifo di porco, E quantunque venga forfe dal Greco ©
dir Capo , onde anche i Latini , chiamano Cephalea un certo dolor di
in Franz. chef fia capo; nondimeno noi non ce ne feruiamo fe non peril
per intendere una facia brutta, e fatta male; © perdl Autore 5 ¥'
tenda , che Perlone dipigne male , chiama cefi quelle facce , che egli dipl
per altro parlando pittorefcamente chiamerebbe Tefte. sped B,
bocc-dZE, E’ una milura fatta di terra cotta inuetriata capace deli
dun fiafco a, 3 ne ogni forta di vafo fia pil par
rande ; che fia perd di quefta materia, e figura. E perch? quetti boccali da
ai , che gli fabbricano in Montelupo (on dione ted eae se fenza un mal
mo dilegno , perd a uno , che dipinga male fi dice Pitror da Boceali 5 0» Pittoest
eMontelupo. oc et le
BASSO ffato,anzi mefchino. Povero mendico; Poverifimo
FVRBO . Propriamente ladro dal latino. fur , ed & parol ingen
tavia fi piglia per 4/tuto, (agace , caltrito,¢ che {a il conto fuo: Qui yuol
fo, perche ha il vizio del giuoco, Fur a furuo , i, migro diétus, Papiat.
_ AVE maneggi pochi, Intendi: maneggi pochi danari. Non gli vengs ,
gran quantita di danari. 2
GIOCHEREBBE {x i pettini-da lino. Intendiamo uno , che giod
ogni iore {comodo , come farebbe , s’ egli ftefle a federe in fui
no, che fon compofti d’ acutifime punte di ferro., pote Ne
‘SI potrebbon fare i fuochi. Si potrebbono fare i fuochi in fegno d’
come d’ una cofa infolita, Detto ufatiffimo, quando fi qualco
guito , che fiamo ftati buon pezzo afpettandola ; Che fi dice anche Sa
doppia, Vedi forto C, 6, Ran. 107. #

  
    
     
 
    
   
      
 
 

  
     
    
     
   
   

  
      
    
 

  

 
 
 
 

  

 
a

QVARTO CANTARE; 195
well PARERBE 4 perder con le tafche rotte. Perderebbe {empre : Farebbe a gara 24
ua chi pili con'te tafche rotte , quantunque quefte perdano tutti li'danari , che

ba ineffe fimettono, _

_ INCACARE, Difprezzare: La natura non sa grado , ¢ non ha obbligo «/’
pCéh arte, non eflendo flato opera dell’ arte , che egli giuochi, ma effetto della natura,
éiy che" ha prodotto con quefto vizio di giuocare. Dan. Pur. C. ro. diffe ;
eye Na la natura gli haverebbe a feorno,
wht VN genio, Vedi fopra C, 1, flan, 31.

“e PRIMA che gli fulfe legato il belico , Subito ch’ egli ulci del ventre della madre:

j,i Bellico’, Diciamo quella parte del corpo, d’ onde é prefo i) noftro primo alimen-

coyitg £0 nel ventre della madre ; 1a qual eae nel venire al mondo é¢ legata dalle nutri-

jumt@ ci. B cid ferua per dichiarazione del prefente detto .

fu Goa SABBO , Mamma, Pappo , e Poppe . Sono delle prime parole , che fi profferi-

ceil = {cono dai bambini , come s'é detto fopra in quefto C. ftan.s. Ma quefto Perlone
prima /pade , bafton , denari , ¢ coppe , che fono li quattro fegni differenti

|
— Srasaoe arte da ginocare , che fi appellano femi, come vedremo fotto C. 8.
. flan. 6, E qui 4 sn fa dire per moftrare,che prima d’ ogni altra cofa quefto Per-
gat Jone ona il giuoco, ¢ che venne fuora con cotefto a eee Ms giuocare.
a ZA XIII, A XIV.
jest Ma Toe voi fappiate il perfonaggio , E' {wo amico , ed ¢ pur feco adeffo
a ofa Saluo Rofata un buom della fua tacca,

i cib'racconta ,é il Franco Vicerofa ,
pi i Cavaliero , del gual non é il | pit, Saggio; Pero che anch ei sabbeverain Permefo,
Scrittor fubblime in ver/oyquate in profa ; E Pittor paffa chiunque tele imbiacca ;
mt Dipinge , ne pus farfi da vantaggio Tratra d ogni ers at ex profeffo ,
, e in qualfivoglia cofa: E in paleo fa fi ben Coviel Patacca ,
cg Vince mel Canto i mufici pit rari , Che fempre ch'ei fi muove,och'ei favela
E nel portare ecchiali non ha pari. Fa proprio seangherarti le mafcella,
‘ STANZA ¥ Vv.

ap 2
4 Hor percht Pranco, ed egli ogni maniera La dove minchionando un po la fiera
: ‘Proceuran fempre ai piacere altrui, Mt Franco diffe lor; Queffoé coli
iA Di Pertone dan conto ye, don’ egli era, Ch in xucca non ha punto,anziragionaft
Di conferua n’ andar con gli altri dui , Diappiccargli alta teffa un'appigionafi .
Accid che fi fappia chi ¢ colui , che da tal notizia di Perlone, dice ; i
haveva nome Franco Vicerofa , cioé Francefco Rovai Cavaliere dotto , Poeta_, .
7 Mutfico , Pittore , ¢ veramente dotato di quelle buone qualita , ¢ virth , che dice
' jl Poeta ,e che flanno beniffimo in {uo pari, come teftificano alcune poche fue.
_ Poefie ftapate dopo Ia di Ini morte,che non {ono anche le migliori,che egli facefie
| Dice che nel portare occhiali non ha pari, perché haveva nafo aquilino aflai grande .
Con eflo & Saiwo Rofata , cioe Saluador Rofa huomo anch’ egli dotto , e Pittores
eccellente , il cui valore ¢ notiffimo, moftrandolo a baftanza le di lui ftimatiffime
Opere; ¢ aoe valeffe nella Poefia fi conofcerebbe da alcune Satire da lui fat-
te, Je quali ra vedere una volta alla tampa . Quefto era amiciffimo dell’Au-
tore , € fu caufa , che egli tirafle avanti la prefente Opera , perfuadendoli , che»
era ce godere I’ aggradimento univerfale , ¢ gli dette anche notizia de lo Cunto
degli Cunti pubblicato in quei tempi . _ Saluator Rofa recitava da Napo-
B 2 Jetuno

!
che egli
PAS Br FRE, FM S

i

Quel! aggiunta di fera ¢ folita metteruifi, ma non so gt a

 

196 MALMANTILE |

Ietano in commedia mirabilmente , ¢ firfaceva.chiamare
fto Franco Vicerofa, e Saluo Rofata infegnarono dunq
defi chi , e dove cra Perlone. a7,
AVOMO della fua tacca, Huomo fimile.a lui. Vniformi di ge
ca detta anche raglia ¢ un pezzo di legnetto feflo in due parti
quale ferue per libro di conti a coloro , che non {anno leggere , ini
Vnifcono dette due parti di legnetto , € nella parte pid {pianata f
tacche , 0 fegni col coltello , 1 - fegni denotano i] numero delle ¢
credenza , 0 dei danari, che fidevono,, o de i lavori fatti , pezzo:
eflo legno rimane appreflo al creditore , ¢ I’ altro appreffo al debitore:
fi voglion dar nuoyi danari , o fegnare nuovi lavori , s’ unifcone detti
vi fi fanno i fegni che occorrono ; E volendo aggiuftare i conti fi
gni, ¢ fi vede la quantita del debito , 0 credito: ne vi pud nalcere i
ché fe in una delle dette parti di legnetto fara fatto un fegno di pid 5 ‘a
far nell’ altra , perch¢ non rifcontrera , fe il debitore , ¢ creditore non!
dono {cambievolmente detti pezzetti. Era in ufo quefta maniera di
anco appreflo ai Latini , che tal Jegnetto , che noi appelliamo Tagia yo
la dicevano tefera;: Swam uterque teferam habet ; ratio conftat . Ha}
un’ altra taglia , che chiamavano Tefera ho/pitalis, la quale ferui
re gli amici , e corrifpondenti di diverfi pacfi , ferbando ciafcuno
goetto ; i] quale fi lafciava anche a gli Eredi; E quando andava
dell’ altro portava Ja parte del legnetto;e unendolo & dava:a conole
te; ¢ pero detti legnetti crano cuftoditi diligentemente . Quefto
Plauto in Pen, Ezo fum ipfus, quem tu quaris . P, hem quid ego audio?
gnatum eje. PB. Sé ita ef , Telferam me conferre bofpitalem Fi vis ceca a
tli, Donde havevano poi , T¢/seram frangere ho/pitalem, che fignifica
hofpity . Dal che fi cava , che homo eix/dem te/sere , fia lo fteflo , che!
medefima taglia, che fignifica delli ftetfi genj , ¢ corrifpondente . Diguihi>
biamo il verbo attaccare , che vuol dire Vnire due materiali infieme ‘ire
bo atagliare , che yuol dire Effer uniti di genio . Ricord, Mal. Sror-Fionapiy
dice: Lucca, Pifoia , e Volterra feciono taglia co’ Fiorentini ,.¢$' i
garono, © fecerolega; E fi trova ne gli antichi noftti Storici &
lega. Om
PASSA chiungue tele imbiacca . Supera ogni Pittore . I co ee
FA sgangherar le mafeella, Fa ridere {cegolatamente , che &,quel Rife quae
che dicemmo fopra C, 3. ftan. 66. alla voce Pimmei . E veramente (
ne gli anni fuoi pid giovenili , che dimord in Firenze recitaya.(
detto ) quefta parte di Napoletano cosi bene , che fi pud. dire, che egli!
Maeftro in far quefto Perfonaggio , wukseuee ;
eANDAR di conferva, Andare infieme.. Detto Marinarelco,
fignificato . . wise oe
BU MINCEUON.ANDO (4 fora. E’ il latino derideo, E tanto yale ilvebom®
chionare , che CO...... Che non fi dice per eflere {porco , ed-ufato ah Wt

         
    
   
  
      
     
 
 
 
   
 
 
    
 

  

Se

   

to fuona il folo verbo mixchionare , {¢ non che

pes

 

  
    

 
 

QVARTO CANTARE: 197
era, efler,detto da coloro, che non avendo voglia di comprare paffeggiano per
ie fete J del prezzo di quefta , 0 di quella cofa, ¢ non offerendo tea

_ te, 0 pochiffimo ; ¢ flanno a vedere , ¢ offeruare chi compra. E venuto poi a»

ee tines affolutamente, ¢ fi dice ancora Adinchionare la Matter. .
edi fonto.C, 7, flan. 15. EB pur qui ancora fenza I’ aggiunta di A¢arrea {uonas

i | £W.xxcea non ha punto ; cio’ punto di fale , €s* intende: Non ha ceruello in te-
nie) © fla, Vedi fopra C, 1, Man. 53. 1] Mauro in lode della Cacia dice :
vVaetss Ed io, che fono un buom materiale ,

  

| pe Cencande cit ben mafirerci cl io false

foots Da dovero una Zucca fenza fale.
Catullo di Quinzia diffe : 2

to.s mica falis,

= 0, $93 Wudla in tam magno off ci
odpiil ATT ACC ARGLI alla tefta un’ appigionafi. Efiendo ia fua tefta vota;per mo~
firare , che ella fi pud afficare fi difcorre a’ appiccargli -appigionafi, che cosi chia-
mo quella cartelia,in cui fla {critco a lettere grandi APPIGIONASI , ¢ s' ap-

miamo que!
icca fopr’ alle porte delle cafe difabitate , affin che fi conofca , che quella é cala

mani
aia ba affittarfi , 0 appigionarfi , appunto come dice , che era la tefta di Perlone,che
feaingt per effer yota di ceruello , era in grado da poterfi affitcare , 0 appigionare . Iny
lawl alcuni i d’ Italia conferuano |’ ufo antico , {crivendo in L. Ef locanda,
ania _ STANZA XVI. STANZA XVIL
cl Spiscqued (ito male ad ambi tanto tanto, Se forfe dice; tu fei frato offefo ,
y , E mentre @ piange,che fi getta via, Che fai tu della {pada il mio piloro?
ch pt 4 pietofa Eravan pianfe al uo pianto e-4 che tenere al fiance quefto pefa
as! Verbigrazia per fargii compagnia ; Per ftartene a mangiante come un boto?
ai Poi tutto liero poftefecti accanto S' al corpo alcun dolor t° have/se poi
Ler cavalo di quella frencfia , Gli é qua chi vende I olia dello Scoro ;
oo Di quelle firida, ¢ pianto si dirotto, Set’ bai bifogno a! oro io ti fo fede ,
Che quaifivegta Banca te lo crede,

,
gt. Che fa per nulla il bretolon mal corto.
ca _ A coftoro difpiacque molto il male di Perlone , ed Eravano dopo haver com~-
' planta quefla (ua ‘difgrazia , fi mefie a confolarlo , ¢ ad efaminarlo ftrettamentes
# per fapere la cagione di si gran fuo pianto.
Ji BLETOLONE mat corto. Huomo {ciocco infipido , fuenevole , appunto come é
Ja bietola : Marzial. 13. Vt /apiunt fatus fabrorum prandia beta, voce Sie-
tola , che viene dal Latino bera , che vuol dire una fpecie d’ erbaggio , tanto nel
 noftro idioma, quanto nel Greco , ¢ nel Latino ferue ancora per efprimere un’
# —huomo {eiocco,ed infipido-, Laerzio nelle vita di Diogene Cinico dice cost; Cir.
| cumfbantibus fe adolefcentibus eff dicentibus: Caveamus ne mordeat nos: Bono inguit
| effete anima filioli 5 carnis enim betis non vefcitur. Plin, lib. 20. cap. 22. moftra, che
i mariti volendo dire villania alle mogli dicevano loro b/irea, raccogliendolo dal-
Je commedie di Menandro ; ¢ fi legge in quelle di Piauto. , intendendo una cofas
f{ciocea , ¢ che non é buona a nulla; E come noi da bieto/a caviamo il verbo svie-
tolare, che yuol dire Scioccamente piangere .( Vedi fotto C, 7. ftan. 93.) ¢ imbie-
solire , che vuol dire Commoverfi, o effeminarfi . ( Vedi fotto C. 9. flan. 57. ) cost
gli antichi hayevano berizare , che ha lo ftcflo , o poco difference figniticato .
A Bie-

 
 

 

   
  
   
     
 
 

198 MALMANTILE \— i
Bierolone dunque fuona lo fteffo , che Scimunito ; ma con I’ aggiu
vuol dire Scimunitiffimo , perché 1a bietola cotta poco 5
della cruda
en ; oe colui, = governa la Nave: dagli an
to Pedorto forfe dal L. pedes prefo per remi , come apprefio Planto
© per funi da mie eolar efecto « Ma quefta voce Piloto ci
mere un’ huomo da poco , ae » irreffoluto, ¢ flemmatico;ed in
é prefo nel prefente luogo, Vien forle in tal cafo dal Lar. pi n
mo , che per havere i piedi troppo piatti, ¢ contraffatti cammina male.
to C. 6, flan.
4 CHE portare, A che fine portare? Che occorre che tu porti?
facis? Ad quid venifti ? nel Greco dice eph’ boo , cio per P appunto
ST ARSENE 4 man ginite come un boro, Boti chiamiamo quei F.
tue , che fi mettono attorno all’ Immagini miracolofe per 8
ricevute,e perd fi dourebbe dir Yori, ma per ifcambiamento di lettera fi

Berni in biafimo d’ un’ huomo brutto .
Fuege da’ ceraioli
Accid che non lo vendan per un boto ; ce
Che anticamente detti Fantocci fi facevano di cera, ¢ per lo pits
giunte in atto d*orare , ¢ per quefto dice /rarfenea man giunte come
s' intende d’ uno, che non fappia , 0 non voglia operare , em
lavorare ; ¢ vnol’ inferire , Che fai tu delle mani , edella {pada , che
doperi a vendicarti , fe v ¢ ftata fatta ingiuria ? Mons, della Cala
boto per modo di dirlo fempre . i
LO Scoro, \ntende di quel Ciarlatano , che vendeva Lattovarj 5
a veleni detto lo Scoto.
TE ls crede, Scherza con I’ equivoco , dicendo ogni banca te lo
banca ti crede , che tu +habbia bifogno dell’ oro , ¢ pare che voglia
banca ti fidera , o preftera l oro.
STANZA XVIIL
Dopo Eravano poi neffun fu muto ,
CP ogaun gli volle fare il uo difcorfo
Offerendo di dargli ancora Aiuto,
Mentre dicefse quanto gli era occorfo ;
STANZA XIx.

Won v' é rimedio amici alla mia forte ; =
” I. tutto é vano y gid che 1a fentenza MU foldato ciat nel ciabattino
E' ftabilita in Ctel.della mia morte y Peri:che miscomuien. ,

Che vnol ch'io muoiase muvia in mia presiza,

Gia l alma ffivalata in fu le porte
Omai dimoftra a? effer di partenza.
Gid con il corpo tutti i fentimenti

Le cirimanie fanne 5 €4 complimen

 

   

Ed ¢ che foto fon come wn
Eldinnanzi a Minos,cagh
Rapprefentar mi devo co,

 

   
      
     
         
        
    
  
      
 

“di noi fi piglia in diverfi

QVARTO CANTARE. 199
STANZA XXL STANZA XKXIL
Wa ecco omai L bora fatale ¢ ginnta , Hormai di vita for nfcito, e pure
"Chto lafci il mio terreftre cordovano; Lon trove al mio penar quicte,o céfarto,
Gid gid la morte corre che par’ unta O Cielo Mondo, 0 Giove, 0 creature
‘fo dime con la gran falce in mano; Dite, s’ udiffe mai cosi gran tor to ?
inge ella il ferro nel bel fendi punta, Se Morte ¢ fin di tutte le feiagure ,
3 iomancar mifento.a mano amano: Come alupar mi fento ancor che morte?
Pero lo {pirto , ¢ il corpo in un fardello Ecome , dove ognuno efce di guai ,
Tivo fuor della vita, e vo all’ avello, Mi 8 agugxa il mulino pile che mai?

_ Anche gli altri dopo Eravano gli offerfero il loro aiuto, ed egli fingendofi paz-
zo\comincia a dire una mano di (cioccherie,e moftrando di creder d’ efler morto,
fi maraviglia , che mors, qua omnia foluit non gli habbia levato I’ appetito di ci-

HAVERE feorfo col ceruello, Effer’ impazzato. Haver dato la volta al ceruel:
Jo. Metafora tolta dail’ orivolo a ruote , che fi dice gualto , quando le ractes
{correndo efcono del Jor moto regolato .
APFFISS AR gli occhi in uno . Guardare fenza punto movere gli occhi ; atto da
pazzo di quella {pecie , che domandano Maniaci .
ALLA mia forte, Di quel che m’ ha da fuccedere . Quefta voce forte appreflo
Woidcari » come feguiva anche appreffo a i Latini , da i
quali fidiceva fors ogni avvenimento di Fortuna . Cic.lib.2.de Divinatione . Quid
enim [ors eft? idem propemodum , quod micare , quod talos iacere , quod tefseras, ed in
fenfo & prefo nel prefente luogo. Si dice tirar le forti,per intender quel

fuper veftem meam miferunt fortes dell’ Buangelifta .

La pigliavano per carica , o incumbenza , fecondo Livio: Si id gravaretur fa-
sere , quod non fue fortis id negotium e/set .

La pigliavano per ftirpe , fecondo Ovid. 6. faft.

a; Si genus afpicitur , Saturnum prima parentem
. Feci , Saturni fors ego prima fui

La dicevano anche il capitale , ¢ quello che noi pure diciamo forte principale;
Plaut. Mott, Quatuor quadraginta illi debentur mina , Et fors,& fenus DA,tantum eff.

Altre volte pigliavano /ors , pro ixdicio fecondo Verg. 6. Aneid,

Nee vero be fine forte data , fine indice fedes ,

Perché ( fecondo Seruio ) non s’ udivano le caule wif per fortem ordinate, nam. ,
quo tempore cause agebantur conueniebant omnes , 7 ex forte dierum ordinem accipie~
bant , quo poft trigefimum diem caufas fuas exequerentur ,

Dicevano forte gli Oracoli , 0 rifpofte , 0 le polizze fopra alle quali fi (criveva-
no le rifpofte. Val. lib. 1. Cuins rei exploranda gratia legati ad Delphicnm oraculit,
vetulerunt : scent fortibus, ut aquam eins lacus emifsam per agros diffunderent , Virg.
in quefto fenlo diffe : Lycie fortes. Apprefio noi aucora ( come ho accennato )
Jorte fi piglia per fortuna , 0 deftino , per condizione , flato , 0 effenza . B dicia-
Mo toccare in forte , che fignifica ottenere la benefiziata , quando s’ eftraggono
Ie polizze , che ¢ quel mittere fortes ; ¢ {e bene in fignificato di fortuna vogliono
alcuni , che fi debba dire forte , ed in fignificato di qualita , 0 condizione fore, ,
hoggi ( almeno nel parlar familiare , ¢ Civile ) non trovo , ches’ ufi tal diftinzio-

ne,

ib aa
200 MALMANTILE —

ne, ma fento ufare alcune volte ’ una per I’ altra indifferentemente} ) ~
ClLABATTINO . Vno che raccomoda {carpe rotte ; iabatta
re Scarpa vecchia , ¢ {carpa all’ Appoftolica , che (ono quelle, che:
Cappuccini. In molti luoghi de’ contorni Fiorentini chiamano C
ra quelli , che fanno-di nuovo ; che noi chiamiamo Calzolai, in Up
fimilusente zapaceros ; € quefto nome di Ciabatraviene
cioé (carpa ferrata con chiodi ; quali (on quelle che ufano i contadi
ciatori. i
TIRAR le quoia , Havendg detto , che di foldate doveva diventar
la ragione perché ; ed ¢ quefta , che gli conuien tirar le quoia, come fanne
battini , ¢ 1 Calzolai, che tirano i quoi per condurglia quella mifura, ¢ i
no, delle quali quoia dice , che fi dee feruire per rimcalzare it pino., cio’ fark
{carpe al pino. Nota che Jo {cherzo dell' equivoco , nafce da rirar le quoia,
vuol dir Morire , ¢ rincalzar con efse il pina , che vuol dire Parfi forerrare
del pino , ¢ cosi alzandogli Ja terra attorno rincalzarlo., che quefto vu
ealzare un’ albero . Ofierua ancora , che facendolo parlar da pazzo’
coloro credano , che egli habbia concepito nel cerucllo quefto fpro 1
a fare le {carpe a i pint ; perché quando un Calzolaio dice ; Io calzoil ti
tende Io gli fo le (carpe. Plut. in Dem, E ca/zandoft dices, Il Gr, erepidas fi
SOTTO , fon come un cammino . Sono {chifo , ed ho le carni fudice ,
cammino , dove fi fa il fuoco, Comparazione ufatifiima particolarn

donne. W s
MINOS , ¢ gli altri Gindici , 1 Giudici dell’ Inferno fecondo le
tichi Poeti , ¢ della Gentilita fono tre , cioé Minos figliuolo di Gi 3
a, che fu Re di Candia, Eaco , che fu figliuolo di Giove , ed’ }
¢ d’un Ifola gia detra Enopia , la quale egli poi dalla madre chiamo Bgi
Radamanto , che fu figliuolo di Giove , ¢ d’ Europa , che fu Re di Li
fti Re,perché furono feveri amatori della giuftizia,dicono i detti Poeti
tone gli cleggefle per Giudici dell’ Inferno , affinché efaminaffero } anime
aflegnaffero loro le pene , che meritavano , ¢ da quello, che di loro ferive
2a. 6. fi pud comprender il lor precilo , ¢ particolar ofizio 5 che di
Quafitor Adinos urnam movet , ille filentum
Concilinmque vocat , vitas , & crimina difcit ,
E di Radamento dice ;
Gnofius hae Rhadamanthus habet duriffima Regna ,
Caftigatque, anditque dolos , fubigitque fateri , §
D’ Eaco parla Ovidio cosi ;

 
  
 
  

 
 
   

 
    
 

   

  
    
   
  
   
   
 
 
   
   
 

    
   
 
   

ytd

 
    

 

Tualque ae

Eacus in penas ingeniofus erit , td
E conchiude i} Pocta, che uno di quefti Giudici efamini, 1" altro giudichidl
zo mandi ad efecuzione . Se ben Dante nel 5. dell’ Inferno.dice:
Stavvi Minofse orribilmente , e ringhia , ss
Efamina le colpe ned entrata , aa a
Giadica , e manda fecondo ch’ avvinghia . oi

CORDOY-ANO . Specie di quoio da fare {carpe , la concia del uae a for

 

 
 

QVARTO CANTARE. zor

inuentata in Cordova,e percid tali quoi chiamanfi riamente cordovani,e fon
poll di Caftroni podta "re: qui ioteade, pelle humana’, ¢ dicendo
tafei ly cordovane intende io muOia , come intendon quelli, che dicono
Terrefire falma;T errena /poglia,e fimili,Cunto de li Cun, Peffoe concio per cordonano.
CORRE che xnta., Corre velocemente ; comparazione dalle carrucole , ©
puleges » 0 altre fimili , le quali quando fono unte con olio , fapone ; 'o altro ,

nate ‘ a os:
PALCE . Strumento,col quale fi fega il fieno ; ¢ col quale fpeffo fi vede dipin-
ta lamorte com efla in mano . anne A>
GVAl , Travagli , fuenture , {ciagure , afflizioni . Vedi fopra'C. 1, flan. 28.
(ALLER ARE, iiceor gran fame’, perché dicono, che i! lupo fempre habbia
gran fame ; quindi il volgo chiama Male della Lupa quello di coloro , che fem-
pre mangerebbono , perche da joro vien preftiftimo fmaltito il cibo con pochiffi-
mo nutrimento , ed ¢ quelia infermita , che i Medici chiamano Fame caninaa .
Vedi forto C. 5. ftan, 61. E da quefto male chiamato della Lupa diciamé allnpare
uno ‘che habbia gran fame -
\AAGVZZARL it mulino, Par-venire s0crefcere V appetito: perche aguzzare
da-macine del mulino yuo! dire Metterla in tagiio in maniera , che fi renda pit
dngorda.. Vedi fatto C. 7. tan. 31.

STANZA XXIII.
Va'a dir che qua fi trovi pane, o vino
O altro da infegnar ballare al mento ;
1 Se nonifi fain cenaidi Salusno ,
» Qaaato a iar non te afeenamito,
oO iepecdineen 70 i ¥
yuihavete a ireal monumento ,
Vor l intendete , che nel catalerto
Com voi portate il pane,ed il fiafchetto,
oa STANZA XXIV,
¢ Let pet old dal cimitero,
S' it Ciel danari's e fanita vi dia
Empiete il buxzo aun morto foreftieroy
O infegnateli almena un ofteria ;
Se ben voi fare qui fempre dinero ,
Perche di carne havete carcftia:
E’ tale appetite che mi {canna
Chiun Diavol corto acor mi parra mina,
STANZA XXV.
Se ben non c ¢ da far cantare un cieco
| Diguefta [pada alOfte foun prefente,
C'ad ogni mo, da pai cht ella fia meco ,
Mai bate colpo, o volle far niente ;
Per una xuppa dollaancor di greco,
Mache gracchioia? Qui neffun mi fente.
Che fo? s* i morti fon di pietd privi

STANZA XXVI.

Qui racquese per fuggir la via fi prefe
Facendo fempre il Nanni,ed il corrive,
Perch'eglie un di quei marti alla Sanefe,
C' han fempre mefcolato del cattivo;
Per haver campo a feorrer il pacfe
We fere pol di quelle con P ulivo
Miffrando egn’bor pitt dar nelle girelle,
E tutto fece per faluar la pelle.

STANZA XXVIL

Perch uno, ch’ it foldato a far #  meffo,
Mentré dal canipo fiigde, e fi rravia,
Sendo trovato, vien fenza proceffo
Caldo catdo mantdato in piccardia ;
Perd £ ei parte non wiiol far lo Refs ,
Ma che lo feufi , ¢ falui la parria ,
Onde minchion minchion factdo itmatto,

Se ne [cantona’, che non par fuo fatto,
STANZA XXVITL ©

4M Fendefiafcappare anch’ei fu leffo

Con gli altri tre correndo a rompicollo ,

Volendo rificar prima un caprefto ,

E maorir con ta fromaco farollo , A

Che refar quiviiamenarfi Pa...

Ed allungare a quella fozgia il cello ;

Li danno certa é fempre da fuggire ,

Meglofara ch jotorniaftartraivivi, Co S'egli avvien peggio puinonc’e che dire,
 

202 MALMANTILE

Petlone feguitando a dire i per effer tenuto matto fi 3 e per fale
var la vita continovo a fare ie iioctiee ie , fapendo , che un to aaa
pa dal campo , ¢ fi parte fenza licenza ¢ reo di morte , ed il Fendefi 5 ¢ gli altri
{camparono anch’ effi. i Seek

Vda dir che qua fi trovi, EB! yanita il credere , 0 dire che qua fi trovi;s' ingan-

na chi crede che qua fi trovi . “vy

INSEGNAR baliare al mento. Mangiare. E’ lo fleffo che Dar il portante a’
deati detto fopra in quefto C, flan, 6. q

FAR la cena di Saluino, Andare a letto fenza cena ; che Ja cena di Saluino era
Pifciare , ¢ andare a letto. *

O SER H/ac,o Abramo , 0 Iecodino ; Lptende tutti gli Ebrei , ¢ feguitandd ¥ opi-
nione del volgo , il quale crede , che quando gii Ebrei feppellifcono i loro morti
metiano lore appreiio del pane , ¢ del vino dice ; Voi / intendete che morendo por
rate con voi il pane,e il vino , poiché nel mondo di qua non fi trova ne da mangia-
re , ne da bere. Wig

CAT ALETTO , Quella barella , entro alla quale fi pottano i morti al {epol-
cro,che i Latini dicevano fererrum . Voce compolta di Lertoye Cara prepofia. Gr.

ORBE , old, alo, E fimili; fono voei , ¢ termini ufati per farfi (entire da chie
alquanto lontano ; come fa il Latiao eas , Orbé , ¢ fatto da Ora bene ; Or beat
Latino age vero ; Alo dal Fr, ailons ; andianne .

CIMITERO , Piazza , nella quale fi fanne i {epoleri per li morti, Voce ches
viene dal Greco she, che {uona dormire y rip » Onde jon, lo
fteflo , che Dormentorio . Quindi i Cretenfi chiamavano Cimeterid una calas
pubblica , la quale ferniva per alloggiare i pellegrini. Vedi forto C, 7. flan, 27.

S°LL Ciel danari, ¢ fanita vi dia, Dice quefto {propofito per acerelcere in
ro la credenza , che egli fia matto , fapendo bene che i morti non hanno bifogno
di fanita , ne fi curano di denari. gt 04

3VZZO. \itendi i] ventre dell’ huomo , da bufto che s' intende tutta quellas
parte del corpo humano , che é dal collo al pettignone , fenza Je braccia . -

FAR di nero, Mangiar di magro. I venerdi , abati , Quarefima , ed aitre vi-
gilie fi chiamano giorni neri , quaG giorni di lutto deftinati alla penitenza, il
Poeta {cherzando con I’ equivoco del acro , col quale é folito farfi !apparato a’
morti , par che voglia dire non mangiate mai carne , perché foggiunge di carne
bavere carefiia , ¢ par che intenda non havete carne da mangiare , ¢ yuol dires
bon havere carne in {a l’ ofa , perché i morti in breve tempo reftano puri (chele-
tri fenza carne. a

e4PPETITO che mi fcanna, Fame cosi grande , che mi fa morire , che mi fa
perder ja canna della gola ; che {cannare uno , vuol dit Tagliarli la canna della
gola. Cunto de li Cunti Giorn, 1.\Se la necefjita non la foannava, 6

MI parr manna, Mi parra buoniffima ; come paruc, ¢ fua gli Ebrei la Man-

na , che mando loro Dio nel deferto , che ricevendola elclamayano Adanu cio. -

‘Che é quefto ? onde forti il nome.

WON ho da far cantare un cieco, Non ho ne meno un quattrino da darlo a un —

cieco,perché canti un’ Orazione . :
LA ogni me, Per: ogni modo. E’ termine affai ufato in Birenze in diversi fen,
per:

 
      
    
 

=

EF ar eae aoe

CaS pc FF ES.

LL ER FEE

 
 

 

 

QVARTO CANTARE: 203

ereeeaeens come nel prefente luogo , Haglio dar via la fpad. ,
4d ogni modo non baste mai colpo , cine perche io non la fimo per non hayer
i rato, O fignifica ne ceffita di fare, 0 non fare una cola per efempio,
a » che ad ogni modo s' hada morire . Significa contentarfi di
‘ sche uno ha confeguito : fo. bo guadagnato poco , ma ad ogni modo io mi
contento . Significa Oftinazione . So hate anaes po muocere 5 ma La voglio
fare ad ogni modo. Vedi fopra Can, tr. ftan, 3. il termine ; /vo danno , che pat che
habbia correlazione al termine , A ogni modo, V.gr. Se ‘0 bo perduta la tal

‘ anno ; ad ogni modo io non mene fervivo, E quel mo per modo &
Ta figura da noi molto ufata come vedremo altrove .

tAl batsé colpo . Diciamo + il tale non batte mai colpe per intendere , il tales
ora mai , ¢ qui intende , che Ja {pada di Perlone nelle fue mani non lavo-

  
   

 
    

_ 4¥PP A. Pane intinto nel vino, 0 in altro liquore, Forfe meglio Suppa, Fran-
0 Sacc, Nov. 86. E fatea la fuppa con le fpexie,fubite porta in tavola il ventre , ¢ la,
Lippe . Stimo che venga dal Tedefco /uppen , che vuol dir Brodo di carne,o ¢’al-
tro , che fi quoca leflo . In quefto fen/o una forta di mineftra chiamiamo xuppa,
Lombarda., Vedi fopra C.2. flan, 7. Ma I’ ufo ha introdotto i) dir corrottamen-
tezuppa ,¢da molti ineuppa ; come 2ulfa ,€¥ézz0 , ¢ zinfonia in vece di solfa,
C2z0 5 | ia , ¢ fimili .
 GRACCHIARE . Dilcorrer fenza propofito , o profitto . Da Graccio Latino
us . Utale mi chiefe dieci {cudi in prefto, ma 10 lo la(ciai gracchiare. Ve-
10 C, 7. flan. 59. ¢ C. 8. ftan. 65. ‘
"il nanni , ed il il corrive, Finger& corrivo , goffo , femplice , baséo .
MATT alla Sanefe. Si dice Sane/i Matti, ma in effetto fon pil fagaci degli al-
tri ene Matti alla Sanefeye’ han fempre mefcolato del cattivo ; cioe dell’ aflu-
fagace , ed ingegnofo . ‘
fece di quelle con I ulivo, Fece delle (cioccherie grandifime . In alcune fo-
a » fuole la generofa pieta del Serenifs, G. Duca liberare dalle carceri aleu-
icon pagare il loro debito , o parte di eflo;¢ guchti tali vanno procef-
tea render grazie a Dio al Tempio della Santifs. Annonziata , o di S.
10: | 5 ¢ quelli che hanno pagato tutto il debito, ¢ (ono affatto liberi por-
tano in mano un ramo d’ olivo a diftinzione di quelli , che per non haver pagato

 
 
    

   
   

tutto il debito , ma parte di effo devono tornare in carcere , i quali nou hanno
Volivo in mano , ma fon legati. Da quetto ramo d’ ulivo, che ia tal congiuntu-

rm to inter , credo che fia nato il dettato ; La tal cofa ¢ com I'uli-
% , che fignifica cofa grande nello fteflo modo , che i Latini differo pa/maris , ed
f un’ azione ardita , che diciamo anche marchiana ; da pigliar con le mole ,
€,, Come s’ intende qui , che vuol dire , che quefto fece cole grandi , ed ardite.
DAR nelle girelle, Impazzire . Vedi fopra C. 3. an, 43. , ¢ fortoC.9.Man.10,
MANDATO in Piccardia caldo catdo. Impiccato fubito prefo fenza far pro-
Gelso : Calde aldo fubito , ¢ prima che la cofa fi raffreddi. Piccardias ,

“in ipl ardore criminis , Provincia della Francia, ferue , {cherzando con la.

Amulitudine della parola , per intendere émpiccare . I Latini pure havevano
MN termine coperto per fare ome impiccare 5 che era diteram longam facere ,
oy cz come

 

 
— 2° Rey: Cena are

  
    

204 MALMANTILES / >

come fi vede in Plauto; il che ha data occafione-a molti letterati didilcorrere per

chiarire qual fulle quefta lettera  ¢ Celio Rod. leét, Ant:

chinde , che fufle il T maiuttolo , che dGmite alla forcay che facevano

Noi ancora diciamo': Andare * Lamia chore un Porto in Tofcana’ ‘te ae

ligno , ciod a fune'y ¢legno ; Dar de’ calei wl vento: “Ballar incampo:

C2, fan, 65. Ballar ‘nel Paretaio web Nemi (00 Cy an, yor B tute:

no Bfer-impiccato . saath t

@UINC HIONE ; Da imitichiet detto fopra in guefto a flan. 15.

SE ne feantona , che non par fio fatto. Se neva via e ee ean

quefto per andarfene , E fete quell’ Agere fe di Te ih

CORRER a rompicolo . Cortet velocemente ; € 3 precip iio fenza wonbiderates —

Ja ftrada buona, o cattiva ,
ARRISCHLARE un caprefto, Avventurare a effere impiceato, Corre it

fto il rifchio a’ andare in fu Je forche , che aa di morir di fame. ;

 
 
     
   

Ss oom see oe

ss

   
 
 

   

  
 
 
 
 
  

MENARSIU A... ... Perder il tempo fenza far nulla. Se vuoi intender
ne quefto.detto , leagi difeor(o d’ Anibal Caro in tin di Ser’ ee
STANZA XXIX. TANZA 1!
Lafciam coftoro , ¢ vadan pure avanti Danae ¢ i guerriero, € via’
oe il vitto Li per quel contornos Cavalcando ne va con fefa, &

Che fe fame glicaccia ye "fon poi fanté Ognor tenendo tt chirarrino in
Da batrerfi ben ben Seco in un forno: Perche il viaggio non ¢1 a
Perched! un era guerrier conit ch’ io cati E' bravo sh, ma‘poi buon
Ate2z0 impaniata, perch'eglihad'iterno E farebbe feruizioinfino
Vasa donna firaniera in veffe. brane, Venga chi vuol , a tute
Ches' afjligge , ¢ fi duol della forenna Se bene fuffeit Carat FB svecchit
STANZA XXXL ’ a
Poiché bella ¢ colei che fi difpera Percio convifo ar , sana cera
Sempre piangendo fens.’ alcun ritegno y Par unEbreoe® beth ee
E vanne,come to diffi, in cioppa mera E as quanto P affligge,e [a trave
Ler dimoftrar di fua mefpiziail fegno, lagrille il Campion quivh
11 Poeta lafeia il difcorfo di quegli afamati,‘¢ fi mette 4 narrare Ja’ fav
vettita di Pfiche; la quale chiede aiuto 2'Calagrillo , che ¢ Carlo Galli
di Cavalli’ 5 gli racconta’i fuoi travagli . -
SON fanti, $? intende fon huomini c’ hanno cuore , € ‘pirito da ‘fare quella
tal cofa; ¢ da pigliare ogni rifolazione .
Dat batterfi ben ben feco in'wn forno . Da combattet con la'fame’ <>
un. forno pien di pane, © mangiandofelo » vincerla’ , ¢ facia
MEZZO impaniato., Ubrogtiato ; Intrigato: Trastato da | alin

   

 

   
      
 
  
  
 
 

SPLF2F RESTS Een eee EE

   
 

 

  
 

  

    
    

vendo toccata Ja pania , volano’st , ma con Uifficulta per tim
. Joro la*pania y.che hanno fopra alle penne.’ +e
BVON papricciane . Huomo dolee , groffotano ,"huomo alla bubint Pisfriccia:
wo {pecie di Paftinaca . 1 dettorantico ¢ Buon pafticcione , cioe eee:
Placidus tanquam aqua filens.
BATT I feravecchio, Molti vogliono , che fi dica il Bratti ferra’
Je fa aa haomo facultofo , ma di cattiva fama : Coftui lafcid poi tutto

  

2 meric 3 sana
 

SSSRew

aS S58 oe

es tS SS

a
3

ee &

QVARTOVCANTARE. 20s.

C di fecolariintitolata in.S, Giofeppe, perché delle renui-
te fine’; come fegue fino al. di d’ hoggi ;:ma.a me pare,
io hia dire il Bares ; ¢ il Batti , ciod i Bactilani s quando noo poilo-
no pii rare "altfa arteyy fi mettono.a fare. il rivendirore di
cenci ,¢ ferri vecchi » e-dall’ andar gridando per ia Gitta Chi ha ferri vec)’, han-
ito il nome ! divFerraveccine ~ E perohé quefte {ono vilifime perfone aed
lle quali fi ha poco riguardo ; + quando vogliamo efprimere , che uno fia di maa-

  
    
   
 

fueta,, ed umil natura , ¢ indifferente.con tutti , fogliamo qualificarlo con quefto
termine | Salute’ yofarebbe fernixio, anche al Batti ferravecchio, Che fe dicette il
calzerebbe tanto bene; pera finalmente il Braeei, fu perfona di qual-
che siguardo’, © Civilta... Zmbratra {oprannome trovali ne) Bocc,
RESACELE E’ nota la favola di Ai pedecleriita maravigliofamente da Apulcio,
il Poeta incalftra in quefta fua Opera, ¢ P immafchera affai aggiultaca-

(Pio aren. Vito afpro sche denota dolore , o altra paffione travagliofa Lar,

TMITT Aces, era, Haver brutta , 0 cattiva:cera vuol dire Faccia , che dal fuo
indi¢hi poca fanita , 0 grave difgufto., che travagliando I animo,
il corpo 5 E bratra cera yuo) dit ancora Fifonomia cattiva .

un’ Ebreo c' habia perduto il peguo. Quand’ uno per qualche difgufto mo.

i¢a ci feruiamo di quefio detto , perché o fia vero, o fia no-

fra opinione’y rarisiimi fono gli Ebrei , che habbiano faccia allegra ; ma un’ E-

breo che habbia perduto i} pegno aggiunge melanconia .a malenconia , ¢ pero
moftra facia. deformatidima .

SPANZA XXXII STANZA XXXIIL

Fn vg incta ) devi fapere E quando poi io 'bo bell, € trovato ,
wr bel marito , ma perch’ io Martinarra, chre fempre lo Scompiglia,

Die eis era contro al {uo volere Fa si. che pur di nuovo m’é é foappato,
er fet anni n° bo pagato il fio; Ed in mia vece.ali' amor {uo s " appigtia,
aallor per farmela vedere Tal ch'io rimange cacciator seraziato ;
Stigtate meco fer’ ando con Dio Scnoprolalepres.e un’ altropoi lapiglia
a lingo ; ebea volerle ritrovare Ti dico quefto ; perchi hanrei voluto
trier volea da ene e Che tu mi deffi a raccatrarle aiuto .
STANZA XXXII.
a ie pomaree ginra sch il marito Edd ella lo ringraziaye del, Lfeguito
‘ pero non fi feomenti. Di sante fue fatiche ,¢ parimenti
“Ef ndpteed « quelche ba smarrito , ( Fata pit lieta per le fue promeffe )
¢ [ei bifognando,e dieci,e vents. Cosi da capo.a raccontar fi meffe .

Psiche efpone a Calagrillo il fuo bifogno,e lo richiede d’aiuto ; Ei glielo pro-
mictte 5 ed ella fatta alleere per tal promefia, incomincio.a difcortere » Narrando
tutte le-fatiche 7 ¢ difagi meee -da lei in ricercare del Marito .

WV-HO pagato i! fio. .N’ ho pagato la poms 3 ¢ibLat. penas.dare, Fio & voces
Fiorentina antica ,che ant dir fexdo . Gio, Villani lib. 5. cap. 1.Scomumicd Fede-

Afoluette tutti li fuoi Baroni da fio, e (aramento, ec, ma da noi hoggi nor.»
ee eneiene 3 nel qualc anche iusd Dante Purg, C, 10, es

 
206

MALMANTILE |
Di tal fuperbia qui fi paga il fio. >»

Ved os Bey aA

Cl ebleva la carta da Navicare, Bra impomibit ritrovac sgun bag. fone haves,

Ja carta da navicare , ola buffola .

ZL! HO bell'e srovate , L ho gia trovato. Vedi fopra C. 3. fan. 14 a foraa di

guefto addiettivo bedo in quefti termini .

4M’ HA feartato, M’ ha rifiutato. Traslato dal giuoco delle carte, ‘che ae

do una carta , che habbiamo in mano non fa per noi,la buttiamo {
delle carte ; il che fi dice fcartare , vedi foto es i
e4 RACCATT-ARLO , Ciot ritcovarlo , riaverlo , ricuperarlo . Il,
fignificato di raccattare ¢ Ragunare , mettere infieme » :
NON fi szomenti, Non fi perda d’ animo , non fi sbigortilca . Petr. ete vila
E fol della memoria mi sgomento .
Dante nel Purg. C. 14. in tignificato attivo . ‘
Cacciator di quei lupi in fu la riva 2 e ih

     
  

—
. fan, 6. alla voce

Vedi fotto C, 1

 

 

i 65

Del fiero finme , ¢ tutti gh sgomenta , * gh
SMARRIRE . E’ un certo perdere con speranza di ritrovare . Dan. Inf, C, rE
Che la diritta via era {marrita vide

_QVATTRO , fei ,¢ dieci., e venti. Scherza facendo,, che Caria proses

pid di quel ch’ é richiefto., come fanno tutti i bravazaoni,¢ in tanto
a una bella donna non mancano mariti.

STANZA XXXV.
Cupido é la mia cara compagnia,
Riccogarzon,feben la carne ba ignuda,
efnzi non é , t’ ho detto una bugia,
Perch'ei no’ mi vol piie covta,ne cruda,
Ma fenti pure , ¢ nota in cortefia:
Quando la madre [ua ch'era la Druda
Del Fiero Atarte,ideff la Dead amore
Gravida fu di queffo traditore ;
STANZA XXXVL
Perch una srippa havea, che conueniva ,
Che dale cigne bomai le fulle retta,
Cagion ch'in Cipro mai ai cofaufc.vs,
Se non con i braccierijed in Seggetta,
Pur fempre co. gran gente, € comitiva y
Com’ a Regina; com’ elle, 8 afpetta,
i J paesi ha dietro,e gli fraffier dinanzs,
E dagt inlati due filar “2 Lanzi .

Se non ch ¢ miei m é pase
Mio padre ch i i ne lo fcanna y
Con un mio Zio ch’ andava zientey

i Cathy

TANZA

 

 

STANZA XKXXVIL )

Effendo cos: fuori una mattina se
Per {uot negorzi se pi
Vito per cafo nna Vacca Trenina >

E tocca a pena, in terra la
Ond’ ella un alta rammanting >
Perch'unalinguaellhache ragliase fede:
Va, che tw facia, quando ne fia otta
Vn fi Leos ice) in forma a wna betta
ZA XXXVIIL
E ae oh? in-vece d'un bel figiia
Di {uo gufto , edi tutti i Te ARAB
Vn rofpo fece come un pan di miglies
Cc ee fatto flomacare i cant;
Che poi crefciuto , feceft configlo
Di dargli un po di moglie ma i merzant
Non Pela pee
ae ee ne voleffe , ofentir nulla
Sperando tutti tre d' ungere il dente,
E dire: O corpo mio fatti cs >
E riparare ad ie lor co

Me gli oferiro;e fecefi impiafiro.

E un mio fratello anch'e
eect Psiche a Calagrillo Ja dolorofa ftoria, ¢ facendofi dalla nalcita fi

Cupido dice , che nacque in forma di rofpo per Ja maladizionc d’ una veschia»€ i
chy poi cre(ciuto fu a lei dato per marito .

   
  
   
   
    
   
    
 
  
 
   
  

Sis cS ee ee 6 ee

egg?

(FEZ EEFE SHEET

SEH 6

£.
 

 

QVARTO CANTARE: 207
een eas Hida Nealeffo, nea rofto. Non mi vuol pi ia,

 
  
 
 

yl Tr. lib. 2. fan. 42. dice:
) Nom gli volle annafar cradi , ne cotté
‘DI « Innamorata , tanto in bene quanto ia male ; perché fi dice amante,
4 » damo , non fempre in fignificato difonefto .
Da C.12z, Dentro vi nacque L amorofo Drude
RIE j : Delia fede Criftiana il S, Atleta, Parla diS. Domenico .
Se bene nel prefente Inogo s’ intende Meretrice , concubina .
| CIGNE . Sono firifcie di quoi, od’ altra materia adattate a foNenere , ¢ te-
"ere infieme qualfivoglia cofa, dette cigne, da cignere .

 BRACCIERL, Coloro , fopr’ alle braccia de’ quali con una mano s’appoggia- -

no le Dame andando a piedi per la Citta.
_ DAGL inlati , Dalle bande , da i lati . Idiotifmo ulato affai in /ati per lati.
_LANZI, Cosi chiamiamo ii foldati Tedefchi della guardia pedeftre del Seren.
G, Duca. Vedi fopra C. 1, flan. 52.
VACC-A Trensina’, Cosi chiamiamo certte donnicciuole poco honelte , sfac-
- Giate , ed ardite , che non portano rifpetto a veruno ; € credo che fi dica cosi pez
militudine , che hanno con le Vacche di Trento , le quali per effer’ avvezze a
fempre per le campagne del Tirolo , (ono faluatiche ,  fecoci .
— Re IANZINA. E*\o fteflo, che rammanzo detto fopra C.1.ft. 52.,¢ che
mbbuffo nel med. C.t 39. Da alcuno ¢ definita cosi: Riprentione fatta con parole
; Minacceyoli , ¢ ingiu:iole.. Forfe dalle dicerie de’ romanzi .
oy yep ieee ae Ȣfende, Ha una cattiva lingua , che dice ogni forta

|

   
   

 

; ‘male fenza rifpetco , o riguardo alcuno , che lacera  alerui ripurazione .

4 HAVREBBE fatto fPomacare i cani , Cosi {porco , ¢ nefando , che haurebbew

7 provocato il vomito fino a i cani per Ja fua {chifezza . ia queflo feafo i Latini
“pure fiferuivano del verbo fomachari .

s  - | DARGLT un po di moglie. La voce poco ¢ ufata da noi in diverfe maniere ; 0

w  declinabile , che fignifica quantita , come dureg/i un poco di carne ; O indeclinabile
Der % io ; come andare un poco a Roma ; Dategli un po di moglie , ¢ {erue per

- eufafial difcorfo , ¢ non per quantita, potendofi dire andare 4 Roma : Dategli mo-

flit, che tanto efprime {chza la voce poco , 1a quale perd nel prefente luogo non
tipithezza , 0 (come diciamo ) borra ; ma € cosi detto per moftrarne I’ uso ,

. che apprefio di noi ¢ frequentiffimo , ma nel cafo come il prefente é tanto ufato ,

6

*

%

‘

w —‘tillima da noi in quefta , ed in altre voci enunciate fopra C. 1, ttan, 36.

oh | MEZZANT, Senfali. Coloro che fono mediatori a conchiudere ogni forta,
y afare,

AL bifegno ne lo Jeanna. E? poveriffimo ;muore di neceffita ; la voce feannare,

a Sula da noi per e{primere an foverchio defiderio di qualfivogtia cola, fe bene il
a Myo pit proprio é della fame , come s' é veduto fopra in quelto C, flan. 24,

‘ PEZLIENT INTE, Povero, che chicde limofina . Deriva dal Latino perere onde,

h, povere pexriente vuol dir pauper petens cleemofinam ; ed € lo fteflo che povero in can.

__ %4, quafi ignudo come una canna ; altri vogliono , che quello ixcanna fia una fo-

iy! Waparola , ¢ voglia dire imcannatore : Che quando un’ huomo fi metre a incanna-

ee ; re

 

 

‘che non pare fi poffa dire altrimenti. Quel po per poco € Ja figura apocope ufa~ -

 
  
   
   

208

re, &fegno, che ¢ miferabile
4] Varchi Stor. Fior. lib, 124:
grnrzolato , ¢ diventarone poveri
Jogi , per guardar fempre it Cielo ¢
VNGER il dente; Mangiar roba , che unga il dente come carne
fempre pane , come fon neceffitati fare i mendichi ; ¢ vuol dire Far
mangiar un po meglio . Wee OE ied dos {ana
DIRE al corpo: fatti capanna « Haver tanto dam
pregare il Cielo, ia diventare il {uo corpo capace quanto.
riporre il soe ir C.
tanta roba . Viiam quefto termine quando veggiamo. uno avvezzo.a
ramente , e che fi trovi poi a ua banchetto Jautifimd.) 6 obasbe
SI fece limpiafiro , Cioé x accorde , fi conchiufe ii negozia .. s
STANZA XXXX. 9 STANZA XXXXE A
Fu volentier la feritta frabilita , Saggiunfers di tui mill aleve bowxey
To dice fol du lor, che fan penfiero Ata quandoda me poi lo
Di non havere a dimenar le dita y :
Ma ben di diventar lupo ceruicra 5
E,, perché e' fon bugiardi per la vita. y
Dimoltrano a me pai il bianco pel nero
Dicendomi, che m' hanno fatta fpofa Ogni volta con mio maggior dolare.
D! un giovanetto sch! ¢ st-belta cofa. Sentivo darmi ana frvccataalcueres
Psiche continova il racconto , ¢ dice , che finalmente-fu-conchinfo db parenta-
do fra lei, ¢ il Rofpo figliuolo di Venere . Z 1 be
ST ABILIT Ala ferieea, Fermato , ¢ conchiufo il contratto del Matrimonio ,
che appreffo di noi fi'dice La feritta del parentado . oma
NON haverea dimenar le dita.Cio’ haveraviver stza iavorarejatan dau
DWENT-AR lupo Ceraiero , Divorare, mangiar yoracemente ycome fail Lupo |
ceruiero. Plin, J. 8.c.22, de Lapis dice cosi: Sunt in eo genere qui Cornary v0~
cantur , qualens é Gallia Pompei, Atagni arena Spectarum diximus 5 buic ae
fame mandenti fi re{pexit , oblivionem cibi furrepere ainnt digrefumque quarere alind »
£ da tale agonia di-mangiare s' aflomiglia un huomo , che mangi sonaednes,
ad un lupo ceruicro .” 23 ices
BOZZE. Intendi bugie , fandonic , trovati non veri ; finziont » )
Quando non vogliamo credere qualche novita , che ci fia raccontatadiciamo:
fot ho per boxza. Traslato da iPitori, che dicono bozze, € abbensare Gace), 1
prime pennellate , che danno in una tela , ¢ gli Sculzori quei primi colpi , che>”
danno in un marmo , O altro ; i quali additano un non so che del vero
faranno col finirle .. Vedi fotto C. 7. ftan, 5. 3 i
‘MI cafco le braccia, M’ abbandonai ; mi perdei d’ animo ; mi {gome
4 STANZA XXXX1 HGS
Now lo voleva ; pur miv’ arrecai Quando pik
Veduto baxendo i tc

sip este
Ma perché non é il Diaual fempre mai
Cotanto brutto com! eglié dipinto,

    
 

      
  
      
 
 
    
  
     
    
    

 

    
  
  
   
     
    

   

 

  
    
  
  
 
  

  

 
 

  
     
 
   
    
   

 
 

     

    

  

   

 
 

 
   
  

QVARTO CANTARE: ty

‘A XXXXIIL STANZA XXXKIV.

un bel g ‘ E perché quivi ancora haurd paura

Chrionon vada a ftucbargli il (uo ripofo,
'\uHlaurd fopr’ ad un monte fepoltura,

Che mai fi vedde ul pik precipiteo ,
Ed alto poi cosi fuor di mifura ,
Che non v'andrebbe il Bartoli ingegnofo;
Oltre che innanxi ch'io vi poffa gingnere
4 Ci-vuol del buono,e ci [ard da ngnere .
forma d'un bel giovane,ia(ciata Ja fozza figura del
jalei fa comadamento,che di.cid in maniera alcuna non parli,perché altri-
ficendo ; fara-cagione , che égli Ja jafci , ¢ f¢ ne vada in luogo da non po-

trovaro.

i. Condefcefi ; acconfentj , mi v’ accomodai ; vedi in quefto Can,
prefo per accomodarfi col corpo ;¢ qui ¢ prefo per accomodarfi con,

  
 
   
  
   

    

 
 

 
   
  
  
 
  
  
   
 
 
    
 
 
 
  
  
 
   

a ee

—

partite vinto. Veduto che la cofa haveva a andare in quella guifas ;
dia diverfi fignificati: perché vuol dire Serutinio , che noi corrot-
n0/quitrino, Vedi foro Can, 6. ftan. 109. , ¢ di qui Viffe il partite
dire Vifto , che il negozio era ftabilito cosi , perché quando il parti-
il negozio §’ intende ftabilito . Adetter il cernello a partite , fignificas
metter in dubbio uno fe deva fare, o non fare una tal cofa. Donna di partite yuol

ice .'Si piglia'in vece 4° accordo’, patro , baratto , 0 condizione , Io vendo
4 col tal partito , ec, Significa ri/oluzione , 0 determinazione. Io ho prefo

=

Rs. Ps -s =

wm i ine . Significa termine , pericolo , 1] tale ficonduffe a mal parti-

4 MO, Clot a catrive termine , 0 4 pericolo di vita, 0 powertd. Ci (erue per efpri-

ch, Mer maniera , modo: lo non vi verrd a partito alcuno. Significa rimedio, e/pe-

i? dente. Prefero per partito di fegargli la gamba,cc.

A “AL Diaval non ¢ brurto con egli ¢ dipinto . 11 Maleinon é poi fempre tanto,quan-

ed taro.

f ila gola ., Immerfo nelle difgrazie. Vedi fopra C. 2. ftan. 44. il fuo
; AQVATT R cechi, A folo a folo , Remoris arbitris ,

ist ‘Sila »de'farté mia’, Non voglia faper pid nulla dime. Tratto dall’an-

g@ ‘ico, Come.fi vede in Pilaco , che col lavarfi'le mani pretefe di non haver, che fa-

gh tenella Sentenza data contro al noftro Sig. Giesi Crifto. Ii Lalli Eneid. Trau,

C4. flan. pa.

oy wid E mi lavo le man de fatti tok

{ AL Bartel ingegnafo di Bartoli , che ha ftampato un trattato dell’architettura,
perch diceingernofo cioe ingegnicre , che appreflo'di-noi vuol dire Architecto ;
_ €non Bartojo legifta(come fi trova in altunr tefti ,,dove dice Bartolo, enon il

«® Bartoli »»perché tratandofi di falire un luogo erto pud giovarpil i fapere d’ un’

oe itetto:, ‘che quello d’ un Legitta . ‘ SE Stetson

tl Ol val del buono. Ci fara molto da faticare , 0 da {pendere, o da camminare,

, — @fimili, feruendoci quefto termine per — tutto quello ci pofla ates necet-
Ss fario

 
 

eee

aio MALMANTILE )

fario in uno affare , fecondo la fubietta materia, come per efempio : A feriver
la prefente Opera ci vuol del buono , € s' intende ci vuol molto tempo , moltas —
fatica , molti fogli , ec. ed é lo fteflo che ci fard da mgnere. Ll che viene dal me-
dicare i feriti,e perd-per Jo pid s' ufa in cofe di poco gutto,e faftidiofe, per efem-

pio: Ll tale amaiazzo uno , vuol haver da ugaere ,cioé yuol hayer

molti trava-

gli, (pele , difficulta , ec. ad aggiuftare il negozio . 11 Mureto lib. 9. cap.13.Var,

cur,
STANZA XXXXV.

Pos ch’ una firada trovero nel piano y
Che veder non fi puo gid mai la peggiay
Poi giunto apie del more alpeftre, eftrane
Con due uncini arrampicar mi fe
Menado alt'erta bor l'una,por U'altra mandy
Come colui , che nnota di {palfecgio y
Ed anche andar con fléma,e co gindizio
‘S’ io non me ne vogl' ire in precipizio,

let, dille; Non parna & panca, fed multa & magna ad hoc effcienduns req

STANZA XXXXVL
Scofcefo é il monte in fomma, e dirupate y
Edl viaggio lunghiffimo, ¢ diferto, —
Cosi diffe Cupido (mafcherato,
Dopo civt ch’ ei mi fi fu feoperto;
Ond? io promeffi di non dir mai fiato y
E che prima la morte bauria foferts,
Che tra(gredir d'un pittoin fatti in detti
Afuoigufti,alucicenni fu

Cupido accenna a Psiche parte delle fatiche , e travagli , che ella havra nell
andare a ricercarlo ; ¢ Psiche gli promette di non dir mai nulla a nefluno,

VNCINI, Strumenti di ferro adunchi , ed aguzzi ,, feruono per a
gualcofa , ¢ fi fanno anche di legno per ufo di corre frutti , ¢ per altre occorren-

ze ruftiche .

eats

RAMPIC-ARE . E proprio dei-gatti 5 ¢ d! altri animali fimili , che falgono

fu per gli alberi , appiceandofi co’ rampi , cioé con I’ ugna delle
fotto in quefto C, ftan. 68. E ci {eruiamo del verbo rampicare per efprimerc uns —

. Vedi

che falga in qualche luogo difficile , ancor chedo faccia fenza rampicare.) Vedi

forto C. 9, ftan. 25.

tere
NVOT A di (paffezgio, Nuotare di {paffeggio diciamo quand? uno effendo tut-
to nell’ acqua dalla tefta in fuori , cava fuora di effa un braccio per volta ordi-
natamente , battendolo fopra all’ acqua per romperla , ¢ (pingerfi avanti.
NON dir fiato, ¢ non fiatare . B lo fteffo che non parlare . Vedi foro C.6. ft
42. Si dice anche non alitare . Non far verbo, Berni Orland. ae
E ferra piit fiatar mi.Pava chiasto, Vedi fopra C, 1. fan. 10.7
_GVSTI, cenni , precerti , In quefto Inogho hanno tutti tre lo fteffo fignificato
di comandamento. Confiderandofi gu/o per il meno ftimato , cenno nel
dJuogo , ¢ precetto pet lo pi ftimato , denotando dominio.

STANZA XXXXVIL

We tal cofa a perfona haurei feopertas
Perche rusravia ta gente {oiocca
Riden del rofpo , ¢ davami ia berta;
Ed io,che quand’ella mi Venne in coca,
Won fo tener un cocamero all’erta,
MMs lafciai finalmente ufcir di bocca,
he quel non era un rofpoyma in efferta
Yn Srariofo y ¢-vago giovanesso »

STANZA XXXKVIIL

E che y fe lo vedeffon poi la notte

“Quandin camerd messin
Exgetra via la feorxa delle bette
Chun fole proprio par Sputate »
Le male lingwe forfe Mparian chiettes
Che si doteiars Se
Pero che now fi pus tiraran peo
Chil comento ay veglian fre ree *

   
 

sn

S@eenvwris =

anes

ear

ES SsHAewvaesak 7a e FES AF ES

wa

 

eae

 
 

ee

= 3

,

a

“

QVARTO CANTARE: zit

_- Vinta Psiche dalla collera , che le venne per effer burlata dal’ altre donnes ,
Scoperle il fegreto 5 E nota che PAutore moftra il coftume delle noftre femmine,
e quelle di tutto il mondo , le quali obligate a narrar qualche loro mancamento;
fi fanno dalla lontana, e “ rfaadere d’ haverlo cx i xo
forzate da’ maggiori mancamenti d’ altri.
pe cme berta, Mi davano la burla , mi beffavano , mi minchionava-
no, Berta fi dice 3 col l¢,impernato fopra i pali, fi fanno le paliz-
7 Patieead barcode fares i pe via di corde fe ras ensets » che (as in
detto ceppo . E il Latino irridere, Raccontano le noftre donne , che quel fagace
villano nominato Campriano , del quale diremo fotto C, 11. ftan. 48. cffendo ve-
nuto in mano della giuftizia per le {uc cattive opere fu condennato a efler mefio
inun facco , ¢ buttato in mare ; In cfecuzione di che fu meffo dentro al facco, ¢
oC to a i famigli, che lo buttaflero in mare . Nell’ andar coftoro ad efegui~
re ini impofti furono per ftrada affaliti da alcuni mafnadieri,i quali fi cre-
derono , che in quel facco fufie roba di valore ; onde i famigli per fcampar la vi-
ta lalci ivi 11 facco. con Campriano , fi fuggirono, Campriano piangendo
fidoleva della {ua difgrazia , il che {entito da uno di quei ma(nadicri gli doman-
‘db perché piangeva, ed a qual fine era flato meflo in quel facco, Il fagace Cam-
‘priano gli rifpofe ; Io piango di quel , che altri gioirebbe , ed & , che quefti SS.
voglion rmi per ie Berta unica figliola del Re noftro, ed io non la voglio,
conofeendomi inabile a tanto grado , per ¢fler’ un povero villano . E perché efi
dicono , che fe ella non fi marita a me, l’oracolo ha detto, che quefto Regno an-
“dra fortofopra , m’ hanno meffo in quefto (acco per condurmi a farmela pigliar
‘forza ;.¢ + oe ela caufa del mio pianto . I] mafnadiero credendo alle paro-
dicoftui , fi concertd con i Compagni d’ andar’ eflo a pigliare quefta buona.
fortuna , ¢ ripartirla con effi : onde fattofi mettere dentro al facco da Cam-
> che non reftava di pregarlo a volergli tar del bene quando fufle poi Re ,
‘feceallontanare i compagni , ¢ ferratolo entro al {acco , ftette afpettando , che
titornaflero coloro , i quall non flettero molto a comparire con nuova gente , es
veduto quivi il facco abbandonato, lo riprefero , ed effendo vicini alla riva del
mate, velo precipitarono , ¢ cosi fpofarono a Berta i) balordo mafnadiero. E
di qui venne dar la berta , 0 /a figlinola del Re, che vuol dir burlare , minchionare. ,
come habbiamo accennato. Si dice anche dar /a madre d’ Orlando, percht das
alcuni ficrede , che la madre d’Orlando Paladino havefle nome Serta ,
— QVAND' ella mi viene in cocca, Quando mi viene in propofito di dire. B fi di-
ce anche ella mi viene in cocca per intendere quand’ io entro in collora, come s’ in-
tende nel prefente luogo . E cocca diciamo quella tacca la quale ¢ nella freccia.
per adattarla in fu la corda dell’ arco da i Latini detta Crena, donde poi diciamo
eee eae 3 © feflura , che é nella parte ae alla punta dell’ ago da
cucire , dal Gr, e-cocche ; effremitd acuta, Dan, Inf, C, 12.
a cere, Chiron prefe lo firale , ¢ con la cocce
oh ; Fece la barba indietro alle mafcello
TUN Etiiee na caccinnre all erie a Dioogee far di meno di non Ja dire. Si
Eft comparazione al cocomero , perche eflendo quefto di, figura sferica., e»
lilcio , facilmente ruotolando pud {correr gil re un’ erta 5 © monte , ¢ facilmen-
War 2 te

Oh

 

 

 

 
 

 

 
    
  
  
  
    
  
 
  
   
   

212 MALMANTULE) 5.

te pud effer anche tenuto fermo ; onde molto -ben. fi-dice Non fa tenet wn ¢
mero ali’ erta d’ uno che fia facile a palelare:un fegreto y che co ug
potria tacerlo, : « . claticwn ftitrranity
PRETTO fputato. Similifiimo a lui : per appunto come jui,e fenza ale
ne alcuna come é il vino oe 2 eae —- een 3
juclla aggiunta di /pacato fi toglic da coloro, che pigliano:le-m cl
eon ae éninale » i quali in qualche Sesion per andi
punto fogliono tirare il filo 5 e {putandovi fopra lafciano cafcar.
Parte, che gli é fotto,e da quello conofcono fe il lavoro ¢ per appunio,
CHIOTTE. Chere. Voce Fiorentina , ma poco ufata fuor di {chet
ne , come poco fopra s’é vifto,! usd il Berni nell’ Orlando E fenza pits
frava chiotto tO Homa} onus
ST danno piato de’ fatti d' altri. Gli danno penfiero ; Gj fono a cuore
altri, Si mecterebbero a litigare per i fatti d’ altri ; Che Piato yuo)’ d
Vedi foto C, 7. ftan, 27. tb orng
VON fi pud sirar un peto ec. Non fi pud far una cofa bench minima ,
popolo non vi voglia far fopra i fuoi difcorfi , Anh sole t
STANZA IL... STANZA Lhioy

 
 

Le ciglia inarca , ¢ tien la bocca feretta
Chiunque da me tal maraviglia afcolta;
Ma quel ch’ importa afordono fu detta,
Che Vener , ch'ogni cofa havea ricolta,
Per veder s' elle verayo barzelletta,
Poiché a dormire ognun fel’ era colta,
Entra in camera,e vien pia, ‘Piano al letto,
E trova il tutto appunte come ha detto.

  
 

  
  
 
  
  
   
   
 
 

STANZA L
E nel vedere in terra quella Spofigla LVon tivo dir com io reftaffi allora Tya,
Che per celarfi al mondoil giorno adopra, Che mi fovvenne fubito di a lw
Di levargliela via le venne voglia , 4 primo di mi fi fueld, ¢ ancora —
Accs con off pik non fi ricwopra : Mi fece l efpertiffimo comando, 2
Cosi la prendeye poi fuor della faglia Chtin alcun rempo io non ta deffi fora,
Fa un gran fuoco, e ve la getta Sopra 5 Ed io fon’ ita fcioceaya farne un bands, é
We mai di ti fi volle partir Venere = poi mi pare rano,e mi arcoy dy,
Infin che non la vedde fatea cenere , Segli ¢ in valigiased ba copratosiporct. hy
STANZA LIL Uae yeas \
Sofpe/a per un pexxo io me ne fretti, Guarda fu pel camin, civo in fai vetti , a
Chi io afpetrave pur ch’ ei ritornalfe; etpro cli armarjye a Scoftar le cafe
A cercarne per A/a poi mi derti We trovanidelo mai, al fin'msi a i
Per le fhanze di foprase per le bafe; Per non fermarmi fin ch'ia no letrove, tay

Il fegreto palefato da siche , venne all’ orecchie di Venere » 1a quale quando
Cupido dormiva gli abbrucié la vefte da rofpo ; il che veduto Cupido la
fe ne faggi ,¢ Psiche fi mede a cercar di lui m

ba,

aca desta a fordo, Fy detta a chi ne fece capitale 4 8 chi importava (a mt
pe Oo « i i

  
   

213

   

|, Haveva fentito’, ¢ intefo ogni cofa ?

Cofa non vera,ma detta per {cherzo. E fi dice Barzellet-
‘lando, ¢ fcherzando .

fo termine , che vuol dire Adagio adagio , fignifica ancora

c }) Senza far punto firepito , 0 romore .

‘LE. Piccolo piumaccio , fopra il quale fi pofa la guancia , quando

; “rn mame guancia , come in diverfi luoghi, ¢ detto origdie-

Rea Ie Oiler

_ Riveftirfi'da rofpo . Ecco la voce generica animale, che noi

le, come accennammo fopra in quefto C. ftan. 4.

dire. E? lo fteflo termine , che penfare voi, vilto fopra in quefto C.

one voglio dirlo, perché da per voi vel’ immaginerete 5

    
    
  

   
      
 
    

fuora, Non la manifeftaffi , ed io n’ bo fatto un bando; ed io Y ho
tutto. Won modo tubam , fed etiam praconem adhibui ,

Scontorcerfi ¢ proprio delle ferpi ferite ; ¢ parlandofi d’ huomini
n certo atto , che denota dolore per qualche difgufto , o travaglio in-

     
 

valigia , E’ in collora , in ira ; Nel bugnolone , nel gabbione, e fimili ,
ni ne habbiamo in quefto fignificato .
AR it porco . Significa andarfene ; ed ¢ come I’ interpetrazione di /ui~
3 quafi voglia dire fuinam , cioe fuillam emere , 0 che pili tofto fia detto /ui-
fi feappar via dalla viena , e fugcirfene , come quei che {On colti a coglic-
we uva nell” alerui vigna . Diciamo batrere sf raccone , batterfela , cor-
je fe ben fon voci , che hanno del furbe(co , fono perd comunementes
pre intefe in quefto fenfo . Vedi {otto C. 11. flan. 11.
ANZA LiV. STANZA LVI.
¢ via v0 fola fola y Ripongo la noccinola, ¢ la caftagna
ancora una giornata y E rimetto le gambe in ful lavoro
ted{perrami figliuolay Per una lunga, ¢ ferile campagna
dietro veggomi una Fata , Difabirata pitt che lo Smannoro ;
mi diede una noccinola, Dupo cinque ani giuntaa una-motagna,
jee lio, difs’ io, d’ una faffatra M1 fi fe inmanzi un grade, eorribil toro,
un’ altra faa compagna Che ha le corna, ei pit tutti d'acciaioy

  
  
     
   
 

mano anch'ela una caftagna, Etira che correbbe nel danaio.
STANZA LVIIL.
ei mangiato i faffi E come Cavalier ch’ al faracino
‘accomodai per darui fu di morfo, Corre per carnovale , 0 altra feta, »
ch'io non La fpiacciaffi, Verfo di me we viene 4 capo chino

gran bifagno non mi fuffe occorfo Con la (ua lancia biforcata in tefba,
ata di cid con eli occhi baffi Lo gid con le budella in un catino
tai del lor difcorfo , Addio dicevo al Atondo,addiochirefta.

   

Seufe,e refe ad ambe eAddio Cupido dove tu ti fia,
le lafcio, dolla a gambe, A rinederct ormai in pellicceria ,
STAN-

 

Digitizers: By
 

aig MALMANTILE’ ©

STANZA LVIII,. tigeesbiadecs WT ORS
O Mamma mia,che pena, e che (pavento Pur come volle il Ciele io ini rammite
Hebbe allor quefta mezxa donniccinala? Del dono delle Fate , ¢ la nocciuola =
Tremavo giufto come giunco al vento, Prefa per cafo prefto fur un fafo
Che quivi mi trovavo inerme , ¢ fola; La {caglio,ella fi rope, en'e/ce un maffo,

Medflafi in viaggio Psiche s’ imbatté in due Fate , dall’ una delle quali wee
una nocciuola , ¢ dal’ altra una caftagna , ¢ le differo , che non le ftiacciaffe , f
non aun gran bifogno. Dopo cingue anni di cammino per un deferto arrivd a»
pi¢ d’ una montagna , dove le venne incontro un toro con le corna d* acciaio;
dal quale {paventata Psiche ftiaccid Ja n occiuola ,¢ ne nacque un maflo,

FATA, Fate fono donne indovine dette fecondo alcuni dal Greco Phatis che

fuona Donna indovina , ¢ quelle forfe che i Latini co’ Greci chiamano /ibilies
ma dalle noftre Balie nel contare le novelle a i fanciulli fon prefe ;
buon genio , ¢ che fanno feruizio al proffimo con le.loro azioni , € & contrarie
all’ Orco , al Bau , ¢ alle Befane , che foo nimici de’ bambini , a i quali queftes
fempre fanno feruizio , ed i] Poeta , col regalo , che fa lor fare a Psiche, moftra
guefta verita . Da gli antichi furono anche chiamate Ninfe ,¢ Dee, el’
nel fuo Puriofo cid afferma , dicendo : sia
Quefte c’ hor Fate , da gli antichi furo Agee
Chiamate Ninfe ,e Dee con pik bel nome, weil
Di quefte Fate difcorre ’ Autore foto, nel Canto fettimo , ed é credibile, che
quefta voce Fate venga dal Latino Fara fatorum ,che Dan, Inf, ¢, 9, diffe le fata,
Che giova nelle fata dar di cozzo? A
QVESTO ¢ meglio a una falfata, Quando fi riceve da uno qualche regalo di po-
co valore , fidice per (cherzo: Queffo ¢ meglio d’ una faffata , 0 vero a’ un calcio di
mofca : volendofi inferire , che da quello , al nocivo, 0 al nulla vi é poca i
za. Plau. in Tr. difle Afelins eff quam deterrimum , E
ALLOTT A haurei mangiati i faffi . Allora havevo cosi gran fame , che haurei
mangiata qualfivoglia cofa,ancor che dura quanto un faflo, lo crederei,che il ve-
ftitore di quefta favola havefle feguitato i compofitori de’ Palmerini , degli Ama-
dis , ed altri Cavalieri erranti, che mai in tanti viaggi , che fanno lor fare , par’
una volta fi trova , che in campagaa mangiaflero ; ma il fentir , che Psiche
{corre di mangiare , ¢ che fu levata dond’ ¢ll’ era , perch¢ non vi moriffe di fame,
mi fa credere diverfamente , cioé che in quefto fuo iungo viaggio le Pare le em-
pieffero il corpo , che clia non fen’ avvedetic , %
SCHIACCLARE , Corrottamente diciamo anche fiacciare, vuol dir Rompere,
6 infragnere , ed ¢ proprio di quelle cofe , che hanno gu(cio , come noci , man-
dorle , uova , ¢ fimili . i
DOLLA agambe. Comincio a camminare; é lo fteflo che rimetto le gambe in [
lavoro, che é nell ottava 56. feguente . I La)l. Eo. Tr. C, 2, flan. 33.
mand’ so la diedi a gambe ,¢ dentro ann foffo
Lafca Nov. 6. Temendo , che colui non gli ufcife dietro , s* ufch di cafa s pleted, é
la dette agambe,e per la fretta fi feordo di ferrar P nfcio, 1 Lat. pure dilfero conijcert
Se in pedes .
ZO Smannoro, Cosi é detta una gran pianura pofta poco lontana per —

Es iae e-em ee

 

 
  
    

PF See Se

  

SEs eEee eat peeiiak

&
=

SPREE FFG

 
 

 

 

QVARTO CANTARE: ary
alla Citta di Firenze , 1a quale dura pid miglia per ogni verfo,fenza mai trovarfi

una cafa, fe bene é tutta coltivata. Si dourebbe dire Ormannoro dalla famiglia
-antica degli Ormanni , la quale era gia padrona di tutte quelle pianure, che fi di-
Ormannorum

A che correbbe in un denaio, Tira cosi aggiuftatamente , che egli correbbes
Piccolo berzaglio, come é un denaro, che é la quarta parte del quattrino
3 con altro nome detto picciolo , ed un giulio ne vale 160.
iCZNO . Cosi chiamiamo quella ftatua , o fantoccio di legno, che figura
0. armato , al quale ( come a berzaglio ) corrono i Cavalieri le lance;
dice anche Buratto, che ¢ ua’ altra forta di berzdglio( il quale fi mettev
vece del Saracino ) ed ¢ una mezza figura fecondo alcuni’, che nella fiailtra_.
i¢ lo feudo , nella deftra la fpada, o baftone ; 1a quale fe non é colpita nel pet-
fi rivolta , ¢ percuote colui , che falll.
Ld biforcata , Intende le corna del Toro .
CON le budelin in un catino . Mi credeva gia morta ; Mi credeva gia effere fia-
‘ta sbudellata dal Toro. Luigi Groto Cieco d’ Adria , in una fua lettera al Petr.
dice: Quei cani con il loro bau bau ci fecero parere d’ havere le budella in ua,
“eatino , E Catino Intendiamo un valo di terra, o d’ altra materia per feruizio di
ucina , ¢ per ufo di lavar piatti , ec.
} A RIWEDERCL in pellicceria, A rivederci fra i morti. Quefto é il comiato ,
ky

es

 
   
  
   

GREERFE Ge

a che noi finghiamo , che G diano le volpi ’ una con I altra , perché (apendo , che
bm devon efler’ ammazzate , ¢ le lor peli vendute, dicono alli lor figli, quando da
fle fi feparano :. A rivederci in pellicceria , che cosi fi chiama in Reena quella
jp firada,, nella quale (ono le botteghe di coloro , che comprano , ¢ vendono pelli
oi  Gianimali per foderare abiti , ec. ed in mano di coftoro , o tardi ,o per tempo
g¢ -fanno che devon capitare .
I O MAMMA mia, O mia madre. Efclamazione di fpavento , e di timores,
wata propriamente da’ fanciullini, quafi dica : O mia madre foccorretemi ins
t icalo .
~ SONNICCIVOLA, Vuo0l dir Donna di {pirito minore di quel che conuerreb-
€ al fuo naturale, da i Latini detta Aduliercu/a, Siche mezza donnicciuola vuol
dir Donna quafi da aulla , ¢ fenza fpirito .
WNCO. Specie di virgulto , che nafce in lwoghi padulofi, del quale fi feruo-
20 i Villani per legare i cralci teneri delle viti , ec.
444550. 8 intende un faflo grande, Quefti noftri {carpellini chiamano il

 
    
  

maffo La cava delle pictre .

_ STANZA LIX. STANZA LX.

Tal pietra per di fuoraé calamita Sfavilla il maffo al batter dell’ acciaro ,

| E ripiena di fucco artifiziato y £ da fuoco al rigiro ch’ ¢ nafcofo ,

Hor mai arriva il Toro, ed alla vita Ea egli a ragzs cb allor ne feapparo

f Con un lancio mi vien tutto infuriato y Vincolpo fatto haver vede a fuo cofto,
Ata dietro al maffe ero fuggita Perch non vi fu feampo , ne riparo,
. re riman quivi {caciato, . Chrei fra le fiamme non fi mucia arroffcs
| CW in fo dando ba ferrata tefta Ed iofeanfatoilfuoco,e ogni altro arate,

da quclte celamica afifo rea, Lieta mi parto , 0 tire innanzs 1 conto,
Ni: fe sss ; rf

 
 

 

216 MALMANTILEY |

Il detto faffo cra per di fuori calamita , ¢ dentro era fuoco lavorate , onde il
Toro perquotendovi con le corna ch’ erano d'acciaio vi rimafero eo
da quella percofla nacque il fuoco , il quale ’s'appiced allt ordigno 5 ‘abbrucid.
il Toro. Psiche libera da quefto incontro feguito il fu viaggio.) ) ae
CALAMIT A.B: \a pictea fimpatica del ferro yo forfe madre dai L

detta Adagnes . Vedi foro C, 8, fan. 45. € 66.

oy logeny!

FYOCO artifiziata, Vuol dire ogni forva di compofizione fatta con
(che diciamo Da archibufo ) tanto per guerra , quanto per fefley.
RIMANE fcaciaro, Riman burlato, E’ lo ftello, che rimaner con un Z

nafo , che vedremo forto C. 6, fan. 5.

mafio.

2o0h Hd

RIGIKO. Intend I’ ordigno di fuoco lavorato.y che ¢ compofto dentro al

v} of sala,

RAZZ, Raggi di fuoco 0 del Sole , 0 d*altro (cintillante..Ma dicendovaf
folutamente razz! , intendiamo quei fuochi artifiziati, che-fi fanno in Occalione
di fefte con poiuere d’ archibulo conttipata,e benisfimo Jegata entro alla —

dotta come pezzi di canna,

TIO innanzi il conto, Seguito il mio viaggio, Vedi fotto C. 6.ftan. 16, Fane

to feruiva tivo innanzs , © fenza metterui if conto fuonava i) medefimo;

nato da quei , che tengono libri di debitori , ¢ creditori ci obliga a dir cost,

» STANZA LXIL

Piit la ritrovo un grand’ uccel erifone,
E tops affai , che giran.come PARR y
Perch’ egli entrato in lor conuer/axione
Gli becca,grafiase ne fa mille [Praxxj y
Di lor mi venne gran compalfione ,
E vo per ovviar,ch’ei,non gli ammarzi,
Ma quei mi séte al moto,einpic firizza,
E per cavarfi, vien con meta flizza,

STANZA LXIL

Quefto animate ha il buffo di cavallo

Di bue la coda,e in fucte fpalleha laley

M capo, e it colo ginffo come sl gallo, «

Li pie di nibbio.vero ye maturale 5).
Gii artigl di fortifimo metalle
Grandi groffise.adunchi in modo tale
Che non vedefti quando leges , 0 ferivi,
Mai de tuoi di pin bei imterrogacini ,
STANZ on
Son? att poie’a far pis acuta
‘ Seecmaige.aedieeie ighe y
Tal che ,s' al vifo fuffinaiwenuto
Con effi, mi lafciava affai pik righe
D’ un sibro di macfiro di linto, ,
Ed una flamperia di falfarighe
Con farmia life come le gratelle.
. Da quocerui le trigtie,e le fardelle , .

 

STANZA LXIVE ©
Hor os tornare , In quel chia hetimere
Ch’ il

mio grifo fia fcherze

La cafagna ch! io in + ecio fuore

La rompo, en’ efce fubite un Ltone,*

Che mi {como non poco tl barvionore»

Perch egli in mia difefa a lui soppone,

E moftrogli bor con Pigna,ed bor ee dati

Jn che mo fi gaftigan gh infolenti.”

STANZA Lky,

L' uccelle anch' eel, che non ha pana

Géi rende molto ben tre per

Ada quel che haver del fuo ‘4

Al contraccambio fubito
E ben ch’ ei owltepauiiae
+ Liafferraye firinge tanto el
Di poi garbatamente gli iefca
Gli flinchi fu's nodelli, e me gli rech:
STANZA LXVL st
AMetto uno firide ye mi ricive imdrete
Loch he paura aller ch'e: mmm ings »
Ata quegii ie

 
 

a Che mai vedelje st

Cio conofcenao sutta egal
Gti lafcia.in terrase va perfacti fuck >
Ed so gti prendo aliora , efsemdo certs
Diaverne ahaver bifogno in figrad'erta

 
  
 
   
 
   
  
  
   
     
  
   
  
   
 
   
  
   
   
 
    
  
  
   
 

ee a

a

Q

3S

——— eS Re

 

 

   

217
STANZA LXVIIL
piedi, Evconuenne talor farfi da piedi
1 \Bawrendo git di erandi flramazzeoni ,
mi Perché non ve dove fermar il pafsa +

| morte brancoloni y Cagion che /pefio mi trovai da ba/so ,
trato i] pericolo del Toro s’ imbatte in‘un’ uccello Grifone , che ha-
@ acciaio , onde roppe Ja caftagna , en’ ufci un Lione , che la difefe

|» ¢ tagliandogli gli artigit , li porté.a lei, Ia quale gli prefe , ¢
ttaccandofi ail’ erto monte , comincid a falirui .
‘che girano come pazzi , Sorci, che vanno in‘qua ¢ in la correndo fenza_
ve determinatamente , appunto come fanno i pazzi.
RS/ la fizza. Sfogar ja collora , la rabbia , I ira .
0... Vecelio di rapina noto . Qui defcrive il Grifone, ¢ lo fa mezzo ca-
mezzo uccello , ¢ con Ja coda di bue , e fe bene da i pi ¢ defcritto mez-
€ mezzo uccello , ¢ nimico mortale de’.cavalli, come fi deduce da Verg.
rantur iam Gryphes Equis,tattavia non fa errore a comporlo di che be-
iuto » perché quefto moftruofo animale in ogni maniera che fia é
fayolofo., fecondo Plinio lib, 10. ¢. aan Pegafos ( dice egli ) equino capite
Gryphes Aurita aduncitate rofiri fabulofos reor , illos in Scythia, hos ins

BRROG-AT IVO , E} un contraflegno d’ ortografia , i} quale fi pone in fine
che conchiudono sn » Orichiedere , ¢ weet ¢ detto Punto
E:perche tal contrafiegno ¢ di figura fimile a un’ uncino, pero a
q igliamo gli artigli degli uccelli , come fa qui il Pocta, affomigliando-
pli a quelli del. grifone ,
LI baaeftro di linto , Intendi libro da mufica,, che fon pieni di righe,afi-
fe di icriverui fopra le note muficali .
PALSARIGHE .. Carte rigate , ¢ lineate dinero , le quali fi mettono {otto al
al quale fi (crive,affine di far i verfi diritti ,‘ed uguali camminando
0, che dalla faifariga per trafparenza fi vede fopra il foglio , ove

PS

LIST] :. Qui vale per firifcette di ferro , 2 le ees compofte le gratelle
firumenti dacucina , che {eruon per metterui fopra 1! pefce , 0 aliro.a quocere»
arrofto Econ mute quefte fimilitudini intende , che fe ucceliot ha vette meta
fli artighi addoffo a Psiche , !’ havercbbe malamente graffiata , ¢ segnata .
G . Vuol dir Faccia di porco, o fimili; ¢ s’ intende alle volte: la faccias
3 gan ame per naneeae » 0 per difprezzo ; € qui il Pocta fe ne ferue per far
bifticcio rh e¢ Grifone ,
x “4 VORE ~Panra 3 tumore. Da quella frequenza di battere, che fa il
a dalla ee del — » quando fi ha qualche fpavento : 1 Latini pure di,
anitah  velcordis percnffio.
INSOLENTE . Acrogante , faftidiolo , petulante ,. Vno che tratta., ¢ proce.
~ per copia « Gli rende pib del fuo dovere., perché.a ‘render
che ¢Ja coppia,fi ms la meta pid del dovere : B con quetto
€ . modo

jiized le
s
Se sna

218 MALMANTILE .¥

modo di dire s' intende , che uno Gi difenda da un’ altro con pare
{empre con vantaggio , che diciamo anche render pane per

NON fi cura haver niente di fue, Intendi Non vuol’ effer da lui fope

e4f PERRARE , Abbrancare, pigliare ftretto ; 1 apprehenfam
NODELLI, Intendi la congiuntura delle gambe co’ piedi, ©
eANDAR carponi. Camminar co’ piedi, ¢ con le mani per terra ,
fieflo , che Andar brancolone , che fi vede nel verfo feguente ; fe non che qu
vuol dir Salire adoperando le mani, ¢ i piedi ; ¢ carponi ¢ camminare alla p
con le mani , ¢ co’ piedi , Dante Inf. C, 26. defcrivendo una fimil falita dice
E profeguende 1a folinga via *
Tra le febegge , ¢ trai rocchi dello feoglia
i pit fenza la man non fi [pedia , ae
STRAMAZZONI. \ntendi Calcate; che per altro #ramazxone intendono gli

{chermitori una {pecie di taglio .
STANZA LXVIIL

Tusei quei topivia ne vengon ratti,

E furon per mangiarmi dalla fefta , ,

Pero che dale granfie io gli hofottracti

Di quella beftia a lor tanto molefta ;

Cost ve rampicando come i gatté

Suil' afpro monte dietro alla lor pefta ,

Sopportando fatiche, frenti, e.guai ,

E fame, ¢ fete quanto fi puo mai,
STANZA LXIxX,

Pur finalmente in capo 4 due altri anni
Giungemmo al luogo tanto defiato ;
Ma non finiron qui mica gli afanni ,
Perché di muro il tutto ¢ circondato;

EB qui s'aggingne ancor malea malanni,
Chto trove Cufcioyma'l trove diacciato;
Penf{a 8 allor mi venne larapina ,
Es’ 10 dscevo della Violina,

 

see
STANZA LXX.

Hora tu fentirai ch’ il dare aiutca

A tutti quanti fempre fi connine y

Percht gid mai quel tempo s'? perdate,
Che dicepirws in foretell,
Non dicofet althuom, ma ico a un britd,
Che forfe immrondo, ¢ inutile fitiene y
E che tx non lo fRimi anche una chioft,
Pero che ognuno é buono agqualehecifa,
STANZA LXXL
Setu giovi al compagno, allor tu fai
( Quafi gli prefti roba ) un capi
Anyi talor per poco, cheghaai
Ti rende psa fei volte che non vale.
Ma non Fase io pretender mai,
Perch? ell é cofa , che ftarebbe male;
Quefod un cenfo il quale achiloprende
Rrchieder non fi pwd s' ci non to vende,
facendole

 
 

1 topi, che Psiche liberd dagli artigli del Grifone la feguitarono
gran fella, ¢ con quella campagaia in capo a due altri anni arrivé Psiche al luo
go dove era Cupido, che era un recinto di mura , dentro al quale non fi poteva
palsare fe non per una fola porta , ¢ quefta era ferrata, oh Reed
VENGONO ratti. Vengono velocemente:dal Latino rapidus, D, Infer, ©. 21.
Perch’ io mi moffi , ed a lui venni ratto Ee

4 whe
Ed habbiamo rartezza,per preftezza, 0 velocita . Varch, Stor. lib. 4. Za ele j
Lae

70 il Sig. Sciarra Colonna fenl ¢on gran rattexza da Roma, ;

FAR fefta anno, Ral

egrarfi conuno. Ricevere, © trovar uno con atti di

amorevolezza:, ¢ cortefia ; Che nelle beltie fi conofce tal rallegramento da i
fj, come nel cane dal dimenar della coda , ne i gatti dal fregarfi u
ed altri animali dal moto degli orecchi, come forfe fi cono{ceva in quei topi .

Lat. adulari fanno yenire alcuni da ad,& wra,che in Greco fignifica coda quafi fit

   
 
 
  
   
 
    
 
 
 
  
  
  
  
  
 
  
 
 
 
 
 
 
 
 
 
 
 
  
  
 
 
  
   
  
 
 
 
 
    
   
     
  
       
 
   
 
  
   
   
   

VARTO CANTARE: 219

ndi falire appiccandofi con gli artigli del Grifone , co.
i. Viene a. che s’ intende ugne di gatto, lione , tigre , €
anche snerpicare ¢ ico Mtrumento ruftico da romper le terre .
Franzefi fopra alle mafchere dice :

| Nom-vi crediate , che qualunque faglie
Haveffe da [ua pofta tanto ardire ,
Chr inerpica/se fopra alle muraglie

fi dice inmarpicare,e annarpicare . Vedi fotto Can. 9.ftan,

w
‘alla lor pefta , Seguitando le lor pedate ,
a icella riempitiva in i er emfafi

ir a ‘B'
appunto come i Latini dicono ne quidem » fe bene & diferente dal
¢ non s’ ufera per affermativa , io voglio mica , come effi dicono ero
che fe bene ¢ per emfafi ha perd qualche parte del negativo , quafi
fo now voglio ne pur’ una mica , che vuol dir minuzolo di pane , o granel-
Petr. Son..91. We mica trove il mio ardente defio .
Dolori di cuore , che fanno quafi venire in angofcia, Petrar.

 

 

Buy Leh >

  

ris

Se la mia vita dal afpro tormento

Si puo tanto febermire , ¢ dagli afanni ,

GER male a malanni . Al male accrefcer male , ¢ peggic .

(0 diacciato . Cio’ porta ferrata . Vedi fopra C. 3. ftan, 3.

la rapina’, Mi venne rabbia , collora, o ftizza.. Rapiva vuol dire ru-

quindi uccello di rapina ; ma dalle noftre donne é prefa ins

» per sfuggir di dire rabbia creduta parola peccaminofa , ¢ dico.

i rv arrabbiare , ed arrabbiato

bicevo del male fra me medefimo , perché le cofe non

mio modo. Quefto so che fignifica Dir della violina , non so gia da.
gine quefto dettato , che ¢ lo fteflo che Dir P oraxione della ber-

ee

i una Chiofa. Non lo ftimi punto. Vedi fopra C. 3. flan, 60, alla

capitale. Metter infieme una fomma confiderabile di denaro per ha-
a ogni fuo bifogno : Si dice anche far un’ afsegnamento,
$0. La namra del cenfo , ¢ che colui, il quale presta danari a cenfo,non
richieder la fomma principale , che egli da , ma folo i frutti d’ efla ; pud ben
steel Ja medefima fomma principale a ogni fuo piacimento ,
la diede @ forzato a riceverla , come dice il Pocta affomigliando co-
ilpiacere a un’ altro , a uno che dia a cenfo , e dice , che colui che
non dee , ne pud pretender la ricompenfa , ma Ja pud bene {perare ,
creditore : ow ben dice Seneca de Beneficijs \ib.3.c.14, Vide etiam
crit , nulla repetitio, B lib, 4.cap. 39. Alia conditio

beneficio .

SEE

 
  
  
      
    

 
 

su Se

Ee 2 ' STAN-

citized bylibgie
*
_ yorture di cofe di fua qualita , ec.

 

  
 
 
 
 
 
   
 
  
 

220 “MALMANTILE)

STANZA LXXIL

Guarda s’ el? é cost; Lo per la mia
Picea di prender di quei topi curay
Da lor vinta refpai ds cortela,
E wt hebbi la pariglia con?) ufura y
Peri ch’ in quefia xxx ricadia ,
Ch’ io ho d'haver trovata claufura.,
Eglino tutti ful cancel faliro ,

E fi fermara , ove ¢ la toppa , ingiro,

STANZA LXXIIL i

E gli denti appiccando.a quel legname, Cupido etmor , che tanti ba
Come s' in bocca haveffero un trapano y Berzaglio qui fi giace della
Prefeo prefto vi fecero un forame Ei chera fuoco,il nafo
Da porre il fiafco,e vender iltrebbiano, i 0
Tal ch’ in terra cafcando ogni ferrame
Spalanco l' xfcio di mia propria mano,
E paffo dentro ,¢ refto pur confufa ,
Perch’ acor qusvie un'altra porta chinfa, Non fard U urna, che glié qui daca
1 Topi fuddetti rimunerarono Psiche , perche rodeudo fino a fette porte, che
erano in quel Serraglio,fecero cafcare i ferrami , ¢ Psiche entrata dentro,trovd il
fepolcro @ Amore , ¢ dail’ In{crizione, che in effo eras comprefe quello’
reftava da fare. t $

HEBBE la parigla . Hebbi il contraccambio .. B’ il Latino Par
Pariglia intendiamo due cole uguali nel giuoco di Carte ,,0dadi,, come due!
due alli , due figure , ec, ¢ di tal voce non ci feruiamo fe non nel giuoco 5 0 nel
cafo del prefente luogo di reader contraccambio si.in bene , come in male }
forto C, 6, ftan, 69. Io I’ ho per voce Spagauola , ed il Varchi nella . 8.
} usd ia ua certo modo come ftraniera dicendo.: Dopo eferfi vendicari
renduto i contraccambio , 0 y come fi fuol dire , la pariglia . y

CON ? xfura , Col frutto . Cioé mi contraccambiarono, facendo
vizio a. me , che non havevo io fatto a loro ; :

ZEZZA, Vitima. E’ voce antica hoggi poco ulata fuor che nel ,
Vedi fopra C. 2, flan, 2, Si trova anche/egee y/ezzaiay 0xerraid, >>

IC.ADIA , Noia , travaglio y avverfita , moleftia, © fimili che vengono\
aun’ altro dilgutto ; da ricadéa , che ¢ quando uno infermo gia quaGi fanato , vie~
ne a riammaiarfi , o per lo mal governo , 0 per altro. Nella ftoria di Semifonte
Trattato terz0 . Con li loro misfatti , dando alli Fiorentini non ic
Sac, Noy. 98. Che ricadiaé que/pa di quefti porci?.. t rs 19:0).
CANCELLO . lntende il legaame , che chiudeama.porta : ma pro}
cancella diciamo una chiufura di porta fatta di ftecconi , orftrifce di legno
ferro feparate |’ una dal’ altra a guia di gabbia. ; ora.gve

TOPPA, Intendiamo quella piaftra di ferro, fopr'alla quale fon
ingegni della ferratura , detta affolutamente , o fenza aggiunta , perché per alt
Toppa G dice ogai pezzo di panne , legno , quoio , ferro, ec. che s’ adatti :

 

 
    
      
 
 
 
   

    
      
   
   
    
   
       
  
    
 
  

          
 
    
  

   
  
 

ia

 
  

 

    
  
  
    
   
    
   
  
  
   
   

Be

EST SPSS SSeS Fo BF

Fee

 

 

QVARTO CANTARE? a2r
fruménto {pecie di fucchiello , col quale fi forano mate.

tre 'metalli , ec. eel odeeaa

@°, Coloro che vendono il vino a
loro » come dicemmo fopra C. 1. flan, 76, ed oltre a,

pit nella porta , o nel muro una fineftrella,per la quale dan-
vendono'; a’quefta’fineftrella atfomiglia il foro fatto da i
0 iano pigliando quefta fpecie di vino per tut-
‘intende effer quefto tale sfondato fimile a quello , che fi fas

vendere if
dere il vino’.

Ateneo

ca
MOR 2a.20

che Cupido ¢'freddo , cioe morto

{chi , appiccano un fialto

‘£. Aprire largamente , quanto fi pud .
bere un novo. Fu cofa faciliffima,come ¢ i] bere un’ uovo : i Gre-
‘in quefto propofito Oxo patio quis ovum forberet je trovali quefta,

0 agrafio . ipingere-a grattio , sgraffio , o graffito ,é un’ imprimer
anwterisacee nell’ ifesthacdties fee(ca de’ muri con deteo ferro,
ja graffio , forfe dal’ antico graphium , che era lo filo di ferro , col

(ONARE ; 0 sbolzonare’. Sacttare , frecciare , da bolzone [pecié di frec-
Mattio Franzefi fopra alla boria dice :
Di qui Amore accorto balefriere
\* “Bolzona quaiche giovane galante
lato’, Ha il nafo freddo. Pighando Ja parte per il tutto,vuol dire,

A? Animiale'noto ; ma qui fi dice una,che chiacchierando affai,non

fe fa tener
ree

we

Feta cofa alcuna; ¢ degli huomini diciamo Cicaloni, Appretfo
‘cicala non {ona male , poich? alle cicale fono da effi raflomigliati in pit

aa peril continovo cantare , che fanno , ¢ quefti, e quelle. EB
quelto Poeta graziofamente chiamo Mufa la cicala fopra C. 1, ftan, 2,

STANZA LXXVL
Non tivo dire adeffo sin quel cafo
ro gli occhi due fontane ,
“E feci'come thi 2 rotto il nafo ,
* Che verfail fing ne,e corre al lavamane;
“malin oorer 4 quel vafo
Durante a lagrimar fei fertimane ,
¥. il pite voglia di piagnere,
inte ceils miebbi iniricobe
Pret aes LXXViL
Quand io ch’ egli era poco meno
“Mn fach Muh ape 4 buon porto,
Valli innanzi ch'e fulfe affatro pieno ,
Ech il marito mio fuffe riforto .
Lavarmi it vifo ye raffettar mi il feno ,
Accio st lorda non m' baveffe {corto ;
Percio miparto ye corro,se in quel monte
Per avventura fuffe qualche fonce .

 

STANZA LXXVIII,
In quel clt io m’ allontano com’ io dico y
AMiartinaxxa,che era in Stregheria,
P fio di la portata dal nimico ,
Che non porette tar per altra via;
E perché fempre fu [uo modo antico
Di far pertutto aalcun qualche agheria ;
Lefie il pitaffio, (quadro Purnaye tenne,
Che li fufse da farne una folenne ,
STANZA LXXIx,
Se qua, dice fra fe , Cupido dorme,
Vito rifuegliarlo per veder un tratto
“ Sregli ty come F dice , ¢ fe conforme
eA quel che dai Pittori vien ritrarto *
Se ben chi lo fa belo , ¢ chi deforme ,
Bafta mi chiarird com’ egli ¢ farto ;
Per quefto ad emprer mettefi quel vafo,
ef cui poco mancava ad efser raso,

Zed bya
 
 
   
   
   
   
 
   
     
 
   
   
  
 
   
 
 
 
 
 
    
  
 
   

222 MALMANTILE 5)

2 STANZA LXXX..%
Con P animo di pianger vis arreca , Al fin fi pone a
4a ponza pontia, lagrima non getta Si che per forza a pianger é
Si prova a far cipiglio , e bocca bieca , Onde la pila in Ai

Ne men qucfiaé pero buona ricetta; Refto colma , ¢ Cupido fe
In ordine al Cartello havendo Psiche con le fue lagrime quafi piena I’ urn
ando a Javarfi il vifo, e raccomodarfi la tefta ; Intanto Martinazza arriyo:
polcro , ¢ con le lagrime fue fini d’ empier l’ urna , ¢ Cupido ufci dal Sepoley
WON ti vo dire, Quefto termine (crue per efprimere. Date puoi ben fa
Sta cola meglio di quelo che io fapeffi dirti; 0 vero so che tu hai da per te tant
gindicar come io rimaneffi , fenza che io te lo dica , Suona lo ftefio che penfa
dica tu ,tu puoi fapere , ec, Vedi fopra in quefto C, flan. 41. flan. 52,5 ¢ fan,
Simile é quello : Non domandar , fe Durlindana taglia. __ +p ae
LAVAMANE, ¥ uno ftrumento di legno , 0 d’ altro , che con tre piedi
ma come una piramide in triangolo equilatere , ¢ fopra effo fi pola Ja catinell
altro vafo per Javarfi le mani. ri;
ERA poco meno che ail’ rie, Era quafi pieno . L’ acqua arrivava
mita del yafo: che queflo vuol dire or/o , che viene dal latino.ora y fi
¥ eftremita di qualfivoglia cofa .
LORDO., Schifo , intrif©. Dal latino Luridus. eee
VA in fregheria . Dicemmmo fopra C.2. flan, 11, donde derivi tal nome di Stre-
ga, c¢dal C. 3. flan. 69, dicemmo effer fama , che tali Streghe vadano la notte:
cavallo in ful caprone a Benevento al congrelso de’ diavoli. E quefto: (
cendo Andare in Stregheria portata dal zimico, che vuol dire i] Demonio , in for-
ma di Caprone . Che quefte donnicciuolucce credure Streghe vadano in ful Cas
prone a Benevento ¢ opinione vulgata ,¢ molti di ceruello _debole I hanno per
indubitata , e le medefime Streghe fe }o credono , perché il Diavolo con illuf
fa loro apparir per vera quefta falfita; Mala graziofa fagacita d’ um Superiore
ne fece chiarire tutti i dubbj in quefta forma . ee
Fu condotta alle carceri una di quefte tali inquifita di maliarda , ed il Gindi
dopo molte efame havendo troyato , che veramente coftei era una donna , che fi
credeva far malic , ftregar bambini , ed altre {cioccherie, ma in effetto non v'era
cofa di conciufione , o di propofito, rifoluette di gaftigarla per la mala i
ne , ed in tanto foddisfare alia propria curiofita. Fattala perd venire a sé 'inter~
rogd fe andava anccr’ ella a Benevento , rifpofe che si, onde egli le die: Tovi
voglio perdonare fe voi andrete quefta notte a Benevento , ¢ domattina mi race
conterete quanto vi fara fucceflo . bifogna che mi diate la liberta,replicd la don-
na, accid io poffa nella mia ftanza fare i miei {congiuci , ¢ le mie unzioni; il
Giudice gliela concedette con quefto che voleya dargli da cena infieme con
compagno : il che accettd la donna , baftandole effer fuori di quel luogo., dove il
Diavolo non poteva capitare. Andata dunque a cafa cend con il detto compa:
goo , che cra un giovanotto ortolano ,¢ con un’ altro giovane, che la donna
© che egli conducefle , ¢ beyuto abbond: come era il {ao coflu
me in tali fere di viaggio , la(ciati i commenfali a tavola fen’ entrd nella folitas
camera , € quivi fpogliatafi, {enza ferrar la porta,ne le fineftre della medefimas
camera »

 
    
 

223

# Yordine del Diavolo ) s'unfe con pit forte di bitumi puzzolen-
liacere in ful letto, fubito s’ addormentd; I due compagni , cosi
camera , ¢ legarono la donna per ecia , ¢ gambe alle
del letto , ¢ beniffimo 1a ftrinfero con funi, ¢ fi meffero a chia-
voc! , ma come fuffe morta non faceva moto , ne dava fegno
onde i detti cominciarono a martirizzarla bruciandole hora
hora una coftia , ¢ finalmente cosi l'impiagarono in diverfe parti del
arfero fino alla cotenna la meta della chioma ; Cominciando a veni-

} donna con fofpiri , ¢ lamenti diede fegno di {uegliarfi, onde i det-

Jegami , ed uno di loro ando per una feggetta , ¢/’ altro Ja rivefti
¢ dal fonno , e molto pid da 1 martorj ; giunta la feggetta,in efla
tarono al Giudice , il quale I’ interrogé {¢ cra ftata a Benevento , ed ella.
che’si, ma che haveva patito gran travagli , ed era ftata baftonata cons

ferro infuocaté, ¢ ftrafcinata , ¢ legata per Je braccia , ¢ per le gambe,
riportata dal fuo Caprone, che nel la(ciarla le haveva abbruciate con la
fa mezze le trecce , ¢ quefto perché ella haveva ubbidito al Giudice , ¢ che
itiva morire dal gran doloredelle piaghe . Il Giudice ordind , che fubito ful-
ita , come fegui ; ed intanto diffe alla donna : Io v' ho fatto {cottare , «
gaftigo del tuo errore , ¢ perché tu conofea, che non altrimenti a.
yma in cafa ua hai ricevuto quefti travagli , ¢ ti rifolua a lafciar que-
Re falfe credenze ; che fe lo farai , io ti perdonerd . Da quefto bel modo di gafii-
cay Pt arguro Giudice quella verita , che apprefio Jui era certiffima .

NON, far per altra via, Non potette cflere in altra manicra , perché
non havrebbe mai potuto falire su quel monte; fe non ve I’ havefles
iavolo .

R/A , Violenza , difpiacere,foprufo. Viene dal Latino greco Anga-
4 cuaétio. Varchi Stor. Fior. lib. 2. E perche i Fiorentini nuovi tributi,
ritrovare havevano .

  
   
     
   

   
    
   
   
   
  
 
 
  
 
   
  
     
 
  
   
   

tuna folenne . Fare un’ angheria delle maggiori , che fi poffano fare .
me & da noi fpefio ufata in vece di grandiffimo , ed é tolra da i riti
t Chiefa , che fi dicono fefte folenni , le maggiori fefte, che feguono nell’An-
0. Cosi bieros , cioe fagro , pretio i Greci , ¢ facer prefio i Latini vale taluolta
Brandifimo, e4nchora facra 5 Adorbus facer , & lo steflo , che Anchora maior ,
us maior, B, Virgilio quando diffe ; uri facra fames, per avventura inteles
ma. .

VIEN ritratto, Vien dipinto . Se il dipinto 2 come il vero. Dice : chi to fa.
9 ¢ chi deforme , per intendere , che i pittori da pochi foldi lo dipingono
“eAD fer rafo, Ad effer pieno affatro. Viene dal mifarare il grano con lo

Maio , che per dare, ¢ ricevere il dovere s’ empie lo ftaio, ¢ quando é pieno fi

- fttifeia fopra con un baftone , ¢ fi fa ca(care quel grano,, che é fopr’ alla boccas

ba quefto fi dice radere , ¢ tal baftone fi dice rafera, € lo ftaio cosi pie-
Oli dice rafe 5 ciot picao per appunto fino all’ orlo della bocca . a
F
Je

 

   
 
   
 
 

224
VIs' arreca, Vis’ accomo
42, s' arrecé con I’ ani
PONZ A po
fizto,quafi riducen
mandano fuora il parto
tare, come fi vede dal rca, che dice; oie
Je riconobbi a ula buom che ponta
L’ Efpofitore dice ideft che ipinga . Vedi 'Alunno fabr, num,
Ed il termine ponza ponza ferue per e{primere uno , che aff ora
da poco ; che fi dice anche trefca trefca . Ticche ticche , denneinne y,
forto C. 5, flan. 51, Za vanum laborare. Se bene qui Deaheaiers
nazza moltiffimo ponzaffe . joke pibale
C/PIGLIO, E! uno increfpamento della fronte fatto in git la
occhi , ed é una guardatura d’ uno adirato , 0 d' uno eftremamente fup
piglio del ciglio, Gli antichi, come,Daate differo Pigtia,la guardatura.
BOCCA bieca, Bocca ftorta . La voce biero Latino obliguus , ¢ ufata affai
Legnaioli per intendere I’ incgualita d' un legno , ¢ digono sbiecare quan
reggiano , ¢ fanno uguale . ; tahoe
PILA, E’ proprio quel fodo , fopra il quale pofano gli archi de i
fi piglia oat fs quel vafo grande di pictra , nel quale fi mette ac
heverare Ie beftie, o per altro ufo fimile ; in (omma per pila intendiamo.
fo di pietra che tenga , 0 riceva acqua. owasell
STANZA LXXXI, STANZA L
Quand’ ella verfo lui volte le ciglia, Fermoffi a Adaimantile ,
E vedde quella [ua ea, figura Lo vole, ¢ gid le moze
Difpofta  ¢ graziofa a meraviglia , Come fai tn ( dirai ) custo dfeguita?
Cie pik are KS ‘far n' una pittura, Lo so, che ub bo a le. :
Gli s avventa di fubito , e lo pigha, Aeeee mi donar quel
E ,fenza ricercar della cattura , Chiin due Aquile effendo trasformatt,
Dat fuoi frafieri renebrofi , ¢ but Perché lafsi facea degli shavighiy WS
Portar fe ne fa via con effo lui, Mt han traportata qua ne hy it Be
Mactinazza porta via Cupido , ed in Malmantile lo piglia per marito; ©
havevano raccontato a Psiche le Fate., le quali trasformate in due A ic Vhave.
yano portata via da quel monte co’ loro artigli . E qui finifce il quarto Cantare,
CATTVRA, Si dice quella fomma di danaro , che fi da a i birri quand’ haa:
no pigliato uno ; ¢ fi dice anche catrura quella polizza, ¢ ordine che fi da alli sbir
ri perché piglino uno. Di qui i! Poeta cava lo (cherzo dicendo , che Marti
za piglio Cupido feng’ haver |’ ordine della cattura , ¢ to port via,¢ nona
t0 , che le futle dato il denaro della cattura , che havea fattadilui,
PACEA degli sbavigli. Si dovrebbe dire shadigi.Dan. Inf. C. 45.
: Anxi co pie fermati shadighava
ie Pur come fonno,ofebbrel’ afjaliffe i
Ma hogei fi dice sbavieli , ¢ sbavigiiare ; che un’ aprimento
do il fiato , ¢ poi mandandolo fuora , i) che per lo pil ¢ cagionay (
penGeri , da triftizia o malinconia , o da altro rincref{cymento ,

 
 
 
 
      
    
    
  

pee

  

 
   
 
 
 
  
    

 
 

 

   

   
        
    
     
  
    
       
   
 
   

'
   
 

   

TO CANTARE. ary

¢ frigidi generati nello ftomaco da ozio , ¢ da pigri-

bocca per la via del cibo , ¢ {pargonfi per le ma(cella , ¢

gli fuora , alita con aperta bocca, il che da i Latini
ili, Significa non haver roba da mangiare , ne»

> ¢d habbiamo una rima , che dice; I

aviclia non pud mentire *

- Ocgliba fete , 0 egli ha fame, o & vuol dormire,

overa Psiche fando in quel Inogo , dove non eva da’ mangiare , ne da
Wa Occafione di sbavigliare non potendo cavarfi la fame,ne la fete,

GLI, Dal Latino articnli . Zampe degli uccelli , o altri animali ditati,

Je mani delle Pate , le quali conuertite in Aquile,havevano artigli in

mani. Se¢ bene diciamo taluolta artigli le mani dell’ huomo. Bocc,Canz.

Go? Amor ys io polfe ufcir de’ tuoi artigh , i

A pena creder poffa ,

Che alcnn altro uncin mai pile mi pigl ,

FINE DEL QVARTO CANTARE;

INTO CANTARE,

 
 

 
  
   
    
 
 
    

 
   

  
    

         
 
  

    

 

Fe
. VERSIE 3
tan’ Conme Coane seiee NS
ie ARGOMENTO, ¢
/ Viol con gl’ incanti dar la eAaga aita ?
F Jn Malmantile al popolo afsediato,
‘an Ma dagli [pirti é cost mal fernita , .
Che trai zimici ¢ il {uo faper beffato ; -e.
me Vien Calagrilio ,e a duellar U inuita y Me
h EP inuito é da lei toffo accettato,
‘il H Fendefi , ¢ altri due com’ é ufanza ,
ie Sparsr di Piaccianteo fan (a pieranza , 2
go CR CIID CR BI CUM DBE
SR Sarath 7s
¢ art
ot STANZA L STANZA IL
ip Gli eftremi non fur mai degni di lode :
Ci vuol la via di merxz0,¢ chi ha ceruelle
Se vere , 0 falfe novitadiesli ode

A crederle al compagno va bel bella:
Le crede , s'ele fom fondate, e fade i
Ma s' elle fhar non poffone a martelfo
Won le gabelia mica di leggieri ,
Come fa il Duca a certi mefsaggieri ,
Ff E Vo-

  
 
    
 

226

Volendo i! Poeta nel prefente Cantare: wince iadia vy
mandati da Martinazza per far diloggiar Baldone’y ¢10
le , per lo quale apparvero a)Baldone diverfamente da quello
che fu caufa , che egli non preftd fede alle loro cmgpcta
Che I’ effer’ huomo teftardo ,.e capone non @ bene ;
bene I effer cosi credulo , che fi dia fede a onan ra che fi fente d
degno di lode colui , che fa pigliare la via del mezzo , dando credito a’ pele
fe , le quali-egli conofce haver fondamento di-veriea y come fece Bald
meflaggieri di Martinazza De ES Ie :
CAPONE, Tellardo. Huomo ofitio nella faa opinione. In ‘Baca
potrebberfi chiamare quelti cali Capirones; da noi alerimenti Capardi ,
TONDO . Huomo groffolano ; femplice , facile, credulo , ec,
4 ai panni lani , che fi dice sends , quando fono groffi , contrario i fini,
diciamo buomo fine , che & il contrario 4’ huome rondo, Laica Novella 2
t0 fu hnomo di si i ereffs pafba , € cost rondo di pelo, che in quattr? anni ai {quola
tette mai imparare t' eAbbicci . Vedi forto C, 6, flan. 80.
MINC HIONE , Semplice . Vedi fopra C, 4. flan. 15.
SE le beve tutte. Crede tutto quello , ch’ ei fente dir .
BAB BV-ASS!, Igooranti , huomini di ceruello groflo’. Vedi frto'!
CREDEREBBON cl? un afin volaffe.. Per e{primer’ uno, » che cred
le cofe impoffibili a crederfi , ci feruiamo di quefto detto. In Empoli |
lenne dell’ anno , fanno una antica fefta , o rapprefentazione di ae
no: Quindi é, che nel Capitolo in lode delp “Ate, che va colle rime:
dice :

  
   
  

 
 

 
 
   
 
        
   
   
   
   
  
    

Ben moftran gli Empolefi aver cervello ,
uanto conkienfi ad ogn' huomo da bene ;
Che ? Afin diventar fanno un uccello ,
NON pué tare a martello. Non cortilponde al vero, Tratto dat Cimento dell?
argento , che quando non fa , ciot non refifte al Martello , ae i e eee
1 Latini pure direbbero in quefto propofito mon ef? aurum ignt prob. 5
NON le gabella. Non le pafla per vere, Non le crede : dal Pelee Ped
Gabeila delle porte , 0 de’ pati ; onde il verbo Gabellare, per ammecterey € a]
vare una cofa per buona , ¢ per vera. <dica particella Heer
enfafi della negativa,come gia,e mai, ec, Lo non vai mai,che fi dica im sis
che fi dica, Io non vud micayche fidica, Vedi fopra C. 4, ftan. 69. ‘

 

 

 

STANZA IIL STANZA v. i

Ma, perche chi m' afcolta intenda bene; Ella ahaa allor, ih deb pers

Tornar a Martinazza. mi bifogna, Chit pigliar Panes

La qual dianzilafciat, levi fovviene , Che per ta pet ce i

Ch’ in ful Caprinfernal, Pigra earegna, Ai farfibravo ye bee

Quel popolaccio ha aggiuntoye lo ritiene Se ben fra tanto: moana

Dal gear via com tantafuavergogna, S* ellacon Gambaforrn,¢ B

Perche quando per lei larafiguray mode

Railenta il corfo , ¢ pifcia la paura ,
 

   
      

QVINTO CANTARE: 227
AL pibieroin: “STANZA VI.
l 0.4 Cid dette balxain cafa, € cold dentro
0 Per aenerfi difpogliafi in capelti ,
_) Ecacciatafi addofso quant’ unguento
Haveva ne! [uci feridi alberetli ,
Vin gran circolo fa nel pavimento ,

   
   
    
    
  
 
 
  
 
 

E con un uafo in man,fcriteiye Cartelli,
Borbortando parole tutravia ,
Che ne men fi direbbono in Turchia ,
STANZA VIL
pari in mezzo al feonoy O colaggiie dal forreraneo Regno

wdo all ordine ogni cof, Cornuti moftri , e gente fpaventofa
ad effetto il [uo difegno Pilizginofi eee Dees
voce firepito/a : Badace a me ; le mie parole udite .
0 a Marcinazza , 1a quale fopra nel C. 3, fan. 76, la(cid, che mon-
cioni in ful Caprone , haveva arrivato quel popolo , che fuggiva per
riconofciutala , la prega a dar’ aiuto a Malmantile , ¢ far , che effi
iano a-combatter,fe fijpud. Bila dice , che Rima neceffario il combat-
che intanto vuol vedere, fe gli riefce cacciar via il nimico per altres
5 ¢ vaflene in cafa a fare i fuoi incantefimi a quefto effetto .
(FERN ALE.Duedizioni.come ridottein una ,fignificante Caprone d'In-
i quel Diavolo in forma divCapra fopr’ al quale era cavalcata,
Za j5e il quale fi favoleggia , che vadano.le Streghe a Benevento ,
fopra C, 3. flan. 69.
Vuol dire Cadavero d’ huomo, o di beftia. Cavalcanti flor,
p. 2. dice ;. Se volere veder quanto la lor per fidia fi diffe/e contro al fan-
macgiori, cercate i Connenti de’Frati , e troveretegli pieni di corpora , e di
vi antichi. Da quefto dire del Cavalcanti m’ indugo a credere, che
na ifichi cadavero d’ huomo ammiazzato con ferite , ¢ ftraziato,
cl ere di tal voce per intendere una beftia piena di mafcalcie , ¢
¢ flinio con Pier Vettori nelle Varic lezioni, che venga da Charonia,
ano gia le voragini del fuoco,, che in diverfe parti del mondo fj tro-
dicevano Charonia da Caronte , perché la fuperftiziofa Gentilica fti-
yche tali vorapini fuflero bocche d’ Inferno , ¢ che per quelle s’ andafle da
E perché hanno fempre puzzo orrendo , che procede da acque (ulfu-*
flo cominciarono a chiamare Charonia tutte quelle cofe , che grande-
vano; E noi feguitando gli antichi diciamo C aragna a tutte le coies ,
wutono , come fanno le beftiaccic guidalefcofe , ¢ le morte. Dicigmo Caro-
un’ huomo , che habbia cattivi fentimenti , perché un’ azione mal fat-
dire 4 putes; onon ha buono odore . , 3
Atenieli chiamayano Charonia quella porta del Pretorio , 0 Palagio del Po-
r ufcivano cojoro, che erano condorti al fupplizio , fecondo
iulio Polluce nell’ Onomattico , ¢ Alex, ab Alex, lib, 4.c. 16. ¢ Cel.
lect. antiq . ¢, 8, ¢ lib, 47. c. 9. Tolta la derivazione di tal vace pures
ate , che conduce J’ anime al hppltsinpatenee in barca, ¢ fi dice mane
2 dar®

 
   
  
 
 
    
   
 
     
  
  
   
  
 
 
 
    
    
 
   
  
  

   

 
 

 
 
   
    
   
    
      
 

228

dar’ uno a Cavonte per intender wala uno alia morte. vile
PISCIA la paxra, Ripiglia animo .. Non ha pit para.
{ no azauffaci fogliono pifciare ; ¢ comunemente dalla plebe fi dice
li paura ; € da quefto diciamo pifciar ta paura 5 quand’ uno -fpaventato
rito,perde quel timore. * Achy
L! ABF-ANNO in fulla pena, Eca aggiunto alla pena , che hebbe per la
) atfanno cagionaro dal correre, Vedi la voce Affanno fopra C.g. ftan, 69.
VEKMENA . Va foil , ¢ giovane ramo d’ una pianta fi dice Vermena
Lavino Himen , Que) patio di Vegezio ; de re militart lib, 1. cap. 11. Quemadns
dut ad feuta viminea , vel ad palos antiqui exercebant tyrones : L' antico vo
tore traduce csi. Come a feudi farti di vermene, 0 paliy tt provavand 4C.
GLI giunta addoffo la piena, Sono accadute loro tutte le maggiori :
piena é prefa nel fenfo detto fopra C, 1. ftan. 84.
eo FAR in mo che non s' habia a metter la {pada al fanco, Far in modd ch
negozio s’ aggiufti , {cnz’ havere ad adoprare le armi , che fi dice Agginftarla
la {pada nel fodero , i
Sé fi puo far ds manco, Se la necefita non forzi a fare in quefta maniera.
GAMB ASTORT A , ¢ Baconero. Nomi di Diavoli inuentati qui dal’ | Poetas }
nelio feflo modo , che inuentati furono i nomi di Barbariccia, € Parfarelte
fimili .
BALZA in cafa, Va velocemente in cafa. Zalcare propriamente fi ¢
faltare , che fa la palla , o pallone perquotendo in terra, Vedi fopra CG;
SPOGLLASLin capelli , Si {poglia ignuda , e {cioglie le trecce dei
vuol intender il Poeta , fe bene fi {crue del detto /pegliar/i im capes che fi
adoperare ogni fuo fapere , ¢ tutta I’ applicazione per fare una tal cofa; per in-
tendere ancora che Martinazza s’era tutta applicata a fat, che Balser per
via d’ incanto diloggi da Malmantile ,
€ACCIANDOS! addoffo,, Metiendofi addoffo, E fe bene il verbo ite uo!’
diy intromettere con violenza , noi lo pigliamo in fenfo di mettere 5 come i vede
nell’ Ottava antecedente cacciar 1a (pada per metter la fpadas
ALBERELLI, Vali diterra , Odi vetro, entro a’ quali fi ccobsladalatan,
guenti , ¢ cole fimili ; ¢ fon forfe quei vafi, che i Latini chiamano alweolt
giiano i] nome da quefti. 2b Pai
BORBOTT ARE, EB’ un certo parlar fra i denti poco intefo da chi I’ Seana
che diciame anche brontolare , E’ il Latino fubmnrmurare. Borboryttein
Greci é il romoréggiare, © mor morare che fanno le budella: Verbi psiiies al rian
fleflo naturale . !
e4 Pit pari, Cio’ a piedi giunti infieme . Quefta voce pari y che per
vuol dire xgwatied di numero , ed il {uo contrario ¢ difpari ( che diciamo fe
i Latini dicono par , © impar , feruc ancora per denotare ugwalita di
corpo » come gui; che s' intende, che un — non era ne pill innanai , er A
indietro dell’ altro Si dice efer pari quando uno $’é vendicato con penn
ha pagato tutto quello che doveva, E ancora + effer pari ¢ gat &
quando non fi pende per neflun verfo . Strada pi ari per. re faut In
wa |’ adoptiamo in tutte quelic cole, dove entri aa Peet

   

    

  

 
 
   
  
   
    
 
   
    

 
 

   
      
  
    
    
      

   
   
 

    
  
   
  

ah
_ STANZA VIIL .
p vi fcongiuro , € vi coman:
Pokies, € virtis ai quefti incanti,

 QVINTO CANTARE:

229

YOST; Affumicati. Tinti da fumo , come fono i cammini , che fon
filiggine , che ¢:compofta di fumo , ¢ d’ umido . Lat. faliginof’.
ATE ame. Attendcte a me; Offeruate le mic parole , ¢ ftate attenti a

STANZA Ix
Per gl’ imbrogli vi chiame,e I inuenzioni,
Che ritrova il Legifta , ed il Notaio,
Quando per pelar meglio i buon pippiont
Gii aggira, che ne anco un’ arcolaio ;
Florsu , pexri di Sacchi di carboni ,

   

|porcheria de’ guardanfanti Per ques ladri del Sarto,e del Adngnaio,
le donne De per cofume , Che ti voglion rubar a tuo difperto ; x
di pulci , ¢ fudiciume . Vycste fuor 7 venite al mio sae .

con diverfi (congiuri chiama gli Spiriti infernali , per (eruirfenes
a far diloggiar Baldone da Malmantile : & I’ Autore moftra il difprezzo, che egli
fa degl’ incantefimi , facendo che Martinazza ¢oftringa i demonj con le cofe ri-
_— ditoleé , che egli mette in quefte due Orcaves «

, SCONGIVRARE. Queito verbo t da noi ufato per inteddere Eforcizzare,ciod

b “il Diavolo per via di giuramenti di formule facre dette per quetto
_ Elorcifmi , cioé (congiuri ; ¢ comunemicnte ¢ prefo in quetto fenfo , ed anche pill
atgamente fi tira , come qui, alla manicra d’ inuocare gli (piritizufata da’Maghi,

fe bene il {uo proprio fignificato é¢ domandare , o chiedere con grande ardenza_,

! edéin to del verbo pregart dicendofi. “i prego , vi fupplico , vi feongin.

ORCHERLA . Si dice non folamente un’ atto fporco , ed illecito, ma ancora
una matetia fchifa , {porca , ¢ brutta , o otal fatta ..Come per efempio : 1/ tale
Seve wis crarione , che rinfed una bella porcheria , La voftra mercanzia non bebbe efite,
perche a una porcheria: I Libri di quel Mercante furono abbruciaci , perche
rat Pieni di parrire falfe , a’ altre porcherie. Varchi nelle Rorie Fiorentine dice :
Era appunto {parfa in Firenze  ufanza a? andar in 2axrera ,  mantello, che era una
Ia por « Quefta voce Porcheria fignificante difprezzo potrebbe venire dal
Latino porcaria , che vuol dir I utero delle Vacche , 0 délle Troie, dopo cht han-
NO partorito, o per dirla colle parole di Plinio lib.11. Cap. 37. Yuluam partu edito,
€tali vulue , particolarmente quaado non avevano condotto il parto , ma fi era~
no feonciate , dagli antichi Romani erano manger per una cofa conser
la Porcaria non la mangiavano tanto voientieri,forfe per efler cofa pid fchifa.Era
chiamata porcaria in un certo modo per difprezzo , € cosi ha portato ay
nol il ignificato , che ritiene di difprezz0 , ed abbominazione . Ma la pil fem-
plice origine € da porco animale immondo ; € cosi deta porcheria , cofa da porci,
Some furfanteria , cofa da furfanci ,¢ fimili .
- GVARDANP ANTE . E? uno ftrumento compolito di cerchi di filo di fetro in
tondo ; il quale portano le donne Spagnuole , © circonda loro fa cintura foto le
Velli , le quali fa gonfiare : E lo dicono guardanfame , perch egli difende dalles
offe I’ infante , cio’ Ja creatura , che hanno le doane prtgne dentro ali’urero,
Perché quefta foggia di veftire , che havevano cominciata ad ufare Ie donne di

. Firenze,

/ 70. Latino obfecro , obre/for .

  
  
 
 
 
 
   
   
   
    
  
Yo!

  
  
 
        

230
Firenze , conofciuta rae
dava a poco a poco difufa

ne il Bando , cioé I’ efilio , ¢ proibizione «

PIPPIONT. Piccioni. S’ intende gente femplice ,
fono i pippioni , colambarum pulli , colombi giovani. £
Cavar danari di mano al corrivo. i
ARCOLAIO , Strumento , fopr’ al quale s’ a
@’ altra materia per incannarle, 0 aggomitolarle co
ed ¢ un moto perpetuo , ¢ perd dice ageira che ne anc!
gira bene , ed aflai: ed aggirare in qucito luogo yuol dire ingannare; ¢
ratore , ingannatore , Coa Bind » fi prende per huomo aggiratore ; ¢
fare per girare, cioé non fi rinuenire col ceruellogL. delirar : \ gg
Ingannare ; Latino circumuenire . : Ns

STANZA X. ; ;
Tutto lt Inferno a cos; gran parele

Vien fibilando , ¢ intorno le faltella

Come dall' alba al tramontar del fole y

Fa quel, ch’ é morfo dalla Tarantella, 1 Com

DIRT ANZA), SL K

Ed a far ch’ei fi pigli quella fpracca Ma perché tu mi voglia far

Senza cagion,gli par ch'ell'abbia iltorto, ‘Di darmi Baconero,
Perché dalla Pro ‘onda fua baracca Perch'jo mi va dell’
A Malmantil non ¢ la via dell orto. 4n cofa che mi preme, ec
Corpo | ( dic’ellayed al C elon l'attacca) Plutone allor quei due fa?
A venir infin qui tu [arai morto} E la frrada fi piglia della p
Ma fenti il mio Pluton , non t adirare, Seguito da i {ugi {udditi ,
Che venir non t’ ho fatto fine quare, Poffon fondar la C ompagnia.
Agli {congiuri di Martinazza le comparifce avanti Plutone con molti!
ed ella gli chiede Baconero, ¢ Gambaftorta , Eile la(cia quivi
nj ,¢con gi altri fe ne torna all’ Inferno . .

SIBILANDO , Soffiando,filchiando. E’ yoce Jatina,che ritiene il fuo.
10. Verg, En. 11, edrrettis horret fquamis , © fibilat ore. Iptendiamo
mente il filchiare de i ferpenti .

SALTELLA . Fa {pelfi, ¢ piccoli falti; & il faltar delle Rane, Vedi
6. ftan. 37,

MORO dalla Tarantella, Per la Calavria , ¢ Puglia dicono fi trovi un
lo ragno detto Tarantola , o Tarantella , il quale nato ex purrs {cappa
fure della terra in tempo di ftate . Geeta mordendo un'huomo,gli mepte ad
una infermita fpecie di rabbia , che Jo forza a ballare contiaovament
vata , al tramontar del fole , ne prova quicte , fe non quando fente
chitarra 0 con altro ftrumento fimile,un’ aria detta percio la Tanane
faono guefto rale attarantato fi affatica a ballare tanto’, che flracco
morto ; € ftato in guefto fuenimento qualche hora,fi rizza , ¢ ccfla,
flando fano per qualche giorno : E perche in quel paefe fi trovano moll

   
 
  

      
    
  
   
 
 
 
 
 
       
  
   
    
      
  
 
   

 
 

    
  
   
   
        
  
   

-QVINTO CANTARE: 231

“Dicono , che tale infermita duri quanto dura la vita di
attarantato , la quale dicono , che non paffi tre anni;
pofta pagati da quei Comuni_, i quali vanao cercando
gli per univerfal benefizio , e ne hanno un tanto
un Rettore a cid deputato . Dicono in oltre, che
ficato provi la detta infermita ogni anno per un méfe poco pill,o
torno a’ quei giorni,ne’ quali fu morficato , che fara intorno al Sol
: trovino di quelli che la provino ogni mefe per qualche giorno,
> 0 Tarantella dalla Citta di Taranto, nel cui territorio forfe
te fi trova . Ti Lalli nell’ En. Tr. lib. 1, ftan, 22. dice,
| Enea quantunque bravo anch’ ei tremante
, Morfo dalla T arantola parea,
GI’ introna la tefta con le ftrida : lo sbalordifce ; lo fa aflordare

‘vecchie. E’ inuecchiato , s’ intende uno che fi tratti da vecchio;

Valo di rame , col quale fi caval’ acqua dai pozzi. Vedi forto

an. 3 . Ba if decto far come le fecchie fenz’ altra aggiunta, fignifica andar in
af come fanno le fecchie infunate nella Carrucola .

Intende abitazione , Che baracca vuol propriamente dire quel

(0 i foldati in campagna per loro abitazione , nel quale 2a

€ capannello di frafche , 0 d’ altro ,col quale fi difendono dal

icque . Viene dal verbo barrare , che vuol dir Circondare , 0 accer-

ice anche frabacca , © corrottamente , © pure ¢0 guid trabibus conftrue

 
    
      
 
   
   

   

  

via dell’ orto, Quelto dettato fignifica ; la via & Junghiffima , e difa-
r ordinario dall’ orto alla cafa non é pil lungo viaggio , che ca~
ri delia porta , la quale di cafa efce nell’ orto, effendo per lo
a Citta gli orti appiceati alle cafe.
PO! ed al Celon ? attacca, Vuol dir Corpo del Cielo. Si dice Corpo del
a del diavolo , ec, Ma quando uno pafia pib ia beftemmiando les
: Bil ateacca al Celone per intendere 5 egli entra nel Ciclo , cio’
jumi Celefti; EB per render pit ofcuro quefto detto,ci feruiamo del-
ne , che wuol dir quel panno, che fi mette fopr’ alla tavola da meafa
di diftenderui fopra 1a tovaglia. .
. Detto ironico per moftrar la poca ftima, che fi fa della Fatica,
durata uno a noftro pro , ¢d il poco grado , che gli fen’ habbia,mafii«
tale ne fa grande oftentazionc. _
: + Voci latine ufate nel fuo fignificato ; ¢ dicefi: Won fine qua-
3, ¢ fignifica non fenza qualche fine , o cagione, Franco Sacch,
de Gli venne vogla d’ andar a strovare il Re Adovardo , ¢ won fine quare , perché

tree molto lodar|o .

IN fondar la Compagnia de’ brutti . Sono'tutti bruttiffimi. Habbiamo in.
tun’ Accademia , o Compagnia detta de’ Brutti , 1a quale fi raguna ogni
giorno di Befana ( che cosi fi dice il giorno dell’ Epifania ) ed in un lau-

tulimo,

 
 
  
 
 

  
   

 
 
  

 
 

  
     
  
 
 
 
 
 
 
 
 
 
 
 
   
   
  
   
  
    
  
  
  
    
  
  
    
    
    

age MALMANTILB, 5
tiffimo , e ftravagante fimpofio fi crea il |
iftim 8: maf Confininanmmnt

la il Fondatore ,¢ fi fa {empre il pil brutto .
Poeta. ;
STANZA XIII

Lafcian Plutone,e corroz dalla Druda
J dug {pirti , afpettando il fuo decreto ,
Ed ella allor che fa da Ceccofuda
Per far si che Baldon dia volta a dreto,
Ed anche fe fi puo ch’ ei vada a Luda,
Gli prega,che le dien qualche fegreto
Da far Jenz’ altre guerre, ovver conte/e,
Che quelle genti sfrattino il pacfe.

STANZA X vo

Pers fe non finghiam ch’ egli le feriva
Ch'il fuo rivale (adeffo ch'egli ha intefo
Chrei #¢ partite) con la gente arriva ,
Per volergliela (u leyar di pefa,

E che fe propria é ver , che per lei viva Hor dunque tu che [ei faputa
( Com’ ei /peffo giurd) d’ amore accefo, Che non (a cedi manco a Ci
E feglit cara lo dimoffri, e prenda, Scrivi la carta , che tu fai ch
Ed armi,e braviye corra,e la difenda, Sian tutti un monte d'a

I Diayoli rrovano I’ inyenzione di far diloggiar Baldone da Mal
guefta € fargli intendere , che la Geva fua dama ¢ in pericolo d’ efler
cono a Martinazza , che {Criva la lettera . i

DRVDA, Innamorata , amante , ec, fe bene non fempre fi piglia
to difoneftofo ; Qui intende dama di Plytone , che era Martinazga , che » some
firega , haveva Jui per innamorato . AY

FA da Cecco fuda , S’ affanna,s’ affatica . Scherza con quefto nome:
perché quand’ uno s’ affatica , ¢ s’ affanna (enza propofito, moftrando d
cofe diclamo: “tale /uda. Di quefta natura era quel Cortigiano defcritt
Berni nelle Rime. Ser Cecco non puo far fenza la Corte, Ne la Corte puo iar
Ser Cecco, : a oa

VADA4« Buda, Vada via per non tornar pi. Proverbio nato dalla guerra
che gia fece il Turco contro Lodovico Re d’ V ria quando acquifto Bud
ca l’anno 1626. , che vi morirono quaii cutti li Criftiant , che yi andaronos ¢
il medefimo Re ; E perd da quel tempo in qua dicendofi : // rale ¢ andato 4 Bu
s’ intende é andato via per non ritornar pil , 0 vero € morto , ed ha il mede
fenfo,e per la medefima cagione ; // rale ¢ andato a fcio. F andato a Patraffoj{chet-
zo fulla Citta d’ Acaia famofa per tl martirio di $. Andrea , come {¢ fi dicefleua
Latino : ivir Patras ; ¢ fulla frale uiata dalla {crittura , fopra quei che muolon0 »
¢ fi feppellifcono , quafi dica ; E’ andato ad patres /uos é
SFRATTINO , 0 sbrattino il paefe. Ripulifcano il paefe , cioé (ene vadanO.
DAR’ a due tavole'a un tratto, Far due negozzi in uno fte(io tempo. Tratt0
dal giuoco di sbaraglino , nel quale con un fol tro , fi dia duc , ¢ tre tavole, 0
girelle. Si dice ie : far’ wn Viaggio , ¢ due fernizj . Vedi {orto C, 6, flan. ae .

  

   
  

     
  
   

 

  
 

 
 
   
  
    
   
  

QVINTO CANTARE, 233
a in due faffe . Attendere a due partiti . Vitumeligere, G alte-
eaiasicinn o. eae ahbe dirfi Mont’ Vghi dalla fa-

illaggio vicino a Firenze .

antichifima Fiorentina . Ricordano Malefpini nella Stor. Fio-
l no ebbe nome V 60 5 quefti anche fue nobilifimo
no sei ql dicefona gli Vghi, ¢ per innanzi il poggio ,
‘fi chiama Montughi , s’é chiamato per loro... Lo fteflo conferma Gio,
RE
7 balan. Allora allora; Subito fubito... Vulla interpofita morula .
INE , Specie di Serpe , detto cosi, perché forfe vada yeloce comes
facta , ¢ credo tuber dei Latini,
y 4A . Propriamente yuol dire Remiganti di galera : Ma qui @ prefa per
, come fi trova anche prefa in pib Storie Fiorentine antiche , ¢ fopras
76.5 ¢ fora C. 11. ftan. 76. dal Latino tarma , fe bene propriamente fi
di foldati a cavailo .
OL ammazzar beftie , ¢ perfone . Vuol difertare il paefe . Quando vogliamo
ler uno , che vanti di voler far gran brayure , ¢ non lo giudichiamo atto a
me Veruna , diciamo Vol ammazzar be/tie,¢ perfone. Ed in tal {enfo di derifio-
‘DEE preio nel prefente luogo . I! Berni nelle rime congiunfe quefte due voci cu-
allor che diffe : Con wn mondo di bestie, ¢ di perfane ,
faputa . Sci dotca ; {ei {Cientifica. Donna fapura , facciura , faccente vuol
di Vna donna , che in tutte le cofe vuol far da macftra . Colla fteffa figura di
fe faccente , diceli e4uuertito , edccorto ,edunifato , ¢ dagli antichi Senrito
ee che avverta , ¢ che s’ accorga delle cofe , ¢ che ftia full’ avvilo, ¢
t, ll participio paffivo in forza di attivo .
| SLAMO una mana a afini , e di buoi , Siamo tanti ignoranti. Per lo pit a que-
fle dug ie ed al Caftrone affomigliamo coloro , che non hanno {cienza alcu-
ha. S¢ bene  Autore fapeva , che il Demonio poffiede tutte le fcienze ( che cosh
a uo Greco nome Daemon , cioé fapiente ; ¢ noi d’ uno che fappia eccel-
a che cofa dichiamo ; Egli ¢ un Demonio ; nondimeno ha voluto,
efi due Diavoli fi dichiarino ignoranti, accid che fi creda. pid facilmence
Vertore , che fecero di {cambiare le palle , come vedremo .
STANZA XVII, STANZA XVIIL
Non ti dé contre atifpond' ella , a quefto , E per dar al negoxio pitt colore ,
Ed che voi vi conofchiate : In forma voglio wr’ io d’ una gomare
| Her si, dice i Demonia , ferivi prefta Della {ua Geva detta Monafiore
parole in tal genere agginftate Confidente del Duca in ogni affare;
’ a; ma vedi, io mi prote(to, Gambafforta verra da fernitore ,
6 portai mai lettere,o imbafciate, Che moftri di venirmi a accompagnare,
Scriniforgiunge queische quato al porta E gid per quefto ho fatte far di cera
ito qui con Gambaftorta . Due pallennach’e bianca,e Laltra nera,

 
    
   
  
 
 
 
  
   
   
  
  
  

——

SS ———

= TaAsS ASD

   
 

s
#
fi
¥
8
i]
¥
4

Eccoms

 

Gg STAN.

 
 

 

  

" i
234 MALMANTILE >
STANZA XIX. shoo eg rah
Quand’ un tien quefia neva in una brica, La nera a lui dare c
Di fubito a un’ huom prende figura; 1

E s'ei vi chinde quellaltra ch’ bianca,
dn femmina fi muta, e trasfigura . v
Si che riguarda ben 8 altro ci mance , La Strega qui gli dice,
E diftendi mai pit quefha feriteura ; Perch ella ferive,e guafpoleha
Chil mio compagno,ed io qua per viaggio Ma lo [cancella, ¢ mettelo in |
Ci marerem Leffigte, ¢ il perfonaggio . ° Cost fie" la carta, ela figilla

STANZA XkKI. van
Le fa la foprafcritra , ¢ poi finifce
A pit a’ un ghirigere ta propria mano,
E con eff quel diavolo /pedifce

   
  

  
  
 
 
       
  
 

    

 
 
 
 

no la medefima Jettera per portarla un di loro trasformato in Mona
1 alco ia un Seruo per via di due palle , e fe ne vanno cosi da Baldo:
havere feambiate le palle , chi dovea apparire la Fiore, appare il
rono {coperti. :
NON portai mai lettere , 0 imbafciate, La maggiore offela , che fi
certe donnicciuole ,¢ il dir loro ; porta lettere ; porta imbafciate ; fa feruig:
polli ( detto credo io dal Franzefe Puuler , che fignifica Jetterino @’ amy
portatrice di lettere amorofe ) perché yuol dire Rufhiana: B perd
Martinazza , che non yuole queft’ offefa addoflo fi dichiara , che non (
portar lettere , 0 ambafciate , cioé da far la ruffiana . vb
ECCOMI lefto, Eccomi pronto: Eccomi all’ ordine. Zefo in ‘ f
vaol dir difinuolto , e fenza imbarazzi . oth SBE!
DAR colore.al negorio, Bar’ apparic = vero quel che é incerto ; Da :
fimilitudine . Quetto fanno appreffo i Rettorici quei , che da loro foro
Colori, Givvenale dice: dic, Quinttiliane , colorem , + yo
COM ARE. Quella che tiene la creawura'al Battefimo . E qui il Pocta
il coftume , che in fimili amori per Jo pit la Balia , la Comare fono ne »
portano le parole, yagi
4ONA. E parola fincopata da Madonna , ed é il titolo che fi da comunt-
mente alle donne d’ infima plebe dicendofi in diminuzione Signora » Madonna»
Monna , come Signore , Meffere, Sere. Ma perché Afonna oltce al fignificato dl
Bertuccia , ha ancora altro fignificato ofceno (almeno in lingua Veneziana) no!
per sfuggir I’ equivoco ; hoggi coftumiamo dire Agora enon Monona:
ALAT pit, Hormai , Cioé finifeila una volta: E’ termine dimoftrative @
certa impazienza , ¢ fi dice : Ob mai piis : ed: il latino eandem aliquande:¢
confa con I imperativo s dicendoGi : Ob mai pis: finitela, OE RB S|
POSTILLA, Nel noftro idioma ha diverfi fignificati; perch? 6 vaol }
( figuratamente fecondo Dante ) immagine d’ un’ oggetto, che ritorni alla
weduta da un yetro, o dall’ acqua chiara , Dan, Par. C. 30,

  
      
       
 
      
 

           
  
    
   
  
   

  
 
        
     

">
 

=

<= SSaeu FE BS

a

QVINTO CANTARE:

 

 
 

Ls 235
Quali per-verri trafparent?, e terff,
Over per acque nitide , ¢ tranquil ,
Wem Wardens « Won si profonde , ch’ i fondi fien perfi';
6 Lernan de noftri vifi le poftiile
Pgs » Debili st, che perla in bianca fronte
at _ Won vien men tofto alle noftre pupille .

O viiol dire annotazioni , 0 glofa, che i Latini dicono expo/tio. O fi piglia per

aggiunta, ed.¢ compofta di due dizioni po? , & ila. Quali dica,

Poftilla verba, cio dopo quelle parole , {crivi , o aggiungi quefto , ¢ quefto, E

annotazioni , glofe , o aggiunte hoggi per /ofilla intendiamo anche»

del libro , cioé quel bianco che fi la(cia di fotto , ¢ di fopra , ¢ dalles

foglio {crivendo , o flampando: Si che /crivere in poffilla yuol dire {cri-

Margine ;¢ s’ intende ogni aggiunta , che fi faccia al tefto {critto ,o

in qualfivoglia Iuogo della carta © fia di {otto , 0 di fopra , 0 dalle ban-

idei verfi ordinati , ¢ regolati; ed in quefto modo , © luogo , dice ches
"Marti

—. E’un tratteggio di penna ufato per Io pitt nelle fopra(crittes

» come moftra i] Poeta nel prefente luogo , che faccia Martinazza..

Ghirigoro da’ noftri antichi era detto in volgare il nome Latino di Gregorio ;
Ghirigoro trovafi fempre coftantemente fcritto nel Malefpini , ¢ nel

Villani ; come era 1a lingua di quel tempo . Ma qui Ghirigoro apparilce per av-

Ventura dal girare , ¢ rigirare della penna cosi detto.. E le parole /n propria mano

8 ufago nelle foprafericte di quelle lettcre , Je quali fi mandano a yno , che fia ne!
luogo , o Citta , 0 vero poco lontano da colui che fcrive .

 
    
 
 

el.

STANZA XXII
Che Baconera ii quale ¢ un' avventato ,
Neb dar la palla alt’ altro di nafcofto,
“Senza grardarla prima havea [cabiato,
a: fattoun grad’ arrofto,
i quand’ a Baldone egli ¢ arrivata
Dice cafe dal ver troppo difcofto,
i afferma d'effer dina,e stbra
Huamo alla barba all abito,e alle mebra,
STANZA XXIIL

ECambaflorca anch'ei balordo, e frolto,

Mencr’apparir ficrede un’ hue dabbene,
Alla Favela, alla prefenza , ¢ al volta
Per Vna fa fernizrj ognun 1a tiene,

» Afoglio intantoil Duca havea lor tolto,
Eveduto lo feritta , ¢ quel contiene ;
Refha certo di quanto eraindovine ,
Ch i furbi vorrian farlo Calandrino,

 

STANZA XXIV.

E poiché gli hanno detto , che la Geva 5
A lui gli manda con quel foglioa pofta
Ma prima che da loro lo riceva
Hann! ordine d’ haverae 1a rifpofta;
E foggiunto y che mentr’ ella {crived,
Getrava gocciolon di quefta pofta ,
Per il trabufograndech'cllahahavuto,
Come potra fentir dal contenuto ,

STANZA XXV,

Egli ¢ (dic? egli ) um gran parabolano y
Chi dice ch’ ell’ ba fcritto la prefente,
Quand'ella no piglio mai penna in mata,
E fo di certo ch' ella n’ ¢ innocente
Che poi tu fia la Geva , ch’ in Venane
A me fu molto nota , ¢ confidente ,

E tu fia buom , 4 dirla in cofcienza,
AA me non pare y¢ nego confeguenza,

Gg STAN:
 
  
   

236 MALMANTILE —
STANZA XKVL»~

1 non compagni a una rifpofta tale - Ed alle

Guardanft in vifo,inquelfendofiaccorti,

Chregli hanno equivocato,e fatto male;

Refhan quivi allibbiti , ¢ merzi morti , ‘Di Baldone , e di tutta

Giunti quei Diavoli da Baldone , credendofi rapprefensare uno
V' altro il Seruo , non effendo accorti d’ havere {cambiate le palle,fe
ambafciata: Ma Baldone , comprefo, che quefta era una furberia, non
cid, quanto dall’ effergli noto , che la Geva non fapeva ferivere 5 fe gli cl
nanzi con una gran quantita di filchiate . ocx prcee
tVENT ATO, Vno che opera fenza confiderazione,e furiofamente.
mo fd 3 € precipitofo ; dal ivo Latino ad. in fi
to d’ avvenirfi , cioé imbatterfi in una cola con velocita »e con furia,
DI nafcofte , B \o feo , che Di foppiatto detto fopra C, 1. flan. 75.
PIGLLAR un eros « Pigliare errore;Intender una cofa per un’ altra, §
pigtiare un granchio a fecco quando uno nel picchiar qualche materialesh !
fi batte i] marteilo fopr’ alle dita, o fi (erra le dita fra due materiali
ecrore intendiamo poi far un’errore, quando diciamo pigtiare un gr Beral +
Che Virgilio ha prefo Vn granciporro, "phase :

FAR un arrofto. Far un’ errore. E' lo fteffo che pigliar un granchio, vad
per avventura dal verbo arrofarfi , che vuol dir affaticarfi {propofitatamente
furiofamente ; ¢ le cofe fatte in furia non fi fanno mai bene , a

BALORODO, ¢ folto, Sinonimi che fignificano Huomo fenza giudi a vO~
ce ftolto é pura latina , e balordo ¢ lo fteflo che in Lat, bardus . aeons )

VNAF A fernixxj . Come's'é detto fopra s’ intende una Rufiana, ©

VOG LION farlo Calandrino , Calandrino , fecondo che dice il
fue Novelle , fu un’ huomo tanto credulo , che gli fu dato ad intendere fino, che
egli era pregno , ¢ perd da coftui diciamo 7% mi vnoi far Calandrino per intend —
re Tu mi vuoi far credere quel che io fo, che non é vero, Si dice anche far
pedino , da uno de’ noftri tempi della natura di Calandrino

HANNO ordine @ haverne 1a rifpofta , 1 Poeta per maggiormente a
caftronaggine di coftoro , fa che chieggano la rifpofta prima di p a

ropotta . ¢

CETTAVA ‘coccioloni di queffa pofta, Lagrimava gagliardamente . Il termine —
Di quefta pofta ignitica grofiezza ; erano pere ds quefta pofta , cioe pere grodidi-
me ,¢ fi ja: » che colui , il quale dice cos, , accompagni il parlare col geft
delic mani dimoftrante la groffezza di quella tal cofa’, Si dice anche samo fatte;
tanto groffe , come vedremo fotto C. 10, ftan. 17. 18. ¢ 36.

TRAMBVSTO, Travaglio , rimefcolamento, follevamento
fa di difgrazie. 3 : aig

PARABOLANO , Bugiardo ; chiacchierone ; [propofitato: Da Parabola, cide
fimilitudiae,o Racconto; ne’Capitoli di Carlo i] Caluo fi legge . Par: itty
fimul , & confiderauernnt . Parlarono infieme , Du Frefne alla V. Parabola,

SO ch’ ella n’ 2 innocente . Intende ; io fo ch’ ella non fa fcrivere. Per efprit
re uno che non babbia ne pure una miaima notizia d’ una tal cola

  
 
   
 

   

  
 
  

 
 
 
 
 
     
 
  
   
    
  
  

   
  
  
     

   
  

   
 

d’animo per eae

        
   
  
 
   

QVINTO ‘CANTARE:

alcxno nella tal cofa 0 ¢ innocente della tal cofa:

go il tutto : perché negando la confé apn viene a
4 e tutto I arguimento , € cosi tutto il dif

un fubito timore 50 wergognie ,e¢ percid
‘solore fmorto y¢ gialliccio , come , feccandofi , diventano le potatu-
che fi chiamano éibbie,dalla qual voce viene alubbitoye altibbire, Ve~
rio della Cra/ca alla Voce “Alibbire + IL Varchi Stor, Fior. lib, 10.
ova il quale incontinente ( quafi le fuffe venuta meno la terra

237

 
 
   
     
 
      
    
 
    
  

: ie,
ATA. Romore di yoci , filchi , urli , battimenti di mani, ¢d’altro ,
uno per dargli la burla, Far le fifchiate.a uno , quel che i Lat.

STANZA XXIX.
Che la padrona il tutto le comparte
Come s'in Malmantil fien due Regine,
 Psiche egnor di attorno , Anzi il bando fi manda da {ua parte ,
ggni quattro palfi fa un Jamento. Perch’ ella foffia il nafo alle galline 5
to tutto quanto il giorno » Cosi poi c bebbe dato libro ,¢ carte ,
cento volte ,¢ cento Entra nell' un vie un, che non ha fine y
Malmantile, ¢ Similmente Coftui,che qusvi s' ¢ poffo a bottega
tinaza ,e fev'é ds prefente. A legger fopra il libro della Strega,
TANZA XXVIIL STANZA XXX,
xn, ch’ al fin la mette per la via Queft altro , che non cerca da coftui
sche queft orrida Befana , Di quefti cingue foldi , havendo fretta,
un.toz%0 haveva,careftia Poichegli ha ee quel che fa per luiy
sitet erba porcedana y Spronailcavalle tutto aun tepo,eshietta,
di.gran foldi.in [ua balia , La donna che trovare il {uo ,colui
una cefacome una Degana , Digiorno in giorno per tal mezro afpetta,
Corteeingrado,e giunta a fegno, Per no loperder adocchio,ech'ei le machi,
i totum continens del Regno , Segue la frarna,egli nasipresi ifianchi,
| Poeta a parlar di Calagrillo ; il quale camminando Psiche s’ imbatte»
he le da avvilo di dove fia ‘Martinazza .
MARCHIARE , Si dice marciare , che vuol dir camminare . Voce Francele ,
ma gia fatta Italiana. Vedi fopra C. 1, flan. 43. EB pit accofto alla pronunzia,
-Oltramontana , dicefi anche Aerciare , forfe da Marcia , contrada , pace , cam-
‘Mino dane/marce diffe il Villani la Danimarca; cioé Danefe Contrada ,
‘ANA. Intendiamo Donna brutta , malfatta. Vedi forto C. 8. ftan, 30.
C.9, fan. 1.
T0ZZO . S intende pezzo di pane. Haver careftia d' un toxxo, Vuol dire ef-
mendico , pezzente .
ST AVA come la porceliana . Ciot terra terra , come l’ erba porcellana, che fer-
2 per terra, € non alza mai virgulti ; detta porcellana dal Latino Portulaca.
quefto detto fignifica Vno che fia in povero ftato , ¢ non habbia modo di falle-
-Varfi, che i ooo pure dicevano humi sacere .
sea balia. In (ao potere,e dominio . Balia ¢ voce fatta venire dal Mroh-

  
 

   
 
    

 

       
 
      
     
    
  

*
BS
5
€

 
 

  
   

   

238 MALMANTILE™)
ni dalla Greca Buleia, che fuona lo fteffo che Bule ; cic

Senato. A noi fuona Potefta,giurifdizione,autorita,e quel che i Latini
poreflas imperium, Dan. Purg, C. 1. bt

 
    
    
     
   

Che pure
Petr.C.36. Afentre ch’ il corpe é vivo y i
Hai tu il freno in balia de’ penfier tuoi
HA una cafa come una Dogana, Ciot picna di robe , come fono le Dog e
ne di mercanzic. set
IL Bando va da parte fua, Ciok , ella comanda’. 7A oP
SOFFLA it nafo alle galline . Ella fa tutte le faccende. E quefi tre
Totum continens de/ Regno;Rando va da parte fuaje fofiail nafoallegallineh
lo fteffo fignificato;ma di quetto ci feruiamo per lo pili per derifione,per in
uno che habbia ambizione d’efler creduto gran miniftro,ed habbiaim
neggi d'un governo,e non fia vero ; che per ifcherzo direbbefi anche
En.Tr.l.4.ft.15.SopratrnrtoaGinnon,che del far ragza E’detta Varcifanfana,
DAR libro ye carte. Dare fata notizia d’alcuno, ‘Viene da’coloro,i
havendo debito co’ Magiftrati , fon mandati in efazione a i Miniftri forenfi
wali Miniftri i Magiftrati mandano il contrafegno del libro, nel quale €
il debito di quel tale , il nome , ¢ cafato di eflo, !'origine , ¢ fomma delde
ed a quante carte ¢ Ja ua partita: E quefto fi dice dar libro, e carte,
in proverbio , fignifica Dar notizia chiara, ed efatta d’ alcuno; 0
habbia fatta un’ azione per altro occulta . 4
ENTRA nell’ un vie uno, Faun difcorfo da non ulcirne mai, cor
be fe uno voleffe feguitare Vn vie uno fa uno, due vit due fa quattra y ec, che sande-
rebbe nell’ infinito . Dice il Varchi nel fuo Ercolano , che in quefto fenforfidice
Cantar la canzone del? uccelline’, Con tal dettato s* elprime un chiacchierone'y
che cicalando , faccia ‘molte digreffioni {propofitate per allungare il fuo cicala~
mento con racconti affai {Conuenevoli, che fi dice; Entrare in un ginepraio
re di palo in frafca, elk
S* £ meffo a bortega , $'& prefo per arte, per fuo mefticro , o negozio. Quan
do uno fa qualche operazione con tutta applicazione, ed attenzione, ¢ con dime
ftrazione di voler durare affai , diciamo; Coftui sé meffo a bottega , oh
LEGGER ful libro d aicuno . Narrar le azioni, qualita , ¢ ftato d’aleuno,
NON cerca quefti cinque foldi, Non cerca , non gl’ importa, non proccuras
fapere quefta cola. Quand’ altri fa un difcorfo , ¢ fa una digreflione fenza tornat”
piu al primo propofito , fe lidice : Voi pagherete la pena de’ cingue foldi . Vedi fot-
to C, 8. flan. 15. E pero dicendo : Non cerco que/ti cingue foldi , § intende;non mi
curo di oes gnefta pena de’ cingue foldi , con obligarti a feguitare il prin>
cipiato difcorto . ai
SBIETT-A. Scappa via prefto , Vedi fotto C. 7. flan, $7, 4
IL fuo colui . Ui fao amante , cioé Cupido . Se
PER non Ws perder a’ occhio, Perché non le efca di vifta, Per non Jo fmarrite. —
SEGVIT A la flarna, Quand’ uno feguita un’ altro per haver da lui qualches
favore , diciamo: Zifeguita la ftarna. E Gi dice la farna , © non altro uccell0s

   
    
 
 

    
   
 
  
   
      
   
   
  
 
     
 
   
  
   

 
 

ie

SEPRTA TES ESS

SS “eS wt a

SAS

a

3. flan.s, Franzele .

 
 

paletettrorcs:

ne ilo
4 STANZA XXXL
— were i

ipeepet dot ons 3e ar fide,

‘inttorna fon piie delle pecchie;
eh foldayed a Suis faite udati ,
Che havido del guerrier noticievecchie,
 Gliva incontro, Vaccoglie ye riverifce ,
t a luicon DY arms 8 offerifce .
TANZA XXXil.
i, fazginnfe, ch? io si preghé
P donna rimaner fernito ,
a quefto ferro lei stimpieghi
| Per conto qua Paces — j
t tanto Cavalier nulla fi nieghi,

~Rifponde acio Baldon tutto complito,
* Tu fei padrone ; fa cio che tu vuoi y
Nom ci van cirimonie fra di noi,
Sita he
Over cl? io me La metta in [ul liuto ,
“Ot veglia tener Poche in paftura ,
"Come quel che ci vada ritenuto

Per mancanza di cuore, v\per paura,

    
 

ss QVAN TO CANTARE:
Q feguitarle,ofleruandole dove fi pofano,e ftraccanioie

239

STANZA XXXIIL
Ti ferusro di feriverti alla banca ,
E in tanto per adeffo io ti ceafegne
M1 gonfalon di queftaciarpa bianca ,
Che tra te {chiere? il noftro cétraffegns:
Tal-che libero il paffo , ¢ feala franca
Haurai per dar’ effetto al tno difegno;
Che non fo qual fi fia , ne lo domando ;
Pero va pur ch'io refto al tuo comande,
STANZA XXXIV,

Ei lo'ringrazia, E ito pit da preja,
Ove fia chinfo di Psiche il bel Sole ,
Ad effa dice:In quanto al tuo intereffo,
Fin qui non t'bo feruito,e me ne duole,
Che tu non penfi, bavendati prome(fos
Ch’ io facciafango delle mie parole ,

E ch'il mioindagio , eilnorifoluer nulla
Sia fatoun voler darti erba trafpulla,
XXXV.

Perché fi come haurai date vedite ’
Won ho fin qui trovata congiuntura
Di chi m’ indirizzaffe qua al Caftello,
Per porerne cavar cappa , 0 mantelle .

~ Jo-con Psiche arriva al Campo , ¢ chiede foldo: Baldone I’ accetta , ¢
nza‘d’ ancare a feruir Psiche , con la quale avviandofi verfo Malmanti.

da,
£ Calagrillo'fi {cufa di non’ haver pri

‘ima feruita .

alla banca, Atrolare uno per foldato: Banca diciamo quel luago
ee foldati , e dove fon loro pagatii denari ae fipend} :

7 (LONE. Vuol propriamente dire vefillo ; ma
@infegna , Vediril Voilio de vitijs fermonis lib, 1. ove di qui

& piglia per ogni forta
la voce.

CLARPA , E’ una legaccia di drappo, che dai foldan fi'cinge come Ja cintu-
‘ra della fpada.. E per altro ciarpa vuol dire quel che accennammo fopra Cans.

Efcharpe.

SCALA franca. Franchigia ; Liberta d’ andare,‘o fare. Paffo libero .

I ston fango delle fue parole. Dilprezzare Ja parola data ¢ non offeruar Ie pro-

DAR erba'eraftulla, AMdetterla ful linto ; miandar voche in paffara hanno tutti tre
verbie dare,

Jo Reffo fignific

ato, che € trattener’ uno con Soa. Lat.
$1

NZA XX

Rifponde Priche a quefta'diceria:

| Lo non entro Signore in quefti meriti y
Non ho parlato mai , ne che ti fia 2
So o fpediro, o ver che tm ti peritis

 

Quel ke tu (fii, tutte tua cortefia,y
Per tal? accettose’l Ciel te lo rimeriti,
‘Co darti invita honor,fama,ericchezza,
Sanita dopo morte 5 ed allegrenza.
STAN-
Se

240 MALMANTILE.
STANZA XXXVIL STANZA XX¥XVHE
Sta quieta , le dic egli , e ti conforta , Van eee @! occhiacci orlati di favore ‘a
Chia vegtia adeffa dar fuoco al ve/paio, 03 addolfa & unt eratte ght [qnaderna

Cost col Corno, il quale Al colle porta’, Che par quand’ il Faina alle fe bare
lanternay.

Chiama la guardia, 0 vero it portinaio, In facia mi {palanca la i
Non ¢ si prefeo il gatto in fu la porta y E mediante.un certo pirxz \ ee
Quand es fente la voce del beccaio; Chrei fente al colo, pixzicotti alterna,
Quanto veloce a quefto fuon la Ronda Ond? alle dita egli ha farsi i digali
Sopr’ alle mura accoftafi alla {ponda D’ imorno a innumerabili mortali, —

Psiche rende grazie a Calagrillo della carita , che Je promerte, ¢ facendo le lor
cirimonie , s' accoftano al Caftella , dove Calagrilio , (onande il Corna, chiama
la fentinella , la quale fubito s’affaccia alle (ponde delle mura, oy

DICERIA , Vuol dire Ragionamento , Difcorfo, Orazione : ma
voce é ufata per lo pid per intendere Ragionamento flugchevole , € odi perla
lunghezza , io

NON entro in quefti meriti . Non parlo di quefte cole. Ma queflo detto has
wna certa forza d’ e(primere : io.non ardifco d'entrar tanto in 1a col difcorfo;mae
niera, che viene dal folerfi dire; il merito della lite , 0 della caufa, cio¢ limpor-
tanza del fatto . re

SANIT A , ed allegrezza dopo morte, E detto giocofo , perché un corpo mor
to non pud haver fanita , ne allegrezza ,ne altre paflioni. Ma fi potrebbe anche
dire , che quefta donna , parlando iperbolico , voglia dire che eglt viva fano, ed
allegro fempre eziam dopo morte , il che ¢ impofhibile , come ¢ impofibile viver
mill’ anni , € pure fi dice: vi prego mille anni di vita, Sanied-é un* augurio sche
corrifponde at Greco hygiainein , ciot far fano , che metteva innanai alle {ue ¢pi-
flole Pittagora devotifimo della {anita ; degrezza corrifponde a quel fal
in principio efpri i Greci co} nelle lor | sperché dove i La
tini pongono Salutem dicit , effi {crivevano Chairein , cioe come tradufle Orazio
in una fua Epiftola Gaxdere , yolendo dire , Ii tale,al tale defidera allegrexea fic
come in quell’ altro modo ufato da Pittagora : il tale al tale,defidera fanita.

DAR es al a 3 Violentare a ulcir fuora uno , che fia dentro ; come {e+
gue , quando fi da fuoco'a un velpaio , che le velpe fon forzate dal tyaco a {eap-
par fuori. Vedi Omero lib, 16, dell’ Iliade , }

LA voce del Beccaio ,. Vanno per Firenze alcuni Beccai, 0 Macellari yendendo
carne per dare a’ gatti , ¢ fanno certe lor voci cosi ben conofcimee da i ot
gatti , foliti havere la carne , che appena coftoro hanno aperta la bocca , che i
gacti fono in fulla porta, A quefti gatti afomiglia la guardia di Malmantile,che
a pena fentito il fuono del corno s’ aftaccia alla muraglia, Delle voci , ¢ de'verli x
che fanno j venditori, che vanno attorno per inuitare i] compratore, Senecacp. 14
56. lam libarij varias exclamationes ,@& botularinm , & eruftularium »@ omnes pope M
narum infeitores , mercem {ua quadam ,& infignita modulatione vendentes $ 4),

t
”

   

SE

  
  

= 2.
seg

 

£ ia eS ezZenre eee

RONDA. Sidice quel Soldato di guardia , che rigita , e pafleggi Ia)
raglia della fortezza , vifitando la Sentinella , detca wa isi andsee incall ee
come i Franzefi dicono , aller en rond , stat

SPONDA, Parapetio delia muraglia ; Quel pezzo di muro, che avanza alle

 
 
   
 

Ss SSREREE BS TELA SSA TAs SAT. SS ae

 QVINTO CANTARE: 2gt

    
   
  
   
   
  

z del terrapieno , ¢ fi dice /ponda quel muretto , 9 fpalice-
ilterreno , a i pozzi , a’ fiumt , ec.

f di favure’, Circondati di cifpa per la fimilitudine,che ha con la cifpa
co; E/avore ¢ uno intingolo fatto di noci , ¢ pane pefto, ¢ liquefatto
+ sec eSnpemeef quell’ umor craffo , che fi conden(a intorno alle pal-
i sli occhi .

a Eicaaepccinns gli occhi addoffo. Subito fifla fopra di lui gli occhi
« Equelto verbo /quadernare s'ula per divolgare , maniichare , ec.

pats Lv 33+
plires,: Cio che per  nniverfo fi fgnaderna

WN, Celebre Luogotenente di Birri cosi chiamato per foprannome ¢
ILANCARE. Aprir quanto fi pud una porta, un’ armario , ¢ fimili : le-
aca , cioé i] palo, che tienc in alcune porte fermato tutta, o unay
slia porta ; aprire affatto. Vedi fotto C. 6. ftan. 43.
ZILOTTO, #’ uno ftringimento , che fi fa in qualche parte del corpo ,
sliando la pelle col dito indice , ¢ ftringendola’co) dito pollice ; ¢ cosi facevas

intorno al collo , a/ternando i pizeicorti , cioe facendoli hor con |’ una , hor

/mano per pigliare i pidocchi, che {ono queghi iznumerabili mortali, che

ue loro gli hanno fatts i ditali , cioe ricoperte le dita; Che ditale inten-
diamo parte del guanto , che cuopre il dito .
— STANZA XXXIX. STANZA XXXX,

  

Non tanto s* abburatta per la rogna , Bu bis, bu by comincia , ch’ il buon ciorno
(Epe brnfeol , che vanno alla goletta, Vorrebbe dar al Cavalier , ch’ ei tiene

n dir non pud quel che bifogna Ii Corrier , mediante il fuon del Corno y

line feiligua ache abacchetta, Del popol d'Israel chor va , hor viene ;

Qual ib quartuccio le bruciate fogna , Van le parole a balzi , ¢ per ifforno
Nefenza quattro fcofe altrui le getta, Prima cal fegno voglian colpir bene ;
Talfi dibarte ,¢ a vite fa la gola Pur pinfe tanto, che gli venne detto;

volta ch! ei manda fuor parela, Buon di Corrier che nuovac'é diGhetito;

(ctive i] Poeta Ja guardia , la quale havendo creduto che Calagrillo fuffes
un’ ', lo faluta come tale .

S'2BBVRATT A. Si dimena: Si dibatte . Abburattare propriamente vuol
dire Separare la farina dalla cru(ca con lo ftacciv.
BRYSCOLI che vanno alla goletta, Intende i pidocchi , che vanno alla pola. ;
Goletta intendiamo Veftremita dell’ abito da Huomo intorno a!la gola . Ed il Poe-
ta copre a detto con I' equivoco di Goerta , fortezza in Barberia , e con las
Voce bru/coli , che fono minutiffime particelle di legno , o paglia, o fimili, ed egli
TART AGLLARE , Intoppare nel profferir le parole; pronunziar con difficul-
i: ¢ /eilingware vuol dir Balbettace . . :
4 BACCHEITA, Ci dare a bacch vuol dire C dare affolut
Mente ¢ difpoticamente in ogni congiuntura , come Re , o Capitano, che porti
{cettro , mazza , 0 Raftone a wf 3 ¢ di qui defi,che coftui eile "
¢ (cilinguava ogni lettera .

LFARTYCCIO, Milura Fiorentina capace della arg aacaner partes

D flaio , ¢ per lo pid ¢ un vafo di legno, BRP.

 

 
 

5s eee

 

   
  
 
 
    
   
   
   
   
     
     
  
  
  
  
 
    

242 MALMANTILE) |

BRVCIATE, Marroni cotti arrofto in padella 5-0 in forn , 0 forto la
FOGNARE , Fogna vuol dire quel vacuo fatto ad arte | :
paiia I’ acqua , ¢ fi conduce feolando al fiume dal Lat, fovea:
mifwra vuol dir wetter la roba nella mifura in maniera , che
ma dentro vi fieno molti vacui , come facilmente fegae nel ia,
quale non fi pofiono flivare i marroni , i quali per cher di figura rotonda non,
ricmpiono lo fpazio , ma fanno nacuraimeate , che rimangano fra I'und, ¢!'al-
tro molti vacui nella mifura ; la quale poi, volendoli votare , io fquo-
tere ; perche s' aftrontano nell’ ulcire , € foqquadrano alla bocca. del quartuccio
in maniera , che non potriano (cappar fuort , fe non Gi fquoteffe il vafo,ed ufcen.
do , fanno un romore fimile a uno che tartagli , le di cui parole pare , che non,
potiano ufcir di bocca , fe egit non fi (quote , dibatte , o Sores equ aae
Jo che egii mette fra uaa parola , ¢ I’ altra lo figura il vacuo che fla fra un
rone,¢l altro. E quctto intende col dire qual il quartuccio le bruciate fogna’, ciok
fogaa Je parole con interuallo di tempo , ¢ non di luogo.. Daa
EAR la gola a vite. Storcer la gola. Vedi lopra C, 2, flan. 9. atlas
PER frorno, Si dice quel ritornare indietro , che fa la palla che ha
nella parte oppofta dove ¢ fata rata o fia muro, © fia altro, ed termine,
prio del ginoco delle pallottole , ¢ s’ intende quand! yno tira per accoftarfial fe
gno per via di detto florno , ¢ non direttamente: E cost indirettamente
di bocca a coftui le parole . Ln fomma vuol dire , che egli impuncava nel parla-
re, tartagliava , ¢ parlava a falti. valighelitya
GHETTO.. Cosi chiamiamo il Serraglio , nel quale flanno in Firenze, edin
altre Citta gli Ebrei: E perch¢ quefti hanno nome di. tener di mano afregheric,
pero dice che il Corriere di quel iuogo ¢ folito Ipetio andare a Malmantile a to-
var la flregha Martinazza.. Ghetto ¢ voce Caldea , che fignifica libello di
dia ; onde noi diciamo Gherro per intender luogo di gente fegregata , ~~.
dal commercio degli altri huomini. Gli Ebrei quando vogliono dire loro
mogli , che le gaftigheranao col repudiarle dicono.; Ti manderé al Gher. a
STANZA XXXXL STANZA XXXKIL
Rifpofe t altro, tal parola udita + 41a che vo il tempo qui buttande vid
D’ elfer corriere gid negar. non poffey In difputar con matti., econ buffoni
Perch’ io Pho corfa afar quefta falitay 4 trattar teco credomi che fia ~~
Ata quato al Ghetto ia ndlavoglioaddoffo; Come a’ Birri contar le fue ragioni;

  
  
    
   

     
  
     
      

Non ho che far con-gente Ifraelita.; We diffi mal , perch hai fifonomia
Benti fara il mio brando ilcappel roffay D' un di color , che cinffan pe’ caltonk y
Ecol darti [ul vifo un foprammano El’ cffer tu coftt , par chvella quadriy

D Ebreo fara mutarts in Siciliano Ch’ i Birri fempre van dove fon ladri.
STANZA XXXXIIL Joa
Dell! alma fala quei:fon foddisfatti'y
Aa, v0i col corpa la portate wa,
Hor baftasfe bs voi tant’ odio corres
difporre.

  

Bench? voi fiste come cani ,egatti ,
Ch'effi non han com, voi gran fimpatia y
Perché peggia de’ dtavol fete fatti,
Vande nel pighar pis tiraunis. 5 Mezlio,a-i.lor danni ti:posras
  

i eh he ke le SE had

}

 

eae tn i

Sessa BPres &
>

243
STANZA XXXXV.
La frefto-devi oprar 5 &° a tei fia farto ;

et cui(perch ei confente in tal baratro
ag porrebbe far le fufa torte;
@i fi cerca efstr mandato Hn tratto
Salt’ afin con due rocche dalla Corte ,
Si che, [tu nol fai, ts rapprefento ,
C” un difordine qui ne puo far cento,

XXXVI.
t Mentre pero Cupido non rimetta :
non impiccate quefta Troia , Ma fe la rende non vi do pik nota,
vK Pigharmi quefta derta Va ditone,enarra a lei quanto tho detto,
i Birro,e in fulle forche il Boia, Chri gui rattendo,e la rifpofta afpetto,

Sadira Calagrillo,che colui I’habbia prefo in cambio del Corriere degli Ebrei,
¢lo minaccia di rompergli ia velta , ¢ sfrepiarlo; ¢ dopo havergli detto molt im.

Oper) » gli ordina , che da fua parte avvifi Martinazza , che renda Cupido ; al-
titi fara render per forza.

LHO corfa .. Ho fatca quefta cofa fenza confiderazione. Quand? altri fa qual-
che rifoluzione , che non riefce poi buona , di¢iamo : E# P ba corfa dall’ armeg~
gist, e'dalcorrere Ja gioflra . Similmente diciamo ; Fare una carriera. Qui fa
giuoco la voce corfa , che ¢ cofa da Corrieri . si
NON Ja veglio aitdoffo . Non la voglio fopportare , Si dice anche non /a veglio in

Uraelita , Intende Ebrei: Popolo d’ Ifrael ,
AL cappello rofo. Gli Ebrei in Firenze portano per contraffegno il Cappello
toflo, Ii Poeta dice; fard ben’ io diventare Ebreo te col farti il cappello roflo col
— E poi d’ Ebreo ti fard diventar Siciliano tagliandoti il vifo , ed intende
quel Siciliano Montambanco, che per accreditare il {uo Olio da Ferite fi facevas
agg perfona , e con effo fe le medicava .
AUMANO . Quel colpo , che fi da con {pada , o baftone , comincian-
do da alto, ¢ calando a batlo., Vedi forto C, ro. ftan. 52, :
SVEPONE » Vino che'fa profeffione di trattener la brigata’con facezie .
DIR Me fue ragioni a Biri, Raccomandarfia chi non pwd’, € non vuol far fer-
vizio’; anzitha caro’il tuo male. Vuol’ anche dire difcorref con und, che nons
a ‘ta dica;o vero buttar le parole al vento, Plauto difle nel Pfeudolo;
spud novercam queri , ‘ . t
CHEFAN pecalzont, Ciov i Bitri ; i quali pigliano pe’ calzoni. [1 verbo
cisfare hha del furbe(co ; €*vuol dir Pigliar con prefa Mabile , ¢ buona, come &
feleste ate, pigliando uno per il ctuffo , cioe pe''capelli. Petrarca’, Le san
‘4vef? jo avoolte entro a’ capegli , fn
“ESSER come vani ,egatts , Eller poco d’ accordo , 0 poco uniti , anzi fempres’
fhimi¢i,come naturalmente fonoi cani , ¢ i gatti . 2
NON ha gran fimpatia. La voce fimpathia Greca fatta Tofcana fignifica incli:
hazione fcambievole , o fimilitudine di _- » di voleri , ¢ d’ aftetti,
. amity 5: 5 :

eu AE.

Mentr'a tofitinonrendailfnoCoforte .

  
 
   
   
   
   
   
 
     
  

 
 
   
 

 
 

  
  
 
   
  
   
     

ve
244 MALMANTILE | —

MAESTRO Baftiano , Intende il Boia , che allora cos
cra ftato Mieftro Biagino . Vedi forto C, 6. tan. 56.
LETTO a tre colonne, Cioé le forche,le quali veramente fon
una flanga fopra a traverfo, ed in molti luoghi fono, ey :
LAVORAR di mano, Vuol dir rubare . {cherza dicendo , «
( cio€ il Boia ) perché effi ricevano qualche ripofo da tanto lavo!
gil mette in ful letto.a tre colonne pee in (ulle forche ) ed in fuftan t
gl’ impicca , perché fon ladri, E Calagrillo , feguitando I’ equivoco del ri
dice aila guardia , che f¢ ella ha punto di pieta,¢ difcrezione, dovrebbe
fto ripofo in ful letto di tre colonne a Martinazza per il fuo tanto
impiccarla , perché¢ ladra, I Latini pure per dir copertameate
manu finifira uti {econdo Catullo in Afinium . ’ sey
Marrucine Afini , manu finiftra
Non belle uteris in ioco y atque vino;
Tollis lintea negligentiorum .
E per dire eopertamente Impiccar’uno,dicevano ; diteram longam.
bsamo notato aitrove .
NON cede un grano: Non cede punto . Che grano fi pud dire ana
inconfiderabile del pelo, poiché 24. grani fanno un danaro , 24.
V oncia ,e 12. once fanno la libbra.
NON uccella a pifpole. Non fi cura di confeguir cofe di poco mon
é fra gli uccelii la pifpola. 1 Latini diflero vom capeat mujcas .
FAR le fufa sorte, Far le corna. Vuol dir quand’ una donna fi Of
altri huomini , che col fuo marito. I Burchiello Poeta capricciofo , VAs
fotto nome d’ Accademico Fiorentino incerto, nella Raccolta delle Rime Piace-
voli del Berni , Cafa , ec, i .
Non ti fidar di femmina , ch’ é ufa si ‘
e4 far le fufa torte al fuo.marito ,
Il Berni nel fuo primo capitolo dell’ orto dice :
E finalmente non fara mai fufa
Donna alcuna per lui torte al marito ; Ky
Si dice fu/a torte per intender copertamente Corna . ;
MANDATO con due rocche in full’ afino. EB coftume in Firenze, al
dclitio del pigliar pid d’ una moglic,aggiugnere una dimoftrazione
che ¢ il far’ andar’ per Ja Citta il delinquente legato fopra ad un’ afino’,
mitra di foglio in capo y ed a-cintola due , 0 pill rocche inconocchiate, che
ficano le due, o pili mogli. ses eeh
, QUEST A troia, Quelta porca . Epitcto vituperofifiimo nelle donnes
ynoldire Laida meretrice:: nell’ huomo non € tanto ingiuriofo i). dirgli
Ml x0 pighar quefta detta, Vud pigliarmi V’aflunto di far quefta
derra vuol dire prometter per un’ aitro , o ftar mallevadore , cioe di far’
cofa , fenon la fara quello , che  priacipalmente obbligato . Comprar.wna 4
yuo] dir comprar un’ avviamento , un credito , ec, Derta & dai plura
Devitt»

 
      
    
      
    
  
   
   
   
  
   
 
      
    
    

fucere

 

\

   
  
 

  
 
 
 
  
 
 
 
    
  
   
   
 
   
 

245
STANZA XXXXVIIL
Lafcia la fentinella , e caracolla
Gin pel caftello, dando quefte nuova
E benche il Adaggioringo della bulla
Gi babbi eff ,r ech'ei fi mova
: ‘Di fargii porre a' piedi la cipolia,
r non poffa dele pacche ; Cercando della morte in bella prova
to havendo si Ciel turbato Vuol avvifar di cio Mona Cofofiola,
¢i par un porcellin grattato, Ch’ é per bafire a quefta battifoffola ,
» che ¢un vero poltrone , fentendo le bravate ai Calagrillo , zitto
/¢ tremando va a dare quefta nuova a Martinazza .
RDA P armi dalle tacche .. Non vuol cavar fuori la {pada, per non la,
:. Intendi che coftui era un codardo , perché per dir copertamente pol-
un foldato , fe gli dice: Rifpiarma foderi .
i Facche . Parole latine corrotte’, ¢ ridotte in una , ufaro affai dalla
- plebe nte per intendere Andare in faluo,ed é il Latino 4d asylum confugere,
Rl LEV-AR delle pacche .. Bulcare , 0 toccar delle ferite , che quefto intendiamo
hey ma ¢detto plebeo. Li Vocabolifia Bolognefe dice che pane fignifica per-
agliarda.. La forza di quefto verbo rilevare vedemmo fopra C. 3. ftan. 67.

hi ftor, Fiorent. lib.6, dice 4/ figlixole del quale nominato Lorenzo,rilevo unas
3

   
   
    
    
      
  

IDO veduto il Ciel turbato, Havendo conofciuto , che coftui era in col-
lice’ anche /4 marina turba ,
iB che pare un porcellin grattato. Similitudine affai ufata per intender uno ,
| nrifpenda alle grida d’ un’ altro o per paura , 0 per riverenza , o per lao
cofcienga macchiata »© per altro; ¢ fi fa la comparazione al porco, perché il
Porto che firide,grattandolo fi quieta , ed i porcai gli rendono maneggiabili col
Peg
OLLA, Il verbo caracolfare yuol propriamente dire Volteggiare col
na non ofante qui torna aflai bene per efprimere , che coftui per las
laffle girando per il caflello , non gli parendo trovare luogo ficuro. B
anche in ufo caracol/are per camminare a piede,volteggiando d’ una ftrada in
altta ,¢ diciamo far un caracel(o per intendere una girata. Viene dalla voce
[ Spagnuola caracol » che ynol dire chiocciola , ‘
ene deta bolla, Termine della lingua furbefca 5 che in Firenze vuol
il Fifcale 3 ma s’ intende per il Superiore in quegli affari di che fi tratta, Va-
> il Maggiore delia Citta , chiamata in quella lingua Bolla dal Greco Polis es
arbaricamente, Polla,
| FARGLI mettere a’ piedi (a cipolla ,, Fargli troncar la tela, ¢ mettergliela a
i: come fi coftuma in Firenze quando , il cadavero del giuftiziato dee Mares
elpofto per qualche ora al pubblico ; che gli mettono la tefta.a i piedi .

PER bafire. Bi per tranfire 5 per (ucnirGi 5 per morirfi. Vedi fopra Cant, 2,
NA Cofofiola, Nome ufato pet intender una donna faccendiera , aftanno-
Ofudatora . Scbbene Ce/ofio/a [ fecondo ul Varchi nei fuo dircojano all. voce

4 : Batti.

    
    
   
   
  

|

 
         
  
 

Rartifoffiola 1810 Reflo che batrifofiola , ¢ fignificand affanno >)
mento grande , ma breve , che cagioni battimento di cuore, 0
il che fi dice fofhare: Franco Sacc, Nov. 44. Ad’ bai data eos),
io non [ard mai pit litro', @ forfe me ne morro . Non credo che fit
guello che dictamo fepraffalro at exore ; lo fteflo che batticudre y affanno
to per paura,o dolore improvvifo dagli Spagauoli detto,/obre/

| 246 MALMANTILE™
i
Corn, Tacito lib. 5. dice : Exterrite unt acri magis qudm dinenrno rimore, Bail n0-

i ftro Davanzati parafrafando quefte parole dice bebbero batrifoffia, woth

| STANZA IL. STANZ Ay Be

I Ella infieme le {chiere ha gia ridotte Atentre del farto poi le da context’ ake

! Di genti,che non vaglionoun piftacchio, Com quell'ambafciaye ling ua di frullone
Cive di quelle , a cus fece la notte Fa ( perche nulia mai fora Fe

} Col [uo carve si grande {pauracchio, Chi lo fente morir di paffione;
Ed hor quivi parare , e dar le botre Ma quellasc'a fenrirlo> forfe avvertAy
Infegna lor , che non ne fan biracchio Li intende un porcost i/ is é
Ma quand innanRi a lei coftni, ififerma Equi fnifoom le legion di wera y |
Cos} tremante,la cave di fcherma , Perch’ella ni'da-pin:ne inCielneinverra,

Martinazza {tava appunto inftruendo quej foldati , che s evan -faggiti per pa
ra de’ fuoi Caproni , quando arrivd un Ja fentinella con Panbotetara a Gale
grillo, che la tarbd cutta’, ond’ ella Ja(cid flare il darlezione, si
NON vagliono un piffacchio. Non fon buoni a nulla. Si dice un piftacchio yun
lupino , una lifca ; una forba , una lappola’, un pelo yun baiocco, wa baa,
un picciolo , un zero , nn’ ctte , un fico , cica.y un iota, una chiarabaldana, ui
puntal di ftringha’,o'd’ aghetto , una fucciola, un foldo:,.an quaterino , un-cor-
no ; tutti per e(primer'la poca flima;che fi facia d’ uno , 0 d’ alcuna cola, Eft
dice anche: non lo ftimo 1) cavolo a merenda, Latino cicwm, titiviliitium ~~
SPAVRACC HIV, Significa quel che accennammo fopra C. pr. ftan, 40, Bab
li fi dice fare /pauracchio a uno per intendere {paventar uno , o mettergli pauras
con fatti, 0 con parole. ,
NOW ne fan biracchio, Nomnefanno nulla. Si dice anche ftraccio , brand, 0
brandello , e fimili , id
! CAV-ARE un di (cherma , Vuol dire far perder il filo del difcorfo a uno, ed &
| Jo fteflo che cavar di'tema.. Ma qui: vuol dir’ anche far lafciare far di-
re, ¢ torna bene , perché Martinazza.la(cio la {cherma', ed ufci di tema, ¢'
propofito per I’ ira,che'le cagiono  ambatciata fattale in‘nome di Calagrillo.
eAMBASCIA, Attanno, o difficile re(pirazione d' alito, Fran. Sace, N.139
T ofto colui di-chi erano frati., (en' ando con l' ambafcia della morte a ripigliarl . oh
LINGY-A di Frulione , Cioe che parla a {alti , 0 a intoppi’y comeeil rumore}
che fa il frullone , che ¢ quell’ ordingo, col quale; per via d’ una»ruota:
fepara la farina dalla cru{ca dsmaxy
NON raccaperza nulla, Non intende nulla. Vedi fotto C, 6, flan rot.”
LP INTENDE per difcrezione, Quando per’ altro ci & noto un-negozio , ¢' che
taluno ce Jo racconti confufamente , o lo {criva con cattivi, ¢ non intelligibili‘ea-
ratteri , fentito,o letto da noi , fogliamo dire ; 2 habbiamo inze/o per diferexionts
cio¢ habbiamo havuto Ja difcrezione di non gli far ripecere il difcorfo cacy

=

a

 

ftreeceée hag feet

we ei

   

2A# FEZ

 
   
     
   
 
     
     
   
   
    
 
    
   

fe”

&

= Sas * BS

id

A= 2% BE

SF

 -

* mente dalla vergogna , la quale perd fi dice anche erubefcenza .

— PE He .

  
 

247
quel fat-

 
 

er qualche: informazione , che havevamo di
orfo, © {critto:. i
,we im terra; E? fuori di fe , Non fa quel che ella fi faccia .
piel 5 we terra ; diflero anche 1 Greci in quefto propofito ; ¢ I’ ula Lu-
Plendamante , ovogliam dire F aifo indovino ,
ANZA LL STANZA LIL
wedefi cambiare Rabbiofa, it capo verfa il crel tentenna,
Quafi col piede il pavimento sfonda ,
tor figratrale chsappe,jvor la corenna,
Hor dice al meffaggiero che rifponda ,
Hor larichiama métr'egli in Chiarcna,
Grida,¢ minaccia,e par che ficunfond ,
Hille difegni tro al penfier racchinde
Lenne inne ye nulla mai conchiude,
LULL.
Che lavandole i! collo lordo,e intrifa
Laghi formano in fen di poxxi neri;
el fin tornara in fe,con la gonnella
Yi come fonagli da (parnicri , S' afcinga,e al meffaggier cosi favella ,
Narra gli accidenti , ed i moti diverfi cagionat: in Martinazza dall’ amba{cia-
tadi | ed in fine Martinazza s' accinge a dar la rifpofta. L’ Autores

 
 

 
 

  

deferive.. per-una folenne {gualdrina poiché dice, che & cost grande il
‘udigitme che-clia ha-addotio., che le Jagrime che Je cafcano dagli occhi fanno
)arerle nel collo tanti laghi di pozzi neri , cioé di cedi , i quali ella s' alciuga

  
 
  
 
 
 

 
 
  

BIANCA come il mia collare, Diventa bianca comie un panno curato , E que-
te mutazioni di colore fon proprie d’ uo che habbia I’ animo alterato si in ma-
€, comein bene , perché la palidezza, ¢ sbiancamento deaota follevamento d’a-
dimo-non effendo altro-, che un mancamento di (angue , il quale per la paura fe
ne fugge al cuore , ¢ lafcia le vene del voito ; ed il roflo denota ira percht quefta

tibollimento di fangue intorno al cuore y che {corre per tucce le venc ,
Ma apparifce pik nella faccia , perché quivi (ono molte vene intercucance , 0 vo-
gliamo dire im pelle ,.che faciimente lo {cuoprono ; ¢€ lo: ftefloeffetto viene pari-

   
  
 
 
 
   
   
  

DOPO ch cgit ha toccata una fpogliaxza. Dopo che egli é ftato fiuftato in ful
© ...,dal maeftro. Spogliazza quali expoliatio , {pogliagione fi dice quando il
Macftro fa cavare i calzoni a uno (colare, ¢ mettendolo fopr’ alle palle d’ un’
altro , gli dd con Ja sferza in fulc..... E quando gli da nella (tela forma , ma,
fenza i mandar gill i calzoni fi dice dare una mula, o un cavallo. A quefto
&,.,. fruftato affomiglia |'Aurore il vifo di Martinazza quando le diventa rotio,
Vna fimile {pogliazza , quafi come a ragazzo infolente , © minacciata Ja nel fe-
dell’ Liiade a quel brutto moftaccio di Terfite , a cui Omero [ fecondo la,
one Latina ad verbum del Gifanio } fa dire da Vlifle: Ne pofthac Viyfi
apne bumeris adfit, Gc, Si non ezo te comprehenfum, & charts veflibus exutum Paltio-
que, © tunica , que pudenda contegunt., Flenvem veloces ad naves dimifero, Cadens ¢
Concioue duris verberibus. TEN-

     
 
     
     

  

 
 

 
 

f

eae

EE ————— ==

—ooeee

  

 
 

248 MALMANTILE! 79

TENTENNA il capo innerfo i! Cielo, Dimena la tefta verlo il
§ fa da molti quando accade Joro cofa di-poco gufto , quafi voglia
il Cielo perché cagiona loro quella tal di(grazia : i Latiabdiflero; caput 9
SEONDA il pavimento col piede. Per la collora batte i piedi in terra
mente , che fa quafi rovinare il Palco. Properzio. Et «repitwm dubi
ede. tf
ST gratta le chiappe , e la cotenna, Si gratta Je natiche , il capo chee
to folito farfi per lo pili dalle d6ne quando fuccede loro qualche difgrazia,Pei
vas’ intende il capo , perché Ja pelle del capo-dell’ huomo fi dice cotenna;
vuol dire 1a pelle del porco , ed impropriamente fi dice la pelle d’ ini
vedi fopra C. 2. ftan. 64. ed in cid noi ci conformiamo co’ Latini,che cuit
ta pelle del capo dell’ huomo, ¢ dicono anche cutem detrabere per {corticare qual-
fivoglia pelle , il proprio vocabolo della quale é pellis . amd Ae
QLVAND?' egit ¢ in chrarenna , Quand’ egii ¢ molto lontano.. Zp oras “
¢ da quetto noi diciamo : Quand’ ezli ¢ in orinci . Viato dal Davanzati nel .
J ENNE inne, Di quefto termine ci {eruiamo per efprimere uno che L
di operare,e non conchiuda . Viene da quello ftento che fanno i ragazzi r
imparano a compitare ; quafi dica compita compita , e mai rileva., ed halo fle
fo fignificato , ¢ forza che ponza ponza detto fopra C, 4. ftan. 80, sea
SON*GLI da fparuieri , Intende lagrime grofie come fono i fonagli, che s'ap-
piccano a i picdi degli fparuieri;comparazione ipcrbolica,ma aflai ‘inten
der grofle lagrime.Ain.1 1.41 lacrymans.guttisghumettat gradibus acon
nels chiamiamo quelle gallozzole , che fa I’ acqua quando pioye, cadendo fopra»
i rigagnogli ; o altrimenti neilo fcorrere. , ‘
2022/1 NER/. Bottini . Quei luoghi forterranei , entro a’ quali Anraes
forta d’ jmmondizia ; ma propriamente pozzo wero é bottino , 0 fogna
del ceflo , a differenza di quella degli acquai , i
STANZA LIV, STANZA LV. >
Torna ,¢ rifpondi a queffo Scaizagatto y Pero s in quefto mentre umor non varia y
Che fi crede ingoiar con le parole : Domani al far del di facciami mottos

  
    
 
 
 

   
  
 
   
   
 

    

      
   

  
   
  
 
  
    
     
   

   

  
   

       
 
 
 
 
  
  

Ch'io no fo quel ch'ei dica,e s'eglié matto E s'io gli faro dar le gambe all! aria
Won ci poffo far' altro ye mene duole, Quella fua landraba da pagar lofeette,
Poi circa alla domanda, ch’ egti ha fatto; Mia fe la forte fofse a me contraria
Che gli daro Cuptdo,e cid ch’ ¢ vnole, Viol ’a me tocchia adar col capo roti,
Se con la {pada in mano,o ver co laa #renda Cupido allor, ch io le prometto
Prima di guadagnario,il cor gli bafta, Lafciarglielo fegnato, e benedetto, —
STANZA LVL
Cid detto partese quei chieralbnomo e/perto Ed in vifo vedendolo fcoperta 5
( Efendo lato Cavailaro,e Atefo ) Ruch’ ba bifegno dice d'un buon lett y
el Cavaliere ad unguem fa il referto Perch'egli é duro,e non punto pupilles
Di quel che Martinazzagh ba comeff[o, Lo conofco bensi, gli¢ Calagrillas —

Martinazza manda a dire a Calagrillo , che gli dara Cupido , s'¢i lo gl
gnera con! armi ; ma fe ella vince , vuol Psiche : la ronda porta I’ ambaleiata,®
riconofce Calagrillo

SCALZ 4GATTO, Huomo vile, Guidone.
 

 
  
 
  
 
 
  
 
  
   
  
  
 
 
  
 
   
  
 
   
   

TO CANTARE. $40
ral eeenesircl neha con Ie chiacchiere’. E fi dice:

teu varia Se fra tanto non fi muta d' opinione .
ina, donna di bordello, ed intende Psiche ; Landra é epi-
pinfami, ¢ Jaide meretrici , quali datrina , che la fogna,

ro. Hada pagare la pena. Pagar lo fcotto vuol dire pagar
sé mangiato , pagar la fua porzione, la fua quota ; Tercnzio
+ Ma qui intende il Latino penas mere, Dan, Purg. C. 30.
» L) alto fato di Dio farebbe rotto
Se Lece fi paffaffe , e tal vivanda
Fuffe guftara fenx’ alcuno {cotta
pentimento , che lagrime panda ,
é « Andar con la peggio ; cioé ch’ io perdeffi il duello ;
ATO,e Back. Liberamente,¢ {enz'eccezione alcuna.Fran.Sacc. Nou.
e ogni hora pur Segnato , ¢ benedetto, Efprime ua dar via qualcofa , o
uno volenticri ,¢ con anime di non rivolerio; Vn licenziare af-
Peecceit vale ,ingust Iola,

. BE’ un famiglio , che porta Je citazioni criminali mandate da
s Trent Cavallaro , pecché flante il largo dominio, ¢ giuri{di-
il {uo tribunale , ¢ necetiario che vada a cavallo ; 4 Ade/s0 € quello
eitazioni i pure de i Miniftri forenfi , ¢ fa i gravamenti , ec. es
lo, perché non gli occorrono lunghe gite , come al Cavallaro ; a
panda Cxrfore; nome fimile al Viator,col quale era dilegnato dagli an-
iil donzello,o fante pubblico .
hem» Per appunto. Frafe latina ufata affai da noi.
“CARA cater. Riferilce . Frafe curiale , che vuol dire quando i] Cavallaro , o

i data Ja citazione, riferifce in atti d’ haverla data, che dicono an-
rapporto. Ev Autore fi ferue ee frale ( per altro non ufata in,
i) perché ha detto, che quefta Guardia era ftato Cavallaro,e Meffo,
tbifogno d'un buon lefso , E’ carne dura, e pero ha bifogno di bollires
7 Sore paaglonds EB detto vulgato per efprimere un’ huomo , che {a il conto fuo,
ye difficile a fuperarfi , che diciamo : O/so date per efempio; Il
“cies @ rodere un’ offo duro .
+ Non ha bifogno di Tutori, fiona lo fteflo che ha bifagno a un
brian é pupille fi rifttinge a faper tare i fatti {uoi , ed ha bi/orno
oe e(prime faper fare i fatti fuoi , ed efler bravo,e valcate in ogni

    
  
  
   
  
   
   
  

‘=
:

STANZA LVIL
wi tie dame Calagrille vefti , Che feguitaron come voi inten defi
giorno rivedremg li poi. Periun , che (en' ando pe’ fatti fuoi ,
ekeongeon apprefti Che trovereme ti , fe venir valere

Per ginger it 7 Fendefi gti altri duoi, Pin presto afsai di quel che vi credete,

oH STAN.

ee <

Digitized

 

  
 

 

 
 
 
 
   
    
    
 
 
  
  
 
 
        
   
   
 

”

250 MALMANTILE.
STANZA LVIIL
Che zio cio fe ne vanno git pel piano
Shattuti com? io diffi dala fame ;
114 non fon iti ancoraun trar di mano A ‘per foddisfar s.
Che fenton raxzolar fra certo firame;  —_—-Hla fatto in quattro di Pil

  

    

Percio con b armi fubito alla mano El: con la fua {pada se

Corron dicendo: Qui c'e del beftiame, Delt’ honor della quale

Si che quando crediamo ditirar minze, Che havendola fancinila

Ji cor pu forfe caverem di grinze, Non gli par ben ch’ ienuda

STANZA LIX. STANZA LX

Curtofi quei che fue di vedere Ata perché un huom pix vsl mas,

Dentr’ a una fialla inabitata entraro y Si pente effer’entrato in talc

E vedder,ch'e: a un'huom poftoagiacere Pero che a fearui folo egti ha,

Sopr’ alia paglia a guifa di fomaro; Che non lo porti via la Trentan

Accanto havea da mangiare ,¢ bere y E perché tutto il giorno quant'es

E gli occhi diftiliava in pranto amaro, Egli ha il mal della lupa,che

E trai difgufti,e il vin ch’ era quifite _ Non va mai fuor #4 cinta

        
     
 
 
 

Pareva in vifo un gambero arrofpico, DL? afcioluar col fuo fiafco nell
STANZA LXIL

Ovungne elie, d'untumi fa un bagordo, Aggira il beccafico ,¢ pela il
Ch’ ognor la gola gli fa lappe lappe ; E a poveri cappon ruba le
Strega le botti di lor fangue ingordo , E prega il Cielche fac

 
    
    
  

E le fuftanze ufurpa delle pappe ; Quant le melagrane,
L’ Autore torna a parlare di Perlone , ¢ degli altri , che Jaicio fopra fi
28. , i quali per la fame s’ andavano ailontanando dal Campo ,e¢ narra 5
floro trovarono in una Capanna quel Piaccianteo , che fu da Bertinellan
fuori a {piare , come vedemmo fopra C, 3. ftan. 45. il quale haveva feco
giare,e da bere . Nella prefente Ottava 62.de(crive aflai vagamente la)
nia di Piaccianteo , 3 ere
G/0' cio. Adagio adagio . B' la figura aphere/is . nf ist
RAZZOLARE , Fregare , rafpare , fragare ; ec. Qui vuol dir quel romor
che fa la paglia , 0 cofa fimile , quando ¢ maneggiata in mafia . ae
STKAME, Paglia , fino, 0 aitra materia fimile per cibo delle beftie +
fopra C, 4. ftan. 2. a
TIRAR minxe . Vuol dite ftentare . Ma s* intende moriré : Si dice milzi
il Poeta fi ferue della licenza , ¢ feguita intanto i pil che dicono ; minga
milzas
C AVARE il corpo di grinze . Mangiare aflai , che in quefta maniera gonb
il ventre s fi levano le grina¢ al corpo. Plauto diffe ventrem diffendere
Georg: difténdunt netare cellas , cioe empiono « Paes
PAREV-A un gambero arroftito, Bra rofio in vifo come fono i
ae aflai ufata per efprimere un roffo in vilo, per il
evuto,
ALA fatto Fillidé mia, Ha finito , ha confumato , o mandato male tutto |
havere. E’ detto fanadatsico Filide per fine , Ma per avventura hala fa 0

   
   
  
 
   
   
      
  
 
 
  

    

bands 2
Lia

© eae anges
 

 
    
    

_ -QVINTO CANTARE, 251

figliuola di Licuego Re de i Traci, 1a quale s! innamoré di Demo-

di Tefeo, ¢ di Fedra , quando nel tornare dalla guerra di Pera
) ftato {pinto da i-venti contrarj nel Regno di Tracia, fu da Fillide rice.
on fegni di grande amorevolezza; ma egli fenza riguardo a i benefizzi das
efla ricevuti , fen’ andd ; per lo che Fillide difperata s’ impicco . Da quefta difpe-
‘rata morte di Fillide , quando diciamo far Fidiide, intendiamo finir la vita, ¢ fini-
re!

   
  
   
   

. - ¥
_ IMPLATT ATO, Nacoto’, Vedi fopra C. 2-fan. 60.
DELL! honor detla quale ba gelofia. Ha gelofia dell’ honor della fua fpada, per:
he havendola tenuta fempre fanciulla , cioe vergine ( che s’ intende now mai
perata ) ftima poco honeflo il lafciarla vedere ignuda , come é veramente po-
@una vergine Jafciarfi vedere ignuda . E con tali {cherzi vuol dire, che
codardo , ¢ vile ,¢ di poco animo , ed uno di coloro che wmbram fam

‘ANC ANNA, Vna beltia ch’ ingoia o tracanna trenta per volta_;
/é una di queile Jarue immaginarie inuentate dalle Balie per far paura a i bam-
‘a come bau , befana, ¢ fimili dette al trove .

_ dL male della Lupa, £ intefo da noi per una infermita , che fa ftare il pazien-
te in continua fame , ed i Medici-la chiamano fame canina . F
¢ | CHElofeanna. BE’ un termine che fignifica grandezza di paflione , ed ha forza
1 davanzare jl fuperlativo , perché dicendofi , Ha ana fame , una fete, un defiderio,

tc. che le feanna , s' intende fame , {ete » 0 defiderio grandiffimo , ¢ pid, Vedi fo-

ei = praC.4. ftan.2z4.
ASCIOLVERE . Solucre il digiuno ; sdigiunarfi , fare colazione . Vedi fopra
“ flan, 35. ma qui é prefo per mangiamento in generale , cioé per la materia

A

Hi Tad! Intende roba da mangiare , che fiaunta, come polli, carnes,

yg
a ec,
' 2 ME -

. ‘ ‘BAGORDO . Bagordare , o far bagordo vuol dir Gioftrare , giuocar d'armi,

+ far conviti , ed ogni altra forta d’ adunanza feftiva , ancorch¢ non d’armi. E
o potrebbe dirfi (cherzando bagordo , quafi vagus ordo , confufione ordinata ; onde
a di gente in confulo » la quale interuiene a tali bagordi , Pigliamo
— ~pol do per commiftione di varie cofe , come nel prefente luogo, che intende
_ melcolanza d’ untumi. Vedi fotto C. 6. ftan. 2. Del refto Bagordo viene da Zi-
tite vuol dire eda. E Bigordare trovafi prefio gli antichi; per corger la

 

» Fazio degli Vberti nel Dittamondo al Canto 32,
; Giovani bigordare alli chintani ,
w hig E gran tornti,e una ,e altre Gioftra
Bete ‘i Farfi veder con giuochi nuovi,e (rani ,
"Poi fi diffe Bagordo ,¢ Bagordare ; ¢ fi trafiero quelte voci a Ggnificare ogni forta
# ‘di ftravizio , ¢ di ricreazione . Che Bigordo voglia dice eda, ciel efempio di
Giovanni Villani lib. - rubric. 132. £ recoffi patio di drappu ad oro fupra capo
i Helfer Amerigo di Nerbona portato fopra bigerdi per pix Cavalieri , Eclgo-
é z da San Gimignano Rimatore antico citato dai Conte Vbaldini nelle Annota-
‘ Meficr Brance(so da Barberi an + Brompere , ¢ ficcar bigards ye lance ,
ee LA

  

 
 

 
  

 
      
 
 

23s MALMANTILE) | 5

LA gola gli fa lappe lappe’, Sigaitica detiderar ardentem
nate dal cao chet il palato con Ja linge £0 VES
za havere nulla in bocea, che ¢ fegno di » qual fuono pare.
lappe ; donde poi il verbo al/ampare , che vuol dire haver gran fam
in Greco , che é lo fteflo , che Lambo in Latino , & fatto dal medefim

ST REG-A (e botti , Stregare yuol dir fucciare il fangue , perch dicono ,
Streghe ficciano il fangue a i bambini ; ¢ perd dicendo frega de borts ini
cia 1 fangue delle bottt 5 che ¢ il vino y del quale ¢ éxgordo , cioe aviditlimo

VSVRP A le (uftanze dele pappe « Divora la carne , che ¢1a foftanza del
del quale fi fanno le pappe. 2 nlohetoendt

cAGGIRA il beccafica , e:pela ibtords, Aggirare, € pelare,metafo!
lando, fignifica ingannar’ uno, ¢ cavargli da dofio danari,; come habbia
nato fopra in quetto C. flan. g. Li Poeta {cherzando pigha decti due verb
vero fenio , ed intende girar nello {piede i beccafichi , ¢ pelare i tordi p
cergli s ¢ mangiarfegli. ‘

LiVA ie cappe ai capponi, Ciok divora la pelle de’ capponi .

E PREG A il Ciel che faccta , che gli agnelli, ec, Dove git agnelli hanno fe
te due granelli , (cio€ tefticoli ) vorrebbe , che ne haveilero’ quanti n’ hs i
melagrane . E cosi defcrive un folenne ghiotto; e crapulone . Similmente un Cet
to Filofleno folenne mangiatore 5 ficcome 'riferitce Ariftotile lib, 3. delle Morali
indirizzate a Nicomaco , cap, 10. defiderava d’ avere il collo pid lunge d’ unas

‘i ate

  

 
 
    
  
  
   
   
  
  
     
 
    
   
  
    
  
 
 
 
  
 

grue {upponendo , che cosi fuffe per eflere il guito maggiore, tae:
STANZA LXIiL comanaal L ake
Vedenda quini comparir repente E quei foggiunge: Adi rallegro,e Gedo ~
L’ infolite cae shigortifce il ghiotta, Che ee | facciate bene,e vi fon febiave;
F dal timor ch’ egli ba di tanta gente Ma s' il patire ¢ fattoa yoo y
Trema da capo a pic, fi pifcia sotto: Penitente di voi non ¢ pin bravo,
Con tutto cio digruma allegramente, Tal ch'io per mevi mando a

E {pefjo fpeffa bacia il fuo barlorto 5 Non nel fettime Ciel, ma
E accio frremara non gli fia la vita Donde ai midani,ea meche[owoileapi,
Non dice mendegnateo aber gliinuitas Pifciar potrete a voftra polkain capo,
STANZA LXIV. STANZA LXVL
e/a i Cavalier famofi a quel plebeo s Ata perch al certo Voftra Reverenza
Che nou proffer: lor della rovelia , Ch’ é frenuata, come-un Carnovale ;
Furon per infeguare il Galatea Hanra fatta fin' hor tant’ affinentay
Con batrergli gik in terraiimama/cella, Che bafti a foddisfar a ogni gran tally
Chi fei? ( difs’ un di loro)e Piaccianteo, flor puo lafciar a noi t
Chie xa pover huorifpode,ein quella Cella dccio baciam ta terra r
Molt? anni in aftinenza ha confumati Per piit mondi accofharfi aquest avans
Per penitenza de’ fuoi gran peccate, Delle retiquie sch’ ell’ ba gi eek
Piaccianteo vedendo comparir coloro armati , hebb’un lef
non per quefto abbandono ii mangiare , anzi fi ——— peril
lomandato

   
   
 
 
 
 

  
 

    
 

_ haveva , che coloro non gli ftremafiero Ja provvifione.

era,rifpote eller uno, che faceva penitenza de'fuoi peccati ia quella cella
caitinenze: Dalla qual rifpofta accortifi, cheegli era un birbone 5 |

*
 
 
   

  

NTO CANTARE: 253
Bli-dice , che lafci un po fare il medefimo digiuno,

4/Si perde d'animo. Vedi fopra Gastan8.Dan,
Cost mi fece sbigortir lo ALefro,

A'S i gli vidi s} turbar La fronte ;
9 Golofo; Avido di mangiar del buono. Lat. ¢éuto ;
‘nol dire haver gran paura . Vedi fopra in quefto C. ftan. 3.
, Intendi mangiare ; fe bene digrumare ¢ il matticare, che fan»
pit feflo , che fi dice anche ruminare dal Latino, che perdchiama
e dette — come habbiamo accennato fopra C. 4. ftan. 6.5 ¢ ve-
9 forta.C, 6. fan. 5.
ACIA il barlotto, Bove . Barlottoéun vafo di legno di figura fimile al barile,
i : hé fara di tenuta o pils,o meno fino a dieci fiafchi,chetenédo
chi fi chiama mezzo barile.Qui pero n6 intende ftrettamente capt fpecie
t@ , ma un vafo da vino portatile addoflo,comunque fi fia o di vetro,o di
una Zucca\, anzi fimo che intenda pil tofto di terra , perché pid git
camo la terra del boccale.
- STREMARE , Vale (cemare , fminuire , quafi ridurre allo ftremo .
 DEGNATE, Eun modo di dire ufato da coloro che mangiano all’ ofteria ,
intorno alla loro tavola alcun lwro conofcente , ¢ dicono : deguate ,
wi di bere, E perché ¢ termine ufatiflimo dalla plebe , il Poeta fa. ,
¢ fi maraviglino , che Piaccianteo non |’ ufi,e fa prendere argumento,
ped afi per paura , che non fia accettato Vinuito., ¢ {cematagli la.

.
CAVALIERS famoft. Cavalieri illuftri , ¢ di fama. Ma qui famofo non deriva
sma allude a fame, ¢ vuol dir Cavalieri aftamati .

We + Vuel dire -huomo di Plebe ; ma ce ne (eruiamo anche per intende.

te ‘infame,fenza honore , ¢ fenza creanza. Qui fe ne ferue per contrap.
lieri famofi , ¢ vuo) dire , che fi come quelli erano famofi , cioé af.

bul cra infame , cioe fenza fame , perché havea ben mangiato ,

, Von, 2} della rovelia , Non offeri nulla; ufandofi {peffo il verbo proferire,

In vece-del verbo oferire ; ¢ la parola della revella & pofta a maggior’ emfafi per

tiprimere non offeri nulla , ne meno una cofa nociva . ,

~, ANSEGNARE i! Gatateo, Infegnare |e creanze, i buoni termini, Galateo. é in-

titolata un’ Operetta di Monfigaor Gio. della Cafa., la quale infegna le buones

 
 
  
 

   
   
 
  
   

  
  

  
 
   

 
  
 

RESleG =

  
  
   

&

 
   
   
   
   

    
    
 
     
   
 

Creanze,

. eae ERGLI. ‘gilt una mafcella, Dargli un tagliovnel vifo , e fargli cadere una
analcia ,

40 vi fon febiavo, Vi {on feruitore . E’ un detto ufato , quando alcuno faccia_,
la azione , che meriti lode , per efempio Il tale fece una beliima Orazione ;

fo gli fon {chiavo, I Caporali nella vita di Mecenate:dice 5

H E fi legge ch’ erugufto un di gli diffe:

Gari Capitan Mecenate io vi fon [chiava, .

 NELL’ ottave Ciclo. L? Autore tenendo !' opinione, che i Cieli ficno otto dice,

Ba : che

  

        

aS SELES SEEPS ES

BAG

  

Digi

 
 

    
      
   
 
 
      
     
  
   
     

254 MALMANTILE™ >

che coftui merita d’ andare nell’ ottavo , ciot nel fup p
penitenza , che merita il fourano pofto nel Ciclo.
MONDAN! . Intende peceatori. Coloro che fono'd
dani. i
ST ENV-ATO come un Carnovale., Magro , come un Carnov:
ironica , che vuol dire Graffiffimo , come fi figura il Carnevale, —
BACLAMO la terra dei boccale , Baciar la terra é un’ atto , che fi
fone divote per umilta s Ma coftui foftenendo I' equivoco del far
haver detro, che gli piace il modo del digiunare , che fa Piaccianteo , d
yuol ancor' egli far’ un' atto d’ uiilta con baciar la terra, ma
boccale , ciot bere. Bocca/e ¢ un vafo di terra capace della meta d’ un
fi piglia per tutti li vafi di terra a quella foggia , ancorché maggiori , ¢
ta di un fiafco anche pid , t
PER accoftarfi pitt mondi , Per accoftarfi
nitenza ,¢ d' umilta con baciar Ja terra.
RELIQV1E . Avanzi , fragmenti ;¢ {cherzando fempre con la bonta
fezione del penitente , par che pigli re/igure nel {enfo {peciale , che I
noi , cioe offa , ed altri fragmenti di Santi ,ed ci vuol poi dire gli avanzid
lui mangiamento. Latino mense relique. Ed in queft ottava |’ equivoco:
ficnuto da coftui in moftrare a Piaccianteo di credere , che egli fuffe u
te , che flefle quivi per fare aftinenza , come haveva detto ; ¢€ per i
tentarfi , che effi ancora 's’ accomodino con lui a far Ja penitenza nell
nicra , che faceva egli.
STANZA LXVIL STANZA LX
Qual madre , che ripara il {uo figlinolo Cosi fam carua di pik rigaglie —
Ch ¢ fopragginnta da mordaci cani , Oltr’ ad un'Oca groffa ar j
Ei cuopre tutto con il. Serraiuols, Ma vedendo pits 1a fra quelle re
Ed eglino gt danno in fale mani; Dun perro d'arme luccicar ,
E col laza del Piccaro Spagnuolo , E del giaco feappare alcune ma lie f
Che dalla menfa-vnel tutts lontani , Da quella {ua cafacca untae
etecio pot a tal cofe non arrivi , Infofpettiron , com’ un’ altra volta
Con due caici lo fan levar di quiri , Patra fentir chi volencier m° afcolta,
Piaccianteo vedendo , che coftoro s’ accoftavano per torgli la roba , cerca di
faluarla,copréndola col ferraiolo , ma effi con una mano di calci V’ allontanaro-
no ,¢ d’accordo fi mefiero a mangiare : Ma intanto,ofieruato , che egli era at-
mato, prefero fofpetto , ¢ fecero quello , che fentiremo fozto nel C. 8. flan. 60,
RIPARARE , Rimediare. Val per difendere. Ed in quefla comparazione> —
‘imita Dante Infer, C. 23. che dice :
Come la madre , ch' al romore ¢ defta ,
E vedo prefo a (e le fiamme accefe ,
Che prende il figlio , ¢ fuge , ¢ non s'arrefta,
Havendo pri di lui , che di fe cura;
Tanto che folo una camicia vefta , -
FERRAIVOLO , Mantello. Vn panno ridotto tondo , ¢ adattato a coprires
tutta Ja perfona fopra agli altri abiet , metcendolo in fy ic fpalle .

  

pil puri , havendo fatto Pau

  

   
 

 
         
    
  
     
 
 
   
    
   
    
 
 
   

_p2Ep SRE EE EE

 

    
  
 
 

O-CANTARE;: 255

nuolo, Gli zingari , quando s’ abbattono nel corrivo;
fa , che gli habbiano vedata , trovano diverfe in-
di farlo ballace , o cantar con loro, o fargli mettere in capo
go , che gli'occupi Ja vifta, o con fargli metter il capo in ua’ arma-
Eee oe, = ae ed inuenzioni per nw ed haver
i rubargli ¢ hanno difegnato , mentr’ egli aftratto da tali ope-
a badaa ry gli Ducts attorno; clea (pets veggiamo eat
commedia , che il feruo aftuto, per truffare il feruo ftolto fi vale di fimili
tic. E quefto fi dice il ¢axo def Piccaro Spagnuolo, cioe inuenzione dello Spa-
0 » Donde poi /azo, dazecgiare fignifica qualunque azione, che fuc-
oi Comici per ¢{primere il ior penficro. E /azo, che in Spagnuolo fignifica
‘prende da noi per quel che i Latini diccbbero capeio , (ophi/ma,commentur,
verfuria , fallacia , artes , doli, Ed in quefto fignificato va profferito con,
» €non cruda , ed afpra , perché con la cruda fignifica fapore afpro ,
ate , come que) della prugna , della forba mal macura , ¢ fimili , che i
il dicono acide ; Dante Inf, C. 15,
ie Ed é ragion , che la trai layzs forbi
ge 7 Si difconuien fruttare il dole fico
Z » perché ¢ frutca di fapore, /axz0, cio’ acide dicefi da gli Spagnuoli
quafi dai Lat. diminutivo acidu/a ,
‘AR carita , Fra i Bacchettoni s’ intende mangiare infieme. E tra gli antichi
( iconuiti , che fi facevano a’ Poveri ; di limofine , fi domandavano dea-
pat , clot Caritadi , EB Pietanza , voce confervatafi tra’ Prati, ¢ tra le Monache ,
Piatto , o mangiare offerto dalla picta, ¢ carita de’ benefattori; non,
indo altro Pieranza, che Piecd, 1] Beato Fra lacopone : Vorria trovar
skune, Che avefe pictanya De lo mio cor afflitto .
ARCT raggiunta, Grathiiima . Vccello foprammodo grafio fi dice raggiunto .

 APCCICARE, Rilpiendere ; Rilucere. Viene da Lucciola .
: CASACCA . Parte d’ abiro da huomo,che copre la perfona da mezza la pan.
vin fy al collo . Cosi Ca/x/a in Lauino ; fe bene altra forta di vette, diver-
fa \Cafacca, fu detta cosi , perche copre tutta la perfona a guifa , che fa la
tala j fe crediamo a Ifidoro nel jib, 19, delli Origini , al cap. 24,

FINE DELQVINTOCANTARE,

 

 

   

Dy

 

S=sSTO

 

 
 
  
   
   
  
  
    
   
   
  

 

 
lll

——

SS OSE See

——

 

  
 
 
  
  
 
 
 
 
 
 
 
  
 
    
 
 
 
 
 
 
 
 
    
  

(ESSE Se ARS th WE
© PA SA a aes oe dP
SESTO CANTAR
Ese WES CSS
ARGOMEN TO, ' 8

5 Nel tenebrofa centro della Terra ,
Ove regna Plutone entra la Strega ,

oF E vnol che [eco per finir la guerra
Di Malmantile entri f Inferno in leva.
" Fanno concilio i moffridi fotterra ,
Ove ciafcun buone ragioni allega ;
2 Certa al fin le promette ? affiffenza ,

Rend’ ella grazie, efa di li partenza.

Be secxrige asin
Bia PS A

aoe
STANZA I STANZA IL
Miler chi mat oprando fi confida: Di chi creas Letcor tu.qui cht ia tratti®’
Far' alla peggivse ch'elia ben gli vada, Tratto di Adartinazza inigua Seregay
Perché chi pyglia il vizio per jua guida, Cha pin peccati, che non ¢ de’ fattiy
Vs contrappelo alla diritta firada . E pel Demonio ogni ben far rinnegay .
E benche qualche repo ei (guarzrijerida Di darfi a lus gid feco ba fattod patti, a
Col vétoin poppain quel che pingli aggrada, ecio ne’ {uci bagordi la ay ‘A,
E' vienposl'ora, ch'ei n' ha arender coto, | Ma frate pur; perché rards,e per-tempe iy
E far del tutto 5 dondola , ch’ io feonto, Lo fconterd ;.da ultima’ ¢ buon tempos hy
STANZA LU. Pee ky
Non fi penft dhaverne a ufcir netra ; E quand’ ei poffa , non fe lo prometta, ty
S'inrighi pur col Diavol, ch’io le dico, Perch'ei, che fempre fu nofire wimite y
Se forfe haver da iui gran cofe a/petta, We puo di ben verun vederci ricchi,
Che nulla dar le puoch'egli ¢ mendico, Vana fune daralle , che L impicchi .

Ji Poeta havendo penfiero di narrar Ja gita , che fece Martinazza ai Regno di
Plutone per muoverio ad aiutarlo a diloggiar Baldone da Malmantile , ed a g*
ftigare Gambattorta , e Baconero , fa ’ introduzione al prefente Cantare cons
una riflefione morale ponderando , che quei , che opera male , non pud fperare
d’ haver mai bene , ¢ principiando come!’ Ariofto C, 6.

Atifer chi mal! oprande fi confida
Conchiude , che Martinazza y Ja quale nou fa fe aon fciagurataggini 5 es’ édata
al Diavolo , non pud {perar d’ haver @ hayer bene , perche il Diavolo e es
 
   
      
      

SESTO CANTARE:

non pt irepral bon paces Hig ak
«10 weet fenza riguacdo alcuno.,
va per il verfo.buong. Va al contrario di quello , ches
da diritta,via.. Sen, epift, 122. Omnia vitia contra naturam
‘dinem deferunt ; boc off Ingcuria propofitum gaudere peruer fis:
dere a reito , fed quam longiffime abire ; deinde ctiam ¢ contrario flare .
andare a.rigrofo dai.Lating retror/um, Dan, Purg. C,.10, in fimil

PPE wii fou
“gy Ofgek Criftian miferi g¢ laff y

) 4 ) (Che della vifta délla mente infermi
pAb Fidanzahavete nes ritrofi paffi.

@’ andar contrappelo ¢ tolta da i pezzi di panno , 0 di pelle pe-
in cucirle infieme.s’ oficrua , che il pelo vada. tutto per un.verlo, ac-
> iano .. A taftar un panno , o pelle pelofa per il verfo, che vail
pilfacile , ¢. non fi trova refillenza alcuna , come.a andar contro as
By Jatin

j
» Goda allegramente . |
:
|
;
|
|

 

  
  
 
    

 

 

|
cou vento in poppa . Secondo che ¢j defidera: Come fuccede quando fi ha il
Vento in poppa della pave :-¢ fignifica ¢ m¢gonzj vanno bene, | Greci pure differo

vento navigare .

i OLA ch’ io fconto, Vuol dire {contera il buon tempo,che ella fié data.,
; alcrewtanti difgutti, E’ detto ufato dalla Plebe , nella quale ¢ nato; ef
endo lato detto.da un maceliaro , a cui ¢ra flata rubata in pil volte gran quan-
ita di Catne 5 ed eflendo facto ritrovato il ladro , fu impi¢cato., ed, il maceliaro
‘appelo alle forche diffe: Dondola , ch’ io feonto; intendendo’a vederti
_ dondolare Sconto il debito,che hai meco per Ja carne rubatami.. Dondolare , &
lo Re ciondolare , come appunto fa |’ impiccato ; ¢ tal Verbo dondolare
i il nome da quel don don,, che fa il fuono delle Campane. E da quello me-
» che faceya quel tanto rinomato vafo dell’ Oracolo di Giove , che
rain. Citta dell’ Epiro , Mima , e con molta ragione , derivarfi il nome
‘ 41, Dodona Abramo Berkelio Olandefe nelle Offeruazioni al Frammento dell’ O-

Peta originale di Stefano de Vrbibus. Dondolare , 0 dondolarfela yuol dire Star-

fenea federe fenza far nulla , di dove Dondolne vuol dire un perdigiorno. Quin-

diua moderng Poeta insendendo di quefti tali diffe :
Voi dal notturno al mattutin crepufcolo
ah hay Vi dondolate ,e¢ fate atu me gli hai,
8st. Soaisconer We conchindete 0 proponete mai ,
Se non rovine al popolo mink{colo . ;

© HA pit peccari , che non ¢.de' fatti « Ha pili peceati-ella fola, che non fono
quelli , che fono flati fatti , 0 commeffi da tutto i] mondo infieme infino a ora ,)

SAGORDI, Fefteggiamenti . Vedi fopra C. 5, ftan. 62,

TARODI, oper tempo. Diciamo anche Tardi , @ accio ( cioe avaccio , parola,
antica , rimafa in contado , che vale tofto ) 0 vero} tardi , o avale; che diflero
& ancora gli antichi agwale ; cioe ora , in quefto punto; vuol dire;quefta {eguira una
s olta:opreflo,otardi. Lat. /erins, ocyus . 3
i. Mare fh Kk DA ‘

uF SA EE

SaaS TS

'

 
 

 

 
    
       
   
 

ase MALMANTILE ”
DA ultimo é buon tempo, Da ultimo verra il ferena Pe
deito ironico, perche fignifica, che da uttimio per Martinazza’
tivo , cice fara gaftigara del fuo mal fare, ©
INT RIGARS?, Vuol dire impacciarG, o intereffarfi : ¢ vuol
giiare, 0 mefcolar una cofa con un’ altra in-manicra di
go per imbroglio . Bihes ae BR
VIA fane daralle, che? impicchi, Quand’ altri ci ha im: niti , pe
gli ,che non merita rimuncrazione , fi fol dire ; Gli vud dare un par
Vn par di funi , 0 una fune , che impicchi . if = ie
STANZA IV. STANZA

    

 
    
 
 
 

 

Horsit tiriamo innanzi , ch’ io ku finito Ella ch’in tanto havuto havea,
Perch’ a quefti difcorfi le perfone Che quei due spirti feiocci ed
Von mi dicefer : Quefto feimunito Havean dinanrs a lis fatto U

 
  
 
 
    
   
 
 
   

Virol farct qualche predica ofermone, Si che dat effo furono Scoperti ,
edirenti dungue. Gid v'havete udito Se la digruma,che ne va il fuo'
L incanto , ch’ elia fece a petixione Mentre gli accordi fe
Di quei det luego , c’ hebbere concerto Rinfciti alla fin tutte,
“Scacciarne il Diuca;ma fuani lefferto, Con un palmo di nafo ne vil
Ii Poeta lafciando da parte la moralita,viene al racconto , ¢ torna alla
tia de] Lettore ’ incanto fatto da Martinazza per cacciare i] Duca
hebbe effetto , per lo che ella é in collera , 4 le pare di perdere<
ma , nella quale era tenuta dai popoli , ¢ foldati di Malmantile.
SCIMVNITO , Sciocco’, fcempiato . Vedi fopra C. 1. ftan. 17:
SVANL: Pofferto , Non riu(ci? effetto : il negozio ando in fumo , 1 Lat.
‘diflero Exanuit , & evanefcere. 198
SE la digruma , Seco ftefia 1a penfa , e maficandola non la pud inj
‘cioé rion la pud fofierire . E fi dice digrumare , ¢ ruminare, ¢ dagli an
‘to rugumare , onde forle & fatto digrumare; (che ¢ il rodere che | Ie
ipi¢ tefio , come vedemmo fopra C.g- flan. 6. ¢ C. 5. flan. 63.) perche
fucceda cofa di poco {uo gulio , fuole per lo pil ftando penfofo ma
fcrare appunto come fanno dette beftic quando digrumano., al che
ebbe riguardo Omero in quel verfo tradotto da Cicerone.
Ipfe funm cor edens , hominum veftivia vitans, a
quafi che chi maninconico rumina ,’e biafcia mafticandola male; ‘moftri di
carfi il cuore. % ; a
RIVSCIT I tutti panzane., Son riufciti tatte vanita , tutte chiacchiere
anzave , bubbole , chiacchiere , ec, vuol dir promettcre » ¢ NON mantenere , ©
dice inzampognare , infinocchiare , ed’ il Lat, Verba dare . 4
RIMANE con tin palrio di nafo, Riman burlata’, beffata , I-Lalli En, trl
flan, 11. dice. tet T stkaigit

 
  
  
    
 

   
   
  

Ed io fon per reftar in quefie caps
Con fei palmi lunghiffini di nafo'e
 

 
 
  
     
   
   

SES TO C’AIN TAR E:

WZARQVEE cad oon il STANZA VU.

fe Bafta, chella fel ¢ legata al-dizo ,

 Etha prefa co’ denti,, ¢ fer! affenns ;
Tal ¢ andarfene in Dite ha feabilito,
Perch ne viol veder quanta la canna ,
Ed oprar,, che Baldon-refti chiarito
Crambifce in Malmiatilfedereaferana;
Hor mentre a quefia volta 8 indivi,
Potra far un viaggio a due feruizj .

non fi perde d’ animo , ¢ vuole in ogai maniera fcac¢iar l'elercito

a Malmantile . Rifolue pera d’ andare all’ inferno ia perfona a tro-

-» per ottener da lut il gaftigo di quei due diavoli , che feccro i’errore,

jo modo di far diloggiar Baldone da Malmantile,

shiz eee fiperde d’ animo ; Non fi gomenta..., Vedi fopra C.

8.6 C. 5. flan. 63.

b lifterrefe. Viebbe finito di cono(cergli , Hebbe viflo quanto effi va-

Si dige Ta m' bai dato il mivrefio: Tu m' hai preno : Son fazio , fon feufo di

r intendere Now mi varro mai pili dell’ opera tua .

hanno fatta di figura, Le hanno fatto una ingiyria grandifima , unas

ima buria. Tratio dal giuoco di primicra , quando uno havendo buon,

ed efiendo per vincer la pofla , un’ altro con figura fa una primiera,e gli

  

   
   
  
   
    

ANNO un caprefto. Reftino impiccati. Chiamano caprefto quella cor-

jie , che i] Boia lega aj cojlo a coloro , che egli impicca , la quale di-

morto il paziente fi rompa ; ¢ perd dice rompano un caprelto ; detto

tidimo per intendere farfi impiccare .

DERRE in lowatura, Ridurre in minutiffimi pezzi. Limatura fi dicong quei
i che cafcano dal ferro, 0 altro metallo, quand’ altri lo lima.

i morfe mai cane , ch’ 10 non voleffi dei fyo pelo, Nefluno mi fece mai in-

40 non mi voleffi vendicare . Nefluno mi morfe, che io non lo rimor-
fi, D che il pelo del cane fia medicamento alle morficature fatte dal me-

'  defimo cane. Vedi fotto C. 9. flan. 58. Eda quefto rimedio ha origine il prefen-
te dettato ; che i latini diflero Nemo impune abyt , qui me aufus fit ledere,

it SEL’ é legara aj dito. Ne ha prefa memoria per vendicarfi. Sogliono molti per

_ haver memoria di qualche negozio, che deyano fare,legarfi un filo intorno a} di-

a = i che ha dato origine al prefente dettato. Ii Lalli En. ‘Lr. Can. 2, flans25,

Tigran Sel’ attaccd , come fuol dirfi , al dito.
Nel Deuteronomio alfefto, Eruntque verba hac , qua ego precipio tibi hodie in corde
tu: © narrabis ea filijs.tuis., & meditaberis (edens in domo tua, & ambulans in itino=
1, dormiens arque confurgens : Cr ligabis quafi fignum in manu tua, B (ono al cap.it,
_Ponite hac verba mea in cordibys 5  animis veftris , & fufpendite ea pre fieno in man
__ tiibus , Bra Giordano Predi antico Domenicano ; nel Vocabolario della.
_ Ctufca alia Voce Filareria. Le filaterie fi erano una carta, ove erano fcritti i co~
Mandamenti della Legge , ¢ portavanla — al braccio apertamente. B quivi

S$4NE2 2 va

 

Dia

 
= Bi Asi - SS ee oe

= Se eS

ee

 

254 MALMANTILE?

va spiegando , cred’ io , il paffo di San Matteo cap. 23.) Di
Jua, B voce Greca ; da phylattein > puardare, di 1
di quoio, o di cartapecora, che gli Ebrei fi legano albraccio’
mente a memoria-i padi delia Scrittura , che.quivi {ono nota
domandano:Tephilim , wa haba sof eo eseytt -

L A profa covdenti » S' & adirata grandemente ,°¢ sé meffa in a
dicarfi. Vuol impiegare ogni fuo-ftadio per vendicaré icalzolai
venire il quoio a quel fegao che loro bilogna; tirarlo co’.denti 5 di
prefente termine , che efprime uno ; che fi fia prefo ‘a cuore di’ un
e'che vogiia impiegare ogni {uo talento:per conchiuderloy

SE »' afanna, Sel & prefaa cuore: N’ ha premura * Sene da pena 5
fiero . , |

 
     
 

ne,|'uno,¢ l'altro nome fignificado ricchezze delle quali,perche fi cavano di
ra, facevano Cuftode,e Padrone quel loro Dio forterranco ; ma qui fi piglia
per la Citta , ¢ per il Regno di Dite, : aa
Ne vnol veder quanto la canna . Cio’ quanto tira, 0 & lungs Ja canna da mifu
rare; ¢s’ intende vederla per la minuta , ¢ quanto fi pud , ¢ fare ogni sforzo pet
arrivare al fuo intento, °
REST I chiarito . Refti (garito : Scaponitox Vedi fopra C. 1/ftan. 12”
SEDERE a feranna , Vuol dire comandare’; effer padrorie { “Scrannay
me diciamo noi ) ci/cranna,é una {pecie di feggiola da i Latini detta fe
Dante Purg. C, 19. dice: ‘ a cS
Hor chi fei tu che vuoi federe a feranna a
Per vindicar da lungi venti miglia alee
Con la veduta corta d’una {panna ? Ue aan
Buratto nell'Apologia contro al Caftelvetro dice + Aon habbiare tanto cermelle, he
baffi , fe ben volete (edere a feranna per giudicare gli altri, uy
FAR un viaggio a due feruizzj , Che dichiamo anche + Fare un viaggio 5 e dies
feruizx). Con un medefimo viaggio far due negozzj , che é impetrar da Plurone
il gaftigo di quei due diavoli , e lo sfratto di/Baldone , Ne i Latini fi trova ims
quefto fenlo Duos parieres de eadem fidelia dealbare ,E fi dice anche Dare a diet
vole 4 un tratto, Vedi fopra C, 3. tanet4. > oe

On

 

., STANZA VILL STANZA ‘IX.
Git da Mammone andar vuolein perfona, Percid s* accontia , e-vie tutta ph y
Che pile non ¢ dover, ch'ella pretenda 5 Col drappoin capo,e vol véraglioin mant
Che {ua bravicorniffima corona © cercar chi? informi della cite;
Salga a fuo conto a veni poco, e fcenda, Ne meglio'fa, che Giulio Padovanes
Chieder grariese dar brighe no céfuona, Chet ba fu 'per le punta delle ditt)
E chi ha bifogno,fi {uol dirys' arrenda , E pik ds Dante, e pi del Manrovans
Per quefto a lei tocca apigliar la firada, Perch eglitio vi furon di a
‘Per calla fin conuien,che chi vuol vada, E quefo ogni tre di vi ‘
, ol vt sain wan?
at oie

t : 3h wath ~ 1 ,

IN Dite. Dite , fecondo il favoloforcreder de i Gentili’é lo fteffo see

 
 
   

  
   
     
  
      
     
  
  
  
   
  
 
  
    
 
    
   
 
 
 
 
 
   
    
 

 

SESTO CANTARE: "255
WaAE PCS. S. - VOW ZA)

  
     
     
    

E poi per abbondare in cautela ,
che in Di prefume) Volendola fernire infino al fiume,
che gente, ¢che loquelay Le porge un fardellin piccolo , € poce
ple dd conto ,¢ lume ; Di robe,che laggth le faran giuoco,

a'rifolue' ae in perfona a aan x oe che
y che quefto Re per lei a ogni fcomodi ; ¢ perd fapendo,che
Padovano é pi aibithaes d oll alevo' della ftrada dell’ Inferno, fe ne va
r da lui informazione , ¢ della gita , e dei coftumi di quei paefi ; ed egli
ice , ¢ per feruirla meglio la vuol accompagnare fino.al fiume Acheron-
intanto le da un fardellino di robe , che laggib verranno.a bifogno’.
VICORNISSIMA corona , Epiteto , ¢ titolo compofto dall’ Autore a Pla.
gh they Lalli’ Bn: Tr. lib. 1: tan. 16. parlando d’ Eolo Re de’ Venti dice;
ie sh » Dunque poi che Giunone alla prefenza
ro) pis) ies Di {ua Real ventofita fu giunta .
'  - MAMMONE , Da Mammon ; parola wfata nell’ Evangelio. Alcuni efpofi-
ue | Sacra Scrittura vogliono , che Mammona fia voce Caldea , ¢ fignifichi
$5 ed altri che fia voce Siriaca , ¢ fignifichi quello, che in Greco fignifica,
» che € divitie , fi che concordano , e tanto é a dir Mammone, che»
 Demonio, ovvero Plutone , che qui s'intende per il Re dell’ Inferno. Vie-
a ne dalla radice Ebrea Taman , che propriamente fignifica a/condere , riporre , es
Per COsi dire-intanare ; onde fi fece AZatmon, ¢ alla Siriaca Adatmona, cio’ ricchez~
Xt nafeofie , © vogliam dire teforo. Mammona poi venne a dirfi per pit agevolez~
za zia
_2. + Dare fcommodi , dar moleltie . La voce briga fignifica opera-
2ioni (coirimode , faticofe , ¢ noiofe . eit
yf CHP ba bifogno # arvenda . Chi ha bifogno non fia fuperbo , ma fi pieghia rac.
Ȣpregare ; Che il verbo arrenderfi val per cedere piegarfi , 0 con-

 

‘ Cc.
ie CHI wolf vada. Chi vuol ottenere una cofa vada a chiederla da per {¢ , ed il
et Ndice Chi non viol manat , e chi viol vada da fe, Che diciamo anche Non i
ie «ibe mee , Che Je fPeffo , 0 vero , Chi va lecca, E chi fha fi fecca , i

ACCONCIARS!. Rinfronzirfi , raffazzonarfi. Vedi fopra'C, 2.ftan, 69,
DRAPPO, Dicendoli drappo affolutamente s’ intende drappo da-donna s che
una ftrifeia di taffetta,o d’ ermifino Jarga fino a due braccia,e lunga fino aquar-

tro, la quale dalle donne Fiorentine di condizione ordinaria € portata in capo ,

Oalle {paile quando vanno fuori di Cafa. In Venezia drappo fignifica ogni forta

diveftimento , fi come prefio i Tofcani antichi {crittori . Vedi foto C.7.Man.az,

VENT AG LIO , Strumento noto ufato dalle donne la ftate per farfi vento ,

ZL INFORMI della gita, Le infegni la ftrada , che conduce all’ iaferno ,

GIVLIO Padovano, Io veramente non ho faputo ritrovare chi fia quefto Giu-
_ lio Padovano , fe forfe non ha intefo di Giulio Hygino ferittore d’ Aitronomia ,

Ma coftui fu liberto , o vogliam dire {chiavo atirancato d’ Augufto; condoreo da

lui ra d’ Aleflandria , fecondo che alcuni vogliono ; i quali percid lo ftima-

no Al rino ; o pure di nazione Spagauoio , fecondo la teltimonianza di Sue~
¢ Maio nel Libro de illuftribus Grammaricrs . L' HA

 

 

Sea S KE =

 
 

 

   
 
 

256 MALMANTILE ©

L' HA fu per le punte delle dita, La fa benitfimo ; Latino im m
do Manuzio nella dedicatoria di Giuvenaie difle : Quando eas tench
digicas ungue(que twos, Cicerone nella Orazione cont i
uid cum accufationis tua membra dividere ceperit 5 @
canfe conftisnere ? Quid, cum unumquodgue tranfigere , expedire ,

DANTE , ¢ il Mantovano , Dante Poeta Fiorentino ; ¢ Vergilio
te finge y-che fuffe fua guida all'Inferno , ¢ pero dice : Egéine vs furon

| OGN tre di, Quefto modo di dire, fe bene € determinate, fignifica tp
fo , ©.a ogni poco indeterminatameate , eden ah

ANDAR via divela, Andar via velocemente , ¢ a dirittura , come |
quando va a vela. » one

PER abbondare in cautela, Cioe per feruirla bene . Diciamo abbondi
quando uno fa pitt di quel che Ga richiefto , o pil di quel che fia n
elempio . lo dard diect feudi a uno , perché mi compri una mercanzia , la«
fo che non vale cosi gran fomma ; ma per aflicurarmi del cafo , che valeffe:
ir cautelato'y

   

 
 
     
     
  
   

pil , li do due altri (cudi per abbondare in caurela , cio per anda
ful ficuro , che non gli manchi denaro,fe ella valefe pi. Qui pegd
Abbondare, ed eccedere in corcefia nel feruirla . rune
LE faranno ginoco . Le torneranno a propofito., Le verranno a bifogno, Le
faranno d’ utile . sod
STANZA XI. STANZA XI

Cosi la Maga fe ne va coneffo , Queftat la via , che mena
* Che f introduce in una bella via Perch'ellac allegrayo
Tutta frorita, st che al primo ingreffo Perché 4 martello poi non we
Par proprio un Paradifo , un' allegria; LA feorre ognor gente di mal afarts —
Ma nipin prefol buom il pic v'ha mefo Le ferpi fono ogni opera ribalda y
Cb’ ella diventa un’ altra mercanzia Chrella ci fale quali a lungo ani
Per i gran morfi,¢ le punture acerbe , Di quanto ha fatto, fcavallata, ¢fenfe
Che fanno i ferpi afcofi fra quel! erbe, Ci fa fentir al cuor qualche rumorfa,
STANZA XIL STANZA XIV. 5

   
     

 
   

    
   
    

    
     
   

Entravi Martinazza,e fente un tratta tMa fe ravvifia un tratto del fue,
Dueyo tre morfi a pit dove calpefta, Bada a tirar innanzé alla balorday —
Percio befemmia,che non par [uofatta, Perch’il vizio rifiglia,e mecte il tally
E dice:O Giulio mio,che cofa e quefia? Vie stpre pix aaggravar/iinfulacerds,

Ed ei ridendo allora come un matto; Ul male inuecchia al fine,e vi fa ilealle 9

Non é nulla ( rifpofe ) vien pur lefta; Siche venga un Serpente parese ‘

Che penfi tu ch'io fia privilegiato so Chrei x6 fente ne meno anctun ribregeny oa
bY 19 mi fento mardere,e non fiato. Cos} peggio che mai la da pel mezts

pa kar "STANZA XV, er

  
   

lla neve fi f4 lo fteffo giusco y ‘ Al fine ei ff rifealda come un fucco
Ebert ment Jul primo dacciafi le dita, Si che non lofarwits mai finita

Poiquelgragelo par che maanchi un poce, We gli darebbe punto di {pavento'

E fempre pine nell! Agitar (a vita; Quandei thavelfe acoraa dorami

Martinazza fe ne va con Giulio , il quale la conduce per una ftrada,
primo mgreiio pare una belia cofa , ma prefto fi conoice , ch'ell’é al

SSLPEE sR Eee Sp Ge eae gis
 

    
  
    
    
  

SESTO CANTARE, zy7

i-€ i i afcofi infra quell’ erbe ; Giulio moftra a Martinazza,

 ftrada , che guida all’ Inferno é facile , e guftofa, ¢ fe bene ¢ ripicna.,
i, non fon fentiti ne conofciuti da quelli, che la camminano , perche
afluefatti ; appunto come fanno coloro , che mettono Je mani nella ne-
i ipio Ja toccano fredda , ¢ col feguitare a maneggiarla , par loro
PARE un Paradifo., Pare wna-cofa tanto allegra , ¢ vaga-, che pid non fi pud
_ fare. Telemaco figliuol d’ Viifie nel quarto del’ Viiflea , arriyvato in Sparta ; nel
_ confiderare attentamente la ricchezza , ¢ |’ ampicaaa del Regio Palazzo di Me-

i — in quella ¢{clamazione : Ta/ dentro ¢ del gran Giove ilgran Pa-
tO,

4 ENT A wn altra mercanzia . Diventa un’ altra cofa. \Véiamo dir mercanzia
i ogni forta di cofa ancor ehe incorporea , come 40 frudiare sé una cer-

py eC,
par (ue fatto + Non par che faccia quella tal cofa. Vedi fopra Can, 4.
aj flan. 16.
-) CASA Calida, Intende V’ Inferno. Il Lalli En. Tr. parafrafando facilis de-

 ferfus Auarni ec. dice :
= Eves mio bello
: A cafa Calda fi va prefto prefto;
’ | seen ‘CHa ritornar infu, quale é il bordello.
] NONE nulla, Quette due negative fecondo la buona regola doverebbono afier-

| Mare, ma é noftro idivtifino tanto inueterato , che I” ufo-ci libera dall’ errore, {e
"‘@ene{eruiamo in quefto modo per negativa . Apprefio i Greci due negative , 0
. = affermano , ma negano maggiormente , ed ¢ maniera , ficcome appref-
40 noi; cosi appreffo loro ufatiffima .
¢ WNONfaa marcello, Non regge alla prova. Noné-com’ ella pare. Metafora
yf ‘tolta'dal Cimento dell’ oro. Vedi fopra C. 5, flan. 2.
‘4 LINGO andare. Col tempo... In procetio di tempo'; Se continoverai lun-

ee . ‘

fo SCAVALLATO , Ciok datafi ogni forta di bel tempo. Si dice anche /correr
i,  latavallna » Virg. 3. Georg. Scilicer ante omnes furor eff infignis equarum , Bt men-
io ‘tem Venus ipfa dedit . E poi: dilas ducit amor trans Gargar astranfeque fonantem,&c,
»  VedifopraC. x. an. .66. :

|) REALCHE vimorfo . Senton rimorder la cofcienza'per gli ¢rrori-commefi.,
@ — ALLAbalorda , Senza-confiderazione .

METT £ il tao. Talliice , fa nuove mefle, Vuol dire:sun vizio ne genera,
g  Mholti, Tallo @ parola veriuta a noi dalia lingua Greca, che fignifica germoglio,
et ‘ufata ancora dagli agricoltori-Latini..

»  VIENE «aggravarfi in fu lacorda . Vien pid che mai a crefeere il male; perché
ido uno tocca i] martirio della corda , ¢s’ aggrava in fu la medefima corda,
~ fa crefcere il dolore; ‘Ed altrimenti 4g¢ravar/? in fu ta corda vuol dire quando uno
#  sfaminato'in fu la-corda dice-cofe, che fanno crefcere I"indizio , che egli hab.
y EMS somuneo-un dette. Ang

§ | “PAilecatlo. Vis’ afiucka. Er ab afuctis non fit paffio , dice, che-non fen-
fi “fe pcmeno.un . c . a RL

 

 

 
Foe,

a ee a

 

 
  
  
   

258 MALMANTILE,

A/BREZZO , Che vuol dire'capriccio di febbre; cioe quel
che fi fente prima , che entri la febbre « Latino rigor avalc
lib,2.cap.21, dice: Antipatro di Sidonia in quel giorno, che egl
gh arrivava qualche ribrezzo di febbre , ¢ tanto continua, s ce
mortale accidente, Ma Dante nell’ Inf. C. fi
Qual é colic! ba si prefa
Della quartana , ¢' ba gia fugna {morte
1 E trema tutto pur guardando il raze.
BalC.za.dice: Pafcia vedd' io mille vifi cagnagri ;
Patti per fredds , onde mi vien riprenroy

E verrd fempre de i gelati guazzi,
Ma noilo pigliamo anche ( come ¢ pre(o nel,prefente Iuogo ) per ogni leggi
follevamento d’ animo , o-{pavento , 0 per un jemplicitiimo dolore , Bdal
te per fattidio , o travaglio per efempio // rale commelfe quel mancamento; ne,
haver de! ribregzs , Vedi foto C, 11, ftlan.2., 9 spkway ¥
La dd pel mezzo, Fa tutto quello, che gli vien yolonta fenza riguardo aleuno.
E’ dedotto da quelli ; che in tempo di pioggia camminando per la Citta yanno
per il mezzo della firada , ¢ non fi guardano dal’ ammollarfi per J’ acqua cadu

 
 
   
 
     
     

     
 
  
     

   

Fe SEES ST ee

=

 
 
   
 
 

  

  
  
   
      

ta, che fcorre pel mezzo , ¢ per quella che vien dal Cielo, <i ied
STANZA XVI, ‘ STANZA XVIL fp
Hor tu m' hai intefo:rafferena il volto, Refta , dic? ella, omai ch’ io ti ringrazia We
Che tu vedrai tirando innanzi il conto Dellinfernxionsch'appiio: li
( Perché di qui a poco non c' é molto ) Promiffio bons viri off obligatiay
Che delle ferpi non farai pin conto, 1 Die egli ; Tho promeffoge intends a
Ma dimmi, c' ba’ tu fatto del rinwolto 2 Ancor feguirti quefto po iay) ha
Lho qui, dic'ella,fempre lefto,e pronto: E quivicon un tibi = "
Sta ben,foggiunge Ginlio,adungue corriy Ail) in qua ripigliands.il mio.cammind

 

Perche qui non ¢ rempo da por porri, Ti lafcio , come io diffial
. Giulio ¢forta.Martinazza a non haver paura, ed a camminare; ed
grazia dell’ inftruzione datale , ¢ lo prega a partire , ed egii ricul di farlo, pet
¢hé l¢ ha promefio di accompagnaria infino al fiume-Achcronte.. 3... 4 ki
D1 qui a poco non ¢? ¢ molto, Quefto termine giocofo ¢ vfato per efprimere r+
ochiffime tempo. ‘ 4 : vowed Sag
TIR ANDO innanzs il conto, Seguitando ll {uo viaggio, EB’ termine mereaatilt,
che vuol dir portare un conto avanti da un libro a un’ altro, oda unacattaae
un’ altra nel medefimo libro, Donde poi tirar innanzé il conto yuol die Cammina
re avanti. Vedi fopra C. 4, ftan. do, ot, nls To i
NON é tempo #a por porri, Noné tempo da perdere , Non & da indugiares.
Quando fi pongono i porri, fono cosi fottili, che Hehe
i a Wnt
| big

 
  

   
 
  
 
 
 
  
  
    

*Zezesaerea

[Ras

porgti; eda quefto habbiamo il prefence proverbio, che fi dice anche +
tempo da dar fleno aoche . i w naobs
PROMISSLO boni viri eff obligatio, Sentenza latina , che vuol dire un’:
da bene ¢ obligato a mantener Ja parola ,.¢d offeruare quel che ha p: a2
CON un tibi me commendo , Detto latino , che fuona con un. mi & a
te; ciog con falutarti.. Quando diciamo: Addio , C1 s’ intende ; vi.raccomandd «
‘ ms Sadun

     

 
 

f ut  -gRSTO CANTARE; 19
alut lo; Catullo : Commtenide tibi me .

  
     
 
    

colonnino, Tial

AANZA XVIIL

ti ‘4 il capo, @ rocca ,

¢ ferpi elt a qualche paura ;
2 x tase fatto def enor roccay
Vi otal teense ntl

is

£.

 

 & Bed

diffe : Vale.

til famofo fume d’ Acherorte ,
ve s'imbarca ognun,che quivi arriva,
| Saffaccia ach'effayma il nocchierC ar ite,
7 che trate ognuxo hebbe dariva,
dietra,grida a lei con toruafrote,
Che quad non pafsa mai anima viva;
~Ond’ ella , meffi fuor certi baiocchi ,

2° Lafcia? ad Colosnine yuo! dir la{ciar uno
laren i mo
vanti alle forche , €'vi leg

Ha colonnetta di legno traforata— ,
inialfactori 1 gli ftrozzano.
STANZA XxX,
Ed égli,che da e/a bebbe il fapone ,
E che fi trove li come tl ranacchia y
Prefo dalla wiedefima al boccone,
Menty" ella faite in barca, chinfe locchia;
La Strega fia quell anime fi pone ,
Luaic nlebrach: fon fino al ginocchio y
Duvendo a’ Sopraffindagi di Dite
Prefentar de’ lor libri le partite.
STANZA XXL
Piangendo , come quando yno ha partite
Le cipolle fortiffime malige :
Pafsan quel fiume,e poi quel di Cocito;
Vitimamente la palnde frige ,
Che a Dite inonda tutto il circuito ,
E in fe racchide furbi,e anime bige
Ove Caronte al fin [endo arrivata

Gli getta un po di poluere negli occhi , Sharco tutti ; ed ognun fu licenziato}

‘Martinazza feguita i] fuo viaggio , e non fa pid ftima delle morficature de i
{espi yed artivati al fiume d’ Acheronte’, Ginlio fi licenzia dalla donna , la qua-
Te ) per entrar nella barca; ma Caronte lo rie dicendo,che non poteva
eotrarui , ond’ ella gli diede un poco di mancia , ed ei finfe di non la vedere en-
Wat in barca , dove ella fi me{cold con gli altri , ¢ fu condotta all’ altra riva ,¢
guivi con effi sbarcata .
oe, « Sidice tocca il cocebio , ¢ fignifica: cammina innanzi . Vedi fopras

41.

ZAMPETT A. Muove le gambe : Cammina. Zampettare fi dice propriamen-
te de'bambini quando cominciano a imparare a andare .

NON fi fente aprir bocca, Non fi fente parlare . Sono infiniti i modi, che hab-
biamo per efprimer il filenzio d’ uno , come far Zitto ; non fiatare ; non far verbo
mmansolire ; ftar chiopto , lafciar la lingua al beccaio, baver viffo il lupo ; diventare Ar»
porrate ec.

GLI difie : Vale . Gli diffe Addio -

ACHERONT £ . | fivmi dell’ Lnferno da i Gentili fi dicevano quattro , ¢ ches
fialteflero dalle Jagrime de’ mortali , per lo Mato de’ quali figura Dante Ja flatua,
the vedde in foguo Nabucdonofor , che havea la telita d’ oro, le braccia, es
Vie d'argenio , il sorpo fino alle cofce di rame , le gambe di ferro, ed i] defiro

di terra cotta ; da quefta dice che {caturifcono le dette lagrime, le quali for-
Mano li detti quattro finmi Infernali , ¢ cosi la deferive nell’ Int. C, 14,
Dencro dal monte fra dritto un gran veglio ,
Che tien volte le Spalle in U Te ;

A% BBt BW BEERS. SS Sak

=

SAFE

E Ra-

 

 
260 MALMANTILE

E Roma guarda fi come {uo [pegtio ,
La fua tefpa ¢ di firt oro formata y
E puro argento fon le braccia, ¢ il , A
Bhd dl nec tcan eee — ah x Tan
Da indi in ginfo¢ tutto ferro eletta , 7 tae
Salno che’! deffro piede é terra cotta y ref 9
E fla in fu quel pinych'in fu U altro, eretto a
Il primo dunque di detti fiumi ¢ Acheronte , che ia un certo modo
privazione d’ allegrezza;da Acheronte na{ce Stige,che fignifica cofa difpi
odiofa , quale ¢ il Dolore ; perche guefto ne viene dopo la privazione dell’
grcaza , Ll rerzo é Flegetunte, che fignfica penfiero ardente travagliolo .
quefti tre fiumi fi genera il quarto, che ¢ Cocito ftagno,o fiume del
pianto. Quefta favolofa opinione de’ Gentili tocca Dante nell’ Inf, C, 14.!
tando i fopraddetti verfi. om

  
        
    
 

  

Ciafcuna parte , fuor che P oro ¢ rotta ity
D’ una fefsura , che lagrime goccia 5 ait
Le quali accolte foran quefta grotta , one

Lor corfo in quefta valle fi diroccia ae
Fanno Acheronte, Stige, e Flegetonta ; ihe
Poi fen va gik per quefea firetta doccia y ;

Infin la dove piit non fi difmonta, ~ alt
Fanno Cocito, e qual fia quello fhagno
Tu’l vederai, perd qui non fi conta,

 

 

ih

at ‘
CARONTE . Notifimo barcarolo dell’ Inferno. Vedi fopra C. a, flan.2g) |,
HEBBE tratto ognun da rina, Hebbe Jevate d' in fu ja riva tutte ! anime, im |
barcandole, hu My
TORVA fronte, B latino ufato da noi; E vuol dire Vifo burbero,afp toy
arcigno. Rs,
ANEMA via, Intendi huomo, che non fiamorto. Virg, 6, Em. Corports | (ty
vina nefas Stygia vettare carina . Sa bene il noftro Poeta,che anime fono immors | i
tali , ma feguita il coftume d’ intendere huomo viveate , quando diciamo animas jij
viva [ Genefi cap,2, Ee fattus eS homo im animam vinentem ] ed imita Dante !
C, 3. che dice; }
E tn che fei coft} anima vina , |
Partiti da codefti , che fon morti, |e
Ii Lalli Ea, Tr. C, 3. flan. 16. ea
E non v' é mai entrata anima vina y iy
GLI gettd wn po di poluere negli occhi . Gii decte un po di mancia, I Latini pute jp
difero : Puluerem oculis offundere, Es’ intende dar mance per corrompere il git %ir
fto , quafi diciamo: Abbagliare gli occhi del gindice con P aro 5 accivcche mom ty
inftizia . 194
£ pire il fapone . Exa flato fubornato , e corrotto con la mancia; Gli er: y

flare infaponate le carrucole (che yuol dire Tirar’ uno al noftro volere , erent
derlo facile a quel che noi bramiamo , ¢ fare che non ftrida contro di noi ) coma
dargit la mancia ; come con! iafaponare nna carrucola, o na ruora fi facilites

| y
:

 
 

   
 
 

  
  
    
    
    
 

261
cOlO 5 }y che non frida, Ed & 10 fteffo che getrar fa poluere negli ecchi
poce Dicefi anche: Vener le mam . Bocc, Nov.6. 2 bucno husmo por
i fece ngner le mani,
come il ranocchio. Obbligato a tacere, per havere havuta la
li faddetti due modi di dire , cio& Havere il fapone , e>
Qui non vorrei che ij Lettore credefle , che i] Poeta
‘egali potcficro corrompere i demonj, fe bea Ja fentenza
lice munera(crede mibi)placant homine/que deo/que,ma (apefie haver’
-moftrare che I' oro arriva a corromper quelli, che ne meno fi
¢ meno dovriano lafciarfi arrivar dail’ oro , ¢ finalmente ha vo-
Ja pofianza , che hanno i regali di far confeguire cid che fi vuole .
per pecuniam fattafune , Si racconta di Filippo Macedone, che haven-
9 riconoicere una fortezza’, ed cflendogli riferito , che era impoffibile il
la, domandafie agli {ploratori,fe vi era modo di farui andare un’ afino ca-
volendo inferire , che dove non potevano I armi , farebbe arrivato
5 uri Sacra fames,quid non mortalia pettora cogis ? BE Orazio. aurum per
ire fatellites , Et perrnmpere amat faxa potentius Iitu fulmines,
YSEL oechio. Finfe di non vedere. £7 il latino connivere, Vedi forto C.

  
 
 

    
  

      
     
     
      
       
 
   

 

 
 

Rieke fino al ginocchio, 11 proverbio Ca/char Le brache & i) medefimo ,
le braccia ; che vuol dir perderfi d’ animo, Omero: <dnimus in pedes de.

il cnore ; cafcd I’ animo a’ piedi , Onde dicendo , che coftoro have-
che fino al ginocchio , intende che eran loro cafcate affatto , cioé erano
> — @ animo , perch¢ doveano render conto delle loro azioni. Vedi
SINDIACT . Cosi chiamiamo noi quel Magiftrato , che ha |’ autori-
cr icontia tutti i Magiftrati, Ofiziali , ¢ Miniitri del dominio Fioren-

  

 
 

 
  

maligia . Specie di cipolla da mangiare , che é fortiffima , ¢ fa ve-
ime a tagliarla ,¢ maneggiaria ; Bocc. gior. 8.n.2, £ talora un mare
cipolle malige , o di Scalogni , 1 Lalii Eo, Tr. C. 3.
Cost dicea , ¢ tutto ii volto molle
Haves di pianto , come [e [chiacciato
> ¥ Vi fuffe fopra il fugo di cipolle.
COCITO , Vedi fopra alla flan, 19, alla parola sAcheronte, ¢ quivi troverai
ancora quel che fia Ja Palude Stige , della quale vedi anche fotto in quefto Cant,

3 Ibige ; Genti fcellerate , ¢ da non fe ne fidare. Per comporre il color
310 i Pittori mefcolano tutti i colori , ¢ lo chiamano il coler dell’ afino ; e perd
lic | huomo bigio s’ intende uno,che ha tutti i vizzj. Va moderno Poeta. ,
€¢ notammo {pra C. 3. ftan. 66. dific parlando d’ uno di quefti tali, che era

   
   
     
        
    

 

  
  

Y ; Chinde un’ anima bigia un corpo nero,
di quefta parola bigio in quelto fignificato ftimo , che nafca da quefto.
in Firenze pe foci paiisti tre ai , una de’ Fautori di Er, Giblins
= Sree ke Savona-

 

 
 

 

 

 
 
 
   
 
  

264 MALMANTILE =~

Savonarola , la quale era detta de’ Piagnons , I altea de? conttarj a detto Fr. Gite: |»
Jamo chiamata gli e4rrabbiati , 0 Compagnacci ; ¢ fta'di loro ¢1 teo'nimi- = |
ci, e difcordi , faluo che univano nel? effer contrarj alla terza fazione;, ¢
de’ fautori de’ Medici , la quale era detta de’ Padefehi , i quali non conue
ne con I’ una , ne con I’ altra fazione . Di quefti che inclinavano alla |
Patiefchi taluolta alcuno per fuoi fini particolari s' univa 0 con Puna , 0 con!
tra delle prime due , ma era ricevuto con fofpetto, che non fale)
ro deliberazioni , ¢ pero dicevano: Won ¢ da fidarfi di lore’, Son Bigi. E
ucftowforfe ha havuto origine quetta voce bigio in fignificato’ 4’ huomo da
fe ne fidare , Vedi la Relazione di Firenze del Fofcari , ¢ il Nardi nell
Florentine fib, 2. :
STANZA XXIL
Cli entrar dovendo in Dite,e faltn,e gira,
Che par quando mi barbera latrottola,
endar non vi vorrebbe , ¢ fi ritira
Grattandofi belande la collottola ;
Pur finalmente forza ve la tira ,
Come fa il pefo al grillo,una pallottola;
Cost ne van quell’ anime nefande
Chi dal piccin tirata ,e chi dal grande,
STANZA XXIIL

   

    
 
 
 
 
 
     
   
 
 
 
 
 
 
 
 
 

Perch gli é offa,e pelle,
Ch'ei par proprio il ritratto dello
STANZA XKV,

Per la gran calca nel paffar le porze Si che quand! ei fi fence il tow
Connenne a ognino andarne cola piena, Lerche la fame quivi ne c
Ma la Strega non hebbe tanta forte, Liingoxxa , che ne manco non gh tocca
Che trenla il can,che quivi Pain catena; Ne di qua , ne di la gih per 3
E perché per tre bocche abbaia forte, Ma fubico gli venne il, ,
Ella dice: Ti dia la Maddalena ; Ond’ei sallunga in terra afar lai

Chil papavero,e il loglioch

Faria dormir un’ orfo, non
ZA X&KVI ’
Sdraiata dorme; e riff com wh tf,

E in tatotrovail pane,e in pexsiltaglia,
E in tre gole ch’ egti apresgliene feaglia,
STAN

  
 
    

‘Hor mentre fa il fonnifero il [uo corfo,
La donna che piit la facea la [corta Lerno da bore fa verfa la porta y
(Peroccht havea timor di qualche morfo) E poi (benchrella fuffe alquanto [rect
Vedendo che Ja beffia , come morta Da unacor[aye in Dite anchrellait
L’ anime rimafte attorno alla Citta di Dite moftrano co’ gefti ,
lentieri vadano dentro alla Citta ; ma i loro peccati @ forza ve le
anime nell entrar della porta fecero cosi gran calca , che la Strega nof
flar con effe , € tanto pili , che ell’ hcbbe ‘paura di Cerbero’ ; ‘onde'pel
fene gli gett del pane fatto col fonnifero ; per lo che il cane’ fi ad
ella entro nella porta . E quiil noftro Poeta imita Verg.nel 6,
dare a Cerbero dalla Sibilla ana fliacciata’col fonnifero , e nelle pr
23. 24.25. parafrafa,fi pud dire,i feguenti verfi del medefimo
Cerberus hse ingens larratn rege i
Perfonat , aduerfo recubans insmanis in dntro
Cui vates borreve videns iam colle colnbris 22

          
    
 
 

       
 

 
 
 

SESTO'CANTARE:

2 9 © medicatam frugibus ofam
Aes, Obijcit ; ille fame rabida tria guttura panders
9 Corripit obietam , atque immania terga refoluit
5 Bafus humi , torque ingens extenditur antro,
Il verbo barberare é ufato da’ nofiri fanciulli per intender quan-
gira a falti, enon va unita per cagione dell’ efler mal contrappela-

  
  

 
   
   
    

LA, Strumento;del quale fi feruono i ragazzi per giuocare, ed é un
foggia di piramide, che fini(ce in una punta di ferro , Vedi fopras
fi fa girare avvoltandola con uno {pago,¢ poi feagliando 3 ter-

 
 
  

3.

14 , tirando con velocita a fe la mano , alla quale ¢ legato detto {pago .
_GRATTANDOS! ta collottola, Grattandofi il capo nella parte di dietro dai
tini detta cerwix . E quefto ¢ un’ atto folito farfi per lo piti dalle donne , ¢ da’
ndo hanno qualche di(grazia , o gran difgufto . Vedi fopra Cant. 3.

 
     
  
 

  
 
  
 

VDO . Vale piangendo : perché fc bene il belare ¢ proprio delle peco-
li , ¢ viene dalla voce , che fanno tali beftie , che fuona be be , ce nes
lamo anche per efprimere il pianto dell’ huomo , ma per derifione ; donde fi
belone , pecorune a uno che pianga affai. Vn moderno Pocta diffe ;
Hor ch’ é per te finita la pafciona ,
Gf Che fai che tu non beli , 0 pecorona?

GRILLO. E' un verme piccolo volatile noto: Ma trattandofi di pallottoles
Grilles intende que!la piccola palla , che fi tira per fegno nel giocare alle pallot-
tole ,0 alle pialtrelle ,omurelle , Vedi fotto in quefto C, ftan.34, e C.9.ftaa.17.
PALLOTTOLA, \ntende quelle palle di legno , che feruono per giuocare. , j
nelle quali fono tre contrappefi di piombo , per via de’ quali fi fanno fare alies ?

F operazioni , e voltamenti , che fi vuole , ’ uno di quefti fi chiama. i
¢atetia , V altro il grande ; ed il terzo il piccino, ed il Poeta, aflomigliando quell’

        
  
   
 

=

ASSES EES:

 
  
 
 
 
  

  
    
 
 
  
  

 

ge  *himea queffe pallottole, dice, che ancor’ efle fon forzate a entrar nell’ inferno \
dal ,¢ chi dal grande , cioé chi da i peccati piccoli , ¢ chi da i grandi .

ye! Quantita grande di popolo ; folla .

5 ANDAR con Ia piena, Andar co’ pid ; andar in truppa con tutte quelle anime,

m lie piema per fimilitudine fignifica inondazione, o furia di popolo . Virg. Georg,
i lane falitantum rotis vomit adibus undam . Andar con la piena fignifica ancora {e-
,g  Biitat Popinione comune ; andar co’ pill .
ee AL Cane che quivi fa in Catena, Cerbero cane con tre tefte,due delle quali ftan-
4  nolempre fuegliate. Hercole lo lego , ed i) noftro Poeta imitando Vergilio co-
‘ me s'€ derto , lo fa addormentare col/pane alloppiato .
4  _TTdiala Maddalena , Pofla tu cfler impiccato . Dicevafi porta di Caronte da-
m gli Atenicfi quella porta del Palagio del Podefla, dond”ufcivano coloro , che an-
;  davano alle forche , come accennammo /opra C. 5. tan. 3. ¢ noi diciamo Ti dia
ta Maddalena , da quella Campana , che ¢ nella torre del Batgello , la quale fuo-
ma, <n va alle'forche , e fi chiama la A¢addalena , perché con tal no-
mee » per efler 1a Cappella di quel Palazzo forto i ticolo di’S, Maria

GLIE.

.

 
oT

 

 
   
      
   
 

264 ’ - MALMANTILE:
GLIENE feaglia, Gliene tira da lontano ; Glien’ av
ra nen {egli volle accoftare. ,
HAVEKIA mangiato Salerno, Haurebbe mangiato i fat, Ve
diffe; fume rabida . E fi trova Satylnm voraret , che batylum chi

piecra , che fi divord Saturno , sDEDy
SER faccents , Si dice : Ser faccenti , o Barbaffori quafi Valvafori,
dajc,a coloro , che tutte le cofe fanno , ¢ dicono magiftralmente , ¢
degli altri ; E’ perd detto (cherzofo , ¢ per burlareuno. Qui intende:
natori dell’ Inferno. E’ parola derivata dail’ anti.o verbo /accio, per,

   
    
    
  
       
     
 
    
    
    
   
    
   
  
 
     
 

 

  

  

Lopio'g > vA
PER il mal gaverno, Per il poco mangiare , che gli danno... Nell
Governare le gailine ; cioé dar loro da mangiare. Similmente i La
{oidati pigliavano un poco di rinfrefco,dicevano ; corpora curare,
Governare gli uliyi difle Pier Vettori,cioe concimargli;quafi quefto fia
Si ferutto che tien ! anima co’ denti, Si macilente , ¢ magro y.che p
lerebhe ? anima, fe non la riteneflc con lo ftringer i denti . Giobbe pe
fe medefimo emaciato , ¢ confunto . Pelli meae , confumptis carnibus y
meu , ce 3
EGLI é ofa, ¢ pelle. Non ha carne addoffo: E’ magriffimo, Plauto d
quefto propofito Offa , atque pellis. E Dant. Purg. C.23. dice;
Wegli occhi era ciafcuna ofcura ,@ CAVA s
Pallida nella faccia , e tanto fcema y
Che dal! offa ta pelie s* informava ,
SPENTO, S intende al maggior fegno magro.
LA fame lo {canna , Muore di fame. Vedi fopra C, 4. flan. 24.
CANNA., Intendi la canna della gola, la quale fi dice canna per.
ne , che ha il gargarozzo con la canna, Dan, Inf,C, 28, f
Reftato a riguardar per meraviglia she, *
Con gli altri , mnanzé agli aleri apr} la canna s cnt
Onde Scannare , sgozzare : tracannare., ingollare , pha
GLI wiene il fonno in cocca , Cioe nell’ eftremita delle palpebre 5 che vengonos
chiuderfi , Gli vien voglia grandiffima di dormire . one
S* ALLVNG A in terra, Si diftende in terra, Lmmania terga refeluit Fufus hima
totoque ingens extenditur antro ; dice Verg. com’ habbiamo accennato fopra.
A FAR (a nanna, A doimite . Termine infegnato dalle Balie it
imparano a parlare , per efier pid facile a dir nanna , che dormire, Lala N.
Non lajcio mai certi detti che haveva imparato da bambino,chiamando pappo il
vino bombo , i quattrini dindi , e quando voleva andare a dormire , diceva
ta nanna . | Launi fimimente !'addormentarfi de‘bambini alla Ninna
tilena delle Balie , da lor detta Ladus , ¢ da’ Greci Wynnini ; dicevano La
MENT RE il fonnifero fa il uo corfo, Ul fonnifero fa la {ua Operazione +
PAP AVERO ,¢ Loglio, li papavero é quell’ erba , il feme , ed eftratto
quale compone |’ oppio , 0 fonnifero ; ed ii loglio € un’ erba , che nafce fi
ni , il feme della quale mangiandolo , dicono , che faccia sbalordire, ¢
no. £ da questi mali effetti del Loglio habbiamo un proverbio, che

  

Lad

   

 

       
 

 

 
  

oo CANTARE?.
fignifica Io non fon balordo.
fopra C,3.ftan. 32. /draiarfi é il verbo recumbere ; E Ver-
u pacule recubans fub tegmine fagi; ftimo che intenda sdra~
te ne ftai all! ombra d’ uno fpaziofo faggio . E nota,
, che vuol dir largo, 0 {paziofo, ¢€ ftaco cavato i! verbo
I eas » ¢ paffare il tempo fenza penfieri, il che chia-

tifmo afiai ufato . *
. Ronfare: Quel romore , che fi fa da molti nel re(pirare dormen- :

  

 

   

ae;
da bette . Vuol dire accoftarfi. Perché le doghe , ¢ I’ altre parti
‘botte fon lavorate in modo, che fi compaginano,, ed unilcono ,

STANZA XXVIIL.
S' ell é come voi dite a quefto modo:
(Fi le rifponde ) andate pur madonna,
Perch’ altrimenti c* entrerebbe il frudo,
E voi fharefti in gogna alla colonna ,
Horsit correte pria che freddi il brodo

 

   
  
 
   
   
 
     
 

dice) ola , che roba ¢ quella?
i ( dic’ ella) nel forame ,

   

non ho qui roba da gabella, Che la Regina poi farebbe donna
— Se non un po a’ alors’ a Proferpina Da farci per la frizza , ¢ pel rovello
Porto , perch’ ella fa la gelatina . Buttar’ a’ pit la forma del cappella,

za havea forto alcune rame d’alloro; ¢ da i gabellieri le fu doman-
+ ma effa con dire, che era per feruizio di Proferpina , fi libera,
a del gabellicre . 11 Poeta imita Vergilio , il quale fa che Enea d’or-

Sibilla porti a Proferpina il ramo di quell’ albero con le foglie d’ oro ,
Come fi vede al lib. 6, dell’ Enetde .

—

 
  
  
   
       
 
  

 

Latet arbore opaca
$ Aureus , & folijs ,& lento vimine ramus
iis Iunini Tnferna dittus facer ,
HELL. Quella cofa mala ; cio’ la (pia.
> (A della fame . Ha grandiffiima fame , perché non guadagna denari
omprar roba per mangiare. Quando i meftieri non lavorano fi dice : i /egna-
, alzolai , ec, arrabbian della fame , cioé non hanno da lavorare .
erai il forame. Per befiar’ uno , che dandofi a creder d’ haver fatto
gno a [pele , ¢ difpetto noftro , e non tha facto, diciamo: Tx ti
. Qui vuol dire: tu credevi di haver guadagnato il quarto, che
sca alle {pie, ma non ¢ flato vero.
»1W.A, Fu figiivola di Giove , e di Cerere, 1a quale fingono gli an-
» che efiendo un giorno a corre i fiori,futfe rapita da Plutone Re del?
» 5 fatta fua moglie : Ma Cerere non potendo comportare,che la figliuo-
imanefle apprefio ai rattore ; fupplicd Giove, che volefic Jevarla dall’lnterno,
eae le , pur che ella non haveffe prefo cibo alcuno; Ma havendo
pina mangiato alcuni granelli di Melagrai:a non potette ulcire ; Cerere di
0 fupplicd , ¢ ftimold tanto Giove , che ottenne, che Profperina fteile fei
‘dell’ anno nell’ Inferno con Pintone, ¢ fei mefi con Ja madre in Cielo. EB
. con

 

     

   
   

  

 

 
 

 

366 MALMANTILE ©

     

cosi Praferpina ref {ei mefi in Cielo , dove ¢ chiamata Luna’, @ fel

  
  

ferno, dove ¢ chiamata Proferpina, ed in Terra é chiamata Diana,
£

triplicata efienza Verg. diffe :

Tergeminamque Hecaten, tria Virginit ora Diane,
E perché la Luna fei mefi dell’ anno crefce , ¢ fei mei feema , perd'i
till finfono , che ella fteffe {ci mefi in Cielo , € fei mefi nell’ Inferno ,

  

 

+ teeth die

goers

 

no splenda in Terra , ed ¢ detta Diana . A quefta finzione allude Dany

Ada mon cinquanta volte fia raccefa
La facia dela Donna che qus regge ¢
GELATINA, Brodo fatto con la carne di porco , € rapprelo; ¢ fi fa
‘ i)

     

brodo di pefce . Vedi fopra C. 2. flan.

2. flan. 15.

STIZZA, Ira, Vedi fopra C. 2, flan. 78, al termine fw piccine, Era
velo , collora , ¢ fimili , i potiono dir finonimi di ftizza , quando é prefa in
fenfo ; che per altro diciamo fizza Vna {pecie di lebbra , che vien€

ad altre beftie .

S:AREBBE derma. Quefto termine significa Haurebbe animo : Si farebbe
to , ardirebbe , non la guarderebbe , ed ha lo fteffo fignificato , che Son | i

detto fopra C, 4, ftan. 29.

BVTT AR 4: pie la forma del cappello , Cio’ buttar 1a tefta a i piedi j) Teoncare

il capo , che é la forma del Cappello .
STANZA XXIK

La Maga fenza dir piit da vantaggio,
Metr'egli a/petta un po di miacia,e intuona;
Ripiglia prontamente il fuo viaggio
E incontra Nepo gid da Galatrona ,
C” havendo dato Id di fe buon faggio,

Jn ongi ¢ favoritoye per la buona ,

Perché Breuffe in oltre a’ premi, ¢ lode

L: ha di pin fatto Diavolo a due code,
s tA NZA XXX,

Hor che gli arriva ail’ improvifo addoffo
4 venir della maga ch'e il [uo cuore,
Lui Mago pur tagliatole a fus doffo
Le [pedifce per fuo trattenitore ,
Alentr’il petardo col cannon pi groffo
Sentefi fargli frrepitofo onore ,
Cavalier Nepo, com’ io diffi dianzi,
Col riverirla fe gis affaccia innanzi,

C'ENTREREBBE il frodo, Ci farebbe la pena d’ haver frodata;
nifeftata la roba , per non pagare il dazio , 0 gabella. |
LN gogna . Alla berlina , che ¢ quel gaftigo vituperofo , che dicemhmo:

   

veh a

     
    
 
  
   
   
  
   
 
 
    
   
 
   
   
   

ira

its

STANZA XXXL
E perché 4 Benevento eff com luk,
Com! ¢i di lei,bavuto havea
Won prima fi riveggon ch’ D
Rifanno il parentado,e 0 a
Tra i diavoli poi vin nei ’
E percht Martinazza v' ¢ novitidy
E non intende il gracidar che @ fam,
L interprete fa egli , ¢ il torcimanits
STANZA XXXIL
Per via informa , ele da molti
Diufanza,¢ lyoghi,e intanto di bua es
Lo guida ai fortumati Campi Elifis
Dove fi mangta,e beve hi we
E tra quei rofolaceé , € fior
Si peat tong far diquatcroeaaty

Chi un baloccose chi ut z
Che li non ¢ un negorio per | if

 
 
  

     
 
 
     
 
 
     
      
  
 
 
      
 
    
 
  
   

CANTARE

PV paar ar XXXIV.
 Quivi fifa al palione,¢ alla pilerta ,
| Parte ne ginoca al Suffise alle Murelle,
Conte carie a Primiera un'alerafroca
A confartini ginoca ,¢ le ciambelle ,
etri fanno a Ci é alla lotta;
indovinells.,¢ chi novelle; (gio
7 lie fiorienn'altrounramo aun fag-
ths cagliata , ¢cou effe canta maggio .
A XMXXV,
Altri pigha , 0 difpenfa del tabacco
Altre piglia le mofche , un’ altro grilli
E cates quanti in quei traftulli immer fe
Sirengonail tenor y(t vanno aiverfi,
Za iL Ȣ-s?incontrd ia Nepo da Galatrona molto
o da Plutone’, il quale per fare onore a Martinazza da Jui tantoramata. ,
i ptrartenitore 5 apendo cheerano amici-, Cost dunque.ac-
a Nepo sche de faceva J’ interprese , perche ella non intendeva il
voli , fe ne pafsd ne i Regai-bui ; edi) primo Juogo, che ved-
; ono. Campi Gli ;,¢ quati il Poeca defcrive ripieai di quei trattenimen-
; ¢ fanciuilelchi., che (on-foliti facli dai bottegai pili vili per le feftivica
1 fubucbani,come fono le Ville degli Strozzi , Pucci , ¢ Gerini , doves
f pola per godere allegramente , ¢ {eaz’ un peofiero al mondo quel-
fa,che concede Ja campagna-, ¢-fofpendere alquaato i penficri aviofi del 1

|, | hvorare

wy

  

 
 
   
 
   

  

  

IL AMANCLA. Vedi fopra C, 2, an, 68.
bh ANTVON ARE, Vuol dive dac priacipio.al canto; Ma qui fignifica chiedere
y/ —$ON inetti , © cennila.maacia ;.¢ ci ferue per intendere domandare con ceani, o
o i quaifivogiia cola : per efempio: Ll talc insuona , vorrebbe andar’ a cena ,
a ta bostega , ec.
o ‘aiatrova’, #uvuno nel contado di Galatrona luogo nel Valdarno di
gt »9. conpolueci fimpatiche , o.con altro medicava tutte le ferite , ¢ 1
yl} buomini,, come di beflic,fenza vedere il paziente ma folo'ia fu le» \
y@ = pezzebagnate nel faaguc di ello, o fopra ua panuo, che havetfe toccato lo Rrop~
IL PiO} per le-betic in qualfivoglialor malore , pigliava la Joro cavezza’, o bri-
ge ia » ¢ fopra quelli diceva alcune parole, ¢ le medicava ; ¢ per que-
‘i fua lica fuperflizione da molti fu ftimato ftregone, come lo ftima il Poe~
i seers. ioe eonofeiuto con Martinazza a Benevento , ¢ che era mago
afuo-doflo.
é DAR bran fore di fe .Pacfi conolcere con le fue azioni-per huomo di garbo ,.
wt = © prudenre, o-virtuolo.
eo ‘ER (a-buona,. S? intende,t per la-buona: firada ;\¢ vuol dire .. B’ in\ buono
yf Mato; Gitira innanzi bene ..

|, BREPSSE.,. Intende Plucone ; ed & Jo fteffo',.che la: Bilisrfa y colla’ qual voce»
inno paura le Balic a’ bambini, furfe dal Lat. &rebus , originato cost; Erehufe, y
le. Me uv Lies

—S

 

 
 

 

   
 
  
      
    
   
    
   
  
   
    
   
   
       
   

# I er
oR

oMALMAN ore *

 

268

   

addoffo una’ lucertola ‘con'due code fia fortunatifimo in :
larmente nel gitloco , ¢'percid vuol dire , che quefto’ 0 atiffi
grandemente privilegiato da Plutones percht haveva le due code }! =)
GLl wrriva addofo , Cio’ fopraggiunge inafpettatamente a Plutone''t
Martinazza tanto amata da lui.) > nImAS a8 : /
T AGLIAT OLE #'fuo dof, Fatto per appunto come lei, che havi mede
nj , ed inclinazioni , che ha lei . Traslato da’ gli abiti 5 che fi dicono sagtiati 4
doffe quando tornano bene in doflo . e 2S arand on
TRATTENITORE , Si dice quel Cortigiano, che vien di a ferniret
Ambatciatore , 0 altro foreftiero , che fia ricevuto , ¢ {pefato'dalla Cortew >
PET ARDO, Specie d’ artiglieria nota , che ferue per'buteare a terra 5 1¢ por
te delle Citta. In Latino fa detea da Famiano Strada con' voce Greta
Pyloclaptrum, Quafi Spezzaporta . op i» 2001y(4 ob oats
RIE ANNO il parentado, et amicizia, Quando due amici flati ‘lango!
Jontani I’ uno dal’ altro fenza vederfi , fi ritrovano infieme, ¢ fanno le:
diciamo ; Rifare il parentado, ¢ P amicixia, NOVICE Tt
VB novizia, Non v’é pratica , perché non v' é mai fata in qu
hofpes , € noi lo traslatiamo ad uno , cheé nuovo , ¢ non praticato in
affare. Lat. nonus, rudis, i ages oe
GRACIDARE, E' proptio delle ranocchie ,'ma'qui intende il
voli, che forfe fe lo figura come quello delle ranocchie’.’ Dan. Inf.
E come a gracidar fi fia la rana,
INTERPRETE , ¢ Turcimanno . Si poflono dit finonimi,{e non che Znterprete¢
propriamente quello, che efplica i fenfi delle parole , ¢ Turcimanno & ¢
parla in vece di coluj , che non intende il hnguaggio,riportando le parole; che»
fente dire nella lingua dell’ uno , ¢ dell’ altro refpettivamente , Da alcuni dicefi
Dragomanno dalla voce Greca Dragomenos , che fignifica Znterprete ufata da’ Greti
Orientali de* tempi baffi; da Tbargum , che in Levante fignifica interprerazione,
Thirghem in Caldeo vale ¢/porre jefplcare, e da quefta radice ¢ detta a
Thargum la Parafrafi Caldea della Scrittura. Ma hoggi Turcimanne da i pill q
tende ruffiano da quel portare le parole . BH
DI buon trotro . Paaiainaids di buon paffo . Trotto diciamo una {pecied’ an-
dare del Cavallo, che é fra il paffo ordinario , ed il correre , ed é il latino ie:
cajare. = .
eal Elifi., Bil creduto Paradifo de i Gentili. Vedi fopra C,2. flan, 68:
e4 BERTOLOTTO. Senza penfare al pagamento , che fi dice anche 4 Vf
a Youne ; a Scrocco; a Salicone, Vedi fopra C, 1ftan.77, ¢ forto C, 7, ftan. 5.
ROSOLACCH , e foralife . Specie di viliffimi fiori Aloette A
PAR di quattro , ed’ orto, Seiben par-che voglia dire giuocare inuitando di
quattro , ¢ d’ otto ; tutra via s’ intende fMarfene fenza far nulla , che fi dice: :
‘ar ateco mec , dondolarfel4 , farea tn me gli hai, ondg un noftro Poeta moderno —

 
  
 
 
   

    
    

  

  

   

       
    
 

  

H ae ‘
SESTO CANTARE:

ative. errno al mattarin crepufcolo
Weddin — me gli bai,
nathan ‘oponete , a concludete mai ,ec,
oe vine Trattenimento «Da Badalueco, che vol dire pro-
: ia.» © leggicre combattimento . Latino velitario , ¢ figurata-
» © trattenimento piacevole . Ma la parola balocco , © balocarfi &
bambini ; e nel contado é prefo per indugiare .
' grandifiimo, quafi dica {paziofo tanto quanto un’ occhio ¢
pO +
UETT£,. Diminutivo di mucchio, che yuol dir quantita di cole riftret-
» quafi monticelletti, Latino cumuli  acerni ; © Cosi mucchietti di gente
{ d otto , 0 dieci tare riftrette infieme. Dan. lof, C, 27.
| B di Prance{chi fangainofo mucchio
i _ » Sorte le branche verdi fi ritrova ..
pure il mondo in carbonata.. Diventi carbone , ¢ abbruci pure il Mondo,
i, ¢ vadia fottofopra il Mondo .
un faftidio di niente .. Non vuol fentir noia , 0 pigliarfi penfierc
che fi vuole , o dibene, o di male,
. Ballare fenz’ ordine , 0 regola. Vien forfe da Ballunchiare
»» chefe bene é parola non ulata ,pur |’usd il Boccaccio Nou. 72. pe:
ballo di contadini . :
Strozzini.Gii Strozzini,come habbiamo d., una villa de’SS.Strozzi po.
a da Firenze,cosi detta.Si come.il Prato del Pucci, ¢ del Gerini fono due
aburbane.de’ SS. Marchefi Pucci , ¢ Gerini ; a’ quali luoghi, fuole I’ eftates
plebe Fiorentina'a {paflarfi , con far merende, balli, ed altro,che le tor-
o,come dice il Poeta nelle prefenti Ottave .
pallone  ¢ alla pillotta , 1) pallone ¢ una groffa palla da giuocare fata di
€tipiena di vento , alla quale fi di con il braccio armato d’ un bracciale
nO: ela pillotta ¢ una palla piccola pure ripiena di vento , ¢ fe le da con
a di legno . Quefii giuochi di palla, fono antichi , perch¢ fecondo Pli-
% 59. furono troyati da un certo Pytho. Herodoto lib,.1, riportato da
slid. Verg. lib. 2. cap. 13. dice , che I’ inuentafiero i Lidi. Alea verd teffe-
» farumgue ludos , & pila, cateraque luforia recreandi animt gratia inuenta_ ,
a» preter quam talaria , Lydi populi Afi omnium primi, cxcogitavere &c. Ac-
-» qui Lydos ciufmodi aleatorias artes non tam voluptatis , quam .compendij,gra-
_» Ua excogitafic idem Herodotus tradit , nam cum gravitate annone patria tem.
» pore Atydis Manis Regis filij premeretur , fic famem confolari foiebant , alte-
$9 £0 quidem die cibum fumentes , altero ludis operam dantes ; atque hoc modo
, -% inediam folantes vixere annis duodeuiginti, E da’ popoli Lyds alcuni voglio-
D0 , ficcome é Ifidoro nelle Origini , che venga la parola Ludus , 0 Ludius, che
lo fteflo , che Iftrione . E ognun fa, che i Lidi dal’ Afia pafiarono in Italia, ¢
popolarono I’ Etruria, ovvero Tofcana ; E da loro i Latini le cirimonie facce,.
dudi , che fi domandavano /cenici particolarmente appreero; EB Hifer in lin.
ica, onde ¢ detto Jfrioni fignificava in Latjno Ledio ficcome dice Tito
3 Poi quefto nome /udus fignificante a cae fpettacolo attenente, o far.
a. m2 to

 
     
   
    
   
    
   
       
   
     
     
   
     
   

 

 

 
 

 
   
  
     
    
   
    
    

aye MADMANTILBES ¢

to per canfa di religione , fi ftele a fignificare: in generale 0
ib 1, ¢ Suida dicono , che Anagallide Gramatica diCorfl i
mento della faltazione a palla,-cioe del gi alla palla at
a Naafica figliuola d’ Alcinoo Re di’ Corfu'y wolendo fare quefta
il vanto d’ una tale invenzione a/una faa paefana , & veramente Naufica:
» Del reflo Di

trodotta fola tra ' Eroine da Omeru a givocare alla palla

attribuifce queft’ inuenzione a’ Sicionj , ¢ Hippafo altro Autore citato da,
a’Lacedemoni,come ache tutti gli altri corporali efercizzj-E che-futie mol
to dagli Spartani , o Lacedemoni lo mottra Properzio in quel vero
veloct fallit per brachia iatiu , delY Elegia che cominicia ,) Atuita tna\, t
vamur inra paleftre, Dal che fi viene in chiar, che il giuoco della;palla fia ante
chiflimo 5 ¢ fi pud credere col Soutero de Jud, Veterum libs 3) Ci 14. e! id,
Verg. lib. 2. cap. 13. che quefta'variazione d’ origini proceda dall’havere havuto
gli antichi diverfe ipecie di-paila , i come habbiamo noi’, ¢ che gli accennati ia-
ventori habbiano ciafcuno ‘inuentata Ja fua fpecies perché:fe noi habbiamoiil pale
» lone ; i Latini havevano, ipfe follis, pila , & ipfis genus ; conftarque V

3» to inflata. Habbiamo Ja pillotca’, & eff ib follicuius , pilay 6 ipfa parva, &
»» fimilicer conttat aluca vento inflata. Simile a queftae la palia i

in-vece d’ efler ripiena di vento ,°¢ ripiena di borra’; Ja’ qual palla hoggi per lo
pil € ufata da i contadini , © quefta havevano anche gli antichi ¢ la diceyano 2-

fa paganica, F 3 Syaapedt
Marz. lib. 14. Hacyqna difficilis turget paganica plama y— . bd
Folle minus laxa oft, © minus artta pila ; ee °

Habbiamo Ja palla fimile alla bonciana’, ma aflai minore ¥ che chiamiamo pala
defina , che pure ? havevano anche \fecondo alewni i Latin, © la dicevano Pila
fixentina , perche forfe nel pacle Fiérentino fi lavoratiero le miglioris Habbiamo
Ja palla facta di cenci impentita , che i Latini pure havevano , e- Ja chiamavand
co’ Greci Phannida , 0 vero Harpafium , pesche te ne feruivano per far il gvoto
y» da noi detto il Calcio fecondo 1] Sipontino , che dice , Harpattumy pile genus
x» elt; grofhor , quam pila paganica , tenuior , quam follis; E panno fere fit,
aliquando ex pelley lana , «omentove impletur , Non repercutitur , fed cum,
» multi fint Judentes in duas partes'divifi , ita ut utrique &-regione fibi inwicems
oy Oppofiti fine , ad fuos quilque tran{mittere pilam constur y quam aduerlari) co
y» Hantur arripere; Alarpajtum diem a Giseco Aarpayin , quod eftirapere, quia
3» proietam pilam mulu fimul conantur ariipere , fed ob cam caufam inuicem
> profternuntur , ve
Marz. lib. 7. ep. 31. Won harpafta vagus pulnerulenta rapis , A aie
Habbiamo la paiia a'corda , che terue per giuocare con‘la-raechertasnelle Manze
fabricate per tale effetto ; ed'etli havevano pram trigomalemscost decta ‘non perché
futie di figura triangolarejma perche era triangolast Ja ftanzajéove conefa
cavano , ¢ per dare a quefta pallayfi fervivano del rericixd y che & JotieNOy chet
racchetta 5 0 laccherta,come accennammo fopra C.3. tan, 58. -Di quefta lacche
ta parla Ovid. lib. 3. 1.4 M
Reticutoque pile leves fundantur aperto , - nt
Nec , mfi quam tollas , ulla movenda pila off .

ss

         

  

 

BB aeFF FEES SEH TF ROPSCFARRP xf RE MASTS SOW HEELS Sees.
SESTO CANTARE.

tebe wit DARE Dawe
tepidum dextra , levaque trigonem .
tichi wfafie la palla-ripiena di borra od‘ altro pelo, fi cava
; tino riportato qui fopra, ¢ dal nome di efa , perché
» che fia decta Pia dal pelo,col quale é ripiena ;.fe bene altri vo-
wenga dal Greco Pefeo:, ideft equo , perche € di figura sferica , che &
ogni parte , 0 pure ( i] che € pil: verilimile) dal pee rh cioé
ibrata , ¢ sbalzaca , ¢ percid anche in Greco, fi come in Tofcana é det-
Dionifodoro antico gramatico , dove nel tefto deil' Viiflea co-
leggevafi Spheran , col qual nome chiamano i Greci da paila; fi di-
i (criveile Patian. come per chiofa, e interpetrazione della voce d'Ome-
queflo vien riferito da Euftazio , che fopra quel Pocta il gran comento
, Che i Greci ancora haveflero motte fpecie di palle,fi pud dedurre non folo
cfere ftati inucatati i giuochi di palia nel tempo , che fiorivano i Greci, es
| flo di oro Ja Spheromachia , |’ Amilla , ed altre fpecie di giua-
ileriti da Giulio Polluce , ¢ dal #ulengero; ma da quello., che {crive
ino lib, 20, C. 14. dove dice,che fra i Greci giuocavano alla pallas
huomini , che le donne ; ¢ cid cava da Homero. Si trova in oltre , che
‘Siracufano giuocava alla palla , ed alla pillotta per ricuperar Je forze .
ex ab Alex. dier. gen. lib. 3... 21. £ fi pud credere, che fi come noi habbiamo
‘diverfe palle , e\diverty modi di giuocar con effe , cost non mancaffero a Joro an-
“coral inuenzioni per foddisfart .
| AL foffi,£! wn givoco folito farGi per lo pit da ragazz’ in quefta maniera . S'uni-
_-eono dues pit ragazzi,¢ pigliano una pietra, € polatala per ritto in terra vi
c ra quel danaro , che fon conucnuti di givocare , ed allontanatifi in,
liftanza ,che fono-d'accordo , tirano una jaftra per uno ordinatamenie
q i¢tra ritta;fopr’alla quale {ono i denari,¢ che fi chiama il Suffije fe que.
A ‘Wien colpico , ¢ fatto cadere,i danari,che cafcano,fono di colui , la lattra.
del quale ha fatto cafcare il fut , fe perd fono pit vicini alla fua Jaftra , che al
- fal moneta , cheé pili vicina al fufli,te gli rimette fopra., ¢ quello a,
cui) ira, ¢ feguitano,come fopra , tanto \che la moneta mefia fopra al ful
“tei finita'di ievare ne) modo , che sé detto . Da quefto giuoco: babbiamo un,
Proverbio che dice Efer w/uffi , il che fignifica efier queliberzagiio, dove ognuno
“tira ,cio€ fopra il quale devon cadere tutte le burle ,-c tore Je minchionature. .
 Quetto giuoco & forfe lo fteflo , che da’ Greci era detto Epbedri/mo, feconge Giu-
Tio Polluce , il Buieng. c. 48. , ed il Meurs, de lud. Graecor, , te bene non ginoca-
~Vano denari, ma colui, che non butcava in terra ai futh,portava a cavalluccio
» quello,che to bucrava,il quale gli turava gli occhi colle mani, finche (enza errare
bb portale alla Jafira , 0 pietra , che fi chiamava diores , cloe Adeta 0 Confine,es
_-facevarquello , che comandava il vincitore , il quale in quefti loro giuoch era,
hiamato'Re , ed il perditore era detto Mida , 0. vero Afino , come habbiamo vi-
“flo aitrove .
_ | MPRELLE . E? giuoco fimile alle pallottole , fe non che in vece di palle ado-
oo eee laftrucce , ed un piccolo faflo per grillo , ¢ tal ginoco fi dice anche pia-
eae wee «®

 
 

27 MALMANTILE? 4

PRIMIERA , Giuoco noto , che fifa con le carte. = =)
FROTT A, Flotta,, o fiotta . Vuol dire quantita di gente unite i

    

muove ; dal Latino fluttus , Virg.Georg. dane Salucantum toris vomit edibus andi,

Varchi Stor.lib. 15. Z vedendo fopra a un monticello non molto
frotta di. contadins . DEIN
CIAMBELLE , ¢ confortini , Sono {pecie di pafie fatte col zucchero.
uova,e quefle fon poe a vender da alcuni pi pel contado, dove fi
¢ raddotti,che in Citta; ¢ quefti portan feco anche le carte per giocare,:
quali hanno diverfe inuenzioni di ginochi, come la mora, il tocco , ec. E
venditori quando giuocano , danno in vece di danari quei confortin’ 5 ¢ cis
fe perdono ; ¢ fe vincono,ricevono danari. L, circali , ernfiula. /
CIVETT-A , Quel giuoco fanciulle(cho , che dicemmo fopra C. 2, flan. 41.
INDOVINELL1, Latino griphi,enigmata;Quello che in latino dal greco i
enigma , noi circofcrivendolo diremmo detto ofcure , ¢ diffeile a i
E la voce enigma s'¢ fatta Tofcana , ¢ |’ ufiamo come I’ usd il Malatefti nellay
fua Sfinge , Vedi eras 8. flan. 26, a vege!
CANT A Haggio, Nel principio di Maggio fogliono le Ragazze plebes
di Firenze , o del Contado sbeceane scomeni foe Pr) eee ¢
di joro-in mano un ramo d’ albero adornato di fiori andar cantando ere
diverfe canzonette per I’ allegria del nuovo maggio, ¢ per bufcar mance da
ro , che fi pigliano i) paflacempo di farle cantare al fuono d’ uno ftrumento.
cembolo , che é un’ afficella ridotta in cerchio , ¢ fondata di cartapecora da una
parte fola a guifa di tamburo. Quefto coftume di rallegrarfi il Maggio viene dal’
antico , ¢ fi trova , che appreffo i Romani Xalendis 5 Nowis , @ Zdsbus maij Lari
Deo facra ficbant afello panibus coronato , Quindi forfe ancora Maggio fi chiamavdl
mefe de gli Afini , che per altro fu detto men/is hilaritatis, Che nel mefedi Mag:
gio fi faceffero allegric forle pil di quello, che comportafie ’ onefta , ¢ lavert-
condia, ne fanno fede gl'Imperatori Arcadio , ¢ Onorio nella loro Coftituzion
inferita. da Giuftiniano nel Codice lib. 11. 45. de maiuma, la quale era una alle
gria , che fi faceva per il Maggio fecondo che fpiega Suida. Da quefto mele quel
ramo d albero, che i contadini piantano la notte di Calen di Maggio avantiall
uicio. delle loro innamorate , fi chiama Adaio; Quefto coftume d? appiceare:
maio alla cafa della Dama ¢ riferito come proprio anche della Francia da Mat
ziale d’ Aluergna ne’ fuoi Arrefti d’ Amore , all! Arrefto quinto , il qualefcritt
re fiori nel 1400. Qual Inogo Benedetto Curzio comentando dice; Prima d
Maij menfis invenes pluribus lndis , ac iocis fefe exercere confueverunt, arborem fapean
mero deportantes , ac in loco publico , aut etiam aute alicnius egreci virt januam 5
frequensiits amica fores pl vel promifinis adamantil
Signijs , atque emblemucibus .

isarote
BRANCO., Quantita di popolo indeterminata ; ma fi dice pili di beftie; com?

branchi di polli di pecore,di buoi , di aGni , ec, Vedi in quefto C.  Ortawa
feguente . 4 sacl
HA mofo} Ofte a facca. Ciot mangiato , ¢ bevuto quanto I Ofte vi haveva-»
nel modo » ¢ con quella furia , che fegue nel dare il {acco a una Cittas ae
fopra:

(EN-

MEZZL brilli , AMez2i briachi , Brillo yuol dir briaco allegro. Vedi &
2, flan. 69. At

 
  
   
  
  
  

  

Bee coc eee e2@Heekse ewer es) =!

 

eseF SF AS Soa

 

 
   

273
a bacco, Vna villantella che fi canta per incitare
Aidlors eg MeL gave

»\\) Faceiam brindis a bacco , © vi :
quefta,va il bicchiere attorno , ed ognuno’ beve,intuonando prima.
perd dice mentre ice 3 cioé mentre il bicchiere va a tor-
/perché tal-coftume é ufatifimo in fimili allegrie , perd il Poeta , che s’ in-
moftrar, che quivi fi fta in felte , ¢ in giuoco , dice che facevano brindis
jot cantavano'bevendo. I Latini dicevano Propinare , cioe prebibere dal
tim , che fuona lo fteflo che il far brindis, ed ufavano anchreffi quefto
bere in giro’, che dicevano ia orbem bibere , & circumferebant feyphums
¢d effi pure cantavano in tale occafione di bere ; come {crive Dione, che
e Roi aC P quando al banch che fece
bevve a un bicchiere , che li fu porto da una bella femmina .
‘brindsfi . Se ben pare che venga dal Tede(co pringen , percht volendo
) a*nazione bere , ‘ed inuitare il compagno,fuol dire: Zk Mellan-
'y che vuol dire’ /o ve lo prefento ; ¢ quefto gia facevano , perché quel vino,
havevano'a bere reftaffe benedetto dal Compagao , il quale foleva rifpondere
nges , che vuol dire Dio lo benedica . Tuttavia il Lalli nella {ua Mofcheide
“61. graziofamente gli da origine dalla Citta di Brindis , dove chi va ad
f é da ogni veflazione curiale tanto Criminale , che Civile, onde a.
_ faevbrindifi par che sinviti uno ad andar ad abitar quella Cited , cioé a lafciar a
¥ ‘parole del Lalli fon-quefte : i
|. \Brindifi bella s io m appongo al vero ,
Date fon meffi i brindifi in ufanza ,
Quafil buom dica; Lafcia ogni penfiero ;
Beviamo allegri , ¢ rinfrefchiam la panga 5
E fe pui il creditor duro ye fevero
Ci fa da’ birri apparecobiar la franzAy
Brindifi habbiamo , Brindifi diletta,
Che quanto pik fi bee , vie pss n’ alletta;
paglie,o /pilli . E’ un giuoco da’ fanciulli, che fi fa cosi: Pigliane
due corte fila di paglia , ¢ pofandole fopra un piano lifcio vanno
‘4spingendole con le dita tanto,che uno di detti {pilli , 0 fli cavalchi l’altro,e quel-
Jo, che refta di fopra vince , giuoco cosi detto dal Ter? , cio’ tog/i , regi. In La-
tind tudere aciculis, E perché queflo ginoco-¢ di niuna, 0 poca conchiufione, hab-
- biamoil proverbio s Fare 4 tes? con gti fpillerti . ; che fignifica affaticarG , ¢ per-
‘Mere il tempo fenz’ utile , © profitto ; ed efprime ancora Far una cofacon fordido

    
    
  
    

   
   

 
 
 
 
 
  
    
 
  
    

       
    

S’ aiutano I’ un I altro , e's’ accordano .
Ww XXXVI.

me ZA
| “Za donna refia litrafecolata , . Per tutta la Cutta vien falutata,
©) Fedendo quanto bene ognun fi {palfa ; E infinle ftanghe,e ogni forcon s'abbaffa,
i he Nepo I ha di gid infor mata, Ed ella bor qua,bur 1a voltando inchini,
ragiona di lorma guarda,e pala: Pare nna bandernola da cammini .

STAN:

ST vengone il tenor, fi vanno a’ verfi .
* STAN

  

 
 
   
 
  

 
 

   
   
 
   
   
 
 
   
 
 
  
 
   
 
 
 
 

a MALMANTILE::
STANZAXXXVAL. STANZA KEI.
é

Pera che tutti quanti queit Demoni y Percio pafjano in cafa ©

Per vederla , n'ufcian di quelle grote, Firatoconta Stree il Reda
‘Ronzando con un brance di mofcioni y. Le da la ben vennta,e vento
Che saggirin d'actorno.a una-botte 5 dt Le ipieaie sete: nar ds
Saleellam per de firade,e fui balconi y Elia per confeguir ogni (uo jutento :
Comal piover a’ agofto fan le borte, hf "

Sees

     

   
 

E fan, vedendo [ue fembianze belie , , bails
Voei altese fiochese fuon di man con elle, -grazia anchiet di dar i
STANZA XXXVIIL, STANZA KXKK Og,
Cosi fra quel diabolico rombarzoa Sta pur , dic’ ey cont anime (ato y g
La fhrega fe ne va con lo firegone » C’ a feruirts mo mo vuo dar di piglia,,
Sic alla fine arrivanoa Palayro Jo.gid,come tu fai, baveo impranate;,
La-dove s' abboccaron con Plucone , Ma il tutco.d andato poi in ifcompigha

   
 

Ma. perché tra di laro entré-nel mare Horfu : fra poco adsunerdil fensta y. ¢
Scwwccamente il eALandragora buffone y E fopra quetto fi fard confizka y.0
Chiin quel cailoquio fesigran fraftuono, eAicio Baldon batta ig ritirates,
Che finalmente ognuno ufcs dt tnono-, E tu reftt consenta,, econfolata,. >

Martinazza refta maravigliata , che coftoro ftieno cos: allegramente 5 ¢ pak
fando pel mezzo a una infinita di Demonj , che-cutti la riveralcono., giunle coms

Nepo a Palazzo , dove fe le fece incoptro Plutone , che la condutie dentro\,,e>

quivi havendole effa detto il fuo bifogno , Plutone fe peomess di confolarla..

REST A trafecolara, Refta.maravigliata : Strabilifce. Vedi fopraC, 1, ft. 28.
ST ANG A. Pezz0 di travicelio ,.cioé ua legno groffo.pii d’ un baftone .
FOKCONE .. E’un’ afta'di legno fopra, alla quale ¢ adattato un tridente di

ferro , ¢ ferue per ufo delle Malle .

INC HINO. Vedi fopra C..1, flan, 34:

BANDERVOL A da Cammini , Bandecuola vuol dir piccola bandiera , o pen

noncello , che é quel pezzetto.di drappo,che gia portavano:j Cavalleggieri appie-

cato vicino alla punta dellatancia a guifa di bandiera ; ed a guifa di quefta ims

Firenze fe.ne vedono fatte,.di lama di ferro potte in piu. emincati luoghi delles

cafe , come (ono le pergamene , dond’ elce il fumo dei cammini, equelte feruo

mo-per far conofcere i venti col lor girare , € voltard in ful. ferro: y.nel quale fono
jnfilate , ¢ bilicate ; ed a quette allomiglia Martinazza , af

RUNZANDO . Ronzare fi dice propriamente delle mofche ; ¢ perd.dice Ce

me fanno i mofcioni , che fono-quei piccoli- vermi alati , che nafcono-dal vino.
COME fanno le botte-al piover a’ Agosto .. S'¢ veduto dalla {perienaa,che Ja piog-

gia di ftate,cafcando nella poluere {caldata dal Sole inuigorifce le rane ,.o bore
nate di poco ,,fe bene molti. hanno creduto, che le faccia nafcere quell" acqua,coo

guel Sole ; il che é falfo, perche prefe (ubito {cappate dalla poluere fi fon trovate

 
   
 
 
  
 
   
  
 
 
   
 
   
 
   
 
  
    

     
     
 
   
 
   
 

 

Exé SE vo ESE ZZ mERZ SHE aes TLS ES

 
 

 
     

col ventricolo pieno-d’ erba ,. Ma fia come fi voglia,bafta che atal”acqua

gono-faliar , ma:d’.un {alto debole ,.¢ fico , spon come il Poeta: sefptl-
“mere yche faltaflero quei Diavoli. Va Posta faccto Piorcntinodel¢rivend a
* sual. cavadi Mancii-ia. un_fuo Sonetta-dice : ; i

  

Seae
 

H SESTO CANTARE. 275

    

ro Si fivergognan che paffan di norte ,
seat oe oe ae feet
Pe " sides \ Trottando , ¢ faltellando come bette ,
OCT alte , ¢ foche , e fuon di man con elle Cost canto Dante Taf. C. 3.
Ke @”. Intendi frida , e'colui, che continova ‘a gridare,afhoca per I’ affati-
"cam 10 dell" afpera artetia , fi che il fecondo nafce dal primo . E /uon di man,
im elle’; cide con quelle voci accompagnano il romore; che fanno co: batter le»

| “ROMBAZZO, Rombazzo vien dal verbo rombare,che yuol dir ronzare,o frul-
Tare , che @ quel romore , che fa perl’ aria una cofa lanciata con yiolenza , ¢ fi
‘Piglia per ogni forta di Arepito’, o fracaflo. I Varchi Stor.lib. 10. in quefto me-
¢ figniticato dice bombagze voce formata dal fnono,nella ftetia manicra,che
; )formd bembus : Torma eAimatloness implernnt cornwa bombas , perche dice
lunge prombertare, ¢ fpampite fatve con invredibile Bombazzo , quafi in tal
Mn paffero ¢ nimici . Ma |'Autore oon Storia di Semifonte dice al trattato 4,
! atone la T erra;allotra fenritofi per quelli della Citta il rombazxu, E V'ulo
puechecl obbiighi adire romibazo v

| le nel mazzo, S'accompagnd con loro, Che diciamo ; incrn/carfi , fic»
é ien'dal giudco del mazzolino detco fopra C. 2. ftan. 46.

‘ a apora’, Coftui era un buffone , © pili cofto un matto di Corte , che
)  chiacehierava (empre’, ¢ fenza propotito , o conchiufione .

j KeVIO . Voce latina fata di rado in Firenze,e yuol dire Ragionamen-

  

LSS" CO

 

s

to, che fanno infieme due , 0 pit perfone . Corrifponde alla Greca Dialogos, che

ifica indo la: parola Znrerlocutio , dilcorlo che fi tiene fra due,o pili perfone;
dai Pranzefi detto Emtrerien quati Trarrenimento ,

VSCH di ruono , Perde¢ il flo del ragionamento,fi dice anche:a/cir di tema, fmar-

rite argumento, il propofito . Vedi fopra C. 2. flan. 47. E’ pref la fimilicudine

3 feherzando fal doppio fignificato della parola /cordar/f |a quale tan-

tOVi dice’ d* un’ huomo , che non fi ricordi piti di quel che ha propofto di dire ;

‘uno ftromento , che non fia in corde, ¢ non fia temperato al giutta

O@ vnc, che non canti ginflo , ¢ fuor del legitimo tuono , il che fi dice

TIRATIS{ de banda. CondottiG in un’ altra parte della ftanza , feparatifi , o
allontanatifi da quel congreflo .
CHE vento? bat [pinta in quelle bande , Qual cagione I’ ha moffa a andar in quel
0". $ .

_ TRABALLARE . E’ quell ondeggiamento , che fa uno , quando non pud fo-

: in piedi, Mattio Franzefi in lode della Pofta dice ,
cd Chi domanda per nome la cavala ,

Chr eels ba fentito dir , ch’ é favorita
Wy} Poi partendo chi trotra ,e chi trabala ,
Qui yuol dire , che Malmantile cra in pericolo di cadere, cio’ effer prefo da Bal-
done, Diciamo in quefto fenfo anche baienare , barcollare , \n certe rime mano-
feritte nella’ Libreria di S, Lorenzo , fi dice d' un cotto , che barcollava: Es'¢
Yalena , @ non batena a (ecco. Qui fi nea ful doppio fignificato di balenare .
? i a Mo

5 FS = BERS SRE SAS ___

   
*

 

 

=

 

     
   
 

276 MALMANTILE ©
40! md, Adeffo adeffo . BE’ il latino , omb:
Firenze. L' usd pit volte Dante nel fuo Poema,fi co
re altre parole Lombarde ; B il Bocce. Nov. 32. 4@ vi
Jaca della donna, ch’era Veneziana . preys
DRO’ di piglio. Dard di mano , cioé comincerd .
ficaya quafi quel,che 1 Latini diflero Expilare; i Franzefi p
dier nel (angue , ¢ nell’ aver di piglio, E.’l {uo cont ) Fazio.
Poema che fece in terza Rima , ove ¢ introdotto Solino a dettare a.
di ecografia , ¢ del mondo; che percid lo intitold Dita mundi,
ando ; dice cosi al canto 142. ( Parla del Saladino ) i
a Coftui per (ua francherza , ¢ gran configlio y
Tolfe la terra fanta a’ Criftiani
Vincendo quegli , e dando lor di pe lion SNe
FLAVEO imprunato. Havevo ordinato il rimedio, Vien da quell’ imp
che dicemmo fopra C, 3. fan, 21, Addio fave. i
HOR sik, Termine efortativo,e conclufivo,e diciamo nello fteffo fenfo. 01
quafi Or via , Latino Eia age, Vedi forto C, 12, flan. 47. Diciamo + hor fu
diciamo bac ipfa hora furge ,& hoc factas , aga
BATT Ala ritirata, Sene vada da Malmantile, Batter Ja ritirata
0] tamburo fi fa quella fonata , per la quale i foldati intendono doverfi
¢ lafciar ! imprefa . Gio, Villani cid difle ; fonare (a ritirara ; quali
il Franzele, retraite. .
STANZA XXXXL STANZA XX
Jo ti ringraxio st, ma non mi place Dico di Gambafporta il tuo v
Percio ( gli rifpond’ ella ) di manieray
Ch'ionon voglia pivlinr lafpada'lgiaco,
Ch in bagnela fon piit di quelch'io era;
Cost con quei due [pirti bavend» ilbaco
Sagginnge (per c’ alor vnol far Japera)
to Vho con quei briccon furfanti indegni, Ches'egl adaffe un polafruftain ve
C’ hanno frurbato tutti i miei difegni, _ Imparerebbon per nn? aitra volts
STANZA XXXXIII,

  
 

 
 
  
     
     
        
     
   
 
 
    
  
 
   
 

   

   
 

E di quel pallerin di Baconero 4
Che ib oo | ginoco con “nt i
Scambiando it color bianco per
Error ,che nol farebbe anc’ un cava
Ma e'vit che gli firapaxzanoil.

See

 
 

7

  
  
     
  
 
 
  
        
    

as

    
     

a oi

a

Rifponde il Re ; Facciam quanto ti piace, Non penfo di reftar gid contumace y —
Ma ti verranno a chieder perdonanzA, S'io non ti feruo,perch'iofo a fidanta. ‘
Si che ru puoi con offi far la pace Dunque ti Lafcio,e fone al to piacere;
Pero racquiera,e vane alla tua fiaza, Fatti feruir da quefto Cavaliere. ,

Martinazza ringrazia Piutone , ¢ dolendofi del danno cagionatoli da G:
florta , e Baconero lo prega a gaftigargli : Plutone I eforta a placarfi,¢
che andranno a chiederle perdono dell’ crrore ; ¢ fatte con effa fue cirimonie
rimanda alle ftanze. a '

WON vogtia pigiar la pada , ¢ ilgiaco, Non voglia armare contro di loro pet
yendicarmi. 4 =i eee
SONO in bugnola , Sono in collera, Bugnola fi chiama un’ arnele fatto dicot —
doni di paglia , entro al quale fi conferua grano, biade , ec, da i Latini dettas
cumera, Bfidice cfler’ in byguola , nel bygnolone, in valigia, nel gabbione ee

eS.

  

=

     
 

asget
 
 

   
 

SESTO CANTARE: 277

ee in cOllera. E tutte quefte manicre vogliono efprimere il gonfiare ,

un fa per l'infiammazione della bile commofla. Orazio Bile wmer iecar ; dove
vaveva detto : mexm iecur vrere bilis, Ovidio ne’ Balti , Intumuit Luna, ciot

onfidzentro in valigia . Gli Spagnuoli fimilmente dicono , embotijar/e .

| HAVENDO il baco, Havendo ira . Traslato da i cani , i quali quando hanno

: & n certo baco nella lingua per di forto , par che fieno (empre adirati, ed il fimile,

 
 
    

 
 

fegue ne i Montoni,quand’ hanno il baco , o tarlo dentro alle corna ,
‘AR la pera. Anticamente s’abbruciavano i corpi morti fopr' ad un montes

 

; = » qual monte quando era accefo , chiamavano ?yra. Lall. En, Tr, lib. 5.
tei!
ae : Gia P alta pira di Didone ardea ,
pate £ vibrava lontan fiamme ,e faville .
‘Edda quefio credo , che venga il noftro far /a pera , € che s'intenda anche am-
na 'uno,quafi fi dica : 40 vagtio far /a pira al tale, S' intende anche far /a [pia

  
   
   
 

‘AR fallo, Far’ errore. E’ termine del giuoco di palla : ¢ perd il Poeta fe ne
ue’; perché I’ errore fu fatto con le palle . Properzio lib.3. we pila veloces fal-
lit per brachia iattus .

NOL farebie ance un cavallo, Error groffiffimo , ¢ che non lo farebbe anches
una beftia ; e fi dice wn cavailo , perché quefto animale pare,che habbia difcorfo, e
giudizio pi che ogni altro animale . I Greci di sippos , che vuol dire cavallo , fe
ne per una particella , che aggiunta a’ nomi , importa grandezza. Hip-
pomarathram id @ il finocchio faluatico , ¢ Aippomyrmeces , certe formiche ,
che paffano di grandezza I' ordinarie , ¢ comuni . Onde errore,o spropofito da.
cavallié mn’ error grande . O pure fi dice cosi, perché fia degno di cavallo , cioé
i _ di gaftigo , qual fi {uol dare nelle {cuole a’ fanciulli .

_ STRAPAZZ ANO il meftiero , Cioé nell' operare , non confiderano quel che

 

f — eANDASSE Ia frufpa in volta. Se la frufla andaffe attorno. Se fuffero di quan-

~ doing baftonati , fruftati.

\ NON penfo di reftar contumace , Termine di cirimonia , che fignifica:non penfo
di commenter mancamento . La voce contumace ¢ Latina ; perd il Lettore fi pud
foddisfare circa i {uoi fignificati ,

FO 4 fidanza , Confido, che per tua cortefia non I’ haurai per male , ¢ mi {cu-

A ferai ; termine ufato fra gli amici intrinfechi ,¢ fidice anche; Fo 4 ficurta ,

____ SONO al tno piacere. Termine ufato da’ fuperiori con gl inferiori , in yece di

fF aetgeps Coenen , torent N

ia ‘avaliero, Intende Nepo,

J STANZA XXXKXIV.

( ses mena allora alle fue fPanve, Ove gli orfi facendo alcune danze

f Cha paraméti havean di quoi darn ats Dan la vivanda , e da lavar le mani;
; Ricamati di fignoli , e di ftianze , Volati al cibo al fin,come gli affori ,

| Efepeans di via de Pelacani Sembrano 4 foe fo due toccateri
= Nn 2 STAN.

 

 

 
 

278
STANZA XXXXV.
Fioritaé la tovaglia , ¢ le faluetce
Di verdi pugnitopi , ¢ di froppioni ,
Saldate con la pece,e in piega frrette,
Infra le chiappe frate de’ Demanj .
Nepo fra tanto a mavinar fi mette 5
E cheto cheto fa di gran boccont ;
Offeruando Caton , ch’ intefe il gioco,
uando diffe: in connito parla poco .
STANZA XXXXVL
Fa Martinazza un bel menar di mani;
Ma pit cheilvitregli occhi al finfi pafce,
E quel pro falie, che fab erba a'cani ,
Che il pan le buca, e sloga le ganafce ,
Perché refee vi fon come trapans,
Ne manco fe ne pro levar con Pafce.
Crudvéilcarnaggio,e si tirante,e duro,
Che non viene apuntareipiedi al muro ,

 

Prexioft liputvl seve ne foaé i
Portati ciafcheduno in fua guafiada
Effendovi aqua fortes inchifro buss
Di quel proprio, c'adopera to Spada,
Ella che quivi frar voleva in tnano, ~
E non cambiar, partendofi,la firada ,
Perché i gran vini al cerebro le danno
Ben ben Vannacqua con agreftojeranno,

STANZA IL

E fatte due tirate da Tedefco
La taxxa butta via fubito sm terra ;
Lero ch'ell'é di morto un tefchio frefco,
Che fuona, e tre di fa n'ando fotterra,

Nepo,che mai alzi vif da defeo,
E intorno.aibuon boccotiratohaaterra;
Anch'egli al fine dato a tutto il

La bocca follevo dal fiero pafto. ‘

Nepo conduce Martinazza alle fue ftanze,dove era imbazidita la menfa, ¢ fu-
bito fi mettono a mangiare. L’ Autore defcrive la qualita de i j dell’
imbandimento , de i trattamenti , € de i cibi, tutto appropriate a uno appartas

miento , ¢ banchetto da Diavoli.

QVOF humani , Pelli a’ huomini. Se ben quoio vuol dir pelle di beftia conciat’
fi piglia ancora per pelle d’ huomo , come s’ é veduto fopra C, 4. ftan, 20, € coms

lo prefe il Rufpoli dicendo :

Vn certo ch’ in full offa ha [ecco il quoio ;
FIGNOLI, Specie d’ apoftema nella cute;da i Medici detti Puruneull . ,
STIANZE . Quelle crofte , che fa nella pelle 1a rogna ; o altre bolle; dai
Latini dette erafe. Varchi Stor. Fior.lib.1.4, G4 trewarono rofo dello fomaco quant
un giutio con una fianga nera fopr’ a quel rofo.
SAPEAN di via de’ Pecalani , Puzzavano di beftia morta di pid giorni, Las
via de’ Pelacani fi dice in Firenze quella ; dove fon le conce delle pelli , nella»
quale ¢ fempre wn puzzo orrendo cagionatoye dalle conce 5 € dalla corruzione di

quelle carni

VOLAT I al cibé come gli aftori, Entrati a tavola veloceiente . Avventacifi al
cibo , come fa I’ afore, il quale, benché habbia il cibo a fuc dominio,vi s\avven-

ta ,¢ lo divora con rapacica grandiffima
SEMBRANO 4 folo a fol due Toccatori,

 

Dicemnmo fopra C. 2. ftan. 66. quel che

* fieno i Toccatori, Quefti fono folamente due , e volendo andare a cena all'ofte-
ria fon foreati andar da lor due foli , che le converfazioni de’ galaathuomini oon

li

 

ee Fa ESOS ee Oe Ren eee se. Se

 

Rey

 

 
  

 

2
7
e
e

=

>

=

SE Tyr

,
'
j
7
{
f

it

 

RET TE in piega, Le faluctte , ¢ tovaglie fi piegano in diverfe maniere , ¢
fifa loro pigliare Ja figura , che fi vuole, col tenerle cost picgate Mrette in un tor-
0, ofirettoio fatto a pofta per tal’ effetto , in vece del quale ftrettoio , guelte
0 ftate firette fra le natiche de i Demonj ; ¢ cid dice. per efprimere, che fons

Deseo
_ ANTESE il giuoco... Sapeva,come era conueniente fare, quando difle: Pauca in
loguere .
_ FAun bel menar di mani, Si tadia ; 8 affatica a mangiare, Vedi fopra C. 1.

LE f il pro,che fal erba a cani , Non le fa pro. Quando i cani mangiano ler.

REST E . Quei fili fottilifimi , che flanno appiccati alla fpiga del grano , dell’
orzo, ¢ della fegale ; dal Lat, aris .
_ TRAPANO., Specie di {ucchiello , o foratoio atto.a bucar pietre 5 ferro , ed
i altra materia per dura che fia , ¢ s' adopra facendolo girare com una corda,
loi ? habbiamo dai'Greco Trypanon , Vedi fopra C. 4. ftan. 73.
NON se ne puo levar cont’ ace, E’ cosi duro , che ci vuol l'afce alevarne uns

iT NON viene,a puntar i piedi al muro, Non fe ne pud ftrappare a fare ogni mag-

\ BAR lo fpiano a cafa a altri, Mangiare a cafa d’ altri (enza fpendere. Vedi

\©, 3. flan. 51. Quefto detto viene dallo fpiano del grano, che vien dato
dal Magiltearo dell” Abbondanza a i Fornai per fmaltire il vecchio, che fi ritrova
hei magazzini pubblici , ¢ da quefto rifinimento /pianare , 0 far lo fpiano a cafa d’
altri intendiamo rifinire , 0 confumare quello, che colui ha di commeftibiic in,

ECASEO barca, e pan Bartolommeo , Precetto della squola de’ ghiotti,che vuol
dire Mangiar 1a midolla del cacio , ¢ la corteccia del pane.

 FREMERE . EB’ voce latina , che conferua appretio noi lo fieflo fignificato :
‘Verg, 1. Bn, Cuntti fimul ore fremebant , E altrove defcrivendo il Furore ; fremir
borridus ore cryento ,

BRANO, Pezzo dicarne (forfe dal Latino membrana ) 0 altro ftrappato
- violenza , ¢ fi dice sbranare; ¢ sbranalo, Vedi lopra C, 2. fan, 52. mandato
abrani.

_ CIBREO , Guazzetto fatto di colli, ¢ ventrigli di poli, 2zinueal. Pud effere
Originata quefta parola dalla Latina Gigeria. Feito Gramatico : Gigeriaex muitis

js ape .
MAGNANO , Quali machinarius fabbricatore di ferti minuti , ¢ di piccoli ia-
i : Begui

 

 

 
 

    
 
 

280.
gegni,come chiavi , toppe ; a diftinzione di Fabbro , che

ine Zappe, vanghe , ec, ¢ del Manefcalco , che fabbrica ferri

ci¢ i Magnani fon fempre tinti di nero , il Poeta dice che il cil

lero interiori,per efprimere, che era nero. ate sh SOR G
VENT RIGL/O , Ventricolo degli uccelli ; in altri luoghi oe theta “i

    
 

 

STRIGOLI. Diciamo quella membrana , 0 rete gratia; che fta
budella degli animali. i

AC QV A alle mule’, BE! un detto di gente baffla , che fignifica date

GVAST ADA , Vafewto di vetro corpacciuto , ¢ col collo lungo , ¢ fire
ferue per lo pia tenerui l’acqua per annacquare il vino , quando fi beve.
antichi diflero Jngaifara. I) Canini la fa venire dalSiriaco Ga/far , che’ v:
ficflo. Potrebbe anche comodamente dedurfi dal Greco Gaffer,che v:
corpo ; ¢ cost Guaftada effer detca dalla figura corpacciuta , nello: fteflo
punto che Graffa voce Siciliana ufata dal Boccaccio neile Novelle in ]
te viene , fi come molte della Sicilia , dalla Greca Ga/ira, un. poco tt
Iettere ; la quale fignifica as vafo che habbia pancia ,

LO Spada, Valerio Spada celeberrimo Macftro di (crivere , huomo fi
¢ che non refta addietro a veruno nella galanteria del tratteggiare con
mano , ¢ frappeggiare , ¢ far paefi con Ja penna ; come d’ intagliare in
bulino , ¢ acqua forte , fu amiciffimo dell’ Autore , ¢ fuo {colare nel
vive ancora ; € ben che d’ eta fopra {ettant’ anni,indefeflamente lavora p
nare il fuo nome.
VOLENDO ftar in tuone, Volendo ftar’ in cervello , ¢ noms’ imb
CAMBIAR Ia ftrada, Quando vogliamo dir copertamente a uno, Tu fei bria-
co;diciamo . 7’ bai fmarrita la Strada , ¢ pero intende;non fi vuole imbriacarey
KANNO . Acqua paffata per cenere , detta anche 4/eia , dal Lat, dixivinm, I
dottitfimo Ferrari nelle Origini della lingua Italiana , dice cosi ; Ranno; lixsvia
Vude vox ortum trabat , omnibus veftigijs indagara battenus fefellit, Chi fa , che non —
fi origini dalla voce Greca Xhanis , che fignifica , Milla , gocciola , perché il sann0
ftilla a gocciola a gocciola da que! vafo , che percid diceli Colatos ? (ee

ia del vinos4

    
  
 
   
  
   
 
     
    

SEP FRE SSS pec he es ee:

a:

F,

4 ef ert Fz Ez.6 =F

    
 
 
 
   
   

FATTE due tirate da Tede/co, Fatte due gran bevute. Manda gii
Latini dicono : pocula obducere;i Franz, avaller ,

SVONA. Di quefto verbo fonare ci feruiamo per intender copertamentes

mere.

P MAL alzé vifa dal defco., Stette empre attento alla roba , che era in tavolas+
Termine ufato per intendere uno,che a tavola mangi con avidita , e non pigli div —
vertimento di forta alcuna ; E de/co fe ben vuol propriamente dire la tavola,dove
fi fla a mangiare , onde il dettato: Chi non mangia al defco Ha mangiato di ,
oggi ¢ poco intefo per altro , che per quel iegno, fopr’ al quale i macellari tagliae
uo la carne , ¢ per quel banco , a) quale nelic Confraternite , 0 compagnie de’
colari ficde il Giovernatore ,
TIR ATO ha aterra ai buon bocconi, Ha mangiato affai de’ buon bocconi 5
Jo fico , che menar le mani detto fopra . aM

La bocca follevs dat fiero pafto. Verlo di Dante Inf, C, 33. Lafcid ftar dimane
giar quell’ orride vivande, . ioe

- STAN-

    
 
 

  
   

   

 
       

  
 

SESTO CANTARE 281

 
     

   
 

on SUBTAN ZA. STANZA LI.
Lafciatii voti, e i piatti fcemi Spargon le rame in varia architettara
in anno al giardino pieno di emente Scheretri bianchi, ¢ rofse anotomie ,

    

Di berlines di mitere,e di remi , Gi aborst,i moftriei gobbi in fu le mura
Edi firumenti da caftrar da gente ; Forman spailiere in lnogo di lumic ;

       
 

 
   
       
    
      

    

ifede in me2x0 sl paretaio del Nemi Dugna , di denti , e fimil’ ffatura z

_, D'un pergolato,il quale a ogni corrente Lnfeliciate fon tutte le vie ;
i con quattro braccia di cavexra, Nun bel fepoteroanicchia il fore butta
. Penoloni , che fono nna bellexra. Del continuo morchia , ¢ colla firutta .

ee STANZA LIL
fono abbroffolite , ¢ feure Sui dadi i torfi nebili feulture

ie del mar venute della rena, (Perch'in rovina il tutto iitempo mena)

‘intorno intorno in varie pofiture Rifpaurati fono , ¢ rifarcité

  
  

d | Sep rem leggiadra feena, Da vere, ¢ frefche tefte divanditi ,

lito che ro di mangiare , Nepo condufie Martinazza nel giardino. Qui
icipia a defcrivere un giardinu da Diavoli moftrandolo ripieno di tutti quei
Malanni , ¢ difgrazic , che-alla giornata accadono a i mortali .
_ LASCLAT Li bicchier vori , e+ piacti feemi . Havendo bevuto,e mangiato quan.

piaciuto .
> GLARDINO . Luogo dove fi piantano fiori , ed altre delizie fimilida i Latini
detto Florarinm 5 fen pomarinm . Vicne quefta voce dal Tedefco Garten , ¢ quefto
dal Latino bortus , fecondo i] Ferrari , ii quale biafima il Perionio , che la fa ve.
nire dal Greco ardevein, innafiare, feguitaco in cid dal Monofini. Ma tanto que-
glinella fua lingua Francefe,quanto quefti nella noftra Tofcana,fono troppo ap-
Paflionati acl far venire le voci dal Greco yilche non & fempre vero , ch’ elle»

     

ee ee

‘vengano .

 SERLINA. Gogna. Vedi fopra C. 2. flan 15.,¢ C, 3. ftan. 62,
HITERA, B) quel berrettone , 0 cartoccio di foglio ; che dalla Giuftizia fi

fa meteere in tefta a coloro » che fono fruftati in full afino. Vedi fotto Can, 12.

Bt

KP.

» 4h Pareraio del Nemi . Intendiamo le forche , perch quefte fon fituate in uns
campo , che era , ¢ forfe é ancora della famiglia de’ Nemi , ¢ Jo diciamo Pareta-
40 per coprire il detto.. Li Pareraio € un bofchetto fatto per uccellare a fringuelli,
ed altri uccelletti fimili nominato Pareraio dalle retiyche s’ adoprano a tal caccia,
Ie small fichiamano parere. Vedi fopra C. 4. ftan,27. al termine mandatoin Pic.
tardia,

~ PERGOLATO., Le viti che foftenute in aria da pali, e pertiche, formano co-
Me Una coperta , o tetto fi dicono pergole » O pergolati , come dicono anche i La-

Fe

_ CORRENTE. E10 fieffo che travicello » cio® un legno lungo,groffo pid d'un
> € 8’ adatea a formare , ¢ foftenere i palchi, ¢ vetti delie cafe ,

| 46 a¥LZZA. § intende quella fune 5 con la quale fi legano per il capo le be-
- . .

| fit ye perd € detta cavezza-quafi capo , ¢ il Poeta Ja chiama cosi , perche é lega.
, #2 per il-collo, ecapo degi’ impiccati a quei correnti, ¢ gli chiama Penzoli , per-
_-Sh€ gli figura grappoli d’vua pendenti a quefta pergola, ;
, BRA : Shek;

 
 

 
  
   
   
    
   
     
  

282 MALMANTILE- 2
SP-ARGON le rame . Gli alberi che fono in quefto.giarditio diftendon
rami in diverfe mani¢re; ma in-vece d’ alberi (ono fchelétri.
tomie. Scheletro , 0 fcheretro diciamo tutta |’ offatura dtumco
ogoi altro animaie,ripulita dalle carni , ¢ rimeflainfiéme con
105 ; e4notomia chiamiamo il corpo d'un’ huomo, ed” altro’ani I
moftra tutti li nerui'y wulcoliy ¢ vene , chefoho foro Jaipelley: soe
SPALLIERE , Quelle piante , ed alberhy che fi fannovdutendere fu perte,
ra con i rami , come limoni , ¢ fufini , ec, fi dicono fpallieré se qui:pig
mie per ogni {pecie di pomi d' agrumi , dice , che in vece di tali pomi
quefti alberi a {palliera gli aborti., i miofiti €s gobli .< ai Se
INSELICIAT E . Seliciavo dal Latino filices diciamo up lafiricd fatto

ma firettamente,intendiamo quei laitrichi fatti-di plete: piccotidinies 5 o- tan
giion fare ne i viali de i giardini a foggia di Mofaico con pietreyperd maggiori di ji
quelle del mo/aico,e minort atial di quelle degit acciortolatiy e fono diary colo- |

ri in manicra che (ene formano figure ,¢c. Come col Mofaico ., Binovece di gi
fic pictruzze,dice che (on fatte d’ ugaa , drdendi , ¢ d’ aitreoflature minute.)
a MORC HLA , Intendiamo la fondata deil’oli0 dai Latuno-amarca, © gets dat
fr, aan . 4 2
CABBROSTOLITE , Abbronzate. e<bbrofolire propriamente viohdire qu
abbruciamento che fi fa agli uccelli pelati , agcio fiabbrucino quei pr ‘the
non fi fon potuti levare con le mani ; ma qui yuol diretince dal fuoco ¢on un
leggieri abbronzamento ¥ che diciamo: abbractacchiate -3 egiggdted
MV MMIE . Sono cadaveri d’ huomini che-hanno Ja-carne appiceata Ws
full’ ofia {eccatavi fopra da balfami, bitumi ,ied aromati, come fon cOlpl,
che fi trovano forto Je Piramndi di Bgicco , 1 quali (ono di perfone péincipaliy che
gli Egizzj havevano per coftume di riewpucre di baifami , ed aromaci, fate
gli con ftrette ftrifce di tela’, odi drappo com murabilitfima maeftria , ¢ pt
hi infieme con qualche Idoletto fatto di metaijo dentro a una cafla, che sate
fe

VANS

 

‘

uy

Ni

iy

i

:

faccia d’ huomo ; cosi gli riponevano {otto quelie piramidi , dove non &
cevano ; ma fi feccava la carne , ¢ fi riduceva tanto queila,che |’ offo come

trito ; per Jo che fi ono-conferuati quet corp: fino a1 tempi nottei , ed f

ne trovano, Polid. Verg. de Rer, imuen, lib. 3. 6, 10. riferifce-con te feguentipe |

3» tole il modo di quefto fotrerrare i cadaveri degh Egizzj:Agypuj Hatin mor hg

>» tuo homine ferro incurto cerebrum per-nares educebant, jocam iiusmedt

we is expl ; deinde lapide A chiopico circa-ilia i ‘a

>» bant,atque illac omnem alucum protrahebaat » & ubi repurgaverant , rorfam lig

»» Odoribus contafis:refarciebaat., ide iterum Contucbine. Vbi hae fecitlent ,fa- toy

>» liebant nitro aduito feptuaginta dics, nam diutius (alire non ticebav; quib

y»» exactis Cadaver findone inuoluebant gummi iilinentes ; Bo deinde .

&

fy

A

w

I

 

 

   

9» Pitiqui ligneam hominis efigiem faciebant , in qua inferebane y

>» lumque ita reponebant ; Eeid, ut arbitror,ica facticabant ju eo acto” >

»» ta cadavera diucurnius incorrupta feruarent . ae
Altri cadaveri fecchi ci vengono pure dagii Egizzj,i quali corpi |

teriori , e-tutto, fecco, ¢ come impictrito; ¢ ono iciza farciature ; equell a

corpi d’ huomiai , che dal vento {ono ftaui fotterrati vivinelia rena 5 € quivi com

   
 
   
   
   
    
   

 

tidal’ tar della rena, Di

 

chi , ma p:

SESTO CANTARE:
fertiatifi forle per caufa de’ venti meridionali , ¢ perd il noftro Poeta dice : Fenn-
quefte Mummie fi feruono i Medici per diverfi farma-
a particolarmente per la Triaca , La'voce Mummiaé Araba ; ¢ il Voffio

tira da Atam , che'in Arabefco vuol dire , cera (de vitijs Sermonis lib, 2. cap.
la cera ¢ '! miele faculta conferuatrice ; ¢ della cera fi feruivano gli

per mantenere i cadaveri fecondo Brodoto , Jib, 1. Ma la pece mefcolata
bitume , era forfe quella materia , per quel che apparifce ,con la quale
gli Egizzj condivano tali corpi , 1a quale in Latino greco dicono Pi

 

289

magi s
?

DI, Intende quelle bafi , fopr’ alle quali fon pofate le ftatue .

R51, Intende torfi d’ huomini , che pittorefcamente parlando vuol dire il

fenza'tefla’, e’braccia , ¢ cofce Latino truncus ; ¢ quefti dice, che fono
ilareiti ; cio€ raccomodati , rappezzati , riftaurati con haverui mefie in veces

- delle lor tefte gia confumate dal tempo , alere tefte nuove , ¢ frefche di banditi;

uol ¢ meta » che alle volte fi veggono al Palazzo della Giuftizia , eo.
fopr'alle forche elpofte alla vifta del popolo , eflendo fate tagliate di poco tem.

po ai maifar
/ STANZA Lil

Inter ‘8 quadri di cipolle
Snip i fior Youyatee nariche;
Soma teiccioni , i fignoli ,e te bolle,

Le pofleme, ta rigna , e le volatiche .

Vil mal Pricefe entrante alle midolle,
CW feminaro dalle male pratithe ,

‘Teticberi , le rabbie 5 ¢ gli altri mali,
- | Che vi mandano gli Offi ,e é Vettnrali,

  

malfartori bandit, « per frelshe

STANZA LIV.

Pefchein{u gli occhi fonui arzurre,egialle,

I marchi , che fiorir debbon le spalle

Ai tagliaborfe , ¢ ladri ancor {colari ;

Le piaghe a maffe , 4 pererecci a bulle,

wend ventofe,¢ gonghe in pik filars , ©
 ¢ il fior di rofolia , € pik rofoni

D' ortefica , vainolo ,e pedignoni .

Cu re for per chi gti porta pari ,

o ita a defcrivere i) giardino dell' Inferno’, ed in quefle duc ottave narras
i - guelche contengono gli fpartimenti . é
<QVADRI di cipolfe . Intende quelli fpartimenti , che fi fanno in terra ne i giar-
Me\gquali fi pongono le cipolle de’ fiori. Latino areole , puluini ,
» PRA foglie , e-natiche . Dice cosi per moftrare , che quefti mali vengono nella
carne ef mente , ¢ pigliando natiche per tutta Ja pelle dell’ huomo , dice che
fra quelle foplie nafcono quefti mali in fu le natiche , intendendo Ja pelle , ¢ per-
ché anche la maggior parte de’ medefimi mali per Jo pili viene in fu le natiche ,
‘come laogo pib carnofo . ;
CHE vi mandano gli Offi 5 ¢ i Vetturali, Quefta forta di gente ha per coftumes
itnprecar fempre male , come venga Ja rabbia , il canchero , la pefte , © firnili ,
» PESCHE in fu gli occhi, Qucei-lividi , che vengono attorno agli occhi , quando
fono flati percofti da pugna , o da altri , e fono di colore azzurriccio , € intorno
: 7 $ Dar le pefche : i Latini dicono /uggillare alquem , vedi fopra C, 3,
‘1. y che noi purt diciaino anche figilli tali lividi,¢ diciamo anche: figillare un’

uno, *
_ GLI sfregi fior per chi gli porta pari , Gli sfregi fon fiori , che anno bene in ful
v ) di coloro che portan as tani , cioé fanno bene il raffiano, che portar i pollé
woo) dir fareil rufhano dalla voce pouler Francefe che vuol dir 5 vigtierro amorofo ,
diciamo porta poulers . Qo CUAR.

SSS HSS ETE SaaS ers

 

 

 
  

290 MALMANTILE

MARCH, Tntende quei fegni,che dalla giuftizia fi fanno
droncelli , gai per effer giovanettinon fono capaci della pena
Higmata , Vedi opra C. 2. ft, 3. alla voce sberlefe. 6

PLAGHE a maffe , peterecci a baile, Piaghe , ¢ peterecci ins rand
ma, Nell’ ulo diciamo anche Patereccio ; ¢ Panareccio dal Greco , ulato ai
da’ Latini Paronychia , poftema che fi forma alla radice dell’ ugna , che i
chiamano Redivias., 0 Reduvias yp eas

GONGHE . Intendiamo gavine infermita che viene nel collo , ¢ quei t t
che fon taluolta /pine ventofe , perché diciamo haver le gonghe os ee e

venga apparentemente nella pelle della gola {otto le ganalce, Latino tonfilla,glan
dule faucium , t VE
Ma perché non paia che io voglia fare un trattato di chirurgia, i
{plicazione di quefti mali ; tanto piu che io ftimo , che faranno inteli per
Italia , nella quale fon chiamati nell’ iftefla , pore differente mamiera ,
intelligenza dell’ opera ferue fapere , che in quefto giardino fono tutte
ta , che vengono agli huomini efteriormente , le quali il Pocta yuol moftrares,
che fi generano nell’ Inferno , come fentina di tutti i mali. 10.04
} LV. STANZALVL,.

Alla ragnaia al fin fi fon condatti
Di fils da toccar la margheritey »
Ove de’ tordi cala , ede’ merlottt
Alla ritrofa quantita infinitay

    
   

ai Lae

   
  

    

  

aes

Si maraviglia , fi fupifce , epanta
Martinazza in veder si vaght fiori ,
Erimirando hor quefta,bor quella pianta
Non fol pafce la vifta in quei colori,

Ata confortar fi fente tutta quanta
Alla fragranza di sh grats odori y
E ai non corne non pxo far di meno

Che fon poi da Biagin pelati ,¢ corti
Sgoxzando de’ pit frolli ana partita y
Altrane/quartaye quellach'e puafrefea

Be ssh ee eh

Vu bel maxxetto, che le adorni il feno,

Nello Stidione infilza alla Turchefca,
NZA LVI.

= BEG ES.

Veduto il tutto, Nepo la conduce

Chi per Ja pizzicata , che produce
A! bagno, on ogni {chinvoy ¢ galeotto )

| dl uazo ,fa tragedie sn ful.capportas _
Opra qualeofa: Vn fa le calze,un cuce, Vn mangia,un fofia nella verrinala y
etltri vende acquavite,altré il bifcotto, Vin trema in fentir dir:fuor camiciuels,
Martinazza refta maravigliata , ¢ fi fupifce , ¢ rimirando tutte quelle piante,
paice la villa , ¢ foddisfa all’ odorato con quella fuave fragranza , ne pud non fa-
re un mazzo di quei fiori galanti per adornarfene il {eno . Vifto il giardino, Ne
po la conduce alla ragnaia, dipoi al Bagno, dove ftanno i galeorti » defcritto ¢-
me é appunto guello di Livorno cirea operazioni , che fanno i galeotti, |
SP ANT ARS!, Dallo Spagnuolo e/pantar/e. Vuol dire efttemamente mart
vigliarfi, ¢ fi dice in augumento maravicliarfi 5 frabilirft , (pantarfi, che & il verbo ui
§paventarfi fincopato . Habbiamo I’ addicttivo/panro, che fignifica eftremament —
marayigliolo . Ma forfc ¢ da Spandere , quali voglia dire largo a ie
de, ampio , ¢ in confeguenza maravigliofo . E di Spamse addicttivo, del, Ay
Spandere cen’ ¢1' efempio in Meffer Cino, Quando ha per gli occhi (ua poem &;
fad 7 wap Sah
4 VN bel marzetto , che le adorni il feno, Bello ornamento del {end d’ una few .
* pa havervi crofte , rogna , ¢ fimili galanictic , delle quali potcya efler fy
~

UGA

 

Serer,

 

 
 
 

SESTO CANTARE)

Poeta {cherza per efprimere la laidezza di Martinazza .
BE’ una felua , o macchia folta pofta per Jo pid lungo i rivi, per
fi pe eee fofpefa a due Mili, e quefta rete fi chiama ragna
‘a imitazione di quei veli,che fanno i ragai per pigliare le mofche ,
ragne . Pietro Angelo da Barga nel fuo Poema della caccia de-
celli: Hos caffes , has ipfa plagas y bac retia quondam Ante alias omnes telamts
ere dotta Innenit dixitque (uo de nomine Arachne, E da quefta rete ragna fi
nna ea » Ove Gi tende per pigliar tordi , beccafichi , ec.
‘teccar la margherita, Cio quelle Aanghe,fopr’ alle quali fi da il mar-
lla Corda, che quetto vuol dir toccar /a margherita .
DI, merlotti , Vuol dir merli giovani , ma dicendofi merlotto, o Tordo
stintende Huomo femplice , corrivo, che cala; che fi lafcia pigliare.

3

 
 

ft. 59.
g Gabola fatta a foggia d’ una trappola da topi , con la quale per
certo Ordigno fi pigliano vivi gli uccelli , detta cosi per efier 1a parte , da
eferrare rivolta in dictro. Vedi fopra in quefto C. ft, 1. alla voce con-
Qui per ritrofa intende Carcere .
; » Maeftro Biagio , o biagino vuol dire il Boia , che cosi havea no-
me, quando I Aurore compofe le prefenti Qrtave ; ed a quelto fucceffle Maeftro
Baltianodetto fopra C. s. tt. 44. ;
0 + Poco gli manca a effere ftantio ; s' intende animale morto di pik
giorni, Vedi fopra C.3. ftan. 24. la voce ftantio ,
INELLARE alla Turchefca , Cioeimpalare ,
BAGNO . Cosi chiamiamo quel ferraglio , entro al quale fi tengono gli {chia:
vi, €coloro’, che per delirti fon condennati alla galera , detti pero Galeotti, i
quali dimorando quivi,fanno i meftieri enunciati dal Poeta, che fi ferue della vo-
_ ¢ bagne per P equivoco,il quale fa credere , che in quefto giardino fia ancora il
g0 da bagnarfi per moftrarlo ripieno d’ogni delizia;come il Paretaio , ¢ la ra-
- _ E.quefto ferraglio di galcotti credo , che fi dica bago, perché in effo quei
iquenti purgano i loro misfacti , come con |’ acqua del bagno fi purgano le»
‘delle membra . Gagno fi difle ancora un luogo fimile, Li Pulci nel Mor-
pante: Dife Aforganre allora : ia fon nel gagno De’ diavoli ,
» PIZZICAT A, Specie di confezione minutiffima , ma per 1a fimilitudine della
figura di effa confezione , ¢ per il fenfo del verbo pizzicareintendiamo ( comes
gui §’ intende ) pidocchi . ; pe
FA ie in ful cappotto , Ammazza pidocchi in ful cappotto , che ¢ quella_,
fo tche portano gli (chiavi , o galcotti , remiganu , ed ogni altro mari-
; detto , ficcome Cappa , 4 capiendo , perche piglia, ¢ cuopre tutta Ja vita,
- SOFPIAR nella verriola . Cio bere , perche bevendo fi fofha, o: refpira col na-
{0 nelia vetriola  cioé nel vetro. Detto che ha del parlar furbelco .. Vetrivéa er-
s herba parietaria detta daalcuni. 11 Monofini lib. 9. Indicare.yo-
Aen muito vino fe ingurgitafle , dicimus . Egis ha toccato ben la verrigla ,
Vesrivola eff herba infeftoribus notiffima , de qua Petrus Cre/centius lib, 6. ¢, ult, pocula
itrea vulgo fiune , q
. do f Auzzino vuol baftonare un galeotto per qualche
Oo 2z fuo

 
 
 

 
 
 
   
  

od
see ce

  
 
  
  
  
  
   
   
   
    
   
   
   
 
   

 
 

  

292

fuo mancamento {uol dire fuor camicinola, intendendo , che

ha da efler baftonato ; ¢ perd dice: Chi trema in (emir dir

trema per il timore delle baftonate. i
CAMICIVOLA . Bun piccolo farletto di panno lino , b2

che fecondo la Ragione fi forto gli altri abiti(opra alla Camicia

derGi dal freddo , come 10 detto fopra alla voce Farfetto : gli

chiamano gix/ecca , vod awieirseles

STANZA LVIIL TAN:

Vanno pik innanzi a’gridi,ed a’romori
Che fanno i rei legati alla catena
Ove a ciafcun fecondo i {noi errors
Datoe il gafligo , e la dovura pena ,
esi primi che for due Proceuratori
Cavar fi vede tl fangue d’ ogni vena ,
E quefto lor avvien , perche ambidui
Furon mignatte delle borfe altrui , Con

STANZA LX, Oy

Quei, dice Nepo,t il Re degli xfurai , 1 gran fe gli marcy dentro a’

Che pel guadagno fcortico il pidocchio , Che nol vendea fe non vaiea

Vn fernizio ad alcun non fece mai , Cost fece det ged hor
Se non col pegno, e dandoli lo (crocchio GP intarla il doffo,e da'fiohfe fh
Paflano avanti a vedere i delinquenti legati alla catena , ¢ gait er lo

falli. I primi fono due Caufidici , ed il fecondo é un’ Viuraio y ti

fecondo il merito . » Seah

PROCCVRATORS, Agitatori di liti. Cautidici tanto Civilijche criminali

MIGNATTE . Sanguifughe. Quei vermi acquaticijde i quali fi feruono 1 Ce»

rufici per cavar fangue ; e perche fi dice , che i danari fono il fecondo fangut

perd effer mignatta delle borfe alerni vuol dir Succhiare , roe ¢avar il de i

altrui borfe , come fa la mignatta fucchiando , ¢ cavando il fangue dalle vent»,

diciamo mignarta , 0 mignella a uno , che € firetto del {uo , ¢ volenti sig :
quello d’ altri: A quefti tali pud quadyare cid, che diffe Orazio . Lon milfura ch
tem nif plena cruoris birudo , a :

V-AGLIARSI , Intendi dimenarfi come fa uno , che habbia rogaa , © altro pe

la vita , che fi dimena , ¢ fcontorce per grattarfi il prudore ; o pizzicore con!’ a

bito , che ha in doffo , ¢ fa con Ja vita un moto fimile a.quello , che fa und , che

vagli il grano. sae
TONCHI, Forfe dal Latino sondere pre(o per mictere Ȣ divorare, Sond vet

mi piccoli , 0 infetti , che fi generano nelle fave, pilelli, ed in altri i

 votano i granelli rodendoli ; da i Latini detti Curcudiones. Virg, 1 z

pulatque ingentem farris acervum Curculio , es visage

TIGNVOLE. Bachi fimili ma fi generano ne i pani 5 ¢ fogi impaftari; dai
latini detti Tee, Di quefte ne nafcono ancora dal grano , ¢ fi chiamano prt

noli. peas ah
™ MOSCIONT. Quei mofcherini 5 che na(cono dal vino , che dicemmo fopra in
quefto C, flan. 37. oe

Son

   

  

      
   
  
   
 
   
    
    
 

   

  

 

 

Se Fern Ee eae FE Sr FEKETE

 

    
 
      
 
 
 
 
 
  
  
 
     
 

SESTO CANTARE: 293
So eae eet fi generano nel legno 5 ¢ lo rodono; da i Jatiai

er) eet .
| RARE « Intende quei farfallini, che fi generano nel grano. Pyrau/fa,cou
4 areca (ono app: = farfalle pil grandi , le quali papain oe al
lume, e vis’abbruciano. Di quefte diffe il Petrarca. Semplicetta farfalla al lume

 

_ “COCCIOLE . Piccoli tumoretti, o enfiature cagionate da’ morfi danimal
come zanzare , bruchi ,¢ fimili . :

‘S8RANF, Rotture; Scorticature. Vedi fopra in quefto C. flan. 47.
PER riftoro. Per ricompenfa . Dan, Par. C. 5. ,

Gabino Dunque che render puoffi per riftore?

Equife ben pare , che il noftro Poeta voglia dire »per riftoramento, o alleggeri-

ia de i-teavagli,¢ pene, nondimeno é tutto il contrario, perché @ parlare -

.
P| ‘ ¢ vuol dire; oltre a gli altri travagli ha di pil , che lo flageliano,¢ pelta-
foe dese pieno di feudi d’ oro’. Quefta voce. rifore vien dal verbo ri~

 

ie derivante dal verbo reffawrare , ed ha quafi, lo: fteflo. fignificato ,
non che quefto vuol dire Acconciare , o raffettar cafe , ed altri materiali ; ¢

i dir Ricompentare , o rifar danni ,
ia Lo,4 Sy Nae ppi aunacordicella; i dendofi per
Py mbello quel facchetto-pieno di {egatura ,0 di cenci , che adoprano i ragazzi
'Perquotere i contadini,come dicemmo fopra C, 1. tt. 59. Zimbello detto, cred’
10, quafi cennelio , civé piccol fegno , argumentandolo dallo Spagnuolo , che il
chia five ‘

© Ub Re degli nfurai . Wmaggiore ufuraio del mondo. Detto che viene da i Gre-
| Gi yiquali chiamavano Re,quello che avanzava,fuperava , e vinceva gli altri ne i

en: 9

i) 29h giuochi fanciuliefchi; ed Afino quel che perdevay come habbiamo detto altro-
Vebs iy) :

Yi , SCORT-1CO? i! pidocchio . Significa effer avido del denaro , ¢ far’ ogni maggior

a per guadagnare ; fi dice /corticar if piducchio , per vender Ja pelle , € con

¢ Planto 6 pod dire, Vel unguinm prafegmana colligere. : 6

, DAR be ferecchio, Preftar danari a ufura »ed in vece di dar. denari effettivi,dar
aoe vaglia dieci oe venti. Vedi fopra C, 3. ft, 74. ¢d é Ja pid efecrandas

‘2, che fi trovi , ¢ forfe la pid praticata .
; MARCIRE , Intendiamo infradiciare , corromperfi, Dal Latino marceres ;
|

_SE non valeva un’ occhio, Se non fi vendeva caro,¢ a prezzo rigorofidimo : Non
vit cola pil cara dell occhio . Onde Catullo . Ni te pins oculis meis amarem
INT ARLARE , Effer mangiato da i tarli, o tignuole, che i Latini dicevano :
Cariem fentire,

E PESTO dai fui foldi, Infcanto dalle percofle di oa facchetto pieno delle
ae monete. Vuol moftrar in fomma il noftro Poeta , che
= Per qua quss peceat , per bec torquetur.

STAN.

 

 
 
 
   
  
    
    

294 MALMANTILE?@

STANZA LXL oe STANZAS
Va! altro ad un balcon balla , e coruetta y Dice la maga queftoe
Ch un diavol con tasferzaacentocorde uand ella i
Chrun grad'occhio dibue ctafcihainverta, | Cofkui ha fates quale
Prima gli da certe picchiate forde y Par non fo nulla,e no
Con una {pinta a baffo poi to getta Domandaa
dn cert’ acque bitumofe 5 elorde , ° Tal penaa chifi debbagda
Chee’ n’ efce poi sch’ ione difgradogli orci, Ed et che per feruirla é 9)
O peggiv d'un Norcin mula wines ‘ Prontamente cosh le da rifpofa,
STAN ZA LKIL + SD
Quei fu Zerbino , ed! amorofe dardo Ma dell’ occhiate fue ben pit ’
Moftrando,il cuor ferito,€ manomeffe, Hor fentene il riverbero', ¢:
Credeva il mio fantocciocon un {guarde E com’.ci gid pensd far alle das
Di (briciolar tuttoil femmineo feffo; Dalla fineftra é tratto in

Quel che fegue ¢ uno che peced d’ ambizione di bello’y¢ lindo 5 ¢ credeva’
Ja {ua bellezza di far’ innamorar tutte Je dame, ed hora'riceve la a
fuo peccato, p 94> being
CORVETT A , Salta. Cornettare & un certo faltellar de*eavalli 5dal Laci eure
uari, Spagnuolo corwar ; piegare , innarcare , torcere» EB quelto’
appropriato in quefto juogo per efprimer i) moto,che faceva coftui , il,
evitare le sferzate,era neceflario che falcella(se a tempoy ed in quella’
to , che fa il cavallo , quando coruetta. : >> SRDS
VN grand? cchio di bue ciafcuna ha in vette , Pone in vetta , cioe nella cima di
quefte corde, tocchio del’bue , ¢ non d’ altro animale ; perché bovis ocnle oculorum
pulchritudo , & nitor fiemfcatur , e trovatene I’ efempio in Omero , dal quale
Giunone é chiamata boopis , ciot bovinos oculos habens 0 vero Dea dagli occhi grandi,
¢ percid maeflofa . E coftui doveva efler gaftigato con la bellezza degli occhi;
perché con la pretefa bellezza de’ fuoi occhi , haveva egli 10 AO
PICCHIATE forde . Picchiate , ¢ percofle gagliarde , Percofle » che facciano
molto male , e non paia che lo facciano ; feruendoci in quefto cafo la voce fords
per la voce occw/to ,.come fi dice ricco fordo , per ricco non palefe , 0 non’ cond:
{ciuto . Ie
LE difgrado , Quel che vaglia quefto termine vedi fopra C. 3. ftan, 37. al ter
mine ho froppato. AMY
ORC/0 . Che cofa fieno orcj. Vedi fopra’C. 1. ft. 7. Qui intende orci da olia,
che fon fempre {chifi . a
NORCINO mula de’ porci, Coloro che in Firenze ammazzano i porci , € cost
morti gli portano fopr’ alle fpalle alle botteghe de’ Macellari,(ono per Jo pit del
paefe di Norcia , ¢ pero gli chiama mule Norcine , cioe portacors da Wercia ¢ O-
ftoro fon fempre tutti unti di gratio di porco , lorditimi , ¢ (chifi di fangue .
QVEST Ac ariofa . Quefta € cofa grande, ardua, e-che arreca:ftupore ; otra
ordinaria , ¢ ftravagante ,¢ che non fi pud credere , me
NON ono far gindizio. Cioé giudizio temerario,e falfo ; Maniera da Ipocriti,
¢ faifi bacchettoni {crupolofi , chelp
ZERSINI , Cosi chiamiamo quei giovani,che perfuadendofi d’ effer belli, faa
Z no

 

=

Ce et ee ee ee ij

aL
= aie oi

Sef asl 2 &

 

 
 
    

- STO CANTARE 295

 vanno lindi credendofi di far innamorare ognuno con la lor

quel dohenae PAriofto nel Furiofo deferive per il pili belio , ¢

¢ di quel tempo. E fi dice anche Mirtillo ;nome cavato'dal Gua-
fins Vedi forto C, 10, ftan, 30,

10 il. cuor ferito , ¢ manomeffo aeasmerslo gacdds Facendo da inna-

   
     
 

2. Nibbiaceo 5 Vecellaccio , ec. tutti feruono per intendere un
cimunito .

aR re in minutifiimi pezzi, o-ridurre in. bricioli,ed in-

morir di ‘patimo, ¢ disfarfi per amor di lui-tutte le dame .
0. +) Sinonimi che figaificano:li riperquotimenti ,che fan-
i del Sole , 0 il fuoco nella parte oppoftaa quella,dove direttamente
i Chimuci dicono ; Fuoco di riverbero., 0 di riflefo.. Qui inten-
‘coftui con quelle fruftate piene d’ occhi , ha il gaftigo dell’ occhiate amo-

egli nel mondo dava alle donne.
Reta is ake dame. Cioé fi come egli pensd che le dame cafcaffe-
Ja fua bellezza , ( il che appreflo di noi vuol dir farle morice
re), cosi egli ¢ buttato da qusi balconi entro al litame , per maggior
0 efti. tali fono {chizzinofi ne poflono. vederfi addofio un,
che guafti la Joro attillatura ,¢lindura,,

] STANZA LXV.

; ANZA LXIV.
an ch’ ¢ legato , e che gli ¢ pofto Qui Nepo fcuopre la di lui magagna ,
berrettin baffo a tagliere, Moftrando ch’ ei fu nobile , ¢ ben nato ,
-colpe colpo da difcofto E fempr'hebbe il Pedante alle caleagna;
fra. gliene facadere. Cc ontuttocio voll’ effer mal creato ;
Mifero fia quivi immoto , e tofto Perche fe e fulfe fhatoil Re di Spagna,
do gli occhi ai colpi dell'arciere, Mi cappello a nellun mai s' é cavato ;
muave Punto, ochinaorizza, Pero s ei fu villano , hora il maeftro
€.42 cultello che & infizra, GU! infegna le creanze col baleftro.
STANZA LXVI.
4 par comune ufanza , Se ¢' faltan la granata,addio Creanza,
4.rifponde al Galatrons ; ae ch? e° fien nati nella Falterona,

 
   
    

    
   
   
    

  
  

  
   
   

   
 

 

      
  
   

     
  
     
     
  

   
  

noi Fanciusi un pocon offer uanz A, Ata per la loro afinita Superba,
eyes: il . ahi baffona. Son poi fuggiti pitt che la mal erba;

   

“« Dattro. che fegue é uno,che nel Mondo non volle mai imparare.i buoni coftu-
Bi. non fi yolle mai cavar il cappello di tefta per riveric nefluno,per grande
j ‘ch ¢gli fulle, onde gli avviene il gafligo , che fi dice nelle prefenti ottave ; E
Masoasaa dice a Nepo , che hoggi di quefta forta mal creati ¢ pieno il Mondo.

pies (TINO a tagiiere. Berretta bafla ¢ piatta,nella quale non fi vede la

del capo , come {ono /e coppole Napoletave ,
eat Qgni volta ch’ ci tira. Vedi fopra C. 2. ftan. 57.
« Sta duro; Sta faldo; Sta fermo; Non fi muove.

(RCIERE . Colui che tira con la balefira « -4rciere in molti luoghi del noftro
do s' intende il Caprone , 0 Becco. Lat, aries.
(AG AGN.A. Mancamento, difétto . E parlandofi d’ huomini s' inane tan-
‘animo , che di corpo, Dante Jaf. C. 33. dice. OGe-

    
     
 
   

  
   
      
 

  

  

 

age 'MALMANTILE”
O Genovefi huomini diverfi 08 OND
“D egni coftume , ¢ pien d! ogni magagnd
Lalli En. Trau, ston erga i j up reeves J
¥ i bE pias vas

Ogni trattate conte ogni magagna |

Magagna in Lat, barb, & detta Mahaminm , Swan) Franz. Aabain ,

¢ vuol dire propriamente mutilazione di membra ,¢ fi ftende a fignificare « i

no ,¢ detrimento . Vedi Du Frefne nel Gloffario alla Parola Aabamiam ,
BEN nato, Nato di nobili , ed honefti parenti ,

       
   

HEBBE fempre it: Pedante alle calcagna , Hebe fempre il Maahre a te 4

gl’ infegnava i buoni coftumi , ¢ termini, cet Sar
MAL creat , Senza creanza, Vao-che non fai buoni termini 0 coftumi ,
VILLANO . Contadino’, Stintende uno fcortele , ¢ mal creato. Planco ra
merum , intende un’ huomo ruftico , fenza civilta , fenga galanteria , un pre
villano, Catullo, Péeniruris, & inficeviarum , [1 contrario di vidlane®, ,
SE faltan la granata, Se effi cfcono di (orto la cura del padre ;'¢ del macho,
Si dice faltar Ja granata ; quand? uno e(ce de’ pupilli che ini diero's ema
re ex Fpbebis, Dicono che quando uno ¢ arruolato per birro,debba far
mee a fare il noviziato ,¢ fnito quefto tempo gli faceian fare una cirion
faltare topr’a una granata,che gli mettono d’avanti in terra,e che fatta quefta azione
refti libero dal noviziato , ed.in un certo modo efca de’ pupilli ; ¢ da que r
monia (che fe non é vera , ¢ aflai vulgata ) credo 10, che habbia origine il pre
fente detto , F ‘hae?
PAIONO nati nella Falterona , Paiono nati in luoghi incolti,e difabigati,come
fono Je montagne della Falterona in Cafentino , dove poche creanze im-
pararfi , non effendo in quei luoghi con chi praticare , fe non con pecore ,€ por
ci, Ci ferniamo perd di quefto termine per efprimere un’ huomo incivile , ¢ foz"
zo, eche tratti da villano ; come ¢ quercubus , aut faxis natus , ae
SON fuggiti pit che a malerba. Nefluno gli vuol praticare . Sono sfuggiti 42
tutti. Malerba intendiamo I’ ortica erba nora , la quale & sfuggita da tuttl » pet

ché pugne .
STANZA LXVIIL STANZA LXVIIL

Ma chi é quel, 0 hai denti di cignale Ora per quefte fue finzioni eterne,
E lingua cost lunga 5 ¢ moffruofa ? Chi egli bebbe fempre nella mercathrs,
Si vede, che fon fuor del naturale Lucciole dands a creder per lanternt y
A me paion radici , 0 fimil cofa . Sharbata gli han la lingnaye denratirss
Wepo rifpofe; Quello ¢ un Senfale Main bocca havids pos di gran cavertty
Che fi i act il Parola ,ma (a glofa Perché non datur vacuum jn naturhy
Huom di fandonie , dice , ¢ di bugie, Glibanno a mifferio in quelle fhanze we
Perche in effe fondo le fenferie . Compofto denti , ¢ lingua di carate «

Segue un Sen(ale , il quale ¢ gaftigato delle bugie , che anna cavato
Ja lingua , ¢ identi , ed in quella vece meflovi delle carore . Ji Poeta fi ferue dell!
affioma Peripatetico! Won datur vacuum in natura , col quale ingende che fulle ae
cefiario riempier quei voti ,cagionati dal’ eftrazione della lingua , € deati » Ae
(cherza , fapendo bene auch’ egli ,che quei medefimi voti erano gid ripienl #
aria. >) ” ae ;

    
 
  

S82 rennrer® BRfs2-nPhee

~

See

 

 
   

‘fon mediatori afar vender una mercanzia :
vin Birenzeun'Senfale di beftic, huomo {ccl-
oy che per le fue farberie:fu impiccato a: forche erette a pofla per
0 ee eealns 3 ed é lo fteflo, che quegli che fu,
chino detto | 3. fhe 55.
OIE ye bagi { aiclonaoe dal vero, ¢ fono fi pud dir finonimi , fe
Ml dir chiacchi vana , ¢ bagia propri vuol dire attefta-

fenferia . S' intende,quando uno di quefti Senfali fa vender qualcofa,e

    
 

per lanterne, Dar a creder una cola per un'altra, 1 Lalli

     
 
 
  
   
  
   
     
   
 
   
          

| Lucciole qui rimiro per lanterne ,
£4, Bi quel vermicello alato, che di notte riluce da i latini detto Ci-
Nottiluca ; da’Tedelchi animaletto di'S. Giovanni, ¢ da’ Greci Lampyris dal
e fai egiare nelle tenebre , come egli fa ;\¢ /anterna & quello arne-
; ‘quale-fi porta il lume la-notte ferrato da talco , offo., 0 vetro per di-

jo dal vento ; ed ¢ voce.pura latina ,

4°) Specie di radica , Latino /iser, Mail proverbio Pisntar , 0 fecar
ca dare a creder bugie . Latino imponere alicui , Onde Impoftura, es
febene fi dice in pil grave fignificato . Vedi fopra C. 2, ft. 70. Dices
yperché vi fon meffe tali carote,é non folamente per riempiere i
perdar il gaftigo a‘coftui delle tante carote , che effo haveva piantate,
era in'vita’, facendogli haver {empre dentro alla bocca effettive , ¢ natu-

a. ‘ANZA LXIX, STANZA LXX,
See volta ha la facia, Vedi colui',¢ al colle ha un’ orinale ;
“Bute diavol legnainolo in ful groppone Cieco , rattratto lacero’, ¢ piagato ?
_— Gli afcia itlegname,fega,ed ipialliaccia, Ei fu Governator a uno fpedale ,
(Ste feruir per [uo pancone ; Ow ei non volle mai pur un malato,
a fu; c'alla pancaccia Ora per pena ogni dolore’, ¢ male ,
aglian le legne addofsv alle perfone , Che gl infermi y' haurebbono portato
tener {a lingua in brigiia ( Mentr’ alia barba lor pappo st bene )
ender. lapariglia , Sopr’ al uo corpo tutto Guanto trene ,
‘il gaftigo dato-a* Mormoratori, ed a quelli, che, eflendo Aati Sopranten-
| Spedali,non hanno havuto carita; ma {olo hanno attefo a crapulare per
Manodion 3 che dovevan fomminiftrare a’ poveri , ed infermi.
edi

    
   
  
 
 
  

VE; Codrione. Le parti di dietro del’huomo fra le reni , € Je nati-
fowo C, 10; ft. 50. LU Perfiani difle ,
“\ Céafcun teme , e fi caca nelle brache
Tn vederus appiccato ful groppane
t ® “Lo frocco da fcannar Je paftinache , -
fi cava che ¢ ufato, ma per lo pit in {cherzo,- Viene fecondo i} Ferrari dal

 
   

 

Orrhopyginm , che fignifica lo fteffo . Hs
ARE , Tagliar con l’afce, che é uno ftrumento da legnaiuoli noto,chia-
Pp mandolo ©

. =.
a ooo

   

 

 
 

Tar ae

298 MALMANTILE ©

mandolo cost anche i Latini, che lo dicono ¢4/cia. Ifidoro neilé Origint lib, 19)
6.19. Afcia ab affulis dita quas-a ligno eximit yenius diminutionm nomen eff afciole
( forfe accetta ) Eff autem manubrio brevijex aduerfa parte referens vel, i
Jexm 5 vel canatum , vel bicorne rafirum, Vitruvio difie Afciare Lib. VI. ¢. 2. Suma.
tur Afcia , © quemadmodum materia ( Qui invende il ego; che gli Spagouoli dal
Latino chiamano , madera ) dolatur , fic calx lacn macerata afcierur , Am,
[MPlALLaccLa , Qui la rima forfe ha necefitato  Autore a feruith di
guefto verbo impiatiacciare in vece del verbo piallare , che vuol nee
-gnami con:la pialla come intende qui 5 ed il verbo impiallacchare vuol dire tito
prire un legname con piallacci ( fefftles lamina , famine pratenues \e diffe Plinio}
fond fottilifene afircelle di noce , con Je quali fi cuopre altro legname pit vilei
far cafle , tavole, ed altro, nella forma che fi fa con ' ebano , granatiglia , ed
tri legnami nobili. Plinio difcorrendo di legnami , de’ quali gli antichi fi (erui-
vano per impiallacciare lib. 17. 43. Que i laminas fecantur , qnorumque Z
weftiatur alia materies , pracipua funt cedrus , terebinthus, etc, E poco 2
prima origo lnxuria 5 arborem alia imtegi , © viliores ligno pretiofiures cortice fier; B
PO, Lvcugitate funt , & ligni brattea , nec fatis, Capere tingi animalinm cornua.
dentes fecari , liguumgue ebore dsfpingui , mox operiri =
LALLA, Chiamano i Legnaiuoli quello ftrumento di legno’, che ha un ferro
incaflato , col quale aflottigliano , appt » pulil 9 ed addiri: ile
gnami, da i Latini,fecondo molti,detto Dolabra, ma forfe con qualche
‘Vn’ antico Grammatico pat che 1a confonda coll’ alcia.. Dolare fabri’ Co
afcia ledere , Si legge in Colum, lib. 3. Qua falce amputari non poffunt, dette doles
bra abradito ,il che pare che voglia dire pili tofto accetta , 0 pennsto , ovanga;
che pialla: E corrobora quefta opinione i! medefimo Colum. lib. 4. ¢-24, feruen
dofene in diminutivo ; Semper circa crus dolabella dimovenila ‘ef revra , clot Inter.
no al cansbo della vite ¢ da levare (a terra con una dccettina , 1) Calepino tiene sche
da pialia fi dica runcina , e porta ’ autorita di Plinio lib. 16. cap. 42. edd éncitares
runcinarum raptus , ove pare , che defcriva appunto P operazione della yf
per infino l arricciolinamento de’ trucioli: Tutto il tefto dice cosi : Br ad quack
que libeat intoftina opera aptifima ( parla de}l’ abeto ) five Graco , five Campanty,
ficulo fabricae artis genere spettabilis , ramentorum crinibus pampinate ae
be fe voluens ad incitates runcinarum raptus , Ma io ardifco conteaddirgli ots
Y autorita d’ Hermolao che dice : Runcine [unt maiores ferra , quibus fabri materis-
rij fecant arborum moles fubiettis canterijs, Si che non la pialla, ma fa fega grande,
‘che adoperano i Marangoni per ricidere i legaami , adattandoli fopra quel C4
Valletti , che noi chiamiamo canreo ( dal Latino cantherins , cio’ cabalus: -e pil
volgarmente pietiche , i quali {ono one di due correnti inchiavardati ial
a guila di cefoie (che propriamente fi dicono pietiche ) ¢ d'un’ aleco pezzo'di cor
rente , che fi merte a traverfo alle pictiche (¢ queito fi dice Canteo ) € a
cosi un triangolo vi adattano per via di piuoli il legao da fegarfi , Runcare é tet
‘mine d’ agricoltura , ‘che vuol dir propriamente tor via, onde fe ne formd per
yeotura la parola antica Latina averruncare , cio’ avertere ; ¢ fe ne Iddio
wAverruncus detto Cosi , perché ab'eo precari folent , ut pericula avertat; fi come dice
Watrronc., E in propofico d’ agricultura (ene fabbricarono le parole aie .

 

   

fF EP Re ewe

a
wt

 

 
 

SESTO CANTARE: 299

Ronconé yle quali fignificano ftrumenti da nettare.i campijda rimondare fructi , ¢
i cinne raat Plinio lib. 18, ¢./21. Siligioem gfe striven. sfemen, shepeleas

accato  farrita 11 iB « Runcatio, cum feres in articulo off , evulfis inu-
. joo eoremng apes radicem indicat, Segetemque difcernit a cefpite. EB Catone cap.
even cremarique ; ie che pili tofto Runcina parrebbe , che avel-
_ fead eflere la roncola , 0 cofa fimile , che la fega , o 1a pialla. Ma forfe non.
tanto. il Calepino , quanto anche il Vocabolario delia Crufca dal levar via , ¢
a faellere ,¢ ripulire ( che quefto fignifica , comes’ ¢ vitto il verbo Runcare ) hanno
dato il nome di rancina alla pialla),perché clla pulifce , appiana , ¢ leva il fover-
da! Jegnami.. Tuttavia anche per quefta ragione 1a dires do/abra , perché fi-
quefla ancora pulilce , ¢ rade , come dice Colum, nel juogo (opra cita-
fia. come efter fi voglia,poco fa sal rth noftram, baftandoci intendere, che

[ene guello ftrumento da legnaiuoli ; che habbiamo accenaato ,
 PANCONE . Chiamano i degnainoli quella loro panca grotia , fopra la quale
ilegnami per lavorargli, deta pancone , perche ¢€ fatta d’ un paycone
aoe un’ afie grofla-circa un. quarto di braccio » che fono affe da rifen-

 

“ALLA pancaccia Cosi fi chiama quel luogo dove in Firenze fi tiene il croc.
dilcorre de’ fatti d'vaitri, edelle nuove . Vedi fopra C. 2. ft. 73, E per-
ildir male del profimo fi dice Tagliar le legne addoffo.a uno. Latino famam.,
icnius lacerare profciadere , pero.a coftoro vien dato il gaftigo adeguato , cons
Halab loro addofio il legname effertivamente .
TENER (a lingua in brigiia , Parlar confideratamente , ¢ con riguardo , ¢ fi di-
ce anche : Tener la lingua a freuo .
ghtNDeR Ja pariglia . Render il contraccambio . Parigiia vuol dire una cofa ,
pud dividerfi in due parti yguali ; come nel numero due fi pud far’ uno,¢ uno.
£ ‘to render pariglia vuol dir render ugual contraccambio .. Vedi fopra C, 4. ft,
ma pwr pari referre de’ Lat. Dan. nel Parad. C. 26. dice :
x Perch’ io lo veggio nef verace /pegtio ,
yh « ~ Che fa di fe pareglie L altre cofes
E nulla fece tui di fe pareglio .
Hoggi perd in quefto fenfo ,¢ maniera , che fi ferue Dante di quefta voce pare-
44 non mi pare , che fi ufi , fe non da’ Franzefi 3 che dicono pareil ,

ALLA barba loro. A fpele loro, Quefto termine efprime Pigliare, 0 confuma-
re una cofa d’ altri contro al guito » € volonta del padrone di efla ; 0 a difpetto,e
208 del medefimo .

AP PO. Cioé mangid, Donde Pappolone uno che mangia aflai che vedem

FER SSLA SES SCES CLESSSE RES

 

Hi

laedinee 1 6.

4 eee NZ A LEXL

5 ‘Chia, coltui , ¢' habbiamo a. dirimpetto Che non ne pag mai un maladetta 5

é (Dice la donna) a cui guegli animali Tenne gran pofto , fe {pele beftiali ;
_ Sharban con le tanaglieilcuor del petto? Ma pai per foddssfare ei non hauria

fw rifponde:.Quefto ¢ un di ques tali, Voluto men trovargli per la via ,

a a

; Pp z STAN-

 

 

 

|
|
 

300 MALMANTILE ©
STANZA LKXILe 0 ooo) oo SDANZA\LERMD
Colni, ¢ hail vifo peffo, eit otto... Riferva il. muragche c' ni
‘Da quei due [pire in femimils fpoglie »._. Donne ,che fe a eH
Hus vile fu, ma bifeaiuoloxe ghiogtoy’ >  D* arir giviellare , ¢ luce

Che fi volle cavar tutte le voglies\ Dar ile... al mario in i
Ocni fera tornava a cafa cote, she Hor le fuperbe pierre; t
E dava col baiton cena alla moglie ; Alla lor liberta fanno il

Hor finti quella Peffa quei demoni, © Pero che tanto erandi , ¢ tant!

Sopra di lui fan trionfar baton...» ».. | Chan fatto per lor carcere’
Termina Ja moftra delle:pene date'a idelinquenti con tre forte
i] primo é dato a coloro, che non vollero mai pagare i loro debiti. Iba,
dato a i crapuloni ftrapazzatoridella mogtiex ll terza & quello dato alle do
ambiziofe ,e vane. ag janguu vik > > sitll
TANAGLIE , Strumento di ferro fatto a foggia divcefoia, e ferue V
chiodi da i legni,, ec. Da i latini detto forcipes , ig Mie ta hea's
NON ne pag un matadetro. Non volle mai pagareun debito , Non pagd mii
un quattrino di debito . L’ epiteto ma/aderto ha la forza d’un becco d'un
no decto fopra C, ¢. f..68. ) vast tye
TENNE gran pofto, Si trattd alla grande), e'fece {pele bepiali', cioe grant
inconfiderate . Lat, immanes . : 23 elit Held eg?
NON hauerebbe volute crovargli per la via, Quand’anche egli’ havefle trovato per
Ja ftrada il denaro , del quale era debitore;non haurebbe ad ogni modo’ pagato il
fuo debito . Quefto termine ci ferue per efprimere’, che nefluna cofa hat
potuto muoverlo dal fuo propofito , ¢ fargli venir'voglia di pagate . ~~
PESTO, Infeanto , ed ammaccato dalle-battonate , che gli danno quel De-
moni finti la fua moglie . E quetto vuol dire trionfar ‘bastoni ..) |» t
AYO M vile, Qui vuol dire huomo di bala condizione . *
RISC AWOLO . Huomo che pratica le bifche «| Bifche diciama quei raddotti
pubblici , dove fi giuoca a carte , ¢ a dadi; nome forfe'venuto dal verbo bi/ear-
sare, che vuol dir Mandar male {propoficatamente il fao havere : ¢ corriff
al Latino prodigere. L’ usd Dante nell’ Inferno C, 10,
Bifcarca ye fonde te fue faculeade , ag A
GHIOTTO . Huomo,a cuipiace mangiar del buono’. Vedi fopra C. es
DAV-A col bafton cena alla moglie » In vece di portar da-cena alla moglie ; la be
ftonava , Coftume affai ufato dalla gente’ d’ infima plebe , imbriacarfi all’ofteri¢,
¢ non penfar’ a mandare da cena a cafa alla moglie , ¢ cosi-briachi'tornare @/¢a-
fa, ¢ perché la povera moglie fi duole d’ eer digidna’, baftonarlas
DAR del c,,. in fal laferone, Quand? un mercante fallifce; diciamo + vitalelt
dato ile ...ful laftrone . Beanetto Latini nel/Patafio-difle Dar det Ad wy
Quefto proverbio é nato'da un’proverbio antico,che era ‘in Firenze; chee *t
i quali fallivano , © rifiutavano-l’ eredita del padre’, “andavano ‘nel mezzo:
Mereatunuovo ( luogo dove fi ragunano i mercanti pers iare)e\gaiei era y
¢d € ancora una gran laftra di marmo tonda,che fi chiama i} carrotcia( perc evi
& poita per fegno,dove fi fermava il carroccio , fopra il quale s' inalberava Pinte
gua generale de’ Piorentini , quando andavano alla guerra ) ¢ fopra —
a:

Shy

  

  
   
  

Bese e see we OE pm ERE ESSE SS ee EE

OPS SSEREG EES a oe

 

 
  

jor

. awifta del popolo, che nell’ hora, che fi doveva fare tal
10; & quefto atto afficurava Ja loro perfona dalle mo-
‘di debito s:ne potevano i creditori moleltare fe non la roba , las
deva ceduta tutta a favore de i Creditori, non effendo per quefto
(0 il debitore a pagaie ultra vires , eflendo quefto come un cedo bonis del
‘dus . Cosi queftra laftra alle perfone de’ falliti , che a quella_rifug-
era come una Ara, © vogiiam dire altare , 0 luogo facro, 0 afilo, o 1
gia , che dail’ efier prefi gli afficurava , ¢ quelto , perché cflendo dedicata
pubblico di foftenere 1i folenne:carro , ¢ Ja tanto famofa infegaa della
endeva per quefto riguardo franchi , ed immuai coloro , che col fe-
 prendevanne folennemente , ¢ con cirimonia il pofscfso . Di qui dar
laftrone yuo) dir-fallire . Edi qui pure,quand’uno cafca, ¢ batte il c...
lle Jaftre diciamo < // tale ha rifiutato il padre, Fallire ancora dichiamo
i pole : Ef rale U’ ba infilate ; che corrifponde al Latino decoxit ,
TTON/, Sono il latino dateres detto fopra C,1. ft. 67. £ fare, 0 dare il
» Vuol dir fare a uno qualche danno grave ; ¢ qui vuol dire ; fono il lor ‘

,¢ pena.

TANZA LXXIV. STANZA LXXV.

in orecchiche mi par che e fuont Dice la Maga; Vo venir anch’ io ,

po tabellaccio del Senato , Perch'il veder pin altro non m' importa,

| Sichee’ mi fa meftier chrior abbandoni , Ed in quefta Cittd cosi a bacio

\ LiPeri cht io non-voglio effer’ appuntaro ; A diria mi par @ effer mexza morta;

i v ch reStavano i lion, Vaglia trattar col Ke d'un fatto mio,

Ma non poffo venir chro fon chiamato, Ed andarmene poi per la pit corta

Ed ecco appunto Diavoli co’ lucchi ; Ed ei le dice in burla; Se ru parti,
Peri lafcia ch'io corra,e m'imbacucchi, Vaviain un'ora,etorna poi intrequarti,

. li fuddetti gaftighi dati a i delinquenti , Nepo fentenda la Campanas

del Senato fi licenzia dalla Strega , ma dovendo efser’ anch’ ella nel Senato per

Bat Re, dice volerlo feguir {in quivi, di dove {pedita fe ne yuo] andare per

  
 
   
 
  

 
 

 
 
 
    
  
  
  
   

'’*

 
   
  
   
   

   

 
 
 
 
    

STAR i orecchie. Afcoltare con attenzione . e4uribus arrettis aufeulrare ,

© TABELL-AC C/O . Cosi é chiamata da molti la Campana del palazzo de} Po-

deft ( hoggi del Bargello, 1a quale & detta la Maddalena ,come yedemmo fopras

ein guefto C, flan. 23. ) forfe dal latino T-abeltiones , che vuol dir Notai, i quali di-

Moravano, ¢ tenevano i lor banchi dentro, ed attorno al detto palazzo,ragunan-

~d0vifi al fuono di detta campana , Ja quale hoggi ¢ detta anche /a furba, perche

lori d’alcune feftc, non fuona, fe non per efecuzioni criminali di tefte, ¢ forche,

_ €/a nowe per moftrar P hora , che non fi pud pill portare armi; 0 pure é cosi

___-detta dal fuono ofcuro ,¢ malinconico , o che almanco rapprefenta cofa mefta ,
‘come il fuono delle tabelle ne’ giorni Santi.

4 NON celia effere appuntato , Coloro che'fon de! Configlio del Dugento, ¢ d’al-

“tri Magiftrati di Firenze fe non vanno al detto Configiio,quando fi raguna a fuo-

: erence condanaati in certa fomma di danaro ; ¢ queflo diciamo

 
   
    
    
   
     
  

2YCCO , Br la fopravvelta,o mantello Curiale di Firenze , ed cra anticamente
nay

V abito 4

 
:

   

    
      

302 MALMANTILE: >

P abito civile ordinario ; e perch queflo haveva gia un
metteva in doflo detto lucco , fi doveva dire imbacuccarfi. Va
141, Subito fu prefo ; ¢ imbacuccata col cappuecio , fu condotto alle carci
C. 11, ft.22. a Pou
e4 B.AC/0 , Campagna , dove batte poco il Sole, che diciamo Al rezi
uggia. Vedi fopraC, 3. ft. 71. alla voce Vria, ¢ foto C, 9. ft. 44, ¢C, 10,
1 contadini in vece di dire : Iuogo o piaggia volta a mezzo giorno , dicono:
Jatio , ed in vece di dire: volta a tramontana , 0 a fettentrione dicono: ab
© a paggino che ¢ i) contrario di folatio, Credo venga dal Latino
fi come natio da natixus . Da molti fi dice meriggio quel luogo,dove
no i raggi del Sole per interpofizione di che che fia,¢ ( pare a prima'
troppo lodevolmente , perche meriggio da meridies vuol dit mezzo giorno ,.
do appunto i raggi del Sole fono pit: quocentije perd andare al meriggio p
be che volefle dir pit tofto andare a fcaldarfi a’ raggi del Sole di mezzo
che andar all’ ombra per difenderfi da i raggi del Sole . Per corroborazion
quefto idiotifmo , fi uova in Autore approvato per buon Scrittor Tofcano
vollero fare il viaggio di notte per lo gran freado , ma si bene in full’ ora met
allora cheil Sole con i fuoi raggs haveffe addolcito i rigori hiemati.Maqueftitalifid
dono con!’ ufo , ¢ potrebbe dirfi anche colla ragione, perché meriggio nel fi
to di luogo ombrofo , e difefo dal Sole ,é lo fiefio, che juogo da paffare I! ore
del mezzo di , la quale cofa i Latini dicevano meridiari , Catullo. dube adte
aeridiathm , Ora dal meriggiare , cioé fare all’ ombra nell’ ore calde é
meriggio , ¢ , da meriggio , rezzo.. Va in un’ ora, ¢ torna pai in tre g.
€ uno (cherzo ufato aliai fra gente bafsa,ed intende Va hora in uno,ciok
¢ torna poi divi(o in tre quarti ; fij impiccato ; fe ben pare che voglia dire: Va»
in un guarto d’ ora, ritorna in tre quarti . Cirimonia da Diavoli,
STANZA LKXVI, STANZA LXXVIL —
Tun vnoi gli rifpos’ ella ,fempre il chiafo; Ed ella per oferta cost magna hah
Wel Con/figtio cost ne va con effo Ringraziamenti fattigli abarellay.
Ove ciafcun P honora, e dalle il paffo , Dice,c'hor mai sbrartar yuol la capagnly
Sbirciandola un po meglio,e piu da preffa, E tornar a dar nuove a Bertinella, —

Ella baciando tl manto a Satanaffo Pluton le dd ticenza,el parse
Fino alla porta ye [i fe ne (gabellay

Lopregad' ofsernar quanto ha prome/so,

Ei.ghe! conferma , e perche ftia ficura, Ond' ellain Dite aun Vetturin saccopty
Per la Palude Stige glielo giura . Che la rimeni a cafa per la pofta *
La Maga cosi {cherzando , ¢ burlando con Nepo fe ne va con effo in Confi-

glio , dove ognuao I’ honora.. Fa riverenza a Piutone , ¢ lo prega a ma

quanto le ha promeffo; Eigliclo giura foleanem:a:2, ¢1 accompagaatala fino
aiia porta del Configlio la liceazia , ed ella va a cercar d’ ua Veccuriag , che la

riconduca per la polta a Cafa, , 8
Tu vuci si chiafso, Tu yuoi la burla. Tu {cherzi. Chiaio nel proprio ¢ 3

ftrezta , vicolo Lat, vicus quali erano le ftrade di Ro oa aatica, edel pri

cerchio in Firenze. Gio, Vill, 10, 29. S apprefe fuoco ix Firense in Borgo S.

Appoftolo nel Clafso tra’ Bonciani ye gli Acciainoli,B pecché in quette ftraducole abl-

tavano taluoita donne di mal aftare , Chuatio ¢ detto forle da Vicus Vicario , Ba-

sas

     
 
   
   
    
 
         

 

       
     
 
 
   
   
   
  
    
   
  

   

 

a2EH@ Ze O- Fe LED

  

 
 

FS Sue

SESS SEAS a”

==

 

ESET ER EC SEE 8 Se

a

 

an SESTO CANTARE jes

ga; in buon Latino Vicinia ) venne a fignificare Pofribolo , ¢ perché in tali difo-
nefti Inoghi fi fa gran baccano ,e ff {cherza, ¢ fi burla fenza rifpetto; percid
‘ iglia per burla,,per ifcherzo. Se bene ¢ molto verifimile, che in quefto

hielo
ultimo fignificato di ftrepico , ¢ di baccano , quale fanno quelli , che licenzioia-
__-menze tratcano,¢ burlano , venga dal Latino de’ tempi baifi ; che il fuono di

» © degli organi, ¢ degli altri ftrumenti domandavano C/aficum ,

tutte le campane
che i buoni Litini dicevano della rromba , a cui fon fuccedute le campane. Lt

lo dice Glas,

| SBIRCLAN'DOLA, Guardandola bene . Vedi fopra C, 1. flan. 9.

9.
 GLIELO ginra per la Pande Stige. Giuramento folenne , ed inuiolabile degli
Dei fecondo la falsa credenza de i Gentili , come fi cava da Omero in pil luoghi
del Lliade 5 ¢da Verg Zn. lib. 6. :

: Stygiamque paludem ,

 

ea
>t: Dif cuins inrare timent , & fallere numen .
« la ragione , per la quale quefto fia giuramento folenne ,fecondo Seruio,¢ quefta

4» Styx'meerorem fignificat , Dij autem lati (uat femper ; ergo qui meerorem non
y» featiunt rant per triftitiam , que res eft fue macure contearia ; ideo Lufiu-
per execrationem habent . L’ altra ragione ¢ , perche havendo Vitto.

iuola di Scige aiutati gli Dei nella guerra contro ai Gigaati Ticani, Gio-

ve per rimuneracla , volle che coloro , che giuravano per Suge di ici madreo ,
fullero privi del nettare delli Dei, e non offeruavano il giuramento. & quefte
“sole furono finte , ¢ credute di Stige , perché (econdo Teofrafto queito Suge era
un fonte in Arcadia , le cui acque , ¢ pefci erano velenofi per la di ini eltremas
frigidita ; ¢ di quefla acqua dice Plin, lib. 30. cap. 16. che Aatipatro voleife da-
re ad Alctiandro Magno , quando volle avvelenarlo per configlio d’ Arifotile .
yy Vogulas tantim mularum repertas , neque ullam aliam materiam , que non,
a» percoderewur a veneno Stygis aque , cum id dandum Alexandro Magno Anti-
a» pater mitteret , memoria dignum eft, magna Ariftotelis infamia excogiratum .
A barelia. ln quantita geande , Si dice a balle a maffe, a facca, ec. fono
pero modi bafli , ¢ pid tofto {cherzofi, ¢ s’ ufano parlando tanto di cofe corporee,

quanto incorporce ,

\ SBRATT_AR la campagna, Andarlene : Sbrattare propriamente fignifica net-
tare 50 ipulire , contrario d' /mbrattare ; fi che sbrattare él paefe yuol dire ripu-
dice il 9 © per confeguenza andarfene da quel luogo .

SENE (gabella . La la(cia ; Sisbriga ; fi libera , ¢ filicenzia da lei. Dedotto
dalla Gabella , che fi paga , perché, come é pagato il dazio , o gabellad’ una.
Mereanzia , fi dice sgabellata , ¢ cosi fi {pedifce , e manda via.

BITE «. Qui la Citta di Plutone , detta cosi da divitie , le _ ci vengono tut-
tedi fotto terra. I Latini chiamarono Due , quel che con Greco yocabolo dice-
vano altrimenti P/zsone , che vuol dire il medefimo, € fignifica il ricco Lddio ,
Addio detie ricchezze , come s’ é veduto fopra .

WET TVRLNO , Coini che prefta cavallia nolo , 0 a vettura .

STAN.

 

’
4
|
 

  
   

304 MALMANTILE: —
3 STANZA LXXVIILA
I Re fatta con tei la dipartenza | Saliro alla,
Al falon del Configho fene torna ,
Onde ciafcuno alla eos
Alza il Civite sé abbaffa gii le corna,
Plutone licenziata la Maga fene torna in
fua refidenza fi prepara a difcorrere .
FATTE le dipartenze. Licenziatifi (cambievolmente .
ALZ A il Civile, Alza le natiche. Civile € una profpetti
fentancte abitazione di Citta ; contraria a quella , che-fi dice of
pagna . I Latini fimil ue entrate principaliin ;/
di quelli che venivano dalla piazza,o dal mercato ; I’ altra di coloro., che fi
geva che venifiero di lontant pacfi , o di fuori dalla Citta; La prima ente:
diceva a foro , 1’ altra 4 peregre , ficcome riferifce Vitruvio . Noi per quelto —
miamo Foro la parte in Paccia della {cena , Lin eRe
RAGNI, Quci veli che fanno i ragni .. Narrano le favole degli antichi
li,che in Lidia fa una femmina detta Arachne nata in contado di bafla
quale fu cosi valorofa nel ricamare , ed in ogni forta d’ artifizio di tela ¢!
che non folo fuperava tutte |’ altre femmine , ma hebbe ardire di co;
la Dea Pallade ; onde Pallade fuperata , ¢ vinta da lei, per difpetto le;
Javoro ,¢ la conuerti in Aragno verme, che é quell’ infetto che fi
veli per pigliar le mofche da noi chiamato , ragno , 0 raguarelo. Ovid,
tam, Dante nel Purg. C. 12. rocca quefta favola , -
O folle-Aragne , fi vedeva io te
Gid mexz? aragna triste in [u gli ftraceé 2
Dell’ opera , che mal per te fi fe. bie é
DR APPELLONI , Cost chiamiamo quei pezzi di drappo i quali ee
no pendenti al cielo de i baldacchini , ¢ delle refidenze de i Prinejpi; ,
rano le Chiefe , ec. Varchi St. Fio. lib. 14. Ed al vano dela Cupola era tirate in fu
Le funi wr beliffimo ortangolo di drappelloni . Matt, Villani lib. 9,.cap. 43 deferiven~
do le nobili efequie fatte nella fepoltura dei Cavaliere Mefier Biordo degli 4
uni. E fopra nha xn drappo a oro con drappelloni pendenti coll’ arme del peice} .
comune ,e di parte Guelfa',¢ degli Vbertiné . Tali drappelloni coll’ arme fi
appiccati in gran numero nella infigne Chicfa Collegiata di $, Lorenzo un
giorno dell’ anno , per memoria di antichi benefattori , ee
SPVT-A un ciabattine. Quando uno per foprabbondanza di catarto ha difficulta
in fpurgarfi , fogliamo dire: gli ha wn ciabattino giis per la gola ye doy
Sputa un ciabattino , —_ a ee oe ¥ - cme se nel Lal
oni, Coll’ o¢chiaia lnvida toffire , e {purar far, te 2 r
Spee Fan STANZA LXXIX. noaniaie mM
Spiegar volendo poi quanto gli occorre, Onde nui fiam quaggit in. fondo di
Comincia il [uo proemio intal maniera; Gente, a cui fi fa notte avanti fers
Voi che di fopra al Sole in queste forre Voich' in malizia,in ogni frode,e
CadcSts meco all’ arta ofcara ,¢ nera , Siateé Adacftri di color che a)

  
    

  
  
 
 

 
      
  
  
  
   
  
 
 
   
 
         
   
  
  
   
   
  
  
 
  

  

ug

Ce eR eek ai kB

   
 

 

 

+

SESTO CANTARE, 305
STANZA LXXXIL
Cominci il primo : Dite, Mdalebranche ,
r ¢ } Quel che e'vi par che qui v'adafe fatto,
‘bazzicar taverne , e chialfi Levato i Tocco, ¢ follevace t anche

5 agnun di voi st bravo ,edotto,
ea vo.
ib pincrife

1 Alor quel Diavol n' un medefmotratto
a

Vn capitombol fa fopr’ alle panche ,

 

  

I a un famiglio a’ Otto ; Efalta ae nel mero com! un gatto s
aunque , benche pare Cittadini Ma perch'il Lucco s'appicco a nn chiodo,
el vieupero ingeg ns peregrini ‘ Si ric e , ¢ parla a questa modo :
TANZA EX xr, STANZA LXXXIII,
tusti'in correfia O Re, cus {plende in mano il gran forcone
Da Martinazxa nofira confidente , Sil Cappello (periale ba quel fegreto y
_ Poithe Baldone ancor cerca ogm via Col qual fi fa feornare un pedignone ,

Dh entrar in Malmansi{con tantagente, fo ? bo da far tornar un’ buomo a dreto.

ar-ch?eglisbandi , e trucchi via So gidche qualche debito ha Baldone
| Adbope Mctitentensy z che ¢ lo vuol pagare in (ul tappeto , ‘
pe ere [opra quefto il fuo parere Percid manda Pedino (a in campagna ,
. ‘the ©! ci fuffe da tencre’, Ch’ ei ginocherd di pofta di Calcagna ,
: io de i Diavoli fr: compofo dail’ Autore; dopo che egli ottenne
meat » nell’ efercitare i} quale conobbe I’ autorita,che fi ufurpano i Can-
: +s anon hoe be » metce per Cancelliere di quefto Configlio un Ciappellet-
celliers che fun notaio (cellerato, fecondo che riferi(ce il Boccaccio nelle {ue Novel-
leye bcontraddica a tutto quello',che vien propofto . I nomi di quetti
colt pis fon cavati da Dante nel {uo Inferno; ¢ fappia il Lettore, che li {pro-
he dicono,fon poco lontani da quelli,che PAutore sétiva dire nel areating
rid iperfonaggi che finge in quelti Diavoli fono fimili alli faoi Colleghi,
ed egli medefimo in leggermi quefto Canto mi diceva ; il tal Diavolo é fimile al
tal mio Coliega ye il tale, al tale; ¢ mi parvero appropriati beniffimo ; non fti-
mo gia bene nominargli. Ma tornando a propofito dico , che Plutone volendo

 
  

 

fens re de’ {uoi Senatori , fatta una breve orazione nella quale inferiice
un ver | Petrarca Gente , 4 cui ft fa notte avanti fera, ed uno da Dante Siese i
Mathri di color che fanno , ordina a Malebranche il dire , quel che egli farebbes
per mandar via Kaldone da Malmantile,ed egli,fatte prime fue diaboliche cirimo-
nie dice che il {uo penfiero farebbe di farlo citare alla Mercanzia da qualche
fuo creditore , salea:

FORRA, Valle lunga , ¢ ftretta pofta fra poggi alti , onde poco dominata dal

3© perd ben detto forra il pacfe infernale dove non batte mai fole .

_ GENTE 4 cui fi fa notte avanti fera , Con quelto yerfo del Petrarca , ? Autore

_ intende che coftoro fon fempre di notte , cioc al buio .

, BABLV ASSO, Huomo fenza giudizio , icimunito. L’ origine fua & {cura ;
forle da Valwaffor parola feudale , dalla quale ¢ fatto anche Barbafforo , lo Netio
che » o-dortoraccio ; faccente ; ¢ che fi da {cioccamente ad intendere di
fapere 0 pure da Bwaccio peggiorativo di bue. Vedi fopraC. 5. flan. 1. Ul Bini

in lode del Malfrancefe dice .
Qq Eri-

 
 
 
   
    
   
 

Mi =p.

 

 
 

TPE

 

305 MALMANTILE | 3
Erifpondendo a certi Babbuaffi , {cea 2c bia
Che voglion dir , che quefta malattia i eS ntn se
Tatto il corpo ci florpi , eck fracali. Ah eat RRR

 

Ed il Molza in lode de’ fichi : My
Hor fa tut argumento , babbuaffo. ono AE
TONDO pii che ? O ds Giotto. Huomo tondo vuol dire huomo groffo d inge-
gno , ed ignorante , come s’ ¢ accennato fopra C. 5. flan. 1, fi che pik rondo dell’'O
: Grotto vuol dire ignoranuthmo , ¢ pil, perché lO, che fece Giotto Pitore fu
tonditfimo , fecondo che riferifce Giorgio Vatari nella vita di eflo Giotto,
BALZICARE, Praticare ; Conuerlare : Bocc. Giorn. 9. Nov. 5. £ vatrene nel
Ja cofa dela paglia , ch’ ¢ sh mighor nego che ci fia , perciocché non vi baxzica mai per
“ote. 1

 

ona.

CHTASS, Bordelli , lupanari , luoghi , ¢contrade , nellequali habitano les
meretrici , come era in Firenze il Chiaffo de’ Buoi , ¢ il luogo , dove ora € il Ghet-
to , detto anticamente Chiafo & perche in tali moghi ula di fare fracatio, e rumo-
re difonefto ; di qui forle ¢ che chia/so , ¢ bordedio fi prende ancora per tumult die
fordinato , infolente , ¢ lafcivo.. : swash

‘PIV’ cattiuo di tre afi, Affo fi dice il numero-uno-de i dadi , che ¢ i)
numero , ¢ per confeguenza nel pil é il peggiore che vi fia tirando tre dadi,)
quetto il prefente termine fignifica cattiviflimo:, che vale per aftutiffiime , ed ¢ lo
ficflo che Pil trsffo a’ un famuglio a’ Orto, che pur vuol dire fagaciffimo ,eche fail
conto fuo, Famigio a' Oro. B’ uno de’ Birri del Magiftrato degli Orto di Bali
di Firenze , che ¢ il Magiftrato Criminale; ¢ perché fi {appone che cofloro fap-
piano tutte le furberi¢e , perd fi dice: Il tale ¢ pis triffo a’ un famiglio d' Onto, per
efprimere ; ¢ huomo fagaciffimo . 1 Greci diflero Cantharo. afturior , che qui
Cantharo fu un’ ofte d’ Atene aftutifimo. 4/*m in antico Latino voleva dire,
foto , fens accompagnatura ; onde chi cantava  fenza ftrumento che L/accompa-
gnaile , fi diceva coftui : canere affa voce, Di qui pud effer venuta la voce Afoes
Kefpare in affo , ciok effer la(ciato folo , {e bene altri gli aflegnano altra origine: 0
pure da «fino che cosi chiamavano ne’ dadi /’ #nita i Greci , dicendola Ones. I
noftro Proverbio : O a/s0 , o/ei i Greci dicevano , o diciotto , orre. O sre fei, ore
afi. ‘Giulio Polluce lib, 9, al cap, di giuochi fanciullefchi , ¢ de’ trattenimenti de

1i antichi. AS
PAZZO Cittadino, Quefo epiteto fi fuol dare 4 colore che fanno sutte le tor elt
4 cAfa , ¢ fenza confiderarione ; ed € lo ftefio che dire ux cernellaccio ,

SBAND-RE, Disfare le bande , cioé licenziare i Soldati.

TKYCCHI via , Se ne vada. E’ modo baflo , cavato forfe dalla parola Ze
ruck Tedefca profferita da i Lanzi, quando con Ie loro alabarde fanno allonta-
nare il popolo ; O forfe dal giuoco del Trucco , che fi dice truccare , 0 trncciart
la palla , quando cogliendola con un’ altra palla fi manda via dal luogo', doves

era ; dal frequentativo Latino tra/fare ufato da Catullo. ' ai

TOCCO. Con il primo o largo ; Specie di berrettone , che anticamente ulava
in Firenze in yece di cappello. Varch. Stor, lib, 11. Cow le calze foppannate: a Ie
jerra bianca ,¢ le berrette , 0 vero tocchi di colore roffo . :

SOLLIFATE ? anche. Alzati i fianchi , cioe rizzatofi da (edere, —

jicias

  
 

ek=s & re:

BEERS Ee Seek Sec s-

Se

Le fF Pe coor Fae

=
=

Sar erree

 

 
 

  
 
  
     
     

ore cubito ,

Dan. Inf. canto 34,
8

 

7 la quale

zioni civil

STANZA LXXXIV.

Pluton diede con tutti una rifata ,

\fiantar fino il brachiere ,

RB difeegi: Va via beftia mcancara
Com’ entra célafsedia il dare,e havere?
Segualalero che vien dela pancata ,

Rizzato Barbariccia da federe
Sichina,ementre abbafsa gti la chioma

Alea le pe,e moftra sf bel di Roma,

STANZA LXXXV.

Poi # intirizzaye dice in rauco fuono:

- Se non fi leva dalle fquadre tl capo,
Quale ¢ Baldone,e non fi da nel buono,
Mai fi verrd di tal negorio a capo ,

| Dove y fe manca lui quanti vi fono,
Reftari come molche fenza capo,

- A poco # poco, a truppe, e alla sfilata
Partendoyn breve disfaran 0 areata ,

——— a

 

Qqz

SESTO CANTARE; 307

parte del corpo , che é fra il fianco ye la cofcia , da Ancon greco

ire gomito ;¢ fi piglia per ogni (ora di piegatura , come Jo moftra il

Citta d? Ancona cost detta dal gomito , che faquivi la fpiaggia ; Pli-

pio libs 3. caps 13. La iifders colonia Ancona appofica promontorio Cumero in ipfo
q fe if

«© Quando noi fursmo la dove 1a cofcia
Gait sis Ss volge appunto ful grofso dell’ anche .
Edi qui fciancato’é un zoppo , che habbia mancamento in tal luogo . Vedi
foto C. 11. flan. 40. B il Latino Coxendices .
ty PITOMBOLO B’ quando uno , pofando il capo in terra , volta fopr' as
quello tutta Ja vita , Vedi {otto C, 7. ft. 20.
| ORB, cui plende in mano sl gran forcone. Fingono che Nettunno Re del mares
atello di Plutone ufi in vece di {cettro una forca con, tre punte , ¢ perd dettas
in realta é una fiocina da pefcatori , Latino fu/cina, e Plutone
tun Bidente, cioé forca con due pente; Equefto & il gran forcone .
er Speziale.B uno Speziale in Fireaze,che fa per infegna un cappelio.
aiubeaowe - Enfiagione che viene ne i piedi’,.¢ nelle mani per caufa del
r Latino Pernio. Vedi 2 C, zt. 6.
- LOamal pagare in ful t: « La vuol pagar per via di Corte , con tutte le fo-
temipebe, non vuol oa Saede non feglt mandano j birti a gravarlo , o cattu-
en dice che Baldone gimecherd di calcagna , cioe fuggira per la paura_
teller prefo per debito , quando vedra Pedino, che cosi G chiamava uno gia bir-
ro della Mercanzia » che éil Magiftrato , per yia de] quale fi mandano I’ efecu-

¥.
re + Subito . Latino ¢ veffigio. Traslato dal giuoco di Tan » che fi dice
dar di pofa — fi da alla palla prima,che tocchi terra. Vedi
e Ly s

forto C. 7,ft.92.
TANZA LXXXVL
Circa il pigliarlo,# ionoal' ho , eglit fallo:
Facciam conto ch'in braco alla paftura
Vin toro fia coftui 0 xn cavalo ;
Tiriamgl addofso qualche accappiatura
Legata innanzia un bel maxzacavalla
Collocato in caftel prefso alle mura,
Ond' ei fi levi un tratto all'aria, e pai
Si tiri dentro ,e dove piace a noi,
STANZA LXXXVIL
Buono , rifpofe il Re ,non mi difpiace ;
Ma il Cancellier di fubito riprefe :
Sia detto ,o Senator ,con voffra pace,
Tant! oltre il poter noftro non 8 effe/e ,
Li tutto (aria nulo , ef foggiace
Ad efser condennato nelle pefe ,
Ed io farei flimato anc’ un Marforio ;
econfentir a Kn! atto perentoria ,
STAN-

 
  
   
 
 
 
 
   
 
 
 

we
308

ae NZA ieee
Perché fempre de ire i

y slrapacs 4 ete 5m rAgion®y
Pei Sella é in morayvienfi a ua! inibitay beg upalerele
E non giovando , alla comminazione
Ch’ in pena cafchi delle forche a vita y
E fe la parte innova lefione y

Aller puo condennarfi , havende ofate
Di far caufa pendente un’ attentato >
Plutone , ridendo con gli altri della coveaaaeens

fecondo, che viene nea pancata,nominata Barbariccia , cheidica i)
e quefto propone che fi tiri un laccio a Baldone} € per vid d'un
s'alzi , ¢ G porti dove pil piacera ; ma cié:non ea C
de Piutone ordina al ter2o nominato Calcabrinal 3 dica il fuo parere
fiui fi rizza,, ¢ fa riverenza al Re per far il difcorfo , meer

 
   
   
   
      

ti Ottave.
‘ SCHLANT ARE , Denes » {pezzare detto da. Splenare, BB
lo , che fidifle fopra C, 3, ft. 5. A

BESTIA incancara, Cost diciamo per e(primere n*huomo feo 9
traslato da quelle beflie , che alle volte conducono. con Joro.i Monts
quali effi fanno far molti giuochi , ¢ dicono che tali: beftie-fieno:
operino per vie diaboliche . Si dice be/tia smcantataa und di poca confi
ed avvedimento , come il Lalli En, Trau. C. 20. 56.
Cosi gridammo, e con.la propria appa
Ci deffimo in ful pie beftie incantare
COAL entra con P afsedio, Significa come s’ accorda , 0 che i che re
I afledio,
IL bel di Roma, Cosi diciamo per intender apertamente c... 5
Roma intende i] Colofseo , da noi corrottamente detto Culifeo,
SINTIRIZZA, Si vizza,fi diftende in fu la — ‘EB’ un’ atto,¢l :
ta una certa fuperbia , e prefunzione di fe fteflo y ed & quella prefopopea’ py ches:
dicemmo fopra C, 1, ft. 72. 5 a eae
NON fi verra a capo dé tal negorio , ec. Non fi conchiudera», 0: terminéra if nt
gozio. ne woagiher
REST ATI come mofche fenxa capo , Ciot fenza oe direzione '0g
Senza fapere che cofa havere a fare. , 0 rifoluere : i infect fo
capo , $’ aggirano inutilmente , ftrafcicando il sane di bey
dove.
ALLA sfilata , Senza ordine ; confulamente , ¢ fenza andare in ila,
nanza: Sbandati ' so29
S' 10 non V ha, ezli éfalle. Yo fon ficuro di pigliarlo . Seionom lo-p
per errore, E’ {pecie di giuramento vantatorio , come: appease
forto C..8. fan. 72. & mio danne che vedremo C, 10, ftan. 49.
ACC APPLATVRA, Vna fune accomodata,, € —- cay
do, che — y ibqual nodo firdice-cappio /corfoio.,

   
 
 
   
   
   
  
   
  
   
  
   

   

page e st FER w Gees = FL TFAF EE

 

=>

 
 

  
   
    
   
 

- SESTO'CANTARE: 309

MALZAC AVALLO’, B un corrente, o pertica grofla congegnata per tra-
——-yerfo, come: acavallo ‘un legno ritto ; la quale's’ alza-da-una parte
‘con tirare a la parte | » E quefto ordingo ¢ ufato affai ne i piani di

Firenze per cavar I" dai i . [ Latini lo differo rolenonem a toliendo ,
dete Smiles quella awbiid, della quale fi feruivano i noftri antichi as
Acagliar pictre:chiamata Azangano .. Livio dice: ariere Tollenonibus Inbramenta
i iy aur fa » fp vobuffos incuriebant , fta hina milirare
fien defcritta da Vegezio cosi; Tollenc dicitur , quoties una trabs in terram praatee
a ry cui in fummo verrice alia tran{uer{4 trabs longior , dimenfa medietate , 6on~
neititur , €o —— 5 at fi unum capue-deprefjeris , alina erigatur , L’ antico Vol-
“garizzamento lralenoé detto , quando una trave alta fi ficca in terra , alia quale nel
J una altra trave pik lunga per lo traverfo , enel meyxo mifurata , fi com-
«mete in tal modo che fe! wno capo fi china , l altro in aito fi leva . Da quefta voce
-alvaleno ( Lat. toileno)fi dice ? e4italena giuoco , che i ragazzi fanno con due travi
* incrociate, ¢ bilicate I’ una fopr’ all’ altra a foggia di Mazzacavallo. Vedi fopra
Ga, flan; 48, Mattio Franzefi contro alle sberrettate dice .
6 Biggetnslo- Ma chi trovalfe il modo a bilicalle ,
ee
os

 

Ma» 2 Sarebbe un [chifanoia , e faria bene
vail, Van contrappefo d! un maxzacavallo ,
SIA detto con voftra pace . Perdonatemi; s'io v’ offendo in dirlo, Non vi adi-
wvioffendete, io lo dico . Frafe de’ Latini Pace rua hoc dicam , Nell’
igen di Quinto Catulo , Pace mibi liceat , Caleftes , dicere veffra . Adortalis
fue pulerior fe Deo, Che Annibal Caro nel primo Sonctto delle fue Rime vol-
CO olfimi j ¢ *icontra a lei mi parue ofcuro , Santi Nums del Ciel , con vofira paces
—— LO vieme’, che dianzi era si bello.
(2 —--—« BSSER condennari nelle fpefe'. Ciot buttar via {a fatica , e il denaro , oleum , &
«Opera perdere. Ma propriamente ¢ffer condannato nelle /pefe vuol dire , quando
=“ UNO!Per aver litigaco wna cofa ingiulta , ¢ dal giudice condannato a rifar cute
le fpefe all avveriario ; ¢ perd quelto Cancellicre dice,che noa vuole acconfenti-
8a tale’atto-per'efiere ingiufto 5 ¢ da efer condannato nelle fpefe .
Ss imato'un Adarforio, Sarei ftimato un’ huomo fenza fentimento ,o giu-
dizio; come @ la‘ftarva di Marforio in Roma,
‘ ATTO fruffratorio  Awo vano , fatto fenza propofito , E quefto termine ,
come tutti gli altri-delle (eguenti ftanze 88. ¢ 89. fon termini Curiali jche veaca-
do dal latino’, ed eflendo praticati in cucti li Tribunali d’ Italia non-dubito , che

 

,
,  farannointefi da'ognuno; perd ne tralalcio la {piegazione .
, ~ STANZA LXXXX. STANZA LXXXX1L
E poi ba fatte riverenze in chiocca Aa in vece di quel cappio da beltrefca,
C0 fuivi pit Lindi-a pianta di pattona, Ch’é il:toffice de ladri , fi prouuegga
Si foffia it nafo ye [parzafi la bocca y Pua bilancia , 0 rete per La pefca ;
- Epoftain equilibrio la perfona Con una lnnga fune , che la regea 5
© Come quel che fi penfa dar’ in.brocca E perch’ sl fatto meglio ci riefca
Tutto sfromato dice: Alta Corona, Si ringa tutta, accio che non fi veega 5
Circa Pordingo , pur fi merra in opra; Einverra quanto ell’ apre, ivi fifpanda,
© Perch*ioconcorrose affermo quatofopra, Fino-ch’ +l porco vengane alla ghianda,
Bate tit : STAN-

 

 

 
OO EE Ls

 

 
 
 
 
 
 
 
  

310

STANZALXXXXL...,
Perche 8 ¢ muovyon |' armi ,di ragione
(Se dal capo l efercita ¢ condotto )
Annan a tutti marcerd Baldone ,
E quand’ ¢i giunga,ed ba la rete fotto,
Fate che lefie allor fien pitt perfone
A farla tirar fu con P avannotto ,
Operando in maniera,ch’ egli infacchi
tn lnogo , ove fi vede il fole a feacchi , Lodando il fa
S.T.AN, ZA \ LXSXXIVi 5 copies
‘Qui , dice il Re , fi da fempreinbudelia , Gli ba fempre pik ritorte cl
‘Siche mi cafcan le braccia, ef ovaiay Mace’ non locredes'einonvaals
Mentre cofini a ogni cofa appeila , ; i
E co’ /uoi punti mena il can per l' aia;
li terzo Diavolo , che ¢ Calcabrina , dopo haver fatta rive
mano di smorfie, come fanno certi Oratori affettati , dice, che app
cavallo , ma che in vece del cappio {corfoio piglierebbe una rete da
i] Cancelliere s’ oppone ; onde Plurone {gridando il medefimo Canc
al quarto Diavoilo , che ¢ Cappelluccio , che dica il fuo parere . at
IN chiocca , In quantita grande , in abbondanza , un diluvio di rive!
PATTONA. Specie di pane fatto di farina di caftagne, che per ¢
pit di figura lunga , s’ aflomiglia a un piede mal fatto di un’ huomo,_
da , Prolusione Plautina prima dice : Qui enim pedibus fant planis ploti:
che piede di parcona fi pud dir plotus dalla voce Latina Plautus, che fig
fo ; ¢ quefta dal Greco Plarys lato , largo ; donde noi a tali huomini ,_
i piedi malfatti-diciamo Pileri . Vedi fopra C. 4. ft. 17. li Franzefe dice P:
Spagnuolo Pata i! {uolo del pit di bue , gatto , oca , ¢ fimili ; dail Gr, Parei
vuol dire battere col pié ; calpeftare ; calcare; EB Patdn fimilmente in
2 il contadino , che porta le {carpe grandi, ¢ grofle, ¢ rozzameate fa
trebbe anche effer detta Partona, in un certo modo quafi Pafona , cic
pata grea s perché¢ quella a fimilitudine d’ un pines groffolano,e
‘Pattume dilie Ser Brunetto nel Pataffio quello , che oggi dichiamo 2.
spaccatura ye mefcnglio di cofe fracide ; ¢ ClO pure cred’ io, dal Greco
peltare . Ed sl pattume vien rammuricando, Il che ha qualche fimili 0
Patrons , cola fordida, ¢ vile, ¢ di brutto colore, { Greci ( per dire anche q
lo fterco , perché fi {carica i) ventre lungi dalla ftrada comunale, che dal?
firada batcuta fi dice Pates ; dificro dpoparema , il che pud aver dato origine al
arole Pattume ,¢ Pastona,, Gli dice findi , ma per ironia , che in, vece d’
picde ben fatto , & attillato , vuol dir piede (concio ,¢ mal fatto . Lindo
ja venuta a noi modernamente di Spagna ; ¢ & come /enda in quella lingua Vi
da) Latino /emita, ¢ linde da) Latino mite ; cosi indo credo che fia d i
mito, cio’ limitato , aggiuftato , ben afletto, compofto. Da Lindo diciamo
che Allindarfi ,e Allindirfi Sp. alindarfe, ‘3 eng ela
$1 /offia it nafo, e fpaxafi la bocca. Eipurga il nao, ¢ fpura , ¢ con Ia lin
netta identi , che fono.quei lezz) , che fanno moiti Ocarori , come porre in

 

     
 
   
   
   
  
   
    
 
   
     

  

    
    
    
 
    
  
   
  
   
  
  
   
   
     
    
  
     
    
     
  

   

SESTO CANTARE: gu

brie ta perfona ; cio’ dopo haver dimenato in qua , ¢ in 1a il corpo,fermarfi in po-
_fitura intir }, come ha detto nell’ Octava antecedente , che fono tutte smor-

fie , che denotano nell’ Oratore una {ciocca fuperbia , ¢ prefunzione di fe fteflo ;

ed il Poeta lo tocca col verfo che fegue, dicendo: Come quello che fe penfa dare in,
b che vuol dire, @ima di haver trovata I’ inucnzione buona , ¢ d’ haver im-

; cioé dato nel fegno .

O sfrontaro. Arditamente , sfacciatamente . I) Franzefe fimilmente ¢f-

ERT ESCA , 0 Bertrefca , o belrrefca ; E’ una {pecie di cateratta , ches’ alza ,
abbaffa , ¢ ferue per riparo di guerra in fu le torri, ein fu le mura fra uns
, ¢P altro ; © cosi fi dice ogni luogo , fopr’ al quale fi falga con pericolo
ecipizio . Di qui viene il verbo berre/care , 0 bertrefcare ulato da molti per
ndere Armeggiare , 0 affaticarfi intorno a un lavoro , ¢ non trovar la via as
hes i per berte/es intende Ja forea ; per fimilitudine delle berte/che , le quali
i di legname, che fi ponevano in alto . Gio, Villani lib. 9. 114. Pers
@ il porto era tutto impalizzato , ¢ incatenato e@ di fopra di eroffo legname imber-
+ Quefte bertefche , 0 torri di legname alzate fu le mura dovcano (eruire
cofe a gettar pictre , onde forfe € la parola pertrechor , che fignifica,
pre i Spagnuoli munizioni , ¢ ripari da guerra , cioé le noftre berre/che , det~
ta forle cost da echar las pedras .
- BILANCTA . Specie di rete da pefcare , detta cosi per effer a foggia di bilan-
sia; firumento , col quale fi pefa la roba.
a ella apre . Cioé quanv’ ella allarga per ogni verfo.
"FINO a ch’ il porce vengane alla ghianda , Fino a che venga a dare nella trappo-
1a; ficali al zinkello . Esintende fino a che Baldone andando alla volta di Mal-
antile dia nella rete fuddetca .
\ SIENO Iefle . Sc bene leflo vuol dir Agile . Vedi fopra C. 1, ft, 11. Tuttavias
far leffo vuol dire ftar pronto , all! ordine , 0 preparato .
~ AFANNOTTO . Pelce piccoliffimo. Voce corrotta da Vguannotto , 0 Vw
annolto 5 che fignifica pefce nato quell’ anno : perch¢ g#ann0,0 wnguanno vuol
ir quel anno , fe bene ufato folo nel contado ,¢ |!’ Autore fe ne feruc in bocca,
@un contadino forto C. 10. ft. 35:1 Latini dicevano Hornus , ed hornotinus unas
colad'tn? anno. Il Poeta da nome d’ avannorto a Baldone , percht dovea effer
prefo con Ja bilancia , che € la rete , con la quale fi pigliano gli avannotti .
IN lnogo, ove fi vezga il Sole a fcacchi , Cie in prigione ; perché le fineftre fer-

a

—

we

=
=

a

ye ‘tate della prigione , battendovi i raggi del Sole, fanno a figura dello {cacchiere,
g! nel luogo dove termina il loro sbattimento , o ombra dei ferri. Da quefte fine-
oo r te , Ograre di ferro delle prigioni , fi formo 1] verbo e4egratighare ufaco
if dal Boce. Nov. 85. Tw m’ hai aggratighato il cuore colla rua ribeba’, Clo€ imprigionas

       

to col {uono della tua rideca , come oggi diremmo : ¢ da Brunetto nel Patafiio .
TVTT At fava. Tutta é una ftefia cola . Sol eff Apollo, ipfe pollo Sol, Di-
Geil Cornazzano Nov. 11. che fu una Signora , Ja quale yolendo riprender co+
Potomee il mario , perche la(ciando lei andava dalle Meretrici , gli fece uns
utidimo definare , ogni vivanda cra condita , ¢ ripiena di fave con diverfi ftra-
‘Vaganti ma delicati fapori. 1) marito le domandava; Che cofa ¢ quelta? ed el.
we la

ee!

 

io.
 

 
    
   

gin MALMAN TILE?
la rifpondeva ; Fava, E queft’ altra? Fava. In fomma
gaor marito {ceglieve quanto volete , perché sattae fava; : egli.
guta, e faceta riprenfione della lie , mut vita, conofcer 3
na ail’ altra non pud effer’ altra differenza , che quella che nafce da un for
sfrenato appetito. E di qui poi venne il dettato 7 ase ¢ fava che fignifica &|
anne daxke Meee a eee so of OO OM
/L Cipolla, Autore noto , che ha {critto.in Criminale.
a Plutone , che fe bene quivi, e/cln/a ogni ragione Civile s* attende
Tuttavia gli Autori criminali non approvano quell’ operazione
rimette dicendo ; Se tu lo comandi,io non ho che replicare , € conc
anche tu Jo voletli far’ impiccare , ¢ {quartare ; che quefto iatende / i
lo {quarto . Tole 4
7 ad in budella, Non fi conchiude cofa di buono , Quefto proverbio.
copertamente: Far come il cane de/ peducciaio , ¢ s' intende dare.in budella. | {
e(prime difcorrer’ aflai , ¢ conchiuder poco , ed ¢ Jo fteflo che dar.in cenci
MI cafcano le braccia ye? ovaia , Mi perdo.d’ animo affatto.. Si dice.
cuore , le braccia , le brache , il fegato , il fiato, eda moltis’ ovaia peri
pertamente é tefficoli, ¢ tutti hanno lo fteflo figniticato, dl perderfi d’ animo.
qui accoppiandone due , ciot /e braccia , ¢ /’ ovaia , efprime perderfi affatod! a
nimo . Latino ovaria, che fi (ono {Coperte ultimamente nelle donne, dagli
erano creduti , ¢ detti 1 loro telticoli. % fi
AOGNI cofa appella. Non ’é cofa che Nia a fuo modo ,, da: difficulta a ogni
cofa ,a ogni cofa ha che dire ; ¢ non fe ne fla, ¢ non fen’ acquieta,
appeharfi termine legale . Toa
CO! fuoi punti mena il can per aia, Co’ fuoi punti legali , e con le difficalta 5
che oppone , manda in lungo Je cole fenza venire a conciufione aleuna. e4its
vien dal latino area , ¢ vuol dir quel pezzo di terra {pianata , ed accomodata per
batterui , ¢ mandarui fopra il grano , ¢ biade , ds
ALA piit ritorte , che faftella. Ha pid ripieghi , ¢ compenfi, che non a
cidenti, che faccedono , Ovvero egli trova fubito riparo a ogni accula «
fi dicono-quei legami fatti di vinciglie d’alberi , coni quali fi legano i falci di
legne , € di fieno , o d’ altro , detti ritorte , perché quella vinciglia fi attorce pet
renderla maneggiabile , ¢ fleffibile a fine d’ adattarla a legare. Dan. Inf, ©. 19:
Che {pexzate bavertan ritorte ,¢ frrambe , et
El non lo crede , Quefto termine fignifica Tu non ti vuoi emendare ; ¢ fi dices
Won crede al Santo, /e non fa miracoli ; cioe non crede d’ haver a efler gaftigatoyin
che ei non prova il gaftigo . Qui dice fe ei non va a degnaia , cio fe egli non & Ie
gnato, ¢ baftonato: Legnaia ¢ un borghetto vicino a Firenze, ed il nome di
gnaia ci {crue per efprimere legnate , o baftonate. Vedi forto C, 11, ft, 116 gr
tar /a tigna., Dove fimettono diverfi modi di dire per intendere Baftonar wn
CAPPVCCIO , Il Varchi Stor, Fiorentina lib. 9. dice; 11 Cappucgio ha te
>» parti: il Mazzocchio, che ¢ un cerchio di borra coperto di pango, che
»» facia d’ attorno alla tefla , € di fopra , foppannato dentro di rovelcio
x» tito i) capo. La 1 opsia € quella, che pendendo in fu Ie fpalle , difende
»> guancia finiftra . Li Becchetto ¢ una ftrilcia doppia del medefimo panes

     
  
      
  
 
  

   

  
 
  

oa
a
rm

BSB eee Fa.

Pe ee ed

=

 

 
  
 
 

BRE eBas FS:

 

 
   

 
    

SESTO CANTARE Re {
ra fi, in fir la fpalla , ¢ bene fpafio s’ avvolge al collo , e» %
eller pil deftri , ¢ pid, intorno alla tela, ec. EB
_che gia portavano le perfone civili 5.¢ del quale parla il
ft. 7. alla voce Adaxxocchio., ¥
STANZA LAXAXV.
che direi ,0 Sire, Perch ei! ha, detto.con si texfo dire ,
te ch? io dica mi vien detto, Ghiioffoper dir che mais’ uds tal detto;
np non ofa, ch’ io non ho che dire, Pero dico.ch' a dir non mi dd il cuore 5
ir quate qui quel? altro ha derto; Elafcio dire a un' altro dicitore ,
ecio, , che é il quarto diavolo , fatee fue cirimonic , fa un dilcorfo fen-
¢, come fi yede nella prefente Octava tutta di {cherzo fopra il yer-
le non richiede {picgazione , ma folo rifleifione al graziofo, ed in.

STANZA XCVIIL
Valeati , dice il Re , {propofitato ;
S? alcuna cofa qui non bas propofta
Come vuoi tu buaccio che'l Senato
Yada in Cancelleria per a rifpofta?
Par fento,rifpond' ¢i ych' in Mdagiftrato
Cosi dir s* hi ed io l'ho detto appofta;
Mas iovifcadolerxo,e alcun m'incolpa
hiandellino. Dica Baciapile, Drerrore in quefto,iomeneredo in colpa,
ANZA XCVIL, STANZA XCIX,
Non occorre brunir co i labbriifaffi ,
Dice Plutone , ofsaccia fenza polpe ,
E fare il torcicollo , e ovunque pafi
Semmar difcipline ,e dir tue colpe ,
Ch to foyche chi per lepre tt compraffi,
Haurebbe almen tre quarti dell.
ua in mexco,bacia terra,ein fine Pera va a fieds ,¢ fegua il Tiritera ;
7 Auago piovon difcipline.. E queis’ affeteaye parla intal maniera +
rende Cappelluccio , ed in tanto il quinto Diavolo , che ¢ Libicoc-
re sbocear’ Arao in Malmaatile , qual configlio ¢ riprovato co.
ile; Oade Plutone ordina al fefta Diavoio ,che ¢ Baciapile,il propor.
-¢quefti dice, che vadano in Cancelleria per la ri/potta, che € lo fteflo ches
Vi
VE

 
  
     
 
 
   
 
 
 
 
 
 
 
   
 
 
 
    
  
 
 
     

 
 
  

 
    
 
    
   
  
   
 
  

Bao SO pero Plutone lo fgrida , ed ordina al Tirirera che ¢ il {ettino

10 dica , ed eglis’ accinge a parlare.

INE. Quel che fignifichi diceamo fopra C, 3, ft.27. E il Latino fearra,

shine. Vn poco poco. E qui , clicndo deteo ironico fignitica; ¢ un,

(pazio da Arno a Maimantile. .

'ASEO , Balordo , melcato , ftupido , bafofo, A quefta voce allude la Pran-

Smarrite, confnfo, quafi sbafito . B far il bafeo vuol dir finger di nou in-

3 erfi huomo fenza giudizio , dal verbo ba/ire vilto fopra C, 2, faa.
Reflo che far /4 carta di mafino , 0 la gatta morta , vio fopra C. 1, ft. 19.

? Ipocrifia, E’ ua SH ipocrito. La voce Ipocrito yi dal

© reco

>
?
+

 
 

 

THe

314 MALMANTILE §

Greco Hypocrinephai , che faona contraffare ; ¢1' Ipocrifia fi difinifee Vina calli.
da , ed afluta palliazione del vizio occulta ; perch¢ Ipocrito fi chiama colui , che
eflendo uno feellerato , nondimeno nell’ abito; negli atti,e :

d’ eficr buono , es’ affatica di parere quel che egli noné,¢ ep rmer iamente J rin
ta fignifica commediante , iitiens ~S. ‘Apofbad nel Sermone da ‘enerdi dopo lan
s» Domenica della Quinquagefima . Hypocrita Greco fermone fiailator ie

>» pretatur, qui, dum intus malus fit, bonum fe palam oftendit.
»» faifum , crifn vero mdicium (onat. Nomen autem hypocrite translacum eft a
3» {pecie eorum , qui eae tecta facie inceduat, diftinguentes vuleum ccerulco,
x) hivcogue colore , & cozteris pigmentis , habentes fimulacra oris lintea gypfata,
yy» & vario colore diftinéta , nonnumquaim colja , & manus creta’ Z
yy utad perfonx colorem peruenirent , & populum , dum in ludis agerent , falle.
3» reat, modo in fpecie viri , modo in forma feminz , & reliquis preeftigijs. I
sy Berni nell Orlando contra gl’ Ipocriti, Won han'da fare lémafchere a ——
i, Quelti (ciagurati fono di tre forte. La prima é di coloro , che fingono |
cofpetto degli huomini d’ effer pieni di religione 5 ed internameare fono ateifli,
La feconda é di coloro , che fanno del bene non moffi dalla virtu , o dall’ amore
del bene y ma per efter creduti buoni. La terza é di coloro, che dimoftrano di
non effer buoni , perché altri credano , che eglino fien buoni da vero, enon,
ipocriti, In quefto Diavolo fi {corgono tutte tre quefte {pecie d’ Ipocriti 5 ches
appreffo di noi fono lo fteflo , che Bacchettoni ; detto fopra C, 2, ftan, 1. Dante
neil’ Inf. C, 23. parlando di loro dice: ih
Laggils trovammo una gente dipinta ;
Che gira attoruo affai con lenti palpi ,
Piangendo , e nel fembiante franca 5 e vinta ; Lai
E gui dite ; i/o /morte , cioe faccia pallida , ¢ {colorita ; e'dice*che pioveno
/cigline per intender uno di tali Bacchettoni falfi 0 diciamo Ipotrito. B foto
neil’ ottava 99. feguente dice , Seminar difcipline , che ha lo fteflo fenfo. Bs’ ula
affai il feruirli di quefti due termini per efprimere : B? paflato per quefta ftradas
un bacchettone. Veramente quefti tali infami non ia(siane di valerGi di tute le
forte d’ apparenze , ed io ne conofeo uno della prima fpecie d’ Ipocriti , che tro-
vandofi in una pubblica adunanza , ia cavarG ii fazzoletto di talea lafcid cadere
una difciplina a vifta d’ ogauno ; ed effendogli detto, che avvertifi', che gli era
cafcato non fo che dalla tafca , egli raceogliendola ‘diffe: Non @ mia roba;’
fon cosi buono s che io adopri tali arnefi. Di/ciplina chiamiamo quella sferza- 5
che le perfone veramente buone adoprano a batterfi per far penitenza ; cost
dal!’ ammunire , ovvero gaftigare-il corpo 5 per rénderlo feruo ubbidieate al fu0
Signore 5 e ben difciplinato ; cioé inftrutto del fuo dovere, che & la fummilfione
alia ragioné. L! ufo frequente della difciplina comincid in Tolcana y'¢ fi diffule
per tutta Iralia,e fi ereflero Compagnie de’ Difciplinanti,o Batcati 1 aring’ 1460,
Sigonius de Regno tralia. i bx ats i
SPROPOSIT ATO, Vino che non fa , ne dice éofa a ptopofite. =.
BV ACCIO. Ignorantaccio . Che fi dice anche edfiraccio y Ci 4
dual , bue di panno , Vedi fopra C, 3. ftan. 49. la voce arfafarco 1 Lz :
hayevano diverfe voci,che efprimevano queito fteflojcome fi vede in rid

  

 

 

  
 

BEG ec kf eke? oe tS eS eee

arr =

u
tty

 

Bee

 
 
 

SESTO CANTARE: ; 2s

. Sc, 1, dove dice.Qui ubique unt, quifuere , quique futu-
ed Gesdepucblici-) fod , fang: ; hard’, Bdeani Buccones , Solus
‘ante eo flultitia ; & moribus indoétis, & Terent. in Heaut. 5.
haram rerum conuenit que {unt diéa in flultum:, caudex ,

plumbeus .
4 plea. B’ quello , che i Latini dicono wltro , confultd , ovvero dedita
ioe non per errore , Oo inconfideratamente .
angolezzo . Il verbo {candolezzo portato dal Greco al Latino, e dal Lati-
 noanoi, ha fignificato d’ inciampare , ¢ d’ adirarfi come vedemmo fopra C. 1.
- flan. 56. ¢ fe gli da anche il fignificato di quelle parole Si oculus tuus fandalizat te
te, come é nel prefeate Iuogo » che prefo in fignificato attivo vuol dire : Se io vi
dé occafione di far errore: {e io vi fono cagione d’ inciampo ; ff ribi offenfioni /um ;
2 | afero y per efempio , fo credeva , che il tale fulfe huome da bene, mail fen-
pai, che ecli da a ufura , mba feandolezrato, cio fatto mutare il concetto , che

 
  
   
    
   
 
  
    
 

   
 

 BRYNIR e@ labbri i fai. Brunire , parlandofi di materiali fodi come ferro ,
ilo , oro 5 ec, yuol dire Dar il luftro , ¢ perd intende qui dar il luftro ai faffi co
labbri , baciandoli fpefio , atto , che fi fa da i Criftiani devoti per {egno d’u-

y fopra C. 2. fan. 9. difle dar il luftro a’ marmi co i ginocchi .
ACCLA Jenza polpe. Carne cattiva ame quando fi compra Ia carne,che
fia con molto offo , fi dice: Vi ¢ poco del buono ; ¢ da quefto dicendofi a un
-huomo o/sa fenza carne s’ intende trifto , ribaldo , o (cellerato .

CHI ti comprafse per lepre , haurebbe almeno tre quarti di volpe , Chi ti credeffes
__femplice , troverebbe poi in te tre quarti almeno di maliziofo , 0 furbo . In La-

tino fidirebbe: Pro fimplci columba , afluta vulpes. In tutta quefta Ottava narra
«| -Moltedi quelle azioni be fanno gl’ Ipocriti , e Bacchettoni falfi .

jp. aS STANZA

i La che fono un! infano, eignaro ogni hora , Finché lo {pirto {porti al foro fora ,

Bb erche faper fupir non voglio , 0 vaglio, Dond’ ti fa i peti,e puted oglio,e d'aclio,
oy ‘al Duca,percht a’ muri ei mora Accio ' accia fu? afpo doppo atdoppi
it Tofoin tefha fi dia Jie meglioun maglio, La Parca ,eil porco con la fhoppa Poppi :
@ _ Wiiritera , che ¢ il fettimo Diavolo propone che fi dia in ful capo a Baldone ,
i €s'ammazzi. 1 Poeta lo fa parlare in bifticcio a imitazione del Pulci nel fuo
f Morgante lib. 23. che dice . ° 3

La cafa cofa parea bretta ,¢ brutta
ah é V inca dal vento /a natta , e la notte y

eF

goo Stilla di flelle , ¢? a tecto era tutta,

BB. Mapnifrnte E fuina , ¢ fuena di botto una borte .
BH Pere havea pure ,¢ qualche frasta frutta ,
io Del pane a pena ne deste a tai dotte

 

Pofeia che pefci , ¢ lafche prefe all’ efca,

 

Y Lt Metta allorra alla frafca fu frefea.
MAGLIO, Dal Lat. malleus . Martello grande di legno per ufo di battere i

 

.

ie

  

- Setchi alle botti , o per ammazzare i buoi, o per altri layor: di legname , nei
{quali richicggano

percufioni gagliarde , ¢ gravi.
Rr z

SPOR-

 

   
   
  
 
 

316
SPORT ARE . Avanzare in fuora , come av.
dclle muraglie delle cafe ; donde Sporti: g tec
ri de! muro maeftro , € rette da’ i , forgozzoni , oc
eAfeniana , che i) Filandro fopra Vitruvio defiatice prorette proiet
tte a Menio, Oc.) € qui vuol dire = feapps , 0 efca fuori lo io EBT
PETO . Quel romore che tail vento {tappando all huo
fo. Lat, peditus . ree mons
| ASLO. E' un baftoncello con due traverfe in croce ‘contra ;
|

 
 
 

ae

 
 
  
      
    

alquanto I’ wna dail’ altra , topra vi qualei raguna il filo/per ridurlo in’
ane dal’ ane pare, nape pelyeapeuaaaeal Gnindolo onde Agenina
PAKCHE, Le tee donne appellate Clra , wAcropo ye Licheft, e dere
quia nemsni parcunt , five quod parce ,@ pene avare vicnm eribuant . La Gi
ilimava , che quelte futicro Figliuole dell’ Erebo., ¢ deta Notte., fe
iVatura Deor, ¢ fecondo altri,che faflero Fighie di Demogorgone; €
figuratiero le tre cole necefiarie all’ hnomo , cioé il nalcere , ii vivere
re ; dicendo che una di loro detta Cioto fila, cheé il nafeere , ta fe ) detta
trope annafpa , che é il vivere , la terza detta Lachefi taghia il t ce il mo
re. Le chiamarono anche Nona , Decima , ¢ Morte. 1. a Ee i
STANZA Cl. STANZA CHL —
Ben tu puxxs ai pazro ch’ ¢ um pexro y Lonon fo fe Baton fornia , 0} iy
Diffe Pluton , beftiaccia, per bifticcio , Perché #¢i vuol,
Perch’ io per me non fo, ne raccaperro Famate i conti 5e conta,
Quel che tu voglia dir neltuocapriceio, Wel zerolho frat wnaye:
Ata non fon Re, s'io non'te ne divezRo,
E perché tu non tent grattaticcio ,
Mentre flima non fai delie bravate ,
Queft altra volta le (aran pectiate . Trémande andranne come
STANZA Cll. STANZA CIV.
via feguite: Sui lo Scamonea Ola , dove fiam nos ( dice P
Si rizza, in vrfo tutto infauguinato , Eche Yi bite chtio
Perch’ ei,ch'eun faftidiofo,appio havea Daro benvio fat a 4
Fatco a graffi con un,che "ha en allato, Si calle fiele iv cnderd it bade

Pero con la bifunta fua geornea, Guarda quel'thetu di barone
La qual traluce come Ciel frellato , E va piit'tefpo ye col ‘aki
Sich'ella un' Argo par fatto alla macchia, Sta ne i vermini-, e parla con gindiciy
Sinettayal Res' inchina,ecostgracchia; Che per min fe ti privo dell’ nfixis.
Plutone dopo haver riprefo il Tiritera , comanda,che ‘dica Scamonea ottav
Diavolo ;'il quale da anch’-egli un configlio {propofitato ,¢ con parole eel
onde Piutone lo fgrida ,minacciandolo di levarglivia degnira Senatoria , ©
non s' avvezza a parlare con termini oneiti ,’¢ rilpettofi . Poe «
BISTICCIO , E: \a figura'che i Greci dicono Parecheff yed & quando fi
due parole che hanno lo fteflo’, 0 poco differente fuono-, diverfo
come fi vede neil’ antecedente ottava 100. ene i due primi veri
101. Detto Bificcio quali Dificcio dal Latino greco Difti¢hnm ; rela
che Biforto & fatto dal Lar, diPortus; Biffemto, dal Lat,

     
  
   
   
 
  
 
 

   

 
  
 

    

£0

Oat FS OSes ten renzEesa _

  
    
      
 

    
   
   

   

   

Be RE fee as

 

 

 
 

  

 SESTIO CANTARE;? 317
»Ci0e maltrattare ,efimili ,. Imperciocché i primi bifticci , de’ quali ci
gli Efempi'yconfiftevano in Diftici , o eel dire coppie di verfi

lia ftefla vocesla quale fignificava duc cole diverfe,fecondo che o piuilar-

b ftretta , o intera , o'dimezzata fiyprofferiva. Fra Guittone d’ Arezzo,

' Poeti‘Antichi di. Mons, Allacci, tutta una Canzone va tefiendo

i diparole ed’ quella che fi trovaa carte 385..nelia Licenza ,

qual Canzone dice cosis

   

    
 
   

 
  
 
  

   
 

ny yedo,
Sen’ ake mido,

aera Edi, che prefofo,

iihies tex i912 cio vuol di tornar fo,
la in'primo Juogo vale ad banc ipfam bor am,ficcome adeffo vale ad boc ipfurs
| fecondo:luogo %d ¢/savuol dire ad ¢/sa mia donnaya les, 1 fait eda,
coll {econdoymeido,L, me dedo, || primo fo vuol dir/ono verbo. I fecon-
-Ne‘fonoefempi in Bindo Bonichi , ed in Francefco da Bzrberino .
raccapegzo .. Non fo'ridurre’a capo: Non rinucrgo : Non rinucngo :

vo: Non intendo . \

C70. Qui vuol dire opinione , o penfiero. Vedi fopra C. 1, ft, 21.
‘fon Re, Laicio d’etier Re , E’ termine giuratorio che efprime Tanto
“€vero che iovho fatta , © fard la tal cofa., quanto é vero cheio fono quale io fo-
#0 Non (60 Padre ui Telemaco , cioe non fono. Viifie feio non ti feufto ; Dit

f 4 Terfite prefio a‘Omero .

ui te ne dwexzo, \S' io non ti fo lafciar quefto vizio ,-0 quefto tuo modo
-diteattate . E> il contrario d’ avvezzare . Vengono da Vizio sens avvitiare pec
eallietare a'un vizio difuxiare per liberare da un vizio. E quelti due verbi
attivi , che neutri hanoo fempre lo iteilo fignificato. Diciamo per efempio
‘Phateeces I del tabasco’, cickie/serfi afiwefarro a pigtiarne .
tem gratcaritcio.,. Twaon fai Rina de i piccoligaftighi; Tu non temi
; enon cri le riprenfioni.. Nelle Raccolce de’ Greci trovafiun certo
ico’, che voltato.in ‘Latino fuona cosi :
sls (6a Tncus maxima nom timer Strepitus .
Egrattaticcio intendiamo grattatura , che leggicrmente offende la cute .
PECCIATE , Petcofie nella peccia , Caici nei ventre. Termine baffo, e pit
toffo feherzofo . Peccia loiflefio , che pancia , {e\bene della parte, che é dallo ito.

 
 

 
  
 
  
  
  
 
 
    
  
  

 

      
   
   
   
   
  
  
  

Maco al none Peccia pare pit verfo lo ftomaco , Pancia pid verfo il petti-
“Btione , Quelta'é dal Latino pancices ; inictini ; quella forfe dallo Spagnuolo pe-
Latino pettus, onde Rimpecciare .

| BISUNT A giornea .. Velte atiai wnta.. ‘BE per giornea’s’ intende la fopravveRe
_ Mdei foldati’, che i Latini dicono Ch/amydem ye Lispigiia per vefte d’ aurorita ,
_ donde habbiamo un proverbio che dice .
AP PIBBIARS! Ja giornea,, Ohe Gignifica prefumerfi molto di fe medefimo . 11
‘Bn, . 102, parlando di Didone dice :

Come

 

 
 
  

 
  

  
   
    
  
    
 
   
 
 
   
  

pugs. -IMALMANTILED®

Came Diana allor che xfcirne acacia ys
Lungo ? Exrota , 0 pure in Cinto (nales damipcabs
Fratutte U' altre la giornea s allaccia

E fuol parer fra le fue Ninfe un fole
Il Forti , parlando della Prammatica delle donne al caps mibi 242.
parole da i libri pubblici di quefta Citta,dice : Wen porevane portare
© mantello 0 altro veftito {parato , ne maniche fparateso tagliate per il lunga
cia . Donde fi deduce , che quefta cra yna fopravvelte , oizimarra aperta t
dinanzi , ufata anche dagli huomini,di conto nelle cafe. .Ma da noi hoggi-
glia per toga , o vefte curiale , che chiamiamo Jucco 5 ¢ nel. prefente 1uogo |

  

dir quefto , *

RALVCE, Trafpare; E s’ intende, che era piena di buchi , perel
giunge pare un’ Argo fatto alla macchia , cio s' aflomiglia a un’ Argo malfa
Argo fu quei paftore , che havea cento occhi., ¢ fu lafciato.da Gi
dia d’ lo figliuola d’ Inaco conuertita da Giove in vacca; ed a sen
miglia i buchi , che crano nella vefte di Scamonea.« Plautoy fe i
chiamé cafa illuftre quella , per Ja quale per eflere il tetto rotto 5 fi vedeva il Cite
lo . Quel che voglia dire aipingere ala macchia , vedilo fopra C. 1. ft. 69, doves
vedrai anche il fignificato di gracchiare . ew

PRAT IC A, Intendiamo Confulta , o Congreffo di Confultori dallo Spagnud-
lo Platica ragionamenio , difcorfo , donde Praticare um megezio yuol dir ,
© maneggiare un negozio . Varchi St. Fior. lib. 14. Ragunafi la eraser
ro , che per effer la Citta ferma , non faceva bifogno fare altra {pefa. wa
volo credo , che intenda furbar la noffra pratica , cioe dar diflurbo a 2.3
noftra amica , perché baver xna pratica fi dice quand’ uno ha ,o fitiene qualche» | a,
donna , 0 innamorata : ¢ corrobora quefta opinione il fapere , che Baldone noms Uy
flurbava il Configlio de’ Diavoli , ne Ji loro congreffi , o pratiche , ma

=

Martinazza con aflediar Malmantile , “ae AG
L! HO nei zero. L' ho nel forame : Non lo ftimo. Zero é la figura tonda: ‘ay
Abbaco detta forfe da Giro , la quale forma le decine , ¢ per fimilitudine s* inten Pie
de il torame , ¢ ci feruiamo di quefta parola per coprire il detto fporcost’ hols | hi
c..., ulatiffimo fra la gente bafia in quefto fignificato di difprezzo; equitormas | %
bene , perché dice con rasta Ja {ua aritmetica , cioe abbaco , io/’bo nel zero, che® | ity
figura di aritmetica . 4

BACCHIO , Baltone , 0 pertica dal Latino baculus . E felleticare qui intendes
perquotere ; ¢ parla ironico , perché le baftonate {ono contrarie del folletico, _

NON fara in gramatica, Non fara difhcile , ¢ che ci voglia grande ftadio.
Gramatica preflo gli antichi volea dire /ingua Latina , come quella’, per Ji
la quale ci bifognava lo ftudio della gramauca . E percid la Greca anticajoyvero I
Ellinica , ¢ litterale ,.che fi conferna folamence nelie (crivure ; a differenzadelld |\,¢
volgare , ¢ moderna, Ja quale oggi fi parla, corrotta da quell’ antica, ¢ fichiama =
Komeca , cioe Greca de’ sempi baffi , ne’ quali i Greci non pit tennero il lor antico
nome di Aellines , ma per gl Imperatori Romani , che in Oriente avevan trasfe-
rito ! imperio Romei comincjaronfi a nominare ; quelia Greca antica , dico , tr-
vafi chiamata gramatica greca; perché gli Odierni Greci per apprenderla my

fogno

2

 

 
  
   
  
    
 
   
  
   
    

 
 

SESTO CANTARE: 319
fognd di gramatica, fi come noi per imparare la Latina . Nel principio dell’ an-
tico ss enero ee delle vite di Plutarco fi legge . Qui comincia la
di Plutarco, la quale fne traslatata di gramatica greca in voleare greco in Rodi,
B perché la Grammatica’é cofa {pinofa , ¢ difficile ; per quefto il dichiarare,¢
re l'intelligenza di qualche fatto , o queftione ofcura , ¢ imbrogliata di-
sgramaticare
RACHE piene , Per la‘paura fi movera loro il ventre , ¢ s’ empicranno le
» Vedi fopra C. 1. ft. 43.
{T1CO., Vno che difiicilmente ha i] benefizio del corpo.
ME paralitica , Cioe tutta tremante come fono | paralitici .
VE fia noi? Dove credi tu d’ ellere ? Termine che fignifica Porta rifpetto
jerfone sed al nega dovetu fei, Alcflandro fentendofi recitare da uno , ches
diftefa la Storia de’ (uoi fatti , una narrazione lontana dal vero ; diffe allo
|5, Evdove eramo noi allora? quali dicefle : Che non ti ricordi ; che io v’ era
5? Altre volte fignifica: Che non hai gindizio? per elempio T dai cexto
tale , che non ha haver 50, , dove fiam noi? cioe dove fiam noi col cerucllo?
E si? Termine ufato per indurre timore , ed ha del giuratorio; E che si,
? quali dica: Giuro che si; ch’ io ti zombero , fe tu nox parli meglio. Si
per fare flar a fegno i fanciulli, E che si , che io vengo cofid, e vi sferzo,
Sidice anche, Vale 0 ginochiamo , 0 ftiamo a vedere , che io visferze ? Vin Poeta
moderno fe-ne ferui per giochiamo , dicendo :,
‘ahscas E che si, padron mio , ch'io m’ indovine
SD ei Del voffro andar girando la cagione ?
SCORRETT ACCIO . Huomo fcorretto diciamo colui , che fenza rifpetto al-
Gund dice parole'{porche' ed ofcene , ed indecenti in ogni laogo .
ZOMBARE. Perquotere . Bil Latino Verberare, Dal faono. Cosi Typro de?
_ Greci sche vuol dire verbero , ¢ verbo fatto dal fuono ; onde ne nacque Typanon ,
=. we yi Tamburo ; dal quale abbiam fatto noi Tamburare , ¢ T ambuffare;
i ympanum , Zombare. Apprefio i Greci bombes ¢ il rombo , 0 romore delle
Pappreflo i Latini bvméxs é il fuono che fa il corno. Appreffo di noi Bom-
bérda ¢\detta dal gran rimbombo neilo (pararfi;; € cosi tutte quefte lingue fi (ono
accordate 5 contraffacendo il fuono medefimo , che da cofe concave ulcendo, ¢
rigitando 5 e:ampliandofi perwene all’ orecchio .
&/MBOMBO . Rifaonamento , I’ Eco , cio’ quel fuono che refta alquanto
ug romore 5 ¢ maffime ne i luoghi cavernofi. Dante Inf..C, 16,
Gid era il loco , ove s' udia il rimbombo
by i Dell’ acqua che cadea nel? altro giro
ac hi ' Simil a quel che  arnie fanno rombo ,
A VA col calzar del piombe , C: ina adagio ; ¢ fd nelle tue op
4 diy Governati con prudenza, Lat. A¢arura ler, Dante Par, C. 13,
yg 2 3 E quefto ti fia fempre pivmbo a* piedi
4 = Per farti muover lento come buom laffo y
o
i

   
  

 

 

RADDA Ed al si, ed al mo, che tu non vedi

 

 
 

     
 
 
 
 
 
 
    
 
 

320 MALIMAN TIE BS @
(STANZA CV. : Z
S* alza Scorpione alloraye wiere da ef
ate it Corno orvibile, propofhoy
Che gli eferciti dive in fuga ba, melo.
Conforme ferive , ¢,accerta, 2 Arioffo.
Si rallegra Pluton ,e dice ; Adefso
Naw ci fara: dal Cancelliere oppofhes
Perche ci calza bene , € certo quefta
Cofa del corno a mevarper Ia testa.
STA
Vuoi forfe darci qualche eceezsone,?
Stiamo in decretis ; diy peta veffito ;
Va ben , rifponde il Sere, ch ex proponey
Cofa , che non deprava ordines9 Ti0% ognun.
Fatta che hebbe Plutone la.bravata a Scamonea,fi riazo Scorpione
volo , e propole, che fi pigliaiie. i| Cormo.d: Afiolfo,, il che piacque a.
per quetto fi volioal Cancelliece domandandoli,fe ci havewa difhicul
provo ; Onde Plutone ordino , che fi.faceffe il partito.. F
SOGGHIGNARE.. Mottrare , 0 far fegno di ridere quali dafu
ere per fegno di. dif

  
   

bene in:faa forza & il latino fubridere » ed ¢ un certo,
zo , 0 di poca ftima,che altri faccia di, qualcofa 5. ¢ fi, chiamayrifo
cio¢ non puro , non vero; ma, fate. ;

JO non fon qui per candeliiere. 1o.non fon qui folamente per far
devo dire ancor’io i mio. parere, quando occorra .

DOTTOR de’ mei firvali. Termine di difprezzo, e vuol dire
Vedifopra C, 4, tt 10. Aewe

PET O-veftite « Che cofa fia peto , vedemmo nell’ ottava roo, d
quando.il vento efce dalle parti da baflo accompagnato con qualcofa altro,
ce peto veltito . Eda guefto il Lettore pud comprendere quel che fignifichi,,
SONATE un doppio, Quand’ altri dopo molte cole mal fatte ne fauna bent
dal medefimo folita farfi di rado., 0 vero dopo , che uno habbia terminata
faccenda con grande ftenta, ed in molto tempo, diciamo. : Sonate wm ¢i0
tutte le campane per I’ allegrezza di quelta cola infolita » © della rerminagions
di quefta faccenda ,che fi penfava non -haveffe a efler terminata may) |

F-AR il partite, Fas.loferutinio , che noi volgarmente diciamo far lo /gxitine}

 
 

© fquittinare .
STANZA CVIIIL, STANZA CIX.

Vanno le fave attorno , edi lupini-, Vauno i danyelti ognun dalla fua banehy
E fentefi fiuonato ,¢ fuor di chiave Ma perché ne ricevan: Be
Alle panche gridar : Tavolaccsné 5 * Che pis neffuna ardy a it.Re comand
Raccogliete pel numero ,¢ le fave Se mon vuolyche a pier popolo fi sferti
Pigliate in man ; che quefti cittadini , Di nuovo attorno s boffoli fi manda
Che in fimil Luogo far dourianfulgrave Da vincerfi il partite pe’ due rere
Rendano( il capo havendo pien ds baie ) E cercate alla fin tutte le panchty _
Male i partivi , e mangian le civaie . Fu vinto non oftante cana:

te

.

RPS R RE SRA HE GRE ERE REESE

 
 
 
  
  
 
 
 
  
 
 
 
  
  
 
   
 
  
 
  

Bess ~~

 

 
  

| j donzelli vanno raccogliendo i vor!
ti in contrario fu vinto , che fi

lone da Malmantile . E qui ters
~ Vedil Ariofo nei fuo Orlando furiofo., che lo finge uns

i fyono fugava Ja gente .
fave arr edi Inpini. E’ coftuine in Firefze , come era anches
di fare i partiti , o (quittini con fave~, e Jupini; ¢ pero havendo i] Poe-
uto, che nel ConGglio grande di Firenze chiamato il Configlio dei Dugen-
yhel quale inte ono centinaia, ¢ cehtinaia di perfone ( come in quefto
Configlio de’ Diavoli ¢ neceflario, che interueniflero fopra 300, Demonj, mentre
to voti non impedivano il yincere il partito) i Tavolaccint , Donzelli van-
F endo le fave , ed i lupini a coloro , che devon rendere i) partico , fas
‘il medefimo coftume nel prefente configlio de’ Diavoli, dove dice che fi fen-
idare fPuonato , ¢ fuor di chiave, cioé in voce, che non intuona , ¢ non accorda,
) procede y perche efiendo pili d’uno , ed in diverfe parti delia ftanza a,
impotfibile che s’ accordino nel tuono , come anche perché dette yoci
ite'fra tanta gente,che bisbiglia., il che le rend ottule , ed offulcate .
YOLACCINO , Seruo ,0.Donzello di Magiftrato ; cosi detto fecondo al-
r abellio detto fopta in quetto C. ft. 74., ma io credo, che i Tavolaccini ,
che fono un ‘numero determinato , ¢ differenti dagli altri Donzelli , fieno quelli
che al rempo della Repubbiicha ftavano fempre in palazzo , ¢ fervivano alla ta-
] vola de’ $5. ciafcuno il fu’, ¢ due n’ haveva il Gonfalonicre, ¢ fi dicevano Ta-
# — volaccini dal feruire alle Tavole ; ¢ che habbiano conferuato il nome , fi come fi
conferua ancora J" ufizio , eflendo coftoro obbligati a andare a feruire alle tavole
eo in palagzo del Serenifs. G, Duca in occafione di Forefticri , 0 di Spolalizzj , ec.
ma per altro aprono ogni mattina , ¢ ferrano ogni (era le Porte della Citta. °
» RACCOG LIE le fave per il numero. A fine di faper con facilica, quanti fieao
coloro , che rendono i} voto , il Tavolaccino pigia in mano’da ciafcuno una fa-
va, © poi fi contano., ed indicano il numero de ivotanu , equelto fi dice
c i numero, E pigliano le fave in mano , enon nel boflolo , per aflicu-
-rarfi che non vi fia chi ne metta pi d’una , ed alteri il numero ,
STAR ful grave . Tener il decoro , la gravita. Star favio ,
HA il capo pien di baie . Sempre vuole icherzare .
RENDER jl parrito , E’ quel dare , 0 mecter 1a fava , 0 lupino nel boflolo, che
fi dice: dare il voto.
4 PIEN popolo, In prefenza , ed a vilta di tutto il popolo .
“BOSSOLO . Quel vafo, nel quale fi metiono i voti dagli Ateniefi detto Camus,

 
  
 
 
  
   
 
  
  
 
 
  
  
 
 
 
  
 
 

   

 

4 Vedi fopra C, 1. ft. 37.

ams

es

if FINE DEL SESTO CANTARE.

J i

fi

h ot

; pw

i Ss \ SET-

 

 

 
 
  
 

  
  
   
 
  
 
 
  
 
 
 
  
 
       
        
   
   
    

SUSE le a
oe

a

®

= =

ARGOMENTO.
Paride dop’ haver molto bevuto
Entra,’ andar’ al campo , in frenefia, ae
E come il fonno havea pel ber perduto , ;
3 Perde nel gir di notte anche la via: te
Cade in un fofso, onde a donargli aiuto ath:
. Corron le Fate ,¢ gli nfan cortefia; eee
Vien condotto sn un’ antro , e per diporte e
La froria gli é narrata di Magorto . :

Sepepapeapeaye pases

joinaatuaiaa
?

Beis -gFf effete fer

STANZA L STANZAII
V Ino tempera te diffe Catone , Perche fe quel s marrage ne einuecebidy Me
Perche fi dee berue a modo,eaverfo, Ed ¢ burlato il tempo di [ua vithy =| ii

E non come cola qualche trincone , Almen Sent ilfapor di quel cb'el

Che, giorno, e notte fempre fa un ver{o; E tien la faccia roffa ye colorita

Ond! et fi quoce,e percié ei va aGirone, Buvlar anche fi fa chi va alia.

La fauola divien dell’ univerfo , E infacea fenza gufto a

E vede poi morendo in tempo breve Che'jo tien sépre bol/oyein man del

Chie ver sche chi pit beve mance beve. Aquall we a Pe morir di tified.
lL

        
    
   
 

Bre aa Ee

STANZA I STANZAIV,
S? il troppo vino fa, che 'hnom foggiace Pero fia chi fi mae egli, é um dappoce
etal’ error di tanto pregiudizio ; Chi imbotta al sayy come gli
Chi non ne beve,e quedo,a cui no piace, 5S! avvegzi a ber del vinoa peeve,
Aquefto cite dunque ba un gra gindizio; Chiei fa che Vacqua fa marcire si pally

 
 
 

 

  

Anzi che nd , fia detto con Sua pace, Aa com’ io dico fi vnol berne pipe ;
Per c ogni eftremo finalmente ¢ virio , Bafta ogni. volta cingue , 6 fe

E fe di biasmo ¢ degno Punose taltro Perch’ egli é poi nocivo il ersncar tame,
Quefto ha il vataggio al mio parer stz'altro, Com' udirere adeffo in quefte C
Volendo il Pocta narrare in quefto Canto I’ accidente occorfo a Paride
ni, per haver troppo bevuto, s' introduce col riflettere, che ficcome ¢ male
molto vino , cosi che fia anche male il bere folamente acqua; ¢ 2 che
dovendofi eleggere uno dei due mali, fia meglio eleggere quello del ber vi ,
ma pero regolatamente , et MO-

    
  
    
 

    
     
  

SETTIMO CANTARE: 323
_AMODO ,¢ a verfo. Regolatamente , E’ il latino vulgato : modis, o formis
dmipblein co : i
_ TRINCONE.. Vno che beva afflai . Da Trinchen Tedelco bere , tirar git .
i fopra C. 1, ft. 6. Si dice anche pecchiare nella prefeate Ottava teraa , quafi

  

‘facciare il vino come fanno le pecchie , ( cioé l’ api che fanno il miele , cosi der-
; ee me Je quali fueciano il dolce da i fiori , ed i vini bianchi geac-
‘tolit edaldetto verbo pecchiare fi dice pecchione a uno, che beve affai ; ¢ pecchione
ichiama ua’ ape faluatica,e maggiore dell’altre,che fuccia il miele prodotto dail’

api da’ Latini chiamato fucus.. Virg, gnauum fucos pecus a rabepibas arcent ,
dice ciencare nella prefente Qttava quarta. Vedi il Landino efpofizione a Di.
nf, C, 9, alla parola cionca nel verfo Che fol per pena ha la /peranza cionca , do-
dice , che cionco é parola Lombarda , ¢ fignifica moxxo , ma cioncarc in Fiorentino
mnifica difordinatamente bere ; Si che quetti tre verbi trincare , pecchiare , ¢ cioncare
fanno lo flefio fignificato , ¢ fe bene hanno del foreftiero , wuttavia fono ulati in

  

os

na

ot .
"SEMPRE fa un'verfo. Sempre fa la medefima cola. Diciamo Ver/o il canto
pce “4 Verfo del rufignuolo, Verfo del fringucilo. E da tal verfo vienes
trato .

‘WA AsGirone. Huomo, che gira; intendiamo pazzo. E perd feruendoci della
voce Girone, che ¢ un Villaggio vicino a Firenze , copertamente intendiamo
uno che fa delle pazzie , come s’ intende nel prefente luogo .

DIVIEN 1a favola dell univerfo, E’ burlato da wtti. 4 ore eff omni populo, It
Lalli Ba, Tr. C. 4. 2. 78.

Son fatca ime la favola del mondo

I) Pett, Ata ben veggio or , fi come al popol tutto Favela fui gran tempo. Tibullo lib,
1, Ne turpis fabula iam . Nella {crittura . Et fattus /um illis in parabolam ,

CHI pitt beve manco beve , Cioé , chi troppo beve s’ ammaia,e muore,e cosi vive
poco , € per confeguenza beve manco , cive dura a bere manco tempo di colui ,
che beve poco. Marz, lib. 6. Jmmodicis brevis eff atas , © rara fenettus , che das
noi poi fi dice in proverbio Poco cs vive chi troppo fparecchia. A fimilitudine di que-
flo fi dice: Chi piie fudia , manco findia ,

OGNTeftremo é vizio. Ogni eftremo ¢ male. Ogni troppo é troppo. Quefta
fentenza wfiamo dirla // troppo, e il poco Guafta id gineco, Al che pare,che facciano
molto a propofito i feguenti verfi di Orazio .

Eft modus sn rebus , funt certi denique fines ,
nos ultra , citraque nequit confiftere rectum ,
E Terenzio mettendo in Latino una fentenza d’ un favio della Grecia diffe ; We

ee a es ee ee

quid nimis ,

SENZ’ altro, Affolutamente ; fenz’ alcun dubbio . Latino fant , procul dubio

VA ala fecchia, Beve acqua. Secchia diciamo quel valo , col quale fi cavas
—_ da i pozzi dal Latino firwla . Vedi fopra C. 5. ft. 10,
: ACCA, Per Gmilitudine diciamo facco al ventre deil’ huomo;quindi dnfacs
tare vuol dir Mandar git nel ventre . Pulci Morg. C. 19. ft. 137.
sot E mangia , ¢ beve , e infacca per due verri
Peril contratio/acar in iipagubelo é trarre , —_— fuori ,

8 2

i ee

scl

 

 
 

 

324 MALMANTILE: >

S¢lelT A. Che non ha fapore alcuno . Deh kagions ia, SR
BOLSO. Vedi — C. 3. ft. 53. Graffo non naturale , con di direlpi-
ro . Cavallo bolfo i Franzefi dicono pou/if dal pullare , ciot | a f
Ja Jena affannata. Lucano lib. 4. Pettora rauca gerunt , qua creber anbelitus urge
Ex defetta gravis longé trabit ila pulfus . r aivye
LN man del Fifico, Col medico fempre attorno ; cioé fempreinfermo, ©
CAL imborta al porzo. Chi beve fempre acqua. E’ lo ftello che dafaceare
to fopra. 0 aged
ANIM ALE . Intende animale irrazionale . Se bene la voce animale & generi-
ca,¢ comprende fotto di fe anche |’ huomo , noi ce ne feruiamo per fpeciale, in~
tendendo folamente le beftie , fiche dicendofi a un’ huomo T# fei wn! animale, in.
tendiamo Tx fei una beftia ; Vat srragionevole , ae
Ss? AVEZZI, 8 afuefaccia., Vedi fopra C, 6, ft, 101. Nth > Aa
FA marcirei pali, Vuol dire:il vino fi guafla annacquandolo , quafi dica;
infradiciare i pali , che reggono le viti , che producono il vino ; o fe fara
infradiciare il vino , che nafce dalle viti., che fono pitt deboli de i pali, mentre
fon da effi foftenute , Dichiamo anche per biafimare I’ ulo dell’ acqua: 2? acqua
rovina é ponti : quafi s’ abbia a intendere + O pepfate, fe non royinera gli foma-
chi deglt huomini , che fono pit deboli ! > KE
BOCC ALE. E una milura capace della meta d' un fiafco Fiorentino, Dice
cinque 0 fei boccali per {cherzo,{apendo bene , che ogni maggiore bevitore non

     

  

 

bevera mai Gi gran quantita in una volta, van
STANZA V. STANZA. Vide
Omai ferra gli ordinghi , ¢ le ciabatte E Paride, ¢' anch'egii fi ritrova
Chiunque lavora,e vive in {ul travaglio, A corpo voto in quelle e hie}!

E difilato a cena fe la batte

A cafayo dove piit gh viene il taglio.
Chi dal compagno aufo il dente sbatte,
Tanti ne vaatavernach'é unbarbaglio,
Parte alla bufea,e infin,pur che firoda,
Per tutto ¢ buona fhanza, on'altri goda,

D Amor chiarito figliod’ una Lous,
Che {ualiziar gli ha fatto le bufecchie y
Dice al villan:Va a coprarmi delves
Ecco fei gink y sonne ben parecelne 5
Piglia del pane ye fopra tutto areca
Buon vino fai\ non. qualche cerboueca,

STANZA VII, Abe

Eset’ avanza poi qualche quattrino y

Spendilo in cacio , non mi portar reffor
Meller fine , rifpofe tl Conradino ,
Jo torva , 8°10. ne trove, ancor corefto.

E partendo gli ride? occhioliney
Sperando haver a far un po dagrefe;
Aa , facendo i {noi conts per la via,
S’ accorgeche e’ non v' ¢ da far calia.

 

Deferive afai vagamente il venir della notte ; fu Ja quale ora Paride affalito
dalla fame comanda a Mco {uo contadino , che vada a comprar roba da-ma
giare,¢ da bere, ¢ per tale effetto gli da fei giuli , con ordine che gli fpendas

i 5G

tuttl. e by . soi 92
ORDINGHT, Intende ogni forta d’ arneGi, ingegni, machines firemsentiod’
Javorare. Diciamo anche Ordigni ; anzi gli anuchi non difero algrimentis

CIABATTE,, Vuol dir propriamente {carpe vecchie , ¢ quelle {carpe all! Ap *

ftolica , che ufano i frati fealzi , ma s’ intende anche i frammento di ma-
tcriali di coloro che lavorano’, ¢ per ogni {orta di mafieriziuole veechies’¢ 6
fumate, che i Latiat diconofcrara., Z WWE

Ay

ZeRm =

— t

Bee eFFezeane28=

zs
_

Fae BF ee SSR est avEeESeF®.

 

 
 

  
     
  
  
    

SETTIMO CANTARE: 325
_ VIVE in faultravaglio, Latino manribus viitum quaritat , Campa delle /uabpaccia .

‘Travagliare in lingua Francefe vuol dir lavorarc , ed in Firenze pure é ufato in.
6 denbo diegndns : cofa ben travagiiata in vece di ben ieleaans edi qui fi di-
in vece di viver col lavoro , o con le fue fatiche , cioé di quel che fi
alavorare. Petr. C, 3.
es ynque animale alberga in Terra,
Se non fe alquanti.c' hanno in odio il Sole ,
: Tempo da travagliare é ,quantoe il giorno:
Ma poi che 'l Ciel accende le (ue Hele ,
Ree rteit. 9 * Qual tornaa cafa equal s' annida in felua ,
ve % portent Per aver pofa almen infin all’ alia.
es ben per altro travagiiare yuo! dire efler’ anguftiato da infermita , 0 da altro.

_ _ DIFILATO . A diriuura: Latino re%a. Con preftezza , ¢ fenza fermarfi.
ze ferue anche fotto in quefto C. ft, 63. Varchi Stor. Fior, lib. 9.

non prima giunto a Firenze che andandofene difilato, fenra pur cavarfi git

i SE la bate ..Se ne va via. E’ termine affai ufato fra la gente bafla per efpri-
ib yia , o partirfi in fretta , ed ha del furbelco batsere /a caleofa , ciok
: utter la trade , andar via , camminare , donde /frada battuta yuol dire ftrada. ,
a 2 camminata , 0 ftrada di paflo . Latino via trita, Lucrezio Avia Pie-
i‘ Tidum peragro loca y nullins ante Trita fol, 1\ Petcarcha difle : Ogni fegnato calle,
—- Prove contrario alla tranquilla vita.
DOVE gli viene il tagio . Dove gli torna pit comodo . Vedi fopra C. 2. ft. 48.
' ', Senza. {pendere . E’ detto plebea... Si {crivono da i Magiftrati di Fi.
_ renee: di,commiflioni ai Miniftri forenfi , le quali da coloro ,che le chieg-
g0n0 , ele prefentang; fi pagano.a i Magiftrati , che Je fanno , ed a i Miniftri ,
I¢ le gicevono ; ¢ quando non fono chielte , ma fon fatte , ¢ mandate per pro.
‘Prio interefle di quel Magi(trato , che Je fa,non vi ¢ {pefa alcuna , ¢ pero afiaché
tall lettere , le quali non {i pagano,fi potiano diftinguer da quelle , che fi pagano,
fetivono nella fopra(critta ¢x ofitio , ma I’ abbreviano {crivendo ex Vifo, ed i ta-
volaccini , o donzelli , che le confegnano non leggono fe non ex fo, ¢ diftin-
Buono quefte due {pecie di lettere , dando a quelle , che fi pagano il nome di let-
tere col diritto , cio’ con la dovuta.fpefa , ¢d all’ altre il nome.dell’ Hf , cioe fen-
za fpela + Edi qui ¢ nato guefto detto a Vo, che vuol dir fenza {pela , ¢ ferve in
gai uccafione .

ckseee=

*~

A SBATTE if dente . Ciot mangia . 3

J E un barbaglio . Son tanti , che fanno abbagliare ; Non fe ne pué raccorre il

f conto fenza sbagliare ; o abbarbagliarfi., cioe errare ; dal Parpaglione , che diffe-
70 gli antichi alla Provenzale;cioé dal Latino papilio;farfalla;di cui ¢ noto lerrare

® — intorno al jume .

é CILLA bufca , Cercando {ua ventura .. Bufeare.. Vuol dir Acquiftare , otte-

iy ere , puadagnare.. E dalla Spagnuola ba/car yenuta a noi quefta voce infiemes

) SON molte altre negli nltimi cempi .

i Si reda, Simangi. Sc bene rodere fi dice de’ topi,de’ tarli , ¢ fimili. Per tutto

y _Pbuena anza on’ alsri gode , Voi bonum , ibe patria. Dove fi tla bene , quello é

‘ 42)" buoa

 

 
—E

326 ' MALMANTILE *©
buon pacfe; E per ogni patfe , ¢ buona Stanza, Difle come in proverbio il Pe.

trarca . ‘ Sp heist tity
CAT APECCHIE , (ntendiamo luoghi orridi , inculti’, € dil . Mattio
Franzefi in lode delle gotte : Alor per ufcir di queffe carapecchie, N
do che pecchra & fatto da apes , apecala , 0 apicu/a cosi verifimilmente carapere
puo dedurfi da apex apichlus , che vuol dire piccola fommitdy ¢ cara prepofizione —
Greca , la quale dice un certo ordine’, o € aggiunta pet maggior forza , come fi
vede nelle parole , Carafalco , Cataictto y Caruno', che differo gli antichi per =
Scheduno ,¢ fimili . > te
CALARITO. Aggiuftato Vedi fopra C. 1. flan. 1.Vuol dir che Amore I’ ha
= accomodato , perché s' era pieno di mal di chiafio , come fi diffe one
in. 11. Oe ie
LOVA. Lorda ; Poltrona . E’ parola d’ingiuria a tina donna. E*yoce fira-
niera ; ¢ yuo) dir Lupa ; che fimilmeate gli Spagnuoli dicono soba; ¢ si
maeretrice. Gio. Vill, lib. 1. cap. 25. parlando di Romulo’; ¢ Remo allevati da
una Lupa dice: Questa Laurenza era bella, ¢ di {uo corpo guadagnava come
ce ,€ pero dai vicini era chiamata Lupa ; onde fi dice furono nutricati da lupa 3 il che
cavo egli da Livio lib. 1. /une qui Laurentiam valgato corpore lupam voratam inter
Stores putent : inde locum fabula , & miraculo datum , +:
SVALIG/ ARE. Cavar della valigia. Qui intende; gli ha fatto‘confamarei
denari, perché ba/ecche fe bene fi dicono i ventricini del porco Boce. git
Noy. 10. Dove Je femmine vanno in xoccoli [u pe i monti riveftendos pores delle lor bu
Jfecchie medefime noi le pigliamo per tafche , 0 borfe , nelle quali i tengono ida
nari. E /wasgiare propriamencte intendiamo , quando i Jadri digtrada rubano @>
uno tutto quello , che egli ha addoffo ; ¢ lo pigliamo per finonimo di /accheggiare;
‘PARECC HIE , Numero indeterminato che efprime , Molti ; dal Lat.
que , {econdo alcuni: Volgarizzamento di Palladio manoferitto ; Nel mele di
Marzo al cap. de ficu . Si metta fotto alle barbe parecchie pietre , *
CEKSONEC.A, Vino fradicio . L' Accademico Fiorentino incerto 5 cos! n0-
minato in una Raccolta di Rime piacevoli , che dicemmo altroye eflere il Bare
chielio , defcrivendo un cattivo vino dice , Te et

 

 

Staccio non pafferebbe ne framigna
Tiant’ é morchiato ,¢ con la feccia miffo;
Sciroppo mi par ber , ma non di vigna; Ly
Chi ne beve non ghigna , ej
Ch’ egh: é ciprigno , e cerboneca fina ; ' b
Chindendo gli occhi , mi par medicina ,

Brunetto Latini nel fuo Pataffio diffe Cerbonea .
Wel ver queft’ ¢ pur nuova Cerbones Hem
Forfe fi dourebbe dir cerconeca , derivando quefta voce da cercone che vuol dit
Fea ease at fi dice cercone dal circolare , che fa il vino quando da Ja volta s §
fig ¥ ac SAM

 

 

 
  

" Gulbinges ver{o ;
Hciibiekh Rife , & argutis, quiddam promifit ocellis ,

agreffo. Avanzare; ma intende d’ avanzo illecito , come farebbe, quan-

acomprare roba, dice havere (pelo pit di quello , che ha (pelo,

quell’ avanzo. Vien da i cantadini, che per rubare al padrone piglia~

f ey va non matura , ( che fi chiama agreffo ) e ne fanno fugo , ¢ lo veadono.

j

{

SETTIMO

CANTARE : 327

QUESSER fine, Vuol dir Meffer si, Ma dice Meffer fine, perché fa parlare a

jun contadino : noffri fic rire toquuntur .

t occhioline . Vuol dir fi rallegra. Ul rider dell’ occhio forle accennd

 

    

le cravaglic de la vita ,

 

\ lo termine ha lo fteffo fignificato anche in Napoli , come fi cava da lo Cun-
to deli Cunti di Gianalefio Abbattutis gior. 1. Cunto 8. dove dice; A¢oPrannole
kefrifolesco’ li quale maritattero turte L autre fizlic , reftannole pure agrefta pe’ gliottere

IN v' ¢ da far calia. Non y’ é da far avanzi. Calta fi dicono quei_rimafu-
a hares argeato , che nel lavorarlo cadono , ¢ fi dicono calia quali calo
a ?

rz 0 dell’ argento , che ridotto poi in proverbio efprime ogni forta di pic-
it Solo avanzo . .
o STANZA VIIL STANZAIX.

   

Perch'egl ¢ tardi,, ed ha voglia di cena,
Poi c ogni cofa ba bell’ ¢ preparato,

st v4;il pane,e ilcacto,es! vin rocac-
ep fatto un guazcabuglio nella [porta , Si frrugge , ¢ fi confuma per la pena 5
un » Le quattro lire slazzera ye fi lpaccia. Che ti non torna il meffo,ne il mandate;
Lialerol afperta agloria,e insis la porta Ma quand’ ei vedde con la {porta piena

. eee seh av ognor s! affaccta, Giunger al fine i] [uo gatto frugaso:
of EB per anticipare , il fuoco accende , O ringraziato , dice, fia Minoffe ,
ae — Lavai bicchicri,e fa L altre faccende, Ch’ una volta le furon buane mofe.
ie | bast S.T.AN.Z.AoX,

Chiappa le, robe, ¢ mentre ch’ ¢i balecca Sbhocconcellandointanto,il fiafco shocca ,
at In quocer t vova,e tl cacioch' ¢ fupendo; Econ due man alzatolo bevendo ,
ih Sente venirfi 2 acquolina in bocca y Dice al villan , che nominate é Meso:

E far lagola come un falifcendo,

Hlorsit ti fo briccone , addio , io beo .

_-M.Comtadino mandato da Paride a provveder la roba , ando all’ Oite per fbri-
arli, compro il tutto. Paride in canto ftava afpettandolo con grande anfieti ;
¢ fubico. giunto , egli mefie a quocer I’ uova ,¢ il cacio , ¢ in ranto viato dal’ im-
-pazienza , ¢ dalla fame comincio a mangiar del pane , ed a bere.
» PER la pi corta , Vuol dir per Ja rada pik corta ; ma qui intendi per sbri-
garfi pitt prefto . -
.» PROCACCIA. Provvede. Vuol propriamente dire cercar di trovare una co-
fa, ¢ trovarla ; Lat. per/equi & affequi , e(primendofi con quefto folo verbo pro-
eacciare la diligenza, che s’ ula in cercare , ¢ andare a caccia d’ una cofa, ¢ las
fortuna, che s’ ha di trovare quel che fi cerca ; onde poi molti dicono : buon pro-
| €aecino uno che s’ ingegna per ogni maniera di guadagnare .
|. GY AZZ ABVGLIO. Mefcolanza , mefcuglio, 11 Cafa acl {uo Capitolo del
Martello di amore dice ;

Nox

 

 
 
  
   
   
 
 

habe Ve > pay
  
 
  
    
   
    
  

328 MALMANTILE! ©
Non eva donna rica yo poverina yo)
Si facea d! ogni cofa un guarrabuglio —
Ogni fhanza eva camera, ¢ cucina, |
Mattio Franzefi nel (uo viaggio di Venezia dice:
Ear a una tavolata allegra cera, «
. Edi var} difcorfi un guazxabuglio >
Il Lafca Nov, 10, Verfarono aceto , vino’, olio , fale, e farina, @ fecero un euat
lio il macgior del mondo, Dal che fi cava y che quefta yoce efprime mefcolanza.
di cofe maceriali , ed anche di non materiali; Voce compofta di Guazzare’, ch
 dibattere cofa liquida , ¢ di Boélire : quali da una Ricetta che dica »Guarzs,e
Bolli ; fattone Guarzabuglio . : {
LIRA, E’ una moneta Fiorentina , che vale un giulio ¢ mezzo, detto anche
Cofime , perché il noftro G, Duca Cofimo I. inuentd , ¢ fa il primo’, che’ bat
in Firenze quefta moneta . : 33 Sepia
SLAZZERA, Cava , conta , mette fuora’, fa venir fyora'a forza’, E*
furbefca , fe bene affai ufata. gree #
S1fpaccia, Sisbriga: Si {pedifce. ;
L' ASPETT Aa gloria* L’ a(petta con gran defiderio, con pazienza efrema,
Si dice anche a/pettare a bocea aperta .. Larus bians . wine
HA bell’ ¢ preparato, Ha di gia mef' all’ ordine . Vedi fopra C. 3... 14 =
NON torna ne il Meffo, ne it Afandato , Non torna lui , e non manda alcunoa
dir quel che fia di Jui. Diciamo anche 4 ho mandate il.corno, dal coruo, che man-
dd Noe fuori dell’ arca , il quale ‘non tornd mai. pee
GATTO frugato , Cosi fon chiamati per ifcherzo da i ragazzi i contadini.
Carus in Latino é cauto , aftuto ; € con quefte nome chiamafi anche il Gatto anl-
male.notas il\quale.quando.¢ fato frugato con pertiche , o con baftoni , non fas
altro , che volgerfi {paurito , € che guatare ; onde vogliono alcuni , che abbia il
nome . Cosi i) contadino , quando {cende alla Citta, Dante Purg. 26,) ©
Non altriments flupido fi turba o }
Lo montanaro , ¢ rimirando ammuta , ae
Quando Hx 9 ¢ faluatico s inurba, ANH
VINA volta furon buone mie, Vana volta ci tornd , Quefto detto wfatiffimo ins
ucfto fignificato , vien da coloro, che flando a veder correre al palio per Jo grat
defiderio , che hanno di vedere arrivare i cavalli , {peflo gridano; Eccagl fe bea
veramente non fono ; ma pure al fine vengono , ed allora dicono, Queffe Jie iy.
buone moffe . 4 che pafiato in proverbio ; fignifica a terminazione di ¢
¢vento , 0 negozio. ite Ty
St balocca . Si trattiene . Si dice anche : Par’ a bada , 0 badaluccare ; Bi YOO Wl

ufata per ibambini. Vedi fopra ©. 6. ft. 32. he i, &

} = 2 ak

  

SPP eee SF

  

rescezrr=

    

STVPENDO . Buoniffimo . Vedi fopra C. 6. &. 35. Cofa maravigliola, hy
perfetta , che induce ftupore , ae
VENIR P acquolsna in bocca . Si fente confumar dal? appetito , © per te

 

foprabbonda Ja faliva in bocca , la qual faliva ¢ caula che /e gola gh fac bi
Salifcendo , perché il gorgozzule gli va ingid , ¢ insu per inghiottir quell’ umid by
E falifcendo & una ftrilcia di ferro , che s’ adatta a lerrar Je porte, apeaaia liy

 

 
 
  
 
 

  

  

'di pane ,¢ mangia .

01 , Onofrio ; ed altri infiniti .
Tl fobriccone. Ti

‘quidice Briccone per brindift,
_ STANZA XI.

Cost per celia cominciando a bere ,
d un forfo,e dagliens il fecondo;
Fesigche dal vedere, e non vedere y

Ei ditde'al vino totalmente fondo ;
tayola di poi meffo 4 federe ,
| Lafeiato it. ixfee voto fopra il tondo,
Viltoffi a dieci pan da Adeo provvsffi y
Eis th momento fece repuliffs .
STANZA XII.
4 i pan dotto,e uinginlio di formaggio
» Non glittoccaron I'ngola,e s'inghiotte
Due par diferque d'uova,e da vataggio,
| Potdice; Meo {pilla quella botte ,
Chet'hai per Popre,e dami il vino afaggio,
To vib feafera anch’ io far le mie lotte ,
| Ben th'io fia bere, fis ripieno,e /uentri,
Percht mi par,ch' una latcata ¢ entri,
STANZA XIII.
URufticoche dar del fuo non ufa:
- Non faper, dice, dove fia il fucchiella,
~ Che per cafa non v* & fhoppa ne fufa ,
E che quel non ¢ vin,ma acquerello .
_Civuol, rifponde Paride,altra fenfa,
By itty » di canna fa un cannello,
-E, in fa ba borte poffo a capo chino ,
~ Con eff , pel cocchiume fuccia il vino.

 
 
 
  
 
     
  
 

   

  
      
 

SASS Ek

  
   
 
   
    
 
  

eee SE Ae

  

SETTIMO CANTARE:

con'alzarla , ed abbaffarla . In quefto fignificato'diciamo ancora:
I » vedi fopra C, 5. ftan. 62.

VCELLANDO. Diciamo sbocconceilare,
compagni a menfa , o che fia portata

SBOCC A il fiafco . Stura il fiafco, ¢ fquotendolo butta fuora il vino, che & nel-
per arlo dall’ immondizie, o fiore, che vi po{’ effete.
hot Bartolomeo » Ela figura Apherefis (peffo ufata da noi ne i nomi
me Cecco per Francefco fatto da Cefco ( che trovafi nel Decamerones
cioé Francefca ) Menico per Domenico ; cos! Lippo , Stagio , Coppo ,
, Noferi , accorciarono i noftri antichi da Filippo , Anaftagio , Iacopo,

329

and’ uno, mentre afpett2,
roba in tavola, piglia de’

 
 
 
  

 

 
 
   
 

Ti fo brindifi . Quefto ¢ quel modo di parlare , che dicono Za.
» come accennammo fopra C, 1. ft. 28, al termine:u(cir del feminato ; ¢

STANZA XIV.

E perch? ¢ buono,e non di quello,il quale
E nato in fu la {chiena de’ ranocchi ,
A Meo , che piit tofto a Carnovale ,
Che per Vopre lo ferba,e/ce degli occhi,
E bada a dire ; Ovvia, vi fard male,
Ma quegliche non vuol ch'ei Pinfinocchi,
Edé la parte fua furbo, e cattivo ,

Gli rifponde : Ob tu fei caritativo.
STANZA XV.

   

= *

Lafciami hie la bocca afcintta,
Che diavol penft tn poisch'ia ne bea?
Jo poppo poppo, ma il cannel non butta,
Rifponde etieo: Po far la noftra Dea,
Che sei buttaffe, la berefti tutta,
O' diferezione s'e’ cen’ ¢ minuzzolo;
Paride beve,e poi gl da lo {pruzolo,

STA I

Non vi fo dir fe Meo sllor tarocca:

Ma P altro, che del vin fu stpre chiotte 5
Di nnovo appicca al fuo canel la bocca,
E lafcia brontolare , ¢ tira fotto ,

Ma tanto efclama,prega,dagli,e tocca,
Chrei lafcia al fin diber gia mexxo cotro,
Dicendo ch'ei non vuol ch’ il vin lo quoca,
eta che chi lo trove non era un’ oca,

if Patide in barla in burla bevendo , votd il fiafco , ¢ poi fi mangid dieci pani,
Prova, il cacio provveduto da Meco , i] quale egli prego,che gli defle a laggio

 
    

 

 

botte , ¢ Meo adduce diverfe {cule per non glielo dare; ondes
; re

Paride

 

 
  
 
 
 

    

330 MALMANTILE

Paride fatto un bocciuolo di canna fi meffe a fucciare il vino per
= s —S cui duole il <a seatats fuo 4!
ere ; ma egli feguita, ¢ per farlo pit arrabbiare gli sbruffa
torna a bere. Al fine gia fazio , laid ftar di Croniede
buona cofa , e che !’ Inuentore fu un gran valent’ huomo; ma
ber pil, per non’ imbriacare . 2 ety X iy
PER celia, Voce ufatitiima in Firenze , per denotare buria , feberze.
una giovane Commediante , la quale era di genio {cherzolo, ¢ burlefco, ¢ face

   
 
  

 
   
    
    
 

 

 
 
    

Ja parte della ferua ; ¢ fi domandava Celia.
| Ui Perfiani . Ji tno canto ¢ pik dolce d’ una auelia; ea
Ma feufami , fe reco io fo la Celia.

DAGLIENE un forfe @c, Cioé bevi un poco , ¢ poi un’ altro p
Ja quantita di vino , o d’ altro liquore , che fi pud bere fenza ripigliar
Latino /orbere . ‘ Mee
FAs) che dal vedere , ¢ non vedere. La cofa andd in maniera , che it
mento ; in un batter d’ ecchio. /n stfu oculi .
DIEDE fondo al vine, Cioe vord il fiafco . Fini il vino. Dar fondo a un
fa yuol dir confumare affatto , Termine marinavelco ; ¢ fi dice dar fondo
Ja nave fi ferma in porto , finito il viaggio . ;
TONDO, Cost chiamiamo quel piatto fpianato di ftagno, o d’ altra}
pra il quale in tavola fi pofano i bicchieri .
FECE repulifti , Fini ; ripuli , confumo ogni cofa , ne volle veder la
inine baffo , e ufato dalla plebe . ;
NON gli toccaron ? ugola . Non gli {cemarono } appetito . Quando a tn gran-
de affamato fi da poco cibo , diciamo: Won gli ha toccato ? ugola , ¢ ancora + Now
elt ba roccato un dente ,e proverbialmente : E ffata una fava in bocca all orfo.
non palatum rigat . Vgola fi dice quella particella carnofa , che pende fra le faucl
per ufo di formar conucnientemente la voce . Latino wa , columella, =
SERQVA, Numero di dodici , ma fi dice d’ vova , di pere ye fimili, che}
altro fi dice dozzina . cay
SPILLA la bore. Buca la botte . Spillare fi dice da spillo,che & quel
to , col quale fi bucano le botti , ¢ quefto forfe dal Latino /picu/am , 0 pure
Spinula , Crefcenzio lib. 4. c. 41. chiama /pina fecaria , € '| fuo antico —
zatore , (pina fecciaia , la cannella pofta nel fondo de’ yafi da vino, per
ufcire la feccia . $
OPERE, Coloro che aiutano lavorare a i contadini , ricevendo il p
Ic loro fatiche giorno per giorno fi dicono opere , 6 opre. In Latino fimi
opera fi dicono | lavoranti . + Om
VVO far le mie lorte , Voglio far le mie forze. Voglio pigliarmi tutte lo
disfazioni poffibili, Diciamo ; i/ sale vnol troppe lorte , troppe inuenie , troppi
troppe cirimonie : quand’ uno in far’ un’-operazione la vuol far con ogni
ancor che fuperfluo , ¢ non neceflario . a
SVENT RI . Scoppi per lo troppo mangiare , e bere. ”
VINA lartata centri, Ci ftia bene una lattata. Diciamo: fare uma lat
do dopo che s’ é mangiato,, ¢ bevuto beac, fi fa venir in tavola nu

   
 
 
 
 
 
   
   

 
 
   
    
  
    
  
  
    
    
 

Pow eaenw Re BB OSS. fo. ee ee

 

      
 
 

 

  

 

    

tetas Hess:
Stes *

= Fs
 

SETTIMO CANTARE:? 331

nuovi bicchieri puliti. Che per altro /atrara & una bevanda fatta con zucchero ,
-orz0 , ¢ femi di popone , che beniffimo pefti , ¢ liquefatti con acqua gli fannd

Pe

yt paflare per ftamigoa_, la quale fi da per Jo pit a’ febbricitanti per rinfrefcare : ed
. t PY pat gr i pad. hese! abbisto pot il nome di /attata ot fuddetto nuovo
bere f » come che vogliano intendere , che quefto fecondo bere non fias
_ $propofitato , ne per gola , ma per rinfre(care I’ ardore del vino bevuto, come fa
alla febbre Ja datara , Ja quale diciamo pid comuncmente orzata .
- S¥CCAIELLO, Diminutivo di /ucchio, che vale lo Neflo. Strumento d’ ac:
ciaio per ufo di bucar legnami: ¢ il Latino Terebra.
NON ha froppa , ve fufa, 1 villano per non dar bere, trova (cufa di non poter
metter fa cannelia alla botce , perch¢ non ha ftoppa da avvoltare in fulla cannel-
Ja per adattarla al buco della botte , ne meno pud bucarla , perché non ha fulas
da turare il buco dello {pillo , delli quali fui ( che per altro feruono alle donne
Se fopra il filo , quando filano a rocca ) ci feruiamo per turare fimili
gi bucht yperché per effer ben tondi , ¢ di figura piramidale,ferran bene ogni buco ,

A di pi r fcula , che quello non ¢ vino, ma acquerello, che é la lavatura

   

ai
io

 

 

gi delle vinacce , ¢ ferue per bevanda de i contadini , da molti detto vinello, ¢ das
2 altri mezzingo, ¢ da i Latini Lorea , 0 Lora. Ma Paride, che molto ben conofce,

che: oat fono tutte inuenzioni , gli dice : C+ vxol altra feufa , ed intende; Non
~ - Mailerrd per quefto di far quel che io ho in animo , cio¢ di bere.

COCCHIV AE . Quel turacciolo di legno , col quale fi tura la buca di fopras
sh della botte ; ¢ fi chiama cosi anche la ftefia buca . | Latini lo dicono do/ij opercu-
him.”

ie SVCCIARE. Attrarre a fe l’ umido , 0 fugo. Dal Latino /were.
a NATO in fu le {chiene de’ ranecchj , Nato ne i pantani,dove ftanno i ranocchi,
j@  chenoné vin buono .
i _ ESCE degli occhs. Non pud vederlo confumare + Jo da mal volentieri, Gli

»duole il veder confumar quel vino , quanto gli dorrebbe il perdere il lume degli
@ Occhi. Detto affai ufato in fimile propofito .
2 NON vnol che ? infinocchi, Non vuol che con le chiacchiere lo ritenga dal bere
gi —_ ‘Tnfinvccbiare & lo fteflo,che dar panzane , bubbole , 0 chiacchiere ed é il Latino Ver-
a) ba dare, 1 Lalli En. Tr. C. 4. ft. 107. dice .

: - Per ch’ il parlar di lei non P infinocchi ,

b OHTY fei caritativo Tu hai la gran pieta dime. E’ detto fcherzofo , ufato in
fimili congiunture , ¢ fi dice + 7% hai caritd pelofa, 0 la caritd di mona Candida,che
Pe biafcicava i confetti agli ammalati per levar loro la fatica ,
is NON fo fe tu minchioni la mattea. Non {0 fe ww burli , Vedi fopra C. 4. ft. 15.

_ Che penfi tu mai ch’ io ne bea? Quanto penfi tu , ch’ io al fine ne beva . Altro~
i ve habbi: detto di quefta particella mai , che altre volte afferma , altre volte
6 hega , ed altre volte fignifica tempo , come qui, che vuol dire , quanto penfi tu ,
4 Piblesnihiximene boda; In Latino dircbbelt, Waid demim cenfer?

40 poppo Poppe « Cioé io attendo a fucciare , ma io tiro fu poco vino, perché il

Cannello ne da 3
of PVO! far la nena Dea, Efclamazione , 0 giuramento di contadini; quafi yo-
a Jendo fignificare /a Dea Pales. Virg. 3. ark) Te quoque magna Pales Oc, <
4 12 Te 2 SE

 
va Se

—

——

 

 

332 MALMANTILE a2, y
SE @ cen’? minwzob , Se cen’ & punto. Se ei cen’ é pur un poto, Ser :
to Latini nel Pataffio. lovee for pata |) me i
GLA da lo {pruRzole . Gli {puta il vino nel vilo a minute fille . Sprags j
ciamo quando comincia a piovere minutamente, onde Spragzaglia offerud il Vet-
tori dirfi da’ contadini una piccola quantita di ope per fimilicudines
7 -AROCC A. Entra in collora ; arrabbia.. Voce ufata in Firenze, e in
Lombardia . Francefco Negri nel fuo Taflo in lingua Bolognefe, portan :
quello il verfo d’ un’ argumento , che dice 4/ Re fi turba alla novela rea, pari
4 Re al fente ,¢ ¢ minza a taruccar ,
SRONTOLARE. E un rammaricarfi , o dolerfi di qualche foprufo, o finifiro
vvenimento con parole non affatto efprefle , ma confule , ¢ male articolate, ¢
fra i denti, che fi dice anche bofenchiare ; ( Nella Valdinievole meer
to il calabrone ) Viene per avventura dal Greco Bronean , che yuol dir P
Virg. in quel verfo , ove nomina i Ciclopi affaccendati a lavorare il ferro, ¢:
mini nella fucina di Vulcano. Bronte/que, Sterope/que © nudus membra y
I primo nome lo cava dal tuono, il fecondo dal folgore , il terzo dall’ ancudine,
¢ dal fuoco , ~
TIRA foto, Attende , continova , (eguita a fare quella tal cola. Si9
DAGLI , ¢ tocca , Quefto termine fignifica , fa , ¢ rifa la tal cofa, ovvero pre:
£2, ¢ riprega ; ¢ fi dice Dagii , picchia , € rocca. Ovvero Dagli , toca, valle
martella , i
MEZZO cotto. Quafi briaco. Vedi fopra C. 6. ft, 35. {
CHE lo trovo non era un'oca . Chi lo trovd non era huomo fenza ceruello, ma
un valent’ huomo . Ceruel d’ oca , 0 capo d’ oca yuol dir huomo di poco giudi-

zio.
STANZA XVII, STANZA XIX,

Wit rn

 
  
  

Che {cufa non gli pare baver, che vaglia,
Che non gli fea a viltade attribuito ;
Cosi ribeve un colpettino , ¢ in cambio
D' andar. 4 lett s'arma,e piglia Pabio,

STANZA XVIIL

Senza lume, ne luce wia spulerza,

E corre al buio,che ne anche il vento
Non ha para mica della brezza ,
Perch’ egli ba in corpo chi lavora dréto;
Per la mora fi ben fi (candolexa ,

Che dando il ¢, ,interraacgni mométo,
Quanto pik cafca,e nella méma pefca,
Tanto pin fentech’elt’? molle, Puta ‘

Poiche dai cibo,e da quel vin che fmaglia, Dopo cht ei fu cafcato ye vicafcatly
Si fente tutto quanto ingazzullito y Ler non fentir quel mollee fre{coacert,
Rifolue ritornar alla bactaglia , Che'lvino,e quato diatiavea i
Donde innocentemente s' ¢ partite , Opra di dentro si, ma non di fuera,

Giiito al mulin dal mexzin giis sbracciate
Si (ciaguatta i calzoni in quella gore
Per dopo nella cafa di quel loco
Farfegli tutti rafciugar al foto,
STANZA XX,
Mentre fi china dando ilc,., a leva;
Ei fece us capitombolo nelacqua -
Ona’ avvien ch'una voltacilacquabews
Sopra delvinyche maiper
Quanto di buon fi ¢; cheisei vplens
Lavar i pauni,jlcorpo anche rifeiacgisy
E divien lacqua si feremexegiallay
Gh’ i pefci vengon sutti quanti a gall

—

Be GBB PRR E S3SsS> Pe

=

ee en EC! . Be FE we 2 ares

 

 
 

SETTIMO CANTARE 333
BiwbineeDAN ZA XXL Hoi

 
  
   
   
    

tutte a lui fon note, Lotanto fi conduce fra le ruote ,
ne per muotar bene sl Romano ;. Che fan girando macinare il grano
ei corpo, confie fa le gore, Ben fen' avvede, e gid mette a entrata
Anna/pa cof piede,e con la mano, Di macinarfi,e fare una fRiacciata,

wide fentendofi inuigorito rifoluette di ritornare al campo ; ¢ cosi (enz’ altro
fi mefie in viaggio , ma fendofi infangato , volle lavare i calzoni in una go-
_ 4, €vicalcd dentro , ¢ fe bene egli fapeva nuotare, ¢ s’ affaticava per ulcir dell”
‘Acqua , tuttavia conobbe , che portava pericolo d’ entrar forto le ruote del muli-
no, ¢ reftarui infranto , fe non gli accadeva quello , che fentiremo appreflo .
» VINO che fmaglia, Vino potente , ¢ generolo . Si dice /magliare , perché il vi-
_ no nel mefcerfi nel bicchiere lafcia nella fuperficie una ftummia , che fa certe cofe
q come maglic » le quali il vino gencrofo rode , ¢ confuma fubito; e quefto disfac
que ie fi dice /magliare , ¢ quando non le disfa , ¢ fegno , che ha poco {pi-
ya ito. Edi quiicicchi hanno un detto : Laloccom! io, 0 vommene ? ed intendono
¢0si didomandar al compagno alluminato , i] quale ha mefciuto nel bicchiere, fe
quella fummia fe ne va , o fi trattiene , ed in confeguenza s'il vino ¢ buono, o
tattivo.,, Lafca Nov, 4. fecero uno /cotto regio con quel vino, che /magliava,
AINGAZZVLLITO. Forfe meglio ingazzurlito. Vuol dir rinuigorito , rin-
» 0 rallegrato di quella allegrezza , che mette addoffo il buon vino.
i dice entrar in xurlo , 0 in zurro 5 corrottamente da rvzzo , ¢ quefto dal Latino

 

 

&

ruere,

ANNOCENTEMENTE # ¢ partito, Dice innocentemente, perché in vero Pa-
tide non haveva errato a partirfi dal campo , poiché n’ era ftato cavato da colo-
TO ,chelo portavano via infermo , come s’é detto fopra C. 3. ft. 25.

YN colpettino . Vn' altra volta, Vn' altro poco. 1 Franzefi fimilmente dicono
per efempio ; boire encore un cowp . Bere un' altra volta. Provarfi a bere un’ altro
poco. Ad é traslato dal provarfi in gioftra .

RIGLIAR ? ambio, Andarfene . Voce corrotta da ambulo latino , che yuol dir
andare, o pur vien da amb io {pecie d’ andatura di cavallo , con altro nome detto
Portame., pecché per efprimere andavfene diciamo Pigtiare il portante ,

SENZA lume , ne luce, Afiatto albuio . Senza lume terreno , ¢ fenza fplen-
dor celeite..

SPFLEZZ.A. Va via furiofamentc. Parmi che poffa venire da (pulare il gra-
no ,che i) vento furiofamente porta via la pula , cioé i gufci del grano; 0 das

pigliare il puleggio detto fopra C. 1. ft. 80. *
_ “OTA, Terra inguppata nell’ acqua , ¢ ridotta quafi liquida . Cosi appreffo
i Franzefi moire & il Latino dus, madidus , ¢ quel che noi diremmo mole ,
pA MEMMA , 0 melma. Quella terra , che nel fondo:de’ fiumi , foffi , laghi, ¢ ,
ae » tidotta liquida , che la diciamo anche bellerta per me/metta Latino Limus
_  Verifimilmente dal Greco Adigma , che vuol dire miffura .
: a. Meflo in corpo. Detto plebeo. Vedi fopra la voce Gubbia-
mC, 1, ft 36.
DA'mexxo in gilt sbracciato . Cosi dice per (cherzo , fapendo bene che sbracciato
Significa ,quand’ upo tirando Ja manica in fu fino al gomito,la(cia ignuda quella, i
‘ parte

ea

SSBB SE”

 

 
 

,

 

 
   
 
     
    

334

arte del braccio , ¢ non quand’ uno fi cava i calzoni’, co!

aride , il che fi dice sbracato ; ma-l' Autore fi ferue della voce
tendere {pogliato ; enon é-vero che habbia a dire sbracato
corretto , non folo perché I’ originale di mano dell’ Autore,che
ed in un {uo primo sbozzo dice sbracciato, ma anche perché fed
mezzo ix giit ' intenderebbe che ei fi fufle tirato fu i calzoni fino a
enon che fe gli fufle affatto cavati , come era neceflario , che ,
voleva lavargli. ¢
SCLAGV-ATT ARE , Dimenare un panno , o altro fimile nell’ acqua.
GORA. Vuol dire un canale d' acqua , che corre , ¢ propriamente s*
quella fofia , per la quale fi conduce |*acqua a i mulini per macinare
ali fofie , o gore fi fanno a quei mulini , che {ono in fa’ rivi,o p
quali ¢ feacfita d’ acqua , non effendo neceflaric a i fiumi reali , nei:
{erui abbondanza d’ acqua , bafta un foftegno , o fteccaia ( che no
fcaia ) che volti l'acqua al mulino, ¢ ferua per Cola, chet ae
alla quale fi raguna tutta I acqua, che porta la gora . Gli antichi finiv
voci in Ora non folumeute quelle ,che aveano fimili udi se col Lat, co
le quattro tempora , come ancor oggi diciamo ; mia anche le Bergora ,
le Campora ; E fimili. Onde i] Sannazzaro nelie Ecloghe della {ua A
fe licenza di dire Pratora per Prats: Gc. Si pote dunque dare benifiimo:
quef’ acque cosi ragunate effi chiamaflero Lacora dal Lat. Jacus , € poi
a ftaccare la voce , ¢ dirfi La gora . Da i Jatini fi trova effer tali, o fi
d’ acqua chiamati Euripi , e dVili , ma credo che fuflero iperboliche a
come fi pud dedurre da Cic, 2. de legibus , dove dice ; Dattus ag
Wilos , Enripofque vocant quis non irriferit ? E veramente ¢ cofa da rid
Euripus é nno ftretto di Mare,ove é il fluflo , ¢ reflufio ; Ed il Nilo ¢ de’
ri fiumi del Mondo; E quette fon fofle femplici , e laghetti , che gli
mani fecero correre infino di vino in occafione di fefte; e da cid piglio
to , che gli adulatori per piacere a’ Signori , le chiamaffero Wii, ed

DANDO ile... aleva, Cioeaizando ile ....ed abbafiando il ca:

FECE un capitombolo , Rivolto il corpo ful capo fottofopra ; fece un
capo , rivoltandofi foctofopra. Vedi C. 6. ftan. 84. ‘

e4 GALLA, Nella fuperficie dell’ acqua. Dai verbo galleggiare.,
origine da galle , che fono quelle leggieriffime palle , che nafcono dalle
donde /eggieri , com’ una galla,

JL Romano, Fu uno Stufaiolo,che infegnava nuotare alla gioventh Fiorentit
MOLTO annajpa. Annafpare vuol dir mettere il filato fopr’ all’ alpo”
durre i) filo in matafle , edipanare , Lat, géomerare ,afhne d’ adattarlo
re, dal Greco ana/pan , che vale retrabere , revellere, E da quefto qu
perde molto tempo a far qualche operazione ,¢ non conchiude cofadi bi

ciamo annafpare . Qui vuol dire,che egli muoveva i piedi ,.¢ le mani,

ve le maui colui che annafpa ;¢ fi puo anche intendere che armeggiava |

nafpava molto , ¢ conchiudeva poco . eal
GLA mette aentrata di far una ftiacciata , Gia tien per certo d’ havere@

infranto dalle ruote del mulino , I caflieri , ed ogn’ altro che tenga libri

%

 
    
   
  
 
 
 
   

   
 
   
 
 
     
    
    

Tg eee ee ae oe ee

 
      
 
  
   

   
 
 

es ES ee
    
 
     
   
  

  

_SETTIMO CANTARE: 335
: trata ,¢ ufcita,mette a entrata , quando ha ricevuto il denaro ; € da quefto noi

mo Tien per certo , o ha gia per ricevuta quella tal cofa.
Stearn ¢ a ccsatabuperr
» che il mefchin gra fi prefume Ognun fi tenga pure il [uo parere ;
andar a far la cena alle feanae 's O quelle , 0 altre, a me non fa farina,
¢ una porta,e in chiaro (ume Baftini per adeffo di fapere ,
ise 2 iar conocchie, Che quefte non fon beftie da dozrina ;
Che le Naiadi Ninfe di quel fume , E, 8’ ella non m' ¢ feata data a bere,
—— Coronate di giunchi ,¢ di Elle fon Fate c’ han virti divina ,
—— Corrono ad aintarlo infin c a viva E che fia il vero , fede ve ne faccia
La dove il di riluce) in falnoarrina, Li Garani feampato dalla fliaccia .
STANZA XXIII. STANZA XXV.
— Bvede all ombra di falcigne frafche 11 quale cost molle , ¢ sbraculato
io Fra le pit brave mufiche acquavle MU cadavero par di Mona becca,
&s == Parte di loro al fuon di bergama/che, Ch’ effendo ftato allor difatterrata ,
we » ¢fefhe ragliar le capriole , Hlabbia fatto alla morte una cilecca ;

ei Chitien che quefte Ninfe fien le lafche
[Chile firene , ed altri le cagznole ;
4 _ Lonon fo chi di lor dia pin nel buono ,
i Ble lafcio nel grado , ch’ elle fono ,

is Ah % $

jd Male Fate , che (pecie fon di pefce,

ys Edbimoilcorpo aftar nell'acqua avvexzo
st Piiche 2 a bagnate a lor rincrefce,

—, cos: fradicio merz0 ;

Si [quote ,¢ trema si,ch’ io ho fopparo
Per San Giovanni il carro della Zecca,
Ementr’ ei ff debatee, e il capo feroiia,
UI pavimento ,¢ i circofhanti ammolla,

TANZA XXxVI.

Percio lo {poglian ; ma perch? riefce,

Quéido un vuol far pits prefpo,fhar un pexro,
Pertrattenerlo(mentr'hor quefia,bor quelia
L) afciuga) una conto queita novella,

Paride ftava con timor d’ affogare , fu foccorfo da alcune Ninfe', les

meffero a {pogliarlo , ed intanto una di loro contd 1a novella , che vedremo

of

; :

3 lo cavarono dell’ acqua , ¢ lo conduffero alle loro ftanze , dove dette Ninfe
af

MESCHINO , lafelice ; Povero. E voce , che denota commiferazione .
ge ANDAR a far la cena de’ ranocchi , Ciot affogare , annegare , ¢ cosi diven-

tar cibo:de’ ranocchi .

yi CONOCCH/z, Pennecchi in falla rocca , che fono quei rinuolti di lino , o
é Jana , 0 altra materia fimile , che Je donne per filarla accomodano in fulla‘rocca
to da effe ufato per filare ; Voce corrotta da cannocchie , fecondo i] Fer-

rati , perché-le rocche per Jo pid (ono di canna; Il Voflio la fa venire dal Lar.

, Golus ; quaGi Rorpiata da colwcula.
i

~DRAPPL, Ciok quei drappi da donna , che dicemmo fopra C. 6. ftan. 9.
5 ~~ CAMPEGGIAR conocchie, Sappotto che Je muca di quelle ftanze fuffono bian-

che,ogni cofa di

alfivoglia colore vi fi difcerne ben fopra , e perd ( feruendoti

; del verbo pittorefco campeggiare ) intende ; fi diftinguevano fopr’a quel bianco i

,  drappi ,

¢ fuentolavano , ¢ le rocche appiccate alle muraglie ,

GIFNCO, Pianta , 0 virgulto noto , che nafce vicino all’ acque , ed in luoghi
| Umidi , ¢ padulofi, ¢ non fa foglie , ne tronchi;ma falti,come paglia,li(ci , (en-
E 2a nodi , fe non uno ia vette , dove nalce il feme. E per quelto habbiamo ua,

 

pro-

 
     
    
       
  

336 | MALMANTILE * ©
proverbio,che dice : Cercar if nodo-in ful ginnco ; Lat, sodum in fcpe ghree
fignifica cercar le difficalta , dove elle non fono. V2 POT en

PANNOCCHIE , Spighe, che fi producono dalle canne, dalla ei

panico, &c, dal Latino Panicu/a , voce ufata ‘da Plinio y ove tratta
Carerum gracilsas nodis difpintta leui faftigio renmatur in Caxmina ,
la coma, ; HeeO SNS

SALCIGNE frafehe, Frondi di falcio albero noto,che nafce ; € vien pil
rofo in luoghi padulofi, Lat. frondes /aligna . a

eAL fuon di bergama(che. Chiamiamo Bergama(ca’ un*ballo compone’ rato
di falti , e capriole , e-perd dice guinre , ¢ fefpe rrgliar le-capriale ; vest

CHZZVOLE , Sono certi animaletti neri,che vivodo sae :
ti pancia , ¢ coda, ¢ col teinpd diventano ranocchie , € met 1¢ gambe,
cafcado loro la coda,mutano colore di nero in verde macchiatoje
mo la meftola da muratori ; Lac, trudla , ¢ che It Abate Baldo da Vebi
zionario fopra Vitruvio dice al fuo paefe chiamarfi Cacchiara,

LE lafcio nel grado ch elle fono, Sieno chi elle fi voglionojio non do’
nome , che un’ altro ; perché cid zon fa fartmarcio’ non m’ importa ; eT
propofito mio. E qui l’ Autore moftra d’ haver notizia delle diverfe
Gentili circa alle Ninfe , le quali tutti concordano effer Figliuole
¢ conchiudono che le piii fuflero Deita aquatiche ; le quali Deita noi”

retiamo, che ficno diverfi effetti, che produce Pumidita, E che parte
infe fieno de i prati , parte de’ bofchi , parte de i monti , ¢ con diverf
Nereidi , Napee , Oreadi 5 ec, : ‘

NON fon beftre da dozzina, Non fon beftie ordinarie , ¢ da farne poca Mima:
Diciamo cofa da doxzina , 0 doxxinale, quella, che ¢ lontana dalla perfe
che @ lavorata con poca diligenza . s

S' ELLA non m’ ¢ frata data a bere, S ella non m’ é ftata data a credere] *

FATE. Vedi fopra C, 4. flan. 54. ‘

STIACCTA, Si dice quella trappola , che fi tende con le laftre a i topi,ed ag
uccelli , cosi detta , perche nel cadere addofio all’ animale, Jo ftiaccia .
SBRACVLATO. Senza brache , e fenza calzoni . 2
C-ADAVERO di Mona Checca, Si (uole in Firenze nel giorno della commen
razione di tutti i morti,ne i fotterranei della Bafilica di $. Lorenzo , che fond il
fepoltuario , efporre uno {cheletro di morto con veli in tea’, ed altri :
menti , ¢ quefto da i ragazzi é detto Atona Checca; cioe Afadonna Fr.
quefto nome poi comuncmente s’ ufa per efprimere uno sbattuto , ed -afflitto dale
la fame , dal freddo , ¢ da altro ftento. Ariftofane portato in Latino dice: AF
bil a Charephonte differ . ww
FARE nna cilecca , 0 feilecca , Far una burla; ciot finger di voler 1
fa , ¢ poi non Ja fare ; Sicché vuol dire: habbia finto d’ efler morto’, &
fia ftato vero. Habbia gabbato la morte. Diciamo anche pare uit m b
rato . Ll Bini nel fecondo Capitolo dell’ orto dice : 2 og ae
' Ho una vaca , ma ell! ba una pecca 1 EY
D! un certo fuo turacciol benedetto ,
C’ ogni volta mi fa qualche cilecca ,

 

  

REF FS ee 8a Ea ee

2.

 

Er

eee ae oe ee a oe

 
 
  
   

ee
1) TP Sr TIMOsCuNTUReE 337
20.

So fips. Qui bao fleffo fignificato , ché mb difetadé detto fopra. 1. 0.
34. ¢C. 6, flan. 61. er altro havere ffoppato uno, vuol dire H2-
negli orecchi , ec. per efempio . Tu mi hai fatto il feruizio tanto tardi , che
ho havuto pid bifogno , ¢ perd io ¢* be fropparo .
lella Zecca , 1) giorno di S. Giovanbatifta ¢ la maggior folennita , che
ri in Firenze per efser del Santo Avvocatu y € Protettore della Citta, ed
oP ome agifteati di Firenze , e tutte le Terre, ¢ Caftella fubordina-
; inio fanno Ja citimonia dell’ offerta al Tempio dedicate al detto Santo,
fra gli altri il Magiftrato delia Zecca offerifce un gran Carro trionfale in figu-
piramidale alto circa 20. braccia , ¢ nella fommita di effo Carro é un’ huomo
tutto coperto di peli , legato.con func a un palo di ferro alto circa un brac-
cio ¢ mezzo , che formando in cima un mezzo circolo gli fafcia lo flomaco,dove
f detto huomo,accid non cachi , il quale rapprefenta San Giovanni nel
to. E perché tal:Carro nell eflere ftra(cicato brandifce , ¢ {quote , perd
che ¢ nella cima del Carros’ agica grandemente ancor’ egli; Ed il Poeca
huomo intende dicendo,che Paride fi {quote pid del Carro della Zecca ,
Colui,che é fopra detto Carro*
CB yO inerefce. Vuol dir venire a noia , 0 a faftidio , ed 8 i! Latino
Bocce, gior.5. Nov. 6. lo fari s} 5 che lavedrai tanto , che ella ti increfeerd .
gnilica haver difpiacere,c’ una cofa fia fatta,, o non fatta. Bocce. Nov. detta.
Ma di cidyche facto , glincrebbe, Significa compaflionare uno , come nel pre-
ee eis quefto C, flan. 50. Significa‘ancora haver difpiacere inten-
dendofi eflerinelle Fate maggiore la compaflione , che havevano di Paride per ve-
derlo cosi mal condotto , che non era il difgufto d’ effer bagnate ; E fono quefti
due fignificati tanto profimi ; che fpeffo col folo verbo rincrefcere s* e(prime.»
Punoe Valtro, come fegae qui , ¢ nel Petr. Son. 44.
dP ote Onde il lafciare el’ 4/pettar m’ increfee,
Rs a intendere mi pefa, mi difpiace il lafciare , ¢ mi viene a noia I’ al-
pettare, Li Perfiani nelia lettera al Sig. Principe D.Lor. diffe :
Ml mio bifegno ho gra detto a parecehi
(art ‘s) EB ciafcun fe ne duole , ¢ gli rincrefce
4Omexo . Coml’, ¢ , firetta , ¢ con una fola ,z, che fa alpro ( per-
ché:con Pelarga., ¢ con due zete,che fanno dolce , fecondo I’ opinione del dot-
4 . Carlo Dati , vuol dire meta ) fignifica bagnato affai ; ¢ la voce fradi.
¢io che wuol dire corrotto , qui fignitica inzuppato d’ acqua. La voce mezo vuol
dire una:cofa tenera per efler troppo matura , come farebbe una mela o pera, ec.
vedi fopra C, 3; ftan. $3.0 una cola intenerita per haver inzuppato molto umi-
do una {pugna intinta nell’ acqua , e quefto ¢ il fenfo de! prefentes
hiogo:.. Adezo & dal Lat, mitis per maturo; ed € il contrario di acerbo , che cosi
chiamiamo Ja:frutca‘n6 per anco matura.V olgarizzamento antico di Palladio,nel
mefe di Gennaio; ‘tit, 15. Serbanfi le forbe:, fe fi colgano dure ; ec. ¢ ivi comin-
)  Glanfigimmerzare . Ul Lat, dice: wbi mitefcere caperint, —

       
 

 
 
 
 
  
   
 
 

      
 

    
  
 
 
    
 
  

 

2
.
"

= BBtesei ES

3
a
7

=

xe x vy STAN.

os

 

 
'
i

 

338 MALMANTILE ~~

STANZA XXVIL
Fro un tratto una dama,eun Cavaliero
Moglie,e Marito in buono,e ricco (Pato,
Che fatti vecchi , contr’ ognipenfiero 5
Dopo a’ haver qualche anno ‘higare
La grinza pelle con il cimitero
Conuenne loro al fin perdere sl piato'y
E fenz’ appello haver a far propofito
Di dar per ficw tal’ offa in depofito,
STANZA XXVIIL
Lufciaron due Figlinoli é pik compliti
Che'l mondahaveffe mai {ule /ue fceney
Perch’ effi havevan tutti srequifiti
Donati aungalat'hnomo,e un bom dabbene;
Aggiunto che di folds eran.gremui ,
(Che quefto infomac quel che vale,e tiene)
Stavan d' accordo , in pace,ed inamore,
Er eran pane ye cacio , anima,e core,

gidi non ufa pil.
PIATO ye piatire.

 
 

 

Perciocche il noftro
gis habent vigorem ,

v

potevano piatire per La lor

La Fata principié a contare la novella (1a quale ¢,tolta'da:lo Cunto deli
ti gior. 4. Cunto 9, ¢ gior. 5. Cunto 9, ) ¢ dice : Furon gia unadama, ¢ ut
valicro marito , ¢ moglie , i quali venendo a morte lafciarono due Figlit my
coftumati , ¢ ricchi , i quali s’ amavano grandemente I' un I’ altro. Qui
fa wna digreffione , ¢ confidera , che quefto modo di trattacfi fra i Fi

Lite , 0 litigare d’ avanti a’ tribunali , detto dal Latybar+
baro placitum per lite ,¢ placitare;la qual voce ritengono bella ¢ intera i Veneaia-
ni, Placitum é il decreto, fentenza del giudice , 0 Magifttato, e quel che i Fran-
z¢fi dicono 4rreffo {econdo il Budeo da arefeein , che in Greco vuol dire placere .
Ne’ Senatuscon(ulti, ovvero Decreti , ¢ Sentenze. del Senato di Roma ulayand
guefta formula : Sexatui placere &c. come fi ricava da Cicerone Filippica 3. € 5+
Nell’ Ordinanze Regie in Francia fi legge fempre in fine: Car tel eff noftre plaifin
iacere é tale. EB nella legge fi dice ; che Principam le

enne poi da’ Latini bafii a tirarfi quefta parola'a fe
il proceffo della lite medefima., fi come anche éudiciwm fignifica Ja’ fentenxa jt
4a lite medefima , che fa nalcere la fentenza. Piatire lo Spagauolo dice;
Franzefe plaider ; wutti dail’ iftefla fonte Latina, Il Doni.nel fuo Cancelli
Sempre ne piati la rovina va innanzi , ¢ chi piatifce ha quant’ ei vysle il
Ed il Varchi Sc, Fior, lib, 14. Erano affegnate le canfe delle pouere
ertd.. E poco appreflo, dice ; Perche
care quel piato al terxo pofefore. Ed in quett ultimi verfi della prefente:
27, dice metaforicamente , che.a coftoro gia fatti vecchi dopo haver fatta

  
   
 
 
 
 
  
 
 
 
 
 
 
 
 
 
  
   

apt eh gS
Ce fare i es
ca é neces

E fr as de cei osre

ll contrariocoftor di chi io favelle
1 quai di cortefia furon due fpe

E trattavan ciafcun da buon Prac
5S’ haurebbon

    
    
  
    
   
 
 
        
  
 

   

il
pene chee
bifegnsos. mi

 

  

SS = 61 PERERA EES eeeeeeee

rar lungo tempo la loro carne a i (epolcri., conuenne morire , ¢ fart

   

Il proverbio piatire i cimiseri vuol dire Effer d’ eta cadente , che Luciano portal
in Latino dice : lterum pedem fepulero , 0 vero in cymba Charontis habere ; Cs

noi pure diciamo ; Havere il pit fu la bara 0 vero il pid nella foffa, —

    
 

 

GA
 

 
  
  
  
  

SETTIMO’CAN'TPARE. 339
GALANT" nemo yed huomo dabbene. Si pofion dir finonimi ; ma Mrettamente
galant’ buomo yuol dire huomo di garbo , ¢ come dicono i Franzeli , ones" uomo ;
oltre acid amorevole , ed alla mano , ed huomo dabbene yuol dire huomo di co-
»huomo d’ anima , ¢ che fa opere buone . Spagn, hombre de bien . L’ uno
I altro comprendono i Greci colla fola parola Caofcagathos . Calos ignitica. ,
adig eh buono , da bene. “y
GR. 4. Ripieni . Bil latino Spifus .. Denfus... E qui vuol dire havevano

 

 

pieno di frutti , un luogo pieno di mofche, o fimili ; perché tal voce 4 do-
ufare in quelle occafioni , nelle quali cade la fimilitudine del proprio di
Hy « Greto-yuol dire terreno’ghiaiofo , ¢ pieno di faili , come {ogliono ri-
S tive de i noftri tinmi , {colata che ¢ I’ acqua piovana , quali rive perd
og — chi )Greto 5 come greto d’arno , greto di mugnone , ec, Ora Grero addict.

fies no di danari ; fe bene ¢ detto improprio,perché gremito s' intende un’

dice i) Vocabolario della Crufca , /o diciamo in fignificato di {peo ; forfe daila
titudine [pela de’ fuffi de’ greti;e diciamo anche in queffo fignificats Gremito. Quan.
ame ,inclinerei a credere,che Gremity dal dirli propriamente degli alberi,quan-
‘ono ‘picni di fiors 5.0 carichi di frutta , venitic da Greminm perciocche if
quella parte , che fuole empicrfi di tali cofe. Gli antichi Volgarizza-
che i Latins diflero diteus eth traduilero greto; laonde potrebbe ad alcuno
 quefta ia fatta da quella . Seneca epilt, 15; Liles repersi en littore cal-
oie ao aouisiem amet delettant c Panciulli & sthcesanc in cofe di
piccol pregio , fi come.tono pictre , che |’ huomo truova nel viaggio , € uel gresv
del mare, ¢ ne’ fiumi. Palladio nc] Gennaio tit. 14. favellando della lattuga .
Candida fieri putancur, fi fiuminis arena , vel litoris frequenter [pargatur in medias ,
rm E pofiono diventare bianche fe entra loro , ¢ intra le loro foglie {pefle volte fi
fpatga rena del fiume , 0 del greto , Qnde a dire gremito di foldi s' invenderebbe_,
= hi fopra il veftito ,o fopr’ alia perfona {parfo gran numero di foldi ,
i somMeenemito di mo/che 8 intende haver molce mofche addofio , € non nella tafea,
i" Oinicafla.; Tuttavia, fc bene. improprio, é alle volte ulato , come qui,
} , ESSER pane , ¢ cacio 5 anima ,e cuore, Andar’ uniti, ¢ d' accordo in ogni ope-
A razione, Bene conmeniunt , © in una fede morantur , .
wht iy oterra al Sole, Se hanno mafierizie, 0 poderi ; per efprimere,
6 uno che. i peep roba diciamo: // tale ba quattro cenct, © fe ha beni ftabili
in terreni: Egli ha della terra al Sole,
» SLAMO di si perfida cottoia, Siamo cosi iniqui , ¢ di mal animo , Quei legu.
Mi, che per moito che fi tengano al fuoco non fi quocono , ne inteaeri(cono
Mai, fidicono di cattiva cotteia  € perd con dire huomo di catriva cottoia , §? in-
tende di genio maligno ,¢ difficile a perfuaderfi al bene. Gr, ateramon ,
~ ESSER al dumcine , Vuol dire eller in cftremo di vita ; € vicue dail’ ufo, che &
nello Spedale di S, Maria Nuova di mettere uo piccolo lume a ua Crocititio al
Jetto di coloro , che fono agonizzanti. Si dice an.ora ; cfier alla candela ,
) NON gli fovverrebbon d’ un tnpine, Non gli darebbono un minimo aiuto. Sov.
yenire neutro yuol dir covalent. Non mi (ovviene , quando fu queftlo . Non mi
eieanam fugqueito. Lat, mentem /ubire in mentem venire , fuccurrere, Fr,
(e fovvenir 5 2

Vv2 " eHOz.

oe

Seba thasa

STEED

  

et

=—s=

o

 

 
 

 
 
  

  
    
   
    
 
   
      
 
 
 
 
      
 
 
 
 
  

340 MALMANTILE) |

1
HOZZORECC HI. Huomo feelleratoyed infame : EB quefto,perch? {
fattori,che per la tenera eta fono elenti dalla ordinari: i
fiizia contraflegnati , come dicemmo fopra C.2. flan. 3. ¢ C. 6. ftan, $4. ¢ fra
gli altri contrafiegni uno é il mozzar loro una parte degli arecchi,. \. mye
. LOKT AR acqua per orecchi, Fare a uno watt i feruizzi i

HAVBEBBON volute indovinare. Quefto termine e(prime la
che uno ha in feruir I’ altro, ¢ compiacerli in tutto Veen .
STANZA XAXL sT ZA XXXL
Effendo un giorno infiemo a um conuito y E tutti quei che feggon quiviamenfa
uad’appanto agurzato hinoil mulino, Lferui 5 i circoftanti,ed f
4 mangian con buoniffima appetite 5 i
4Von focome il maggtore dette Nardino
Nell’ affettar il pan taglioffi um dito,
51 ch’ egli infanguind st tovagtioline
E parwegli si bello a quel mo intrifo,

  

Ch ei fi pofe 4 guardarlo fifo fife. Lfangue:
STANZA XxXIl, STANZA XXXIV,
E refta a feder li tutsa infenfato , Che gli par di veder ymentre in

CL! ei par di legno anch'er come la fedia, i for ver
Luo is ( tanto nel wifo é dilavato ) qualche Dea di Ciele
Con la tovaglia i fimili in commedia Compofta colafsis di rofe , ¢, Z
E mirando quel panno infanguinato E si gli piace: y.¢ tanta ght
Hor mai tant! allegria mutain tragedia, Che finalmente mentre ch’
AMeatre nel pin bel fuon delle fcodelle Vna moglie d’ un tal componimene
Si vede — ripofar le mafeelle, ‘Non fad de i fuci di mai pil contente,
Edendo gli faddetti giovani a un conuito, Nardino , che era.il maggiore,afiet-
tando il pane,fi taglio un dito , ed infanguind il tovagliolino , e nel mirae quel
bel roflo in ful bianco , s’ innamoré in maniera,che fi propofe di non haver mai
a reflar confolato , s’ ci non piglava una moglic compotta di quel colore del 10+
vagliolino infanguinato , 5 nhy
CONMITO. Definare , 0 cena {plendida , Dal latino Consivixm, 0 c
da Conuitare nel fenfo che gli Spagnuoli pigliano il loro. Combidar , per fuaicare ,
¢ nel quale il prefe il Boccaccio , che difie , Commie « mangiare:,. E, Conuirati alt
ravole , ii B
AGVZ ATO il muline.. All’ ordine con la fame per mangiare.. Cosh eratta lt
fimilitudine dal mulino; dicefi Adacinarea due palmenté , c10e mulini ; di ¢
preftezza , © voracita maftica da amendue i lath aun. tranoy
itanza 22. > ¢ ORE ong B
APPETITO. Vuol dir. appetenza , ¢ defiderio im generaley»ma
detto aflolutamente ,¢ {en2? aggiunta,vuol dir Fame © voglia 4 0) *
giare. Vedi fopra C. 4. tt. 8, # mal che viene in bocca allagalina, > 2 sb Oi
TOV-AG LIOLINO . Quafi piccola tovaglia . Quel pezzo di panne line
tiene avanti,quando fi mangia etiendo a menfa , Boccaccio difle
Jo dichiamo anche falwietta dalla voce Spagnuola. Seruillera , perché ferue mol
al minificro , ¢ al {cruizio della tavola, hae Be = HERR,

og

 
  
 
  
 
  
 
   
 
   
  
  
  
 
  
  
 
 
  
      

#&eFRESELLCFE - Pee stsiz=

Tr

gg #E 2 se FZ.

=

2 ge (Ba
  
  
        

 

 

SETTIMO CANTARE: 34

INT RISO » La poluere } © altra materia fimile ftemperata con liquore , come
e:farii Wa fidice:imtrifo , ¢ intridere Ma fignifica ancora imbrat-

| tao, [porcato , ec. come fignifica in quefto |
PISO fifo. ‘Senza batter’ oechio 5 wha greta attenzione: dntentis , inconni-

eculis . 1 Greci dicono in una parola e4/cardamytti , che @ lo fteflo che:

 

 

{ o irca’y

8 28S Cash vede/sio fifo,
BSB NG Come Amor dolcemente gli governa
Seles Sol un giorno da prefso ,

2m 8 Senza volger giamai rota fuperna ,'

Bp%b-on pers | We penfaffi a’ alerui , ne di me Steff ,

 Purcdiwg sci El batter eli occhi miei non fuffe [peffo.

- DILAV. idito . Smorto . Si dice dilavato ogni colore , che nons

‘ “ATO ; Impallidi

-ariva alla perfezione della fua effenza: come roffo dé/avato fi dice un color roto,
‘che fia pitt sbiancato ,e pity'chiaro del vero roflo . Latino difurus .

PPO far con la tovagiia i fimili in commedia , Intende ch’ egli ¢ bianco appunto
come? latovaglias Latino’ non oxnm: fic ouo fimile, I due fimili ® wn fuggetto di

', come quello de Menechmi di Plauto, a molti vi hanno (cherzato,per-
 fecondo d’ intrecci .
A, Specie di tela lina fatta a un’ opera , che fi chiama renfa, detta cosi

dalla Citta'di Rein Francia . Cosi ‘Perpignano forta di panno dalla Citta della.
Navarra di quefto nome . e4razzi dalla Citta d’ Arras in Fiandra’:¢ Dxagio al

) Boccaccio fi diceva un panno,che veniva di Dovay Citta di-Fiandra. ,
che Gio; Villani fecondo |’ ufo de’ fuoi tempi, chiama Doagio. Latino Duacum .
Baldacchino j deappo di Levante; da Babbillonia, che i Levantini chiamano 24-
g4ady inoftrivrantichi Ba/dacco, Gio: Villani |. 7. E'me/so fuori delta Citta , fopra
(a {ua perfona umricco palio di Baldacchini di seta ed’ oro,

LENZA ,olenfa, Lat, tinea, filum pifcatorinm , detta cosi quafi dal Latino
linea, Quelia’cordicelia fatta di crini di cavallo , o di feta cruda , con 1a quale
filegaitiamo da petcare:» Franco Sacchetti Nov. 163. Egii haves prefo P allumi-
white ale lente acfcandole con 200, Fiorini a’ ord, Lalca Nov. 166. Fau'nn pefcatore

i puceote i e/cando con lami, e con lenze,

* Wid oer. Acetta . Pezo di tela ia larghezza det fuo effere, € lungheza

4d libitum ; come un telo di lenzuolo , 6 di paramento sdrucito in tutta la lun-
ta#di effo lenzuolo , 0 paramento , Diciamo ref da pane quella rovaglietra,

O.trifcia'di panno lino, con la quale fi cuopre i} pane in fu !’ afle’, Qui intende

iktovaglinolo . Te/econ I", ¢ slargo ufato da alcuni in Poefia , vuol dire il dar-

do. Lat. telam . .

“'GLfvaapelo, Glivaa genio. Se gli confa: ¢ fecondo il {uo gufto; él) op-

pofto:d” contrappelo detto fopra C. 6, itan. r. }

"2 a ete mine XKXV,. 2
da figura nel penfiero, E come chet la vegga daddovero
ieee paket, eas Divoto fe le inchina y € le favelld ,
Co’ {uci capelli-d’ oro , ¢ t* occhio nero, E le promette, # egli haurd moneta,

“Che pit ne men la matturing frella . Di pagarie la fiera al? Lmproneca,
: STAN.

 

ig”

 
 

 

 
 
 
    

34%
E vuol mandarle il cuore in un pafticcio y
Perch’ sila fe ne ferua a colarione;
E gli s' interna si coral capriccio
E tanto (ene va ip contemplarione s :

Nardino s' immagina , ¢ fi compone nel penfigro una’
parendogli d’ haverla veramente avanti a gliocchi., le parla , ¢ fe
Ie dona il cuore ; ed in quefta guifa s' inaamora ardentemente d’ una b
maginaria . sie eye Han HOR ee IPE

PRES C.A. Trattandofi d’ huomo s’ intende Vino dipoca eta; ed h
donna frefchi s’ intende fani , gagliardj 5. di buona cera,quantunque
grave. Virg. cruda deo , viridi/que fenetius . Frefeo Secondo il Ferrari

  
 
  
      
     
    
 
  

  

ney

       

Origine dal Latino vire/cens. La marturina Hella. Virg. Qualisie
fer undis, ; to

PAG ARLE Ia fiera all’ Improneta, Pagarle un regalo all
giorno di S, Luca 18, d' Oucbre all’ Impruncta’, la quale'é una Chiefa
tana da Firenze, celebre,¢ frequentata per un’ immagine miracolofa
fima vergine , che ¢ quivi; la quale in tempo di calamita , ¢ di
portata folennemente a Firenze; ¢ nella venuta di quefla Immagine f
una Lauda in una Raccolta antica di Laude fpiriwali ,

E SE gli imerna si cotal capriccio, Gli fi ficca nel ceruello , o-gli
mente quefto capriccio , fantafia , opinione. Vedi fopra C, 15M. a4
S' INMAMORA come un miccio., S' innamora come un’ afino
mente , perché I’ afino ¢ oftinatifiimo,¢capone. 5 itt

STANZA XXXVII STANZA X
Cos} a credenza infacca nel frngnuolo ,
Ma da un catoegliharagion davidere,
Che s'egli -veryc’eAmor vuol effer folo,
Rivale non ¢ qui con chi contendere,
Ma Brunettoilfratel, che n'hagra duolo,
Poich'il{uo male alcun non pus ‘coprédere .
Tien per la prima un’ ottima ricetta 5
‘Di rimandarlo a cafa in una feggetta,
STANZA da
Ei che vagheggia fott' alle lenzuola Replica quello , ¢ feccafi la.gola :
i gentil volto,e le dorate chiome, Lo fruga, tira, e chiamalo per nome
Ne anche gli rifponde una parola 5 Ed ¢i pianta una vignaye nulla
Won che gli voglia dir ne chene come, Pur tanto Caltrofa ,ch' ei fi rifente,
Cosi Nardino sianamora ardentemente fenza faper di chi. Brunetto fu)
tello lo fece portare a cafa , dove lo meflero in {ul Jetto, ¢ vennero,
Speziali a vifitarlo , ma non conofcevano ne meno effi il di lui male ; onde
netto fi mefie a pregarlo , che gli diceffe quel che egli havea ; e Nardino!
Ja {ua contemplazione non rifpondeva ; pure alla fine vinto da tanti
fratello parlo nella maniera , che vedremo nell’ Octave feguenti.
eA CREDENZ A, Vuol dire, quando fi compra qualche mercanzia ,

  
 
  
   
 
 
 
 

 

  
  
 
  
 
   
 
   
 
 

  
 

          
 
   
 
 

See aS SELB RTE SE &

ASE

=~

 

SETTIMO’CANTARE. 343
fi sborfa il danaro allora , as garlo in altro tempo. Ma qui vuol
dire feniza propofito , 0 fenza fon *

mento. fH Varchi nel Cap. dell’ vova fede.
©) Chiba fquadrato ben la quintefenza,
“. 9) Dite ch’ ella non ha color neffuno-,
© 5 Bebe quel giallo v ¢ pofto a credenza.
pTr@ng) Rir67." > ° ’
Contro di noi bravavano a credenza.
Quefta maniera é corrifpondente al graris de’ Latini . Perfecuti funt me gratis, La
verfion Greca dice dorean ; in dono , cioe di lor cortefia , fenza che io il meritaffi .
INS ACCA nel frugnuolo: S' innamora ; Se bene entrar nel frugnolo yuol dire
anche entrar’ in collera . Frugnuolo é _ Janterna; con la quale fi va di notte
a cacciaagli vecelli ,ed’a pefcare; ed ¢ parola corrotta da fornuolo , perché tal
trnaefiendo fimile alla bocca d’un forno , cosi ¢ chiamata .
EGLI ha ragion da vendere Gli avanza delia ragione. Ha grandiffima ragione.
\ SEGGETT-.A ; Seggiola portatile con due ftanghe. Vedi fopra C. 1. ft. 48.
 GOMITO , La congiuntura del braccio dalla parte di fuori , dove fi piega a.
mezzo il braccio,, dal Latino cubirs .

_VAGHEGGIA, Fa all’amore , amoreggia , con defiderio d’ avere la cofa amata,
Yagguarda., come difie il Buti cittadino , ¢ Lettore Pifano nella fua lettura fopras
Dante, Vedi fotto C. ro. ft. 44. Dan, Purg. C. 16.
aT A Efce di mano a [ui , che la vaghegvia ,

\ S.* Prima che fia a guifa di fanciula .
Enel Parad) 10. | Eli comincia a vagheggiar nell’ arte Di quel Macfire .
Fazio degli Vberti nel Dictamundi ; canto 143.
ale Efe @' udirlo proprio ti vaghegsi .
(cioé feivagho ; ardentemente defideri ) E canto 144.
we Bios va pur, che quanto priego', e chieggio
Al fommo bene , ¢ fol, che tofto fia

2 Vaeay Wel paefe , ch’ i bramo , ¢ ch* i vagheggio.
cio’ defidero , ne fon vago ; col quale io fo all’ amore ; ea cui mi pare un’ ora,
mille anni di ritornare » Vagheggiare il Ferrari deduce dal Latino viftare, frequen-
ter ae, citaa ptopofico i verfi di Lucrezio lib, 1. che defcrivono Marte, che

Venere. uae
—— in gremium qui fepe tuum fe
‘yO Reycit aterno devinitus vulnere ‘amoris ,
Arque ita fufpiciens tereti cervice repofta ,
Pafcit amore avides inbians in te Dea vifus ,
O:pure view da Vago , avido ; perché chi ¢ avido di godere la cofa amata , va at-
torno percercarla , ¢ fi'rigira come farfalla intorno al lume della bellezza di
quella, Dante in un fuo Sonetto .
. To fon fe vago dela bella luce
Degli occhi traditor , che m' anno occifo ,
Che la dov’ io fon morto , ¢ fon derifo,
La gran vagherza pur mi riconduce .
NE che , ne come. Intendi , che non folo non gli volle dire ne il male , ne la,
Caul@ di efig , ma ne meno yolle parlare , SEC-

    
  

  

 

‘a

 
344

MA LMDANMTILE ©

  

SECC-ASI la gola., Se glia icequanie fauci per. isemnan eb Li strc d
PIANT A una vigna.,. Non bada , 0-non attende.a, ice «Che noi

diciamo anche far orecchte di mercante » che &:'

titi, che lif

propongono , attento folo al, fuo vantaggio., irc 57 ‘Ear conte che

L Imperatore 0 far conto, che uno cants.. Per il conteario schi parla
non bada , o non yuol badare , dicefi Predicare al defer

C. 10, ft,

bere.
Studio iaktabat inani ,

46. Jn Latino pyre.trovanGi molt: detti in quetto
Vento loqui., Surdo canere, Frufira 3 velin wannm cantare y cum pifce,
Aliam rem agere » Oc, Virg. Ech 2, tbi bec sanouiteelis

@ gente,
10 5 Predicare:
fignificato , come:
Lire

z ve é

 

SZrifene . Ciok fi sloeglia da quella applicazicne 0 filamin unis sili

 
    

 

STANZA X STANZA XXEXIL os
Dicendo ; Fratel mio, fe nae mi vuoi Kedi jSoggsnnfest’ altro, och’ io m' adirey’
Quel benyche tu dicei volermi a faced, 0. O\par. Rpiuernred etme
Non mi dar noia,va pe’ fatti.tuor y Hai tus quiftione? hai tu qualche rigire
Lerche ii mio mal non é male da biacca, Tx me 0s 4 dire in tutte le manicre,
Al quale ad ogni mo trovar non puoi _ Lardin rifpofe, dopo on ne

Vn rimedia,che vaglia una patacca , Tu fei importuno oi pbanmal

Perch'eglie ¢ firavagante, ed alla moday,
Che non fe ne rinuien Caposne coda s
STANZA XXxXX1L.

Brunetto udito il cafoje quanto e' fia

Ma da chiio devo, iro eccomi prow; 4
Cosi guivi-di tutto fa unragcontes.)

STANZA XXXXUL ]k

E conofcenda, c' a ridurlo in fefto

Ji {uo cordogtio,anch’ eidolente refea y Ci vuol'alero che il medico,oilbarbiert,
Se ben per fargli cnor moftra allegria Vifi spenda la visa,e vadailrefios. |
AMa(com'io dico)dentro ¢ chi la pefia Vuol rimediarui in tutte le ae

Perch’ in veder si gran malinconia ,
Ed un umor si fifo nella tefta ,

E sav Ff rifolue pr
D andar girando il i meal kh
Jn quanto a lui gli par che la fucchielli, Di trovargli una mogtic di {uo gfe,
Per terminare il giuoco a! pagrerelli , Com’ ¢i gliel' ha dipinta ginfto gino,
Fratel mio , fe veramente.tu,mi porti quell’ affetto , che ww-dici, lafeiami Mary |
¢ non mi dir pill altro, perché ad ogni modo tw pm rimediare al mio mal
che & grandithmo, Brunctto di nuovo Jo prega , onde Nardino , vii
importunita gli racconta tutto il cafo , e Brunetto , fe bene dentro ha
travaglio facea buon vifo, ¢ datogli animo fi rifolué d’ andar girando il Mondo
per veder di trovare una donna Jecondo il gufto di Nardino,, ¢ cavarlo di quella ‘
frenefia .
Vna cfortazione,, ¢ richiefta fimile a quella , che fa Brunetto a Nardino, fail
Ma {cherone allo Gnocco meek faper fa. di hui affiizione come fi vede ne i
verfi dello Stef
Atto pr. Sc. pr. » ae riporto qui , perché il Letore veda, che a un’ Let
rerato ( come era Jo Stetonio ) non fi dildice alle volre Ja(ciare gli fludj pil fer
per le bizzarric fanciulle(che., ¢-{pero, che non fara Ailgaee queita poca di digrel-
ne.

 
 

 

 
 

SETTIMO'CANTARE, 34y
6G NOCCHVS, ET MACCHERO:?
Ga ne Mundo traviare venivi ,
pay ” _Cur non tum morui , cum primim lucis in auras
- » Sborfavit genitrix ? Cur me difgratia emper
49 Perfeguitat manigolda fenem? Cur ladra placerum
gy, Abftulis', & cunctis caricas me feeva malannis ? ;
7 Quando finalmentum dabitur mifura teavai ?
x9 Quando refinabis ftreghidima filia ftreghae?
» Dum me-penfabam biancain repofare vechiezam ,
x Mille diabolicis ftraziorque , creporque ruinis .
» Vh me mefehinum | Poterit quis ferre focorfum ?
‘Ma. 55 Appuntum Gnoccum video . Quid brontolas? ola !
gy Fronte malinconica , quid tecum, Gnocche , favellas ?
_ Sigy-Deh poverome , pares viridas magnafle lucertas ;
> 5, Tam demagratus , tam difuenutus apares .
gy Tefta dolet forfan ? Sciatica ? Fiftula ? peius ?
»» Ag potius placidam flurbant penferia mentem ?
y» Dig mihi quzfo tuam (cannat quid , Gnocche , coradam 2?
3» Vade viam , Macherone , tuam . Pradele , fogare

a ~~ 4, Mevolo’, nec quidquam poteris fuccurrere Gaocco .
i Ma ,, Ohimé! cur {prezas fradelli verba pregantis ?
o 3» Quis {cit ? parlando paffabit forte dolorus ,

| gy, Praefertim-caro dum palefatur amico ,

Ga 4, Deh nolis , quxfo , nolis mihi rumpere teftam:
ea 5, Deh lafia me ftar ; fum pienus ; vade bonhoram ,
p » Nec des impaccium , quoniam mihi crefcis afannum .
wt | May »y Deh poffar mundus ! Tortum mihi facis adeffum .

io ‘ »» Cur mihi , Gnocche , tuum non vis sfogare'Jamentum ?
ie x» Sum pro te chi 16: prafum dic, quafo , travaium’.
eh Gn. ,, Pur ibi: Vade tuum , cancar ! tu vade viagginm .

 

" . » Me miferum ! ad mundum veni trafcinare cordam ;
oe x Mancum nonne malum fuerat non nafcere , vel i
i 9 Nalcere debebam , plus prattum nafcere fungus ,

ib 9» Quam malé ftentando {contentus vivere femper 5
pit » Omnibus & giornis centum morire fiatis ?

Ma. 5. Maide ! Cordoglio (ciappas , & (pernis aitam?
ry » Vadis & ad guilam matt , Lanzique briachi ?
ia a» Infuper , & fdegnas , fi quis tua vulnera-curat?

Gn, !
> a »» O bellum tempus , Machero , poca(que facendas !
oi > Otmnes'confilium femper dare novimus'aletis

i yy Sed fibi medefimis nolunt procurare parerum .

a - ay Bene dicit .vaigi proverbium : Ducere danzam ,

i » ‘nuces OMnes , qui fedent , bactere norunt ,
ys Cum funt ad terram . Me lafits dico , malhoram .
Ma, __ » Ah Zucarine meus, meus , ah Gaocchine , galantus ,
a Xx

»» Quid

 
346

Gn.

Ma.

Ga,
Ma,

Ma.

MALMANTILE 4)

9 Quid facies hofti , fidedegnaris amico?) ) — wanes ,
yo Cur mihi nafoonsdisy pbc vulnera
»» Non ego partibo , nifi contes ante’ marezam .

x» Su, fradelle , euam crep: oaconien ieee raconta .
x» Non parlas ? Deh butta fora , mefchine , venenum ;
» Dic mihi , que-carpunt faftidia triftia mentem, ) 5)
5» Que lacerant cuce 5 que te fulpiria rumpunt?.
>» Nonue recordaris tirictos nos effe parentes ? HPiy
>» Eft tua mamma mee carnalis,Gaocche , forella,
x» Atque ego nawura fi,non caenalis , amore
3» Sum tibi fradeiius plus quam carnalis : aitam ,

3» Quam potero tibi , Gnocche , dabo,; fac denique provam eu

»» Nam abi porto beaum , nec me fradelle licenties . ai
x» Namque amo te plus quam me fteflum » ane
»» Dicito cunéa mihi , nec ce mefchine’ fa
»» Confilium forfan potero tibi dare galantum . &

2» Quid turbulentus guardas ? fu butta deh foras;

»» Eia, valent’ homus ; non finghiottire bilognat ;

"5 Valneris afcofti nunquam medicina trovatur ; ‘5 7)
x» Atsborfando foras fanatur fepe dolorus ; 2
»» Fiftula , qua tumuit , totos corrumperet artus , itt
» Ni lancetta viam barbieri Jefta taiaret , #
3, Sufum , Gnocche valens » cordolia dire comenza . 4b

»» O fortuna mihi nimium traverfa tapino,
sy Que mihi per forzam non firappas ventre magonem ;
x» Eft-ne pofhbilum , quod non sborfare fiatum ,

» Vnam nec potero gambam diftendere voltam ?
x» Sum defperatus : volo me impiccare da verum ;
x» Cerne, mei, Machero , cavezam porto fomari.
» Impiccare ? mai. Non impiccare te ,non.non;
a» Mattelcis; coftat troppum impiccare : nieatum

» Tu facies. Guardes gambam ! impiccare ?,Diavol !

» Et te, meque fimul piccares , Gnocche « Gas ‘orlanauting
” Maidé » quis tantum milzam tibi rodit afannus?

x» Dic , faporite meus , que te fuentura chiapavit ?

») Sime impiccabo , cunctos {cappabo, travaios .
>» Pur iJluc; iftam mattezam manda malhoram .

», Sola meum ftentum poterit sbandire,caveza\. Mt bed
>» Ab nimium certé te {teflum , Gnocche’, fafinas: ) )
5 Mancum donna timet , mancum fe donna sgomentat: a
»y Ne facias cofam talem pazelcis adeflum,;

»» Incidis ia brafam cupiens evitare padellam , te
>» Qui fugiens damnum , foccor{um a Morte rechiedisy «

»> Qua nullum maius damoum reperitur inorbe, >)
» Dicas, quid peius furca maginare poteftur?, 5 «4

  

   

 

  
    

FF pe pe

=

 

 
   

 

SETTIMO CANTARE;

> Nonne vides furcas ipfos odiare (afinos y

» Millantas furcas meritant qui mille fiatis ?

x» Forfe putas bellam cofam piccare feltefium ?

» Nullos audifti, nullos nec , Gnocche , latrones

93 Effe volenterum piccatos., Canchere | robbam

x Perdere , poderos,, filios , atque moieram

»» Poffumus ; at contum non muttit perdere vitam |

xy Parlemusd’ altro , bona notte ; porge cavezam ,

» Fac fennum matti , caveas non talopram .

» Si fennum matti facerem , mattiffimus effem ; ’
y» Sum deliberatus cannam truncare una volta ;

2» Nec parles ; quoniam mandas tua verba Patraflum ;
x» Et liquidas tentas accoglicre retibus auras ;

 

347

_ 5» Dextra orecchia bibit , fed verfat lava parolas ;

y» Surdo verba canis; oleum fimul opera perdis .

»» Qui pro te robbam propriam, vitamque gitarem ,

»» Pocum flimo malum pro te gittare parolas .

»» Indarnum gracchias , indarnum dico , va viam .

» Litera vis tandem ficri longiffima ? Ga, Certum .

2 Et godis cortum laqueo difrumpere collum ?

» Audis, Ma. Et tandem cornacchis effere paftum ?

» Sentis. Ma. Bavofam buccam torquere ? Gn, Cofiaum ;
2» Et tralunatos oculos moftrare ? Gn, Davanzum .

» Lucentem faciem , lucentia bracchia , fula

» Vifcera , contradam totam peftare fetore ,

2 Et vitiare diem vitiato vifcere letum ?

»» Sinum ; fi dico , finum ; volo rumpere cannam.

» Heu ipfis fugiende lupis , buttande fofatis ,

» Terribilis ftratiande modis , privande facrato .
» Denique penferus nullus te , Gnocche , tuorum

yy Tangit ? Cui laffas pupillos, paze , chiatinos ?

x» Cui robbam ? cui confortem ? miferofque parentes ?

» Teque finalmentum ? Cafe qui (cribitur heres ?

»» Vis proprias carnes tecum mandare Patrafium ?

»» Vis proprios natos panem cattare per ufcios ,

2 Difperios pueros pitocorum more per urbes 2

x» Et poft de fora veniet qua fama da verum ?

»» Gloria que Caf laflatur? Refpice tandem :
2» Teque , ruofque fimul , mifere miferere fameia 5
» Et miferere tui , qui proijciere fofato ,

x Indignum facro corpus recoprire tereno..

2 Forfan ad Stygias ibis ? (eu forfan Acheum

»» Ibis ad Infernum ? penfa , pover’ home , to feétos ;
2 Penfala , dico , benum : facile eft calare dcoiium ,
»» Sed montare fuper cancar ; ftentare bifognat ;

Xx 2 >> Sed

 
sj

 

348

Ga,
Ma.
Ga,
Ga,

Ma.
Ma.

Ma.

Gn.

 

Sum contentus; abi, grarum fed fiafcum, —
‘ 4 Nam ftio cesta 6 rampas beulaon 6 a 4

 
 

MALMANTILE ©).

j» Sed nec ftentando brutto feapulabis ab Oreo. »

»» Horfus tornemus ca(as ; fu, Gnocche : cavezam

>» Cafe mitte tue. Penfas piccare? bel opram; «

»» Effere non vellem, Veneto pro boia teforo,

» At tw, te fteflum fi piccas , boia farabis.

» Ah tibi , ne quel, tibi fis ne boia medemo, ~~

2 Et qui pro centum mundis non effere velles ;

sv Bflere pro nihilo nolis . pec porge,
ico

pocum , ff P 3
Forfitan ipfa dies (aldabie , Gnocche , fericam, — é
Dura remolicfeune paicis,& tempore forbay 99
2 Nefpula dura die mivefcunt , nefpula dura
»» Guarda mo-, fi Gnocchi poterit mitefcere noia
Tu bene cicalas ; dottorus , & effe videris::
Sed cicala purem ; gicttas nam carmina faxis.. \ «
Al facias i » Gnocche , pl >
»» Extremumque mihi praftes , care Gnocche; favorem .
2» Quem nam ? dil, Ma, ura ; facies , quod certe domando
»» Dummodo fare queam , fabo , fta fupra parolam, —
Et potes , & legrus facies. Gn. Dic ergo ,quid optas
Eft mihi botazus vinetti, Gnocche , rubentis,
xs Quod difamoratis poffet rubare coradam ,
»» Illius humore taze cum plena plaaura cht ,
Saltitat , & brillat , brillando lumina frezaty
Et rubor in vitro liquefatti more rubini ,
»» Ac dicto citius fpumat ; hune inde dileguat
» Puri sbottigliata meri vis fernida , qualis
»» Cum foffiat Boreas , nubes sfrattare per auras
»» Cernitur , & Calum laté purgare ferenum .
ay Sat fcio,, fi nafum preettabis ad ante bicherum',
»» Optabis fieri totum te , Gaocche, nafonem ; $ ”
»» Piccantum retinet pulerum , garbumque galantum, ©
2» Quod refucitaret mortos. De hoc, quefo,pochertum «
»» Guftes , ante tuum claudas quam tofte fiaum’, : sie
xy Atque mei hoc portes extremi pignus amoris , f
x Vis rechem chi 16? Gn, Reches , fed frettola paffym. - P
» Nigotta proderic , cum fim piccandus adi ys ta
»» Auamen hanc lafles , dum torno , Gnocche, cavezam, «
» Ne te gire viam tua tantum fpafima cogant,
»» Et fine guftando vinum , morire , galantum’,

  
  
  

     

   
 
 

      
   

 

 
 
 

 
   
 
 
    
   
      
  
 
   

   

VOLER bene a facea. Portar granditfimo affetto.. BE’ f
Va pe’ fatti tuoi . Cioe vattene,e bada a te, Res tuas ribi babero ;

 

riti anticamente alle mogli , quando fecondo le Romane
Vedi fopra C, 5. ft. 57. 7 —

 

  

ah
8S Geta
 

 
   
 
 
    
  
   
   
 

SETTIMO:\CANTARE: 349

| NON 2 mal da bideca’, Non é male ordinario,e che firifani con pocdrimedio,
perché la Biacca ; che ¢un biahco cavato dal piombo , ¢d é adoprato da i Pitio-
ri, ferue anche per fare un‘ unguento buono.a poco altro, che ad alleggerire il
ng Lp erat perd dicendofi.: Won ¢ mat da biacca , s' inten-
de, Manatee Sri aia :

_ CHE waglia una patacca, Che,vaglia nulla . Che patacca é moneta , che. in,
Firenze non vale, Paracomé una moneta di ramevufata in Portogallo , che vale
guattrini , Cos} noi d*una cofa da noi tenuta in poco pregio , diciamo . Wo

ale un foldo, Nonne darei un foldo See

ALLA moda. Vuol dite all ufanza , come vedemmo fopra C. 2. ft. 54. mrs
in quefto luogo vuol dire ftravagante , 0 nuovo y ¢ non pid fentito , 0 vilto, ¢ del
tutto infolito ; Diciamo: ceruello alia moda per fignificare ceruello fravagance , ©
fantaftico ; dal mutar che fi fa tutto giorno dela moda nel veltire .
— WON fi rinniene ne capo , ne coda. Non frritrova, ne il-principio, ne la fine di
wm 5 yeofa . Non’fi fa , non’s' intende , o non fi ritrova come Ja cola fi tlia, Vee
_—- baput, nec pedes, difle Cic, EB’ traslato dalle mataffe del filo , ¢ fi dice anche Now
ritrov , che @ il principio della matafla .
ih) «| AL tu quiftione? Antendiamo havere inimicizic .

| HAL eu qualche rigiro? Har ta qualche innamorata ? Che la voce rigiro ufata.,
] ‘come nel prefente luogo,vuol dir Pratica di donne per vizio ; che per altro,rigiro

‘fignifica Ripiego , dicendofi : 1] tale fa molte faccende , percht egli ha molui ri-
giré , cive neigh ed oe di vendere la fua roba. Alle volte Gi piglia per

« Vedi fopra C.4. ft. 60,
. DENT RO é chi |e pefhe’ Quand’ uno fi sforza di moftrarfi nel vil allegro , ed

‘ha travagli da ftar malinconico,diciamo; Ei fa beon vifo , ma dentro é chi la pefa,

hes dentro fta in altra guifa . Kifus in ore y fletus in corde, Virg. Spem vulti fimu-
lat , premit altum corde dolorem .

5 Heaiore fie in teha, Penfiero , o fantafia oftinata . Vedi fopra C. 1. ff. 10,
PAR ch ei la fnechieli . Egli fla fra i) st, ¢ il no di fare una tal cofa,che direm-

mo Irrefoluto . Dante Inf. 8.

Padi age Che’lsi,e'l no nel capo mi tenzona,

Traslato dal ginoco delle carte , che firdice fucchielare quando fi tira fu la carta,

‘adagio adagio 3 if che pure é traslato dal bucar col fucchiello , che ¢ una aziones

fimile al tirar fu Ja carta. Qui vuol dire, Pare che quefta fua fiflazione lo voglia.

adagio adagio fare impazzire ,¢ ridurlo a i Pazzerelli , che ¢ lo {pedale,dove fi
mettono i pazzi.
RIDVRLO in feffo. Ridurlo alla giufta mifura ; Raggiuftarlo , rimetterlo in,

buon’ effere : fargli ritornare i} giudizio .. Vedi fopra C. 1. ft. 15.

SU fpenda la Vita , € vada il reffo. Si (penda ta vita, ¢ la toba. Tratto dal giuo-
0, le ff faole feommettere , ¢ dire . Yada il refto; fo del reffo, Bquic det.
to per figura ; perch¢ quando é andata la vita , che ¢ la pil cara cola, che noi

ha iamo, par che non ci refti quafi altro da buttar via.
g 6 "O cinffo, Perappunto. Ela replica ha la folita forza di fuperlativo .

—— Catullo.. Af igis magis increbrefeunt . Nell'Ebraico Azeod , che vuole dice af/ai, rol

ot , raddoppiato vuol dire afaiffine,moltijimo . ,

   

 

|

 

STAN-

   

 

 
 

359
STANZA XLIVs.0%
Percio d abiti , ¢ foldi fi provvede, |
E da buone [peranze alfno Nardino.

Efce di cafa,e mettefi in cammino ,

Shirciandofempreinqua,e inlafevede >

Donna di vifo bianco , echermifino;
E fe ei ne incontra mai di quella tinta,
Vuol poi chiarirfi,vellae verayo finta,

STANZA

Di modo cl ei non vuol reftarui colto
Ma fharui lefto , ¢ rivederla bene
E per quefto una [pugna feco ba tolto 5
E fempre in molle accanto fe la tiene,

_ Suegetto y che gis occorra farne prove,
Brunetto date buone {peranze.al {uo fratello,montd a cavallo ; ed
0 un’ huomo a piedi , fen’ ando cercando d* una donna bianca’, ¢ rofladi
naturalmente , ¢ fapendo che tutte le donne hoggi fi lifciano , haveva
fpugna bagnata,per far con quella la prova,fe il colore era finto,04
per molto , che egli cercafle,non trove mai donna , nella quale occorrefie far tal
Prova , perché fi conofceva fenza farla , che twtte eran tinte y e-lifciate. Quello
colore finto , che chiamiamo li(cio , o belletto,fi dice anche frco-, che |
buona a tignere i pani ; da i Latini detea faces 5 ¢ P intendevano
fi per quetto lilcio , o belletto. Plaut. Moft. 4.118. Vetule edentula ,
corporis fuco occultant . EB di qui i Latini per fuco intendono una forta d
che ricopre con artifizio un mancamento in una mercanzia , ec, onde ;
cere,

pk
S8IRCLANDO , Guardando attentamente . Vedi fopraC, 1.f9,
CHER4MISINO . Rofo di Chermisi , 0 Cremesi , E’ il roflo porporino,chef
fa col fangue di certi vermi chiamati con voce Spagnuola Cocciniglia dal Latino
coccineus color , colore di grana, colore vermuelio ; ed & il pid nobile , ed a 0
te , che fi trovi , ne mai perde il {uo colore:e da quefto nel prefente luoge inte
de rofio naturale a perfezione , ¢ che non perde , come farebbe il finto +
o Karmes in Arabico yuol dire grana.. Latino coccum,(econdo lo Scaligero eferti
tazione 325.

D1 quetia tinta . Di quel colore . E’ termine pittorefco , cotumandosi
dire: La tale ha una carn i

di carne.
VVOL chiarirfi, Vuole accertarfi.

MALIMW/AN TILE © 9%

  

4 STANZAKLV. | ‘
— \

> Che non fi mini 0 fi taftri le quai;
Epre{o un bud cavalloyeunbuomoapiedey 00 B
Cc

ragione nella quale feno beile tinte , per intendere Belli

     
   
  
 
    
  
   
 
   
    
     

‘

Pas.

“Wella pare il ritratto

Quattro dita vi lafcia fu di loiay
Oe
Chrella par proprio un’ Angiolin di
XLV" a pots
Con che paffan Je fopraiil volte ,
Vedra: s'il color 0 fe vii

Aa gira girasin fasti ei non yitroyh,

i=

Ss
sa

 
 
 

fucum

a eee

aachil
ea

NON fi minij . Non fi tinga, Minio & {pecie di color roffa cavato ¢
8

\> adit
ae 1 fi:
no ; € miniare € una (pecie di dipignere con finidimi colori fopra cole e
me cartapecora , ec,

S/ luftri le quoia . Si \ifci la pelle .

MOST ACCIO infrigno . Vilo grinzolo,

refroigne .

© crefpofo,o rinfrignato of
ANCROIA, L’ Ancroia é finta una donna brava in un Poema

   

 
 

 

SETTIMO.CANTARE - 351

~ Regina Ancroia ; ¢ perché quefto Poema é degli antichi , che fi trovino nella.

_ lingua noftra , mi do a credere , che quando fi dice I’ Ancroia ,' intenda una vec-
chia. di Berni , de(crivendo Ja fua ferua in un Sonetto dice .
' Lobo per cameriera mia l’ eAncroia,
. Madre di Ferrak , Zsa di Morgante ,
y w _ )  ehicavola maggior dell’ Amoftante ,
phe Baliadel Turco , ¢ fuocera del boia
_ Ma pnd effer ancora,che queita voce Ancroia fia un’ addiettivo , che venga da
i@ , che vuol dire Zotico, ¢ duro dai Lat. corium quali inquoito , fatto duro , co-
il quoio . )
thee Col pugno gli percoffe I epa croia ,

 

ae Da quefta voce croio habbiamo il verbo tncroiare,che vuol dire aggrinzare ,.¢d

= =
ee

ce
a

 

SRSLERESE

 

renee per intender pelle grinza , ¢ fecca, ¢ indurita , come € quel -

vecchie , alle quali perd fi dice per (cherzo Aduna incroia , che nel parlare
‘Pultima lettera di 44ona confonde , ¢ mangia la prima d’ ixcroia , viene a
ancroia, che vuol dir vecchia grinzola. Jncroiato fi dice un qaoio,che per
flato preflo al fuoco , fia divenuto duro , ¢ grinzofo , ed il fimile una carca-
abbruciacchiata . Si dice incroiaro anche un panno divenuto {odo per gli
mi’, ¢ lordure ; ma di quefto'é pid proprio incorezzato , dal Lat. corryia. Il
bolifta Bolognefe dice, che Ancroia fignitica vecchia,che va crollando il ca-
po, ¢che viene dal Greco Craein che vuol dir croilare . Ma venga donde fi vo-
glia , bafta che apprefio di noi vaol dir Donna vecchia , ¢ brutta y ed im quelto
Aenlo ¢ prefa nel prefente luogo .
_ LOLA, Sudiciume . Terra ftemperata con acqua , ¢ ridotta liquida ,che cons
altro nome chiamiamo mota. Qui vuol dir quelle materie , che fi mettono in ful
vilo le donne y le quali s’ imbellettano , Voce fatta per avventura dal L, i/luvies..
. IMPIAST RA, S' unge con materic bituminofe ,¢ vifcofe come é l unguento.
STVCC.A, Stucco é quella compofizione di geffo , ¢ colla,¢ d’ altre materie
4tenaci, che ferue per riturar feflurejo magagne ne i legnami. E facco é una {pe-
cic di geflo , o terra , 0 altra compofizione , con che fi tanno le figure di rilievo,
i per fucco intende quelle materie , che le dunne fi mettono fopra il vifo per
Ja faccia , ¢ turarfi le margini del vaiolo,,o altre cicatrici ; che il
verbo fuccare yuo) dire intafare , cio¢ riempiere i buchi , ¢ ragguagliare una.
I¢ ; donde gli Orefici dicono fluccare , quando con una certa loro lima
detta lima ftucca , {pianano i lavori d’ argento . Sewccare yuol dire ancora quan-
-do wn cibo ci apporta naufea , 0 i difcorfi d’ alcuno ci vengono a faftidio .
. ViN* Angioline di Lucca.. A Lucca fabbricano certi figurini di cera, di geffo,
Od’ altra materia,a’ quali dopo formati danno il-colore di carne con ua rotio lu-
Sirante; per quefto d’ una donna lifciaca diciamo; Pare un’ Angiolino di Lucca.
Gosi iGreci,che Je belle perfone afiomigliano alic ftatue ben fatce , le chiamano
Agalmata,e Properzio,ditie che il colurito del vifo della {ua donna era giutto co-
»me quello,che fi {corgeva nelle pitture del famofo Pittore Apelle. Quals Apelicis
<¢f color in tabulis . in un’ Bpigramma Greco una faccia impellettata , ¢ lilciata,
con clegante bifliccio vien detta Profopeion , con Profopon , ciot mafchera'y © non
faccia « Vedi Cel, Rod. Leét, antig. lib. 29. C. 7.

AON

 
 

    
  

55%
NON vnol refParni colto., Non ins rimanare q
ST ARV1 lefto . Stare'accorto , O.avvertito, 9) © 96
G/RA gira, Cammina in diverfi luoghi ; cammina
IN fatti, E’ lo fteflo ,che in fomma, o in pen? L,
STANZA XLVIL..
Dopo che tanto a ricercare ¢ itoy
Che i calli alc, . ha fattoinfulafellay
Giunfe una fera al luego d'un Romito,
C’ areftar L innita nella fan Cella y quel delle ¢
A lui parne toccar il Ciel col dito Di che fpeffa ciafcun
(Per non baver a iar fuori ala ftella) Stettero a croschio ii
4M pafjar dentro, ed egli,e il feraitore,
Ringragiando idbuon buom di val favore,
STANZA XLVIILL
Veftia di bigio il Vecchio Mdacslente ,
Facendo penitenca per Adacone Dice chi fia echo di cafael
E perch’ ei fu nell accattar frequente , Aon per fa conta,mad! un fi
Per nome fi chiamo fra Pigolone . Del quale infino alt’ Tad
Coftui y( com’ io diceva ) allegramente Perché gli pare ufcito die
dn Cella raccetto le lr perfoue , Non fifas ei fifia pile carne ,
Spoglia il cavalio,e gli trito la paglia; Cosh piangendo in far di cid'm
Sul defco por diftele Ja tovaglia, Per la mmuta conragl la.
Capito Brunetto una fera alla Cella d’ un Romito, dove efleada
tato , ftando a tavola raccontd al Romito ibcafo del Fratello , dicendo
fuora per far feruizio al medefimo fuo Fratello,
TOCC AR il Ciel col dito, Confeguir I impoffibile . Ap
ST AR alla fella. Dormire all’ aria ; a ciclo {coperto ; alla fella diana 5 Lat
ub dio, NNR
MACILENTE . Mal fano; Ciot magro per lo ftento , ¢ giallo i
ione .
: EV frequente nell’ accattare, Duc tefti di mano dell’ Autore dicono uno
te, edé) ultimo; ¢P altro feruente , equefto ¢ la prima bozza, ¢ fe i
¢ Iraltro pud ftare , io pighierei ? ultimo, perch¢ in fultanza vuol dire che
€ra attento ,¢ diligente nell’ accattare , ¢ fempre chiedeva , che da,
importunita , s acquiftd il nome di fra Pigolone che cost chiamiamo
fempre chieggono , ¢ che moftrando una certa ingordigia di roba,fi
pre dello ftato loro . Pigolare €il verfo de’ puicini , che beccano. Lat .
Spagn. piar dal fare pio pio,che cosi ¢ il lor verfo . ovat
DESCO, Tavola , fopra la quale fi pongono le vivande , quando fi
dal Lat. difcus , che ¢ pierra rotenda , 0 laftra da fcagliarfi, Vedi fopra!
TVTTO accatrato . Ogni cofa havuta per limofina.
FIORITO quanto un Adaggio . Fiorititimo;percht ii mefe di ‘Maggio’ la {
ne de i fiori; O pure perche queili,che vanuo a cantar maggio,porta
ad aealbire tutto picno di-diverti tiori , il qual camo @ albero ch
Bio » ° maio. Diciamo: vizo foecisay quando o per ctier ab ton

 
   

  
  
 
 

 
  
 
   
         
 
  
    
 
   
  
   
  
 
    
    
   
    
 
  

 
  
  

 
  

SETTIMO°CANTARE, 333
per altro mancamentoj il vino dofi nel bicchieresha ‘nella fuperficie minu-
tifimi frammenti d’ una cerca {pecie di muffa‘bidrica ; che é il panno, che fi fas
dal vino , equ ‘chiamano fort ; fi che quis" intende,che il vino era vicino al
fondo dell ', 0 havea altro mancamento,che’ produce la detta muffa ; fe be-
I ‘chevoglia dire Vino ifquifito; perche /ro itoeattribuco di perfezione in tut-
4 syeccetto che nelvino , che |’ efler fiorito é fegno d’ imperfezione.
‘ centuna bette. Quefto numero centuna , benché fia determinato,
dee | t per indeterminato; ¢ vuol dire Cavato da infinite botti di coloro,
}haveyan dato per limofina, E quefto pure é imperfezione del vino, che
_perde lo (pirito., ¢ la bonta in tanti travafamenti , ¢ mefcolamenti , ,
__ STETTERO a crocchio . Stettero chiacchicrando.. Vedi fopra C, 1, ft. gt., €
Bietemin cosi detto dallo ftrepito , che fi fa ridendo , ¢ chiacchicrando
ielle conuerfazioni di trattenimento, percid dette Crocchi, Dal romore fimilmen-
_ teedal faono che rendono , fono dette da’ Prancefi Cloches le Campane. Cosi
i — ‘saccordano nel rapprefentare con |’ arte i femplici fuoni inartico-
jt lati che (ono un’ inalterabil linguaggio della natura. ‘
ed batte dove il dente duoie , Si dilcorre empre volentieri di quelle cofe ,
j@ dove hala paffione’, o fia di guito , o di difgutto.
‘ a il campanello. Parlaya {empre lui, Quefto detto viene da i Magiftra-
/
Cd

  
 

       
  

Bet
*

tidi ¢ , ne i quali uno dei Colleghi fi chiama il Propofto , e quefto fempre
Sa aj litiganti, ¢ chiama , ¢ licenzia dall’ udienze, ed i compagni
‘ a cheti ; ¢ quefto Propofto tiene allato alla fua feggiola un campa-
nellowE da quefio, quand’ uno in una conuerfazione fempre parla lui, diciamo :
yp Bi tleneil caimpanetio . ,
APINCRESCE fino all anima’ Gli ho grandiffima compaffione’; Vedi fopras
in quefto ©; ft, 26. Mi'dilpiace , mi pefa. Dante Inf. 6,
fe RDS “Mi pefa’st , ch’ a lagrimar m’ innita ,
DU Greco dice Achthomai, mi dolgo; ¢ 10 Spagnuolo fimilmente pe/ame . Onde quel
che'in Tofcano fi dice’ dare i! mi difpiace , effo dice , dar ef pefame : La ftefla forza
ha MMe Y AP inere/ee , quali mibi merave/cir , {econdo il Ferrari ; mi grava, e»
BS Pope nage Amore ¢ pefoscomincid Dante una Canzone . £’ m’ incre/ce di
” Z PMO. Sus ‘
WON fap ei ff fra carne’; 0 pefee. Non fa quel ch’ ei fi fia. Noné in cervelio,
Non ha’ vO conofeimento . Awevo pefee dicevano gli antichi un’ biomo /Pra-

Eee ,

ae

 

 

 

 

wm math
ip! neces ANZA LL STANZA LIL
ee Sta Pigolone attenro a collo rorto Egli ha un giardino poffo in un bel piano ,
i Ad ‘afcaltarlo s€ poi ch’ egli ba finito; Ch’ ¢ ognor frorito,e verde tutto quato ;
“a: F igiiuol ,rifponde a Ini 5 datti conforto, Giardiniero non v’ t, ne Ortolano,
oy + Bfappi, che th fei nato veftito , Che a entrarui nefjin pus darfi vanto,
» Che qui? Pbbnom falnatico Magorto, Da per fe lo lavora di fua mano,
a Ch'e un beftione , un diavol traveftito , E da fe (0 fondo per via a! incanto ,
o & “Che fe te lo vedeffi vb eglie pir brutto! Con una cafa bella di frupore ,
ye Balke a {uo tempo conterotti it rxtto, Che vi potrebbe ar  Imperadare .
5 4 vy STAN:

 

  
  
  
    

 
354 MALMANTILE

STANZA LIIL
Ma io ti uno dar’ adeffo un? abbozzata
Lui prefto preffo della fua figura.
Ei nacque a’ un Folletto, ed’ nna Fata
A Fiefol n’ una buca delle mura,
Ed ¢ si brutto, poiche (a brigata
Solo al {uo nome crepa di paura ;
O quefto ¢ il cafo a por fra i nocentini
ed far manciar la pappa a quesbabini,
STANZA LIV,
Oltre ch’ ei pute come una carogna
Ede pin nero della mezza norte ,
Ha il ceffo d'Orfoye it collo diC arogna,
Ed una pancia » come una gran borte
Va in {ui balefirs, ed ba bocea di fogna
Da dar ripiego a un tin di mele cotte
Zanne ba di porco,e nafo di csvetta ,
Che pifcia in bocca,e del continuo getta,

STANZA

ea lafciando per bor  altre da parte,
Cocomeri vi fon di certa raza,
Che chi ne puo haver uno,e poi la parte,
Vi trova una belliffima ragazza ,

Pigolone incefo il bifogno di Brunetto, gli da animo con dirgli, che
huomo faluatico ha quivi un’ orto,dove fon cocomeri, che tagliandoli n’elce fuo-
ra una bella fanciuila , la quale chiede da bere, ma {ce feglida., ella
Deicrive ancora in quefte quattro Orta ve la qualita di
SE/ naro veftito, Hai havuto buona fortuna, 0 qu wd
quefto termine per efprimere,quand’ uno defiderando qualcofa difficile a y
s abbatte accidentalmente a trovarla per appunto ,,come ci la defideraya , eda
propolito del (uo bifogno . Dicono te Levatrici., che taluolta nafcono ban
con una certa {poglia fopr’ alia pelle , la quale fpoglia non fi leva loro fubitos
ti , ma fi jafcia , ¢ ca(ca poi da per fe in proceflo di giorni.; eral creatur
fi dice vara vefiira ed € prefo per augurio di felicita di quella tal.

ha dato origine al prefente dettato.

VN diavol traveftito. Vin diavolo immafcherato da huomo; intende un’

brutto , quanto il Diavolo.
BELLA di flup
yvede;
VO!

I Pittori dicono Abbozzare

Jerto ¢ Gianni Schicchi, dice che i P

STANZALVL. —
Dell’ ofa poi ne fa fi ce

ere. Belliffima mirabilis vifu . Tanto bella , che fa flupire
ma per venire la yoce /tupore dal latino,pud ognuno intendere il {uo}
IG LIO darti un’ abborzata , Ciok ti vo;

Y > quelle prime pennellate,che danno in.una tela ;
trove , dove voglion fare una pittura . Vedi fopra C. i uy

FOLLETTO . Vno di quelli fpiriti infernali , che dicono che ftieno per Hari
1) Ferrari nell’ Origini alla Voce Fuile ,citando Dante Inf, 30, efi diffe , quelf

   
 

z

    
    
       

E della pelle ne fa maccheroni, =

Diente in Jomma v't , che vada ma
Sreche Brunetto figlinol mio, tn femti,
Ch’ egli é un cattivo,ed orrido ammale,
Hora torniamo a {uci {compartimemi,
Ove fon frutte buone quanto il
Vaghe piante, bes fiori, ed altre vole
Com’ to ti potrei dir maraviglofe,
LVI, AYE
Che per effer aftuta la fua parte,
Dirattiche tu gli tpia una fun rare ,
A un di quei fonti li fi chiari,e freddi,
Ma fe la ferni, a Lucca ti i

efto Mi i
ota. cadena

nis

glio de(crivere alquanto , 0
ft, 41.

?olletti fono la/civ) genij ac Lemures rifu aS
domos implentes , . = E.

 
 

 

  
 
 
   
 
   

SETTIMO CANTARE. 335
FAT A. Vedi fopra C, 4. ft. 45. i
eA FIESOL 0 una bucadelie mura, A Ficlole fi veggono ancora alcune reliquie

delle mura di antica Citta , ed in ci frammenti di muraglie fra I’ altre fi
_ yede una gran buca di »od’altra cofa fimile , la quale dalle donnicciuole é
- ereduta, ed é dataac ai fanciuili per abitazione delle Fate , ¢. pero vol-
ms @idetta /a buca delle Face. E quefta é quella buca , nella quale dices
-» che Magorto era nato d’ un Follerto , ed’ una Fara. Angelo Poliziano
~ al titolo Lamia dice : Vicinus quoque adbuc Fefulano rufcnlo meo lucens Fon-
ticuins eff, fecreta in umbra delite/cens , ubi fedem effe nunc quoque Lamiarnm narrant

      

a imuliercule « Quefta credo fia quella caverna , che-oggi fi chiama /a fonte fotterra
¢,  luogo orrido  ¢ (paventevole , ma fempre pieno di limpidiffima , ¢ frefchiffims
lt NS TEV ‘ 2

ry “SNocewr 17 « Cioé quei ragazzi,che s’ allevano nello Spedale deg!’ Innocen-
so erensa +5: ton

ca ~ CAF AR mangiar la pappa a quei bambini, Cosi diciamo d' un’ huomo , o donna
nil a ¢ brutti, quaGi che fieno come i] Bau, Ja Befana, ¢ fimili larue in-
ga  uentate dalle Balic per render i bambini ubbidienti, ¢ fare che per il timore man-
- Cai wa. Vedi fopra C. g. ft. 3. E.quetto putire da i Latini_ cra efpreffo
0 co} paragone , perché dicevano vixum cadaver . 11 Monofini .

ut nero della mezza notte . Negritimo , piii nero del buio ,

 

 VAin fui balefri » Ha le gambe fottili , e torte come fono i baleftri , compa-
‘fazione vulgata), {endoci una cantilena di Balie , che dice .
° Ben ne venga Mignamau,

Saif oe j Cha le gambe a baleftrucci .
O81. 3 ¢ Sbilenco , dicefi chi ha le gambe torte ; ¢ ancora Aver Je bilie ;
tratta la fimilitudine da certi legni torti , o randelli , co” quali i vetiurali legano
itetto, ¢ arrandeliano le fome ; da loro dette bilie .
_ BOCCA di fogna. Alla bocca delle fogne maeftre , o principali , che ricevo-
‘no acqua delle ftrade,quando piove , ¢ la conducono nel fiume d’ Arno , é figu-
rato up ma(cherone di pietra , il quale ingoia |’ acqua ed oga’ altra {porci-
zia. , € di quefte intende il Poeta ; ¢ da quefto diciamo : Bocca di fogna a uno, che
mangia , ed ingoia ogni forta di cibo, fe bene fporco , fenza diftinzione , o ri-
yaleuno . Latino bedwo , gurges. Quefte fogne in altri luoghi d’ Italia fono
lette Chiaviche dal Latino Cloaca .

DA dar ripiego., Cioe dove entrerebbono tante mele cotte, quante n’ entrereb-
be in un sina, che quel gran valo di Iegno , entro al quale fi mette 1’ uva pigia-
$a bollire per farne vino. rare
», ANNE, Denti : Propriamentes' intende di quei denti Junghi , che hanno i
Signali , i lupi ,i.cani, ec. che noi li chiamiamo anche denti Adac/tri ; 0 Adaeftre .
Vedi £ aes, ft. 64. Borle & meglio dir /anne , ed ¢ pill conforme all’ origine ,
Onde /ubfannare buriarfi d’ uno ridendo , in maniera , che tutti identi , come di-
$e il Boce. fi poteflero trarre ; moftrando le {anne . Daa. Inf, C, 6,

Quando ci feorfe Cerbero il gran vermo,
Le bocche aperfe,¢ —— le fanne.
yz

 

SER GF BRLLSES weeks

Se

it
4

eC,

   
 

  
        
 
    
  
    
 

356 MALMANTILE?D*)2

¢C.22. E Ciriatto , a cui di bocca nfcia > Vo
D’ ogni parte una Janna come 4 porcoy eres Pepe)

Gli fa fentir-comel’ nie sdrucia. » 8
NASO , che pifeiain bocce , Ciok nafo aquilino’ che ha la a
la bocca , ¢ pare che vi colidencro . vb~orn saokay
BERLING ACCIO « i Giovedi geaflo, chee I ultimo giovedt

detto Serlingacciv da Berlingare, che vuol dire bere, ¢ mangiare',

mente , come fi fa in quel giorno : ¢ cosi Magorto, quando pi a
faceva conto , che quel giorno fufle il Berlingaccio , toleani:

menti , pappalecchi, e Gorxeviglie,daligodere, Latino ga

    

   

ennizzandolo con
c wifare , conic fi
antico Glofiario , onde lo Spagaualo gozar., godere pel noftro gavayzare
ti finonimi , che voglion dir ghiotcornic Bocc, g. 8.n. 2. Sé cues
pis volte infieme fecero corzovighe sec. << i8927 toil? :
MIGLLACCIO , Sangue di porco , o d’ altro animale mefeolato'con
farina’, ¢ poi frittoynella padella a ufo di frictata'da aleuni* Latiot
chus ; fe bene quetta era una compofizione-dicacio , ¢ falamevdal
che vuol dir cacio , etarichas , che vuol dir falame. 1
STVZZIC ADENTI, Nettadenti : Sottiliffimi,ed acuti ftecchi
d’ offo , o d’ altra maceria per ufo di nettare i denti + Latino
BYONI quanto il fale. Saporitifimi .. Vaaevivanda con molto fale’
rita , che vuol dire i] contrario di {ciocca , oinGipida ye: fenza Yale 7 ¢
faporito é meglio al gulto , che Pinfipido,'¢ pero per faporito i i
¢ dicendofi; buoni quanto il fale , s’ incendefaporiciimi , cioe guftofifimi

fapore . es
TCOCOMERO . Specie di mellone acquofo di fapore'dolce , che nella
ftagione calda per rinfrefcarfi. In moiti luoghi d’ Italia chiama
cost la chiama i] Mattiolo , e dice che era incognita a iLatini , fe bene G crovas
cuckmis , ma intendono il cetriuolo , che pure in alcuni luoghi fi chiama eeeome™
Anguria, dice il Perrari , ¢ detta quafi cucumus anguinens 5 © cost quefto nome
che era proprio del cecriuolo,per mancanza di vocabolo fu tratto a
fructo , che noi Tofcani chiamiamo cocomero . i
e4 LVCC Ati riveddi , Quefto detto fignifica Non 1a vedrai pitt,
Buoni da Lucca nel fuo teforo de“Proverbi dice , che havendo un
Lucchefe veduto ua Gentilhuomo Pifano a Lucca,usd feco cortefia
definare a cafa {ua , dove condotto , fu trattato con ogni forta <r
titofi il Pifano , e ritornaro alla patria,avvenne che fra poco tempo’
ando a Pila , dove paruegli conucnevole vilitare il Pifano fuddetto : Ts
pero alla cafa di eflo , dopo haver molte volte buflatosal fine sa ffaccit
© gli difle che nom lo conofceva ; onde il Lucchete diffe: 2: e4 Lucta ci-vede
Pifa ti conobbi , © con quetto fiticenzid . Cost forive un Lavcchefe:, ma tt
voltano il proverbio dicendo: 4 Pifativedat pea Lacca si i
grato ,¢ {cortefe quello da Luccaje:non quello da Pita’) Seibene il
era ne Lucchefe ne Pifano nella fua En. Te. C. 3. ft. 4. dice: ;
E dicon fpefo alernis Ti veddi a Lucca,

 
    
  
 
 
  
   
 
 
     
 
  
 
  
 
 
  
   
 
 
 
     
 

 

  
 
 
      
    
 
  

—nw ese E-epeEE PBB TL ete

   
 

 

 
  
      

SETTIMO CANTARE! 337

STANZA LIX.
Efe en ean
Dirad , che tu buon Cavalier non fia,
entre conforme all’ oblige non vft
Seruitie con le Dame, e cortefia .
ea lafcia dire,e tien gli orecchichinfi,
Non ti piccar di cio, fia pure al quia ,
Gracchi a {ua poftayty non le dar bere,
Accio non fugga;e poi ti fia il dovere .

Con quefta, che fara farta a pennello, Vientene dunque meco,e fia in ceruelo.
| Come te cerchi , lenerai dal cuore Cammina piano, ¢ fa poco romare ,
—  Ogni dogti i affanno al tuo fratello, Chefee' ci fentea forte, fcuopreil cane,
16 --Edioten' entro gia mallevadore . No occor’ altro;noi habbiam fatto ilpane,

ite feguita:a narrar la favola del Cocomero , ed inftcuito Brunetto di co-
a‘ ome contenere , perché la fanciulla non gli {cappi , s’ avvia con effo alla
volta del giardino di Magorto. -
s far conto @ haveria vifea. Ti puoi dare a credere d' hayerla veduta qua.
4 : i a.vedere , perch¢ non la rivedrai pill. 2
Al uno ftivale . Refterai beffato, Retterai uno fcimunito. Vedi fopra
yg  (Geqfero, LGreci diflero Bagas con/fieyti , da un tale detto Baga , 0 pure Bagoas
a} ‘nome da Eunuco ; che fu un’ huomo infipidifimo ; Donde poi noi diciamo Bag-
sg) 60» © Baggiano , a ua’ huomo {cimunito fe non forle da Ba/eo,, ¢ da Babbano ; 0
da Bageiano forta di fave maggjore dell’ aitre .
of va di forche 5 edi moine, Vina quantita grandiffima di finte carezze ,€
‘Nezzi; i Latini diflero blandicie, Ed in quelto propofito tanto é dire far le forche ,
5 « dexai , quanto mome , fignificando cutte tre una forta di Jufinghe fatte con
wit Bti,ocon parole , ¢ fono quafi lo fteflo che adulazione ; perché ancor le»
dt “nine sec, {on atti , gefti , ¢ difcorGi , i quali contengono , fe noa falfe lodi, co-
»meccontienc ? adulazione , almeno falfe dimoftrazioni d' affetto affine di com-
eo. jiacere ye di acquiftar ia grazia di colui , a cui fi parla ,¢ quefte fon proprie di
ie di femmine , ¢ |’ adulazione ¢ conuentente ad ogat forta di perfone ,
ma é fempre indizio d’ animo vile , ed effeminato . Ll Landino nell’ efpofiziones
a Dante Inf, C. 18. dice ,che gli adulatori in lingua Fiorentina fidicono moinieri;

   

f=

¥ »Ma quefta voce non fi dicendo in oggi , ac avendo autorita di Scrittore nell’ an.
fi tico , mi fa credere , che il Landino la derivaile a capriccio daila voce Fiorenti-
i na Meive non trovando parola corrifpondente alla Latina ddulatores, I) Cala

nel Galateo volendo mettere in volgare il Latino ada/ari , lo efprelie colla paro-

SSL.

» la Piaggiare , L Bini in lode del mal Francefe dice:
uhangl Jo non roppi gid mai; ne carfiiancia 5
Machi mi va con si fatse moine,
Vorrei porergli sfondolar 1a pancia .
_ La Stor. di Semifonte Trattato 4. Quand! altri ha ofefo un fupremo, non ¢ da fi~
< darfi di lui’, ne delle fue aftute moine ,¢ Lufinghe. ,
-  NON+i piccare. Non v' offendere none’ adirace ; Non cntrare in gara; Non
ats ti

 

©!

=

 

 
as

ae
mI cL

358 MALMANTILE 4) |

ti flimare ingiuriato. Vedi fopra C, 3. ftan, 20. Tanto il Franzefe quan:
to lo Spagnuolo Picar voglion dire Pugnere ; forfe da Picca ;

  
 
  
   

    
     
  

   

colina

wale Omero appella nyttein, cioe pungere. Vino piccame & que
iecda Ȣ che punga , lee they cP amma
bio; Tienle caro. ll Perfiani Tesan taeda bea
Va menati l agrefto 5 :
Ceruellaccio peftato per Lambiceo ?

Che 'l tuo mordente ha trove poco appicco .

Di quefto iv non mi picco “ct

Che s* io non ho la nobilta a bigonce , yet

Mi bafta ds non effer a’ undici once, (cioe baftardo) —

PICC ARS! , Vuol dir anche perfuaderfi , o darfia creder d’ etfer eccellentes
in una cofa , come piccarfi di bravo , di bello , di dotto , ec, ¢ vale quanto efier am
biziofo , o haver ambizione . sx pee

SSE £8 ® Ss ome

 

ST-A al quia, Sta {odo : Non badare a quel che ella dice; enon tilafeiares | tf
fuolgere , o perfuadere a darle da bere. Dante. State contenti , wmanagenits, |v,
al quia, ‘ hive ow

GRACCHI a fua pofta . Gridi , cicali , eflami pure quanv’ella vuole; lafciala | yy
dire , Ja(ciala cantare . Quand’ uno vuol quaicofa da un’ altro 5 ed ‘ ty
mandarglieia ,¢ colui non glicla yuol dare,{uol replicare a i detti di: Rs
chia , gracchia ; quafi dica : Tanto mi muove il tuo dire 5 quanto il geacchiar) | ti,
d’ una cornacchia . Vedi forto C, 8, flan. 64. far Fup

TJ (tia il dovere , Ti fucceda quel che w meriti. aa

SAKA fatta a pennelo , Cioe fara fimilifima , ed appunto come cs

T” entro Mallevadore , Te ne afficuro. Ti fo ficurta,che leverai u
Fratello quefla frenefia. Adadevadore ¢ il Latino Fdeinffor , quafi afidarore , afi
curatore ; detto Maiievadore {econdo il Menagio , dal /evare in alto la.
fegno d' afficurazione , Lo Spagnuolo lo chiama Fiador , la qual voce in
co Vo)garizzamento Tofcano manofcritto delle Vite di Plutarco tra
lingua Aragonefe, refid {enza interpretazione infieme con alcune altre y il)
guiva in gucfte tali traduzioni , o per vezzo del traduttore , 0 per i
gine , o perche non ne fapefle pi la. Caro mon volle il dipofito , ma fiette t
tutti, wah:

NOL habbiam fatto il pane, Noi habbiam dato nel laccio . Noi i
vuro la difgrazia fenza rimedio. Diciamo ancora ; Voi babbiam fritto, Vou
fouo C. 8. fan, 54. sega

STANZA LXI. STANZA LXIL ©
Zitti dunque 5 nefjun parts , 0 rifponda + A cafa lo strafcina ete lo fica |

     
 
 

 

eAndiamo che e's’ ha a ir poco lontano,
Cosi va innanzi, ¢ I altro lo feconda ,
Oikjernitor lo fegue anch’ ei Piano piano,
Ma quel Demonio,che va, fempr’ inroda,
Gii fente , e gli vnol vincer della mano,
Perche gli afperta,e il vecchioc'alla fiepe
Vien primoghiappa/ ,come dir : pepe .

7s facta ,e conlacorda ve

E fatto quefto a un canapol'

Che vien dal palco oueea a vertas
E per pigtiar il refto della critthy
Ejce poi fuora , ma ncl fate’
Che quand ei prefe q ‘
Ad afpettarlo havute

 
 
 
  

SS ee. ew SE FF ee eR Keke eee

 
 

   
 
  
 
 
  

 

SETTIMO CANTARE: 359

Soot a Selb SRRSTANZA LXE
| Edoggimai fi trovano in franchigia , Sfogarfi intende ,¢ a quella vefte bigia
ene Vuole un po meglio feardalfar le lane ,
Kabel: june, en'érantoin valigia Percio /u verfo il bofeo col pennato
Che ne manco daria la pace 4 un cane; A tagliar un Quercinal va difilato .

Pigolone efortando i compagni a far romore,s'avvia con effi ver{o il giar-
dino, ma appena giun(ero alla fiepe , che Magorto gli {enti , ¢ prefe il Vecchio,
che era op vicino alla detta fiepe , ¢ condottolo a cafa lo ferro in ua facco ,¢

palco , tornd per pigtiare il reito , ma non gli trovando , fen’ andd

| alco 5
al bo{co per fare un buon battone , col quale haveva ia animo di baftonare Pi.

_ 2ITTL, Cheti, Vedi fopra C. 1. ftan. 10.

LO feconda , Gli va dietco: Lo feguita , Petr. Canz. 8.

pies Ed un gran vecchio il fecondava appre/so .

__ EB fpelfo in ronda . Gira per orto facendo la guardia . Ronda dal Lat, retun-

dus; dal quale é fatto il Franzefe Rond ritondo .

_ GLI enel vincer della mano, Vuole efler pis diligente , ¢ pitt lefto di loro ; gi
wool prevenire . E traslato da quei givochi di dadi , ec, ne 1 quali il punto ugua-
Ie noné pace , ma vince quello , che ¢ il primo a tirare ; per efempio , io fond il
primo a tirare , ¢ {cuopro fei ; tira il {econdo , ¢ parimente fcuopre fei, ¢ fe be-
neil punto ¢ uguale , vinco io , che fono ftato il primo a tirare ; ¢ quefto fi dice
Vincer della mano , perch colui » che ¢ il primo a tirare,fi dice baver la mano.
tanto bafta ai noftco propofito, f€ bene moiti altri giuochi di carte danno quefto
Privilegio alla mano .

{ SIEPE, Chiudenda , 0 riparo fatto di pruni , ed’ altri fterpi agli orti , eda
icampi, E’ yoce latina . Franco Sacc. Nov. 83. E giungende dove era la vigna,
qucftaera molto affoffara , ¢ con una buona fiepe .

CHLARPA fu , come di pepe . Piglia fubito ,¢ fenza contrafto , o fatica alcu-
na. Credo , che quefto detcato fia corrotto ,¢ che fi debba dire : Come dir : pepe,
che é facilidimo a profferirfi , come tutto labiale ,¢ di fillaba raddoppiata ; ¢ che
da quefta facilita fi cavi il fgaiticaco di facilita in dire 50 fare una tal cofa , per-
ché'a dire; ‘Come di pepe non ci fo trovar figaificato , 0 fale alcuno. Chiappare
dal pecaaere . Da Arripere fece il Bocce. Arrapare, Nella Lettera del medefi-
mo fcrittay a Meffer Francefco Priore di Santo Appoftolo , E fimalmente can
pit largo parlare ferivi , che io non doveva cosi {ubito il partire , anzi la fuga dal tuo
Mecenate arrapare, Volle efprimere il Lat. fugam arripere con dare a quel verbo
wna terminazione Tofcana. Cosi #rappare abbiamo fermato da extra , € rapere .

STRASCIN ARE. Stra(cicare un materiale per terra fenza follevarlo ,o por-
Jo fopra veicoli. Lat, Trabere .

FICC-ARE, Vuol dir mettere una cofa in un recipiente con violenza dal La-
tino figere,

_ CRICC-A, § intende conuerfazione , 0 compagnia di pid perfone: metaforico
da quei giuochi di carte , ne i quali tre figure uguail infieme fi chiamano cricea ,
come tre Re, tre Dame , 0 tre Fanti.

» | AVRIANO banuto del bue . Haurebbono havuto poco giudizio , poco avve-
be

.

 

 
———

360

SI trovano in franchigia . Si ttovano in ficuro, in Inogo, dove n

refi; che franchigia intendefi un luogo immune per pri

tincipi, Lat, asy/m y che pure alcuai Tofcani dico alte
ine eT

mofi di yoci nuove,dallo Spagnuolo dicono amparo ,
RIMANE un bel minchone . Riman buriato , riman beffato. weno
flan. 15. fi dice ancora reffare uno fivaie fopra in quettoC, spo
E in valigia, Erin collera . Si dice anche im bigencia yin’ nel
nel gabbione , ec, come habbiamo notato fopra C. 6. tans 41. &
un’ arnele di quoio , entroal quale fi mettono cofe necefiarie per la
fona , quando fi viaggia , e's’ adatta in fulla groppa dei cavalo, e quelli
vanno a piedi la portano in {u le reni , ma quefta propriamenfe fi dice
NON darebbe la pace a un cane. Non darebbe Ja pace a Veruino ; ciot
ftizza , 0 collera , che egli ha, che fe gli venifle avanti un’ amico ,
be come nimico , perche la.rabbia gli ha fatto perdere il conofeimento , Si dice
xn cane , © non un’ altro animale, perché |’ alo noftroé di dire + Wow
do guardi in vifo ; Non ha cane che cli vogiia bene ; nom ba cane che lo foccorra 6p ai
t# , € queflo perche il cane ¢ timbolo della fedelra’, ne-fi trova animale pill
liare , ed amico dell’ huomo , che il cane ; e pero dovendofi pigliare un’
vicino all’ humanita , ¢ profiimo al ragionevole ; nel prelente luogo 5
i {opraddetti proverbi , pigliamo il cane. ta
SFOG ARS/ intende . $i yuol cavar la rabbia. Vuole sfogar ¥ ira;
all’ ira, come fi fa del fuoco, del fummo , che gli fi da apertura,
VVOLE un po meglio fcardaffar la lana A quetia vefte bigia, Scardaflar’
vuol dir battere , e pettinar la lana ; con denti di fil di ferro a i an.
che cara: ( dalla fimilitudine del cardo erba fpinofa ) raffinare Ja lana , accioeeht
fi pofia fiare. Vedi fopra C, 3. flan. 60. ¢ per metafora fignifica baflonare ind;
¢ perd qui dicendo , vole fcardaffare , ec. intende Vuol battonare ue
torna bene I’ equivoco , perché par che voglia dire rilavorare,¢ di ;
re la lana , con la quale ¢ fatta Ja vefte di Pigoione. Li Puici nel Morgantes: ”
Adattera it bartaglio ancor dal Cielo ee
In qualche modo a feardaffargli il pelo, a
PENNATO , Coitcllone adunco , il quale ferue per potar le viti 5 app
forte cosi da quella crefta , © penna tagjicnte , che ha nella parte di
nio Marcello alla Voce Bipennis dice cosi: Bipennis manifefium ef id we
utraque parte fir acutum , Nam nonnulli gubernaculorum partes tenuores ad D
mulitudinem pinnas vocant eleganter , Pennato ancora é epiteto , che ¢ ftato’
Latino a’ yolatili .. Onde tcherzando {ull — » ditie 1) Boece, Gi ]
18. / vidi volare i pennati y cofa incredibile a chi non gli aveffe veduti, EB n0i'
a raccontare gualche novella , per renderla pit credibue, factiamo
fegnito nell’ antico afiai , quando gli huomini eram pid femplici’, &
che volavano i pennati, Palladio de Re ruftica tit. 43. difcorrendo de’
deContadini vi nomina é pennati,e gli chiama falces a rergo acutas , atque laiitl,
DIFILATO . E jo fietlo che Andar di vela,di filo , addirinura,
C. 6. ftan, 10, Vedi fopra in quefto C, ftan. 5.

ob”

 

  
 
 

 

MALMANTILE!D 04%

eaEFRS rere &F Peerae

oe. pers er kro 2e028 825 5

 
  

Bot
STAN ZA LXV.

Ed ei le corde alfacco aun tratto feialte ,
£ fatto quel mefchino ufcirne fuore ,
Che lo ringraria , ¢ bacta mille volte ,

el cht del vecchio. E fa un falto poi per qnell’ amore,

chiufo in quel {acco iltrova pohe >. > Vi merce il can cin e guarda le ricolte y

oe. 4 mal por! Dandogti aint, ed ezli se il feruitore ,

Poi con i piatri ye pie vafi di terra

Due fiafchi di vin rojo, ¢ lariferra,

LxVL
Quando Magorto in gik viene a ricifa
Con una fhanga in man cotanto fara ,
fesse crspands delle rifa Perchée gli par mill’ anni con quel tronco
wove con quegli altri firimpiatta ; Difar vedere altrui ch’ ¢i non é monco,

0, che ftava naicofto a offeruare , veduco partirii Magorto , corfe alla

9 ¢ trovato il vecchio nel {acco jo cavo., ¢ vi mefie dentro i] cane con

di terra, ¢ duc falchi di vino , ¢ rattaccatolo come ftava prima fi na.

oo vedde venir Magorto con una grande ftanga in mano.
felice, ’ paroia di commilerazione ,come mefchino ,¢ fimili.

YANDOS! 4 mal porto, Trovandoli a cattivi termini.

arrucolada poxzo, Carrucola ¢ una catiecta di legno , e tal volta di fer-

alla quale ¢ impernata una gircila {canalaca , ¢ (ope’a tal girella s’a-

, 0 catena per tirar fu pefi con facilita , e quelta carrucola fi tiene co-

ente appiccata al pozzo per tirar fu acqua , ed i] moto , che fa cal girelia

ta cagiona per lo piu ftrepito , al quale il Poeta atiomiglia i fofpiri ,
; ‘igolone .
ae SFA fais, per quell amore. E’ un detto faceto , col quale s' efprime Ja gran-
a, ¢ contento d’alcuno: E tal detto viene da quzi Ciechi, che per
i Popolo fanno nelle piazze giocolare i cani, ¢ fra gli altri giuochi gli

{ ¢ al baftone con dire : fa un falto per amor d' un pane, ed il cane tutto

» © per il contrario dicendogli ; /alta per uaa mano di baftonate , il ca-

ein atto di mordere , ¢ non {aita ; ed il termine per qucil’ amore figati-

lazione , O in riguardo ; come Lo fo la tal cola per amor tuo , s! in-

bh tende Io la fo in riguardo , 0 a contemplazione tua per |’ amore ch’ 10 ti porwo ,

_ SERATT-A, Vedi fopra C. 5. ftan. 13.

flere f delle rifa. Rider gagliardamente . Rider come fece Margutte , che

baenpp:s fecondo che favoleggia il Pulci nel fuo Morgante; Ll’ verbo

a altro yuol dire allentarfi gi’ inceltini , vale anche quaato /eoppiare,

parities pur fidice: Scoppiare ,¢ morire dalle rifa, Bd & quel re quati che

“habbiamo decto fopra C, 3. ttan. 65. Li Pulci nella Beca dice:

Petty wit ‘ Ta fet nel letto , e crepi dake rifa .

st enone Sitorna a nafcondere . Vedi fopra C. 20, ftan. 60. ¢ forto C.9.

bis he fa cht ek s* appiartd miffer gli denti .

ia era i emi a Trattaco ——_ dice: Quejte cofe ho cavate da un {ix

bro

   
 
 
 
 
 
 
 
 
 
 
 
 
  
   
 
  
 
  
 
 
    
 
    
    
  
    
  
  
   

oraeeaal

 

  

 
 

   
 
   
   
  
 
   
 
    
 
   
   
 
  
 
   
   
 
 
   
  
 
  
      
    

362 MALMANTILE —

bro del Comune , che fu impiattato da uno de’ Buonhuomini ,¢
4 ricifa, Senz’ intermidione ; fenza fermarfi , a p
difilato detto poco fopra Octava 63. antecedente . I) Pulein
ES io mi metto a cantar a ricifa, a
COT ANTO fata, Grofia in quefta guifa. Vedi fopra C. 5.
flan. 36. Tam
Par veder , ch’ ei non é monco, Far conofcere ch’ egli ha le mani; 0
non ha mancamento alle braccia. Jonco yuol dir uno che ha manco
tutte due le mani. Lat. A¢ancas,
STANZA LXVIIL.
errriva in cafa, ¢fbracciafi ,.€ fi mette Ed ei , ch’ ¢ 1 fulle furie non vi
( Serrato V ufcio ) con il fue randello Che infin.ch' ei non fisfoga
Sopr' aquel [acco afar le fue venderte, Sta intato il vecchio all'ufcio,
Suonanao Zuant'ei pao fodo a martello, Ad origliare per udir qualedfa s
Ll Romito che ftava-ale velette , E fente dire: O lecca
Perche? nfcio hadi fuora il chiaviftello Carne feantia , barba pi
Andi ( benché tremando,e con fpavento Ribaldo , Santinfizza, €
Che havea di lus) e ve lo ferro drento , C’aquel d’ altri pon cingue, el
STANZA LXIx, 1

   
 

  

  

  

Guardate qui la gatra di Mafino , Ma quel’ hai toltoa me,
Che riprendeva il virio , ed il peccato , Won dubitar ti coftera fa
+Se il monello-ha le man fatte a uncino Che tante volte al poxzova'
Per gire a [erafignar pel vicinato? Ch’ ella vi lafeia il manicog

   

  
 

  
    

Magorto , arrivato a cafa , fi meffe a baftonar quel facco , credendo che vi
fufle dentro Pigolone.; Ma quelto-efiendo ufcito di-cafa mefle il ci
di fuori alla porta , ¢ fermatofi alquanto quivi , fenti che Magorto mn
facco gli diceva una mano d’ improperj . win ti
SBKACCLARS!, Vuol dire Denudarfi'il braccio da mezzo in git te |
mano come accennammo fopra in quefto C. flan. 19, B sbracciarfi ; ee
mente parlando vuol dire Impiegare ogni fua forza, diligenza , ed mol
in un’ affare. Lat, mamibus , pedibu/que eniti . want 1
SVONANDO a martelio . Cioe baftonando . Suonar’ a martello fi <<, m
do la campana fuona a rintocchi., come fa il martello full ancudine, ii che i | %
quando fi vuol ragunare il popolo per li bifogni della Citta. Il verbo fumaretil | &
Latino puifo , vale appretio di noi , come apprefio i Latini per fuonare, ¢ pet |
perquotere . Vedi fopra C. 3. ftan. 7. aie
ST AVA alle velette., Stava offeruando . Veletta , 0 vedetta diciamo
to, che fta in fulle mura d’ una Citta, o Fortezza a far la guardia d
munemente/entinells., edil lwogo dove fta detto foldato fi dice velerra
Sumo che fia trasiato da i Marinari , che tengono la detta guardia_
albero delia nave , ¢ dicono metter I’ huomo aila vela , 0 veletta forfe
piccola vela,che fia in quel luogo . Tarcagnotta Stor. lib, 5. p. 3. 7
Partitofi pero il Priore Stroxzi da Marfilia con 2 3. Galere , ed una g
welette in mare lo venne ad sacontrare. Dal che ficava che fi chi:
-gune barche , le quali camminino avanti a una armata con huo
 

 

SETTIMO CANTARE. 363

Je, opure da vedere vederta'e poi corrottamente veletta . Si come da /pecio anti-
¢ Latino fignificante lo veggio » fi fece /pecula luogo eminente che figno-

_reggi molto paele . Ma fia come fi fia bafta il {apere, che ftare alle velette vuoi
dire Stare a offeruare .
| Bin fale furie, E'colmo d’ ira .
ORIGLIARE , Star in orecchi , Star a fentire , ¢ vedere con attenzione , edi
iy! cofto.Pranzele oreillier . Spagn, otear forfe dal Gr, Ora,orecchie , che i Fiaa-
fini {piega :/piare , eguardare da (uogo aito , come fanno le fentinelle .
__ PEVERADA , Brodo di carne , o d’ altro, E /ecca peverada yuol dir Brodaio,
fe Beiniignifica porco , perché il porco mangia volentieri ogni forta di broda.. .

_ War, St, Fior. lib. 14. dice: Gli diede una mineffrina bolita , cotta in peverada di

- pollo. Detta Pewerada dal Penere , cioé dal pepe , che per dar (apore fi metteva.s

fa le mineftre , come fu da altri dottamente offeruato .

CARNE ftantia , Carnaccia vecchia, ¢ frolla. Vedi fopra C. 3. flan. 24. ¢ $4.
ye _ SAKBA piattolofa , Termine ingiuriofo per un vecchio;¢ vuol dire barba {chi-
i 2, epiena di pidocchi , ¢ d’ altre lordure ,

_SANTINFIZZ A. \pocrito; de i quali a baftanza s’ é detto altrove ; EB per
yy  {atinfieza's' intendono certi Torcicolli , che ftanno tutto il giorno d’ avanti a
_ una immagine d’ un Santo , perché fi creda che effi facciano orazione .
yi , GABBADEL . Rinncgato . Vno che gabba , cioé inganna le Deita, adoran-
fio, Oggi una , e domani un’ altra , rinnegando ja prima . Se bene Deus non ir-
yal ‘Tetur. Si dice ancora Gabbafanti . §
ay, PON cingue , ¢ teva fei, Vuol dire Tu (ci ladro ; perché ponendo cinque dita
we della mano , fai il numero di fei con aggiugnere alle cinque dita la roba , ches
sath porti via. Plauto diffe: Trism literarum Homo , cioe tees Habbiamo diverfi
modi di dire copertamente Ladro , come Sgrafignare . Havere le mani a oncim ,
: che fi vedono nella prefente Otttava 69. BefPemmar con le mani, Andar aCarpi,
¢ 64 Borfelli., Par il Lanzo ( che in lingua lanadattica yuol dire Ladro ) gixocar , 0
# lavorar di mano ,¢ Gimili .
i ‘é _LAgatia di Mafiro, Quetta fingeva d’ effer morta ,¢ nen era ,e perd vuol
é " dire huomo finto . Huomo che fa il femplice,e non é . Lat, Lepus dormiens , Te-
nere gli occhi aperti, baver L occhio, ed aprir I’ occhio vuol dire andar cauto nell’
Operare: ¢ perché tanto Ia lepre, che il gatto tengono gli occhi aperti anche dor-
mendo , feruono a i Latini , ed a noi per efprimer un’ huomo vigilante , cd ay-
yeduto , e che moftri di non efiere . Vedi fopra C, 1. ftan. 19.

MONELLO . Cosi chiamiamo quei guidoni , che per Firenze battono mari-

_ Ma, comes’é detto fopra C. 4. flan. 8. Siccome Guidone di nome proprio fié
fatto appellativo , cosi forfe anche Monello , in principio diminutivo dt Adone ,
accorciato dal nome proprio di Simone é venuto a fignificare una tal razza di

 
  

perfone.
'. ASSASSINO . Vuol dir ladro di ftrada , ma quié detto in vece di furbo ,o
‘ »¢ pud anche intenderfi ladro di ftrada ,
NON dubisar ti coferd falato . Sta ficuro , che ti ha da coftare aflai , 0 che ne

-pagherai un gran fio .
— TANTO va la fecchia al poxzo , ec, Tante volte fi torna a fare un male , ches
Seri tey ‘ Zz

2 una
i
i ~
"

ba ee a

   

 
364 MALMANTILE ©
una volta vi fi riman colto . Vna volta’ fa per molte ; e diciamo ancora; Tate ) wij

volte va la gatta al lardo , che unavolta vi lafcia la zampa\, ec

violantium malus eff , ed orecchie della fecchia diciamo quelle due | tL

rate , nelle quali ¢ infilato il manico di efia (ecchia . sear Ne at
S 4

TANZA LXXx. STANZA LXXL ai
Poi fente, ch'egli dopo una gran bibbia Ben ch ci creda finua ‘
D ingiurie dd nel facco una percuffa , Tira di nuovo, eda vicino
Che rurte le frovigiie /perra,e tribbia, Ed il fuo cane acchiappa i
Ech eidiceva; Horsugiihorottol ofa; Che fa-urliche van nell
E che di nuovo un’ aitro ne rafibbia., » Ona’ egli fiupefarto afjai ne
E che ( facendo il vin la terra rofja ) Dicendo: Qui é quand iomi
Soggiunge:O quanto/ague banelievene! Se nce’ il fangue egli ha di gi
Quella ghicttene, a me, beeva bene. Come a gridar puo egli
Seguitando Magorto a dire ingiurie , da una baftonata in ful facco , €
i piatti , ¢ fa verlare il vino , ¢ credendolo il fangue di Pigolone refta
to, che ne pofia haver tanto ; € replicando un’ aitra batlonata , ae
po il cane; 11 quale comincid a urlare , ed ei credendo, che fuifero ftrida di
lone , ftrabilifce ¢ non retta capace , che egli pola haver piu forza di 7
ara

  

ae

PoetEstEe

frida , mentre ha verfato tutto ul fangue .
DOP PO una gran bibbia, Dopo una lunga diceria , 0 filaftrocca ;

Dopo haver dette tante ingiurie , che farebbono un gran libro , da Biblia Greco
Latino , che vuol dir br: ; E fe bene la voce Bibbia oggi comunemente ¢ istela
per il libro deila Sagra Scrittura , cuctavia noi la pigiiamo ancora ne i cafcome
il prefence nel detto fenfodi libro , o di lettera, © di difcorfo lungo , come pate
che la pigliaficro gli antichi fecondo Herodoto lib, 1. dove dice > Alarpagum
clufife , leporis ventri biblion.ad Cyrum , Se bene qui ¢ Viguerro 5 Jetrera, Dal po
ma d’ Omero intitolato I' Lliade , il quale é d’ una prodigiofa quantita di vert,
come quelli , che a(cendono al numero di quindicimila (etcecento oreantatre ; ut
gran moltitudine di cofe , 0 di parole, diflero i Latini 4ias , o Hiades , Propeaid
41.2, clegia 1.

oe rRs oe FT

Tune vero longas condimus Iliadas ,
Seu quicquad fecit , fine eff quodcumqne locura
Ataxima de nibilo nafcitur bifforia , ne

RAFFIBEIA, Replica. ‘Irasiaco dal congingner con fibbia bottoni ,
il che fi dice -4fibbiare , Vedi fopra C, 2. tt. 81.

STOVIGL/A, Intendiamo ogai forta di piatti , e vafellami di terra per nfo di
cucina. Ll Ferrari, Seovigue, Fittiia, vafenla, & frivola. Vandena, to
comperi. 1o fimo che fia parola florpiata dalla Latina . Veenfilia , Crefe, 12:12.
E molti altei arnefi,e fovg 4 di bilogno . Pallad. volgarizzato lib, 1. tity 6 Faber
da far terramenti , ¢ dilegname , e di /ovigli da vino , da Javorare, eda
Quefto ultimo non é nel Launo , ed é aggiunto nella traduzione per impiegate
voce Mowtgli, 7 3

TX/BSLARE , Propriamente vuol dire Batter i] grano in fulltaia dab Latino
Tribula tribule , 0 tribulum tribuli , che vuol dire una {pecie di carro , ¢ol gia
fquoceva il grano in fit!" ata , come fi cava da Colum. ‘lib, 2, cap. ie

@uifa eo PEEP ~ fe Ee oe eT

 

 
 
   
   
 
  

   

SETTIMO CANTARE. 365

‘unt adijcere Tribulum , & trabam pofis , ¢ Varr. lib. 1.C, 25. E'/picisin area

cM tivwencis iunitis', © tribula . B queflo dal Greco eribein peftare , trita-

. Latino terere , o da thlibein {chiacciare , dal qual verbo viene il Latino trib.

travaglio  dettoanche da’ Santi Padri prefura, ‘

£. Quefto termine fignifica A mio giudizio ; Secondo me. Secondo il
4/0 intendimento ; ¢ per ie fi dice replicatamente 4 mé a me, Quan-

» cl0é per quanto io giudico i Franzefi Quant’ a moi, 1 Greci fimilmente

» ciot fecondo me , fecondo il mio giudizio.

DE haver finita la fefia . Crede haver terminato il negozio , ciot d’ haver’

Pigolone, Similitudine trata dalla folenaita , colla quale fon facti

i, che fi giuftiziano.

CHIAPP A. Coglie : perch? fe bene «cchiappare vuol dir pigliare uno con
¢ violenza , ci ferue er efprimere colpir bene. Latino Certo iétw a/-

Spagnuolo , acertar, Vedi C. 2. ft. 41.

EF ATTO. Rimafto ftupido per la meraviglia grande. Latino ob/upe-

STANZA LXXII. STANZA LXXIIL
in questo mentre:col fuo fante Perch'ei del certain quanto a contentarla
_ Haven di gid {correndo pel giardino Non ci ha ne meno un minimopenfiero,

   

   
      
   
    
  

 
  
   
     

 
 
       
   

it pritrovato , e quelle piante E pero quante volte ella ne parla,
we coles,che chede sl {uo Nardino , Mura difcorfo, ela riduce al zero;
vot! tha trata fuor belle galante , Ma perch'ellae mozzinaye con laciarla
tif be mon fi vedde mai il pit bel fennino, Le Afonache trarria del Monaftero ,
Econ un {no bocchin da fciorre aghetts Vedeyche s*ella bada troppo a dire
i  Chiede da ber ma non gid fel’ afperti, Si lafcerebbe forfe connertire ,
" av ‘ STANZA LXAlV.
wil) Peri per non cadere in queffo errore E ch'ei ne venga ch’ ei l'afpetta fuore ,
st | Lapigtianun tratto,efe la portain frrada, eAccio con effi anch’ egli fe ne vada,
yl Ed ai vecchio fa dir pel feruitore , Che i non vuol lafciarlo nelle pefte y
* Che'pit tempo non é di frar’ a bada , 44a condurlo al pacfe alle lor felke .

“Mentre che Magorto fi fludia a baffonare, il favio Brunetto col feruitore eras
andatoinell’ orto , ed havea trovato i] Cocomero , ¢ tagliatolo n’ era ulcita las
fanciulla ‘che egii cercava , la quale fi mefle a pregario , che egli I’ empictic las
tazza, maei non volle contentarla , anzi la prefe , ¢ la porto in firada , e man-
dO i teruidore a chiamar Pigolone per condurlo {eco alle nozze di Nardino .
a ANTE, Si dice i) feruitore ; dail’ intero infanre,fi come in Latino Puer figni-
" fica ferno , da noi detto anche garzone , {e ben Fante perd comunemente vuol dire

" - foldaro'a piede , perche ne’ tempi dell’ Imperio baflo , che la milizia comincid a ri- |
of a tarfi pil per ja cavalleria , che per Ja foldatefca a piede ; il pedone G venne as |

‘ttimare come miniftro., ¢ feruitore del Cavaliere ; ¢ percid fu detto fanre ,
| SENNINO. E’ una parola , che fi dice per vezzi a una femmina bella, favia ,
~¢ pulita , ¢ che operi cen giudizio con fenno,¢ con puntualita. Latino /cita pue/~
la,feitula . z
~~ BOCCA dat feiorre agherti., Cosi diciamo di quelle femmine, le quali per parer
“belle tengono la bocca ferrata , ¢ ridotta forzatamente pi Mretta del. {uo nau.
ss rale;

    
  
 
  
  
  
 
 
 
  
 
   
   
   
 
  

366 MALMANTILE ©

sale; ne muovono i labbri di come fe gli fono accomodati allo fpecc
par proprio , che habbiano la bocca accomodata a feiorre un ng
Aghetto é quello, che vedemmo fopra C.2.f. 10,
‘NON fe? afpetti. Non lo fperi. Ciot non afperti, che le dia bere « |
gnuolo ¢/perar ¢ lo fteflo , che a/pettare . bes pe a My
LA riduce al vero, La riduce al nulla; Zero quella figura d'abbaco, che
fe ftefla non rileva numero alcuno , ed accompagnata, forma le decine , ¢ eile
per efprimer # nuda, ’
eHOZZINA, Huomo aftuto , triflo , ¢ che fa il conto fino, mas'inte
genio maligno. Latino Vulpis reliquie . Quefta voce vien forfe da orecehi m
che cosi fon fegnati quei furbi , che meriterebbono le forche , ma perla
eta non ne fon capaci , fopra C. 6. ft. 54., ed in quefto C, ft. 30. Ȣ credo
perché diciamo Azoyzorecchi in vece di moxzina nello fteflo fignificato,
TRARRIA le eAtonache dal eAionaftero , Confeguirebbe I’ impotlibile con fas
fua induftria , periuafiva » ed cloquenza . Diogene diffe: Oratio non ex ani

Sfn2i* 8

proficifcens , fed ad gratiam compofita meleus eff laquens , quod [cilicer blandé x
ens hominem ingulet . té
NON é tempo di ftar' a bade, Non & tempo di trattenerGi. Non v' étempods | &
erdere . ; R
LASCTAR' uno nelle pefte . Abbandonar’ uno nel pericolo, Vino fa’ R
folenza , o mala creanza , ¢ per non efler percoflo fugge viaye la(cia i Ra
€ quefto fi dice /a/ciar nelle pefte, cioé nelle pedate , o nella ftrada , che &
mancamenti ha fabbricato ai pericolo,colui che é fuggito;fi pronunzia cont ay
ma c ftretta a differenza di pefte infermita, che fi pronunzia con l’é lagaies | 4
pero quefta rima ha un a di falfita , ma tollerabile, ed¢ ammefla. &
STANZA L&XV. STANZA LXXVL s
Cost di la poi ructi fer partita ; Brunetto fi ridea di Pigolone mm 1A
Ma piis dogni altro allegra la faciullay Perch’ ei parea nel vifo un fico vittty | %i
Perché non prima fu dell’ orto ufcita E menaua a due cambe di ee pth
Crognt incanto,agnt vogliain lei s'anulla, Com’ egli haveffe hauuto i Birri drt; aay
Anzi ai lor preghi in ful caval, Salita , E la donna diceva : Grambracont, hey
Che la duri; ed il vecchio manfuttt, | 4

Senza pitt ragionar di ber , ne nulla,
Va sipreinnazs ag altri wn trar di mano Che fi vedeua fatto il lor xsmbello:
Fiera, ¢ bizzarra come un Capitano . Dagli pur (rifpondea) ch'eglie fafitl. | "sp
Vicita che fu {a fanciulia dell’ orto cefsd incantefimo , e la voglia del bere4y
anzicon la maggior’ allegria del mondo monté a cavallo {cherzando, € moe | &
teggiando il vecchio , il quale era ancor pailido per lo {pavento havuto, i)
"RIZZARKO . Wuol dir lracondo , Suzzofo , o cola fimile , fecondo chelule | &
rono gliantichi, Ma fi piglia anche per {pirito(o , ¢ vivace , come é !
prefente luogo . In Spagnuolo Zixarro fignifica uno che vada bello, ¢ fupecbo nel Ge
veitire. B fimilmente roba bizarra, che 1 Pranzcfi direbbero bigearree , ie) 4
roba , cio’ vette bellifima , varia, ¢ pompofa, donde poi da noi fi prende Bare |
ro per capricciofo , firano , ftravagante . eae Bk
FICO vieto. Fico annebbiato , o afato. Vn fico , il quale al colore , ¢ tene
rezza par maturo , non é, ma dalla nebbia é ridotto giallo, come fe fulle ma:

 

 
 

SETTIMO CANTARE? 367
furo : comparazione , che efprime affai bene la faccia gialla , e grinza di Pigolo-
ne. El’epiteto Viero ¢ proprio delia carne falata , lardo , burro, ¢ olio, quando

_. per eflere ftantij , ¢ corrotti mutano il colore , I’ odore , ed il fapore .
, _MENAR di {padone 4 due gambe . Fuggire ; Correre. Spadone a due mani fi
quella pada pil grande delle {pade comuni ordinarie , la quale s' adopra
-ambe ie mani , ¢ per derifione di coloro, che, vantandofi di bravi , all’ occa-
poi fuggono , col folo dire ; meno di Spadone , 0 gioco di spadone , s' intende a

ye gambe , che vuol dir Fuggi. Vedi (otto C. 10, ft. 3.
COM egli havefe havuto s Birri arety, Detto ufato per efprimere, che uno
corra velocemente

GIAMBRACONE, che a duriDubito,che voi non fiate per durare a cammina-
re. Giambracone fu un mato , che fempre andava gridando: Che /a duri, e»
| perd quando noi veggiamo , che uno faccia un’ Operazione con grande attenzio.
‘Re ,€ che noi dubitiamo , che egii non fia per durare fogliamo dire Giambracone ,

an) © (enza dire , che /a ders intendiamo ; piaccta al Cielo, che egli continovi , € cost & Co-

 

intefo .
BATT O il loro Zimbello. Divenuto lo {cherzo. Zimbello,oltre al fignificato ,
i @ 0 (opra C, 1. ft, 59.,vuol dire aacora quell’ ucceilo, che fi lega per
un piede allato al bo{chetto de’ paretai , 0 altri luoghi , dove fi tende per pigliare
ig uecelli , che tirandofi quelia cordicella , che ha legata al piece fi fa fuolazzares
Per incitare gli altri uccelli a calarfi. Latino amis illex , € dallo ftrapazzo , che
ry tale uccello riceve diciamo Zimbellouno quando ¢ burlato , beffato , ¢ ftrapazato
ad da tutti ; nel qual fenfo ¢ prefo nel prefente luogo ; ¢ forto C. 9. ft. 66.
7 — DAGLI ch’ egii¢ faffedo . Dag, ch’ ci lo merita. Olleruifi che 1 verbo Dare
nei cafi come i prefente,vale per continovare , feguitare , durare, ec. ¢ con dire
rin folamente dagés icnz’ altra aggiunta s’ intende /eguita ; ma s'aggiunge ch’ egli ¢ faf-
Selle per una certa vaghezza , ¢ per un genio ,¢ naturale inciinazione , che han-
N01 Fiorentini'd: paciar per proverbio , metafore, comparazioni , o fimilitudini ;
i

 

- € forle ¢ aggiunto per confondere,ed ofcurare il detto,perché dare al fafedo vuol
fe dir perquoterio , ¢ nov vuoi dic feguitare . Habbiamo due fpecie di tordi , cioé
: botraces ye fafjedi 5 1 primi fon meno aftuti , ¢ piit facili a la(ciarGi pighare, i fecon-
| di fono pid aftuti , ¢ ad ogni poco di romore {cappano , pero quando la notte col
# "s frugauolo fi {cuoprono , fi dice dagli con la ramata , che qucfto ¢ fa/sello , che alpet-
«’ ta poco. In fuftanza nel prefente luogo vuol dire continuate , 0 Seguitare , a burlar~

i mi, beffarmi , ¢ firapazarmi , ch’ io lo merico, Da quefta aftutezza del faffello fi di-
a fi S¢ fafsello a un’ huomo , che {a il conto fuo , ed efercita il fuo fapere a vantaggio,
my" pretendendo fempre pil del giuflo , e del dovere , avido di guadagnare , € tenace
* el fuo pid del conueniente .
w STANZA LXXVIL

uf Cosi feberzando , com io dico ,in brigiia Percio dopo baver fatte molte miglia ,
i Ne vanno Lenya mai fentirfi franchi y E che tor parue un tratto d'e/serfrachi,
) Efempr’ ognun pin calda fe la pigtia , Tutts affannati per st lunga via y
wo. Percheilcimor glifpinge,e /pronai fachi; D' accordo fi fermaro a un’ Ofteria,

i!
ty
# STAN.

 

 
 

MALMANT DLE) S G0 —

  
  
 
   
 
  
   

368
STANZA Laie au a
Dove il padron che intende fare. pafto >» Ben. ”, '
Trovagran vroba per yi garbato 5 ‘Guamioiinfa ba a0,
Chreitien che afar no habia rroppornafpo, E che quelia »
Mae? non fache enon hanno definate; Che's,

Brunetto con Ja fua compagnia feguita allegramente ‘il {uo vi
do per il timore , che hanno di Magorto y ma fti ia 7
un’ Ofteria , dove mangiaron pit di quello , che il padrone non: z

SC HERZARE in briglia, Quetto detto,che fignifica uno,che flando|
faculta , ¢ d’ogni commodo , non oftante G duole dello ftato fuoy éd
anche per intender’uno , che ftia allegramente ,¢ {cherzando fenzae
che egli é in grandiflimo pericolo ;¢ cosi s’intende nel prefente luoga,.

fcherzano fenza pen(are al pericolo,nel quale fono » che Mas arrivi
dofio , da-chanhp, peas

OGNVNO fe la pigtia pits calda ,~Ogauno fe ne piglia maggior
fto pigtiarfela caida i Franzefi efprimono col verbo chalsir , ¢ noi cal
dal Lat. calere ; Boccaccio nel Poema in ottava ‘rima intitolate il

 

de’ fatti di Tefeo 1, 2.
Oude li fe nuova vifion vedere ; s OSes i
Perche di ritornar li fu in calere. 2 Mey
E appreflo. Vici d’ Atene , ne li fu in calere , ae
D' Ipolita amor dolce , e pudico. me
Spiegd la forza di quefto verbo il Petrarca quando diffe } ee

We dentro fento , ne di fuor gran caldo;
Che fa come una fpiegazione de’ due verfi immediate precedenti:.
Ne del volgo mi cal’, nedi fortuna ; oy gaialah
Ne di me molto me ‘di cofa vile, ome
GLI parne d' effer franchi, Parue loro d’ effer in ficuro , ed effer liber da Mo

orto. hoped
OO ARE 4 pafto. Si dice quando I Ofte fenza prezzare cofa per cola di quell

che mette ia tavola vuole ua tanto per perfona , ¢ mette in tavola quello yee

are a lui .
f NON habbiano a far troppo guafto. Non habbiano.a mangiar molto, Le.

  

aPEFER

 

feo incognito dice .
Jo ero fario , ¢ non fei troppo guafto,
Il Berni in lode delle pefche +
Diofcoride , Plinio ye Tecfrafto
Lon hanno feritto delle pefche bene
Lerché non ne facevan troppo guafto ,
Cioé non ne mangiavano molte , perché ‘non gli piacevano.
V? & rimafto, L’ ha fgarrata. E’ rimafto ingannato,, come chi
trappola. ssf
LON vi refta fate. Non vi refta nulla. Vedi fopra in queflo ©, flan. 7}
Mattio Franzefi contr’ alle sberrettate dice; ; cals
“Hed
Ae

FREER

we

 

 

ss,
 
 
 

  

SETTIMO CAN TARE 369

A cavarfela , ¢ metter pitt di cento
> *Folte per hora , +l che non ferue a fiato ,
va dietro alla cafserta, Cioe non fi gaadagna , ma pil tofto fi perde,
TANZA LXXIX. - “STANZA LXXX,
sntante | frracco S’ 16 percoffi quel vecchio marivolo
wil randello a quel partito , Com’ ha io fatto, diffe, un canicidio?
‘ciolte,ed apertohavedo omai quel/acco Sa ch’ io lo prefi ye la ferras qui folo y
sencinar la carne del Romito , Che gnun porea vedermi,o dar faftidio,
Ed in quel cambio viftovi il fuo bracco Won fo s' 0 fono il Graffo Legnaiuolo
coceh 5 vetri macolo ,¢ bafito, A quefe metamorfofi d' Ovidio ,
fa miaravigliato in una forma Che fono in ver meranigliofe, e frane

     
   
  
  
 
  
     
  

  
       

Ct ei non fa s'ei fia defto,os'ei fi dorma, Poi cnn Romito: mi dinenta un cane,
; STANZA LXXXL
e'| povera Melampo Lo ho una rabbia addofso ch'io avvam:
Che | nétte gua tencé oui Jaci; Con quel veechiaccio barba d'Olo sae

  

Chi pit fard la guardia al mio bel capo Ch al certo fatto m' ba cost bel ginoco ;
defio, che t’ hai cliinfe le lanterne? Che dubbio! metcerei le man nel fuoco ,
eo Magorto'dal baftonar quel facco lo fpiccd dal palco , ed apertolo vi
il {uorcane ; ¢€ reftando maravigliato , fuppone che fia ftato Pigolo-
li habbia fatca quefta burla .
ere In quella guifa ; in quella forma , in quella maniera ,
kntendi frammenti di piatti , pentole, ed altri vafi di terra ,
Pe .Badro , giuntatore . E’ voce Napoletana ,'ma gia facta Fioren-
tina, A 4 7
CHE gine pores darmi faftidio, Che niuno poteva impedirmi, La voce gaxno
per nino , hogei @ ufata folo da 1 noftri contadini ,
NON fos’ io fono 11 Grafo Legnainolo, Non sos’ io mi fia diventato ur’ altro, il
‘Graflo Legnaiuolo fa vn Fiorentino , il quale fu tanto femplice , che gli fu dato
@ credere , che non era pili lui ma diventato un’ altro  ¢ per quefto tale fu mefio
! ¢ alloppiato , ¢ fatto dormire quando fi rifenti , s* accordé a paga-
Te le tpee je le cancellature per il precefo delitto , del quale fu affoluto , benché
havefle Confeffato'd’ haverlo commefio come nuovo perlonaggio , ¢ pagd il dena-
10 un fratello di quello , che il Graflo fi credeva d' eflere , ¢ duro in quefta cre-
_ _ denza qualche temipo ; ¢ fin che li fuoi veri parenti lo fecero riconofcerfi, ¢ ritor.
share: che egii cra. La Novella pare a me, é ftampata dietro alle cento No.
vellea dell’ edizione de’ Giunti . Da coftui digiamo i Graff Legnaiuolo per
intendere un’ huomo fempliciffimo., ¢ facile a creder ogni-cola , bench’ ei fappia
non efler vera , ed effer’ impofiibile , che ella fia . Si dice ancora Calandrino , ¢
Cap, ,comie aecennammo fopra Cy 5, ft23. t ,
VE Romito mi dineneaun cane, Se bene intende, cheil Romito era diventato un
‘caN';/perché nel! facco trove il cane, vi haveva meflo i) Romito, fi potrebbes
‘anche ‘che intendefie parergli gran metamorfofi , che un Romito y cioé ua’
i bene jdiventi un cane, cioé nno (cellerato . i a
- MAL chitfe-ie lanterne'; Hai chiufi gli occhi ;-ed.intende fei morto,, Chiamanfi
Anche gli Occhi /ccicanré'in lingua furbefca ; € Cosi li chiamo in un verio del {uo
fio Brunctto Lauini Macftro di Dante . Aaa 10

an

+225 28

 
  
  
  
  
    
   
 
 
  
    
  
  
 
 

wi tie Ea =

   
  
 

 
   
   
    
    
  
  
 

370 MALMANTILE ©&|

10 ho una rabbia addoffe ch’ io avvampo, Latino Jn fermento totus fi
collora , un’ ira grandiffima. vvampare fignifica abbruciare leg)
cfempio ; Vn panno bianco accoftato a una fiamma s’ infuocola,e piglia
fi dice arfo , o abbronzato , o avvampato . » # Sie

BARBA d' Oloferne. Barbaccia . E’ nota la Storia facra di Iuditta,
Ja tefta ad Oloferne . Nel pepeeiones detta ftoria , li Pittori per far
Oloferne per un’ huomo crudele , dipingono la di lui tefta tagliata brate:
barba lunga , folta , ¢ rabbuffata ; ¢ da quefto il dire a uno barba a’ Olofer
giuriofo , perché fuona anche lo fteflo , che refta d’ impiceato , “

 

&8& set ise

wr

   
  

   
   
   
 

METTEREL 1a mano nel fuoco. Mi par d’ effer cosi certo di gaefta cola, cheio | »,
Ja giurerei con metter la mano nel fuoco . Vno de’ giudizzi,che chiamavano’ wi
vini , appreflu i Safloni era la prova , che faceva il reo , via del fuo > 1
nendo in mano ferro infocato. E le folennita , colle quali fi veniva a qu -
va , fono defcritte puntualmente dietro all’ Iftoria clica di Polidoro Vi j "

TANZA LXXAIL STANZA LEXXIV be
Oimé le mie stoviglie, ¢ il vin di Chianti Ma perch’ ei vede quivi le a Tis

Chrio tolfi in dar la caccia aun vetturale Volte al giardino,e poi verfolavity | Gy

eA cagion di quel trifto Graffiafanti CheBrunetto,equegiialtri: tu

in um tempo ¢ verfato , ¢ ito male, Quando v'entraro,e quando andarovia R

Giuroal Ciel ch'io non una ch'ei fene vati, Infofpettito , lafcia andar il Frate, Pe

E,s'¢i non vola , puo far capitale Ed entra nel giardino,e a y i

Chr io voglia ritrovarlo,e s'es c’ incappa Scorge quel {uo cocomero dit an

Che mi venga (a rabbia s'ei mi feappa, Ch'e frato il fargli un fregio fopr w

STANZA LXXXIIL STANZA LXXXV, | 'y
Lo trovero bensi , perch’ io vue ire Poiché levata gli han quella fighualay | x

Qua intorno per veder s'io lorintraccio; Chiin effa(cons' io bo dette) fi trouwvs, |»

Cos} corre alla porta per ufcire y Per la ftizza non puo formar )

Ma cei nd puofarlo,perché e’ v's il chianaccio Si soraffia, barre s denti, efa z

Lo fquote , e shatte per volerlo ape eS E {palancando poi tanto di gola dey

Edhor v'attacca ’uno,hor altro braccio; Verla,befemmia il Ciel, z ne

Noiato al fine vanne ,e corre ad alto, Dicendo ; QO Macomettoe ¢

E dai balconi in frrada fa un falto, Che fi facciano al mondo i")

STANZA LXXXVL mime | i
fa quanto a te chi ti pifciaffe addofso Sapro ben’ io a coftor far shy. «thy \ y

So ben che th non ne farefti cao; Credilo pur, percht , fe fi da il cafe h

Ma io che da miei di mai bevvi grofso, (Che fi dara feny’altro) chia, eT is

E le mofche levar mi so dal nafo 4o me gli vue di pofta ingoiar vit, t

Seguita Magorto a dolerfi della fua di(grazia ; poi fata rifoluzione d’é ty
cercar del Romito , falta dalla fineftra in ftrada , dove vedute alcune. kj
fo il giardino , infofpettito la(cid i] penfiero d’ andar cercando di Pi bry
ne va alla volta del giardino  ¢ quivi accortofi del ratto della fanci u
di yoler trovare coloro , che gli hanno fatto quefto torto, ¢ di, volergli tu &
goiar vivi. Nota che il noftro Pocta in qn ottava 84. ¢ ftato cri dy
ché s'¢ feruito della voce ia in tutte tre le rime , ma tal fottigliezza fb iy
tofto chiamare ignoranaa , perché {e bene & fempre la ftefla yoce , “—— 7 j

 

 
SETTIMO CANTARE: 371

fempre diverfo fignificato , perché la prima fignifica ftrada ; la feconda fignifica,
altrove , 0 moto da un luogo a un’ altro , ¢ la terza fignifica modo , guila , ma-
~niera , ec, E di fimili rime troverai altrove in quefta Opera , ¢ fempre le vedrai

    
  
    

 

_ lodevoli per I’ artifizio , pit tofto , che biafimevoli per ta poca avvertenza .

- AOL . Elclamazione , che efprime difgulto , 0 dolore. Latino Hei mibi ;
- CHIANTT, E’ una regione in ‘Toleana dove nafce vino buonifimo. E Vettu-

 intendiamo colui, che fopra alle beftic conduce vino , ed altre robe da un,
ogo all’altro ; a differenza di Vetvurino , che prefta , ed accompagna caval-
a lettighe , ec, a i Viaggianti . Vedi fopra C. 6, tt. 37.
| DAR fa caccia, Correr dietro a uno. E propriamente fi dice Dar /a caccias ,
_ quando i birri corron dietro a uno per pigliarlo .
git = GRAFFIASANT!, Bacchettone , lpocrito, E’ lo fteffo, che Santinfizza det-
igi to fopra in quefto C, ft. 68.
pit PVO? far capitale, Pud efler certo. Qufta voce Capitale fignifica lo ftato , o
Ill faftanze d’ uno: tale ba 10, m, fendi di capirale . Significa aflegnamento. Chi
yt del mio fn capitale detto fopra C. 2. ft. 7. Significa forte principale . Latino Sors,

i detta
i

   

,

yah

   

qs dat Greci cephalaion , ciot caput ; dagli Spagnuoli candal , che corrifponde»
niall noftro Capitale , ¢ Candalofo dicono colui , che ha gran capitale , cioé grandi
we = fallanze . U/ rale ba havuto la fentenza contro, ed ¢ Stato condennato nelle [pefe , ed 4

are cento fendi di frutti , e mille di capitale . Significa quello vedremo foto C. 8,

 

u@ 1.65. Qui fignifica pud credere , pud effer ficuro .
jah «SET c' inciappa . S’ ci mi da nelle mani. Se, c’ incoglie. S’ egli cafca ne’ mici
ei) Meguati

i) UI venga la rabbia , Giuramento imprecativo contro fe fteffo. Giuro di voler

yj far latalcofa,, ¢ fe non la fo , mi fottopongo a ogni maggior tormento .

ait 8" 10 to rintraccio, Traccia fignifica orma , 0 veltigio ; onde tracciare vuol dir

«at ‘{eguitare le pedate , ¢ per confeguenza qui intende: Se io lo ritrovo ; Traccia fi

iti dice quella ftrada , che fa il cane per la paflata della lepre , o d' altro animale

im fiutando; viene quefto verbo rintracciare , che vuol dir Ritrovare , ¢ rraccia-

jus) ecetcare, Latino vesticare . x

ynt © CHLAFACCIO . Elo fteffo , che chiaviftello detto fopra C. r. ft. 69. che i Sa

se nefi dicono pestio dal Latino pefu/us . 11 Conte Vgolino preflo Dante Inf. 33. Ed
io fent) chiavar ? nfcio di fotto all’ orribile torre ; cioe mettere il chiavaccio.

rit «= A QVELL Avia. A quella foggia. Inquellaguifa.

ull PARGLI uno sfregio in ful vifo. Fargli una ingiuria ignominiofa , fi come fono

iii Bl sftegi. Vedi fopra ©. 2. ft. 3.¢C. 6. tt. 54.

PAlabava. Intendi ha gran rabbia. Latino fromachatur , Che bava quell’
que. UmMOre vifcofo , che da per fe fleflo ca(ca dalla bocca come fchiuma, come fi vede
eS fe i cani arrabbiati , donde ¢ prefa la prefente metafora. Si dice ancora: Afi
: fs venir (a bava’; di chi mi fa entrare in collora , ¢ di chi noia forte,

La ML Ciel minaccia', e brava. Sgrida , ¢ minaccia il Ciclo. Vedi fopra C. 5. ft
ff Gx, che dice Rabbiofa,il capo verfo sl Ciel tentenna , che & quel minacciare il Cielo .
2 ‘Di quefto verbo bravare, che vien dal Provenzale il Varchi ne fa un lungo difcor-
ut fonel fuo Hercolano , ¢ lo giudica molto efprimente il latino obrargare . Catullo .
ty :
6

é Aaa z Gel:

 

 

anes

Ate

 
| 372 MALMANTILE >

t+
] Gellins audierat , patruum obiurgare. falerb § 5 SOV si Dayne
| St qnis delitias oe. 9 aut factret , et)
| TANTO di gla; Gola affai larga. Vedi fortoC, 16. ft, 18, ‘
| ce tanto ufata in quefti termini, é tote Fig i
NON ne farefti caso,quand’ uno ti pifciafe addofso, Non ti
j non t’ importerebbe quand’ uno ti pilciafle addofio ; ed intend: Sei
ne, ¢ codardo,che fopporterefti qualfivoglia grandifima ingiuria
ne. Vn'antico Pocta per voler efprimere uno fcellerato,¢ ingiuri
| ria di fuo padres dice:patrios mincerit in cineres , B Pittagora in uno de’
boli per dinotare il rifpetto , che fi dee portare alla Divinica,
non fi pifci in faecia al Sole. : itty
NON bevvi grofo, Non fopportai mai ingiuria alcuna. Ber
Non la guarda cosi per la minuta , ma m5 8 ogni ingiuria fenza
ne , fingendo non fen' avvedere. Tratto dal bere Je medicine , le quali
faporano , ma fi mandano git a occhi chiufi, » 0: Soa sos Say
| MI fo levar le mofcbe a’ intorno al nafo ~ Mi s0,vendicare dellvingiuri
cilitd , Omero nell’ liiade La preftezza , colla quale un Dio fa tornare indietroi
colpi avvelenati contro a un’ Eroe compara al cacciare d’ una mofea , che fa las
Madre dal corpo del {uo figliualo. © J oenetige
FAR ilc,. .rofso auno, Galtigar’ uno .. Tratto da i Pedanti, i quali )
i ragazzi perquotendola in ful ¢.. 5 ¢ gliclo fanno roffo con'le)
fopra C, 4. ft. §1. i : 1 a 0b if Settee SIR
oe . Subito , Viene dal giuoco di palla, che fi dice Dar di nr.

  

 
   
  

 

 

fi da di primo tempo , cio¢ avanti, che Ja pallatocchi terra «Lavinond seftigio«
INGOLARE, E' lo fteflo , che ingollare detto fopra Cas ft: 63, ¢ vuol die mat:
dar la roba gil: nello ftomaco . ‘ ol Owe
STANZA LXXXVIL, STANZA LXXXIX:
Ma dove col ceruel fon’ io trafcorfor Quel detia Cella del Romite eiil
Pits buesdi me non ¢ forto le frelle , Ove trovaede if ibrvepuenia
Perchinnanzi ch'io babbia pref Porfo Intana dentro, e non wi fcongeniing)
Vio (come fi fuol dir) vender ta pelle 5 Fruga,erifrugainquaye iia aea
Fatei ci voglion qua , perch’ il difcorfa Sgomina cia che v'édafommoywinty
Fuor che ai Senfali non frutto covelle, M14 tutto in vanoyondeghial,
E mal per chi ba tépo,e tempo afpettas Sen efce con le man pieae ns yente s
Che mitre pifcia il can,ia lepre sbierta, Ma dieci volte pit dimal talento,
STANZA LXXXVIIL STAN ZAULX Xia ly
E peri prima , che 4 vila a gamba « Entra neh bafeayeogni ,
Vaa fuga mi fuanin divconcerto E in fammea ne cored pax
A cafa Pigolon vuogt ix-di gamba s | LB wedde' 5 fanz ain ‘

Che vi [ard coi compliciidel certo, vorrde pigiato effer bidval far decent.
ares fice ribinfe

Cost conchinfo, correch? diff gamba y 4 «| Onde nelifine alf

N

Ne np bp cw @ eee FZ cee Fees SF ee hese KS

E come un bracco taper-quel deferto Che pur mol vendicar segtandiane

Tutti quanti quei benghi a uno 4 uno Cosi v’ ar rivera. po’ pai in
Cercando.s'ei vi fcmopee,o fentecaloxne 1 Seveifuffe( di i ne

K aia MS

 
    
   
    
 
 
 
   
   
  
 
 
 
  
  
  
 
 
 
  
   
 

  

SETDEMO CANTARE: 373
° STANZA LXXXXIL;
Poiche Brunetro , ¢ le fue camerate

Pagaron L offe ,( il quale affai contefe,
Perche le gole lar. difabnare
Gli eran parute meeety Spefe)-

» . Partiron,, ¢ poi dopo-altre fermate y,
Sie SUnapdeledee

  

i di quanto have 10 E giunto a cafa,ringrazianda il Cielo
Ve pit 5 ne manco ne fegui  effetto . Entra in fala ,¢ di pofea fa un belo.
yi STANZA LXXXXIUL
Trovan Nardino acor di male oppreffo,
E sbietolar Jo veggono ancor Lui,
L' Alhante, che porgevali horxata
faper. 9 me men per cui, >, Purine faceva lafua quattrinata
Magorto lai lamenti , ¢ fi,mette a ceecar di coloro., che gli havevano ru-
t Ja Figliuola Ȣ nen gli trovando nella Cella del Romito , ne in alcun altro
ricorfe-a gl’.incanti; co i quali coftrinfe tutti della cafa di Brunerto a pian-
3 onde Brunetto con i compagai arrivato a cafa fubito comincid , ed
icompagni a piangerc. ;
fon’ ia feorfo coi ceruello’ Che armegg’ io? Che giro.io? Che frenetich’ io?
SD pWVOMe fetta le frelle it pit buedi me... Ao {ono il maggiore ignorante che fa nel
Mondo Vedi fopra C. 6. flan. 98. Sarr /a Luna ; i Petrarca..,Arda.s 0 mora,o
dangnifer sun pin gentile Stato del mo non ¢ forte la Luna, ‘
thsi Bi: da.pelle dell’ orfo. prima di pigliarlo. Fax aflegnamento fopna una cola,
che ancora non s’ ¢ confeguita , ed ¢ anche molto dubbio/o.Ji confeguirla ..Eflen-
ido anati cre Giovani per ammazzare \un’ orfo'si| quale faceva molto danno ,

prima che atrivafiero.al luogo dove foleva trovarfi I’ orfo, fi fermarono a un’
Bieria ed havendo aflai ben mangiato , ditlero all’ Ofte., che lo paghercbbono
son denaci: del danatiy a, che haure bbano dato loro Je, Comunia per V’orfo ,
\uhe walevano ammageare 5 ¢d.amapifi verlo dove ftavala ficra , fubito, che las
veddero fidiedero a fuggire , ¢ uno di loro fali fopra ad un’ albero.s 1’ altro {cap-
- PP Viayyediiltcrzo fu lypraggisnto dail’ Oxlo ei quale bavendolelg,caceiato fot-
~todtinfran bene beng sdi'por gli.accoflo digrife ail’ orecchio,, ed intanto quel
mefchino fe neftava come monto, fenza muoyerf punto ; ¢ perche J’ orfo naw.
cralmente ({econdo. dicona alcuni.) quando: ee ede,che.! apimale da Jui. afaltato
fia morto , non gli da pit fattidio , credendo che coftui fatie morto,fen! andd ,7¢
‘€Gluiideyd fi 5 ed ay vind vero Ja Citta cutto mal coacio .. Quella y.che,era fa
_ litodimaulimalbero:iccle-s ed.accompagnatoli coneflo , gli domandé quelche,gli
_ havefleidetto.? orfo nell’ orecchia y sd egii rifpole = Mijha derto , che io non mi
“fidi pili difimilicompagai come fei G28 Che 19. now veoda.a, pelle.dell or(o. fe
jeteemeete ho prefo,.: B.da quefla.novelia-habbiamg il prefeuce proverbio, che
idictlanche : Vender t' uscelio in fu la frafcas L Geesi ditiero : Anrequam pisces

  

    
 
    
 
    
    

 
  

BSELEAL:

 
   
   
 

 
 

  
     

smuriam mifces.. J .
- MAI frutd covelle. Non fa d utile alcuno , Covelle & voce romagnuola.e vuol
dire. Qualcofa , E’ poco ufata nok Fiogcatino fuor sie da qnalche.consadino . Il
TANS : : valore

ERLETELLE

=

=

 
    
  
     

374 MALMANTILE ©

valore di quefta voce é affai copiofamente efpreffo dal Copetta inun f
fopra i! non covelle , Nel Decameron trovati Cavelle per lo fteffo_
Lat. quod velles .

 
     
     
 
   
   
 
 
 
 
     
  
  
  
    

epee ae!
E’ mal per chi ba tempo, € tempo a/petta,Che mentre, ec, Mal fa colui ,che ha
do P occafione pronta perde il tempo , ¢ non Ja piglia,perché mentre si
J occafione fugge: E’ noto il verfo : Fronre capillaca poft i
verbo sbiettare !’ habbiamo anche fopra C. 5, ftan. 30, Adentre il can pifcia,
fe ne va. 1 Latini ditfero Semper nocuit diferre pararis ; fecondo Lucano , di¢
forfe Dante nell’ Inf. C. 28. diffe: : 7
Quefti feacciato il dubitar fommerfe $
In Cefare affermando , che il fornite
Sempre col danno L) attender foferfe .
PRIMA che a viola a gamba, ec. Incende prima che d’ accordo fe ne
Viola a gamba ¢ il baflo di viola, Fuga é {pecie di fonata a capriccio,
vuol dir Suonata concertata con diverfi Rrumenti , ec. Econ quefti
tende quel che s’ ¢ accennato . }
INT ANA, Entra dentro . Si ferue di quefto verbo anche forto
25. fe bene ¢ improprio ; perché vuol dire Entrare in una tana , 0 buca
rebbe intanare una volpe , un taflo , un granchio, ec, cutcavia ¢ pur
to come nel prefente luogo . ei
AMO. Niuno. Dal Lat nemo. Voce oggi ulata folo dai contadiai ell
noftro Poeta fe ne ferue anche forto C, ro. ftin. 37. in bocca d’ un
SGOMINA, Si dice anche (gombinare ,( contrario di combinare ,

piare , unire ) e vuol dir mettre in confulione 5 o fortofopra tutto | che fi

maneggia . Lat. perturbare, .
DA fommo aime, Frafe latina, che fignifica Da capo a piedi: Dalla fomal-
ta della cafa , fino a i fondamenti di effa : Petrarca Trionfo della Fama, ene
Onde da imo Perduffe al fommo t edificio fanto , + ae
LE man piene di vento. Cio’ fenz’ haver trovato , 0 conchiufo nulla. Nellie
Scrittura . Et nibil inuenerunt in manibus (nis; che diciamo ancora Con le troment
facco Ter, diffe Infetta re. ee |
DI mal talento, 1 collera ,e con volonta di far del male y ¢ di vendicatl:
Varchi Stor. lib. 4. Erano verfo i nobis di maliffimo talento , ne altro per manvweit:
gli alpettavano , che quel che avvenne . BY frale ufaca dai Boccaccio. =)
NE cerco per mars e monti , Quefto detto iperboiico é uiacutimo per efprimett
Ne cercd da per tutto; Viene dal Latino. ee
SENZA metteria in forfe, Senza dubicar pit. Senza metterla in dubbio . Dd
mettere in forfe fece Dante il verbo inforfare , feguivato in cid dal Petrarca +
IL pigiato effer (ui al far de* conti . A confideraria bene} offefo', e-beffaroem
folamente lui. Quattro giuocano infieme’, tre vincono , ed un di loro folamet”
te perde ; quefto tale fi dice é/ pigiato , cio® quello , che ha gli altri addofio , et
cui G fpreme il denaro. Bs’ intende in ogui cafo, che ja difgrazia tocchia ue
folo della conuerfazione , ¢ tutti gli altri habbiano (oddiienioasl “, outile dal
danno di lui. : Ae:
POPOL in quel fondo, Vedi fopra C, 2, tian, ,3.

 

 
 
 
  

   
 

y ‘"SETTIMO CANTARE.: 375

_ BeANNO avanga. Vanno fecondo il defiderio. Ex animi eins fententia ille cr
wnt. Noil habbtamo da i Contadini , che quando fi rende loro facile i! lavo-
‘Ja terra con la dicono ; 4 lavoro va 4 vanga , ciot bene ,¢ come fi de-
« Bvangad ftrumento ruftico fatto a foggia di pala,ma di ferro pil
» © pill acuta , del quale i contadini fi {eruono per rivoltolar la terran .

edi fopra C. 6, fan. 69, al verbo impiallacciare., Columella lib, 3, 1a chiama do-
ira ,¢ perché quefto nome vuol dire pil tofto la piaila, forfe Columella inten-
ee flrumento ufato a fuoi tempi, che faceva fopra alla terra I’ effetto che
t pialla fopra il legno , ( come ¢ hoggi la marra {copaiola , della quale fi fer-
_Uono i contadini per ripulire ,¢ radere i bofchi di {cope per difporgli alla {emen-
_ ta della fegale ) perché,fe volefie dire la vanga,haurebbe detto acuta dolabra fodi-
%,¢ non abradito: E la vanga fi trova bipalinm , in Varrone: /d priss bipalio

 

  
  

  

  

STVMMIA di furfanti , Scelleratiffimi , ex omni vitiorum colluvione concreti .

i ‘Stammia ,fcbiuma , 0 {puma , & quello efcremento , che nel bollire una pentola.
* piena di carne , ¢ di acqua manda alla fuperficie , il quale fi butta via , perch ¢

th Imm ia ; onde fummia di furfanti , 11 peggio,che fia nella furfanteria . i

hay! ( difabicata. Lat. gurges , Cosi diciamo di colore , che fempre mangia-

yi —— fi veggono fazzi.

“an paruti cari per le (pee. Exa parfo all’Ofte,che coftoro haveflero man-

ot 9 troppo. D’ uno che fia buono a poco , ¢ mangi aflai , ¢ che vada a feruire
od | }; Beli ¢ caro per le {pefe; ¢ intendefisfe gli da pil del dovere ,¢ di que! che
wr en fua abilita a dargli {olamente mangiare , fenza dargli danari per prov-
a ¢. li Lalli nelia fua En, Ir. C. 2. ftan. 130.

Sit Non vagiio un pel ; fon caro per le {pefe . :
Dit! DI pofta fa un belo Subito comincia a piangere a tn Vedi fotto C.9, ft.21.
ig) © SSHETOLARE., Cio piangere. Vedi fopra C. 4. ftan, 16.

AST ANTE, Intende colui,che aififle al feruizio di Nardino infermo, 4fan-
oii! 34 fi dicono quei Seruenti ,che affiftono a feruire gl’ infermi negli Spedali,¢ guefti
it Aoglion efler chiamati dalle perfone comode ad aififtere alli loro infermi , ¢ pe-

10 qui lo chiama col nome 4’ 4/fante, (upponendolo uno di quefti tali .
in ORZ ATA, Bevanda rivfrefcativa fatta di feme di popone , orzo , ¢ zucche-
yo 79 deniflimo peiti ¢ liquefatti con acqua,¢ paffati per (tamigna , fi da per Jo pib
ra febbricitanti;detta anche /arrara come habbiamo veduto fopra in quefto C. ft. 12.
NE faceva la Sua quattrinata , Cie faceva la {ua parte del pianto,

Ae STANZA XCIV. STANZA XCV.

oe /ardin vede colei bell’ ye verzofa Mettere pur cosi le mani innanzs

ro Com’ appunto ? haueva nel penficro 5 ( Rifpond’ ella) Signer per non cadere,

i E dices Benuenuta la mia [pola , AMentre ,temendo ch’ io non mici feanzi,

1) Ko 9 tt piacere a fe da Cavaliero . Specorare fi bench? é un piacere:

we 4a voi piangere ? ditemi una cofa Ch io mi vi levi, ditems , dinanzi ,
Bs Koi ci venite a malincorpo , ¢ ¢’ vero? Che voi non mi porete pix vedere 5)
RD. Non vogliate rifponder che ¢' non fia , Senza darmilaburla,ch’ io m' acquicta,

" Perche

#

 

bé vei mi direfti una bugia .

E Senza replicar do voita a dreto.

STAN-

 
      
 
      
      
 

+
  
 
 
 
 
 
  
   
   
 
 
 
  
   
  
   
   
   
   
   
   
  
  
  
   
  
  
 

376

STANZA XCVIV

Ne fofopra la man non volterciy
j Che Pandarese lo far mi fontitt nia,
| Eben c'al mondo t6\fia come gli Bbreiy
| Che non han terra fer mayo patriaalcuna
i Andrd penfardo incanto a farki ries
Per veder di trovar migtior fortuna’,
ct fond » come diceva Afona Berta: vr

t Chi non mi vuol fegn’e che non mi merta, - Pero non vogliar
? STANZA XCVHL: %

   

Ella foggiunge, ed Egli ribadi/ee j
| Ella non cede , ed ei rifponde a thiond.; © ” ogmoraincafa 5 fuora y
Pur gliacquiera Brunetto,e al fin glsunifee, _ ( Perch fempre fi fnmena,
Sicché 2 un 0 altro chiedefi perdond} « ©) Hantioa tener agli ovcbi |
Nardino vede la Fanciulla ; ¢ la trova per appunto'comie fel" era’
| ; ma vilto che ella pi angeva le dice,che dubita , che ella’ fia venuta mi
ed ella gli rifponde , che dubita,che pit tofto egh non Ja riceva vo!
pra quefto feguitavano a contraftare, ma Brunetto al finé gli raj
tutto quefto ognuno feguitava a piangere . i ¢
j VOI ci venite a matincorpo, Voici venite malvolentieri ,'¢ con
| foddisfazione;cétra’ flomaco,cOtra voglia,fatrone'taa fola parol: come
METT ETE te mani innanzi, Queito certnine ci ferae per efprimete
accufa un’ altro di quaiche mancamento , del quale merita dj effet 4
per efempio: I ragazai dello Spedale degi*imhdceati , i quali fi
fieno tutti baftardi, in occafione di contraftare’con alcri ragazzi ,
giuria che dicano a quelli ¢ , 7 /ei bastardo , pecche non fia detro a”
fio fidice: Aderrer le mani sananzé + © vi fi aggiugne anche ; per
prevertere,occupare . -
NON mi ci fanzi. Non mi fermi in que(ta Cafa per fempre .'
SPECORATE . Piangere . Diciamo be/are per piangere per 1a
che ha cobbelar degti agnelli , € delle pecore certo planco Jango’, che!
rei bambini , come accenuammo fopra C. 6, ttn, 22. e'da queito
Specorare in vece di belare , ¢ s intende piangere . ; nist
St ben ch’ é un piacere. Tanto bene , che € un gufto a fentirui, ¢ vedertis
NON ne volterei la mano fortofopra , In quetta cola io fouo imdife;
poco ny importa il faria,o non farla, Vicne da i Latini che discvano
Ne manum quidem verterem , + tye
ESSER come gis Ebrei , Ciok non haver luogo che fia\fuo propri
ra il Poeta medefimo dicendo > Wun bo terra jermayche intende terta’y
abitazione fermata , ¢ ftabilita per-lei , che per altro Lerra* termat
paefe , che non Ifota di mare’, Lar, continens’, + syageael
VOI vi levate in barca , Vou cntrate in colicra, Vedi\fopra Ce
dice anche abarcare 5 ¢ 2 iracondo , o vero facile al? iray chek
» derocholor& dette da NOt Aacmo as poca levacurascioe che'ci'vudl poco |
rein collcra .

~ ida non per
Co

 

   

a
3
 
 

SETTIM/O CANTARE; 377
Qui vuol dire fofferenza , 0 pazzienza , che per altro’ Flemmas
accennammo C. 3. flan ages % ho
$4, Iraconda, Vedi fopra C, 1, Aan, 29, Alewni critici hanno
dra quelta refs ; giudicandola rimafaifa in'tigaardo-dell’ 5s , dolce di
‘studardi ai/perte/a.,¢ dell’ , 0 , Jargo diquelle , ¢ fretto di quefte ,
voglio quictare , ¢ difendere il holtro Poeta col Rifcelli, 0 cons
‘non mi fon'voluto pigliar la briga di vedergli come non neceflaria ,
bem loro un' efempio d’ Autore claifico il quale dice.
Hag. Lb 9.30 La verginellae fimile alia rofa ~
9 5 Nei bel giardin /u ta nativa {pina ,
Mentre fola ,¢ ficura fi ripofa
Ne gregge , ne\paftore fe le aunicina
, Li aura fuave , el’ alba rugiadofa, ec,
Eg con quefto efempio.( il quale fia per regola , 0 per licenza) di fal-
ire il noftro Pocta , ¢ quietargii , aucor per J’ altre, che hanno offeruatese fopra
ftan, 13. Rofa  prola , ¢ cola ; ¢ foro in quefto C, flan, 103. Spofa , cola,

BADIRE , Ribattere , conficcare
Vedi fopra’C, 2. tan. 79.
DE 4 two, Rilponde aggiuttatamente, ed a propofito di quel, che fi
verbum audit , tale dicit . Si dice anche Rifpondere per le Rime. Las
iilitudine € tracta dalla Mufica; la feconda dalla Poefia ; B allude al co-
he de’ Poeti che indirizzando /’ uno all’ altro Sonetti ,e¢ proponendofi que-
levano,e le {cioglievano in altra eguale compofizione teffuta delle
t eliine rime , il qual coftume venuto dall' antico , fi mantiene anche in oggi .
ful - Sl fmacica , ¢ B cola , $i manda efcremeati dal nafo ,¢ lagrime dagli occhi
oi M ‘piaato, che/moccicare vuol dire mandar fuori mocei , che € quello
_ eleremento del ceruelio , che efce dal nafo detto da i Latini mens.
PEZZVOL A. Pazzoletto , 0 Moccichino ; ed & quel pezzo di panno lino,che

  

 
  
 
     
     
   
  
   

dall’ altra parte un chiodo, Vale per re-

 
     
 
   

 

i fi lo di fe per ufo di nettarfi i nafo, —
sl) py, SEANZA XCIX, STANZA ©,
fs wa in un continuo pianto , E veduto ch’ ell! ¢ tra buona gente ,
; om - ees: 3 i &
| Phangona i fer wise piangon gls animali, Moglie a! un ricco, € nobil Baccalare,
i Onde th guazzo per terraétale,e tanto, Eche Sok le puo mancar niente ,
iB Chee Portan tutti quanti gli frinali . Per ch’ elinein unacafacome un mare,
, iamoa Magorto, che fra tanto Won vi fo dir ¢ ei gongola , e ne ente
a Per faper quel che fia di quefti tai, Contento grande , € guffo fingolare ,
i! Ez dowe 1a jua figlia fi rnrovi , De-modo ch'ei fi pente,affligge ,e duole
a = Hes fato ab confuse incanti nuovi. ao —_ ba fatto ye rifarcir lo vuole,
RS ae abled. 5) o STANZ yt
Pi un fae cogno , E poi che dentro pitt non ne puo porre 5
& Sumnonipencace, a nae foot ; Biphedi pte "Luo afpetto ¢ molto brutto
lage ¢ 0,enecomincia a corre, Si lata, vipilifce ye raffaxcona
a ~ Darando fin che hebbe pieno tuto ; E rimbeuilce tutta ta perfona ,
0 SS AAMERabs he ig

Bbb STAN-

 

 
  
  
  
    
  

 
  
  
 
   
   
    
   
 
  
   
   
   
   
 
   
    
   
  
  
 
 

378 MALMANTILE; =.

 
 

. STANZA CIL
E prefa addoffo poi quella {ua cafsa ,. Che al fuo ver
CL? ¢ tanto grave,ch’ ei vi crepa forto y Mirando in rif
St metic 1a wa , @ prefto fene passa. ;  Everfad pomiin
Ow ¢ la figlia s ¢ il flebile raddotso Poi ft

Mentre che coftoro piangono , Magorto per via-de’
& la Figliuola , snaolcendonche ella é bene aipentt fi. %
folue di regalare gli fpofi d’ una quantita grande di pomi d’ co + A008 ;
to, € Cosi fece, ed ap arrivo oo in cafa degli {pofi tutti ceflarono di pia
GV.ALZO . Luogo picno,d’acqua , dove fiipofla guazzare , ciot pall
picde {cnza navilio , che noi dal Jatino diciamo, vadoy o gaado; onde ilk
Vaaa cost detto perché quel luogo dicevai Yada Volaterrana's ¢ guadare x
¢ paflare : Ma i piglia ancora per ogni grande ammollamento , che fi fa
neile cafe , 0 altrave in {ul {uoloy come € prefo nel prefente ye: aque
calo viene da guazza , la quale cade dal Cielo , altrimenti detta 4 at
prvina come gelata diffe Dante dal Lat, gelu ; ¢ non da guazzare il flume; Sefore
ie non voleflimo pigliarlo per parlare iperbolico , come ¢  adoperare
per paflar-tal molle , che é in quella lanza. 0) oop oe
8ACC-ALARE. Huomo di fima . Vno dei principali del paefe
anche Barbafforo, Baccalare da Baccalaureus fi dice colui , che nelle
acquiftaco un grado proffimo al Dottorato , 0 Maeftrato detto altriments I
ziato ; il che ula nelle Fraterie, ¢ corrottamente lo dicono Saccelliere
grado fi ritrovava anche nell’ ordine della Cavalleria . :
E una cafa come un mare. Cioe {empre picna di zoba ed abbonda
bene , fi come il mare,che ¢ immenfo , detto percid. da Omero atrygi ih
non ha fin y ne fondo , Si dice anche Vna cafa come una Dogana, a
GONGOLA. Greco cancharei » Giubbila :-Si rallegra 2-5i
certa allegrezza interna . B’ voce ufata afiai dalla piebe. : Re
KIS-AKCIRE , Riftorare ; Rifare il danno ,0 ricompen(argli qd havergli tenn.
ti tanto in pianto. E per altro quefto verbo ri/arcire yuol dir raffer
vifto fopra,C, 6. flan, 52,
cOGAO, E’ una mifura,immaginaria di vino , che contiene dieci barill
quale corrottamente fi dice Cento; Deriva.dal Lat. congims . Onde Bigonet §
da un Lat, bicongins ; a Piftoia percid dette pili profimamente all’ origine Bite
Gio, Villani lib. 8, rubr, 116. Valle lo fraio del grano in Firenze folds 8. Leagan del
mofto in certe parti meno di (oldi 40, Ma qui é prefo , come & coflume », per une
certa forte di cafla , o pid toflo cefta fatta , ¢ contefla di firilce dal
corbelli , ma é di foggia Junga ,.ed ha il coperchio , come hanno Je, c
S/rafazzona . Si ripulilce ; Si rinfronzilce . V.cdi ops. Cy 2 flange
fi rifa ; fi rimette in fazione , in abito ; fala Pre tecr yi la bella
nicra . Gli autichi dal Provenaale diflero agenzare , cio’ &, ‘
ce Gente ufata dagli antichi Tofeani ancora per Gentile, Br
voi , donna gente M’ ha prefo.amor , non é.gia maraviglia . L
il fenno , ¢ li gents coraggi.. I Beato lacopong diffe che la 2
geaz@ , clot non rilciacqua , come fpiegd alcuno , ma rafaxzuna , ri
'

 
 
 

SETTIMO CANTARE: 37
forto pet lo fverthio pefo j ed'il verbo crepare , che»
, come vedemmo et: flan. 18.qui¢ nel fuo vero
ire , perché quella gran fatica pud cagionare I’ allentamento .
0€ fi cava di Se. ; fila errcor tact propriamente il piles
eflendo i] noftro cappello pili tofto il pera/us .
Fras ) STANZA‘CIV.
2) Eperché qualfinoglia-donniccinola ,
2 Lpauen: Porta la dote , ed il corredo appreffo ,
digni cofa,.—_—*. Acciacch’ in quella cafa la Figliuola
Sdegho*toritmenté ha fpento; ~~ ~Polfa moftrar a baker qualche regre/so,
<1 SS Ne che gli abbin a aner quelcalcioingola
“C’ un piccolo ne anche v’ habia meffo ,
‘La vuol dotar conforme al grado loro
pss ‘Con quel gran-monte di bei pomi a! ora,
SM Evopny shaves Ps ANN FAs +P; .
llor brillando con Brunette © Edegli poi al fin com ogni afferto
gr axiejefanaratanccoglitza; °°\° Riker? rutti, é volle far partenza ,
inato un grande ye bel banchetto” > Ledandofi del furto del Romita,
“ar le noxze in fua prefenra, ‘Che si grand’ allezrezza ha partorito ;
orto fi fa. conolcere per il padre della, Spofa , ed aificurando Pigolonc , ¢
i perdonato , ed’ haver guito’, che fegua quel parentando , colti-
vl lla caffa piena di pemi d’oro. Si fanno perd di nuovo gli {pon.
il banc! ¢¢ Magorto fe ne corna al fuo paéfe, dando molte lodi a,
per efler’egli ftato autore di cosi gran ‘conteato . E qui con Ja fine de]-
la‘raccontata dalle Pate a Paride termina il fettimo cantare ,
_  AMAN vote, Senza nulla in mano : cioé fi mariti(enza dare dote aleuna ;
we} «= CORREDO , Quegli arnefi, abiti , ed altre robe j’che fi'danno alle Femmine ,
_ Oltre alla dote , quando fi maritano , che i Giureconfulti digono Parapherna dal
yep Greco Para , che vuol dire oltre , ¢ pherna , che vuol dir dote,
git HAVER regrefio Termine legale , che vuol dire haver azione di domandare ,
_ Sontro\a tino , per rifarfi del pagato ad un’ altro; Vedi (otto C, 8. ft. 42, E co-
jj MUunemente fignifica un certo ardire , ed autoritd fopra ad una perfona , o fopra
i i {uoi beni ed effetti: 1 rale gli ba prefo regre/so addefvo , per intendere ha prefo
yg) atdire fopra di Jui.

a gli abbin a haver quel calcio ia gola , Non habbiano a poter rinfacciar-

  
  
    
   
   

   
     

 
  
 
 
  
 
 
 
 
    
 
  
 
   
   
   
  
  
 

  

uf om ; “ oe non v’ habbia portato nulla: Noa habbiano a ha.
‘caufa di conculcarla , ‘

| ELANDO . Giubbilando , Vedi fopra C. 2. ft. 69.

4 CACCOGLIENZE . Vedi fopra C. 1. ft. 34.

le mgxe. Cioé di nuovo fi fecero gli {ponfali , e folennemente ff
di fpofi.

   
  

‘ ‘Gohan

FINE DEL SETTIMO CANTARE:
Bbb 2 OTTA-

 

  
 

tk
 

 

   

    

    

  

&3 ARGOMENTO.
oY
D’ un! avventura grande é po

OTTAVO.CAN
a ; J 3
SERS]
Dalle fue Fate Paride veftita. ous. a ae
Vede lagalleria di quel? albergo;\.\ a
Seco eer eenin

    
      
    
    
   
   
 
   
  
 

   

Ond’ ei pigliavicenza , e voltaiLcergo.,

 

  

      

 

a

5 Vien Piaccianteo condotto al Generale y.... »

ey Che non glivolle far ne ben, nemale,

BAP 25 CEPR CUE MFI CED
N75 h
ENA AAACN |
STANZA STANZA THM. Ta
Orrei , che mi diceffe un di coffora, La notte, diffe y¢nn wafo dit
Che giofran tusta notte per le vie: y Che verfa affronti, rifichi, € iy.
Che gufto v’ ¢,perch' a ridurla a ora Pero, he nel:fiuo sempo shucan fuora kg
Non v'é guadagno, ¢ fon tutte parries Tutes i ribald ydadrt , e rompicali ;

Poiche ( lafciando , che enon ¢ decoro ) Onde fia ben'riporfi di buon’ boa, !
L? aria cagiona cento malattie, E dene efempio Lhuom pig is al iy
Mille difgrazie poffono accadere , Che  unds loro al piis vale. at T
Mille malanni, Diauoli , e Verfiere, E pria ch’ il. fol sr amoncé fe ripome thy
STANZA IL. STANZA LWe be
Sapete,che e’ s'inciampa, ¢.che e’ fi cafen, Edegli, che at un. mondo affai pitt vale t
Si puo in cambio a’ un’ altro effer'offefo , Sta fuori tutta notse,o dineth, Pa
O dar mun, fet’ bai monete intafca , £ gira al bmo come un’ ani 4
C alleggerir ti voglia di quel pefo 5 hie
Maca: qual ma fi pudcorrer burrafca, t
Perd vi ginro, chiio non ho'mai intefo ny
La fin di quefti tali , € tengo a mente de
Quel c'wn tratto mi diffe un buom valéte, to modo, che non ve da le
STANZA V. » DSR hebeR Cy
Perché le fon tutte cofe provate , Come al Garani quand! a gal K
E vere, che non v'? {pina ne offo , e-4ndato era la norte n
E non fi tronan poi fempre le Fates Che , mentie oi by
Che vengano 4 leuarti il mal da doffo , Da effe ebbe un fanor di &

 
 

 

   
    
  
  
 

2

OTTAVO CANTARE 381

Pocta ifeguitare a narrare quanto avvenne a Paride s’ introduce col
che nocumento fia ’ andar fuori di notte , ¢ che perd fia cofa da,
, dente il'non confiderare quanti pericoli fi poffono correre ; Ed
nigliando la notte al Vafo di Pandora conchiude , che fi dourebbe imparar
i polli , che vanno a dormir fubito , chee’ s'é ripofto il fole , ¢ cosi sfuggires
te le disgrazie , perché non fi trova fempre chi liberi dal male , come avvenne
a Paride , che dalle Fate fu liberato dal pericolo di morte .
_ GIOSTRARE. O armeggiare . Mctaforicamente s’ intende andar girando , o
pafleggiando fenza faper dove , 0 fenza fine determinato , che fi dice anche anda-
» Oagironi.
ARIDVRLA « oro. Per ridurla alla. conchiufione. Vedi fopra C, 3.ft. 43.
e malanni Diavoli ,e Verfiere. E' un modo di dire affai ufato in fimili
aioe per efprimere poffono avvenire tutte le forte di difgrazic .
_VEKSIERA , orl infernale , che dalle nottre donnicciuole ¢ intefa per una
fla moglie del Diavolo . Forfe viene dal Latino Ver/usia , che vuol dir
‘malizia ; ¢ fi dice Verfiera un ragazzo maliziofo , faftidiofo , ¢ infolente, ma
Spill veri » che venga dal Latino aduer/arins,col quale nome ¢ difegnato il
7 nella ferittura., ddnerfarins nofter diabolus, Petearca .
5 1 ; Si che anendo le reti indarno tefe ,
U1 mio duro avverfario fe ne fcorni,
Da aduerfarius nelio fteflo modo, che 1 Francefi fecero aduerfaire, cost i noftri an-
: iz i's Auuerfiere ? anuerfiere , ¢ poi finaimente /a Verfiera . Ll Beato lacopone da

 
     
 
    
 
 
 

 
 

i

 

   

if

Hl canto 62.

ih “A Lo nemico ingannatore

oe o8aiiin::; * Anerfier de la Signore.

xs) Eecant2r. | Fata gli anerfere venire ,

8 ‘ Chel degian accompaguare ,

an Nell? ufo dicefi Far la Verfiera , fare il Diauolo, e peggio ,

ib  INCTAMP.ARE , B il latino ofendere. Vedi fopra C. 1. ft. 13.

i OT ASC.4 . Quella facchetta, che fi porta comunemeate appiccara agli abiti per
nfo diyportar roba necellaria alla giornata , come denari , ¢ fimili da’ Latiai detta

it Pera, o Zona. 8

/ @ALLEGGERIRE di quel pefo. Cio portar via i denari, ¢ cos} alleggerirlo del

', pelo, edelia noia , che per quello gli veniva. : :

a MANC A in che mo , Cioe {ono infiniu i modi. Il termine mance in quefto ca-

oe a0 sufato ironicawente , perché s’ intende : Vou mancano s modi .

a _ CORRER burrafca , E termine Marinare(co , che fignifica Correr pericolo , ed
in queflo Ggnificato ¢ prefo comunemente , {e bene berrafca vuol propriamentes

dire follevamento di mare per il cattivo temporale di venti , ec.

$ & VASO di Pandora, E) nota la fayoladi Pandora , 1a quale fauna Femmina. ,

( che Giove fece fabbricare da Vulcano , ¢ darle.in dono di ciafcuno degli Dei je

 

‘ parti , affine di farne innamorar Prometco ,.¢d indurlo ad aprire un va-
7 fo pieno di tutti i mali , che Giove haveva dato alla medefima , che lo donaffea
jf Promotco » che vuol dire Prevvidente ; che antivede , per vendicarf dell’ ingiu-
, tia da eifo fattogli quando rubo u fuoco celefte , ma non!’ havendo Prometeo
oe voluto

 

rie =
*

   
  
   
   
    
   
   
     
     
  
 
    

   

382 MALMANTILE ©

voluto accettare , lo prefe Epimeteo fuo fratello , 4
fatto , il quale |’ aperfe., ¢ vennero fuori tuttii mali, che
quefto ¢ i] vafo , che il Poeta intende nel prefente luogo , ¢'
ni nel fecondo capitolo della pefte dicendo :
To leffi gid & un vafo di Pandora ,
Che n’ era drento il canchero , ¢ la febbre ,
E mille morti , che n' ufciron fuora

 
 

Orazio lib, 1, Ode 3.
Poft ignem atheria domo
Subduftum , macies ,& nova febrium
Terris incubuie cobors ,
La favola , e raccontata da Esiodo . é
RISICO., Riftio , 0 rifico dal verbo arrificarf? , arrifchiarfi ,o d
vuol dire Efporfi al cimento , 0 avventurarfi a qualche icolo. In‘
Rifco fignifica , rups pricipizio , luogo pericolofo . Cie, fe bene mi f
quam in diffictle , & fcopulofo loco verfer , rificofo . “he a ;
TRACOLLI, Da tracollare : altrimenti barcollare ,che & fm
i] Latino metare, o ritubare ; ¢ qui vuol dir Difgrazia,, o pericolo, by
ROMPICOLLI . Huomini ; che configliano , o inducono altri a far 2 |

Latino in omnem audaciam proieéti . A a Cu
T#STONE . Moneta Fiorentina , che vale tre giuli,o paoli, 9 | 4
VAL pik a! un mondo., Quefta iperbole fignifica non vi ¢ prezzo, Ta

Star difcoffo un mondo, difle il Bronzino nelle rime burlelche ; cio Me

Spario. u

CERCAR di Frignuccio, Cercar le difgrazie. Andar incontro a’ pores
Frignuccio dalle noftre donnicciuole é prefo per il Diavolo, e diciamo ane ny
cercar il male come i eMedicr, I Latini in quefto propofito diflero; Camarinam me \ 4s
xere da una pianta ,.4a quale ha le foglie cosi fetenti , che movendole , @tocta | Mi

 
  
  
 

dole lafciano un puzzo terribile : o forfe da una palude detta Camarina! do
cina al caftello detto Camarina in Sicilia , la qual palude , perché cagi \
detto Caftello la pefte , i pacfani domandarono ad Apollo , fe era bene Q
re detta palude , ¢ I’ Oracolo rifpole : Camarinam non effe mouendam 5 I
fatto poco conto di detta rifpofta,vollero feccarla , ¢ n’ hebbero il gaft q

i nimici paflando per quella palude gia fecca , entrarono nel Caftello, €
“TW bella proua . A pola ; ¢ I addicttivo beds s*ula in quefi ca pes eal
IN bella proua. a; ¢ I addiettivo s’ula in i cal ’
r seaen un Gpuinivonmal dica in prouffima. Vedi fopra se
nell' ufo: L' ho bell’ ¢ fatea quefta , o quella cofa ; cioé I’ ho fatta fa
terminata , fornita . x
CHI cerca trona , Detto fentenziolo , che fignifica , che colui , che
al male , merita che gli fucceda :
NON 0’ ¢ {pina , ne offo.. E’ negozio fpianato . E cofa lifcia, Non’
bitare , non ci ¢ da incontrare difficulta alcuna i
AGAMBE alzate , Ciot col capo all’ ingit’. Si dice anche ve a
are, Vs0 quelta frafe Agambe alzare. Ser Brunetto Latini maeftro di D

  
   

tase n= ~sxse.

 

 
 
 

  
 

-OTTAVO CANTARE, 3 83

ovvero Capitoli pieni di gerghi , ¢ di vocaboli Fiorentini ; ¢ vole {pie-
‘atto di chi iemwomnla in terra per ifcaticare il ventre . Zvidi a eae als

ecanteee: con riverenza , cacava ) che quefto vuol dire torrire in

ested col, ut

“EGGLAV A con la morte . Faceva conto di morire. Temeva di morires

nel mulino .
STANZA VI. STANZA IX.
efto vuol pur ch’ io di Ini difeorra , Circa quefto,penfiero elle non hanno,
Onde di nuono a i fatti {uoi ritorno . Ne di fare altre {pefe, come accade
Le Ninfe , eb? il vedean barter la borra Ad ogni galant’ huomo.a capo a anno
Tutte gli fon co’ panni caldi attorno , D acconci,taffe, laftrichi di trade:
B gid tra loro par che fi difeorra UL vito,e il freddo non puo far lor dino,
Di fargli dare una fcaldata in forno, Perch il tetto,che feorre,e mai non cade

 
  

 

 

  
 
 
 
 
  
    
   
 

   

 
  

ta perché quefto in danno {uo rifulta LZ? Inuerno fui pilaftri di coralto
+ volleil/uo parere anch'ei inCfulta, Si ferma,e forma un palco di criftallo.
_ :S8TANZA VIL. STANZA X.
ino di non farn? altro; ond’ effe Di Stare il Sole giu ne’ {uci quartieri
riveftire a [pefe loro ; Non puo col frugnolone haver l'ingre/io,
7 icia nuova una gli meffe, Tal ch’ elle flanno bene , e volentieri,
« C'ba dal colloe da man trina ¢ lavoro, E gedono un pacifico pufse/so ,

 

Pr altra il ginbbone yn'altra le bracheffe Paride intanto infra tazze,e bicchieri ,
"un ricco ,¢ nobil quoio a’ oro , E di piu forte vini, ¢ fructe apprefso ,

  
 
      
  
  
 

« Fit altra gli ravvia la capeliera , Con é/se ritrovandofi in cantina, —_
J&€ ette il benduccto ,e la montiera. Valle provarne almeno una trentina,
oe STANZA VILL STANZA XI,
Alpalfe poi lo menan per la mano Ve per quefto alterato egli ne reftay

4 ta lor bella abitaxione , Ovengach'egli¢ avverzoin dlemagna,

Ma poi pits buona,benché fia in patano, Oc’ a faluar quel vin faccia la tefta,

ea pagar non hanno Ja pigione , Ed in quel cambio dia nelle calcagna ;

¢,un negoriv odio/o, e frrano Ragio,che quadra bene,e quedaye quefta,
: quell’ infolente del padrone Perch’ ei non urta mai chi P accépagna,

oad na a cafa,e co fi poca graria Ma sipre in tuono,e dritto com un fufo

Chiedeilfemeftrech'ei nov'é una crazia, Con efse per le fcale torna fufo,

-olox gh pic's STANZA XIL

nv! Ow egli entrato in una bella fala , Di li poi falgon fopr’ 4 un’ altra {ala
6 Ch ella fia P Accademia fi figura, Di bafton congegnati infra due mura,

' he vi fon aratolo, ¢ la pala Donde, arpicando come fan le gatre

ti d Strumenti da Siudiar ? agricoltara , Vanno a pafsar per certe cateratte.,

, DiParide dunque vuol feguitare a difcorrer il Rocta , ¢ dice , che conofcendo
i ke Ninfe,, che eglifentiva un gran freddo 5 volevano metterlo a rafciugare , es

 Hilcaldarfi in un forno.,ma.egli non volle , onde efle gli fecero un veftito nuovo

 
 
 
      
 
      
 
   
 
  
 
 
 

   

‘ (pefe nella maniera,.che viene efprefio, in guetta Stanza fettima; Di poi
Jomenarono a vedere la loro abitazione, ed in cantina dove bewve afiai ;e.nony
-danno per le ragioni ; che adduce il Pocta ; ¢ di cantina falirono alles

   
 

0

   

2 er

 

  
    

 

 
  
   
  
  
      
        
  
    
 
    
  
   
    
    
   
   
  
 
     
      
     
    

22,

384 MALMANTILE?

BATTER [a borra , Iorendiamo Tremare , ¢ battere i dei
do: E fi dice cosi per la fimilitudine , che ha tal b
che G fa della borra :1a quale ¢ {peci¢ di lana triturata col coltello ¢ |
empiere i bafti delle bettie da foma , ec. ¢ per liberar devta borra dalla)
fi mette fopra a un’ afle forata con piccoli:, € {peffi fori , © filbatte
di corde adattate a quefto effetto ; ¢ quefto battere fa uno firep: ct
che fimiliiudine col batter de i denti, che faccia uno tremante per ¢:
do , ec, Si dice anche batter la Diana; tremare tutto , flando allt aria, a€
{coperto ; Latino /ib dio... Vedi foto C. 9. tt..6. os 8 Coe Raa

BRACHESSE , Brache , caizoni , Voce Veneziana taluolea wfata anc

  

On ‘ '
QVOF d' oro, Pelli di beftic conciate , e dorate’, feruono per
ze in vece di drappi. ‘ at

GLI ranua la capellicra , Gli pettina la zazzera yO chioma, 3

benaa. Strifcia di panno lino bianca , che s' appicca pendente alla
cintola de i bambini , perché fi pofiano con effa nettare ij malo, -

MONTIERA, Specie di berretta ufata dai bambini. Dallo §;
tera , berrettino,

PANT ANO . Palude’, che diciamo anche padule , luogo pieno:d?
ma , che renda il terreno inzuppato , riducendolo come fango, da i
detto Palus , paludis , ,

PIGIONE , Cioé quel denaro , che fi paga per fitto d’una cofa; E
con termini proprj tro fi dice quel danaro , che'fi paga per poderi , et
pigione fi dice quel denaro , che fi paga per Cafe , 0 botteghe , dicendo
borteghe,o cafamenti : Ed appigionare cafe,e botteghe. Di quelte fi dice
ma dei terreni mai fi direbbe appigionare. Pigione dal Latino
forfe da fexdum , fio, e quefto dal Latino fides , ¥

STRANO., Stravagante . Qui intende noiofo , odiofo , faftidiolo . .
frrano dal Latino extraneus ritiene anche appretio di noi il fignificato di
ro , 0 lontano dal parentado noftro . Vi/o ffrano , vuol dir vilo arcigno
© crucciofo ; vifo rane yuol diranche faccia macilente , ¢ pallida,”

SEMEST RE , Numero di fei mefi ; ma intendi il denaro, che fi
pigione di fei mefi . Le

T ASSE , ¢ laftrichi di firade . Spe(e, che occorrono farfi alla giornata
ro-, che pofleggono cafe in Firenze ; che /africhi , intende quella {pela
partice fra i padroni delle cafe per raflettamento , ¢ Jaftricamento
della Citta.

TETTO , che fempre feorre ,¢ mai non cade., Abitano forto acqua 51
il loro retto’, che fempre feorre , e mai non cade," 1 abt
PILAST Ri di Coralie, Pilattri fi dicono quelle colonne fatte di
altri {afi , per foftener volte, Latino pile. B percht‘il corallo nafo
finge , che quefto tetto f regga loprai pilaftri di coralloje vaol di
werno s’ agghiaccia |’ acqua , ¢ fiferma . 1 boy poner
NON pro col frugnolone baver ? ingreffo . Non pud il Sole trama
netrare i {uoi raggi fotco I’ acqua, Fragnojone da Frugnuolo detto

  
 
 

  

 

2.2. -Seeeepeee seers eee =

_ =

BRipnmaes..

 
 
 
  
 
  
 

OTTAVO CANTARE. 385

ae ‘RATO . Commoffo , o perturbato da qualfifia accidente . Ed alterato
dal vino yuol dir Briaco. Onde gli Alterati Accademici gia famofi in Firenze
Bee ates

 

 

o per Impre(a un Tino ; in cui Gi pigiava I’ vua , ¢ ogni Accademico ula-
per imprefa particolare cofe attenenti a vino ; fi come quella della Crufca_ ,
le fuccedé , ula per imprefa tutte cofe attenenti a grano .
ACCIA a faluar la tea. Non offenda cot fuoi fumi la tefta , perché ¢ vino
- Detto (cherzofo tratto da quelli , che giuocando di fcherma non fanno
gioco , ma pattuifcono di faluare la tefta , cioé non fi colpire nella te(ta.
GION che quadra benese quellaye quefa . Tanto pud efler per quefta ragione,
per quella , che egli non Sa rimatto alterato dal tanto bere.

NON urta chi 0 accompagna , ma é fempre in tnono. Non barcolla come fanno i
riachi , ¢ non da fpinte a chi é {eco , ma fta in ceruello , ¢ va dritto.

| ARATOLO.. Si dice anche aratro dal Latino. EB erato fi trova nell’ antico
“Volgarizzamento di Palladio:; donde ¢ fatco il diminutivo drarolo. Strumento

quale i villani‘rompono la terra , facendolo tirar da i buoi .

‘PIC ANDO, Bi il verbo arrampicare fiacopato , ¢ vuol dire il falire , che

oi gatti fopr’ a ua’ albero , 0 fimili , ¢ viene da rampicone , che & un ferro
inde Die , che ufano i marinari per pigliare , ¢ fermar le navi. Latino
ro  harpagonis ; da che noi pure lo diciamo anche arpazone , ¢ arpagonare,
CAT ARATTE. Et voce latuna , che vien dalla Greca catarrbattes , con las
intendiamo ancora quelle buche fatte ne i palchi,per le quali fi pafia di for-
entrarein luoghi fuperiori con fcala a pioli , come farebbe falire per di
faltetto + E per lo piij tali cateratte s’ ufano per entrar nelle colambaic;
a forta era la cateratta , che dice in quefto lyogo .

 
  
  

  
  

  

 
 
   
   
 

TANZA XIII, STANZA XV.
4 qui la Mula vnol ch’ io mi difchiari Horsi per ch’ io non cafchi nella pena
Circa il deferiner quefte loro Stanze, De cingue folds ; ecco ritorna a bomba
Che stio vi pongo addobbi un poordinari, A Brache d'or, che nel (alire arrena
Non per dir bugie, ne ffranaganze; Per quella {eala,che va fu per tromba,
© Péréhiile Ninfe han folo i necefsari , Perché fe bene ei fail Adagia da Siena
wv ie moderne usAnre , Gli é difadatto, e pefa chregls {piomba ,

i
Per infeonare'a noi c babbian le borie E con le Ninfe a correr non puo porfi ,

 
 
 

4 adri',¢ letti d'ore,e tante lorie, Maffime liche v’ ¢ un falir da Orfi,
atl STANZA XIV. STANZA XVI
i; Ch ognun vuol far il Principealdid'eggi, Elle di gid , com’ to diceua ade/so

   
 

rl | Seben chi la vole/se rivedere y Vicite fon di fopra a ftanze nuone ,
Melti firvegvon far grandexze,e sfoeei, eApettando,che facia anch'ei Liste/so,
cae,

 
 
   
 
  

(pecchto poi col rigattiere: C’ appunto com il cambero fi muone ;
a sa lnfsobgrande, egiaregnain{ns poggi, Onde connien poi loro andar per efso ,
_ Efon nelle capanne le portiere + Ed aiutarlo fin, che piacque a Gioue ,

4 | Bera icannelt infin qualfiuoglia unto Che quafi manganato,e per ftrettuio

Had fuck fhiperti , ¢ feggiole di punto. Pafsafse ad alto il Caualier di quoio ,
Pr I Autore di voler dire la yerita , prega il Lettore a non pigliare
wzione , fe in defcrivere Je maficrizie delle Ninfe metterad addobbi, ed ar-
Refi un poco ordinarj , perche in eftctto ead cosi ; ¢ da quefto pigiia occafione di
tye Lee biafi-

 
 

SS

 

 
 

  
  
    

   

    
    
    
 
  

 
 
   
 
 
 
 
  
  
 
 
  

 
 

  
 
 
  
 
   
   
   
  
  
    
    
   
    
  
   
    
 
    
   
   
 
  
  

386 MALMANTILE

biaiimare il Info , che é oggi in Firenze. Di poi tornando
che le Ninfe falirono alle ftanze di fopra , doye con gran fa
de , il quale chiama il Cavalier di quoio , perché era v
demo. shee
ADDOBS! , Mafferizie , ed arnefi per ufo 5 ed ornamento
verbo addebbare, che vuol dire Adornare. Du Frefne nel Gloffari
dig Latinitatis. addobbare , armis inffruere , militare cingulum alic
confetka ex adoptare , quod qui aliquem armis instruit , ac militem ne
modo adopter in filixm, Si che Addobbare fecondo quefto autore vi
folennita del vettire i Cavalieri,
ZORIA, Aibagia. Vanagioria. » oe
SFUGGI. Vlanze fontuote canto di veftire , quanto d’ addobbamenti
fatti con (picndideza , € pil del confueto ; Donde fi dice fare sfoggio , 0
quando i trutti fanyo quantita grandiffima di frutte , 0 quando chi
piu del folico ; ed in fomma s' intende d' ogai operazione , che efca del
© cel naturale ; come fi dice frutta sfoggrata quella , che eccede img
beilezza , ¢ fupera I’ altre fructe della fua fpecie. EB la forza della
venendo da feggia , cioé ulanza , el folito , antepoftavi I’, s , vuol dir.
foggia , cice tuor del folito , e del confueto. Gio, Villani quel che noi
foggi , chiamna difordinati ornamenti lib. 9. ¢, 245.5 ¢ lib. 10. Cap.t
mo autore lib, 12, cap. 4. £ mon ¢ da Lafciare as fare memoria d' una,
mintazion d’ abito , che ci recaro di nuovo i France(chi. EB poco foro,
natura fiamo dijpofti moi vani cittadint alle mutazioni de’ nuomi abiti
trafare. Sfoggio dunque vale fuori di foggia., cloé dellafuzione , 0 VO
y.cniera’ di fare ordinaria , ¢ ufitata; che il Villani comes’é villo
sformata mutazione a’ abito ; ¢ disordinati , ¢ fconuenenoli , ¢ difonefi, ef
mienti ye nuoni, ¢ iffrani abiti , :
CHI (a volefse rinedere. Cio’ chi la volefle bene efaminare, 0 rice!
maniera quefii cali poflano fare Gimili sfoggi + ‘i
SONO a fpecchio. Hanno debito. Traslato da coloro , che hanno d
decime , che fi pagano al Principe y i quali &i dice effer'.a {pecchio , p
notati a un libro , che fi chiama lo fpecchio.; Qui dicendo. ; fono
ghattiere , da:due colpi , uno che coltoro , che fanno tante borie non
gate , ¢J''altroy che quefti loro sfoggi fono di robe ulate, € vedute
poiché I’ ha prefé'dal rigattiere, che yuol dire Vao , che vende mafleri
ed abiti ufati. Vedi fopra C. 3. ft 5.
POKTIERA, Paramento di drappo , o d’altro, che ferue per
porte delle Ranze nelle cafe Civili. Da alcuni detta in Latino velum ada
TRA icanneili . Vuoldire fra la gente pili vile ; perche fra i cannelli.
mo fra i tefitori di lana , che fon gente d’ infima plebe , ed.¢ lo ftefio
qualfineglia unto; perché quelti tal: maneggiando fempre lane unte
fempre unti ; ¢ qui aggiungendo al detto fra i canned, il fi
intende , che fino i Batulani , che fra gli unti fono i pid vili »fanno le!
SEGGIOLE di punto, Cioé feggiole ricamate , o trapuntate di
mo: Panto Vaghere ,0 punto Franzce, 3

 

   

BABeRew eft gFen2n2 fe 8s =

=

a=
Se =z

SURFER
 

 

 

*’ OTTAVO CANTIARE: 387
CASCAR nella pena de’ cingue foldi . Quand’ altri nel difcorfo fa una digrettio-
ne, €non torna mai a) primo propofito , gli diciamo: Voi cafcherere nella pert.
de’ cingxe foldi, 1 Varchi nel {uo Hercolano pariando di quefta pena dice: E chi
cominciate alcun ragionamento ,e pot eutrato in un’ altro, non fi ricordaa prit di

nave 4 bomba 2 fornire il primo , pagava gid , fecondo teftimonio dal Burchiello , ni.

te

‘a

4
wail
a

 

offo , #1 g 'o non valeua per aunentura in quel rempo pit di quei cingue foldi, che
a ced Nae quali Lacie vegghiamo , che FT rarchs fi ferue del detto
Tornare « Bomba per tornare a {egno, 0 al propofito del primo difcorfa , come fa
il noftro Autore nel prefence luogo . L’ Ariofto Satira prima dice ;

= Ma perché i cingue foldi da pagarte ,

t Tx che leggi , non ho , ritornar vuglio

7 La mia favola , donde ella fi parte.

_ eARREN A, Intoppa ; Si ferma ; Non feguita il viaggio. Traslato dalle na-

Viquando fi fermano , perché tuccano il lette dell’ acqua , che fi dice arrenare , 0

incagliare , De 1 gual) verbi ci feruiamo per e(primere non tanto i! fermarfi in un

‘Wiaggio , quanto il fermarfi in un dilcorfo , o nel profeguimento di qualfivoglia.s
‘aaione , negozio. Latino hnerere .
_ PAil mangia da Siena, Fa il bravo. Fa il valorofo. Il Mangia da Siena é
\ di metailo atiai grande, la quale é pofta fopra la Torre dell’ orivolo
del Comune di quella Citta , la qual figura dicono , che fia il fimulacro d@’ uno an-
tico huomo bravo detto ii Mangia ; Ma io fon d’ opinione , che ella fia i] fimu-

lacro di qualche antico Podefta di Siena , e che habbia acquiftato i] nome di 423-

$4 da qualche inferizione , che havefle appreflo, la qual diceffe Il eA/agna di Sio-
WAS COR i) ALagnifico di Siena , che s' incendeva gia il Podefta: Ma fia com ef-

fer fivoglia , a noi bafta fapere , che quefto detto ferue per in tender con derifio-
ne un bravo , o valente ; quafi voglia mangiare le perfone , e ingoiarle ,

DISADATTO , Contrario d’ Atto , deftro , agile , ec, Vno che duri gran,
fatica a maneggiarfi ,o muoverfi per la gravezza , o per altro accidente, Sciar-
feancora ¢ contrario di arto , ¢ fignifica uno , che fa male , o negligentemente
quel ‘ch’ e’ fa ; poco pulito nelle fue faccende , ¢ nella perfona ,

CON Ie Noha correr non puo porfi, Non pud gareggiare con le Ninfe a chi pil
corre. Interide , che le Ninte al ficuro lo fuperercbbeno nel corfo ;

VP Bun falir da Orfi. V' & cattivo , o difhcil falire. L’ Orfo é un’ animale,che
f ben, ir goffo , e difadatto , nondimena ¢ afiai deftro , ¢ facilmente fale anche
in ionghi inaccefibili ; donde noi habbiamo: Efer come L’ Orfa, cioe Sofuse deftro,

Ui Berni nel Cap, al Fracaftoro dice :
Shy . Conniene ivi lafciar t" xfato corfa ,
«ta £ falir fs per una certa [cala ,
i Dove hauria rotto il colle ogni deftr’ Orfo ,
‘Ostiero nell Iliade al nono chiama una rupe 50 balza Aigitips , cio8 dalle capre abe
“Vandonata ; © queito medefimo nome di Leeips danno gli antichi a una Citta dell’
Afola di Cefaionia , € ua’ aitra deil’ Epico. Noi diciamo di Jaoghi fimili erti ri-
‘Pidi , © feofceli: Won vi falirebbero le capre , le ee Virgilio nell’ Egloghe dide:
repe. Quella montagna altidima nell’ India; fu'la quale fu il primo Ale(-
fandro Magno a falire , fu detta da’ Greci eornos , cioe fenza uccelii, quafi mon.
oksal Cec 2 tagna

 

 

ca i
 

   
     
   
 
  
 
 
   
  
 
 
 
 
 
 
   
   
 
     
     
    
  
  
      
    
      
 
 
     
 
    
 
  

 

388 MALMANTILE

tagna da non poterfi ne anche da chi aveffe I’ ale formontare
S7 muove come il gambero . Cio’ va all’ indictro , Wepam
MANG.ANATO , lnfranto; Mangano ( dal Greco mage:
na, con la quale fi diftendono 5 ¢ fi-da il juftco.a i panni, ¢
fare a forza di rulli fotto un graviffimo pefo , ¢ tal panno 5 0
fi dice poi manganato. E Mangano come s‘accennd fopra Cw.
na militare della quale i noftri antichi fi (eruivano per (cagliar'p
fiediate ,'¢ con effa fcagli anche | ini, che dicevano poi
ganati, cio’ sflagellati , ¢ pelti dalla percofla ; ¢ cosi fi potrebbe inten
ride; ma perché foggiunge paffaro per frretrow , che é un’ altra machina, ¢

 

   

 

 
  

 

 

ue per ftringer ulive, ec. , © per mettere in piega a panni , fi vede , che
quel mangano da panni .
Ss ZA XVIL
WN un Dormentorio grande, ma diverfo,
Ove ciafcuna in proprio ha la fuaceltay
Che fia com’ io dir per quefto verfo ,
( Se non erra Turpin , che ne favella )
Vana fanga a mez aria tuna travorfo, ¥
Dow’ alla tien fe calze,e la gonnella, w
i penzol delle forbe, ¢ del trebbiano, ag
E quel che pit le par di mano in mano; fa
TANZA XVIIL. ae
Pitt git da banda un tavolin fi vede, er
Che {a i tre/polifa la mnna nanna, Sxpenfa,che vi ina
E fia fpatiiera al muro, ove fi vede Ada trova im ozso tutti gli 1
Via fiuvia di giunchi,e fottil'canna, E i piatei ripulisi come fp fai
Evvi una madia zoppa da un piede, Teglie, e padelle, inutile a
E il filatoio con la fua cifcranna, Star'appiccare al muro per gli fi
Won v'é letti , fe non un per micliaio , Ed anche fon per fParut pilt a et
‘Che tutte quante dormono al pagliaio, Perché il gattoa dormir vedein, | dy
: STANZA XX14, sth ng
Ond? egli offefo molto fe me tiene , ») (Gliaccanan ch'ei vedra fel ta
Ch’ una mantita per la golatocca ; Ed ei ghignando allor pits noms (i
Ma quelle, che s' avveggon molto bene, E con effe ne va di compng §
Chregli bal'arme diSienarpreffain bocca Per ultimo a veder la Gi 7 hk
De(crive nelle prefenti Octave il dormentorio delle INinfe,e ledorowmafieria® | Yu
Arrivano ee cucina, dove Paride a (aol e de pr ta
arata ‘cofa alcuna ‘per mangiare ; Ma ie Ninfe Jo quictano‘con dirgli , ch e
ey ada eiare} 0d de lo'condi a veder la Galleria , Pe
‘DiVERSO , Differente., 0 ‘diffimile aghi-altri Dormentorj, perche: «
Celle non ‘fon fatte di muraglia , ma fon tutte in-una grande flanza-y §
vile con ftanghe app al:palco ciondolot foa r
quali poneng jo ciafcuna'le fue ‘robe , e panni Je ‘fa feruire per muro
.cosi vengono ‘fortnate'le Celle:. ‘Si pud anche dire , yche la voce
dae figaificati il primo ,'che yuol dire diferente ( ¢ gquefto f a
 

 

   
  
   
 

OTTAVO CANTARE 389

‘meffo per contrappofto , come la-tal cofa ¢ diverfa dalla tale ) il (condo quando
po ‘ate che vuol dire ftrano, o ftravagante, il Poeta lo piglia ias
Recornto fignificato . —_ lo piglid Dante Inf. C. 7,
_ Entrammo g pen via diver[a, Oc,
Cavaleamti nelle fue ftorie lib, 12 parlando di Cammillo quando ifefe il
lio dice :, Non guardo all’ ingiufto cacciamento , ma con grandiffi-
i yy mo efercito corfe alla dife(a della patria, ¢ liberolla da cosi diverfa fortuna .
7 es , Ricordano Malefp. Stor, Fior. cap. 80. dice: E cid fu per I’ inuidia della Si-
> ae i » che non ¢ra.al loro volere, ¢ fu diverfa , ed afpra guerra. Vedi. fo-
int 2, flan, 3.
bis Beene del trebbiano. Che cofa intendiamo per penzolo vedemmo fopra C.
6. flan. 50. ¢ Trebbiano € {pecie d’ uva bianca , ma.qui ¢ prefo in generale eee
(IL ogni forta.d’uva ,\che's’ appicca nelle ftanze per ferbare all’ Inuerno .
_ DI mano in mano. Di tempo in tempo. Lat. Deinceps , che s’ intende fuccelfiue
neat:  Cic, 7, Ep. Fam, difle De manu in manum . Dan. Par. 6, dice:
jas E [otto t’ ombradelle Sacre penne
we \Governo il mondo li di mano.in mano.
yin
ye
fi
yh
ue
ae
lf

  

. Bd  detto figuratamente-dal far paflaggio una cofa dalla mano d’ uno nella.
‘mano dell’ alero.. Dal ginoco. detto Lampade dromia, nel quale colu aveva il
-vanto che va una fiaccola accefa correndo., ¢ cosi bella, ¢ accefa la confe-
gnava.a chi aveva.a correre dopo di lui ; difle Lucr. lib, 2. Augescunt alva LEnbes 9
lia minuuntur , Inque breni [patio musantur fecla animantum , Et quafi curfores vite
Po eamema 9 Gi0e fuccede t'.uno vomo all’ altro, l.nne vinente all’ altro di mano

0 roc, Dal Lat. tripus , odes , E un pezzo di legno, 0 ceppo., in cui
a fon fite tre mazze , fope’ alle quali pofando., ferue;per fohence tavole, ¢ defchi,
oe da i Latinidetto Trapecophorus.s'quali menfam ferens .

’ | PAlaninnananna. Non Ma forte in terra, ma dimena o:per |’ inegua lita de-
ri N ‘te tre mazzc, 0 del fuolo,,.o per altro mancamento; ¢ diciamo far /a ninna manna
Od eda,quel dimenare;che fi fa-della,culla.de ibambini , ‘quando dallebalie fi procu-

ache dormano , che fi dice ZVmnare , spetche per lo pit fogliono accompagnare

4 -talmoto.con una lor cantilena,che dice Ninna nanna il mivbambino. Vedi fopra
iS ‘Cae Renaes. Quefto dimenare fidice anche:ewlare pur dalla Quila de’ bambini..
, SPALLIER A, Quella'parte della feggiola , alla quale:s' appoggiano Je (palle
mA Aedendos |B per /paliiere intendiamo quelle nuragli¢ 5 alle-qualt fono.appoggiate

spianted'agrami_, ec. come's'¢ detto fopra\C. 6, ftan.'51. Quefto artitizio-di
Farele:mura:coile piante-dicefi-da alcuni in Lat.-opus topiarium.. Equi nee
‘quel-‘muro parato di ftuoie tatte di giunchi , o-canne paluftri , che fourafta.alla.
oat »fopr’ alla: quale dice;che fedevano Je Ninfe, ¢ferue,per {pallicra alla .me~

 

J

3 STVOLA, B il Latino Storeache conlerua appreffo noi il {uo:fignificato.,

i! HUADIA, Dal Latino maétra,i| qual pure ¢ Geeco;.ed una cafla sadatrata
it fopra-quattro:piedi, dentro alla quaic fi lavora la patta per far-il pane; La dices
3 Zoppa.da‘un'piede perché le: mancava., 0 crarottouno diguefli piedi.. Zoppa fi-
‘i! siete den tee cain tavon della vecchierella Bayside Ja ;preffo rida

 
 

390 “ MALMANTILE

lib. 8. delle Trasformaziuni ; ma ella la fece ftare pari con me!
to ; menfam fuccintta , tremenfque Ponit anus ; menfa fed erat pes ters
fia parem fecit , i , 2/1 ihe
FILATO/O . Strumento col quale per via d’ una gran ruota fi fila Jan:
napa, ec, e fi fanno le‘funi. 1) OL HRS
CISCRANNA . Specie di feggiola come accennammo fopra C,
DORMONO «i pagiiaio. Cio’ dormono in fu la paglia.
HVOMO alia buona, Huomo {chietto, fincero,e fenza malizia ; Huo

za cirimonie , ¢ nimico del luffo , e delle boric fine fuco, © fallacijs , | Ve

sornm , ed Hxomo pofirixo intendiamo uno,che non fa sfoggi nel veltire , ¢
ogni cofa fi tratca {enza lufflo. SOS
SENTITOSI allegare i dents. Vuol dite fentitofi (timolare dalla golae dal
defiderio di mangiare ; fe bene allegare i dents vuol dire quando i deat pert
matfticata qualcola acida , 0 agra. coine ‘il limone:, ec. s*iavormentileono, ¢ i
fente una certa diffculta nel mafticare.Ma ufandofi come nel pretente iuogo,vu0
dir venir yoglia di mangiare . :
TEGLIA. Specie di tegame fatto di rame ftagaato per di dentro , ferue pe
quocerui torte , ¢ migliacci , ec. (| Monofini lo fa venire dal Greco Telia y a
gual voce tra I’ altre cofe fignitica 1’ a/se da pane, e"| turacciolo,o coperchio del fum
maiuolo , 0 vogliam dire di quel canale,che gli antichi , in vece di cammino ,ave
vano per feruizio di cucina , buono folo a ricevere ,¢ porcar via it A
dicendolo molt Tegehia , ¢ gli antichi in particolare , mi muovo a
venga pil cofto dal verbo Latino Tegere. Quefte teglie hanno nell’
ta una campanelia di ferro per comodita d’ appiccarla , ¢ le padelle hanno un
anclio in cima al magico per il medefimno effetco’; € quefti fond gli orecchi de’qua-
li parla il Poeta dicendo : Stanno appiccate al muro per els orecchi., Ovidio lid. &
Metam, erat aluens illic Taginexs , dura clauo fufpenfus z anfa, hia
TORN/ZE , Parlando di gatei s’ intende quel ronfare che fanno ; perché ¢ &
mile a quel romore , che fa il tornio quando gira . + Aba
TOCC A una mentita per la cola, Dar una mentita per la gola a uno ¢ quando
fe gli dice,che egli afferma il failo, ed ¢ grandiffima ingiuria , e che muove al
¢ pero il Poeta {cherzando dice , che\Paride fi adira per I offela,che ri
quella mentita per la gola , cio¢ di quel tuppofto che vi fufle roba per la golasy
che fu falfo, WS EE
Li arme di Siena imprefsa in bocca, L' arme di Siena é una Lupa , ed il mal dé
Ja lupa ¢ intefo comunemente per una infermita , che fa ftare i] pazziente in col
tinova fame ; onde quando vogliamo intendere ; il tale ha gran fame diciamd:
Egil ba il mat della iupa , ¢ pis copertamente Egli ha ? arme adi Senay es’
la lupa , cioé Ja fame. Vedi fopra C. 3. ftan. 22. Kh
VEDKA' , # il corpo teene, Cioé mangiera , ¢ bera. Detto affai afato”
gente di vil condizione . » 1. 3a
GHIGNANDO., Ridendo leggiermente . Lat. fubridere,
GALLERIA , Cosvin voce ftraniera chiamiamo aicune Manze piene
nate di-galanterie , ¢di-cofe fingolari , ¢ maravigtioe 3 quali ttaazes
fon dee Pmacorheca dal Greco Pmax , che {uona tabula pita, © theca

oe

 

 

  

  

 
   

    
  
   

oe. >Eeerec-se ._= ez ER PTE

 
  
   

  

orre al
STANZA XXIL

fb Principi ritvatti ye di Patrizzi ,
_ Originali farti gid in Fiarenga

4) Da quel,che gis vendea fotto gl’ ufizri,

_» Ed euns dello feeffo una Sibilla

_ Eduna bella Cittadina in villa.

Bt STANZA XXIIL

me

SE aa

tapelPa fole , e sgabelli
intorna inalzan fopra al piano
_ Statue eccellents di quet Prafiteli
BH Gia shalt danno il moto in. Settignano,
Ce * Buonarrnoti, e i Donated

Caer

— Aquel baffa ritseva di lor mano

z

 

ORTAVO CANTARE. jor
erste: Sper altro Galletia voce militare ¢ {pecie di fortificazione.
xX s

TANZA XkXIV.

Si che que? opre , che non hanno pari ,
Quantoi (uddetti quadri,c’ han del vago
Non fi polfon pagar mai con danari,
Perché fon giore 5 che non hanno pago ;
Vao feaffale v' ¢ di libri vari,

Ch’ eran La libreria di Simon Mago ,

C’ abbellita di feorie 5 ¢ di romanzi

Fu pot venduta lor dal Pocauanzi ,
STANZA XXV.

Exni un tomo fra gli altri ferittoa penne,
C’ ame par bello, ¢ piace fine fine,
One fi legge in carta di cotenna
Tradotte le librettine in feftine,

E che Gateno, ¢ il medico eduicenna
In mufica mettean le medicine;

ww &

Leo, s' sl corpo fempre a chi le piglis

GC as paari fcalzi pur fi vede ancora
‘ Gorgheggia ,¢ canta,noné meraxiglist,

| Sw t arco della porta per di fuora,
+3 9A da principio’a defcrivere la Galleria deile Pate , ¢ narra la bellezza
4 aicune pitture , ¢ ftatue non diffimili dal refto delle maflerizie , per efler’ opra
ade ad pilicimuniti: Artefici 5 (e bene fcherzando gli efaita fopra i pid eccellenti
Macitri, Oitre alle picture ve anche wo foaffale pieno di libri dei medelima yaio-
IE ye. » che fono te pitture , ¢ {colture.
FRONTESPIZZ/ . Vedi {orto C. 9. ftan. 15. i
MAIOLIC 4 . Specie di piatti , ed altri valeilami di terra , la quale meglio ,
Che im aites iuogh: fi Javora oggi in Faenza ;¢ quefta terra é detta maiolica dall
Mola di Adsiorica , 0 Adaiorca dove gia fi fabbricava ; €1' lfola che diciamo oggi
Maiorca gia fi diceva Maiolica,, come fi vede in Gio; Villani lib. 4. cap. 30. WVe-
$4 anni ds Cristo 1117. Gui Pifani fectono nna grande armata di Galees e Navi, ed
ndaronofopr’ ali’ Hola di Adavolica., B che iaquelta lola fi fabbricafiero tali va-
lami fi deduce, non folo da] nome , che ritengono di Maiolica , ma anche dal
Vederfi nelle fabbriche antiche di Pifa 4 etparticolarmente nelle facciate delies
Chiele murati di tai piatti come per trofeo , ¢ memorie delle vittorie havute da
i Pilani contro ai Maiorchin: 5
VNA belia Cutadina in vila, Era gia in Firenze un Pittore da pochi foldi, il
quale faceva ritratci di Principi , di donne Fiorentine in abito da Villa , ¢ da Cit-
ta, de Sibuie yle Mules ec, -¢ tucto.cosi malfatto , che non ¢ran comprate tali
picture fe non da genti di contado ,¢ per vilidimo prezzo . Dette pitture fi ven-
devano forto le Logge, che (ono d’ avantia quelle flanze, dove fi radunano i Ma-
giltraui di Firenze,c quefto luogo fi dice forto gh Vfizei y ¢ per una bella Cittadina in
Villa , € una Sibilia incende di quefte belle pitture .

D1 quei Prafitelli , Di quelli Scultori valorofi , ¢ celcbri , come fu Prafiteles ;
/parla perd ironicamence , ¢ per derifione. Praffirelle detto poeticamente come
_Annibaile , Eetorre y ¢ fimili per ia rima ,in vece di Praffitele , Annibale , Ben ‘

“hoo 0

=SERS BQ SSESHER ESS

AS

2s
Se

=

a= EE

   
    

392 MALMANTILE™
Cosi i Latini raddoppiarono Ja Lat. in Relligio , x Relient a ¢

la legge del verfo. 4
CHE a i faffi i. daensit mined Settignano . Dar il moto ai fafti,
fi vuol dire Formar figure di pictra: Virg. vines ducene de marmore
Settignano Borgo vicino a Firenze abitano quafi cutti fearpellini
fabbricano poco altro che ftipitt , (Caglioni Primi if
che di cafe , ec, taluolta Javorano anche delle figure » ma per lo
le fuddette pitture ; ¢ pero il Posta (cherzando dice , dannoi moto
che voglia dire animano i fafi,, fabbricando ftatue , che peiolowive
de , che danno il moto ai (atli , cioé gli muovono , ed ¢ { I
quali fono ia.quei monti di Settignano , lyogo detto cost quali
dere , 0 pofieilione della cafa Seprimia , antica Romana, ficcome
della Perronia , ¢ altri molu iuoghi dello Stato 5 che risepgane anon
padroni , nobili Cittadini dell antica Roma ,

QLVEL bafso ritievo di tor mano, Fe, Perche fir’ c d
erano quefte ftatue » porta I efempio d’ una figura y che &: nell irchi
ste della Chiefa di 3. Paolo de i Carmelitani Sealzi ,che é-una

afio rilievo , la quale rapprefenta , o almeno dourebbe rap,
Jo, maé lavorata cosi maravigiiolamente male ,.ches'é rela
{ua ftorpiataggine ; ed €-compagna delle (tupende pitture del Pamo
Zannino da Campugnano . Jntendendo dunque al omen Boera
tre figure,che le fono attorno fatte della medefima maniera vuol
che fi vedevano in quella Gaileria eran maligimo fatte,

NON hanno pago. Non hanno prezzo: E? parlareironico , e-wuol
hanno prezzo , clot non s' apprezzano 5 non fi fimano , non vaglion A

SC.AFFALE . Armadio aperto fatto.a palchetti per ufo di tener libri. |
nome di Sehapha , e di Scapbos fidicono in Greco molti arnefi,e ftramenti , 1
tutti'o concavi , o- (cavati per ufo di tener roba 5 dal verbo /capresm
re cauare ,fcanare, Onde /caffaie , arnefe y che ha varie capacita 5 €
ne’ quali fi ordinano ,¢ fi pongono i libri Lat. platens armarium

SHMON Mago, Fu lt Autore , ¢ capo de’. Simoniaci ; effendo-
che tentafie di comprar da 5, Piero i beni.Sacri , ¢ Spirituali , come fi
acti degli Apoftoli. & che cofa fia Mago , Vedi fopra C, 1. flan, 20,5

POC AVANZI, Fu un Libraio Fiorentino ¢osi detto , ii quale nel’
 Autore compofe la prefente Opera era ridotto in poverta, € vendeva’
che leggende .

CART A di cotenna. Intende Cartapecora .

LIBRETTINE , Quel libretto, che infegna conolcere le figure dell"
¢ Je prime regole del medefimo. Il Burchicilo ..Vedilo andar ,¢h e* par
tine, cioe ¢ tanto magro , fecco ye ee C' pare una fignrad
tini un macilente , efienuato » ¢ deforme nelio fieflo modo.
grammo , ioe delineato {olamente y¢ fattovi il (olo,¢ puro din

o colorito.
MEDICINA . Quando fi dice femplicemente medicins da noi?
_ bevanda folutiva , che fi beve'con Ja preparazione , oe ¢
ta prima con alcuni {ciloppi , cc.

 
    
  
 
 
 
  
   
  
 
   

 
 

  
  
 
  
    
     
  
   
    
   
        
   
  
  
 
   
 

 
  

OTTAVO CANTARE., 393

| GORGHEGGIARE , E} termine mufico da i Lat. detto Vibrifare , ed & un tril-
lo di voce fatto con Ja gola , al quale in un certo modo € fimile quel romore, che
fa nel.corpo il vento , 0 altra follevazione d’ umori cagionata dalla medicina ,
ed il Poeta ii » di quefto romore , che fa il corpo dice , che il pazziente>
pud far di meno-di non cantar cosi , poicht Galena , cd Avicenna haveyano
in mufica tali medicine

  
  

b STANZA XXVI,
ave n't in rima,che Ja Sfinge ¢ detto Perch’ ci,chefa,chee Sale bebbe concetto,
| Seelta d Enigmi,che non hanno nguali, Accid che i verfi {uct fieno immorcali ,
Perch! agnune ¢ diftinto in um fonetto, £ i vermi dell'obblio non dien or noia
we Che if Poeta ha ripien tutto di fali ; Porgli fra fale,e inchioftro in falamoa .

Bra quefti libri delle Fate fi crova anche la Sfinge , che é una {celta d’ Indoyi-
i diftinsi ciafcuno in un fonetto , opera del Sig. Antonio Malatefti ; la qua-
Ieil noftra Poeta ( facendo di efla quella ftima che merita ) non haverebbe metia
ion le , fe il medefimo Malatelti non ! havefle forzato a farlo,com-
lo egli medefimo la prefente Ortava nog alterata punto dal noftro Poeta.
fale Opera conticae ( come habbiamo detro ) Indovinelli , il Malateiti
il nome di Sfioge, che fu un Maitro appreflo a Tebe , Figliuolo ( fecondo
no )del Gigante Titone, ¢ di Echidna, che fignifica Vipera ; ¢ Fratel carnale,
il , della {paventola Gorgone, del Can Cerbero , del Serpente
di pi tefle chiamato Idra_, ¢ di pi altri moftri ¢ animalacci , il qual moftro di-
4 -tmorava in-un monte contiguo a Tebe fopr’ ad uno {coglio vicino alla ftrada , ed
| a chiunque paflava proponeva wx dubbie[ che i Greci dicono evigma , i Latini
nt ‘uphas pure dal Greco ; ¢ noi indoninello come sé detto fopra C. 6, flan, 34.]e
; Leqlioa ace Jo fcioglieva , il moftro improvvifamente lo pigliava ,e}' uccide-
‘i va. Agcadde , che Edipo figlio di Laio Re di Tebe fu quivi mandato , ed il Mo-
, fico gli propote: Qual’ era quell'Animale, che da principio andaya con quattro
“ piedi , poi con due , ed in ultimo con tre = Edipo rifpole , quefto effer ’ huomo ,
"i, che da bambino va carponi con le mani , ¢ co1 piedi , € cosi con quattro piedi ,
se poi rittoin fa due piedi , ed in vecchiaia con tre , perché va col baitone ; E con
tal folygione vinfe il moftro , che percio fi mori. Pe
_ RIPLENO di foi. Ripicno di belli , ed arguti penfieri . I Latini ancora chia-
vif mavang falil' arguzie , trovandofi in Orazio.. Nofri proaui Plautinos landanere
fale, Giulto Liptio Aatig. lect. dicit fe amare elegantes Plauti fates, Lucano : Non
ie Solici ifere fates. Tor. in Eun, Qui baber falem , qui in te edt , intende {cienza , fa-
“if ‘pere. Ma qui.’ Autore fcherzando con I’ equivoco del fale dice : Che i] Mala-
teftisil, (a che cofa ¢ il fale ,¢ che cifecti partoriica [ perché egli era guar-
dane azzini del Sale di Firenze] ‘ia meffo de i fali.ne i fuoi fonecti , per
#1 fr loro falamoia con } inchioftro , athaghé i fuoi verfi fi conferuino , ¢ G
mw?  difendano da i tarli della dimenticanza , fapendo , che il fale conferua , © difen-
we ic ins ; ¢ le compofizioni fi conferuano da i vermi dell’ obbiio con,
g@  Icriverle , € queito fi fa con ".inchioftro , ¢ pero lo chiama falamoia, I Latini
cone la 1a Murra , del che noi componghiamo la voce /alamoia, quali
falis mursa , 1’ iachiottro da Monfignor Ciampoli fu chiamato dal con(eruare {¢
Orie € i noint degli huomini Bai/amo della fama,
t Ddd STAN.

   

  
 
 

“

a

 

  
        
     
      
          
      
   
 
  
 
    
  
 
     
 
   
  
      
    
 

394 MALMANTILE ©

STANZA XXVIL
Altri Poemi poi vi fono ancora , E uncerto Mal
£d hanno caparrato alla Condotta Ecco fubito bell! ¢
Grillo ilGiambarda, Ipolito,e Dianora Le Deecol Babi,chel ba
1 ferre Dormienti , e Donna Ifotta; i Z
Narra che moir’ altri Poemi fono in detto fcaffale , ¢ mette t
frottole compofte da’ Cicchi per Je donnicciuole, ¢ per i fanciulli.
genie dice, che fara ancora la prefente fuaopera, '
{NC AP ARR ATO , Data la caparra cioe dato danari innanzi per fert
mercanzia per conto proprio. ( Voce formata , dice il Perrari , da cape a
Qui vuol dire che hanno chicfto lu MALMANTILE, Gili antichi d
rare da Arra, caparra . "
ALLA Condotta , Cosi & chiamata a Firenze una ftrada , nella quale
botteghe i Librai , e alcuni Stampatori , ed ¢ cosi appellata , perch¢
fima ftrada haono i magazzini coloro , che tengono 1 muli per lao
mercanzie a Roma , a Bologna ed altrove. .
MESSE in rotta le Dee col Bambi . 11 Bambi era uno, che vendeva
maggio , ec. che noi chiamiamo Pixzicagnoli. Dice che le Ninfe fono p
car lite con detto Bambi , perché eflo impedira , che elle non habbiano il B
di MALMANTILE,, volendolo egli per farne alle accinghe tance ¢:
per inuoltar falumi. Ed in fuftanza vuol dire,che la prefenve fua Opera’

2 2 eee.

r=

na per vendere a pefo per carta al pizzicagnolo ; che cosi diciamo
che un libro non habbia in fe di buono altro che Ja carta .. E qui fe
dice quefto per fua umilta , ¢ modeflia [ non effendo 1a fua Opera da
pelo per carca j tuttavia , non fapendo che la mia penna dovea farle meritare
tal fine , fece buon pronoftico , € non dubito , che havera dato nel fegno I
Lalli nella fua Franceide C. 4, flan, 21. Si ferui di
E le cartacce lor feruono al fine
Per avvolger U acciughe ¢ le Tonine ,

STANZA XXVIII.
Bovvi anch' un libro ds fegreti, il quale
Gioua a chi legge , ¢ infegna di bei tratti
Ed infra’ altre a far che le cicale
Cantsn fenza che'l corpo fe le gratti ,
Ea far ch’ i tordi magri con? occhiale
Guardandogli divengan tanto fatti ,
Deferive pos moltiffimi rimedi
Per chi parifce de i calli de’ piedi,

STANZA XXX.

Perchi la donna come altera , e vana
Sopr’ agli sfoges ognor pen[a,e vaneggia,
ae cht el?” abi un ceffo di befana
Pompofa,e riceavuol che ogni la veggia;

 

   
     
  
     
 
   
  
    
  
   
    

quefta medefima frale. ee

STANZA XXIXi
S? io vi narraffi tutto il
Coftui , direfti, ba it cera j

Pur vuo! contarnen’ wna folamentt R
Chie vera , ne crediare eb io sarfilh "i
Racconta a! una tal parturientt :
Ch’ una carrenea {een faeae 5 &
E ch’ una voglia fu,che bawen bavwsy b
Ed io lo crederé fenza difpura.
Percio colei bebbe 1a voplit ;
Della grandexza dell’ I
eanceeioeadeg robe ik i
Le girelle vorrian , ebe'tfa 1
E é
   
  

rts

a

SB SSE CRELESE EGE

SEES SRSA Seth

i

Ma hafti circa i libri quanto ho detto ,

 

OTTAVO CANTARE. 393
“STANZA XXXL

ed qualch' error novoglio far fuggerte,
_ Perch'ioche negli Pudi non m'imbrog lio, Che pur eroppin' ho fatti for’ al fogiio ,
eee altri non bo letto E pot perche fom tanti ,¢ tanti i tomi ,

Lorfe i fatti lor faper non voglio y Che ne anche fo dir d'unterzo: nom: ,
eos il racconto de i libri , che fono nello (caffale ,¢ narraado un favolofo

=, _ iperbolico parto , fa una leggieri fatira contro al luo delle donne.

_ 10 sfarfaili . lo aggiunga al vero: Io m’ avvantaggi acl racconto . Dalla far.
falla , che gira ¢ s' avvolge or qua , or la, ¢ detto sfarfallare .

_ ¥NA vglia fu, Che cola fia voglia in quefto propofito. Vedi fopraC. 2. ft. 42.
— ALTIERA,e vana. Altiero, fi pud dir finonimo di fuperbo , pigliandofi
fpeffo ' uno per I altro ; fe bene a/tiero fi dice colui , che per grandezza d’ animo
non riguarda ,¢ non applica a cofe vili , anzi dimoftra verf di quelle una cerca
{chifezza generola , ¢ fenza vizio ,¢/uperbo G dice colui , che per vizio , ¢ per

apriccio {propofitato difprezza tutti , ¢ tutte te cole indifferentemente , ¢ fenza
Thasoee alcuna. Qui , dicendo a/tera intende piena di prefunzione di fe {tel-
fa, che ¢ lo ftetlo che /uperbo ; e Vana dedita alle vanita , o vanagloriofa , boria.
fa, li Petrarca diftingue quefte due voci , dicendo nella Caaz, 22,

costs + Ch’ in vifta vada altiera , ¢ difdegnofa ,
Non fuperba , ¢ ritrofa.

_ BEF ANA, Significa Donna malfatta: perché befana diciamo un fantoccio fat-
todicenci, che fi fuole da alcuni mettere alle tine{tre il giorno deil’ Epifania , il

jale da Epifania ¢ detto. corrottamente il giorno di Befana . Vedi forta C. 9,

Sis

I,

TREGG/A . lntende carrozza. Se ben tregeia é un veicolo ruftico fenza ruo-
te per ulo di portar paglia , ¢ legne , ec. facendolo tirar ftrafciconi da i buoi .
Servio fopra quel verfo di Virg. 1. Georg. Tribulaque , traheaque , © iniquo ponde-
re rafirs dice cost. Traha genus vebiculs dittum a trahendo ; nam non haber rotas ,
edé la noftra Treggia . F :

4L fangue tira , L? inclinazione , 0 genio le fpinge , le forza , Intende che le»

irelle,che le donne hanno in tefta , havendo fimpatia coal’ altre girelle , fanno
Seiderare alle donne quelle della carrozza .

NON m! imbroglio negli fudi . Ciot ; non attendo agli fPudi ; nan ho che fare con,
loro; nom mi intrometto di fiudiare ; nan me ne impaccio,

PUR troppi n’ bo fasti ful foglio . Per modettia intende ; Pur trappi fono gli er-

rori che ho fatti nel comporre la prefente Storia .

STANZA XxXU,
Perd feguiam con Paride le Dee
A veder cofe belle , e Strauaganti ;
E prima tronerem di gran mifcee ,
_ Corpi di Mummie,ed ofa di Giganti ;
_ Her in corpo a pefce due galee ,
Tmpietrive com turti i naujganti y
eps 9 li quali effe han per tradizione
Ci

fur fatti del gingerol di Nerone .
Larti del gingguol di Ner Hid

STANZA XXXIIL

Chinfe nel vafo poi vedrem le cotte
C’ bebbe quel Vecchio Chioccia di Sileng,
El afta che fu , dicon , di Nembrotte
Con che voile infilzar 2 Arcobateno ;
Benché fi creda pit di Don Chifciotte,
E veramente non puo far di meno,
Perché in vetta nel mexzo delia lama
V' é feritto Dulcineach' erafuadama,

2 STAN.

 
396 MALMANTILE

STANZA XXXIV.
Pende dal palco un fecco gran Serpente,
Che uh al Cocodrilo s' afomigtia ,
E dicon che 1a coda folamenre
Per laliighexza arrina a cingne miglia;
A1a quel che pitt curiofo di niente
E' certo, é una grandifima conchiglia , /
Ouxe fra minuta alga , ¢ poca rena Chi vi dipana fa quant’
Sta congelaro un’ uouo di Balena, C” al fin @ ogni gomitol fi
La(ciato il raceonto de’ libri , torna I’ Autore a narrar le cofe mai
fingolari,che fono in quefta Galleria, E perché in tali Gallerie i proc
le fa di riporui cofe flravaganti , ed ancicagliec ragguardevoli, ¢ molte da
ne fingono per accreditare il luogo , € pero 11 noftro Poera mette anche
mano di cofe iperboliche , come fono due galec impictrite in corpo 4 |
€ favolofe , come un vafo pieno di gotte , ec, Vedi Liaciano neil’ fitoria
ove delcrive terre , ed huomini in corpo'a una'balena; B Efiodo, ove
il vafo di Pandora , ove erano tutti i malori, ¢ tutti i malaoni ,
AUSCEE . Intendiamo bazzecole , mafieriziuole , ed arneli vecchi di
prezzo,che habbiano del curiofo ; metcuglio di bagattelle , di curiofita ¥
AV MME, Vedi (opra C, 6. fan. 52. i
GWVGGIOLO di Nerone , Habbiamo un ‘noftro detto,che é: Meron
ginggiole , che {erue per efprimere ; 4 fortuna mi s’ artranerfa; Ml Diaual
difce I’ efecuzione del mio penfiero, E viene non da Nerone iimperadore ,
contadino chiamato Neri , il quale ftava fopra un giuggiolo , offeruando
che entravano in cafa fua pee rubare , ¢ toftoro accortifi a’ efler ;
moftrare che gli volevano fare una burla , ¢. non rubare: gli ditiero; 4b WV
\w fei in ful giuggiolo , intendendo : Noi t’ havevamo ben veduro. E del lepname
di quefto giuggiolo dice,che eran fatte le due alee impietrite incorpo.al pele.

VECCHIO chioccia, Vecchio malandato . ' uno, che fia alquanto infermo de
ciamo chiocciare ; dalla chioccia , gallina vecchia,e {pelata , che cova i puleitl,
come il malato cova il letto; ¢ !Autore chiama Suleao vecebio chioccia
Icno Pedante, ed Aio di Bacco fi faceva portare topra aun’ afino, C
mezzo infermo ; ed i Gentili dicevano , che egii fi trattava in quelta forma, pet-
ché eflendo egli il maeftro di Bacco , il quale € numerato fra gli Dei el
amici delle comodita , ¢ del piacere , era gitilto , che fudde un’ huomo di tutti!
{uoi comodi. ar tag

VOLLE infilzar  Arcobaleno, Volle infilzar \Areo-Celefte ; che i! chia
mavano Iride , ¢ la dicevano infieme co’ Greci. Atmbafeiatrice degii x

wee

4B. 5.
: Frinde Colo mifit Saturnia Tino ,
Ed il noftro Poeta dicesche Nembrotce vole injfilgar & Arcobitteno
fu quello ; che Pe eer fi pensd di voler guerreppiar col Cielo ed.
to fabbricd la famofa Torre'di Babel, cioé della confufione . ar
DIN Chifeiorre , Che in noftra lingua voreébbe dire: Di
mile, Fu ua Giteadino-delaMuntia , il quate havendo letti molti

     
   
  
  
  
 
    
  
  
 
 
    
 

 

  
    
  
      
 
 
   
    
   

 
  
     

pA seop ok FLFR e wn lere2t sere res e-2s

ae

 
 

    
   
 
    
    
   
    

OTTAVO CANTARE 397
valleria , cio? Amadis di Gaula, Palmerino d’ Oliva , ec. s’ imbriacd , eddinuaghi
dei meftiero di Cavaliere Brrace di tal maniera,che fi meffe ad immitare le azioni
di detti Cavalieri » facendofi armare con quelle cirimonie , che eran foliti fare
“quei ; anch’ egli a cercare I’ avventure , come graziofamente rac-
conta 26 Michel Ceruates’nel {uo D6 Chi(ciotte,il quale fu molto bene tradotco
noftro volgare da Lorenzo Franciofini da Caftel Fiorentino, affai benemerito
' a Spagnuola ; (1 aggiunta , 0 fecondo libro del qual racconto' voglio-
_ no,che fia flato compofto da Carlo V. Imperatore ) E perché i Cavalieri Erranti
Ron erano ftimati veri Cavalieri , fe non havevano I’ innamorata , perd quelto
@ Don Chifciotte fi finfe ancor egli la fua,che fu Dulcinea del Tobofo; E da queftas
ae il noftro Poeta prova (cherzofamente, che quefla Atta fulle pid tafo di
| Don Chilciotte , perché nella lama’, che era.in cima alla detta afta v’ era (eritvo
‘Daulcinea , ed intende , che quefto ferro era doice , cio¢ di cattiva tempera.
FN gran Serpente, Quetta iperbole del Serpente ¢ pofta qui ad immitazionc, o

iat per dir megho , in derifione di coloro , che {crivono le Storie d' Etiopia , che>
wi di -eflerui tali Serpenti, che ingoiano un Ceruio , 0 un Bue intero per volta

_ €fono di lunghezza di piii di trenta piedi ; E che M. Attilio Regulo nella prima
til > oda ai Cartaginefi ne uccidefle uno in Affrica preflo al fiume Bagra-
it che era lungo 120, piedi .

MANTICE , 0 mantaco. Vedi fopra C. 1. flan. 55.
si) = SARCOL AIO, Steumento fatto di canne rifefie , 0 ftecche dilegno , fopra il
wai ‘wales’ adatta Ja matafla per comodita di dipanarla , 0 incangarla come s’ ¢ det~
wi WifoprarC. 5, flan. 9. E dipanare € raccorre il filo,formandone una palla per co-
shi imetterlo in opera , ¢ tal palla fi dice gomitolo dal Latino glomerare , e+
i Soma che il gomirolo , che a Roma ancora fi dice glomero.
4) STANZA XXXVI. STANZA XXXVUL
se Van Sfera bellifima fi vede , S’.in Grecia fatra fu la criftullina ,
nis © (Ch fopr’ aun ben tornito piediffallo , E quelta di vefciche vien da Troia ,
ae ‘Che per ginfiexza tutze l’ akre eccede, Che a Fiefol fu portara a Catilina
ail on farte di legno , 0 di metalios ue norte ch’ ei ee verfo Piftoia ,
es pure , ¢ fotrerrifi Archimede Ch’ ei non giunfe ne anc! alla mattina,

i Con lla fua, ch'ei fece ai Criftalla , Chet arate wi tio le quoia ,
am % Che bifogna guardarla,e /parfi addietra Sicché due Capitan fue camerate
; ie “ “Per'timor di non romper qualche verro, La prefero ye la diedero alle Fate,
2 STANZA XXXVIL STANZA XXXIX,
Che quefia 5 che con ogni diligenza Mentre s ammira cosi-bel lauoro

Di purgate vefciche fu commilfa ,

E vi fi fanno fu cento argumenti ,

iv Se perdisgraia , 0 per inavvertenza Paride guarda 5¢ vede una di loro
ae Perquote ocade,ell’ ¢ Sempre la fiefa; Canarfi un’ occhiolaparrncta,esdenti,
E sel criftalio ba in fe larrafparenza , E dargli aun’ altra,perch’inturto ilcoro

_ LA vefcica al Diafano s' apprefia , Delle. Naiadi ch’ ini fom prefenti ,
if --Edé\un corpoyche giammai non varia, O fuora (che pur anche fon parecchi )
eo E-quel fi cangia ognor fecondo t' aria, Ha fol quei détrynn'ecchio,e due cernecebi

Se.

 

STAN-
 
 
 
 
 
 
 
 
  
  
  
 
  
   
 
     
   
 
     
     
  
   

398 MALMANTILE. ©

STANZA XXXX.
Pero ch’ elle fon cieche 5 e vecchie tutte y
E loro i denti fon di bocca nfctti ,
Ma ni per questo ell’ apparifcon brute,
Ch’ ell! hanno volti bel , ¢ coloriti ,
E fe mangiar non poffoncarne , efrutte
Elle s’ aintan con de’ pambolliti
Perche quei denti,come gli occhi,eiriccs
Non hanno pin virtit , che fom pofticci .
STANZA XXXXIL
Cosi per offernar le lor vicende
lucha ch’ io dico fe gli caua adcffo, Cedendo ogni ragione y¢ ogni
Gia ritornata dalle fue faccende , Perch’ inqueff'oraa é
Perch’ il portagli pin non le ¢ permeffa, La fronte escape ,erife
Defcrive una Sfera fatta di ve(ciche di Porco, ¢ moftra , che fia
re di quella di Crifallo , che fece Archimede Siracufano , perché ¢ pili f
pid ficura . Mentre che Paride ftava mirando , ¢ dilcorrendo fopra ilb
della Sfera di vefciche, una delle Ninfe fi cavo la Parrucca sun’ Occhio, 1
¢ dette il tutto a un’ altra , perche cosi ¢ l' ordine fra loro, Qui pate, che:
alle Lamie, Donne , o Larne per dir m lio, che con carezze allettatrici
ftimate da’ fuperftiziofi Gentili mangiarfi i bambini ; le quali fea cutre
no un’ occhio folo , ¢ quello ufavano a viceada hor quetta her quella,fe
deicrive Angelo Poliziano lib. 3. tit, Lamia , che dice: Lamia h
excmptiles , hoc eft quos fibi eximunt, detrahuntque cum libuit , curl
y» cum libuit refumunt, atque affgunt ; alice vero ctiam dentibus utuntur eque
y» exemptilibus , quos noéte non aliter reponunt, quam togam, ficut ba CO
>> mam (uam illam dependuiam, & cincinnos , &c. Sed lamia hac quoties do
egreditur oculos fuos fibi afhgit , vagatur per fora ee plateas, &c, domum ye
»» ro cum revenic , in ipfo fatim limine demit illos fbi oculos , abijcitque in le
»» culos ; ita femper domi cca , foris oculata , ,
PIEDIST ALLO . Bi quceila pictra , che ¢ foto al dado , fopra il quale pola
colonna : ¢ qui ¢ prefo per tutta la bafe , che regge quefta fua Sfeca , comet prt:
fo comunemente . aia
VADA , efotterrifi eArchimede . E’o(curata la gloria d’ Archimede ; Quan?
uno fa un’operazione meglio d’ un’ altro diciamo al fuperato ; T# ti pwoi ire ari:
porre,oafotterrare . Intendendo ; Tu hai perduto cutto il credico , o la flimas
che ¢ quella fenza la quale uno é tra gli huomini come morto; i che
che non fi dee pit far taata Mima della Sfera d’ Archimede fatta di cri
ché quefta facta di veiciche I’ ha tuperata . 2
DATvoia, Non dalla Citta di Troia , come pare , che ogi dice
Troia femmina del porco , delle cui vefciche era formata guetta sfera
V1 tira e quia. Vimori. Vedi fopra C, 4. tt. 20, Qui cocea Ja con
nione , che Catilina famofo capo di congiura defcritco da Salu(tio mori
floia «
Vi fanno cento argumenti . Cio’ difcorrono afiai fopra quefta Sfera. —

xy

  
    
   
 
 

 

 

 
 

 

OTTAVO CANTARE. 399

Ml PARRVCO-A, Voce ftraniera fattz noftrale , ¢ vuol dire Zazzera , dichioma
®@ finta , che diciamo : Zazzera pofticcia dal Francele . Perroxque, chioma . Potreb-
» be forfe dirfi in Latino capidamentum .
- CERNECCHI, Capelli pendenti alla tefta ; qui intende quella parrucca,o ca-
peli poftice: ; fe ben cernecebi Gi dicono quei foli capelli , che pendono dalle tem-
_* pie agli orecchi con altro nome dette faceagore , che i Latini , fecondo i) Polizia-
no nel luogo fopra citato dicevano cincinnos , E noi diciamo cincinns quei ciondo-
_ lidi pelo , che fogliono haver i capretti , ed i Becchi foro la gola, i quali hanno
ft qualche fimilitudine con quetti capelli , che noi chiamiamo cernecchi .
PAN bollite , Pappa fatta di pane , bollito in acqua .
_ MASC-ALCIA, Magagna ; Difetto ; mancamento. E’ lo fteffo , che guida-
" Ielco , ma quefto fi dice folo nelle bettie , e mafcalcia , che farebbe veramente {o-
Todelle beftic, ’ ufGiamo anche per gli huomuni , ¢ taluolta per i materiali, Vie
‘un’ antico libro Tofcano intitolato Libro d+ e#a/calcra , che @ dell’ arte del ma-
nefcalco , de re veterinaria ,
DA quella via 01a quelia via, Subito . Senza metter tempo in mezzo. Latino

=

SEE

ill extemplo , ¢ veffigio . Se bene fi potrebbe intendere ancora per In quella manicra;

ind in quella guifa , come ¢intefo fopraC. 7.0.84. 0

na oN ogni regreffo. Cede ogni azione ; ogni autorita . Vedi fopra C. 7.ft.104.

git. RIFERRAR (a bocca, Incende rumettere i denti. Bocca sferrata fi dice a uno,

em =the habbia meno i denti dinanzi dal ferrare le beltic , ¢ rimeteer loro i chiodi a’

il pied, fi fono sferrate .

mw STANZA XXXXIIL STANZA XXXXIV.

0 Pitna di cibi intanco nna credenza Credilo a me ch' eglié del eloriofo

pal Wit pari pari aperta spalancata , Pero qua dentroyvia,diftendi il braccio,

8 OB fatta da vicin la rinerenza Che trouerai del buono , ¢ del guftofo,

ye Parole pronunzio di guefpa daca: Se tu voleffi ben del Caftagnaccio,

pi ° Caualier , fe tu voi far penitenza, Paride fece un po del vergognofo,

on ‘Ein parte a noi piacere,e cofa grata Hla nel veder le bombole nei ghiaccio,
440 munizion da caricar la canna , Mando prefto dabanda la vergogna,

gl E poi da bere un vin ch’é una manna. E fece come i Ciechi da Bologna,

oF STANZA XXXXV.

Levategli poi'via la calamita Sicch? in quanto ad bauer taglioo ferita
ost Di quel buon vino,e maffime del bianco Jn altra parte era ficuro ,¢ franco,
ipo Gli fararon le Dee tutta /a vita Poi dangli un brando con la {ua cintura,
gil  WDalla baferra in fror del laro manco, E del trattarlo  intavolatura ,
ye Mentre ftavano guardando le fuddette galanterie,comparue und credenza aper-
i ta piena di roba da mangiare, ¢ da bere, ed inuiid Paride a foddisfarfi; egli dopo

haver fatto alquanto lo Riiakian’> mangid , ¢ bevve ; Terminato il mangiare fe

46  ‘Ninfe lo fatarono,rendendogli impenctrabile tutta la perfona , eccetto che la ba-
‘fetta mancina + Qui il Poeta immuta |’ Autore , che favoleggia Orlando impene-
o@ _ trabile in tutta la perfona , eccetto che nelle piante de’ piedi .
@ | CREDENZA. Cosi chiamiamo un’ armadio , entro al quale fi ripongono , ¢
tonferuano gli arnefi , ed avanzi della menfa ; il quale armario fi dice ancora,
4 tredenziera  perche quei bicchicri vaii , ¢ baciji d’ argento, ec. che fi mettono al-
Ic

 
400 MALMAN TILES ©
le tavole de’ Grandi per feruizio ; 0. per apparato della
diti tutti infieme , fi dicono credenza , ¢ i fi rij
vriano riporre in detto armadio , che perd lo chiaa
tino eroacns re

SPALANC AT A. Affatto aperta. Vedi fopra'C. gf. 38, Pal
cato diciamoa Ja chindenda\, o riparo fatto con é pali, a un >
vuol dir Senza palanca , ¢ per conleguenza totalmente aperto, ©
tegno , 0 inipedimento . ei é

PAROLE di quefia data , Parole (mili a\quefte , 0 di
ta , la quale fi attende moltifiimo nel gioco delle'carte’, per ef
chiate; Onde fi dice: Ha farrauna buona ,o una cattiva data,

SE tu vuoi far penitenza, Se tw vuoi mangiare . Termine ufato per :
inuitar’ uno a definare , o cenar connoi , guafi ditiamo:venite a digi ;
ché la nofira menfa ¢ povera , ¢ (eatlardi cibi . Sidicc.amcoralfar carisa 5,00

s'é vifto fopra C. 5. ft. 68, otis verb 9!
HO munirione per caricar la canna. Ho roba da mangiare\, eda
care Ja canna della gola., ¢ non quella dellarchibufo <2 5, Gate
VN vin, ch’ una manna , Vino ifquifitifimo, che tale fi legge fulle!
che mando Dio nel deferto al Popolo cletta., Vedi ferro Cip.ft58..Ma
ftranicra , ma fatta nofirale,che fignifica una brina:condenfata: tenera’y’
detta cos} dali’ Ebraico, Azanbm ; ioe Quid of bee: come: fi dice nell?
16. Poiché maravigliati gli Ebrei di quefto nuovo, ¢ faporolo cibo 5
uno all’ altro; Che ¢ cid, che no? mangiamo ? Dalquefta dolcezza viene ‘
noftro detto . 1 Latini dicevano in quefto propolito dowis Velar..
EGLI¢ det gloriofo. I Battilani chiamano vino gloriofo il vino pga 4
rofo , e buoniffimo , e dicono groliefe in vece di gloriofe ; cio’ valor lo
vaalle ftelle. In certe Profe Tofcane antiche, delle quali alcune fi ritroyanom
nuicritte nella Libreria di S, Lorenzo date fuora dal Doni, vi ¢ una leteera amt:
rofa , nella quale ¢ accennato Amore con dire: Quel gloriofe ; titolo dato

da’ nofiri Battilani al vino; ¢ veramente Amore non imbriaca meno di

fi faccia il vino il pith gloriofo, a
VIA. Quetto termine ferue per follecitare , © incitare uno . Latino Eia ae,
CAST AGNI AC C/0.Pane fatto di farina di Caftagne : qui vuol

per opera d’ incanti quella credenza dava tutto quello, che uno fapeva

itis g

tern gab

   
    

rare. : a
FECE il vergenofo. Finfe dinon fi ardire a mangiare. Moftrava vergogaitt
d’ accettar I’ inuito , che gli faceva quella credenaa. ig
BOMBOLE. Vali di vetro , i quali feruono per mettere il vino in
ghiaccio , 0 neve , detti cosi ( fecondo alcuni ) dai fuono , che fanno-nel |
fuori il vino , che par che fuoni bombof . Al Rotenano' vuole , che i Latini
da tal fuono le diceflero amphore bilbina ; ma pud anch’ eflere , che agi ie
cost da bombo voce.puerile 5 che vuol dir bevanda , decta cosi dal fueno, —
COME i Ciechi da Bolugna, Si da loro un toldo , perche cominci

c bifogna poi dargliene duc , perché fi chetino . Ci iciue per efpri :

 
   
  
 

 

 
 
    
      
    

i
mt

SBER ote

=
=

Ask RESS Pa SeETLE

 

T~s-e

>

 
    
   
     
   

   
 
   
 

   
    
   
 

fs

rad
ike

i
wb 1

-

4

  

4
4
of

4

   

a 4
_ Pigliual
| Ainqud franortese thcome,e ilquddo,e il doue,

 
 

Escciam

Ciechi da Bi

 
 

re.
~ STANZA XXXXVI'
‘Eperche if tempo ormai era rrafcorfo y
jarlo dowean di quini altroue
Prima in fua lode fatto un bel difcorfo,

( differo ) quanto t° ¢ occorfo

Py ae tutto per appunto,

Ot é gus ra nofira giunto,

oc SPAN ZA KLE X Vil.

Akcibsuvada incontro a un'aunentura

< d prod’ un pover huomo quefta notte ;
Quelle é un tal cognominato il Tura,
CH in Parion confiaua le pilotte .

0 Eta tebellenze un mofiro di natura,

Sicebe tutse /e donne n' eran corte 5

Elafeiando i roccberti , ed i cannelli

2 Per-lui chee cht ¢ facevano a capelli,

“STANZA XXXXVIIL

Non ch? eine deffe loro accafione ,

— Come qualcbe narcify inesbertato ,
C'una caffia, che e' vegga a un verone

?0/ «far lo {pafimato ;

‘ Anis wn diqueie'al Addo ta 4pigione,

«A biofcio nel veftire , ¢ feiammannato,
©’ addoffo i panni ognor tutti mincfira

0) Tirati gli parean dalla finefira ,

“OTTAVO CANTARE,

maghana a Adarte,al Soleje aGiove,

gor

 ptegare a far’una tal.cofa moftrando non voler farla , ¢ bifogaa
ehe refti di farla , Orazio . :
» Omnibus hoc vitinum eff cantoribus , inter amicos
0) Wt munquam inducant animum cantare rogati ,
drinffi numquam defiftant ,
Sid > da Ferrara , 0 da Milano, 1 Latini in quefto propofito
‘ditlero Arabicus Tibicen. Qui intende , che Paride fi fece pregare a mangiare ,'¢
fe ¢ poi non fi trovava 11 modo , che egli reftafie .

LAMIT A,B! \a pictra Adagnes , la quale ha proprieta d’ attrarre i] ferro,
punto ha il vino di tirare a fe Paride , ed é fra etio, ed’ il yino Ja ftetla,
»cheé fra jacalamita , ¢ il ferro. Vedi fopra C, 5. ft. 59. B fotto ing
— guelto C, tt. 66,
| | Di trattario ? intavolatura’, L? inftruzione di come fi debba adoprar quella {pa-
t+ Intavolatura ¢ fcritwura , che per yia di note, ¢ di numeci regola la mano del

STANZA IL,
Ed effeeran capone; ma chiarite ;
el fin lafciando quel fuocnor di fmalto,
Fecer come la Volpe a quelia vite
C* hauea si bell’ Una ye tanto ad alto,
Che dopo mille prone , anzé infinite
Arrinar non potendoni col falto ,
Glié mé,dilfe,chio cerchi altra paftura,
Che quefpa da ogni mo non é matura,
STANZA L,
Cosh non Ia faldo era Martinarza ,
La qual non vitrouado anch'elja attacco,
Poicht gran tempo andata ne fu parra
Hanédo if rerzo,e il quarto,e ognuno firac
Codurreun giorno fecelo alla mazza, (co
E per via d’ un che le teneua il facco
Aunezro a tofar pecore , ed agnelli ,
Mentr’ ci dormina gli taglios capelli,
NZA LI.

Quei capelii c un rempo hanea chiamati

Del fuo falcto mortal funi , ¢ risorte,
Le bioae chiome wh Dio,queicriniaurati
Che ricoprivan rante piazze morte,
Onde feoperti furo s trincerati

Onc il nimico fi facea si forte ;

Perché (per quanto un Autore accenna)
Lo rimondavon fino alla cotenna ,

ale fate dopo haver lodato Paride per bravo , per bello, ¢ per valoro(o gli dif-
fero sche I! havevan fatto capitar quivi , perche egli andatie a liberar il Tora,
quale loda ironicamente , ¢ dice , che tutte le donne crano innamorate di iui; ma

 

Ece

accor-

 
   
     
     
   
 
     
   

qo2 MALMANTILE (9

accortefi , chenon corri leva a niffuna , lo lafciarono, ¢:

egli non volle mai corrifponderle , haveva fattagli Ja malia
ottave feguenti .

1 DE
AVVENTVRA. 1 Romanzatori Spagauoli in quei loro Amadis di Gaul
Palmerini d’ Oliva chiamavano ayventure ( awenturas) quegli i
ne i quali s'imbattevano i Cavalieri Ecranti, ¢ perd il noftro:
creato i] Cavalier di Quoio,vuol, che ancor’ egli fia fimato i

che vada a provare I’ avventura di liberare il Tura dail’ incantefimo. 1
fimilmente diflero aduentures, Bi noftri Tofcani ancora , fentendofi:
termine cavallerefco , chiamarono gli accidenti , che accadevano

davan loro materia di fare prodezze dawenture. L’ Alamanni nel
principio .

 
  

3 teal

Narrero di Girone P alte auwenture. a ad

E da cid i] Boce, Tels. lib. 5. diffe: 5 aabiitly
eetterfi in aunentura : . ee

Ma non li parne via ben ben ficura . poe

Pero non fe ne mife in auuentira, sung

4L Tura, Coftui cra un pover’ -huomo , che gonfiava le pillotte in] }
che in Firenze é la ftrada, dove fi giuoca alla pillotea detta cosi da
perché in efla anticamente haveano le botteghe coloro, che lavoravano:
mi, 0 pure ( il che forfe ¢ pitt verifimile ) quafi Ripar Regio Ripe Roine
tale ftrada sbocca ful Pafleggio di Lung’ Arno; [n Roma ancora vi ¢ la
da di Parione detta fimilmente cosh detta quali Rione a Ripa, Regio Riper
pure é cosi chiamata , quafi Parte di Rione ; Pars regionis , come mi vien riferito
leggerfi in alcune Carte , o Contratti. E perché veramente coftui era brut
di faccia , ed haveva la zazzera avviluppata’, elorda , lo chiama moffro
va in bellegza, ed intende Deforme , fe ben par, che voglia dire , di belleam
fopranaturali. aise
PILLOTT A , Specie di palla da giuocare , Vedi fopra C. 6. ft. 34.
WV’ ERAN corte. Erano abbruciate dal fuoco d’ amore per lui Virg, Mir
infelix Dido: dice briache del {uo amore , cs’ intende innamoratiffime di Iai. Lt
ebrie amore. Piauto nel eAilit glorifo,o Soldato,al quale da nome di
cio’ di Abbartitore di Torri ,e di Citta; 0, came noi diremmo Taghacantoiy©
Spacca Montagne ; fa dirgli da e4rtorrage , ciot in noftra lingue Sparapane Pata
to, fuo adulatore ; che tutte le donne fono di lui fieramente innamorate » Le
tibi ego dicam ; quod omnes mortales (ciunt ; Pyrgapolinicem te unum in tere rn
Virtute 5 & forma , & fattis muittiffimus? Amant te omnes mulieres,neque herce i
ria, Qui fis tam puicher . Ed egli {prezzatore altero di tali amori iange !
Jamente Ja fua difgrazia , beccandofi fu quefte lodi; dell effer hwoWd, |
da fare innamorare di Jui tutto il Mondo , WVimis eff miferia
pimis. ‘

LASCIANDO i roccherti 5 ed i'cannelli. Lafciando ftar di lavorare.
prefe tanto forte I’ amore; ¢ tanto le teneva fifle nell’ amorofo penk
non potevano pili atcendere a’ loro ufati lavori. Quando Didone fi
‘rata d’ Enca., non tirava innanzi gli edifizj.,.¢ le fabbriche della faa

  
  

 

 

   
  

 

preaereseei ns

 
  

OTTAVO CANTARE:

Virgilio ebbe a dire: pexdemt opera interrupta , minaque Adiroxam ingentes) come
che era occupata da pitt poffente penfiero. Co! prefente detto di la{ciare
droccherti , 4 canneilé , s' intende quefto , perché le donne a’ infima plebs (che ta-

1 epeweneenye > che erano I’ innamorate di coftui ) per lo pit non hanno
lavoro

» che ? incannace , ¢ teffere , a’ quali lavori s adoprano i Rocchetti (che

fon legnetti tondi forati per lungo , ¢ feruono per ragunarui fopra la feta , ed

‘,

oo LU BRE

AaeTEEsS

TS A A Se ee, ae

altro filo : ed i Cauneii , che fono pezzuoli di canna tagliata fra un nodo, e
Paltro, dai Latini perd detti ixternodia , ¢ feruono per lo medefimo effetto d’ a-
dunarui fopra Ja feta, ec. per adattaria a telsere ; I che fi dice incannare.
hone éch’é, Ad oraad ora; Di momento in momento. Vedi fopra Can. 3.

 PACEVANO a i cape, Si perquotevano, S'azzuffavano. Quando due donne
combattono tra di loro diciamo fare «4 capelli ; perché il lor perquoterfi, ¢ per lo
_ pit) pigliarfi 1 una I altra per i capelli.
_ CVFFIA. Berretta a foggia di facchetto , entro alla quale le donne & ferrano
icapelli in tefta , ¢ quando noi diciamo nel modo , che ¢ detto ne! prefente luogo
tna ¢xfia , un ciapperone , ¢ fimili arnefi ufati dalle donne, intendiamo una Don-
na, Cosi dal portare lancia, o barbuta ; i toldati medefimi fi chiamavano Lance,
¢ Barbute, come ficava da Matteo Villani, 11. 81. ¢ Erodoto volendo dire, che
pera fi ritrovavano avere in. piedi ottomila foldati , che portavano rotel.
90 brocchiere ; diffe ottaci/chilian afpida , cioé {cudi militari,o rotelle ottomila.
VEKONE . Latino moenianum , podium , pergula, e in Greco fecondo alcuni Pe-
ribolas da peribaliern abbracciare , circondare , che i Francefi dicono enxironner .
Propriamente vuol dire andito , o terrazzo fcoperto’: Qui credo , che habbia a dir
Baleone ,¢ non Verone . Verone & detto quafi girone , cioe giro , dall’ andarui fopra
¢rigirare , eAndito , che ¢ lo fteflo par facto da e4ndare. Latino ambulatio ,
EGLiéa Pigione al mondo . Cosi diciamo d’ un’ huomo {penfierato , {ciatto,{en-
2a confiderazione , ¢ che vive a cafo , che fi dice anche Auomo a biofcio , fciaman:
nato ( cioe male ammannato,male all’ ordine ) e che i panu# gli paiono tirati addofsa
finefira,B cd quefti quattro modi di dire |'Autore delcrive l'attilatezza del Tu-
ra;del.refto , parlando fecondo moralita , ognuno dourebbe flare in quefto mon-
do, come a pigione; perché la noftra propria cafa é nel Cielo. E nel Salmo 118,
4ncola ego fum in terra , i) Greco dice Parcecos, ¢ alcuni Salteri dicevano , come ri-
ferifce $. Agoftino fopra i Salmi , inquilinus , ciot pigionale . F
CAPONE, Oftinato Latino. Pertinax . Pertiwax.
FAR come (a voipe alla Vite. La Volpe dopo haver molto faltato, e dopo efferfi
i per arrivare un grappolo d’ vua , ¢ non I havendo potuto arci-
vare difle: La voglio la(ciare flare , perché ad ogni modo ella non ¢ matura..
Pud aver data occafione a quefta novelletca quella d’ Efopo , della Volpe, € del
Pruno ; in cui la Volpe , che voleva falire una fiepe , mi fuppongo , per mangiar
Tuva , della quale é ghiottidiima , penfando di troyare il Pruno buon’ amico , «.-
fto ingannata del fuo penficro ; poiché attaccandovili refto intaccata , e ! appog-
gio le fu ferita , e volendola poi difputar con lui , ebbe il-torto: E quefto detio
¢i ferue per efprimere uno , che habbia ufata ogni poflibil diligenza per confegui-
Te una tal cofa ,¢ non ’ havendo potuta ottenere , 9 habbia abbandonata i’ im-
Ece 2 prefa

ae
F
7

 
 

404 MALMANTILE ©

prefa come impofiibile , o fia quella tal cofa fata data a un’ al
vanti di non |’ haver voluta , perché:non era buona, 0 non!
diciamo : farfi honore d una cofa n
COST’ non fa falde eartinazza y Cos\ non fini, 0 termind ?
nazza la quale non troxando attaeco , cioé non trovando luogo di f
fuo amore ver(o il Tura , del quale ando paz ca,cio’ fterte innamoratiti
CONDFRRE uno alla mayza, Tradir’ uno: Condarre uno con inga
finghe in mano de’ fuoi nimici , o della giuftizia , o in qualche altro
come fi fuol dire ; al macedo , Latino /n infidias ducere . re
TENER il facco, Tener di mano. Aiutare a cometter un delitto . Ha
un proverbio fentenziofo , che dice: Tanto ne va a chi ruba , quanto a
Jucco, che efprime Agentes, & confentientes pari pena puniuntur . B diciamo anche
Tenerfi il facco l'un t altro; che efprime il detto di Teren. Tradere operas mutnas
FEY NI, ¢ ritorte del {uo fafcio mortale . Metafora amorofa + Si ne
ritorte tengono unite pill legne in un falcio , 0 faftello , cosi i capelli del Tura,
quafi funi , € ritorte tengono unita col corpo |'anima , cioé tengono in
Amanti del medefimo Tura, E riorre dicemmo , che cofa fieno fopra C.
PLAZZE morte, Si dicono i luoght vacanti de i foldati ; per efempio
no € pagato per cento foldati , ¢ non ne ha f€ non novanta ; quei dieci
cento , che mancano fi dicono piazze morte. Ma qui intende quelle pi
la(ciano Je margini , o cicatrici de i mali , che vengono nel capo ; fopr”
li non natcono capelli. “|
1 TRINCIERAT 1, V luoghi , dove erano le trinciere. Intende ,
gliargli i capelli fi fono fcoperti quei Inoghi, i quali con- elle: margini
una campagna piena di trincicre . done tl nimico fi facena forte s clot di 0
devano i pidocchi . we tage
TRINCIERA, oT rincea, EB’ un’ alzamento di terreno: condotto a foggia d
baltione , nel ricinto del quale dimorano i foldati per difenderti dallartiglierigg®
de i nimici. Franzefe rrenchée , ciot tagtiata , yene
LO rimondaron fino alla cotenna , Gili tagliarono i capelli fino rafente la pelle.
Rimondare vuol dir Tagliare a un’ albero i rami: B curenna’s invende folo lap
le del porco , ma quando fi tratta del capo s’ intende anche quelia dell huom
Vedi fopra C. 5. ft, 52. * 4a?
STANZA LIL STANZA LALO
E cos) Aartinaz2a hebbe il fuo fine E quefto Lupo raggirar fi vede 4 2
Volendo vendicarfi per tal via y Intorno a un montnofo cafameme—
Pero sche buona parte di quel crine,
Ch’ aleun non few avvedde , leppo vidy
E fabbriconne al Tura le rove
Con una potentifima malia,
Che revifrata in Dite al pratocollo
fa un Lupo rapace trasfor mola,

  
 
 

 

   

4

NS ne ears ESR ene

D' una Genteycheymentre mice ilpitl

 
 
  
 
   
    
 
  

r

yor bS

 

Zexv7ek
 

OTTAVO CANTARE 403,
o» STANZAEIV. 5 STANZA LV;
d vanne,e perche tu non facia Eli la prende con il libro infieme ,
— Qualche marronyma vegas arar drittoy Dicendo , che varraffi dell! avvifo y
- Acco tal: ero fi disfaccis

f Pi disfaccia y E.ched’ incano ye dianoli non teme ,
- Percht feattado un pel tu baurefti. ifritto, Perch? eglirehuom,che fa moftrar ilvifo,

 

Yi he quefto libro qui faccia per faccia Si parte,e per c'al capo andar glipreme,
hl ordine, ei modo fi ritrova [critto, da due parti vorrebbe efer divifo ;
ipa Portalo teco, e.accio che rm difeerna , Pur vnol feruirle , perche fi figura
Perch! egli ¢ buio to questa laterna , Che non ci vada gran manifattura ,
i Metono: STANZA LVL
wit poi nel fuo ceruello , Ricerca nel [uo maftro feartabello
ia Che sa quel luego a bambera s'inuia Di quei pacfi la Geografia y ;
| Potrebbe andar a Roma per eMugello, Aa quel(per quato noi potré coprendere)
oPerchvei non fi rinnien dow’ ei fi fia 5 Non fi vorria da lui lafciar’ intendere,

Hi
id ~~ Martinazza:hebbe il (uo intento, perché prefa buona parte de i capelli del Tu-
sf 4 con effi gli fece una malia , che Jo trasformd in lupo, ¢ lo confind in ua mon.
ati tevicino:a Maimaatile . Finito quefto racconto le Fate licenziaron Paride ,¢ gli
r diedero un libro , dove era fcritto i] modo da tenerfi per disfar quell’ incanto , ed
we wna lanterna per farfi lume ; ¢ Paride fi parti con rifoluzione di sbrigar quefta_s
we faccenda prima d’andare al Campo.
_ LEPPO' via , Portd via di nalcofto . Il verbo /eppare ci ferue per efprimere
svélocita ‘hell’ andar via 0 nel levar via quaicofa .
4 MALIA. IncanteGimo , fattucchieria , ftregoneria. i
1” © PROTOCOLLO . Libro pubblico tenuto da i Notai per fcrivervi fopra i con-
oe trattiyeteftamenti 5 ¢ cost ¢ intefo da noi; fe ben protocode vuol dire libro da re.
| giftrarvifopra , che che fia. 11 Berni fonetto in biafimo d’ una mula dice :
4 > — E troppo fia diginna
ie ts srry? Cht il prorecollo memoria non fanne .
Perché veramente Prorocoo.é ua libretto,fopra il quale 4 fegnano, ¢ regiftrano
r brevementé le cole , per diflefderne poi {crittura pil largamente , ed autentica.
i  Msnce: deteo cosi quali 5 primo libra incallaro , ¢ legato. Liber ex glutine compattus y
o in guematta referuntur, Ma il noftco Poeta jo piglia nel fenfo , che oggi ufiamo
di libroida Notai , ¢ inteade che Martinazza haveva fatto contratto cal Dia-
mM volo di quefta malia ; il qual contratto era gia mefio al libro del Notaio del Dia-
it volo yeper quefto detta malia era autenticata , ¢ non fi poteva alterare , perche
' era paflaca per mano di Notaio , ¢ regiftraca al {uo protocollo .
#  - CASAMENTO montnofo, Intende i] Caftello di Montelupo , che ogg? quali
# — diftrutco siperd pili colto Cafotare , che Ca/iello, e lo dice montuofo , perché &
fopra un monte come lo moftra il nome medefimo, E nota, che ancor qui il no-
® fico Pocta vaimitandoi Romazatori Spagnuoli , che fanno parlare ofcuramente,
| e come gli Oracoli quei loro:Alchifi , Zirtee , Wrgatide , cc, incantatori.
« MENT-RE move il pit fopra alia terra v’ ¢ rinuolea drento : Le reliquie di quefto
/ — Caftelio fono abirate da perfone , che fabbricano valellami di terra 5 come pen-
tole , boceali, ec, quali fi fabbricano per via d’ una ruota , la quale va moffa cot

piedi , ¢ fa effctto del tornio , ¢ perché in muover desta ruota , € fabbricare il
it valo,

  

aiies

 

 
   
  
 
  

oh MALMANTILE
vafo , la terra {chizza addofio a chi lavora perd dice Ademtre shane il pit fopra alla

terra v' é rinuolta drento ,
FAR’ un marrone ; Far’ un error grandiffimo ; wx crrorome,
e4¢KAK aritto , Operar giuftamente, Non fare errori. Tolto dal Bifo
ciamo ancora , rigar diritto . 4a) aA
SCATT ANDO un pelo . Se tu ufcifi punto dell inftruzione 5 che tu’
tare , 0 fcoccare , fi dice della freccia quando {eappa dalla cocca , t
di gui € tolta la metafora , o forfe dal!’ orivolo'a ruote . See
TV hauerefti fritto. \\ Proverbio dice: Come difela Tinca ai Ti
altra aggiunta s' intende : oi habbiam feito, Qui intende tu hanrelti
tu haurelti rovinato quefto negozio, EB’ lo fteilo che: Noi habbiam
ne detto fopra C. 7. ft. 60. ow
HVOM che fa moftrar il vifa. Huomo ardito , e che non fee cen
ABAMBERA, Acaflo. Latino Jrconfultd. Vien forfe da ¢ é
vuol dir ragazzuolo,(enza giudizio, B il ragazzo in alcunt luoghi chiamato Bar
berottolo. Dicefianche . 4 fanfera . 2 6a
ANDAR 4 Roma per Adugello, Far’ una ftrada al tutto contraria , come &
rebbe andar da Firenze a Roma, ¢ pigliar la flrada per il Mugello , che € dirt
tamente contraria . oa
NON fi rinuiene, Ciok non riconofce in che parte ci fi fia , ¢ non fa quelchid
fi debba fare. aed
MAST RO fcartabello . Tntende quel libro , che gli haveano dato naa

  
  
  

  
 
 

é il {uo maeftro , e direttore ; Quetta voce /cartabello, ¢ corrotta da C.
anticamente era intefa per un libro di ftima ; come moftra il Dortifimo 5:
ditiffimo Sig. Francefco Redi nelle annotazioni al fuo bellifimo Ditiramboac
18. Gli Spagnuoli chiamano Careape/ una {crittura continuata nel foglio fens
voltarlo , comes’ ufa negli editti ; dal’ efere cred’ io , non ripiegata , come ifr
gli, ma ftefa , come una pelle ; 0 perché Gi diftendeflero tali forte di {eriteure a0.
3

in carte ordinarie , ma in pelli , ovvero in cartapecore.

 

” we an ee ae Le Lae eeasS we

STANZALVIL STANZA LVIIL ~~
Fu Paride perfona letterata , Ma benche 1a lettura fia fantafticay
Che gid udiato havea pit d'un faleero, A un che , fi pu dir non fa niente 5
AAa pei, non ne volendo pik fonata, E 0? altro di virtit non ba fe
Alla squola fiudi di Prete Pero, Che pelle pelle 1 Alfabeto a mentt,
Pera s' ei non ne intende boccicara Tanto la biafcia , strologaye rimafice
EB? da {cufarlo; e poi, per dire il vero, C’ 4 compito leggendo finalmente —
Lettere , ed armi van di rado unite 41 funto apprende,e fra Lalere fue ciate
Per ¢’ han di precedenza eterna ite. Ripone il libroye sprona i le foarpe.
. STANZA LIX, otra
Cos} commina, ¢ a quel Caftello arriva, Aa perche’ non ¢ tempo cb’
Paffa dentro, lo gira ,¢ fi fiupyfce , Quanto col Tura a Paride
Che quixi non fi vede anima vina Con buona gratia vofra Fare pasft »

Perc'aquell ora in cafa ognun poltrifce, Per difinir di Piaccanseo la catfte

 
 

» Jee cold

OTTAVO CANTARE; 407
STANZA LX.

 

Che da.quei trifti, com! io diffi ananti Di poi gli feeffi fel cacciaro innanzi
| ( Fatto mentre pappana affegnamento Ginfto come un Villano in fu il giuméto,
_ Diinfaccarfi per lor -qnei boccon fanti ) Pungolandolo, come un’ animale
_ Tocco de é pie nell’ Crsenal del vento; Fin , che lo {pinfer dove ¢ il Generale,

_. Deferive le qualita di Paride , ¢ dice , che egli cra letterato , perch? havea let-

@ to pidd’ un faltero,, che ¢ quel libricciuolo , contenente alcuni Salmi ; che fi da a

  

Aeggere a’ ragazzi quand’ hanno imparato a cono(cer Je lettere dell’ Abbicc: ; E
Tamuiodke , samedi che vm fapeva troppo leggere ; ¢ dice, che non ¢
da far meravigiia di quefto , perché I’ armi , ¢ le lettere mai furon d' accordo ,¢
»perd egli , che era armigero , cra {cufabile , fe non era letterato ; con tutto cid
‘compitando leffe in quel libro , ed intefe quel ch’ ei doveva fare ; ed arrivato al
-Cafamento montuofo trovd che ogauno dormiva . E qui!’ Aurore lafcia il parlac
i lui, ¢ torna a parlar di Piaccianteo , che la(cid fopra nel fine del Canto 5. e>
dice , che a-furia di calci ¢ pungolate fu da coloro condotto doy’ era il Generale,
© NON ne volendo fapere pin fuonata. Non volendo pii: {entirne difcorrere di tare
tuna tal cofa , ¢ qui intende non volendo pit ftudiare .
LA fquola ds Pretepero, \n{egnava dimenticare .
© NON intende boccicata , Non ne intende punto. Non conofce a pena le lette-
“res perché boccicata ftimo che venga da abbiccs , ae dica non fa lt Abbicci , che
pt oma , che con i Greci ancor noi diciamo diphabero, el’ ufa il noftro Poeta,
fente Oxtava 58. Procopio nelia Storia {egreta narrando I’ ignoranza di
- Giuttino imperadore , che poi fi adottd Giuftiniano; dice che egli era Analfubeto,
»tioé sche non fapeva I abbicci ; ac {crivere il {uo nome.
» PELLE pete. Superficialmente. &’ lo ftetlo che baccia buccia detto fopra C.

fe flans27, b 4
© BIASCIARE . Mafticare fenza denti ; cioé con la lingua ,¢ col palato . Qui
‘lntende quello ftudiare,che fanno i fanciulli , quando imparano a leggere , ches

“prima di rilevare , o profferic ja parola,che leggono , la compitano [utte voces,
‘con la bocca i] medefimo gefto , che fa uno, che biafcia ; ¢ lo fteilo vuoi

dire quel rimaftica , ec, ¢ firoluga intend: cerca d indovinare quel che dica queila

{crittura .

. + Leggere a compito , ¢ quello accoppiar le lettere , ¢ fillabe , ches

fannoi fanciuili , quando commciano a unparare a leggere , il che fi dice compi-

tare. cio contare a una a una Ie lettere , per poi fommarle, per cost dire, in

“ana parola ; il che fi dice , rileware ,

~ CLARPE , Bazzecole , Vedi fopra C. 3. flan. 5.

SPRONAR te fearpe. Detto afato per burlar’ uno , che viaggi a picdi

ANIMA vina . Ancor fopra C. 6. flan. 19, fi ferue di quefto detvo aflai ufate
“anol, fe ben fi fa che l’anima fempre vive;e qui vuol dire,che tutti dormivano.

POLT RIRE, Dormire. Vien da Poltro , che vuol dir letto ; circa che vedi
forto C, 9. itan. 39.

FA-CIAM panfa , Ripofiamoci : 0 fermiamoci . Frafe Latina venuta dal
Greco , ufata anco da noi, i quali da Paufa abbiamo fatto Poa, eda Pan/are
‘ufato pure da’ Latini de’ tempi bali , ’y/are~

; ae.

 

 
 

  
 
 
   
 
 
  
 
  

408 MALMAONTILE TO
ARSENAL del vento. Ripoftiglio deViyento § cioe il
dire una ftanza , entro-alla quale &i fabbricanoi Dante laf. <
unle neil Arzana de’ Keneviani, » 6 6) es
Mu hoggi fi dice:ax/enale , e credo che fia parola corrott. 3
@rx manalis , ia quale origine viene approvata-dal Ferrari.)
PYNGOLARE ; Stimolare . Pangolo ¢-quel-baftone con tuna’ punta a
@ acciaio in'cima j\del quale fi erupnoi congadihi per re
camminino ; Lat. fimulus 7B quetto fidicepangalare’,: caiy iss%
STANZA LXA vogcl now pS STPPARNg a
etppunto rl Generale a faris' epofto . * ‘Cofteroal fine fegli

a

      
      
 
  
 
 
   
     
    
    
   
       
     
  

 

 

Alle minchiace ,¢d ¢ cof ridicola. la) Rersdingli del prigion
Jl vederlo ingrugnato , ¢ mal difpofo. Ma e¢ poffom predicar
Perchigli ¢ fata morta una verzicila, ©. Perch'eglich’e' mele
Le carte ba dato mal, non ba rifpofto, ©. Oo “Eiperde una gram

E poi-dt non-conrare anco\pericola E gliene duole} ¢ now
Sendofcopertobaner di\pin nna carta yi. . Lornonddrerta,e:

Perch dtrado , quando rubay foatee, 00° Pletofamente fa qucfte fi
-Cofloro , che conducevano Piaccianted parrivarono al Ge eye
va ginocando alle Minchiate , ma perchéegli-+haveva fatto-unaln a
perdeva , ¢ perd rain colicra , in-vece d afeoitare quel che effi dice
fe a dolerfi della Fortuna -; come fentiremo apprefio, D i nonedyy
MINC HI ATE ,£ un ginoco affai notodetto anche 7%
Germini » Ma perché ¢-poco ufato' fuori della noftra Tofcana,o d
mente da quelche uGiamoinoi:, pér intelligenza delle prefents Otraves
ceflario faperfi, che il giuco delle Minchiate fifa nella maniera che
E’ compofto quefto ginoco di novantafette carte , delle quali 56. d
racce , € 40. fi dicond'Tarocchi', ed una,che fi dice i/ matto : 1é carte
in quactro {pecie, che fi dicono femi , che in quattordici {ono efigiati De
da Galeonto Marzio diconGi eflere pani antichi contadinefchi ) 10 1.4: Coppe
14. Spade, ed in 14. Baftoni , ed in cia(Cuna carta di que(ti fea cominciad
(che fi dice afo ) fino a dieci’,'¢:nell' undecima ¢ figuraro un Bante,
Cavallo , nelia 13. una Regina , € nella 14. un Re, ¢ pucte quelte carte
fuor che i Re fi dicon cartacce, Le 40. fidicono Germmis 0 Tarocabiy ©
yoce Tarocchi vuole il Monofino che venga dal Greco: Etaroohi ; quah
egli con ' Alciato , demotunrur fodales s1li'y gus cibi caufarad dufum-com
quella voce non fo »che fia ; fo bene , che Afereroi, ¢ Hetaroiwuol dire
da quefta voce diminuita all’ ufanza: latina fi pud-etiere fatto 4
Compagnont, Germini forfe da Gemini fegno celette,che ¢-fra Ta
é il maggiore . In quefte carte di Tarocchi (onw efigiati diverti
Segni celetti, ¢ ciafcuna ha il {uo numero da uno fino a 35,,¢0"
no a 40. non hanno numero , ma. fi diftingue dalla figuea am
maggioranza , che é in quefto ordine Strela , Luna , Sole , Afonds eT)
éla idre , ¢ farebbe il numero 4o. L’ allegoria ¢ ,iche ficcome Je)
vinte di juce dalla Luna ,¢ la Luna dal Sole, cosi il Mondo ¢ maggi
ela Fama figurata colle Trombe , vale piu che ii Moado ;

ale
a

  
  

   
    
  
  
   

  
  
  

REF FSSRSETFESTLRR SLE EESS EE

 

 

SRE

“PRFPEREVF eRe SBBEZE
 
 

OTTAVO CANTARE; 409

‘I huomo n’ é ulcito , vive in effo per fama , quando ha fatte azioni glo-
. li Petrarca fimilmente ne Trionfi te come un giuoco , perche Amore € fu-
daila Caftita , la Caftita dalla Morte , la Morte dalla aw > ¢ la Famas
| Diviniva , la quale eternamente regna. Non ¢ numerata ne anche la carta
ma vi é imprefia la figura d’ ua AZarro , ¢ quefta fi confa con ogni carta ,e»
ogai auacco , ed ¢ fuperata , da ogni carta, ma non muor mal, cioe non
i umat nel monte dell’ avverfario , il quale riceve ia cambio dei detto Maco
altra cartaccia da quello,che detce il aa ¢, ( alla fine del giusco queltv,
‘dette il Matto, non ha mai prefo carte all’ avverfario , conuiene che gii dia
(0, On havendo altra carta da dare im fua vece , ¢ quefto é il cafo , nel
fi perde ii matto ; Di tali Tarocchi aleri fi chiamano mobili perche conjano
chi gli ha tn mano vince quei punti , che etli vagliono ) altri ignobili , per=
snon'contand. Nobili ono Ve, due , tre , quattro ,e cingue , che la Cartas

%
del Fao conta cingue ,¢ I’ altre quattro contano tre per ciafcuaa. Ui numero 10.
:

   
     
  
  
    
    
  
  
  

43.20. € 28. tino ai 35. inclufive contano cinque per ciafcuaa , ¢ I’ ultime cingue
| Guutanio dieci per ciafcuna,e fi chiamano sd, [i Afatro conta cinque , ¢d ogni
Re conta cingute, ¢ ono aacor’ eifi fra ie carte nobili, {1 numero 29 non contas
fe niin quando ¢ in verzicola , che allora conta cingue , ed una voita meno delle»
‘compagae felpetcivamente ; Delic dette carte nobyli fi formano le Verzicole , che
Ordini , ¢ (egucnze almeno di fre carte uguuli , come tre Re, o quattro Re;
: di tre carte andanti , come xe , due , © tre , quattro , ¢ cingue , 0 compote, co-
i bet ‘hie wo y 13.¢28. Vo , matto , ¢ quaranta , che fono le Trombe, Dieci 20. € 30.
tq OVEFO 20 30.¢ go. E queite verzicole vanno moftrate prima,che fi cominci il
ginoco ,¢ meife ia tavola , il che fi dice acenfare la Verzicola , Con tutte le verzi-
i ‘evi fi confa il matto , ¢ conta doppiamente , o triplicatamente come fanno I'al-
a “tre , che fono in verzicola,la quale efilte fenza matto , € non fa mai yerzicola fe
i ‘non nell’ une , matto , ¢ trombe , Di queite carte di verzicola fi conta il numero
“che vagliono, tre volte , quando pero I’ avverfario non ve la guafti ammazando-
a “Vene wna Carta , 0 pir , con carte fuperiori , che in quelto cafo quelie, che refta-
f ‘DO; coMtano due voite , fe perd non reftano in (eguenza di tre , per efempio: Io
a. “‘moltrO 4 principio del giuoco 32. 33. 34.¢ 35. fe mi mi muore il 33.0 il 34. cho
os rompone 1a feguenza di cre,la verzicola ¢ guaflata, ¢ quelle,che vi reftano con-
tano foiametice due volte per una , ma fe mi muore il 32. 0 il 35. vi refta la fe.
Suchza di tre ,€ per confeguenza ¢ verzicola , ¢ contano il lor valore tre volte»
ciafeheduna . Mf Atato, come s’é detto , non fa feguenza , ma couta fempre
é 1i (uO valore due voice , 0 tre fecondo , che conta la verzicola 9 guafta , o falua-
a ta} © quando s’ ha pill d’ una verzicola , con tutte va i Adarto, ma una fol volta
i) conta tre ed il refto conta due ; € quefto s’ intende delle verzicole accufate ,/es
mottrace , prima,che fi conunci 1! giuoco , perché quelle fatte gon Ie carte am-
" mazzate agh avverfarj , come farebbe ; fe hayendo io il 32. ed il 33. ammazzaii
ai’ avverfario ti) 31, 0 ii 34. ho fatca la yerzicola, ¢ queita conta duc volte,
Quando ¢ ammazzata alcuna delle carte nobili , ciafcuno avverfario fegaa a co-
‘lav, a cui € thata morca canti fegni , 0 punt: , quanti ae yaleva quella tal carca ;
Ccvetto perd di quelie , che fono ftate moftracte in verzicola , delle quali , {endo
Auiazeate non fi gaa cola alcuna ( . a da quello , che per privil:gio non
3 f giuo-

 

 
     
    
    
    
 
  
   
   
  
    
   
   
 
   
  
   
    
   
    
  

4r0 MALMAN TILES

» giuoca ) perché tali fegni vengono dagli avyerfarj guadagaati
del valore di efla verzicola , che douria contar tre volte ,
ed il 29, morendo la verzicola , dove effo eatrava , conta folo cing
carte poi , le quali fi dicono carte ignobili , ¢ cartacce non contand
mazzano tal volta le aabili, che coa;ano comei tarocchi dal aunero 6,
amwmazzaa tutti i piccin , cioe I’ 1.2. 3.4. ¢ 5. dal 14. in fu am nazzano
i] tredici, ¢ dal 21. in fa ammazzano anche il 20, ed ogai tarocea
Re ) ma feruvno per rigirare i giuoco ; il qual giuoco appreilo di noi non
non in quattro perfone al pid , ed allora fi danao 21, Carta per ci
do fi giuoca in due, o ia tre , (¢ ne danno 25. E giocandofi in quai
primo che feguita dopo quello,che ha mefcolace le carte in fala mano
dice bauer (4 mano] ha Ja faculta di non giuocare , ¢ paga fegai trenta
che nel giuoco pigita ’ uitima carta , ¢ quefto che piglia P ultima |
dice far f' ultima ) guadagna a ciafcuno ai gueili , che hanno giuocato
Colui,che non giuoca guadagna ancor’ egli de i morti , cioe fegna,
lore della carta a colut,al quale ¢ amimazzata deta carta . Se
giuoca , il fecondo ha la faculta di non giuocare pagando go. fegni,{e
ca il 3. ha derta faculea pagando go. fegal , fe il 3. giuoca paifa la faci
che paga 60, fegni come fopra. Ma fe il giuoco ¢ folamente tn tre
ci quefta faculta di non giodare .

Me/colate che fono le carte, quello de i giocatori , che é a mano
guello,che ha mefcolato,n’ alza una parte,e fe v’é volta nel fondo di guel
del mazzo , che gli refta in mano una delle carte aobili , o ua tarocco dal
27. inclufive , 12 piglia , ¢ feguita a pigiiarle fino a che aoa vi trova und
ignobile: Quello,che ha me(colate le carte dopo haverne date a ciale

PSPS PPTs

fe fteflo dodici la prima girata , ¢ tredici la feconda , ¢ {coperta a %
carta la feuopre anche a fe medefimo , ¢ poi guarda quella,che fegue ,¢! a
fe fara carta nobile , o tarocco dal 21,al 27. ¢ feguica a pigliarne come +

gu-fto fi dice rubare , ¢ quefte carte , che fi rubaao,¢ fi {cuoprono , {endo a
guadagnano a colui,a chi fi {Coprono ,o che le ruba , tanti fegni ,
gliono ; ¢ coloro,che le rubano € neceffario , che {cartino ; ciae fi levino:
altrettante carce a loro elezione , quaate ne hanao rubate per ridurre le
al numero adeguato a quello de i compagai ; ¢ chi non {carta , o per
dente di carte mal contate , fi trova da uitimo con pil carte , o con.
avverfarj per pena del {uo errore non conta i puati , che vagliono le fu
ma fe ne va a monte; Colui, che da le carte , fe ne da pid, o meno d
flabilito , paga 20. puati a ciafcuno degli avverfarj , e chi fe ne trova
pill , ¢ deve fcartare quelle , che ha di pid ; ma non pud far vacanza
deve rimanere di quel feme,, che egli fcarta; Se ne ha meno , la deve
monte a fua elezione , ma fenza vederla per di dentro , cioe chieder la qu
o la fefta , ec. di quelle , che fono nel monte , € quello,che mefcoid le
fi dice far /e carte ) fattele alzare gli da quella , che ha chiefto ,
Cominciafi il giuoco dal moftrar le verzicole , che uno ha in mano
mo dopo quello,che ha mefcolate le carte in fu la mano deftra ,
quna carta , ( il che fi dice dare ) quegli altri , che feguono devon dare

 
4

 

OTTAVO CANTARE. 4it

mo feme’, fe ne hanno ; € non ne havendo devono dar tarocco , e quello fi dice
nes » E dando del medefimo feme fi dice ri/pondere . Chi non rifponde ,
ed | reece, feme , che é ftato meffo in tavola , paga un feflanta punti
ciafcuno , ¢ lia carta nobile , che haveffe ammazzato ; per efempio il

Toes ai di danari , ed il econdo benché habbia denari in mano, da.un

 

acco fopra il Re , ¢ l ammazza ; {coperto di haver in mano denari , rende
acolui dichiera , ¢ paga agli avverfarj feflanta punti per ciacuno , come
MY gE deo. Ogni tarccco piglia tutti i emi, ¢ fra lor taroccht il maggior numero

 

 

__-pigiia il minore , ed i matto non piglia mai’, ¢ non é prefo , fe non nel Calo dec
| todi fopra . Cosi fi feguita dando le carte yeu il primo a dare ¢ quello che piglia
; Tecarte date ; ed ognuno fi fludia di pigliare all’ avverfario le carte , che conta-
—-BO€ quando s'¢ finito di dare tutte le carte ,che s’ hanno in mano ciafcuno con-
b “tale carte , che ha prefe’, ed havendone di pil delle fue 25. fegna a chi I’ ha me-
ge MO taati punti , quante fono le carte che ha di pid 5 dipoi conta i fuoi onort , cod
P il valore delle carte nobili , e verzicole , che fi trova in effe fue carte , ¢ fegoas
all’ avverfario tanti punti , quanti con li fuoi onori conta pitt di efl> 5 ed ogni
F et fi mecte da bai.da un fegao , il quale fi chiama wn fefanca , € que'ti
Sfane valutavano {econdo il concordato . E tanto mi pare che bafti per facili-
tare |’ intelligenza delle prefenti ottave a chi nonfulle pratico del giwoco delles
Minchiate , che ufiamo noi Tofcani , che é affai differente da quello 5 che con le
‘medefime carte ufano quelli dala Liguria , che lo dicono Gsmellini ; perch Adin~
¢hiate in quei paefi é parola ofcena . Da quefto'giuoco vengono molte manicres
dire; come E/sere il matro fra tarecchi ; entrare in tutte le verzicole ; Efsere le»
Trombe 5 carracce ; Contare ; nom contare ; ¢ fimili .
\ INGRVGNATO . In collera . Chi s’ adira , 0 entra in collera fuol moftrarlo
con la Mutazione di volto , torcendo la bocca , 0 increfpando la fronte , com,
atti fimili , che fi dice anche far mufo , ¢ far grugno , 0 ingrugnare . Vedi fopra C,
2, Man. 57, Lafca Nov. 10 Ata Beco non la porendo [goxzare fene fRaua ingrugnato
_ anki che no, Viceli anche portare tener broncto ; imbroncrare ,. Nonio Marcelio an~
tico Gramatico . Bronci /unt producto ore , & dentibus prominentibus .
DS AMMAZZ AT A una Verzicola, Ammarzare ,rubare ,foartare , dar mal les
artenon contare , verzicola , non rifpondere y fe/santi , ec. lepgi queiche habbiamo
~detto qui fopra alla voce Afinchiare . read
» HVOMO roto. Huomo collerico . Lat. praceps in ira , che fi dice ancora in

yb + mn ena huomo precipitolo .

ES

BE

RUEBRERLER SSeS.

,¢ © NON ci puo fear foto, Non la pud foffrire . Lat. /uftinere , pati.
gi | LOR nem da retra’, Non bada , 0 non attende a que! che eff dicono . Non da
. Lat. mon faciem accomodat aurem. Dar vetta in altro fenlo diflero gli
“f antichi nelle cofe di guerra , per quello che i Latini diflero , imperum /uftinere ,
yy » GAGNOLARE . Rammaricarfi. Vedi fopra C. 4. ftan. 9.
n.% STANZA LXIIL
| Che t* bo io fatto mai fortuna ria Lucho non fi farebbe anch' in Turchia,
/ Chee? bas con me si grand’ inmicizia; Lie proprio un'impierade un'inginftizia;
, Mentre tu mi fai perder tuttauia Vedi , non lo negar che tu? kai meco ;
5 Che @ nem mi tocca pureadir:Galizia? E poi fen’. aunedrebbe Nanni cieco ,
2 Es £2 STAN.

 

ee Low

 
© SSR SS aes eee

a

 

 
  
 
 
 
 
 
 
  
 
 
 
 
   
   
   
   
 
   
   
  
   
  
        

412 MALMANTILE

STANZA LXIV,
eMa , fe volubil fei quanto /degnofa
Facciam la pace , manda via lo fdegno;
E [e tu fei de’ miferi pietofa , a
Danne,col farmi vincer, qualche fegno, Si 3} 5 ma bafta po
2» Fu il vincer fempre mai lodeual cofa y O Baccellacero
a» Vincafi per fortuna 50 per ingerno y

ee FFE CEES SP rsa ererersEas

Percio de’ danni miei refhando faria, Capitale\ Sarthe
La Fortuna mi fia non la Difgraria , Se tu nd voi pil

STANZA
E cosh finiran tanti [chiamazri
Dichiamar la Fortuna,ei ginochi inginffi, — Ov'io ritrowo ognor exttit
Che mentre vi ti ficchie vit ammarzi Per forza al ginoco mi ric e
T4 [pendi, e paghi il Boiacherifrupi, Appunto , come il ferre-a cal
1i Generale fi duole della Fortuna perché gli ¢ contraria , ¢ lo fa
pre: la prega a volerfi mutare , ed effergli una volta favorevole: © 0
fto C. 15. ftan. 1. dice Px i vincere , ec, Ma poi accorgendoli , che il fuo
é inutile , riprende fe medefimo , del vizio ,che ha di giocare , ma conok
P ammonizioni non (ono abili a farlo defiftere dal giuocare ..
WON mi tocca a dir ; Galizia, Non ho punto il conte mio. I
de della Galea diffe :
E fe non ne facean tanto romore
Non {aria lor toccato a dir : Galizia 3
Tanta gente » andaua per amore. ae
Ed il Perfiani dolendofi , che un fuo fratello cra pid lefto , epi aflute
diffe ;
E prima: 1 mio fratello é una ciuftizia y
Che mi riuede moito bene il pelo ,
1 credeu' eljer furbo , e giuro al Ciele
Che feco non mi tocca a dir ; Galizia, ket
Da quiefto che dice il Perfiani pud,chi legge,comprendere il vero
fto detto . 3
NON fi farebb! ancl’ in Turchia , Non fi farebbe in luogo veruno 5
foaa del mondo , (¢ ben fulle il maggior noftro nimico , come ¢id Treo
fopra C. 5. ftan. 6. i, Suagtod Cane
SEN’ avvedrebbe Nanni cieco, Lo conofcerebbe uno,che non havelle
Lo vedrebbe un Cieco ,come era Nanni. li Proverbio dice: éome:
cieco , ¢ (enz’ altra aggiunta s’ intende , vedere 5 perché quefto Nanni
va fempre ; vedere , Si dice anche femplicemente Vamnicreca., © 8!
defimo . Si dice anche : Le vedrebbe Cimabues vibe ncn ciecos 05\che
>

eed

vt

occhi di panno , detto h » venendo da
tura in Firenze , non perché eghi fuffe cicco » ma & voieva denota
fufle nato al mondo cieco y vive affatto al buio del difegno . 1
THM.

444A che gracchia io? Ma che fto io a ciarlare in vano. @

 
 

OTTAVO CANTARE, 43

re della Cornacchia yo del graccio , quafi Lat. graceutare, Ma ci ferue per clpri-
un cicalare fenza mento 5 fenza frutto , oal vento, Vedi fopraC. 1.
ftaa, 69. C. 4. flan. 25..¢C, 7. Rao. 59. Ser Brunetto Latini nel Patathio ; in quel

-weelo: Adi aific 5 #10 non fo.y ch’ aurem cornacclie ? volle dire in gergo ; alludendo
eal faono della cornacchia ; Che auremo per il ores di domani, Lat.cras,

a

_ DISDETT 4. Dilgrazia . Maia fortuna . B' ii contrario di Desta, che vuol
_ dir buona forcuaa nei Eee , Oinaltro. Sp. defdicha L, malum fatum,mala fors .
_ FINCER Ja pofta. Guadagnare quello,che va in giuoco. Vedi forto in quefto

wp ©. flan, 7..¢ vuol dire vincere una volta foia .
— PORRE 4 Caualiere , Rimaner fuperiore, Caxaliere fi chiama quella Torretta,
4 nelic Fortezzeavanza fopra a tutte le muraglic della medcfima fortceza ;¢ di

Effere yo fiare a Canalere , vuol dire Effcr fuperiore , © avanzare il compa-

- gno, Varcit Stor, lib, 9. Zara quefta parte delle mura di qua d’ Arno non banendo

wile Me monti, ne colli fopraccap!, non puo dal di fopra, (come fi dice)a canalicre effere offefa.

rd _ BACCELLACC I/O, Scimunito 5 Sciocco ; Infeafato. Auguilo Imperadore»
all” diceva bacelus . €

‘isp -Lorfe fogna pere, Ognuno Gi figura di goder quel ch’ ei vorrebbe , ogauno fo-

: ch'er bramna. Virg. ed. 8. 4% qui amant ipfi fioi fomma fingunt. Vedi [0-
_ pra C, 2. fla, 7. B per qual caula ti dica / oro , ¢ non altri aaunali, Vedi C. 1.
ca 31. Teocrito ditie; Omnis canis panem fomniat , ec.
| ) @APIT ALE, Quetto termine oltr’a i fignincati, che dicemmo fopra C. 7.
Mian, 82, protterito nel modo , che ¢ nel prefence uogo , ha la forza del Latino
Fiinam © yuoi dire piaccia a Dio,che non fia per effere,¢ che non fegua, ins
contrario .

y SCHLAMAZZO . Romore , Strepito . Traslato dalle galline, il gridar delle
quali Gi dice ichiumazzare , Ll Vocaboiitta Bologaele dice , che 1 verbo schia~
-Mbazzare fignifica Kiciamare io darao , dal Verbo Greco Sciamocheo , che vale

} mare cum umbra, Ma ¢ yvanita ; perche {chiamazzo vien dal Latino exc/smatio,
V1 ficchi evi  ammazzs , ba queito cafo {on quaGi Sinonimi , € figaificano
_— immergerti, o applicacti cutto a una cola, A "4
bp PAGE boia che +i frajti , Spend per haver danno. Teognide diffe: Sibé

Oe sph vineula cudie .

bp LABRICCINO del Paonazei , Lntende carte da giocare , perché gia un tale de’

te Paonazzi fabbricava dewte carte.

APPYNTO come ti ferro 4 calamita, Per fimpatia , come fa la calamita al fer-

i ro; Beeausito detta da Franzeli simaat , cive Pica amante .

df SA ANGA LXVIL STANZA LXVILL

is E Sard. ver , ch’ so babbia a star feggetto Datemi dungue un marzo in {ula tefia
. | a una cofa , che mi da tormento? Vedere ; eccoms qui ch io non mi muaitay

   
 

St

a

Come tormento ? oibo \ s' 10 ci bo dilette.

» St ua intanco per lui vine fcontenta.

O per fido giuocaccto | ¢ maledetto
Cin ¢ ba trouate,e me, chetifrequente y

Ne voi farete cofa men che bonefta ,
Se dal.giocar , morendo, io mi rimond y
Soc! ogni di farebbe quefta fefta,

C! altro dilerto , che giocar mon proKo y

i Tu non cs bai colpa tu, 4 me il gaftiga Ed a ginocare omai fon tanto avenge
a — 2 poicht cou te m' intrig? he'd pentirms non giana £4 Se

 
  
   
  
 

414 MALMANTILE ©
STANZA LXIX.
LD ufare ogni fapere , ogni mi
Non vale a far mi cotro al gioco,
Imperocch’ io t ba fitto si nell’ ofa
C? amos mio mal qual afferato inferno,
E forfe giochero dentr’ alla fofia,
Che forfe? diciam pur :tengo per fermo;
E fe trouar le carte ini non pufio y
Fari, ( pur chee fi eiecbi} all aliofiso. + I quarti anro,oo'far
Seguita il Generale a lamentarfi , ¢ combattendo in lui la voglia
con la ragione , ¢ con la conucuieuza’, prega gli amici 5 che Pai
ché vede , che non c’é altro modo ; che egli fi rimanga di’giocare}
d’ efier certo d’ havere a giocare anche dopo morte, ¢ che alla fepoltura
dare con le carte da giocare nel feretro nelia manicra,, che efprime
va 70. *
b7z0" « Quefta voce ha diverfi fignificati, perché ce ne feruiamo
come nel prefente luogo : per dimoftrazione'di naufea , come oii 5
e quefia ? (orto C, 10, tt. 23. per riprenfione , © difapprovazione: Oibe.
cofa ,ed efprime il latino Kab , & espace, E gue) che i Greci diftero e4ib
ciamo anche: aibo , eibo , ¢ tbo, 4 Oe
SCONTENTO, Scontclato , difguftaro » La \ettera’, sy aggiunta
pio di nomi , verbi , ec, ha nel pariar nofire la forza , che apprefio
particelia i» privativa di Circa di che vedi il Varchi nell’ Hercola
de alla particella ex .
MAZZO , Quei martellone di legno , che adoprano i. Macellati
Ja tela a’ buoi , donde mazznola queiia , che a Roma adoprano per
i malfattori . Si dice anche magéio , nia quetto € propriamenteq
prano i bottai a cerchiar le botti , Dal Latino malleus . 18
FARE {chermo contro al gioco, Difenderfi , 0 ripofarti dat non gioeare.
dal verbo /chermire , che vuoi dire Eiercitarfi per imparare a difenderhi
il qual viene dal Germano be/chirmen , ficcome vuole il Voto. Dan,
O Grscopo dicea da Sant’ Andrea, La ie
Che t' é viovato ds me fare {chermo? Y
i Petr, Son, 17. Ch’ i non fon forte ad afpettar la luce
Di quefta donna , ¢ non fo fare {ebermo
Di luoghi tenebrofi ,e a’ hore tarde ? i rue
L’ HO fitto nell ofa. Ho wn defiderio di giocare internatifimo ;
giovane innamorato difie , Georg. lib. 3. Quid ivvenis magnum cui
ignem Durus amor? Bil Petrarca. : Dee ai
eee

 
 
 
   
   
   
    
  
  
      
   

    
     
   
    
    
  
   
   

E ricercami le midolle yet ofa, © ~ A
AMO il mio mal qual ajerato infermo . Come brama: il febbricitante di bees
che gli ¢ nocivo , cost bramo io di giocare , che mi ¢ dannofo , she?
e4ALIOSSO . Come habbiamo detto fopra C, 1. ft. 9. tutti li gi i
da i Latini fi dicono alea: da che io deduco , che quelta voce Aliso
Latino alea , & of, ¢ fignifichi , come in efictto fignifica ofo da gu

   
   

 
 
    
  
 
 
  

   
 
 
  
 
   
  

 

pe

Ste

:

s

it
r|

 

 

OTTAVOCANTARE. is

i P aftragalo.de i.Greci., Dicefi ancora Catrioffo ; quai. gasdy
uct otio bared et gambe didietro di tutti Pen o
on ¢ nell’ agnell

 

> Pagnello , bue, ec, che negli animali d’ ugna {ode , come il
ec, © ditate come il Lione , ec. non fi trova , eccetto , che nell’ Alicor-
o Pol. Virg, lib. 2, cap. 13.. ¢ Dianel Soutero de Aleatoribus lib, primo
+» Buleng. de lud, Veter. c, 58. ed ¢ un’ offetto di figura quadrilunga das
concavo,¢ dall’ altra conueflo: Nel mezzo del concavo apparifce un
co, , ed il conuetio., che é la parte oppofta al concavo , forma: in cia-
luefiancate duc piccoli buchi ; nelle teftate del fianco al concavo ,¢
flo | due fuperficie quafi piana , fe non-che in una fi vede un fegno come
»¢ hell’ altra un fegno come un 8., € quefle duc parti quando |’ Alioffo fi
in tavola fono le pid difficili a rimanere {coperte , perche ono di pit dificil
del concavo , ¢ del conveffo , ¢ I’ altre due fiancate non reftano mai (co,

    

  

f wee perché niuna per la fua rotondita pud pofare «<I noftri ragazai dell’infima
_ pitbe,nel giuocare con quett' offo s' adaitano a quei fegni, ferucndofene per nu-

_ Miero.con fare il concavo il numero 4nd , il conuctfo farina , cioe naila, per effler
qusito 11 pity facile arimanerefcoperto , la parte dove ¢ il fegno 8. vince otto
tiene Ja figura di quci numero ;>¢ da’ Greci quello numero, di otto negli
chiamato Srefichora ,{¢ la parte dove é il. fegno.S, vinca dodici ,) perché
haf ‘a quafi di libra , che fi divide in 12. parti; 0 {econdo , che conuengono ,
ado, o variando quefto giuoco , fecondo i patti :(B I’ ufano detti_ra-
dalla Pafqua di Refurrezione ( nel qual tempo s' ammazzano gli agaclli ,
mpe de’ quali fi trovano.quelti offi } fino a che vengono le pelche , ed al-
lato.’ Auuofo.,.¢ giuocano ai nocciolisne i modi detti fopra C, 3. ft. 37.
qual giuoco durauo.a giuocare , fino a che ftiacciati i noccioli vendono I’ ani-
me di ef aghi fpcaziali., che fara per tutto Octobre in circa  ¢ da quelto tempo
fino a Quaiefima givocano alla ruila , 0 alle buche com Ja palla di legno nel
Che fi difie opra C, 3... 57. ; ¢ per tuttaJa Quarefima giocano alla trot-
« E cosi diftribu{cono 1 loro trattenimenti per tutto |’ anno ., Ma tornando
all’ Aliofse ; appretio agli antichi Romani era ufato dagli huomini pit fenfati , ed
in diverfe maniere ; ¢ fra I altre il concavo.era chiamato Cane, 0 canicula forfe
da. fiella lucida , che fi yede nella bocca del Gane Celefte ;, ftella cattiva , ¢
malefica ; € colui , che tirando faceva apparire detto lato ,, pofava in tavola due
denari , o guello.che ¢rauo conuenuti fra loro i giocatorl , ed era cattivo , onde
Pecfio dif. Damnofa Canicula quantum Kaderet y la parte oppolta,a, detta era
ohiamata Venus ftella benigna, ¢ benelica; ¢ fignificava il num, fer Latino Serio,
da noi detto Size, nel giuoco dello Sbaraglino , quafi Seino da’ Greci chiamato
 Hexites ,¢ chi tirando {copriva quelta Venere guadagnava [ei , ¢ tutto ello,
che haveyano polazo in tavola coloro,, che havevano {coperto Cane , 0 Canico-
la, Giulio Poiluce lub. 9. dice 5 che da ipill , il Sei era chiamato C0 ,.¢ il Cane,
| Ovvero!’ Aflo ; Chio: ¢ che in quefto lor talo non havevano , ne il duc, ne il cia-
gue . Con queiio offo giocavano tanto i Greci, quanto 1 Laci in altre maniere,
~ © fino con fei, ¢ oti offi per volta , ma a me balla haver accennata la fuddetta_»
“per teflimonio , che anticamente ancora era in ufo quefto-giuoco; ¢ tralalcio di
harrare J’ altre manicre che fon molie , perché non fa a propofito noltro ee

 
 
 
 

FE

 
 
  
 
 
 
 
  
  
     
  
  
  
 
     
  
   

o0ti(‘(“‘ Cz AT

fe i] Lettore ne faffe curiofo legga Polid. Verg. lib. 2.
Aleatoribus lib, pr, cap. 29, Buieng, de lud, Vet. Gap ye
rum gen. lib, 3. Cap, 21. Ho decto , che quefto Aljotio ogg
zi, ed il noftro Autore ci addita quetla verita , faccndo r
prrché fi giochi , all’ altofso , Se trovar le carte ivi mon pofio; ¢ intend: V
fempre , ¢ f€ non troverd carte,giuocherd all’ a/io/so , quantungue
fagazzi , pur ch’ io foddisfaccia al viziofo genio , che ho di giocare .

VAN co libri, ec, A’ Dottori , quando portati alla fepoltura j

mettere nel feretro,o bara i librised a i Cavalieri la (pada al fiance f

dice , che fara fatto a lui , che per far conolcere , che meiitre ville era |

re, gli faranno una ghirlanda di quei fiort , che fono itmpreifi nelle

vefte fara ricamata di picche , e di cuori , ¢ forto la tefta git c

mattoni ; ed in quefta maniera hawia anch’ egii actorno tucei quattro’

fono impretfi nelle carte da giocare a primicra . + a ee
Far sn quarto @ germini , Giocare in quaccro alle minchiace y Vedi fop

aS etek

  

   
 
  
 
     
  
     
    
   
  
  
      
 
      
   
  
  
  

   
 

quetto C. ft. 61. 5 00? th Se Ta
STANZA LXXI. STANZA LXX th
Volea feguir , ma tutti della lanza Amoltance ch ¢ buow ai bei
Gii dieron fu la voce con il dire , E por da bene,acor chia a
Che il perdere ¢ comune,e fhar’ nfanza, Dt quefto {uo viocar , don't fi he
E perde una miferia ds tre lire , i p ow
Pero fi qusets pure,e habbsa fperanta Dicendo &° a impiccarie hon, y
C’ an giorno la difdetta ba da finire, L! bauer femp.icemienteunpo dm o
‘Pero che i tempi variabili fono , Ma quand snch ezti havefse ivan Ga
E dopo il triffo n’ ha a venire il buono, Del far la [pia non fe we fa pr by
STANZA LXXIl, STANZA Lax th
Intanto gli moftraron il Prigione Ed al prigion preterito imperferto” oy
Che fort’ il manto deit lpoerifia Rinolto con le carve im man P itty \ >
In carwa , dicendo , in divozione Gid fattofele porre a dirimpette &,
Faceva lo fcultore , ideft la fpia; i giocar a’ nna crazia la tay
Berd, perch’ im effetro egli é un euidone Ouver fi metra fuor in [i a |
L  impicchi s* ei vuoi far opera pi: Vn teftoncino , ¢ fia guerra finitay | fy
Serragli pur, dicean, la gola, ¢ poi y Cosi lo prega, lo fconginra,e inpatt |
S’ ci dice pik nulla, apponlo a noi, Bada pur fempre « mefcolar leo
Voleva il Generale contiouare il {ao lamento , ma 1 circoftanti lo &
tare confolandolo ,¢ moftranuogli , ch’ ei fi faceva (corgere a far tanto ke ¥
per una perdita disi pochi foidi : Intanto gh prelentarono Piaccianted
it, che lo facefle impiccare, perche eglt era Spia; Ma il Generale buonhio® )

jo fece liberare , dicendo , che un poco d’ indizic non era baftance a
care, ed oltre a queflo del fara fpia non fe ne fa ne meno procetllo 5
che fe s' haveflero a fare impiccare tutte le {pic ci farebbe facceada, |
medefimo Generale inuita Piaccianteo a giocar {eco di poco,¢ (olo per’
Nei che il Poeta efprime i! vizio internaco di giuocare,che era
ché nello fteflo tempo , che determina di non voler mai pit gi i
terfi a giocare fino con ua vil prigione , con /’ anticta y che muitea Ma

 
   
        
  
  

   

OTTAVO CANTARE. 417
\der fempre a mefcolar le carte ; come fanno coloro , che punti dal gino-
per haver perduto, vorrebbono pur trovare con chi giocare per ricattarfi .
‘LI dieron fu la voce, Lo fecero chetare. Latino .: Vocem alicui comprimere,.
CDE una miferia di tre lire, Perde poco. La voce miferia , che per altro
ifica infelicita , 0 avarizia , ufata in quelti termini ferue per avvilire ; ¢ pero
ime qui una fomma di niuna confiderazione.
i SOTTO a manto d@’ Ipocrifia , Sotto feula, (otto pretefto , {otto coperta di far

  
   
    
 

teh

: - BACEVA ta feultore , Cie faceva I alcoltatore , ¢ non lo ftatuario, ed inten.
de, Stava alla feolta , cio fava afcoltando 1 ditcorfi d’ altri per ridirgii ; ¢ cons
| termine equivoco viene a dir copertamente Far /a /pia , come dichiara il

medefimo
-G71DONE . Furfante , Huomo d’ infima plebe (enza riputazione , Vedi fopra

 
  

Gr, 63. ‘
AP. a noi, Mlins crimen affinge nobis , See’ fa pit la {pia , gaftiga noi.
4 ‘Tiathcuriamo » OP entriamo mallevadori , che e’ non fara pil la (pia, Elo
[ll fleflo', che mo danno , che vedremo [otto C, 11, ft. 49. cioe mio fia it danno , fe non
‘jail Segue tort , Come iv dico ,

_ HVO.MO di buona paffa. Huomo di buona natura, Latino Oleo tranquiltior .
i Plauets in Poenuio', dra hunc canem faciam pibi-oleo tranquilliorem, fard ftare zitto,

   

| 60m’ olio ,
yet = — DOP ci fr enafta, Dove egli pecca. Con che egli varia la fua buona natura .
4 ~_ DEL far la {pia non fe ne fa proce(so, Gaftigar uno fenza far proceffo vuol dir

iio fommariamente . Latino indica canfsa, 0 pid tolto , de plano , cioe
ein nea ita di giudizio , fenza federe a banco di ragione , © come fi dice an-
4 che volgarmente pro tribunals ; ma qui par che voglia due’, che le {pie noa folo

we non fi gaitigano , ma ne anche fe ne fa proceffo . %

yal. PRIGION preterito imperfetto . La voce preterito , che fuona paffato , qui vuok

wie dir, che il prigione era dictro al Generale; ¢ 1a voce smperfetro denota Vimperfe-
zione , ¢ vighiaccheria di Piaccianteo .

uli . “
_ FN teffoncino . Teflone ¢ una moneta , che vale tre paoli , ¢ da molti in occa-
il fione di ginoco fi dice Vm re/toncino , per intendere giochiamo folo un teftone,¢ fis

ai S774 fimta , coe non fi giuochi pit. ; ;

BADA a mefcolare ve carte, Con quefta azione di badare ( ciot continovare ) a
mecolar le carve inuitando colui a giocare efprime , come habbiamo detto, las
ye BFAD vVoglia , che il Generale ha di giocare ,
; “STANZA LidY, STANZA LXXVIL
i Queeli che compracerto nun gis cofka, Duraro a battagliar forfe tre hore ,
£ vede bauer!’ hauuta a buon mercato; Poi la levaron quafi che del pari;

4 Li inusto tiene , ¢ regge a ogns poita , Se non ch’ il General fu vincitore

‘i Ben ch'ei non habbia un bagattino atiato, Di certa po di fomma di danari ,

= E dice , al pit faremo una batofpa E perche gli domanda, ¢ fa. Sealpore, :

4 Kuddei mi vincaye vogiia effer pagato, Quei,che gli {pefe in cene,e in definari,

“p Li rapa fangue non fi pnd cavare Lon bauer ( dice) manco affegnamento
Ne far due cofe y perdere , ¢ pagare « Tal ct Amoftance refta al fallimento,

a tele : gre en

 

 
    
   
      
     
        

ge MALMANTILE =

Piaccianteo actetta I’ innito, e mefiiGi.a giocare il
@ alquanti denari ; ma perché Piancianteo non ne haveva
grit Cosi fa la Fortuna , quando perleguita un giuocatore
o|amente quando oon vi é modo d ciler pagato
L! HA baunta a buon mercate. Ha (campato un gran perict
non ha havuta quella pena , o gafligo., che egli conolceva c
TIENE L inuto, Accetta |’ inuito ,¢ s' accorda a giocare,..
REGGE a ogni pofta, Pofta ( trattaadofi di giuoco ) vuol dir
danaro ? che 1 giuocatori concordano , che corra volta per, eal
fi dice inuirare , e reggere 4 ogni pofta , 8’ intende tenere tutti gl’ inuil
BAGATTINO. La quarta parte del quattrino Fiorentino , con altro none
deito Picciolo, Latino We obolum quidem , Voce , € moncta Veneziana,
FAKE una batofta, Combattere , ¢ gueftionare con parole , ec. Latino,
cari , ed habbiamo aucora ii verbo barofare , per combattere prope
ria di Semifonte trattato quarto, Non havendo tanta.geate , che
Terra batoftare, E pili (otto. Hor dt qud , bor di la fi baoftafe., j
NON fi pus cavar di rapa fangue. Non i pud cavare una cofa di, wee
&. Latino. gum é pumice poftalare. Plauto. Nam tu aquam ¢ pumice nun pr

ftulas , qui ipfus fitiat . iva
LA levaron quafi del pari, Cis’ intende /a /crittura ; Non vi corle qa iene,

cio’ fi vinfe , ¢ fi perdé poco. mitidiy
FA fealpore ; Fa romore ; Contende alzando la voce . 5 Oe

NON hauer manco afegnamento. Non haver danari , ne modo da trovames.
Ela voce manco in quelti termini ha la forza del Latino , nec etiam, ome
quidem , che noi pure diciamo , ne pare , ne meno , ne ance. lo credo,

Ce corrotta da ne anco.

REST A al fallimento, Refla con quel credito da non tlguoter mai pt

fallito s’ intcnde colui , che non ha denasi , ne aflegnamenti ,

FINE DELL’ OITTAVO CANTARE,

  

ae

 

 
  
  

 

  
 

: ARGOMENTO
" Ginnti i rinfre{cbi , ¢ inusgorito il Campo

 

ie

Qe

a
Corre all’ affalto , ¢ fegue afpra baruffa ;

~ Malmantil quaft ¢ prefo, ond’ al ua feampo %,
f Chiama all’ accordo , e termina la rufa, [ae
i Chi tratta pik di guerra hor trova inceampo , >

a = Perché nell’ allegrezze ognun fi tnffa ,
5 Faffi in Corte il conuito , ¢ poi , dal vino
. Rifcaldati quei Principi , il feltine. Ol
«ERPS EARS Pb Pape Pape ce ere? 7
: Nie
as 48
STANZAI. STANZAILIL
ye Aguerra , ch’ in Latino ¢ detta bello Si che e' mi par ben tondo,ed un corriva,
Parbrutta ame in volgar per fei Befane, Chi pus fear bene in cafa allezro,e fano,
Non cr altro fe e comincia quel bordello E lafcia il proprio per ? appeliativo
Di quell’ artigtierie , che fom mal fane , Cercando miglior pan,che que! di grano,
| Eche enon v'é da metter’ in caffello ; Cen’ un' altra ancorch'io non arrixe 5
E Slenti poi per altro com’ un cane Ch’ ¢ quell'afsalir un con Parmi in mano,
Sere’ un quattrino,e pien di vitupero Che non fol non m' ha fatto viliania ,
( Ditelo vei , fe quefto ¢ un bel meftiero, Ma, che mai viddi in vifo in vita mia,
STANZAIL _ STANZA IV,
E pur la gente corre ,¢ vi s' accampa Florsit cerchi chi vuot bartagliae rifse ,
- Ognit per farfiua'huomo,e acquiftar gradi, E fi chiarifeaye prow: un po le chiare,
~ Quafi degli buomms cola fia la frampa Che s' io. credeffi farmi un altro Viifse
Mentr! il canarne I’ ofsa avvien aradi, L'armi,percioné m'hano ainzapognare,
LA gli buomin fi disfanno,e chi ne feapa Ognuno ha il [uo capriccio , come difse
#14 tirato diciatto con tre dadi , Quel Lanzoghe volea farft impiccare,
E pria ch’ ei ginnga a efser Caporale ‘Pero mi quiero, ma perc’ bora brama
CHangierd certo, piu d'un fraiodi fale, Atoftrarus il vero;attenti,e cominciamo,

,Per introduzione de! prefente Cantare , nel quale il Poeta vuol deferiver | af.
Ito dato. a Afalmantie , fi ferue della dimoftrazione , che la guerra (ia una brut.

ta cofa , e che pero habbiano poco giudizio coloro , che vt vauno ; perche fe be»
nei Latini la chi 0 Bello ( il che fecondo alcunt facevano per aatifrali, cig’

Gee 2 pec

 
   
 
  
  
 
     
   
   
    
     
    
  
    
  
   
 
   
 
  
 
     
 
 
     
 

420 MALMANTILE

per una figura di parlare contraria a » che s'intende, c
bofco , che € fenza luce; Parce le , che memine proctnt
guerra , che non ha in fe cofa aleuna di bello, egli nondimeno
tifima , e ripiena di pericoli , come farebbe a dire i colpi 3
abbondante di patimenti, ¢ ftenti come farebbe il non haver, che
non haver mai denari; onde un Poeta per ifpicgar la-bructezga di
Lelia horrida bella, Oltre a quefto @ contro alle ragioni della
gnar I armi a danno di chi mai ci fece ingiuria alcuna , difle un G
lum a beluis dicirur, perche & cofa darbeftie , Si maraviglia pero
vada volentieri ingannata dalla fperanza , che in quella fi face
¢ non s’ accorgono , che pit tofto vi Gi disfanno , ¢ quand’ anche g
ci vuol degli anni prima,che uno confeguifca i minori gradi della o
la guerra Vx fol ne premia , € un million ne ammiazra . Conchivde p
vo di giudizio colui , che potendo ftare a cafa fua con ogni commodo,
trigarfi con la guerra, ¢ che quanto a fe , quand’ anche fufle certo d
ventare i] maggior’ huomo del mondo, non fi lafcera mai lufingare da
ranze: Ma perché egli fa , che ognuno pud far di fe a fuo modo, {ofp
{correr pil de i mali , che nafcono dalla guerra , ¢ s’ accinge a ;
con deferivere !’ affalto dato a Malmantile dall’ efercito di Baldone.
IN volgare , Cioe a parlare chiaro , fuor di gramatica . '
BRVTT A per jei befane, Befane come dicemmo {i C..8, ft. 30. vu
Panioccio fatto di cenci,e di qui per Befana intendiamo non folamente
na brutta , e mal fatta; Ma le Balie fi feruono della voce befana per i
una di quelle Larue, che nuocono a i bambini , come il Baw, er. ;'¢ gli p
no, che ci fia la Befana cattiva ,¢ la buona , ¢ che venga nellecale perk
del cammino del focolare , ¢ perd la notte avanti al giorno dell’ Bpifania,
Gio, Villani lib. 7, ¢ I noftro popolo anc’ oggi chiama Befania , onde ;
mente vien queflo nome di Befana , come s’& detto fopra , fanno che i
appicchino Je calze a i cammini , perché le dette Befane gliel’ empiano di
buona , 0 cattiva , fecondo , che effi fono ftati 6 buoni , o cattivi ze tali
buone , o eattive fi figurano fempre brutte ; onde bratro per fei befane vuol dit
eftremamente brutto. J Filofofi icolaftici per efprimer pid 1a, che i
dicono M2 «fo , dando alle qualita gradi fino in otto , e volgarmente per elprimt
lo Reflo fi dice Sei , come di fei corre , ec, fe bene ¢ un termine , che ha
furbelco, Cscala per fei putte, ¢ fimili. Ll Ferrari cavando Ja definizione
na dal Politi Aucor Sanefe la defcrive cosi: Larwale fimulacrum , ——
nia puerss terriculamentum Sufpenditur ; unde nomen inuenit, B foggiunfe w
mulieres deformes Befane dicuntur larua illa turpiores . Dice finalmente, che i Frar
cel dicono T#phanie dal Greco Tbeophama , cioe Apparizione.d' iddio.
nocte danno ad intendere le fuperftiziot(e,¢ ignoranti femmine a’ femplici
li, che feguano molte cofe fuor dell’ ordine della natura., miracoloic,
per effer Ja vigilia della fefta de’ Magi , né {anno , che con quefto nt ¢
Perfiani , ond’ ebbe origine , eran chiamati i Savi, ¢ intendenti
Natura , delle Stelle , e de! Ciclo. ia 3 NM

    

  

at

EPS &

eee FF FaTRRSEs

oe: esx EB >is

 

WEL bordello, La voce bordello , che propriamente vuol ie i igo

 
 

 
    
  

NONO CANTARE. qat

blico dove abitano:le meretrici., ¢ prefa da noi in pit fenfi , come per frepito 5 0

per una cofa flucchevole:,¢ noiofa , come é prefa nel prefente Juogo, ¢ altri la_

iglian inteoder Difficulta , o fatica »comela prefe il Lalli nella {ua Ea.tr.

le paroled: Verg: Hoc opus bic labor , :

enn Ene aio bello 5

8 et Cafacalda fi va prefto prefto ;

}) gameboeene| 2-1 | Ma vitornar in fu , quefioe il bordello ,

aa "0 é da mettere in Castelo . Specie di pariar Janadattico , del quale par-

2 a. fopra C, 1. ft, 29. alla voce /eminato , es’ intende Non v’ é da mettere in
~ ,

   

» che fignifica poi ; non v' ¢ reba da mertere sm corpo, cioe non y'é da man-

'. In furbefco ; Non v’é da smorfire ; Non-v' ¢ da empiere il fuflo, che cost

, dicefiil corpo nello fteflo modo, che in Greco volgare fi dice Cormi da literale
a Corner ,che vuol dire Fuffo,o Ceppo, Latino /ipes, candex .

| ~ STENT A come uncane, Patilci , ed hai careftia delle cofe neceflarie.al vivere.

: eo della Caccia lib. 5. Ergo age duro dffuefcant vittu catuli, Si dice frentar

bracco , quando uno per la fua poverta ha male il modo di provvederfi il

we

ie mee

-PIENO ai vitupero . Pieno di pidocchi , rogna , ed altre tattere , ¢ porcheries
4 i indivitbil della foldzvefea yi chet dice anche : Pieno “4s Bobbio , dal

_— Latinovepprobrinm , ebbrobrio ) ¢ Peno di faftidio ; del refto Vitupere fignifica infamia

bye vergegna, Bocc, Nov. 63.
in ET . . Abi vitupero del guasto mondo
ptt T] medefimo Boccaccio nella Teleide lib. 1.

BOs Abi vitupero delia gente Achiva,

ee Omero ,¢ Epimenide citato da‘, Paolo diticro in quefto fenfo mala probra , cio’
id vituperofi .
o Per farff haomo . Per diventar un’ huomo valorofo : Che e/sere un huomo , 6
sit an’huome , {erue apprelso di noi per intender quello, che intendeva Diogene,
of ES diceva 1 Aominem quero , diccfi Efler un’ huomo Givven. f wis efe aliquis ,
ie ferittara Confortamini , & effere robuffi, Omero , Viri effore, & forte cor fumite,
af VCHivefcampa . Scampare vuol dire fuggire , {cappare , © liberarfi da un peri-
PI colo + € qui intende chi eicé vivo, o avanza alla guerra, Scampare 5 quali u/cire
J dal.campo ; dalla battaglia . !
| © MA twraterdiciotto con tre dadi . Ha havuto la maggior fortuna , che fi pofla,
haere y/perché i! cum, 18, ¢)i] maggiore , che fi potia fare con tre adi. 1.Gre~
J cl pure ond eae dicevano : Ter /ex iattare , come fi ricava da Giulio
| Pollucesnell? Onomattico . Sy aah its
Bi CAPORALE . Capo di fquadra , che fra gli Vfiziali ¢ il minor grado , che fi
j dia nella milizia , Caporati differ gli antichi per Principale,Latino Capitalis. Gio,
Villani 1. 28. parlando di Roma dice:

ee Fu caporale regno di fe medefima
— Biib. 12.89. eA tutte le caporali Citia a’ tralia .
La voce é formata dail’ antico plurale ‘Capora', come Campora , Borgora’, es

fimili . °° :
- MANGERA pri: duno fraio di fale , Significa-confumesa molto tempo , perch
x molto

 

 
—

  
    
  
   
 
    
   
   
 
    
   
   

4uz MALMANTILE™
molto tempo ci vuole aun’ huomo folo a confumare ‘uno f

chi , quando volevano fignificare-un ‘tempo lungo ; dicevano com
che sled da mangiare a d' un -moggio di sale , Cicerone de Ami
que illud eff , quod vulgo dicitur wultos modior ‘falis fimul edendos efse
nus expletiim fit, Queftamaniera proverbiale pure in. pro
ufata da Piurarco nel libro della multiplicita degli amici » Si pud
che inghiottira pil d’ un boccone amaro 5: ¢ di poco fuo:guflo .
con troppo fale fi dice amara ; ¢ pero mangiando molto faleman
amaro. ‘ ‘ +.
TONDO , e-corrivo .. Si poflon dir-finonimi ; ¢ il primo fignific
fo , ed inGpido , ed il fecondo 5 che:fi dice\anche: Corribo., huomo leg
cile a creder' ogni cola. Latino credu/xs:.. 1\Napolerani dicono ¢
minchionare , burlare., .¢ dar. pafto'a uno; fopra-C..6..f, 80, difle.«
tondi pik dell’ O di Giotto, che{uona loyftefla, Tonsa fimiimente: pre
}i vale balordo , dappoco., femplice , goffoxCunto degli cunti?
Bue. :
LASCIAR il proprio per  appellative . Maniera di dire tratta dalla
in cui fi danno nomi di due forte , alcuni chiamati propri , aleri appell
dire ; Lafciar il certo per It incerto, Bar come il Cand’ Biopo ci
che haveva in bocca,per pigliar quella,della quale yedeva,lo shattimento 4
qua , che gli pareva magguore, ¢ Jo ftefto fignificato ha ; Cercarmmighor
grano, Eliodo Poeta Greco: Folle ¢ colui , che lafcia andar le cofe facili
¢ con certa fpeme fegue le pin difficili , ¢ lomrane ..\4\ pene
10 non arrino , Cioe lo noa comprendo + lo non arrivo col mio giudizio a it
tendere . In lingua furbefca.. foo» ammafco s non redo  cive non piglio, nonae
zanno, non comprendo . Lat. non affeguor . iru

ESPs SGP SSE

ee ee rset

 

VILLANIA , [ngiuria ,Soprufo y mal termine { LG

S? io credeffi farmi un nuouo Viiffe ec, 8’ io credefGi di diventare il maggior hut
mo del mondo. Diciamo Va nuoxe Orlando . L Greci Alter Hercules , ‘gh

SI chiarifca col pronar le chiare , S' accerti di que(ta cofa con provare le feri
perché chiara intendiamo quell’ albume deli’ uova , i quale s’ adopra a medicit
Je ferite , vedi fopra C, 1. flan. 60. ed il Pocta feruendofi del verbo ehrarive che
vuol dire (caponire , o (gannare ,€ della voce chiare fa nafcere lo (cherad.
_ INZAMPOGNARE., Ingannar con infinghe . Lat, Verba dare :ed ¢ 10 hielo
che iatinocchiare detto fopra C, 7, flan, 14. Dalla natura del fuono , e della Me
fica.; incancatrice delle meoti degli nomin.. Fra tutti gli trumenti perd. que d
fiato , levano pili di (efto , ¢ pare , che percuotano I’ anima pid gagit ¢
onde furono , ad efclufione degli altri ,ulati nelle battaglie , nelle quali facevad
meitieri tor via da cuori |' appreafione del pericolo., ¢ infonderni la a
fperanza . Noi habbiamo un Proverbio. Far come i Pifferi di montagna (|
nator di piffero ftrumento di fiato contadine(co.) che andarono per piferare
rono pifferats. Volcano minchionare gli altri coi, darne , ¢. furonc.
col toccaine . Fare uno cornamu/a apprefio i\ Puici, ¢’] Burchiclioe
inzampognare verbo facto da fampogna {trumento di fiato rufticale , cost a
Symphonia , della qual yoce {erucndofi Daniello al cap. 3. neil’ Litoria

 
   
  
 

cn x aie

 
 

_ ciulli 5

 

   

= a

ae

Ss

S=etis &. BEx es a i EO

, NONO CANTARE.

43

¢ narrando che efi non attefero-punto il cenno., che per comando Regio
fi dava, @ adorare Ja Statua , col fuono di tromba , di cevera , di finfonia:, ¢ di

ees ae fuoni ; sg fi pud dire [ fiami lecito qui dt feruicmi. di quefta baga ma-
inzampognare,come git altri. Tromper in Vranz,

=» ¢ pur dal Latino carmina ,

we

be Dis LEAN ZAWV..
2 aurora ye come diligente
azza le ftelle in Cielo, ¢fapulito ,
eae ffi alla fineftra d! Oriente
Evora l orinal del fuo Marito,
. Ma perche il Carretton ricco, e lucente

. Acciocch'ei non la vegga/cociase/ciarta,
eee: amegerneved ei iirimpiatta.
: STANZA :
Quande il vutto easone ‘ (rinfrefco ‘
St che,chi hauea col mafticar dinieto ,
pe oma iecamente il corpo al defco y
E come fi /uot dir ) riebbe il pero ;
ae Hi General, che tutta notte al frefco
nda con? Afirolabio innanzi,e indreto,
Batrendo la Diana in ful lunario
Hanea fatto di Stelle un calendario ,

~ Edi noftro Autore dice =

 

. Gid muone il Sole ,ed ella U’ ha fentiza, .

as = forle a corno , 0 tromba de’ ciurmatori: E Charmer Ancantayes >

UGNFNO ba il fuo capriccio. Virg. Quifque fuas patimur manes, Ogauno ha je
: fantalic, Vo Lanzo , effendo riprelo, perché faceva cofe da efler impiccato,
ve Che folerce tire » lafciatte far a ie 5 percht ho ancor ie mie pelle capricce, BE chi
ba Lanzw, Vedi (opra C, 1. flan, 52. ¢C. 4. ftan, 36.

STANZA VII.
seaienaat era anch’ egli riuedere
Tutto quanto aggrez.rato al pappalecco,
Done per hauer meglio it {uo doxere
Fece in principioun bel murare afecca:
Quando fu pieno,al fin chiefe da bere ,
E poi ch’ egti hebbe in molle polto ilbecco:
Fighnoli , 3 4iffe, omai venutat I’ hora,
Ch' e' fi tratta d' hanerla acauar fuora.
STANZA VIIiL.

S’ a men{a ognun di voi tanto s' affolta,
Atangia per quattro,e bewerpoi per fecte 5
Che par proprio che fia giuntoa ricolta ,
Anrich'egls bablia afar le fuevendette,
Tat ch’ io penfai vedern' anc’ una volta
La tonaglia ingoiare ,¢ le faluette ,
Ed bebbi un tratto anche di me paura ,
Per una fpalla dauola ficura,

“I noftro Poeca de(crivendo la levata del Sole imita Daate nel Purg, C.2.dove,
defcrivendo anch’ egli il parcir dell” Aurora dice:

65 bid Sicche le bianche ,¢ le vermiglie.guance
La doue io era, dela bella Aurora ,
Per troppa etade dieninan rance .

Accio ch’ ei non la vegea feoncia ye [ciatta ,
Manda git impannata ,¢ fi rimpiatta.

Bd intendono Vaoy et Alero, che quei colore , 1: quale appariva nell’ Oriz-
lente per caula deil’ Aurora , era quai (parito ; ed in iu queit’ hora comparue la
munizione da bocca , edi foldati i rinfrefcarono . Dopo di che 1) Generale det-
‘We principio a far 1’ orazione per inanimire i Soldau j quaic Orazione militace fi
Soutiene nelle prefenti ftanze fettima , ¢ ottava , ¢ nelle quaccro fegueati .

ere de fielle in Cielo, ¢ fa pulico , . L’ Aurora coi ivy ipicndore , offufcas

quelio

 
 
  
 
 
   
  
     
   
 
  
   
   
  
  
  
   
    

  

4z4 MALMANTILE ~

quello delle Stelle, € cosi le leva dai Cielo ¢ lo fgombra ;
VOT AP orina{ del uo Mdarito, Cioe del vecchio Titone favol
Ja Aurora , Virg, Tithont crovenm tinquens Aurora cubile. D
cubina di Titon antico gid s* imbiancaua al balzo a’ Oriente Puor delle
dolce amico, Qui pero defcrive Aurora nei {uo primo app
Ja parola # imbiancana . Li noitro Poeta poi-, per ‘Vorsmale ¢ J
tende quella rugiada , la quale caica fopr’ alla terra cicca’l apparic
Ja qual’ hora P Alba , 0 Aurora fi perde 5 pero dice Adanda gin 0 impan
rimpiatta , ciot ferra le tinciire, es’ afconde J “
SCONCLA , ¢ feiatta . Si potion dir Sinonimi.. Se bene /oon cia
mente dire una Donna , che non fi fia ancora accomodata icapelli
quale accomodaiento di capelit dice Accunciatu ase feiatea vaold
{compotta , e che habbta gu abiti male adactati,y¢ agguitats
fconcio ¢ pil generica , che nome la voce /sarto 5 core:
tine . Znconcinnus , inbonestus , wdecens , incompofitus',
1M? ANNAT A, Cosi chiawiiamo queiteiat di legao {portellatt
tono alle fineilre per chiuderle con carta , tela 50 vetsr, che vi fi
fenderG dai treddo ,o dal Soe 5 & mandar git  émpannaca yuoldir fe
tclio di gueito telaio , ¢ chiuder la fineitra ; perché per lo pile deceit:
aggiu(tati in manicra , che per aprire , ¢ Chiudére s'\ alzano , ed. abbul
diciamo tar fu , ¢ manaar gin. 6
SJ rimpiatta , S? a(conde . Vedi {opra C, 7. ftam. 66.
HAVE A col mafwar dimcto, A chi cra vietato i mangiare t
havevano ) traslato da 1 Magittract di Firenze y Re’ quaii ti dice baxer
non poter confeguirgli , ¢ aver proibizione per quaiche tempo di et
jut, che v' habbra parenu , oche gi habbia efereman di corto , Oo) per  @
givni ttabilite dalle Jeggi. Dan. Purg, C. 14. one
Lav’ ¢ meffier ds conforto Diniero , asthe
Negli Statuti Fiorentini diceti barbaramecate Dewerum ou itl
LIET AMENT E , Vuol dire-Ailegramente da lito; $e bene i noltti Contile
pi dicono /eramenre in vece di prettamente ; ¢ forfe qui i Aurore Jo Cee
fio tenlo ; perché fi pud credere, che 1 foldatis’ accoftafiero & mangiare:
gramente , ¢ preftamente . Li Lat edacer donde ¢ venutu il Pofcano Allegri s*
1 Frangele Alaigre ( che pil mostra Ja iua origine ) vale pronto, H
E /efo per avventura puo eiler fatto da serus ae
AP POGGLARE il corpo al defco . Si dice anche di chi rifeuote danari 0 prove
fione da banco , 0 Juogo pubbico . Cie accoltarti alla menia per mangiare.
RIEBBE wf pero , Svritociiia + Ripreie forza sok pero quello tia, vedi
6. ttan, 107. Del riavere i) peto vedi wna curiofa noveilettain Giovannt:
te,derto Gioviano Poatano,ne! Diaiogo iniacolato earenio p
cipio. Del maic che:fa al vento caccaiuly , © del beue,che neit:
cice 5 fe ne legge un’epig) Greco di Nv » melita 3 1
dire Fiorita Kaccolta de’ medetunt bpigramun 51 quaic cradon ave
fuona cost. Peditus occadst muitos incinjus in aluo ; Lipiojts batoo,
Seriat y@ occidie rurfum fi peditus; ergo Regibus auguftis quis

 

  

Fz ER THELIST.

Be

eee 2 baw ere

  
  
  
  
   
   

 
 

cue NONOCANTARE ~- — 45

BATTENDO 14 Diana in ful lunario, Tremando dal freddo per effere thao
all’ aria a confiderar Je ftelle . Batrer la Diana , Vuol dir battere i! tamburo all’
pparir del giorno , quando fi vede la Stella mattutina, ovvero Stella Diana, cioé

del di. Ma per mecafora intendiamo battere i denti per il freddo , che di- ae

mo anche barter la bora, Vedi fopra C. 8. fan. 6, >, a

_ TVTTO aggrexzato , Intirizzato per il freddo ; Affiderato ; Agghiacciato ; ;

sghiadato ; morto di freddo. sggrinzato truovafi nell’ antico per {ecco , es
liato di carne , quali fogliono reftare i morti ( appellati percid da Greci /i-

res , ci0é privi d’ umidore , fecondo che vuole Pjutarco nel libro intitolaro ‘J
inal fia de’ due pitt profitrenole ; ! acqua , 0 pure i fuoco ,¢ quali fi veggono cffere

 

is mie ftructe , (munte, ¢ fecche. da Aggrinzaro forte ¢ nato Aggrizzato , ps
| PAPPALECCO, Antende al mangiamento in generale : che per altro Pappa-

 decco se - leccornia , ghiottornia ( Franzcle ; friandife ) come habbiamo veduto

1C.7. flan. 55.
i hes Os niece il fuo donere , ec, Moftra che il Generale , effendo affamato ,
yi aifolratle anch’ egli a mangiare , acciocché gli toccaffe la fua parte ; intenden-
j ' do che mangio aflai prima di bere Tee murare a fecco , yuol dir murare fenza
eaicina 0 alcro bicume , ma con i foii {afi , ¢ trateandofi di mangtare vuol cir
jot Mangiare fenza bere. Neil’ antico facevano la parte a mangiare , ¢ a cia~
feheduno toccava ja fua ; il Juffo poi levd quefta ufanza ; dice Plucarco nelle Que-
 ftioni Conviviali lib, 2. g. 10.
; MESSE il becco in molie, Vuol dit bere , pigliandofi la voce becco , che vuol dir
re il rofteo degli uccellr, per la bocea deli huomo , Queito detto merrer il becco ins
molle Gguinca auche parlare , aprir 1a bocca. Gli Spagauoli la faccia dell’ humo

dicot roffro da quella degli uccelli .
i ‘S' afolta’. S? atfacica con furia , ¢ con vehemenza .
im STA Gitmo 4 ricotra , Cioe ch’ e’ G fia nell’ abbondanza maggiore , come fi fup-
pone che e’ fi fia nel tempo,che fi faono le raccolte : Se forfe nua voletim» dire,
che coftoro mangiando facevano uno fparecchiare fimile a quello, che tanao co-
loro che fegano 1 grano , ec.

PAR cbt egli habbia a far le fue vendette. Quand’ altri mangia ,¢ beve aflai ,o
fa quaififia operazione fen’ iatermiiione , ripofo , o rifpiarmo , ci ferviano di
queito'detto , affomigliando quel tale a uno , che per vendicarfi portato dail’ ira
Opert veementemente .

PER una fpalla davola ficura.M’era entrato cosi gran timore,che non mangiaflero
anche me,che d’accordo havrei daca una delle mie {palle per confecuarim: 1 ceito,

STANZA IX. STANZA xX.
Redeamus ad rem ; Se ( come ho detto’) Che quafi fui per dar nelle girelle ,
Qua fufte al ber infer mie al magiar fani , Perché dopo ch’ i punti della Luna

  

 

Eco+ coltelli sn man, (Pandoui a petto y Hebbs deferitti , ¢ che extse'le Helle
| Runfeiste si brani (parapant , fic Haneuo rafsegnate ad una ad una
bli battaglia vederui ancora afpetto Trouo [marrite bauer le Gallinelle:
| Con la fpada cost menar le mant y Ma dopo é, ch’ io mi dauo alla fortund,
Ona ib aimico vino , ed abbartuto Che fra le elle fiffe , efral’ erranti ,
NNe fia , come franotre bo preveduto « Won vedenone anche i Mercatanti ,
VR Hhh <2 Ska

CRRREBALERE EMASE.

 

=

 
 

 

 

      
 
  
   
   
 
    
 
  
 
     
    
 
    
   
     
       
 
  
 
 
  
   
  
  
     
   
  

26 MALMANTILE

STANZA XI.

M€a diffi poi da me , che poco importa
Se quel branco di Polli non fi troua
Ani che quefto a noi rifparrio apporta y
Peroche magian molto,e non fann' nova;

E [e ne anche alcuna Stella ho {corta
De! Mercatanti , gui creder mi gioua,
Che e'fieno in fierayo vero al lor viaggio,
Per laViaLatrea a mercatar formaggio, Effi cerchin la roba , ¢ mo
Seguita il Generale Ja fua orazione militare , con la quale dopo hai
fuoi Soldati di bravi nella maniera,che fi vede, termina {uo
che fi vada ad affaitare il nimico , perché {pera y che fieno per h ;
tuna per le ragioni , che dice , con le quali da un poco di bur! ara
FVSTE al bere infermi , al mangiar fani, Bevelte , ¢ mang te aflai,
gi’ Intermi per lo pil vorrcbbono fempre bere , ed i fani mangiano
caflai. :
ST-ANDOV1 a petio co’ coltelli in mano. Par che voglia dire ,
fronte per far alle coltellate » ed intende , che flayano a menfa uno
altro co’ coltelli in mano per tagliar pane , ¢c,, ec.
SPAR AP ANT , Cosi diciamo per derifione a un bravazzone , ¢ qui ton
ne , perché quefti foldati mangiavano gran quantita di pane, 4 ‘
PIT per dar nelle gireile . Fui per dare la volta al cerucllo. Vedi fopraC.t.
GALLINELLE , Quelle {ette Stelle , che fi veggono fra il Tauro, ef
dette Pleradi ; in Lat. Vergilie, Il comento d' Arato Latino. Pleiades 4 plartits
te Graci vocant, Latini eo guod Vere exoriantur Vergilias dunt . Aicum dil
Pleiades fieno nominati, quafi Plefiades cio che fi Ranno accoflo,per.
ci le chiamaton anche B try , cioé Grappol d’ uva,¢ noi Galinelley p
piccole ,¢ in un mucchio. Lt Vberti nel Dittamondo .
Poi diffe: guarda nella frome a quelle y
Le qua’ da’ fani ‘Pliadi [on dette ,
E che i volear le chiaman Gallinelle , 4
AU! dauo alla fortuna, Mi tribolayo: Mi difperavo : Si dice an
alle freghe , al diauolo , alla verfiera , alle bertucce , a’ cani ¢ fimili ,
fortuna : tratto per avventura , da’ Marinari , quando difperati, ab
in braccio alla borra(ca ; 1a quale da’ nofiri Tofcani fortuna di mare 5¢
folutamente vien detta . Il Petrarca s' era dato in un certo, modo alla
quando,defcrivendo il fuo ftato infelice diceva . a wi
Fra si contrari venti in frale barca .
Ui trouo in alto mar fenza gouerno,.
E poi. Ch’ ia mede/mo non fo quel ch’ io
MERC AT ANT1. Le tre ftelle del cingold @
Tauro , cosi dette perché fono infeme , ¢ paion compagae,
ragione . Adercatante dicevano gli antichi quel che noi. oggi p
-reante . L’ arte de’ Mercatanti nella noftra Citta ancora al,
feruato I’ antico nome . , : % ‘

SREERGERE

2RERSES

ee ae

  
 
 
   
  
      
 
   
  

 
  

NONO CANTARE. 27

__ BRANCO 4i polli.Latende le Gallinelle dette di fopra.ll Ferrarialla voce Branca
dice in fondo : Branco eréam pro grege.Vin branco-di pecore.Vaa mano di pecore .
Mon n pro mulritudine , ec, Manus autem eff branca , ut alibi anumaduerfurm ,
REDER mi vious che fien per la Via Latcea, ec. Scherzando con queiti aoint di
clot Gallinelle , ¢ Mercatanti difcorce di effe , come fe quelle fudero gaiiine ,
che fon difatili,perché mangiano , ¢ non fanno uova ,¢ che quetti Mer-
i non eran nel Cicio, percné erano andati a provvederd di formaggio
Via Lactea y 1a quale egii fuppoae di latte , ¢ che pero vi fia il formaggio a
Mercato ; ¢ conchiade , che ancor quefti fono difutili , perch¢ fond intenti
ente a’ guadagni , ¢ aon fi curano di gloria di guerre; ¢ pero che ¢ bene,che
. quelti non Gi trovino ia Cielo , perché torna a ior favore , ¢ pero fi poilas
8 “ entrar’ in guerra con buono augurio. Ridicole confeguenze altrologiche , con le
‘quali mottra Ja poca ftima ,che egli fa dell’ Aftrologia come di cofa frivola,e vana,
_— Fra laren, 8 quel circolo bianco , che divide da una parte all’ altra I’ Oriz-
"-zonté , edi nose i vede 1m Ciclo la meta , il quale dicono tia formato di miaucil
fime fielle ; Da molti é cniamato /a va Romana, Dan. nel Parad, C. 14. la chia-
| m0 Galafia, dalla voce Greca, colla quale queito yalibul cercnio del Ciclo fi caia-
Ma Galaxsas , cive laccco,
| Come diftinta da minori in maggi
ee Lum biancheggia tras pols del mondo ,
a Galafiass , che fa dubbiar ben Sages,
SON boti; Son huoauni di geflo, ¢ di Aucco ; che s'intende huomini buo-
ot at ia yilolidi; Lat. frpites , caudices. Vedi fopra C. 4. tan. 17. ¢ {otto C.
ws Tt, fap, 41. Similicudine tratta da quell’ immagini , che appicca nelic Chicle chi
ge 8 botato. In ifpagayoio Sore ¢ (puatato , che ha il cagho morto, Lat, hebes,
age tt Oftde boro de ingenso vale huomo d’ ingegao poco vivace ; ouylo .
fe | DANNO te ferste con (a penna , Ciot terilcono sella borla , quando fcrivono
Te partite in debico a uao . EB verameute le partite in debita fono ferite , perche
GidiceL denars sono it fecondo fangue , i) quale con tali ferite fi cava d' addofio al
Proilimo, Cosi i dice volgarmcnte Tarare ana frecesa, calui, che chiede a uo’ al-
tro in danari,vedi topra ( 2.¢ insdguinarti chi comincia a toccar guattrini,
sh) Dl dar foro, Deve dare , coe divicae lor dzbitore , ¢ per P equivoco inten-
de deve Perquocergli ; ¢ da cio cava Ja coalegueuza , che noa fiea buont per las
Suerra, poiche fe cia piantaav una partica ( snteadi difpongono una parte , una
# quaama di Soidati Jogauno gh dee dare (taccadi perquocere tali Soldati ) es
j gueilt che da tutti ac coccane , boo fon buoui per la guerra, Psancare wna par-
Ma Cinferire , o defcrivere nel Giorudle , 0 uubro di uegozio uaa parte , 0 arcico-
lo, capo di (crittura , che da dcbuo , € credito a chi s' alpetia ; 1 che fi dices
anche decendere una parsita y decendere uno debore ye creanwe , toric dal Latino
recerfere, deicciverc y regiltrare .
STANZA XIIL

     
   

   
 
   
 
  
  
    
  

 

| Nun prima fabili P andare in GMErTA y Com un bratcod uccelliil quale in terra
Che vede/ts pie prefto ch’ 10 nol dico Sts calato a beccar grano , 0 paniva;
Vitleuaiena, «ur trattoyun ferra ferray Va che fi muons bafta , che quct folo
Ed ir correnas contr’ Ali! inimne. £4 fuoice pyuare a tats nw volo,
é : Hhh z STAN:

 

zat,

 

 
 

 
 
 
 
 
 
 
 
 
 
 
 
 
  
  
   
 
 
  
     
     
    
   
 
   
   
   
    
  
 
 
  
 
 

428 MALMANTILE™
STANZA XIV.
J coraggioft al primo , che fi moffe ,
Gli altri (gid fendo meglio [ui piccenali )
Non poterono star pit alle moffe ,
Ma corfero ancor lor come Terzuoli 5
Giunti di Malmantile in fu le fofe ,
Drizrate al muro afsai feale a pinuoli
i falirui renewano una baia
Com’ andar pe’ piccions in colombaia .
STANZA
Gh fiipits , le foglie, e gli architraui
A quclto efecto efsendo gid (murati
Per via di curri, dargani, ¢ ditrani
Gli hanevan {u le mura firafcinati , Faceano un venga addofsoat
Stabuito d’ entrare in guerra ,¢ dar U affalto a Malmantile i pid ¢
rono i primi a muoverdi , ¢ gli altri meno coraggiofi (eguicarono. &
Dante , che nei Purg. C, 2, dice:
Come quando cogliendo o biada , 0 loglio
J colombi adunati alla paffura
Quieti fenza moftrar U ufato orgoglio 5
Se cofa appar ond’ effi habbian paura
Subitamente lafciano fRar ? efca
Perché affaliti fon da maggior cura,
Arrivati dungue alle mura di Malmantile , credendofi di trovar fac
s' ingannarono , perché quei di fopra gagliardamente fi difendevano
altro. Qui ¢ da confiderare , che fe bene Capitelliye Srontefpizzs fon me
shitettura , il Pocta (cherzando con I equivoco.di capi, ¢ fronti, ¢ ferue
verbo Pampare nel fenfo , che lo pigliano i Legnaiuoli , ec, che dicen
C. 1, tt. 8., vuol die, che tali merii pictre , ed altro devano fopra 1 2
alic fronti dei {oldati , ¢ gli fampavane , cive gli faceyano di quei-
chiamano fampe , ed in fuftanza vuol dire , che rompevano tefle ,¢
fuono , che rendono i corpi battuti fecero i Greci il lor verbo typrein,
re; da queito verbo ne venne Typus voce pur Greca accettata da’|
una forma imprefia , o cavata fuori col battere : Se ne fece ancora 7}
tamburo , che Omero pil conforme all’ origine diffe Tympanon feguito
Catullo nel Poema Gailiambico . Noi abbiamo voci da riferire a quelte' \
come farebbe Stampa , Stampita , Stampare , Stampanare, Ma in pro
fiampe fatte ful moftaccic d’ un’ antico Giucatore di pugna, evvi un
gramma del Greco Lucilio , che in noftra lingua voltato dice Cosi;
2 un vagho , Appollofane , il tua capo ,
O qual fu mai pin traforato arnefe ,
Son tane di formiche 90r dritte , or torte,
E par , che con bizzarre , e varie nore
Vn Lirico eccellente il Lidio v' abbia
Inravolate fopra , ol Frigio canto.

esceftie?

jie te te en ee i a.
 
   

  

NONO CANTARE,
6 Or franco vibra il minacevol pugno
 Ecombarci pur liero in duro arringo 5
_ Che fe colpo novelio a te difcende , :
Quel ch’ ai rifcoffo , aurai, ma non gid nuond
et Capir nel capo tuo potra ferita ,
PIP prefo chrio nol dico,, Preftitiino confumaron manco tempo a far tal cofa,di
silo , che io confumo a dirlo. Latino dicto citsus .
“N lena leua , un ferra ferra , Quando vogliamo intendere, che una gran quan:
: di popolo adunata in qualche luogo fi fia partita in un fubito , ¢ velocemente
ia one di quefto.detto 5B figniticano quafi lo fteflo , fe aon che I’ ultimo ef-

    
   
      

» quando uno é da altri incaizato a correr , ec, vedi (opra C. 1, ft. 63. e»

ke
- f hail

pero nel p luogo fi potrebbe anche dere, che i primi volon-
 tarj , ed 1 fecondi forzati dalla riputazione . 11 Varchi Stor, lib. 2. dice : Pa /ubir
| Wegridato: armi armi , lena lena , ferra serra , ec, Dal che fi cava , che quefto detto
tog fignifichi Leva la roba di fopr’ alle;moftce delle botteghe , ¢ ferrale come (eguiva
at | Firenze nelle follevaziont di popolo , ¢ che ii medefimo detto fia poi facto co-

Mune a oga: forta di tumulto , ¢ per ¢fprimer un moto turiofo di quaatita di po-

4
Ll

| AR correnda . Andar correndo . Il verbo ire venendo dal Latino, vale appreffo
di aot quaato il verbo anaare , ma ci feruiamo folo deil’ Intinito ire , del partici-
Pi9 ito , © folo , o accompagnato col verbo efere , e dell’ Lmperfetto ina , ixano ,
che fi dice poi , giva, ¢ giwano, Nella vita di Cola di Rienzo (critta in lingua Ro-

Mana antica trovali jio , ¢ seffero , ¢ fimili , che i Tofcani cangiando |’ [ coafonan-
foi *ealpra nella doice Jettera G dicono gio, cioé andd, € gifero, cioe andaflero.
wi fimiimente prende alcuai tempi , come farebbe i prefenci di tutti i modi,
‘i dai verbo Vado, io vo; ancorche Dante viatle forefticramente , edadi per Vada ;
gg © 0i0 cofretto dalia rima .

» ST ANDO mestio in fui piccinoli, Effendo pi gagliardi nelle gambe ; ¢ quefto
gi AVVeniva , perche havevano mangiato . £ piccinols , che & il gambo delle fruttes.
g Latino pedicutus , ¢ pref comunemente in quefto cafo per le gambe dell’ huomo,
ia NON porersero Rar falc alie moffe . Non potettero contenerfi , che non corref-
a fero . Toho da j Wavalli Bacbari , i quali corrono a i palj , che eflendo tenuti per

Jo freno dai loro Stallon: al luogo donde a) fuono della tromba deeono partirfi,
7 che fi dice le moffe ( Latino carceres ) molte volte fcappano , prima che fia dato i)
‘ detto fegno,e quefto fi dice non far ferme aie moffe , che poi paflato in proverbio
! non haver pazzicnza , © lofferenza , ma per il gran defiderio d’ arriva-
i Tea Uo luogo, partirfi prima del dovere ; ed efprime quella inquietudine, che uno
, hanelitafpercar , che /egua una tal cofa da iui anfiofamente bramata. Del Ca-

vallo generofo Virg. Georg. 3.
Stare loco nefcit , micat auribus ,G tremit artus ,
Colettumgue premens volvit {ub naribus ignem ,

CORSERO come terzxoli , Corfero con la ftefla velocita,con ia quale vola alla.
preda il terzuoio (pecie di falcone. Perché cosi fia detto rende ta ragione il Tua-
No de re accipitraria lib. 1, edtrque ad co. cum tres foetu enitatur eodem Predones gene~
rofa parens mas kitinus imo defpectus letto incet appeliatur y & inde Tertius ,

SC_A-

 
 
    
   
   
    

4jo MALMANTYPLE &
SCALE 4 pivoli . Scale fabricate di due corredti «
glioni fono pivoli ficcati fra ’uho ¥'¢ I altrore C
fine in diftanza uguale a rifcontro , ovvero'i detti f
© ftecche , © regoli di legno conficcati in deeti correntt Mampati
rifcontro. B pinole, ( Latino clanicx/a , civxt cavicchio ; ovvero
de ogni pezzo di-baftone adattato a porerd mettere in un buco,
TENEV ANO una baia , Stimavand cof: facile ;*Stima
burla , ec. Latino mage, Ii Ferrari dice poter venire quefta voce da
iflar’ a bada , in ozio , Latino wataré, © O01 i
COLOA18 ALE, Quelle flanze fabricate per lo pity nelle form
per ufo de i colombi , € nelle quali‘wafcono i piceionit) «>
FEC ERO parergli altro fuono , Fecero lor conotcere , che |
ment. . voy
ewERLI, Qvei picco}i murelli'; in diftanza uguale'y ned quali per!
mioano te muraghie delle Citta , ¢feruond per: parapetti’. ad foldati,
per difefa delia muraglia ; cost dette quali. murnlesdice il Berrari; fume
primes parus murs,Dichiamo @una-cola;che ancora abbia delle dific
rarfi ,¢che non Gi fiano per anco {puntate: £ ci é de merio , cio’ non é elpy
to il cutto, Ci rea ancora qualche parte da abbattere 2 Vedi foro © 12)
ISSO fatro. Subito . Due voci Latine corrotte , ¢ ridotte Tofcane,
loro lo feflo fignincato .
DISEATTO (e reftuggini, Infrante le Teftuggini animali Terreftri,
che hanno Ja coccia , © gufcio durifimo da alcuni‘detti Tartaruche
he , da altri bexzache ( dal bezzicare ,.ch’ elle fanno rafpando in terra
atinl Tefudmes, E § potriaanche dire , che ? Autore intendetle di qu
razions da guerra , che ufavano gli antichi dette Te/udines; nelle gi ;
no foo alle mura , reggendofi fulie {palie gli uni gli altri , ¢ aiutandofia m
tarui {opra , coperti turu di feudi , € terran iteme per ripararfi da’ colpi , che
fi (cagliavano per di fopra; E quetta operazione s' addimandava refixggine spe
ché flavano col capo , € ‘colla vita dentro agli icudi , come flanno le
(in Lp. torragas in Beanz. ortaes ) dentro aile loro {codelle 3 le quali )
dette da’ quei dello ftato di Muang , come racconta il Ferrari bi/se fo
bijce (codeliaie , perche anno 1i capo di bilcia, ¢ ftanno rinchiufe cone i
delia ; Onde potrebbenfi dire, dom:porte, come un’ antico Poeta chiamé le chien
de . Autione famoso ceteratore ¢ fatto parlare da Pacuvio cost , delcrive
tetluggine con que’ verfi portati da Cicerone de divin, ub. 2. Q@madrapes 1am
da , agreftis , bumilis , afpera , Capite breui , cernice anguina , adjpettu trad
ruche ,¢ BR2uhe , {ovo voci ufate dai Caro ne’ Mauiaccint; ¢ i} Veneziaiol
chiama Gv/ane dal Gr. Chelonei , da noi fi dicono anche butte {eodellaic,
BAST le NO Seré, Celebre , ¢ nouttime {crittore d’ archucuura.
EbDIF/Z/0 , Preto largamente s' inteuce Ogni forta di faborica , €
ma prefo ttrettamente vuol dir faia , ec, Cafe, ed altre niuraghe , |
ades , @ facio; ed in queito andiamo uniti co’ Latini , che per earfien
no ogni forta di ferittura . Gio, Villani t, 128. Pauose/f ad afcdin, 00, ©
difici , ¢ per cane per forza ebbe, Li lib, del conquiito, Per joraa a

 

 
       
  
  
     
   
   
  
  
 
  
  
 
 
  
 
   
 

 
 
   
  

gE Es PSs SEES. =

 

‘> ie PS

= Fo

 

 
 
 

  

NONO CANTARE.

1. Capiteli , ¢ frontejpizi ,, Columnarnm capitula , © fronts befpitii, >
(ATT H Srglie.s ¢. aui, Stipi (ono le pietre de i tianchi y¢ foglie quel-
a parneey quelle dilopra , che tutte infieme formano una por-
a» Suipice dal Latino #:pes.. Architrave ; quafi trave principa-

: « Quei ruotoli di legno,che feruono per facilitare lo ftrafcico de i pei;
atini li ditiero Palange, Vedi fopra C. 2, ft. 65. Dichiamo: mertere une fal exr-
Spiguerlo a poco.a poco , ¢ condurlo doicemente a fare alcuna cola, La
Voce viene probabilmente dal Latino baiudare ; quefto aggiuttar’ un corpo
}a un’ altro in maniera , che quello Jo porti con ficurezza . E la feconda
| Latino xmbdicus , cioe punto ne) mezzo, Bilicare quali ponere in umbitica ,
ARGANO . Strumento , che feruc pct tirar fu pefi in alto , che da huomini é
" moflo in giro per via di leve. Alcuni Latini lo dicono Sucu/e , i Greci oniffi, cioe
 Afineli: , ¢ quelto & V argano,fecondo il Filandro , cum axe iacente, quello pui cum
axe ereite , dice che in Latino ¢ Ergeta , cioé macchina da lavoro; donde, 0 da
voce(lecondo i} Baido fopra Vitruvio)é fatta la noltra Argano,
MSADATT 1. Scommodi; Non atti a efier portati , o Arafcicati .
MC ATI, Meili in bilico-,-0. equilibrio., Latino Jibratis .. Diciamo.bilico
ofitura d’ un corpo fopra ad un’ altro in maniera , che pofando quafi in un
non penda , o aggravi pil da un lato, che dall’ alo. L noftri Scarpellini
_ dicono baggiclare per biluare. i
it. BOTT O porto, Si dice . Ch’ é cb’ € 5 colpo colpe sec. ¢ 8’ intende Spefiime volte

      
   
   
      
      
    
    

 
  
 
 
 

PAR* un venga, Tirar roba da alto a batio fopra auno , che fia foo.

 

 

 
 

“a ay STANZALXVI. STANZA XVIIL
a Le Donne anch? effe corron co’ figtinoli y Chi , perché gik non piglin l imbeccata
f i 2 dy che troxan, gettan dalle muray Cuopre i capi con tegoli y ¢ mattoni ,
o con la conca , 0 vafo da vinolt Chi verf{a git bollente la rannata,

a 9» Pighia a qualcun del capo la. mifura ; Che pela i vifie porta via i bordoni,
a8 Profuma il pifcio i panni, ei ferraixoli Nei? olto un'altra intigne la granata,
yet Ne guardan vc v'é penail far bruttura, E fal afperges fopra i morioni,
ps Chi tira gi: unjastrone alic cerned y Altre buttan le caffe,accio i foldaté
ie Che fe ewe orili ferua per murella, Partir fi debban , poiché fon cafjati ,

ie ooNarraiil Poera la difefa, che facevano queidi Malmantile , ¢ defcrive diverle
we" Opérazioni militari adeguate alla compofizione burie(ca di cutta. opera .
CONCA, Valo grande fatto di terra cotta, entro al quale fi fanno i bucati
Ke ASO da viyoli, Sono vatetti di terra cotta fimili alle conche , ma piccoli, en.
| 80a! quali G_pongono vivoli , cd altre pianterelle d’ erbe, 0 fiori. Dice che.con
v — gucfi pigliano la mifura a.ijcapi y perch¢ hanno il vacuo capace della tetta d? unt
Td huomo ; al quale quando i Cappellat voglion pigliare la mifura della tefta, metto-
u# ~—-'NO in capo un tappelio ; € ceftaco di Malmanzile per pigliar tal milura , in vece
sso un cappelio., mettevano-un valoda vivoli : ¢ cosi{cherzando intende y che ti-
@ — ravano (ule tefte a i foldati di Baldone i deni vali. ;
@ \SEvi dipenail far brurture', Se\vi ¢ pena il fare fporcizie; Dice che tirano fino
Dorina , ¢ non guardano ,-fe. cid fia proibito,: ¢ con quefo dire, accenna i} co-
ef flume, che ¢ in Firenze a” affiggere alle muraglic dove non fi vuole , che fien fat.
r te

 
    
 

432 MALMANTILE

te {porcizie , certe tavolette di pietra , nelle quali & feritto il
flrato degli Otto , che proibifce , ¢ mette la pena a’chi fa
niuno fi pofia pretendere ignoranza ; Ed intende anche di
¢ grave pena , che é in Firenze a buttare dalle fineftre nel
torno a’ quali difpone anche la ragion comune , come fi vede
De his , qui deiecerine , vel effuderint , ' ‘
SE v’ ¢grilli, Sopra nel C, 6, ft. 22. dicemmo , che grille fi cl
cola palla , che fi tira per fegno , giucando alle palloctole ; ed all
firelle , qual giuoco dicemmo come ti facia fopra ia detto C.6.t,
rché tirandofi , or qua , or la alla ventura , 0 alla volonta
a il falto del grillo , che dopo un breve falteilare fi ferma, e-poi
-dicefi ancora Lecco, quali i/ex eMurelle chiamanfi anco
nelle fue Rime. orate
Ch’ io do fempre nel lecco alle murelle OP R
dal Tofcano antico e#ora , che ¢ lo fteflo , che il Latino Moles }ép
fi dice di pictre. A’awer la refta piena di griili s’ intende uno, che ha capric
vaganti ; ¢d il Poeta {cherzando'con quefto equivoco di’ grille dice
quelle laftre a’ grilli , che fono neile tette di'coloro , come fe piocatietd
ftrelle , 0 murelle. Dal pazzo fimilmente,¢ curiofo faito del grillo fon detti
icapricci , ¢ fantafie firavaganu , che faltano in capo , ¢ per cosi dire
PIG LIAR’ un’ imbeccata , Infreddare : B diciamo ancora : Pighare df mitt
caffrone , perche il beccd , ed il caftrone hanno una tal raucedine, che
pre, che cofiano , appwato come fanno gl infreddati .
Té£GOL/, Pezzi di terra cotta adattati a coprire i tetti delle cafe.

ap

   
 

 

     
    
  
  
 
   
   
    
   
  
 
  
  
  
    

 

HlAe .
: RANNAT 4, Lifcia forte; che é quell acqua bollita'con cenere ; ¢
dalla conca , quando fi fanno i bucati. Lacino /ixininm ,
BORLON/, Inteudiamo quelle penne , che non de} cutto fpun
{corgono dentro alla pelie degli uccelli , ¢ per fimilitudine intendiamo il)
{punta nella faccia degli huomini « way
FAI alperges con la granata, Diciam far ? afperges quando con {pugha
tra cofa fi (pruzza acqua , © altro liquore ,.a minute ftilie ; 1a qual cola il
chiama e4/pergere , qui dice, che {pruzzavan' Olio con le pranate ;
aiciato un mazzo di {cope , © d’ altro fimile adattato per (pazzare ,)

ftanze .

SOLDAT! caffati . S' intendono quelli , che fono ftati pri
Ja milizia , perché cafare vuol dire cancedare: Ed il Poeta{
guivoco di <afaté , cioe percotli dalle cafie’, dice, che fe fon
nou dal Campo , perche non fon pid nel numero de” float,

SLANZA XIX,

 

Vi? altro con un gatto vwol la berta, Ed il primo ch' et trova
Legato il cala,ond’ es fra quei.d'Vgnano - Che dou'ticbiappar
Sguawnialugna , econ la bocca aperta

Griaa ina/prio in fue parlar Soriano s

  
   
       
   

  

oF

ee ee ee ee eT oe eS Ol ee

=~ aw

 

 
 

NONO CANTARE: 433

arnt) re Bie XX, e
Miagola, ¢ foffia it gatto, es’ arronciglia y
Ed Gite endian heerees
janes quel che oa " trattopigla
Beli é miracol poi fe pite gli feappa ;
thie oat peter tee cos riglia y
jie Lo tira fu con quaiche bella cappa ,
a «Ci qustcheciarpayo qualche pinacchiera,
ye  Ecosi gli rie{ce di far fiera,
ame cool (STANZA XXL
due Quand una volta lafciale calare
ib oi iaers al buffo di Grazian Molletto,
Che fu;di pofta per ifpiritare ,
«Quel pelliccion vedendo intorno al petto,
we Le beftia intanto falta , ¢ dal coliare
‘hoe "

=

 

Tutto prima gli firaccia un bel gigisetto,
fet  Dipos fi lanciaye al capo fe gli ferra,
ebst _ Si che il cappelio gli mando per terra,

STANZA XXIL
Non.sa Grarian,che Diauol fi fia quello:
Pur tanto fac’ al fine ei fe ne sbriga,
Ea aiza il vif per farne un maceilo ,
¢Ha vedendo il rigiro, e ch'ei s' intriga
Con dame , vuol canarfi di cappello ;
Ma perch’ il micio gis ha tolto La briga,
La Dama accsuetrata, anzi civetta
Lo burla, che gli é corfa la berretta,
STANZA XXIILL
Ed ei , che da colei punger fi fente y
Onde al nafo lo fironzolo gli fale,
Perde il rifpetto ,¢ quiui fi rifente
Con dirgli, Atona merda , ¢ ogni male,
Vain quefto al aria ungraromar digéete,
Che 4 terra feende a mafse dalle fcale
Fiaccate,erotte ach'elfe dagii /prazrolt
‘Di pierre, c' ancor grattana § cocuzzoli,

oa Continova il Poeta a narrare gli accidenti , che (eguono nell’ aflalto di Mal-
ie mantile , e dopo haver detcritto una Donna , la quale con un gatto legato a uns
" i miazzacavallo andava levando rcba da dofio a quetlo , ¢ a quello , come fegue a
ol Graziano Molletto ( che ¢ il Sig. Conte Lorenzo Magalotti ceicbre per aobilta ,
HF 'e dottrina ) dice che le {cale degli AGalitori furon rotte dagli Allediati : ¢ con i
r faffi , ¢ con altro , che tiraco di fopra alle mura, dava ancora addoffo a i foldati.
at IL (a berta, Vuol la burla ( vedi fopra C, 4. ft. 47..) onde shertare , lo ftef-
4 fo y che beffare. [i Davanzati ped dite Swerrare nella (ua traduzione di Tacito .
mY Corte poefie fenza antore , che fuertavano le fue crudeid . Se bene in quefto luogo fi
; poirebbe intender per berta quello ftrumento , che ferue per ficcare i pali ne i
ea pfiumi nel far Je fleccaie , che ¢ un gran ceppo di legno ferrato, il quale infilato in
“ln pernio , 0 ago di ferro confitto fopr’ alia tefta d’ un palo , s'alza per via di fu-

ni, ¢ fi lafcia ca(care fopr’ alla tefta del detto palo ( gia fitto in terra) per fario

sf andar pita drento. E perché in quefta medefima guila faceva Colci coi gatto,in-

yo teade , che defie cosi /a berra , eruendofi del mazzacavaljo, che appretio gli an-
ti"  tichi era ufato per arnefe militare , come s' ¢ toccato fopra C.6. ft. 86. In propo-
i)" fito di Berta per Bxrla , il Ferrari dice cost: ognuno poi la creda , come gli pare
4 f verifimile , Dopo aver detto , che que’ delio flato di Milano chiamano Berta
8 ta Gazzera , ¢ cid dal balbettare,ch’ ella fa ; foggiugne ; (aoniam autem fanne,
gil! At que irrifionis [pecies eff aliena verba imitando reperere,inde Berta pro Inda,ae derifione
gi accipitur , © fare una berta illudere , & decipere. O pure finalmente ¢ forte pit
credibile , che venga quefla manicra di dire dalla novella raccontata fopra nelle
Annotazioni alla St, 47. del quarto Cantare ,
d& —. SGVAINA I agna, Cava fuori’ ugna , che tiene alcofte dentro alla pelle , la
we bed gli ferue per guaina , ed il Pocta {cherza , dicendo /guaina U' ugna. lock ques
gnano

Wo + Appropriando beniffimo wns, a Vgnano. _
yd 4NASPRITO . Incollorito , meflo in ira , in ftizza, in rabbla. Latino exa/~
lii 1N

peratis,

i

 

 
 
  
   

$4

IN parlar Soriano , Ciok oer gatti in ling
fi dice quello , che ha la pelle di color lionato ferpato d
ché fi dia in altri animali, o in panai, non fi dice foriana ; fe
perché i gatti di tal colore fien venuti di Soria , come ai
di Perfia quelli di color di topo portati da Pietro della Vaile,
chiamati Perfiani , o per Perfianini . ‘one

DISERT A, Cioe ttroppia ; concia male . Guafta .

VVOL ievarne il brano, Brano dal Latino barbaro mn.
il pezzo , Vedi fopra C. 6. ft. 47.

MIAGVLARE , 0 ignaulare . Bi ii

  
   

  

   
 
    
 
 
   
  
 
   
  
  
    
   
  
   
 
  
      
 
  
 
 
  

  

I gridar de i gatti ; + il’fofiare dic
quello ftrepito , che fanno aprendo la gola , quando fond in rabb

S' ARRONCIGLIA., $i torce in {fe fteflo , come fa la ferpe quan
viene da ronca , roncola , ronciglia ; fpecie d’ arme ; 0 pitt” 2
agricoltori , ed ¢ fata come una {pada , ma é torta in cima a guifa d
ferue per eflirpare i pruni: o pure da Ronciglio, ufato’ da’ Dante per graf
fauto a ufo d’ uncino. i "

E MIRACOL 8 egli feappa , E cola foprannaturale, © impofiibile,
degli artigli, £1 Petrarca. soe eee
E cio , ch’ in me non era

Mi pareua un miracolo in altrui
cioé una cofa , che non potefie ftare . ‘

LO tiene in brigla , Cioe 10 maneggia bene , facendolo operat

CLARPA , Dal Franzefe e/charpe , banda , bandiera .. Quel draj
tano i foldati cinto:de’ foldati era proprio il cintolo , onde cinguoie fol
dalla milizia , Vedi fopra C. 5. ft. 33. 5) ie 7

FAR fiera, Bufcar , 0 acquiftar roba = per efempio ends pirando per
torni , ¢ chi gli dette pane , cht voua , chi una cofa , ¢ chi un’ altra tanto,
Satta un poco di fiera , fe ne tornd, mn .

D1 pofta, Subito : Di primo tempo. Vedi fopra C. 7. ft. 92. BY
giuoco di palla , che fi dice dar ai posta quando fi da alla palla, prima
terra , ed ¢ il Latino ilico , ¢ vefigio, Gli antichi dillero: Di colpo y
fo , che di Borto. 7

FV per fpiritare . Hebbe un grandiffimo fpavento , o paura.

GIGLIETTO . Specie di trina con punte ; cosi detta , perch ha
col giglio .

Avr. Cioé gnell’ ordigno , col quale la donna alza , ed ab
Vedi fopra C. 4. ft. 69. Se bene & pud ree la voce rigire nel
mo fopra C. 7, ft. 41, , ed intender, che Graziano , alzando il ca
giro, cioé la donna , ¢ dedurre quefta opinione da quel,che foggiung
Vedendo , che s’ intriga con Dame, ,

ACCWETT AT A, Afiuta ; Sagace . Tolto dagli uccelletti ,
civertats, quando havendo altre volte veduta la civetta fono dit
non fi la(ciano lufingare a volarle attorno , come fanno quelli
mai pil veduta. ae

eANZL cinetta . Pili toto troppo ardita , ¢ sfacciata . Si dice’

, eel ee8 TR

e— lL eBeuwe ete

_— 7 ate

= ~ - — =

 
 

x A juomo da poco , perd con tale equivoco

 

NONO CANTARE: 435

vane troppo ardita nel trattar con gli huomini , quafi faccia con effi, come la_
corerasss gi uccelletti, che cerca con gli (uoi gefti di tirargli a fe. Vedi for-
to in quefto C, ft. 60, E Plin, lib. 10. cap. 17. ;
CHE gli ¢ corfa la berresta; Che il gatto.ha fatto preda , ¢ gli ha portaro yia il
ppello.. Ma perché, La/ciarficorrer , pee via la berretta, yuol dice Elicres
mentando G h diveherch iandnban tones
raziano womo da poco dal veder , che fi lafcia rubare , € portar
via il cappello , gli an burla ; di che egli s’ adira , perché fi fente fete
if r¢ dali’ etiere burlato da que(ta donna ,
-, GLI fale lo frronzufo ai nafo. Derro {porco , che fignifica entrare in collera , ma
= poco ulato, dicendofi pil tolto fair la muffa , 0 Ja fenapa, 0 la moftarda, 0 it
herimo , ec. Vedi fopra C. x, ft. 39. Bil Lalli En, Trau, C, 2. ft, 65,
Waapn 6 Airs Corebo un tale firazio,e tanto,
i Con la moffarda al nafo ,e nol comporta,
AGli Ebrgi.colla fiefla voce fignificano , ¢'/ na/o , ¢ ira , perciocché par, che qui-
_¥iclla particolarmente rifegga , ficcome difle Teocrito + acris bits ad nafum fedet,
Onde noi dichiamo Arric¢iare il nafo per ifdegaarG ; fimile in parte quel che dice-
_wano gli anuichi Leware il miffo. La voce Ebrei fie Aph, in Siriaco Apha; ondes
. itorcec: ¢ venuta la noftra 4fa, colla quale a ete una cola fomi-
giiaptitima alle vampe dell’ ira ; cio¢ un vapore , ¢ yn caldo fallidiofo , ¢ affan-

    

 

BO!»
t ‘sop SLrifenre . S'adira : Entra in collera , perché ¢ burlato ,
pjat ‘A merda, Detio ingiuriofo ulato fra le donne di vil condizione , ¢ del-
Ta voce mona vedi fopra C, 5. tt, 18.1 Lagini similmente (asum , conum , frerquili-

me,

. FLACCATE. Spezate , Fiaccare & verbo proprio per efprimer , quando un Je-
£00 , © altro. materiale fi rompe in mezzo per fouerchio pelo, Latino fari/cere.,
_ springs . Donde poi bxeme fiacco vuol dir huomo affaticato , ¢ ftracco ; fe bene &

ver) imile sche venga dal Latino faces , faccidus , dichiamo , fiaccare |e braccia

A uno, clive infragnerglicle , ¢ romperglicle colle baftonate .

_SPKVZZOLARE . Vedi fopra C, 7. ft. 15. E qui é derto ironico , ed intende

f Bingge pict 7 ‘
V2ZZOLO . Latino vertex, cacumen. La parte di fopra del capo diffefi an-
she. Zwecolo 5 ficcome da Cocuzza de’ Napoletani ( Latino cacarbita ) ¢ fi dice an-
Gora. comiznole , fe bene quefto ¢ proprio delle fommita de’ tetti , ¢ de’ camumini ;
dal Latino cudmen quali culminnlum «
ares ST NZA XXIV, STANZA XXV.
Chi con , chi per banda ,.¢ chi fupino Quantungue il.campo annaffi tal rugiada,
i fe ne viene , ¢ fa certe cafcate , Come le zucche, annarpican le [cale ,
Che manco ie farehbe un’ Arlecchino y Onde pit a’ xno in gik verfala firada
, Quand in commedia fa Je fue fealare; Fa pur di nnono un bel [alto mortaic;
Si che y stinnanzi fecero il fantino , M44, piché ammonti ne traboechije cada ,
Le brache in fasti glieran pui cafcate, Sardonello [2a forte , ¢ in alto fale ,
> B infranes , ¢ pefti andando gis nel foffo E trai mimici al fine a lor mal grado

Mette [u il piede,e agli altri ye Uguado
2 PAN.

_, Hanatolere a quchlo nuove {cate adddfe, y
geet ii

 
 

436 MALMANTILE™
STANZA XXVI.
Chi vidde in un pollaio, ove fisrona | *
Vn numero di polli fenza fine
Tra lor cafcar qualche pokafira nuona,’
Che roft addoff' elt ba gullie galline
Ciafcun per far di lei P ultima'prona 5
Eye aa folelapariea athee, 1
Che la difende , ¢ da beccar (e porta Ma Eravan , che
Stroppiata rimarrebbe , ¢ forfe morta, Aiuto a un cempo,ed
Rotte le fcale coloro , che erano fopra di effe cafcarono nel fofla
r0 corpi furon polate nuove {cale , in fa le quali intrepidamente
neilo falto ful muro y ¢ feel nella Terra , dove fu da mojti di quei
falito: Ma Eravano , che lo vedde in pericolo d’ efler ammazzato
¢gli dentro a dargli aiuto .

BOCCONI! . Dittefo in terra , 0 altrove con la pancia , e faccia ve
no, Lat. pronus contrario di Sapino , fulle reni ; Lat. fupinus ye Per ,
Ja doppia poficura che refta , diverfa dall' una, ¢ dal’ altra , la diciamo 4 x
Per franco ,¢ Per latv, Lat. in latus . Bocconi  detto colla ftefla forma , che!
nocchioni , Brancoloni , Saltelloni , ¢ fimile 5 che fi -dicono anche Boccone
vhione , ec, anzi quefta ultima maniera é l’ ufata dagli Autori antichi Ti

eARLECCHINO . Va fecondo Zanni , cioé un feruo.femplice in
Cosi nominato , il quale faceva affai bene le fcalate , che fon quei giuoc
Ai fuol fare detto Zanni in commedia con una {cala a pivoli , fopra alla
affaticandofi di voler falire , cafca in diverfe manicre . f

FECERXO il fantine , Pecero il bravo , l' ardito , il coraggiofo , Si
gura. Egli ¢ fantino cioe perfona , da fare queffo ,e altro, Fantino di
faate . Lat, infans, cioe Ragazzino ulato dagli antichi in generale, @
oggi a ua fignificato particolare . Chiamando noi fantini quei R i, ¢
pr’ a cavalli {pogliati corrono al palio , Si dice anche fare if Baiardina, da
lardo celebre Cavatlo di Rinaldo Paladino, cosi detto dal fuo mantello y
yea efiere Baio accefa . :

GLI eran cafcate le brache . G\i era entrata la paura addofio
animo. Vedi fopra C, 6. flan, 20, Lat. aninsum defponderant ,

ANNAF FI tal rugiada, Annaffiare vuol dire Ammollare 5 0 af
giada vuol dire quel che accennammo fopra C. 2. ftan. 55. alla voce gr
Ma qui da nome di ragiada a quelle pietre ec, che buttavan gid gli

-dnnafiare detto da Adacqware , che fi dice anche /anacquare,e Annacquare y
Ui duc ultimi verbi diconfi propriamente del remperare coll acqua il vino;
equare propriamente ¢ dare [ acqua alle piante . + Ia
INARPICARE, Aggrapparfi , forfe dal Gr. herpein chet in
Pere , reptare , Salire in alco , appiccandofi con Je mani , € co’ piedi y
no i gatti . Si dice anche rampicare fopra C. 4, lan. 68. ed-«
vedremo nella feguente ottava 28, :
SALTO mortale . Chiamano i Giocolatori falto mortale,quando
tecra Con le Mani’, o con alcro faltano , voltandy la perfona fo}

 
       
   
 
  
  
 
 
   
 

> eran

  
 
 
 
 
 
 
  
  
     
   
  
   
 
 
 
    
        
  
   
 
     

WEITAF.

wee Se peg RRFESZLTE=

 

  
 

NONO CANTARE, 437

verifimilmente facevano coloro , che ca(cavano , o erono gittati da alto 'a batfo .
) TRABOCCARE , Intende precipitare , 0 cafcare da alto a baflo, romperfi
la bocca ; andar colla bocca per terra. E {e bene il proprio fignificato di trabuc-
“care é quando mettendofi in un vafo maggior quantica di liquore , 0 d’ altro , di
PS yche pofla capire’, cafca dalla bocca del vafo quel,che vi ¢ di pili; onde per
figura fi dice'un Trabocco di fangue , ec, tuttavia fi piglia ancora in fenfo di calca-
te. Traboceo ne i vizzi, ec).
hie = ROMPE il guado. Apre \a ftrada, 0 il paffo. Ovid. de arte amandi,comandando
‘ex che fi rompa il guado per via di viglietto , dice : Cera vadum tenter , Guado vuol
s dir quel luogo ne i fiumi,per dove fi pud paflare fenza navilio , che fi dice guada-
ve; Eda quelto guadare, o rompere il guado s’ intende aprirfi il paflo in qual.
“Voglia occafione , o congiuntura.. Parrebbe che fletie meglio vado dal Latino
mis » ficcome fi dice ancora yolgarmente il porto di Yada, dal Lat. Wada VYo-
_ taterrana; perch cosi Gi fuggi V equivoco di guado (pecie di tintara , mas
ivell quelli ftitichi , i quali fi vergognano , che la noftra lingua fia aiutata dalla {ua+
frit madre Latina,non ci concorrerebbono, ¢ darebbono una turbativa a chil’ ufaiic .
hist = MANDAK 4 Purafso. Par morire; E perché fignifica il medefimo che man-
aoe » 0 4 Scio credo che derivi da i foccorfi maadati in diverfe occafoni ,
| “tempi ai detti tre Juoghi , da i quali non effendo tornato veruno di quelli , che
al —andarono , quando fi vedeva mancare uno in paefe , fi cominciafle a dire. Eel
stl ¢ andato a Buda , a Scio ,0 4 Patrafso ; per intendere egli € andato in luogo , don-
de non tornera mai pit, duc, unde negat redire quemquam ; ¢ s' intende egli é
i = Morto. Vedi fopra C. 5. itan. 13.
j TIRAR P ainxolo. Vuol dir morire , dalle cunvulfioni della perfona , che pa-
§&  tilcono quei,che fi muoiono , Aixslo é (pecie di rete da pigliare uccelli. E la for-
2a, che fa ’ uccellatore nel tirare l’ aiuoio , o fimil forta di rete, ¢ deferitta das
id Petro de Angelis da Barga in que’ verfi +
0! Tum vero innitens pedibus confurgit ,& omnes
Intendens neruos magno trabit impete funem .
4 ZO feorge debito. 1.0 vede in pericolo di morte . ‘
STANZA XXVIII. STANZA XXIX,
1 ‘ Chinmgue é 'n Caftelle allor pien di paura - Auitiene a lor ne pri , ne meno un’ iota
% Corré per far © auanti et prit non vada, Com’ ai fancinlli , quando per la via,
Fan la tura at rigagnol con la mota ,

 
    
  

 

«RB memrtil vuol rifpinger dalle mura ,
' “\ Ch altri pik la 2 arrampica non bada ; El! acqua ne comincia a portar via,
| | itr db ouniare anco di gua proccura Che,mentr' affodan quixi ov'ellaé vota,
| Main fete Ini ghit ged farts la frrada, Effa diftende altrone la corfia ,
, E fe riparan la , prt qua fracafsa ,

| E a cogs intorno tanto il popol crefce ,
C" ogni riparo innalido riefce . Tal ch’ ella rompe,e a lor difpetto pa/sa,

«\ [Soldati di Baldoné fuperate tutte le difficuita , finalmente entrarono in Mal-
© mantile, éd il Poeta paragonando quefta cacrata ad un’ acqua corrente, che rom-
_ pe, € paffa ogni oftacolo, che le 4 pari avanti , efprime I" inutil difela , che fan-
“no i Terrazzani .
ARRAMPLARE . E' jo ficflo che inarpicare detto poco fopra , ed é il Latino
Perreptare.
VN

™"

 

a
 
     
   
 
 
      
     
    
  
    

438 _ MiALLIMIAN TLE Bot

VN ita, Vn niente , detto fopra C. 1» flan. 18.)

RIG AG NOLO . Diminutivé di ome 5 Piccolo riva,
é proprio per intendere da parte pit: bafla,che ¢ nel 0
di Firenze per dove feotre P acqua’s che piove y efic 7
incende nel prefente luogo 4 € ¢' aacénide comuacmente s che ua
rigo , 0 rio diremmo rixolo 0 ra/celloy dewro cost da Riuiceday la
preffo alcuno antico. Se bene Dante nell’ Inf. C, 1g.dices Bd
Sente rigagno , ec. ed intende quel fiatnieell@ , 0 rivos il, 0
nali. Li Varchi Stor. Fior, libro 13. Commiciarono ad nfcar fuara
e che i rigagnoli correuano, ele ve erat piene di motayedifarge rt
Nov. 16. 4 rigagnolo delia qual via corre, chepare un fiumicclian |

MOT A, ‘Lerra ben inzuppata acl? acqua. Ai Percariz, Lupums
 immora, Per intelligenza della \iuddetta comparazione ¢ ince
i ragazai dell’ 1afima piebe di ae ‘fogliono per loro pa
dopo ja pioggia (corre !’ acqua per detti rigagaoh pigliate del
ond ae come un Danian oppofte ai corfo dell’ aon
paflaggio al fume , ¢ quefta chiamano la twra.; ma fiocome d’
quel iuogo fempre va cre{cendo ,)cosi 0. per 10 pelo, rompe
bondanza traboccando Ja fuperay ¢ pada via noa oltaace dri
v' appiichine , come dice il Pocta . Qunero nell’ Aliads ib, a 5s,

De! Troiani fereci allagranturbay. ..
dt folgorante eApollo andanainnanze
Tenendo in mano il preziufo fondo:

Ei degls Achini il muro aterra Sefe;
Ne coffogli fatica , appunto.come
Lungo il mare il fanciulfacoll arena y
Che poich¢ fabbricato ba per. fuo gsoco
Va gentil fanciullec{co alto lanoro ;
Colle mani , ¢ co’ pie {cherzando il guafta ,

A lor dijpetto,, Contro alor voglia. Lat. ijs innitis , Il Boce, diffe
Per di(peto. A Dante prima, ¢ poi al Petrarca ia uecedlica della rl
il feruirli della parola De/picto accordandofi in cid, ficcome ima
col dialetto. Provenzale, o Francelco. Virg, ecl. 2. Defpectus tibi Jum ne
queris, Tu m’ hai in difpetto,ne ti cale il fapere,chi io mi fia, Confiache
la ftrada , che é.per il mezzo della galera ; onde que) groilo Canaone.
diceli Cannone di corfia, S’ intende ancora per la correate dell’ acqua..

       
  
      
   
      

FeSlFaer- aes

eet

it Se OR Peas aw

   
 
 

STANZA Xxx, opqas ae a
Gid tutti fon di fopr’ alla muraglia , Celidora a due man 4
Che 1a circonda un lunge terrapieno ; Che ne-anche un vi
Gia fi fiorifee in si crudel barcagha Tanti fil d'erba gol

Di fanguinacci la gran madre il feno: Lane’ buomini cost
 

4
NONO CANTARE: _ 439.
o- - STAN ZAXXEL oo. STANZA XXX.
ee, jth Amiffame —. — Adafa di Coccio a quefto,e quel comand,
Da toccatori fan col brandifpocco , Ed all'un dane,e aun'altronepromerte ,
d h Lacompagnia del Furbainnanci mada,

“Pere che della morte almen Ceffane ,

‘Se non prigion fi fa chi é da lor tocce , Che refti ai fianchia Batiston commette
AIP incontro ritrovafi Sperante ; Com Pippoyil quale (Pa dal’ altrabanda,
WA) + Che fa menando (a fua pata , il fiocco, Ma egli imretreguardia poi fi mette ,
Wh E fe gid le fuftanze ha difipace , E mentr’ognun favanza agloriasmtente
a Ei fiede a gambe larghe,¢ fi fa vento.

 
 

Hor mand'a male gli buomini a palate,
+ Effendo gia wtci i Soldati di Baldone faliti fopr’ alla muraglia , ¢ padati oclla
PS di dentro fi mettono alla difefa, Sinarra la bravura di Celidora , di
y edi Amoftante , s' accenna 1l valor di-Sperante , !a diligenza di Mafo
S eraccc pane wragtoe ut Coot cies A
La gran madre fi fe i fanguinacci 11 feno » Ci terra s'afperge di fanguc:
#88 Ounero nell” Lliade (petisind « = :
pm 8 di fangue la terra intrifa corre.
® La Gran madre per la Terra intefe if Petrarca nel Trionfo della Morte .
elf SEG ORY O ciechs 5 if tanto affaticar che giova?
jeu 08 Tutti tornate alla gran madre antica 5
pone E'L nome voftro appena fi ritrova .
_  TOCCATOR?, Vedi fopra C. 2. tan. 60.¢ C. 6. flan. 44. 3
_ “ BRANDISTOCCO.. Specie a’ armein afta ; fimile alla picca , ma I’ afta pik
corta, ed i ferro pid’ largo y ¢’pily lungo , che non ¢ quel della picca ; ¢ credo
venga dal Tedefco froch , che vuol dir battone , € brando che da’ Pocti Eroici mo-
derat fi prende per Iipada , ¢ fignifichi Spada in ful baffone.. Stocco ¢ dal Greco
Felechot Lat. Pipes , candex , da cui é facta anche la voce feecco ,  perciocché pri-
ma per batterfi fi adoprarono le-mazze , ¢ poi fi venne a ferri ;( Orazio Serm.
1.1. Sat, 3. Vaguibus © pugnis dem fuftibus , atque ita porro Pugnabant armis » que
‘pelt fabricaverat nfus i nomi potleduti gia dall’arme di legno , furono ereditati
‘dalle arme di ferro , che a quelle fuccederono . Onde Stocco , che in Germanico é
baitone , a nOi fignifica /pada corta, ¢ floccata ia ferita,che fi da con quella . Brand
* jn Saflonico ¢ riz one , 0 fuoco; onde Brandifpoccbi poterouo eflere cio che Virgi-
“tio lib. 7.¢ 11. chiaina /fipires , © /udes pranffas , ovvero obuftas cioe baftoni , 0

mazze appuntate col fuoco . 3
' CESSANTE . Si dice quel debitore , che effendo ftato toccato da i toccatori

“pud effer fatto prigione dopo le 24. hore da che é lato toccato , ( del quale ato
me rt e. (a 60. ¢ C. 6. flan. 44.) ed il Poeta {cherzando coll’

‘Paclammo fopra ©.
egnivoco toccare, cide efler percoffo ; dice che quello , che da coftoro é tocco di-
viene almeno Cefante della moree , fe non prigione , ed intende che quello, che da
coftoro é ferito o muore ; 0 refta vicino al morire , com” é proto ad andar in

Prigione colui che ¢ tocco « ‘ <

FAR il focco , Fioccare vuol dir quando nevica gagliardamente , € da quefto
diciamo fare i! fioceo per efprimere un’ abbondanza di che che Ga , per elempio fi
fa ii fioece delli uccelli , 0 de’ pefci , 0 de’ denari, ec. fi direbbe a uno,che pigliaf.

fe molti uccellt , molei pefci , o molti danari , ¢¢. & cosi nel preteate luogo inten-
de

%

 

 

a

 
 
   
 
   
  
       
    

440 MALMANTILBE

de che Sperante ammazzafle molti huomini con
il vello della lana Lat. foccus., Si trae anche come's’ ¢detto
ve , che Marziale appella tacitarum vellera aquarum, La
in abbondanza , fi dice Fioceare; ¢ ftendefi anche r
aver dewo di Mcnelao: Poco dicea , ma bene , viene a dire d’
Atandaua fuor diluvi di parole 5 ‘
Come allor che di verno ilnembo fiocca y.
E fu pe’ monti nena a! ogn’ intarnos dohlgioee
MANDAR male a palate . Vuol dire mandar male il fay
gamente ed inconfideratamente. E qui ii Rocta ia Spe
vendo havuto per coftume di mandar male ii tuo a 0
P antica ulanza di mandar male a palate ancora gli huomini 5 ¢d
con quella {ua pala , concia male moltihuomini , '
A chine dd, ¢achi ne promette . Diciamo cosid’ uno infolente
che tutto il giorno facia rifle , perquotendo quand’ uno  .< quand"!
con quefto dettato il Pocta deferive la,natura. di Malo di Coccio , il
s' ¢ detto fopra al {uo Inogo ) era huomo di conuerfazione ,¢ nelie tel
ordi , ne 1 quali fi trovava , foieva vOler (empre fopraftare gli aluri
¢ ca Cf farfi ubbidire con le grida , ¢ tainolta con ie butie,
# gambe larghe . S’ elprime con quefto termine la commeaita , ¢ (pe
ginc,con la quale uno ficce a pigharh ripofo ;(¢ fi dimotira un pimuo ¢
sare , ed amico dell’ ozio , ¢ delja pigrizia ) che fidice: Stare iw Rane
C, 3. flan. 72, € C. 3. flan, 1, 60m s¢ mani in mano; Con ie mani in cintola, —
STANZA -XXXiLL STANZA AARIV,
Amoftante alt incontro un nuoko eAarte.  Vedendo i Terrazzanigbe ftannoin fa
Senbra fra tutti anants alia testata 5 Che il nimico ad S[padeye gioca
Lo fegue Pao C orbi da una parte s Ler non far Mole 1H fab MALCON KG i
E aa quest aitra Egeno alta franceta, Ritsranfi , ¢ non sengon pik
Vengonfiin tanto a mefcolar le carte Ma fperon ben ( moftcanaoas
E vien /pade ,ebaston per ogni armata, Denari,e coppe)indurghs a far p
Ectidam puche ,e 4 gsmocar none leflo a) fi
Vs perde ia figkra, ¢ fa acl r¢fto. Speaifcon , che pario in
eile preicnu due otrave il Poeta dopo haver lodato per vaiorolo
feguicato dai Corbi, ¢ da Egeno , icherza in {ull' equivaco del ginoco 5 &
{ucne rai as/corfo dai pronerbio « Vengonfs a mefcolar le carte ,( che
€ \¢ ne Locca , O fe ne 1iceve , Come vedremo {otto C. 10, Ble
auibedue 1 campi vanno ( cioe s’ adoprano ) /pade , ¢ ba/tom , ¢ che chi
che ( ive urta nelle picche ) perde /a figura (che € una di quelle carte, nell
Ji fono efhgiaui gues fantocci , che ne 1 ginochi di daia tono te carte,
cive perde Ja propria perlona,e fa del refto ( cioé muore ). £ Terr
in fors , C1U¢ hanno i lor punto in fiori , ( ed incende tanao ip
Bria ) vedende che 41 nimico ad /pade ( cioe adopra ic ipade) . Per non,
+ maitom: ( cloe per non fare un monte di mori in iu 4 mattoni, ¢ ¥
fui terreno.) ff r#tir ano da chore ( cide Jafciano J’ ardire ,) me tengon
Vuoi dike HU VoOguon pid giuocare y ed intends non vogiion pil

  

 
 
 
  
   
   
    
    
 
    
 

  
  
    
 
 
       
   
   
 
  
  
   
  
  

 

gs gp emer Ee ae PP ee EsP soo eee eee FEE

 
 

« NONO CANTARE;

ano di ridurgli a far partite , cioe accordarli , moftrandogli

 

44t
i ddwari', e coppe , cine

_ ofterendo loro dell oro: E pee quefto mandano al Campo un’ Ambafciadore_ ,

che parld nella maniera che fe
_ STANZA XxXxy.

_ Spida Signori ? armi ognun fofpenda ,

Ache far quefta guerra afpra,e mortalel
Fermi per grazia ; pit non fi contenda,
Per c! alsrimenti vi farete male.
Fate che la cagion aimen s' intenda,

| Ca cherichedi a quefto mo.non vale

F

ni

it

ee

| Bchi pretende venga con le buone ,
Che dara glifard foddisfarione ,

‘ntiremo nelle feguenti ottave .

“STANZA XXXVI.

Con queiyche dona per amor non s' nf4
4n tal modo ta forza ,e la rapina ,
Chiedere,imperciocch? giammai ricufa
Ui ginfto, ed st douer la mia Regina,
No entraron mai mofchein bocca chinfa,
E con chi tace qua non s indonina ?
Pofs’ egli accomodarla con danayi?
Dungue parlace , ¢ vengafi ai ripari,

_ L Ambaiciadore de 1 Terrazani efpone la fua amba(ciata , ¢ chiedendo tregua,
-elolpenfione d’ armi conchiyde che la Regina di Malmanule ¢ pronta a dar loro
fodistazione , pero domandino , che faranno efauditi .
|. SPiDA, Quefta é una parola ulata da j ragazzi ne i loro giuochi fanciul-
ye non hay ( ch’io fappia ) fignificato nefluno univerfalmente , ma nel
modo , che fe ne feruono i ragazz: fignitica fofpenfione di giuoco, o permuffione
@ eleacarfi per alquanto da efio fenza pregiudizio , appunto come fi fa con la fo-
fpenfione d’ armi in occafione di distide o particolari , 0 generali , ond’ io crede-
rei che G potefle dire , che quefta voce /pida futle corrotta da ssida , 0 disfida, I
- Fagazai fi feruono di queita voce cosi, per efempio. Wel ginoco de’ birri , ¢ ladri
detto fopra C, 2. flaa. 32, quand’ uno occa bomba 0 per qualche fua faccendas
on attenente al giuoco , vuol partire,per afficurarfi dal’ efler catturato dice ;
Spida, E con queita parola s’intende per lui fatta fofpenfione di giuoco: E quan-
do il ragazzo,che & Ggnore del giuoco dice Spida s' intende fofpenfione generale .
Ed il Poeta » che fi ricorda che egli fcrive una Novella per i fanciulli s' accomo-
daa i termini da loro praticati,ed intefi , facendo feruirfi a quefto Ambafciadore
della voce Spida per farfi intendere che vorrebbe fofpenfion d? armi .

Cae hericheli + Chetamente ; occultamente , fenza parlare. Varchi St. Fior.
lib, 15. Per Ze cafe fi facenano delle ragunate a chetichelli .

WON vale. Quefto pure ¢ termine fanciullefco , fe ben taluolta ufato anche
dagli huomini d’ eta , ¢ fignifica Non é dovere , Non conuiene , Non fta bene ,
ec. Prefo per avvenitura dal giuoco , in cui chi fcommette dice per efempio; Va-
dedi tanto ? E quegli che non accetta dice : Non vale, cioé non fo buona quefta
erate 90 pure quando fi fa contra Je leggi del giuoco, fi dice fimilmentes

NON entraron mai mofche in bocea chiufa, Chi non chiede , non confeguifce ;
chi non parla non. intefo. Lo Stefonio nella fua Gnoccheide ato primo {ce-
a prima dice ,

hee pak Vulneris alcofti nunquam medicina paratur ,

£ viene a fonar lo ftetlo che con chi tace , qua non ' indoxina , Plauto nel Pseu-
dolo Att. 1, fe. x. ove introduce lo [chiavo , che cost parla al fuo giovane Padro-
Ae Innamorato ,

Kkk Si

 
 
  
     

PrViot ovoy

  

E poi conchiiile:
ina fuggire i litigi .

dice Cosi

“STANZA AdXxVIL
A quel sl General ,c' ha un pod’ ingegno
Kusene i] colpo,e in dietro fi difcofta
Che fi fer mina i fuoi , aipoi fa fegno ,
' Pala parola , ¢ manda gente 4 pofta,
Ne bado molto a fargli har a fegno ,
Chela materia fi trove difpofta ;
Crafcun a! amie le parci ferte faldo,
pC? ognun cerca fuggire il ranno caido.
STANZA XxxXvVill,
ch della pelle ba punto panto cara y'
ch che von vorrebbe effer nccifo
‘empre de feiarre di fuggir procexra s
~ BYe mai c entra, ba caro effer ditsfo,
' Ben ch} ei. mostré non, baner pasra
S? in quel Cimento lo guardate in vif
Lifciato lo vedrete d’ un bellerto
* Compofto di giuncate , ¢ di brodetto ,

 
   

* Ordiaa i) Generale, che fi fermi il combatteré , ¢ trova i'Sol
dieatidivai » perche a ogauno piace il vivere; ¢ fia wno'coraggiols
mai effere, al cimento poi non haura careftia di timiore. Fermato gue’
battére , Chi era ferito s’ ando a far medicare ¢ ah

PASS AR parola, & termine militare , che fignifica far fapete |
Capitano’ per tucco P efercico con dirlo a uno , che'lo dica a ua?
vada feguitands fiaché lo fappia ogauno fenza che fi faccia
Qi. Gli aatichi Capicani fa

fiziali fubordinati wa
fi conteneva I’ ordine di cio, ch

‘di yoo! ior cl icvar qiaao da ij i

aes ie Hic Sen aes

eee UM AN TY Le Mi

v hominum parfi vem
6 °)\Nees te rogandi , © eileen ye x
Nunc quoniam id fieri non poteft
| Me fubiget , ut ve rogitem'; Fjord
Eloquere ut quod ego ne/cio , id tet ‘
“PVOS S? egis actomviarla con danari . Ci é egli modod
trovVar rant denaro, che aggiuitj quefta ca

*'‘Dungue parlare, Queft’ ultimo verlo pat tolto: di oda g
1, ove Teti patia al'iuo Figiinolo addolarato ;

Parla; sis Wb habs sit digi by beds, ue

Tener la'cofd Wath tua'beentt afcofa'y mins aA

eAiciocché tu ae thee as
TA

  

oe

sa Pees

Dewo uli

 
     
  
  
 

1h te ada,

0G ease
Bao i as
ade

     
    
     

SE FLSFFSRPSR oS Staetes

A
Sien: Projo brau,
Se mai vengono a
Crediare che elo fan
Perec’ a rutei viene il bi
Ech ela palferebban
Se lo potefser far con tor
snithewsate i a quella opiniiie
Di veder Cuanro viner fa
STANZA XXWK
E quefti che badauane ax
in Malmanty, 8 accorfe
Che que; none meflier
Pero fi contetaron dell”
Gai tagle alcuno impi
Hitri rimette braccia,e ¢
Altri da capo a
Echi fifa uae ed

SEF

pest

     
  
      
    
 
  
 
 
 
 
 
  
     

as
. =.

 

ih
nye
aaa

- o = & wo _Z.

 
 

f es haat 1803 ae Udsarntare Teffera . Amminiand’

 
 

eee -

NONOPQANTA RE

} #88
‘Siliodtalico,, eee etn ee ff
pases temmalumeaee €.con ording ,.0 de Da by eat re
ea - oa Leib se sittin

ST ROV Oar niaseria,difpofta . Veove. prontezza d! ubbidire » perch? cialcur

- inclinava a lafciare il combattere . Sante eT ae
\ \ EVGGARE it ranto valde  Buggire i pericoli ,o le fatiche, ~
| HA care eferdinifoHa caro che-qualcuno entri di mezzo ,.¢ impedi(ea i
tocombatreresiche queito vuoldire diwidere una quiftione . Lac, pugaam dir e.
elLilcio Lateadiamo tutte quelle mefture ,, con le quali aicune»
sper parce-bellefi lifciang. ta faccia 5 che diciaino imbelietrarfe : decto [:con-
do aleuaisda wRerlerra. cio’. melmay fango . In Franaefe il (cio dicefi Fard , onde
ciog unbratwace 5 ¢44re ana farda , ¢ wna fardaca , il che figuratameate>
- Sluergognare uno.con mato pungente in pubblico , che alccimenti dice fi; dur /a
 Ceretata 5 E. dare una cenciata fudices , ccacta dal coftume de’ Ragazzi Fiorentini,
che il'di di yuezza Quarefima , quando ( per ufare un loro idvoti{mo ) fi (ega las
eal cioé viene ad.ctlere partita per mezzo quella Stagione di penitenga;
Peete ior abufo,ednfolenza batcono el vifo alla gente grotlolana , 0 fenipiice
pd al COntado.cenci intinti.nell’tnchiottro , 0 in altro fudiciuine.. Branco Saccheci
diffe Dane ca Fare, ¢ dare nna zapare , per offeadere coa marto. Vedi fopra, a,

ae Pilla 45.0 : base wid Ca Te ya
jit «OM Ne ATA, Latte rapprefo,, ¢ (errato in fogli¢ di farfara con giynchi,, e»
Gdecta ginncata , la.quale mefcolaca con broderro,che ¢ mincitra, fata d’
Wlovauidette liquide con brodo , o acqua ,¢ agrelto , o fugo.di limone,,. farebbe
ua color¢ fra ij, giallo 5,¢ il bianco , appunto come diventa ia faccia di coloro,che
& i da fubito timore , sink, 1
ASN AD/ERL, Huomini fanguinarij : Da Mafnada,che yuol dire truppa.
P di Soldatic: what, militum manus Ma per lo piii intendiamo compaguia di ajiaii-
at poeaid Aieada.
TIRARLA

fuori , Cio .cavar fuori la {pada per combattere . Virg. vagina.

VOISN Gx

  

aetkbEiks =

TREES

RESEES

er aay ; .
= SATEICVORE. Ecceffiva paura , ¢ fpavento. Dicefi folo dal frequente bat-
‘eres che fi fence dalla parte del cuore in uno,che habbia timore. Se bene af but-
ter del cuore ¢ indizio ancora d’ altre paftioni , che futte anno quivi lor, feggio ;
“eae gran defio , congiunto colla fperanza di vicino conféguimeato del defi.
rato bene, la quale pero dai timore , non é mai io tuto disgiunca , :
sualptelten’arctben 4 Jeegiert, Paciimente lafcerebbono (tire dt far quella quittio.
‘Re. ln un frammcnto di Storia, Fioreaciaa manofcritca,che dame oa tisfa di -
i ncarui il principio G legge: 5, Gli difero ua monte di villagia , ¢
ond 'ingiurie, ma.il Cattellano , che era di uci Soldat, che avg iano canioin
dt ight doula Cavalleria, fe la palso di leggiers,¢ la Ciaaiogii gracchiare ,
sgnattendeva.a ftar. deatro ;.ed a i fuoi Suldaci , che Jo pregavag» a ulcire , ¢ dare
vs, addoffo.al nimico,, rifpondeva ; Lo noa vogity ultirs , percaé nog voglio cae
Se CEDER guurs [a vivere un poltrone . Con quelto termine defcriviamo 490, che
yuo brighe, ac faciche , o.penfiert , a¢ meno f yuule efporce.2 rifthi , o.pe~
SS Ree : ye oe

 

MSS ERE ES

ae
ar eet

  
 
  
  
 
 
 
    
   
  
 
 
    
  
  
 
   
     
     
     

444 MALMANTILE

ricoli di forta alcuna.. Il Ferrario feguitando il Salmafio nel |
le che la voce poltrone venga da Police trunco , dicendo che:
andare alla guerra fi trova che fi troncaffero a pofta da lor
dito grofio ; B dovea effere ufata tanto quefta furfanteria , ¢
tali il {oprannome , e furono appellati Azurci fecondo che
Cellino lib, 15. il che.volea dire poltreni ; poiché Murcia pret
mava la Dea dell’ oziofita , e della poltroneria , Origine et
non la credo vera , ftimando che 1a voce polerone venga pill: fto da
poledro , ( come alcuni fpiegano quel be/fie poltre di Dante Purg. )
Poltrone a.uno , che non vuole , 0 non pud durar fatica , appu 0
dro , il quale non é ancora atto alla fatica . Ovvero da poltro ,che
fecondo 1 Landino fopra quel patio di Dante Inf. 24. che dice

Hor mai conuien che tu cosh ti fpoirre ,

Diffe it maeftro ; che feggendo in pinma

1n fama non fi vien , ne fotto coltre .

Donde poltroni gli huomini pigri  ¢ dormiglioti , dice il

zione di quefto patfo .

PREG Sk FS oye = oe

— meftiero da abborracciare, E’ cofa da farfi confideratan t
cafo,
_LMPLAST R ARSI con le chiare, Medicarfi con le chiare d’ uovo le ae
di fopra in quetto C, ftan. 4 A a Re
PARSI aar de’ punti in ful cefs, Ricucired tagli, che ha nel vifo ,: quale cae 9 pe
ma cefo , perche guatto da i tagli , non merita nome di faccia. Cefe o Fran a
fe € parola nobile , che fignifica Capo, come alcuai vogliono , dal Gr. gi grps mH
nol ¢ parola di difpregio , ¢ fignifica vifaccio brutto. ae ‘a
STANZA XXxxl. STANZA XXKX ui
Baldane in quefto per la pit ficura * Et effi andaron con la lor patente tp
Due gran Dottori atrattamentiinuia, Di poter dire ye fare, € alto ¢
Lun Fitfolan Branducci che proccura Lor camerata fa tra? a
D' haver fe non po in Pifa,oin Paxia, Che gli feguia curiofo per. =
<ilmeno in refettorio una lettura 5 Baldino Filippucei lor yy
ZL! altro é Meinforcon da Scarperia , Huom, che pis tofto canta py
ChefeVbuom vine per mangiar vi ginro, Crefcer volea come gli altri appa e
Ch’ ei vuol campar mill anni del ficuro, 3 44a fi pent),quand’a e
. STANZA XXXXIL. STANZA XXXXIV, 9 &
Calfandro Cala Cheleri fra tanto Son alti gli altri due fuor di mifar «
Del Duca allora il primo Segretaria Ond! ei nel me? 0 camm
“ 7° loro un difcorfo di quel tanto Refha aduggiato sv hed)
evan dire al lo aunerfario Ne men pro cre{cer pits
Cacciatof, Giosieiae: ar
Efcorfo turto if [uo vocabolario

Scriffe in manierayefeceun tale Spoglio y
Che mefe un mar diCrufcain mexico feglio,

 
 

 

NONO CANTARE: 445

PRES os | HROMOVE TH ANZ AX X "BV. lov
ella pure alor quiui's'inchina , Purche il nome conferui di Regina,
Dando a ciafennoi fut debiti riroli., Luando per t annenire altras' intitoli,
Econ effi ferme I! altra mattina Che queftons le nieghin, chiede al mato.
| Mdifcorrere , ¢ far patti, e capitoli , Wel resto por da loro il foglio bianco,

manda fuoi Amba(ciadori a Bertinelia, i quali con efla fermarono di
flabilire i capitoli della pace per la matuna feguente , promettendo Ja medefima
| Bertinella d' acconfentire a tutto,pur che le retti il titolo di Regina .
DE gran Dertors, Dice due grandi , perche veramente erono ambedue di. fta~
a ce alta , ed un folo di effi era veramente Dottore , cioé Ficlolano Branducci ,
ai che ¢ Frdncefco Baldovini giovane dotto , ¢ {piritofo ; ma perché nel tempo ,che
i fu compofta la pretente Opera era afiai difapplicato, pero lo motteggia, dicendo,
che egii proccura d’ havere una lettura in un refettorio , {e egli non la pud otte-
_ Berein Pifa ; 0 in Pavia. Ma non voglio gia io lafciar nelle menti di chilegge-
_ fala prefente Opera I imprefiione’, che quefto Baidovini fulie Jettore da’ Retet-
fod t0rj , € pero dico , che le (ue beile , ed erudite compofizioni lo fecero conolcere»
infin in Parigi , dove eflendu fate fenuite in diverfe Accademie dall’ Em. Sig.
ym Card, Chigi tino di la lo fece chiamare a Roma, ¢ lo diede per Segr. all’ Em. Sig.
» Cardinal Nini , la qual carica eghi efercito pi anni molto Jodevoimente ; mas
kit Beceilitato dalla poca buona fanita , che godeva in quel clima , fe ne tornd allas
| patria , dove efiendo ftato prowvilto d’ una Pieve, quivi fe ne vive godendo mag-
b,@ Blor quiere, ¢ miglior faluce , che non godeva a Roma. i
él MELN forcon da Scarperia , Pierfrance(co Mainardi grandiffimo di ftatara, ma
G8 ware dottore . Quefto per effer,fi pud dire,un colotio , ed in ful fiore della gio~
veotl thangiava ati ,¢ perd il Poeta dice , che fe 1 mangiare fa campare , ¢gli
(Ill Per viver molto tempo. L’iperbole di mile anni (e bene & di numero determi-
‘ge ato; fi piglia per indeterminaco , ¢ fignitica lunghiffimo tempo .
I * CASS ANDRO Cheieri, Cive il Sig. Aictlandro de’ Cerchi Cavaliere , e Sena-
we tore Fiorentino Segretairo della Sereni(s. Granduchefla , e perd ii Poeta lo fa pri-
mo Segretario del Duca. E perché veramente egli € un Gentilhuomo di gutto
"i isquifito , ¢ d’ una cloquenza aggiuftaciflima , dice , che con la direzione del Boc-
sil caccio (le cuj opere regolano Ja lingua Fiorentina per efier’ egli il noftro Cicero-
Ne ) ¢feorrends il fuo Vocabulario ( cive il Vocabolario della Crufca ) meffe um mare
di crufea in mezzo fostio , ¢ (cherzando |’ Autore con I’ equivoco di Crufca buccia.s
uv del grad, ee CRVSCA Accademia Fiorentina, intende, che quefto‘Caflandro fe-
id ‘ce un diflefo compotto di parole approvate dalla medefima Accademia della,
», ‘Crufea, nella quale fi fa proteifione di pariare , € {criver pulitamente la veras
“| lingua Fiorentina .
7 PER far un diffefo di quello , che doveano dire, Cioé per metter loro in {critto
I Iattruzione di come doveano’contenerai in trattar 'accordo,fi come fi faa tutti
gli Ambafciadori,e plenipotenziari,che G mandano da’ Principi, Repubbliche ec,
_ FAR to {poglio a! wn libro , Mercantilmente's’ intende copiare le partitede’ i de-
__ bitori ; ¢ per altro s'intende quando fi cavano da un libro quei concetti, tentenze,
‘parole , delle quali ci voguamo feruire in far qualche compofizione .
POTER dire ,¢ fare , ¢ alto, ¢ bao, Potcr negoziare , ¢ conciudere a lor gu-
Os

d)
i

 
e,
flo’, € vélonta, the ih
dicono: Peni; j
patentee Bis

libero.

_LALDINO Filippucci , Filippo Baldintcci d
e quelto intende il Poeta dicendo Huomo’, che canta'ben
¢reicera pid , perché egli ¢ duggiato da quei due huomini lunghi
e Meio , de’ quali egli lo dice’parevte , non perché vera

eg ee ee

e accomodarfi alla rima. Queito¢
jamo detto fopra nel Proemro . ~
* LVYOGO

5 STANZA XXXXVIL
Eperché ore gia finian del giorno
Siconfuled, che fulfe fatrafera s
_, Percio tutti alle fpanze fer ritorno
Com! un fatco digatti, fuor di Schiera,
I Cittadini Pavan @ ogn! intorno

* Welle radesfu i cantize alla fronciera, Che non fi
Bicivcgh’ ognun fecondo il {uo porere Gis teiehnzs Gene Bl
© 5 foreftieri in ala dia quartiere, Sti Mab fpefa dicey men Wid

a ST AN-Z/A*REXRWVNA ome DAVIN
©"Del Principe a’ Vgnan pot fi domanaa , Poeperre ners

“\  perche la labarda anch' egls appoogs
* ‘Staffer attorno a rivercar fi manaa ;

  

  

un facco, a quait
LA quarticre »
fied a ME i

ee

 

uae

 

BG

anggiaco, Vuol dir luogd , dove nonatt
Pinterpofizione di muraglic } 0 d” altro, EY Gail doghile pian 00
tate , € con poco vigore , ¢ i dicona auggiare ; da Yggia » ombra , ;
TENNE un mexro miglio di pace. ‘Per mbitrar’, che queni t
haveano Je gambe lunghe, fi feruc di quefte"iperbole'd? un imezzo mi
DA loro il fogito bianco, Apptova tutto quello’; che effi conchi
Joro Jil foglio, bianco firmato di tua mano,acctocche vi ferivano lee
capitoli delia pace , come pili piacera loro, ‘Che ¢ Jo fteflo, chedit
in voi in tuto, ¢ pertutto , In quelto fenfo dific il Petrarca ». my

"Chi Lhabbia racceteato , e chil’ alloggi; x
Etiendofi gia fatta (era ciafcuno sbandd , €d i Terrazami tts
sex dar’ alloggio a | foldati di Baidone . Bertinelia iawn Pala x

¢d il Generale , 1 quali accctcarono Pinuito . Si'cered deiDuca per co

‘ch’ eli in Palazzo , dove-bnalmente egli venne dopo quaiche di

© che non voleva parurfi dalla iocanda , nella quale s’ era accomodato..

COME un facco ds Gatts , Cr0e lenz’ Ordine , o'regola 5 ma con!

~ tende, che ifoldau sbandarono , chi io qua y chivin Jay come

Gi dja! andare.
rova aliogyio , Dar

   

  
   
  

aan ta
a i

 
 
   
 
  
  
 
   
  
 
  
  
  
 
 
 
 
   
 
   
   
 
   
  
 

os MBA

aa

: wa
STANZA XXEK
Grants a palarro Bi
In Amofpame eC
E-wuol che (gli odj mai:
Stien feco 5 ma ciafe
» Puer' finalmence ne i preg

Se, es 8. SERS PES EL EETE RBPRERPS SR

S” era décniarovoue
Priaichiei n'wferfs
Nand per:

 
  
  
  
  
   

  
   
    
  

dort ~ 1
quarticre fignificara
ae Swan grote hk ey

  

      

 
 
   

em, Sista 30a sobre ipaaies
dA 12. epill. 33. quidem,
r Fer ime —m fed egoa egh , ur eee - Croe noo
wesmercnen gliteci croppe cirimonie. E apprefio . Pall pot C. Ca~
» Hlorum ego vix attigi penulam ; ramen remanferunt Dichia=;
e ferraiyo'o jinuitare uo. aitaseawate » © pregario a voler ‘rima-
co noi. £ ta/ciarfi tirare pel ferrainole , ¢ non accettarc |’ inuito » € ari
Koa > ‘
CH! vs difagio, Quand’ altri ¢ inuitato a un conuito aed
teatro. datalcuno.y.per licenziarfi da chi Jo tratticne ta full’ ora del ¢o.

s te Ja-caufa speria quale ei i parte, fuol fernirfi di qu:flo ates
al eons (a, non dia aifaeio : cioe {e 10 fon caula , che egli (peade, aun ¢ dovere 5
‘difagio-col tarmi afpettare .
“ ee ~Andar a mangiar a cafa d' altri fenza (pendere...:
operat ferraiuolo, o ¢appa.s perché 1n vece di quello ia porcano ful-
i:Alabardieri + i quali in occafione 4’ avere aire a tavola  {¢ ne, {pa-
ae appoggiuala-aila parece 5 ¢ perdo.con queft) decto intendiamo. Pofare ra
ior (ad! aters5c.quivi mangiare , fe bene Pe/are tl ferrainolo.s’ ay
“4 ‘aucora‘un giovane, che non ha provifione , ma ferue in uo banco,, 0 ‘in who ff.
2ibegravissy baitandogl d’ edereimpiegato , ¢ d” abuuart per poter goder€ col
oe
MWAMBRA locanda . Incendiamo reli Alberghi , 0 vero Offerie , che danno, das
 dOrmice a vforetticri .
SERA nce wiare . a era nome eed Havea eletto quel lyogo per’ Abto
Fipotor, exis Wiens t
VOLLE mille Porei . Vole iacpdofiaith di citimonie y¢ lufinghe : ed. ¢ io. neiio hc
‘chevwererderto: itopra’< Com fran Bche Janene , cost dewto dal Latido vente c1oe
di corpo, ¢ gi fl
“WCODAZZO\, Incende feguito di gente “dictto.« Warchi Stor, Fior, lib. I2.Faé
al Primt Cittadini eli fecero codazrodietro , accompagnandolo , ¢raccompagnandolo gaila
we ius Cufanl Palarrxo ; comes’ ei fufféril padrone di Firenze,

      

hat

Ltd fate

oh “WHSPANZA thy STANZA L...,
A cena (perche il giorne in quefto loco dn cambio di guarir dell’ appetica ~~

a: * Lblebbertvairra faccenda le brigate , Facenano un collo come nna. Giz eff

; 8 arta cucinave intorno al foco ) Se vien frictate, og un Sana accinits ,
wt Senses furia ds friteate’, Che per aria chi puofe.la fearaffa ;
od ¢ nem ipresba si 5 ma duran poco , Si riduffero in brene a tal partita.,
3 \Che-uppena farte ellveran gid ingoiate, C’ ogms volta faceanoa rufa raffac,
a Presse gente a rauolaera molta’, tn ultimo feguendo Bertinella . >»
gi sR, We" miangiawan dueye tre per wolta, L! andanano @ cauar.dela padella.
gf oWDelerivetarcena fatta'da’ Bertinella a i Foreftieriy la. aleconfiflettga, in,
pt fritcate » mangiate con fa fiiria , che egli dice: paflo Reale , e cirimonie conue-
if fe a una Regina di Malmantile.

iin fueria di fritrare , Beitvate in quantita ; 3 Waa gran quantica di Fricta.
fopra C, 3. ft. 50. EXIT:

eet

  

    
   
    
  

   
 
 
   
 

.

448 aan F EDR
PRITT APA SEE viv eda! factard WOVa bE
felid'pddella’ asfoge ia aveortah,ielde mene a
125 appreffo ‘atirort baslerebe dine 5 petcheirgioy
{ce Sen eal : as tra ng “
GIRAFF-A, ‘Avimale quadeupede § ikqualess fe bene
fidema ,€ s citaaaiea Dencaaneg eine toy -havil €onouid
a’quello del’ Cammello'ylégambe'dinanai abo i quelled
coda j ed é del colore meuctia® ,- che q
i Latini lo dicono’ Camelopardalis y cio’ bela Yeheticne! I
‘Pantera, Pannoil-coo comenine ewafnd inwndealiangas Lio eel
interpretare 5 che non’ fifazialleroy" perchemméareare | dial
cibo con gran‘deiiderio’;Latino-¥ehiare } 0) chesaliuagatiero ene
betas

 
  
  
 
  
    
 
 

a

 
 
  

    
 
 
     
  
 
  
   
     
 

pet vedere donde, ¢ quandowenivanolle Feiecace ena
refize'a tempo tuo fa menzione'ibPolizignd-nelie® pellance 5 » Gitiog

 

Scaligero' fimil dit quetto :
ail Efercitazionie 209. nutn 3s OVedice "hei Perfiani Girmafa P. f
E Abts il BOM Gina Parse Omit » Ha" eH) bho mee
o STAPA accinite, Sravarateetito'y Teo} oprepataco sidal Laci
-didiatho ftavalattento’, <u’all’ ordiné cones, tnleMtaro.chigmaw,

bo tifato i ahtivo's particolarmenté dation Villanty*s fempre in

fpele fei ptovvedere'danatir: "Ora /peritintratctared! Origine softy
nendofi il danaro a fructo , la Corte’ prititipale 9 ficcome da"Greciy dalla
detta Capo scost-da nO1'fi ehiamo Capitalc ; ¢ Fondo! ancora, dai tei idére.y 6
la petunia data a intereties a:penfa’ di fondo 5 e»pedere!, orpotieth
ta’; Che'pérd:' nftra y come geactata dai danaro \y-the! ayprincipi
Greti Chiamarono Torr stioeParre , 1 Latiniyemes siqua nig
fu Ud Varrdne’, ¢ da Norio Marctlio Oticrpabome apiraies p
pofito'; ff diffe Ia forte’ pquafipecinia capitales principal ndan
che'da quelta pechaiirpolta 12/a%phincipio s hevenivd poirdngu

da’ Holtri anticht Crvaren, voce che finulmentestrovatiun Gio «
la) éhé i Franzefi didero chewanee,'cioe rendita envratayda Chef, capo.Ora
cinire , che anche dillero , Cixamgare, ¢ lo fleflo , che Provvedere ti
<cidé & chiefact , aflegnar fondi's*¢ ludghi da rifchotere ; foraire ye:

 
 
  

See. > ea ae PERFEPE RSET ERE RRR ES

    

rnito , nogeiLefto » sircensp
. oP OPP Dee oes ai
y uP Via Con firia , come fi-fardellescara

atrornd Peiitrelehh Voce alle vdice ufacar; enn Jaycredo

i rofto ‘fied’ per bi ‘iaS* a 1a asad
Pi Tn ape. Si dice’ ido fono pili gente d’ act
Gialcuno # affatina con preftezza’s € (eti2"Ordine 5 O-regola dip
‘egli pud'dic Shae Sad repair med, toa? inciutlese

i ¢ da notare ’Poeta | ‘ i
Pin pane fopraveiene fiipro
fritearemifttvie? dalle macenier Unica feu

 
         
   
  

ve

 
 
   
      
    
   
  
   
  

|

!
!

 

 

Stanchi di mangiar , non {azz}

Finito

BPA ficsass-

 

STANZAL
‘at anna
Tal mufica fini po poi in quel fondo ;
Ma perché dopo cena sl vin lauora
Facean parzie le ‘ior del mondo y

| Fra’ akre Bertinella , e Celidora
inganancieree per burla un bale tando ,
_ Eapooa

4 0. entrouni altra brigata
Tal che fi fece poi veglia formata.

sien STANZA LIL

‘Fano poi com’ é  ufanka
Moite candele intorno alla muraglia ,

_ Lefplendor delle quali in quella franca
E sale, e tanto , chelagente abbaglia,

+ he diffinte fi vedeva in danza

bt meglio capriole intreccia , ¢ taglia
Wannaccio in tanto [opr’ alla fpinetta 2
S’ era mefioa xappar la Spagnoletta .

NONO CANTARE.

Z rel taano gestive insane difcadess lnnetira nazioneda
orit quali dicono,che i Fiorentini fanno je frittate d’un’ uova !'una per rilparmiare;
_ & perd dices che durano poco , ¢ per quefto ce ne vogliono molte pi + fi che per
sta ragione non é vero , che fi facciano fortili per rifparmiare , eflendo certo,
he tanto. 3¢ tanto unto fi con/uma a far’ una frittata d’un’ uovo [olo,quan-
wm to a farne una-di {ci ; onde fi viene a confumare cinque volte pill , perch¢ unas
- fristata di {ei uova faziera tre perfone , ¢ fet frittate d’ un’ uovo |’ una.non {azic-
un’ huomo folo. Si che non di fordidi , ma di ghiotti in quefto partico-
potion effer tatiati i Fiorentini , che fanno ie frittate di poche uova |’ una ,
inché fieno pid cotte , ¢ pid guftofe . Di quefta verita fi puo chiarire, chi non
erede , con fare a quattro perfone due frittate di fei uova I’ una , ¢ vedra , che
eranno fatica a finirle » come le finiranno ben prefto quattr’ altri , a’quait fa
dieno dicci anche di due uova I’ una , purché ben cotte , ¢ quetti fi ridurrando
a rufa raffa , ed a rubarle anche dalla padella , come facevano coioro di
tile, Raffa raffa & lo fteflo , che il Latino rape , rape , dal Latino rapere ,
 fifece rabare , ¢ fi poté ancora formare , rappare , come il Boccaccio in una {ua
‘manolcritta da fugam arripere , formd Arrapare , © dillero la fuga.
r « Leppare , voce della lingua furbefca puo venire di qui, o pil toflo da
vare , fignificando portar via con preftezza, La figura é ja medefima , comes
Tose dice Prometter Roma , ¢ toma , per avvcatura dallo Spag. tomar; quali;
E piglia , ch' 10 Ja fo gia ua, ¢ tela dd. Tre agiole ,¢ barugule. L. naga, varie,

mgé. Daa rufa é facto gure ; {compigliare .

 

449
ei detrat-

STANZA LIID

Vn gobbo [no compagno wn tal delfino
C’ alle borfe. pit rofto , che nel mare
Tempesta induce ; prefe un violino,
Che fonando parea pien di zanzare ,
Intanto un ben dipinto mefolina
Si porge in mano a quei ch'ha dainitare,
Et Ygnanefe , al quale il balle tocca
Sciorina a Kertinella in fulle nocca,

STANZA LIV.

2’ grave il colpo ,¢ gingne in modo tale ,
Che quanto piglia tanta pelle sbuccia :
La Danna , bench fentafi far male
Senx' alterarfi in burla fe la fugcia,
No vol parer ma infel’ha poi per male,
E dice l’ orazion della bertuccia
Sorride , ma nel fin par che riefca
tn un rider pit tofto alla Tedefca .

» che ebbero di cenare i Conuitati cominciarono a ballare cosi in burla,

Ma crefcendo i] popolo riufci poi veglia formata . Cost per lo pid fegue fra lay

dalla quale nel tempo di Carnevale , dopo le cene folite farfi

x i, fi da ne i fuoni, ¢ cominciano a ballare fra di loro pa-

Ren, ¢ fenvefi da chi patia per le Se ¢ da i viciui vi concorre altro Boge
it : 1 e

 
 
 
   
     
   
       
   
   
    
  
   
      

ayo MALMANTILE

e fi fa vera veglia di ballo , come fegui fra quefti connitati
quali effendo toccato a fare da mai ‘del batto alla meffola
egli inuité Bertinella , perquotendola co! meffolino
che le sbuccid le nocca , di ché la donni's’adird,fe bea non ta
ballo alla mefola fi coftuma in quefte veglie per introdu
Jo , che é eletto Maeftro rocca con que! meftolino le mania
vita al ballo, e poi tocca Je mani ad alcrertanti huomini, ¢q
vitate vanno a ballare , e nel ballare i] Maeltro da il me!
ella va con effo a toccare tanti huomini , ¢ tante donne, € cost
tri ufano quefto ballo con fare , che il Maeftro tocchi ante:
lato che hanno alquanto fra di loro , vanno fenza meftola a
mini come ¢ folito , ¢ fi feguita fenza adoprar pitt la'mefola’,” Q
fi dice batlo alla meffola , {i ta anche colla pezzuola , 0 Oy
lando fi getta a quello , che fi vuole inuitare , ¢ cost di mano in
chiamato Ballo alla pexruola, 6
ST ANCH di mangiare , non faxxj . Stanchi dal? affaticarfi a maflicar pi
ma non gia fatolli, perch¢ havevano mangiato poca roba. Ll Petrarca nel T
fo d' Amore, nel principio : ; ne
Sranco gia di mirar , non fagio ancora ,
Giuvenale Sat. 4. ragionando di Meffalina moglie di Claudio
Et laffata viris , nondum fatiata receffit.
TAL mifura fini po poi in quel fondo, Alla fine delle fini tal’ opet
nd: Pur una volta fini. Latino ad extremum , tandem , aliquando,
C, 4. ft. 9. in quefto C, ft. 1, alla voce Bordello , € forto C. 10,
ne po pot , ec, Vedi fopra C. 2, ft. 73.

sR SERS TSE RPESEES

=
=

Ba FPR a=

  

   
  
 
     
 

a W

iL vin laxora , 1\ vino opera,fa la fua operazione con dar” alla teflaye ‘
briacare . Del fuo lavoro , € della fua operazione fi pud dire quel che difie} ka
delle pecchie . Ferner opus . i ty

B ALLO tondo , Specie di ballo , che fi fa , pigliando pit perfone per! »
¢ formando cosi di tutti loro un circolo , ch’ é forfe Latino Choreas m
noftri Tofcani detto Carolare . ee Ye

VEGLIA formata. Veglia vera , ¢ folenne con tutte 'le formalita , i 4
Vedi fopra C. 2, ft, 46. dove teoverai Jutrecciare , ¢ tagliar capriole , & ie 4
ft A

23. q
Nunn acco . Quefto fu un tale nominato Giovanni , € fi diceva
cio per la fua (ciattezza , ¢ {penficrataggine [ poicht fo nome &
del vero nome Giovanni ; fopra il qual nome é da vedi tole
della’ Cafa ] ; Quefto infegnava fonare la chitarra 4/ed if
pochifimo come quello , che non haveva cognizione cna della
rd dice epee 4a {pagnoletta ( {pecie di danza ) aflomighando il
cato delle dita in fu lo Arumento , a uno , che zappi: ¢ Spinerra
balo , o Bonaccordo , ,
VN gobbo. Intende il gobbo Trafedi , il quale faceva p

violino , ma fonava affai male , ¢ per quefto iI Poeta dice: ch
@i xanxare , aflomigliando il fonar di lui al ronzare delle

   
    
  

   
    

   
 
 
  

 

d NONO CANTARE,
‘It! mipiccoli alati , co acutidimo pungiglione , Quefto Gobbo ferul alla Sere-
oleemmt aioe. quaita di Nano , ¢ per le fue facete manicre piacque
" salia Serentis, Arciduchelia Anna d’ Auftria, chg o condufle con {e, quando an-
do dove entro tanto in grazia al Serenils, Arciduca Ferdinando Car-
Jodi lei marico , che  arricchi non folo con li fuoi gro fipendj , € molto piit
con I regaii', ma ancora con 4 denari , che quefto generofo Principe fi lafciava.
da efio nel Bins delle mane » nel quale il Trafedi era aftutidimo , ¢ face-
_ ‘Ya grofle,potte , perche fapeva , che perdendo $, A, S.non voleva eller pagata ,
lige fe vinceva cra pagato puuwwalmente . E per quefto il Poeta dice , che ip un di
Wh quei Delfini , che'predicono rempesta alse borfe , come vogliono ; che il pelce Delfino
ica Ja tempelta nel Mare , ¢ perche quelto pefce pare , che fia gobbo , perd
i ) per coltyine chiamar Lojfini , + gobbi, Mori poi quefto Trafedi , ¢ 1a-
jit {cid mece.ie fue faculta a una donna di camera della Serenifs. Arciducheffa , della
Co qual donna haveva tatco {cmpre¢ da innamorato , con patto , che fi maritafle con
un Fiorentino {uo amuco , che era in Infprug , come fegui.
1 MESTOLINO , Cucchiaio di jegno per ufo di cucina: Diminutivo di 4zefo-
#4, la quale in Lombardia chiamano 44¢/cola , dal mefcolare ,
Ada inuitare . k4a da chiamare ai bailo , ‘
—— SCWWRINA , Chog batte gagliardamente, Il proprio di /ciorinare & quando fi
get ort > abit: di paano fuori delle caffe ne i tempi di State, ¢ fi diften-
_ dono per targlt pigliac aria, batcendogii con (curifci ,( che dichiamo camari dal
pot Greev camaces) donde feamarare fi dice quelto battere , per cavargli la poluere ,
ft © Per liberacgis dalle cigauole - E da queito fcamatare , o perquotere j panni, ec.
igel Pighamo il verbo fciorinaré per perquotere , E /ciorinarf? intendiamo uno, che per
_ A gran caldo Gi leyi gli abiti daddotia; Dal Latino ara detta poi ora coll’ o lar-
f £9 , quale Gi fence , quando.ia plebe de’ ragazzi con {ua antica canzone grida al-
sath Ie matchere u carnovale efiora Ter, in Adelph, Accipiunds , © muffitanda in iyria
adalefcentium eff . L’ huomo fe ladeve fucciare. Quivi Donato, Adafitare enim,
pe
4

4

 
   
   
  
     
   
 
 
 
 
 
 

Proprit'ef? difimulandi canfatacere. E Sopra. eHufficanda; Patienda , confideranda
cum filentio, Gc, ¢ dal {ao diminutivo non ufato orina , cioe auretra , ne riufei il
verbo Sciorinarfi, che ¢ lo ftelio , che fe dicetle ,con Latino barbaro , ¢ ridico-
fo exawrinare. Netia Valdiaicvole dicono ; /cfobacare quando exopacare , cavares
i day’. opaco, ;
1N buria fe la fuccia, La comporta come fatta in ifcherzo ; dal fucciare-, che
"| fifa, quando G feate grave dolore ; tirando a fe il fiato,
| NeMivuel parere , mat’ ba poi per male. Non vorrebbe , ch’ e' fi conofceffe ;
mane ha veramente havuato diigulto. Virg. premit alcum corde dolorem ,
DICE Porazione della bertuccia, Dice de] male borbottando, o brottolando
fotto voce, e cost facendo con Ja bocca quei getti , che fa la bertnceia , 0 fcimmia,
“quando@in rabbia , che pare , che elJa borbouti , ¢ difcorra dentxo a i denti; che
-diciamo comunemente , che ella dica orazioni . — ;
| RISO alla Tedefca , Rifus fardonicus . Kifo finto ,¢ che par pid tofto pianto.
In lingua Tedefca ridere fi dice Jache ; ond’ io credo , che il noflro Autore, che
“haveva qualche cognizione di quella lingua per effere ftato alquanto tempo ia Ia-
fprug » habbia detto ri/o alla Tede/ca ° aon perché Bertinella ridetie , come fanno
12 i Te-

 

    
 
  

 

 
 
   
   
   
   
  

fT¢
pero pla
argla
Fase bine ) Che fiand' fimilt 1
meazione.
STANZA’ ae ai
Al Det veramebie pare Bratig? 2G Ya beffii ?
vse "babii bya A onde or Bhreded 00 ee arvlnesy z

Perch gii par a’ baverle dato prano , Ci morde in quaiche part
ernei d baverla tocca a malo envoy “" Ech) fe
Ma quando fanguinar vedde la mano,” “ts ee
Io mi difdico , diffe , ¢ me ne penta y?\ "Faia
Finalmente to ho tl diauol nelle braccia, uel mefPolino
E [ono ,¢ faro fempre una beftiacci@ ba vette!

STANZA LVR?
Wer carargliene pena, é ‘Biri
nonfacome,al paren h
Dror

  

Sl ¢ WSRoERRER

    
 
  
     
    
     
   
     
  
   
   

 
 

ae
Rin arap in Canberit ih fablerro 2 « © 1009 Syaadermapuoraee:
2” aaanai pit TPinig ane Se raceme ie Casaliadonma,

: STANZA LVIL : STANZA Lk
He Principe'a quel oriad ) Wigule? emairep ‘LLG ridsa\ Dortma ator come’
3 ‘a foggiradrd ¥2 Wictrtdto here’, 192% «Bldipolaize mance,

= call tro 'du hhh VeAbite dhe JOU 2 « 28 IRE feowe/l ah aldorne gal
Co amore in tui vuol far le fue vendette, —- OUR GNMEGC cated fareRiifwe

Ui quel vive fhiattin combean picébio’, "0 « Ds iene yx

erkriwet:

     

ess

“CG abiriih aiiplerofkdW maiekirE\2) 6 OG RLU enareeinura pei

I! mefolina 5 © quei , che glienc dette Di non moftrar in ranto 8
«NB per BP laa bdr qa Ol 19 LY pPERRE OS Gece vel wsuforat medics
SO Po igeres ia terrain Cente rnild pores |. 129 UD anguenre che teyfan

   

“0: Bj doe 'G mara vighias che ta Donna*faccit st gran/laniento pparéndyy
Ofporer haverle® ad maida (anpucuccortifi,, che ib male
2 Fe G7 QUEP EY pli HOU eredcvaYy-ripreh de fe Reffo |) 2 fi metre ivo!
WY CON Medica Me HE biedtlediAratite fi feuop! namiorato 0
OCP ARS, enande torino.’ Rifentirts'¢ dolerf tanta) 1c! ol amie le
>. Ona ABD Read, “ANfatiday AppehalNon giipard’ haverla quai
Sreata ye da Stevicate; ¢ Srénravee dal Liarino fettentarey come owii ¢
z ati Cie, Fi adPAtcic, ALE wir msiferdque fuftento..10 MeHor} cioe paul
yea thiala pera mi Condued's @ ati'reggo’.° Non folametite dicta
Sich ea oial caeueanae ” | @ mala faricay
C'ferto'y Batinovint' ech ytenid} catkanten\ B fitcometi dies 464
Bebe » cio’ grandifima . Ho auuta una buona malartiey
1
Ou s

ee ae on ane:

imal¥orza‘; pochiffimo,.»vsn wed wy
  

 
 

’

ere as
icp. divenfamnenne 4 ne

races ne aa Ow at

 
  
 
  

sor we! iy 51
nym ORG Shots ovo

vale CMOS en no Mle ghinibizzef .
a RS nme S'S on ioeaeemaneenenie gee

 

 

wh ib sen oh
+ CANom artes adenine
a “od medefimo in lode dellt,Vimor.malancolico,..,. A drow Seqeeay Bw '9 {ConUL AE

611 » Bvan fuggendo ogns altra compagnia, ) 2 ASWA bE
aaa SOL op Ae Cd ghizibizein 98 concerti y ¢ 4 CARTE 5 96 sui grktis WY
yo viens ©, Lheecompagman pur fempre.vada.. a8 iA sce s\c08 DAN
j Story Bior-liba.t5e.dige < acca, LAA AER, Seonpes ghicibiczands
Plow dy Dihoeae’D sua | ee
@bArvcca il-ticchio,.. Giiwien, quefta volonts ».pen eG @ gape 0 afoul dal
« Branzcle,%ix.,. mofca caninay;,Sumili , ma Aatnabate Penie, bvalilla.s.¢ Al
adaliailillo.»che ¢,una-molca pungentitima » che infelta 4 i da noi ia
i coenaadad pacerba faransy quo tora aermorndeyert peta NB
mda S 4 APS AS
ae Relacanantciocanstiye! dolore. che; prova uo pazeiente,, quap-
See una fericafirmettelale, accto , o,altra.cola, fimile.4.che, moktihica , e>
Corrodede. partielle de’iquali carpi acri , ¢ mordags fembragg..al, ate a
_Buila difrecoje feriicanms ©PRAZRHO. — .sisshwo9 90. 9\ml)o 5
RAR ua tira 4 an Suniendettar uo mal Ceri: c0 a che a iactia a
UNOswiaw A oie ar re we-sfhow vow wh srsivsusily stls,, in
h\STAACCLA comme 4 pleebiawsE grand a.collera Bg i
‘ sofehiacciare Ggnificabatiene identi per 1a collera, —- per a ae ; ed ha
piquetto: Gignificato fenz! aggivagerus come vom picrhio ma,tal Gmilitnding s ay
quefto. uccelioiha propriesa: naturale dij batter, fceau cere
rofted.in fu sramiideg|t aibert per ; fueginsdefarmu ¢ sliggual ce
concbellittima giz »che;¢ queiasiMope haysr, molgo., eet 2
'¢ ville ufcir le formiche fi diflende some morte fopra, quel amo,» €, Ca’
ladingua g,che éJunga 5 ¢.carnola.,¢ quella.diftends opera il, medefimo a an 2c
ose formighe, vi vanao {opra.per.palcerti, ¢ quando.al Picchio, pare di haveruenes
——— abaftanea), ura ate taolinguayeddngoia , aDa.quett ‘0 uccelio deco in
» Gea Oryscalaptes » 0198: Pictinatere di quencery € InnLaty pics li .¢.formaso ,probabil-
ovamente il, verbo Picchiare,cioe. batierese. chi batted demtlperila ftizea,paresche face
lorfiedle romore,ca.tdenuyche fa ak prcchio cal becco », Plasto spel, pro-
Seeremersaniice srond Sai S108 OH, . ai )
= MANDA git Trinigante ,eAacomisto,. Beftepymia >maledice tua tal Be,

WAKA

 

=

adil SeEEELEe dpe

 

   
    
  
  
 
  
  
 
   
    
  
   
       

a

454 MALAEAN TILE Oe
¢ fuoi falfi Profeti ¥ pn eee
colle maladizioni , coprecesteats ¢ beftemmie oe
GV AIRE, RawmaricarS , eoeee aie: i's
gagnolare . Vedi fopra-C, 4. flan. vventura da wagire ee
guaina ; perché i cani quando ne ae tocche,fanno um mug
gito de’ bambitii’. ‘Si pud anche dire , che venga da #1 ase i
rammaricarff dell’ huomo. 1 wales Now, 2 bn R
comincia # ffridere , ¢ Puaire., a he wl
METTE 4 foqquadro , Solleva , ¢ mette axofgr tutti i vi
re, Soqquadro & voce ufata dat muratori’y eee »¢ fimili, ¢ v1
{quadro , che ¢ quando per accidente d*
mancamento un pelo tirato , 0 ftrafeiaatonon pud fare” ib {uo corlo,
rd cagiona , che git fteomenti del veicolo , o treno facciano fi ito at
per lo sforzo , ed affaticamento yche riceyono, eda Yale
drare , ¢ mettere a fogquadro iv vecedi Rordirecobromorey) &
/MBLETOLIRE , Commuoyerti } Intensrire « Vedi fopra C.
tini pure in vece di /anguere , dicevano'volzarmente ne! sane
eficr cénero., e mofcio , pigliando Ja fimilitudine das real ¢
fignitica erbageio 20 ortaggio; Auguito Imperadore formé una 5
rola, e dilie Serizare pigiiando ia fimiludine dalle bietule , ~per vi
languids’; non iftar bene. Vedi Suetonio'nella Vita d Augulto » Ove:
voci,¢ maniere particolari , che quetto Principe ulavaynel. par
Celio Rodigino lib. 15. c10. Now fimilmente, diciamo! fauna
fi, illanguidirfi per il aah d'amore, B Bretolone pre a hu
mii fatta ;
BESTIA fcimunita , Spend {propofitato fenza jmenlitnaiasiiya -
zio affatto. Lifca Nov. 2, dts perche. ellaera ponera, a quefte fe
torre fenza dote , ec, Scimunito ; {ciacco , Scimunito’é lo fteffo che wren
Lat. incaftigarns, Gr. acolafes , che not riceve'lammoniziani ;) €
fictti , monitoribus aff ah E perche quefti , 0 fimili a loro fogiiono eflere
ale il giovane deloritto da Orazio y Sublimis cupidusque ,o amara reli
nix ; E qual’é quei , che difvaol cio , che volle : come difle Dante nfs
ro nell’ Fliade al terzo libto ; Delle giowani genti rigogliofe Sempre per:
tere menti; cioe per dirla volgarmente hanno il ceruello fopra Jab :
@ che Scimwunito’, che di fua natura yale Non ammonito , non riprefo 5
ftigato , o che non vuol eflere amimonito , ne riprefo, ne galtigato; ¢
rio , € mentecatti fanno ; venga’ a ‘figniticare /eiocco , e haomo dt
to, L’ efempio del Bocce. nel Filocolo lib, 4. dove: parlando come
Il tno diletto ¢ dimorar ne! vani occhi delle foimunte femmine , pwd elle
voglia dire ancora licenziofe , immodette , intemperanti, ¢ non
ze folamente ,
RAGNATELO. Ragno , infetto noto , Dicono che perm
dej cane fi pigiia del (uo pelo , ¢ fiipone fopr' alla parte offela ,
HO fopra C.6 flan, 6.¢ che il ragno, ¢ 'o fcorpione aumpa
foper a Ja piaga che hahao faita coi loro morfo,{uaino il pazziene

   

Page

eFEEEES

 REEES

 
   
      
     
  
    
    
    
  
  
    
   

ea

   

Se Ss See Se
 
 

‘*
NONO CANTARE: 455
necredendo chest pezzi delmeftolino, babbiano la ftefla virta ; lega fopralia fe-
rita,che ha fatta col meftolino a-Bertinella, idetti pezzi Maforle Baldone:, co-
me Soldato bravo » haveva notizia della jancia,con la quale Achille feci Telefo ,
ee nea sehen havea detto J’ Oracolo, i, Qua
. iabit medebirur, Donde Dante afer. C, 31, difle: , ‘
lo) loi Cosnod! toche foiena la lancia y

he 14 5 0D! Achille y¢ del fu padre effer cagione

tHe Prima di trifta ,e poi di buona mancia,
| -\Biérede ; che il meftolino babbia la medefima virti della detta lancia .

>

Buk

qt ALAN del Cielo » Quali che Adanna def Cielo, ¢s' intende orto rimedio per
at fanar male ,»come fu ottimo rimedia per liberar.daila fame 11 popojo eleno
wiytt inane che. Dio git mando nel deferto.. diFirenzuola in lode del iegno fante
io, 3 > <oSy
fe) sbiaib shoizwe2 S& uno'non mangia , s' un non fi ripofa,
lags i Osha il fegato guafto , ole budella ,
Rab > Bgli é a man del Crelo a ogni cofa .

** Nota!che in:quefto detto la parola:4¢an:non vuol dir mano , non, effendo pa-
Ola figurata‘per apocope , ma nell*intera fua efieaza Adem , che cost fi trovan
feritta nelSacro Tefto quella , che Dio mando al (uo.Popolo (che noi poi chia~
jamO manna )¢tal man fi dice nelia Sapienza al capo 16. che havetle ogni buon
x vien chiamata quivi Paze approncato , e appreftato dal Cielo fenya fatica
© pero iniqucito detto credo che fr debba intender e#Zanna y ¢ non mano per fi-
fe uba cofa ottima in ogni gencre ; e-che cid fia vero y quando fopravvie-
he a*yao® qualcofa di fuo gulto,{uoi dire: #’ wa manna ,e non mano : ¢ fe uno
ricercaté | fe per un {u6 conuito una tal vivanda gli piacera prifponde fard Atan-
adScome fi Vede 'fopra G, 8. ftan. 43. Se bene potrebbe anche dirfi», che collas
feta parola Gi aljndetle a due fignincati , ¢ a quello che ora di fopra fi é detto ,
WMtan ; cioe manna , ¢ dian , cioe mano, E ALano de! cielo potrebbe parer det-
ta'Colla medcfima forma’, con cui diciamo di qualche rimedio., o medicamenio
cfitace Kyi ¢ (Paro Ja man di Dio, il che coceifponde a cid»; che dice Piutarco
fOnumM Conuiuialiam lib. 4. quacit.1.)cheun certo Filone medico,aicuni me-
‘Witdinenti Reali, cosi decti perché erano da Re ) enon da Poveri , 0 per efiere
i*! fepreti di Ré jo per 1a loro eccellenza ; ¢ che dal (occor(o potente , che fe ne ri-
; ceveva y erano-chiainati /exipbarmaca , appclld com-particolare.appellaziones
mani degl Idaij .
jd) WPREGLAT-A, ¢ neva, Intrifa , {porcata , tinta , Da i venti,che portanan via le
i mmelecine Bal gran vento , che per le parti da baflo gli ufciva dal corpo accom-
ip ‘pagnato da qualche altra cofa ; la-quale ricoprendo le'mele che fono quella par-
ce pitt eafnola delle:cofce,che forma il federe } ” alconde alia vita © costin un,
w? Cert modo fe’ porta via ; Si che il Pocta Meoppiando quel verlo 5 che dice . «Dai
md ‘venti, che Portanan via le vele , intende , che la Camicia di Baidoneera tinta dallo
z)
6
é

RELILE wtuateale

‘fterco ’ :
SQVADERNA fuori , Cava fuori de i calzoni,¢ la diftende . Morg. Le chiap-
/quaderno con rinerenza, Dante Par. 33. Cio che per 0 uninerfo fi /quaderna.

! tele , cid che ¢ {ciolto , ¢ {(parfo per I’ univerfo , prendendo Ja fimulicudine da’

J libri fciolti , ¢ {quadernati. DR

  
 
   
 
   
     
 
 
  
 
  
   
 
   
   
      
 

‘436 MALMANTILE |

DIRGLI manco che meffere , ec. Dirgli iurie
tia differo i Lat, ed il Lalli Bitar kon by eHloee Lich
é Teitt m' ba detco peggio che meffere. 6)
Molti dicono + Ase/sere él’ afino : ond’ io ftimo:che dic
che meficre s' intenda , I’ ingiurid pit che fe gli havefle
Comico Fiorentino nella Moglie Ato 4. {c, 10. in-derifione del t
dice : Si; Adefsere ¢  safine , che va nel mezzo. Quali dica :
quando paffa per le ftrade gli fa largo, eva nel mezzo,
BEL vedere , 1 bel di Roma ¥' intende it Colofico ycheinoi
Ciamo Culilco ; eda quefto per belmadere y Obel di Kama iptei
che Bertinella pericolava di moftrare alzando le gambe .
Bellofguardo , {on nomi di juoghi , ¢ ville nobilidime nel Fic
vato , ¢ donde fi {corge molto , ¢ bel paefe.
eHEDICO da fucciole , Medico {propofitato ,e dipoca:fcienza ,
mo i marroni cotti col gufcio nell’ acqua , ¢ preadono tal nome dal /ucciare 5§
fanno i ragazzi per trarne {enza aprir wutto 1 gu(cio , la pafta , che vi & dent
E perché quefto cibo ¢ vilifimo ; pero foros iamo da fi i
nulla. I Latini diflero bomo manct cioé di niua pregio ,
fico ; per Naucum intendendo il Gufer , o buccia di quaifivoglia cof
la , che fi bucta via , ¢ aon buona a aulla,
LE fa veder le luccicle, Le fa pianger per il dolore , Quando uno}
tale , che gli muova. le lagrime , pare al pazziente di veder per ari
14 di minutiffime ftelle , fimili alle lucciole , il che ¢ cagionato dall’

lagrime , ¢ che pafiando fopra alle pupille offende , ed altera Ja virth v vas
Oe STANZA LXL STANZA LXUL,

Non dimoftra la taccia cost mefta S’ impiccherebbe , ma dall' altro:
Quel ragagxo fcolar,quel cauczzmola y Ei va pos retinente ,e¢

Allor che motti giorni e (ato fefta
E che finita poi quella vignuala ,
Ji mataderto tempo ecco s' apprefta ,
Ch' e's ba di nuouo atornar alla/quola,
We fi gualta belando si la bocca
uuand il matftro col bajion to chiecca, Gli vada in (u le forche
STANZA LXiL STANZA Lxl
Qrante cambiate in vifo ,¢ mal contento, Poiche '1 cundotto delle pat
Adefo pare il pouero Baldone , S' ha da ferrar(dic’ egli,
\ © ba nna ftizna ,ch'ei fi rode drento, Perche fi ia leva alle fue.
Per non bauer ceruel , ne difcrizione ,
Che benc' altrni la morte dia [pauento,
Se e' non fuffe che e'c' ¢ condenvariong
Achis ammarza pena della vita,
Con una fune baurebbela finita ,

Con quella mane’ alei dif

 
 
  
  

un mutmameess ofa. sealers a2 ESEG~0FE

oe Se
 

  

pas sro LIDUE

 

8 ois! me nyo 34 fan ne‘iipauni,
Ps em wre - dif oe vs onngoy eRereheymensre th’ I ami, ella v anuifea
‘chap yh Ob bu) Chomsany, u lite
a ye fooppia dali th Gi Sent habbia on acgquauite .
intovaci) Poetara Sabeshens cepaeha cheba baleen
ps9 sorim re eran moped Da quefio a srcorgentof Bering
di.lei 3-4 h

 

a iste ine goticiod fi sbaseni tmodh 1b i od LE": sy sais: Nas. aoe}
ianbopkematendonstetniens faney Olaltea forta di)logame, con
eaeioees ed<alere:beftiefimili + Evcaves.t; filidice ancora,

fa merce) collooa: malfastori pquandog)' i =
Oy: 6. Saatgo! aah Eda emadl noiidiciamora un rapazzo-malig

    

 lioiywenn\ LiVai facendorparlare tn Pedantésdice 2a (09 11309 isos to att?
doped d jwoda, Hee, Seana & osjuy 1k1gGs esm2) 901813 Iq issagayi Onnst
ab omtetharion Seieababtanitioher ants omiiiiv $ odio olaup dioieg ¥
of Mey aba so, O folerro-vrifar ciferory PO Over owsd Orstid imaed bellow
OR iiteade 4a sesete ft O, eyed |i ccaaiate marae) 19g ¢ Colt
’ A quella id nines ehocien en quel-pdatemps,
itp aeaast iid? eavpolonannavOaiee:
ivcen se sar bebe vignnola’, P ela dnrage pes inten:
credo che fia pata He Tulnadaeeibe deseo opera

mad C.9, nf eke an la) Petite Tn: Eictirhnadutemne ‘a a’ Buohs

cL aa

   
 
   
   

uh fa gia‘an val Sse da ‘Panzano , ib quale havendo din’ fola pic
ue co) Poe facevz a ex faye barilittivino ; ede ae,
on rei hi achiorbailt , ed haVeva 2*OB Hi Torte frattesyichie tro -
7 AO’ al aon eglnop it fi meow rubsiidoltuva,

mee ore Ha" mvae ¢ fempre dieeva";' ‘Chie’ratdoplievaer bani bofa
nefla ; rem POscoPLe®, thie per fuor bifog ni" tai vende iadetenvigng yes
fre do pia Fitoperta della derta'vight’, HoH potevarabare ‘come
ee: or Salis fo S*aFri(chiava a imbdttare ance witio's pet fo” ches
dom abdate alii fo? amici ‘da CHe*procedeva’ phe eel" err ‘vino 5
edali Prifporidéva , che era fina la vignuota ae a teiccoe dice: il
pad eifer che” venga it i decacos ee ae a) “wip reheas
ov mang an =q oon

rig Pgh Syeepoda balie Roca Pane ©.°6. fan.
nares @¢ fo fteffd', titi duc verb} ftei'dal Tao's IP BalerNov, 7,
ane A ractomandana ' a phi poteres; e coloré antendtnand a vbinctirts ‘chi di

pinseebed ls sha olehnon aol,
te wna fine baurebbela fin ita: Haurebbe fiaito ee ‘fub" CRN aziO’ ton-im-

TANTO , 0 quanto, Termine , che fignifica piccola quantita , ed 2 lo feffo .
che par un poco; alquanto, Petrarca. E tu , fe taxto, 0 quanto.d' Amor fenti ,

4 un fopractieni’, Parca waa folpenfione , un preety di foptatrenere ;
‘ ‘Profiiagato il termine . Mm coNn-

bie —

 

 
  
  

 
 
  
 
 

   

453 MALMANTILE >

CONDOTTO delle pappardelle\, Cioe la cannaidella gola ,
del cibo detto da’ Greci Ocfophages , ¢ da noi {cherzofamente él condotto de! bo
che rifponde alla parola Greea fignificante il porta cibo, o i} Port i
piglia pappardelle , che {ono lafagne corte nel brodo di carne pet ogni cibo,
ti chiamano pappardelle la ricotta ftempcrata con acqua rola), eu ova 5 a
¢€ poi fritta a toggia di frittelle .

TLR AR le quoia , Signitica morire , come dicemmo LC. 4. 20,
{cherza , moitrando , che pet la legge del Taglione fi gattigar le gu
( civé la pelle ) dei Duca per haver egli commeffo un delitto nel
nella , rompendogli quella della mano , € feguita lo (cherzo dicendo , «
morire in /« tre degus  [ che vuol dire in {ule forche ] perché con un
col meftolino J fece la decta ferita nella mano di Bertinella ; ¢ di pil
Ballerino a vento (che vuol dire ballerino da qulla ) per moftrare:
egli commefio ' errore bailando , farebbe gaftigaco con effer fatto mori
do, come pare che muoia colui che ¢ impiceato , Vedi fopra\C, 2, ft
re un ballo in campo aczurro; che ¢ lo fteflo , che Tirar de’ calci a Ronaio ,
vento Borea , 0 Tramontano, Quel che fopra dice : in /u tre legni per i

forche ; é fimile a quel di Plauto , che volemdo intender Far , cioe ladro 5 difles
trinm literarum homo, vel
FACENDO il Nanni. Facendo il goffo. Fingendo di non badare , oofferm
re,Vedi fopra C, 4. ftan, 26. Moftrando di non s’accorger di quel che faceva Bal-
done , facendo le vifte di non vedere. *
SCOPPLA dalle rifa, Ride {ecgolatamente . Vedi C. 3, flan, 66, alla yo
Pimmei , ¢ C, 7. ftan. 66. 0S

 

 
  
     
    
 
   
   
   
  
   
  
 
 
    
   
    
 
 
 
   
     

&
Been SSB SRS Ewe ow o=

PER P ablegrezza non puo fear nei panni. Si rallegra geandemente.
capir nella pelle. Per il gran guito Gi rallegra tanto , che non trova qui
di fopra C, 2. @aa, 69, Piatone acl Carmide , poco dopo al principio , volend
efprimere una gran paiione di piacere , ¢ di gioia fa dire a Socrate, &
pitt in me flefo. i 0 ae

cANDARE in fumo ad acquauite, Rifolucre in nvila . Suanire. Lat, 4
re. Sidice anche in tu:no d’ elisire , Od’ eferuite , fopra C, 3. han. 52.

STANZA LXVL STANZA LXVIL-
Atentre Baldon qual fempluerto uccello , 4a ridan pure , ¢ faccian ci
Coast d! tntorno alla cinetta armeegia Per ch' ci vuol far orecehieds Merci,

 
  
 

Lo burtino te genti , Amor ta,
C” ad ogni mo fard fido ,¢
Come talor 3! abbrucia ico
“i garto al fuoco , ¢ frau

ed tuti guint ferue per zimbello ,
Senzache mai vi badi, o fen’ annecgia
Ogun lo burla, e dice ; Pelle vello;
Crafexn dice la fua ,ciafcun motteggia, , «

 

Beato chi pu bella te la feianta Baldon gia fenve tl fuocose
E pot leuanfi crofci dell ottanta, Aa com un pan di ”
; STANZA LXVIH. 6

ne ot See wa

E cos} wa,per ca principio Amore , Ma nel getrarla allor
Par bella cofa , efembra ginsto ginfto Perché riftringeye rides
Vira pera cotogna , il cui colore y Ecosi Amor, al primae ani
Odor » faper aslesia ,¢ piace al gxfboy C! allerta y ¢ piace 54

     
 
 
 

lal

aa,t
ie
ib,

NONO CANTARE. 459

STANZA LXIX,

Ed agli cht impaniato, € 4 qualche fegno ta lafciamla per hor cbt io'fo diferno,

_ Credeil fuo amor da lei effer gradiro , Che quefho canto refti qui finito ,
 Altero vanne , ¢ fhima a’ effer degno, Perch? dife un Dottor da Paleftrina

——— Diinuidia pitt che d'effer moftro a dito, Breuis oratio penetra in cantina,

. era cosi fit di la , che faceva mille me-
lenfaggini 5 per le quali era da ogauno burlato , ed egli Fingeva di non fe n’ ac-

c » © continovava a fare (cioccherie oftiaato in quell’ Amore , come tal

volta @ un gatto oftinato a ftare intorno al fuoco , ancorché fi feata abbruciare .

4 Poeta adsaugiia Amore alle pere cotogne , le quali dilettano con I’ odore , col

colore , ¢daano gufto nel mangiarle , ma fi dura poi fatica a digerirle , € diven-

do che Baldone fi reputava pitt degno d’ effer inuidiato , che compatito , termina

il nono Cantare .
| CWETT A, Vedi fopra in quefto C. ftan. 22.

SERVE per zimbello. Seruc per {cherzo di tutti. O pure per allettatore degli
altriamanti a venire ad amar la fia Dama . Ii Malatefti parlando in perfona d
un villano mandato d’ oggi in domani , ¢ burlato dalla fua Dama , diffe ;
‘ Da poi ch io bo fernito per zimbello ,

E fon andato trenta mefi aiont

Gridando per la rabbia’, e pel ronello
vibr od Come fa il gatte quando ha i pedignoni
ud id « Alla mia Betta ho pur dato? anello , ec, 7
DICE : vello vello, Termine, che fenifica Derifione’, quafi dica ; guarda_, >
guarda lo feiocco , il pazzo,o fimili , ed ¢ lo fteflo che Effer moffrato a dito per de-
rifione , che vedremo apprefio nell’ ottava 69. ¢ che far lima lima dittro a uno vi-
fto fopra C. 3. ftan. 37.

MOTT EGGIARE . Burlare , o beffare copertamente uno con detti acuti, e+
mordaci . 1 Greci di C diare uno ; noi p biarlo 5 egiarlo, Da
motto , parla; che fi piglia anche dagli antichi per fentenza , 0 concetto , o det-
to intero ; B Azorsetto , cine breve detto, e fentenziofo , come fon quelli intitolati
Motterti ne’ documenti d’ amore di mefler Francefco da Barberino . Asutire, loqui
difle Fefto , foggiugnendo 1’ autorita d’ Ennio nel Drama intitolato Telefo . 2a.
am missive piebero piaculum eft, EB ttimato un delitto a ud plebeo il far motto,cioe
aprir bocea , ¢ parlare : onde Azertegesare non é altro , che parlare con qualche.
bel dettoy @acuto . Dal Greco Azythos viene il Latino murire , €'| noltro Adorze,
Ui Cafa)perd nel Galateo col definire i Motti /pectal pronrexza , e leggiadria , ed
oftano movimento a’ animo; pare che in un certo modo lo faccia venire 5 O pures
{cherzaquafi, che venga'da A4oto , movimento .

BEAT O chi pik belo te ta frranta , ‘BE’ lodaco colui , che la dice pid bella in bef-
famento di Baldone; ci feruiamo dell’ cpiteto bearo per felice , avventurato ,
fortunato.,'efimili (come fe ne ferue il Poeta anche fopra C. 1. ft. 29. come nel
prefente:luogo-, cheefprime , Fanno a gara a chi pit bene lo burla : Latino Cer-

 

 

 

sare conuitijs ) Petr. i
NED Beato venir men che 'n lor (Rhos:
Me piit caro il morir , che viner fenxa;
= Mmm 2 Le
& ca

q

 
    
 
 
 
 
 
 
  
 
 
    
  
   
  
   
   
   

460 MALMANT ELE:

LEV AN crofci dell! ottanta., Si ride fmoderatamente. La vt
quel bollore gagiiardo , che fa la pentola , Fee cra 9 Op
€ fi dice crofcrare dal fuono +. ik gal verbo
Dan, Inf. C. 24,

O giuftizia di Dio quant ats
Che carai colpi per venderta crefeia oi

Tl termine dedi! otrawta fignifica {quifitexza., 0 ps
ne logico a to: © forfe daile, ralce {pecie dipannine; le quali
tanta paiole fono a buonidimo grado di perfezione 50 finezza..

Ck ALECC!, 0 cicalices . Dilcorfi faytida pil perfone infieme
priamente dire Difcorfi dell! azioni , ed snteredi altrui con.
di bene : éd intended per Jo pid, Cigalamenti fatti dadonn
digiorni , novellieri ; per quefto quando fi fente Ree nuova
dice € un cicaleccio , 0 una cicalata . >

FARK orceciue dt mercante. Finger di nom afcoltare 2 © nan.
che altri ti difcorra. E propriamente s’ intende far oregchie di mercante coll,
che efiendo richiefto di qualcofa , 0 riprefo d’ 4leun vizio non
richiefte , 0 non fi emenda agli avvertimenti , 0 riprenfioni... Si dice piantare me
vyna lopra C. 7. It. 39. Far conto , chee paffi l. Leperadore . ne to. fi

COSTERECCT, intendi le Coftole : Li coftato..

EVN certo imbroglio, E’ un certo negozio imbrogliato, is difficile , cele
mo anche ana cofa cost fatta , intendendo una cola. che pon ha eo del banat
del giufto , dell’ onefto 0 del fattibile. ons ,

WEL gettarla, Dicono , che la pera cotogna viloinga il venton-a coed stil
mangia , ¢ lo rifecchi rendendolo ftiticho , ¢ perd dive ;Vel.gerranla da dolore se
piu lotto dice ; Nel fine ti vogtio , nello {maitirla si man. ia fuori
mu dica le ti riefce cosi di gufto come pel principio s:cio¢iquando lama

41d impaniato , E’ rimatto prefo alla pania, come rimane-il pettiroflo
do ja Civetta , intende s’ é innamorato 4moris yorte dmplicitus y aK or
parazione , che ha fatta fopra dicendo ,
etientre Baldon qual, femplicerta angela 2

" . Cost d intorna alla Civetta armecoia..
Quando uno ha male grave , da non ne potere ( non iisimene errs
dichiamo ; £g/i ha impaniato , eq o¢ eam

ALTERO vanne , Vedi fopra C. 8. ft. 30, Qui-yuol dire gout,
mando , che quefto amore Jo renda degno d’ eGere inuidiato per haver
bene , come ftima !' amore.di Bertinella , che d’ eles ¢ompatito del
d’ cllerfi innamorato di coftei. B cosi fi da.a,credere digodere ogni
fapendo , che come difle Erodoto nel libro intjtolaca, Talia 5-2 meglio
diato , che compatito ; 1a quale fentenza colle efi parole appunta , a
fa I’ usd Erodoto , dichiamo noi comunemence tutto giorno; E.chee ji: ue
ce Pindaro nella Raccolta morale dello Stobea eHMiglian Minuidiak F ,
le quali fentenze dalla noftra plebe ridotte in una Cantilena Fiores
Cost ¢ sa sincoomate

Meglio ¢ inuidia fop| tare h

Che di fe compajfion dare ,

 
 
 
     
 
 
 
   
 
    
 
 

 

    
    
  
 

    

  
 

NONO CANTARE. 451

__ DOTTOR! di Paleftrina , Se ioffapeti , che Catone haveffe detto . Brevis ora

Caios crederei 5 ‘che volefle dir di lui, perché fu originario di Tufculo ,

di Prafeai »eche havette pigliato Palefrina , cioe l'antico Prenelte per Fra-

7 € S'i0" fapeti » che un montambanco , i] quale fi faceva chiamare il Dotto-
redi Paleftrina , e faceva da Attrologo fufle folito dire tal fentenza, ftimerci,che
ee quefto , Ma intenda di chi egli vuole , bafta che con quefta fencenza
dai opps ha voluto fignificare , che i difeort brevi piacciono inating ai

2 icantinieri , ( perché ne’ fuoi Originali trovo una volta im excints ,

‘ra volta i in cantina ) ed in fultanza intende , che ancora gi’ idioti amano ,e>

ei eee idifcorfi brevi .
fo
ime i

   

  

nt FINE DEL NONO CANTARE.

DECIMO CANTARE,
_Peeabasdlasibabastiasdbarls 8

ARGOMENTO,
Per far la Adaga col Rival quiftione
Va y ma in vederlo pot le {palie volta,
E , con lui dietro ,
Ove ¢ la gente per balare accolta ,
Del Lupo in traccia Paride fi pone,
Ui trova ,e’l prende con induftria molta , we

ugge nel falone,

i E uccifo quel, da fine alf avventura ,

STANZA I.
wanti ci fan , che veftono armatura
0° Dartor di feberme, e ingoiator di fquole
4 ditminedaces » che fanno altrui paura ,
© Premar la Terrase [paventare tl Sole;
' © BE ratcontande ognor qualche branura
f

os

Sempre ogn'un cone parole;
St fda sl cafo di venire all’ ergo ,
Labial om! olia , poi voltano «| tergo.

STANZA

tpien mofirain zucca bauer del Sale,
hb ee jon [fanio fempre fugge ta guiftione,
Anxi veder facendo quanto ei vale
odMebpicare al bifogna di fpadone ,

| Ed wu tal guifaé liberate il Tura,

| pene Reps ep pe geste eer aS

we

STANZA II,

Mae fon da compatir fee fanno errore ,
benché non fembri mancamento quefto ,
Se chi 4 menar le man nonglidailcuore
In quel cambio a menare é piedi é leo,
Ob mi direte: Vanne del tuo bonore

Si, ma un po di vergogna pala prefto,
Helio é dir: Vn Poltron gui fi fugvi,
ee : qui fermofi un bravo,e si mori .

L,

E che ( chi a neffun vorria far male )
Sa ritirarfi dalt’ occaftune y

E Senza pagar tafteso chi lo medichi
La campo,che ai ni re fe Fee

«dh theme

 

i eee}
 
  
  
 
 
  
  
 
  

462 MALMANTILE.

STANZA TV.
Ma voi , che di question fate bottega
Credendo immortalarui; e che vi giova
Far la {pada ogni di com! una fega y imparate
E porni a rifchise far ogni gran prova, eg
Il noftro Poeta volendo deferivere nel prefente Cantare la di
lagrillo a Martinazza , per la paura , ¢ poltroneria della
fegui , s' introduce con dire , che quei Bravazzoni ,ed Amm
pre difcorrono di far riffle , ¢ quiftioni , quando fi vien poi ai
ratamente , ¢ loda il lor penfiero , contiderando , che 1
Ja vita , che far fermo , ed effer’ ammazzato per il vano pretefto di rij
eche non pud effer biafimato colui , che non havendo cuore a menar |
mena in quel. cambio i piedi , ¢ fa intanto un’ azione degna di lode , fug;
male . Conchiude al fine , che tali bravi , che cercano d*immortalarai
ro bravure , ¢ {margiafferie s’ ingannino , perché dopo la lor morte:
ur minima menzione di loro: Git eforta pero ad imparare da i
DOT TORI di {cherme,e Ingoiatori di (quole .. Cioé che fanno da mae!
ma, ¢ che fi prefumono di faper tenere in mano la fpada meglio di chi
da nelle {quole di {cherma . Ma qui {cherzando.con I’equivoco di (quola’
che cofioro fon bravi mangiatori , poiché ingetano /e /axole, che fo
ne fatto di farina mefcolata con anici , ed € chiamato fquola ,
figura d’ uno ftrumento,col quale fi tefe,detto corrottamente /guola
dixs , come vuole il Ferrari ; ed é quella cafletta fatta a foggia di na
ro chiamata anche navicella)entro alla quale s’ adatta il cannello pieno di!
paflarlo a riempier I’ ordito: Si dovrebbe dire (paola , ma I’ ufo ha
ja notizia di tal voce. Dan. Inf. C. 20,
Vedi le triffe , che lafciaron U ago
La (puola ,e il fufo, e fecerfi indovine ,
E nel Purgatorio Can. 31.
E , tirandofi me dietro , fen giva .
Sour! effo ? acgua liene come pola. ?
FANTONSACC!/, Huomaccioni; Huomini di ftatura grande; ma dicendol
Fantonacei §' intende in un certo modo grardi,e poleroni,o difutili . B dict:
Galeonaces , @Uanizoldacei , ec, Omero nell’ Liiade lib, 3. introduce Extore,
del male a Paride {uo fratello. £ tra gli altri mali, che gli dice , unoedi
marlo , Eidos ariffe, cioe un bel fantone , d'ottime fattezze ; 0 come meer
fignificando la bellezza del corpo,difgiunta daila virti dell’ animo;un
wn Dongelione , 0 come dice qui il noitro Poeta ; un Fantonaccio , cio? che!
moftra , ma ¢ poco buono a auila , *
AMMAZLAR con le parole, Legiones difflare fpiritu,come diffe Pl
dato millantatore . Pretender di farfi ftimare , ¢ temere col dilcorrer
ritie , quiftioni , ammazzamenti , ¢ con efercitar fempre con chi fil
arrogante fuperiorita . Di-quetti parla Famiano Strada Jib, 2, Pro
Gloriofi ifti duces. Det homsnumque contemprores , & gut fe atijs faci
Calo minitabundi gre ‘p ATLis, Guam profil d 08

   
   
   

      
    
   
  
   
   
 
  
   
  
 
 
   
   
  

DR Pweg er ge ep roeae : =. wa

Sener

  
 
   
  
  

DJECIMO CANTARE: 493.
tini chiamano milites gloriofos , quefti vantatori poltroni, de i quali intende il Poe.
ta nel prefente luogo,e fe ne dichiara col dire : Se view mas il ca/o di venire all'ergo,
~ ifica , fe vien mai il cafo d’ haver ad adoprar |’ armi , non parlano pili, ¢
fuggono , che € quell’ abijcere Clypexm de i Latini .

VN poco di vergogna paffa prefto. Quel poco di roflore , che fi ha per una cofa
mal fatta fuanifce , efi difperde: Seatenza ulata , ¢ praticata da coloro ,
che fanno poca ftima della riputazione .

(i MEG LIO ¢ dire: Vin Poltron qui fi fuggi , ec. Buona fentenza ,¢ vera , ¢ prati+
jig cata da coloro , ehe bramano pe tofto vivere con poca riputazione , che glorio-

gi famente morite ; il che bene efprime il detto Latino Vir fugiens denuo pugnabit .

m Der , che s'era srmato , ed havea fatto (Crivere nel (uo {cudo a caratteri
iamt d' oro BON FORT VN& vantandofi di voler-far gran bravure , (¢ egli entra
é,g Va in guerra ; quando fi venne al combatterc , buttd via lo (cudo, ¢ fi fuggi, ed
misit a. coloro , che lo taflavano poi di codardo difle: Vir qui fugie , ruxfus redinregra-
nme bit pralinm, indicans ueilins Patria fugere , quam pralio mori, mortuus enim non pi.
sen grat (che noi diciamo: / morti non fan pin guerra; ) at qui falurem quefiuit in fuga,
poet pote/? sm multis pralijs patria u/ui efe. Tuttavia anche appreffo gli Antichi era vitu-
dda Perofo quefto tuggire ; ¢ fi trova , che 1 Lacedemoni bandirono Archiloco {ola-
digi mente , perche havea {critto , che cra meglio abijcere clypeum , quam interire ,
jue a del fale im <ueca, Kaver giudizio. Vedi fopra C. 4. ft. 15. ¢ C. 8. ft,
wi, © CHOCAR di fpadone, Par che voglia dire , che quefto tale fi difenda con gio-
jgad care di {padone a due mani , ma incende , che gioca di spadone a due gambe. ,
yal Slot fugge : motteggiamento ufatiffimo verlo coloro , che fuggono per paura il
ie dite Ginora ben di /padone , ¢ \enza dite a due gambe s’ intende fuggi. Vedi fopras

| C, 7.0.76. Giuocar di {padone fi ula ancora di dire in propofito d’ una cafa, che

fia igauda , ¢ (pogliata di maflerizie ; in quefta manicra . Vi fi puo ginocare di [pa-
done , ciaé Non vi ¢ cola alcuna , che poffa arreftare, 0 impedire quefto efercizio,
che ha bifogno di iuogo largo , ¢ difimbarazzato .
T#aSTE, Vedi fopra C. 1. ft 60. Talte fila , che G mettono nelle ferite, dette
cost dal taflare , che fanno la lunghezza, ¢ larghezza di quelle. Latini panicidi
ai Vulnerary , lineamenta, i
g DAR campo , che fi predichi di ivi, Dac’ occafione , che fi difcorra di lui cons
wm) lode. £1 ver! predicare ufato in quefti termini figaifica Far’ cn:omj, 0 lodare,
| Quand’ uno fa quaiche azione bella , ¢ di cia G pavoneggia, (ogliamo dire in de-
Be 2 Chese ne predich ,
PAR botreca di quiftioni , Viuer di rifle . Haver care le riffe per guadagnares .
E tanto quefto detto quanto far da fpada come una fega, cioe intaccaria nel far qui-
fione , come é intaccata , o denotata una fega ) fono detti deriforj a tali Bravaz-
zoni,¢ Tagliacantoni .
LA morte vi fi piega, Voi morite , ¢ dopo la voftra morte non fi difcorre pit
de! voftri gran fatti , ¢ fi perde la unemoria delle voitre azioni » ¢ vanne del pari

Ja bravura , ¢ la codardia » Quell’ importuno , che per la via facra s’avvid dictro

a Orazio , enon lo voleva lafciare ; domandatorda lui , fe avava netiuno de’fuoi,

che’ afpettafiero a caia ; pee maggior {uo dolore gli rilpofe : Omues compo/ui,(a-

no accomodati , la morte gli ha ripicgaui tucci, Sa) th ee SUN

 
 

 
  
 
 
 

44

Colei c ha fatto buio
Paga di fogni i debiti a ciafiuno, _.  (Benche fi
Quella , che dianzi tolfe al di la vitay Per fuggir
Cagion, che tutto il mondo porta ae Comincea a
Defcrive con vaga maniera in queft’ Ottava V apparir
con equivoci ; uae far buio vuol dit Confumar tutto il fuo Sed ‘
tendédo della notte)vuol dire ha ofcurato: ¢ fe ha confamato | !
¢ fallita » ¢ non prod pagare i fuoi debitife non con i
ricca fe non di fogni ;e pagar di fogns vuol dir pagar di moneta
non pagare, Vedi fopra C. 2. ft. 7. fugge dungue la notte per
giona non folamente , perché é fallita , ma ancora i ella te
fia fatta la {pia , che ella poco diana. uccife il giorno perché la
ofcurita uccide il giorno ) per la qual morte tutto i) mondopi ee
dir , che per tutto il mondo Ja notte ¢ buio, enter: bruno,e €0
te di gualche nofiro conginen i fe bene ella non dourebbe temere di tal!
zione 5 perche Si chinde gli ocehi a , che fgets on off .
re, finger di non fapere ; ¢ il eos connivere., Vedi fopra C, 6.8 vit
vuol dire che fi chiudano effettivameate gli occhi , perché og ne
fuggir I iba c’ ha le calze gialle , per fuggix V Alba , A ¢ fpia del gi
che ha le calze gialle, perché il primo albore del giorno é i colore frail
€ giallo , ¢ cosi s' accomoda all’ equivoco delle calze gialle shee
ze il contraflegno delle fpie , 0 de i toccatori come accenn fopra C
fan. 60. 03 99g
COMINCTA a ragionar dt far le balle + Comincia a ragionare, 0 r
partenza , che quefto intendjamo quando diciamo: 4 rale fa te baile

fa colligar :
TANZA VI, STANZA VII
E denna » che di quci balletti Jf aftidita poi da ranto fran) +
Sarebbe in corte tutto il condimento , Suvi mulinelli , forge cL

  
 
   

  
   

  
      
 
   
    
    
 
  
 

 
 

  
   
 

Ler ch’ in un tempo fol con i calcerti £ data nna Seofferra come i fae
Ballando{uona al par d' ogni firumento, La laciachiede, britdospi 7
Lupo cena per degni {uci rifperti Perché il mmico all? alba de’ Ta
Prefe dag altri un canto in pagamento, Vuol trucidare in fingolar
E fopra un pagliericcio angufto y ¢ fod Ed a fargli feruixio , pil
Fino ad horas’ ¢ cotta nel | fuo brodo. Vuol ee
STANZA VIL. ANZ

Pero che wel penfar che la mattina i vi intrepid
Entrar in campo dee alla tenzone , Efpaccia il Baiardino, eit
Fa ginfto , come quella Nocentina , Chi la fringeffe,
C's giorno andar douendo a proceffione, Pagherebbe quaicofa ay

Occhio non chinde , ¢ tuttania mulina, Ma tutto quefto

 
 
    

(ZRF, BORED ER RESEP RL Bae. ELLER ew eee

Tanto che ud capoell' bacome unceftone; La faccia tofta 7
Cost la Strega in cella folitaria Sperando
eAtrende afar mille cafpelli in aria, Chie! non fen,

  

101 Sig

 
 
  
 

 

DECIMOCANTARE. 465

__-‘Martinazza , che farebbe ftata la perfezione di quella veglia , fe ne ritiro ins
camera , ¢ poflafi in ful letto ftava penfando alla battaglia , che doveva fare con
jagrillo , ed alla fine , fe ben veramente non farebbe voluta andare a combat-
ere, finge coraggio per non effer cae codarda , ed in {ul far del giorno chie-
le fue armi , (perando pure, che habbia a fucceder qualcofa ; che impedilca , ¢
® fia caufa che non fegua il detto duello .
SAREBBE fata ii condimento , Cioe Carebbe ftata la perfezione di quei bali ,
# di quell’ allegria . Cosi quando fopraggiugne qualche perfona gradita in una con-
" jone , fi dice per ilcherzo , Venir ella , come il cacto fu maccheroni , come lo
- xuechero in fulle fragote , 0 fulle vinande ; valendo con quefte batie fimilitudini fi-
gaificare cid che pid nobilmente fidirebbe . Effere ella il condimento della con-
tm ucriazione , ¢ non vi mancare altro per renderla guftofa , faporita , ¢ perfera .
hued SVON-A al par d’ ogni firumento , Ghediio vogliamo dir copertamente, che una
wet cofa pute diciamo : La talcofa fuona, Vedi fopra C. 6 ftan: 49, ed il Poeta cava
da cid lo {cherzo dell’ equivoco , moftrando di dire che Martinazza fuoni d’ ogni
mit ftrumento , ed intende che Je putano affai i piedi , poiche dice , che ella /uona co’
‘mj ¢alcetti , che fono {carpini di panno lino,che fi portano in piedi in {u ja carne fot-
{hay to te calze ; ¢ fi dicono cascerti ancora quelle {carpe di quoio forcile , fenza fuolo,
gum ma con la fola piantella , che ufano i ballerini , e che ulavano gia l¢ noftre donne
ga di portare fopr’ alla calza quando portavano le pantofole .
ott _ PIGLIAR un canto in Pagamento. Significa Andarfene. I debitori , che volen-
rag ticri (cantonano i {uoi creditori ,fi dicono dare un canto in pagamento ,ciot fug-
gigi gite il creditore per non pagarlo , ¢ per non avere occafione di trattare con Jui:
| . PAGLIERICCIO . E quel gran {acco pieno di paglia , che ufiamo tenere in fu
gig Fletti forto le materaffe , detto anche /accone .
wt ~~ 8° 6 cotta nel {uo brodo , Non ha havuto veruno d’ attorno. Quando alcuno f2
: qualche rifoluzione , che non é approvata , 0 non piace agli altri , e non ¢ da ve-
yi tuno in quella feguitato diciamo ; E /* quocerd nel /no brodo , cice {enza che altri
vi a & nulla del fuo ; 0 vero Fard come gli (pinaci , ¢ s' intende che G quo-
cono ir brodo
gia FA come quella Nocentina, Nello Spedale deg!*Innocenti di Firenze (che & quel
“4 nel guale s’ allevano i nati per lo pid di copula iliecita , fi come accennam-
i Te fopra C, 1. ftan, 85. ) ftanno riferrate molte Fanciulle , che noi chiamiamo
a Mocentine le quali non efcon fuori fe non una voita ! anno, che é Ja mattina,
, della vigilia'di San Gio: Batifta , che vanno per la Citta procethionalmente ; es
Pe ciafcuna di Joro ha gran defiderio di far tal gita , non vi € aubbio, che
f {peranza d’ haver a godere si bramata foduistazione , fa, che pare a’ ciafcuna
_ mill’ anni ,che venga il giorno y¢ che per tal penficro poco derma Ja notte avan.
_ £1, rivoltando per Ja mente wweti li modi di comparire atullata , ¢ bene all’ ordi-
ne ; il che é caula , che Ja mattina ella ha poi un capo c me un ceffone , cice grof-
© €pieno di confuficni per haver poco dormito , ed affaticaia la mente in quei
Penfieri ; € guefte fon quelle , alle quali il Poeta aflomigiia Martinazza .
MVLINARE + Penfare; Difegnare , andar vagando con la immaginazione ;
che diciamo anche: Ghiribizzare . Vedi fopra C. 9 flan. 56. Viene dal Latino
molior » che yuol dir wacchinare ,O ne dal volgare Aduino , quali girare coi pen-
aa ficro

  

   
    
   
   
 

te
 

 

 

      

466 MALMAN TILLEY

ficro come un mulino . Virg, diffe {pedifimo +| Corde:
che fanno le perfone innamorate peulando fidamen
giamente ne diede la defcrizione in Didone ,
Multa viri virtus animo , multufque
Gentis honos  barent infixs pe‘tore vultus
Verbaque , nec placidam membris dat
Tutta la notte va mulinando « E lo ftefio , chevaculer . Ho
Quid brexi ‘fortes iaculamur auo
multa ?
E’ detto ballo {cagliarfi col penfiero ora in una cofa ora, inua)
Mattio Franzefi acl Capitolo delle Nuove,
Lafciamo aftroiegare a chi indovina
Per wie di conetiure , ¢ di difeorfi,
E col vernel fantaitica , ¢ mulinay.

HLA il capo come un cefeone. Gli fi confonde ik cerucilo, Pai p
do diciamo fa ii capo grofio , 9 fe gli ingrofia il capo , intendiamo
de il giudizio: EB Cefone & un gran paniere fatto di vinciglic dt
te , ed & capace di mezza (oma , ¢ perche ha la figura a :
queta comparazione . vil

CAST ELLO in aria, Penfieri fenza fondamento , ed affegnamenti
nt, ¢ che non poflono riufcire . Laili Ba, Tr. C. 2. ft, 2470 ADA

Fra me facea mille Caffelli im aria ode
Ariftofane intitola una fua Commedia, in cui. Gi burla di
Nuuole ; ¢ lo fa falire , ¢ pafleggiare in aria.y per moftrate , pr !
vana, ¢ fenza fondamento Ja (ua filofofia . Noi quando vogliamo dire!
badare a’ difcorfi terij , ¢ avere il capo aitrove ,¢ a bagatelle ; Dichiamo i
fare a’ nuuoli , (e non vuol dire pitt toflo in lingua Lanadattica : Pen/area milla.

MVLINELLO . E uno firumenco di ferro , che ferue per follevar peli
derivandojo dal verbo malinare detto (opra fignifica inucnzioni ,
ne , difegni , ec,

DATA una {cofetta come i cani, S intende , che Martinazza I
veilita , ¢ levandofi dal paglicriccio , fece come fanno 1 Cani , quando,
no, che per lo pid fi fquotono . A

ALBA de’ Tafani , Si dice quell’ ora del giorno sche il Bolee
re vigore , nelia qual’ ora i Tafani fono pit vivaci , Tafano. Lati
un verme volatile fimile alla vefpa nel colore , ¢ nella figura ; ma
aflai maggiore , ed ha ancor’ egii un’ acuto pungiglione 5, ficche
de’ Tafani s' intende leyarfi di la da mezzo giorno wi) |) say Pr
PAR vegnia uno, Far cortefie , 0 carcazeasuno, an iL
no affetcate , fidicoao /exzi, quali iddicia'y © intedtus » come k i
{ca Novella 10. Sérallegro con Nencio [pofo della Ragaread y #6 &
bene , ¢ le faceffe verxi . Col dire.  farls feruizia, e pin chee
orecclu fieno i maggiar pexzé ,intende , che Martinazza gli fara g
tarlo in pezzi cost minuti , che un’ orecchio intero fia: 1 mag)
trovi del {uo corpo »detto acim per suena ua

 
 
 
 

   
  
  
   
  
  
    
    
   
 
 
    
  
 
    
      
   
 
      
  

 
 

a
om

DECIMO‘CANTARE 467

‘SP ACCTA il Baiardino , ¢ il Rodomonre , Si fa Mimar bravo , come favoleggia,
' Ariofto , che fulle il Cavallo di Rinaldo Paladino appellato Baiardo , € quel
¢ Saracino detto Rodomonte. Pud anche effere , che far il Baiardino , fignifi-

chi far il bravo da un tal Pietro Terragiio foprannomniato Baiardo , che fr uns
foldato di-valore , ¢ d’inufitate forze , il quale mori forro Milano militando al

_-feruizio del Re Francefco di Prancia , come narra il Varchi Stor. Fior. lib, 2,
| CHI la fringeffe fra ufcio, ¢’ mare, Chi l efaminatie bene ; chi glielo do-

mandafle da folo a folo,

fegua , 0 non vada Ja pofta , o P inuito
tutte Je cofe , che intenzionate, non s’ ¢!

 PAGHEREBBE quatcofa a farne monte , Spenderebbe qualcofa a non far queflo
“duello . in ructi i giuochi fi dice far monte , quando fi reita d’ accordo , che nons
ropofto ; ¢ quefto ¢ fatto poi comune a.

i(cono : per efempio / tal matrimonio,

he era gid conchinfo', ando'poi 4 monte, cioe non fi ftabili. lo voleva andare a Ro.

with
joie

ma , ma poi ne feci monte , cio non andai .

IN fe tien duro, Lo tien fegreto in fe. Non fi confida con veruno ,

FA factia tofta,, La faccia fol’ efier dimoftratice delle interne paffioni ; ¢ pe-
ydiciamo; / rale fa faccia toa , intendiamo il tale fi sforza di non {co-

iia prit co mutamenti del voito 1 fuoi fegreti , eflendone richieflo , ¢-di non confet-

waco

a

git
:
3
eo
é
:

i

:

“STANZA X.
Spada,e lancia fra taro un Seruo apprefia

i Col perto.a borta in man Laltro galoppa,

‘a altro 0 elmo da coprir la testa
Da diftder unalcro,e bracciaye groppa,
Di che coperta in ricca fopranuefta
Par un pulcin rinuolto nella floppa ,
Ed allestica in ful cantar del gallo
eitro quivi non refta, che il Canalo

fare itdelino »eflendone claminato . Latino frontem perfricit ,

STANZA XL

Percio fa comandare a i Barberefchi ,
Che lo menin n' un campo di gramigna
Accioech’ei pafca un poco,e fi rinfrefchi
Perché per altro il poverm digriena .
La marca bebbe del Reeno,es enidalefcbi
Gis hanno rifatta quella di Sardigna ,
Maglie,e reti ha negli occhi,ode per cena
Vanne a pefcar nel lago di Bolfena

B ferui di Martinazza le portano I’ armi , delle quali armatafi , ordina , che le
fia condotto:i} Cavallo , quale il Pocta de(crive per una folennifima Carogna .
“GALOPP A, Cioe Corre’, Verbo ufato in quefto fignificato ,ma perd impro-

prio, perché galoppare , o gualappare & fpecie di correr di Cavallo ; la qual voce
concorrono gli eruditi a farla venire dal Greco calpareia ,

GROPPA, Si dice la parte di dietro del cavallo , o fimile animale, ma qui in-

tende la {chiena di Martinazza .

PARE un pajein rinnolto nelia floppa, Quando fi vede uno , che non fa portare

I’ abito in dotfo , ¢ che pare impaftoiato nel camminare per caufa deg!i abbiglia-
menti , che had’ attorno , I’ aflomigliamo a un palcixe , o poliaftrello rinuolto
nella floppa ; e non fiamo is cid diffimili dai Latini , che in quefto propofito
didero. Herer ranguam mus in pice .

SVL cantar del gato, All’ apparir del giorno , che a talora fogliano per Io pik

cantare i Galli Vedi forto C. rr. ft. 5. Orazio.

etd galli cantum con fultor ubi oftia pulfar ,
BARBERESC HI, lntende gli Stalioni; (¢ bene Sarbere/chi chiamiamo coloro,
N ai

on 2 i quali

 

Bikes e, 4
  
 
 
  

468 ; MALMANTILE —

t quali cvflodi(cono , ¢ gevernano i Cavalli Barbari , ¢
Pocta gli chianya cosi per derifione del Cavallo di Martina
Firenze 1 Cavaili , che corrono a i palj della Citta, ton
frica , che noi chiamiamo Barberia , {
CRAMIGNA, Erba nota buona per pafcolo degli Afini pil
li, ma a quelio di Martinazza non par poco haver di quefta,
zerin digrigna , clue {¢ nou havefle di quefta , non havrebbe .
ci feruiamo del verbo digrignare per incendere flentar per la fa
nare , ¢ acrocare i denti per non hauer altro , in che ado
canl, ec. che fi dice digrigware , quando per la rabbia
Tat Cas.
x Non vedi tu , che digrignano i denti
Econ le cigha ne minacctan anoli?
Ed egliame: Non vno , che cu paventi y
Lafcsagli digrignar pure 4 lor fenno, ,
MARCA , Contraflegno. Es’ intende quel fegao, che hannoi

li, o di razza in una cofcia , o nel collo, perché da effi fi pofla
razza fono. Virg. 3. Georg. Continuoque notas 5 nomina gentis inurunt ,
che quelto Deftriero di Martinazza havea gia la Marca del Regno di
fono oggi i migliori) ma che i guidade/chi gue n’ haveano mutata in
digna , € non intende dell’ Liola di Sardigna , ma di quel luogo fuori
Firenze , dove fi {corticano le beftie morte detto la Sardigna , came
pra C, 1. ft. 2g. , ed intende , che quefto Cavallo per li guidaleichi, ed
fetti , che haveva, era buono a mandare in Sardigna allo Scorticatoio
te/co diciamo ogni {corticatura fatta alle Beftie dalle {elie , balti, o altro. Mau

Franzcli defcrivendo un cavallo fintile a quefto diffe ; wig
Dinanzi ¢i non ¢ 21d troppo gagliardo ; “iy
Ma in {a ja fcbiena ha quaiche curdalefcho ,
E le {pronate mostran , ch’ ¢ infingardo, ™

MAGLIE ,¢ veri, Cosi chiamiamo alcuni mancamenti , che vengono si
occhi aile beftie ; ed i} Poeta feruendofi dell’ equivoco dice , che con ‘quelle ra
pud andar a pefcare nel Lage dé Bolfena ; ed intende , che il cavallo-era bof
dicemmo fopra C, 3. ft. 53. » che cola fia. E cost fotto quetti equivoci iroa

mente loda il-Cavailo di Martinazza . sagt
STANZA XIl. STANZA XIIL
Hor mentre pajce 1 mifero animale , E ti faluta ,e tt fi raccomanda, —
Eche fi fala cerca aclla fella , E per cha intefo, che rm fai duclly
Giunge un Diavol pit ner aet caviale Vn rotelion di fughero ti manda,
Con un marteile in mano,e una rorella, Spada non gid,ma ben gnejto:
Ed un liquor botiente ix un pitale Con una potentiffima benanday —
‘Ed inchinato a lei cos favella: Ch’ 10 ti prefemo emr'a
I Re dell’ Infernal Diavoleria Bell! ¢ caiduceta come la.

Con quefte trescherelle a te m’ innia , edilo [pedal fi ad ta medicinal
; : aie

    
 
 
 
 
 
  
  
    
 
   
 
 
 
  
  
 

 

 
 

DECIMO CANTARE. 469

. STANZA XIV. STANZA XVI.

Hor fenrj ; che qui batte sl fondamento Ma fe per non haver buon corridore’
Quand’ ih nimico ti verra a ferire Quivi a canfares tu non fulfe leffA ,
Va pure innanzi,e non haver {pavento, O per altra difgrazia ; 0 per errore
el ferro quefta targa a offerire, Ei r'appoggiaffi qualche calpo in tej 5

_ E tuffo ch’ ei la paffa per di drento , Vorlio , che tu per ficurtd maggiore

» Sia prefto col martello a ribadire, Hor per allor4 ti tracanni quefts ,

Ma lafciagnene fubito alla {pada Quale ¢ una bevanda sh squifita ,
Peich'egli a fe tirando, tu non cada, Che chi Lha in corpo no pua u{cir di vita.
Ni STANZA. XV. STANZA xVIL
Face’ egli poi con effa quanto vuole Cosi le fa rngoiar tanty dt micca
| Che pix di punta non puo farts offe/a, D! una colla renace di tal forte ,
» Di taglio manco, effendo c’ una male Che dove per fortuna ella fi fcca
Si fata a maneggiar pur troppo pela; wl mondo non é prefa la pik forte ;
Portila dunque per ombrello al Sole, Luefta ( die’ egli) Uanima t appicca
» Perc alia refta non gis muona {cefa Ben ben col corpo, ¢ s'aitre non ¢ morte
 Edigli( gid che queila non ¢ il cafe) ©? una fepararion di que/ts Aussi ,
Che # egli ti vuol dar , ti dia di nafo, Oxgi timor non hai de’ fasti [uci .

» che Martinazza afpetta i} fuo Cavallo riceve un regalo da Plutone. 5
confiftente in armi jd in. una bevanda per difenderfi dalle ferite ,¢ dalla morte,
Nota che in queflo bel regalo il Poeta immita coloro , che hanno fcritto Je pro-

_ dezze d? Amadis di Gaula , ed altri Romanzatori , i quait , quando il loro Erces
dee efporfi a quaiche battaglia pericolofa , fanno fempre , che qualche Mago
“amico di efso Eroe io mandi a regaiare d’ armi incantate , © altri difenfivi , ed
inttruziom , ‘

St fata cerca delia fella. Si ta cercando della fella, Dice cosi per moftrar,che
ae cra tanto iniolito ad adoprar Ja (ella , che non fi lapeva piu dov’
ella fufle , >

PIT ALE. Alberello , 0 vafo di terra, come dichiara i] medefimo Autore nell’
Ottava feguente dicendo ; ch io ti prefento entr' a quefto aiberelio, Se ben Pitale &
Piopriamente quel va, che fi mette centro alle predelle con altro nome detto
tantero..L’ uno , ¢ i! altro nome dai Greco , quello da Pitharion , piccol valo di
terra , doioiwm ; quetio da Cantharos voce ufaca anche da’ Latini . © fignifica ua
vao lungo , ¢ ftretto in fondo. E con manichi 5 quale ¢ queilo, che fi vede cal-
volta figurato in mano a Bacco.

« TRESCHERELLE, Lato trice , Bagatelle ; Coferelle di poco prezzo, Ve-
di fotto in queito C, f, 28.

SVGHERO . Pianta aota fimile alla Quercia , ¢ fa le ghiande ferotine , ¢ las
faa leggierisfima (corza ferue per far lavort da refiitere all’ acqua , come farebbe
caiietce per metterui bomboie di vetro piene di vino , o d’ altro per diacciare .

_ BELL ¢ calduccia,, Temperatamente calda; ¢ come fi da la medicina , che intea-
diamo bevanda folutiva . Vedi fopra C, 8. tt. 25.

CHUVOVERE fiefa , Fer yenire l infreddatura . Scefa diciamo una diftillazione ,
© catarro , che daila tefta cafea nell’ altre membra per caufa del freddo .

Tl dia di nafo. Detto iporco ufatisfimo nelia Picbaglia ia fegno di difprezzo,e

sin-

 
————————————< =

i

se

472 MALAANTILES @
s intende di nafoine,... che per ricoprire fi dice 6
ferue = efprimere la poca ftima , che fi fa della —_—

NON fuffi lefiaa canfarci. Noa fai prefta a-fuggirli,.
Effugere , delinearesy nes lisdab Greco compre ara
detto cosi quali CG x F

TRACANNI, they bevay logolli i

TANT Adi mica, Vina gran quantita di ininefied = "
tore del Capitolo in lode de’ Peducci , parlando:della min Secccea i
E gli bo tutes per cari, non che buoni

Von oftante , che fia chi dica efpreffa ,

Che tanta micca e cofa da bricconiy
Ser Brunetto Latini feruendofi di quefta voce nel fuo\librovco
tutto di gerghi ,¢ vocaboli ,¢ proverbi Hinsanwsats 7 intitolaco
che fia antica Cittadina-di Firenze, 1s

Non ti darei una mica di beata ; ¢
Se bene qui par , che voglia dire wn bricivlo , dal tele
tanta fi pronunzia col gelto, che accennammo fopra OC. soft.
Luefia pea, e vedremo (orto nell Ortava 18.¢ 36. feguenti

FICC-A. Ficcare vuol dir Mexeré » 0)Cacciar per forza’.

NON é prefa (a pix forte, Diciamo fan prefax, quando la collay cal
© fimili s’ appiccano gagliardamente in quet noghi »ne\bquali-fono

L'ANIMA & appieca, Si ricordi il Lettore , che quella 6
fu le burle , ¢ particolarmente dove G trata diyincanti,ne iquali, q
trava luogo di fare apparir quaiche azione {propofitata,non lafera
fegue in guefla bevanda , Ja quale dice , che appicca ’ Anima al
che egli creda , o voglia periuadere , che cid pofla per incanto farfi
firare la goflaggine di Martinazza , ¢ di coloro j che hanno tanta a
caatelimi , ¢ ne i Demouj ,

STANZA XVIIL
Quando la Maga vede un tal prefente,~
C' ba in fe tanta virth , tanto valores
Da. morte 4 vita riauer fi fente y

Si ringalluxza, ¢ fa tanto di cuore 5 ‘Cusiabe ‘hontai i fe
E dove fares ita un po.arilente Percio fatracal ronzin ba fell
Nel far con Calagrillo il bellumore , Vi monta fopra, € poi te xomb

 

    
     
 
   

Hor ¢ ha la barca afficuraca in porto

Perc! adeffo ch' eg ba ratte
Per fette volte almanco lo vnol morte, rddy

Camminerebbe pik in
STANZA &X

Perch? ei bada a fPudiar declinazioni Pur.grazia del mated
Pin non fi pua farlo levare « panca ; Tentenna tanto y
Le polizze non Pwo, parca i i frafconi, Chiesvien. done n'
E con lo fpalle s*¢ givcaio un’ anca ; M14 « carinetie il fang

Martinazza inansmita dai regalo mandatole da Piutone , etlendo-
Sole , monta a cavallo , ¢ taaro io fruga con gli {proni,, e col im.
zoppicando pur alia fiac  conduile ai luogo dove hayea ote \

fi
at

  
  
 
    
 
 
 
   
 
 
   
  
  
  
   
  
   
   
   

reese erProczsleezeTEt.2f:r2=...

 
 

DECIMO CANTARE. 471
ST fente viauer da morteavies .Cioe le pafla quel timore , c’ havea dvefieres
- ammazzata da Calagrillo | x , .
mi. SI ringaliarza. Si caliegea. Lat. Gefire, Si dice ringalluzzarfi , quafi mo-
~ firarli ficro.,.¢4 animofo come fanno 1 Galletci ,quando fi preparaao per -
“ ter fra oro , @ dopo che hanne combattuco , e vinto. Lucilio 4ib, 8, {atyr. dice:

eee Galli nacens cum victor fe Gallus bonefte
ee ite © Sufulie in digitos , primore/que erigit ungues .
@ [Lalli En, Te. C.5. dan. a6. ditle 3. Jn guetta nacas anor fi ringalluzza . Stor:
di Seumifonte TLratt, 321 Semifoutefi , credendo d'hawer ogni dsfficuied fopita , rinesl-
_— bnzxaronfi , @ fidandofi di (ua valentia y ec, B pi Lowe dice: Veds quanto noi fama
om 4 iti, e 8 mimici ringalluRrati, eC.
gibi FAltanto-de cuore, Pigiia animo , le crefce V' ardire . E il termine Tanto nel fix
infos gail » che diceauno nell’ Occava 17, antecedeate ed altrove , .¢ fi fuppones
i sho deteo.aicrove:), che colui, che per la faczia la dimottrazioac con las
| Mano accennando la grotiezza , ¢ pants di quella cal cofa, Quei che i.La-
ect ual daimus , vooltci quali fempre dicono coraggia, ¢ cxore.
ia | SAKEBSE a a rilente., Sarcbbe andata adagio. Circofpetta ) O rattenuta a4
¥ rifoluerfi,, )L? haurebbe penfata , 0 contidcrata. Significa infomma operar coa
tuwore. Leace per lento , ficcome Violente per Violento dicefi da alcuat ; come
Queito filo, queita corda ¢ fenre ,.cloe non tela , non urata . Da Lente fi fece Ri-
ing (fn sche noo ti ula fe von in quelta Manicra : eFadare a rilente , ¢ fignifica lo
cai ftetio , che Lente cioc ientameate., Nello tteflo modo che!’ antica voce Diricapo
ail ufatardal?-anuco volgatizzatore di Virgilio;c lo iteilo che Dacapo,
PAR ih bed: umore. de dea huomo dell’ umore, vuoi dire huomo faceto , es
SFAZ1010,5-come vedenyno fapra C. 1, flan, 1o..¢.58. s*intende anche wao. , che
si Vogiia: {Opcattarc 1 Compagna-di parole , ¢ di fatti, ¢c, comes’ inteade nel. pre-
feute Moga, f
aia AOR eC ba la barca aficarats in porto, Ciok le par d’ haver afficurata la vita col
regalo mandatoie da Piuwone
ih nae Vheche racing a | bucats wi fu i terrazzi', Cio’ il Sole, che afciuga i panui
“moi deabucati, Dereazzo! »( quali Terrazzo) diciamo quella parce fupériore >
jul dele cafe la quale per loipiu edafciata da ana banda aperta’, ¢ feoza muro, in
Fe vece dei quaie lita, solteacre al tetta- da colonac , ¢ fom fabbricati in quetta forma
ww per comodita d’ havere idole epercid das Latinidetti Solarimm , eda i Greci
Wi! hewocaminus’, Cio fornace del Sole ,
iM CAM AMNGREBBE pit in tre di che in uno, None dubbio., che qualfivoglia
m! — Animaie.camuninerebbe pith ia tee giorni , che.in uao , ma uGamo quetto modo
wih di dure per moitrar Ja fiacchezza d’uao Aaiuale y quafi diciamo: Quel viaggio
che egit na da farein ua giorno, 10 farcbb¢e pwd voleauieri in tre giorni , che ins
yun foiow ue ¥
ul BADA a fhudiare-declinagioni , Attende ,:o-continovaad accennare di -cadere
w” ~— perladebolezza. Deciinare’s’ intende uno , che*etfendo in buono ftato: 5.0 dt fa~
ie hita ,o di roba , cominci amancare neil’ uuo , O.nell' altra; equi (cherza cont
4) equivoco delle declinaziom de 1 noim 5’ed-insende , che-il cavalo per Ja deboica-
#) 2a cra feinpre per. cafcare..
0 Wow
  
 
 
 
 
 
 
   

47% MALMANTILE ©

NON fi pui far lenare 4 panca, Non fi pud farlo riavei
flar ritto : quand’ uno é ftato lungo tempo afflitto da i difaftri
to per terra, 0 vero terra terra) € che a poco a poco fi va
Comincia a rizzarfi a panca; BE’ traslato da 1 Bambini , quando:
dar ritti appoggiandofi alle panche; onde habbiamo un detto per
uno fia pil aftuto d’ un’ altro, che dice : Quando it Diauolo del tale nat
del? altro andaua alle panche , Franc, Sac: Nov. 158. dice: ach 60)
noftra mercanzia , che non ce ne rizzerems piit a per quefto anno, ©
NON pwd le polizze . Non ha tanta forza ch’ ei pofla portare una po
Latini pure diflero: We folium quidem fufpinet . t

PORT Ai frafconi , ec. Diciamo portare i frafconi uno , che fia alg
mo , traslato dagli uccelli , ne i quali ¢ contraflegno d’ infermita Y
abbaflate , che paiono beftie cariche di faftella di frafconi . Vedi. y
g. alla voce grado. Qui vuol dir che il Cavallo era infermo , ¢ malandato pet lt
vecchiaia . | Lb
CON 10 [palo s* ¢ giuscato un anca, Scherza con I’ equivoco del giuoco di
nel 8 quand’ uno piglia tante carte , che col lor contare
31. fi dice /pallace , 0 ba baxuto lo /pallo , ¢ perde , fi che intende che il

 
 
     
 

&

  
  
   

  
   
     
 

   
    
  

Martinazza é fpallato . von lil
GRAZIA del martello , e degli fproni, Con ' aiuto del martello , che le mand)
Plucone , ¢ degli {proni , cio perquotendolo col martello, epi 1
gli fproni: Diciamo anche mercé del martello , ec, er
S* arranca , Diciamo arrancarfi , quand” uno per qualche difetto non pot
muover le gambe s’ affatica per camminare , e/forle ¢ il verbo p
pato. Vi chi lo fa venire da Anca , che é !' offo tra "I fianco, ela coleiay
quefta dalla Greca Ancon,colla quale fi fignifica il gomito , ¢ fi ftende ad
gature , che fomigliano quella del gomito , Onde Sciancaro , quafi ex:
pun ha intere , enon fenza mancamento |’ anche. B Arrancarfi quafi tirarh, 2
firaicinarfi distro P anche . 15h) aga
NE ba da ire il fangue a catinelle , ¢ ha bigonce, Ha da verlari moltifimo far
ue. Vedi fopra C. 2. flan. 57. (perbole ufaca quando due Poitroni
ducllo, Vedi fopra C. 1, ftan. 62. in altro figniticato. BC, 3. Ran, 29, che ol
fia bigoncia , Quando I’ indugio piglia vizio , e-che fa di bifogno 1a preftezza jl
altro propofito dichiamno . ee ne va il fangue a catinelle, aah
STANZA XXI. STANZA XXIL
Quand! ti Nimico, ch’ ius faa difagio Se tu fapeffi , come tu non faiy
A tal pigrizia ,grida ad alta voce , C’ armi fon quefte pie
Vieni Afinaccia , moniti Santagw Farefte forfe il brauo mance ,
Cb’ so fon qui pronto acaricarti anoce, O parlerefti almen-a' altro ling
Ella rifponde: A noce? Biagio; Ma gid che tn venifti a tno
Fate un popian Barbier ohe'lranoquoce; i
S' altro vifo non haivallo a procura,

    
 
    
  

 

    
   
 
   
 
 
  
 
   
 
 

SeweR.> es > sre

cr

a repo RB re =

atten

o

alee

 

 
   

Lerche codefto non mi fa para. rrotté
Arrivata Martinazza al luogo dove s' haveva a fare il duello.wi tr
¢o Calagrillo , il quale vedendola venire cosi adagio ia fgrida y ela

 

 
 

SSk ELS

8 EE Sei ESei a oak

=
&

DECIMO CANTARE. 4B
ella gli rifponde ;che non ha tanta furia , dicendogli ch’ ei non’ farebbe tante»

bravure,(¢ egli fapefie di che armi ell’ ¢ armata),¢ che ella veniva per ammaz-

zarlo.

“STA 4 difagio. Patifce afpettando : Sente incommodo in afpettarla ,
_ eASINACC/A. Parola ingiuriofa , ¢ benifimo iesire in quefto cafo as

Martinazza , perché veniva pigramente , come fal’ Afino.

. SANT AGIO, Si dice veramente Ser egio; che fu un Medico cosi nominato,

perché taceva tucte Je fue faccende con ogni maggior {uo agio ,¢ commodiia fino

a tirighare , ¢ ripulire Ja faa mula, fenza muoverfi dal Jetto ; ed ¢ paflato poi in

verbio , ¢ yuo! dir Huomo di turti i {uoi comodi , e tardo nell' operare , che

ju una parola diciamo - Agiato. O forfe-dalla voce To{cana, che vuol dire Len-
fecha y Comodird ,

A caricarts a noce , Quando il noce é carico di noce , fi fcarica con le baflona-

te, ¢ pero dice , che wuol caricarla alla foggia , che fi carica i] noce, pec (cari-

Carla poi-con le perco!

fe

» @LAGIO Biagso. Modo di dire ufatifimo , ¢ particolarmente de i Fanciulli ,
€ credo che fidica per caula della rima , ¢ del bifticcio , perché per altro il nome
Biagio ¢ (uper fluo all”'e(preffione , valendo tanto il dir folamente adagio , quanto
adagio Biagio , S¢ bene ci ¢ una favola notifima d’ un certo Contadino nominato
Biagio , i quale perché non gli fullero rubati i fuoi fichi , fe ne ftava cutea la not-
te a far loro la guardia ; onde alcuni Gioyanotti per levarlo da tal guardia, es
poter a Jor guito corre 1 fichi , fintifi Demonj una notte s' accoftarono al capan-
nettoid) Biagio mentr’ era dentro , ¢ difcorrendo fra Joro di portar via la gente,
ciafcuno narrava le {ue bravure; ed uno di coltoro diffe ad alta voce; Se voglia-
mo fare un’ opera buona catriamo nella Capanna , ¢ portiamo via Biagio ; Bia-
B10 cid -udito,{cappd dai capannetto tutto pieno di paura gridando Adagio adagio.
o di qui puo forfe havere origine il prefente dettato Adagio Biagio , 0 adagio diffe

ago,

FAT £ pian Barbiere che'l ranno quoce . Di quefto dettato ci feruiamo , quando
Ron voghamo accon{cutre che fi faccia qualcola in noftro danno .

» COT ESTO vifa non mi fs paura, Quando vogliamo moftrare di non temeres
diciamo: Ha tu altro vifo?e qui Martinazza dice: Va 4 cerca d! wn’ altro vifo
perche corefho non mi fa paura.

SEVER AGGIO . Invende quella colla fehe Ie ha fatta bere il Diavolo, 1] Fran-
zele dice bexarage corcifpondentemente alla noftra voce .

A tao ma’ guar, Cioe a tuoi mali guai; Mal per te , che ci venifti , Ci {ci ve-
puto per wrovare il tuo danno, Cusi 44a’ paffi diceli alcuna volta per cattivi pal-
fi; ome ‘Piano a ma' paffi ,

MANDA 1 faggio . Quando fi da una piccola porzione di quella mercanzia ,
che fi vaol yendere:, acciocché il compratore pofla riconofcere 1a qualica di etla
mercanzia fi dice ; dare , 0 mandare il faggio . B Martinazza dice a Calagrillo,
che intanto mandi il faggio delia fua carne ai vermini , perche fra poco yuol
mandargii nell’ avello tutto il corpo .

NON volti portar bafto. Non {on folita fopportare ingiurie .

Ooo STAN:
414 MALMANTILRS 2
STANZA XXIIL t ]
Horsh , dic’ egli, all armiv apparecchia,
E vedrem fe farai tante corenne ,
© quefto fuono allor mona Pennecchia

y

Dice fra fe: No,no:Non taro Ammenne, \-  E ch* io t° infegni far

Sard meglio qui far da lepre vecchia , Cost tn ch’ items

E fenva ftar a dir pur al ©... vienne , Milafis a}

Fa proua ( gid dilcefa dal deftriero ) Ma fa pur quitof

Se le gambe (c\dicon meglio il vero. Bt ual, fe eu fu z.
STANZAXXV20 1)

  
 

S? al cimento , dic’ ella , del duello C
A furta corfi , bor fuggolo qual pefte y Perd che dop’ al muro f
Pero va ben , che chi non ba ceruello Grid egli quanto vnol,
Habbia gambe , e cost mena le [este , Che per le grida it Lupofe
Mortinazza , vedendo , che Calagrillo non cede alle fue bravate ,

che fara meglio per Jei non indugiar pil a fuggirfene , pero (non fi:

cavallo) fmonto , e fuggi cosi a piede verlo il Caftello!
rimproverandole i] mancamento , ma effa ftimando pit il peri

la perdita della riputazione fen’ entra in Malinantile , ¢ lo lafcia
SE farat tante corenne , Se farai tante bravure. Detto di derifione a wu

vantatore . wR
MONA Pennecchia , Detto derifivo alle donne. Da Pennecchio', ig

priamente fi é quella quantita di lino , 0 lana , o cofa fimile , che fi

rocca per filarla , detta cosi quafi penficulum .. Dal Lat. penfum .
NON tanto ammenne . Non fara cosi . Ogai parola non vuol rifpofta Per

io non voglio poi anche fidarmi in tutto di Platone’. Amen & parola Bl

vale In verita , Per verita . er
FAR da lepre vecchia , Cioé tornare in dietro, La lepre vecchia per’

gnar terreno , quando ¢ feguitaca dal levriero da in dictro , (il cS atto

La un ganchero , Vedi fopra C. 2, flan. 76. ) ed il cane furiofo fe

fcappa innanzi , ¢ perde |’ occafione di pigliarla . L’ aftuta maniera

delia Lepre é defcritta mirabilmente da Eliano nella Storia degli animali’
cap. 14. . are

SENZA dire alc ,,., vienne, Andarfene fubito , © fenza ‘merter tempo it

mezzo. II Pulci nel Morgante , £ non é tempo da dire ale.... vienne.
SE le gambe le dicon meglio il vero. Se cilia fara pit prefto a fuggire

a cavallo, Quando le gambe , braccia , 0 altre membra fanno bene 1a

razione diciamo: Le gambe , ec, mi dicono sl vero , cioe non mii fallifeone

mancano foto. , Wee

Cl haueffi detro ulmen Salamelech , Almeno ci’ havefii'ta detto .

Turchefca ufata da noi per (cherzo ; ¢ fignifica , Pace , 0 Salurea voi. ~

FARM le feilecche . Betfarmi.. Vedi fopra C, 7. flan. 25, 11 Vor

goefe dice , che Cilecca wien dal Greco Cileo , che wwol dir mulceo far

feilecca far tl contrario di carezne, civ far burle. Ma pud effere , che |

licta Gi fece Lezei forca di delicatczzc cosh Scileccke il contrario , che A

aliettare 5 ¢ poi burlare ,

E intana di riterno

 
    
 
   
  
   
   
 
   
   
      
   
  
 
        
 
  
     

   

 

 
 
 
 
 
 
 
 
 
      
    
 

= pe eet S*e2 ce oc We ieee se*8. BS screes._es.ess
  

SSeS

 

 

DECIMO CANTIARE. 475
WMI lafci a prima giunta in fulle fecche . Subito,m') abbandoni + Milafci:fenz’

- alcoltarmi . .B' lo ftetio.che la{ciar in Naffo,vifto (opra C, 1, an. 79, Si dice an-

che /afciare in Seco; lafciar fulle fecche di Barberia. Lat. Syrtos .

AO teco il sarlo. Ho.rabbia teco., perché ilxoder. deila rabbia s' affomiglia al
roder del tarlo nel legname :,Per il contrario fidice: auer baco.con wna perfona y
cio¢ averci paiione. Petrarca: Afentre che il cuar daeli amorofi vermi fu confumato

TI vegito fe tu fuffi in gremboa Carlo, Tiarriverd per.tuto, Diciamo; J.
grembo 4 Carlo, cioe Carlo Magno Imperadore , per moftrare che fi vuole arri-
vare uno , ¢ vendicarfi in ogni maniera , quand’ egli anche fi fuggifle fotco la pro-
tezione del pid porente , valorofo Principe del. mondo , come fu Carlo Magno;
econ i Latini diciamoanche,in grembo a Gione.. at

| CORRER a furia,, Eo fefios che far una cofa fenza confiderazione.. Vedi
fopra C. 5. ftan, 41. E qui (cherzaintendendo fe corfe nel venice corre anches
nel tornare in dietro. ‘
_ CHL nan hs ceruello habbia gambe . Significa chi non ha havuto giudizio ,o me-
moria di pigliare , o fare tutto quello , che egli doveva in uo viaggio , habbia gam-
be , cloe lo faccia in due , 0 pit viaggi , ma qui il Pocta {cherza ,.¢ motteggian-
do Martinazza fi ferue del proverbio , per intender , che fe ella non hebbe cer-
ucllo ad accettare , ¢ venire al cimento del, duello , habbia hora le gambe per

ire
MENA le fefte, Pa {peti ,¢ lunghi padi, Le (ele , cioé il compaffo , s’ affo-
miglia alle gambe dell’ huomo ; ¢ pero mexar /e feffe s! intende adoprar prelto les
gambe , cioc camminar velocemente , correre .
ANT ANA. Intendi {e n’ entea nel Caftello di Malmantile. /ntanare da Tana;
cava foterranea .
DIET RO ai muro faluns efte, Chi ha un parapetto di muraglia non ¢ dubbio ,
che € ficuro dalle ftoccate.. B/fe dal Lat, &é , formato all’ ufanza noftra , de’
li niuna parola intera finifce in confonante. Ii Burchiello nelia fine del primo
Sonetto . on funt non funt pifces pro Lombardi . I] primo fant va {eritto , ¢ letto
funte come qui Efe , acciocché il vero torni . E in quel verfo , per dire anche
spe 28’ aliude a un vero Racconto , che fi trova (critta nelle Craniche de’ Pre-
icatori , alla vita di Giovanni da Vercelli Generale.
DALLE grida fcampa il Lupo, Detto ulaciflino per moftrar la poca Rima ,
che fi fa di coloro , che gridano ,

STANZA XXVL STANZA XXVIII.
Poich* egli vede in fomma che coftei , CHartinarza , che teme del {uo male,
Alsrimenti non torna , fa i {uot conti , Vedendo che 'l nimico fe le accaita ,
Che fara ben ch’ ei vada a trouar Lei, Tre (caglionc’ba la porta,a un tépo fale,
Come faceua Macometto a i montis E gli da nel moftaccio dell! impofta «
E perch’ ell ha due gambe , ed egli fei Ds poi dandola a gambe per ie {oale y
( Mentre pero di fella ei non i/monts ) Senta dar tempo altempo,apigliar fofta
L arvriuerd:ne primaildeftrier punge, Infacca nel falon , la done ¢ il ballo ,
GC? all entrar dé Palazzocs te lagsunge. Ed ei la fegue foefo da cauallo.
O00 2 STAN-
.
 

   
   
 
   
   
   
   
   
    
 
    
  
       
 
    
    
      
 
  
   

476 MALMAN TIVE) 0

STANZA XXVIIL:
Appunto era feguito in ful feftino , \
(Come interuienein ee
Che due di quei che fannoda xerbine
S' cron per Donne disfidari «morte
L' un foreftiero, e /mentico pel vino he
L’ a mi lafera,anch’eicenddoincorte } -
Ha Spada accato il Cortigian,ch'é l'altro,
Ma piit per ornamento, che per altro, Alle fpalle

ca STANZA XKX.
In quel ch'ei morde i guati,efaquei sees © Ohe im

Che van de plano all arte del Adirrilto, ©

Ech'egtihafempriall'ufeioguoctes :

Dietro alla Serega giunge Calagrillo y Pit des pie e

Calagrillo feguitando Martinazza entra con Lei nel is oO
che gia fatto giorno ) continovavano a ballare ,'¢ mette paura a
larmente a un-zerbiaello, che ¢flendofi sfidato con ua {uo tivale
fufle quelio , ¢ pero fi fuggi codardamente. 3
COME faceva Macometto ai monti , cioe {e NON VengZORd Pr aendi fi
noi da loro , che cosi ¢ fama , che dicefle Macomeito , per mofti
miracolo , comando a i monti , che {cenueilero gilda iui, “e veer
venivano ‘dicefse; Horst: andremo noi da loro.

HA fei gambe, Cioe due fua , e- quattro del Cavallo. 0

GL1 da P impofta nel moftaccio, Gii ferra la porta in faecia Che T
mo quel legname , che'chiude Je porte ,‘¢ fincitre da: Launo poiter) B
Serrar la porta in faccia:, per intendere operare - fare in modo 5 the =
vicino alla porta non entri,’¢ ferrar (4 porta tn fa le calcagna, ‘incendere
uno fuori di cafa., come vedemmo fopra C, 3. ft. 50. ‘Nenehe serial
T impofta nel vifo., o ne i piedi. ae
DANDOLA a gambe , Cominciando a correre. Vedi fopra 0.4
SOST-A. Ripofo.. Vien dai verbo /ofare , chee: il. Laune/ ey
re ,o fifere,

FESTINO, Trattenimento di giuoco;o di ballo. Vedi ropa Ca
celi Fefino., quafi felta piccola,, come quella, che'fi fa\felle’ca!
delle grandi , che fi fanno-nel pubbiico..

TRESC.A, Cosi-anticamente dicevafiuna fpeeie di allo dal qual
hoggi Tre/cone fpecie di'bailo , come vedremo {otto C; 11g. U
Purg. c. ‘10, Ja piglia per {pecie diballo, dicendot
Trefcando alzatol’ umile §. rosacea
E nel prefente logo ¢ prefa per adunanza wi. gence’, che’
che Ja piglia il medefimo nell’ daf.C.14.

-  Senza'vipofo mai eralatrefea
Da trefea ; trefeare , ches’ intende operaré ; ¢ Tre)
telle , che yuol dir cofe di poco prezzo , o ftima. Vedi a
£ANNO da Zerbino, Fanno dei bello, ¢ del galante,

ao Pe oes SP THE SS Es

-

a aie aa Ae

 

 
 

¥

DECIMO CANTARE 477

\ TVTT AY architettura’s ec. Vuol dite , che quel tale ufava nel veltire ogni ar~
te, € s’ aggiuftava con ogni maggior lindura , diligenza ,edifegno.
GONFIO, Alticro, ¢ fuperbo per la fua bellezza , come fa 11 Pavone ,. che al

i detto delle perfone pid femplici ,' gonfia perche fi ftima bello ; donde poi pavonez-

giarfi , che vuol dir confiderarfi , ¢ vagheggiarfi per bello; E guefto verbo <(pri-
‘Me quel che vuol dir i/ Poeta nel prefente luogo .

CREDE turar le Dame in Veffunio, Crede far perder ‘tutte le Dame per il fuo
amore. Crede, che Ja fua bellezza fia per far’ ardere del {uo amore 3 ¢ Vefusio &
il monte del Regno di Napoli , dove fono le voragini di fuoco . . rf
_ HA paura del dilnnio, Cioe del diluuto delle percofle , le-quali fpengono amor
nel cuore, e ' accendono nelle fpalle ma differentidimo .

» VAN deplano all’ arte del Mirtilio. Son-douute ,-¢ fi richiedono all’ arte dell’ ia-
namorato , da que! Mirtillo introdotto per innamorato dal Guarino avila fuss
Tragicommedia incitolata Pafforyfide .

HA gli occhi a’ mobi, Bada , oflerua , fta vigilante. E diciamo «’ mochi, ¢ non
allfaltre biade di maggior valore , perché eflendo i Mochi cibo proprio de i Co-
lombi , fono da’ effi prt, che I’ alcre danneggiati quando fono di poco feminati ,
€ peroé neceflario haver J’.occhio , ¢ badare con piit attenzione a i mochi , ches

Ll alte biade .

[pochi , Detto ironico , ‘che fignifica moltifimi .
ba pik cnor dun grille, EB’ codardo, non ha animo, Sotto C.11.z9.dice,
Han facte di Leoni, ¢ cnor di fcriccioli , Appretlo i Greci per il contrario trovafi
Thymaleon , cioe Cuar dt leone , per vomo valoro(o , forte , cortaggiofo.

FA pit capicale de’ piedi , che del ferro, Si confida pil ne i piedi , che nella {p2-
da ; cioe Mimd pid ficuca dife(a quella del fuggire , che quella dell’ armi: ¢ circas
queita voce capit aie. Vedi fopra C. 7. It. 82. ¢ C. 8. ft. 6.

STANZA XX&K1. STANZA XXXIL
Toffo tornando l' amicizia in parte , Prima , che tra coftoro altro ci nafca ,

Si viene allarmi, che ciafcuna armata
Cid tien del altra un fegno fatto adarte
Per darle atradimento la-pierrara :
‘Di qui fi viene a mefcolar le-carre ,
Tal ch’ in vederlatante fcompigiara ,
Rittrandofi a dir badan le Dame:
Baha bafta; non pik ; dentro le lame.

£ che la rabbia affatto entri frat cani,
E ms conuien fattar di palo in frafca ,
E ripighar la Storia. del Garani ,

Chre dietro a far che'l Turacirinafia,
eAccio,tornato pot come i Criffiani ,
Ad onca della Strega-ogni mattina
Ritorni a vifitar ta’ Kegolina

Di queflo follevamento ciafcuna-del'e-parti prefe fofperto di tradimento,.e per=
cid fi venne all’ armi dentro al medefimo falone.. ‘Qui l’ Autore.lafcia coftoro , ¢
torna a Paride Garani , il quale egli latciodopra C. 8. ft. 59. .

TORNO' ? amicizia inparce. L amicizia fi divile; cive ritornd inimicizi

“mMeera*prima . Parre t quella ; che i Latini‘dicevano parter , ‘ciot fetta , fazione ;
‘onde Parziale , cioe affezionato,difenditore. Quel che fia parte per womo di fpa-
da ch’ egli era, ¢ non di lettere , Jo defini affai bene Farinata degli Vberti ti vec-
‘chio , ‘pretio.a Gio. Villani |. 12. Volere , ¢ disuolere ; € per oltraggi , ¢ grazie ri-
Ceuute,

DAR ta pietrata , Dar colpo mortale ; 0 conclufivo , dare a tradimento la pic-

trata
  
   
   
    
   

478 -MALMANTILE &

trata é ver in quel verfo di Plauto ; leera manu fere lapidem ; panem oftemit
altera , Che rifponde anche per appunto al noftro proverbio ane y¢ (a
Sajata. ; o% SW
ST viene a mefcolar Je carte , Si me(cold la zuffa. Vedi fopra C. 9. ft, 35.
SCOMPIGLLAT A. Confula. Qui intendi, rottalapace.
LA rabbia , e fra i cani, Cosi diciamo quando yogliamo efprimere »
s’ azzuffano indiftintamente: Ii Latino Xabies inter canes, Ee
SALT AR di palo im frafca, Paflar da un difcorfo ad un’ altro affai
dal primo. Far digreffione . 11 Monofini dice , che con quefta noftra
s’ accorda quella de’ Latini ufata da Tertulliano, De calcaria in carbonariam..
Ma guefta s’ accorda pid con quell’ altra,. Dalla padelia nella brace. I |
Tertulliano nel lib, de Carne Chrifti dice cosi . dgitwr de calcaria, quod:
carbonariam ; a Adarcione ad Apellen, i ose
LA regolina, Cosi chiamano i Ragazzi dell’ ipfima Plebe Pornia ube
ga, la quale fta aperca in tempo di Quarefima, ed ivi fi vendono frittelle ,
Ii, baccala fritto , ed altre forte d’ untumi fimili , praticata, ¢ frequentata da’ ra
gazzi , ed altre genti viliffime , come era il Tura , che {peflo v' andava.
STANZA XXXIIL STANZA XXXV.__

 

Paride giunto in mexxo ai cafolari,

Ove meffer Morfeo aun tempo folo
Fa dir di sia molts in Pian Giullaré
Strepitando fugeir lo fece a volo,
Sicognun deffo vanne a'fusi affari,

Ed ci,che Star non viol quivi a piuolo

eAnzi dare al negurio [pedizione,

Domanda di quel luogo infor marione,
STANZA XXXIV.

Vn gran Villano, un bnom acta matura

De’ Quarantotti li di quel Contado,

Che perche ei non ha troppa [effitura,

Ed ¢ profontuofo al quinto grado

Junanzi fe gi fece a dirittura ,

E concerts (uoi inchin da Fraccurrado,

Benevenga diffe , Voftra fignoria,

E Le buone Calende sl Ciel vs dia,

Jn quanto al Lupo egli ¢ un! animales
Aa che aninial dich? io bue,
Via fiftol ds quei veri , un faci
C' ha fatto per sngenito gran dant,
E gid con i forconi, ¢ con le pales
J popoli affilliti rurto mguanno
Quin’ oltre gli enno feati tutti riete
Per levar quefto marbo da nn

STANZA XXXVL

Ma gli ¢ un fetanalfo foatenato,

Che non teme legami , ne ea.
S? ¢ carpito pits voilti 5 ed ammagliatty
Ed ha ricifo funi tantogrofe,
Le bastonare non gli fanno fates
Chie’ navha abriga rocehechethaledt
D' ammayyario co’ ferri non c' ¢ viy
Cb’ egit ¢ come frncar n’ uaa matity

 

STANZA XXXVIL

  

La entro a quella felua ei fi rappiarra, Che tutti gl' animaliycht ei raccats

* Perch’ elia égrande,dirupata ye fitta y Cudfando gli trafcina lvirittay
wacciocche nimu.un tratto lo cumbatta, , E chi guatar poteffe ; io.fopenfierd
Quand egli ba dato a'Socci la sconfitra, Chie’ v' habbia fatto a’ ofa uns "

Paride entraso ne i Calolari di Montelupo trovo , che tutti dormiyand,|
con firgpitare fece (uegliargli , ed havendo caro di sbrigarfi , proccurd
intormazione da qualcuno delle qualita ed abitazione del Lupo, ¢s' ak W
un Villano Sateapo del paefe, che gliene diede puntual ragguaglio. Ecol dif
fo., che.fa fare a quefto Villano , moftra il modo di parlare del cont
KODZCp

      
  
 
 

 

 

tel on oe Ge Cee:

i el i
 

 

 

ee.

tak

 

Zo
7
a
a
q
i
5
j
:

%

DECIMO CANTARE:? 479

CASOL ARI, Intendiamo pid cafe infieme in campagna fcoperte , € spalcate ;
qui intende di Montelupo , il quale fe bene ¢ Caftello , ha pil figura di Cafolares
per effer le Cafe cutte quafi rovinate , ¢ diftrutte .

MOREFEO , Favolofo Miniftro del Sonno , il quale i Gentili tenevano , che a»
i comandamenti del Sonno fuo padrone fi trasformafie nella facia, nel paflare, €
ne i coftumi in qualfivoglia vivente , ¢ perd fu {critto : Hominum fittor Morpheus’,
beftiarum imirator . Ed altri, Atorpheus , © varijs fingit nova vultibus ora , detto
Morfeo da Morphe , che in Latino vuol dire forma , faccia ; onde noi Smorfie,
per brutto arto, o gefto fvenevole , che fi facial particolarmente col vio. E
roan in furbe(co ; mangiare . Qui dal nofiro Poeta Morfeo, ¢ prefo per'lo Nef

fo fonno . \

FA dir di si a molti in Pian Ginllari , Fa dormir molti ; perché colui , che dor-
me fenza pofar Ja tefla , l’ inchina , ¢ fa con efla i] medefimo atto , che fa colui’,
il quale con efia accenna di dir di si. In Piaw Gintlaré intende nel letto , che anti-
camente’fi coftumava il dire. / vo in’ Pian Ginllari per intenderc , io voa letio,
© mi pongo gil a dormire : Ma quefto detto come oggi poco ufaco é ancora poco
intefo. Per altro Pian Gindari & chiamato un Borghetto di Cafe nel concorno de’
Vilage di Firenze non troppo diftante dalla Citta, che anticamente era de'Giul-
Jari cafata Fiorentina . Giullari , e Giulleria , dal Latino iaculares , vuol dir butio-
ne, ¢ buffoneria , 0 allegria . Vedi il Varchi nei {uo Hercolano ; ed il medefimo
nelle Stor. Fior. lib. 15. Won gridavan con quella fefta , ¢ ginlleria ch' eran foliti .

STKEPIT ANDO fuegir lo fece 4 volo. Facendo romore , fece fuggir Morfeo ,
cioé fueglid i popoli .

NON vuol far a pivolo, Non vuole ftar’ a difagio afpettando ; diciamo : Tener
uno a pivolo , quando lo facciamo afpettar pili del dovere , o pil di quel che egli
vorrebbe , quafi che egli flia legato alla noftra volonta contro a {ua voglia, come
fi fanno ftar legate le bettie a i pinoli, che (ono pezzi di baftone, che fitti per le»
mura feruono a i Contadini per legarui le beftie .

DE’ Uuarantotte del contado, De i pid riputati , ¢ Aimati del paefe; perché il
Quarantotto in Firenze é la dignita Senatoria , la quale ¢ il maggior grado , ches
godano i Cittadini Fiorentini .

NON ba feffitura . E’ huomo ardito , e libero nel parlare’, non ha vergogna, o
-riguardo 0 timore , che lo ritenga ; ¢ s’ intende anche Vn’ huomo , che operi, c
viva inconfideratamente , Sefirwra chiamano le Donne guella filza di puoti radi,
che fon folite fare da piedi , 0 nel mezzo delle loro vefti per farle divenir pib cor-
te , © per aliungarlo con sdrucire detti punti fecondo , che torna loro in acconcio
dal Latino /ectura , come vuole il Ferrari, Le Romane moderne 1a dicono ritrep-
pio, quafi piccol ritiramento deila velte , ed ¢ lo fteflo , che imbaftitura , che ve-
dremo fotto C, 12. ft. 33.

PRESONTVOSO , Pili che ardito , e poco men , che impertinente : Vno che
prefume afiai di fe medefimo, e s’ arroga piii di quel ch’ ei merita. Vn’ arrogan-
te. Daa. Purg. C. 11, dice.

Bd ¢ qui perch fu prefontnofo

DA Fraccurrado, Da Fantoccino ; da burattino ; che intendiamo quei bam-

bocci , che dicemmo fopra ©. 2. ft. 46, 11 Bini nel Capitolo del Bicchicre <
Kuch

 
 

 

    
 
  
   

MALMANTILE

Quefti perche fon grandi , ancor fon belli
Sends poca betta fenza grandeRra y
\ wei paion Fraccurradi y¢ Spivitellig
Tra’ canti Carna(cialefchi vi ¢ un canto intitolato . Canta, ni

Fraccurradi , e Bagattelle , ove {ono defcritti , i giuachi, che
© giucatori di mano con tali legnetti , ¢ burattini , detti-Frac

LE buone Calende il Ciel vi dia. Virconceda il Cielo, tutti i.
dia ij buon’ anno .

SVE di panno, Sciocchiffimo ch’ io fone , Io ho manco giu
dicenci. Vedi fopra C, 6. ft. 98. Lyf

VN fistolo. Le noftre Donnicciuole intendono Demonio , Diavolo. Vi
male maladetto, Bocce, gior. 7, Nou, 6. dufino a tanto, che il fiftolo uf
Juo marito . Cosi detto dal filchiare de’ ferpenti , a’ quali egli ¢ affo

F AC/MALE , Huomo maligno , ¢ da fare cout it
lefactor, Cavalcanti Storia lib, 9. cap, 11. Cerri huomini befti
i quali mai alcun bene fecero , ¢ now hanrebbono faputo farne y huomini faci
futili, n't

PER ingenito. Per naturale inftinto , che quefto vuol’ intender quel Ci

eASSILLIT I, \oucleniti , adirati. L’ Affillo é un vermicello volati
alla zanzara , ma pill grande , ed ha un forte , ¢ lungo pungiglione
quando il Bue ¢ punto , entra in grandiilima {mania ,¢ tem eda qu
tadini quando vogliono intendere , che uno é in collera dicono ; Eel:
o¢ afiduo, Sula in Firenze ancora quefto termine , ma per ischerzo, y
con ammogliati con i quali farebbe termine ingiuriofo , quando non fulle ula
in burla , perché ¢ un dirgli Bxe , ve hfe

¥GVANNO. Queft’ anno , Vedi fopra C. 6. ft. 92. alla voce auannotte,

SMINOLT RE glienno feati tutti rieto, Qui intorno gli fono ftati cust dietro ct
cando di pigliarlo, Enno, ¢ la terza perfona del numero plurale dell’ indi
del verbo efere , hoggi poco ufato in guefla forma fuor , che da i contadini;¢!
uso Dante Parad. C, 13, me

Non per faper lo numero , che enno oot

PER levar quefio morbo da tappeto Per levar queita pefte , ¢ quefta tribolazion
dal mondo ; J sappete feruiva gia in Firenze per firato ai Supremi
quindi /euare uno da tappero figuibca levario , 0 privario di quella dignita
quale ¢ pofto , che por pafiato in proverbio yuoi dire privare , © levar uno
qualfivoglia luogo , come qui che s’ intende levar dal mondo,

SET AN ASSO, Satana; Demonio , dai Latino Saranas,come
nuovo teflamento. Appelliamo Saranafo uno , che fia fiero , ¢
{crua di tal jua forza per far del male: ¢ ufato perd dalle donne contro,
ciulli fieri, ¢ vivaci, 1 quali chiamano anche WVabifi. In Ebraico:
onde il noftro Dante. i? acai

48a

 

 
     
 
 
  
    
   
  

 

 

Pape Satan pape fatan aleppe. ) aries

Evuol dire Aduer/arins , Aduer/arins nofter dsabolus , ate
CakPITO. Cioé pigiiato con violenza , dal Latino carpere.
i Contadini . ‘sili

  

zeseseEer: |

ae pee ee ee

— a
 

ees

=

fet

he

ete

RERt ES

DECIMOCANTARE: 481

2. Vedi fopra in quefto C. ft, 18. il termine santo di cuore,
NN git fanno fata, Non gli fanno male , 0 danao

‘TANTO

_ NON? ha 4 briga tocche , che Ube feoe . Subito , che ¢gli ? ha toccate gli pal-
fa il-doiore, non ftima 'e percoise . Quando i Cant hanno toccato delle baftona-
te fi {quotano , ¢ reftano di guarite, che ¢ indizio, che non fentono, O non cura~
no pid il doiore , ¢ di qui viene quefto fignificato di {quotere |e bufic , ¢ ne hab-
biamo il dettato Tw fai come i Cams, es’ intende cu (quoti le bufie , che fignificas
Non le cur: , non le fenti , non ne fai thma, ec. Vedi forto C. 11, tt 44,

MACT A. Con Vi longa . Monte di fatii dal Latino Adaceria,

Sl rimpiatra, Sinaconde , Vedi fopra C. 9, fh. 5,

_ dVia40.. iano « Latino nemo . Won fopra C, 7. ft. 89.
. £0 combatra . Gli dia noia;! impedifca,,

QVAN DL egis ha dato a’ Soccs la feonfitta, Quand’ egli ha meffo fortofopra, o in
contufione Je mandrie’, cioe fatti fuggire i bettiami afialtandogli: Che Socciv.s'in-
teade quel beltiame, il quale fi da a ua Contadino per far’ a mezzo del guadagno,
quafi dica a Sccio , cloe a compagnia. L’azione , che nafce dal contratto di So-
Gita , fi domanda da’ Legifti Azione Pro focio; Ma noi per Seccio intendiamo
una focieta , 0 compagnia particolare , ovvero una Accomandita di beltiame,che
fi.da altrui., perche lo cuftodi(ca, e governi » a mezzo guadagno , ¢ perdita . So-
Zi /poj pure dal Latino Sectas intendiamo quel , che i Latini diflero /edatis iures
Sodalitijs iunétus , 0 Buon forse dichiamo a colui , che non guafta mai , ¢ che acco-,
da le conuerfazioni ,

CA’ ei raccatta, Ch’ ei raduna , Ch’ ci trova, ¢ piglia,

CIVEF ANDO. Cioé¢ pigiiando con voracita ; rubando .

LU ritza, Cioe in quel luogo li, Termine ruttico , Dal Latino #i rea, Qui-
via diritto ; in quella dirittura , 0,, come 1 Franceli dicono, en cer endroit ,

10. fo penfiere ch’ e v? habia fatto a’ ofa un cimitero , lo credo ch’ ci v’ habbia ra-
gunato una gran quantita-d’ offa . Che Cimitero diciamo 1] luogo , dove fi forter-
Tano imorty. Vedi fopra C. 4. ft.2g.¢ C. 7. tt, 27,

STANZA ‘AxXVill
Sta Paride afenttrio molto attenta y

Ada pai vedendo quant’ ei fi prolunga

. Frafe dice; Coftui cs ha dato drento
Come quelche vuol far mela ben lunga,
Gli ¢ me troncargli qui il ragionamento

_ checio prima, che il ds mi fopraggiunga
40 polfa lafciar P opera compira ,
Peri gis. dice: O via falia finita .

STANZA XXXIX,

Poi ch’ egls ba intelo dow’ ei poffa bartere

e4.un diprefja 4 rinuergare il Tura ,
Lell'efer foleo il bofeose a’ altre tartére,
Che gli narri coftui , faper non cura:
La laterna apre,e il libro ,od'alcarattere
Poa, vedendo., dar’ una lettura,
Cost leggendo fenti darfi norma

Di quanto debba fare, in quefta forma,

STANZA XXxX,

Vicino al bofchereccio Scannatoia
Mentr’ il froco di ftipa vi riluca ,
Palton groffe, Bracctale ,¢ Schizzatoia
Co’ Gucators a palleggiar conduca ;

Ai rumbombar del fuo diletto quoia
Toffe vedra , che 'l Gocciolone sbuca
Keuei ricchi arnefi vago di mirare,
Che gid in Firenze lo facean gunfiare >

, Sta Paride attento al difcorig dei Villano ;ma conofcendo ch’ egli era entraco
ip on difcorio da non finir mai , lo fece chetare, ¢ prefo il libro , da cflo compic-
BEDS tate ;

fe quel ch’ ci doveva fare.

co.
 

 
    
 

482 MALMANTILE

COSTVI ci ba dato drento. Coftui & entrato in un difcorfo da non’
fine ; ¢ me la vuol far (unga , Ciok vuol far’ una lunga diceria,
OVVTA, E' \o fteBo , che ors . Latino’ Eia age. Termine, che
fpedizione . ms " i ast
DOP’ ci pud battere , Ciot da qual parte egli habbia andare per ir : .
Tura. Tey et
APN diprefso, Alquanto vicino a dove egli fia. Si dice 0 a ited “tid
vel circa, Dal dirfi per efempio : Furono tanti , quanti io v’ ho detto vel cireay
Cit , 0 in quel torno , haa alle fe
RINVEKG ARE . Rinuenire ; Ri ; Ri iare ; Raccapezzare. © ist
ALTKE tattere , Altre zacchere , minuzie , © circoftanze di poca confidera- a
vione . Se ben Tattere per (cherzo s' intende una fpecie di malore, che viene in 4/4
torno al fefio per crefcenza di carne . i ?
CARATTERE , La forma , 0 figura delle letcere dell’ Abbiccl. Voce latinas
tolta dal Greco Character , ¢d i) Monofino vuol che itia lio dir carartolo , ma fi
non fo per qual cagione , fe non fufle per allontanarfi dal Latino , che per altro te
non ho letto tai , ne fentito dir carartofo , fe non a qualche Villano del tutto ru- ne
fico. eee
SCANNATO10 , S) intende il luogo dove s' ammazzano i buoi, edaltrebe>
flie , ma qui intende quella felua , entro alla quale fi nafcondeva il Tura,¢las Ki
chiama /cannatoio , perché quivi i] Lupo fcannava le beftic ,
BRACCIALE, Manica di legno dentata , della quale s' arma il braccio pet laf
giocare al pallon groflo. Vedi fopra C. 6. ft. 34. any Ta
SCHIZZ AT O/O ( gui intende il piccolo). Strumento d* ottone , o d’ altro Ne
metallo fatto a foggia di canna da crilteri , ma aflai minore , e ferue per metter ‘
vento in qualunque luogo con violenza , come fi faa gonfiar palloni ,0 pillotte, wt
o per (chizzar liquori ; ¢ ‘i maggiore per far feruiziali . Latino e/yfer detto cosi , ,

quafi frumento inondante , ¢ lavativo . Vedi fopra C, 3. ft. 14. Che

PALLEGGIARE , Dare alla palla , 0 Pallone , mandindoio , ¢ rimandandolo Che
per tra(tullarfi , ¢ per avviare 1) giuoco ; ma non giocare regolatamente , Onde» Ry
quando uno tira ia luogo un neguzio , coll’ avviare chi glielo raccomanda , 2 un? Cad
altro , ¢ che quello lo rimanda , al primo , ¢ tutti due fi accordono a burlare il Tes,

pover’ huomo ; fidice + Tra loro fe ¢a palieggiane ; che in Latino forle i direbbe Giaj
Coludunt , ast SOE?
GOCCIOLONE . Si dice a uno , che ta guardando una cofa con grande atten- Tu

zione , ¢ con defiderio a’ ortencria , ¢ propriamente fi dice di quelli innamoratt » Beri
che ftanno i giorni interi appit d’ una cafa a guardar la dama, che € alla finellsa, Ng
¢ fi coniumano , € fi ftruggono a poco a poco , ¢ per cosi dire a filia a flilla, © vu
perd dice Gocetolone al Tura , ¢ vuol' elprimere , che egli cra innamorato di que L
guarnefi. Lucrezio lib, 4. Pariando deg!’ innamorati . ee fog
Wamque voluptatem prafagit multa cupido , Ri
Hac Venus eft vobis , bine autem eff nomen amiorit; - Ca
Hine ila primum Veneris dulcedinis in cor B g

Stilauit gutta, © fucceffit frigida cura, ?
CHE gid lo facean gorfiare, La yoce gontiare vyol dire Andar fuperbo , comes oy

4
ra
53,

 
 

DECIMO CANTARE.
dicemmo fopra in quefto.C. ft, 29. sed il Poeta (cherzando con l'equivoco di gon-

fiar

3 ma in effetto vuol

483

Ic pillotte , ¢ palloni , che era il mefticro del Tura , come acccnnammo fopra

Gf ft, 47. pare , che voglia dire , che quegli arnefi eran caula , che il Tura (eo

andava {up poi dire , che quegli aracfi eran caula ch’ei

J Sontava Je milous » ¢i palloni, ¢ che egli gonfiava Ja pancia, bufcando per mez-
zo

imi arnefi da comprar roba per empictia .

— STANZA XXXXL
Paride in soofe fatice ubbidifce 5
Accender fa le feope , ¢ intorno al fi
_ Gid quefloje quel ft {paglia ed alleftifce
 M faa braccialeye fi comincia il gimoco;
Al {uan del qual! Amico comparifee,
M4 ritenuto, perch’ e vede il fuoce ,
_ Elemento, che vien dali’ animale
Fugritaper inftinto naturale .
STANZA XXX AIL
NGarani che fava alle-velette ,
_ Fedendo che'd Compar viene alla cefta,
Che te feope fi (pengano commerte y
_ Edin ua tempo a i Giscator da fefta :
WD un batrer docchio il ginoco fi difmeste
La fipa fi sparpagha ye fi calpefta;
Tal che ficuro t' animal ridotto ,
Va Paride pian piano, e fa fagotto.
5

STANZA XXXXIII.
Cid ch’é in ginoco in nn fafcio egh ravvia,
E tra gambe la ferada poi fi caccia
A tutto firafcicanao per la via
Con una fune a otto, 0 dieci braccia .
Spinto dal genio a quella ghiortornia
Da lunge il Tura fcguita la traccia,
Come fa il Gatto dietro alle vivande ,
E il Porco a’ beveroni,ed alle ghiande.
STANZA XXXXIV.
Vaghecgialo, s'aliunga, xappa,e mugola ,
Talor 8 appre/sa,econlexampe iltoces,
Hor moftra shavigliando aperta l'ugola
Hor per leccarlo appoggiavi la bocca ,
Tutto lo fina, lo roniftia,e frugola ;
Cosi mentre il {uo cnor givia trabocca
Ej, che non rocea per letizia terra,
Entra nel Borgo, e in gabbia fi riferra :

TANZA XXxxv.

Perche Paride fa ferrar le porte ,
E poi comanda a un branco di Famigli,
Che quiui farti bauea venir di Corte,
Che di loy mano l' Animal fi pieli;

Ma i Birri,che bufcar temean la morte
Non voglion accercar fimil configli ,
E fan conto ( fe ben’ ei fa lor cuore )
Che @ paffi cutrania 2 Imperadore ,

 

Paride in ordine a quel che trovd fcritto nel libro datogli dalle Fate, fece acc
cender il fuoco d’ avanti a) bofco , ed attorno vi meffe gente a giocare a} pallo-
Ne : a quel romore i! Tura ulci dal bofco , ed allora Paride fece un falcio de’brac-
Ciali , pallone, ed altri arnei , ¢ legatolo a una fune lo fece ftrafcicare per las
fcada , Ja qual conduce al Caftello di Monte Lupo , dentro al quale i conduffe il
Tura, feguitando quegli arnefi , e Paride fece ferrar le porte , ed ordind ad alcuni
Bi tri, che quivi haveva per quefto fatti venire , che lo pigliaffero, ma effi impau-
titi aon yollero accoftarfi. ;

C4ALLEST/RE. Metter’ all’ ordine : Approntare

L) AMICO comparifce . Cioe i] Tura efce dal bofco,¢ vien fuora fpinto dal gu-
fto di vedere il pallone .

RITENVTO., Renitente ; cicé non alla libera , ma con qualche timore per
¢aufa det fuoco:, del quale i! Lupo n1cura'mente ha timore .

ST AVA alle velette, Stava offeru'ndo . Vedi fopra C. 7. ft. 67. It Burchiello
nella Novella del Medico Bolognef= . ¢ dello Scolar femplice dice : Andando ¢ri-
dando cerci tutta ia cafa , ¢ tronarlo non gli fu ordine , onde tratte dalla difperarione fi

Ppp 2 parti,

 
484 MALMANTELE :
parth , e lo Scolare , che flaua alle velette Vitorhato in'cafay ec,
. IL Compar viene alia cefta, Cioe Animale vien fuor
zimbello de i braceiali 5'¢ palloni 5 ec, iy HgOWs
DA fia ai Ginocatori , Ba veftardi giocare ; Licenzia iG)
agli Scolari vuol dir Licenziar la Squola , ¢ di qui dicendofi dar,
cenziare ogni forta di lavoro, Daag
IN un barter ad? occhio, Inun momento. 1 Latini pure ditono-Jr%é
SPARPAGLIATE , Spandere contufamente , ¢ ae ee
come fi fa della paglia , quanido‘fi batte , ¢ fi spoglia il'grano’. 1 Pulei dite:
Sopr' alle fpallela treccinfperpagtia., 0
FA fagotto, Fa un fa(ciode rbracciali , paltomt ec. Par fagotto ; ¢ 10!
quafi , che far le baile per bacterfela , per andarfene. Latino v4/a colligere. ~
Sl caccia la via fra gambe , Comincia a camminare . Latino + viem,
SEGVIT A Ia traceia, Seguita, 0 va dictro allapefta , oalla sed étol-
to dai Bracchi , i quali fi dice/egwitar /a traccsa , quando mel cercar della %.
ec. fiutando feguitano quella firada , ¢ quel tratto’, per dove ella ha tirato;
per dove ¢ paflata : di qui habbiamo il verbo inzracetare-detto fopra C, 7. ft,
BEVERON! , Cosi chiamano i -noftri Comtadini quella bevanda grofia fatta di
crufca, € d’acqua , ec, la-quale danno a i Porci. Vega eh aie
LO vagheggia, ‘Lo guarda aftewuofamente.. Sivaledi-quefto verbo vaghectiog
per efprimer il gulto , col quale 1) Tura guardava quegli arnefi ;:eflendo tal ¥
proprio degl’ innamorat', Vedi fopra C. 7. ft79. (4 aay
MVGOLARE, Buna voce indiltinta , ¢ che aon finitamuore fra i denti.
ROVISTIARE ; Ravoltolare , netter {oflopray Forte aeglio-romifia dal verbo
rovistare , che vuol dir Muovere da un ldogo all*aliro. Ji Pulci., Morgante vas
rouiftando ognt cofa.. Hh Wx
PER letizia non tocca terra, Sopra C. 9..ft.63, Per V allegrezza nom pud*ftar
n¢ i panni , ‘che € lo ftetio ; ¢ figaifca haver'aliegrezza’, o gufo grandiffimo ; Si
dice ancora; ma in modo batio, Lacamicianon gli tocca il federes Ml Boccaccio
Novella 32. iam
FAMIGLI , Qui sintende Famighi di Giuftizia,cio’ Birri;la famiglia debPode-
fla,dal Boccaccio detti fergents, quali ferxientes, ficcome'da noisfamigli,cic’ fa
FA conto y che pafii 'dmperadere,, Finge di-non intendere , o-di nom lentire qu
che fi dica\, Detto forfe quefto dal tempo’, quando'era I'dmper: rec
vanni Paleologo-in Firenzeal Conciiio »che per cferfi gia tata familiate la ua
vifta , ¢ forfe , mancandogli i danari., non comparendo:cosi pompola, ne Cos
bella compagnia ; ¢ appagata anche dalla prima volta in fu , lacuriofita; quan-
do paflava per le ftrade , non doveva far muovere la ‘gente come prima, ¢ come
ando egli arrivd ; Onde fi venne a dire, quasdo uno non fi cura di quaiche co-
f: Facciam conto , che paffi lo Laperadore , t $a7%8 one
: ST, AN 2, AoREXKEV End OVS
Poiché gran perso ha i porri bapredicate, Senza pin (har a burtear via il fiato,

 

  

 

E che fan conto tuttania cb'eb cantiv, Totti di mano abc. iiguanti,
Pero che.da i Ribaldi gli vien dato : Bifogna , dice , con quefia canaglia
Li udienz.a,che da il Papa i furfanti, Far come il Podeftdidi

‘AN:

 

   
   
 

 
   

DECIMO CANTARE: 485

ZAXXXXVIL ©» STANZA XXXXVIIL.
ds caps Si refta il Lupo, e’l Tura buomo diviene;
Ma non pero, che libero ne fia ,

    
   
  
     

 

ad una delle [ue legacce

    
 

a addoffo al’ Animale C’ ambi fone appiccati per le rene
eee a ufo di bifacce : Formandoun Leong ha ¢ la Bugia,
r di tal.concia dé cauiale Dice Turpino,e par ch' et dica bene,

 

Ch’ effendo quefta si crudel malia ,
ina di iupo,ed una d'buomo fembra, Lon erano.a disfaria mai baftani
di {ua [pecie oguunna ba le/ue mzbra, » Gli odor birrefchi femplici dei guanti ,
4)! STANZA IL
opri tal mafferizia E Paride , che gra ' chbe notizia
molto pin fatto le mani , Da quel {uo libro,fi da quint ai cani,
_ Percheglincants in man delaGinffizia Perché pin oltre il libro non ifpiega ,
i fichi- alla nebbia vengon van, ‘Ona’ et fa conto al fin di tor ia lega,
ide veduto che i Birri non ubbidivano , ed havendo per avvertimento dal
bro datogli dalle Fate , che gl’ incanti rimangon vani iv mano della Giuftizia ,
{diede a credere che haveffero tal virti ancora i guanti dei birri , ¢ per queflo
f eae al Caporale , ¢ gli mefle addoffo alla beftia , la quale fi conuerti
Induce corpi appiccati infieme , che uno d’ huomo ,¢ I altrodi lupo. A tal me-
tamorfofi refta Paride ftupefatto , ¢ non fapendo che cofa farfi , perché il libro
bon inlegna da vantaggio » rifolué di chiamar due fegatori per(eparar It Animal
bruto*dal razionale . In quefto moftro il noftro Poeta imita Dante nell’ Inf. C.
_ 25. nella commiftione di que! Serpe con ' anime di quei cingue Cittadini Fioren-
i € Ja delcrizion di tal moftro comincia al verfo: Se tu fei hor Lertore acreder
dente,
 PREDICARE #3 porri . Predicare al deferto,, Affaticarli in-vano a efortares
uno.a far bene ,che i Latini differo vento logui; Surdocanere.
_ PANNO conto ch’ ei cantiB lo ftetlosche dar Pandienza che da il Papa ai furfanti
che ia fuitéza vuol dire n6 fare ftima delle parole d’un0,0n6 badare a quel ch’es dice.
CAPOXKALE . ‘Capo di {quadra di birri, Grado che fi di anche fia i Soldati.
Vedi fopra'C.'9. flan. 2.
_ BAR come il Podefid di Sinigaglia , Ciot comandare , ¢ farda fe.. I) Duca di
Calauria Sigifmondo havea aflediato Sinigaglia,nella qual Terra era per Gover-
shatore foltituto da Gio: de Caftro , Petruccio Piccolomini ; Coftui tentd di ab-
‘ la Terra , dicendo efler. meglio uccello di-campagna , che di gabbia_ ,
¢d a lutaderiva il Podefta , ma i Cictadini featendo quefto differo di volergli get-
“tare dalle fiaeftre fe pit parlavano d’ abbandonare la Citta , ¢ vennero tanco ins
odio ved in difprezzo de i Cittadini , che. quando comandavano noa erono ubbi-
Giti, edi qui venne il Proverbio: Far.come it Poreftd di Sinigaglia , cioe Coman-
dare , e'far da fe, Cavalc. Scor. :
$ _Deeaa + S'intende quei Jegami , con i quali ff legano Je calze , cingendo
ambe .
MSACCE , Cosi chiamiamo due facchetti appiccati "uno contro all’ altto a
‘due cigne , i quali fi mettono a traver(o ai cavallo , ec. fopra il quale fi cavalca.,
'€ feruono per porcar robe , come fi fa con una valigia , (ono appellate —

  
   
  
 
 

me

 
 
 
     
  
   
    
 
   
  
 
    
   
   
  
  
     

vit

 

 

 
486 MALMANTILE

bis facche , due volte facche , 0 facche a doppio. Lat, AMdantica Bocce,
nov. 10.5, Haveva Frace Cipolla comandato che bea guardafle, che let
»» (ona ada toccaile le cofe fue , ¢ (pectalmeace le fue bilacce nelle q
»» cofe rare. B pil otto nella medeGa aovella. La prima cof che venac
» prefa fu la bifaccia , agila quale era la peana. weet
CONZI 4A. Quando Gi dice coacia di guaati s’ iateade profumameato ,

fi dice guanti di coacia di Rona , di Venezia » di Spagaa 7 ec. ¢ 8 intend
mati alla foggia di Roma, ec. Qui dice concia di Cauiale y cioe feteatt »
fragore , 0 feagraaza ¢ Detto ironico . , Were ta
LA Sugia. La Bugia Gi figura uaa Femmina con due facce differenti , comes
@' orfo 0d’ huo ny, o di lupo, ¢ d’ huomo , come é aci prefeace luogO,
DICE Tarpino, Scherza cone fa fopra C. 2, ftan, 31, autorizzando en

te (ua Novella com i detti di Turpino , come fa ? Aciofto. lve
MALIa, Iacantefimo. Suregoneria. Vedi fopra C, 8, ftan. 52. Donde 44-
liarda una ftrega . iy
T AL maferizia , Iatende i guanti del birro . cust cee
Sd aicani, S'adica, Quando uno per la ftizza grida , ¢ fa alere dimoftra+
zioni d’ impazzienza , 0 di rabbia diciamo; Si daa’ can, Vedi (opra C. the 10,
STANZA L STANZA LiL...

 
 

 
   
  

 

Per cid fatti venir due Marangoni y E morta re la dd per cofacertay =
Con tutto quell’ ordingo , che s’ adopra M4 quel Demonio infieme firappicch ,
eA fegare i legnami , edi panconi , E qual porco ferito agolaaperta

ef dinider il Moffro metre in opra ;
Mitre la fegaim mero ai dusigropponi
Scorre cosiva il mondo fortofopra
Mediante il rumor de i due parrienti,
Che un fa d' urli el altro dilamenti,
STANZA LL
Pur fenza ch’ inraccato elit habbia un offo
La [ez infino ail’ uitsmo aileefe
Lafciando il Tura libero, ma roffo,
Dietro ds fangue com’ un Genone/e ;
La Be(tia gli volea tornare addoffo ,
Ma Paride , che (ubito ? intefe
Prefa la (pada la cagio pel mexrd
Penfando di madarla un trattoalrezro,

  

aa
Per dinorarlo forte fe gli ficeay - -
Ed eslic! alt’ incontro flana all! eta y
Ja [a Latefta un fopramman gli appicedy
Ch in due parti diuifela di netto
Com’ una tefticcinola di capretto.
STANZA Luh
M4 ritornato a penna,¢4calamaio
Pur quello Heffo a Paride fi volta y
Che per veder il fin di quel mofeaio
See’ fulfe mai poffibile una volta y
Mena le man chee pare un Berrettait,
Ed a chius’ occhi pur fuonas r4ccilta
E dagli , e picchia,rifuona se mA 7
4a forbice , t ¢ fempre bella.

Paride fatti venir due Segatori d’ affe, fece fegare il Moftro in fu It artasatu-
ra deli’ huomo con la beltia , e cost gli fepard ; Ma la Beftia tentava di
carfi onde Paride caglid la Beftia pel mezzo , ma eifa prefto firappiccd 5 B qui
il noftro Autore immita I’ Ariofto nella favola d’ Orillo ; levata da Vergilioacil
Eneide , che finge un tal’ Erillo Re di Paleftrina che haveva tre anime , onde cra
neceflario tre volte ammazzarlo per finirlo a.

tHARANGONT, Si dicono i Garzoni de i Legnaiuoli che lavorano peropra,
quando in una bottega , ¢ quando in un’ altra a tanto il giorno , ¢ non jn
una boticga a falario di tanto il mefe; ma qui I’ Autore intende Segatori di le-

 

goami ,

 
a ese

—————

eee

me

=

- +e

 

 

 

DECIMO CANTARE. 487

| guaind 3 €gli ordinghi , che ? adopra , {ono la fega a due mani , lima per metteres
} Gags denti , ¢ il cavalletto per adattarui fopra quel materiale , che i dec (e-
ox. cavalletco fi chiama pietiche . Vedi fopra C. 6. flan, 6g. alla voce im-

_ PANCONT. Sono afi groffe circa un quinto di braccio , le quali fi rifendouo
per farne o affi pit fortili , che fi dicono panconcelli , o per farne correnti.

GROPPONE . S' intende la parte di poms di tutti gii animali, o bipedi,o

yadrupedi , ¢ lo diciamo ancora codione , ed ¢ propriamente quella parte che re=

fra le natiche , ¢ le reni . Vedi (opra C. 6, fan. 69.

VA fottofopra il mondo , Lo ftrepito confonde I’ univerfo . I Latini pure dicono
Mundi fumma readit ima , © ima fumma ;¢ vuol dire , che jo ttrepo cra gran.
‘dithmo per Je firida del Tura , ¢ per gli urli del Lupo .

_ ROSSO come un Genoxeje , &* in Firenze una Compagnia , 0 Confraternita di
Secoiari detta de’ Genovefi , perch¢ ¢ formata di gente di quella Naziwne s Co-
ftoro hanno per coitume d’ andar proceflionalmente Ja fera dei Giovedi Santo as
vifitare le Chiefe , fi battono le reni ignude con mazzi di corde entrovi alcune
ficiie di metailo acute come quelle degli {proni, ¢ quefle forando la pelle ne trag-
gono il fangue , il quale bagna loro Je reni , ele tigne di roflo; E di quelti in-
tende il noftro Pocta nel prefente luogo .

. tH ANDARE uno al rexzo . Mandare uno nell’ altro mondo , df frefco , ciok
il corpo fuo forto terra. Ammazzar’ uno . Rezo , vuol dire un luogo dove non
arrivano i raggi del Sole per interpofizione di che che fia , ¢ fidice anche , me-
riggiv , bacio , ombra ,¢ uggia . Vedi fopra C. 6. flan. 75 ¢C. g. ftan. 44.

ST.AV-A aif erta , Stava ocnlato ; flava avvertito . Erta fi dice la talita d'un

BRIO ; ¢ are all’ erta ¢ termine di caccia , percht la Lepre ha per propria di
for fempre alla volta della fommita de’ monti , per non efler cosi facilmente
arrivata , ¢ pigliando i {uoi ripofi , {coprir paefe , ¢ minchionarc icani ; ¢ pera

in caccia State al? erta s? intende Habbiate |’ occhio, ofieruate ; il che ¢

poi pafiato in dettato comune a ogni cola .

PN fopramman gii appicca, Gli da un foprammano , che é quel colpo,che fi da
¢ fpada , baftone , ec. cominciando da alto , ¢ calando a balio. Vedi fopras

5- ftan. 41.

D1 netto . S' intende lo taglid pulitamente in un fol colpo ,

TESTICCIWOLA, Le telte degli Agnelli , ¢ de i Capretti da noi fi chiamano
Teftucinote , ¢ per friggerie fi tagliano nel mezzo per lo lyago in duc parti ugua~
li; eda quefto taglio afiomiglia quello , che fa Paride alia tetta det Lupo.

4 penna ,¢ 4 calamaio, Per ! appunto. Vedi fopra C. 2, flan, 19.

VEDER il fin di quel mofeaio, Veder il tine di quetia cola noioia. Vedi fopras
C, 4. ftan. 9.¢C. 9. flan. 51.

MEN A le man ch’ ¢i par ux Berrettaio, Menar le mani dicemmo fopra C, 1, ft,
7. quel che fignifichi , ¢ qui intende che. mcnava le mani con ceierua,come fauna
1 Berrettai , ¢ Cappellai , che nel felcrare i cappelli , o berrette menano le mani
Prelto in riguardo dell’ acqua bollente , con ia quale fi fa tal lavoro,

_ SVONA a raccolra , Continova a perquoter a jungo , che cost fuona Ja campa-

Ra ; quando fuona a raccolta di popolo per le prediche , ¢c, ed 1 verbo fonare fi

° gailca
+

ee

 

   
   
  

- — “gk ae
2 488
a
488 MALMANTILE |
gnifica anche perquotere , ¢d ¢ della medefima natura, che il
habbiamo detto altrove. 6 ay eee
DAGLI, picchia , rifuona ,e marvella. Quefto di dire
re uno , che adopri ogni fea induftria , per fare una cola perf
do pitt vole le diligeaze . Vedi (opra C. 7. ftan, 16. Similitudine ,
tratta da’ fabbri , quando Javorano il ferro fopra I’ incudias ; Qui:
d' Orazio incudi reddere verfus , mettergli alP incudine , forto
critica . Cio¢ efaminargli , rivedergii di nuovo.con fomma , rigorofa
diligenza . La nottra maniera ; Barrere il ferro quando é caida , ebbe
meate da quefta prontezza , ¢ macitria talieme,che fi adopra per lavorat
nalmente |’ dcudir degli Spagauoli , che vale aixeare , voce ormai fi
é fatta dal Latino ddcudere , ciod battere infieme il medefimo ferro .
dichiamo per efempio . La prego a volere accudive « quefke megorio; © fi
FORSICE.. Quefto termine fignifica oftinazione,, per elempio. fo 2,
che tu non faccia la tal cofa; e tu forbice , cioe Tu oftinato I’hai voluta |
modo. Dicono che venga da uaa Donna offinata , ¢ capona,, 1a quale
chieito al Marito un par di Forbice , e non havendogiicie il marito mai:
te,ella ad ogni cofa,che i} marito le domandava rifpondeva : Forbice ;
impazzientato da queita {ciocca oftinazione,le proibi il dirlo
piu Jo diceva 5 per Jo che il marito la baflond, ma. non per: ella fe
maneva , ficche egii un giorno fopraffauto dalla collera Ja gewo in ump
cila fino che potette parlare fempre dite; Forbice, ed in ultimo goa p
valerfi della voce , fi valfe delle mani cavandolg fuori del’ aequa con Je
giori alzate ed allargate in figura di forbice,per mottrare che moriva |
oitinazione , ¢ caponeria . Quefta novella ¢ vulgatitiima fra le noftre
io ho trovata tra una raccolra di efempi facta da ua Buontempr
mano del medefimo tengo fra i miei nianofcritci . 2 eS
Lit fempre quella bella; L’ ¢ {empre quella medefima . Quefto yien da un
co , 1] quale andava accartando ,¢ cantava una cerca orazione al fuono di un
tarrino , fermandofi alle porte de’ {uoi benefators i giorni deftinati ;
venuco a faftidio, do fempre la defima cola, inci:

  
   
   
 
    
   
 
  
    
    
  
   
    
    
    
   
   

 

  

 
 

 
 

 

  
    

quelli , che gli facevano I’ elemofina a dirgli , che fe non cantava q ‘ae
orazione non gli haurebbero dato pil nulia , ed egii rifpandéva; Pa
se’, cht domani ve ne vaglio cantare una bella, Ma pecche il Povererto ai 4
fe.non quella , tornaya I’ altra mattina , e cantava la fteila , laonde i f ”
fattori.accortifi , che il Me{chino non ne. fapeva altre compathonaadolo , git te
cevono . L’ é fempre quella bella , ed intendewano |’ ¢ tempre quella 1 ig
che ¢ poi venuro in detrato , ¢ fignifica noi fiam fempre-alie medelim a
quanto racconto ancora fra gli {criti del medcfimo Bugnrempi top z
pucato ali’ origine del prefente dettato . ren “a i
S. TAN ZiAisbl Moni tap 'y¢ :
Tal ch’ ei fi {cofta none , e dieci paffi, Pervia gli anuenta m
E piglia fato , perch! es pronar vuvle, i
Selavirtude a forte gli giouaffi , i;

C* hanno! erbe , le pictre, ¢ le parole ;

 
  
     
 
 
  

489
gout STANZA _ i
recaffe a fcorno , Resta in parata’, molto gira il cnaré
alle gioftre,e alle quitaney | ‘Pimceis pikes anc ielibbienicfie,
we b gli vada incorna , > Merce ch ei fache'l Diauvloe bugiardo,
EB latrartigo’ faffi , come un cane ; “E quanto en fia furtile ,¢ filigroffo ;
i, ver ch' e' fufse ! apparir del giorno, oPercia fi merte un pezro a bellofguardo,
; L! Ombre,il Bau, ele Befane oCredendoognor che gli faltafse addofso,
Sparyce affatco, e pit non fi rinede, Aa poi ch’ ei vedde omas d' ¢/ser ficuro
Ma Paride per quefto non gli crede | = Ando all Ofte , ¢ cauollo di pan duro ,
Vedendo Paride , che quel Moftro fi rappiccava fempre » ¢ che ci non trovava
‘modo di liberarfene per ferite ,che glisdette, gli venne'in penficro , che fe era la
Werita 5 che in herbis , verbir, & lapidibus ftefle la virtt, poteiic eflere che alcune
di quette cofe havetie virti di fare {parire, ¢ {vaniresl Moltro ; ¢ pero prefo il
_ [xa dove , il quale era pieno di parole , ¢ dliverle erbe , € de i faili ogni cola tird
addotio a quel Moftro , ¢ l'indovind , perch fubito egli {pari , ed il Tura rima-
fe libero”, ‘Con tutto quefto,Paride non fi fidando , ftette buon pezzo a offeruare ;
ma veduro , che il Lupo non compariva pil fi parti, ¢ ando all’ olteria a man-

Patt. i
Ors ‘ fiato , Cioé fi ripofa ,
. MLAEST RO Grillo Contadino , 8 nota Ja favola‘di Grillo Contadino , i! quale
per fardifpetta @un fue fratello Medico sche non gli volle dar parte d’ ua tefo-
F0,che infizme! havevano trovato, fi fece Medico anch’ egii,¢ con i {ui forcuna-
a fiti's' acquifto la grazia del fuo Re, non folo per havergli: rifanata las
cavandoie una tilca di pefce della gola con ungerle ilc,... , ma ancora
per haver faputo indovinare i fegreti dél medefimo Re , ¢ chi erano coloro , che
 aluirubato hayevano, in fomma fece diverfe fcioccheric , le quali tutte per gli
} spares fidondarono in ftima del fuo valore , ¢ !’ accreditarono per un valoro(o
Medico , ¢ grandiffimo Indovino , come fi legge nella di Jui favolofa vita , 0 di-
Ciamo fpiritola Satira .
WINT ANA} Bruna campanella ,che fi tien fofpefa in aria (oftenuta da una
molla dentro a un canacilo , alla quale per infilarla corrono 4 Cavaiiert con las
Jancia’, come fanno anche’al Saracino , che dicemmo fopra C. 4, than, 57. € fi di.
Ce ancora Chintana , Varchi Stor, Fior, lib, 15. Fecera metrer delia rena a! avanti
al palazze , ed appiccare /a chintana, Dai noltri Ragazzi ¢ detta corrottamentes
Timana 9 ed ¢ iatelo quel lor patiatempo , che fanno , infilando una zucca frefca
in una corda , ¢ pottala in aria attraverfo a una Arada corrono con alle 1a mano
@ dare in detta zucca , unmitando i Cavalieri , i quali corrono alla quintana , 0
al Saracino , Dice che Paride era avvezzo alle gaintane, ¢ alle gioffre [che nel
Prelence inogo fon fitioninu ; {¢ ben gioftra’s’ intende quando i Cayaiieri corrono
a corpo a om 70 al Saracino, ¢ quintana fignifica quello ,che diciaino qui fo-
Pra) perche Paride haveva pil aout militato im Spagna , dove haveva cfercitaco
1 jor! gradi della mulizia, ¢ tornato alla Patria tu dal Serenityaio Gran Duca
fatco Governatore deija forcezza veochia di Livorno , ed hunorato del titoio di
Macttro di Capo , I nome tuo era Andrea Parigi , fu fratello d Aifon(o , ¢ di
Paoio detto fopra Papirio Gola , & Figliuolo di Giulio, ¢ fu come custi quelti va-
$a = Qqq jen-

     
 
   
 
  
  

 
   

       
 

   
 
 
  
 
   

 

-

 
   
   
  
  
 
 
 
  
  
   
 
 

Fe

:

©

i

=i

    
    
 
  

aa

 

 

 
—

‘ Ye
*

490 MALMANTILE™
lentifimo Tngegnere , € periti archi Qui
Ferrari cusi. Ludus equeltris ,cum diretta in encun fimulachrnn:
gehtat , bala incurritur , Alcunt han detto come Vguecione Pifano.s
zionario , che Gia cosi detta dalla quinta parte della piazza yin
tri , come Balfamone fopra Fozio da un certo Quinto inuentore 2ed
Ja vera origine mottra il Pertari eflere da Comrus.ciot ee i
punta di ferro ; ¢ fi raccoglie dabtitolo nel Godice:, de i Y
radore chiama quelto giuoco con voce Greca Kynranos., In ordi
Chintano , ¢ non Chincana pare , che lo chiamaile , fe sha a
ma , Fazio degli Vberti nel Dittamondo . :
Gionani bigordare alli Chintani y
E gran tornei, ed. una, ed altrag
Far fi vedea con ginochi nuoui se ferant. -\

  

   

jofira ‘
>

CALAPPOLERIE . Cofa di poca ftima : oda farne poco conto i “Apine; ‘

triceque ,¢ buttubata, V. Feito , ¢ ivi fopra lo Scaligerc.,
BAV ,e Befane , S' intendono quelle Larue inueatate dalle Balie per far paura
ai Bambini , come habbiamo decto fopra C. 2. ftan. 50. et
REST A sn parata , Si ferma in guardia , cioé con 1a {pada pronta , ed in pofi-
tura comoda a ferire , E’ termine da {chermitori . yori
MERCE’ , Con la prima , €;, firetta , ela {econda longa , vuol dir mercede
che profferito al contrario vuol dir mercanzia : Nel modo che:é detta nel pre-
fente luogo , ed in molt’ altre occafioni merc vuol dire per caufa di cid: qual di
ca io riconofco tal mercede , tal benefizio da quefta cofa , o da i,
ec, ficome Paride riconofce quefta mercede , 0 benefizio di non fi fidare del Dia.
volo dal fapere, che quello ¢ bugiardo, ed ingannatore . Queftoiderto ¢ lo'ftelio,
che Grazia del marcello , ¢ degli foroni , che vedemmo fopra in que(to C, fran, 20,
1L Diauoloé futtile , ¢ fiia grofo. 11 Diavolo & fagace, ed inganna P huomo ,
facendo il goffo , ed il balordo . * inet
REST Aa bellu (guards, Reled guardando attentamente . Bello fguarde® unas
villa poco lontana da Firenze: ¢ per 1a fimilitudine che ha quefto nome bella/enar-
do con il verbo guardare fi piglia in detto fignificato . pn amaetir
_ CAPOLLA di pan duro, Mangid adai . Gii mangid tutto il pane, che haveva
in cafa , gliclo rifint. Detto ufatifiimo per efprimere Aéangiare affai ee,
spy paler

ays Nae

FINE DEL DECIMO CANTARE . eae

 
  
   
  
 

 
 
   

 
  
 
      
 
  
 
 
 

 
    
    
     
  
    
 
   

 
  
 

|

 

eee
eh ARGOMENTO, ‘
St

Cangia le dance in rifsa un? accidente, aS
iSe

Fuggonfi Bertinella , ¢ Martinacza,

-VNDECIMO CANTARE,

Vien fuor Biancone , ¢ fa morir gran gente ;

5 Ma gli Orbi a tui fan poi fentir la mazza, 6%
es Da Celidora , ¢ da Baldon pofsente 33
ee Mezza defirntta ¢ quella trifta razza ; th

Taghanfi a pezei in quelle squadre , ¢ in quefte ‘“*
E cost in ata? fanfi le feffe . e a ge

2 —
RAPALA AAAS AS

om STANZAL STANZA Ik
Chi mi.dard Ja voce , ele parole ui ci vorria chi fcortica L’ agnello,
‘antia dir la guerra indiavolata ; Es al mondoé perfona pil inumana,
Ond’ oggimas dara le barbe al Sole © defcriver la frrage,ed il flagello
Bertinella con tutta la {ua armata ; Che feguir fi vedrd di carne humana ;
C'alCiel Gagliarde alzando,e Capriole, Ch’ io gid oni fento , mentre ne favello,
\Fard.verfo Volterra la Calata , A tremito venir della quartana.,
. Efe d' amor canto con cetra in mano, E n' ho si gran terror, ch'io vi confefi0,
<Derd col ferro il ve/pro Sicilsano ? Che mai piu de'miei di fard quel de/so,

Tinoftro Pocta volendo.nel prefente Cantare narrar Ja battaglia feguita ia Mal-
mantile, ¢ le crudcita grandi , che (uccefiero nel Palazzo della Regina , dice, che
a fac tale defcrizione vorrebbe efier un’ huomo fanguinario , quanto é colui , che
feortica git agnelli ; che non fi {pavencerebbe , come fa egli acl rammentarfi i]
grande firazio , che fu fatco di carne humana in tal batcagiia. Qui immuta Dan-
te-nel principio del C..8. dell’ Inf. che dice;

i Chi porrsa mas pur con parole fcialte
Dicer del fangue , ¢ delle piaghe a pieno
Ch! 10 hora vidi , per narrar pitt volte ?
4 mi lingua per certo uerria meno, L
— avventura feguita Vergilio nei 6. deli’ Kncid., che dice, imitando pures

°

Qqq 2 Non
 

 
 
   
 
  
     
   
      
     
 
  

492 MALMANTILE
Non mibi'y fi Sassen or ag
Omnia penaru ee omina polfem .—

E cosi rende I’ uditore attento ,  curiofo , col promettere di vol
venimenti cosi maravigliofi , che non ¢ per trovar parole adegu
ne efprimere . A! > bet: F e

‘DARA Ie barbe al fole, Morira ,. E’ traslato dalle piante , le qu
cioé fi feccano , quando fi fuelgono , ¢ fi voltano loro le barbe al So

GAGLIARDA, e Calata, Sono-duc {pecie di danza, ob
{cherza con la voce ealata., che vuol dir caduta , oftela, d
ver fatte qui Gagliarde , e capriole fara la calata,, cioé calera verfo
comunemente s* intende andar forterra , cioe morire . Jay +e

DIRA il Vefpro Siciliano, Dopo haver cantato verfi amosgh ante fj
Siciliano, che s’ intende ; vedra , € provera ftragi. B’ nora la follev ne de
ciliani (orto Gianni di Procida contro a i Francefi nel cempo,che quefti ti g
giavano la Sicilia nella qual follevazione fu il egno , che un determina gi
al fuona del Velpro ciafcuno fi moveffe contro a i Prancefi, come fe
fuccefle granditfima Mrage di effi Franceli ; E da quefto & nato il ‘
Vefpro Siciliano ; che vuol dir fare fteagi,ammazzare. Vedi Gio. Villani!
61.¢ Giachecto Male(pini nella Continuazione della Storia di Ricor
cap. 209, >

Hil festive P agetib + Sona tial yarenaisinmeeltaie
i quali nel tempo , che fono gli agnelli , vanno per Firenze gridando . Ch
fcorticar P dgnello ; per bulcar denari in ammazzare , ¢ {corticare Metti ani
il noftro:Poeta da quello (canaare ,\¢:fcorticar un’ intinica di‘effranil ,
puta huomini crudeli , ¢ fenza pieta , ¢ quefta per'accomodarfi abgenioy"e cap.
cita de i fanciulli , che ftimano quell’ atto una granditlima inumanita,
nando quelle beftiuole innoceati . ny sieht

FLAGELLO . Qui é prefo in fignificato di eopine, farts ee CA

   

   
    
   
   
 
      
     
 
 

  

  

di. Vedi fopra C.1. tt. 45. invaltro figniticato. In Gio. Villani trovafi nel fen ayy
ufato qui dal Poeta ; F/agello, ¢ Fragelio ; come coftuma di dire anche aug
piebe Fiorentina , ¢ come diflero i Greci , ¢ fi legge ne} tefto Greco dell® ; pac
fey

uy

dy

 
 

hio , Phragellion per quello , che i Latini dicono Fracetium Omcto
{grazia ,sferza , 0 2agello ds Givve vinci Node tibro 12) verlo 397%
831. Attila Re degli. Vani tu foprannominato per quelo , Frage! t
TREMITO dela quartana, Quci brividi , che G-fentono’ dal pazavente nell’en-
trare della febbre quartana , i quali fono aflai maggiori diquegli 5 che foglions fie
venire , quand’ uno ha qualche {pavento}-eperd ‘con dives VA tvemiep dela pias edy
sana , intende , che lo (pavento cra grandifima ,€ fuori dell’ ordinario : E «ali tend
brividi , 0 tremiti vengon’ allt huomo:, perthéla’patira fringe il cuore ; per lo

  
 

che il fangue corre tuctovin aiuto di eflo ;¢percio--membri efteriori, ¢ ie parti te
fuperficiali , ed cftreme rimangon\fredde ; edi fteddo facendo riftrit i pori, be
cagiona quel che i Latini dicono rigor » che farizeare i capelli , © pels "€ Cagio~ ni
na il cremito , il quale fi domanda capriccio , ¢ rsbrexzo, Vedi C. 6 Gig

MAL pit, de’ miei di fare quel defo, Spaurifco tanto, che efco

 
 

 

 
    
   

cro prima .

ets DAN ZALHI8.’

be il galio apportator del giorno
La notte nera pits d! un Calabrone,

Bil {ua buio,e quant'abre eli’ba dintorne
Diognise qualungue grado,e condizione,
| Acid ficuri omai faccian ritorno
\ Gli nccei, cantando il lor falfo bordone ,

AIncitr’al Sol,ch'in quefPa parte,e in quella
| Fa pel lor gorxo nafcer le granella
lead ety

 
    

Perché-crafeun » che quini fi ritrova ,

 

VNDECIMOCANTARE. 493

“fino a che viverd » non fard mai pid allegro , come era mio folito , perché quelto
- fpavento m’ ha fatto mutar compicifione , ¢ temperamento : Non {aro piii , quel

STANZA IV.

Quand’ infra Dame, ¢ Cavalieri erranti,
C’ al trefcone in Palazzo eran intentiy
Comprefeun dietro all'alero i duellanti,
Armati tutti due, come fergenti ,

-Si shallo il ballo, andar da cantoicanti,
Ele chitarre , ei mufics Srumenti
Ai proprj fuonatori , ¢ balierini
Divenner rante cnfie ,¢ berrestini,

STANZA V.

Si fa pero bifbigtio , ¢ fi rinnuous

_ Kedendo entrar quell’ armi coid dentro, L? odio fra te farion gid quafi {perto,
 Subirovdiffe : Qui garta cicacca: Che tirando ai rifpere: gu la bufa,

: =, £ trama di qualche tradimento, Ruppe la tregua , e rappicce la xufa,

4 iver la Jevata del Soley ¢ dice , che in fu quell’ hora entrarono nella ftan-
22, ove fi faceva il ballo , Martinazza , ¢ Calagrillo, che la feguitava con l’armi
F inmano:; per lo che fi lafcid flar il baliare , ¢ fi venne all’ armi., rompendo las

tregua , perché ciafcuna delle parti fofpetto d’ effer tradita, ¢ che quefto fufle uno
— militare , come i ditie fopra C, 10, flan.31. dove laicid quefti duel-
EL gaily apportaror del giarno sbandina 1a notte. 1) gallo ¢ folito cantare in full'ap-
pariridel giorno , ¢. perd dice ch’ eglié apporrasore del giorno , e che da 11 ban-
do alia notte col {uo cantare. Somniaque excuffit nuncia lucis aus , difle ua Poeta;

Excubitorque diem cantu predixerat ales , canto un’ altco , & erifta /pettabilis alta,

Auroram gallus vecat applandentibus als, Difle il Poliziano nel fuo Villano .

CALABRONE. E! uva {pecie d' infetto , o verme alato di figura fimile allas
mofta »maatlai pil grande, ¢ di colore ncgriflimo , ed ha un jungo , forte, e»
acutifime pungigiione. Con quefto nome chiamiamo,ancora il tafano detto fo-
Corot. 8. 1 Greci Prouerbilti ditiero fearabao mgrior , Pitt nero dello {cara-
B10, che ¢ un’ altra fpecie:di mofconaccio . i
4N comro.al Sole, Giivucceili vanno incontro al Sole cantando in ringraziamen-

to delbenefizio , ch’ ci fa joro , maturando le biade per loro alimento .

* GOZZO . E! il primo ventre degli uccelli , cloe quella vefcica , che hanno ap.

Ppit-del colio , dove fi ferma il cibo , che beccano , edi guivia poco a poco fi di-

Mtribuilce al ventricolo ; ¢ da noi fi piglia ancora per Ja gola dell’ huomo, perché

vien da gutrur . re

° CAVALIERI erranti , Cosi fon chiamati quei Cayalieriavyenturieri , che fon

defcritti ne i Romanzi Spagnvoli da loro detti Cauaheros andanes ; wa qui inten-

de , che erravano perche ftavano ballando aliora , che bilognava combateere .

“| TRESCONE , Specie di ballo , cos detto da Tre/ca balio anuco . Vedi iopra

G. 10. ft, 28, Dante Purg. 10.

 

: ee S=TePiie etre

= ee

a

i

 

 
 

  
 
 
 
  

494 MALMAINTILE® (0
Li precedewa al benedetto Vafo

Trefcando alzato , 0 umile S. ane cond
cioé faltando , ballando. M As +
SBALLO’. \\ verbo shallare vuol dire disfare le balle ; ma qui

re il balio, In buon Tofcano non fi direbbe shallare il dar fine al
pis la forza della lettera 5s , aggiunta al principio di verbo, 0
ignificato contrario si come la particella, i», appreffo i latini, |
tare , {piantare ; grariofo , fgrariaso , ec, ma il Poeta fe ne {
{cherzo di ballare , e sballare , e (eguita il bitticcio # dar da canto s canti
figuratamente sbaf/are , per eccedere la verita ne’ racconti ; ¢ © ¢
numeri di cofe con vantaggio , ¢ con caricatura . " *
DIVENT AR caffe , ¢ berrettini , ec, Cuffia , come s’e detto fopra C, 8, fh. 48:
una berretta fatta di velo , o di tela.a foggia di facchetio ufata dalle |
ferrar dentro i capelli in capo ; dice , che gli Prumenti vennero caffe ye
perché le chitarre , ed aitri ftrumenti fimill corpacciuti, eflendo bateuti in
capi di coloro , ¢ per la loro fottigliezza sfondandofi , fecero I effetto
be in ful capo la cuftia , o berrettino , cioé lo ricoperfero , e ferrarono in
E’ detto ufatitfimo . Ti faro wm berretrino delia chitarra , per intendere i
chitarra in fu la cefta. Vina timil frafe venne in capo a Omero nell’ Iliade, quan-_
do difle , Lapidea indui tunica , per voler dire, Effere tapidato', quafi il ricoprires
uno di faffate , fia uo fargli un veftito di pietre , che gli ftia bene alla vita.
GATT A cicova, Ci€é mifterio foro. Ci ¢ inganno, eum Tras tiled
i Latini. tet ain
TRAMA, Si dice quella feta , ec. , che ferue per riempiere le a
renza dell’ altra , che ferue per ordire , che fi dice orfoio ; che per la pid n
fi dicono ordito , ¢ ripieno. Dante Parad. C. 17. t {oi aged Rl hn,

  
   

  
  
  
      

    

  

    
   
    
    
 
  
         

Poiché racendo fi moftro [pedita Tat

L! anima fanta di metter la trama Che

Jn quella tela , ch’ io le porsi ordita , (SRE LS Sir

‘Ma trama Gi piglia per concerto , ene habbiamo il verbo tramare, cheiwuol dir bag
negoziare copertamente , ¢ forco mano, dilegnare,, concertare, Mraletrami ge ba,

fio affare ,ec, Bdicendo: Queffaé trama ds quaiche tradimento , intendes/Queho Oe)

  
  

 

@ tradimento concertato . Latino /ute/a doi. Varchi Stor, Fior, lib Cm
d’ una conuenzione facta fenza faputa d’ un terzo dice + Orazio fe ne? ada
rugia,fenzache il Sig, Gentile fufpicalfenon che fapelfe cofa alcuna di quefta' i
trama di gocciola per intédere {pecie d apopicsia ,quafi una coperta apoplethiaye da 4
quefto fi potrebbe intendere per rrama , uaa (pecic ; ¢ dire quefta é fpecie di qual i
Che tradimento. Storia di Scmifonte Trattat, 3. dice. 4 popolo fa fallewe 5 ¢ grida tha

na , [ufpwcando , che trama ui falfe , contro di lus, speotepecnh aot ¥

  

BIS BIGLIARE , Dilcorrer in fegretor, che fi dice anche Far Pith pifft; ij
Pifpigiiare , che usd Dante Parg. C. 5. Skit ise Bap w
Che fi fa cio , che quini fi pifpielia , “ ¥ they

E fi dice pi/pigio., ¢ pifpigiio , forta di cicalamento ; e viene da quel fafurrio , che: hi

featiamo da coloro , che parlano in fegecto.. toggi pia comunemente fidiceb® =
Soighiare , bifvigtio , ¢ bifrigtio, 5 te te
Th Ry

 

ae ae
 
 

  
 

na , O rifpetto
 STANZAVL
metre man da buon Soldato y
imico ritorna a Bertinella ,
f quale in quel punto cafco il fiato,
UM fegato , la milza, ¢ le budella ,
Vedendo , quando men’ hauria penfato,
_ Vicire i pefei fuor deta padelia ,
_ Mtentre 1a fa venir Adarte vighacco

     
   
  

Col fuo Baldone alle peggio del (acco .

STANZA Vil,

 

 

 

} VNDECIMO CANTARE.
| | TIRANDO git La buffa.a i rifpessi . Non havendo pili rifpetto , 0 riguardo al-
cuno. Sxffa intendiamo una berretta , la quale ¢ fatta a

f » € mandata gil cuopre anche tutta la faccia, ¢ i collo: Eda quefto
la faccia , mandar gis /a buffa., yuol dire oprare fenza riguardo , ¢ {caza

 
   
   
 

495

foggia di morione , che

STANZA VIIL.

Mentre 8 alcun t' offerua , ella pon mente
Per canfarfi enon effer appoftaca 5
Ecco in un tratto vedefi prefente
Martinazza la {ua confederata ,

Che poco dianzi anch’ ells fimiimeute

Di man di Calagrillo ¢ feapolata ,

E feco vanne in luoghi occulti , ¢ fenré

A fare wncanti , es faliti (congiuri ,
STANZA Ix,

eit a © un certo vento non le gufta, Nes quali aiuto ella chiede a Plutone ,
Che fa le (pade,e ognor per I’ aria fi{chia, Ed ¢i comparfo quixi in uno ifpante

wii] —- E.grd vedendo che (a morte aggiufta Dice , c' ba fatto a lor riquifizione
yee] Chipievnol far det brano,e pin starrifchia, Gid [pedire un tacche per un gigante
at Bel bello fuigna , ¢ vanne alla rifrufta Qual’ é quel famofifime Brancone ,
it | Dun luego da faluarfi da tal mifchia , Che col bartaglio,ch' era di Morgante ,
pt} —- Adtifchiayche non gli par di porer credere, Verrd quini tra poco in lor foccorfo
- Ee Percio fofpira , ¢ non fi puo difcredere , ef dar picchiate,e’ hanno a pelar  orfa,
& votes : f STANZA X. xi

Ed eccolo ( foggiunfe) ovvé battaglio\ E 8 anuedra ,c' al fin pifcio nel vaglio ,

© desi fo dir ych'il primo,ch' egit accoppa,
Tatra P armata a irfene in sharaglio
Che la barba penso farci di froppa ;

E che al pigliar un Reeno non é loppa ;
Cot fcaciata abbaffera la crefta
dn veder , che de’ {uci non campa testa,

Si rappicca la battaglia , ¢ Bertinella eflendofi perduta d’ anima , per vederes
i ritornato {uo nimico., quand’ ella penfava d’ haverlo tutto dalla fua, es
-temendo di non efler ammazzata in quella Foote » meditava di faluarfi in qual-
4 che ficuco , ed appunto-s’ imbatcé in Martinazza {campata da Calagrillo ,
J € con effa en’ ando in iuogo appartato a fare incanteGimi, per coftringer Plutone
F -ad aiutarle ; ed Egli comparfo quivi dice, che fi fara venire il Gigante Biancone,
il uals in quefto dire arrivO quivi , ¢ Piutone rincuora le donne con raccontare
la bravura di flo , dalla quale da loro per diftrutta I’ armata di Baldone .
LE cafca il fate. Si perde d’ animo. E foggiungeado: 4 fegato , la milza ye,
, te budelia , intende Si perda d’ animo affatto
— QFeANDO men fet é penfaro. Quando meno dubitava. Non expettato valvus
ab hoffe culit . .

VSCIRE i pefci fuor della padella. Perder quel ches’ era acquiftato , ¢ fopra di
che s’ era fatto aflegnamento certo , ¢ ficuro .

VENIR alla peegio del facco, Venire al maggior fegno di difcordia , e di rottu-
ta, Nelle guerre il peggior grado , che fia , € , quando le Citta ,0!' Armate fon
meffe a facco ; ¢ perd dicendofi /e peggio de! facco in peggior grado , ¢ condizio-
ne, che é haver il facco. VL

,

 
|

 

 

   
 
 
 
   
   
   
      
     
    
 
    
 
     
   
     
    
   
  

ee
496 — MALMANTILE © 7
VIGLIACCO,, Vile , codarda., EB voce {pagnuola , vells
fignifica furbo ,¢ furfante , poltrone. i
SEL bella, Con bella maniera , ¢ fenza dar © del
antichi differ ; bedlamente,manonéinufo..
SVIGNA. Se ne va con preftezza , o fugge. Forfeda quefto
viene ¢ omprare if porco , che vuol dite anch’ egli Andarfene
fuinam , 010% fuillans emere. Ed ¢ ulate quetto verbo fuignare
befco. Vedi fopra C. 4. flan. 51, Si potrebbe anche dire , come pei
erudito , che quefto verbo fuignare ligniticaado {cappar dalla Vigna , s°:
{cappare di foro Ja Vigna , ftrumeato o macchina milicare , che feruiva
tichi per andare (otto ie muraglic a combatier Je Piazze , con le quali”
difeadevano gli atiecianti da i (aii , ed altre cofe , che erano: buttace lor
dagli affediati , le quali necetiitavano quelil , che vi erano.coperti a
forto alle medefime vigne ; extra vineam exire , che (uona fuignare.
VANNE ala rifrufia, Vuol dice cerca mioutamente , ¢ con diligenza
NUN si pus difcreaere, Non pud non credere . Non pud creder , che
a cffer cosi , ¢ non habbia a eficre altrimenti. Non pud capacitarli
SCAPULAT A, Fuggita; Scuppata . 3’ intende {campato il pericolo
LACCHE', Ragazzi , cae corrogo appiedi per feruizio de’ loro
di fopra C. 2, fan. 29. 2 ae
BLANCONE . B' quel coloffo di marmo bianco., fattura dell’ Ammannato, il
quale ¢ pofto in Firenze nella Piazza dei Gran Duca , dentro a una valea gran-
de , la quale riceve !’ acqua da diverfe fontane , che fcacurifcono da detto: fo
¢ fuoi annetii ; ¢ fe bene rappreicnta Nettugno , ¢ chiamaco da cutta M Biancones — ui,
ai ee ray Vaca ; 1 hi
MORGANT E , 11 Pulci in un fuo Poema intitolato il Morgante narra’; che

  

  

   
    

 

  

Ms
quetto era un Gigante , 1 quale nog adoprava per coubattere alt’: he ua Ya
gran battaglio da campana, joe alo tf

PICCALATE ¢' hanno a pelar  orfo. Picchiate gagliarde , perché il) pelo dell’ Oh
orfo efiendo difficile a {uellere , ¢ pelare » non fi fa caicare con’ ky
fe leggieri, Pelare , wattandofi di muraglie , 0 pietre yuol dire-{pace; ol
fi , 0 (crepolare , onde potrebbe dirli hanno a peiare  orfo , cioe tare fore yi
rompere l’ orfo , che Gi dice quel pictronc , che adoprano gii fiyfaiuol FN
lire i piano delle ftufe , onde nabbiamo poi menar 1’ orfo.a Atoaan Pre
re ripulir Modana , ¢ Ggnitica mecterii a far una cofa umpolsibue uk

PENSO' farci la barba di Stoppa, S'intcnde ; E poi dargh tuoco.
Penso ingaonarci ,.¢ por farci ogni maggior danuo , ie
PISCIO' nel vagiio, Blo ftetio che far 1a zuppa nel paniere desto fopra C.
flan.7. E.che cola fia vaglio, Vedi fopra C, 2. ftan, 79. Luciano in ab
co volendo fpiegare , che il far bene a’ crifti ¢ come un tar Ja 2upp.
perché 1 benetizzi riceuti (cappano jaro prettifimo dalla memora; 4
buomo cattivo,e {conoicente a una bog forata , che uo quello, che va i met. tes,
te, fi ver(a. Plauto nei Pfeudolo , o vogiiam dire Bugiardello ; 2Vae piuris refert, ="
quam fi imbrem in cribrum geras , Corcifponde quefta maniera alia noltes (char
nel vagtio, Luciano nei Live dic ; come da in cofano forato, ©

ey

  
 

 
 

VNDECIMO CANTARE:

 
  

497

)zuppa nel panicre. Playto pure nel Pleudolo , Ja pertu/um ingerimus. dicta do-

opera ludimus. La favola delle Danaidi ha fatto luog
nifica non ¢ cofa facile. Loppa ; che fi dice

19 al prouerbio .

 
 

NON: ne Detto bafio , che
anche lolla ,
anche

ce » gli c levata.

a STANZA XL
Qui tacqueil Diawol,perch’ ¢ fatto rece,
“ él aria al capo git é maligna,
anuerzo 4 flar fempre nei fuco,
Vatea alle donne il dietroacafa,e/uigna,
EB lafesaus il Gigante nel fua law ,
Che douendo a Baldon grattar latigna,
 Sull ulcio det falon gid perwenuto ,
edge Hf batragiinje questo fu il faluto,
STANZA Ali,
Sei braccia era ti bascagtio aito, e ds paffo,
| Bm injragnena aimen arciotto,o vent,

  

4; Ma dando fu nei patcormando a baffo
cha _ Van trang intatiata , ¢ tre correnti,

; E fece tai frafiuano ,e cal fracaffo
14 Che shalord? « un tratto i combattenti ,
oh OE per pawra , a chi non fu percoffo

we |, Nomrimafe sr quel punto/anguc addoffo,

il gulcio , che fi leva di fopr’ al grano quando fi bacte, che fi chia-
inche pyle. Lat, apinde {econdo Nonio Marcello gramatico . 5

Y SCACIAT A ~ Rimanere fcaciato ; vuol dir Rimaner buriato , ches’ intendes
; nd’ ugo credendofi confeguice una cofa , ¢ facendolela {ua , 0 non Ia confe-

 
   
 
    
  
   
 

 ABBASSERA la crefta . Gli feemera!* umore , o I alterigia, I Galli d’ In-
dia , quand’ entrano in frenefia , gonfiano,, ¢ crefce loro la crefta , € patleggiano
on una certa intronizzatura , che par (uperbia ; ed ufciti di quella frenefia, fce~
ma , ed abbaifa loro ja creita ,¢ di qui vicne il prefente dettaco , che fignificas
readerGi umiie , contrario di Rizzar (4 crea, ;

STANZA XIII,

Ed infra gli altri Piaccianteo , il quale
S' era {chermito bene infizo aliora ,
Vedendo un fantoccion si badiale ,
Dopo il terror di tanre [pade fuora ,
Di quel derto farebbe capuaie ,

a9 C’ un bel fuggir faina la vita ancora,
444 perche in quae in la v'é mal rifcotro,
Vede hauer vifo di fentenra coniro. .

STANZA XIV,

Poiché non fa tronar modo , ne via
Per nellun verfe da [campar laguerra,
Ech’ ovis ¢ forza , che chi v'é vi ftia ,
Pond morto , gettafi gilt in terra,

E ritrouando Ja botrigleria

Apre t armadia , ¢ dentro vi fi ferra,
Con penficro di fiarni fempre occulto ,
Fin che fi quiet cost gran tumulto ,

! Plutone fi paite dalle Donne,e Ja(cia quivi il Gigante Biancone , il quale andd
, alla lanza ,dove fi faceva ia zuffa , ed arrivato in fu la porta alzd il battaglio ,
: per comigciar con effo a perquotere , ma al primo colpo dette in una traye , la.
quale per efler fradicia , fi fraca(so infieme con pill correati. Tal colpo fpauri

» tutti coloro , che eran quivi , ¢ particolarmente Piacciantco , il quale fino allora

8 era ben difefo , ma per Jo {pavento , che hebbe dei Gigante ,

getto in terra ,

s fingendofi morto , ed a poco a poco fi condufie all’ armadio della bortiglicria ’

b nel quale entrato vi fi erro.

 

fi #ATIOr«0, Divenuto fioco. Vno , che per catarro , 0 per altro impedi-
} mento aell’ afpera arteria ha perduta la chiarezza della voce , li dice rancus,don-
y de rancedine , ¢ reco. Dan, Int. C. 14. :

A, Erendele a colui ch’ era gia reco, £ .
(| Li aria glié maligna, 1? aria gli nuoce , gli cagiona danno ,

1, dietra a cafa ,¢fuignua. Volta le reni, ¢ fe ny « Bil verbo /ujznare,detto
rr GR.

Poco fopra neil’ ovtava fectuma , _

AT.

 

 
  

   
  

  

MALMANTILE® ©
S' initende perquotere . ‘Cosi I intende
. fo direi anche , maio temo, che'ella

ny Won s apparecchi a grattarmi la tigne .

Si dice anche cacciar 1a mo/ca da deffo , in quefto C. ftan. 20, !
dajfar la lana , fopra C, 7, ftan,63. Adandare a Legnaia ,fopra C.
ter Ta poluere , {orto C, 12. ftan, 1, E tutti hanno lo fteffo figni

INFRAGNERE . Ammaccare , 0 pigiare una cofa tani
forma., come farebbe Peftare un fico maturo , ec, ¢ il Lat. ¢/
Vedi fopra C, 4. ftan. 76, ¢ forto in quefto C, flan. 17.) *

INT ARLAT A, Rofa dai tarli , che fono quei vermi , li
dentro al legname.,, ¢ di ele fi nutricono ; da i Latim detti rer
'C.6. flan. 59.

FRASTVONO, Fracaffo . Sinonimi , che fignificano Romore , ftre

NON gli rimafe fangute in deffo. Acbbero cosi grande {pavento, che

mate {pirito, Dicono , che a uno , che habbia ha'vuro un granditimo {p
© paura , fe in quel punto gli fule tagliata una vena , non gliu
per le ragioni accennate (opra in guefto C, flan, 2, d

S’ eva Jchermito bene . Cio’ ,s’ ra difefo.. Havea {campato il toccatne

BAD/LALE , Grande, Si dice anche machofo , imperialc , € fimili ,
‘{cherzo ; ¢ fignifica grande pit del naturale. Kose ee
VN bel fuggir falna la vita ancora, Alla (entenza che dice Vn'bel morir tutta le
vita honora , rifpondono coloro , che flin:ano pit il vivere 5, oe

 
  
   
 
  
      
   
   
   
    
   
    
  

Sa ney

bony

         
  
    
  

Vo bei fuggir fainn In vita ancora, 7 ag
V" é mal vifcontro, V' é male il modo. Non W'@ buona congiuntura, ~ io
VPEDE baker vifo di fentenza contro, Conofce di non‘haver ragione , ciot,che il Mt

‘ncgozio non é per feguire , com’ ei vorrebbe. A tthe? Wy
CAl ve vi fia, Chi ha havuta la difgrazia , fe la ianga: E fi dice: Obi v'é ui

vi fia, ¢ chs nén 0° é non v" entri , qui perd intende ; chi in quella ftanza viftia, tt

perché aon fe ne pud uftire . Reet eee
BOTTIGIERIA, Armadio ,o flanza , ove fi tengono V afi da Vino ’ a)

¢ feruizio della menfa . Voce , che vicn dal Francele Borteille , che e Cl

‘fiafco , 0 altro vafo fimile da vino . 4 4

STANZA XV. 7

‘Col battaglio di nucuo agile , ¢ prefto © gia ch’ egli non puo tt

Tira il Gigante 5 ¢ da nella lumiera’, etrmeggiar col bat: ‘et lento y f k,
Ls qual cadendo fece del fuo refto , ‘Pero che il toga non ba gran diffanza, fr
Perché i [penfe , ¢ rope cid che v' era; “Cagion ch’ ei trowa fempre' mento; a
Hor , 8 eglie in beftia,dicanelo quefo, ‘Lafeialo andar bawendo pin fidana x
Mentre ch’ ei da ne’ lumi intal manera, Nelle fue manych’ in fimile Srumenb hei
E dice che’! Demonio lo jtafila, E piglia guells ciurma abbietra,e sbricia a
Poiché eli fa faltir due colpi in fila. ‘eA4 menate 5 com’ anici im camicia yj
‘ STANZA XVIL . oe ee fg
‘Cosi tutto arrabbiato , come un-cane Talche'l me/chin non mangera pit par ad
Piglia un pel coho ,¢ feactialo nel muro, Percid gli amics [uci , a
Di forta , che disfatto ei ne rimane “We voglion , che il ribaldo,
Som’ wie ficaccia piattalo matures Gli andaron alta-vien tush quant

Stet a. ac? ae
 
     
    
  
   

VNDECIMOCANTARE. 492
STANZA XVIIL STANZA XIX.

"sion cofforo un brance di galletti , E come la mia Serua, quana’ in fretta
Quando la fhate, a tempo di ricolta, Dee fare ilpefce a uovo,e che fi caccia,
Antorne a qualche bica units, ¢ fpretti Trama due nova einfigmele picchiet:s,
nun di loro a berricar s' afolta, Sicche in untempoturte due le(chiaccias

Pere il Gigante fa certi feambierti , Bs che dall’ tra ¢ [pinto alia yenderca
Che re ne [uifa quattro,o fe per volta; Softien quei due,es' apreneliebraccia ;
Infaffidico al fin da quel baccano , Poryciacche,pacte infieme quello,e quefie;

Si china,ed aggavignane un per mano, Stcche e diwentan prit che pollo pefto.

__. Biancone con un coipo fracafia la lumiera , ¢ {pegne tutti i lumi. Nota che,
i _fe bene era di giorno , la lumiera era tuctavia accela , il che {peflo aveiene in ta-
lioccafioni di veglie y che i segiivorl diftratti dal gufto del ball ,fanno mezzo
— fenz’ avvederfi , che fia pafiata la notte, Ll Gigante in collera Jafcia il
ttaglio , ¢ comincia a pigliar quelia gente , ¢ bacteria per le mura , onde tut-
tian tratto gli corfero addoffo , ma egl fi difendeva , facendo di loro ua gran
-maccilo .
LVMIER A, EB vno ftrumento , col quale fi {oftengono in aria pitt lumi acce-
fi, che i Latini dicono Lychauchus penfius , luceraiere in aria .
FECE del fue refto . Far dei retto s’ intende fipire ja roba, la vita , ec. qui dun-
que vuol dire fi fpeafero atfatto 1 lumi . <
B in beftia, B in collera., Dar ne i lumi, vuol dire entrar grandemente'ty col-
Tera , dar_ nelle (candefcenze ; ed é Jo fteilo che dar nelle furie , ed il Poeta (cherza
_ con quefta metafora di dar ac’ lumi , ed intende dare etfettivamente col batcta-
- glio ne i lumi della lumiera .

ail ; 4L Dianol to feafiia , 11 Diavolo lo perfeguita ; Gli ¢ contrario.

IN fila, Vo doppo I’ altro , fenz’ intramezzo .
ot CARMEGGIARE , Quefto metaforicamente fignifica Aggirarfi, o affaticarfi in
ibs vano ; ¢ fignitica anche ingaonarfi, per efempio : Tu armeggi , fe tu (peri d’ ot-

tenere , ec, ma qui ¢ prefo anche nel {uo proprio figniticato di mineggiar lara;

gh Cnell’ altro d' aggirarGi. —

wo) CWRMA. Genraccia vile. Vedi fopra C, 3, fan. 76 ¢ C. g. flan, 16,

ABBIETT A , ¢ sbricia, Sinonimi,che figaincano vilitfima, minurifsima gente,

A manate , Da i pili fi dice menare . Quanti a’ entrano in uaa mano; ¢ per la
grandezza della mano del Gigante fuppone il Poeta , che fica moltiimi per vol-
ta, perche dice: came anici sn camicia , che fono anici coperti di 2ucchero , de i
quali con una mano fe ae pigliauo le centinaia . $

FICO piattole, E’ una {pecie di fico detta cosi.

NON voglion ch' ci fe ne vanti. Lo voglion gattigare , perch’ ci non s’ habbia a.
gloriare d’ hayer ammazzato quel loro amico .

» . BlC-AQuafi da il Lat. Barbaro apica dal buono -dpex. Cosi chiamano i Conta-
dini quel monte di grano in paglia a mazzi , da loro cosi accomodato , affinché
fi flagioni , pec poterlo cavar dalla fpiga ; deta da 1 Latini rrieict congeries. Das
quefta voce bica habbiamo il verbo sdbicare per accamulare . Dante laf, C. 9.

Come le rane innanzi alla mmica ,
Bifcia per I’ acqua fi dileguan tucte
Per ¢ alla terra ciafcuna s abbua, Rrr2z~ BEZ-

SEE CERCA ES

 

 

 
 

 
  

500

MALMAN TYLER: 1 v

BEZZIC ARE , MW beccare'de i pollaftrelli fi dice bezs
FA certi feambietti , Ciot contraccambia Je percofie ,

  

ra

Scambietto * termine di ballo , che fignifica mutanea’
INF AST IDITO da quel baccano, Klicndogii v«

  
 
 

 
 

fi
fopra C, 4. flan. 9.

 

 
 
 

 

Allor Bieco non ba pite fofferenza ,
E giura , che di queftot: Bacchillone
Von andra al Prete per la penitenza,
Perch'ei vnol,chee' la faccia col baffone;
Ei fui, che di ral arme ban da teenza
Gite ne daran a una fanta ragione’s
Cosi guida i fuvi ciechiyow' ¢ il coloffe ,
Accto gli caccin le mofche da defo.
STANZA XXL.
Eglino tutti quini fermi a tiro
Preffo.a Biancone aun fifcbioco' baitoni,
Senza tramexzo alcun , fenza refpiro
We diedero un carpiccio di queé buoni,
Ed egli con un piede alzato in giro
Fa lor fentir , s' egli ha fodii talloni,
E mentre quefto paffa ye quel rientra,
‘Con quel pedino te li chiappa,e /uentra,

 

‘Bieco veduto quefto fa vehire-i fuoi Ciechi,i quali tutti in giro ini

Ja importunita. La voce baccano,che fignifica combat esett
piglia nel fenfo,che fi piglia mufica , felta » bordello, '

Quand! ecco rt veccbio Paolino

 

 

Ve anti

AGG AVIGNA, Piglia ,¢s' intende cinger con Ja ‘mano ‘tu

glia, in manicra, che fi poffa tenere ftretto con factiita,
PESCE a’ weno, Vova fritte »0 frittata , che dicemmo fopra C. 9
s’ intende propriamente la frittata , che dopo eer cotta 5 0
ruotolo , pure nella padella; rifritca, ¢ ridorta in figura “di p
ta pefce d'uono, La Compagnia deila Lefina dice : La 'consner
antichi , i quali conrenti a’ un pefce d’ uouo di due woun al pile

ClACCHE . Quefta parola non ha verun fignificato’, ma folo
no , che fanno I’ uova , ed alere cofe fimilt, quando fi rompono , edil
ne ferue pr efprimer quel bateere, che fa il Gigante di‘quei due hi
tr’ all’ altro, ed immita Dante , che nell’ laf. C,32,dice:
LVon hauea pur dail’ orlo fatto Crich
E feguita i Latini , che pure ‘hanno a finta voce Tax,
come fi vede in Plauto in Perla ; dove per intender buie dice > Tax
meo. E noi pure diciamo'tach , ¢ pach ; anzi le percotie da molti in F
cono pacche , come dice anche il noltro Poeca fopraC, 5. ft. 47. Da
ta la parola Fiorentina dcciaceare,, che ¢ lo ttetio , che’ Pefeare
dicefi ‘Pepe acciaccary ; modeftamente infranto ,e Acciaceo {opi
do uno per cosi dire calpefta ,¢ maktratta un’alero., ”
5 j

3¢ per

  

la hile elbétae ,

a ;
STANZA XXN

Aquat fa pits cagon,cblT efti,e'
E ( perchegti e bizzarre) bam
Condotti com’ ei fuole,un par a

OveSalito a Petigion di

Vavol matel,ch'egis' ha de ee

T aftando,owe il Gigs

 
  
 
 
 
 
 
 
 
   
 
 
 

£ darel ccc ieP bocca

 
 
 
 
 
 
  

SEafi Fi =F eo &

      
      
   
       
     
     

 

oy
Wes,

ee
a

aie

 
  
      

VNDECIMO CANTARE. yor

affaltano co baftoni , ¢ Paolino falito fopr’ a i {uoi trampoli metie i) {uo
iuolo {opr’ alla faccia-di eflo Biancone , i] quale perd s' adira , ¢ beltemmia

i {uot falfi Dei. Pah
| BACCHILLONE , 0 Bacchiglone, E nome d'un fiume , che paffa dalla Cita
| Vicenza , in Latino detto Azedoacus minor (econds Fra Leandro Alberti ; ed ¢
ida Dante Inferno 15.-ove difcorre d’ uno, a cui fu permutato il Velco-
irenze in quello di Vicenza , che dal feruo de' ferui Fu trafmmutato d’ Arno
one. Da quelto fatto di Mefier’ Andrea Mozzi , che cosi fi domanda-
Vefcovo, o pure dal verfo di Dante nacque in Firenze il proyerbio ; del
fanno teftimonianza il Varchi nell’ Ercolano, ¢ il Borghini . Sacare d’'e4r-
in Baechilione , aitudendo al {aito dal Vefcovado di Firenze a quello di Vicen-
y che fignifica faltar d’un propofivo in un’ altro s Saitar ai palo i frafea: Ma
-quelta voce Bacchillona aggiunta a huomo fignifica huomo infipido , ¢ buono 4»
. oe » ancorché di perfona grande ; ¢ fuona lo tteflo , che Gaicone, Palamidonc,
i: » ¢ fimili ,.¢ credo , che fia il medefimo dire a un! huomo Lacchillone ,
scheCaftrone , e che venga da Bacchio , che in alcuni juoghi di Tofcana yuol dire
we — agnello,e cos: Bacchi/one voglia dire agnelio grade,cioe Caffrone. O pure viene dal
© | Lat. bacuius,quati Perticone , Scurifcione, O vero & deo quali Baleceone; che fi
»¢€ non fa niente dibuono , ne di ferio .

WON andra al Prete per la penitenza . Quelto modo di dire ufiamo per fare in-
‘tendere , che ci vogliamo vendicare del oprufo , 0 torto fattoci, o che yogliamo
galligare uno di qualche mancamento commeffo ; quafi diciamo: lo medefimo
i dard Ja pena di quefto {uo fallo , (enza che egit vada per efla al Confefore sed

il Poeta l’ e(prime dicendo : Perché vnol , ch’ ei la facia col baffane.,
| AIANNO ficenza-di porter tale arme , Cioe hanno permiflione di portare il ba-
it {cherza , peso ivciechi portano il bafione per necefita , per farfi lan

  

  
   
 
   
  
    

  
   

 

 
 

    

 
 
   
     

QW VINA fanca ragione , Gli daranno le'baftonate ,.come vanno date, ¢ quella

pi |  WoCe Sama, fe ben pare riempitura per emfali , nondimeno detta in quefti termi-
sf ablignifica perfezione , quafi dica divera , ¢ di tutta ragione , ¢ d’ intera giufti-
a Zia, che la voce Sanétus fiacopata da Suncitus vuol dire Nabilito , determinato.,

» Nov. 10. £ battnrala adungque d’ una fanta regione, cioe.con una folenne ma-
niera ; dateglicie delie'buone . Vedi I’ Orava 25. feguente..

GLI caccino le'‘mofebe da defo, Lo battonino. Vedisopra in quefto C. ft, 11,

SENZA tramexzo , ¢ fenza refpiro, Senz’ intermiffione di tempo , ¢ fenza pi-
igliare ripofo .

NE dettero un-carpiccio di quei buon, Ne detterouna buona ,'¢ gran quantita.
Carpiccio viene dal verbo carpire ,-¢ pero yuol dire. manata., 0 manciata, ¢ cence
Aeruiamo per intender quantita., ma per Jo pili di bufie , come!’ intefe ilFiren-
-2uola neil’ Afin d' oro + £ pofcia , che per nua volta gle x’ hebbe dati un-carpiccia de

i

TALLONI + Quella parte del piede ,che ¢ tra Ja noce., ¢ il calcagno,:ma qui
‘piglia la parte per cueto il piede. Vien dai Latino Tans. C. 8, ft..69.
_ PEDINO, Deito ironico., ed.intende gran picde , pedone ,

SPER

 

 
 

 

    
 

goz MALMANTILE

SVENTRA. Rompe , fpezza , 0 sfonda il ventre ,
attivo , che fventrare neutro ha il figaitca
PAOLINO Cieco . Quefto fu un Cieco compo!
zonette, le quali fi fentono ancora cantar per Firenze da al
azzi , ¢ per quefto il noftro Poeta dice: Fs pil canzoni , ch
oeti celeberrimi del noftro fecolo . Tali fue canzoni anda’
le piazze , dove per adunare il popolo faceva fare diverfi
cani , ed egli medefimo , benché affatto cieco , ¢ decrepito, 2
trampoli di legno a i piedi, Queftitrampoli erano duc pertiche y in
ciafcuna,delle quali era fitto un pivolo , ¢ fopr’a quefti dae pivoli falis
fopr’ ad effi i piedi , ¢ foftenendo la perfona col rimanente di de
con adattarfele forto le braccia,camminava con granditima franchezza
poli da’ Latini fi domandano Graiie , ‘ccondo Nonio Marcelle ; ¢ quei,
minano {u’ trampoli , Gratlatores . Feito dice ; Grattarores i
ni, qui, ut in faltatione tmitarentur agipanas , adiettis perticis furculas h
que in bis fuperftances aa fimilitudinem crurum eins generis gradiebantir
prer diffcuteacem confiffendi, Plauto Vinceretis curfuceruas ,© gallatorem.
D1 cento fcampolt , Tutto rappezzato ; che /campole Jiciamo quel pezzo d
no , 0 drappo , ec, che al mercante avanza d’uua tela quafi pezzo,cosi
pato , cioe avanzato a far’ un' abito 1nccro ; ¢ qui intende toppes 0 pezei
anno. ere a
. (MB ACVCC ARE . S! intendé coprire il capo ,¢ ilwifo .. Vedi
fi. 73. Varchi Stor. Fior, ub. 1.4 Subso fu prefo ,¢ smbacnceato col eapp
dotto alle carceri,
Sl feandolexza, S' adira . Vedi fopra C.
di fcandolezzare ¢ quel , che dicemmo fopra C.
BREZZA, Vento freddo ; Vedi fopra C. 7. ft. 18. ue
PAbP AICO. E' un pezzo di drappo incre(pato da una parte, e ridotto quai Ht
in forma di facco , quale portano in capo le donne per difenderfi freddo, ed ‘afl
oggi lo chiamano anche cufia, Mattio Franzefi in lode delle Malehere dice ¢ » all

  

  
   
 
 
 
     
    

 
 

 
 

  
 
    
  
     
  
    
   
  
  
     
   
   

£Lvvi un fegreto, che a noi dir fi puore , vet ~ Yay
Che la mafcheraé me’ a! un pappafico , fi
E pero si vente in van. cufola , ¢ [quote ty
Ed il medefimo in lode delia Potta uso il verbo impappaficarft di aay
Chi ale tempse fi fafcia gli vechiati ‘ake
Chi fopr’ a i berrettin impappafica, ine
PORCO, Aggiunto a huomo vuol dire Schifo . ps a
0740". Intend , Che schitezza ¢ quefla? Vedi fopraC, 8.67 yy
ALLEZZA, Vedi fopra C.-3. ft, 64. & wota,che il verbo allezeare tantoat =”
tivo , quanto neutro ha Jo ftetio fignificato , ; 3 sur (oy
SA di refe azzurro, Per tigncre in azzurro adoprano i Tintori ere
fetore orrendo , o fia galla , 0 fia guado’, 0 uno , 1’ altro infiemes 1M

rimane per qualche rempo in fu la roba tinta , ¢ particolarmence in ful 1in0
pero dice quel cenciaccio fa ai refe azzurro, ed intende. Ha gran fetores'
verbo appeftare ha lo fictio fignificato , ¢ natura , che ha il verbo 4
di al detto C, 3, ft. 54.

bee

    
   
    
 
    
  

STANZA XXIV.
levare intanto hawea Perlone
| La srane dal Gigante roninata ;
Abe ancor quini ciondolone,

he la lumiera gid tenea legata ,
“Ed 4 foggia d! eAriere , 0 eMontone

7 nla addietro , e dannole l' andata

- Verfo quel torvion , che fi diftefe ,

   
     

> STANZA XxV.
Hor’ quando ( perch’ egls sbalordito,
~ Etutto intenebrato in terra giace )
 LCieehi pik che mai fanno pulito,
_ Edegli fe le piglia in fanta pace ,
OB fra le maxe innolto.a quel partite
Vn facco diventato par di brace ,
© Eben quel panno al vifo gli é dovuto,
—— Dovendofi si-cappuecto aun batturo,

 
  

   
       
  
   
 

 

lo fteffo fignificato.

= =.
—

orribili Giganti.

 
 
 
 
 
 
   
 

| Col si pile voite in bocca del Franzefe, Perché quivinon é troppo-buon’ aria.

 

VNDECIMO CANTARE 503

3 TI uno! dar P incenfo con le pera . In vece di farti honore , ed incenfarti , voglio

Reece > offerendoti cofe puzzolenti , come fuol’ efler il peto , del quale. Ve-
| di fopra C. 6. ft. 100, Orazio. Vin tu Curtis Ludais oppedere?

STANZA XXVI.

Mentre gli rompon Poa, € poi gli fanne

Cosi t incannucciata co’ randelli
E talor , non wedendo ove fi danno ,
Si tamburan fra lor come vitellt',

 

Gli altri foldati a gambe fe la danno ,

Ed ognun dice: alla larga seabells ;
Euege la parte amica,e la contrariay

 

STANZA XXVIII,

Ma reftin pure a rinfrefcarle gli orbi ,

Con quell? snfalatina di mazzocchi ,
Ed et ripofi all? ombra di quei forbi ,
Che gli grattan la rognaco' lor nocchi
Mentre quivi per far difpetto aicorbi,
Sotto quel cencio tien-coperti gli ovcht 5
Che sugnun parte,ed io mi partoacoray
‘Pen tornare a Baldone ,¢ Celidora :

   

~ Con inucazione , ¢ macchina di Perlone , il Gigante ¢ atterrato , ed i Ciechi
‘gli vanno tutti addoflo col baftone , ed in quefto grado Jo Jafcia il Pocta , ¢ torna
‘a dilcorrer di Baldone , ¢ di Celidora .

CIONDOLONE . Vna cola, che fla pendente da alto a baffo [enz’ effer ferma
‘in verun’ altro luogo , che dove é appiccata, come farebbe il battaglio.nella cam-
. » fidice far ciondolone , 0 ciondojoni dal verbo ciondolare, come dal verbo pen-
Gee fi dice pendotoni , 0 penzoloni; da dondolare., dondoloni, che tutti hanno quali

ARIET E , 0 montone , Macchine ,'0 ftrumenti bellici antichi , de’ quali fi ferui-
| -vanoiper rovinare Je muraglie ; Sono notidimi., parlandone tutti gli Storici La-
“tin; ma particolarmente Giulio Cefare ne’ {uoi comentarj .
quel sorrione . Cost & chiamato dal noftro Poeta il Gigante , perché
avanza fopra gli altri huomini., come avanzano i torrioni fopra lemuraglic ; ed
anche perché fervendofi dell’ Ariete , 0 Montone , lo deve adoperare, non in un’
huomo , ma inuna torre , come é folito adoprarfi fimili arnefi. Da quefta gi-
‘gantefca flatura , per Ja quale.effi {ono affomugliati alle torri ; fece Dante il ver-
‘bo Torreggiare afiai galantemente . Inf. 31. Vorreggiavan dé mezzra ta perfona Gli

* COL si del Franzefe in becca .'Gridando®: bud , but, che voce dimoftrativa di
p| dolore , ed in lingua-Franzefe vuol dire si. 4

¥ - SBALORDITO . Siordito , fuori del fentimento'per le percoffe ricevute..
¥) © INTENEZRATO, Si pwd dir finonimo di sbalordito: ¢ qui vale per intormen-
i * ‘tito daile percotie . Vn fatio , muraglia , 0 altro fimile materiale folido , ¢ dura,

 

 

“Ai dice intenebrato , quando, per le peccolic., che fe gli danno per romperlo., ¢ 4i-

Bio,

 
    
     
   
      
      
    
    
   

504 MALMANTILBE. | ¥

dotto in termine , che dal fuono fi conofce , che fi comincia.a
F ANNO puiite, Vuol dire Ripulire 5 ma detto in quefti te
da vero , o perfettamente ; E' Jo fteflo , che Far_di buono detto fo
SE le pigia in fdnca pace. Se le piglia con tutta, ed intera quiete . Ci
baftonare , e non fi rivolta , ne's'adira. E la voce Santa ha la forza,
detto fopra in quefto C, ft. 20. ‘ olay
KINVOLTO fra le mazze. Coloro , che portano la brace a vende
ze , la mettono ne i facchi ; ¢ per ammagiiarii, ¢ legargli (opra tie,
tatamente gli rinuojtano in alcunc imazze; ed il Poeta {chereando dice
gante ¢ fimile a uno di quefti facchi picni di brace , perche egli € rinu
mazze ,¢ intende di quelle mazze , con le quali i ciechi Jo baftonano .
BATTVTO. Chiamiamo Barrasi coloro delice Contraternite fecolari
proceflionalmente vanuo con velti line in dowlo , le quali chiamiamo faccht
figueino vefti di penitenza ) cappe, 0 velti da bactu,, cioe , che f bane,
fi difciplina , ed il capo , ¢ faccia coperta con ua cappuccio appiccato a dettas
vefle . Ed i] Poeta fcherzando con I’ adicttivo barrute , cio baftonate , ¢ col {a
flantivo barruto , cioé humo di Confraternita , dice, che ai Biancone flaya
il Cappuccio , perche cra barrato ; ¢ per cappuccio piglia quel ferraiuolo , che
lino Cieco havea meflo in capo al Gigante . '
INC ANNVCCIAT A co’ randelli. A coloro , che G sompono braccia,gambe,
© cofce , ec, Nel raflettare tal rottura , «fhuche ’ off Rando fermo al luogo,ac-

  

  
 

 
 
 

comodato fi rappicchi , fanno una fa(ciatura con pezzi d’ afficeile, o ttecche, la is
gual fa(ciatura chiamano / incannucerata , ¢ pero dice , che , hayendo rouse | offa Us

al Gigante , gli fanno hora I’ incannucciata co’ randelli, cioe con quei afloat, Dk
0’ quali lo perquotono . 37h Une)

$1 tamburano come vitelli . Si baftonano ben bene. Quando i Macellari hanne Ni
ammazzato un Vitello , o Bue , ec. lo gontiano , ed acciocché il vento pall Da
da per tutto faccia fpiccare la pelle daija carne , baftonauo la beftia con alcune>

f
mazze , ¢ quefto fi dice tamburare » 0 tambu/sare , che vedemmo fopra C. 2. AL34. ie
ed a quefto ramburare aflomiiglia le baftonace , che fi danno fra loro i Ciechi 5 whe

wuol dire molte , fode , e {pete « Sidice samburare , perche date in quelle pelli di fi
Bue , ec. gonfie , fanno i] fuono fimile a quello del tamburo ftrumento 3 i

E per altro ramburare uno vuol dire quereiario ; ¢ quefto perché anti

Firenze 4 tenevano in alcuni Juoghi pubbiici de’ Magiltrati certe ;

hi da chiunque fi voleva , crano meile le denunzie legrete , € guefte calle C
vano tambari , ¢ da effi tamburare , era il medefimo , che acculare, 0 quetelare .
Vedi gli Staunti di Firenze al libro intitolato. Ordinamenta snfpicia contra Magnares
(citau aicune -voite da Gio, Villani ) al capitolo, ove Gi tratta del mettere nel tam~
buro. ieee
ALLA larga /gabelli'. Allontaaiamoci . Quando dopo Ja cena fi fa balla, 0al-
tro paflatempo fimile nella medefima stanza , nelia quale s’ ¢ cenato sche 1 com
menfali fi rizzano , ¢ per dar Juogo fi fanno levar via le tavole , Je feggiole, ¢ Blt
sgabelli ; ed ogn’ altro , che potetic dare impedimento, fi {uol dire: alla

belli , ¢ s' intende ; fi levi di mezzo ogui impedimento ; il che ¢ in ¢
che fignifica ; facciafi ala , o fi taccia largo 5 ma per lo pil s’ 1

 
 
 
 

HOEZBE SE 2E x PRZ

  
 
 

VNDECIMO CANTARE. 505

er ae: ' [
troppo buon aria, Li gon’ y'é buono tare ; Intendi : v'é pericolo di
- MAZZOCC H! , Cosi chiamiamo i Talli del radicchio, ne i quali nafce il fe-
de iquali fi fanno infalate , che fono rinfrefcative , ed il Poeta , (cherzan-

‘con I equivoco dimazzocchio., che vuol dire: baftone , dice che con quelti

1 i taano al Gigante I’ infalata per rinftefcarlo , ed intende ; /e ba/tonare,
SURAT. 1 baftoni de'Ciechi per to pid fono di forbo,o a’ altro legnaine fimile
chiuto , fodo.,.¢ grave ,¢ dicendo 1) Pocta ; Si ripoff alPombra di guei sorbi , che
i grartan ia rogna co’ lor nocehi , inteade + fi ripofi forto quelle baftonate de i

 

      

Ai “@ t
© PER far difperco a i corbi tiem coperti gli ccchi , Per fare Mizza a i corui per la.
» che hanno di non poter beccare , ¢ cavare glivocchi al Gigante, poiché gli
, © difefi col mantello di Paolino cicco ,
PANZAXXVIIL 5 STANZA XXIX,
Che la-nel mezzo a's fuoi nimics comba Su via figtixoli ; orto buon piceini ,

     
  
  

“Di moda, ch? effi foeman per bollire , Faccian di quepti furbiyun tracto,ciccioli,
Che dove i colpi ella indirizza,e pidha, Nim remete di quefti spadaccins ,
Te sli manda in un {ubito a dormire, C' alcimento non vaglion poi tre pictiali ;

    
 

“Che ne meno col fhan della (ua tromba

¢ Es in vista.vi paton Paiadini
NCamprian eli fardbbe rifentire

Han facce di Lionije cuor di fericciell ;

 
   
  
 
 
   
   
  

|B quamo brava, fimilmente accorta’, Efel-gridare,e ilbravar lor v' afforda ,
ait Acombatrere i {uci cosi conforta, Ut can chabbaiayraro avyien che morda,
ya ov Deferive laibravaca , ¢ prudenza di Celidora, ¢ riferifce #* orazidne da effa fat-

z Pe inanimire i foidati , la quale € veramente appropriata al perfonaggio, che
ae : :

~ ZOMBA , Perquote ». Vedi fopra C, 6. ft. 104.
| SCEALAN per bollire, Vuol dire {minuifcono , e quell’ aggiunta per bollire, fi
‘poneiper un coftume introdotto da un quoco goffo , ¢ ghiotto, il quale havendo
mMeflo'a quocere Ieile alcune merle , fe ne mangid pil della meta , ¢ portate il re-
A ‘in’ #gli domando i! padrone , che cofa havea fatto dell’ altre merle? ed
“il'quoco gli rifpofe ; Sig, fono /cemate per bollive, E da quefta goffa aftuzia quando
diciamhoe Laral cofaé (cemara per bodire , intendiamo , che una tal cofa ¢ (cemata
aflai , fenza poterfene ritrovare i] conto , o faperfi la caufa del mancamento .
~~ PIOMBA .. Precipita’; lafcia calare , 0 calcare il colpo .
"LA tromba di Campriano , Quefto Campriano fu ua contadino aftuto , come sé
‘accennatosfopra C. 4. ft. 47. ,¢ Come fi vede dalla {ua fayolofa foria ttampata,
€Ol titolo Storia di C ampriano , 11 quale per‘far denati trovd diverfe inucnatoni di
gabbare le perfone femplici ; ¢ fra I’ altre quelia d* una pentola, che bolliva fen-
(2a fuloco yperché da efio levata , mentre gagliardamente bolliva , € portat®in,
~mezzo a) una ftanza , la fece vedere al corrivo , a cui voleva venderla ; coflui ve-
dmala veramente bollire , fenz’ haver fuoco avanti , fubito fe ne inuaghi , ed ac.
» Sordofi di compraria per il prezzo , che conyewncro . Giunto poi guetlo tale a,
cafa con la pentola , e volendo fenza fuoco farla bollire, ¢ non gli riuicendo , fi
~ quereld con Campriano , dicendogli,che I ne ingannato; Campriano chiamé
ss la

   
   
   
  

SU OS eae st ey

 
 
   
   

 

 

    
   
  
    

 
 

    

506 MALMANTILE

la moglie , ¢ la {gridd , dicendo , che non potev efler 5

cambtata, La donna fingendo un gran timore , con gran la;

per haverla inavvertentemente rotta , glien’ havyeva data un’ a!

paura , che havea del marito. Di che Campriano moftrand

to , cavo fuori un colrello , ¢ con effo feri la moglie nel petto

afcofa fotto i panni una gran vefcica piena di fangue , il quale fg

che ufcitie dalla terita factale da Campriano ; per la quale fingendo |

fer morta , calcd in terva. LU gonzo fi doleva , che Campriano per ¢ C

gicra havetle commedo un delitto cost grave ;_ Ma Campriano con facia.

£'i die + Sc ben 1a donna € morta , 10 1apro rifulcitarla , quando vorro

bafta , ch’ io fuoni quetta trombetta ; ¢ ftimolato dal fempiice a far

piacque » ¢ fonata ja tromba , ja donna ff rizzo , moftrando di rifalc

il (emplice con grand’ inftanza chiefe la tromba a Campriano, il quale d

te preghicre a gran prezzo gliela vendé: Coftui andato a cala prefe o

gridar con la moglic , ed in fine le diede una pugnalata, con la quale

€ poi fi mefle a (onar la tromba , ava quella infelice elendo veramente morta,n0a

rifalcitd altrimenti, B per quefla caula , ¢ per altre fue fei. aggini fu Cam:
riano condaanato alla morte , che dicemmo fopra C, 4. ft. 27, E di quefta trom-

Es parla i) Poeta nelprefente luogo . sot then is ERE
SOTTO buon piccini. Efortazione , che fi fa a’ cani , quando s’ incitano,o am- —

mettono contro qualche fiera , come vedemmo fopra C. 2. £.87, ; ed il

fi {oftiene fempre in fu le burle, fa che quetta Capitanefla eforti , edit

fuoi foldati con quefti termini da cani . : - 1

cicciold . Frammenti di graffo di porco , che avanzano nel tegame,o altro

va(o , quando i fa lo ftrutto , 0 lardo , da alcuni detti ancora dardings, ficche> — jy

  
   
      
   
       
  
 

     
   
  
  
   

  

 
    
  
  
   

 
 

vuol dire facciamo di coftoro minutiffimi pezzi . Ciccio/o diminutivo , che vieoe> (
da Ciccia ; la quale nel linguaggio delle Balie 5 ¢ de” fancimili vale'apprefodinot
Carne ; ficcome appreffo i fanciuili Greci Tria . ei Whi
SPADACCINI, Cosi fi dicono per derilione coloro , che portano laspadas =p,
folo per pompa . juin seemnigadgtit Une
PALADINI, Ciok Conti Palatini, Quegli huomini bravi, evalorolidifran- —j,
cia cantati dal Boiardo , dail’ Ariofto , ¢ da altri ; ¢ da quefti dicen ty

Mena (e mani come un Paladino , intendiamo buome valorofo; poiche t O tay,

   

do. Cosisapprefio gli Antichi,Ercole , e Achille fi veniva a chia a
rofo ,¢ dicevano : editer Hercules ,¢ di Lucio Sicinio Denotato agg
mano braviflimo , riferifce Gellio lib. 2. cap. 11.5 che per la gt: eras Oy
appellato Achilles Romanus, Di guefti Conti Paladini , ‘0 del Palagzo intele il ei

 
 
 

Petrarca nel Trionfo della Fama Cap. 2, ro) Ne
Cingean coftus + uci dodici robufti, + ’ d } yy

FACCE di Lioni , ecuor di /criccioli , Moftrano d’ efler bravi', ed animofit te

codardi . Lo fcricciolo eflendo il pit piccolo uccello , che ti trovi , ha per conie- » .

guenza il cuore piccoliffimo , ed huomo di piccol cuore s'i huomo timido,

e codardo. Vedi fopra C. 10. ft. 30, Latino pari , © angu(ti anim Mi-

eropsychvs«

4

 
 
 
  
   
  

 
 
 
 

 

I ee

 

 

VNDECIMO CANTARE.

597

   

eet. di rado morde:,. Chi fa molte parole, fuol far pochi faci. FE

ape

Suol far poche parole .
STANZA XXX.

bb wel ch? Ella da ritto,e da rove/cio,
— Condicende , va fenando a doppio ,

Da ful vifaal Cornacchiann marove/cio

Cun mighio si fenti lonan sale 3
et =
anc’ egli eA cantocneall pio,
| Mail fapor non gufto gia de’ buon oo
Come chi prefe il {uo de’ cartoccini ,
» STANZA XXXI,
Sperance per di id gran colpi tira
Con quell’ “infornapan della fia pala ,

We barte in terra,fempre ch’ ¢ la gira,

5 shafiti per la fala,
Tal che ciafcuno indietro fi ritira ,
»O per franco schifandolo fa aia ,

Bhi l' afperta , come bavete intefo,

‘ ek elon Ai) ie i pf

   

Perch Alsicardo, c' al pafsol? attende y
4 goxz0 gli trafora col pugnale
Ete lo manda a far le fue faccende;

ANZA
r ome il fuggir quefta volta non gli vale ,

proverbio con dire, Caneche murde y non abbaia s' e{primera la

Curzio:, Aleffima queque siumina minimo labuntur fono; ed. anches
Polidoro. Vergilio : Cave ribs
lontano il detto di Catone

fe flefic fentenze,habbiamo in ufo. anche\nel pariar noftro dicendofi:

a @ acque chete, Guardati dail? acque chere; Chi far di farsi vuole ;

acane mnto., & ab aqua filenics A
1 Demiffos.animos , tacitos vitare me-

    
 

STANZA XXXIL

eAmoftante , che vede tal fiagello

D? se! arme non ufata pit in battaglia,
erica la spadaye quando vede il bello ,
Tira unfenditeein mezoglicla taglia;
Riman brusto Sperante , e per rowcllo
Li refto,che gli auanza all aria fcaglia;
Vola il trovone, eil Dianol fack' eicafchi
Sula bottiglierra tra vetri, ¢ fiafchi,
STANZA X&XLIL

Dalle diacciate bombole ,e guaktade

i vino sprigionato bianco ye rofso
Fugge per b afse, ¢ dann felso cade
Git dow’ é Piaccianteo,e dagli addofso:
Ei che nel capo ba fempre ftoccht,e spade,
04 quel fre{co di (ubita rifcofso ,
Pen{ando fia qualche [pada , ocoltello ,
Si lancia fuora,evia farpa fraselle.,
XXXIV.

Cosi dal gozz0 venne ogni (uo male, .
Per tui fadi , per Ini la vita {pende;

E vanne al Diavolche di nuouo piccalo,
A ustolare a menfa appie di Tantaio ,

-Celidora efortando i fuoi a combattere non lafcia di menare le mani; Si nac-

tano diverfi avvenimenti , ¢ la morte del Cornacchia , e di Piaccianteo .
_SVONA a doppio , Intendi perquote inceflantemente. Suonare a doppio inten-
do tucte le campane , o la maggior parte dicfle , che fono in un,

fampaaile » fyonano infieme .

Vedi fopra C, 6, ft. 107. Sonare per percuotere,,

il Boccaccio Novella 67. E alzato il baflone i comincié a fonare. Latino

are.

_ MANROVESCIO , E} quel colpo,che fi da col braccio all’ indietro,cio¢ con la
|p conuefia della mang , ¢ da quella parte con baflone , 0 altro , che s’ habbia

in mano ,
ako feepi fe h seid Meneoee un miglio, Il romore fi fenti molto da lontano , Ioke:

ropofito .

een rere fopra C, 3. ft. 21.
SICLIANDO un fempiterno aloppio « qu 20 alloppiarfi , 0 pigliar L oppia ;

© cor-

 
 

  
 
   

. | ee
so8 MALMANTILE 10%
© corrottamente'? alloppie vuol dire addormentarfi da Opis
Sicch€ qui intende , che prefe un fonnoeterno , cioe mori.)
Oui dura quies oculos , & ferrens unger Somnus; in arernam cli
Dice ; che per —— ¥ oppi rein perché It haveva dato.
tempo , per moftrare , che quis peccat,per hac torquetar,,
di Pincha y che per caula dab guacnisoeetipibans F c
zo muore, " sees

INFORNAP ANE , Ciot Ja pala da infornare il pane,
per arme, oh obi
SBASIT!, Morti, VedifopraC.2. f. 79.0 a a
FA ala, Fa largo; fa piazza’. ‘Latino Viam prebere 3 win decedere y fun
HA finito il pefo. $a fivito di fare quel , che gli era flare ordinaro; ha
compito ; € s' intende ha fino ta vita: Metaforico di qaetta porgione di
che fi da alli bactilani dali loro Capodieci di tance libbre@i lana, che
vorare , la qual porzione chiamano wn pefo , ¢ dicono bauer finito il pefo
peafum , quando hanno finito di lavorar quel tanto’, che era ftato loro daro .
QUANDO vedde il bello, Quando vedde il deftro ; il tempo a propofito.. >
REST A brute. Kiman bettato , eflendogli avvenuto quello y.che egh non s'al-
pettava 5 nel qual cafo il vifo refta macchiato di triftezza'y € i.
confufione . . We
SOMBOLA. Vedi fopra C. 8. ft. 44. , .
FESSO . Fetura apertura di legname , 0 d’ alera’materia, ©
vafi di terra cotta, Latino Rima, ' ‘ 3
WEL capo frucchi, e fpade . Dubita , che tutto quello, che egli fente, fieno ar-
mi per I’ immaginazione depravata della paura ; per la quale #%¢ rifeofo
tremore , che viene per qualche accidente inafpettato ; ‘che. ci cugioni

   

  
 
 
  
 
  
   
  
  
 
    
   
 
 
 
  
  
 
 
  
    
  
 
 
     
   
 
   
   

per lo spavento , ches’ abbia di qualche cofa improwvifa . Vedi fo he
C. ft.2. se RitEe ‘
SARPA. Se neva. E verbo marinarclco . Latino foluir , anchoram vellit. Bog.
l' aggiuata della voce fratello é pofta per emfafi , e quali per un giuro "g hl
LO manda a far le fne faccende . Lo {pediice . Quis’ intende ’ammazzay — te
PIANT ALO a uftolare . Latino ardere , inbiare. Lo mettevaliato a Tantalo 2 i
defiderar ancor’ egli il cibo. Ed ufuiare é latino ; ‘quafi dica + re dal ri
defiderio d*haver quella tal cofa , che egli vede. ‘Ovidio negli Ai ¢
indomitis ignem exercentibus curis Fertilis , accenfis menfibus arder %
propofito ci feraiamo anche del verbo spirare . Vedi fopra C. 1, A. 31, diciamo ch
anche Vrolare ; particolarmente de'cani, che fanno col mofo atte vie
vande , € per cost dire le mangiano coal occhi , € col defiderio . ee, &
TANT ALO. E’ nota la favola di Tantaio hglivolo di Gioves-e di Plotenin'2, 7
il quale per far prova del valore degli Dei ‘gli convitd, € diede loroim tavola cot. i
to, ¢ fpezzaco un {uo figlinolo detto Pelope ; Ma gli Deis’ aftenaero op
cibo , eccecto Cerere , che mangié le (chiene , le quali le furono'poi Fit 1
Dei , che lo fecero rifalcitare , ¢ confinarono all’ Inferna T: r be
cendolo patire di concinova fame , ¢ fete, per thaggior fuo te
‘metcere fopra il flame Ereditaao, che moltra acque doiciifims,a! ; .
 
 

VNDECIMO CANTARE:; 509
1 felabbra , ma non tanto , che ne pofla bere , ¢ fopra alla tefla ha un’
albero-carico di frutte belliffime le quali s’ allontanano quand’ egli s* allunga per
‘pigiiarle’ 41 noftro Poeta ,che-ha de(critto Piaccianteo per un’ huomo golofo

'y che morendo,egli fara confinato all Inferno, € per quefto fuo peccato di
ola fara mefio allato a Tantalo a #/flare anch’ egli, come fa Tantalo,vedendo

ba da faziarfi , e che non pofla haverla. Bologninus . i
Tantalus bic etram fitiens potare vetatur ,
Ba ‘a quod Pelopis Dijs epulanda dedit ,
quali Omero nell’ 11. dell’ Viiflea defcrive la pena di Tantalo , tradot-
Latini fuonane cosi :
Stat mifer in medio; medijs exardet in undis

Tantalus ,& fruftra circumfert pallidus ora,
Proximus illudit mento circumfluus humor
Et prope Yorantes contingunt corpora grtra y
Et crines ,@ barba madent a/pergine crebra ;
Dumque undam captar fitienti Tantalus ore

  

 
 

  
 

   
    
   

 
     
 
 
        
   

STANZA XXXV.
Era ua camer ata un tal Guglieimo ,
Cha la labarda,e ifuci calzonia ftrifce
Virbigonicinolohaincapo in vece d’elmo,
E'tutto il reffo armaro a flocchefifce .
» “Alemnnno é coftui Perneiter [celmo,
» Econ quel dir che brava,ed atterrifce,
Sbruffi ferenti (earicando; e rutti
Ln un tempo spaventa,e ammorba tutti,

STANZA XXKVL

Humoremque cavis , fentat tomprendere palmis .
Hen /upito , ben longe fugitura recurfitat unda ,

STANZA XXXVIL
Perché voltando il ferro della cappa

Verfo Alticardo a vendicar [ amica,
Quei ghetafcafaye glittra forto,e'l chinppa
Con la spada meixo del bekico ,
Ond'sl vim pretto in maggior copiafcappa,
Che no mefce in tre dil Inferno,e tl Fico,
Ala non va mal, perch'e: caduto allotta,
Hentre boecheegia tutto lo rimborta,

STANZA XXXVIIL

 

     
    
 
 

 
   

 
       
 

 

 

1 Coftud a quel ghiortone a tutte Uhore Gira Sperante pegeio a’ un mulino

Fu buon compagno a ber la maluagia, Perch'arme alcuna in manpiiend.gl refta,
rer non cadere adeffo in qualch'errore, Par trova un tratte un pie d'un tavolino,
yi E far’ un torto alla cavaleria , £ Ciro incontra,e gls vuol far la fefta,
a] Pur'anco gli vuoi far ,mentre chrei muore Aa quei prefo di quivi un sharagline,
, Con farfi dar due crocchie, compagnia, Voa cafa con effo a ini fain refia,
E non duri molta farica in quefto , Perche paffando ? offo oltr* alta pelle,
: ~ Chet trove chi [pedilo'e bene,e prefto. Nel capo gli raddoppra le cirelle.,
. Seguitando il Poeta a narrare gli accident occorfi in quefta zaffa , dice , ches

 

   

 

Alticardo ammazz0 Guglielmo Lanzo , che volle feguicare in morte Piaccianteo,
come !’ haveva feguitato fempre all oiterie ; B Ciro Serbatondi ammazza Spe-
rante , con battergii un tavoliere da giocare a-sbaraglino in fu la tefta,
GVGLIELMO Tedefco. Fu quetto Ledelco Soldato della Guardia pedeftre del
Serenitfimo Gran Duca , Ja quale ¢ compofta d' Alabardicri veltiti a livrea con,
brache larghe fatte a ftrifce paonazze , ¢ role ,¢ fi chiamano Linzi. Vedi fo-
pra C. 4. flan. 4. E perché quefti non portano ferraiuolo , o cappa , diciamo per
afcherzo ferraiuolo , o cappa quella labarda , che portano in fpalla., come vedre-
me

 
      
 
  

 

  
 

 
 

 

  
   
 

gre MALMA NETLLTW

mo appreffo flan. 27. ¢ s' ¢ accennato fopra €. 9. fan. 48..€ _ r
date , 0 percofie colla jabarda. Coftui era molto amico di-

aiuto a mandar male la roba , ¢ perd il Poeta dice , ch’ ei lo vuol

in morte 2) La OORT

BIGONCIVOLO.. Diminutivo di bigoncia , detto fopra C, 10, flan. 7
coftui con un bigonciuolo , arnefe,che per lo pi s' adopra al vino,
che in tutte le (ue operaziont egli haveva I’ animo al viao, ¢ con
( che vuol dir pefce baftone , vivanda aflai ufata dai Tede(cht ) per m
alla voglia del vino haveva unita ancora quelia del mangiare. Si. ars
ancora,che il Poeta voglia moftrare,che coftui era fudicio ,.¢ c
in effetto egli era , ¢ come per lo pili (ono quefti Lanzis a caula forle di
pelce , che veramente ha fempre malo odore.

BEKNEIDEK Scelm. Voci Todefche le quali in noftra ipa fuonane
cone , feellerato,

ATT ERRISCE , Spaventa . La pronunzia Todefca ha un certo accento,
fa credere , che colui , che parla bravi fempre , € per quelta rozzezza di ‘al
gua dicono che ella fia propria , ed il cafo a comandare eferciti , come la Fran-
ccle a aoe con dame , la Spagnuola al comando politico , ie cuaanaraerey
guefte cofe Pr

SBRVEFL, BE? quel mandar fuori per bocca il vento jonato in carded
prabbondanza di ae E ratti fi ie dire lo ftefio , aso che per rasse inten
diamo il puro vento ,¢ sbruffo fi dice quando il vento vicn fuor del corpo «
no firepito , che non viene il rutto , ma accompagnato con un poco sae;
¢fiendo lo cheafare un mandar fuori di bocca con violenza vino, 0 altro i
AMMORBA, Fa putire. Vedi fopra in quefto Cant. ftan, 23. quie pe

fignificato attivo , cio¢ appefta ; mette Ja pefte in tutti . ‘
GHIOTTONE,, Gran go.olo ; Gran ghiorto . lntende di Pisesbame a

MALV AGIA , Specie di vino afiai noto ; ed a noi viene di Vi qui ie
pigliando Ja fpecie per il genere, intende che gli fu fempre compo be a tain
forta di vino.

  

 
       
          
 

 

  
 
 
  

CROCCHIE ; Percofle , Da ereechiare che in fignificato attiyo vuol dire Ps
motere

2 SPEDILLO bene ye prefo. In poco tempo gli diede buona {p t 7"
ammazzo prefto , ed affatto . Quefto detto bene , ¢ preffo era il mol 3 7
cademia Fiorentina detta de’ Rifritti , ed il Pocta fe ne (erue, p pil
fu gia di detta Accademia , ed immita un’ altro Poeta , che nelj' umprovvila , © \s
byona morte d’ uno pure di detta Accademia difie ; ‘aban bar
E per moftrar , come Rifritto ville y pide eh e to

Mor: , come Kifriteo E PRESTO, E BENE, Ca

EEE ¢ il Fico, Sono due Ofterie di ‘Eirenze cosi nomina’ die oo i;
Infega

Bosc HEGGIARE . Quel moto , che fanno con aprire , ¢ (errage la bosaia
mandar fuora gli ultimi {piriti coloro , che muoiono 4

LO rimborra, Rimette nella bors 9 lO¢ in corpo 5 ‘ribeve -
che gli cra ulcito di corpo .

 

   
 

 

 

 

 

.

VNDECIMO CANTARE) Sat

CLI vel far la feta, Cide lo vuole finire , lo vuole ammazzare ‘
GLI fauna cafa in tefta. Nel giuoco di sareg eee una cafa,vaol dire rad-
iar le girelle , 0 tavole fopr’ a uno de’ 24. fegni , che fono nel tavoliere , cd
i. {cherza con quefto addoppiar Ie gireile con dire che batrendogti il ta-
_ yoliere in cefta gli raddoppia le girelie , che quiui haveva , ¢ cost gli fa una cafe,
- intefta, che haver girelle in tefta s' intende tuomo col cerucllo che gira. Vedi

C. 9. flan: 10.
STANZA XXXX.

-) STANZA XXXIX.
Ritvaffe gid Perlone un certo Marte, Tofelloych in fere.ra ad bxom non cede
Riefce adefio qus tutto garbato ,

© Cthaucua il nafo da fiurar poponi ,

| E perch ei nol pago mai del ritratto y Perch'ei rifana un zoppo da un piede,
Pere fa feco adeffo agli fgrugnoni ; Cregnor fu quella parte andd {eiancato,

| Ediegtien' un si forte . ch’ in quell’ atto Mentre di taglio un fopramanglidiede

Gli fi fhianto la firinga de’ calzoni , dn quel,che fano havea dali’ alrra late ,
| Che qual tenda calando alle calcagna Che pareggiolio, ond’ ei fu poi di quci
_ Scopri {cena di bofco , ¢ di campagna, Che dicon : qui¢ mioye qua vorres ,

% STANZA XXXXIL
Grazian di fangue in terra ha fatt'un bagno Che vie da un trcbettier di Carla Atagne
Onde glié ya 4 chi va gin che nnoti; Quando le molfe dar fece ai tremors ;
_ Afetta un Salta,e xn Birrocolcopagns Toglie ad unl'afta,tl qual fail Paladine
1 E frroppia uneal,che fale erucce aiboti, Se ben con efsa fu [parzacammino ,

“Seguita a narrare varj accidenti occorfi in quella zutla , ¢ le racconca le bravu-
re di Tofello Gianni , ¢ di Grazian Molletto .
SU ffianto la firinga de’ caizoni , Si roppe la ftringa , cioé quel legame,che ferra
calzoni in fulia pancia .
TENDLe4 . Incende nel prefente luogo quella tela, che fi mette d’ avanti a i
chi »fopra i quali fi rapprefentano Commedie, affinché cuopra le fcene per
Doprine nel dar principio alia Commedia ; Lat. /iparinm , e perd dice,che i {noi
calzoni. eflendogli cafcati,, {coperfono {cena di bofco , ec, cioé quel,che da loro
‘eraycoperto . Cafo veramente feguito a Perione , che,per voler ae pagato d'ua
Fitratto., che egli havea’fano a uno, gli conuenne fare alle pugna , ed ia quel
re gli cafcarono i calzoni.
SCIANC ATO. Vno,che va zoppo per haver difecto nell’ anche, offo princi-
pale delle cofce . Vedi fupra C. 6, ftan. 82.
. CHE dicon ; quie mio, ¢ qua worrei , Cosi diciamo di quelli zoppi , che vanno
a gambe larghe per difecco,che habbiano nell’ anche, o in ambedue le ginocchia,
€ non pofano i piedi in dritto , fecondo J’ ufo comune , ma pare , che vogliaao
can un piede andare in un iuogo ye con’ altro in un’ altro, ¢ che accennino qui
# mio, €qua vorrei , Di quelti tali diciamo ancora Andare a feiacquabarili, perch
fanno lo fieflo moto con ia perfona , che fa uno,che (ciacqui un barile «
APPETT-A, Taglia da una parte all’ altra , come fi fa al pane, del quale pro-
-Priamente fi dice affectare ,o far fette.
VN Sata, Si chiamano Salti quei famigli , ¢ donzelli dell’ Arte dell’ honefta
“(che in Firenze € i] Magiftrato , al quale fon fottopofte le Meretrici ) i quali fan-
‘NO ogni forta d’ cfecuzione tanto Civile , quanto Criminaie contro le Meretricé ,
t VN

me

i ee
 

 

le figure di carta petta:, le
di boto , ¢ d’ haver ricevuto ae
cono Bari. Vedi fopra C, 4. tts

 
 

 
 

fente ‘uogo il noftro Poeta ,

5 tz MALMANTILE | (¥
VN tal che fale erucce a boti « Intende' uno {eultore dappoed 5 che.
qnaii @ mettono alle Immagini facre.
razia ; ¢ quefte figure:co
c . £7. Gruccia é dal Lac, barb:
€ baflone fatto a croce ; onde in alcuni Juoghi della ’
Far le grucce a una figura, s intende fra.i pittori.
fan, 27, Intendi dunque , che coftui era (cultore ftroppiatore dit
fabbricava fe non faacecci di carta pella , formati confornie di gi
no di quella bellezza , che pud yedere?chi andra nelle! Chiele
miracolofi ; ¢ quefte figure faceva cosi male , che le ftrop
da fapere che /eultor da bori {uona fra gli fcultori lo ftefio , che fra i
Pittor da fgabelli , dewo fopra ©, 4. ftan, 10, Quefto tale ancorché fulie:
¢ nato d’ intima plebe ,  ttimava un Buonarruot ; efi piccavaidi nobile
dice,che yen da wn tromberta di Carlo Adagno , quandevle moffe dar fac
Cioe ha origine da un trombettiere,dei
re.i bandi , che dar de mofe a’ tremori , yuol dir comandarfo\

ticamente , fe bene in deco fcherzolo ,¢ per derifione, come fe ne ferue nel pre.
> aa

arias
java affarto ¢ Ino

quale:Carlo Magno i ferviva per manda

Ae

SPALZZACAMMENO.. Vanno per Firenze aleuni © Marchigiani o Lom-
bardi con una pertica in fpalia gridando : Spazcacammina y acciocthé pia
che efi ripulifcono Je cappe., 0 gole de i cammini-dalle filiggine » Vino:

tait era cului y il quale con queli’ alta, clod con ta pertica tr

ladino.
STANZA XXXXIL
Tutto tinte ne va Puccio Lamoni
Stoccheggianda nel merzo della Vuffa ,
£ in Pippa un tratte da del Caftitioni,
Che majcherato ancor tira di buffa ;
Ea ci che nel fentir quei farfallont ,
Venir pitt tofta fentefi la mufa ,
Paffandalo pel petto banda banda
Ai far rider le piattole lo manda,
STANZA XXAXKTL
Nanniruffa ba piu la pien di ferite ,
Pericolo , che fu [copa meftieri ,
Fu pailaio, Senfale , etitor di lite ;
Srette Bargelioy ed abbaco di xeri
Prefel appalte alfin dell! acquavire :
Ala pris fuaniro i fuvi peafieri ,
Lon pite il wana fhillando, ma il cernello
Per metterui poi il moffo,el'acquerelio,

Continoya a narrar quel , che fegue neheombattimento y

mazzamenti,

TVTTO tinte, Vuol dire adirato , ma il Poeta fi ferue'di
ché detco Puccio é di faccia bruna , come s'€ detto fopra C.

6 OP
STANZA XXXREV) ©
Con Duriano ii Purba eccoalle wank ©
Di ferro da fradceri i Safe,
Ev altro una paletta tq
E con eff a tui cerca,e sbracia
Ma percht quei le fqnete, tome canis
Gi fraricatt fuaf hs chibufo, (4

Chreghi ba a’ Monnini,evane

Fatto d ognun polpette
S’ 4 tanto mal non fe :
Col dar ful grifo-a tui Salue Rofata y
Chef. oui 2
Vuol ch’ e facia pere

Cb? effendo prefa
Lo {pinge fuor

   
  

  
      
  

     
     
 
 
  
    
 
  
    
 
      
     
   
  
  
 
     
 
 
 
 
 
 
 
   

«=. FREER

=a.

 
 

wii =

a a ae

Pre Sst = = -

Sate Sb tes eee

 

VNDECIMO CANTARE 513

_ TIRA4di bufa , Fa i buffone. Le buffe , come accennammo fopra C. 2. ftaa.
2. alla voce bu/chette , {ono pezzetti di mazza rifefla , e formano quaai un dado,
fe non che hanno te parti piane , ed una conuefia , ¢ fi tirano come idadi , fa-
eendo Con effe quei giuochi , che fi refta d' accordo con fei , orto , 0 pil di cali

__ buffe ; ¢ per me ftimo , che s’ ufino , come s’ ufavano dagli aatichi gli aliogi : ma

: é i ¢ giuoco da fanciulli,percio habbiamo il detto sirar ds buffayche yuol
ire Far cofe da fanciulli , ec. da perfone di poco giudizio , che poi da quefto in
una parola fi dice buffone , e far i! buffone ; che i Latini dicendolo /carra lo delcri-
vono per uno , che rifum ab audientibus caprar, non habita ratione verecundie , aut di-
gnitatis , © cosi per uno , che non habbia I’ intero giudizio da diftinguere i tempi,
ee wetl ne le perfone , come ¢ per lo pit il giudizio d’ un fanciullo . UI P.
’, Vincenzo Maria Carmelitano Scalzo nel {uo viaggio all’ [adic Oricntali lib. 4,
¢.26. defcrivendo un’ uccello detto Buffo [ che é forfe.quello che i Launi Bubo,
€ noi chiamiamo Gufo } dice cosi ,, I nottri antichi lo chiamaron Buffo , onde_:
y» forfe hebbe origine il nome di buffone , poiché é incredibile , quanto quetto
a uscello fia inclinato agli {cherzi , ed alle burle , con Ie quali bene (peffo atcer-
y rice di notte , ed inganna la gente .
|. BARFALLONI, Denti {propofitati , ¢ fciocchi .

SENT ES! venir la muffa . Si fente venir V’ ira; Entra in collera.

LO manda a far rider te piattole, Lo manda a far il buffone nell’ altro mondo,
dice /e piartole , perché quefti fon vermi , che ftanao negli aucili , ed hanno oc-
¢afione di rallegrarfi per 11 nuovo cibo che a lor viene dall’ andar egli nell’ avello,

PERICOLO , che fa Scopameftieri , Si dice Scopameftieri colui , il quale feguita
poco tempo a far un’ arte , ma Jafciandola ftare ne vada a fare un’ altra , perché
la prima non gli piaccia ,come appunto fece quefto Aleilandro Violant detto
Pericolo , nominato fopra C. 3. ftan. 58. il quale veramente fece tutti i mefticri
enunciati nella prefente Ottava 43. ed in ultimo fi diede.a trovare invenzioni di
Mettere appalti ; comincid dal Tabacco , e poi I’ Acquavite , i quali fenza fuo
utile , 0 pochiffisno conchiufe per altri. Dice,che abbaco di zeri ,perché veramen-
te ci fu un grandiffimo abbachifta ,e per quefto havendo faputo trovar degli erro-
ri. contro a’ miniftri grandi , fu da effi perfeguitato si , che fu mandato in gaiera;
Ma havendo le notizie date da lui fatto al fine (coprir la verita , furono i delin-
Foe gaftigati , ed egli cavato di galera. Dice abbaco ; ma percht quefto verbo

gnifica ancora flar dietro a fare una cofa ,¢ non trovare Ja via a terminarla,
per non haver tanto giudizio , o {cienza che a cid bafti , il Poeta piglia tal detto
in quefto luogo neil’ uno , ¢ nell’ altro fenfo , cioé,che egli fulle veramente gran-
de abbachifta , ¢ che egli abbacaffe , cioé armeggiafle col ceruello fenz’ utile es

conchinfione , ¢ perd v’ aggiunge di zeri, perché, fia pur grande un’ abba-
¢hifta quanto fi vuole , che mai non rilevera fomma alcuna , fe non fi feruira d’
altra hgura che del zero, Cos} in effetto fu coftus che con tutto il fuo grand’ ab-
baco non pes mai far conto , che gli tornafle bene , ¢ con tutte le fue arti, ed
invenzioni fi pud dire che abbacaffe , perché in ultimo fi mori quafi di fame ,

PIGLIAR ? appatto. Quand’ uno col pagare ai Principe una fomma convenuta
Piglia ’ afunto di provvedere uno Stato d’ una mercanzia , ¢ fa proibire che -al-
tri la pofla vendere , o fabbricare fenza {ua licenzia , diciamo pigiare appaito, che
Sil Las, Adonopolinm . Tre MET-
 

 
   
    

514 MALMANTILE fi

MET TERVI il mofto , eI acquerello « Confumarui tanto le bu
tive fuftanze.. Oleam , & operam perdere,

FVSO da StradieriChi fiend gli Stradieri dicemmo fopra C. 3.
fto lor fufo ¢ un ferro fortile lungo , ed acuta , col quale forano i
altro a fine di vedere , fe vi fia occulrata roba 5 che paghi gabella.

PALETT A da Caldani, B’ una meltoletta di ferro con manico: g0 » ches
{erue per iftuzzicare i fuoco ‘nel caldano,0 focone,il quale,che cola fia, Vi
C. 3. ftanza 3. ; ee

SBRACTARE.. Vuol dire iftuzzicar la brace, perché s' acceada , 0 P’accelas tie
{pandere alquanto, ¢ qui diceado: gf sbracta il mufo , intende, 10 perquote con 1a

   
  
    
       

paletta nel vifo , ¢ gli¢ lo {Cortica . 1. 20.) aga Deke
4 LE {quote come fanno icani, Non ttima, Non cura le buffe.. Vedi fopra C, 10, Suan
anza 36. 5 ‘sun Obey Mec
eARe HIBESO ch’ egli ba a’ Monnini , Doriano fa morire il Fucba con ‘uno: Sino
quei {uoi Monnini detti fopra C, 1. @. 44. i quali Monnini ij Poeta infieme cons Nan
ogai aitro flimava tanco {ciocchi,e odiofi, che credeva fuflono abil a far morire Celido
uno di naufea , ; al fen

SQV ARCINA , Spada corta ,e larga ,altrimenti detta colrellao mexza/pada. Te
POLPETT A... Vivanda nota fatta di carne beniffimo bactuta con coltello sed
impaftata con uova , cacio , pan grattaco , fale, {pezierie,ecs > an Oh Difer
CERVELLAT A, & {pecie di falficcia fatta di carne , © di ceruelli di poreo La
triturati , ed imbudellati come la falficcia. E dicendo far poiperte 5 ¢ ceruellaths Tere
4’ huomini intende far macello , ¢ ftrage d’ huomini. OLS AES
CONT ADINA., Specie di danza ufata nel Carnovale 5 la quale confifte tutta hag

 

   

in forze in quefta manicra,, Octo-, 0 dieci huomini fi fermano ritti col im Cel
fieme in giro con Je braccia alla coliottola I’ uno all’ altro; opr’ alle. di ha
quciti faigono quattro, o fei», fopra i fei altri tre ,¢ foprai tre wao ,¢ fatea que- atts
ita regolata mafia vanno girando a tempo di fuono,, ed in ultimo quello,che € to
cima (opra a tutti,fa ua capitombolo fopr’ alle {palle di quei tre alla volta delter= e
reno, dove ¢ ripigliato da due , che fono quivi a tale effetto ; :nello feflo modo ty
fanno poi i tre ,¢ poi i (ei, ¢ dopo quefti gli otto , 0 i dieci fanno iltcapitombolo ile
in terra; ¢ quefta dicon far /a tombeiata. EB percht Mato di Coccio.in: for- te
ta di bal'o era Maeftro, € perd dice,che Salvo Rofara fapendo, bea la re
Contadina, lo fa fare la tombolata gil perla fcala. aan Ui
STANZA XAXXXVL STANZA XX¥EKVIL- ti
Palamidone in tanto con la mano, Quafi di viver Bariftone ufo, > ’
In tafca a Belmaforto andana in volta, Egeno affronta con un prmerwoloy hes
Per tirarne la borfa in {uw pran piano , E perche quei |" uccedia come nn gifoy i
Per carita che non gli fuffe tla; Salea ch’ ei pare'un gailestovmapanele ip
Mail buon penfier ch’ egii bayrie/ce vano E raito fa cht iL manbeaeenpe y ta
Perch’ egli col pugnal fe gli rinolta's Manda My
E fa per carizade anch’ e che muoiar, E por to pi: the
‘extecio fa vita non gli tolga il boa, ‘Per dario per un 1

 
 

 

EB paffagli un veftir

ee

STANZA XXXXVIIL

  

Exquei gti duol che'l rinnono quell anno,
- Bfee’ fi muor vnol che gli paghi il danno,

ae VNDECIMOCANTARE, 515

STANZA XXXXIx,

Romolo infilza “to mezzo al bufto > L' armi Papirio ad un Prandron guadagna,
‘tes iyoenunise un canto erafugviafco, Che. fae apiacuhine lo Swillerra ;
Efe ne muor con molto fuo difeuito , Ma # a parole gli é Spaccomontagna,
«Perch? egli haveva a effer aun fiasco ; AUP ergo poi riefce Spada fanta ,
Tira inun tempo fifo aun bell imbufto, Perchheifactee it al Cel dar lecalcagna,
demmafeo , ‘Won una voir dice , ma cinguanta :

Sta[uch'in terra i pari miei non danno
Ed ei rifponde : S'io fto (uy mio danno,
L

 

STANZA
riga il Mula, ePOfte degli allori , E nelle parti git pofferiori
» Son mandati per fempre a far un fonno, Panfiloagginfta Meoyche vendeil tonno,
_ Miccioge’l Baggina da Strazildo Nori Tal che s* allor putina,bor chi accofta

Sono inuiati done ando il lor Nonno , Sente che raddoppiata egli ha la potta,

\ Narra‘ morte d’ alcuni difenfori di Mal mantile , ¢ le bravure de’ Soldati di

a, Se’brami tanto d’ intendere i nomi anagrammatici , quanto di fapere
chifieno gli altri. Vedi fopra al C. 1. ed ai C. 3,
STVEO . Sazio . Annoiato. 2

_ PENT ERVOLO.. Piccolo file di ferro aeuto , del quale infra gli altri fi feruo-
no i farti per far buchi agli abiti .

DB aecelta > Lo baria ; lo {chernifce . Dice come un gufo , cioe come fanno gli
ucceilétth al: gato , che é uno uccello notturno , ¢ fimile alla Civetta , ma aflai pid
grande } chey Latini dicono babenem , donde bubbofone fi dice a uno {propofitato
chiacehierone; ¢ bubbole i racconti {propofitati , ¢ non’ veri ( forfe da Bubbola uc-
cello , Lat. «pupa . ) In quefto uccello detto gufo , o barbagianni , favoleggiano
git atichy Poeti,che fufle mutaco da Proferpina quell’ Ascalafo , che fece la spia
a.Proferpina d’ haver ella mangiato ja melagrana , il che fu caufa , che ella non

¢ ulcir daii' Inferno. Ovid, 5. Met. Quefto uccello € forfe lo fteflo,che quel

Pgeedel quale habbiamo detto fopra in quefto C. ftan. 42.

~ GALLETTO marzxolo, | galli , che nafcono del mefe di Marzo, quando poi
fifega il grano fon pil grandi ,e fs gagliardi di quelli, che nafcona d’ Aprile ,
eper queitofaicano piii alto alle fpighe del grano , onde col dire: Salea come un
galletto marxvalo, s’ intende falta gagliardamente .

_ LL mal tarenfo , Vuol\ dire huomicciuolo di cattivo animo , che i Latini purer
dicono boma fungini generis .

4VEFETTO . lntendiamo una fpecie di tavolino ; ma quis’ intende un colpo,
che fi da‘col dito di mezzo accomodato a guila di molla a! dito pollice,o ( come
diciamo ) dito geoffo , ¢ poi lafciato (appar con violenza al Juogo , dove fi vuol
colpire «| Moiti pero per bufferto , o buffertune , intendono.colpo di tutta la mano ;
¢ appreffo gli muoli Boferada , 0 Boferon vuol dire moftaccione , guanciata. ,

Macon quetto huomicciuolo , che non era da pugna , 0 fimili, fi pud credere,
che intenda veramente pufferro dato con un fol dito.

BAR querciuole, Ciok con le gambe alzate all’ aria , € s’ intende ft ammazza,

-Lnoftri ragazzi dicono far querciuolo , quando no pola le mani , ea tefta in,

terra , ¢ manda le gambe all’ aria ; quaft moftrando qd effere una.pianta , la fc
od Tee "2 a

 
 

——

  
 
     
    
  
   
     
   
 

 
   

516 MALMANTILE | ©

 

 
   
 
 
 
  
 
 
 
      
 
   
 

 

  
  

ba,della quale fia il capo, il corpo fia il futto,e i rami le zampe . bo
feguente dice dar /e ca/cagna al Cielo sche vuol dir caduto in terra b Bul
cosi fi moftrano le calcagna al Cielo , ¢ fi'dice anche mandare a gambe | no
FVGG/ASCO . Riurato , fuggitivo. Vao,che per paura de’ birri sg
vedere , fe non ne i luoghi immuni. we ky
HAVEVA a offer a un fiafco, Croe 8 haveva a trovare a bere i 5
Quando alcuni voglion bere infieme un fiafco di vino , € pagarne i
ii valore per mettere infieme la cricca , dicono Chi vaol effere 4 un fiafco? Mi,
tende chi vuol accordarfi a bere , € pagar cia(cuno Ja {ua parte? BY termiae! Bad
fo , ed ufato fra I’ infima plebes ate a0
BELL imbuffe . Bella preteaza, Va di coloro , che Manno in fa la ky
quaii non hanno di buono che Ja prefenza , da 1 Launi foprannominati 4
per metatora , perche /folones fi dicono quci bet rami , che noa ab
donde noi diciamo folly a uno che non € buoao fe non a far comparla,o v
za,come fi dice qui #7 bell’ smbuffo , che diciamo ancora wa bel coram Vobis . A
Tulipano , diciamo a uno,che abbia buono afpetco ; ¢ poche altre quali Ti,
fimilitudine del fore cosi detto , venutoci di Turchia , che va imitando 1a! hare
¢ la vaghezza delia Tulipa , 0 del turbante Turche(co,ondehailnome, =u
DOMMASCO , Deito cosi dalla Citta di Damatco in Levante. Specie di v
drappo fottile di feta fatto a fior1 , 0 ( come diciamo ) a opera. os baa
RINNOVO! qued!’ anno , Se ' era fatto di nuovo quell’ aano, Pare che fia foli-
to quando altri fi fa un veltito nuovo per li primi giorai , che -adopra havers = nd
git qualche riguardo di pit , come faceva coftui , che per effer ii (uo vettito nuo- T
vo , !’ apprezzava pili della propria vita , poiché rinfaccia, e proreiladeldanno
del veftito , ¢ di quello della vita non ne dilcorre, oem oie ¢
FlaNDROWE , Huomo di Fianiira, Ma perché huomo di Fiandea diciamo j
Fiammingo , la voce Fiandrone ci fertic per efprimere Vino fpaccone, éhe fi vanti P
di bravo,raccoatando Je prodezzc tacte da im fuori di qua , ed uno di quelli,che b
i Latin dicono milires gloriofos , ed in quefto fenfo Jo piglia il Poeta nel prefente i
luogo , fe ben (cherza con I’ equivoco ; Ed egli fteflo lo dichiara dicendoy Che» I
fan Taghiacantons,e lo Smillanta ; all’ ergo poi riefce Spada fanta , ciok fa da bravo 5 ha
ma dovendo venire a i fatti, ¢ alia conclufione , riefce una (pada,che aon fa mal ¢
veruno , ¢ pero Santa ; ed in fultanza un poitrone. Dicefi nell’ ufo, "i
buona pada; cioe € huomo , che fa bene adoprare Ja {pada . Nel Pianto che't Pe
Carlo Magno neila morte di Rolando da’ noftri Poeti detto Orlando, appreflo vy
Tarpino Arcivefcovo di Rems , ¢ compagno in guerra del medefimo Carlo: 6 die fing
ce. O brachium dextrum corporis mei , barba optima, decus Gallarums, inf hg
Carlo chiama Oriando Spada deila giuftizia alludendo alla formidabile {pada da ie
Turpino detta durenda , da’ duri colpt ch’ egli dava con etia da’ poeti Darindana, Hs
oh wrath rf 5 © fmill. dich ua nottro pi bio in. di,
che dice La fradera del’ kiba , che vuol div vantatore di gran cole 50 ;
re; Equefto perché Ja ftadera dell’ Elba ; che ferue per pefare barche piene ey
ferro , acile fue tacche comincia a contar da/ mille , ¢ feguita { -a migiiaia
Tagliacantoni , cioe , che tira gill pezzi di muraglia corrifj | Pyrgo ii
wices di Riawto, Che vorrcbbe dire in noltra Lingua Atrerrasor ty

 
 

ee eS

4
Ai. si

a
VNDECIMO CANTARE. 537

 

Lo Smillanea , ciok Smillantatore fi efprime dal Greco Thrafon , ciot Audace.,

BHES Ske

ee

Sesh © £F

SPTVSILRS Pee Thr ees CUR ET

 

Baldanzofo ; ¢ dal Latino Adiles gloriofus. E la parolaé fatta da Adidanea , (cher~
‘zofamente ufato dal Boce. in vece di mille ; dandogli la definenza di quaranta ,
cinguanta, ¢ fimili; quafi uno non fia contento di dire Ja femplice parola di mil-
le , ma la voglia go > ¢ far parere 1a cofa pil di quel ch’ ell’ ¢ in efferto .
'S’ io Ho fu , mio danno, Non mi rizzo al certo . Quefto termine mio danno ula-
to in quefta forma , ¢ {pecie di giuramento , ed ha Ja forza del termine appon/o 4
noi , decto fopra C. 8. ftan, 72. € 3° io non’ ho,egli ¢ fallo,detto fopra C. 6. (tan, 86,
MiCC4O , Cosi era nominato un garzone della pallaa Corda, che ¢ uno di
coloro i quali {tanno nel mezzo della ftanza , mentre fi gioca, a raccorve la pals
Ja, ¢ rammentare il giuoco .
BAGGILANA , o Baggina , Eva un Battilano , che in occafione di felte feruiva
ai Bawtilant per tamburino .
DOVE anao il lor Monno , Ciok nell’ altro Mondo . Vedi fopra C. 4, tan. 2.
__ NELLE parti pofferiori . Cioé nel c ....0 come baflamente fi dice , nel prete-
rito , dove dice che ¢ prima putiva , hora pute il doppio , che quefto vuoi dire
ha raddoppiato la pofta .
e4GGIVST A . B’ prefo ne) fenso medefimo , che é prefo fopra C, 2. ftan. 41.
CHEO che vende il Tonno, Fu un venditore di peice falato ,¢ tali huomini
hanno (empre addofio cattivo odore .

STANZA LI. STANZA LIL
In abito Scarnecchia da Coviello, Guftavo Faibi con un foprammane ,

Tinta de brace l una,el altra guancia, Di nerty il capo fmoccola a Santella
EB per {ua [pada sfodera un fufcelio , Scaramuccia fi muar fotte Erauano ,
C" al pome a’ una bella melarancia , C' aimazza anche Gaba da Berzighella,
Rinolto con queft’ armi a Sardonello , E fuentra quel birbon dell Ortolano ,
Perma , gis dice , guardati la pancia , Che fa il minchion per non pagar gabella,
Ed enrifponde: ueftoé penfier mio, Ma colto poi vi reffa ad ogni modo ,
z rant un colpo, ete lo manda a Scio. _ Mentr' adeffo gli va la vita iv frodo ,
Defcrive } abito , ed armi di Scarnecchia , che refto morto da Sardonello ;

Eravano ammazza Scaramuccia , Gaban da Berzighella ,¢ P Ortolano.
COVIELLO . Ciot lacoviello mafchera , che finge un bravo fciocco Napole-
tano, ‘a quale s’ aggrotte(ca con fargli i bafi alla Spagauola col nero 41 b ace,es~
PerO dice Tinto di brace? una, el’ akira guancia ,¢ con armaria d’ una {pada faca
d’ una mazza, che ha in vece di pome una mela , o melarancia , o altra frutras
fimile per rendere il perfonaggio piu cidicolo,e cosi veftiva quelto Montambanco,
facendofi.chiamare Scarnecchia... Vedi fopra C, 3. tt. 62, Cosi Cola, ¢ Zanni,
perfonaggi ridicoli di Commedia fono nomi proprj de’ loro paefi , donde fi fingo~
.no»s accorciati dag!’ interi nomi Niccola , ¢ Giovanni ; onde va in terra lorigine
di Zanni , che alcuni ingegnofamente hanno tirato dal Latino Sannio , mis .
LO manda-a Scio , Lo manda all’ altra vita, ed ¢ lo ftelio, ¢ fi dice per ia me-
defima ragione , che mandar a Pasraffo ,0 a Buda , derto fopra C, 5. ft. 134
- SMOCCOLA il capo. Taglia il capo... Smoccolare fi dice tagliare i} Lucignolo
di una candela, o altro lume per levar quegli efcrementi , che fa la fiaccola, che
hiamali f i . » che queiti Spagi sear’

 
 

   
   
  
   
   
     
  
 

8 MALMANTILE

 

  
   
 
   
  
 

desfavilar quafi exfavillare ; il Vives difle exfungare formando 1a all
Virg. 1. Georg, Scintillare oleum , & putres concrefcere fungos , ol
SCARAMVCCIA, Vo’ aitea mafchera , come Scarnecehia - tit
Ourava 51. , ma quefto era Iftrione , e non Montambanco. i owe Roi
GABAN da Berrighella, Quetto pure era Iftrione , ce rapprefentava wo |
dt un Romagauolo ttoito . ' = Oe
LORTOLANO, Coftui fa un yeechio aftuto , che: per ein
dovutali per aicuni delitti commeili , s' era finto ae 1 Del
chion per non pagar gabella, Menandro , Rufticum effete fimulas , tam Par
vi refta colto , cioe viene (caperta quefta fua malizia da Bravano , che Per
vita in frodo, a colui,che non volea pagar la gabella,e¢ vuol wae Sin
in vece di frede folamente I’ ufiamo di dire dalla fraude , che fi comm el
pagare Ja gabella . Ta
STANZA LIL is
e4rmato a priuileo} omai Rofaccio Che piove al
Marte sguaina , e Venere influente , Ond’'ci in quel pumoandada Nan
Ma ae Sardonello ful moftaccio Vede le elle, e linac t altrasfera ua
Gli fece con la {pada un’ afcendente, Nel vifo ectifia,e dice: Ty
Rofaccio ricoperto di privilegj cava fuora Marte; ¢ Venere; che Pe
tivi influffi , ma Sardonello fece piombare fopr’ a di luiun pefimo % Tee
tagliandogli con un foprammano parte del vilo , ¢ del collo, ed un braccio Rani
il qual dolore egli vede le ftelle , ed eclifiando |’ una , € laltra sfera del Coats
ferrando gli occhi dice : Buona fera , cioe perme , fatto buio , «B Mi
fto Rolaccio fi piccava d' Aftrclogo , come s’ ¢ detto fopra C, 31M. 63.5 11 Pocta tg
con la prefente Otrava defcrive Ja di lui morte con equivoci di termini affrolo- pred
ici. f Lapa
: ARMATO 4 priniteg| . Quefto Rofaccio, come ancora gli altri Montamban- ro
chi per accreditare i rimedj , che da effi fon difpen{ati , moftrano una infiuna di iw)
privilegj concefli loro da diverfi Principi ; ¢ pero ii Poeta lo fa axmato di privi- the
legi . Uontanlald ken

SGVAINA , Virgilio vagina eripit , Sfodera Marte , © Venere ; che predicono
rovine; B dice /gnaina , che vuol dir cavar Ja spada dal fodero, o guaiaa, perche
s*intende , che aon haveva alcr’ armi offenfive, che Venere , ¢ Marte wnfluili
cattivi ‘a duaaiead

ASCENDENTE , Termine aftrologico , col quale qui intende colpo di taglio, Un

che viene da alto a baflo , piovendo , ciot calando in ful capo, ec,

OCCIDENTE , Intendiamo I’ occalo del Sole’, maqui intende ocealo y cio’ &
morte di Rofaccio, ily oni aalaan ey baatee tm
VEDE le free. Quand’ uno fence gran dolore ; fi dice + Eeli ha veduto le fielle y hi
perché le lagrime , che vengono in (ugli occhi per il dolore, G

Ja refcazione della luce , che yi batte , una cofa fimile a una quantita di mi -

nute ftelle in Ciclo , che pil volgarmente diciamo veder ‘nce i

mo fopra C. 9, ft. 60, 5 ma qui fi ferue di quefto , perché.gii

re di farlo morire aftrologicamente , i
ECLISS.A. Chiude , cuopre ; ficome alla Luna ‘reftano i

»hajean > x

#3

    
 

VNDECIMO CANTARE. sip
_ dail interpofizione della Terra 1 raggi del Sole, quando feguono I ecliffi .
DICE buona fera , Cioe fi fa buio per lui, ven donate 10. ft. 5. Qui intende
_ & finito il giorno del mio vivere . Virgilio in-aternum clauduntur lumina noflem , ©
i asadostndelcginnanalo » che, havendo:manco un’ occhio, ¢ Ji fa ca-
vato l'altro, difle: Buona worte per tutto lo tempo, ‘

i STANZA LIV. STANZA LV. i
Mein per fiancofentefi percoffo Gid per la franca il fangue era a tal fegue
Datlo ftidion del cuciniere Melicche, C” andar vi fi potea co’ mauicelli —

_ Parafiraccio porco grande , ¢ groffo Istrion Vefpi tutto furia ,e sdegne
Perch’ il ghiowso fi fa di buone micche; Rinualto ha quivi tl povero Adaffelli,
| Sirivolta eAeino ,¢ da al coleffo E col coltel da Pedrolin di legno
Nelda gola ch' egli ha pien di pafticche, Su pel capo eli squotola icapellig,
«Tal che morendo dolcemente il guitto: acleciopratcane poi la lifoa , el Lota...
» Addio cucina dice , ch'iobo frito, Pius bella faccian la conocehia a Cloto.,
ils STANZA LVL
NGatsi,, e Paol Corbi inueleniti A tal ch'i pacfani sbigottiti ,
| Quali villan ch’ i tronchsyed i rampolli E dal difagio feonquaffati, e froki
 Taglin di marzo ai fratti ed alle viti (Oktre che a’ pachi il numero é ridotto)
| Potanda i bafts braccia, gambe,e colli; Cominctaron le gambe a tremar forto,
. Termina con te prefenti Ortave i} racconto del combattimenco (egaito in Mal-
mantile,, ¢ dice Ja morte:di Melicché, ¢del.Mailelli., ¢ qui tinifce ’ Vadecimo

re

 MELICC HE, Vedi fopra C, 3. ft. 59, lo chiama Parafiraccio,perché era huo-
»¢del continuo havrebbe mangiato: EB quefla voce Parafito , che ap»
prefio'dinoi ha dell’ ingiuriofo,non era cosi apprefio gli antichi,come fi pud de-
durre da molti Autori tra’guali Luciano ; ma particolarmente da Piutarco , dove
fitrova’: Parafitos nontancumappellabant strici adalatores illos, qui apud Dinitums
tmenfas wutriuntnr , fedietianm tos,qus ob rem egrecit geftam,publico /umptu in Prytaneo
atebautur Oc, Ondedelie Stinche di Firenze, nel capitolo in lode del Debito, il

Bernt; é
Voi fore quel famofo Priranco y é
Ab bower yas Doe renews in grafsoin fisoi baront
I popaly che difcefe' due F efeo,

Exit Atheneo Parafiti olim appelabuntur foci , 7 fideles Pontificum, eAMagiftratiiz,
Ibmedefimo Plutarco.. ¥

_PASTICCHE., Specie di confezione fatta col zucchero\mufchiato y:ec; ¢ perd
dice more ‘doicemente , perché ha gii per Ja gola 11 zucchero, Pa/feca voce Spa-
boas » ficcome anche ‘Pa(figiia, che vale Jo {tedo ; ¢ {ono tutte due diminutivi di
pafta.

GVITTO , Huomo vile , abbietto , fudicio , sporco., e fciatto. Vedi fopras
C, 3.\f..9.é:voce Napoletana, ma ufata oggi anche da noi, ‘Nella raccolta de’
_ Poeti antichi-deil’ Aliacci , Pra Guittone-{crivendo un Sonetto , ficcome da effo fi
raccoglie.a Meflere Oneito da Bologna ‘Pocta , ¢ amico fuo ; {cherza {ul nome di
turer € die, *

— SAS QF Cisasn sees

=

Pita

_A.. SSeS

 

 

 
 

  
  
  
   
   
 
 
  

g20 MALMANTILE | © ¥

Voktre nome , Mefsere ,¢ caro ,e onratoj
Lo meo afsai ontofo ,e vil penfando, =
Ma al voftro non vorrei auercangiato, =
10 ho fritto, Scherza col verbo friggere, che vuol dir Quocere carne,0
padella con lardo , 0 olio ; ed il detto ho fritto , che fignifica il
in malora . Latino Attum eff de me ; perij . Vedi fopra C. 8. ft. $4, tor
nel prefente luogo , perché par che dica ; Addio cucina , ti lafio non
pili bilogno di te , perché io ho gia fritto , ed intende ho finito di vivere.
IST RION Vefpi . Pietro Sufini . Quefto fu cognato dell’ Autore , ¢ git
grandiffimo (pirito , copiofiffimo d’ inuenzioni , come fi vede in una
commedie da lui compofte, ¢ da altre fue Opere poetiche , B pecige p
fentava in commedia ottimamente tutte le parti , ma in fpecie quella del fe
zanni , ( cioe feruo {ciocco Lombardo ) che ula armare con un coltello di Tegao
fimile a quello,col quale fi batte , ¢ fi {cotola il lino per purgarlo dalla lifca ,
percid chiamafi Scotola ; perd il Poeta lo fa azzuttare col Mafielli, ¢ {c
con quel coltello la zazzera . Dice coltello da Pedrotino , perché con tal
ceva chiamare in commedia detto Sufini nella parte di feruo feiocco. Quefto mo-
ri giovane poco dopo I’ Autore ; ¢ con effo fi pud dire , che in Birenze morifles 4
la moderna arte comica , 0 almeno la franchezaa , ¢ leggiadria nel maneggiarlag =
SQVOTOLARE . Vuol dire battere il lino. Ma qui intende squotere i capelli
per facilitare a Cloto , una delle tre Parche , il farne 1a conocchia, aleism
INVELENIT/, Ancrudeliti , inviperiti , inafpriti , incancheriti , arrabbiati
fon finonimi per intender’ uno , che fopraffatto dalla collera operi of
te, ¢ con ira , in maniera , che non fappia quafi diftinguer ch’eififaccias, =
Similitudine prefa dal ferpente in collera ; di cui Virgilio lib, 2, En, tcolentems
tras ,& coerula colla tumentem . wm abean
POT-ANO . Latino amputant,demetunt , obtrancant , tutte fimilitudini trates
dal’ agricoltura . Potare fi dice de’ traici delle viti , € de’ rami degli alberi; ma il
Poeta fi ferne di quefto verbo per corri{ponder’ alia fimilicudine , haveado dewto
quafi viltan ch! e' tronchi , eds rampolli taglin ds Marzo , ec, . sd
SCONGVASSATI, Stanchi , € rovinati walla fatica del combattere.
FROLL], Qui vale per ftanchi , ed indebolits 5 t¢ ben per altro Frode vuol di-
re ftantio . Vedi fopra C. 3. ft. 55. alla voce Leazo, iahesh
TREMAR le gambe foto. Vuvi dir haver paura. Virg. Eo. ry.
ae folvuntur frigore membra , Se ben fi puo anche intendere, che le §
mente tremafiero per la debolezza , ¢ thancneaza .

FINE DELL’ VNDECIMO CANTARE. +

  
   
   
    
   
    
  

 

 
   

Ze

 
 
 
 

Berbnanhéansr ~ekk dé

 
  

BER TEER BEES

 

 

 

 

HifwMAMALAM

Fea dattacbute

A R GOMENTO,
e A, Montelupo. da Paride 1). nome,
Poi gapigar la Maga ,e Biancon vede,
Rimef[a sn. Trano é. Colidera 3 3 e1come.
~ Aarito , al general dd ln fuafede .
a Baldon , che la fortuna ha per.le chieme
Con Calagrille aVgnan rivolgeil picde y
E al {uo bel. Regno con Amor va, Psiche
A corre il frutto, delle fue fariche.

 

 

 

ae

speyrepeage

sae PP Regen

STANZA HL
Che fono fratt com! io diffi fopra ,

were STANZA I.
Swanco gid di vangar tutta mattina

   
    

“Abconcadino al fin a va-a rifelnere ,
- Te forniar Vopresed-in chiamar la T ina

* Cokmerize guarto,eil petal dell'afoioluerc ;

( Nella Maga affidatifi) afperranda
Da’ Diavoit im lor pro veder quale'eprs;

ea chi-vive a {peranta muor acids;

eee tn Caffelle ancor non firifina Perch in Dite fon tutti fottofopra ,
Phu quei-marei di squoterfi la poluere; ‘Per non faper dove, come, ne quando
Onde: Badldon quei popoli-di/per de Laftiaffe il Cornocapolfo,c! ale (chiere
Tal che a’ joldati Malmantileé al verde, Effer tromba dovea nelle carricre .
STANZA Il. TANZAILV

E vase Sta, perché porevan dianzi ,

vedean col peggivandar ficuro,

~\ Cederil campo, @ non tirare innanzi

ra Star avoler coxzar col mura:

< E cost va, che quefti fon gli avanei ,

Che fafempre colsie’ha is capo duro,

» Che dentro.a feifi reputa un’ Oracolo,

Ne crede al Santo,fe non fa miracolo,

Di modo , che Plurone omai fcornato ,

Poiche quel corno pitenon fi ricrova ,
Pel Proconfolo dice baver pefearo +
Pero connien penfare a inuenzion nuova;
Ha innanss ch’ ¢i-rifolua col Senato ,
Eche'l:foccor{o 4 Atalmantil fi muova,
Ch'egli babbia.aeffer proprio pot savvifa
Di Meffinail foccorfo , v quel di-Pifs ,

wey introduce i Poeta in quefto Duodecimo Cantare con 1a rifleilione , che i (ol-
7

vv dati

|

 
 

    
     

22 IRKTVUA MAL MAN TILE > 1009

dati-di Bertinella non, haurebbono ricevyto,cosi gran danno\ |
fono accordati,, ¢ non fufione faut in tanta ‘tingaiones Ja. ;
in loro per Ja {peranza , che havevano neg!’ incanti di,Mar
havevano havuto effetto alcuno , 1 Diavoli non feppe:
dove fufie ii Corno d’ Aftvlfo., aon fi ricordando , che. an
quando Affolfo ando per il fenno.d’ Orlando , comedice.|'/
| KANG AKE...Lavorar la. cerca conia vanga... Bipalio

FERALAR P opre ,.Cioe far defiltere dal lavorare eer an
raion Depera. fra. i\contadini.s’ intendedlJayoro;, che fa.un’
no , ¢ s’ intende.ancora lo Relio huomo.s.che ya.alavorare a
io.ho; chamato due, opere , per iacender_ due huomint; In quetto lavoro ci
dicci opere, per intender dicci giorai di iavoro , ec,

p44 Ting . La Caterina , intende ladonna del Contadino

MEZZO, quarto... Cosi chiamano i Contadini ua gran valo —
foggia. da boccale 5 .del.quale fi leswo, fespartag da bere ai Javor.
po, ¢ gli danno quefto cee perché’e forfe di ce een
ftaio . x ae
PER SL afeigluere 1 cont ania ebaainesior il definare afciolvere , Seren csnidal
foluere il digiung, dali. sdigunarfi , ¢d.il definare, lo chismano wien
terzo mangiare Aicono./a cen

eA non fi rifina., ‘Nanay refla,,
efprima una.op ¢ feng’
Ciog perquoterii,, baftonarfi.. Vedi rae C7. tt. 63. t by

ESSER? al verde ., Eji¢r’ ajla.fine . ‘Tratto dalle candele. di, fe ce seieprion dig
fon unte di verde nel piede.. Viano nel Magiftrato del Sale di.
Je tafe dell'Oftcric.,,¢ darle.al pili cfferente., ¢ agl.tempo , che aMtneenapi thd
coliffima candela di cera tinta da picde di color verde ognuno puo.otferine, es ida
conlumata quelia noo pud. pil veyung offerire fopr’ a quell’ offcria, ma shintende Ps
reflata.a, colui), che ha cfiertoyi maggior prezzo, ovver0 non arrivando.lotier. Cm
ta.aldovere, ’ Ofteria ai suoyo fi dubafta un’ altro giorno con nuova candelerta’, deat
EB digui habbiamo il dewato hs ha che dir,dicada candela ¢ al.verde, che fignifica ted

      
    
  

 

on, fifa fine. Ma — che p00 iar

   

BINED a ET SERRE SSESE

 

 

 
 
 

sbrighiamoci y che il cmpo fugge .. E quelto eficr’ al verde ¢ pafiato in thes
per tutte le cole , come.cficr’ al verde ot danari , vuol dire effer’ alia ka
pari,...Va mpderan Pocta teleth fcritto neil’ Ofteria di Radicofaa Pre
trata » qénm ° Re he age

| Cohanr, p Spebater ridotto alverde. : ee ua ie ‘ hy

 

. » Gineca, uper ricattarfi , ¢ Sempre perdes

COzzAR col muro», Tentac P impotlibile ,. Contrattar con chiha pb: forea di re
poi, Clavam, ¢. manu Herculis extorquere . Diceli anche: saree a co4hi co! mre othe
ciuoli . Nell’ Ecolcfiafico cap, 3. Ditiori re ne focins fuerss 3 Qu Ep
cabys ad ollam ? Quandoenim fe colliferint , confringerur. La Feces Fest
tole nel. fume galleggianti 2 una di rame , I’ altra dicerra fa a. ¢ oka
quale viene, Anaae ad. Efopo,¢ troyafi refa in verfi Latini gala ,
C_APL dur ag te oftinati. Dure pom ebb —o Ne
‘ST tacas lends » Amico della {ua opinionc,¢. che Li, Gh
uw

rs

 
 

~~ oN eeenrvc

S88

SSBB ER CEES £5:

 

DVODECIMDJSIVLTIMO'CANTARE, 8
reat fate’, ¢ dit meglio w ogni altro. Huomo di quefd naritat dice de’
= Setpe thea Nim di lapete;e d-etere wngtan” buotao - Baxi.
; edi fe'micltefimo ',¢ pereid ine diviene contumace 3! &

PUR Hess ONT I OM 9G; ty
VENOM orede al Sarito’;\fe*hon fa miracoli, Non crede'; che una cofa pli poma'ii-
teruenire’, fe'rion 1a vede fegitire .Generario prava quarit fignum videre. B per lo
pid s’ ufa in’ occafione' dammonire , 0 rinfacciare j'come ¢ nel/pretente hiozo} ll
tale é lato pir volte: avvertito dition contindvare @ fat’guella tale operazione-,
perche'gliene' potrebbe feguir male’; ma’ egli oftinato wor erede at Bantoy fe nor fe
miracoh , cioe non da retca agli aveertimenti ; ma! vudl-feguitare?,finvhe 1a die
ee fucceda » 4’ Proverbifti Greci mettond un Proverbio ,-che dice: Primes
a rem. PURI Toe h) LL Mes bas! Pe TNE Dey $99@ 1951b

CHI vive con (peranzia mior cacando . Detto. fporco » ¢d ufato per lo pil fa,

genterviles;\c vuol dire -'chiMfi palce di {peranza-,'muore di faine'y"ed-in fulliinza

a €*vanitaril'fondarfi nelle fperanze . “ai /pe'neratar 5 wi reer

mats ig 2. 0g

  

SON tutti fortofopra, Sono in grandiffima confufiune . ’
{I DOVEA fer tromba alle carriere .’ Dovea' fare feappat tut? peome facev't il
Corno 4’ Aftolfo : e'come fa’ {cappare dalle motfe i cavaili’barbati'y che edrreno
al palio quella tromba , che fuona il banditore , pet dare if feghd della [otpperied
SCORN ATO’ »» Vuol dif beffato; ma qui 410 (cherzu di /eorWard\, che Vadeiie
fenza corna , come era rimafo Piutone fenza'¢orno , cine fenza it Corti dA Nbk
fo. Var animale , che abbia perdute ; © tronche le cortia , vient ad avere per
del décoro; onde fcornato diciamo per beffato . Acheloo 'fiume ; ‘efleatlogli d2*Er-
colelévato un corno’, rimafe feornato ; ¢ fuergonato . Onde Ovidio 9; Met Muh
tas Achelons agrefies 5 Et! Laverne cornu; meuijs capur abdidlit sndis,  Hivtc tanith abla
ti dommie idibure decoris, Gc} 229 10109 Lb 2h 4 HW) 199 > piri
> SPBSC_AR per il Proconfalé. Bo Neff, che durar fatica per impoverite; sean,
CG operam perdere . \\'Proconfolo’é ia Firenze il Magiftrato', che foprdutenie 21
dottori ,¢ Notai , ed ha la‘{aa refideriza otto Ie logge,dove fono git altri Viizzi,
acl!’ ultima abitazione verforil fiume d’ Arno; il qual fiume pet quello fpazio’,
che ¢ fra I’ un ponte , e/’ altro} ', 6 almeho efa gia fortopotto alla 'giurifdizione
del medefimo Magiftrato del Proconfolo'; come pefca'ad elo rifetuata ne’ vr ti
poteva pefcare fenza licenza del detto Magiftracs 3! non vi era-gia ditra pena aIfi
contraffacienti, fe non la perdita delle reti , ¢ del pefce , che hanno prefo 5 feads
acchiappati in ful fatto; E Pett utes! aie s

STAN ZAV E>) > FTN ZA VEU.
ae Paride ritorno , OO Ada qnegti 5 ¢° obligate fi non Witende ,
onGhte nellvoffe alla quarta sboccatura ;\ Wor vnol phr quanto un capo di spillerto;
« Eperché dal pacfeegls ha in quelgiornd E {ubito ogni cofa indietro vende ,
0 Foleo ogni nota’, liberando ib Tura; Ringrariande cinfcsin det buon’ afetto ,
8| La\gente quini corre @ intornd E'dwe', che da lor nulla precende ;

ed rallegrarfidelia fuabravkrh >) 0! 2) EB Te aiteddisfarle bhnho concerto ,
Ne lo ringraxiasewrallegrarfi intenta, °°! Perital niémoria gh fara pitt griro
Chi gti da chin lt dona z'chi gli-avvina, ©: Che it tuogo'Aonteliipa fia chizmaro ;

ang, ~ :

Vvv 2 STAN-

 
 

 

 

  
  
 
 
  
   
  
 
 
  

524

Si si, ch’ eli é dover da tutte quanté
Gli fu rifpofte , ed in un tempo ftefo
Li editto pel Cafpello fu pe i canti
Per notizia de’ Popols fu meffory>
Che dinuleato pos di te avanti yo. «\
Fu offeruato si , che finoadefo-..- \
Lucho nome confernan quelle mura,
E'l manterranno,fin che'l mondo.duna, ¢

STANZA AX ) f \“ A

E che fuor del Cafpella il,popoh proves: «3000 (- ) SERB ;

Che ognor ne feappa quaiche sfucinata,

Per to piit gemse yeh? a peta 31 i loxofexes fer o

Cotantaé rifinita ,¢ maltrattata, — Qui pinto innansi stwile fentiva) >

Tornaril Poeca a difcorrer di Paride , il quale hayendo ridosto il Tura nely

fino flaco , haveva liberato quei popoli, i quali per riconofeimento del

ordinarono , che que! luogo fi chiamafie daallora avanti Montelupo

torna al campo , ¢ trova ogni cofa murata. .
LA quarta sboccarura, Cie ha sbaccato , ‘cioe.: manomeflo

yuol dire : ha bevuto tre fiafchi di vino , ec cominciato ibquarto2\Ipérbole, ches

fignifica : ha bevuto molto vino , sborcare propriamrnte Qgettare via

vino , che ¢ nel collo del fialco., per purgarlo affaordall!’ obia.yec, LiAQpesas!
CHI gli da , chi gh dona, ¢ chi git avventa iB’ detto giecofo nfato per burlares

uno, che figlorij d’ eflere {fpeflo:regalato ; es) intende ; chido ee 1

avventa, cice fafiate, ec. € lo fcherzo dell’equivoco’t:nelwerbomare, ¢ ‘
NON enol , quant’ un puntale d' agherto, Racufarurto.. Vedisfopra Capt, 10;
RINGR AZIO' del buono affetto . Termine di cirimonia julaci fi

ringrazia uno’del regalo, ¢ nello fefid tempo fi Ficnfa di rice -dicia- f

mo ;non voglio ,© non ftimo il regalo, feruendo , per obligarmiy Pinclinazio- ’

ne , che io veggio in voi di farmelo ; ¢ quefta teftimonianza  chehio dal :voltro
affetto verfo di me. ist
eH/ONTE Lupo. Finge , che Montelupo Caftelio wicino a
anch’ eg) quafi diftrutto bavefle nome da quota azionedi:
biamo per tradizione vuigata , che eglilfufleanticamente
flare il Caftelio di Capraia luogo allora forte’ fituato rincontro
cendo coloro, che.’ edificarono: Perdifiragger quefta Capra ‘Non sci guole altro, che
un Lupo, ¢ percid lo nominarono Caftello Lupo, che. per effer Mopraiun monte fi
detto Monte Lupo. Coca bg ED
GLI venne il grille. Gli venne voglia : E’ 1o-fteflo , che tocedsill

fopra G. 9. ft. 56. con Fp bei
ST &VGGIMENTO . Vn continuo ardente penfiero 50% I

iftruggimento vuol guarire , cioe vuol’ adempire queflo.fne-defiderio

all armata. Ii Burchiello , fe'ben mi ricorda ; Se/piri:d.amo rd

 

 

  
  
 
  
  
  
    
 
 
 
  
 
   
   

 

  
  
 
   
   

    
   
  
 

SPARITO cid che v' era, Non v' cra pil perfonasalcuna, ip
Baldone era diloggiato , ed entrato’in Malmanule. 495

   
 

 

“DVODECIMO;ETVLTIMOCANTARE. 525
SEVCINAT A, mola neg Vana gran quantita, Fuciaayyicn dal

che wuol-dic
ani

  

(O facina @ i

» 0 luogo dove Gi ri

no mercangic ; es
be capire una fucina prela. pec
operasoni

le ic
’ et re Bocce, Nov. 2. ee ane eena di diaboliche

igion dire ; 4

fi erika, ‘vuol anche dire il Barccaten de’ fabbri 0 delle fonderie,.ec,
| RIELNIT. « Malconcia,@aaca , finica , sopunatai ¢s.intcnde di fanita,e roba,

or STANZA a
pala 5 ¢ ne ri/contra un branco,
Preeti lemgean,
‘bi dietro fr afcicar fivedeun fianco
; | gh gi

   

 

STANZA XL

Chi ha fcatole , chi {acchi ,¢ chi sieehiee

Di givie di mifoee 5 dibiancheriay \
Va" altro ba una ranaca di forittwe 4,
Ch’ agli ha @un Pinto della dtercariby

agli fenza.adar albaco, £ piange ,ch).¢i le vede mal ficure y
& nee Sete egli ha rifcoffo; Pero che *l vento gliene porta via ;
_ Ciafcuno hail/uofardel) diguelletre/ibe, Vat altro dopo bauer mille imbarazzi,
a a og fi ba potuco beans ee Port’ addofso nna gerla di ragaxzi,
STANZA XIilL
reimbacuccato Arete feretto Le dine agliocchihantutteilfazzoletto,

a ria > ‘eJpelse, Ipe/so fi szattiene 9

E sgombrane 2 Py rocche,e pergamene

 

tra ys’ elle, le stanno. _

Chi'lf il ye chi >

Chi porta nngatto,e La caninainbraccio,

Sono

 

lex
te yede una gran quantita di gente , che fugge da Maimantile, per (cam-
parila vita , e porta feco,le.cofe pid grate; nel che il Poeta s’ accomoda a’ gen) di
quelle tali perionc, che fuggono , ed a quello che,per Jo piu,luol {eguire ia fimili
Seapets;

at ENC INGO . Se ben significa quantita di polli, o di pecore, o fimili , tuttavia

ne feruiamo per elprimere ancora quantita d’ huomini , Lat. bomnum manus.
Vedi aC, 6, tan. 35;

T ASCIC A dittro on Fucwene Va zoppo., per efler Mroppiato da.un fianco .
HA »ifeoffo Senza afpettare al abate, Glioperarj ordinariamente ri(quotono le
ro mercedi , ¢ prezzidelliloro lavori il giorno del fabato 5. ed il Poeta {cherza

col. verbo rifquotere , che yuol dire ricever denari ¢ ce.ne feruiamo ancora per
intendere Ricever butle.

GVIDALESCO. Malcalcia; Scorticatura . Vedi fopra C, 10, fan. 11.

TRESCHE. Qui intende bagattelle , bazzecole, arneli di poco,prezzo; Lar,
trica, Vedi fopra C, 10. ftan, 12,

SCATOLA, Lat, cap(ule . Sono caflette con fondo ,e coperchio , fatte con,
fottiliffime afiicelle in varie figure , fecondo che richiede Ja roba, che dentro ay
efie fi ripone.

SLANCHER/E, S' intende ogni forta di panno.lino., come tovaglic , lenzuo-
la, camice , ec.

PLATO, Lite civile » dal Lat, placitum , VedifopraC. 7, ftan.27.

MERC ANZI A. Altrimenti Afercatanzia. Cosi chiamiamo, in Firenze quel
Foro , 0 Magittrato, al quale fi ricorre, per far I’ efecuzioni civili,¢ ai we fon

fone-

 
 

ee

 

    
  

526
fortopofti tutti li Mercanti , ec. il quale ha particolari fat
‘MB ARAZZI, Spagnuolo, Embarazes » Roba’, th
6 feommodo ; ed’ aBBHaED il verbo imbarakzare y cht
nefi(, te tina Qanza ¥ ec * » ASSAM vaya
GERLA Da gero Latino’,'che vuol dire
Ma Voce il noftro Chimentelli nel’ Azsr nie?
di baftoni a guifa di gabbia da uceelli
larga je fondato hella’ parte pity tretca 5 det
per portare'i! pane eotto da un Iuogo all*aitro’y adatrandofelo”
alle reni; € et eeitind nim
firo Autore nella-eecéra alia Seréniilima Atciducheda’Clauiia ,
nelProemio'j dove Wie’ Che i Prafcica diecro’lina gerbi- tdi farfaallond COR
gran quantita dipropofiti) Pad bene atiche efere Che il’ Poeta intend’
mente ger/a , ¢ che voglia dire, ché haveflero due }o tre bambini'in u
talé gerle §'per:portari'pilr comodamente's coiné veggiamo'tutco”ll B
parire povere donne della Garfagnatiay ¢ d* altrove , che portino due , 0
gaaai addotio imgerie y 6 altri trabicvolifimill /-)) >) 9
tM BACVEC ATO} Copertd5¢ Ito ‘bene, ¢'s* ihtende pi
pert ibcaporn Vedi fopra C)1't, Man22: fe bendal Cy 6, fan. "64.
ne ferue per intendere Metterfi |’ abito addofio , tuttavia ¢ da norare ,’
intende il lucco , che ¢ l’ abito Curiale "if quale‘aiicledménte haveva il
per coprir Ja tefta , ¢ perd mietrerfivtal"abito fi diceva Pmbackecarff ; Si
inbavagliare. Giovanbatifta Bufia? asBehedetto 'V archi lettéra nona, | ¢
da AMona'coiet, “ed imbavagliatala la conddffero alle Palle se 3
OLE rifconera » Cive riconta la moncta 7 per vedere} i ‘
traruino y vuol dire imbatcerfi in-who Pma rifconttare libri ;’ ferirrtite 5
danariy contipecyvuolidit Rivederé je rorha®l Ay
Z HO29! ib. 920074 2» MH299) _ tegal cuba
» conteaffegnd Wi piabtd’, 6'di dolore’
il fazzokeio agli ocehi’, Veli topra G..9! ftan.48! ahaa
SCOMBRANO « Portan via Seombrare [ quati dal Latind excumiiliré , ton?
trario d? Ingombrare , che ¢’come fe fotle dal Lit! ficidmindare] deco
t6, ci (eruc per intendere. portar le? mafferivie"die ania cafa a tn” altras
mo in-vece'del verbo diloegiare’, sieggiare ‘Biaiken archi
BeASPL rocthe je pergamene' 7 re fruthenti atcenenti
habbiamodenowoped nel ©. ye Rad.'>\\E -pertamend intertddid tes aS
Carta con‘la quale'fermano ia ‘condechia’ ia (u'l# roced "per fadifitarell Blare
Ja‘dicond paielibéad #pérshe per tovpit: ol efier facta di carta pecora,€ he ti

  
   
        
  

     
     
     
   
    

  

et anche , carta perdaminay i 9
199 9.99 roel vp SAB AON Z AOR 249 Pr cxortert
Entra: Paride al fimdentro alla portarys 9 * Ma quel che mardni¢ha p tt dppor
OOue\g lipar a! entrasdentroun matelles \ -<Si% st’ veder tn’ pi m Capen
Chr ad ogni palje troua gente morta. Di {cope , e di fafcine f
1Oiper lo-mer 5 che (Ps per far fardeliay > ani? i iy :
Oud i IMGselis ¢ edusiiwi wh sil? - otary ate wage "A Ve Loi 4

   

aw
 
   

——— a

BOE Gg CUaio ‘Set etree eee

 

 

DVODECIMO;EDVETIMOCANTARE: = 527

oo ye STAN ZAXWE oo cen! Singeatte :
arriuato in pragza, Egliftaben, perc una fimil raza,
Perchi(domanda) ésigran fuoceaccefor . . C? ba fatto fe @ ogni lana un pefo,
 Egh érifpofto:egheper Martinarra, E' fi vorrebbe ( Dia me lo perdoni )
bid v'e dentro ,efcrine:latopre/o; . Gaftigar a milura di carboni . 7
‘aride entra oe! Caftello , ¢ vede molta gente morta , o malameate ferita., e+
jartinazza mefia nel fuoco per gaftigo deilc fue ftregonerie .
 MACELLO, Beccheria . Luogo dove s' ammazzano Je beftic per vitto dell?
mo: E per macedo intendiamo Strage , 0 difipamento di che che, fia. Qui
Iptende , che a-Paride par d’ entrare in una bottega di ua macellaro in riguardo
=| molto fangue , che vede (parfo per il Caftello. Cosi quel che dice Dante, che
V go Ciapecta tofle figliuolo d’ un beccaio di Parigi., Sccfano Pafquier va interpe-
trando , che abbia voluco dire di un bravo soldato, quale era {uo Padre, che per
Ja @trage che faceva , era riputato , come un maceliaro . ‘
CHE fea per far fardelio, Lat, vafa collig't , Che & vicino a morte ;, fla-per an-
darlene da quefto mondo . Vedi fopra C, 4, ftan, 21.
CAPANNELLY di feope. Piccola capagna , mucchio , monte di (cope., ec, i!
eee quando era per I’ cffetto , che era fatto, quelto , cra dat Lacini :detto eon

Inc reca Pyra dal Greco Pyr , vuol dir fucco,e noi pure lo diciamo Pira, Dang
126. : ii
i Chi é in quel fuoco , che vien.si dinife ,
eed z Ds fopra , che par furger dalla pira,
iets : Ove Exeocle cul fratel fu mifo. »
SCRIVE ; lato prefo , Antendi ; ha cieito per {c quel luogo - /edem occupauit ;ma
Per maggior chiarczza di quefto detro , ¢ da fapere , che in Firenze G fanno ogni
@nco tra gli altri quattro mercati_, uno per Quartiere , che il primo nel Quar-
Gere , ¢ in fu ja piazza di S, Maria Novella il primo giorno di Quarefima, ach
quale Gi vendono Icgumi , feccumi , ¢ frutte . Li fecondo nel giorno di S$, Simone
nel Quartiere,, ¢ in fu la piazza di S. Croce, Li terzo la.vigila dituitii Santenel
Quartiere ¢ in fu la piazza di S, Giovanni , acl quale fi vendevano oche.; mas
quefto € andaco in defuciudine ; perche ¢ perduta l''ulanza di regalar |’ oca lay
mattina di cucti i Santi. LI quarto nel giorno di San Martino nel Quartiere ,.e»
in fu la piazza di S. Spirito . 1n quefto , come nel fecondo Gi veadono abiti , pans
Hine , ed ogni forta d’ arncfi , ¢ maflerizic;.¢ come-che acile dette fire concorros
Bo molti mercanti di panni , ed altri artefici d’ ogai forta.,. cost alle. volremanca
doro il luogo , dove polarfi , per farui.ia quel giorno la lox boxtega; onde. piglia-
Ho il luogo qualche giorno avant, ¢ fegnano jo {pazio dei juogo, , che piguano
con getio 50 altra unta , ¢ vi (crivono in leere cubicali LATO PRESO,, ¢ que-
flo feruc per impedire , che altri entrino in quel juogo.: Edi. gui diccadofi ; I
tale ha (critto /ato prefo in quelia cala, ec, intendiamo : quella cala, ec. ¢ per iui,
wne gli pud efier tolta . Cosi. dice, che Martinazza fcriveva dace pre/o in quel mon=
te di fcope , per iagendere y.chc havea tatto in modo , che.qucl fuoco.non Je po-
teva effer tolto ... . 4g Neds e402 an ic
|fatto a’ ogni Jana un pefo, Ha commefio ogni forta di de'i\to-fenza riguar-
do alcuno . “Si dice anche far d’ ogms erba fa/cio, Che in (uitaaza s’ jntende un’ nue.
mo

  

 

|

 
 

 

 

 

 

 
  
     

38 “1 ALMA NIP 1

mo feellerato , di cofcienza larga fhe Hon‘ tetne
giuttizia; che'in Latino’ pure fi ditebbe , ex guoliber,
mea quella; Aivdum fie-pratum , quod non persranfeie lit

b10 me loperdent, Detto da Ipocriti , perch ¢ in’ un’ certo
cenza a Dio di fare un peccato impune.1 Latini havevano'una i
che parte fimili + Si Dijs«placet’, " eee
_. GASTIG ARE a mifura di-carboni . Dar maggior gattigo di
il detingtente . 11 carbone ¢ fra le pitt vili/ mercanzie; chef
mifura , € per queflo nom ff guards cost: per la minuta in darne
bra , ¢ pero habbiamo quefto detrato , che fignifica : dar’ pile |
nel Morgante . ef mifura di crufea , ¢ dt carboni, + 0 RE Oa

STANZA XV. ; STANZA HVE)!

Ia quefpo., ¢ ognum parla della Strega , i i a
Si fente dire ; A voi ; largo , Signori ,
E un bnomaccion pit lungo a’ unalega,

Dal Palazzo fi vede conaur- fuori, Per effer vogavanti di galere

 
      
  

     
  

  

Poi fopra il Carro , ove Birrenoil leva, Chetal fa d Amoktante
E cinto ( come gia gl Lmperadort ) Eperch'egli@un ?
Dialorowmvece, a’ uncarton le chioma, Sentengtaro I hanea’ nfarey
Va trionfante al Remo, non a Roma, Che Atalmantil non ba legniyne Mare
STAN ZA XWPLY

Percid , mentre che tutto ignudo nato , Lat confulte it decreto ha renocate 5 ~
Senonch' egli ba due frafche per brachetra, Sicche di luimndn' ordine 8 >
Sh) bel trofeo fi muone , ed ¢ tirato Ed ¢\ Stato fpedito un Cancel res
Da quattro canallaccs dacarretta , €on pik famigli « farlo-ratzenere o>

. I. Gigante Biancone legato ignudo fopra un carro ¢ condorto fuori di: Palazzo
per efler menazo in Galera ; ma quella efecuzione refta fofpela , perehé Malman+
tile non haveva’, ne Mare, ne galere-, Haba 3 sun-

LARGO Signori’, Date luogo; Fate ala. I Latini far far largo dicevano Sum
monere, Orazio. Neque confularis Summoner liter. Vedi fopra C, 11. fam

PIV" lungo a wna lega, Iperbole ufatifiima per efprimere Lunghitiimoy Di
atiche pis /ungo a una picea, 6 LO alae

BIKRENO , Intende birro , e'fi dice’cos) per 1a. fimilitadine
con Kirreno , che fu amante d’ Olimpia,fecondo |" Arioito’, dal! snes .
capertamente birro diciamo : lo /pofo a’ Olimpia , th ial ene

CINTA di cartone (a chioma, A coloro,, che per delitti-fon las
frufta , afino , o berlina , fogiiono per maggior vilipendio meceereinteta un bet
réttone di fogiio’, che per-efier a foggia ai mitra-epiicopale lovehiamano milena,
quali’ fono ‘quelle , colle quali farono:dipinti nelle itira del PalagiodehPorelta
oggi derro del Bargello’ , 1 feguaci del caceato Duca @ Areael, le
per P antichita appéna fi veggono’. VeditopraG, 6. Man, 56, €eque
per cartone , che pet altro vuol dire quella carta grotia,che (erie
incartar pauat , cc, r

HAVO MO abandiera , Haomo a cafo , inconfiderato » volubil
riofo nelle {ue operazioni . +k Saga, Url al

 
 
   

 

 

SS ae
DVODECIMO,ETVLTIMOCANTARE, 525
IGNVDO nate . Affaito'igdudo % Vedi fopra ©, 2. fan.-64. I! Coloffo ad*noi
; e"mto ignudo ; faluo.che ba due frafche per braghertas cio’ duc
fogliedi vite-fatte di ferro 5.0 d’ altro metailo dorato , che gli cuoprono. le parti

  
   
 

& ¢ SESLOU Re Ub =e a
« CAVALLAGCCE da carretta , Coloro., che in Firenze tengono carrette a vet
ra? per-portar mercanzie yed arnefi da un luogo.a un’ altro hanno fempre caval-
lacci vecchi , rifiniti , ¢-ai poco valore , ¢ pero dicendofi cavalio da carretta ys"
intende cavailaccio di tal forta . Qui il Poeta finge , che il Gigance Biancone fal
{melo fopra.avun carro tirato da quattro di quefti cavallacci » perche 1l Colofio
detto Biancone fta fopra ad un carro , che fi. figura tirato da quattro, Cavalli
anarini ,. > ’ 2 a
1A vinocate il Procefso.. Intendi ha: mutata Ja fentenza, o decreto della galera
havendo confiderato , che non fe li poteva dare efecuzione , perch¢ Malmantile
non ha gaiere,ne dominio di mare . ; ‘

» » STANZA XVIIL STANZA XIX,
~Hragaxzi infrattanto , che fon triffi , E perch! ei.nonha in doffo alcuna vefay
© Aveder cio che fuffe , efendo corfi , Lo fegnan colpo colpo in modo,tate

Epaich' eglié un prigion ,/i fona avvifti, Ch’ mmnanzi ch' e finifcan quella feta,
let Bich eglieben legara, ¢ non puo fciarfi, Ne lo fuifaron , ¢ conciaron male;
ly) Wnitamente in un balen provuifi E al miteron ,che atorre haueainsefa,
Di bucce , di meluzze , rape , etorfi, ( Bench giammaifpuntate auefel’ aig,)
* Cominciarono a far achi pri tira, Conquei {uot merli,che non ban lepeane,
Ed anche non tiranan fuor di mira. Pigtiar volo alt aria al fin conuenne,
Narra gli {trapazzi , ed infulti , che yengon fatti al Biancone , ¢ con quefto
smoftra il coflume de i ragazzi Fiorentini, i quali quando ua malfattore ¢ condot-
-to per Ja Citta in full’ afiao, 0 metio alia berlina , lo trattano nella forma , ches
dice del Biancone , tirandogii torli , cio¢ gambi di cavoli 5 bucce di poponi, ¢ fi-
“mili immendizie. £ nota che havendo egli.detto , che Biancone haveva Jamice-
»ra, perché il Coloffo detto Biancone ada ha yeramente la mitera » fa che i sa-
~gazzi la levino co i faifi di capo-al Gigante Biancone». i
-4N-nn baleno . Subito ; In.un batter d’ occhio , detto fopra C, 11, flan, 42. Di-
ciamo anche: in men che noo,balena ; eflendo il baleno , o il Jampo 4. ficcomeyil
vento ,¢’l fulmine cofa velocifiima , Onde noi d' uno yche corra ¢ {parifca.yia
fuggendo , diciamo = £' pare il vento, Ha fatto comenu baleno. Corre y come units
Yacsta, Pare che"! vento fe loporti, Virg. En. |. 5- J ,
Primus abit , longeque ante omnia corpora Nifus ont
Emicat ,& ventis 5 & fulminis ocyor alis ,
Dove quell’ Emicat vaic : Scappa fuora ,¢ innanzi agli aleri , come um lampo, Si
Swede correr la piazza in un baleno,
«»LVON tiran fuer di mira. Colpivano nel Inogo , dove fegnavano.. Vedi fopra
~C. 1. flan..37. dove troverai co/po colpo , che fignifica ogai coipo, che ¢’ tirana.
Che diciamo anche Zorto bette, Mira ¢ lo ftelio che Scopus , voce Greca ufata.da’
« Latini,; facta da Scopein, mirare, ;
le PkIa Ache finife ques foffa. Primachee’ finifle quell? operazione ; Si dice
anche + quel/a mufica; quel baccano ; aes Ȣfimili , Vedi fopra C. ae fh 53+
c xx. ; +, tbe

rad

=eSh 2

See CUR SSS

=e

~ _ a ae

MBAS, %
eng

 

 

 
ss

 

 

     

530 MALMANTILE

MITERONE a torre . Quel foglio , che per derifione fi métte i
fattori detto mitera , come habbiamo accennato poco |
doi! capo al delinquente , apparifce a i ciecoftanti una roronda t
la parte di fopra di detto foglio molte volte I’ intagliano a guifa d
farfi fopr’ alle muraglie delle Citta ; ¢ cosi havevano fatto a quelle
e'perd il Poeta {cherza con la voce merlo,che é un’ uccello note
glia dicendo , che fe bene i merli , che haveva in capo Biancone n
mar mefle le penve , ¢ non havevano mai {puatate / ali ; tuttavia
vouare ,ed intende , che quel Afirerone fu fatto volare dalle bucciate ,
che gii tirarono quei ragazai , con le quali glielo levarono di tefta, s
STANZA AX. STANZA XXIL oy
Paolin Cieco , il qual non ba fuvi pari Ed ci lo donaa Bieco,e a Pasian
Nel fare in piazza ginocolar’ i cani , Col carro ,e tutcel' altre ap,
E vendea l' operetic , ed ¢ lunari ,
E proprio ha genioa fpar coi Ciarlatani,
Penf[ato ch’ ex farebbe eran denari ,
Se quel bestion veniffe alle fue mani ,
Pere’ baurebbe,a moffrarfi,quel Gigante
Pix caica , che non hebbe l Eiefante .
STANZA XX1
Cost prefa fra fe rifoluzione , ; Subito qui Paolino feende ,
| Vain Corte a Bieco,¢ lo conduce fuora; Per trouar qualche ft buon.
Gili dice il {uo penfiero , € lo difpone Havendolo ferrato fra due ee
etchieder il Gigante a Celidora; Accio non fia veduti da perfona, Vey

  
 
 

 
  
  
  
    
      
    
 
 
 
 
 
 
  
 
  
  
      
   
 
   
 
 
   
     

E Bieco andato a ritronar Baldone Bieco a tenerlo con due altri atendey
Tanto P infipilla , ¢ allora allora E fe lo vede muouer 510 ba ;
Ei corre alla cugina , e gliene chiede ; Ma egliha fortuna, perch écni grande,
Ed ella volentier elielo concede , Che non gli arrina mancod

fande,
Paolino Cieco ottiene da Celidora in dono il Gigante infieme co! carro , ful quale
era, ¢ ful quale lo condufle a Firenze, ¢ fi fermo ia fu la Piazza della Signoria ,
havendo chiufo dewto Gigante fra due tende ; affinché non fatie venduto , ¢ men-
tre.cost flando , Paolino cerca d’ una flanza , per metteruelo, ¢ farlo poi vedere
a coloro , che havefiero pagato un tanto per uno , come fi faceva dell’ Biefaate ,
fuccetle quel, che fentiremo appretio , * ie
“ ELEF ANTE, ¥u condotto in Firenze pid anni ono un’ Elefant
il popolo per Ja curiofica correva in gran numero a vederio forto ie logge
Signoria ( hoggi detta de’ Lanzi , perché quivi € il quartiere de’ Trabanti , 0 fan-
ti della guardia del Serenils, Gran Duca da noi chiamati Lanai’) dove fava rin-
chiufo in un tavolato , ¢ fi pagavano alcune crazie per entrarvia vederlos ¢
fio animale fingulare ne i noltri paefi , mori in Firenze per lo gra freddy elas
fua pelle ripiena, ¢ lo fcheletro nettato, ¢ meffo inficme fi confervano nella Gal-
leria del Serenifs. Gran Duca. ucoensini aie
INZIPILLO’ . Inttigo , ftimold , pregd inftantemente , ¢ forle voce corrottas —
Sill :

    
   
   
 
      

da hbillare , Latino foilare , infufurrare , trovandolt nella’ flor

traccaco feume: Di-ninwa miferedenca era fhato antore 5 ¢ nulla male:
date, ta

 
   
 

DVODECIMO,ED VLTIMOCANTARE, = 31

TRAINO. Diciamo guella quantita di roba, che poffono ftrafcinare duc buoi,
che i contadini dicono trainare , ed il veicolo chiamano traino, 0 treggia, La-
tino traba , 0 trahea , a trahendo, Virg. Georg. 1, Tribulaque , trabeaque , © ini-
que pondere rafiri . Si dice anche sraine una mafura di travi , che contiene quattro
Breccia quadre. Qui intende quel carro, fopra il quale era il Biancone con tutti
phate arnefi, ¢ pigia la voce sraino nel fignificato della voce rreno ufata per

rfi intendere carro , ¢ bagaglio dell’ artiglierie ; !a qual voce s'accorda ‘colla.s
Franzefe Train. Noi percio ja diciamo ora Treno,rapprefentando quella prooun-
zia ; ora 77a:mo coll’ accento fulia prima, non facendo conto della pronunzias
Oltramontana , ma della (crittura. Qui il Poeta dice Traine coll’ accento fulla.
penultima ; per accomodarfi alla neccitsta della rima . Franco Sacchetti nelle Ri-
me fimiimente pofe quelta voce nelia fine d’ un verfo,

Per tirar colti piedi un gran traino,

LA Piazza dela Synoria , La Piazza, che hoggi fi dice Piazza del Gran Du-
a ,¢ fi diceva de’ Signori, 0 delia Signoria, perché é d’ avanti al Palazzo de’
Priori , ¢ Gonfalonicri di Firenze , che fi dicevano Ja Signoria , nella qual Piazza
@ la fuddewta loggia , detta de’ Lanzi

CHE non gli arrsva manco alle matande , Cioé non gli arriva ai bellico , perché
mutande chiamiamo propriamente certe piccole brache , le quali fi potiany,quan-
do fi va a bagnarfi in Arno, per coprire le parti vergognofe , le quali mutagde»
per ordinario cuoprono dai bellico fino al principio della colcia .

STANZA XAlV, STANZA XXV.
Piange Siancone, e chiede altrui mercede, Quei tre yc ognor came cuciti a i fianchi,

E mentre il Fato,e la Fortuna accufa ,

Euor delle tende si guardo gira ,e vede
« Perfeoy'ha in man la tefta di Medufa,

E immoto refta li da capo a piede ,

Ne piit fi duol yma tien la bacca chinfa,

Perché col Carro, e tutta la fua muta

De cavallaccs in marmo fi tramuta,

Gi favan quivi,accioch'ei nofeappaffe,
Privi di fenfo allora,e freddi,e bianchi
eAnch' eglino fi fanna immobil faffo .
Ata perchs'l protungarmi non vi ftachi,
Giie me' ,c’ a Malmantile io mene palji,
Ove git amici Paride ritrova ,

E fente ,¢’ ogni cofa fi rinnova ,

 

Ii Gigante Biancone era cosi grande , che avanzava il capo fopr’ alle tende_ ;
nel girare , che egii fece la telta verlo la loggia de’ Lanzi, vedde i tcichio di Me-
dufa tenuta in mano da Perfeo ; per Ja qual vilta rimafe immobile , ¢ diveanes
faflo tanto lui, quanto il carro , i cavalli, ¢ coloro , che gli crano d’ attorno; B
cosi il Poeta da la fua fine , ¢ fi sbriga dal Gigante ; di poi ritcorna a dilcorrer di

_ quel che fi faceva a Malmantile .

PERS EO , ¢’ ha in man la tefta di Medufa, Quefta é una ftatua di bronzo , las
ae € fiuata fotto un’ arco di detta loggia de’ Lanzi; opera di Benucnuto Cel-

i; ¢ rapprefenta Perfeo con la tefta di Medula in mano , verfo ia quale flacua,
guarda il Coloffo detto Biancone , percht ¢ di marmo bianco. E nota la fayola
di Perfeo figliuolo di Giove , edi Danac , 11 quale uccile Medufa figiiuola di For-
co ftrupata da Nettunao nel Tempio di Pallade, la quale percio sdegnata conuer-
tii capelli di Medwla in ferpi, ¢ fece che la fua facia faceili diventare di faffo
coloro , che la guardaflero : Ma il detto Perfeo havuti da Mercurio gli ftivali, ¢
la {cimitarra , mentre Medufa dormiva s le tagho ia celta , 1a quale pot ee

xX 2 mefle

 
 

 

 

   
  
 

sg IFATHAI OM ES +9 vi h
AS3HL ) pire Ofisip MALM he thy ; 4 ie on fi
miéfie nel proprio 'feudo , Di quefta favola fi ferué il! Poeta 4 } 7
gante ;dicendo, che per haver’ eghi mirato quefta tettadé-]
marmo , € cosi da graziofamente una favoloia origine a quefto f
rapprefenta Nettenno Dio del Mare’, ed! é»pofto nella: Piazza’ del G
fopr'ad un carro tirato da-quattro cavalli marini nel mézzoa una
quale riceve I acqua’, ‘che featurifee davaleuni niechi, © conchiglic
in mano da alcune ftatue di Tritoni-alte quanto le gamberdel d
or dette flawe flanno attorno :"E quefte il Poera finge’, che fieno

mipagni , che dice fargli cucits a i fianchi, ¢ che non gli arrinano a le
dé’; E cosi viene a conformarfi col gruppo , che fi vede di quefte ttarue
fo tutto di marmo ,

CVCIT 1 ai fanchi , Stretti attorno , come fe fuffero euciti , Detto uk
per'efprimere uno , che mai fi levi-d’ attorno a un’ altro ;€ qui corna bene ,|
Ché quelle flatue fono cosi ftrette attorno aj Coloffo, che paiono cavate:
fo marmo , del quale ¢ cavato il Colofio .

GLle me’, Glié meglioy. Vedi fopra C. 2. ft, 10. <a? S

PSTANZA XXVEy STANZA XXVHE
Poicht Baldone eAalmantile ba prefo , Cos} cercando le grandexe i |

E tutte quelle povere brigate Soe @altrihor feo Ve, ?
Saluopera chi non fi fuffe arrefo ) Onde tornata Celidora , il Lage
mii fe ne fon ite a gambe akace 5 De i popoli padrona , ¢ dello Stato
Sitché'da queite havendo al fin coprefo Temendo ancor de’

 
  
    
        
     
 

.

    
     

 

    
   

 

 

       
 
 
  

‘Pot Bertinella , ch ella l ba infilate ; Nuovi Miniffri fa , nuove ve i
Perammaxzarfi sfodera un pugrale, Se ben de i primi poco ha da temere
Ada quei,ch'é buono,non le vuol far male, Che tutes ban ripiegate le bandiert i
STANZA XXVIL. STANZA BALK
Clienon fo come gli efce fra le dita , E per eftinguer la memoria i
B/fulta in Strada,che le gabe ha deftre, Di Bertinella in ogni gente ye-loco }
Ov" ella a ripigliarlo’é pos /pedita Si levan le fue armi , il fuo ritratto,

Tagliato in croce fi condanna al fusce's
aE perch'elt habbia a raccorciar, la gita, Vn bando va di poi, & averum patto
Le fa pigliar la via dalle finefire; Neffan ne parti pite punto ye poco
NEMa wa sh 5 ma poco poi le importa Sotto pena di fear in fu la fume
MT rovaricht amarza,fe viginnge morta, Quattro mefi al palarzo del Com
Celidora tornata padrona di Malmantile fa buttar Bertinelia
ordina nuovi Magiftrati , ¢ comanda, che non fi parli pitt di Berti

villime'pene. jo faites

Dix'chi dopo di lei fa le mineftre ;

    
        
 
    

  

ELLAL ha infilate . \ofilar le pentole , vuol dire Effer rovinato
ver finito-, 0 perduto la roba , ¢ la vita, ec, clie di tutto s*in cok
mente. “tale ? ba inflate. Latino decoxit . +20 sett

LE gambe ha dere, Non, che quel pugnale haveffe gan
dire } cheetlendo grave , gli fu facile andar’ a baffo in ftrada’;
perie ‘fineftre anche Bertinella da chi fa le minefre , ciok dachi
avichi comand; chee Celidora ritornata padrona di Malmanule .
gacge ae peccato, Ha la pena det fuo fallire, ¢ che ha m
whet ; Fs

   
   

  

 
 

  

1 v¢ t E LAN 7
-.  DVODECIMO;EDVLTIMOCANTARE. | 533
' flaver voluto per ftrade indirette farfi Regina’, ufirpando queld' altri iio! is > /
be i. icsanlig voghamosintendere uno,che piocenea oe taper fare Ogni-cofa
meglio degii altri diciamo ; M.raleéit Lagi, Che il Lagi fu anticamente un Sen-
icato wv Firenze , che faceva tutti i negozzj della piazza’: Si dices
rO per {cherzo , ¢ per una certa ironia , ¢ derifione. bo “ogee
* HANNO ripiegato le bandiere , Cioéhanno finito ; Son morte, Il Petfiani ,
parlando di fe medefimo in quetto propofito diffe + ty
core edi primo tramontano a queft® afcintte ae)
$i Be Ditems pure sl requie ,¢ il Miferere , :
Perch’ so fo vela , e piego le bandiere ; ’
_ E buona notte ; a rinederci tutti ,
LE fue’ arm , Intendi'}*infegne della fua cafata , 0 ftirpe . ues
7 ~ STAR in [u la fune quattro mefi. Now & pofibile' ftar in fa 1a coda quattro
y hore , non che quattro mefi., ond’ io penfo, che con quefta iperbole voglia iaten-
: fia condennato alla morte , alludendo agi’ impiccati , che in un certo modo

 

quando pendono daile forche a vifta del

popolo; ft poflono dire are in fulla corde,

be in fulla fune .
¢ STANZA XXX. “STANZA* XXXIIL
jee | Yr Orarore intanto de’ pitt brani Spiegafi fe desea 4 ttn tavolotto
" ACelidura Aaimamue inuia , Vol abito mavi di mexzalana ,
a ‘Che det Caffelo ad effa da le chiavi, Che infu fianchi appiccato ba per diforto
Evende omaggio con la diceria ; Pn lindo rid aief alla Romana;

 

Ed ella in detti macffofi , ¢ gravi
Pronta rifp a tant’ Ambafceria ;
Inds le chiavi piglia ye nn’ altro mazzo
= Wi quelle delle flanze det palazzo.
ae STANZA AXAIL,

E perch? gti é un perro , ch’ eli’ ba voglia
Di riveder , come ad arnefi e pieno ;
Del Mamoye d'altri addobbsfi di/poglia,
E comincia a girarlo dal terreno ;
4Guardarobi afpetta, ead ogm fogiia,

Poi viene un verde nuouo camiciotto
Con bianche imbaftiture alla balana ;
E poi due trincterate camicinole ,
Che fanno piatza d’ arme alle tignuole,
STANZA XXXIV,

Vua Rimarra pur difaianera, ~~
Per dove fifa a’ faffi arcifquifita ,
Perché gli aliorti, ¢ it banero a spalliera
Pavan la teftaye in giu meza la vita,
Portandola alle

‘i; te ,o0anna fitra,
\C* ad aprer gli ufci patono it baleno; Torre,e comprar fi pio roba infinica ,
é E fubito poi lefto-uno safiere Cb elt" hadue manicon s) badiali,
mn Quand’ elta palfa , le alza le portiere . Che é ine quattordici arfenali .
f STANZA XXAIL, STANZA -XXXV,
Ed ella fe ne va ficura ,e franca, Vina cappa tane bella, e pula
b Sapendo ogms traforo a munadito , Di cotone ; fe ben vefta indecifo ,
ie Perché troppo.non é , ch'ella ne manca, S’ ell’¢ di drappo , o pur ringiovanita ,
EP abito, fin quando havea mario, Perché non fe ie vede pelo in vifo ,
‘0 Scefe ; )£i70 y fali ne mat fu hance s Evvi @ abiti pur copiainfinica ,
2 t Sin che non hebbe di veder finite; Mia chi unto , chi roto, ¢ chi ricifo ;
2 All’ ulssvia fi fece in guardaroba Che il tempo guafta tutto; ¢ per marura
% eAprir gli armadi,e cavar fuor 1a roba, Cofa bella quagzit pala, ¢ non diira,
4 Malmantue manda un fuo Ambalciadore , 0 Depataco a renaer’ wbienes :
a Ce.
f .

 

 

E
 

  
       

44.534
a Celidora ; ¢d ella attualmente , ¢ corporalmen
tutte le flanze del Palazzo, ed in Guardaroba fa la
veramente adeguati a una Regina'di Malmantile.. 3
RENDE a Ja diceria , Cioé fece una Orazione d’
mone , 0 Difcorfo , col quale refe ubbidienza. is. 4
HA voglia di rinedere. Ii Poeta (prime beniffimo il genio unit
fire donne , quale é di rivedere tutte le caffe 5 armadi , ec. fubito, che
© maritaggio entrano in una cafa a loro nuova, ho isch ete
TERRENO., S' intendono qui , fecondo I’ ufo., le prime fanze d’ una cal
che fono al piano della frada, Del reo Terrenoé la tetra ftefla cosi,0 cosi ¢
dizionata . Latino terrenum ; folum, ager.» - send 5 -
PALONO il baieno , Ciok tannopretto, Dante Pars 25. Subito
di baleno . Inf, 22. i2 men, che non bafena, vatiot ”
OGNI traforo . Antendi ogni porta , ognicriufcita ,/ogni minima:
4A MENA dito, Sa benitimo . Latino caller, Le fono notifime ft
L ABITO’ fin quando banea marito , Celidora , comes’ ¢ detto fopra C,
Fu moglie del Re di Malmantile , ¢ da lui haveva ereditato i Regno, i)
MAVE , Color wrchino chiaro. Azzurro sbiancato , i
GV ARDINF ANTE. Vedi {opra C. 5. tt. 8. * : geomet
MEZZ ALANA, Tela fatta di lino ,é lana , che inuna fola parola fi dices
ancora acce//ana , quali accia , ¢ (ana ; roba aflai da i noftri Contadini.) |
C.AMICIOTTO.. Cosi chiamano le Contadine,quella'velteda donna, cheles
Fiorentine chiamano fortana , Et
CON bianche imbaftiture alla baixana , Coftumano le noftre Contadine di fares
nelle Joro vefti yicino a terra una cintura con punti di refe bianco in ful nero jun-
ghi , acciocché fi veggano da lontano , ¢ queiti punti foftengono una piegaturas
fatta nel giro di detta velte per accortarla, ¢ ferue a loro per ornamento ,0 guat-
nizione , ¢ fi danno ad intendere di far creder nuova la medefima — caufa
di quella punteggiatura , ¢ che aliora fia ufcita deile mani del Sarto; il ee
quando vuole imbaftire ,.0 dar priueipio a cucire yo’ abito per mettere int 9
eda fegno i pezzi, che vuol cucire , ¢ {olito fare tal punteggiatura larga y das
queito imbaffire fi dice imbaftitura altrimenti feffitura, © ritreppio, Latino /ubfutnr4.
E quefto verbo smba/tire feruc per intendere ogni cola principiata,e non perfezio-

 
    
      
        
  

   

  
 
    
   
 
   
 
  
    
   

    
 
 
 

 
   

nata ; come éo ho imba(Pito L' oraxione , che debbo recitare 5 ed in poche ere ” :
che diciamo abboyzare . we

BALZ ANA. Iniendono il giro da piedi della vefte ; altrove Pideos ‘Latino
limbus « LF

TRINCIER AT E.camicinole.. Vuol dit camiciuole confumate dalle tignuoles »
per la fimilitudine , che ¢ tra una campagna pieaa di trinciere , ed.un panno ple
no d’ intignature , che percio apparifce bucato, € trinciato , Vedi fopraC. 8. ft
51. E.che cofa fia camiciuola . Vedi fopra C, 6, ft: 57, at otwe att

BANNO piagza a arme alle tignuole, Vedi opra Co. 51. -quefto medefimo
concetto fopra il capo del Tura; B che fia tignuola al C, 6. ft. 54. € Cs 10. (h 12+

ZIMARRA, Abito , che gia ulavano portare le Donne Fiorenti all?
altro abito detto /orrana ; il quaic da i Latini ¢ detto amiculam , il qual’

' YY

 

 

  
   
 

tie
a

“= SSeeresiut

 

 

 

DVODECIMO,EDVLTIMOCANTARE. ~ 535

‘veramente affai decorofo, e modeflo , ¢ non come quello , che ufano hoggi, del
quale fi pud dire:con Quinto Curzio lib. 5. Feminarum conniusa inenntinm in prin-
cipio modeftus eft habitus , dewde fumma quaque amicula exuunt , panlatimque pudoré
profanant , ad ultimum ima corporum velamenta proyjciunt , Ma tornando a propofi-
to: Quelta {pecie d’abito detto Zimarra haveva intorno al collo un collare gean-
de (che chiamayano bavero ) fatto di tela incollata,e cartone,e ripieno di ftecche
d' offo di balena ; ed in fu le {palle , dove ha principio il braccio un giretto actor-
no al braccio farto della ftefla roba , che il bavero ) qual giretto il noftro Autore
appella aliotti , perch cosi fi chiama , ed alle volte fi dice piffagne ) dal quaies
pendeva una manica larga.,¢ grande quanto una buona sporta , la qual manicas
non s’ imbracciava , ma ferviva cosi pendente per ornamento , ¢ per una certas

“grave accompagnatura ; ed oltre a quefto dava commodita di riporvi fazzoletto ,

Oaltro , che occorretie. Di quefte maniche , tali {e ne fon vedute a’ mici giorni ,
che farebbono fiate capaci di cinquanta libbre di grano I’ una , ¢ pili; © perd il
Poeta dice , che fono 11 cafo per andare alle nozze , ed ai mercati , perché vi {i
pud mettere molta roba dentro: E gli-aliorri , ¢ banero difenderebbono da un col-
Po in riguardo della roba, di cui fon compolli; E dice /a rea; perche quefti ba-
veri , nafcondevano dentro di loro tutto 11 capo di chi gli portava ; ¢ tali aliorti
fi fono veduti , i quali coprivano pili di inezzo il braccio .

DOVE fi fa ai fafi , Dove fi tirano le fafiate ; il che fegue in Firenze in Mer-
cato nuovo , dove 1 garzonetti delic butteghe de’ Setaioli quindici, o venti giorni
avanti alla Solennica di S, Gio, Batilla fra il mezzodi , ¢ il vefpro fanno fra- di
loro alle fafiate , ¢ necetiitano tutti li bottegai di quelle contrade intorno al Mer-
cato nuovo a ftar ferrace per quell’ ore ; ¢ quefto fanno per folennizzare la decta
fefta quel tempo innanai ; ¢ per quefta ragione tutte le botteghe, che {ono in quel-
la firada, dove tirano i fatfi, hanao la riufcita in aleca ftrada per di dietro , di
dove entrano i macitri , ¢ lavoranti , fenza aprire Jo (portello principale , ¢ quivi
attendendo a i Jor Javori , laiciauo che i loro ragazzr Gi piglino per quell’ore tale
{paffo 5 anzi ci (ono taiuoica de i maeitri , che comandano a1 loro ragazzi, che
vadano a pigliarii , fpaveatati da un profetico detto: Guai a Firenze , quando in.
Mercato non fi fara ai faffi ; V {ano di fare a’ fai anche in Roma i ragazzi Tra-
fleverini. E fare a’ faji , Hgucacaiente s’ iotende » Mandar male , rovinarfi, get-
tar via il fuo , Latino di/apidare , fare alla peggio , ¢ operare fenza giudizio ; fi
faceva a’ faffi ancora in Firenze per accafione d’ allegreeze pubbliche, ¢ una fine-
fica di rame traforata fu pofta al Palazzo de’ Medici,oggi de’ Marchefi Riccardi
Per vedere quefto {pettacolo , come ¢ {ato da altri {critto , ed offeruato .

ARCISLVISITO , Ui cafifimo , buoniitimo,, attifimo, ¢ pil, {e pid fi pud E

dire. B’ un termine , ches’ ufa per farfi intendere ; pit fu, che il fuperiativo, di-
cendofi buono , pil buono , buoniflimo , ed arcibuonitime . Ma dicendofi buo-
NO , Migliore , in vece di pil buono , ¢ ifquifito in vece di buoniffimo , che fa.
V effetto del fuperlativo di buono , non pare che fia ben detto pili ifquifito , e»
isquifitifimo,facendosi cosi'un fuperlativo di fuperlativo ; tuttavia per J'ulo inteo-
dotto non farebbe riprefo chi Jo facetle ; ed io crederei, che fufle meno biatime-
vole dire , arcifquifite , che isquifititfimo , perch non trovo troppo in ufo il dire
pil isquifito 5 onde non pud s' ufo antrodurre isquifitisimo.s che toguirebbe al pitt
squi.

ae

 
    
     

536, uuie nohace dae aaa
isquifito..;L Latini dicono bonus, melior 9 che: Q d F
byono,, migliore 5 ¢ i/quifite ; ed io conde cic i che Bctedfinn piste ass {
miffiaus:, che faonerebbe pit isquifito , isquifitifimo, fe it I

trova eptimiffimus.. Appretio det noftri Autori Tofcani fi trova , 1
molto , aijai , ¢ fimili a i fuperlativi , come notammo Coat 17. >} ia r
buona grazia di efi , lo flimo.errore , perché molto , piu 5" » Hiufilis
faculta di scemare , e non cre(cere il fuperlativo ,. aa

er efempio if tale ¢ Luoniffimo , vuol dite il tale é perferramente
iamo molto , certo’, che (cemiamo la perfezione di buono 5
molto buono , ma von perfettamente buono , eficado maolte una’
{2.5 € non indeterminata , come ¢ i} fuperlativo: EB — » che
1iguifito , ¢ isquifieifimo , © arcifquifite , hanno prefa la vace {
tivo da per fe, ¢ non come per fuperlative di buano ; il che vi 7

tofna poi all’ addigteivo aigiiore , che non riseve alerazione 5 nomdicendof » nondicendof i ;
migliore , Be miglioriffimo , le hen fidicewmalte migiione 5 e:alfai mia
marlod' eflenza ;. come ia bbiia thes detto s»perché solo 5.0 affat miglit
men buono,che non fa migéiore aflolutamente detto:, fe non comparando:
all: altra quale fia.di loro meglio, st Amr
i ZANE, Colore fra il paonazo ,¢ i} lionato . 2p
OTONE, Vuoldire bambagia non filata, Manoi per cotone:
forta dipanao col pelo annodato ; come.é la {aia rovelcia 50 il rovele [
Hon fi dicono corone fe non. hanno il pelo aanodato, che allora fi dicano di coteney
© actoronati, Dice , che num ¢ certo fe fia rowefcia 5 0 drappa 5) pceaim
la feta. 2 ellendogli caduto il pelo , per efler logoro je perché:é fenza pelo dice

che € riagiouanito ; Sicch¢ in fuftanza vyol dire che: era ufato, i lal.
R(CISO , Qui vale per intendere confumato nelle piegature d'un di !

  

   
 
 

    
    
   
 
     
  
 

  

epanno 5, per efiere ftato cosi piegato lungo tempo; che per altro ri
“un Jegho , o altro materiale tagliato ne] mézzo yed ¢ il contrario’ div rife se

   
 
 
 

nel oy pela per illungo, Vedi fopra ©. 11, tan, 36, ricife,
ANZA XXXVI. STANZA X&K .
Bafta es eve qualcofa un po cattind , Due altre ‘armadj poi i fur

Che Celidora ha quint abiti , ¢ panni , Che ? anoe tutto
Che al certo (tuttanolta ch’ ella vina )
Puofrancamence andar in lq co gli anni y
Ma perehe al [uo char magnonos'arriua E un’ altro di pin tr
‘Di certe roppe , fcampoli , e foppanni + Bealze ye fearpe ye)
Top Wimpaccio vollese a quella gente, Chea vederfi p er
"Ch edt'ha a’ intorno,farne un belpreséte, Ve poi'la nidoigi
. STANZA xxxViil ft
mui fe fi parte ed Apre uno Riperto A
2 intagli,e a’ arabe/chi ornato,e ricco,
“E trois due cafferse di belletto
Cort! altre di pezrette , ¢ @ orichitco
va il Pocta a narrare glia arnefi,e
hon fi parte dallo feh
‘ faye | Ae a ee

  
  
 
 
 
 
 
 

 
 

n

    
  
  

ME

   

  
 

Si elcr ibs.

a

SERRE ES © =

SSE ST Pesrsr st i ta.

 

DVODECIMO,EDVLTIMOCANTARE: $37

contro alle donne , moftra ; che fe ufano il belletto , ed il lifcio, hanno anche
bifogno della medicina da rogna , ¢ del rottorio .

VN po cattiua , Quel po vuol dir poco per la figtira Apocope ; ed un poco cat-
tiva 5 trattandofi di abiti , ¢ d’ altri materiali , s' intende per lo pit’ , confumati ,
2

vecchi .
TVTT AVOLT A, ch’ ella viva , Pub francamente andar in da con eli anm , Pav

che voglia dire , che fe Celidora vivera , ha tanti abiti, che le bafteranno molti.

anni fenza farfene di nuovo ; Ma dall’ effere gli abiti della detta qualita , fi com-
prende , che fcherzando vuol dire , che fe Celidora vive , inuecchiera , percht
andar in Id con gli anni yuol dire inuecchiare , come s’ accennd fopra C, 2, ftan. 2.

(siginines Ritagli , pezzi di panno , 0 drappo. Scampoli, vedi fopra C. 11;

in. 22.

SOPP:ANNI, Fodere , cioe tele vecchie , che hanno feruito per fodere d’abiti.
Scherzando burla la generofita di Celidora , la quale con quefte galanti ciarpe ,
che fon fondacci d’ una bottega di rigatticre , o ferravecchio , regala i (uoi pil:
cari per non apparir meno generofa di Bertincila , che regalo 1a patcona , come
vedemmo fopra:C. 1. flan. 81. a :

D* oronetro: Par che dica d’ oro pulito , ¢ puro , ma intende wetto d’ oro , ciot
puro ; fenz’ oro , Equivoco ufatifiimo in'quelto propotite ,

LA miaferizia per la cafa. Incendiamo 11 Cariello’, 0 turacciolo del ceffo ; es

flo 5 perché un tale detto Galeno,che andava per Firenze vendendo tali cariel-
li, gridava shi vnol la mafferizia per la cafa , in vece di dire , chi vuol Carielli ; od
¢ra bene intefo.da cutti ,

RABESCHI , 0 Arabe/chi , Specie di pittura fatta a fogliami, fiori, ma{che-
roni., © altro , tutto aggrottelcato , cioc fproporzionato dal naturale , detto co-
si, perché forfe tal maniera fia venuta d’ Arabia , fecondo che fi pud dedurres
da. Cel. Rodig. Jib. 29. ¢. 5. dove trattando delle Lamie , ¢ delic Sircae , dice ;
LaAmmiam vero opera parerga ex Arabia maftichen vocant ,

SELLETTO. Lifcio. Meftura , con Ja quale fi lifciano , ed imbellettano les
donne « Vedi fopra C. 9. ftan, 38.

PEZZETT E . Sano pezzi di tela bambagina tinti col cremisi , ¢ zucchero , ed
altre fono'di carta fabbricate in Spagna, ¢ fc ne feruono le femmuae per colorirai
di rofio la \faccia .

ORM AlCCO., Gomma di Ciriegio , di Pefco , 0 di Sufino , ec. della quale fi
feruono Je femmine per luftrarfi la faccia , e per appiccarfi veli in fu la teita .

“PER Jambicco. Adagio adagio {caturendo da piccioli fori fatci nel coperchio
del fiafchetto., come s’ufa dei’ acque odorifere . Lambicco ¢ il nao della campa~
na ,¢ d’ ogni cappelio per ufo di ftillare , donde /ambiceare , ¢ pafsar per lambicco,
# incende ftillare ; B /ambiccare , 0 lambiccarfi it ceruello , ¢ lo ftello che mulmare ,
detto fopra C, 10, ftan.7.

ALLERA, Pianta nota, le di cui foglie eruono per cauteri ; ¢ ¢osi i ceci bian.
chi , li quali per tal effetto erano ia quello (tipo.. Da queite cofe vili comprenda
il Lettore , che il Poera fi maaticne fempre in fu gli (cherai , deferivendo una Re-
gina , ¢ Palazzo ricchi di quegli addobbi , che fon conuenienti a una beac {tant
cOntadina , ¢ decenti alla grandezza d'una Regina di Maimantile, ,

% Yyy . STAN,
 

 
  
 
 
 
 
 
 
 
 
 
  

Sh. MALMAN TILE 1980
STANZA XXXIX,.. , ith NZ 3
dun caffon diferro vada REREO y c i i co#lor |
L Quiuitvoua i] morto , nia dd vero, s -
Che i diamantiye le givie di gram pregro
Lon v'bano che far nidla,e fono un zero;
Lerche fi tratta, che vi. Safe un wero
bi perle , che fe ben pendeana in nero
Examsi groffe , che ft [parfe vace , ;
Ch' ell’ eran poca manco d’ una. noce ; Sun i quartrini 5 i precioli ye i bateati,
STANZA XXXX STANZA -XXXXIL
D? anells ya! orecchini Vé1h marame} ‘Poi ne venixan gli occhidiciueste;
“Tanti gioie!ls pot , ch’ ¢ un fracaffo ; Ma il profeguir piu olere fa interrortes,
Perc’ alla donwa:

Di medaghe dorate 50, vavindi-rame’ a
dir , che" Duca levolea far

Vn. moggio ne mifurano, , @ di palo;
Ala quella ¢ {parr ates, ed nn litame Ond? ella il tatto nelcafjon rimette

    
 
  
 

  

    
 
        
      
 
 
 

Rifperto alle monere 5 che pit baffo E riferrato feende giwdi (orto,
Le piit belle comparfero del mondo ; Oue Baldon ? afpercarn iftinali,
Ch! in faseri pofes creffi flanvo al fonda: -» one partir di quini fha'im ful? ali >

: STANZA: EMBKI MO vinnd cow 2

Per ¢ agginftare omas tutte le cole 5 In punto, @ quefto fine aller
Che pin defiderar non fi potea in : ier ‘bined
Egli , ch’ eva per far come le/pofe La puliva.per metterie la fellay

  

LA ritornata s idef? alla Dacea, Licenrioffs costidullaforellay © >
Celidora trova il caffone de’.danari, , ¢ coi tal-occatione i Poera’
monete Fiorentine eficttive , ed immaginarie. . kn tanto che Celidora va vedendo
guefte ricchezze ; vien da lei Baldone-fuo cugino per liceoziativ) 9
TROFA it morta, Cioé trova il buond . Diciamo rrewar it morta, 0 fare nits
morto , qnand’ uno trova ripod quaiche gran vallente , © fa in gua-
dagno.. A . P
LON 0 ba che far nulla, Par che voglia dire non fi fRimano, vifpette al? altres
Givie , che fono in.quet /uege ; ma in eisai vuol dire ; che quedo non ¢ luoge per toro
cioe non ve ue fond, i b tone Ses
Sf trata. Si difcorre; Termine aflai ufato per efprimere una ches
s' habbia di qualche cola’; quafi-dica > Si difeorre comunemente , che’
cosh . .
AL marame. Voa quantita grandidioma . aferame propriamente.wuol dite ogni
rifiuto di mercanzia, come quella, che dak mare ¢-geteata’a’ iva’ bi i”
tum, Ma quando diciamo marame nel modo; che! & detto: nel eel
intendiamo abbondanza cosi grandé.d’ una cofa y che generi naulea, €
difprezzabile la medefima cola. Fra i nottci Contadini Gedice
tendefi ? avanzo 5 ¢ rifiuto delle frutte rimatte lord, dopo. la celta’, o° vel
delle migliori » noa fo {¢ effi Rroppiano'la noftraparola y o-feonoi Cori
la loro 5 dico bene che mi pare pit fighificante; Amaramejehe J
Fiorentino quello 5 che quefto 5 che per cost dire’, ha del Nape
Vedi il Vocabolario della Cru(ca alla voce Cerna’,

      
     

y

 
   
  

   
 

DVODECIMO,EDVLTIMOCANTARE 439

RN fraceffo iB 10 feflo che un flagello ; umbarbaglioderts fopraC. 7; ftan, 5,
VA moggio. Ai.noftro moggio é di ftaia 24, lo ftaio ¢ di tibbre 50.-di grano ,¢
la noftra libbra ¢ once dodici, Ma qui ¢ detto iperbolico , ¢ fignifica quantita
graodifima. 5 1 F

. RISPETTO aquefto, A paragone di quelto; ciot'a paragone delle monete 5
che fon pi bafio, .* i iy

1 pefes eroffi franno al fonds , Detto , che fignifica : 11 megtio‘fta nel fondo.

PLAST RA, Elo Scudo , 0 Ducato d’ argento Fiorentino , che vate lire fette
¢d.¢ moneta effettiva . 1} Fiorino€ moneta immaginaria , ¢ valeva quando pia ,
€ quando meno, efiendoci anche il fiorino d’ oro , che forfe ¢ quello che habbia-
mo ancora hoggi @ oro effettivo , ¢ lo chiamiamo zecchino gigliato , ma il fiori-
ho ne ismmaginario , ne effettivo appreflo di noi non ¢ pid in ufo, Scudo d’oro
@ moneta immaginaria ufata da i Mercanti per facilita di ferittura , vaiutandolo

ire fette,.¢ mezzo , fe ben molti per (cudo d’ oro intendono la mezza doppia .

a Lira¢moneta d’ argento effettiva , ¢ fi chiama Cofimo, ¢ vale dodici crazic.
Ul Giusy), che fi chiama.anche Pavolo € moneta d’ argento , ¢ vale orto crazie,
T1Cacligo »pur d’ argento effettivo ne vale {ci ; ed il Teftone val duc lire ; que-
fia moneta gia in Firenze G chiamd Riccio , dail’ impronta della tefta del Duca,
Aletiandro ce' Medici, che era ricciuta. La.mezza piattra ¢ a’ argento effectiva,
¢ vaje lire tre , emezzo. La crazia ¢ moncta d' argento baffo , ed é V ottava.
parte de) giulio . Il quattrino ¢ moncta di bronzo effettiva ,ed & 1a quinta parie
della crazta . Li foido moneta immaginaria che vale tre quattrini ; ed il battuco
neval.due : hoggt |’ habbiamo ambedue di bronzo effettive « H'quattrino fi di-
vide ig quattro denari di bronzo effettivi , ma hoggi non fe ne vedono , (¢ non in
occafione di tributi Ecclefiaftici , che fono prefeatati , ¢ fon poi refi , perche gli
potiano haver un‘aitr’ anno . n ,

OCC HI di Cinetsa , {ntende le moncte d’ oro , come il doblone , che vale lire
quaranta. La doppia,che vale lice venti. La mezza doppia,che vale lire dicci , 11
quacto di doppia,che vaie lire cinque. L’ ottavo di doppia , che vale lire due , es
me¢zzo5che tutce {ono d’oro effettive . Habbiamo ancora il zecehino , il quale»
chiamiamo gigiiato', che vale lire dodici , ¢d € il pid purgato , oro che fi conij, ¢
fi pud-dire ii noflco upghero . Si teovano ancora de’ dobloni di quattro ,-¢ cin-
que 5 €.disle} doppie1’.uno , di conio Fiorentino, sha yt :

SPAKTIMENT! , Divifioni , feparamenti . Chiamiamo {partimenti quelle,
divifioni di'tereeno , che Gi fanno ne 1 giardini per piantarui le cipolle da tiori .

ali (partimenti: , fe bene fono di diverle figure , fi dicono anche quairi . Vedi
pe C,6,-ttan. 63..E per fimilitudine aiciamo {partimenti te divifioni » che fi
trovano ineafiecte , 0 fcatole , come crano queiti delle monere ,

_VENNERO pris hafere. Intendi Avvifi , 0 imbalciace 5 che Staferta appreffo
di noi ,¢:1o fteflo, che Corriere. Sp. efafera .

\ BAR matte’. Elo fteflo che abbaccarli con uno ¢ parlargli, Vedi fopra C,
2, flan, 59.,in altro fignificato, — > ne

STA sm full aii. EP all’ ordine per partirfi . SST

. FAR come le fpofe . Significa ritornare ; lo dichiara il Poeta medefimo,dicendo:
Tdeft 1a ritornata ; E quetto perché gia coflumavafi , ¢ forfe ancora in alcuni Iu

   
 

es:

aS

, ee ee ee

=e 2 SS Sw

PR 6S we

d-
Koyy 1s ghi

 

 
=

 

 
 

s4o
hi@eoRitma , che le si dopo'effere ftate dicti’, 0 pre:
foie rotniao alla cafa paceraa’s Fer sephe qui git

Teniarns dell Achinea . Taupe lo fallone , ‘che* cated
che Achinea , 0 Chinea , intendiamo il cavailo buon rer
éuina {pecie di cavaili particolare «Sp, bacanea . Franz, bacquenen’y
STANZA XXXXIV, ts “STANZA. Xx
O mai é tempo , cara Celidora ,
Ch! inver{o li miei [udditi m' apprefi >
Che 'l trattene*mi di vanvargio faora y

  
    
   
 
 
   
    
 
 
 
 
 
 
 
 
 
 
  
 
 

Pregsndicar potrebbe a' miei intereffi Dite , non ci oi fulle corda ,

Pero qui refea tu co! tuoi ,sn buon bee, Bifog a Lmeteae epee a
E farti anwe , e rifpercar da effi y ( Rifpsfe il General) 3 ella 8

Ed in ordine a quefto i conviene ee ome t

Fare anche un’ altra cofa per tuo bene,
STANZA XXXKXV.
Perché , s' io parte ei »cugina mia,
Non fo 2 fe tn ci havraituttiitnsigufti,
Che qui non é neffun., che per te fia ,
Mentre forsee poi nuowi difeusti ,
Ma voglia il Ciel , ch’ io dica la bugias

Ed ogni modo vo’ , che tut’ aggiuiti , ‘tipo prefto sles of
Per ficurtd con an compatie » Ugquale Vuolotu? parla. a Her =e |

S accafi teco ,¢ qucfto, ¢ il Generale, D: mat pitt si, ¢ daccela in fa

STANZA XXKXVL STANZA AKAM ;

LT byei hati difender fi da vanto, Ed ella nel fentir , cons eit affrin

Che tn vedi,egli ¢branoquarun Marte, A dar pronta: rifpotta atal do

E fe finor per noi ha fatto tanto , D' un modefto roffor tutta,

Pifa quel che eifara,s'egli entra aparte,

Orsit  daglt la mano; cana [it ilgnanto;

E voi non ve ne fiate pitt in difparte,

Cafa Latoni , 0 Amoftante noftro

Fareui innanzi , dite il fatto voftro

STANZA L

Degli dunque la mano in mia prejenzas 3 Ma per non recar tedio

E voi , 0 General , datela a les , Ideft a chi afcolta i verfi mitiy

Ch io ‘voglio prima della mia partenca

Veder folennizzar quefti Flimenci . La[cidgliyadiame;

Baldone da per fpofa Celidora al Generale Amoitaate Latoni “ai

dopo haver narrato il difcorfo fatto da Baldone a paliow per indurlaa

tarfi d’ haver quefto marito, ed i foliti lezzi donnefchi farti da 2

dir di si; paffa a di(correr d’ un’ altra fpofa , che ¢ Psiche , cone ee i

Lagere ouave .

hai neJunsche per te fia « Non hai nefiyno , she aid

  
     
 
 
 
 
   
  
 

 
 
  

a

    
‘

DVODECIMO,EDVLTIMOCANTARE. 548

OVVTA . Termine che fignifica {pedizione , © incalzamenio a far prefto. BE’ il
Latino Hia ,age . Vedi fopra C. 6, fan, go. alla voce, horse, aiegh 3
PASS ATE gud. Venite qua. Lat. ade/dum. B: modo di dire , che fignificas
comandar con imperio ,.¢ con (everita , ed ha del bravatorio. R59,
SE vi piace la pannina , Se vi piace la mereanaia y cio¢ Celidora .
NON ¢i tenete piit in fulla corda . Non ci fate pid Aentage ,o defiderar la rifpo-
fla . Nom cé renere piis coll’ animp dubbio , ¢ fofpefe, :
SON bell’ ¢ accordato. lo fono,affatto d’ accordo ; fon contentiffimo . Vedi fo-
pra C. 3. faa, 14, Quefto termine bee, ‘
TERREL d bauerne di beato, Lo riputerei mia gran felicita , Stimerei d’ haver
gran forte, WV’ avrei di carti, Mi terrei d’ etfer beato , ee
EGL1¢ dower fentir  altca campana. E’ cofa giulta fentir I altra parte ,
TRANA, Quefta voce non havrebbe alcun fignificaco , fe bene ¢ affai ufata 5
ma perche pace,che immiti il fuono della tromba , quando fi da la moffa a i ca~
vali , che corrono al palio; ci ferue per efprimer mxovité  /pedi/citi , sbrigati a.
far la tal cofa, Q pure ¢ detto Trana, cioe tra’ pur/d tira avanti ; dal verbo Tra-
nare , che vale trarre con fatica qualche cofa , ¢ ftrafcinarla .

, ALAL pitt. Quetto termine ulato nel modo , che é nella prefente Ortava, ci é
familiarifimo , ¢d ha quati lo ftefio figniticato che evvia detto poco fopra, e s'ula
Pua per F altro in occatione di ftimolar quaicheduno a fpedirfi ; ed efprime unas
certa impazzienza di colui , che ftimola. E’ il Lat. ea tandem. Finifcila ,. dille
ana volta,

DAG ELA in fanore . Rifpondi fecondo il nofiro defiderio, Quando fi vince
una lite , fi dice haner 1a fentenza in fanore . .

CUOKIE con (a ghirlanda . Significa morir vergine. A coloro che muoiono in
¢oncetto di vergini , quando fi portano al fepolcro , coftumafi di porre in telta
una ghirlanda di fori in fegno della loro caftita . Qui il Pocta fcherza , come &

, folito farfi , quando fi difcorre d’ una donna impudica , che Gdice Elba giurate
di morir con (a ghirlanda , ¢d & detto ironicamence , ¢ per intendere , e//a vual por=
i tare il vanto ye La corona delle donne impudiche , Ma non per queito il Poeta (ches
molto ben fi ricorda , che Celidora , per effere flaca moglie del Re di Malmanti-
le , non é pi da ghirlanda , intende , che Celidora fofle impudica , ma dice gosh
per ifcherzo , ¢ per fegu tare il coftume della plebe , la quale , quand’ uao nomi-
t na forella , madre , 0 moglic, fuol dire ; purtana di me, ¢ fimui . Se fi parla d?
amumogilati fuol dire becco del diavolo , ee. Tal cohtume moitrd il Pocta ancor fo-
praC. 2. flan, 21. dove dicendo : 4 faper quante paia fan tre buoi , foggiugne {fybi-
to Se ben dat padre , ec. ¢ vuole intender padre bue , fecondo lo {cherzo fuddetto :
' Non é pero queito ftimato offefa , percht avvien fempre detto per ifcherzo; ma
4 ricice bene odiolo , ¢ riaferefcevole I’ eder.u/aco fpeflo , ed in ogni congiuntura ,
y come é ufato fra i pil vili, che lo fanno per parer fagaci , ¢ concettof.
¢ Sl riftringe nelle [paile . Cioe 8’ accorda , ed accop/ente a quel , che altri dice,
ib
v

aS Sen

© propone. EB’ un’ arto folito farfi da quelli , che & rimettono , o aderifcono alla
yoloata d’ uno , per non poter fare alttuncnti , 0 conuinti dalle ragioni, o indo
ti dalla necedlica , quafi dicano: Parienza; Bifogna frarct. Bocc. Giorn, 2, nov, 8,
‘ Ada pure nelle [palte rifiretto casi quella dagiar a daee setae mole Mas /ihoorre AnGa,
yar 2. fe

 

 
 

   
 
    
   
 
  
   
  

Sate
Eefubetiesaivolen nos fi faceia effert “7
volta della tefta 5 non dimend dictarho >. re ;

0 garbate : O cost fta'bene Lat, edge, perphtore belle Te
fue ii contento |, che's’ ha», he una’cofasfucceda’ fecondo chefi defid
APREST Oye male, te cone dafser Meglio’¢ farimale'y¢ pre!
{i mai col penfiero “dis volér far benew Chi fa o|,,emale pfiaalineare’
cha facenuy adagio 5 ¢ bene , mainon conchiude , o-termina‘quel’cheha
moidi fare , non fi puddin che facciay ¢ yeramente nonfa'y e pend nell'c
dei fare ¢ meglio far male , che non fare. ‘
DATE (a mano Dar ia mado (Latinoviuagere\ dexreras yO la.
nia, che fr faccia negii fpontaiizai') ¢ dice impalmare O:far Limpalmamentos,
STANZA Lie oS BAN ZAMLDL oub
Sogwitoical {xb Lvoe gud Phiche bakes ,
(olLanSenegee |y. che sn: last frggiafh patra
mand eiskincdrfecon La ging iddaes, y\

 

 
   

  

ee © al dueilo non volle la gatta; Per eat sa i
quefamnalivnara Medesy) >) sion Lagwale
ieplaeaens ere ot mqe) asBe Pe ep

neater (grades ust & Biche trt/ud honor ae {
ones’, aldan pian, Ces ie perdadayy 00% ~ nel e es aero ae we }
ia

    
   
 
   
 
   
    

STANZA LIL; oralga on nenaaet
Bit won potends bauer Cupide {pofoy 08 90 Pereincomintance/m: ets a4 F
hori: Amardai martha tontana >: \\Bacendo com? il'can delParcolano,, ©)
Ou Revael,s elapher (can iia Ucanegdlah iG 9% (O'all! infatara now P
Che pur veduto fia da corpo humane: E non pao ines eae
Martinazza haveddo prdiilto , che dovea effer fatta imoriré, eiche per Gupi
do non dovea effer piirfucsfpolo,, inttidiofa , che.quefto'be ne havetioa epodie dd
alteiy: !-haveva incancatoun-udga igacto per impedire: yoche‘altrinon havefe
\ EFOGIFA ratta., Boggiva velotemente, Ratto viene dal ae eee
verbio Fiorentino ; .Cbrva-pianosa ratto , corri(pondente ai Latiaoy
GING ADEA , Intend lafpada , come s‘intende: conunenene wb al
deta dail’impugnarficé tutte cinque le dicasete bene itbaftone pure simpugna coeur hur
te cinque de-dita , non fi di¢e-cinquadea y pecché quefto fipud im}
digch jal che\non Gi — fare delianfpads ordinariayy 0fe pur ff
© con difficulta VS RSH As
wuollagstea, a vuolattendete pNomwuol'badarey
Rissmnneiir quel tal:negozio. Hl Berni nell\Orlando y=
« Chey come fi fuol dir , voglit la gutta, ~~
OVA Aeden B+ uora lacrudela, che eh Medea fi
Oza Re de’ Colchi: ,»verfaril fratello Absyreo- opr )
fo, Glauca fua rivale y¢' co yet th fuo ne per 4

  

"ihe ‘Vee Rte Sceey cr

jeicodeind Mamie) mateo; A Gatto ata

goD

 
Se peeve fats = &

-

ie
6

|

 

DVODECTMO,ED VLTIMOGANTARE ‘5432

 

 

 

ne fuan'o\pibes inet faitem a <, be aL a oe
do)0-da quaiche donni at iftra ye wih won ; sfiss alist ctloe

TIRA per dado. “Conia aplageresrnoraands pil Bettilenels
la milizia, foldati infieme habbiano commefio qualche delitto ‘ca-

pitale , farmorice tn di loro’y,¢ falvar.la:vita a tutti gli alert, facendo loro tiz
rariila-forte ne s€ perd 5.1 orcas dettirdadi, ¢ da-credere , che ace
compagnine tal funzione con, fo! i xe con pianti ;.¢ fimo perd sche il Poetas
digcndo ztiraper-dade y intenda , toipieay © plange pill di cuore che mai; /eguirae
Piangeress pisces gagardamene yes sie pare, she non heaas here aim > 6 fia -
da principio,

“hssan wage. Effer defiderofa d' una tahoe . « Saiwere vago , che vuol dir be
lo , adarne.yec. Sig igiia\ ancora in quefto fendi bramefo, ec. Tiraleé — divbes cir
yuol dire : Zi tale genio’, ha gufto di betle burle , ¢ feberzi.

HA gid fanoil-pianto. Liha gia pianto per perduto. Termine affai ufarorims:
Gimili congiunture.. Pianto & quellamento , che fi fa-fopra il morto,decgo:cosi.dal
batterti, per.dolore il petto .» Latino planitus, rodalia = voce Lavina:hanne fat-
ta Gmiimente i Pranzefi la loro Péainte , seh ats eh

eA LZ AR capanne , eo, Ciot quei monti di fcope 9 ec. chevsaveno fatei per &b-
biuciar Martinazza come fredetto fopra in quetto G.t.-3» equeftefonove 2o/e
as Fusco, ie quali dices che sshanno a fare per-hongrdi Jet; \cheper altrovyquan-

do diciamo:: s! banno a fare.cofe di ey on st afarcofe, bole ; wine.

frofe , ¢ fuori del confueto ,

FAR come il cane dell’ ortolano y Ciot non voleres ° ‘non. potere havert uaa cO+
fa 5 ed-impedire y che altri J*habbia , come fa ilcané dell*orcolano’} che nin,
fuangia-! crbaggio y ¢.non vuole che altet lo. Piglt Canis in Prafepi + Provetbio
nlato da bucianoy

eT AN Ze: bItbbs

  

 

 

“iquid 6 of STANZA LV M
tio, \e Bsiche bebbtrogeuife Cos ‘byes: affanni ¥¢ le fatiche 0!) Ob
< WDE extte quello eb’ é fegiito'ie Corre; » Soffente per rant’ anni, ¢ lafri ee

~Gbda il teiogoappinte now fi farprecifoy

Risrovatofi, Amore ; ed egli, e Priche

AtRena fi fainsaprer'tattele porte; » Rappattumato fu dai cavalieri 30
fe Amanro crofeiar fenve/i wrgran rife, >: Onde foordats deli"ingiurie amicbe 5 a
atest \obie peg gio; poi fwonsr; ma forte Eriuniti pit che volentieri :
& Aeafivmare.di-pefe sr aboccanti 5 + vad regp spofi fero i bactabaffi yoo 9
i obSemea sunofceriehi rece, Contantics’ bq Reftando # parte diver foe je (pai
STANZA LV. STANZA shVTR cor 6
“Gir per peniefcate ognnn preftoaddirizza y (i Gluntis cialdéni pots e fare i bile,
Che dal timor gli # arricciane é peli . Ml Duca diede al fin 2 ultimo Addie’ -
Ma C alagrillo aitiero , ¢-pien di fiicxa “E Jubiro conagni [uo vaffallo
ib oGem talus frrifeia fa colps cradels 5 =: dnnerfa Venano Spiele it-pendio
wi Wa per le ftance fende,taglia.e infizza, E Catagrilio ix groppa al-fud <aualld
jlut 444 mon cliappa, fe vende’ logences cil’ © \Preforoon Pficle it Raretrare Dia ,)

~ien Rar tde inns ¢ i¢ok fucertibroinranty y.
E il Diavol cacciaye manda vialincato,

o8e\Gupido per opra dij Pakide Aixicrova je per inekzo di-quei Cavalieri’

“3uVO

‘¢Aashrei pars); eintefolildor 20
Gb ricomduffe ‘ali’ Amoroso. as
 

 
   
  

344
con Psiche , fi fanno le fefte delio {polalizio di
Jo di loa Bache con iain beioenta ~ dy,
lo accompagaa Psiche se Regno d’,
€ROSCLAR an ria. Rider gagliatdamenre- Vedi
GRAV , traboce: Gravi:pil del giufto pelo ,
on delle nee ee on fe ne feeuc per
¢ feguita 5 chi recé contanti ( che termine proprio
= intender , chi dava fe heeds ‘
e4ADDIKIZZ 4Ciok va via. Fugge per la pitt hota temas
STKISCLA; Intendila spada , come intefe (apra:C. 2: ft, 60. ° ae
CHIAPPA , Coglig s ritrova , perquote s¢aipilce . Vedi fopra C7.
RAGNATELI, Ragni,piccolt vermis o inferti nati... Vedi fopra.
Le flanze piens di ragnateli tignifica vote dogui.altea fa. Siauimente |
yolendo dire il borficchio voto , dite; Plexys facculius eff arancaram,
RAP PATTV MATL. \ocendiamo rappacificati. Da molti fi dice
ge di pace donde : O vincere , o patrare , clo’ pareggiare ; far pace» ae
gredo venga quetto verbo rappatramare , il quale ¢ atlai ufato , mala) 4
da pochi fuori della plebe . : :
CLALDONL, Specie di pafta confetta , condorta fowtile come V oft atone
ta, ¢ridotta oer un grofio.canneilo di canna
STANZA LVIILED VETIMA.,
Finito ¢ il noftro fcherzo : bor facciam efi »
Perché la Storia mia non va pil avanti y
Sicche da fare adefo alte» non refhay.
Se non ch’ io riverifca gli afcoltanti ¢
Ond' io percio cavandomi di tefta y i dante at
ei v' inchino , € ringrazso tutrs quanti; eee
Stretta [a foglia fia , larga la via: oe a
Dite la voftra , ch’ é ho detto la mia,
SCHERZO , Qui vale per trattenimento, Latino /w/us, Sogtionai ‘nofti Cons
tadini , speedo. fanno le loro veglie di balio 2 saee she hanno ua nen ballatoy
d alche intermedio , rappr di forge» O
altro ,¢ sun chiamano/o /cherzo , che per lo pit fiaflee in burlar qua

  
      
 
  
    
 
    
 

    

plice , ¢ dat’ occafione di ridere , ¢ quefto tale € poi anche detto da, I iy
} inrendiamo conmncmente ; ¢d il nofiro Pocta molto bea I’ efprit fe wo
fene nella fua lettera alla Sereni(s. Arciduchela Claudia d'Auftria,riportata fopra

nel Proemio ,dicendo: Conrentandomi io,che la mia Lergenda,come papas

smi facia seers alle genti, Ge had
FATE feta, Cros fiate licenziati , Vedi fopra C. 10, ft. 42.

av wo Bs OF y

Nota, amorevole Lettore , che il Poeta per terminare la osteo it fe Does .

ringraziando con quefta ultima Oxtava gli vditori, fi ferue della chinf

ed ufata dalle donnicciuole , quand’ hanno raccoutata una novella ;
Stretea la foglia fia , larga la via ;

Die la Voftra,ch' 10 ho detto la mia, _
E conchinde , che ha coutata una Novella, come 7

 

  
 

 

DVODECIMO,EDVLTIMOCANTARE. FAK

io di quef’ Opera, Ed io pure me ne feruo per incitare altri.a-dir qualeolarme-
lio di ee iubhia teat aena.to4 10 wi dice wal dicharates o,pures
confondere , ed — quello che nella — Opera ho ftimato poco intel-
ligibile fuori della Citta di Firenze , ¢ prego il difcreto Lettore a compatir
me ,/che per ubbidire ho pigliato a far’ un volo fuperiore alle\mie forze ,-¢d ais
contentarfi di biafimar me folo , ¢ non quei , che mi comandd ; perch¢ habbias

 

fawtoerrore nell’ elezione . E fo punto
PKCR
FINE DEL XILEDVLTIMOCANTARE. 4

, ie

' 4
ts

:

i

n | 25.

,

¥

bi

yf

it

a

+

a

4 i

u

a

; {

,

% ”

 
 

 
  
 
   
 
   
 
 
    
   
    
 
   
   
    
   
    
 
 
   

é ; MEE LY Big

I Molto Rev, Sig Gio; Domenico
cia di riconoleeté con ogni di
. Opera fouo il Titolo di Malmant
Zipolt , vi fia cov alecuna , che ¢
~ Cattolica , eda’ buoni ‘Coftumi 4
» Maggio 1686. ; =e aoe ’

Niccolo Caftellam Vic. Cen. Fiorent, dam Ry ia

fi
Mluftrifs. e Rev. Sig, g
Ho attentamente Jetto Oe cor Operetta al
le Racqusftato di Perlone Zipolt , infieme con le fae note
fpiegazioni , ¢ per non ayervi trovato cofa , ie
alla Santa Fede Cattolica ,ed a’ buoni coftumi,
mano mi fofcrivo. Firenze 20, Settem. 1686,
Gro. Domenico del Bruno en Sac, Ti

Attefa la foptaddetta selazione fi ‘flampi > offervati gli ordini
foliti, Data z0.Settemb, 1686. > Z ;
Niccold Cafteliani Vic. Ge

I Molto Rev. Padre Lettore Dolci Minor Otfrvante Conf i
tore del Sant’ Vfizio di Firenze legga attentamente la
fente Opera di, Perlone Zipoli ,. intitolata Malmantile
quiftato , ¢ ritrovandoyi cofa tepugnante alla Sat
Cattolica , ¢ buoni coftumi, riferilca , Dal 9, Vfizio.
renze 17, Ottobre 1686. : :
Fr, Francefio Agoftino Gambaroua Min,
Del S, Vyizso.

Reverendifs. Padre , :
Ho rivifta, ¢ ben confiderata I Opera intitolata A
le di Perlone Zipoli , ¢ per non ecfleryi cola repu
-aggiunte , ftimo poffa
D'Ogni Santi li 24. Febbr. 1686, Pe
Fr, Bragio Dolei Ain, Offer. Conf, del S. j

Attenta prefata relazione .
Imprimatur ;

Fr. Ces. Pallarvicinus Ordimis Min, Convent, Vie, Ge
S. Off. Florentia .

Ruberto Pandolfini Senat. ¢ Aud. di S, A. S.

 
  
   
   

 

Stel 3% oy rm i

 

BT 2Mh ood ww
 
